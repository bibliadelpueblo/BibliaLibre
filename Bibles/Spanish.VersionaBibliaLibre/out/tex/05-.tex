\hypertarget{section}{%
\section{1}\label{section}}

\bibverse{1} El Señor le habló a Moisés en el Tabernáculo de Reunión
mientras estaban en el desierto del Sinaí. Esto fue el primer día del
segundo mes, dos años después de que los israelitas salieran de Egipto.
Le dijo: \bibverse{2} Censen a todos los israelitas según su tribu y su
familia. Cuenten a cada hombre y mantengan un registro del nombre de
cada uno. \bibverse{3} ú y Aarón deberán registrar a todos los mayores
de veinte años que sean aptos para prestar el servicio militar según sus
divisiones en el ejército israelita. \bibverse{4} ara ayudarlos habrá
estar con ustedes un representante de cada tribu, que es el jefe de cada
familia:

\bibverse{5} Estos son los nombres de los hombres que trabajarán con
ustedes:

De la tribu de Rubén, Elisur, hijo de Sedeur;

\bibverse{6} de la tribu de Simeón, Selumiel, hijo de Zurisadai;

\bibverse{7} de la tribu de Judá, Naasón, hijo de Aminadab;

\bibverse{8} de la tribu de Isacar, Nataanel, hijo de Zuar;

\bibverse{9} de la tribu de Zabulón, Eliab, hijo de Helón;

\bibverse{10} de los hijos de José: de la tribu de Efraín, Elisama, hijo
de Amihud; y de la tribu de Manasés, Gamaliel, hijo de Pedasur;

\bibverse{11} de la tribu de Benjamín, Abidán, hijo de Gedeoni;

\bibverse{12} de la tribu de Dan, Ajiezer, hijo de Amisadai;

\bibverse{13} de la tribu de Aser, Pagiel, hijo de Ocrán;

\bibverse{14} de la tribu de Gad, Eliasaf, hijo de Deuel;

\bibverse{15} y de la tribu de Neftalí, Ahira, hijo de Enán.''

\bibverse{16} Estos fueron los hombres elegidos de la comunidad
israelita. Eran los jefes de las tribus de sus padres; los jefes de las
familias de Israel. \bibverse{17} Moisés y Aarón convocaron a estos
hombres que habían sido seleccionados por nombre. \bibverse{18} Hicieron
que todos los israelitas se reunieran el primer día del segundo mes, y
registraron la genealogía del pueblo según su tribu y familia, y
contaron los nombres de todos los que tenían veinte años o más,
\bibverse{19} como el Señor le había dicho a Moisés que hiciera. Moisés
llevó a cabo este censo en el desierto del Sinaí.

\bibverse{20} Los descendientes de Rubén, (que era el hijo primogénito
de Israel), hombres de veinte años o más, fueron registrados por nombre
según los registros genealógicos de su tribu y familias. Y todos los
hombres registrados que estaban aptos para servir en el ejército
\bibverse{21} n la tribu de Rubén sumaron 46.500.

\bibverse{22} Los descendientes de Simeón, hombres de veinte años o más,
fueron registrados por nombre según los registros genealógicos de su
tribu y sus familias. Todos hombres registrados que estaban aptos para
servir en el ejército, \bibverse{23} de la tribu de Simeón, sumaron
59.300.

\bibverse{24} Los descendientes de Gad, hombres de veinte años o más,
fueron registrados por nombre según los registros genealógicos de su
tribu y sus familias. Todos los hombres registrados que estaban aptos
para servir en el ejército, \bibverse{25} de la tribu de Gad, sumaron
45.650.

\bibverse{26} Los descendientes de Judá, hombres de veinte años o más,
fueron registrados por nombre según los registros genealógicos de su
tribu y sus familias. Todos los hombres inscritos, que estaban aptos
para servir en el ejército, \bibverse{27} de la tribu de Judá, sumaron
74.600.

\bibverse{28} Los descendientes de Isacar, hombres de veinte años o más,
fueron registrados por nombre según los registros genealógicos de su
tribu y sus familias. Todos los hombres inscritos que eran aptos para
servir en el ejército, \bibverse{29} de la tribu de Isacar, sumaron
54.400.

\bibverse{30} Los descendientes de Zabulón, hombres de veinte años o
más, fueron registrados por nombre según los registros genealógicos de
su tribu y sus familias. Todos los hombres inscritos que estaban aptos
para servir en el ejército, \bibverse{31} de la tribu de Zabulón,
sumaron 57.400.

\bibverse{32} Los descendientes de José: los descendientes de Efraín,
hombres de veinte años o más, fueron registrados por nombre según los
registros genealógicos de su tribu y sus familias. Todos los hombres
registrados que estaban aptos para servir en el ejército \bibverse{33}
de la tribu de Efraín, sumaron 40.500.

\bibverse{34} Y los descendientes de Manasés, hombres de veinte años o
más, fueron registrados por nombre según los registros genealógicos de
su tribu y sus familias. Todos los hombres registrados que estaban aptos
para servir en el ejército \bibverse{35} de la tribu de Manasés, sumaron
32.200.

\bibverse{36} Los descendientes de Benjamín, hombres de veinte años o
más, fueron registrados por nombre según los registros genealógicos de
su tribu y sus familias. Todos los hombres registrados que estaban aptos
para servir en el ejército, \bibverse{37} de la tribu de Benjamín,
totalizaban 35.400.

\bibverse{38} Los descendientes de Dan, hombres de veinte años o más,
fueron registrados por nombre según los registros genealógicos de su
tribu y sus familias. Todos los hombres registrados que estaban aptos
para servir en el ejército, \bibverse{39} de la tribu de Dan, sumaron
62.700.

\bibverse{40} Los descendientes de Aser, hombres de veinte años o más,
fueron registrados por nombre según los registros genealógicos de su
tribu y sus familias. Todos los hombres inscritos que estaban aptos para
servir en el ejército, \bibverse{41} de la tribu de Aser, sumaron
41.500.

\bibverse{42} Los descendientes de Neftalí, hombres de veinte años o
más, fueron registrados por nombre según los registros genealógicos de
su tribu y sus familias. Todos los hombres inscritos que estaban aptos
para servir en el ejército, \bibverse{43} de la tribu de Neftalí,
sumaron 53.400.

\bibverse{44} Estos fueron los totales de los hombres contados y
registrados por Moisés y Aarón, con la ayuda de los doce líderes de
Israel, que representaban cada uno a su familia. \bibverse{45} De esta
manera, todos los hombres israelitas de veinte años o más que pudieron
servir en el ejército de Israel fueron registrados según sus familias.
\bibverse{46} La suma total de los registrados fue de 603.550.

\bibverse{47} Sin embargo, los levitas no estaban registrados con los
demás según su tribu y sus familias. \bibverse{48} Esto se debió a que
el Señor le había dicho a Moisés: \bibverse{49} No registres la tribu de
Leví, ni los cuentes en el censo con los otros israelitas. \bibverse{50}
Pon a los levitas a cargo del Tabernáculo y del Testimonio,\footnote{1.50
  El Testimonio se refiere a las tablas de piedra de los Diez
  Mandamientos contenidas en el interior del Arca.} así como de todo su
mobiliario y de todo lo que hay en él. Ellos serán los responsables de
llevar el Tabernáculo y todos sus artículos. Deben cuidarlo, y hacer su
campamento alrededor de él. \bibverse{51} Cuando llegue el momento de
trasladar el Tabernáculo, los levitas lo bajarán, y cuando llegue el
momento de acampar, los levitas lo levantarán. Cualquier forastero que
se acerque al Tabernáculo debe ser condenado a muerte. \bibverse{52} Los
israelitas acamparán por tribus, cada uno estará en su propio
campamento, bajo su propia bandera. \bibverse{53} Pero los levitas deben
levantar su campamento alrededor del Tabernáculo del Testimonio para
evitar que alguien me haga enojar con los israelitas.+ 1.53
Presumiblemente impidiendo que cualquiera que no fuera sacerdote se
acercara demasiado al Tabernáculo.Los levitas son responsables de cuidar
el Tabernáculo del Testimonio''.

\bibverse{54} los israelitas hicieron todo lo que el Señor les ordenó a
través de Moisés.

\hypertarget{section-1}{%
\section{2}\label{section-1}}

\bibverse{1} El Señor les dijo a Moisés y Aarón: \bibverse{2} Los
israelitas deben establecer su campamento alrededor del Tabernáculo de
Reunión pero a cierta distancia de él. Cada miembro de cada tribu
acampará bajo su propia bandera y estandarte familiar. \bibverse{3} La
tribu de Judá acampará bajo su bandera en el lado este. Su líder es
Naasón, hijo de Aminadab, \bibverse{4} tiene 74.600 hombres.
\bibverse{5} La tribu de Isacar acampará junto a ellos. Su líder es
Natanael, hijo de Zuar, \bibverse{6} y tiene 54.400 hombres.
\bibverse{7} La siguiente es la tribu de Zabulón. Su líder es Eliab,
hijo de Helón, \bibverse{8} y tiene 57.400 hombres. \bibverse{9} Así que
el total de hombres en el territorio de Judá e de 186,400. Y cuando
llegue la hora de marcharse,\footnote{2.9 ``Cuando llegue la hora de
  marcharse'': añadido para mayor claridad.}ellos irán a la cabeza.

\bibverse{10} La tribu de Rubén acampará bajo su bandera en el lado sur.
Su líder es Elisur, hijo de Sedeúr, \bibverse{11} cuenta con 46.500
hombres. \bibverse{12} La tribu de Simeón acampará junto a ellos. Su
líder es Selumiel, hijo de Zurisadai, \bibverse{13} y cuenta con 59.300
hombres. \bibverse{14} La siguiente es la tribu de Gad. Su líder es
Eliasaph, hijo de Deuel, \bibverse{15} y cuenta con 45.650 hombres.
\bibverse{16} Así que el número total de hombres en el área del
campamento de Rubén es de 151.450. Ellos marcharán en segundo lugar.

\bibverse{17} El Tabernáculo de Reunión que está en el centro del
campamento acompañará a los levitas. Deben marchar en el mismo orden en
que levantaron el campamento, cada uno en el lugar que le corresponde,
bajo su bandera.

\bibverse{18} La tribu de Efraín acampará bajo su bandera en el lado
oeste. Su líder es Elisama, hijo de Amiud, \bibverse{19} y cuenta con
40.500 hombres. \bibverse{20} La tribu de Manasés acampará junto a
ellos. Su líder es Gamaliel, hijo de Pedasur, \bibverse{21} y cuenta con
32.200 hombres. \bibverse{22} La siguiente es la tribu de Benjamín. Su
líder es Abidán, hijo de Gedeoni, \bibverse{23} y cuenta con 35.400
hombres. \bibverse{24} Así que el número total de hombres en el área del
campamento de Efraín es de 108.100. Ellos marcharán en tercer lugar.

\bibverse{25} La tribu de Dan acampará bajo su bandera en el lado norte.
Su líder es Ajiezer, hijo de Amisadai, \bibverse{26} y cuenta con 62.700
hombres. \bibverse{27} La tribu de Aser acampará junto a ellos. Su líder
es Pagiel, hijo de Ocrán, \bibverse{28} y cuenta con 41.500 hombres.
\bibverse{29} A continuación estará la tribu de Neftalí. Su líder es
Ajirá, hijo de Enán, \bibverse{30} y cuenta con 53.400 hombres.
\bibverse{31} Así que el total de hombres en el área del campamento de
Dan es de 157.600. Ellosmarcharán en último lugar, con sus banderas''.

\bibverse{32} Este es un resumen del censo de los israelitas, hecho por
familia. El total final de los contados en los campamentos por tribus
fue de 603.550. \bibverse{33} Sin embargo, los levitas no fueron
contados entre los demás israelitas, siguiendo las instrucciones que el
Señor le dio a Moisés.

\bibverse{34} Los israelitas hicieron todo lo que el Señor le ordenó a
Moisés. Establecieron sus campamentos bajo sus banderas en sus
posiciones asignadas, y marchaban en el mismo orden, cada uno con su
propia tribu y familia.

\hypertarget{section-2}{%
\section{3}\label{section-2}}

\bibverse{1} Este es el relato sobre Aarón y Moisés cuando el Señor le
habló a Moisés en el Monte Sinaí. \bibverse{2} Los nombres de los hijos
de Aarón eran: Nadab (primogénito), Abiú, Eleazar e Itamar. \bibverse{3}
Estos eran los nombres de los hijos de Aarón que fueron ungidos y
ordenados para servir como sacerdotes.

\bibverse{4} Nadab y Abiú murieron en la presencia del Señor cuando
ofrecieron el fuego prohibido ante el Señor en el desierto del Sinaí.
Como no tenían hijos, Eleazar e Itamar sirvieron como sacerdotes
mientras su padre Aarón vivía.

\bibverse{5} El Señor le dijo a Moisés, \bibverse{6} Reúne a la tribu de
Leví y preséntalos ante el sacerdote Aarón para que le ayuden en el
ministerio. \bibverse{7} Deben cumplir con sus deberes en su nombre y en
nombre de todos los israelitas en el Tabernáculo de Reunión, cuidando el
servicio del Tabernáculo. \bibverse{8} Serán responsables de cuidar todo
el mobiliario del Tabernáculo de Reunión, sirviendo a los israelitas a
través de su trabajo en el Tabernáculo. \bibverse{9} Los levitas deben
trabajar exclusivamente para Aarón y sus hijos de porque esta es su
asignación entre los israelitas. \bibverse{10} Tú designarás a Aarón y a
sus hijos para que tengan la responsabilidad del sacerdocio. Cualquier
otro que intente actuar como sacerdote debe ser ejecutado.''

\bibverse{11} El Señor le dijo a Moisés: \bibverse{12} He tomado a los
levitas de entre los israelitas en lugar de cada uno de sus
primogénitos. Los levitas me pertenecen \bibverse{13} porque todos los
primogénitos son míos. Cuando maté a cada primogénito en Egipto, separé
como sagrado para mí a todos los primogénitos de Israel, humanos y
animales. Son míos. Yo soy el Señor''.

\bibverse{14} El Señor le habló a Moisés en el desierto del Sinaí, y le
dijo: \bibverse{15} Censa a los levitas según la genealogía de su padre
y su familia. Cuenten cada varón de un mes o mayor''. \bibverse{16}
Entonces Moisés los registró, siguiendo las instrucciones del Señor, tal
como se lo había dicho.

\bibverse{17} Estos eran los nombres de los hijos de Levi: Gersón, Coat
y Merari. \bibverse{18} Estos eran los nombres de los hijos de Gersón
por familia: Libni y Simeí. \bibverse{19} Los hijos de Coat por familia
eran Amram, Izar, Hebrón y Uziel. \bibverse{20} Los hijos de Merari, por
familia, eran Majlí y Musí. Estas eran las familias de los levitas,
según el linaje de su padre.

\bibverse{21} La familia de Libni y la familia de Simeí procedían de
Gersón. Estas eran las familias de Gersón. \bibverse{22} El total de
todos los varones de un mes o más era de 7.500. \bibverse{23} El
campamento de las familias de Gerson estaba al oeste, detrás del
Tabernáculo. \bibverse{24} El líder de las familias de Gerson era
Eliasaf, hijo de Lael. \bibverse{25} Su responsabilidad asignada para el
Tabernáculo de Reunión era cuidar del Tabernáculo y la tienda, su
cubierta, la cortina de la entrada del Tabernáculo de Reunión,
\bibverse{26} las cortinas del patio, la cortina de la entrada del patio
que rodea el Tabernáculo y el altar, las cuerdas y todo lo relacionado
con su uso.

\bibverse{27} Las familias de Amram, Izar, Hebrón y Uziel procedían de
Coat. Estas eran las familias de Coat. \bibverse{28} El total de todos
los varones de un mes o más era de 8.600. Su responsabilidad asignada
era cuidar del santuario. \bibverse{29} El campamento de las familias de
Coat estaba en el lado sur del Tabernáculo. \bibverse{30} El líder de
las familias de Coat era Elisafán, hijo de Uziel. \bibverse{31} Su
responsabilidad asignada era cuidar el Arca, la mesa, el candelabro, los
altares, los artículos del santuario usados con ellos, el velo, y todo
lo relacionado con estos artículos. \bibverse{32} El jefe de los líderes
de los levitas era Eleazar, hijo del sacerdote Aarón. Él estaba a cargo
de los responsables de servir en el santuario.

\bibverse{33} La familia de Majlí y la familia de Musíprocedían de
Merari. Estas eran las familias de Merari. \bibverse{34} El total de
todos los varones de un mes o más era de 6.200. \bibverse{35} El líder
de las familias de Merari era Zuriel, hijo de Abijaíl. Su campamento
estaba en el lado norte del Tabernáculo. \bibverse{36} Su
responsabilidad asignada era cuidar de los marcos del Tabernáculo,
barras transversales, postes, soportes, todo su equipo y todo lo
relacionado con su uso, \bibverse{37} así como los postes del patio
circundante con sus soportes, estacas y cuerdas.

\bibverse{38} El campamento de los hijos de Moisés, Aarón y Aarón estaba
al Este del santuario, con vista al amanecer, frente al Tabernáculo de
Reunión. Eran responsables del santuario en nombre de los israelitas.
Cualquier otro que intentara actuar como sacerdote debía ser ejecutado.
\bibverse{39} La suma total de levitas registrados por Moisés y Aarón
como el Señor ordenó fue de 22.000. Esto incluía a todos los varones de
un mes o mayores.

\bibverse{40} El Señor le dijo a Moisés:``Haz un censo de todos los
primogénitos varones israelitas de un mes o más, y registra sus nombres.
\bibverse{41} Aparta a los levitas para mí. Yo soy el Señor. Ellos están
en lugar de todos los primogénitos de los israelitas. El ganado de los
levitas está en lugar de todo el ganado primogénito de los israelitas''.

\bibverse{42} Moisés realizó un censo de todos los primogénitos de los
israelitas, tal como el Señor le había instruido. \bibverse{43} La suma
total de los primogénitos varones de un mes o más, registrados por
nombre, fue de 22.273.

\bibverse{44} El Señor habló con Moisés y le dijo: \bibverse{45} Debes
tomar a los levitas en lugar de todos los primogénitos de Israel, y el
ganado de los levitas en lugar de su ganado, porque los levitas me
pertenecen. Yo soy el Señor. \bibverse{46} Para poder comprar los 273
primogénitos de Israel que son más que el número de levitas,
\bibverse{47} se recaudan cinco siclos para cada uno de ellos, (usando
la norma del siclo del santuario de veinte geras). \bibverse{48}
Entregarás el dinero a Aarón y a sus hijos como precio de redención para
cubrir el exceso de los israelitas que sobran.'' \bibverse{49} Moisés
recaudó el dinero de redención para aquellos israelitas que excedían el
número redimido por los levitas. \bibverse{50} Recolectó el dinero dado
en nombre de los primogénitos de los israelitas. Llegó a recolectar
1.365 siclos, (usando el estándar del siclo del santuario).
\bibverse{51} Moisés dio este dinero de redención a Aarón y sus hijos
como el Señor se lo había dicho, siguiendo las instrucciones del Señor.

\hypertarget{section-3}{%
\section{4}\label{section-3}}

\bibverse{1} El Señor le dijo a Moisés y Aarón: \bibverse{2} Registra a
los descendientes de Coat de la tribu de Leví, de acuerdo a su familia y
línea paterna. \bibverse{3} Cuenta a los hombres de treinta a cincuenta
años y que tengan derecho a hacer el trabajo de servir en el Tabernáculo
de Reunión. \bibverse{4} Este trabajo que deben hacer en el Tabernáculo
de Reunión implica cuidar las cosas más sagradas. \bibverse{5} Cada vez
que muevan el campamento, Aarón y sus hijos entrarán, quitarán el velo y
lo colocarán sobre el Arca del Testimonio. \bibverse{6} Sobre esto
pondrán una fina cubierta de cuero, extenderán un paño de color azul
sólido sobre ella, y luego insertarán las varas para transportarlo.
\bibverse{7} Que extiendan también un paño azul sobre la mesa de la
Presencia, y que pongan sobre ella los platos y las copas, así como los
cuencos y las jarras para la ofrenda de bebida. La ofrenda permanente de
pan debe permanecer sobre ella. \bibverse{8} Sobre todas estas cosas
deben extender un paño carmesí, luego una fina cubierta de cuero, y
luego insertar las varas para transportarla.

\bibverse{9} Con un paño azul cubrirán el candelabro de luz, junto con
sus lámparas, pinzas de mecha y bandejas, así como los frascos de aceite
de oliva que se usan para llenarlos. \bibverse{10} Luego deben
envolverlo junto con todos sus utensilios dentro de una fina cubierta de
cuero y colocarlo enel bastidor para transportarlo. \bibverse{11} Deben
extender un paño azul sobre el altar de oro, cubrirlo con cuero fino, y
luego insertar sus varas para transportarlo. \bibverse{12} Deben colocar
todos los utensilios usados para el servicio en el santuario en un paño
azul, cubrirlos con cuero fino y colocarlos en el bastidor para
transportarlos.

\bibverse{13} Que limpien las cenizas del altar de bronce y extiendan un
paño morado sobre él, \bibverse{14} y que pongan sobre él todo el equipo
usado en los servicios del altar: los fogones, los tenedores para la
carne, las palas y los aspersores. Extiendan sobre él una fina cubierta
de cuero y luego inserten las varas para transportarlo.

\bibverse{15} Una vez que Aarón y sus hijos hayan terminado de cubrir
estas cosas sagradas y todo el equipo relacionado con ellas, cuando el
campamento esté listo para moverse, los sacerdotes de la familia Coat
vendrán y las llevarán. Pero tienen prohibido tocar cualquier cosa
sagrada, de lo contrario morirán. Estas son sus responsabilidades a la
hora de trasladar el Tabernáculo de Reunión.

\bibverse{16} Eleazar, hijo del sacerdote Aarón, supervisará la
obtención del aceite de oliva para las lámparas, el incienso aromático,
la ofrenda de grano diaria y el aceite de la unción. Estará a cargo de
todo el Tabernáculo y todo lo que hay en él, todas las cosas sagradas y
el equipo''.

\bibverse{17} Entonces el Señor le dijo a Moisés y Aarón: \bibverse{18}
Asegúrense de que las familias de Coat no sean eliminadas entre los
levitas. \bibverse{19} Esto es lo que tienes que hacer para que vivan y
no mueran por acercarse demasiado a un objeto sagrado: Aarón y sus hijos
deben entrar y decirle a cada uno de ellos lo que tienen que hacer y lo
que tienen que llevar. \bibverse{20} Pero no deben entrar y mirar las
cosas más sagradas, ni siquiera por un momento, de lo contrario
morirán.''

\bibverse{21} El Señor le dijo a Moisés: \bibverse{22} Registra a los
descendientes de Gersón, según su familia y el linaje paterno.
\bibverse{23} Cuenta a los hombres de treinta a cincuenta años que
tengan derecho a hacer el trabajo de servir en el Tabernáculo de
Reunión. \bibverse{24} Así es como las familias de Gersón servirán en
cuanto a trabajo y el traslado: \bibverse{25} levarán las cortinas del
Tabernáculocon su fina cubierta de cuero, las cortinas de la entrada del
Tabernáculo de Reunión, \bibverse{26} las cortinas del patio, la cortina
de la entrada del patio que rodea el Tabernáculo y el altar, las cuerdas
y todo lo relacionado con su uso. Las familias de Gersón son
responsables de todo lo que se requiera en relación con estos artículos.
\bibverse{27} Todo lo que hagan estará bajo la supervisión de Aarón y
sus hijos, así como todo el trabajo y las tareas que lleven a cabo.
Debes decirles todo lo que deben llevar. \bibverse{28} Estas son sus
responsabilidades para el traslado del Tabernáculo de Reunión, realizado
bajo la dirección de Itamar, hijo del sacerdote Aarón.

\bibverse{29} Registra los descendientes de Merari, según su familia y
linaje paterno. \bibverse{30} uenta a los hombres de treinta a cincuenta
años que tengan derecho a realizar el trabajo de servir en el
Tabernáculo de Reunión. \bibverse{31} Así es como servirán en el manejo
del Tabernáculo de Reunión: llevarán los marcos del Tabernáculo con sus
travesaños, postes y soportes, \bibverse{32} los postes del patio
circundante con sus soportes, estacas y cuerdas, todo su equipo y todo
lo relacionado con su uso. Debes decirles por su nombre lo que cada uno
debe llevar. \bibverse{33} Estas son sus responsabilidades por todo su
trabajo en el traslado del Tabernáculo de Reunión, realizado bajo la
dirección de Itamar, hijo del sacerdote Aarón.''

\bibverse{34} Moisés, Aarón y los líderes israelitas registraron a las
familias de Coat según el linaje de su familia y de su padre.
\bibverse{35} Contaban a los hombres de treinta a cincuenta años que
tenían derecho a hacer el trabajo de servir en el Tabernáculo de
Reunión. \bibverse{36} El total por familias fue de 2.750. \bibverse{37}
Este fue el total de las familias de Coat, y eran todos los que tenían
derecho a hacer el trabajo de servir en el Tabernáculo de Reunión.
Moisés y Aarón los registraron de acuerdo con las instrucciones que el
Señor le dio a Moisés.

\bibverse{38} Las familias de Gersón fueron contadas, de acuerdo a su
familia y linaje paterno, \bibverse{39} hombres de treinta a cincuenta
años de edad todos ellos con derecho a hacer el trabajo de servir en el
Tabernáculo de Reunión. \bibverse{40} El total por familias y linaje
paterno fue de 2.630. \bibverse{41} Este fue el total de las familias de
Gersón, todos los que tenían derecho a hacer el trabajo de servir en el
Tabernáculo de Reunión. Fueron registrados por Moisés y Aarón de acuerdo
con las instrucciones del Señor.

\bibverse{42} Las familias de Merari fueron contadas, según el linaje
familiar y paterno, \bibverse{43} hombres de treinta a cincuenta años de
edad, todos ellos con derecho a realizar el trabajo de servir en el
Tabernáculo de Reunión. \bibverse{44} El total por familias fue de
3.200. \bibverse{45} Este fue el total de las familias de Merari
registradas por Moisés y Aarón de acuerdo con las instrucciones del
Señor.

\bibverse{46} Así es como Moisés, Aarón y los líderes israelitas
registraron a todos los levitas de acuerdo a su familia y linaje
paterno. \bibverse{47} Contarona los hombres de treinta a cincuenta años
que tenían derecho a hacer el trabajo de servir en el Tabernáculo de
Reunión y llevarlo. \bibverse{48} La suma total fue de 8.580.
\bibverse{49} Fue en respuesta a las instrucciones del Señor que fueron
registrados por Moisés. A cada uno de los inscritos se les dijo qué
hacer y qué llevar, tal como el Señor se lo había ordenado a Moisés.

\hypertarget{section-4}{%
\section{5}\label{section-4}}

\bibverse{1} Entonces el Señor le dijo a Moisés: \bibverse{2} Ordena a
los israelitas que expulsen del campamento a cualquiera que tenga una
enfermedad de la piel, o que tenga una secreción, o que esté sucio por
tocar un cuerpo muerto.\footnote{5.2 ``Sucio por tocar un cuerpo
  muerto'': Esta parece ser una exclusión temporal. Ver Levítico 11:24.}
\bibverse{3} Ya sea hombre o mujer, debes expulsarlos para que no
ensucien su campamento, porque ahí es donde yo habito con ellos.''

\bibverse{4} Los israelitas siguieron estas instrucciones y expulsaron a
esas personas del campamento. Hicieron lo que el Señor le había dicho a
Moisés que debían hacer.

\bibverse{5} El Señor le dijo a Moisés: \bibverse{6} Dile a los
israelitas que cuando un hombre o una mujer es infiel al Señor pecando
contra alguien más, son culpables \bibverse{7} y deben confesar su
pecado. Tienen que pagar el monto total de la compensación más un quinto
de su valor, y darlo a la persona a la que han agraviado. \bibverse{8}
Sin embargo, si esa persona\footnote{5.8 Esta disposición se refiere a
  una situación en la que la persona agraviada ha muerto.} no tiene un
pariente que pueda recibir la compensación, ésta le pertenece al Señor y
será entregada al sacerdote, junto con un carnero de sacrificio con el
que se justifica al culpable. \bibverse{9} Todas las ofrendas sagradas
que los israelitas traigan al sacerdote, le pertenecen a él.
\bibverse{10} us santas ofrendas les pertenecen, pero una vez que se las
dan al sacerdote, le pertenecen a él.''

\bibverse{11} El Señor le dijo a Moisés: \bibverse{12} Dile a los
israelitas que estas son las instrucciones a seguir\footnote{5.12
  ``Estas son las instrucciones a seguir'': añadido para mayor claridad.}en
caso de que la esposa de un hombre tenga una aventura amorosa, siéndole
infiel a él \bibverse{13} por acostarse con otra persona. Puede ser que
su marido no se entere y que su acto sucio no haya sido presenciado. No
la atraparon. \bibverse{14} Pero si su marido se pone celoso y sospecha
de su mujer, sea culpable o no, \bibverse{15} debe llevarla ante el
sacerdote. También debe llevar en su nombre una ofrenda de un décimo de
efa de harina de cebada. También debe llevar para ella una ofrenda de un
efa de harina de cebada. No debe verter aceite de oliva o poner incienso
sobre ella, ya que es una ofrenda de grano por los celos, una ofrenda
recordatoria para recordarle a las personas sobre el pecado.

\bibverse{16} El sacerdote debe guiar a la esposa hacia adelante y hacer
que se presente ante el Señor. \bibverse{17} Luego llenará una vasija de
barro con agua sagrada y rociará sobre ella polvo del suelo del
Tabernáculo. \bibverse{18} Una vez que el sacerdote haya hecho que la
mujer se ponga de pie ante el Señor, le soltará el pelo y le hará
sostener la ofrenda de grano recordatoria, la ofrenda de grano que se
usa en casos de celos. El sacerdote sostendrá el agua amarga que
maldice. \bibverse{19} Pondrá a la mujer bajo juramento y le dirá: 'Si
nadie más ha dormido contigo y no has sido infiel ni te has vuelto
impura mientras estabas casada con tu marido, que no te perjudique esta
agua amarga que maldice. \bibverse{20} Pero si has sido infiel mientras
estabas casada con tu marido y te has vuelto impura y has tenido
relaciones sexuales con otra persona\ldots'' \bibverse{21} Aquí el
sacerdote pondrá a la mujer bajo juramento de la maldición como
sigue).``Que el Señor te envíe una maldición que todo el mundo conoce,
haciendo que tus muslos se encojan y tu vientre se hinche. \bibverse{22}
Que esta agua que maldice entre en tu estómago y haga que tu vientre se
hinche y tus muslos se encojan.''

La mujer debe responder: ``De acuerdo, estoy de acuerdo.''\footnote{5.22
  Literalmente, ``Amén, Amén.''}

\bibverse{23} El sacerdote debe escribir estas maldiciones en un
pergamino y luego lavarlas en el agua amarga. \bibverse{24} Hará que la
mujer beba el agua amarga que maldice, y le causará un dolor amargo si
es culpable.\footnote{5.24 ``Si es culpable'': implícito.} \bibverse{25}
El sacerdote le quitará la ofrenda de grano por los celos, la agitará
ante el Señor y la llevará al altar. \bibverse{26} Entonces el sacerdote
tomará un puñado de la ofrenda de grano como porción de recuerdo y lo
quemará en el altar, y hará que la mujer beba el agua.

\bibverse{27} Después de hacerla beber el agua, si ella se ha hecho
impura y ha sido infiel a su marido, entonces el agua que maldice le
causará un dolor amargo. Su vientre se hinchará y sus muslos se
encogerán. Se convertirá en una mujer maldita entre su pueblo.
\bibverse{28} Pero si la mujer no se ha hecho impura por ser infiel y
está limpia, no experimentará este castigo y aún podrá tener hijos.

\bibverse{29} Esta es la regla a seguir en casos de celos cuando una
mujer tiene una aventura y se hace impura mientras está casada con su
marido, \bibverse{30} o cuando el marido empieza a sentir celos y
sospecha de su esposa. Su esposa deberá presentarse ante el Señor, y el
sacerdote deberá cumplir cada parte de esta regla. \bibverse{31} Si es
hallada culpable,\footnote{5.31 ``Si es hallada culpable'': implícito.}su
marido no será responsable. Pero la mujer cargará con las consecuencias
de su pecado.

\hypertarget{section-5}{%
\section{6}\label{section-5}}

\bibverse{1} El Señor le dijo a Moisés: \bibverse{2} Dile a los
israelitas: Si un hombre o una mujer hace una promesa especial de
convertirse en nazareo,\footnote{6.2 ``Nazareo'': significa
  ``dedicado.''}para dedicarse al Señor, \bibverse{3} no deben beber
vino u otra bebida alcohólica. No deben ni siquiera beber vinagre de
vino o cualquier otra bebida alcohólica, o cualquier jugo de uva o comer
uvas o pasas. \bibverse{4} Durante todo el tiempo que estén dedicados al
Señor no deben comer nadaque sea fruto de una vid, ni siquiera las
semillas o las cáscaras de uva.

\bibverse{5} No deben usar una navaja de afeitar sobre sus cabezas
durante todo el tiempo de esta promesa de dedicación. Deben permanecer
santos hasta que su tiempo de dedicación al Señor haya terminado. Deben
dejar crecer su cabello.

\bibverse{6} Durante este tiempo de dedicación al Señor no deben
acercarse a un cadáver. \bibverse{7} Incluso si es su padre, madre,
hermano o hermana los que han muerto, no deben ensuciarse, porque su
pelo sin cortar anuncia su dedicación a Dios. \bibverse{8} Durante todo
el tiempo de su dedicación deben ser santos para el Señor.

\bibverse{9} Sin embargo, si alguien muere repentinamente cerca de
ellos, convirtiéndolos en inmundos, deben esperar siete días, y al
séptimo día cuando se limpien de nuevo deben afeitarse la cabeza.
\bibverse{10} El octavo día llevarán dos tórtolas o dos pichones al
sacerdote que está a la entrada del Tabernáculo de Reunión.
\bibverse{11} El sacerdote ofrecerá una como ofrenda por el pecado y la
otra como holocausto para corregirlas, porque se hicieron culpables por
estar cerca del cadáver. Ese día deben volver a dedicarse y dejar que
les vuelva a crecer el cabello. \bibverse{12} Deben volver a dedicarse
al Señor por el tiempo completo que prometieron originalmente y traer un
cordero macho de un año como ofrenda por la culpa. Los días anteriores
no cuentan para el tiempo de dedicación porque se volvieron inmundos.

\bibverse{13} Estas son las reglas que se deben observar cuando el
tiempo de dedicación del nazareo termine. Deben ser llevadas a la
entrada del Tabernáculo de Reunión. \bibverse{14} Allí deben presentar
una ofrenda al Señor de un cordero macho sin defectos de un año como
holocausto, un cordero hembra sin defectos de un año como ofrenda por el
pecado y un carnero sin defectos como ofrenda de paz. \bibverse{15}
Además deben traer una cesta de pan sin levadura hecha de la mejor
harina mezclada con aceite de oliva y obleas sin levadura recubiertas
con aceite de oliva así como sus ofrendas de granos y bebidas.
\bibverse{16} El sacerdote presentará todo esto ante el Señor, así como
el sacrificio de la ofrenda por el pecado y el holocausto. \bibverse{17}
También sacrificará un carnero como ofrenda de paz al Señor, junto con
la cesta de pan sin levadura. Además el sacerdote presentará la ofrenda
de grano y la ofrenda de bebida.

\bibverse{18} Luego los nazareos se afeitarán la cabeza a la entrada del
Tabernáculo de Reunión. Se quitarán el cabello de sus cabezas que fueron
dedicadas, y lo pondrán en el fuego bajo la ofrenda de paz.
\bibverse{19} Una vez que los nazareos se hayan afeitado, el sacerdote
tomará la espaldilla hervida del carnero, un pan sin levadura de la
cesta, y una oblea sin levadura, y los pondrá en sus manos.
\bibverse{20} El sacerdote los agitará como ofrenda mecida ante el
Señor. Estos artículos son sagrados y pertenecen al sacerdote, así como
el pecho de la ofrenda mecida y el muslo que fue ofrecido. Una vez que
esto termine, los nazareos podrán beber vino. \bibverse{21} Estas son
las reglas que deben observarse cuando un nazareo promete dar ofrendas
al Señor en relación con su dedicación. También pueden traer ofrendas
adicionales si tienen los medios para hacerlo. Cada nazareo debe cumplir
las promesas que ha hecho cuando se dedica.''

\bibverse{22} El Señor le dijo a Moisés: \bibverse{23} Dile a Aarón y a
sus hijos: Así es como debes bendecir a los israelitas. Esto es lo que
deben decir:

\bibverse{24} Que el Señor te bendiga y te cuide. \bibverse{25} Que el
Señor te sonría y sea misericordioso contigo. \bibverse{26} Que el Señor
te cuide y te dé la paz.' \bibverse{27} cuando los sacerdotes bendigan a
los israelitas en mi nombre, yo los bendeciré.''

\hypertarget{section-6}{%
\section{7}\label{section-6}}

\bibverse{1} El mismo día que Moisés terminó de montar el Tabernáculo,
lo ungió y lo dedicó, junto con todo su mobiliario, el altar y todos sus
utensilios. \bibverse{2} Los líderes israelitas, que eran los jefes de
sus familias, vinieron y dieron una ofrenda. Eran los mismos líderes de
las tribus que habían trabajado en el registro\footnote{7.2 Ver el
  capítulo 1.} de los israelitas. \bibverse{3} Trajeron al Señor una
ofrenda de seis carros cubiertos y doce bueyes. Cada líder dio un buey,
y dos líderes compartieron la ofrenda de un carro. Los presentaron
frente al Tabernáculo.

\bibverse{4} El Señor le dijo a Moisés: \bibverse{5} Acepta lo que te
dan y úsalo para el trabajo del Tabernáculo de Reunión. Dáselo a los
levitas para que los usen según sea necesario''.

\bibverse{6} Moisés aceptó las carretas y los bueyes y los entregó a los
levitas. \bibverse{7} Dio dos carros y cuatro bueyes a las familias de
Gersón para que los usaran según sus necesidades. \bibverse{8} Dio
cuatro carros y ocho bueyes a las familias de Merari, para que los
usaran según sus necesidades. Todo el trabajo debía hacerse bajo la
dirección de Itamar, hijo del sacerdote Aarón. \bibverse{9} No dio
carros ni bueyes a los coatitas porque su responsabilidad era llevar
sobre sus hombros los objetos sagrados asignados bajo su cuidado.

\bibverse{10} El día que el altar fue ungido, los líderes se presentaron
con sus ofrendas dedicatorias, presentándolas delante de él.
\bibverse{11} Entonces el Señor le dijo a Moisés:``Haz que un líder
venga cada día y presente su ofrenda para la dedicación del altar.''

\bibverse{12} El primer día Naasón, hijo de Aminadab, de la tribu de
Judá se adelantó con su ofrenda. \bibverse{13} Su ofrenda era una placa
de plata que pesaba ciento treinta siclos, y un tazón de plata que
pesaba setenta siclos, (usando la tasación del siclo según el
santuario). Ambos estaban llenos de la mejor harina mezclada con aceite
de oliva como ofrenda de grano. \bibverse{14} También presentó un plato
de oro que pesaba diez siclos llenos de incienso. Como sacrificios trajo
\bibverse{15} un novillo, un carnero y un cordero macho de un año como
holocausto, \bibverse{16} una cabra macho como ofrenda por el pecado,
\bibverse{17} y una ofrenda de paz de dos bueyes, cinco carneros, cinco
cabras macho, y cinco corderos macho de un año. Esta fue la ofrenda de
Naasón, hijo de Aminadab.

\bibverse{18} El segundo día se presentó Natanael, hijo de Zuar, el
líder de la tribu de Isacar. \bibverse{19} La ofrenda que presentó fue
una placa de plata que pesaba ciento treinta siclos, y un tazón de plata
que pesaba setenta siclos, (usando la tasación del siclo según el
santuario). Ambos estaban llenos de la mejor harina mezclada con aceite
de oliva como ofrenda de grano. \bibverse{20} También presentó un plato
de oro que pesaba diez siclos llenos de incienso. Como sacrificios trajo
\bibverse{21} un novillo, un carnero y un cordero macho de un año como
holocausto, \bibverse{22} una cabra macho como ofrenda por el pecado,
\bibverse{23} y una ofrenda de paz de dos bueyes, cinco carneros, cinco
cabras macho, y corderos macho de cinco años. Esta fue la ofrenda de
Natanael, hijo de Zuar.

\bibverse{24} El tercer día se presentó Eliab, hijo de Helón, el líder
de la tribu de Zabulón. \bibverse{25} La ofrenda que presentó fue una
placa de plata que pesaba ciento treinta siclos, y un cuenco de plata
que pesaba setenta siclos, (usando la tasación del siclo según el
santuario). Ambos estaban llenos de la mejor harina mezclada con aceite
de oliva como ofrenda de grano. \bibverse{26} También presentó un plato
de oro que pesaba diez siclos llenos de incienso. Como sacrificios trajo
\bibverse{27} un novillo, un carnero y un cordero macho de un año como
holocausto, \bibverse{28} una cabra macho como ofrenda por el pecado,
\bibverse{29} y una ofrenda de paz de dos bueyes, cinco carneros, cinco
cabras macho, y corderos macho de cinco años. Esta fue la ofrenda de
Eliab, hijo de Helón.

\bibverse{30} El cuarto día se presentó Elisur, hijo de Sedeúr, el líder
de la tribu de Rubén. \bibverse{31} La ofrenda que presentó fue una
placa de plata que pesaba ciento treinta siclos, y un cuenco de plata
que pesaba setenta siclos, (usando la tasación del siclo según el
santuario). Ambos estaban llenos de la mejor harina mezclada con aceite
de oliva como ofrenda de grano. \bibverse{32} También presentó un plato
de oro que pesaba diez siclos llenos de incienso. Como sacrificios trajo
\bibverse{33} un novillo, un carnero y un cordero macho de un año como
holocausto, \bibverse{34} una cabra macho como ofrenda por el pecado,
\bibverse{35} y una ofrenda de paz de dos bueyes, cinco carneros, cinco
cabras macho, y corderos macho de cinco años. Esta fue la ofrenda de
Elisur, hijo de Sedeúr.

\bibverse{36} El quinto día se presentó Selumiel, hijo de Zurisadai, el
líder de la tribu de Simeón. \bibverse{37} La ofrenda que presentó fue
una placa de plata que pesaba ciento treinta siclos, y un cuenco de
plata que pesaba setenta siclos, (usando la tasación del siclo según el
santuario). Ambos estaban llenos de la mejor harina mezclada con aceite
de oliva como ofrenda de grano. \bibverse{38} También presentó un plato
de oro que pesaba diez siclos llenos de incienso. Como sacrificios trajo
\bibverse{39} un novillo, un carnero y un cordero macho de un año como
holocausto, \bibverse{40} una cabra macho como ofrenda por el pecado,
\bibverse{41} y una ofrenda de paz de dos bueyes, cinco carneros, cinco
cabras macho, y corderos macho de cinco años. Esta fue la ofrenda de
Selumiel, hijo de Zurisadai.

\bibverse{42} El sexto día se presentó Eliasaf, hijo de Deuel, el líder
de la tribu de Gad. \bibverse{43} La ofrenda que presentó fue una placa
de plata que pesaba ciento treinta siclos, y un cuenco de plata que
pesaba setenta siclos, (usando la tasación del siclo según el
santuario). Ambos estaban llenos de la mejor harina mezclada con aceite
de oliva como ofrenda de grano. \bibverse{44} También presentó un plato
de oro que pesaba diez siclos llenos de incienso. Como sacrificios trajo
\bibverse{45} un novillo, un carnero y un cordero macho de un año como
holocausto, \bibverse{46} una cabra macho como ofrenda por el pecado,
\bibverse{47} y una ofrenda de paz de dos bueyes, cinco carneros, cinco
cabras macho, y corderos macho de cinco años. Esta fue la ofrenda de
Eliasaf, hijo de Deuel.

\bibverse{48} El séptimo día se presentó Elisama, hijo de Ammihud, el
líder de la tribu de Efraín. \bibverse{49} La ofrenda que presentó fue
una placa de plata que pesaba ciento treinta siclos, y un tazón de plata
que pesaba setenta siclos, (usando la tasación del siclo según el
santuario). Ambos estaban llenos de la mejor harina mezclada con aceite
de oliva como ofrenda de grano. \bibverse{50} También presentó un plato
de oro que pesaba diez siclos llenos de incienso. Como sacrificios trajo
\bibverse{51} un novillo, un carnero y un cordero macho de un año como
holocausto, \bibverse{52} una cabra macho como ofrenda por el pecado,
\bibverse{53} y una ofrenda de paz de dos bueyes, cinco carneros, cinco
cabras macho, y corderos macho de cinco años. Esta fue la ofrenda de
Elishama, hijo de Amiúd.

\bibverse{54} El octavo día se presentó Gamaliel, hijo de Pedasur, el
líder de la tribu de Manasés. \bibverse{55} La ofrenda que presentó fue
una placa de plata que pesaba ciento treinta siclos, y un tazón de plata
que pesaba setenta siclos, (usando la tasación del siclo según el
santuario). Ambos estaban llenos de la mejor harina mezclada con aceite
de oliva como ofrenda de grano. \bibverse{56} También presentó un plato
de oro que pesaba diez siclos llenos de incienso. Como sacrificios trajo
\bibverse{57} un novillo, un carnero y un cordero macho de un año como
holocausto, \bibverse{58} una cabra macho como ofrenda por el pecado,
\bibverse{59} y una ofrenda de paz de dos bueyes, cinco carneros, cinco
cabras macho, y corderos macho de cinco años. Esta fue la ofrenda de
Gamaliel, hijo de Pedasur.

\bibverse{60} El noveno día se presentó Abidán, hijo de Gideoni, el
líder de la tribu de Benjamín. \bibverse{61} La ofrenda que presentó fue
una placa de plata que pesaba ciento treinta siclos, y un tazón de plata
que pesaba setenta siclos, (usando la tasación del siclo según el
santuario). Ambos estaban llenos de la mejor harina mezclada con aceite
de oliva como ofrenda de grano. \bibverse{62} También presentó un plato
de oro que pesaba diez siclos llenos de incienso. Como sacrificios trajo
\bibverse{63} un novillo, un carnero y un cordero macho de un año como
holocausto, \bibverse{64} una cabra macho como ofrenda por el pecado,
\bibverse{65} y una ofrenda de paz de dos bueyes, cinco carneros, cinco
cabras macho, y corderos macho de cinco años. Esta fue la ofrenda de
Abidán, hijo de Gedeoni.

\bibverse{66} El décimo día se presentó Ahiezer, hijo de Amisadai, el
líder de la tribu de Dan. \bibverse{67} La ofrenda que presentó fue una
placa de plata que pesaba ciento treinta siclos, y un tazón de plata que
pesaba setenta siclos, (usando la tasación del siclo según el
santuario). Ambos estaban llenos de la mejor harina mezclada con aceite
de oliva como ofrenda de grano. \bibverse{68} También presentó un plato
de oro que pesaba diez siclos llenos de incienso. Como sacrificios trajo
\bibverse{69} un novillo, un carnero y un cordero macho de un año como
holocausto, \bibverse{70} una cabra macho como ofrenda por el pecado,
\bibverse{71} y una ofrenda de paz de dos bueyes, cinco carneros, cinco
cabras macho, y corderos macho de cinco años. Esta era la ofrenda de
Ahiezer, hijo de Amisadai.

\bibverse{72} El undécimo día se presentó Pagiel, hijo de Ocrán, el
líder de la tribu de Aser. \bibverse{73} La ofrenda que presentó fue una
placa de plata que pesaba ciento treinta siclos, y un cuenco de plata
que pesaba setenta siclos, (usando la tasación del siclo según el
santuario). Ambos estaban llenos de la mejor harina mezclada con aceite
de oliva como ofrenda de grano. \bibverse{74} También presentó un plato
de oro que pesaba diez siclos llenos de incienso. Como sacrificios trajo
\bibverse{75} un novillo, un carnero y un cordero macho de un año como
holocausto, \bibverse{76} una cabra macho como ofrenda por el pecado,
\bibverse{77} y una ofrenda de paz de dos bueyes, cinco carneros, cinco
cabras macho, y corderos macho de cinco años. Esta era la ofrenda de
Pagiel, hijo de Ocran.

\bibverse{78} El duodécimo día se presentó Ahira, hijo de Enán, el jefe
de la tribu de Neftalí. \bibverse{79} La ofrenda que presentó fue una
placa de plata que pesaba ciento treinta siclos, y un cuenco de plata
que pesaba setenta siclos, (usando la tasación del siclo según el
santuario). Ambos estaban llenos de la mejor harina mezclada con aceite
de oliva como ofrenda de grano. \bibverse{80} También presentó un plato
de oro que pesaba diez siclos llenos de incienso. Como sacrificios trajo
\bibverse{81} un novillo, un carnero y un cordero macho de un año como
holocausto, \bibverse{82} una cabra macho como ofrenda por el pecado,
\bibverse{83} y una ofrenda de paz de dos bueyes, cinco carneros, cinco
cabras macho, y corderos macho de cinco años. Esta fue la ofrenda de
Ahira, hijo de Enan.

\bibverse{84} Así que el día en que el altar fue ungido, las ofrendas
dedicatorias traídas por los líderes israelitas fueron doce platos de
plata, doce cuencos de plata y doce platos de oro. \bibverse{85} Cada
plato de plata pesaba ciento treinta siclos, y cada cuenco pesaba
setenta siclos. El peso total de la plata era de dos mil cuatrocientos
siclos, (usando la tasación del siclo según el santuario). \bibverse{86}
Los doce platos de oro llenos de incienso pesaban diez siclos cada uno,
(usando la tasación del siclo según el santuario). El peso total del oro
era de ciento veinte siclos. \bibverse{87} Los animales presentados como
holocausto eran doce toros, doce carneros y doce corderos machos de un
año, así como sus ofrendas de grano, y doce cabras machos como ofrenda
por el pecado. \bibverse{88} Los animales presentados como ofrenda de
paz eran veinticuatro toros, sesenta carneros, sesenta machos cabríos y
sesenta corderos machos de un año. Esta era la ofrenda de dedicación
para el altar una vez que había sido ungido.

\bibverse{89} Cada vez que Moisés entraba en el Tabernáculo de Reunión
para hablar con el Señor, oía la voz que le hablaba desde la tapa de
expiación del Arca del Testimonio entre los dos querubines. Así es como
el Señor le habló.

\hypertarget{section-7}{%
\section{8}\label{section-7}}

\bibverse{1} El Señor le dijo a Moisés: \bibverse{2} Dile a Aarón,
`Cuando pongas las siete lámparas en el candelabro, asegúrate de que
brillen hacia el frente.''' \bibverse{3} Así que eso es lo que hizo
Aarón. Colocó las lámparas hacia el frente del candelabro, como el Señor
le había ordenado a Moisés.

\bibverse{4} El candelabro estaba hecho de oro martillado desde su base
hasta los adornos florales de la parte superior, de acuerdo con el
diseño que el Señor había mostrado a Moisés.

\bibverse{5} El Señor le dijo a Moisés: \bibverse{6} Separa a los
levitas de los demás israelitas y purifícalos. \bibverse{7} os
purificarás así: Rocíalos con el agua de la purificación. Deben
afeitarse todo el pelo de sus cuerpos y lavar su ropa para que estén
limpios. \bibverse{8} Haz que traigan un novillo con su ofrenda de grano
de la mejor harina mezclada con aceite de oliva, y debes traer un
segundo novillo como ofrenda por el pecado. \bibverse{9} Toma a los
levitas y haz que se paren frente al Tabernáculo de Reunión y llama a
todos los israelitas para que se reúnan allí. \bibverse{10} Cuando
lleves a los levitas al Señor, los israelitas pondrán sus manos sobre
ellos. \bibverse{11} Aarón presentará a los levitas a Jehová como
ofrenda agitada de los israelitas para que hagan la obra de Jehová.
\bibverse{12} Los levitas pondrán sus manos sobre las cabezas de los
toros. Uno será sacrificado como ofrenda por el pecado al Señor, y el
otro como holocausto para reconciliar a los levitas con el Señor.
\bibverse{13} Que los levitas se pongan de pie delante de Aarón y sus
hijos y los presenten al Señor como ofrenda de ofrenda. \bibverse{14}
Así separarás a los levitas del resto de los israelitas, y los levitas
me pertenecerán a mí. \bibverse{15} Pueden venir a servir en el
Tabernáculo de Reunión una vez que los hayas purificado y presentado
como ofrenda mecida.

\bibverse{16} Los levitas han sido completamente consagrados a mí por
los israelitas. Los he aceptado como míos en lugar de todos los
primogénitos de los israelitas. \bibverse{17} Todo primogénito varón de
Israel me pertenece, tanto humano como animal. Los reservé para mí
cuando maté a todos los primogénitos de Egipto. \bibverse{18} He tomado
a los levitas en lugar de todos los primogénitos de los israelitas.
\bibverse{19} De todos los israelitas, los levitas son un regalo mío
para Aarón y sus hijos para servir a los israelitas en el Tabernáculo de
Reunión, y en su nombre para enderezarlos, para que no les pase nada
malo cuando vengan al santuario''.

\bibverse{20} Moisés, Aarón y todos los israelitas hicieron todo lo que
el Señor había ordenado a Moisés que hicieran con respecto a los
levitas. \bibverse{21} Los levitas se purificaron y lavaron sus ropas.
Entonces Aarón los presentó como ofrenda mecida al Señor. Aarón también
presentó el sacrificio para que estuvieran bien con el Señor para que
estuvieran limpios. \bibverse{22} Después los levitas vinieron a
realizar su servicio en el Tabernáculo de Reunión bajo la dirección de
Aarón y sus hijos. Siguieron todas las instrucciones sobre los levitas
que el Señor había dado a Moisés.

\bibverse{23} El Señor le dijo a Moisés: \bibverse{24} Esta regla se
aplica a los levitas. Los mayores de veinticinco años servirán en el
Tabernáculo de Reunión. \bibverse{25} Sin embargo, una vez que alcancen
la edad de cincuenta años deben retirarse del trabajo y no servirán más.
\bibverse{26} Todavía pueden ayudar a sus compañeros levitas en sus
tareas, pero no deben hacer el trabajo por sí mismos. Estos son los
arreglos en el caso de los levitas.''

\hypertarget{section-8}{%
\section{9}\label{section-8}}

\bibverse{1} El Señor le habló a Moisés en el desierto del Sinaí en el
primer mes, dos años después de que Israel dejara Egipto. Le dijo:
\bibverse{2} ``Los israelitas deben celebrar la Pascua en el momento
designado. \bibverse{3} La observarán a la hora requerida, en la tarde
después de la puesta del sol del día catorce de este mes, y lo harán de
acuerdo con sus reglas y normas.''

\bibverse{4} Moisés hizo un llamado a los israelitas para que observaran
la Pascua. \bibverse{5} Así que celebraron la Pascua en el desierto del
Sinaí, comenzando por la tarde después de la puesta del sol del día
catorce del primer mes. Los israelitas siguieron todas las instrucciones
que el Señor había dado a Moisés.

\bibverse{6} Sin embargo, había algunos hombres que eran impuros porque
habían estado en contacto con un cadáver, por lo que no podían celebrar
la Pascua ese día. Fueron a ver a Moisés y Aarón el mismo día
\bibverse{7} y le explicaron a Moisés: ``Somos inmundos por causa de un
cadáver, ¿pero por qué eso significa que no podemos dar nuestra ofrenda
al Señor con los demás israelitas en el momento oportuno?''

\bibverse{8} Quédense aquí mientras averiguo cuáles son las
instrucciones del Señor respecto a ustedes'', respondió Moisés.

\bibverse{9} Entonces el Señor le dijo a Moisés: \bibverse{10} Dile a
los israelitas: 'Si tú o tus descendientes están sucios por causa de un
cadáver, o están viajando, aún pueden celebrar la Pascua del Señor.
\bibverse{11} La observarán por la tarde, después de la puesta del sol,
en el día catorce del segundo mes. Comerán el cordero con el pan sin
levadura y las hierbas amargas. \bibverse{12} No deben dejar nada de él
hasta la mañana siguiente y no deben romper ninguno de sus huesos.
Deberán observar la Pascua de acuerdo con todas las normas.

\bibverse{13} Sin embargo, cualquiera que esté ceremonialmente limpio y
no viaje lejos y que no observe la Pascua debe ser expulsado de su
pueblo, porque no presentó la ofrenda del Señor en el momento apropiado.
Ellos serán responsables de las consecuencias de su pecado.
\bibverse{14} Cualquier extranjero que viva entre ustedes y que quiera
observar la Pascua del Señor puede hacerlo siguiendo las normas y
preceptos de la Pascua. Las mismas reglas se aplican a los extranjeros
como a ustedes.''

\bibverse{15} La nube cubrió la Tienda del Testimonio (el Tabernáculo)
el día en que fue erigida, y se vio como fuego sobre ella desde la noche
hasta la mañana. \bibverse{16} Siempre era así. La nube cubría el
Tabernáculo durante el día\footnote{9.16 ``Durante el día'': Tomado de
  la Septuaginta.} y por la noche parecía fuego. \bibverse{17} Cuando la
nube se levantaba sobre la Tienda, los israelitas marchaban, y cuando la
nube se detenía, los israelitas acampaban allí. \bibverse{18} Los
israelitas se movían cuando el Señor les decía, y levantaban el
campamento cuando el Señor les decía. Mientras la nube permanecía sobre
el Tabernáculo, ellos permanecían acampados allí. \bibverse{19} Aunque
la nube no se moviera durante mucho tiempo, los israelitas hicieron lo
que el Señor les decía y no seguían adelante. \bibverse{20} A veces la
nube sólo permanecía sobre el Tabernáculo durante unos pocos días. Como
siempre, siguieron la orden del Señor de acampar o seguir adelante.
\bibverse{21} A veces la nube sólo se quedaba durante la noche, así que
cuando se levantaban por la mañana seguían avanzando. Cada vez que la
nube se levantaba, de día o de noche, se marchaban. \bibverse{22} Si la
nube se quedaba en un lugar durante dos días, o un mes, o más tiempo,
los israelitas se quedaban donde estaban y no se iban mientras la nube
permaneciera sobre el Tabernáculo. Sin embargo, una vez que se
levantaba, se iban. \bibverse{23} Acampaban cuando el Señor les decía, y
se iban cuando él les decía. Ellos seguían las instrucciones del Señor
le daba a Moisés.

\hypertarget{section-9}{%
\section{10}\label{section-9}}

\bibverse{1} El Señor le dijo a Moisés: \bibverse{2} Haz dos trompetas
de plata martillada. Se usarán para convocar a los israelitas y para
hacer que el campamento se mueva. \bibverse{3} Cuando se toquen las dos
trompetas, todos los israelitas se reunirán ante ti en la entrada del
Tabernáculo de Reunión. \bibverse{4} Pero si sólo se toca una, sólo los
líderes de la tribu se reunirán ante ti.

\bibverse{5} Cuando se toque la trompeta, que es la señal de alarma para
salir, los campamentos del lado este deben salir primero. \bibverse{6}
Cuando se toca la trompeta por segunda vez, los campamentos del lado sur
deben marchar. Esa es su señal para empezar a moverse. \bibverse{7} Para
convocar a la gente, soplen las trompetas normalmente, no la señal de
alarma fuerte. \bibverse{8} Los descendientes de Aarón deben tocar las
trompetas. Esta regulación seguirá vigente en todos los tiempos y para
todas las generaciones futuras.

\bibverse{9} Cuando estés en tu propia tierra y tengas que ir a la
batalla contra un enemigo que te haya atacado, toca la señal de alarma y
el Señor tu Dios no te olvidará: te salvará de tus enemigos.
\bibverse{10} Toquen las trompetas cuando celebren también, en sus
fiestas regulares y al principio de cada mes. Es decir, cuando traigas
tus holocaustos y tus ofrendas de comunión que serán como un
recordatorio para ti ante tu Dios. Yo soy el Señor tu Dios.''

\bibverse{11} ntonces la nube se levantó del Tabernáculo del Testimonio
el vigésimo día del segundo mes del segundo año. \bibverse{12} Los
israelitas abandonaron el desierto del Sinaí y se desplazaron de un
lugar a otro hasta que la nube se detuvo en el desierto de Parán.
\bibverse{13} Esta fue la primera vez que salieron siguiendo el mandato
del Señor a través de Moisés.

\bibverse{14} Las divisiones de latribu de Judá fueron las primeras en
marchar bajo su bandera, con Naasón, hijo de Aminadab, al mando.
\bibverse{15} Natanael, hijo de Zuar, estaba a cargo de la tribu de
Isacar, \bibverse{16} y Eliab, hijo de Helón, estaba a cargo de la
división tribal de Zabulón. \bibverse{17} Entonces el Tabernáculo fue
desmontado, y los guersonitas y los meraritas que lo llevaban se
pusieron en marcha.

\bibverse{18} Luego vinieron las divisiones de la tribu de Rubén,
quienes marcharon bajo su bandera, con Elisur, hijo de Sedeur, a cargo.
\bibverse{19} Selumiel, hijo de Zurishaddai, estaba a cargo de la tribu
de Simeón, \bibverse{20} y Eliasaf, hijo de Deuel, estaba a cargo de la
tribu de Gad. \bibverse{21} Entonces los coatitas se pusieron en marcha,
llevando los objetos sagrados. El tabernáculo se colocaría antes de que
llegaran.

\bibverse{22} Luego vinieron las divisiones de la tribu de Efraín, y
marcharon bajo su bandera, con Elisama, hijo de Amihud a cargo.
\bibverse{23} Gamaliel, hijo de Pedasur, estaba a cargo de la tribu de
Manasés, \bibverse{24} y Abidán, hijo de Gedeón, estaba a cargo de la
tribu de Benjamín.

\bibverse{25} Finalmente llegaron las divisiones de Dan que marcharon
bajo su bandera, defendiendo la retaguardia de todos los grupos
tribales, con Ahiezer, hijo de Amisadai, a cargo. \bibverse{26} Pagiel,
hijo de Ocrán, estaba a cargo de la tribu de Aser, \bibverse{27} y
Ajirá, hijo de Enán, estaba a cargo de la tribu de Neftalí.
\bibverse{28} Este era el orden en el se desplazaban las tribus de
Israel.

\bibverse{29} Moisés le explicó a Hobab, el hijo del suegro de Moisés,
Reuel, el madianita,\footnote{10.29 Esto convertía a Hobab en el cuñado
  de Moisés.}``Nos vamos al lugar que el Señor prometió diciendo: `Te
daré esta tierra'. Ven con nosotros y seremos buenos contigo, porque el
Señor le ha prometido cosas buenas a Israel.''

\bibverse{30} No, no me iré, volveré a mi país y a mi pueblo'',
respondió Hobab.

\bibverse{31} Por favor, no nos abandones ahora'', le dijo Moisés,
``porque tú eres el único que sabe dónde debemos acampar en el desierto
y puedes guiarnos. \bibverse{32} Si vienes con nosotros, todo lo bueno
que el Señor nos de como bendición lo compartiremos contigo''.

\bibverse{33} e fueron de la montaña del Señor para hacer un viaje de
tres días, y El Arca del Pacto del Señor les mostró el camino durante
estos tres días para encontrar un lugar para acampar. \bibverse{34} La
nube del Señor estuvo sobre ellos durante el día mientras se alejaban
del campamento.

\bibverse{35} Cada vez que el Arca avanzaba, Moisés gritaba:
``Levántate, Señor, y que tus enemigos se dispersen, y que los que te
odian huyan de ti''.

\bibverse{36} cada vez que se detenía, Moisés gritaba: ``Vuelve, Señor,
a los miles y miles del pueblo de Israel.''

\hypertarget{section-10}{%
\section{11}\label{section-10}}

\bibverse{1} No pasó mucho tiempo antes de que la gente empezara a
quejarse de lo mucho que estaban sufriendo. Cuando el Señor escuchó lo
que decían, se enfadó. El fuego del Señor los quemó, destruyendo algunos
que iban por los extremos del campamento. \bibverse{2} l pueblo clamó a
Moisés por ayuda. Entones él oró al Señor y el fuego se apagó.
\bibverse{3} Ese lugar se llamó Taberá,\footnote{11.3 ``Taberá''
  significa ``arder.''} porque el fuego del Señor los quemó.

\bibverse{4} Entonces un grupo de alborotadores\footnote{11.4
  Generalmente asociado con una ``multitud mixta'' que salió de Egipto
  con los Israelitas (ver Éxodo 12:38)}que estaba entre ellos tenían
antojos de comida tan intensos que afectaron a los israelitas que
empezaron a llorar de nuevo, preguntando ``¿Quién va a conseguirnos algo
de carne para comer? \bibverse{5} Recuerden todo el pescado que comíamos
en Egipto y que no nos costaba nada, así como los pepinos, los melones,
los puerros, las cebollas y el ajo. \bibverse{6} ¡Pero ahora nos estamos
desvaneciendo aquí! ¡Lo único que vemos es este maná!''

\bibverse{7} El maná tenía la apariencia de semillas de cilantro, de
color claro como la resina. \bibverse{8} l pueblo salía a recogerlo, lo
molíancon un molino o lo trituraban en un mortero;luego lo hervirían en
una olla y lo convertirían en pan plano. EL sabor era como de pasteles
hechos con el mejor aceite de oliva. \bibverse{9} Cuando el rocío
descendíasobre el campamento por la noche, el maná bajaba con él.

\bibverse{10} Moisés escuchó a todas las familias llorando a la entrada
de sus tiendas. El Señor se enfadó mucho, y Moisés también se enfadó.
\bibverse{11} Le preguntó al Señor: ``¿Por qué me has puesto las cosas
tan difíciles a mí, tu siervo? ¿Por qué estás tan descontento conmigo
que me has puesto la pesada responsabilidad de toda esta gente?
\bibverse{12} ¿Acaso son mis hijos? ¿Los di a luz para que me dijeras:
``Sujétalos en tu pecho como una nodriza que lleva un bebé''y luego
tener que llevarlos a la tierra que les prometiste a sus antepasados?
\bibverse{13} ¿De dónde se supone que voy a sacar carne para todos
ellos? Se siguen quejando de mí, ``¡Consíguenos algo de carne para
comer!'' \bibverse{14} No puedo seguir soportando a todo este pueblo yo
solo.¡Es demasiado! \bibverse{15} Si esta es la forma en que me vas a
tratar, entonces por favor mátame ahora para no tener que enfrentarme a
esta depresión que me abruma. Por favor, concédeme esta petición''.

\bibverse{16} Entonces el Señor le dijo a Moisés: ``Trae ante mí setenta
ancianos israelitas que sepas que son respetados como líderes por el
pueblo. Llévalos al Tabernáculo de Reunión. Se quedarán allí contigo.
\bibverse{17} Yo bajaré y hablaré contigo allí. Tomaré un poco del
Espíritu que tienes y se lo daré. Ellos tomarán parte de la
responsabilidad del pueblo para que no tengas que soportarlo todo tú
solo.

\bibverse{18} Dileal pueblo: `Purifíquense, porque mañana tendrán carne
para comer, pues se han quejado y el Señor ha oído su petición: `¿Quién
nos va a dar carne para comer? Estábamos mejor en Egipto'. Así que el
Señor va a proveerles carne para comer. \bibverse{19} La comerán, no
sólo por un día o dos, ni por cinco, diez o veinte días. \bibverse{20}
La comerán durante un mes entero hasta que vomiten y les salga por las
narices, porque han rechazado al Señor, que está aquí con ustedes, y se
han quejado de él diciendo: ``¿Por qué se nos ocurrió salir de Egipto?''

\bibverse{21} Pero Moisés respondió: ``Estoy aquí con 600.000 personas y
me dices: `Les voy a dar carne y la comerán durante un mes'?
\bibverse{22} ún si todos nuestros rebaños y manadas fueran
sacrificados, no sería suficiente para ellos. Incluso si todos los peces
del mar fueran capturados, ¡no sería suficiente para todos ellos!''

\bibverse{23} ¿No tiene el Señor el poder de hacer eso?'', respondió el
Señor. ``¡Ahora vas a ver con tus ojos si lo que he dicho sucederá o
no!''

\bibverse{24} ntonces Moisés fue y compartió con el pueblo lo que el
Señor dijo. Convocó a setenta ancianos del pueblo y los hizo ponerse de
pie alrededor de la tienda. \bibverse{25} Entonces el Señor descendió y
le habló. El Señor tomó algo del Espíritu que Moisés tenía y se lo dio.
Ellos profetizaron, pero esto no volvió a suceder.

\bibverse{26} Sin embargo, dos hombres llamados Eldad y Medad se habían
quedado en el campamento, y el Espíritu vino sobre ellos también.
(Habían sido puestos en la lista de los setenta ancianos, pero no habían
ido a la tienda. Pero profetizaron donde estaban en el campamento de
todos modos). \bibverse{27} Un joven corrió y le dijo a Moisés: ``Eldad
y Medad están profetizando en el campamento''.

\bibverse{28} Josué, hijo de Nun, que había sido asistente de Moisés
desde joven, reaccionó diciendo: ``¡Moisés, mi señor, tienes que
detenerlos!''

\bibverse{29} ¿Estás celoso de mi reputación?''respondió Moisés.
``¡Deseo que cada uno en el pueblo del Señor sea profeta y que el Señor
les dé su espíritu a todos!'' \bibverse{30} Entonces Moisés volvió al
campamento con los ancianos de Israel.

\bibverse{31} el Señor envió un viento que sopló codornices desde el mar
y las hizo caer cerca del campamento. Cubrieron el suelo hasta una
profundidad de unos dos codos y se extendieron durante un día de viaje
en todas direcciones del campamento. \bibverse{32} Durante todo ese día
y noche, y durante todo el día siguiente, el pueblo siguió recogiendo
codornices. Todos recolectaron al menos diez homers,\footnote{11.32
  Estimado en un volumen de 220 litros.} y las repartieron por todo el
campamento.

\bibverse{33} Pero mientras la gente seguía mordiendo la carne, incluso
antes de que la masticaran, el Señor mostró su ardiente ira contra
ellos, matando a algunos de ellos con una grave enfermedad.
\bibverse{34} Llamaron a ese lugar Quibrot-Hatavá,\footnote{11.34 Que
  significa: ``sepulturas de glotonería.''} porque allí enterraron a la
gente que tenía estos intensos antojos de comida.

\bibverse{35} Luego se trasladaron de Quibrot-HataváhaciaJazerot, donde
permanecieron durante algún tiempo.

\hypertarget{section-11}{%
\section{12}\label{section-11}}

\bibverse{1} Pero Miriam y Aarón criticaban a Moisés por su esposa,pues
Moisés se había casado con una mujer etíope.\footnote{12.1 ``Etíope'':
  literalmente, ``Cusita,'' refiriéndose a la tierra que quedaba en el
  sureste de Egipto.} \bibverse{2} ''¿Acaso el Señor solo habla a través
de Moisés?'', cuestionaban. ``¿No habla también a través de nosotros?''
Y el Señor escuchó todo esto.

\bibverse{3} Moisés era un hombre muy humilde, más que nadie en la
tierra.

\bibverse{4} De repente el Señor llamó a Moisés, Aarón y Miriam,
diciéndoles: ``Ustedes tres, vengan al Tabernáculo de Reunión.''Y los
tres lo hicieron.

\bibverse{5} El Señor bajó en una columna de nube y se paró en la
entrada de la Tienda. Llamó a Aarón y a Miriam y ellos se adelantaron.
\bibverse{6} Escuchen mis palabras, les dijo. Si tuvieran profetas, yo,
el Señor, me revelaría a ellos en visiones; me comunicaría con ellos en
sueños. \bibverse{7} Pero no es así con mi siervo Moisés, que de todo mi
pueblo es el que me es fiel. \bibverse{8} Yo hablo con él personalmente,
cara a cara. Hablo claramente, y no con acertijos. Él ve la semejanza
del Señor. Entonces, ¿por qué no tuvieron miedo al criticar a mi siervo
Moisés?'' \bibverse{9} Entonces el Señor se enfadó con ellos, y se fue.

\bibverse{10} Cuando la nube se elevó sobre la Tienda, la piel de Miriam
se volvió repentinamente blanca por la lepra. Aarón se volvió a mirar y
vio que tenía lepra. \bibverse{11} le dijo a Moisés: ``Señor mío, por
favor no nos castigues por este pecado que hemos cometido tan
estúpidamente. \bibverse{12} Por favor, no dejes que me convierta en un
moribundo cuya carne ya se está pudriendo cuando nace!''

\bibverse{13} Moisés clamó al Señor: ``¡Dios, por favor, cúrala!''

\bibverse{14} Pero el Señor le respondió a Moisés: ``Si su padre le
hubiera escupido en la cara, ¿no habría sido deshonrosa durante siete
días? Mantenla aislada fuera del campamento durante siete días, y luego
podráregresar''.

\bibverse{15} Miriam quedó en aislamiento fuera del campamento durante
siete días, y el pueblo no avanzó hasta que fue llevada de vuelta.
\bibverse{16} Entonces el pueblo se fue de Jazerot y se instaló en el
desierto de Parán.

\hypertarget{section-12}{%
\section{13}\label{section-12}}

\bibverse{1} El Señor le dijo a Moisés, \bibverse{2} Envía algunos
hombres a explorar la tierra de Canaán, el país que le doy a los
israelitas. Escoge a uno de los líderes de cada una de las tribus para
que vaya y haga esto''.

\bibverse{3} Moisés hizo lo que el Señor le había ordenado y envió a los
hombres desde el desierto de Parán. Todos ellos eran líderes de los
israelitas. \bibverse{4} Sus nombres eran:

Samúa hijo de Zacur, de la tribu de Rubén.

\bibverse{5} Safat, hijo de Hori, de la tribu de Simeón.

\bibverse{6} Caleb, hijo de Jefone, de la tribu de Judá.

\bibverse{7} Igal, hijo de José, de la tribu de Isacar.

\bibverse{8} Oseas,\footnote{13.8 También llamado Josué. Ver el
  versículo 16.}hijo de Nun, de la tribu de Efraín.

\bibverse{9} Palti hijo de Raphu, de la tribu de Benjamin.

\bibverse{10} Gaddiel, hijo de Sodi, de la tribu de Zabulón.

\bibverse{11} Gaddi, hijo de Susi, de la tribu de Manasés (una tribu de
José).

\bibverse{12} Amiel, hijo de Gemalli, de la tribu de Dan.

\bibverse{13} Sethur, hijo de Miguel, de la tribu de Aser.

\bibverse{14} Nahbi, hijo de Vophsi, de la tribu de Neftalí.

\bibverse{15} Geuel, hijo de Machi, de la tribu de Gad.

\bibverse{16} Estos eran los nombres de los hombres que Moisés envió a
explorar el país. Moisés le puso por nombre Josué aOseas.

\bibverse{17} Moisés los envió a explorar la tierra de Canaán,
diciéndoles: ``Pasen por el Néguev y entren en las montañas.
\bibverse{18} Vean cómo es el lugar, y averigüen acercade la gente que
vive allí, ¿son fuertes o débiles? ¿Son muchos o pocos? \bibverse{19}
¿La tierra donde viven es buena o mala? ¿Son sus ciudades como campos
abiertos, o tienen muros defensivos? \bibverse{20} ¿Es el suelo
productivo o no? ¿Es forestal? Sean valientes y traigan algunos de los
frutos del país''. (Era el comienzo de la vendimia.)

\bibverse{21} Así que los hombres fueron y exploraron la tierra desde el
desierto de Zin hasta Rejob, el oso Lebó Jamat. \bibverse{22}
Atravesaron el Néguev y llegaron a Hebrón donde vivían Ahiman, Seshai y
Talmai, los descendientes de Anac. Esta ciudad fue construida siete años
antes que la ciudad egipcia de Zoán.

\bibverse{23} Cuando llegaron al Valle de Escol cortaron una rama que
tenía un solo racimo de uvas. Tenían que cargarla en un palo sostenido
entre dos hombres. También recogieron algunas granadas e higos.
\bibverse{24} (El lugar fue llamado el Valle de Escol\footnote{13.24
  ``Escol'' significa ``manojo.''} por el racimo de uvas que tomaron de
allí.)

\bibverse{25} Cuarenta días después los hombres regresaron de explorar
el país. \bibverse{26} Fueron a ver a Moisés y Aarón, y todos los
israelitas se reunieron allí en su campamento en Cades, en el desierto
de Parán. Dieron un informe ante todos y les mostraron los frutos que
habían traído del país.

\bibverse{27} Este es el informe que dieron a Moisés: ``Fuimos y
exploramos el país al que nos enviaste, y es definitivamente muy
productivo, como si fluyera leche y miel. ¡Miren algunas de sus frutas!
\bibverse{28} Pero la gente que vive allí es fuerte, y sus pueblos son
grandes y tienen muros defensivos. También vimos algunos descendientes
de Anac allí. \bibverse{29} Los amalecitas viven en el Néguev. Los
hititas, jebuseos y amorreos viven en las colinas. Los cananeos viven en
la costa del mar y también al lado del Jordán.''

\bibverse{30} Entonces Caleb pidió silencio mientras la gente se paraba
delante de Moisés y les decía: ``Vamos a tomar la tierra. Podemos
conquistar el país, ¡sin duda!''

\bibverse{31} Pero los hombres que habían ido con él no estaban de
acuerdo. ``¡No podemos ir a luchar contra este pueblo! ¡Son mucho más
fuertes que nosotros!'' \bibverse{32} difundieron un informe negativo
entre los israelitas sobre el país que habían explorado. Le dijeron al
pueblo: ``El país que exploramos destruye a la gente que vive allí.
Además todas las personas que vimos eran muy grandes! \bibverse{33}
Incluso vimos gigantes allí, ¡son descendientes del gigante Anac!
Comparados con ellos pareceríamos saltamontes, ¡y así debimos parecerles
a ellos también!''

\hypertarget{section-13}{%
\section{14}\label{section-13}}

\bibverse{1} Entonces todos los que estaban allí gritaron toda la noche.
\bibverse{2} Todos los israelitas fueron y se quejaron a Moisés y Aarón,
diciéndoles: ``¡Ojalá hubiéramos muerto en Egipto, o aquí en este
desierto! \bibverse{3} ¿Por qué el Señor nos lleva a este país sólo para
que nos maten? ¡Nuestras esposas e hijos serán capturados y llevados
como esclavos! ¿No sería mejor que volviéramos a Egipto?''

\bibverse{4} Se dijeron unos a otros: ``Elijamos un nuevo líder y
volvamos a Egipto''.

\bibverse{5} Moisés y Aarón se postraron en el suelo frente a todos los
israelitas reunidos. \bibverse{6} Josué, hijo de Nun, y Caleb, hijo de
Jefone, estaban allí. Habían sido parte del grupo que había ido a espiar
la tierra. Se rasgaron la ropa,\footnote{14.6 En señal de duelo y
  emoción intensa.} \bibverse{7} y les dijeron a los israelitas: ``El
país que viajamos y exploramos tiene muy buena tierra. \bibverse{8} Si
el Señor está contento con nosotros, nos llevará allí y nos la dará, una
tierra tan productiva que es como si fluyera leche y miel. \bibverse{9}
No se rebelen ni luchen contra el Señor. No hay que tener miedo de la
gente que vive en el campo, ¡podemos cogerlos fácilmente! Están
indefensos y el Señor está con nosotros. ¡No les tengan miedo!''

\bibverse{10} En respuesta, todo el pueblo gritó: ``¡Apedréenlos!'' Pero
la gloria del Señor apareció de repente en el Tabernáculo de Reunión,
justo en medio de los israelitas.

\bibverse{11} El Señor le dijo a Moisés: ``¿Hasta cuándo me va a
rechazar este pueblo? ¿Cuánto tiempo va a rechazar esta gente la
confianza en mí, a pesar de todos los milagros que he hecho delante de
ellos? \bibverse{12} Voy a enfermarlos con una enfermedad y matarlos.
Entonces los convertiré en una nación más grande y fuerte que ellos''.

\bibverse{13} Pero Moisés le dijo al Señor: ``¡Los egipcios se enterarán
de esto! Fue por tu poder que sacaste a los israelitas de entre ellos.
\bibverse{14} Ellos le contarán todo al pueblo que vive en este país. Ya
han oído que tú, Señor, estás con nosotros los israelitas, que tú,
Señor, te muestras cara a cara, que tu nube los vigila, y que los
conduces por una columna de nube durante el día y una columna de fuego
por la noche. \bibverse{15} Si matas a toda esta gente de una sola vez,
las naciones que han oído hablar de ti dirán: \bibverse{16} El Señor
mató a este pueblo en el desierto porque no pudo llevarlos al país que
prometió darles. Los ha matado a todos en el desierto''.

\bibverse{17} Ahora, Señor, por favor demuestra el alcance de tu poder
tal como lo has dicho: \bibverse{18} El Señor es lento para enojarse y
está lleno de amor confiable, perdonando el pecado y la rebelión. Sin
embargo, no permitirá que los culpables queden impunes, trayendo las
consecuencias del pecado de los padres a sus hijos, nietos y bisnietos.
\bibverse{19} Por favor, perdona el pecado de estas personas ya que tu
amor digno de confianza es tan grande, de la misma manera que los has
perdonado desde que salieron de Egipto hasta ahora''.

\bibverse{20} Los he perdonado como me lo pediste'', respondió el Señor.
\bibverse{21} ``Pero aún así, juro por mi vida y toda mi gloria en la
tierra, \bibverse{22} ue ni uno solo de los que vieron mi gloria y los
milagros que hice en Egipto y en el desierto, sino que me provocaron y
se negaron a obedecerme una y otra vez; \footnote{14.22 ``Una y otra
  vez'': literalmente, ``diez veces,'' pero se cree que es una expresión
  que se refiere a múltiples ocasiones.} \bibverse{23} ni uno solo de
ellos va a ver el país que prometí a sus antepasados. Ninguno de los que
me rechazaron lo verá.

\bibverse{24} Pero como mi siervo Caleb tiene un espíritu totalmente
diferente y está totalmente comprometido conmigo, lo llevaré al país que
visitó, y sus descendientes serán los dueños. \bibverse{25} Como los
amalecitas y los cananeos viven en los valles, mañana deberán dar la
vuelta y volver al desierto, tomando la ruta hacia el Mar Rojo''.

\bibverse{26} El Señor le dijo a Moisés y Aarón, \bibverse{27} ¿Cuánto
tiempo más me van a criticar estos malvados? Ya he oído lo que dicen,
haciendo quejas en mi contra. \bibverse{28} Ve y diles: ``Juro por mi
propia vida, declara el Señor, que cumpliré sus deseos, ¡créanme!
\bibverse{29} Todos ustedes morirán en este desierto, todos los que
fueron registrados en el censo que contó a los mayores de veinte años, y
será porque se quejaron contra mi. \bibverse{30} Ninguno de ustedes
entrará en el país que prometí darles, excepto Caleb, hijo de Jefone, y
Josué, hijo de Nun. \bibverse{31} Sin embargo, me llevaré a sus hijos,
los que dijeron que serían llevados como botín, al país que ustedes
rechazaron, y ellos sí lo apreciarán. \bibverse{32} Pero todos ustedes
van a morir en este desierto. \bibverse{33} us hijos vagarán por el
desierto durante cuarenta años, sufriendo por su falta de confianza,
hasta que todos sus cuerpos estén enterrados en el desierto.

\bibverse{34} Así como han explorado el país durante cuarenta días, su
castigo por sus pecados será de cuarenta años, un año por cada día, y
verán lo que ocurre cuando me opongo a ustedes. \bibverse{35} ¡Yo, el
Señor, así lo he dicho! Verán por ustedes mismos que haré esto con estos
malvados israelitas que se han unido para oponerse a mí. Sus vidas
acabarán en el desierto, y morirán allí''.

\bibverse{36} Los hombres que Moisés había enviado a explorar el país --
los que regresaron y porque dieron un mal informe hicieron que todos los
israelitas se quejaran contra el Señor -- \bibverse{37} los hombres que
dieron el mal informe murieron ante el Señor de la peste. \bibverse{38}
Los únicos que vivieron fueron Josué hijo de Nun y Caleb hijo de Jefone
de los que fueron a explorar el país.

\bibverse{39} Cuando Moisés dijo a los israelitas lo que el Señor había
dicho estaban muy, muy tristes. \bibverse{40} Se levantaron temprano a
la mañana siguiente planeando ir a las colinas. ``Sí, realmente
pecamos'', dijeron, ``pero ahora estamos aquí e iremos donde el Señor
nos dijo''.

\bibverse{41} Pero Moisés se opuso. ``¿Por qué desobedecen la orden del
Señor? ¡No tendrán éxito en su plan! \bibverse{42} No intenten irse,
porque sus enemigos los matarán, pues el Señor no está con ustedes.
\bibverse{43} Los amalecitas y cananeos que viven allí los atacarán, y
morirán por espada. Porque rechazaron al Señor, y no les ayudará''.

\bibverse{44} ero ellos fueron arrogantes y subieron a las colinas,
aunque Moisés y el Arca del Pacto del Señor no se movieron del
campamento. \bibverse{45} los amalecitas y cananeos que vivían allí en
las colinas bajaron y atacaron a esos israelitas y los derrotaron, y los
persiguieron hasta Jormá.

\hypertarget{section-14}{%
\section{15}\label{section-14}}

\bibverse{1} Entonces el Señor le dijo a Moisés: \bibverse{2} Dile a los
israelitas, 'Estas son la instrucciones sobre lo que deben hacer una vez
que lleguen al país que les doy para vivir: \bibverse{3} Cuando traiga
una ofrenda al Señor de tu ganado o rebaño (ya sea un holocausto, un
sacrificio para cumplir una promesa que hiciste, o una ofrenda de libre
albedrío o de fiesta) que sea aceptable para el Señor, \bibverse{4}
entonces también presentarás una ofrenda de grano de una décima parte de
un efa de la mejor harina mezclada con un cuarto de hin de aceite de
oliva. \bibverse{5} Añade un cuarto de hin de vino como ofrenda de
bebida al holocausto o al sacrificio de un cordero.

\bibverse{6} Cuando se trate de un carnero, presenta una ofrenda de
grano de dos décimas de efa de la mejor harina mezclada con un tercio de
hin de aceite de oliva, \bibverse{7} junto con un tercio de hin de vino
como ofrenda de bebida, todo ello para ser aceptable al Señor.

\bibverse{8} Cuando traigas un novillo como holocausto o sacrificio para
cumplir una promesa que hiciste o como ofrenda de paz al Señor,
\bibverse{9} también llevarás con el novillo una ofrenda de grano de
tres décimas de efa de la mejor harina mezclada con medio hin de aceite
de oliva. \bibverse{10} Añade medio hin de vino como ofrenda de bebida.
Todo esto es una ofrenda para ser aceptable al Señor.

\bibverse{11} Esto debe hacerse por cada toro, carnero, cordero o cabra
que se traiga como ofrenda.\footnote{15.11 ``Que se traiga como
  ofrenda'': añadido para mayor claridad.} \bibverse{12} Esto es lo que
tienes que hacer para cada uno, sin importar cuántos sean. \bibverse{13}
Todo israelita debe seguir estas instrucciones cuando presente una
ofrenda que sea aceptable para el Señor. \bibverse{14} Esto también se
aplica a todas las generaciones futuras que si un extranjero que vive
entre ustedes o cualquier otra persona entre ustedes desea presentar una
ofrenda aceptable para el Señor: deben hacer exactamente lo que ustedes
hacen. \bibverse{15} Toda la congregación debe tener las mismas reglas
para ustedes y para el extranjero que vive entre ustedes. Esta es una
ley permanente para todas las generaciones futuras. Tú y el extranjero
deben ser tratados de la misma manera ante la ley. \bibverse{16} Las
mismas reglas y normas se aplican a ustedes y al extranjero que vive
entre ustedes.''

\bibverse{17} El Señor le dijo a Moisés: \bibverse{18} Diles a los
israelitas: 'Cuando lleguen al país al que yo los llevo \bibverse{19} y
cománde los alimentos que allí se producen, darán parte de ellos como
ofrenda al Señor. \bibverse{20} Darán como ofrenda una parte de la
harina que usen para hacer los panes, y la presentarás como una ofrenda
de la era. \bibverse{21} Para todas las generaciones futuras, darás al
Señor una ofrenda de la primera de tus harinas.

\bibverse{22} Ahora bien, si pecan colectivamente sin querer y no siguen
todas estas instrucciones que el Señor ha dado a Moisés, \bibverse{23} s
decir, todo lo que el Señor les ha ordenado hacer a través de Moisés
desde el momento en que el Señor les dio y para todas las generaciones
futuras, \bibverse{24} y si se hizo sin querer y sin que todos lo
supieran, entonces toda la congregación debe presentar un novillo como
holocausto para ser aceptado por el Señor, junto con su ofrenda de grano
y su libación presentada según las reglas, así como un macho cabrío como
ofrenda por el pecado. \bibverse{25} De esta manera el sacerdote debe
hacer que toda la congregación de Israel esté bien con el Señor para que
puedan ser perdonados, porque el pecado fue involuntario y han
presentado al Señor un holocausto y una ofrenda por el pecado, ofrecida
ante el Señor por su pecado involuntario. \bibverse{26} Entonces toda la
congregación de Israel y los extranjeros que viven entre ellos serán
perdonados, porque el pueblo pecó sin intención.

\bibverse{27} En el caso de un individuo que peca sin intención, deben
presentar una cabra hembra de un año como ofrenda por el pecado.
\bibverse{28} El sacerdote hará que la persona que pecó sin querer esté
en su derecho ante el Señor. Una vez que hayan sido expiados, serán
perdonados. \bibverse{29} Aplicarás la misma ley para el que peca por
error a un israelita o a un extranjero que viva entre ustedes.

\bibverse{30} Pero la persona que peca a manera de desafío, ya sea un
israelita o un extranjero, está blasfemando\footnote{15.30
  ``Blasfemando'': en el sentido de abusar deliberadamente del Señor.}
al Señor. Serán expulsados de su pueblo. \bibverse{31} Deben ser
expulsados, porque han tratado la palabra del Señor con desprecio y han
quebrantado su mandamiento. Son responsables de las consecuencias de su
propia culpa''.

\bibverse{32} Durante el tiempo en que los israelitas vagaban por el
desierto, un hombre fue sorprendido recogiendo leña en el día Sábado.
\bibverse{33} as personas que lo encontraron recogiendo leña lo llevaron
ante Moisés, Aarón y el resto de los israelitas. \bibverse{34} Lo
pusieron bajo vigilancia porque no estaba claro qué le iba a pasar.
\bibverse{35} El Señor le dijo a Moisés: ``Este hombre tiene que ser
ejecutado. Todos los israelitas deben apedrearlo fuera del campamento''.
\bibverse{36} Así que todos tomaron al hombre fuera del campamento y lo
apedrearon hasta la muerte como el Señor había ordenado a Moisés.

\bibverse{37} Poco después el Señor le dijo a Moisés: \bibverse{38}
Diles a los israelitas que para todas las generaciones futuras harán
borlas para los dobladillos de tu ropa y deberán atarlas con un cordón
azul. \bibverse{39} Cuando miren estas borlas recordarán que deben
guardar todos los mandamientos del Señor y que no sean infieles,
siguiendo sus propios pensamientos y deseos. \bibverse{40} De esta
manera serecordarán que deben guardar todos mis mandamientos y serán
santos para Dios. Yo soy el Señor su Dios que los sacó de Egipto para
ser su Dios. \bibverse{41} ¡Yo soy el Señor su Dios!''

\hypertarget{section-15}{%
\section{16}\label{section-15}}

\bibverse{1} Coré, \footnote{16.1 Coré era primo de Moisés y Aarón, y el
  celo por su posición pudo haber sido la causa de su rebelión.}hijo de
Izhar, hijo de Coat, hijo de Levi, trató de asumir el liderazgo, junto
con Datán y Abiram, hijos de Eliab, y On, hijo de Pelet, que eran de la
tribu de Rubén. \bibverse{2} stos se rebelaron contra Moisés, y se les
unieron 250 respetados líderes israelitas y miembros de la asamblea.
\bibverse{3} Se unieron en oposición a Moisés y Aarón, diciéndoles:
``¡Ustedes se han adueñado del poder! Cada uno de los israelitas es
santo, y el Señor está entre ellos. Entonces, ¿por qué se ponen ustedes
por encima de la asamblea del Señor?''

\bibverse{4} Cuando Moisés oyó lo que decían, cayó al suelo boca abajo.
\bibverse{5} Entonces le dijo a Coré y a todos los que estaban con él:
``Por la mañana el Señor va a demostrar quién es suyo y quién es santo,
y permitirá que esa persona se acerque a él. Sólo permitirá que se
acerque a él quien él elija. \bibverse{6} Esto es lo que tú, Coré, y
todos los que están contigo van a hacer. Toma unos quemadores de
incienso, \bibverse{7} y mañana pon incienso en ellos y enciéndelo en la
presencia del Señor. Entonces el hombre que el Señor elija es el que es
santo. ¡Son ustedes, los levitas, los que están tomando demasiado poder
para ustedes mismos!''

\bibverse{8} Moisés también le dijo a Coré: ``¡Escuchen, levitas!
\bibverse{9} ¿Les parece poco que el Dios de Israel los haya elegido
entre todos los demás israelitas y les haya permitido acercarse a él y
realizar la obra en el Tabernáculo del Señor, estar ante los israelitas
y servirles? \bibverse{10} estedado el privilegio de acercarte a él, a
ti, Coré, a y a todos los demás levitas, ¡pero ahora también quieren
tener el sacerdocio! \bibverse{11} Así que en realidad tú y los que se
han unido a ti están luchando contra el Señor, porque ¿quién es Aarón
para que murmurencontra él?''

\bibverse{12} ntonces Moisés convocó a Datán y a Abiram, los hijos de
Eliab, pero ellos respondieron: ``No vamos a comparecer ante
ustedes!\footnote{16.12 En otras palabras, se negaron a reconocer la
  autoridad de Moisés para exigirles que comparecieran ante él para ser
  juzgados.} \bibverse{13} ¿No has hecho suficiente alejándonos de una
tierra que fluye leche y miel para matarnos aquí en el desierto?
¿También tienes que hacerte un dictador y gobernante? \bibverse{14}
Además, no nos has llevado a una tierra que fluye leche y miel ni nos
has dado campos y viñedos para que los poseamos. ¿De verdad crees que
puedes engañar a todo el mundo?+ 16.14 La expresión usada aquí ``¿Le
sacarás los ojos a estos hombres?'' se entiende como algo así como
``¿Les vas a tirar de la lana en los ojos?''¡No, no asistiremos!''

\bibverse{15} Moisés se enfadó mucho y le dijo al Señor: ``No aceptes
sus ofrendas. Nunca les he quitado ni un burro ni he tratado mal a
ninguno de ellos''.

\bibverse{16} Moisés le dijo a Coré:``Tú y todos los que se han unido a
ti deben presentarse ante el Señor mañana, todos ustedes y Aarón
también. \bibverse{17} Cada uno tomará su quemador de incienso, lo
pondrá en él y lo ofrecerá ante el Señor. Los 250 usarán sus quemadores
de incienso y Aarón también''.

\bibverse{18} Entonces cada uno tomó su incensario, puso incienso en él,
lo encendió, y se paró junto con Moisés y Aarón a la entrada del
Tabernáculo de Reunión. \bibverse{19} Cuando Coré reunió a todo su grupo
rebelde a la entrada del Tabernáculo de Reunión, la gloria del Señor
apareció ante toda la congregación.

\bibverse{20} El Señor dijo a Moisés y Aarón, \bibverse{21} Apártense de
estos israelitas y los destruiré enseguida.''

\bibverse{22} Pero Moisés y Aarón cayeron al suelo boca abajo y dijeron:
``Dios -- Diosde todo lo que vive --si es un solo hombre el que peca,
¿tienes que enfadarte con todos?''

\bibverse{23} Entonces el Señor le dijo a Moisés: \bibverse{24} Dile al
pueblo que se aleje de las casas de Coré, Datán y Abiram''.

\bibverse{25} Entonces Moisés se acercó a Dathan y Abiram, y los
ancianos israelitas de Israel le siguieron. \bibverse{26} ordenó al
pueblo: ``Apártense de las tiendas de estos malvados y no toquen nada
que les pertenezca, de lo contrario serán destruidos junto con ellos en
todos sus pecados''.

\bibverse{27} El pueblo se alejó de las casas de Coré, Datán y Abiram.
Dathan y Abiram salieron y se pararon en las entradas de sus tiendas
junto con sus esposas, hijos y pequeños.

\bibverse{28} Moisés dijo: ``Así es como sabrán que el Señor me envió
para llevar a cabo todo lo que he hecho, porque no fue nada que surgiera
de mi pernsamiento.\footnote{16.28 ``Nada que surgiera de mi
  pensamiento'': literalmente, ``no salió de mi mente,'' porque se creía
  que el corazón era el lugar donde se generaban los pensamientos.}
\bibverse{29} Si estos hombres mueren de muerte natural, experimentando
el destino de cada ser humano, entonces el Señor no me envió.
\bibverse{30} Pero si el Señor hace algo totalmente diferente, y la
tierra se abre y se los traga junto con todo lo que les pertenece para
que bajen vivos al Seol, entonces sabrán que estos hombres han actuado
con desprecio ante Señor.''

\bibverse{31} Tan pronto como Moisés terminó de decir todo esto, la
tierra debajo de los rebeldes se abrió, \bibverse{32} y la tierra se los
tragó así como a sus hogares, y a todos los que estaban allí con Coré y
todo lo que les pertenecía. \bibverse{33} Bajaron vivos al Seol con todo
lo que tenían. La tierra se cerró sobre ellos, y ya no estaban.

\bibverse{34} Cuando oyeron sus gritos, todos los israelitas cercanos
salieron corriendo, gritando: ``¡Cuidado! La tierra podría tragarnos a
nosotros también''. \bibverse{35} ntonces fuego salió del Señor y quemó
a los 250 hombres que ofrecían incienso.

\bibverse{36} Entonces el Señor dijo a Moisés, \bibverse{37} Dile a
Eleazar, hijo del sacerdote Aarón, que recoja los incensarios sagrados
de entre los que se han quemado, y que esparza las brasas usadas para el
incienso bien lejos del campamento. \bibverse{38} Haz que los
incensarios de los que pecaron a costa de su propia vida sean
martillados en láminas de metal como cobertura para el altar, porque
fueron ofrecidos ante el Señor, y así se han hecho santos. Serán un
recordatorio para los israelitas de lo que pasó''.

\bibverse{39} Así que el sacerdote Eleazar recogió los incensarios de
bronce que usaban los quemados y los hizo martillar como cubierta para
el altar, \bibverse{40} siguiendo las instrucciones que le dio el Señor
a través de Moisés. Esto era para recordar a los israelitas que nadie
que no sea descendiente de Aarón debe venir a ofrecer incienso ante el
Señor, de lo contrario podrían terminar como Coré y los que están con
él.

\bibverse{41} Al día siguiente todos los israelitas se quejaron a Moisés
y Aarón, diciendo: ``¡Han matado al pueblo del Señor!'' \bibverse{42}
Pero cuando el pueblo se reunió para enfrentarse a ellos, Moisés y Aarón
se acercaron al Tabernáculo de Reunión, y de repente la nube lo cubrió y
apareció la gloria del Señor. \bibverse{43} Moisés y Aarón fueron y se
pararon al frente del Tabernáculo de Reunión, \bibverse{44} y el Señor
le dijo a Moisés: \bibverse{45} Aléjate de este pueblo, pues acabaré con
ellos inmediatamente.'' Moisés y Aarón cayeron al suelo boca abajo.

\bibverse{46} Moisés le dijo a Aarón: ``Pon algunas brasas del altar y
algo de incienso en tu incensario. Luego corre donde está el pueblo y
ponlos delante del Señor, porque el Señor está enojado con ellos y una
plaga ha comenzado.''

\bibverse{47} Aarón tomó el incensario tal como le había dicho Moisés y
corrió al centro de la asamblea. Vio que la peste había empezado a
afectar al pueblo, así que ofreció el incienso e hizo que el pueblo se
pusiera en pie ante el Señor. \bibverse{48} Se interpuso entre los
muertos y los vivos, y la peste se detuvo. \bibverse{49} Sin embargo,
14.700 personas murieron por la plaga además de los que murieron por
culpa de Coré.

\bibverse{50} Entonces Aarón regresó a Moisés a la entrada del
Tabernáculo de Reunión porque la plaga había sido detenida.

\hypertarget{section-16}{%
\section{17}\label{section-16}}

\bibverse{1} El Señor le dijo a Moisés: \bibverse{2} Dile a los
israelitas que traigan doce bastones, uno del líder de cada tribu.
Escriban el nombre de cada hombre en el bastón, \bibverse{3} y escriban
el nombre de Aarón en el bastón de la tribu de Leví, porque tiene que
haber un bastón para el jefe de cada tribu. \bibverse{4} Coloca los
bastones en el Tabernáculo de Reunión frente al
Testimonio\footnote{17.4 El testimonio se refería a las dos tablas de
  piedra de los Diez Mandamientos que se guardaban dentro del Arca.}donde
me encuentro contigo. \bibverse{5} El bastón que pertenece al hombre que
yo elija brotará ramas, y pondré fin a las constantes quejas de los
israelitas contra ti''.

\bibverse{6} Moisés explicó esto a los israelitas, y cada uno de sus
líderes le dio un bastón, uno para cada uno de los líderes de sus
tribus. Así que había doce bastones incluyendo el de Aarón. \bibverse{7}
Moisés colocó los bastones ante el Señor en la Tienda del Testimonio.

\bibverse{8} Al día siguiente Moisés entró en la Tienda del Testimonio y
vio que el bastón de Aarón que representaba a la tribu de Leví, había
brotado y salieron ramas de él, y estaba florecido y había producido
almendras. \bibverse{9} Moisés tomó todos los bastones de la presencia
del Señor y los mostró a todos los israelitas. Ellos los vieron, y cada
hombre recogió su propio bastón.

\bibverse{10} El Señor le dijo a Moisés: ``Pon el bastón de Aarón de
nuevo delante del Testimonio, para que lo guardes allí como un
recordatorio para advertir a cualquiera que quiera rebelarse, para que
dejen de quejarse contra mí. De lo contrario, morirán''. \bibverse{11}
Moisés hizo lo que el Señor le ordenó.

\bibverse{12} Entonces los israelitas vinieron y le dijeron a Moisés:
``¿No ves que todos vamos a morir? ¡Nos van a destruir! ¡Nos van a matar
a todos! \bibverse{13} El que se atreva a acercarse al Tabernáculo del
Señor morirá. ¿Nos van a aniquilar a todos?''

\hypertarget{section-17}{%
\section{18}\label{section-17}}

\bibverse{1} El Señor le dijo a Aarón: ``Tú y tus hijos y los otros
levitas son responsables de los pecados relacionados con el santuario.
Sólo tú y tus hijos son responsables de los pecados relacionados con su
sacerdocio. \bibverse{2} Haz que tus hermanos de la tribu de Leví, la
tribu de tu padre, se unan a ti para ayudarte a ti y a tus hijos con tu
servicio en la Tienda del Testimonio. \bibverse{3} Ellos se encargarán
de tus responsabilidades y de las relacionadas con la Tienda, pero no
deben acercarse demasiado a los objetos sagrados del santuario o del
altar, de lo contrario morirán, y tú también. \bibverse{4} Deben
ayudarte y cuidar de las responsabilidades del Tabernáculo de Reunión,
haciendo todo el trabajo en la Tienda, pero no se les permite estar
contigo durante tu ministerio sacerdotal.\footnote{18.4 ``Durante tu
  ministerio sacerdotal'': añadido para mayor claridad.}

\bibverse{5} Debes llevar a cabo las responsabilidades relacionadas con
el santuario y el altar, para que mi ira no vuelva a caer sobre los
israelitas. \bibverse{6} Mira, yo mismo he elegido a tus hermanos los
levitas de los israelitas como mi regalo para ti, dedicado al Señor para
hacer el trabajo que relaciona el Tabernáculo de Reunión. \bibverse{7}
Pero sólo tú y tus hijos son responsables de tu sacerdocio, haciendo
todo lo que concierne al altar y está detrás del velo. Sólo tú debes
realizar ese servicio. Te estoy dando el don de tu sacerdocio, pero
cualquier otro que se acerque al santuario debe ser ejecutado.''

\bibverse{8} El Señor le dijo a Aarón, ``Escucha, te he puesto a cargo
de oficiar mis ofrendas. Todas las santas contribuciones de los
israelitas que traen están reservadas para ti, y esta es una regla
permanente. \bibverse{9} Parte de las ofrendas más sagradas tomadas de
los holocaustos son tuyas. Parte de todas las ofrendas que me dan como
ofrendas sagradas, ya sean ofrendas de grano o de pecado o de culpa, esa
parte pertenece a ti y a tus hijos. \bibverse{10} Lo comerás en un lugar
santísimo.\footnote{18.10 Según lo requería la ley levítica: Ver por
  ejemplo Lev. 6:16, Lev. 16:26; Lev. 7:6.}A todo macho se le permite
comerlo. Deben considerarlo como algo sagrado.

\bibverse{11} También te pertenecen los regalos voluntarios y las
ofrendas de los israelitas. Te he dado esto a ti y a tus hijos e hijas
como una regla permanente. Todos los de tu casa que estén
ceremonialmente limpios pueden comerlo. \bibverse{12} Les doy el mejor
aceite de oliva y el mejor vino y grano que los israelitas dan como
primicias al Señor. \bibverse{13} Las primicias de todas las cosechas
que produzcan en su tierra y que traigan al Señor son tuyas. Todos los
miembros de tu familia que estén ceremonialmente limpios pueden
comerlas.

\bibverse{14} Todo lo que en Israel se dedica al Señor es tuyo.
\bibverse{15} Todo primogénito, ya sea humano o animal, que se ofrezca
al Señor es tuyo. Pero debes redimir todo primogénito y todo primogénito
de los animales inmundos. \bibverse{16} Cuando tengan un mes de edad,
pagarás el precio de redención de cinco siclos de plata (usando el
estándar de siclos del santuario), equivalente a veinte gueras.

\bibverse{17} Pero no se te permitirá redimir al primogénito de un buey,
una oveja o una cabra porque son sagrados. Esparcirás su sangre sobre el
altar y quemarás su grasa como holocausto aceptado por el Señor.
\bibverse{18} Su carne es tuya, de la misma manera que el pecho y el
muslo derecho de la ofrenda ondulada son tuyos.

\bibverse{19} e doy todas las ofrendas voluntarias que los israelitas
presentan al Señor así como a tus hijos e hijas como una regla
permanente. Es un acuerdo permanente de sal\footnote{18.19 ``Acuerdo
  permanente de sal'': se refiere a un acuerdo que no puede romperse. La
  sal se usaba como conservante, y las ofrendas al Señor incluían sal
  (ver Levítico 2:13).} ante el Señor para ti y tus descendientes.''

\bibverse{20} ``No tendrás propiedades en su país, y no tendrás una
parte de sus tierras. Yo soy tu parte y tu posesión entre los
israelitas. \bibverse{21} En cambio, he dado a los levitas todos los
diezmos de Israel como compensación por el servicio que prestan al hacer
el trabajo en el Tabernáculo de Reunión.

\bibverse{22} A los israelitas ya no se les permite acercarse al
Tabernáculo de Reunión, de lo contrario cometerán una ofensa y morirán.
\bibverse{23} Los levitas deben realizar el trabajo en el Tabernáculo de
Reunión, y deben asumir la responsabilidad de los pecados que se
cometan. Esta es una regla permanente para todas las generaciones
futuras. Los levitas no recibirán una parte de la tierra entre los
israelitas. \bibverse{24} En su lugar, he dado a los levitas como
compensación el diezmo que los israelitas dan al Señor como
contribución. Por eso les dije que no recibirían una parte de la tierra
entre los israelitas''.

\bibverse{25} El Señor le dijo a Moisés: \bibverse{26} Habla con los
levitas y explícales: Cuando recibas de los israelitas el diezmo que te
he dado como compensación, debes devolver parte de él como ofrenda al
Señor: un diezmo del diezmo. \bibverse{27} Tu ofrenda será considerada
como si fueran las primicias del grano de tu era o del jugo de uva del
lagar. \bibverse{28} De este modo, deberás contribuir con una ofrenda al
Señor de cada diezmo que recibas de los israelitas, entregando la
ofrenda del Señor al sacerdote Aarón. \bibverse{29} De todas las
ofrendas que recibas debes contribuir como ofrenda del Señor con lo
mejor, la parte más sagrada de cada ofrenda.'

\bibverse{30} Así que di a los levitas, ``Cuando hayas presentado la
mejor parte, será considerada como tu contribución producida por tu
trilladora o lagar. \bibverse{31} stedes y sus familias pueden comerla
en cualquier sitio porque es la compensación por su servicio en el
Tabernáculo de Reunión. \bibverse{32} No se considerará que han pecado
si han presentado la mejor parte. Pero si tratan las sagradas ofrendas
de los israelitas con falta de respeto morirán.'''

\hypertarget{section-18}{%
\section{19}\label{section-18}}

\bibverse{1} El Señor le dijo a Moisés y Aarón: \bibverse{2} Esta es una
norma que el Señor ha ordenado, diciendo, 'Dile a los israelitas que te
traigan una vaca roja\footnote{19.2 ``Vaca'': la palabra utilizada aquí
  se traduce generalmente como ``novilla'' que en inglés se refiere a
  una joven vaca hembra que no ha tenido un ternero. Sin embargo, como
  se desprende de 1 Samuel 6:7, la palabra también se utiliza para
  describir a una vaca que ha tenido un ternero y está produciendo
  leche.} sin defectos, que nunca haya sido uncida. \bibverse{3}
Entrégasela al sacerdote Eleazar, y él la llevará fuera del campamento
ylamandará a masacrar. \bibverse{4} El sacerdote Eleazar pondrá un poco
de su sangre en su dedo y la rociará siete veces hacia la entrada del
Tabernáculo de Reunión. \bibverse{5} Luego la vaca debe ser quemada
mientras él observa. Todo debe ser quemado: su piel, carne y sangre, así
como sus excrementos. \bibverse{6} El sacerdote arrojará madera de
cedro, hisopo e hilo carmesí sobre la vaca en llamas.

\bibverse{7} Entonces el sacerdote lavará sus ropas y su cuerpo en agua,
y después podrá entrar en el campamento, pero permanecerá impuro hasta
la noche. \bibverse{8} La persona que quemó la vaca también lavará sus
ropas y su cuerpo en agua, y él también permanecerá impuro hasta la
noche.

\bibverse{9} El que esté limpio recogerá las cenizas de la vaca y las
guardará en un lugar limpio fuera del campamento. Las guardarán los
israelitas para preparar el agua de purificación que sirve para
purificar del pecado. \bibverse{10} El hombre que recogió las cenizas de
la vaca lavará también sus ropas, y permanecerá impuro hasta la noche.
Esta es una regla permanente para los israelitas y para el extranjero
que vive con ellos.

\bibverse{11} Si tocas un cadáver serás impuro durante siete días.
\bibverse{12} Debes purificarte con el agua de la purificación al tercer
día y al séptimo día, y entonces estarás limpio. Pero si no te purificas
en el tercer y séptimo día, no estarás limpio. \bibverse{13} Si tocas un
cadáver y no te purificas, harás impuro el Tabernáculo del Señor y
deberás ser expulsado de Israel. Sigues siendo impuro porque no se te ha
rociado con el agua de la purificación y tu impureza permanece.

\bibverse{14} La siguiente norma se aplica cuando una persona muere en
una tienda. Todo el que entre en la tienda y todo el que ya esté en ella
será impuro durante siete días. \bibverse{15} Cualquier recipiente
abierto que no tenga una tapa cerrada es impuro. \bibverse{16} Si estás
al aire libre y tocas a alguien que ha muerto por la espada o que ha
muerto de forma natural, o si tocas un hueso humano o una tumba,
entonces serás impuro durante siete días.

\bibverse{17} Este es el proceso para la purificación si eres impuro.
Toma algunas de las cenizas del holocausto para la purificación y ponlas
en un frasco con agua fresca. \bibverse{18} El hombre que esté limpio
tomará un hisopo y lo mojará en el agua. Luego rociará la tienda y todo
lo que haya dentro de ella, y a todos los que estuvieran allí. También
deberá rociarlo a usted si ha tocado un hueso, o una tumba, o alguien
que ha muerto o ha sido asesinado.

\bibverse{19} El hombre que está limpio debe rociarte tanto al tercer
día como al séptimo día. Después de que te purifiques al séptimo día,
debes lavar tu ropa y a ti mismo en agua, y esa noche estarás limpio.
\bibverse{20} Pero si no te purificas, serás expulsado de los
israelitas, porque has hecho impuro el Tabernáculo del Señor. El agua de
la purificación no ha sido rociada sobre ti, y sigues siendo impuro.
\bibverse{21} Esta es una regla permanente para todos. El hombre que
rocía el agua de purificación debe lavar su ropa, y cualquiera que toque
el agua de purificación será impuro hasta la noche. \bibverse{22} Todo
lo que toque la persona impura será impuro, y cualquiera que lo toque
será impuro hasta la noche.''

\hypertarget{section-19}{%
\section{20}\label{section-19}}

\bibverse{1} Fue durante el primer mes del año que todos los israelitas
llegaron al desierto de Zin y establecieron un campamento en Cades.
(Aquí fue donde Miriam murió y fue enterrada.)

\bibverse{2} Sin embargo, allí no había agua para que nadie bebiera, así
que la gente se reunió para enfrentarse a Moisés y Aarón. \bibverse{3}
Discutieron con Moisés, diciendo: ``¡Si hubiéramos muerto con nuestros
parientes en la presencia del Señor! \bibverse{4} ¿Por qué has traído al
pueblo del Señor a este desierto para que nosotros y nuestro ganado
muramos aquí? \bibverse{5} ¿Por qué nos has sacado de Egipto para venir
a este horrible lugar? Aquí no crece nada, ni grano, ni higos, ni viñas,
ni granadas. Y no hay agua para beber''.

\bibverse{6} Moisés y Aarón dejaron el pueblo y se fueron a la entrada
del Tabernáculo de Reunión. Allí cayeron boca abajo en el suelo, y la
gloria del Señor se les apareció. \bibverse{7} El Señor le dijo a
Moisés, \bibverse{8} Toma el bastón y haz que la gente se reúna a tu
alrededor. Mientras miran, tú y tu hermano Aarón ordenarán a la roca y
derramará agua. Traerán agua de la roca para que el pueblo y su ganado
puedan beber''.

\bibverse{9} Moisés recogió el bastón que estaba guardado en la
presencia del Señor, como se le había ordenado. \bibverse{10} Moisés y
Aarón hicieron que todos se reunieran frente a la roca. Moisés les dijo:
``¡Escuchen, pandilla de rebeldes! ¿Tenemos que sacar agua de esta roca
para ustedes?'' \bibverse{11} Entonces Moisés tomó el bastón y golpeó la
roca dos veces. Salieron chorros de agua para que la gente y su ganado
pudieran beber.

\bibverse{12} Pero el Señor les dijo a Moisés y a Aarón: ``Como no
confiaron en mí lo suficiente para demostrar lo santo que soy a los
israelitas, no serán ustedes los que los lleven al país que les he
dado''. \bibverse{13} El lugar donde los israelitas discutían con el
Señor se llamaba las aguas de Meribá, y era donde les revelaba su
santidad.

\bibverse{14} Moisés envió mensajeros desde Cades al rey de Edom,
diciéndole: ``Esto es lo que dice tu hermano Israel. Tú sabes todo sobre
las dificultades que hemos enfrentado. \bibverse{15} Nuestros
antepasados fueron a Egipto y nosotros vivimos allí mucho tiempo. Los
egipcios nos trataron mal a nosotros y a nuestros antepasados,
\bibverse{16} así que pedimos ayuda al Señor, y él escuchó nuestros
gritos. Envió un ángel y nos sacó de Egipto.

Escuchen, ahora estamos en Cades, un pueblo en la frontera de su
territorio. \bibverse{17} Por favor, permítanos viajar a través de su
país. No cruzaremos ninguno de sus campos o viñedos, ni beberemos agua
de ninguno de sus pozos. Nos quedaremos en la Carretera del Rey; no nos
desviaremos ni a la derecha ni a la izquierda hasta que hayamos pasado
por su país.''

\bibverse{18} Pero el rey de Edom respondió: ``Se les prohíbe viajar por
nuestro país, de lo contrario saldremos y los detendremos por la
fuerza''.

\bibverse{19} Nos mantendremos en el camino principal'', insistieron los
israelitas. ``Si nosotros o nuestro ganado bebemos tu agua, te pagaremos
por ella. Eso es todo lo que queremos, sólo pasar a pie.''

\bibverse{20} Pero el rey de Edom insistió: ``¡Tienen prohibido viajar
por nuestro país!'' Salió con su gran y poderoso ejército para
enfrentarse a los israelitas de frente. \bibverse{21} Como el rey de
Edom se negó a permitir que Israel viajara por su territorio, los
israelitas tuvieron que volver.

\bibverse{22} Todos los israelitas dejaron Cades y viajaron al Monte
Hor.

\bibverse{23} En el monte Hor, cerca de la frontera con el país de Edom,
el Señor dijo a Moisés y Aarón, \bibverse{24} Aarón pronto se unirá a
sus antepasados en la muerte. No entrará en el país que he dado a los
israelitas, porque ambos desobedecieron mi orden en las aguas de Meribá.
\bibverse{25} Que Aarón y su hijo Eleazar se unan a ustedes y suban
juntos al monte Hor. \bibverse{26} Quítale a Aarón la ropa de sacerdote
y pónsela a su hijo Eleazar. Aarón va a morir allí y se unirá a sus
antepasados en la muerte''.

\bibverse{27} Moisés hizo lo que el Señor le ordenó: Subieron al monte
Hor a la vista de todos los israelitas. \bibverse{28} Moisés se quitó
las ropas sacerdotales que llevaba Aarón y se las puso a su hijo
Eleazar. Aarón murió allí, en la cima del monte. Entonces Moisés y
Eleazar volvieron a bajar. \bibverse{29} Cuando la gente se dio cuenta
de que Aarón había muerto, todos lloraron por él durante treinta días.

\hypertarget{section-20}{%
\section{21}\label{section-20}}

\bibverse{1} El rey cananeo de Arad, que vivía en el Néguev, se enteró
de que los israelitas se acercaban por el camino de Atharim. Fue y atacó
a Israel y tomó a algunos de ellos prisioneros. \bibverse{2} Así que
Israel hizo una promesa solemne al Señor: ``Si nos entregas a esta
gente, nos comprometemos a destruir completamente sus pueblos.''

\bibverse{3} El Señor respondió a su invitación y les entregó a los
cananeos. Los israelitas los destruyeron completamente a ellos y a sus
pueblos, y llamaron al lugar Horma.\footnote{21.3 ``Horma'' significa
  ``destrucción.''}

\bibverse{4} Los israelitas dejaron el Monte Hor por el camino que lleva
al Mar Rojo para evitar viajar por el país de Edom. Pero el pueblo se
puso de mal humor en el camino \bibverse{5} y se quejó contra Dios y
contra Moisés, diciendo: ``¿Por qué nos sacaste de Egipto para morir en
el desierto? No tenemos ni pan ni agua, y odiamos esta horrible
comida!''\footnote{21.5 ``Horrible comida'': refiriéndose al maná.}

\bibverse{6} Así que el Señor envió serpientes venenosas para atacarlos,
y muchos israelitas fueron mordidos y murieron.

\bibverse{7} El pueblo fue a ver a Moisés y le dijo: ``Nos equivocamos
al presentar quejas contra el Señor y contra ti. Por favor, ruega al
Señor que nos quite las serpientes de encima.'' Moisés rezó al Señor en
su nombre.

\bibverse{8} El Señor le dijo a Moisés: ``Haz una maqueta de una
serpiente y ponla en un palo. Cuando alguien que haya sido mordido la
mire, vivirá.'' \bibverse{9} Moisés hizo una serpiente de bronce y la
puso en un poste. Aquellos que la miraron vivieron.

\bibverse{10} Los israelitas salieron y acamparon en Obot. \bibverse{11}
Luego se fueron de Obot y acamparon en Iye-abarim en el desierto en el
lado este de Moab. \bibverse{12} Se fueron de allí y acamparon en el
Valle de Zered. \bibverse{13} Luego se trasladaron de allí y acamparon
en el lado más alejado del río Arnón, en el desierto cerca del
territorio de Amorite. El río Arnón es la frontera entre Moab y los
amorreos. \bibverse{14} Por eso el Libro de las Guerras del Señor se
refiere ``al pueblo de Vaheb en Sufa y al barranco de Arnón,
\bibverse{15} a las laderas del barranco que llegan al pueblo de Ar que
está en la frontera con Moab''.

\bibverse{16} Desde allí se trasladaron a Beer, el pozo donde el Señor
le dijo a Moisés, ``Haz que el pueblo se reúna para que pueda darles
agua''.

\bibverse{17} Entonces los israelitas cantaron esta canción: ``¡Echen
agua en el pozo! ¡Cada uno de ustedes, cante! \bibverse{18} Los jefes de
las tribus cavaron el pozo; sí, los jefes del pueblo cavaron el pozo con
sus varas de autoridad y sus bastones''.

Los israelitas dejaron el desierto y siguieron hasta Matanaá
\bibverse{19} Desde Mataná viajaron a Nahaliel, de Nahaliel a Bamot,
\bibverse{20} y de Bamoth al valle en el territorio de Moab donde la
cima del Monte Pisga mira hacia los páramos.

\bibverse{21} Entonces Israel envió mensajeros a Sehón, rey de los
amorreos, con la siguiente petición: \bibverse{22} Por favor, permítanos
viajar a través de su país. No cruzaremos ninguno de sus campos o
viñedos, ni beberemos agua de ninguno de sus pozos. Permaneceremos en la
Carretera del Rey hasta que hayamos pasado por su país.''

\bibverse{23} Pero Sehón se negó a permitir que los israelitas viajaran
por su territorio. En su lugar, llamó a todo su ejército y salió al
encuentro de los israelitas de frente en el desierto. Cuando llegó a
Jahaz, atacó a los israelitas. \bibverse{24} Los israelitas los
derrotaron, matándolos con sus espadas. Se apoderaron de su tierra desde
el río Arnón hasta el río Jaboc, pero sólo hasta la frontera de los
amonitas, porque estaba bien defendida.

\bibverse{25} Los israelitas conquistaron todos los pueblos amorreos y
se apoderaron de ellos, incluyendo Hesbón y sus pueblos vecinos.
\bibverse{26} Hesbón era la capital de Sehón, rey de los amorreos, que
había luchado contra el anterior rey de Moab y le había quitado todas
sus tierras hasta el río Arnón. \bibverse{27} Por eso los antiguos
compositores escribieron: ``¡Vengan a Hesbón y hagan que la
reconstruyan; restauren la ciudad de Sehón! \bibverse{28} Porque un
fuego salió de Hesbón, una llama de la ciudad de Sihón. Quemó a Ar en
Moab, donde los gobernantes viven en las alturas de Arnón. \bibverse{29}
¡Qué desastre enfrentas, Moab! ¡Vais a morir todos, pueblo de
Quemos!\footnote{21.29 Quemos era un dios al que se le presentaban
  sacrificios humanos.}Entregaste a tus hijos como exiliados y a tus
hijas como prisioneras a Sehón, rey de los amorreos. \bibverse{30} ¡Pero
ahora hemos derrotado a los amorreos! El gobierno de Heshbon ha sido
destruido hasta Dibon. Los aniquilamos hasta Nofa y hasta Medeba''.

\bibverse{31} Los israelitas ocuparon el país de los amorreos.
\bibverse{32} Moisés envió hombres a explorar Jazer. Los israelitas
conquistaron los pueblos de los alrededores y expulsaron a los amorreos
que vivían allí. \bibverse{33} Luego continuaron en el camino hacia
Basán. Og, rey de Basán, dirigió a todo su ejército para enfrentarse a
ellos de frente, y luchó contra ellos en Edrei. \bibverse{34} El Señor
le dijo a Moisés: ``No tienes que temerle, porque yo te lo he entregado,
junto con todo su pueblo y su tierra. Hazle lo que hiciste con Sehón,
rey de los amorreos, que gobernó desde Hesbón''.

\bibverse{35} Así que mataron a Og, a sus hijos y a todo su ejército.
Nadie sobrevivió, y los israelitas se apoderaron de su país.

\hypertarget{section-21}{%
\section{22}\label{section-21}}

\bibverse{1} Los israelitas avanzaron y acamparon en las llanuras de
Moab al este del Jordán, frente a Jericó. \bibverse{2} alac, hijo de
Zippor, había visto todo lo que los israelitas habían hecho a los
amorreos. \bibverse{3} Los moabitas estaban aterrorizados de los
israelitas porque eran muchos. Los moabitas temían la llegada de los
israelitas \bibverse{4} y dijeron a los líderes de Madián, ``Esta horda
se comerá todo lo que tenemos, como un buey se come la hierba del
campo''. (Balac hijo de Zippor, era rey de Moab en ese momento.)
\bibverse{5} Envió mensajeros para llamar a Balaam, hijo de Beor, que
vivía en Petor, cerca del río Éufrates en su propio país.

``Escuchen, ha llegado aquí un grupo de personas que vinieron de
Egipto'', dijo Balac en su mensaje a Balaam. ``Hay hordas de ellos y
representan una verdadera amenaza para nosotros. \bibverse{6} Por favor,
ven inmediatamente y maldice a estas personas por mí, porque son más
fuertes que yo. Tal vez entonces pueda atacarlos y expulsarlos de mi
país porque sé que quienquiera que bendiga es bendecido, y quienquiera
que maldiga es maldito''.

\bibverse{7} Entonces los líderes moabitas y madianitas partieron,
llevándose el pago de la adivinación con ellos. Cuando llegaron, le
dieron a Balaam el mensaje de Balac.

\bibverse{8} ``Pasen la noche y les haré saber la respuesta que me da el
Señor\footnote{22.8 Aunque Balaam no es un israelita, usa su nombre para
  Dios.},'' les dijo Balaam. Así que los líderes moabitas se quedaron
allí con Balaam.

\bibverse{9} Dios vino a Balaam y le preguntó: ``¿Quiénes son estos
hombres que están contigo?''

\bibverse{10} Balaam le dijo a Dios: ``Balac, hijo de Zipor, el rey de
Moab, me envió este mensaje: \bibverse{11} 'Escucha, ha llegado aquí un
grupo de gente que ha venido de Egipto. Hay hordas de ellos. Por favor,
ven inmediatamente y maldice a esta gente por mí. Tal vez así pueda
luchar contra ellos y expulsarlos de mi país''.

\bibverse{12} Pero Dios le dijo a Balaam, ``No debes volver con ellos.
No debes maldecir a este pueblo porque están bendecidos.''

\bibverse{13} Por la mañana Balaam se levantó y dijo a los mensajeros de
Balac, ``Vuelve al lugar de donde viniste porque el Señor se ha negado a
permitirme ir contigo.''

\bibverse{14} Los líderes moabitas se fueron. Volvieron donde Balac y le
dijeron: ``Balaam se negó a volver con nosotros''.

\bibverse{15} Entonces Balac envió más líderes, unos que eran más
prestigiosos que los anteriores. \bibverse{16} Cuando llegaron le
dijeron a Balaam: ``Esto es lo que dice Balac hijo de Zipor: 'Por favor,
no dejes que nada te impida venir a verme, \bibverse{17} porque te
pagaré mucho y seguiré todos los consejos que me des. Por favor, ven y
maldice a este pueblo por mí''.

\bibverse{18} Pero Balaam le dijo a los oficiales de Balac, ``Aunque
Balac me diera todo su palacio lleno de plata y oro, no podría
desobedecer el mandato del Señor mi Dios de ninguna manera.\footnote{22.18
  ``De ninguna manera'': literalmente, ``Ya sea por poco o mucho.''}
\bibverse{19} Ahora también deberías pasar la noche para ver si el Señor
tiene algo más que decirme.''

\bibverse{20} Dios vino a Balaam durante la noche y le dijo, ``Ya que
estos hombres han venido por ti, levántate y ve con ellos. Pero sólo haz
lo que yo te diga.'' \bibverse{21} Por la mañana Balaam se levantó, puso
una silla en su burro y se fue con los líderes moabitas.

\bibverse{22} Dios se enfadó porque Balaam había decidido irse. El ángel
del Señor vino y se paró en el camino para enfrentarlo. Balaam iba
montado en su burro, y estaba acompañado por sus dos sirvientes.
\bibverse{23} El burro vio al ángel del Señor de pie en el camino con
una espada desenvainada, así que se apartó del camino y se fue a un
campo. Así que Balaam lo golpeó para que volviera al camino.

\bibverse{24} Entonces el ángel del Señor se paró en una parte estrecha
del camino que pasaba entre dos viñedos, con muros a ambos lados.
\bibverse{25} El burro vio al ángel del Señor e intentó
pasar.\footnote{22.25 ``E intentó pasar'': añadido para mayor claridad.}Empujó
contra la pared y aplastó el pie de Balaam contra ella. Así que lo
golpeó de nuevo.

\bibverse{26} Entonces el ángel del Señor se adelantó y se paró en un
lugar estrecho donde no había espacio para pasar, ni a la derecha ni a
la izquierda. \bibverse{27} El burro vio al ángel del Señor y se acostó
bajo Balaam. Se enfadó y lo golpeó con su bastón.

\bibverse{28} El Señor le dio al burro la habilidad de hablar y le dijo
a Balaam: ``¿Qué te he hecho para que me golpees tres veces?''

\bibverse{29} ¡Me has hecho quedar como un estúpido!'' Balaam le dijo al
burro. ``¡Si tuviera una espada, te mataría ahora!''

\bibverse{30} Pero el burro le preguntó a Balaam, ``¿No soy yo el burro
que has montado toda tu vida hasta hoy? ¿Alguna vez te he tratado así
antes?''

``No'', admitió.

\bibverse{31} Entonces el Señor le dio a Balaam la habilidad de ver al
ángel del Señor de pie en el camino con una espada desenvainada. Balaam
se inclinó y cayó al suelo boca abajo.

\bibverse{32} El ángel del Señor le preguntó: ``¿Por qué golpeaste a tu
burro tres veces? Escucha, he venido a enfrentarme a ti porque estás
siendo obstinado. \bibverse{33} El burro me vio y me evitó tres veces.
Si no me hubiera evitado, ya te habría matado y dejado vivir al burro''.

\bibverse{34} He pecado porque no me di cuenta de que estabas parado en
el camino para enfrentarme'', dijo Balaam al ángel del Señor, ``Así que,
si esto no es lo que quieres, volveré a casa''.

\bibverse{35} El ángel del Señor le dijo a Balaam, ``No, puedes ir con
los hombres, pero sólo di lo que yo te diga.'' Así que Balaam continuó
con los oficiales de Balac.

\bibverse{36} Cuando Balac se enteró de que Balaam estaba en camino, fue
a reunirse con él en el pueblo moabita en la frontera del río Arnón, el
punto más alejado de su territorio. \bibverse{37} Le dijo a Balaam,
``¿No pensaste que mi llamada para que vinieras era urgente? ¿Por qué no
viniste a mí inmediatamente? ¿Pensaste que no podía pagarte lo
suficiente?''

\bibverse{38} Mira, estoy aquí contigo ahora, ¿no?'' Balaam respondió.
``¿Pero crees que puedo decir cualquier cosa? Sólo puedo decir las
palabras que Dios me da para que las diga.''

\bibverse{39} Así que Balaam se fue con Balac y llegaron a
Quiriath-huzot. \bibverse{40} alac sacrificó ganado y ovejas, y
compartió la carne con Balaam y los líderes que estaban con él.

\bibverse{41} A la mañana siguiente Balac llevó a Balaam hasta
Bamot-baal.\footnote{22.41 ``Bamot-baal'': que significa ``Los Altos
  Lugares de Baal''. Algunos han llegado a la conclusión de que un
  templo pagano a Baal ocupaba este punto alto.} Desde allí pudo ver la
extensión del campamento israelita.

\hypertarget{section-22}{%
\section{23}\label{section-22}}

\bibverse{1} Entonces Balaam le dijo a Balac, ``Constrúyeme siete
altares aquí, y prepárame siete toros y siete carneros para un
sacrificio''. \bibverse{2} alac hizo lo que Balaam había dicho, y juntos
ofrecieron un toro y un carnero en cada altar.

\bibverse{3} Balaam le dijo a Balac, ``Espera aquí junto a tu holocausto
mientras voy a ver si quizás el Señor vendrá y se reunirá conmigo.
Cualquier cosa que me revele, la compartiré contigo.'' Entonces Balaam
se fue a escalar un peñasco rocoso.

\bibverse{4} Dios se encontró con él allí, y Balaam dijo. ``He
construido siete altares y en cada uno de ellos he ofrecido un toro y un
carnero.''

\bibverse{5} El Señor le dio a Balaam un mensaje para compartir. Le
dijo, ``Vuelve a Balac y esto es lo que debes decirle.''

\bibverse{6} Así que volvió a Balac, que estaba esperando allí junto a
su holocausto, junto con todos los líderes moabitas.

\bibverse{7} Esta es la declaración que Balaam dio:

``Balac me trajo de Aram; el rey de Moab me trajo de las montañas del
este. Dijo: '¡Ven a maldecir a Jacob por mí! Ven y condena a Israel''.

\bibverse{8} Pero ¿cómo puedo maldecir lo que Dios no ha maldito? ¿Cómo
puedo condenar lo que el Señor no ha condenado? \bibverse{9} Porque yo
los miro desde lo alto de los peñascos; los observo desde las colinas.
Veo un pueblo que vive por su cuenta, diferente de las otras naciones.

\bibverse{10} ¿Quién puede contar los descendientes de Jacob? ¡Son
tantos que son como el polvo! ¿Quién puede contar hasta una cuarta parte
de los israelitas?

¡Me gustaría morir como muere una persona buena! ¡Que el fin de mi vida
sea como el fin de ellos!''

\bibverse{11} Entonces Balac se quejó a Balaam, ``¿Qué esesto que me has
hecho? Te traje aquí para maldecir a mis enemigos, ¡y ahora mira! ¡Todo
lo que has hecho es bendecirlos!''

\bibverse{12} Pero Balaam respondió: ``¿No crees que debería decir
precisamente lo que el Señor me dice?''

\bibverse{13} Entonces Balac le dijo: ``Por favor, ven conmigo a otro
lugar donde puedas verlos. Pero sólo verás una parte de su campamento,
no los verás a todos. Puedes maldecirlos por mí desde allí''.
\bibverse{14} Lo llevó al campo de Zofim en la cima del Monte Pisga.
Allí construyó siete altares y ofreció un toro y un carnero en cada
altar.

\bibverse{15} Balaam le dijo a Balac, ``Espera aquí junto a tu
holocausto mientras me encuentro con el Señor allí.'' \bibverse{16} El
Señor se encontró con Balaam y le dio un mensaje para compartir. Le
dijo, ``Vuelve a Balac y esto es lo que debes decirle.'' \bibverse{17}
Así que volvió a Balac, que estaba esperando allí junto a su holocausto,
junto con todos los líderes moabitas.

``¿Qué dijo el Señor?'' Preguntó Balac.

\bibverse{18} Esta es la profecía que Balaam cumplió:

``¡Levántate, Balac, y presta atención! ¡Escúchame, hijo de Zipor!

\bibverse{19} Dios no es un ser humano que mentiría. No es un simple
mortal que cambia de opinión. ¿Acaso él dice que va a hacer algo, pero
no lo hace? ¿Acaso hace promesas que no cumple?

\bibverse{20} Mira, se me ha ordenado dar una bendición. Dios ha
bendecido, y no puedo cambiar eso.

\bibverse{21} No esperes que le pase nada malo a Jacob; no se prevé
ningún problema para Israel. El Señor su Dios está con ellos; lo
celebran como su rey.

\bibverse{22} Dios los sacó de Egipto con gran poder, tan fuerte como un
buey.

\bibverse{23} No se puede lanzar ningún hechizo contra Jacob; no se
puede usar ninguna magia contra Israel. La gente hablará de Jacob e
Israel, diciendo: ``¡Qué cosas tan asombrosas ha hecho Dios por ellos!

\bibverse{24} ¡Miren! Los israelitas salen a cazar como una leona;
persiguen como un león. No descansan hasta que comen su presa, y beben
la sangre de su víctima muerta.''

\bibverse{25} Entonces Balac le dijo a Balaam, ``¡Si no puedes darles
ninguna maldición, al menos no les des ninguna bendición!''

\bibverse{26} Pero Balaam respondió: ``¿No te he explicado que tengo que
hacer todo lo que el Señor me diga?''

\bibverse{27} Por favor, ven conmigo y te llevaré a otro lugar'', dijo
Balac. ``Tal vez Dios te permita maldecirlos por mí desde allí.''
\bibverse{28} alac llevó a Balaam a la cima del Monte Peor, que mira
hacia los páramos.

\bibverse{29} Balaam le dijo a Balac, ``Constrúyeme siete altares aquí,
y prepárame siete toros y siete carneros para sacrificar.''
\bibverse{30} alac le dijo lo que Balaam le dijo, y ofreció un toro y un
carnero en cada altar.

\hypertarget{section-23}{%
\section{24}\label{section-23}}

\bibverse{1} Cuando Balaam vio que el Señor quería bendecir a Israel,
eligió no usar la adivinación como lo había hecho anteriormente. En su
lugar se volvió hacia el desierto, \bibverse{2} y al mirar a Israel
acampado allí según sus respectivas tribus, el Espíritu de Dios vino
sobre él. \bibverse{3} Hizo una declaración, diciendo:

\bibverse{4} ``Esta es la profecía de Balaam, hijo de Beor, la profecía
de un hombre que ve con los ojos bien abiertos,\footnote{24.4 ``Con los
  ojos abiertos'': Esta palabra sólo aparece aquí y en el versículo 15.
  Se traduce como ``cerrado'' o ``abierto'', sin embargo, el significado
  es esencialmente claro en que Balaam se refiere a la visión profética.
  La Vulgata Latina tiene ``Ojos que están bloqueados'' mientras que la
  Septuaginta Griega dice:``el que ve realmente.''} la profecía de uno
que oye las palabras de Dios, que ve la visión dada por el Todopoderoso,
que se inclina con respeto con los ojos abiertos.

\bibverse{5} ¡Qué bien puestas tus tiendas, Jacob; los lugares donde
vives, Israel! \bibverse{6} Parecen valles boscosos, como jardines junto
a un río, como árboles de áloe que el Señor ha plantado, como cedros a
la orilla del agua. \bibverse{7} Los israelitas derramarán cubos de
agua; sus descendientes tendrán mucha agua. Su rey será más grande que
el rey Agag; su reino será glorioso. \bibverse{8} Dios los sacó de
Egipto con gran poder, tan fuerte como un buey, destruyendo a las
naciones enemigas, rompiéndoles los huesos, atravesándolos con flechas.
\bibverse{9} Son como un león que se agacha y se acuesta. Son como una
leona que nadie se atreve a molestar. Quienes los bendigan serán
bendecidos; y quienes los maldigan serán malditos''.

\bibverse{10} alac se enfadó con Balaam, y se golpeó los puños. Le dijo
a Balaam, ``Te traje aquí para maldecir a mis enemigos, ¡y ahora mira!
Sigues bendiciéndolos, haciéndolo tres veces. \bibverse{11} ¡Vete ahora
mismo! ¡Vete a casa! Prometí pagarte bien, pero el Señor se ha asegurado
de que no recibirás ningún pago.''

\bibverse{12} Pero Balaam le dijo a Balac: ``¿No le expliqué ya a los
mensajeros que enviaste \bibverse{13} que aunque me dieras todo tu
palacio lleno de plata y oro, no podría hacer nada de lo que quisiera ni
desobedecer el mandato del Señor mi Dios de ninguna manera? Sólo puedo
decir lo que el Señor me dice. \bibverse{14} ¡Escucha! Ahora vuelvo a
casa con mi propio pueblo, pero primero déjame advertirte lo que estos
israelitas van a hacer con tu pueblo en el futuro''.

\bibverse{15} Entonces Balaam hizo una declaración, diciendo: ``Esta es
la profecía de Balaam, hijo de Beor, la profecía de un hombre con los
ojos bien abiertos \bibverse{16} la profecía de uno que escucha las
palabras de Dios, que recibe el conocimiento del Altísimo, que ve la
visión dada por el Todopoderoso, que se inclina con respeto con los ojos
abiertos.

\bibverse{17} Lo veo, pero esto no es ahora. Lo observo, pero esto no
está cerca. En el futuro, un líder como una estrella vendrá de Jacob, un
gobernante con un cetro llegará al poder desde Israel. Aplastará las
cabezas de los moabitas, y destruirá a todo el pueblo de Set.\footnote{24.17
  ``El pueblo de Set'': si esto se tomara literalmente, tal descripción
  también incluiría a los israelitas como descendientes de Set. En el
  contexto de la poesía paralela hebrea aquí probablemente se refiere
  específicamente a los moabitas. En el pasaje paralelo de Jeremías
  48:45 se lee ``pueblo rebelde.''} \bibverse{18} El país de Edom será
conquistado, su enemigo Seir+ 24.18 Seir era el nombre antiguo de
Edom.serán conquistados, y los israelitas saldrán victoriosos.
\bibverse{19} Un gobernante de Jacob vendrá y destruirá a los que queden
en la ciudad.''

\bibverse{20} Balaam dirigió su atención a los amalecitas y dio esta
declaración sobre ellos, diciendo, ``Amalec fue el primero entre las
naciones, pero terminarán siendo destruidos''.

\bibverse{21} Dirigió su atención a los ceneos y dio esta declaración
sobre ellos, diciendo, ``Donde vives está seguro y protegido, como un
nido en la cara de un acantilado. \bibverse{22} Pero Kain será quemado
cuando Asiria los conquiste.''

\bibverse{23} Luego Balaam hizo otra declaración, diciendo: ``¡Es una
tragedia! ¿Quién puede sobrevivir cuando Dios hace esto? \bibverse{24}
Se enviarán barcos desde Chipre para atacar Asiria y Eber, pero también
serán destruidos permanentemente.''

\bibverse{25} Entonces Balaam se marchó y volvió a su país, y Balac se
marchó también.

\hypertarget{section-24}{%
\section{25}\label{section-24}}

\bibverse{1} Cuando los israelitas se alojaban en Sitím los hombres
empezaron a tener sexo con mujeres moabitas \bibverse{2} que los
invitaban a los sacrificios hechos a sus dioses. Los israelitas comían
las comidas paganas y se inclinaban ante estos dioses. \bibverse{3} De
esta manera los israelitas se dedicaban a la adoración de Baal de Peor,
y el Señor estaba enojado con ellos.

\bibverse{4} El Señor le dijo a Moisés, ``Arresta a todos los líderes
israelitas y mátalos ante el Señor donde todos puedan ver\footnote{25.4
  ``Donde todos puedan ver'': literalmente, ``delante del sol.''} para
alejar la furiosa ira del Señor del pueblo.''

\bibverse{5} Así que Moisés instruyó a los jueces de
Israel,\footnote{25.5 Estos eran ``jueces líderes'' que desempeñaban más
  que un rol legal en la sociedad israelita.}``Cada uno de ustedes tiene
que matar a todos sus hombres que se han dedicado a adorar a Baal de
Peor.''

\bibverse{6} En ese momento un hombre israelita llevó a una mujer
madianita a la tienda de su familia a la vista de Moisés y de todos los
israelitas mientras lloraban a la entrada del Tabernáculo de Reunión.
\bibverse{7} Al ver esto, Finees, hijo de Eleazar, hijo del sacerdote
Aarón, abandonó la asamblea, agarró una lanza \bibverse{8} y siguió al
hombre a su tienda. Allí, Finees atravesó con la lanza a ambos, al
israelita y al estómago de la mujer. Esta acción detuvo la plaga que
había empezado a matar a los israelitas, \bibverse{9} pero ya habían
muerto 24.000.

\bibverse{10} El Señor le dijo a Moisés, \bibverse{11} Finees hijo de
Eleazar, hijo del sacerdote Aarón, ha alejado mi ira de los israelitas,
porque de todos ellos estaba fervorosamente dedicado a mí, así que no
destruí a los israelitas en mi apasionada ira. \bibverse{12} Así que
anunciad que le concedo mi acuerdo de paz. \bibverse{13} Será un acuerdo
que asegura un sacerdocio permanente para él y sus descendientes, porque
se dedicó apasionadamente a su Dios y enderezó a los israelitas''.

\bibverse{14} El nombre del israelita que fue asesinado con la mujer
madianita era Zimri, hijo de Salu, un líder de la familia de la tribu de
Simeón. \bibverse{15} El nombre de la mujer madianita que fue asesinada
era Cozbi, hija de Zur, un líder de familia de la tribu de Madián.

\bibverse{16} El Señor le dijo a Moisés: \bibverse{17} Ataca a los
madianitas y mátalos, \bibverse{18} porque te atacaron engañosamente,
llevándote por mal camino al usar a Peor y a su mujer Cozbi, la hija del
líder madianita -- la mujer que fue asesinada el día que llegó la plaga
por su devoción a Peor --.''

\hypertarget{section-25}{%
\section{26}\label{section-25}}

\bibverse{1} Después de que la plaga terminó, el Señor le dijo a Moisés
y Eleazar, hijo del sacerdote Aarón, \bibverse{2} Censen a todos los
israelitas por familia, todos aquellos de veinte años o más que sean
elegibles para el servicio militar en el ejército de Israel''.

\bibverse{3} Allí, en la llanura de Moab, junto al Jordán, frente a
Jericó, Moisés y Eleazar el sacerdote dio la orden, \bibverse{4} Censar
a los hombres de veinte años o más, siguiendo las instrucciones que el
Señor dio a Moisés''.

El siguiente es el registro genealógico de los que dejaron la tierra de
Egipto.

\bibverse{5} Estos eran los descendientes de Rubén, el primogénito de
Israel:

Hanoc, antepasado de la familia hanocítica; Falú, antepasado de la
familia faluita; \bibverse{6} Hezrón, antepasado de la familia
hezronita; y Carmi, antepasado de la familia carmita. \bibverse{7} Estas
fueron las familias descendientes de Rubén y fueron 43.730. \bibverse{8}
El hijo de Falú era Eliab, \bibverse{9} y los hijos de Eliab eran
Nemuel, Datán y Abiram. (Fueron Datán y Abiram, líderes escogidos por
los israelitas, los que se unieron a la rebelión contra Moisés y Aarón
con los seguidores de Coré cuando se rebelaron contra el Señor.
\bibverse{10} La tierra se abrió y se los tragó, junto con Coré. Sus
seguidores murieron cuando el fuego quemó a 250 hombres. Lo que les
sucedió fue una advertencia para los israelitas. \bibverse{11} Pero los
hijos de Coré no murieron). \bibverse{12} Estos fueron los descendientes
de Simeón por familia:

Nemuel,\footnote{26.12 Or ``Jemuel,'' see the parallel lists in Genesis
  46:10 y Exodus 6:15.}antepasado de la familia Nemuelita; Jamin,
antepasado de la familia Jaminita; Jacín, antepasado de la familia
Jaquinita; \bibverse{13} Zera,+ 26.13 Escrito también como ``Zojar'' en
las listas paralelas de Génesis 46:10 y Éxodo 6:15.ancestro de la
familia Zeraita; y Saul, ancestro de la familia Saulita. \bibverse{14}
Estas eran las familias descendientes de Simeón y eran 22.200.

\bibverse{15} Estos fueron los descendientes de Gad por familia:

Sefón,\footnote{26.15 Escrito también Zefón en Génesis 46:15.}ancestro
de la familia sefonita; Haggi, ancestro de la familia Haggite; Shuni,
ancestro de la familia Shunite; \bibverse{16} Ozni, ancestro de la
familia Oznite; Eri, ancestro de la familia Erite; \bibverse{17} Arod,+
26.17 Escrito también como Arodí en Génesis 46:16.antepasado de la
familia Arodita; Areli, antepasado de la familia Arelite. \bibverse{18}
Estas eran las familias descendientes de Gad y eran 40.500.

\bibverse{19} Los hijos de Judá que murieron en Canaán fueron Er y Onan.
Estos eran los descendientes de Judá por familia:

\bibverse{20} Sela, antepasado de la familia selaíta; Fares, antepasado
de la familia faresita; Zera, antepasado de la familia zeraíta.
\bibverse{21} Estos fueron los descendientes de Fares: Hezrón, ancestro
de la familia hezronita; y Hamul, ancestro de la familia hamulita.
\bibverse{22} Estas eran las familias descendientes de Judá y
sumaban76.500.

\bibverse{23} Estos fueron los descendientes de Isacar por familia:

Tola, antepasado de la familia tolaíta; Púa,\footnote{26.23 Escrito como
  ``Puah'' en algunas traducciones antiguas.} antepasado de la familia
punita; \bibverse{24} Jasub, antepasado de la familia jasubita; y
Simrón, antepasado de la familia simronita. \bibverse{25} Estas eran las
familias descendientes de Isacar y sumaban 64.300.

\bibverse{26} Estos eran los descendientes de Zabulón por familia:

Sered, antepasado de la familia seredita; Elón, antepasado de la familia
elonita; y Jahleel, antepasado de la familia jahleelita. \bibverse{27}
Estas eran las familias descendientes de Zabulón, y eran 60.500.

\bibverse{28} Estos fueron los descendientes de José por familia a
través de Manasés y Efraín: \bibverse{29} Los descendientes de Manasés:
Maquir (era el padre de Galaad), antepasado de la familia maquirita; y
Galaad, antepasado de la familia galaadita.

\bibverse{30} Los descendientes de Galaad: Izer, antepasado de la
familia Iezerita; Heled, antepasado de la familia helequita;
\bibverse{31} Asriel, antepasado de la familia asrielita; Siquem,
antepasado de la familia siquemita; \bibverse{32} Semida, antepasado de
la familia Semidita; y Hefer, antepasado de la familia heferita.
\bibverse{33} (Zelofehad, hijo de Hefer, no tuvo hijos, sólo hijas. Se
llamaban Maala, Noa, Hogla, Milca y Tirsa). \bibverse{34} Estas eran las
familias que descendían de Manasés, y eran 52.700.

\bibverse{35} Estos eran los descendientes de Efraín por familia:

Sutela, antepasado de la familia sutelaíta; Bequer, antepasado de la
familia bequerita; y Tahán, antepasado de la familia tahanita.
\bibverse{36} El descendiente de Suthelah era Erán, ancestro de la
familia eranita. \bibverse{37} Estas eran las familias descendientes de
Efraín, y sumaban 32.500. Estas familias eran descendientes de José.

\bibverse{38} Estos eran los descendientes de Benjamín por familia:

Bela, antepasado de la familia Belaite; Asbel, antepasado de la familia
asbelita; Ahiram, antepasado de la familia ahiramita; \bibverse{39}
Sufán,\footnote{26.39 O ``Sefufán.''}antepasado de la familia sufamita;
y Hufam, antepasado de la familia hufamita. \bibverse{40} Los
descendientes de Bela fueron Ard, ancestro de la familia de arditas; y
Naamán, ancestro de la familia Naamita. \bibverse{41} Estas fueron las
familias descendientes de Benjamín, y sumaban 45.600.

\bibverse{42} Estos fueron los descendientes de Dan por familia:

Súham, antepasado de las familias Suhamitas. \bibverse{43} Todas eran
familias suhamitas, y eran 64.400.

\bibverse{44} Estos eran los descendientes de Aser por familia:

Imnah, antepasado de la familia Imnite; Isvi, antepasado de la familia
isvita; y Bería, antepasado de la familia beriaita. \bibverse{45} Los
descendientes de Bería fueron Heber, antepasado de la familia heberita;
y Malquiel, antepasado de la familia malquielita. \bibverse{46} El
nombre de la hija de Aser era Sera. \bibverse{47} Estas eran las
familias descendientes de Aser, y sumaban 53.400.

\bibverse{48} Estos eran los descendientes de Neftalí por familia:

Jahzeel, antepasado de la familia jahzeelita; Guni, antepasado de la
familia gunita; \bibverse{49} Jezer, antepasado de la familia jezerita;
y Silem, antepasado de la familia silemita. \bibverse{50} Estas eran las
familias descendientes de Neftalí, y sumaban 45.400.

\bibverse{51} El total de todos los contados fue de 601.730.

\bibverse{52} El Señor le dijo a Moisés: \bibverse{53} Divide la tierra
que se va a poseer en función del número de los censados. \bibverse{54}
Dale una mayor superficie de tierra a las tribus grandes, y una menor
superficie a las tribus más pequeña. Cada tribu recibirá su asignación
de tierra dependiendo de su número contado en el censo.

\bibverse{55} La tierra debe ser dividida por sorteo. Cada uno recibirá
su tierra asignada en función del nombre de la tribu de su antepasado.
\bibverse{56} Cada asignación de tierra se dividirá por sorteo entre las
tribus, ya sean grandes o pequeñas.''

\bibverse{57} Estos fueron los levitas censados por familia:

Gerson, antepasado de la familia gersonita; Coat, antepasado de la
familia coatita; y Merari, antepasado de la familia merarita.
\bibverse{58} Las siguientes fueron las familias de los levitas: la
familia libnita, la familia hebronita, la familia mahlita, la familia
musita y la familia coraíta. Coat era el padre de Amram, \bibverse{59} y
el nombre de la esposa de Amram era Jocabed. Era descendiente de Levi,
nacida mientras los levitas estaban en Egipto. Tuvo hijos con Amram:
Aarón, Moisés y su hermana Miriam. \bibverse{60} Los hijos de Aarón
fueron Nadab, Abihu, Eleazar e Itamar, \bibverse{61} pero Nadab y Abihu
murieron cuando ofrecieron fuego prohibido en presencia del Señor.

\bibverse{62} El número de los levitas censados ascendía a 23.000. Esto
incluía a todos los varones de un mes o más. Sin embargo, no fueron
contados con los otros israelitas, porque no se les dio ninguna
asignación de tierras con los otros israelitas.

\bibverse{63} Este es el registro de los que fueron censados por Moisés
y Eleazar el sacerdote cuando contaron a los israelitas en las llanuras
de Moab al lado del Jordán frente a Jericó.

\bibverse{64} Sin embargo, no incluyeron ni uno solo que hubiera sido
censado previamente por Moisés y el sacerdote Aarón cuando contaron a
los israelitas en el desierto del Sinaí, \bibverse{65} porque el Señor
les había dicho que todos morirían en el desierto. No quedó nadie
excepto Caleb, hijo de Jefone, y Josué, hijo de Nun.

\hypertarget{section-26}{%
\section{27}\label{section-26}}

\bibverse{1} Las hijas de Zelofead vinieron a presentar su
caso.\footnote{27.1 Ver también Josué 17:3-6.}Su padre Zelofehad era
hijo de Hefer, hijo de Galaad, hijo de Maquir, hijo de Manasés, y era de
la tribu de Manasés, hijo de José. Los nombres de sus hijas eran Maala,
Noa, Hogla, Milca y Tirsa. Vinieron \bibverse{2} y se presentaron ante
Moisés, el sacerdote Eleazar, los líderes y todos los israelitas a la
entrada del Tabernáculo de Reunión. Dijeron, \bibverse{3} Nuestro padre
murió en el desierto, pero no era uno de los seguidores de Coré que se
unieron para rebelarse contra el Señor. No, murió por sus propios
pecados, y no tuvo hijos. \bibverse{4} ¿Por qué debería perderse el
nombre de nuestra familia simplemente porque no tuvo un hijo? Danos
tierra para que la poseamos junto a nuestros tíos''.

\bibverse{5} Moisés llevó su caso ante el Señor. \bibverse{6} El Señor
le dio esta respuesta, \bibverse{7} Lo que las hijas de Zelofehad están
diciendo es correcto. Debes darles tierra para que la posean junto a sus
tíos, dales lo que se le habría asignado a su padre. \bibverse{8}
Además, dile a los israelitas: ``Si un hombre muere y no tiene un hijo,
dale su propiedad a su hija. \bibverse{9} Si no tiene una hija, da su
propiedad a sus hermanos. \bibverse{10} Si no tiene hermanos, dé su
propiedad a los hermanos de su padre. \bibverse{11} Si su padre no tiene
hermanos, déle su propiedad a los parientes más cercanos de su familia
para que puedan ser dueños de ella. Esta es una regulación legal para
los israelitas, dada como una orden del Señor a Moisés.''

\bibverse{12} El Señor le dijo a Moisés: ``Sube a los montes de Abarim
para que veas la tierra que he dado a los israelitas. \bibverse{13}
Después que la hayas visto, también te unirás a tus antepasados en la
muerte, como lo hizo tu hermano Aarón, \bibverse{14} porque cuando los
israelitas se quejaron en el desierto de Zin, ambos se rebelaron contra
mis instrucciones de mostrar mi santidad ante ellos en lo que respecta
al suministro de agua''. (Estas fueron las aguas de Meribá en Cades, en
el desierto de Zin).

\bibverse{15} Entonces Moisés suplicó al Señor, \bibverse{16} Que el
Señor, el Dios que da la vida a todos los seres vivos, elija un hombre
que guíe a los israelitas \bibverse{17} que les diga qué hacer y les
muestre dónde ir, para que el pueblo del Señor no sea como ovejas sin
pastor''.

\bibverse{18} El Señor le dijo a Moisés: ``Llama a Josué, hijo de Nun,
un hombre que tiene el Espíritu en él, y pon tus manos sobre él.
\bibverse{19} Haz que se ponga delante del sacerdote Eleazar y de todos
los israelitas, y dedícalo mientras ellos velan. \bibverse{20} Entrégale
algo de tu autoridad para que todos los israelitas le obedezcan.
\bibverse{21} Cuando necesite instrucciones deberá ir ante Eleazar, el
sacerdote, quien orará al Señor en su nombre y consultará la decisión
usando el Urim.\footnote{27.21 Elemento que se usaba para determinar la
  voluntad del Señor. Ver Éxodo 28:30, Levítico 8:8.} Josué les dará
órdenes a todos los israelitas sobre todo lo que deben hacer.''

\bibverse{22} Moisés siguió las instrucciones del Señor. Hizo que Josué
viniera y se pusiera delante del sacerdote Eleazar y de todos los
israelitas. \bibverse{23} Moisés puso sus manos sobre Josué y lo dedicó,
tal como el Señor le había dicho que hiciera.

\hypertarget{section-27}{%
\section{28}\label{section-27}}

\bibverse{1} El Señor le dijo a Moisés: \bibverse{2} Dales las
siguientes normas a los israelitas:\footnote{28.2 Este pasaje es
  paralelo a las instrucciones dadas en Éxodo 29:38-41.} 'Debes
presentarme en el momento apropiado mis ofrendas de comida para que las
acepte. \bibverse{3} Diles que debes presentar al Señor cada día dos
corderos machos de un año como holocausto continua. \bibverse{4} Ofrece
un cordero por la mañana y otro por la tarde antes de que oscurezca,
\bibverse{5} junto con una décima parte de una efa de la mejor harina
para una ofrenda de grano, mezclada con un cuarto de hin de aceite de
oliva prensado.

\bibverse{6} Este es un holocausto continuo que se inició en el Monte
Sinaí como una ofrenda aceptable para el Señor. \bibverse{7} La ofrenda
de bebida que acompaña a cada cordero debe ser un cuarto de hin. Vierte
la ofrenda de bebida fermentada al Señor en el santuario. \bibverse{8}
Ofrecerás el segundo cordero por la tarde antes de que oscurezca, junto
con las mismas ofrendas de grano y bebida que por la mañana. Es un
holocausto aceptable para el Señor.

\bibverse{9} En el día de reposo, presentarás los corderos machos de dos
años, sin defectos, junto con una ofrenda de grano de dos décimas de efa
de la mejor harina mezclada con aceite de oliva, y su libación.
\bibverse{10} Este holocausto debe ser presentado cada sábado además del
holocausto continuo y su libación.

\bibverse{11} Al comienzo de cada mes, presentarán al Señor un
holocausto de dos novillos, un carnero y siete corderos machos de un
año, todos ellos sin defectos, \bibverse{12} junto con ofrendas de grano
que consisten en tres décimas de una efa de la mejor harina mezclada con
aceite de oliva para cada toro, dos décimas de una efa de la mejor
harina mezclada con aceite de oliva para el carnero, \bibverse{13} y una
décima de una efa de la mejor harina mezclada con aceite de oliva para
cada uno de los corderos. Este es un holocausto aceptable para el Señor.

\bibverse{14} Sus respectivas libaciones serán medio hin de vino por
cada toro, un tercio de hin por el carnero y un cuarto de hin por cada
cordero. Este es el holocausto mensual que se presentará cada mes
durante el año. \bibverse{15} Además del holocausto continuo con su
libación, presentar un macho cabrío al Señor como ofrenda por el pecado.

\bibverse{16} La Pascua del Señor es el día catorce del primer mes.
\bibverse{17} Habrá una fiesta a los quince días de este mes, y durante
siete días sólo se comerá pan sin levadura. \bibverse{18} Celebrarán una
reunión sagrada el primer día de la fiesta. No hagan ninguna actividad
de su trabajo normal. \bibverse{19} Preséntense ante el Señor con las
siguientes ofrendas: un holocausto de dos novillos, un carnero y siete
corderos de un año, todos ellos sin defectos. \bibverse{20} Sus ofrendas
de grano se harán con la mejor harina mezclada con aceite de oliva: tres
décimas de efa para cada toro, dos décimas de efa para el carnero,
\bibverse{21} y una décima de efa para cada uno de los siete corderos.
\bibverse{22} Presenten también una cabra macho como ofrenda por el
pecado para hacerte justicia. \bibverse{23} Deben presentar estas
ofrendas además del continuo holocausto de la mañana. \bibverse{24}
Presenta las mismas ofrendas todos los días durante siete días como
holocausto para ser aceptado por el Señor. Deben ser ofrecidas con su
libación y el continuo holocausto. \bibverse{25} Celebren una reunión
sagrada el séptimo día del festival. Ese día no harán su trabajo usual.

\bibverse{26} Durante la celebración del Festival de las
Semanas,\footnote{28.26 También llamado el ``Festival de la Cosecha'' en
  Éxodo 23:16.}celebrarán una reunión sagrada el día de las primicias
cuando presenten una ofrenda de grano nuevo al Señor. No hagan ningún
tipo de trabajo. \bibverse{27} Presenten un holocausto de dos novillos,
un carnero y siete corderos de un año para ser aceptados por el Señor.
\bibverse{28} Deben ir acompañados de sus ofrendas de grano de la mejor
harina mezclada con aceite de oliva: tres décimos de un efa para cada
toro, dos décimos de un efa para el carnero, \bibverse{29} y un décimo
de un efa para cada uno de los siete corderos. \bibverse{30} Presenten
también una cabra macho como ofrenda para que los justifique.
\bibverse{31} Presenten estas ofrendas junto con sus libaciones además
del holocausto continuo y su ofrenda de grano. Asegúrate de que los
animales sacrificados no tengan defectos.''

\hypertarget{section-28}{%
\section{29}\label{section-28}}

\bibverse{1} ``Celebren una reunión sagrada el primer día del séptimo
mes. No hagas nada de tu trabajo normal. Este es el día en que tocarás
las trompetas. \bibverse{2} Presenten un holocausto de un novillo, un
carnero y siete corderos machos de un año, todos ellos sin defectos,
como sacrificio aceptable al Señor, \bibverse{3} junto con sus ofrendas
de grano de la mejor harina mezclada con aceite de oliva: tres décimos
de un efa para el toro, dos décimos de un efa para el carnero,
\bibverse{4} y un décimo de un efa para cada uno de los siete corderos
machos. \bibverse{5} Presenten también una cabra macho como ofrenda por
el pecado para hacerte justicia. \bibverse{6} Estas ofrendas se suman a
los holocaustos mensuales y diarios junto con las ofrendas de grano y
las ofrendas de bebida requeridas. Son ofrendas quemadas aceptables para
el Señor.

\bibverse{7} Celebrarás una reunión sagrada el décimo día de este
séptimo mes, y practiquen la abnegación. No hagas nada de tu trabajo
normal. \bibverse{8} Presenta un holocausto de un novillo, un carnero y
siete corderos machos de un año, todos ellos sin defectos, aceptables
para el Señor. \bibverse{9} Deben ir acompañados de sus ofrendas de
grano de la mejor harina mezclada con aceite de oliva: tres décimas de
efa para el toro, dos décimas de efa para el carnero, \bibverse{10} y
una décima de efa para cada uno de los siete corderos. \bibverse{11}
Presenta también un macho cabrío como ofrenda por el pecado, además de
la ofrenda por el pecado para corregirte y el holocausto continuo con su
ofrenda de grano y su libación.

\bibverse{12} Celebrea una reunión sagrada el día quince del séptimo
mes. No hagas nada de tu trabajo normal. Debes celebrar un festival
dedicado al Señor durante siete días. \bibverse{13} Presenta como
holocausto aceptable al Señor: trece novillos, dos carneros y catorce
corderos machos de un año, todos ellos sin defectos. \bibverse{14} Se
acompañarán con sus ofrendas de grano de la mejor harina mezclada con
aceite de oliva: tres décimas de efa de la mejor harina mezclada con
aceite de oliva por cada uno de los trece toros, dos décimas de efa por
cada uno de los dos carneros, \bibverse{15} y una décima de efa por cada
uno de los catorce corderos. \bibverse{16} También presentarás un macho
cabrío como ofrenda por el pecado además del holocausto continuo con su
ofrenda de grano y su libación.

\bibverse{17} El segundo día, presente doce novillos, dos carneros y
catorce corderos machos de un año, todos ellos sin defectos.
\bibverse{18} Deben ir acompañados por sus ofrendas de grano y bebidas
para los toros, carneros y corderos, todo según el número requerido.
\bibverse{19} Presenta también un macho cabrío como ofrenda por el
pecado, además del continuo holocausto con su ofrenda de grano y su
libación.

\bibverse{20} Al tercer día, presenta once novillos, dos carneros y
catorce corderos machos de un año de edad, todos ellos sin defectos.
\bibverse{21} Deben estar acompañados por sus ofrendas de grano y
libaciones para los toros, carneros y corderos, todo de acuerdo al
número requerido. \bibverse{22} Presenta también un macho cabrío como
ofrenda por el pecado además del holocausto continuo con su ofrenda de
grano y su libación.

\bibverse{23} Al cuarto día presentarás diez novillos, dos carneros y
catorce corderos machos de un año, todos ellos sin defectos.
\bibverse{24} Deben ir acompañados por sus ofrendas de grano y
libaciones para los toros, carneros y corderos, todo de acuerdo al
número requerido. \bibverse{25} También presentarás un macho cabrío como
ofrenda por el pecado, además del continuo holocausto con su ofrenda de
grano y su libación.

\bibverse{26} El quinto día presentarás nueve novillos, dos carneros y
catorce corderos machos de un año, todos ellos sin defectos.
\bibverse{27} Deben ir acompañados por sus ofrendas de grano y
libaciones para los toros, carneros y corderos, todo de acuerdo con el
número requerido. \bibverse{28} Presentarás también un macho cabrío como
ofrenda por el pecado, además del continuo holocausto con su ofrenda de
grano y su libación.

\bibverse{29} Al sexto día presentarás ocho novillos, dos carneros y
catorce corderos machos de un año, todos ellos sin defectos.
\bibverse{30} Deben estar acompañados por sus ofrendas de grano y
libaciones para los toros, carneros y corderos, todo de acuerdo al
número requerido. \bibverse{31} También se presentará un macho cabrío
como ofrenda por el pecado, además del continuo holocausto con su
ofrenda de grano y su libación.

\bibverse{32} Al séptimo día presentar siete novillos, dos carneros y
catorce corderos machos de un año de edad, todos ellos sin defectos.
\bibverse{33} Deben estar acompañados por sus ofrendas de grano y
libaciones para los toros, carneros y corderos, todo de acuerdo al
número requerido. \bibverse{34} También se presentará un macho cabrío
como ofrenda por el pecado, además del continuo holocausto con su
ofrenda de grano y su libación.

\bibverse{35} En el octavo día todos ustedes se reunirán juntos. No
hagan nada de su trabajo normal. \bibverse{36} Presenta como holocausto
aceptable al Señor: un toro, dos carneros y siete corderos machos de un
año, todos ellos sin defectos. \bibverse{37} Deben ir acompañados de sus
ofrendas de grano y de las libaciones para los toros, carneros y
corderos, todo según el número requerido. \bibverse{38} También se
presentará un macho cabrío como ofrenda por el pecado, además del
continuo holocausto con su ofrenda de grano y su libación.

\bibverse{39} Presenta estas ofrendas al Señor en los momentos en que se
te requiera, además de tus ofrendas para cumplir una promesa y las
ofrendas de libre albedrío, ya sean holocaustos, ofrendas de grano,
libaciones o sacrificios de paz''.

\bibverse{40} Moisés repitió todo esto a los israelitas como el Señor se
lo ordenó.

\hypertarget{section-29}{%
\section{30}\label{section-29}}

\bibverse{1} Moisés dijo a los jefes de las tribus de Israel: ``Esto es
lo que nos ordena el Señor: \bibverse{2} Si un hombre hace una promesa
solemne al Señor, o promete hacer algo jurando, no debe romper su
promesa. Debe hacer todo lo que dijo que haría.

\bibverse{3} Si una mujer joven que aún vive en la casa de su padre hace
una promesa solemne al Señor o se compromete a hacer algo mediante un
juramento \bibverse{4} y su padre se entera de su promesa o juramento
pero no le dice nada, todas las promesas o juramentos que ha hecho se
mantendrán. \bibverse{5} Pero si su padre las rechaza tan pronto como se
entere, entonces ninguna de sus promesas o juramentos serán válidos. El
Señor la liberará de cumplirlas porque su padre las ha desautorizado.

\bibverse{6} Si una mujer se casa después de haber hecho una promesa
solemne o un juramento sin pensarlo \bibverse{7} y su marido se entera
de ello pero no le dice nada inmediatamente, todas las promesas o
juramentos que haya hecho se mantendrán. \bibverse{8} Pero si su marido
las rechaza cuando se entera de ello, entonces ninguna de sus promesas o
juramentos permanecen válidos y el Señor la liberará de cumplirlos.

\bibverse{9} Toda promesa solemne hecha por una viuda o una mujer
divorciada debe cumplirse.

\bibverse{10} Si una mujer que vive con su marido hace una promesa
solemne al Señor o se compromete a hacer algo mediante un juramento,
\bibverse{11} y su marido se entera de su promesa o juramento pero no le
dice nada y no lo desautoriza, entonces ninguna de sus promesas o
juramentos permanecen válidos. \bibverse{12} Pero si su marido las
rechaza tan pronto como se entera de ello, entonces ninguna de sus
promesas o juramentos siguen siendo válidos. El Señor la liberará de
mantenerlas porque su marido las ha rechazado.

\bibverse{13} Su marido también puede confirmar o rechazar cualquier
promesa o juramento solemne que la mujer haga para negarse a sí misma.
\bibverse{14} Pero si su marido no le dice nunca una palabra al
respecto, se supone que ha confirmado todas las promesas y juramentos
solemnes que ella ha hecho. \bibverse{15} Sin embargo, si él las rechaza
algún tiempo después de enterarse de ellas, entonces él tendrá la
responsabilidad de que ella las rompa.''

\bibverse{16} stos son los preceptos que el Señor dio a Moisés sobre la
relación entre un hombre y su esposa, y entre un padre y una hija que es
joven y todavía vive en casa.

\hypertarget{section-30}{%
\section{31}\label{section-30}}

\bibverse{1} El Señor le dijo a Moisés, \bibverse{2} Castiga a los
madianitas por lo que le hicieron a los israelitas. Después de eso te
unirás a tus antepasados en la muerte''.

\bibverse{3} Moisés instruyó al pueblo: ``Que algunos de tus hombres se
preparen para la batalla, para que puedan ir a atacar a los madianitas y
llevar a cabo el castigo del Señor sobre ellos. \bibverse{4} Debes
contribuir con mil hombres de cada tribu israelita.''

\bibverse{5} Así que se eligieron mil hombres de cada tribu israelita,
haciendo doce mil tropas listas para la batalla. \bibverse{6} Moisés los
envió a la batalla, mil de cada tribu, junto con Finees, hijo del
sacerdote Eleazar. Llevó consigo los objetos sagrados del santuario y
las trompetas usadas para dar señales. \bibverse{7} Atacaron a los
madianitas, como el Señor le había dicho a Moisés, y mataron a todos los
hombres. \bibverse{8} Entre los muertos estaban los cinco reyes de
Madián, Evi, Rekem, Zur, Hur y Reba. También mataron a Balaam, hijo de
Beor, con la espada.

\bibverse{9} Los israelitas capturaron a las mujeres y niños madianitas,
y tomaron como botín todas sus manadas, rebaños y posesiones.
\bibverse{10} Prendieron fuego a todos los pueblos y campamentos
madianitas donde habían vivido, \bibverse{11} y se llevaron todo el
saqueo y el botín, incluyendo personas y animales.

\bibverse{12} Llevaron los prisioneros, el saqueo y el pillaje a Moisés,
al sacerdote Eleazar y al resto de los israelitas donde estaban
acampados en las llanuras de Moab, junto al Jordán, frente a Jericó.
\bibverse{13} Moisés, Eleazar el sacerdote y todos los líderes
israelitas salieron del campamento para encontrarse con ellos.

\bibverse{14} Moisés estaba enfadado con los oficiales del ejército, los
comandantes de miles y los comandantes de cientos, que volvieron de la
batalla. \bibverse{15} ¿Por qué dejaste vivir a todas las mujeres?'' les
preguntó. \bibverse{16} ¡Noten que estas mujeres sedujeron a los hombres
israelitas, llevándolos a ser infieles al Señor en Peor, siguiendo el
consejo de Balaam! Por eso el pueblo del Señor sufrió la plaga.
\bibverse{17} Así que ve y mata a todos los niños y a todas las mujeres
que se hayan acostado con un hombre. \bibverse{18} Deja vivir a todas
las chicas que son vírgenes. Son tuyas. \bibverse{19} todos aquellos que
mataron a alguien o tocaron un cadáver deben permanecer fuera del
campamento durante siete días. Purifíquense y purifiquen a sus
prisioneros al tercer y séptimo día. \bibverse{20} También purifiquen
toda su ropa y cualquier cosa hecha de cuero, pelo de cabra o madera.''

\bibverse{21} El sacerdote Eleazar dijo a los soldados que habían ido a
la batalla: ``Estos son los preceptos legales que el Señor ha ordenado
llevar a cabo a Moisés: \bibverse{22} Todo lo que esté hecho de oro,
plata, bronce, hierro, estaño y plomo, \bibverse{23} todo lo que no se
queme, debe ser puesto al fuego para que quede limpio. Pero todavía
tiene que ser purificado usando agua de purificación. Todo lo que se
quema debe ser pasado por el agua. \bibverse{24} Lava tu ropa en el
séptimo día y estarás limpio. Entonces podrás entrar en el campamento.''

\bibverse{25} El Señor le dijo a Moisés, \bibverse{26} Tú, el sacerdote
Eleazar, y los líderes de la familia israelita deben tomar un registro
de las personas y animales que fueron capturados. \bibverse{27} Luego
divídanlos entre las tropas que entraron en batalla y el resto de los
israelitas. \bibverse{28} Tomen como contribución al Señor de lo que se
asigna a las tropas que fueron a la batalla una de cada quinientas
personas, ganado, asnos u ovejas. \bibverse{29} Tomen esto de su media
parte y denlo al sacerdote Eleazar como ofrenda al Señor.

\bibverse{30} De los israelitas; la mitad de la parte, toma una de cada
cincuenta personas, ganado, asnos u ovejas, u otros animales, y dáselos
a los levitas que cuidan del Tabernáculo del Señor''.

\bibverse{31} Moisés y el sacerdote Eleazar hicieron lo que el Señor
había ordenado a Moisés.

\bibverse{32} Esta era la lista de los botines que quedaban y que habían
sido saqueados por las tropas: 675.000 ovejas, \bibverse{33} 72.000
vacas, \bibverse{34} 61.000 burros, \bibverse{35} y 32.000 vírgenes.

\bibverse{36} Esta era la mitad de los que habían ido a luchar: 337.500
ovejas, \bibverse{37} con una contribución para el Señor de 675
\bibverse{38} 36.000 bovinos, con una contribución para el Señor de 72,
\bibverse{39} 30.500 burros, con una contribución para el Señor de 61,
\bibverse{40} y 16.000 personas, con una contribución para el Señor de
32. \bibverse{41} Moisés dio la contribución al sacerdote Eleazar como
ofrenda al Señor, como el Señor había ordenado a Moisés.

\bibverse{42} La mitad de la parte de los israelitas se fue después de
que Moisés diera la mitad de la parte a las tropas que habían ido a
luchar, \bibverse{43} consistió en: 337.500 ovejas, \bibverse{44} 36.000
vacas, \bibverse{45} 30.500 burros, \bibverse{46} y 16.000 personas.
\bibverse{47} Moisés tomó de la mitad de los israelitas una de cada
cincuenta personas y animales y les dio los levitas que cuidan del
Tabernáculo del Señor, como el Señor le había ordenado.

\bibverse{48} Los oficiales del ejército, los comandantes de millares y
los comandantes de centenas, se acercaron a Moisés \bibverse{49} y le
dijeron: ``Nosotros, tus siervos, hemos comprobado las tropas que
mandamos y no falta ni un solo hombre. \bibverse{50} Así que hemos
traído al Señor una ofrenda de los objetos de oro que cada hombre
recibió: brazaletes, pulseras, anillos, pendientes y collares, para que
podamos estar bien ante el Señor''.

\bibverse{51} El sacerdote Moisés y Eleazar aceptaron de ellos todos los
objetos de oro. \bibverse{52} El oro que los comandantes de miles y
cientos de personas dieron como ofrenda al Señor pesaba en total 16.750
siclos. \bibverse{53} (Los hombres que habían luchado en la batalla
habían tomado cada uno un botín para sí mismos.) \bibverse{54} Moisés y
el sacerdote Eleazar aceptaron el oro de los comandantes de miles y
cientos y lo llevaron al Tabernáculo de Reunión como ofrenda
conmemorativa para los israelitas en presencia del Señor.

\hypertarget{section-31}{%
\section{32}\label{section-31}}

\bibverse{1} Las tribus de Rubén y Gad tenían grandes cantidades de
ganado y vieron que la tierra de Jazer y Galaad era un buen lugar para
criarlos. \bibverse{2} Entonces vinieron a Moisés, al sacerdote Eleazar
y a los líderes israelitas y dijeron, \bibverse{3} Las ciudades de
Atarot, Dibón, Jazer, Nimra, Hesbón, Eleale, Sebam,\footnote{32.3
  También conocido como Sibma en el versículo 38.}Nebo y Beón,
\bibverse{4} que el Señor conquistó a la vista de los israelitas, son
adecuados para el ganado que poseemos tus siervos''.

\bibverse{5} Continuaron: ``Por favor, responde favorablemente a nuestra
petición: danos esta tierra. No nos hagas cruzar el Jordán''.

\bibverse{6} En respuesta Moisés preguntó a las tribus de Gad y Rubén:
``¿Esperas que tus hermanos vayan a luchar mientras tú te quedas aquí
sentado? \bibverse{7} ¿Por qué desanimar a los israelitas para que no
crucen al país que el Señor les ha dado? \bibverse{8} Esto es lo que
hicieron sus padres cuando los envié desde Cades-barnea a explorar la
tierra. \bibverse{9} Después de que sus padres viajaron por el valle de
Escol y exploraron la tierra, desalentaron a los israelitas,
persuadiéndolos de que no entraran en el país que el Señor les había
dado. \bibverse{10} Como resultado, hicieron enojar mucho al Señor ese
día, y él hizo este juramento, \bibverse{11} Ni uno solo de los que
salvé de Egipto, que tenga veinte años o más, verá jamás la tierra que
prometí con el juramento de dar a Abraham, Isaac y Jacob, porque no
estaban completamente comprometidos conmigo, \bibverse{12} nadie excepto
Caleb, hijo de Jefone, el cenesita, y Josué, hijo de Nun, porque estaban
completamente comprometidos conmigo''. \bibverse{13} El Señor se enojó
con Israel y los hizo vagar por el desierto durante cuarenta años, hasta
que murió toda la generación que había hecho el mal ante sus ojos.

\bibverse{14} ¡Miraos ahora, hijos de pecadores que han venido a ocupar
el lugar de sus padres para hacer que el Señor se enfade aún más con
Israel! \bibverse{15} Si dejas de seguirlo, él volverá a abandonar a
esta gente en el desierto, y su muerte será culpa tuya!''

\bibverse{16} Entonces las tribus de Gad y Rubén vinieron a Moisés y le
dijeron: ``Planeamos construir muros de piedra para mantener a salvo
nuestro ganado y pueblos para nuestros hijos. \bibverse{17} Pero aún así
nos prepararemos para la batalla, y estaremos preparados para liderar a
los israelitas hasta que puedan ocupar su tierra con seguridad. Mientras
tanto, nuestros hijos se quedarán atrás, viviendo en los pueblos
fortificados para protegerlos de la población local. \bibverse{18} No
regresaremos a nuestros hogares hasta que cada israelita esté en
posesión de su tierra asignada. \bibverse{19} Sin embargo, no poseeremos
ninguna tierra al otro lado del Jordán porque hemos recibido esta tierra
para poseerla en este lado oriental del Jordán''.

\bibverse{20} Moisés respondió: ``Si esto es lo que realmente harán, si
se preparan para la batalla bajo la dirección del Señor, \bibverse{21} y
si todas sus tropas cruzan el Jordán con el Señor hasta que haya
expulsado a sus enemigos delante de él, \bibverse{22} entonces una vez
que el país sea conquistado con la ayuda del Señor entonces podrán
regresar, y habrán cumplido sus obligaciones con el Señor y con Israel.
Serás dueño de esta tierra, que te ha sido concedida por el Señor.
\bibverse{23} Pero si no lo haces, claramente estarás pecando contra el
Señor, y las consecuencias de tu pecado te alcanzarán. \bibverse{24}
Adelante, construye ciudades para tus hijos y muros de piedra para tus
rebaños, pero asegúrate de hacer lo que has prometido''.

\bibverse{25} Las tribus de Gad y Rubén prometieron a Moisés, ``Señor,
nosotros, tus siervos, haremos lo que tú has ordenado. \bibverse{26}
Nuestras esposas e hijos, nuestro ganado y todos nuestros animales,
permanecerán aquí en los pueblos de Galaad. \bibverse{27} Pero nosotros,
tus siervos, estamos listos para la batalla, y todas nuestras tropas
cruzarán para luchar con la ayuda del Señor, tal como tú has dicho,
señor''.

\bibverse{28} Moisés les dio las siguientes instrucciones sobre ellos al
sacerdote Eleazar, a Josué, hijo de Nun, y a los jefes de familia de las
tribus de Israel. \bibverse{29} Moisés les dijo: ``Si las tribus de
Gaditas y Rubén cruzan el Jordán contigo, con todas sus tropas listas
para la batalla con la ayuda del Señor, y la tierra es conquistada a
medida que avanzas, entonces dales la tierra de Galaad para que la
posean. \bibverse{30} Pero si no se preparan para la batalla y cruzan
contigo, entonces deben aceptar su tierra asignada entre ustedes en el
país de Canaán.''

\bibverse{31} Las tribus de Gad y Rubén respondieron: ``Haremos lo que
el Señor nos ha dicho, como sus siervos. \bibverse{32} Cruzaremos y
entraremos en el país de Canaán listos para la batalla con la ayuda del
Señor, para que podamos tener nuestra parte de tierra asignada a este
lado del Jordán.''

\bibverse{33} Moisés dio a las tribus de Gad y Rubén y a la media tribu
de Manasés, hijo de José, el reino de Sehón, rey de los amorreos, y el
reino de Og, rey de Basán. Esta tierra incluía sus ciudades y sus
alrededores. \bibverse{34} Los pueblos de Gad reconstruyeron Dibon,
Ataroth, Aroer, \bibverse{35} Atarot-sofán, Jazer, Jogbeha,
\bibverse{36} Bet-nimra y Bet-arán como ciudades fortificadas, y
construyeron muros de piedra para sus rebaños.

\bibverse{37} El pueblo de Rubén reconstruyó Hesbón, Eleale, Quiriataim,
\bibverse{38} así como Nebo y Baal-meón (cambiando sus nombres), y
Sibma. De hecho, cambiaron el nombre de los pueblos que reconstruyeron.

\bibverse{39} Los descendientes de Maquir, hijo de Manasés, atacaron a
Galaad y lo capturaron. Expulsaron a los amorreos que vivían allí.
\bibverse{40} Entonces Moisés entregó a Galaad a la familia de Maquir,
hijo de Manasés, y se establecieron allí. \bibverse{41} Jair, un
descendiente de Manasés, atacó sus pueblos y los capturó. Los llamó las
Aldeas de Jair. \bibverse{42} Noba atacó a Kenat y la capturó, junto con
sus aldeas. La nombró Nobah en su honor.

\hypertarget{section-32}{%
\section{33}\label{section-32}}

\bibverse{1} Este es un registro de los viajes realizados por los
israelitas al salir de Egipto en sus divisiones tribales lideradas por
Moisés y Aarón. \bibverse{2} Moisés registró las diferentes partes de su
viaje según las instrucciones del Señor. Estos son los viajes que
hicieron listados en orden desde donde comenzaron:

\bibverse{3} Los israelitas dejaron Ramsés el día quince del primer mes,
el día después de la Pascua. Salieron triunfantes mientras todos los
egipcios observaban. \bibverse{4} Los egipcios enterraban a todos sus
primogénitos que el Señor había matado, porque el Señor había hecho caer
sus juicios sobre sus dioses. \bibverse{5} Los israelitas dejaron Ramsés
e instalaron un campamento en Sucot.

\bibverse{6} Se fueron de Sucot y acamparon en Etam, en la frontera con
el desierto.

\bibverse{7} Se alejaron de Etam, volviendo a Pi-hahiroth, frente a
Baal-zefón, y acamparon cerca de Migdol.

\bibverse{8} Se mudaron de Pi-hahirot\footnote{33.8 Ver Éxodo 14:2.}y
cruzó por el medio del mar hacia el desierto. Viajaron durante tres días
al desierto de Etham y establecieron un campamento en Marah.

\bibverse{9} Se desplazaron desde Mara y llegaron a Elim, donde había
doce manantiales de agua y setenta palmeras, y acamparon allí.

\bibverse{10} Se trasladaron de Elim y acamparon al lado del Mar Rojo.

\bibverse{11} Se trasladaron desde el Mar Rojo y acamparon en el
Desierto del Pecado.

\bibverse{12} Se trasladaron del desierto de Sin y acamparon en Dofca.

\bibverse{13} Se mudaron de Dofca y acamparon en Alús.

\bibverse{14} Se mudaron de Alús y acamparon en Refidím. No había agua
allí para que la gente bebiera.

\bibverse{15} Se fueron de Refidim y acamparon en el desierto del Sinaí.

\bibverse{16} Se fueron del desierto del Sinaí y acamparon en
Kibroth-hataava.

\bibverse{17} Se mudaron de Kibroth-hattaavah y acamparon en Hazerot.

\bibverse{18} Se trasladaron de Hazerot y establecieron un campamento en
Ritma.

\bibverse{19} Se trasladaron de Ritma y establecieron un campamento en
Rimón-fares.

\bibverse{20} Se trasladaron de Rimmon-fares y acamparon en Libna.

\bibverse{21} Se trasladaron de Libna y establecieron un campamento en
Rissa.

\bibverse{22} Se trasladaron de Rissa y establecieron un campamento en
Ceelata.

\bibverse{23} Se trasladaron de Ceelata y acamparon en el Monte Sefer.

\bibverse{24} Se trasladaron del Monte Sefer y acamparon en Harada.

\bibverse{25} Se trasladaron de Harada y acamparon en Macelot.

\bibverse{26} Se trasladaron de Macelot y acamparon en Tahat.

\bibverse{27} Se fueron de Tahat y acamparon en Tara.

\bibverse{28} Se mudaron de Tara y acamparon en Mitca.

\bibverse{29} Se mudaron de Mitca y acamparon en Hasmona.

\bibverse{30} Se fueron de Hasmona y acamparon en Moserot.

\bibverse{31} Se mudaron de Moserot y acamparon en Bene-jaacán.

\bibverse{32} Se mudaron de Bene-jaacán y acamparon en Hor-haggidgad.

\bibverse{33} Se trasladaron de Hor-haggidgad y acamparon en Jotbata.

\bibverse{34} Se mudaron de Jotbata y establecieron un campamento en
Abrona.

\bibverse{35} Se mudaron de Abrona y acamparon en Ezión-geber.

\bibverse{36} Se trasladaron de Ezion-geber y acamparon en Cades, en el
desierto de Zin.

\bibverse{37} Se trasladaron de Cades y acamparon en el monte Hor, en la
orilla de Edom. \bibverse{38} El sacerdote Aarón subió al monte Hor como
el Señor le había ordenado, y murió allí el primer día del quinto mes,
en el cuadragésimo año después de que los israelitas hubieran salido de
Egipto. \bibverse{39} Aarón tenía 123 años cuando murió en el Monte Hor.

\bibverse{40} (El rey cananeo de Arad, que vivía en el Néguev en el país
de Canaán, se enteró de que los israelitas estaban en camino).

\bibverse{41} Los israelitas se trasladaron del Monte Hor y
establecieron un campamento en Zalmona.

\bibverse{42} Se trasladaron de Zalmona y acamparon en Punón.

\bibverse{43} Se trasladaron de Punón y acamparon en Obot.

\bibverse{44} Se trasladaron de Oboth y acamparon en Iye-abarim, en la
frontera de Moab.

\bibverse{45} Se mudaron de Iye-abarim\footnote{33.45 Como se escribe en
  el versículo 21:11. Aquí el nombre se menciona como ``Iyim.''}y
acamparon en Dibon-gad.

\bibverse{46} Se mudaron de Dibon-gad y acamparon en Almon-diblataim.

\bibverse{47} Se mudaron de Almon-diblataim y acamparon en las montañas
de Abarim, frente a Nebo.

\bibverse{48} Se trasladaron de las montañas de Abarim y acamparon en
las llanuras de Moab, junto al Jordán, frente a Jericó.

\bibverse{49} Allí, en las llanuras de Moab, acamparon al lado del
Jordán, desde Beth-jesimot hasta Abel-sitim. \bibverse{50} Aquí fue
donde, en la llanura de Moab junto al Jordán, frente a Jericó, el Señor
le dijo a Moisés, \bibverse{51} Dile a los israelitas:'Tan pronto crucen
el Jordán y entren en el país de Canaán, \bibverse{52} deben expulsar a
todos los que viven en la tierra, destruir todas sus imágenes talladas e
ídolos de metal, y derribar todos sus templos paganos.\footnote{33.52
  ``Templos paganos'': literalmente, ``lugares altos.''} \bibverse{53}
Debes tomar el país y establecerte allí, porque te he dado la tierra y
te pertenece. \bibverse{54} Debes dividir la tierra y asignarla por
sorteo a las diferentes familias tribales. Dale una porción más grande a
una familia más grande, y una porción más pequeña a una familia más
pequeña. La asignación de cada uno se decide por sorteo, y todos ustedes
recibirán una asignación dependiendo de su tribu.

\bibverse{55} Pero si no expulsan a todos los que viven en la tierra,
las personas que dejen permanecer serán como arena en sus ojos y espinas
en sus costados. Les causarán problemas cuando se establezcan en el
país. \bibverse{56} Eventualmente, el castigo que planeé para ellos se
los infligiré a ustedes.''

\hypertarget{section-33}{%
\section{34}\label{section-33}}

\bibverse{1} El Señor le dijo a Moisés, \bibverse{2} Dales esta orden a
los israelitas: 'Cuando entren en el país de Canaán, se les asignarán
las posesiones con los siguientes límites:\footnote{34.2 Otros pasajes
  que incluyen demarcaciones de límites son: Josué 13:8-33, Josué
  14:1-19:51, Ezequiel 47:13-20.}

\bibverse{3} La extensión sur de su país será desde el desierto de Zin a
lo largo de la frontera de Edom. Su frontera sur correrá hacia el este
desde el final del Mar Muerto, \bibverse{4} cruzará al sur del Paso del
Escorpión, hasta Zin, y alcanzará su límite sur al sur de Cades-barnea.
Luego irá a Hazar-addar y a Azmon. \bibverse{5} TAllí la frontera girará
desde Azmon hasta el Wadi de Egipto,\footnote{34.5 Normalmente se
  identifica como Wadi El-Arish. No se cree que se refiera al Nilo.}
terminando en el Mar Mediterráneo.

\bibverse{6} Su frontera occidental será la costa del Mar Mediterráneo.
Este será su límite al oeste.

\bibverse{7} Tu frontera norte irá desde el Mar Mediterráneo hasta el
Monte Hor. \bibverse{8} Desde el Monte Hor la frontera irá a Lebo-hamat,
luego a Zedad, \bibverse{9} a Zifrón, terminando en Hazar-enan. Este
será su límite al norte.

\bibverse{10} Su frontera oriental irá directamente de Hazar-enan a
Sefan. \bibverse{11} Luego la frontera bajará de Sefam a Ribla en el
lado este de Aín. Pasará a lo largo de las laderas al este del Mar de
Galilea. \bibverse{12} Luego el límite bajará a lo largo del Jordán,
terminando en el Mar Muerto. Esta será su tierra con sus fronteras
circundantes.''

\bibverse{13} Moisés dio la orden a los israelitas, ``Asignen la
propiedad de esta tierra por sorteo. El Señor ha ordenado que sea
entregada a las nueve tribus y media. \bibverse{14} Las tribus de Rubén
y Gad, junto con la media tribu de Manasés, ya han recibido su
asignación. \bibverse{15} Estas dos tribus y media han recibido su
asignación en el lado este del Jordán, frente a Jericó.''

\bibverse{16} El Señor le dijo a Moisés, \bibverse{17} Estos son los
nombres de los hombres que se encargarán de asignar la propiedad de la
tierra para ustedes: Eleazar el sacerdote y Josué, hijo de Nun.
\bibverse{18} Que un líder de cada tribu ayude en la distribución de la
tierra. \bibverse{19} Estos son sus nombres:

De la tribu de Judá: Caleb, hijo de Jefone.

\bibverse{20} De la tribu de Simeón: Semuel, hijo de Amiud.

\bibverse{21} De la tribu de Benjamín: Elidad, hijo de Quislón.

\bibverse{22} Un líder de la tribu de Dan: Buqui, hijo de Jogli.

\bibverse{23} Un líder de la tribu de Manasés, hijo de José: Haniel,
hijo de Efod.

\bibverse{24} Un líder de la tribu de Efraín: Kemuel, hijo de Siftán.

\bibverse{25} Un líder de la tribu de Zabulón: Eli-zafán, hijo de
Parnac.

\bibverse{26} Un líder de la tribu de Isacar: Paltiel, hijo de Azán.

\bibverse{27} Un líder de la tribu de Aser: Ahiud, hijo de Selomi.

\bibverse{28} Un líder de la tribu de Neftalí: Pedael, hijo de Amiud''.

\bibverse{29} Estos son los nombres de los que el Señor puso a cargo de
la asignación de la propiedad de la tierra en el país de Canaán.

\hypertarget{section-34}{%
\section{35}\label{section-34}}

\bibverse{1} El Señor le habló a Moisés en las llanuras de Moab junto al
Jordán, frente a Jericó. Le dijo, \bibverse{2} Ordena a los israelitas
que provean de sus ciudades de asignación de tierras para que los
levitas vivan y pasten alrededor de las ciudades. \bibverse{3} Las
ciudades son para que vivan en ellas, y los pastos serán para sus
rebaños y para todo su ganado. \bibverse{4} Los pastos alrededor de las
ciudades que le des a los levitas se extenderán desde el muro mil codos
por todos lados. \bibverse{5} Mide dos mil codos fuera de la ciudad al
Este, dos mil al Sur, dos mil al Oeste y dos mil al Norte, con la ciudad
en el medio. Estas áreas serán sus pastos alrededor de las ciudades.

\bibverse{6} Seis de los pueblos que le das a los levitas serán pueblos
santuarios,\footnote{35.6 Ver también Josué 20.} donde una persona que
mata a alguien puede correr para protegerse. Además de estas ciudades,
dale a los levitas cuarenta y dos más . \bibverse{7} El número total de
pueblos que le darás a los levitas es de cuarenta y ocho, junto con sus
pastos. \bibverse{8} Las ciudades que asignes para ser entregadas a los
levitas serán tomadas del territorio de los israelitas, y tomarás más de
las tribus más grandes y menos las más pequeñas. El número será
proporcional al tamaño de la asignación de tierras de cada tribu''.

\bibverse{9} El Señor le dijo a Moisés, \bibverse{10} Dile a los
israelitas: 'Cuando cruces el Jordán hacia Canaán, \bibverse{11} escoge
pueblos como tus pueblos de santuario, para que una persona que mate a
alguien por error pueda correr allí. \bibverse{12} Estas ciudades serán
para ustedes santuario de los que buscan venganza, para que el asesino
no muera hasta que sea juzgado en un tribunal.

\bibverse{13} Las ciudades que elijan serán sus seis ciudades santuario.
\bibverse{14} Elijan tres ciudades al otro lado del Jordán y tres en
Canaán como ciudades de refugio. \bibverse{15} Estas seis ciudades serán
lugares de santuario para los israelitas y para los extranjeros o
colonos entre ellos, de modo que cualquiera que mate a una persona por
error pueda correr allí.

\bibverse{16} Pero si alguien golpea deliberadamente a alguien con algo
hecho de hierro y lo mata, esa persona es un asesino y debe ser
ejecutado. \bibverse{17} Si alguien tomaun trozo de piedra que pueda ser
usado como arma y golpea a alguien con ella, y lo mata, esa persona es
un asesino y debe ser ejecutado. \bibverse{18} Si alguien tomaun trozo
de madera que pueda ser usado como arma y golpea a alguien con ella, y
lo mata, esa persona es un asesino y debe ser ejecutado.

\bibverse{19} El vengador\footnote{35.19 ``El vengador:'' este era el
  pariente más cercano a la víctima: literalmente, ``el vengador de la
  sangre''.} debe ejecutar al asesino. Cuando encuentre al asesino, lo
matará. \bibverse{20} De la misma manera, si uno odia al otro y lo
derriba o le tira algo deliberadamente, y lo mata; \bibverse{21} o si
alguien golpea a otro con su mano y mueren, el que lo golpeó debe ser
ejecutado porque es un asesino. Cuando el vengador encuentra al asesino,
debe matarlo.

\bibverse{22} Pero si alguien derriba a otro sin querer y sin odiarlo, o
le tira algo sin querer hacerle daño, \bibverse{23} o deja caer
descuidadamente una piedra pesada que lo mata, pero no como enemigo o
con intención de hacerle daño, \bibverse{24} entonces la comunidad debe
juzgar entre el asesino y el vengador siguiendo este reglamento.
\bibverse{25} El tribunal debe proteger al asesino de ser atacado por el
vengador y debe devolverlo a la ciudad santuario a la que corrió, y debe
permanecer allí hasta la muerte del sumo sacerdote, que fue ungido con
el óleo santo.

\bibverse{26} Pero si el asesino sale de los límites de la ciudad
santuario a la que huyó, \bibverse{27} y el vengador lo encuentra fuera
de su ciudad santuario y lo mata, entonces el vengador no será culpable
de asesinato, \bibverse{28} porque el asesino tiene que permanecer en su
ciudad santuario hasta la muerte del sumo sacerdote. Sólo después de la
muerte del sumo sacerdote se les permite volver a la tierra que poseen.
\bibverse{29} Estas normas se aplican a todas las generaciones futuras
dondequiera que vivan.

\bibverse{30} Si alguien mata a una persona, el asesino debe ser
ejecutado basándose en las pruebas aportadas por los testigos, en
plural. Nadie debe ser ejecutado basándose en la evidencia dada por un
solo testigo.

\bibverse{31} No se aceptará el pago en lugar de ejecutar a un asesino
que ha sido declarado culpable. \bibverse{32} Tampoco se le permite
aceptar el pago de una persona que huye a una ciudad santuario y le
permite regresar y vivir en su propia tierra antes de la muerte del sumo
sacerdote.

\bibverse{33} No contaminen la tierra donde viven porque el
derramamiento de sangre contamina la tierra, y la tierra donde se
derrama la sangre no puede ser purificada excepto por la sangre de quien
la derrama. \bibverse{34} No hagas impura la tierra donde vives porque
yo también vivo allí. Yo soy el Señor, y vivo con los israelitas.''

\hypertarget{section-35}{%
\section{36}\label{section-35}}

\bibverse{1} Los jefes de familia de los descendientes de Galaad, hijo
de Maquir, hijo de Manasés, una de las tribus de José, vinieron y
hablaron ante Moisés y con los líderes israelitas, que eran otros jefes
de familia. \bibverse{2} les dijeron: ``Cuando el Señor te ordenó, mi
señor, que asignaras la propiedad de la tierra a los israelitas por
sorteo, también te ordenó que dieras la parte de nuestro hermano
Zelofehad a sus hijas. \bibverse{3} Sin embargo, si se casan con hombres
de las otras tribus de Israel, su asignación les quitaría la parte de
nuestros padres y se le añadiría a la tribu de los hombres con los que
se casan. Esa parte de nuestra asignación sería pérdida para nosotros.
\bibverse{4} Así que cuando llegue el Jubileo para los israelitas, su
asignación se añadirá a la tribu con la que se casen, y se le quitará a
la tribu de nuestros padres''.

\bibverse{5} Siguiendo lo que el Señor le dijo, Moisés dio estas órdenes
a los israelitas, ``Lo que dice la tribu de los hijos de José es
correcto. \bibverse{6} Esto es lo que el Señor ha ordenado con respecto
a las hijas de Zelofehad: Pueden casarse con quien quieran siempre que
lo hagan dentro de una familia que pertenezca a la tribu de su padre.
\bibverse{7} No se podrá pasar ninguna asignación de tierras en Israel
de tribu a tribu, porque cada israelita debe aferrarse a la asignación
de la tribu de su padre. \bibverse{8} Toda hija que posea una herencia
de cualquier tribu israelita debe casarse dentro de un clan de la tribu
de su padre, de modo que todo israelita poseerá la herencia de sus
padres. \bibverse{9} No se podrá pasar ninguna asignación de tierras de
una tribu a otra, pues cada tribu israelita debe mantener su propia
asignación''.

\bibverse{10} Las hijas de Zelofehad siguieron las órdenes del Señor a
través de Moisés. \bibverse{11} Maala, Tirsa, Hogla, Milca y Noa, hijas
de Zelofehad, primos casados por parte de su padre. \bibverse{12} Se
casaron dentro de las familias de los descendientes de Manasés, hijo de
José, y su asignación de tierras permaneció dentro de la tribu de su
padre.

\bibverse{13} Estas son las órdenes y normas que el Señor dio a los
israelitas a través de Moisés en las llanuras de Moab, junto al Jordán,
frente a Jericó.
