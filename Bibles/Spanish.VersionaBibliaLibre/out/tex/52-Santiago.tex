\hypertarget{section}{%
\section{1}\label{section}}

\bibverse{1} Esta carta viene de parte de Santiago, siervo de Dios y del
Señor Jesucristo. Es enviada a las doce tribus dispersas en el
extranjero\footnote{\textbf{1:1} Refiriéndose a las doce tribus de
  Israel, por supuesto.}. ¡Mis mejores deseos para ustedes!

\bibverse{2} Amigos míos, elijan estar felices aun cuando se atraviesen
todo tipo de problemas en su camino, \bibverse{3} porque ustedes saben
que la paciencia surge al enfrentar desafíos en cuanto a su fe en Dios.
\bibverse{4} Que su paciencia se fortalezca tanto como sea posible, para
que estén completamente maduros, sin ningún defecto.

\bibverse{5} Si alguno de ustedes necesita sabiduría, pídala a Dios, que
da a todos generosamente y sin retenciones. \bibverse{6} Pero cuando
pidan, recuerden confiar en Dios. Háganlo sin dudas. Porque quien duda
es como las olas del mar que son llevadas de un lado al otro por el
viento. \bibverse{7} Y ninguna persona que sea así debe pensar que podrá
recibir algo del Señor, \bibverse{8} porque sus mentes van errantes, y
son inestables en todo lo que hacen.

\bibverse{9} Los creyentes que han nacido en la pobreza deben
enorgullecerse en la posición de grandeza que se les ha dado,
\bibverse{10} mientras que los ricos deberían ``jactarse'' en la humilde
posición que ahora tienen\footnote{\textbf{1:10} 1:9, 10. Refiriéndose
  principalmente a la manera como Dios los ve, no necesariamente a su
  posición en la sociedad\ldots{}}, pues se marchitarán como flores en
el campo. \bibverse{11} Porque el sol se levanta junto al viento
devastador y chamusca la hierba. Las flores se caen y su belleza muere.
Del mismo modo, todo lo que el rico obtiene se marchitará.

\bibverse{12} Feliz es quien soporta con paciencia la tentación, porque
cuando haya demostrado que es fiel, recibirá la corona de vida que Dios
promete a los que le aman. \bibverse{13} Cuando alguien es tentado, no
debe decir: ``Estoy siendo tentado por Dios.'' Porque Dios no es tentado
por el mal, ni él tienta a nadie. \bibverse{14} Las tentaciones vienen
de nuestros propios malos deseos que nos descarrían y nos atrapan.
\bibverse{15} Tales deseos nos llevan al pecado, y el pecado, al haberse
desarrollado en plenitud, causa la muerte.

\bibverse{16} Mis queridos amigos, no se dejen engañar. \bibverse{17}
Todo lo que es bueno, todo don perfecto, viene de arriba, y desciende
del Padre que hizo las luces del cielo. A diferencia de ellas, él no
cambia, él no varía ni arroja sombras\footnote{\textbf{1:17} Santiago
  parece referirse a los distintos movimientos de cuerpos celestes y
  eclipses (sombras).}. \bibverse{18} Él eligió darnos una nueva vida
por medio de la palabra de verdad, para que de toda su creación nosotros
fuésemos muy especiales para él\footnote{\textbf{1:18} A menudo se cree
  que se refiere al llamado de Dios y su provisión para que volvamos a
  nacer espiritualmente.}.

\bibverse{19} Recuerden esto, mis queridos amigos: todos deberían ser
prontos para escuchar, pero lentos para hablar y lentos para enojarse,
\bibverse{20} porque el enojo humano no refleja el verdadero carácter de
Dios\footnote{\textbf{1:20} Literalmente, ``alcanzar la justicia de
  Dios.''}. \bibverse{21} Así que despojémonos de todo lo que es sucio y
maligno. Acepten humildemente la palabra que ha sido implantada en
ustedes, porque esto es lo que puede salvarlos. \bibverse{22} Hagan, más
bien, lo que dice la palabra. No escuchen solamente ni se engañen
ustedes mismos. \bibverse{23} Si solo oyen la palabra y no la ponen en
práctica, es como si miraran sus rostros en un espejo. \bibverse{24} Ahí
se ven a sí mismos, pero luego se van, y de inmediato olvidan cómo se
veían. \bibverse{25} Pero si miran la ley perfecta de la libertad, y la
siguen, no como quien solo escucha y olvida, sino como quien la pone en
práctica, entonces serán bendecidos en lo que hagan. \bibverse{26} Si
piensan que son piadosos, pero no controlan lo que dicen, se están
engañando a ustedes mismos y su religión no tiene sentido. \bibverse{27}
Ante los ojos de nuestro Dios y Padre, la religión pura y genuina
consiste en visitar huérfanos y viudas que sufren, y guardarse de la
contaminación del mundo.

\hypertarget{section-1}{%
\section{2}\label{section-1}}

\bibverse{1} Mis amigos, como creyentes con fe en nuestro glorioso Señor
Jesucristo, ustedes no deben mostrar favoritismo. \bibverse{2} Imaginen
que a su sinagoga llega un hombre usando anillos de oro y ropas finas, y
luego entra un hombre pobre vestido de harapos. \bibverse{3} Si atienden
de manera especial al hombre bien vestido y le dicen: ``Por favor,
siéntate aquí en esta silla de honor,'' mientras que al pobre le dicen:
``Siéntate allá, o siéntate en el piso, a mis pies;'' \bibverse{4}
¿acaso no han discriminado y juzgado con razones equivocadas?
\bibverse{5} Escuchen, mis queridos amigos: ¿Acaso Dios no eligió a los
que el mundo considera pobres para que fueran ricos en su fe en él, y
para que heredaran el reino que prometió a quienes lo aman? \bibverse{6}
Pero ustedes han tratado al pobre de manera vergonzosa. ¿No son los
ricos quienes los oprimen y los arrastran a las cortes? \bibverse{7}
¿Acaso no son ellos quienes insultan el honorable nombre\footnote{\textbf{2:7}
  A menudo se entiende que es el nombre de Jesús.} de Aquél a quien
pertenecen y los llamó?

\bibverse{8} Si ustedes realmente observan la ley real de la Escritura:
``Amarás a tu prójimo como a ti mismo,'' entonces hacen bien.
\bibverse{9} Pero si demuestran favoritismo, están pecando. La ley los
condena como culpables de su incumplimiento. \bibverse{10} Quien observa
todo lo que está en la ley pero incumple una sola parte, es culpable de
incumplirla toda. \bibverse{11} Dios les dijo que no cometan adulterio,
y también les dijo que no maten. De modo que si no cometen adulterio
pero matan, de igual modo son quebrantadores de la ley. \bibverse{12}
Deben hablar y actuar como personas que serán juzgadas por la ley de la
libertad. \bibverse{13} Todo aquél que no muestra misericordia, será
juzgado sin misericordia. ¡Sin embargo, la misericordia triunfa sobre el
juicio!\footnote{\textbf{2:13} El significado exacto de este versículo
  es objeto de debate, pero el punto fundamental es enfatizar el
  carácter misericordioso de Dios.}

\bibverse{14} Amigos míos, ¿qué de bueno hay en decir que tenemos fe en
Dios si no hacemos lo correcto? ¿Puede salvarnos tal ``fe''?
\bibverse{15} Si un hermano o hermana no tiene ropas, o comida para el
día, \bibverse{16} y tú vas y le dices: ``¡Que Dios te bendiga!
¡Mantente cálido y disfruta de la comida!'' pero no provees lo que esta
persona necesita para sobrevivir, ¿qué de bueno hay en eso?
\bibverse{17} Porque la fe basada en la confianza en Dios por sí misma
está muerta y no sirve para nada si no haces lo recto.

\bibverse{18} Hay quien podría debatirme: ``Tú tienes tu fe en Dios pero
yo tengo mis buenas obras.'' Pues bien, ¡muéstrame tu fe en Dios sin
buenas obras, y yo te mostraré mi fe en Dios con mis buenas obras!
\bibverse{19} ¿Tú crees que Dios es un solo Dios? Eso es bueno, pero los
demonios también creen en Dios, ¡y se aterran de él! \bibverse{20}
¡Ustedes son necios! ¿No saben que la fe en Dios sin hacer lo recto no
tiene sentido? \bibverse{21} ¿No fue nuestro Padre Abrahán
justificado\footnote{\textbf{2:21} O ``probado justo.''} por lo que hizo
al ofrecer a su hijo Isaac en un altar? \bibverse{22} Sepan que su fe en
Dios iba de la mano con lo que hizo, y por medio de lo que hizo su fe en
Dios fue completa. \bibverse{23} De este modo, se cumplió lo que dice la
Escritura: ``Abrahán creyó en Dios, y esto le fue contado como obra de
justicia, y fue llamado amigo de Dios.''

\bibverse{24} Vemos entonces que somos justificados por lo que hacemos y
no solo por nuestra fe en Dios. \bibverse{25} Del mismo modo, ¿no fue
justificada Rahab, la prostituta, por lo que hizo cuando cuidó de los
mensajeros y los envió luego por un camino distinto? \bibverse{26} Así
como el cuerpo está muerto sin el espíritu, la fe en Dios está muerta si
no obramos con justicia.

\hypertarget{section-2}{%
\section{3}\label{section-2}}

\bibverse{1} Mis amigos, no muchos de ustedes deberían ser maestros,
porque ustedes saben que quien enseña tiene una responsabilidad mayor
ante el juicio. \bibverse{2} Todos cometemos errores de muchas maneras.
El que no comete errores en lo que dice es realmente una persona buena
que puede mantener todo su cuerpo bajo control. \bibverse{3} Nosotros
ponemos frenos en nuestra boca como los caballos para que nos obedezcan,
y así poder dirigirlos hacia donde queremos. \bibverse{4} Miremos
también los barcos: aunque son muy grandes y son impulsados por vientos
fuertes, son conducidos por un pequeño timón hacia la dirección que el
piloto quiere ir.

\bibverse{5} Del mismo modo, la lengua es una parte del cuerpo muy
pequeña, ¡pero hace grandes alardes! ¡Piensen cuán grande incendio puede
provocar una pequeña llama! \bibverse{6} Y la lengua es una llama. Es
una espada del mal en medio de las partes del cuerpo. Puede estropearte
por completo como persona, y puede derrumbar toda tu vida, pues la
enciende el fuego de Gehena\footnote{\textbf{3:6} Gehenna: el basurero
  que estaba afuera de Jerusalén, donde se quemaba la basura. Esta
  palabra se usa de manera simbólica como destino final de los malvados.}.
\bibverse{7} La gente puede dominar todo tipo de animales, ya sean aves,
reptiles, y criaturas del mar, \bibverse{8} pero nadie puede dominar la
lengua. Porque es maligna, difícil de controlar, llena de veneno mortal.
\bibverse{9} La misma lengua que usamos para bendecir a nuestro Señor y
Padre, la usamos para maldecir a otras personas que están hechas a
imagen de Dios. \bibverse{10} ¡Emanan bendiciones y maldiciones de la
misma boca! Amigos, ¡esto no debe ser así! \bibverse{11} ¿Acaso puede
brotar de la misma fuente agua dulce y amarga a la vez? \bibverse{12}
Amigos míos, así como una higuera no puede producir olivas, y una viña
no puede producir higos, una fuente de agua salada no puede producir
agua dulce y fresca.

\bibverse{13} ¿Quién entre ustedes tiene sabiduría y entendimiento? Pues
permita que su buen vivir demuestre lo que hace, actuando rectamente,
con sabia bondad y consideración.

\bibverse{14} Pero si tienes celos amargos y ambición egoísta en tu
corazón, no te jactes de ello ni quieras mentirle a la verdad.
\bibverse{15} Esta clase de ``sabiduría'' no viene de arriba, sino que
es terrenal, carente de espiritualidad, y demoníaca. \bibverse{16}
Dondequiera haya celos y ambición egoísta, también habrá confusión y
todo tipo de prácticas malas.

\bibverse{17} Sin embargo, la sabiduría que viene de arriba es pura
sobre todas las cosas, y también trae paz. Es noble y abierta a la
razón. Está llena de misericordia y produce cosas buenas. Es genuina y
no hipócrita. \bibverse{18} Los que siembran paz recogerán la paz de lo
que es recto en verdad.

\hypertarget{section-3}{%
\section{4}\label{section-3}}

\bibverse{1} ¿De dónde surgen las contiendas y discusiones que hay entre
ustedes? ¿Acaso no son por las pasiones sensuales que luchan dentro de
ustedes? \bibverse{2} Están ardiendo de deseo, pero no reciben lo que
quieren. Son capaces de matar por lo que anhelan con lujuria, pero no
encuentran lo que buscan. Pelean y discuten pero no logran nada, porque
no lo piden en oración. \bibverse{3} Oran, pero no reciben nada, porque
oran con motivos equivocados, queriendo gastar lo que reciban en
placeres egoístas. \bibverse{4} ¡Adúlteros! ¿No se dan cuenta que la
amistad con el mundo es enemistad contra Dios? Los que quieren ser
amigos del mundo se convierten en enemigos de Dios.

\bibverse{5} ¿Creen que la Escritura no habla en serio cuando dice que
el espíritu que puso en nosotros es celoso en gran manera\footnote{\textbf{4:5}
  O ``Dios ha puesto en nosotros un espíritu lleno de fuertes deseos.''}?
\bibverse{6} Pero Dios nos da todavía más gracia, como dice la
Escritura: ``Dios está en contra de los arrogantes, pero da gracia a los
humildes.'' \bibverse{7} Colóquense, pues, bajo la dirección de Dios.
Confronten al enemigo, y él huirá de ustedes. \bibverse{8} Acérquense a
Dios y él se acercará a ustedes. Laven sus manos, pecadores. Purifiquen
sus corazones, ustedes que tienen lealtades divididas. \bibverse{9}
Muestren algo de remordimiento, lloren y laméntense. Cambien su risa por
lamento, y su alegría por tristeza. \bibverse{10} Sean humildes ante el
Señor y él los exaltará.

\bibverse{11} Amigos, no hablen mal unos de otros. Todo el que critica a
un hermano creyente y lo condena,\footnote{\textbf{4:11} O ``jueces.''}
critica y condena la ley. Si ustedes condenan la ley, entonces no la
están cumpliendo, porque están actuando como jueces. \bibverse{12} Hay
un solo dador de la ley y juez, el único que puede salvarnos o
destruirnos, así que, ¿quién eres tú para juzgar a tu prójimo?

\bibverse{13} Atiendan, ustedes los que dicen: ``Hoy o mañana iremos a
tal y tal ciudad, pasaremos un año allí haciendo negocios y obtendremos
ganancia.'' \bibverse{14} ¡Ustedes no saben qué pasará mañana! ¿Acaso
qué es su vida? Es apenas una niebla que aparece por un poco tiempo y
luego se va. \bibverse{15} Lo que deberían decir es: ``Si Dios quiere,
viviremos de esta manera, y haremos planes para hacer aquello.''
\bibverse{16} Pero ahora están solo llenos de ideas vanas. Y toda esta
jactancia es maligna. \bibverse{17} Porque es pecado si sabes hacer lo
bueno y no lo haces.

\hypertarget{section-4}{%
\section{5}\label{section-4}}

\bibverse{1} ¡Ustedes, ricos! Deberían llorar y lamentar todos los
problemas que les vienen encima. \bibverse{2} Toda su riqueza está
podrida, y sus ropas han sido devoradas por polillas. \bibverse{3} Su
oro y su plata están corroídos, y la corrosión hablará en contra de
ustedes, devorando sus carnes como fuego. Ustedes han amontonado
riquezas en los últimos días. \bibverse{4} Miren, los salarios de sus
obreros del campo que han estafado ahora claman contra ustedes, y los
lamentos de los obreros han llegado a los oídos del Señor Todopoderoso.
\bibverse{5} Han disfrutado una vida de lujos aquí en la tierra, llenos
de placer y autocomplacencia, engordándose para el día del sacrificio.
\bibverse{6} Han condenado y han asesinado al inocente que ni siquiera
les opuso su resistencia.

\bibverse{7} Amigos, sean pacientes y esperen el regreso del Señor.
Consideren al agricultor que espera con paciencia la preciosa cosecha de
la tierra mientras crece con la lluvia temprana y tardía. \bibverse{8}
Ustedes también necesitan ser pacientes. Manténganse fuertes porque el
regreso del Señor está cerca. \bibverse{9} Amigos míos, no se quejen
unos de otros, para que no sean juzgados. ¡Miren, el juez está justo a
las puertas! \bibverse{10} Tomen como ejemplo a los profetas, amigos
míos. Miren cómo hablaban en nombre del Señor mientras sufrían y
esperaban con paciencia. \bibverse{11} Observen que siempre decimos que
son benditos los que perseveran. Han oído hablar de la paciencia de Job,
y han visto cómo el Señor condujo todo a un fin positivo, porque el
Señor está lleno de compasión y misericordia.

\bibverse{12} Por encima de todo, amigos, no juren. No juren por el
cielo, ni por la tierra, ni hagan ningún otro tipo de juramento.
Simplemente digan sí, o no, para que no caigan en condenación.
\bibverse{13} ¿Está sufriendo alguno entre ustedes? Ore. ¿Hay alguien
alegre entre ustedes también? Cante canciones de alabanza. \bibverse{14}
¿Está alguno enfermo? Llamen a los ancianos de iglesia para que oren y
le unjan con aceite en el nombre del Señor. \bibverse{15} Porque tal
oración, con fe, sanará a los enfermos, y el Señor los hará estar bien.
Y si ha cometido pecados, le serán perdonados. \bibverse{16} Admitan
unos delante de otros los errores que han cometido, y oren unos por
otros para que sean sanados. La oración sincera de los justos es eficaz.
\bibverse{17} Elías era un hombre que tenía la misma naturaleza humana
que nosotros. Él oró con sinceridad para que no lloviera, y no llovió en
la tierra durante tres años y medio. \bibverse{18} Luego oró una vez
más, y el cielo envió la lluvia sobre la tierra, y la tierra dio su
cosecha.

\bibverse{19} Amigos míos, si alguno de ustedes se descarría de la
verdad y alguien le trae de vuelta, \bibverse{20} háganle saber que todo
el que rescata a un pecador del error de su camino lo salvará de la
muerte y ganará perdón de muchos pecados.
