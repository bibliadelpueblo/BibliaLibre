\hypertarget{section}{%
\section{1}\label{section}}

\bibverse{1} Esta carta viene de Pablo, siervo de Jesucristo. Fui
llamado por Dios para ser apóstol. Él me designó para anunciar la buena
noticia \bibverse{2} que anteriormente había prometido a través de sus
profetas en las Sagradas Escrituras. \bibverse{3} La buena noticia es
sobre su Hijo, cuyo antepasado fue David, \bibverse{4} pero que fue
revelado como Hijo de Dios por medio de su resurrección de los muertos
por el poder del Espíritu Santo. Él es Jesucristo, nuestro Señor.
\bibverse{5} Fue a través de él que recibí el privilegio de convertirme
en apóstol para llamar a todas las naciones a creer en él y obedecerle.
\bibverse{6} Ustedes también hacen parte de los que fueron llamados a
pertenecer a Jesucristo.

\bibverse{7} Les escribo a todos ustedes que están en Roma, que son
amados de Dios y están llamados para ser su pueblo especial. ¡Gracia y
paz a ustedes de parte de Dios nuestro Padre y del Señor Jesucristo!

\bibverse{8} Permítanme comenzar diciendo que agradezco a mi Dios por
medio de Jesucristo por todos ustedes, porque en todo el mundo se habla
acerca de la forma en que ustedes creen en Dios. \bibverse{9} Siempre
estoy orando por ustedes, tal como Dios mismo puede confirmarlo, el Dios
al cual sirvo con todo mi corazón al compartir la buena noticia de su
Hijo. \bibverse{10} En mis oraciones siempre le pido que pronto pueda ir
a verlos, si es su voluntad. \bibverse{11} Realmente deseo visitarlos y
compartir con ustedes una bendición espiritual para fortalecerlos.
\bibverse{12} Así podemos animarnos unos a otros por medio de la fe que
cada uno tiene en Dios, tanto la fe de ustedes como la mía.
\bibverse{13} Quiero que sepan, mis hermanos y hermanas, que a menudo he
hecho planes para visitarlos, pero me fue imposible hacerlo hasta hora.
Quiero ver buenos frutos espirituales entre ustedes así como los he
visto entre otros pueblos\footnote{\textbf{1:13} Literalmente, ``los
  gentiles.''}. \bibverse{14} Porque tengo la obligación de trabajar
tanto para los civilizados como los incivilizados, tanto para los
educados como los no educados. \bibverse{15} Es por eso que en verdad
tengo un gran deseo de ir a Roma y compartir la buena noticia con
ustedes.

\bibverse{16} Sin lugar a dudas, no me avergüenzo de la buena noticia,
porque es poder de Dios para salvar a todos los que creen en él, primero
a los judíos, y luego a todos los demás también. \bibverse{17} Porque en
la buena noticia Dios se revela como bueno y justo\footnote{\textbf{1:17}
  Literalmente, ``la justicia de Dios.''}, fiel desde el principio hasta
el fin. Tal como lo dice la Escritura: ``Los que son justos viven por la
fe en él\footnote{\textbf{1:17} Las palabras reales en el texto original
  son: ``el que es recto vivirá por fe.''}.''

\bibverse{18} La hostilidad\footnote{\textbf{1:18} Literalmente,
  ``ira.'' Existen debates en cuanto a la atribución de emociones
  humanas negativas a Dios.} de Dios se revela dese el cielo contra
aquellos que son impíos e injustos, contra aquellos que sofocan la
verdad con sus malas obras. \bibverse{19} Lo que puede llegar a saberse
de Dios es obvio, porque él se los ha mostrado claramente. \bibverse{20}
Desde la creación del mundo, los aspectos invisibles de Dios---su poder
y divinidad eternos---son claramente visibles en lo que él hizo. Tales
personas no tienen excusa, \bibverse{21} porque aunque conocieron a
Dios, no lo alabaron ni le agradecieron, sino que su pensamiento
respecto a Dios se convirtió en necedad, y la oscuridad llenó sus mentes
vacías. \bibverse{22} Y aunque aseguraban ser sabios, se volvieron
necios. \bibverse{23} Cambiaron la gloria del Dios inmortal por ídolos,
imágenes de seres, aves, animales y reptiles. \bibverse{24} Así que Dios
los dejó a merced de los malos deseos de sus mentes depravadas, y ellos
se hicieron, unos a otros, cosas vergonzosas y degradantes.
\bibverse{25} Cambiaron la verdad de Dios por una mentira, adorando y
sirviendo criaturas en lugar del Creador, quien es digno de alabanza por
siempre. Amén.

\bibverse{26} Por eso Dios los dejó a merced de sus malos deseos. Sus
mujeres cambiaron el sexo natural por lo que no es natural,
\bibverse{27} y del mismo modo los hombres renunciaron al sexo con
mujeres y ardieron en lujuria unos con otros. Los hombres hicieron cosas
indecentes unos con otros, y como resultado de ello sufrieron las
consecuencias inevitables de sus perversiones. \bibverse{28} Como no
consideraron la importancia de conocer a Dios, él los dejó a merced de
su forma de pensar inútil e infiel, y dejó que hicieran lo que nunca
debe hacerse. \bibverse{29} Se llenaron de toda clase de perversiones:
maldad, avaricia, odio, envidia, asesinatos, peleas, engaño, malicia, y
chisme. \bibverse{30} Son traidores y odian a Dios. Son arrogantes,
orgullosos y jactanciosos. Idean nuevas formas de pecar. Se rebelan
contra sus padres. \bibverse{31} No quieren entender, no cumplen sus
promesas, no muestran ningún tipo de bondad o compasión. \bibverse{32}
Aunque conocen claramente la voluntad de Dios, hacen cosas que merecen
la muerte. Y no solo hacen estas cosas sino que apoyan a otros para que
las hagan.

\hypertarget{section-1}{%
\section{2}\label{section-1}}

\bibverse{1} Así que si juzgas a otros, no tienes excusa, quienquiera
que seas. Pues en todo lo que condenas a otros, te estás juzgando a ti
mismo, porque tú haces las mismas cosas. \bibverse{2} Sabemos que el
juicio de Dios sobre aquellos que hacen tales cosas está basado en la
verdad. \bibverse{3} Pero cuando tú los juzgas, ¿realmente crees que de
alguna manera podrás escapar del juicio de Dios? \bibverse{4} ¿O es que
menosprecias su maravillosa bondad y tolerancia, sin darte cuenta de que
Dios, en su bondad, está tratando de conducirte al arrepentimiento?
\bibverse{5} Ahora por tu corazón endurecido y tu rechazo al
arrepentimiento, estás empeorando tu situación para el día de la
recompensa, cuando se demuestre la rectitud del juicio de Dios.
\bibverse{6} Dios se encargará de que todos reciban lo que merecen,
conforme a lo que han hecho. \bibverse{7} Así que los que han seguido
haciendo lo correcto, recibirán gloria, honor, inmortalidad y vida
eterna. \bibverse{8} Pero los que solo piensan en sí mismos, rechazando
la verdad y eligiendo deliberadamente hacer el mal, recibirán castigo
con furia y hostilidad. \bibverse{9} Todos los que hacen el mal tendrán
pena y sufrimiento. Primero los del pueblo judío, y luego los
extranjeros también. \bibverse{10} Pero todos los que hacen lo bueno
tendrán gloria, honor y paz. Primero los del pueblo judío, y luego los
extranjeros también. \bibverse{11} Pues Dios no tiene favoritos.

\bibverse{12} Aquellos que pecan aunque no tienen la ley
escrita\footnote{\textbf{2:12} Refiriéndose a la ley escrita por Moisés.
  Los que no tiene la ley escrita son los ``extranjeros,'' y los que sí
  tienen la ley escrita son los judíos.} están perdidos, pero aquellos
que pecan y sí tienen la ley escrita, serán condenados por esa misma
ley. \bibverse{13} Porque el solo hecho de oír lo que dice la ley no nos
hace justos ante los ojos de Dios. Los que hacen lo que dice la ley son
los que reciben justificación. \bibverse{14} Los extranjeros no tienen
la ley escrita, pero cuando hacen por instinto lo que la ley dice, están
siguiendo la ley aunque no la tengan. \bibverse{15} De esta manera,
ellos demuestran cómo obra la ley que está escrita en sus corazones.
Pues cuando piensan en lo que están haciendo, su conciencia los acusa
por hacer el mal o los defiende por hacer el bien. \bibverse{16} La
buena noticia que yo les comparto es que viene un día cuando Dios
juzgará, por medio de Jesucristo, los pensamientos secretos de todos.

\bibverse{17} ¿Qué hay de ti, que te llamas judío? Confías en la ley
escrita y te jactas de tener una relación especial con Dios.
\bibverse{18} Conoces su voluntad. Haces lo recto porque has aprendido
de la ley. \bibverse{19} Estás completamente seguro de que puedes guiar
a los ciegos y que eres luz para los que están en oscuridad.
\bibverse{20} Crees que puedes corregir a los ignorantes y que eres un
maestro de ``niños,'' porque conoces por la ley toda la verdad que
existe. \bibverse{21} Y si estás tan afanado en enseñar a otros, ¿por
qué no te enseñas a ti mismo? Puedes decirle a la gente que no robe,
pero ¿estás tú robando? \bibverse{22} Puedes decirle a la gente que no
cometa adulterio, pero ¿estás tú adulterando? Puedes decirle a la gente
que no adore ídolos, pero ¿profanas tú los templos\footnote{\textbf{2:22}
  O, ``robar templos.''}?

\bibverse{23} Te jactas de tener la ley, pero ¿acaso no das una imagen
distorsionada de Dios al quebrantarla? \bibverse{24} Como dice la
Escritura, ``Por tu causa es difamado el carácter de Dios entre los
extranjeros.''\footnote{\textbf{2:24} Isaías 52:5. Literalmente, ``el
  nombre de Dios,'' que fundamentalmente tiene que ver con su carácter.}
\bibverse{25} Estar circuncidado\footnote{\textbf{2:25} La circuncisión,
  dada por Dios a Israel en el Antiguo Testamento, era una señal de que
  ellos eran su pueblo especial.} solo tiene valor si haces lo que dice
la ley. Pero si quebrantas la ley, tu circuncisión es tan inútil como la
de aquellos que no están circuncidados. \bibverse{26} Si un hombre que
no está circuncidado\footnote{\textbf{2:26} No circuncidado, queriendo
  decir que no era judío, o que era un ``extranjero.''} guarda la ley,
debe considerársele como si lo estuviera aunque no lo esté.
\bibverse{27} Los extranjeros incircuncisos que guardan la ley te
condenarán si tú la quebrantas, aunque tengas la ley y estés
circuncidado. \bibverse{28} No es lo externo lo que te convierte en
judío; no es la señal física de la circuncisión. \bibverse{29} Lo que te
hace judío es lo que llevas por dentro, una ``circuncisión del corazón''
que no sigue la letra de la ley sino la del Espíritu. Alguien así busca
alabanza de Dios y no de la gente.

\hypertarget{section-2}{%
\section{3}\label{section-2}}

\bibverse{1} ¿Tienen entonces los judíos alguna ventaja? ¿Tiene algún
beneficio la circuncisión? \bibverse{2} Sí. ¡Hay muchos beneficios! En
primer lugar, el mensaje de Dios fue confiado a los judíos. \bibverse{3}
¿Qué pasaría si alguno de ellos no creyera en Dios? ¿Acaso su falta de
fe en Dios anula la fidelidad de Dios? \bibverse{4} ¡Claro que no!
Incluso si todos los demás mienten, Dios siempre dice la verdad. Como
dice la Escritura: ``Quedará demostrado que tienes la razón en lo que
dices, y ganarás tu caso\footnote{\textbf{3:4} O, ``serás vindicado.''}
cuando seas juzgado.''\footnote{\textbf{3:4} Salmos 51:4.}

\bibverse{5} Pero si el hecho de que estamos equivocados ayuda a
demostrar que Dios está en lo correcto, ¿qué debemos concluir? ¿Que Dios
se equivoca al pronunciar juicio sobre nosotros? (Aquí estoy hablando
desde una perspectiva humana). \bibverse{6} ¡Por supuesto que no! ¿De
qué otra manera podría Dios juzgar al mundo? \bibverse{7} Alguno podría
decir: ``¿Por qué sigo siendo condenado como pecador si mis mentiras
hacen que la verdad de Dios y su gloria sean más obvias al
contrastarlas?'' \bibverse{8} ¿Acaso se trata de ``Vamos a pecar para
dar lugar al bien''? Algunos con calumnia nos acusan de decir eso.
¡Tales personas deberían ser condenadas!

\bibverse{9} Entonces, ¿son los judíos mejores que los demás?
¡Ciertamente no! Recordemos que ya hemos demostrado que tanto judíos
como extranjeros estamos bajo el control del pecado. \bibverse{10} Como
dice la Escritura: ``Nadie hace lo recto, ni siquiera uno. \bibverse{11}
Nadie entiende, nadie busca a Dios. \bibverse{12} Todos le han dado la
espalda, todos hacen lo que es malo. Nadie hace lo que es bueno, ni
siquiera uno. \bibverse{13} Sus gargantas son como una tumba abierta;
sus lenguas esparcen engaño; sus labios rebosan veneno de serpientes.
\bibverse{14} Sus bocas están llenas de amargura y maldiciones,
\bibverse{15} y están prestos para causar dolor y muerte. \bibverse{16}
Su camino los lleva al desastre y la miseria; \bibverse{17} no saben
cómo vivir en paz. \bibverse{18} No les importa en absoluto respetar a
Dios.''\footnote{\textbf{3:18} Este texto incluye referencias a Salmos
  14:1-3, 5:9, 140:3, 10:7, Isaías 59:7, 8, Proverbios 1:16, Salmos
  36:1.}

\bibverse{19} Está claro que todo lo que dice la ley se aplica a
aquellos que viven bajo la ley, para que nadie pueda tener excusa
alguna, y para asegurar que todos en el mundo sean responsables ante
Dios. \bibverse{20} Porque nadie es justificado ante Dios por hacer lo
que la ley exige. La ley solo nos ayuda a reconocer lo que es realmente
el pecado.

\bibverse{21} Pero ahora se ha demostrado el carácter bondadoso y
recto\footnote{\textbf{3:21} Ver el versículo 1:17. También 3:22.} de
Dios. Y no tiene nada que ver con el cumplimiento de la ley, aunque ya
se habló de él por medio de la ley y los profetas. \bibverse{22} Este
carácter recto de Dios viene a todo aquél que cree en Jesucristo,
aquellos que ponen su confianza en él. No importa quienes seamos:
\bibverse{23} Todos hemos pecado y hemos fallado en alcanzar el ideal
glorioso de Dios. \bibverse{24} Sin embargo, por medio del regalo de su
gracia, Dios nos hace justos, a través de Jesucristo, quien nos hace
libres. \bibverse{25} Dios presentó abiertamente a Jesús como el don que
trae paz\footnote{\textbf{3:25} O, ``lugar de expiación.''} a aquellos
que creen en él, quien derramó su sangre. Hizo esto con el fin de
demostrar que él es verdaderamente recto, porque anteriormente se
contuvo y pasó por alto los pecados, \bibverse{26} pero ahora, en el
presente, Dios demuestra que es justo y hace lo recto, y que hace justos
a los que creen en Jesús.

\bibverse{27} ¿Acaso tenemos algo de qué jactarnos? Por supuesto que no,
¡no hay lugar para ello! ¿Por qué? ¿Acaso es porque seguimos la ley de
guardar los requisitos? No, nosotros seguimos la ley de la fe en Dios.
\bibverse{28} Entonces concluimos que somos hechos justos por Dios por
medio de nuestra fe en él, y no por la observancia de la ley.
\bibverse{29} ¿Acaso Dios es solamente Dios de los judíos? ¿Acaso él no
es el Dios de los demás pueblos también? ¡Por supuesto que sí!
\bibverse{30} Solo hay un Dios, y él nos justifica por nuestra fe en él,
quienesquiera que seamos, judíos o extranjeros. \bibverse{31} ¿Significa
eso que por creer en Dios desechamos de la ley? ¡Por supuesto que no! De
hecho, afirmamos la importancia de la ley.

\hypertarget{section-3}{%
\section{4}\label{section-3}}

\bibverse{1} Miremos el ejemplo de Abraham. Desde la perspectiva humana,
él es el padre de nuestra nación. Preguntemos: ``¿Cuál fue su
experiencia?'' \bibverse{2} Porque si Abraham hubiera sido justificado
por lo que hizo, habría tenido algo de lo cual jactarse, pero no ante
los ojos de Dios. \bibverse{3} Sin embargo, ¿qué dice la Escritura?
``Abraham creyó en Dios, y por ello fue considerado justo.''
\bibverse{4} Todo el que trabaja recibe su pago, no como un regalo, sino
porque se ha ganado su salario. \bibverse{5} Pero Dios, quien hace
justos a los pecadores, los considera justos no porque hayan trabajado
por ello, sino porque confían en él. \bibverse{6} Es por ello que David
habla de la felicidad de aquellos a quienes Dios acepta como justos, y
no porque ellos trabajen por ello: \bibverse{7} ``Cuán felices son los
que reciben perdón por sus errores y cuyos pecados son cubiertos.
\bibverse{8} Cuán felices son aquellos a quienes el Señor no considera
pecadores.''

\bibverse{9} Ahora, ¿es acaso esta bendición solo para los judíos, o es
para los demás también? Acabamos de afirmar que Abraham fue aceptado
como justo porque confió en Dios. \bibverse{10} Pero ¿cuándo sucedió
esto? ¿Acaso fue cuando Abraham era judío o antes? \bibverse{11} De
hecho, fue antes de que Abraham fuera judío por ser circuncidado, lo
cual era una confirmación de su confianza en Dios para hacerlo justo.
Esto ocurrió antes de ser circuncidado, de modo que él es el padre de
todos los que confían en Dios y son considerados justos por él, aunque
no sean judíos circuncidados. \bibverse{12} También es el padre de los
judíos circuncidados, no solo porque estén circuncidados, sino porque
siguen el ejemplo de la confianza en Dios que nuestro padre Abraham tuvo
antes de ser circuncidado.

\bibverse{13} La promesa que Dios le hizo a Abraham y a sus
descendientes de que el mundo les pertenecería no estaba basada en su
cumplimiento de la ley, sino en que él fue justificado por su confianza
en Dios. \bibverse{14} Porque si la herencia prometida estuviera basada
en el cumplimiento de la ley, entonces confiar en Dios no sería
necesario, y la promesa sería inútil. \bibverse{15} Porque la ley
resulta en castigo,\footnote{\textbf{4:15} Castigo por el incumplimiento
  de la ley, que por supuesto incluye a todos.} pero si no hay ley,
entonces no puede ser quebrantada.

\bibverse{16} De modo que la promesa está basada en la confianza en
Dios. Es dada como un don, garantizada a todos los hijos de Abraham, y
no solo a los que siguen la ley\footnote{\textbf{4:16} Pablo no está
  diciendo que los que obedecen la ley de Moisés son justificados ante
  Dios. Ya había tratado ese tema. Sencillamente está señalando que los
  que no siguen la ley de moisés no son excluidos por Dios.}, sino
también a todos los que creen como Abraham, el padre de todos nosotros.
\bibverse{17} Como dice la Escritura: ``Yo te he hecho el padre de
muchas naciones.''\footnote{\textbf{4:17} Génesis 17:5.} Porque en
presencia de Dios, Abraham creyó en el Dios que hace resucitar a los
muertos y trajo a la existencia lo que no existía antes. \bibverse{18}
Contra toda esperanza, Abrahán tuvo esperanza y confió en Dios, y de
este modo pudo llegar a ser el padre de muchos pueblos, tal como Dios se
lo prometió: ``¡Tendrás muchos descendientes!'' \bibverse{19} Su
confianza en Dios no se debilitó aun cuando creía que su cuerpo ya
estaba prácticamente muerto (tenía casi cien años de edad), y sabía que
Sara estaba muy vieja para tener hijos. \bibverse{20} Sino que se aferró
a la promesa de Dios y no dudó. Por el contrario, su confianza en Dios
se fortalecía y daba gloria a Dios. \bibverse{21} Él estaba
completamente convencido que Dios tenía el poder para cumplir la
promesa. \bibverse{22} Por eso Dios consideró justo a Abraham.

\bibverse{23} Las palabras ``Abraham fue considerado justo''\footnote{\textbf{4:23}
  Génesis 15:6.} no fueron escritas solo para su beneficio.
\bibverse{24} También fueron escritas para beneficio de nosotros, para
los que seremos considerados justos porque confiamos en Dios, quien
levantó a nuestro Señor Jesús de los muertos. \bibverse{25} Jesús fue
entregado a la muerte por causa de nuestros pecados\footnote{\textbf{4:25}
  Ver Isaías 53:4, 5}, y fue levantado a la vida para justificarnos.

\hypertarget{section-4}{%
\section{5}\label{section-4}}

\bibverse{1} Ahora que hemos sido justificados por Dios, por nuestra
confianza en él, tenemos paz con él a través de nuestro Señor
Jesucristo. \bibverse{2} Porque es por medio de Jesús, y por nuestra fe
en él, que hemos recibido acceso a esta posición de gracia en la que
estamos, esperando con alegría y confianza que podamos participar de la
gloria de Dios. \bibverse{3} No solo esto, sino que mantenemos la
confianza cuando vienen los problemas, porque sabemos que experimentar
dificultades desarrolla nuestra fortaleza espiritual\footnote{\textbf{5:3}
  O ``perseverancia.''}. \bibverse{4} La fortaleza espiritual, a su vez,
desarrolla un carácter maduro, y este carácter maduro trae como
resultado una esperanza que cree. \bibverse{5} Ya que tenemos esta
esperanza, nunca seremos defraudados, porque el amor de Dios ha sido
derramado en nuestros corazones a través del Espíritu Santo que él nos
ha dado. \bibverse{6} Cuando estábamos completamente indefensos, en ese
momento justo, Cristo murió por nosotros los impíos. \bibverse{7} ¿Quién
moriría por otra persona, incluso si se tratara de alguien que hace lo
recto? (Aunque quizás alguno sería suficientemente valiente para morir
por alguien que es realmente bueno.) \bibverse{8} Pero Dios demuestra su
amor en que Cristo murió por nosotros aunque todavía éramos pecadores.

\bibverse{9} Ahora que somos justificados por su muerte\footnote{\textbf{5:9}
  Literalmente, ``sangre.''}, podemos estar totalmente seguros de que él
nos salvará del juicio que viene. \bibverse{10} Aunque éramos sus
enemigos, Dios nos convirtió en sus amigos por medio de la muerte de su
Hijo, y así podemos estar totalmente seguros de que él nos salvará por
la vida de su Hijo. \bibverse{11} Además de esto celebramos ahora lo que
Dios ha hecho por medio de nuestro Señor Jesucristo para reconciliarnos
y convertirnos en sus amigos. \bibverse{12} Porque a través de un hombre
el pecado entró al mundo, y el pecado condujo a la muerte. Y de esta
manera la muerte llegó a todos, porque todos eran pecadores.
\bibverse{13} Incluso antes de que se diera la ley, el pecado ya estaba
en el mundo, pero no era considerado pecado porque no había ley.
\bibverse{14} Pero la muerte gobernaba desde Adán hasta Moisés, incluso
sobre aquellos que no pecaron de la misma manera que lo hizo Adán.

Pues Adán prefiguraba a Aquél que vendría\footnote{\textbf{5:14} En
  otras palabras, Adán era un símbolo o tipo de Jesús, quien vendría.}.
\bibverse{15} Pero el don de Jesús no es como el pecado de
Adán\footnote{\textbf{5:15} Haciendo explícito lo que quiere decir con
  ``don'' y ``pecado''.}. Aunque mucha gente murió por culpa del pecado
de un hombre, la gracia de Dios es mucho más grande y ha sido dada a
muchos a través de su don gratuito en la persona de Jesucristo.
\bibverse{16} El resultado de este don no es como el resultado del
pecado. El resultado del pecado de Adán fue juicio y condenación, pero
este don nos justifica con Dios, a pesar de nuestros muchos pecados.
\bibverse{17} Como resultado del pecado de un hombre, la muerte gobernó
por su culpa. Pero la gracia de Dios es mucho más grande y su don nos
justifica, porque todo el que lo recibe gobernará en vida a través de la
persona de Jesucristo. \bibverse{18} Del mismo modo que un pecado trajo
condenación a todos, un acto de justicia nos dio a todos la oportunidad
de vivir en justicia. \bibverse{19} Así como por la desobediencia de un
hombre muchos se convirtieron en pecadores, de la misma manera, a través
de la obediencia de un hombre, muchos son justificados delante de Dios.
\bibverse{20} Pues cuando se introdujo la ley, el pecado se hizo más
evidente. ¡Pero aunque el pecado se volvió más evidente, la gracia se
volvió más evidente aun! \bibverse{21} Así como el pecado gobernó sobre
nosotros y nos llevó a la muerte, ahora la gracia es la que gobierna al
justificarnos delante de Dios, trayéndonos vida eterna por medio de
Jesucristo, nuestro Señor.

\hypertarget{section-5}{%
\section{6}\label{section-5}}

\bibverse{1} ¿Cuál es nuestra respuesta, entonces? ¿Debemos seguir
pecando para tener aún más gracia? \bibverse{2} ¡Por supuesto que
no!\footnote{\textbf{6:2} Literalmente, ``¡que no ocurra así!'' Esta
  reacción enérgica es traducida en diversas maneras así: ¡Por supuesto
  que no! ¡De ninguna manera! ¡Que Dios no lo quiera! También en el
  versículo 6:15 etc.} Pues si estamos muertos al pecado, ¿cómo
podríamos seguir viviendo en pecado? \bibverse{3} ¿No saben que todos
los que fuimos bautizados en Jesucristo, fuimos bautizados en su muerte?
\bibverse{4} A través del bautismo fuimos sepultados con él en la
muerte, para que así como Cristo fue levantado de los muertos por medio
de la gloria del Padre, nosotros también podamos vivir una vida nueva.
\bibverse{5} Si hemos sido hechos uno con él, al morir como él murió,
entonces seremos levantados como él también.

\bibverse{6} Sabemos que nuestro antiguo ser fue crucificado con él para
deshacernos del cuerpo muerto del pecado, a fin de que ya no pudiéramos
ser más esclavos del pecado. \bibverse{7} Todo el que ha muerto, ha sido
liberado del pecado. \bibverse{8} Y como morimos con Cristo, tenemos la
confianza de que también viviremos con él, \bibverse{9} porque sabemos
que si Cristo ha sido levantado de los muertos, no morirá más, porque la
muerte ya no tiene ningún poder sobre él. \bibverse{10} Al morir, él
murió al pecado una vez y por todos, pero ahora vive, y vive para Dios.
\bibverse{11} De esta misma manera, ustedes deben considerarse muertos
al pecado, pero vivos para Dios, por medio de Jesucristo. \bibverse{12}
No permitan que el pecado controle sus cuerpos mortales, no se rindan
ante sus tentaciones, \bibverse{13} y no usen ninguna parte de su cuerpo
como herramientas de pecado para el mal. Por el contrario, conságrense a
Dios como quienes han sido traídos de vuelta a la vida, y usen todas las
partes de su cuerpo como herramientas para hacer el bien para Dios.
\bibverse{14} El pecado no gobernará sobre ustedes, porque ustedes no
están bajo la ley sino bajo la gracia.

\bibverse{15} ¿Acaso vamos a pecar porque no estamos bajo la ley sino
bajo la gracia? ¡Por supuesto que no! \bibverse{16} ¿No se dan cuenta de
que si ustedes se someten a alguien, y obedecen sus órdenes, entonces
son esclavos de aquél a quien obedecen? Si ustedes son esclavos del
pecado, el resultado es muerte; si obedecen a Dios, el resultado es que
serán justificados delante de él. \bibverse{17} Gracias a Dios porque
aunque una vez ustedes eran esclavos del pecado, escogieron de todo
corazón seguir la verdad que aprendieron acerca de Dios. \bibverse{18}
Ahora que han sido liberados del pecado, se han convertido en esclavos
de hacer lo recto.

\bibverse{19} Hago uso de este ejemplo cotidiano porque su forma humana
de pensar es limitada. Así como una vez ustedes mismos se hicieron
esclavos de la inmoralidad, ahora deben volverse esclavos de lo que es
puro y recto. \bibverse{20} Cuando eran esclavos del pecado, no se les
exigía que hicieran lo recto. \bibverse{21} Pero ¿cuáles eran los
resultados en ese entonces? ¿No se avergüenzan de las cosas que
hicieron? ¡Eran cosas que conducen a la muerte! \bibverse{22} Pero ahora
que han sido liberados del pecado y se han convertido en esclavos de
Dios, los resultados serán una vida pura, y al final, vida eterna.
\bibverse{23} La paga del pecado es muerte, pero el regalo de Dios es
vida eterna por medio de Jesucristo, nuestro Señor.

\hypertarget{section-6}{%
\section{7}\label{section-6}}

\bibverse{1} Hermanos y hermanas, (hablo para personas que conocen la
ley\footnote{\textbf{7:1} El uso que Pablo hace de la palabra ley puede
  tener varios significados, pero a menudo se refiere al sistema de
  creencias judías. Parte de esto tiene que ver con el cumplimiento de
  las reglas.}), ¿no ven que la ley tiene autoridad sobre alguien solo
mientras esta persona esté viva? \bibverse{2} Por ejemplo, una mujer
casada está sujeta por ley a su esposo mientras él esté vivo, pero si
muere, ella queda libre de esta obligación legal con él. \bibverse{3} De
modo que si ella vive con otro hombre mientras su esposo está vivo, ella
estaría cometiendo adulterio. Sin embargo, si su esposo muere y ella se
casa con otro hombre, entonces ella no sería culpable de adulterio.

\bibverse{4} Del mismo modo, mis amigos, ustedes han muerto para la ley
mediante el cuerpo de Cristo, y ahora ustedes le pertenecen a otro, a
Cristo, quien ha resucitado de los muertos para que nosotros pudiéramos
vivir una vida productiva\footnote{\textbf{7:4} Literalmente, ``que
  lleve fruto para Dios.''} para Dios. \bibverse{5} Cuando éramos
controlados por la vieja naturaleza, nuestros deseos pecaminosos (tal
como los revela la ley) obraban dentro de nosotros y traían como
resultado la muerte. \bibverse{6} Pero ahora hemos sido libertados de la
ley, y hemos muerto a lo que nos encadenaba, a fin de que podamos servir
de un nuevo modo, en el Espíritu, y no a la manera de la antigua letra
de la ley.

\bibverse{7} ¿Qué concluimos entonces? ¿Que la ley es pecado? ¡Por
supuesto que no! Pues yo no habría conocido lo que era el pecado si no
fuera porque la ley lo define. Yo no me habría dado cuenta de que el
deseo de tener las cosas de otras personas estaba mal si no fuera porque
la ley dice: ``No desees para ti lo que le pertenece a otro.''
\bibverse{8} Pero a través de este mandamiento el pecado encontró la
manera de despertar en mí todo tipo de deseos egoístas. Porque sin la
ley, el pecado está muerto. \bibverse{9} Yo solía vivir sin darme cuenta
de lo que la ley realmente significaba, pero cuando comprendí las
implicaciones de ese mandamiento, entonces el pecado volvió a la vida y
morí. \bibverse{10} Descubrí que el mismo mandamiento que tenía como
propósito traerme vida, me trajo muerte en lugar de ello, \bibverse{11}
porque el pecado encontró su camino a través del mandamiento para
engañarme, y lo usó para matarme.

\bibverse{12} Sin embargo, la ley es santa, y el mandamiento es santo,
justo y recto. \bibverse{13} Ahora, ¿acaso podría matarme algo que es
bueno? ¡Por supuesto que no! Pero el pecado se muestra como pecado
usando lo bueno para causar mi muerte. Así que por medio del mandamiento
se revela cuán malo es el pecado realmente. \bibverse{14} Comprendemos
que la ley es espiritual, pero yo soy totalmente humano\footnote{\textbf{7:14}
  Literalmente, ``carne.''}, un esclavo del pecado. \bibverse{15}
Realmente no entiendo lo que hago. ¡Hago las cosas que no quiero hacer,
y lo que odio hacer es precisamente lo que hago! \bibverse{16} Pero si
digo que hago lo que no quiero hacer, esto demuestra que yo admito que
la ley es buena. \bibverse{17} De modo que ya no soy yo quien hace esto,
sino el pecado que vive en mí \bibverse{18} porque yo sé que no hay nada
bueno en mí en lo que tiene que ver con mi naturaleza humana pecaminosa.
Aunque quiero hacer el bien, simplemente no puedo hacerlo. \bibverse{19}
¡El bien que quiero hacer no lo hago; mientras que el mal que no quiero
hacer es lo que termino haciendo! \bibverse{20} Sin embargo, si hago lo
que no quiero, entonces ya no soy yo quien lo hace, sino el pecado que
vive en mí.

\bibverse{21} Este es el principio que he descubierto: si quiero hacer
lo bueno, el mal también está siempre ahí. \bibverse{22} Mi ser interior
se deleita en la ley de Dios, \bibverse{23} pero veo que hay una ley
distinta que obra dentro de mí y que está en guerra con la ley que mi
mente ha decidido seguir, convirtiéndome en un prisionero de la ley de
pecado que está dentro de mí. \bibverse{24} ¡Cuán miserable soy! ¿Quién
me rescatará de este cuerpo que causa mi muerte\footnote{\textbf{7:24}
  Literalmente, ``cuerpo de muerte.''}? ¡Gracias a Dios, porque él me
salva a través de Jesucristo, nuestro Señor! \bibverse{25} La situación
es esta: Aunque yo mismo elijo en mi mente obedecer la ley de Dios, mi
naturaleza humana obedece la ley del pecado.

\hypertarget{section-7}{%
\section{8}\label{section-7}}

\bibverse{1} Así que ahora no hay condenación para los que están en
Cristo Jesús. \bibverse{2} La ley del Espíritu de vida en Jesucristo me
ha libertado de la ley del pecado y muerte. \bibverse{3} Lo que la ley
no pudo hacer porque no tenía el poder para hacerlo debido a nuestra
naturaleza pecaminosa\footnote{\textbf{8:3} ``Naturaleza pecaminosa,''
  literalmente ``carne,'' refiriéndose a la naturaleza física pecaminosa
  y caída de la humanidad. A menudo se usa esta palabra en los
  versículos que siguen para hacer un contraste con la naturaleza
  espiritual.}, Dios pudo hacerlo. Al enviar a su propio Hijo en forma
humana, Dios se hizo cargo del problema del pecado\footnote{\textbf{8:3}
  O ``hacienda un sacrificio de sí mismo por el pecado.''} y destruyó el
poder del pecado en nuestra naturaleza humana pecaminosa. \bibverse{4}
De este modo, pudimos cumplir los buenos requisitos de la ley, siguiendo
al Espíritu y no a nuestra naturaleza pecaminosa. \bibverse{5} Aquellos
que siguen su naturaleza pecaminosa están preocupados por cosas
pecaminosas, pero los que siguen al Espíritu, se concentran en cosas
espirituales. \bibverse{6} La mente humana y pecaminosa lleva a la
muerte, pero cuando la mente es guiada por el Espíritu, trae vida y paz.
\bibverse{7} La mente humana y pecaminosa es reacia a Dios porque se
niega a obedecer la ley de Dios. Y de hecho, no puede hacerlo;
\bibverse{8} y aquellos que siguen su naturaleza pecaminosa no pueden
agradar a Dios. \bibverse{9} Pero ustedes no siguen su naturaleza
pecaminosa sino al Espíritu, si es que el Espíritu de Dios vive en
ustedes. Porque aquellos que no tienen el Espíritu de Cristo dentro de
ellos, no le pertenecen a él.

\bibverse{10} Sin embargo, si Cristo está en ustedes, aunque su cuerpo
vaya a morir por causa del pecado, el Espíritu les da vida porque ahora
ustedes están justificados delante de Dios. \bibverse{11} El Espíritu de
Dios que levantó a Jesús de los muertos, vive en ustedes. Él, que
levantó a Jesús de los muertos, dará vida a sus cuerpos muertos a través
de su Espíritu que vive en ustedes. \bibverse{12} Así que, hermanos y
hermanas, no tenemos que seguir\footnote{\textbf{8:12} O ``no tenemos
  obligación.''} nuestra naturaleza pecaminosa que obra conforme a
nuestros deseos humanos. \bibverse{13} Porque si ustedes viven bajo el
control de su naturaleza pecaminosa, van a morir. Pero si siguen el
camino del Espíritu, dando muerte a las cosas malas que hacen, entonces
vivirán. \bibverse{14} Todos los que son guiados por el Espíritu de Dios
son hijos de Dios. \bibverse{15} No se les ha dado un espíritu de
esclavitud ni de temor una vez más. No, lo que recibieron fue el
espíritu que los convierte en hijos, para que estén dentro de la familia
de Dios. Ahora podemos decir a viva voz: ``¡Dios es nuestro Padre!''
\bibverse{16} El Espíritu mismo está de acuerdo con nosotros\footnote{\textbf{8:16}
  Literalmente, ``nuestro espíritu.''} en que somos hijos de Dios.
\bibverse{17} Y si somos sus hijos, entonces somos sus herederos. Somos
herederos de Dios, y herederos junto con Cristo. Pero si queremos
participar de su gloria, debemos participar de sus sufrimientos.

\bibverse{18} Sin embargo, estoy convencido de que lo que sufrimos en el
presente no es nada si lo comparamos con la gloria futura que se nos
revelará. \bibverse{19} Toda la creación espera con paciencia, anhelando
que Dios se revele a sus hijos. \bibverse{20} Porque Dios permitió que
fuera frustrado el propósito de la creación. \bibverse{21} Pero la
creación misma mantiene la esperanza puesta en ese momento en que será
liberada de la esclavitud de la degradación y participará de la gloriosa
libertad de los hijos de Dios. \bibverse{22} Sabemos que toda la
creación clama con anhelo, sufriendo dolores de parto hasta hoy.
\bibverse{23} Y no solo la creación, sino que nosotros también, quienes
tenemos un anticipo del Espíritu, y clamamos por dentro mientras
esperamos que Dios nos ``adopte,'' que realice la redención de nuestros
cuerpos. \bibverse{24} Sin embargo, la esperanza que ya ha sido vista no
es esperanza en absoluto. ¿Acaso quién espera lo que ya puede ver?
\bibverse{25} Como nosotros esperamos lo que no hemos visto todavía,
esperamos pacientemente por ello.

\bibverse{26} De la misma manera, el Espíritu nos ayuda en nuestra
debilidad. Nosotros no sabemos cómo hablar con Dios, pero el Espíritu
mismo intercede con nosotros y por nosotros mediante gemidos que las
palabras no pueden expresar. \bibverse{27} Aquél que examina las mentes
de todos conoce las motivaciones del Espíritu\footnote{\textbf{8:27} O,
  ``la mente del Espíritu.''}, porque el Espíritu aboga la causa de Dios
en favor de los creyentes. \bibverse{28} Sabemos que en todas las cosas
Dios obra para el bien de los que le aman, aquellos a quienes él ha
llamado para formar parte de su plan. \bibverse{29} Porque Dios,
escogiéndolos de antemano, los separó para ser como su Hijo, a fin de
que el Hijo pudiera ser el primero de muchos hermanos y hermanas.
\bibverse{30} A los que escogió también llamó, y a aquellos a quienes
llamó también justificó, y a quienes justificó también glorificó.

\bibverse{31} ¿Cuál es, entonces, nuestra respuesta a todo esto? Si Dios
está a nuestro favor, ¿quién puede estar en contra de nosotros?
\bibverse{32} Dios, quien no retuvo a su propio Hijo, sino que lo
entregó por todos nosotros, ¿no nos dará gratuitamente todas las cosas?
\bibverse{33} ¿Quién puede acusar de alguna cosa al pueblo de Dios? Es
Dios quien nos justifica, \bibverse{34} así que ¿quién puede
condenarnos? Fue Cristo quien murió---y más importante aún, quien se
levantó de los muertos---el que se sienta a la diestra de Dios,
presentando nuestro caso.

\bibverse{35} ¿Quién puede separarnos del amor de Cristo? ¿Acaso la
opresión, la angustia, o la persecución? ¿O acaso el hambre, la pobreza,
el peligro, o la violencia? \bibverse{36} Tal como dice la Escritura:
``Por tu causa estamos todo el tiempo en peligro de morir. Somos
tratados como ovejas que serán llevadas al sacrificio.''\footnote{\textbf{8:36}
  Salmos 44:22.} \bibverse{37} No.~En todas las cosas que nos suceden
somos más que vencedores por medio de Aquél que nos amó. \bibverse{38}
Por eso estoy plenamente convencido de que ni la muerte, ni la vida, ni
los ángeles, ni los demonios, ni el presente, ni el futuro, ni las
potencias, \bibverse{39} ni lo alto, ni lo profundo, y, de hecho,
ninguna cosa en toda la creación puede separarnos del amor de Dios en
Jesucristo, nuestro Señor.

\hypertarget{section-8}{%
\section{9}\label{section-8}}

\bibverse{1} Yo estoy en Cristo, y lo que digo es verdad. ¡No les
miento! Mi conciencia y el Espíritu Santo confirman \bibverse{2} cuán
triste estoy, y el dolor infinito que tengo en mi corazón \bibverse{3}
por mi propio pueblo, por mis hermanos y hermanas. Preferiría yo mismo
ser maldecido, estar separado de Cristo, si eso pudiera ayudarlos.
\bibverse{4} Ellos son mis hermanos de raza, los israelitas, el pueblo
escogido de Dios. Dios les reveló su gloria e hizo tratados\footnote{\textbf{9:4}
  Literalmente, ``pactos.''} con ellos, dándoles la ley, el verdadero
culto, y sus promesas. \bibverse{5} Ellos son nuestros antepasados,
ancestros de Cristo, humanamente hablando, de Aquél que gobierna sobre
todo, el Dios bendito por la eternidad. Amén.

\bibverse{6} No es que la promesa de Dios haya fallado. Porque no todo
israelita es un verdadero israelita, \bibverse{7} y no todos los que son
descendientes de Abraham son sus verdaderos hijos. Pues la Escritura
dice: ``Tus descendientes serán contados por medio de
Isaac,''\footnote{\textbf{9:7} Génesis 21:12.} \bibverse{8} de modo que
no son los hijos reales de Abrahán los que se cuentan como hijos de
Dios, sino que son considerados como sus verdaderos descendientes solo
los hijos de la promesa.

\bibverse{9} Y esta fue la promesa: ``Regresaré el próximo año y Sara
tendrá un hijo.''\footnote{\textbf{9:9} Génesis 18:10, 14.}
\bibverse{10} Además, los hijos gemelos de Rebeca tenían el mismo padre,
nuestro antepasado Isaac. \bibverse{11} Pero incluso antes de que los
niños nacieran, y antes de que hubieran hecho algo bueno o malo, (a fin
de que pudiera continuar el propósito de Dios, demostrando que el
llamado de Dios a las personas no está basado en la conducta humana),
\bibverse{12} a ella se le dijo: ``El hermano mayor servirá al hermano
menor.''\footnote{\textbf{9:12} Génesis 25:23.} \bibverse{13} Como dice
la Escritura: ``Yo escogí a Jacob, pero rechacé a Esaú.''\footnote{\textbf{9:13}
  Malaquías 1:2, 3.}

\bibverse{14} Entonces, ¿qué debemos concluir? ¿Diremos que Dios es
injusto? ¡Por supuesto que no! \bibverse{15} Como dijo a Moisés:
``Tendré misericordia de quien deba tener misericordia, y tendré
compasión de quien deba tener compasión.''\footnote{\textbf{9:15} Éxodo
  33:19.} \bibverse{16} De modo que no depende de lo que nosotros
queremos o de nuestros propios esfuerzos, sino del carácter
misericordioso de Dios. \bibverse{17} La Escritura registra que Dios le
dijo al Faraón: ``Te puse aquí por una razón: para que por ti yo pudiera
demostrar mi poder, y para que mi nombre sea conocido por toda la
tierra.''\footnote{\textbf{9:17} Éxodo 9:16.} \bibverse{18} De modo que
Dios es misericordioso con quienes él desea serlo, y endurece el corazón
de quienes él desea\footnote{\textbf{9:18} En el Antiguo Testamento esta
  expresión se usa para describir un rechazo obstinado por Dios, tal
  como la experiencia del Faraón de Éxodo. En Éxodo 9 Faraón es
  presentado en varias ocasiones con corazón endurecido, o menciona que
  Dios endurecía su corazón, o en voz pasiva, diciendo que su corazón
  era endurecido. De manera que este versículo en el libro de Romanos no
  debe usarse para decir que Dios deliberadamente endurece el corazón de
  las personas y luego los castiga por ello. El endurecimiento del
  corazón es un rechazo a la gracia divina.}. \bibverse{19} Ahora bien,
ustedes discutirán conmigo y preguntarán: ``Entonces, ¿por qué sigue
culpándonos? ¿Quién puede oponerse a la voluntad de Dios?''
\bibverse{20} Y esa no es manera de hablar, porque ¿quién eres tú, ---un
simple mortal---, para contradecir a Dios? ¿Puede alguna cosa creada
decirle a su creador: ``por qué me hiciste así?'' \bibverse{21} ¿Acaso
el alfarero no tiene el derecho de usar la misma arcilla ya sea para
hacer una vasija decorativa o una vasija común?\footnote{\textbf{9:21}
  Literalmente, ``vasijas de valor y deshonra.''}

\bibverse{22} Es como si Dios, queriendo demostrar su oposición al
pecado\footnote{\textbf{9:22} Literalmente ``mostrar su ira.''} y para
revelar su poder, soportara con paciencia estas ``vasijas destinadas a
la destrucción,'' \bibverse{23} a fin de revelar la grandeza de su
gloria mediante estas ``vasijas de misericordia,'' las cuales él ha
preparado de antemano para la gloria. \bibverse{24} Esto es lo que
somos, personas que él ha llamado, no solo de entre los judíos, sino de
entre los extranjeros también\ldots{}

\bibverse{25} Como dijo Dios en el libro de Oseas: ``Llamaré mi pueblo a
los que no son mi pueblo, y a los que no son amados llamaré mis
amados,''\footnote{\textbf{9:25} Oseas 2:23.} \bibverse{26} y ``sucederá
que en el lugar donde les dijeron `tú no eres mi pueblo' serán llamados
hijos del Dios viviente.''\footnote{\textbf{9:26} Oseas 1:10.}

\bibverse{27} Isaías clama, respecto a Israel: ``Aun cuando los hijos de
Israel han llegado a ser tantos como la arena del mar, solo unos
cuantos\footnote{\textbf{9:27} Literalmente, ``remanente.''} se
salvarán. \bibverse{28} Porque el Señor terminará rápida y completamente
su obra de juicio sobre la tierra. \bibverse{29} Como había dicho antes
Isaías: ``Si el Señor Todopoderoso no nos hubiera dejado algunos
descendientes, nos habríamos convertido en algo semejante a Sodoma y
Gomorra.''\footnote{\textbf{9:29} Isaías 1:9.}

\bibverse{30} ¿Qué concluiremos, entonces? Que aunque los extranjeros ni
siquiera procuraban hacer lo recto, comprendieron lo recto, y por medio
de su fe en Dios hicieron lo recto. \bibverse{31} Pero el pueblo de
Israel, que seguía la ley, para que ella los justificara con Dios, nunca
lo logró. \bibverse{32} ¿Por qué no? Porque dependían de lo que hacían y
no de su confianza en Dios. Tropezaron con la piedra de tropiezo,
\bibverse{33} tal como lo predijo la Escritura: ``Miren, en Sión pongo
una piedra de tropiezo, una roca que ofenderá a la gente. Pero los que
confían en él, no serán frustrados.''\footnote{\textbf{9:33} Isaías
  28:16, 8:14.}

\hypertarget{section-9}{%
\section{10}\label{section-9}}

\bibverse{1} Mis hermanos y hermanas, el deseo de mi corazón---mi
oración a Dios---es la salvación del pueblo de Israel. \bibverse{2}
Puedo dar testimonio de su ferviente dedicación a Dios, pero esta
dedicación no está basada en conocerlo como él realmente es.
\bibverse{3} Ellos no comprenden cómo Dios nos hace justos, y tratan de
justificarse a sí mismos. Se niegan a aceptar la manera en que Dios
justifica a las personas. \bibverse{4} Porque Cristo es el cumplimiento
de la ley. Todos los que confían en él son justificados. \bibverse{5}
Moisés escribió: ``Todo el que hace lo recto mediante la obediencia de
la ley, vivirá.''\footnote{\textbf{10:5} Levítico 18:5.} \bibverse{6}
Pero la disposición de hacer lo recto que proviene de la fe, dice esto:
``No preguntes `¿quién subirá al cielo?' (Pidiendo que Cristo descienda
a nosotros),'' \bibverse{7} o ``\,`¿quién irá al lugar de los
muertos\footnote{\textbf{10:7} Literalmente, ``el abismo,'' pozo sin
  fondo.}?' (Pidiendo que Cristo regrese de entre los muertos).''
\bibverse{8} Lo que la Escritura realmente dice es: ``Este mensaje está
muy cerca de ti, está en tu boca y en tu corazón.''\footnote{\textbf{10:8}
  Estas son alusiones a Deuteronomio 30:11-14. Originalmente se
  aplicaban a la ley, y servían para indicar que la ley no era distante
  e inalcanzable, negando claramente que fuera difícil su observancia.
  Ahora Pablo lo aplica a la persona de Cristo, aclarando que este
  ``mensaje de la ley'' se cumplió en él.} De hecho, lo que estamos
mostrando es este mensaje, basado en la fe. \bibverse{9} Porque si
declaras que aceptas a Jesús como Señor, y estás convencido en tu
corazón de que Dios lo levantó de los muertos, entonces serás salvo.
\bibverse{10} Tu fe en Dios te hace justo, y tu declaración de
aceptación a Dios te salva. \bibverse{11} Como dice la Escritura: ``Los
que creen en él no serán frustrados.''\footnote{\textbf{10:11} Isaías
  28:16. Frustrados: o ``avergonzados.''}

\bibverse{12} No hay diferencia entre judío y griego, porque el mismo
Señor es Señor de todos, y da generosamente a todos los que le piden.
\bibverse{13} Porque todo el que invoque el nombre del Señor será
salvo.''\footnote{\textbf{10:13} Joel 2:32.} \bibverse{14} Pero ¿cómo
podrá la gente invocar a alguien en quien no creen? ¿Cómo podrían creer
en alguien de quien no han escuchado hablar? ¿Y cómo podrían escuchar si
no se les habla? \bibverse{15} ¿Cómo podrán ir a hablarles si no se les
envía? Tal como dice la Escritura: ``Bienvenidos son los que traen la
buena noticia.''\footnote{\textbf{10:15} Isaías 52:7.} \bibverse{16}
Pero no todos han aceptado la buena noticia. Como pregunta Isaías:
``Señor, ¿quién creyó en la noticia de la que nos oyeron
hablar?''\footnote{\textbf{10:16} Isaías 53: 1.} \bibverse{17} Creer en
Dios viene de oír, de oír el mensaje de Cristo.

\bibverse{18} Y no es que no hayan oído. Muy por el contrario: ``Las
voces de los que hablan de Dios\footnote{\textbf{10:18} Implícito.} se
han oído por toda la tierra. Su mensaje se extendió por todo el
mundo.''\footnote{\textbf{10:18} Salmos 19:4.} \bibverse{19} Así que mi
pregunta es: ``¿No sabía Israel?'' Primero que nada, Moisés dice: ``Les
haré sentir celos usando un pueblo que ni siquiera es una nación; los
haré enojarse usando extranjeros ignorantes.''\footnote{\textbf{10:19}
  Deuteronomio 32:21.} \bibverse{20} Luego Isaías lo dijo con mayor
vehemencia: ``Fui encontrado por personas que ni siquiera me estaban
buscando; me presenté a personas que ni siquiera estaban preguntando por
mí.''\footnote{\textbf{10:20} Isaías 65:1.} \bibverse{21} Como dice Dios
a Israel: ``Todo el día extendí mis manos a un pueblo desobediente y
terco.''\footnote{\textbf{10:21} Isaías 65:2.}

\hypertarget{section-10}{%
\section{11}\label{section-10}}

\bibverse{1} Pero entonces pregunto: ``¿Acaso Dios ha rechazado a su
pueblo?'' ¡Por supuesto que no! Yo mismo soy israelita, de la tribu de
Benjamín. \bibverse{2} Dios no ha rechazado a su pueblo escogido. ¿Acaso
no recuerdan lo que dice la Escritura acerca de Elías? Cómo se quejó de
Israel ante Dios, diciendo: \bibverse{3} ``Señor, han matado a tus
profetas y han destruido tus altares. ¡Soy el único que queda y también
están tratando de matarme!''

\bibverse{4} ¿Cómo le respondió Dios? ``Aun me quedan siete mil personas
que no han adorado a Baal.''\footnote{\textbf{11:4} 1 Reyes 19:10, 14.}
\bibverse{5} Hoy sucede exactamente lo mismo: aún quedan algunas
personas fieles, escogidas por la gracia de Dios. \bibverse{6} Y como es
por medio de la gracia, entonces claramente no se basa en lo que la
gente hace, ¡de otro modo no sería gracia!

\bibverse{7} ¿Qué concluiremos, entonces? Que el pueblo de Israel no
logró aquello por lo que estaba luchando. Solo los escogidos, mientras
que el resto endureció su corazón. \bibverse{8} Como dice la Escritura:
``Dios opacó sus mentes para que sus ojos no pudieran ver y sus oídos no
pudieran oír, hasta el día de hoy.''\footnote{\textbf{11:8} Deuteronomio
  29:4; Isaías 6:9, 10; 29:10.} \bibverse{9} David agrega: ``Que sus
fiestas se conviertan en una trampa para ellos, una red que los atrape,
una tentación que traiga castigo. \bibverse{10} Que sus ojos se vuelvan
ciegos para que no puedan ver, y que sus espaldas siempre estén dobladas
de abatimiento.''\footnote{\textbf{11:10} Salmos 69:22, 23.}

\bibverse{11} Ahora, ¿estoy diciendo que ellos tropezaron y fracasaron
por completo? ¡Por supuesto que no! Pero como resultado de sus errores,
la salvación llegó a otras naciones, para ``hacerlos sentir celos.''
\bibverse{12} Ahora pues, si su fracaso beneficia al mundo, y su pérdida
es de beneficio para los extranjeros, ¡cuánto más benéfico sería si
ellos lograran lo que debían llegar a ser!\footnote{\textbf{11:12}
  Implícito.}

\bibverse{13} Ahora déjenme hablarles a ustedes, extranjeros. En tanto
que soy un misionero para los extranjeros, promuevo lo que hago
\bibverse{14} para que de alguna manera pueda despertar celo en mi
pueblo y salvar a algunos de ellos. \bibverse{15} Si el resultado del
rechazo de Dios hacia ellos es la reconciliación del mundo con Dios,
¡entonces el resultado de la aceptación de Dios hacia ellos sería como
si los muertos volvieran a vivir! \bibverse{16} Si la primera parte de
la masa del pan es santa, también lo es todo el resto; si las raíces de
un árbol son santas, entonces también lo son las ramas. \bibverse{17}
Ahora, si algunas de las ramas han sido arrancadas, y tú---un brote
silvestre de olivo---has sido injertado, y has compartido con las demás
ramas el beneficio de las raíces del árbol de olivo, \bibverse{18}
entonces no debes menospreciar a las demás ramas. Si te sientes tentado
a jactarte, recuerda que no eres tu quien sustenta a las raíces, sino
que las raíces te sustentan a ti. \bibverse{19} Podrías presumir,
diciendo: ``Las ramas fueron cortadas, por ello pueden injertarme a
mí.'' \bibverse{20} Todo eso estaría bien, pero estas ramas fueron
cortadas por su falta de fe en Dios, y tú sigues allí por tu fe en él.
De modo que no te tengas en un alto concepto, sino sé respetuoso,
\bibverse{21} porque si Dios no perdonó a las ramas que originalmente
estaban allí, a ti tampoco te perdonará. \bibverse{22} De modo que debes
reconocer la bondad y también la dureza de Dios, pues fue duro con los
caídos, pero es bondadoso contigo siempre que confíes en su bondad, de
lo contrario también serías cortado. \bibverse{23} Si estas ramas no se
niegan más a confiar en Dios, podrán ser injertadas también, porque Dios
puede injertarlas nuevamente. \bibverse{24} Si tú pudiste ser cortado de
un árbol de olivo, y luego injertado de manera artificial en un árbol de
olivo cultivado, cuánto más fácilmente podrán ser injertadas nuevamente,
de manera natural, las ramas de su propio árbol.

\bibverse{25} Hermanos y hermanas, no quiero que pasen por alto esta
verdad que estaba oculta anteriormente, pues de lo contrario podrían
volverse arrogantes. El pueblo de Israel en parte se ha vuelto terco,
hasta que se complete la venida de los extranjeros. \bibverse{26} Así es
como Israel se salvará\footnote{\textbf{11:26} Esto no busca enseñar
  sobre una salvación universal, sino que a este punto todo Israel (que
  está conformado tanto por extranjeros como por judíos que aceptan la
  salvación por medio de la gracia de Dios) serán salvados.}. Como dice
la Escritura: ``El Salvador vendrá de Sión, y él hará volver a Jacob de
su rebeldía contra Dios. \bibverse{27} Mi promesa para ellos es que
borraré sus pecados.''\footnote{\textbf{11:27} Isaías 59:20, 21; 27:9.}

\bibverse{28} Aunque ellos son enemigos de la buena noticia, ---y esto
los beneficia a ustedes---aún son el pueblo escogido y amado por causa
de sus ancestros. \bibverse{29} Los dones de Dios y su llamado no pueden
retirarse. \bibverse{30} En un tiempo ustedes desobedecieron a Dios,
pero ahora Dios les ha mostrado misericordia como resultado de la
desobediencia de ellos. \bibverse{31} De la misma manera que ellos ahora
son desobedientes como lo eran ustedes, a ellos también se les mostrará
misericordia como la que ustedes recibieron. \bibverse{32} Porque Dios
trató a todos como prisioneros por causa de su desobediencia, a fin de
poder ser misericordioso con todos.

\bibverse{33} ¡Oh cuán profundas son las riquezas, la sabiduría y el
conocimiento de Dios! ¡Cuán increíbles son sus decisiones, y cuán
extraordinarios son sus métodos! \bibverse{34} ¿Quién puede conocer los
pensamientos de Dios? ¿Quién puede darle consejo? \bibverse{35} ¿Quién
le ha dado alguna vez a Dios algo que luego él tuviera la obligación de
pagárselo? \bibverse{36} Todo proviene de él, todo existe por medio de
él, y todo es para él. ¡Gloria a Dios para siempre, amén!

\hypertarget{section-11}{%
\section{12}\label{section-11}}

\bibverse{1} Así que yo los animo, mis hermanos y hermanas, por la
compasión de Dios\footnote{\textbf{12:1} O ``misericordia.''} por
ustedes, que dediquen sus cuerpos como una ofrenda viva que es santa y
agradable a Dios. Esta es la manera lógica de adorar. \bibverse{2} No
sigan los caminos de este mundo; por el contrario, sean transformados
por la renovación espiritual de sus mentes, a fin de que puedan
demostrar cómo es realmente la voluntad de Dios: buena, agradable, y
perfecta. \bibverse{3} Déjenme explicarles a todos ustedes, por la
gracia que se ha dado, que ninguno debería tener un concepto de sí mismo
más alto que el que debería tener. Ustedes deben tener un autoconcepto
realista, conforme a la medida de fe que Dios les ha dado.

\bibverse{4} Así como hay muchas partes del cuerpo, pero no todas hacen
lo mismo, \bibverse{5} del mismo modo nosotros somos un cuerpo en
Cristo, aunque somos muchos. Y todos somos parte de los otros.
\bibverse{6} Cada uno tiene dones diferentes, que varían conforme a la
gracia que se nos ha dado. De modo que si el don consiste en hablar de
Dios, entonces debes hacerlo conforme a tu medida de fe en Dios.
\bibverse{7} Si se trata del ministerio del servicio, entonces debes
servir; si se trata de enseñar, debes enseñar; \bibverse{8} si el don
consiste en animar a otros, entonces debes animar; si el don consiste en
dar, entonces da generosamente; si es el don del liderazgo, entonces
lidera con compromiso; si el don consiste en ser misericordioso,
entonces hazlo con alegría.

\bibverse{9} El amor debe ser genuino. Odien lo malo; aférrense a lo
bueno. \bibverse{10} Dedíquense por completo unos a otros en su amor
como familia, valorando a los demás más que a ustedes mismos.
\bibverse{11} No sean perezosos para el trabajo arduo; sirvan al Señor
con un espíritu entusiasta. \bibverse{12} Permanezcan alegres en la
esperanza que tienen, soporten las pruebas que se presenten, y no dejen
de orar. \bibverse{13} Participen en la provisión para las necesidades
del pueblo de Dios, y reciban con hospitalidad a los extranjeros.
\bibverse{14} Bendigan a quienes los persiguen, bendíganlos y no los
maldigan. \bibverse{15} Alégrense con los que estén alegres; lloren con
los que lloran. \bibverse{16} Piensen los unos en los otros\footnote{\textbf{12:16}
  O, ``Vivan en armonía unos con otros.''}. No se consideren ustedes
mismos más importantes que los demás; vivan humildemente. No sean
arrogantes. \bibverse{17} No paguen mal por mal. Asegúrense de demostrar
a todos que lo que hacen es bueno, \bibverse{18} y en cuanto esté de
parte de ustedes, vivan en paz con todos. \bibverse{19} Queridos amigos,
no procuren la venganza, más bien dejen que Dios sea quien haga
juicio\footnote{\textbf{12:19} Literalmente, ``dar lugar a la ira.''}---tal
como señala la Escritura: ``\,`Es a mí a quien corresponde administrar
la justicia, yo pagaré,' dice el Señor.''\footnote{\textbf{12:19}
  Deuteronomio 32:35.} \bibverse{20} Si quien los odia tiene hambre,
denle de comer; si tiene sed, denle de beber; pues al hacer esto
acumulan carbones ardientes sobre sus cabezas\footnote{\textbf{12:20}
  Queriendo decir que esto les causará gran vergüenza y remordimiento.}.
\bibverse{21} No sean vencidos por el mal, sino conquisten el mal con el
bien.

\hypertarget{section-12}{%
\section{13}\label{section-12}}

\bibverse{1} Todos deben obedecer a las autoridades de gobierno, porque
nadie tiene el poder de gobernar a menos que Dios se lo permita. Estas
autoridades han sido puestas allí por Dios. \bibverse{2} Y quien quiera
que se resista a estas autoridades, se opone a lo que Dios ha
establecido, y los que lo hacen recibirán el merecido juicio por esto.
\bibverse{3} Porque los gobernantes no producen temor a los que hacen el
bien, sino a los que hacen el mal. De modo que si ustedes no quieren
vivir temerosos de las autoridades, entonces hagan lo recto, y tendrán
su aceptación. \bibverse{4} Los que están en el poder son siervos de
Dios, que han sido puestos allí para el propio bien de ustedes. De modo
que si ustedes hacen mal, deben tener temor, ¡no en vano las autoridades
tienen el poder para castigar! Ellos son siervos de Dios, que castigan a
los infractores. \bibverse{5} Por eso es importante que ustedes hagan lo
que se les dice, no por la amenaza de castigo, sino por lo que sus
propias conciencias les dicen. \bibverse{6} Por ello es que ustedes
tienen que pagar impuestos, porque las autoridades son siervos de Dios
que se ocupan de estas cosas. \bibverse{7} Paguen todo lo que deban: los
impuestos a las autoridades de impuestos; muestren respeto a los que
deben recibir respeto, y rindan honra a los que deban recibir honra.
\bibverse{8} No le deban nada a nadie, excepto amor unos a otros, porque
los que aman a su prójimo están cumpliendo la ley.

\bibverse{9} ``No cometan adulterio, no maten, no roben, no deseen para
ustedes las cosas con envidia\footnote{\textbf{13:9} Literalmente,
  ``codicia.''}''---los demás mandamientos están resumidos en esta
declaración: ``Ama a tu prójimo como a ti mismo.'' \bibverse{10} El amor
no hace daño a nadie\footnote{\textbf{13:10} O, ``no lastima a nadie.''},
y de esta manera el amor cumple la ley. \bibverse{11} Ustedes deben
hacer esto porque pueden darse cuenta de cuán urgente es este tiempo,
que ha llegado la hora de que despierten de su sueño. Porque la
salvación está más cerca de nosotros ahora que cuando por primera vez
creímos en Dios. \bibverse{12} ¡La noche casi termina, el día casi está
aquí! Así que despojémonos de nuestras malas obras y vistámonos de la
armadura de la luz. \bibverse{13} Tengamos una conducta apropiada,
demostrando que somos personas que vivimos en la luz. No debemos perder
el tiempo yendo a fiestas y embriagándonos, o teniendo amoríos y
actuando de manera inmoral, o metiéndonos en peleas y andar con celos.
\bibverse{14} Por el contrario, vístanse del Señor Jesucristo y
olvídense de seguir sus deseos pecaminosos.

\hypertarget{section-13}{%
\section{14}\label{section-13}}

\bibverse{1} Acepten a los que todavía están luchando por creer en Dios,
y no tengan discusiones por causa de opiniones personales. \bibverse{2}
Es posible que una persona crea que puede comer de todo, mientras otra,
con una fe más débil, solo come vegetales\footnote{\textbf{14:2} 14:1,
  2. Esto no guarda relación alguna con el tema del vegetarianismo o la
  dieta, sino con la comida ofrecida a ídolos. (Tal como también sucede
  en 1 Corintios 8).}. \bibverse{3} Los que comen de todo no deben
menospreciar a los que no, y los que no comen de todo no deben criticar
a los que sí lo hacen, porque Dios ha aceptado a ambos. \bibverse{4}
¿Qué derecho tienes tú para juzgar al siervo de otro? Es su propio amo
quien decide si está haciendo bien o mal. Con ayuda de Dios, ellos
podrán discernir lo correcto.

\bibverse{5} Hay quienes consideran que algunos días son más importantes
que otros, mientras que otros piensan que todos los días son iguales.
Todos deben estar plenamente convencidos en su propia mente.
\bibverse{6} Los que respetan un día especial, lo hacen para el Señor; y
los que comen sin preocupaciones,\footnote{\textbf{14:6} Comer o no
  comer se refiere a si era correcto o no comer alimentos que habían
  sido llevados como ofrenda a ídolos paganos.} lo hacen también, puesto
que dan las gracias a Dios; mientras tanto, los que evitan comer ciertas
cosas, también lo hacen para el Señor, y del mismo modo, dan gracias a
Dios.

\bibverse{7} Ninguno de nosotros vive para sí mismo, o muere para sí
mismo. \bibverse{8} Si vivimos, vivimos para el Señor, o si morimos,
morimos para el Señor. De modo que ya sea que vivamos o muramos,
pertenecemos al Señor. \bibverse{9} Esta fue la razón por la que Cristo
murió y volvió a la vida, para así ser Señor tanto de los muertos como
de los vivos. \bibverse{10} ¿Por qué, entonces, criticas a tu hermano
creyente? Pues todos estaremos en pie delante del trono en el juicio de
Dios.

\bibverse{11} Pues las Escrituras dicen: ``\,`Tan cierto como yo estoy
vivo,' dice el Señor, `toda rodilla se doblará delante de mí, y toda
lengua declarará que yo soy Dios.'\,''\footnote{\textbf{14:11} Isaías
  45:23.} \bibverse{12} Así que cada uno de nosotros tendrá que rendir
cuenta de sí mismo a Dios. \bibverse{13} Por lo tanto, no nos juzguemos
más unos a otros. Por el contrario, decidamos no poner obstáculos en el
camino de nuestros hermanos creyentes, ni hacerlos caer.

\bibverse{14} Yo estoy seguro---persuadido por el Señor Jesús---que nada
es, en sí mismo, ceremonialmente impuro. Pero si alguno considera que es
impuro, para él es impuro. \bibverse{15} Si tu hermano creyente se
siente ofendido por ti, en términos de comidas, entonces ya tu conducta
no es de amor. No destruyas a alguien por quien Cristo murió por la
comida que eliges comer. \bibverse{16} No permitas que las cosas buenas
que haces sean malinterpretadas--- \bibverse{17} porque el reino de Dios
no tiene que ver con la comida o la bebida, sino con vivir bien, tener
paz y gozo en el Espíritu Santo. \bibverse{18} Todo el que sirve a
Cristo de este modo, agrada a Dios, y es apreciado por los demás.
\bibverse{19} Así que sigamos el camino de la paz, y busquemos formas de
animarnos unos a otros. \bibverse{20} No destruyas la obra de Dios con
discusiones sobre la comida. Todo es limpio, pero estaría mal comer y
ofender a otros. \bibverse{21} Es mejor no comer carne, o no beber vino
ni nada que pueda ser causa del tropiezo de tu hermano creyente.
\bibverse{22} Lo que tú crees, de manera personal, es algo entre tú y
Dios. ¡Cuán felices son los que no se condenan a sí mismos por hacer lo
que creen que es correcto! \bibverse{23} Pero si tienes dudas en cuanto
a si está bien o mal comer algo, entonces no debes hacerlo, porque no
estás convencido de que es correcto. Todo lo que no está basado en la
convicción\footnote{\textbf{14:23} O, ``fe.''} es pecado.\footnote{\textbf{14:23}
  O, ``Pecado es hacer algo que no crees que es correcto.''}

\hypertarget{section-14}{%
\section{15}\label{section-14}}

\bibverse{1} Los que de nosotros son espiritualmente fuertes deben
apoyar a los que son espiritualmente débiles. No debemos simplemente
complacernos a nosotros mismos. \bibverse{2} Todos debemos animar a
otros a hacer lo recto, edificándolos. \bibverse{3} Así como Cristo no
vivió para complacerse a sí mismo, sino que, como la Escritura dice de
él: ``Las ofensas de los que te insultaban han caído sobre
mí.''\footnote{\textbf{15:3} Salmos 69:9.} \bibverse{4} Estas Escrituras
fueron escritas en el pasado para ayudarnos a entender, y para animarnos
a fin de que pudiéramos esperar pacientemente en esperanza.

\bibverse{5} ¡Que Dios, quien nos da paciencia y ánimo, los ayude a
estar en armonía unos con otros como seguidores de Jesucristo,
\bibverse{6} a fin de que puedan, con una sola mente y una sola voz,
glorificar juntos a Dios, el Padre de nuestro Señor Jesucristo!

\bibverse{7} Así que acéptense unos a otros, así como Cristo los aceptó
a ustedes, y denle la gloria a Dios. \bibverse{8} Siempre digo que
Cristo vino como siervo a los judíos\footnote{\textbf{15:8}
  Literalmente, ``de la circuncisión.''} para mostrar que Dios dice la
verdad, manteniendo las promesas hechas a sus antepasados. \bibverse{9}
También vino para que los extranjeros pudieran alabar a Dios por su
misericordia, como dice la Escritura, ``Por lo tanto te alabaré entre
los extranjeros; cantaré alabanzas a tu nombre.''\footnote{\textbf{15:9}
  Samos 18:49.} \bibverse{10} Y también dice: ``¡Extranjeros, celebren
con este pueblo!''\footnote{\textbf{15:10} Deuteronomio 32:43.}
\bibverse{11} Y una vez más, dice: ``Todos ustedes, extranjeros, alaben
al Señor, que todos los pueblos le alaben.''\footnote{\textbf{15:11}
  Salmos 117:1.} \bibverse{12} Y otra vez, Isaías dice: ``El
descendiente de Isaí vendrá a gobernar las naciones, y los extranjeros
pondrán su esperanza en él.''\footnote{\textbf{15:12} Isaías 11:10.
  ``Descendiente de Isaí.'' Se refiere a Isaí, el padre del Rey David,
  quien inició el linaje real.}

\bibverse{13} ¡Que el Dios de esperanza los llene por completo de todo
gozo y paz, como sus creyentes, a fin de que puedan rebosar de esperanza
por el poder del Espíritu Santo! \bibverse{14} Estoy convencido de que
ustedes, mis hermanos y hermanas, están llenos de bondad, y que están
llenos de todo tipo de conocimiento, de modo que están bien capacitados
para enseñarse unos a otros. \bibverse{15} He sido muy directo en la
forma como les he escrito sobre algunas de estas cosas, pero es solo
para recordarles. Porque Dios me dio la gracia \bibverse{16} de ser un
ministro de Jesucristo para los extranjeros, como un sacerdote que
predica la buena noticia de Dios, a fin de que puedan convertirse en una
ofrenda agradable, santificada por el Espíritu Santo.

\bibverse{17} Así que, aunque tenga algo de qué jactarme por mi servicio
a Dios, \bibverse{18} (no me atrevería a hablar de ninguna de estas
cosas, excepto cuando Cristo mismo lo ha hecho a través de mi), he
conducido a los extranjeros a la obediencia a través de mi enseñanza y
ejemplo, \bibverse{19} a través del poder de señales y milagros
realizados por el poder del Espíritu Santo. Desde Jerusalén hasta
Ilírico, por todos lados he compartido enteramente la buena noticia de
Cristo. \bibverse{20} De hecho, con mucho agrado compartí el evangelio
en lugares donde no habían escuchado el nombre de Cristo, para no
construir sobre lo que otros habían hecho. \bibverse{21} Como dice la
Escritura: ``Los que no han oído de la buena noticia la descubrirán, y
los que no han oído entenderán.''\footnote{\textbf{15:21} Isaías 52:15.}

\bibverse{22} Por ello muchas veces me fue imposible venir a verlos.
\bibverse{23} Pero ahora, como no hay más lugar aquí donde trabajar, y
como he deseado visitarlos desde hace años, \bibverse{24} espero verlos
cuando vaya de camino a España. Quizás puedan brindarme ayuda para el
viaje, después de pasar juntos por algún tiempo. \bibverse{25} Ahora voy
de camino a Jerusalén para ayudar a los creyentes que están allá,
\bibverse{26} porque los creyentes en Macedonia y Acaya pensaron que
sería bueno enviar una ayuda a los pobres que están entre los creyentes
de Jerusalén. \bibverse{27} Estuvieron felices de ayudarlos porque
tienen esta deuda con ellos\footnote{\textbf{15:27} Queriendo decir que
  los extranjeros están en deuda con los judíos por compartir la buena
  noticia de Dios. Este ejemplo en particular se aplica de manera
  específica a los creyentes en Jerusalén, es decir, que los extranjeros
  están felices de enviarles un regalo para ayudarlos.}. Ahora que los
extranjeros son partícipes de sus beneficios espirituales, están en
deuda con los creyentes judíos\footnote{\textbf{15:27} Implícito.} para
ayudarlos con cosas materiales. \bibverse{28} De modo que cuando haya
terminado con esto, y les haya entregado de manera segura esta
contribución, los visitaré a ustedes de camino a España. \bibverse{29}
Sé que cuando venga, Cristo nos dará su plena bendición.

\bibverse{30} Deseo animarlos, mis hermanos y hermanas, mediante nuestro
Señor Jesucristo y mediante el amor del Espíritu, a que se unan y oren
mucho por mí. \bibverse{31} Oren para que pueda estar a salvo de los no
creyentes de Judea. Oren para que mi labor en Jerusalén sea bien
recibida por los creyentes de allí. \bibverse{32} Oren para que pueda
regresar a ustedes con alegría, conforme a la voluntad de Dios, para que
podamos disfrutar juntos, en compañía. \bibverse{33} Que el Dios de paz
esté con todos ustedes. Amén.

\hypertarget{section-15}{%
\section{16}\label{section-15}}

\bibverse{1} Les encomiendo a nuestra hermana Febe, quien es diaconisa
en la iglesia de Cencrea. \bibverse{2} Por favor, recíbanla en el Señor,
como deben hacerlo los creyentes, y ayúdenla en todo lo que necesite,
porque ha sido de gran ayuda para mucha gente, incluyéndome a mí.
\bibverse{3} Envíen mi saludo a Priscila y Aquila, mis compañeros de
trabajo en Cristo Jesús, \bibverse{4} quienes arriesgaron su vida por
mí. No solo yo estoy agradecido con ellos, sino con todas las iglesias
de los extranjeros también\footnote{\textbf{16:4} Refiriéndose a las
  Iglesias no judías.}. \bibverse{5} Por favor, también salúdenme a la
iglesia que se reúne en su hogar. Den mis mejores deseos a mi buen amigo
Epeneto, la primera persona en seguir a Cristo en la provincia de Asia.
\bibverse{6} Envíen mis saludos a María, que ha trabajado mucho por
ustedes, \bibverse{7} y también a Andrónico y a Junías, judíos como yo,
y compañeros en la cárcel. Ellos son muy bien conocidos entre los
apóstoles y se convirtieron en seguidores de Cristo antes que yo.
\bibverse{8} Envíen mis mejores deseos a Amplias, mi buen amigo en el
Señor; \bibverse{9} a Urbano, nuestro compañero de trabajo en Cristo; y
a mi querido amigo Estaquis. \bibverse{10} Saludos a Apeles, un hombre
fiel en Cristo. Saludos a la familia de Aristóbulo, \bibverse{11} a mi
conciudadano Herodión, y a los de la familia de Narciso, que pertenecen
al Señor. \bibverse{12} Mis mejores deseos a Trifaena y Trifosa,
trabajadores diligentes del Señor, y a mi amiga Pérsida, que ha
trabajado mucho en el Señor. \bibverse{13} Den mis saludos a Rufo, un
trabajador excepcional\footnote{\textbf{16:13} O, ``uno del pueblo
  especial de Dios.''}, y a su madre, a quien considero como mi madre
también. \bibverse{14} Saludos a Asíncrito, a Flegontes, a Hermes, a
Patrobas, a Hermas, y a los creyentes que están con ellos. \bibverse{15}
Mis mejores deseos a Filólogo y Julia, a Nereo y a su hermana, a Olimpas
y a todos los creyentes que están con ellos. \bibverse{16} Salúdense
unos a otros con afecto. Todas las iglesias de Cristo les envían
saludos.

\bibverse{17} Ahora les ruego, mis hermanos creyentes: cuídense de los
que causan discusiones y confunden a las personas de la enseñanza que
han aprendido. ¡Aléjense de ellos! \bibverse{18} Estas personas no
sirven a Cristo nuestro Señor sino a sus propios apetitos, y con su
forma de hablar lisonjera y palabras agradables engañan las mentes de
las personas desprevenidas.

\bibverse{19} Todos saben cuán fieles son ustedes y eso me llena de
alegría. Sin embargo, quiero que sean sabios en cuanto a lo que es
bueno, e inocentes de lo malo. \bibverse{20} El Dios de paz pronto
quebrantará el poder de Satanás y lo someterá a ustedes. Que la gracia
de nuestro Señor Jesucristo esté con ustedes.

\bibverse{21} Timoteo, mi compañero de trabajo, envía sus saludos, así
como Lucio, Jasón y Sosípater, quienes también son judíos. \bibverse{22}
Tercio---quien escribe esta carta---también los saluda en el Señor.
\bibverse{23} Gayo, quien me dio hospedaje, y toda la iglesia de aquí
también los saludan. Erasto, el tesorero de la ciudad, envía sus mejores
deseos a ustedes, así como nuestro hermano Cuarto. \bibverse{24}
\footnote{\textbf{16:24} Los primeros manuscritos no incluyen el
  versículo 24.}

\bibverse{25} Ahora, a Aquél que puede fortalecerlos, mediante la buena
noticia que yo comparto y el mensaje de Jesucristo,

Conforme al misterio de verdad\footnote{\textbf{16:25} Literalmente,
  ``misterio,'' un término que se refiere a una verdad secreta o a un
  plan secreto que es conocido solo por los creyentes religiosos. Ver
  también, versículo 26.} que ha sido revelado,

El misterio de verdad, oculto por la eternidad, \bibverse{26} y ahora
visible.

A través de los escritos de los profetas, y siguiendo el mandato del
Dios eterno,

El misterio de la verdad es dado a conocer a todos, en todos lados a fin
de que puedan creer y obedecerle;

\bibverse{27} Al único Dios sabio,

A través de Jesucristo.

A él sea la gloria para siempre. Amén+ 16.27 Estos últimos versículos
parecen ser un poema o canción, por ello están estructurados de esta
manera..
