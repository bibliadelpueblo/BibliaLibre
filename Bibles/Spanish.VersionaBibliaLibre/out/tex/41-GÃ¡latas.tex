\hypertarget{section}{%
\section{1}\label{section}}

\bibverse{1} Esta carta viene de Pablo, un apóstol no designado por
ninguna organización o autoridad humana\footnote{\textbf{1:1}
  Literalmente, ``no por hombres, ni a través de hombre.''}. Muy por el
contrario, fui designado por Jesucristo y Dios, el Padre, quien levantó
a Jesús de entre los muertos. \bibverse{2} Todos los hermanos y hermanas
que están aquí conmigo se han unido para enviar esta carta a las
iglesias de Galacia.

\bibverse{3} Que la gracia y la paz de Dios el Padre y de nuestro Señor
Jesucristo esté con ustedes.

\bibverse{4} Jesús se entregó a sí mismo por nuestros pecados para
liberarnos de este mundo actual de maldad, siguiendo la voluntad de
nuestro Dios y Padre. \bibverse{5} A él sea la gloria por siempre y para
siempre. Amén.

\bibverse{6} Estoy sorprendido de cuán rápidamente están abandonando al
Dios que los llamó por la gracia de Cristo. Se están convirtiendo a otro
tipo de buena noticia \bibverse{7} ¡una que no es ninguna buena noticia
en absoluto! Hay algunos por ahí confundiéndolos, queriendo pervertir la
buena noticia de Cristo. \bibverse{8} Pero si alguno, incluso nosotros
mismos, o incluso si un ángel del cielo promoviera cualquier otro tipo
de buena noticia\footnote{\textbf{1:8} Claramente Pablo no cree que esto
  sea en absoluto una buena noticia, así que posiblemente debería ir
  entre comillas, a manera de ironía: ``buena noticia.''} que la que ya
les hemos enseñado, que sea condenado. \bibverse{9} Les vuelvo a decir
lo que ya les he dicho antes: ¡si alguno promueve cualquier otro tipo de
buena noticia\footnote{\textbf{1:9} Tal como en 1:8.} distinta a la que
ya ustedes han aceptado, que sea condenado!

\bibverse{10} ¿De quién creen que quiero aprobación? ¿De la gente o de
Dios? ¿Creen que intento agradar a la gente? ¡Si quisiera hacerlo, no
sería un siervo de Cristo!

\bibverse{11} Permítanme aclarar esto, amigos míos, respecto a la buena
noticia que estoy declarando: Que no vino de ningún ser humano.
\bibverse{12} No la recibí de nadie, y nadie me la enseñó. Fue Cristo
Jesús mismo quien me la reveló. \bibverse{13} Ustedes oyeron sobre mi
conducta como seguidor de la religión judía, y cómo perseguí con
fanatismo a la iglesia de Dios, tratando de destruirla de manera
salvaje. \bibverse{14} Incluso superé a mis contemporáneos en la
práctica de la religión judía porque era un seguidor celoso de las
tradiciones de mis ancestros.

\bibverse{15} Pero en el momento que Dios (quien me había separado desde
mi nacimiento) me llamó por su gracia, y se complació \bibverse{16} en
revelarme a su Hijo, a fin de que pudiera anunciar la buena noticia a
las naciones\footnote{\textbf{1:16} O ``gentiles.''}, y esto no lo
discuto con nadie. \bibverse{17} No fui a Jerusalén para hablarle a los
que me precedieron como apóstoles; en lugar de ello fui a Arabia, y más
tarde regresé a Damasco. \bibverse{18} Después de tres años fui a
Jerusalén a visitar a Pedro. Me quedé allí dos semanas con él.
\bibverse{19} Tampoco vi a otros apóstoles, excepto a Santiago, el
hermano del Señor. \bibverse{20} (¡Permítanme asegurarles ante Dios que
no miento sobre las cosas que les estoy escribiendo!) \bibverse{21}
Luego fui a Siria y a Cilicia. \bibverse{22} Aún así, los que estaban en
las iglesias de Judea no me habían visto personalmente. \bibverse{23}
Ellos solo escuchaban a la gente decir: ``¡El hombre que solía
perseguirnos ahora está esparciendo la fe que una vez intentó
destruir!'' \bibverse{24} Y alababan a Dios por causa de mí.

\hypertarget{section-1}{%
\section{2}\label{section-1}}

\bibverse{1} Catorce años más tarde, regresé a Jerusalén con Bernabé.
Entonces llevé conmigo a Tito. \bibverse{2} Fui por causa de lo que Dios
me había mostrado\footnote{\textbf{2:2} Literalmente, ``según la
  revelación.''}. Me reuní en privado con los líderes reconocidos de la
iglesia allí y les expliqué sobre la buena noticia que estaba
compartiendo con los extranjeros\footnote{\textbf{2:2} Literalmente,
  ``gentiles.''}. No quería continuar el camino que hasta ese momento
había seguido, y por el cual había trabajado tanto, y que al final fuera
en vano. \bibverse{3} Pero sucedió que estando allá nadie insistió en
que Tito, quien iba conmigo, fuera circuncidado, aunque él era griego.
\bibverse{4} (Ese asunto solo surgió porque algunos falsos cristianos se
habían infiltrado para espiar la libertad que tenemos en Cristo Jesús,
tratando de convertirnos en esclavos. \bibverse{5} Pero nunca cedimos a
ellos, ni siquiera por un momento, sino que queríamos asegurarnos de
mantener la verdad de la buena noticia intacta para ustedes.)

\bibverse{6} Pero aquellos considerados como importantes, no añadieron
cosa alguna\footnote{\textbf{2:6} O, ``no hicieron ningún cambio.''} a
lo que dije. (No me importa qué clase de líderes eran, pero Dios no
juzga a las personas del mismo modo que yo lo hago.) \bibverse{7} Por el
contrario, cuando se dieron cuenta de que se me había dado la
responsabilidad de compartir la buena noticia con los extranjeros, del
mismo modo que a Pedro se le había dado la responsabilidad de compartir
la buena noticia con los judíos, \bibverse{8} (pues el mismo
Dios\footnote{\textbf{2:8} Literalmente, ``el Único.''} que obraba en
Pedro como apóstol a los judíos, también obraba a través de mi como
apóstol a los extranjeros), \bibverse{9} y cuando reconocieron también
la gracia que me había sido dada, entonces Santiago, Pedro y Juan,
quienes llevaban la responsabilidad\footnote{\textbf{2:9} 2:9.
  Literalmente, ``considerados como pilares.''} de ejercer el liderazgo
de la iglesia, estrecharon sus manos conmigo y Bernabé, aceptándonos
como sus compañeros de trabajo. \bibverse{10} Nosotros trabajaríamos por
los extranjeros, mientras ellos trabajarían por los judíos. Su única
instrucción fue que recordáramos cuidar de los pobres, algo con lo que
ya estaba muy comprometido.

\bibverse{11} Sin embargo, cuando Pedro fue a Antioquía, tuve que
confrontarlo directamente, porque evidentemente estaba equivocado en lo
que hacía. \bibverse{12} Antes de que algunos de los amigos de Santiago
llegaran, Pedro solía comer con los extranjeros. Pero cuando estas
personas llegaron, dejó de hacerlo y se alejó de los extranjeros. Él
temía ser criticado por los que insistían en que los hombres debían ser
circuncidados. \bibverse{13} Así como Pedro, otros judíos cristianos se
volvieron hipócritas también, al punto que incluso Bernabé fue
persuadido a seguir su misma hipocresía.

\bibverse{14} Cuando comprendí que no tenían una posición firme en
cuanto a la verdad de la buena noticia, le dije a Pedro delante de
todos: ``Si eres judío pero vives como los extranjeros y no como judío,
¿por qué obligas a los extranjeros a vivir como judíos? \bibverse{15}
Podemos ser judíos por nacimiento, y no `pecadores' como los
extranjeros, \bibverse{16} pero sabemos que nadie es justificado por
hacer lo que la ley exige, sino solo por la fe en Jesucristo. Nosotros
hemos confiado en Cristo Jesús a fin de que pudiéramos ser justificados
al poner nuestra confianza en Cristo, y no por hacer lo que la ley dice,
porque nadie es justificado por la observación de los requisitos de la
ley.''

\bibverse{17} Porque si, al intentar ser justificados en Cristo,
nosotros mismos demostramos ser pecadores, ¿significa eso que Cristo
está al servicio del pecado?\footnote{\textbf{2:17} La idea que se
  expresa aquí es que al renunciar a la observancia de la ley judía, nos
  convertimos en pecadores, y Cristo nos ha conducido al pecado, un
  concepto que Pablo rechaza enérgicamente.} ¡Por supuesto que no!
\bibverse{18} Pues si tuviera que reconstruir lo que he destruido,
entonces solo demuestro que soy un transgresor de la ley\footnote{\textbf{2:18}
  En otras palabras, si vuelvo al antiguo sistema de usar ley como medio
  para ser justificado delante de Dios, lo único que lograré es
  demostrar que estoy violando la ley como pecador.}. \bibverse{19}
Porque a través de la ley morí a la ley para poder vivir para Dios.
\bibverse{20} He sido justificado con Cristo, de modo que ya no soy yo
quien vive, sino Cristo quien vive en mí. La vida que ahora vivo en este
cuerpo, la vivo confiando en el Hijo de Dios, quien me amó y se entregó
por mí. \bibverse{21} ¿Cómo podría rechazar la gracia de Dios? ¡Pues si
podemos ser justificados por guardar la ley, entonces la muerte de
Cristo fue en vano!

\hypertarget{section-2}{%
\section{3}\label{section-2}}

\bibverse{1} ¡Oh, gálatas, cuán insensatos!\footnote{\textbf{3:1} La
  palabra que se utiliza aquí a menudo es traducida como ``necios''; sin
  embargo, hoy se ha vuelto más un epíteto. El punto de Pablo es que no
  están pensando las cosas, la palabra realmente significa
  ``irracionales.'' La misma palabra se usa nuevamente en el versículo
  3.} ¿Quién los puso bajo hechizo? ¡La muerte de Jesucristo en una cruz
les fue mostrada claramente para que pudieran ver! \bibverse{2} Díganme,
entonces, ¿recibieron el Espíritu por guardar la ley o por creer en lo
que habían oído? \bibverse{3} ¡En realidad han perdido la sensatez!
Comenzaron a vivir\footnote{\textbf{3:3} Que quiere decir: ``Ustedes
  comenzaron su vida cristiana.''} en el Espíritu. ¿Realmente creen que
pueden volverse perfectos por sus propios esfuerzos humanos\footnote{\textbf{3:3}
  O, ``por medios humanos.''}? \bibverse{4} ¿Sufrieron tanto para nada?
(Realmente no fue para nada, ¿o sí?) \bibverse{5} Permítanme
preguntarles esto: ¿Acaso Dios\footnote{\textbf{3:5} Literalmente, ``El
  Único.''} les dio el Espíritu y realiza tantos milagros entre ustedes
por el hecho de que ustedes guardan la ley, o porque confían en lo que
han oído?

\bibverse{6} Es como Abraham, que ``confío en Dios, y fue considerado
como hombre justo.''\footnote{\textbf{3:6} Génesis 15:6.} \bibverse{7}
De modo que ustedes deben reconocer que los que creen en Dios son los
hijos de Abraham. \bibverse{8} En la Escritura estaba predicho que Dios
justificaría a los extranjeros que creyeran en él. La buena noticia fue
revelada a Abraham de antemano con las palabras: ``A través de ti serán
benditas todas las naciones.'' \bibverse{9} En consecuencia, los que
creen en Dios son bendecidos junto a Abraham, que confió en Dios.
\bibverse{10} Todos los que dependen del cumplimiento de la
ley\footnote{\textbf{3:10} Como medio de salvación.} están bajo
maldición, porque como dice la Escritura: ``Maldito es todo aquél que no
guarda cuidadosamente todo lo que está escrito en el libro de la ley.''
\bibverse{11} Está claro que nadie es justificado delante de Dios por el
intento de guardar la ley, porque ``los justos vivirán por su fe en
Dios.''\footnote{\textbf{3:11} Habacuc 2:4.} \bibverse{12} Y la
obediencia a la ley no tiene que ver con la fe en Dios. La Escritura
solo dice: ``Vivirán si observan todo lo que la ley exige.''\footnote{\textbf{3:12}
  Levítico 18:5.} \bibverse{13} Pero Cristo nos ha rescatado de la
maldición de la ley al convertirse en maldición por nosotros. Como dice
la Escritura: ``Maldito todo aquél que es colgado en un
madero.''\footnote{\textbf{3:13} Deuteronomio 21:23.} \bibverse{14} De
modo que a través de Cristo Jesús la bendición de Abraham pudo llegar
también a los extranjeros, y nosotros pudimos recibir la promesa del
Espíritu por nuestra fe en Dios.

\bibverse{15} Hermanos y hermanas, aquí tenemos un ejemplo de la vida
diaria. Si se alista un contrato y este es acordado, firmado y sellado,
nadie puede ignorarlo o añadirle más cosas. \bibverse{16} Pues las
promesas les fueron dadas a Abraham y a su hijo.\footnote{\textbf{3:16}
  Literalmente, ``semilla.''} No dice ``hijos,'' en plural, sino en
singular: ``y a tu hijo,'' queriendo decir, Cristo. \bibverse{17}
Déjenme explicarles: La ley, que llegó cuatrocientos treinta años
después, no cancela el pacto anterior que Dios había hecho, quebrantando
la promesa. \bibverse{18} Si la herencia se deriva de la obediencia a la
ley, ya no proviene de la promesa. Pero Dios, por su gracia, le dio esta
herencia a Abraham por medio de la promesa.

\bibverse{19} ¿Qué sentido tiene la ley, entonces? Fue dada para mostrar
lo que realmente es el mal, hasta que el Hijo vino a los que se les
había hecho la promesa. La ley fue introducida por ángeles, por mano de
un mediador. \bibverse{20} Pero no se necesita de un mediador cuando hay
una sola persona involucrada. ¡Y Dios es uno!\footnote{\textbf{3:20} El
  concepto que se expresa aquí es que el Antiguo Testamento necesitaba
  un mediador (Moisés). Pero en el caso de la promesa, esta fue hecha
  directamente a Abrahán, y de acuerdo al argumento que Pablo desarrolla
  aquí, la promesa se cumplió directamente mediante Jesucristo. De este
  modo, dice Pablo, la promesa y su cumplimiento son superiores a la
  ley.}

\bibverse{21} ¿De modo que la ley obra en contra de las promesas de
Dios? ¡Por supuesto que no! Porque si hubiera una ley que pudiera dar
vida, entonces nosotros podríamos ser justificados por el cumplimiento
de ella. \bibverse{22} Pero la Escritura nos dice que todos somos
prisioneros del pecado. El único modo en que podemos recibir las
promesas de Dios es por la fe en Jesucristo. \bibverse{23} Antes de que
confiáramos en Jesús permanecíamos bajo custodia de la ley hasta que se
reveló este camino de la fe. \bibverse{24} La ley fue nuestro guardián
hasta que vino Cristo, para que pudiéramos ser justificados por la fe en
él. \bibverse{25} Pero ahora que ha llegado este camino de fe en Jesús,
ya no necesitamos de tal guardián. \bibverse{26} Porque ustedes son
hijos de Dios por medio de su fe en Jesucristo. \bibverse{27} Todos los
que de ustedes fueron bautizados en Cristo se han vestido de Cristo.
\bibverse{28} Ya no hay más judío o griego, esclavo o libre, hombre o
mujer, pues ustedes todos son uno en Cristo Jesús. \bibverse{29} ¡Si son
de Cristo, son hijos de Abraham, y herederos de la promesa!

\hypertarget{section-3}{%
\section{4}\label{section-3}}

\bibverse{1} Permítanme explicarles lo que estoy diciendo. Un heredero
que es menor de edad no es distinto a un esclavo, aunque el heredero sea
el propietario de todo. \bibverse{2} Pues un heredero está sujeto a los
guardias y administradores hasta que llegue el tiempo establecido por su
padre. \bibverse{3} Lo mismo sucede con nosotros. Cuando éramos niños,
éramos esclavos sujetos a las reglas básicas\footnote{\textbf{4:3} La
  palabra traducida como ``reglas'' aquí está sujeta a amplia
  interpretación. Originalmente, la palabra se refería al alfabeto. Más
  tarde tomó el significado de ``Abecés de la vida. Pablo compara la ley
  ceremonial con tales letras y símbolos, que son instrucciones básicas
  y útiles para la existencia pero que no tienen poder para salvar y
  sanar. La misma palabra se usa en el versículo 9.} de la ley.
\bibverse{4} Pero en el momento apropiado Dios envió a su hijo, nacido
de una mujer, nacido bajo el gobierno de la ley, \bibverse{5} para poder
rescatar a los que fueron cautivos bajo el dominio de la ley, a fin de
que pudiéramos recibir la heredad de hijos adoptivos.

\bibverse{6} Para demostrar que ustedes son sus hijos, Dios envió al
Espíritu de su Hijo a nuestros corazones, haciéndonos clamar: ``Abba,''
que quiere decir ``Padre.'' \bibverse{7} Puesto que ya no eres un
esclavo, sino un hijo, y si eres su hijo, entonces Dios te ha convertido
en su heredero.

\bibverse{8} Cuando ustedes no conocían a Dios, estaban esclavizados por
los supuestos ``dioses'' de este mundo. \bibverse{9} Pero ahora han
llegado a conocer a Dios, o mejor aún, han llegado a ser conocidos por
Dios. ¿Cómo pueden volver, entonces, a esas reglas inútiles y sin valor?
¿Quieren ser esclavos de esas reglas nuevamente? \bibverse{10} Ustedes
observan días especiales y meses, temporadas y años\footnote{\textbf{4:10}
  Esto se refiere a la observancia de días de fiestas especiales y
  épocas en el sistema del Antiguo Testamento.}. \bibverse{11} Y me
preocupa que todo lo que hice por ustedes haya sido tiempo perdido.

\bibverse{12} Les ruego, mis amigos: sean como yo, porque yo me volví
como ustedes\footnote{\textbf{4:12} En otras palabras, un ``gentil
  liberado.''}. Ustedes nunca me trataron mal. \bibverse{13} Recuerden
que compartí la buena noticia con ustedes porque estaba enfermo durante
mi primera visita\footnote{\textbf{4:13} Parece ser que como Pablo
  estaba retrasado por su enfermedad, tuvo la oportunidad de compartir
  la buena noticia con los gálatas.}. \bibverse{14} Y aunque mi
enfermedad fue muy incómoda para ustedes, no me rechazaron ni me
despreciaron, sino que de hecho, me trataron como a un ángel de Dios,
como a Jesucristo mismo.

\bibverse{15} ¿Entonces qué ha pasado con su gratitud? ¡Déjenme decirles
que en ese tiempo, si ustedes hubieran podido sacarse los ojos para
dármelos a mí, de seguro lo habrían hecho! \bibverse{16} ¿Qué es lo que
ha ocurrido, entonces? ¿Me he convertido en enemigo de ustedes por decir
la verdad? \bibverse{17} Estas personas anhelan tener el apoyo de
ustedes, pero no es con buenas intenciones. Por el contrario, quieren
alejarlos de nosotros a fin de que ustedes se entusiasmen para
apoyarlos. \bibverse{18} Por supuesto, es bueno hacer el bien. ¡Pero
debería ser todo el tiempo, no solo cuando yo estoy aquí con
ustedes!\footnote{\textbf{4:18} Esto sugiere que estas personas querían
  ser de ayuda a los Gálatas solamente para lograr sus propios fines.}
\bibverse{19} Mis queridos amigos, quiero trabajar a su lado hasta que
el carácter de Cristo se haya duplicado en ustedes. \bibverse{20}
Desearía poder acompañarlos ahora mismo y así podrían notar cómo cambio
el tono de mi voz\ldots{} Estoy muy preocupado por ustedes.

\bibverse{21} Respóndanme esto, ustedes que quieren vivir bajo la ley:
¿No escuchan lo que la ley está diciendo? \bibverse{22} Como dice la
Escritura: Abraham tenía dos hijos, uno de la sierva y otro de la mujer
libre. \bibverse{23} Sin embargo, el hijo de la sierva nació por planes
humanos\footnote{\textbf{4:23} Refiriéndose al plan de Sara para tener
  un hijo por medio de la esclava.}, mientras el hijo de la mujer libre
nació como resultado de la promesa. \bibverse{24} Esto nos muestra una
analogía: estas dos mujeres representan dos pactos. Un pacto es del
Monte Sinaí---Agar---y ella da a luz hijos esclavos. \bibverse{25} Agar
simboliza al Monte Sinaí en Arabia, y corresponde a la Jerusalén actual,
porque ella está en esclavitud con sus hijos. \bibverse{26} Pero la
Jerusalén celestial es libre. Ella es nuestra madre.

\bibverse{27} Como dice la Escritura: ``¡Regocíjense las que no tienen
hijos y las que nunca han parido! ¡Griten de alegría, las que nunca han
estado en labores de parto, porque la mujer abandonada tiene más hijos
que la mujer que tiene esposo!''\footnote{\textbf{4:27} Isaías 54:1.}
\bibverse{28} Ahora, amigos míos, nosotros somos hijos de la promesa tal
como Isaac. \bibverse{29} Sin embargo, así como el que nació por planes
humanos persiguió al que nació por el Espíritu, del mismo modo ocurre
hoy. \bibverse{30} Pero ¿qué dice la Escritura? ``Despidan a la sierva y
a su hijo, porque el hijo de la sierva no será heredero junto al hijo de
la mujer libre.''\footnote{\textbf{4:30} Génesis 21:10.} \bibverse{31}
Por lo tanto, mis amigos, no somos hijos de la sierva, sino de la mujer
libre.

\hypertarget{section-4}{%
\section{5}\label{section-4}}

\bibverse{1} Cristo nos libertó para que pudiéramos tener verdadera
libertad. Así que estén firmes y no se agobien nuevamente por el yugo de
la esclavitud. \bibverse{2} Permítanme decirles francamente: si dependen
del camino de la circuncisión, Cristo no les será de beneficio en
absoluto. \bibverse{3} Permítanme repetir: todo hombre que es
circuncidado tiene que cumplir toda la ley. \bibverse{4} Los que entre
ustedes creen que pueden ser justificados por la ley, están separados de
Cristo y han abandonado la gracia.

\bibverse{5} Porque por medio del Espíritu creemos y aguardamos la
esperanza de ser justificados. \bibverse{6} Porque en Cristo Jesús, ser
circuncidado o no circuncidado no logra nada; lo único que importa es la
fe que obra por el amor. \bibverse{7} ¡Lo estaban haciendo muy bien!
¿Quién se interpuso en el camino y les impidió convencerse de la verdad?
\bibverse{8} Esta ``persuasión'' sin duda no proviene de Aquél que los
llama. \bibverse{9} Ustedes solo necesitan un poco de levadura para que
crezca toda la masa. \bibverse{10} Estoy seguro en el Señor que ustedes
no cambiarán su manera de pensar, y que el que los está confundiendo
afrontará las consecuencias\footnote{\textbf{5:10} O, ``juicio.''},
quienquiera que sea.

\bibverse{11} En cuanto a mí, hermanos y hermanas, si aún estamos en
favor de la circuncisión, ¿por qué me siguen persiguiendo? Si eso fuera
cierto, eliminaría el tema de la cruz, que tanto ofende a la gente.
\bibverse{12} ¡Ojalá quienes los agobian fueran más allá de la
circuncisión y se castraran!\footnote{\textbf{5:12} No debe tomarse de
  manera literal, por supuesto, sino simbólicamente, llevando la
  filosofía de la circuncisión al extremo.}

\bibverse{13} ¡Ustedes, mis hermanos y hermanas, fueron llamados para
ser libres! Simplemente no usen su libertad como excusa para satisfacer
su naturaleza pecaminosa. En lugar de ello, sírvanse unos a otros en
amor. \bibverse{14} Pues toda la ley se resume en este mandamiento:
``Amarás a tu prójimo como a ti mismo.'' \bibverse{15} Pero si se atacan
y se destruyen unos a otros, cuídense de no destruirse ustedes mismos
por completo. \bibverse{16} Mi consejo es que caminen por el Espíritu.
No satisfagan los deseos de su naturaleza pecaminosa. \bibverse{17}
Porque los deseos de la naturaleza pecaminosa son contrarios al
Espíritu, y los deseos del Espíritu son opuestos a la naturaleza
pecaminosa. Se pelean entre sí, de modo que ustedes no hacen lo que
quieren hacer. \bibverse{18} Pero si el Espíritu los guía, no están bajo
la ley.

\bibverse{19} Es claro lo que la naturaleza pecaminosa trae como
resultado: inmoralidad sexual, indecencia, sensualidad, \bibverse{20}
idolatría, hechicería, odio, rivalidad, celos, rabia, ambición egoísta,
disensión, herejía, \bibverse{21} envidia, embriaguez, banquetes, y
cosas semejantes. Tal como les advertí antes, les vuelvo a advertir:
ninguna persona que se comporte de esta manera heredará el reino de
Dios.

\bibverse{22} Pero el fruto del Espíritu es amor, gozo, paz, paciencia,
benignidad, bondad, fe, \bibverse{23} mansedumbre y dominio propio. ¡No
hay ley que se oponga a estas cosas! \bibverse{24} Los que pertenecen a
Cristo han clavado en la cruz su naturaleza humana pecaminosa, junto con
todas sus pasiones y deseos pecaminosos. \bibverse{25} Si vivimos en el
Espíritu debemos caminar también en el Espíritu. \bibverse{26} No nos
volvamos jactanciosos, ni nos irritemos unos a otros, ni tengamos
envidia unos de otros.

\hypertarget{section-5}{%
\section{6}\label{section-5}}

\bibverse{1} Mis amigos, si alguno se extravía por causa del pecado,
ustedes, que son espirituales, deberían traerle de regreso con espíritu
de mansedumbre. Y cuídense de no ser tentados también. \bibverse{2}
Sobrelleven unos las cargas de los otros, pues de esta manera cumplen la
ley de Cristo. \bibverse{3} Los que creen que son importantes---cuando
realmente no son nada---se engañan a sí mismos. \bibverse{4} Examinen
cuidadosamente sus acciones. Así podrán estar satisfechos de ustedes
mismos, sin compararse con nadie más. \bibverse{5} Debemos ser
responsables de nosotros mismos.

\bibverse{6} Aquellos que reciben enseñanza de la Palabra deben tratar
bien a sus maestros, compartiendo con ellos todas las cosas buenas.
\bibverse{7} No se dejen engañar, Dios no puede ser tratado con
desacato: todo lo que siembren, eso cosecharán. \bibverse{8} Si ustedes
siembran conforme a su naturaleza humana pecaminosa, de esa misma
naturaleza segarán autodestrucción. Pero si siembran conforme al
Espíritu, cosecharán vida eterna. \bibverse{9} No nos cansemos nunca de
hacer el bien, pues segaremos en el momento apropiado, si somos
perseverantes. \bibverse{10} Así que mientras tengamos
tiempo\footnote{\textbf{6:10} Es decir, la oportunidad.}, hagamos bien a
todos, especialmente a los que pertenecen a la familia de la fe.

\bibverse{11} ¡Miren cuán grandes son las letras, ahora que les escribo
con mi propia mano! \bibverse{12} Esas personas que solo quieren dar una
buena impresión los están obligando a circuncidarse para no ser
perseguidos ellos por la cruz de Cristo. \bibverse{13} Incluso los que
están circuncidados no guardan la ley, pero quieren que ustedes se
circunciden para poder jactarse de ustedes y decir que ustedes son sus
seguidores+ 6.13 La jactancia de estas personas está en que han
convencido a otros de seguir su creencia en cuanto a la importancia del
rito judío de la circuncisión (y otras prácticas judías, que es el
problema que se destaca a los largo del libro de Gálatas)..
\bibverse{14} Ojalá yo nunca me jacte de nada, excepto en la cruz de
nuestro Señor Jesucristo. Por medio de esta cruz, el mundo ha sido
crucificado para mí, y yo he sido crucificado en lo que tiene que ver
con el mundo. \bibverse{15} La circuncisión o la incircuncisión no
importan. ¡Lo que importa es que fuimos creados nuevamente!
\bibverse{16} ¡Paz y misericordia a todos los que siguen este principio,
y al Israel de Dios! \bibverse{17} Por favor, no me agobien más, porque
llevo en mi cuerpo las marcas de Jesús+ 6.17 En otras palabras, las
heridas que Pablo recibía cuando era perseguido por seguir a Jesús. .
\bibverse{18} Mis hermanos y hermanas, que la gracia de nuestro Señor
Jesucristo esté con el espíritu de todos ustedes. Amén.
