\hypertarget{section}{%
\section{1}\label{section}}

\bibverse{1} Esta carta es enviada por Pablo, prisionero de Jesucristo,
y de nuestro hermano Timoteo, a Filemón, nuestro buen amigo y compañero
de trabajo; \bibverse{2} a nuestra hermana Apia, a Arquipo, quien lucha
junto con nosotros, y a nuestra iglesia que está en tu casa.
\bibverse{3} Recibe gracia y paz de parte de Dios nuestro Padre y del
Señor Jesucristo.

\bibverse{4} Siempre le doy gracias a Dios por ti, al recordarte en mis
oraciones, \bibverse{5} pues escucho sobre tu fe en el Señor Jesús y tu
amor por todos los creyentes. \bibverse{6} Oro para que esa generosidad
que caracteriza tu fe en Dios puedas ponerla en acción al reconocer las
cosas buenas de las que participamos en Cristo. \bibverse{7} Tu amor, mi
querido hermano, me ha causado mucha felicidad y ánimo. ¡Has reanimado
los corazones de nosotros, los que somos creyentes!

\bibverse{8} Por eso, aunque soy suficientemente valiente en Cristo para
darte orden de hacer tu trabajo, \bibverse{9} prefiero pedirte este
favor en nombre del amor. El viejo Pablo, ahora también prisionero de
Cristo Jesús, \bibverse{10} te ruega en nombre de Onésimo, que ha venido
a ser como mi hijo adoptivo durante mi encarcelamiento. \bibverse{11} En
el pasado él no fue útil para ti, ¡pero ahora es útil tanto para ti como
para mí! \bibverse{12} Lo envío, pues, con mis más sinceros
deseos.\footnote{\textbf{1:12} 12. Literalmente ``con aprecio de
  corazón.''} \bibverse{13} Habría preferido que se quedara aquí conmigo
para que me fuera de ayuda como me habrías ayudado tú mientras estoy
encadenado por predicar la buena noticia. \bibverse{14} Pero decidí no
hacer nada sin tu permiso. No quería obligarte a hacer el bien, sino que
lo hicieras de buen agrado. \bibverse{15} ¡Quizás lo perdiste por un
tiempo para ahora tenerlo para siempre! \bibverse{16} Ya no es más un
siervo, porque es más que un siervo. Es un hermano especialmente amado,
principalmente para mí, e incluso más para ustedes, tanto como persona y
también como hermano creyente en el Señor.\footnote{\textbf{1:16} 16.
  Literalmente, ``en la carne y en el Señor.''}

\bibverse{17} Así que si me consideras un compañero de trabajo en el
Señor,\footnote{\textbf{1:17} 17. ``Un colega que trabaja contigo por el
  Señor.'' La palabra griega es ``socio,'' pero requiere explicación
  debido a los usos modernos de esta palabra.} recíbelo como si me
recibieras a mí. \bibverse{18} Y si ha cometido algún error, o te debe
algo, cárgalo a mi cuenta. \bibverse{19} Yo, Pablo, escribo esto con mi
propia mano: Te pagaré. Sin duda no diré lo que me debes, ¡incluyendo tu
propia vida! \bibverse{20} Sí, hermano, espero este favor de tu parte en
el Señor; por favor, dame esa alegría en Cristo.

\bibverse{21} Te escribo sobre esto porque estoy convencido de que harás
lo que te estoy pidiendo. ¡E incluso sé que harás más que eso!
\bibverse{22} Mientras tanto, por favor, prepara una habitación para mí,
pues espero poder regresar a verte pronto, como respuesta a tus
oraciones. \bibverse{23} Epafras, que está aquí conmigo en prisión, te
envía su saludo, \bibverse{24} así como mis colaboradores Marcos,
Aristarco, Demas, y Lucas. \bibverse{25} Que la gracia de nuestro Señor
Jesucristo esté con todos ustedes.
