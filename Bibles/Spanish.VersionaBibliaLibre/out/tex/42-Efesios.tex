\hypertarget{section}{%
\section{1}\label{section}}

\bibverse{1} Esta carta proviene de Pablo, un apóstol de Cristo Jesús
conforme a la voluntad de Dios, para los cristianos\footnote{\textbf{1:1}
  Literalmente, ``santos.''} en Éfeso y para los que creen en Cristo
Jesús. \bibverse{2} Gracia y paz a ustedes de Dios nuestro Padre y del
Señor Jesucristo.

\bibverse{3} Alabado sea Dios, el Padre de nuestro Señor Jesucristo,
quien nos ha bendecido en Cristo con todo lo que es espiritualmente
bueno en el mundo celestial, \bibverse{4} tal como nos eligió para estar
en él\footnote{\textbf{1:4} O, ``en unión con él.''} antes del principio
de este mundo, a fin de que en amor pudiéramos ser santos y sin falta
delante de él. \bibverse{5} Él decidió de antemano adoptarnos como sus
hijos, obrando mediante Jesucristo para traernos a hacia él. Se
complació en hacerlo porque así él lo quiso. \bibverse{6} Por eso lo
alabamos, por su gloriosa gracia que con tanta bondad nos dio en su Hijo
amado.\footnote{\textbf{1:6} Aquí se asume que es el Hijo. El griego
  dice ``amado.''} \bibverse{7} A través de él obtenemos la salvación
mediante su sangre, el perdón de nuestros pecados, como resultado de su
preciosa gracia \bibverse{8} que con tanta generosidad nos dio, junto
con toda la sabiduría y conocimiento.

\bibverse{9} Él nos reveló su voluntad que hasta ese momento estaba
oculta, y por medio de la cual se complació en llevar a cabo su plan
\bibverse{10} en el momento apropiado para reunir a todos\footnote{\textbf{1:10}
  Haciendo paralelo con Colosenses 1:20.} en Cristo, tanto los que están
en el cielo, como los que están en la tierra. \bibverse{11} En Él fuimos
escogidos de antemano, según el plan de Aquél que obra todas las cosas
conforme a su voluntad, \bibverse{12} con el fin de que
nosotros,\footnote{\textbf{1:12} ``Nosotros'' queriendo decir Judíos
  Cristianos.} los primeros en guardar la esperanza en Cristo,
pudiéramos alabar su gloria. \bibverse{13} En Él ustedes\footnote{\textbf{1:13}
  ``Ustedes'' queriendo decir Gentiles Cristianos.} también han
escuchado la palabra de verdad, la buena noticia de su salvación. En Él,
puesto que creyeron en él, fueron sellados con el sello de la promesa
del Espíritu Santo, \bibverse{14} que es el anticipo de nuestra herencia
cuando Dios redima lo que ha preservado para sí mismo: nosotros, quienes
le adoraremos y le daremos gloria.

\bibverse{15} Esa es la razón, pues he escuchado de su fe en el Señor
Jesús y el amor que ustedes tienen por todos los cristianos,
\bibverse{16} por lo cual nunca dejo de dar gracias a Dios por ustedes y
recordarlos en mis oraciones. \bibverse{17} Oro para que el Dios de
nuestro Señor Jesucristo, el Padre glorioso, les conceda un espíritu de
sabiduría para que lo vean y lo conozcan como él es realmente.
\bibverse{18} Que sus mentes sean iluminadas a fin de que puedan
entender la esperanza a la cual él los ha llamado: \bibverse{19} las
gloriosas riquezas que él promete como heredad a su pueblo fiel. Oro
para que también puedan comprender el maravilloso poder de Dios
\bibverse{20} que fue demostrado al levantar a Cristo de los muertos.
Dios sentó a Cristo a su diestra en el cielo, \bibverse{21} por encima
de cualquier otro gobernante, autoridad, poder o señor, o de cualquier
líder, sin importar los títulos, y no solo en este mundo sino también en
el mundo por venir. \bibverse{22} Dios ha sujetado todas las cosas a la
autoridad de Cristo, y le ha dado la responsabilidad como cabeza sobre
todas las cosas para la iglesia, \bibverse{23} que es su cuerpo. Cristo
llena y completa a la iglesia, pues él llena y da plenitud a todas las
cosas.

\hypertarget{section-1}{%
\section{2}\label{section-1}}

\bibverse{1} En un tiempo ustedes estaban muertos en sus pecados y
maldad, \bibverse{2} viviendo según los caminos del mundo, bajo el
dominio del diablo,\footnote{\textbf{2:2} Literalmente ``el gobernante
  del poder del aire.''} cuyo espíritu trabaja en aquellos que
desobedecen a Dios. \bibverse{3} Todos una vez fuimos así, y nuestra
conducta estaba determinada por los deseos de nuestra naturaleza humana
pecaminosa y nuestros malos pensamientos. Como todos los demás, en
nuestra naturaleza éramos hijos de la ira.\footnote{\textbf{2:3} El
  idioma griego dice literalmente: ``Hijos por naturaleza de la ira.''
  Siguiendo el pensamiento anterior sobre la naturaleza humana, esto
  podría significar que nosotros tenemos ``ira por naturaleza'' o que
  somos rebeldes hacia Dios. Otra posibilidad sería ver esto como si
  nosotros fuéramos objeto de la ira divina, aunque a Dios no se le
  menciona aquí de manera específica.}

\bibverse{4} Pero Dios, en su gran misericordia, por el maravilloso amor
que tuvo por nosotros \bibverse{5} incluso cuando estábamos muertos en
nuestros pecados, nos ha resucitado junto a Cristo. ¡Creer en él los ha
salvado! \bibverse{6} Él levantó a Cristo, y en Cristo Jesús nos sentó
con él en el cielo, \bibverse{7} para demostrar por toda la eternidad el
enorme alcance de su gracia, al mostrarnos su bondad a través de Cristo
Jesús. \bibverse{8} Porque ustedes han sido salvos por gracia, por la fe
en él, y esto no por ustedes mismos, ¡es el regalo de Dios! \bibverse{9}
La salvación no depende del esfuerzo humano, así que no se
enorgullezcan. \bibverse{10} Somos el resultado de la obra de Dios,
creados en Cristo para hacer el bien que Dios ya planeó para nosotros.

\bibverse{11} Así que ustedes, que son ``extranjeros'' humanamente
hablando, llamados ``incircuncisos'' por los que son
``circuncisos''\footnote{\textbf{2:11} Es decir, los judíos
  (circuncidados) y los gentiles (incircuncisos).} (que es apenas un
procedimiento realizado por seres humanos), necesitan recordar
\bibverse{12} que una vez no tenían relación con Cristo. Ustedes estaban
excluidos como extranjeros de ser ciudadanos de Israel, extraños
respecto al pacto que Dios había prometido. No tenían esperanza y vivían
en el mundo sin Dios. \bibverse{13} Pero ahora, En Cristo Jesús, ustedes
que una vez estaban lejos, han sido acercados por la sangre de Cristo.

\bibverse{14} Cristo es nuestra paz. Por su cuerpo\footnote{\textbf{2:14}
  Por el contexto, parece que aquí Pablo se está refiriendo a la
  crucifixión de Jesús.} él convirtió dos en uno solo, y rompió el muro
de hostilidad que nos dividía, \bibverse{15} liberándonos de la ley con
sus requisitos y normas. Él lo hizo para crear en sí mismo a una nueva
persona a partir de los dos y lograr la paz, \bibverse{16} y así
reconciliarlos por completo con Dios a través de la cruz como si fueran
un solo cuerpo, habiendo destruido nuestra hostilidad unos por otros.

\bibverse{17} Él vino y compartió la buena noticia de paz con los que
estaban lejos y con los que estaban cerca, \bibverse{18} porque por él
ambos podemos tener acceso al Padre, por medio del mismo Espíritu.
\bibverse{19} Esto significa que ya ustedes no son extranjeros, sino
conciudadanos del pueblo de Dios y pertenecen a la familia de Dios
\bibverse{20} que está siendo edificada sobre el fundamento de los
apóstoles y profetas, del cual Cristo es la piedra angular.
\bibverse{21} En él toda la edificación está unida, creciendo para
formar un santo templo para el Señor. \bibverse{22} Ustedes también
están siendo edificados en él como un lugar para que habite Dios por el
Espíritu.

\hypertarget{section-2}{%
\section{3}\label{section-2}}

\bibverse{1} Es por esto que yo, Pablo, prisionero de Jesucristo por
causa de ustedes los extranjeros, \bibverse{2} (pues, asumo que ustedes
han oído que Dios me dio la responsabilidad específica de compartir la
gracia de Dios con ustedes), \bibverse{3} por lo que Dios me mostró,
aclaró el misterio que estaba oculto anteriormente. Yo les escribí
brevemente sobre esto, \bibverse{4} y cuando lean esto podrán entender
mi opinión sobre el misterio de Cristo. \bibverse{5} En las generaciones
pasadas esto no se le había explicado a nadie, pero ahora ha sido
revelado a los santos apóstoles de Dios y a los profetas por medio del
Espíritu, \bibverse{6} que los extranjeros son herederos también, parte
del mismo cuerpo, y en Cristo Jesús comparten en la promesa por medio de
la buena noticia.

\bibverse{7} Me convertí en ministro de esta buena noticia por medio del
regalo de la gracia de Dios que se me dio por su poder que obraba en mí.
\bibverse{8} Esta gracia me fue dada a mí, al menos importante de todos
los cristianos, con el fin de compartir con los extranjeros el increíble
valor de Cristo, \bibverse{9} y para ayudar a todos a ver el propósito
del misterio que desde el mismo principio estaba oculto en Dios, quien
hizo todas las cosas. \bibverse{10} El plan de Dios fue que los
distintos aspectos de su sabiduría fueran revelados por medio de la
iglesia a los gobernantes y autoridades en el cielo. \bibverse{11} Esto
fue conforme al propósito eterno de Dios que llevó a cabo en Cristo
Jesús nuestro Señor. \bibverse{12} Por él y nuestra fe en él podemos
acercarnos a Dios con total confianza y libertad. \bibverse{13} Por eso
les pido que no se desanimen por mi sufrimiento, ¡es por ustedes y
deberían apreciarlo!

\bibverse{14} Por eso me arrodillo ante el Padre \bibverse{15} de quien
todas las familias del cielo y de la tierra reciben su naturaleza y
carácter, \bibverse{16} le ruego que, de sus riquezas de gloria, los
fortalezca con poder en lo más íntimo de su ser por medio de su
Espíritu. \bibverse{17} Que Cristo viva en sus corazones a medida que
confían en él, a fin de que sembrados profundamente en amor
\bibverse{18} adquieran el poder para comprender, junto a todo el pueblo
de Dios, la amplitud, la longitud, la altura y la profundidad del amor
de Cristo. \bibverse{19} Que conozcan el amor de Cristo que sobrepasa
todo conocimiento, para que puedan ser llenos y alcancen la plenitud que
proviene de Dios.

\bibverse{20} Que por su poder que obra dentro de nosotros, Aquél que es
poderoso para hacer más de lo que le pedimos o siquiera alcanzamos a
pensar, \bibverse{21} sea él glorificado en la iglesia y en Cristo Jesús
por todas las generaciones, por siempre y para siempre. Amén.

\hypertarget{section-3}{%
\section{4}\label{section-3}}

\bibverse{1} Así que yo, ---este prisionero en el Señor---los animo a
que vivan conforme a los principios a los cuales fueron llamados.
\bibverse{2} No se enorgullezcan de ustedes mismos; sean amables y
pacientes, demostrando tolerancia unos por otros en amor. \bibverse{3}
Esfuércense por seguir siendo uno en el Espíritu mediante la paz que los
une. \bibverse{4} Pues hay un cuerpo, y un Espíritu, así como fueron
llamados a una esperanza. \bibverse{5} El Señor es uno, nuestra
confianza en él es una, y hay un solo bautismo; \bibverse{6} hay un solo
Dios y Padre de todos. Él es sobre todo, a través de todo y en todo.

\bibverse{7} A cada uno de nosotros se nos dio gracia en proporción al
generoso don de Cristo. \bibverse{8} Como dice la Escritura: ``Cuando
ascendió a las alturas llevó cautivos con él, y otorgó dones a la
humanidad.''\footnote{\textbf{4:8} Citando Salmos 68:18.} \bibverse{9}
(En cuanto a esto: dice que ascendió, pero eso indica que también
descendió primero a nuestro mundo inferior. \bibverse{10} El que
descendió es el mismo que también ascendió a lo más alto del cielo, a
fin de poder hacer que todo el universo estuviera completo).

\bibverse{11} Los dones que él dio fueron tantos que algunos pudieron
ser apóstoles, otros profetas, otros evangelistas, otros pastores y
otros maestros, \bibverse{12} con el fin de preparar al pueblo de Dios
en la obra de ayudar a otros, para ayudar al crecimiento del cuerpo de
Cristo. \bibverse{13} Así crecemos hasta llegar a ser uno en nuestra fe
y en el conocimiento del Hijo de Dios, y crecer hasta alcanzar la plena
madurez en Cristo. \bibverse{14} Ya no deberíamos ser más como niños,
sacudidos por cualquier viento de doctrina, confundidos por los engaños
humanos, y conducidos al error por personas astutas que hacen planes
engañosos; \bibverse{15} sino que hablando la verdad en amor debemos
crecer en todas las cosas en Cristo, que es nuestra cabeza.
\bibverse{16} Es por él que funciona todo el cuerpo, y cada coyuntura lo
mantiene unido, mientras que cada una de las partes cumple su debida
función, y así crece todo el cuerpo, edificándose en amor.

\bibverse{17} Así que permítanme decirles esto---de hecho, insisto en
ello en el Señor---que no deberían vivir más de manera frívola, como lo
hacen los extranjeros. \bibverse{18} Ellos, en la oscuridad de sus
mentes no entienden, y han sido separados de la vida de Dios porque no
saben nada y por su terquedad tampoco quieren saber. \bibverse{19} Y
como no les importa, se dejan llevar por la sensualidad, y
codiciosamente hacen todo tipo de cosas desagradables.

\bibverse{20} ¡Pero eso no fue lo que ustedes aprendieron acerca de
Cristo! \bibverse{21} ¿Acaso no escucharon hablar de él? ¿No se les
enseñó acerca de él? ¿No aprendieron la verdad sobre Jesús?
\bibverse{22} ¡Entonces abandonen su antigua forma de vivir, y dejen esa
vieja naturaleza que los destruye con sus deseos engañosos!
\bibverse{23} Déjense renovar mental y espiritualmente, \bibverse{24} y
vístanse de esta nueva naturaleza que Dios creó para que lleguen a ser
como él, rectos y santos en la verdad.

\bibverse{25} Rechacen las mentiras y díganse la verdad unos a otros,
porque nos pertenecemos unos a otros. \bibverse{26} No pequen por el
enojo; no dejen que anochezca estando aun enojados, \bibverse{27} y no
le den ninguna oportunidad al diablo. \bibverse{28} Los que son
ladrones, dejen de robar y trabajen productivamente y con honestidad con
sus manos, para que tengan algo que brindar a quienes lo necesitan.
\bibverse{29} No usen lenguaje sucio. Digan palabras que animen a las
personas cuando sea necesario, de tal modo que sean palabras de ayuda
para quienes los escuchan. \bibverse{30} No decepcionen al Espíritu
Santo de Dios que los señaló como pertenencia suya para el día de la
redención. \bibverse{31} Abandonen todo tipo de amargura, enojo, ira,
abuso verbal e insultos, así como toda forma de maldad. \bibverse{32}
Sean amables y compasivos unos con otros, perdonándose unos a otros, así
como Cristo los perdonó a ustedes.

\hypertarget{section-4}{%
\section{5}\label{section-4}}

\bibverse{1} Así que imiten a Dios, pues ustedes son sus hijos amados.
\bibverse{2} Vivan en amor, como Cristo los amó. Él se entregó por
nosotros, y fue un don y ofrenda de sacrificio para Dios como un perfume
con dulce aroma. \bibverse{3} Nunca debería mencionarse la inmoralidad
sexual o ningún tipo de indecencia o codicia al hablar de ustedes, pues
el pueblo de Dios no debería estar haciendo tales cosas. \bibverse{4}
Las conversaciones obscenas, las charlas necias, y los chistes con doble
sentido son totalmente inapropiados. Por el contrario, deberían dar
gracias a Dios. \bibverse{5} Ustedes saben que ciertamente ninguna
persona que cometa inmoralidad sexual, indecencia, que sea codiciosa, o
idólatra heredará cosa alguna en el reino de Cristo y de Dios.
\bibverse{6} No dejen que nadie los engañe con mentiras, porque por
tales cosas el juicio de Dios es transmitido a los hijos de la
desobediencia. \bibverse{7} Así que no participen con ellos en esto.
\bibverse{8} En un tiempo ustedes estaban en tinieblas, pero ahora
ustedes son luz en el Señor. Deben vivir como hijos de luz \bibverse{9}
(y el fruto de la luz es todo lo bueno y verdadero), \bibverse{10}
demostrando lo que el Señor realmente desea.

\bibverse{11} No tengan ningún tipo de relación con las cosas inútiles
que produce la oscuridad, más bien, expónganlas. \bibverse{12} Es
incluso vergonzoso hablar de las cosas que tales personas hacen en
secreto, \bibverse{13} pero cuando algo es expuesto por la luz, entonces
es revelado como realmente es. La luz hace visibles todas las cosas.
\bibverse{14} Por eso se dice: ``Levántense, ustedes los que duermen,
levántense de entre los muertos, y Cristo brillará sobre ustedes.''
\bibverse{15} Así que tengan cuidado en cuanto a su forma de vivir, no
con necedad, sino con sabiduría, \bibverse{16} haciendo el mejor uso
posible de las oportunidades, porque los días están llenos de maldad.
\bibverse{17} Así que no sean ignorantes y averigüen cuál es la voluntad
de Dios. \bibverse{18} No se emborrachen con vino, porque esto arruinará
sus vidas, más bien llénense del Espíritu. \bibverse{19} Compartan
juntos unos con otros por medio de salmos, himnos y cantos sagrados,
cantando y creando música para el Señor con sus corazones. \bibverse{20}
Siempre den gracias a Dios el Padre por todas las cosas en el nombre de
nuestro Señor Jesucristo.

\bibverse{21} Cada uno de ustedes debe estar dispuesto a aceptar lo que
los demás les dicen a partir de la reverencia por Cristo. \bibverse{22}
Esposas, hagan lo que sus esposos les dicen, como lo harían si se los
dijera el Señor. \bibverse{23} El esposo es cabeza de la esposa del
mismo modo que Cristo es la cabeza de la iglesia, así como su cuerpo y
salvador. \bibverse{24} Del mismo modo que la iglesia hace lo que Cristo
dice, las esposas deben hacer lo que sus esposos les dicen en todo.
\bibverse{25} Esposos, amen a sus esposas de la misma manera que Cristo
amó a la iglesia y se entregó por ella. \bibverse{26} Él la santificó,
la limpió al lavarse en el agua del mundo,\footnote{\textbf{5:26}
  Probablemente como alusión al bautismo.} \bibverse{27} así pudo
apropiarse de la iglesia, sin ningún defecto o mancha, sino santa e
irreprochable. \bibverse{28} Los esposos deben amar a sus esposas de
esta manera, así como aman sus propios cuerpos. Un hombre que ama a su
esposa se ama a sí mismo, \bibverse{29} pues nunca nadie aborrece su
propio cuerpo, sino que lo alimenta y lo cuida, así como Cristo lo hace
por la iglesia, \bibverse{30} pues nosotros somos partes de su cuerpo.
\bibverse{31} ``Es por esto que un hombre deja a su padre y a su madre,
y se une a su esposa, y los dos se unen, siendo ahora uno
solo.''\footnote{\textbf{5:31} Citando Génesis 2:24.} \bibverse{32} Esta
es una verdad profunda oculta, pero hablo de Cristo y de la iglesia.
\bibverse{33} Sin embargo, cada esposo debe amar a su propia esposa como
a sí mismo, y la esposa debe respetar a su esposo.

\hypertarget{section-5}{%
\section{6}\label{section-5}}

\bibverse{1} Hijos, hagan lo que sus padres les dicen, porque esto es lo
correcto. \bibverse{2} ``Honra a tu padre y a tu madre.'' Este es el
primer mandamiento que tiene una promesa unida: \bibverse{3} ``para que
te vaya bien y tengas larga vida en la tierra.''\footnote{\textbf{6:3}
  Citando Deuteronomio 5:16.} \bibverse{4} Padres, no enojen a sus
hijos, sino cuiden de ellos, disciplinándolos e instruyéndolos acerca de
Dios. \bibverse{5} Siervos, obedezcan a sus amos en la tierra, con el
debido respeto y admiración, haciendo las cosas con sinceridad, como si
sirvieran a Cristo. \bibverse{6} No trabajen simplemente cuando los ven
o para recibir aprobación, sino trabajen como siervos de Cristo,
haciendo con honestidad la voluntad de Dios, \bibverse{7} sirviendo con
alegría, como si lo hicieran para el Señor y no para la gente.
\bibverse{8} Ustedes saben que todo el que hace lo bueno será
recompensado por el Señor, sea siervo o libre. \bibverse{9} Amos, traten
a sus siervos del mismo modo. No los amenacen, recuerden que el Señor en
el cielo es tanto su amo como el de ustedes, y él trata a las personas
con igualdad, sin favoritismo.

\bibverse{10} Por último, manténganse firmes en el Señor, y en su poder.
\bibverse{11} Vístanse con toda la armadura de Dios para que puedan
estar firmes ante los ataques del enemigo. \bibverse{12} No estamos
peleando contra fuerzas humanas, sino contra poderes y gobernantes
sobrenaturales, contra los señores de las tinieblas de este mundo,
contra las fuerzas espirituales de maldad que están en los cielos.
\bibverse{13} Tomen las armas que Dios les da para que puedan estar
firmes en el día del mal y que sigan en pie aun después de la lucha.
\bibverse{14} Así que levántense, pónganse el cinturón de la verdad,
pónganse la coraza de justicia y rectitud, \bibverse{15} y colóquense el
calzado de la prontitud para compartir la buena noticia de paz.
\bibverse{16} Pero sobre todas las cosas, tomen el escudo de la fe en
Dios, por el cual podrán soportar todos los dardos de fuego del enemigo.
\bibverse{17} Usen el casco de la salvación, y lleven la espada del
Espíritu, que es la palabra de Dios. \bibverse{18} Siempre oren en el
Espíritu al hacer todo esto. Estén despiertos y sigan orando por todo el
pueblo de Dios. \bibverse{19} Oren por mí para decir las palabras
adecuadas, y para poder explicar con toda confianza las verdades ocultas
de la buena noticia. \bibverse{20} Soy un prisionero embajador por causa
de la buena noticia, así que les ruego que oren para que pueda hablar
sin temor, como es debido. \bibverse{21} Tíquico, nuestro buen amigo y
ministro fiel, les dará todas las noticias sobre mí y les explicará
todo, para que sepan cómo estoy. \bibverse{22} Por ello lo envío a
ustedes, para que les diga lo que nos ha sucedido y se animen.
\bibverse{23} Paz a todos los cristianos allí, de parte de Dios el Padre
y del Señor Jesucristo, con amor y fe en él. \bibverse{24} Gracia a
todos los que aman eternamente a nuestro Señor Jesús.
