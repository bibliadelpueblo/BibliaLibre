\hypertarget{section}{%
\section{1}\label{section}}

\bibverse{1} Querido Teófilo\footnote{\textbf{1:1} Quiere decir:
  ``Alguien que ama a Dios,'' ya sea una persona específica, o más de
  manera genérica. El ``libro anterior'' que se menciona es el evangelio
  de Lucas.}, en mi libro anterior escribí acerca de todo lo que Jesús
hizo y enseñó desde el comienzo \bibverse{2} hasta el día en que fue
llevado al cielo. Eso sucedió después de haberles dado instrucciones a
sus apóstoles escogidos a través del Espíritu Santo. \bibverse{3} Él se
les apareció durante cuarenta días después de la muerte que sufrió,
demostrando con evidencia convincente que estaba vivo. Se les aparecía y
les hablaba acerca del reino de Dios. \bibverse{4} Mientras aún estaba
con ellos\footnote{\textbf{1:4} O, ``mientras compartía una comida con
  ellos.''} los instruyó: ``No salgan de Jerusalén. Esperen hasta
recibir lo que el Padre prometió, tal como lo oyeron de mí. \bibverse{5}
Es cierto que Juan bautizaba con agua, pero en pocos días ustedes serán
bautizados con el Espíritu Santo.''

\bibverse{6} Así que cuando los discípulos se encontraron con Jesús, le
preguntaron: ``Señor, ¿es este el momento en que restablecerás el reino
de Israel?''

\bibverse{7} ``Ustedes no necesitan saber acerca de las fechas y los
tiempos que son establecidos por la autoridad del Padre,'' les dijo.
\bibverse{8} ``Pero recibirán poder cuando el Espíritu Santo descienda
sobre ustedes, y serán mis testigos en Jerusalén, por toda Judea y
Samaria, y hasta en los lugares más lejanos de la tierra.''

\bibverse{9} Y después que les dijo esto, fue alzado mientras ellos lo
veían y una nube lo ocultó de la vista de ellos. \bibverse{10} Y
mientras observaban el cielo con atención, de repente dos hombres
vestidos de blanco se pusieron en pie junto a ellos. \bibverse{11}
``Hombres de Galilea, ¿por qué están ahí parados mirando al cielo?''
preguntaron ellos. ``Este mismo Jesús que ha sido llevado al cielo
delante de ustedes vendrá de la misma manera en que lo vieron irse.''

\bibverse{12} Entonces los discípulos regresaron del Monte de los Olivos
hacia Jerusalén, lo que equivale al camino de un día de reposo desde
Jerusalén\footnote{\textbf{1:12} En otras palabras, una distancia
  relativamente corta.}. \bibverse{13} Cuando llegaron, subieron las
escaleras del lugar donde posaban hasta la habitación de arriba. Allí
estaba Pedro, Juan, Santiago y Andrés; Felipe y Tomás; Bartolomeo y
Mateo; Santiago el hijo de Alfeo, Simón el Celote, y Judas, el hijo de
Santiago. \bibverse{14} Todos ellos se reunieron para orar, junto con
las mujeres y María, la madre de Jesús, y sus hermanos.

\bibverse{15} Durante esta ocasión Pedro se puso en pie y se dirigió a
una multitud de aproximadamente ciento veinte creyentes que se habían
reunido.

\bibverse{16} ``Mis hermanos y hermanas,'' dijo, ``Las Escrituras,
habladas por el Espíritu Santo a través de David, tenían que cumplirse
en cuanto a Judas, quien guio a los que arrestaron a Jesús.
\bibverse{17} Él fue contado como uno de nosotros, y compartió este
ministerio.''

\bibverse{18} (Judas había comprado un campo con sus ganancias ilícitas.
Allí cayó de cabeza, y su cuerpo estalló, derramando todos sus
intestinos. \bibverse{19} Todos los que vivían en Jerusalén oyeron
acerca de esto, así que este campo fue llamado en su idioma
``Acéldama,'' que quiere decir ``Campo de Sangre.''). \bibverse{20} Tal
como está escrito en el libro de Salmos, ``Sea hecha desierta su
habitación, y no haya quien more en ella; y tome otro su
oficio.''\footnote{\textbf{1:20} Citando Salmos 69:25 y 109:8.}

\bibverse{21} ``De modo que ahora necesitamos escoger a alguien que haya
estado con nosotros durante todo el tiempo que Jesús estuvo con
nosotros, \bibverse{22} desde el tiempo cuando Juan estuvo bautizando
hasta el día en que fue llevado al cielo ante nosotros. Uno de estos
debe ser elegido para que se una a nosotros como testigo, dando fe de la
resurrección de Jesús.'' \bibverse{23} Entonces se postularon dos
nombres: José Justo, también conocido como Barsabás, y Matías.
\bibverse{24} Luego oraron juntos, diciendo: ``Señor, tú conoces los
pensamientos de cada uno\footnote{\textbf{1:24} 1:24 ``Tú conoces los
  pensamientos de cada uno''---literalmente, ``conocedor de los
  corazones.''}; por favor, muéstranos a cuál de estos dos tú has
elegido \bibverse{25} para sustituir a Judas como apóstol en este
ministerio al cual él renunció para irse a donde pertenece.''
\bibverse{26} Entonces echaron suertes, y fue elegido Matías. Y fue
contado como apóstol junto a los otros doce.

\hypertarget{section-1}{%
\section{2}\label{section-1}}

\bibverse{1} Cuando llegó el día del Pentecostés, todos estaban reunidos
en un mismo lugar. \bibverse{2} De repente se escuchó un ruido que
provenía del cielo, como un viento que aullaba y llenó toda la casa
donde estaban. \bibverse{3} Y vieron lo que parecía como diferentes
llamas con forma de lenguas que se posaron sobre cada uno de ellos.
\bibverse{4} Todos fueron llenos del Espíritu Santo y comenzaron a
hablar en diferentes lenguas a medida que el Espíritu se los permitía.

\bibverse{5} En ese momento había allí judíos devotos provenientes de
todas las naciones de la tierra, que vivían en Jerusalén. \bibverse{6}
Cuando escucharon este ruido, se reunió una gran multitud de ellos.
Estaban perplejos porque todos escuchaban hablar en su propio idioma.
\bibverse{7} Y estaban totalmente sorprendidos, diciendo: ``Miren, ¿no
son Galileos todos estos que están hablando? \bibverse{8} ¿Cómo es
posible que les escuchemos hablar en nuestros propios idiomas?''
\bibverse{9} Partos, Medos y Elamitas; gente de Mesopotamia, Judea y
Capadocia, Ponto y Asia, \bibverse{10} Frigia y Panfilia; desde Egipto y
el área de Libia alrededor de Cirene; visitantes de Roma, tanto judíos
como conversos, \bibverse{11} cretenses y árabes, les escuchamos hablar
en nuestros propios idiomas acerca de todas las cosas grandes que Dios
ha hecho.''

\bibverse{12} Y todos estaban sorprendidos y confundidos. Y unos a otros
se preguntaban ``¿Qué significa esto?'' \bibverse{13} Pero había otros
que se burlaban y decían: ``¡De seguro han estado tomando mucho vino!''

\bibverse{14} Entonces Pedro se puso en pie con los otros once
discípulos y habló en voz alta: ``¡Hermanos judíos y todos los que viven
aquí en Jerusalén: préstenme atención y les explicaré todo esto!
\bibverse{15} Estos hombres no están ebrios, como ustedes insinúan.
¡Noten que apenas son las nueve de la mañana! \bibverse{16} Lo que está
sucediendo es lo que fue predicho por el profeta Joel: \bibverse{17}
`Dios dice: En los últimos días, derramaré mi Espíritu sobre toda la
gente. Sus hijos e hijas profetizarán. Sus jóvenes tendrán visiones, y
sus ancianos tendrán sueños. \bibverse{18} En esos días derramaré mi
Espíritu sobre mis siervos ---hombres y mujeres por igual--- y
profetizarán. \bibverse{19} Y haré maravillas arriba en los cielos y
señales abajo en la tierra: sangre, fuego y nubes de humo. \bibverse{20}
El sol se oscurecerá, y la luna se pondrá roja como la sangre antes de
que llegue el día grande y glorioso del Señor. \bibverse{21} Pero todo
el que invoque el nombre del Señor será salvo.'''\footnote{\textbf{2:21}
  Citando Joel 2:28-32.}

\bibverse{22} ``Pueblo de Israel, escuchen esto: como bien lo saben,
Jesús de Nazaret fue un hombre confirmado por Dios ante ustedes por
medio de los milagros poderosos y las señales que Dios hizo por medio de
él. \bibverse{23} Dios, sabiendo de antemano lo que sucedería, siguió su
plan y resolvió entregarlo en manos de ustedes. Por mano de hombres
malvados, ustedes lo mataron, clavándolo en una cruz. \bibverse{24} Pero
Dios lo levantó nuevamente a la vida, libertándolo de la carga de la
muerte, porque la muerte no tuvo poder para tenerlo prisionero.

\bibverse{25} ``David dice de él: `Vi al Señor siempre conmigo. No seré
sacudido, porque él está aquí a mi lado. \bibverse{26} ¡Con razón mi
corazón está contento, y mi lengua grita sus alabanzas! Mi cuerpo
descansa en esperanza. \bibverse{27} Pues tú no dejarás mi alma entre
los muertos ni permitirás que tu Santo se pudra en la tumba.
\bibverse{28} Me has mostrado el camino de la vida y me llenarás con la
alegría de tu presencia\footnote{\textbf{2:28} Citando Salmos 16:8-11.}.'

\bibverse{29} ``Mis hermanos y hermanas, permítanme decirles claramente
que nuestro antepasado David murió y fue sepultado, y su tumba está aquí
con nosotros hasta el día de hoy. \bibverse{30} Pero él era un profeta,
y sabía que Dios había prometido bajo juramento poner en su trono a uno
de sus descendientes. \bibverse{31} David vio lo que sucedería y habló
sobre la resurrección de Cristo, porque Cristo no fue abandonado en su
tumba, ni sufrió descomposición.

\bibverse{32} ``Dios ha levantado a este Jesús de entre los muertos, y
todos nosotros somos testigos de eso. \bibverse{33} Ahora él ha sido
exaltado a la diestra de Dios y ha recibido del Padre al Espíritu Santo,
el cual prometió, y ha derramado lo que ustedes están viendo y oyendo.
\bibverse{34} Porque David no ascendió al cielo, pero dijo: `El Señor
dijo a mi Señor; siéntate a mi diestra, \bibverse{35} entretanto que
pongo a tus enemigos por estrado de tus pies.'''\footnote{\textbf{2:35}
  Citando Salmos 110:1. Una señal de victoria.} \bibverse{36} Ahora esté
todo Israel convencido de esto: ¡Dios ha puesto a este Jesús, a quien
ustedes mataron en una cruz, como Señor y Mesías!''\footnote{\textbf{2:36}
  Mesías (Hebreo) es equivalente a Cristo (Griego).}

\bibverse{37} Cuando la gente escuchó esto, sintieron remordimiento de
conciencia.\footnote{\textbf{2:37} Literalmente, ``se les partió el
  corazón.''} Entonces le preguntaron a Pedro y a los apóstoles:
``Hermanos, ¿qué debemos hacer?''

\bibverse{38} ``¡Arrepiéntanse!'' les dijo Pedro. ``Todos deben
bautizarse en el nombre de Jesús para perdón de sus pecados, y recibirán
el don del Espíritu Santo. \bibverse{39} Esta promesa es para ustedes,
para sus hijos, y para todos los extranjeros. Para todo aquél a quien el
Señor nuestro Dios llama.''

\bibverse{40} Luego Pedro siguió hablando, dándoles más evidencias. Les
advirtió: ``Sálvense de esta generación perversa.'' \bibverse{41}
Aquellos que aceptaban lo que él decía, eran bautizados, sumándose así
cerca de tres mil personas al grupo de creyentes en ese día.

\bibverse{42} Ellos se comprometieron a seguir lo que los apóstoles les
habían enseñado, y a la hermandad de los creyentes, ``partiendo el
pan''\footnote{\textbf{2:42} Esto probablemente hace referencia a la
  Cena del Señor, y no solo a comidas regulares, aunque también estarían
  incluidas.} y orando juntos. \bibverse{43} Todos estaban asombrados, y
a través de los apóstoles se realizaban muchos milagros y señales.
\bibverse{44} Y todos los creyentes estaban juntos y compartían todo lo
que tenían. \bibverse{45} Ellos vendían sus propiedades y pertenencias,
compartiendo las ganancias con todos, en cuanto lo necesitaban.
\bibverse{46} Día tras día siguieron reuniéndose en el templo, y comían
juntos en sus casas. Disfrutaban de las comidas con humildad y alegría.
Alababan a Dios, y todos pensaban bien de ellos. \bibverse{47} Cada día
el Señor agregaba al grupo aquellos que iban siendo salvos.

\hypertarget{section-2}{%
\section{3}\label{section-2}}

\bibverse{1} Pedro y Juan iban de camino, subiendo hacia el templo, a la
hora de la oración de la tarde, cerca de las 3 p.m. \bibverse{2} Y
habían llevado allí a un hombre que había estado paralítico desde su
nacimiento. Todos los días lo ponían allí, junto a la puerta del templo
que se llamaba ``La Hermosa,'' para que pudiera pedir limosna a las
personas que entraban al templo. \bibverse{3} Entonces este hombre vio a
Pedro y a Juan cuando iban a entrar al templo, y les pidió dinero.
\bibverse{4} Entonces Pedro lo miró fijamente, y Juan también.

``¡Míranos!'' le dijo Pedro. \bibverse{5} Entonces el hombre paralítico
puso toda su atención en ellos, esperando recibir algo. \bibverse{6}
``No tengo plata ni oro,'' le dijo Pedro, ``pero te daré lo que tengo:
En el nombre de Jesucristo de Nazaret, ¡camina!''

\bibverse{7} Entonces Pedro lo tomó por la mano derecha y lo ayudó a
levantarse. Y de inmediato sus pies y rodillas se volvieron fuertes.
\bibverse{8} Y el hombre se puso en pie de un brinco y comenzó a
caminar. Luego entró con ellos al templo, caminando y saltando, y
alabando a Dios. \bibverse{9} Todos los que estaban allí lo vieron
caminando y alabando a Dios. \bibverse{10} Entonces lo reconocieron como
el mendigo que solía sentarse junto a la puerta del templo, La Hermosa,
y estaban sorprendidos y maravillados ante lo que le había sucedido a
este hombre. \bibverse{11} Entonces él se agarró fuertemente de Pedro y
Juan mientras todos corrían por el Pórtico de Salomón\footnote{\textbf{3:11}
  Una sección del templo, ver también el versículo 5:12.}en completo
asombro por lo que había ocurrido.

\bibverse{12} Y cuando Pedro vio esta oportunidad, les dijo: ``Pueblo de
Israel, ¿por qué están sorprendidos por lo que le ha sucedido a este
hombre? ¿Por qué nos miran como si hubiéramos hecho caminar a este
hombre por nuestro propio poder o fe? \bibverse{13} El Dios de Abraham,
Isaac, y Jacob---el Dios de nuestros antepasados---ha glorificado a
Jesús, su siervo. Él fue al que ustedes traicionaron y rechazaron en
presencia de Pilato, incluso después de que Pilato había decidido
soltarlo. \bibverse{14} Ustedes rechazaron a Aquél que es Santo y Justo,
y exigieron que dejaran libre a un asesino. \bibverse{15} Ustedes
mataron al Autor de la vida, Aquél a quien Dios levantó de los muertos,
y nosotros somos testigos de esto. \bibverse{16} Por medio de la fe en
el nombre de Jesús este hombre fue sanado. Ustedes ven a este hombre
aquí, y lo conocen. Es por la fe en Jesús que este hombre ha recibido
sanidad completa aquí delante de todos ustedes.

\bibverse{17} ``Ahora sé, hermanos y hermanas, que ustedes hicieron esto
por ignorancia, así como sus dirigentes. \bibverse{18} Pero Dios cumplió
lo que había prometido: que su Mesías iba a sufrir. \bibverse{19} Ahora,
arrepiéntanse, y cambien sus caminos, para que sus pecados puedan ser
limpiados, a fin de que el Señor pueda enviarles oportunidad para
sanarse y restaurarse, \bibverse{20} y envíe a Jesús, el Mesías
designado para ustedes. \bibverse{21} Porque él debe permanecer en el
cielo hasta el momento en que todo sea restaurado, como Dios lo anunció
a través de sus santos profetas hace mucho tiempo.

\bibverse{22} ``Moisés dijo: `El Señor tu Dios levantará entre tus
hermanos a un profeta como yo. A él lo escucharás. \bibverse{23}
Cualquiera que no lo escuche será eliminado del pueblo por completo.'
\bibverse{24} A partir de Samuel, todos los profetas profetizaron acerca
de estos días. \bibverse{25} Ustedes son hijos de los profetas, y del
acuerdo\footnote{\textbf{3:25} O ``pacto.''}que Dios hizo con sus padres
cuando le dijo a Abrahán: ``Por medio de tus descendientes todas las
familias de la tierra serán benditas. \bibverse{26} Dios preparó a su
siervo y lo envió primero a ustedes, para bendecirlos al convertirlos de
sus malos caminos.''

\hypertarget{section-3}{%
\section{4}\label{section-3}}

\bibverse{1} Mientras hablaban a la gente, los sacerdotes, el capitán
del templo y los saduceos llegaron donde ellos estaban. \bibverse{2}
Estaban enojados porque ellos estaban enseñándole a la gente,
diciéndoles que por medio de Jesús hay resurrección de la muerte.
\bibverse{3} Entonces los arrestaron y los pusieron bajo custodia hasta
el día siguiente, pues ya era de noche. \bibverse{4} Pero muchos de los
que habían escuchado el mensaje lo creyeron, y el número total de
creyentes aumentó hasta cerca de cinco mil.

\bibverse{5} El día siguiente, los gobernantes, los ancianos y los
líderes religiosos se reunieron en Jerusalén. \bibverse{6} E incluyeron
al Sumo Sacerdote Anás, Caifás, Juan, Alejandro y a otros miembros de la
familia de sacerdotes. \bibverse{7} Y trajeron a Pedro y a Juan delante
de ellos, y les preguntaron: ``¿Con qué poder o autoridad han hecho
esto?''

\bibverse{8} Entonces Pedro, lleno del Espíritu Santo, les respondió.
``Gobernantes del pueblo, y ancianos: \bibverse{9} ¿Se nos está
interrogando por un bien que se le hizo a un hombre que no podía hacer
nada por sí mismo, y cómo fue sanado? \bibverse{10} Si es así, todos
ustedes deben saber, y todo el pueblo de Nazaret también, que fue en el
nombre de Jesucristo de Nazaret, al que ustedes mataron en una cruz y a
quien Dios levantó de los muertos. Es gracias a él que este hombre está
en pie delante de ustedes, completamente sanado. \bibverse{11} Él es la
piedra que ustedes los constructores rechazaron, pero ha sido puesta
como piedra angular\footnote{\textbf{4:11} Citando Salmos 118:22.}.'
\bibverse{12} No hay salvación en ningún otro; no hay otro nombre debajo
del cielo, dado a la humanidad, que pueda salvarnos.''

\bibverse{13} Cuando vieron la confianza de Pedro y Juan, y se dieron
cuenta de que eran hombres sin instrucción, hombres comunes, se
sorprendieron mucho. También reconocieron a los demás compañeros de
Jesús. \bibverse{14} Y como veían al hombre que había sido sanado justo
ahí junto a ellos, no tuvieron nada que decir en respuesta a lo que
había sucedido.

\bibverse{15} Entonces les dieron orden de esperar fuera del concilio
mientras debatían el asunto entre ellos. \bibverse{16} ``¿Qué debemos
hacer con estos hombres?'' preguntaron. ``No podemos negar que por medio
de ellos ha ocurrido un milagro importante. Todos los que viven aquí en
Jerusalén saben de ello. \bibverse{17} Pero para evitar que se difunda
mucho más entre la gente, debemos amenazarlos para que no vuelvan a
hablarle a nadie en este nombre\footnote{\textbf{4:17} Claramente se
  refiere al nombre de Jesús, pero ellos no querían ni siquiera
  mencionar el nombre verdadero\ldots{}}.''

\bibverse{18} Entonces los llamaron para que entraran nuevamente y les
dieron orden de no volver a hablar o enseñar en el nombre de Jesús.
\bibverse{19} Pero Pedro y Juan respondieron: ``Decidan ustedes si es
correcto ante los ojos de Dios obedecerlos a ustedes antes que a él.
\bibverse{20} ¡No podemos dejar de hablar sobre lo que hemos visto y
oído!''

\bibverse{21} Después de proferir más amenazas contra ellos, los dejaron
ir. No pudieron resolver cómo podían castigarlos porque todos
glorificaban a Dios por lo que había ocurrido. \bibverse{22} Porque el
hombre que había recibido este milagro tenía más de cuarenta años de
edad. \bibverse{23} Después de que los discípulos fueron liberados,
fueron donde estaban otros creyentes y les contaron todo lo que los
jefes de los sacerdotes y los ancianos les habían dicho. \bibverse{24}
Cuando estos oyeron lo que había sucedido, oraron juntos a Dios:

``Señor, tú hiciste el cielo, la tierra y el mar, y todo lo que hay en
ellos. \bibverse{25} Tú hablaste por medio del Espíritu Santo a través
de David, nuestro padre y tu siervo, diciendo: `¿Por qué se enojaron los
pueblos de otras naciones? ¿Por qué conspiran insensatamente contra mí?
\bibverse{26} Los reyes de la tierra se prepararon para la
guerra\footnote{\textbf{4:26} ``Para la guerra,'' implícito.}; los
gobernantes se unieron contra el Señor y contra su Escogido.'\footnote{\textbf{4:26}
  Literalmente, ``Ungido.'' La cita proviene de Salmos 2:1, 2.}

\bibverse{27} ``¡Ahora esto en verdad ha sucedido aquí, en esta misma
ciudad! Tanto Herodes como Poncio Pilato, junto con los extranjeros y el
pueblo de Israel, unidos todos contra el Santo, tu santo siervo Jesús, a
quien tú ungiste como Mesías. \bibverse{28} Ellos hicieron todo lo que
tú ya habías decidido porque tú tuviste el poder y la voluntad para
hacerlo.

\bibverse{29} ``Ahora Señor: ¡mira todas sus amenazas contra nosotros!
Ayuda a tus siervos a predicar tu palabra con valor. \bibverse{30} Y que
al ejercer tu poder para sanar, las señales y milagros sean hechos en el
nombre de tu santo siervo Jesús.''

\bibverse{31} Cuando terminaron de orar, la edificación donde estaban
reunidos tembló. Y todos ellos fueron llenos del Espíritu Santo, y
predicaban con valor la palabra de Dios. \bibverse{32} Todos los
creyentes tenían un mismo pensar y un mismo sentir. Ninguno de ellos
consideraba nada como suyo sino que compartían todas las cosas unos con
otros. \bibverse{33} Los apóstoles daban su testimonio respecto a la
resurrección del Señor Jesús con gran poder, y Dios los bendecía a todos
en gran manera. \bibverse{34} Y ninguno de ellos necesitaba nada porque
los que tenían tierras o propiedades las vendieron. \bibverse{35}
Entonces tomaron las ganancias y las llevaron a los apóstoles para
compartirlas con los que tenían necesidad. \bibverse{36} José, al que
los apóstoles llamaban Bernabé (que quiere decir ``hijo de la
consolación''), era un Levita, nativo de Chipre. \bibverse{37} Este
vendió un campo que era suyo. Luego trajo el dinero y lo presentó a los
apóstoles.

\hypertarget{section-4}{%
\section{5}\label{section-4}}

\bibverse{1} Había un hombre llamado Ananías, que vendió una propiedad
junto con su esposa, Safira. \bibverse{2} Él guardó para sí parte del
dinero que recibieron, y llevó el resto a los apóstoles. Y su esposa
sabía lo que él estaba haciendo.

\bibverse{3} Entonces Pedro le preguntó: ``Ananías, ¿por qué Satanás ha
entrado a tu corazón para mentir al Espíritu Santo y reservarte parte
del dinero de la tierra que vendiste? \bibverse{4} Mientras tuviste la
tierra, ¿no te pertenecía? Y después que la vendiste ¿no tenías aun el
control sobre lo que hacías con el dinero? ¿Por qué decidiste hacer
esto? ¡No le has mentido a los hombres sino a Dios!''

\bibverse{5} Y al oír estas palabras, Ananías cayó al suelo y murió. Y
todos los que oyeron lo que había sucedido estaban horrorizados.
\bibverse{6} Algunos de los jóvenes se levantaron y lo envolvieron en un
sudario. Luego lo sacaron de ahí y lo enterraron.

\bibverse{7} Cerca de tres horas después llegó su esposa, sin saber lo
que había sucedido. \bibverse{8} Pedro le preguntó: ``Dime, ¿vendiste la
tierra por este precio?''

``Sí, ese fue el precio,'' respondió ella.

\bibverse{9} Entonces Pedro le dijo: ``¿Cómo pudieron ponerse de acuerdo
para engañar\footnote{\textbf{5:9} Literalmente, ``tentar.''} al
Espíritu del Señor? Mira, los que sepultaron a tu esposo acaban de
regresar, y te llevarán a ti también.''

\bibverse{10} Y de inmediato ella cayó al suelo y murió a los pies de
Pedro. Entonces los jóvenes entraron nuevamente y la encontraron muerta,
así que la sacaron de allí y la sepultaron junto a su esposo.
\bibverse{11} Y se difundió un gran temor en toda la iglesia, así como
entre todos aquellos que oían lo que había sucedido.

\bibverse{12} Y se llevaban a cabo muchas señales milagrosas entre el
pueblo a través de los apóstoles. Y todos los creyentes solían reunirse
en el pórtico de Salomón\footnote{\textbf{5:12} Ver nota al pie para el
  versículo 3:11.}. \bibverse{13} Ningún otro se atrevía a unirse a
ellos aunque eran respetados en gran manera. \bibverse{14} Sin embargo,
muchos hombres y mujeres comenzaron a creer en el Señor. \bibverse{15}
Como resultado de ello, la gente traía a los enfermos a las calles y los
acostaban allí en sus camas y alfombrillas para que la sombra de Pedro
cayera sobre ellos al pasar por ahí\footnote{\textbf{5:15} Con la idea
  de que incluso el toque de la sombra de Pedro podía sanar.}.
\bibverse{16} Y venían multitudes de los pueblos de Jerusalén, trayendo
a sus enfermos y endemoniados. Y todos eran sanados.

\bibverse{17} No obstante, el sumo sacerdote y los que estaban con él
(que eran Saduceos) estaban muy celosos y decidieron intervenir.
\bibverse{18} Entonces arrestaron a los apóstoles y los metieron a la
cárcel pública. \bibverse{19} Pero durante la noche, un ángel del Señor
abrió las puertas de la prisión y los hizo salir. \bibverse{20} ``¡Vayan
al templo y cuenten a la gente todas las cosas acerca de este nuevo
estilo de vida!'' les dijo. \bibverse{21} Entonces ellos hicieron como
el ángel les dijo y fueron al templo, cerca del amanecer, y comenzaron a
enseñar.

Entonces el sumo sacerdote y sus seguidores convocaron una reunión de
concilio con todos los líderes de Israel. Y mandaron a buscar a los
apóstoles a la prisión. \bibverse{22} Pero cuando los oficiales fueron a
la prisión, no pudieron encontrarlos, así que regresaron y dijeron al
concilio: \bibverse{23} ``Encontramos la prisión cerrada con llave y con
guardias en sus puertas. Pero cuando les pedimos que nos abrieran, no
encontramos a nadie adentro.''

\bibverse{24} Así que cuando el capitán de la guardia del templo y los
jefes de los sacerdotes oyeron esto, quedaron totalmente desconcertados,
y se preguntaban qué estaba sucediendo. \bibverse{25} Entonces alguien
entró y dijo: ``¡Miren, los hombres que ustedes metieron en la prisión
están ahí en el templo enseñándole a la gente!''

\bibverse{26} Entonces el capitán fue con sus guardias y los trajo
adentro, pero no a la fuerza, porque temían que la gente los apedreara.
\bibverse{27} Los apóstoles fueron llevados adentro y los hicieron
permanecer en pie frente al concilio.

\bibverse{28} ``¿Acaso no les dimos orden de no enseñar en este
nombre?'' preguntó el Sumo Sacerdote, con tono exigente. ``¡Miren, han
saturado a toda Jerusalén con su enseñanza, y ahora ustedes tratan de
culparnos por la muerte de él!''

\bibverse{29} Pero Pedro y los apóstoles respondieron: necesitamos
obedecer a Dios antes que a los hombres. \bibverse{30} El Dios de
nuestros antepasados levantó a Jesús de los muertos, al que ustedes
mataron, colgándolo en una cruz. \bibverse{31} Dios lo exaltó a una
posición de honor, a su diestra, como Príncipe y Salvador, como una
forma de traer arrepentimiento a Israel, y para perdón de pecados.
\bibverse{32} Nosotros somos testigos de lo que sucedió, y del mismo
modo lo es el Espíritu Santo, a quien Dios ha dado a aquellos que le
obedecen.''

\bibverse{33} Y cuando el concilio escuchó esto, se pusieron furiosos y
querían matarlos. \bibverse{34} Pero entonces uno de los miembros del
concilio se levantó para hablar. Era Gamaliel, un fariseo y doctor de la
ley que era respetado por todos. Él ordenó que los apóstoles salieran
por un momento.

\bibverse{35} Entonces Gamaliel se dirigió al concilio: ``Líderes de
Israel, tengan cuidado con lo que planean hacerles a estos hombres.
\bibverse{36} Hace un tiempo Teudas quiso hacerse famoso, y cerca de
cuatrocientos hombres se le unieron. Fue asesinado y todos los que lo
seguían quedaron dispersos y no lograron nada. \bibverse{37} Luego,
después de él, Judas de Galilea vino también durante el tiempo del
censo, y logró atraer algunos seguidores. Él también murió, y aquellos
que lo escuchaban quedaron dispersos. \bibverse{38} Del mismo modo, en
este caso, yo recomiendo que dejen en paz a estos hombres, y que los
dejen ir. Si lo que ellos están planeando, o lo que están haciendo viene
de sus propias ideas humanas, entonces lograrán derrota. \bibverse{39}
Pero si viene de Dios, ni siquiera ustedes podrán derrotarlos. ¡Incluso
podrían terminar ustedes mismos peleando contra Dios!''

\bibverse{40} Entonces ellos aceptaron lo que él dijo. Así que llamaron
a los apóstoles para que entraran nuevamente, los mandaron a azotar, y
les ordenaron que no dijeran nada en el nombre de Jesús. Luego los
dejaron ir. \bibverse{41} Los apóstoles salieron del concilio, felices
por ser considerados dignos de padecer afrenta por causa del nombre de
Jesús. \bibverse{42} Y cada día seguían enseñando y proclamando a Jesús
como el Mesías, en el templo y de casa en casa.

\hypertarget{section-5}{%
\section{6}\label{section-5}}

\bibverse{1} Durante este tiempo, cuando el número de creyentes crecía
rápidamente, los creyentes que hablaban en idioma griego comenzaron a
discutir con los creyentes que hablaban en idioma Arameo\footnote{\textbf{6:1}
  Literalmente ``Helenistas'' y ``Hebreos.''}. Ellos se quejaban de que
sus viudas estaban siendo discriminadas en cuanto a la distribución
diaria de alimento.

\bibverse{2} Entonces los doce apóstoles convocaron una reunión de todos
los creyentes y les dijeron: ``No es apropiado que nosotros dejemos de
predicar la palabra de Dios por servir las mesas. \bibverse{3} Hermanos,
elijan entre ustedes a siete hombres fieles, llenos del Espíritu y de
sabiduría. Nosotros les entregaremos esta responsabilidad a ellos.
\bibverse{4} Nosotros mismos dedicaremos toda nuestra atención a la
oración y al ministerio de la predicación de la palabra.''

\bibverse{5} Todos estuvieron contentos con este acuerdo, y eligieron a
Esteban, (un hombre lleno de fe en Dios y del Espíritu Santo), Felipe,
Prócoro, Nicanor, Timón, Parmenas, and Nicolás, (quien era originalmente
un judío de Antioquía que se había convertido). \bibverse{6} Estos
hombres fueron presentados a los apóstoles, quienes oraron y pusieron
sus manos sobre ellos para bendecirlos. \bibverse{7} La palabra de Dios
seguía siendo esparcida, y el número de discípulos en Jerusalén aumentó
grandemente, incluyendo a un gran número de sacerdotes que se
comprometieron a creer en Jesús.

\bibverse{8} Esteban, lleno de gracia y del poder de Dios, realizaba
milagros maravillosos entre el pueblo. \bibverse{9} Pero algunos
comenzaron a discutir con él. Eran de la sinagoga llamada ``de los
libertos,''\footnote{\textbf{6:9} Se cree que era una sinagoga
  conformada por personas que habían estado antes bajo esclavitud.} y
también había algunos de Cirene, Alejandría y gente de Cilicia y de Asia
menor. \bibverse{10} Pero estas personas no podían enfrentarse a la
sabiduría de Esteban ni al Espíritu con el que hablaba. \bibverse{11}
Así que sobornaron a algunos hombres para que dijeran: ``¡Hemos oído que
este hombre dice blasfemias contra Moisés, y también contra Dios!''

\bibverse{12} Y estos hombres incitaron al pueblo, y junto con los
ancianos y los maestros de la ley, fueron a arrestarlo. Luego lo
llevaron ante el concilio, \bibverse{13} y llamaron falsos testigos para
que testificaran en su contra, diciendo: ``este hombre siempre está
difamando el santo templo\footnote{\textbf{6:13} Literalmente, ``este
  lugar santo,'' también en el versículo 6:14.}. \bibverse{14} Lo hemos
oído decir que este Jesús de Nazaret destruirá el templo y cambiará las
leyes\footnote{\textbf{6:14} También se traduce como ``costumbres''; sin
  embargo, en este contexto tiene que ver mucho más con los requisitos
  legales y ceremoniales.}que recibimos de Moisés.''

\bibverse{15} Y todos los que estaban sentados en el concilio miraban
atentamente a Esteban, y su rostro brillaba como el rostro de un ángel.

\hypertarget{section-6}{%
\section{7}\label{section-6}}

\bibverse{1} ``¿Son ciertas estas acusaciones?'' preguntó el sumo
sacerdote.

\bibverse{2} ``¡Hermanos y padres, escúchenme!'' respondió Esteban.
``Dios apareció en su gloria a nuestro padre Abrahán, cuando vivía en
Mesopotamia, antes de que se mudara a Harán.

\bibverse{3} ``Dios le dijo: `Deja tu tierra y tu parentela, y vete a la
tierra que yo te mostraré.' \bibverse{4} Y Abrahán se marchó de esa
tierra de los Caldeos y vivió en Harán. Después de la muerte de su
padre, Dios lo envió aquí a este país donde ahora viven ustedes.
\bibverse{5} Dios no le dio a Abrahán una herencia aquí, ni siquiera un
metro cuadrado. Pero Dios le prometió a Abrahán que le daría a él y a
sus descendientes la posesión de la tierra, aunque no tenía hijos.
\bibverse{6} También Dios le dijo que sus descendientes vivirían en un
país extranjero y que allí serían tomados como esclavos y maltratados
durante cuatrocientos años. \bibverse{7} Luego Dios dijo: `Yo castigaré
a la nación que los tome como esclavos. Y al final saldrán de allí y
vendrán aquí a adorarme.' \bibverse{8} Dios también le dio a Abrahán el
pacto de la circuncisión\footnote{\textbf{7:8} O ``acuerdo.''}, y por
eso, cuando nació Isaac, Abrahán lo circuncidó al octavo día. Isaac fue
el padre de Jacob, y Jacob el padre de los doce patriarcas.

\bibverse{9} ``Los patriarcas, quienes estaban celosos de José, lo
vendieron como esclavo en Egipto. Pero Dios estaba con él, \bibverse{10}
y lo rescató de todos sus problemas. Le dio sabiduría y lo ayudó a ganar
el favor del Faraón, quien lo puso como gobernador sobre Egipto y sobre
la casa real.

\bibverse{11} ``Entonces hubo una gran hambruna en todo Egipto y Canaán,
causando terrible miseria y nuestros padres no tenían alimento.
\bibverse{12} Cuando Jacob escuchó que había grano en Egipto, envió a
nuestros antepasados a hacer una primera visita. \bibverse{13} Pero
durante su segunda visita, José les reveló a sus hermanos quién era, y
el Faraón descubrió el origen familiar de José. \bibverse{14} Entonces
José envió a buscar a su padre y a todos sus parientes: Setenta y cinco,
en total. \bibverse{15} Así que Jacob viajó hacia Egipto, y murió allí,
como también nuestros antepasados. \bibverse{16} Sus cuerpos fueron
traídos de regreso a Siquem y fueron puestos en la tumba que Abrahán
había comprado con plata de los hijos de Jamor, en Siquem.

\bibverse{17} ``Cuando se acercaba el tiempo para el cumplimiento de la
promesa que Dios le había hecho a Abrahán, el número de los habitantes
de Egipto aumentó. \bibverse{18} Y subió un nuevo rey al trono, que no
conocía de José. \bibverse{19} Este rey se aprovechó de nuestro pueblo y
trató mal a nuestros antepasados, obligándolos a abandonar a sus bebés
para que murieran. \bibverse{20} Fue en este tiempo cuando nació Moisés.
Era un niño hermoso, y durante tres meses recibió cuidado en la casa de
su padre. \bibverse{21} Cuando llegó el momento de ser abandonado, la
hija del Faraón lo rescató y cuidó de él como su propio hijo.

\bibverse{22} ``Moisés recibió instrucción en todas las áreas del
conocimiento conforme a los Egipcios, y se convirtió en un gran orador y
líder. \bibverse{23} Sin embargo, cuando tenía cuarenta años de edad,
decidió visitar a sus parientes, los israelitas. \bibverse{24} Y vio a
uno de ellos que era maltratado, por lo cual intervino para defenderlo.
Entonces, tomó venganza en favor del hombre y mató al egipcio.
\bibverse{25} Moisés pensaba que sus hermanos, los israelitas, verían
que Dios los estaba rescatando a través de él, pero no lo vieron así.
\bibverse{26} Al día siguiente, cuando llegó, dos israelitas estaban
peleando. Entonces trató de hacerlos reconciliar para que dejaran de
pelear. `¡Señores! ¡Ustedes son hermanos!' les dijo. '¿Por qué se atacan
el uno al otro?''

\bibverse{27} ``Pero el hombre que había comenzado la pelea empujó a
Moisés. ``¿Quién te puso como guardián de nosotros? ¿Acaso ahora eres
nuestro juez?'' le preguntó. \bibverse{28} ``¿Vas a matarme como mataste
ayer al egipcio?'' \bibverse{29} Y cuando escuchó esto, Moisés huyó.
Entonces se fue a vivir como exiliado en la tierra de Madián, donde tuvo
dos hijos.

\bibverse{30} ``Cuarenta años más tarde, en el desierto del Monte Sinaí,
un ángel se le apareció en las llamas de una zarza que ardía.
\bibverse{31} Y cuando Moisés vio esto, se sorprendió, y se aproximó
para ver más de cerca. Entonces la voz del Señor le habló: \bibverse{32}
`Yo soy el Dios de tus padres, el Dios de Abrahán, Isaac y Jacob.'
Entonces Moisés tembló de temor y no se atrevía a levantar la vista.
\bibverse{33} El Señor le dijo: `Quita tus sandalias, porque el lugar
donde estás es santo. \bibverse{34} Yo he visto el sufrimiento de mi
pueblo en Egipto, y he oído sus clamores. He descendido para
rescatarlos. Ahora ven, porque voy a enviarte a Egipto.'

\bibverse{35} ``Este fue el mismo Moisés que el pueblo había rechazado
cuando dijeron: `¿Quién te puso como gobernante y juez sobre nosotros?'
Dios lo envió para que fuera tanto gobernante como libertador, por medio
del ángel que se le apareció en la zarza. \bibverse{36} Entonces Moisés
los sacó después de realizar señales milagrosas en Egipto, en el Mar
Rojo, y siguió haciéndolo en el desierto durante cuarenta años.
\bibverse{37} Este es el mismo Moisés que le prometió a los israelitas:
`Dios les enviará un profeta como yo proveniente de su propio pueblo.'
\bibverse{38} Y Moisés estaba con el pueblo de Dios reunido cuando el
ángel le habló en el Monte Sinaí, y ahí junto con nuestros antepasados
recibió la palabra viva de Dios para que nos la diera a nosotros.
\bibverse{39} Él fue al que nuestros padres no escucharon. Ellos lo
rechazaron y decidieron regresar a Egipto. \bibverse{40} Y le dijeron a
Aarón: ``Elabora dioses para que nos guíen, porque no sabemos qué ha
ocurrido con Moisés, el que nos sacó de la tierra de Egipto.'
\bibverse{41} Entonces elaboraron un ídolo en forma de becerro, le
presentaban sacrificios, y celebraban lo que ellos mismos habían hecho.

\bibverse{42} ``Así que Dios desistió de ellos. Y los dejó adorar las
estrellas del cielo. Esto es lo que escribieron los profetas: `¿Acaso
ustedes los israelitas me dieron ofrendas o hicieron sacrificios para mi
durante los cuarenta años en el desierto? \bibverse{43} No, ustedes se
llevaron el tabernáculo del dios Moloc y la imagen de la estrella del
dios Refán, imágenes que ustedes hicieron para adorarlas. Por lo tanto
yo los enviaré a exilio hasta más allá de Babilonia.'

\bibverse{44} ``Nuestros antepasados tenían el Tabernáculo del
testimonio\footnote{\textbf{7:44} Quiere decir que este tabernáculo
  transmitía el mensaje de Dios y evidenciaba su presencia.} en el
desierto. Dios le había dicho a Moisés cómo tenía que hacerlo, siguiendo
el modelo que había visto. \bibverse{45} Más tarde, nuestros antepasados
lo llevaron junto con Josué para asentarse en la tierra tomada de las
naciones que el Señor expulsó delante de ellos. Y permaneció ahí hasta
el tiempo de David. \bibverse{46} David halló el favor de Dios y pidió
hacer una morada permanente para el Dios de Jacob. \bibverse{47} Pero
fue Salomón el que construyó un templo\footnote{\textbf{7:47}
  Literalmente, ``casa.''} para él. \bibverse{48} Por supuesto que el
Todopoderoso no vive en los templos que jnosotros hacemos. Como dijo el
profeta: \bibverse{49} `El cielo es mi trono, y la tierra es el lugar
donde pongo mis pies. ¿Qué tipo de morada pueden ustedes construir para
mí?' pregunta el Señor. `¿Qué cama podrían hacerme ustedes para
descansar? \bibverse{50} ¿Acaso no lo hice yo todo?'

\bibverse{51} ``¡Pueblo arrogante y terco! ¡Nunca escuchan!\footnote{\textbf{7:51}
  Literalmente, ``incircuncisos de oídos y corazón.''} ¡Ustedes siempre
pelean contra el Espíritu Santo! ¡Actúan como lo hacían sus padres!
\bibverse{52} ¿Hubo acaso algún profeta que sus padres no persiguieran?
Mataron a todos los que profetizaban sobre la venida de Aquél que es
verdaderamente bueno y recto. Él es Aquél a quien ustedes traicionaron y
asesinaron. \bibverse{53} Ustedes, los que recibieron la ley por medio
de los ángeles, pero se negaron a guardarla.''

\bibverse{54} Cuando oyeron esto, los miembros del concilio se
enfurecieron, y le hacían gruñidos, crujiendo sus dientes. \bibverse{55}
Pero Esteban, lleno del Espíritu Santo, elevó su mirada al cielo y vio
la gloria de Dios, y a Jesús a la diestra de Dios. \bibverse{56}
``Miren,'' dijo él, ``Veo el cielo abierto, y al Hijo del Hombre a la
diestra de Dios.''

\bibverse{57} Pero ellos taparon sus oídos con sus manos y gritaban tan
fuerte como podían. Todos se apresuraron juntos hacia él, \bibverse{58}
lo llevaron a rastras fuera de la ciudad, y comenzaron a apedrearlo. Sus
acusadores pusieron sus abrigos junto a un joven llamado Saulo.
\bibverse{59} Y mientras lo seguían apedreando, Esteban oró: ``Señor
Jesús, recibe mi espíritu.'' \bibverse{60} Entonces se arrodilló,
clamando: ``¡Señor, por favor no les tengas en cuenta este pecado!'' Y
después de decir esto, murió\footnote{\textbf{7:60} Literalmente, ``cayó
  dormido.'' En el Nuevo Testamento a menudo se habla de la muerte en
  términos de un sueño.}.

\hypertarget{section-7}{%
\section{8}\label{section-7}}

\bibverse{1} Saulo estaba de acuerdo con que era necesario matar a
Esteban. Ese mismo día se inició una terrible persecución contra la
iglesia en Jerusalén, y todos, excepto los apóstoles, se dispersaron por
toda Judea y Samaria. \bibverse{2} (Algunos seguidores fieles de Dios
sepultaron a Esteban, con gran lamento). \bibverse{3} Pero Saulo comenzó
a destruir a la iglesia, yendo de casa en casa, sacando a hombres y
mujeres de ellas y arrastrándolos hasta la prisión.

\bibverse{4} Los que se habían dispersado predicaban la palabra
dondequiera que iban. \bibverse{5} Felipe fue a la ciudad de Samaria, y
les habló acerca del Mesías. \bibverse{6} Cuando las multitudes oyeron
lo que Felipe decía y vieron los milagros que hacía, prestaron atención
a lo que les estaba diciendo. \bibverse{7} Y muchos fueron liberados de
posesión de espíritus malignos que gritaban al salir, y muchos que
estaban cojos o discapacitados fueron sanados. \bibverse{8} La gente que
vivía en la ciudad estaba feliz en gran manera.

\bibverse{9} Había, pues, un hombre llamado Simón, que vivía en la
ciudad donde se solía practicar la hechicería. Él afirmaba ser muy
importante, y había asombrado al pueblo de Samaria, \bibverse{10} de
modo que todos le prestaban atención. Desde la persona más pequeña hasta
la más grande en la sociedad decían: ``Este hombre es `El Gran Poder de
Dios.'\,'' \bibverse{11} Y estaban impresionados de él porque los había
asombrado con su magia por mucho tiempo.

\bibverse{12} Pero cuando creyeron en lo que Felipe les dijo acerca de
la buena nueva sobre el reino de Dios y el nombre de Jesucristo, hombres
y mujeres se bautizaron. \bibverse{13} Y Simón también creyó y fue
bautizado. Y acompañó a Felipe, sorprendido por las señales milagrosas y
las maravillas que veía.

\bibverse{14} Cuando los apóstoles estuvieron de regreso en Jerusalén y
oyeron que la gente de Samaria había aceptado la palabra de Dios,
enviaron a Pedro y a Juan a visitarlos. \bibverse{15} Y cuando llegaron,
oraron por los conversos de Samaria para que recibieran el Espíritu
Santo. \bibverse{16} Este no había sido derramado sobre ninguno de estos
conversos aun, pues solamente habían sido bautizados en el nombre del
Señor Jesús. \bibverse{17} Así que los apóstoles pusieron sus manos
sobre ellos, y recibieron el Espíritu Santo.

\bibverse{18} Cuando Simón vio que el Espíritu Santo era recibido por
las personas cuando los apóstoles colocaban sus manos sobre ellas, les
ofreció dinero. \bibverse{19} ``Dénme este poder también,'' les pidió,
``para que cualquiera sobre el cual yo coloque mis manos, reciba el
Espíritu Santo.''

\bibverse{20} ``Ojalá tu dinero sea destruido contigo, por pensar que el
don de Dios puede comprarse'' respondió Pedro. \bibverse{21} Tú no eres
parte de esto. No tienes parte en esta obra, porque ante los ojos de
Dios tu actitud está completamente equivocada. \bibverse{22}
¡Arrepiéntete de tu mal camino! Ora al Señor y pídele perdón por pensar
de esta manera. \bibverse{23} Puedo ver que estás lleno de una amarga
envidia, y estás encadenado por tu propio pecado.''

\bibverse{24} ``¡Por favor, ora por mí para que no me ocurra nada de lo
que has dicho!'' respondió Simón.

\bibverse{25} Después de haber dado su testimonio y de haber predicado
la palabra de Dios, regresaron a Jerusalén, compartiendo la buena nueva
en muchas aldeas de Samaria a lo largo del camino.

\bibverse{26} Y un ángel del Señor le dijo a Felipe: ``Alístense y vayan
al sur, al camino desierto que lleva de Jerusalén a Gaza.''
\bibverse{27} Entonces Felipe emprendió el viaje y se encontró con un
hombre etíope, un eunuco que tenía una posición importante en el
servicio de Candace\footnote{\textbf{8:27} Candace no es el nombre
  personal de la reina, sino su título, como Faraón.''}, reina de
Etiopía. Este eunuco era el tesorero jefe. Había ido a Jerusalén para
adorar, \bibverse{28} y venía de regreso de su viaje, sentado en su
carruaje. Estaba leyendo en voz alta una parte del libro de Isaías.

\bibverse{29} Entonces el Espíritu le dijo a Felipe: ``Ve y acércate más
a ese carruaje.'' \bibverse{30} Y Felipe corrió hacia allá, y escuchó al
hombre que leía un texto del profeta Isaías.

``¿Entiendes lo que estás leyendo?'' le preguntó Felipe.

\bibverse{31} ``¿Cómo podría entender, si no hay quien me explique?''
respondió el hombre. Entonces invitó a Felipe a subirse al carruaje y
sentarse junto a él. \bibverse{32} Y el texto de la Escritura que estaba
leyendo era este: ``Como oveja, fue llevado al matadero; y como cordero
que enmudece ante su trasquilador, ni siquiera abrió su boca.
\bibverse{33} Lo humillaron y no le hicieron justicia. ¿Quién describirá
su descendencia? Porque su vida fue arrancada de la tierra.\footnote{\textbf{8:33}
  Literalmente, ``su vida fue tomada de la tierra.''}''

\bibverse{34} Entonces el eunuco le preguntó a Felipe: ``Dime, ¿de quién
está hablando este profeta? ¿Es acaso de sí mismo, o de otra persona?''
\bibverse{35} Entonces Felipe comenzó a explicarle, partiendo de este
texto, y hablándole de Jesús. \bibverse{36} A medida que continuaban el
camino, llegaron a un lugar donde había agua. Entonces el eunuco dijo:
``Mira, aquí hay agua, ¿por qué no me bautizas?'' \bibverse{37}
\footnote{\textbf{8:37} El siguiente versículo (37) que se encuentra en
  algunas biblias, no se encuentra en los primeros manuscritos.}
\bibverse{38} Entonces dio la orden para que detuvieran el carruaje. Y
Felipe y el eunuco descendieron juntos al agua y Felipe lo bautizó.
\bibverse{39} Cuando salieron del agua, el Espíritu del Señor se llevó a
Felipe. Y el eunuco no lo vio más, pero siguió su camino con alegría.
Felipe se encontró entonces en Azoto. \bibverse{40} Y allí predicaba la
buena nueva en todas las ciudades por las que pasaba, hasta que llegó a
Cesarea.

\hypertarget{section-8}{%
\section{9}\label{section-8}}

\bibverse{1} Pero mientras tanto, Saulo estaba enviando amenazas
violentas contra los discípulos del Señor, deseoso de matarlos. Así que
fue donde el sumo sacerdote \bibverse{2} y solicitó cartas de
autorización para ir a las sinagogas de Damasco, y para tener permiso de
arrestar a todos los creyentes que encontrara en El Camino\footnote{\textbf{9:2}
  ``Creyentes en El Camino,'' es un término antiguo que se refería a los
  seguidores de Jesús.}, hombres o mujeres, y traerlos de regreso a
Jerusalén como prisioneros.

\bibverse{3} Pero cuando Saulo se aproximaba a Damasco, de repente fue
rodeado por una luz brillante que descendía del cielo. \bibverse{4}
Entonces Saulo cayó al suelo, y escuchó una voz que decía: ``Saulo,
Saulo, ¿por qué me persigues?''

\bibverse{5} ``¿Quién eres, Señor?'' preguntó Saulo.

``Yo soy Jesús, al que persigues,'' le respondió. \bibverse{6}
``Levántate, ve a la ciudad y allí se te dirá lo que debes hacer.''

\bibverse{7} Y los hombres que iban de viaje con Saulo estaban sin
palabras. Habían oído la voz que hablaba, pero no vieron a nadie.
\bibverse{8} Entonces Saulo se puso en pie, y cuando abrió sus ojos no
podía ver. Entonces sus compañeros de viaje lo tomaron de la mano y lo
llevaron hasta Damasco. \bibverse{9} Durante tres días Saulo no pudo
ver, y no comió y bebió nada.

\bibverse{10} En Damasco vivía un seguidor de Jesús. Su nombre era
Ananías, y el Señor le habló en una visión.

``¡Ananías!'' llamó el Señor.

``Estoy aquí, Señor,'' respondió Ananías.

\bibverse{11} ``Levántate y ve a la Calle Derecha,'' le dijo el Señor.
``Pregunta en la casa de Judas por un hombre llamado Saulo de Tarso. Él
está orando. \bibverse{12} Ha visto en visión a un hombre llamado
Ananías que llega y pone sus manos sobre él para que recobre su vista.''

\bibverse{13} ``Pero Señor,'' respondió Ananías, ``He oído muchas cosas
acerca de este hombre, y sobre todas las cosas malas que hizo a los
creyentes de Jerusalén. \bibverse{14} Los jefes de los sacerdotes le han
dado poder para arrestar a todos los que te adoran y te siguen.''

\bibverse{15} Pero el Señor le dijo: ``Ve, porque él es la persona a la
cual he escogido para llevar mi nombre a los extranjeros y reyes, así
como a Israel. \bibverse{16} Yo le mostraré que él tendrá que sufrir por
causa de mi nombre.''

\bibverse{17} Entonces Ananías salió y fue a la casa que el Señor le
mostró. Y puso sus manos sobre Saulo. ``Hermano Saulo,'' le dijo, ``El
Señor Jesús, quien se apareció delante de ti en el camino cuando
viajabas hacia acá, me ha enviado para que recobres tu vista y seas
lleno del Espíritu Santo.'' \bibverse{18} De inmediato, de sus ojos
cayeron como escamas, y su vista fue restaurada. Entonces se levantó y
fue bautizado. \bibverse{19} También comió y se sintió más fuerte.

Y Saulo pasó varios días con los discípulos en Damasco. \bibverse{20}
Entonces comenzó de inmediato a predicar en las sinagogas, diciendo:
``Jesús es el Hijo de Dios.'' \bibverse{21} Y todos los que lo oían
predicar estaban asombrados, y preguntaban: ``¿Acaso no es este el
hombre que causó tantos problemas a los creyentes de Jesús en Jerusalén?
¿Acaso no vino aquí para arrestar y llevar encadenados a los creyentes
ante los jefes de los sacerdotes?'' \bibverse{22} Y Saulo crecía cada
vez más, así como su fe, demostrando de manera muy convincente que Jesús
era el Mesías, tanto que los habitantes de Damasco no podían refutar lo
que decía.

\bibverse{23} Tiempo después, los judíos conspiraron para matarlo,
\bibverse{24} pero Saulo se enteró de sus intenciones. De día y de noche
esperaban en las puertas de la ciudad, buscando una oportunidad para
matarlo. \bibverse{25} Así que durante la noche sus seguidores lo
tomaron y lo hicieron descender en una canasta, desde una abertura del
muro de la ciudad. \bibverse{26} Cuando Saulo llegó a Jerusalén, trató
de encontrar a los discípulos, pero todos le tenían miedo, porque no
estaban convencidos de que él realmente fuera discípulo. \bibverse{27}
Sin embargo, Bernabé lo llevó donde estaban los apóstoles, y les explicó
cómo Saulo había visto al Señor durante el camino y cómo el Señor le
había hablado. Bernabé también explicó cómo Saulo había hablado con
vehemencia en nombre del Señor en Damasco.

\bibverse{28} Saulo se quedó con los apóstoles y los acompañó hasta
Jerusalén, \bibverse{29} predicando abiertamente en nombre del Señor. Y
Saulo hablaba y debatía con los judíos de habla griega, pero ellos
trataron de matarlo. \bibverse{30} Pero cuando los hermanos supieron
acerca de esto, lo llevaron a Cesarea y lo enviaron a Tarso.

\bibverse{31} Durante este tiempo, toda la iglesia en Judea, Galilea y
Samaria estuvo en tranquilidad. Y la iglesia se fortalecía y aumentaba
en número a medida que los creyentes vivían en reverencia para con el
Señor\footnote{\textbf{9:31} Literalmente, ``en el temor del Señor.''},
animados por el Espíritu Santo.

\bibverse{32} Pedro andaba de viaje y fue a visitar a los creyentes que
vivían en Lida. \bibverse{33} Allí conoció a un hombre llamado Eneas,
quien era paralítico y había quedado confinado a estar en su cama desde
hacía ocho años. \bibverse{34} Entonces Pedro le dijo: ``¡Eneas,
Jesucristo te sana! ¡Levántate y recoge tu camilla!'' Y de inmediato
Eneas se levantó. \bibverse{35} Y todos los que vivían en Lida y Sarón
lo vieron, y se convirtieron en creyentes del Señor.

\bibverse{36} En Jope vivía una seguidora llamada Tabita, (Dorcas en
griego\footnote{\textbf{9:36} Tabita/Dorcas significa ``gacela.''}).
Ella siempre hacía el bien y ayudaba a los pobres. \bibverse{37} Sin
embargo, durante esos días ella se enfermó y murió. Y después de lavar
su cuerpo, la acostaron en una habitación que estaba en la parte de
arriba. \bibverse{38} Lida estaba cerca a Jope, así que los discípulos
que estaban en Jope, al saber que Pedro estaba en Lida, enviaron a dos
hombres con el siguiente mensaje: ``Por favor, ven acá de inmediato.''
\bibverse{39} Así que Pedro se alistó y se fue con ellos. Y cuando llegó
lo llevaron a la habitación de arriba. Todas las viudas estaban ahí
llorando, y le mostraban a Pedro los abrigos y ropas que Dorcas había
hecho mientras estuvo con ellas. \bibverse{40} Entonces Pedro les pidió
que salieran de la habitación, y se arrodilló y oró. Entonces dio vuelta
al cuerpo de Tabita y dijo: ``Tabita, levántate.'' Entonces ella abrió
los ojos, y cuando vio a Pedro se sentó. \bibverse{41} Luego Pedro la
tomó de la mano y la levantó. Y entonces llamó a los creyentes y a las
viudas, y la presentó viva delante de ellos. \bibverse{42} Y la noticia
se esparció por toda la ciudad de Jope, y muchos creyeron en el Señor.
\bibverse{43} Pedro se quedó mucho tiempo en Jope, hospedándose en la
casa de Simón el curtidor.

\hypertarget{section-9}{%
\section{10}\label{section-9}}

\bibverse{1} En Cesarea vivía un hombre llamado Cornelio, quien era un
centurión romano del ejército italiano. \bibverse{2} Este era un hombre
devoto que, junto con todos los que vivían en su casa, tenían reverencia
por Dios. Este hombre daba a los pobres con generosidad, y oraba a Dios
con regularidad. \bibverse{3} Un día, cerca de las 3 p.m. Cornelio tuvo
una visión en la que vio claramente a un ángel de Dios que venía hacia
él y lo llamaba, diciendo: ``¡Cornelio!''

\bibverse{4} Entonces Cornelio, asustado, lo miró y preguntó: ``¿Qué
quieres, Señor?''

``Dios ha escuchado tus oraciones, y ha reconocido tu generosidad para
con los pobres,'' le dijo el ángel a Cornelio. \bibverse{5} ``Ahora
envía hombres a Jope, y trae a Simón, también llamado Pedro,
\bibverse{6} el cual se está hospedando en la casa de Simón el curtidor,
cuya casa está junto al mar.''

\bibverse{7} Y cuando el ángel se fue, Cornelio llamó a dos de sus
siervos y a un soldado de su guardia personal, quien era un hombre
devoto. \bibverse{8} Después de haberles explicado todo lo que había
sucedido, los envió a Jope.

\bibverse{9} Al día siguiente, mientras ellos iban de camino y se
aproximaban a la ciudad, Pedro subió a la azotea\footnote{\textbf{10:9}
  Las casas de esa época tenían techos planos, y servían como
  habitaciones al aire libre.} para orar. Era cerca del mediodía,
\bibverse{10} y ya sentía hambre, y deseaba comer. Pero mientras aun
preparaban la comida, Pedro entró en trance, \bibverse{11} y vio el
cielo abierto. También vio que algo descendía y era como una sábana
grande que estaba sostenida por sus cuatro esquinas, y descendía hacia
la tierra. \bibverse{12} Adentro había todo tipo de animales, reptiles y
aves. \bibverse{13} Entonces oyó una voz que dijo: ``¡Levántate, Pedro!
¡Mata y come!''

\bibverse{14} Pero Pedro respondió: ``¡Por supuesto que no, Señor! Nunca
he comido nada que sea inmundo e impuro.''

\bibverse{15} Entonces escuchó nuevamente la voz: ``¡No llames impuro lo
que Dios ha limpiado!'' \bibverse{16} Y esto sucedió tres veces, y
entonces la sábana fue rápidamente llevada de nuevo al cielo.

\bibverse{17} Mientras Pedro aún estaba perturbado por saber qué
significaba aquella visión que había tenido, los hombres enviados por
Cornelio habían encontrado la casa de Simón y estaban en pie frente a la
puerta. \bibverse{18} Ellos llamaban, preguntando si Simón, también
llamado Pedro, se hospedaba allí. \bibverse{19} Mientras Pedro aún
reflexionaba sobre la visión, el Espíritu le dijo: ``Mira, hay tres
hombres buscándote. \bibverse{20} Levántate, baja, y ve con ellos. No te
preocupes en absoluto, porque yo los envié.''

\bibverse{21} Entonces Pedro descendió para encontrarse con los tres
hombres. ``Yo soy a quien buscan,'' dijo. ``¿Por qué están aquí?''

\bibverse{22} ``Venimos de parte de Cornelio, un hombre bueno, devoto,
que tiene temor de Dios y es respetado entre el pueblo judío,''
respondieron. ``Un ángel lo instruyó para que enviara a buscarte y
llevarte hasta su casa para escuchar lo que tú tienes para decirle.''
\bibverse{23} Entonces Pedro los invitó a entrar y ellos se quedaron
allí.

Al día siguiente, Pedro se levantó y se fue con ellos. Y algunos de los
hermanos de Jope también fueron con ellos. \bibverse{24} Al otro día
llegaron a Cesarea, donde los esperaba Cornelio junto a sus parientes y
amigos cercanos, a quienes había reunido. \bibverse{25} Cuando Pedro
entró a la casa, Cornelio salió a su encuentro y cayendo a sus pies, lo
adoró. \bibverse{26} Pero Pedro lo hizo retroceder, diciéndole:
``¡Levántate! ¡Yo soy solo un hombre!''

\bibverse{27} Entonces Pedro habló con Cornelio y luego siguieron hacia
adentro, donde había muchas otras personas esperándolo. \bibverse{28} Y
Pedro les dijo: ``Sin duda alguna, ustedes saben que no se le permite a
un judío reunirse o visitar extranjeros. Pero Dios me ha mostrado que no
me corresponde a mí llamar impuro o inmundo a ninguno. \bibverse{29} Es
por eso que vine sin ningún problema cuando enviaron a buscarme. Así que
ahora quiero saber la razón por la cual me mandaron a buscar.''

\bibverse{30} ``Hace cuatro días, cerca de esta misma hora---tres de la
tarde---yo estaba orando en mi casa,'' explicó Cornelio. ``Cuando de
repente vi a un hombre en pie frente a mí, vestido con ropas que
brillaban. \bibverse{31} Y me dijo: `Cornelio, tus oraciones han sido
escuchadas, y Dios ha reconocido tu generosidad para con los pobres.
\bibverse{32} Envía a alguien hasta Jope y trae a Simón Pedro. Él se
está hospedando en la casa de Simón, el curtidor, junto a la orilla del
mar.' \bibverse{33} Así que de inmediato envié a buscarte, y me complace
que hayas venido. Por eso estamos todos aquí, reunidos delante de Dios,
listos para escuchar todo lo que el Señor te ha dicho.''

\bibverse{34} Entonces Pedro respondió: ``Estoy plenamente convencido de
que Dios no tiene favoritos. \bibverse{35} En toda nación, Dios recibe a
aquellos que lo respetan y hacen lo recto. \bibverse{36} Ustedes conocen
el mensaje que Dios envió a Israel, compartiendo la buena noticia de paz
que viene de Jesucristo, quien es Señor de todos. \bibverse{37} Ustedes
saben que esta buena noticia fue predicada por Judea, comenzando en
Galilea, siguiendo el llamado de Juan al bautismo. \bibverse{38} Es la
buena noticia sobre Jesús de Nazaret y cómo Dios lo ungió con el
Espíritu Santo, con poder, y cómo anduvo por todas partes haciendo el
bien, sanando a todos los que estaban bajo control del diablo, porque
Dios estaba con él.

\bibverse{39} Nosotros podemos dar testimonio de todo lo que él hizo en
Judea y en Jerusalén. Lo mataron, colgándolo en una cruz. \bibverse{40}
Pero Dios lo levantó nuevamente a la vida en el tercer día, y lo hizo
aparecer, \bibverse{41} no a todos, sino a los que son testigos elegidos
por Dios, incluyéndonos a nosotros, quienes comimos y bebimos con él
después de que se levantó de los muertos. \bibverse{42} Él nos dio la
responsabilidad de contar públicamente esto a la gente, de testificar
que él es el Escogido de Dios como Juez de los vivos y los muertos.
\bibverse{43} Él es Aquél del cual hablaron todos los profetas, para que
todo el que crea en él reciba perdón por medio de su nombre.''

\bibverse{44} Mientras Pedro aún hablaba, el Espíritu Santo fue
derramado sobre todos los que estaban ahí oyendo el mensaje.
\bibverse{45} Entonces los creyentes judíos\footnote{\textbf{10:45}
  Literalmente, ``los que creían en la circuncisión.''} que habían
venido con pedro estaban asombrados, porque el don del Espíritu Santo
también había sido derramado sobre los extranjeros. \bibverse{46} Y los
oían hablando en lenguas, glorificando a Dios. \bibverse{47} Entonces
Pedro preguntó: ``¿Impedirá alguien que estos sean bautizados en agua,
siendo que han recibido el Espíritu Santo, igual que nosotros?''
\bibverse{48} Entonces Pedro dio orden de que fueran bautizados en el
nombre de Jesucristo. Entonces le pidieron que se quedara más tiempo con
ellos.

\hypertarget{section-10}{%
\section{11}\label{section-10}}

\bibverse{1} Los apóstoles y los hermanos de Judea oyeron que algunos
extranjeros también habían aceptado la palabra de Dios. \bibverse{2} Y
cuando Pedro llegó a Jerusalén, los que creían que la
circuncisión\footnote{\textbf{11:2} Literalmente, ``los de la
  circuncisión,'' lo cual puede simplemente significar ``los judíos.''
  Sin embargo, por el contexto podría parecer que estos eran judíos
  cristianos preocupados por las relaciones con los ``extranjeros.''}
todavía era esencial, comenzaron a discutir con él. \bibverse{3}
``Fuiste a casa de hombres incircuncisos y comiste con ellos,'' dijeron.

\bibverse{4} Entonces Pedro comenzó a explicarles todo lo que había
ocurrido. \bibverse{5} ``Mientras estaba en la ciudad de Jope y oraba,
estando en trance vi una visión. Algo que parecía como una gran sábana
era bajada por sus cuatro extremos desde el cielo, y bajó hacia mí.
\bibverse{6} Cuando miré había animales adentro, bestias salvajes,
reptiles y aves.

\bibverse{7} ``Entonces oí una voz que me dijo: `Levántate, mata y
come.'

\bibverse{8} ``Pero yo respondí `¡Por supuesto que no, Señor! ¡Nunca ha
entrado en mi boca nada inmundo ni impuro!'

\bibverse{9} ``Entonces la voz del cielo habló otra vez, y dijo: `¡No
llames inmundo a lo que Dios ha limpiado!' \bibverse{10} Esto sucedió
tres veces, y luego todo esto se devolvió al cielo. \bibverse{11} En ese
mismo momento había tres hombres frente a la puerta de la casa donde nos
estábamos hospedando. Habían sido enviados desde Cesarea para verme.
\bibverse{12} Entonces el Espíritu me dijo que fuera con ellos, y que no
me preocupara acerca de quiénes eran. Estos seis hermanos que están aquí
también fueron conmigo, y entramos a la casa del hombre. \bibverse{13}
Él nos explicó cómo un ángel se le había aparecido en su casa, el cual
le dijo: `Envía a alguien a Jope, y haz venir a Simón, también llamado
Pedro, \bibverse{14} quien te dirá lo que necesitas escuchar para que
seas salvo, tú y toda tu casa.'

\bibverse{15} ``Cuando comencé a hablar, el Espíritu Santo se derramó
sobre ellos, como sucedió con nosotros al principio. \bibverse{16}
Entonces recordé lo que el Señor dijo: ``Juan bautizaba con agua, pero
ustedes serán bautizados con el Espíritu Santo.'' \bibverse{17} Si Dios
les dio el mismo don que nos dio a nosotros cuando creímos en el Señor
Jesucristo, ¿qué autoridad tendría yo para estar en contra de Dios?''

\bibverse{18} Después de escuchar esta explicación, no volvieron a
discutir con él, y alababan a Dios, diciendo: ``Ahora Dios ha concedido
también a los extranjeros la oportunidad de arrepentimiento y de tener
vida eterna.''

\bibverse{19} Sucedió que los que habían quedado esparcidos por causa de
la persecución que ocurrió cuando Esteban fue asesinado, viajaron hasta
Fenicia, Chipre y Antioquía. Y solo predicaban la buena noticia entre
los judíos. \bibverse{20} Pero cuando algunos de ellos que eran de
Chipre y Cirene llegaron a Antioquía, comenzaron a predicar la buena
noticia también a los griegos, hablándoles acerca del Señor Jesús.
\bibverse{21} Y el poder del Señor estaba con ellos y una gran cantidad
de gente creyó en el Señor y se convirtió a él. \bibverse{22} Entonces
se difundió la noticia acerca de lo que había ocurrido y llegó hasta la
iglesia en Jerusalén, y enviaron a Bernabé hasta Antioquía.
\bibverse{23} Cuando llegó y vio con sus propios ojos cómo estaba
obrando la gracia de Dios, se deleitó en esto. Y animó a todos a que se
consagraran por completo a Dios y a mantenerse fieles. \bibverse{24}
Bernabé era un buen hombre, lleno del Espíritu Santo y había puesto toda
su confianza en Dios. Y muchas personas eran traídas al Señor.
\bibverse{25} Entonces Bernabé se fue a Tarso para buscar a Saulo,
\bibverse{26} y cuando lo encontró, lo hizo regresar con él a Antioquía.
Y durante el transcurso de un año trabajaron juntos con la iglesia,
enseñando el mensaje a multitudes. Y fue en Antioquía que a los
creyentes se les llamó por primera vez ``Cristianos.''

\bibverse{27} Durante este tiempo algunos profetas fueron de Jerusalén a
Antioquía. \bibverse{28} Y uno de ellos, llamado Ágabo se puso en pie y
profetizó una advertencia por medio del Espíritu que habría una hambruna
terrible que afectaría a todo el mundo\footnote{\textbf{11:28}
  Literalmente, ``el mundo no habitado,'' refiriéndose básicamente a
  Imperio Romano.}. (Esto se cumplió durante el reinado del Emperador
Claudio). \bibverse{29} Los creyentes decidieron enviar fondos para
ayudar a los hermanos que vivían en Judea, dando cada uno conforme a lo
que tenía. \bibverse{30} Hicieron esto y enviaron el dinero con Bernabé
y Saulo a los líderes de la iglesia de Judea.

\hypertarget{section-11}{%
\section{12}\label{section-11}}

\bibverse{1} Durante estos días, el Rey Herodes comenzó a perseguir a
algunos miembros de la iglesia. \bibverse{2} Y mandó a matar a espada a
Santiago, el hermano de Juan. \bibverse{3} Y cuando vio que los judíos
se complacían en esto, mandó a arrestar a Pedro también. (Esto sucedió
durante la Fiesta de los Panes sin Levadura.) \bibverse{4} Después de
arrestar a Pedro, lo enviaron a la prisión, con cuatro escuadrones de
soldados para vigilarlo. Su plan era traer a Pedro a un juicio público
después de la Pascua.

\bibverse{5} Y mientras Pedro estaba en la cárcel, la iglesia oraba
fervientemente a Dios por él. \bibverse{6} La noche antes de que Herodes
lo llevara a juicio, Pedro estaba durmiendo entre dos soldados,
encadenado a cada uno de ellos, y había guardas que vigilaban la puerta.
\bibverse{7} De repente, un ángel del Señor apareció, y una luz
resplandeció en la celda. El ángel sacudió a Pedro para despertarlo,
diciendo: ``¡Rápido! ¡Levántate!'' Entonces las cadenas se cayeron de
sus manos, \bibverse{8} y el ángel le dijo: ``Vístete y ponte tus
sandalias.'' Y así lo hizo Pedro. Entonces el ángel le dijo: ``Ponte tu
abrigo y sígueme.'' \bibverse{9} Así que Pedro lo siguió hasta afuera. Y
no se daba cuenta de que lo que el ángel hacía estaba sucediendo en
realidad, pues pensaba que estaba teniendo una visión.

\bibverse{10} Luego pasaron la primera y segunda guardia, y llegaron
hasta la puerta de hierro que conducía hasta la ciudad. Y esta se abrió
por sí sola. Entonces salieron y descendieron por la calle, cuando de
repente el ángel lo dejó. \bibverse{11} Cuando Pedro volvió en sí, dijo:
``¡Ahora me doy cuenta de que esto realmente sucedió! El Señor envió un
ángel para rescatarme del poder de Herodes, y de todo lo que el pueblo
judío había planeado.''

\bibverse{12} Y ahora que Pedro estaba consciente de lo que había
sucedido, fue a la casa de María, la madre de Juan Marcos. Y muchos
creyentes se habían reunido allí y estaban orando. \bibverse{13} Cuando
Pedro tocó la puerta, una sierva llamada Rode salió a abrirle.
\bibverse{14} Pero al reconocer la voz de Pedro, en su emoción, no abrió
la puerta sino que corrió hacia adentro, gritando: ``¡Pedro está en la
puerta!''

\bibverse{15} ``¡Estás loca!'' le dijeron. Pero ella siguió insistiendo
en que era cierto. Entonces dijeron: ``Debe ser su ángel\footnote{\textbf{12:15}
  ``Su ángel.'' En ese tiempo, algunos creían que las personas tenían un
  equivalente espiritual que existía ya fuera que el individuo estuviera
  vivo o muerto. Probablemente la expresión hoy en día sería: ``¡Es su
  fantasma!''}.'' \bibverse{16} Pero Pedro siguió tocando a la puerta.
Cuando finalmente la abrieron, lo vieron y estaban conmocionados.

\bibverse{17} Pedro levantó su mano para indicarles que guardaran
silencio, y entonces les explicó cómo el Señor lo había sacado de la
cárcel. ``Hagan saber de esto a Santiago y a los hermanos,'' les dijo, y
luego se marchó a otro lugar.

\bibverse{18} Cuando llegó el amanecer, había una total
confusión\footnote{\textbf{12:18} Literalmente, ``una gran confusión.''}
entre los soldados respecto a lo que le había sucedido a Pedro.
\bibverse{19} Herodes mandó a realizar una minuciosa búsqueda de él,
pero no lo encontraron. Y después de interrogar a los soldados, Herodes
ordenó la ejecución de todos ellos\footnote{\textbf{12:19} En realidad,
  el griego dice: ``que se los llevaran.'' Sin embargo, la gran mayoría
  de los comentaristas entienden esto como ``que se los llevaran para
  matarlos,'' puesto que el castigo por permitir que los presos
  escaparan era la ejecución.}. Entonces Herodes se fue de Judea y se
quedó en Cesarea.

\bibverse{20} Ahora Herodes estaba furioso con el pueblo de Tiro y
Sidón. Entonces enviaron una delegación para verlo y lograron ganarse el
favor de Blasto, el asistente personal del rey, para que los ayudara.
Ellos suplicaban paz a Herodes porque dependían del territorio del rey
para conseguir el alimento. \bibverse{21} Cuando llegó la hora de
encontrarse con el rey, Herodes se puso sus vestidos reales, se sentó en
su trono, y dio un discurso para ellos. \bibverse{22} La audiencia gritó
como respuesta: ``¡Esta es la voz de un dios, no de un hombre!''
\bibverse{23} De inmediato el ángel del Señor lo derribó, porque no le
dio la gloria a Dios. Y fue consumido por los gusanos y murió.

\bibverse{24} Pero la Palabra de Dios se esparcía, y cada vez más
personas creían. \bibverse{25} Bernabé y Saulo regresaron de Jerusalén
una vez terminaron su misión, trayendo conmigo a Juan Marco con ellos.

\hypertarget{section-12}{%
\section{13}\label{section-12}}

\bibverse{1} La iglesia de Antioquía tenía profetas y maestros: Bernabé,
Simón (llamado el Negro), Lucio de Cirene, Manaén (amigo de la infancia
de Herodes, el tetrarca), y Saulo. \bibverse{2} Mientras estaban
adorando al Señor y ayunando, el Espíritu Santo dijo:
``Aparten\footnote{\textbf{13:2} O, ``dediquen.''} a Bernabé y a Saulo
para que hagan el trabajo para el cual los he llamado.'' \bibverse{3}
Después de ayunar, orar y colocar sus manos sobre ellos para
bendecirlos, los enviaron.

\bibverse{4} Entonces Bernabé y Saulo, dirigidos por el Espíritu Santo,
fueron a Seleucia. Y de allí navegaron hasta Chipre. \bibverse{5} Al
llegar a Salamis, proclamaron la palabra de Dios en las sinagogas
judías. Y Juan\footnote{\textbf{13:5} Este Juan es Juan Marcos (ver.
  12:25).} estaba con ellos como asistente. \bibverse{6} Viajaron por
toda la isla y finalmente llegaron a Pafos. Allí encontraron a un mago
judío, un falso profeta que tenía por nombre Barjesús. \bibverse{7} Era
cercano al gobernador, Sergio Paulo, un hombre inteligente. Este
gobernador invitó a Bernabé y a Saulo para que fueran a visitarlos pues
quería escuchar la palabra de Dios. \bibverse{8} Pero el mago Elimas (su
nombre griego) se les opuso, tratando de evitar que el gobernante
creyera en Dios.

\bibverse{9} Saulo, también llamado Pablo, estaba lleno del Espíritu
Santo, y lo miró fijamente. \bibverse{10} ``¡Estás lleno de engaño y de
todo tipo de mal, hijo del diablo, enemigo de todo lo recto! ¿Nunca
dejarás de pervertir los caminos verdaderos del Señor? \bibverse{11}
Mira, la mano del Señor está sobre ti y quedarás ciego. Y no verás el
sol por un tiempo.'' De inmediato, sobre él cayó neblina y oscuridad, y
tuvo que encontrar a alguien que pudiera llevarlo de la mano.
\bibverse{12} Cuando el gobernante vio lo que había ocurrido, creyó en
Dios, asombrado por la enseñanza sobre el Señor.

\bibverse{13} Entonces Pablo y los que estaban con él navegaron desde
Pafos y se fueron a Perga, en Panfilia, pero Juan los dejó y regresó a
Jerusalén. \bibverse{14} Entonces fueron por toda Perga y luego
siguieron hasta Antioquía de Pisidia. El sábado fueron a la sinagoga y
se sentaron. \bibverse{15} Después de leer la Ley y los Profetas, los
líderes de la sinagoga les enviaron un mensaje, diciendo: ``Hermanos,
por favor, compartan con la congregación toda palabra de ánimo que
puedan.''

\bibverse{16} Entonces Pablo se levantó, moviendo su mano para obtener
su atención, y comenzó a hablar. ``Hombres de Israel, y todos los que
reverencian a Dios, escúchenme. \bibverse{17} El Dios del pueblo de
Israel eligió a nuestros antepasados, y le dio prosperidad a nuestro
pueblo durante su estadía en la tierra de Egipto. Con su gran poder los
sacó de Egipto \bibverse{18} y los trató con paciencia en el desierto
durante cerca de cuarenta años.

\bibverse{19} ``Después de haber derrocado a siete naciones que vivían
en la tierra de Canaán, Dios dividió su tierra entre los Israelitas y se
las dio como heredad. Esto tomó cerca de cuatrocientos cincuenta años.
\bibverse{20} Luego los dotó de jueces como líderes hasta los días del
profeta Samuel. \bibverse{21} Entonces el pueblo pidió un rey, y Dios le
dio a Saúl, hijo de Quis, de la tribu de Benjamín, quien gobernó durante
cuarenta años. \bibverse{22} Entonces Dios quitó a Saúl, y puso a David
como su rey. Dios aprobó a David, diciendo: `He encontrado en David,
hijo de Isaí, un hombre conforme a mi corazón. Él cumplirá todos mis
propósitos.'

\bibverse{23} ``Jesús es descendiente de David; él es el Salvador que
Dios prometió enviar a Israel. \bibverse{24} Antes de que Jesús viniera,
Juan anunció el bautismo de arrepentimiento a todo el pueblo de Israel.
\bibverse{25} Y cuando Juan estaba finalizando su misión, dijo: `¿Quién
creen que soy? Yo no soy al que ustedes buscan. Pero después de mi viene
uno cuyas sandalias no soy digno de desatar.'

\bibverse{26} ``Hermanos míos, hijos de Abrahán y todos ustedes que
reverencian a Dios: ¡Este mensaje de salvación ha sido enviado a
nosotros! \bibverse{27} La gente que vivía en Jerusalén y sus líderes no
reconocieron a Jesús ni entendieron las palabras que los profetas habían
dicho y que leen cada sábado. ¡De hecho, ellos mismos cumplieron las
palabras proféticas al condenarlo! \bibverse{28} Aunque no pudieron
encontrar ninguna prueba para sentenciarlo, pidieron a Pilato que lo
mandara a matar. \bibverse{29} Después de haber cumplido todo lo que se
había predicho que ellos le harían a Jesús, lo bajaron de la cruz y lo
sepultaron en una tumba. \bibverse{30} Pero Dios lo levantó de los
muertos, \bibverse{31} y él se apareció durante muchos días a aquellos
que lo habían seguido desde Galilea hasta Jerusalén. Ellos son ahora sus
testigos ante la gente.

\bibverse{32} ``Nosotros estamos aquí para traerles a ustedes la buena
noticia de la promesa que Dios hizo a nuestros antepasados,
\bibverse{33} que ahora ha cumplido en nuestro favor al levantar a Jesús
de los muertos. Tal como está escrito en el libro de Salmos 2: `Tú eres
mi Hijo, y hoy me he convertido en tu Padre.' \bibverse{34} Dios lo
levantó de los muertos, para que no muriera más, tal como lo indicó al
decir: `Yo les daré cosas santas y fieles, como se lo prometí a
David.'\footnote{\textbf{13:34} Refiriéndose a Isaías 55:3.}
\bibverse{35} Tal como lo dice otro salmo: `No dejarás que tu Santo
conozca la putrefacción.'\footnote{\textbf{13:35} Citando Salmos 16:10.}
\bibverse{36} Pero David murió, después de haber hecho la voluntad de
Dios a su tiempo, y fue sepultado con sus antepasados, y su cuerpo
sufrió descomposición. \bibverse{37} El que Dios levantó de los muertos
no sufrió descomposición.

\bibverse{38} Hermanos míos, quiero que entiendan que lo que les estamos
diciendo es que por medio de este hombre hay perdón de pecados.
\bibverse{39} Por medio de él todo el que cree es justificado de todos
sus errores, y de una manera que nunca podría ser posible mediante la
ley de Moisés. \bibverse{40} Asegúrense de que no les suceda lo que
dijeron los profetas: \bibverse{41} ``¡Miren, burlones! ¡Asómbrense y
desaparezcan! Estoy por hacer en estos días una obra que ustedes nunca
creerán, aunque alguien se la explique.''\footnote{\textbf{13:41}
  Citando Habacuc 1:5.}

\bibverse{42} Y cuando salían, la gente les suplicaba que les explicaran
más el siguiente sábado. \bibverse{43} Después de reunirse en la
sinagoga, muchos de los judíos y conversos al judaísmo siguieron a Pablo
y a Bernabé, quienes hablaban con ellos, animándolos a seguir firmes en
la gracia de Dios. \bibverse{44} El sábado siguiente casi toda la ciudad
se presentó para escuchar la palabra de Dios. \bibverse{45} Sin embargo,
cuando los judíos vieron las multitudes, se pusieron extremadamente
furiosos, contradiciendo lo que Pablo decía y maldiciéndolo.

\bibverse{46} Entonces Pablo y Bernabé hablaron con firmeza, diciendo:
``Primero teníamos que predicarles la palabra de Dios a ustedes. Pero
ahora que la han rechazado, ustedes están determinando que no son dignos
de la vida eterna. Pues ahora predicaremos a los extranjeros.
\bibverse{47} Porque eso es lo que el Señor nos ha mandado a hacer: `Yo
los he convertido en luz para los extranjeros, y a través de ustedes la
salvación llegará hasta los fines de la tierra.'\footnote{\textbf{13:47}
  Citando Isaías 49:6.} \bibverse{48} Cuando los extranjeros escucharon
esto, se alegraron en gran manera, alabando la palabra del Señor, y
todos los elegidos para la vida eterna creyeron en Dios. \bibverse{49}
Así que la palabra de Dios fue esparcida por toda la región.
\bibverse{50} Pero los judíos incitaron a mujeres devotas e influyentes
y también a líderes de la ciudad para perseguir a Pablo y a Bernabé, y
los expulsaron de su territorio. \bibverse{51} Entonces Pablo y Bernabé
sacudieron el polvo de sus pies hacia ellos en señal de protesta, y se
fueron a Iconio. \bibverse{52} Y los creyentes seguían siendo llenos de
gozo y del Espíritu Santo.

\hypertarget{section-13}{%
\section{14}\label{section-13}}

\bibverse{1} En Iconio ocurrió lo mismo: Pablo y Bernabé fueron a la
sinagoga judía y hablaron con tanta seguridad que muchos adoradores,
tanto de habla griega como judíos, creyeron en Jesús. \bibverse{2} Pero
los judíos que se negaron a creer en Jesús provocaron sentimientos
negativos en los extranjeros\footnote{\textbf{14:2} En otras palabras,
  la población que no era judía.}, y difamaban de los creyentes delante
de ellos. \bibverse{3} Pero Pablo y Bernabé permanecieron allí por mucho
tiempo, hablando audazmente en el Señor, quien confirmaba su mensaje de
gracia mediante señales milagrosas que ellos podían realizar.
\bibverse{4} Los habitantes de la ciudad estaban divididos, pues algunos
apoyaban a los judíos y otros a los apóstoles. \bibverse{5} Pero
entonces los extranjeros y los judíos, junto a sus líderes, decidieron
atacar con piedras a Pablo y a Bernabé. \bibverse{6} Sin embargo, ellos
supieron sobre este plan y huyeron a la región de Licaonia, a las
ciudades de Listra y Derbe, \bibverse{7} y allí siguieron compartiendo
la buena noticia.

\bibverse{8} En la ciudad de Listra había un hombre paralítico, que
tenía lisiados los dos pies. Había nacido en esta condición y nunca
había podido caminar. \bibverse{9} Entonces este hombre se sentó allí
para escuchar a Pablo predicar. Y cuando Pablo lo miró directamente a él
y se dio cuenta de que este hombre estaba creyendo en el Señor para ser
sanado, \bibverse{10} dijo en voz alta: ``¡Levántate y ponte de pie!'' Y
este hombre de un salto se puso en pie y comenzó a caminar.
\bibverse{11} Y cuando la multitud vio lo que Pablo había hecho,
gritaron en el idioma de Licaonia, ``¡Los dioses han descendido hasta
nosotros en forma de humanos!'' \bibverse{12} Entonces identificaron a
Bernabé como el dios griego Zeusy a Pablo como el dios Hermes, porque él
era el que predicaba la mayor parte del tiempo.

\bibverse{13} Entonces los sacerdotes del templo de Zeus que está justo
a las fueras de la ciudad, trajeron bueyes y guirnaldas\footnote{\textbf{14:13}
  Guirnaldas. Eran puestas sobre los animales antes de ser sacrificados.}a
las puertas de la ciudad. Planeaban hacer un sacrificio frente a la
multitud. \bibverse{14} Pero cuando los apóstoles Bernabé y Pablo se
enteraron de lo que estaba sucediendo, rasgaron su ropa\footnote{\textbf{14:14}
  En las culturas antiguas era una señal de gran aflicción.}, y
corrieron hasta la multitud, gritando: \bibverse{15} ``Señores, ¿qué
hacen? Nosotros somos seres humanos con la misma naturaleza de ustedes.
Vinimos a traerles buenas noticias para que ustedes abandonen estas
cosas inútiles y se vuelvan a un Dios que realmente está vivo. Él es
quien hizo el cielo, la tierra, el mar y todo lo que hay en ellos.
\bibverse{16} En tiempos pasados, él dejó que las naciones siguieran sus
propios caminos. \bibverse{17} Pero aun así demostró quién era al hacer
el bien, enviándoles lluvia del cielo y cosechas a su tiempo, dándoles
todo lo que necesitaban, y llenando sus corazones de alegría.''
\bibverse{18} Con estas palabras apenas lograron detener a las
multitudes para que no les ofrecieran sacrificios.

\bibverse{19} Pero entonces ciertos judíos de Antioquía e Iconio
llegaron y se ganaron la simpatía de la multitud. Y apedrearon a Pablo,
y lo arrastraron hasta las afueras de la ciudad, pensando que estaba
muerto. \bibverse{20} Pero cuando los creyentes se reunieron a su
alrededor, Pablo se levantó, y regresó a la ciudad. Al día siguiente,
Pablo y Bernabé partieron a Derbe. \bibverse{21} Y después de predicar
la buena noticia con las personas de esa ciudad, y después de que muchos
se convirtieran en creyentes, regresaron a Listra, Iconio y Antioquía.
\bibverse{22} Entonces animaron a los creyentes a mantenerse firmes y a
seguir creyendo en Jesús. ``Tenemos que pasar por muchas pruebas para
entrar al reino de Dios,'' decían.

\bibverse{23} Después de haber escogido ancianos para cada iglesia, y de
haber orado y ayunado con ellos, Pablo y Bernabé los encomendaron al
Señor, Aquél en quien creían. \bibverse{24} Y pasaron por Pisidia, y
llegaron a Panfilia. \bibverse{25} Predicaron la palabra de Dios en
Perga, y siguieron hasta Atalía. \bibverse{26} De allí navegaron de
regreso hasta Antioquía\footnote{\textbf{14:26} Antioquía en Siria,
  donde habían iniciado su viaje (ver 13:1).} donde habían comenzado,
pues ahí habían sido dedicados en la gracia de Dios para la obra que
ahora habían logrado. \bibverse{27} Y cuando llegaron, reunieron a toda
la iglesia y les informaron todo lo que el Señor había hecho por medio
de ellos y cómo había abierto las puertas para que los extranjeros
creyeran en él. \bibverse{28} Y se quedaron allí con los creyentes por
mucho tiempo.

\hypertarget{section-14}{%
\section{15}\label{section-14}}

\bibverse{1} Entonces llegaron unos hombres de Judea que comenzaron a
enseñarles a los creyentes, y les decían ``A menos que estén
circuncidados conforme a las normas establecidas por Moisés, no podrán
salvarse.'' \bibverse{2} Y Pablo y Bernabé debatieron y discutieron
mucho con ellos. Así que Pablo y Bernabé y otros más fueron nombrados
para ir a Jerusalén y hablar con los apóstoles y los líderes de allí
sobre este asunto. \bibverse{3} Entonces la iglesia los envió de viaje,
y mientras viajaban por Fenicia y Samaria, explicaban cómo los
extranjeros se estaban convirtiendo, y esto alegraba mucho a los
creyentes. \bibverse{4} Cuando llegaron a Jerusalén fueron recibidos por
los miembros de la iglesia, los apóstoles y los ancianos. Explicaron
todo lo que Dios había hecho a través de ellos. \bibverse{5} Pero
sufrieron oposición de parte de algunos de los creyentes que pertenecían
a la división de los Fariseos. Ellos decían: ``Estos conversos tienen
que circuncidarse e instruirse para que observen la ley de Moisés.''

\bibverse{6} Entonces los apóstoles y los ancianos se reunieron para
debatir el asunto. \bibverse{7} Y después de tanto debatir, Pedro se
levantó y les dijo: ``Hermanos, ustedes saben que hace un tiempo Dios me
escogió de entre ustedes para que los extranjeros pudieran oír el
mensaje de la buena noticia y creyeran en Jesús. \bibverse{8} Dios,
quien conoce nuestros corazones\footnote{\textbf{15:8} En otras
  palabras, conoce nuestra forma de pensar.}, ha demostrado que los
acepta, dándoles el Espíritu Santo a ustedes así como lo hizo con
nosotros. \bibverse{9} Él no hace distinción entre nosotros y ellos, y
limpió sus corazones cuando ellos creyeron en él.

\bibverse{10} ``Entonces, ¿por qué ustedes quieren oponerse a Dios y
colocar sobre los creyentes cargas que nuestros padres no fueron capaces
de soportar, y que nosotros tampoco podemos? \bibverse{11} Estamos
convencidos de que somos salvos mediante la gracia del Señor Jesús, así
como ellos.'' \bibverse{12} Y todos escuchaban con atención a Bernabé y
a Pablo cuando ellos les hablaban de las señales milagrosas que Dios
había realizado entre los extranjeros a través de ellos.

\bibverse{13} Después que terminaron de hablar, Santiago tomó la
palabra, diciendo: ``Hermanos, escúchenme. \bibverse{14}
Simón\footnote{\textbf{15:14} Simón Pedro.} ha descrito cómo Dios
primero reveló su interés por los extranjeros escogiendo entre ellos un
pueblo comprometido con él. \bibverse{15} Esto está en conformidad con
las palabras de los profetas, tal como está escrito: \bibverse{16} `En
el futuro, volveré, y reconstruiré la casa caída de David; yo
reconstruiré sus ruinas y las enderezaré. \bibverse{17} Haré esto para
que los que han quedado por fuera vengan al Señor, incluyendo los
extranjeros que invocan mi nombre. \bibverse{18} Esto es lo que el Señor
dice, el que reveló estas cosas hace mucho tiempo.'

\bibverse{19} ``Así que mi decisión es que no debemos ser estorbo para
los extranjeros que se convierten a Dios. \bibverse{20} Debemos
escribirles y decirles que eviten la comida sacrificada a los
ídolos\footnote{\textbf{15:20} Literalmente, ``contaminación de
  ídolos.''}, la inmoralidad sexual, la carne de animales que hayan sido
estrangulados, y de consumir sangre. \bibverse{21} Porque la Ley de
Moisés ya ha sido enseñada en cada ciudad por mucho tiempo, pues es
leída en las sinagogas cada sábado.''

\bibverse{22} Entonces los apóstoles y los ancianos, en reunión con toda
la iglesia, decidieron que sería bueno elegir a algunos representantes y
enviarlos a Antioquía con Pablo y Bernabé. Y eligieron a Judas Barsabás
y a Silas, líderes entre los hermanos, \bibverse{23} y los enviaron con
esta carta:

``Saludos de parte de nosotros, los apóstoles y ancianos, a los hermanos
no judíos\footnote{\textbf{15:23} Literalmente, ``Gentiles.''} de
Antioquía, Siria y Cilicia: \bibverse{24} Hemos oído que algunos de
nuestro grupo los han confundido con sus enseñanzas, causándoles
problemas. Sin duda alguna nosotros no les dijimos que hicieran esto.
\bibverse{25} Así que decidimos elegir algunos representantes y
enviarlos hasta donde ustedes están, junto con nuestros hermanos muy
amados, Bernabé y Pablo, \bibverse{26} quienes han arriesgado sus vidas
por el nombre de nuestro Señor Jesucristo.

\bibverse{27} ``Así que hemos enviado a Judas y Silas, quienes podrán
confirmarles verbalmente lo que les estamos diciendo. \bibverse{28} El
Espíritu Santo y nosotros consideramos que es mejor no colocarles
ninguna carga pesada aparte de estos requisitos. \bibverse{29} Deben
evitar: cualquier cosa sacrificada a ídolos, sangre, carne de animales
estrangulados, e inmoralidad sexual. Harán bien al observar estos
requisitos. Dios los bendiga.''

\bibverse{30} Los hombres fueron enviados a Antioquía. Y cuando
llegaron, convocaron a todos a una reunión y entregaron la carta.
\bibverse{31} Después de leerla, estaban muy felices por el mensaje de
ánimo. \bibverse{32} Entonces Judas y Silas, que también eran profetas,
animaron a los hermanos, enseñándoles muchas cosas, y dándoles
fortaleza. \bibverse{33} Después de pasar un tiempo allí, fueron
enviados de regreso por los hermanos, con su bendición, a los creyentes
de Jerusalén. \bibverse{34} \footnote{\textbf{15:34} Se cree que el
  versículo 34 referente a Silas no hace parte del original.}
\bibverse{35} Pero Pablo y Bernabé permanecieron en Antioquía, enseñando
y proclamando la palabra de Dios en compañía de muchos otros.

\bibverse{36} Algún tiempo después, Pablo le dijo a Bernabé:
``Regresemos y visitemos a los creyentes de cada ciudad donde hemos
predicado la palabra de Dios, y veamos cómo están.'' \bibverse{37}
Entonces Bernabé hizo planes para llevarse también a Juan Marcos.
\bibverse{38} Pero Pablo no consideró que fuera buena idea llevarlo con
ellos, pues él los había abandonado en Panfilia y no había seguido
trabajando con ellos. \bibverse{39} Y tuvieron un descuerdo tan grande,
que se separaron. Entonces Bernabé tomó a Juan Marcos y navegó hacia
Chipre. \bibverse{40} Pablo eligió a Silas, y al marcharse, los
creyentes los encomendaron a la gracia del Señor. \bibverse{41} Y Pablo
viajó por Siria y Cilicia, animando a las iglesias de esos lugares.

\hypertarget{section-15}{%
\section{16}\label{section-15}}

\bibverse{1} Entonces Pablo fue primero a Derbe, y luego a Listra, donde
conoció a un creyente llamado Timoteo. Este era el hijo de una madre
cristiana judía, y su padre era griego. \bibverse{2} Y los hermanos en
Listra e Iconio hablaron bien de él. \bibverse{3} Pablo quería que
Timoteo viajara con él, así que lo circuncidó porque todos los judíos de
la región sabían que el padre de Timoteo era griego. \bibverse{4} Y
mientras pasaban por las diferentes ciudades, enseñaban los requisitos
que los apóstoles y ancianos en Jerusalén habían dicho que debían
observarse. \bibverse{5} Las iglesias fueron fortalecidas en su fe en el
Señor, y cada día aumentaban los miembros.

\bibverse{6} Viajaron también por los distritos de Frigia y Galacia,
pues el Espíritu Santo los advirtió de ir a la provincia de Asia para
predicar la palabra. \bibverse{7} Cuando llegaron a la frontera de Misia
trataron de entrar a Bitinia, pero el Espíritu de Jesús no los dejó
entrar. \bibverse{8} Así que pasaron por Misia y descendieron a Troas.

\bibverse{9} Allí Pablo vio en visión durante la noche a un hombre de
Macedonia en pie, rogándole: ``¡Por favor, ven a Macedonia a
ayudarnos!'' \bibverse{10} Y después que Pablo tuvo esta visión,
hicimos\footnote{\textbf{16:10} El cambio de pronombre a ``nosotros''
  indica que el escritor, Lucas, se había unido a ellos.} arreglos de
inmediato para ir a Macedonia, pues concluimos que Dios nos había
llamado para predicar la buena noticia con ellos.

\bibverse{11} Entonces partimos navegando desde Troas directo hasta
Samotracia. Al día siguiente continuamos hasta Neápolis, \bibverse{12} y
de allí hasta Filipos, que es la ciudad más importante de Macedonia, y
también una colonia romana. Y nos quedamos en esta ciudad durante varios
días. \bibverse{13} El sábado salimos por las puertas de la ciudad hacia
las orillas del río, donde pensábamos que la gente iría a orar. Entonces
nos sentamos y hablamos con las mujeres que se habían reunido allí.

\bibverse{14} Una de ellas se llamaba Lidia, era de la ciudad de Tiatira
y vendía paños de púrpura. Era una adoradora de Dios, y nos escuchó.
Entonces el Señor abrió su mente a lo que Pablo le decía, y ella aceptó
lo que él le dijo. \bibverse{15} Después que ella y toda su casa se
bautizaron, nos rogó: ``Si ustedes realmente creen que estoy
comprometida con el Señor, vengan y quédense en mi casa.'' Y siguió
insistiendo hasta que aceptamos.

\bibverse{16} Un día, cuando descendíamos al lugar de la oración,
conocimos a una joven esclava que estaba poseída por un espíritu
maligno\footnote{\textbf{16:16} Literalmente, ``espíritu pitón,'' que
  era un espíritu de adivinación.}. Ella ganaba para sus amos mucho
dinero a través de la adivinación. \bibverse{17} Esta chica siguió a
Pablo y al resto de nosotros por todos los lugares, gritando: ``Estos
hombres son siervos del Dios Todopoderoso. ¡Ellos dicen cómo se puede
ser salvo!'' \bibverse{18} Y siguió haciendo esto por varios días. Pero
esto molestó a Pablo, así que se dio vuelta y le dijo al espíritu: ``¡En
el nombre de Jesucristo te ordeno que salgas de ella!'' Y el espíritu
salió de ella inmediatamente.

\bibverse{19} Pero cuando sus amos vieron que la joven había perdido sus
medios para ganar dinero, agarraron a Pablo y a Silas y los llevaron a
rastras ante las autoridades que estaban en la plaza del mercado.
\bibverse{20} Y los llevaron ante los magistrados, acusándolos: ``Estos
hombres judíos están causando grandes disturbios en nuestra ciudad,''
decían. \bibverse{21} ``Están enseñando ideas que son ilegales para
nosotros como romanos, y que no aceptamos ni practicamos.''
\bibverse{22} Entonces la multitud se reunió para atacarlos. Los
magistrados rasgaron la ropa de Pablo y Silas, y ordenaron que fueran
golpeados con varas. \bibverse{23} Y después de darles una golpiza
severa, los metieron en la prisión, ordenándole al carcelero que los
mantuviera bajo llave. \bibverse{24} El carcelero siguió las órdenes. Y
metió a Pablo y a Silas en la celda interna y encadenó sus pies en el
cepo.

\bibverse{25} Cerca de la media noche Pablo y Silas estaban orando y
cantando alabanzas a Dios, y los demás prisioneros los escuchaban.
\bibverse{26} De repente un terrible terremoto sacudió los cimientos de
la cárcel. De inmediato todas las puertas se abrieron y las cadenas de
todos se cayeron.

\bibverse{27} Entonces el carcelero despertó y vio que las puertas de la
cárcel estaban abiertas. Y entonces sacó su espada, y estaba a punto de
matarse, pensando que los prisioneros habían escapado. \bibverse{28}
Pero Pablo gritó: ``¡No te hagas daño, todavía estamos aquí!''

\bibverse{29} Entonces el carcelero pidió que le trajeran lámparas y se
apresuró. Temblando de miedo cayó a los pies de Pablo y Silas.
\bibverse{30} Luego los acompañó hasta la puerta y les preguntó:
``Señores, ¿qué necesito hacer para ser salvo?''

\bibverse{31} ``Cree en el Señor Jesús y serás salvo, tú y toda tu
casa,'' respondieron. \bibverse{32} Luego predicaron la palabra del
Señor con él y con todos los que vivían en su casa. \bibverse{33} Y
aunque era tarde en la noche, lavó sus heridas y fue bautizado allí
mismo, junto a su familia. \bibverse{34} Y los llevó a su casa y mandó a
preparar comida para ellos. Y el carcelero y toda su familia estaban
felices porque creían en Dios.

\bibverse{35} Siendo temprano, al día siguiente, el magistrado envió
oficiales donde el carcelero, diciéndole: ``Libera a esos hombres.''
\bibverse{36} Entonces el carcelero le dijo a Pablo: ``Los magistrados
han enviado orden para dejarte libre. Así que puedes irte, y ve en
paz.''

\bibverse{37} Pero Pablo les dijo: ``¡Ellos nos golpearon públicamente
sin un juicio, y nosotros somos ciudadanos romanos! Luego nos echaron a
la cárcel. ¿Ahora quieren dejarnos ir discretamente? ¡No, ellos deben
venir personalmente y liberarnos!''

\bibverse{38} Entonces los oficiales regresaron e informaron esto a los
magistrados. Cuando oyeron que Pablo y Silas eran ciudadanos romanos, se
preocuparon mucho, \bibverse{39} y fueron a disculparse con
ellos\footnote{\textbf{16:39} Era ilegal castigar a un ciudadano romano
  sin un juicio previo.}. Y los acompañaron afuera y les rogaron que se
fueran de la ciudad. \bibverse{40} Entonces Pablo y Silas salieron de la
cárcel y se fueron a la casa de Lidia. Allí se encontraron con los
creyentes, los animaron y siguieron su camino.

\hypertarget{section-16}{%
\section{17}\label{section-16}}

\bibverse{1} Después que Pablo y Silas pasaron por Anfípolis y Apolonia,
llegaron a Tesalónica, donde había una sinagoga judía. \bibverse{2} Como
de costumbre, Pablo entró a la sinagoga y durante tres sábados debatió
con ellos, usando las Escrituras. \bibverse{3} Entonces les explicó su
significado, demostrándoles que el Mesías tenía que Morir y resucitar.
``Este Jesús del cual les hablo, es el Mesías,'' les dijo. \bibverse{4}
Y algunos de ellos se convencieron y se unieron a Pablo y a Silas, junto
con muchos adoradores griegos\footnote{\textbf{17:4} Los adoradores que
  hablaban griego: el término a menudo se aplica a ``paganos'' que
  habían aceptado la creencia en el Dios del judaísmo pero no se habían
  vuelto judíos por circuncisión.} y algunas mujeres influyentes de la
ciudad.

\bibverse{5} Pero los judíos se pudieron celosos y junto a unos
agitadores que encontraron en la plaza del mercado\footnote{\textbf{17:5}
  Literalmente, ``hombres malos que había en el mercado.''} formaron una
turba. Y se amotinaron en la ciudad, y atacaron la casa de Jasón.
Entonces trataron de encontrar a Pablo y a Silas para presentarlos ante
la gente. \bibverse{6} Pero como no pudieron encontrarlos, arrastraron a
Jasón y a otros creyentes ante los líderes de la ciudad, gritando: ``A
estas personas se les conoce por estar causando problemas y desorden.
Ahora vinieron aquí, \bibverse{7} y Jasón los ha recibido en su casa.
Todos ellos desafían los decretos del César, cometiendo traición al
decir que hay otro rey, llamado Jesús.'' \bibverse{8} El pueblo y los
líderes de la ciudad estaban muy perturbados al escuchar esto.
\bibverse{9} Entonces obligaron a Jasón y a los otros a pagar fianza
antes de dejarlos ir.

\bibverse{10} Los creyentes hicieron salir a Pablo y Silas hacia Berea
esa misma noche. Y cuando llegaron allí, fueron a la sinagoga judía.
\bibverse{11} La gente de allí tenía mejor actitud que los de
Tesalónica, pues aceptaron rápidamente la palabra, y examinaban las
Escrituras cada día para asegurarse de que era correcto lo que les
enseñaban. \bibverse{12} Como resultado de esto, muchos se convirtieron
en creyentes, así mismo algunas mujeres y hombres griegos que tenían
cargos importantes.

\bibverse{13} Pero cuando los judíos de Tesalónica oyeron que Pablo
también estaba predicando la palabra en Berea, fueron hasta allá, y
causaron los mismos disturbios, provocando a las multitudes.
\bibverse{14} De inmediato los creyentes enviaron a Pablo a la costa,
mientras que Silas y Timoteo se quedaron. \bibverse{15} Y los que
acompañaban a Pablo lo llevaron muy lejos, hasta Atenas, y regresaron
con instrucciones que Pablo envió a Silas y a Timoteo para que ellos
fueran a acompañarlo tan pronto como fuera posible.

\bibverse{16} Mientras los esperaba en Atenas, Pablo estaba muy
perturbado al ver la idolatría que se practicaba en la ciudad.
\bibverse{17} Él debatía en la sinagoga con los judíos y con los
adoradores de Dios\footnote{\textbf{17:17} Se cree que era la misma
  ``clase'' de creyentes que se mencionaron en 17:4: extranjeros que
  habían aceptado al Dios de Israel pero no se habían convertido en
  judíos.}, así como también lo hacía en las plazas del mercado con los
que se encontraba cada día. \bibverse{18} Algunos filósofos epicúreos y
estoicos también discutían con él. ``¿De qué habla este
hombre?''\footnote{\textbf{17:18} Literalmente, ``¿Qué está tratando de
  decir este recolector de semillas?'' ``Recolector de semillas'' se
  refería a pájaros parloteadores que recolectaban semillas en el
  mercado; también puede traducirse como ``hablador.''} Se preguntaban.
Y otros concluían: ``Parece que enseña sobre dioses extranjeros,''
porque hablaba sobre Jesús y la resurrección. \bibverse{19} Entonces lo
llevaron al Aerópago\footnote{\textbf{17:19} Una especie de reunión de
  debate para filósofos.}, y le pidieron: ``Por favor háblanos sobre
esta nueva enseñanza que estás promoviendo. \bibverse{20} Hemos oído de
ti cosas que para nosotros son extrañas, por eso nos gustaría saber lo
que significan.'' \bibverse{21} (Todos los atenienses, incluyendo a los
extranjeros que vivían allí, pasaban todo el tiempo sin hacer nada más
que explicar o escuchar sobre cosas nuevas).

\bibverse{22} Entonces Pablo se puso en pie en medio del Aerópago y
dijo: ``Pueblo de Atenas, puedo ver que ustedes son muy devotos en todo.
\bibverse{23} Y mientras caminaba, viendo sus santuarios, encontré un
altar que tenía la inscripción, ``A un Dios no conocido.'' Este Dios no
conocido a quien ustedes adoran es el que yo les estoy describiendo.
\bibverse{24} El Dios que creó el mundo y todo lo que hay en él, el
Señor del cielo y la tierra, no vive en los templos que nosotros
hacemos. \bibverse{25} Él no necesita que le sirvamos, como si él
necesitara de alguna cosa, porque él es la fuente de vida de todo ser
vivo. \bibverse{26} De un solo hombre él hizo a todos los pueblos que
viven en la tierra, y decidió de antemano cuándo y dónde debían vivir.
\bibverse{27} El propósito de Dios era que ellos lo buscaran, esperando
que ellos se acercaran a él y lo encontraran, aunque él no está lejos de
ninguno de nosotros. \bibverse{28} En él vivimos, nos movemos y
existimos. Tal como escribieron los mismos poetas de entre ustedes:
``Somos su familia.''

\bibverse{29} Ya que somos su familia, no debemos pensar que Dios es
como el oro, la plata o una piedra moldeada por arte y pensamiento
humano. \bibverse{30} Dios pasó por alto la ignorancia de la gente en el
pasado, pero ahora llama a todos, en todas partes, al arrepentimiento.
\bibverse{31} Porque él ha establecido un tiempo en el cual juzgará con
justicia al mundo por medio del hombre que él ha elegido, y les ha
demostrado a todos que él es el escogido al resucitarlo de los
muertos.''

\bibverse{32} Algunos de ellos se burlaron cuando escucharon acerca de
la resurrección de los muertos, pero otros dijeron: ``Por favor, regresa
más tarde para que podamos oír más sobre esto.'' \bibverse{33} Entonces
Pablo se fue. \bibverse{34} Y unos cuantos hombres se unieron a él y
creyeron en Dios, incluyendo a Dionisio, un miembro del Aerópago, así
como una mujer llamada Damaris, y otros más.

\hypertarget{section-17}{%
\section{18}\label{section-17}}

\bibverse{1} Entonces Pablo partió de Atenas y se fue a Corinto,
\bibverse{2} y allí conoció a un judío llamado Aquila. Este era de
Ponto, y acababa de llegar de Italia con su esposa Priscila porque
Claudio\footnote{\textbf{18:2} El Emperador romano.} había deportado a
todos los judíos expulsados de Roma. Y Pablo fue a verlos, \bibverse{3}
y como estaban en el mismo negocio de fabricar tiendas, se quedó con
ellos. \bibverse{4} Y Pablo debatía en la sinagoga cada sábado,
convenciendo tanto a griegos como a judíos. \bibverse{5} Cuando Silas y
Timoteo llegaron desde Macedonia, Pablo sintió que necesitaba ser más
directo en lo que predicaba, y les dijo a los judíos que Jesús era el
Mesías. \bibverse{6} Y cuando ellos se le opusieron y lo maldijeron,
sacudió su ropa\footnote{\textbf{18:6} Un acto simbólico que declaraba
  inocencia.} y les dijo: ``¡La sangre de ustedes está sobre sus propias
cabezas! Soy libre de toda culpa, y desde ahora iré a los extranjeros.''

\bibverse{7} Entonces se marchó y se fue a quedar donde Tito Justo,
quien adoraba a Dios y cuya casa estaba al lado de la sinagoga.
\bibverse{8} Crispo, líder de la sinagoga, creía en el Señor y también
toda su casa. Y muchas personas de Corinto que escucharon el mensaje se
convirtieron en creyentes y fueron bautizados.

\bibverse{9} Entonces el Señor le dijo a Pablo en una visión de noche:
``No tengas miedo. Habla, no te quedes callado \bibverse{10} porque yo
estoy contigo, y nadie te hará daño, pues muchas personas en esta ciudad
son mías.'' \bibverse{11} Y Pablo se quedó allí durante dieciocho meses,
enseñando la palabra de Dios.

\bibverse{12} Sin embargo, cuando Galión\footnote{\textbf{18:12} Galión
  era el hermano de Séneca, el filósofo romano estoico.} se convirtió en
el gobernante de Acaya, los judíos se unieron para atacar a Pablo y lo
llevaron ante la corte\footnote{\textbf{18:12} Literalmente ``tribunal
  de juicio,'' o ``estrado.'' Ver también en 18:16, 17.}. \bibverse{13}
``Este hombre está persuadiendo al pueblo para adorar a Dios
ilegalmente,'' declararon.

\bibverse{14} Pero cuando Pablo estaba a punto de defenderse, Galión les
dijo a los judíos: ``Si ustedes los judíos me trajeran cargos criminales
o una ofensa legal grave, habría razón para que yo escuchara su caso.
\bibverse{15} Pero como solo están discutiendo por las palabras y
nombres y respecto a la propia ley de ustedes, entonces encárguense
ustedes mismos. Yo no voy a gobernar respecto a tales asuntos.''
\bibverse{16} Después de esto Galión mandó a sacarlos de la corte.
\bibverse{17} Entonces la multitud tomó a Sóstenes, líder de la
sinagoga, y lo golpearon justo a las afueras de la corte, pero a Galión
no le preocupó esto en absoluto.

\bibverse{18} Pablo se quedó por un tiempo. Entonces dejó a los
creyentes de allí y partió hacia Siria, llevando consigo a Priscila y
Aquila. En Cencrea mandó a afeitar su cabeza, porque había hecho un
voto\footnote{\textbf{18:18} Voto: probablemente un voto nazareo (ver
  Números 6).}.

\bibverse{19} Entonces llegaron a Éfeso, donde Pablo había dejado a los
otros. Y se dirigió a la sinagoga para razonar con los judíos.
\bibverse{20} Y ellos le pidieron que se quedara por más tiempo, pero
Pablo no aceptó. \bibverse{21} Entonces se despidió y emprendió su viaje
desde Éfeso, diciéndoles: ``Regresaré y los veré nuevamente si es la
voluntad de Dios.''

\bibverse{22} Después de desembarcar en Cesarea fue a saludar a los
miembros de iglesia\footnote{\textbf{18:22} Posiblemente los miembros de
  la iglesia en Jerusalén.}, y entonces siguió hasta Antioquía.
\bibverse{23} Y se quedó un tiempo allí y luego fue de ciudad en ciudad
por la región de Galacia y Frigia, animando a los creyentes.

\bibverse{24} Durante este tiempo, un judío llamado Apolo, de
Alejandría, llegó a Éfeso. Era un orador con mucho talento, que conocía
bien las Escrituras. \bibverse{25} Se le había enseñado el camino del
Señor. Era apasionado por lo espiritual, y en su hablar y su enseñanza
presentaba a Jesús de manera precisa, pero solo sabía acerca del
bautismo de Juan. \bibverse{26} Entonces comenzó a hablar de manera
abierta en la sinagoga. De modo que cuando Priscila y Aquila lo
escucharon, lo invitaron a unirse a ellos y le enseñaron con mayor
amplitud el camino del Señor. \bibverse{27} Cuando decidió ir a Acaya,
los hermanos lo animaron y le escribieron a los discípulos de allí,
diciéndoles que lo recibieran. Y cuando llegó fue de gran ayuda a los
que por gracia creían en Dios, \bibverse{28} porque podía refutar
enérgicamente a los judíos en debates públicos, demostrando con las
Escrituras que Jesús era el Mesías.

\hypertarget{section-18}{%
\section{19}\label{section-18}}

\bibverse{1} Mientras Apolos estaba en Corinto, Pablo tomó camino tierra
adentro y llegó a Éfeso, donde encontró a algunos creyentes.
\bibverse{2} ``¿Recibieron al Espíritu Santo cuando creyeron?'' les
preguntó.

``No, no hemos escuchado nada acerca de un Espíritu Santo,'' le dijeron.

\bibverse{3} ``Entonces ¿qué bautismo recibieron?'' preguntó.

``El bautismo de Juan,'' respondieron ellos.

\bibverse{4} ``Juan bautizaba con el bautismo del arrepentimiento,''
dijo Pablo. ``Él enseñaba a las personas que debían creer en el que
vendría después de él, es decir, que debían creer en Jesús.''
\bibverse{5} Cuando oyeron esto, fueron bautizados en el nombre del
Señor Jesús. \bibverse{6} Y después que Pablo puso sus manos sobre
ellos, el Espíritu Santo descendió sobre ellos y hablaron en lenguas y
profetizaron. \bibverse{7} Había aproximadamente doce de ellos en total.

\bibverse{8} Entonces Pablo fue a la sinagoga y durante los siguientes
tres meses habló de forma clara a los que estaban allí, debatiendo con
ellos y tratando de convencerlos acerca del mensaje del reino de Dios.
\bibverse{9} Pero algunos de ellos eran tercos, y no quisieron aceptar.
Ellos condenaban a El Camino\footnote{\textbf{19:9} ``El Camino'': otro
  término antiguo para referirse a los cristianos.} ante la multitud.
Así que Pablo se dio por vencido respecto a ellos y se fue a la
sinagoga, llevando consigo a los creyentes. Entonces comenzó a hacer
debates cada día en el salón de Tirano.

\bibverse{10} Así sucedió durante los siguientes dos años, logrando que
todos los que vivían en la provincia de Asia, tanto judíos como griegos,
escucharan la palabra del Señor. \bibverse{11} Y Dios realizaba milagros
extraordinarios a través de Pablo, \bibverse{12} tanto así, que la gente
tomaba los pañuelos o delantales que Pablo había tocado para sanar a los
enfermos y para expulsar espíritus malignos.

\bibverse{13} Y ciertos judíos que iban por ahí haciendo exorcismos,
decidieron usar el nombre del Señor Jesús cuando sacaban a los espíritus
malignos. Y decían: ``Te ordeno que salgas en el nombre de Jesús, del
que habla Pablo.'' \bibverse{14} Y los que hacían esto eran los siete
hijos de Esceva, un judío y jefe de sacerdotes.

\bibverse{15} Pero un día, un espíritu maligno respondió: ``Yo conozco a
Jesús, y conozco a Pablo, pero ¿quién eres tú?'' \bibverse{16} Y el
hombre con el espíritu maligno saltó sobre ellos y los dominó a todos.
Los golpeó tan fuertemente que ellos salieron corriendo de la casa,
desnudos y malheridos.

\bibverse{17} Las personas que vivían en Éfeso, tanto judíos como
griegos, oyeron sobre esto. Y todos estaban asombrados por lo que había
sucedido, y el nombre del Señor Jesús recibió grande respeto.
\bibverse{18} Y muchos llegaron a creer en el Señor y confesaron sus
pecados, admitiendo abiertamente sus prácticas pecaminosas.
\bibverse{19} Muchos de los que practicaban la brujería recogieron sus
libros de magia y los llevaron para ser quemados públicamente. Y sacaron
cuentas sobre el valor de los libros, y el total era de cincuenta mil
monedas de plata. \bibverse{20} De este modo la palabra del Señor se
fortalecía y era predicada por todas partes.

\bibverse{21} Cierto tiempo después de esto, Pablo decidió ir a
Jerusalén, pasando primero por Macedonia y Acaya. ``Después de estar
allí, iré a Roma,'' dijo. \bibverse{22} Entonces envió a dos de sus
ayudantes, Timoteo y Erasto, a Macedonia, y mientras tanto él se quedó
por un tiempo en la provincia de Asia.

\bibverse{23} Durante este tiempo hubo serios problemas concernientes a
El Camino. \bibverse{24} Y un hombre llamado Demetrio, quien era un
artesano de plata, estaba produciendo pequeñas réplicas en material de
plata del templo de la diosa Artemisa. Este negocio requería mucho
trabajo para los artesanos. \bibverse{25} Entonces Demetrio los llamó a
una reunión, junto con otros que trabajaban en el mismo oficio, y dijo:
``Compañeros, ustedes saben que nosotros ganamos dinero gracias a este
negocio. \bibverse{26} Sin duda alguna, como ustedes ya saben por lo que
han visto y oído---no solo aquí en Éfeso sino por toda Asia---este tal
Pablo ha convencido y confundido a mucha gente, diciéndoles que no hay
dioses hechos por manos humanas. \bibverse{27} No se trata de que
nuestro negocio esté en peligro de perder el respeto, sino que el templo
de la gran diosa Artemisa perderá su valor ante los ojos de la gente. La
misma Artemisa será destronada de su alta posición como la única a la
cual todos en Asia y en el mundo entero adoramos.''

\bibverse{28} Y cuando ellos escucharon esto se pusieron furiosos, y
gritaron: ``¡Grande es Artemisa de los Efesios!'' \bibverse{29} Y la
ciudad estaba en total caos. La gente corrió hacia el anfiteatro,
arrastrando con ellos a Gayo y a Aristarco, quienes eran compañeros de
viaje de Pablo, y eran de Macedonia. \bibverse{30} Pablo creyó que era
su deber confrontar la turba, pero los demás creyentes no se lo
permitieron. \bibverse{31} Entonces algunos de los oficiales de la
provincia\footnote{\textbf{19:31} Literalmente, ``Asiarcas.''} que eran
amigos de Pablo enviaron un mensaje también, rogándole que no entrara al
anfiteatro.

\bibverse{32} Y algunos gritaban una cosa, y otros gritaban otra, pues
la multitud que se había reunido estaba totalmente confundida. Y muchos
de ellos no sabían por qué estaban allí. \bibverse{33} Y los judíos que
estaban en la multitud empujaron a Alejandro hacia el frente. Entonces
Alejandro movió su mano indicando que guardaran silencio, queriendo
explicar las cosas a la gente. \bibverse{34} Pero cuando se dieron
cuenta de que era un judío, comenzaron un canto que duró cerca de dos
horas, gritando: ``¡Grande es Artemisa de los Efesios!''

\bibverse{35} Después de que el secretario de la ciudad logró silenciar
a la turba, les dijo: ``Pueblo de Éfeso, ¿quién no sabe que la ciudad de
los Efesios es guardiana del templo de la gran Artemisa y de su imagen
que cayó del cielo? \bibverse{36} Ya que estos hechos no pueden negarse,
ustedes deben estar tranquilos y no hagan nada a la ligera.
\bibverse{37} Han traído aquí a estos hombres, pero ellos no han robado
ningún templo, ni han blasfemado contra nuestra diosa. \bibverse{38} De
modo que si Demetrio y los demás artesanos tienen alguna queja contra
alguien, entonces vayan a las autoridades\footnote{\textbf{19:38}
  Literalmente, ``procónsules.''} y a las cortes. Ellos podrán presentar
los cargos correspondientes. \bibverse{39} Si hay alguna otra cosa,
puede llevarse a asamblea legal. \bibverse{40} De hecho, nosotros mismos
corremos el peligro de ser acusados como responsables de un motín hoy,
pues no había razón para ello, y no podemos justificar por qué
sucedió.'' \bibverse{41} Y cuando terminó de hablar, despidió a la
multitud.

\hypertarget{section-19}{%
\section{20}\label{section-19}}

\bibverse{1} Una vez se disipó el alboroto, Pablo llamó a los creyentes
a una reunión y los animó. Entonces se despidió de ellos y partió hacia
Macedonia. \bibverse{2} Y anduvo por toda la región, compartiendo muchas
palabras de ánimo con los creyentes que estaban allí, y entonces
continuó su viaje hasta Grecia. \bibverse{3} Después de haber estado
tres meses allí, y cuando estaba a punto de embarcarse rumbo a Siria, se
supo que los judíos estaban organizando un complot contra él. Así que
decidió regresar por Macedonia. \bibverse{4} Y estas fueron las personas
que viajaron con él: Sópater de Berea, hijo de Pirro, Aristarco y
Segundo de Tesalónica, Gayo de Derbe, Timoteo, Tíquico y Trófimo de la
provincia de Asia. \bibverse{5} Ellos se fueron primero y nos esperaron
en Troas. \bibverse{6} Después de la Fiesta de los Panes sin Levadura,
nos embarcamos rumbo a Filipo, y nos encontramos con ellos cinco días
después en Troas, donde permanecimos una semana.

\bibverse{7} Pablo estaba predicando el primer día de la semana y
estábamos reunidos para partir el pan. Él estaba planeando partir en la
mañana, y siguió predicando hasta la media noche. \bibverse{8} (La
habitación de arriba, donde estábamos reunidos, estaba iluminada por
muchas lámparas.)

\bibverse{9} Y un joven llamado Eutico estaba sentado en la ventana y
comenzó a sentir mucho sueño. Mientras Pablo seguía predicando este
joven se durmió profundamente y se cayó del tercer piso. Cuando lo
recogieron se dieron cuenta de que estaba muerto.

\bibverse{10} Entonces Pablo descendió, extendió sus manos hacia él y lo
abrazó. ``No se preocupen, está vivo,'' dijo.

\bibverse{11} Entonces volvió a subir, partió pan y comió con ellos. Y
siguió hablando con todos hasta que llegó la mañana y entonces se
marchó. \bibverse{12} Luego se llevaron al joven a casa, vivo y sano, y
estaban muy agradecidos por ello.

\bibverse{13} Entonces continuamos nuestro camino hasta la embarcación y
partimos hacia Asón. Allí debíamos recoger a Pablo, pues ese fue su plan
al elegir viajar a pie. \bibverse{14} En efecto, nos encontramos con él
en Asón. Lo recogimos allí y nos fuimos hacia Mitilene. \bibverse{15}
Partimos de allí y llegamos a Quios, y al día siguiente nos detuvimos
por un rato en Samos, y al día siguiente llegamos a Mileto.
\bibverse{16} Pablo había planeado seguir directo hasta Éfeso para no
demorarse en la provincia de Asia. Tenía prisa de llegar a Jerusalén
para estar a tiempo para el Día del Pentecostés.

\bibverse{17} Desde Mileto, Pablo envió un mensaje a los ancianos de la
iglesia de Éfeso. \bibverse{18} Y cuando llegaron, les dijo: ``Ustedes
saben cómo me he comportado siempre con ustedes desde el primer día que
llegué a la provincia de Asia. \bibverse{19} Serví al Señor con humildad
y lágrimas. Enfrenté los problemas y el estrés que me causaban los
complots de los judíos. \bibverse{20} Sin embargo nunca me negué a
compartir con ustedes todo cuanto pudiera beneficiarlos, y les enseñé en
público, yendo de casa en casa. \bibverse{21} Fui testigo tanto para
judíos como para griegos de que era necesario arrepentirse y volverse a
Dios, y creer en nuestro Señor Jesucristo. \bibverse{22} Ahora el
Espíritu insiste en que vaya a Jerusalén, y no sé qué me sucederá allí.
\bibverse{23} Lo único que sé es que en cada ciudad que visito, el
Espíritu Santo me advierte que me espera prisión y sufrimiento.
\bibverse{24} Pero considero que mi vida no tiene ya valor para mí
mismo. Solo quiero terminar mi misión y el ministerio que el Señor Jesús
me dio de ser testigo de la buena noticia de la gracia de Dios.

\bibverse{25} ``Ahora estoy seguro de que ustedes no volverán a ver mi
rostro, ustedes entre los cuales compartí la noticia del reino.
\bibverse{26} Así que hoy les declaro que no soy responsable de que
alguno se pierda\footnote{\textbf{20:26} Literalmente, ``No soy culpable
  de la sangre de nadie''}. \bibverse{27} No dudé en enseñarles todo lo
que Dios desea que ustedes sepan. \bibverse{28} Cuídense ustedes mismos
y cuiden el rebaño, el cual les ha sido encomendado por el Espíritu
Santo para que cuiden de él. Alimenten la iglesia del Señor, la cual ha
comprado con su propia sangre. \bibverse{29} Yo sé que después de
marcharme vendrán lobos rapaces entre ustedes, y querrán destruir el
rebaño. \bibverse{30} Dentro del grupo de ustedes se levantarán hombres
queriendo pervertir lo recto para lograr que los creyentes los sigan a
ellos. \bibverse{31} ¡Así que estén atentos! No olviden que durante tres
años los instruí de día y de noche, a menudo con lágrimas. \bibverse{32}
Ahora los encomiendo al cuidado de Dios y al mensaje de su gracia, el
cual puede edificarlos y darles la heredad que pertenece a los que son
santificados. \bibverse{33} Nunca quise la plata, ni el oro, ni la ropa
de nadie. \bibverse{34} Saben que trabajé con mis propias manos para
sustentar mis propias necesidades, así como las de aquellos que estaban
conmigo. \bibverse{35} Les he dado ejemplo en todo: trabajen para ayudar
a los débiles, recordando las palabras del Señor Jesús: ``Más bendición
hay en dar que en recibir.''

\bibverse{36} Y cuando terminó de hablar, se arrodilló y oró con todos
ellos. \bibverse{37} Entonces todos lloraron mientras lo abrazaban y lo
besaban. \bibverse{38} Lo que más los atribulaba era lo que él había
dicho acerca de no volverlo a ver\ldots{} Entonces descendieron hasta la
embarcación con él.

\hypertarget{section-20}{%
\section{21}\label{section-20}}

\bibverse{1} Después de habernos despedido de ellos, navegamos
directamente hasta Cos, y al día siguiente continuamos hasta Rodas.
Desde allí nos fuimos hacia Pátara \bibverse{2} donde nos encontramos
con una embarcación que iba hacia Fenicia. Nos embarcamos en ella y
zarpamos. \bibverse{3} Luego avistamos Chipre por el lado izquierdo y
continuamos hasta Siria, luego bajamos a tierra en Tiro, que era donde
debía desembarcar la nave. \bibverse{4} Allí buscamos a los creyentes y
nos quedamos en ese lugar durante una semana. Y por medio del Espíritu
Santo los creyentes le dijeron a Pablo que no fuera a Jerusalén.
\bibverse{5} Cuando se acabó nuestro tiempo de estar allí, partimos y
regresamos al barco para seguir nuestro viaje. Todos los creyentes, y
las esposas e hijos, nos acompañaron al marcharnos de la ciudad. Allí en
la playa nos arrodillamos y oramos, y nos despedimos. \bibverse{6}
Entonces nos subimos al barco y regresamos a casa. \bibverse{7} Nuestro
viaje desde Tiro terminó en Tolemaida, y allí saludamos a los creyentes
y nos quedamos con ellos durante un día.

\bibverse{8} Al día siguiente partimos de allí y nos fuimos hasta
Cesarea. Nos quedamos en la casa de Felipe el evangelista (uno de los
siete)\footnote{\textbf{21:8} Uno de los siete escogidos para ayudar con
  la distribución de la comida (ver Hechos 6:5).}. \bibverse{9} Y Felipe
tenía cuatro hijas solteras que profetizaban. \bibverse{10} Después de
habernos quedado allí durante varios días, un profeta llamado Ágabo
llegó desde Judea. \bibverse{11} Al acercarse a nosotros, tomó el
cinturón de Pablo, y ató sus propias manos y pies. Entonces dijo: ``El
Espíritu Santo dice: `Así es como los judíos de Jerusalén cegarán al
hombre que posee este cinturón, y lo entregarán en manos de los
extranjeros.'\,''

\bibverse{12} Cuando oímos esto, nosotros y los creyentes le rogamos a
Pablo que no fuera a Jerusalén. \bibverse{13} Sin embargo, Pablo
respondió: ``¿Qué hacen? Están llorando y rompen mi corazón. No solo
estoy listo para ser amarrado en Jerusalén, sino también para morir allí
por causa del Señor Jesús.'' \bibverse{14} Y como no pudo ser persuadido
de lo contrario, nos dimos por vencidos y dijimos: Que se haga la
voluntad del Señor.''

\bibverse{15} Después de esto hicimos nuestras maletas y nos dirigimos
hacia Jerusalén. \bibverse{16} Y algunos de los creyentes de Cesarea
vinieron con nosotros, y nos llevaron hasta la casa de Nasón, donde
íbamos a quedarnos. Él venía de Chipre y fue uno de los primeros
creyentes.

\bibverse{17} Cuando llegamos a Jerusalén, los creyentes nos recibieron
calurosamente. \bibverse{18} Al día siguiente Pablo fue con nosotros a
ver a Santiago y todos los líderes de la iglesia estaban allí.
\bibverse{19} Después de saludarlos, Pablo comenzó a contar con detalles
todo lo que Dios había hecho por los extranjeros por medio de su
ministerio.

\bibverse{20} Y cuando oyeron lo que había sucedido, alabaron a Dios y
le dijeron a Pablo: ``Hermano, ahora puedes ver cuántos miles de judíos
han llegado a creer en el Señor, y todos guardan la Ley cuidadosamente.
\bibverse{21} A ellos les han dicho que tú enseñas a los judíos que
viven entre los extranjeros a que ignoren la ley de Moisés, diciéndoles
que no circunciden a sus hijos y que no sigan nuestras costumbres.

\bibverse{22} ``¿Qué debemos hacer al respecto? Sin duda la gente
escuchará que llegaste aquí. \bibverse{23} Queremos que hagas lo
siguiente: Cuatro de nuestros hombres han hecho un voto. \bibverse{24}
Ve con ellos y haz los rituales de purificación con ellos, pagándoles
para que les afeiten sus cabezas. De este modo todos sabrán que los
rumores que han escuchado acerca de ti no son ciertos, sino que tú mismo
guardas la Ley en tu forma de vivir. \bibverse{25} En cuanto a los
extranjeros que han creído en el Señor, ya escribimos una carta respecto
a nuestra decisión de que deben abstenerse de comer alimentos
sacrificados a los ídolos, consumir sangre o cualquier animal
estrangulado, y de inmoralidad sexual.''

\bibverse{26} Así que Pablo llevó consigo a estos hombres, y al día
siguiente fue y se purificó con ellos. Entonces fue al templo para dar
aviso respecto a la terminación del tiempo de purificación y de la
ofrenda que se daría por cada uno de ellos.

\bibverse{27} Se acercaba el fin de los siete días cuando ciertos judíos
de Asia vieron a Pablo en el templo y lo agarraron. \bibverse{28}
``¡Hombres de Israel, vengan a ayudarnos!'' gritaron. ``Este es el
hombre que está enseñando por todas partes para que se opongan a nuestro
pueblo, a la Ley y al templo. Además ha traído griegos al templo,
contaminando este lugar santo.'' \bibverse{29} (Decían esto porque
anteriormente lo habían visto en la ciudad con Trófimo, el efesio y
supusieron que Pablo lo había traído al templo). \bibverse{30} Y toda la
ciudad estaba impactada por este hecho y la gente llegaba corriendo.
Entonces agarraron a Pablo y lo sacaron a rastras del templo. De
inmediato se cerraron las puertas. \bibverse{31} Mientras trataban de
matarlo, el comandante del batallón romano recibió la noticia de que
toda la ciudad de Jerusalén estaba alborotada.

\bibverse{32} De inmediato el comandante tomó a unos centuriones y
descendió corriendo hasta donde estaba la turba. Cuando la multitud vio
al comandante y a los soldados, dejaron de golpear a Pablo.
\bibverse{33} Entonces el comandante llegó y arrestó a Pablo, ordenando
que lo ataran con dos cadenas. Entonces preguntó quién era él y qué
había hecho. \bibverse{34} Y algunos gritaban y decían una cosa y otros
decían otra. Y como el comandante no pudo saber la verdad por todo el
ruido y la confusión, ordenó que Pablo fuera llevado a la fortaleza.

\bibverse{35} Cuando Pablo llegó a las escaleras tuvo que ser llevado
por los soldados porque la turba era muy violenta. \bibverse{36} Y la
gente de la multitud que seguía gritaba: ``¡Acaben con él!''
\bibverse{37} Y cuando estaba a punto de ser ingresado a la fortaleza,
Pablo le dijo al comandante: ``¿Puedo decirte algo?''

``¿Sabes griego?'' le preguntó el comandante. \bibverse{38} ``¿Acaso no
eres el egipcio que hace poco incitó una rebelión y condujo a
cuatrocientos asesinos al desierto?''

\bibverse{39} ``Yo soy judío, ciudadano de Tarso, en Cilicia, una ciudad
reconocida,'' respondió Pablo. ``Por favor, déjame hablarle al pueblo.''

\bibverse{40} Entonces el comandante le dio permiso para hablar. Así que
Pablo se puso en pie en las escaleras e hizo señal para que hicieran
silencio. Cuando todo estuvo en silencio, comenzó a hablarles en arameo.

\hypertarget{section-21}{%
\section{22}\label{section-21}}

\bibverse{1} ``Hermanos y padres,'' dijo, ``escuchen, por favor, pues
presentaré ante ustedes mi defensa.'' \bibverse{2} Y cuando lo
escucharon hablando en arameo, todos se quedaron en absoluto silencio.

\bibverse{3} ``Soy judío, nacido en Tarso de Cilicia,'' comenzó. ``Sin
embargo, fui criado en esta ciudad, y me senté a los pies de Gamaliel.
Fui enseñado para guardar de manera estricta la ley de nuestros padres.
Yo era un hombre celoso por Dios, tal como ustedes aquí hoy,
\bibverse{4} y perseguí a las personas de El Camino, mandándolos a matar
y enviándolos a la cárcel, tanto a hombres como a mujeres.

\bibverse{5} ``Y como el sumo sacerdote y el concilio de anciano pueden
verificar, recibí de ellos cartas de autorización dirigidas a los
hermanos judíos en Damasco, y fui allí para arrestar a estas personas y
traerlas como prisioneras a Jerusalén para darles castigo.

\bibverse{6} ``Cerca del mediodía, mientras iba de camino y me acercaba
a Damasco, de repente una luz brillante vino del cielo iluminando todo a
mi alrededor. \bibverse{7} Entonces caí al suelo y escuché una voz
diciéndome: `Saulo, Saulo, ¿por qué me persigues?'

\bibverse{8} ```¿Quién eres, Señor?' respondí.

```Yo soy Jesús de Nazaret, a quien tu persigues,' me dijo.

\bibverse{9} ``Y los que viajaban conmigo vieron la luz, pero no oyeron
la voz que me habló.

\bibverse{10} ```¿Qué debo hacer, Señor?' pregunté.

``Y el Señor me dijo: `Levántate y ve a Damasco, y allí se te dirá todo
lo que ya se ha dispuesto para que hagas.'

\bibverse{11} ``Y como no podía ver por el brillo de la luz, los que
estaban conmigo le llevaron de la mano hasta Damasco. \bibverse{12} Allí
había un hombre llamado Ananías que fue a verme. Era un hombre devoto
que guardaba la ley, y era muy respetado por los judíos que vivían en la
ciudad. \bibverse{13} Se paró frente a mí y me dijo: `Hermano Saulo,
recobra tu vista.' Y en ese momento pude ver nuevamente, y lo miré.

\bibverse{14} ``Entonces me dijo: `El Dios de nuestros padres te ha
designado para que conozcas su voluntad, para que veas a Aquél que es
verdaderamente recto\footnote{\textbf{22:14} Refiriéndose a Jesús.}, y
escuches lo que él quiere decirte. \bibverse{15} Testificarás en su
nombre a todos acerca de lo que has visto y oído. ¿Qué esperas entonces?
\bibverse{16} Levántate, bautízate y lava tus pecados invocando su
nombre.'

\bibverse{17} ``Entonces regresé a Jerusalén, y mientras oraba en el
templo, caí en trance. \bibverse{18} Tuve una visión del Señor
diciéndome: `¡Apresúrate! Debes irte pronto de Jerusalén, porque no
aceptarán lo que estás enseñando acerca de mí.'

\bibverse{19} ``Entonces respondí: `Señor, seguramente ellos saben que
fui de sinagoga en sinagoga, golpeando y enviando a la cárcel a los que
creían en ti. \bibverse{20} Cuando Esteban fue asesinado por testificar
sobre ti, yo estuve allí en pleno acuerdo con los que lo mataron,
sosteniendo sus abrigos.'

\bibverse{21} ``Y el Señor me dijo: `Sal ahora, porque yo te voy a
enviar muy lejos, donde están los extranjeros.'\,''

\bibverse{22} Hasta ese momento la gente había escuchado lo que Pablo
decía, pero entonces comenzaron a gritar: '¡Eliminen a este hombre de la
tierra! ¡No merece vivir!'' \bibverse{23} Y gritaban y rasgaban sus
ropas y lanzaban tierra al aire. \bibverse{24} Entonces el comandante
ordenó que Pablo fuera enviado a la fortaleza, y que fuera interrogado
usando latigazos para descubrir la razón por la cual la gente gritaba
tanto en contra de Pablo. \bibverse{25} Al extenderlo y atarlo para
darle los azotes, Pablo le preguntó al centurión que estaba allí: ``¿Es
legal azotar a un ciudadano romano que no ha sido llevado a juicio?''

\bibverse{26} Cuando el centurión escuchó lo que Pablo dijo, fue hasta
donde estaba el comandante y le preguntó: ``¿Qué estás haciendo? Este
hombre es ciudadano romano.'' \bibverse{27} Entonces el comandante fue y
le preguntó a Pablo: ``Dime, ¿eres ciudadano romano?''

Y Pablo respondió: ``Sí, lo soy.''

\bibverse{28} ``Pagué mucho dinero para comprar la ciudadanía romana,''
dijo el comandante.

``Pero yo nací siendo ciudadano,'' respondió Pablo.

\bibverse{29} Entonces los que estaban a punto de interrogar a Pablo se
fueron de inmediato. Y el comandante estaba preocupado porque le había
puesto cadenas\footnote{\textbf{22:29} Era ilegal encadenar a un
  ciudadano romano si no había sido hallado culpable previamente.}.

\bibverse{30} Al día siguiente, queriendo descubrir la razón por la cual
los judíos acusaban a Pablo, dio orden de que lo liberaran y lo llevaran
ante los jefes de los sacerdotes y ante todo el concilio, al cual
convocó para una reunión. Entonces mandó a traer a Pablo y que lo
pusieran frente a ellos.

\hypertarget{section-22}{%
\section{23}\label{section-22}}

\bibverse{1} Entonces Pablo, mirando al concilio, dijo: ``Hermanos,
hasta ahora siempre me he conducido delante de Dios con una conciencia
limpia.''

\bibverse{2} Y Ananías, el sumo sacerdote, ordenó a los oficiales que
estaban junto a Pablo que lo golpearan en la boca.

\bibverse{3} Entonces Pablo le dijo: ``¡Dios te golpeará, hipócrita! ¡Te
sientas allí para juzgarme conforme a tu ley, pero das orden para que me
golpeen siendo que es infracción de la ley!''

\bibverse{4} Entonces los oficiales que estaban junto a Pablo le
dijeron: ``¿Cómo te atreves a insultar al sumo sacerdote?''

\bibverse{5} ``Hermanos, no sabía que era el sumo sacerdote,'' respondió
Pablo. ``Como dicen las escrituras, `No maldigas al jefe de tu
pueblo.'\,''

\bibverse{6} Y cuando Pablo se dio cuenta de que algunos miembros del
concilio eran Saduceos y otros eran Fariseos, exclamó: ``¡Hermano, yo
soy Fariseo, hijo de un Fariseo! Estoy en este juicio por mi esperanza
en la resurrección de los muertos.''

\bibverse{7} Y cuando dijo esto, se despertó un tremendo debate entre
los Fariseos y los Saduceos que dividió al concilio. \bibverse{8} (Los
Saduceos dicen que no hay resurrección de la muerte, ni ángeles, ni
espíritus; pero los Fariseos sí creen en estas cosas.)

\bibverse{9} Y surgió gran conmoción y uno de los maestros de la ley
Fariseos se puso en pie y argumentó con firmeza: ``¡Consideramos que
este hombre no es culpable! Es posible que un espíritu le haya hablado,
o un ángel.''

\bibverse{10} Y el debate se estaba saliendo de las manos, así que el
comandante, preocupado de que fueran a descuartizar a Pablo, ordenó a
los soldados que fueran a rescatarlo a la fuerza del concilio, y que lo
llevaran de regreso a la fortaleza. \bibverse{11} Después de esto,
durante la noche, el Señor se puso en pie junto a Pablo y le dijo:
``¡Ten valor! Así como has dado testimonio de mí en Jerusalén, así mismo
deberás hacerlo en Roma.''

\bibverse{12} Al día siguiente los judíos organizaron juntos un complot,
e hicieron voto de no comer o beber hasta que hubieran matado a Pablo.
\bibverse{13} Cerca de cuarenta personas hacían parte de esta
conspiración.

\bibverse{14} Y fueron donde los jefes de los sacerdotes y los líderes y
dijeron: ``Hemos tomado un voto solemne de no comer ni beber hasta que
hayamos matado a Pablo. \bibverse{15} Así que ustedes y el concilio
deben enviar la orden al comandante para que traiga a Pablo para
reunirse con ustedes, como si quisieran estudiar su caso más
detalladamente. Estamos listos para matarlo en el camino.''

\bibverse{16} Pero el sobrino de Pablo (el hijo de su hermana) escuchó
sobre esta emboscada que habían planeado, y entró a la fortaleza y le
contó esto a Pablo. \bibverse{17} Entonces Pablo llamó a uno de los
centuriones, y le dijo: ``Lleva a este hombre donde el comandante, pues
tiene información para darle.''

\bibverse{18} Entonces el centurión tomó al sobrino de Pablo y lo llevó
ante el comandante y le dijo: ``El prisionero Pablo me llamó y me pidió
que te trajera a este joven. Tiene algo que decirte.'' \bibverse{19}
Entonces el comandante tomó al joven de la mano y lo llevó aparte.
``¿Qué tienes que decirme?'' le preguntó en voz baja.

\bibverse{20} ``Los judíos han hecho un plan para pedirte que lleves a
Pablo ante el concilio mañana como si quisieran hacer preguntas más
detalladas sobre su caso,'' le explicó. \bibverse{21} ``Por favor, no
les hagas caso, porque han planeado una emboscada con más de cuarenta
hombres que han hecho un voto para no comer ni beber hasta que lo hayan
matado. Ya están listos, esperando que tú aceptes la petición.''

\bibverse{22} Entonces el comandante envió al joven de camino,
advirtiéndole: ``No le digas a nadie que me has dicho esto.''
\bibverse{23} Y llamó a dos centuriones y les dijo: ``Alisten a
doscientos soldados para ir a Cesarea, junto con setenta hombres a
caballo y doscientos hombres con lanzas. Estén listos para salir esta
noche a las nueve. \bibverse{24} Preparen caballos para Pablo, de tal
modo que llegue con seguridad hasta donde el Gobernador Félix.''

\bibverse{25} Además escribió una carta que decía así:

\bibverse{26} De Claudio de Lisiasa Su Excelencia, el Gobernador Félix.
Saludos. \bibverse{27} Este hombre fue tomado por los judíos y estaban a
punto de matarlo cuando llegué a la escena con soldados y lo rescatamos,
porque he sabido que es ciudadano romano. \bibverse{28} Quise saber la
razón de sus acusaciones, y por eso lo llevé ante el concilio.
\bibverse{29} Descubrí que los cargos presentados contra él están
relacionados con la ley de los judíos, pero él no era culpable de
ninguna cosa que amerite muerte o prisión. \bibverse{30} Cuando supe que
había un complot organizado contra este hombre lo envié a ti de
inmediato, dando orden a los acusadores de que presenten sus quejas
delante de ti.''

\bibverse{31} Entonces los soldados siguieron sus órdenes y llevaron a
Pablo durante la noche a Antípatris. \bibverse{32} A la mañana siguiente
lo enviaron con la caballería, y regresaron a la fortaleza.

\bibverse{33} Cuando la caballería llegó a Cesarea, entregaron la carta
al gobernador y presentaron a Pablo delante de él. \bibverse{34} Después
de leer la carta, el gobernador le preguntó a Pablo de qué provincia
venía. Y cuando supo que era de Cilicia, le dijo: \bibverse{35}
``Investigaré tu caso cuando lleguen tus acusadores.'' Y ordenó que
mantuvieran a Pablo detenido en el palacio de Herodes.

\hypertarget{section-23}{%
\section{24}\label{section-23}}

\bibverse{1} Cinco días más tarde, Ananías, el sumo sacerdote, llegó con
algunos de los líderes judíos, y con un abogado llamado Tértulo. Y
presentaron acusaciones formales contra Pablo ante el gobernador.
\bibverse{2} Y cuando Pablo fue llamado, Tértulo comenzó a presentar el
caso contra él. Y dijo: ``Su Excelencia Gobernador Félix, hemos
disfrutado de un largo periodo en paz bajo su gobierno, y como resultado
de su sabio juicio se han hecho reformas para el beneficio de la nación.
\bibverse{3} Todos en el país estamos muy agradecidos por esto.

\bibverse{4} ``Pero para no cansarlo, por favor sea amable en prestarnos
atención por un momento. \bibverse{5} Nos dimos cuenta de que este
hombre es una verdadera peste\footnote{\textbf{24:5} Literalmente,
  ``plaga.''}, levantando rebeliones entre los judíos, y es cabecilla de
la secta nazarena. \bibverse{6} Trató de contaminar el templo y por eso
lo arrestamos. \bibverse{7} \footnote{\textbf{24:7} 24:6b-8a. Existen
  dudas respecto a la originalidad de estos versículos y han sido
  omitidos del texto griego estándar.} \bibverse{8} Al interrogarlo,
usted mismo descubrirá la verdad de nuestras acusaciones.'' \bibverse{9}
Y los judíos se le unieron, diciendo que todo esto era verdad.

\bibverse{10} Entonces el gobernador hizo señas a Pablo para que
respondiera. ``Reconociendo que usted ha sido juez de esta nación
durante muchos años, gustosamente presentaré mi defensa,'' comenzó
Pablo. \bibverse{11} ``Usted puede verificar fácilmente que yo llegué a
Jerusalén para adorar hace apenas doce años. \bibverse{12} Nadie me
encontró nunca discutiendo en el templo con nadie, o incitando a la
gente a amotinarse en ninguna sinagoga o en ninguna otra parte de la
ciudad. \bibverse{13} Tampoco pueden probarle a usted ninguna de sus
demás acusaciones contra mí.

\bibverse{14} ``Pero le admitiré esto: Sirvo al Dios de nuestros padres,
siguiendo las creencias de El Camino, a lo que ellos llaman una secta
hereje. Yo creo en todo lo que la ley enseña y lo que está escrito en
los libros de los profetas. \bibverse{15} Tengo la misma esperanza en
Dios que ellos tienen, creyendo que habrá una resurrección de los buenos
y de los malvados. \bibverse{16} Por ello trato de asegurarme de tener
siempre una conciencia tranquila ante Dios y ante todos.

\bibverse{17} ``Después de haber estado lejos durante unos años, regresé
para traer dinero para ayudar a los pobres y para dar ofrendas a Dios.
\bibverse{18} Eso fue lo que me encontraron haciendo, culminando la
ceremonia de purificación. Y no había ninguna multitud o disturbio.
\bibverse{19} Pero ciertos judíos de la provincia de Asia estaban allí,
y que deberían estar aquí presentes ante usted para presentar sus cargos
contra mí, si es que tienen alguno. \bibverse{20} De lo contrario,
permita que estos hombres expliquen de qué crimen me hallaron culpable
cuando estuve ante el concilio, \bibverse{21} excepto el hecho de que
exclamé: `Estoy bajo juicio hoy porque creo en la resurrección de los
muertos.'

\bibverse{22} Entonces Félix, quien estaba bien informado sobre el
Camino, aplazó el juicio. ``Cuando el comandante Lisias venga, entonces
tomaré mi decisión respecto a tu caso,'' dijo. \bibverse{23} Entonces
ordenó al centurión que mantuviera a Pablo bajo custodia pero que le
permitiera tener algunas medidas de libertad y que dejara que sus amigos
cuidaran de él sin restricciones.

\bibverse{24} Algunos días después, regresó junto a su esposa Drusila,
quien era judía. Y envió a buscar a Pablo y lo escuchó hablar sobre la
fe en Jesucristo. \bibverse{25} Y Pablo debatió con ellos sobre vivir
rectamente, sobre el dominio propio, y sobre el juicio venidero. Félix
se intranquilizó y le dijo a Pablo: ``Vete ahora, y enviaré a buscarte
cuando tenga la oportunidad.'' \bibverse{26} Esperando que Pablo lo
sobornara con dinero, a menudo Félix mandaba a buscar a Pablo y hablaba
con él. \bibverse{27} Pasaron dos años y Félix fue sucedido por Porcio
Festo. Pero para mantener el favor de los judíos, Félix dejó a Pablo en
la cárcel.

\hypertarget{section-24}{%
\section{25}\label{section-24}}

\bibverse{1} Ocurrió que tres días después de que Festo había llegado a
la provincia,\footnote{\textbf{25:1} La provincial romana de Judea.} se
marchó de Cesarea para ir a Jerusalén. \bibverse{2} Los jefes de los
sacerdotes y los líderes judíos vinieron ante él y presentaron sus
cargos contra Pablo. \bibverse{3} Le rogaron a Festo que como favor
enviara a Pablo a Jerusalén, conspirando para hacerle una emboscada y
matarlo en el camino. \bibverse{4} Pero Festo respondió que Pablo estaba
bajo custodia en Cesarea y que él mismo estaría allá dentro de poco
tiempo. \bibverse{5} ``Sus líderes pueden venir conmigo, y presentar sus
acusaciones contra este hombre, si es que ha hecho algo malo,'' les
dijo.

\bibverse{6} Después de estar allí con ellos durante cerca de ocho o
diez días, Festo regresó a Cesarea. Al día siguiente, tomó su asiento
como juez, y ordenó que Pablo fuera traído delante de él. \bibverse{7}
Cuando Pablo entró, los judíos que habían ido desde Jerusalén lo
rodearon y presentaron acusaciones serias contra él, pero que no podían
probar.

\bibverse{8} Pablo se defendió, diciéndoles: ``No he pecado en absoluto
contra la ley judía, ni contra el templo ni contra el César.''
\bibverse{9} Pero Festo, quien buscaba el favor de los judíos, le
preguntó a Pablo: ``¿Estás dispuesto a ir a Jerusalén y ser juzgado ante
mí allí sobre estos asuntos?''

\bibverse{10} ``Yo estoy ante la corte del César para ser juzgado, justo
donde debería estar,'' respondió Pablo. ``No le he hecho nada malo a los
judíos, como bien lo saben. \bibverse{11} Y si he cometido algo que
merezca la muerte, no pido ser perdonado de la sentencia de muerte. Pero
si no hay pruebas para estas acusaciones que ellos hacen contra mí,
entonces nadie tiene derecho a entregarme a ellos. ¡Yo apelo al César!''
\bibverse{12} Entonces Festo deliberó con el concilio, y respondió:
``¡Has apelado al César y al César irás!''

\bibverse{13} Varios días después, el Rey Agripa y su hermana Berenice
llegaron a Cesarea para rendir honores a Festo. \bibverse{14} Y estaban
quedándose por un tiempo, así que Festo le presentó al rey el caso de
Pablo, explicando: ``Hay un hombre que Félix dejó aquí como prisionero.
\bibverse{15} Cuando fui a Jerusalén, los jefes de los sacerdotes judíos
y líderes vinieron y presentaron cargos contra él y me pidieron dar
sentencia. \bibverse{16} Yo respondí que conforme a la ley romana no
está permitido condenar a alguien sin dejarle ver la cara de sus
acusadores y debe dársele la oportunidad de defenderse de sus cargos.
\bibverse{17} Así que cuando llegaron sus acusadores, no dejé pasar
mucho tiempo sino que convoqué el juicio para el día siguiente. Y di
orden para que trajeran a este hombre. \bibverse{18} Sin embargo, cuando
los acusadores se levantaron, no presentaron cargos de acciones
criminales, como yo esperaba. \bibverse{19} En lugar de ello surgieron
controversias sobre asuntos religiosos, y sobre un hombre llamado Jesús,
que estaba muerto, pero Pablo insistía en que estaba vivo todavía.
\bibverse{20} Como yo estaba indeciso respecto a cómo proceder en la
investigación de tales asuntos, le pregunté si estaba dispuesto a ir a
Jerusalén para ser juzgado allí. \bibverse{21} No obstante, Pablo apeló
por su caso para que fuera escuchado por el emperador, así que di orden
de que fuera detenido hasta que pudiera enviarlo al César.''

\bibverse{22} ``Me gustaría escuchar yo mismo a este hombre,'' dijo
Agripa a Festo.

``Haré los arreglos para que lo escuches mañana,'' respondió Festo.

\bibverse{23} Al día siguiente, Agripa llegó con Berenice con gran
esplendor ceremonial y entraron al auditorio con los comandantes y
principales ciudadanos. Entonces Festo dio orden para que trajeran a
Pablo.

\bibverse{24} ``Rey Agripa, y todos los que están presentes aquí con
nosotros,'' comenzó Festo, ``ustedes ven que delante de ustedes está
este hombre, de quien todo el pueblo judío, tanto aquí como en
Jerusalén, se han quejado ante mí, gritando que no debería seguir con
vida. \bibverse{25} Sin embargo, descubrí que él no ha cometido ningún
crimen que merezca la muerte, y como él ha apelado al emperador, decidí
enviarlo allí. \bibverse{26} Pero no tengo nada específico que escribir
sobre él a Su Majestad Imperial. Por ello lo he traído aquí delante de
ustedes para poder tener algo concreto que escribir. \bibverse{27} No me
parece justo enviar a un prisionero sin explicar los cargos presentados
en su contra.''

\hypertarget{section-25}{%
\section{26}\label{section-25}}

\bibverse{1} Entonces Agripa le dijo a Pablo: ``Eres libre para hablar a
tu favor.''

Haciendo un gesto con su brazo, Pablo comenzó su defensa. \bibverse{2}
``Estoy complacido, Rey Agripa, de presentar mi defensa ante usted hoy,
respecto a todas las cosas de las que soy acusado por los judíos,
\bibverse{3} especialmente porque usted es un experto en todos los
asuntos y costumbres judías. Le ruego su paciente atención al escuchar
lo que tengo que decir.''

\bibverse{4} ``Todos los judíos conocen la historia de mi vida, desde
mis primeros días en mi propio país y luego en Jerusalén. \bibverse{5}
Me han conocido por mucho tiempo y pueden verificar, si eligen hacerlo,
que he seguido la escuela religiosa que observa nuestra fe, de la manera
más estricta. Pues vivía como Fariseo.

\bibverse{6} ``Ahora estoy aquí para ser juzgado respecto a la esperanza
prometida que Dios dio a nuestros padres, \bibverse{7} que nuestras doce
tribus esperaban recibir si se consagraban al servicio de Dios. ¡Sí, es
por esta esperanza que soy acusado por los judíos, Su Majestad!
\bibverse{8} ``¿Por qué pensaría alguno de ustedes que es increíble que
Dios resucite a los muertos?

\bibverse{9} Anteriormente estaba convencido con sinceridad de que debía
ser todo lo posible para oponerme al nombre de Jesús de Nazaret.
\bibverse{10} Eso es lo que hacía en Jerusalén. Puse a muchos de los
creyentes en la cárcel, habiendo recibido la autoridad para hacer esto
de parte de los jefes de los sacerdotes. Cuando fueron sentenciados a
muerte hice mi voto en contra de ellos. \bibverse{11} Los mandé a
castigar en todas las sinagogas, tratando de hacer que se retractaran. Y
me opuse a ellos con tanta furia que fui a las ciudades que están fuera
de mi país para perseguirlos.

\bibverse{12} ``Esa es la razón por la que un día yo iba de camino a
Damasco con autoridad y órdenes de los jefes de los sacerdotes.
\bibverse{13} Y cerca de la hora del mediodía, Su Majestad, vi una luz
que venía del cielo y era más brillante que el sol. Iluminó todo a mi
alrededor y a los que iban viajando conmigo. \bibverse{14} Todos caímos
al suelo. Entonces escuché una voz que me hablaba en idioma arameo:
`Saulo, Saulo, ¿por qué me persigues? ¡Es duro para ti pelear contra
mí\footnote{\textbf{26:14} Literalmente, ``dar coces contra el
  aguijón''---la imagen de aguijones que se usaban para guiar al ganado.}!'

\bibverse{15} ```¿Quién eres, Señor?' pregunté.

```Yo soy Jesús, a quien tu persigues,' respondió el Señor.
\bibverse{16} `Pero levántate y ponte de pie. La razón por la que he
aparecido ante ti es para designarte como mi siervo, para que seas mi
testigo, contando a otros lo que has visto y todo lo que voy a
revelarte. \bibverse{17} Te salvaré de tu propio pueblo y de los
extranjeros. Yo te envío a ellos \bibverse{18} para abrir sus ojos y que
así puedan volverse de las tinieblas a la luz, del poder de Satanás
hacia Dios, y que así reciban perdón por sus pecados y un lugar con
aquellos que son justificados por creer en mí.'

\bibverse{19} ``Claramente, Rey Agripa, no podía desobedecer esta visión
del cielo. \bibverse{20} Primero en Damasco, luego en Jerusalén y luego
en toda Judea y también a los extranjeros les prediqué el mensaje de
arrepentimiento: cómo deben volverse a Dios, demostrando su
arrepentimiento por medio de sus acciones. \bibverse{21} Por eso los
judíos me agarraron en el templo y trataron de matarme.

\bibverse{22} ``Dios ha cuidado de mi para que hoy pueda estar aquí como
testigo para todos, tanto para las personas comunes como para las
personas más prestigiosas. Yo solo estoy repitiendo lo que Moisés y los
profetas dijeron que sucedería: \bibverse{23} cómo sufriría el Mesías, y
que al resucitar de los muertos él anunciaría la luz de la salvación de
Dios\footnote{\textbf{26:23} Implícito. El original dice simplemente
  ``luz.''} tanto para los judíos como para los extranjeros.''

\bibverse{24} Entonces Festo interrumpió a Pablo mientras presentaba su
defensa, exclamando: ``¡Pablo, te has vuelto loco! ¡Todo tu conocimiento
te está llevando a la locura!''

\bibverse{25} ``No estoy loco, Su Excelencia Festo,'' respondió Pablo.
``Lo que estoy diciendo es verdad y tiene sentido. \bibverse{26} El rey
reconoce esto, y lo estoy explicando de manera muy clara. Estoy seguro
de que él sabe lo que está sucediendo, porque ninguna de estas cosas ha
sucedido en secreto.

\bibverse{27} ``Rey Agripa, ¿cree usted en lo que dijeron los profetas?
¡Estoy seguro que sí!''

\bibverse{28} ``¿Crees que puedes convencerme para convertirme en
cristiano tan rápidamente?'' le preguntó Agripa a Pablo.

\bibverse{29} ``No importa si toma poco o mucho tiempo,'' respondió
Pablo. ``Pero mi oración a Dios es que no solo usted, sino todos los que
me escuchan se vuelvan como yo, excepto por estas cadenas.''

\bibverse{30} Entonces el rey se levantó, junto con el gobernador y
Berenice, y todos los que estaban sentados con él. \bibverse{31} Y
deliberaron juntos después de que Pablo había salido de allí. ``Este
hombre no ha hecho nada que merezca la muerte o la cárcel,''
concluyeron. \bibverse{32} Entonces Agripa le dijo a Festo: ``Podría
haber quedado libre si no hubiera apelado al César.''

\hypertarget{section-26}{%
\section{27}\label{section-26}}

\bibverse{1} Cuando llegó nuestro momento de zarpar a Italia, Pablo y
algunos otros prisioneros fueron entregados a un centurión llamado
Julio, que pertenecía al Régimen Imperial. \bibverse{2} Nos embarcamos
en un barco que estaba registrado en Adramitio y que se dirigía hacia
los puertos costeros de la provincia de Asia, y comenzamos a navegar.
Aristarco, un hombre de Tesalónica, Macedonia, iba con nosotros.
\bibverse{3} Al día siguiente, hicimos una breve pausa en Sidón, y
Julio, con mucha amabilidad, permitió que Pablo saliera del barco y
visitara a sus amigos para que pudieran darnos provisiones necesarias.

\bibverse{4} Luego partimos de allí y navegamos protegidos por Chipre
porque el viento venía de manera contraria. \bibverse{5} Entonces
navegamos directamente por mar abierto hasta la costa de Cilicia y
Panfilia, llegando al Puerto de Mira en Licia. \bibverse{6} Allí el
centurión encontró un barco que iba hacia Italia, e hizo los arreglos
para que nos fuéramos en él.

\bibverse{7} Navegamos lentamente durante varios días y finalmente
llegamos a Gnido. Pero como los vientos no nos permitían seguir,
navegamos al amparo de Creta, cerca de Salmona. \bibverse{8} Pasamos por
toda la costa con dificultad hasta que llegamos a un lugar llamado
Buenos Puertos, cerca de la ciudad de Lasea. \bibverse{9} Habíamos
perdido mucho tiempo, y el viaje se hacía peligroso porque ya había
pasado la celebración del Ayuno\footnote{\textbf{27:9} ``El Ayuno'': El
  Día de la Expiación, probablemente celebrado en octubre, por lo cual
  navegar en esta temporada podía ser peligroso.}. Pablo les advirtió:
\bibverse{10} ``Señores, puedo ver que este viaje traerá adversidades y
pérdidas, no solo de la carga sino también de nuestras propias vidas.''
\bibverse{11} Pero el centurión prestó más atención al consejo del
capitán del barco y de su dueño que a lo que dijo Pablo.

\bibverse{12} Y como el Puerto no era suficientemente grande para para
el invierno, la mayoría estuvieron a favor de que nos fuéramos e
hiciéramos lo posible por llegar a pasar el invierno en Fenice, un
puerto que está en Creta, y que da de frente con el noreste y el
sureste.

\bibverse{13} Y cuando empezó a soplar un viento moderado, pensaron que
podían hacer lo que habían planeado. Entonces elevaron el ancla y
navegaron por la orilla a lo largo de la costa de Creta. \bibverse{14}
Pero no pasó mucho tiempo cuando de la tierra comenzó a soplar un viento
como de huracán, llamado ``nordeste.'' \bibverse{15} Entonces el barco
quedó atrapado en el mar y no podía hacerle frente al viento. Así que
tuvimos que desistir y dejarnos llevar por el viento. \bibverse{16}
Finalmente pudimos entrar al abrigo de un islote llamado Cauda, y con
dificultad pudimos sujetar a bordo el bote salvavidas del
barco\footnote{\textbf{27:16} ``Bote salvavidas''--- pequeño bote
  similar a un bote inflable o salvavidas, que en ocasiones era
  remolcado por detrás de un barco, y otras veces estaba atado desde la
  cubierta. Ver también versículo 30.}. \bibverse{17} Después de subirlo
a bordo, amarraron cuerdas alrededor del casco del barco para
reforzarlo. Luego, preocupados de que pudiera romperse en los bancos de
arena de Sirte, bajaron el ancla flotante y dejaron el barco a la
deriva.

\bibverse{18} Al día siguiente, como la tempestad arremetía con mucha
fuerza contra nosotros, la tripulación comenzó a lanzar por la borda la
carga que llevaba el barco. \bibverse{19} El tercer día con sus propias
manos tomaron el engranaje del barco y lo lanzaron al mar. \bibverse{20}
Y no habíamos visto el sol ni las estrellas durante muchos días mientras
nos golpeaba la tormenta, así que habíamos perdido toda esperanza de ser
salvados.

\bibverse{21} Y ninguno había comido nada por mucho tiempo. Entonces
Pablo se puso en pie delante de la tripulación y les dijo: ``Señores,
debieron haberme prestado atención y no partir de Creta. Así hubieran
evitado todo este apuro y pérdida. \bibverse{22} Pero ahora les aconsejo
que mantengan el valor, porque nadie se perderá, sino solo el barco.
\bibverse{23} Anoche un ángel de mi Dios\footnote{\textbf{27:23}
  Literalmente, ``el Dios al cual pertenezco.''} y al cual sirvo, se
puso en pie junto a mí.

\bibverse{24} ``\,`No tengas miedo, Pablo,' me dijo. `Debes ir a juicio
ante el César. Mira, por su gracia Dios te ha dado a todos los que
navegan contigo.' \bibverse{25} ¡Así que tengan valor! Yo creo en Dios y
estoy convencido de que las cosas pasarán tal como se las he dicho.
\bibverse{26} Sin embargo, vamos a naufragar en alguna isla.''

\bibverse{27} Cuando era cerca de la media noche, durante la
decimocuarta noche de tormenta, y mientras aún éramos arrastrados por el
Mar Adriático, la tripulación presintió que se acercaban a tierra.
\bibverse{28} Entonces revisaron la profundidad y se dieron cuenta que
era de cuarenta metros, y un poco más adelante volvieron a revisar y era
de treinta metros.

\bibverse{29} Y estaban preocupados de que pudiéramos chocar contra las
piedras, así que lanzamos anclas desde la popa, y oramos para que
pudiera salir la luz del día.

\bibverse{30} La tripulación trató de abandonar el barco, y ya habían
bajado el bote salvavidas al agua con el pretexto de que iban a lanzar
anclas desde la proa. \bibverse{31} Pero Pablo le dijo al centurión y a
los soldados: ``Si la tripulación no permanece en el barco, perecerá.''
\bibverse{32} Así que los soldados cortaron las cuerdas que sostenían el
bote salvavidas y lo dejaron suelto.

\bibverse{33} En la madrugada, Pablo exhortó a todos para que comieran
algo: ``Han pasado catorce días y no han comido nada porque han estado
muy ocupados y angustiados,'' les dijo. \bibverse{34} ``Por favor, hagan
lo que les digo y coman algo. Eso les ayudará a tener fuerzas. Porque no
se perderá ni un cabello de sus cabezas.'' \bibverse{35} Y cuando
terminó de hablar, tomó un trozo de pan y dio gracias a Dios por él
delante de todos. Luego lo partió y comenzó a comer. \bibverse{36} Y
todos se sintieron animados y comieron también. \bibverse{37} El número
total de personas a bordo era de doscientas setenta y seis.

\bibverse{38} Cuando quedaron saciados, la tripulación disminuyó el peso
del barco lanzando las provisiones de trigo por la borda. \bibverse{39}
Cuando llegó la mañana no reconocieron la costa, pero vieron una bahía
que tenía playa. Entonces hicieron el plan para tratar de encallar el
barco allí. \bibverse{40} Así que cortaron las cuerdas que sostenían las
anclas, y las dejaron en el mar. Al mismo tiempo desataron las cuerdas
que sostenían los timones, elevaron el trinquete al viento, y llegaron a
la playa.

\bibverse{41} Pero el barco chocó contra un banco de arena y encalló
allí. La proa chocó y quedó atascada con tanta fuerza que no podía
moverse, mientras que la popa comenzó a romperse por culpa del embate de
las olas.

\bibverse{42} Los soldados planeaban matar a los prisioneros para que
ninguno pudiera nadar y escaparse. \bibverse{43} Pero como el centurión
quería salvar la vida de Pablo, les advirtió que no lo hicieran, y dio
orden para que los que pudieran nadar se lanzaran del barco primero y
llegaran a tierra. \bibverse{44} El resto se agarró de tablas y restos
del barco, para que así todos pudieran llegar a tierra a salvo.

\hypertarget{section-27}{%
\section{28}\label{section-27}}

\bibverse{1} Cuando estuvimos a salvo en la orilla, supimos que
estábamos en la isla de Malta. \bibverse{2} La gente de allí era muy
amable, y encendieron una fogata y nos llamaron para que pudiéramos
estar abrigados de la lluvia y el frío. \bibverse{3} Pablo recogió un
atado de leña y la lanzó al fuego. Pero de la leña salió una serpiente
venenosa por causa el calor, y picó a Pablo, enroscándose en su mano.
\bibverse{4} Cuando la gente que estaba allí vio la serpiente colgando
de su mano, se dijeron unos a otros: ``Este hombre debe ser un asesino.
Aunque escapó de la muerte en el mar, la justicia no lo dejará vivo.''

\bibverse{5} Sin embargo, Pablo sacudió la serpiente al fuego y no
sufrió ningún daño. \bibverse{6} Y todos estaban esperando que sufriera
hinchazón, o que cayera muerto repentinamente. Pero tras esperar largo
rato, vieron que nada malo le ocurría, así que cambiaron de opinión y
decidieron creer que quizás él era un Dios.

\bibverse{7} Y cerca de allí había tierras que pertenecían a Publio, el
funcionario principal de la isla. Él nos recibió y cuidó de nosotros
durante tres días con mucha hospitalidad. \bibverse{8} Pero el padre de
Publio estaba enfermo, acostado en una cama y sufría con fiebre y
disentería. Entonces Pablo entró a verlo, y oró por él, puso sus manos
sobre él y lo sanó. \bibverse{9} Después que sucedió esto, todos los
demás enfermos de la isla venían y eran sanados. \bibverse{10} Entonces
nos dieron muchos regalos, y cuando tuvimos que marcharnos nos dieron
provisiones necesarias para el viaje.

\bibverse{11} Después de permanecer allí tres meses zarpamos en un barco
de Alejandría que tenía por insignia a los Gemelos
Celestiales\footnote{\textbf{28:11} Recibía este nombre por los dioses
  gemelos Castor y Pólux.} que había pasado el invierno en la isla.
\bibverse{12} Nos detuvimos en Siracusa, y pasamos allí tres días.
\bibverse{13} De allí navegamos hacia Regio. Al día siguiente sopló un
viento del sur, y el segundo día llegamos al Puerto de Poteoli,
\bibverse{14} donde encontramos algunos creyentes. Y nos pidieron
permanecer con ellos por una semana.

Así que fuimos a Roma. \bibverse{15} Y cuando ciertos creyentes de roma
oyeron que habíamos llegado, fueron a encontrarse con nosotros en Foro
de Apio y las Tres Tabernas. Y cuando Pablo los vio, agradeció a Dios y
se sintió animado. \bibverse{16} Al llegar a Roma, a Pablo se le
permitió permanecer bajo arresto domiciliario con un soldado que lo
custodiaba.

\bibverse{17} Tres días después, Pablo invitó a los líderes judíos para
que fueran a verlo. Y cuando estaban reunidos les dijo: ``Hermanos,
aunque no tengo nada en contra del pueblo o de las costumbres de
nuestros antepasados, fui arrestado en Jerusalén y entregado a las
autoridades romanas. \bibverse{18} Después de interrogarme querían
dejarme en libertad porque yo no había hecho nada que ameritara mi
ejecución. \bibverse{19} Pero los líderes judíos se opusieron a esto, y
fui obligado a apelar al César, y no porque tuviera alguna acusación
contra mi propio pueblo. \bibverse{20} Es por eso que pedí verlos y
hablar con ustedes, porque es por la esperanza de Israel que estoy
encadenado de esta manera.''

\bibverse{21} ``Nosotros no hemos recibido ninguna carta de Judea
respecto a ti, ni ninguna persona en nuestro pueblo ha traído informes
contra ti, ni han dicho algo malo de ti,'' le dijeron. \bibverse{22}
``Pero queremos oír de ti lo que crees, especialmente respecto a esta
secta, que sabemos que está condenada en todos lados.''

\bibverse{23} Entonces concertaron una cita para reunirse con él. Y ese
día muchos fueron al lugar donde él estaba. Y Pablo les enseñaba desde
la mañana hasta la noche, hablándoles sobre Jesús y sobre el reino de
Dios. Trataba de convencerlos acerca de Jesús, usando los escritos de la
ley de Moisés y los profetas. \bibverse{24} Algunos aceptaron lo que
Pablo decía, pero otros se negaron a creer. \bibverse{25} Y no podían
ponerse de acuerdo entre ellos, y se marcharon cuando Pablo les dijo
esto: ``El Espíritu Santo lo dijo bien a través del profeta Isaías, el
profeta de sus antepasados, \bibverse{26} `Ve a este pueblo y dile:
``Aunque ustedes oigan, nunca entenderán, y aunque vean, nunca
comprenderán. \bibverse{27} Porque el corazón de este pueblo se ha
vuelto insensible; se les han embotado los oídos, y se les han cerrado
los ojos. De lo contrario, verían con los ojos, oirían con los oídos,
entenderían con el corazón y se convertirían, y yo los sanaría.''\,'

\bibverse{28} ``Por lo tanto, sepan que esta salvación que viene de Dios
ha sido enviada a los extranjeros y ellos escucharán.'' \bibverse{29}
\footnote{\textbf{28:29} Este versículo no está en todos los
  manuscritos, y algunos comentaristas creen que pudo haber sido
  añadido.}

\bibverse{30} Y durante dos años completos Pablo permaneció allí en la
casa que alquiló, recibiendo a todos los que iban a verlo. \bibverse{31}
Y les hablaba del reino de Dios, y enseñaba sobre el Señor Jesucristo
audazmente. Y nadie se lo impedía.
