\hypertarget{section}{%
\section{1}\label{section}}

\bibverse{1} Después de la muerte de Josué, los israelitas preguntaron
al Señor: ``¿Qué tribu de entre nosotros debe ir primero a atacar a los
cananeos?''

\bibverse{2} ``Judá debe ir primero'', respondió el Señor. ``Les he
entregado la tierra''.

\bibverse{3} Los hombres de Judá dijeron a sus parientes de la tribu de
Simeón: ``Vengan con nosotros a la tierra que nos ha sido asignada y
luchen juntos con nosotros contra los cananeos. Entonces haremos lo
mismo con ustedes y con la tierra que les fue asignada.''\footnote{\textbf{1:3}
  Las asignaciones de tierra estaban al lado de las otras.}Así que la
tribu de Simeón se unió a ellos.

\bibverse{4} Los hombres de Judá atacaron a los cananeos y a los
ferezeos, y el Señor los entregó derrotados. Mataron a diez mil enemigos
en la ciudad de Bezec. \bibverse{5} Allí se enfrentaron a
AdoníBezec\footnote{\textbf{1:5} Significa ``el señor de Bezek.''}y
lucharon con él, derrotando a los cananeos y a los ferezeos.
\bibverse{6} AdoníBezec huyó, pero ellos lo persiguieron y lo
capturaron, y luego le cortaron los pulgares y los dedos gordos de los
pies.

\bibverse{7} AdoníBezec dijo: ``Tuve setenta reyes con los pulgares y
los dedos gordos de los pies cortados recogiendo las sobras de debajo de
mi mesa. Ahora Dios me ha devuelto de la misma manera lo que les hice a
ellos''. Lo llevaron a Jerusalén, donde murió.

\bibverse{8} Los hombres de Judá atacaron Jerusalén y la
conquistaron.\footnote{\textbf{1:8} Es evidente que no se trataba de una
  conquista permanente, ya que David tuvo que tomar Jerusalén varios
  siglos después.}Mataron a los habitantes a espada y quemaron la
ciudad. \bibverse{9} Después de esto, los hombres de Judá fueron a
luchar contra los cananeos que vivían en la región montañosa, en el
Néguev y en las estribaciones de las tierras bajas. \bibverse{10}
Atacaron a los cananeos que vivían en Hebrón (antes conocida como
Quiriat Arba) y derrotaron a Sesay, Ajimán y Talmai.

\bibverse{11} De allí pasaron a atacar a los que vivían en Debir (antes
conocida como QuiriatSefer). \bibverse{12} Caleb anunció: ``Daré a mi
hija Acsa en matrimonio al que ataque y capture QuiriatSefer''.
\bibverse{13} Otoniel, hijo de Quenaz, hermano menor de Caleb, fue quien
la capturó, así que le dio a su hija Acsa en matrimonio.

\bibverse{14} CuandoAcsa se acercó a Otoniel,
laconvención\footnote{\textbf{1:14} Texto hebreo. Algunas versiones de
  la Septuaginta dicen: ``La animó.''}para que le pidiera un terreno a
su padre. Cuando ella se bajó del asno, Caleb le preguntó: ``¿Qué
quieres?''

\bibverse{15} ``Por favor, dame una bendición,''\footnote{\textbf{1:15}
  ``Bendición'': se refiere a la costumbre del padre de la novia de dar
  a su hija una bendición especial con motivo de su matrimonio.}respondió
ella. ``Me diste una tierra que es como el desierto, así que por favor
dame también manantiales de agua''. Así que Caleb le dio los manantiales
superiores e inferiores.

\bibverse{16} Los descendientes del suegro de Moisés, el ceneo, fueron
con el pueblo de Judá desde la ciudad de las palmeras hasta el desierto
de Judá, en el Néguev, cerca de Arad, donde se establecieron entre el
pueblo.

\bibverse{17} Entonces Judá se unió a Simeón y derrotó a los cananeos
que vivían en Zefat. Destruyeron completamente la ciudad, por lo que la
llamaron Horma\footnote{\textbf{1:17} ``Horma'': significa ``dedicado a
  la destrucción.''} \bibverse{18} Judá también capturó las ciudades de
Gaza, Ascalón y Ecrón, cada una con su territorio circundante.

\bibverse{19} El Señor estaba con Judá, y se apoderaron de la región
montañosa, pero no pudieron expulsar a los habitantes de la llanura
porque tenían carros de hierro.

\bibverse{20} Como Moisés había estipulado, Hebrón fue entregada a
Caleb, quien expulsó de ella a los descendientes de tres hijos de Anac.
\bibverse{21} Sin embargo, Benjamín no pudo expulsar a los jebuseos, los
habitantes de Jerusalén, por lo que los jebuseos viven entre el pueblo
de Benjamín en Jerusalén hasta el día de hoy.\footnote{\textbf{1:21}
  Véase también Josué 15:63, donde Judá fue igualmente incapaz de tomar
  Jerusalén.}

\bibverse{22} Los descendientes de José\footnote{\textbf{1:22} Significa
  la tribu de Efraín y la media tribu de Manasés.}fueron y atacaron la
ciudad de Betel, y el Señor estaba con ellos. \bibverse{23} Enviaron
espías a investigar Betel, que antes se llamaba Luz. \bibverse{24} Los
espías vieron a un hombre que salía de la ciudad y le dijeron: ``Por
favor, muéstranos cómo entrar en la ciudad, y te trataremos bien.''

\bibverse{25} El hombre les mostró el camino para entrar en la ciudad, y
mataron a todos los habitantes, excepto al hombre y a su familia, a
quienes dejaron ir. \bibverse{26} El hombre se trasladó al país de los
hititas, construyó allí una ciudad y la llamó Luz, que es su nombre
hasta hoy. \bibverse{27} Sin embargo, Manasés no expulsó a los
habitantes de las ciudades de BetSeán, Taanac, Dor, Ibleam, Meguido y
sus aldeas circundantes porque los cananeos insistieron en vivir en la
tierra. \bibverse{28} Cuando los israelitas se hicieron más fuertes,
obligaron a los cananeos a realizar trabajos forzados, pero nunca los
expulsaron del todo.

\bibverse{29} Efraín no expulsó a los cananeos que vivían en la ciudad
de Gezer, así que los cananeos siguieron viviendo allí entre ellos.

\bibverse{30} Zabulón no expulsó a los habitantes de las ciudades de
Quitrón y Nalol, por lo que los cananeos siguieron viviendo entre ellos.
Sin embargo, los cananeos fueron obligados a realizar trabajos forzados
para el pueblo de Zabulón.

\bibverse{31} Aser no expulsó a la gente que vivía en las ciudades de
Aco, Sidón, Ajlab, Aczib, Jelba, Afec y Rejob, \bibverse{32} así que el
pueblo deAser siguió viviendo allí entre los habitantes cananeos de la
tierra porque no los habían expulsado.

\bibverse{33} Neftalí no expulsó a los habitantes de las ciudades de
Bet-semes y Bet-anat. Así que el pueblo deAser siguió viviendo allí
entre los habitantes cananeos de la tierra porque no los habían
expulsado. Sin embargo, el pueblo deBet-semes y Bet-anat fue obligada a
realizar trabajos forzados para el pueblo de Neftalí.

\bibverse{34} Los amorreos hicieron retroceder al pueblo de Dan a la
región montañosa; no los dejaron bajar a las tierras bajas.
\bibverse{35} Los amorreos insistieron en quedarse en el monte Heres,
Ajalón y Salbim, pero cuando las tribus de José se hicieron más fuertes,
los amorreos fueron obligados a hacer trabajos forzados. \bibverse{36}
La frontera con los amorreos iba desde el Paso del Escorpión, pasando
por Sela y subiendo desde allí.

\hypertarget{section-1}{%
\section{2}\label{section-1}}

\bibverse{1} El ángel del Señor fue de Gilgal a Boquín y le dijo al
pueblo: ``Yo los saqué de la tierra de Egipto y los traje a esta tierra
que les prometí a sus antepasados, y les dije que nunca rompería el
acuerdo que hice con ustedes. \bibverse{2} También les dije que no
hicieran ningún acuerdo con los pueblos que vivían en la tierra y que
derribaran sus altares. Pero ustedes se negaron a obedecer lo que les
dije. ¿Por qué hiciste esto? \bibverse{3} También os advertí: `No los
expulsaré delante de ustedes, y serán trampas para ustedes, y sus dioses
serán trampas para ustedes.'\,''\footnote{\textbf{2:3} Véase Números
  33:55; Josué 23:13}

\bibverse{4} Después de que el ángel del Señor explicó esto a todos los
israelitas, el pueblo lloró a gritos. \bibverse{5} Por eso llamaron al
lugar Boquín,\footnote{\textbf{2:5} ``Boquín'' significa ``llanto.''}y
presentaron allí sacrificios al Señor.

\bibverse{6} Después de que Josué despidió al pueblo, los israelitas
fueron a tomar posesión de la tierra, cada uno a su tierra asignada.
\bibverse{7} El pueblo siguió adorando al Señor durante toda la vida de
Josué, y durante toda la vida de los ancianos que le sobrevivieron, los
que habían visto todas las cosas maravillosas que el Señor había hecho
por Israel.

\bibverse{8} Josué, hijo de Nun, siervo del Señor, murió a la edad de
ciento diez años. \bibverse{9} Lo enterraron en Timnat-Jeres, en la
región montañosa de Efraín, al norte del monte Gaas, la tierra que le
había sido asignada.

\bibverse{10} Una vez que pasó esa generación, la siguiente no conoció
al Señor ni lo que había hecho por Israel. \bibverse{11} Los israelitas
hicieron lo que era malo a los ojos del Señor, y adoraron a los
baales.\footnote{\textbf{2:11} ``Baales'': dioses paganos.}
\bibverse{12} Abandonaron al Señor, el Dios de sus antepasados, que los
había sacado de Egipto. Siguieron a otros dioses, inclinándose en
adoración a los dioses de los pueblos que los rodeaban, haciendo enojar
al Señor. \bibverse{13} Abandonaron al Señor y adoraron a los ídolos
Baal y Astarot. \bibverse{14} Como el Señor se enojó con Israel, los
entregó a los invasores que los saquearon. Los vendió a sus enemigos de
alrededor, enemigos a los que ya no podían resistir. \bibverse{15} Cada
vez que Israel entraba en batalla, el Señor luchaba contra ellos y los
derrotaba, tal como les había advertido y como había jurado que haría.
Estaban en un gran apuro.

\bibverse{16} Entonces el Señor les dio jueces,\footnote{\textbf{2:16}
  ``Jueces'': o ``líderes.''}que los salvaran de sus invasores.
\bibverse{17} Pero aun así, se negaban a escuchar a sus jueces y se
prostituyeron siguiendo a otros dioses, inclinándose ante ellos.
Rápidamente abandonaron el camino que habían seguido sus antepasados, y
no obedecieron los mandamientos del Señor como lo habían hecho sus
antepasados.

\bibverse{18} Cuando el Señor proveyó a Israel de jueces, estuvo con
cada juez y salvó al pueblo de sus enemigos durante la vida de ese juez,
porque el Señor se compadecía de su pueblo, que gemía bajo sus opresores
y perseguidores. \bibverse{19} Pero cuando el juez moría, el pueblo
recaíay hacía cosas incluso peores que sus antepasados, siguiendo a
otros dioses y adorándolos. Se negaron a dejar lo que hacían y se
aferraron a sus costumbres obstinadas.

\bibverse{20} Como resultado, el Señor se enojó con Israel y les dijo:
``Debido a que esta nación ha quebrantado el acuerdo que ordené a sus
antepasados que obedecieran, y no ha prestado atención a lo que dije,
\bibverse{21} de ahora en adelante no expulsaré ante ellos a ninguna de
las naciones que Josué dejó al morir. \bibverse{22} Esto es con el fin
de usarlas para probar a Israel y ver si guardan el camino del Señor y
lo siguen como lo hicieron sus antepasados.'' \bibverse{23} Esta es la
razón por la que el Señor permitió que esas naciones permanecieran, y no
las expulsó inmediatamente entregándolas a Josué.

\hypertarget{section-2}{%
\section{3}\label{section-2}}

\bibverse{1} Las siguientes son las naciones que el Señor dejó y utilizó
para poner a prueba a todos aquellos israelitas que no habían conocido
lo que era formar parte de ninguna de las guerras en Canaán.
\bibverse{2} (Lo hizo para que las generaciones posteriores de Israel,
especialmente a los que no la habían experimentado antes, aprendieran de
la guerre). \bibverse{3} Son: los cinco jefes de los filisteos, todos
los cananeos, los sidonios y los heveosque viven en las montañas del
Líbano, desde el monte Baal-hermón hasta Lebo-jamat. \bibverse{4} Fueron
dejados allí para que probar a los israelitas, comprobarsi éstos
guardarían los mandamientos del Señor que él había dado a sus
antepasados por medio de Moisés. \bibverse{5} Vivían entre cananeos,
hititas, amorreos, ferezeos, heveos y jebuseos. \bibverse{6} Los
israelitas se mezclaron con ellos, se casaron con sus hijas, dieron sus
propias hijas a sus hijos y adoraron a sus dioses.

\bibverse{7} Los israelitas hicieron lo que era malo a los ojos del
Señor. Ignoraron al Señor, su Dios, y adoraron las imágenes de los
baales y de los ashires. \bibverse{8} El Señor se enojó con Israel, y
los vendió a Cusán-Risatayín, rey de AramNajarayín. Los israelitas
estuvieron sometidos a Cusán-Risatayín durante ocho años.

\bibverse{9} Pero cuando los israelitas clamaron al Señor para que los
ayudara, él proveyó a alguien para rescatarlos, Otoniel, hijo de Quenaz,
hermano menor de Caleb, y él los salvó. \bibverse{10} El Espíritu del
Señor vino sobre él, y se convirtió en juez de Israel. Fue a la guerra
con Cusán-Risatayín, rey de Aram, y el Señor entregó al rey a Otoniel,
que salió victorioso. \bibverse{11} Como resultado, el país estuvo en
paz durante cuarenta años, hasta que murió Otoniel, hijo de Cenaz.

\bibverse{12} Pero una vez más los israelitas hicieron lo que era malo a
los ojos del Señor, y por eso el Señor le dio poder a Eglón, rey de
Moab, para que conquistara a Israel. \bibverse{13} Eglón hizo que los
amonitas y los amalecitas se unieran a él, y luego atacó y derrotó a
Israel, tomando posesión de la Ciudad de las Palmas.\footnote{\textbf{3:13}
  ``Ciudad de las Palmas'': Jericó.} \bibverse{14} Los israelitas
estuvieron sometidos a Eglón, rey de Moab, durante dieciocho años.

\bibverse{15} Nuevamente los israelitas clamaron al Señor para que los
ayudara, y él les proporcionó a alguien que los rescatara, Aod, hijo de
Guerá el benjamita, un hombre zurdo. Los israelitas lo enviaron a pagar
el tributo a Eglón, rey de Moab. \bibverse{16} Aod se había hecho una
espada de doble filo de un codo de largo, y se la ató al muslo derecho
debajo de su ropa. \bibverse{17} Llegó y presentó el tributo a Eglón,
rey de Moab, que era un hombre muy gordo.

\bibverse{18} Después de entregar el tributo, envió a casa a los que
habían ayudado a llevarlo. \bibverse{19} Pero cuando llegó a los ídolos
de piedra cerca de Gilgal, se volvió. Fue a ver a Eglón y le dijo: ``Su
Majestad, tengo un mensaje secreto para usted''. El rey les dijo a sus
asistentes: ``¡Silencio!'' y todos se fueron.

\bibverse{20} Aod se acercó entonces a donde Eglón estaba sentado solo
en su fresca habitación del piso superior, y le dijo: ``Tengo un mensaje
de Dios para usted''. Cuando el rey se levantó de su asiento,
\bibverse{21} Aod tomó su espada con la mano izquierda desde su muslo
derecho y la clavó en el vientre de Eglón. \bibverse{22} La empuñadura
entró con la hoja y la grasa se cerró sobre ella. Entonces Aod no sacó
la espada, y el rey defecó.

\bibverse{23} Entonces Aod cerró y echó el cerrojo a las puertas, y
escapó por la letrina.\footnote{\textbf{3:23} ``Letrina'': El
  significado de la palabra es incierto, algunos creen que significa
  ``cobertizo'', sin embargo parece que Aod consiguió salir de la
  habitación en secreto. Descender por una letrina abierta parece ser la
  mejor conclusión.} \bibverse{24} Cuando se hubo marchado, llegaron los
criados y vieron que las puertas de la habitación estaban cerradas con
llave. ``Debe estar usando la letrina'', concluyeron. \bibverse{25} Así
que esperaron hasta que no pudieron aguantar más, y como todavía no
había abierto las puertas de la habitación, fueron a buscar la llave y
abrieron las puertas. Allí estaba su señor, muerto en el suelo.

\bibverse{26} Mientras los sirvientes se demoraban en actuar, Aod
escapó, pasando los ídolos de piedra y dirigiéndose a Seirat.
\bibverse{27} Al llegar allí, hizo sonar una trompeta en la región
montañosa de Efraín, y los israelitas se le unieron. Bajaron de las
colinas, con Aod a la cabeza. \bibverse{28} Él les dijo: ``Síganme,
porque el Señor les ha entregado a Moab, su enemigo''. Así que lo
siguieron hacia abajo y se apoderaron de los vados del Jordán que
llevaban a Moab. No dejaron que nadie cruzara. \bibverse{29} Luego
atacaron a los moabitas y mataron a unos 10.000 de sus mejores y más
fuertes combatientes. Ni uno solo escapó. \bibverse{30} Ese día Moab fue
conquistado y sometido a Israel, y el país estuvo en paz durante ochenta
años.

\bibverse{31} Después de Aod fue Samgar, hijo de Anat, quien mató a
seiscientos filisteos con un carro de bueyes. También rescató a Israel.

\hypertarget{section-3}{%
\section{4}\label{section-3}}

\bibverse{1} Después de la muerte de Aod, los israelitas volvieron a
hacer lo que era malo a los ojos del Señor. \bibverse{2} Así que el
Señor los vendió a Jabín, rey de Canaán, que gobernaba desde Hazor. Su
comandante del ejército era Sísara, que vivía en JarosetGoyim.
\bibverse{3} Los israelitas clamaron al Señor para que los ayudara,
porque Sísara tenía novecientos carros de hierro y los maltrató
cruelmente durante veinte años.

\bibverse{4} Débora, esposa de Lapidot, era profeta y dirigía a Israel
como juez en ese momento. \bibverse{5} Se sentaba bajo la palma de
Débora, entre Ramá y Betel, en la región montañosa de Efraín, y los
israelitas acudían a ella para que tomara sus decisiones.\footnote{\textbf{4:5}
  Aunque a Débora se le llama ``juez'', su papel es mucho más que el de
  un magistrado. Las decisiones que tomó eran de importancia nacional,
  más que la mera resolución de disputas legales. Enestecaso, ``juzgar''
  tendría el significado de ``gobernar''.} \bibverse{6} Mandó llamar a
Barac, hijo de Abinoam, desde la ciudad de Cedes, en Neftalí, y le dijo:
``El Señor, el Dios de Israel, te lo ordena: 'Ve al monte Tabor, y toma
contigo diez mil hombres de Neftalí y Zabulón, y llévalos allí.
\bibverse{7} Yo llevaré a Sísara, el comandante del ejército de Jabín,
con sus carros y sus tropas, hasta el río Cisón, y te lo entregaré''.

\bibverse{8} Barac respondió: ``Si vienes conmigo, iré; pero si no
vienes conmigo, no iré''.

\bibverse{9} ``Definitivamente iré contigo'', respondió Débora, ``pero
si vas a tomar ese camino, no recibirás ningún respeto, porque el Señor
entregará a Sísara en manos de una mujer.'' Débora se levantó y fue con
Barac a Cedes. \bibverse{10} Barac convocó a los ejércitos de Zabulón y
Neftalí, y diez mil hombres se reunieron bajo su mando. Débora también
estaba allí con él.

\bibverse{11} (Heber el ceneo se había separado de los demás ceneos, los
descendientes de Hobab, el suegro de Moisés, y había instalado su tienda
en el gran árbol de Zaanannim, que está cerca de Cedes).

\bibverse{12} Sísara se enteró de que Barac, hijo de Abinoam, había ido
al monte Tabor, \bibverse{13} así que convocó a todos sus novecientos
carros de hierro y a todos sus hombres para que vinieran desde
JarosetGoyim hasta el río Cisón.

\bibverse{14} Entonces Débora le dijo a Barac: ``¡Ponte en marcha! Hoy
el Señor te ha entregado a Sísara. ¿No marchó el Señor delante de ti?''
Entonces Barac bajó del monte Tabor, acompañado de diez mil hombres.
\bibverse{15} CuandoBarac atacó, el Señor hizo entrar en pánico a Sísara
y a todos sus carros y guerreros. Sísara saltó de su carro y huyó.
\bibverse{16} Barac persiguió a los carros y a las tropas hasta
JarosetGoyim. Todo el ejército de Sísara murió; no sobrevivió ni un solo
hombre.

\bibverse{17} Mientras tanto, Sísara había huido a la tienda de Jael, la
esposa de Heber el ceneo, porque había un tratado de paz entre Jabín,
rey de Hazor, y la familia de Heber el ceneo. \bibverse{18} Jael salió
al encuentro de Sísara y le dijo: ``Entra, señor mío, entra conmigo. No
tengas miedo''. Así que él entró en su tienda, y ella lo cubrió con una
gruesa manta.

\bibverse{19} ``Por favor, dame un poco de agua para beber, porque tengo
sed'', le pidió Sísara. Así que ella abrió un odre de leche, le dio de
beber y lo volvió a tapar.

\bibverse{20} ``Haz guardia en la puerta de la tienda'', le dijo él.
``Si viene alguien y te pregunta: ``¿Hay alguien aquí?'', sólo di que
no''.

\bibverse{21} PeroJael, la esposa de Heber, tomó una estaca de la tienda
y un martillo y se acercó sigilosamente a él, donde yacía profundamente
dormido y agotado. Le clavó la estaca de la tienda en la sien hasta
atravesarlahasta la tierra, y así murió.

\bibverse{22} Cuando Barak pasó, buscando a Sísara, Jael salió a su
encuentro y le dijo: ``Ven aquí y te mostraré al hombre que buscas''. Él
entró con ella, y allí yacía Sísara, muerto, con la estaca de la tienda
atravesada en la sien.

\bibverse{23} Ese día Dios derrotó a Jabín, rey de Canaán, en presencia
de los israelitas. \bibverse{24} A partir de entonces Israel se hizo
cada vez más poderoso hasta que destruyó a Jabín, rey de Hazor.

\hypertarget{section-4}{%
\section{5}\label{section-4}}

\bibverse{1} Aquel día Débora y Barak, hijo de Abinoam, entonaron esta
canción:

\bibverse{2} ``Los líderes de Israel se hicieron cargo, y el pueblo se
comprometió totalmente. ¡Alabado sea el Señor!

\bibverse{3} ¡Escuchen, reyes! ¡Prestad atención, gobernantes! Yo, sí
yo, cantaré al Señor; alabaré al Señor, el Dios de Israel, con un canto.

\bibverse{4} Señor, cuando saliste de Seir, cuando marchaste del país de
Edom, la tierra tembló, la lluvia cayó del cielo, las nubes derramaron
agua.

\bibverse{5} Las montañas se derritieron en presencia del Señor, el Dios
del Sinaí, en presencia del Señor, el Dios de Israel.

\bibverse{6} En los días de Shamgar, hijo de Anat, en los días de Jael,
la gente no usaba las carreteras principales y se quedaba en caminos
sinuosos.

\bibverse{7} La vida de las aldeas en Israel estaba
abandonada\footnote{\textbf{5:7} Es de suponer que la gente se trasladó
  a ciudades fortificadas para protegerse.}hasta que yo, Débora, entré
en escena como madre en Israel.

\bibverse{8} Cuando el pueblo eligió nuevos dioses,\footnote{\textbf{5:8}
  O ``Cuando Dios eligió nuevos líderes.''}entonces la guerra llegó a
sus puertas. Ni siquiera un escudo o una lanza podían encontrarse entre
los cuarenta mil guerreros de Israel.

\bibverse{9} Mis pensamientos están con los comandantes israelitas y con
la gente que se ofreció como voluntaria. ¡Alabado sea el Señor!

\bibverse{10} Ustedes, que van montados en asnos blancos, sentados en
cómodas mantas, viajando por el camino, observen

\bibverse{11} de lo que habla la gente cuando se reúne en los
abrevaderos. Describen los actos justos del Señor y los de sus guerreros
en Israel. Entonces el pueblo del Señor se dirigió a las puertas de la
ciudad.

\bibverse{12} `¡Despierta, Débora, despierta! ¡Despierta, despierta,
canta una canción! ¡Levántate, Barac! Captura a tus prisioneros, hijo de
Abinoam.'

\bibverse{13} Lossobrevivientes\footnote{\textbf{5:13}
  ``Sobrevivientes'': refiriéndose al ``remanente'' de Israel. Los
  ``nobles'' y los ``poderosos'' se refieren a los señores cananeos.}fueron
a atacar a los nobles, el pueblo del Señor fue a atacar a los poderosos.

\bibverse{14} Algunos vinieron de Efraín, tierra que solía pertenecer a
los amalecitas; la tribu de Benjamín te siguió con sus hombres. Los
comandantes vinieron de Maquir; de Zabulón vinieron los que llevan el
bastón de mando de un militar.

\bibverse{15} Los jefes de Isacar apoyaron a Débora y a Barac; corrieron
hacia el valle siguiendo a Barac. Pero la tribu de Rubén estaba muy
indecisa.

\bibverse{16} ¿Por qué se quedaron en casa, en los rediles, escuchando a
los pastores que silbaban por sus rebaños? La tribu de Rubén realmente
no podía decidir qué hacer.

\bibverse{17} Galaad se quedó al otro lado del Jordán. Dan se quedó con
sus barcos. Aser se quedó en la costa, sin moverse de sus puertos.

\bibverse{18} El pueblo de Zabulón arriesgó su vida, al igual que
Neftalí en los campos de batalla de altura.

\bibverse{19} Los reyes vinieron y lucharon, los reyes cananeos lucharon
en Tanac, cerca de las aguas de Meguido, pero no obtuvieron ningún botín
de plata.\footnote{\textbf{5:19} No recibieron el botín que esperaban al
  unirse a la batalla contra los israelitas.}

\bibverse{20} Las estrellas lucharon desde el cielo. Las estrellas en
sus cursos lucharon contra Sísara.

\bibverse{21} El río Cisón los arrastró: ¡el viejo río se convirtió en
un torrente impetuoso!\footnote{\textbf{5:21} La participación de las
  estrellas del cielo y la tormenta que provocó la inundación del río
  son significativas, ya que los dioses cananeos estaban asociados con
  el clima y las estrellas, mostrando a los involucrados la supremacía
  del Señor sobre tales ``dioses.''}¡Pero yo marché con valentía!

\bibverse{22} Then the horses' hooves flailed loudly, his stallions
stampeded.

\bibverse{23} 'Maldice a Meroz,'\footnote{\textbf{5:23} ``Meroz'': el
  lugar no se menciona en ninguna otra parte de las Escrituras. Se
  piensa que puede referirse a los israelitas que se habían
  ``cananizado'' tanto que se negaban a ayudar a sus compatriotas.}dice
el ángel del Señor. `Maldice totalmente a los que viven allí, porque se
negaron a venir a ayudar al Señor, a ayudar al Señor contra los
poderosos enemigos.'

\bibverse{24} Jael, la esposa de Heber el ceneo, es la más alabada entre
las mujeres. Ella merece ser alabada por encima de todas las demás
mujeres que viven en tiendas.

\bibverse{25} Él pidió agua y ella le dio leche. En un recipiente digno
de los nobles le llevó suero de leche.

\bibverse{26} Con una mano tomó la estaca de la tienda y con la derecha
sostuvo un martillo de obrero. Golpeó a Sísara y le rompió el cráneo; le
destrozó y perforó la sien.

\bibverse{27} A sus pies se desplomó, cayó, quedó inmóvil. A sus pies se
desplomó, cayó; donde se desplomó, allí cayó, su vida le fue
arrebatada.\footnote{\textbf{5:27} Aunque hay mucha repetición en este
  versículo, se mantiene en la traducción por su efecto dramático.
  También se mantiene en la traducción la última palabra del texto
  hebreo, que significa ``saqueado'' o ``expoliado'', en lugar de decir
  simplemente que estaba muerto, ya que se le quitó la vida de forma
  similar a la de un soldado que saquea la casa de una víctima.}

\bibverse{28} La madre de Sísara se asomó a la ventana. A través de la
ventana enrejada gritó: ``¿Por qué tarda tanto en llegar su carro? ¿Por
qué se retrasa tanto el sonido de su carro?

\bibverse{29} La más sabia de sus damas le dice, y ella se repite a sí
misma las mismas palabras:

\bibverse{30} 'Están ocupados repartiendo el botín y asignando una o dos
jóvenes\footnote{\textbf{5:30} ``dos jóvenes'': literalmente,
  ``vientres,'' un término despectivo para referirse a las mujeres.}
para cada hombre. Dentro del botín seguro habrá ropas de colores para
Sísara; en el botín habrá ropas de colores bellamente bordadas; un
botín\footnote{\textbf{5:30} La repetición es de nuevo significativa: La
  palabra ``saqueo'' se utiliza tres veces: se imagina a la madre de
  Sísara pensando en todo el maravilloso botín que recibirá. Sin
  embargo, es Sísara quien ha sido ``saqueado'' (la palabra que se
  utiliza allí suele significar simplemente destruido, pero puede
  incluir el saqueo y el pillaje), y por supuesto la madre de Sísara se
  sentirá amargamente decepcionada.} de ropas de doble bordado que
llegan hasta el cuello.'

\bibverse{31} ¡Que todos tus enemigos mueran así, Señor! ¡Pero que los
que te aman brillen como el sol en todo su esplendor!'' La tierra estuvo
en paz por cuarenta años.

\hypertarget{section-5}{%
\section{6}\label{section-5}}

\bibverse{1} Pero los israelitas hicieron lo que era malo a los ojos del
Señor. Así que el Señor los entregó a los madianitas durante siete años.
\bibverse{2} La opresión madianita era tan grande que, a causa de ellos,
los israelitas se hicieron de escondites en montañas, cuevas y
fortificaciones. \bibverse{3} Cada vez que los israelitas sembraban sus
cosechas, los madianitas, amalecitas y otros pueblos del este venían a
atacarlos. \bibverse{4} Instalaban sus campamentos y destruían las
cosechas del país hasta Gaza. No dejaban nada para comer en todo Israel,
y tomaban para sí todas las ovejas, el ganado y los asnos. \bibverse{5}
Llegaron en gran número con su ganado y sus tiendas como enjambres de
langostas, con tantos camellos que no se podían contar. Invadieron la
tierra para devastarla por completo. \bibverse{6} Los israelitas se
vieron desesperadamente empobrecidos por los madianitas y pidieron ayuda
al Señor.

\bibverse{7} Cuando los israelitas clamaron al Señor por ayuda a causa
de los madianitas, \bibverse{8} el Señor envió a los israelitas un
profeta. Éste les dijo: ``Esto es lo que dice el Señor, el Dios de
Israel: `Yo os saqué de Egipto; yo os saqué del lugar\footnote{\textbf{6:8}
  Literalmente, ``casa.''}donde erais esclavos. \bibverse{9} Lossalvé
del poder de los egipcios y de todos los que os oprimían. Los expulsé
delante de ustedes y les di su tierra. \bibverse{10} Yo te advertí: Yo
soy el Señor, tu Dios. No debes adorar a los dioses de los amorreos, en
cuya tierra vives ahora'. Pero no me escuchaste''.

\bibverse{11} El ángel del Señor vino y se sentó bajo la encina de Ofra
que pertenecía a Joás el abiezerita. Su hijo Gedeón estaba trillando
allí el trigo en un lagar para ocultarlo de los madianitas.
\bibverse{12} El ángel del Señor se le apareció y le dijo: ``¡El Señor
está contigo, gran hombre valiente!''

\bibverse{13} ``Perdona, mi señor, pero si el Señor está con nosotros,
¿por qué nos ha pasado todo esto?'' respondió Gedeón. ``¿Dónde están
todos sus maravillosos milagros que nos recordaban nuestros antepasados
cuando decían: `¿No fue el Señor quien nos sacó de Egipto?'. Pero ahora
el Señor nos ha abandonado y nos ha entregado a los madianitas''.

\bibverse{14} El Señor se dirigió a él y le dijo: ``Ve con la fuerza que
tienes y salva a Israel de los madianitas. ¿No soy yo quien te envía?''

\bibverse{15} ``Perdona, mi señor, pero ¿cómo puedo salvar a
Israel?''respondió Gedeón. ``¡Mi familia es la menos importante de la
tribu de Manasés, y yo soy la persona menos importante de esa
familia!''.

\bibverse{16} ``Yo estaré contigo'', le dijo el Señor. ``Derrotarás a
los madianitas como si fueran un solo hombre''.

\bibverse{17} ``Por favor, Señor, si piensas bien de mí, dame una señal
de que realmente eres tú quien me dice esto'', pidió Gedeón.
\bibverse{18} ``No te vayas hasta que regrese y te presente mi
ofrenda''.

``Me quedaré aquí hasta que vuelvas'', respondió.

\bibverse{19} Gedeón fue y cocinó un cabrito y coció panes sin levadura
con un efa de harina. Puso la carne en una cesta y el caldo en una olla.
Los sacó y se los presentó al ángel bajo la encina.

\bibverse{20} El ángel de Dios le dijo: ``Coloca la carne y los panes
sin levadura sobre esta roca y vierte el caldo sobre ellos''. Así lo
hizo Gedeón.

\bibverse{21} El ángel del Señor extendió el báculo que tenía en la mano
y tocó la carne y los panes ácimos con la punta. De la roca salió fuego
y quemó la carne y los panes sin levadura. Luego el ángel desapareció.

\bibverse{22} Cuando Gedeón se dio cuenta de que era el ángel del Señor,
gritó: ``¡Oh, no, Señor Dios! He visto al ángel del Señor cara a cara''

\bibverse{23} Pero el Señor le dijo: ``¡Paz! No te preocupes, no vas a
morir''.

\bibverse{24} Así que Gedeón construyó allí un altar al Señor y lo llamó
``El Señor es la Paz''. Todavía hoy está allí, en Ofra de los
abiezritas.

\bibverse{25} Esa noche, el Señor le dijo a Gedeón: ``Toma el toro de tu
padre y un segundo toro de siete años, y derriba el altar de Baal de tu
padre, y corta el poste de Asera que está al lado. \bibverse{26} Luego
construye un altar al Señor, tu Dios, en la forma debida, en la cima de
la colina. Con la madera del poste de Asera que cortaste como leña, toma
el segundo toro y preséntalo como holocausto''.

\bibverse{27} Gedeón, acompañado por diez de sus siervos, hizo lo que el
Señor le había dicho. Sin embargo, como tenía miedo de su familia y de
el pueblo del pueblo, lo hizo durante la noche y no de día.

\bibverse{28} Por la mañana, cuando el pueblo del pueblo se levantó, vio
que el altar de Baal había sido derribado y el poste de Asera que estaba
a su lado había sido cortado, y que el segundo toro había sido
sacrificado en el altar que acababa de ser construido. \bibverse{29} Se
preguntaron unos a otros: ``¿Quién ha hecho esto?''. Enetoncesindagaron
hasta que les dijeron: ``Lo hizo Gedeón, hijo de Joás''.

\bibverse{30} ``Entrega a tu hijo'', le ordenó el pueblo del pueblo a
Joás. ``Debe morir, porque ha derribado el altar de Baal y ha cortado el
poste de Asera que estaba junto a él''.

\bibverse{31} Joás respondió a todos los que se enfrentaban a él:
``¿Acaso están peleando a favor de Baal? ¿Tienen que salvarlo?
Cualquiera que pelee a favor de él será condenado a muerte por la
mañana. Si es un dios, que luche por sí mismo contra los que derribaron
su altar''.

\bibverse{32} Aquel día llamaron a Gedeón Jerub-baal, que significa
``Que Baal luche con él'', porque había derribado su altar.

\bibverse{33} Todos los madianitas, amalecitas y otros pueblos del
Oriente se reunieron y cruzaron el Jordán. Acamparon en el valle de
Jezreel. \bibverse{34} El Espíritu del Señor vino\footnote{\textbf{6:34}
  Literalmente, ``vistió.''} sobre Gedeón, y tocó la trompeta, llamando
a los del clan de Abiezerpara que se les unieran. \bibverse{35} Envió
mensajeros por todo el territorio de Manasés, llamándolos para que se
unieran a él, y también a Aser, Zabulón y Neftalí, para que también
vinieran y se unieran a los demás.

\bibverse{36} Gedeón dijo a Dios: ``Si salvas a Israel por medio de mí,
como lo prometiste, \bibverse{37} entonces mira: pondré un vellón de
lana en la era. Si el vellón está mojado por el rocío pero la tierra
está seca, entonces sabré que vas a salvar a Israel a través de mí como
lo prometiste.''

\bibverse{38} Eso fue lo que ocurrió. Cuando Gedeón se levantó temprano
a la mañana siguiente, presionó el vellón y exprimió el rocío,
suficiente agua para llenar un tazón.

\bibverse{39} Entonces Gedeón le dijo a Dios: ``Por favor, no te enfades
conmigo. Sólo déjame hacer una petición más. Déjame hacer una prueba más
con el vellón. Esta vez deja que el vellón esté seco y que toda la
tierra se cubra de rocío''.

\bibverse{40} Esa noche Dios hizo exactamente eso. Sólo el vellón se
secó y toda la tierra se cubrió de rocío.

\hypertarget{section-6}{%
\section{7}\label{section-6}}

\bibverse{1} Jerub-baal (Gedeón) y los que estaban con él se levantaron
temprano y fueron a acampar junto a la fuente de Jarod. El campamento
madianita estaba al norte, en el valle cercano a la colina de Moré.

\bibverse{2} El Señor le dijo a Gedeón: ``Hay demasiados soldados
contigo para que les entregue a los madianitas, pues de lo contrario
Israel se jactará ante mí diciendo: `Me salvé con mis propias fuerzas'.
\bibverse{3} Así que dile a los soldados: `Cualquiera que esté
preocupado o tenga miedo puede abandonar el monte Galaad y volver a su
casa'\,''. Veintidós mil de ellos volvieron a casa, pero diez mil se
quedaron.

\bibverse{4} Entonces el Señor le dijo a Gedeón: ``Todavía hay
demasiados soldados. Llévalos al agua y yo los reduciré \footnote{\textbf{7:4}
  O ``probaré,'' ``tamizaré,'' ``purgaré.''}por ti. El que yo te diga:
`Irá contigo', irá. Pero el que te diga: `No irá contigo', no irá''.

\bibverse{5} Gedeón llevó a los soldados al agua. El Señor le dijo a
Gedeón: ``Pongan a un lado a los que lamen el agua con la lengua, como
hace un perro, y al otro lado a los que se arrodillen para beber.''
\bibverse{6} Trescientos lamieron el agua de sus manos a la boca. Todos
los demás se arrodillaron para beber el agua.

\bibverse{7} El Señor le dijo a Gedeón: ``Con estos trescientos hombres
que lamieron te salvaré y te entregaré a los madianitas. Deja que el
resto de los soldados se vaya a casa''.

\bibverse{8} Los trescientos se hicieron cargo de las provisiones y las
trompetas de los demás. Gedeón envió a todo el resto a casa, pero se
quedó con los trescientos hombres.

El campamento madianita estaba debajo de él en el valle. \bibverse{9}
Esa noche el Señor le habló a Gedeón: ``Levántate, baja y ataca el
campamento, porque te lo he entregado. \bibverse{10} Pero si tienes
miedo de bajar, ve con tu siervo Furá al campamento. \bibverse{11} Oirás
lo que hablan y entonces tendrás el valor de atacar el campamento''. Así
que tomó a su siervo Furá con él y se dirigió al borde del campamento,
donde había hombres armados de guardia.

\bibverse{12} Los madianitas, los amalecitas y todos los pueblos de
Oriente llenaban el valle como una nube de langostas, y en cuanto a sus
camellos, eran tan incontables como la arena de la orilla del mar.
\bibverse{13} Justo cuando llegó Gedeón, un hombre le contaba a su amigo
un sueño que había tenido. Decía: ``He tenido este sueño. Soñé que veía
una hogaza redonda de pan de cebada llegar rodando al campamento
madianita. Golpeaba una tienda de campaña y la ponía patas arriba, en el
suelo''.

\bibverse{14} ``Esto sólo puede representar la victoria por la espada de
Gedeón, hijo de Joás, un hombre de Israel'', respondió su amigo. ``Dios
le ha entregado a los madianitas y a todos los que están acampados
aquí''.

\bibverse{15} Cuando Gedeón escuchó el sueño y lo que significaba, se
inclinó en señal de agradecimiento a Dios.\footnote{\textbf{7:15} ``En
  señal de agradecimiento a Dios'': implícito. El hebreo dice
  simplemente dice: ``se inclinó.''}Volvió al campamento israelita y
anunció: ``¡De pie! Porque el Señor te ha entregado el campamento
madianita''.

\bibverse{16} Dividió a los trescientos hombres en tres compañías. A
todos les entregó trompetas y jarras vacías con antorchas en su
interior. \bibverse{17} ``Observadme y seguid mi ejemplo'', les dijo.
``Cuando llegue al límite del campamento, haced exactamente lo que yo
haga. \bibverse{18} Inmediatamente, yo y los que están conmigo tocaremos
las trompetas, y luego ustedes tocarán sus trompetas desde todo el
campamento y gritarán: ``¡Por el Señor y por Gedeón!''

\bibverse{19} Gedeón y los cien hombres que lo acompañaban llegaron a
las afueras del campamento alrededor de la medianoche,\footnote{\textbf{7:19}
  Literalmente, ``el comienzo de la media vigilia''}después de que se
cambiaron los guardias. Hicieron sonar sus trompetas y rompieron las
jarras que llevaban. \bibverse{20} Las tres compañías tocaron sus
trompetas y rompieron sus jarras. Tenían las antorchas en la mano
izquierda y las trompetas en la derecha, y gritaban: ``¡Una espada para
el Señor y para Gedeón!''

\bibverse{21} Cada uno se puso en su lugar rodeando el campamento, y
todos los soldados enemigos corrieron gritando; luego huyeron.
\bibverse{22} Cuando tocaron las trescientas trompetas, el Señor hizo
que todos los hombres del campamento se atacaran unos a otros con sus
espadas. El ejército enemigo huyó hacia BetShitá, cerca de Zererá, hasta
la frontera de Abel Meholá, cerca de Tabbá. \bibverse{23} Los soldados
israelitas fueron convocados desde Neftalí, Aser y todo Manasés, y
persiguieron a los madianitas. \bibverse{24} Gedeón envió mensajeros por
toda la región montañosa de Efraín diciendo: ``Vengan a atacar a los
madianitas y tomen el control de los vados del Jordán delante de ellos
hasta Bet-Bara.'' Así que todos los hombres de Efraín fueron convocados,
y tomaron el control de los vados del Jordán hasta Bet-Bara.
\bibverse{25} También capturaron a Oreb y Zeeb, dos de los comandantes
madianitas. Mataron a Oreb en la roca de Oreb, y a Zeeb en el lagar de
Zeeb. Siguieron persiguiendo a los madianitas y llevaron las cabezas de
Oreb y Zeeb a Gedeón, que estaba al otro lado del Jordán.

\hypertarget{section-7}{%
\section{8}\label{section-7}}

\bibverse{1} Entonces los hombres de Efraín le preguntaron a Gedeón:
``¿Por qué nos has tratado así? ¿Por qué no nos llamaste cuando fuiste a
atacar a los madianitas?''. Discutieron furiosamente con él.

\bibverse{2} ``¿Y qué he logrado yo en comparación con ustedes?'' Gedeón
respondió. ``¡Incluso las uvas sobrantes de Efraín son mejores que toda
la cosecha de uvas de Abiezer! \bibverse{3} Dios te entregó a Oreb y
Zeeb, los dos comandantes madianitas. ¿Qué he conseguido yo en
comparación con ustedes?''. Cuando les dijo esto, su animosidad hacia él
se calmó.

\bibverse{4} Entonces Gedeón cruzó el Jordán con sus trescientos
hombres. Aunque estaban agotados, continuaron la persecución.
\bibverse{5} Cuando llegaron a Sucot, Gedeón pidió a el pueblo de allí:
``Por favor, denles algo de pan a los hombres que están conmigo porque
están agotados; estoy persiguiendo a Zeba y Zalmnna, los reyes
madianitas.''

\bibverse{6} Pero los dirigentes del pueblo de Sucot respondieron:
``¿Por qué habríamos de dar pan a tu ejército cuando aún no has
capturado a Zeba y Zalmnna?''

\bibverse{7} ``¡En ese caso, una vez que el Señor me haya entregado a
Zeba y Zalmuna, volveré y os azotaré con espinas y cardos del
desierto!'' Gedeón respondió.

\bibverse{8} Se fue y fue a Penuel y les preguntó lo mismo, pero el
pueblo dePenuel respondió lo mismo que el pueblo deSucot. \bibverse{9}
Entonces les dijo: ``¡Cuando regrese victorioso, demoleré esta torre!''.

\bibverse{10} Zeba y Zalmuna estaban en Carcor con sus ejércitos de unos
quince mil hombres. Estos eran todos los que quedaban de los ejércitos
del pueblo de Oriente; ya habían muerto ciento veinte mil espadachines.
\bibverse{11} Gedeón tomó la ruta de las caravanas hacia el este de Noba
y Jogbehah, y atacó a su ejército, tomándolos desprevenidos.
\bibverse{12} Zeba y Zalmuna huyeron, pero él persiguió a los dos reyes
madianitas y los capturó, derrotando a todo su aterrorizado ejército.

\bibverse{13} Entonces Gedeón, hijo de Joás, regresó de la batalla por
el paso de Heres. \bibverse{14} Allí capturó a un joven de Sucot y lo
interrogó. El hombre le escribió los nombres de los setenta y siete
líderes y ancianos de Sucot. \bibverse{15} Gedeón fue y les dijo a los
líderes del pueblo de Sucot: ``Aquí están Zeba y Zalmuna, de los que te
burlaste cuando dijiste: ``¿Por qué debemos darle pan a tu agotado
ejército cuando aún no has capturado a Zeba y Zalmuna?'' \bibverse{16}
Así que tomó a los ancianos de la ciudad de Sucot y les dio una lección
usando espinas y cardos del desierto. \bibverse{17} También derribó la
torre de Peniel y mató a los hombres del pueblo.

\bibverse{18} Entonces Gedeón preguntó a Zeba y a Zalmuna: ``¿Cómo eran
los hombres que mataste en el Tabor?''

``Se parecían a ustedes'', respondieron. ``Cada uno de ellos tenía la
estatura de un príncipe''.

\bibverse{19} ``Esos eran mis hermanos, los hijos de mi madre'', estalló
Gedeón. ``¡Vive el Señor, si los hubieras dejado vivir, no te
mataría!''.

\bibverse{20} Le dijo a Jéter, su hijo mayor: ``¡Anda, mátalos!''. Pero
el muchacho se negó a sacar la espada, porque era joven y tenía miedo.

\bibverse{21} Zeba y Zalmuna le dijeron a Gedeón: ``¡Vamos, hazlo tú!
Muéstrate como un hombre y mátanos''. Entonces Gedeón se acercó y mató a
Zeba y a Zalmuna, y tomó los adornos en forma de media
luna\footnote{\textbf{8:21} Probablemente de oro e indicaba que los
  camellos pertenecían a los reyes.}del cuello de sus camellos.

\bibverse{22} porque nos has salvado de los madianitas.''

\bibverse{23} ``Yo no seré su gobernante, y mi hijo tampoco'', respondió
Gedeón. ``El Señor será tu gobernante''.

\bibverse{24} Entonces Gedeón dijo: ``Tengo una petición que pedirles:
que cada uno de ustedes me dé un pendiente de su botín.'' (Sus enemigos
eran ismaelitas y llevaban pendientes de oro).

\bibverse{25} ``Te los daremos con gusto'', respondieron. Extendieron un
manto, y cada uno de ellos echó sobre él pendientes de su botín.
\bibverse{26} El peso de los pendientes que había pedido era de 1.700
siclos, sin incluir los adornos, sino los colgantes y las prendas de
color púrpura que llevaban los reyes madianitas ni las cadenas que
llevaban al cuello de sus camellos.

\bibverse{27} Con el oro, Gedeón hizo un efod,\footnote{\textbf{8:27} El
  pectoral que llevaba el sumo sacerdote. Esta acción de Gedeón sugiere
  que pensaba que debía establecerse un centro de culto en Ofra.}que
colocó en su ciudad natal de Ofra. Todo Israel se prostituyó allí
adorándolo como un ídolo,\footnote{\textbf{8:27} ``Adorándolo como un
  ídolo'': se ha añadido para mayor claridad.}y se convirtió en una
trampa para Gedeón y su familia.

\bibverse{28} Así fue como los madianitas fueron subyugados ante los
israelitas y no volvieron a ganar poder. Así, la tierra estuvo en paz
durante cuarenta años en vida de Gedeón. \bibverse{29} Jerob-baal, hijo
de Joás, se fue a su casa, viviendo en su propia casa. \bibverse{30}
Gedeón tuvo setenta hijos, todos suyos, porque tenía muchas mujeres.
\bibverse{31} Su concubina, que vivía en Siquem, también tuvo un hijo.
Lo llamó Abimelec. \bibverse{32} Gedeón, hijo de Joás, murió a una edad
avanzada y fue enterrado en la tumba de su padre Joás, en Ofra de los
abiezritas. \bibverse{33} Pero en cuanto murió Gedeón, los israelitas
volvieron a prostituirse, adorando ante los baales. Hicieron de
Baal-berit su dios. \bibverse{34} Se olvidaron del Señor, su Dios, que
los había salvado de todos los enemigos que los rodeaban. \bibverse{35}
No mostraron ningún respeto a la familia de Jerob-baal (Gedeón) por todo
el bien que había hecho por Israel.

\hypertarget{section-8}{%
\section{9}\label{section-8}}

\bibverse{1} Abimelec, hijo de Jerob-baal, se dirigió a los hermanos de
su madre en Siquem y les dijo a ellos y a todos los parientes de su
madre: \bibverse{2} ``Por favor, preguntad a todos los dirigentes de
Siquem: `¿Qué es lo mejor para ustedes? ¿Que setenta hombres, todos
ellos hijos de Jerob-baal, gobiernen sobre ustedes, o un solo hombre?'
Recuerda que soy de tu propia sangre''.

\bibverse{3} Los hermanos de su madre compartieron su propuesta con
todos los dirigentes de Siquem, y decidieron seguir a Abimelec, porque
dijeron: ``Es nuestro pariente.'' \bibverse{4} Le dieron setenta siclos
de plata del templo de Baal-berit. Abimelec utilizó el dinero para
contratar a unos alborotadores arrogantes como su banda. \bibverse{5}
Fue a la casa de su padre en Ofra, y de una pedrada mató a sus setenta
hermanastros, los hijos de Jerob-baal. Pero Jotam, el hijo menor de
Jerob-baal, escapó escondiéndose.

\bibverse{6} Entonces todos los jefes de Siquem y Bet-millo se reunieron
junto a la encina en la columna de Siquem y nombraron rey a Abimelec.

\bibverse{7} Cuando Jotam se enteró de esto, subió a la cima del monte
Gerizim y gritó en voz alta: ``¡Escúchenme, jefes de Siquem, y que Dios
los escuche!

\bibverse{8} Érase una vez que los árboles estaban decididos a ungir un
rey que los gobernara. Le dijeron al olivo: `Tú serás nuestro rey'.
\bibverse{9} Pero el olivo replicó: ``¿Debo dejar de dar mi rico aceite,
que beneficia tanto a los dioses como a los hombres, sólo para ir de un
lado a otro de los árboles?'' \bibverse{10} Entonces los árboles
pidieron a la higuera: ``Ven tú y sé nuestro rey''. \bibverse{11} Pero
la higuera respondió: ``¿Debo dejar de dar mi buen y dulce fruto para ir
a balancearme sobre los árboles?'' \bibverse{12} Entonces los árboles le
preguntaron a la vid: ``Ven y sé nuestro rey''. \bibverse{13} Pero la
vid respondió: ``¿Debo dejar de dar mi vino, que hace felices a los
dioses y a los hombres, para ir a balancearme sobre los árboles?''
\bibverse{14} Entonces todos los árboles le preguntaron al espino: ``Ven
y sé nuestro rey''. \bibverse{15} El arbusto espinoso respondió a los
árboles: `Si de verdad son sinceros al ungirme como su rey, venid a
refugiaros a mi sombra. Pero si no, ¡que salga fuego del espino y queme
los cedros del Líbano!'

\bibverse{16} ¿Has actuado con sinceridad y honestidad al hacer a
Abimelec tu rey? ¿Has actuado con honestidad con Jerub-baal y su
familia? ¿Lo has respetado por todo lo que hizo? \bibverse{17} ¡No
olvides cómo mi padre luchó por ti y arriesgó su propia vida para
salvarte de los madianitas!

\bibverse{18} Pero hoy te has rebelado contra la familia de mi padre.
Has matado a sus setenta hijos de una sola piedra y has hecho a
Abimelec, el hijo de su esclava, rey de los dirigentes de Siquem
simplemente porque es pariente tuyo. \bibverse{19} ¿Has actuado hoy con
sinceridad y honestidad con Jerub-baal y su familia? Si es así, ¡que
seas feliz con Abimelec, y que él sea feliz también! \bibverse{20} Pero
si no lo has hecho, ¡que salga fuego de Abimelec y que queme a los
líderes de Siquem y de Bet-millo, y que salga fuego de los líderes de
Siquem y de Bet-millo y que queme a Abimelec!'' \bibverse{21} Entonces
Jotam escapó y huyó. Fue a Beer y se quedó allí por la amenaza de su
hermano Abimelec.

\bibverse{22} Abimelec gobernó sobre Israel durante tres años.
\bibverse{23} Entonces Dios envió un espíritu maligno para causar
problemas entre Abimelec y los líderes de Siquem. Los líderes de Siquem
traicionaron a Abimelec. \bibverse{24} Esto sucedió por el asesinato de
los setenta hijos de Jerob-baal y para que la responsabilidad de su
sangre recayera en Abimelec, su hermano, que los mató, y en los líderes
de Siquem, que proporcionaron los medios para matar a sus hermanos.
\bibverse{25} Los jefes de Siquem enviaron hombres a los pasos de la
colina para que acecharan y atacaran a Abimelec, y, mientras tanto,
robaban a todos los que pasaban por el camino. Abimelec se enteró de lo
que ocurría.

\bibverse{26} Gaal, hijo de Ebed, se había trasladado a Siquem con sus
parientes, y se ganó la lealtad de los dirigentes de Siquem.
\bibverse{27} En la época de la cosecha salieron al campo, recogieron
las uvas de sus viñedos y las pisaron. Lo celebraron haciendo una fiesta
en el templo de su dios, donde comieron y bebieron, y maldijeron a
Abimelec.

\bibverse{28} ``¿Quién es ese Abimelec?'', preguntó Gaal, hijo de Ebed.
``¿Y quién es Siquem, para que tengamos que servirle? ¿No es él el hijo
de Jerub-baal, mientras que Zebul es el que manda en realidad? Deberías
servir a la familia de Hamor, el padre de Siquem. ¿Por qué tendríamos
que servir a Abimelec? \bibverse{29} ¡Si yo fuera el encargado de
ustedes, me desharía de Abimelec! Le diría: ``¡Reúne a tu ejército y ven
a luchar!''.

\bibverse{30} CuandoZebul, el gobernador de la ciudad, escuchó lo que
decía Gaal, se enojó mucho. \bibverse{31} Envió secretamente mensajeros
a Abimelec para decirle: ``Mira, Gaal, hijo de Ebed, y sus parientes han
llegado a Siquem, y están incitando al pueblo a rebelarse contra ti.
\bibverse{32} Así que ven de noche con tu ejército y escóndete en el
campo. \bibverse{33} Por la mañana, en cuanto salga el sol, ve a atacar
la ciudad. Cuando Gaal y sus hombres salgan a combatirte, podrás
hacerles lo que quieras''.

\bibverse{34} Abimelec partió de noche junto con su ejército, y se
separaron en cuatro compañías que acecharon cerca de Siquem.
\bibverse{35} CuandoGaal, hijo de Ebed, salió y se puso a la puerta de
entrada de la ciudad, Abimelec y su ejército salieron de donde se habían
escondido. \bibverse{36} Gaal vio que el ejército se acercaba y le dijo
a Zebul: ``¡Mira, hay gente que baja de las cumbres!''.

``Eso son sólo sombras hechas por las colinas que parecen hombres'',
respondió Zebul.

\bibverse{37} ``No, en realidad, la gente está bajando de las alturas'',
repitió Gaal. ``Además, hay otra compañía que viene por el camino que
pasa por el roble de los adivinos''.

\bibverse{38} ``¿Dónde está tu bocaza ahora? Tú eres el que dijo:
``¿Quién es ese Abimelec, para que tengamos que servirle?''. le dijo
Zebul. ``¿No es ésta la gente que detestabas? Pues vete y lucha con
ellos''.

\bibverse{39} Así que Gaal condujo a los líderes de Siquem fuera de la
ciudad y luchó con Abimelec. \bibverse{40} Abimelec los atacó y los
persiguió a él y a sus hombres mientras huían, matando a muchos de ellos
cuando trataban de regresar a la puerta del pueblo. \bibverse{41}
Abimelec regresó a Arumá, mientras Zebul expulsaba a Gaal y a sus
parientes de Siquem.

\bibverse{42} Al día siguiente el pueblo deSiquem salió a los campos, y
Abimelec fue informado de ello. \bibverse{43} Dividió a su ejército en
tres compañías y las hizo emboscar en los campos. Cuando vio que la
gente salía de la ciudad, los atacó y los mató. \bibverse{44} Abimelec y
su compañía corrieron a ocupar la puerta de entrada de la ciudad,
mientras que las dos compañías corrieron a atacar a todos en los campos
y matarlos. \bibverse{45} La batalla por la ciudad duró todo el día,
pero finalmente Abimelec la capturó. Mató a la gente, demolió la ciudad
y esparció sal por el suelo.\footnote{\textbf{9:45} Para evitar que algo
  creciera.}

\bibverse{46} Cuando todos los líderes de la torre de Siquem se dieron
cuenta de lo que había sucedido, se refugiaron en la cámara acorazada
del templo de El-berit. \bibverse{47} CuandoAbimelec se enteró de que
todos los líderes de la torre de Siquem se habían reunido allí,
\bibverse{48} él y todos los hombres que lo acompañaban subieron al
monte Zalmón. Abimelec tomó un hacha y cortó una rama de los árboles. Se
la subió al hombro y les dijo a sus hombres: ``¡Rápido! Ya vieron lo que
hice. Hagan ustedes lo mismo''. \bibverse{49} Cada uno de ellos cortó
una rama y siguió a Abimelec. Colocaron las ramas contra la cámara
acorazada y le prendieron fuego. Así murió toda la gente que vivía en la
torre de Siquem, unos mil hombres y mujeres.

\bibverse{50} LuegoAbimelec fue a atacar Tebez y la capturó.
\bibverse{51} Pero había una torre fuerte dentro de la ciudad. Todos los
hombres y mujeres y los líderes de la ciudad corrieron hacia allí y se
atrincheraron, y luego subieron al techo de la torre. \bibverse{52}
Abimelec subió a la torre para atacarla. Pero cuando se acercaba a la
entrada de la torre para prenderle fuego, \bibverse{53} una mujer dejó
caer una piedra de molino sobre la cabeza de Abimelec y le abrió el
cráneo.

\bibverse{54} Rápidamente llamó al joven que llevaba sus armas y le
ordenó: ``Saca tu espada y mátame, para que no digan de mí que lo mató
una mujer''. Entonces el joven lo atravesó con su espada, y murió.
\bibverse{55} Cuando los israelitas vieron que Abimelec estaba muerto,
se fueron todos a sus casas.

\bibverse{56} Así pagó Dios a Abimelec el crimen que cometió contra su
padre al asesinar a sus setenta hermanos. \bibverse{57} También pagó al
pueblo de Siquem por su maldad, y la maldición de Jotam, hijo de
Jerob-baal, cayó sobre ellos.

\hypertarget{section-9}{%
\section{10}\label{section-9}}

\bibverse{1} Después del tiempo de Abimelec, Tola, hijo de Fuvá, hijo de
Dodóde la tribu de Isacar, entró en escena para salvar a Israel. Vivía
en la ciudad de Shamir, en la región montañosa de Efraín. \bibverse{2}
Dirigió a Israel como juez\footnote{\textbf{10:2} Véase la nota en el
  vserículo 4:5.}durante veintitrés años. Luego murió y fue enterrado en
Shamir.

\bibverse{3} Después de Tola vino Jair, de Galaad, quien dirigió a
Israel como juez durante veintidós años. \bibverse{4} Tenía treinta
hijos que montaban treinta asnos. Tenían treinta ciudades en la tierra
de Galaad, que hasta hoy se llaman las Ciudades de Jair. \bibverse{5}
Jair murió y fue enterrado en Camón.

\bibverse{6} Una vez más los israelitas hicieron lo que era malo a los
ojos del Señor. Adoraron a los baales y a los astoretas, así como a los
dioses de Aram, Sidón y Moab, y a los dioses de los amonitas y los
filisteos. Rechazaron al Señor y no lo adoraron. \bibverse{7} Entonces
el Señor se enojó con Israel, y los vendió a los filisteos y a los
amonitas. \bibverse{8} Ese año y durante dieciocho años más acosaron y
oprimieron a los israelitas, a todos los israelitas que vivían al este
del Jordán, en Galaad, la tierra de los amorreos. \bibverse{9} Los
amonitas también cruzaron el Jordán para atacar a Judá, Benjamín y
Efraín, causando terribles problemas a Israel.

\bibverse{10} Los israelitas clamaron al Señor por ayuda, diciendo:
``Hemos pecado contra ti, rechazando a nuestro Dios y adorando a los
baales.''

\bibverse{11} El Señor respondió a los israelitas: ``¿No los salvé de
los egipcios, los amorreos, los amonitas, los filisteos, \bibverse{12}
los sidonios, los amalecitas y los maonitas? Cuando te atacaron y
clamaste a mí por ayuda, ¿no te salvé de ellos? \bibverse{13} Pero
ustedes me han rechazado y han adorado a otros dioses, así que no
volveré a salvarlos. \bibverse{14} Ve y pide ayuda a los dioses que has
elegido. Deja que ellos te salven en tu momento de angustia''.

\bibverse{15} Los israelitas le dijeron al Señor: ``¡Hemos pecado!
Trátanos de la manera que creas conveniente, ¡sólo que por favor
sálvanos ahora!''. \bibverse{16} Así que se deshicieron de los dioses
extranjeros que tenían y adoraron al Señor. Y el Señor no pudo soportar
más la miseria de Israel.

\bibverse{17} Los ejércitos amonitas habían sido convocados y estaban
acampados en Galaad. Los israelitas se reunieron y acamparon en Mizpa.
\bibverse{18} Los comandantes del pueblo de Galaad se pusieron de
acuerdo entre ellos: ``El que dirija el ataque contra los amonitas se
convertirá en gobernante de todos los que viven en Galaad.''

\hypertarget{section-10}{%
\section{11}\label{section-10}}

\bibverse{1} Jeftéel gaaladitaera un fuerte luchador. Era hijo de una
prostituta, y su padre era Galaad. \bibverse{2} La mujer de Galaad le
dio hijos, que cuando crecieron, echaron a Jefté, diciéndole: ``No
heredarás nada de nuestro padre porque eres hijo de otra
mujer.''\footnote{\textbf{11:2} ``Hijo de otra mujer'': esto es lo que
  dice el hebreo, sin embargo, probablemente tenga el significado de
  ``el hijo de una prostituta''. Ciertamente, así lo entendieron los
  traductores de la Septuaginta.}

\bibverse{3} Jefté huyó de sus hermanos y se fue a vivir a la tierra de
Tob. Se unió a él una banda de alborotadores y los dirigió en sus
incursiones.\footnote{\textbf{11:3} El hebreo simplemente dice:
  ``salieron con él'', sin embargo el contexto indica que eran una banda
  de mercenarios.}

\bibverse{4} Más tarde, los amonitas estaban en guerra con Israel.
\bibverse{5} Mientras los amonitas atacaban a Israel, los ancianos de
Galaad vinieron a buscar a Jefté a la tierra de Tob. \bibverse{6} ``Ven
y sé nuestro comandante del ejército'', le pidieron a Jefté, ``para que
podamos luchar contra los amonitas''.

\bibverse{7} ``¿No fueron ustedes los que me odiaron y me expulsaron de
la casa de mi padre?''Jefté les preguntó: ``¿Por qué vienen a mí ahora
que están en problemas?''.

\bibverse{8} ``Sí, por eso hemos acudido a ti ahora'', le respondieron
los ancianos de Galaad. ``Ven con nosotros a luchar contra los amonitas,
y serás el jefe de todo el pueblo de Galaad''. \bibverse{9} ``Entonces,
si vuelvo con ustedes y lucho contra los amonitas, y el Señor me hace
victorioso, ¿seré su líder?''lepreguntó Jefté a los ancianos de Galaad.

\bibverse{10} ``El Señor será testigo entre nosotros'', respondieron.
``Haremos lo que tú digas''.

\bibverse{11} Así que Jefté se fue con los ancianos de Galaad, y el
pueblo lo nombró su líder y comandante del ejército. Y Jefté repitió
todas sus condiciones ante el Señor en Mizpa.

\bibverse{12} EntoncesJefté envió mensajeros al rey de los amonitas para
preguntarle: ``¿Qué tienes contra mí para que quieras atacar mi
tierra?''

\bibverse{13} El rey de los amonitas respondió a los mensajeros de
Jefté: ``Israel se apoderó de mi tierra cuando vino de Egipto. Se
extendía desde el río Arnón hasta el río Jaboc, y hasta el río Jordán.
Devuélvemela y no habrá combates''.

\bibverse{14} Jefté envió mensajeros al rey de los amonitas
\bibverse{15} para decirle: ``Esta es la respuesta de Jefté: Los
israelitas no tomaron ninguna tierra de Moab ni de los amonitas.
\bibverse{16} Cuando salieron de Egipto, los israelitas atravesaron el
desierto hasta el Mar Rojo y llegaron a Cades. \bibverse{17} Enviaron
mensajeros al rey de Edom, diciendo: ``Por favor, déjanos pasar por tu
país'', pero el rey de Edom se negó a escuchar. También enviaron la
misma petición al rey de Moab, y éste también se negó. Así que se
quedaron en Cades.

\bibverse{18} Finalmente, los israelitas atravesaron el desierto,
evitando las tierras de Edom y Moab. Llegaron al lado oriental de la
tierra de Moab y acamparon al otro lado del río Arnón. Pero no entraron
en el territorio de Moab, pues el río Arnón era su frontera.

\bibverse{19} Entonces los israelitas enviaron mensajeros a Sehón, rey
de los amorreos, que gobernaba desde Hesbón, y le pidieron: ``Por favor,
déjanos pasar por tu tierra hasta nuestro propio país. \bibverse{20}
PeroSehón no confiaba en que los israelitas pasaran por su territorio.
Así que reunió a su ejército, acampó en Yahaza y atacó a los israelitas.
\bibverse{21} Sin embargo, el Señor, el Dios de Israel, entregó a Sehón
y a todo su pueblo a los israelitas, que los derrotaron. Así, los
israelitas se apoderaron de toda la tierra habitada por los amorreos.
\bibverse{22} Ocuparon todo el territorio de los amorreos desde el río
Arnón hasta el río Jaboc, y desde el desierto hasta el río Jordán.

\bibverse{23} Fue el Señor, el Dios de Israel, quien expulsó a los
amorreos delante de su pueblo Israel, así que ¿por qué has de apoderarte
de ella? \bibverse{24} ¿Por qué no se quedan ustedes con lo que les dio
su dios Quemos, y nosotros nos quedamos con lo que nos ha dado el Señor,
nuestro Dios? \bibverse{25} ¿Te crees mucho mejor que Balac, hijo de
Zipor, rey de Moab? ¿Acaso él se peleó con Israel o lo atacó?

\bibverse{26} Hace trescientos años que los israelitas viven en Hesbón,
en Aroer, en sus aldeas y en todos los pueblos de la ribera del río
Arnón. ¿Por qué no los hiciste regresar durante ese tiempo?
\bibverse{27} Yo no he pecado contra ti, pero tú me has hecho mal al ir
a la guerra contra mí. Que el Señor, el Juez, decida hoy entre los
israelitas y los amonitas''.

\bibverse{28} Pero el rey de Amón no prestó atención a lo que decía
Jefté.

\bibverse{29} Entonces el Espíritu del Señor vino sobre Jefté. Pasó por
Galaad y Manasés, y luego por Mizpa de Galaad. Desde allí avanzó para
atacar a los amonitas. \bibverse{30} Jeftéhizo una promesa solemne al
Señor, diciendo: ``Si me haces victorioso sobre los amonitas,
\bibverse{31} dedicaré al Señor todo lo que salga de la puerta de mi
casa para recibirme a mi regreso seguro de la batalla. Lo presentaré
como holocausto''.

\bibverse{32} Jefté avanzó para atacar a los amonitas, y el Señor le dio
la victoria sobre ellos. \bibverse{33} Los derrotó con contundencia,
capturando veinte ciudades desde Aroer hasta los alrededores de Minit,
hasta Abel-Queamín. Así fue como los amonitas fueron conquistados por
los israelitas.

\bibverse{34} CuandoJefté llegó a su casa en Mizpa, su hija salió a
recibirlo con panderetas y bailes. Era su única hija; no tenía ningún
hijo ni hija aparte de ella. \bibverse{35} En cuanto la vio, se rasgó
las vestiduras en agonía y gritó: ``¡Oh, no, hija mía! ¡Me has aplastado
por completo! Me has destruido, pues hice una promesa solemne al Señor y
no puedo echarme atrás''.

\bibverse{36} Ella respondió: ``Padre, has hecho una promesa solemne al
Señor. Haz conmigo lo que prometiste, porque el Señor trajo la venganza
de tus enemigos, los amonitas''.

\bibverse{37} Entonces ella le dijo: ``Sólo déjame hacer esto: déjame
caminar por las colinas durante dos meses con mis amigos y afligirme por
el hecho de que nunca me casaré.''

\bibverse{38} ``Puedes irte'', le dijo él. La envió por dos meses, y
ella y sus amigas se fueron al monte a llorar porque nunca se casaría.
\bibverse{39} Cuando pasaron los dos meses, volvió a su padre, y él hizo
con ella lo que había prometido, y quedó virgen. Este es el origen de la
costumbre en Israel \bibverse{40} de que cada año las jóvenes de Israel
salgan durante cuatro días a llorar en conmemoración de la hija de Jefté
el Galaadita.

\hypertarget{section-11}{%
\section{12}\label{section-11}}

\bibverse{1} Entonces los efraimitas fueron convocados y cruzaron el
Jordán hasta Zafón. Le dijeron a Jefté: ``¿Por qué fuiste a luchar
contra los amonitas sin convocarnos para que fuéramos contigo? Vamos a
quemar tu casa contigo dentro''.

\bibverse{2} ``Yo era un hombre responsable de una gran lucha'',
respondió Jefté. ``Yo y mi pueblo estábamos luchando contra los
amonitas. Cuando te pedí ayuda, no vinieron a salvarme de ellos.
\bibverse{3} Cuando me di cuenta de que no iban a ayudarme, me hice
cargo de mi propia vida y fui a luchar contra los amonitas, y el Señor
me hizo victorioso sobre ellos. Entonces, ¿por qué han venido hoy a
atacarme?''.

\bibverse{4} Jefté convocó a todos los hombres de Galaad y luchó contra
los efraimitas. Los hombres de Galaad los mataron porque los efraimitas
se burlaban de ellos, diciendo: ``Ustedes los galaaditas no son más que
fugitivos que viven entre Efraín y Manasés.''

\bibverse{5} Los galaaditas tomaron el control de los vados sobre el río
Jordán que conducían al territorio de Efraín, y cuando un efraimita que
escapaba\footnote{\textbf{12:5} La palabra es la misma que se usó para
  burlarse de los Galaaditas en el verso anterior. Ahora los efraimitas
  son los ``fugitivos.''}de la batalla venía y pedía: ``Déjenme
cruzar'', los galaaditas le preguntaban: ``¿Eres efraimita?'';si
respondían``No'', \bibverse{6} le decían: ``Di Shibboleth''. Y los que
eran de Efraín, dirían``Sibboleth'' porque ellos no podían pronunciarlo
bien, y así los agarrarían y los matarían allí en los vados del Jordán.
Un total de 42.000 fueron asesinados en esa ocasión.

\bibverse{7} Jefté dirigió a Israel como juez durante seis años. Luego
murió y fue enterrado en una de las ciudades de Galaad.

\bibverse{8} Después de Jefté, Ibzán de Belén dirigió a Israel como
juez. \bibverse{9} Tuvo treinta hijos y treinta hijas. Casó a sus hijas
con hombres de otras tribus, y trajo a treinta esposas de otras tribus
para que se casaran con sus hijos. Ibzán dirigió a Israel como juez
durante siete años. \bibverse{10} LuegoIbzán murió y fue enterrado en
Belén.

\bibverse{11} Después de él, Elón de Zabulón dirigió a Israel como juez
durante diez años. \bibverse{12} Luego murió y fue enterrado en Ajalón,
en el territorio de Zabulón.

\bibverse{13} Después de él, Abdón, hijo de Hilel, de Piratón, dirigió a
Israel como juez. \bibverse{14} Tuvo cuarenta hijos y treinta nietos,
que montaban setenta asnos. Dirigió a Israel como juez durante ocho
años. \bibverse{15} Luego murió y fue enterrado en Piratón, en el
territorio de Efraín, en la región montañosa de los amalecitas.

\hypertarget{section-12}{%
\section{13}\label{section-12}}

\bibverse{1} Los israelitas siguieron haciendo lo malo ante los ojos del
Señor, así que el Señor los entregó a los filisteos para que los
gobernaran durante cuarenta años.

\bibverse{2} En aquel tiempo había un hombre llamado Manoa. Era de la
tribu de Dan y vivía en la ciudad de Zora. Su mujer no podía concebir y
no tenía hijos.

\bibverse{3} El Ángel del Señor se le apareció y le dijo: ``Es cierto
que no podías concebir y no tienes hijos, pero ahora vas a quedar
embarazada y darás a luz un hijo. \bibverse{4} Así que ten cuidado de no
beber vino ni ninguna otra bebida alcohólica, y no comas nada impuro.
\bibverse{5} Vas a quedar embarazada y a tener un hijo cuya cabeza no
debe ser tocada por una navaja de afeitar, porque el niño será un
nazareo, dedicado a Dios desde su nacimiento. Él iniciará el proceso de
salvar a Israel de los filisteos''.

\bibverse{6} La mujer fue y le dijo a su marido: ``Un hombre de Dios
vino a mí. Parecía el Ángel de Dios, realmente aterrador. No le pregunté
de dónde venía, y no me dijo su nombre. \bibverse{7} Pero me dijo: `Vas
a quedar embarazada y darás a luz un hijo. No debes beber vino ni
ninguna otra bebida alcohólica, y no comas nada impuro. Porque el niño
ha de ser nazareo, dedicado a Dios desde su nacimiento hasta el día de
su muerte'\,''.

\bibverse{8} EntoncesManoa oró al Señor: ``Por favor, Señor, que el
hombre de Dios que nos enviaste regrese a nosotros para explicarnos qué
debemos hacer con el niño que va a nacer.''

\bibverse{9} Dios respondió a la petición de Manoa, y el Ángel de Dios
regresó a la mujer mientras ella estaba sentada en el campo. Sin
embargo, su esposo Manoa no estaba con ella. \bibverse{10} Entonces ella
corrió rápidamente a decirle a su marido: ``¡Mira! El hombre que se me
apareció el otro día ha vuelto''.

\bibverse{11} Manoa se levantó, volvió con su mujer y le preguntó:
``¿Eres tú el hombre que le habló a mi mujer antes?''

``Sí, soy yo'', respondió él.

\bibverse{12} EntoncesManoa dijo: ``¡Que se cumpla tu promesa! ¿Qué se
decidirá\footnote{\textbf{13:12} La palabra empleada aquí es la misma
  que la utilizada para describir las decisiones de Débora en 4:5.}para
el niño, y cuál será su vocación?''

\bibverse{13} ``Asegúrate de que tu esposa sea cuidadosa y siga las
instrucciones que le di'', respondió el ángel del Señor. \bibverse{14}
``No debe comer nada que provenga de la vid ni beber vino, ni ninguna
otra bebida alcohólica. No debe comer nada impuro. Tu esposa debe seguir
todo lo que le indiqué''.

\bibverse{15} Manoa le dijo al ángel del Señor: ``Por favor, déjanos
retenerte aquí mientras te preparamos una comida de un cabrito.''

\bibverse{16} El ángel del Señor respondió: ``Me quedaré, pero no comeré
tu comida. Sin embargo, si preparas un holocausto, puedes presentarlo al
Señor''. (Manoa no sabía que era el ángel del Señor).

\bibverse{17} Manoa le preguntó al ángel del Señor: ``¿Cuál es tu
nombre, para que cuando se cumpla tu promesa podamos honrarte?''

\bibverse{18} ``¿Por qué preguntas esto?'', respondió el ángel del
Señor. ``Mi nombre es incomprensible''.

\bibverse{19} Manoa tomó un cabrito y una ofrenda de grano y los
presentó sobre una roca al Señor. Mientras Manoa y su esposa observaban,
el Señor hizo algo sorprendente. \bibverse{20} Mientras la llama del
altar ardía en el cielo, el ángel del Señor ascendió en la llama. Manoa
y su esposa vieron lo que sucedía y cayeron con el rostro en tierra.
\bibverse{21} El ángel del Señor no volvió a aparecer a Manoa ni a su
esposa, y Manoa se dio cuenta de que era el ángel del Señor.

\bibverse{22} ``Definitivamente vamos a morir'', le dijo Maonaa su
esposa, ``¡porque hemos visto a Dios!''

\bibverse{23} Pero su esposa le respondió: ``Si el Señor hubiera querido
matarnos, no habría aceptado nuestro holocausto y nuestra ofrenda de
grano. No nos habría mostrado todas estas cosas, y no habría venido
ahora a anunciarnos esto''.

\bibverse{24} Dio a luz un hijo y lo llamó Sansón. El niño creció, y el
Señor lo bendijo. \bibverse{25} ElEspíritu del Señor comenzó a
impulsarlo\footnote{\textbf{13:25} Literalmente, ``perturbar'',
  ``agitar'' o ``impulsar.''}en MajanéDan, un lugar entre Zora y Estaol.

\hypertarget{section-13}{%
\section{14}\label{section-13}}

\bibverse{1} Un día Sansón fue a Timná, donde le llamó la atención una
joven filistea. \bibverse{2} Volvió a su casa y les dijo a su padre y a
su madre: ``Una mujer filistea en Timná me ha llamado la atención.
Tráiganmela porque quiero casarme con ella.''

\bibverse{3} Pero su padre y su madre le respondieron: ``¿No puedes
encontrar una joven de nuestra tribu o de nuestro propio pueblo? ¿Tienes
que ir donde los filisteos paganos\footnote{\textbf{14:3} Literalmente,
  ``incircuncisos.''} para conseguir una esposa?''

Pero Sansón le dijo a su padre: ``Sólo búscamela, ella es\footnote{\textbf{14:3}
  ``Ella'': En el hebreo aparece con énfasis, indicando la determinación
  de Sansón de que esta mujer en particular se convirtiera en su esposa.}
la que me gusta.''

\bibverse{4} (Su padre y su madre no se daban cuenta de que esto estaba
en los planes del Señor, que buscaba una oportunidad para enfrentarse a
los filisteos, porque en ese momento los filisteos gobernaban sobre
Israel).

\bibverse{5} Sansón fue a Timná con su padre y su madre. Cuando pasaron
por los viñedos de Timna, de repente salió un león joven rugiendo para
atacarlo. \bibverse{6} El Espíritu del Señor se apoderó de él y desgarró
al león con sus propias manos\footnote{\textbf{14:6} ``Con sus propias
  manos'': Literalmente ``pero no había nada en su mano,'' en otras
  palabras, no tenía ningún arma.}, con la misma facilidad con que se
desgarra un cabrito. Pero no le dijo a su padre ni a su madre lo que
había hecho. Luegosiguiósucamino. \bibverse{7} When Samson talked with
the woman and decided she was right for him.

\bibverse{8} Más tarde, cuando Sansón volvió para casarse con ella, se
apartó del camino para buscar el cadáver del león. Dentro del cuerpo
había un enjambre de abejas y su miel. \bibverse{9} Raspó un poco de
miel en sus manos y la comió mientras caminaba. Cuando volvió con su
padre y su madre, les dio un poco y se la comieron. Pero no les dijo que
había tomado la miel del cadáver de un león.\footnote{\textbf{14:9}
  Incluso el mero hecho de tocar cualquier cosa de un cadáver los habría
  hecho ceremonialmente impuros.}

\bibverse{10} Mientras su padre iba a visitar a la mujer, Sansón
organizó una fiesta para beber, porque esa era la costumbre entre los
jóvenes de clase alta. \bibverse{11} Cuando el pueblo filisteo lo vio,
dispuso que treinta hombres lo acompañaran.\footnote{\textbf{14:11} Más
  bien como ``cuidadores'' que como ``asistentes'', ya que parece que
  los filisteos tenían bastante miedo de lo que Sansón pudiera hacer.}

\bibverse{12} ``Dejen que les diga un acertijo'', les dijo Sansón. ``Si
pueden encontrar su significado y explicármelo durante los siete días de
la fiesta, yo les daré treinta mantos de lino y treinta mudas de ropa.
\bibverse{13} Pero si no pueden explicármelo, ustedes me darána mi
treinta mantos de lino y treinta mudas de ropa''.

``De acuerdo'', respondieron. ``¡Queremos oír tu acertijo!''

\bibverse{14} ``La comida salió del que come, y la dulzura salió del
fuerte'', dijo. Tres días después todavía no lo habían resuelto.
\bibverse{15} Alcuarto\footnote{\textbf{14:15} ``Cuarto'': Según la
  Septuaginta. En el hebreo dice ``séptimo.''} díase acercaron a la
mujer de Sansón y le dijeron: ``Utiliza tus encantos para que tu marido
te explique el acertijo y luego nos lo cuentes, o te quemaremos a ti y a
toda tu familia hasta la muerte. ¿Nos has traído aquí sólo para
robarnos?''

\bibverse{16} Entonces la mujer de Sansón fue llorando hacia él,
diciendo: ``¡Realmente me odias, no es así! ¡No me amas en absoluto! Has
planteado un enigma a mi pueblo, pero ni siquiera me lo has explicado''.

``¿Y qué?'', respondió él. ``¡Ni siquiera se lo he explicado a mi padre
o a mi madre! ¿Por qué debería explicártelo a ti?''.

\bibverse{17} Ella lloró delante de él durante todo el tiempo que duró
la fiesta, y al final, al séptimo día, se lo explicó porque ella lo
regañaba mucho. Entonces les explicó el significado del acertijo a los
jóvenes filisteos.

\bibverse{18} Antes de que se pusiera el sol del séptimo día, los
hombres de la ciudad se acercaron a Sansón y le dijeron: ``¿Qué es más
dulce que la miel? ¿Qué es más fuerte que un león?''

``Si no hubieran usado mi vaca para arar, no habrían descubierto el
significado de mi acertijo'', respondió Sansón.

\bibverse{19} El Espíritu del Señor se apoderó de él y se dirigió a
Ascalón, mató a treinta de sus hombres, tomó sus ropas y se las dio a
los que habían explicado el enigma. Furioso, Sansón devolvió a la casa
de su padre. \bibverse{20} La mujer de Sansón fue entregada a su
padrino, que lo había acompañado en su boda.

\hypertarget{section-14}{%
\section{15}\label{section-14}}

\bibverse{1} Algún tiempo después, cuando se estaba cosechando el trigo,
Sansón fue a visitar a su mujer, llevando consigo un cabrito de regalo.
``Quiero ir a ver a mi mujer a su habitación'', le dijo al
llegar,\footnote{\textbf{15:1} ``Al llegar'': Añadido para mayor
  claridad.} pero su padre no lo dejó entrar.

\bibverse{2} ``Pensé que la odiabas por completo y por eso se la di a tu
padrino'', le dijo a Sansón. ``Pero su hermana menor es aún más
atractiva; ¿por qué no te casas con ella en su lugar?''.

\bibverse{3} ``Esta vez no se me puede culpar por los problemas que le
voy a causar a los filisteos'', declaró Sansón. \bibverse{4} Entonces
fue y atrapó trescientas zorras y les ató las colas, de dos en dos.
\bibverse{5} Ató una antorcha a cada una de las colas atadas y les
prendió fuego. Luego las soltó en los campos de cereales de los
filisteos y prendió fuego a todo el grano, cosechado y no cosechado, así
como a los viñedos y olivares.

\bibverse{6} ``¿Quién ha hecho esto?'', preguntaron los filisteos. ``Fue
Sansón, el yerno del hombre de Timná'', les dijeron. ``Ese hombre le dio
la mujer de Sansón al padrino de Sansón''. Entonces los filisteos fueron
y la quemaron a ella y a su padre hasta la muerte.

\bibverse{7} Sansón les dijo: ``¡Si así van a actuar, no pararé hasta
vengarme de ustedes!'' \bibverse{8} Entonces los atacó
violentamente,\footnote{\textbf{15:8} ``Los atacó violentamente'':
  Literalmente, ``les golpeó la cadera y el muslo,'' que quiere decir:
  ``completamente.''}matándolos, y luego se fue a vivir a una cueva en
la roca de Etam.

\bibverse{9} Entonces el ejército filisteo llegó y acampó en Judá,
preparado para la batalla cerca de Lehi. \bibverse{10} El pueblo de Judá
preguntó: ``¿Por qué nos han invadido?''

``¡Hemos venido a capturar a Sansón, para hacerle lo mismo que nos ha
hecho a nosotros!'', respondieron.

\bibverse{11} Tres mil hombres de Judá fueron a la cueva de la roca de
Etam y le preguntaron a Sansón: ``¿No entiendes que los filisteos nos
dominan? ¿Qué crees que estás haciendo con nosotros?''

``Sólo hice lo que ellos me hicieron a mí'', respondió.

\bibverse{12} ``Pues bien, hemos venido a tomarte prisionero y a
entregarte a los filisteos'', le dijeron.

``Sólo júrenme que no me van a matar ustedes'', respondió Sansón.

\bibverse{13} ``No, no lo haremos'', le aseguraron. ``Sólo te ataremos y
te entregaremos a los filisteos. Desde luego, no te vamos a matar''. Lo
ataron con dos cuerdas nuevas y lo sacaron de la roca.

\bibverse{14} Cuando Sansón se acercó a Lejí, los filisteos corrieron
hacia él, gritándole. Pero el Espíritu del Señor lo invadió, y las
cuerdas que le ataban los brazos se debilitaron como el lino quemado, y
sus manos se soltaron. \bibverse{15} Agarró la mandíbula
fresca\footnote{\textbf{15:15} En otras palabras, el hueso no estaba
  seco y quebradizo.}de un asno y con ella mató a mil filisteos.

\bibverse{16} Entonces Sansón declaró: ``Con la quijada de un burro he
amontonado a los muertos. Con la quijada de un burro he matado a mil
hombres''.

\bibverse{17} Cuando Sansón terminó su discurso, tiró la quijada y llamó
al lugar Colina de la Quijada. \bibverse{18} Ahora tenía mucha sed, y
Sansón clamó al Señor diciendo: ``Tú has logrado esta asombrosa
victoria\footnote{\textbf{15:18} Literalmente, ``salvación.''}por medio
de tu siervo, pero ¿ahora tengo que morir de sed y ser capturado por los
paganos?''

\bibverse{19} Entonces Dios abrió una hondonada en Lejí, y salió agua de
ella. Sansón bebió, recuperó las fuerzas y se sintió mucho mejor. Por
eso le puso el nombre de Manantial del que clama, y hasta el día de hoy
sigue allí en Lejí.

\bibverse{20} Sansón dirigió a Israel como juez por veinte años durante
el tiempo de los filisteos.

\hypertarget{section-15}{%
\section{16}\label{section-15}}

\bibverse{1} Sansón fue a Gaza. Allí vio a una prostituta y fue a
acostarse con ella esa noche. \bibverse{2} Los hombres de Gaza se
enteraron de que Sansón estaba allí, así que se reunieron para pasar la
noche acechándolo a las puertas de la ciudad. Estuvieron callados toda
la noche, susurrando entre ellos: ``Lo mataremos cuando amanezca''.

\bibverse{3} Pero Sansón sólo se quedó hasta la mitad de la noche.
Agarró las puertas de la ciudad junto con sus dos postes y las arrancó,
junto con la barra de la cerradura. Se las puso sobre los hombros y las
llevó a la colina frente a Hebrón.\footnote{\textbf{16:3} De Gaza a
  Hebrón hay unas 40 millas.}

\bibverse{4} Más tarde se enamoró de una mujer llamada Dalila que vivía
en el valle de Sorec. \bibverse{5} Los jefes filisteos se acercaron a
ella y le dijeron: ``A ver si puedes seducirlo y conseguir que te
muestre el secreto de su increíble fuerza, y averiguar cómo podemos
dominarlo y atarlo para que no pueda hacer nada. Todos te daremos mil
cien siclos de plata cada uno''.

\bibverse{6} Dalila fue y le suplicó a Sansón: ``Por favor, dime de
dónde viene tu increíble fuerza y qué se puede usar para atarte y que no
puedas hacer nada.''

\bibverse{7} ``Si me atan con siete cuerdas flexibles que no se hayan
secado, me volveré igual de débil'', le dijo Sansón.

\bibverse{8} Los jefes filisteos le trajeron siete cuerdas de arco
flexibles que no se habían secado, y ella lo ató con ellas. \bibverse{9}
Después de hacer que los hombres se escondieran en su habitación, listos
para atacarlo, ella gritó: ``¡Sansón, los filisteos han venido a por
ti!'' Pero él rompió las cuerdas del arco como se rompe un hilo cuando
lo toca una llama. Así que nadie supo de dónde provenía su fuerza.

\bibverse{10} Más tarde, Dalila le dijo a Sansón: ``¡Me has hecho quedar
como una estúpida, diciéndome estas mentiras! Así que ahora, por favor,
dime qué se puede usar para atarte''.

\bibverse{11} ``Si me atan bien con cuerdas nuevas que no se hayan usado
antes, me debilitaré como cualquier otro'', le dijo él.

\bibverse{12} Así que Dalila consiguió unas cuerdas nuevas y lo ató con
ellas. Gritó: ``¡Sansón, los filisteos han venido a por ti!''. Como
antes, los hombres se escondieron en su habitación. Pero de nuevo Sansón
rompió las cuerdas de sus brazos como si fueran finos hilos.

\bibverse{13} Dalila le dijo a Sansón: ``¡Sigues haciéndome quedar como
una estúpida, diciéndome estas mentiras! Sólo dime qué se puede usar
para atarte''.

``Si tejieras las siete trenzas de mi cabello en la red del telar y las
apretaras con el alfiler, me volvería tan débil como cualquier otro'',
le dijo él. Así que, mientras él dormía, Dalila tomó las siete trenzas
de su cabeza, las tejió en la red, \bibverse{14} y apretó el alfiler.
Gritó: ``¡Sansón, los filisteos han venido a por ti!''. Pero Sansón se
despertó y arrancó del telar tanto el alfiler como la
telaraña.\footnote{\textbf{16:14} El hebreo de este verso y del
  siguiente aparecen dañados. Aquí se utiliza la versión de la
  Septuaginta.}

\bibverse{15} Entonces Dalila se quejó con Sansón: ``¿Cómo puedes
decirme: `Te amo', cuando no me tienes confianza?\footnote{\textbf{16:15}
  ``No me tienes confianza'': Literalmente, ``tu corazón no está
  conmigo.''}¡Tres veces me has hecho quedar como una estúpida, sin
decirme de dónde viene tu increíble fuerza!''

\bibverse{16} Ella se quejaba y lo regañaba todo el tiempo,
fastidiándolo hasta que deseó morir. \bibverse{17} Finalmente, Sansón le
confió todo. ``Nunca me he cortado el pelo, porque estoy dedicado como
nazareo a Dios desde mi nacimiento. Si me afeitan, mi fuerza me
abandonará, y me volveré tan débil como cualquier otro''.

\bibverse{18} Dalila se dio cuenta de que realmente le había confiado y
compartido todo, y envió un mensaje a los líderes filisteos diciéndoles:
``Vuelvan una vez más, porque esta vez me ha confiado y me ha contado
todo.'' Así que los líderes filisteos regresaron, trayendo consigo el
dinero para dárselo a ella.

\bibverse{19} Dalila lo calmó durmiendo en su regazo y luego llamó a
alguien para que le afeitara las siete trenzas de pelo. Empezó a
atormentarlo, pero él no pudo hacer nada, pues le abandonaron las
fuerzas. \bibverse{20} Ella gritó: ``¡Sansón, los filisteos han venido a
por ti!''.

Sansón se despertó y pensó: ``Haré como antes y me liberaré''. Pero no
sabía que el Señor lo había abandonado.

\bibverse{21} Los filisteos lo agarraron y le sacaron los ojos. Luego lo
llevaron a Gaza, donde lo encarcelaron con cadenas de bronce. Lo
obligaron a trabajar moliendo grano en el molino de la prisión.

\bibverse{22} Pero su cabello comenzó a crecer de nuevo después de haber
sido afeitado.

\bibverse{23} Los jefes filisteos se reunieron en una gran fiesta
religiosa para sacrificar a su dios Dagón y celebrar, diciendo:
``¡Nuestro dios nos ha entregado a Sansón, nuestro enemigo!''

\bibverse{24} Cuando el pueblo lo vio, alabó a su dios y dijo: ``Nuestro
dios nos ha entregado a nuestro enemigo, el que devastó nuestra tierra y
mató a tantos de nosotros.''

\bibverse{25} Cuando empezaron a emborracharse, gritaron: ``¡Convoca a
Sansón para que nos entretenga!'' Así que llamaron a Sansón de la cárcel
para que los entretuviera, y lo hicieron colocarse entre las dos
columnas principales del edificio.

\bibverse{26} Sansón dijo al criado que lo llevaba de la mano: ``Déjenme
junto a las columnas sobre las que se apoya el templo para que pueda
palparlas y apoyarme en ellas.'' \bibverse{27} El templo estaba lleno de
gente. Todos los gobernantes filisteos estaban allí, y en el techo
estaba la gente común observando lo que hacía Sansón.

\bibverse{28} Sansón clamó al Señor: ``Señor Dios, por favor acuérdate
de mí y dame fuerzas. Por favor, Dios, hazlo sólo una vez más, para que
con un acto pueda pagar a los filisteos en venganza por la pérdida de
mis dos ojos''. \bibverse{29} Sansón se acercó a los dos pilares
centrales que sostenían el templo. Con la mano derecha apoyada en uno de
los pilares y la izquierda en el otro, \bibverse{30} Sansón gritó:
``¡Dejadme morir con los filisteos!'' y empujó con todas sus fuerzas. El
templo se derrumbó sobre los gobernantes y toda la gente que estaba en
él. Así, los muertos a su muerte fueron más que los que él mató en vida.

\bibverse{31} Luego vinieron sus hermanos y toda su familia, lo llevaron
de vuelta y lo enterraron entre Zora y Estaol, en la tumba de su padre
Manoa. Dirigió a Israel como juez durante veinte años.

\hypertarget{section-16}{%
\section{17}\label{section-16}}

\bibverse{1} Un hombre llamado Miqueas, de la región montañosa de Efraín
\bibverse{2} , le dijo a su madre: ``Esos mil cien siclos de plata que
te robaron y que te oí maldecir, yo tengo la plata. Yo fui quien la
tomó''.

Entonces su madre le dijo: ``¡Hijo mío, que el Señor te
bendiga!''\footnote{\textbf{17:2} Esto puede significar que la madre
  intentaba neutralizar la maldición con una bendición, ya que afectaba
  a su hijo, o que se alegraba de que hubiera reconocido el robo.}

\bibverse{3} Élle devolvió a su madre los mil cien siclos de plata. Su
madre anunció: ``Dedico este dinero por completo al Señor. Se lo voy a
entregar a mi hijo para que haga tallar un ídolo, una imagen hecha con
plata fundida.\footnote{\textbf{17:3} No está claro si se refiere a dos
  objetos o a uno. Véase 18:17, que parece sugerir dos objetos, mientras
  que 18:20 y 18:31 se refieren a uno solo.}Así que ahora te lo
devuelvo.''

\bibverse{4} Después de devolverle la plata a su madre, ella le dio
doscientos siclos a un platero que los convirtió en un ídolo tallado,
una imagen hecha con plata fundida. Los guardó en la casa de Miqueas.
\bibverse{5} Miqueas hizo construir un santuario para el ídolo. También
hizo un efod y algunos dioses domésticos, y ordenó a uno de sus hijos
como sacerdote. \bibverse{6} En aquel tiempo Israel no tenía un rey:
cada uno hacía lo que le parecía correcto.\footnote{\textbf{17:6} Esto
  es exactamente lo contrario de la frase habitual ``hizo lo que era
  correcto a los ojos del Señor''. En lugar de un elogio, esto debe
  verse como la ``democratización de la maldad''. La misma expresión se
  utiliza en 21:25.}

\bibverse{7} Un joven, levita de la tribu de Judá,\footnote{\textbf{17:7}
  No está claro cómo este hombre podía pertenecer tanto a la tribu de
  Leví como a la tribu de Judá, a menos que sus padres fueran de tribus
  diferentes.} que vivía en Belén de Judá, \bibverse{8} se fue deBelén
para buscar otro lugar donde vivir. Mientras viajaba por la región
montañosa de Efraín, llegó a la casa de Miqueas.

\bibverse{9} ``¿De dónde eres? le preguntó Miqueas.

``Soy un levita de Belén de Judá'', respondió el hombre. ``Estoy
buscando un lugar para vivir''.

\bibverse{10} ``Ven y quédate aquí conmigo. Puedes ser mi `padre' y
sacerdote, y te daré diez siclos de plata al año, además de tu ropa y
comida''. Así que el levita entró \bibverse{11} y aceptó quedarse con
él. El joven se convirtió en un hijo para él. \bibverse{12} Miqueas
ordenó al levita como su propio sacerdote y vivió en la casa de Miqueas.

\bibverse{13} ``Estoy seguro de que el Señor me bendecirá ahora, porque
tengo un levita como sacerdote'', concluyó Miqueas.

\hypertarget{section-17}{%
\section{18}\label{section-17}}

\bibverse{1} En aquella época Israel no tenía rey. La tribu de Dan
buscaba un territorio en el que poder vivir, porque hasta entonces no
habían conseguido la posesión de la tierra que se les había concedido
entre las tribus de Israel. \bibverse{2} Así que los danitas eligieron
de entre ellos a cinco hombres destacados de Zora y Estaol para que
recorrieran la tierra y la exploraran.

``Vayan y exploren la tierra'', les dijeron. Cuando los hombres llegaron
a la región montañosa de Efraín, llegaron a la casa de Miqueas, donde
pasaron la noche. \bibverse{3} Mientras estaban allí, reconocieron el
acento del joven levita, así que se dirigieron a él y le preguntaron:
``¿Quién te ha traído aquí y qué haces en este lugar? ¿Por qué estás
aquí?''

\bibverse{4} ``Miqueas me arregló las cosas y me contrató como
sacerdote'', les dijo.

\bibverse{5} ``Por favor, pide al Señor por nosotros para saber si
nuestro viaje tendrá éxito'', le pidieron.

\bibverse{6} ``Id en paz'', respondió el sacerdote. ``El viaje que están
haciendo está siendo observado por el Señor.''\footnote{\textbf{18:6}
  Nótese que el sacerdote no está declarando el éxito o lo contrario.
  Literalmente dice que el viaje ``está delante del Señor'', lo que
  puede significar o bien que el Señor está guiando, o bien que el Señor
  está escudriñando sus acciones. La rápida respuesta del sacerdote
  también hace sospechar que en realidad no dedicó mucho tiempo a pedir
  una respuesta al Señor.}

\bibverse{7} Los cinco hombres partieron y se dirigieron a la ciudad de
Lais. Observaron que el pueblo de allí vivía con seguridad y seguía las
costumbres de los sidonios. El pueblo estaba desprevenido y confiado en
su seguridad, en su casa, en una tierra productiva. No tenían un
gobernante fuerte, vivían muy lejos de los sidonios y no tenían otros
aliados que los ayudaran.\footnote{\textbf{18:7} ``Que lo ayudaran.''
  Añadido para mayor claridad. Los espías, obviamente, se preocupan por
  saber quién podría acudir en ayuda de esta ciudad si fuera atacada. La
  distancia de Sidón y la falta de alianzas aparentes los animaba a
  pensar que un ataque tendría éxito. Además, la falta de un
  ``gobernante fuerte'' (literalmente, ``un poseedor de moderación'')
  significaba que la defensa de la ciudad no estaría a cargo de un
  poderoso comandante militar.}

\bibverse{8} Los hombres regresaron a Zora y Estaol, y sus parientes les
preguntaron: ``¿Qué han hecho?''

\bibverse{9} ``¡Vamos a atacarlos!'', interrumpieron los hombres.
``¡Hemos inspeccionado el terreno y es excelente! ¿No harán algo? ¡No se
tarden en ir y ocupar el terreno! \bibverse{10} Cuando lleguen allí
verán que la gente es desprevenida y que la tierra es extensa. Dios te
ha dado un lugar donde no falta nada''.

\bibverse{11} Así que seiscientos hombres armados danitas salieron de
Zora y Estaol, listos para atacar. \bibverse{12} En el camino acamparon
en Quiriat-Yearín, en Judá. Por eso el lugar al oeste de Quiriat-Yearín
se llama hasta hoy el Campo de Dan. \bibverse{13} Luego partieron de
allí y se dirigieron a la región montañosa de Efraín y llegaron a la
casa de Miqueas.

\bibverse{14} Entonces los cinco hombres que habían ido a explorar la
tierra de Lais dijeron a los demás miembros de la tribu: ``¿Se dan
cuenta de que aquí, en estas casas, hay un efod, dioses domésticos y un
ídolo tallado, una imagen hecha con plata fundida? Así que ya saben lo
que deben hacer''. \bibverse{15} Los cinco hombres dejaron el camino y
fueron a donde vivía el joven levita en la casa de Miqueas para
preguntar cómo estaba. \bibverse{16} Los seiscientos hombres armados
danitas estaban a la entrada, junto a la puerta. \bibverse{17} Los cinco
hombres entraron y tomaron el ídolo tallado, el efod, los ídolos
domésticos y la imagen hecha con plata fundida. El sacerdote estaba
junto a la puerta con los seiscientos hombres armados.

\bibverse{18} Cuando el sacerdote los vio llevarse todos los objetos
religiosos\footnote{\textbf{18:18} El texto repite los elementos
  enumerados en el versículo 17.}de la casa de Miqueas, les preguntó:
``¿Qué están haciendo?''

\bibverse{19} ``¡Cállense! ¡No digan nada! Vengan con nosotros, y podrán
ser nuestro `padre' y sacerdote. ¿No sería mejor para ti que en lugar de
ser sacerdote de la casa de un solo hombre fueras el sacerdote de una
tribu y una familia israelita?''

\bibverse{20} Esto le pareció una buena idea al sacerdote y se fue con
ellos. Llevando el efod, los ídolos de la casa y la imagen hecha con
plata fundida, marchó con el pueblo a su alrededor. \bibverse{21}
Continuaron su viaje, poniendo por delante a sus hijos, su ganado y sus
posesiones.

\bibverse{22} Losdanitas ya estaban bastante lejos de la casa de Miqueas
cuando los hombres del pueblo de Miqueas los alcanzaron, \bibverse{23}
gritándoles. Los danitas se volvieron para mirarlos y le preguntaron a
Miqueas: ``¿Qué te pasa? ¿Por qué llamas a estos hombres para que vengan
a por nosotros?''

\bibverse{24} ``Se han robado los dioses que hice, y también a mi
sacerdote, y luego se han ido. ¿Qué me han dejado? ¿Cómo pueden
preguntarme: `Qué te pasa'?''

\bibverse{25} ``¡No te quejes con nosotros!'' contestaron los danitas.
``¡Algunos de los que están de mal humor aquí podrían atacarte y tú y tu
familia perderían la vida!'' \bibverse{26} Los danitas siguieron su
camino. Miqueas vio que eran demasiado fuertes para luchar, así que se
dio la vuelta y regresó a su casa.

\bibverse{27} Así que los danitas se llevaron los ídolos que había hecho
Miqueas, así como a su sacerdote. Atacaron a Lais con su gente pacífica
y desprevenida, los mataron a espada y quemaron la ciudad. \bibverse{28}
Nadie pudo salvarlos porque estaban muy lejos de Sidón y no tenían otros
aliados que los ayudaran. La ciudad estaba en el valle que pertenece a
Bet-rejob. Los danitas reconstruyeron la ciudad y vivieron allí.
\bibverse{29} Cambiaron el nombre de la ciudad a Dan, en honor a su
antepasado, el hijo de Israel. Laish era su nombre anterior.
\bibverse{30} Losdanitas erigieron el ídolo tallado para adorarlo, y
Jonatán, hijo de Gersón, hijo de Moisés, y sus hijos se convirtieron en
sacerdotes de la tribu de Dan hasta el momento en que el pueblo salió
del país en cautiverio. \bibverse{31} Ellos adoraron el ídolo tallado
que Miqueas había hecho todo el tiempo que el Templo\footnote{\textbf{18:31}
  Aunque no se hace ninguna referencia específica a la construcción de
  un Templo en Silo, se cree que allí existía una estructura más
  permanente, ya que de lo contrario se habría hablado del lugar como
  ``Tienda de la Reunión''. El relato del principio de 1 Samuel apoya
  esta opinión.} de Dios estuvo en Silo.

\hypertarget{section-18}{%
\section{19}\label{section-18}}

\bibverse{1} En aquella época Israel no tenía rey. Un levita que vivía
en una zona remota de la región montañosa de Efraín se casó con una
esposa concubina\footnote{\textbf{19:1} En otras palabras, una esposa de
  ``segunda clase'', que no tenía el estatus social de una verdadera
  esposa.} de Belén de Judá. \bibverse{2} Pero ella le fue
infiel\footnote{\textbf{19:2} ``Infiel'': Literalmente, ``actuaba como
  una prostituta''. Sin embargo, algunas versiones antiguas tienen ``se
  enojó con él.''}y lo abandonó, volviendo a la casa de su padre en
Belén. Allí se quedó durante cuatro meses.

\bibverse{3} Entonces su marido fue tras ella para hablarle con
amabilidad y traerla de vuelta a casa. Con él iba su criado y dos
burros. Lo llevó a la casa de su padre y, cuando éste lo conoció, lo
acogió con gusto. \bibverse{4} Su padre le presionó para que se quedara
con ellos, así que se quedó tres días, comiendo, bebiendo y durmiendo
allí. \bibverse{5} Al cuarto día, él y su concubina se levantaron
temprano por la mañana y se prepararon para irse, pero el padre de ella
le dijo a su yerno: ``Te sentirás mejor si comes algo antes de irte''.
\bibverse{6} Así que los dos hombres se sentaron a comer y beber juntos.
El padre le dijo a su yerno: ``¡Accede a pasar otra noche aquí, y podrás
disfrutar!'' \bibverse{7} El hombre se levantó para irse, pero su suegro
le presionó para que se quedara, así que al final pasó la noche allí.

\bibverse{8} Al quinto día se levantó de madrugada para marcharse. Pero
su suegro le dijo: ``Come antes de irte, y luego vete por la tarde''.
Así que comieron juntos. \bibverse{9} Cuando se levantó para irse con su
concubina y su criado, su suegro le dijo: ``Mira, es tarde, ya es de
noche. Pasa la noche aquí. El día está a punto de terminar. Quédate aquí
la noche y diviértete, y mañana podrás levantarte temprano y volver a
casa''.

\bibverse{10} Pero el hombre no quería pasar otra noche, así que se
levantó y se fue. Se dirigió hacia la ciudad de Jebús (ahora llamada
Jerusalén) con sus dos asnos ensillados y su concubina. \bibverse{11}
Cuando se acercaban a Jebús, el siervo le dijo a su amo: ``Señor, ¿por
qué no nos detenemos aquí, en esta ciudad jebusea, para pasar la
noche?''

\bibverse{12} Pero su amo le contestó: ``No, no vamos a detenernos en
esta ciudad donde sólo viven extranjeros y ningún israelita. Seguiremos
hasta Gabaa''. \bibverse{13} Entonces le dijo a su siervo: ``Vamos,
tratemos de llegar a Gabaa o a Ramá y pasemos la noche en algún lugar''.
\bibverse{14} Así que siguieron adelante y llegaron a Guibeá, en el
territorio de Benjamín, justo cuando se ponía el sol. \bibverse{15} Se
detuvieron en Gabaa para pasar la noche, y se sentaron en la plaza
principal del pueblo, pero nadie los invitó a quedarse.

\bibverse{16} Pero más tarde, esa misma noche, llegó un anciano que
volvía de trabajar en el campo. Era de la región montañosa de Efraín,
pero ahora vivía en Gabaa, en el territorio de Benjamín. \bibverse{17}
Se asomó y se fijó en el viajero de la plaza y le preguntó: ``¿Adónde
van y de dónde vienen?''

\bibverse{18} ``Venimos de Belén, en Judá, y nos dirigimos a una zona
remota en la región montañosa de Efraín'', respondió el hombre. ``Soy de
allí y fui a Belén, y ahora voy al Templo del Señor.\footnote{\textbf{19:18}
  ``Al templo del Señor'': La Septuaginta dice: ``a mi casa.''}Aquí
nadie me ha invitado a quedarme. \bibverse{19} Hay paja y comida para
nuestros burros, y nosotros, tus siervos, tenemos pan y vino, suficiente
para mí, la mujer y mi siervo. Tenemos todo lo que necesitamos''.

\bibverse{20} ``Son bienvenidos a quedarse conmigo'', respondió el
hombre. ``Puedo dejarles todo lo que necesiten. Pero no pases la noche
aquí en la plaza''. \bibverse{21} Lo llevó a su casa y dio de comer a
los burros. Los viajeros se lavaron los pies y luego se pusieron a comer
y beber.

\bibverse{22} Mientras se divertían, llegaron unos depravados del
pueblo, rodearon la casa y golpearon la puerta, gritando al anciano
dueño de la casa: ``Saca al hombre que vino a quedarse en tu casa para
que tengamos sexo con él.''

\bibverse{23} El dueño de la casa salió y les dijo: ``¡Hermanos míos, no
actúen con tanta maldad! Este hombre es un invitado en mi casa. ¡No
cometan semejante acto tan repugnante! \bibverse{24} Miren, aquí están
mi hija virgen y la concubina de ese hombre. Dejen que las saque y
podrán violarlas y hacer con ellas lo que quieran. Pero no hagan algo
tan repugnante con este hombre''.

\bibverse{25} Pero los hombres se negaron a escuchar, así que el hombre
agarró a su concubina y se la echó fuera. La violaron y abusaron de ella
toda la noche hasta la mañana, y sólo la abandonaron al amanecer.
\bibverse{26} Cuando la noche se convirtió en día, ella regresó a la
casa donde estaba su amo y se desplomó frente a la puerta en la mañana.

\bibverse{27} Su amo se levantó por la mañana y abrió la puerta de la
casa. Salió para continuar su viaje y allí estaba su concubina, tendida
en la puerta de la casa, con las manos agarradas al umbral.

\bibverse{28} ``Levántate, vamos'', le dijo, pero no hubo respuesta.
Entonces el hombre la subió a su burro y se fue a su casa. \bibverse{29}
Cuando llegó a su casa, tomó un cuchillo y, aferrándose a su concubina,
la cortó, miembro por miembro, en doce pedazos, y envió estos pedazos a
todas las partes de Israel. \bibverse{30} Todos los que la
vieron\footnote{\textbf{19:30} ``la'': La mayoría de las traducciones
  usan ``lo'', pero el hebreo literal dice: ``Todos los que veían
  dijeron\ldots{}'' Está claro que aquí se habla de las partes
  desmembradas del cuerpo de la concubina, por lo que el pronombre
  femenino tiene más sentido.} dijeron:``Nunca se había visto nada
igual, desde que los israelitas salieron de Egipto hasta ahora.
¡Deberían consideraresto que le pasó a ella! ¡Decidan qué vanhacer!
¡Hablen ahora!''

\hypertarget{section-19}{%
\section{20}\label{section-19}}

\bibverse{1} Todos los israelitas, desde Dan hasta Beerseba, incluyendo
la tierra de Galaad, fueron y se reunieron en Mizpa ante el Señor. La
asamblea estaba unida en su propósito. \bibverse{2} Los líderes de todo
el pueblo de cada tribu israelita tomaron sus posiciones asignadas en el
ejército reunido del pueblo de Dios, cuatrocientos mil soldados armados
con espadas. \bibverse{3} La tribu de Benjamín se enteró de que los
israelitas se habían reunido en Mizpa. Los israelitas preguntaron:
``Díganos, ¿cómo pudo ocurrir un acto tan horrendo?''.

\bibverse{4} El levita, esposo de la mujer asesinada, explicó: ``Yo y mi
concubina vinimos a pasar la noche a la ciudad de Guibeá, en el
territorio de Benjamín. \bibverse{5} Los dirigentes de Guibeá vinieron a
atacarme por la noche. Rodearon la casa con la intención de
matarme.\footnote{\textbf{20:5} Es interesante que el levita pase por
  alto la pretendida violación homosexual y en su lugar afirme que se
  trata de un caso de intento de asesinato.}Violaron a mi concubina y
ésta murió. \bibverse{6} Tomé a mi concubina y la corté en pedazos, y
envié estos pedazos a todas las partes del país que habían sido
entregadas a Israel, porque esos hombres habían hecho algo vergonzoso y
repugnante en Israel. \bibverse{7} ¡Así que todos ustedes, israelitas,
tienen que decidir aquí y ahora qué van a hacer al respecto!''

\bibverse{8} Todos se pusieron de pie y declararon unidos: ``¡Ninguno de
nosotros va a volver a su casa, ni a sus tiendas! Ninguno de nosotros
volverá a sus casas. \bibverse{9} Esto es lo que vamos a hacer a Guibeá:
la atacaremos con nuestras fuerzas elegidas por sorteo. \bibverse{10}
Tomaremos diez hombres de cien de todas las tribus israelitas, luego
cien de mil, y luego mil de diez mil, para organizar la comida del
ejército, de modo que cuando las tropas lleguen a Guibeá, en Benjamín,
puedan pagarles por todas estas cosas repugnantes que han hecho en
Israel.''

\bibverse{11} Todos los hombres de Israel se pusieron de acuerdo y se
reunieron para atacar la ciudad. \bibverse{12} Las tribus israelitas
también enviaron hombres a todo el territorio de Benjamín, preguntando a
la gente: ``¿Qué están haciendo con este terrible mal que ha ocurrido
entre ustedes? \bibverse{13} ¡Entreguen a estos malvados para que
podamos ejecutarlos y deshacernos de este mal de Israel!'' Pero los
benjamitas se negaron a escuchar lo que sus compañeros israelitas tenían
que decir. \bibverse{14} Salieron de sus ciudades y se reunieron en
Guibeá para ir a luchar contra los demás israelitas. \bibverse{15} Ese
día se convocó a un total de veintiséis mil hombres armados con espadas
de las ciudades de Benjamín, además de los setecientos guerreros
experimentados de Guibeá. \bibverse{16} Formaban parte de este ejército
setecientos soldados experimentados que usaban la mano izquierda. Todos
ellos podían disparar una honda y no fallar ni por asomo. \bibverse{17}
El ejército israelita (excluyendo a Benjamín) contaba con cuatrocientos
mil guerreros experimentados, todos armados con espadas.

\bibverse{18} Los israelitas fueron a Betel y le preguntaron a Dios:
``¿Quiénes de nosotros deben ser los primeros en ir a luchar contra los
benjamitas?''

``Judá debe ir primero'', respondió el Señor. \bibverse{19} A la mañana
siguiente, los israelitas salieron y acamparon cerca de Guibeá.
\bibverse{20} Luego salieron a la batalla con el ejército de Benjamín,
tomando sus posiciones para atacar Guibeá. \bibverse{21} Pero los
benjamitas salieron de Guibeá y mataron a veintidós mil israelitas en el
campo de batalla ese día.

\bibverse{22} Pero los israelitas se animaron unos a otros a tener
confianza, y tomaron las mismas posiciones que tenían el primer día.
\bibverse{23} Los israelitas fueron y clamaron ante el Señor hasta el
atardecer y preguntaron: ``¿Debemos ir a atacar de nuevo a los
benjamitas, nuestros parientes?''

``Vayan y atáquenlos'', respondió el Señor.

\bibverse{24} Así que el segundo día avanzaron para atacar al ejército
de Benjamín. \bibverse{25} Sin embargo, los benjamitas salieron de
Guibeá una vez más y mataron a dieciocho mil israelitas, todos armados
con espadas.

\bibverse{26} Entonces todos los israelitas y todo su ejército fueron a
Betel y se sentaron allí a llorar ante el Señor. Ese día ayunaron hasta
la noche y dieron holocaustos y ofrendas de comunión al Señor.
\bibverse{27} Los israelitas le preguntaron al Señor qué debían hacer.
En aquel tiempo se guardaba allí el Arca del Acuerdo de Dios.
\bibverse{28} Finés, hijo de Eleazar y nieto de Aarón, era el sacerdote.
Los israelitas le preguntaron al Señor: ``¿Debemos ir y luchar de nuevo
contra nuestros parientes de Benjamín, o no?''

``¡Sí, vayan! Mañana se los entregaré'', respondió el Señor.

\bibverse{29} Entonces los israelitas prepararon una emboscada alrededor
de Guibeá. \bibverse{30} Al tercer día tomaron las mismas posiciones de
antes. \bibverse{31} Los benjamitas salieron a atacarlos y se alejaron
de la ciudad mientras comenzaban a matar a los israelitas como lo habían
hecho antes. Unos treinta israelitas murieron en el campo de batalla y a
lo largo de los caminos, el que va hacia Betel y el que vuelve hacia
Guibeá.

\bibverse{32} ``Los estamos derrotando, como antes'', gritaron los
benjamitas.

Pero los israelitas dijeron: ``Huyamos de ellos y alejémoslos del pueblo
hacia los caminos.''

\bibverse{33} El grueso del ejército israelita salió de donde estaba y
tomó posiciones en Baal-tamar, mientras que los que estaban en la
emboscada al oeste de Guibeá salieron a atacar desde donde se habían
escondido. \bibverse{34} Diez mil aguerridos guerreros israelitas
atacaron Guibeá, y la lucha fue tan intensa que los benjamitas no se
dieron cuenta de que estaban al borde del desastre. \bibverse{35} Así
que el Señor derrotó a Benjamín ante Israel. Ese día los israelitas
mataron a veinticinco mil cien benjamitas, todos armados con espadas.
\bibverse{36} Los benjamitas vieron que estaban derrotados.

Los israelitas habían retrocedido ante los benjamitas porque confiaban
en que la emboscada que habían preparado cerca de Guibeá tendría éxito.
\bibverse{37} Los hombres de la emboscada corrieron a atacar el pueblo,
y mataron a todos los que estaban en él. \bibverse{38} El acuerdo fue
que enviarían una gran nube de humo para mostrar que la ciudad había
caído.\footnote{\textbf{20:38} ``Mostrar que la ciudad había caído'':
  suministrado para mayor claridad.} \bibverse{39} El ejército israelita
se volvió para atacar a los benjamitas, que ya habían matado a unos
treinta israelitas. Los benjamitas decían: ``¡Los estamos derrotando
completamente, como en la primera batalla!''.

\bibverse{40} Sin embargo, cuando los israelitas vieron que las columnas
de humo se elevaban hacia el cielo hasta formar una gran nube sobre toda
la ciudad, \bibverse{41} se volvieron contra sus enemigos. Los
benjamitas se horrorizaron al verlo y se dieron cuenta de que estaban
condenados. \bibverse{42} Se volvieron y huyeron de los israelitas hacia
el desierto, pero la batalla los alcanzó, y los israelitas también
mataron a los que salieron de las ciudades en el camino. \bibverse{43}
Persiguiendo a los benjamitas, los israelitas los rodearon y los
alcanzaron fácilmente al este de Guibeá. \bibverse{44} Murieron
dieciocho mil benjamitas, todos ellos valientes guerreros. \bibverse{45}
Algunos de los benjamitas que quedaron corrieron hacia la Roca de la
Granada en el desierto, y los israelitas mataron a otros cinco mil
hombres en el camino. Persiguieron a otro grupo de benjamitas hasta
Guidón y mataron a otros mil.

\bibverse{46} Así que ese día mataron a veinticinco mil benjamitas,
todos armados con espadas y todos guerreros valientes. \bibverse{47}
Hubo seiscientos que huyeron a la Roca de la Granada, en el desierto, y
permanecieron allí cuatro meses. \bibverse{48} Los israelitas volvieron
a entrar en el territorio de los benjamitas, y yendo de pueblo en
pueblo, mataron todo: personas, animales, todo lo que encontraron. Luego
quemaron todos los pueblos en su camino.

\hypertarget{section-20}{%
\section{21}\label{section-20}}

\bibverse{1} Los hombres de Israel habían hecho un juramento en Mizpa:
``Ninguno de nosotros permitirá que nuestras hijas se casen con un
benjamita.'' \bibverse{2} Los israelitas fueron a Betel y se sentaron
allí ante Dios hasta el atardecer, llorando a gritos de angustia.
\bibverse{3} ``Señor, Dios de Israel, ¿por qué le ha sucedido esto a
Israel?'', preguntaron. ``Hoy falta una de nuestras tribus en Israel''.

\bibverse{4} Al día siguiente se levantaron temprano, construyeron un
altar y trajeron holocaustos y ofrendas de comunión. \bibverse{5}
``¿Cuál de todas las tribus de Israel no asistió a la asamblea que
celebramos ante el Señor?'', preguntaron. Porque habían hecho un
juramento sagrado de que cualquiera que no se presentara ante el Señor
en Mizpa sería ejecutado sin excepción.

\bibverse{6} Los israelitas se compadecieron de su hermano Benjamín,
diciendo: ``¡Hoy una tribu ha sido arrancada de Israel! \bibverse{7}
¿Qué haremos con las esposas para los que quedan, ya que hemos jurado
ante el Señor que no permitiremos que ninguna de nuestras hijas se case
con ellos?''

\bibverse{8} Entonces preguntaron: ``¿Quién de todas las tribus de
Israel no asistió a la asamblea que celebramos ante el Señor en Mizpa?''
Descubrieron que nadie de Jabés Galaad había acudido al campamento para
la asamblea, \bibverse{9} pues una vez hecho el recuento, no había nadie
de Jabés Galaad.

\bibverse{10} Así que la asamblea envió allí a doce mil de sus mejores
guerreros. Les dieron órdenes, diciendo: ``Vayan y maten a los
habitantes de Jabes de Galaad con sus espadas, incluso a las mujeres y a
los niños. \bibverse{11} Esto es lo que tienen que hacer:
Destruyan\footnote{\textbf{21:11} Literalmente, ``dediquen para su
  destrucción'': esto era lo que Dios les había ordenado a los
  israelitas que hicieran con las ciudades cananeas (por ejemplo,
  Jericó). Ahora se aplica ilegítimamente a otra ciudad israelita.} a
todo varón y a toda mujer que haya tenido relaciones sexuales con un
hombre.''

\bibverse{12} Lograron encontrar entre los habitantes de Jabés Galaad
cuatrocientas vírgenes que no habían tenido relaciones sexuales con un
hombre. Las llevaron al campamento de Silo, en la tierra de
Canaán.\footnote{\textbf{21:12} Parece que esta referencia a Israel como
  ``la tierra de Canaán'' es deliberada y pretende mostrar hasta qué
  punto Israel había caído en la idolatría.} \bibverse{13} Entonces toda
la asamblea envió un mensaje a los benjamitas en la Roca de Rimónpara
decirles: ``¡Paz!'' \bibverse{14} Así que los hombres de Benjamín
regresaron a sus casas y les dieron como esposas a las cuatrocientas
mujeres de Jabes Galaad que se habían salvado. Sin embargo, no había
suficiente para todos ellos.

\bibverse{15} El pueblo se compadeció de los benjamitas porque el Señor
había hecho este vacío entre las tribus israelitas. \bibverse{16} Los
ancianos de la asamblea preguntaron: ``¿Qué haremos para suplir a las
esposas restantes, porque todas las mujeres de Benjamín han sido
destruidas?'' \bibverse{17} Añadieron: ``Tiene que haber herederos para
los supervivientes benjaminitas; una tribu israelita de Israel no puede
ser aniquilada. \bibverse{18} Pero no podemos permitirles que tengan a
nuestras hijas como esposas, ya que nosotros, como pueblo de Israel,
hicimos un juramento sagrado, diciendo: `¡Maldito aquél que le dé una
esposa a un benjaminita!'\,''

\bibverse{19} Entonces dijeron: ``¡Mira! Todos los años se celebra la
fiesta del Señor en Silo. Se celebra al norte de Betel, y al este del
camino que va de Betel a Siquem, al sur de Lebona''.

\bibverse{20} Entonces ordenaron a los benjamitas: ``Vayan y escóndanse
en las viñas. \bibverse{21} Estén atentos, y cuando vean a las jóvenes
de Silo salir a hacer sus danzas, salgan corriendo de las viñas, y cada
uno de ustedes secuestre\footnote{\textbf{21:21} La palabra utilizada
  aquí es inusual y significa apoderarse de alguien por la fuerza.} una
esposa para sí mismo y regrese a su casa en la tierra de Benjamín.
\bibverse{22} Si sus padres o hermanos vienen a quejarse a nosotros, les
diremos: ``Por favor, hágannos un favor, porque no pudimos encontrar
suficientes esposas para ellos en la guerra.\footnote{\textbf{21:22}
  Refiriéndose al ataque a Jabes Galaad.} Y no es que ustedes sean
culpables de romper el juramento, ya que no las dieron en matrimonio''.

\bibverse{23} Losbenjaminitas hicieron lo que se les ordenó. Cada hombre
tomó a una de las mujeres bailarinas hasta donde fue necesario y se la
llevó para que fuera su esposa. Luego regresaron a su tierra, donde
reconstruyeron sus ciudades y habitaron en ellas.

\bibverse{24} Luego los israelitas partieron y se fueron a sus tribus y
familias, cada uno a la tierra que poseía. \bibverse{25} En aquel tiempo
Israel no tenía un rey; cada uno hacía lo que le parecía correcto.
