\hypertarget{section}{%
\section{1}\label{section}}

\bibverse{1} Dios, que en el pasado habló a nuestros padres por medio de
los profetas en distintas épocas y de muchas maneras, \bibverse{2} en
estos días nos ha hablado por medio de su Hijo. Dios designó al Hijo
como heredero de todo, e hizo el universo por medio de él. \bibverse{3}
El Hijo es la gloria radiante de Dios, y la expresión visible de su
verdadero carácter. Él sostiene todas las cosas con su poderoso mandato.
Cuando hizo provisión para limpiar el pecado, se sentó a la diestra de
la Majestad del cielo. \bibverse{4} Y fue puesto en un lugar más elevado
que los ángeles porque recibió un nombre más grande que ellos.
\bibverse{5} Dios nunca le dijo a ningún ángel: ``Tú eres mi hijo; hoy
me he convertido en tu Padre,'' o ``Seré su Padre, y él será mi
Hijo.''\footnote{\textbf{1:5} Hebreos contiene muchas citas y alusiones
  al Antiguo Testamento, algunas de las cuales no están citadas de
  manera exacta o son presentadas de manera resumida. Por eso, en
  ocasiones es difícil identificar la fuente exacta y con el fin de no
  sobrecargar el texto con tantos pie de página, las citas del Antiguo
  Testamento a menudo no aparecerán aquí.}

\bibverse{6} Además, cuando trajo a su Hijo primogénito\footnote{\textbf{1:6}
  ``Primogénito'': Este término no debe usarse como si hubiera algún
  tiempo en que Jesús no existió; más bien se usa para señalar un rango,
  mas no una cronología.} al mundo, dijo: ``Adórenlo todos los ángeles
de Dios.'' \bibverse{7} En cuanto a los ángeles, él dijo: ``Él
transforma a sus ángeles en vientos, y a sus siervos en llamas de
fuego,'' \bibverse{8} pero respecto al Hijo, dice: ``Tu trono, oh Dios,
perdura por siempre y para siempre, y la justicia es el cetro de tu
reino. \bibverse{9} Tú amas lo recto, y aborreces el desorden. Es por
eso que Dios, tu Dios, te ha puesto por encima de todos los demás,
ungiéndote\footnote{\textbf{1:9} La Antigua práctica de poner aceite
  sobre la cabeza de una persona tenía como fin indicar que la persona
  era escogida para una posición específica, un alto honor.} con el
aceite del gozo.''

\bibverse{10} ``Tú, Señor, pusiste los fundamentos de la tierra en el
principio. Los cielos son producto de tus manos. \bibverse{11} Un día se
acabarán, pero tú seguirás. Se desgastarán como se desgasta la ropa,
\bibverse{12} y los enrollarás como un manto. Los cambiarás como cambiar
la ropa, y tu vida no cesa jamás.''\footnote{\textbf{1:12} Literalmente,
  ``tus años nunca terminan.''} \bibverse{13} Pero nunca le dijo a
ningún ángel: ``Siéntate a mi diestra hasta que sujete a tus enemigos
debajo de tus pies.'' \bibverse{14} ¿Qué son los ángeles? Son seres que
sirven, que han sido enviados para ayudar a los que recibirán la
salvación.

\hypertarget{section-1}{%
\section{2}\label{section-1}}

\bibverse{1} Por lo tanto deberíamos estar aún más atentos a lo que
hemos aprendido para no descarriarnos. \bibverse{2} Si el mensaje que
los ángeles trajeron es fiel, y si cada pecado y acto de desobediencia
trae su propia consecuencia,\footnote{\textbf{2:2} Literalmente,
  ``recibe su recompensa.''} \bibverse{3} ¿cómo escaparemos si no
atendemos esta gran salvación que el Señor anunció desde el principio, y
que después nos confirmó por medio de quienes lo oyeron? \bibverse{4}
Dios también dio testimonio por medio de señales y milagros, por actos
que demuestran su poder, y por medio de los dones del Espíritu Santo,
que repartió como quiso.

\bibverse{5} No serán los ángeles los encargados del mundo venidero del
cual hablamos. \bibverse{6} Sino que, como se ha dicho: ``¿Qué son los
seres humanos para que te preocupes por ellos? ¿Quién es el hijo de
hombre\footnote{\textbf{2:6} ``Hijo de hombre'': En su uso normal se
  refiere solo a un ser humano; sin embargo, Jesús aplicó este término
  genérico a sí mismo.} para que cuides de él? \bibverse{7} Lo hiciste
un poco inferior a los ángeles; lo coronaste con gloria y honra, y lo
pusiste por encima de toda tu creación.\footnote{\textbf{2:7} En lugar
  de referirse solo a la humanidad, también puede referirse a Jesús:
  ``Lo hiciste un poco menor que los ángeles, y luego lo coronaste de
  gloria y honra.'' Todo el texto puede verse de manera dual,
  refiriéndose a Jesús como el hijo de hombre, siendo tanto
  representante como Salvador de la humanidad.} \bibverse{8} Le diste
autoridad sobre todas las cosas.''\footnote{\textbf{2:8} Una vez más,
  esto puede aplicarse a la humanidad, a Dios dando autoridad sobre las
  criaturas como se menciona en Génesis 1, o puede aplicarse a la
  autoridad de Jesús como Señor.} No quedó nada por fuera cuando Dios le
dio autoridad sobre todas las cosas. Sin embargo, vemos que no todo está
sujeto a su autoridad todavía.

\bibverse{9} Pero vemos a Jesús, puesto en un lugar un poco inferior al
de los ángeles, coronado de gloria y honra por el sufrimiento de la
muerte. Por medio de la gracia de Dios, Jesús experimentó la muerte por
todos.

\bibverse{10} Era conveniente que Dios, quien crea y sostiene todas las
cosas, preparara por medio del sufrimiento a Aquél que los lleva a la
salvación, para llevar a muchos de sus hijos a la gloria. \bibverse{11}
Pues tanto el que santifica como los que son santificados pertenecen a
la misma familia.\footnote{\textbf{2:11} Literalmente, ``todos de una.''}
Por eso no vacila en llamarlos ``hermanos'' \bibverse{12} al decir:
``Anunciaré tu nombre a mis hermanos; te alabaré entre tu pueblo cuando
se reúna.''\footnote{\textbf{2:12} ``se reúna'': la palabra griega es
  ``ecclesia'' que más adelante llegó a significar ``iglesia.''}
\bibverse{13} Y también dice: ``Pondré mi confianza en él,'' y ``Aquí
estoy, junto a los hijos que Dios me ha dado.''

\bibverse{14} Y como los hijos tienen en común carne y sangre, él
participó de su carne y sangre del mismo modo, para así destruir por
medio de la muerte a aquél que tiene el poder de la muerte---el
diablo--- \bibverse{15} y liberar a todos los que habían estado
esclavizados toda la vida por miedo a la muerte.

\bibverse{16} Sin duda alguna, los ángeles no son su preocupación; él se
preocupa por ayudar a los hijos de Abrahán. \bibverse{17} Por ello le
fue necesario volverse como sus hermanos en todo, para poder llegar a
ser un sumo sacerdote, misericordioso y fiel, en las cosas de Dios, para
perdonar los pecados de su pueblo. \bibverse{18} Y como él mismo sufrió
la tentación, puede ayudar a los que son tentados.

\hypertarget{section-2}{%
\section{3}\label{section-2}}

\bibverse{1} Así que, mis hermanos y hermanas que viven para Dios y
participan de este celestial llamado: necesitamos pensar con cuidado
acerca de Jesús, el que decimos que fue enviado por Dios,\footnote{\textbf{3:1}
  Literalmente, ``apóstol.''} y quien es el Sumo Sacerdote. \bibverse{2}
Él fue fiel a Dios en la obra para la cual fue elegido, así como Moisés
fue fiel a Dios en la casa de Dios.\footnote{\textbf{3:2} La palabra
  ``casa'' aquí significa más que el edificio: se refiere a los miembros
  de una casa, la familia, el hogar.} \bibverse{3} Pero Jesús es
merecedor de mayor gloria que Moisés, del mismo modo que el constructor
de una casa merece más crédito que la misma casa. \bibverse{4} Cada casa
tiene su constructor; Dios es el constructor de todo. \bibverse{5} Y
como siervo, Moisés fue fiel en la casa de Dios. Él nos dio evidencia de
lo que sería anunciado después. \bibverse{6} Pero Cristo es un hijo, a
cargo de la casa de Dios. Y nosotros somos la casa de Dios siempre y
cuando nos aferremos con confianza a la esperanza en la cual decimos que
creemos con orgullo.

\bibverse{7} Por eso el Espíritu Santo dice: ``Si oyen lo que Dios les
está diciendo hoy, \bibverse{8} no endurezcan sus corazones\footnote{\textbf{3:8}
  ``Endurezcan sus corazones,'' queriendo decir, volverse tercos u
  obstinados.} como en aquél tiempo en que se rebelaron contra él,
cuando lo pusieron a prueba en el desierto. \bibverse{9} Los padres de
ustedes me pusieron a prueba, y probaron mi paciencia, y vieron la
evidencia que les mostré durante cuarenta años.

\bibverse{10} ``Tal generación despertó mi enojo\footnote{\textbf{3:10}
  Como siempre, Dios aquí usa términos humanos. No debemos entender que
  Dios se enoja de la manera que nosotros lo hacemos, especialmente
  cuando se trata de ``perder la paciencia'' y actuar sin amor o
  irracionalmente. Lo mismo aplica para el versículo 3:11.} y por ello
dije: `Siempre se equivocan en su manera de pensar. No me conocen ni
saben lo que estoy haciendo.' \bibverse{11} Por ello, en mi frustración
hice un juramento: `No entrarán a mi reposo.'\,''\footnote{\textbf{3:11}
  ``Reposo.'' Este concepto se desarrolla más en el capítulo 4 y se
  relaciona con el Sábado, la Tierra Prometida, y la invitación de Dios
  de venir a él. Aunque no es la más fácil de las frases, ``entrar en su
  reposo'' quizás es la mejor traducción pues mantiene la base de lo que
  más adelante se desarrolla en el texto, e incluye todas las alusiones.}

\bibverse{12} Hermanos y hermanas, asegúrense de que ninguno de ustedes
tenga un pensamiento malvado y alejado de la fe en el Dios de la vida.
\bibverse{13} Anímense unos a otros cada día mientras dure el ``hoy,''
para que ninguno de ustedes pueda ser engañado por el pecado ni se
endurezcan sus corazones. \bibverse{14} Porque somos socios con Cristo
siempre y cuando mantengamos nuestra confianza en Dios de principio a
fin.

\bibverse{15} Como dice la Escritura: ``Si oyen lo que Dios les dice
hoy, no endurezcan sus corazones como aquél tiempo en que se rebelaron
contra él.'' \bibverse{16} ¿Quién se rebeló contra Dios aun habiendo
oído lo que él dijo? ¿No fueron acaso los que fueron sacados de Egipto
por Moisés? \bibverse{17} ¿Contra quienes estuvo enojado Dios durante
cuarenta años? ¿No fue contra aquellos que fueron sepultados en el
desierto? \bibverse{18} ¿De quién hablaba Dios cuando hizo juramento de
que no entrarían en su reposo? ¿No fue de los que lo desobedecieron?
\bibverse{19} Así vemos que ellos no pudieron entrar, porque no
confiaron en él.

\hypertarget{section-3}{%
\section{4}\label{section-3}}

\bibverse{1} Por lo tanto seamos cuidadosos y asegurémonos de no
perdernos la oportunidad de entrar a su reposo, aunque Dios ya nos dio
la promesa. \bibverse{2} Porque hemos oído buenas noticias tal como
ellos lo hicieron, pero eso no fue suficiente porque ellos no aceptaron
ni creyeron lo que oyeron. \bibverse{3} Sin embargo, los que creen en
Dios ya han entrado al reposo mencionado por Dios cuando dijo: ``En mi
frustración hice un juramento: `No entrarán a mi reposo.'\,'' (Esto es
así aunque los planes de Dios ya estaban completos cuando creó el
mundo). \bibverse{4} En cuanto al séptimo día, hay un lugar en la
Escritura que dice: ``Dios reposó el séptimo día de toda su obra.''
\bibverse{5} Y como lo afirmaba el pasaje anterior: ``Ellos no entrarán
a mi reposo.''

\bibverse{6} El reposo de Dios aún está disponible para que entremos en
él, aunque aquellos que habían oído antes la buena noticia no lograron
entrar por su desobediencia. \bibverse{7} Así que Dios una vez más
coloca un día---hoy---diciéndonos mucho tiempo después por medio de
David,\footnote{\textbf{4:7} Refiriéndose a Salmos 95:7.} como lo hizo
antes: ``Si oyen lo que Dios les dice hoy, no endurezcan sus
corazones.'' \bibverse{8} Porque si Josué hubiera podido darles reposo,
Dios no habría dicho nada después sobre otro día. \bibverse{9} De modo
que el reposo del Sábado todavía permanece para el pueblo de Dios.
\bibverse{10} Porque todo el que entra al reposo de Dios también
descansa de su labor, así como Dios lo hizo.

\bibverse{11} En consecuencia, debemos esforzarnos por entrar al reposo
de Dios para que nadie caiga al seguir el mismo ejemplo de
desobediencia. \bibverse{12} Pues la palabra de Dios es viva y eficaz, y
más afilada que espada de dos filos, que penetra hasta separar la vida y
el aliento,\footnote{\textbf{4:12} Las palabras griegas ``psuche'' y
  ``pneuma,'' en ocasiones traducidas como ``alma'' y ``espíritu,''
  aunque es difícil entender el significado ya que no hay diferencia
  entre ``alma'' y espíritu''.'' Se emplea la traducción de ``vida'' y
  ``aliento'' porque se considera que expresa mejor el pensamiento
  original.} así como los tendones y los tuétanos, juzgando los
pensamientos y las intenciones de la mente. \bibverse{13} No hay ser
vivo que esté oculto de su vista; todo está expuesto y es visible ante
aquél a quien hemos de rendirle cuentas.

\bibverse{14} Y como tenemos tal sumo sacerdote que ha ascendido al
cielo, Jesús, el Hijo de Dios, asegurémonos de mantenernos en lo que
decimos creer. \bibverse{15} Pues el sumo sacerdote que tenemos no es
uno que no pueda entender nuestras debilidades, sino uno que fue tentado
de la misma forma que nosotros, pero no pecó. \bibverse{16} Así que
deberíamos acercarnos confiados a Dios, en su trono de gracia, para
recibir misericordia, y descubrir la gracia que nos ayuda cuando
realmente la necesitamos.

\hypertarget{section-4}{%
\section{5}\label{section-4}}

\bibverse{1} Todo sumo sacerdote es elegido dentro del mismo pueblo y
está designado para trabajar por el pueblo en cuanto a su relación con
Dios. Él presenta a Dios tanto sus dones como sus sacrificios por sus
pecados. \bibverse{2} El sumo sacerdote comprende cuán ignorantes y
engañadas se sienten las personas porque él también experimenta las
mismas debilidades humanas que ellos. \bibverse{3} En consecuencia, él
tiene que ofrecer sacrificios por sus pecados así como por los del
pueblo. \bibverse{4} Nadie puede tomar la posición de sumo sacerdote por
sí mismo, sino que debe ser elegido por Dios, como lo fue Aarón.
\bibverse{5} Del mismo modo en que Cristo no se atribuyó honra a sí
mismo convirtiéndose en sumo sacerdote. Sino que fue Dios quien le dijo:
``Tú eres mi hijo. Hoy yo me convierto en tu Padre.'' \bibverse{6} Y en
otro versículo, Dios dice: ``Eres un sacerdote por siempre, siguiendo el
orden de Melquisedec.'' \bibverse{7} Jesús, mientras estuvo aquí, en
forma humana, oró y clamó a Dios con grandes gemidos y lágrimas, al
único que tenía el poder de salvarlo de la muerte. Y Jesús fue escuchado
por su respeto hacia Dios. \bibverse{8} Aunque era el Hijo de Dios,
Jesús aprendió de manera práctica el significado de la obediencia a
través del sufrimiento.\footnote{\textbf{5:8} La traducción común de que
  Jesús ``aprendió obediencia por medio del sufrimiento'' podría sugerir
  que originalmente Jesús no era obediente, o que le era necesario
  sufrir para aprender, las cuales son ideas extrañas en lo que se
  refiere a Jesús, el hijo pre-existente de Dios.} \bibverse{9} Y cuando
su experiencia culminó,\footnote{\textbf{5:9} Evitar el término
  ``habiendo sido perfeccionado,'' que en la mente podría sugerir que no
  era perfecto desde el principio.} se convirtió en la fuente de
salvación eterna para todos los que hacen su voluntad, \bibverse{10}
habiendo sido designado por Dios como sumo sacerdote, conforme al orden
de Melquisedec.

\bibverse{11} Hay mucho que decir acerca de Jesús, y no es fácil
explicarlo porque ustedes parecen no entender. \bibverse{12} Para esta
hora, ustedes ya han tenido suficiente tiempo para ser maestros, pero
todavía necesitan de alguien que les enseñe los fundamentos, los
principios de la palabra de Dios. ¡Es como si necesitaran volver a beber
leche en lugar de comida sólida! \bibverse{13} Los que beben leche no
tienen la experiencia para vivir de manera correcta, pues apenas son
bebés. \bibverse{14} La comida sólida es para los adultos, para los que
han aprendido siempre a usar su cerebro para poder decir la diferencia
entre el bien y el mal.

\hypertarget{section-5}{%
\section{6}\label{section-5}}

\bibverse{1} Así que no nos estanquemos en las enseñanzas básicas acerca
de Cristo, sino progresemos a un entendimiento más maduro. No
necesitamos volver una y otra vez a los conceptos sobre el
arrepentimiento de lo que solíamos hacer, o sobre la fe en Dios,
\bibverse{2} o enseñanzas acerca del bautismo, la imposición de manos,
la resurrección de los muertos, y el juicio eterno. \bibverse{3}
Avancemos en la medida que Dios nos lo permite.

\bibverse{4} Es imposible que los que una vez comprendieron y
experimentaron el don celestial de Dios---que participaron del
recibimiento del Espíritu Santo, \bibverse{5} que habían conocido la
palabra de Dios y el poder de la era que está por venir--- \bibverse{6}
y luego abandonaron por completo a Dios, vuelvan al arrepentimiento una
vez más. Ellos mismos han crucificado al Hijo de Dios una y otra vez, y
lo han humillado públicamente. \bibverse{7} La tierra que ha sido regada
por la lluvia, y produce cosecha para quienes la trabajan, tiene la
bendición de Dios. \bibverse{8} Pero la tierra que solo produce monte y
espinas no sirve para nada, y está condenada. Y al final lo único que
puede hacerse es quemarla.

\bibverse{9} Pero queridos amigos, nosotros deseamos cosas mejores para
ustedes, y también su salvación, aunque les hablemos así. \bibverse{10}
Dios no hubiera sido injusto como para olvidarse de lo que ustedes han
hecho y del amor que le han demostrado mediante el cuidado que han
brindado a los hermanos creyentes, lo cual es algo que todavía siguen
haciendo. \bibverse{11} Queremos que cada uno de ustedes demuestre el
mismo compromiso y confianza en la esperanza de Dios, hasta que sea
cumplida. \bibverse{12} No sean espiritualmente perezosos, sino sigan el
ejemplo de los que por medio de su fe en Dios y paciencia son herederos
de lo que Dios ha prometido. \bibverse{13} Cuando Dios le dio su promesa
a Abrahán, no pudo jurar por alguien superior, así que hizo un juramento
consigo mismo, \bibverse{14} diciendo: ``Sin duda alguna te bendeciré, y
multiplicaré tus descendientes.'' \bibverse{15} Y así, después de
esperar pacientemente, Abrahán recibió la promesa.

\bibverse{16} Las personas juran por cosas que son superiores a ellas, y
cuando tienen alguna discusión, hacen un juramento como la última
palabra sobre tal asunto. \bibverse{17} Es por ello que Dios quería
demostrar más claramente a los que heredarían la promesa, que él nunca
cambiaría su decisión. \bibverse{18} De modo que por estas dos
acciones\footnote{\textbf{6:18} Es decir, la promesa y el juramento.}
que no pueden cambiarse, y, como Dios no puede mentir, podemos tener
plena confianza en que al huir buscando seguridad, podemos aferrarnos de
la esperanza que Dios nos presentó. \bibverse{19} Esta esperanza es
nuestra ancla espiritual, es segura y confiable, y nos lleva más allá de
la cortina, a la presencia de Dios. \bibverse{20} Allí entró Jesús en
nuestro favor, porque tenía que convertirse en un sumo sacerdote
conforme al orden de Melquisedec.

\hypertarget{section-6}{%
\section{7}\label{section-6}}

\bibverse{1} Melquisedec fue rey de Salem y sacerdote del Dios Supremo.
Conoció a Abrahán, quien venía de regreso después de haber derrotado a
los reyes, y lo bendijo. \bibverse{2} Y Abrahán le dio diezmo de todo lo
que había ganado. El nombre Melquisedec significa ``rey de justicia''
mientras que el rey de Salem significa ``rey de paz.'' \bibverse{3} No
tenemos información sobre su padre o su madre, o sobre su genealogía. No
sabemos cuándo nació ni cuándo murió. Así como el Hijo de Dios, sigue
siendo sacerdote para siempre.

\bibverse{4} Consideremos la grandeza de este hombre ante los ojos de
Abrahán, el patriarca, que incluso le entregó diezmo de lo que había
ganado en la batalla. \bibverse{5} Sí, pues los hijos de Leví, que son
sacerdotes, tienen mandato por la ley de recibir diezmo del pueblo, que
son sus hermanos y hermanas, y que son descendientes de Abrahán.
\bibverse{6} Pero Melquisedec, sin pertenecer a esta descendencia,
recibió diezmos de Abrahán, y bendijo al que tenía las promesas de Dios.
\bibverse{7} No existe duda de que quien recibe bendición es inferior a
quien bendice. \bibverse{8} En el primer caso, los que reciben el diezmo
son hombres mortales, pero en el otro caso, se dice que los recibió uno
que sigue viviendo. \bibverse{9} Entonces podríamos decir que Leví, el
que recibe los diezmos, ha pagado diezmos por ser descendiente de
Abrahán, \bibverse{10} pues aún no había nacido de su padre\footnote{\textbf{7:10}
  Literalmente ``en hombros de su padre.''} cuando Melquisedec conoció a
Abrahán.

\bibverse{11} Ahora, si hubiera sido posible lograr la perfección por el
sacerdocio de Leví (pues así fue como se recibió la ley), ¿Por qué había
necesidad de otro sacerdote que siguiera el orden de Melquisedec, y no
del orden de Aarón? \bibverse{12} Si se cambia el sacerdocio, la ley
necesitaría cambiarse también. \bibverse{13} Pero este de quien hablamos
viene de otra tribu, una tribu que nunca ha provisto sacerdotes que
sirvan en el altar. \bibverse{14} Está claro que nuestro Señor es
descendiente de Judá, y Moisés nunca hizo mención sobre sacerdotes que
provinieran de esta tribu. \bibverse{15} Y esto queda aún más claro
cuando vemos que aparece otro sacerdote similar a Melquisedec,
\bibverse{16} que no llegó al sacerdocio por virtud de su ascendencia,
sino por el poder de una vida que no puede ser destruida. \bibverse{17}
Por eso dice: ``Tú eres sacerdote para siempre, conforme al orden de
Melquisedec.''

\bibverse{18} De modo que la norma anterior ha sido anulada porque era
débil e inútil, \bibverse{19} (porque la ley nunca perfeccionó nada).
Pero ahora ha sido reemplazada por una esperanza mejor, por la cual
podemos acercarnos a Dios. \bibverse{20} Esto\footnote{\textbf{7:20}
  Refiriéndose a una nueva forma de acercarse a Dios.} no se hizo sin un
juramento, aunque los que se convierten en sacerdotes lo hacen con un
juramento. \bibverse{21} Pero él se convirtió en sacerdote con un
juramento porque Dios le dijo: ``El Señor ha hecho un juramento solemne
y no cambiará de opinión: Tú eres sacerdote para siempre.''
\bibverse{22} Es así como Jesús se convirtió en la garantía de una
relación con Dios\footnote{\textbf{7:22} ``Un acuerdo de relación con
  Dios.'' Esto traduce una sola palabra que en griego se traduce
  tradicionalmente como ``pacto.'' Sin embargo, la palabra ``pacto''
  normalmente no se usa en nuestro lenguaje coloquial y por ello se ha
  convertido en una palabra ``teológica.'' Se ha escrito mucho sobre
  este concepto y los términos usaos, y ``pacto'' a menudo se ha
  preservado porque parece no haber una manera eficaz de explicar lo que
  se quiere decir aquí. El concepto de pacto se desarrolla más
  ampliamente en los capítulos 8 y 9. Y existen problemas con palabras
  alternativas. La palabra ``contrato'' puede significar el resultado de
  una negociación, que no es el caso. Del mismo modo, ``tratado'' o
  ``acuerdo,'' desde el punto de vista humano, puede referirse a
  negociaciones mutuas. Pero aquí la palabra hace referencia a la
  iniciativa de Dios, y sin duda no se lleva a cabo entre dos
  semejantes. Quizás un mejor concepto sería ``una promesa que se pacta
  con obligaciones correspondientes,'' pero tal palabrería sería más
  engorrosa.} que es mucho mejor.

\bibverse{23} Ha habido muchos sacerdotes porque la muerte les impidió
continuar su sacerdocio; \bibverse{24} pero como Jesús vive para
siempre, su sacerdocio es permanente. \bibverse{25} En consecuencia,
tiene el poder para salvar por completo a los que se acercan a Dios por
medio de él, viviendo siempre para rogar su caso a favor de ellos.

\bibverse{26} Él es justamente el sumo sacerdote que necesitamos: santo
y sin falta, puro y apartado de los pecadores, y con un lugar en lo más
alto de los cielos. \bibverse{27} A diferencia de los sumos sacerdotes
humanos, él no necesita ofrecer sacrificios diarios por sus pecados y
los de las personas. Él lo hizo una vez, y por todos, cuando se dio a sí
mismo como ofrenda. \bibverse{28} La ley designa hombres imperfectos
como sumos sacerdotes, pero después de la ley, Dios hizo un juramento
solemne, y designó a su hijo, que es perfecto para siempre.

\hypertarget{section-7}{%
\section{8}\label{section-7}}

\bibverse{1} El punto principal de lo que estamos diciendo es este:
tenemos tal sumo sacerdote que está sentado a la diestra de Dios, que
está sentado en majestad sobre su trono en el cielo. \bibverse{2} Él
sirve en el santuario, el verdadero tabernáculo que fue establecido por
el Señor y no por seres humanos. \bibverse{3} Como es responsabilidad de
todo sumo sacerdote ofrecer dones y sacrificios, este sumo sacerdote
también tiene algo que ofrecer.

\bibverse{4} Ahora bien, si él estuviera aquí en la tierra, no sería un
sacerdote en absoluto, porque ya hay sacerdotes para presentar las
ofrendas que exige la ley. \bibverse{5} Pero el lugar donde ellos sirven
es una copia, una mera sombra de lo que hay en el cielo. Y eso fue lo
que Dios le dijo a Moisés cuando iba a construir el tabernáculo: ``Ten
cuidado de hacer todo conforme al modelo que se te mostró en la
montaña.''

\bibverse{6} Pero a Jesús se le ha dado un ministerio mucho mejor, pues
él es el único mediador de una relación mejor entre nosotros y Dios. Una
relación basada en mejores promesas. \bibverse{7} Si el primer pacto
hubiera sido perfecto, no se habría necesitado un segundo pacto.
\bibverse{8} Y hablando sus fallos,\footnote{\textbf{8:8} Aclarando que
  el problema con el ``primer pacto'' no se debió a un acuerdo
  defectuoso sino a que el pueblo de Dios no cumplió con las
  responsabilidades del acuerdo.} Dios le dijo a su pueblo: ``Estén
atentos, dice el Señor, porque vienen días en que haré un nuevo pacto en
relación a la casa de Israel y la casa de Judá. \bibverse{9} No será
como el pacto prometido que hice con los ancestros cuando los llevé de
la mano fuera de la tierra de Egipto. Porque ellos no cumplieron con su
parte en la relación que habíamos acordado, y por eso los abandoné, dice
el Señor.''

\bibverse{10} ``Esta es la relación que le prometo a la casa de Israel:
Después de ese tiempo, dice el Señor, yo pondré mis leyes en sus mentes,
y las escribiré en sus corazones. Yo seré su Dios, y ellos serán mi
pueblo. \bibverse{11} Nadie tendrá que enseñarle a su prójimo, y nadie
necesitará enseñar en su familia, diciendo: `Debes conocer al Señor.'
Porque todos me conocerán, desde el más pequeño hasta el más grande.
\bibverse{12} Y yo seré misericordioso cuando se equivoquen, y me
olvidaré de sus pecados.''

\bibverse{13} Al decir ``pacto de una nueva relación,'' Dios abandona el
primer pacto. Ese pacto que ya está obsoleto y desgastado, y que casi ha
desaparecido.

\hypertarget{section-8}{%
\section{9}\label{section-8}}

\bibverse{1} El antiguo sistema tenía instrucciones sobre cómo adorar, y
un santuario terrenal. \bibverse{2} En la primera sala del tabernáculo
estaba el candelabro, la mesa, y el pan sagrado. A este lugar se le
llamaba el Lugar Santo. \bibverse{3} Al pasar el segundo velo, se
encontraba la sala que se llamaba el Lugar Santísimo. \bibverse{4}
Dentro de este lugar estaba el altar de oro del incienso, y el ``arca
del pacto,'' cubierta de oro. \footnote{\textbf{9:4} 9:4a. Traducida
  comúnmente como ``arca del pacto,'' era una caja de madera que
  simbolizaba un sitio de reunión, de reconciliación, y acuerdo entre
  Dios y su pueblo.} Dentro del arca se encontraba una taza de oro que
contenía maná, la vara de Aarón que reverdeció, y las inscripciones del
pacto sobre piedras.\footnote{\textbf{9:4} 9:4b. Se creía que era la
  piedra con las inscripciones de los 10 mandamientos.} \bibverse{5} Y
encima del arca estaba el ángel querubín protegiendo el lugar de la
reconciliación. Pero ahora no podemos hablar de esto en detalle.

\bibverse{6} Cuando todo esto estuvo establecido, los sacerdotes ya
podrían entrar con regularidad a la primera sala para llevar a cabo sus
labores. \bibverse{7} Pero solo el sumo sacerdote entraba a la segunda
sala, y solo una vez al año. Incluso en ese momento tenía que hacer un
sacrificio que incluyera sangre,\footnote{\textbf{9:7} La sangre es un
  tema muy frecuente en la última parte del libro de Hebreos. Es un
  símbolo abreviado de la vida, y la sangre derramada significa muerte,
  y aunque el contexto original del Sistema de sacrificios es literal,
  sin duda alguna, su uso en el libro de Hebreos, al aplicarlo a Cristo,
  es principalmente como símbolo de lo que él logró con su vida, muerte
  y resurrección.} el cual era ofrecido por sí mismo y por los pecados
que el pueblo hubiera cometido por ignorancia.

\bibverse{8} Con esto, el Espíritu Santo indicaba que el camino al
verdadero Lugar Santísimo no se había revelado mientras aún existía el
primer tabernáculo.\footnote{\textbf{9:8} El significado de esta
  afirmación es tema de debate. En general, podríamos concluir que a la
  luz de la nueva revelación de Dios por medio de Jesús, que es el
  centro del nuevo testamento, y particularmente del libro de Hebreos,
  este pasaje se refiere a Jesús como la plena revelación de Dios,
  proporcionando un ``acceso'' hacia él, lo cual no había sucedido bajo
  el antiguo sistema (ver como referencia la afirmación de Jesús en Juan
  14:6).} \bibverse{9} Esta es una ilustración para nosotros en el
presente, demostrándonos que los dones y sacrificios que se ofrecen no
pueden limpiar la conciencia del adorador. \bibverse{10} Pues esos son
solamente requisitos religiosos, que tienen que ver con la comida y la
bebida, y diversas ceremonias que implican el lavamiento, las cuales
fueron impuestas hasta que llegó el tiempo en que Dios estableció una
nueva forma de relacionarnos con él.

\bibverse{11} Cristo ha venido como sumo sacerdote de todas las buenas
experiencias que ahora tenemos. Entró a un tabernáculo más grande y
completo que no fue hecho por manos humanas, ni es parte de este mundo
creado. \bibverse{12} Él no entró por medio de la sangre de cabras y
becerros, sino por medio de su propia sangre. Entró una sola vez y por
todas, en el Lugar Santísimo, liberándonos para siempre.

\bibverse{13} Pues si la sangre de cabras y toros, y las cenizas de vaca
rociadas sobre lo que está ritualmente impuro pueden hacer que el cuerpo
esté ceremonialmente puro, \bibverse{14} ¿cuánto más la sangre de
Cristo, quien se ofreció a Dios teniendo una vida sin pecado por medio
del Espíritu eterno, puede limpiar sus conciencias de sus antiguas vidas
de pecado, para que puedan servir al Dios vivo?

\bibverse{15} Por eso él es el mediador de una nueva relación de pacto.
Puesto que la muerte ha ocurrido para liberarlos de los pecados
cometidos bajo la relación del primer pacto, ahora los que son llamados
pueden recibir la promesa de una herencia eterna. \bibverse{16} Pues
para que se cumpla un testamento, quien lo hace debe morir primero.
\bibverse{17} El testamento solo es válido cuando hay muerte, y nunca se
cumple mientras la persona aun esté viva. \bibverse{18} Por eso el
primer pacto fue establecido con sangre.

\bibverse{19} Después que Moisés presentó todos los mandamientos de la
ley a todo el pueblo, tomó la sangre de cabras y becerros junto con agua
y roció el libro\footnote{\textbf{9:19} El libro de la ley.} y también a
todo el pueblo, usando lana escarlata e hisopo. \bibverse{20} Y les
dijo: ``Esta es la sangre de la relación de pacto que Dios les ha dicho
que quiere tener con ustedes.'' \bibverse{21} Del mismo modo, Moisés
roció la sangre en el tabernáculo y en todo lo que se usaba para el
culto. \bibverse{22} Conforme a la ley ceremonial, casi todo se
purificaba con sangre, y sin derramamiento de sangre, nada quedaría
ritualmente limpio de la mancha del pecado. \bibverse{23} De modo que si
las copias de lo que hay en el cielo necesitaban limpiarse de esta
manera, las cosas que están en el cielo necesitaban limpiarse con
mejores sacrificios.

\bibverse{24} Porque Cristo no ha entrado al Lugar Santísimo construido
por seres humanos y que es apenas un modelo del original. Él entró al
cielo mismo, y ahora aparece en representación de nosotros, hablando a
nuestro favor en presencia de Dios. \bibverse{25} Esto no tiene como fin
ofrecerse repetidas veces, como un sumo sacerdote que tiene que entrar
al Lugar Santísimo después de un año, ofreciendo sangre que no es suya.
\bibverse{26} De otro modo, Cristo habría tenido que sufrir muchas veces
desde la creación del mundo. Pero no fue así: fue solo una vez al final
de la era presente que él vino a eliminar el pecado al sacrificarse a sí
mismo. \bibverse{27} Y así como los seres humanos mueren una sola vez, y
luego son juzgados, \bibverse{28} del mismo modo ocurre con Cristo. Pues
al haber sido sacrificado una sola vez para quitar los pecados de
muchos, vendrá otra vez, no para hacerse cargo del pecado, sino para
salvar a quienes lo esperan.

\hypertarget{section-9}{%
\section{10}\label{section-9}}

\bibverse{1} La ley es apenas una sombra de las cosas buenas que
vendrían, y no de la realidad como tal. De modo que no podía justificar
a los que venían a adorar a Dios por medio de sacrificios repetitivos
que se ofrecían cada año. \bibverse{2} De otro modo ¿no se habrían
detenido los sacrificios? Si los adoradores hubieran sido limpiados una
vez y para siempre, nunca más habrían tenido conciencias culpables.
\bibverse{3} Pero tales sacrificios, en efecto, le recuerdan a la gente
los pecados año tras año, \bibverse{4} porque es imposible que la sangre
de toros y cabras quite los pecados.

\bibverse{5} Por eso, cuando Cristo\footnote{\textbf{10:5} El original
  dice simplemente ``él;'' Se infiere que es Cristo por los versículos
  9:24, 28.} vino al mundo dijo: ``Tú no querías sacrificios ni
ofrendas, sino que preparaste un cuerpo para mí. \bibverse{6} Las
ofrendas quemadas y los sacrificios por el pecado no te agradaron.'
\bibverse{7} Entonces dije: `Dios, considera que he venido a hacer tu
voluntad, tal como se dice de mí en el libro.'\,''\footnote{\textbf{10:7}
  En realidad dice ``el encabezamiento de un rollo,'' queriendo decir,
  las Escrituras.} \bibverse{8} Como se menciona arriba: ``No quisiste
sacrificios ni ofrendas. Las ofrendas quemadas y los sacrificios por el
pecado no te agradaron,'' (aunque eran ofrecidos conforme a los
requisitos de la ley). \bibverse{9} Entonces él dijo: ``Mira, he venido
a hacer tu voluntad.'' Entonces él abandona el primer pacto para
establecer el segundo, \bibverse{10} por medio del cual todos somos
santificados a través de Jesucristo, quien ofrece su cuerpo una vez y
para siempre.

\bibverse{11} Todos los sumos sacerdotes ofician en los servicios cada
día, una y otra vez, ofreciendo los mismos sacrificios que no pueden
quitar los pecados. \bibverse{12} Pero este Sacerdote, después de
ofrecer un solo sacrificio por los pecados, que dura para siempre, se
sentó a la diestra de Dios. \bibverse{13} Y ahora espera hasta que todos
sus enemigos sean vencidos, y vengan a ser como banquillo para sus pies.
\bibverse{14} Porque con un solo sacrificio él justificó para siempre a
los que están siendo santificados. \bibverse{15} Tal como nos dice el
Espíritu Santo, por haber dicho: \bibverse{16} ``Este es el pacto que
haré con ellos más adelante, dice el Señor. Pondré mis leyes en sus
corazones, y las escribiré en sus mentes.'' Entonces añade:
\bibverse{17} ``Nunca más me acordaré de sus pecados e iniquidades.''
\bibverse{18} Después de estar libres de tales cosas, las ofrendas por
el pecado ya no son necesarias.

\bibverse{19} Ahora tenemos esta seguridad, hermanos y hermanas, de
poder entrar al Lugar Santísimo por la sangre de Jesús. \bibverse{20}
Por medio de su vida y muerte,\footnote{\textbf{10:20} ``Su vida y
  muerte'': literalmente ``su cuerpo.''} él abrió a través del velo que
nos lleva hacia Dios, una nueva forma de vivir. \bibverse{21} Siendo que
tenemos este gran sacerdote que está a cargo de la casa de Dios,
\bibverse{22} acerquémonos a Dios, con mentes sinceras y plena
confianza. Nuestras mentes han sido rociadas para purificarlas de
nuestros malos pensamientos, y nuestros cuerpos han sido lavados y
limpiados con agua pura. \bibverse{23} Así que aferrémonos a la
esperanza de la cual les hablamos a otros, y sin dudar, porque el Dios
que prometió es fiel. \bibverse{24} Pensemos en cómo podemos animarnos
unos a otros a amar y hacer el bien. \bibverse{25} No deberíamos
desistir en cuanto a reunirnos, como algunos lo han hecho. De hecho,
deberíamos animarnos unos a otros, especialmente cuando vemos que el
Fin\footnote{\textbf{10:25} Literalmente ``el Día.''} se acerca.

\bibverse{26} Porque si seguimos pecando deliberadamente después de
haber entendido la verdad, ya no hay sacrificio para los pecados.
\bibverse{27} Lo único que queda es el temor, la espera de un juicio
inminente y el fuego terrible que destruye a los que son rebeldes con
Dios. \bibverse{28} Quien rechaza la ley de Moisés es llevado a muerte
sin misericordia, ante la evidencia de dos o tres testigos.
\bibverse{29} ¿Cuánto más merecedores de castigo creen que serán quienes
hayan pisoteado al Hijo de Dios, siendo que han menospreciado la sangre
que selló el pacto que nos santificaba, considerándolo como ordinario y
trivial, y que han abusado del Espíritu de gracia? \bibverse{30}
Conocemos a Dios, y él dijo: ``Me aseguraré de hacer justicia; le daré a
la gente lo que merece.'' También dijo: ``El Señor juzgará a su
pueblo.'' \bibverse{31} ¡Cosa terrible es caer en manos del Dios vivo!

\bibverse{32} Recuerden el pasado, cuando después de entender la
verdad,\footnote{\textbf{10:32} Literalmente ``fueron iluminaos.''}
experimentaron gran sufrimiento. \bibverse{33} En ocasiones fueron
mostrados como espectáculos, siendo insultados y atacados. En otros
tiempos ustedes se mantuvieron siendo solidarios con los que sufrían.
\bibverse{34} Mostraron compasión con los que estaban en la cárcel, y
entregaron con alegría sus posesiones cuando les fueron confiscadas,
sabiendo que cosas mejores vendrán para ustedes, cosas que realmente
perdurarán.

\bibverse{35} Así que no pierdan su fe en Dios, porque será recompensada
con abundancia. \bibverse{36} Es necesario que sean pacientes, para que
habiendo hecho la voluntad de Dios, reciban lo que él ha prometido.
\bibverse{37} ``En poco tiempo vendrá, tal como lo dijo, y no tardará.
\bibverse{38} Los que hacen lo recto vivirán por su fe en Dios, y si se
retractan de su compromiso, no me agradaré de ellos.''\footnote{\textbf{10:38}
  10:37-38. Esta es más bien una referencia libre a Isaías 26:20 y a
  Habacuc 2:3-4. Sin duda el que prometió regresar, en este contexto, es
  visto como Jesús.} \bibverse{39} Pero nosotros no somos la clase de
personas que se retracta y termina en la perdición. Nosotros somos los
que creemos en Dios y su salvación.

\hypertarget{section-10}{%
\section{11}\label{section-10}}

\bibverse{1} Ahora bien, nuestra fe en Dios es la seguridad de lo que
esperamos, la evidencia de lo que no podemos ver. \bibverse{2} Los que
vivieron hace mucho tiempo, creyeron en Dios y eso fue lo que les hizo
obtener la aprobación de Dios. \bibverse{3} Mediante nuestra fe en Dios
comprendemos que todo el universo fue creado por su mandato, y que lo
que se ve fue hecho a partir de lo que no se puede ver.

\bibverse{4} Por la fe en Dios Abel ofreció a Dios mejor sacrificio que
Caín, y por eso Dios lo señaló como alguien que vivía rectamente. Dios
lo demostró al aceptar su ofrenda. Aunque Abel ha estado muerto por
mucho tiempo, todavía Dios nos habla por medio de lo que él hizo.
\bibverse{5} Por fe en Dios Enoc fue llevado al cielo para que no
experimentara la muerte. Y no pudieron encontrarlo en la tierra porque
fue llevado al cielo. Y antes de esto, a Enoc se le conocía como alguien
que agradaba a Dios.

\bibverse{6} ¡No podemos esperar que Dios se agrade de nosotros si no
confiamos en él! Todo el que se acerca a Dios debe creer que él existe,
y que recompensa a quienes lo buscan.

\bibverse{7} Noé creyó en Dios, y él mismo le advirtió sobre cosas que
nunca antes habían sucedido. Y como Noé atendió lo que Dios le dijo,
construyó un arca para salvar a su familia. Y por fe en Dios, Noé mostró
que el mundo estaba equivocado, y recibió la recompensa de ser
justificado por Dios.

\bibverse{8} Por la fe en Dios Abrahán obedeció cuando Dios lo llamó
para ir a la tierra que él le daría. Y partió sin saber hacia dónde iba.
\bibverse{9} Por fe en Dios vivió en la tierra prometida, pero como
extranjero, viviendo en tiendas junto a Isaac y Jacob, quienes
participaron con él al ser herederos de la misma promesa. \bibverse{10}
Porque Abrahán buscaba una ciudad construida sobre fundamentos
duraderos, siendo Dios el constructor y hacedor de ella.

\bibverse{11} Por su fe en Dios, incluso la misma Sara\footnote{\textbf{11:11}
  Algunas versiones dicen Abrahán.} pudo concebir un hijo aunque fuera
muy vieja para hacerlo, pues creyó en Dios, que había hecho la promesa.
\bibverse{12} Por eso, los descendientes de Abrahán, (¡que ya estaba a
punto de morir!), se volvieron numerosos como las estrellas del cielo e
innumerables como la arena del mar.

\bibverse{13} Y todos ellos murieron creyendo aún en Dios. Aunque no
recibieron las cosas que Dios prometió, todavía las esperaban, como
desde la distancia y lo aceptaron gustosos, sabiendo que eran
extranjeros en esta tierra, pasajeros solamente.

\bibverse{14} Quienes hablan de esta manera dejan ver que esperan un
país que es de ellos. \bibverse{15} Porque si les importara el país que
habían dejado atrás, habrían regresado. \bibverse{16} Pero ellos esperan
un mejor país, un país celestial. Por eso Dios no se defrauda de ellos,
y se alegra de llamarse su Dios, porque él ha construido una ciudad para
ellos.

\bibverse{17} Abrahán creyó en Dios cuando fue puesto a prueba y ofreció
a Isaac como ofrenda a Dios. Abrahán, quien había aceptado las promesas
de Dios, incluso estuvo listo para dar a su único hijo,\footnote{\textbf{11:17}
  Por supuesto que Isaac no era literalmente el único hijo de Abrahán;
  el término griego indica primacía.} como ofrenda \bibverse{18} aun
cuando se le había dicho: ``Por medio de Isaac se contará tu
descendencia.'' \bibverse{19} Abrahán consideró las cosas y concluyó que
Dios podía resucitar a Isaac de los muertos. Y en cierto modo eso fue lo
que sucedió: Abrahán recibió de vuelta a Isaac de entre los muertos.

\bibverse{20} Por la fe en Dios, Isaac bendijo a Jacob y a Esaú,
considerando lo que el futuro traería. \bibverse{21} Confiando en Dios,
Jacob, casi a punto de morir, bendijo a los hijos de José, y adoró a
Dios apoyado en su bastón. \bibverse{22} Por fe en Dios, José, cuando se
acercaba su hora de muerte también, habló sobre el éxodo de los
israelitas, e instruyó sobre lo que debían hacer con sus huesos.
\bibverse{23} Por fe en Dios, los padres de Moisés lo ocultaron durante
tres meses después de nacer. Reconocieron que era un niño especial. Y no
temieron ir en contra de la orden que se había dado.

\bibverse{24} Por fe en Dios, Moisés, siendo ya adulto, se rehusó a ser
conocido como el hijo adoptivo de la hija del Faraón. \bibverse{25} Sino
que prefirió participar de los sufrimientos del pueblo de Dios antes que
disfrutar los placeres pasajeros del pecado. \bibverse{26} Y consideró
que el rechazo que experimentaría por seguir a Cristo sería de mayor
valor que la riqueza de Egipto, porque estaba concentrado en la
recompensa que vendría.

\bibverse{27} Por fe en Dios, salió de Egipto y no tuvo temor de la ira
del Faraón, sino que siguió adelante con sus ojos fijos en el Dios
invisible. \bibverse{28} Por fe en Dios, Moisés observó la Pascua y la
aspersión de la sangre en los dinteles, para que el ángel destructor no
tocara a los israelitas.\footnote{\textbf{11:28} ``Ángel'' e
  ``Israelitas'' por contexto.} \bibverse{29} Por fe en Dios, los
israelitas cruzaron en Mar Rojo como si caminaran por tierra seca. Y
cuando los egipcios quisieron hacer lo mismo, murieron ahogados.
\bibverse{30} Por la fe en Dios, los israelitas marcharon alrededor de
los muros de Jericó durante siete días, y los muros cayeron.
\bibverse{31} Por fe en Dios, Rahab, la prostituta, no murió junto a los
que rechazaban a Dios, porque había recibido a los espías israelitas en
paz.

\bibverse{32} ¿Qué otro ejemplo podría mostrarles? El tiempo no me
alcanza para hablar de Gedeón, Barac, Sansón, Jefté; o sobre David,
Samuel y los profetas. \bibverse{33} Ellos, por su fe en Dios
conquistaron reinos, hicieron lo recto, recibieron las promesas de Dios,
cerraron la boca de leones, \bibverse{34} apagaron incendios, escaparon
de la muerte por espada, eran débiles pero se volvieron fuertes,
lograron grandes cosas en guerras, y dirigieron ejércitos.

\bibverse{35} Muchas mujeres recibieron a sus familiares con vida por
medio de la resurrección. Otros fueron torturados, al negarse a rechazar
a Dios para ser perdonados, porque querían ser parte de una mejor
resurrección. \bibverse{36} E incluso otros recibieron insultos y
latigazos; y fueron encadenados y encarcelados. \bibverse{37} Algunos
fueron apedreados, tentados, muertos a espada. Algunos fueron vestidos
con pieles de corderos y cabras: destituidos, oprimidos y maltratados.
\bibverse{38} Les digo que el mundo no era digno de tener a tales
personas errantes en los desiertos y montañas, viviendo en cuevas y en
huecos debajo de la tierra.

\bibverse{39} Todas estas personas, aunque tenían la aprobación de Dios,
no recibieron lo que Dios había prometido. \bibverse{40} Él nos ha dado
algo aún mejor, para que ellos no llegaran a la plenitud sin nosotros.

\hypertarget{section-11}{%
\section{12}\label{section-11}}

\bibverse{1} Por eso, siendo que estamos rodeados de tal multitud de
personas que demostraron su fe en Dios, despojémonos de todo lo que nos
detiene, del pecado seductor que nos hace tropezar, y sigamos corriendo
la carrea que tenemos por delante. \bibverse{2} Debemos seguir con la
mirada puesta en Jesús, el autor y perfeccionador de nuestra fe en Dios.
Pues por el gozo que tenía delante, Jesús soportó la cruz, sin
importarle su vergüenza, y se sentó a la diestra del trono de Dios.
\bibverse{3} Piensen en Jesús, quien soportó tal hostilidad de un pueblo
pecador, y así no se cansarán ni se desanimarán.

\bibverse{4} Hasta ahora, la lucha contra el pecado no les ha costado su
sangre. \bibverse{5} ¿Acaso han olvidado\footnote{\textbf{12:5} O
  ``Ustedes han olvidado.''} el llamado de Dios cuando les habla como a
hijos suyos? Él dice: ``Hijo mío, no tomes con ligereza la disciplina de
Dios, ni te des por vencido cuando te corrige. \bibverse{6} Porque el
Señor disciplina a los que ama, y castiga a todos los que recibe como
sus hijos.'' \bibverse{7} Así que sean pacientes cuando experimenten la
disciplina de Dios, porque quiere decir que los está tratando como a sus
hijos. ¿Qué hijo no experimenta la disciplina de su padre? \bibverse{8}
Si no reciben disciplina, (la cual todos hemos recibido), entonces son
ilegítimos, y no son hijos de verdad. \bibverse{9} Porque si
respetábamos a nuestros padres terrenales que nos disciplinan, ¿cuánto
más deberíamos estar sujetos a la disciplina de nuestro Padre
espiritual, que nos conduce a la vida? \bibverse{10} Ellos nos
disciplinaron por un tiempo, en lo que ellos consideraban inapropiado,
pero Dios lo hace por nuestro bien, a fin de que podamos participar de
su carácter santo. \bibverse{11} Cuando la recibimos, la disciplina nos
parece dolorosa, y no sentimos que traiga felicidad. Pero después
produce paz en los que han sido entrenados de esta forma para hacer lo
recto.

\bibverse{12} Así que fortalezcan sus manos cansadas, y sus rodillas
débiles. \bibverse{13} Tracen un camino recto sobre el cual caminar,
para que los que son inválidos no se descarríen, sino que sean sanados.
\bibverse{14} Esfuércense por estar en paz con todos y buscar la
santidad, pues de lo contrario no verán al Señor. \bibverse{15}
Asegúrense de que no les falte la gracia de Dios, en caso de que surja
alguna causa de amargura y tribulación y termine corrompiendo a muchos
entre ustedes. \bibverse{16} Asegúrense de que ninguno sea sexualmente
inmoral o profano como Esaú. Él vendió su primogenitura por una sola
comida. \bibverse{17} Recuerden que incluso quiso recibir la bendición
después que le fue negada. Y aunque lo intentó, y lloró amargamente, no
pudo cambiar lo que había hecho.

\bibverse{18} Ustedes no han llegado a una montaña de verdad\footnote{\textbf{12:18}
  Sin duda en este contexto se hace referencia al Monte Sinaí.} que
pueda tocarse, ni a un lugar que arda con fuego, ni tampoco a un lugar
de tormenta u oscuridad, \bibverse{19} donde se haya escuchado una
trompeta o voz que habla, y quienes oyeron esa voz rogaron no volver a
oírla nunca más. \bibverse{20} Porque no pudieron obedecer lo que se les
dijo, como por ejemplo: ``Incluso si un animal toca la montaña, será
apedreado hasta la muerte.'' \bibverse{21} Semejante panorama era tan
aterrador, que el mismo Moisés dijo: ``¡Tengo tanto miedo que estoy
temblando!''

\bibverse{22} Pero ustedes han llegado al Monte de Sión, la ciudad del
Dios viviente, la Jerusalén celestial, con sus miles y miles de ángeles.
\bibverse{23} Han venido a la iglesia de los primogénitos cuyos nombres
están escritos en el cielo; a Dios, el juez de todos, y donde están las
personas buenas, cuyas vidas están completas. \bibverse{24} Han venido a
Jesús, quien participa con nosotros de esta nueva relación de pacto; han
venido a la sangre esparcida que tiene más valor que la de
Abel.\footnote{\textbf{12:24} Probablemente quiere decir que Jesús
  derramó su sangre en un espíritu de perdón, mientras que en el
  contexto de la primera muerte Dios hace referencia a la sangre de
  Abel, como pidiendo venganza.} \bibverse{25} ¡Asegúrense de no
rechazar al que les está hablando! Si ellos no pudieron escapar cuando
rechazaron a Dios en la tierra, sin duda alguna nosotros tampoco
podremos escapar si volvemos nuestra espalda a Dios, quien nos advierte
desde el cielo. \bibverse{26} En ese tiempo la voz de Dios agitó la
tierra, pero ahora su promesa es: ``Una vez más voy a agitar no solo la
tierra sino también el cielo.'' \bibverse{27} La expresión ``una vez
más,'' indica que toda la creación será agitada y removida para que solo
permanezca lo inconmovible.

\bibverse{28} Siendo que estamos recibiendo un reino inconmovible,
tengamos una actitud llena de gracia, para que sirvamos a Dios de una
manera que le agrade, con reverencia y respeto. \bibverse{29} Porque
``nuestro Dios es fuego consumidor.''\footnote{\textbf{12:29} Cita de
  Deut. 4:24.}

\hypertarget{section-12}{%
\section{13}\label{section-12}}

\bibverse{1} ¡Que siempre permanezca el amor que tienen unos por otros
como hermanos y hermanas! \bibverse{2} No olviden mostrar amor por los
extranjeros también, porque al hacerlo muchos han recibido ángeles sin
saberlo. \bibverse{3} Acuérdense de los que están en la cárcel, como si
ustedes estuvieran presos con ellos. Acuérdense de aquellos que son
maltratados, como si ustedes sufrieran físicamente con ellos.

\bibverse{4} Todos deben honrar el matrimonio. Los esposos y esposas
deben ser fieles unos a otros.\footnote{\textbf{13:4} Literalmente, ``la
  cama no contaminada.''} Pues Dios juzgará a los adúlteros.
\bibverse{5} No amen el dinero. Estén contentos con lo que tienen. Dios
mismo dijo: ``Nunca te defraudaré; nunca te abandonaré.'' \bibverse{6}
Por eso podemos decir con toda confianza: ``Es Señor es mi ayudador, por
lo tanto no temeré. ¿Qué puede hacerme cualquier persona?'' \bibverse{7}
Recuerden a los líderes que les enseñaron la palabra de Dios. Miren
nuevamente los frutos de sus vidas, e imiten su fe en Dios. \bibverse{8}
Jesucristo es el mismo ayer, hoy y para siempre.

\bibverse{9} No se distraigan con distintas clases de enseñanzas
extrañas. Es mejor que la mente esté convencida por gracia y no por
leyes en lo que concierne a los alimentos.\footnote{\textbf{13:9} Aquí,
  la palabra simplemente es ``comida,'' pero el contexto que sigue se
  refiere a la ley ceremonial y a los tipos de comida que se permitían.}
Los que seguían tales leyes no lograron nada. \bibverse{10} Tenemos un
altar del cual no pueden comer los sacerdotes del tabernáculo.
\bibverse{11} Los cuerpos muertos de animales, cuya sangre es llevada
por el sumo sacerdote al lugar santísimo como ofrenda para el pecado,
son quemados a las afueras del campamento. \bibverse{12} Del mismo modo,
Jesús, murió también fuera de las puertas de la ciudad para santificar
al pueblo de Dios por medio de su propia sangre. \bibverse{13} Así que
vayamos a él, fuera del campamento, y experimentemos su vergüenza.
\bibverse{14} Pues no tenemos una ciudad permanente en la cual vivir
aquí, sino que esperamos un hogar que está por venir. \bibverse{15}
Ofrezcamos, pues, por medio de Jesús, un sacrificio continuo de alabanza
a Dios, es decir, hablando bien de Dios, y declarando su
carácter.\footnote{\textbf{13:15} Literalmente, ``nombre,'' que a menudo
  se refiere a la naturaleza y carácter de la persona que se describe.
  Esto se logra ver en algunas expresiones como ``ser de buen nombre,''
  para indicar el carácter.} \bibverse{16} Y no olviden hacer lo bueno,
y compartir lo que tienen con otros, porque Dios se agrada cuando hacen
tales sacrificios. \bibverse{17} Sigan a sus líderes, y hagan lo que
ellos les piden, porque ellos cuidan de ustedes y darán cuenta. Actúen
de tal manera que ellos puedan hacerlo con alegría, y no con tristeza,
pues eso no sería bueno para ustedes.

\bibverse{18} Por favor, oren por nosotros. Pues estamos seguros de que
hemos actuado bien y con buena conciencia, procurando siempre hacer lo
correcto en cada situación. \bibverse{19} De verdad quiero que oren
mucho para que pueda ir pronto a verlos.

\bibverse{20} Ahora pues, que el Dios de paz que resucitó de los muertos
a nuestro Señor Jesús, el gran pastor de las ovejas, y lo hizo con la
sangre de un pacto eterno, \bibverse{21} provea todo lo bueno para
ustedes a fin de que puedan cumplir su voluntad. Que obre en nosotros,
haciendo su voluntad, por medio de Jesucristo, a él sea la gloria por
siempre y para siempre. Amén.

\bibverse{22} Quiero animarlos, hermanos y hermanas, a que pongan
cuidado a lo que les he dicho en esta pequeña carta. \bibverse{23} Sepan
que Timoteo ha sido liberado. Si llega pronto aquí, iré con él a verlos.
\bibverse{24} Envíen mi saludo a todos sus líderes, y a todos los
creyentes que hay allá. Los creyentes que están aquí en Italia envían
sus saludos. \bibverse{25} Que el Dios de gracia esté con todos ustedes.
Amén.
