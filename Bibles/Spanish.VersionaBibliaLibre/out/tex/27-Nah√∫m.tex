\hypertarget{section}{%
\section{1}\label{section}}

\bibverse{1} Profecía acerca de Nínive: El rollo de la visión que vino a
Nahúm de Elcos. \bibverse{2} El Señor es un Dios celoso y vengador. El
Señor Dios es vengador y lleno de enojo. El Señor toma venganza de sus
enemigos, y está enojado con los que son hostiles hacia él. \bibverse{3}
El Señor es lento en enojarse, tiene gran poder y no dejará al culpable
sin castigo. Él camina en medio del torbellino y la tormenta; las nubes
son como polvo bajo sus pies. \bibverse{4} Él da la orden y el mar se
seca, así mismo seca todos los ríos. Basán y Carmelo\footnote{\textbf{1:4}
  Estos don dos sitios famosos por tener buenos pastizales.} se
marchitan; la prosperidad del Líbano se desvanece. \bibverse{5} Los
montes tiemblan en su presencia y las colinas se derriten. La tierra
tiembla ante él, todo el mundo y los que en él habitan. \bibverse{6}
¿Quién puede resistir su furia? ¿Quién puede soportar el ardor de su
ira? Su enojo brota como fuego derretido y rompe las rocas en pedazos.

\bibverse{7} El Señor es bueno, es un lugar seguro en momento de
tribulación. Él cuida de los que confían en él, \bibverse{8} pero los
que están contra él serán arrastrados por una gran inundación hasta ser
destruidos. Él va tras sus enemigos hasta hacerlos llegar a la oscuridad
de la muerte. \bibverse{9} ¿Por qué conspiran contra el Señor? Él
acabará con su conspiración por completo, y la miseria no se levantará
dos veces. \bibverse{10} Ellos\footnote{\textbf{1:10} Los enemigos de
  Dios.} se enredan como los que quedan atrapados en medio de arbustos
con espinas; como borrachos que han bebido y se han embriagado. Serán
quemados por completo como la paja seca.\footnote{\textbf{1:10} Este
  versículo es reconocido como uno de los versículos más difíciles de
  traducir en la Biblia, ya que la interpretación específica es
  incierta. Sin embargo, queda claro el punto principal sobre la
  destrucción de los que se oponen a Dios.} \bibverse{11} Uno de ustedes
conspira el mal contra el Señor, uno que maquina maldad.

\bibverse{12} Esto es lo que dice el Señor: Aunque sean fuertes y
numerosos, ellos serán destruidos y morirán. Aunque yo les he causado
angustia, ya no lo haré más. \bibverse{13} Ahora romperé el yugo que han
puesto sobre sus cuellos y romperé las cadenas con que los han atado.

\bibverse{14} Esto es lo que el Señor ha dicho respecto a ti:\footnote{\textbf{1:14}
  Refiriéndose al pueblo de Nínive.} No tendrán descendientes que lleven
tu nombre. Yo destruiré los dioses en tus templos, todos los ídolos de
madera y de metal. Cavaré tu tumba, porque te has depravado.

\bibverse{15} ¡Mira! Viene un mensajero desde las montañas y trae las
buenas nuevas, proclamando paz. Celebra, Judá, tus festivales religiosos
y guarda tus votos, porque los enemigos malvados no invadirán tu tierra
nunca más. Serán destruidos por completo.

\hypertarget{section-1}{%
\section{2}\label{section-1}}

\bibverse{1} ¡El que dispersa\footnote{\textbf{2:1} O ``hace pedazos.''}
ha venido a atacarte! ¡Cuiden las fortalezas! ¡Vigilen los caminos!
¡Prepárense! ¡Saquen a cada soldado!

\bibverse{2} (Porque el Señor restaurará el esplendor del pueblo de
Jacob, así como restaurará el esplendor de Israel, pues los invasores
los han saqueado y han destruido su
tierra.\textsuperscript{{[}\textbf{2:2} Literalmente, ``ramas de
vid.''{]})}{[}\textbf{2:2} Esta oración se ha puesto entre paréntesis
para indicar que no es parte de la descripción del ejército atacante ni
de su comandante. {]}

\bibverse{3} Los escudos de sus soldados principales están teñidos de
rojo; los guerreros visten de escarlata. Sus carruajes brillan como
fuego bajo la luz del sol al prepararse para la batalla. Levantan y
sacuden sus lanzas con ástiles de madera\footnote{\textbf{2:3} La
  palabra para lanza aquí es la misma que para mencionar su madera, y se
  debate entre si es pino, criprés o abeto.}. \bibverse{4} Los carruajes
se precipitarán por las calles, yendo de aquí para allá por las plazas.
Tan brillantes como antorchas, corren como relámpagos. \bibverse{5} Él
alza su voz dando órdenes a sus oficiales. Ellos tropiezan mientras se
precipitan para atacar la muralla. La embestida está lista. \bibverse{6}
Las puertas de los ríos se abren, y el palacio queda
destruido.\footnote{\textbf{2:6} O ``se diluye en su temor.''}
\bibverse{7} ``La reina'' Nínive\footnote{\textbf{2:7} El significado de
  la palabra usada aquí es incierto y no aparece en ninguna otra parte
  en el Antiguo Testamento.} queda despojada y es llevada en exilio, con
sus siervas que lloran como palomas mientras golpean sus pechos.
\bibverse{8} Nínive es como un estanque lleno de agujeros, y sus
habitantes son como agua que se sale del estanque. ``¡Deténganse!
¡Deténganse!'' grita la gente, pero nadie vuelve su rostro. \bibverse{9}
¡Tomemos el botín de plata! ¡Tomemos el oro! Hay innumerables cosas para
tomar, hay de todo lo que puedas desear. \bibverse{10} ¡Nínive queda
desierta, destruida y desvastada! Los corazones desfallecen, las
rodillas tiemblan, hay dolor en los estómagos. Los rostros de todos
palidecen.

\bibverse{11} ¿Dónde está el foso de los leones? ¿Cuál es el sitio donde
se alimentan los leobnes jóvenes? ¿Dónde está el león, la leona y su
cachorro, quienes no temían a nadie?\footnote{\textbf{2:11} El símbolo
  del león era ampliamente usado por los asirios, y refleja también el
  trato cruel que le daban a sus víctimas.} \bibverse{12} El león
despedaza la carne para sus cachorros, y estrangula las presas para su
leona. Llena el foso con presas, y su guarida con cadáveres.

\bibverse{13} ¡Anda con cuidado! Porque yo estoy contra ti, declara el
Señor Todopoderoso. Prenderé fuego a tus carruajes y se consumirán hasta
reducirse a humo. Tus jóvenes fuertes\footnote{\textbf{2:13}
  Literalmente, ``leones jóvenes.''} morirán a espada. Yo impediré que
sigas saqueando a otros pueblos.\footnote{\textbf{2:13} Literalmente,
  ``quitaré tu presa de la tierra.''} No se oirán más las exigencies de
tus emisarios\footnote{\textbf{2:13} Emisarios: los asirios enviaban a
  sus representantes a otras naciones para exigir sometimiento y
  tributos.}.

\hypertarget{section-2}{%
\section{3}\label{section-2}}

\bibverse{1} ¡Cuán grande es el desastre que viene sobre esta ciudad
sanguinaria, llena de traición! Se ha llenado con la riqueza que ha
robado y sus víctimas son incontables.\footnote{\textbf{3:1}
  Literalmente, ``su presa nunca se va.''} \bibverse{2} ¡Escuchen el
sonido, el chasquido de los látigos, el estruendo de las ruedas, los
caballos galopando, y los carruajes se sacuden! \bibverse{3} ¡Jinetes a
cargo, con espadas y lanzas que brillan! Muchos difuntos, montones de
cadáveres e innumerables cuerpos, tantos que la gente tropieza con
ellos.

\bibverse{4} Todo esto es el resultado de la prostitución de Nínive, la
prostituta, la bella amante con sus mortales encantos con los que seduce
a las naciones a la esclavitud por su esclavitud y hechicería.
\bibverse{5} ¡Anden con cuidado! Porque yo estoy contra ustedes, declara
el Señor Todopoderoso. Yo levantaré tus faldas sobre tu cara y dejaré
que las naciones vean tu desnudez, y que los reinos vean tu vergüenza.
\bibverse{6} Yo echaré inmundicia sobre ti, te trataré con desprecio, y
serás un espectáculo ante todos. \bibverse{7} Entonces todos los que te
vean te rechazarán, diciendo: ``¡Ha caído Nínive! ¿Pero quién lamentará
tu pérdida?'' ¿Dónde encontraré a alguien que pueda consolarte?
\bibverse{8} ¿Eres mejor que la ciudad de Tebas\footnote{\textbf{3:8}
  Literalmente, ``No Amón,'' la ciudad del dios egipcio Amén. Había sido
  destruida anteriormente por los asirios.} en el río Nilo, rodeada de
agua? El agua fue su defensa, y el agua fue su muralla,\footnote{\textbf{3:8}
  El Qumran pesher (comentario) sobre el libro de Nahúm aclara que el
  pronombre se refiere a la ciudad (1QpNah).} \bibverse{9} La ciudad
gobernó a Egipto y Etiopía.\footnote{\textbf{3:9} Literalmente, ``Cus.''}
Put y Libia fueron sus aliados. \bibverse{10} Sin embargo, su pueblo
también fue exiliado, y llevado en cautividad. Sus bebés fueron
descuartizados por las calles. Sus nobles fueron atados con cadenas y
llevados como sirvientes, elegidos al azar.

\bibverse{11} Tú también te comportarás como un borracho. Te ocultarás
con temor, tratando de refugiarte del enemigo. \bibverse{12} Todos tus
castillos son como higueras con fruto maduro. Cuando sacuden el árbol,
el fruto cae en la boca de los que comen. \bibverse{13} ¡Mira! tus
soldados son mujeres de entre tu pueblo. Las puertas de tu nación están
abiertas de par en par ante tus enemigos. Los barrotes de las puertas
serán quemados. \bibverse{14} ¡Guarda agua para que estés lista para el
asedio! ¡Refuercen sus castillos! Vayan a las barredas y mezclen bien el
cemento. ¡Preparen los moldes de ladrillos pronto! \bibverse{15} Pero
aún allí+ 3.15 Refiriénrose al asedio. el fuego los consumirá, y serán
destruidos con espada. Serán destruidos como si fueran devorados por una
plaga de langostas. Así que multiplíquense ustedes también como
langostas, como una plaga de langostas. \bibverse{16} Tú multiplicaste
tus comerciantes, tanto que son más que las estrellas del cielo. Pero
como las langostas, desnudan todo lo que encuentran a su paso y se van.
\bibverse{17} Tus líderes son como langostas, tus oficiales son como una
plaga de langostas. Yacen en los muros en el día frío, pero cuando el
sol sale, se van volando y nadie sabe a dónde han ido. \bibverse{18} El
rey de Asiria, tus pastores están dormidos, tus príncipes están
adormecidos.+ 3.18 En las Escrituras la muerte a menudo se compara al
sueño. Por lo tanto, este versículo quiere decir que todos los líderes
que cuidaban del pueblo están muertos. Tu pueblo está disperso por las
montañas y nadie puede reunirlo. \bibverse{19} No hay forma de sanar tus
lesiones, y estás herido de gravedad. Todos los que oyen esta noticia
aplaudirán por lo que te ha sucedido, porque ¿acaso hay quien haya
escapado de tu constante crueldad?
