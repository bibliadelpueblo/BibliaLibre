\hypertarget{section}{%
\section{1}\label{section}}

\bibverse{1} Este es el mensaje que el Señor le dio a Sofonías. Él era
el hijo de Cusí, hijo de Guedalías, hijo de Amarías, hijo de
Ezequías.\footnote{\textbf{1:1} Probablemente el rey Ezequías, uno de
  los antiguos reyes de Judá.} Esto pasó cuando Josías, hijo de Amón,
era rey de Judá.

\bibverse{2} Yo destruiré por complete todo de la faz de la tierra,
declara el Señor. \bibverse{3} Yo destruiré a toda persona y animal,
destruiré a las aves del vielo, y los peces del mar. Acabaré\footnote{\textbf{1:3}
  Literalmente, ``obstáculos,'' lo que hace que el significado no esté
  claro.} con los malvados, y destruiré a la raza humana de la faz de la
tierra. \bibverse{4} Golpearé\footnote{\textbf{1:4} Literalmente,
  ``extenderé mi mano contra.''} a Judá y a todos los que viven en
Jerusalén. Además destruiré todo lo que queda de su culto a Baal, junto
con sus sacerdotes paganos para que hasta sus nombres sean
olvidados.\footnote{\textbf{1:4} Implícito.} \bibverse{5} Destruiré
también a los que suben a las azoteas para inclinarse ante el sol, la
luna y las estrellas. Ellos también se inclinan y juran fidelidad al
Señor, pero lo mismo hacen con Milcón.\footnote{\textbf{1:5} O
  ``Molec,'' un dios pagano.} \bibverse{6} Destruiré a os que una vez
adoraron al Señor y dejaron de hacerlo. Ellos no buscan al Señor ni
piden mi ayuda.

\bibverse{7} ¡Callen ante el Señor Dios! Porque el día del Señor está
cerca: el Señor ha preparado un sacrificio consagrado a sus
invitados.\footnote{\textbf{1:7} En este contexto, Israel es el
  sacrificio, y los babilonios son los ``invitados.''} \bibverse{8}
Entonces, en el día del sacrificio del Señor yo castigaré a los
oficiales y a los hijos del rey, y a los que siguen los caminos
paganos\footnote{\textbf{1:8} Literalmente, ``los que se visten con
  ropas extranjeras.''}. \bibverse{9} También castigaré a los que saltan
por encima del umbral.\footnote{\textbf{1:9} Se debate sobre este
  significado. Algunos piensan que era una costumbre pagana (ver, por
  ejemplo, 1 Sam. 5:4-5). Otros lo vinculan con el siguiente versículo y
  lo ven como un deseo de robar a los pobres.} Ese día castigaré a los
que llenan las casa de sus amos con violencia y engaño. \bibverse{10}
Ese día, declara el Señor, un grito de lamento saldrá de la Puerta del
Pez, un lamento saldrá del Segundo Barrio, y un fuerte estrépito de las
montañas. \bibverse{11} Los que viven en el Barrio del
Mercado\footnote{\textbf{1:11} Literalmente, ``El mortero.''} se
lamentarán porque los mercaderes\footnote{\textbf{1:11} Literalmente,
  ``el pueblo de Canaán.''} estarán destruidos, así como los que
comercian la plata. \bibverse{12} En ese tiempo, buscaré por toda
Jerusalén con lámparas y castigaré a los jactanciosos, que son como los
residuos de vino podrido, y que dicen para sí mismos: ``El Señor no hará
bien ni mal.''\footnote{\textbf{1:12} En otras palabras, rechazan al
  Señor porque no creen que él se preocupa por ellos.} \bibverse{13} Sus
posesiones serán saqueadas y sus casas quedarán destruidas. Construirán
casas pero no vivirán en ellas. Plantarán viñedos, pero no beberán el
vino.

\bibverse{14} El gran día del Señor está cerca y se aproxima con
prontitud. Será un día amargo, e incluso los guerreros clamarán en voz
alta. \bibverse{15} Será un día de enojo,\footnote{\textbf{1:15} La
  expresión de que Dios está enojado o lleno de ira es una imagen
  frecuente en los escritos proféticos, pero no debe entenderse de la
  misma manera que la ira humana. La ira de Dios no es emocional como
  una especie de ``niebla roja'' irreflexiva, sino una oposición de
  principios a todo lo que es malo. Dios usa esta ira para tratar de
  convencer a los que están equivocados de hacer lo correcto por su
  propio bien, no porque él ``se enoje'' y arremeta contra ellos. La ira
  humana es egocéntrica; la ira divina está centrada en el otro.} un día
de tribulación y angustia; un día de ruina y desastre; un día de
oscuridad y penumbra, un día aciago con nubes negras; \bibverse{16} un
día de sonido de trompetas y gritos de guerra contra ciudades
fortificadas y torres de vigilancia. \bibverse{17} Traeré angustia sobre
la humanidad, haciéndolos caminar como ciegos porque han pecado contra
el Señor. Su sangre se derramará como una gran cantidad de polvo, y sus
intestinos como el excremento. \bibverse{18} Su plata y su oro no los
salvarán el día de la ira del Señor. Toda la tierra será consumida con
el fuego del celo\footnote{\textbf{1:18} ``Celo'' cuando se aplica a
  Dios no es lo mismo que hablar de los celos humanos. Se refiere al
  fuerte deseo de Dios de que las personas lo sigan solo a él, ya que
  solo él puede salvarlos. Quiere una relación exclusiva porque sabe que
  cualquier otra cosa lleva al desastre.} de su ira. Él se asegurará de
que la destrucción de los habitantes del mundo sea repentina y completa.

\hypertarget{section-1}{%
\section{2}\label{section-1}}

\bibverse{1} Reúnanse, sí, reúnanse, nación sin valor, \bibverse{2}
antes de que se emita el decreto, antes de que se marchiten y mueran
como la flor;\footnote{\textbf{2:2} Texto tomado de la Septuaginta, que
  también dice: ``antes de que desaparezcas como la paja en el viento.''}
antes de que la ira del Señor caiga sobre ustedes; antes de que el día
de la ira del Señor venga sobre ustedes. \bibverse{3} Miren al Señor,
todos ustedes, habitantes de la tierra que son humildes y siguen sus
mandamientos. Procuren hacer lo recto, y traten de vivir con humildad.
Quizás serán protegidos\footnote{\textbf{2:3} Literalmente,
  ``ocultados.''} en el día de la ira del Señor. \bibverse{4} Gaza será
abandonada, Ascalón será destruida; Asdod será saqueada de noche, y
Ecrón será arrancada de raíz. \bibverse{5} ¡Grande es el desastre que
viene sobre ustedes, filisteos, ustedes habitantes de las costas y de la
tierra de Canaán! El Señor ha emitido juicio sobre ustedes. Los
destruiré, y no habrá sobrevivientes. \bibverse{6} La costa de su
territorio se convertirá en pastizales, con praderas para los pastores y
será lugar de rediles de ovejas. \bibverse{7} Le pertenecerá al
remanente de Judá. Allí apacentarán sus rebaños, y los pastores dormirán
en las casas abandonadas de Ascalòn. Porque el Señor su Dios estará con
ellos y los hará prósperos nuevamente.

\bibverse{8} He oído las burlas de los moabitas y los escarnios
desdeñosos de los amonitas que han insultado a mi pueblo y que han
enviado amenazas contra su territorio. \bibverse{9} Por ello, juro por
mi vida, declara el Dios Todopoderoso, el Dios de Israel, que los
moabitas serán como Sodoma, y los amonitas como Gomorra. Su tierra será
un lugar lleno de ortigas y sembrados de sal y ruinas para siempre. Y
los que quedan en mi pueblo los saquearán y ocuparán su tierra.
\bibverse{10} Esto es lo que recibirán como pago por su orgullo, porque
se burlaron y amenazaron al pueblo del Señor Todopoderoso. \bibverse{11}
El Señor los atemorizará, y hará morir de hambre a todos los dioses
terrenales. Todas las naciones adorarán al Señor, dondequiera que se
encuentren, en todo el mundo.

\bibverse{12} Ustedes, etíopes, morirán a espada. \bibverse{13} El Señor
golpeará a los asirios del norte y los destruirá. Desolará a Nínive, y
será una tierra valdía y desierta. \bibverse{14} El ganado se tumbará en
medio de la ciudad. Será el hogar de los animales salvajes. Las lechuzas
y los búhos\footnote{\textbf{2:14} Las aves que realmente se mencionan
  aquí sin inciertas, pero se encuentran en Levíticos y Deuteronomio
  como parte de los animales impuros.} se posarán en sus pilares. Su
clamor hará eco por las ventanas. Los escombros bloquearán las puertas,
y la madera de cedro quedará expuesta. \bibverse{15} Esto es lo que le
sucederá a esta ciudad triunfante que creyó estar segura. ``¡Mírenme!''
decías con arrogancia. ``¡No hay ciudad cuya grandeza sea como la mía!''
Pero has quedado desolada, y eres apenas el hogar de animales salvajes.
Todos los que pasan te señalarán con el dedo y se burlarán de ti con
desdén.

\hypertarget{section-2}{%
\section{3}\label{section-2}}

\bibverse{1} ¡Grande es el desastre que viene sobre ti, corrupta y
rebelde Jerusalén, que oprimes a la gente!\footnote{\textbf{3:1}
  ``Jerusalén'': implícito por contexto.} \bibverse{2} Tú \footnote{\textbf{3:2}
  Literalmente, ``ella,'' pero usar el pronombre en segunda persona hace
  que la advertencia sea más vívida.} no prestas atención a nadie ni
aceptas la corrección; no confías en el Señor, ni pides su
ayuda.\footnote{\textbf{3:2} Literalmente, ``no te acercas a Dios.''}
\bibverse{3} Tus líderes son codiciosos como leones rugientes. Tus
jueces son como lobos hambrientos que no dejan para para el día
siguiente. \bibverse{4} Tus profetas son hombres arrogantes y mentirosos
que corrompen lo sagrado, y quebrantan abiertamente la ley.\footnote{\textbf{3:4}
  O ``usan la ley en su propio beneficio.''}

\bibverse{5} Pero el Señor que hace justiciar aún está entre ustedes, y
no hará mal. Cada mañana emite su juicio, y cada día sin falta. Pero los
que actúan injustamente no tienen vergüenza. \bibverse{6} Yo he
destruido naciones. Sus castillos están abandonados, sus calles vacías,
y sus ciudades destruidas. No hay en ellas sobrevivientes. No siquiera
uno.

\bibverse{7} Me dije a mi mismo: ``De seguro ellos\footnote{\textbf{3:7}
  Refiriéndose al pueblo de Jerusalén.} me respetarán y aceptarán mi
correción. Entonces su hogares no serán destruidos para enseñarles la
lección.'' Pero por el contrario persistes en tu deseo de hacer el mal.

\bibverse{8} Solo espera, declara el Señor. Viene el día en que me
levantaré para mostrar la evidencia. Porque he decidido juntar a todas
las naciones y a los reyes para derramar mi ira sobre ellos, así como mi
furia y mi enojo. Toda la tierra será consumida con el fuego del
celo\footnote{\textbf{3:8} Ver en el versículo 1:18 la nota sobre el
  celo.} de mi ira. \bibverse{9} Porque entonces haré que las naciones
hablen con pureza, para que puedan orar y adorar juntas al Señor.
\bibverse{10} Desde lejos los ríos de Etiopía, mi pueblo esparcido, mis
adoradores, vendrán a traerme sus ofrendas.

\bibverse{11} Ese día no serás avergonzado por lo que hiciste al
rebelarte contra mi, porque yo quitaré de entre tu pueblo a los
orgullosos y jactanciosos. Nunca más mostrarás orgullo en mi monte
santo. \bibverse{12} Dejaré entre tu pueblo a los mansos y humildes, a
los que confían en el nombre del Señor. \bibverse{13} El pueblo de
Israel que queda no actuará con maldad, ni hablará con mentira. No se
engañarán unos a otros. Podrán comer en paz y dormir seguros, porque no
tendrán ningún temor.

\bibverse{14} ¡Canta, Jerusalén! ¡Grita Israel! ¡Alégrate y celebra con
todo tu corazón, Jerusalén! \bibverse{15} Porque el Señor se ha
arrepentido de castigarte, y ha enviado lejos a tus enemigos. El Señor,
el rey de Israel está contigo y nunca más tendrás que temer al desastre.
\bibverse{16} Ese día, el mensaje al pueblo de Jerusalén será: ``¡No
temas, ni te desanimes!''\footnote{\textbf{3:16} ``No se desanimen'':
  Literalmente, ``no debiliten sus manos.''} \bibverse{17} El Señor tu
Dios está en medio de ustedes como un poderoso guerrero que te salvas.
Se alegrará en ti. Renovará\footnote{\textbf{3:17} Septuaginta. En
  hebreo: ``él callará su amor'' no concuerda con la frase que le
  precede y le antecede.} su amor por ti. Cantará fuertemente celebrando
tu existencia. \bibverse{18} Yo reuniré a los que lloran por las fiestas
religiosas, y nunca más tendrán que soportar la vergüenza.\footnote{\textbf{3:18}
  Esta es una interpretación del hebreo que no está clara. El
  significado es que, cuando estaban en el exilio, los israelitas no
  podían celebrar sus festividades religiosas como lo deseaban y esto
  era motivo de desgracia para ellos.}

\bibverse{19} ¡Miren lo que haré! En ese tiempo me encargaré de los que
te han oprimido. Salvaré a los indefensos y traeré de regreso a los que
estaban dispersos. Convertiré su vergüenza en alabanza, y todo el mundo
los respetará. \bibverse{20} En ese tiempo, los traeré a casa, y los
reuniré. Les daré una buena reputación, y serán alabados por todos los
pueblos de la tierra, cuando yo restaure tu posición ante tus propios
ojos, dice el Señor.
