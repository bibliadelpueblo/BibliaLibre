\hypertarget{section}{%
\section{1}\label{section}}

\bibverse{1} En el principio, Dios creó los cielos y la tierra.
\bibverse{2} La tierra carecía de forma y estaba vacía, y la oscuridad
cubría la superficie del abismo. El Espíritu de Dios se movía sobre la
superficie de las aguas.

\bibverse{3} Y Dios dijo: ``¡Que haya luz!'' y hubo luz. \bibverse{4}
Dios vio que la luz era buena, y separó a la luz de la oscuridad.
\bibverse{5} Dios llamó a la luz ``día'' y a la oscuridad le llamó
``noche''. Así que hubo noche y mañana, lo cual fue el primer
día.\footnote{1.5 Es importante decir que el ``día'' se mide desde la
  oscuridad a la luz, que sigue siendo el método judío para calcular los
  días.}

\bibverse{6} Entonces Dios dijo: ``Que haya expansión\footnote{1.6
  ``Expansión:'' Las traducciones más antiguas a menudo han convertido
  esta palabra en ``firmamento'', tomando prestado del latín
  ``firmamentum''. Esto se refería a una antigua creencia de que el
  cielo era como una cúpula de metal forjado, y por lo tanto era un
  objeto físico tangible. Ahora se ha demostrado que esta es una idea
  equivocada. De hecho, las traducciones latinas de los siglos XVI y
  XVII suelen utilizar la palabra ``expansionem''.}en medio de las aguas
para dividirlas''. \bibverse{7} Así que Dios hizo una expansión para
separar las aguas que estaban arriba de las aguas, de las aguas que
estaban debajo. Y así sucedió. \bibverse{8} Dios llamó a la expansión
``cielo''.Entonces hubo noche y mañana, lo cual fue el segundo día.

\bibverse{9} Dios dijo: ``Que las aguas que están debajo del cielo se
junten en un solo lugar para que aparezca la tierra''.Y así sucedió.
\bibverse{10} Entonces Dios llamó al suelo ``tierra'' y a las aguas les
llamó ``mares''.Y Dios vio que era bueno.

\bibverse{11} Dios dijo: ``Que la tierra produzca vegetación: plantas
que produzcan semillas y árboles que produzcan frutos con semillas, cada
uno de su propia clase''.Y así sucedió. \bibverse{12} La tierra produjo
vegetación: plantas que producen semillas y árboles que producen frutos
con semillas, cada uno de su propia clase. Entonces Dios vio que era
bueno. \bibverse{13} Así que hubo noche y mañana, lo cual fue el tercer
día.

\bibverse{14} Dios dijo: ``Que haya luces en el cielo para separar el
día de la noche, y para que exista una forma de marcar las estaciones,
los días y los años. \bibverse{15} Habrá luces en el cielo que brillen
sobre la tierra''.Y así sucedió. \bibverse{16} Dios creó dos grandes
luces:\footnote{1.16 El hebreo tiene palabras para el sol y la luna,
  pero no se usan aquí, tal vez para evitar cualquier tentación de
  adorar al sol y a la luna.} la más grande a cargo del día, y la más
pequeña a cargo de la noche. También creó las estrellas. \bibverse{17}
Dios puso estas luces en el cielo para que brillaran sobre la tierra,
\bibverse{18} para que estuvieran a cargo del día y de la noche, y para
separar la luz de la oscuridad. Y Dios vio que era bueno. \bibverse{19}
Así que hubo noche y mañana, lo cual fue el cuarto día.

\bibverse{20} Y Dios dijo: ``Que las aguas se llenen de criaturas
vivientes, y que las aves vuelen por encima de la tierra, en el cielo''.
\bibverse{21} Así que Dios creó enormes animales marinos y todos los
seres vivos que nadan y que habitan en las aguas, cada uno de su propia
clase; así como cada ave que vuela, cada una según su especie. Y Dios
vio que era bueno. \bibverse{22} Dios los bendijo y dijo:
``Reprodúzcanse y multiplíquense, y llenen las aguas del mar, y que se
multipliquen las aves en toda la tierra''. \bibverse{23} Y así hubo
noche y después mañana, lo cual fue el quinto día.

\bibverse{24} Entonces Dios dijo: ``Que la tierra produzca criaturas
vivientes, cada una según su especie:rebaños, ganado, las criaturas
reptiles, los animales salvajes, cada uno de su propia clase''.Y sucedió
así. \bibverse{25} Dios hizo a los animales salvajes, al ganado, y a los
reptiles, a todos según su propia especie. Y Dios vio que esto era
bueno.

\bibverse{26} Entonces Dios dijo: ``Hagamos seres humanos según nuestra
imagen, y que sean como nosotros.\footnote{1.26 Este aspecto de ser
  ``como'' Dios transmite la idea de ser ``modelado'' por Dios. La
  palabra también se traduce como ``similitud'', ``figura'' o ``forma''.
  El aspecto más esencial de esta semejanza es seguramente el del
  carácter.}Ellos tendrán autoridad sobre los peces del mar y sobre las
aves que vuelan por los aires, sobre los animales y sobre toda la tierra
y las criaturas que se muevensobre ella''. \bibverse{27} Así que Dios
creó a los seres humanos según su propia imagen. Los creó a la imagen de
Dios, como varón y hembra.+ 1.27 La repetición de ``creado'' en este
versículo es significativa sin duda, por lo que se coloca en primer
lugar en cada frase. \bibverse{28} Dios los bendijo y les dijo:
``Reprodúzcanse y multiplíquense; vayan por toda la tierray gobiérnenla.
Tengan autoridad sobre los peces que están en el mar y sobre las aves
que vuelan por los aires, y sobre cada criatura que se mueve sobre la
tierra''.

\bibverse{29} Y Dios dijo: ``Miren, les he dado como alimento cada
planta que produce semilla de toda la tierra, y cada árbol que produce
fruto con semilla. \bibverse{30} Todas las plantas verdes las he dado a
todos los animales de la tierra, a las aves, y a cada criatura que se
mueve sobre la tierra, es decir, a todo ser vivo''.Y así sucedió.

\bibverse{31} Entonces Dios vio todo lo que había creado, y una vez más
vio que era muy bueno. Así hubo tarde y luego mañana, lo cual fue el
sexto día.

\hypertarget{section-1}{%
\section{2}\label{section-1}}

\bibverse{1} La creación de los cielos, la tierra y todo lo que hay en
ellos\footnote{2.1 ``Todo lo que hay en ellos'': literalmente, ``todo el
  conjunto de ellos''. La palabra usada para ``conjunto'' es usualmente
  un término militar que designa la formación de un ejército de
  soldados.}quedó terminada. \bibverse{2} Cuando llegó el séptimo día,
Dios había terminado el trabajo que había hecho, y descansó en el
séptimo día de todo el trabajo que había estado haciendo. \bibverse{3}
Dios bendijo el séptimo día, y lo apartó como día santo, porque en él
descansó de todo el trabajo que había hecho en la creación.

\bibverse{4} Este es el relato de la creación del Señor Dios, cuando
hizo los cielos y la tierra.

\bibverse{5} Hasta este momento no había plantas silvestres
\footnote{2.5 ``Plantas silvestres'': el término también puede
  significar ``arbustos'' o ``matorrales''.}ni cultivos creciendo sobre
la tierra, porque el Señor Dios no había enviado la lluvia, y porque no
había quien cultivara el suelo. \bibverse{6} El rocío brotaba de la
tierra y hacía que la superficie del suelo estuviera húmeda.
\bibverse{7} Entonces el Señor formó al hombre Adán+ 2.7 La palabra para
hombre es Adán, así que sirve para identificar tanto al primer hombre
como su nombre personal. Como no siempre está claro si el término se
refiere al hombre genéricamente o a la persona de Adán, esta versión ha
optado en la mayoría de los casos por traducir ``hombre'' como Adán, y
luego por extensión, ``la mujer'' como Eva, aunque no se la nombre
específicamente hasta el versículo 3:20. De esta manera el relato se
lleva a un nivel más personal. Además, nótese que la palabra para tierra
es ``adamah'', mostrando la estrecha conexión del hombre con la
tierra.con polvo de la tierra. Y sopló en sus fosas nasales el aliento
de vida, y Adán se convirtió en un ser vivo.

\bibverse{8} Entonces el Señor sembró un jardín en Edén, al oriente.
Allí puso al hombre Adán, al cual había creado. \bibverse{9} El Señor
creó toda clase de árboles para que crecieran en el jardín, árboles
hermosos y árboles que producían frutas agradables al paladar. El árbol
de la vida se encontraba en medio del jardín, así como el árbol del
conocimiento del bien y del mal. \bibverse{10} De Edén salía un río que
regaba el jardín, y desde allí se dividía en cuatro brazos.
\bibverse{11} El primero fue llamado Pisón, y pasaba por toda la tierra
de Havila, donde había oro. \bibverse{12} (El oro de esta tierra es
puro. Allí también hay bedelio\footnote{2.12 ``Bedelio'': referencias
  posteriores afirman que se trata de una resina aromática. No se sabe
  si es la misma sustancia que se menciona aquí.}y ónice.) \bibverse{13}
El segundo brazo fue llamado Gijón y rodea toda la tierra de Cus.+ 2.13
``Cus'': en gran parte del Antiguo Testamento este es otro nombre para
Etiopía; es incierto saber si es el caso aquí. \bibverse{14} El tercer
brazo fue llamado Tigris+ 2.14 ``Tigris'': literalmente ``Hidekel'', que
se cree que es el antiguo nombre hebreo del Tigris. Ver también Daniel
10:4.y rodeaba el oriente de la ciudad de Asur. El cuarto brazo fue
llamado Eufrates.+ 2.14 ``Eufrates'': literalmente ``Parat'', usualmente
considerado como sinónimo del Éufrates.

\bibverse{15} El Señor Dios puso al hombre en el Jardín de Edén para que
lo cultivara y cuidara de él. \bibverse{16} El Señor Dios le dio orden a
Adán: ``Eres libre de comer el fruto de todos los árboles del jardín,
\bibverse{17} pero no debes comer del árbol del conocimiento del bien y
del mal, porque el día que comas de él, será seguro que morirás''.

\bibverse{18} Entonces el Señor Dios dijo: No es bueno que Adán esté
solo. Haré a alguien que lo ayude, alguien que sea como él''.

\bibverse{19} El Señor Dios usó la tierra para hacer a los animales
salvajes y a todas las aves. A todos los animales los trajo hasta donde
estaba Adán para que les pusiera nombre, y Adán puso nombre a cada
criatura viviente. \bibverse{20} Adán le puso nombre a todo el ganado, a
todas las aves, y a los animales salvajes. Pero Adán no encontraba a
nadie que pudiera ayudarlo.

\bibverse{21} Así que el Señor hizo que Adán durmiera profundamente, y
mientras dormía Dios quitó una de las costillas de Adán y después volvió
a cerrar el lugar del cual tomó el tejido. \bibverse{22} Y el Señor hizo
a una mujer, usando la costilla que había tomado de Adán, y entonces se
la presentó a Adán.

\bibverse{23} ``¡Por fin!'' dijo Adán. ``Esta es hueso de mis huesos y
carne de mi carne. Ella será llamada mujer, porque fue sacada del
hombre''.\footnote{2.23 ``Hombre'': Tanto aquí como en el siguiente
  versículo se usa una palabra hebrea diferente.}

\bibverse{24} Esta es la razón por la cual el hombre deja a su padre y a
su madre y se une\footnote{2.24 ``Se une'': literalmente, ``se aferra''
  o ``se pega''.}a su esposa, y los dos se vuelven un solo ser.
\bibverse{25} Adán y su esposa Eva+ 2.25 Ver la nota en el versículo
2:7.estaban desnudos, pero no sentían vergüenza de ello.

\hypertarget{section-2}{%
\section{3}\label{section-2}}

\bibverse{1} La serpiente era más astuta que cualquiera de los otros
animales salvajes que el Señor Dios había hecho. Y le preguntó a Eva:
``¿En serio Dios dijo que no pueden comer del fruto de todos los
árboles\footnote{3.1 ``Cualquiera'': la palabra también podría ser
  traducida como ``todos''; sin embargo, esto significaría entonces que
  la serpiente estaba sugiriendo que Dios le había dicho a Adán y Eva
  que no comieran la fruta de ningún árbol del jardín, lo que parece
  menos probable.}del jardín?''

\bibverse{2} Entonces Eva le respondió a la serpiente: ``Podemos comer
de los árboles del jardín, pero no del fruto del árbol que está en medio
del jardín. \bibverse{3} Dios nos dijo: `No deben comer de ese árbol, y
ni siquiera tocarlo, pues de lo contrario morirán.'\,''\footnote{3.3
  ``De lo contrario, morirás:'' La palabra utilizada para ``de lo
  contrario'' puede indicar una posibilidad de que algo suceda, en lugar
  de una certeza absoluta. Así que la frase podría ser traducida, ``de
  lo contrario podrías morir'', una diferencia con la clara prohibición
  de Dios, también afirmando que Dios había dicho que el fruto no debía
  ser tocado.}

\bibverse{4} ``Por supuesto que no morirán'',le dijo la serpiente a Eva.
\bibverse{5} ``Lo que sucede es que Dios sabe que tan pronto coman de
él, verán las cosas de una manera distinta, y serán como Dios,
conociendo lo que es el bien y el mal''.

\bibverse{6} Eva vio que el fruto del árbol lucía bueno para comer. El
fruto se veía muy atractivo. Y Eva lo deseaba para obtener sabiduría.
Así que tomó del fruto y lo comió, y lo compartió con su esposo, que
estaba con ella, y él también comió. \bibverse{7} Tan pronto como
comieron del fruto, vieron todo diferente y se dieron cuenta de que
estaban desnudos. Así que cosieron hojas de higuera para cubrirse.

\bibverse{8} Al caer la noche y cuando soplaba la brisa del atardecer,
escucharon al Señor caminando en el jardín. Entonces Adán y Eva se
escondieron de la presencia del Señor entre los árboles del jardín.

\bibverse{9} Entonces el Señor llamó a Adán: ``¿Dónde estás?''

\bibverse{10} ``Te escuché caminando por el jardín y me asusté porque
estaba desnudo, y por eso me escondí'',respondió Adán.

\bibverse{11} ``¿Quién te dijo que estabas desnudo?'' le preguntó el
Señor Dios. ``¿Acaso comiste del árbol que te dije que no comieras?''

\bibverse{12} ``Fue la mujer que me diste quien me brindó del fruto del
árbol, y yo lo comí'',respondió Adán.

\bibverse{13} Entonces el Señor le preguntó a Eva: ``¿Por qué has hecho
esto?''

``La serpiente me engañó, y yo lo comí'',respondió ella.

\bibverse{14} Entonces el Señor le dijo a la serpiente: ``Por lo que has
hecho, serás maldita entre todos los animales. Te arrastrarás sobre tu
vientre y comerás polvo mientras vivas. \bibverse{15} Me aseguraré de
que tú y tus hijos, así como la mujer y sus hijos sean enemigos. Uno de
sus hijos aplastará tu cabeza, y tú herirás su talón''.

\bibverse{16} Dios le dijo a Eva: ``Haré que el embarazo sea más penoso,
y que dar a luz sea más doloroso. Sin embargo, tendrás deseo por tu
esposo y él te gobernará''.\footnote{3.16 ``Él tendrá el control sobre
  ti'' o ``también te deseará''.}

\bibverse{17} Y Dios le dijo a Adán: ``Por haber hecho\footnote{3.17
  ``Hecho'': la palabra es ``escuchado'', pero no en el sentido de sólo
  escuchar algo. Significa actuar conforme a lo que se ha escuchado, u
  obedecer.}lo que te dijo tu esposa, y comiste del fruto del árbol
sobre el cual te dije `No comas del fruto de este árbol,' el suelo ahora
estará maldito por tu culpa. Tendrás que trabajar arduamente para
cultivar los alimentos durante toda tu vida. \bibverse{18} Los cultivos
tendrán cardos y espinas, y tendrás que comer plantas silvestres.+ 3.18
Las plantas fueron originalmente asignadas a los animales. Ver 1:30.
\bibverse{19} Tendrás que sudar para cultivar suficiente comida hasta
que mueras y regreses a la tierra. Porque fuiste hecho del polvo de la
tierra, y al mismo polvo regresarás''.

\bibverse{20} Adán le puso por nombre Eva a su esposa, porque ella sería
la madre de todos los seres humanos. \bibverse{21} El Señor hizo
vestiduras con piel de animales para Adán y Eva y lo vistió.

\bibverse{22} Entonces el Señor miró una vez más: ``Veo que los seres
humanos\footnote{3.22 ``Los seres humanos'': literalmente, ``el
  hombre'', pero debe entenderse de manera inclusiva ya que Eva también
  había caído.}se han convertido en uno más como nosotros, y conocen
ahora tanto el bien como el mal. Ahora bien, si llegan a tomar el fruto
del árbol de la vida y lo comen, ¡vivirán para siempre!'' \bibverse{23}
Así que el Señor los expulsó del jardín de Edén. Envió a Adán a cultivar
el suelo del cual él mismo fue hecho. \bibverse{24} Después de sacarlos
del jardín, el Señor puso al oriente del jardín ángeles y una espada que
daba su resplandor en todas las direcciones. Esto con el fin de que no
pudieran acceder al árbol de la vida.

\hypertarget{section-3}{%
\section{4}\label{section-3}}

\bibverse{1} Adán durmió con su esposa Eva y ella quedó embarazada. Y
dio a luz a Caín, y dijo: ``Con la ayuda de Dios he hecho a un hombre''.
\bibverse{2} Después dio a luz a su hermano Abel. Él se convirtió en un
pastor de ovejas, mientras que Caín era un agricultor.

\bibverse{3} Algún tiempo después, Caín trajo el fruto de su cosecha
como ofrenda al Señor. \bibverse{4} Abel también trajo una ofrenda: el
primogénito de su rebaño, eligiendo las mejores partes como ofrenda. El
Señor se sintió agradado de Abel y su ofrenda, \bibverse{5} pero no se
agradó de Caín ni de su ofrenda, lo cual enojó a Caín en gran manera y
frunció el ceño con enfado.

\bibverse{6} Entonces el Señor le preguntó a Caín: ``¿Por qué estás
enojado? ¿Por qué te ves tan enfadado? \bibverse{7} Si hicieras lo
correcto, te verías contento.\footnote{4.7 ``te verías contento'':
  literalmente, ``animado''. En el versículo anterior, el significado
  literal es que el ``rostro de Caín decayó''. Así que lo opuesto sería
  que su rostro fuera ``levantado'', en otras palabras, se vería feliz.}Pero
si no haces lo correcto, el pecado será como animales agazapados en la
puerta de tu casa, listos para atacarte. El pecado desea apoderarse de
ti, pero tú debes mantener el control''.

\bibverse{8} Más tarde, mientras Caín hablaba con su hermano
Abel,\footnote{4.8 La Septuaginta y algunas otras versiones antiguas
  añaden aquí, ``salgamos a los campos''. La forma en que la frase está
  estructurada en el hebreo sugiere que faltan algunas palabras.}e iban
por los campos, Caín atacó a su hermano y lo mató.

\bibverse{9} ``¿Dónde está tu hermano Abel?'' le preguntó el Señor a
Caín.

``¿Cómo podría saberlo?'' respondió Caín. ``¿Acaso se supone que debo
ser el cuidador de mi hermano?''

\bibverse{10} ``¿Qué has hecho?'' le preguntó el Señor. ``La sangre de
tu hermano clama a mi desde la tierra. \bibverse{11} Por esto ahora
estarás más maldito que la tierra, porque la has impregnado con la
sangre de tu hermano. \bibverse{12} Y cuando cultives la tierra, no
producirá cosechas para ti. Siempre serás un prófugo, errando por toda
la tierra''.

\bibverse{13} ``Mi castigo es más de lo que puedo soportar'',respondió
Caín. \bibverse{14} ``¡Mira! Me expulsas en este instante, maldiciendo
la tierra y echándome de tu presencia. Ahora tendré que esconderme y
seré siempre un prófugo, errante por toda la tierra. ¡Y cualquiera que
me encuentre me matará!''

\bibverse{15} Pero el Señor respondió: ``No, Caín. Cualquiera que te
mate será castigado siete veces más''.El Señor puso una marca sobre Caín
para que ninguno lo matase.

\bibverse{16} Así que Caín se fue de la presencia del Señor y se fue a
vivir al país llamado Nod, al oriente de Edén.\footnote{4.16 ``Nod''
  significa ``deambulando''.}

\bibverse{17} Caín se acostó con su esposa y ella quedó embarazada. Y
tuvo un hijo llamado Enoc. En ese tiempo Caín estaba construyendo una
ciudad, y le puso el mismo nombre que su hijo Enoc. \bibverse{18} Enoc
tuvo un hijo llamado Irad. E Irad fue el padre de Mejuyael, y luego
Mejuyael fue el padre de Metusael, y Metusael fue el padre de Lamec.
\bibverse{19} Lamec se casó con dos mujeres. La primera se llamaba Ada,
y la segunda se llamaba Selá. \bibverse{20} Ada tuvo un hijo llamado
Jabal. Él fue el padre\footnote{4.20 ``Padre'' también puede significar
  ``ancestro''.}de los que viven en tiendas y tienen rebaños.
\bibverse{21} Él tuvo un hermano llamado Jubal. Y Jubal era el padre de
todos los que tocan instrumentos de cuerda y de viento. \bibverse{22}
Selá también tuvo un hijo que se llamaba Tubal-Caín y era un herrero que
forjaba toda clase de herramientas de hierro y bronce. La hermana de
Tubal-Caín se llamaba Naamá.

\bibverse{23} En cierta ocasión, Lamec le dijo a sus esposas: ``Ada y
Selá, escúchenme. Ustedes, esposas de Lamec, presten atención a lo que
les voy a decir. Yo maté a un hombre y él me hirió. Maté a un hombre
joven porque atentó contra mi. \bibverse{24} Si la sentencia por matar a
Caín era de ser castigado siete veces más, entonces si alguien me mata a
mí, el castigo debería ser setenta y siete veces más''.

\bibverse{25} Adán volvió a acostarse con su esposa otra vez, y tuvieron
un hijo llamado Set,\footnote{4.25 ``Set'', significa ``sustituto'', o
  ``regalado''.}con la explicación: ``Dios me ha dado otro hijo para
tomar el lugar de Abel, el que mató Caín''. \bibverse{26} Después Set
tuvo un hijo llamado Enós,+ 4.26 ``Enós'', significa ``humanidad'' o
``gente''.porque en ese tiempo las personas habían comenzado a adorar al
Señor por su nombre.

\hypertarget{section-4}{%
\section{5}\label{section-4}}

\bibverse{1} Este es el registro de los descendientes de Adán. Cuando
Dios creó a los seres humanos, los hizo semejantes a él. \bibverse{2}
Los creó varón y hembra, y los bendijo. El día que los creó, los llamó
``humano''.\footnote{5.2 ``Humano'': literalmente, ``Adán'', u
  ``hombre''.}

\bibverse{3} Cuando Adán cumplió la edad de 130 años, tuvo un hijo
semejante a él, y a su imagen, y le puso como nombre Set. \bibverse{4}
Después del nacimiento de Set, Adán vivió 800 años más, y tuvo más hijos
e hijas. \bibverse{5} Y Adán vivió en total 930 años, y entonces murió.

\bibverse{6} Cuando Set cumplió la edad de 105 años, tuvo a Enoc.
\bibverse{7} Después del nacimiento de Enoc, Set vivió 807 años más, y
tuvo más hijos e hijas. \bibverse{8} Enós vivió en total 912 años, y
entonces murió.

\bibverse{9} Cuando Enós cumplió la edad de 90 años, tuvo a Cainán.
\bibverse{10} Después del Nacimiento de Cainán, Enoc vivió 815 años más
y tuvo más hijos e hijas. \bibverse{11} Enoc vivió en total 905 ños y
entonces murió.

\bibverse{12} Cuando Cainán cumplió la edad de 70 años, tuvo a Malalel.
\bibverse{13} Después del Nacimiento de Malalel, Cainán vivió 840 años
más, y tuvo más hijos e hijas. \bibverse{14} Y Cainán vivió en total 910
años, y entonces murió. \bibverse{15} Cuando Malalel cumplió la edad de
65 años, tuvo a Jared. \bibverse{16} Y después del Nacimiento de Jared,
Malalel vivió 830 años más y tuvo más hijos e hijas. \bibverse{17} Y
Malalel vivió en total 895 años, y entonces murió.

\bibverse{18} Cuando Jared cumplió la edad de 162 años, tuvo a Enoc.
\bibverse{19} Después del Nacimiento de Enoc, Jared vivió 800 años más y
tuvo más hijos e hijas. \bibverse{20} Y Jared vivió en total 962 años, y
entonces murió.

\bibverse{21} Cuando Enoc cumplió 65 años, tuvo a Matusalén.
\bibverse{22} Y Enoc tuvo una relación muy estrecha con Dios. Después
del nacimiento de Matusalén, Enoc vivió 300 años más y tuvo más hijos e
hijas. \bibverse{23} Y Enoc vivió en total 365 años. \bibverse{24} Pero
Enoc tenía una relación tan estrecha con Dios que no murió,\footnote{5.24
  ``No murió'': añadido con fines explicativos. Ver Hebreos 11:5.} sino
que desapareció porque Dios se lo llevó.

\bibverse{25} Cuando Matusalén cumplió la edad de 187 años, tuvo a
Lamec. \bibverse{26} Después del Nacimiento de Lamec, Matusalén vivió
782 años más y tuvo más hijos e hijas. \bibverse{27} Y Maturalén vivió
en total 969 años, y entonces murió.

\bibverse{28} Cuando Lamec cumplió 182 años, tuvo un hijo. \bibverse{29}
Y le puso por nombre Noé,\footnote{5.29 ``Noé'': nombre asociado con el
  significado de ``alivio'', ``descanso'', y ``consuelo''.} con la
explicación ``Él nos dará alivio del arduo trabajo que debemos hacer
para cultivar la tierra que el Señor maldijo''. \bibverse{30} Después
del Nacimiento de Noé, Lamec vivió 595 años más y tuvo más hijos e
hijas. \bibverse{31} Y Lamec vivió en total 777 años, y entonces murió.

\bibverse{32} Noé vivió 500 años antes de tener a Sem, Cam y
Jafet.\footnote{5.32 A partir de las pruebas internas en el Génesis,
  parece que Jafet era el mayor y Cam era el más joven. Normalmente los
  hermanos se enumeran por orden de nacimiento en el Antiguo Testamento,
  aunque por ejemplo Moisés, a pesar de ser más joven que Aarón, aparece
  en primer lugar. Aquí parece que Sem es considerado más importante, y
  por lo tanto aparece en primer lugar.}

\hypertarget{section-5}{%
\section{6}\label{section-5}}

\bibverse{1} Y los seres humanos comenzaron a multiplicarse y a
esparcirse por toda la tierra. Y tenían hijas, \bibverse{2} y los hijos
de Dios\footnote{6.2 ``Hijos de Dios'': algunos han visto esto como una
  referencia a los ángeles, pero Jesús dijo claramente que los ángeles
  no se casan (Mateo 22:30), y en el siguiente versículo el castigo
  recae sobre todos como seres humanos. Los hijos de Dios pueden ser
  identificados como aquellos en el linaje de Set, distinguidos de estas
  mujeres que son descendientes de Caín. Acaban de presentarse las
  genealogías de ambos grupos (capítulos 4 y 5).} se dieron cuenta de
que estas mujeres eran hermosas, y tomaban para sí las que querían.

\bibverse{3} Entonces el Señor dijo: ``Mi espíritu de vida no
permanecerá con este pueblo para siempre, porque son mortales. Ahora el
tiempo de vida será de 120 años''.\footnote{6.3 Que esto se refiera a un
  nuevo máximo de vida parece poco probable, ya que muchos después de
  este tiempo vivieron mucho más de 120 años. El hebreo dice
  literalmente, ``Sus días serán 120 años''. Aquí los ``días'' pueden
  ser tomados simplemente como tiempo, o incluso tiempo restante, hasta
  que llegara el Diluvio.}

\bibverse{4} Y en esos días había gigantes\footnote{6.4 ````Gigantes'':
  literalmente, ``Nefilim''. Esta palabra se traduce como ``gigantes''
  en la Septuaginta. Sin embargo, algunos toman la palabra como base de
  la palabra hebrea ``caído''. A estos ``gigantes'' también se les hace
  referencia más adelante (ver Números 13:33). En la traducción griega
  de Symmachus, el término ``Nefiilim'' queda traducido como ``los
  violentos''.} en la tierra, y aún después los hubo también. Estos
nacieron después de que los hijos de Dios se acostaran con las hijas de
este pueblo. Sus hijos se volvieron grandes guerreros y hombres de
renombre en la antigüedad.

\bibverse{5} Y el Señor se dio cuenta de cuán malvados se habían vuelto
los habitantes de la tierra, pues cada uno de los pensamientos en sus
mentes estaban llenos de maldad. \bibverse{6} El Señor se lamentó de
haber creado a los seres humanos para habitar la tierra, y le
entristeció este pensamiento. \bibverse{7} Así que el Señor dijo: ``Voy
a eliminar de la tierra a estas personas que he creado; y no solo a
ellos, sino también a los animales, a los reptiles y a las aves, porque
me lamento de haberlos creado''.

\bibverse{8} Pero el Señor se agradó de Noé.

\bibverse{9} Esta es la historia de Noé y su familia. Noé era un hombre
íntegro, que vivía una vida con principios morales entre las personas de
su época. Él tenía una relación estrecha con Dios. \bibverse{10} Y Noé
tenía tres hijos: Sem, Cam, y Jafet.

\bibverse{11} Dios vio cuán inmoral se había vuelto el mundo entero,
lleno de violencia y de personas que actuaban sin ley. \bibverse{12}
Dios se dio cuenta de que la perversión del mundo se debía a que todos
vivían vidas inmorales. \bibverse{13} Entonces Dios le dijo a Noé: ``He
decidido poner fin a todos los habitantes de la tierra porque todos son
violentos y viven sin ley. Yo mismo los voy a destruir a todos, y a la
tierra misma junto con ellos.

\bibverse{14} Construye un arca\footnote{6.14 La palabra usada aquí para
  ``arca'' es diferente a la usada más tarde para describir el Arca del
  Pacto del Señor.}de madera de ciprés. Haz habitaciones dentro del
arca, y cúbrela con alquitrán, por dentro y por fuera. \bibverse{15} Y
así es como deberás construirla: El arca debe medir 300 codos de largo,
50 codos de ancho, y 30 codos de alto. \bibverse{16} Hazle un techo al
arca, dejando una ventana del tamaño de un codo entre el techo y la
parte superior de los lados.+ 6.16 El significado hebreo de esta última
frase no está claro.Coloca una puerta lateral en el arca, y haz el arca
de tres cubiertas.

\bibverse{17} Yo mismo voy a enviar un diluvio a la tierra que destruirá
todo lo que respire. Todo ser vivo sobre la tierra morirá. \bibverse{18}
Pero yo guardaré mi pacto contigo. Tu entrarás al arca, tomarás contigo
a tu esposa, a tus hijos y a sus esposas. \bibverse{19} Tomarás un par,
macho y hembra de cada especie de animal, y te asegurarás de
preservarlos con vida. \bibverse{20} Harás lo mismo con cada especie de
ave, ganado, y con los reptiles: un par de cada uno vendrá a ti para que
puedas mantenerlos con vida. \bibverse{21} Lleva contigo toda clase de
alimentos y almacénala para que tú y los animales tengan suficiente
alimento para comer''.

\bibverse{22} Y Noé hizo exactamente lo que Dios le ordenó que hiciera.

\hypertarget{section-6}{%
\section{7}\label{section-6}}

\bibverse{1} El Señor le dijo a Noé: ``Entra al arca con toda tu
familia. Porque he visto que eres un hombre íntegro, que vive una vida
moral en medio de la gente de esta generación. \bibverse{2} Toma contigo
siete pares, macho y hembra, de cada especie de animal limpio; y un par,
macho y hembra de cada especie de animal impuro. \bibverse{3} Además,
toma siete pares, macho y hembra, de todas las aves, para que todas las
especies de toda la tierra puedan sobrevivir. \bibverse{4} En siete días
hare llover por cuarenta días y cuarenta noches. Voy a erradicar de la
superficie de la tierra a todos los seres que he creado''.

\bibverse{5} Y Noé hizo exactamente lo que el Señor le ordenó que
hiciera.

\bibverse{6} Noé tenía 600 años cuando las aguas inundaron la tierra.
\bibverse{7} Noé entró al arca, junto con su esposa y sus hijos, y las
esposas de sus hijos, por causa del diluvio. \bibverse{8} Animales
impuros e impuros, aves y reptiles, \bibverse{9} entraron en el arca que
construyó Noé. \bibverse{10} Después de siete días, las aguas cayeron
sobre la tierra.

\bibverse{11} Noé había cumplido 600 años, cuando en el día número
diecisiete del segundo mes, todas las aguas que estaban debajo de la
tierra estallaron y atravesaron el suelo, y una fuerte lluvia cayó del
cielo. \bibverse{12} La lluvia siguió cayendo sobre la tierra durante
cuarenta días y cuarenta noches.

\bibverse{13} Ese fue el día\footnote{7.13 ``Ese fue el día'': se
  refiere al día mencionado en el versículo 11.}en que el Noé, su
esposa, sus hijos Sem, Cam y Jafet,junto a sus esposas, entraron en el
arca. \bibverse{14} Con ellos entró toda especie de animales
salvajes,ganado, reptiles y aves, así como todo animal alado.
\bibverse{15} Todos entraron con Noé al arca;todos los seres vivos, y en
pares. \bibverse{16} De cada criatura entró el macho con su hembra, tal
como Dios le dijo a Noé. Entonces el Señor cerró la puerta.

\bibverse{17} Y la lluvia cayó sobre la tierra por cuarenta días,
haciendo flotar el arca por encima del suelo. \bibverse{18} Las aguas
subieron cada vez más y se hacían profundas, pero el arca flotaba en la
superficie. \bibverse{19} Finalmente, el agua aumentó tanto de nivel que
hasta las montañas más altas quedaron cubiertas, y solo se podía ver el
cielo. \bibverse{20} El agua subió tanto, que sobrepasó la altura de las
montañas hasta quince codos más. \bibverse{21} Y todo lo que habitaba
sobre la tierra pereció: las aves, el ganado,los animales salvajes,
todos los reptiles, y todas las personas también. \bibverse{22} Murió
todo ser vivo que estaba sobre la tierra y que podía respirar.
\bibverse{23} El Señor exterminó a todo ser viviente: desde los seres
humanos, hasta el ganado, los reptiles y las aves. Todos murieron y solo
sobrevivieron los que estaban con Noé en el arca. \bibverse{24} Y la
tierra permaneció inundada por 150 días.

\hypertarget{section-7}{%
\section{8}\label{section-7}}

\bibverse{1} Pero Dios no se había olvidado de Noé y de todos los
animales salvajes y el ganado que estaba con él en el arca. Dios envió
un viento fuerte sobre la tierra, y las aguas comenzaron a bajar.
\bibverse{2} Las aguas subterráneas se cerraron, y la lluvia se detuvo.
\bibverse{3} Poco a poco, las aguas comenzaron a retirarse de la tierra.
Bajaron tanto que 150 días después del diluvio \bibverse{4} el arca se
posó sobre el monte Ararat. Esto sucedió en el día diecisiete del
séptimo mes. \bibverse{5} Las aguas siguieron bajando hasta que el
primer día del décimo mes, ya se podía ver la cumbre de las montañas.

\bibverse{6} Cuarenta días después, Noé abrió la ventana que había hecho
en el arca, \bibverse{7} y envió a un cuervo fuera del arca. El cuervo
iba y venía hasta que el agua sobre la tierra se hubo secado.
\bibverse{8} Entonces Noé envió una paloma para comprobar si las aguas
habían bajado lo suficiente como para que hubiera tierra seca.
\bibverse{9} Pero la paloma no pudo encontrar ningún lugar donde
posarse. Así que regresó a Noé porque el agua aún cubría toda la tierra.
Noé sacó su mano y tomó a la paloma, y la trajo consigo de nuevo dentro
del arca. \bibverse{10} Entonces Noé esperó siete días más y volvió a
enviar a la paloma fuera del arca. \bibverse{11} Cuando la paloma
regresó en la noche, trajo en su pico una hoja fresca de olivo, de modo
que Noé supo así que las aguas se habían ido en gran parte de la tierra.
\bibverse{12} Una vez más, Noé esperó otros siete días más, y entonces
volvió a enviar a la paloma, pero esta vez la paloma no regresó.

\bibverse{13} Noé había cumplido ahora 601 años, y era el primer día del
primer mes, cuando las aguas se habían secado por completo. Noé retiró
la cubierta del arca y pudo ver que el suelo se estaba secando.
\bibverse{14} En el vigésimo séptimo día del segundo mes, la tierra
estaba seca.

\bibverse{15} Entonces Dios le dijo a Noé: \bibverse{16} ``Salgan del
arca tú, tu esposa, tus hijos, y sus esposas. \bibverse{17} Dejen ir a
todos los animales; a las aves, a los animales salvajes, a los reptiles,
para que se multipliquen y llenen en la tierra''. \bibverse{18} Así que
Noé y su esposa, así como sus hijos y sus esposas salieron del arca.
\bibverse{19} También todos los animales, los reptiles y las aves, todo
ser vivo que estaba en el arca salió, todosagrupados por especie.

\bibverse{20} Entonces Noé construyó un altar y sacrificó a algunos de
los animales limpios, así como a algunas aves, a manera de ofrenda.
\bibverse{21} El Señor aceptó\footnote{8.21 ``Aceptó'': literalmente,
  ``olió un aroma agradable''. Esta es una ``extensión figurativa'' de
  este proceso sensorial que indica que del mismo modo que nos gusta
  algo y por extensión lo aceptamos, así mismo lo hace Dios.}tal
sacrificio, y dijo para sí mismo: ``No volveré a maldecir a la tierra
por culpa de los seres humanos, aunque cada uno de sus pensamientos sea
perverso desde su niñez. Y no volveré a destruir a los seres vivos como
lo acabo de hacer. \bibverse{22} En tanto exista la tierra, no faltará
la temporada de siembra y de cosecha, el frío y el calor, el verano y el
invierno, así como el día y la noche''.

\hypertarget{section-8}{%
\section{9}\label{section-8}}

\bibverse{1} Y Dios bendijo a Noé y a sus hijos, y les dijo:
``¡Reprodúzcanse, multiplíquense y llenen toda la tierra! \bibverse{2}
Todos los animales te temerán, incluso las aves, las criaturas que se
arrastran por el suelo, y los peces del mar. Ahora estás a cargo de
ellos. \bibverse{3} Todo ser vivo que se mueve será alimento para ti,
así como las plantas verdes.\footnote{9.3 De acuerdo con Gén. 1:30, las
  plantas verdes estaban originalmente destinadas a los animales. Ahora,
  tanto las plantas como los propios animales están permitidos como
  alimento humano. Después del diluvio habría habido poca comida
  disponible inmediatamente.} \bibverse{4} Pero no comerás carne que aún
tenga sangre de vida en ella. \bibverse{5} Si tu sangre es derramada por
causa de un animal, yo pediré cuentas por ello; y si tu sangre es
derramada por otra personas, yo se lo reclamaré. \bibverse{6} Si alguno
derrama sangre de otro ser humano, otro ser humano derramará su sangre
también. Porque Dios hizo a los seres humanos según su imagen.
\bibverse{7} ¡Así que reprodúzcanse, multiplíquense y llenen la tierra
de muchos descendientes!''

\bibverse{8} Entonces Dios le dijo a Noé y a sus hijos que estaban con
él: \bibverse{9} ``Escuchen, yo hoy hago mi pacto con ustedes y con sus
descendientes, \bibverse{10} y también con todos los animales ---las
aves, el ganado y todos los animales salvajes de la tierra, así como
todo animal que estuvo en el arca. \bibverse{11} En este pacto yo les
prometo que no volveré a destruir a los seres vivos por medio de un
diluvio, y que no habrá nuevamente un diluvio destructor como este''.

\bibverse{12} Entonces Dios dijo: ``Les daré una señal para confirmar el
acuerdo que hago hoy entre mi y ustedes, y todos los seres vivos. Un
acuerdo que durará por todas las generaciones. \bibverse{13} He puesto
mi arcoíris en las nubes, y esta será la señal de mi acuerdo contigo y
con toda la vida que hay sobre la tierra. \bibverse{14} Cada vez que
haya nubes sobre la tierra y aparezca el arcoiris, \bibverse{15} me
recordará de mi pactoentre mi y ustedes, así como cada criatura
viviente, de que las aguas no volverán a destruir todo ser viviente
sobre la tierra. \bibverse{16} Y veré el arcoíris en las nubes, y me
acordaré de este acuerdo eterno entre Dios y cada ser vivo que habita
sobre la tierra''.

\bibverse{17} Entonces Dios le dijo a Noé: ``Esta es la señal del
acuerdo que hago hoy con cada criatura sobre la tierra''.

\bibverse{18} Los hijos de Noé que salieron del arca eran Sem, Cam y
Jafet. (Cam fue el padre de los cananeos.) \bibverse{19} Y todos los
seres humanos que están esparcidos por el mundo son descendientes de
estos tres hijos de Noé.

\bibverse{20} Noé comenzó a cultivar la tierra como un granjero, y
plantó un viñedo. \bibverse{21} Entonces bebió del vino que produjo su
viñedo, se emborrachó y se quedó dormido desnudo en su tienda.
\bibverse{22} Cam, el padre de Canaán vio las partes íntimas de su padre
y fue y se lo dijo a sus hermanos que estaban afuera. \bibverse{23}
Entonces Sem y Jafet tomaron un manto, y poniéndolo sobre sus hombros,
caminaron de espaldas y cubrieron las partes privadas de su padre. Y se
aseguraron de mirar hacia otro lado, a fin de no ver las partes privadas
de su padre.

\bibverse{24} Cuando Noé se levantó de su sueño por la embriaguez, se
dio cuenta de los que su hijo menor había hecho, \bibverse{25} y dijo:
``¡Maldito seas, Canaán\footnote{9.25 Por qué Canaán es el maldito y no
  Cam es un tema de debate desde hace mucho tiempo. Una sugerencia es
  que los cananeos posteriores fueron los enemigos particulares de
  Israel y fueron subyugados por ellos, por lo que Canaán era
  proféticamente más significativo como símbolo.}! ¡Serás el esclavo de
menor clase, y servirás a tus hermanos!''

\bibverse{26} Entonces Noé continuó: ``Bendito sea el Señor, Dios de
Sem, y que Canaán sea su esclavo. \bibverse{27} Que Dios le de a Jafet
mucho espacio para sus descendientes, y que vivan en paz con el pueblo
de Sem, y que Canaán sea su esclavo también''.

\bibverse{28} Y después del diluvio, Noé vivió 350 años más.
\bibverse{29} Noé vivió en total 950 años, y entonces murió.

\hypertarget{section-9}{%
\section{10}\label{section-9}}

\bibverse{1} Las siguientes son las genealogías\footnote{10.1 Estas
  genealogías se repiten en 1 Crónicas 1:5-27.} de los hijos de Noé:
Sem, Cam, y Jafet. Ellos tuvieron hijos después del diluvio.

\bibverse{2} Los hijos\footnote{10.2 Nótese que ``hijos'' en este
  capítulo también puede significar ``descendientes''.} de Jafet: Gomer,
Magog, Madai, Jabán, Tubal, Mésec y Tirás.

\bibverse{3} Los hijos de Gomer: Asquenaz, Rifat, and Togarmá.

\bibverse{4} Los hijos de Jabán: Elisá, Tarsis,Quitín,
yRodanín.\footnote{10.4 ``Rodanín'': la Septuaginta presenta la
  ortografía de Rodanín como lo hace el pasaje paralelo en 1 Crónicas
  1:7. Nótese que los dos últimos nombres por lo menos son probablemente
  los de un grupo de personas en lugar de un nombre personal.}
\bibverse{5} Los descendientes de estos ancestros se esparcieron por las
áreas costeras, cada grupo con su propio idioma, con sus failias que se
convirtieron en diferentes naciones.

\bibverse{6} Los hijos de Cam: Cus, Misrayin, Fut, y Canaán.

\bibverse{7} Los hijos de Cus: Seba, Javilá, Sabtá, Ragama y Sabteca.

Los hijos de Ragama: Sabá y Dedán.

\bibverse{8} Cus también fue el padre de Nimrod, quien se destacó como
el primer tirano en la tierra. \bibverse{9} Fue un guerrero que
desafió\footnote{10.9 ``Desafió'': En la Septuaginta se lee como
  ``contra'' o ``en contra''.} al Señor, y es la razón por la que existe
el dicho: ``Como Nimrod, un poderoso guerrero que desafió al Señor''.
\bibverse{10} Su reino comenzó en las ciudades de Babel,+ 10.10
``Babel'' o ``Babilonia''. Nimrod es la primera persona en la Escritura
descrita como poseedor de un reino, normalmente asociado con un gobierno
impuesto por la fuerza. Erec, Acad, y Calné, todas ellas ubicadas en la
tierra de Sinar.+ 10.10 ``Sinar'': o ``Babilonia''. \bibverse{11} De
allí se mudó a Asiria+ 10.11 ``Asiria'': en Miqueas 5:6 Asiria es
llamada ``la tierra de Nimrod''. y construyó las ciudades de Nínive,
Rejobot Ir, Cala, \bibverse{12} y Resén, la cual queda entre Nínive y la
gran ciudad de Cala.

\bibverse{13} Misrayin fue el oadre de los ludeos, los anameos, los
leabitas, los naftuitas \bibverse{14} los patruseos, los caslujitas y
los caftoritas (ancestros de los filisteos).\footnote{10.14 Ver Jeremías
  47:4 y Amós 9:7.}

\bibverse{15} Canaán fue el padre de Sidòn, su primogénito, y de los
hititas,\footnote{10.15 ``Los hititas'': literalmente ``Heta''.}
\bibverse{16} de los jebuseos, de los amorreos, de los gergeseos,
\bibverse{17} de los heveos, los araceos, los sineos, \bibverse{18} los
arvadeos, los zemareos y los jamatitas.

Luego las tribus de Canaán se esparcieron \bibverse{19} y el territorio
de los caananitas se extendió desde Sidón hasta Guerar y hasta Gaza,
luego hacia Sodoma, Gomorra Admá, y Zeboyín, hasta Lasa. \bibverse{20}
Estos fueron los hijos de Cam según sus tribus, idiomas, territorios y
nación.

\bibverse{21} Sem, cuyo herman mayor\footnote{10.21 Ver la nota en Gén.
  5:32.}era Jafet, también tuvo hijos. Sem fue el padre de todos los
hijos de Eber.

\bibverse{22} Los hijos de Sem: Elam, Asur, Arfaxad, Lud, y Aram.

\bibverse{23} Los hijos de Aram: Uz, Hul, Guéter, y Mas.\footnote{10.23
  ``Mas'': En la Septuaginta y 1 Crónicas 1:17 se lee ``Mesec''.}

\bibverse{24} Arfaxad fue el padre de Selaj. Y Selaj fue el padre de
Éber.

\bibverse{25} Éber tuvo dos hijos. Uno se llamó Péleg,\footnote{10.25 La
  palabra significa ``dividido''.}porque en su tiempo se dividió la
tierra; y el nombre de su hermano era Joctán.

\bibverse{26} Joctán fue el padre de Almodad, Sélef, Jazar Mávet, Yerah,
\bibverse{27} Hadorán, Uzal, Diclá, \bibverse{28} Obal, Abimael, Sabá,
\bibverse{29} Ofir, Javilá y Jobab. Todos estos fueron hijos de Joctán.
\bibverse{30} Ellos vivieron en la región entre Mesáhasta Sefar, en la
región montañosa oriental.

\bibverse{31} Estos fueron los hijos de Sem, sus tribus, idiomas,
territorios y naciones.

\bibverse{32} Todas estas fueron las tribus descendientes de los hijos
de Noé, según su descendencia y naciones. A partir de estos ancestros se
formaron las distintas naciones de la tierra que se expandieron en todo
el mundo después del diluvio.

\hypertarget{section-10}{%
\section{11}\label{section-10}}

\bibverse{1} En ese tiempo se hablaba en todo el mundo un solo idioma y
todos usaban palabras con el mismo significado. \bibverse{2} Al
trasladarse hacia el oriente, descubrieron una llanura en la región del
Sinar y se asentaron allí.

\bibverse{3} Y se dijeron unos a otros: ``Vengan, juntemos ladrillos y
cocinémoslos con fuego''. (Ellos usaron ladrillo en lugar de piedra, y
alquitrán en lugar de cemento).\footnote{11.3 Esto se debía a que en la
  llanura babilónica no había piedras para usarse en las construcciones.}
\bibverse{4} Y entonces dijeron: ``Construyamos ahora una ciudad para
nosotros mismos con una torre cuya cúspide llegue hasta el cielo. Así
lograremos tener una gran reputación y no andaremos dispersos por todo
el mundo''.

\bibverse{5} Pero el Señor descendió para mirar la ciudad y la torre que
estas personas estaban construyendo. \bibverse{6} Y el Señor dijo:
``Miren cómo estas personas están unidas y hablando el mismo idioma. ¡Si
pueden lograr todo esto tan solo comenzando, nada les será imposible si
se fijan un propósito! \bibverse{7} Necesitamos bajar allí y confundir
su idioma para que no puedan entender lo que se dicen unos a otros''.

\bibverse{8} Y entonces el Señor los expulsó de allí e hizo que se
dispersaran por todo el mundo, y dejaron de construir la ciudad.
\bibverse{9} Por eso la ciudad fue llamada Babel,\footnote{11.9 El
  sonido Babel es como la palabra hebrea que se usa para el término
  ``confundir''.}porque el Señor confundió el idioma que se hablaba en
el mundo.

\bibverse{10} La siguiente es la genealogía de Sem. Cuando Sem tuvo 100
años, nació su hijo Arfaxad. Esto sucedió dos años después del diluvio.
\bibverse{11} Después del nacimiento de Arfaxad, Sem vivió 500 años más
y tuvo más hijos e hijas.

\bibverse{12} Cuando Arfaxad tuvo 35 años, nació su hijo Selaj.
\bibverse{13} Después del nacimiento de Selaj, Arfaxad vivió 403 años
más y tuvo más hijos e hijas.

\bibverse{14} Cuando Selaj tuvo 30 años, nació su hijo Éber.
\bibverse{15} Después del nacimientode Éber, Selaj vivió 403 años más y
tuvo más hijos e hijas.

\bibverse{16} Cuando Éber tuvo 34 años, nació su hijo Péleg.
\bibverse{17} Después del nacimiento de Péleg, Éber vivió 430 años más y
tuvo más hijos e hijas.

\bibverse{18} Cuando Péleg tuvo 30 años, nació su hijo Reú.
\bibverse{19} Después del nacimiento de Reú, Péleg vivió 209 años más y
tuvo más hijos e hijas.

\bibverse{20} Cuando Reú tuvo 32 años, nació su hijo Sarug.
\bibverse{21} Después del nacimiento de Sarug, Reú vivió 207 años más, y
tuvo más hijos e hijas.

\bibverse{22} Cuando Sarug tuvo 30 años, nació su hijo Najor.
\bibverse{23} Después del nacimiento de Najor, Sarug vivió 200 años más
y tuvo más hijos e hijas.

\bibverse{24} Cuando Najor tuvo 29 años, nació su hijo Téraj.
\bibverse{25} Después del nacimiento de Téraj, Najor vivió 119 años más,
y tuvo más hijos e hijas.

\bibverse{26} Cuando Téraj tuvi 70 años, nacieron sus hijos Abrán, Najor
y Aram.\footnote{11.26 Una vez más (ver nota del 5:32) estos hijos no
  están listados en orden de nacimiento. Abram aparece en primer lugar
  debido a su importancia}

\bibverse{27} Esta es la genealogía de Téraj. Téraj fue el padre de
Abrán, Najor y Aram. Aram fue el padre de Lot. \bibverse{28} Sin
embargo, Aram murió cuando si padre Téraj aún vivía en Ur de los
caldeos, la tierra donde nació. \bibverse{29} Tanto Abrán como Najor se
casaron. La esposa de Abrán se llamaba Sarai, y la esposa de Najor se
llamaba Milcá. (Milcá era hija de Aram, quien era padre tanto de Milca
como de Jiscá). \bibverse{30} Sarai no podia quedar embarazada, por lo
tanto no tenía hijos.

\bibverse{31} Térajtomó a su hijo Abrán, a su nieto Lot, (quien era el
hijo de Arám), a su nuera Sarai, (que era la esposa de su hijo Abrán), y
se fue de Ur de los caldeos para mudarse a Canaán. Llegaron hasta Harán
y se quedaron a vivir allí. \bibverse{32} Téraj vivió 205 años y murió
en Harán.

\hypertarget{section-11}{%
\section{12}\label{section-11}}

\bibverse{1} Y el Señor le dijo a Abrán: ``Deja tu país, a tu familia, y
el hogarde tu familia,\footnote{12.1 ``el hogar de tu familia'':
  literalmente, ``la casa de tu padre''.}y vete al país que yo te
mostraré. \bibverse{2} Te convertiré en el predecesor de una gran nación
y te bendeciré. Te daré una gran reputación y haré que seas una
bendición para otros. \bibverse{3} Bendeciré a los que te bendigan, y
maldeciré a los que te maldigan. Todos en la tierra serán benditos a
través de ti''.

\bibverse{4} Así que Abrán siguió las instrucciones del Señor, y Lot
también se fue con él. Abrán tenía 75 años cuando se fue de Harán.
\bibverse{5} Junto a él iba su esposa Sarai, su sobrino Lot, y llevaron
consigo todas las posesiones que habían acumulado, así como a las
personas que se les unieron\footnote{12.5 ``Las personas que se les
  unieron'': esto incluiría a los sirvientes, pero el término utilizado
  no es específico y se aplica a cualquiera que se haya unido al grupo
  de Abram por cualquier razón.}en Harán. Salieron y se fueron hacia la
tierra de Canaán.

Cuando llegaron allí, \bibverse{6} Abrán viajó por todo el país hasta
que llegó a un lugar llamado Siquén, haciendo una pausa en el roble que
estaba en Moré. En ese tiempo, el país estaba ocupado por los
caananitas.

\bibverse{7} Entonces el Señor se le apareció a Abrán y le dijo: ``Esta
tierra te la daré a ti y a tus descendientes''.Así que Abrán construyó
un altar al Señor allí porque allí se le apareció el Señor. \bibverse{8}
Y entonces se mudó hacia la región montañosa, al oriente de Betel y armó
su campamento allí. Betel estaba en el occidentey Ay quedaba en el
oriente. Abrán construyó allí un altar al Señor y lo adoró. \bibverse{9}
Después se fue de allí, camino al Neguev.\footnote{12.9 ``El Neguev'':
  la zona desértica al sur.}

\bibverse{10} Pero sucedió que una gran hambruna había azotado esta
tierra. De modo que Abrán siguió hacia Egipto, con planes de vivir allí,
pues la hambruna era muy severa. \bibverse{11} Al acercarse a Egipto,
cuando estaba a punto de cruzar la frontera, Abrán le dijo a su esposa
Sarai: ``Yo sé que eres una mujer muy hermosa. \bibverse{12} Y cuando
los egipcios te vean, dirán, `ella es su esposa,' y me matarán, pero no
a ti. \bibverse{13} Así que mejor diles que eres mi hermana, para que me
traten bien por ti, y así mi vida estará a salvo gracias a ti''.

\bibverse{14} Cuando Abrán llegó a Egipto, la gente allí notó lo hermosa
que era Sarai. \bibverse{15} Los oficiales del faraón se dieron cuenta
también y le hablaron al faraón bien de Sarai. Así que Sarai fue llevada
a su palacio para convertirse en una de sus esposas.\footnote{12.15
  ``Convertirse en una de sus esposas'': añadido para mayor claridad.}
\bibverse{16} Y el faraón trataba bien a Abrán por causa de ella, y le
dio ovejas y ganado, así como asnos y asnas, y además sirvientes tanto
hombres como mujeres, y camellos.

\bibverse{17} Pero el Señor hizo que el faraón y los que habitaban en su
palacio sufrieran una terrible enfermedad por causa de Sarai, la esposa
de Abrán. \bibverse{18} Así que el faraón ordernó que trajeran a Abrán
delante de él, y le dijo: ``¿Qué me has hecho? ¿Por qué no me dijiste
que ella era tu esposa? \bibverse{19} ¿Por qué dijiste `ella es mi
hermana,' y me dejaste traerla para convertirse en una de mis esposas?
¡Aquí tienes a tu esposa! ¡Llévatela y vete!''

\bibverse{20} Y el faraón dio orden a sus guardas para que lo expulsaran
a él y a su esposa del país, junto a todos los que iban con él y todas
sus posesiones.

\hypertarget{section-12}{%
\section{13}\label{section-12}}

\bibverse{1} Así que Abrán se fue de Egipto y regresó al Neguev con
Sarai, Lot, y todos los que iban con él, así como sus posesiones.
\bibverse{2} Abrán era muy rico, y tenía muchas manadas de ganado y
mucha plata y oro. \bibverse{3} Se fue el Neguevy comenzó su viaje por
partes hasta Betel, de regreso al lugar donde había acampado antes,
entre Betel y Ay. \bibverse{4} Fue aquí donde había construido un altar
por primera vez. Entonces adoró al Señor allí, como lo había hecho
antes.

\bibverse{5} Lot, quien viajaba con Abrán, también tenía muchos rebaños,
manadas y tiendas, \bibverse{6} tantas que la tierra disponible no
alcanzaba para que ambos vivieran allí, pues tenían tanto ganado, que ya
no podrían habitar juntos en el mismo lugar. \bibverse{7} Los granjeros
de Abrán discutían con los de Lot; además los cananeos y los fereceos
también habitaban la tierra en ese momento.

\bibverse{8} Así que Abrán le dijo a Lot: ``Por favor, evitemos las
discordias entre nosotros y entre nuestros granjeros, pues somos
familia. \bibverse{9} ¿Ves toda esta tierra disponible delante de ti?
Debemos dividirnos. Así que si decides ir por la izquierda, yo iré por
la derecha; y si decides ir por la derecha, yo iré por la izquierda''.

\bibverse{10} Lot miró todo el valle del Jordán, en dirección a Zoar, y
vio que estaba bien abastecido de agua, y que lucía como el jardín de
Edén, como la tierra de Egipto. (Esto era antes de que el Señor
destruyera a Sodoma y Gomorra.) \bibverse{11} Así que Lot eligió todo el
valle del Jordán y se fue hacia el oriente, y así los dos se separaron.
\bibverse{12} Abrán se fue a vivir a la tierra de Canán mientras que Lot
se asentó entre las ciudades del valle, estableciendo su campamento en
Sodoma. \bibverse{13} (El pueblo de Sodoma era muy perverso, y cometían
pecados terribles que ofendían al Señor).

\bibverse{14} Después de separarse de Lot, el Señor le dijo a Abrán:
``Desde donde estás, mira a tu alrededor, hacia el norte, el sur, el
oriente y el occidente. \bibverse{15} Toda esta tierra que ves, te la
daré a ti y a tus descendientes para siempre. \bibverse{16} Y tendrás
tantos descendientes que serán como el polvo de la tierra. ¡Quien quiera
contara tus descendientes será quien pueda contar el polvo de la tierra!
\bibverse{17} Ve y camina por toda la tierra, en todas las direcciones,
porque yo te la he dado''.

\bibverse{18} Así que Abrán se fue a vivir a Hebrón, y estableció su
campamento entre los robles de Mamré, donde construyó un altar al Señor.

\hypertarget{section-13}{%
\section{14}\label{section-13}}

\bibverse{1} En aquél tiempo Amrafel era el rey de Sumeria,\footnote{14.1
  ``Sinar'': un antiguo nombre para Babilonia.}y se había aliado con
Arioc, rey de Elasar, Quedorlaómer, rey de Elam, y con Tidal, rey de
Goyim. \bibverse{2} Juntos atacaron a Bera, el rey de Sodoma, a Birsá,
rey de Gomorra, a Sinab, rey de Admá, a Semeber, rey de Zeboyín, y al
rey de Bela (que también se le conocía como Zoar).

\bibverse{3} Todos estos en el segundo grupo\footnote{14.3 ``En el
  segundo grupo'', añadido para mayor claridad}se aliaron en el Valle de
Sidín (el valle del Mar Muerto). \bibverse{4} Habían estado bajo el
gobierno de Quedorlaómer durante doce años, pero en el decimotercer año
se rebelaron contra él. \bibverse{5} En el decimocuarto año,
Querdolaómer los invadió junto con sus reyes aliados. Vencieron a los
refaítas en Astarot Carnayin, a los zuzitas en Jam, a los emitas en Save
Quiriatayin, \bibverse{6} y a los horeos en su propia región montañosa
de Seir, llegando hasta El Parán, junto al desierto. \bibverse{7}
Entonces regresaron y atacaron a Enmispat (conocida también como Cades)
y conquistaron todo el país que le pertenecía a los amalecitas, así
como.a los amorreos que vivían en Jazezón Tamar.

\bibverse{8} Entonces los reyes de Sodoma, Gomorra, Admá, Zeboyín y Bela
(es decir, Zoar), marcharon y se prepararon para la batalla en el Valle
de Sidín. \bibverse{9} Pelearon contra Quedorlaómer, rey de Elam; Tidal,
rey de Goyim; Amrafel, rey de Sumeria; y Aric, rey de Elasar. Eran
cuatro reyes uno al lado del otro contra cinco.

\bibverse{10} En ese tiempo, había muchos pozos de alquitrán en el Valle
de Sidín, y cuando los reyes de Sodoma y Gomorra huían tras ser
vencidos, algunos de sus hombres\footnote{14.10 ``Algunos de sus
  hombres'': aunque el texto parece sugerir que los reyes cayeron en los
  pozos de alquitrán, el versículo 17 deja claro que al menos el rey de
  Sodoma no había muerto.}cayeron en ellos, mientras los demás corrieron
hacia las montañas. \bibverse{11} Los invasores saquearon todas las
posesiones y alimento de Sodoma y Gomorra, y se fueron. \bibverse{12}
También capturaron a Lot, el sobrino de Abrán, y tomaron sus posesiones,
porque también vivía en Sodoma. \bibverse{13} Pero uno de los capturados
escape y fue a decírselo a Abrán el hebreo+ 14.13 ``Abrán, el hebreo'':
es la primera vez que Abrán es llamado hebreo, y puede ser la forma en
que era identificado por la gente de la época.lo que había sucedido.
Abrán vivía entre los robles de Mamré el amorreo, quien era hermano de
Escol y Aner. Todos ellos eran aliados de Abrán.

\bibverse{14} Cuando Abrán se enteró de que su sobrino había sido
capturado, convocó a 318 hombres Guerreros que habían nacido en su hogar
para que los persiguieran hasta llegar a Dan. \bibverse{15} Allí dividió
a sus hombres en grupos y atacaron por la noche, venciendo al enemigo y
persiguiéndolos hasta llegar aHobá, al norte de Damasco. \bibverse{16}
Abrán recuperó todo lo que ellos habían tomado, incluyendo a Lot y sus
posesiones, y además trajo consigo a las mujeres y a otras personas que
habían sido capturadas.

\bibverse{17} Cuando Abrán regresó después de conquistar a Quedorlaómer
y a sus aliados, el rey de Sodoma salió a su encuentro en el Valle de
Save (o Valle del Rey). \bibverse{18} Melquisedec, rey de Salem, trajo
pan y vino. Él era sacerdote del Dios Altísimo. \bibverse{19} Y bendijo
a Abrán, diciéndole: ``Que tú, Abrán, seas bendito por el Altísimo,
Creador del cielo y de la tierra. \bibverse{20} Que el Altísimo sea
alabado, por entregar en tu mano a tus enemigos''.Entonces Abrán le dio
a Melquisedec una décima parte de todo.

\bibverse{21} El rey de Sodoma le dijo a Abrán: ``Devuélveme a mi gente,
y quédate con todo lo demás''.

\bibverse{22} Pero Abrán le respondió al rey de Sodoma: ``Levanto mi
mano, haciendo una promesa solemne al Señor, al Dios Altísimo, Creador
del cielo y de la tierra, \bibverse{23} que me niego a guardar
cualquiera de tus pertenencias, ni siquiera un hilo ni la correa de una
sandalia. De lo contrario podrías decir: `¡Yo fui quien hizo rico a
Abrán!' \bibverse{24} Por lo tanto no me quedaré con nada, excepto lo
que mis hombres han comido, y la parte de los que me han acompañado, es
decir, permite que Aner, Escol, y Mamré conserven su parte''.

\hypertarget{section-14}{%
\section{15}\label{section-14}}

\bibverse{1} Después de todo esto, Dios habló con Abrán en un vision, y
le dijo: ``¡No tengas miedo, Abrán! ¡Yo soy tu protector, y tu gran
recompensa!''

\bibverse{2} Pero Abrán respondió: ``Señor Dios, ¿De qué me beneficiará
cualquier cosa que me des? No tengo hijos, y el único heredero de todo
lo que tengo es Eliezer de Damasco''.\footnote{15.2 Era una práctica
  común en la época que las parejas sin hijos nombraran a su sirviente
  de más confianza como su heredero.} \bibverse{3} Y Abrán continuó,
expresando tu queja: ``¡Mira, no me has dado hijos, y me toca darle toda
mi herencia a un sirviente de mi casa!''

\bibverse{4} Pero entonces el Señor le dijo: Este hombre no será tu
heredero. Tu heredero será tu propio hijo''.

\bibverse{5} Entonces el Señor llevó a Abrán afuera y le dijo: ``Mira al
cielo, y dime si puedes contar las estrellas.¡Así será la cantidad de
descendientes que tendrás!''

\bibverse{6} Y Abrán creyó en lo que el Señor le dijo, y el Señor
concluyó que Abrán y él tenían una relación perfecta.

\bibverse{7} El Señor también le dijo: ``Yo soy el Señor, que te saqué
de Ur de los caldeos para darte esta tierra''.

\bibverse{8} ``Pero Señor, ¿cómo podré estar seguro de que es mía?''
preguntó Abrán.

\bibverse{9} Entonces el Señor me dijo: ``Tráeme una vaca, una cabray un
carnero, todos de tres años de edad, y además una paloma adulta y una
paloma joven''. \bibverse{10} Así que Abrán mató a los tres animales,
luego los cortó por la mitad, y puso cada mitad frente a la otra. Sin
embargo, no cortó a las aves por la mitad. \bibverse{11} Cuando los
buitres descendían para comerse los cadáveres, Abrán los espantaba.

\bibverse{12} Cuando se puso el sol, Abrán sintió mucho sueño, y a la
vez una oscuridad espesa y terrible se puso sobre él. \bibverse{13}
Entonces el Señor le explicó a Abrán: ``Puedes estar seguro de que tus
descendientes serán extranjeros en otras naciones, donde sufrirán
esclavitud y maltratados por 400 años. \bibverse{14} Sin embargo, yo
castigaré a la nación que los tendrá como esclavos, y después tus
descendientes saldrán, llevándose muchas posesiones de gran valor.
\bibverse{15} Pero en lo que a ti concierne, morirás en paz y serás
sepultado después de haber vivido una buena vida. \bibverse{16} Cuatro
generaciones más tarde, tus descendientes volverán para vivir aquí,
porque ahora mismo los pecados de los amonitas no han logrado su máximo
alcance''.

\bibverse{17} Después de que el sol se puso y se hizo oscuro, de repente
apareció un horno echando humo y una antorcha encendida que pasaba entre
las mitades de los cadáveres de los animales. \bibverse{18} Así fue como
el Señor hizo un acuerdo con Abrán ese día, prometiéndole: ``Yo le daré
esta tierra a tus descendientes. Se extiende desde el Wadi de
Egypto\footnote{15.18 ``Wadi de Egipto'': No el Nilo, sino lo que hoy se
  conoce como el Wadi Arish. Ver Números 34:5; Josué 15:4, Josué 15:47.}hasta
el gran Río Éufrates, \bibverse{19} e incluye el territorioo de los
quenitas, los quenizitas, los cadmoneos, \bibverse{20} los heteos, los
ferezeos, los refaítas, \bibverse{21} los amorreos, los cananeos, los
gergeseos, y los jebuseos''.

\hypertarget{section-15}{%
\section{16}\label{section-15}}

\bibverse{1} Sarai, la esposa de Abrán, no había podido darle hijos. Sin
embargo, ella poseía una esclava egipcia cuyo nombre era Agar,
\bibverse{2} así que Sarai le dijo a Abrán: ``Por favor, escúchame. El
Señor no me permite tener hijos. Así que por favor ve y acuéstate con mi
esclava. Así quizás podré tener una familia por medio de ella''. Abrán
aceptó la sugerencia de Sarai. \bibverse{3} Así que Sarai, la esposa de
Abrán, tomó a su esclava egipcia Agar, y se la entregó a su esposo como
su esposa. Abrán había estado viviendo en la tierra de Canaán por diez
años cuando esto sucedió.

\bibverse{4} Abrán durmió con Agar y ella quedó embarazada. Cuando ella
se dio cuenta de que estaba embarazada, comenzó a tratar a Sarai con
desdén.\footnote{16.4 ``comenzó a tratar a Sarai con desdén'',
  literalmente, ``su señora se veía pequeña ante sus ojos''. Otra
  traducción sería ``miraba a su señora con desprecio''.}

\bibverse{5} Entonces Sarai se quejó con Abrán: ``¡Esto que estoy
sufriendo es por tu culpa! Te entregué a mi esclava para que te
acostaras con ella, y ahora que sabe que está embarazada me trata con
menosprecio. ¡Que el Señor decida entre los dos quién es el culpable, si
tú o yo!''

\bibverse{6} ``¡Es tu esclava!'' respondió Abrán. ``Puedes hacer con
ella lo que quieras''.Entonces Sarai trató a Agar con tanta crueldad,
que Agar huyó.\footnote{16.6 ``Huyó'': el hebreo dice ``huyó de ella'',
  pero Agar no solo huyó de Sarai, sino del campamento.}

\bibverse{7} Entonces el ángel del Señor vino al encuentro de Agar junto
un manantial en el desierto que está de camino alsur.

\bibverse{8} Y le preguntó: ``¿De dónde vienes, Agar, esclava de Sarai,
y hacia dónde vas?''

``Estoy huyendo de mi señora Sarai'',respondió Agar.

\bibverse{9} ``Vuelve a donde tu señora y obedécele'',le dijo el ángel
del Señor. \bibverse{10} Y continuó diciendo: ``Yo te daré muchos
descendientes, y serán tantos que no podrán contarse''. \bibverse{11} Y
siguió diciéndole: ``Escucha, ahora estás embarazada y tendrás un hijo.
Le pondrás por nombre Ismael,\footnote{16.11 Ismael significa ``Dios
  oye''.}porque el Señor ha escuchado cuánto has sufrido. \bibverse{12}
Él será como un asno salvaje, que peleará con todos, y todos pelearán
con él. Siempre estará en discordia con sus familiares''.

\bibverse{13} Desde ese momento, Agar clamó al Señor que habló con ella:
``Eres el Dios que me ve'',porque ella dijo: ``Aquí vi al que me ve''.
\bibverse{14} Es por ese que ese pozo\footnote{16.14 Este pozo es la
  misma fuente de agua a la que se le llama manantial en el versículo 7.}se
llama ``el pozo del Ser Viviente que me ve''.Aún existe entre Cades y
Béred.

\bibverse{15} Agar dioa luz un hijo para Abrán, y Abrán le puso por
nombre Ismael. \bibverse{16} Cuando Agar tuvo a Ismael, Abrán tenían 86
años.

\hypertarget{section-16}{%
\section{17}\label{section-16}}

\bibverse{1} Cuando Abrán tenía 99 años, el Señor se le apreció y le
dijo: ``Yo soy el Dios Altísimo. Vive en mi presencia y haz el
bien.\footnote{17.1 ``Vive en mi presencia y no hagas el mal'',
  literalmente, ``camina delante de mí y sé inocente''.} \bibverse{2} Yo
haré mi pacto contigo, y te daré muchos descendientes''.

\bibverse{3} Abrán se inclinó y puso su rostro en el suelo; y Dios le
dijo: \bibverse{4} ``¡Escucha, Abrán! Este es el acuerdo que hago
contigo. Serás el padre de muchas naciones, \bibverse{5} así que tu
nombre ya no será más Abrán. En su lugar, tunombre será
Abraham\footnote{17.5 El cambio de nombre suele interpretarse como un
  cambio de Abrán (``padre exaltado'') a Abraham (``padre de muchos'').}porque
yo te haré padre de muchas naciones. \bibverse{6} Yo me aseguraré de que
tengas un gran número de descendientes. Ellos se transformarán en muchas
naciones, y algunos de sus reyes también vendrán de tu linaje.
\bibverse{7} Yo te prometo guardar mi pacto contigo, y con tus
descendientes, por todas las generaciones futuras. Este es un pacto
eterno. Yo siempre seré tu Dios y el Dios de tus descendientes.
\bibverse{8} Yo te daré a ti y a tus descendientes todo el país de
Canaán---donde has vivido como extranjero---como tu tierra para siempre,
y yo seré su Dios''.

\bibverse{9} Entonces Dios le dijo a Abraham: ``Tu parte consiste en
guardar mi pacto, tanto tú como tus descendientes, por todas las
generaciones futuras. \bibverse{10} Este es mi acuerdo contigo y con tus
descendientes, el acuerdo que debes guardar: Todo hombre entre ustedes
será circuncidado. \bibverse{11} Vas a circuncidar la carne de tu
prepucio, y esta será la señal del pacto entre mi y ustedes.
\bibverse{12} Desde ahora y por todas las generaciones, todo hombre
entre ustedes será circuncidado a los 8 días después de nacer. Esto no
solo se aplicará a tus hijos sino a todo varón que nazca en tu casa, o
que sea comprado de los extranjeros. \bibverse{13} Debes circuncidar a
los varones nacidos en tu casa o comprado de los extranjeros, como señal
externa de mi pacto. \bibverse{14} Cualquier varón incircunciso que se
niegue a circuncidarse será expulsado del pueblo, porque habrá
quebrantado mi pacto''.

\bibverse{15} Entonces Dios le dijo a Abraham: ``Ahora, en lo que
concierne a Sarai, tu esposa. No la llamarán Sarai nunca más. En su
lugar, su nombre será Sara. \bibverse{16} Yo la bendeciré y prometo
darte un hijo por medio de ella. Yo la bendeciré para que se convierta
en la madre de todas las naciones, y habrá reyes entre sus
descendientes''.

\bibverse{17} Abraham se inclinó y puso su rostro en el suelo. Pero por
dentro se reía, y se preguntaba: ``¿Cómo podré tener un hijo a la edad
de cien años? ¿Cómo podría Sara tener un hijo a sus noventa años?''

\bibverse{18} Abraham le dijo a Dios: ``¡Que Ismael viva siempre con tu
bendición!''

\bibverse{19} ``¡No, será tu esposa Sara quien te dará un hijo!''
respondió Dios. ``Lo llamarás Isaac.\footnote{17.19 Isaac significa ``él
  se ríe''.}Yo guardaré mi pacto con él y con sus descendientes como un
pacto eterno. \bibverse{20} Ahora bien, en lo que a Ismael se refiere,
escuché lo que dijiste y también lo bendeciré. Me aseguraré de que tenga
muchos descendientes. Será el padre de doce príncipes, y yo lo
convertiré en una gran nación. \bibverse{21} Pero guardaré mi pacto con
Isaac, el hijo que tendrá Sara para estos días el próximo año''.
\bibverse{22} Cuando Dios terminó de hablar con Abraham, se retiró de su
presencia.

\bibverse{23} Ese día Abraham circuncidó a su hijo Ismael y a todos los
que habían nacido en su casa, así como a los que había comprado, y todos
los varones que habitaban en su casa, tal como Dios se lo dijo.
\bibverse{24} Abraham tenían 99 años cuando fue circuncidado,
\bibverse{25} y su hijo Ismael tenía 13 años. \bibverse{26} Tanto
Abraham como su hijo Ismael fueron circuncidados en el miso día.
\bibverse{27} Todos los hombres en la casa de Abraham, incluyendo los
nacidos o comprados como esclavos extranjeros, fueron circuncidados con
él.

\hypertarget{section-17}{%
\section{18}\label{section-17}}

\bibverse{1} El Señor se le apareció a Abraham en medio de los robles de
Mamré. Abraham estaba sentado a la entrada de su tienda, pues hacía
mucho calor ese día. \bibverse{2} Abraham levantó la vista y de repente
vio a tres hombres en pie. Al verlos, corrió a su encuentro y se inclinó
hasta el suelo.

\bibverse{3} Entonces les dijo: ``Señor,\footnote{18.3 Parece que
  Abraham se dirigía a uno de ellos, tal vez viéndolo como su líder.}si
le parece bien, no siga su camino sin antes hospedarse conmigo, en la
casa de su siervo. \bibverse{4} Permítanme traerles agua para lavarse
sus pies, y para que descansen junto al árbol. \bibverse{5} También
permítanme traerles algo de comer para que puedan recobrar sus fuerzas
cuando sigan el camino, ahora que ha venido a visitarme''.

``Nos parece bien'',respondieron ellos. ``Haz lo que has dicho''.

\bibverse{6} Abraham se apresuró a la tienda y le dijo a Sara:
``¡Apresúrate! Prepara pan con tres medidas grandes\footnote{18.6
  ``Medidas grandes'': literalmente ``seahs'', que se estiman como
  aproximadamente 20 kilos o 44 libras.}de la mejor harina. Amasa la
masa y prepara el pan''. \bibverse{7} Entonces Abraham corrió hasta
donde estaba el ganado y eligió un becerro bueno y joven, y se lo dio a
su siervo, quien lo mató y lo cocinó rápidamente. \bibverse{8} Entonces
Abraham tomó un poco de yogurt yleche, y cocinó la carne. Luego trajo la
comida delante de los tres hombres y se quedó cerca junto a un árbol
mientras ellos comían.

\bibverse{9} ``¿Dónde está tu esposa Sara?'' le preguntaron.

``Está allá adentro, en la tienda'',les contestó.

\bibverse{10} Entonces uno de ellos le dijo: ``Te prometo que el próximo
año volveré a visitarte por estos días, y tu esposa Sara tendrá un
hijo''.Y Sara estaba escuchando mientras se ocultaba a la entrada de la
tienda, detrás de él.

\bibverse{11} Abraham y Sara ya estaban viejos y eran de edad avanzada.
Y Sara ya había pasado su edad fértil. \bibverse{12} Sara se estaba
riendo dentro de la tienda, y decía para sí: ``¿Cómo podría experimentar
placer alguno ahora que estoy vieja y cansada? ¡Mi esposo también está
viejo!''

\bibverse{13} Entonces el Señor le preguntó a Abraham: ``¿Por qué Sara
se rió, y preguntó `¿cómo podré tener un hijo ahora que estoy tan
vieja?' \bibverse{14} ¿Acaso hay algo difícil para el Señor? Volveré el
próximo año durante la primavera, tal como te lo dije, y para entonces
Sara tendrá un hijo''.

\bibverse{15} Entonces Sara tuvo temor y negó el hecho, diciendo: ``Yo
no me reí''.

``Sí, te reíste'',respondió el Señor.

\bibverse{16} Entonces los hombres se fueron. Miraron en dirección a
Sodoma\footnote{18.16 Claramente podían ver a Sodoma abajo en el valle,
  desde su punto de vista más alto.}y se dirigieron hacia allá. Y
Abraham los acompañó parte del camino.

\bibverse{17} Entonces el Señor dijo: ``¿Debería ocultarle a Abraham lo
que voy a hacer? \bibverse{18} Abraham sin duda será una nación grande y
ponderosa, y todas las naciones de la tierra serán benditas a través de
él. \bibverse{19} Yo lo he elegido para que le enseñe a sus hijos y a su
familia a seguir el camino del Señor haciendo lo que es bueno, a fin de
que yo, el Señor pueda cumplir lo que le he prometido a Abraham''.

\bibverse{20} Entonces el Señor continuó diciendo: ``Hay muchas quejas
expresadas contra Sodoma y Gomorra a causa de su pecado descarado.
\bibverse{21} Voy a ver si estas quejas son ciertas. Si no lo son, de
seguro lo sabré''.

\bibverse{22} Los dos hombres se dieron la vuelta y se dirigieron a
Sodoma, pero el Señor se quedó con Abraham.

\bibverse{23} Entonces Abraham se le acercó y le preguntó: ``¿En serio
vas a destruir a las personas buenas junto con las malvadas?
\bibverse{24} ¿Qué si hay cincuenta buenas personas en la ciudad? ¿Vas a
destruir la ciudad a pesar de que haya cincuenta personas buenas allí?
\bibverse{25} ¡No puedes hacer algo así! No puedes matar a las personas
buenas junto con las malvadas, pues estarías tratando a buenos y malos
del mismo modo. ¡No puedes actuar así! ¿No actuará con justicia el Juez
de toda la tierra?''

\bibverse{26} ``Si encuentro a cincuenta personas buenas en Sodoma,
perdonaré a toda la ciudad por causa de ellos'',respondió el Señor.

\bibverse{27} ``Como ya comencé, permíteme seguir hablando con mi Señor,
aunque no soy nadie sino apenas polvo y cenizas'',continuó Abraham.
\bibverse{28} ``¿Qué si hay cuarenta y cinco personas buenas -- solo
cinco menos -- ¿Aun así vas a destruir toda la ciudad solo porque son
menos personas buenas?''

``No la destruiré si encuentro cuarenta y cinco personas
buenas'',respondió el Señor.

\bibverse{29} Entonces Abraham habló nuevamente y le preguntó al Señor:
``¿Qué pasaría y solo hay cuarenta?''

``No lo hare por causa de las cuarenta personas'',respondió el Señor.

\bibverse{30} ``Mi Señor, no te enojes conmigo'',continuó Abraham.
``Pero permíteme preguntarte esto: ¿Qué pasaría si hay treinta?''

``No lo hare por causa de las treinta personas'',respondió el Señor.

\bibverse{31} ``Debo admitir que he sido osado en hablar de esta manera
con mi Señor'',dijo Abraham. ``¿Qué sucedería si solo hubiera veinte
personas buenas?''

``No lo haré por causa de las 20 personas'',respondió el Señor.

\bibverse{32} ``Por favor, no te enojes conmigo, mi Señor'',dijo
Abraham. ``Solo permíteme preguntar una cosa más. ¿Qué pasaría si hay
solamente diez personas buenas?''

``No la destruiré por causa de las diez personas'',respondió el Señor.

\bibverse{33} Entonces el Señor se fue cuando terminó de hablar con
Abraham, y Abraham se fue a casa.

\hypertarget{section-18}{%
\section{19}\label{section-18}}

\bibverse{1} Los dos ángeles\footnote{19.1 ``Ángeles'': el relato
  alterna entre llamar a los dos visitantes ``ángeles'' y ``hombres''.}llegaron
esa noche a Sodoma. Y Lot estaba sentado en la puerta de la ciudad. Y
cuando vio a los hombres se levantó para recibirlos, y se inclinó con su
rostro en tierra.

\bibverse{2} ``Señores, por favor, entren y quédense en mi casa esta
noche'',les dijo. ``Pueden lavar sus pies y seguir su camino temprano
por la mañana''.

Pero ellos le respondieron: ``No te preocupes. Pasaremos la noche aquí
en la plaza''.

\bibverse{3} Pero Lot insistió, y los dos hombres fueron con él a su
casa. Les preparó alimentos y coció pan para que comieran. \bibverse{4}
Pero ellos ni siquiera se habían ido aún a la cama, cuando unos hombres
de Sodoma, jóvenes y adultos, de cada parte de la ciudad, vinieron y
rodearon la casa. \bibverse{5} Entonces le gritaron a Lot: ``¿Dónde
están los hombres que se hospedaron en tu casa esta noche? Tráelos, pues
queremos tener sexo con ellos''.

\bibverse{6} Entonces Lot saió a hablar con ellos en la entrada de su
casa, cerrando la puerta al salir.

\bibverse{7} ``¡Amigos, por favor, no cometan tal perversidad!
\bibverse{8} Como verán, tengo dos hijas vírgenes. Puedo traerlas para
que hagan con ellas lo que quieran, pero por favor no le hagan nada a
estos hombres. Yo soy responsable de cuidarlos''.\footnote{19.8 ``Es mi
  responsabilidad cuidar de ellos'': literalmente, ``Han venido para
  estar seguros bajo mi techo''.}

\bibverse{9} ``¡Apártate de nuestro camino!'' gritaron. ``¿Quién crees
que eres, que vienes a vivir aquí, y ahora tratas de juzgarnos? ¡A ti te
haremos peores cosas que las que íbamos a hacerles a ellos!'' Entonces
empujaron a Lot y trataban de derribar la puerta.

\bibverse{10} Pero los hombres que estaban dentro de la casa salieron y
tomarona Lot, lo trajeron dentro y cerraron la puerta de golpe.
\bibverse{11} Entonces hicieron que todos los hombres que estaban en la
entrada de la casa, jóvenes y adultos, quedasen ciegos, así que no
podían encontrar la puerta.

\bibverse{12} Entonces los dos hombres le preguntaron a Lot: ``¿Hay
alguien más en tu familia, como yernos, hijos e hijas, o alguna otra
persona en la ciudad? Si es así, asegúrate de que se vayan,
\bibverse{13} porque estamos a punto de destruir este lugar. Las quejas
que han subido hasta el Señor son tan graves que él nos ha enviado a
destruirla''.

\bibverse{14} De inmediato Lot fue a hablar con los hombres que estaban
comprometidos con sus hijas. ``¡Levántense y salgan de aquí!'' les dijo,
``porque el Señor está a punto de destruir la ciudad!'' Pero ellos
pensaron que se trataba de una broma.

\bibverse{15} Al atardecer, los ángeles le rogaron a Lot que se
apresurara, diciéndole:``¡Apúrate! Sal ahora mismo con tu esposa y con
tus dos hijas de aquí, de lo contrario serán destruidas cuando caiga el
castigo sobre la ciudad''. \bibverse{16} Pero Lot dudó. Entonces los
hombres tomaron a Lot, a su esposa y a sus hijas por la mano, y los
arrastraron hasta llevarlos fuera de la ciudad. El Señor fue
misericordioso en hacer esto con ellos.

\bibverse{17} Tan pronto se encontraron fuera de la ciudad, uno de los
hombres dijo: ``¡Corran y salven sus vidas! ¡No miren hacia atrás, y no
se detengan en ninguna parte para ir por el valle! ¡Corran hacia las
montañas, o serán destruidos!''

\bibverse{18} ``¡Señor, por favor, no me hagas esto!'' respondió Lot.
\bibverse{19} ``Si bien te parece, ya que has sido tan misericordioso en
salvar mi vida, no me hagas correr hacia las montañas, pues no podré
lograrlo. ¡La destrucción me alcanzará y moriré! \bibverse{20} Hay una
ciudad cerca, a la cual puedo correr y es muy pequeña. Por favor, déjame
correr hasta allí, pues es muy pequeña y así podré salvar mi vida''.

\bibverse{21} ``Está bien. Haré lo que me pides'',respondió el Señor.
``No destruiré la ciudad que me has mencionado. \bibverse{22} Pero
apresúrate y vete allí rápidamente, porque no podré continuar hasta que
estés allí''. (Por esto esta ciudad de llamó Zoar.)\footnote{19.22 Zoar
  significa ``lugar pequeño''. Originalmente se llamaba Bela (ver 14:2).}

\bibverse{23} Cuando Lot llegó a Zoar ya había salido el sol.
\bibverse{24} Entonces el Señor hizo llover desde el cielo fuego y
azufre sobre Sodomay Gomorra. \bibverse{25} Y destruyó las ciudades por
completo con todos sus habitantes, así como el valle y todos los
cultivos que estaban creciendo allí. \bibverse{26} Pero la esposa de
Lot, que se había quedado atrás, miró hacia atrás y de inmediato se
convirtió en una estatua de sal.

\bibverse{27} A la mañana siguiente, Abraham se levantó temprano y
regresó al lugar donde había hablado con el Señor. \bibverse{28} Y miró
en dirección de Sodoma y Gomorra, así como todo el valle, y vio la
tierra ardiendo en llamas, expulsando humo como si fuera un horno.

\bibverse{29} Cuando Dios destruyó las ciudades del valle, no se olvidó
de la promesa que le había hecho a Abraham, y salvó a Lot de la
destrucción de las ciudades donde él vivía.

\bibverse{30} Lot tuvo miedo de quedarse en Zoar, así que salió de la
ciudad y se fue a vivir con sus dos hijas en una cueva, en las montañas.
\bibverse{31} Algún tiempo después, la hija mayor de Lot le dijo a su
hermana menor: ``Nuestro padre está envejeciendo, y no queda ningún
hombre que nos pueda dar hijos como a las demás. \bibverse{32}
Emborrachemos a nuestro padre con vino, y acostémonos con él para que
podamos hacer crecer esta familia''.

\bibverse{33} Así que esa noche emborracharon a su padre con vino. La
hija mayor se acostó con él, y él no se dio cuenta cuando ella se
acostó, ni cuando se levantó.

\bibverse{34} Al día siguiente, la hija mayor le dijo a la hija menor :
``Anoche yo me acosté con nuestro padre. Emborrachémoslo esta noche otra
vez para que tú puedas acostarte con él y podamos hacer crecer esta
familia''.

\bibverse{35} Así que una vez más, esa noche emborracharon a su padre
con vino, y la hija menor fue y se acostó con él. Y Lot no se dio cuenta
cuando ella se acostó ni cuando se levantó.

\bibverse{36} Fue así como ambas hijas de Lot quedaron embarazadas de su
propio padre. \bibverse{37} la hija mayor tuvo un hijo, al que llamó
Moab.\footnote{19.37 ``Moab'': Se entiende que significaba ``hijo de mi
  padre''.}Él es el ancestro de los moabitas hasta hoy. \bibverse{38} La
hija menor también tuvo un hijo, al que llamó Ben-ammi.+ 19.38
``Ben-ammi'': ``hijo de mi pueblo''.Él es el ancestro de los amonitas
hasta hoy.

\hypertarget{section-19}{%
\section{20}\label{section-19}}

\bibverse{1} Abraham emprendió viaje hacia el Neguev, y se quedó entre
Cades y Sur. Después se mudó y se fue a vivir a Gerar. \bibverse{2}
Mientras vivía allí, cada vez que hablaba de Sara decía ``Es mi
hermana''.De modo que Abimelec,\footnote{20.2 ``Abimelec'' significa
  ``mi padre es el rey'', o ``mi padre es Molec'', un dios cananeo. Esto
  bien podría haber sido un título formal en lugar de un nombre personal
  (ver también 26:8).}rey de Gerar, mandó a llamar a Sara y la tomó para
que fuera una de sus esposas.+ 20.2 ``Convertirse en una de sus
esposas'': Añadido para mayor claridad.

\bibverse{3} Pero Dios se le apareció a Abimelec en un sueño, y le dijo:
``¡Presta atención! Morirás, porque la mujer que has tomado ya está
casada. Ella tiene un esposo''.

\bibverse{4} Abimelec no había tocado a Sara, y preguntó: ``Señor,
¿acaso tú matas a las personas buenas? \bibverse{5} ¿Acaso no me dijo el
mismo Abraham `ella es mi hermana,' y acaso no dijo ella misma `él es mi
hermano'? ¡Hice esto siendo inocente y mi conciencia está limpia!''

\bibverse{6} Dios le dijo en el sueño: ``Sí, sé que hiciste esto con
toda inocencia, y evité que pecaras contra mi. Por eso no permití que la
tocaras. \bibverse{7} Envía a esta mujer con su esposo. Él es un
profeta, y orará por ti, y tú vivirás. Pero si no la envías de regreso,
debes saber que tú y toda tu familia morirán''.

\bibverse{8} Abimelec se levantó temprano a la mañana siguiente y reunió
a todos sus sirvientes. Les explicó lo ocurrido, y todos estaban
aterrorizados. \bibverse{9} Entonces Abimelec mandó a llamar a Abraham y
le preguntó: ``¿Qué has venido a hacernos? ¿Qué mal te he hecho para que
me trates de esta manera, trayendo este pecado terrible sobre mi y mi
reino? ¡Has hecho cosas que nadie debería hacer!''

\bibverse{10} YAbimelec le preguntó a Abraham: ``¿En qué estabas
pensando cuando hiciste esto?''

\bibverse{11} ``Pues yo dije para mí: `Nadie respeta a Dios en este
lugar. Me matarán para quedarse con mi esposa,'\,'' respondió Abraham.
\bibverse{12} ``De cualquier modo, ella es mi hermana, porque es la hija
de mi padre, pero no de mi madre, y yo me casé con ella. \bibverse{13}
Ya que Dios me hizo dejar a mi familia, le dije: `Si de verdad me amas,
dondequiera que vayas conmigo dirás: Él es mi hermano.'\,''

\bibverse{14} Entonces Abimelec le dio a Abraham dones de ovejas,
Ganado, esclavos y esclavas, y le devolvió a Sara. \bibverse{15} Y le
dijo: ``Contempla mi tierra, y elige dónde quieres vivir''.
\bibverse{16} Y a Sara le dijo: ``Ten en cuenta que le he dado a tu
esposo mil piezas de plata. Esto es para compensar el mal que te hemos
hecho ante los ojos de los que estaban contigo, y para que tu nombre
quede limpio ante todos los demás''.

\bibverse{17} Entonces Abraham oró a Dios, y Dios sanó a Abimelec ya su
esposa; y también sanó a sus esclavas, a fin de que nuevamente pudieran
engendrar hijos. \bibverse{18} Porque el Señor había hecho que las
mujeres fueran infértiles porque se habían llevado a Sara\footnote{20.18
  ``Se habían llevado a Sara'': añadido para mayor claridad.}, la esposa
de Abraham.

\hypertarget{section-20}{%
\section{21}\label{section-20}}

\bibverse{1} Y el Señor vino a ayudar a Sara, tal como se lo había
prometido. El Señor cumplió la promesa que le había hecho a Sara.
\bibverse{2} Sara quedó embarazada y tuvo un hijo de Abraham cuando ya
era viejo, en el tiempo exacto, como Dios lo había dicho. \bibverse{3}
Abraham llamó a su hijo Isaac. \bibverse{4} Y lo circuncidó a los ocho
días de nacido, según el mandato de Dios. \bibverse{5} Y Abraham tenía
100 años cuando nació Isaac.

\bibverse{6} Sarah entonces declaró: ``Dios me ha hecho
reír,\footnote{21.6 Isaac significa ``él se ríe''.}y todos los que
escuchen acerca de esto se reirán conmigo''. \bibverse{7} Además dijo:
``¿Habría podido alguien decirle a Abraham que Sara tendría que dar de
mamar a un hijo suyo? ¡Y ahora he tenido un hijo de Abraham aún su
vejez!''

\bibverse{8} Y el bebé creció, y el día que fue destetado Abraham hizo
una gran fiesta. \bibverse{9} Pero Sara se dio cuenta de que Ismael, el
hijo que la esclava egipcia Agar había tenido para Abraham, se burlaba
de Isaac. \bibverse{10} Entonces Sara fue donde Abraham y le dijo:
``Tienes que deshacerte de esa mujer esclava y de su hijo! ¡Un hijo de
esa esclava no será coheredero con mi hijo Isaac!''

\bibverse{11} Abraham se sintió muy triste porque Ismael era su hijo
también. \bibverse{12} Pero Dios le dijo a Abraham: ``No te sientas mal
en cuanto al hijo de la mujer esclava. Haz lo que Sara te pide, porque
tu descendencia será contada a través de Isaac. \bibverse{13} Así que no
te preocupes, porque yo también convertiré al hijo de la esclava en una
gran nación, porque él también es tu hijo''.

\bibverse{14} A la mañana siguiente, Abraham se levantó temprano. Empacó
alimentos y un odre con agua para Agar, y puso todo esto en sus hombros.
Entonces los despidió. Ella se fue y anduvoerrante por el desierto de
Beerseba.

\bibverse{15} Cuando se le acabó el agua, dejó al niño en medio de unos
arbustos. \bibverse{16} Entonces se fue y se sentó a cierta distancia, a
unos cientos de yardas de distancia,\footnote{21.16 ``A unos pocos
  cientos de metros de distancia'': literalmente, ``a un disparo de
  arco''.}pues pensaba: ``¡No podré soportar ver a mi hijo morir!'' Y al
sentarse, reventó en llanto.

\bibverse{17} Dios escuchó el llanto del niño, y el ángel de Dios llamó
a Agar desde el cielo y le preguntó: ``¿Qué ocurre, Agar? ¡No tengas
miedo! Dios ha escuchado el llanto del niño desde donde está.
\bibverse{18} Levántate, ve a ayudar a tu hijo y consuélalo, porque yo
lo convertiré en una gran nación''.

\bibverse{19} Entonces Dios abrió sus ojos y ella pudo ver un pozo que
estaba cerca. Así que fue y llenó su odre de agua y le dio de beber al
niño.

\bibverse{20} Dios bendijo a Ismael y él creció, viviendo en el
desierto. Se convirtió en un arquero con una gran habilidad.
\bibverse{21} Vivió en el desierto de Parán. Su madre le eligió una
esposa de la tierra de Egipto.

\bibverse{22} En aquél mismo tiempo, Abimelec y Ficol, jefe de su
ejército, vinieron a ver a Abraham. Y Abimelec le dijo: ``Dios te
bendice en todo lo que haces''. \bibverse{23} Y continuó: ``Así que
júrame aquí y hora que no me traicionarás, ni a mis hijos, ni a mis
descendientes. Del mismo modo que te he demostrado mi lealtad, haz lo
mismo conmigo y con mi nación, en la cual vives''.

\bibverse{24} ``Así lo juro'',respondió Abraham. \bibverse{25} Entonces
Abraham planteó ante Abimelec un problema relacionado con un pozo del
que los siervos de Abimelec se habían apoderado a la fuerza.

\bibverse{26} ``No sé quién hizo esto, y no lo habías mencionado antes.
Nunca había oído acerca de esto hasta hoy'',respondió Abimelec.

\bibverse{27} Entonces Abraham le dio a Abimelec algunas de sus ovejas y
ganado, y los dos hicieron un pacto. \bibverse{28} Abraham también
apartó siete corderas del rebaño.

\bibverse{29} ``¿Qué significan esas siete corderas que has apartado del
rebaño?'' le preguntó Abimelec.

\bibverse{30} ``Te doy estas siete corderas como compensación por tu
reconocimiento de que yo cavé este pozo'',respondió Abraham.
\bibverse{31} Por eso llamaron ese lugar Beerseba,\footnote{21.31
  Beerseba significa tanto ``pozo del juramento'' como ``pozo de las
  siete''.}porque ahí los dos juraron e hicieron un pacto.

\bibverse{32} Después de haber hecho el pacto en Beerseba, Abimelec y
Ficol -- el comandante de su ejército --, se fueron y llegaron a la
tierra de los filisteos. \bibverse{33} Abraham plantó un árbol de
tamarisco en Beerseba y allí adoró al Señor, al Dios eterno.
\bibverse{34} Y Abraham vivió en el país de los filisteos por muchos
años.

\hypertarget{section-21}{%
\section{22}\label{section-21}}

\bibverse{1} Algún tiempo después, Dios puso a prueba a Abraham. Y lo
llamó: ``¡Abraham!''

``Aquí estoy'',respondió Abraham.

\bibverse{2} Entonces Dios le dijo: ``Ve con tu hijo, el hijo al que
amas, tu único hijo, a la tierra de Moriah y sacrifícalo como una
ofrenda quemada sobre el altar en una de las montañas que yo te
mostraré''.

\bibverse{3} A la mañana siguiente, Abraham se levantó temprano y
ensilló su asno. Tomó consigo a dos siervos y a Isaac, y se fue a cortar
leña para quemar la ofrenda. Y se fue con ellos al lugar que Dios le
había dicho.

\bibverse{4} Después de viajar por tres días, Abraham pudo finalmente
ver el lugar a la distancia. \bibverse{5} Y le dijo a sus siervos:
``Esperen aquí con el asno mientras yo voy con mi hijo y adoro a Dios.
Después regresaremos''.

\bibverse{6} Entonces Abraham hizo que Isaac cargara la leña para la
ofrenda que debía quemar, mientras que él llevaba el fuego y el
cuchillo, y caminaron juntos.

\bibverse{7} Isaac le dijo a Abraham, ``Padre\ldots{}''

``Dime, hijo\ldots{}'' respondió Abraham.

``Puedo ver que tenemos el fuego y la madera, pero ¿dónde está el
cordero para la ofrenda que vamos a quemar?'' preguntó Isaac.

\bibverse{8} ``Dios proveerá el cordero para la ofrenda que vamos a
quemar, hijo mío'',respondió Abraham, y siguieron caminando juntos.

\bibverse{9} Cuando llegaron al lugar que Dios les había mostrado,
Abraham construyó un altar y puso sobre él la leña. Entonces amarró a su
hijo Isaac y lo puso sobe el altar sobre la madera. \bibverse{10} Y
Abraham tomó el cuchillo, listo para sacrificar a su hijo.

\bibverse{11} Pero el ángel del Señor le gritó fuerte desde el cielo,
diciendo ``¡Abraham! ¡Abraham!''

``Sí, aquí estoy'',respondió.

\bibverse{12} Entonces el ángel le dijo: ``¡No toques al niño! No le
hagas nada, porque ahora sé que realmente obedeces a Dios, pues no te
negaste a darme a tu hijo, a tu único hijo''.

\bibverse{13} Abraham entonces elevó su mirada y vio a un carnero que
estaba enredado con sus cuernos en medio de los arbustos. Trajo al
carnero y lo sacrificó como ofrenda en lugar de su hijo. \bibverse{14} Y
Abraham llamó a aquél lugar ``El Señor proveerá''.Esa es una frase que
la gente usa aun hoy: ``El Señor proveerá en esta montaña''.

\bibverse{15} Entonces el ángel del Señor gritó otra vez a Abraham desde
el cielo: \bibverse{16} ``Te juro por mí mismo, dice el Señor, que por
lo que has hecho y por no haberte negado a darme a tu hijo, a tu único
hijo, \bibverse{17} puedes estar seguro de que te bendeciré y te daré
muchos descendientes. Serán tan numerosos como las estrellas del cielo y
la arena del mar, y conquistarán a sus enemigos.\footnote{22.17
  ``Conquistar a sus enemigos'': literalmente, ``tomar posesión de las
  puertas de sus enemigos''.} \bibverse{18} Y todas las naciones de la
tierra serán benditas por tus descendientes porque tú me obedeciste''.

\bibverse{19} Entonces Abraham regresó donde estaban sus siervos, y se
fueron juntos a Beerseba, donde vivía Abraham.

\bibverse{20} Algún tiempo después, a Abraham le informaron: ``Milca ha
tenido hijos con tu hermano Nacor''. \bibverse{21} Uz fue el
primogénito, luego nació su hermano Buz, después Quemuel (quien vino a
ser el ancestro de los arameos), \bibverse{22} Quesed, Hazo, Pildas,
Jidlaf, y Betuel. \bibverse{23} (Betuel fue el padre de Rebeca.) Milca
tuvo estos ocho hijos con Nacor, el hermano de Abraham. \bibverse{24}
Además, su concubina Reúma tuvo a Tebahj, a Gajam, a Tajas, y a Maaca.

\hypertarget{section-22}{%
\section{23}\label{section-22}}

\bibverse{1} Sara vivió hasta los 127 años, \bibverse{2} y entonces
murió en Quiriat-Arba (o Hebrón) en la tierra de Canaán. Abraham fue
adentro\footnote{23.2 ``Fue adentro'': posiblemente a la tienda, donde
  yacía el cuerpo.}para lamentar su muerte y llorar por ella.

\bibverse{3} Entonces Abraham se levantó y fue a hablar con los líderes
de los hititas. \bibverse{4} ``Yo soy un extranjero, un extraño que vive
entre ustedes'',les dijo. ``Por favor, permítanme comprar un lugar de
sepultura para que pueda sepultar a mi difunta esposa''.

\bibverse{5} Entonces los hititas le respondieron a Abraham, diciéndole:
\bibverse{6} ``Escucha, mi señor, tú eres un príncipe muy respetado
entre nosotros. Elige el mejor lugar para sepultar a tu difunta. Ninguno
de nosotros se opondrá''.

\bibverse{7} Abraham se levantó y después se inclinó ante estos hititas,
\bibverse{8} y les dijo: ``Si les parece bien ayudarme a sepultar a mi
difunta, escuchen mi propuesta. ¿Podrían, por favor, pedirle a Efrón,
hijo de Zojar, \bibverse{9} que me venda la cueva de Macpela que está
ubicada en el extremo del campo que es de su propiedad?Estoy dispuesto a
pagarle el precio total aquí en presencia de ustedes, para así yo tener
un lugar de sepultura''.

\bibverse{10} Efrónel hitita estaba allí sentado en medio de su pueblo.
Y le respondió a Abraham en presencia de los hititas que estaban en las
puertas de la ciudad. \bibverse{11} ``No, mi señor'',le dijo. ``Por
favor, escúchame. Yo te regalaré el campo y la cueva que está allí. Te
lo regalo y mi pueblo es testigo. Por favor, ve y sepulta a tu
difunta''.

\bibverse{12} Abraham se inclinó ante los habitantes locales,
\bibverse{13} y para que todos lo escucharan, le dijo a Efrón: ``Por
favor, escúchame. Yo pagaré el precio del campo. Toma el dinero y déjame
ir a sepultar a mi difunta allí''.

\bibverse{14} Efrón le respondió a Abraham, diciéndole: \bibverse{15}
``Mi señor, escúchame, por favor. La tierra vale cuatrocientas piezas de
plata.\footnote{23.15 ``Cuatrocientas piezas de plata'': Se ha aceptado
  de manera general que era una cantidad exorbitante.}¿Pero qué valor
tiene eso para nosotros? Ve y sepulta a tu difunta''.

\bibverse{16} Abraham aceptó la oferta de Efrón y Abraham calculó el
peso y le dio a Efrón las cuatrocientas piezas de plata que había dicho,
usando el peso estándar que usaban los mercaderes, y delante de los
hititas como testigos.

\bibverse{17} De esta manera, la propiedad se traspasó legalmente.
Incluía el campo de Efrón en Macpela, cerca de Mamré, tanto el campo
como la cueva estaban incluidos en el precio, así como los árboles
plantados dentro del campo, y toda el área hasta los límites.
\bibverse{18} Todo esto vino a ser entonces propiedad de Abraham, y los
hititas que se encontraban a las puertas de la ciudad fueron testigos de
esta transacción.

\bibverse{19} Entonces Abraham fue y sepultó a su esposa Sara en la
cueva que estaba en el campo de Macpela, cerca de Mamré (o Hebrón) en la
tierra de Canaán. \bibverse{20} La propiedad del campo y de la cueva fue
transferida de los hititas a Abraham para que fuera su lugar de
sepultura.

\hypertarget{section-23}{%
\section{24}\label{section-23}}

\bibverse{1} Abraham ya estaba muy avanzado en años, y el Señor lo había
bendecido de todas las formas posibles. \bibverse{2} Y Abraham le dijo a
su siervo más viejo que estaba a cargo de toda su casa: ``Pon tu mano
bajo mi muslo,\footnote{24.2 Era costumbre hacer esto al jurar y hacer
  pacto.} \bibverse{3} y júrame por el Señor, el Dios del cielo y de la
tierra, que no dejarás que mi hijo se case con una de las hijas de los
caananitas entre los cuales vino. \bibverse{4} En lugar de ello, irás a
mi tierra donde viven mis familiares, y encontrarás allí una esposa para
mi hijo Isaac''.

\bibverse{5} ``¿Y qué pasará si la mujer se niega a venir conmigo a este
país?'' preguntó el siervo. ``¿Debería entonces traerme a tu hijo al
país de donde vienes?''

\bibverse{6} ``No, no debes llevarte a mi hijo para allá'',respondió
Abraham. \bibverse{7} ``El Señor, el Dios del cielo, me tomó del seno de
mi familia y de mi propio país. Habló conmigo y me juró, haciendo un
voto con la promesa:`Yo le daré esta tierra a tus descendientes.' Él es
quien enviará a su ángel delante de ti para que puedas encontrar a una
esposa para mi hijo. \bibverse{8} Sin embargo, si la mujer se rehúsa a
venir aquí contigo, entonces quedarás libre de este juramento. Pero
asegúrate de no llevarte a mi hijo para allá''.

\bibverse{9} El siervo puso su mano bajo el muslo de su señor Abraham y
juró un voto de hacer conforme él se lo había dicho. \bibverse{10}
Entonces el siervo preparó diez camellos de su señor para llevar todo
tipo de regalos de parte de Abraham y se fue hacia la ciudad de Nacor,
en Aram-Najaraim.\footnote{24.10 ``Aram-naharaim'': or ``Mesopotamia''.}
\bibverse{11} Al llegar en la noche, hizo que los camellos se
arrodillaran junto a un fuente que estaba a las afueras de la ciudad.
Esta era la hora en que las mujeres salían a buscar agua.

\bibverse{12} Y el siervo oró: ``Señor, Dios de mi señor Abraham, por
favor, haz que este sea un día exitoso, y muestra tu
fidelidad\footnote{24.12 ``Fidelidad'': esta palabra, a menudo traducida
  como ``amor fiel'', en este escenario tiene que ver realmente con
  ``lealtad'', ``compromiso'', e incluso ``amabilidad''.}hacia mi señor
Abraham. \bibverse{13} Ahora pues, me encuentro junto a esta Fuente, y
las mujeres jóvenes de la ciudad vendrán a buscar agua. \bibverse{14}
Haz que suceda de la siguiente manera: Que la joven a quien yo le diga
`Por favor, sostén tu cántaro para que yo pueda beber,' y me responda:
`Por favor bebe tú y tus camellos también,' que sea ella la mujer que
has elegido como esposa de tu siervo Isaac. De esta forma sabré que has
mostrado fidelidad a mi señor''.

\bibverse{15} Y aún no había terminado de orar cuando vio a Rebeca que
venía a buscar agua, llevando el agua en un cántaro sobre su hombro.
Ella era la hija de Betuel, hijo de Milcaj. Milcajera la esposa de
Nacor, el hermano de Abraham. \bibverse{16} Ella era una mujer muy
hermosa,y era virgen porque nadie se había acostado con ella. Ella
descendió hasta la fuente, llenó su cántaro, y regresó. \bibverse{17}
Entonces el siervo se apresuró para alcanzarla y le preguntó: ``Por
favor, déjame beber unos cuantos sorbos de agua de tu cántaro''.

\bibverse{18} ``Por favor, bebe, mi señor'',respondió. Y rápidamente
tomó su cántaro de sus hombres y lo sostuvo para que él pudiera beber.
\bibverse{19} Cuando terminó de ayudarle a beber, le dijo: ``Permíteme
darle de beber a tus camellos también, hasta que se sacien''.

\bibverse{20} Ella vació rápidamente su cántaro en el bebedero de los
camellos, y corrió hasta la fuente para buscar más agua. Y trajo
suficiente agua para sus camellos.

\bibverse{21} El hombre la observaba en silencio para ver si el Señor
había hecho de este un día exitoso. \bibverse{22} Cuando los camellos
terminaron de beber, él le dio un zarcillo de oro y puso dos brazaletes
pesados en sus muñecas.\footnote{24.22 Los pesos se describen como medio
  siclo para el aro de la nariz, y diez siclos para los brazaletes. Como
  no se conoce el precio del oro en ese momento, es imposible estimar su
  valor. Sin embargo, fueron regalos significativos.}

\bibverse{23} Entonces le preguntó: ``¿Quién es tu padre? ¿Sabes si
puedo encontrar posada en la casa de tu padre para pasar la noche?''

\bibverse{24} Y ella respondió: ``Soy la hija de Betuel, el hijo de
Milcáj y Nacor''.Y continuó: ``Tenemos suficiente lugar paja y comida
para los camellos, \bibverse{25} y desde luego tenemos posada para ti
esta noche''.

\bibverse{26} El hombre se arrodilló y se inclinó en actitud de
adoración al Señor. \bibverse{27} Y oró: ``Gracias Señor, Dios de mi
señor Abraham''. ``No has olvidado tu promesa y tu fidelidad con mi
señor. ¡Y Señor, tú me has guiado directamente al hogar de los
familiares de mi señor Abraham!''

\bibverse{28} Ella corrió a la casa de su madre y le contó a su familia
lo que había sucedido. \bibverse{29} Rebeca tenía un hermano llamado
Labán, y él corrió al encuentro del hombre que se había quedado junto a
la fuente. \bibverse{30} Labán había notado el zarcillo y los brazaletes
que ella estaba usando, y había escuchado a su hermana Rebeca
decir:``Esto es lo que me dijo aquél hombre''.Cuando llegó, todavía el
hombre estaba de pie con sus camellos junto a la fuente.

\bibverse{31} ``Por favor, ven conmigo, bendito del Señor'',le dijo
Labán. ``¿Qué esperas aquí? Tengo la casa preparada para ti, y un lugar
donde los camellos pueden estar''.

\bibverse{32} Así que el hombre se fue con él a su casa. Labán descargó
los camellos y les dio paja y comida. También trajo agua para que el
hombre lavara sus pies, y también para los hombres que venían con él.
\bibverse{33} Entonces Labánmandó a traer alimentos.

Pero el hombre le dijo: ``No voy a comer hasta que les haya dicho por
qué estoy aquí''.

``Por favor, explícanos'',le respondió Labán.

\bibverse{34} ``Soy el siervo de Abraham'',comenzó el hombre.
\bibverse{35} ``El Señor a bendecido abundantemente a mi señor y ahora
es un hombre rico y poderoso. El Señor le ha dado ovejas, Ganado, plata
y oro, siervos y siervas, así como camellos y asnos. \bibverse{36} Su
esposa Sara tuvo un hijo de mi señor incluso siendo avanzada de edad, y
mi señor le ha dado todo lo que posee. \bibverse{37} Mi señor Abraham me
ha hecho jurarle un voto, diciendo: `No debes buscar esposa para mi hijo
entre las hijas caananitas entre quienes vivo. \bibverse{38} Sino ve a
la tierra done vive mi familia, y busca allí una esposa para mi hijo
Isaac.'

\bibverse{39} Y yo le dije a mi señor Abraham: `¿Qué pasa si esta mujer
no desea venir aquí conmigo?'

\bibverse{40} Y él me dijo: `El Señor, en cuya presencia he vivido mi
vida, enviará a su ángel contigo, y hará que tu viaje sea exitoso.
Encontrarás una esposa entre mi familia, de la familia de mi padre.
\bibverse{41} Serás liberado de tu juramento si al ir a mi familia,
ellos se niegan a dejarla regresar contigo.'

\bibverse{42} Hoy cuando llegué a la fuente, oré al Señor:``Dios de mi
señor Abraham, por favor haz que sea un día exitoso. \bibverse{43} Ahora
que me encuentro aquí junto a la fuente, haz que si una joven viene a
buscar agua, y yo le pida diciendo `Por favor, dame de beber un poco de
agua,' \bibverse{44} y ella me diga: `Por favor, bebe y yo traeré agua
para los camellos también', que esa sea la que has elegido como esposa
para tu siervo Isaac''.

\bibverse{45} ``Yo aún no había terminado de orar en silencio cuando vi
a Rebeca que iba a buscar agua, cargando su cántaro sobre su hombro.
Descendió hasta la fuente para buscar agua, y le dije: `Por favor, dame
de beber.' \bibverse{46} Y ella de inmediato tomó el cántaro de su
hombre y me dijo: `Por favor, bebe, y yo traeré para tus camellos
también.' Así que bebí y ella trajo agua para los camellos.

\bibverse{47} Yo le pregunté: `¿Quién es tu padre?' Y ella respondió:
`Soy la hija de Betuel, el hijo de Milcáj y Nacor.' Así que puse un
zarcillo en su nariz, y los brazaletes en su muñeca.

\bibverse{48} Entonces me arrodilé y me incliné para adorar al Señor. Le
agradecí al Señor, al Dios de mi señor Abraham, porque me condujo
directamente hasta la sobrina de mi señor Abrahan para su hijo.
\bibverse{49} Así que díganme ahora su ustedes mostrarán compromiso y
fidelidad a mi señor Abraham? Dígan me si aceptarán o no para que yo
pueda decidir qué hacer''.

\bibverse{50} Entonces Labány Betuel respondieron: ``Sin duda todo esto
viene del Señor, así que no podemos oponernos de ninguna manera.
\bibverse{51} Rebeca está aquí, puedes tomarla y llevártela. Ella puede
ser la esposa del hijo de tu señor Abraham, tal como lo ha decidido el
Señor''.

\bibverse{52} Tan pronto como el siervo de Abraham escuchó esta
decisión, se inclinó y adoró al Señor. \bibverse{53} Entonces desempacó
joyas de plata y oro, así como ropas finas, y se las dio a Rebeca.
También le dio regalos de valor a su hermano y a su madre. \bibverse{54}
Entonces el siervo y sus hombres comieron y bebieron, y pasaron la noche
allí. Cuando se levantaron por la mañana, dijo ``Es mejor que me vaya
ahora a cada de mi señor Abraham''.

\bibverse{55} Pero su hermano y su madre dijeron: ``Déjala permanecer
con nosotros diez días más, y después podrá irse''.

\bibverse{56} ``Por favor no me hagan demorar'',les dijo él. ``El Señor
me ha dado éxito en este viaje, así que déjenme ir donde está mi
señor''.

\bibverse{57} ``Llamemos a Rebeca y preguntémosle lo que ella desea
hacer'',sugirieron ellos.

\bibverse{58} Entonces llamaron a Rebeca y le preguntaron: ``¿Quieres
irte ahora mismo con este hombre?''

``Sí, me iré'',respondió ella.

\bibverse{59} Entonces dejaron que Rebeca, la hermana de Labán se fuera
con el siervo de Abraham, junto a la criada que la había cuidado desde
pequeña. \bibverse{60} Pidieron una bendición sobre ella diciendo:
``Nuestra querida hermana, que seas la madre de miles de descendientes,
y que tus hijos conquisten a sus enemigos''. \bibverse{61} Entonces
Receba y su sierva se subieron en sus camellos. Siguieron al siervo de
Abraham, y se fueron.

\bibverse{62} Mientras tanto, Isaac, que vivía en el Neguev, acababa de
regresar de Beer-lahai-roi. \bibverse{63} Salió a los campos una tarde
para pensar las cosas.\footnote{24.63 ``Pensar las cosas'': a menudo se
  traduce como ``meditar'', pero el significado de estas palabras es
  incierto. Sin embargo, Isaac podría saber que su futura esposa estaba
  por venir, el cual era un evento de gran importancia para su vida.}Entonces
miró a la distancia y vio venir los camellos.

\bibverse{64} Rebeca también miraba desde la distancia. Y cuando vio a
Isaac, descendió del camello. \bibverse{65} Y le preguntó al siervo:
``¿Quién es ese que viene en camino a nuestro encuentro?''

``Él es mi señor, Isaac'',\footnote{24.65 No se identifica
  específicamente a Isaac en este texto; sin embargo, el sirviente
  simplemente dice: ``Él es mi amo'', lo que normalmente significaría
  Abraham.}respondió. Entonces ella se puso el velo para cubrirse.

\bibverse{66} Y el siervo le dijo a Isaac todo lo que había hecho.
\bibverse{67} Entonces Isaac tomó a Rebeca y la llevó a la tienda de su
madre Sara y se casó con ella. La amó, y ella le dio consuelo por la
muerte de su madre.

\hypertarget{section-24}{%
\section{25}\label{section-24}}

\bibverse{1} Abraham se casó con otra mujer. Su nombre era Quetura.
\bibverse{2} Con ella tuvo los siguientes hijos: Zimrán, Jocsán, Medán,
Madián, Isbac y Súaj. \bibverse{3} Jocsán fue el padre de Seba y Dedán.
Los descendientes de Dedán fueron lo asureos, los letusitas y los
leumitas. \bibverse{4} Los hijos de Madián fueron Efa, Efer, Hanoc,
Abida y Eldaa. Todos estos fueron los hijos de Quetura.

\bibverse{5} Abraham le dejó todas sus posesiones a Isaac. \bibverse{6}
Pero mientras aún estaba con vida, le dio regalos a los hijos de sus
concubinas y las envió al oriente, para que vivieran lejos de Isaac.

\bibverse{7} Abraham vivió hasta la edad de 175 años \bibverse{8} cuando
dio su último suspiro y murió en buena vejez, habiendo vivido
suficientes años. Había vivido una vida plena y ahora se había unido a
sus antepasados en la muerte. \bibverse{9} Sus hijos Isaac e Ismael lo
sepultaron en la cueva de Macpela, cerca de Mamré, en el campo que antes
había pertenecido a Efrón, hijo de Zojar, el hitita. \bibverse{10} Esre
fue el terreno que Abraham le había comprador a los hititas. Y Abraham
fue sepultado allí, junto a su esposa Sara. \bibverse{11} Después de la
muerte de Abraham, Dios bendijo a su hijo Isaac, quien vivía cerca de
Beer-lahai-roi.

\bibverse{12} Esta es la genealogía de Ismael, hijo de Abraham: Su madre
Agar fue la esclava egipcia de Sara. \bibverse{13} Y estos fueron los
nombres de los hijos de Ismael, según su familia genealógica: Nebayot,
(primogénito), Quedar, Adbeel, Mibsam, \bibverse{14} Misma, Duma, Masá,
\bibverse{15} Hadad, Tema, Jetur, Nafis, y Quedema. \bibverse{16} Estos
fueron los hijos de Ismael, y también fueron los nombres de los lugares
donde vivieron y acamparon: Las doce familias líderes de sus tribus.
\bibverse{17} Ismael vivió hasta la edad de 137 años. Entonces dio un
último suspiro, y se unió a sus antepasados en la muerte. \bibverse{18}
Los descendientes de Ismael habitaron la región de Havila hasta Sur,
cerca de la frontera de Egipto, en dirección de Asur. Y siempre estaban
peleando unos con otros.\footnote{25.18 El significado hebreo de este
  versículo no está claro. Sin embargo, nótese el versículo 16:12.}

\bibverse{19} La siguiente es la genealogía Isaac, hijo de Abraham:
Abraham fue el padre de Isaac. \bibverse{20} Cuando Isaac cumplió los 40
años, se casó con Rebeca, la hija de Betuel el arameo de Padán-Aram y
hermana de Labán el arameo.

\bibverse{21} Isaac oró al Señor pidiendo su ayuda en favor de su esposa
porque no podía tener hijos. El Señor respondió su oración y ella quedó
embarazada. \bibverse{22} Los dos hijos que tuvo eran gemelos y peleaban
dentro de su vientre. Entonces ella le preguntó al Señor: ``¿Por qué me
pasa esto a mí?''

\bibverse{23} ``Tienes dentro de ti dos naciones'',respondió el Señor.
``Darás a luz a dos hijos que competiránel uno contra el otro. Uno será
más fuerte que el otro, y el mayor será siervo del menor''.

\bibverse{24} Cuando llegó el momento, Rebeca tuvo dos gemelos.
\bibverse{25} El primero en salir estaba rojo al nacer, y estaba
cubierto de mucho cabello, como si fuera un cabrito. Por eso lo llamaron
Esaú.\footnote{25.25 Esaú suena como la palabra usada para ``cabello''.}
\bibverse{26} Luego salió su hermano gemelo, quien salió de la matriz
agarrando el talón de Esaú. Por eso lo llamaron Jacob.+ 25.26 Jacob
suena como las palabras ``talón'' o ``engañador''. Isaac tenía 60 años
cuando sus dos hijos nacieron.

\bibverse{27} Estos dos hijos de Isaac crecieron, y Esaú se convirtió en
un cazador de gran talento en el campo. Jacob era tranquilo y se quedaba
en casa, en las tiendas. \bibverse{28} Isaac amaba a Esaú porque le
preparaba comida con los animales que cazaba, mientras que Rebeca amaba
a Jacob.

\bibverse{29} Cierto día, Jacob estaba preparando un guisado cuando Esaú
regresó del campo, cansado y muy hambriento. \bibverse{30} ``Dame un
poco de ese guisado rojo'',le dijo Esaú a Jacob. ``¡Muero de hambre!''
(Fue así como Esaú obtuvo su otro nombre: ``Edom'',que significa
``rojo''.)

\bibverse{31} ``Primero véndeme tus derechos de primogénito'',le
respondió Jacob.

\bibverse{32} ``¡Mira! ¡Me estoy muriendo! ¿De qué me sirven esos
derechos de primogénito?'' dijo Esaú.

\bibverse{33} ``Primero tienes que jurármelo'',exigió Jacob. Así que
Esaú hizo juramento vendiéndole a Jacob sus derechos de primogénito.
\bibverse{34} Entonces Jacob le dio a Esaú un poco de pan, y un guisado
de lentejas. Esaú comió y bebió hasta que se sació y se fue. Al hacer
esto, Esaú demostró cuán poco le importaban sus derechos como hijo
primogénito.

\hypertarget{section-25}{%
\section{26}\label{section-25}}

\bibverse{1} Y hubo un gran hambruna en el país, no la misma que la que
ocurrió en el tiempo de Abraham, sino más tarde. Así que Isaac se mudó a
Gerar, en el territorio de Abimelec, rey de los filisteos.

\bibverse{2} El Señor se le apareció a Isaac y le dijo: ``No vayas. A
Egipto, sino al país que yo te mostraré. \bibverse{3} Quédate aquí en
este país. Yo estaré contigo y te bendeciré, porque voy a darte a ti y a
tus descendientes todas estas tierras. Yo guardaré la promesa solemne
que yo le juré a tu padre Abraham. \bibverse{4} Yo haré que tu
descendencia sea tan numerosa como las estrellas del cielo, y les daré
todas estas tierras. Todas las naciones de la tierra serán benditas por
tus descendientes, \bibverse{5} porque Abraham hizo lo que yo le dije, y
siguió mis órdenes, mis mandamientos, mis preceptos y mis leyes''.

\bibverse{6} Así que Isaac se quedó en Gerar. \bibverse{7} Cuando los
hombres de esa tierra le preguntaron por su esposa, él respondió
diciendo: ``Ella es mi hermana'',porque tuvo miedo. Pues pensó para sí
mismo: ``Si digo que ella es mi esposa, me matarán para quedarse con
Rebeca, pues es muy hermosa''. \bibverse{8} Pero más tarde, después de
haber vivido allí por un tiempo, Abimelec, rey de los filisteos, miró
por la ventana y vio a Isaac acariciando a su esposa Rebeca.

\bibverse{9} Abimelec entonces mandó a buscar a Isaac y le planteó su
queja. ``¿Según lo que vi, ella es claramente tu esposa!'' le dijo.
``¿Por qué decidiste decir `es mi hermana'?''

``Porque pensé que me matarían por causa de ella'',respondió Isaac.

\bibverse{10} ``¿Por qué nos hiciste esto?'' le preguntóAbimelec. ``¡Uno
de los hombres aquí pudo haberse acostado con tu esposa, y tú nos
habrías hecho culpables a todos!''

\bibverse{11} Abimelec emitió una orden a todo el pueblo,
advirtiéndoles: ``Cualquiera que toque a este hombre o a su esposa, será
ejecutado''.

\bibverse{12} Isaac volvió a sembrar ese año, y el Señor lo bendijo con
una cosecha que fue cien veces más grande que lo que había sembrado.
\bibverse{13} Se volvió un hombre rico, y su riqueza creciócontinuamente
hasta que se volvió muy rico. \bibverse{14} Y poseía muchos rebaños de
ovejas, manadas de bueyes, y muchos esclavos. Tanía tantas riquezas que
los filisteos comenzaron a sentir celos de él. \bibverse{15} Así que los
filisteos usaron basura para tapar todos los pozos que su Padre Abraham
había cavado.

\bibverse{16} Entonces Abimelec le dijo a Isaac: ``Tienes que abandonar
nuestro país, porque te has vuelto demasiado poderoso para nosotros''.

\bibverse{17} Así que Isaac se fue y estableció su campamentoen el valle
de Gerar, donde se quedó a vivir. \bibverse{18} Allí destapó los pozos
que había cavado su padre Abraham en su tiempo, los que los filisteos
habían tapado después de la muerte de Abraham. Y les puso los mismos
nombres que su padre les había puesto.

\bibverse{19} Los siervos de Isaac también cavaron un pozo en el valle y
encontraron una fuente de agua. \bibverse{20} Pero los pastores de Gerar
comenzaron a tener discordias con los pastores de Isaac, diciendo:
``¡Esa agua es nuestra!'' Así que Isaac llamó a aquel pozo
``Discordia'',porque allí tuvieron discordia con él. \bibverse{21} Isaac
tenía otro pozo cavado, y allí también hubo discordias. A ese pozo le
llamó ``Oposición''.\footnote{26.21 ``Oposición'': La palabra es, de
  hecho, la forma femenina de la palabra ``satán'', que quiere decir
  oponente o adversario.} \bibverse{22} Así que se fueron de allí y
cavaron otro pozo. Esta vez no hubo ninguna discordia respecto a este
pozo, y lo llamó ``Libertad'',+ 26.22 ``Libertad'': literalmente,
``espacio amplio/abierto'', que a menudo se utiliza en hebreo como
sinónimo de libertad, ya que entonces se le da a la gente espacio para
moverse. Véase, por ejemplo, Job 36:16; Salmos 118:5.diciendo: ``Ahora
el Señor nos ha dado libertad para crecer y ser exitosos en esta
tierra''.

\bibverse{23} De allí Isaac se fue hacia Beerseba. \bibverse{24} Y esa
noche el Señor se le apareció y le dijo: ``Yo soy el Dios de tu padre
Abraham. No tengas miedo, porque yo estoy contigo. Te bendeciré y te
daré muchos descendientes por causa de mi siervo Abraham''.
\bibverse{25} Isaac entonces construyó un altar y adoró al Señor.
También estableció sus tiendas, y sus siervos cavaron un pozo en ese
lugar.

\bibverse{26} Algún tiempo después, Abimelec vino desde Gerar con su
consejero Ahuzat y con Ficol, jefe de su ejército, para ver a
Isaac.\footnote{26.26 Ver 21:22. En vista del tiempo que transcurre
  entre estos eventos, es poco probable que sean los mismos individuos.
  Probablemente se trataba de títulos oficiales más que de nombres
  personales.} \bibverse{27} ``¿Por qué han venido a verme?'' les
preguntó Isaac. ``¡Antes me odiaban y me pidieron que me fuera!''

\bibverse{28} ``Ahora nos hemos dado cuenta de que el Señor está
contigo'',le respondieron. ``Así que hemos decidido hacer un pacto
contigo con juramento. \bibverse{29} Tu nos prometerás que no nos harás
daño, así como nosotros nunca te hemos hecho daño. Reconocerás que
siempre te hemos tratado bien, y cuando te pedimos que te marcharas lo
hicimos con bondad. ¡Y mira cómo el Señor te bendice ahora!''

\bibverse{30} Así que Isaac mandó a preparar una comida especial para
celebrar este pacto. Comieron y bebieron, \bibverse{31} y se levantaron
temprano en la mañana y cada uno hizo juramento al otro. Entonces Isaac
los dejó ir, y ellos se fueron en paz.

\bibverse{32} Ese fue el día en que los siervos de Isaac que habían
estado cavando un pozo vinieron a decirle: ``¡Hemos hallado agua!''
\bibverse{33} Así que Isaac llamó a ese pozo ``juramento'',y por eso
hasta el día de hoy,el nombre de esa ciudad es ``Pozo del juramento''
(Beerseba).

\bibverse{34} Cuando Esaú cumplió 40 años, se casó con Judit, hija de
Beri,el hitita, y también con Basemat, hija de Elón, el hitita.
\bibverse{35} Y ellas le causaron muchas amarguras a Isaac y Rebeca.

\hypertarget{section-26}{%
\section{27}\label{section-26}}

\bibverse{1} Isaac estaba viejo y se estaba quedando ciego. Así que
llamó a su hijo mayor Esaú, diciendo: ``Hijo mío''. Y Esaú contestó:
``Aquí estoy''.

\bibverse{2} ``Ya estoy viejo'',dijo Isaac, ``Es posible que muera
pronto. \bibverse{3} Así que toma tu arco y tus flechas, ve a cazar y
tráeme algo de carne. \bibverse{4} Prepárame una comida de buen sabor
para que yo coma, y para bendecirte antes de morir''.

\bibverse{5} Rebeca escuchó lo que Isaac le dijo a Esaú. Así que cuando
Esaú se fue al campo a buscar carne de caza, \bibverse{6} Rebeca le dijo
a su hijo Jacob: ``¡Escucha! Acabo de oír a tu padre decirle a tu
hermano, \bibverse{7} `Tráeme carne de caza y prepárame una buena comida
de buen sabor para comer y bendecirte en presencia del Señor antes de
morir.' \bibverse{8} Ahora, hijo, escúchame y haz exactamente lo que yo
te diré. \bibverse{9} Ve al rebaño y tráeme dos cabras jóvenes. Yo las
cocinaré y prepararé la comida de buen sabor que a tu padre más le
gusta. \bibverse{10} Entonces se la llevarás a tu padre para comer, para
que te bendiga en la presencia del Señor antes de morir''.

\bibverse{11} ``Pero escucha'',le respondió Jacob a su madre Rebeca,
``mi hermano Esaú es un hombre velludo, y yo soy lampiño. \bibverse{12}
De pronto mi padre se dará cuenta cuando me toque. Entonces parecerá
como que lo estoy engañando y en lugar de bendición, recibiré una
maldición''.

\bibverse{13} ``Que la maldición caiga sobre mí, hijo mío'',respondió su
madre. ``Solo haz lo que te digo. Ve y tráeme las cabras jóvenes''.

\bibverse{14} Así que Jacob fue a buscarlas y las trajo para su madre, y
ella preparó una comida de buen sabor, tal como le gustaba a su padre.
\bibverse{15} Entonces Rebeca fue y tomó la mejor ropa de su hijo mayor
Esaú que ella tenía en su casa, y se la puso a Jacob, su hijo menor.
\bibverse{16} Y puso la piel de los corderos en sus manos, y en su
cuello. \bibverse{17} Entonces le entregó a su hijo Jacob la comida y el
pan que había preparado.

\bibverse{18} Jacob entró a ver a su padre, y lo llamó diciendo: ``Padre
mío, aquí estoy''.

``¿Cuál de mis hijos eres?'' preguntó Isaac.

\bibverse{19} ``Soy Esaú, tu hijo mayor'',le dijo Jacob a su padre.
``Hice lo que me pediste. Por favor, siéntate y come la carne de caza
que hice para ti, para que puedas bendecirme''.

\bibverse{20} ``¿Cómo pudiste encontrar tan rápido a un animal, hijo
mío?'' le preguntó Isaac.

``Es porque el Señor lo envió para mí'',respondió Jacob.

\bibverse{21} ``Ven aquí para que pueda tocarte, hijo mío'',le dijo
Isaac a Jacob, ``así sabré si eres en verdad Esaú''.

\bibverse{22} Así que Jacob se acercó a su padre Isaac, quien al tocarlo
dijo: ``Es la voz de Jacob, pero son las manos de Esaú''. \bibverse{23}
Isaac no se dio cuenta de que era Jacob porque tenía las manos con
vellos como Esaú, así que se preparó para bendecirlo.

\bibverse{24} ``En realidad eres tú mi hijo Esaú?'' preguntó de nuevo.
``Si, soy yo'',respondió Jacob.

\bibverse{25} Entonces dijo: ``Hijo mío, tráeme de la comida de caza que
me has preparado para comer, y así podré darte mi bendición''.Así que
Jacob trajo para su padre Isaac comida para comer y vino para beber.

\bibverse{26} Después Isaac le dijo a Jacob: ``Ven y bésame, hijo mío''.
\bibverse{27} Entonces Jacob se inclinó y lo besó, e Isaac pudo oler la
ropa que Jacob estaba usando. Así que procedió a darle su bendición,
pensado para sí: ``El olor de mi hijo es como el olor de un campo que el
Señor ha bendecido''.

\bibverse{28} ``¡Que Dios use el rocío del cielo y la tierra fértil para
darte ricas cosechas de grano y vino nuevo! \bibverse{29} Que los
pueblos de distintas naciones te sirvan y se inclinen ante ti. Que todos
los que te maldigan sean malditos, y que sean benditos todos los que te
bendigan''.

\bibverse{30} Después de que Isaac terminó de bendecir a Jacob---de
hecho, Jacob ya se había ido de la presencia de su padre---Esaú regresó
de su viaje de caza. \bibverse{31} También había preparado una comida de
buen sabor, y se la trajo a su padre. Entonces Esaúle dijo a Isaac,
``Siéntate padre, y come de mi carne de caza para que puedas
bendecirme''.

\bibverse{32} ``¿Quién eres?'' le preguntó Isaac.

``Soy tu hijo Esaú, tu primogénito'',respondió.

\bibverse{33} Isaac entonces comenzó a temblar y preguntó: ``¿Quién fue
el que se fue de cacería y me trajo la comida? Ya la comí toda antes de
que llegaras y lo bendije. No puedo retirar su bendición ya''.

\bibverse{34} Cuando Esaú escuchó las palabras de su padre, gritó de
rabia y amargura, y le rogó a su padre: ``Por favor, bendíceme a mí
también, padre!''

\bibverse{35} Pero Isaac respondió: ``Tu hermano vino y me engañó. ¡Él
se ha robado tu bendición!''

\bibverse{36} ``¡Con toda razón su nombre es Jacob, el
impostor!''\footnote{27.36 ``Engañador''. Ver Gén. 25:26.}dijo Esaú.
``Me ha engañado dos veces. Primero se apropió de mi primogenitura, ¡y
ahora se ha robado mi bendición! ¿No has guardado una bendición para
mi?''

\bibverse{37} Entonces Isaac le contestó a Esaú: ``Lo he hecho tu señor,
y he dicho que todos sus parientes serán sus siervos. He declarado que
no le faltará el grano ni el nuevo vino. ¿Qué puedo dejar para ti, hijo
mío?''

\bibverse{38} ``¿Acaso solo tienes una bendición, padre mío?'' preguntó
Esaú. ``¡Por favor bendíceme a mi también!'' Entonces Esaú comenzó a
llorar a gritos.

\bibverse{39} Entonces su padre Isaac declaró: ``¡Escucha, hijo! Vivirás
lejos de la tierra fértil, lejos del rocío que cae del cielo.
\bibverse{40} Te ganarás el sustento con espada, y serás el siervo de tu
hermano. Pero cuando te rebeles, quitarás su yugo de tu cuello''.

\bibverse{41} Desde entonces Esaú sintió odio hacia Jacob, por causa de
la bendición de su padre. Esaú se dijo a sí mismo: ``Pronto llegará el
tiempo en que lamentaré la muerte de mi padre. ¡Y entonces mataré a mi
hermano Jacob!''

\bibverse{42} Sin embargo, Rebeca escuchó lo que había dicho Esaú, y
mandó a llamar a Jacob. ``Mira'',le dijo, ``tu hermano Esaú siente
consuelo en hacer planes de matarte. \bibverse{43} Así que, escúchame
atentamente, hijo mío, lo que te voy a decir. Vete inmediatamente a
donde mi hermano Labán en Arám. \bibverse{44} Quédate con él hasta que
la rabia de tu hermano se calme. \bibverse{45} Cuando esté en calma y se
le olvide, yo te mandaré a buscar de nuevo. Porque no quisiera perderlos
a los dos en un solo día''

\bibverse{46} Entonces Rebeca fue y le dijo a Isaac: ``Estoy cansada de
estas mujeres hititas . ¡Están arruinando mi vida! ¡Prefiero morir antes
de que Jacob llegue a casarse con una mujer hitita como ellas, una de
esas habitantes locales!''

\hypertarget{section-27}{%
\section{28}\label{section-27}}

\bibverse{1} Isaac llamó a Jacob y lo bendijo. ``No te cases con una
mujer cananea'',le ordenó. \bibverse{2} ``Vete ahora mismo a Padán Aram,
a la casa de Betuel, el padre de tu madre. Busca allí una esposa, una
hija de Labán, el hermano de tu madre. \bibverse{3} El Dioa Altísimo te
bendiga y que tus descendientes sean tan numerosos que llegues a ser el
ancestro de muchas naciones. \bibverse{4} Que Dios te concede a ti y a
tus descendientes la misma bendición que le dio a Abraham, para que
poseas la tierra en la que eres extranjero, la tierra que Dios le dio a
Abraham''.

\bibverse{5} Así que Isaac envió a Jacob, y Jacob se fue de viaje a
Paddan-aram, a la casa deLabán, hijo de Betuel, el arameo. Labán era el
hermano de Rebeca, la madre deJacob y Esaú.

\bibverse{6} Esaúdescubrió que Isaac había bendecido a Jacob y que lo
había enviado a Paddan-aram para encontrar allí una esposa, y que cuando
lo bendijo, le dijo: ``No te cases con una mujer cananea''. \bibverse{7}
También se enteró de que Jacob había obedecido a su padre y ahora se
dirigía hacia Paddan-aram. \bibverse{8} Esto hizo que Esaú descubriera
cuánto su padre aborrecía a las mujeres cananeas. \bibverse{9} Así que
Esaú fue a donde la familia de Ismael, y se casó con otra mujer llamada
Majalat, la hija de Ismael, hijo de Abraham, y hermana de Nebaiot.

\bibverse{10} Mientras tanto, Jacob había salido de Beerseba e iba de
camino hacia Arán. \bibverse{11} Llegó allí después de la puesta de sol,
y se quedó esa noche en un lugar. Tomó una piedra, la puso bajo su
cabeza, y se acostó a dormir.

\bibverse{12} Y Jacob tuvo un sueño en el que veía una escalera que
comenzaba en la tierra, y llegaba hasta el cielo. Vio a los ángeles de
Dios que subían y bajaban en ella. \bibverse{13} Entonces vio al Señor
que estaba en pie sobre él,\footnote{28.13 ``Sobre él'': o, ``sobre ella
  (la escalera)''.}y que dijo: ``Yo soy el Señor, el Dios de tu padre
Abraham, y el Dios de Isaac. Yo te doy a ti y a tus descendientes la
tierra en la que estás acostado ahora. \bibverse{14} Tus descendientes
serán tantos como el polvo de la tierra, y se esparcirán de oriente a
occidente, y de norte a sur. Todos sobre la tierra serán benditos por
tus descendientes. \bibverse{15} ¡Escucha! Yo estoy contigo y te cuidaré
dondequiera que vayas. Yo te traeré de regreso a esta tierra. No te
abandonaré, porque voy a hacer lo que te prometí''.

\bibverse{16} Cuando Jacob se despertó, se dijo a sí mismo: ``¡El Señor
está aquí, en este lugar, y no me había dado cuenta!''\footnote{28.16
  Jacob parece sorprenderse de que el Señor esté presente en cualquier
  lugar y no en algún ``lugar sagrado'' regular.} \bibverse{17} Entonces
se asustó y dijo: ``¡Este es un lugar terrible! Debe ser la casa de Dios
y la entrada al cielo''.

\bibverse{18} Cuando Jacob se levantó en la mañana, tomó la Piedra que
había puesto bajo su cabeza, y la colocó en forma vertical, como un
pilar de piedra, y roció aceite de oliva sobre ella. \bibverse{19} Y le
puso por nombre a ese lugar ``Betel'',\footnote{28.19 ``Betel''
  significa ``casa de Dios''.} (anteriormente su nombre era Luz).
\bibverse{20} Jacob también hizo una promesa solemne, diciendo: ``Dios,
si vas conmigo y me cuidas durante mi viaje, y me das alimento y bebida,
así como ropa para vestir \bibverse{21} para que pueda regresar a salvo
a la casa de mi padre, entonces tú, Señor, serás mi Dios. \bibverse{22}
Y este pilar que he levantado aquí, será la casa de Dios,+ 28.22 En
otras palabras, un lugar de culto.y yo te daré la décima parte de lo que
me des''.

\hypertarget{section-28}{%
\section{29}\label{section-28}}

\bibverse{1} Jacob se apresuró y se puso en marcha,\footnote{29.1 ``Se
  apresuró y se puso en marcha'': literalmente, ``levantó los pies''.}y
llegó a la tierra de los orientales. \bibverse{2} Al contemplar a su
alrededor, vio un pozo en un campo y a tres rebaños de ovejas acostadas
junto a él, esperando recibir agua. Una gran piedra cubría la boca del
pozo. \bibverse{3} La práctica común era que+ 29.3 ``La práctica común
era que'': añadido para mayor claridad.cuando todos los rebaños
llegaban, los pastores rodaban la piedra de la boca del pozo y le daban
agua a las ovejas, y luego colocaban la piedra en su lugar nuevamente.

\bibverse{4} Y Jacob les preguntó: ``Hermanos míos, ¿de dónde son
ustedes?''

``Somos de Harán'',respondieron.

\bibverse{5} ``¿Conocen a Labán, el nieto de Nacor?'' les preguntó.

``Si, lo conocemos'',respondieron.

\bibverse{6} ``¿Cómo está él?'' preguntó.

``Está bien'',respondieron. ``¡Mira!De hecho, la que viene allí con las
ovejas es su hija Raquel''.

\bibverse{7} ``Todavía es temprano'',dijo Jacob. ``Es muy pronto para
guardar las ovejas. ¿Por qué no les dan agua de beber y las dejan pastar
un poco más?''

\bibverse{8} ``No podemos hasta que hayan llegado todos los rebaños'',le
dijeron los pastores. ``Entonces rodamos la piedra del pozo y las
dejamos beber agua''.

\bibverse{9} Mientras aún hablaban, llegó Raquel con el rebaño que
pastoreaba para su padre. \bibverse{10} Cuando Jacob vio a Raquel, la
hija de Labán, que era el hermano de su madre, subió y rodó la piedra
del pozo para que las ovejas de Labán pudieran beber agua. \bibverse{11}
Entonces Jacob besó a Raquel y lloró de alegría. \bibverse{12} (Le había
dicho ya que él era el hijo del hermano de Labán y de Rebeca.) Y ella
corrió y le contó a su padre lo que había sucedido.

\bibverse{13} Tan pronto como Labán escuchó la noticia acerca de Jacob,
salió corriendo a su encuentro. Lo abrazó y lo besó, y se lo llevó a
casa. Después de que Jacob le explicó todo a Labán, \bibverse{14} Labán
le dijo: ``¡No hay duda alguna, eres carne de mi carne y sangre de mi
sangre!'' Y Jacob se quedó con Labán durante un mes.

\bibverse{15} Un día, Labán le dijo: ``¡Eres mi pariente, así que no es
justo que trabajes sin una compensación por ello! Dime entonces, ¿cuánto
debo pagarte?''

\bibverse{16} Labán tenía dos hijas. La mayor se llamaba Lea, y la más
joven se llamaba Raquel. \bibverse{17} Lea tenía ojos que expresaban
amabilidad\footnote{29.17 ``amabilidad'': literalmente, ``suave'' o
  ``gentil''.}, pero Raquel tenía un cuerpo armonioso y una apariencia
hermosa. \bibverse{18} Jacob se enamoró de Raquel, así que le prometió a
Labán: ``Trabajaré siete años+ 29.18 ``trabajaré siete años'': a
diferencia de lo que hizo el siervo de Abraham, Eliezer (capítulo 24),
Jacob había llegado sin regalos y sin dote, por lo que ofreció su
servicio como pago en especie.para ti por Raquel, tu hija menor''.

\bibverse{19} ``Pues para mi es mejor dártela a ti que a cualquier
otro'',respondió Labán. ``Así que quédate y trabaja para mi''.
\bibverse{20} Y Jacob trabajó para Labán por siete años, pero para él
fueron como días, porque realmente la amaba.

\bibverse{21} Entonces Jacob le dijo a Labán: ``Ha llegado el tiempo que
acordamos. Ahora dame a tu hija para que sea mi esposa''.

\bibverse{22} Y Labán organizó un banquete de bodas\footnote{29.22 ``Un
  banquete de bodas'': la palabra en realidad significa ``una fiesta
  para beber'', que es probablemente la única manera en que el engaño
  pudo haber tenido éxito.} y invitó a todos para que vinieran al
banquete. \bibverse{23} Pero al caer la noche, Labán trajo donde Jacob a
su hija Lea, y Jacob se acostó con ella. \bibverse{24} (Labán se había
encargado de que su sierva Zilpá fuera la criada personal de Lea.)

\bibverse{25} ¡Al amanecer, Jacob se dio cuenta de que era Lea! Así que
fue donde Labán y con enojo le preguntó: ``¿Qué me has hecho? ¡Trabajé
para ti por Raquel! ¿Por qué me has engañado?''

\bibverse{26} ``Aquí no entregamos a la hija menor en matrimonio antes
que a la hija mayor'',respondióLabán. \bibverse{27} ``Deja que la semana
de celebración por la boda termine, y entonces te daré a mi otra hija
también, pero con la condición de que trabajes siete años más para mi''.

\bibverse{28} Jacob estuvo de acuerdo. Terminó la semana de celebración
por la boda con Lea, y entonces Labán le dio a su hija Raquel como
esposa también. \bibverse{29} (Labán también se encargó de que su sierva
Bila fuera la criada personal de Raquel.) \bibverse{30} Así que Jacob se
acostó con Raquel, y amó a Raquel más que a Lea. Y trabajó para Labán
siete años más por Raquel.

\bibverse{31} Y el Señor vio que Lea no era amada, y la ayudó a tener
hijos. Pero no hizo lo mismo con Raquel. \bibverse{32} Así que Lea quedó
embarazada, y tuvo un hijo a quien llamó Rubén,\footnote{29.32 ``Rubén
  significa ``¡Miren, un hijo!'' y también suena como ``él vio mi
  sufrimiento''.}pues dijo: ``¡El Señor vio lo mucho que he sufrido y
ahora mi esposo me amará!''

\bibverse{33} Entonces Lea volvió a quedar embarazada, y tuvo un hijo. Y
dijo: ``El Señor ha esuchado que no soy amada, y me ha dado un
hijo''.Así que le puso por nombre Simeón.\footnote{29.33 ``Simeón''
  significa ``él escucha''.}

\bibverse{34} Lea volvió a quedar embarazada por tercera vez, y tuvo
otro hijo. Y dijo: ``Finalmente mi esposo se sentirá unido a mi, porque
ahora le he dado tres hijos''.Por eso le puso por nombre
Leví.\footnote{29.34 ``Leví'' significa ``conectado'' o ``unido''.}

\bibverse{35} Una vez más, Lea quedó embarazada y tuvo otro hijo. Lo
llamó Judá,\footnote{29.35 ``Judá'' significa ``alabanza''.}pues dijo:
``¡Ahora realmente puedo alabar al Señor!'' Y no tuvo más hijos después
de esto.

\hypertarget{section-29}{%
\section{30}\label{section-29}}

\bibverse{1} Cuando Raquel se dio cuenta de que no podía tener hijos con
Jacob, sintió celos de su hermana. Entonces Raquel puso su queja con
Jacob: ``¡Moriré si no me das hijos!''

\bibverse{2} Jacob se enojó con Raquel y le dijo: ``¿Acaso soy Dios?
¿Crees que soy el que impide que puedas tener hijos?''

\bibverse{3} ``Aquí está mi criada personal, Bila'' respondió Raquel.
``Acuéstate con ella para que ella tenga hijos por mi, y así yo también
tenga una familia''. \bibverse{4} Y Raquel le dio a Jacob a su criada
personal Bila como esposa, y Jacob se acostó con ella. \bibverse{5}
Entonces Bila quedó embarazada y tuvo un hijo de Jacob. \bibverse{6} Y
Raquel dijo: ``¡Dios ha juzgado a mi favor! Me escuchó y me ha dado un
hijo''.Y a este hijo lo llamó Dan.\footnote{30.6 Dan significa ``juez''.}
\bibverse{7} Bila, la criada personal de Raquel volvió a quedar
embarazada y tuvo un segundo hijo de Jacob. \bibverse{8} Y Raquel dijo:
``He tenido una contienda con mi hermana, pero ahora he ganado''.Y a
este hijo lo llamó Neftalí.+ 30.8 Neftalí significa ``lucha''.

\bibverse{9} Lea se dio cuenta de que no estaba teniendo más hijos, así
que le dio a Jacob a su criada personal Zilpá como esposa para Jacob.
\bibverse{10} Y Zilpá tuvo un hijo de Jacob. \bibverse{11} Entonces Lea
dijo: ``¡Cuán afortunada soy!'' Y a este hijo lo llamó Gad.\footnote{30.11
  Gad significa ``afortunado''.} \bibverse{12} Entonces Zilpá, la criada
personal de Lea volvió a quedar embarazada y tuvo otro hijo de Jacob.
\bibverse{13} Entonces Lea dijo: ``Soy muy feliz, y las otras mujeres
también lo dirán!'' Y a este hijo lo llamó Aser.+ 30.13 Aser significa
``feliz''.

\bibverse{14} Durante el tiempo de la cosecha de trigo, Rubén encontró
algunas plantas de mandrágoras mientras andaba por los campos de
cultivos. Y los llevó a su madre Lea. Entonces Raquel le dijo a Lea:
``Por favor, dame algunas de las mandrágoras que tu hijo encontró''.

\bibverse{15} ``¿Acaso no te basta con haberme robado a mi esposo?''
respondió Lea. ``¿Vas a tomar también las mandrágoras de mi hijo?''

``Bueno, podrás acostarte con él esta noche si me das algunas
mandrágoras a cambio'',respondió Raquel.

\bibverse{16} Cuando Jacob regresó de los campos esa noche, Lea fue a su
encuentro. ``Debes acostarte conmigo esta noche porque he pagado por ti
con las mandrágoras de mi hijo. Así que Jacob durmió con ella aquella
noche. \bibverse{17} Dios escuchó la petición de Lea, y quedó embarazada
y tuvo un quinto hijo con Jacob. \bibverse{18} Entonces Lea dijo: ``El
Señor me ha premiado por haberle dado a mi esposo mi criada personal''.Y
a este hijo lo llamó Isacar.\footnote{30.18 Isacar significa
  ``recompensa''.} \bibverse{19} Entonces Lea volvió a quedar embarazada
y tuvo un sexto hijo con Jacob. \bibverse{20} Y Lea dijo: ``Dios me ha
dado un buen regalo. Ahora mi esposo me honrará porque le he dado seis
hijos''.Y a este hijo lo llamó Zabulón.+ 30.20 Zabulón se asocia con las
palabras para ``regalo'' y ``honra''. \bibverse{21} Tiempo después, Lea
tuvo una hija a la cual llamó Dina.

\bibverse{22} Entonces Dios prestó atención a Raquel, y escuchó sus
plegarias, y la ayudó a tener hijos. \bibverse{23} Entonces Raquel quedó
embarazada y tuvo un hijo. Y dijo: ``Dios ha quitado mi desgracia''.
\bibverse{24} Y a este hijo lo llamó José,\footnote{30.24 José puede
  significar tanto que ``que añada'' o como ``él quita'', refiriéndose a
  la ``desgracia'' de Raquel.}diciendo: ``Que el Señor me de un hijo
más''.

\bibverse{25} Cuando José nació, Jacob le dijo a Labán: ``Déjame ir para
volver a casa y a mi propio país. \bibverse{26} Dame a mis esposas y a
mis hijos porque he trabajado por tenerlos. Déjame ir ahora porque
conoces todo el trabajo que he hecho para ti''.

\bibverse{27} ``Por favor, quédate'',respondió Labán, ``porque he
descubierto\footnote{30.27 ``Descubierto'': o, ``me enteré por
  adivinación''.}que el Señor me ha bendecido por tu causa''.
\bibverse{28} Entonces Labán continuó: ``Dime cuánto debo pagarte''.

\bibverse{29} ``Tú sabes bien cuánto trabajo he hecho para ti, y el buen
cuidado que he provisto a tus rebaños. \bibverse{30} ¡Cuando yo llegué
no tenías mucho, pero ahora tienes muchas cosas! El Señor te ha
bendecido por lo que yo he hecho. ¿Cuándo podré proveer para mi propia
familia?''

\bibverse{31} ``¿Qué me propones como pago hacia ti?'' preguntó Labán
nuevamente.

``No tienes que darme nada'',respondió Jacob. ``Si quieres hacer algo
por mí, qué te parece esto: Seguiré cuidando y alimentando tus rebaños.
\bibverse{32} Permíteme visitar tus rebaños hoy, y yo tomaré todas las
ovejas que tienen pecas o manchas, y todas las de lana oscura, y de
igual manera con los cabritos. Eso será mi pago. \bibverse{33} En el
futuro, podrás probar que he sido honesto. Cuando miren mis rebaños,
cualquier cabrito y oveja que no tenga manchas, o que no sea de lana
oscura, podrá considerarse que fue robada de tus rebaños''.

\bibverse{34} ``Muy bien'',aceptó Labán. ``Lo haremos como has dicho''.
\bibverse{35} Sin embargo, ese mismo día, Labán salió y apartó a todos
los cabritos machos con manchas y con rayas, así como todas las cabras
con manchas y de pelaje oscuro. Pidió a sus hijos que los cuidaran y los
mandó lejos \bibverse{36} para que estuvieran a tres días de camino
separados de Jacob, mientras que Jacob cuidaba del resto de los rebaños
de Labán.

\bibverse{37} Entonces Jacob cortó unas varas de álamo, de almendro, y
de plátano cuya madera era blanca bajo la corteza. Peló la corteza, e
hizo varas que lucían con rayas blancas. \bibverse{38} Y puso las varas
que había pelado en los bebederos de los rebaños, pues ahí era donde se
apareaban. \bibverse{39} Los rebaños se apareaban frente a las varas y
producían crías con rayas, con pintas y con manchas. \bibverse{40}
Entonces Jacob separó a todos estos. Entonces hizo que su rebaño se
pusiera de frente al rebaño de Labán que tenía rayas y era de pelaje
oscuro. Así fue como pudo mantener a su rebaño separado del rebaño de
Labán.

\bibverse{41} Cuando las hembras estaban a punto de dar a luz, Jacob
puso las varas en los bebederos donde los rebaños pudieran verlas
mientras se apareaban. \bibverse{42} Pero Jacob no hizo esto con las
hembras más débiles. Las más débiles se fueron del lado de Labán, y las
más fuertes se fueron del lado de Jacob. \bibverse{43} Así Jacob se
volvió muy rico, con un gran rebaño, y con muchos esclavos y esclavas,
así como camellos y asnos.

\hypertarget{section-30}{%
\section{31}\label{section-30}}

\bibverse{1} Jacob descubrió que los hijos Labán estaban decían: ``Jacob
se ha quedado con todo lo que le pertenecía a nuestro padre. Toda su
riqueza la obtuvo de nuestro padre''. \bibverse{2} Jacob también se dio
cuenta de que Labán lo había comenzado a tratar de manera diferente.

\bibverse{3} Entonces El Señor le dijo a Jacob: ``Regresa al país de tus
antepasados, al hogar de tus padres. Y yo estaré contigo''.

\bibverse{4} Jacob mandó a buscar a Raquel y a Lea, pidiéndoles que
vinieran a su encuentroen los campos donde estaba apacentando los
rebaños. \bibverse{5} ``He notado que su padre me está tratando
diferente a la manera como me trataba antes'',les dijo. ``Pero el Dios
de mi padre estará conmigo. \bibverse{6} Ustedes saben cuán duro he
trabajado para su padre. \bibverse{7} ¡Pero me ha estado engañando, y ha
reducido mi salario diez veces! Sin embargo, Dios no ha dejado que me
haga mal. \bibverse{8} Cuando dijo: `Te pagaré con cabras pintadas,'
entonces en todo el rebaño solo había cabritas jóvenes pintadas. Cuando
dijo: `Te pagaré con cabras con rayas,' entonces en todo el rebaño solo
había cabras jóvenes con rayas. \bibverse{9} Por eso Dios tomó todo el
rebaño de su padre y me lo dio a mi. \bibverse{10} Cuando el rebaño se
estaba apareando, tuve un sueño en el que vi a las cabras macho
apareándose con el rebaño donde todas las cabras eran de rayas, con
pintas o manchas. \bibverse{11} Entonces en el sueño, el ángel del Señor
me habló y me dijo: `Jacob!' y yo respondí: `Aquí estoy.'

\bibverse{12} Y me dijo: `Mira y te darás cuenta de que las cabras macho
que se aparean con el resto del rebaño, tienen rayas, o tienen pintas o
manchas, porque he visto lo que Labán te hizo. \bibverse{13} Yo soy el
Dios de Betel, donde echaste el aceite de oliva sobre el pilar de piedra
y me hiciste una promesa solemne. Ahora prepárate para salir de esta
tierra, y devuélvete a la tierra de tus padres.'''

\bibverse{14} ``No tenemos heredad de nuestro padre de todas
formas'',respondieron Raquel y Lea. \bibverse{15} ``Él nos trata como
extranjeras porque nos vendió a ti, y ahora ha gastado todo ese dinero.
\bibverse{16} Toda la riqueza que Dios le ha arrebatado ahora nos
pertenece a nosotras y a nuestros hijos, ¡así que haz lo que Dios te ha
dicho!''

\bibverse{17} Así que Jacob se alistó. Ayudó a sus hijos ya sus esposas
a subir a los camellos, \bibverse{18} y condujo a su rebaño frente a él.
Llevó consigo todas las posesiones que había ganado mientras vivió en
Padan-Harán, y partió de allí para volver a la tierra de su padre en
Canaán.

\bibverse{19} Mientras Labán estaba lejos de casa esquilando sus ovejas,
Raquel robó los ídolos de la casa\footnote{31.19 ``Ídolos de la casa'':
  pequeñas figuras consideradas importantes y ``de suerte'',
  representativas de los dioses paganos y consultadas para la toma de
  decisiones. A menudo eran figuras femeninas, y se asociaban con la
  fertilidad. También parecen ser importantes para determinar asuntos de
  propiedad y tierras, que es quizás otra razón por la que Raquel las
  tomó y por la que Labán tenía tanto interés en tenerlas de vuelta.}que
le pertenecían a su padre. \bibverse{20} Jacob también engañó a Labán el
arameo al no decirle que se escaparía. \bibverse{21} Así que Jacob se
apresuró para irse con todo lo que tenía, cruzó el río Eufrates, y se
encaminó hacia la región montañosa de Galaad.

\bibverse{22} Tres días después, Labán descubrió que Jacob había huido.
\bibverse{23} Tomando consigo a algunos de sus familiares, salió a
perseguir a Jacob y se encontró con él siete días más tarde en el país
montañoso de Galaad. \bibverse{24} Pero por la noche Dios visitó a Labán
en un sueño y le dijo: ``Ten cuidado con lo que le dices a Jacob. No
trates de persuadirlo para que regrese, ni lo amenaces''.\footnote{31.24
  ``No trates de persuadirlo para que regrese, ni lo amenaces'':
  literalmente, ``de bueno a malo''. Esta expresión idiomática cubría el
  rango de posibles enfoques que Labán pudo haber estado tentado a
  tomar, ya fuera tratar de inducir a Jacob a regresar ofreciéndole
  alguna recompensa, hasta amenazarlo por la fuerza o imponerle algún
  tipo de castigo.}

\bibverse{25} Jacob había establecido su campamento con tiendas en
Galaad cuando Labán se encontró con él. Así que Labán y sus familiares
hicieron lo mismo. \bibverse{26} ``¿Por qué me engañaste de esta
manera?'' le preguntó Labán a Jacob. ``¡Tomaste a mis hijas como si
fueran prisioneras llevadas al cautiverio con espadas! \bibverse{27}
¿Por qué te fuiste huyendo en secreto, tratando de ponerme una trampa?
¿Por qué no viniste a decírmelo?Si lo hubieras hecho, te habría
preparado una Buena despedida, con música con panderetas y liras.
\bibverse{28} ¡Ni siquiera me dejaste despedirme de mis nietos y nietas!
¡Has actuado de forma muy insensata! \bibverse{29} Podría castigarte,
pero el Dios de tu padre me habló anoche y me dijo: `Cuidado con lo que
le dices a Jacob. No intentes persuadirlo para que vuelva, ni tampoco lo
amenaces.' \bibverse{30} Es evidente que querías irte y volver a la casa
de tu familia, pero ¿por qué tenías que robar mis ídolos?''

\bibverse{31} ``Huí porque tenía miedo'',le explicó Jacob a Labán.
``Tenía miedo de que me quitaras a tus hijas a la fuerza. \bibverse{32}
En cuanto a tus ídolos, cualquiera que los tenga morirá. Puedes buscar
delante de nuestra familia, y si encuentras cualquier cosa que te
pertenezca, puedes tomarla''. (Jacob no sabía que Raquel había robado
los ídolos de la casa.)

\bibverse{33} Entonces Labán comenzó a buscar en las tiendas de Jacob,
Lea y las dos criadas personales, pero no encontró nada, y entonces
entró a la tienda de Raquel. \bibverse{34} Raquel había puesto los
ídolos de la casa en una alforja de camello y estaba sentada en ella.
Labán buscó cuidadosamente en toda la tienda pero no los encontró.
\bibverse{35} Entonces le dijo a su padre: ``Señor, por favor, no se
enfade conmigo por no estar en pie en su presencia, pero tengo mi
periodo menstrual''.Labán buscó en todas partes, pero no encontró los
ídolos.

\bibverse{36} Jacob se enojó con Labán y lo confrontó, diciendo: ``¿De
qué crimen soy culpable? ¿Qué mal te he hecho para que vengas a
buscarme? \bibverse{37} Has buscado entre todas mis posesiones. ¿Has
encontrado algo que te pertenezca? ¡Si es así, tráelo aquí delante de
nuestras familias y que sean ellos los que decidan quién tiene razón!

\bibverse{38} He trabajado para ti durante estos últimos veinte años, y
durante ese tiempo ninguna de tus ovejas y cabras ha abortado, ni yo he
comido ni un solo carnero de tu rebaño. \bibverse{39} Si alguno de ellos
fue asesinado por los animales salvajes, ni siquiera te traje el cadáver
para demostrarte la pérdida, sino que yo mismo la soporté. Pero tú, por
el contrario, siempre me has hecho compensar por los animales robados,
ya fuera de noche o a plena luz del día.

\bibverse{40} ya fuera sudando en el calor del día o congelándome en el
frío de la noche cuando no podía dormir, seguí trabajando para tu casa
durante veinte años. \bibverse{41} Trabajé catorce años por tus dos
hijas, y seis años más con tus rebaños. ¡Y me redujiste el sueldo diez
veces! \bibverse{42} Si no fuera por el Dios de mi padre, el Dios de
Abraham, el increíble Dios\footnote{31.42 ``El increíble Dios'':
  literalmente ``El Miedo''.}de Isaac, quien me cuidó, me habrías
despedido sin nada. Pero Dios vio mi sufrimiento, lo duro que trabajé y
te condenó anoche''.

\bibverse{43} Labán respondió: ``¡Estas son mis hijas, estos son mis
hijos, y estos son mis rebaños! De hecho, ¡Todo lo que ven aquí es mío!
Sin embargo, ¿qué puedo hacer ahora con mis hijas y sus hijos?
\bibverse{44} Así que hagamos un acuerdo solemne entre tu y yo, y será
testimonio de nuestro compromiso mutuo''.

\bibverse{45} Entonces Jacob tomó una piedra y la puso en pie como un
pilar. \bibverse{46} Luego le dijo a sus parientes: ``Vayan y recojan
algunas piedras''.Y todos\footnote{31.46 ``Todos'': incluyendo ambos
  grupos.}construyeron un pilar de piedras y se sentaron junto a él para
comer. \bibverse{47} Labán lo llamóYegar-Saduta, pero Jacob lo llamó
Galaad.+ 31.47 Ambos nombres significan ``pila de piedras'', el primero
es en arameo, el segundo es en hebreo.

\bibverse{48} Entonces Labán anunció: ``Este montón de piedras sirve de
testigo entre nosotros dos''.Por eso se le llamó Galeed. \bibverse{49}
También se le llamó Mizpa,\footnote{31.49 ``Mizpa'': significa ``torre
  de vigilancia''.}porque como dijo Labán: ``Que el Señor nos vigile de
cerca a los dos cuando no estemos juntos. \bibverse{50} Si tratas mal a
mis hijas, o te casas con otras esposas además de ellas, ¡Dios verá lo
que haces aunque nadie más se entere!''

\bibverse{51} Entonces Labán le dijo a Jacob: ``Mira este altar de
piedras que he construido en memoria del acuerdo \footnote{31.51 ``En
  memoria del acuerdo'': añadido para mayor claridad.}entre los dos.
\bibverse{52} También son testimonio de nuestras solemnes promesas
mutuas: No las pasaré por alto para atacarte, nitú las pasarás por alto
para atacarme. \bibverse{53} Que el Dios de Abraham y el Dios de Nacor,
el Dios de nuestros antepasados, sea el que juzgue entre nosotros
cualquier disputa''. Jacob, a su vez, hizo la solemne promesa en nombre
de maravilloso Dios de su padre Isaac.

\bibverse{54} Luego ofreció un sacrificio en la montaña e invitó a todos
sus parientes a comer allí. Pasaron la noche en la montaña \bibverse{55}
Labán se levantó temprano por la mañana y dio un beso de despedida a sus
nietos e hijas. Los bendijo y luego se fue para volver a casa.

\hypertarget{section-31}{%
\section{32}\label{section-31}}

\bibverse{1} Jacob siguió su camino y algunos ángeles de Dios vinieron a
su encuentro. \bibverse{2} Cuando los vio dijo: ``¡Este debe ser el
campamento de Dios!''Y llamó al lugar ``Dos campamentos''.

\bibverse{3} Entonces envió mensajeros a su hermano Esaú, que vivía en
la región de Seír, en el país de Edom. \bibverse{4} Y les dijo: ``Esto
es lo que deben decirle a mi señor Esaú:Tu siervo Jacob te envía este
mensaje. He estado con Labán hasta ahora, \bibverse{5} y tengo ganado,
asnos, ovejas y cabras, así como esclavos y esclavas. He enviado a estos
mensajeros para explicarte esto, mi señor, esperando que te alegresde
verme''.

\bibverse{6} Los mensajeros volvieron a Jacob y le dijeron: ``¡Su
hermano Esaú viene a recibirle con 400 hombres armados!'' \bibverse{7}
Cuando Jacob escuchó esto, estaba absolutamente aterrorizado. Dividió a
toda la gente con él, junto con las ovejas, las cabras, el ganado y los
camellos, en dos grupos, \bibverse{8} diciéndose a sí mismo: ``Si Esaú
viene y destruye un grupo, el otro puede escapar''.

\bibverse{9} Entonces Jacob oró: ``¡Dios de mi abuelo Abraham, Dios de
mi padre Isaac! Señor, tú fuiste quien me dijo: `Vuelve a tu país y a la
casa de tu familia, y te trataré bien' \bibverse{10} No merezco todo el
amor y la fidelidad que has mostrado a tu siervo. Crucé el Jordán hace
años\footnote{32.10 ``hace años'': añadido para mayor claridad.} con
sólo mi bastón, y ahora tengo dos grandes campamentos. \bibverse{11} Por
favor, sálvame de mi hermano; ¡defiéndeme de Esaú! Me aterra que venga a
atacarme a mí, a mis mujeres y a mis hijos. \bibverse{12} Tú mismo me
dijiste: `Sin duda alguna te trataré bien. Haré que tus descendientes
sean tan numerosos como la arena de la playa, demasiados para
contarlos.'\,''

\bibverse{13} Jacob pasó la noche allí. Luego escogió animales como
regalo para su hermano Esaú: \bibverse{14} 200 cabras hembras, 20 cabras
machos; 200 ovejas, 20 carneros; \bibverse{15} 30 camellos hembras con
sus crías, 40 vacas, 10 toros; 20 burros hembras, 10 burros machos.
\bibverse{16} Puso a sus sirvientes a cargo de cada uno de los rebaños y
les dijo: ``Adelántense y mantengan una buena distancia entre los
rebaños''.

\bibverse{17} A los que tenían el primer rebaño les dio estas
instrucciones: ``Cuando mi hermano Esaú se encuentre con ustedes y les
pregunte: `¿Quién es su amo,a dónde van, y de quién son estos animales
que vienen con ustedes' \bibverse{18} deberán decirle: `Tu siervo Jacob
envía estos como regalo a mi señor Esaú, y viene detrás de
nosotros'\,''. \bibverse{19} A los que tenían el segundo, el tercero, y
todos los rebaños subsiguientes lesdio las mismas instrucciones,
diciéndoles: ``Esto es lo que deben decirle a Esaú cuando se encuentre
con ustedes. \bibverse{20} Ytambién deben decirle: ``Tu siervo Jacob
viene justo detrás de nosotros'''.

\bibverse{21} Así que los regalos iban adelante mientras Jacob pasaba la
noche en el campamento.

\bibverse{22} Se levantó durante la noche y tomó a sus dos esposas y a
las dos criadas personales y a sus once hijos, y cruzó el río Jaboc.
\bibverse{23} Después de ayudarles a cruzar, también les envió todo lo
que les pertenecía. \bibverse{24} Pero Jacob se quedó allí solo. Un
hombre vino y luchó con él hasta el amanecer. \bibverse{25} Cuando el
hombre se dio cuenta de que no podía vencer a Jacob, golpeó la cavidad
de la cadera de Jacob y la desarticuló mientras luchaba con él.

\bibverse{26} Entonces el hombre dijo: ``Déjame ir porque ya casi ha
amanecido''.

``No te dejaré ir a menos que me bendigas'', respondió Jacob.

\bibverse{27} ``¿Cómo te llamas?'' le preguntó el hombre.

``Jacob'',respondió él.

\bibverse{28} ``Tu nombre no será más Jacob'', dijo el hombre. ``En su
lugar te llamarás Israel, porque luchaste con Dios y con los hombres, y
ganaste''.

\bibverse{29} ``Por favor, dime tu nombre'',preguntó Jacob.

``¿Por qué me preguntas mi nombre?'' respondió el hombre. Entonces
bendijo a Jacob allí.

\bibverse{30} Jacob nombró el lugar Peniel, diciendo: ``¡Vi a Dios cara
a cara y todavía estoy vivo!'' \bibverse{31} Y cuando Jacob se fue de
Peniel, ya salía el sol, e iba cojeando por su cadera fracturada.
\bibverse{32} (Por eso, aún hoy, los israelitas no se comen el tendón
del muslo que está unido a la cuenca de la cadera, porque ahí es donde
el hombre golpeó la cuenca de la cadera de Jacob).

\hypertarget{section-32}{%
\section{33}\label{section-32}}

\bibverse{1} Jacob vio a Esaú a lo lejos, viniendo hacia él con
cuatrocientos hombres. Así que dividió a los niños entre Lea, Raquel y
las dos sirvientas personales. \bibverse{2} Colocó a las dos sirvientas
personales con sus hijos primero, luego a Lea y sus hijos, y al final a
Raquel y José. \bibverse{3} Luego Jacob se adelantó a ellas y se inclinó
hasta el suelo siete veces antes de acercarse a su hermano. \bibverse{4}
Esaú corrió hacia él y lo abrazó. Puso sus brazos alrededor de su cuello
y lo besó. Los dos lloraron.

\bibverse{5} Entonces Esaú miró a su alrededor, a las mujeres y los
niños. ``¿Quiénes son estas personas que están contigo?'' preguntó.

``Son los hijos que Dios le dio a tu siervo'', respondió Jacob.

\bibverse{6} Las sirvientas personales y sus hijos se acercaron y se
inclinaron. \bibverse{7} Entonces Lía y sus hijos se acercaron y se
inclinaron. Por último, José y Raquel se acercaron y se inclinaron..

\bibverse{8} ``¿Para qué eran todos los animales que encontré en el
camino?'' Preguntó Esaú.

``Son un regalo para ti, mi señor, para que me trates bien'', respondió
Jacob. \bibverse{9} ````¡Tengo más que suficiente, hermano mío! Guarda
lo que tienes'', dijo Esaú.

\bibverse{10} ````¡No, por favor!'' Jacob insistió. ``Si eres feliz
conmigo, entonces por favor acepta el regalo que te estoy dando. Ahora
que he vuelto a ver tu rostro es como ver el rostro de Dios, ¡y me has
acogido tan amablemente! \bibverse{11} Por favor, acepta el regalo que
te he traído porque Dios me ha tratado muy bien y tengo mucho''. Así que
Esaú lo aceptó.

\bibverse{12} ````Sigamos nuestro camino'', dijo Esaú. ``Yo iré delante
de ti''.

\bibverse{13} ````Mi señor puede ver que los niños son débiles'',
respondió Jacob. ``También las cabras, las ovejas y el ganado están
amamantando a sus crías, y si los presiono demasiado, todos morirán.
\bibverse{14} Sigue adelante, mi señor, y tu siervo vendrá lentamente,
caminando con los niños, y me reuniré contigo en Seir''.

\bibverse{15} ````Bien, pero déjame dejar algunos de mis hombres
contigo'', dijo Esaú.

``Eres muy amable, pero no hay necesidad de hacer eso'', respondió
Jacob.

\bibverse{16} Así que Esaú comenzó su camino de regreso a Seír ese día.
\bibverse{17} Pero Jacob se dirigió a Sucot, donde se construyó una casa
y refugios para el ganado. Por eso el lugar se llama Sucot.\footnote{33.17
  ``Sucot'' significa ``refugios'' o ``establos''.}

\bibverse{18} Más tarde Jacob continuó su viaje desde Paddan-aram. Llegó
a salvo a Siquem en el país de Canaán donde acampó en las afueras del
pueblo. \bibverse{19} Compró el terreno donde acampaba a los hijos de
Hamor, el fundador de Siquem, por 100 monedas.\footnote{33.19
  ``Monedas'': literalmente, ``kesitah'', cuyo valor es desconocido.}
\bibverse{20} Construyó un altar allí y lo llamó El-Elohe-Israel.+ 33.20
``El-Elohe-Israel'': que significa ``Dios es el Dios de Israel''.

\hypertarget{section-33}{%
\section{34}\label{section-33}}

\bibverse{1} Dina, la hija de Jacob y Leah, fue a visitar a algunas de
las mujeres locales. \bibverse{2} Siquem, hijo de Hamor el heveo, el
gobernante de esa zona, la vio. La agarró y la violó. \bibverse{3} Sin
embargo, luego se enamoró profundamente de Dina y trató de que ella
también lo amara. \bibverse{4} Fue y le pidió a su padre, ``Trae a esta
joven para que me case con ella''.

\bibverse{5} Jacob descubrió que Siquem había violado\footnote{34.5
  ``Violado'': La palabra usada aquí está vinculada con ser impuro.} a
su hija Dina, pero como sus hijos estaban lejos cuidando los rebaños en
los campos no dijo nada hasta que volvieron a casa. \bibverse{6}
Mientras tanto, Hamor, el padre de Siquem, llegó para hablar con Jacob.
\bibverse{7} Cuando los hijos de Jacob regresaron de los campos se
molestaron mucho al oír la noticia y se enojaron mucho porque Siquem
había hecho algo indignante en Israel al tener relaciones sexuales con
la hija de Jacob, algo que nunca debería hacerse.

\bibverse{8} Hamor les dijo: ``Mi hijo Siquem está muy enamorado de su
hija y de su hermana Dinah.\footnote{34.8 ``su hermana Dina'': añadido
  para mayor claridad, pues Hamor se está dirigiendo tanto a Jacob como
  a los hijos de Jacob.} Please allow him to marry her. \bibverse{9} De
hecho, podemos tener más matrimonios. Pueden darnos a sus hijas y pueden
tener a nuestras hijas. \bibverse{10} Puedes vivir entre nosotros y
establecerte donde quieras. Podéis ir donde queráis y comprar tierras
para vosotros mismos''.

\bibverse{11} Entonces el propio Shechem habló y le dijo al padre y a
los hermanos de Dina: ``Por favor, acéptenme a mi y a mi propuesta, y
haré lo que me pidan. \bibverse{12} Puedes poner el precio de la novia
tan alto como quieras, y yo lo pagaré junto con todos los regalos que
daré. Sólo déjame tener a la chica para poder casarme con ella''.

\bibverse{13} Los hijos de Jacob no fueron honestos cuando le
contestaron a él y a su padre Hamor porque Siquem había violado a su
hermana Dina. \bibverse{14} Les dijeron: ``¡No podemos hacer esto! No
podemos permitir que nuestra hermana se case con un hombre que no está
circuncidado. Eso nos traería la desgracia. \bibverse{15} Sólo lo
aceptaremos con esta condición: todos ustedes deben ser circuncidados
como nosotros. \bibverse{16} Entonces os daremos nuestras hijas y
tomaremos vuestras hijas, y viviremos entre vosotros y nos convertiremos
en una familia. \bibverse{17} Pero si no estáis de acuerdo con nosotros
en que debéis circuncidaros, entonces tomaremos a nuestra hermana y nos
iremos''.

\bibverse{18} Hamor y su hijo Siquem estuvieron de acuerdo con lo que se
propuso. \bibverse{19} El joven Siquem no perdió tiempo en arreglar esto
porque estaba encaprichado con la hija de Jacob, y se le consideraba la
persona más importante de la familia de su padre. \bibverse{20} Hamor y
Siquem fueron a la puerta del pueblo y hablaron con los otros líderes
allí.

\bibverse{21} ``Estos hombres son nuestros amigos'', les dijeron.
``Hagamos que vivan aquí en nuestro país y permitámosles ir a donde
quieran, es lo suficientemente grande para todos ellos también. Podemos
tomar a sus hijas como esposas, y podemos darles nuestras hijas para que
se casen. \bibverse{22} Pero sólo aceptarán esto con una condición: sólo
se unirán a nosotros y se convertirán en una familia si cada hombre de
entre nosotros es circuncidado como ellos. \bibverse{23} Si eso ocurre,
¿no acabarán perteneciéndonos todo su ganado y sus propiedades, todos
sus animales? Sólo tenemos que estar de acuerdo con esto y ellos vendrán
a vivir entre nosotros''.

\bibverse{24} Todos los que estaban en la puerta del pueblo estaban de
acuerdo con Hamor y Siquem, así que todos los hombres del pueblo fueron
circuncidados. \bibverse{25} Tres días después, mientras aún sufrían
dolor, Simeón y Levi, dos de los hijos de Jacob y los hermanos de Dina,
llegaron con sus espadas a la ciudad. Sin oponerse, mataron a todos los
hombres. \bibverse{26} Mataron a Hamor y a Siquem con sus espadas,
tomaron a Dina de la casa de Siquem y se fueron.

\bibverse{27} Los otros hijos de Jacob llegaron, robaron los cadáveres y
saquearon la ciudad donde su hermana había sido violada. \bibverse{28}
Se llevaron sus ovejas, cabras, ganado y burros. Tomaron todo lo que
había en el pueblo y en los campos, \bibverse{29} es decir, todas sus
posesiones. Capturaron a todas sus mujeres y niños, y saquearon todo lo
que había en sus casas.

\bibverse{30} Pero Jacob criticó a Simeón y a Leví, diciéndoles: ``¡Me
habéis causado muchos problemas! Me has hecho como un mal olor entre la
gente de este país, entre los cananeos y los ferezeos. Sólo tengo unos
pocos hombres, y si esta gente se reúne para atacarme, yo y toda mi
familia seremos aniquilados''.

\bibverse{31} Pero ellos respondieron: ``¿Deberíamos haber dejado que
tratara a nuestra hermana como una prostituta?''

\hypertarget{section-34}{%
\section{35}\label{section-34}}

\bibverse{1} Entonces Dios le dijo a Jacob: ``Prepárate para ir a Betel
y vivir allí. Construye allí un altar a Dios, que se te apareció cuando
estabas huyendo de tu hermano Esaú''.\footnote{35.1 Ver 28:11 en
  adelante.} \bibverse{2} Entonces Jacob le dijo a su familia y a todos
los que estaban con él: ``Deshazte de los ídolos paganos que tienes
contigo. Purifíquense y cambien su ropa. \bibverse{3} Debemos
prepararnos e ir a Betel para construir un altar a Dios que me respondió
en mi tiempo de angustia. Él ha estado conmigo donde quiera que haya
ido''.

\bibverse{4} TEntregaron a Jacob todos los ídolos paganos que tenían,
así como sus pendientes,\footnote{35.4 ``Pendientes'': algunos
  comentaristas creen que estos pendientes también tenían alguna
  conexión religiosa.}y los enterró bajo el roble de Siquem.
\bibverse{5} Al partir en su viaje, el terror de Dios se extendió por
todos los pueblos de alrededor, así que nadie intentó tomar represalias
contra los hijos de Jacob.

\bibverse{6} Jacob y todos los que lo acompañaban llegaron a Luz
(también conocida como Bethel) en el país de Canaán. \bibverse{7}
HConstruyó un altar allí y llamó al lugar El-Bethel,\footnote{35.7
  ``El-Bethel'': que significa ``el Dios de Betel''. Betel a su vez
  significa ``la casa de Dios'' (ver 28:19).} porque allí se le había
aparecido Dios cuando huía de su hermano Esaú.

\bibverse{8} Deborah, la enfermera de Rebekah, murió y fue enterrada
bajo el roble cerca de Bethel. Así que se le llamó ``el roble del
llanto''.

\bibverse{9} Dios se le apareció de nuevo a Jacob y lo bendijo después
de su regreso de Paddan-aram. \bibverse{10} Dios le dijo: ``Jacob no
será más tu nombre. En lugar de Jacob tu nombre será Israel''. Así que
Dios le llamó Israel.

\bibverse{11} Entonces Dios dijo: ``¡Yo soy el Dios Todopoderoso!
Reproduce, aumenta, y te convertirás en una nación - de hecho un grupo
de naciones - y los reyes estarán entre tus descendientes. \bibverse{12}
Te daré a ti y a tus descendientes la tierra que también di a Abraham e
Isaac''. \bibverse{13} Entonces Dios dejó el lugar donde había estado
hablando con Jacob. \bibverse{14} Después Jacob puso un pilar de piedra
en el lugar donde Dios había hablado con él. Derramó una ofrenda de
bebida sobre ella, y también aceite de oliva. \bibverse{15} Jacob llamó
al lugar Betel, porque allí había hablado con Dios.

\bibverse{16} Luego se fueron de Betel. Cuando aún estaban a cierta
distancia de Efrat, Raquel se puso de parto y tuvo grandes dificultades
para dar a luz. \bibverse{17} Cuando tuvo los peores dolores de parto,
la comadrona le dijo: ``No te rindas, tienes otro hijo'' \bibverse{18}
Pero ella se estaba muriendo, y con su último aliento le puso el nombre
de Benoni.\footnote{35.18 ``Benoni'' significa ``hijo de mi
  sufrimiento''.}Pero su padre le puso el nombre de Benjamín.+ 35.18
``Benjamín'' significa ``hijo de mi mano derecha''. La mano derecha se
consideraba más favorable. \bibverse{19} Raquel murió y fue enterrada
camino de Efrat (también conocida como Belén). \bibverse{20} Jacob
colocó una piedra conmemorativa sobre la tumba de Raquel, y sigue ahí
hasta hoy.

\bibverse{21} Israel\footnote{35.21 ``Israel'': refiriéndose por
  supuesto a Jacob después de su cambio de nombre.} Israel siguió
adelante y acampó más allá de la torre de vigilancia en Eder.
\bibverse{22} Durante el tiempo que vivió allí, Rubén fue y se acostó
con Bilhá, la concubina de su padre, e Israel se enteró de ello.+ 35.22
La Septuaginta añade: ``y fue muy angustioso para él''.

Estos fueron los doce hijos de Jacob:

\bibverse{23} Los hijos de Lea: Rubén (el primogénito de Jacob), Simeón,
Leví, Judá, Isacar, y Zabulón.

\bibverse{24} Los hijos de Raquel: Joséy Benjamín.

\bibverse{25} Los hijos de Bila, la criada personal de Raquel: Dan y
Neftalí.

\bibverse{26} Los hijos de Zilpá, la criada personal de Lea: Gad y Aser.

Estos fueron los hijos de Jacob, que nacieron cuando vivía en
Padán-Arán.

\bibverse{27} Jacob regresó a casa de su padre Isaac en Mamre, cerca de
Kiriath-arba (también conocida como Hebrón), donde habían vivido Abraham
e Isaac. \bibverse{28} Isaac vivió hasta la edad de 180 años,
\bibverse{29} cuando respiró por última vez y murió a una edad avanzada.
Había vivido una vida plena y ahora se unió a sus antepasados en la
muerte. Sus hijos Esaú y Jacob lo enterraron.

\hypertarget{section-35}{%
\section{36}\label{section-35}}

\bibverse{1} La siguiente es la genealogía de Esaú (también llamado
Edom). \bibverse{2} Esaú se casó con dos mujeres cananeas: Ada, hija de
Elón el hitita, yAholibama, hija de Aná, y nieta de Zibeón el heveo.
\bibverse{3} Además se casó también con Basemat, hija de Ismael, y
hermana de Nebayot.

\bibverse{4} Adah tuvo un hijo para Esaú llamado Elifaz. Basemath tuvo a
Reuel. \bibverse{5} Aholibama tuvo a Jeús, Jalán y Coré. Estos fueron
los hijos de Esaú, que le nacieron en Canaán.

\bibverse{6} Esaú tomó a sus esposas, hijos e hijas, y a todos los de su
casa, junto con su ganado, todos sus otros animales y todas las
posesiones que había ganado mientras estaba en Canaán, y se fue a vivir
a un país lejano de su hermano Jacob. \bibverse{7} Lo hizo porque la
tierra en la que vivían no podía mantenerlos a ambos con todo su ganado.
\bibverse{8} Esaú se estableció en la región montañosa de Seír.

\bibverse{9} La siguiente es la genealogía de Esaú, padre de los
edomitas, que vivía en las colinas de Seír:

\bibverse{10} Estos eran los nombres de los hijos de Esaú: Elifaz, hijo
de la esposa de Esaú, Ada, y Reuel, hijo de la esposa de Esaú, Basemath.
\bibverse{11} Los hijos de Elifaz eran: Teman, Omar, Zefo, Gatam y
Cenaz. \bibverse{12} Timna, la concubina del hijo de Esaú, Elifaz, tenía
a Amalec como hijo de Elifaz. Estos eran los descendientes de la esposa
de Esaú, Ada.

\bibverse{13} Estos fueron los hijos de Reuel: Najat, Zera, Sama y Mizá.
Eran los descendientes de Basemat, la esposa de Esaú.

\bibverse{14} Estos fueron los hijos de la esposa de Esaú, Aholibama,
hija de Aná y nieta de Zibeón, a quien tuvo con Esaú: Jeús, Jalam y
Coré.

\bibverse{15} Estos fueron los jefes de las tribus de los hijos de Esaú.
Los jefes de las tribus de los hijos de Elifaz (el primogénito de Esaú)
eran Temán, Omar, Zefo, Quenaz, \bibverse{16} Coré,\footnote{36.16
  ``Coré'': de la manera que aparece aquí se considera a menudo que fue
  un error del copista, ya que aparece como un hijo de Esaú en el
  versículo 14.} Gatán y Amalec. Fueron los jefes de las tribus de
Elifaz en el país de Edom, y eran los descendientes de Ada.

\bibverse{17} Estos fueron los hijos del hijo de Esaú, Reuel: los
líderes de las tribus Najat, Zera, Sama y Mizá. Fueron los jefes de las
tribus descendientes de Reuel en el país de Edom, y eran los
descendientes de la esposa de Esaú, Basemath.

\bibverse{18} Estos fueron los hijos de la esposa de Esaú Aholibama: los
jefes de las tribus Jeús, Jalán y Coré; fueron los jefes de las tribus
descendientes de la esposa de Esaú Aholibama, hija de Aná. \bibverse{19}
Todos ellos eran hijos de Esaú (también llamado Edom), y fueronlos jefes
de sus tribus.

\bibverse{20} Estos fueron los hijos de Seír el horeo, que vivían en el
país: Lotán, Sobal, Zibeón, Aná, \bibverse{21} Disón, Ezer y Disán; eran
los jefes de la tribu de los horeos, los descendientes de Seír en la
tierra de Edom.

\bibverse{22} Los hijos de Lotán fueron Hori y Hemam. Timna era la
hermana de Lotan.

\bibverse{23} Estos fueron los hijos de Sobal: Alvánn, Manajat, Ebal,
Sefó y Onam.

\bibverse{24} Estos fueron los hijos de Zibeón: Ayá y Aná. (Este fue el
mismo Aná que descubrió las fuentes termales\footnote{36.24 ``fuentes
  termales'': el significado de este versículo en el hebreo es incierto.}
en el desierto mientras cuidaba los asnos de su padre Zibeón).

\bibverse{25} Estos fueron los hijos de Aná: Disón y Aholibamah, hija de
Aná.

\bibverse{26} Estos fueron los hijos de Disón: Hemdán, Esbán, Itrán y
Querán.

\bibverse{27} Estos fueron los hijos de Ezer: Bilán, Zaván y Acán.

\bibverse{28} Estos fueron los hijos de Disán: Uz y Arán.

\bibverse{29} Estos fueron los jefes de las tribus de los horeos: Lotán,
Sobal, Zibeón, Anaá \bibverse{30} Disón, Ezer y Disán. Eran los jefes de
las tribus de los horeos, listados según sus tribus en el país de Seír.

\bibverse{31} Estos fueron los reyes que gobernaban en la tierra de Edom
antes de que hubiera un rey que gobernara sobre los israelitas:

\bibverse{32} Bela, hijo de Beor, gobernaba en Edom y el nombre de su
ciudad era Dinaba.

\bibverse{33} Cuando murió Bela, Jobab, hijo de Zera de Bosra, asumió el
cargo de rey.

\bibverse{34} Cuando murió Jobab, Jusán, de la tierra de los temanitas,
asumió el cargo de rey.

\bibverse{35} Cuando murió Husam, Hadad, hijo de Bedad, asumió el cargo
de rey. Fue él quien derrotó a los madianitas en el país de Moab, y el
nombre de su ciudad era Avit.

\bibverse{36} Cuando murió Hadad, Samla de Masreca se hizo cargo de la
corona.

\bibverse{37} Cuando Samla murió, Saúl de Rejobot en el Éufrates se hizo
cargo como rey.

\bibverse{38} Cuando Saúl murió, Baal Janán, hijo de Acbor, se hizo
cargo como rey.

\bibverse{39} Cuando Baal Janán, hijo de Acbor, murió, Hadad se hizo
cargo como rey. El nombre de su pueblo era Pau, y el nombre de su esposa
era Mehitabel. Yera hija de Matred, hija de Mezab.

\bibverse{40} Estos fueron los nombres de los jefes de las tribus
descendientes de Esaú, según sus familias y el lugar donde vivían,
enumerados por nombre: los jefes de las tribus Timná, Alvá, Jetet,
\bibverse{41} Aholibamah, Elá, Pinón, \bibverse{42} Quenaz, Temán,
Mibzar, \bibverse{43} Magdiel e Iram. Estos fueron los jefes de las
tribus de Edom, listados según los lugares donde vivían en el país. Esaú
fue el antepasado de los edomitas.

\hypertarget{section-36}{%
\section{37}\label{section-36}}

\bibverse{1} Jacob se estableció y vivió en Canaán como lo había hecho
su padre.

\bibverse{2} Esta es la historia de Jacob y su familia: José tenía
diecisiete años y ayudaba a cuidar el rebaño junto con sus hermanos, los
hijos de Bila y Zilpá, las esposas de su padre. José le contó a su padre
algunas de las cosas malas que sus hermanos estaban haciendo.

\bibverse{3} Israel\footnote{37.3 ``Israel'', es decir, Jacob.}amaba a
José más que a cualquiera de sus otros hijos, porque José le había
nacido cuando ya era viejo. E hizo una túnica de colores y de mangas
largas para José. \bibverse{4} Cuando sus hermanos se dieron cuenta de
que su padre lo amaba más que a cualquiera de ellos, lo odiaron y no
tenían nada bueno que decir de él.

\bibverse{5} José tuvo un sueño, y cuando se lo contó a sus hermanos, lo
odiaron aún más. \bibverse{6} ``Escuchen este sueño que tuve'',les dijo.
\bibverse{7} ``Estábamos atando fardos de grano en los campos cuando de
repente mi fardo se levantó, y sus fardos se acercaron y se inclinaron
ante el mío''.

\bibverse{8} ``¿De verdad crees que vas a ser nuestro rey?'' le
preguntaron ellos. ``¿De verdad crees que vas a gobernar sobre
nosotros?'' Y lo odiaron aún más por su sueño y por cómo lo describía.

\bibverse{9} Luego José tuvo otro sueño y se lo contó a sus hermanos.
``Escuchen, tuve otro sueño'', explicó. ``El sol y la luna y once
estrellas se inclinaban ante mí''.

\bibverse{10} También se lo contó a su padre y a sus hermanos, y su
padre se lo recriminó, diciendo: ``¿Qué es este sueño que has tenido?
¿Vamos a venir nosotros, tu madre y tus hermanos a inclinarnos hasta el
suelo ante ti?''

\bibverse{11} Los hermanos de José se pusieron celosos de él, pero su
padre no entendía el significado del sueño.

\bibverse{12} Un día los hermanos de José llevaban los rebaños de su
padre a pastar cerca de Siquem.

\bibverse{13} Israel le dijo a José: ``Tus hermanos cuidan las ovejas
cerca de Siquem. Prepárate porque quiero que vayas a verlos''.

``Así lo haré'', respondió José.

\bibverse{14} Así que Jacob le dijo: ``Ve a ver cómo están tus hermanos
y los rebaños, y vuelve y házmelo saber''. Así que lo despidió, y José
partió del Valle de Hebrón, \bibverse{15} y llegó a Siquem. Un hombre lo
encontró vagando por el campo, y le preguntó: ``¿Qué buscas?''
\bibverse{16} ``Estoy buscando a mis hermanos'', respondió José.
``¿Puedes decirme por favor dónde están cuidando el rebaño?''

\bibverse{17} ``Ya se han ido'', respondió el hombre. ``Les oí decir:
`Vamos a Dotán'\,''. Así que José siguió a sus hermanos y los alcanzó en
Dotán.

\bibverse{18} Pero lo vieron venir a lo lejos, y antes de que llegara a
ellos, hicieron planes para matarlo. \bibverse{19} ``¡Mira, aquí viene
el Señor de los Sueños!'' se dijeron entre ellos. \bibverse{20} ``Vamos,
matémoslo y arrojémoslo a una de las fosas. Diremos que algún animal
salvaje se lo ha comido. ¡Entonces veremos qué pasa con sus sueños!''

\bibverse{21} Cuando Rubén escuchó todo esto, trató de salvar a José de
ellos. \bibverse{22} ``No lo ataquemos ni lo matemos'', sugirió. ``No lo
asesinen, sólo arrójenlo a esta fosa aquí en el desierto. No necesitamos
ser culpables de violencia''.\footnote{37.22 ``No necesitamos ser
  culpables de violencia'': literalmente ``no debemos poner una mano
  contra él''. Rubén está sugiriendo que no tienen que matar activamente
  a José, pero si lo arrojan a una fosa morirá sin que sean culpables de
  cometer un asesinato.} Rubén dijo esto para poder regresar más tarde y
rescatar a José de ellos y llevarlo a casa con su padre.

\bibverse{23} Así que cuando llegó José, sus hermanos le arrancaron la
túnica -- la colorida túnica de manga larga que llevaba puesta --
\bibverse{24} lo agarraron y lo arrojaron a una fosa. (La fosa estaba
vacía y no tenía agua). \bibverse{25} Estaban sentados para comer cuando
vieron una caravana de ismaelitas que venía de Galaad. Sus camellos
llevaban especias aromáticas, bálsamo y mirra para llevarlos a Egipto.

\bibverse{26} ``¿Qué sentido tiene matar a nuestro hermano?'' preguntó
Judá a sus hermanos. ``¡Entonces tendríamos que encubrir su muerte!
\bibverse{27} En vez de eso, ¿por qué no lo vendemos a estos ismaelitas?
No tenemos que matarlo. Después de todo, es nuestro hermano, nuestra
propia carne y sangre''. Sus hermanos estuvieron de acuerdo.

\bibverse{28} Así que cuando los ismaelitas (que eran comerciantes de
Madián)\footnote{37.28 En ocasiones el texto se refiere a ellos como
  ismaelitas, y a veces como madianitas, pero claramente son el mismo
  grupo. Ver también el versículo 36.} llegaron, sacaron a José de la
fosa y se lo vendieron por veinte piezas de plata. Los ismaelitas lo
llevaron a Egipto.

\bibverse{29} Cuando Rubén regresó más tarde y miró en la fosa, José se
había ido. Rasgó sus ropas en señal de dolor. \bibverse{30} Regresó con
sus hermanos. ``¡El muchacho se ha ido!'', gimió. ``¿Qué voy a hacer
ahora?''

\bibverse{31} Mataron una cabra y mojaron la túnica de José en la
sangre. \bibverse{32} Luego enviaron la colorida túnica a su padre con
el mensaje: ``Encontramos esto. Por favor, examínalo y ve si es la
túnica de tu hijo o no''.

\bibverse{33} El padre la reconoció de inmediato y dijo: ``¡Esta es la
túnica de mi hijo! Algún animal salvaje debe habérselo comido. El pobre
José ha sido despedazado, ¡no hay duda de ello!''

\bibverse{34} Entonces Jacob rasgó sus ropas en señal de lamento y se
vistió con un saco. Lloró la muerte de su hijo durante mucho tiempo.
\bibverse{35} Todos sus hijos e hijas trataron de consolarlo, pero él
rechazaba sus intentos. ``No'', dijo, ``bajaré a mi tumba llorando por
mi hijo''. Así que el padre de José siguió llorando por él.

\bibverse{36} Mientras tanto, los ismaelitas habían llegado a Egipto y
le habían vendido a José a Potifar. Potifar era uno de los oficiales del
faraón, era el capitán de la guardia.

\hypertarget{section-37}{%
\section{38}\label{section-37}}

\bibverse{1} Por esta época, Judá dejó a sus hermanos y montó su
campamentoen Adulán, cerca de un hombre local llamado Hirá. \bibverse{2}
Allí Judá vio por casualidad a la hija de un cananeo llamado Súa y se
casó con ella. Se acostó con ella, \bibverse{3} y ella quedó embarazada
y tuvo un hijo, al que llamó Er. \bibverse{4} Luego ella quedó
embarazada de nuevo y tuvo un hijo que llamó Onán. \bibverse{5} Luego
tuvo otro hijo llamado Selá que nació en Quezib.

\bibverse{6} Mucho más tarde, Judá hizo que Er, su primogénito, se
casara con una mujer llamada Tamar. \bibverse{7} Pero Er hizo lo que era
malo a los ojos del Señor, así que el Señor le dio muerte. \bibverse{8}
Judá le dijo a Onán, ``Ve y acuéstate con la mujer de tu hermano para
cumplir los requisitos de un cuñado para tener hijos en nombre de tu
hermano''``.

\bibverse{9} Onán se dio cuenta de que los hijos que tuviera no serían
suyos, así que siempre que se acostaba con la mujer de su hermano se
aseguraba de que no se quedara embarazada retirando y derramando su
semen en el suelo. De esta manera evitaba que nacieran niños en nombre
de su hermano. \bibverse{10} Pero lo que hizo fue malo a los ojos del
Señor, así que también mató a Onán.

\bibverse{11} Entonces Judá le dijo a su nuera Tamar, ``Ve a la casa de
tu padre y vive allí como una viuda hasta que mi hijo Selá crezca''.
Porque pensó: ``Quizá él también muera, como sus hermanos''. Así que
Tamar se fue y se quedó en la casa de su padre.

\bibverse{12} Mucho tiempo después murió la esposa de Judá, la hija de
Súa. Cuando Judá terminó el tiempo de luto, fue a visitar a sus
esquiladores en Timná con su amigo Hirá de Adulán. \bibverse{13} A Tamar
le dijeron: ``Tu suegro va a Timná a esquilar sus ovejas'' \bibverse{14}
Así que se quitó la ropa de viuda y se cubrió con un velo,
disfrazándose. Se sentó junto a la entrada de Enayin, que está en el
camino a Timná. Se había dado cuenta de que aunque Selá había crecido,
no se había hecho nada para que se casara con él.

\bibverse{15} Judá la vio y pensó que debía ser una prostituta porque se
había cubierto la cara. \bibverse{16} Se acercó a ella a un lado de la
carretera y le dijo: ``Quiero acostarme contigo''. Pero no sabía que era
su nuera.

``¿Qué me darás si te dejo dormir conmigo?'' preguntó ella.

\bibverse{17} ``Te enviaré una cabra joven de mi rebaño'', respondió él.

``¿Qué garantía me darás para asegurarme de que la enviarás?'' preguntó
ella.

\bibverse{18} ``¿Qué garantía tengo que darte?'' preguntó él.

``Tu sello de sello y su cordón, y tu bastón que sostienes'', respondió
ella. Él se los entregó. Se acostó con ella y quedó embarazada.
\bibverse{19} Se fuea casa, se quitó el velo y se puso la ropa de viuda.

\bibverse{20} Judá envió a su amigo Hirá de Adulán con una cabra joven
para que le devolviera sus pertenencias que había dejado como garantía
de la mujer, pero no pudo encontrarla. \bibverse{21} Hirále preguntó a
los hombres de allí, ``¿Dónde está la prostituta de culto que se sienta
en el camino de entrada a Enayin?''

``Aquí no hay ninguna prostituta de culto'',respondieron.

\bibverse{22} Hirá regresó a Judá y le dijo: ``No pude encontrarla, y
los hombres de allí dijeron: `Aquí no hay ninguna prostituta de
culto.'\,''

\bibverse{23} ``Que se quede con lo que le di'', respondió Judá.
``Quedaremos en ridículo si seguimos buscando. En cualquier caso,
intenté enviarle la cabra joven como prometí, pero no la encontraron''.

\bibverse{24} Unos tres meses después le dijeron a Judá: ``Tamar, tu
nuera ha tenido relaciones sexuales como una prostituta y ahora está
embarazada''.

``¡Sáquenla y quémenla hasta la muerte!''ordenó Judá.

\bibverse{25} Cuando la sacaron, envió un mensaje a su suegro, diciendo:
``Estoy embarazada del hombre que posee estas cosas''. Luego añadió:
``Por favor, miren con atención este sello, el cordón del sello, y este
bastón. ¿A quién le pertenecen?''

\bibverse{26} Judá los reconoció de inmediato y dijo: ``Ella ha honrado
la ley más que yo, porque no la entregué en matrimonio a mi hijo Selá''.
Y no volvió a acostarse con Tamar.

\bibverse{27} Cuando llegó el momento en que Tamar debía dara luz, se
descubrió que llevaba gemelos. \bibverse{28} Un bebé extendió su mano, y
la comadrona le ató un hilo escarlata en su muñeca y dijo: ``Este salió
primero'' \bibverse{29} Pero entonces él retiró su mano y su hermano
nació primero, y ella dijo, ``¿cómo saliste tú?'' Así que lo
llamaronFares.\footnote{38.29 ``Fares'' significa ``irrumpir''.}
\bibverse{30} Después nació su hermano con el hilo escarlata en la
muñeca, y lo llamaron Zera.+ 38.30 ``Zera'' significa ``naciente'' (como
en ``sol'') con la implicación quizás del color rojo.

\hypertarget{section-38}{%
\section{39}\label{section-38}}

\bibverse{1} José había sido llevado a Egipto por los ismaelitas,
quienes lo habían vendido a Potifar, un egipcio que era uno de los
oficiales del faraón, el comandante de la guardia real.

\bibverse{2} El Señor estaba con José y lo hacía triunfar. Vivía en la
casa de su amo egipcio. \bibverse{3} Su maestro se dio cuenta de que el
Señor estaba con él y le otorgaba éxito en todo lo que hacía.
\bibverse{4} Potifar apreció a José y su servicio, y lo puso a cargo de
su casa y lo hizo responsable de todo lo que poseía. \bibverse{5} Desde
que puso a José a cargo y le confió todo lo que tenía, el Señor bendijo
la casa de Potifar por causa de José. El Señor bendecía todo lo que
tenía, tanto en su casa como en sus campos. \bibverse{6} Así que Potifar
dejó que José cuidara de todo lo que tenía. No se preocupaba de nada,
excepto de decidir qué comida iba a comer.

Ahora José era guapo, tenía un buen físico, \bibverse{7} y algún tiempo
después llamó la atención de la esposa de su amo. Ella le propuso
matrimonio, diciendo: ``¡Ven aquí! ¡Acuéstate conmigo!''

\bibverse{8} Pero él la rechazó, diciéndole a la esposa de su amo:
``Mira, mi amo confía tanto en mí,\footnote{39.8 ``Confía tanto en mi'':
  añadido para mayor claridad.}que ni siquiera se molesta en averiguar
cómo funciona su casa. Me ha puesto a cargo de todo lo que posee.
\bibverse{9} ¡Nadie en esta casa tiene más autoridad que yo! No me ha
ocultado nada excepto a ti, porque eres su esposa. Entonces, ¿cómo
podría hacer algo tan malo como esto, y pecar contra Dios?''

\bibverse{10} Día tras día ella insistía en preguntarle, pero él se
negaba a dormir con ella y trataba de evitarla. \bibverse{11} Pero un
día entró en la casa para hacer su trabajo y ninguno de los otros
sirvientes estaba allí. \bibverse{12} Ella lo agarró por la
ropa,\footnote{39.12 ``Ropa'': la palabra hebrea es una palabra general
  para la ropa y no es específica. Sin embargo, el arte egipcio antiguo
  muestra a los sirvientes usando sólo taparrabos, y es probable que
  esto sea lo que José estaba usando en ese momento. También encaja con
  la historia el hecho de que este trozo de tela podría haber sido
  fácilmente arrancado de su cuerpo. Sin embargo, ya que no se menciona
  nada específico, aquí usamos el término general.}y le exigió:
``¡Acuéstate conmigo!'' Pero dejando su ropa en su mano, salió corriendo
de la casa.

\bibverse{13} Viendo que había dejado su ropa en su mano y había salido
corriendo de la casa, \bibverse{14} ella gritó a sus sirvientes:
``¡Miren! ¡Él\footnote{39.14 Refiriéndose a su marido. Es interesante
  que ella simplemente se refiera a su marido como ``él'', demostrando
  su falta de respeto hacia él, también evidenciada por su voluntad de
  cometer adulterio.}trajo a este esclavo hebreo aquí para deshonrarnos!
Este hombre vino a tratar de violarme, pero yo grité con todas mis
fuerzas. \bibverse{15} Cuando me oyó gritar pidiendo ayuda, dejó su ropa
a mi lado y salió corriendo''

\bibverse{16} Y guardó la ropa de José hasta que su marido volvió a
casa. \bibverse{17} Luego le contó su historia. Y esto fue lo que le
dijo: ``Ese esclavo hebreo que trajiste aquí trató de venir a
deshonrarme. \bibverse{18} Pero tan pronto como grité y pedí ayuda, dejó
su ropa a mi lado y salió corriendo''.

\bibverse{19} Cuando Potifar oyó la historia que le contó su mujer,
diciendo: ``Esto es lo que tu siervo me hizo'', se enfadó. \bibverse{20}
Tomó a José y lo metió en la prisión donde estaban los prisioneros del
rey, y allí se quedó. \bibverse{21} Pero el Señor estaba con José,
mostrándole un amor digno de confianza, e hizo que el carcelero
principal se complaciera con él. \bibverse{22} El carcelero principal
puso a José a cargo de todos los prisioneros y le dio la responsabilidad
de dirigir la prisión. \bibverse{23} El carcelero principal no se
preocupaba de nada porque José se ocupaba de todo, pues el Señor estaba
con él y le daba éxito.

\hypertarget{section-39}{%
\section{40}\label{section-39}}

\bibverse{1} Más tarde, el copero y el panadero cometieron alguna ofensa
contra su amo, el rey de Egipto. \bibverse{2} El Faraón se enojó con
estos dos oficiales reales -- el copero y el panadero principal---
\bibverse{3} y los encarceló en la casa del comandante de la guardia, la
misma prisión donde estaba José. \bibverse{4} El comandante de la
guardia les asignó a José como su asistente personal. Fueron mantenidos
en prisión por algún tiempo.

\bibverse{5} Una noche, mientras estaban en la cárcel, el copero y el
panadero del rey de Egipto tuvieron un sueño. Eran sueños diferentes,
cada uno con su propio significado. \bibverse{6} Cuando José llegó a la
mañana siguiente notó que ambos parecían deprimidos. \bibverse{7} Así
que le preguntó a los oficiales del faraón que estaban presos con él en
la casa de su amo, ``¿Por qué te ves tan deprimido?''

\bibverse{8} ``Los dos hemos tenido sueños, pero no encontramos a nadie
que nos explique lo que significan'', dijeron.

Así que José les dijo, ``¿No es Dios el que puede interpretar el
significado de los sueños? Cuéntame tus sueños''

\bibverse{9} El copero principal le contó a José su sueño. ``En mi sueño
había una vid justo delante de mí'', explicó. \bibverse{10} ``La vid
tenía tres ramas. Tan pronto como brotaba, florecía y producía racimos
de uvas maduras. \bibverse{11} Yo sostenía la copa de vino del Faraón,
así que recogí las uvas y las metí en la copa y se la di al Faraón''.

\bibverse{12} ``Este es el significado'', le dijo José. ``Las tres ramas
representan tres días. \bibverse{13} Dentro de tres días el Faraón te
sacará de la cárcel y te devolverá tu trabajo, y tú le entregarás al
Faraón su copa como solías hacerlo. \bibverse{14} Pero cuando las cosas
te vayan bien, por favor, acuérdate de mi y habla con el Faraón en mi
nombre, y por favor sácame de esta prisión. \bibverse{15} Fui cruelmente
secuestrado en la tierra de los hebreos, y ahora estoy aquí en esta fosa
a pesar de que no he hecho nada malo''.

\bibverse{16} Cuando el jefe de los panaderos vio que la interpretación
era positiva, le dijo a José: ``Yo también tuve un sueño. Tenía tres
cestas de pasteles en mi cabeza. \bibverse{17} En la cesta de arriba
estaban todos los pasteles y pastas para que los comiera el Faraón, y
los pájaros se los comían de la cesta de mi cabeza''.

\bibverse{18} ``Este es el significado'', le dijo José. ``Las tres
cestas representan tres días. \bibverse{19} Dentro de tres días el
Faraón te sacará de la cárcel y te colgará en un palo, y los pájaros se
comerán tu carne''.

\bibverse{20} Tres días más tarde era el cumpleaños del Faraón, y
organizó un banquete para todos sus oficiales. Hizo que el copero y el
panadero jefe fueran liberados de la prisión y llevados allí ante sus
oficiales. \bibverse{21} Le devolvió el trabajo al copero jefe y volvió
a sus deberes de entregar al Faraón su copa. \bibverse{22} Pero colgó al
jefe de los panaderos tal como José había dicho cuando interpretó sus
sueños. \bibverse{23} Pero el copero jefe no se acordó de decir nada
sobre José;de hecho, se olvidó de él.

\hypertarget{section-40}{%
\section{41}\label{section-40}}

\bibverse{1} Dos años más tarde, el Faraón soñó que estaba de pie junto
al río Nilo. \bibverse{2} Vio siete vacas que subían del río. Parecían
bien alimentadas y sanas mientras pastaban entre los juncos.
\bibverse{3} Luego vio otras siete vacas que subían por detrás de ellas.
Se veían feas y flacas mientras estaban junto a las otras vacas en la
orilla del Nilo. \bibverse{4} Luego las vacas feas y flacas se comieron
a las vacas bien alimentadas y sanas. Entonces el Faraón se despertó.

\bibverse{5} El Faraón se durmió de nuevo y tuvo un segundo sueño. En un
tallo crecían siete cabezas de grano, maduras y sanas. \bibverse{6}
Entonces siete cabezas de grano crecieron después de ellas, delgadas y
secas por el viento del este. \bibverse{7} Las siete cabezas de grano
delgadas y secas se tragaron las maduras y sanas. Entonces el Faraón se
despertó y se dio cuenta de que había estado soñando.

\bibverse{8} A la mañana siguiente el Faraón estaba preocupado por sus
sueños,\footnote{41.8 ``Por sus sueños'': añadido para mayor claridad.}así
que mandó llamar a todos los magos y sabios de Egipto. El Faraón les
habló de sus sueños, pero nadie pudo interpretar su significado para él.

\bibverse{9} Pero entonces el copero principal habló. ``Hoy acabo de
recordar un grave error que he cometido'', explicó. \bibverse{10} ``Su
Majestad se enfadó con algunos de sus oficiales y me encarceló en la
casa del comandante de la guardia, junto con el panadero jefe.
\bibverse{11} Cada uno de nosotros tuvo un sueño. Eran sueños
diferentes, cada uno con su propio significado. \bibverse{12} Un joven
hebreo estaba allí con nosotros, un esclavo del comandante de la
guardia. Cuando le contamos nuestros sueños, nos interpretó el
significado de nuestros diferentes sueños. \bibverse{13} Todo sucedió
tal y como él dijo que sucedería, me devolvieron mi trabajo y colgaron
al panadero''.

\bibverse{14} El Faraón convocó a José, y rápidamente lo trajeron de la
prisión. Después de que se afeitara y se cambiara de ropa, fue
presentado al Faraón.

\bibverse{15} El Faraón le dijo a José: ``Tuve un sueño, pero nadie
puede interpretar su significado. Pero he oído que cuando alguien te
cuenta un sueño sabes cómo interpretarlo''.

\bibverse{16} ``No soy yo quien puede hacer esto'', respondió José.
``Pero Dios explicará su significado para tranquilizar la mente de Su
Majestad''

\bibverse{17} El Faraón le explicó a José, ``En mi sueño estaba parado a
la orilla del Nilo. \bibverse{18} Vi siete vacas que subían del río.
Parecían bien alimentadas y sanas mientras pastaban entre los juncos.
\bibverse{19} Luego vi otras siete vacas que subían por detrás de ellas.
Se veían enfermas, feas y flacas. ¡Nunca había visto vacas tan feas en
todo Egipto! \bibverse{20} Estas vacas flacas y feas se comieron las
primeras siete vacas de aspecto saludable. \bibverse{21} Pero después no
se podía saber que se las habían comido porque se veían tan flacas y
feas como antes. Entonces me desperté.

\bibverse{22} Luego me volví a dormir. En mi segundo sueño vi siete
cabezas de grano creciendo en un tallo, maduras y sanas. \bibverse{23}
Después de ellos crecieron siete cabezas de grano, marchitas y delgadas
y secas por el viento del este. \bibverse{24} Las siete delgadas cabezas
de grano se tragaron las sanas. Les dije todo esto a los magos, pero
ninguno de ellos pudo explicarme su significado''

\bibverse{25} ``Los sueños del faraón significan lo mismo'', respondió
José. \bibverse{26} Las siete buenas vacas y las siete buenas cabezas de
grano representan siete buenos años de cosecha.\footnote{41.26 ``De
  cosecha'': añadido para mayor claridad.}Los sueños significan lo
mismo. \bibverse{27} Las siete vacas flacas y feas que vinieron después
de ellas y las siete finas cabezas de grano secadas por el viento del
este representan siete años de hambruna. \bibverse{28} Es tal como le
dije a Su Majestad: Dios ha mostrado al Faraón lo que va a hacer.
\bibverse{29} Van a ser siete años con mucha comida producida en todo el
país de Egipto. \bibverse{30} Pero después de ellos vendrán siete años
de hambruna. La gente olvidará la época en que había mucha comida en
todo Egipto. La hambruna arruinará el país. \bibverse{31} El tiempo de
abundancia se olvidará por completo porque la hambruna que le sigue será
terrible. \bibverse{32} El hecho de que el sueño se repitiera dos veces
significa que definitivamente ha sido decidido por Dios, y que Dios lo
hará pronto.

\bibverse{33} Así que Su Majestad debería elegir un hombre con
perspicacia y sabiduría, y ponerlo a cargo de todo el país de Egipto.
\bibverse{34} Su Majestad también debe nombrar funcionarios para que
estén a cargo de la tierra, y hacer que recojan una quinta parte del
producto del país durante los siete años de abundancia. \bibverse{35}
Deben recoger todos los alimentos durante los años buenos que se
avecinan, y almacenar el grano bajo la autoridad del Faraón,
manteniéndolo bajo vigilancia para proporcionar alimentos a los pueblos.
\bibverse{36} Esto será una reserva de alimentos para el país durante
los siete años de hambruna para que la gente no muera de hambre''.

\bibverse{37} El Faraón y todos sus oficiales pensaron que la propuesta
de José era una buena idea. \bibverse{38} Así que el Faraón les
preguntó: ``¿Dónde podemos encontrar a un hombre como este que tiene el
espíritu de Dios en él?'' \bibverse{39} Entonces el Faraón habló con
José, diciéndole: ``Puesto que Dios te ha revelado todo esto, y no hay
nadie como tú con tanta perspicacia y sabiduría, \bibverse{40} tú
estarás a cargo de todos mis asuntos, y todo mi pueblo obedecerá tus
órdenes. Sólo yo, con mi condición de rey \footnote{41.40 ``Mi condición
  de rey'': literalmente ``el trono''.}seré más grande que tú''.

\bibverse{41} Entonces el Faraón le dijo a José: ``Mira, te pongo a
cargo de todo el país de Egipto''. \bibverse{42} El Faraón se quitó el
anillo del sello del dedo y lo puso en el dedo de José. Lo vistió con
ropas de lino fino y le puso una cadena de oro alrededor del cuello.
\bibverse{43} Hizo que José viajara en el carro designado para su
segundo al mando mientras sus asistentes se adelantaron gritando,
``¡Inclínate!''\footnote{41.43 ``¡Inclínate!'' Esta palabra prestada del
  idioma egipcio cuenta con varias traducciones: ``¡Atención!'' ``¡Abran
  paso!'' ``¡Alaben!'' ``¡Gloria!'' Todos se relacionan con la honra a
  un gran dignatario.} Así es como el Faraón le dio a José autoridad
sobre todo Egipto.

\bibverse{44} Entonces el Faraón le dijo a José, ``Yo soy el Faraón,
pero sin tu permiso nadie levantará una mano o un pie en todo el país''.
\bibverse{45} El Faraón le dio a José el nombre de Zafnat
Paneajab,\footnote{41.45 Quiere decir ``El Dios habla y él (el sujeto)
  vive''.}y arregló que se casara con Asenat, la hija de Potifera,
sacerdote de On. Así es como José se elevó al poder en todo Egipto.

\bibverse{46} José tenía treinta años cuando empezó a trabajar para el
Faraón, rey de Egipto. Después de dejar al Faraón, José viajó en una
gira de inspección\footnote{41.46 ``Gira de inspección'': añadido para
  mayor claridad.}por todo Egipto. \bibverse{47} Durante los siete años
de buenas cosechas, la tierra produjo muchos alimentos. \bibverse{48}
Recolectó todos los alimentos durante los siete años buenos, y almacenó
el grano producido en los campos locales de cada ciudad. \bibverse{49}
JJosé apiló tanto grano que era como la arena de la orilla del mar.
Eventualmente dejó de llevar registros porque había mucho.

\bibverse{50} Fue durante este tiempo, antes de que llegaran los años de
hambruna, que José tuvo dos hijos con Asenat, la hija de Potifera,
sacerdote de On. \bibverse{51} José nombró a su primogénito
Manasés,\footnote{41.51 ``Manasés'' significa ``que hace olvidar''.}porque
dijo: ``El Señor me ha hecho olvidar todos mis problemas y toda la
familia de mi padre''. \bibverse{52} A su segundo hijo le puso el nombre
de Efraín,+ 41.52 ``Efraín'' significa ``fructífero''.porque dijo:
``Dios me ha hecho fructífero en el país de mi miseria''.

\bibverse{53} Los siete años de abundancia en Egipto llegaron a su fin,
\bibverse{54} y los siete años de hambruna comenzaron, tal como José
había dicho. Había hambruna en todos los demás países, pero todo Egipto
tenía comida. \bibverse{55} Cuando todo Egipto tuvo hambre, la gente
clamó al Faraón por comida, y él les dijo a todos: ``Vayan a ver a José
y hagan lo que él les diga''. \bibverse{56} La hambruna se había
extendido por todo el país, así que José abrió todos los almacenes y
vendió el grano al pueblo de Egipto. La hambruna era muy mala en Egipto,
\bibverse{57} de hecho, la hambruna era muy mala en todas partes, así
que la gente de otros países de todo el mundo vino a Egipto para comprar
grano a José.

\hypertarget{section-41}{%
\section{42}\label{section-41}}

\bibverse{1} Cuando Jacob se enteró de que había grano disponible en
Egipto, preguntó a sus hijos: ``¿Por qué siguen mirándose para hacer
algo? \bibverse{2} He oído que hay grano en Egipto. ¡Ve allí y compra un
poco para nosotros para que podamos seguir vivos, si no, vamos a
morir!''

\bibverse{3} Así que diez de los hermanos de José fueron a Egipto a
comprar grano. \bibverse{4} Pero Jacob no envió al hermano de José,
Benjamín, con sus otros hermanos, porque dijo: ``Tengo miedo de que le
pase algo malo''. \bibverse{5} Así que los hijos de Israel fueron a
comprar grano junto con todos los demás, porque también había hambre en
Canaán.

\bibverse{6} José era el gobernador del país y vendía grano a todo el
pueblo de allí. Los hermanos de José fueron a él y se inclinaron ante él
con el rostro en tierra. \bibverse{7} José los reconoció en cuanto los
vio, pero se comportó como un extraño con ellos y les habló con
severidad, diciendo: ``¿De dónde vienen?''

``Del país de Canaán'', le respondieron. ``Hemos venido a comprar
comida''.

\bibverse{8} Aunque José reconoció a sus hermanos, ellos no lo
reconocieron a él. \bibverse{9} José pensó en los sueños que había
tenido con ellos y les dijo: ``¡No! ¡Son espías! ¡Habéis venido a
descubrir las debilidades de nuestro país!''

\bibverse{10} ``¡Eso no es cierto, mi señor!'' respondieron. ``Nosotros,
sus siervos, hemos venido a comprar''

\bibverse{11} ``Todos somos hijos de un hombre y somos honestos. ¡No
somos espías!''

\bibverse{12} ``¡Claro que no! ¡Ustedes han venido a descubrir la
debilidad de nuestra nación!'' insistió.

\bibverse{13} ``Tus siervos son doce hermanos, hijos de un hombre que
vive en el país de Canaán'', explicaron. ``El más joven está ahora mismo
con nuestro padre, y uno ha fallecido''.

\bibverse{14} ``Como dije antes, ¡son espías!'' declaró José.
\bibverse{15} ``Así es como se comprobará su historia: Juro por la vida
del Faraón que nunca dejarán este país a menos que su hermano menor
venga aquí. \bibverse{16} Uno de ustedes tendrá que regresar y traer a
su otro hermano aquí. Los demás se quedarán aquí en la cárcel hasta que
esté claro que dicen la verdad. Si no, entonces juro por la vida del
Faraón que esa será la prueba de que son espías''

\bibverse{17} Así que José los puso a todos en prisión por tres días.
\bibverse{18} Al tercer día les dijo: ``Como respeto a Dios, hagan lo
que les digo y vivirán. \bibverse{19} Si son verdaderamente honestos,
elijan a uno de sus hermanos para que se quede aquí en la cárcel. El
resto de podrá volver a casa con grano para sus familias hambrientas.
\bibverse{20} Pero deben traer a su hermano menor aquí para demostrar
que lo que dicen es verdad. Si no, todos ustedes morirán''.Ellos
estuvieron de acuerdo en hacer esto.

\bibverse{21} ``Claramente estamos siendo castigados por lo que le
hicimos a nuestro hermano'', se decían unos a otros. ``Lo vimos en
agonía suplicándonos misericordia, pero nos negamos a escucharlo. Es por
eso que tenemos todos estos problemas''.

\bibverse{22} Rubén les dijo: ``¿No les dije: `No le hagan daño al
muchacho'? Pero no me escucharon. Ahora estamos pagando el precio por lo
que le hicimos.''\footnote{42.22 Literalmente, ``Ahora se requiere su
  sangre''. El concepto es que la sangre de la víctima clama por
  venganza.} \bibverse{23} No se daban cuenta de que José entendía lo
que decían porque le hablaban a través de un intérprete. \bibverse{24}
José se alejó de ellos porque empezó a llorar. Volvió cuando pudo
hablarles de nuevo. Eligió a Simeón y lo tuvo atado mientras ellos
miraban.

\bibverse{25} José dio la orden de llenar sus sacos con grano, y también
de devolver el dinero que habían pagado poniéndolo también en los sacos.
También ordenó que se les proveyera de comida para el viaje de vuelta a
casa. Todo esto se hizo. \bibverse{26} Los hermanos cargaron el grano en
sus asnos y luego se pusieron en marcha.

\bibverse{27} En el camino, se detuvieron para pasar la noche y uno de
ellos abrió su saco para darle algo de comer a su asno y vio su dinero
allí en la parte superior del saco. \bibverse{28} Entonces les dijo a
sus hermanos: ``Me han devuelto mi dinero. ¡Está aquí mismo en la parte
superior de mi saco!'' ¡Estaban horrorizados! Temblando de miedo se
preguntaron: ``¿Qué es esto que Dios nos ha hecho?''

\bibverse{29} Cuando llegaron a casa en Canaán, le contaron a su padre
Jacob todo lo que había pasado. \bibverse{30} ``El hombre que es el
gobernador del país nos habló de manera severa y nos acusó de espiar la
tierra'', explicaron. \bibverse{31} ``Le dijimos: `Somos hombres
honestos. ¡No somos espías! \bibverse{32} Somos doce hermanos, hijos de
un solo padre. Uno ha fallecido y el más joven está ahora mismo con
nuestro padre en el país de Canaán.' \bibverse{33} Entonces el
gobernador del país nos dijo: `Así es como sabré si dicen la verdad:
dejen a uno de sus hermanos aquí conmigo mientras los demás llevan grano
a casa para sus familias hambrientas. \bibverse{34} Entonces tráiganme a
su hermano menor. Así sabré que no son espías, sino que dicen la verdad.
Les entregaré a su hermano y podrán quedarse en el país y hacer
negocios.'\,''

\bibverse{35} Mientras vaciaban sus sacos, la bolsa de dinero de cada
uno estaba allí en su saco. Cuando ellos y su padre vieron las bolsas de
dinero, se horrorizaron. \bibverse{36} Jacob, su padre, los acusó: ``Me
han quitado a José, ¡se ha ido! ¡Simeón también se ha ido! ¡Ahora
quieren llevarse a Benjamín! ¡Soy yo el que está sufriendo por todo
esto!''\footnote{42.36 La frase es literalmente, ``sobre mí están todas
  estas cosas''. La construcción hebrea se centra ``sobre mí'' dejando
  claro que Jacob los hace responsables de su sufrimiento.}

\bibverse{37} ``Puedes matar a mis dos hijos si no te lo devuelvo'', le
aseguró Rubén. ``Confíameloa mí, y yo mismo te lo traeré a casa''.

\bibverse{38} ``¡Mi hijo no irá allí con ustedes!'' declaró Jacob. ``Su
hermano está muerto, y es el único que me queda. Si le pasa algo malo en
el viaje, la tristeza llevará a este viejo a la tumba''.

\hypertarget{section-42}{%
\section{43}\label{section-42}}

\bibverse{1} La hambruna continuó siendo muy grave en Canaán,
\bibverse{2} así que una vez que se acabó el grano que habían traído de
Egipto, su padre les dijo: ``Tienen que volver y comprar más grano''.

\bibverse{3} Pero Judá respondió: ``El hombre fue firme cuando nos
advirtió: `No los veré a menos que su hermano venga con ustedes.'
\bibverse{4} Si envías a nuestro hermano Benjamín con nosotros, entonces
iremos a comprar comida para ti. \bibverse{5} Pero si no lo envías,
entonces no iremos, porque el hombre fue muy claro, `No los veré a menos
que su hermano venga con ustedes.'\,''

\bibverse{6} ``¿Por qué me hancomplicado las cosas al decirle al hombre
que tenían otro hermano? Preguntó Israel.

\bibverse{7} ``El hombre seguía haciendo preguntas directas sobre
nosotros y nuestra familia, como: `¿Su padre sigue vivo?' y `¿Tienen
otro hermano?'\,'' respondieron ellos. ``Sólo respondimos a sus
preguntas. ¿Cómo íbamos a saber que él diría, `¡Traigan a su hermano
aquí!'\,''

\bibverse{8} Judá le dijo a su padre Israel: ``Envía al muchacho bajo mi
cuidado, y nos iremos inmediatamente, para que podamos seguir vivos y no
morir, ¡y eso te incluye a ti, a nosotros y a nuestros hijos!
\bibverse{9} Prometo cuidarlo y seré personalmente responsable de
traerlo de vuelta a ti. ¡Si no lo hago, entonces cargaré siempre con la
culpa! \bibverse{10} Ahora vamos, porque si no hubiéramos dudado, ya
podríamos haber ido y vuelto dos veces''.

\bibverse{11} ``Si tiene que ser así, entonces esto es lo que harán'',
respondió Israel. ``Llévense lo mejor que produce nuestro país. Empaquen
sus bolsas con regalos para este hombre: bálsamo, un poco de miel,
especias, mirra, pistachos y almendras. \bibverse{12} Lleven el doble
del dinero que les han devuelto en tus sacos, tal vez fue un error.
\bibverse{13} Tomen a su hermano y regresen de inmediato donde este
hombre. \bibverse{14} Que Dios Todopoderoso haga que este hombre los
trate bien para que cuando se presenten ante él libere a su otro hermano
y envíe a Benjamín de regreso. En cuanto a mí, si voy a perder a todos
mis hijos, que así sea''.

\bibverse{15} Así que empacaron los regalos, tomaron el doble de dinero
y se fueron, acompañados por Benjamín. Llegaron a Egipto y fueron a
tener una audiencia con José. \bibverse{16} Cuando José vio que Benjamín
estaba con ellos, le dijo al encargado de su casa: ``Lleva a estos
hombres a mi casa. Maten un animal y preparen una comida, porque van a
comer conmigo al mediodía''.

\bibverse{17} El hombre hizo lo que le ordenó José y los llevó a la casa
de José. \bibverse{18} Ellos estaban muy asustados de que los llevaran a
la casa de José. ``Es por el dinero que se estaba en nuestros sacos la
primera vez que vinimos'', se dijeron entre ellos. ``¡Por eso nos traen
para acusarnos y atacarnos! ¡Nos convertirá en sus esclavos y se llevará
nuestros asnos!''

\bibverse{19} Así que fueron y hablaron con el supervisor de la casa de
José en la entrada de la casa. \bibverse{20} ``Por favor, discúlpenos,
mi señor'', dijeron. ``Bajamos la primera vez para comprar comida,
\bibverse{21} y cuando nos detuvimos para pasar la noche, abrimos
nuestros sacos y cada uno de nosotros encontró su dinero, la cantidad
exacta, en la parte superior de nuestros sacos. Así que lo trajimos de
vuelta con nosotros. \bibverse{22} También hemos traído más dinero para
comprar comida. ¡No tenemos ni idea de quién puso nuestro dinero en
nuestros sacos!''

\bibverse{23} ``¡Todo está bien!'' les dijo. ``¡No se preocupen! Su
Dios, el Dios de su padre, debe haberles dado el
tesoro\footnote{43.23 ``Tesoro'': la palabra se refiere, por supuesto,
  al dinero, y es la que se usa para describir el dinero que está
  escondido o enterrado.}escondido en sus sacos. Yo tengo su dinero''.
Luego trajo a Simeón para que se encontrara con ellos. \bibverse{24} El
mayordomo los llevó dentro de la casa de José, les dio agua para que se
lavaran los pies, y les dio comida para sus asnos. \bibverse{25}
Prepararon sus regalos para cuando José viniera al mediodía, porque se
habían enterado de que iban a comer allí.

\bibverse{26} Cuando José llegó a la casa le dieron los regalos que le
habían traído y se inclinaron hasta el suelo ante él. \bibverse{27} José
preguntó cómo estaban, y luego les preguntó: ``¿Cómo está su anciano
padre del cual me hablaron? ¿Sigue vivo?''

\bibverse{28} ``Sí, tu siervo, nuestro padre, sigue vivo y está bien'',
respondieron, y se inclinaron en señal de respeto.

\bibverse{29} Entonces José miró a su hermano Benjamín, el hijo de su
propia madre. ``¿Es este su hermano más joven del que me hablaron?''
preguntó. ``Dios sea misericordioso contigo, hijo mío'', dijo.

\bibverse{30} José tuvo que salir corriendo rápidamente porque se estaba
poniendo muy emotivo al ver a su hermano.\footnote{43.30 No había visto
  a Benjamín por más de 20 años.}Buscó un lugar para llorar, y se fue a
su habitación para llorar allí. \bibverse{31} Luego se lavó la cara,
controló sus emociones y volvió a salir. ``Sirvan la comida'', ordenó.

\bibverse{32} José fue servido en una mesa para él solo, y sus hermanos
fueron servidos en una mesa separada. A los egipcios también se les
sirvió en otra mesa, porque los egipcios no podrían comer con los
hebreos, porque les resultaba repulsivo.\footnote{43.32 Parece que como
  los egipcios veneraban a la diosa vaca Isis, considereaban inmundo a
  cualquiera (incluyendo a los hebreos) que comiera carne.}
\bibverse{33} Los hermanos se habían sentado frente a él en orden de
edad, desde el primogénito, el mayor, hasta el más joven, y se miraron
con absoluta sorpresa.+ 43.33 Por supuesto, esto habría sido imposible
para cualquiera que no conociera la intimidad de la familia.
\bibverse{34} La comida se les sirvió de la mesa de José, y Benjamín
recibió cinco veces más que cualquier otro. Así que comieron y bebieron
mucho con él.

\hypertarget{section-43}{%
\section{44}\label{section-43}}

\bibverse{1} José le ordenó al mayordomo de la casa: ``Llena los sacos
de los hombres con todo el grano que puedan contener y pon el dinero de
cada hombre en la parte superior de su saco. \bibverse{2} Luego pon mi
taza de plata especial en la parte superior del saco del más joven,
junto con el dinero para su grano''.El mayordomo hizo lo que le dijo
José. \bibverse{3} Al amanecer, fueron enviados de camino con sus asnos.
\bibverse{4} Apenas habían salido de la ciudad cuando José le dijo al
mayordomo de su casa: ``Ve tras esos hombres, y cuando los alcances,
pregúntales: `¿Por qué han devuelto el bien con el mal, robando la copa
de plata de mi amo?\footnote{44.4 ``Robando la copa de plata de mi
  amo'': Adición de la Septuaginta, para mayor claridad.} \bibverse{5}
Esta es la copa de la que él personalmente bebe, y que usa para
adivinar.+ 44.5 ``Divination'': a way of discovering secrets or hidden
knowledge. Sometimes this is close to magic, but, in this case, it may
be that José is using a common superstition to cover up his plan. Lo que
han hecho es realmente malo!'\,''

\bibverse{6} Cuando los alcanzó, les dijo lo que José había dicho.

\bibverse{7} ``Señor mío, ¿qué estás diciendo?'' le contestaron.
``¡Nosotros, tus siervos, no haríamos algo así! \bibverse{8} Recuerda
que trajimos el dinero que encontramos en la parte superior de nuestros
sacos cuando volvimos de Canaán. ¿Por qué robaríamos plata u oro de la
casa de tu señor? \bibverse{9} Si alguno de nosotros es encontrado con
él, morirá, y todos nosotros nos convertiremos en tus esclavos''.

\bibverse{10} ``Lo que ustedes digan'', respondió el hombre, ``pero sólo
el que sea encontrado con él se convertirá en mi esclavo, ya que el
resto de ustedes estarán libres de toda culpa''. \bibverse{11} Todos
descargaron sus sacos y los pusieron en el suelo. Cada uno abrió su
propio saco. \bibverse{12} El supervisor de la casa registró los sacos,
empezando por el más viejo y bajando hasta el más joven. La taza fue
encontrada en el saco de Benjamín. \bibverse{13} Los hermanos rasgaron
sus ropas en señal de lamento. Luego cargaron los sacos en sus burros y
se dirigieron a la ciudad.

\bibverse{14} José todavía estaba en casa cuando Judá y sus hermanos
llegaron, y cayeron al suelo delante de él. \bibverse{15} ``¿Por qué
hicieron esto?'' preguntó José. ``¿No saben que un hombre como yo puede
darse cuenta de estas cosas por medio de la adivinación?''

\bibverse{16} ``Mi señor, ¿qué podemos decir?'' respondió Judá. ``¿Cómo
podemos explicarte esto? ¿De qué manera podemos probar nuestra
inocencia? Dios ha expuesto la culpa de tus siervos. Mi señor, somos tus
esclavos, todos nosotros, incluyendo el que fue encontrado con la copa''

\bibverse{17} ``¡Yo no haría nada de eso!'' respondió José. ``Sólo el
hombre que fue encontrado con la copa se convertirá en mi esclavo. El
resto de ustedes puede regresar con su padre''.

\bibverse{18} Judá se acercó y le dijo: ``Si te complace, mi señor, deja
que tu siervo diga una palabra. Por favor, no te enfades con tu siervo,
aunque seas tan poderoso como el propio Faraón. \bibverse{19} Mi señor,
antes nos preguntaste: `¿Tienen un padre o un hermano?' \bibverse{20} Y
respondimos, mi señor: ``Tenemos un padre anciano y un hermano menor,
que nació cuando nuestro padre ya era anciano. El hermano del muchacho
está muerto. Es el único de los hijos de su madre que queda, y su padre
lo quiere mucho.'

\bibverse{21} Entonces tú nos ordenaste: `Tráiganlo aquí para que pueda
verlo.' \bibverse{22} Y te dijimos: `El muchacho no puede dejar a su
padre; porque si lo hiciera, su padre moriría.' \bibverse{23} Pero tú
nos dijiste: `Si su hermano menor no viene con ustedes, no me volverán a
ver.'

\bibverse{24} Así que cuando volvimos con tu siervo, nuestro padre, le
explicamos todo lo que nos habías dicho. \bibverse{25} Sin embargo, más
tarde, nuestro padre nos dijo: `Vuelvan y compren más comida.'
\bibverse{26} Pero nosotros le dijimos: `No podemos volver a menos que
Benjamín, nuestro hermano menor, vaya con nosotros, porque no podremos
ver a este hombre si Benjamín no va con nosotros.'

\bibverse{27} Entonces mi padre nos dijo: `Os dais cuenta de que mi
mujer\footnote{44.27 ``Mi esposa'': Refiriéndose a Raquel. Evidentemente
  Jacob la consideraba como su verdadera esposa.} tuvo dos hijos para
mí. \bibverse{28} Uno ya no está, sin duda quedó hecho pedazos,+ 44.28
Ver 37:33.porque no lo he visto desde entonces. \bibverse{29} Si me
quitan a éste también, y le pasa algo malo, la tristeza llevará a este
viejo a la tumba.'

\bibverse{30} Así que si el muchacho no está con nosotros cuando regrese
a mi padre, cuya vida depende de la vida del muchacho, \bibverse{31} tan
pronto como vea que el muchacho no está allí morirá, y realmente
enviaremos a este anciano, nuestro padre, a su tumba con tristeza.
\bibverse{32} De hecho me di a mí mismo como garantía del muchacho a mi
padre. Le dije: `¡Si no lo traigo de vuelta a ti, siempre cargaré con la
culpa!'

\bibverse{33} Así que, por favor, déjame quedarme aquí como esclavo de
mi señor en lugar del niño. Deja que vuelva a casa con sus hermanos.
\bibverse{34} Porque, ¿cómo podría volver a mi padre si el niño no
estuviera conmigo? No podría soportar ver la angustia que causaría a mi
padre''.

\hypertarget{section-44}{%
\section{45}\label{section-44}}

\bibverse{1} José no pudo controlar sus emociones por más tiempo
mientras todos sus asistentes estaban allí, así que gritó: ``¡Todos
déjenme solo!'' Así que no había nadie más cuando José reveló quién era
a sus hermanos. \bibverse{2} Pero gritó tan fuerte que los egipcios
pudieron oírlo, y se lo contaron a la casa del Faraón.

\bibverse{3} ``¡Yo soy José!''les anunció a sus hermanos. ``¿Mi padre
sigue vivo?'' No pudieron responderle porque se sorprendieron mucho al
verle cara a cara.

\bibverse{4} ``Por favor, acérquense a mí'', les dijo a sus hermanos. Se
acercaron a él. ``Soy su hermano José, a quien vendieron como esclavo en
Egipto. \bibverse{5} Pero no se preocupen ni se enojen con ustedes
mismos por haberme vendido como esclavo aquí, porque fue Dios quien me
envió antes que ustedes para salvar vidas. \bibverse{6} El país ha
sufrido de hambruna durante dos años ya, pero habrá cinco años más sin
arar ni cosechar. \bibverse{7} Dios me envió delante de ustedes para
asegurarse de que todavía tuvieran descendencia, para salvar sus vidas
de esta forma milagrosa.\footnote{45.7 ``De esta forma milagrosa'': o
  ``con muchos sobrevivientes''.} \bibverse{8} Así que no fueron ustedes
quienes me enviaron aquí, sino Dios. Él fue quien me convirtió en el
consejero principal+ 45.8 ``El consejero principal del Faraón'':
literalmente, ``un padre para el Faraón''. del Faraón a cargo de todos
sus asuntos, y gobernante de todo el país de Egipto.

\bibverse{9} ¡Ahora apúrense! Vuelvan donde está mi padre y díganle:
``Este mensaje es de tu hijo José: `Dios me ha hecho gobernante de todo
Egipto. Ven a mí ahora, sin tardar. \bibverse{10} Vivirás en la tierra
de Gosén para estar cerca de mí con tus hijos y nietos, y con tus
rebaños y manadas y todo lo que te pertenece. \bibverse{11} Me aseguraré
de que tengan comida, porque aún quedan cinco años de hambruna por
venir. De lo contrario, tú y tu familia y tus animales van a morir de
hambre.'\,''

\bibverse{12} Entonces José dijo a sus hermanos,\footnote{45.12
  ``Entonces José dijo a sus hermanos'': añadido para mayor claridad,
  mostrando que José se dirige directamente a sus hermanos de nuevo.}
``¡Todos pueden ver por ustedes mismos, incluyendo a mi hermano
Benjamín, que realmente soy yo quien les habla! \bibverse{13} Díganle a
mi padre cuánto me respetan en Egipto. Cuéntenle todo lo que han visto.
¡Deprisa! ¡Traigan a mi padre aquí rápidamente!'' \bibverse{14} Abrazó a
Benjamín, y Benjamín le devolvió el abrazo. Ambos lloraron de alegría.
\bibverse{15} Besó a todos sus hermanos y lloró por ellos, y después de
eso, sus hermanos pudieron empezar a hablar con él.

\bibverse{16} Pronto llegó al palacio del faraón la noticia de que los
hermanos de José habían llegado. El Faraón y sus oficiales se alegraron
de escuchar la noticia.

\bibverse{17} El faraón le dijo a José: ``Dile a tus hermanos: Esto es
lo que deben hacer: Carguen sus asnos con grano y vuelvan a la tierra de
Canaán. \bibverse{18} Entonces traigan a su padre y a sus familias y
vuelvan aquí conmigo. Les daré la mejor tierra de Egipto y comerán la
mejor comida que el país pueda ofrecerles'.

\bibverse{19} Diles que hagan esto también: ``Tomen algunos carros de
Egipto para sus hijos y sus esposas. Tráiganlos a ellos y a su padre
aquí. \bibverse{20} No se preocupen por traer sus posesiones, porque lo
mejor de todo Egipto ya les pertenece.'''

\bibverse{21} Así que los hijos de Israel hicieron justamente eso. José
les proporcionó carros, como el faraón lo había ordenado. También les
dio provisiones para su viaje. \bibverse{22} Les dio a cada uno de ellos
ropa nueva. Pero a Benjamín le dio cinco juegos de ropa y 300 piezas de
plata. \bibverse{23} José también envió a su padre lo siguiente: diez
asnos que llevaban las mejores cosas de Egipto, y diez asnas que
llevaban el grano y el pan y los suministros necesarios para el viaje de
su padre.

\bibverse{24} Luego vio a sus hermanos irse, y cuando se fueron les
dijo: ``¡No discutan en el camino!'' \bibverse{25} Así que salieron de
Egipto y volvieron a la casa de su padre Jacob, en el país de Canaán.

\bibverse{26} ``¡José sigue vivo!'' le dijeron. ``¡Es verdad, y él es el
gobernante de todo el país de Egipto!'' Jacob se quedó atónito con la
noticia, ¡no podía creerlo! \bibverse{27} Pero cuando le contaron todo
lo que José les había dicho, y cuando vio los carros que José había
enviado para llevarlo a Egipto, Jacob volvió en sí. \bibverse{28} ¡Está
bien, les creo! ¡Mi hijo José sigue vivo! Voy a ir a verlo antes de
morir''.

\hypertarget{section-45}{%
\section{46}\label{section-45}}

\bibverse{1} Así que Israel se fue a Egipto con todo lo que tenía.
Cuando llegó a Beerseba ofreció sacrificios al Dios de su padre Isaac.
\bibverse{2} Durante la noche Dios habló a Israel en una visión.
``¡Jacob! ¡Jacob!'' llamó.

``Estoy aquí'', respondió.

\bibverse{3} ``¡Yo soy Dios, el Dios de tu padre! No temas ir a Egipto,
porque te convertiré a ti y a tus descendientes\footnote{46.3 ``Y a tus
  descendientes'': añadido para mayor claridad.} en una gran nación.
\bibverse{4} Iré a Egipto contigo, y prometo traerte de vuelta. Y José
personalmente cerrará tus ojos cuando mueras''.

\bibverse{5} Entonces Jacob dejó Beerseba. Sus hijos lo llevaron a él, a
sus hijos y a sus esposas a Egipto usando los carros que el Faraón había
enviado. \bibverse{6} También se llevaron todo su ganado y todas las
pertenencias personales que habían acumulado en el país de Canaán.

Así pues, Jacob y todos los miembros de su extensa familia fueron a
Egipto, \bibverse{7} incluyendo todos sus hijos y nietos, hijas y
nietas.

\bibverse{8} La siguiente es la genealogía de Israel y sus hijos que
fueron a Egipto: Rubén, el primogénito de Jacob.

\bibverse{9} Los hijos de Rubén: Janoc, Falú, Jezrón y Carmi.

\bibverse{10} Los hijos de Simeón: Jemuel, Jamín, Ohad, Jachín, Zojar y
Saúl, hijo de una mujer cananea.

\bibverse{11} Los hijos de Leví: Gersón, Coat y Merari.

\bibverse{12} Los hijos de Judá: Onán, Selá, Fares y Zera. Sin embargo,
Er y Onán murieron en Canaán.

Los hijos de Fares: Hezrón y Hamul.

\bibverse{13} Los hijos de Isacar: Tola, Fuvá, Job,\footnote{46.13
  ``Job'' se presenta como ``Jasub'' en Números 26:24 y 1 Crónicas 7:1.}
y Simrón.

\bibverse{14} Los hijos de Zabulón: Séred, Elón y Yalel.

\bibverse{15} Estos son los hijos que Lea tuvo para Jacob en Padán Aram,
así como su hija Dina. El número total de hijos e hijas y nietos fue de
treinta y tres.

\bibverse{16} Los hijos de Gad: Zefón, Jaguí, Suni, Esbón, Erí, Arodí y
Arelí.

\bibverse{17} Los hijos de Aser: Imná, Isvá, Isví, Beriá, y su hermana
Sera.

Los hijos de Beriá: Heber y Malquiel.

\bibverse{18} Estos son los hijos que Jacob tuvo con Zilpá, la sierva
dada por Labán a su hija Lea. Fue en total dieciséis hijos y nietos.

\bibverse{19} Los hijos de la esposa de Jacob, Raquel: José y Benjamín.

\bibverse{20} Los hijos que José tuvo en la tierra de Egipto con Asenat,
hija de Potifera, sacerdote de On: Manasés y Efraín.

\bibverse{21} Los hijos de Benjamín: Bela, Béquer, Asbel, Guerá, Naamán,
Ehí, Ros, Mupín, Jupín y Ard. \bibverse{22} Estos son los hijos que
Raquel tuvo con Jacob, y fueron en total catorce hijos y nietos.

\bibverse{23} El hijo de Dan: Jusín.

\bibverse{24} Los hijos de Neftalí: Yasel, Guní, Jéser y Silén.

\bibverse{25} Estos son los hijos que Jacob tuvo con Bilhá, la sierva
dada por Labán para su hija Raquel. Fue un total de siete hijos y
nietos.

\bibverse{26} Todos los que formaban parte de la familia de Jacob que
vinieron a Egipto (sus parientes de sangre, aparte de las esposas de los
hijos de Jacob) sumaban un total de sesenta y seis. \bibverse{27}
Incluyendo los dos hijos que José tuvo en Egipto, el número total de la
familia de Jacob que se encontraba en Egipto era de setenta.

\bibverse{28} Jacob envió a Judá por delante para que se reuniera con
José y averiguara el camino a Gosén. Cuando llegaron a Gosén,
\bibverse{29} José ordenó que prepararan su carro y fue a encontrarse
allí con su padre Israel. Tan pronto como llegó, abrazó a su padre y
lloró por mucho tiempo.

\bibverse{30} ``Ahora puedo morir en paz porque he visto tu rostro de
nuevo y sé que sigues vivo'', le dijo Israel a José.

\bibverse{31} José dijo a sus hermanos y a la familia de su padre: ``Voy
a ir a informar al Faraón y a decirle: `Mis hermanos y la familia de mi
padre han llegado del país de Canaán para unirse a mí. \bibverse{32} Son
pastores y tienen ganado. Han traído con ellos sus rebaños y manadas y
todas sus posesiones.'

\bibverse{33} Cuando el Faraón los llame y les pregunte: `¿Qué trabajo
hacen ustedes?' \bibverse{34} díganle: `Tus siervos han cuidado ganado
desde que éramos niños, tanto nosotros como nuestros padres antes que
nosotros.' Así podrán vivir aquí en Gosén, porque los egipcios
desprecian a los pastores.'\,''

\hypertarget{section-46}{%
\section{47}\label{section-46}}

\bibverse{1} José fue a informar al faraón y le dijo: ``Mi padre y mis
hermanos, junto con sus rebaños y manadas y todas sus posesiones, han
llegado de la tierra de Canaán y ahora están aquí en Gosén''.
\bibverse{2} José tomó a cinco de sus hermanos para que lo acompañaran y
se los presentó al Faraón.

\bibverse{3} El faraón les preguntó a los hermanos: ``¿Qué trabajo hacen
ustedes?''

``Nosotros, sus siervos, somos pastores, como nuestros padres antes que
nosotros'', respondieron ellos.

\bibverse{4} ``Hemos venido a vivir a Egipto porque no queda hierba en
Canaán para que nuestros rebaños coman'', explicaron. ``La hambruna es
muy grave en Canaán. Así que nos gustaría pedir permiso para vivir en
Gosén''.

\bibverse{5} El faraón le dijo a José: ``Ahora que tu padre y tus
hermanos han llegado para unirse a ti, \bibverse{6} puedes elegir el
lugar que quieras en Egipto, el mejor lugar, para que ellos vivan.
Déjalos vivir en Gosén. Si conoces a alguno de ellos que sea bueno en lo
que hace, ponlo también a cargo de mi ganado''.

\bibverse{7} Entonces José fue con su padre Jacob y le presentó al
faraón. Después de que Jacob bendijera al faraón, \bibverse{8} el faraón
le preguntó: ``¿Cuánto tiempo has vivido?''

\bibverse{9} ``He estado andando de aquí para allá durante 130 años'',
respondió Jacob. ``Mi vida ha sido corta y difícil; no he vivido tanto
como mis antepasados que también vagaban de un lugar a otro''.
\bibverse{10} Entonces Jacob bendijo de nuevo al faraón antes de
dejarlo.

\bibverse{11} Entonces José dispuso que su padre y sus hermanos vivieran
en Egipto y les dio tierra en la mejor parte, cerca de Ramsés, como el
Faraón lo había ordenado. \bibverse{12} También les proporcionó
alimentos a todos ellos: a su padre, a sus hermanos y a toda la familia
de su padre, según el tamaño de cada familia.

\bibverse{13} No quedaba comida en todo el país porque la hambruna se
había vuelto muy grave. A lo largo de Egipto y Canaán la gente se moría
de hambre. \bibverse{14} Mediante la venta de grano, José recogió todo
el dinero en Egipto y Canaán, y lo colocó en el tesoro del Faraón.
\bibverse{15} Una vez que el dinero de Egipto y Canaán se había acabado,
todos los egipcios vinieron a José y le exigieron: ``¡Danos comida!
¿Quieres que muramos delante de ti? ¡Hemos perdido todo nuestro
dinero!''

\bibverse{16} ``Tráiganme su ganado'', les dijo José. ``Si se han
quedado sin dinero, les daré grano a cambio de su ganado''.

\bibverse{17} Así que los egipcios le trajeron a José su ganado, y él
les dio grano a cambio de sus caballos, ovejas, cabras, ganado y burros.
Durante ese año, José les dio grano a cambio de su ganado.

\bibverse{18} Pero cuando terminó ese año, vinieron a él al año
siguiente y le dijeron: ``Mi señor, no podemos ocultarte el hecho de que
nuestro dinero ha desaparecido y que ahora eres dueño de nuestro ganado.
Todo lo que nos queda por darte son nuestros cuerpos y nuestra tierra.
\bibverse{19} ¿Quieres que muramos delante de ti? Entonces cómpranos a
nosotros y a nuestra tierra a cambio de comida. Entonces nuestra tierra
pertenecerá al Faraón, y seremos sus esclavos. Danos grano para que
podamos vivir y no muramos, y así la tierra no quedará abandonada''.

\bibverse{20} Así que José compró toda la tierra de Egipto para el
Faraón. Todos y cada uno de los egipcios vendieron sus campos, porque la
hambruna les estaba haciendo mucho daño. La tierra terminó siendo
propiedad del Faraón, \bibverse{21} y todo el pueblo se convirtió en
esclavos suyos,\footnote{47.21 ``Y todo el pueblo se convirtió en
  esclavos suyos'': Dicho por la Septuaginta y otras traducciones
  antiguas. El hebreo dice ``los trasladó a las ciudades''.} de un
extremo a otro de Egipto. \bibverse{22} La única tierra que no compró
fue la de los sacerdotes porque tenían una asignación de alimentos que
les proporcionó el Faraón, así que no tuvieron que vender sus tierras.

\bibverse{23} Entonces José le dijo al pueblo: ``¡Escúchenme! Ahora que
los he comprado a ustedes y a su tierra para el Faraón, les daré
semillas para que siembren los campos. \bibverse{24} Sin embargo, cuando
recojan la cosecha, tienen que dar una quinta parte al Faraón. Las otras
cuatro quintas partes las podrán guardar como semilla para los campos y
como alimento para ustedes mismos, sus hogares y sus hijos''.

\bibverse{25} ``Nos has salvado la vida'', declararon. ``Ojalá sigas
tratándonos bien, mi señor, y seremos esclavos del Faraón''.

\bibverse{26} Así que José hizo una ley para Egipto, que sigue vigente
hoy en día: Que una quinta parte de todos los productos de la tierra
pertenecen al Faraón. Sólo la tierra de los sacerdotes estaba exenta ya
que no pertenecía al Faraón.

\bibverse{27} Los israelitas se establecieron en Gosén, en Egipto, donde
se convirtieron en prósperos terratenientes y aumentaron rápidamente en
número de habitantes. \bibverse{28} Jacob vivió en Egipto durante
diecisiete años, por lo que vivió en total 137 años.

\bibverse{29} Cuando llegó el momento de su muerte, Israel llamó a su
hijo José y le dijo: ``Si me consideras, pon tu mano debajo de mi muslo
y promete tratarme con amor y fidelidad. No me entierres aquí en Egipto.
\bibverse{30} Cuando muera, entiérrame con mis antepasados. Deben llevar
mi cuerpo desde Egipto hasta la tumba familiar y enterrarme con ellos''.

``Haré lo que tú digas'', prometió José.

\bibverse{31} ``Júrame que lo harás'', dijo. Y José juró que lo haría.
Entonces Israel se inclinó en actitud de adoración en la cabecera de su
cama.

\hypertarget{section-47}{%
\section{48}\label{section-47}}

\bibverse{1} Algún tiempo después de esto, le dijeron a José: ``Tu padre
está enfermo''. Así que José fue a verlo, llevándose a sus dos hijos
Manasés y Efraín.

\bibverse{2} Cuando le dijeron a Jacob: ``Tu hijo José ha venido a
verte'', reunió sus fuerzas y se sentó en la cama. \bibverse{3} Jacob le
dijo a José: ``El Dios Todopoderoso se me apareció en Luz, en el país de
Canaán, y me bendijo allí. \bibverse{4} Me dijo: `¡Escucha! Te haré
próspero y haré que tu descendencia sea tan numerosa que te convertirás
en el antepasado de muchas naciones, y daré esta tierra a tus
descendientes para que la posean para siempre.'

\bibverse{5} Cuento como míos a tus dos hijos Efraín y Manasés que
nacieron aquí en Egipto antes de que yo llegara, así como Rubén y Simeón
son míos. \bibverse{6} Cualquier otro hijo que tengas después de ellos
será tuyo, y compartirás su herencia dentro de la tierra de sus hermanos
mayores. \bibverse{7} Hago esto porque\footnote{48.7 ``Hago esto
  porque'': añadido para proporcionar contexto. El sentido parece ser
  que debido a que Raquel murió al dar a luz teniendo a Benjamín, no
  pudo tener más hijos, así que Jacob en su mente ve a los hijos de José
  como una especie de recompensa.} trágicamente para mí, cuando
regresaba de Padán Aram, Raquel murió en Canaán, a cierta distancia de
Efratá. La enterré allí de camino a Efratá'' (también conocida como
Belén).

\bibverse{8} Israel vio a los hijos de José y dijo: ``¿Son estos son tus
hijos, entonces?''

\bibverse{9} ``Sí, estos son los hijos que Dios me dio aquí'', le dijo
José a su padre.

``Tráelos aquí para que pueda bendecirlos'', dijo.

\bibverse{10} La vista de Israel estaba fallando debido a su edad y no
podía ver bien, así que José los acercó a su padre, y él los besó y los
abrazó. \bibverse{11} Entonces Israel le dijo a José: ``Nunca pensé que
volvería a ver tu cara, y ahora Dios me ha dejado ver a tus hijos''.

\bibverse{12} José tomó a sus hijos de entre las rodillas de Israel, y
se inclinó con el rostro hacia el suelo. \bibverse{13} Entonces José
puso a Efraín a su derecha para que estuviera a la izquierda de Israel,
y a Manasés a su izquierda para que estuviera a la derecha de Israel, y
luego los trajo a Israel. \bibverse{14} Pero cuando Israel extendió sus
manos, las cruzó y colocó su mano derecha sobre Efraín, el hijo menor, y
colocó la izquierda sobre Manasés, el primogénito. \bibverse{15} Bendijo
a José, diciendo:

``Que el Dios que mi abuelo Abraham y mi padre adoraron, el Dios que me
ha cuidado como un pastor a lo largo de mi vida hasta ahora,
\bibverse{16} el Ángel que me ha salvado de todo tipo de problemas,
bendiga a estos muchachos. Que mi nombre y los nombres de mi abuelo
Abraham y de mi padre Isaac continúen a través de ellos, y que tengan
muchos descendientes que se extiendan por toda la tierra''.

\bibverse{17} José se sintió infeliz cuando vio que su padre había
puesto su mano derecha sobre Efraín, así que tomó la mano de su padre
para tratar de moverla de la cabeza de Efraín a la de Manasés.
\bibverse{18} ``Así no, padre, este es el primogénito; pon tu mano
derecha sobre su cabeza'', le dijo José.

\bibverse{19} Pero su padre se negó, diciendo: ``Yo sé lo que hago.
Manasés también se convertirá en un pueblo importante, pero su hermano
menor será más grande que él, y sus descendientes se convertirán en una
gran nación''.

\bibverse{20} Así que Israel los bendijo ese día y dijo: ``En el futuro,
el pueblo de Israel usará sus nombres para dar una bendición, diciendo:
`Que Dios los bendiga como lo hizo con Efraín y Manasés'. Al decir esto,
puso a Efraín antes que a Manasés.

\bibverse{21} Entonces Israel dijo a José: ``Voy a morir pronto, pero
Dios estará con ustedes y los devolverá a la tierra de sus padres.
\bibverse{22} También te doy algo además de lo que le doy a tus
hermanos: un trozo de tierra en la ladera de la montaña de
Siquem\footnote{48.22 La palabra utilizada aquí que significa
  ``hombro'', y se refiere tanto a la ladera de una montaña como a la
  ciudad de Siquem, que lleva el nombre de dicha ladera. En 33:18, se
  registra que Jacob compró un pedazo de tierra en Siquem, y en Josué
  24:32 se afirma que José fue enterrado allí más tarde. También se hace
  referencia en Juan 4:5 como la tierra que Jacob le dio a José.} que le
quité a los amorreos con mi espada y mi arco''.

\hypertarget{section-48}{%
\section{49}\label{section-48}}

\bibverse{1} Jacob entonces llamó a sus hijos y les dijo: ``Reúnanse
para que pueda decirles lo que les va a pasar en el futuro. \bibverse{2}
Vengan aquí, hijos de Jacob, y escuchen a su padre Israel.

\bibverse{3} Rubén: Tú eres mi primogénito, concebido cuando era fuerte,
nacido cuando era vigoroso. Estuviste por encima de todos los demás en
posición y en poder. \bibverse{4} Pero tú hierves como el agua, así que
ya no estarás más por encima de nadie, porque fuiste y te acostaste con
mi concubina;\footnote{49.4 Ver 35:22.}violaste mi lecho matrimonial.

\bibverse{5} Simeón y Levi son dos de la misma clase, usan sus armas
para la violencia destructiva.\footnote{49.5 Ver 34:25.} \bibverse{6} Me
niego a ser parte de sus decisiones; me niego a participar en lo que
hacen. Porque mataron a los hombres en su ira; lisiaron al ganado sólo
por diversión. \bibverse{7} Maldigo su ira porque es demasiado dura;
maldigo su furia porque es demasiado cruel. Separaré a sus descendientes
a través de Jacob; los dispersaré por todo Israel.

\bibverse{8} Judá: tus hermanos te alabarán. Derrotarás a tus enemigos.
Los hijos de tu padre se inclinarán ante ti en señal de respeto.
\bibverse{9} Mi hijo Judá es un joven león que vuelve después de
comersea su presa. Se agacha y se acuesta como un león. Así como un
león, ¿quién se atrevería a molestarlo? \bibverse{10} Judá siempre
sostendrá el cetro, y el bastón de la autoridad estará siempre a sus
pies hasta que venga Siloé\footnote{49.10 ``Siloé'': hay un considerable
  desacuerdo entre los comentaristas sobre esta palabra. Muchos ven esto
  como una profecía relacionada con el Mesías.}; las naciones le
obedecerán. \bibverse{11} Ata su asno a la vid, y el potro de su asno a
la mejor vid. Lava sus ropas en vino, sus túnicas en el jugo rojo de las
uvas.+ 49.11 La intención de este versículo es que los descendientes de
Judá tuvieran tal prosperidad que pudieran permitirse atar asnos a sus
viñas y tener tanto vino que pudieran lavar sus ropas con él.
\bibverse{12} Sus ojos brillan más que el vino, y sus dientes son más
blancos que la leche.

\bibverse{13} Zabulón vivirá a la orilla del mar y será un puerto para
los barcos; su territorio se extenderá hacia Sidón.

\bibverse{14} Isacar es un asno fuerte, acostado entre dos
alforjas.\footnote{49.14 ``Alforjas'': o, ``rediles''.} \bibverse{15} Ve
que el lugar donde descansa es bueno, y la tierra es encantadora, por lo
que está dispuesto a inclinar la espalda para aceptar la carga y
trabajar como esclavo.

\bibverse{16} Dan juzgará\footnote{49.16 Dan significa juez, ver Gén
  30:6.}a su pueblo como una de las tribus de Israel. \bibverse{17} Dan
será tan peligroso como una serpiente al lado del camino, una víbora por
el sendero que muerde el talón del caballo, haciendo caer a su jinete de
espaldas.

\bibverse{18} Confío en ti para que me salves, Señor.

\bibverse{19} Los jinetes atacarán a Gad, pero él atacará sus talones.

\bibverse{20} Aser tendrá una comida deliciosa, producirá comida de lujo
para la realeza.

\bibverse{21} Neftalí es un ciervo que puede correr libremente; da a luz
a hermosos cervatillos.\footnote{49.21 ``da a luz a hermosos
  cervatillos'': o ``transmite hermosas palabras''.}

\bibverse{22} José es un árbol fructífero, un árbol fructífero al lado
de un manantial, cuyas ramas trepan por la pared. \bibverse{23} Los
arqueros lo atacaron vilmente; le dispararon sus flechas con odio.
\bibverse{24} Pero él mantuvo su arco firme, y sus brazos y manos se
movieron rápidamente con la fuerza del Poderoso de Jacob, que se llama
el Pastor, la Roca de Israel. \bibverse{25} El Dios de tu padre te
ayudará y el Todopoderoso te bendecirá con bendiciones de los cielos de
arriba, con bendiciones de las profundidades abajo, con bendiciones para
muchos hijos.\footnote{49.25 ``Muchos hijos'': literalmente ``pechos y
  matriz''.} \bibverse{26} Las bendiciones que recibió tu padre fueron
mayores que las bendiciones de sus antepasados, más que las bendiciones
de las montañas eternas.+ 49.26 El hebreo de este versículo no está
claro.Que estén sobre la cabeza de José, en la frente del que se apartó
como líder de sus hermanos.

\bibverse{27} Benjamín es un lobo feroz. Por la mañana destruye a sus
enemigos,\footnote{49.27 ``Destruye a sus enemigos'': literalmente
  ``come de la presa.''} por la tarde divide el botín''.

\bibverse{28} Estas son todas las doce tribus de Israel, y esto es lo
que les dijo su padre al bendecirlas, cada una según sus respectivas
bendiciones.

\bibverse{29} Luego les dio las siguientes instrucciones: ``Voy a morir
pronto. Entiérrenme con mis antepasados en la cueva que está en el campo
de Efrón el hitita. \bibverse{30} Esta es la cueva que está en el campo
de Macpela, cerca de Mamré, en Canaán, y que Abraham compró junto con el
campo de Efrón el Hitita para tenerlo como lugar de sepultura.
\bibverse{31} Abraham y su esposa Sara fueron enterrados allí, Isaac y
su esposa Rebeca fueron enterrados allí, y yo enterré a Lea allí.
\bibverse{32} El campo y la cueva fueron comprados a los hititas''.

\bibverse{33} Cuando Jacob terminó de dar estas instrucciones levantó
los pies en el lecho, respiró por última vez y se unió a sus antepasados
en la muerte.

\hypertarget{section-49}{%
\section{50}\label{section-49}}

\bibverse{1} José fue y abrazó a su padre, llorando sobre él y
besándolo. \bibverse{2} Luego José instruyó a los médicos que trabajaban
para él que embalsamaran el cuerpo de su padre. Así que los médicos
embalsamaron a Israel. \bibverse{3} Esto tomó 40 días, el tiempo normal
para el proceso, y los egipcios lloraron por él durante 70 días.

\bibverse{4} Una vez terminado el tiempo de luto, José dijo a los
oficiales del Faraón: ``Si son tan amables, por favor hablen con el
Faraón en mi nombre y explíquenle que \bibverse{5} mi padre me hizo
hacer un juramento, diciéndome: `Debes enterrarme en la tumba que he
preparado para mí en Canaán'. Por favor, permíteme ir a enterrar a mi
padre y luego volveré''.

\bibverse{6} El Faraón respondió: ``Ve y entierra a tu padre como te
hizo jurar''.

\bibverse{7} José fue a enterrar a su padre, y todos los oficiales del
Faraón fueron con él, todos los consejeros principales del Faraón y
todos los líderes de Egipto, \bibverse{8} así como la familia de José,
sus hermanos y la familia de su padre. Sólo dejaron a los niños pequeños
y sus rebaños y manadas en Gosén. \bibverse{9} Fueron acompañados por
carros y jinetes, una procesión muy grande.

\bibverse{10} Cuando llegaron a la era de Atad, al otro lado del Jordán,
lloraron con gritos de dolor. José celebró una ceremonia de siete días
de luto por su padre allí. \bibverse{11} Los cananeos que vivían allí
vieron la ceremonia de duelo en la era de Atad. Dijeron: ``Este es un
momento muy triste de luto para los egipcios'', así que rebautizaron el
lugar como Abel-Mizraim,\footnote{50.11 ``Abel-Mizraim'': significa
  ``luto de los egipcios''.}que está al otro lado del Jordán.

\bibverse{12} Los hijos de Jacob hicieron lo que él les había ordenado.
\bibverse{13} Llevaron su cuerpo a Canaán y lo enterraron en la cueva de
Macpela, en el campo cerca de Mamre, el cual Abraham le había comprado a
Efrón el hitita como lugar de sepultura.

\bibverse{14} Después de enterrar a su padre, José y sus hermanos
regresaron a Egipto junto con todos los que habían ido con ellos.
\bibverse{15} Sin embargo, ahora que su padre había muerto, los hermanos
de José se preocuparon diciendo: ``Tal vez José nos guarde rencor y nos
pague por todas las cosas malas que le hicimos''.

\bibverse{16} Así que enviaron un mensaje a José para decirle: ``Antes
de que tu padre muriera, dio esta orden, \bibverse{17} `Esto es lo que
debes decirle a José: Perdona a tus hermanos sus pecados, las cosas
malas que te hicieron, tratándote de una manera tan desagradable.'
Ahora, por favor, perdona nuestros pecados, nosotros que somos siervos
del Dios de tu padre''. Cuando José recibió el mensaje, lloró.

\bibverse{18} Entonces sus hermanos vinieron y se postraron ante José y
le dijeron: ``¡Somos tus esclavos!''

\bibverse{19} ``¡No necesitan tener miedo!'' les dijo. ``No me pongo en
el lugar de Dios, ¿verdad? \bibverse{20} Aunque ustedes conspiraron
cosas malas para mí, Dios planeó para el bien, a fin de que muchas vidas
pudieran ser salvadas.\footnote{50.20 Ver 45:5, 45:7.} \bibverse{21} Así
que no se preocupen. Seguiré cuidando de ustedes y de sus hijos''.
Hablando amablemente así los calmó.

\bibverse{22} Y José permaneció en Egipto, junto con toda la familia de
su padre. Vivió hasta los 110 años, \bibverse{23} y vio tres
generaciones de su hijo Efraín, y los hijos de Maquir, el hijo de
Manasés, fueron puestos en su regazo cuando nacieron.

\bibverse{24} ``Voy a morir pronto'', les dijo José a sus hermanos,
``pero Dios estará con ustedes, y los llevará fuera de este país a la
tierra que juró dar a Abraham, Isaac y Jacob''.

\bibverse{25} José hizo jurar a los hijos de Israel, diciendo: ``Cuando
Dios venga a estar con ustedes, deben llevarse mis huesos cuando se
vayan''.+ 50.25 ``Cuando te vayas'': añadido para mayor claridad.
\bibverse{26} José murió cuando tenía 110 años. Después de que su cuerpo
fue embalsamado, fue colocado en un ataúd en Egipto.
