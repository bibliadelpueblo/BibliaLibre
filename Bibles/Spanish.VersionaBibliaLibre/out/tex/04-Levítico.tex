\hypertarget{section}{%
\section{1}\label{section}}

\bibverse{1} El Señor llamó a Moisés y le habló desde El Tabernáculo de
Reunión, diciendo: \bibverse{2} ``Ve y habla con los israelitas y diles:
'Cuando presenten una ofrenda al Señor, pueden traer como ofrenda un
animal del rebaño de ganado o del rebaño de ovejas o cabras.

\bibverse{3} Si su ofrenda es una ofrenda quemada de un rebaño de
ganado, debe ofrecer un macho sin ningún defecto. Llévalo a la entrada
del Tabernáculo de Reunión para que sea aceptado por el Señor.
\bibverse{4} Pon tu mano en la cabeza de la ofrenda quemada, para que
pueda ser aceptada en tu nombre para tu justificación.\footnote{\textbf{1:4}
  ``Para tu justificación'': o ``para tu expiación''. El concepto es
  que, debido al pecado, la relación con Dios se ha fracturado. Los
  diversos rituales descritos en el Levítico son símbolos de cómo la
  relación puede ser restaurada, o ``arreglada'' ante los ojos de Dios.
  Además, las personas y los objetos (por ejemplo, el altar) también
  pueden ser ``arreglados'' en el sentido de ser purificados, por lo que
  este término también se utiliza en esta traducción.} \bibverse{5}
Debes matar el toro en presencia del Señor, y los hijos de Aarón, los
sacerdotes, deben tomar la sangre y rociarla por todos los lados del
altar a la entrada del Tabernáculo de Reunión. \bibverse{6} Entonces
debes desollar la ofrenda quemada y cortarla en pedazos. \bibverse{7}
Los hijos del sacerdote Aarón encenderán un fuego en el altar y le
pondrán leña. \bibverse{8} Entonces los sacerdotes colocarán
cuidadosamente las piezas, incluyendo la cabeza y la grasa, en la madera
que se quema sobre el altar. \bibverse{9} Lavarás las entrañas y las
piernas con agua, y el sacerdote lo quemará todo en el altar como una
ofrenda quemada, una ofrenda de comida, para ser aceptada por el
Señor.\footnote{\textbf{1:9} ``Aceptada por el Señor'': Literalmente,
  ``con un aroma agradable''. Esta es una ``extensión figurativa'' de
  este proceso sensorial que indica que de la misma manera que nos gusta
  algo, y por extensión, lo aceptamos, también lo hace Dios. También en
  los versículos 13 y 17, etc.}

\bibverse{10} Situ ofrenda es una ofrenda quemada de un rebaño, ya sea
de ovejas o de cabras, debes ofrecer un macho sin ningún defecto.
\bibverse{11} Debes matarlo en el lado norte del altar en presencia del
Señor, y los hijos de Aarón, los sacerdotes, deben tomar la sangre y
rociarla en todos los lados del altar. \bibverse{12} Entonces lo
cortarás en pedazos, y el sacerdote colocará cuidadosamente los pedazos,
incluyendo la cabeza y la grasa, en la madera que se quema sobre el
altar. \bibverse{13} Lavarás las entrañas y las piernas con agua, y el
sacerdote lo quemará todo en el altar como un holocausto, una ofrenda de
comida, para ser aceptada por el Señor.

\bibverse{14} Si tu ofrenda al Señor es un holocausto de pájaros, debes
ofrecer una tórtola o una paloma joven. \bibverse{15} El sacerdote lo
llevará al altar, le quitará la cabeza y lo quemará en el altar. Su
sangre será drenada en el lado del altar. \bibverse{16} Debe quitar el
buche y las plumas, y tirarlas al lado este del altar en el montón de
cenizas. \bibverse{17} Lo abrirá por las alas, pero no completamente. El
sacerdote lo quemará en el altar, sobre la madera ardiente. Es una
ofrenda quemada, una ofrenda de comida, agradable Señor.

\hypertarget{section-1}{%
\section{2}\label{section-1}}

\bibverse{1} Cuando traigas una ofrenda de grano al Señor, tu ofrenda
debe ser de la mejor harina. Vierte aceite de oliva y ponle incienso,
\bibverse{2} y llévaselo a los hijos de Aarón, los sacerdotes. El
sacerdote tomará un puñado de la mezcla de harina y aceite de oliva, así
como todo el incienso, y lo quemará como un ``recordatorio'' en el
altar, una ofrenda de comida, agradable al Señor. \bibverse{3} El resto
de la ofrenda de grano es para Aarón y sus hijos; es una parte muy
sagrada de las ofrendas dadas al Señor como ofrendas de comida.

\bibverse{4} Situ ofrenda es de grano cocido en un horno, debe ser hecha
de harina fina sin usar levadura. Pueden ser pasteles mezclados con
aceite de oliva o barquillos con aceite de oliva untado en ellos.
\bibverse{5} Si tu ofrenda es una ofrenda de grano cocido en una
plancha, debe ser hecha de harina fina mezclada con aceite de oliva sin
usar levadura. \bibverse{6} Rómpelo y vierte aceite de oliva sobre él;
es una ofrenda de grano. \bibverse{7} Si tu ofrenda es una ofrenda de
grano cocido en una sartén, debe ser de harina fina con aceite de oliva.
\bibverse{8} Trae al Señor la ofrenda de grano hecha de cualquiera de
estas maneras. Preséntala al sacerdote, quien la llevará al altar.
\bibverse{9} El sacerdote debe tomar ``el recordatorio'' de la ofrenda
de grano y quemarla en el altar como una ofrenda de comida, agradable
para el Señor. \bibverse{10} El resto de la ofrenda de grano es para
Aarón y sus hijos; es la parte más sagrada de las ofrendas de comida
dadas al Señor.

\bibverse{11} Ninguna ofrenda de grano que traigas ante el Señor puede
hacerse con levadura. No quemes ninguna levadura o miel como ofrenda al
Señor. \bibverse{12} Puedes dárselas al Señor cuando presentes tus
ofrendas de primicias, pero no deben ser ofrecidas en el altar para ser
aceptadas por el Señor. \bibverse{13} Todas tus ofrendas de granos deben
ser sazonadas con sal. No dejes la sal del pacto de Dios fuera de tu
ofrenda de grano. Añade sal a todas tus ofrendas.

\bibverse{14} Cuando traigas al Señor una ofrenda de primicias de grano,
ofrece cabezas aplastadas de grano nuevo asadas en el fuego.
\bibverse{15} Pon aceite de oliva e incienso en él; es una ofrenda de
grano. \bibverse{16} El sacerdote quemará el ``recordatorio'' del grano
triturado y el aceite de oliva, así como todo su incienso, como ofrenda
de alimento al Señor''.

\hypertarget{section-2}{%
\section{3}\label{section-2}}

\bibverse{1} ``Cuando quieras hacer una ofrenda de paz y ofrezcas un
animal de una manada de ganado, ya sea macho o hembra, debes presentar
uno sin ningún defecto ante el Señor. \bibverse{2} Pon tu mano en la
cabeza de la ofrenda y mátala a la entrada del Tabernáculo de Reunión.
Entonces los hijos de Aarón los sacerdotes rociarán la sangre por todos
los lados del altar. \bibverse{3} De la ofrenda de paz debes traer una
ofrenda de comida al Señor: toda la grasa que cubre las entrañas,
\bibverse{4} ambos riñones con la grasa en ellos por los lomos, y la
mejor parte del hígado, que debes quitar junto con los riñones.
\bibverse{5} Los hijos de Aarón deben quemar esto en el altar sobre la
ofrenda quemada que está sobre la madera ardiente, como una ofrenda de
comida, agradable al Señor.

\bibverse{6} Cuando quieras hacer una ofrenda de paz y ofrezcas un
animal de un rebaño de ovejas o cabras, ya sea macho o hembra, debes
presentar uno sin ningún defecto ante el Señor. \bibverse{7} Si das un
cordero como ofrenda, debes presentarlo ante el Señor. \bibverse{8} Pon
tu mano en la cabeza de la ofrenda y mátala delante del Tabernáculo de
Reunión. Entonces los hijos de Aarón los sacerdotes rociarán la sangre a
todos los lados del altar. \bibverse{9} De la ofrenda de paz debes traer
una ofrenda de comida al Señor hecha de su grasa: la cola
entera\footnote{\textbf{3:9} Las colas de las ovejas en Israel eran muy
  grandes y se consideraban un manjar.}removida de la base de la
rabadilla, toda la grasa que cubre el interior, \bibverse{10} ambos
riñones con la grasa en ellos por los lomos, y la mejor parte del
hígado, que debes quitar junto con los riñones. \bibverse{11} Entonces
el sacerdote debe quemar esto en el altar como una ofrenda de comida,
una ofrenda de comida al Señor.

\bibverse{12} Si tu ofrenda es una cabra, debes presentarla ante el
Señor. \bibverse{13} Pon tu mano en su cabeza y mátalo frente al
Tabernáculo de Reunión. Entonces los hijos de Aarón, los sacerdotes,
rociarán la sangre a todos los lados del altar. \bibverse{14} De tu
ofrenda debes presentar una ofrenda de comida al Señor hecha de toda la
grasa que cubre las entrañas, \bibverse{15} ambos riñones con la grasa
en ellos por los lomos, y la mejor parte del hígado, que debes quitar
junto con los riñones. \bibverse{16} Entonces el sacerdote debe quemar
esto en el altar como una ofrenda de comida, una ofrenda al Señor usando
fuego. Toda la grasa es para el Señor. \bibverse{17} No debes comer
ninguna grasa o sangre. Esta regulación es para todos los tiempos y para
todas las generaciones futuras dondequiera que vivan''.

\hypertarget{section-3}{%
\section{4}\label{section-3}}

\bibverse{1} Entonces el Señor le dijo a Moisés: \bibverse{2} ``Dile a
los israelitas que estas son las reglas para manejar los casos de
aquellos que pecan involuntariamente contra alguno de los mandamientos
del Señor y hacen lo que no está permitido.

\bibverse{3} Si es el Sumo Sacerdote quien peca y trae la culpa sobre
todos, debe presentar al Señor un novillo sin defectos como ofrenda por
su pecado. \bibverse{4} Debes llevar el toro a la entrada del
Tabernáculo de Reunión ante el Señor, poner su mano sobre su cabeza y
matarlo ante el Señor. \bibverse{5} Entonces el sumo sacerdote llevará
parte de la sangre del toro al Tabernáculo de Reunión. \bibverse{6} El
sumo sacerdote mojará su dedo en la sangre y rociará un poco de ella
siete veces delante el Señor, frente del velo del santuario.
\bibverse{7} El sacerdote pondrá un poco de sangre sobre los cuernos del
altar de incienso aromático que está ante el Señor en el Tabernáculo de
Reunión. El resto de la sangre del toro la derramará en el fondo del
altar de los holocaustos, a la entrada del Tabernáculo de Reunión.
\bibverse{8} Entonces quitará toda la grasa del toro de la ofrenda por
el pecado: toda la grasa que cubre las entrañas, \bibverse{9} ambos
riñones con la grasa en ellos por los lomos, y la mejor parte del
hígado, que debe eliminar junto con los riñones \bibverse{10} de la
misma manera que la grasa se quita del toro de la ofrenda de paz.
Entonces el sacerdote quemará esto en el altar de la ofrenda quemada.

\bibverse{11} Pero la piel del toro, toda su carne, cabeza, patas,
interiores y desechos, \bibverse{12} y todo el resto, tiene que llevarlo
fuera del campamento a un lugar que esté ceremonialmente limpio, donde
se arrojen las cenizas, y debe quemarlo en un fuego de leña allí en el
montón de cenizas.

\bibverse{13} Si todo el pueblo de Israel se extravía sin querer, y
aunque no sean conscientes de hacer lo que no está permitido por ninguno
de los mandamientos del Señor, siguen siendo todos culpables.
\bibverse{14} Cuando se den cuenta de su pecado, deben traer un toro
joven como ofrenda por el pecado y presentarlo ante el Tabernáculo de
Reunión. \bibverse{15} Los ancianos de Israel pondrán sus manos sobre su
cabeza y lo matarán delante el Señor. \bibverse{16} Entonces el sumo
sacerdote llevará parte de la sangre del toro al Tabernáculo de Reunión.
\bibverse{17} Mojará su dedo en la sangre y lo rociará siete veces ante
el Señor delante del velo. \bibverse{18} Pondrá un poco de sangre en los
cuernos del altar que está delante del Señor en el Tabernáculo de
Reunión. Luego derramará el resto de la sangre del toro en el fondo del
altar de los holocaustos a la entrada del Tabernáculo de Reunión.
\bibverse{19} Luego le quitará toda la grasa al toro y lo quemará en el
altar. \bibverse{20} Ofrecerá este toro de la misma manera que lo hizo
para la ofrenda por el pecado. Así es como el sacerdote los expiará, y
serán perdonados. \bibverse{21} Entonces tomará el toro fuera del
campamento y lo quemará, tal y como quemó el toro anteriormente
mencionado. Es la ofrenda por el pecado de todo el pueblo.

\bibverse{22} Si un líder peca involuntariamente y hace lo que no está
permitido por ninguno de los mandamientos del Señor su Dios, es
culpable. \bibverse{23} Cuando se dé cuenta de su pecado, debe traer un
macho cabrío sin defectos como ofrenda. \bibverse{24} Debe poner su mano
en la cabeza del cabrito y matarlo en el lugar donde se presenta la
ofrenda quemada ante el Señor. Es una ofrenda por el pecado.
\bibverse{25} Entonces el sacerdote debe tomar parte de la sangre de la
ofrenda por el pecado con su dedo y ponerla en los cuernos del altar de
la ofrenda quemada, y derramar el resto de la sangre en la base del
altar. \bibverse{26} Quemará toda su grasa en el altar como la grasa de
las ofrendas de paz. De esta manera el sacerdote expiará el pecado del
hombre y será perdonado.

\bibverse{27} Si cualquier otro israelita peca involuntariamente y hace
lo que no está permitido por ninguno de los mandamientos del Señor su
Dios, es culpable. \bibverse{28} Cuando se dé cuenta de su pecado, debe
traer una cabra sin defectos como ofrenda por ese pecado. \bibverse{29}
Debe poner su mano en la cabeza de la ofrenda por el pecado y matarla en
el lugar del holocausto. \bibverse{30} Entonces el sacerdote debe tomar
un poco de su sangre con su dedo y ponerla en los cuernos del altar de
la ofrenda quemada, y derramar el resto de la sangre en la base del
altar. \bibverse{31} Le quitará toda su grasa como la grasa de las
ofrendas de paz y la quemará en el altar y será aceptada por el Señor.
De esta manera el sacerdote expiará el pecado del hombre y será
perdonado.

\bibverse{32} Si trae un cordero como ofrenda por el pecado, debe traer
una hembra sin defectos. \bibverse{33} Debe poner su mano en la cabeza
de la ofrenda por el pecado y matarla como ofrenda por el pecado en el
lugar donde se hace el holocausto. \bibverse{34} Entonces el sacerdote
debe tomar un poco de su sangre con su dedo y ponerla en los cuernos del
altar de la ofrenda quemada, y derramar el resto de la sangre en la base
del altar. \bibverse{35} Le quitará toda su grasa como la grasa del
cordero se quita de las ofrendas de paz y la quemará en el altar y será
aceptada por el Señor. De esta forma el sacerdote expiará el pecado del
hombre, y será perdonado.

\hypertarget{section-4}{%
\section{5}\label{section-4}}

\bibverse{1} Si tú pecas por no suministrar las pruebas necesarias en un
caso legal, ya sea que tú mismo hayas visto o escuchado algo al
respecto, eres responsable de tu culpabilidad. \bibverse{2} Si tocas
algo sucio como el cadáver de un animal salvaje impuro\footnote{\textbf{5:2}
  El concepto de impureza ceremonial es frecuente aquí y en otros libros
  del Antiguo Testamento. Es principalmente un concepto ``religioso''
  pero se basa en algunas áreas relacionadas con los aspectos de salud e
  higiene.}o animales de granja o bichos, incluso si no eres consciente
de ello, serás impuro y culpable. \bibverse{3} Si tocas algo sucio de
alguien que es sucio, incluso si no eres consciente de ello, serás
culpable. \bibverse{4} Si juras imprudentemente hacer algo, (ya sea
bueno o malo, y de cualquier manera que la gente pueda jurar
impulsivamente), incluso si no eres consciente de que está mal, cuando
finalmente te das cuenta, serás culpable.

\bibverse{5} Si te vuelves culpable de una de estas formas, debes
confesar tu pecado, \bibverse{6} y debes llevar tu ofrenda de culpa de
un cordero o cabra hembra al Señor como una ofrenda por tu pecado. El
sacerdote expiará tu pecado. \bibverse{7} Si no te alcanza para comprar
un cordero, puedes ofrecer al Señor como compensación por tu pecado dos
tórtolas o dos palomas jóvenes, una como ofrenda por el pecado y otra
como holocausto. \bibverse{8} Debes llevarlos al sacerdote, quien
presentará el primero como la ofrenda por el pecado. Él debe arrancarle
la cabeza del cuello sin quitarla completamente. \bibverse{9} Luego debe
rociar parte de la sangre de la ofrenda por el pecado en el lado del
altar mientras el resto de la sangre es derramada en la base del altar.
Es una ofrenda por el pecado. \bibverse{10} El sacerdote debe entonces
preparar la segunda tórtola como holocausto según las normas. De esta
manera el sacerdote te justificará por tus pecados, y serás perdonado.

\bibverse{11} Si no te alcanza para comprar dos tórtolas o dos pichones,
puedes traer un décimo de efa de la mejor harina como ofrenda por el
pecado. No le pongas aceite de oliva o incienso, porque es una ofrenda
por el pecado. \bibverse{12} Llévala al sacerdote, que tomará un puñado
como ``recordatorio'' y lo quemará en el altar sobre las ofrendas
quemadas al Señor. Es una ofrenda por el pecado. \bibverse{13} Así es
como el sacerdote expiará cualquiera de estos pecados que hayas
cometido, y serás perdonado. El resto de la ofrenda pertenecerá al
sacerdote, al igual que la ofrenda de grano''.

\bibverse{14} El Señor le dijo a Moisés: \bibverse{15} Si alguno de
ustedes descuida involuntariamente todo lo que el Señor ha declarado que
le pertenece y es santo,\footnote{\textbf{5:15} Esto incluiría todo lo
  que el Señor ha dicho que su pueblo debe darle, incluyendo, por
  ejemplo, primicias, diezmos, el primogénito, etc.}debes llevar tu
ofrenda de culpabilidad al Señor: un carnero sin defectos de tu rebaño o
sólo uno de valor equivalente en siclos de plata (según el estándar del
siclo del santuario). Es una ofrenda por la culpa. \bibverse{16} En
cuanto a cualquier requisito sagrado que no hayas aportado, debes pagar
una compensación añadiéndole un quinto de su valor y luego dárselo al
sacerdote, que lo arreglará con el carnero como ofrenda por la culpa, y
serás perdonado. \bibverse{17} Si pecas y quebrantas alguno de los
mandamientos del Señor, aunque no seas consciente de ello, sigues siendo
culpable y asumes la responsabilidad de tu culpa. \bibverse{18} Debes
llevar al sacerdote un carnero sin defectos y de valor apropiado como
ofrenda de culpabilidad. Entonces expiará por ti el mal que hiciste en
la ignorancia, y serás perdonado. \bibverse{19} Es una ofrenda de culpa
porque fuiste culpable en lo que concierne al Señor''.

\hypertarget{section-5}{%
\section{6}\label{section-5}}

\bibverse{1} El Señor le dijo a Moisés: \bibverse{2} ``Si pecas contra
el Señor, rompiendo tu compromiso con él, entonces esto es lo que debe
suceder.\footnote{\textbf{6:2} ``Entonces esto es lo que debe suceder:''
  añadido para mayor claridad.}Puede que le hayas mentido a tu vecino
sobre algo que cuidabas para ellos, o sobre algún depósito pagado, sobre
algo que fue robado, o tal vez tratabas de engañarlos. \bibverse{3}
Puede que hayas encontrado una propiedad que alguien perdió, y mentiste
e hiciste declaraciones falsas sobre ello, o has pecado de otras maneras
en tales situaciones. \bibverse{4} Si has pecado y te vuelves culpable
debes devolver lo que has robado o engañado a tus víctimas, el depósito
que tomaste, la propiedad perdida que encontraste, \bibverse{5} o
cualquier otra cosa que deba ser devuelta y sobre la que hayas mentido.
Debes pagar la compensación completa más una quinta parte del valor, y
dársela al dueño tan pronto como aceptes que eres culpable de pecado.
\bibverse{6} Luego debes llevar al sacerdote tu ofrenda de culpabilidad
para el Señor: un carnero sin defectos del valor apropiado del rebaño.
\bibverse{7} Así es como el sacerdote te hará justicia ante el Señor, y
se te perdonarán todos los pecados que hayas cometido y de los que seas
culpable''.

\bibverse{8} El Señor le dijo a Moisés: \bibverse{9} ``Instruye a Aarón
y a sus hijos respecto al holocausto: La ofrenda quemada debe dejarse
ardiendo sobre altar durante toda la noche hasta la mañana, y el fuego
en el altar debe mantenerse encendido. \bibverse{10} El sacerdote se
pondrá sus ropas de lino y su ropa interior, y tomará del altar las
cenizas grasosas del holocausto que el fuego ha quemado y las pondrá al
lado del altar. \bibverse{11} Luego se cambiará de ropa y llevará las
cenizas fuera del campamento a un lugar que esté ceremonialmente limpio.
\bibverse{12} El fuego del altar debe mantenerse encendido, no dejarlo
apagarse. Cada mañana el sacerdote debe añadir leña al fuego, colocar
cuidadosamente la ofrenda quemada en él, y quemar las partes gordas de
las ofrendas de paz en él. \bibverse{13} El fuego debe mantenerse
encendido en el altar continuamente, no lo dejes apagar.

\bibverse{14} Estas son las regulaciones para la ofrenda de grano: Los
hijos de Aarón deben presentarla ante el Señor, delante del altar.
\bibverse{15} El sacerdote quitará un puñado de la mejor harina mezclada
con aceite de oliva, así como todo el incienso de la ofrenda de grano, y
quemará la ``parte recordatoria'' en el altar para ser aceptada por el
Señor. \bibverse{16} El resto es para que Aarón y sus hijos lo coman.
Debe comerse sin levadura en un lugar santo, el patio del Tabernáculo de
Reunión. \bibverse{17} No debe ser horneado con levadura. Lo he
proporcionado como su parte de mis ofrendas de comida. Es muy sagrado,
como la ofrenda por el pecado y la ofrenda por la culpa. \bibverse{18}
Cualquiera de los descendientes masculinos de Aarón puede comerlo. Es
una asignación permanente de las ofrendas de comida al Señor para las
generaciones futuras. Todo lo que los toque se convertirá en sagrado''.

\bibverse{19} El Señor le dijo a Moisés: \bibverse{20} ``Esta es la
ofrenda que Aarón y sus hijos deben presentar al Señor cuando sean
ungidos: una décima parte de una efa de la mejor harina como ofrenda de
grano habitual, la mitad por la mañana y la mitad por la tarde.
\bibverse{21} Cocínalo con aceite de oliva en una plancha. Llevarlo bien
amasado y presentarlo como una ofrenda de grano partido en pedazos, para
ser aceptado por el Señor. \bibverse{22} Debe ser cocinado por el
sacerdote que es uno de los descendientes de Aarón y que debe ser ungido
para tomar su lugar. En este caso, como está asignado permanentemente al
Señor, debe ser quemado completamente. \bibverse{23} Cada ofrenda de
grano para un sacerdote debe ser quemada completamente. No debe ser
comida''.

\bibverse{24} El Señor le dijo a Moisés: \bibverse{25} ``Dile a Aarón y
a sus hijos que estas son las normas para la ofrenda por el pecado. La
ofrenda por el pecado debe ser matada donde el holocausto es matado ante
el Señor, y es muy sagrada. \bibverse{26} El sacerdote que ofrece la
ofrenda por el pecado debe comerla. Debe comerse sin levadura en un
lugar santo, el patio del Tabernáculo de Reunión. \bibverse{27} Todo lo
que lo toque se convertirá en sagrado y si algo de la sangre se salpica
en la ropa, debe lavarse en un lugar santo. \bibverse{28} La olla de
barro que se usa para hervir la ofrenda por el pecado debe romperse. Si
se hierve en una olla de bronce, la olla debe ser limpiada a fondo y
lavada con agua. \bibverse{29} Cualquier varón entre los sacerdotes
puede comerla, es muy sagrada. \bibverse{30} Pero no se puede comer
ninguna ofrenda por el pecado si su sangre ha sido llevada al
Tabernáculo de Reunión como medio para arreglar las cosas en el Lugar
Santo. En ese caso debe ser quemada''.

\hypertarget{section-6}{%
\section{7}\label{section-6}}

\bibverse{1} ``Estas son las regulaciones para la ofrenda de la culpa,
es muy sagrada. \bibverse{2} La ofrenda de culpa debe ser matada donde
se mata el holocausto, y el sacerdote rociará su sangre a todos los
lados del altar. \bibverse{3} Toda la grasa de ella será ofrecida: la
cola gorda, la grasa que cubre las entrañas, \bibverse{4} ambos riñones
con la grasa sobre ellos por los lomos, y la mejor parte del hígado, que
el sacerdote debe quitar junto con los riñones. \bibverse{5} Los quemará
en el altar como ofrenda al Señor; es una ofrenda por la culpa.
\bibverse{6} Cualquier varón entre los sacerdotes puede comerla. Debe
comerse en un lugar santo, es muy sagrado. \bibverse{7} La ofrenda por
la culpa es como la ofrenda por el pecado; las normas son las mismas
para ambas. El sacerdote que presenta la ofrenda que ``hace las cosas
bien'' debe tenerla. \bibverse{8} En el caso de los holocaustos
ordinarios, el sacerdote debe tener la piel del animal. \bibverse{9} De
la misma manera, todas las ofrendas de grano que se cocinan en un horno
o en una cacerola o en una plancha son para el sacerdote que las
presenta, \bibverse{10} y todas las ofrendas de grano, ya sea mezcladas
con aceite de oliva o secas, son para todos los descendientes de Aarón.

\bibverse{11} Estas son las reglas para la ofrenda de paz que puedes
presentar al Señor. \bibverse{12} Sila ofrecen con espíritu de
agradecimiento, entonces junto con el sacrificio de acción de gracias,
deben ofrecer pan, obleas y pasteles bien amasados de la mejor harina,
todo ello hecho sin levadura y mezclado o cubierto con aceite de oliva.
\bibverse{13} Además de tu ofrenda de paz de acción de gracias de los
panes hechos sin levadura, presentarás una ofrenda de panes hechos con
levadura. \bibverse{14} Presentenuno de cada tipo de pan de la ofrenda
como contribución al Señor. Es para el sacerdote que rocía la sangre de
la ofrenda de paz. \bibverse{15} La carne del sacrificio de tu ofrenda
de paz de acción de gracias debe comerse el mismo día que la ofrezcas.
No dejes nada de eso hasta la mañana. \bibverse{16} Si el sacrificio que
ofreces es para pagar un voto o una ofrenda voluntaria, se comerá el día
que presentes tu sacrificio, pero lo que quede puede comerse al día
siguiente. \bibverse{17} Sin embargo, cualquier carne del sacrificio que
quede al tercer día debe ser quemada. \bibverse{18} Si comes algo de la
carne de tu ofrenda de paz al tercer día, no será aceptada. No recibirás
crédito por ofrecerla. De hecho, será tratada como algo asqueroso, y
cualquiera que la coma será responsable de su culpa.

\bibverse{19} Si esta carne toca algo impuro no debe ser comida; debe
ser quemada. Esta carne puede ser consumida por aquellos que están
ceremonialmente limpios. \bibverse{20} Si alguien que es inmundo come
carne de la ofrenda de paz dada al Señor, debe ser expulsado de su
pueblo. \bibverse{21} Cualquiera que toque algo impuro, ya sea de una
persona, un animal impuro o una cosa inmunda repugnante, y luego coma
carne de la ofrenda de paz dada al Señor, debe ser expulsado de su
pueblo''.

\bibverse{22} El Señor le dijo a Moisés: \bibverse{23} Dales estas
instrucciones a los israelitas. Diles: `No debes comer nada de la grasa
de un toro, una oveja o una cabra. \bibverse{24} Puedes usar la grasa de
un animal encontrado muerto o muerto por bestias salvajes para cualquier
propósito que desees, pero no debes comerla. \bibverse{25} Cualquiera
que coma la grasa de un animal de una ofrenda de comida presentada al
Señor debe ser expulsado de su pueblo. \bibverse{26} No deben comer la
sangre de ningún pájaro o animal en ninguno de sus hogares.
\bibverse{27} Cualquiera que coma sangre debe ser expulsado de su
pueblo.'''

\bibverse{28} Entonces el Señor le dijo a Moisés: \bibverse{29} Dales
estas instrucciones a los israelitas. Diles que si presentas una ofrenda
de paz al Señor debes traer parte de ella como un regalo especial para
el Señor. \bibverse{30} Debes traer personalmente las ofrendas de comida
al Señor; deben traer la grasa así como el pecho, y mecer el pecho como
ofrenda mecida ante el Señor. \bibverse{31} El sacerdote quemará la
grasa en el altar, pero el pecho es para Aarón y sus hijos.
\bibverse{32} Dale el muslo derecho al sacerdote como contribución de tu
ofrenda de paz. \bibverse{33} El sacerdote como descendiente de Aarón
que ofrece la sangre y la grasa de la ofrenda de paz tiene el muslo
derecho como su parte. \bibverse{34} He requerido de los israelitas el
pecho de la ofrenda mecida y la contribución del muslo de sus ofrendas
de paz, y se las he dado a Aarón el sacerdote y a sus hijos como su
parte de los israelitas para siempre''.

\bibverse{35} Esta es la parte de las ofrendas de alimentos entregadas
al Señor que pertenece a Aarón y sus hijos desde el día en que fueron
designados para servir al Señor como sacerdotes. \bibverse{36} Desde el
día en que fueron ungidos, el Señor ordenó que esto les fuera dado por
los hijos de Israel. Es su parte para las generaciones futuras.

\bibverse{37} Estas son las regulaciones con respecto al holocausto, la
ofrenda de grano, la ofrenda por el pecado, la ofrenda por la culpa, la
ofrenda de ordenación y la ofrenda de paz. \bibverse{38} El Señor se las
dio a Moisés en el Monte Sinaí en el momento en que ordenó a los
israelitas que le dieran sus ofrendas en el desierto del Sinaí.

\hypertarget{section-7}{%
\section{8}\label{section-7}}

\bibverse{1} El Señor le dijo a Moisés: \bibverse{2} Ve con Aarón y sus
hijos, y toma sus vestidos sacerdotales, el aceite de la unción, el toro
de la ofrenda por el pecado, dos carneros y la cesta de los panes sin
levadura, \bibverse{3} y haz que todos se reúnan a la entrada del
Tabernáculo de Reunión''. \bibverse{4} Moisés hizo lo que le ordenó el
Señor, y todos se reunieron a la entrada del Tabernáculo de Reunión.
\bibverse{5} Moisés les dijo: ``Lo siguiente es lo que el Señor ha
ordenado que se haga''.

\bibverse{6} Moisés hizo pasar a Aarón y a sus hijos y los lavó con
agua. \bibverse{7} Vistió a Aarón con la túnica, le ató el cinto, le
puso el manto y luego el efod. Ató la cintura del efod alrededor de él,
sujetándolo. \bibverse{8} Luego Moisés sujetó el pectoral a Aarón y puso
el Urim y Tumim en el pectoral. \bibverse{9} Puso el turbante en la
cabeza de Aarón y colocó la placa de oro, la corona sagrada, en la parte
delantera del turbante, como el Señor le había ordenado.

\bibverse{10} Entonces Moisés tomó el aceite de la unción y ungió el
tabernáculo y todo lo que había en él para dedicarlo todo. \bibverse{11}
Roció siete veces el aceite sobre el altar para ungirlo y todos sus
utensilios, así como la palangana con su soporte para dedicarlo.

\bibverse{12} Moisés derramó parte del aceite de la unción sobre la
cabeza de Aarón para ungirlo y dedicarlo.

\bibverse{13} Entonces Moisés hizo que los hijos de Aarón se acercaran,
les vistió con sus túnicas, les ató fajas y les envolvió con cintas para
la cabeza, como el Señor le había ordenado. \bibverse{14} Moisés trajo
el toro para la ofrenda por el pecado, y Aarón y sus hijos pusieron sus
manos sobre su cabeza. \bibverse{15} Moisés entonces mató al toro y usó
parte de la sangre para untarla en los cuatro cuernos del altar con su
dedo, para de esta manera consagrarlo y purificarlo. Luego vertió el
resto de la sangre en la parte inferior del altar, y consagró el altar
para poder hacer expiación por el pueblo sobre él.

\bibverse{16} Moisés tomó toda la grasa que cubre las entrañas, la mejor
parte del hígado, ambos riñones con la grasa en ellos, la quemó toda en
el altar. \bibverse{17} Pero el resto del toro - la piel, la carne y los
desechos - lo quemó fuera del campamento, como el Señor le había
indicado.

\bibverse{18} Moisés trajo el carnero para el holocausto, y Aarón y sus
hijos pusieron sus manos sobre su cabeza. \bibverse{19} Moisés mató el
carnero y roció la sangre a todos los lados del altar. \bibverse{20}
Dividió el carnero en pedazos y quemó la cabeza, los pedazos y la grasa.
\bibverse{21} Lavó las entrañas y las piernas con agua y quemó todo el
carnero en el altar como una ofrenda quemada, una ofrenda de comida para
ser aceptada por el Señor, como el Señor le había ordenado a Moisés.

\bibverse{22} Moisés trajo el segundo carnero, el carnero de la
ordenación, y Aarón y sus hijos pusieron sus manos sobre su cabeza.
\bibverse{23} Moisés mató al carnero y tomó un poco de su sangre. La
puso en el lóbulo de la oreja derecha de Aarón, en el pulgar de su mano
derecha y en el dedo gordo de su pie derecho. \bibverse{24} Entonces
Moisés hizo que Aarón y sus hijos se acercaran y pusieran un poco de la
sangre en el lóbulo de su oreja derecha, en los pulgares de su mano
derecha y en los dedos gordos de su pie derecho. Luego roció la sangre a
todos los lados del altar. \bibverse{25} Moisés tomó la grasa,
incluyendo la cola gorda, toda la grasa de las entrañas, la mejor parte
del hígado, ambos riñones con la grasa encima Moisés tomó la grasa junto
con el muslo derecho. \bibverse{26} Tomó una barra de pan sin levadura,
una barra hecha con aceite de oliva y una oblea de la cesta de pan sin
levadura que estaba en la presencia del Señor. Los colocó encima de las
porciones de grasa y en el muslo derecho. \bibverse{27} Luego se los dio
a Aarón y a sus hijos, y los agitó ante el Señor como ofrenda mecida.
\bibverse{28} Después de esto, Moisés los tomó y los quemó en el altar
con el holocausto. Esta era una ofrenda de ordenación, una ofrenda de
comida para ser aceptada por el Señor. \bibverse{29} Moisés entonces
tomó el pecho, su parte del carnero de ordenación, y lo agitó ante el
Señor como ofrenda mecida, como el Señor le había ordenad.

\bibverse{30} Moisés tomó entonces parte del aceite de la unción y parte
de la sangre del altar. Roció ambos sobre las ropas de Aarón y sus
hijos. Así es como dedicó las ropas de Aarón y sus hijos.

\bibverse{31} Moisés dijo a Aarón y a sus hijos: ``Deben hervir la carne
a la entrada del Tabernáculo de Reunión, y luego comérosla allí con el
pan que está en el cesto de las ofrendas para la ordenación, como
ordené: ``Es para que coman Aarón y sus hijos''. \bibverse{32} Después
debes quemar lo que queda de la carne y el pan.

\bibverse{33} No deben abandonar la entrada del Tabernáculo de Reunión
durante siete días hasta que la ceremonia de ordenación haya terminado,
porque la ordenación tardará siete días. \bibverse{34} Lo que se ha
hecho hoy ha sido ordenado por el Señor como un medio para justificarte.
\bibverse{35} Deben permanecer a la entrada del Tabernáculo de Reunión
durante siete días, día y noche, y seguir las órdenes del Señor para que
no mueran, porque esto es lo que se me ha mandado a hacer''.

\bibverse{36} Aarón y sus hijos hicieron todo lo que el Señor les ordenó
a través de Moisés.

\hypertarget{section-8}{%
\section{9}\label{section-8}}

\bibverse{1} Al octavo día después de la ordenación,\footnote{\textbf{9:1}
  ``Después de la ordenación'': añadido para mayor claridad.}Moisés
llamó a Aarón y a sus hijos, y a los ancianos de Israel, para reunirse
con él. \bibverse{2} Le dijo a Aarón: ``Debes traer un novillo como
ofrenda por el pecado y un carnero como holocausto, ambos sin defectos,
y presentarlos ante el Señor. \bibverse{3} Entonces les dijo a los
israelitas: ``Traigan las siguientes ofrendas: un macho cabrío como
sacrificio por el pecado; un becerro y un cordero, (ambos de un año y
sin defectos), para un holocausto; \bibverse{4} un toro y un carnero
como sacrificio de paz para presentarlos ante el Señor; y una ofrenda de
grano mezclado con aceite de oliva. Haced esto porque hoy el Señor se va
a revelar a ustedes hoy'''

\bibverse{5} Siguiendo las órdenes de Moisés trajeron lo que había dicho
al frente del Tabernáculo de Reunión. Todos vinieron y se presentaron
ante el Señor. \bibverse{6} Moisés dijo: ``Esto es lo que el Señor me
ordenó que te dijera, para que veas su gloria''.

\bibverse{7} Entonces Moisés le dijo a Aarón: ``Ve al altar y sacrifica
tu ofrenda por el pecado y tu holocausto para que tú y el pueblo estén
bien. Luego sacrifica las ofrendas traídas por el pueblo para
enderezarlas, como el Señor lo ordenó''.

\bibverse{8} Así que Aarón fue al altar y mató el becerro como ofrenda
por el pecado para sí mismo. \bibverse{9} Sus hijos le trajeron la
sangre. Él mojó su dedo en la sangre y la puso en los cuernos del altar.
Derramó el resto de la sangre en la parte inferior del altar.
\bibverse{10} Quemó la grasa, los riñones y la mejor parte del hígado de
la ofrenda por el pecado en el altar, como el Señor le había ordenado a
Moisés. \bibverse{11} Sin embargo, quemó la carne y la piel fuera del
campamento.

\bibverse{12} Aarón mató la ofrenda quemada. Sus hijos le trajeron la
sangre y él la roció a los lados del altar. \bibverse{13} Le trajeron la
cabeza y todos los demás pedazos del holocausto, y él los quemó en el
altar. \bibverse{14} Lavó las entrañas y las piernas y las quemó con el
resto del holocausto en el altar.

\bibverse{15} Entonces Aarón presentó las ofrendas del pueblo. Mató al
macho cabrío como ofrenda por el pecado del pueblo, y lo ofreció de la
misma manera que su propia ofrenda por el pecado. \bibverse{16} Presentó
la ofrenda quemada, haciéndolo de acuerdo con las regulaciones.
\bibverse{17} Presentó la ofrenda de grano. Tomó un puñado de él y lo
quemó en el altar, además de la ofrenda quemada presentada esa mañana.

\bibverse{18} Aarón mató al toro y al carnero como ofrenda de paz para
el pueblo. Sus hijos le trajeron la sangre, y él la roció a los lados
del altar. \bibverse{19} También le trajeron las porciones de grasa del
toro y del carnero: el rabo gordo, la grasa que cubre las entrañas, los
riñones y la mejor parte del hígado \bibverse{20} y las pusieron sobre
los pechos. Aarón quemó las porciones de grasa en el altar,
\bibverse{21} pero agitó los pechos y el muslo derecho como ofrenda
ondulante ante el Señor, como Moisés le había ordenado.

\bibverse{22} Entonces Aarón levantó las manos hacia el pueblo y lo
bendijo. Después bajó del altar, habiendo completado la ofrenda por el
pecado, el holocausto y el sacrificio de paz.

\bibverse{23} Moisés y Aarón entraron en el Tabernáculo de Reunión.
Cuando salieron, bendijeron al pueblo, y la gloria del Señor se reveló a
todos. \bibverse{24} Un fuego salió de la presencia del Señor y quemó el
holocausto y las porciones de grasa en el altar. Cuando todos vieron
esto, gritaron de alegría y cayeron con el rostro en el suelo.

\hypertarget{section-9}{%
\section{10}\label{section-9}}

\bibverse{1} Los hijos de Aarón, Nadab y Abihu, encendieron sus
quemadores de incienso usando fuego ordinario\footnote{\textbf{10:1}
  ``Usando fuego ordinario'': añadido para mayor claridad.} y
encendieron sus quemadores de incienso usando fuego ordinario y pusieron
incienso, y de esta manera ofrecieron fuego prohibido en la presencia
del Señor, algo que él no había autorizado. \bibverse{2} El fuego salió
de la presencia del Señor y los quemó. Murieron en la presencia del
Señor. \bibverse{3} Moisés explicó a Aarón: ``Esto es lo que el Señor
estaba hablando cuando dijo: `Mostraré mi santidad a los que se acerquen
a mí; revelaré mi gloria para que todos la vean'\,''. Pero Aarón no
respondió.

\bibverse{4} Moisés llamó a Misael y a Elzafán, hijos del tío de Aarón,
Uzziel, y les dijo: ``Vengan y lleven los cuerpos de sus primos y
llévenlos fuera del campamento, lejos del frente del santuario''.
\bibverse{5} Vinieron, los recogieron por sus ropas y los llevaron fuera
del campamento, como Moisés había ordenado.

\bibverse{6} Entonces Moisés dijo a Aarón y a sus hijos Eleazar e
Itamar: ``No dejensus cabellos sin peinar, ni rasguen sus vestidos de
luto,\footnote{\textbf{10:6} ``De luto'': añadido para mayor claridad.}de
lo contrario, morirán y el Señor se enfadará con todos. Pero tus
parientes y todos los demás israelitas pueden llorar por los que el
Señor mató con fuego. \bibverse{7} No salgas de la entrada del
Tabernáculo de Reunión, o morirás, porque has sido ungido por el
Señor''. Hicieron lo que dijo Moisés.

\bibverse{8} El Señor le dijo a Aarón: \bibverse{9} Tú y tu descendencia
no deben beber vino ni ningún otro tipo de alcohol cuando entren al
Tabernáculo de Reunión, de lo contrario morirán. Esta norma es para
siempre y para todas las generaciones futuras. \bibverse{10} para que
puedas enseñar a los israelitas todas las regulaciones que el Señor les
ha dado a través de Moisés''. \bibverse{11} para que puedas enseñar a
los israelitas todas las regulaciones que el Señor les ha dado a través
de Moisés''.

\bibverse{12} Moisés dijo a Aarón y a sus dos hijos que quedaban,
Eleazar e Itamar: ``Tomen la ofrenda de grano que sobra de las ofrendas
dadas al Señor y comedla sin levadura junto al altar, porque es muy
santa. \bibverse{13} Deben comerlo en un lugar santo, porque es la parte
que os corresponde a ustedes y a sus descendientes de las ofrendas dadas
al Señor. Esto es lo que me han ordenado.

\bibverse{14} Tú y tus descendientes masculinos y femeninos pueden comer
el pecho de la ofrenda ondulada y la contribución del muslo en cualquier
lugar que esté ceremonialmente limpio, porque a ti y a tus descendientes
se les ha dado esto como su parte de las ofrendas de paz de los
israelitas. \bibverse{15} La contribución del muslo y el pecho de la
ofrenda mecida, así como las porciones de grasa de las ofrendas de
comida hechas, deben ser traídas y mecidas como ofrenda mecida ante el
Señor. Te pertenecen a ti y a tus hijos para siempre, como el Señor ha
ordenado''.

\bibverse{16} Moisés comprobó lo que había sucedido con el macho cabrío
de la ofrenda por el pecado, y descubrió que había sido quemado. Se
enfadó con Eleazar e Itamar, los hijos que Aarón había dejado, y les
preguntó, \bibverse{17} ``¿Por qué no tomaron la ofrenda por el pecado y
la comieron en el lugar santo, porque es muy santa y se les dio para
quitar la culpa del pueblo, haciéndolo justo ante el Señor?
\bibverse{18} Como su sangre no fue llevada al lugar santo, debieron
comerla en el área del santuario, como yo ordené''.

\bibverse{19} Entonces Aarón le explicó a Moisés: ``Mira, fue hoy cuando
presentaron su ofrenda por el pecado y su holocausto ante el Señor.
Después de todo lo que me acaba de pasar, ¿se habría complacido el Señor
si yo hubiera comido la ofrenda por el pecado hoy?''

\bibverse{20} Cuando Moisés escuchó lo que Aarón tenía que decir, aceptó
la explicación.

\hypertarget{section-10}{%
\section{11}\label{section-10}}

\bibverse{1} El Señor les dijo a Moisés y a Aarón: \bibverse{2} ``Den
estas instrucciones a los israelitas. Estos son los animales que se les
permite comer: \bibverse{3} cualquier animal que tenga una pezuña
dividida y que también mastique el bolo alimenticio. \bibverse{4} in
embargo, si mastica el bolo alimenticio, o tiene una pezuña dividida,
entonces no puedes comerlo. Estos incluyen: el camello, que aunque
mastica el bolo alimenticio no tiene una pezuña dividida, por lo que es
inmune para ti. \bibverse{5} ldamán de las rocas, que aunque mastica el
bolo alimenticio no tiene una pezuña dividida, por lo que es inmune para
ti. \bibverse{6} a liebre, que aunque mastica el bolo alimenticio no
tiene una pezuña dividida, así que es inmunda para ti. \bibverse{7} l
cerdo, que aunque tiene una pezuña dividida no mastica el bolo
alimenticio, así que es inmundo para ti. \bibverse{8} o debes comer su
carne ni tocar sus cuerpos de cuentas. Son inmundos para ti.

\bibverse{9} Puedes comer cualquier criatura con aletas y escamas que
viva en el agua, ya sea en el mar o en agua dulce. \bibverse{10} ero no
puedes comer ninguna de las muchas criaturas que no tienen aletas y
escamas y que viven en el agua, ya sea en el mar o en agua dulce.

\bibverse{11} Son repulsivos.\footnote{\textbf{11:11} ``Repulsivos'': no
  sólo de manera sensorial, sino también en el sentido de que son
  ceremonialmente impuros.}No debes comer su carne, y debes tratar sus
cadáveres como impuros. \bibverse{12} Todas esas criaturas acuáticas que
no tienen aletas y escamas deben ser reprobables para ti.

\bibverse{13} En cuanto a las aves,\footnote{\textbf{11:13} No hay
  certeza de las aves específicas en la lista que sigue. Sin embargo, se
  refiere principalmente a las aves de rapiña y otras aves que comen
  carroña o animales inmundos. Además, la palabra traducida como
  ``aves'' significa realmente ``criaturas voladoras'', lo que explica
  la inclusión de murciélagos al final de la lista.}estas no deben ser
consumidas porque son impuras: águila, buitre leonado, quebrantahuesos,
\bibverse{14} ratonero, milano y aves de presa similares, \bibverse{15}
cualquier cuervo o cuervo, \bibverse{16} cárabo, búho chico, gaviotas,
cualquier tipo de halcón, \bibverse{17} mochuelo, búho pescador, búho
real, \bibverse{18} lechuza de los establos, búho del desierto, buitre
egipcio, \bibverse{19} cigüeñas y cualquier tipo de garza, abubilla y
murciélagos. \bibverse{20} Todos los insectos voladores que se
arrastran\footnote{\textbf{11:20} ``Que se arrastran'': Literalmente,
  ``que van en cuatro patas''. Sin embargo, esto no significa que los
  insectos sólo tengan cuatro patas, cuando en realidad tienen seis,
  simplemente se refiere a la forma habitual en que se mueven los
  animales, pues la mayoría tiene cuatro patas.} serán impuros para ti.
\bibverse{21} Pero puedes comer los siguientes tipos de insectos
voladores que se arrastran: Los que tienen patas articuladas y que usan
para saltar. \bibverse{22} sí que en esta categoría puedes comer
cualquier tipo de langosta, langosta calva, grillo o saltamontes.
\bibverse{23} odos los demás insectos voladores que se arrastran serán
reprobables para ti, \bibverse{24} y te harán impuro. Si tocas sus
cadáveres serás impuro hasta la noche, \bibverse{25} y si recoges uno de
sus cadáveres debes lavar tu ropa, y serás impuro hasta la noche.

\bibverse{26} odo animal con pezuñas que no estén divididas, o que no
mastique la baba, es impuro para ti. Si tocas alguno de ellos serás
impuro. \bibverse{27} ualquier animal de cuatro patas que camine sobre
sus patas es impuro para ti. Si tocas sus cadáveres serás impuro hasta
la noche, \bibverse{28} y si recoges uno de sus cadáveres debes lavar tu
ropa, y serás impuro hasta la noche. Ellos son inmundos para ti.

\bibverse{29} Los siguientes animales\footnote{\textbf{11:29} No hay
  certeza de los animales específicos mencionados en la lista que sigue
  a continuación.}que corren por el suelo son inmundos para ti: ratas,
ratones, cualquier tipo de lagarto grande, \bibverse{30} eco, lagarto
monitor, lagarto de pared, el eslizón y el camaleón. \bibverse{31} Estos
animales que corren por el suelo son inmundos para ti. Si tocas uno de
ellos, estarás sucio hasta la noche.

\bibverse{32} ualquier cosa que uno de ellos muera y caiga sobre él se
vuelve inmundo. Sea lo que sea, algo hecho de madera, ropa, cuero, tela
de saco o cualquier herramienta de trabajo, debe ser lavado con agua y
será impuro hasta la noche. Entonces se volverá limpio. \bibverse{33} i
uno de ellos cae en una vasija de arcilla, todo lo que hay en ella se
vuelve impuro. Debes romper la olla. \bibverse{34} i el agua de esa olla
toca algún alimento, ese alimento se vuelve impuro, y cualquier bebida
de una olla como esa también se vuelve impura. \bibverse{35} ualquier
cosa sobre la que caiga uno de sus cadáveres se vuelve inmunda. Si se
trata de un horno o una estufa, debe ser aplastado. Es permanentemente
impuro para ti. \bibverse{36} Por otro lado, si es un manantial o una
cisterna que contiene agua, entonces permanecerá limpia, pero si tocas
uno de estos cadáveres en ella estarás sucio. \bibverse{37} De igual
manera, si uno de sus cadáveres cae sobre cualquier semilla utilizada
para la siembra, la semilla permanece limpia; \bibverse{38} pero si la
semilla ha sido empapada en agua y uno de sus cadáveres cae sobre ella,
es impura para ti.

\bibverse{39} i muere un animal que se te permite comer, cualquiera que
toque el cadáver será impuro hasta la noche. \bibverse{40} i comes algo
del cadáver debes lavar tu ropa y serás impuro hasta la noche. Si
recoges el cadáver debes lavar tu ropa y estarás sucio hasta la noche.

\bibverse{41} odo animal que se arrastra por el suelo es repulsivo, no
debes comerlo. \bibverse{42} o comas ningún animal que se arrastre por
el suelo, ya sea que se mueva sobre su vientre o camine sobre cuatro o
muchos pies. Todos esos animales son repulsivos. \bibverse{43} o se
contaminen con ningún animal que se arrastre. No se contaminen por causa
de ellos, \bibverse{44} porque yo soy el Señor su Dios; así que
dedíquense y sean santos, porque yo soy santo. No se contaminen con
ningún animal que se arrastre por el suelo. \bibverse{45} o soy el Señor
que os sacó de Egipto para ser su Dios. Así que sed santos, porque yo
soy santo. \bibverse{46} stas son las normas sobre los animales, las
aves, todo lo que vive en el agua, y todos los animales que se arrastran
por la tierra. \bibverse{47} ebes reconocer la diferencia entre lo sucio
y lo limpio, entre los animales que se pueden comer y los que no''.

\hypertarget{section-11}{%
\section{12}\label{section-11}}

\bibverse{1} El Señor le dijo a Moisés: ``Dales estas instrucciones a
los israelitas. \bibverse{2} oda mujer que quede embarazada y tenga un
niño, será impura durante una semana, de la misma manera que es impura
durante su período. \bibverse{3} l prepucio del niño debe ser
circuncidado al octavo día. \bibverse{4} a mujer debe esperar otros
treinta y tres días para la purificación de su sangrado. No se le
permite tocar nada sagrado, y no se le permite entrar en el santuario
hasta que termine el tiempo de purificación. \bibverse{5} i una mujer
tiene una hija, estará impura durante dos semanas, de la misma manera
que lo está durante su período menstrual. La mujer debe esperar otros
sesenta y seis días para la purificación de su sangre. \bibverse{6} na
vez que el tiempo de purificación haya terminado para un hijo o una
hija, la mujer debe traer un cordero de un año como ofrenda quemada y
una paloma joven o una tórtola como ofrenda de purificación. Debe llevar
sus ofrendas al sacerdote a la entrada del Tabernáculo de Reunión.
\bibverse{7} l sacerdote las presentará al Señor para purificarla y que
quede limpia de su sangrado. Estas son las normas para una mujer después
de haber tenido un hijo o una hija.

\bibverse{8} i una mujer no puede permitirse traer un cordero, debe
traer dos tórtolas o dos pichones. Una es para el holocausto y la otra
para la ofrenda de purificación. El sacerdote las ofrecerá para
purificarla, y ella quedará limpia''.

\hypertarget{section-12}{%
\section{13}\label{section-12}}

\bibverse{1} El Señor le dijo a Moisés y a Aarón: \bibverse{2}
Cualquiera que tenga una hinchazón, un sarpullido o una mancha en la
piel que pueda ser una enfermedad infecciosa de la piel debe ser llevado
a Aarón el sacerdote o a uno de sus descendientes. \bibverse{3} l
sacerdote inspeccionará lo que sea que esté en la piel. Si el pelo se ha
vuelto blanco y si el problema parece ser más que algo en la superficie,
es una enfermedad grave de la piel, y el sacerdote que lo inspeccione
declarará a la persona impura.

\bibverse{4} ero si la mancha es sólo una decoloración blanca y no
parece ser más que superficial, y si el pelo de la mancha no se ha
vuelto blanco, el sacerdote pondrá a la persona en aislamiento durante
siete días. \bibverse{5} l séptimo día el sacerdote realizará otra
inspección, y si descubre que la mancha no ha cambiado y no se ha
extendido sobre la piel, el sacerdote debe poner a la persona en
aislamiento durante otros siete días. \bibverse{6} l séptimo día después
de esto el sacerdote lo inspeccionará de nuevo. Si la mancha se ha
desvanecido y no se ha extendido sobre la piel, el sacerdote declarará a
la persona limpia ya que era un sarpullido. Deben lavar su ropa y
estarán limpios. \bibverse{7} in embargo, si el sarpullido se extiende
después de que la persona ha sido inspeccionada por el sacerdote y ha
sido declarada limpia, la persona debe volver para ser inspeccionada de
nuevo. \bibverse{8} i el sacerdote descubre que el sarpullido se ha
propagado, debe declarar a la persona impura porque es ciertamente una
enfermedad de la piel.

\bibverse{9} ualquier persona que desarrolle una enfermedad infecciosa
de la piel debe ser llevada al sacerdote. \bibverse{10} l sacerdote los
inspeccionará, y si hay una hinchazón blanca en la piel y el pelo se ha
vuelto blanco, y hay una herida abierta en la hinchazón, \bibverse{11}
es una enfermedad grave de la piel y el sacerdote debe declararlos
inmundos. No necesita poner a la persona en aislamiento porque sea
impura.

\bibverse{12} in embargo, si la enfermedad de la piel afecta a toda su
piel de manera que cubre su piel de la cabeza a los pies, en todos los
lugares que el sacerdote pueda ver, \bibverse{13} el sacerdote los
inspeccionará, y si la enfermedad ha cubierto todo su cuerpo, declarará
a la persona limpia. Como todo se ha vuelto blanco, están limpios.
\bibverse{14} ero si al inspeccionar a alguien se encuentra una herida
abierta, serán inmundos. \bibverse{15} uando el sacerdote descubre una
herida abierta, debe declarar a la persona impura. La herida abierta es
impura; es una enfermedad infecciosa de la piel. \bibverse{16} ero si la
herida abierta se cura y se vuelve blanca, la persona debe volver al
sacerdote. \bibverse{17} l sacerdote los inspeccionará de nuevo, y si la
herida se ha vuelto blanca, el sacerdote debe declarar a la persona
limpia; entonces están limpios.

\bibverse{18} uando un forúnculo aparece en la piel de una persona y
luego se cura, \bibverse{19} y en su lugar aparece una hinchazón blanca
o una mancha blanca-rojiza, debe mostrarse al sacerdote. \bibverse{20} l
sacerdote lo inspeccionará, y si parece ser más que algo en la
superficie, y si el pelo allí se ha vuelto blanco, el sacerdote lo
declarará impuro. Es una enfermedad grave de la piel que ha infectado el
furúnculo. \bibverse{21} in embargo, si cuando el sacerdote lo
inspeccione, no tiene pelo blanco en él y no parece ser más que
superficial, y se ha desvanecido, el sacerdote deberá poner a la persona
en aislamiento durante siete días. \bibverse{22} Si entonces la mancha
se ha extendido más en la piel, el sacerdote las declarará impuras; es
una enfermedad. \bibverse{23} ero si la mancha permanece igual y no se
extiende, es sólo la cicatriz del furúnculo, y el sacerdote las
declarará limpias.

\bibverse{24} Si alguien tiene una quemadura en la piel y donde está
abierta se convierte en una mancha blanca o rojiza, \bibverse{25} el
sacerdote debe revisarla. Si el vello que crecesobre la mancha se ha
vuelto blanco y la mancha parece ser más profunda, es una enfermedad
grave de la piel que ha infectado la quemadura, y el sacerdote que la
inspeccione declarará a la persona impura. Es una enfermedad infecciosa
de la piel. \bibverse{26} in embargo, si cuando el sacerdote la
inspeccione, no tiene pelo blanco en ella y no parece ser más que
superficial, y se ha desvanecido, el sacerdote deberá poner a la persona
en aislamiento durante siete días. \bibverse{27} l séptimo día el
sacerdote inspeccionará a la persona de nuevo. Si entonces la mancha se
ha extendido más en la piel, el sacerdote la declarará impura; es una
enfermedad grave de la piel. \bibverse{28} ero si la mancha permanece
igual y no se ha extendido sobre la piel, sino que se ha desvanecido, es
la hinchazón de la quemadura, y el sacerdote los declarará limpios
porque es sólo la cicatriz de la quemadura.

\bibverse{29} i alguien, hombre o mujer, tiene una llaga en la cabeza o
en el mentón, \bibverse{30} el sacerdote la inspeccionará, y si parece
ser más que superficial y el pelo en ella se ha vuelto pálido y fino, el
sacerdote debe declararlos inmundos; es una infección que produce
costras, una enfermedad grave de la cabeza o del mentón. \bibverse{31}
Sin embargo, si el sacerdote inspecciona la infección de la costra y no
parece ser más que superficial y no tiene vello pálido,\footnote{\textbf{13:31}
  ``Pálido'': el texto hebreo dice ``negro'' pero esto es probablemente
  un error de los escribas.} el sacerdote debe poner a la persona en
aislamiento durante siete días. \bibverse{32} El séptimo día el
sacerdote inspeccionará a la persona de nuevo y si la infección de la
costra no se ha extendido y no hay pelo pálido en ella, y no parece ser
más que superficial, \bibverse{33} entonces la persona debe afeitarse
excepto en la zona escamosa. El sacerdote debe poner a la persona en
aislamiento por otros siete días. \bibverse{34} l séptimo día el
sacerdote inspeccionará la infección de la costra, y si no se ha
extendido en la piel y no parece ser más que superficial, el sacerdote
debe declarar a la persona limpia. Deben lavar su ropa y estarán
limpios. \bibverse{35} in embargo, si la infección de la costra se ha
propagado en la piel después de haber sido declarada limpia,
\bibverse{36} el sacerdote debe inspeccionarlos, y si la infección de la
costra se ha propagado efectivamente en la piel, el sacerdote no
necesita comprobar si hay pelo pálido; la persona está impura.
\bibverse{37} ero si el sacerdote ve que la infección de la costra no ha
cambiado, y le ha crecido pelo negro, entonces se ha curado. La persona
está limpia, y el sacerdote debe declararlo. \bibverse{38} Si alguien,
hombre o mujer, tiene manchas blancas en la piel, \bibverse{39} el
sacerdote las inspeccionará, y si las manchas aparecen de un blanco
apagado, es sólo un sarpullido que se ha desarrollado en la piel; la
persona está limpia.

\bibverse{40} i un hombre pierde el pelo y se queda calvo, sigue estando
limpio. \bibverse{41} i tiene un retroceso del cabello y se queda calvo
en la frente, sigue estando limpio. \bibverse{42} ero si aparece una
llaga rojiblanca en su cabeza o frente calva, es una enfermedad
infecciosa que se está desarrollando. \bibverse{43} l sacerdote debe
inspeccionarlo, y si la hinchazón de la llaga en su calva o frente se ve
rojiblanca como una enfermedad de la piel, \bibverse{44} entonces tiene
una enfermedad infecciosa; está sucio. El sacerdote debe declararlo
impuro por la infección en su cabeza. \bibverse{45} Cualquiera que tenga
tales enfermedades debe usar desgarrada y dejar que su cabello
permanezca despeinado. Debe cubrirse la cara\footnote{\textbf{13:45}
  ``Caras'': Literalmente, ``labio superior''.}y gritar: ``¡Inmundo,
inmundo!'' \bibverse{46} Permanecen inmundos mientras dure la infección.
Tienen que vivir solos en algún lugar fuera del campamento.
\bibverse{47} Las siguientes reglamentaciones se refieren \footnote{\textbf{13:47}
  ``Las siguientes regulaciones se refieren'': añadido para mayor
  claridad.}a cualquier material que se vea afectado por el
moho,\footnote{\textbf{13:47} ``Moho'': la palabra utilizada es la misma
  que la de la enfermedad infecciosa de la piel mencionada
  anteriormente.}como la ropa de lana o lino, \bibverse{48} cualquier
cosa tejida o de punto hecha de lino o lana, o cualquier cosa hecha de
cuero: \bibverse{49} Si la mancha es verde o roja en el material, ya sea
cuero, tejido o tejido de punto o algún otro artículo de cuero, entonces
está infectado con moho y debe ser mostrado al sacerdote. \bibverse{50}
l sacerdote debe inspeccionar el moho y poner el artículo en aislamiento
durante siete días. \bibverse{51} l séptimo día el sacerdote deberá
inspeccionarlo de nuevo, y si la mancha de moho se ha extendido en el
material, ya sea cuero, tejido o tejido de punto o algún otro artículo
de cuero, entonces es un moho dañino; el artículo está sucio, sea cual
sea el uso que se le dé. \bibverse{52} l sacerdote debe quemarlo, ya sea
que el artículo afectado sea de lana, lino o cuero. Debido a que el moho
es dañino, el artículo debe ser quemado. \bibverse{53} in embargo, si
cuando el sacerdote lo inspeccione de nuevo, el moho no se ha extendido,
\bibverse{54} el sacerdote ordenará que el artículo afectado se lave y
se ponga en aislamiento durante otros siete días. \bibverse{55} na vez
lavado, el sacerdote debe inspeccionarlo de nuevo, y si el objeto con el
molde no ha cambiado su aspecto, está sucio. Aunque el moho no se haya
extendido, debe quemar el artículo, tanto si el daño del moho está en el
interior como en el exterior. \bibverse{56} i el sacerdote lo
inspecciona y la mancha de moho se ha desvanecido después de lavarlo,
debe cortar la parte afectada el material, ya sea cuero, tejido o tejido
de punto. \bibverse{57} in embargo, si el moho regresa, entonces se está
extendiendo. En ese caso debe quemar la parte afectada. \bibverse{58} i
el moho desaparece después de lavarlo, entonces hay que lavarlo de
nuevo, y estará limpio.

\bibverse{59} stas son las normas sobre lo que hay que hacer cuando el
moho contamina la lana o el lino, ya sea tejido o tejido de punto, o
cualquier artículo de cuero, para declararlo limpio o sucio''.

\hypertarget{section-13}{%
\section{14}\label{section-13}}

\bibverse{1} El Señor le dijo a Moisés, \bibverse{2} ``Estas son las
normas relativas a los que han tenido una enfermedad de la piel cuando
se declaran limpios habiendo sido llevados al sacerdote. \bibverse{3} l
sacerdote debe salir del campamento e inspeccionar a la persona. Si la
enfermedad de la piel se ha curado, \bibverse{4} el sacerdote hará que
le traigan dos pájaros ceremoniales limpios, también algo de madera de
cedro, hilo carmesí e hisopo, en nombre de la persona que se va a
limpiar.

\bibverse{5} l sacerdote ordenará que se mate a uno de los pájaros sobre
una vasija de arcilla llena de agua fresca. \bibverse{6} omarás el
pájaro vivo junto con la madera de cedro, el hilo carmesí y el hisopo, y
los mojará en la sangre del pájaro que fue matado sobre el agua fresca.
\bibverse{7} sará la sangre para rociar siete veces a la persona que
está siendo limpiada de la enfermedad de la piel. Luego el sacerdote los
declarará limpios y dejará que el pájaro vivo se vaya volando.

\bibverse{8} l que se limpia debe lavar su ropa, afeitarse todo el pelo
y lavarse con agua; entonces se limpiará ceremonialmente. Después de eso
pueden entrar en el campamento, pero deben permanecer fuera de su tienda
durante siete días. \bibverse{9} l séptimo día se afeitarán todo el
pelo: la cabeza, la barba, las cejas y el resto del cabello. Deben lavar
su ropa y lavarse con agua, y estarán limpios.

\bibverse{10} El octavo día traerán dos corderos machos y una hembra,
todos de un año de edad y sin defectos; una ofrenda de grano que
consiste en tres décimos de una efa de la mejor harina mezclada con
aceite de oliva, y un ``tronco''\footnote{\textbf{14:10} ``Tronco'' es
  una medida líquida, un poco menos de una pinta.} de aceite de oliva.
\bibverse{11} El sacerdote que dirige la ceremonia presentará al Señor
la persona a ser limpiada, junto con estas ofrendas, a la entrada del
Tabernáculo de Reunión. \bibverse{12} l sacerdote tomará uno de los
corderos machos y lo presentará como ofrenda por la culpa, junto con el
tronco de aceite de oliva; y lo agitará ante el Señor como ofrenda
mecida.

\bibverse{13} uego degollará el cordero cerca del santuario donde se
degüella la ofrenda por el pecado y el holocausto. La ofrenda por el
pecado y la ofrenda por la culpa pertenecen al sacerdote; es muy
sagrada. \bibverse{14} l sacerdote pondrá parte de la sangre de la
ofrenda por la culpa en el lóbulo de la oreja derecha, en el pulgar
derecho y en el dedo gordo del pie derecho de la persona que se está
limpiando. \bibverse{15} l sacerdote echará un poco del tronco de aceite
de oliva en su palma izquierda, \bibverse{16} mojará su dedo índice
derecho en él, y con su dedo rociará un poco de aceite de oliva siete
veces ante el Señor. \bibverse{17} l sacerdote usará entonces parte del
resto del aceite de oliva que queda en su palma sobre la persona que se
está limpiando, y lo pondrá sobre la sangre de la ofrenda de culpa. Esto
estará en el lóbulo de su oreja derecha, en su pulgar derecho y en el
dedo gordo de su pie derecho, sobre la sangre de la ofrenda de culpa.
\bibverse{18} o que quede del aceite de oliva en su palma, el sacerdote
lo pondrá sobre la cabeza de la persona que se está limpiando y luego lo
hará justo ante el Señor. \bibverse{19} l sacerdote sacrificará la
ofrenda por el pecado para hacer a la persona correcta, de modo que
ahora esté limpia de su impureza. Después de eso, el sacerdote matará el
holocausto \bibverse{20} y lo ofrecerá en el altar, junto con la ofrenda
de grano, para enderezarlos, y estarán limpios.

\bibverse{21} ero los que son pobres y no pueden pagar estas ofrendas
deben traer un cordero macho como ofrenda de culpa para ser agitado para
hacerlos rectos, junto con una décima parte de la mejor harina mezclada
con aceite de oliva para una ofrenda de grano, un tronco de aceite de
oliva, \bibverse{22} y dos tórtolas o dos pichones de paloma, lo que
puedan pagar. Una se usará como ofrenda por el pecado y la otra como
holocausto.

\bibverse{23} l octavo día deben llevarlos al sacerdote a la entrada del
Tabernáculo de Reunión ante el Señor para que los limpie. \bibverse{24}
l sacerdote tomará el cordero para la ofrenda por la culpa, junto con el
tronco de aceite de oliva, y los agitará como ofrenda mecida ante el
Señor. \bibverse{25} espués de matar el cordero para la ofrenda por la
culpa, el sacerdote tomará un poco de la sangre de la ofrenda por la
culpa y la pondrá en el lóbulo de la oreja derecha del que se está
limpiando, en el pulgar derecho y en el dedo gordo del pie derecho.

\bibverse{26} Entonces el sacerdote verterá un poco de aceite de oliva
en su palma izquierda \bibverse{27} y con su dedo índice derecho,
rociará un poco de aceite de su palma izquierda siete veces ante el
Señor. \bibverse{28} l sacerdote pondrá también un poco de aceite de
oliva en su palma, en el lóbulo de la oreja derecha de la persona que se
está limpiando, en el pulgar derecho y en el dedo gordo del pie derecho,
en los mismos lugares que la sangre de la ofrenda de culpa.
\bibverse{29} o que quede del aceite de oliva en su palma, el sacerdote
lo pondrá en la cabeza de la persona que se está limpiando y luego lo
pondrá delante del Señor. \bibverse{30} uego deberán sacrificar una de
las tórtolas o palomas jóvenes, según sus posibilidades, \bibverse{31}
una como ofrenda por el pecado y la otra como holocausto, junto con la
ofrenda de grano. Así es como el sacerdote hará a la persona correcta y
limpia ante el Señor. \bibverse{32} stas son las normas para aquellos
que tienen una enfermedad de la piel y no pueden permitirse las ofrendas
habituales para hacer a la gente limpia''.

\bibverse{33} Entonces el Señor les dijo a Moisés y a Aarón:
\bibverse{34} ``Cuando lleguen a Canaán, la tierra que yo les doy, si
pongo\footnote{\textbf{14:34} Este es otro ejemplo en que el Señor es
  ``acreditado'' por una situación porque es el Dios todopoderoso que
  gobierna el universo. No significa necesariamente que Dios actúe
  directamente de esta manera.} un poco de moho en una casa y la
contamino, \bibverse{35} el dueño de la casa debe venir y decirle al
sacerdote:``Parece que mi casa tiene moho''. \bibverse{36} l sacerdote
debe ordenar que se vacíe la casa antes de entrar a inspeccionar el
moho, para que nada en la casa sea declarado impuro. Una vez hecho esto,
el sacerdote debe entrar e inspeccionar la casa. \bibverse{37} xaminará
la casa y verá si el moho de las paredes está hecho de hendiduras verdes
o rojas que van bajo la superficie, \bibverse{38} el sacerdote saldrá a
la puerta y sellará la casa durante siete días. \bibverse{39} l séptimo
día el sacerdote volverá e inspeccionará la casa de nuevo. Si el moho se
ha extendido en las paredes, \bibverse{40} ordenará que las piedras
afectadas se retiren y se eliminen en un área impura fuera de la ciudad.
\bibverse{41} uego ordenará que todo el yeso del interior de la casa sea
raspado y arrojado en una zona impura fuera de la ciudad. \bibverse{42}
e deben usar diferentes piedras para reemplazar las que se han quitado,
y se necesitará un nuevo yeso para volver a enlucir la casa.
\bibverse{43} i el moho vuelve y afecta de nuevo a la casa, incluso
después de haber quitado las piedras y de haber raspado y vuelto a
enlucir la casa, \bibverse{44} el sacerdote debe venir a inspeccionarla.
Si ve que el moho se ha extendido en la casa, es un moho dañino; la casa
está sucia. \bibverse{45} ebe ser demolida, y todas sus piedras, maderas
y yeso deben ser tomadas y arrojadas en un área impura fuera de la
ciudad. \bibverse{46} ualquiera que entre en la casa durante cualquier
tiempo que esté sellada será impuro hasta la noche. \bibverse{47} uien
duerma o coma en la casa debe lavar su ropa.

\bibverse{48} in embargo, si cuando el sacerdote venga a inspeccionarla
y encuentra que el moho no ha reaparecido después de que la casa haya
sido tapizada, declarará la casa limpia porque el moho ha desaparecido.
\bibverse{49} raerá dos pájaros, madera de cedro, hilo carmesí e hisopo
para limpiar la casa. \bibverse{50} atará a uno de los pájaros sobre una
vasija de arcilla llena de agua fresca. \bibverse{51} umergirá el pájaro
vivo, la madera de cedro, el hilo carmesí y el hisopo en la sangre del
pájaro muerto y en el agua fresca, y rociará la casa siete veces.
\bibverse{52} impiará la casa con la sangre del pájaro, el agua fresca,
el pájaro vivo, la madera de cedro, el hisopo y el hilo carmesí.
\bibverse{53} uego dejará que el pájaro vivo se vaya volando fuera de la
ciudad. Así es como hará la casa bien, y estará limpia.

\bibverse{54} Estas son las regulaciones para cualquier enfermedad
infecciosa de la piel, para una infección de costra, \bibverse{55} para
el moho en la ropa y en una casa, \bibverse{56} así como para una
hinchazón, sarpullido o mancha. \bibverse{57} e utilizan para decidir si
algo está limpio o sucio. Estas son las normas relativas a las
enfermedades de la piel y el moho.''

\hypertarget{section-14}{%
\section{15}\label{section-14}}

\bibverse{1} El Señor dijo a Moisés y a Aarón: \bibverse{2} ``Díganles a
los israelitas: Cuando un hombre tiene una secreción de sus genitales,
la secreción es impura. \bibverse{3} a impureza proviene de su flujo, ya
sea que su cuerpo lo permita o lo bloquee. Lo hace impuro. \bibverse{4}
ualquier cama en la que se acueste el hombre con la secreción será
impura, y cualquier cosa en la que se siente será impura. \bibverse{5} l
que toque su cama tiene que lavar su ropa y lavarse con agua, y será
impuro hasta la noche. \bibverse{6} l que se siente sobre cualquier cosa
en la que se haya sentado el hombre tiene que lavar su ropa y lavarse
con agua, y será inmundo hasta la noche. \bibverse{7} ualquiera que
toque el cuerpo del hombre tiene que lavar sus ropas y lavarse con agua,
y serán inmundos hasta la tarde. \bibverse{8} i el hombre con la
secreción escupe sobre alguien que está limpio, tiene que lavar su ropa
y lavarse con agua, y será inmundo hasta la noche. \bibverse{9} odo
aquello en lo que el hombre se siente cuando está montando será inmundo.
\bibverse{10} odo el que toque lo que estaba debajo de él será impuro
hasta la noche. Cualquiera que recoja estas cosas tiene que lavar su
ropa y lavarse con agua, y serán inmundos hasta la tarde. \bibverse{11}
i el hombre con la secreción toca a alguien sin lavarse las manos
primero con agua, la persona que fue tocada tiene que lavar su ropa y
lavarse con agua, y será impura hasta la noche. \bibverse{12} ualquier
objeto de arcilla tocado por el hombre debe romperse, y cualquier objeto
de madera debe lavarse con agua.

\bibverse{13} na vez que la secreción se haya curado, el hombre debe
asignar siete días para su proceso de limpieza, lavar sus ropas y
lavarse con agua fresca, y estará limpio. \bibverse{14} l octavo día
debe tomar dos tórtolas o dos pichones, presentarse ante el Señor a la
entrada del Tabernáculo de Reunión y entregárselos al sacerdote.
\bibverse{15} l sacerdote las sacrificará, una como ofrenda por el
pecado y la otra como holocausto. Así es como el sacerdote pondrá al
hombre delante del Señor por su descarga.

\bibverse{16} uando un hombre tiene una descarga de semen, debe lavar
todo su cuerpo con agua, y será impuro hasta la noche. \bibverse{17}
ualquier ropa o cuero sobre el que caiga la secreción de semen debe ser
lavada con agua, y permanecerá impura hasta la noche. \bibverse{18} Si
un hombre se acuesta con una mujer y hay una liberación de semen, ambos
deben lavarse con agua, y permanecerán inmundos hasta la noche.

\bibverse{19} uando una mujer tenga una secreción de sangre en su
cuerpo, será impura por su período durante siete días, y cualquiera que
la toque será impuro hasta la noche. \bibverse{20} odo aquello sobre lo
que se acueste o se siente durante su período será impuro, \bibverse{21}
y todo aquel que toque su cama deberá lavar su ropa y lavarse con agua,
y será impuro hasta la noche. \bibverse{22} ualquiera que toque lo que
estaba sentado tiene que lavar su ropa y lavarse con agua, y será impuro
hasta la noche. \bibverse{23} Ya sea que se trate de una cama o de algo
en lo que estaba sentada, cualquiera que lo toque será impuro hasta la
noche. \bibverse{24} i un hombre duerme con ella y le toca la sangre de
su período, será impuro durante siete días, y cualquier cama en la que
se acueste será impura.

\bibverse{25} uando una mujer tenga secreción de sangre durante varios
días,siendo que no es el momento de su período, o si continúa después de
su período, será impura durante todo el tiempo que esté sangrando, y no
sólo durante los días de su período. \bibverse{26} ualquier cama en la
que se acueste o cualquier cosa en la que se siente mientras tenga la
secreción será impura, al igual que su cama durante su período.
\bibverse{27} uien los toque tiene que lavar su ropa y lavarse con agua,
y serán inmundos hasta la noche. \bibverse{28} na vez que el flujo de la
mujer se haya curado, deberá destinar siete días para su limpieza, lavar
su ropa y lavarse con agua fresca, y estará limpia. \bibverse{29} l
octavo día debe tomar dos tórtolas o dos pichones, presentarse ante el
Señor a la entrada del Tabernáculo de Reunión y entregarlos al
sacerdote. \bibverse{30} l sacerdote las sacrificará, una como ofrenda
por el pecado y la otra como holocausto. Así es como el sacerdote la
pondrá delante del Señor por su descarga.

\bibverse{31} sí es como debes evitar que los israelitas se vuelvan
ceremonialmente inmundos, para que no mueran al hacer que mi Tabernáculo
se vuelva impuro, el lugar donde vivo con ellos.

\bibverse{32} stas son las normas para el hombre que tiene una
secreción, el hombre que tiene una secreción de semen que lo hace
inmundo, \bibverse{33} la mujer durante su período, cualquier hombre o
mujer que tiene una secreción, y el hombre que se acuesta con una mujer
inmunda''.

\hypertarget{section-15}{%
\section{16}\label{section-15}}

\bibverse{1} El Señor habló a Moisés después de la muerte de dos de los
hijos de Aarón cuando fueron a la presencia del Señor. \bibverse{2} l
Señor le dijo a Moisés: ``Adviértele a tu hermano Aarón que no venga al
Lugar Santísimo en cualquier momento que lo desee, de lo contrario
morirá. Porque ahí es donde aparezco en la nube sobre la cubierta de
expiación del Arca, detrás del velo.

\bibverse{3} stas son las instrucciones para que Aarón entre en el
santuario. Debe venir con un toro joven para una ofrenda por el pecado y
con un carnero para una ofrenda quemada. \bibverse{4} ebe llevar la
túnica de lino sagrada, con ropa interior de lino. Tiene que atar una
faja de lino a su alrededor y ponerse el turbante de lino. Estas son
ropas sagradas. Debe lavarse con agua antes de ponérsela. \bibverse{5}
el pueblo de Israel debe traer dos machos cabríos para la ofrenda por el
pecado, y un carnero para el holocausto. \bibverse{6} Aarón presentará
el toro como su propia ofrenda por el pecado para hacerse a sí mismo y a
su familia. \bibverse{7} uego traerá los dos machos cabríos y los
presentará ante el Señor a la entrada del Tabernáculo de Reunión.
\bibverse{8} Aarón echará suertes para elegir entre los cabritos, uno
para el Señor y otro para el chivo expiatorio. \bibverse{9} resentará el
macho cabrío elegido por sorteo para el Señor y lo sacrificará como
ofrenda por el pecado. \bibverse{10} l macho cabrío elegido por sorteo
como chivo expiatorio será presentado vivo ante el Señor para arreglar
las cosas enviándolo al desierto como chivo expiatorio.

\bibverse{11} Aarón debe presentar el toro para su ofrenda por el pecado
para hacer las cosas bien para él y su casa matando el toro como su
propia ofrenda por el pecado. \bibverse{12} uego llenará un quemador de
incienso con carbones encendidos del altar que está ante el Señor, y con
sus manos llenas de incienso de olor dulce finamente molido, los llevará
detrás del velo. \bibverse{13} Pondrá el incienso en presencia del
Señor, sobre las brasas, y el humo del incienso cubrirá la tapa de la
expiación sobre el Testimonio,\footnote{\textbf{16:13} El ``Testimonio''
  se refiere a las dos tablas con los Diez Mandamientos dentro del Arca.}
para que no muera. \bibverse{14} Tomará un poco de sangre del toro y con
su dedo la rociará en el lado este de la tapa de expiación. También
rociará un poco con su dedo siete veces delante de la tapa de la
expiación.

\bibverse{15} ntonces Aarón sacrificará el macho cabrío para la ofrenda
por el pecado del pueblo y traerá su sangre detrás del velo, y con su
sangre deberá hacer lo mismo que hizo con la sangre del toro: La rociará
contra el propiciatorio y delante de él. \bibverse{16} sí es como
corregirá el Lugar Santísimo y lo purificará de la inmundicia de los
israelitas, de sus actos de rebelión y de todos sus pecados. Hará lo
mismo con el Tabernáculo de Reunión que está en medio de su campamento,
rodeado de sus vidas inmundas. \bibverse{17} adie puede estar en el
Tabernáculo de Reunión desde que Aarón entra para purificar el Lugar
Santísimo hasta que sale, después de haber arreglado las cosas para él,
su casa y todos los israelitas.

\bibverse{18} ntonces irá al altar que está delante del Señor y lo
purificará. Tomará la sangre del toro y del macho cabrío y la pondrá en
todos los cuernos del altar. \bibverse{19} ociará con su dedo siete
veces la sangre para dedicarla y purificarla de la inmundicia de los
israelitas. \bibverse{20} na vez que Aarón haya terminado de purificar
el Lugar Santísimo, el Tabernáculo de Reunión y el altar, presentará el
macho cabrío vivo. \bibverse{21} uego pondrá ambas manos sobre la cabeza
del macho cabrío vivo y confesará sobre él todas las faltas de los
israelitas, todos sus actos de rebelión y todos sus pecados. Los pondrá
sobre la cabeza del macho cabrío y lo enviará al desierto, llevado allí
por un hombre elegido para hacerlo. \bibverse{22} l macho cabrío tomará
sobre sí mismo todos sus pecados y se irá a un lugar lejano, y el hombre
lo enviará al desierto.

\bibverse{23} Aarón volverá al Tabernáculo de Reunión, se quitará las
ropas de lino que se puso antes de entrar en el Lugar Santísimo, y las
dejará allí. \bibverse{24} e lavará con agua en el santuario y se pondrá
su propia ropa. Luego saldrá a sacrificar su holocausto y el holocausto
del pueblo que le da la razón a él y al pueblo. \bibverse{25} ambién
debe quemar la grasa de la ofrenda por el pecado en el altar.

\bibverse{26} l hombre que fue y envió al chivo expiatorio debe lavar
sus ropas y lavarse con agua; entonces podrá volver al campamento.

\bibverse{27} os restos del toro usado para la ofrenda por el pecado y
la cabra para la ofrenda por el pecado, cuya sangre fue traída al Lugar
Santísimo para purificarla, deben ser llevados fuera del campamento. Su
piel, carne y desechos deben ser quemados. \bibverse{28} a persona que
los queme debe lavar sus ropas y lavarse con agua; luego puede volver al
campamento.

\bibverse{29} ste reglamento se aplica a todos los tiempos. El décimo
día del séptimo mes es un día de ayuno para ustedes. No debes hacer
ningún trabajo. Esto se aplica a todos los nacidos en el país y también
a cualquier extranjero que viva entre ustedes, \bibverse{30} porque en
este día se hará el proceso de corregiros, para haceros limpios de todos
sus pecados, limpios ante el Señor. \bibverse{31} s un sábado de los
sábados, un día de descanso y de ayuno. Esta regulación se aplica para
todo el tiempo. \bibverse{32} l sacerdote que es ungido y dedicado a
suceder a su padre como sumo sacerdote llevará a cabo la ceremonia de
hacer las cosas bien, y se pondrá las ropas de lino sagrado.
\bibverse{33} levará a cabo la purificación del Lugar Santísimo, el
Tabernáculo de Reunión y el altar, corrigiendo también a los sacerdotes
y a todo el pueblo. \bibverse{34} sta regla se aplica a ustedes para
siempre: una vez al año los israelitas harán expiación por todos sus
pecados''. Moisés hizo todo lo que el Señor le ordenó.

\hypertarget{section-16}{%
\section{17}\label{section-16}}

\bibverse{1} El Señor le dijo a Moisés, \bibverse{2} ``Dile a Aarón, a
sus hijos y a todos los israelitas que esto es lo que el Señor ordena:
\bibverse{3} Cualquier israelita que mate un toro, un cordero o una
cabra, ya sea dentro o fuera del campamento, \bibverse{4} en lugar de
llevarlo a la entrada del Tabernáculo de Reunión para presentarlo como
ofrenda al Señor allí, esa persona será considerada culpable de
asesinato ilegal.\footnote{\textbf{17:4} ``Asesinato ilegal'':
  Literalmente ``sangre'', donde está implícita la idea de sangre
  derramada.}Han derramado sangre y deben ser expulsados de su pueblo.

\bibverse{5} Por eso los israelitas tienen que llevar al Señor los
sacrificios que actualmente ofrecen en los campos. Deben llevarlos al
sacerdote a la entrada del Tabernáculo de Reunión y ofrecerlos como
ofrendas de paz al Señor. \bibverse{6} El sacerdote rociará la sangre
sobre el altar del Señor a la entrada del Tabernáculo de Reunión, y
quemará la grasa, agradable al Señor. \bibverse{7} No deben seguir
ofreciendo sus sacrificios al diablo de las cabras con el que han
actuado como prostitutas. Esta es una regulación para todos los tiempos
y para todas las generaciones futuras.

\bibverse{8} dviértanles que cualquier israelita o extranjero que viva
entre ellos que ofrezca un holocausto o un sacrificio \bibverse{9} sin
llevarlo a la entrada del Tabernáculo de Reunión para sacrificarlo al
Señor debe ser expulsado de su pueblo. \bibverse{10} esconoceré a
cualquier israelita o extranjero que viva entre ellos y coma sangre y
los expulsaré de su pueblo. \bibverse{11} a vida está en la sangre del
cuerpo. Os la he dado para que, poniéndola en el altar, sus vidas puedan
ser rectificadas, porque es la sangre que rectifica la vida de la gente.
\bibverse{12} or eso les advierto a los israelitas: Ninguno de ustedes
puede comer sangre, y ningún extranjero que viva entre ustedes puede
comer sangre.

\bibverse{13} ualquier israelita o extranjero que viva entre ellos y que
cace y mate un animal salvaje o un pájaro que esté permitido comer, debe
drenar la sangre de su cuerpo y cubrirlo con tierra, \bibverse{14}
porque la vida está en la sangre del cuerpo. Por eso he advertido a los
israelitas: No se permite comer la sangre de nada viviente, porque la
vida está en la sangre del cuerpo. Cualquiera que la coma debe ser
expulsado.

\bibverse{15} odo israelita o extranjero que coma algo que haya sido
encontrado muerto o matado por animales salvajes debe lavar sus ropas y
lavarlas con agua, y estarán inmundos hasta la noche. Entonces estarán
limpios. \bibverse{16} ero si no lavan sus ropas y se lavan con agua,
entonces son responsables de su culpa.''.

\hypertarget{section-17}{%
\section{18}\label{section-17}}

\bibverse{1} El Señor le dijo a Moisés: \bibverse{2} ``Dile a los
israelitas: Yo soy el Señor tu Dios. \bibverse{3} o sigas los caminos de
Egipto, donde vivías, y no sigas los caminos de Canaán, donde te
llevaré. No adoptes sus prácticas. \bibverse{4} az lo que te digo y
guarda mis reglas. Yo soy el Señor tu Dios. \bibverse{5} i guardas mis
reglas y haces lo que te digo, vivirás. Yo soy el Señor.

\bibverse{6} o tengas relaciones sexuales con un pariente cercano.
\bibverse{7} o avergüences a tu padre teniendo sexo con tu madre. Ella
es tu madre; no tengas sexo con ella. \bibverse{8} o tengas sexo con
ninguna de las esposas de tu padre y avergüenza a tu padre. \bibverse{9}
No tengas sexo con tu hermana,\footnote{\textbf{18:9} Esto claramente
  incluye a una media hermana o una hermanastra.}si es hija de tu padre
o de tu madre, o si nació en la misma casa que tú o en otro lugar.
\bibverse{10} o tengas sexo con tu nieta, la hija de tu hijo o la hija
de tu hija, porque eso sería algo vergonzoso para ti. \bibverse{11} o
tengas sexo con la hija de ninguna de las esposas de tu padre y tu
padre, ella es tu hermana. \bibverse{12} o tengas sexo con la hermana de
tu padre. Ella es un pariente cercano de tu padre. \bibverse{13} o
tengas relaciones sexuales con la hermana de tu madre. Ella es un
pariente cercano de tu madre. \bibverse{14} o avergüences a tu tío
teniendo sexo con su esposa. Ella es tu tía. \bibverse{15} o tengas sexo
con tu nuera. Ella es la esposa de tu hijo. No tengas sexo con ella.
\bibverse{16} o te acuestes con la esposa de tu hermano y avergüénzate
de él. \bibverse{17} o tengas sexo con una mujer y su hija. No tengas
sexo con la hija de su hijo o la hija de su hija. Son sus parientes
cercanos. Eso es algo que detesto. \bibverse{18} o te cases con la
hermana de tu esposa y tengas sexo con ella mientras tu esposa esté
viva. Serán esposas hostiles entre sí. \bibverse{19} o tengas sexo con
una mujer durante el tiempo que esté sucia debido a su período.
\bibverse{20} o cometas ningún acto sexual con la esposa de otro hombre.
Esto te contaminaría y te haría sucio.

\bibverse{21} No le des ninguno de tus hijos como sacrificio humano a
Moloc,\footnote{\textbf{18:21} Moloc era un dios pagano. Los sacrificios
  eran de niños que eran traídos vivos en las manos de los ídolos de
  metal que habían sido calentados con fuego.}porque no debes deshonrar
el carácter de tu Dios. Yo soy el Señor.

\bibverse{22} No tengas sexo con un hombre como con una mujer. Eso es
algo asqueroso. \bibverse{23} o tengas sexo con ningún animal. Esto te
contaminaría y te ensuciaría. \bibverse{24} na mujer no debe entregarse
a un animal para tener sexo con él. Eso es algo asqueroso. No se
contaminen y se ensucien haciendo algo así. Esa es la razón por la que
estoy expulsando a estas naciones de la tierra, se han contaminado a sí
mismos por todas estas prácticas. \bibverse{25} ncluso la tierra se ha
contaminado, así que la estoy castigando por los pecados cometidos por
la gente que vive allí, y la tierra los vomitará.

\bibverse{26} eroustedes deben hacer lo que yo os diga y guardar mis
reglas. No deben hacer ninguno de estos actos repugnantes, ya sea un
israelita o un extranjero que viva entre ustedes. \bibverse{27} as
personas que vivían en la tierra antes de ti practicaban todas estas
cosas repugnantes, y la tierra se contaminó. \bibverse{28} i contaminas
la tierra, te vomitará como lo hizo con las naciones anteriores a ti.

\bibverse{29} or consiguiente, cualquiera que haga cualquiera de estas
cosas repugnantes debe ser expulsado de su pueblo. \bibverse{30} ebes
aceptar mi demanda de que no sigas ninguna de estas prácticas
repugnantes hechas antes de tu llegada. No se contaminen ni se hagan
impuros. Yo soy el Señor su Dios''.

\hypertarget{section-18}{%
\section{19}\label{section-18}}

\bibverse{1} El Señor le dijo a Moisés, \bibverse{2} ``Dile a todos los
israelitas: Sean santos porque yo soy santo; yo soy el Señor su Dios.

\bibverse{3} Muestra respeto por tu madre y tu padre y guarda mis
sábados. Yo soy el Señor tu Dios.

\bibverse{4} Noacudas a los ídolos en busca de ayuda ni hagas imágenes
metálicas de dioses. Yo soy el Señor tu Dios.

\bibverse{5} uando sacrifiques una ofrenda de paz al Señor, asegúrate de
hacerlo correctamente para que Dios te acepte. \bibverse{6} ebe ser
comida el día que la sacrificas, o al día siguiente. Lo que quede al
tercer día debe ser quemado. \bibverse{7} i comes algo de ello al tercer
día, el sacrificio se vuelve repulsivo y no será aceptado. \bibverse{8}
l que lo coma se hará responsable de su culpa, porque ha ensuciado lo
que es santo para el Señor. Deben ser expulsados de su pueblo.

\bibverse{9} uando cosechen los cultivos de su tierra, no lo hagan hasta
los límites del campo, ni recojan lo que se ha perdido. \bibverse{10} o
recojas hasta la última uva de tu viñedo ni recojas las que han caído.
Déjalas para los pobres y los extranjeros. Yo soy el Señor tu Dios.

\bibverse{11} o robes. No mientas. No engañes.

\bibverse{12} No hagas juramentos en mi nombre que no sean verdaderos,
de lo contrario difamarás el carácter\footnote{\textbf{19:12}
  Literalmente, ``nombre''. A través de toda la Escritura el nombre se
  relaciona al carácter.}de tu Dios. Yo soy el Señor.

\bibverse{13} No engañes a los demás ni les robes. No te niegues a pagar
los salarios que se deben a los trabajadores hasta la mañana.

\bibverse{14} o hables mal de los sordos. No pongas objetos en el camino
de los ciegos para hacerlos tropezar. En vez de eso, muestra respeto a
tu Dios. Yo soy el Señor.

\bibverse{15} o seas un juez corrupto. No muestres favoritismo a los
pobres o a los ricos. Juzga a los demás con justicia.

\bibverse{16} No vayas por ahí difundiendo falsos rumores sobre la
gente. No te quedes callado cuando las vidas de otros están en
peligro.\footnote{\textbf{19:16} Literalmente, ``No te quedes callado en
  cuanto a la sangre de tu prójimo''.} Yo soy el Señor.

\bibverse{17} No te aferres a sentimientos de odio hacia los demás.
Habla honestamente con tus vecinos, para no pecar por ellos.
\bibverse{18} o busques venganza ni guardes rencor a nadie, sino ama a
tu prójimo como a ti mismo. Yo soy el Señor.

\bibverse{19} ¡Haz lo que te digo! No hagas que diferentes tipos de
ganado se reproduzcan juntos. No siembren sus campos con dos tipos
diferentes de semillas. No uses ropa confeccionada con dos materiales
distintos.

\bibverse{20} i un hombre tiene relaciones sexuales con una sirvienta a
la que se le ha prometido ser la esposa de otro hombre, pero que aún no
ha sido comprada o liberada, entonces se debe pagar una compensación.
Sin embargo, no deben ser asesinados, porque ella no ha sido liberada.
\bibverse{21} ero el hombre debe llevar un carnero como ofrenda de culpa
al Señor a la entrada del Tabernáculo de Reunión. \bibverse{22} l
sacerdote arreglará las cosas para él ante el Señor usando el carnero de
la ofrenda por el pecado que ha cometido, y su pecado será perdonado.

\bibverse{23} Cuando entres en la tierra y plantes cualquier tipo de
árbol frutal, trata la fruta al principio como impura.\footnote{\textbf{19:23}
  ``Impura'': Literalmente, ``incircunciso''.}Durante tres años tienes
prohibido comerlo. \bibverse{24} l cuarto año todo el fruto debe ser
dedicado al Señor como ofrenda de alabanza. \bibverse{25} in embargo, el
quinto año podrán comer el fruto y así tendrán una cosecha aún mayor. Yo
soy el Señor tu Dios.

\bibverse{26} o comas carne con sangre. No uses la adivinación o la
brujería.

\bibverse{27} ose corten el cabello a los lados de la cabeza ni se
corten la barba,\footnote{\textbf{19:27} Se cree que esta prohibición
  está asociada con la siguiente en relación con alguna ceremonia
  pagana.} \bibverse{28} no se corten el cuerpo en algún ritual pagano
para los muertos y se hagan tatuajes. Yo soy el Señor.

\bibverse{29} ocausen vergüenza a su hija convirtiéndola en una
prostituta, de lo contrario la tierra se volverá moral y espiritualmente
depravada.

\bibverse{30} uarden mis sábados y mostrad respeto por mi santuario. Yo
soy el Señor.

\bibverse{31} o intentes encontrar ayuda de médiums o espiritistas, ni
siquiera vayas a buscarlos, de lo contrario te corromperán. Yo soy el
Señor tu Dios.

\bibverse{32} Levántate y respeta a los ancianos. Muestra reverencia por
tu Dios. Yo soy el Señor.

\bibverse{33} o maltrates a los extranjeros que viven en tu país.
\bibverse{34} Trátalos como a un conciudadano, y átalos como a ti mismo,
porque una vez fuisteis extranjeros viviendo en Egipto. Yo soy el Señor
tu Dios.

\bibverse{35} o uses pesos y medidas deshonestas. \bibverse{36}
Asegúrate de que tus balanzas y pesos sean exactos, que tus medidas de
efa y hin sean correctas. Yo soy el Señor tu Dios que te sacó de Egipto.

\bibverse{37} uarda todas mis reglas y regulaciones, y asegúrate de que
las sigues. Yo soy el Señor''.

\hypertarget{section-19}{%
\section{20}\label{section-19}}

\bibverse{1} El Señor le dijo a Moisés: \bibverse{2} ``Diles a los
israelitas: estas normas son para los israelitas y los extranjeros que
viven entre ellos. Cualquiera que sacrifique sus hijos a Moloc debe ser
ejecutado. La comunidad debe apedrearlos hasta la muerte. \bibverse{3}
Los repudiaré y los expulsaré de su pueblo, porque al sacrificar sus
hijos a Moloc, han contaminado mi santuario y deshonrado mi reputación.
\bibverse{4} Si la comunidad decide mirar hacia otro lado y no ejecutar
a aquellos que sacrifican sus hijos a Moloch, entonces yo mismo tomaré
medidas contra ellos.\footnote{\textbf{20:4} ``Yo mismo tomaré medidas
  contra ellos'': añadido para mayor claridad.} \bibverse{5} Los
repudiaré a ellos y a su familia, y los expulsaré de su pueblo, y no
sólo a ellos, sino a todos los que los sigan en la prostitución
espiritual con Moloc.

\bibverse{6} ambién repudiaré y expulsaré de su gente a cualquiera que
vaya a médiums o espiritistas y de esta manera cometa prostitución
espiritual con ellos.

\bibverse{7} sí que dedíquense y sean santos, porque yo soy el Señor su
Dios. \bibverse{8} uarden mis reglas y pónganlas en práctica. Yo soy el
Señor que los santifica.

\bibverse{9} Cualquiera que maldiga a su padre o madre debe ser
ejecutado. Han maldecido a su padre o madre; ellos son los responsables
de su castigo.\footnote{\textbf{20:9} ``Son responsables de su
  castigo'': Literalmente, ``su sangre está sobre él''.}

\bibverse{10} Cualquier hombre que cometa adulterio con la esposa de
otro debe ser ejecutado, así como la mujer.

\bibverse{11} n hombre que tiene sexo con la esposa de su padre ha
traído la vergüenza a su padre. Tanto el hombre como la mujer deben ser
ejecutados. Ambos deben ser ejecutados; ellos son responsables de su
castigo.

\bibverse{12} n hombre que tiene relaciones sexuales con su nuera debe
ser ejecutado, así como ella. Han hecho algo perverso; ellos son los
responsables de su castigo.

\bibverse{13} os hombres que tienen relaciones sexuales con otros
hombres como con una mujer habrán hecho algo repugnante. Deben ser
ejecutados; ellos son culpables de su castigo.

\bibverse{14} n hombre que se casa con una mujer y con su madre ha
actuado de manera perversa. Deben ser quemados hasta la muerte para que
no haya tal maldad entre ustedes.

\bibverse{15} l hombre que tenga relaciones sexuales con un animal debe
ser ejecutado, y el animal debe ser matado también.

\bibverse{16} na mujer que se entrega a un animal para tener sexo con él
debe ser ejecutada, junto con el animal. Ambos deben ser asesinados;
ellos son responsables de su castigo.

\bibverse{17} n hombre que se casa con su hermana, ya sea hija de su
padre o de su madre, y tienen relaciones sexuales, ha hecho algo
vergonzoso. Deben ser expulsados de su pueblo delante de todos. Ha
avergonzado a su hermana; él tiene la responsabilidad de su castigo.

\bibverse{18} n hombre que tiene relaciones sexuales con una mujer que
está teniendo su período ha expuesto de dónde proviene el flujo de
sangre de ella, y ella también lo ha hecho. Ambos deben ser expulsados
de su pueblo.

\bibverse{19} o tengas relaciones sexuales con tu tía, ya sea por parte
de tu padre o de tu madre, porque eso avergüenza a tu propia familia.
Ambos serán responsables de su pecado.

\bibverse{20} n hombre que tiene sexo con la esposa de su tío trae
vergüenza a su tío. Ellos serán responsables de su pecado; morirán sin
tener hijos.

\bibverse{21} n hombre que se casa con la mujer de su hermano hace algo
que es impuro. Ha avergonzado a su hermano; la pareja no tendrá hijos.

\bibverse{22} sí que guarda todas mis reglas y regulaciones, para que la
tierra donde te llevo a vivir no te vomite. \bibverse{23} o sigas las
prácticas de las naciones que estoy expulsando por ti. Las detesté
porque hicieron todas estas cosas malvadas. \bibverse{24} ero te he
prometido que te harás cargo de su tierra. Te la voy a dar en propiedad,
una tierra que fluye con leche y miel. Yo soy el Señor tu Dios, que te
ha hecho un pueblo distinto de todos los demás.

\bibverse{25} sí que asegúrate de observar la diferencia entre las aves
y animales limpios e inmundos. No se vuelvan inmundos por causa de
ningún animal o ave, ni por nada que corra por la tierra. He dejado
clara la diferencia: son inmundos para ustedes. \bibverse{26} stedes
serán santos para mí porque yo soy santo. Yo soy el Señor, y los he
hecho un pueblo distinto de todas las demás naciones. Ustedes me
pertenecen.

\bibverse{27} odo hombre o mujer que sea médium o espiritista debe ser
ejecutado. Deben ser apedreados hasta la muerte; ellos son responsables
de su castigo''.

\hypertarget{section-20}{%
\section{21}\label{section-20}}

\bibverse{1} El Señor le dijo a Moisés: ``Dile a los hijos de Aarón, los
sacerdotes: Un sacerdote no debe ensuciarse tocando el cadáver de
ninguno de sus parientes. \bibverse{2} as únicas excepciones son para su
familia inmediata. Esto incluye a su madre, padre, hijo, hija o hermano,
\bibverse{3} o a su hermana soltera ya que es un pariente cercano porque
no tiene marido. \bibverse{4} o debe hacerse impuro por aquellos que
sólo están relacionados con él por matrimonio; no debe hacerse
ceremonialmente impuro.

\bibverse{5} os sacerdotes no deben afeitarse las zonas calvas de la
cabeza, ni recortar los lados de la barba, ni cortar el
cuerpo.\footnote{\textbf{21:5} Ver19:27.} \bibverse{6} Deben ser santos
para su Dios y no deshonrar la reputación de su Dios. Ellos son los que
presentan las ofrendas de comida al Señor, la comida de su Dios. Por
consiguiente, deben ser santos.

\bibverse{7} l sacerdote no debe casarse con una mujer que se ha vuelto
impura por la prostitución\footnote{\textbf{21:7} Esto puede incluir la
  prostitución culta en los templos paganos.} o que se ha divorciado de
su marido, porque el sacerdote debe ser santo para su Dios. \bibverse{8}
o considerarán santo porque presenta las ofrendas de comida a su Dios.
Él será santo para ti, porque yo soy santo. Yo soy el Señor, y te
elegícomomi pueblo especial.\footnote{\textbf{21:8} ``Te elegí como mi
  pueblo especial'': Literalmente, ``te aparté''.} \bibverse{9} La hija
de un sacerdote que se hace impura por la prostitución hace impura a su
padre. Ella debe ser ejecutada por medio del holocausto.

\bibverse{10} El sumo sacerdote tiene el lugar más alto entre los otros
sacerdotes. Ha sido ungido con aceite de oliva vertido en su cabeza y ha
sido ordenado para llevar la ropa sacerdotal. No debe dejar su pelo
despeinado ni rasgar su ropa.\footnote{\textbf{21:10} Estas eran señales
  de angustia o de luto.} \bibverse{11} No debe acercarse a ningún
cadáver. No debe hacerse impuro, aunque sea por su propio padre o madre.
\bibverse{12} o debe salir para tratar asuntos de un difunto\footnote{\textbf{21:12}
  ``Tratar asuntos de un difunto'': añadido para mayor claridad.Eso no
  significa que el sumo sacerdote nunca pudiera dejar el santuario.} o
hacer que el santuario de su Dios sea impuro porque ha sido dedicado por
el aceite de la unción de su Dios. Yo soy el Señor.

\bibverse{13} ólo puede casarse con una virgen. \bibverse{14} o debe
casarse con una viuda, una mujer divorciada, o con una que se haya
vuelto impura a través de la prostitución. Tiene que casarse con una
virgen de su propio pueblo, \bibverse{15} para que no haga inmundos a
sus hijos entre su pueblo,\footnote{\textbf{21:15} Tener una esposa
  extranjera significaría que cualquier hijo del matrimonio sería
  considerado impuro. Tampoco podrían seguir a su padre en el
  sacerdocio, y de hecho él tampoco podría continuar como sacerdote.}
porque yo soy el Señor que lo hace santo''.

\bibverse{16} l Señor le dijo a Moisés, \bibverse{17} ``Dile a Aarón:
Estas reglas se aplican a todas las generaciones futuras. Si alguno de
sus descendientes tiene un defecto físico, no se le permite venir a
presentar las ofrendas de su Dios. \bibverse{18} o se permite a ningún
hombre hacer esto si tiene algún defecto, incluyendo a cualquier persona
ciega, lisiada, desfigurada en la cara o con deformidades, \bibverse{19}
cualquier persona que tenga un pie o brazo roto, \bibverse{20} cualquier
persona que sea jorobada o enana, o que tenga cataratas, llagas en la
piel o costras, o un testículo dañado.

\bibverse{21} ningún descendiente del sacerdote Aarón que tenga un
defecto se le permite venir y presentar las ofrendas de comida al Señor.
Debido a que tiene un defecto, no debe venir y ofrecer la comida de su
Dios. \bibverse{22} e le permite comer la comida del Lugar Santísimo de
su Dios y también del santuario, \bibverse{23} pero como tiene un
defecto, no se le permite acercarse a la cortina o al altar, para que mi
santuario y todo lo que hay en él no se vuelva impuro, porque yo soy el
Señor que los hago santos''. \bibverse{24} Moisés repitió esto a Aarón y
a sus hijos, y a todos los israelitas.

\hypertarget{section-21}{%
\section{22}\label{section-21}}

\bibverse{1} El Señor le dijo a Moisés, \bibverse{2} ``Dile a Aarón y a
sus hijos que trabajen con dedicación\footnote{\textbf{22:2} ``Con
  dedicación'': o ``que traten con respeto''. El sentido es que como las
  ofrendas hechas por el pueblo se daban en dedicación, los sacerdotes
  debían tratarlas de la misma manera.} al tratar con las ofrendas
sagradas que los israelitas me han ofrecido, para que no deshonren mi
santo nombre. Yo soy el Señor.

\bibverse{3} Diles: Estas reglas se aplican a todas las generaciones
futuras. Si alguno de sus descendientes en estado inmundo se acerca a
las ofrendas sagradas que los israelitas dedican a honrar al Señor, esa
persona debe ser expulsada de mi presencia. Yo soy el Señor.

\bibverse{4} i uno de los descendientes de Aarón tiene una enfermedad de
la piel o una secreción, no se le permite comer las ofrendas sagradas
hasta que esté limpio. Cualquiera que toque algo que se haya vuelto
impuro por un cadáver o por un hombre que haya tenido una liberación de
semen, \bibverse{5} o cualquiera que toque un animal o una persona
impura, (cualquiera que sea la impureza), \bibverse{6} cualquiera que
toque algo así permanecerá impuro hasta la noche. No se le permite comer
de las ofrendas sagradas a menos que se haya lavado con agua.
\bibverse{7} l atardecer se limpiará, y entonces se le permitirá comer
de las ofrendas sagradas porque le proporcionan su comida. \bibverse{8}
o debe comer nada que haya muerto o haya sido matado por animales
salvajes, porque eso lo haría impuro. Yo soy el Señor. \bibverse{9} os
sacerdotes deben hacer lo que yo les exija, para que no se hagan
culpables y mueran por no haberlo hecho, tratando mis exigencias con
desprecio. Yo soy el Señor que los hace santos.

\bibverse{10} ualquiera que no sea parte de la familia de un sacerdote
no puede comer las sagradas ofrendas. Esto también se aplica al huésped
de un sacerdote o a su trabajador asalariado. \bibverse{11} in embargo,
si un sacerdote usa su propio dinero para comprar un esclavo, o si un
esclavo nace en la casa del sacerdote, entonces a ese esclavo se le
permite comer su comida. \bibverse{12} i la hija del sacerdote se casa
con un hombre que no es sacerdote, no se le permite comer las ofrendas
sagradas. \bibverse{13} ero si la hija de un sacerdote sin hijos es
viuda o divorciada y regresa a la casa de su padre, se le permite comer
la comida de su padre como lo hacía cuando era niña. Pero nadie fuera de
la familia del sacerdote puede comerla.

\bibverse{14} ualquiera que coma una ofrenda sagrada por error debe
pagar una compensación añadiendo un quinto a su valor, y dárselo todo al
sacerdote. \bibverse{15} os sacerdotes no deben hacer inmundas las
ofrendas sagradas que los israelitas presentan al Señor \bibverse{16}
permitiendo que el pueblo las coma y al hacerlo asuman el castigo por la
culpa. Porque yo soy el Señor que las hace santas''.

\bibverse{17} l Señor le dijo a Moisés, \bibverse{18} Dile a Aarón, a
sus hijos y a todos los israelitas: Si tú o un extranjero que vive
contigo quiere hacer un regalo como holocausto al Señor, ya sea para
cumplir una promesa o como una ofrenda voluntaria, esto es lo que debes
hacer. \bibverse{19} i va a ser aceptado en tu nombre debes ofrecer un
macho sin defectos de los rebaños de vacas, ovejas o cabras.
\bibverse{20} o presente nada que tenga un defecto porque no será
aceptado en su nombre.

\bibverse{21} Si quieres presentar una ofrenda de paz al Señor de la
manada o rebaño para cumplir una promesa o como una ofrenda de libre
albedrío, para ser aceptable el animal debe ser perfecto, completamente
sin defectos. \bibverse{22} o presente al Señor un animal que esté
ciego, herido o dañado de alguna manera, o que tenga verrugas, llagas en
la piel o costras. No coloques ningún animal que tenga esto en el altar
como ofrenda de comida al Señor.

\bibverse{23} in embargo, puedes presentar una ofrenda voluntaria de un
toro o una oveja que tenga una pata demasiado larga o demasiado corta,
pero no si se trata de un sacrificio para cumplir una promesa.
\bibverse{24} o presente al Señor un animal con los testículos dañados
accidental o deliberadamente. No se le permite sacrificar ninguno de
estos animales dañados en su tierra. \bibverse{25} ampoco se le permite
aceptar tales animales de un extranjero para ser dañados y
defectuosos''.

\bibverse{26} l Señor le dijo a Moisés, \bibverse{27} ``Cuando nace un
toro, una oveja o una cabra, debe permanecer con su madre durante siete
días. Después de ocho días puede ser aceptado como una ofrenda de comida
al Señor. \bibverse{28} in embargo, no se debe matar un toro o una oveja
y sus crías en el mismo día. \bibverse{29} uando presente una ofrenda de
agradecimiento al Señor, asegúrese de hacerlo de manera que sea aceptada
en su nombre. \bibverse{30} ebe ser comida el mismo día. No dejes nada
de eso hasta la mañana. Yo soy el Señor. \bibverse{31} uarda mis reglas
y ponlas en práctica. Yo soy el Señor. \bibverse{32} o deshonres mi
santo nombre. Seré santo ante tus ojos. Yo soy el Señor que tesantifica.
\bibverse{33} oy el que te sacó de Egipto para ser tu Dios. Yo soy el
Señor''.

\hypertarget{section-22}{%
\section{23}\label{section-22}}

\bibverse{1} El Señor le dijo a Moisés, \bibverse{2} ``Diles a los
israelitas que estas son mis fiestas religiosas, las fiestas del Señor
que debes llamar como tiempos sagrados en los que nos reuniremos.
\bibverse{3} ienen seis días para trabajar, pero el séptimo día es un
sábado de completo descanso, un día sagrado de reunión. No trabajarán.
Es el Sábado del Señor en todos los lugares donde vivas.

\bibverse{4} stas son las fiestas religiosas del Señor, las reuniones
sagradas en las que comieron para anunciar, en su fecha señalada:
\bibverse{5} La Pascua del Señor comienza en la tarde del día catorce
del primer mes. \bibverse{6} a fiesta del Señor de los panes sin
levadura comienza el día quince del primer mes. Durante siete días el
pan que coman debe ser hecho sin levadura. \bibverse{7} l primer día se
celebra una reunión sagrada. No debes hacer ninguno de sus trabajos
habituales. \bibverse{8} urante siete días presentarás ofrendas de
comida al Señor. Habrá una reunión sagrada el séptimo día. No debes
hacer nada de tu trabajo habitual''.

\bibverse{9} l Señor le dijo a Moisés, \bibverse{10} ``Diles a los
israelitas que cuando entren en la tierra que yo les doy y recojansus
cosechas, deben llevarle al sacerdote gran parte del grano de las
primicias de su cosecha. \bibverse{11} l agitará la pila de grano ante
el Señor para que sea aceptada en su nombre. El sacerdote hará esto el
día siguiente al sábado. \bibverse{12} uando agites la pila de grano,
presentarás al Señor un cordero de un año sin defectos como holocausto,
\bibverse{13} junto con su ofrenda de grano de dos décimas de efa de la
mejor harina mezclada con aceite de oliva (una ofrenda de comida al
Señor para ser aceptada por él) y su ofrenda de bebida de un cuarto de
hin de vino. \bibverse{14} o comas pan, grano tostado o grano nuevo
hasta el momento en que lleves esta ofrenda a tu Dios. Esta norma es
para siempre y para las futuras generaciones en todos los lugares donde
vivas.

\bibverse{15} uenta siete semanas completas desde el día después del
sábado, el día que trajiste la pila de grano como ofrenda ondulada.
\bibverse{16} uenta cincuenta días hasta el día después del séptimo
sábado, y en ese día presenta una ofrenda de grano nuevo al Señor.
\bibverse{17} raigan dos panes de sus casas como ofrenda mecida.
Hacedlos de dos décimas de efa de la mejor harina, cocidos con levadura,
como primicias para el Señor. \bibverse{18} demás del pan, presenten
siete corderos machos de un año sin defectos, un novillo y dos carneros.
Serán un holocausto para el Señor, así como sus ofrendas de grano y sus
ofrendas de bebida, una ofrenda de comida para el Señor para ser
aceptada por él. \bibverse{19} resenten una cabra macho como ofrenda por
el pecado y dos corderos macho de un año como ofrenda de paz.
\bibverse{20} l sacerdote agitará los corderos como ofrenda mecida ante
el Señor, junto con el pan de las primicias. El pan y los dos corderos
son sagrados para el Señor y pertenecen al sacerdote. \bibverse{21} se
mismo día anunciará una reunión santa, y no deberá hacer ningún trabajo
habitual. Este reglamento es para todos los tiempos y para las
generaciones futuras, dondequiera que vivan.

\bibverse{22} Cuando coseches los cultivos de tu tierra, no lo hagas
hasta los bordes del campo, o recoge lo que se ha perdido. Déjalos para
los pobres y los extranjeros. Yo soy el Señor tu Dios''.

\bibverse{23} l Señor le dijo a Moisés: \bibverse{24} ``Diles a los
israelitas que el primer día del séptimo mes deben tener un sábado
especial de completo descanso, una reunión santa que se anuncia con el
sonido de las trompetas. \bibverse{25} o hagas nada de tu trabajo
habitual, sino que debes presentar una ofrenda de comida al Señor''.

\bibverse{26} l Señor le dijo a Moisés, \bibverse{27} ``El Día de la
Expiación es el décimo día de este séptimo mes. Celebrarán una reunión
sagrada, negándose a sí mismos,\footnote{\textbf{23:27} ``Negándose a sí
  mismos:'' Esto a menudo se refiere al ayuno y a la abstinencia de
  placeres.}y presentarán una ofrenda de comida al Señor. \bibverse{28}
En este día no debes hacer nada de tu trabajo habitual porque es el Día
de la Expiación, cuando las cosas se arreglan para ti ante el Señor tu
Dios. \bibverse{29} ualquiera que no practique la el ayuno en este día
debe ser expulsado de su pueblo. \bibverse{30} estruiré a cualquiera de
ustedes que haga cualquier trabajo en este día. \bibverse{31} o hagan
ningún tipo de trabajo. Este reglamento es para siempre y para las
futuras generaciones dondequiera que vivan. \bibverse{32} erá un sábado
de completo descanso para ustedes, y ayunarán. Observaránsu sábado desde
la tarde del noveno día del mes hasta la tarde del día siguiente''.

\bibverse{33} l Señor dijo a Moisés, \bibverse{34} ``Di a los israelitas
que la fiesta de los tabernáculos para honrar al Señor comienza el día
quince del séptimo mes y dura siete días. \bibverse{35} l primer día
tened una reunión sagrada. No debes hacer nada de tu trabajo habitual.
\bibverse{36} urante siete días presentarás ofrendas de comida al Señor.
El octavo día tendrás otra reunión santa y presentarás una ofrenda al
Señor. Es una reunión para la adoración. No debes hacer nada de tu
trabajo habitual.

\bibverse{37} (Estas son las fiestas sagradas del Señor, que anunciarán
como reuniones sagradas para presentar ofrendas de comida al Señor.
Estas incluyen holocaustos, ofrendas de grano, sacrificios y ofrendas de
bebida, cada una de ellas de acuerdo al día específico. \bibverse{38}
odas estas ofrendas son adicionales a las de los sábados del Señor.
También son adicionales a tus regalos, a todas tus ofrendas para cumplir
promesas, y a todas las ofrendas voluntarias que presentas al Señor).

\bibverse{39} elebrarán una fiesta en honor del Señor durante siete
días, el día quince del séptimo mes, una vez que hayas cosechado sus
cosechas. El primer día y el octavo día son sábados de completo
descanso. \bibverse{40} l primer día recogerás ramas de árboles grandes,
de palmeras, de árboles frondosos y de sauces de río, y celebrarás ante
el Señor tu Dios durante siete días. \bibverse{41} elebraránesta fiesta
para honrar al Señor durante siete días cada año. Este reglamento es
para todos los tiempos y para todas las generaciones futuras.

\bibverse{42} Vivirás en refugios temporales\footnote{\textbf{23:42}
  Hecho de las ramas de los árboles mencionados en el versículo 40.}por
siete días. Todo israelita nacido en el país debe vivir en refugios,
\bibverse{43} para que sus descendientes recuerden que yo hice vivir a
los israelitas en refugios cuando los saqué de Egipto. Yo soy el Señor
tu Dios''.

\bibverse{44} sí que Moisés les contó a los israelitas todo sobre las
fiestas del Señor.

\hypertarget{section-23}{%
\section{24}\label{section-23}}

\bibverse{1} El Señor le dijo a Moisés, \bibverse{2} Ordena a los
israelitas que te traigan aceite de oliva puro y prensado para las
lámparas, para que siempre estén encendidas. \bibverse{3} esde la tarde
hasta la mañana Aarón debe cuidar las lámparas continuamente ante el
Señor, fuera del velo del Testimonio en el Tabernáculo de Reunión. Esta
regulación es para todos los tiempos y para todas las generaciones
futuras. \bibverse{4} ebe cuidar constantemente las lámparas puestas en
el candelabro de oro puro ante el Señor.

\bibverse{5} sando la mejor harina hornea doce panes, con dos décimas de
un efa de harina por cada pan. \bibverse{6} Colócalos en dos pilas, seis
en cada pila, sobre la mesa de oro puro que está delante del Señor.
\bibverse{7} oner incienso puro al lado de cada pila para que vaya con
el pan y sirva de recordatorio, una ofrenda al Señor. \bibverse{8} ada
sábado se pondrá el pan delante del Señor, dado por los israelitas como
una señal continua del acuerdo eterno. \bibverse{9} s para Aarón y sus
descendientes. Deben comerlo en un lugar santo, pues deben tratarlo como
una parte santísima de las ofrendas de alimentos dadas al Señor. Es su
parte de las ofrendas de comida para siempre''.

\bibverse{10} Un día un hombre que tenía una madre israelita y un padre
egipcio entró en el campamento israelita y tuvo una pelea con un
israelita. \bibverse{11} El hijo de la mujer israelita maldijo el nombre
del Señor. Así que lo llevaron ante Moisés. (Su madre se llamaba
Selomit, hija de Dibri, de la tribu de Dan.) \bibverse{12} Lo detuvieron
hasta que quedó claro lo que el Señor quería que hicieran al respecto.

\bibverse{13} l Señor le dijo a Moisés, \bibverse{14} Lleva al hombre
que me maldijo fuera del campamento. Que todos los que le oyeron
maldecir pongan sus manos sobre su cabeza; y que todos le apedreen hasta
la muerte. \bibverse{15} Diles a los israelitas que cualquiera que
maldiga a su Dios será castigado por su pecado. \bibverse{16} ualquiera
que maldiga el nombre del Señor debe ser ejecutado. Todos ustedeslo
apedrearán hasta la muerte, tanto si es un extranjero que vive con
ustedes como si es un israelita. Si maldicen mi nombre, deben ser
ejecutados.

\bibverse{17} ualquiera que mate a alguien más debe ser ejecutado.
\bibverse{18} ualquiera que mate a un animal tiene que reemplazarlo -
una vida por otra. \bibverse{19} i alguien hiere a otra persona, lo que
haya hecho debe serle hecho: \bibverse{20} un hueso roto por un hueso
roto, ojo por ojo, diente por diente. Sea cual sea la forma en que hayan
herido a la víctima, se les debe hacer lo mismo. \bibverse{21} ualquiera
que mate un animal tiene que reemplazarlo, pero cualquiera que mate a
alguien más debe ser ejecutado. \bibverse{22} los extranjeros que viven
con ustedes se les aplican las mismas leyes que a los israelitas, porque
yo soy el Señor su Dios''.

\bibverse{23} Moisés dijo esto a los israelitas, y ellos llevaron al
hombre que maldijo al Señor fuera del campamento y lo apedrearon hasta
la muerte. Los israelitas hicieron lo que el Señor le ordenó a Moisés
que hiciera.

\hypertarget{section-24}{%
\section{25}\label{section-24}}

\bibverse{1} El Señor le dijo a Moisés en el Monte Sinaí, \bibverse{2}
Dile a los israelitas: Cuando entrena la tierra que les daré, la tierra
misma debe también observar un descanso sabático en honor al Señor.
\bibverse{3} eis años puedes cultivar tus campos, cuidar tus viñedos y
cosechar tus cultivos. \bibverse{4} ero el séptimo año ha de ser un
sábado de completo descanso para la tierra, un sábado en honor al Señor.
No planten sus campos ni cuiden sus viñedos. \bibverse{5} o cosechen lo
que haya crecido en sus campos, ni recojan las uvas que no hayan
cuidado. La tierra debe tener un año de completo descanso. \bibverse{6}
ueden comer lo que la tierra produzca durante el año sabático. Esto se
aplica a ti mismo, a tus esclavos y esclavas, a los trabajadores
asalariados y a los extranjeros que viven contigo, \bibverse{7} y a tu
ganado y a los animales salvajes que viven en tu tierra. Todo lo que
crezca puede ser usado como alimento.

\bibverse{8} uenta siete años sabáticos, es decir, siete veces siete
años, para que los siete años sabáticos sumen cuarenta y nueve años.
\bibverse{9} uegohaz sonar la trompeta por todo el país el décimo día
del séptimo mes, que es el Día de la Expiación. Asegúrate de que esta
señal se oiga en todo el país. \bibverse{10} Dedicarás el año cincuenta
y anunciarás la libertad en todo el país para todos los que viven allí.
Este será su Jubileo, cuando cada uno de ustedes vuelva a reclamar su
propiedad y a formar parte de su familia una vez más.\footnote{\textbf{25:10}
  Esto significaba que cualquier propiedad vendida durante los 50 años
  anteriores volvía a su dueño original, y que cualquiera que se hubiera
  convertido en esclavo era liberado y se le permitía volver a su propia
  familia.} \bibverse{11} El 50º año será un jubileo para ti. No
siembren la tierra, no cosechen lo que haya podido crecer en sus campos,
ni recojan las uvas de sus viñedos que no hayan cuidado. \bibverse{12} s
un Jubileo y debe ser sagrado para ustedes. Podrán comer todo lo que
produzca la tierra. \bibverse{13} n este año jubilar, cada uno de
ustedes volverá a su propiedad.

\bibverse{14} i venden tierra a su vecino, o le compran tierra, no se
exploten mutuamente. \bibverse{15} uando compren a su prójimo, calculen
cuántos años han pasado desde el último Jubileo, pues él les venderá
según los años de cosecha que queden. \bibverse{16} uantos más años
queden, más pagarán; cuantos menos años queden, menos pagarán, porque en
realidad les está vendiendo un número determinado de cosechas.
\bibverse{17} ose exploten los unos a los otros, sino respeten a Dios,
porque yo soy el Señor su Dios.

\bibverse{18} uarden mis reglas y observen mis mandamientos, para que
puedan vivir con seguridad en la tierra. \bibverse{19} ntonces la tierra
producirá una buena cosecha, para que tengas suficiente comida y
vivasseguro en ella. \bibverse{20} ero si preguntas: ``¿A qué iremos en
el séptimo año si no sembramos o cosechamos nuestros cultivos?''
\bibverse{21} Yo te bendeciré en el sexto año, para que la tierra
produzca una cosecha que sea suficiente para tres años. \bibverse{22}
omo sembrarán en el octavo año, seguirán comiendo de esa cosecha, que
durará hasta su cosecha en el noveno año.

\bibverse{23} a tierra no debe ser vendida permanentemente, porque
realmente me pertenece. Para mí ustedes son sólo extranjeros y viajeros
de paso. \bibverse{24} Así que cualquier tierra que compren, deben hacer
arreglos para devolverlo a su dueño original.\footnote{\textbf{25:24}
  ``Devolverlo a su dueño original'': Literalmente, ``la redención de la
  tierra''.} \bibverse{25} Si uno de los tuyos se vuelve pobre y te
vende parte de su tierra, su familia cercana puede venir y comprar de
nuevo lo que ha vendido. \bibverse{26} in embargo, si no tienen a nadie
que pueda volver a comprarla, pero mientras tanto su situación
financiera mejora y tienen suficiente para volver a comprar la tierra,
\bibverse{27} trabajarán cuántos años han pasado desde la venta, y
devolverán el saldo a la persona que la compró, y volverán a su
propiedad. \bibverse{28} Si no pueden reunir lo suficiente para pagar a
la persona por la tierra, el comprador seguirá siendo su propietario
hasta el Año Jubilar. Pero en el Jubileo la tierra será devuelta para
que el propietario original pueda volver a su propiedad.

\bibverse{29} i alguien vende una casa situada en una ciudad amurallada,
tiene derecho a comprarla de nuevo durante un año completo después de
venderla. Puede ser comprada de nuevo en cualquier momento durante ese
año. \bibverse{30} i no se recompra al final del año, la propiedad de la
casa en la ciudad amurallada se transfiere de forma permanente al que la
compró y a sus descendientes. No será devuelta en el Jubileo.
\bibverse{31} ero las casas de las aldeas que no tienen muros a su
alrededor deben ser tratadas como si estuvieran en el campo. Pueden ser
compradas de nuevo, y serán devueltas en el Jubileo.

\bibverse{32} in embargo, los levitas siempre tienen el derecho de
volver a comprar sus casas en los pueblos que les pertenecen.
\bibverse{33} odo lo que los levitas poseen puede ser comprado de nuevo,
incluso las casas vendidas en sus ciudades, y debe ser devuelto en el
Jubileo. Eso es porque las casas en las ciudades de los levitas son lo
que se les dio en propiedad como su parte entre los israelitas.
\bibverse{34} in embargo, los campos que rodean sus ciudades no deben
ser vendidos porque pertenecen a los levitas permanentemente.

\bibverse{35} Si alguno de los tuyos se vuelve pobre y no puede
subsistir, debes ayudarlos de la misma manera que ayudarías a un
extranjero o a un extraño, para que puedan seguir viviendo en tu
vecindario. \bibverse{36} o les hagas pagar ningún interés o exigir más
de lo que pidieron prestado, pero respeta a tu Dios para que puedan
seguir viviendo en tu zona. \bibverse{37} o les prestes plata con
intereses ni les vendas comida a un precio exagerado. \bibverse{38}
ecuerda, yo soy el Señor tu Dios que te sacó de Egipto para darte la
tierra de Canaán y ser tu Dios.

\bibverse{39} i alguno de los tuyos se hace pobre y tiene que venderse
para trabajar para ti, no le obligues a trabajar como esclavo.
\bibverse{40} az que vivan contigo como un trabajador asalariado que se
queda contigo por un tiempo. Trabajarán para usted hasta el año del
Jubileo. \bibverse{41} ntonces ellos y sus hijos deben ser liberados, y
pueden volver a su familia y a la propiedad de su familia. \bibverse{42}
Los israelitas no deben ser vendidos como esclavos porque me pertenecen
como mis esclavos - los saqué de Egipto. \bibverse{43} o los traten con
brutalidad. Tengan respeto por su Dios.

\bibverse{44} ompra tus esclavos y esclavas de las naciones vecinas.
\bibverse{45} ambién puedes comprarlos a los extranjeros que han venido
a vivir entre ustedes, o a sus descendientes nacidos en tu tierra.
Puedes tratarlos como tu propiedad. \bibverse{46} Puedes pasarlos a tus
hijos para que los hereden como propiedad después de tu muerte. Puedes
convertirlos en esclavos de por vida, pero no debes tratar brutalmente
como esclavo a ninguno de tu propio pueblo, los israelitas.

\bibverse{47} i un extranjero entre ustedes tiene éxito, y uno de los
suyos que vive cerca se empobrece y se vende al extranjero o a un
miembro de su familia, \bibverse{48} todavía tienen derecho a ser
comprados de nuevo después de la venta. Un miembro de su familia puede
volver a comprarlos. \bibverse{49} Un tío o primo o cualquier pariente
cercano de su familia puede volver a comprarlos. Si tienen éxito, pueden
volver a comprarse a sí mismos. \bibverse{50} l interesado y su
comprador calcularán el tiempo desde el año de la venta hasta el año del
jubileo. El precio dependerá del número de años, calculado con la tarifa
diaria de un trabajador asalariado. \bibverse{51} i quedan muchos años,
deberán pagar un porcentaje mayor del precio de compra. \bibverse{52} i
sólo quedan unos pocos años antes del Año Jubilar, entonces sólo tienen
que pagar un porcentaje dependiendo del número de años que les queden.
\bibverse{53} eben vivir con su propietario extranjero como un
trabajador asalariado, contratado de año en año, pero procuren que el
propietario no lo trate brutalmente. \bibverse{54} i no son recomprados
de ninguna de las maneras descritas, ellos y sus hijos serán liberados
en el Año Jubilar. \bibverse{55} Porque los israelitas me pertenecen
como mis esclavos. Son mis esclavos, yo los saqué de Egipto. Yo soy el
Señor tu Dios''.

\hypertarget{section-25}{%
\section{26}\label{section-25}}

\bibverse{1} ``No hagas ídolos en ninguna parte de la tierra ni te
inclinespara adorarlos, ya sean imágenes o altares sagrados, o
esculturas de piedra. Porque yo soy el Señor tu Dios.

\bibverse{2} Guarda mis sábados y respetami santuario. Yo soy el Señor.

\bibverse{3} Si sigues mis reglas y guardas mis mandamientos,
\bibverse{4} me aseguraré de que llueva en el tiempo adecuado para que
la tierra crezca bien y los árboles den su fruto. \bibverse{5} u tiempo
de trilla durará hasta la cosecha de la uva, y la cosecha de la uva
hasta el momento en que tengas que volver a sembrar. Tendrán más que
suficiente para comer y vivirán seguros en su tierra. \bibverse{6} e
aseguraré de que tu tierra esté en paz. Podrás dormir sin tener miedo de
nada. Me desharé de los animales peligrosos de la tierra, y no sufrirás
ningún ataque violento del enemigo.\footnote{\textbf{26:6} ``no sufrirás
  ningún ataque violento'': Literalmente, ``No pasará espada por tu
  tierra''.} \bibverse{7} Perseguirás a tus enemigos y los matarás con
la espada. \bibverse{8} inco de ustedes matarán a cien, y cien de
ustedes matarán a diez mil. Destruirás a tus enemigos.

\bibverse{9} Vendré a ayudarte, para que te reproduzcas y aumentes en
número, y confirmaré mi acuerdo contigo. \bibverse{10} eguirás comiendo
tu viejo stock de grano cuando necesites deshacerte de él para poder
almacenar el nuevo grano. \bibverse{11} endré a vivir contigo, no te
rechazaré. \bibverse{12} iempre estaré a tu lado. Seré tu Dios, y tú
serás mi pueblo. \bibverse{13} o soy el Señor tu Dios, que te sacó de
Egipto para que no tuvieras que ser más esclavo de los egipcios. Rompí
el yugo que te mantenía agachado y te ayudaba a mantenerte erguido.

\bibverse{14} ero si no me prestas atención y haces lo que te digo;
\bibverse{15} si rechazas mis leyes, odias mis reglamentos y te niegas a
seguir mis mandamientos y, por consiguiente, rompes mi acuerdo,
\bibverse{16} entonces esto es lo que te voy a hacer: Te haré entrar en
pánico y sufrirás enfermedades como tuberculosis y fiebre, que te
dejarán ciego y te consumirán. Será inútil para ti sembrar en tus campos
porque tus enemigos se comerán la cosecha. \bibverse{17} e volveré
contra ti y serás derrotado por tus enemigos. La gente que te odia
gobernará sobre ti, ¡y huirás incluso cuando nadie te esté persiguiendo!

\bibverse{18} i después de todo esto todavía te niegas a obedecerme,
pasaré a castigarte siete veces por tus pecados. \bibverse{19} omperé tu
fuerza autosuficiente de la que estás tan orgulloso, y haré que tu cielo
sea duro como el hierro y tu tierra dura como el bronce. \bibverse{20}
Tu fuerza será en completamente en vano\footnote{\textbf{26:20} En otras
  palabras, cultivar la tierra no tendrá sentido.}porque su tierra no
producirá cultivos, y sus árboles no darán frutos.

\bibverse{21} i continúas oponiéndote a mí y negándote a hacer lo que te
digo, haré que tus castigos sean siete veces peores, basados en tus
pecados. \bibverse{22} nviaré animales salvajes a matar a sus hijos, a
eliminar su ganado y a haceros tan pocos que no habrá nadie en sus
caminos.

\bibverse{23} in embargo, si a pesar de toda esta corrección no cambian
sino que siguen en rebeldía contra a mí, \bibverse{24} entonces tomaré
medidas contra ustedes. Te castigaré siete veces por tus pecados.
\bibverse{25} nviaré a los enemigos con espadas para que te ataquen por
quebrantar el pacto. Aunque te retires a tus ciudades para defenderte,
te plagaré de enfermedades y serás entregado a tus enemigos.
\bibverse{26} nviaré una hambruna para que haya escasez de pan. Un horno
servirá para las necesidades de diez mujeres que hacen pan. Se
distribuirá por peso para que coman, pero no tendrán suficiente.

\bibverse{27} in embargo, si a pesar de todo esto no me obedecen, sino
que siguen en oposición a mí, \bibverse{28} entonces actuaré contra
ustedes con furia, y yo mismo los castigaré siete veces por sus pecados.
\bibverse{29} Se comerán los cuerpos de sus propios hijos e hijas.
\bibverse{30} Destruiré sus lugares altos\footnote{\textbf{26:30}
  ``Lugares altos'': a menudo asociado a la adoración de ídolos.}de
adoración, destrozaré sus altares de incienso, y apilaré sus cadáveres
sobre lo que queda de sus ídolos, que tampoco tienen vida alguna. Los
despreciaré de verdad. \bibverse{31} emolerésus ciudades y destruiré sus
santuarios paganos, y me negaré a aceptar sus sacrificios. \bibverse{32}
o mismo devastaré su tierra, para que sus enemigos que vengan a vivir en
ella se horroricen de lo que ha sucedido. \bibverse{33} os dispersaré
entre las naciones. Serán perseguidos por ejércitos con espadas mientras
su tierra queda en ruinas y sus pueblos son destruidos. \bibverse{34} l
menos entonces la tierra podrá disfrutar de sus sábados todo el tiempo
que esté abandonada mientras ustedes estén exiliados en la tierra de sus
enemigos. La tierra finalmente podrá descansar y disfrutar de sus
sábados. \bibverse{35} Todo el tiempo que la tierra esté abandonada,
observará los sábados de descanso que no pudo guardar mientras ustedes
vivían en ella.\footnote{\textbf{26:35} Otra reprimenda, ya que
  claramente la regla de dejar la tierra sin cultivar un año de cada
  siete no se cumplía adecuadamente.}

\bibverse{36} Haré que aquellos de ustedes que sobrevivan se desanimen
tanto que mientras vivan en las tierras de sus enemigos incluso el
sonido de una hoja soplando en el viento les asustará para que huyan!
Huirán como si fuerais perseguidos por alguien con una espada, y caerán
aunque nadie los persiga. \bibverse{37} ropezarán unos con otros como si
huyeran del ataque, aunque no venga nadie. No tendrás poder para
resistir a tus enemigos. \bibverse{38} orirás en el exilio y serás
enterrado en un país extranjero. \bibverse{39} quellos que logren
sobrevivir en los países de sus enemigos se marchitarán por su culpa,
pudriéndose al compartir los pecados de sus padres.

\bibverse{40} ecesitan confesar sus pecados y los de sus padres,
actuando de manera tan infiel hacia mí, oponiéndose a mí. \bibverse{41}
or eso tomé medidas contra ellos y los exilié en los países de sus
enemigos. Sin embargo, si humildemente abandonan su actitud obstinada y
aceptan el castigo por sus pecados, \bibverse{42} entonces cumpliré el
acuerdo que hice con Jacob, Isaac y Abraham, y no olvidaré mi promesa
sobre la tierra. \bibverse{43} orque la tierra quedará vacía para ellos,
y disfrutará de sus sábados siendo abandonada. Pagarán por sus pecados,
porque rechazaron mis reglas y regulaciones.

\bibverse{44} ero a pesar de todo esto, aunque vivan en la tierra de sus
enemigos, no los rechazaré ni los odiaré tanto como para destruirlos y
romper mi acuerdo con ellos, porque yo soy el Señor su Dios.
\bibverse{45} or ellos renovaré el acuerdo que hice con sus padres, a
los que saqué de Egipto como las otras naciones observaron, para ser su
Dios. Yo soy el Señor''.

\bibverse{46} stas son las normas, reglamentos y leyes que el Señor
estableció entre él y los israelitas a través de Moisés en el Monte
Sinaí.

\hypertarget{section-26}{%
\section{27}\label{section-26}}

\bibverse{1} El Señor le dijo a Moisés, \bibverse{2} Dile a los
israelitas: Cuando haces una promesa especial de dedicar a alguien al
Señor, estos son los valores que debes usar. \bibverse{3} El valor de un
hombre de veinte a sesenta años es de cincuenta siclos de plata, (usando
el estándar del siclo del santuario). \bibverse{4} l valor de una mujer
es de treinta siclos. \bibverse{5} El valor de alguien de cinco a veinte
años es de veinte siclos para un hombre y diez siclos para una mujer.
\bibverse{6} l valor de alguien de un mes a cinco años es de cinco
siclos de plata para un hombre y tres siclos de plata para una mujer.
\bibverse{7} l valor de alguien de sesenta años o más es de quince
siclos para un hombre y diez siclos de plata para una mujer.
\bibverse{8} in embargo, si al cumplir su promesa es más pobre que el
valor fijado, debe presentar a la persona ante el sacerdote, quien
entonces fijará el valor dependiendo de lo que pueda pagar.

\bibverse{9} i al cumplir tu promesa traes un animal que esté permitido
como ofrenda al Señor, el animal dado al Señor será considerado santo.
\bibverse{10} o se le permite reemplazarlo o cambiarlo, ya sea por uno
mejor o peor. Sin embargo, si lo reemplazas, ambos animales se
convierten en sagrados.

\bibverse{11} i al cumplir tu promesa traes algún animal impuro que no
esté permitido como ofrenda al Señor, entonces debes mostrar el animal
al sacerdote. \bibverse{12} l sacerdote decidirá su valor, ya sea alto o
bajo. Cualquier valor que el sacerdote le dé es definitivo.
\bibverse{13} i luego decide comprar el animal de nuevo, debe agregar un
quinto a su valor en pago.

\bibverse{14} i usted dedica su casa como santa al Señor, entonces el
sacerdote decidirá su valor, ya sea alto o bajo. Cualquier valor que el
sacerdote le ponga será definitivo. \bibverse{15} ero si quieres volver
a comprar tu casa, tienes que añadir un quinto a su valor en pago.
Entonces le pertenecerá de nuevo.

\bibverse{16} i dedicas parte de tu tierra al Señor, entonces su valor
se determinará por la cantidad de semilla necesaria para sembrarla:
cincuenta siclos de plata por cada homer de semilla de cebada utilizada.
\bibverse{17} i dedicas tu campo durante el año jubilar, el valor será
la cantidad total calculada. \bibverse{18} ero si dedicas tu campo
después del Jubileo, el sacerdote calculará el valor dependiendo del
número de años que queden hasta el siguiente Año Jubilar, reduciendo así
el valor. \bibverse{19} ero si quieres volver a comprar tu campo, tienes
que añadir un quinto a su valor en pago. Entonces le pertenecerá de
nuevo. \bibverse{20} ero si no compras el campo de nuevo, o si ya lo has
vendido a alguien más, no puede ser comprado de nuevo. \bibverse{21}
uando llegue el Jubileo, el campo se convertirá en sagrado, de la misma
manera que un campo dedicado al Señor. Se convertirá en propiedad de los
sacerdotes.

\bibverse{22} i le dedicas al Señor un campo que has comprado y que no
era de tu propiedad original, \bibverse{23} el sacerdote calculará el
valor hasta el próximo año del Jubileo. Ese día pagará el valor exacto,
dándoselo como una ofrenda sagrada al Señor. \bibverse{24} n el Año
Jubilar, la propiedad del campo volverá a la persona a la que se lo
compraste, al propietario original del terreno. \bibverse{25} (Todos los
valores usarán el estándar del siclo del santuario de veinte gerahs al
siclo.)

\bibverse{26} adie puede dedicar el primogénito del ganado, porque el
primogénito pertenece al Señor. Ya sea que se trate de ganado vacuno,
ovino o caprino, son del Señor. \bibverse{27} ero si se trata de un
animal impuro, se puede volver a comprar según su valor, añadiendo un
quinto extra. Si no se vuelve a comprar, entonces se vende según su
valor.

\bibverse{28} Todo lo que dediques\footnote{\textbf{27:28} La palabra
  usada aquí y en el siguiente versículo es un término religioso que
  significa dar algo al Señor (apartado), ya sea destruyéndolos o
  presentándolos como una ofrenda.} deforma especialal Señor, ya sea una
persona, animal o tu tierra, no podrá ser vendido o rescatado. Todo lo
que sea dedicado especialmente será santo para el Señor.

\bibverse{29} Ninguno que sea especialmente dedicado para la destrucción
podrá ser redimido. Debe ser asesinado.

\bibverse{30} l diezmo de sus cosechas o de sus frutos le pertenece al
Señor; es santo para el Señor. \bibverse{31} i quieren volver a comprar
parte de su diezmo, deben añadir un quinto a su valor.

\bibverse{32} uando cuentes tus rebaños y manadas, cada décimo animal
que pase bajo la vara del pastor es santo para el Señor. \bibverse{33} o
necesitarás examinarlo para ver si es bueno o malo, y no debes
reemplazarlo. Sin embargo, si lo reemplazas, ambos animales serán
sagrados; no podrán ser comprados de nuevo''.

\bibverse{34} stas son las leyes que el Señor dio a Moisés para los
israelitas en el Monte Sinaí.
