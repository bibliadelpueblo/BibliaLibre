\hypertarget{section}{%
\section{1}\label{section}}

\bibverse{1} Había una vez un hombre de Ramataim de Zofim, en la región
montañosa de Efraín. Su nombre era Elcana, hijo de Jeroham, hijo de
Eliú, hijo de Tohu, hijo de Zuf, de la tribu de Efraín.

\bibverse{2} Elcana tenía dos esposas. El nombre de la primera esposa
era Ana, y el de la segunda, Penina. Penina tenía hijos, pero Ana no
tenía ninguno.

\bibverse{3} Todos los años Elcana salía de su ciudad y se iba a adorar
y sacrificar al Señor Todopoderoso en Silo, donde los dos hijos de Elí,
Ofni y Finees, eran los sacerdotes del Señor.

\bibverse{4} Cada vez que Elcana ofrecía un sacrificio, daba porciones
del mismo a Penina, su esposa, y a todos sus hijos e hijas. \bibverse{5}
Y le daba una porción\footnote{\textbf{1:5} Al dar una porción extra,
  Elcanah estaba mostrando a todos que trataba a Ana como si tuviera un
  hijo.} extra a Ana, para mostrar su amor por ella aunque el Señor no
le había dado ningún hijo. \bibverse{6} Su rival -- la otra esposa -- se
burlaba de ella para entristecerla porque el Señor no le había dado
hijos.

\bibverse{7} Esta situación duró años, y cada vez que Ana iba al templo
del Señor, Penina se burlaba de ella hasta que Ana lloraba y no podía
comer.

\bibverse{8} Su esposo le preguntaba: ``Ana, ¿por qué lloras? ¿Por qué
no comes? ¿Por qué estás tan alterada? ¿No soy mejor para ti que diez
hijos?''

\bibverse{9} En cierta ocasión, después haber comido y bebido en Silo,
Ana se levantó y se dirigió al Templo\footnote{\textbf{1:9} ``Y se
  dirigió al templo'': Añadido para mayor claridad.}. El sacerdote Elí
estaba sentado en su silla junto a la entrada del Templo del Señor.
\bibverse{10} Ana estaba terriblemente disgustada, y le oraba al Señor
mientras lloraba inconsolablemente. \bibverse{11} Allí hizo un voto,
pidiendo: ``Señor Todopoderoso, si tan sólo te fijas en el sufrimiento
de tu sierva y te acuerdas de mí, y no me olvidas, sino que me das un
hijo, lo dedicaré al Señor durante toda su vida, y ninguna navaja de
afeitar tocará su cabeza.''

\bibverse{12} Mientras Ana seguía orando ante el Señor, Elí observaba su
boca. \bibverse{13} Ana oraba mentalmente, y aunque sus labios se
movían, su voz no producía ningún sonido, y por eso Elí pensó que debía
estar ebria.

\bibverse{14} ``¿Tienes que venir aquí estando ebria?'', le preguntó.
``¡Ya deja el vino!''

\bibverse{15} ``No es eso, mi señor'', le respondió Ana. ``Soy una mujer
muy desdichada. No he estado bebiendo vino ni cerveza; sólo estoy
derramando mi corazón ante el Señor. \bibverse{16} ¡Por favor, no
pienses que soy una mala mujer! He estado orando a causa de todos mis
problemas y penas''.

\bibverse{17} ``Ve en paz, y que el Dios de Israel te conceda lo que le
has pedido'', respondió Elí.

\bibverse{18} ``Gracias por tu bondad con tu sierva'', dijo ella. Luego
siguió su camino, comió algo y ya no se veía triste.

\bibverse{19} A la mañana siguiente, Elcana y Ana se levantaron temprano
para adorar al Señor y luego se fueron a su casa en Ramá. Elcana hizo el
amor con su esposa Ana, y el Señor accedió a su petición. \bibverse{20}
A su debido tiempo, Ana quedó embarazada y dio a luz un hijo. Le puso el
nombre de Samuel, diciendo: ``Porque se lo pedí al Señor''.

\bibverse{21} Elcana y toda su familia fueron a hacer el sacrificio
anual al Señor y a cumplir sus votos. \bibverse{22} Pero Ana no fue. Le
dijo a su marido: ``Una vez destetado el niño, lo llevaré para
presentarlo al Señor y que se quede allí para siempre.''

\bibverse{23} ``Haz lo que creas conveniente'', le respondió su marido
Elcana. ``Quédate aquí hasta que lo hayas destetado, y que el Señor
cumpla lo que ha dicho.''\footnote{\textbf{1:23} ``Lo que ha dicho'':
  refiriéndose al Señor. En la Septuaginta y en un rollo de Qumrán se
  lee ``lo que has dicho'', refiriéndose a Ana.} Así que Ana se quedó y
amamantó a su hijo hasta que lo destetó.

\bibverse{24} Cuando hubo destetado al niño, Ana se lo llevó junto con
un novillo de tres años, un efa de harina y un odre con vino. Aunque el
niño era pequeño, lo llevó al Templo del Señor en Silo. \bibverse{25}
Después de sacrificar el novillo, presentaron el niño a Elí.

\bibverse{26} ``Por favor, mi señor'', dijo Ana, ``con toda seguridad,
mi señor, yo soy la mujer que estuvo aquí con usted orando al Señor.
\bibverse{27} Yo oré por este niño, y como el Señor me ha dado lo que le
pedí, \bibverse{28} ahora se lo entrego al Señor. Mientras viva estará
dedicado al Señor''. Entoncesadoró\footnote{\textbf{1:28} ``Adoró'': Se
  presume que se refiere a Elcana. Algunas versiones cambian el
  pronombre y dice ``Adoraron.''} al Señor en ese lugar.

\hypertarget{section-1}{%
\section{2}\label{section-1}}

\bibverse{1} Ana oró: ``¡Estoy tan feliz en el Señor! ¡Él me ha dado
poder! Ahora tengo mucho que decir en respuesta a los que me odian.
¡Celebro su salvación! \bibverse{2} ¡No hay nadie santo como el Señor,
nadie aparte de ti, ninguna Roca como nuestro Dios!

\bibverse{3} ¡No hables con tanta arrogancia! ¡No hablen con tanta
arrogancia! Porque el Señor es un Dios que lo sabe todo: ¿acaso no juzga
lo que hacen?

\bibverse{4} Las armas\footnote{\textbf{2:4} ``Armas'': literalmente,
  ``arco.''} de los poderosos son destrozadas, mientras que los que
tropiezan se vuelven fuertes. \bibverse{5} Los que tenían mucha comida
ahora tienen que trabajar para ganarse un mendrugo, mientras que los que
tenían hambre ahora han engordado. La mujer que no tenía hijos ahora
tiene siete, mientras que la que tenía muchos se desvanece.

\bibverse{6} El Señor mata y otorga vida; a unos los manda a la tumba,
pero a otros los resucita. \bibverse{7} El Señor empobrece a unos, pero
enriquece a otros; abate a unos, pero levanta a otros. \bibverse{8}
Ayuda a los pobres a levantarse del polvo; saca a los humildes del
muladar y los sienta con la clase alta en lugares de gran honor. Porque
los cimientos de la tierra son del Señor, y sobre ellos ha colocado el
mundo.

\bibverse{9} Él cuidará de los que confían en él, pero los malvados se
desvanecen en las tinieblas, pues el hombre no puede triunfar por sus
propias fuerzas. \bibverse{10} El Señor aplasta a sus enemigos, truena
desde el cielo contra ellos. Él gobierna\footnote{\textbf{2:10}
  ``Gobierna'': o ``juzga.''} toda la tierra; fortalece a su rey y
otorga poder al que ha ungido.''

\bibverse{11} Entonces Elcana se fue a su casa en Ramá, mientras el niño
se quedó con el sacerdote Elí sirviendo al Señor.

\bibverse{12} Los hijos de Elí eran hombres inútiles que no tenían
tiempo para el Señor \bibverse{13} ni para su función como sacerdotes
del pueblo. Enviaban a uno de sus siervos con un tenedor cuando alguien
venía a ofrecer un sacrificio. \bibverse{14} El siervo metía el tenedor
en la olla mientras se hervía la carne del sacrificio, y les llevaba a
los hijos de Elí la carne que salía en el tenedor. Así trataban a todos
los israelitas que llegaban a Silo. \bibverse{15} De hecho, incluso
antes de que se quemara la grasa del sacrificio, el sirviente venía y
exigía al hombre que sacrificaba: ``Deme la carne para asarla para el
sacerdote. Él no quiere la carne hervida sino cruda''.

\bibverse{16} El hombre podía responder: ``Déjame, primero quemar toda
la grasa, y luego puedes tener toda la que quieras''.

Pero el criado del sacerdote le contestaba: ``No, debes dármela ahora.
Si no lo haces, la tomaré por la fuerza''. \bibverse{17} Los pecados de
estos jóvenes eran extremadamente graves ante los ojos del Señor, porque
estaban tratando las ofrendas del Señor con desprecio.

\bibverse{18} Pero Samuel servía ante el Señor: era un muchacho vestido
de sacerdote, con un efod de lino. \bibverse{19} Cada año, su madre le
hacía un pequeño manto y se lo llevaba cuando iba con su marido a
ofrecer el sacrificio anual. \bibverse{20} Elí bendecía a Elcana y a su
esposa, diciendo: ``Que el Señor le dé hijos de esta mujer para
reemplazar al que ella dedicó al Señor''. Luego regresaban a casa.
\bibverse{21} Y el Señor bendijo\footnote{\textbf{2:21} ``Bendijo'':
  literalmente, ``Le presto atención.''} a Ana con tres hijos y dos
hijas. El niño Samuel creció en la presencia del Señor.

\bibverse{22} Elí era muy anciano, pero se había enterado de todas las
cosas que sus hijos hacían con el pueblo de Israel, y de cómo seducían a
las mujeres que servían a la entrada del Tabernáculo de Reunión.
\bibverse{23} Entonces les preguntó: ``¿Por qué se comportan de esta
manera? Sigo oyendo las quejas de todo el mundo por sus malas acciones.
\bibverse{24} No, hijos míos, lo que escucho sobre ustedes de parte del
pueblo del Señor no es bueno. \bibverse{25} Si un hombre peca contra
alguien, Dios puede interceder por él; pero si un hombre peca contra el
Señor, ¿quién podrá interceder por él?'' Pero no prestaron atención a lo
que les dijo su padre, pues el Señor planeaba darles muerte.

\bibverse{26} El niño Samuel crecía en estatura, y también crecía en
cuanto a la aprobación del Señor y del pueblo.

\bibverse{27} Un hombre de Dios se acercó a Elí y le dijo: ``Esto es lo
que dice el Señor: ¿Acaso no me revelé claramente a la familia de tu
antepasado cuando era gobernado por el faraón en Egipto? \bibverse{28}
Yo lo elegí\footnote{\textbf{2:28} Refiiriéndose a Aarón.} de todas las
tribus de Israel como mi sacerdote, para ofrecer sacrificios en mi
altar, para quemar incienso y llevar un efod en mi presencia. También le
di a la familia de tu antepasado todos los holocaustos de los
israelitas. \bibverse{29} ¿Por qué, entonces, has tratado con desprecio
mis sacrificios y las ofrendas que he ordenado para mi lugar de culto?
Ustedes honran más a sus hijos que a mí, se engordan ustedes con las
mejores partes de todas las ofrendas de mi pueblo Israel.

\bibverse{30} En consecuencia, esta es la declaración del Señor: Hice la
promesa definitiva de que tu familia y la de tu padre me servirían
siempre como sacerdotes. Pero ahora el Señor declara: ¡Ya no más! En
cambio, honraré a los que me honran, pero trataré con desprecio a los
que me desprecian. \bibverse{31} Se acerca el momento en que pondré fin
a tu familia y a la de tu padre.\footnote{\textbf{2:31} ``Pondré fin a
  tu familia'': literalmente, ``cortaré tu fuerza.''} Nadie vivirá hasta
la vejez. \bibverse{32} Verás tragedia en el lugar de
adoración.\footnote{\textbf{2:32} Quizás refiriéndose a la pérdida del
  Arca a manos de los filisteos.} Mientras Israel prospere, ninguno en
tu familia volverá a vivir hasta la vejez. \bibverse{33} Cualquiera de
tu familia que no haya sido apartado para servir en mi altar, te hará
llorar y te causará dolor. Todos tus descendientes morirán aún estando
llenos de vida. \bibverse{34} He aquí una señal para ti de que esto
sucederá con respecto a tus dos hijos Ofni y Finees: ambos morirán el
mismo día. \bibverse{35} Yo elegiré para mí a un sacerdote digno de
confianza que hará lo que realmente quiero, lo que tengo en mente. Me
aseguraré de que él y sus descendientes sean dignos de confianza y que
siempre sirvan a mi ungido. \bibverse{36} Cada uno de tus descendientes
que quede vendrá y se inclinará ante él, pidiendo dinero y comida,
diciendo: `Por favor, dame trabajo como sacerdote para que pueda tener
comida.'''

\hypertarget{section-2}{%
\section{3}\label{section-2}}

\bibverse{1} El niño Samuel servía ante el Señor bajo la supervisión de
Elí. En aquella época no se escuchaba un mensaje del Señor con
frecuencia, y las visiones no eran comunes. \bibverse{2} Una noche, Elí
se había ido a acostar en su habitación. Sus ojos estaban tan débiles
que no podía ver. \bibverse{3} La lámpara de Dios aún no se había
apagado, y Samuel estaba durmiendo en el Templo del Señor, donde estaba
el Arca de Dios.

\bibverse{4} Entonces el Señor lo llamó: ``¡Samuel!''

Samuel entonces respondió: ``Aquí estoy''. \bibverse{5} Entonces corrió
hacia Elí y le dijo: ``Aquí estoy ¿Me llamabas?''.

``Yo no te he llamado'', respondió Elí. ``Vuelve a la cama''. Así que
Samuel volvió a la cama.

\bibverse{6} Entonces el Señor volvió a llamar: ``¡Samuel!''. Y Samuel
se levantó, fue a ver a Elí y le dijo: ``Aquí estoy ¿Me llamabas?''.

``Yo no te he llamado, hijo mío'', respondió Elí. ``Vuelve a la cama''.

\bibverse{7} (Samuel aún no había llegado a conocer al Señor y no había
recibido ningún mensaje de él).

\bibverse{8} El Señor volvió a llamar por tercera vez: ``¡Samuel!''.
Éste se levantó, fue a ver a Elí y le dijo: ``Aquí estoy ¿Me
llamabas?''. Entonces Elí se dio cuenta de que era el Señor quien
llamaba al muchacho.

\bibverse{9} Así que Elí le dijo a Samuel: ``Vuelve a la cama, y si
escuchas el llamado, dile: `Habla, Señor, porque tu siervo te
escucha'\,''. Así que Samuel volvió a su cama.

\bibverse{10} El Señor llegó y se quedó allí, llamando igual que antes:
``¡Samuel! Samuel!'' Entonces Samuel respondió: ``Habla, porque tu
siervo te escucha''.

\bibverse{11} El Señor le dijo entonces a Samuel: ``Presta atención,
porque voy a hacer algo en Israel que sorprenderá a todos los que lo
escuchen.\footnote{\textbf{3:11} ``Sorprenderá a todos los que lo
  escuchen'': literalmente, ``estremecerá los oídos de todos los que lo
  escuchen.''} \bibverse{12} Es entonces cuando cumpliré todo lo que he
dicho, de principio a fin, contra Elí y su familia. \bibverse{13} Le
dije que juzgaré a su familia para siempre por los pecados que él
conoce, porque sus hijos blasfemaron contra Dios y él no trató de
detenerlos. \bibverse{14} Por eso le juré a Elí y a su familia: `La
culpa de Elí y de sus descendientes no se quitará nunca con sacrificios
ni con ofrendas'\,''.

\bibverse{15} Samuel permaneció en la cama hasta la mañana. Luego se
levantó y abrió las puertas del Templo del Señor como de costumbre.
Tenía miedo de contarle a Elí la visión. \bibverse{16} Pero Elí lo llamó
y le dijo: ``Samuel, hijo mío''.

``Aquí estoy'', respondió Samuel.

\bibverse{17} ``¿Qué te ha dicho?'' preguntó Elí. ``No me lo ocultes.
Que Dios te castigue muy severamente si me ocultas algo de lo que te
dijo''.

\bibverse{18} Así que Samuel le contó todo y no le ocultó nada.

``Es el Señor'', respondió Elí. ``Que haga lo que le parezca bien''.

\bibverse{19} Samuel siguió creciendo. El Señor estaba con él y se
aseguraba de que todo lo que decía era fiel. \bibverse{20} Todos en todo
Israel, desde Dan hasta Beerseba, reconocían que Samuel era un profeta
del Señor digno de confianza. \bibverse{21} El Señor siguió apareciendo
en Silo, porque allí se revelaba a Samuel y le entregaba sus mensajes,

\hypertarget{section-3}{%
\section{4}\label{section-3}}

\bibverse{1} y las palabras de Samuel se comunicaron a todos los
israelitas.

Los israelitas marcharon para enfrentarse a los filisteos en la batalla.
Acamparon en Ebenezer, mientras los filisteos lo hacían en Afec.
\bibverse{2} Los filisteos atacaron a los israelitas en formación, y
cuando la batalla se extendió, los filisteos derrotaron a los
israelitas, matando a 4.000 de ellos en el campo de batalla.
\bibverse{3} Cuando el ejército israelita regresó al campamento, los
ancianos de Israel preguntaron: ``¿Por qué el Señor nos ha derrotado hoy
ante los filisteos? Vayamos a buscar el Arca del Pacto del Señor a Silo,
para que nos acompañe y nos salve de nuestros enemigos.''

4 Así que el ejército envió hombres a Silo, y trajeron de vuelta el Arca
del Pacto del Señor Todopoderoso, el que está sentado en su trono entre
los querubines. Ofni y Finees, los dos hijos de Elí, estaban allí con el
Arca del Pacto del Señor. \bibverse{5} Cuando el Arca del Pacto del
Señor llegó al campamento, todos los israelitas dieron un grito tan
fuerte que hizo temblar el suelo.

\bibverse{6} Cuando los filisteos oyeron todo el griterío, preguntaron:
``¿Qué significa este griterío en el campamento israelita?'' Cuando se
enteraron de que el Arca del Señor había llegado al campamento,
\bibverse{7} los filisteos se asustaron. ``Un dios ha llegado al
campamento'', dijeron. ``Estamos en problemas, pues nunca antes había
sucedido algo así. \bibverse{8} ¡Esto es un desastre para nosotros!
¿Quién nos salvará del poder de estos poderosos dioses? Estos son los
dioses que atacaron a los egipcios con toda clase de plagas en el
desierto. \bibverse{9} ¡Sean valientes y luchen como verdaderos hombres,
filisteos! De lo contrario, terminarán como esclavos de los israelitas,
tal como ellos fueron sus esclavos. Sean hombres de verdad y luchen''.

\bibverse{10} Así que los filisteos lucharon, y los israelitas fueron
derrotados: cada uno huyó a su casa. El número de muertos fue muy
grande: treinta mil de la infantería israelita murieron. \bibverse{11}
El Arca de Dios fue capturada y murieron Ofni y Finees, los dos hijos de
Elí.

\bibverse{12} Un hombre de la tribu de Benjamín huyó aquel día de la
batalla hasta Silo. Su ropa estaba rota y tenía tierra en la
cabeza.\footnote{\textbf{4:12} ``Su ropa estaba rota y tenía tierra en
  la cabeza''. Esto simbolizaba una gran angustia.} \bibverse{13} Cuando
llegó, Elí estaba sentado en su silla junto al camino, atento a las
noticias porque estaba preocupado por el Arca de Dios. Cuando el hombre
llegó a la ciudad y dio su informe, todo el pueblo lloró a gritos.

\bibverse{14} Elí oyó el llanto y preguntó: ``¿Qué es todo este
ruido?''. El hombre corrió hacia Elí y le contó lo que había sucedido.

\bibverse{15} Elí tenía noventa y ocho años, y sus ojos estaban fijos
porque no podía ver.

\bibverse{16} ``Acabo de llegar de la batalla'', dijo el hombre. ``Hoy
he huido de ella''.

``¿Qué pasó, hijo mío?'' preguntó Elí.

\bibverse{17} ``Israel huyó de los filisteos; fuimos derrotados'',
respondió el mensajero. ``También tus dos hijos, Ofni y Finees, fueron
asesinados, y el Arca de Dios ha sido capturada''.

\bibverse{18} En cuanto se mencionó el Arca de Dios, Elí cayó de
espaldas de su silla junto a la puerta de la ciudad. Como era viejo y
pesado, se rompió la nuca y murió. Elí había sido el líder de Israel
durante cuarenta años.

\bibverse{19} Su nuera, la esposa de Finees, estaba embarazada y a punto
de dar a luz. Cuando escuchó la noticia de que el Arca de Dios había
sido capturada, y que su suegro y su marido habían muerto, se puso de
parto y dio a luz, pero sus dolores de parto fueron demasiado fuertes.
\bibverse{20} Y justo antes de morir, las mujeres que la atendían le
dijeron: ``No te rindas, has dado a luz un hijo''. Pero ella no contestó
ni dio ninguna respuesta.

\bibverse{21} Entonces llamó al niño Icabod, diciendo: ``La gloria se ha
ido de Israel'', porque el Arca de Dios había sido capturada, y su
suegro y su marido habían muerto. \bibverse{22} Ella dijo: ``La gloria
ha dejado a Israel, porque el Arca de Dios ha sido capturada.''

\hypertarget{section-4}{%
\section{5}\label{section-4}}

\bibverse{1} Después de que los filisteos capturaron el Arca de Dios, la
llevaron de Ebenezer a Asdod. \bibverse{2} Llevaron el Arca de Dios al
Templo de Dagón y la colocaron junto a Dagón. \bibverse{3} Cuando el
pueblo de Asdod se levantó temprano al día siguiente, vio que Dagón
había caído de bruces frente al Arca del Señor. Así que tomaron a Dagón
y lo volvieron a colocar. \bibverse{4} Cuando se levantaron temprano a
la mañana siguiente, vieron que Dagón había caído de bruces frente al
Arca del Señor, con la cabeza y las manos rotas, tirado en el umbral.
Sólo su cuerpo permanecía intacto. \bibverse{5} (Por eso los sacerdotes
de Dagón, y todos los que entran en el templo de Dagón en Asdod, no
pisan el umbral, ni siquiera hasta ahora).

\bibverse{6} El Señor castigó\footnote{\textbf{5:6} ``El Señor
  castigó'': literalmente, ``La mano del Señor fue pesada.''} a los
habitantes de Asdod y sus alrededores, devastándolos y plagándolos de
hinchazones.\footnote{\textbf{5:6} Algunos piensan que estas
  ``hinchazones'' o ``tumores'' estaban relacionados con la peste
  bubónica. La Septuaginta añade al final de este versículo: ``y las
  ratas pululaban por toda la tierra, y había muerte y destrucción en la
  ciudad.''} \bibverse{7} Cuando los habitantes de Asdod vieron lo que
sucedía, dijeron: ``No podemos dejar que el Arca del Dios de Israel se
quede aquí con nosotros, porque nos está castigando a nosotros y a
Dagón, nuestro dios.'' \bibverse{8} Así que mandaron llamar a todos los
gobernantes filisteos y les preguntaron: ``¿Qué debemos hacer con el
Arca del Dios de Israel?''

``Lleven el Arca del Dios de Israel a Gat'', respondieron. Así que la
trasladaron a Gat. \bibverse{9} Pero una vez que trasladaron el Arca a
Gat, el Señor también actuó contra esa ciudad, sumiéndola en una gran
confusión y atacando a la gente de la ciudad, jóvenes y ancianos, con
una plaga de hinchazones.

\bibverse{10} Entonces enviaron el Arca de Dios a Ecrón, pero en cuanto
llegó, los dirigentes de Ecrón gritaron: ``¡Han trasladado aquí el Arca
del Dios de Israel para matarnos a nosotros y a nuestro pueblo!''
\bibverse{11} Así que mandaron llamar a todos los gobernantes filisteos
y les dijeron: ``Que el Arca del Dios de Israel se vaya, vuelva al lugar
de donde vino, porque si no nos va a matar a nosotros y a nuestro
pueblo.'' La gente moría en toda la ciudad, creando un pánico terrible,
pues el castigo de Dios era muy duro. \bibverse{12} Los que no morían
estaban plagados de hinchazones, y el grito de auxilio del pueblo
llegaba hasta el cielo.

\hypertarget{section-5}{%
\section{6}\label{section-5}}

\bibverse{1} Después de que el Arca del Señor estuvo en el país de los
filisteos durante siete meses, \bibverse{2} los filisteos convocaron a
los sacerdotes y adivinos y les preguntaron: ``¿Qué debemos hacer con el
Arca del Señor? Explíquennos cómo devolverla al lugar de donde vino''.

\bibverse{3} ``Si van a enviar de vuelta el Arca del Dios de Israel, no
la envíen con las manos vacías, sino asegúrense de enviar junto con ella
un regalo de ofrenda por la culpa para él'', respondieron. ``Entonces
serán sanados y entenderán por qué los ha tratado así''.

\bibverse{4} ``¿Qué clase de ofrenda por la culpa debemos enviarle?'',
preguntaron los filisteos. ``Cinco objetos de oro en forma de tumor y
cinco ratas de oro que representen el número de gobernantes de los
filisteos'', respondieron. ``La misma plaga los atacó a ustedes y a sus
gobernantes. \bibverse{5} Haz modelos que representen tus hinchazones y
las ratas que destruyen el país, y honra al Dios de Israel. Tal vez deje
de castigarte a ti, a tus dioses y a tu tierra. \bibverse{6} ¿Por qué
ser tercos como los egipcios y el faraón? ¿Acaso cuando Dios los castigó
no dejaron ir a los israelitas para seguir su camino?

7 Así que preparen un nuevo carro, tirado por dos vacas con crías y que
nunca hayan sido uncidas. Aten las vacas al carro, pero quiten sus
terneros y pónganlos en un establo.\footnote{\textbf{6:6} El propósito
  de esto era forzar a las vacas a hacer algo inusual dejando
  voluntariamente sus terneros. De este modo, el pueblo estaría seguro
  de que esta acción contaba con la aprobación de Dios si la hacía
  realidad.} \bibverse{8} Recojan el Arca del Señor, pónganla en el
carro y coloquen los objetos de oro que envían como ofrenda por la culpa
en un cofre junto a ella. Luego envíen el Arca. Dejen que se vaya por
donde quiera, \bibverse{9} pero no dejen de vigilarla. Si sube por el
camino hacia su patria, hacia Bet-Semes, entonces es el Señor quien nos
ha causado todo este terrible problema. Pero si no lo hace, entonces
sabremos que no fue él quien nos castigó, sino que nos ocurrió por
casualidad''.

\bibverse{10} Entonces el pueblo lo hizo así. Tomaron dos vacas con
crías y las ataron al carro, y guardaron sus terneros en un establo.
\bibverse{11} Pusieron el Arca del Señor en el carro, junto con el cofre
que contenía las ratas de oro y los modelos de sus hinchazones.
\bibverse{12} Las vacas subieron en línea recta por el camino de
Bet-Semes, mugiendo mientras avanzaban, yendo directamente por el camino
principal y sin girar ni a la izquierda ni a la derecha. Los jefes
filisteos las siguieron hasta la frontera de Bet-Semes.

\bibverse{13} Los habitantes de Bet-semes estaban cosechando trigo en el
valle. Cuando levantaron la vista y vieron el Arca, se alegraron mucho
de verla. \bibverse{14} El carro entró en el campo de Josué de Bet-semes
y se detuvo allí junto a una gran roca. El pueblo cortó la madera del
carro y sacrificaron las vacas como holocausto al Señor. \bibverse{15}
Los levitas bajaron el Arca del Señor y el cofre que contenía los
objetos de oro, y los pusieron sobre la gran roca. El pueblo de
Bet-semes presentó holocaustos e hizo sacrificios al Señor ese día.
\bibverse{16} Los cinco jefes filisteos vieron todo lo que sucedió y
regresaron a Ecrón ese mismo día.

\bibverse{17} Los cinco modelos de oro de las hinchadas enviados por los
filisteos como ofrenda de culpa al Señor eran de los gobernantes de
Asdod, Gaza, Ascalón, Gat y Ecrón. \bibverse{18} Las ratas de oro
representaban el número de ciudades filisteas de los cinco gobernantes:
las ciudades fortificadas y sus aldeas circundantes. La gran roca sobre
la que colocaron el Arca del Señor sigue en pie hasta el día de hoy en
el campo de Josué de Bet-semes como testigo de lo que allí ocurrió.

\bibverse{19} Pero Dios mató a algunos de los habitantes de Bet-semes
porque revisaron el interior del Arca del Señor. Mató a
setenta,\footnote{\textbf{6:19} Algunos manuscritos parecen decir
  50.070, pero esta es una cifra improbable para un pequeño
  asentamiento.} y el pueblo se lamentó profundamente porque el Señor
había matado a tantos. \bibverse{20} El pueblo de Bet-semes preguntó:
``¿Quién puede estar frente al Señor, este Dios santo? ¿Adónde debe ir
el Arca de aquí en adelante?''

\bibverse{21} Entonces enviaron mensajeros al pueblo de Quiriat-jearim
para decirles: ``Los filisteos han devuelto el Arca del Señor.
Desciendan y llévensela a casa.''

\hypertarget{section-6}{%
\section{7}\label{section-6}}

\bibverse{1} Entonces el pueblo de Quiriat-jearim vino y se apropió del
Arca del Señor. La pusieron en la casa de Abinadab, en la colina.
Designaron a su hijo Eleazar para cuidar el Arca del Señor. \bibverse{2}
El Arca permaneció allí, en Quiriat-jearim, desde aquel día, durante
mucho tiempo, hasta veinte años. Todos en Israel se lamentaron y,
arrepentidos, volvieron al Señor.

\bibverse{3} Entonces Samuel le dijo a todo Israel: ``Si desean
sinceramente volver al Señor, desháganse de los dioses extranjeros y de
las imágenes de Astoret, entréguense al Señor y adórenlo sólo a él, y él
los salvará de los filisteos.'' \bibverse{4} El pueblo de Israel se
deshizo de sus baales e imágenes de Astoret y sólo adoró al Señor.

\bibverse{5} Entonces Samuel dijo: ``Que todo el pueblo de Israel se
reúna en Mizpa, y yo oraré al Señor por ustedes.''

\bibverse{6} Una vez reunidos en Mizpa, sacaron agua y la derramaron
ante el Señor. Ese día ayunaron y reconocieron: ``Hemos pecado contra el
Señor''. Y Samuel se convirtió en el líder\footnote{\textbf{7:6}
  Literalmente ``juez,'' que era el equivalente a ``líder.'' Ver
  también, vers. 15.} de los israelitas en Mizpa.

\bibverse{7} Cuando los filisteos se enteraron de que los israelitas se
habían reunido en Mizpa, sus gobernantes dirigieron un ataque contra
Israel. Cuando los israelitas se enteraron de esto, se aterraron por lo
que los filisteos podrían hacer. \bibverse{8} Le dijeron a Samuel: ``No
dejes de rogarle al Señor nuestro Dios por nosotros, para que nos salve
de los filisteos''. \bibverse{9} Samuel tomó un cordero joven y lo
presentó como holocausto completo al Señor. Clamó al Señor por ayuda
para Israel, y el Señor le respondió.

\bibverse{10} Mientras Samuel presentaba el holocausto, los filisteos se
acercaron para atacar a Israel. Pero aquel día el Señor tronó muy fuerte
contra los filisteos, lo que los confundió totalmente, y fueron
derrotados ante la mirada de Israel. \bibverse{11} Entonces los hombres
de Israel salieron corriendo de Mizpa y los persiguieron, matándolos
hasta llegar a un lugar cercano a Bet-car.

\bibverse{12} Después de esto, Samuel tomó una piedra y la colocó entre
Mizpa y Sen.~La llamó Ebenezer, diciendo: ``¡El Señor nos ayudó hasta
aquí!''.

\bibverse{13} Fue así como los filisteos se mantuvieron bajo control y
no volvieron a invadir Israel. A lo largo de la vida de Samuel, el Señor
usó su poder contra los filisteos. \bibverse{14} Los filisteos le
devolvieron a Israel las ciudades que les habían arrebatado, desde Ecrón
hasta Gat, e Israel también liberó el territorio vecino de manos de los
filisteos. También hubo paz entre Israel y los amorreos.

\bibverse{15} Y Samuel fue el líder de Israel por el resto de su vida.
\bibverse{16} Todos los años recorría el país, yendo a Betel, Gilgal y
Mizpa. En todos estos lugares atendía los asuntos de Israel.
\bibverse{17} Luego regresaba a Ramá, porque allí vivía. Desde allí
gobernaba a Israel, y también construyó un altar para el Señor.

\hypertarget{section-7}{%
\section{8}\label{section-7}}

\bibverse{1} Cuando Samuel envejeció, nombró a sus hijos como
jefes\footnote{\textbf{8:1} Nuevamente la palabra utilizada es
  ``jueces'', pero en este período de la historia de Israel, antes de
  que tuvieran reyes, los jueces no sólo resolvían casos legales, sino
  que actuaban como gobernantes.} de Israel. \bibverse{2} Su primer hijo
se llamaba Joel, y su segundo hijo se llamaba Abías. Ambos fueron
gobernantes en Beerseba. \bibverse{3} Sin embargo, sus hijos no
siguieron su camino. Eran corruptos, ganaban dinero aceptando sobornos y
pervertían la justicia.

\bibverse{4} Así que los ancianos de Israel se reunieron y fueron a
buscar a Samuel a Ramá. \bibverse{5} ``Mira'' -- le dijeron -- ``tú ya
eres viejo y tus hijos no siguen tus caminos. Elige un rey que nos
gobierne como a todas las demás naciones''.

\bibverse{6} A Samuel le pareció que era una mala idea cuando le
dijeron: ``Danos un rey que nos gobierne'', así que oró al Señor al
respecto. \bibverse{7} ``Haz lo que el pueblo te diga'', le dijo el
Señor a Samuel, ``porque no es a ti a quien rechazan, sino a mí como su
rey. \bibverse{8} Están haciendo lo mismo que siempre han hecho desde
que los saqué de Egipto hasta ahora. Me han abandonado y han adorado a
otros dioses, y lo mismo están haciendo contigo. \bibverse{9} Así que
haz lo que quieran, pero dales una advertencia solemne: explícales lo
que hará un rey cuando los gobierne''.

\bibverse{10} Samuel repitió delante de todo el pueblo todo lo que el
Señor le había dicho en cuanto al pueblo pidiéndole que les diera un
rey. \bibverse{11} Entonces les dijo: ``Esto es lo que hará un rey
cuando gobierne sobre Israel: Tomará a sus hijos y los hará servir como
soldados y jinetes, y para que corran como guardia delante de sus
propios carruajes. \bibverse{12} A algunos de los asignará como
comandantes de millares y comandantes de cincuentenas, y otros tendrán
que arar sus campos y segar su cosecha. A algunos los destinará a
fabricar armas y equipos para los carros de guerra. \bibverse{13} Tomará
a las hijas de ustedes y las hará trabajar como perfumistas, cocineras y
panaderas. \bibverse{14} Tomará de entre ustedes los mejores campos,
viñedos y olivares y se los dará a sus funcionarios. \bibverse{15}
Tomará la décima parte de las cosechas de grano de ustedes, así como el
producto de sus viñedos y la asignará a sus jefes y funcionarios.
\bibverse{16} Tomará a los siervos y siervas de ustedes, así como a sus
mejores jóvenes y asnos, y los pondrá a trabajar para él. \bibverse{17}
Tomará la décima parte de los rebaños de ustedes, y ustedes mismos serán
ahora sus esclavos. \bibverse{18} Ese día ustedes suplicarán ser
rescatados del rey que han elegido, pero el Señor no les responderá.''

\bibverse{19} Pero el pueblo se negó a escuchar lo que Samuel decía.
``¡No!'', insistieron. ``¡Queremos nuestro propio rey! \bibverse{20} Así
podremos ser como las demás naciones. Nuestro rey nos gobernará y nos
guiará cuando salgamos a pelear nuestras batallas''.

\bibverse{21} Samuel escuchó todo lo que el pueblo decía y se lo repitió
al Señor. \bibverse{22} Entonces el Señor le dijo a Samuel: ``Haz lo que
ellos te piden y dales un rey''. Entonces Samuel les dijo a los
israelitas: ``Vuelvan a sus casas.''

\hypertarget{section-8}{%
\section{9}\label{section-8}}

\bibverse{1} Había un hombre rico e influyente de la tribu de Benjamín,
que se llamaba Cis, hijo de Abiel, hijo de Zeror, hijo de Becorat, hijo
de Afía, descendiente de la tribu de Benjamín. \bibverse{2} Cis tenía un
hijo llamado Saúl. Este era el joven más guapo de todo Israel. Era más
alto que cualquier otro.

\bibverse{3} En cierta ocasión, los burros del padre de Saúl, Cis, se
extraviaron. Cis le dijo a su hijo Saúl: ``Por favor, ve a buscar los
burros. Puedes llevar a uno de los siervos contigo''. 4 Saúl buscó en la
región montañosa de Efraín y luego en la tierra de Salisa, pero no
encontró los burros. Entonces buscaron en la región de Saalim, pero
tampoco estaban allí. Luego buscaron en la tierra de Benjamín, pero
tampoco pudieron encontrarlos allí.

\bibverse{5} Cuando llegaron a la tierra de Zuf, Saúl le dijo a su
criado: ``Vamos, volvamos, porque si no mi padre no se preocupará
solamente por los burros, sino también por nosotros.''

\bibverse{6} Pero el criado le respondió: ``¡Espera! Hay un hombre de
Dios en esta ciudad. Tiene muy buena fama, y todo lo que dice se cumple.
Vamos a verle. Tal vez él pueda decirnos qué camino debemos tomar''.

\bibverse{7} ``Pero si vamos, ¿qué podemos darle?'' respondió Saúl.
``Todo el pan de nuestras bolsas se ha acabado. No tenemos nada que
llevarle al hombre de Dios. ¿Qué tenemos con nosotros?''

\bibverse{8} ``Mira, tengo un cuarto de siclo de plata conmigo. Se lo
daré al hombre de Dios para que nos indique el camino que debemos
tomar'', le dijo el criado a Saúl.

\bibverse{9} (Antiguamente, en Israel, alguien que iba a consultar a
Dios decía: ``Ven, vamos a ver al vidente'', porque a los profetas se
les solía llamar videntes).

\bibverse{10} ``Me parece bien'', le dijo Saúl a su criado. ``Vamos
entonces''. Y se fueron al pueblo donde estaba el hombre de Dios.

\bibverse{11} Mientras subían la colina hacia el pueblo, se encontraron
con unas jóvenes que salían a sacar agua y les preguntaron: ``¿Está el
vidente aquí?''

\bibverse{12} Ellas les respondieron: ``Está más adelante. Pero tendrán
que apresurarse. Hoy ha venido a la ciudad porque el pueblo está
celebrando un sacrificio en el lugar de adoración. \bibverse{13} Cuando
entren a la ciudad podrán encontrarlo antes de que suba a comer en lugar
de adoración. El pueblo no comerá antes de que él haya llegado, porque
él tiene que bendecir el sacrificio. Después comerán los que han sido
invitados. Si se van ahora, lo alcanzarán''.

\bibverse{14} Así que siguieron su camino hasta la ciudad. Cuando
llegaron allí estaba Samuel yendo en dirección contraria. Se encontraron
con él cuando subía al lugar de adoración.

\bibverse{15} El día anterior a la llegada de Saúl, el Señor le había
dicho a Samuel: \bibverse{16} ``Mañana a esta hora te voy a enviar un
hombre de la tierra de Benjamín. Nómbralo como gobernante de mi pueblo
Israel, y él los rescatará de los filisteos. He visto lo que le pasa a
mi pueblo y he escuchado su ruego de ayuda''.

\bibverse{17} Cuando Samuel vio a Saúl, el Señor le dijo: ``Este es el
hombre del que te hablé. Es el que va a gobernar a mi pueblo''.

\bibverse{18} Saúl se acercó a Samuel en la puerta y le preguntó:
``¿Podrías decirme dónde está la casa del vidente?''

\bibverse{19} ``Yo soy el vidente'', le dijo Samuel a Saúl. ``Sube
delante de mí y comeremos juntos. Luego, por la mañana, responderé a
todas tus preguntas y te enviaré por el camino. \bibverse{20} En cuanto
a los burros que perdiste hace tres días, no te preocupes por ellos
porque los han encontrado. Pero ahora, la esperanza de todo Israel
descansa en ti y en tu linaje''

\bibverse{21} ``¡Pero yo soy de la tribu de Benjamín, la más pequeña de
Israel, y mi familia es la menos importante de todas las familias de la
tribu de Benjamín!'' respondió Saúl. ``¿Por qué me dices esto?''

\bibverse{22} Entonces Samuel llevó a Saúl y a su criado al salón, y los
sentó a la cabeza de las treinta personas que habían sido invitadas.
\bibverse{23} Entonces Samuel le dijo al cocinero: ``Trae el trozo de
carne especial que te di y que te dije que reservaras''.

\bibverse{24} Así que el cocinero tomó el muslo superior\footnote{\textbf{9:24}
  A Saúl se le dio la carne que sólo debían comer los sacerdotes. Ver
  Levítico 10:14-15.} de la carne y lo puso delante de Saúl. Entonces
Samuel le dijo: ``Mira, esto es lo que estaba reservado para ti. Cómelo,
pues estaba apartado para ti, para este momento en particular, desde que
dije: `He invitado al pueblo'\,''. Así que Saúl comió con Samuel aquel
día.

\bibverse{25} Cuando descendieron del lugar de adoración en lo alto a la
ciudad, Samuel habló con Saúl en el techo de su casa.\footnote{\textbf{9:25}
  A falta de otras habitaciones, la azotea de la casa se utilizaba como
  alojamiento temporal.} \bibverse{26} Al amanecer del día siguiente,
Samuel llamó a Saúl desde el tejado: ``¡Levántate! Tengo que enviarte de
regreso''. Así que Saúl se levantó y salió con Samuel. \bibverse{27}
Cuando se acercaban a las afueras de la ciudad, Samuel le dijo a Saúl:
``Dile a tu siervo que se vaya adelante, antes que nosotros. Cuando se
haya ido, quédate aquí un rato, porque tengo un mensaje de Dios para
ti''. Así que el criado se adelantó.

\hypertarget{section-9}{%
\section{10}\label{section-9}}

\bibverse{1} Entonces Samuel tomó un frasco de aceite de oliva y lo
derramó sobre la cabeza de Saúl, y lo besó diciendo: ``El Señor te ha
ungido como gobernante de su pueblo elegido.\footnote{\textbf{10:1} Esta
  línea se da en forma de pregunta, pero es mejor traducirla como una
  declaración, ya que una pregunta puede implicar incertidumbre.}
\bibverse{2} Cuando me dejes hoy, te encontrarás con dos hombres cerca
de la tumba de Raquel en Selsa, en la frontera del territorio de
Benjamín. Te dirán que han encontrado los burros que fuiste a buscar.

Ahora tu padre no está preocupado por ellos, sino por ti, y se pregunta:
``¿Qué pasará con mi hijo?''.

\bibverse{3} Saldrás de allí y seguirás hasta la encina de Tabor, donde
te encontrarás con tres hombres que van a adorar a Dios en Betel. Uno
llevará tres cabritos, otro llevará tres panes y otro llevará un odre de
vino. \bibverse{4} Te saludarán\footnote{\textbf{10:4} Literalmente,
  ``shalom,'' el saludo usual de la época.} y te darán dos panes que
deberás tomar.

\bibverse{5} A continuación llegarás a Guibeá de Dios, donde los
filisteos tienen una guarnición. Al entrar en la ciudad, te encontrarás
con una procesión de profetas que desciende del lugar alto, tocando
arpas, panderetas, flautas y liras, y estarán profetizando. \bibverse{6}
Entonces el Espíritu del Señor vendrá sobre ti con poder. Profetizarás
con ellos, y te convertirás en un hombre diferente. \bibverse{7} Después
de que hayan ocurrido estas señales, haz lo que tengas que hacer, porque
Dios está contigo. \bibverse{8} Luego ve delante de mí a Gilgal. Te
aseguro que iré y me reuniré contigo para presentar holocaustos y
ofrendas de paz. Espera allí siete días hasta que yo vaya a verte y te
haga saber lo que debes hacer''.

\bibverse{9} En el momento mismo que Saúl se volvió y dejó a Samuel,
Dios le dio a Saúl una forma de pensar diferente,\footnote{\textbf{10:9}
  ``Una manera de pensardiferente'': literalmente ``hizo que su corazón
  fuera otro''. Dado que en hebreo el corazón era donde se pensaba, esto
  se relaciona con la mente. En muchos sentidos, esto se corresponde con
  el concepto griego de un ``cambio de mente'', que es el verdadero
  significado de la conversión. Así que en cierto sentido se podría
  decir que Saúl se ``convirtió'' en ese momento.} y todas las señales
se cumplieron aquel día. \bibverse{10} Cuando Saúl y su criado llegaron
a Guibeá, había una procesión de profetas que salía a su encuentro. Y el
Espíritu de Dios vino sobre Saúl con poder, y él también comenzó a
profetizar con ellos.

\bibverse{11} Todos los que conocían a Saúl y lo veían profetizar con
los profetas se decían: ``¿Qué pasa con el hijo de Cis? ¿Acaso Saúl es
también uno de los profetas?''

\bibverse{12} Un hombre que vivía allí respondió: ``¿Pero quién es su
padre?''\footnote{\textbf{10:12} En otras palabras, el don profético no
  depende de la genealogía.} Así que se convirtió en un dicho: ``¿Es
Saúl también uno de los profetas?''

\bibverse{13} Cuando Saúl terminó de profetizar, fue al lugar alto de
adoración. \bibverse{14} El tío de Saúl le preguntó a éste y a su
criado: ``¿Dónde estaban?''

``Estábamos buscando los burros'', respondió Saúl. ``Como no los
encontramos, fuimos a ver a Samuel''.

\bibverse{15} ``Por favor, díganme qué les dijo'', preguntó el tío de
Saúl.

\bibverse{16} ``Nos aseguró que los burros habían sido encontrados'',
respondió Saúl. Pero Saúl no le dijo a su tío lo que Samuel le había
dicho que sería rey.

\bibverse{17} Entonces Samuel convocó al pueblo de Israel a presentarse
ante el Señor en Mizpa. \bibverse{18} Y les dijo a los israelitas:
``Esto es lo que dice el Señor, el Dios de Israel: ``Yo saqué a Israel
de Egipto y los salvé de los egipcios y de todos los reinos que los
oprimían. \bibverse{19} Pero ahora ustedes han rechazado a su Dios, el
que los salva de todos sus problemas y aflicciones. Y le han dicho:
`Tienes que nombrar un rey que nos gobierne'. Así que ahora preséntense
ante el Señor por tribus y grupos familiares''.

\bibverse{20} Entonces Samuel hizo que todo Israel se presentara por
tribus, y la tribu de Benjamín fue elegida por sorteo. \bibverse{21}
Luego hizo que la tribu de Benjamín se presentara por sus grupos
familiares, y fue elegido el grupo familiar de Matri. Por último, se
eligió a Saúl, hijo de Cis. Pero cuando lo buscaron, no lo encontraron.
\bibverse{22} Y le preguntaron al Señor: ``¿Ya está aquí?''.

Y el Señor respondió: ``Vayan a buscarlo; está escondido entre el
equipaje''.

\bibverse{23} Así que corrieron y trajeron a Saúl. Cuando se puso de pie
entre la gente, era más alto que los demás.

\bibverse{24} Samuel les dijo a todos: ``¿Ven al que el Señor ha
elegido? No hay nadie como él en ninguna parte''.

Y todo el pueblo gritó: ``¡Viva el rey!''.

\bibverse{25} Entonces Samuel le explicó al pueblo todo lo que haría un
rey. Lo escribió en un pergamino y lo puso ante el Señor. Luego Samuel
los envió a todos a casa.

\bibverse{26} Saúl también regresó a su casa en Guibeá, acompañado de
los guerreros a quienes Dios había convencido para que lo ayudaran.

\bibverse{27} Pero algunos hombres odiosos preguntaron: ``¿Cómo podría
salvarnos este hombre?''. Lo odiaron y no le trajeron ningún regalo;
pero Saúl no tomó represalias.\footnote{\textbf{10:27} En el texto
  hebreo tradicional el capítulo termina aquí. Sin embargo, en un rollo
  encontrado en Qumrán hay la siguiente información adicional que se
  relaciona con el siguiente capítulo y se incluye aquí por su interés.
  ``Nahas, rey de los amonitas, había estado oprimiendo severamente al
  pueblo de Gad y Rubén. Les sacaba el ojo derecho y no dejaba que nadie
  los ayudara. No quedó nadie de los israelitas al otro lado del Jordán
  a quien Nahas, rey de los amonitas, no le hubiera sacado el ojo
  derecho. Sin embargo, había siete mil hombres que habían escapado de
  los amonitas y se habían ido a vivir a Jabes de Galaad.''}

\hypertarget{section-10}{%
\section{11}\label{section-10}}

\bibverse{1} Nahas el amonita llegó con su ejército\footnote{\textbf{11:1}
  ``Con su ejército'': añadido para mayor claridad.} y sitió Jabes de
Galaad. Todo el pueblo de Jabes le dijo: ``Haz un acuerdo de paz con
nosotros, y seremos tus súbditos.''

\bibverse{2} Pero Nahas el amonita respondió: ``Haré un tratado de paz
con ustedes con una condición: que les saque a todos el ojo derecho para
avergonzar a todos los israelitas.''

\bibverse{3} ``Déjanos siete días para que podamos enviar mensajeros por
todo Israel'', respondieron los ancianos del pueblo de Jabes. ``Si nadie
viene a ayudarnos, nos rendiremos ante ustedes''.

\bibverse{4} Cuando los mensajeros llegaron a Guibeá de Saúl y dieron el
mensaje mientras el pueblo escuchaba, todos lloraron a gritos.

\bibverse{5} Justo en ese momento Saúl volvía de arar un campo con sus
bueyes. ``¿Por qué están todos tan alterados?'', preguntó. Entonces le
contaron lo que habían dicho los hombres de Jabes.

\bibverse{6} Cuando se enteró de esto, el Espíritu de Dios se apoderó de
Saúl, y se enojó mucho. \bibverse{7} Entonces tomó un par de bueyes y
los cortó en pedazos. Luego los envió con los mensajeros por todo Israel
con el mensaje: ``Esto es lo que pasará con los bueyes de cualquiera que
no siga a Saúl y a Samuel''. Y el Señor hizo que el pueblo se pusiera
ansioso\footnote{\textbf{11:7} ``El Señor hizo que el pueblo se pudiera
  ansioso'': literalmente ``El temor del Señor cayó sobre el pueblo''.
  Esto podría interpretarse como que el Señor es la fuente del temor, o
  el objeto de temor. En cualquier caso, el resultado es que el pueblo
  apoya a Saúl.} por hacerlo, y el pueblo salió como si fueran uno solo.
\bibverse{8} Cuando Saúl los contó en Bezek, había 300.000 hombres de
Israel y 30.000 de Judá.

\bibverse{9} A los mensajeros que llegaron les dijeron: ``Diganles a los
hombres de Jabes de Galaad: ``Mañana serán rescatados, para cuando el
sol caliente al medio día''. El pueblo de Jabes se puso muy contento
cuando los mensajeros les dieron este mensaje. \bibverse{10} Entonces
les dijeron a los amonitas: ``Nos rendiremos a ustedes mañana, y
entonces podrán hacer con nosotros lo que quieran.''

\bibverse{11} Al día siguiente, Saúl organizó al ejército en tres
divisiones. Atacaron el campamento amonita antes del amanecer y
siguieron matándolos hasta que llegó el medio día. Los sobrevivientes
estaban tan dispersos que ni siquiera quedaban dos de ellos juntos.

\bibverse{12} Entonces el pueblo le preguntó a Samuel: ``¿Dónde están
los que dijeron `¿Por qué debemos tener a Saúl como rey?' Entreguen a
estos hombres para ejecutarlos''.

\bibverse{13} Pero Saúl respondió: ``Nadie va a ser ejecutado hoy,
porque éste es el día en que el Señor ha salvado a Israel.''

\bibverse{14} Entonces Samuel le dijo al pueblo: ``Vengan conmigo,
vayamos a Gilgal y renovemos el reino''.

\bibverse{15} Y todos fueron a Gilgal, y confirmaron a Saúl como rey
ante el Señor. Sacaron ofrendas de paz para el Señor, y Saúl, junto con
todos los israelitas, hizo una gran celebración.

\hypertarget{section-11}{%
\section{12}\label{section-11}}

\bibverse{1} Entonces Samuel le dijo a todo Israel: ``He prestado
atención a todo lo que me han pedido, y les he dado un rey para que los
gobierne. \bibverse{2} Ahora su rey es su líder. Yo soy viejo y canoso,
y mis hijos están aquí con ustedes. Los he guiado desde que era un niño
hasta hoy. \bibverse{3} Aquí estoy ante ustedes. Traigan cualquier
acusación que tengan contra mí en presencia del Señor y de su
ungido.\footnote{\textbf{12:3} ``Su ungido'': Refiriéndose al rey.} ¿Me
he apropiadodel buey o del burro de alguien? ¿He perjudicado a alguien?
¿He oprimido a alguien? ¿He aceptado un soborno de alguien para hacerme
el de la vista gorda? Díganmelo y les pagaré por ello''.

\bibverse{4} ``No, nunca nos has engañado ni nos has oprimido'',
respondieron, ``y nunca has tomado nada de nadie''.

\bibverse{5} Samuel les dijo: ``El Señor es testigo, y su ungido es
testigo hoy, en este caso que les concierne, de que no soy culpable de
nada.''\footnote{\textbf{12:5} ``No soy culpable de nada'':
  literalmente, ``no han encontrado nada en mi mano.''}

``El Señor es testigo'', respondieron.

\bibverse{6} ``El Señor es testigo,\footnote{\textbf{12:6} Tomado de la
  Septuaginta.} el que designó a Moisés y a Aarón'', continuó Samuel.
``Él sacó a sus antepasados de la tierra de Egipto. \bibverse{7} Así
pues, permanezcan aquí mientras les presento, en presencia del Señor, la
prueba de todas las cosas buenas que el Señor ha hecho por ustedes y por
sus antepasados.

\bibverse{8} Después de que Jacob fue a Egipto, sus padres clamaron al
Señor por ayuda, y él envió a Moisés y a Aarón para que ayudaran a sus
antepasados a salir de Egipto y para establecerse aquí. \bibverse{9}
Pero se olvidaron del Señor, su Dios, y éste los abandonó en manos de
Sísara, comandante del ejército de Hazor, de los filisteos y del rey de
Moab, que los atacó.

\bibverse{10} Ellos clamaron al Señor por ayuda y dijeron: `Hemos
pecado, pues hemos rechazado al Señor y hemos adorado a los baales y a
Astoret. Por favor, sálvanos de las manos de nuestros enemigos, y te
adoraremos'. \bibverse{11} Entonces el Señor envió a Gedeón,\footnote{\textbf{12:11}
  ``Gedeón'': Llamado aquí ``Jerub-Baal.''} Barak,\footnote{\textbf{12:11}
  Tomado de la Septuaginta y version siríaca. El hebreo era ``Bedán.''}
Jefté y Samuel, y los salvó de los enemigos que los rodeaban para que
pudieran vivir con seguridad.

\bibverse{12} Pero cuando vieron que Nahas, rey de los amonitas, venía a
atacarlos, me dijeron: ``No, queremos nuestro propio rey'', aunque el
Señor, su Dios, era su rey. \bibverse{13} Así que aquí está el rey que
ustedes han elegido, el que pidieron. Miren: ¡el Señor se los entrega
ahora como su rey!

\bibverse{14} Si honran al Señor, lo adoran, hacen lo que les dice y no
se rebelan contra las instrucciones del Señor, y si tanto ustedes como
su rey siguen al Señor su Dios, ¡entonces todo estará bien!
\bibverse{15} Sin embargo, si se niegan a hacer su voluntad y se rebelan
contra las instrucciones del Señor, entonces el Señor estará contra
ustedes como lo estuvo contra sus antepasados.

\bibverse{16} Ahora quédense quietos y observen lo que el Señor va a
hacer, ante sus propios ojos. \bibverse{17} ¿No es el tiempo de la
cosecha de trigo?\footnote{\textbf{12:17} En esta época no solía
  producirse la lluvia.} Pues bien, le pediré al Señor que envíe truenos
y lluvia. Entonces se darán cuenta del mal que han hecho ante los ojos
del Señor cuando exigieron su propio rey''.

\bibverse{18} Entonces Samuel oró al Señor, y ese mismo día el Señor
envió truenos y lluvia. Todos estaban totalmente asombrados del Señor y
de Samuel.

\bibverse{19} ``¡Por favor, ruega al Señor tu Dios por nosotros, tus
siervos, para que no muramos!'', le rogaron a Samuel. ``Porque hemos
añadido a todos nuestros pecados la maldad de pedir nuestro propio
rey''.

\bibverse{20} ``No tengan miedo'', respondió Samuel. ``Aunque en verdad
hayan hecho todas estas maldades, no dejen de seguir al Señor, sino
dedíquense por completo a adorarlo. \bibverse{21} No adoren a ídolos sin
valor que los tales no pueden ayudarlos ni salvarlos, porque no son
nada. \bibverse{22} Lo cierto es que, gracias a la clase de persona que
es el Señor, no abandonará a su pueblo, porque se alegró de reclamarlos
a ustedes como suyos.

\bibverse{23} En cuanto a mí, ¿cómo podría pecar contra el Señor dejando
de orar por ustedes? También seguiré enseñándoles el camino del bien y
la rectitud. \bibverse{24} Asegúrense de honrar a Dios y adorarlo
fielmente, con total dedicación. Piensen en las maravillas que ha hecho
por ustedes. \bibverse{25} Pero si siguen haciendo lo malo, ustedes y su
rey serán eliminados.''

\hypertarget{section-12}{%
\section{13}\label{section-12}}

\bibverse{1} Saúl tenía treinta años cuando llegó a ser rey, y reinó
sobre Israel durante cuarenta y dos años. \bibverse{2} Saúl había
elegido a tres mil hombres de Israel. Dos mil de ellos estaban con Saúl
en Micmas y en la región montañosa de Betel, y otros mil estaban con
Jonatán en Guibeá de Benjamín. Y envió al resto del ejército a casa.

\bibverse{3} Tiempo después, Jonatán atacó la guarnición de los
filisteos en Geba. Los filisteos no tardaron en enterarse, así que Saúl
hizo sonar la trompeta de llamada a las armas por todo el país,
diciendo: ``Hebreos,\footnote{\textbf{13:3} ``Hebreos'': el término es
  el nombre dado por otros a los israelitas, y así utilizado aquí
  recuerda a los israelitas que son dominados por otras naciones.
  Algunos han sugerido incluso que el término se utilizaba para los
  israelitas que eran esclavos de los extranjeros.} presten atención!''

\bibverse{4} Entonces todo Israel escuchó la noticia: ``¡Saúl ha atacado
la guarnición filistea, y ahora los filisteos odian a Israel!'' Así que
todo el ejército fue convocado para unirse a Saúl en Gilgal.

\bibverse{5} Los filisteos se reunieron para pelear contra Israel.
Tenían tres mil\footnote{\textbf{13:5} El texto hebreo dice ``30.000'',
  lo que parece excesivo. La versión luciana de la Septuaginta y la
  versión siríaca dicen 3.000.} carros, seis mil jinetes y soldados tan
numerosos como la arena en la orilla del mar. Avanzaron y acamparon en
Micmas, al este de Bet-aven.

\bibverse{6} Cuando los hombres israelitas se dieron cuenta de la
difícil situación en la que se encontraban y de que el ejército estaba
recibiendo una paliza, se escondieron en cuevas, agujeros, rocas, pozos
y cisternas. \bibverse{7} Algunos de los hebreos incluso cruzaron el
Jordán hacia el territorio de Gad y Galaad, pero Saúl se quedó en
Gilgal, y todos los hombres que estaban con él temblaban de miedo.
\bibverse{8} Saúl esperó allí siete días el tiempo que Samuel había
dicho, pero Samuel no llegó a Gilgal, y el ejército comenzó a
abandonarlo.

\bibverse{9} Entonces Saúl ordenó: ``Tráiganme el holocausto y las
ofrendas de paz'', y presentó el holocausto.

\bibverse{10} Justo cuando terminó de presentar el holocausto, vio
llegar a Samuel. Saúl fue a recibirlo y a saludarlo.

\bibverse{11} ``¿Qué has hecho?'' le preguntó Samuel.

Saúl respondió: ``Bueno, vi que mis hombres me abandonaban, y que tú no
habías llegado cuando dijiste que lo harías, y que los filisteos se
estaban reuniendo en Micmas para atacar. \bibverse{12} Así que pensé:
`Los filisteos están a punto de atacarme en Gilgal, y no he pedido la
ayuda del Señor'. Así que sentí que debía presentar yo mismo el
holocausto''.

\bibverse{13} ``Has sido muy estúpido'', le dijo Samuel. ``No has
cumplido los mandatos del Señor, tu Dios. Si lo hubieras hecho, el Señor
habría asegurado tu reino sobre Israel para siempre. \bibverse{14} Pero
ahora tu reino no durará. El Señor ha encontrado para sí un hombre que
piensa como él, y lo ha elegido para que sea el gobernante de su pueblo,
porque tú no has cumplido los mandatos del Señor.''

\bibverse{15} Entonces Samuel se fue de Gilgal. El resto de los soldados
siguió a Saúl para reunirse con el ejército, yendo de Gilgal a Geba, en
Benjamín.\footnote{\textbf{13:15} En el texto hebreo falta una parte de
  este versículo, probablemente debido a un error de los copistas. Aquí
  se sigue la Septuaginta.} Saúl contó el número de soldados que estaban
con él y eran unos seiscientos. \bibverse{16} Saúl, su hijo Jonatán y
los soldados que estaban con ellos se alojaban en Geba de Benjamín,
mientras los filisteos estaban acampados en Micmas. \bibverse{17} Tres
grupos de asaltantes salieron del campamento filisteo para ir a atacar.
Un grupo se dirigió hacia Ofra en la tierra de Shual, \bibverse{18} otro
hacia Bet-horón, y otro hacia la frontera que da al Valle de Seboim por
el desierto.

\bibverse{19} En esos días no había un herrero en ninguna parte de
Israel. Los filisteos lo impedían para que los hebreos no hicieran
espadas y lanzas. \bibverse{20} Todos los israelitas tenían que acudir a
los filisteos para afilar sus rejas de hierro, picos, hachas y hoces.
\bibverse{21} La tarifa era de dos tercios de siclo\footnote{\textbf{13:21}
  ``Dos tercios de siclo'': literalmente ``un pim.''} por rejas de arado
y picos, y un tercio de siclo para afilar las hachas y las picas de
ganado.

\bibverse{22} Así que cuando llegó el día de la batalla ninguno de los
soldados que acompañaban a Saúl y a Jonatán tenía espadas ni lanzas;
sólo Saúl y su hijo Jonatán tenían esas armas.

\bibverse{23} Una guarnición filistea había tomado el control del paso
de Micmas.\footnote{\textbf{13:23} Este versículo es mejor tomarlo como
  parte del siguiente capítulo.}

\hypertarget{section-13}{%
\section{14}\label{section-13}}

\bibverse{1} Un día Jonatán, hijo de Saúl, le dijo al joven escudero:
``Vamos, crucemos a la guarnición filistea del otro lado''. Pero no le
hizo saber a su padre acerca de sus planes. \bibverse{2} Saúl se
encontraba cerca de Guibeá, bajo un granado\footnote{\textbf{14:2} ``Un
  granado'' (árbol): o ``la roca de Rimón.''} en Migrón. Tenía unos
seiscientos hombres con él, \bibverse{3} incluyendo a Ahija, que llevaba
un efod.\footnote{\textbf{14:3} ``Efod'': Un accesorio sacerdotal.} Era
hijo del hermano de Icabod, Ahitob, hijo de Finees, hijo de Elí,
sacerdote del Señor en Silo. Nadie se dio cuenta de que Jonatán se había
ido. \bibverse{4} A ambos lados del paso que Jonatán planeaba cruzar
para llegar a la guarnición filistea se erigían dos acantilados, uno
llamado Boses y el otro Sene. \bibverse{5} El acantilado del norte
estaba en el lado de Michmash, el del sur en el lado de Geba.

\bibverse{6} Jonatán le dijo al joven que llevaba la armadura: ``Vamos,
crucemos a la guarnición de estos hombres paganos\footnote{\textbf{14:6}
  ``Paganos'': literalmente, ``incircuncisos.''}. Tal vez el Señor nos
ayude. Al Señor no le cuesta ganar, sea por muchos o por pocos''.

\bibverse{7} ``Tú decides qué hacer'', respondió el escudero. ``¡Estoy
contigo sin importar lo que decidas!''

\bibverse{8} ``¡Vamos entonces!'' dijo Jonathan. ``Cruzaremos en su
dirección para que nos vean. \bibverse{9} Si nos dicen: `Esperen allí
hasta que bajemos a ustedes', esperaremos donde estamos y no subiremos a
ellos. \bibverse{10} Pero si nos dicen: ``Suban hacia nosotros'',
subiremos, porque eso será la señal de que el Señor nos los ha
entregado''.

\bibverse{11} Así que ambos se dejaron ver por la guarnición filistea.
``¡Mira!'', gritaron los filisteos. ``Los hebreos están saliendo de los
huecos\footnote{\textbf{14:11} ``Huecos'': la palabra se utiliza a
  menudo para describir las madrigueras donde viven los animales.} donde
se escondían.''

\bibverse{12} Los hombres de la guarnición llamaron a Jonatán y a su
escudero: ``¡Suban aquí y les mostraremos un par de cosas!''.

``Sígueme arriba'', dijo Jonatán a su escudero, ``porque el Señor los ha
entregado a Israel''.

\bibverse{13} Así que Jonatán subió de manos y pies, con su escudero que
iba justo detrás de él. Jonatán los atacó y los mató,\footnote{\textbf{14:13}
  ``Jonatán los atacó y los mató'': literalmente, ``cayeron ante
  Jonatán.''} y su escudero le siguió haciendo lo mismo. \bibverse{14}
En este primer ataque, Jonatán y su escudero mataron a unos veinte
hombres en un área de media hectárea. \bibverse{15} Entonces los
filisteos entraron en pánico, en el campamento, en el campo y en todo su
ejército. Incluso los que estaban en los puestos de avanzada y los
grupos de asaltantes se aterrorizaron. La tierra se estremeció. Era
terror proveniente de Dios.

\bibverse{16} Los vigías de Saúl en Guibeá, en Benjamín, vieron cómo el
ejército filisteo se desvanecía y se dispersaba en todas direcciones.
\bibverse{17} Saúl les dijo a los soldados que estaban con él: ``Pasen
lista y averigüen quiénes no están con nosotros''. Cuando pasaron lista,
descubrieron que Jonatán y su escudero no estaban allí.

\bibverse{18} Saúl le dijo a Ajías: ``Trae el Arca de Dios aquí''. (En
esa época el Arca de Dios viajaba con los israelitas).

\bibverse{19} Pero mientras Saúl hablaba con el sacerdote, el alboroto
que venía del campamento filisteo era cada vez más fuerte. Así que Saúl
le dijo al sacerdote: ``¡Olvídalo!''\footnote{\textbf{14:19}
  ``¡Olvídalo!'': literalmente, ``Quita tu mano''. El sacerdote estaba a
  punto de intentar determinar la voluntad del Señor con respecto a un
  ataque contra los filisteos, tal vez consultando el Urim y el Tumin en
  el efod o mediante el uso del Arca de Dios de alguna manera.
  Cualquiera que sea el caso, Saúl revocó su orden anterior de guía
  divina diciéndole al sacerdote que detuviera lo que estaba a punto de
  hacer.}

\bibverse{20} Entonces Saúl y todo su ejército se reunieron y entraron
en batalla. Descubrieron que los filisteos estaban en total desorden,
atacándose unos a otros con las espadas. \bibverse{21} Los hebreos que
antes se habían puesto del lado de los filisteos, y que estaban con
ellos en su campamento, cambiaron de bando y se unieron a los israelitas
que estaban con Saúl y Jonatán. \bibverse{22} Cuando todos los
israelitas que se habían escondido en la región montañosa de Efraín se
enteraron de que los filisteos estaban huyendo, también se unieron para
perseguir a los filisteos y atacarlos. \bibverse{23} Ese día el Señor
salvó a Israel, y la batalla se extendió más allá de
Bet-aven.\footnote{\textbf{14:23} La Septuaginta añade lo siguiente en
  este punto: ``y el ejército que acompañaba a Saúl contaba con unos
  diez mil hombres''. La batalla se extendió por la región montañosa de
  Efraín.''}

\bibverse{24} Aquel día fue difícil para los hombres de Israel porque
Saúl había ordenado al ejército hacer un juramento, diciendo: ``Maldito
el que coma algo antes de la noche, antes de que me haya vengado de mis
enemigos''. Así que nadie del ejército había comido nada. \bibverse{25}
Cuando todos entraron en el bosque, encontraron panales de miel en el
suelo. \bibverse{26} Mientras estaban en el bosque, vieron que la miel
se acababa, pero nadie la recogió para comerla porque todos tenían miedo
del juramento que habían hecho. \bibverse{27} Pero Jonatán no se había
enterado de que su padre había ordenado al ejército hacer ese juramento.
Así que metió la punta de su bastón en el panal, cogió un trozo para
comer y se sintió mucho mejor.\footnote{\textbf{14:27} ``Se sintió mucho
  mejor'': literalmente, ``sus ojos brillaron.'' Igual que en el
  versículo 29.} \bibverse{28} Pero uno de los soldados le dijo: ``Tu
padre hizo que el ejército hiciera un juramento solemne, diciendo:
``¡Maldito el que coma algo hoy! Por eso los hombres están agotados''.

\bibverse{29} ``Mi padre nos ha causado un montón de problemas a
todos,''\footnote{\textbf{14:29} ``A todos'': literalmente, ``la
  tierra.''} respondió Jonatán. ``Mira qué bien estoy porque he comido
un poco de esta miel. \bibverse{30} ¡Habría sido mucho mejor si el
ejército hubiera comido hoy en abundancia del botín tomado a sus
enemigos! ¿Cuántos filisteos más habrían matado?''

\bibverse{31} Después de derrotar a los filisteos ese día, matándolos
desde Micmas hasta Ajalón, los israelitas estaban totalmente agotados.
\bibverse{32} Se apoderaron del botín, tomando ovejas, vacas y terneros,
y los sacrificaron allí mismo en el suelo. Pero se los comieron con la
sangre.

\bibverse{33} Entonces le dijeron a Saúl: ``Mira, los hombres están
pecando contra el Señor al comer carne con la sangre''.

``¡Infractores de la ley!'', les dijo Saúl. ``¡Tira una piedra grande
aquí ahora mismo!'' \bibverse{34} Luego les dijo: ``Recorran todo el
lugar donde están los soldados y díganles: ``Cada uno debe traerme su
ganado o sus ovejas y sacrificarlos aquí, y luego comer. No pequen
contra el Señor comiendo carne con sangre'\,``. Cada uno del ejército
trajo lo que tenía\footnote{\textbf{14:34} ``Lo que tenía'': Tomado de
  la Septuaginta.} y lo sacrificó allí aquella noche. \bibverse{35}
Entonces Saúl construyó un altar al Señor. Este fue el primer altar que
construyó al Señor.

\bibverse{36} Saúl dijo: ``Vamos a perseguir a los filisteos durante la
noche y a saquearlos hasta el amanecer, sin dejar sobrevivientes''.

``Haz lo que creas conveniente'', respondieron. Pero el sacerdote dijo:
``Preguntémosle primero a Dios''.

\bibverse{37} Saúl preguntó a Dios: ``¿Debo bajar y perseguir a los
filisteos? ¿Los entregarás a Israel?'' Pero ese día Dios no le
respondió.

\bibverse{38} Entonces Saúl dio la orden: ``Todos los comandantes del
ejército, vengan aquí para que podamos investigar qué pecado ha ocurrido
hoy. \bibverse{39} ¡Juro por la vida del Señor que salva a Israel que,
aunque sea mi hijo Jonatán, tendrá que morir!'' Pero nadie en todo el
ejército dijo nada.

\bibverse{40} Saúl les dijo a todos: ``Ustedes pónganse a un lado, y yo
y mi hijo Jonatán nos pondremos en el lado opuesto''.

``Hagan lo que les parezca mejor'', respondió el ejército.

\bibverse{41} Saúl oró al Señor, el Dios de Israel: ``Que el Tumím nos
muestre.''\footnote{\textbf{14:41} En otras palabras, que el Tumin
  muestre quién es el culpable.} Jonatán y Saúl fueron identificados,
mientras que todos los demás fueron absueltos.

\bibverse{42} Entonces Saúl dijo: ``Echen suertes entre mi hijo Jonatán
y yo''. Jonatán fue seleccionado.

\bibverse{43} ``Dime qué has hecho'', le preguntó Saúl a Jonatán.

``Sólo probé un poco de miel con la punta de mi bastón'', le dijo
Jonatán. ``Aquí estoy, y tengo que morir''.

\bibverse{44} Saúl dijo: ``¡Que Dios me castigue muy severamente si no
mueres, Jonatán!''

\bibverse{45} Pero el pueblo le dijo a Saúl: ``¿Tiene que morir Jonatán,
el que logró esta gran victoria en Israel? ¡De ninguna manera! Juramos
por la vida del Señor que ni un solo cabello de su cabeza caerá al
suelo, pues fue con la ayuda de Dios que logró esto hoy.'' El pueblo
salvó a Jonatán, y éste no murió.

\bibverse{46} Entonces Saúl dejó de perseguir a los filisteos, y los
filisteos se fueron a su propio país.

\bibverse{47} Después de que Saúl aseguró su dominio sobre Israel, luchó
contra todos sus enemigos de alrededor: Moabitas, amonitas, edomitas,
los reyes de Soba y los filisteos. En cualquier dirección que tomara,
los derrotaba a todos. \bibverse{48} Luchó con valentía, conquistando a
los amalecitas y salvando a Israel de los que los saqueaban.

\bibverse{49} Los hijos de Saúl fueron Jonatán, Isvi,\footnote{\textbf{14:49}
  Llamado tambiénIsboset.} y Malquisúa. Los nombres de sus dos hijas
eran Merab, (la primogénita), y Mical, (la menor). \bibverse{50} El
nombre de su esposa era Ahinoam, hija de Ahimaas. El nombre del
comandante del ejército de Saúl era Abner, hijo de Ner, y Ner era tío de
Saúl. \bibverse{51} Cis, padre de Saúl, y Ner, padre de Abner, eran
hijos de Abiel.

\bibverse{52} Durante toda su vida Saúl estuvo en guerra constante con
los filisteos. Saúl reclutó para su ejército a todo guerrero fuerte y a
todo luchador valiente que encontró.

\hypertarget{section-14}{%
\section{15}\label{section-14}}

\bibverse{1} Entonces Samuel le dijo a Saúl: ``El Señor me ha enviado
para ungirte como rey de su pueblo Israel. Así que presta atención a lo
que el Señor dice. \bibverse{2} Y esto es lo que dice el Señor
Todopoderoso: He observado lo que los amalecitas le hicieron a Israel
cuando los emboscaron en su camino desde Egipto. \bibverse{3} Ve y ataca
a los amalecitas y extermínalos a todos. No perdones a nadie, sino que
mata a todo hombre, mujer, niño y bebé; a todo buey, oveja, camello y
asno''.

\bibverse{4} Saúl convocó a su ejército en Telem.\footnote{\textbf{15:4}
  Aquí se escribe Telaim, pero se cree que es la misma ciudad llamada
  Telem en Josué 15:24.} Había 200.000 infantes israelitas y 10.000
hombres de Judá. \bibverse{5} Saúl avanzó hacia el pueblo de Amalec y
preparó una emboscada en el valle. \bibverse{6} Saúl envió un mensaje
para advertirles a los ceneos: ``Salgan de la zona y dejen a los
amalecitas para que no los destruya a ustedes con ellos, porque ustedes
mostraron bondad con todo el pueblo de Israel en su camino desde
Egipto.'' Así que los ceneos se alejaron y dejaron abandonaron a los
amalecitas.

\bibverse{7} Saúl derrotó a los amalecitas desde Havila hasta Shur, al
oriente de Egipto. \bibverse{8} Capturó vivo a Agag, rey de Amalec, pero
exterminó a todo el pueblo a espada. \bibverse{9} Saúl y su ejército
perdonaron a Agag, junto con las mejores ovejas y ganado, los terneros y
corderos gordos, y todo lo que era bueno. No quisieron destruir eso,
sino que destruyeron por completo todo lo despreciable y que no tenía
valor.

\bibverse{10} El Señor envió un mensaje a Samuel, diciendo:
\bibverse{11} ``Lamento haber hecho rey a Saúl, porque ha dejado de
seguirme y no ha hecho lo que le ordené''. Samuel se molestó y clamó al
Señor durante toda la noche.

\bibverse{12} Entonces Samuel se levantó de madrugada y fue a buscar a
Saúl, pero le dijeron: ``Saúl se ha ido al Carmelo. Allí incluso ha
erigido un monumento para honrarse a sí mismo, y ahora se ha marchado y
ha bajado a Gilgal''.

\bibverse{13} Cuando Samuel lo alcanzó, Saúl dijo: ``¡El Señor te
bendiga! He hecho lo que el Señor me ha ordenado''.

\bibverse{14} ``¿Qué es ese balido de las ovejas que escuchan mis oídos?
¿Qué es ese mugido del ganado que estoy oyendo?'' preguntó Samuel.

\bibverse{15} ``El ejército las trajo de los amalecitas'', respondió
Saúl. ``Les perdonaron las mejores ovejas y reses para sacrificarlas al
Señor, tu Dios, pero nosotros destruimos por completo el resto''.

\bibverse{16} ``¡Cállate!'' le dijo Samuel a Saúl. ``Déjame contarte lo
que el Señor me dijo anoche''.

``Dime lo que dijo'', respondió Saúl.

\bibverse{17} ``Antes no solías pensar mucho en ti mismo, ¿pero no eres
ahora el líder de las tribus de Israel?'' preguntó Samuel. ``El Señor te
ungió como rey de Israel. \bibverse{18} Luego te dio una orden,
diciéndote: `Ve y extermina a esos pecadores, los amalecitas. Atácalos
hasta destruirlos a todos'. \bibverse{19} ¿Por qué no hiciste lo que el
Señor te ordenó? ¿Por qué te abalanzaste sobre el despojo e hiciste lo
malo ante los ojos del Señor?''

\bibverse{20} ``¡Pero si hice lo que el Señor me ordenó!'' respondió
Saúl. ``Fui e hice lo que el Señor me mandó hacer. Hice regresar a Agag,
rey de Amalec, y destruí por completo a los amalecitas. \bibverse{21} El
ejército tomó ovejas y ganado del botín, lo mejor de lo que estaba
apartado para Dios, para sacrificarlo al Señor, tu Dios, en Gilgal.''

\bibverse{22} ``¿Qué crees que prefiere el Señor? ¿Los holocaustos y los
sacrificios? ¿O que seas obediente a su palabra?'' le preguntó Samuel.
``¡Escucha! ¡La obediencia es mejor que los sacrificios! Prestar
atención es más importante que ofrecer la grasa de los carneros.
\bibverse{23} La rebelión es tan mala como la brujería, y la arrogancia
es tan mala como el pecado de la idolatría. Porque has rechazado los
mandatos del Señor, él te ha rechazado como rey''.

\bibverse{24} ``He pecado'', confesó Saúl a Samuel. ``Desobedecí las
órdenes del Señor y tus instrucciones, porque tuve miedo del pueblo y
seguí lo que ellos decían. \bibverse{25} Así que, por favor, perdona mi
pecado y vuelve conmigo, para que pueda adorar al Señor''.

\bibverse{26} Pero Samuel le dijo: ``No voy a volver contigo. Has
rechazado las órdenes del Señor, y el Señor te ha rechazado como rey de
Israel''.

\bibverse{27} Cuando Samuel se dio la vuelta para marcharse, Saúl se
agarró del dobladillo de su túnica, y ésta se rasgó.

\bibverse{28} Samuel le dijo: ``¡El Señor te ha arrancado hoy el reino
de Israel y se lo ha dado a tu prójimo, a uno que es mejor que tú!
\bibverse{29} ¡Además la Gloria de Israel no miente ni cambia de
opinión, porque él no es un ser humano!''

\bibverse{30} ``Sí, he pecado'', respondió Saúl. ``Por favor, hónrame
ahora ante los ancianos de mi pueblo y ante Israel; vuelve conmigo, para
que pueda adorar al Señor, tu Dios''. \bibverse{31} Así que Samuel
regresó con Saúl después de todo, y Saúl adoró al Señor.

\bibverse{32} Entonces Samuel dijo: ``Tráeme a Agag, rey de los
amalecitas''. Agag se acercó a él confiado, pues pensó: ``La amenaza de
muerte debe haber pasado ya''.

\bibverse{33} Pero Samuel le dijo: ``De la misma manera que tu espada ha
dejado sin hijos a las mujeres, también tu madre quedará sin hijos entre
las mujeres.'' Entonces Samuel descuartizó a Agag ante el Señor en
Gilgal.

\bibverse{34} Samuel se fue a Ramá, y Saúl se fue a su casa en Guibeá de
Saúl. \bibverse{35} Hasta el día de su muerte, Samuel no volvió a
visitar a Saúl. Samuel se lamentó por Saúl, y el Señor se arrepintió de
haber hecho a Saúl rey de Israel.

\hypertarget{section-15}{%
\section{16}\label{section-15}}

\bibverse{1} El Señor un día le preguntó a Samuel: ``¿Hasta cuándo vas a
seguir llorando a Saúl porque lo he rechazado como rey de Israel? Llena
tu frasco\footnote{\textbf{16:1} ``Frasco'': literalmente, ``cuerno.''}
con aceite de oliva y vete. Ve donde Isaí de Belén, porque he elegido un
rey para mí de entre sus hijos''.

\bibverse{2} ``¿Cómo puedo ir a hacer eso?'' preguntó Samuel. ``¡Saúl se
enterará y me matará!''.

El Señor respondió: ``Lleva contigo una novilla y di: `He venido a
sacrificar al Señor'. \bibverse{3} Invita a Isaí al sacrificio, y yo te
enseñaré lo que tienes que hacer. Unge para mí al que yo te diga''.

\bibverse{4} Samuel hizo lo que el Señor le había dicho y fue a Belén.
Cuando los ancianos de la ciudad le salieron al encuentro, se asustaron
y le preguntaron: ``¿Vienes en son de paz?''

\bibverse{5} ``Sí, vengo en son de paz'', respondió. ``He venido a
presentar sacrificio al Señor. Purifíquense y vengan conmigo a hacer el
sacrificio''. Entonces purificó a Isaí y a sus hijos y los invitó al
sacrificio.

\bibverse{6} Cuando llegaron y Samuel vio a Eliab, pensó para sí:
``¡Este tiene que ser el ungido del Señor!''.

\bibverse{7} Pero el Señor le dijo a Samuel: ``No te fijes en su aspecto
exterior ni en su altura porque lo he rechazado. Porque el Señor no mira
como los seres humanos. Los seres humanos sólo ven con sus ojos lo que
está en el exterior, pero el Señor mira la forma de pensar de las
personas en su interior.''

\bibverse{8} Entonces Isaí llamó a Abinadab y lo hizo venir ante Samuel,
quien dijo: ``El Señor tampoco ha elegido a éste.''

\bibverse{9} Entonces Isaí hizo que Simea se presentara. Pero Samuel
dijo: ``El Señor tampoco ha elegido a éste''.

\bibverse{10} Isaí hizo que siete de sus hijos se presentaran ante
Samuel, pero éste le dijo: ``El Señor no ha elegido a ninguno de
éstos.''

\bibverse{11} Entonces le preguntó a Isaí: ``¿No tienes más hijos?''.

``Bueno, aún queda el más joven'', respondió Isaí, ``pero está fuera
cuidando las ovejas''.

``Manda a buscarlo y tráelo aquí, porque no nos vamos a sentar a
comer\footnote{\textbf{16:11} ``sentar a comer'': literalmente,
  ``rodear.'' Normalmente se piensa que significa rodear una mesa antes
  de sentarse, pero también podría significar ``rodear'' un altar, es
  decir, el comienzo de los rituales de sacrificio.} hasta que llegue
aquí'', le dijo Samuel a Isaí.

\bibverse{12} Así que Isaí mandó a buscarlo y lo trajo delante de
Samuel. Tenía una tez roja y unos ojos hermosos, y tenía buen parecer.
El Señor dijo: ``Ve a ungirlo, porque es él''.

\bibverse{13} Samuel tomó el frasco de aceite de oliva y lo ungió en
presencia de sus hermanos, y el Espíritu del Señor vino sobre David con
poder desde aquel día. Luego Samuel se fue y regresó a Ramá.

\bibverse{14} El Espíritu del Señor había abandonado a Saúl, y un
espíritu maligno del Señor lo atormentaba.\footnote{\textbf{16:14} Como
  en otras partes de la Escritura, a veces se presenta a Dios como si
  hiciera algo que en realidad no impide. La eliminación del Espíritu
  del Señor dejó a Saúl abierto al control de otro espíritu. La forma en
  que los siervos reaccionan muestra que esta era una visión común de la
  época: se responsabiliza a Dios de los problemas de Saúl.}
\bibverse{15} Los siervos de Saúl le dijeron: ``Sin duda es un espíritu
maligno de Dios el que te atormenta. \bibverse{16} Danos aquí la orden
de encontrar a alguien que sea bueno tocando el arpa, para que cuando el
espíritu maligno de Dios venga sobre ti, pueda tocar y te sientas mucho
mejor.''

\bibverse{17} Saúl dio la orden a sus siervos: ``Busquen a alguien que
sea bueno tocando el arpa y tráiganlo aquí''.

\bibverse{18} Uno de los criados respondió: ``Conozco a un hijo de Isaí,
de Belén, que es bueno tocando el arpa. Es un hombre valiente, buen
luchador, de buen hablar y guapo, y el Señor está con él.''

\bibverse{19} Saúl envió mensajeros a Isaí, diciéndole: ``Envíame a tu
hijo David, el que cuida las ovejas''.

\bibverse{20} Así que Isaí cargó un burro con pan, un odre de vino y un
cabrito y los envió con su hijo David a Saúl. \bibverse{21} David llegó
a Saúl y comenzó a trabajar para él. Saúl lo apreciaba mucho, y David se
convirtió en su escudero.

\bibverse{22} Saúl envió un mensaje a Isaí, diciendo: ``Por favor,
permite que David siga trabajando para mí, porque estoy complacido con
él''.

\bibverse{23} Así, cada vez que el espíritu de Dios se apoderaba de
Saúl, David tomaba su arpa y tocaba, y Saúl se aliviaba y se sentía
mejor, y el espíritu maligno lo dejaba.

\hypertarget{section-16}{%
\section{17}\label{section-16}}

\bibverse{1} Los ejércitos filisteos se reunieron para la batalla en
Soco, en Judá. Acamparon en Efes-damim, entre Socoh y Azeca.
\bibverse{2} Saúl y los israelitas se reunieron y acamparon en el Valle
de Ela y tomaron sus posiciones para comenzar la batalla contra los
filisteos. \bibverse{3} Los filisteos estaban en una colina y los
israelitas en otra, con el valle entre ellos.

\bibverse{4} Entonces salió del campamento filisteo un
campeón\footnote{\textbf{17:4} ``Campeón'': literalmente ``un hombre del
  espacio intermedio''. Suele entenderse como un campeón que luchará
  contra otro en una especie de batalla por delegación, pero su
  significado preciso es incierto, ya que sólo aparece aquí y en el
  versículo 23 en todo el Antiguo Testamento.}. Se llamaba Goliat, de
Gat, y medía seis codos y un palmo.\footnote{\textbf{17:4} ``Seis codos
  y un palmo de altura''. Esto equivale a unos nueve pies y medio. La
  Septuaginta y un manuscrito de Qumrán tienen cuatro codos y un palmo,
  lo que equivale a seis pies y medio.} \bibverse{5} Tenía en la cabeza
un casco de bronce y llevaba una cota de malla de bronce que pesaba
cinco mil siclos. \bibverse{6} En las piernas llevaba una armadura de
bronce y una jabalina\footnote{\textbf{17:6} ``Jabalina'': algunos creen
  que se trata más bien de una espada curva o una cimitarra, y
  ciertamente se hace referencia a una espada en el versículo 51.}
colgada entre sus hombros. \bibverse{7} El asta de su lanza era tan
gruesa como una viga de tejedor, con una punta de hierro que pesaba
seiscientos siclos. Su escudero caminaba delante de él llevando su
escudo.\footnote{\textbf{17:7} ``Llevando su escudo'': añadido para
  mayor claridad.}

\bibverse{8} Goliat se puso de pie y gritó a las filas de soldados
israelitas: ``¿Por qué han venido y se han puesto en fila para la
batalla? Yo soy el filisteo, y ustedes son los siervos de Saúl. Elijan a
uno de sus hombres y hagan que descienda a pelear conmigo. \bibverse{9}
Si él puede pelear conmigo y logra matarme, entonces los filisteos
seremos sus esclavos. Pero si lo venzo y lo mato, entonces ustedes serán
nuestros esclavos y trabajarán para nosotros''.

\bibverse{10} Entonces el filisteo dijo: ``¡Me burlo de las líneas de
batalla de Israel hoy! Dénme un hombre para que podamos luchar los
dos''.

\bibverse{11} Saúl y todos los soldados israelitas quedaron destrozados
y absolutamente aterrados cuando oyeron lo que dijo el filisteo.

\bibverse{12} David era uno de los hijos de Isaí, un efrateo de Belén de
Judá que tenía ocho hijos. En la época en que Saúl era rey, Isaí era muy
viejo. \bibverse{13} Los tres hijos mayores de Isaí se habían unido a la
guerra como parte del ejército de Saúl. Ellos eran Eliab (el
primogénito), Abinadab (el segundo) y Simea (el tercero). 14 David era
el más joven. Los tres hijos mayores estaban con Saúl, \bibverse{15}
mientras que David iba con Saúl y luego volvía para cuidar las ovejas de
su padre.

\bibverse{16} Todas las mañanas y las tardes, durante cuarenta días, el
filisteo salió y se puso en pie en el mismo lugar.

\bibverse{17} Isaí le dijo a su hijo David: ``Por favor, lleva a tus
hermanos este efa de grano tostado y estos diez panes para tus hermanos.
Llévalos rápidamente al campamento de tus hermanos. \bibverse{18}
Además, lleva estos diez trozos de queso a su comandante. Comprueba con
cuidado cómo están tus hermanos y tráeme noticias de ellos''.
\bibverse{19} Sus hermanos estaban con Saúl y todo el ejército israelita
en el Valle de Ela, luchando contra los filisteos.

\bibverse{20} David se levantó de madrugada y dejó el rebaño con un
pastor. Tomó las provisiones y se puso en marcha como se lo había dicho
Isaí. Llegó al campamento justo cuando el ejército marchaba hacia su
línea de batalla, gritando el grito de guerra. \bibverse{21} Los
israelitas se colocaron en su línea de batalla y los filisteos en la del
lado opuesto. 22 David dejó sus provisiones con el responsable y corrió
a la línea de batalla. Cuando llegó allí, preguntó a sus hermanos cómo
estaban. \bibverse{23} Mientras hablaba con ellos, Goliat, el campeón
filisteo de Gat, salió de sus filas y gritó su desafío como antes, y
esta vez David escuchó lo que decía.

\bibverse{24} Todos los soldados israelitas huyeron al verlo, porque
tenían un miedo terrible. \bibverse{25} ``¿Han visto a ese hombre que no
deja de salir para burlarse de Israel?'', preguntaron. ``El rey hará muy
rico al hombre que lo mate. También le dará a su hija en matrimonio, y
su familia vivirá libre de impuestos en Israel''.

\bibverse{26} Entonces David les preguntó a los hombres que estaban a su
lado: ``¿Qué recibirá el hombre que mate a este filisteo y elimine esta
vergüenza de Israel? ¿Quién se cree que es este Filisteo
pagano\footnote{\textbf{17:26} ``Pagano'': literalmente,
  ``incircunciso.'' Del mismo modo ocurre en el versículo 36.} para
burlarse del Dios vivo de los ejércitos?''.

\bibverse{27} Los soldados repitieron lo que habían dicho, diciéndole:
``Esto es lo que recibirá el que lo mate''.

\bibverse{28} Cuando Eliab, el hermano mayor de David, lo oyó hablar con
los hombres, se enojó con él. ``¿Qué haces aquí?'', le preguntó. ``¿Con
quién has dejado esas pocas ovejas en el desierto? ¡Sé lo orgulloso y
malvado que eres! Sólo has venido a ver la batalla''.

\bibverse{29} ``¿Qué he hecho ahora?'' preguntó David. ``¿No puedo ni
siquiera hacer una pregunta?''. \bibverse{30} Se acercó a otros y les
hizo la misma pregunta, y ellos le dieron la misma respuesta que antes.
\bibverse{31} Alguien escuchó lo que dijo David y se lo comunicó a Saúl,
que mandó a buscarlo.

\bibverse{32} David le dijo a Saúl: ``Que nadie se desanime por culpa de
este filisteo. Yo, tu siervo, iré a luchar contra él''.

\bibverse{33} ``No puedes ir a luchar contra ese filisteo'', respondió
Saúl. ``Tú eres sólo un muchacho, y él es un guerrero entrenado desde su
juventud''.

\bibverse{34} David respondió: ``Tu siervo ha estado cuidando las ovejas
de su padre. Cuando venía un león o un oso y se llevaba un cordero del
rebaño, \bibverse{35} yo lo perseguía, lo derribaba y salvaba el cordero
de su boca. Si se volvía para atacarme, le agarraba el pelo, lo golpeaba
y lo mataba. \bibverse{36} He matado leones y osos, y este pagano
filisteo será como uno de ellos, pues se ha burlado de los ejércitos del
Dios vivo.''

\bibverse{37} David concluyó: ``El Señor, que me salvó de las garras del
león y del oso, y del mismo modo me salvará de este filisteo.''

``Ve, y que el Señor esté contigo'', respondió Saúl.

\bibverse{38} Saúl le dio a David su propia ropa de combate para que se
la pusiera, le colocó un casco de bronce en la cabeza y le puso una
armadura. \bibverse{39} David se puso la espada sobre la armadura, pero
no podía caminar porque no estaba acostumbrado.

``No puedo caminar con todo esto'', le dijo David a Saúl. ``No estoy
acostumbrado''. Así que David se quitó toda la armadura. \bibverse{40}
Tomó su bastón, escogió cinco piedras lisas del arroyo y las puso en su
bolsa de pastor. Llevando su honda en la mano, se acercó al filisteo.

\bibverse{41} El filisteo se acercó a David, cada vez más cerca, con su
escudero al frente. \bibverse{42} Cuando el filisteo miró de cerca, pudo
ver que David era sólo un joven apuesto de cara roja, y entonces trató a
David con desprecio.

\bibverse{43} ``¿Piensas que soy un perro para venir a pelear conmigo
con un palo?'', le preguntó el filisteo a David, y lo maldijo por sus
dioses. \bibverse{44} Entonces el filisteo le gritó a David: ``Ven aquí,
y daré de comer tu carne a las aves y a los animales salvajes.''

\bibverse{45} David le respondió al filisteo: ``Tú vienes a atacarme con
espada, lanza y jabalina. Pero yo vengo a atacarte en nombre del Señor
Todopoderoso, el Dios de los ejércitos de Israel, del que te has
burlado. \bibverse{46} Hoy el Señor te entregará en mis manos, y yo te
derribaré; te cortaré la cabeza y entregaré los cadáveres de los
soldados filisteos a las aves y a los animales salvajes. Entonces todo
el mundo sabrá que hay un Dios que actúa por Israel. \bibverse{47} Todos
los aquí reunidos se darán cuenta de que el Señor salva, pero no con
espada y lanza. Porque la batalla es del Señor, y él nos entregará a
todos los filisteos''.

\bibverse{48} Cuando el filisteo avanzó para atacarlo, David corrió
hacia la línea de batalla para enfrentarlo. \bibverse{49} David metió la
mano en su bolsa, sacó una piedra y la disparó con su honda, golpeando
al filisteo en la frente. La piedra se le clavó en la frente, y Goliat
se desplomó boca abajo en el suelo.

\bibverse{50} Así fue como David derrotó al filisteo con sólo una honda
y una piedra; sin espada en la mano, David derribó al filisteo y lo
mató. \bibverse{51} Entonces David corrió y se paró sobre el filisteo.
Tomó la espada del filisteo y la sacó de su vaina. Lo mató y luego le
cortó la cabeza con la espada. Cuando los filisteos vieron que su
campeón estaba muerto, dieron la vuelta y huyeron.

\bibverse{52} Entonces los hombres de Israel y de Judá se lanzaron al
grito de guerra y persiguieron a los filisteos hasta Gat y hasta las
puertas de Ecrón. Sus cuerpos fueron esparcidos a lo largo del camino de
Saaraim hacia Gat y Ecrón.

\bibverse{53} Cuando los israelitas regresaron de su acalorada
persecución a los filisteos, saquearon sus campamentos. \bibverse{54}
David tomó la cabeza del filisteo y la llevó a Jerusalén, pero puso las
armas del filisteo en su propia tienda.

\bibverse{55} Cuando Saúl vio que David salía a luchar contra el
filisteo, le preguntó a Abner, el comandante del ejército: ``Abner, ¿de
quién es hijo ese joven?''

``Por su vida, Su Majestad, no lo sé'', respondió Abner.

\bibverse{56} ``Averigua de quién es hijo este joven'', ordenó el rey.

\bibverse{57} En cuanto David regresó de matar al filisteo, Abner lo
tomó y lo llevó ante Saúl. David todavía tenía la cabeza del filisteo en
la mano.

\bibverse{58} ``¿De quién eres hijo, joven?'' preguntó Saúl.

``Soy hijo de tu siervo Isaí de Belén'', respondió David.

\hypertarget{section-17}{%
\section{18}\label{section-17}}

\bibverse{1} Después de que David terminó de hablar con Saúl, Jonatán se
hizo gran amigo de David. Amaba a David como a sí mismo. \bibverse{2}
Desde entonces, Jonatán hizo que David trabajara para él y no lo dejó
volver a su casa. \bibverse{3} Jonatán hizo un acuerdo solemne con David
porque lo amaba como a sí mismo. \bibverse{4} Jonatán se quitó la túnica
que llevaba puesta y se la dio a David, junto con su túnica, su espada,
su arco y su cinturón.\footnote{\textbf{18:4} Estas acciones fueron una
  forma de confirmar el acuerdo.}

\bibverse{5} David tenían éxito al hacer todo lo que Saúl le pedía, así
que Saúl lo nombró oficial del ejército. Esto complació a todos, incluso
a los demás oficiales de Saúl.

\bibverse{6} Cuando los soldados regresaron a casa después de que David
había matado al filisteo, las mujeres de todos los pueblos de Israel
salieron cantando y bailando al encuentro del rey Saúl, celebrando
alegremente con panderetas e instrumentos musicales. \bibverse{7}
Mientras bailaban, las mujeres cantaban: ``Saúl ha matado a sus miles, y
David a sus decenas de miles''.

\bibverse{8} Lo que cantaban enojó mucho a Saúl, pues no le pareció
bien. Se dijo a sí mismo: ``A David le han dado el crédito de haber
matado a decenas de miles, pero a mí sólo a miles. Lo único que falta es
darle el reino''. \bibverse{9} Desde entonces Saúl miró a David con
recelo.

\bibverse{10} Al día siguiente, un espíritu maligno de Dios se apoderó
de Saúl con fuerza, y despotricó\footnote{\textbf{18:10} ``Despotricó'':
  la palabra se traduce normalmente como ``profetizar'' (véase, por
  ejemplo, 10:10 cuando se aplica a Saúl), pero la función principal de
  un verdadero profeta de Dios era entregar mensajes de Dios. Que la
  fuente fuera ``un espíritu maligno'' no encaja en tal imagen, incluso
  si el espíritu maligno ``viniera de Dios.''} dentro de la casa
mientras David tocaba el arpa como lo hacía habitualmente. Resulta que
Saúl tenía una lanza en la mano, \bibverse{11} y se la lanzó a David,
mientras pensaba: ``Clavaré a David en la pared''. Pero David logró
escapar de él dos veces.

\bibverse{12} Saúl tenía miedo de David, porque el Señor estaba con él,
pero se había rendido ante Saúl. \bibverse{13} Así que Saúl despidió a
David y lo nombró comandante de mil soldados, dirigiéndolos de ida y
vuelta como parte del ejército.

\bibverse{14} David siguió teniendo mucho éxito en todo lo que hacía,
porque el Señor estaba con él. \bibverse{15} Cuando Saúl vio el éxito de
David, le tuvo aún más miedo. \bibverse{16} Pero todos en Israel y en
Judá amaban a David, por su liderazgo en el ejército.

\bibverse{17} Un día Saúl le dijo a David: ``Aquí está mi hija mayor,
Merab. Te la daré en matrimonio, pero sólo si me demuestras que eres un
guerrero valiente y luchas en las batallas del Señor''. Porque Saúl
pensaba: ``No hace falta que sea yo quien lo mate; que lo hagan los
filisteos''.

\bibverse{18} ``Pero, ¿quién soy yo, y qué categoría tiene mi familia en
Israel, para que me convierta en yerno del rey?'' respondió
David.\footnote{\textbf{18:18} David may have been concerned at the cost
  of providing a dowry, especially as this is a condition of marriage
  mentioned later in verse 25.}

\bibverse{19} Sin embargo, cuando llegó el momento de entregar a Merab,
la hija de Saúl, a David, ésta fue dada en matrimonio a Adriel de Meholá
en su lugar.

\bibverse{20} Mientras tanto, la hija de Saúl, Mical, se había enamorado
de David, y cuando se lo dijeron a Saúl, se alegró de ello.
\bibverse{21} ``Se la daré a David'', pensó Saúl. ``Ella puede ser la
carnada para que los filisteos lo atrapen''. Entonces Saúl le dijo a
David: ``Esta es la segunda vez que puedes ser mi yerno''.

\bibverse{22} Saúl les dio estas instrucciones a sus siervos: ``Hablen
con David en privado y díganle: ``Mira, el rey está muy contento contigo
y todos te queremos. ¿Por qué no te conviertes en el yerno del rey?''

\bibverse{23} Los sirvientes de Saúl hablaron en privado con David, pero
él respondió: ``¿Creen que no es nada hacerse yerno del rey? Soy un
hombre pobre y no soy importante''.

\bibverse{24} Cuando los sirvientes de Saúl le explicaron lo que David
había dicho, \bibverse{25} Saúl les dijo: ``Díganle a David que la única
dote que el rey quiere para la novia son cien prepucios de filisteos
muertos, como forma de vengarse de sus enemigos.'' El plan de Saúl era
hacer que los filisteos mataran a David.

\bibverse{26} Cuando los sirvientes le informaron a David de lo que el
rey había dicho, éste se alegró de ser el yerno del rey. Mientras había
tiempo, \bibverse{27} David partió con sus hombres y mató a doscientos
filisteos, y trajo sus prepucios. Los contaron todos ante el rey para
que David se convirtiera en yerno del rey. Entonces Saúl le dio a su
hija Mical en matrimonio.

\bibverse{28} Saúl se dio cuenta de que el Señor estaba con David y de
que su hija Mical estaba enamorada de David, \bibverse{29} por lo que se
volvió aún más temeroso de David, y fue enemigo de éste por el resto de
su vida.

\bibverse{30} Cada vez que los comandantes filisteos atacaban, David
tenía más éxito en la batalla que todos los oficiales de Saúl, por lo
que su fama se extendió rápidamente.

\hypertarget{section-18}{%
\section{19}\label{section-18}}

\bibverse{1} Entonces Saúl ordenó a su hijo Jonatán y a todos sus
funcionarios que mataran a David. Pero Jonatán apreciaba mucho David,
\bibverse{2} así que le advirtió: ``Mi padre Saúl está tratando de
matarte. Así que ten cuidado mañana por la mañana: busca un lugar donde
esconderte y permanece oculto. \bibverse{3} Yo saldré con mi padre y me
pondré en el campo cerca de donde te escondes. Hablaré con él sobre ti y
veré lo que puedo averiguar, y luego te avisaré''.

\bibverse{4} Entonces Jonatán habló positivamente de David a su padre
Saúl, y le dijo: ``El rey no debe hacer nada malo a su siervo David,
porque él no le ha hecho nada malo; siempre le ha servido bien.
\bibverse{5} Se tomó la vida en sus manos cuando mató al filisteo, y el
Señor logró una gran salvación para todo Israel. Tú lo viste y te
alegraste, así que ¿por qué pecar y derramar sangre inocente matando a
David sin tener ninguna razón?''

\bibverse{6} Saúl aceptó lo que Jonatán tenía que decir y prometió con
un juramento ``Juro por la vida del Señor que no lo matarán''.

\bibverse{7} Más tarde Jonatán llamó a David y le contó todo lo que se
había dicho. Luego lo llevó ante Saúl, y David trabajó para Saúl como lo
había hecho antes.

\bibverse{8} La guerra estalló de nuevo, y David fue a luchar contra los
filisteos. Los atacó con tanta fuerza que huyeron derrotados.

\bibverse{9} Algún tiempo después, un espíritu maligno del Señor se
apoderó de Saúl mientras estaba sentado en su casa con su lanza en la
mano. Mientras David tocaba la lira, \bibverse{10} Saúl intentó clavar a
David en la pared con la lanza. Pero David logró esquivar la lanza que
se incrustó en la pared. Entonces David escapó y huyó en la noche.

\bibverse{11} Saúl envió algunos mensajeros a la casa de David para que
vigilaran y lo mataran por la mañana. Pero Mical, la mujer de David, le
advirtió: ``Si no te escapas esta noche, mañana te matarán''.

\bibverse{12} Mical bajó a David desde una ventana, y él salió
corriendo, logrando escapar. \bibverse{13} Luego tomó un ídolo de
casa\footnote{\textbf{19:13} ``Ídolo de casa'': la palabra hebrea
  utilizada aquí es teraphim y se mencionan por primera vez en Génesis
  31. Eran objetos de culto que se utilizaban para determinar la
  voluntad del ``dios'', véase Ezequiel 21:21; Zacarías 10:2. El hecho
  de que tales ídolos estuvieran allí en la casa de David muestra el
  grado en que la ``religión pura'' se había corrompido con el tiempo.}
y lo acostó en la cama, le puso una peluca de pelo de cabra en la cabeza
y lo cubrió con la ropa de cama.

\bibverse{14} Cuando Saúl envió a los mensajeros a detener a David,
Mical les dijo: ``Está enfermo''.

\bibverse{15} Saúl envió a los mensajeros a ver a David, diciendo:
``Tráiganmelo en la cama para que lo mate''. \bibverse{16} Pero cuando
los mensajeros entraron en el dormitorio, allí estaba el ídolo en la
cama con la peluca de pelo de cabra en la cabeza.

\bibverse{17} ``¿Por qué me has engañado así, ayudando a mi enemigo a
escaparse para que pueda huir?'' preguntó Saúl a Mical.

``Me dijo: '¡Apártate de mi camino! No quiero tener que matarte''.
respondió Mical.

\bibverse{18} Así fue como David se alejó y escapó. Fue a ver a Samuel
en Ramá y le explicó todo lo que Saúl le había hecho. Luego, él y Samuel
se fueron a hospedar en Naiot. \bibverse{19} Cuando Saúl se enteró de
que David estaba en Naiot, en Ramá, \bibverse{20} envió mensajeros para
arrestarlo. Pero cuando vieron a un grupo de profetas que profetizaban
con Samuel al frente, el Espíritu de Dios vino sobre los mensajeros de
Saúl y ellos también comenzaron a profetizar. \bibverse{21} Saúl fue
informado de lo que había sucedido, así que envió más mensajeros, y
ellos también comenzaron a profetizar. \bibverse{22} Por tercera vez
Saúl envió mensajeros, y ellos también comenzaron a profetizar.
\bibverse{23} Al final, Saúl fue él mismo a Ramá y llegó a la gran
cisterna de Secu. ``¿Dónde están Samuel y David?'', preguntó.

``En Naiot, en Ramá'', le dijeron.

Así que Saúl se dirigió a Naiot en Ramá, pero el Espíritu de Dios
incluso vino sobre él, y estuvo profetizando mientras caminaba hasta que
llegó a Naiot. \bibverse{24} Entonces Saúl también se quitó la ropa y
también profetizó en presencia de Samuel. Luego se postró y estuvo
desnudo todo ese día y toda esa noche. Por eso se dice: ``¿Es Saúl
también uno de los profetas?''

\hypertarget{section-19}{%
\section{20}\label{section-19}}

\bibverse{1} David corrió desde Naiot en Ramá hasta donde estaba Jonatán
y le preguntó: ``¿Qué he hecho? ¿Qué mal he hecho? ¿Qué cosa terrible le
he hecho a tu padre para que quiera matarme?''

\bibverse{2} ``¡Nada!'' Respondió Jonatán. ``¡No vas a morir! ¡Escucha!
Mi padre me cuenta todo lo que planea, sea lo que sea. ¿Por qué iba mi
padre a ocultarme algo así? No es cierto''.

\bibverse{3} Pero David volvió a jurar: ``Tu padre sabe muy bien que soy
tu amigo, y por eso seguro ha pensado: `Jonatán no puede enterarse de
esto, porque si no se enfadará mucho'. Te juro por la vida del Señor, y
por tu propia vida, que mi vida pende de un hilo.''\footnote{\textbf{20:3}
  ``Mi vida pende de un hilo'': literalmente, ``sólo hay un paso entre
  mí y la muerte.''}

\bibverse{4} ``Dime qué quieres que haga por ti y lo haré'', le dijo
Jonatán a David.

\bibverse{5} ``Bueno, la fiesta de la Luna Nueva es mañana, y tengo que
sentarme a comer con el rey. Pero si te parece bien, pienso ir a
esconderme en el campo hasta la noche de dentro de tres días.
\bibverse{6} Si tu padre me echa de menos, dile: `David ha tenido que
pedirme urgentemente permiso para bajar a Belén, su ciudad natal, a
causa de un sacrificio anual que se celebra allí para todo su grupo
familiar'. \bibverse{7} Si dice: `Está bien', entonces no hay problema
para mí, tu siervo, pero si se enfada, sabrás que pretende hacerme daño.
\bibverse{8} Así que, por favor, trátame bien, como prometiste cuando
hiciste un acuerdo conmigo ante el Señor. Si he hecho mal, ¡mátame tú
mismo! ¿Por qué me llevas a tu padre para que lo haga?''

\bibverse{9} ``¡De ninguna manera!'' respondió Jonatán. ``Si supiera con
certeza que mi padre tiene planes para hacerte daño, ¿no crees que te lo
diría?''

\bibverse{10} ``Entonces, ¿quién me va a avisar si tu padre te da una
respuesta desagradable?'' preguntó David.

\bibverse{11} ``Vamos, salgamos al campo'', dijo Jonatán. Así que ambos
salieron al campo.

\bibverse{12} Jonatán le dijo a David: ``Te prometo por el Señor, el
Dios de Israel, que mañana a esta hora o pasado mañana interrogaré a mi
padre. Si las cosas se ven bien para ti, te enviaré un mensaje y te lo
haré saber. \bibverse{13} Pero si mi padre planea hacerte daño, que el
Señor me castigue muy severamente, si no te lo hago saber enviándote un
mensaje para que puedas salir a salvo. Que el Señor esté contigo, como
lo estuvo con mi padre. \bibverse{14} Mientras viva, por favor,
demuéstrame un amor digno de confianza como el del Señor para que no
muera, \bibverse{15} y por favor, no retires tu amor fiel a mi familia,
aunque el Señor haya eliminado a todos tus enemigos de la tierra.''

\bibverse{16} Jonatán hizo un acuerdo solemne con la familia de David,
diciendo: ``Que el Señor imponga su castigo a los enemigos de
David.''\footnote{\textbf{20:16} Éste y los versos anteriores tienen una
  serie de problemas de traducción.} \bibverse{17} Jonatán se lo hizo
jurar a David una vez más, basándose en el amor que le profesaba, pues
Jonatán ya amaba a David como a sí mismo.

\bibverse{18} Entonces Jonatán le dijo a David: ``La fiesta de la Luna
Nueva es mañana. Se te echará de menos, porque tu lugar estará vacío.
\bibverse{19} Dentro de tres días, ve rápidamente al lugar donde te
escondiste cuando todo esto empezó, y quédate allí junto al montón de
piedras. \bibverse{20} Yo lanzaré tres flechas a su lado, como si
estuviera disparando a un blanco. \bibverse{21} Luego enviaré a un
muchacho y le diré: ``¡Ve a buscar las flechas! Si le digo
concretamente: ``Mira, las flechas están a este lado; tráelas aquí'',
entonces te juro por la vida del Señor que puedes salir sin peligro.
\bibverse{22} Pero si le digo al muchacho: ``Mira, las flechas están más
allá de ti'', entonces tendrás que salir, porque el Señor quiere que te
vayas. \bibverse{23} En cuanto a lo que tú y yo hablamos, recuerda que
el Señor es testigo entre tú y yo para siempre''.

\bibverse{24} Así que David se escondió en el campo. Cuando llegó la
fiesta de la Luna Nueva, el rey se sentó a comer. \bibverse{25} Se sentó
en su lugar habitual, junto al muro, frente a Jonatán. Abner se sentó
junto a Saúl, pero el lugar de David estaba vacío. \bibverse{26} Saúl no
dijo nada ese día porque pensó: ``Seguramente le ha pasado algo a David
que lo hace ceremonialmente impuro; sí, seguro está impuro''.

\bibverse{27} Pero el segundo día, el día después de la Luna Nueva, el
lugar de David seguía vacío. Saúl le preguntó a su hijo Jonatán: ``¿Por
qué el hijo de Isaí no ha venido a cenar ni ayer ni hoy?''.

\bibverse{28} Jonatán respondió: ``David tuvo que pedirme urgentemente
permiso para ir a Belén. \bibverse{29} Me dijo: ``Por favor, déjame ir,
porque nuestra familia va a celebrar un sacrificio en la ciudad y mi
hermano me dijo que tenía que estar allí. Si piensas bien de mí, por
favor, déjame ir a ver a mis hermanos'. Por eso se ausentó de la mesa
del rey''.

\bibverse{30} Saúl se enojó mucho con Jonatán y le dijo: ``¡Rebelde hijo
de puta! ¿Crees que no sé que prefieres al hijo de Isaí? ¡Qué vergüenza!
¡Eres una vergüenza para la madre que te dio a luz! \bibverse{31}
Mientras el hijo de Isaí siga vivo, tú y tu reinado no estarán seguros.
Ahora ve y tráemelo, porque tiene que morir''.

\bibverse{32} ``¿Por qué tiene que morir?'' preguntó Jonatán. ``¿Qué ha
hecho?''

\bibverse{33} Entonces Saúl lanzó su lanza contra Jonatán, tratando de
matarlo, por lo que supo que su padre definitivamente quería a David
muerto. \bibverse{34} Jonatán abandonó la mesa, y estaba absolutamente
furioso. No quiso comer nada en el segundo día de la fiesta, pues estaba
muy molesto por la forma vergonzosa en que su padre había tratado a
David.

\bibverse{35} Por la mañana, Jonatán fue al campo, al lugar que había
acordado con David, y un muchacho iba con él. \bibverse{36} Entonces le
dijo al muchacho: ``Corre y encuentra las flechas que yo tire''. De modo
que el muchacho comenzó a correr y Jonatán le disparó una flecha.
\bibverse{37} Cuando el muchacho llegó al lugar donde había caído la
flecha de Jonatán, éste le gritó: ``¿No ves que la flecha está más
adelante? \bibverse{38} ¡Apúrate! ¡Hazlo rápido! ¡No esperes!'' El
muchacho recogió las flechas y se las llevó a su amo. \bibverse{39} El
muchacho no sospechaba nada; sólo Jonatán y David sabían lo que
significaba. \bibverse{40} Jonatán le dio el arco y las flechas al
muchacho y le dijo: ``Llévatelas a la ciudad''.

\bibverse{41} Después de que el muchacho se había ido, David se levantó
de donde estaba, junto al montón de piedras, se tiró al suelo boca abajo
y se inclinó tres veces. Entonces él y Jonatán se besaron y lloraron
juntos como amigos, aunque David fue el que más lloró.

\bibverse{42} Jonatán le dijo a David: ``Vete en paz, porque los dos
hemos hecho un juramento solemne en nombre del Señor. Dijimos: `El Señor
será testigo entre tú y yo, y entre mis descendientes y los tuyos para
siempre'\,''. Entonces David se marchó, y Jonatán volvió a la ciudad.

\hypertarget{section-20}{%
\section{21}\label{section-20}}

\bibverse{1} David fue a la ciudad de Nob para ver al sacerdote
Ahimelec. Cuando se encontró con David, Ahimelec temblaba de miedo, y le
preguntó: ``¿Por qué estás aquí solo? ¿Por qué no hay nadie contigo?''.

\bibverse{2} ``El rey me ha dado un encargo'', respondió David. ``Me
dijo: `Nadie debe saber nada de la misión que te he enviado a cumplir'.
En cuanto a mis hombres, les he dicho dónde encontrarme. \bibverse{3}
¿Qué tienes a la mano para comer? Dame cinco panes, o lo que puedas
encontrar''.

\bibverse{4} ``No hay pan ordinario'', le dijo el sacerdote a David,
``pero hay pan sagrado, siempre que tus hombres no se hayan acostado con
ninguna mujer últimamente.''

\bibverse{5} ``No nos hemos acostado con ninguna mujer'', respondió
David. ``De hecho, esa es la norma cuando dirijo las tropas en misión.
Se mantienen puros incluso durante las misiones ordinarias, y con mayor
razón en este momento''.

\bibverse{6} Entonces el sacerdote le dio el pan sagrado, ya que allí no
tenían otro pan que el ``Pan de la Presencia'', que había sido retirado
de la presencia del Señor\footnote{\textbf{21:6} En otras palabras,
  colocado en la Tienda de la Reunión.} ese día y lo sustituyeron por
pan fresco.

\bibverse{7} Uno de los siervos de Saúl estaba allí ese día, tratando de
enmendarse\footnote{\textbf{21:7} ``Enmendarse'': literalmente
  ``detenerse''. Parece que Doeg estaba ofreciendo un sacrificio por
  algún pecado que había cometido y que el sacerdote Ahimelec conocía.
  Esta parece ser una de las razones por las que Doeg delató a David
  (22:9) y ejecutó la orden de Saúl de matar a Ahimelec y a los demás
  sacerdotes.} con el Señor. Era Doeg el edomita, el pastor principal de
Saúl.

\bibverse{8} ``¿Tienes aquí una lanza o una espada?'' le preguntó David
a Ahimelec. ``No traje mi espada ni ninguna de mis armas, porque lo que
el rey necesitaba que hiciera era urgente''.

\bibverse{9} Entonces el sacerdote respondió: ``Tengo aquí la espada de
Goliat, el filisteo que mataste en el Valle de Ela. Está envuelta en un
paño detrás del efod. Puedes cogerla si quieres. Es el único que hay
aquí''.

``¡Es mejor que cualquier otra espada! Por favor, dámela'', respondió
David.

\bibverse{10} Ese día David huyó de Saúl y se dirigió a Aquis, rey de
Gat.\footnote{\textbf{21:10} Gat era una ciudad filistea.}

\bibverse{11} Pero los oficiales de Aquis preguntaron al rey: ``¿No es
éste David, el rey de ese país? ¿No cantaban sobre él en sus danzas:
`Saúl ha matado a sus miles, y David a sus decenas de miles'\,''?

\bibverse{12} David escuchó atentamente lo que decían y esto le hizo
temer mucho a Aquis, el rey de Gat. \bibverse{13} Así que cambió su
forma de actuar con ellos y se hizo el loco. Hizo marcas en las puertas
de la ciudad y dejó que su saliva corriera por su barba. \bibverse{14}
Aquis les dijo a sus oficiales: ``¡Como ven, ese hombre está
completamente loco! ¿Por qué me lo han traído? \bibverse{15} ¿Acaso
necesito más locos para que me traigan a este hombre y que se vuelva
loco delante de mí? ¿Creen que voy a dejar que entre en mi casa?''

\hypertarget{section-21}{%
\section{22}\label{section-21}}

\bibverse{1} Después David escapó y se fue a la cueva de Adulam. Cuando
se enteraron de dónde estaba, sus hermanos y todo el resto de su familia
fueron y se reunieron con él allí. \bibverse{2} Todos los que tenían
problemas o deudas o estaban resentidos también acudieron a él y se
convirtió en su líder. Ahora tenía unos cuatrocientos hombres con él.

\bibverse{3} Luego David se fue a Mizpa, en el país de Moab. Le pidió al
rey de Moab: ``Por favor, deja que mi padre y mi madre vengan y se
queden contigo hasta que averigüe lo que Dios planea para mí''.
\bibverse{4} Así que los dejó con el rey de Moab, y se quedaron con el
rey todo el tiempo que David vivió en la fortaleza.\footnote{\textbf{22:4}
  ``Fortaleza'': probablemente refiriéndose a su campamento en la cueva
  de Adulam.}

\bibverse{5} Pero entonces el profeta Gad le dijo a David: ``No te
quedes en la fortaleza. Vuelve a la tierra de Judá''. Así que David se
marchó y se dirigió al bosque de Haret.

\bibverse{6} Saúl se enteró de que David había regresado y de dónde
estaba. Saúl estaba sentado bajo el tamarisco en la colina de Guibeá.
Tenía su lanza en la mano, con todos sus oficiales rodeándolo.

\bibverse{7} Entonces Saúl les dijo: ``¡Escúchenme, hombres de Benjamín!
¿Acaso el hijo de Isaí les va a dar a todos ustedes campos y viñedos y
los va a hacer comandantes y oficiales del ejército? \bibverse{8} ¿Es
por eso que todos ustedes han conspirado contra mí? Ni uno solo de
ustedes me dijo que mi propio hijo había hecho un acuerdo con el hijo de
Isaí. Ni uno solo de ustedes ha demostrado que se preocupa por mí, ni me
ha explicado que mi hijo lo ha animado para que intente matarme. ¡Eso es
lo que está haciendo ahora!''

\bibverse{9} Doeg el edomita, que estaba allí con los oficiales de Saúl,
habló diciendo: ``Vi al hijo de Isaí visitar a Ahimelec, hijo de Ahitob,
en Nob. \bibverse{10} Ahimelec pidió consejo al Señor para él y le dio
comida. También le dio la espada de Goliat el filisteo''.

\bibverse{11} El rey envió un mensaje para convocar al sacerdote
Ahimelec, hijo de Ahitob, y a toda su familia, que eran sacerdotes en
Nob. Todos ellos acudieron al rey.

\bibverse{12} ``Ahora escucha, hijo de Ahitob'', le gritó el rey.

``¿Qué pasa, mi señor?'' preguntó Ahimelec.

\bibverse{13} ``¿Por qué tú y el hijo de Isaí han conspirado contra mí?
Le diste pan y una espada, y le pediste consejo a Dios para que se
rebelara contra mí y tratara de matarme, ¡que es lo que está haciendo
ahora!''

\bibverse{14} ``¿Quién de todos tus oficiales es tan confiable como
David, el yerno del rey? ¡Él está a cargo de su escolta, y es muy
respetado en su familia!'' respondió Ahimelec. \bibverse{15} ``¿Y fue
ese día la primera vez que pidió consejo a Dios en su favor? ¡Por
supuesto que no! El rey no debe acusarme a mí, tu siervo, ni a nadie de
mi familia, pues yo no sabía nada de todo esto''.

\bibverse{16} ``¡Vas a morir por esto!'', declaró el rey. ``¡Tú y toda
tu familia!''

\bibverse{17} Entonces el rey se dirigió a sus guardaespaldas que
estaban allí y les ordenó: ``¡Maten a estos sacerdotes del Señor, porque
están del lado de David! Sabían que era un fugitivo y, sin embargo, no
me lo dijeron''. Pero los guardias del rey se negaron a atacar a los
sacerdotes del Señor.

\bibverse{18} Entonces el rey le ordenó a Doeg: ``¡Mata tú a los
sacerdotes!'' Doeg el edomita atacó y mató a los sacerdotes, matando a
ochenta y cinco hombres que llevaban puesta su ropa sacerdotal.
\bibverse{19} Luego se dirigió a Nob, la ciudad de los sacerdotes, y
mató a sus hombres y mujeres, niños y bebés, ganado, asnos y ovejas.

\bibverse{20} Pero uno de los hijos de Ahimelec, hijo de Ahitob, logró
escapar. Se llamaba Abiatar, y huyó y se unió a David. \bibverse{21} Le
dijo a David que Saúl había matado a los sacerdotes del Señor.
\bibverse{22} Entonces David le dijo a Abiatar: ``Yo sabía que ese día,
cuando Doeg el edomita estaba allí, iba a contárselo a Saúl. Es mi culpa
que toda tu familia haya muerto. \bibverse{23} Pero puedes quedarte
conmigo y no debes tener miedo, porque el hombre que quiere matarte
también quiere matarme a mí. Yo cuidaré bien de ti.''

\hypertarget{section-22}{%
\section{23}\label{section-22}}

\bibverse{1} Un día David escuchó la noticia: ``Los filisteos están
atacando Keila y están robando el grano de las eras''. \bibverse{2}
Entonces David le pidió consejo al Señor: ``¿Debo ir a atacar a esos
filisteos?''.

Y el Señor le dijo a David: ``Ve y ataca a los filisteos y salva a
Keila''.

\bibverse{3} Pero los hombres de David le dijeron: ``Incluso aquí en
Judá sentimos miedo. Si fuéramos a Keila a luchar contra los ejércitos
filisteos, ¡estaríamos absolutamente aterrorizados!''

\bibverse{4} Entonces David volvió a pedir consejo al Señor, y éste le
dijo: ``Ve inmediatamente a Keila, porque te daré la victoria sobre los
filisteos.''

\bibverse{5} Entonces David y sus hombres fueron a Keila y lucharon
contra los filisteos. Los mataron y expulsaron su ganado. De esta manera
David salvó al pueblo de Keila. \bibverse{6} (Cuando Abiatar, hijo de
Ahimelec, huyó hacia David en Keila, llevó consigo el efod).

\bibverse{7} Cuando Saúl se enteró de que David había ido a Keila, dijo:
``Dios me lo ha entregado, porque se ha encerrado en una ciudad con
puertas que se pueden cerrar con barrotes.'' \bibverse{8} Entonces Saúl
convocó a todo su ejército para ir a atacar a Keila y sitiar a David y a
sus hombres.

\bibverse{9} Cuando David se enteró de que Saúl estaba tramando
atacarlo, le pidió al sacerdote Abiatar: ``Por favor, trae el efod''.

\bibverse{10} David oró: ``Señor, Dios de Israel, a mí, tu siervo, me
han dicho claramente que Saúl planea venir a Keila y destruir la ciudad
por mi culpa. \bibverse{11} ¿Van a entregarme los jefes de la ciudad de
Keila? ¿Va a venir Saúl, como he oído? Señor, Dios de Israel, por favor,
dímelo''.

``Sí, vendrá'', respondió el Señor.

\bibverse{12} ``¿Y los jefes de la ciudad de Keila me entregarán a mí y
a mis hombres a Saúl?'' preguntó David.

``Sí, lo harán'', respondió el Señor.

\bibverse{13} Así que David y sus hombres, que eran unos seiscientos,
salieron de Keila y se desplazaron de un lugar a otro. Cuando Saúl
descubrió que David había escapado de Keila, no se molestó en ir allí.
14 David acampó en las fortalezas del desierto, quedándose en las
montañas del desierto de Zif. Saúl lo buscó continuamente, pero Dios no
permitió que David fuera capturado.

\bibverse{15} Mientras David se alojaba en Horesh, en el desierto de
Zif, descubrió\footnote{\textbf{23:15} ``Descubrió'': o ``temió.''} que
Saúl iba a matarlo. 16 El hijo de Saúl, Jonatán, fue a ver a David a
Horesh y lo animó a seguir confiando en Dios, diciéndole: \bibverse{17}
``No te preocupes, porque mi padre Saúl nunca te va a encontrar. Vas a
ser rey de Israel y yo seré tu sustituto. Hasta mi padre Saúl lo sabe''.
\bibverse{18} Los dos hicieron un acuerdo ante el Señor. David se quedó
en Horesh y Jonatán se fue a su casa.

\bibverse{19} Entonces los hombres de Zif fueron a ver a Saúl a Guibeá y
le dijeron: ``David se esconde en nuestra zona, en las fortalezas de
Hores, en la colina de Haquila, en los páramos del sur. 20 Así que, Su
Majestad, venga cuando quiera, y nos aseguraremos de entregárselo''.

\bibverse{21} Y Saúl le respondió: ``Que el Señor te bendiga por pensar
en mí. \bibverse{22} Por favor, ve y asegúrate de saber exactamente
dónde está -- dónde se hospeda y quién lo ha visitado -- porque la gente
me dice que es muy taimado. \bibverse{23} Busca y anota todos sus
escondites. Luego vuelve a mí cuando estés seguro, y yo volveré contigo.
Si está aquí en el campo, lo cazaré entre todo el pueblo de Judá''.

\bibverse{24} Así que los hombres de Zif se pusieron en marcha,
regresando a Zif por delante de Saúl. David y sus hombres estaban en el
desierto de Maón, en el valle de Araba\footnote{\textbf{23:24} ``El
  valle de Araba'': otro nombre para el Valle del Jordán.} en los
páramos del sur. \bibverse{25} Saúl y sus hombres comenzaron a buscarlo.
Cuando David se enteró, bajó a la roca y se quedó en el desierto de
Maón. Y cuando Saúl se enteró, persiguió a David en el desierto de Maón.

\bibverse{26} Saúl iba por un lado de la montaña, mientras que David y
sus hombres iban por el otro lado, apurando la marcha. Pero justo cuando
Saúl y sus hombres se acercaban a David y a los suyos, a punto de
capturarlos, \bibverse{27} llegó un mensajero para decirle a Saúl:
``¡Ven de inmediato! Los filisteos han invadido el país''.

\bibverse{28} Así que Saúl tuvo que dejar de perseguir a David y fue a
enfrentarse a los filisteos. Por eso el lugar se llama ``Roca de la
Fuga''. \bibverse{29} Entonces David partió y se fue a vivir a las
fortalezas de En-gadi.

\hypertarget{section-23}{%
\section{24}\label{section-23}}

\bibverse{1} Cuando Saúl volvió de perseguir a los filisteos, le
informaron: ``David está en el desierto de En-gadi''. \bibverse{2} Así
que Saúl tomó tres mil hombres especialmente escogidos de todo Israel y
fue a buscar a David y a sus hombres en los alrededores de las Rocas de
las Cabras Salvajes. \bibverse{3} Cuando Saúl pasó por los corrales de
las ovejas en el camino, había una cueva, y entró a hacer sus
necesidades. David y sus hombres estaban escondidos en lo profundo de la
cueva.

\bibverse{4} Los hombres de David le dijeron: ``Hoy es el día que el
Señor te prometió al decirte: `Escucha, voy a entregarte a tu enemigo,
para que hagas con él lo que quieras'\,''. Entonces David se acercó
sigilosamente y cortó un trozo del borde del manto de Saúl.

\bibverse{5} Pero después David se sintió muy mal porque había cortado
un trozo del manto de Saúl. \bibverse{6} Y les dijo a sus hombres: ``Que
el Señor me impida hacer algo así\footnote{\textbf{24:6} ``Algo así'':
  probablemente refiriéndose al deseo de sus hombres de atacar al rey.}
a mi amo, el ungido del Señor. Nunca lo atacaré, porque es el ungido del
Señor''. \bibverse{7} Y reprendió a sus hombres, y no les permitió
atacar a Saúl.

Saúl se levantó y siguió su camino. \bibverse{8} Un poco más tarde,
David salió de la cueva y gritó: ``¡Mi amo el rey!''. Cuando Saúl miró a
su alrededor, David se inclinó con el rostro hacia el suelo.

\bibverse{9} ``¿Por qué haces caso a la gente que dice que yo quiero
hacerte daño''? preguntó David. \bibverse{10} ``¡Sólo mira! Hoy has
visto con tus propios ojos que el Señor te entregó a mí en la cueva.
Algunos me instaron a matarte, pero yo te mostré compasivo y dije: `Me
niego a atacar a mi amo, porque es el ungido del Señor'. 11 ¡Mira, padre
mío! ¿Ves este pedazo de tu túnica que estoy sosteniendo? Sí, te lo he
cortado, pero no te he matado. Ahora puedes verlo por ti mismo y puedes
estar seguro de que no he hecho nada malo ni rebelde. No he pecado
contra ti, pero tú me persigues, tratando de matarme.

\bibverse{12} Que el Señor decida entre tú y yo quién de los dos tiene
razón, y que el Señor te castigue, pero yo nunca intentaré hacerte daño.
\bibverse{13} Como dice el viejo refrán: ``Del malvado salen actos
malvados'', pero yo nunca trataré de hacerte daño. \bibverse{14} ¿A
quién persigue el rey de Israel? ¿A quién persigue? ¡A un perro muerto!
¡Sólo una pulga! \bibverse{15} Que el Señor decida y elija entre tú y
yo. Que preste atención a mi caso y lo apoye; que me salve de ti''.

\bibverse{16} Cuando David terminó de decir esto, Saúl preguntó: ``¿Eres
tú el que habla, David, hijo mío?'', y lloró en voz alta. \bibverse{17}
Entonces le dijo a David: ``Tú eres mejor persona que yo, porque me has
pagado con el bien, pero yo te he pagado con el mal. \bibverse{18} Hoy
has demostrado lo bien que me has tratado, pues cuando el Señor me
entregó a ti, no me mataste. \bibverse{19} Porque si un hombre agarrara
a su enemigo, ¿lo dejaría escapar ileso? ¡Que el Señor te recompense
bien por cómo me has tratado hoy! \bibverse{20} Escucha, sé que
definitivamente serás rey, y tu gobierno sobre el reino de Israel será
seguro. \bibverse{21} Ahora júrame por el Señor que no destruirás a mis
descendientes que me siguen y que no borrarás mi nombre de mi linaje.''

\bibverse{22} Así que David le prometió esto a Saúl con un juramento.
Entonces Saúl regresó a su casa, pero David y sus hombres volvieron a la
fortaleza.

\hypertarget{section-24}{%
\section{25}\label{section-24}}

\bibverse{1} Samuel murió. Todos en Israel se reunieron para llorar por
él, y lo enterraron en su casa de Ramá. David partió y se fue al
desierto de Parán.

\bibverse{2} Había un hombre de Maón que era muy rico. Tenía propiedades
en el Carmelo y poseía mil cabras y tres mil ovejas. Estaba en el
Carmelo esquilando las ovejas. \bibverse{3} El hombre se llamaba
Nabal,\footnote{\textbf{25:3} ``Nabal'' significa ``tonto.''} y su
esposa se llamaba Abigail. Era una mujer sabia y hermosa, pero su marido
era cruel y trataba mal a la gente. Era descendiente de Caleb.
\bibverse{4} David estaba en el desierto y se enteró de que Nabal estaba
esquilando ovejas. \bibverse{5} Entonces David envió a diez de sus
jóvenes y les dijo: ``Vayan a ver a Nabal al Carmelo. Salúdenlo en mi
nombre y salúdenlo de mi parte. \bibverse{6} Díganle: '¡Te deseo una
larga vida! Paz a ti y a tu familia, y que todo lo que hagas prospere.
\bibverse{7} Me he enterado de que estás ocupado esquilando. Cuando tus
pastores estuvieron con nosotros, no los maltratamos, y nada de lo que
les pertenecía fue robado en todo el tiempo que estuvieron en el
Carmelo. \bibverse{8} Pregúntales a tus hombres y ellos te lo
confirmarán. Por favor, sean amables con mis hombres, sobre todo porque
hemos venido en este día de fiesta. Por favor, danos la comida que
puedas a nosotros y a tu buen amigo David''.

\bibverse{9} Los jóvenes de David llegaron, le dieron a Nabal este
mensaje de parte de David y esperaron su respuesta.

10 ``¿Quién se cree ese `David, hijo de Isaí'\,''? respondió Nabal.
``¡Hoy en día hay muchos siervos que huyen de sus amos! \bibverse{11}
¿Por qué habría de tomar el pan y el agua que he suministrado, y la
carne que he sacrificado para mis esquiladores, y entregárselos a estos
extraños? ¡Ni siquiera sé de dónde son!''.

\bibverse{12} Así que los hombres de David se dieron la vuelta y
regresaron por donde habían venido. Cuando regresaron, le contaron a
David todo lo que Nabal había dicho.

\bibverse{13} ``¡Todos, tomen las espadas!'' ordenó David. Y todos se
pusieron las espadas, y David también lo hizo. Unos cuatrocientos
hombres siguieron a David, mientras que doscientos se quedaron atrás
para custodiar sus pertrechos.

\bibverse{14} Mientras tanto, uno de los hombres de Nabal le dijo a
Abigail, la esposa de Nabal: ``David envió a unos mensajeros del
desierto para que le trajeran saludos a nuestro amo, pero él sólo los
insultó. \bibverse{15} Los hombres de David siempre fueron muy buenos
con nosotros y nunca nos maltrataron. Todo el tiempo que estuvimos en el
campo con ellos no nos robaron nada. \bibverse{16} Fueron como un muro
protector para nosotros, tanto de día como de noche, durante todo el
tiempo que estuvimos con ellos cuidando las ovejas. \bibverse{17} Debes
saber lo que ha pasado y pensar en lo que debes hacer al respecto. El
desastre está a punto de golpear a nuestro amo y a toda su familia,
¡pero es tan odioso que nadie puede hacerlo entrar en razón!''

\bibverse{18} Abigail recolectó rápidamente doscientos panes, dos cueros
de vino, cinco ovejas ya sacrificadas, cinco seahs de grano tostado,
cien tortas de pasas y doscientas tortas de higos, y luego cargó todo en
los asnos. \bibverse{19} Entonces les dijo a sus hombres: ``Vayan
ustedes adelante. Yo los seguiré''. Pero no le dijo nada a su marido
Nabal.

\bibverse{20} Mientras Abigail montaba en su burro por un valle de la
montaña, vio que David y sus hombres bajaban hacia ella, y les salió al
encuentro. \bibverse{21} David acababa de quejarse: ``¡De nada sirvió
proteger las pertenencias de este hombre en el desierto! No le han
robado nada en absoluto y, sin embargo, ¿qué hace? ¡Me devuelve mal por
bien! \bibverse{22} ¡Que Dios me castigue muy severamente si dejo vivo a
uno solo de sus hombres para la mañana!''

\bibverse{23} Cuando Abigail vio a David, se bajó rápidamente del asno y
se inclinó ante él, con el rostro en el suelo. \bibverse{24} Cayendo a
sus pies en señal de respeto, le dijo: ``Señor, acepto toda la
responsabilidad por lo que ha sucedido. Por favor, escuche lo que yo, su
sierva, tengo que decir. \bibverse{25} Por favor, no te inquietes por
ese despreciable de Nabal. Su nombre significa ``tonto'', y él es
realmente tonto. En cuanto a mí, tu siervo, ni siquiera vi a los hombres
que enviaste.

\bibverse{26} Ahora, señor, vive el Señor y vives tú, el Señor te ha
impedido derramar sangre y tomar tu propia venganza. Señor, que tus
enemigos y los que quieren hacerte daño sean como Nabal. \bibverse{27}
Te ruego que aceptes este presente que yo, tu sierva, te he traído,
señor, y se lo des a tus hombres. \bibverse{28} Por favor, perdona
cualquier ofensa que yo, tu sierva, haya cometido, porque el Señor está
seguro de establecer una dinastía para ti que durará mucho tiempo,
porque tú, señor, peleas las batallas del Señor. La maldad no debe
encontrarse en ti mientras vivas.\footnote{\textbf{25:28} Tal vez
  Abigail está sugiriendo que la misión actual de David no está
  sancionada por Dios y que seguir con ella sería comprometer su
  reputación, especialmente como futuro rey de Israel.} \bibverse{29} Si
alguien te persigue y trata de matarte, tu vida quedará ligada a los que
el Señor, tu Dios, cuida, a salvo en su cuidado. Pero él tirará las
vidas de tus enemigos como piedras de una honda. \bibverse{30} Así que
cuando el Señor haya hecho por ti, señor, todo el bien que te prometió,
y te haya hecho gobernar sobre Israel, 31 no tendrás sentimientos de
remordimiento ni conciencia culpable por el derramamiento innecesario de
sangre ni por tomar tu propia venganza. Y cuando el Señor haya hecho
estas cosas buenas por ti, señor, por favor acuérdate de mí, tu
sierva''.

\bibverse{32} Entonces David le dijo a Abigail: ``¡Alabado sea el Señor,
el Dios de Israel, que te ha enviado hoy a mi encuentro! \bibverse{33}
Que seas recompensada por tus sabias decisiones, por haber evitado que
hoy derramara sangre y me vengara. \bibverse{34} Por el contrario, vive
el Señor, el Dios de Israel, que me ha impedido hacerte daño, si no
hubieras salido corriendo a mi encuentro, definitivamente no habría
quedado vivo ni uno solo de los hombres de Nabal al amanecer.''

\bibverse{35} David aceptó de Abigail lo que le había traído y le dijo:
``Puedes irte a casa en paz, porque estoy de acuerdo con tu consejo y te
concedo tu petición.''

\bibverse{36} Cuando Abigail volvió a casa de Nabal, éste estaba en la
casa, de fiesta como un rey. Se sentía muy alegre y estaba muy borracho.
Así que ella no le dijo nada hasta la mañana. \bibverse{37} A la mañana
siguiente, cuando Nabal estaba sobrio, su mujer le contó lo que había
sucedido. Cuando él la escuchó, le dio un ataque al corazón y se quedó
paralizado.\footnote{\textbf{25:37} ``Quedó paralizado'': literalmente,
  ``estaba como una piedra.''} \bibverse{38} Unos diez días después, el
Señor abatió a Nabal y éste murió.

\bibverse{39} Cuando David se enteró de que Nabal había muerto, dijo:
``Alabado sea el Señor, que me ha apoyado contra la injuria de Nabal y
me ha impedido hacer el mal. Porque el Señor hizo que la maldad de Nabal
recayera sobre él''. Entonces David envió un mensaje a Abigail,
pidiéndole que se casara con él.

\bibverse{40} Cuando los hombres de David llegaron al Carmelo, le
dijeron a Abigail: ``David nos ha enviado a traerte para que seas su
esposa.''

\bibverse{41} Ella se levantó, se inclinó y dijo: ``Soy la sierva de
David. Estoy dispuesta a servir y a lavar los pies de los siervos de mi
señor''. \bibverse{42} Abigail subió rápidamente a un burro y, con sus
cinco sirvientas, regresó con los hombres de David y se convirtió en su
esposa. \bibverse{43} David también se había casado con Ahinoam de
Jezreel. Así que ambas fueron sus esposas. \bibverse{44} Sin embargo,
Saúl había dado a su hija Mical, esposa de David, a Paltiel, hijo de
Laish. Él era de Galim.

\hypertarget{section-25}{%
\section{26}\label{section-25}}

\bibverse{1} El pueblo de Zif fue a ver a Saúl a Guibeá y le dijeron:
``David se esconde en la colina de Haquilá, frente a los páramos''.

\bibverse{2} Así que Saúl se dirigió al desierto de Zif junto con tres
mil hombres de Israel especialmente escogidos para buscar a David allí.
\bibverse{3} Saúl acampó junto al camino en la colina de Haquilá, frente
a los páramos, cerca de donde David vivía en el desierto. Cuando se dio
cuenta de que Saúl había ido a buscarlo allí, \bibverse{4} envió espías
y descubrió que Saúl había llegado definitivamente.

\bibverse{5} Una noche\footnote{\textbf{26:5} ``Una noche'': implícito.},
David se levantó y fue al campamento de Saúl y vio dónde dormía éste,
junto con Abner, hijo de Ner, el comandante del ejército. Saúl estaba
acostado en medio del campamento, con sus hombres a su alrededor.
\bibverse{6} David les preguntó a Ahimelec el hitita y a Abisai, hijo de
Sarvia,\footnote{\textbf{26:6} Servia era hermana de David y madre de
  Joab, Abisai y Asahel.} hermano de Joab: ``¿Quién quiere acompañarme
al campamento a ver a Saúl?

``Iré contigo'', respondió Abisai.

\bibverse{7} Así que David y Abisai fueron al campamento del ejército
por la noche. Saúl estaba durmiendo en el campamento con su lanza
clavada en el suelo junto a su cabeza, y Abner y sus hombres dormían a
su alrededor.

\bibverse{8} Abisai le dijo a David: ``Dios te ha entregado hoy a tu
enemigo. Así que, por favor, déjame clavarle la lanza en el suelo de una
sola vez. No necesitaré hacerlo dos veces''.

\bibverse{9} Pero David le dijo a Abisai: ``¡No, no lo mates! ¿Quién
puede atacar al ungido del Señor y no ser culpable de un crimen?
\bibverse{10} Vive el Señor, el Señor mismo lo matará. O le llegará su
hora y morirá, o irá a la batalla y lo matarán. 11 Que el Señor me
impida atacar al ungido del Señor. Recoge la lanza y el cántaro de agua
junto a su cabeza, y vámonos''.

\bibverse{12} David tomó la lanza y la jarra de agua junto a la cabeza
de Saúl, y se fueron. Nadie vio nada; nadie supo lo que había pasado;
nadie se despertó. Todos se quedaron dormidos, porque el Señor los había
hecho caer en un profundo sueño.

\bibverse{13} Entonces David volvió al otro lado, y se situó en la cima
de la colina, lo suficientemente lejos -había una distancia considerable
entre ellos. \bibverse{14} Gritó al ejército y a Abner, hijo de Ner:
``¿No vas a responderme, Abner?''.

``¿Quién es el que grita, molestando al rey?'' respondió Abner.

\bibverse{15} David llamó a Abner: ``¿No estás destinado a ser ese gran
hombre? ¿Hay alguien en Israel que sea mejor que tú? ¿Por qué no
protegiste a tu amo el rey cuando alguien vino a matarlo? \bibverse{16}
No has hecho nada bien. Vive el Señor, que todos ustedes merecen morir,
porque no protegieron a su amo, el ungido del Señor. Miren a su
alrededor. ¿Dónde están la lanza y el cántaro del rey que estaban junto
a su cabeza?''

\bibverse{17} Saúl reconoció la voz de David y preguntó: ``¿Eres tú
quien habla, David, hijo mío?''

``Sí, soy yo, mi señor y rey'', respondió David.

\bibverse{18} ``¿Por qué me persigue mi señor, su siervo? ¿Qué es lo que
he hecho? ¿De qué crimen soy culpable?'', continuó. \bibverse{19} ``Por
favor, escúchame, mi señor y rey. Si el Señor se ha enfadado conmigo,
que se alegre de aceptar una ofrenda. Pero si son los hombres los que lo
han hecho, ¡que sean malditos ante el Señor! Durante todo este tiempo me
han expulsado de vivir entre el pueblo elegido por Dios, diciéndome:
`Vete y adora a otros dioses'. 20 Por favor, no me dejes morir tan lejos
de la presencia del Señor. El rey de Israel ha venido a perseguir una
pequeña pulga, cazándome como quien caza una perdiz en el monte.''

\bibverse{21} ``He hecho mal'', respondió Saúl, ``vuelve, David, hijo
mío. No volveré a intentar hacerte daño, porque hoy me has valorado y me
has perdonado la vida. ¡He sido tan estúpido! He cometido un gran
error''.

\bibverse{22} ``Tengo aquí la lanza del rey'', dijo David. ``Envía a uno
de tus hombres a recogerla. \bibverse{23} El Señor recompensa a todos
los que hacen lo correcto y son fieles. El Señor me ha entregado hoy a
ti, pero me he negado a dañar al ungido del Señor. \bibverse{24} De la
misma manera que hoy he valorado tu vida, que el Señor valore la mía y
me rescate de todos mis problemas.''

\bibverse{25} Saúl entonces le dijo a David: ``Que seas bendecido,
David, hijo mío. Lograrás muchas cosas y siempre tendrás éxito''. Y
David se fue, y Saúl volvió a su casa.

\hypertarget{section-26}{%
\section{27}\label{section-26}}

\bibverse{1} Pero David pensó para sí mismo: ``Un día de estos Saúl va a
atraparme. Creo que será mejor que huya a la tierra de los filisteos.
Así Saúl dejará de buscarme por todo Israel y no me atrapará''.

\bibverse{2} Así que David y los seiscientos hombres que lo acompañaban
se pusieron en marcha, cruzaron la frontera y se dirigieron a Aquis,
hijo de Maoc, el rey de Gat. \bibverse{3} David y sus hombres se
instalaron con Aquis en Gat. Todos los hombres tenían a sus familias con
ellos, y David tenía a sus dos esposas, Ahinoam de Jezreel y Abigail del
Carmelo, la viuda de Nabal. \bibverse{4} Cuando Saúl se enteró de que
David había huido a Gat, no siguió buscándolo.

\bibverse{5} David le dijo a Aquis: ``Por favor, hazme un favor:
asígname un lugar en una de las ciudades del campo para que pueda vivir
allí. Yo, tu siervo, no merezco vivir en la ciudad real contigo''.

\bibverse{6} Aquis le dio de inmediato Siclag, y la ciudad sigue
perteneciendo a los reyes de Judá hasta el día de hoy. \bibverse{7} Y
David vivió en el país de los filisteos durante un año y cuatro meses.

\bibverse{8} Durante ese tiempo, David y sus hombres hicieron
incursiones contra los guesuritas, los girzitas y los amalecitas. Estos
pueblos habían vivido en la tierra hasta Sur y Egipto desde tiempos
antiguos. \bibverse{9} Cuando David atacaba un lugar, no dejaba a nadie
con vida. Tomaba los rebaños y las manadas, los asnos, los camellos y la
ropa. Luego regresaba a Aquis. \bibverse{10} Cuando Aquis le preguntaba:
``¿Dónde has estado asaltando hoy?'' David respondía: ``En el
desierto\footnote{\textbf{27:10} ``Desierto'', literalmente ``el
  Negev'', la región árida del sur.} de Judá'', o ``el desierto de
Jerameel'', o ``el desierto de los ceneos''.

\bibverse{11} David no dejó a nadie con vida que pudiera ir a Gat porque
pensó: ``Podrían delatarnos y decir: `David hizo esto'\,''. Así hizo
todo el tiempo que vivió en el país de los filisteos. \bibverse{12}
Aquis confió en David y pensaba: ``Se ha hecho tan ofensivo para su
pueblo, los israelitas, que tendrá que servirme para siempre.''

\hypertarget{section-27}{%
\section{28}\label{section-27}}

\bibverse{1} Por aquel entonces, los filisteos convocaron a sus
ejércitos para ir a la guerra contra Israel. Entonces Aquis le dijo a
David: ``Esperamos que tú y tus hombres me acompañen como parte del
ejército''.

\bibverse{2} ``¡Está bien!'' respondió David. ``Entonces tú mismo
descubrirás lo que yo, tu siervo, puedo hacer''.

``Eso también está bien'', respondió Aquis. ``Te haré mi guardaespaldas
de por vida''.

\bibverse{3} Para entonces Samuel había muerto, y todo Israel lo había
llorado y enterrado en Ramá, su ciudad natal. Saúl se había deshecho de
los médiums y espiritistas del país.

\bibverse{4} Los ejércitos filisteos se reunieron y acamparon en Sunem.
Saúl convocó a todo el ejército israelita y acampó en Gilboa.
\bibverse{5} Cuando Saúl vio al ejército filisteo, se aterrorizó y
tembló de miedo. \bibverse{6} Pidió consejo al Señor, pero éste no le
respondió ni por sueños, ni por Urim, ni por profetas. \bibverse{7}
Entonces Saúl les dijo a sus oficiales: ``Búsquenme una mujer que sea
médium para que pueda ir a pedirle consejo''.

``Hay una mujer que es médium en Endor'', respondieron sus oficiales.

\bibverse{8} Saúl se disfrazó vistiendo ropas diferentes. Fue con dos de
sus hombres a ver a la mujer por la noche. Saúl le dijo: ``Tráeme un
espíritu para que pueda hacer algunas preguntas. Te daré el nombre''.

\bibverse{9} ``¿No sabes lo que ha hecho Saúl?'', respondió ella. ``Se
ha deshecho de los médiums y espiritistas del país. ¿Intenta tenderme
una trampa y hacer que me maten?''

\bibverse{10} Saúl le hizo un juramento por el Señor: ``Vive el Señor,
no serás considerada culpable por hacer esto''.

\bibverse{11} ``¿A quién quieres que traiga para ti?'', preguntó la
mujer.

``Trae a Samuel'', respondió él.

\bibverse{12} Pero cuando la mujer vio a Samuel, gritó con fuerza y le
dijo a Saúl: ``¿Por qué me has engañado? ¡Tú eres Saúl!''

\bibverse{13} ``No te asustes'', le dijo el rey. ``¿Qué ves?''

``Veo un dios que sale de la tierra'', respondió la mujer.

\bibverse{14} ``¿Qué aspecto tiene?'' preguntó Saúl. ``Un anciano está
subiendo'', respondió ella. ``Saúl pensó que debía ser Samuel y se
inclinó hacia abajo en señal de respeto.

\bibverse{15} Entonces Samuel le dijo a Saúl: ``¿Por qué me molestas
haciéndome subir?''.

``Estoy en un problema terrible'', respondió Saúl. ``Los filisteos me
atacan, y Dios me ha abandonado. Ya no me responde, ni con profetas ni
con sueños. Por eso te he llamado para que me digas qué hacer''.

\bibverse{16} ``¿Por qué vienes a preguntarme si el Señor te ha
abandonado y se ha convertido en tu enemigo?'' preguntó Samuel.
\bibverse{17} ``El Señor ha hecho contigo exactamente lo que te dijo a
través de mí, pues el Señor te ha arrancado el reino y se lo ha dado a
tu vecino, David. \bibverse{18} El Señor te ha hecho esto hoy porque no
hiciste lo que el Señor te mandó y no ejecutaste su furia sobre los
amalecitas. \bibverse{19} El Señor te entregará a ti y a Israel a los
filisteos. Mañana tú y tus hijos morirán y estarán conmigo. El Señor
también entregará el ejército israelita de Israel a los filisteos''.

\bibverse{20} Saúl se derrumbó boca abajo en el suelo, aterrorizado por
lo que Samuel había dicho. No tenía fuerzas, porque no había comido nada
en todo ese día y esa noche. \bibverse{21} La mujer se acercó a Saúl y
vio que estaba absolutamente aterrado. Ella le dijo: ``Mire, señor, yo
hice lo que usted me pidió. Arriesgué mi vida e hice lo que usted me
dijo. \bibverse{22} Ahora, por favor, haga lo que le digo. Deje que le
traiga un poco de comida. Cómasela y tendrá fuerzas para seguir su
camino''.

\bibverse{23} Pero él se negó, diciendo: ``No puedo comer nada''. Pero
sus hombres y la mujer le animaron a comer, y él hizo lo que le dijeron.
Se levantó del suelo y se sentó en la cama.

\bibverse{24} La mujer tenía un ternero cebado en la casa, y rápidamente
fue a sacrificarlo. También cogió harina, la amasó y coció panes sin
levadura. \bibverse{25} Luego ella sirvió la comida a Saúl y a sus
hombres, y ellos la comieron. Luego se levantaron y se fueron, esa misma
noche.

\hypertarget{section-28}{%
\section{29}\label{section-28}}

\bibverse{1} Los filisteos reunieron todos sus ejércitos en Afec, y los
israelitas acamparon junto al manantial de Jezreel. \bibverse{2} Los
jefes filisteos marchaban en sus divisiones de cientos y miles de
personas, con David y sus hombres en la retaguardia con el rey Aquis.
\bibverse{3} Pero los jefes filisteos preguntaron: ``¿Qué hacen aquí
estos hebreos?''\footnote{\textbf{29:3} Esto también podría traducirse
  como ``¿Quiénes son estos hebreos?'', ya que el texto simplemente dice
  ``Qué estos hebreos?''}

Entonces Aquis les respondió a los comandantes filisteos: ``Ese es
David, un oficial del rey Saúl de Israel. Lleva mucho tiempo conmigo,
incluso años, y no he encontrado ninguna falta en él desde el día en que
se pasó a nuestro lado hasta ahora.''

\bibverse{4} Pero los comandantes filisteos se enojaron con Aquis y le
dijeron: ``Envíalo de vuelta al lugar de donde vino, a la ciudad que le
asignaste. No puede ir con nosotros a la batalla. ¿Y si se vuelve contra
nosotros durante la lucha? ¡Qué buena manera de complacer a su amo,
entregando las cabezas de nuestros hombres! \bibverse{5} ¿No es éste el
David que cantan en sus danzas? ``Saúl ha matado a sus miles, y David a
sus decenas de miles''?

\bibverse{6} Entonces Aquis llamó a David y le dijo: ``Vive el Señor, tú
eres honesto y has hecho lo correcto por lo que veo. Por lo que a mí
respecta, debes marchar conmigo a la batalla porque no he encontrado
ningún fallo en ti desde el día en que llegaste hasta ahora. Pero los
otros líderes no te aprueban. \bibverse{7} Así que vuelve a tu casa en
paz, y así no harás nada que moleste a los líderes filisteos.''

\bibverse{8} ``¿Pero, qué he hecho?'' preguntó David. ``¿Qué falta has
encontrado en mí, tu siervo, desde el día en que vine a ti hasta ahora,
que me impida ir a luchar contra los enemigos de mi señor el rey?''

\bibverse{9} ``Por lo que a mí respecta, eres tan bueno como un ángel de
Dios'', respondió Aquis. ``Pero los comandantes filisteos han declarado:
`No puede entrar en batalla con nosotros'. \bibverse{10} Así que
levántate temprano mañana y sal con tus hombres en cuanto amanezca''.
\bibverse{11} David y sus hombres se levantaron de madrugada y volvieron
al país de los filisteos. Pero los filisteos avanzaron hacia Jezreel.

\hypertarget{section-29}{%
\section{30}\label{section-29}}

\bibverse{1} Tres días después, David y sus hombres llegaron de nuevo a
Siclag. Unos amalecitas habían hecho una incursión en el Néguev y en
Siclag. Habían atacado Siclag y la habían incendiado. \bibverse{2}
Habían capturado a las mujeres y a todos los demás allí, jóvenes y
ancianos. No habían matado a nadie, pero se llevaron a todos con ellos
al marcharse.

\bibverse{3} Cuando David y sus hombres volvieron a la ciudad, la
encontraron quemada hasta los cimientos, y a sus mujeres e hijos
capturados. \bibverse{4} David y sus hombres lloraron a gritos hasta no
poder más. \bibverse{5} Las dos esposas de David también habían sido
tomadas como prisioneras: Ahinoam, de Jezreel, y Abigail, la viuda de
Nabal, de Carmel. \bibverse{6} David estaba en un gran apuro, porque los
hombres estaban tan molestos por la pérdida de sus hijos que empezaron a
hablar de apedrearlo.

Pero confiando en el Señor, su Dios, \bibverse{7} David fue a ver al
sacerdote Abiatar, hijo de Ahimelec, y le dijo: ``Tráeme el efod''. Y
Abiatar se lo trajo. \bibverse{8} Entonces David le preguntó al Señor:
``¿Debo perseguir a estos asaltantes? ¿Los alcanzaré?''

``Sí, persíguelos'', contestó el Señor, ``porque definitivamente los
alcanzarás y rescatarás a los prisioneros''.

\bibverse{9} David y seiscientos de sus hombres partieron hacia el valle
de Besor. \bibverse{10} Doscientos de ellos se quedaron allí porque
estaban demasiado cansados para cruzar el valle, mientras que David
siguió adelante con cuatrocientos hombres.

\bibverse{11} Se encontraron con un egipcio en el campo y se lo llevaron
a David. Le dieron de comer y de beber. \bibverse{12} También le dieron
un trozo de una torta de higos y dos tortas de pasas. Se los comió y se
recuperó, porque llevaba tres días y tres noches sin comer ni beber.

\bibverse{13} ``¿De quién eres esclavo y de dónde vienes?'' le preguntó
David.

``Soy egipcio -- respondió --, esclavo de un amalecita. Mi amo me
abandonó hace tres días cuando me enfermé. \bibverse{14} Asaltamos a los
queretanos en el Neguev, así como la parte que pertenece a Judá y el
Neguev de Caleb. También quemamos Siclag''.

\bibverse{15} ``¿Puedes guiarme hasta esos asaltantes?'' preguntó David.

``Si me juras por Dios que no me matarás ni me entregarás a mi amo,
entonces te llevaré hasta ellos'', respondió el hombre.

\bibverse{16} Entonces llevó a David hasta donde los amalecitas, quienes
estaban esparcidos por todo el lugar, comiendo, bebiendo y bailando
debido al gran botín que habían tomado de las tierras de los filisteos y
de Judá. \bibverse{17} David los atacó desde el atardecer hasta la noche
siguiente. Nadie escapó, excepto cuatrocientos hombres que lograron
huir, montados en camellos. \bibverse{18} David recuperó todo lo que los
amalecitas habían tomado, incluidas sus dos esposas. \bibverse{19} Todo
fue contabilizado: todos los adultos y niños, así como todo el botín que
los amalecitas habían tomado. David recuperó todo. \bibverse{20} También
recuperó todos los rebaños y manadas. Sus hombres los llevaron por
delante del resto del ganado, gritando: ``¡Este es el botín de David!''.

\bibverse{21} Cuando David recuperó a los doscientos hombres que habían
estado demasiado cansados para seguir con él desde el valle de Besor,
salieron a recibirlo a él y a los hombres que lo acompañaban. Cuando
David se acercó a los hombres para saludarlos, \bibverse{22} todos los
hombres desagradables y buenos para nada de los que habían ido con David
dijeron: ``Ellos no estaban con nosotros, así que no compartiremos el
botín que tomamos, excepto para devolverles a sus esposas e hijos. Que
los tomen y se vayan''.

\bibverse{23} Pero David intervino diciendo: ``No, hermanos míos, no
deben hacer esto con lo que el Señor nos ha dado. Él nos ha protegido y
nos ha entregado a los asaltantes que nos habían atacado. \bibverse{24}
¿Quién los va a escuchar cuando digan tales cosas? La parte que reciban
los que fueron a la batalla será la misma que la de los que se quedaron
para guardar las provisiones''. \bibverse{25} David hizo que esta fuera
la regla y norma para Israel desde ese día hasta ahora.

\bibverse{26} Cuando David regresó a Siclag, envió parte del botín a
cada uno de sus amigos entre los ancianos de Judá, diciendo: ``Aquí
tienen un regalo para ustedes del botín de los enemigos del Señor.''
\bibverse{27} David lo envió a los que vivían en Betuel,\footnote{\textbf{30:27}
  ``Betuel'': mucho más probable que ``Betel'' como aparece en el texto
  hebreo.} Ramot Néguev, Jattir, \bibverse{28} Aroer, Sifmot, Eshtemoa,
29 Racal, y las ciudades de los jeraelitas y ceneos, \bibverse{30}
Hormah, Bor-ashan, Athach, \bibverse{31} Hebrón: todos los lugares a los
que David y sus hombres habían ido.

\hypertarget{section-30}{%
\section{31}\label{section-30}}

\bibverse{1} Mientras tanto, los filisteos habían atacado a Israel, y el
ejército israelita huyó de ellos, y muchos murieron en el monte Gilboa.
\bibverse{2} Los filisteos persiguieron a Saúl y a sus hijos, y mataron
a los hijos de Saúl: Jonatán, Abinadab y Malquisúa. \bibverse{3} La
lucha se hizo muy intensa en torno a Saúl, y las flechas de los arqueros
filisteos encontraron su objetivo, hiriendo gravemente a Saúl.

\bibverse{4} Entonces Saúl le dijo a su escudero: ``Toma tu espada y
mátame, o estos hombres paganos\footnote{\textbf{31:4} ``Paganos'':
  literalmente, ``incircuncisos.''} vendrán a matarme y a torturarme''.

Pero el escudero no quiso hacerlo porque tenía demasiado miedo. Entonces
Saúl tomó su propia espada y cayó sobre ella. \bibverse{5} Cuando su
escudero vio que Saúl estaba muerto, también cayó sobre su propia espada
y murió con él. \bibverse{6} Saúl, sus tres hijos, su escudero y todos
los hombres que estaban con él murieron el mismo día.

\bibverse{7} Cuando los israelitas que vivían a lo largo del valle y los
del otro lado del Jordán se dieron cuenta de que el ejército israelita
había huido y de que Saúl y sus hijos habían muerto, abandonaron sus
ciudades y también huyeron. Entonces llegaron los filisteos y se
apoderaron de ellas.

\bibverse{8} Al día siguiente, cuando los filisteos fueron a despojar a
los muertos, encontraron a Saúl y a sus tres hijos tendidos en el monte
Gilboa. \bibverse{9} Le cortaron la cabeza a Saúl, lo despojaron de su
armadura y enviaron mensajeros por todo el país de los filisteos para
que anunciaran la noticia en los templos de sus ídolos y a su pueblo.
\bibverse{10} Entonces colocaron su armadura en el templo de Astoret y
clavaron su cuerpo en el muro de la ciudad de Bet-San.

\bibverse{11} Sin embargo, cuando el pueblo de Jabes de Galaad se enteró
de lo que los filisteos le habían hecho a Saúl, \bibverse{12} todos sus
fuertes guerreros se pusieron en marcha, viajaron toda la noche y
descolgaron los cuerpos de Saúl y de sus hijos de la muralla de Bet-sán.
Cuando volvieron a Jabes, quemaron allí los cuerpos. \bibverse{13} Luego
tomaron sus huesos y los enterraron bajo el tamarisco en Jabes, y
ayunaron durante siete días.
