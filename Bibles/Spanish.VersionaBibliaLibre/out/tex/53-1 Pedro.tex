\hypertarget{section}{%
\section{1}\label{section}}

\bibverse{1} Esta carta viene de Pedro, apóstol de Jesucristo, y es
enviada al pueblo escogido de Dios: a los exiliados que están dispersos
por todas las provincias de Ponto, Galacia, Capadocia, Asia, y Bitinia.
\bibverse{2} Ustedes fueron elegidos por Dios, el Padre, en su
sabiduría, y son un pueblo santo por el Espíritu, que obedece a
Jesucristo y que está rociado con su sangre. Tengan gracia y paz cada
vez más.

\bibverse{3} ¡Alabado sea Dios, el Padre de nuestro Señor Jesucristo!
Por su gran misericordia hemos nacido de nuevo y se nos ha dado una
esperanza viva\footnote{\textbf{1:3} O ``una esperanza que nos trae
  vida.''} por la resurrección de Jesucristo de entre los muertos.
\bibverse{4} Esta herencia es eterna, y nunca se daña ni se desvanece, y
está ahí segura para ustedes. \bibverse{5} Por la fe de ustedes en él,
Dios los protegerá con su poder hasta que venga la salvación. La
salvación que está lista para ser revelada en el último día.

\bibverse{6} Así que estén felices por esto, aunque estén tristes por un
poco de tiempo, mientras soportan distintas pruebas. \bibverse{7} Estas
demuestran que su fe en Dios es genuina---aunque también puede ser
destruida---y esa fe es más valiosa que el oro. De este modo, su fe en
Dios será reconocida y ustedes recibirán alabanza, gloria y honra cuando
Cristo aparezca.

\bibverse{8} Ustedes lo aman aunque nunca lo han visto. Aunque no pueden
verlo ahora, creen en él y están llenos de una felicidad maravillosa e
indescriptible. \bibverse{9} ¡Y por creer en él, su recompensa será la
salvación! \bibverse{10} La salvación que buscaban y de la cual
investigaban los profetas cuando hablaban de la gracia que estaba
preparada para ustedes. \bibverse{11} Trataton de descubrir cuándo y
cómo esto sucedería, porque el Espíritu de Cristo dentro de ellos
hablaba de manera clara sobre los sufrimientos de Cristo y la gloria que
vendría. \bibverse{12} A ellos se les explicó que lo que hacían no era
para ellos mismos, sino para ustedes, pues aquello de lo que ellos
hablaron, ustedes lo han aprendido de aquellos que compartieron la buena
noticia con ustedes por el Espíritu Santo que el cielo envió. ¡Hasta los
ángeles quieren saber sobre esto!

\bibverse{13} Asegúrense de que sus mentes estén alerta. Tengan un
pensamiento claro. Fijen su esperanza exclusivamente en la gracia que
les será dada cuando Jesús sea revelado. \bibverse{14} Vivan como hijos
obedientes. No se permitan a ustedes mismos ser moldeados por sus
antiguos deseos pecaminosos, cuando no conocían algo mejor.
\bibverse{15} Ahora necesitan ser santos en todo lo que hagan, así como
Aquél que los llamó es santo. \bibverse{16} Tal como dice la Escritura:
``Sean santos, porque yo soy santo.''

\bibverse{17} Puesto que ustedes le llaman Padre, y reconocen que él
juzga a todos de manera imparcial, basado en sus obras, tomen en serio
su vida aquí en la tierra, guardando reverencia hacia él.

\bibverse{18} Ya saben que no fueron liberados por su vana forma de
vivir que heredaron de sus antepasados, por cosas que no tenían valor
duradero, como el oro o la plata. \bibverse{19} Sino que fueron
liberados con la preciosa sangre de Cristo, que fue como un cordero sin
mancha ni defecto.

\bibverse{20} Él fue elegido antes de la creación del mundo, pero fue
revelado en estos últimos días\footnote{\textbf{1:20} O ``al final del
  tiempo.''} para beneficio de ustedes. \bibverse{21} Por medio de él,
ustedes creen en Dios, quien lo levantó de los muertos, y lo glorificó,
para que ustedes puedan confiar y tener esperanza en Dios. \bibverse{22}
Ahora que se han consagrado a seguir la verdad, ámense unos a otros con
sinceridad, como una verdadera familia\footnote{\textbf{1:22} O ``con
  amor fraternal.''}. \bibverse{23} Ustedes han nacido de nuevo, no son
el producto de una ``semilla'' mortal,''\footnote{\textbf{1:23} Aquí en
  énfasis está en el hecho de que distintas ``semillas'' producen
  distintas clases de ``vida.''} sino inmortal, por la palabra viva y
eternal de Dios. \bibverse{24} Porque: ``Todas las personas son como la
hierba, y su gloria es como flores del campo. La hierba se seca y las
flores se marchitan. \bibverse{25} Pero la palabra de Dios permanece
para siempre.''\footnote{\textbf{1:25} Isaías 40:6-8.} Esta palabra es
la buena noticia de la que les hablaron antes.

\hypertarget{section-1}{%
\section{2}\label{section-1}}

\bibverse{1} Así que renuncien a las malas obras que hacen: la
deshonestidad, la hipocresía, el hablar mal de los demás. \bibverse{2}
Deben volverse como bebés recién nacidos que solo quieren leche
espiritual pura, para que puedan crecer en la salvación \bibverse{3}
ahora que han probado cuán bueno es el Señor. \bibverse{4} Cuando se
acerquen a él, la piedra viva que la gente rechazó como si fuera inútil,
- pero que es elegida por Dios y preciada para él -- \bibverse{5}
ustedes también se convierten en piedras vivas, edificadas en una casa
espiritual. Ustedes son sacerdocio santo que ofrece sacrificios
espirituales y que Dios recibe con agrado por medio de Jesucristo.
\bibverse{6} Como dice la Escritura\footnote{\textbf{2:6} Isaías 28:16.}:
``¡Miren! Yo establezco en Sión su piedra angular, una piedra escogida
de manera especial y valiosa. Todo el que crea en él no será
defraudado\footnote{\textbf{2:6} O ``avergonzado.''}.'' \bibverse{7} Él
es muy valioso para todos ustedes los que creen. Pero para los que no
creen, ``La piedra que los constructores rechazaron, y que llegó a ser
la piedra angular del fundamento.''\footnote{\textbf{2:7} Salmos 118:22.}
\bibverse{8} es ``La piedra que hace tropezar y los hace
caer.''\footnote{\textbf{2:8} Isaías 8:14.} La gente tropieza con este
mensaje porque se niegan a aceptarlo, lo cual es completamente
predecible en cuanto a ellos.

\bibverse{9} En cambio, ustedes son una familia elegida de manera
especial, un sacerdocio real, una nación santa, un pueblo que pertenece
a Dios. Por eso, pueden revelar las cosas maravillosas que él ha hecho,
al sacarlos de la oscuridad a su luz admirable. \bibverse{10} En el
pasado, ustedes no eran nadie, pero ahora son el pueblo de Dios. En el
pasado carecieron de misericordia, pero ahora la han recibido.

\bibverse{11} Amigos míos, les ruego como si fueran
extranjeros\footnote{\textbf{2:11} ``Peregrinos y extranjeros'' que no
  ven este mundo como su hogar.} en este mundo, que no se rindan ante
los deseos físicos que están en oposición a lo espiritual. \bibverse{12}
Asegúrense de actuar apropiadamente cuando estén en compañía de quienes
no son cristianos, para que incluso si los acusaran de hacer lo malo,
ellos puedan ver sus buenas obras y glorifiquen a Dios cuando
venga\footnote{\textbf{2:12} Literalmente, ``día de visitación.''}.

\bibverse{13} Obedezcan a la autoridad humana, por causa del Señor, ya
sea al rey, como autoridad suprema, \bibverse{14} o a los gobernantes
que Dios designa para castigar a los que hacen el mal y dar
reconocimiento a los que hacen el bien. \bibverse{15} Dios quiere que al
hacer el bien ustedes hagan callar las acusaciones ignorantes de los
necios. \bibverse{16} ¡Sí! ¡Ustedes son un pueblo libre! Así que no usen
la libertad para disimular la maldad, sino vivan como siervos de Dios.
\bibverse{17} Respeten a todos. Muestren su amor por la comunidad de
creyentes. Reverencien a Dios. Respeten al rey. \bibverse{18} Si eres un
siervo, entonces mantente sujeto a tu amo, no solo a los que son buenos
y nobles, sino también a los que son duros. \bibverse{19} Porque en esto
consiste la gracia: soportar el dolor de la vida y el sufrimiento
injusto, pero manteniendo la mente enfocada en Dios. \bibverse{20} Sin
embargo, no hay crédito si eres castigado por hacer el mal. Pero si
sufres por hacer lo recto, y lo soportas, entonces la gracia de Dios
está contigo.

\bibverse{21} En efecto, a esto han sido llamados, porque Cristo sufrió
por ustedes y les dio un ejemplo, para que siguieran sus pasos.
\bibverse{22} Él nunca pecó, ni mintió; \bibverse{23} y cuando fue
maltratado, no replicó. Cuando sufrió, no amenazó con venganza.
Simplemente se puso en manos de Aquél que juzga siempre con justicia.
\bibverse{24} Tomó nuestros pecados sobre sí mismo, sobre su cuerpo en
la cruz, para que nosotros pudiéramos morir al pecado y vivir en
justicia. ``Por sus heridas, somos sanados.''\footnote{\textbf{2:24}
  Isaías 53:5.} \bibverse{25} En un tiempo ustedes eran como ovejas que
habían perdido su camino, pero ahora han regresado al pastor, al que
cuida de ustedes.

\hypertarget{section-2}{%
\section{3}\label{section-2}}

\bibverse{1} Esposas, acepten la autoridad de sus esposos de la misma
manera, para que si ellos se niegan a aceptar la palabra, puedan ser
ganados sin palabras, por la conducta de ustedes, \bibverse{2}
reconociendo que su conducta es pura y reverente. \bibverse{3} No se
concentren en el atractivo físico, ni en el corte de cabello, ni en las
joyas de oro, o en las ropas elegantes; \bibverse{4} sino por el
contrario, que el atractivo sea interior, que sea el de un espíritu
manso y pacífico que nace desde el interior. Porque eso es lo que Dios
estima. \bibverse{5} Así es como en el pasado, las mujeres santas que
ponían su fe en Dios, se embellecían, con la ternura que brindaban a sus
esposos, \bibverse{6} como Sara, que obedecía a Abrahán, y lo llamaba
``señor.''\footnote{\textbf{3:6} Or ``maestro.'' Hoy esta formalidad en
  el matrimonio es inusual.} Ustedes son sus hijas si hacen lo recto y
sin temor.

\bibverse{7} Esposos, del mismo modo, sean considerados con sus esposas
en su vida diaria juntos. Aunque tu esposa no sea tan fuerte como tú,
debes honrarla, porque ella heredará en igual proporción junto a ti el
don de la vida de Dios. Asegúrense de hacer estas cosas para que nada
estorbe sus oraciones.

\bibverse{8} Finalmente, tengan todos un mismo propósito. Sean amables y
amorosos unos con otros. Sean compasivos y humildes. \bibverse{9} No
paguen mal por mal, ni reclamen cuando otros sean abusivos, sino
bendíganlos, porque a eso fueron llamados, para que puedan recibir
bendiciones ustedes mismos también. \bibverse{10} Recuerden: ``Los que
quieren amar sus vidas y ver días felices, deben abstenerse de hablar el
mal, y no decir mentiras. \bibverse{11} Aléjense del mal y hagan el
bien; ¡busquen la paz y síganla! \bibverse{12} Porque Dios está atento a
los justos y escucha sus oraciones, pero aborrece a los que hacen el
mal.''\footnote{\textbf{3:12} Salmos 34:12-16.}

\bibverse{13} ¿Quién les hará daño si la intención de ustedes es hacer
el bien? \bibverse{14} Porque incluso si experimentan sufrimiento por
hacer lo recto, ustedes están mucho mejor. No teman las amenazas de la
gente, no se preocupen por esas cosas, \bibverse{15} solo tengan en su
mente a Cristo como Señor. Estén siempre listos para dar explicaciones a
todo el que pregunte la razón de su esperanza. Y háganlo con mansedumbre
y respeto. \bibverse{16} Asegúrense de tener una conciencia limpia, para
que si alguno los acusa, sean avergonzados por hablar mal sobre la buena
manera de vivir de ustedes, en Cristo. \bibverse{17} Sin duda alguna, es
mejor sufrir haciendo el bien, (si eso es lo que Dios quiere), que
sufrir haciendo el mal. \bibverse{18} Y Jesús murió por culpa de los
pecados, una vez y para siempre, el Único que es completamente verdadero
y justo, por aquellos que somos malos,\footnote{\textbf{3:18}
  Literalmente, ``el justo por los injustos.''} para poder llevarnos a
Dios. Fue llevado a muerte en su cuerpo, pero vino a la vida en el
espíritu.

\bibverse{19} Él fue a hablar a los que estaban ``presos''\footnote{\textbf{3:19}
  O ``almas prisioneras.'' Ha existido mucho debate sobre esta frase.
  Debemos notar que la misma palabra que se usa para ``almas'' aquí, se
  usa también en el versículo 10. Algunos entienden que ``prisioneras''
  se refiere a las personas que vivían en la época del diluvio y que
  estaban ``cautivas'' por su pecaminosidad (ver Gén. 6:5).}
\bibverse{20} y que se negaban a creer, siendo que Dios con paciencia
esperó, durante los días de Noé, cuando estaban construyendo el arca.
Apenas unos cuantos---de hecho, ocho personas---se salvaron ``por el
agua.'' \bibverse{21} Esta agua simboliza el bautismo que los salva
ahora, no limpiando la suciedad de sus cuerpos, sino como una respuesta
positiva a Dios, que surge de una conciencia limpia. La resurrección de
Jesús es la que posibilita la salvación. \bibverse{22} Después de haber
ascendido al cielo, él está en pie a la diestra de Dios, con ángeles,
autoridades, y poderes puestos bajo su control.

\hypertarget{section-3}{%
\section{4}\label{section-3}}

\bibverse{1} Y como Cristo padeció sufrimiento físico, ustedes deben
prepararse con la misma actitud que él tuvo, porque los que sufren
físicamente, han abandonado el pecado\footnote{\textbf{4:1} Este es un
  versículo difícil, pues sin duda el sufrimiento no implica que no haya
  pecado. Queda implícito que así como Jesús sufrió injustamente, cuando
  los cristianos sufren, participan de la experiencia de Cristo.}.
\bibverse{2} Ustedes no vivirán el resto de sus vidas siguiendo los
deseos humanos, sino haciendo la voluntad de Dios. \bibverse{3} En el
pasado vivieron mucho tiempo siguiendo los caminos del mundo:
inmoralidad, complacencia sexual, orgías, fiestas, borracheras, e
idolatría abominable. \bibverse{4} La gente piensa que es extraño que
ustedes ya no participen con ellos de este estilo de vida lleno de
excesos, y por eso los maldicen. Pero ellos tendrán que dar cuentas de
lo que han hecho contra Aquél que está listo para juzgar a los vivos y a
los muertos. \bibverse{5} Por eso, la buena noticia fue compartida con
los que ya murieron, \bibverse{6} para que aunque hayan sido juzgados
correctamente según la justicia humana y pecaminosa, ellos puedan vivir
en el espíritu según la justicia de Dios.

\bibverse{7} ¡Todo llegará a su fin! Así que piensen con claridad y
manténganse vigilantes cuando oren. \bibverse{8} Por encima de todo,
ámense unos a otros con amor profundo, porque el amor cubre muchas de
las faltas que la gente comete. \bibverse{9} Muestren hospitalidad unos
con otros y no se quejen. \bibverse{10} Cualquiera sea el don que hayan
recibido, compártanlo con otros entre ustedes, como un pueblo que
demuestra sabiamente la gracia de Dios, en todas sus formas.
\bibverse{11} Todo el que hable, hágalo como si Dios hablara a través de
él. Todo aquél que quiera ayudar a otros, hágalo por medio de la fuerza
que Dios le da, para que en todo Dios sea glorificado por medio de
Jesucristo. Que la gloria y el poder sean suyos por siempre y para
siempre. Amén.

\bibverse{12} Amigos míos, no se sorprendan ante las ``pruebas de
fuego''\footnote{\textbf{4:12} Literalmente ``una prueba de fuego para
  probarlos.''} que están experimentando, como si estas fueran algo
inesperado. \bibverse{13} Estén contentos en la medida que participan
del sufrimiento de Cristo, porque cuando aparezca en su gloria, ustedes
serán muy felices. \bibverse{14} Si alguien los maldice en el nombre de
Cristo, en realidad son bendecidos, porque el espíritu glorioso de Dios
reposa sobre ustedes. \bibverse{15} Y si sufren, no será como asesinos,
como ladrones, como criminales o como chismosos, \bibverse{16} sino que
si es como un cristiano, entonces no tendrán de qué avergonzarse. Más
bien, oren para que sean llamados cristianos.

\bibverse{17} Porque el tiempo del juicio ha llegado, y comienza por la
casa de Dios. Y si comienza por nosotros, entonces ¿cuál será el fin de
los que rechazan la buena noticia de Dios? \bibverse{18} ``Si ya es
difícil salvarse para los que son justos, ¿qué será de los pecadores que
aborrecen a Dios?''\footnote{\textbf{4:18} Proverbios 11:31.}
\bibverse{19} De modo que los que sufren conforme a la voluntad de Dios,
del Creador fiel, deben asegurarse de que están haciendo el bien.

\hypertarget{section-4}{%
\section{5}\label{section-4}}

\bibverse{1} Quiero animar a los ancianos que están entre ustedes. Pues
yo también soy un anciano, un testigo de los sufrimientos de Cristo, y
participaré de la gloria que está por venir. \bibverse{2} Cuiden del
rebaño que se les ha encomendado, no porque estén obligados a
vigilarlos, sino con agrado, como Dios quiere que sea. Háganlo de buena
gana, sin buscar beneficio de ello. \bibverse{3} No sean arrogantes,
enseñoreándose de aquellos que están bajo su cuidado, sino sean un
ejemplo para el rebaño. \bibverse{4} Cuando aparezca el Pastor supremo,
ustedes recibirán una corona de gloria, que nunca se dañará.

\bibverse{5} Jóvenes, hagan lo que los ancianos les dicen. Sin duda
deberían todos servirse unos a otros con humildad, porque ``Dios
aborrece a los orgullosos, pero obra en favor de los
humildes.''\footnote{\textbf{5:5} Proverbios 3:34.} \bibverse{6}
Humíllense ante la mano poderosa de Dios, para que los exalte en su
debido tiempo. \bibverse{7} Entreguen todas sus preocupaciones a él,
porque él tiene cuidado de ustedes. \bibverse{8} Sean responsables, y
estén vigilantes. El diablo, su enemigo, anda por ahí, como león
rugiente, buscando a quién devorar. \bibverse{9} Manténganse firmes
contra él, confiando en Dios. Recuerden que sus hermanos creyentes en
todo el mundo están viviendo dificultades similares. \bibverse{10} Pero
después de que hayan sufrido un poco, el Dios de toda gracia, que los
llamó a su gloria eterna en Cristo, él mismo los restaurará, los
sostendrá, los fortalecerá y les dará un fundamento sólido.
\bibverse{11} El poder sea suyo, por siempre y para siempre. Amén.

\bibverse{12} Esta carta se las envío con ayuda de Silvano, a quien
considero como un hermano fiel. En estas pocas palabras que les he
escrito, quiero animarlos y testificar que esta es la verdadera gracia
de Dios. ¡Manténganse firmes en ella! \bibverse{13} Los creyentes de
aquí, de ``Babilonia,''+ 5.13 Literalmente, ``los que están en
Babilonia.'' En el Nuevo Testamento Babilonia es un símbolo de Roma.
escogidos junto a ustedes, les envían su saludo, así como Marcos, mi
hijo. \bibverse{14} Salúdense unos a otros con un beso de amor. Paz a
todos ustedes que están en Cristo.
