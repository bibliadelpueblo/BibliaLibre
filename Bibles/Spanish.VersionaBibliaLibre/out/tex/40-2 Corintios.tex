\hypertarget{section}{%
\section{1}\label{section}}

\bibverse{1} Esta carta viene de parte de Pablo, apóstol de Jesucristo,
conforme a la voluntad de Dios, y de parte de Timoteo, nuestro hermano.
Es enviada a la iglesia de Dios en Corinto, así como a todo el pueblo de
Dios que está por toda la región de Acaya. \bibverse{2} Reciban gracia y
paz de Dios, nuestro Padre, y del Señor Jesucristo.

\bibverse{3} ¡Alaben a Dios, el padre de nuestro Señor Jesucristo! Él es
el Padre misericordioso, y Dios de toda consolación. \bibverse{4} Él nos
consuela en todas nuestras aflicciones, para que podamos consolar
también a otros con el consuelo que recibimos de Dios. \bibverse{5}
Cuanto más participamos de los sufrimientos de Cristo, tanto más
abundante es el consuelo que recibimos de él. \bibverse{6} Si estamos
angustiados, es para su consuelo y salvación. Si estamos siendo
consolados, es para consuelo de ustedes, que los ayuda a soportar con
paciencia los mismos sufrimientos que nosotros padecemos. \bibverse{7}
Confiamos en gran manera en ustedes\footnote{\textbf{1:7} Literalmente,
  ``nuestra esperanza en ustedes está firme.''}, sabiendo que así como
participan de nuestros sufrimientos, también participan de nuestro
consuelo.

\bibverse{8} Hermanos y hermanas, no les ocultaremos los problemas que
tuvimos en Asia. Estábamos tan agobiados que temíamos no tener las
fuerzas para continuar, tanto así que dudábamos de que pudiéramos salir
con vida. \bibverse{9} De hecho, era como una sentencia de muerte dentro
de nosotros. Esto nos sirvió para dejar de depender de nosotros mismos y
comenzar a confiar en Dios, quien levanta a los muertos. \bibverse{10}
Él nos salvó de la muerte, y pronto lo hará otra vez. Tenemos plena
confianza en que Dios seguirá salvándonos. \bibverse{11} Ustedes nos
ayudan con sus oraciones. De este modo, muchos agradecerán a Dios por la
bendición que Dios nos dará en respuesta a las oraciones de muchos.

\bibverse{12} Nos enorgullecemos en el hecho---y nuestra conciencia lo
confirma---de que hemos actuado de manera apropiada con las personas,
especialmente con ustedes. Hemos seguido los principios de Dios de
santidad y sinceridad, no conforme a la sabiduría mundanal, sino por la
gracia de Dios. \bibverse{13} Porque no escribimos ninguna cosa
complicada que ustedes no puedan leer o comprender. Espero que ustedes
al final entiendan, \bibverse{14} aunque ahora solo entiendan en parte,
a fin de que cuando el Señor venga, ustedes estén orgullosos de
nosotros, como nosotros de ustedes.

\bibverse{15} Como yo estaba tan seguro de su confianza en mí, hice
planes para venir a visitarlos primero. Así ustedes se habrían
beneficiado doblemente, \bibverse{16} pues iría desde donde están
ustedes a Macedonia, y luego volvería desde Macedonia a donde ustedes
nuevamente. Luego yo les habría pedido que me enviaran de camino a
Judea. \bibverse{17} ¿Por qué cambié mi plan original? ¿Creen que tomo
decisiones a la ligera? ¿Creen que cuando hago planes soy como cualquier
persona del mundo que dice Sí y No al mismo tiempo? \bibverse{18} Así
como Dios es digno de confianza, cuando nosotros les damos nuestra
palabra, no es Sí y No a la vez. \bibverse{19} La verdad del Hijo de
Dios, Jesucristo, fue anunciada a ustedes por medio de
nosotros---Silvano, Timoteo y yo---y no fue Sí y No.~¡En Cristo la
respuesta es definitivamente Sí! \bibverse{20} No importa cuántas
promesas Dios haya hecho, en Cristo la respuesta siempre es Sí. Por él,
respondemos diciendo Sí\footnote{\textbf{1:20} Literalmente, ``Amén,''
  que significa ``Sí,'' o ``Estoy de acuerdo.''} a la gloria de Dios.
\bibverse{21} Él nos ha dado a nosotros y también a ustedes la fuerza
para permanecer firmes en Cristo. Dios nos ha designado, \bibverse{22}
ha puesto su sello de aprobación sobre nosotros, y nos ha dado la
garantía del Espíritu en nuestros corazones. \bibverse{23} Pongo a Dios
como mi testigo que la razón por la que decidí no ir a Corinto fue para
no causarles dolor. \bibverse{24} El propósito de esto no es dictarles
la manera en que deben relacionarse con Dios, sino porque queremos
ayudarlos a tener una experiencia de gozo, porque es a través de la fe
en Dios que permanecemos firmes.

\hypertarget{section-1}{%
\section{2}\label{section-1}}

\bibverse{1} Por eso decidí que evitaría otra visita triste con ustedes.
\bibverse{2} Porque si les causo tristeza, ¿quién estará allí para
alegrarme a mí? ¡No serán ustedes mismos, a quienes entristecí!
\bibverse{3} Por eso escribí lo que escribí, para no estar triste por
los que deberían causarme alegría. Estaba muy seguro de que todos
ustedes participarían de mi felicidad. \bibverse{4} Lloré mucho cuando
les escribí, en gran angustia y con un corazón cargado, no para
entristecerlos, sino para que supieran cuánto los amo.

\bibverse{5} Sin exagerar, pero la persona que causó mi tristeza,
provocó más dolor a todos ustedes que a mí. \bibverse{6} Esta persona
sufrió suficiente castigo por parte de la mayoría de ustedes,
\bibverse{7} así que ahora deben perdonarlo y ser amables con él. De lo
contrario, podría hundirse en el remordimiento. \bibverse{8} Así que yo
los animo a que públicamente confirmen su amor hacia él. \bibverse{9}
Por eso escribí, para poder Conocer el carácter de ustedes y comprobar
si están haciendo lo que se les enseñó. \bibverse{10} A todo el que
ustedes perdonen, yo también perdono. Lo que he perdonado, sea lo que
sea, lo he perdonado ante Cristo, en beneficio de ustedes. \bibverse{11}
De este modo, Satanás no podrá llevarnos hacia el pecado, porque
conocemos las trampas que él inventa.

\bibverse{12} Cuando llegué a Troas para predicar la buena noticia de
Cristo, el Señor puso delante de mí una oportunidad. \bibverse{13} Pero
mi mente no estaba en paz porque no podía encontrar a mi hermano Tito.
De modo que me despedí y me fui hacia Macedonia\footnote{\textbf{2:13}
  Viajar de Troas a Macedonia implicaba realizar un cruce por el mar.}.

\bibverse{14} ¡Pero gloria a Dios, que siempre nos guía hacia la
victoria en Cristo, y revela un dulce aroma de su conocimiento a través
de nosotros, dondequiera que vamos! \bibverse{15} Somos como una
fragancia de Cristo para Dios, que se eleva entre los que son salvos así
como entre los que mueren. \bibverse{16} Para los que mueren, es el
aroma de la descomposición, pero para los que son salvos, es el aroma de
la vida. ¿Pero de quién depende esta tarea? \bibverse{17} No somos como
la mayoría, que hacen negocios con la palabra de Dios por conveniencia.
Muy por el contrario: somos sinceros al predicar la palabra de Dios en
Cristo, sabiendo que él nos ve.

\hypertarget{section-2}{%
\section{3}\label{section-2}}

\bibverse{1} ¿Acaso estamos empezando a hablar bien de nosotros mismos
una vez más? ¿O necesitamos una carta de recomendación para ustedes, o
de parte de ustedes, como algunos? \bibverse{2} Ustedes son nuestra
carta de recomendación, escrita en nuestros corazones, la cual todo el
mundo conoce y puede leer. \bibverse{3} Ustedes demuestran que son una
carta de Cristo, entregada por nosotros; no escrita con tinta, sino con
el Espíritu del Dios vivo; no escrita sobre piedras, sino en corazones
humanos. \bibverse{4} Tenemos plena confianza ante Dios por medio de
Cristo. \bibverse{5} No porque consideremos que nosotros mismos podemos
hacerlo, sino que Dios nos da este poder. \bibverse{6} También nos da la
capacidad de ser ministros de un nuevo acuerdo\footnote{\textbf{3:6} O
  ``pacto.''}, no basado en la letra de la ley, sino en el Espíritu. La
letra de la ley mata, pero el Espíritu da vida. \bibverse{7} Sin
embargo, la antigua forma de relacionarnos con Dios, escrita en piedras,
terminó en muerte, aunque fue entregada con la gloria de Dios, tanto
así, que los israelitas no pudieron soportar ver el rostro de Moisés
porque era muy brillante, aunque esa gloria se estaba desvaneciendo.
\bibverse{8} Si fue así, ¿no debería venir con mayor gloria la nueva
forma de relacionarnos con Dios en el Espíritu? \bibverse{9} ¡Si la
antigua forma que nos condena trae gloria, la nueva forma, que nos
justifica, trae consigo mucha más gloria todavía! \bibverse{10} Porque
las cosas viejas que una vez fueron gloriosas, no tienen gloria en
comparación con la increíble gloria de lo nuevo. \bibverse{11} Si lo
viejo, que se desvanece, tenía gloria, lo nuevo, que no se acaba, tiene
mucha más gloria.

\bibverse{12} ¡Y como tenemos esta esperanza segura, hablamos sin temor!
\bibverse{13} No tenemos que ser como Moisés, que tuvo que ponerse un
velo para cubrir su rostro y así los israelitas no fueran enceguecidos
por la gloria, aunque ya se estaba desvaneciendo. \bibverse{14} No
obstante, sus corazones se endurecieron. Porque desde ese entonces hasta
ahora, cuando se lee el antiguo pacto, permanece el mismo ``velo''.
\bibverse{15} Incluso hoy, cada vez que se leen los libros de Moisés, un
velo cubre sus mentes. \bibverse{16} Pero cuando se convierten y aceptan
al Señor, el velo se quita. \bibverse{17} Ahora bien, el Señor es el
Espíritu, y dondequiera está el Espíritu del Señor, hay libertad.
\bibverse{18} Así que todos nosotros, con nuestros rostros descubiertos,
vemos y reflejamos al Señor como en un espejo. Estamos siendo
transformados conforme a la misma imagen del espejo, cuya gloria es cada
vez más brillante. Esto es lo que hace el Señor, que es el Espíritu.

\hypertarget{section-3}{%
\section{4}\label{section-3}}

\bibverse{1} Así pues, como Dios en su misericordia nos ha proporcionado
esta nueva manera de relacionarnos con él, no nos rendimos. \bibverse{2}
Pero sí hemos renunciado a los actos secretos y vergonzosos. No actuamos
con engaño ni distorsionamos la Palabra de Dios. Nosotros demostramos lo
que somos al revelar la verdad ante Dios, a fin de que todos puedan
decidirse a conciencia. \bibverse{3} Aún si la nueva noticia que
compartimos está velada, lo está para los que mueren. \bibverse{4} El
dios de este mundo ha cegado las mentes de los que no creen en Dios.
Ellos no pueden ver la luz de la buena noticia de la gloria de Cristo,
quien es la imagen de Dios.

\bibverse{5} No nos anunciamos\footnote{\textbf{4:5} Literalmente,
  ``predicamos.''} a nosotros mismos, sino a Cristo Jesús como Señor. De
hecho, somos siervos de ustedes por causa de Jesús. \bibverse{6} Porque
el Dios que dijo: ``Que brille la luz en medio de la oscuridad,'' brilló
en nuestros corazones para iluminar el conocimiento de la gloria de Dios
en el rostro de Jesucristo. \bibverse{7} Pero tenemos este tesoro en
vasijas de barro, para demostrar que este poder supremo proviene de Dios
y no de nosotros.

\bibverse{8} Nos atacan por todos lados, pero no estamos derrotados.
Estamos confundidos en cuanto a qué hacer, pero nunca desesperados.
\bibverse{9} Estamos perseguidos, pero nunca abandonados por Dios.
¡Estamos derribados, pero no destruidos! \bibverse{10} En nuestros
cuerpos siempre participamos de la muerte de Jesús, para así también
poder demostrar la vida de Jesús en nuestros cuerpos. \bibverse{11}
Aunque vivimos, estamos siempre bajo amenaza de muerte por causa de
Jesús, a fin de que la vida de Jesús pueda revelarse en nuestros cuerpos
mortales. \bibverse{12} En consecuencia, enfrentamos la muerte para que
ustedes tengan vida.

\bibverse{13} Como tenemos el mismo espíritu de confianza en Dios al que
se refiere la Escritura cuando dice: ``Creí en Dios, por tanto hablé,''
nosotros también creemos en Dios y hablamos de él. \bibverse{14} Sabemos
que Dios, quien resucitó a Jesús, también nos resucitará con él, y nos
llevará a su presencia con ustedes. \bibverse{15} ¡Todo es por ustedes!
Cuantos más alcance la gracia de Dios, mayor será nuestro agradecimiento
a él, a su gloria. \bibverse{16} Por eso no nos rendimos. Aunque
nuestros cuerpos físicos están cayéndose a pedazos, nuestro interior se
renueva cada día. \bibverse{17} Estas tribulaciones triviales que
tenemos, apenas duran un poco de tiempo, pero producen para nosotros
gloria eterna. \bibverse{18} No nos interesa lo visible, porque
aspiramos a lo invisible. Lo que vemos es temporal, pero lo que no vemos
es eterno.

\hypertarget{section-4}{%
\section{5}\label{section-4}}

\bibverse{1} Sabemos que cuando esta ``tienda de campaña''\footnote{\textbf{5:1}
  El simbolismo que vemos aquí es que el cuerpo terrenal es como una
  tienda de campaña, y un cuerpo celestial es una casa, y ambos
  ``visten'' a la persona.} terrenal en la que vivimos sea derribada,
tenemos una casa preparada por Dios, no hecha por manos humanas. Es
eterna, y está en el cielo. \bibverse{2} Suspiramos en nuestro anhelo
por esto, deseando con ansias ser vestidos de este nuevo hogar
celestial. \bibverse{3} Cuando tengamos este vestido, ya no nos veremos
desnudos. \bibverse{4} Aunque estamos en esta ``tienda'' suspiramos,
agobiados por esta vida. No deseamos tanto ser desvestidos de lo que nos
ofrece esta vida, sino que ansiamos aquello con lo que seremos
revestidos, para que lo mortal sea aplastado por la vida. \bibverse{5}
Dios mismo preparó todo esto para nosotros, y nos dio al Espíritu como
garantía. \bibverse{6} Por ello mantenemos la fe, sabiendo que aunque
estamos en casa, con nuestros cuerpos físicos, estamos lejos del Señor.
\bibverse{7} (Pues vivimos por la fe en el Señor, y no por vista).
\bibverse{8} Como les digo, estamos seguros, deseando estar lejos del
cuerpo para poder estar en casa con el Señor. \bibverse{9} Por eso
nuestra meta, ya sea que estemos en nuestro cuerpo o no, es agradarle.
\bibverse{10} Porque todos debemos comparecer ante el tribunal de
Cristo. Y cada uno de nosotros recibirá lo que merece por lo que hayamos
hecho en esta vida, ya sea bueno o malo.

\bibverse{11} Sabiendo lo que es el temor al Señor, tratamos de
convencer a otros. Para Dios es claro lo que somos, y espero que esté
claro en sus mentes también. \bibverse{12} Una vez más, no intentamos
hablar bien de nosotros mismos, sino que tratamos de darles a ustedes la
oportunidad de que se sientan orgullosos de nosotros, a fin de que
puedan responderle a los que se enorgullecen de lo exterior y no de lo
interior\footnote{\textbf{5:12} Literalmente, ``en el corazón.''}.
\bibverse{13} Si estamos ``locos''\footnote{\textbf{5:13} Eso era
  posiblemente una crítica hecha por los de corinto respecto a Pablo y
  sus compañeros.} es por Dios. Si somos sensatos, es por ustedes.
\bibverse{14} El amor de Cristo nos obliga, porque estamos completamente
seguros de que él murió por todos y así todos murieron. \bibverse{15}
Cristo murió por todos para que ya no vivieran para sí mismos, sino para
él, quien murió y resucitó para ellos.

\bibverse{16} De ahora en adelante ya no miramos a nadie desde el punto
de vista humano. Aunque una vez vimos a Cristo de esta manera, ya no lo
hacemos. \bibverse{17} Por eso todo el que está en Cristo es un nuevo
ser. ¡Lo viejo ya se ha ido y ha llegado lo nuevo! \bibverse{18} Dios lo
hizo transformándonos de enemigos en amigos por medio de Cristo. Dios
nos encomendó este mismo trabajo de convertir a sus enemigos en sus
amigos. \bibverse{19} Porque Dios estaba en Cristo trayendo al mundo de
regreso de la hostilidad a la amistad con él, sin contar sus pecados, y
dándonos este mensaje para convertir a sus enemigos en sus amigos.
\bibverse{20} De modo que somos embajadores de Cristo, como si él rogara
por nosotros: ``Por favor, vuelvan a él y sean sus amigos''
\bibverse{21} Dios hizo que Jesús, quien nunca pecó, experimentara las
consecuencias del pecado para que nosotros pudiéramos tener un carácter
recto, así como Dios es recto\footnote{\textbf{5:21} O, ``pudiéramos
  llegar a ser rectos como él es recto.''}.

\hypertarget{section-5}{%
\section{6}\label{section-5}}

\bibverse{1} Como colaboradores de Dios, también les rogamos que no
acepten la gracia de Dios en vano. \bibverse{2} Tal como Dios dijo: ``En
el momento apropiado te escuché, y en el día de salvación te
salvé.''\footnote{\textbf{6:2} Isaías 49:8.} Créanme, ¡ahora es el
momento apropiado! ¡Ahora es el día de salvación! \bibverse{3} Nosotros
no ponemos obstáculos en el camino de nadie para que ningunno tropiece,
asegurándonos de que nadie critique la obra que hacemos. \bibverse{4} En
lugar de ello tratamos de demostrar que somos buenos siervos de Dios en
todas las formas posibles. Con mucha paciencia soportamos todo tipo de
problemas, dificultades y angustias. \bibverse{5} Hemos sido azotados,
llevados a la cárcel y atacados por turbas. Nos han hecho trabajar hasta
el cansancio, soportando noches sin dormir y con hambre. \bibverse{6}
Viviendo vidas irreprensibles en el conocimiento de Dios, con mucha
paciencia, siendo amables y llenos del Espíritu Santo, mostrando amor
sincero. \bibverse{7} Hablamos con fidelidad\footnote{\textbf{6:7} O
  ``palabra de verdad,'' refiriéndose al evangelio.}, viviendo en el
poder de Dios. Nuestras armas son lo verdadero y lo recto; atacamos con
nuestra mano derecha y nos defendemos con la izquierda\footnote{\textbf{6:7}
  Literalmente, ``armas de derecha e izquierda.'' Esto posiblemente se
  refiere al uso de una espada en la mano derecha, y un escudo en la
  mano izquierda.}. \bibverse{8} Nosotros seguimos, no importa si
recibimos honra o deshonra, si somos maldecidos o alabados. La gente nos
llama fraude, pero nosotros decimos la verdad. \bibverse{9} Somos
menospreciados, aunque somos reconocidos; nos han dado por muertos, pero
aún estamos vivos; nos han dado latigazos pero no hemos muerto.
\bibverse{10} ¡Nos han considerado como miserables, pero siempre estamos
gozosos; como pobres, pero hacemos ricos a muchos; nos han considerado
como desamparados, pero lo tenemos todo!

\bibverse{11} Les he hablado con franqueza, mis amigos de Corinto,
abriéndoles todo mi corazón. \bibverse{12} No les hemos negado nuestro
amor, pero ustedes sí lo han hecho. \bibverse{13} ¡Como si fueran mis
hijos, les ruego que correspondan, y amen con todo el corazón!

\bibverse{14} No se junten con los que no creen. ¿Acaso qué relación
tiene el bien con el mal? O ¿qué tienen en común la luz con las
tinieblas? \bibverse{15} ¿Podrían alguna vez estar de acuerdo Cristo y
el Diablo\footnote{\textbf{6:15} Literalmente, ``Belial.''}? ¿Cómo
podrían compartir juntos un creyente con un incrédulo? \bibverse{16}
¿Qué compromiso podría existir entre el templo de Dios con los ídolos?
Pues nosotros somos templo del Dios vivo, tal como Dios dijo: ``Viviré
en ellos y caminaré en medio de ellos. Yo seré su Dios, y ellos serán mi
pueblo.''\footnote{\textbf{6:16} Levítico 26:12 y Ezequiel 37:27.}
\bibverse{17} ``Así que abandónenlos y apártense de ellos, dice el
Señor. No toquen nada impuro, y los aceptaré.''\footnote{\textbf{6:17}
  Isaías 52:11 y Ezequiel 20:34, 41.} \bibverse{18} ``Seré como un Padre
para ustedes, y ustedes serán mis hijos e hijas, dice el Señor
Todopoderoso.''\footnote{\textbf{6:18} 2 Samuel 7:14 o 1 Crónicas 17:13.}

\hypertarget{section-6}{%
\section{7}\label{section-6}}

\bibverse{1} Queridos amigos, dado que tenemos estas promesas,
limpiémonos de todo lo que contamina nuestro cuerpo y espíritu,
procurando la santidad que nace de la reverencia a Dios. \bibverse{2}
¡Por favor, abran un espacio para nosotros en sus corazones! No le hemos
hecho mal a nadie, no hemos corrompido a nadie, ni nos hemos aprovechado
de nadie. \bibverse{3} No lo digo para condenarlos a ustedes, pues como
ya les dije, ustedes son muy importantes para nosotros, tanto, que
estamos dispuestos a vivir y morir con ustedes. \bibverse{4} Les hablo
con confianza porque estoy orgulloso de ustedes. Son una fuente de ánimo
para mí. Y estoy muy contento de ustedes a pesar de todas nuestras
dificultades.

\bibverse{5} Cuando llegamos a Macedonia, no tuvimos ni un minuto de
paz. Recibimos ataques por todas partes, por causa de conflictos
externos así como de miedos internos. \bibverse{6} Aun así, Dios, quien
alienta a los abatidos de corazón, nos animó con la llegada de Tito.
\bibverse{7} Y no solo con su llegada, sino con el ánimo que ustedes le
dieron a él. Él nos contó cuánto deseaban verme, cuán tristes y
preocupados estaban por mí, lo cual me hizo aún más feliz. \bibverse{8}
Aunque los hice entristecer con la carta que les escribí, no me
arrepiento, aunque sí me arrepiento porque la carta los haya
entristecido, pero fue solo por un poco tiempo. \bibverse{9} Ahora estoy
feliz, no por entristecerlos, sino porque esa tristeza los hizo cambiar.
Llegaron a sentir la tristeza de una manera que Dios aprueba, por lo
tanto no les hicimos daño de ninguna manera. \bibverse{10} La tristeza
que Dios quiere que sintamos es la que nos lleva al arrepentimiento y
trae salvación. Esta clase de tristeza no trae consigo ningún tipo de
remordimiento, pero la tristeza mundanal trae muerte. \bibverse{11}
Miren, por ejemplo, lo que ocurrió cuando tuvieron esta misma
experiencia de tristeza que viene de Dios. Recuerden cuán empeñados y
afanados se volvieron por defenderse, cuánto enojo sintieron por lo que
había sucedido, con cuanta seriedad asumieron las cosas, y cuánto anhelo
tenían por hacer lo recto; estaban muy preocupados y deseosos de que se
hiciera justicia. En todo esto ustedes demostraron que eran sinceros en
su deseo de hacer las cosas rectamente\footnote{\textbf{7:11} Pareciera
  que Pablo se está refiriendo a problemas anteriores, que necesitaban
  atención. Por ejemplo, el capítulo 2.}.

\bibverse{12} Así que cuando les escribí, no era para hablarles respecto
al agresor ni del agredido, sino para mostrarles cuán fieles son ustedes
a nosotros, ante los ojos de Dios. \bibverse{13} Esto nos anima en gran
manera. Además de este ánimo, nos alegró ver cuán feliz estaba Tito
porque ustedes le dieron fortaleza. \bibverse{14} Me
enorgullecí\footnote{\textbf{7:14} Aquí y en el resto de esta carta,
  Pablo habla de su jactancia. Esto debe tomarse como un cumplido
  dirigido a los otros, más que como orgullo respecto a sí mismo.} de
ustedes al hablar con él, y no me defraudaron. Así como todas las demás
cosas que les digo son verdaderas, mis elogios sobre ustedes hacia Tito
resultaron ser verdaderos también. \bibverse{15} Él se preocupa por
ustedes aún más al recordar que ustedes hicieron todo lo que él les
pidió y lo recibieron con mucho respeto. \bibverse{16} Me siento muy
feliz de poder confiar plenamente en ustedes.

\hypertarget{section-7}{%
\section{8}\label{section-7}}

\bibverse{1} Hermanos y hermanas, queremos contarles sobre la gracia de
Dios hacia las iglesias de Macedonia. \bibverse{2} Aunque han sufrido
mucha angustia, rebosan de felicidad; y aunque son muy pobres, también
rebosan de generosidad. \bibverse{3} Puedo dar testimonio de que dieron
todo lo que pudieron y, de hecho, más que eso. Por decisión propia
\bibverse{4} siguieron rogando con nosotros para tener parte en este
privilegio de participar en el ministerio al pueblo de Dios.
\bibverse{5} No solo hicieron lo que esperábamos que hicieran, sino que
se entregaron completamente al Señor y luego a nosotros, como Dios lo
quería. \bibverse{6} Así que hemos animado a Tito---ya que él fue quien
inició esta obra con ustedes---para que regrese y termine con ustedes
este ministerio de gracia.

\bibverse{7} Ya que ustedes tienen abundancia en todas las
cosas---confianza en Dios, conocimiento espiritual, total dedicación, y
amor por nosotros--- asegúrense de que esta abundancia que poseen
también llegue a este ministerio de dadivosidad. \bibverse{8} No los
estoy obligando a hacer esto, sino a que demuestren la sinceridad de su
amor, comparado con la dedicación de los otros\footnote{\textbf{8:8} Se
  presume que se refiere a las otras Iglesias, como las de Macedonia.}.
\bibverse{9} Porque ustedes conocen la gracia de nuestro Señor
Jesucristo. Que aunque era rico, se volvió pobre por ustedes, a fin de
que a través de su pobreza ustedes pudieran llegar a ser ricos.
\bibverse{10} Este es mi consejo: sería bueno que terminaran lo que
comenzaron. El año pasado ustedes fueron no solo los primeros en dar
sino también los primeros en querer hacerlo. \bibverse{11} Ahora,
terminen los planes que hicieron. Sean prestos para terminar así como lo
fueron para hacer planes, y den según lo que puedan dar. \bibverse{12}
Si hay disposición, es bueno que den de lo que tengan, y no lo que no
tienen. \bibverse{13} El propósito no es hacer que las cosas sean
fáciles para los demás y difíciles para ustedes, sino justas.
\bibverse{14} En este momento ustedes tienen más que suficiente para
suplir sus necesidades, y a la vez, cuando ellos tengan más que
suficiente podrán satisfacer las necesidades de ustedes. De esta manera
todos reciben un trato justo. \bibverse{15} Como dice la Escritura: ``El
que tenía mucho, no tenía en exceso, y el que no tenía mucho, tampoco
tenía muy poco.''\footnote{\textbf{8:15} Esto hace referencia a la
  recolección del maná, en Éxodo 16:8.}

\bibverse{16} Gracias a Dios que le dio a Tito la misma devoción que yo
tengo por ustedes. \bibverse{17} Aunque aceptó hacer lo que le dijimos,
viene a verlos porque realmente desea hacerlo, y porque ya lo había
decidido. \bibverse{18} También enviamos con él a un hermano que es
elogiado por todas las iglesias por su obra en la predicación de la
buena noticia. \bibverse{19} También fue designado por las iglesias para
que fuera con nosotros a entregar esta ofrenda que llevamos con
nosotros. Lo hacemos para honrar al Señor y para mostrar nuestro
ferviente deseo de ayudar a otros. \bibverse{20} Queremos evitar que
alguno pueda criticar la manera como usamos este regalo. \bibverse{21}
Nos interesa hacer las cosas de manera correcta, no solo a los ojos del
Señor, sino también ante los ojos de todos. \bibverse{22} También
enviamos con ellos a otro hermano que ha demostrado en muchas ocasiones
ser un hombre de confianza, y que está dispuesto a ayudar. Ahora tiene
aún más disposición de ayudar por la gran confianza que tiene en
ustedes. \bibverse{23} Si alguno pregunta sobre Tito, digan que es mi
compañero. Trabaja conmigo en favor de ustedes. Los otros hermanos son
representantes de las iglesias y que honran a Cristo. \bibverse{24} Así
que les ruego que los reciban antes que todas las demás iglesias y les
muestren su amor, demostrando así que tenemos razón en estar muy
orgullosos de ustedes.

\hypertarget{section-8}{%
\section{9}\label{section-8}}

\bibverse{1} Realmente no necesito escribirles sobre esta ofrenda para
el pueblo de Dios. \bibverse{2} Sé cuán prestos están para ayudar. De
hecho, elogié esto en Macedonia, diciendo que en Acaya ustedes han
estado prestos por más de un año, y que su entusiasmo ha animado a
muchos de ellos a dar. \bibverse{3} Pero envío a estos hermanos para que
los elogios que hago de ustedes no sean hallados falsos, y que estén
preparados, tal como dijeron que lo harían. \bibverse{4} Esto lo digo en
caso de que algunos de Macedonia lleguen conmigo y ustedes no estén
listos. Nosotros, -- y sabemos que ustedes también -- nos sentiríamos
muy avergonzados de que este proyecto fracasara. \bibverse{5} Por eso
decidí pedir a estos hermanos que los visiten antes, y finalicen los
arreglos necesarios para recoger esta ofrenda, de tal modo que esté
lista como un regalo y no como una obligación.

\bibverse{6} Quisiera recordarles esto: Si siembran poco, cosecharán
poco; pero si siembran con abundancia, cosecharán abundancia.
\bibverse{7} Cada uno debe dar según lo que haya decidido dar, y no de
mala gana o por obligación, porque Dios ama a los que dan con espíritu
alegre. \bibverse{8} Dios puede proveerles todo para que nunca les falte
nada; con abundancia, para que ayuden a otros también. \bibverse{9} Como
dice la Escritura: ``Él da con generosidad a los pobres. Su generosidad
es eterna.''\footnote{\textbf{9:9} Salmos 112:9. En el contexto del
  salmo, se refiere a un hombre generoso.} \bibverse{10} Dios, quien
provee la semilla para el sembrador y da el pan para la comida, proveerá
y multiplicará su ``semilla'' y aumentará sus cosechas de generosidad.
\bibverse{11} Serán ricos en todas las cosas, a fin de que puedan ser
siempre generosos y su generosidad lleve a otros a estar agradecidos con
Dios. \bibverse{12} Cuando sirvan de esta forma, no solo se satisfacen
las necesidades del pueblo de Dios, sino que muchos darán gracias a él.
\bibverse{13} Al dar esta ofrenda, demuestran su carácter y los que la
reciben agradecerán a Dios por su obediencia, pues ella demuestra su
compromiso con la buena nueva de Cristo y su generosidad al darles a
ellos y a todos los demás. \bibverse{14} Entonces ellos orarán por
ustedes con más amor, por la abundante gracia de Dios obrando por medio
de ustedes. \bibverse{15} ¡Gracias a Dios porque su don es más grande
que lo que las palabras pueden expresar!

\hypertarget{section-9}{%
\section{10}\label{section-9}}

\bibverse{1} Yo mismo, Pablo, los insto personalmente, por la bondad y
la ternura de Cristo. El mismo Pablo que es ``tímido'' cuando está con
ustedes, pero que es ``osado'' cuando no está allá\footnote{\textbf{10:1}
  Pablo pareciera estar enfrentando alguna acusación que se había hecho
  contra él.}. \bibverse{2} Les ruego para que la próxima vez que esté
con ustedes, no tenga que ser tan duro como pienso que tendré que ser,
confrontando abiertamente a los que piensan que nosotros nos comportamos
de forma mundana. \bibverse{3} Aunque vivimos en este mundo, no peleamos
como el mundo. \bibverse{4} Nuestras armas no son de este mundo, pero
tenemos el poder de Dios que destruye fortalezas del pensamiento humano,
y derriba teorías engañosas. \bibverse{5} Todo muro que se interpone
contra el conocimiento de Dios es derribado. Todo pensamiento rebelde es
capturado y conducido a un acuerdo de obediencia a Cristo. \bibverse{6}
Cuando ustedes estén obedeciendo a Cristo por completo, entonces
estaremos listos para castigar cualquier desobediencia.

\bibverse{7} ¡Miren lo que tienen delante de sus ojos! Todo el que crea
que pertenece a Cristo debe pensarlo dos veces, porque así como ellos
pertenecen a Cristo, nosotros también le pertenecemos. \bibverse{8}
Aunque pareciera que me enorgullezco mucho de nuestra autoridad, no me
avergüenzo de ello. El Señor nos dio esta autoridad para edificarlos a
ustedes, no para destruirlos. \bibverse{9} No intento asustarlos con mis
cartas. \bibverse{10} La gente dice: ``Sus cartas son duras y severas,
pero en persona es débil, y es un orador inútil.'' \bibverse{11} Este
tipo de personas deberían comprender que lo que decimos por cartas
cuando no estamos allá, lo haremos cuando sí estemos allá. \bibverse{12}
No somos tan arrogantes como para compararnos con los que se tienen en
un concepto muy alto. ¡Los que se miden a sí mismos, y se comparan
consigo mismos, son totalmente necios! \bibverse{13} Pero no nos
jactamos con términos extravagantes que no puedan medirse. Sencillamente
medimos lo que hemos hecho usando el sistema de medida que Dios nos ha
dado, y eso los incluye a ustedes. \bibverse{14} No estamos abusando de
nuestra autoridad al decir esto, como si no hubiéramos estado entre
ustedes, porque realmente sí estuvimos allí y compartimos con ustedes la
buena noticia de Cristo\footnote{\textbf{10:14} Pablo está diciendo que
  él estaba trabajando dentro del marco de su comisión para predicar el
  evangelio cuando vino a Corinto. Puede ser que algunos estaban
  diciendo que Corinto realmente no era parte de la jurisdicción de
  Pablo.}. \bibverse{15} Nosotros no nos estamos jactando con términos
extravagantes que no puedan medirse, reclamando crédito por lo que otros
han hecho. Por el contrario, esperamos que a medida que su fe en Dios
aumenta, nuestra obra entre ustedes crezca en gran manera. \bibverse{16}
Entonces podremos compartir la buena noticia en lugares que están más
allá, sin jactarnos de lo que ya ha sido hecho por otros\footnote{\textbf{10:16}
  Pablo desea evitar problemas en cuento a quién recibe crédito por
  hacer una cosa y otra, y preferiría seguir hacia adelante con la obra
  de la predicación de la buena noticia.}. \bibverse{17} Si alguno
quiere jactarse, que se jacte en el Señor.''\footnote{\textbf{10:17}
  Jeremías 9:24.} \bibverse{18} No reciben respeto los que se elogian a
sí mismos, sino a los que el Señor elogia.

\hypertarget{section-10}{%
\section{11}\label{section-10}}

\bibverse{1} Espero que puedan soportarme unas cuantas necedades más.
¡Bueno, de hecho, ya me soportan a mí mismo! \bibverse{2} Sufro de una
agonía por el celo divino que siento por ustedes, pues les prometí un
solo esposo---Cristo---a fin de presentarlos a ustedes como una mujer
virgen y pura para él. \bibverse{3} Me preocupa que, de algún modo, así
como la serpiente engañó a Eva con su astucia, ustedes puedan ser
descarriados en su forma de pensar sobre su compromiso sincero y puro
con Cristo. \bibverse{4} Si alguno llega a hablarles sobre un Jesús
distinto al que nosotros hemos compartido con ustedes, fácilmente
ustedes concuerdan con ellos\footnote{\textbf{11:4} En otras palabras,
  son muy tolerantes con los que traen una comprensión muy distinta de
  la buena noticia.}, aceptando un espíritu diferente al que han
recibido, y una buena noticia distinta a la que creyeron.

\bibverse{5} No me considero inferior a estos ``súper apóstoles.''
\bibverse{6} Aunque no sea muy talentoso para dar discursos, sé de lo
que hablo. Les hemos explicado esto claramente y de todas las maneras
posibles. \bibverse{7} ¿Fue un error que me humillara para exaltarlos a
ustedes, siendo que compartí la buena noticia con ustedes sin beneficio
económico alguno? \bibverse{8} Despojé a otras iglesias, recibiendo pago
de ellas para poder trabajar en favor de ustedes. \bibverse{9} Cuando
estuve allá con ustedes y necesité algo, no fui carga para nadie, porque
los creyentes que venían de Macedonia se hicieron cargo de mis
necesidades. Estuve decidido a no ser carga para ustedes y nunca lo
seré. \bibverse{10} Esto es tan cierto como la verdad de que Cristo está
en mí: ¡No hay nadie en toda Acaya que me impida jactarme de esto!
\bibverse{11} ¿Y por qué? ¿Acaso es porque no los amo? ¡Dios mismo sabe
que sí los amo! \bibverse{12} Y seguiré haciendo lo que siempre he
hecho, para eliminar cualquier oportunidad que otros puedan tener de
jactarse de que su obra es igual a la nuestra. \bibverse{13} Estas
personas son falsos apóstoles, obreros deshonestos, que
fingen\footnote{\textbf{11:13} Literalmente, ``se transforman en.''
  También aparece en el versículo 14.} ser apóstoles de Cristo.
\bibverse{14} No se sorprendan de esto porque incluso Satanás mismo
finge ser un ángel de luz. \bibverse{15} Así que no se extrañen de que
los que le sirven finjan ser agentes del bien. Pero su final será
conforme a sus obras.

\bibverse{16} Permítanme decirlo nuevamente: por favor, no crean que
estoy siendo necio. No obstante, si así lo creen, acéptenme como un
necio, y permítanme jactarme un poco\footnote{\textbf{11:16} Pablo
  sugiere que a él también debería permitírsele jactarse como lo hacían
  los falsos apóstoles.}. \bibverse{17} Lo que estoy diciendo no es como
lo diría el Señor, con todo este orgullo. \bibverse{18} Pero como muchos
andan por ahí jactándose como lo hace el mundo, entonces permítanme
hacerlo también. \bibverse{19} (Ustedes son felices de soportar necios,
pues son muy sabios\footnote{\textbf{11:19} Evidentemente, es un
  comentario sarcástico o irónico, así como lo que sigue al versículo
  \ldots{}}) \bibverse{20} Soportan a personas que los esclavizan, que
les roban, que los explotan, que los humillan con su arrogancia, y que
los abofetean. \bibverse{21} ¡Lamento tanto que nosotros fuimos muy
débiles para soportar algo así! Pero sean cuales sean las razones por
las cuales la gente se jacta, me atrevo a hacerlo también. (En esto
hablo como necio una vez más).

\bibverse{22} ¿Es porque son hebreos? Yo también. ¿Es porque son
israelitas? Yo también. ¿Es porque son descendientes de Abrahán? Yo
también lo soy. \bibverse{23} ¿Es porque son siervos de Cristo? (Esto
podría sonar como una locura). Pero yo he hecho mucho más. He trabajado
con más esfuerzo, me han llevado preso en muchas más ocasiones, me han
azotado más veces de las que puedo contar, he enfrentado la muerte una y
otra vez. \bibverse{24} Cinco veces he recibido de los judíos cuarenta
latigazos menos uno. \bibverse{25} Tres veces fui golpeado con palos,
una vez fui apedreado, tres veces naufragué. Una vez duré veinticuatro
horas a la deriva en el océano. \bibverse{26} Durante muchas ocasiones
he afrontado los peligros de cruzar ríos, encontrarme con pandillas de
atracadores, ataques de mis propios conciudadanos, así como de
extranjeros\footnote{\textbf{11:26} Literalmente, ``gentiles.''}. He
enfrentado peligros en las ciudades, en los desiertos, y en el mar. He
enfrentado el peligro de parte de personas que fingen ser cristianos.
\bibverse{27} He enfrentado trabajo duro y luchas, muchas noches sin
dormir, hambre y sed, a menudo he estado sin comida, con frío, y sin
ropa para cubrirme del frío.

\bibverse{28} Aparte de todo esto, cada día enfrento las preocupaciones
de ocuparme de todas las iglesias. \bibverse{29} ¿Quién es débil? ¿Acaso
no me siento débil también? ¿Quién es conducido a pecar sin que yo arda
de enojo? \bibverse{30} Si tengo que jactarme, me jactaré en lo débil
que soy. \bibverse{31} El Dios y Padre del Señor Jesús---sea él alabado
por siempre---sabe que no miento. \bibverse{32} Mientras estaba en
Damasco, el gobernador que estaba bajo autoridad del Rey Aretas mandó a
custodiar la ciudad para capturarme. \bibverse{33} Pero me ayudaron a
descender en una canasta por el muro de la ciudad, y hui de él.

\hypertarget{section-11}{%
\section{12}\label{section-11}}

\bibverse{1} Supongo que tengo que jactarme, aunque eso no ayuda
realmente. Permítanme hablarles ahora de las visiones y revelaciones de
parte del Señor. \bibverse{2} Conozco a un hombre en Cristo que hace
catorce años fue llevado al tercer cielo (si fue físicamente con su
cuerpo, o si fue fuera del cuerpo, no lo sé, pero Dios sabe).
\bibverse{3} Sé que este hombre (si fue físicamente con su cuerpo, o
fuera de él, no lo sé, pero Dios lo sabe), \bibverse{4} fue llevado al
Paraíso, y escuchó cosas tan maravillosas que no se pueden explicar, en
palabras tan sagradas que ningún ser humano podría decir. \bibverse{5}
De algo como eso me jactaría, pero no me jactaré de mí mismo, sino de
mis debilidades. \bibverse{6} No sería un necio si quisiera jactarme,
porque estaría diciendo la verdad. Pero no me jactaré, para que nadie me
tenga en un concepto más alto que lo que ve que hago o me oyen decir.
\bibverse{7} Además, como las revelaciones fueron tan asombrosas, y para
que no pudiera enorgullecerme de ello, se me dio una ``espina en la
carne''\footnote{\textbf{12:7} Probablemente se refiere a algún problema
  físico en el cuerpo de Pablo.}---un mensajero de Satanás, para herirme
a fin de que no me volviera orgulloso. \bibverse{8} Le rogué al Señor
tres veces para deshacerme de este problema. \bibverse{9} Pero él me
dijo: ``Mi gracia te bastará, pues mi poder se hace eficaz en la
debilidad.'' Por eso me jacto felizmente de mis debilidades, para que
habite en mí el poder de Cristo. \bibverse{10} Por lo tanto valoro las
debilidades, los insultos, los problemas, las persecuciones y las
dificultades que sufro por causa de Cristo. ¡Porque cuando soy débil,
entonces soy fuerte!

\bibverse{11} Estoy hablando como necio, pero ustedes me obligaron a
hacerlo. Ustedes deberían haber estado hablando bien de mí, pues de
ninguna manera soy inferior a estos ``súper apóstoles\footnote{\textbf{12:11}
  Ver 11:5.}, aunque no soy nada. \bibverse{12} Sin embargo, las señales
de apostolado fueron presentadas pacientemente ante ustedes: señales,
maravillas, y milagros poderosos. \bibverse{13} ¿Acaso en qué fueron
ustedes inferiores a las demás iglesias, sino en el hecho de que no fui
una carga para ustedes? ¡Les ruego que me perdonen por hacerles
mal!\footnote{\textbf{12:13} Otra vez, una afirmación que debería
  considerarse como irónica; tal como en el versículo 16.} \bibverse{14}
Estoy preparándome para visitarlos por tercera vez y no seré carga para
ustedes. ¡No quiero las cosas que tienen, los quiero a ustedes! Después
de todo, los niños no deben cuidar de los padres, sino los padres de los
hijos. \bibverse{15} Gustosamente me gastaré y me desgastaré por
ustedes. Si los amo mucho más, ¿acaso me amarán menos ustedes?
\bibverse{16} Pues, incluso si es así, no fui carga para ustedes.
¡Quizás estaba siendo taimado y los engañé con mis estrategias astutas!
\bibverse{17} ¿Pero acaso me aproveché de ustedes mediante alguno de los
que envié? \bibverse{18} Obligué a Tito para que fuera a verlos, y envié
a otro hermano con él. ¿Acaso Tito se aprovechó de ustedes? No, porque
ambos tenemos el mismo espíritu y usamos los mismos métodos.
\bibverse{19} Quizás ustedes están pensando que todo este tiempo hemos
estado tratando de defendernos a nosotros mismos. No, hablamos de Cristo
ante Dios. Todo lo que hacemos, amigos, es por beneficio de ustedes.
\bibverse{20} Cuando voy de visita, me preocupo de no encontrarlos como
quisiera, y de que ustedes no me vean como quisieran verme. Me temo que
habrá discusiones, celos, enojo, calumnia, chisme, arrogancia, y
desorden. \bibverse{21} Me temo que cuando vaya de visita, mi Dios me
humillará en presencia de ustedes, y que estaré lamentándome por muchos
que han pecado antes, y que aún no se han arrepentido de impureza,
inmoralidad sexual, y los actos indecentes que cometieron.

\hypertarget{section-12}{%
\section{13}\label{section-12}}

\bibverse{1} Esta es mi tercera visita. ``Todo cargo debe ser verificado
por dos o tres testigos.''\footnote{\textbf{13:1} Deuteronomio 19:15.}
\bibverse{2} Ya advertí a los que entre ustedes estaban en pecado cuando
fui por segunda vez. Aunque no estoy allí, les advierto a ellos una vez
más---y al resto de ustedes---que cuando los visite no dudaré en tomar
medidas contra ellos, \bibverse{3} puesto que están demandando una
prueba de que Dios está hablando a través de mí. Él no es débil para
tratarlos; más bien obra con poder en medio de ustedes. \bibverse{4}
Aunque fue crucificado en debilidad, ahora vive mediante el poder de
Dios. Nosotros también somos débiles en él, pero ustedes podrán ver que
vivimos con él mediante el poder de Dios. \bibverse{5} Examínense
ustedes mismos y vean si están confiando en Dios. Pónganse a prueba. ¿No
se dan cuenta de que Jesucristo está en\footnote{\textbf{13:5} O ``unido
  a.''} ustedes? A menos que hayan fallado en la prueba\ldots{}
\bibverse{6} No obstante, espero que comprendan que nosotros no hemos
fallado.

\bibverse{7} Rogamos a Dios que ustedes no hagan nada malo, no para que
nosotros podamos mostrar que pasamos la prueba, sino para que ustedes
puedan hacer lo recto, aunque nos haga parecer como un fracaso.
\bibverse{8} No podemos hacer nada contra la verdad, solo en favor de la
verdad. \bibverse{9} Nos alegra cuando somos débiles, y ustedes son
fuertes. Oramos para que sigan mejorando. \bibverse{10} Por eso les
escribo sobre esto ahora que no estoy con ustedes, para que cuando sí
esté allá, no tenga necesidad de tratarlos con dureza e imponiendo mi
autoridad. El Señor me dio autoridad para edificar, no para destruir.

\bibverse{11} Finalmente, hermanos y hermanas, me despido. Sigan
mejorando espiritualmente. Anímense unos a otros. Estén en armonía.
Vivan en paz, y que el Dios de amor y paz esté con ustedes.
\bibverse{12} Salúdense unos a otros con amor cristiano. \bibverse{13}
Todos los creyentes aquí les envían su saludo. \bibverse{14} Que la
gracia del Señor Jesucristo, el amor de Dios, y la comunión del Espíritu
Santo esté con todos ustedes.
