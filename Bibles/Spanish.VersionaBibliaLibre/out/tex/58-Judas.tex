\hypertarget{section}{%
\section{1}\label{section}}

\bibverse{1} Esta carta viene de parte de Judas, siervo de Jesucristo, y
hermano de Santiago. Escribo a los que son llamados y amados por Dios,
el Padre, y que son guardados a salvo por Jesucristo: \bibverse{2} ¡Que
la misericordia, la paz y el amor de Dios aumente en su experiencia!

\bibverse{3} Amigos, desde antes ya anhelaba la oportunidad de
escribirles acerca de la salvación de la cual somos partícipes. Pero
ahora necesito escribirles urgentemente y animarlos a defender
enérgicamente la verdad acerca de Dios,\footnote{\textbf{1:3} 3.
  Literalmente: ``fe,'' o ``creencia.''} que fue dada una vez y para
siempre\footnote{\textbf{1:3} 3. ``Dada una vez y para siempre'': puesto
  que Dios reveló la verdad sobre sí mismo en muchas ocasiones a lo
  largo de la historia, probablemente la idea aquí es la revelación
  suprema del mismo Dios por sí mismo en la persona de Jesús.} al pueblo
santo de Dios. \bibverse{4} Pues algunos se han infiltrado entre
ustedes. Ya antes se escribió acerca de ellos y fueron condenados,
porque son personas malvadas que pervierten la gracia de Dios,
convirtiéndola en una licencia para la inmoralidad, mientras que también
niegan a nuestro Señor y maestro Jesucristo.

\bibverse{5} Aunque esto ya lo saben, quiero recordarles que aunque el
Señor salvó a su pueblo de la tierra de Egipto, después destruyó a los
que eran incrédulos. \bibverse{6} Incluso a los ángeles que estaban
inconformes con las posiciones que Dios les había dado y que abandonaron
sus debidos sitios, él los ha puesto eternamente\footnote{\textbf{1:6}
  6. Eternamente en el sentido de su consecuencia, no en su duración,
  como es evidente por el contexto en que este aspecto ``externo'' acaba
  con el juicio. Lo mismo se aplica al ``fuego eterno'' en el versículo
  7 que se ejemplifica por Sodoma y Gomorra: los efectos son eternos,
  pero tales ciudades no están ardiendo en fuego ahora, ni ``para
  siempre''.''} en cadenas de oscuridad hasta el gran Día del Juicio.
\bibverse{7} Del mismo modo, Sodoma y Gomorra, y las naciones cercanas
que practicaban la inmoralidad y perversión sexual, son presentadas como
ejemplo de aquellos que experimentan el castigo del fuego eterno.

\bibverse{8} Del mismo modo, estos soñadores\footnote{\textbf{1:8} 8. Se
  refiere a las personas mencionadas en el versículo 4.} contaminan sus
cuerpos, desprecian la autoridad, e insultan a los seres celestiales.
\bibverse{9} Incluso el arcángel Miguel, cuando discutía con el diablo
sobre el cuerpo de Moisés, no se animó a condenarlo con insultos
difamatorios, sino que dijo: ``Que el Señor te reprenda.'' \bibverse{10}
Pero estas personas ridiculizan lo que no comprenden; y lo que
entienden, eso siguen, por instinto, como animales que no tienen razón.
Esto es lo que los destruye. \bibverse{11} ¡Cuán grande problema tienen!
Pues han seguido el camino de Caín. Así como Balaam y su engaño, ellos
se han dejado llevar por el afán de lucro. Como la rebelión de Coré, se
han destruido a sí mismos. \bibverse{12} Estas personas participan con
ustedes de las comidas de compañerismo y las echan a perder, porque son
pastores egoístas que no tienen el mínimo sentido de vergüenza, pues
solo se preocupan de ellos mismos. Son como nubes llevadas por el viento
y que no producen lluvia. Son árboles sin hojas ni frutos, muertos doce
veces y extraídos desde las raíces. \bibverse{13} Son olas violentas del
océano, que arrojan la espuma de su propia desgracia. Son estrellas
falsas, condenadas para siempre a la más negra oscuridad.

\bibverse{14} Enoc, quien vivió siete generaciones después de Adán,
profetizó sobre estas personas: ``¡Miren! El Señor viene con miles y
miles de sus santos \bibverse{15} para juzgar a todos, para revelar las
cosas malas que han hecho, y todas las cosas terribles que los pecadores
hostiles han dicho contra él.'' \bibverse{16} Tales personas son
gruñonas, que siempre están quejándose. Siguen sus propios deseos malos,
y hablan con jactancia de sí mismos, y halagan a otros para lograr sus
propios fines.

\bibverse{17} Pero ustedes, mis queridos amigos, recuerden, por favor,
lo que les dijeron los apóstoles de nuestro Señor Jesucristo.
\bibverse{18} Porque ellos les explicaron que en los últimos tiempos
vendrían mofadores, que seguirían sus propios deseos malvados.
\bibverse{19} Ellos causan divisiones, son personas mundanas que no
tienen el Espíritu.

\bibverse{20} Pero ustedes, amigos míos, deben edificarse a sí mismos
por la fe en Dios. Oren en el Espíritu Santo, \bibverse{21} manténganse
a salvo en el amor de Dios, y esperen la misericordia de nuestro Señor
Jesucristo, que otorga vida eterna. \bibverse{22} Muestren bondad con
los que dudan. \bibverse{23} Salven a todos los que puedan,
arrebatándolos del fuego. Muestren misericordia, pero con mucho cuidado,
aborreciendo incluso las ``vestiduras'' contaminadas por la naturaleza
pecaminosa de los humanos.\footnote{\textbf{1:23} 23. En otras palabras,
  mientras somos misericordiosos con el pecador, debemos tener cuidado
  con la ``vestidura pecaminosa'' que tienen los seres humanos, para
  evitar ``infectarnos'' nosotros mismos.}

\bibverse{24} Ahora, a Aquél que puede guardarlos sin caer,

y que puede llevarlos a su gloriosa presencia sin falta, y con gran
gozo,

\bibverse{25} al único Dios, nuestro Salvador, por medio de Jesucristo
nuestro Señor, sea la gloria, la majestad, el poder y la autoridad,
desde siempre, ahora, y para siempre.

Amén.
