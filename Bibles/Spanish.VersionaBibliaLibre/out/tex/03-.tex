\hypertarget{section}{%
\section{1}\label{section}}

\bibverse{1} Estos fueron los nombres de los hijos de Israel (Jacob) que
vinieron con él y sus familias a Egipto: \bibverse{2} Rubén, Simeón,
Leví y Judá; \bibverse{3} Isacar, Zabulón y Benjamín; \bibverse{4} Dan y
Neftalí, Gad y Aser. \bibverse{5} Allí Jacob tuvo 70 descendientes,
incluyendo a José, que ya estaba en Egipto.

\bibverse{6} Finalmente José, todos sus hermanos, y toda esa generación
murieron. \bibverse{7} Sin embargo, los israelitas tenían muchos hijos y
su número aumentaba rápidamente. De hecho, eran tantos que se volvieron
muy poderosos, y el país estaba lleno de ellos.

\bibverse{8} Entonces subió al poder un nuevo rey que no tenía ningún
conocimiento acerca deJosé.\footnote{1.8 Se cree que se refiere a una
  dinastía egipcia diferente.} \bibverse{9} Este rey se reunió con sus
compatriotas egipcios y les dijo: ``Debemos tener cuidado con estos
israelitas, pues son más numerosos y más poderosos que nosotros.
\bibverse{10} Tenemos que hacer un plan para evitar que sigan
multiplicándose, porque que si llega a haber una guerra, se pondrán del
lado de nuestros enemigos, lucharán contra nosotros, y huirán del
país''.

\bibverse{11} Entonceslos egipcios comenzaron a obligarlos a hacer
trabajos forzados y asignaron capataces para que estuvieran a cargo de
ellos. Los usaron para construir las ciudades de almacenamiento de Pitón
y Ramsés. \bibverse{12} Pero cuanto más maltrataban a los israelitas,
más se multiplicaban y se extendían, y también los egipcios los
detestaban\footnote{1.12 ``Detestaban'' o ``temían''.} aún más.
\bibverse{13} Los egipcios trataban a los israelitas con violencia,
\bibverse{14} haciendo de sus vidas una miseria. Los obligaban a hacer
trabajos duros, construyendo con mortero y ladrillos, y haciendo todo
tipo de trabajo pesado en los campos. Y en medio de todo este trabajo
duro los trataban con crueldad.

\bibverse{15} Entonces el rey les dio órdenes a las parteras hebreas
Sifra y Fúa. \bibverse{16} Y les dijo: ``Cuando ayuden a las mujeres
hebreas durante el parto, si ven que es un niño, mátenlo; pero si es una
niña, déjenla vivir''. \bibverse{17} Pero como las parteras respetaban a
Dios, no hicieron lo que el rey de Egipto les había ordenado, sinoque
dejaban vivir a los niños también.

\bibverse{18} Entonces el rey de Egipto llamó a las parteras y les
preguntó: ``¿Por qué han hecho esto, y han dejado vivir a los niños
varones?''

\bibverse{19} ``Las mujeres hebreas no son como las egipcias'', le
dijeron las parteras al Faraón. ``Dan a luz más fácilmente, y tienen a
sus hijos antes de que lleguen las parteras''.

\bibverse{20} Y Dios trató bien a las parteras, y el pueblo aumentó en
número, así que había aún muchos más de ellos. \bibverse{21} Y como las
parteras reverenciaban a Dios, él les dio familias propias.

\bibverse{22} Entonces el Faraón emitió esta orden a todo su pueblo:
``Arrojen al Nilo a todo niño hebreo que nazca, y por el contrario,
dejen vivir a las niñas''.

\hypertarget{section-1}{%
\section{2}\label{section-1}}

\bibverse{1} Fue por esta época que un hombre de la tribu de Levi se
casó con una mujer, también levita. \bibverse{2} Ella quedó embarazada y
tuvo un hijo. Y viendo que era un bebé precioso, lo escondió durante
tres meses. \bibverse{3} Pero cuando ya no pudo esconderlo más, cogió
una cesta de papiro y la cubrió con alquitrán. Luego puso a su bebé en
la cesta y lo colocó entre los juncos, a lo largo de la orilla del Nilo.
\bibverse{4} Yla hermana del niño esperaba a cierta distancia,
vigilándolo.

\bibverse{5} Entonces la hija del Faraón llegó para bañarse en el Nilo.
Sus criadas caminaban por la orilla del río, y cuando ella vio la cesta
entre los juncos, envió a su criada a buscarla y traérsela. \bibverse{6}
Al abrirla, vio al niño que lloraba y sintió pesar por él. ``Este debe
ser uno de los niños hebreos'', dijo.

\bibverse{7} Entoncesla hermana del niño le preguntó a la hija del
Faraón: ``¿Desea que vaya a buscar a una de las mujeres hebreas para que
lo cuide por usted?''

\bibverse{8} ``Sí, ve y hazlo'', respondió ella. Así que la niña fue y
llamó a la madre del bebé para que viniera.

\bibverse{9} ``Toma a este niño y amamántalo por mí'', le dijo la hija
del Faraóna la madre del niño. ``Yo misma te pagaré''. Así que su madre
se lo llevó a casa y lo cuidó.

\bibverse{10} Cuando el niño creció, se lo llevó a la hija del Faraón,
quien lo adoptó como su hijo. Ella lo llamó Moisés\footnote{2.10
  ``Moisés'' suena como la palabra hebrea ``sacar''. En egipcio es una
  abreviatura que significa ``hijo de''.}, porque dijo: ``Yo lo saqué
del agua''.

\bibverse{11} Más tarde, cuando Moisés había crecido, fue a visitar a su
pueblo, los hebreos. Los vio haciendo trabajos forzados. También vio a
un egipcio golpeando a un hebreo, uno de su propio pueblo. \bibverse{12}
Entonces miró a su alrededor para asegurarse de que nadie estuviera
mirando, yluego mató al egipcio y enterró su cuerpo en la arena.

\bibverse{13} Al día siguiente, regresó y vio a dos hebreos peleando
entre sí. Entonces le dijo al culpable: ``¿Por qué golpeas a uno de los
tuyos?''

\bibverse{14} ``¿Quién te ha encargado como juez sobre nosotros?'',
respondió el hombre. ``¿Acaso vas a matarme como lo hiciste con el
egipcio?''

Entonces Moisés se asustó por esto y se dijo a sí mismo: ``¡La gente
sabe lo que he hecho!''

\bibverse{15} Cuando el Faraón se enteró, trató de mandar a matar a
Moisés, pero Moisés huyó del Faraón y se fue a vivir a Madián.

Un día, mientras estaba sentado junto a un pozo, \bibverse{16} las siete
hijas del sacerdote de Madiánvinieron a buscar agua para llenar los
bebederos a fin de que el rebaño de su padre pudiera beber.
\bibverse{17} Entonces llegaron unos pastores y las echaron de allí,
pero Moisés intervino y las puso a salvo, y le dio de beber a su rebaño.

\bibverse{18} Cuando llegaron a casa, su padre Reuel les preguntó:
``¿Cómo es que hoy han regresado tan rápido?''

\bibverse{19} ``Un egipcio nos rescató de unos pastores que nos
atacaron'', respondieron. ``Incluso nos trajo agua para que el rebaño
pudiera beber''.

\bibverse{20} ``¿Y dónde está?'' le preguntó Reuel a su hija. ``No lo
dejaste allí, ¿verdad? ¡Ve e invítalo a comer con nosotros!''

\bibverse{21} Y Moisés aceptó quedarse con el hombre, quien arregló que
su hija Séfora se casara con Moisés. \bibverse{22} Ella tuvo un hijo, y
Moisés le puso el nombre de Gersón,\footnote{2.22 ``Gersón'' suena como
  ``Allí hay un extranjero''.} porque dijo: ``Soy un exiliado que vive
en un país extranjero''.

\bibverse{23} Años más tarde, el rey de Egipto murió. Pero los
israelitas seguían gimiendo por su duro trabajo. Su clamor pidiendo
ayuda en medio de sus dificultades llegó hasta Dios. \bibverse{24} Dios
escuchó sus gemidos yse acordó de su pacto con Abraham, Isaac, y Jacob.
\bibverse{25} Además Dios mirabacon compasión a los israelitas, y se
preocupaba por ellos.\footnote{2.25 ``Se preocupaba por ellos'':
  literalmente, ``sabía''.}

\hypertarget{section-2}{%
\section{3}\label{section-2}}

\bibverse{1} Moisés era un pastor que cuidaba el rebaño de
Jetro,\footnote{3.1 ``Jetro'': Otro nombre de Reuel.}su suegro, el
sacerdote de Madián. Condujo el rebaño lejos en el desierto hasta que
llegó al monte de Dios, el monte Horeb.+ 3.1 ``Monte Horeb'': Otro
nombre para el Monte Sinaí. \bibverse{2} Allí el ángel del Señor se le
apareció en una llama de fuego desde dentro de un arbusto. Moisés miró
con atención y vio que, aunque la zarza estaba ardiendo, no se estaba
quemando.

\bibverse{3} ``Iré a echar un vistazo'', se dijo a sí mismo Moisés. ``Es
muy extraño ver un arbusto que no se queme''.

\bibverse{4} Cuando el Señor vio que Moisés venía a echar un vistazo,
Dios le llamó desde dentro de la zarza: ``¡Moisés! ¡Moisés!

``Aquí estoy'', respondió Moisés.

\bibverse{5} ``¡No te acerques más!'' le dijo Dios. ``Quítate las
sandalias porque estás parado en tierra sagrada''. \bibverse{6} Luego
dijo: ``Soy el Dios de tu padre, el Dios de Abraham, el Dios de Isaac y
el Dios de Jacob''. Moisés se cubrió el rostro, porque tuvo miedo de
mirar a Dios.

\bibverse{7} ``Soyplenamente consciente de la miseria de mi pueblo en
Egipto'', le dijo el Señor. ``Los he escuchado gemir por culpa de sus
capataces. Sé cuánto están sufriendo. \bibverse{8} Por eso he descendido
para rescatarlos de la opresión egipcia y para llevarlos desde ese país
a una tierra fértil y amplia, una tierra donde fluye leche y miel, donde
actualmente viven los cananeos, los hititas, los amorreos, los ferezeos,
los heveos y los jebuseos. \bibverse{9} Escucha ahora: El clamor de los
israelitas ha llegado hasta mí, y he visto cómo los egipcios los
maltratan. \bibverse{10} Ahora debes irte, porque yo te envío donde
elFaraón para que saques a mi pueblo Israel de Egipto''.

\bibverse{11} Pero Moisés le dijo a Dios: ``¿Por qué yo? ¡Yo soy un don
nadie! ¡No podré ir ante el Faraón y sacar a los israelitas de Egipto!''

\bibverse{12} ``Yo estaré contigo'', respondió el Señor, ``y esta será
la señal de que soy yo quien te envía: cuando hayas sacado al pueblo de
Egipto, adorarás a Dios en este mismo monte''.

\bibverse{13} Entonces Moisés dijo a Dios: ``Mira, si yo fuera donde los
israelitas y les dijera: `El Dios de sus padres me ha enviado a
ustedes', y ellos me preguntaran: `¿Cómo se llama?', ¿qué les diré
entonces?''.

\bibverse{14} Dios le respondió a Moisés: ``Yo soy'' el que soy. Dile
esto a los israelitas: ``Yo soy'' me ha enviado a ustedes''.

\bibverse{15} Entonces Dios le dijo a Moisés: ``Diles a los israelitas:
'El Señor, el Dios de sus padres, el Dios de Abraham, el Dios de Isaac y
el Dios de Jacob, me ha enviado a ustedes. Este es mi nombre para
siempre, el nombre con el que me llamarás en todas las generaciones
futuras''.

\bibverse{16} Ve y llama a todos los ancianos de Israel para que se
reúnan contigo. Diles: ``El Señor, el Dios de tus padres, se me ha
aparecido, el Dios de Abraham, Isaac y Jacob. Él dijo: ``He prestado
mucha atención a lo que te ha pasado en Egipto. \bibverse{17} He
decidido sacarlos de la miseria que están teniendo en Egipto y llevarlos
a la tierra de los cananeos, hititas, amorreos, ferezeos, heveos y
jebuseos, una tierra que fluye leche y miel''.

\bibverse{18} ``Los ancianos de Israel aceptarán lo que tú digas.
Entonces debes ir con ellos al rey de Egipto y decirle: ``El Señor, el
Dios de los hebreos se nos ha revelado. Así que, por favor, hagamos un
viaje de tres días al desierto para poder ofrecer sacrificios al Señor
nuestro Dios allí''.' \bibverse{19} Pero sé que el rey de Egipto no te
dejará ir a menos que se vea obligado a hacerlo por un poder más fuerte
que él.\footnote{3.19 ``Un poder más fuerte que él'': literalmente,
  ``una mano poderosa''.} \bibverse{20} Así que usaré mi poder para
infligir a Egipto todas las cosas aterradoras que estoy a punto de
hacerles. Después de eso los dejará ir. \bibverse{21} Haré que los
egipcios los traten bien como pueblo, para que cuando se vayan, no se
vayan con las manos vacías. \bibverse{22} Toda mujer pedirá a su vecina,
así como a cualquier mujer que viva en su casa, joyas y ropa de plata y
oro, y se las pondrá a sus hijos e hijas. De esta manera se llevarán la
riqueza de los egipcios con ustedes''.

\hypertarget{section-3}{%
\section{4}\label{section-3}}

\bibverse{1} ``Pero,¿qué pasa si no me creen, o no escuchan lo que
digo?'' Preguntó Moisés. ``Podrían decir: `El Señor no se te apareció'.

\bibverse{2} El Señor le preguntó: ``¿Qué tienes en la mano?''

``Un bastón'', respondió Moisés.

\bibverse{3} ``Tíralo al suelo'', le dijo a Moisés. Así lo hizo Moisés.
Se transformó en una serpiente y Moisés huía de ella.

\bibverse{4} ``Ahora extiende la mano y agárrala por la cola'', le dijo
el Señor a Moisés. Moisés lo hizo y se convirtió en un bastón en su
mano.

\bibverse{5} ``Debes hacer esto para que crean que yo, el Señor, me
aparecí delante de ti. E Dios de sus padres, el Dios de Abraham, Isaac y
Jacob''.

\bibverse{6} El Señor le dijo: ``Pon tu mano dentro de tus ropas cerca
de tu pecho''. Así que Moisés hizo lo que se le dijo. Cuando sacó su
mano, estaba blanca como la nieve, con una enfermedad de la piel.

\bibverse{7} ``Vuelve a meter la mano dentro de tu ropa'', dijo el
Señor. Y Moisés lo hizo. Cuando la sacó de nuevo, su mano había vuelto a
la normalidad.\footnote{4.7 ``A la normalidad'': literalmente, ``como su
  carne''.}

\bibverse{8} ``Si no te creen y no les convence la primera señal,
creerán por la segunda señal'', explicó el Señor. \bibverse{9} ``Pero si
todavía no te creen o no te escuchan debido a estos dos signos, entonces
debes tomar un poco de agua del Nilo y ponerla en el suelo. El agua del
Nilo se convertirá en sangre en el suelo''.

\bibverse{10} Entonces Moisés dijo al Señor: ``Discúlpame, pero no soy
bueno con las palabras, ni lo he sido en el pasado, ni desde que
comenzaste a hablar conmigo, tu siervo. Soy de hablar lento y no digo
las cosas bien''.\footnote{4.10 ``Soy de hablar lento y no digo las
  cosas bien'': literalmente: ``Me pesa la boca y la lengua''.}

\bibverse{11} ``¿Quién le dio la boca a la gente?'' le preguntó el
Señor. ``¿Quién hace a la gente sorda o muda, capaz de ver o ciega? Soy
yo, el Señor, quien lo hace. \bibverse{12} Ahora ve, y yo mismo seré tu
boca, y te diré lo que debes decir''.

\bibverse{13} ``Por favor, Señor, ¡envía a otra persona!''
respondióMoisés.

\bibverse{14} El Señor se enojó con Moisés y le dijo: ``Ahí está tu
hermano Aarón, el levita. Sé que habla bien. Viene camino para
encontrarse contigo y se alegrará mucho de verte. \bibverse{15} Habla
con él y dile qué decir. Yo seré tu boca y la suya, y te diré lo que
debes hacer. \bibverse{16} Aarón hablará en tu nombre al pueblo, como si
fuera tu boca, y tú estarás en el lugar de Dios para él. \bibverse{17}
Asegúrate de llevar tu bastón contigo para que puedas usarlo para hacer
la señales''.

\bibverse{18} Entonces Moisés regresó dondeJetro su suegro y le dijo:
``Por favor, permíteme volver con mi propio pueblo en Egipto para ver si
alguno de ellos sigue vivo''.

``Ve con mi bendición'', respondió Jetro.

\bibverse{19} Mientras Moisés estaba en Madián, el Señor le dijo:
``Vuelve a Egipto porque todos los que querían matarte han muerto''.

\bibverse{20} Moisés puso a su esposa e hijos sobre un asno y regresó a
Egipto, llevando el bastón que Dios había usado para hacer
milagros.\footnote{4.20 ``El bastón que Dios había usado para hacer
  milagros:'' literalmente, ``el bastón de Dios''. Esta interpretación
  se refiere a los milagros que se registran en los versículos 3 y 4.}

\bibverse{21} El Señor le dijo a Moisés: ``Cuando regreses a Egipto,
asegúrate de ir al Faraón y realizar los milagros que te he enseñado
para que los hagas. Lo volveré terco\footnote{4.21 ``Terco'':
  literalmente, ``endureceré su corazón,'' traducido de manera similar a
  lo largo del libro. La misma experiencia se describe como acto de
  Dios, también como una acción del propio Faraón, y también en voz
  pasiva sin un agente identificado.}y no dejará ir al pueblo.
\bibverse{22} Pero esto es lo que debes decirle al Faraón: ``Esto es lo
que dice el Señor:`Israel es mi hijo primogénito. \bibverse{23} Te
ordené que dejaras ir a mi hijo para que pueda adorarme. Pero te negaste
a liberarlo, así que ahora mataré a tu hijo primogénito'\,''.

\bibverse{24} Pero mientras iban de camino, el Señor llegó al lugar
donde se encontraban, queriendo matar a Moisés. \bibverse{25} Sin
embargo, Séfora usó un cuchillo de pedernal para cortar el prepucio de
su hijo. Le tocó los pies con él y le dijo: ``Para mí eres un marido de
sangre''. \bibverse{26} (Llamarlo marido de sangre se refiere a la
circuncisión.)\footnote{4.26 El término utilizado aquí no está muy
  claro. Puede significar algo como: ``A través de esta sangre que he
  derramado, ahora estás emparentado conmigo a través del matrimonio''.
  Algunos intérpretes creen que la palabra significa ``alguien que está
  circuncidado''.}Después de esto el Señor dejó a Moisés tranquilo.

\bibverse{27} El Señor le había dicho a Aarón: ``Ve a encontrarte con
Moisés en el desierto''. Así que Aarón fue y se encontró con Moisés en
el monte de Dios y lo saludó con un beso. \bibverse{28} Entonces Moisés
le explicó a Aarón todo lo que el Señor le había mandado a decir, y
todos los milagros que le había ordenado hacer. \bibverse{29} Moisés y
Aarón viajaron a Egipto. Allí reunieron a todos los ancianos israelitas.
\bibverse{30} Aarón compartió con ellos todo lo que el Señor le había
dicho a Moisés, y Moisés realizó los milagros para que pudieran verlos.
\bibverse{31} Los israelitas estaban convencidos. Cuando oyeron que el
Señor había venido a ellos, y que había sido tocado por su sufrimiento,
inclinaron sus cabezas y adoraron.

\hypertarget{section-4}{%
\section{5}\label{section-4}}

\bibverse{1} Después de esto, Moisés y Aarón fueron donde el Faraón y le
dijeron: ``Esto es lo que el Señor, el Dios de Israel dice: `Deja ir a
mi pueblo para que me haga una fiesta religiosa en el desierto'\,''.

\bibverse{2} ``¿Quién es este `Señor' para que yo escuche su petición de
dejar ir a Israel?''respondió El Faraón. ``¡No conozco al Señor y
ciertamente no dejaré que Israel se vaya!''

\bibverse{3} ``El Dios de los hebreos vino a nosotros'', añadieron.
``Por favor, permítenos hacer un viaje de tres días al desierto y
ofrecer sacrificios al Señor nuestro Dios. De lo contrario nos matará
por enfermedad o por espada''.

\bibverse{4} ``Moisés y Aarón, ¿por qué intentan distraer al pueblo de
su trabajo?'' Preguntó el Faraón. ``¡Vuelvan a su trabajo!'' ordenó.

\bibverse{5} ``Mira aquí'', continuó. ``Hay mucha gente tuya aquí en
nuestro país y estás impidiendo que hagan el trabajo que se les ha
asignado''. \bibverse{6} Ese mismo día ordenó a los capataces y a los
encargados del pueblo: \bibverse{7} ``No les den más paja para hacer
ladrillos como antes. Que vayan ellos mismos a recoger la paja.
\bibverse{8} Pero que sigan produciendo la misma cantidad de ladrillos
que antes. Son un pueblo perezoso, por eso gritan, pidiendo: ``Por
favor, déjanos ir y ofrecer sacrificios a nuestro dios. \bibverse{9}
¡Hagan que su trabajo sea más duro, para que puedan seguir trabajando y
no se distraigan con todas estas mentiras!''

\bibverse{10} Así que los capataces salieron y le dijeron al pueblo de
Israel: ``Esto es lo que el Faraón ha ordenado: 'No lesproveeré más
paja. \bibverse{11} Vayan y recojan la paja donde puedan encontrarla,
porque su cuota de trabajo no se reducirá''. \bibverse{12} Así que la
gente iba por todo Egipto recogiendo rastrojos para la paja.

\bibverse{13} Loscapataces seguían forzándolos, diciendo: ``¡Todavía
tienen que hacer el mismo trabajo que hacían cuando recibían la paja!''
\bibverse{14} Golpeaban a los supervisores israelitas que ellos habían
puesto a cargo, gritándoles: ``¿Por qué no han cumplido con su cuota de
ladrillos como lo hicieron antes?''

\bibverse{15} Los supervisores israelitas fueron a quejarse al Faraón,
diciendo: ``¿Por qué nos tratas así a tus siervos? \bibverse{16} No nos
das nada de paja, pero tus capataces exigen que hagamos ladrillos, ¡y
nos golpean! ¡Tu pueblo nos trata mal!''

\bibverse{17} ``¡No, ustedes solo son unos vagos, unos perezosos!''
respondió el Faraón. ``Por eso siguenrogando: `Por favor, déjanos ir y
ofrecer sacrificios al Señor'.' \bibverse{18} ¡Ahora salgan de aquí y
vayan a trabajar! ¡No se les dará paja, pero aún así tendrán que
producir la cuota completa de ladrillos!''

\bibverse{19} Los supervisores israelitas se dieron cuenta de que
estaban en problemas cuando les dijeron: ``No deben reducir la
producción diaria de ladrillos''.

\bibverse{20} Se acercaron a Moisés y Aarón que los esperaban después de
su encuentro con el Faraón, \bibverse{21} y dijeron: ``¡Que el Señor vea
lo que han hecho y los juzgue por ello! Han hecho que el Faraón y sus
oficiales se enojen con nosotros. ¡Han puesto una espada en sus manos
para matarnos!''

\bibverse{22} Moisés volvió donde el Señor y se quejó: ``¿Por qué le has
causado todos estos problemas a tu propio pueblo, Señor? ¿Fue para esto
que me enviaste?

\bibverse{23} Desde que fui a ver al Faraón para hablar en tu nombre, él
ha sido aún más duro con tu pueblo, ¡y no has hecho nada para
salvarlo!''

\hypertarget{section-5}{%
\section{6}\label{section-5}}

\bibverse{1} Pero el Señor le dijo a Moisés: ``Ahora verás lo que le
haré al Faraón. Con mi gran fuerza lo obligaré a dejarlos ir; por mi
poder los enviará fuera de su país''.

\bibverse{2} Dios habló a Moisés y le dijo: ``¡Yo soy Yahvé!\footnote{6.2
  ``Yahvé'': Este término suele traducirse como ``Señor'', pero dado que
  se identifica específicamente por su nombre, parece apropiado utilizar
  ``Yahvé'' aquí.} \bibverse{3} Me revelé como Dios Todopoderoso a
Abraham, a Isaac y a Jacob, pero ellos no conocían mi nombre, `Yahvé''
\bibverse{4} También confirmé mi acuerdo solemne con ellos de darles la
tierra de Canaán, el país donde vivían como extranjeros. \bibverse{5}
Además he escuchado los gemidos de los israelitas que los egipcios
tratan como esclavos, y no he olvidado el acuerdo que les prometí.

\bibverse{6} Así que di a los israelitas: ``Yo soy el Señor y los
salvaré del trabajo forzoso que les imponen los egipcios; yo los
liberaré su esclavitud. Los rescataré usando mi poder e imponiendo
fuertes castigos. \bibverse{7} Yo los convertiré en mi propio pueblo.
Entonces sabrán que soy el Señor su Dios, que los rescató de la
esclavitud en Egipto. \bibverse{8} Los llevaré a la tierra que prometí
solemnemente darles a Abraham, Isaac y Jacob. Se las daré y será de
ellos. ¡Yo soy el Señor!''

\bibverse{9} Moisés le explicó esto a los israelitas, pero ellos no lo
escucharon, porque estaban muy desanimados, y por el duro trabajo que se
veían obligados a hacer.

\bibverse{10} El Señor le dijo a Moisés: \bibverse{11} ``Ve y habla con
el Faraón, rey de Egipto. Dile que deje a los israelitas salir de su
país''.

\bibverse{12} Pero Moisés respondió: ``Ni siquiera mi propio pueblo me
escucha. ¿Por qué me escucharía el Faraón, sobre todo si soy tan mal
orador?''

\bibverse{13} Pero el Señor les habló a Moisés y a Aarón, y les dijo lo
que debían hacer con respecto al pueblo de Israel y conFaraón, rey de
Egipto, para sacar a los israelitas de Egipto.

\bibverse{14} Estos eran los jefes de la familia de Israel: Los hijos de
Rubén, el primogénito de Israel, eran Hanok y Pallu, Hezrón y Karmi.
Estas fueron las familias de Rubén.

\bibverse{15} Los hijos de Simeón fueron Jemuel, Jamín, Oad, Joaquín,
Zojar y Saúl (hijo de una mujer cananea). Estas fueron las familias de
Simeón.

\bibverse{16} Estosfueron los nombres de los hijos de Leví según sus
registros genealógicos: Gersón, Coaty Merari. Leví vivió durante 137
años.

\bibverse{17} Los hijos de Gersón, por familias, fueronLibní y Simí.

\bibverse{18} Los hijos de Coat fueron Amram, Izar, Hebrón y Uziel. Coat
vivió durante 133 años.

\bibverse{19} Los hijos de Merari fueron Majli y Mushi. Estas eran las
familias de los levitas según sus registros genealógicos.

\bibverse{20} Amram se casó con la hermana de su padre, Jocabed, y ella
tuvo sus hijos Aarón y Moisés. Amram vivió durante 137 años.

\bibverse{21} Los hijos de Izar fueron Coré, Neferón y Zicrí.

\bibverse{22} Los hijos de Uziel fueron Misael, Elzafán y Sitri.

\bibverse{23} Aarón se casó con Elisabet, hija de Aminadab y hermana de
Nasón. Ella tuvo sus hijos Nadab y Abiú, Eleazar e Itamar.

\bibverse{24} Los hijos de Coré fueron Asir, Elcana y Abiasaf. Estas
eran las familias de Coré.

\bibverse{25} Eleazar, hijo de Aarón, se casó con una de las hijas de
Futiel, y tuvo su hijo Finés. Estos son los ancestros de las familias
levitas, listados según sus clanes. Eleazar hijo de Aarón se casó con
una de las hijas de Futiel, y ella dio a luz a su hijo, Finés. Estos son
los jefes de las familias levitas, listados por familia.

\bibverse{26} Aarón y Moisés mencionados aquí son los que el Señor dijo,
``Saquen a los israelitas de Egipto, divididos en sus respectivas
tribus''. \bibverse{27} Moisés y Aarón también fueron los que fueron a
hablar con el Faraón, rey de Egipto, sobre la salida de los israelitas
de Egipto. \bibverse{28} Cuando el Señor habló a Moisés en Egipto,
\bibverse{29} le dijo: ``Yo soy el Señor. Dile al Faraón, rey de Egipto,
todo lo que te digo'' \bibverse{30} Pero Moisés respondió: ``No soy un
buen orador, ¿por qué me escucharía el Faraón?''

\hypertarget{section-6}{%
\section{7}\label{section-6}}

\bibverse{1} Entonces el Señor le dijo a Moisés: ``Mira, te haré parecer
como Dios ante elFaraón, y tu hermano Aarón será tu profeta''.
\bibverse{2} Debes repetir todo lo que te digo, y tu hermano Aarón debe
repetirlo al Faraón para que deje salir a los israelitas de su país.
\bibverse{3} Pero le daré al Faraón una actitud terca, y aunque haré
muchas señales y milagros en Egipto, no te escuchará. \bibverse{4}
Entoncesatacaré\footnote{7.4 Literalmente, ``pondré mi mano sobre''.}a
Egipto, imponiéndoles fuertes castigos, y sacaré a mi pueblo, los
israelitas, tribu por tribu. \bibverse{5} De esta manera los egipcios
sabrán que yo soy el Señor cuando actúe contra Egipto y saque a los
israelitas del país''.

\bibverse{6} Moisés y Aarón hicieron exactamente lo que el Señor había
ordenado. \bibverse{7} MMoisés tenía ochenta y Aarón ochenta y tres años
cuando fueron a hablar con el Faraón.

\bibverse{8} El Señor les dijo a Moisés y a Aarón: \bibverse{9} ``Cuando
el Faraón te pregunte: `¿Por qué no haces un milagro, entonces?' dile a
Aarón: 'Toma tu bastón y tíralo delante del Faraón, y se convertirá en
una serpiente''.

\bibverse{10} Moisés y Aarón fueron a ver al Faraón e hicieron lo que el
Señor había ordenado. Aarón arrojó su bastón delante del Faraón y sus
oficiales, y se convirtió en una serpiente. \bibverse{11} Pero el Faraón
llamó a sabios y hechiceros, y estos magos egipcios hicieron lo mismo
usando sus artes mágicas. \bibverse{12} Cada uno de ellos arrojó su
bastón y también se convirtieron en serpientes, pero el bastón de Aarón
se tragó todos sus bastones. \bibverse{13} Sin embargo, el Faraón tenía
una actitud dura y terca, y no los escuchaba, como el Señor había
predicho.

\bibverse{14} El Señor le dijo a Moisés: ``Faraón tiene una actitud
obstinada, se niega a dejar ir al pueblo. \bibverse{15} Así que mañana
por la mañana ve a Faraón mientras camina hacia el río. Espera para
encontrarte con él en la orilla del Nilo. Lleva contigo el bastón que se
convirtió en una serpiente. \bibverse{16} Dile: ``El Señor, el Dios de
los hebreos, me ha enviado a decirte: `Deja ir a mi pueblo para que me
adoren en el desierto'. Pero no me has escuchado hasta ahora.
\bibverse{17} Esto es lo que el Señor te dice ahora: `Así es como sabrán
que yo soy el Señor'\,''.

``¡Miren! Con el bastón que tengo en la mano, voy a golpear el agua del
Nilo, y se convertirá en sangre. \bibverse{18} Los peces del Nilo
morirán, el río tendrá mal olor, y los egipcios no podrán beber nada de
su agua''.

\bibverse{19} El Señor le dijo a Moisés: ``Dile a Aarón: `Toma tu bastón
en tu mano y sostenlo sobre las aguas de Egipto, sobre sus ríos y
canales y estanques y albercas, para que se conviertan en sangre. Habrá
sangre por todo Egipto, incluso en los recipientes de madera y
piedra'\,''.

\bibverse{20} Y Moisés y Aarón hicieron exactamente lo que el Señor les
dijo. Mientras el Faraón y todos sus oficiales miraban, Aarón levantó su
bastón y golpeó el agua del Nilo. ¡Inmediatamente todo el río se
convirtió en sangre! \bibverse{21} Los peces del Nilo murieron, y el río
olía tan mal que los egipcios no podían beber su agua. ¡Había sangre por
todo Egipto! \bibverse{22} Pero los magos egipcios hicieron lo mismo
usando sus artes mágicas. El Faraón mantuvo su actitud terca y no quiso
escuchar a Moisés y Aarón, tal como el Señor había predicho.
\bibverse{23} Entonces el Faraón volvió a su palacio y no prestó
atención a lo que había sucedido \bibverse{24} Todos los egipcios
cavaron a lo largo del Nilo porque no podían beber su agua.
\bibverse{25} Siete días pasaron después de que el Señor llegara al
Nilo.

\hypertarget{section-7}{%
\section{8}\label{section-7}}

\bibverse{1} El Señor le dijo a Moisés: ``Ve a ver al Faraón y dile:
Esto es lo que dice el Señor: ``Deja ir a mi pueblo, para que me adoren.
\bibverse{2} Si te niegas a dejarlos ir, enviaré una plaga de ranas por
todo tu país. \bibverse{3} Saldrán en enjambre del Nilo, y entrarán en
tu palacio y se meterán en tu dormitorio y saltarán a tu cama. Entrarán
en las casas de tus funcionarios y saltarán alrededor de tu gente,
incluso en tus hornos y tazones de pan. \bibverse{4} Ranas saltarán
sobre ti, tu pueblo y todos tus oficiales'''.

\bibverse{5} El Señor le dijo a Moisés: ``Dile a Aarón: `Extiende tu
bastón en tu mano sobre los ríos, canales y estanques, y haz que las
ranas se extiendan por todo Egipto'\,''. \bibverse{6} Aarón extendió su
mano sobre las aguas de Egipto, y las ranas subieron y cubrieron la
tierra. \bibverse{7} Pero los magos egipcios hicieron lo mismo usando
sus artes mágicas. Criaron ranas en Egipto.

\bibverse{8} El Faraón llamó a Moisés y a Aarón y les suplicó: ``Oren al
Señor y pídanle que me quite las ranas a mí y a mi pueblo. Entonces
dejaré ir a tu pueblo para que pueda ofrecer sacrificios al Señor''.

\bibverse{9} ``Ustedes tendrán el honor de decidir\footnote{8.9
  ``Ustedes tendrán el honor de decidir'': Literalmente, ``Glorifícate a
  ti mismo sobre mí''.}cuándo oraré por ustedes, sus funcionarios y su
pueblo para que os quiten las ranas a ustedes y a sus casas.
Permanecerán sólo en el Nilo''.''.

\bibverse{10} ``Hazlo mañana'', respondió el Faraón.

Moisés dijo: ``Sucederá como has pedido para que sepas que no hay nadie
como el Señor nuestro Dios. \bibverse{11} Las ranas los dejarán y
abandonarán sus casas, las casas de tus funcionarios y detodo tu pueblo,
y sólo permanecerán en el Nilo''.

\bibverse{12} Moisés y Aarón dejaron al Faraón, y Moisés le suplicó al
Señor por las ranas que había enviado contra el Faraón. \bibverse{13} El
Señor hizo lo que Moisés le pidió. Las ranas de las casas, los patios y
los campos murieron. \bibverse{14} El pueblo las recogió montón tras
montón, y todo el país olía fatal. \bibverse{15} Pero cuando el Faraón
se dio cuenta de que la plaga había pasado, decidió volver a ser duro y
terco, y no quiso escuchar a Moisés y Aarón, tal como el Señor había
predicho.

\bibverse{16} Entonces el Señor le dijo a Moisés: ``Dile a Aarón:
`Recoge tu bastón y golpea el polvo del suelo. El polvo se convertirá en
un enjambrede mosquitos\footnote{8.16 El nombre exacto del insecto que
  se menciona aquí no se conoce con certeza. El hebreo sugiere ``insecto
  molesto'', y ha sido traducido como piojos, mosquitos o pulgas además
  de mosquitos. Sin embargo, alguna forma de pequeño insecto volador que
  muerde como un mosquito encajaría mejor en el contexto de ``polvo''.}
por todo Egipto'\,''.

\bibverse{17} Así que hicieron lo que el Señor dijo. Cuando Aarón
levantó su bastón y golpeó el polvo de la tierra, los mosquitos
pululaban sobre las personas y los animales. El polvo de todo Egipto se
convirtió en mosquitos. \bibverse{18} Los magos también trataron de
hacer mosquitos usando sus artes mágicas, pero no pudieron. Los
mosquitos se mantuvieron tanto sobre las personas como sobre los
animales.

\bibverse{19} ``Este es un acto de Dios'', le dijeron los magos al
Faraón. Pero el Faraón eligió ser obstinado y duro de corazón, y no
quiso escuchar a Moisés y Aarón, como el Señor había predicho.

\bibverse{20} El Señor le dijo a Moisés: ``Mañana por la mañana
levántate temprano y bloquea el camino del Faraón mientras baja al río.
Dile: ¨Esto es lo que dice el Señor: Deja ir a mi pueblo, para que me
adoren. \bibverse{21} Si no dejas que mi pueblo se vaya, enviaré
enjambres de moscas sobre ti y tus funcionarios, y sobre tu pueblo y tus
casas. Todas las casas egipcias, e incluso el suelo sobre el que se
levantan, se llenarán de enjambres de moscas. \bibverse{22} Sin embargo,
en esta ocasión trataré a la tierra de Gosén de manera diferente, que es
donde vive mi pueblo, y no habrá allí ningún enjambre de moscas. Así es
como sabrán que yo, el Señor, estoy aquí en su país. \bibverse{23}
Distinguiré\footnote{8.23 El hebreo menciona aquí ``redención'', pero
  parece ser una errata. En este versículo hemos usado el término que
  propone la Septuaginta.} a mi pueblo de su pueblo. Verás esta señal
que lo confirma mañana''.

\bibverse{24} Y el Señor hizo lo que había dicho. Enormes enjambres de
moscas entraron en el palacio del Faraón y en las casas de sus
oficiales. Todo Egipto fue devastado por estos enjambres de moscas.

\bibverse{25} El Faraón llamó a Moisés y a Aarón y les dijo: ``Vayan y
ofrezcan sacrificios a su Dios aquí dentro de este país''.

\bibverse{26} ``No, eso no sería lo correcto'', respondió Moisés. ``Los
sacrificios que ofrecemos al Señor nuestro Dios serían ofensivos para
los egipcios. ¡Si nos adelantáramos y ofreciéramos sacrificios ofensivos
a los egipcios, nos apedrearían! \bibverse{27} Debemos hacer un viaje de
tres días al desierto y ofrecer allí los sacrificios al Señor nuestro
Dios como nos ha dicho''.

\bibverse{28} ``Los dejaré ir para que ofrezcan sacrificios al Señor su
Dios en el desierto, pero no vayan muy lejos'', respondió el Faraón.
``Ahora ora por mí para que esta plaga termine''.\footnote{8.28 ``Que
  esta plaga termine''. Idea implícita.}

\bibverse{29} ``Tan pronto como te deje, oraré al Señor'', respondió
Moisés, ``y mañana las moscas dejarán al Faraón y a sus oficiales y a su
pueblo. Pero el Faraón debe asegurarse de no volver a ser mentiroso,
negándose después a dejar que el pueblo vaya a ofrecerle sacrificios al
Señor''.

\bibverse{30} Moisés dejó al Faraón y oró al Señor, \bibverse{31} y el
Señor hizo lo que Moisés le pidió, y quitó los enjambres de moscas del
Faraón y sus funcionarios y su pueblo. No quedó ni una sola mosca.
\bibverse{32} Pero una vez más el Faraón eligió ser obstinado y duro de
corazón y no dejó que el pueblo se fuera.

\hypertarget{section-8}{%
\section{9}\label{section-8}}

\bibverse{1} El Señor le dijo a Moisés: ``Ve y habla con el Faraón.
Dile: `Esto es lo que dice el Señor': Deja ir a mi pueblo, para que me
adoren. \bibverse{2} Si te niegas a dejarlos ir y sigues deteniéndolos,
\bibverse{3} te castigaré trayendo una plaga muy severa sobre tu ganado:
en tus caballos, tus asnos, tus camellos, así como tus rebaños y
manadas. \bibverse{4} Pero el Señor distinguirá entre el ganado de los
israelitas y el de los egipcios, de modo que ninguno de los que
pertenecen a los israelitas morirá''' \bibverse{5} El Señor ha fijado un
tiempo, diciendo: ``Mañana esto es lo que va a pasar aquí en el país''.

\bibverse{6} Al día siguiente el Señor hizo lo que había dicho. Todo el
ganado de los egipcios murió, pero no murió ni un solo animal de los
israelitas. \bibverse{7} ElFaraón envió a los oficiales y descubrió que
no había muerto ni un solo animal de los israelitas. Pero el Faraón fue
terco y no dejó que el pueblo se fuera.

\bibverse{8} El Señor les dijo a Moisés y a Aarón: ``Vayan y saquen unos
puñados de hollín de un horno. Luego Moisés deberá arrojarlo al aire
delante del Faraón. \bibverse{9} Se esparcirá como polvo fino por todo
el país de Egipto, y aparecerán llagas abiertas en la gente y en los
animales de toda la tierra''. \bibverse{10} Entonces sacaron hollín de
un horno y fueron a ver al Faraón. Moisés lo arrojó al aire, y se
comenzaron a abrir llagas en las personas y los animales. \bibverse{11}
Los magos no pudieron venir nicomparecer ante Moisés, porque ellos y
todos los demás egipcios estaban cubiertos de llagas. \bibverse{12} Pero
el Señor puso en el Faraón una actitud obstinada, y el Faraón no los
escuchó, tal como el Señor le había dicho a Moisés.

\bibverse{13} El Señor le dijo a Moisés: ``Mañana por la mañana
levántate temprano y ve al Faraón, y dile que esto es lo que el Señor,
el Dios de los hebreos, dice: `Deja ir a mi pueblo para que me adore'.
\bibverse{14} Esta vez dirigiré todas mis plagas contra ti y tus
funcionarios y tu pueblo, para que te des cuenta de que no hay nadie
como yo en toda la tierra. \bibverse{15} A estas alturas ya podría haber
extendido mi mano para atacarte a ti y a tu pueblo con una plaga que te
habría destruido por completo.\footnote{9.15 ``Te habría destruído por
  completo'': Literalmente, ``Habrías perecido en la tierra''.}
\bibverse{16} Sin embargo, te he dejado vivir para que veas mi poder, y
para que mi reputación sea conocida por toda la tierra. \bibverse{17}
Pero en tu orgullo sigues tiranizando a mi pueblo, y te niegas a dejar
que se vaya. \bibverse{18} ¡Así que ten cuidado! Mañana a esta hora
enviaré la peor granizada que haya caído sobre Egipto, desde el
principio de su historia hasta ahora. \bibverse{19} Así que haz guardar
a tu ganado y todo lo que tienes en el campo. Porque toda persona y todo
animal que permanezcan fuera morirán cuando el granizo caiga sobre
ellos''.

\bibverse{20} Aquellos oficiales del Faraón que tomaron en serio lo que
el Señor dijo, se apresuraron a traer a sus sirvientes y a su ganado
adentro. \bibverse{21} Pero aquellos a los que no les importó lo que el
Señor decía, dejaron a sus sirvientes y ganado afuera.

\bibverse{22} El Señor le dijo a Moisés: ``Levanta tu mano hacia el
cielo para que caiga una tormenta de granizo sobre todo Egipto, sobre la
gente y sobre los animales, y sobre todo lo que crece en los campos de
Egipto''.

\bibverse{23} Moisés levantó su bastón hacia el cielo, y el Señor envió
truenos y granizo, e hizo caer rayos al suelo. Así es como el Señor hizo
llover granizo sobre Egipto. \bibverse{24} Cuando el granizo cayó, vino
acompañado de relámpagos por todas partes. El granizo que cayó fue tan
severo como nunca se había visto en todo Egipto desde los comienzos de
su historia. \bibverse{25} A lo largo de todo Egipto el granizo golpeó
todo en los campos, tanto a las personas como a los animales. Derribó
todo lo que crecía en los campos, y desnudó todos los árboles.
\bibverse{26} Sólo en la tierra de Gosén, donde vivían los israelitas,
no había granizo.

\bibverse{27} ElFaraón llamó a Moisés y a Aarón y les dijo: ``Admito que
esta vez he pecado. ¡El Señor tiene razón, y yo y mi pueblo estamos
equivocados! \bibverse{28} Rueguen al Señor por nosotros, porque ya ha
habido más que suficiente de los truenos y granizos de Dios. Dejaré que
se vayan. No necesitan quedarse más tiempo aquí''.

\bibverse{29} ``Una vez que haya dejado la ciudad, oraré al Señor por
ti'', le dijo Moisés. ``Los truenos cesarán y no habrá más granizo, para
que te des cuenta de que la tierra pertenece al Señor. \bibverse{30}
Pero sé que tú y tus funcionarios aún no respetan realmente al Señor
nuestro Dios''.

\bibverse{31} (El lino y la cebada fueron destruidos, porque la cebada
estaba madura y el lino estaba floreciendo. \bibverse{32} Sin embargo,
el trigo y la escanda no fueron destruidos porque crecen más tarde).

\bibverse{33} Moisés dejó al Faraón y salió de la ciudad, y oró al
Señor. Los truenos y el granizo se detuvieron, y la tormenta de lluvia
terminó. \bibverse{34} Cuando el Faraón vio que la lluvia, el granizo y
los truenos habían cesado, volvió a pecar, y eligió volver a ser
obstinado, junto con sus funcionarios. \bibverse{35} Debido a su actitud
terca, el Faraón no permitió que los israelitas se fueran, tal como el
Señor había predicho a través de Moisés.

\hypertarget{section-9}{%
\section{10}\label{section-9}}

\bibverse{1} El Señor le dijo a Moisés: ``Ve a ver al Faraón, porque fui
yo quien le dio a él y a sus oficiales una actitud obstinada para que yo
pudiera hacer mis milagros ante ellos. \bibverse{2} Esto es para que
puedas contar a tus hijos y nietos cómo hice que los egipcios parecieran
tontos\footnote{10.2 ``parecieran tontos'': La palabra sugiere que el
  Señor se está burlando de los egipcios, y esto sería principalmente
  por su devoción a ídolos inútiles.}haciendo estos milagros entre
ellos, y para que sepas que yo soy el Señor''.

\bibverse{3} Moisés y Aarón fueron a ver al Faraón y le dijeron: ``Esto
es lo que dice el Señor, el Dios de los hebreos: '¿Hasta cuándo te
negarás a humillarte ante mí? Deja ir a mi pueblo, para que me adore.
\bibverse{4} Sino dejas que mi pueblo se vaya, mañana enviaré una plaga
de langostas a tu país. \bibverse{5} Habrá tantas que cubrirán el suelo
para que nadie pueda verlas. Comerán los cultivos que haya dejado el
granizo, así como todos los árboles que crezcan en tus campos.
\bibverse{6} Entrarán en enjambres en tus casas y en las casas de todos
tus funcionarios, de hecho en las casas de todos los egipcios. Esto es
algo que ninguno de tus antepasados ha visto desde que llegaron a este
país''. Entonces Moisés y Aarón se volvieron y dejaron al Faraón.

\bibverse{7} Los oficiales del Faraón se acercaron a él y le
preguntaron: ``¿Cuánto tiempo vas a dejar que este hombre nos cause
problemas?\footnote{10.7 ``Nos cause problemas'': Literalmente, ``sea
  una trampa para nosotros''.}Deja que esta gente se vaya para que
puedan adorar al Señor su Dios. ¿No te das cuenta de que Egipto ha
quedado destruido?''

\bibverse{8} Moisés y Aarón fueron traídos nuevamente ante el Faraón.
``Vayan y adoren al Señor su Dios'', les dijo. ``Pero ¿quién de ustedes
irá?''

\bibverse{9} ``Todos iremos'', respondió Moisés. ``Jóvenes y viejos,
hijos e hijas, y llevaremos nuestros rebaños y manadas con nosotros,
porque vamos a celebrar unafiesta para el Señor''.

\bibverse{10} ``¡El Señor realmente tendrá que estar con ustedes si dejo
que sus hijos se vayan!'' respondió el Faraón. ``¡Claramente estás
planeando algún tipo de truco maligno! \bibverse{11} ¡Así que no! Sólo
los hombres pueden ir y adorar al Señor, porque eso es lo que has estado
pidiendo''. Entonces hizo que echaran a Moisés y a Aarón.

\bibverse{12} El Señor le dijo a Moisés, ``Levanta tu mano sobre Egipto,
para aparezcan las langostas y se coman todas las plantas del país, todo
lo que haya dejado el granizo''.

\bibverse{13} Moisés extendió su bastón sobre Egipto, y durante todo ese
día y noche el Señor envió un viento del este que soplaba sobre la
tierra. Cuando llegó la mañana, el viento del Este había traído las
langostas.

\bibverse{14} Las langostas pululaban por toda la tierra y se asentaron
en cada parte del país. Nunca había habido tal enjambre de langostas, y
no lo habrá nunca más. \bibverse{15} Cubrieron el suelo hasta que se vio
negro, y se comieron todas las plantas de los campos y todos los frutos
de los árboles que había dejado el granizo. No quedó ni una sola hoja
verde en ningún árbol o planta en ningún lugar de Egipto.

\bibverse{16} ElFaraón llamó urgentemente a Moisés y a Aarón y dijo:
``He pecado contra el Señor tu Dios y contra ti. \bibverse{17} Así que,
por favor, perdona mi pecado sólo esta vez y suplica al Señor tu Dios,
pidiéndole que al menos me quite esta plaga mortal''.

\bibverse{18} Moisés dejó al Faraón y rezó al Señor. \bibverse{19} El
Señor cambió la dirección del viento, de modo que un fuerte viento del
Oeste arrastró a las langostas hasta el Mar Rojo. No quedó ni una sola
langosta en ningún lugar de Egipto. \bibverse{20} Pero el Señor hizo que
el Faraón se obstinara y no dejara ir a los israelitas.

\bibverse{21} El Señor le dijo a Moisés: ``Levanta tu mano hacia el
cielo para que caiga la oscuridad sobre Egipto, una oscuridad tan espesa
que se pueda sentir''.

\bibverse{22} Moisés levantó su mano hacia el cielo, y todo Egipto quedó
completamente a oscuras durante tres días. \bibverse{23} Nadie podía ver
a nadie más, y nadie se movió de donde estaba durante tres días. Pero
todavía había luz donde vivían todos los israelitas.

\bibverse{24} Finalmente el Faraón llamó a Moisés. ``Vayan y adoren al
Señor'', dijo. ``Dejen a sus rebaños y manadas aquí. Incluso puedes
llevarte a tus hijos contigo''.

\bibverse{25} Pero Moisés respondió: ``También debes dejarnos animales
para los sacrificios y los holocaustos, para que podamos ofrecerlos al
Señor nuestro Dios''. \bibverse{26} Nuestro ganado tiene que ir con
nosotros también.No se dejará ni un solo animal. Necesitaremos algunos
para adorar al Señor nuestro Dios, y no sabremos cómo debemos adorar al
Señor hasta que lleguemos allí''.

\bibverse{27} Pero el Señor hizo que el Faraón se obstinara, y no los
dejó ir. \bibverse{28} ElFaraón le gritó a Moisés: ``¡Fuera de aquí! ¡No
quiero volver a verte nunca más! ¡Si te vuelvo a ver, morirás!''

\bibverse{29} ``Que sea como tú dices'', respondió Moisés. ``No volveré
a verte''

\hypertarget{section-10}{%
\section{11}\label{section-10}}

\bibverse{1} El Señor le dijo a Moisés: ``Hay una última plaga que
derribaré sobre el Faraón y sobre Egipto. Después de eso os dejará
marchar, pero cuando lo haga, os expulsará a todos del país.
\bibverse{2} Ahora ve y dile a los israelitas, tanto hombres como
mujeres, que pidan a sus vecinos egipcios objetos de plata y oro''.
\bibverse{3} El Señor hizo que los egipcios miraran favorablemente a los
israelitas. De hecho, el propio Moisés era muy respetado en Egipto tanto
por los oficiales del Faraón como por la gente común.

\bibverse{4} Moisés dijo: ``Esto es lo que dice el Señor: `Alrededor de
la medianoche recorreré todo Egipto. \bibverse{5} Todo primogénito en la
tierra de Egipto morirá, desde el primogénito del Faraón sentado en su
trono hasta el primogénito de la sirvienta que trabaja con un molino de
mano, así como todo primogénito del ganado. \bibverse{6} Habráfuertes
gritos de luto en todo Egipto, como nunca antes se ha hecho, y nunca más
se hará. \bibverse{7} Pero entre todos los israelitas ni siquiera el
ladrar de un perro molestará a las personas o a sus animales. Así sabrán
que el Señor distingue entre Egipto e Israel''' \bibverse{8} Todos tus
oficiales vendrán a mí, se inclinarán ante mí y me dirán: ``¡Vete y
llévate a todos tus seguidores! Después de eso me iré''. Moisés se
enfadó mucho y se fue de la presencia delFaraón.

\bibverse{9} El Señor le dijo a Moisés: ``El Faraón se niega a
escucharte para que pueda hacer más milagros en Egipto''. \bibverse{10}
Moisés y Aarón hicieron estos milagros ante el Faraón, pero el Señor le
dio al Faraón una actitud obstinada, y no dejó que los israelitas
salieran de su país.

\hypertarget{section-11}{%
\section{12}\label{section-11}}

\bibverse{1} El Señor le dijo a Moisés y a Aarón cuando aún estaban en
Egipto: \bibverse{2} ``Este mes será para ti el primer mes, el primer
mes de tu año. \bibverse{3} Diles a todos los israelitas que el décimo
día de este mes, cada hombre debe elegir un cordero\footnote{12.3
  ``Cordero'': O una cabra joven. La palabra usada aquí se aplica a
  ambos.}para su familia, uno para cada hogar. \bibverse{4} Sin embargo,
si la casa es demasiado pequeña para un cordero entero, entonces él y su
vecino más cercano pueden elegir un cordero según el número total de
personas. Dividirán el cordero según lo que cada uno pueda comer.
\bibverse{5} El cordero debe ser un macho de un año sin ningún defecto,
y puede ser tomado del rebaño de ovejas o del rebaño de cabras.

\bibverse{6} Guárdalo hasta el día catorce del mes, cuando todos los
israelitas sacrificarán los animales después de la puesta del sol y
antes de que oscurezca. \bibverse{7} Tomarán un poco de sangre y la
pondrán a los lados y en la parte superior de los marcos de las puertas
de las casas en las que coman. \bibverse{8} Asarán la carne en el fuego
y la comerán esa noche, junto con pan sin levadura y hierbas amargas.
\bibverse{9} No deben comer la carne cruda o hervida en agua. Todo debe
ser asado sobre el fuego, incluyendo la cabeza, las piernas y los
intestinos. \bibverse{10} Asegúrense de que no quede nada hasta la
mañana. Si sobra algo, deben quemarlo por la mañana.

\bibverse{11} Así es como deben comer la comida. Deben estar vestidosy
listos para viajar, con las sandalias en los pies y el bastón en la
mano. Deben comer rápido, pues es la Pascua del Señor. \bibverse{12} Esa
misma noche recorreré todo Egipto y mataré a todos los primogénitos de
las personas y los animales, y traeré la condenación a todos los dioses
de Egipto. Yo soy el Señor. \bibverse{13} Marcaré las casas con sangre,
y cuando vea la sangre, pasaré de largo. Ninguna plaga mortal caerá
sobre ustedesni los destruirá cuando ataque a Egipto.

\bibverse{14} Este será para ustedes un día para recordar. Lo celebrarán
como un festival para el Señor por las generaciones futuras. Observarán
esto por todos los tiempos venideros. \bibverse{15} Durante siete días
sólo comerán pan hecho sin levadura. El primer día deben deshacerse de
la levadura de sus casas. Cualquiera que coma algo con levadura desde el
primer día hasta el séptimo debe ser excluido de la comunidad israelita.
\bibverse{16} Tanto el primer como el séptimo día deben tener una
reunión sagrada. No deben trabajar en esos días, excepto para preparar
la comida. Eso es lo único que pueden hacer.

\bibverse{17} Celebrarán la fiesta de los panes sin levadura porque en
este mismo día yo saqué a sus tribus de Egipto. Deben observar este día
de aquí en adelante. \bibverse{18} En el primer mes deberán comer pan
sin levadura desde la tarde del día catorce hasta la tarde del día
veintiuno. \bibverse{19} Durante siete días no debe haber levadura en
sus casas. Si alguien come algo con levadura, debe ser excluido de la
comunidad israelita, sea extranjero o nativo de la tierra. \bibverse{20}
No comerán nada que contenga levadura. Coman sólo pan sin levadura en
todas sus casas''.

\bibverse{21} Entonces Moisés convocó a todos los ancianos de Israel y
les dijo: ``Vayan enseguida y elijan un cordero para cada una de sus
familias y maten el cordero de la Pascua. \bibverse{22} Cojan un manojo
de hisopo, mójenlo en la sangre de la palangana y pónganlo en la parte
superior y en los lados del marco de la puerta. Ninguno de ustedes
saldrá por la puerta de la casa hasta la mañana.

\bibverse{23} Cuando el Señor pase a castigar a los egipcios, verá la
sangre en la parte superior y en los lados del marco de la puerta.
Pasará porencima de la puerta y no permitirá que el destructor entre en
sus casas y los mate. \bibverse{24} Ustedes y sus descendientes deberán
recordar estas instrucciones para el futuro. \bibverse{25} Cuando entren
en la tierra que el Señor prometió darles, celebrarán esta ceremonia.
\bibverse{26} Cuando sus hijos vengan y les pregunten: ``¿Por qué es
importante esta ceremonia para ustedes?'' \bibverse{27} deben decirles:
``Este es el sacrificio de Pascua para el Señor. Él fue quien pasó por
encima de las casas de los israelitas en Egipto cuando mató a los
egipcios, pero perdonó a nuestras familias''. El pueblo se inclinó en
adoración.

\bibverse{28} Entonces los israelitas fueron e hicieron lo que el Señor
les había dicho a Moisés y a Aarón. \bibverse{29} A medianoche el Señor
mató a todo primogénito varón en la tierra de Egipto, desde el
primogénito del Faraón, que estaba sentado en su trono, hasta el
primogénito del prisionero en la cárcel, y también todo el primogénito
del ganado \bibverse{30} El Faraón se levantó durante la noche, así como
todos sus oficiales y todos los egipcios. Hubo fuertes gritos de agonía
en todo Egipto, porque no había una sola casa en la que no hubiera
muerto alguien. \bibverse{31} El Faraón llamó a Moisés y a Aarón durante
la noche y les dijo: ``¡Fuera de aquí! ¡Dejen a mi pueblo, ustedes dos y
los israelitas! Váyanse, para que puedan adorar al Señor como lo han
pedido. \bibverse{32} ¡Llévense también a sus rebaños y manadas, como lo
dijeron antes y váyanse! Oh, y bendíceme a mí también''.

\bibverse{33} Los egipcios instaron a los israelitas a dejar su país lo
más rápido posible, diciendo: ``¡Si no, moriremos todos!'' \bibverse{34}
Así que los israelitas recogieron su masa antes de que se levantara y la
llevaron sobre sus hombros en tazones de amasar envueltos en ropa
\bibverse{35} Además, los israelitas hicieron lo que Moisés les había
dicho y pidieron a los egipcios objetos de plata y oro, y ropa.
\bibverse{36} El Señor había hecho que los egipcios miraran tan
favorablemente a los israelitas que aceptaron su petición. De esta
manera se llevaron las riquezas de los egipcios.

\bibverse{37} Los israelitas partieron a pie desde Ramsés hacia Sucot y
fueron unos 600.000 hombres, así como mujeres y niños.\footnote{12.37
  ``Mujeres y niños'': Literalmente, ``dependientes''.} \bibverse{38}
Además, muchos extranjeros se les unieron. También se llevaron consigo
grandes rebaños y manadas de ganado. \bibverse{39} Como su masa de pan
no tenía levadura, los israelitas cocinaron lo que habían sacado de
Egipto en panes sin levadura. Esto se debió a que cuando fueron
expulsados de Egipto tuvieron que salir de prisa y no tuvieron tiempo de
prepararse la comida.

\bibverse{40} Los israelitas habían vivido en Egipto durante 430 años.
\bibverse{41} El mismo día en que terminaron los 430 años, todas las
tribus del Señor, por sus respectivas divisiones, salieron de Egipto.
\bibverse{42} Siendo que el Señor veló esa noche para sacarlos de la
tierra de Egipto, ustedes deben velar esa misma noche como una
observancia para honrar al Señor, que será guardada por todos los
israelitas para las generaciones futuras.

\bibverse{43} El Señor les dijo a Moisés y a Aarón: ``Esta es la
ceremonia de la Pascua. Ningún extranjero puede comerla. \bibverse{44}
Pero cualquier esclavo que haya sido comprado puede comerla cuando lo
hayas circuncidado. \bibverse{45} Los visitantes extranjeros o los
contratados de otras naciones no podrán comer la Pascua. \bibverse{46}
Se debe comer dentro de la casa. No se permite sacar nada de la carne
fuera de la casa, ni romper ningún hueso. \bibverse{47} Todos los
israelitas deben celebrarla. \bibverse{48} Si hay un extranjero que vive
con ustedes y quiere celebrar la Pascua del Señor, todos los varones de
su casa tienen que ser circuncidados. Entonces podrán venir a celebrar y
ser tratados como nativosdel país. Pero ningún hombre que no esté
circuncidado puede comerla. \bibverse{49} La misma regla se aplica tanto
al nativo como al extranjero que vive entre ustedes''.

\bibverse{50} Entonces todos los israelitas siguieron estas
instrucciones. Hicieron exactamente lo que el Señor había ordenado a
Moisés y Aarón. \bibverse{51} Ese mismo día el Señor sacó a las tribus
israelitas de Egipto, una por una.

\hypertarget{section-12}{%
\section{13}\label{section-12}}

\bibverse{1} Entonces el Señor le dijo a Moisés: \bibverse{2} ``Todo
varón primogénito será dedicado a mí. El primogénito de cada familia
israelita me pertenece, y también cada animal primogénito''.

\bibverse{3} Así que Moisés le dijo al pueblo: ``Recuerden que este es
el día en que dejaron Egipto, la tierra de su esclavitud, porque el
Señor los sacó de allí con su asombroso poder. (Nada con levadura en él
será comido). \bibverse{4} Hoyustedes están en camino, este día en el
mes de Abib. \bibverse{5} El Señor los llevará a la tierra de los
cananeos, hititas, amorreos, heveos y jebuseos, la tierra que le
prometió a sus antepasados, una tierra que fluye leche y miel. Así que
deben observar esta ceremonia en este mes. \bibverse{6} Durante siete
días sólo comerán pan sin levadura, y el séptimo día celebrarán una
fiesta religiosa para honrar al Señor. \bibverse{7} Durante esos siete
días solo podrán comer pan sin levadura. No deben tener levadura; de
hecho, no debe haber levadura en ningún lugar donde vivan.

\bibverse{8} Ese día digan a sus hijos: ``Esto es por causa de lo que el
Señor hizo por mí cuando salí de Egipto''. \bibverse{9} Cuando celebren
esta ceremonia\footnote{13.9 ``Cuando celebren esta ceremonia'': añadido
  para mayor claridad.}será como una señal en su mano y un recordatorio
en la frente de que esta enseñanza del Señor debe ser contada con
regularidad. Porque el Señor los sacó de Egipto con su gran poder.
\bibverse{10} Es por eso que deben observar esta ceremonia año tras año,
en esta fecha. \bibverse{11} Una vez que el Señor los lleve a la tierra
de los cananeos y se las entregue, como se los prometió a ustedes y a
sus antepasados, \bibverse{12} deben presentar al Señor todos los
primogénitos varones, humanos o animales. Todos los primogénitos de su
ganado le pertenecen al Señor. \bibverse{13} Deben rescatara cada asno
primogénito a cambio de un cordero, y si no lo hacen, deberán romperle
el cuello. Deberánrescatar a cada primogénito de sus hijos.

\bibverse{14} Cuando en el futuro sus hijos vengan a ustedes y les
pregunten: ``¿Por qué es importante esta ceremonia?'' deberán decirles:
``El Señor nos sacó de Egipto, la tierra de nuestra esclavitud, mediante
su asombroso poder. \bibverse{15} El Faraón se negó obstinadamente a
dejarnos ir, así que el Señor mató a todos los primogénitos de la tierra
de Egipto, tanto humanos como animales. Por eso sacrificamos al Señor el
primogénito de cada animal, y compramos todos los primogénitos de
nuestros hijos''. \bibverse{16} De esta manera, será como una señal en
la mano y un recordatorio en la frente, porque el Señor nos sacó de
Egipto por su asombroso poder''.

\bibverse{17} Cuando el Faraón dejó salir a los israelitas, Dios no los
llevó por la tierra de los filisteos, aunque era un camino más corto.
Porque Dios dijo, ``Si se ven obligados a luchar, podrían cambiar de
opinión y volver a Egipto''. \bibverse{18} Así que Dios los llevó por el
camino más largo a través del desierto hacia el Mar Rojo. Cuando los
israelitas dejaron la tierra de Egipto eran como un ejército listo para
la batalla.

\bibverse{19} Moisés llevó los huesos de José con él porque José
leshabía hecho a los hijos de Israel una promesa solemne, diciendo:
``Dios definitivamente cuidará de ustedes, y entonces deben llevarse mis
huesos cuando salgan de aquí''.

\bibverse{20} Viajaron desde Sucot y acamparon en Etam, a la entrada del
desierto. \bibverse{21} El Señor iba delante de ellos como una columna
de nubes para mostrarles el camino durante el día, y como una columna de
fuego para proporcionarles luz por la noche. Así podían viajar de día o
de noche. \bibverse{22} La columna de nubes durante el día y la columna
de fuego por la noche iban siempre delante del pueblo.

\hypertarget{section-13}{%
\section{14}\label{section-13}}

\bibverse{1} Entonces el Señor le dijo a Moisés: \bibverse{2} ``Diles a
los israelitas que vuelvan y acampen cerca de Pi-Ajirot, entre Migdol y
el mar. Deben acampar junto al mar, frente a Baal-Zefón. \bibverse{3} El
Faraónsacará su conclusión respecto a los israelitas: ``Están vagando
por el país con gran confusión, y el desierto les ha impedido salir''
\bibverse{4} Daré a Faraón una actitud terca para que los persiga a fin
derecuperarlos.\footnote{14.4 ``A fin de recuperarlos'': añadido para
  mayor claridad.}Pero ganaré honra por lo que le sucederá al Faraón y a
todo su ejército, y los egipcios sabrán que yo soy el Señor''. Así que
los israelitas hicieron lo que se les ordenó.

\bibverse{5} Cuando el rey de Egipto se enteró de que los israelitas se
habían marchado apresuradamente, el Faraón y sus oficiales cambiaron de
opinión sobre lo que había sucedido y dijeron: ``¿Qué hemos hecho? Hemos
dejado ir a todos estos esclavos israelitas.'' \bibverse{6} Así que el
Faraón hizo preparar su carro y se puso en marcha con su ejército.
\bibverse{7} Tomó 600 de sus mejores carros junto con todos los demás
carros de Egipto, cada uno con su oficial a cargo. \bibverse{8} El Señor
le dio al Faraón, rey de Egipto, una actitud terca, así que persiguió a
los israelitas, que salían con los puños levantados en triunfo.
\bibverse{9} Los egipcios salieron en persecución, con todos los
caballos y carros del Faraón, así como jinetes y soldados. Alcanzaron a
los israelitas mientras estaban acampandojunto al mar cerca de
Pi-Ajirot, frente a Baal-Zefón.

\bibverse{10} Los israelitas miraron hacia atrás y vieron al Faraón y al
ejército egipcio acercándose. Estaban absolutamente aterrorizados y
pidieron ayuda al Señor. \bibverse{11} Se quejaron a Moisés: ``¿No había
tumbas en Egipto que nos tuvieras que traer aquí en el desierto para
morir? ¿Qué nos has hecho al hacernos salir de Egipto? \bibverse{12}
¿Acaso no te dijimos en Egipto: ``Déjanos en paz para que sigamos siendo
esclavos de los egipcios''? ¡Hubiera sido mejor para nosotros ser
esclavos de los egipcios que morir aquí en el desierto!''

\bibverse{13} Pero Moisés le dijo al pueblo: ``No tengan miedo. Quédense
donde están y verán cómo el Señor nos salvará hoy. Los egipcios que ven
ahora, ¡no los volverán a ver nunca más! \bibverse{14} El Señor va a
luchar por ustedes, así que no necesitan hacer nada''.

\bibverse{15} El Señor le dijo a Moisés: ``¿Por qué clamas a mi con
gritos? Dile a los israelitas que sigan adelante. \bibverse{16} Debes
tomar tu bastón y sostenerlo en tu mano sobre el mar. Divídelo para que
los israelitas puedan caminar por el mar en tierra seca. \bibverse{17}
Pondré en los egipcios una actitud obstinada y dura para que los
persigan. Entonces me ganaré su honra por lo que le sucederá al Faraón y
a todo su ejército, así como a sus carros y jinetes. \bibverse{18} Los
egipcios sabrán que soy el Señor cuando me gane su respeto a través del
Faraón, sus carros y su caballería''.

\bibverse{19} El ángel de Dios, que había estado guiando a los
israelitas, se movía detrás de ellos, \bibverse{20} posicionándose entre
los campos de los egipcios y de los israelitas. La nube estaba oscurapor
un lado, pero iluminaba la noche por el otro. Nadie de ninguno de los
dos campamentos se acercaba al otro durante la noche.

\bibverse{21} Entonces Moisés extendió su mano sobre el mar, y durante
toda la noche el Señor hizo retroceder el mar con un fuerte viento del
este, y convirtió el fondo del mar en tierra firme. Así que el agua se
dividió, \bibverse{22} y los israelitas caminaron por el mar en tierra
seca, con muros de agua a su derecha y a su izquierda.

\bibverse{23} Los egipcios los persiguieron, con todos los caballos,
carros y jinetes del Faraón. Siguieron a los israelitas hasta el mar.
\bibverse{24} Pero al final de la noche el Señor miró al ejército
egipcio desde la columna de fuego y nube, y les causó pánico.
\bibverse{25} Hizo que las ruedas de sus carros se atascaran, por lo que
les resultaba difícil conducir. Los egipcios gritaron: ``¡Retírense!
¡Debemos huir de los israelitas porque el Señor está luchando en favor
de ellos contra nosotros!''

\bibverse{26} Entonces el Señor le dijo a Moisés: ``Extiende tu mano
sobre el mar, para que el agua caiga sobre los egipcios, sus carros y
jinetes''. \bibverse{27} Entonces Moisés extendió su mano sobre el mar,
y al amanecer el mar volvió a la normalidad. Mientras los egipcios se
retiraban, el Señor los arrastró al mar. \bibverse{28} El agua cayó
sobre ellos y cubrió los carros y los jinetes, así como todo el ejército
del Faraón que había perseguido a los israelitas hasta el mar. Ni uno
solo de ellos sobrevivió.

\bibverse{29} Pero los israelitas habían caminado por el mar en tierra
seca, con muros de agua a su derecha y a su izquierda. \bibverse{30} El
Señor salvó a los israelitas de la amenaza de los egipcios. Y los
israelitas vieron a los egipcios muertos en la orilla. \bibverse{31}
Cuandovieron el gran poder que el Señor había usado contra los egipcios,
los israelitas se quedaron asombrados del Señor y confiaron en él y en
su siervo Moisés.

\hypertarget{section-14}{%
\section{15}\label{section-14}}

\bibverse{1} Entonces Moisés y los israelitas cantaron esta canción al
Señor:

``¡Cantaré al Señor, porque él es supremo! Ha arrojado al mar a los
caballos y a sus jinetes.

\bibverse{2} El Señor me da fuerza. Él es el tema de mi canción. Él me
salva. Él es mi Dios, y yo lo alabaré. Él es el Dios de mi padre, y yo
lo honraré.El Señor me da fuerza. Él es el tema de mi canción. Él me
salva. Él es mi Dios, y yo lo alabaré. Él es el Dios de mi padre, y yo
lo honraré.

\bibverse{3} El Señor es como un guerrero. Su nombre es el Señor.

\bibverse{4} Arrojó los carros del Faraón y su ejército al mar. Los
mejores oficiales del Faraón se ahogaron en el Mar Rojo.

\bibverse{5} El agua los cubrió como una inundación. Cayeron a las
profundidades como una piedra.

\bibverse{6} Tu poder, Señor, es verdaderamente asombroso. Tu poder,
Señor, aplastó al enemigo.

\bibverse{7} Con tu majestuoso poder destruiste a los que se te oponían.
Tu cólera ardió y los quemó como un rastrojo.

\bibverse{8} Túsoplaste\footnote{15.8 Literalmente, ``por el aliento de
  tu nariz''.}y el mar se amontonó. Las olas se alzaron como un muro.
Las profundidades del océano se volvieron sólidas.

\bibverse{9} El enemigo se jactó: ``Los perseguiré y los alcanzaré''.
Dividiré el botín. Los comeré vivos. Bailaré con laespada. Con mi mano
los destruiré'.

\bibverse{10} Pero tú soplaste con tu aliento y el mar los arrastró. Se
hundieron como el plomo en las aguas revueltas.

\bibverse{11} ¿Quién es como tú entre los dioses, Señor? ¿Quién es como
tú, glorioso en santidad, asombroso y maravilloso, que hace milagros?

\bibverse{12} Tú actuaste, y la tierra se tragó a los egipcios.

\bibverse{13} Guiaste a las personas que salvaste con tu confiable amor.
Los guiarás en tu fuerza a tu santo hogar.

\bibverse{14} Las naciones oirán lo que ha sucedido y temblarán de
miedo. El pueblo que vive en Filistea experimentará una angustia
agonizante.

\bibverse{15} Los jefes edomitas estarán aterrorizados. Los líderes
moabitas temblarán. La gente que vive en Canaán se derretirá en pánico.

\bibverse{16} El terror y el miedo caerán sobre ellos. Señor, debido a
tu gran poder, estarán quietos como una piedra hasta que tu pueblo pase,
hasta que pase el pueblo que compraste..

\bibverse{17} Tomarás a tu pueblo y lo plantarás en el monte que tú
posees, el lugar que tú, Señor, has preparado como tu casa, el santuario
que tus manos han construido, Señor.

\bibverse{18} ¡El Señor reinará por siempre y para siempre!''

\bibverse{19} Cuando los caballos, carros y jinetes del Faraón entraron
en el mar, el Señor hizo que el agua se precipitara sobre ellos. Pero
los israelitas caminaron por el mar en tierra seca.

\bibverse{20} La profeta Miriam, hermana de Aarón, cogió una pandereta y
todas las mujeres la siguieron bailando y tocando la pandereta.
\bibverse{21} Miriam les cantó: ``¡Canten al Señor, porque él es
supremo! Ha arrojado al mar a los caballos y a sus jinetes''.

\bibverse{22} Entonces Moisés llevó a Israel lejos del Mar Rojo y al
desierto de Sur. Durante tres días caminaron por el desierto pero no
encontraron agua. \bibverse{23} Cuando llegaron a Mara, el agua allí era
demasiado amarga para beber. (Por eso el lugar se llama Mara.)

\bibverse{24} Entonces el pueblo se quejó a Moisés, preguntando: ``¿Qué
vamos a beber?'' \bibverse{25} Moisés le pidió ayuda al Señor, y el
Señor le mostró un trozo de madera. Cuando lo arrojó al agua, se volvió
dulce.

Allí el Señor les dio reglas e instrucciones y también puso a prueba su
lealtad hacia él.\footnote{15.25 ``Lealtad hacia él'': añadido para
  mayor claridad.} \bibverse{26} Les dijo: ``Si prestan atención a lo
que dice el Señor su Dios, hagan lo que es correcto ante sus ojos,
obedezcan sus órdenes y cumplan todos sus reglamentos, entonces no les
haré sufrir ninguna de las enfermedades que les di a los egipcios porque
yo soy el Señor que los sana''.

\bibverse{27} Luego viajaron a Elim, que tenía doce manantiales de agua
y setenta palmeras. Allí acamparon junto al agua.

\hypertarget{section-15}{%
\section{16}\label{section-15}}

\bibverse{1} Toda la comunidad israelita dejó Elim y se fue al desierto
de pecado, entre Elim y Sinaí. Esto fue el día quince del segundo mes
después de que dejaran la tierra de Egipto. \bibverse{2} Allí, en el
desierto, se quejaron a Moisés y a Aarón.

\bibverse{3} ````¡El Señor debería habernos matado en Egipto!'' les
dijeron los israelitas. ``Al menos allí podíamos sentarnos junto a ollas
de carne y comer pan hasta que estuviéramos llenos. ¡Pero tenías que
traernos a todos aquí en el desierto para matarnos de hambre!''

\bibverse{4} El Señor le dijo a Moisés: ``Ahora haré llover pan del
cielo para ustedes. Cada día la gente debe salir y recoger lo suficiente
para ese día. Voy a ponerlos a prueba con esto para saber si seguirán
mis instrucciones o no. \bibverse{5} El sexto día deben recoger el doble
de lo habitual y prepararlo''.

\bibverse{6} Entonces Moisés y Aarón explicaron a todos los israelitas:
``Esta tarde tendrán la prueba de que el Señor fue quien los sacó de
Egipto \bibverse{7} y por la mañana verán la gloria del Señor desplegada
al responder a las quejas que los ha oído hacer contra él. ¿Por qué
debería quejarse con nosotros? ¡No somos nadie!'''

\bibverse{8} Entonces Moisés continuó: ``El Señor les dará esta tarde
carne para comer y por la mañana todo el pan que quieran, porque ha oído
sus quejas contra él. ¿Por qué se queja ante nosotros, nadie? Tus quejas
no están dirigidas contra nosotros, sino contra el Señor''.

\bibverse{9} Entonces Moisés dijo a Aarón: ``Dile a toda la comunidad
israelita: `Preséntense ante el Señor, porque ha oído sus quejas'\,''.

\bibverse{10} Mientras Aarón aún hablaba a todos los israelitas, miraron
hacia el desierto y vieron aparecer la gloria del Señor en una nube.

\bibverse{11} Entonces el Señor le dijo a Moisés: \bibverse{12} ``He
oído las quejas de los israelitas. Diles: `Por la tarde comerás carne, y
por la mañana tendrás todo el pan que quieras'. Entonces sabrán que yo
soy el Señor su Dios''.

\bibverse{13} Esa noche las codornices volaron y aterrizaron, llenando
el campamento. Por la mañana, el rocío cubrió el suelo alrededor del
campamento. \bibverse{14} Una vez que el rocío se había ido, había una
capa delgada y escamosa en el desierto, que parecía cristales de
escarcha en el suelo. \bibverse{15} Cuando los israelitas lo vieron, se
preguntaron ``¿Qué es?'' porque no tenían ni idea de lo que era.

Así que Moisés les explicó, ``Es el pan que el Señor ha provisto para
que coman. \bibverse{16} Esto es lo que el Señor les ha ordenado hacer:
``Todos ustedes recogerán lacantidad que les sea necesaria. Tomen un
gómer por cada persona en su tienda''.

\bibverse{17} Los israelitas hicieron lo que se les dijo. Algunos
recolectaron más, mientras que otros recolectaron menos. \bibverse{18}
Pero cuando lo midieron en gomeres, a los que habían recogido mucho no
les sobraba nada, mientras que a los que sólo habían recogido un poco
les sobraba. Cada persona recolectó tanto como necesitaba para comer.

\bibverse{19} Entonces Moisés les dijo: ``Nadie debe dejar nada para
mañana''. \bibverse{20} Pero algunos no escucharon a Moisés. Dejaron un
poco para el día siguiente, y estaba lleno de gusanos y olía mal. Y
Moisés se enfadó con ellos.

\bibverse{21} Así que cada mañana todos recogían todo lo que
necesitaban, y cuando el sol se calentaba, se desvanecía. \bibverse{22}
Sin embargo, en el sexto día, recogieron el doble de esta comida, dos
gomeres por cada persona. Todos los líderes israelitas vinieron y le
dijeron a Moisés lo que habían hecho. \bibverse{23} Moisés respondió:
``Estas son las instrucciones del Señor: `Mañana es un día especial de
descanso, un sábado santo para honrar al Señor. Así que horneen lo que
quieran, y hiervan lo que quieran. Luego aparten lo que quede y
guárdenlo hasta la mañana.'\,''

\bibverse{24} Así que lo guardaron hasta la mañana como Moisés había
ordenado, y no olía mal ni tenía gusanos. \bibverse{25} Moisés les dijo:
``Coman hoy, porque hoy es un sábado para honrar al Señor. Hoy no
encontrarán nada ahí fuera. \bibverse{26} Pueden salir a recolectar
durante seis días, pero el séptimo día, el sábado, no habrá nada que
puedan recolectar''. \bibverse{27} Aún así, el séptimo día algunas
personas todavía salieron a recolectar, pero no encontraron nada.

\bibverse{28} El Señor le dijo a Moisés: ``¿Cuánto tiempo te negarás a
obedecer mis órdenes e instrucciones? \bibverse{29} Debes entender que
el Señor te ha dado el sábado, así que el sexto día te dará comida para
dos días. El séptimo día, todos tienen que quedarse donde están, y nadie
tiene que salir'' \bibverse{30} Así que el pueblo no hizo ningún trabajo
en el séptimo día.

\bibverse{31} Los israelitas llamaron a esta comida maná.\footnote{16.31
  Que significa, ``¿Qué es esto?'' Ver versículo 15.}Era blanca como la
semilla de cilantro y sabía a obleas con miel. \bibverse{32} Moisés
dijo: ``Esto es lo que el Señor ha ordenado: `Guarda un gomer de maná
como recordatorio para las generaciones futuras, para que puedan ver la
comida que usé para alimentarlos en el desierto cuando los saqué de
Egipto'\,''. \bibverse{33} Así que Moisés le dijo a Aarón: ``Toma un
frasco y pon un gomer de maná en él. Luego ponlo ante el Señor para que
lo guarde como un recordatorio para las generaciones futuras''.
\bibverse{34} Aarón lo hizo y colocó la jarra delante del Testimonio,+
16.34 El significado de este término en el contexto es incierto.
Normalmente se refiere a las dos tablas de los Diez Mandamientos (ver
25:16, 40:20 etc.) El recipiente con maná fue finalmente colocado dentro
del Arca del Pacto junto con las tablas de piedra de los Diez
Mandamientos, pero ni el arca ni las tablas existían todavía (ver
capítulos 25 y 26). para que se conservara tal y como el Señor se lo
había ordenado a Moisés. \bibverse{35} Los israelitas comieron maná
durante cuarenta años, hasta que llegaron a la tierra en la que se
asentarían; comieron maná hasta que llegaron a la frontera de Canaán.
\bibverse{36} (Un gómer es una décima parte de una efa).

\hypertarget{section-16}{%
\section{17}\label{section-16}}

\bibverse{1} Todos los israelitas dejaron el desierto de Sin, yendo de
un lugar hacia otro, según las órdenes del Señor. Acamparon en Refidim,
pero no había agua para que el pueblo la bebiera. \bibverse{2} Algunos
de ellos vinieron y se quejaron a Moisés, diciendo: ``¡Danos agua para
beber!'' Moisés respondió,

``¿Por qué se quejas conmigo?'' Preguntó Moisés. ``¿Por qué intentan
desafiar al Señor?''

\bibverse{3} Pero el pueblo estaba tan sediento de agua que se quejó a
Moisés, diciendo: ``¿Por qué tuviste que sacarnos de Egipto? ¿Intentas
matarnos a nosotros y a nuestros hijos y ganado de sed?''

\bibverse{4} Moisés le gritó al Señor: ``¿Qué voy a hacer con esta
gente? ¡Un poco más de esto y me apedrearán!''

\bibverse{5} El Señor le dijo a Moisés: ``Ve delante del pueblo y
llévate a algunos de los ancianos de Israel contigo. Lleva contigo el
bastón que usaste para golpear el Nilo, y sigue adelante. \bibverse{6}
Mira, me pararé a tu lado junto a la roca en Horeb. Cuando golpees la
roca, el agua se derramará para que la gente beba''. Así que Moisés hizo
esto mientras los ancianos de Israel observaban. \bibverse{7} Llamó al
lugar Masá y Meribá\footnote{17.7 Masá significa ``prueba'' y Meribá
  significa ``queja''.}porque los israelitas discutieron allí, y porque
desafiaron al Señor, diciendo: ``¿Está el Señor con nosotros o no?''

\bibverse{8} Entonces vinieron unos amalecitas y atacaron a los
israelitas en Refidim. \bibverse{9} Moisés le dijo a Josué: ``Escoge
algunos hombres y sal a combatir a los amalecitas. Mañana me pararé en
la cima de esta colina con el bastón de Dios''.

\bibverse{10} Josué hizo lo que le dijo Moisés y luchó contra los
amalecitas, mientras que Moisés, Aarón y Hur subieron a la cima de la
colina. \bibverse{11} Mientras Moisés sostenía el bastón\footnote{17.11
  ``El bastón'': implícito.} con sus manos,los israelitas eran los que
ganaban, pero cuando los bajaba, eran los amalecitas. \bibverse{12} Así
que cuando las manos de Moisés se volvieron pesadas, los otros tomaron
una piedra y la pusieron debajo de él para que se sentara. Aarón y Hur
se pararon a cada lado de Moisés y le levantaron las manos. De esta
manera sus manos se mantuvieron firmes hasta que el sol se puso.
\bibverse{13} Como resultado, Josué derrotó al ejército amalecita.

\bibverse{14} El Señor le dijo a Moisés: ``Escribe todo esto en un
pergamino como recordatorio y léeselo en voz alta a Josué, porque voy a
eliminar por completo a los amalecitas para que nadie en la tierra se
acuerde de ellos''.

\bibverse{15} Moisés construyó un altar y lo llamó ``el Señor es mi
bandera de la victoria''. \bibverse{16} ``¡Levanten el estandarte de la
victoria del Señor!'', declaró Moisés. ``¡El Señor seguirá luchando
contra los amalecitas por todas las generaciones!''

\hypertarget{section-17}{%
\section{18}\label{section-17}}

\bibverse{1} EntoncesJetro\footnote{18.1 También llamado Reuel en el
  capítulo 2.}, el suegro de Moisés y sacerdote de Madián, escuchó todo
lo que Dios había hecho por Moisés y su pueblo, los israelitas, y cómo
el Señor los había sacado de Egipto. \bibverse{2} Cuando Moisés envió a
casa a su esposa Séfora, su suegro Jetro la acogió, \bibverse{3} junto
con sus dos hijos. Uno de los hijos se llamaba Gersón,+ 18.3 Ver 2:22.ya
que Moisés había dicho: ``He sido un extranjero en tierra extranjera''.
\bibverse{4} El otro hijo se llamaba Eliezer,+ 18.4 Que significa, ``mi
Dios es mi ayuda''.porque Moisés había dicho: ``El Dios de mi padre fue
mi ayuda, y me salvó de la muerte de la mano del Faraón''.

\bibverse{5} El suegro de Moisés, Jetro, junto con la esposa y los hijos
de Moisés, fue a verlo en el desierto en el campamento cerca de la
montaña de Dios. \bibverse{6} A Moisés se le dijo de antemano: ``Yo, tu
suegro Jetro, vengo a verte junto con tu esposa y sus dos hijos''.

\bibverse{7} Moisés salió al encuentro de su suegro y se inclinó y le
besó. Se preguntaron cómo estaban y luego entraron en la tienda.
\bibverse{8} Moisés le contó a su suegro todo lo que el Señor había
hecho al Faraón y a los egipcios en favor de los israelitas, todos los
problemas que habían experimentado en el camino y cómo el Señor los
había salvado.

\bibverse{9} Jetro se alegró de escuchar todas las cosas buenas que el
Señor había hecho por Israel cuando los había salvado de los egipcios.
\bibverse{10} Jetro anunció: ``Bendito sea el Señor, que te salvó de los
egipcios y del Faraón. \bibverse{11} Esto me convence de que el Señor es
más grande que todos los demás dioses, porque salvó al pueblo de los
egipcios cuando actuaron tan arrogantemente con los israelitas''.

\bibverse{12} EntoncesJetro presentó un holocausto y sacrificios a Dios,
y Aarón vino con todos los ancianos de Israel para comer con él en
presencia de Dios.

\bibverse{13} Al día siguiente Moisés se sentó como juez del pueblo, y
le presentaron sus casos desde la mañana hasta la noche. \bibverse{14}
Cuando su suegro vio todo lo que Moisés estaba haciendo por el pueblo,
preguntó: ``¿Qué es todo esto que estás haciendo por el pueblo? ¿Por qué
te sientas solo como juez, con todo el mundo presentándote sus casos de
la mañana a la noche?''

\bibverse{15} ``Porque el pueblo viene a mí para consultar la decisión
de Dios'', respondió Moisés. \bibverse{16} ``Cuando discuten sobre algo,
el caso se presenta ante mí para decidir entre uno de ellos, y les
explico las leyes y reglamentos de Dios''.

\bibverse{17} Jetro le dijo: ``Lo que estás haciendo no es lo mejor.
\bibverse{18} Tú y los que vienen a ti se van a agotar, porque la carga
de trabajo es demasiado pesada. No pueden manejarlo solos. \bibverse{19}
Así que, por favor, escúchame. Voy a darte un consejo, y Dios estará
contigo. Sí, debes continuar siendo el representante del pueblo ante
Dios, y llevarle sus casos a él. \bibverse{20} Sigue enseñándoles las
leyes y los reglamentos. Muéstrales cómo vivir y el trabajo que deben
hacer. \bibverse{21} Pero ahora debes elegir entre el pueblo hombres
competentes, hombres que respeten a Dios y que sean dignos de confianza
y no corruptos. Ponlos a cargo del pueblo como líderes de miles,
cientos, cincuenta y decenas. \bibverse{22} Estos hombres deben juzgar
al pueblo de manera continua. Pueden traertelos asuntos más grandes,
pero podrán decidirpor sí mismos respecto a todos los asuntos pequeños.
De esta manera su carga se hará más ligera a medida que la compartan.
\bibverse{23} Si sigues mi consejo, y si es lo que Dios te dice que
hagas, entonces podrás sobrevivir, y toda esta gente podrá volver a casa
satisfecha de que sus casos han sido escuchados''.\footnote{18.23
  ``satisfecha de que sus casos han sido escuchados'': Literalmente,
  ``en paz''. La palabra shalom, sin embargo, significa más que paz,
  pues también tiene el significado de bienestar y armonía dentro de la
  comunidad.}

\bibverse{24} Moisés escuchó lo que dijo su suegro y siguió todos sus
consejos. \bibverse{25} Así que Moisés eligió hombres competentes de
todo Israel y los puso a cargo del pueblo como líderes de miles,
cientos, cincuenta y decenas. \bibverse{26} Y actuaron como jueces del
pueblo de manera continua. Llevaban los casos difíciles a Moisés, pero
juzgaban los pequeños asuntos por sí mismos.

\bibverse{27} Entonces Moisés envió a Jetrode camino, y regresó a su
propio país.

\hypertarget{section-18}{%
\section{19}\label{section-18}}

\bibverse{1} Dos meses después del día\footnote{19.1 ``Dos meses después
  del día'': Literalmente, ``El día de la tercera luna nueva''.}en que
habían salido de Egipto, los israelitas llegaron al desierto del Sinaí.
\bibverse{2} Habían partido de Refidim, y después de entrar en el
desierto del Sinaí acamparon allí frente a la montaña.

\bibverse{3} Moisés subió al monte de Dios. Y el Señor habló con Moisés
desde la montaña y le dijo: ``Esto es lo que debes decirles a los
descendientes de Jacob, los israelitas: \bibverse{4} `Vieron con sus
propios ojos lo que hice con los egipcios, y cómo los llevé sobre alas
de águila, y cómo los traje hacia mí. \bibverse{5} Ahora bien, si
realmente obedecen lo que digo y cumplen el acuerdo conmigo, entonces,
de todas las naciones, serán mi pueblo especial. Aunque que el mundo
entero es mío, \bibverse{6} para mí serán un reino de sacerdotes, una
nación santa.' Esto es lo que debes decirles a los israelitas''.

\bibverse{7} Entonces Moisés bajó, convocó a los ancianos del pueblo y
les presentó todo lo que el Señor le había ordenado decir. \bibverse{8}
Todos respondieron: ``Prometemos hacer todo lo que el Señor diga''.
Entonces Moisés llevó la respuesta del pueblo al Señor.

\bibverse{9} El Señor le dijo a Moisés: ``Voy a ir hacia ti en una nube
espesa para que el pueblo me oiga hablar contigo y así siempre confiarán
en ti''. Entonces Moisés le informó al Señor lo que el pueblo había
dicho.

\bibverse{10} El Señor le dijo a Moisés: ``Baja y prepáralos
espiritualmente\footnote{19.10 ``Prepáralos espiritualmente'':
  Literalmente, ``conságralos, apártalos,'' quizás mediante algún
  ritual. Ver también versículos 14 y 22.} hoy y mañana. Deben lavar sus
ropas \bibverse{11} y estar listos al tercer día porque es cuando el
Señor descenderá al Monte Sinaí a la vista de todos. \bibverse{12}
Establezcan un límite alrededor de la montaña y adviértanles:'Tengan
cuidado yno intenten subir a la montaña, ¡ni siquiera la toquen! Porque
cualquiera que toque la montaña seguramente morirá. No toquen a ninguna
persona o animal que haya tocado la montaña. \bibverse{13} Asegúrate de
que sean apedreados o disparados con flechas, pues no se les debe
permitir vivir. Sólo cuando escuchen un fuerte sonido de cuerno de
carnero,el pueblo podrá subir a la montaña''.

\bibverse{14} Moisés bajó de la montaña y preparó al pueblo
espiritualmente y lavó sus ropas. \bibverse{15} Luego instruyó al
pueblo: ``Prepárense para el tercer día, y no tengan relación íntima
con\footnote{19.15 ``No tengan relación íntima con'': Literalmente, ``no
  se acerquen a mujer alguna''.}una mujer''.

\bibverse{16} Cuando llegó la mañana del tercer día hubo truenos y
relámpagos, y una nube espesa cubrió la montaña. Hubo un fuerte sonido
de cuerno de carnero, y todos en el campamento temblaron de miedo.
\bibverse{17} Moisés condujo al pueblo fuera del campamento para
encontrarse con Dios. Se pararon al pie de la montaña \bibverse{18} El
humo se derramó sobre todo el Monte Sinaí porque la presencia del Señor
había descendido como el fuego. El humo se elevó como el humo de un
horno, y toda la montaña tembló furiosamente. \bibverse{19} A medida que
el sonido del cuerno de carnero se hacía cada vez más fuerte, Moisés
hablaba, y Dios le respondía con una voz fuerte y atronadora.
\bibverse{20} El Señor descendió a la cima del Monte Sinaí, y llamó a
Moisés para que subiera allí. Así que Moisés subió, \bibverse{21} y el
Señor le dijo: ``Vuelve a bajar, y adviértele al pueblo que no se
esfuercen en cruzar el límite para intentar subir donde está el Señor o
morirán. \bibverse{22} Incluso los sacerdotes, que vienen ante el Señor,
deben prepararse espiritualmente, para que el Señor no los castigue''.

\bibverse{23} Pero Moisés le dijo al Señor: ``El pueblo no puede subir
al monte Sinaí. Tu mismo nos advertiste diciendo: `Establezcan un límite
alrededor de la montaña, y considérenla como sagrada.'\,''\footnote{19.23
  ``Trátenla como sagrada'': se utiliza la misma palabra que para
  preparar o consagrar al pueblo espiritualmente. Sin embargo, es
  evidente que un objeto inanimado como una montaña no puede ser
  ``consagrada'' de la misma manera que una persona.}

\bibverse{24} El Señor le dijo: ``Baja y trae a Aarón contigo. Pero los
sacerdotes y el pueblo no deben tratar de subir donde estáel Señor, o él
los castigará''

\bibverse{25} Entonces Moisés bajó y le explicó al pueblo lo que el
Señor había dicho.\footnote{19.25 ``Lo que el Señor le había dicho'':
  añadido para mayor claridad.}

\hypertarget{section-19}{%
\section{20}\label{section-19}}

\bibverse{1} Dios dijo todas las siguientes palabras:

\bibverse{2} ``Yo soy el Señor tu Dios, que te sacó de Egipto, de la
tierra de tu esclavitud.

\bibverse{3} Notendrás a otros dioses aparte de mi.

\bibverse{4} Noharás ningún tipo de ídolo, ya sea que se parezca a algo
arriba en los cielos, o abajo en la tierra, ni debajo en las aguas.
\bibverse{5} No debes inclinarte ante ellos ni adorarlos, porque yo soy
el Señor tu Dios y soy celosamente exclusivo. Yo pongo las consecuencias
del pecado de los que me odian sobre sus hijos, sus nietos y sus
bisnietos; \bibverse{6} pero muestro mi amor fiel a las miles de
generaciones que me aman y guardan mis mandamientos.

\bibverse{7} No debes usar mal el nombre del Señor tu Dios, porque el
Señor no perdonará a nadie que use su nombre de forma incorrecta.

\bibverse{8} Recuerda el día de reposo para santificarlo. \bibverse{9}
Tienes seis días para trabajar y ganarte el sustento, \bibverse{10} pero
el séptimo día es el sábado para honrar al Señor tu Dios. En este día no
debes hacer ningún trabajo, ni tú, ni tu hijo o hija, ni tu esclavo o
esclava, ni el ganado, ni el extranjero que esté contigo. \bibverse{11}
Porqueen seis días el Señor hizo los cielos y la tierra, el mar y todo
lo que hay en ellos, y luego descansó en el séptimo día. Por eso el
Señor bendijo el día de reposo y lo hizo santo.

\bibverse{12} Honra a tu padre y a tu madre, para que vivas mucho tiempo
en la tierra que el Señor tu Dios te da.

\bibverse{13} No cometerás asesinato.

\bibverse{14} No cometerás adulterio.

\bibverse{15} No robarás.

\bibverse{16} No darás falso testimonio contra otros.

\bibverse{17} No desearás tener la casa de otro. No desearás a su
esposa, nia su esclavo o esclava, ni a su buey o asno, ni cualquier otra
cosa que le pertenezca''.

\bibverse{18} Cuando todo el pueblo oyó el trueno y el sonido de la
trompeta, y vio el relámpago y el humo de la montaña, temblaron de miedo
y se alejaron. \bibverse{19} ``Habla con nosotros y te escucharemos'',
le dijeron a Moisés. ``Pero no dejes que Dios nos hable, o moriremos''.

\bibverse{20} Moisés les dijo: ``No teman, porque Dios sólo ha venido a
probarlos. Quiere que le tengan miedo para que no pequen''.
\bibverse{21} Entonces el pueblo se alejó mucho cuando Moisés se acercó
a la espesa y oscura nube donde estaba Dios.

\bibverse{22} El Señor le dijo a Moisés: ``Esto es lo que les debes
decir a los israelitas: ``Vieron con sus propios ojos que les hablé
desde el cielo. \bibverse{23} Noharán ningún ídolo de plata o de oro ni
lo adorarán aparte de mí. \bibverse{24} Háganme un altar de tierra y
sacrifiquen sobre él sus holocaustos y ofrendas de paz, sus ovejas, sus
cabras y su ganado. Dondequiera que decida que me adoren, vendé a
ustedes y los bendeciré. \bibverse{25} Ahora bien, si me hacen un altar
de piedras, no lo construyas con piedras cortadas, porque si usan un
cincel para cortar la piedra, dejan de ser sagradas. \bibverse{26}
Además, no deben subir a mi altar con escalones, para que no se vean sus
partes privadas''.

\hypertarget{section-20}{%
\section{21}\label{section-20}}

\bibverse{1} ``Estos son los reglamentos que debe presentarles:

\bibverse{2} Si compran un esclavo hebreo, debe trabajar para ustedes
durante seis años. Pero en el séptimo año, debe ser liberado sin tener
que pagar nada. \bibverse{3} Si era soltero cuando llegó, debe irse
soltero. Si tenía una esposa cuando llegó, ella debe irse con él.
\bibverse{4} Si su amo le da una esposa y ella tiene hijos con él, la
mujer y sus hijos pertenecerán a su amo, y sólo el hombre será liberado.

\bibverse{5} Sin embargo, si el esclavo declara formalmente: ``Amo a mi
señor, a mi esposa y a mis hijos; no quiero ser liberado' \bibverse{6}
entonces su señor lo llevará ante los jueces.\footnote{21.6 La palabra
  utilizada aquí también puede referirse a Dios, pero en este contexto
  parece que se está hablando de un tribunal civil. Ver también 22:8, 9.}Luego
lo pondrá de pie contra la puerta o el poste de la puerta y usará una
herramienta de metal para hacerle un agujero en la oreja. Entonces
trabajará para su amo de por vida.

\bibverse{7} Si un hombre vende a su hija como esclava, no será liberada
de la misma manera que los esclavos. \bibverse{8} Si el hombre que la
eligió para sí\footnote{21.8 ``La eligió para sí'': probablemente
  refiriéndose a ella como concubina.}no está satisfecho con ella, debe
dejar que sea comprada de nuevo. No podrá venderla a los extranjeros, ya
que ha sido injusto con ella. \bibverse{9} Si decide dársela a su hijo,
debe tratarla como a una hija. \bibverse{10} Si toma a otra mujer, no
debe reducir los subsidios de comida y ropa, ni los derechos maritales
de la primera. \bibverse{11} Si no le da estas tres cosas, ella es libre
de irse sin pagar nada.

\bibverse{12} Todo aquel que golpee y mate a otra persona debe ser
ejecutado. \bibverse{13} Sin embargo, si no fue intencional y Dios
permitió que sucediera, entonces arreglaré un lugar para ustedes donde
puedan correr y estar seguros. \bibverse{14} Pero si alguien planea
deliberadamente y mata a propósito a otro, debe alejarlo de mi
altar\footnote{21.14 ``De mi altar'': donde la gente iba considerándolo
  santuario.} y ejecutarlo.

\bibverse{15} Cualquiera que golpee a su padre o madre debe ser
ejecutado. \bibverse{16} Cualquiera que secuestre a alguien más debe ser
ejecutado, ya sea que la víctima sea vendida o que aún esté en su
posesión.

\bibverse{17} Cualquiera que desprecie a su padre o a su madre debe ser
ejecutado.

\bibverse{18} Si los hombres están peleando y uno golpea al otro con una
piedra o con el puño, y el hombre herido no muere pero tiene que
permanecer en cama, \bibverse{19} y luego se levanta y camina afuera con
su bastón, entonces el que lo golpeó no será castigado. Aún así, debe
compensar al hombre por el tiempo perdido de su trabajo y asegurarse de
que esté completamente curado.

\bibverse{20} Cualquiera que golpee a su esclavo o esclava con una vara,
y el esclavo muera como resultado, debe ser castigado. \bibverse{21} Sin
embargo, si después de un día o dos el esclavo mejora, el dueño no será
castigado porque el esclavo es de su propiedad.

\bibverse{22} Si los hombres que están peleando golpean a una mujer
embarazada para que dé a luz prematuramente,\footnote{21.22 ``Dé a luz
  prematuramente'': o, ``tiene un aborto espontáneo''.}pero no se
produce ninguna lesión grave, debe ser multado con la cantidad que el
marido de la mujer demande y según lo permitan los jueces. \bibverse{23}
Pero si se produce una lesión grave, entonces debe pagar una vida por
otra vida; \bibverse{24} ojo por ojo, diente por diente, mano por mano,
pie por pie, \bibverse{25} quemadura por quemadura, herida por herida y
moretón por moretón.

\bibverse{26} El que golpee a su esclavo o esclava en el ojo y lo
ciegue, debe liberar al esclavo como compensación por el ojo.
\bibverse{27} El que golpee el diente de su esclavo o esclava debe
liberar al esclavo como compensación por el diente.

\bibverse{28} Si un buey usa sus cuernos para matar a un hombre o una
mujer, el buey debe ser apedreado hasta morir, y su carne no debe ser
comida. Pero el dueño del buey no será castigado. \bibverse{29} Pero si
el buey ha herido repetidamente a la gente con sus cuernos, y su dueño
ha sido advertido pero aún no lo tiene bajo control, y mata a un hombre
o una mujer, entonces el buey debe ser apedreado hasta morir y su dueño
también debe ser ejecutado. \bibverse{30} Pero si en lugar de ello se
exige el pago de una indemnización, el propietario puede compensar su
vida pagando la totalidad de la indemnización exigida. \bibverse{31}
Pero si en lugar de ello se exige el pago de una indemnización, el
propietario puede salvar su vida pagando la totalidad de la
indemnización exigida. \bibverse{32} Si el buey usa sus cuernos y mata a
un esclavo o esclava, el propietario del buey debe pagar treinta siclos
de plata al amo del esclavo, y el buey debe ser apedreado hasta la
muerte.

\bibverse{33} Si alguien quita la tapa de una cisterna o cava una y no
la cubre, y un buey o un asno cae en ella, \bibverse{34} el dueño de la
fosa debe pagar una compensación al dueño del animal y quedarse con el
animal muerto.

\bibverse{35} Si el buey de alguien hiere al de otro y éste muere, debe
vender al buey vivo y compartir el dinero recibido; también debe
compartir al animal muerto. \bibverse{36} Pero si se sabe que el buey ha
herido repetidamente a personas con sus cuernos, y su dueño ha sido
advertido pero aún no lo tiene bajo control, debe pagar una compensación
completa, buey por buey; pero el dueño puede quedarse con el animal
muerto''.

\hypertarget{section-21}{%
\section{22}\label{section-21}}

\bibverse{1} ``Quien robe un buey o una oveja y la mate o la venda,
deberá devolver cinco bueyes por un buey y cuatro ovejas por una oveja.

\bibverse{2} Si se descubre a un ladrón entrando en la casa de alguien y
es golpeado hasta la muerte, nadie será culpable de asesinato.
\bibverse{3} Pero si ocurre durante el día, entonces alguien es culpable
de asesinato. El ladrón debe devolver todo lo robado. Si no tiene nada,
entonces debe ser vendido para pagar lo que fue robado. \bibverse{4} Si
lo que fue robado es un animal vivo que todavía tiene, ya sea un buey,
un asno o una oveja, debe devolver el doble.

\bibverse{5} Si el ganado pasta en un campo o en un viñedo y su dueño lo
deja vagar para que pasten en el campo de otro, el dueño debe pagar una
compensación con lo mejor de sus propios campos o viñedos.

\bibverse{6} Si se inicia un incendio que se extiende a los arbustos
espinosos y luego quema el grano apilado o en pie, o incluso todo el
campo, la persona que inició el fuego debe pagar una compensación
completa.

\bibverse{7} Si alguien le da a su vecino dinero o posesiones para que
las guarde y se las roban de la casa del vecino, si el ladrón es
atrapado debe pagar el doble. \bibverse{8} Si el ladrón no es atrapado,
el propietario de la casa debe comparecer ante los jueces para averiguar
si se llevó la propiedad de su vecino.

\bibverse{9} Si hay una discusión sobre la propiedad de un buey, un
asno, una oveja, una prenda de vestir, o cualquier cosa que se haya
perdido y alguien dice: ``Esto es mío'', ambas partes deben llevar su
caso ante los jueces. Aquel al que los jueces encuentren culpable debe
devolverle el doble al otro.

\bibverse{10} Si alguien pide a un vecino que cuide un asno, un buey,
una oveja o cualquier otro animal, pero éste muere o se lesiona o es
robado sin que nadie se dé cuenta, \bibverse{11} entonces se debe
prestar un juramento ante el Señor para decidir si el vecino ha tomado
la propiedad del dueño. El propietario debe aceptar el juramento y no
exigir una compensación.

\bibverse{12} Sin embargo, si el animal fue realmente robado al vecino,
debe compensar al propietario. \bibverse{13} Si fue matado y despedazado
por un animal salvaje, el vecino deberá presentar el cadáver como prueba
y no necesita pagar indemnización.

\bibverse{14} Si alguien toma prestado un animal del vecino y éste
resulta herido o muere mientras su dueño no está presente, debe pagar
una indemnización en su totalidad. \bibverse{15} Si el propietario
estaba presente, no se pagará ninguna compensación. Si el animal fue
alquilado, sólo se debe pagar el precio del alquiler.

\bibverse{16} Si un hombre seduce a una virgen no comprometida para
casarse y se acuesta con ella, debe pagar el precio completo de la novia
para que se convierta en su esposa. \bibverse{17} Si el padre de ella se
niega rotundamente a dársela, el hombre debe pagar la misma cantidad que
el precio de la novia por una virgen.

\bibverse{18} No se debe permitir que viva una mujer que practique la
brujería.

\bibverse{19} Todo aquel que tenga relaciones sexuales con un animal
debe ser ejecutado.

\bibverse{20} Cualquiera que se sacrifique a cualquier otro dios que no
sea el Señor debe ser apartado y ejecutado.\footnote{22.20 ``Set apart
  and executed'': the term used here means ``devoted to destruction'' in
  the sense they now are to suffer God's punishment.}

\bibverse{21} No se debe explotar o maltratar a un extranjero. Recuerden
que ustedes mismos fueron una vez extranjeros en Egipto.

\bibverse{22} Nose aprovechen de ninguna viuda o huérfano. \bibverse{23}
Si los maltratan, y ellos me piden ayuda, responderé a su clamor.
\bibverse{24} Me enfadaré y mataré a quien se aproveche de ellos con
espada. Entonces sus esposas se convertirán en viudas y sus hijos
quedarán huérfanos.

\bibverse{25} Si le prestas dinero a mi pueblo porque son pobres, no te
comportes como un prestamista con ellos. No debes cobrarles ningún
interés.

\bibverse{26} Si necesitas la capa de tu vecino como garantía de un
préstamo, debes devolvérsela antes de la puesta de sol, \bibverse{27}
porque es la única ropa que tiene para su cuerpo. ¿De otro modo, con qué
dormirá? Y si me pide ayuda, le escucharé, porque soy misericordioso.

\bibverse{28} Nodesprecies a Dios ni maldigas al líder de tu pueblo.

\bibverse{29} Noretengas las ofrendas requeridas de tus productos,
aceite de oliva y vino.\footnote{22.29 ``aceite de oliva y vino'':
  Literalmente, ``mejores vendimias''.}Debes darme el primogénito de tus
hijos. \bibverse{30} También debes darme el primogénito de tus vacas,
ovejas y cabras. Podrás dejarlos con sus madres durante los primeros
siete días, pero debes darmelos al octavo día.

\bibverse{31} Ustedes deben ser un pueblo santo para mí. No coman ningún
cadáver de animal que encuentren en el campo y que haya sido asesinado
por animales salvajes. Láncenlos a los perros para que se lo coman''.

\hypertarget{section-22}{%
\section{23}\label{section-22}}

\bibverse{1} ``No ayudes a difundir historias que son mentiras. No
ayudes a la gente mala dando mal testimonio.

\bibverse{2} No sigas a la multitud haciendo el mal. Cuando testifiques
en un juicio, no corrompas la justicia poniéndote del lado de la
mayoría. \bibverse{3} Tampoco\footnote{23.3 ``Tampoco'': añadido para
  mayor claridad. La justicia tiene que ser imparcial, así que mostrar
  favoritismo a cualquier parte está mal. Sin embargo, el problema más
  usual es la negación de la justicia a los pobres (ver por ejemplo el
  versículo 6.} muestres favoritismo hacia los pobres en sus casos
legales.

\bibverse{4} Si te encuentras con el buey o asno de tu enemigo que se ha
extraviado, devuélveselo. \bibverse{5} Si ves el asno de alguien que te
odia y que ha caído porel peso de su carga, no lo dejes ahí. Debes
detenerte y ayudarle.

\bibverse{6} No debes impedir que los pobres obtengan justicia en sus
demandas. \bibverse{7} No tengas nada que ver con hacer falsas
acusaciones. No maten a los inocentes ni a los que hacen el bien, porque
no dejaré que los culpables queden impunes.

\bibverse{8} No aceptes sobornos, porque un soborno ciega a los que
pueden ver y socava las pruebas de los honestos.

\bibverse{9} No abusen de los extranjeros que viven entre ustedes, pues
ustedes saben muy bien lo que es ser extranjeros, ya que una vez fueron
extranjeros en Egipto.

\bibverse{10} Seis años deben sembrar la tierra y cosechar los cultivos,
\bibverse{11} pero en el séptimo año deben dejarla descansar y dejarla
sin cultivar, para que los pobres puedan comer lo que crece
naturalmente\footnote{23.11 ``Lo que crece naturalmente'': añadido para
  mayor claridad.} en el campo y los animales salvajes puedan terminar
lo que queda. Sigan el mismo procedimiento para sus viñedos y olivares.

\bibverse{12} Tendrán seis días para hacer su trabajo, pero el séptimo
día deben dejar de trabajar, para que su buey y su asno puedan
descansar, y las familias de sus esclavos puedan recuperar el aliento,
así como los extranjeros que viven entre ustedes.

\bibverse{13} Asegúrate de prestar atención a todo lo que te he dicho.
Que no pase por tu mente invocar el nombre de otros dioses, ni siquiera
debes mencionarlos.

\bibverse{14} Tres veces al año celebrarán una fiesta dedicada a mí.
\bibverse{15} Deben observar el Festival de los Panes sin Levadura como
se los he instruído.\footnote{23.15 Ver el capítulo 13.}Deben comer pan
sin levadura durante siete días en el momento apropiado en el mes de
Abib, porque ese fue el mes en que saliste de Egipto. Nadie puede venir
delante mí sin traer una ofrenda.

\bibverse{16} Ytambién observarán el Festival de las Cosechas cuando
presenten las primicias de los productos de lo que hayan sembrado en los
campos. Por último, deben observar el Festival de la Cosecha\footnote{23.16
  El nombre más familiar, dado más tarde, es el Festival de los
  tabernáculos.} al final del año, cuando recojan la cosecha del resto
de tus cultivos en el campo. \bibverse{17} Todo varón israelita debe
presentarse ante el Señor Dios en estas tres ocasiones cada año.

\bibverse{18} No ofrecerán la sangre de mis sacrificios junto con nada
que contenga levadura, y la grasa de las ofrendas presentadas en mi
festival no debe dejarse hasta la mañana.

\bibverse{19} Traigan las mejores primicias de sus cosechas a la casa
del Señor su Dios. No cocinarán a un cabrito en la leche de su madre.

\bibverse{20} Yoenvío un ángel delante de ti para que te proteja en el
camino y te lleve al lugar que te he preparado. \bibverse{21} Asegúrate
de prestarle atención y hacer lo que te diga. No te opongas a él, porque
no perdonará la rebelión, pues lleva mi autoridad.\footnote{23.21
  ``lleva mi autoridad'': Literalmente, ``mi nombre está en medio de
  él''.}

\bibverse{22} Sin embargo, si le escuchas atentamente y haces todo lo
que te digo, entonces seré enemigo de tus enemigos y lucharé contra los
que luchan contra ti. \bibverse{23} Porque mi ángel irá delante de ti y
te llevará a la tierra de los amorreos, hititas, ferezeos, cananeos,
heveos y jebuseos, y los aniquilaré. \bibverse{24} No debes inclinarte
ante sus dioses ni adorarlos, ni seguir sus prácticas paganas. Más
biendestruirás sus ídolos y derribarás sus altares.

\bibverse{25} Adorarásal Señor tu Dios, y él bendecirá tu comida y tu
agua. Me aseguraré de que ninguno de ustedes se enferme. \bibverse{26}
Ninguna mujer tendrá un aborto espontáneo ni se quedará sin hijos. Me
aseguraré de que vivan una larga vida.

\bibverse{27} Enviaré un terror sobre mí delante de ustedes que hará que
todas las naciones que los conozcan entren en pánico. Haré que todos sus
enemigos se den la vuelta y huyan. \bibverse{28} Y enviaré
avispones\footnote{23.28 ``Avispones'': El significado de la palabra
  utilizada aquí aún se debate. Algunos lo ven de manera similar al
  ``terror'' del verso anterior que causa pánico.}delante de ti para
expulsar a los heveos, cananeos e hititas. \bibverse{29} No los
expulsaré en un año, porque la tierra se volvería desolada y tendrías
que enfrentarte a un mayor número de animales salvajes. \bibverse{30}
Poco a poco los expulsaré delante de ti, hasta que haya suficientes para
tomar posesión de la tierra.

\bibverse{31} Fijaré sus fronteras desde el Mar Rojo hasta el Mar de los
Filisteos,\footnote{23.31 ``Mar de los filisteos'': El mediterráneo.} y
desde el desierto hasta el río Éufrates. Te entregaré los habitantes de
la tierra y tú los expulsarás. \bibverse{32} No debes hacer ningún
acuerdo con ellos ni con sus dioses. \bibverse{33} No se les debe
permitir permanecer en tu tierra, de lo contrario te llevarán a pecar
contra mí. Porque si adoras a sus dioses, definitivamente se convertirán
en una trampa para ti''.

\hypertarget{section-23}{%
\section{24}\label{section-23}}

\bibverse{1} El Señor le dijo a Moisés: ``Subana la presencia del Señor,
tú y Aarón, Nadab y Abiú, y setenta de los ancianos de Israel. Deben
adorar a distancia. \bibverse{2} Sólo Moisés puede acercarse al Señor,
los demás no deben acercarse. El pueblo no puede subir al
monte\footnote{24.2 ``Al monte'': añadido para mayor claridad.} con
él''.

\bibverse{3} Moisés fue y le dijo al pueblo todas las instrucciones y
reglamentos del Señor. Todos respondieron juntos: ``¡Haremos todo lo que
el Señor diga!'' \bibverse{4} Moisés escribió todo lo que el Señor había
dicho. Se levantó temprano a la mañana siguiente y construyó un altar al
pie de la montaña, y levantó doce pilares para cada una de las doce
tribus de Israel. \bibverse{5} Luego envió a algunos jóvenes israelitas
que fueron y ofrecieron holocaustos y sacrificaron toros jóvenes como
ofrendas de paz al Señor. \bibverse{6} Moisés puso la mitad de la sangre
en tazones y roció la otra mitad en el altar.

\bibverse{7} Luego tomó el Libro del Acuerdo y se lo leyó al pueblo.
Ellos respondieron: ``Haremos todo lo que el Señor diga. Obedeceremos''.

\bibverse{8} Entonces Moisés tomó la sangre, la roció sobre el pueblo y
dijo: ``Mira, esta es la sangre del pacto que el Señor ha hecho contigo
siguiendo estos términos''.

\bibverse{9} Entonces Moisés y Aarón, Nadab y Abiú, y setenta de los
ancianos de Israel subieron al monte, \bibverse{10} y vieron al Dios de
Israel. Bajo sus pies había algo así como un pavimento de azulejos hecho
de lapislázuli, tan azul claro como el propio cielo. \bibverse{11} Pero
Dios no hirió\footnote{24.11 ``hirió'': esto se debió a la expectativa
  de que cualquiera que viera a Dios moriría (Génesis 32:30; Jueces
  6:22), respaldado por el mismo Dios (33:20)}a los líderes de Israel.
Ellos lo vieron, y luego comieron y bebieron una comida sagrada.+ 24.11
``Una comida sagrada'': añadido para mayor claridad.

\bibverse{12} Entonces el Señor le dijo a Moisés: ``Sube a mí al monte y
quédate aquí, para que te dé las tablas de piedra, con las instrucciones
y órdenes que he escrito para que las aprendan''.

\bibverse{13} Así que Moisés se fue con Josué su ayudante y subió a la
montaña de Dios. \bibverse{14} Les dijo a los ancianos: ``Quédense aquí
y esperen a que volvamos. Aarón y Hur están contigo. Si alguien tiene un
problema, puede hablar con ellos''.

\bibverse{15} Cuando Moisés subió a la montaña, la nube la cubrió.
\bibverse{16} La gloria del Señor descendió sobre el Monte Sinaí,
cubriéndolo durante seis días. En el séptimo día, el Señor llamó a
Moisés desde dentro de la nube. \bibverse{17} Para los israelitas la
gloria del Señor parecía un fuego ardiente en la cima de la montaña.
\bibverse{18} Moisés subió a la nube cuando subió a la montaña, y
permaneció en la montaña durante cuarenta días y noches.

\hypertarget{section-24}{%
\section{25}\label{section-24}}

\bibverse{1} Entonces el Señor le dijo a Moisés: \bibverse{2} ``Ordena a
los israelitas que me traigan una ofrenda. Recibirás mi ofrenda de todos
los que quieran darla.

\bibverse{3} Estos son los artículos que debes aceptar de ellos como
contribuciones: oro, plata y bronce; \bibverse{4} hilos azules, púrpura
y carmesí; lino y pelo de cabra finamente hilados; \bibverse{5} pieles
de carnero curtidas y cuero fino; madera de acacia; \bibverse{6} aceite
de oliva para las lámparas; especias para el aceite de oliva usado en la
unción y para el incienso fragante; \bibverse{7} y piedras de ónix y
otras gemas para ser usadas en la fabricación del efod y el pectoral.

\bibverse{8} Me harán un santuario para que pueda vivir entre ellos.
\bibverse{9} Debes hacer el Tabernáculo\footnote{25.9 La palabra
  ``Tabernáculo'' viene del latín para ``tienda de campaña'', y traduce
  el hebreo que se refiere a una morada, o lugar donde se habita.}y
todos sus muebles según el diseño que te voy a mostrar.

\bibverse{10} Deben hacer un Arca de madera de acacia que mida dos codos
y medio de largo por codo y medio de ancho por codo y medio de alto.
\bibverse{11} Cúbranla con oro puro por dentro y por fuera, y hagan un
adorno de oro para rodearla. \bibverse{12} Fundirán cuatro anillos de
oro y fijarlos a sus cuatro pies, dos en un lado y dos en el otro.
\bibverse{13} Harán palos de madera de acacia y cubrirlos con oro.
\bibverse{14} Colocarán las varas en los anillos de los lados del Arca,
para que pueda ser transportada. \bibverse{15} Las varas deben
permanecer en los anillos del Arca; no las saques. \bibverse{16} Pongan
dentro del Arca el testimonio que os voy a dar.

\bibverse{17} Harás una tapa de expiación\footnote{25.17 ``Cubierta de
  expiación'': la palabra usada aquí significa ``cubrir'', en el sentido
  de tratar con los pecados. La traducción tradicional de
  ``propiciatorio'' se originó en Martín Lutero. Desde un punto de vista
  físico era la ``tapa'' del Arca.}de oro puro, de dos codos y medio de
largo por codo y medio de ancho. \bibverse{18} Haz dos querubines+ 25.18
Una clase de ángel.de oro forjado para los extremos de la cubierta de la
expiación, \bibverse{19} y pon un querubín en cada extremo. Todo esto
debe ser hecho a partir de una sola pieza de oro. \bibverse{20} Los
querubines deben ser diseñados con alas extendidas apuntando hacia
arriba, cubriendo la cubierta de expiación. Los querubines se colocarán
uno frente al otro, mirando hacia abajo, hacia la cubierta de expiación.
\bibverse{21} Pondrán la cubiertade expiación encima del Arca, y también
podrán el testimonio que les daré dentro del Arca. \bibverse{22} Me
reuniré contigo allí como está dispuesto sobre la tapa de la expiación,
entre los dos querubines que están de pie sobre el Arca del Testimonio,
y hablaré contigo sobre todas las órdenes que daré a los israelitas.
\bibverse{23} Entonces harás una mesa de madera de acacia de dos codos
de largo por un codo de ancho por un codo y medio de alto. \bibverse{24}
Cúbrela con oro puro y haz un adorno de oro para rodearla. \bibverse{25}
Haz un borde a su alrededor del ancho de una mano y pon un ribete de oro
en el borde. \bibverse{26} Haz cuatro anillos de oro para la mesa y
sujétalos a las cuatro esquinas de la mesa por las patas. \bibverse{27}
Los anillos deben estar cerca del borde para sostener los palos usados
para llevar la mesa. \bibverse{28} Haránlas varas de madera de acacia
para llevar la mesa y las cubrirán con oro. \bibverse{29} Harán platos y
fuentes para la mesa, así como jarras y tazones para verter las ofrendas
de bebida. Todos serán de oro puro. \bibverse{30} Pongan el Pan de la
Presencia sobre la mesa para que esté siempre en mi presencia.

\bibverse{31} Haz un candelabro de oro puro, modelado con martillo. Todo
debe ser hecho de una sola pieza: su base, su eje, sus copas, sus
capullos y sus flores. \bibverse{32} Debe tener seis ramas que salgan de
los lados del candelabro, tres en cada lado. \bibverse{33} Tiene tres
tazas en forma de flores de almendra en la primera rama, cada una con
capullos y pétalos, tres en la siguiente rama. Cada una de las seis
ramas que salen tendrá tres tazas en forma de flores de almendra, todas
con brotes y pétalos.

\bibverse{34} En el eje principal del candelabro se harán cuatro tazas
en forma de flores de almendra, con capullos y pétalos. \bibverse{35} En
las seis ramas que salen del candelabro, colocarás un capullo bajo el
primer par de ramas, un capullo bajo el segundo par y un capullo bajo el
tercer par. \bibverse{36} Los brotes y las ramas deben hacerse con el
candelabro como una sola pieza, modelada con martillo en oro puro.
\bibverse{37} Hagan siete lámparas y colóquenlas en el candelabro para
que iluminen el área que está delante de él. \bibverse{38} Las pinzas de
la mecha y sus bandejas deben ser de oro puro. \bibverse{39} El
candelabro y todos estos utensilios requerirán un talento de oro puro.
\bibverse{40} Asegúrate de hacer todo de acuerdo con el diseño que te
mostré en la montaña''.

\hypertarget{section-25}{%
\section{26}\label{section-25}}

\bibverse{1} Harás diez cortinas para el Tabernáculo de lino finamente
hilado, usando hilos azules, púrpura y carmesí. Háganlas bordar con
querubines por alguien que sea hábil en el bordado. \bibverse{2} Cada
cortina debe medir 28 codos de largo por 4 codos de ancho, y todas las
cortinas deben ser del mismo tamaño.

\bibverse{3} Junta cinco de las cortinas y haz lo mismo con las otras
cinco. \bibverse{4} Usa material azul para hacer lazos en el borde de la
última cortina de ambos juegos. \bibverse{5} Haz cincuenta lazos en una
cortina y cincuenta lazos en la última cortina del segundo juego,
alineando los lazos entre sí. \bibverse{6} Luego haz cincuenta ganchos
de oro y une las cortinas con los ganchos, para que el Tabernáculo sea
una sola estructura.

\bibverse{7} Haz once cortinas de pelo de cabra como una tienda de
campaña para cubrir el Tabernáculo. \bibverse{8} Cada una de las once
cortinas debe ser del mismo tamaño: 30 codos de largo por 4 codos de
ancho. \bibverse{9} Unirás cinco de las cortinas como un conjunto y las
otras seis como otro conjunto. Luego dobla la sexta cortina en dos en la
parte delantera de la tienda. \bibverse{10} Haz cincuenta lazos en el
borde de la última cortina del primer juego, y cincuenta lazos a lo
largo del borde de la última cortina del segundo juego. \bibverse{11}
Harás cincuenta ganchos de bronce y póngalos en los lazos para unir la
tienda como una sola cubierta. \bibverse{12} La media cortina extra de
esta cubierta de la tienda se dejará colgada en la parte trasera del
Tabernáculo. \bibverse{13} Las cortinas de la tienda serán un codo más
largas en cada lado, y la longitud extra colgará sobre los lados del
Tabernáculo para que quede todo cubierto. \bibverse{14} Harás una
cubierta para la tienda con pelo de cabra y pieles de carnero curtidas,
y colocarás una cubierta extra de cuero fino sobre ella.

\bibverse{15} Hagan un marco vertical de madera de acacia para el
Tabernáculo. \bibverse{16} Cada estructura debe tener diez codos de
largo por uno y medio de ancho. \bibverse{17} Cada marco tendrá dos
clavijas para que los marcos puedan ser conectados entre sí. Hagan todos
los marcos del Tabernáculo así. \bibverse{18} Haz veinte marcos para el
lado sur del Tabernáculo. \bibverse{19} Haz cuarenta soportes de plata
como apoyo para los veinte marcos usando dos soportes por marco, uno
debajo de cada clavija del marco. \bibverse{20} De manera similar para
el lado norte del Tabernáculo, harás veinte marcos \bibverse{21} y
cuarenta soportes de plata, dos soportes por marco. \bibverse{22} Harás
seis marcos para la parte trasera (lado oeste) del Tabernáculo,
\bibverse{23} junto con dos marcos para sus dos esquinas traseras.
\bibverse{24} Unirás estos marcos de las esquinas en la parte inferior y
en la parte superior cerca del primer anillo. Así es como debes hacer
los dos marcos de las esquinas. \bibverse{25} En total habrá ocho marcos
y dieciséis soportes de plata, dos debajo de cada marco.

\bibverse{26} Haz cinco barras transversales de madera de acacia para
unir los marcos del lado sur del Tabernáculo, \bibverse{27} cinco para
los del norte y cinco para los de la parte trasera del Tabernáculo, al
oeste. \bibverse{28} El travesaño central que se coloca a mitad de
camino de los marcos irá de un extremo al otro. \bibverse{29} Cubrid los
marcos con oro, y haced anillos de oro para sujetar los travesaños en su
sitio. Cubrir los travesaños con oro también. \bibverse{30} Ensambla el
Tabernáculo siguiendo el diseño que te mostré en la montaña.

\bibverse{31} Haz un velo de hilo azul, púrpura y carmesí, y de lino
finamente hilado, bordado con querubines por alguien que sea hábil en el
bordado. \bibverse{32} Con ganchos de oro, cuélgalo de cuatro postes de
madera de acacia cubiertos de oro, sostenidos por cuatro soportes de
plata. \bibverse{33} Coloca el velo bajo el gancho y pon el Arca del
Testimonio dentro, detrás del velo. El velo separará el Lugar Santo del
Lugar Santísimo.

\bibverse{34} Pon la cubierta de expiación en el Arca del Testimonio en
el Lugar Santísimo. \bibverse{35} Pon la mesa fuera del velo en el lado
norte del Tabernáculo y pon el candelabro enfrente en el lado sur.
\bibverse{36} Haz una pantalla para la entrada de la tienda usando hilos
azules, púrpura y carmesí, y lino finamente hilado y hazlo bordado.
\bibverse{37} Haz cinco postes de madera de acacia con ganchos de oro
para colgar el biombo, y funde cinco soportes de bronce para sujetarlos.

\hypertarget{section-26}{%
\section{27}\label{section-26}}

\bibverse{1} Haz un altar de madera de acacia. Debe ser cuadrado y debe
medir cinco codos de largo por cinco codos de ancho por tres codos de
alto. \bibverse{2} Haráscuernos para cada una de sus esquinas, todos de
una sola pieza con el altar, y cubrirás todo el altar con bronce.

\bibverse{3} Harás todos sus utensilios de bronce: cubos para quitar las
cenizas, palas, tazones para rociar, tenedores para la carne y
cacerolas. \bibverse{4} Hagan una rejilla de malla de bronce para él con
un anillo de bronce en cada una de sus esquinas. \bibverse{5} Coloquen
la rejilla bajo el saliente del altar, de modo que la malla llegue hasta
la mitad del altar. \bibverse{6} Haz postes de madera de acacia para el
altar y cúbrelos con bronce. \bibverse{7} Las varas deben ser colocadas
en los anillos para que las varas estén a cada lado del altar cuando sea
llevado. \bibverse{8} Hagan el altar hueco, usando tablas, tal como te
lo mostré en la montaña.

\bibverse{9} Haz un patio para el Tabernáculo. Para el lado sur del
patio haz cortinas de lino finamente hilado, de cien codos de largo por
un lado, \bibverse{10} con veinte postes y veinte soportes de bronce,
con ganchos y bandas de plata en los postes. \bibverse{11} Del mismo
modo, en el lado norte se colocarán cortinas en una disposición
idéntica. \bibverse{12} Las cortinas del lado oeste del patio tendrán
cincuenta codos de ancho, con diez postes y diez soportes. \bibverse{13}
El lado este del patio que da al amanecer tendrá 50 codos de ancho.
\bibverse{14} Las cortinas de un lado deben tener quince codos de largo,
con tres postes y tres soportes, \bibverse{15} y las cortinas del otro
lado deben ser iguales.

\bibverse{16} La entrada al patio debe tener veinte codos de ancho, con
una cortina bordada con hilos azules, púrpura y carmesí, y lino
finamente hilado, sostenida por cuatro postes y cuatro soportes.
\bibverse{17} Todos los postes alrededor del patio tendrán bandas de
plata, ganchos de plata y soportes de bronce. \bibverse{18} Todo el
patio tendrá cien codos de largo y cincuenta de ancho, con cortinas de
lino finamente hilado de cinco codos de alto, y con soportes de bronce.
\bibverse{19} Todo el resto del equipo usado en el Tabernáculo,
incluyendo las estacas de la tienda y las del patio, serán de bronce.

\bibverse{20} Debes ordenar a los israelitas que te traigan aceite de
oliva puro, prensado a mano, para las lámparas, para que puedan seguir
encendidas, dando luz. \bibverse{21} Enel Tabernáculo de Reunión, fuera
del velo delante del Testimonio, Aarón y sus hijos mantendrán las
lámparas encendidas en presencia del Señor desde la tarde hasta la
mañana. Este requisito debe ser observado por los israelitas durante
todas las generaciones.

\hypertarget{section-27}{%
\section{28}\label{section-27}}

\bibverse{1} Haz que tu hermano Aarón venga a ti, junto con sus hijos
Nadab, Abihu, Eleazar e Itamar. Ellos, de todos los israelitas, me
servirán como sacerdotes. \bibverse{2} Harás que se hagan ropas sagradas
para tu hermano Aarón para que se vea espléndido y digno. \bibverse{3}
Debes dar instrucciones a todos los obreros hábiles, a los que han
recibido de mí sus habilidades, sobre cómo hacer la ropa para la
dedicación de Aarón, para que pueda servirme como sacerdote.
\bibverse{4} Estas son las ropas que deben hacer: un pectoral, un efod,
una túnica, una túnica plisada, un turbante y una faja. Estos son los
vestidos sagrados que harán para tu hermano Aarón y sus hijos para que
puedan servirme como sacerdote. \bibverse{5} Los trabajadores usarán
hilo de oro, junto con hilo azul, púrpura y carmesí, y lino finamente
hilado.

\bibverse{6} Harán el efod de lino finamente tejido y bordado con oro, y
con hilos azules, púrpura y carmesí, hábilmente trabajado. \bibverse{7}
Dos piezas de hombro deben ser unidas a las piezas delanteras y
traseras. \bibverse{8} La cintura del efod será una pieza hecha de la
misma manera, usando hilo de oro, con hilo azul, púrpura y carmesí, y
con lino finamente tejido.

\bibverse{9} Escribe en dos piedras de ónice los nombres de las tribus
de Israel, \bibverse{10} seis nombres en una piedra, y seis en la otra,
en orden de nacimiento.\footnote{28.10 ``En orden de nacimiento'':
  Literalmente, ``según su generación''.} \bibverse{11} Escribe los
nombres en las dos piedras de la misma manera que un joyero graba un
sello personal. Luego coloque las piedras en un adorno de oro.
\bibverse{12} Ata ambas piedras a las piezas del hombro del efod como
recordatorio para las tribus israelitas. Aarón debe llevar sus nombres
en sus dos hombros para recordar a los israelitas que los representa
cuando va a la presencia del Señor. \bibverse{13} Hagan adornos de oro
\bibverse{14} y dos cadenas trenzadas de oro puro, y sujetar estas
cadenas a los adornos.

\bibverse{15} También debe hacer un pectoral para las
decisiones\footnote{28.15 ``Para las decisiones'': el pectoral debía
  sostener el Urim y el Tumim utilizados para determinar la voluntad del
  Señor y las decisiones sobre diferentes cuestiones (véase el versículo
  30).}de la misma manera hábil que el efod, para ser usado en la
determinación de la voluntad del Señor. Háganlo usando hilo de oro, con
hilo azul, púrpura y carmesí, y con lino finamente tejido. \bibverse{16}
Tiene que ser cuadrado cuando se pliega, midiendo alrededor de nueve
pulgadas+ 28.16 ``Nueve pulgadas'': Literalmente, ``un espacio,'' la
distancia entre el pulgar y el dedo meñique cuando la mano está
estirada. de largo y ancho. \bibverse{17} Adjunta un arreglo de piedras
preciosas en cuatro filas como sigue: + 28.17 Ninguna de las siguientes
piedras ha sido identificada con certeza.En la primera fila cornalina,
peridoto y esmeralda. \bibverse{18} En la segunda fila turquesa,
lapislázuli y sardónice. \bibverse{19} En la tercera fila jacinto, ágata
y amatista. \bibverse{20} En la cuarta fila topacio, berilo y jaspe.
Coloca estas piedras en los adornos de oro. \bibverse{21} Cada una de
las doce piedras se grabará como un sello personal con el nombre de una
de las doce tribus israelitas y las representará.

\bibverse{22} Haz cordones de cadenas trenzadas de oro puro para sujetar
el pectoral. \bibverse{23} Harás dos anillos de oro y sujételos a las
dos esquinas superiores del pectoral. \bibverse{24} Ata las dos cadenas
de oro a los dos anillos de oro de las esquinas del pectoral,
\bibverse{25} y luego ata los extremos opuestos de las dos cadenas a los
adornos de oro de los hombros de la parte delantera del efod.
\bibverse{26} Haz dos anillos de oro más y fíjelos a las dos esquinas
inferiores del pectoral, en el borde interior junto al efod.
\bibverse{27} Haz dos anillos de oro más y póngalos en la parte inferior
de las dos hombreras de la parte delantera del efod, cerca de donde se
une a su cintura tejida. \bibverse{28} Ata los anillos del pectoral a
los anillos del efod con un cordón de hilo azul, para que el pectoral no
se suelte del efod.

\bibverse{29} Así, cada vez que Aarón entre en el Lugar Santo, llevará
los nombres de las tribus israelitas sobre su corazón en el pectoral,
como un recordatorio constante ante el Señor. \bibverse{30} Coloca el
Urim y Tumim en el pectoral de la decisión, para que ellos también estén
sobre el corazón de Aarón siempre que venga a la presencia del Señor.
Aarón llevará continuamente los medios de decisión sobre su corazón ante
el Señor.

\bibverse{31} Haz la túnica que va con el efod exclusivamente de tela
azul, \bibverse{32} con una abertura en el medio en la parte superior.
Cose un cuello tejido alrededor de la abertura para fortalecerla y que
no se rompa.

\bibverse{33} Haz las granadas con los hilos azul, púrpura y carmesí y
pégalas alrededor de su dobladillo, con campanas de oro entre ellas,
\bibverse{34} teniendo las campanas de oro y las granadas alternadas.

\bibverse{35} Aarón debe llevar la túnica siempre que sirva, y el sonido
que haga se oirá cuando entre o salga del santuario al entrar en la
presencia del Señor, para que no muera.

\bibverse{36} Haz una placa de oro puro y grabad en ella como un sello,
``Consagradoal Señor.'' \bibverse{37} Pónganlo en la parte delantera del
turbante con un cordón azul. \bibverse{38} Aarón lo llevará en la
frente, para que se responsabilice de la culpa de las ofrendas que hagan
los israelitas, y esto se aplica a todas sus santas ofrendas. Debe
permanecer siempre en su frente para que el pueblo sea aceptado en la
presencia del Señor.

\bibverse{39} Teje la túnica con lino finamente hilado y haz el turbante
del mismo material, y también haz la faja y con bordado. \bibverse{40}
Haz túnicas, fajas y tocados para los hijos de Aarón, para que tengan un
aspecto espléndido y digno.

\bibverse{41} Haz que tu hermano Aarón y sus hijos vistan esta ropa y
luego úngelos y ordénalos. Dedícalos para que puedan servirme como
sacerdotes. \bibverse{42} Elabora calzoncillos de lino para cubrir sus
cuerpos desnudos, desde la cintura hasta el muslo. \bibverse{43} Aarón y
sus hijos deben usarlos cuando entren a el Tabernáculo de Reunión o
cuando se acerquen al altar para servir en el Lugar Santo, para que no
seanhallados culpables y mueran. Esta es una ley para Aarón y sus
descendientes para siempre.

\hypertarget{section-28}{%
\section{29}\label{section-28}}

\bibverse{1} Así es como debes proceder para dedicarlos y que me sirvan
como sacerdotes. Coge un novillo y dos carneros sin defectos.
\bibverse{2} Luego, con la mejor harina de trigo, haced lo siguiente sin
levadura: pan, pasteles mezclados con aceite de oliva y barquillos
espolvoreados con aceite de oliva. \bibverse{3} Ponlos todos en una
cesta y tráelos como ofrenda, junto con el toro y los dos carneros.

\bibverse{4} Lleva a Aarón y a sus hijos a la entrada delTabernáculo de
Reunión y lávalos con agua.\footnote{29.4 Esta era una limpieza
  ceremonial, no era como la limpieza diaria.} \bibverse{5} Toma los
vestidos y pónselos a Aarón: la túnica, el manto del efod, el efod mismo
y el pectoral. Ata el efod sobre él con su cinturón. \bibverse{6}
Envuelve el turbante en la cabeza y ata la corona sagrada al turbante.
\bibverse{7} Luego usa el aceite de la unción para ungirlo, vertiéndolo
sobre su cabeza.

\bibverse{8} Luego que vengan sus hijos y les pongan las túnicas.
\bibverse{9} Ata las fajas alrededor de Aarón y sus hijos y ponles los
tocados. El sacerdocio les pertenece para siempre.

Así es como debes ordenar a Aarón y a sus hijos. \bibverse{10} Lleva el
toro al frente del Tabernáculo de Reunión, y Aarón y sus hijos deben
poner sus manos sobre su cabeza. \bibverse{11} Luego mata el toro en
presencia del Señor a la entrada del Tabernáculo de Reunión.
\bibverse{12} Toma un poco de la sangre del toro y úntasela con el dedo
en los cuernos del altar. Luego vierte el resto de la sangre en la base
del altar. \bibverse{13} Tomen toda la grasa que cubre los intestinos,
las mejores partes\footnote{29.13 ``Mejores partes'': Se cree que se
  refiere al epiplón.}del hígado y los dos riñones con su grasa, y
quemadlos en el altar. \bibverse{14} Pero quema la carne del toro, su
piel y sus excrementos fuera del campamento, pues es una ofrenda por el
pecado.

\bibverse{15} A continuación, que Aarón y sus hijos pongan sus manos en
la cabeza de uno de los carneros. \bibverse{16} Sacrifiquen el carnero,
tomen su sangre y salpicaalrededor del altar. \bibverse{17} Corta el
carnero en pedazos, lava los intestinos y las piernas, y ponlos con los
otros pedazos y con la cabeza. \bibverse{18} Luego quema todo el carnero
en el altar. Es una ofrenda quemada al Señor para ser aceptada por él.

\bibverse{19} Entonces haz que Aarón y sus hijos coloquen sus manos
sobre la cabeza del otro carnero. \bibverse{20} Luegosacrifica el
carnero y pon un poco de su sangre en los lóbulos de las orejas derechas
de Aarón y sus hijos, en los pulgares de sus manos derechas y en los
dedos gordos de sus pies derechos. Salpica el resto de su sangre
alrededor del altar. \bibverse{21} Toma un poco de la sangre del altar y
un poco del aceite de la unción y rociadlo sobre Aarón y sus ropas, y
sobre sus hijos y sus ropas. Entonces él y sus ropas serán sagradas, así
como sus hijos y sus ropas.

\bibverse{22} Toma la grasa del carnero, incluyendo la grasa de su
amplio rabo, la grasa que cubre los intestinos, las mejores partes del
hígado, los dos riñones con su grasa, así como el muslo derecho (porque
este es un carnero para la ordenación). \bibverse{23} Toma también una
barra de pan, una torta de pan hecha con aceite de oliva y una oblea de
la cesta de pan hecho sin levadura que está en la presencia del Señor.
\bibverse{24} Dáselos todos a Aarón y a sus hijos para que los
mezan\footnote{29.24 Algunos estudiosos creen que en lugar de ``mecer''
  la ofrenda ante el Señor, la elevaban hasta él. Sin embargo, esto
  parecería ser lo mismo que lo que tradicionalmente se llama la ofrenda
  ``levantada''.}ante el Señor como ofrendamecida. \bibverse{25} Luego
toma los diferentes panes y quémalos en el altar sobre el
holocausto,para que sean agradables para el Señor.

\bibverse{26} Toma el pecho del carnero de la ordenación de Aarón y
mécelo ante el Señor como ofrenda mecida. Esta es la parte que puedes
guardar.\footnote{29.26 De aquí en adelante esta porción estaba
  reservada para los sacerdotes.} \bibverse{27} Separa para Aarón y sus
hijos el pecho de la ofrenda mecida y el muslo de la ofrenda mecida,
ambos tomados del carnero de la ordenación. \bibverse{28} De ahora en
adelante, cuando los israelitas levanten las ofrendas de paz al Señor,
estas partes pertenecerán a Aarón y a sus hijos para siempre como una
parte regular de los israelitas.

\bibverse{29} Las vestiduras sagradas que tiene Aarón serán transmitidas
a sus descendientes, para que las lleven cuando sean ungidos y
ordenados. \bibverse{30} El descendiente que le suceda como sacerdote y
entre al Tabernáculo de Reunión para servir en el Lugar Santo deberá
llevarlas durante los siete días de su ordenación.\footnote{29.30 ``De
  su ordenación'': añadido para mayor claridad.}

\bibverse{31} Toma el carnero de la ordenación y hierve su carne en un
lugar sagrado. \bibverse{32} Aarón y sus hijos comerán la carne del
carnero y el pan que está en la cesta, a la entrada del Tabernáculo de
Reunión, \bibverse{33} Comerán la carne y el pan que formaban parte de
las ofrendas que simbolizaban el perdón requerido\footnote{29.33 ``Que
  simbolizaban el perdón requerido'': añadido para mayor claridad. La
  palabra hebrea es sencillamente ``cubrir sobre,'' y se usa para
  describir perdón y reconciliación.} para su ordenación y dedicación.
Nadie más puede comerlos, porque son sagrados. \bibverse{34} Si alguna
de las carnes de la ordenación o algún pan permanece hasta la mañana
siguiente, quemen lo que sobre. No debe ser comido, porque es sagrado.

\bibverse{35} Este es el proceso que debes seguir para Aarón y sus
hijos, observando todas las instrucciones que les he dado. La ordenación
durará siete días. \bibverse{36} Cada día debes sacrificar un toro como
ofrenda para el perdón por el pecado. Al hacer esto, el altar necesita
ser purificado. Úngelo para hacerlo sagrado. \bibverse{37} Durante siete
días purificarás el altar y lo consagrarás. Entonces el altar se volverá
completamente santo, y todo lo que toque el altar se volverá santo.

\bibverse{38} Ofrecerás dos corderos de un año en el altar, diaria y
continuamente. \bibverse{39} Por la mañana ofrece un cordero y por la
tarde, antes de que oscurezca, ofrece el otro.\footnote{29.39 ``Por la
  tarde, antes de que oscurezca'': Literalmente, ``entre las noches''.}
\bibverse{40} Con el primer cordero ofrece también una décima parte de
una efa de harina de la mejor calidad, mezclada con un cuarto de hin de
aceite de oliva, y una libación de un cuarto de hin de vino.
\bibverse{41} Entonces ofrece el segundo cordero por la tarde, con las
mismas ofrendas de grano y bebida que por la mañana, un holocausto al
Señor y aceptado por él.

\bibverse{42} Estos holocaustos se harán continuamente por todas las
generaciones a la entrada del Tabernáculo de Reunión en presencia del
Señor. Allí me reuniré para hablar con ustedes. \bibverse{43} Me reuniré
con los israelitas allí, y ese lugar será sagrado por mi gloria.
\bibverse{44} De esta manera dedicaré el Tabernáculo de Reunión y el
altar, y dedicaré a Aarón y sus hijos a servirme como sacerdotes.

\bibverse{45} Entonces viviré con los israelitas y seré su Dios.
\bibverse{46} Ellos sabrán que soy el Señor su Dios, que los sacó de
Egipto, para poder vivir con ellos. Yo soy el Señor su Dios.

\hypertarget{section-29}{%
\section{30}\label{section-29}}

\bibverse{1} ''Haz un altar de madera de acacia\footnote{30.1 Esta es
  una adición al altar que se menciona en el capítulo 27.}para quemar
incienso. \bibverse{2} Será cuadrado, medirá un codo por codo, de dos
codos de alto, con cuernos en sus esquinas que son todos de una sola
pieza con el altar. \bibverse{3} Cubre su parte superior, su lado y sus
cuernos con oro puro, y hace un adorno de oro para rodearlo.
\bibverse{4} Hagan dos anillos de oro para el altar y pónganlos debajo
de la moldura, dos a ambos lados, para sostener las varas para llevarlo.
\bibverse{5} Haz las varas de madera de acacia y cúbrelas con oro.
\bibverse{6} Pon el altar delante del velo que cuelga delante del Arca
del Testimonio y la tapa de expiación que está sobre el Testimonio+ 30.6
``Testimonio'': se refiere a las tablas de piedra donde se escribieron
los Diez Mandamientos. donde me reuniré con ustedes.

\bibverse{7} Aarón debe quemar incienso fragante en el altar cada mañana
cuando cuida las lámparas. \bibverse{8} Cuando enciendas las lámparas
por la noche, se debe quemar incienso de nuevo para que hay incienso
siempre en la presencia del Señor por las generaciones futuras.
\bibverse{9} No ofrezcas en este altar ningún incienso no
aprobado,\footnote{30.9 ``Incienso no aprobado'': En otras palabras,
  incienso no preparado según las instrucciones dadas en los versículos
  34-38.} ni ningún holocausto ni ofrenda de grano, y no derrames sobre
él ninguna libación.

\bibverse{10} Una vez al año, Aarón debe realizar el ritual de expiación
poniendo en los cuernos del altar la sangre de la ofrenda por el pecado
para la expiación. Este ritual anual de expiación debe ser llevado a
cabo por las generaciones futuras. Este es el altar sagrado del Señor.''

\bibverse{11} El Señor le dijo a Moisés: \bibverse{12} ``Cuando hagas un
censo de los israelitas, cada hombre debe pagarle al Señor el rescate
por su vida cuando sea contado. Así no sufrirán la plaga cuando sean
contados. \bibverse{13} Cada uno que pase a esos condados debe dar medio
siclo, (usando el estandarte del siclo del santuario, que pesa veinte
geras). Este medio siclo es una ofrenda al Señor. \bibverse{14} Esta
ofrenda al Señor se exige a todos los que tengan veinte años o más.
\bibverse{15} Cuando ofrezcan esta ofrenda como rescatepor sus vidas,
los ricos no deben dar más de medio siclo y los pobres no deben dar
menos. \bibverse{16} Tomen este dinero pagado por los israelitas y
úsenlo para los gastos de los servicios del Tabernáculo de Reunión.
Servirá como recordatorio para que los israelitas hagan expiación por
sus vidas en presencia del Señor''.

\bibverse{17} Y el Señor le dijo a Moisés: \bibverse{18} ``Haz una
palangana de bronce con un soporte de bronce para lavar. Colócalo entre
el Tabernáculo de Reunión y el altar, y pon agua en él. \bibverse{19}
Aarón y sus hijos la usarán para lavarse las manos y los pies.
\bibverse{20} Cada vez que entren en el Tabernáculo de Reunión, se
lavarán con agua para no morir. Cuando se acerquen al altar para
presentar los holocaustos al Señor, \bibverse{21} también deben lavarse
para no morir. Este requisito debe ser observado por ellos y sus
descendientes por todas las generaciones''.

\bibverse{22} Entonces el Señor le dijo a Moisés: \bibverse{23} ``Toma
las especias de mejor calidad: 500 siclos de mirra líquida, 250 siclos
de canela de olor dulce, 250 siclos de caña aromática, \bibverse{24} 500
siclos de casia, (pesos usando el estándar del siclo del santuario), y
un hin de aceite de oliva. \bibverse{25} Mezcla todo esto en el aceite
de la unción sagrada, una mezcla aromática como el producto de un
experto perfumista. Úsalo como aceite de la unción sagrada.
\bibverse{26} Úsalo para ungir el Tabernáculo de Reunión, el Arca del
Testimonio, \bibverse{27} la mesa y todo su equipo, el candelabro y su
equipo, el altar de incienso, \bibverse{28} el altar de los holocaustos
y todos sus utensilios, y la vasija más su soporte. \bibverse{29}
Dedícalos para que sean especialmente santos. Todo lo que los toque será
sagrado.

\bibverse{30} Unjan a Aarón y a sus hijos también y dedíquenlos para que
sirvan como sacerdotes para mí. \bibverse{31} Diles a los israelitas:
``Este será mi aceite santo de unción para todas las generaciones
futuras. \bibverse{32} No lo usen en la gente común y no hagan nada
parecido usando la misma fórmula. Es santo, y debes tratarlo como si
fuera santo. \bibverse{33} Cualquiera que mezcle aceite de unción como
éste, o lo ponga sobre alguien que no sea un sacerdote,\footnote{30.33
  ``Alguien que no sea sacerdote'': Literalmente, ``un extraño''.} será
expulsado de su pueblo''.

\bibverse{34} El Señor le dijo a Moisés: ``Toma cantidades iguales de
estas especias aromáticas: resina de bálsamo, perfume, gálbano e
incienso puro. \bibverse{35} Añade un poco de sal y haz incienso puro y
santo mezclado como el producto de un experto perfumista. \bibverse{36}
Muele un poco en polvo y colóquelo delante del Arca del Testimonio en el
Terbenáculo de Reunión, donde me reuniré contigo. Será especialmente
sagrado para ti. \bibverse{37} Nopreparen ningún incienso como éste
usando la misma fórmula. Deben considerar este incienso como sagrado
para el Señor. \bibverse{38} Cualquiera que se haga un incienso como
este para su propio deleite será expulsado de su pueblo''.

\hypertarget{section-30}{%
\section{31}\label{section-30}}

\bibverse{1} El Señor le dijo a Moisés: \bibverse{2} ``He escogido por
nombre a Bezalel, hijo de Uri, hijo de Hur, de la tribu de Judá.
\bibverse{3} Lo he llenado con el Espíritu de Dios dándole habilidad,
creatividad y experiencia en todo tipo de artesanías. \bibverse{4} Puede
producir diseños en oro, plata y bronce, \bibverse{5} puede tallar
piedras preciosas para colocarlas en los marcos, y puede tallar madera.
Es un maestro de todas las artes.

\bibverse{6} También he elegido a Aholiab, hijo de Ahisamac, de la tribu
de Dan, para que le ayude. También he dado a todos los artesanos las
habilidades necesarias para hacer todo lo que te he ordenado hacer:

\bibverse{7} El Tabernáculo de Reunión, el Arca del Testimonio y su tapa
de expiación; y todos los demás muebles de la Tienda: \bibverse{8} la
mesa con su equipamiento, el candelabro de oro puro con todo su equipo,
el altar de incienso, \bibverse{9} el altar del holocausto con todos sus
utensilios, y la palangana más su soporte; \bibverse{10} así como las
ropas tejidas tanto para Aarón el sacerdote como para sus hijos para
servir como sacerdotes, \bibverse{11} así como el aceite de unción y el
incienso fragante para el Lugar Santo. Deben hacerlos siguiendo todas
las instrucciones que les he dado''.

\bibverse{12} El Señor le dijo a Moisés: \bibverse{13} ``Dile a los
israelitas, 'Es absolutamente esencial que guarden mis sábados. El
sábado será una señal entre ustedes y yo para las generaciones futuras,
para que sepan que yo soy el Señor que los santifica. \bibverse{14}
Guardarán el sábado porque es santo para ustedes. Cualquiera que lo
deshonre debe ser asesinado. Cualquiera que trabaje en ese día debe ser
cortado de su pueblo. \bibverse{15} Seis días podrán trabajar, pero el
séptimo día será un día de descanso, santo para el Señor. Cualquiera que
trabaje en el día de descanso debe ser asesinado. \bibverse{16} Los
israelitas deben guardar el sábado, observando el sábado como un acuerdo
eterno para las generaciones futuras. \bibverse{17} Es una señal entre
los israelitas y yo para siempre, porque el Señor hizo los cielos y la
tierra en seis días, pero en el séptimo día se detuvo y descansó''.

\bibverse{18} Cuando el Señor terminó de hablar con Moisés en el Monte
Sinaí, le dio las dos tablas del Testimonio, tablas de piedra escritas
por el dedo de Dios.

\hypertarget{section-31}{%
\section{32}\label{section-31}}

\bibverse{1} Cuando el pueblo se dio cuenta de cuánto tiempo tardaba
Moisés en bajar de la montaña, fueron juntos a ver a Aarón. Le dijeron:
``¡Levántate! Haznos unos dioses que nos guíen porque este hombre,
Moisés, que nos sacó de la tierra de Egipto, no sabemos qué le ha
pasado''.

\bibverse{2} ''``Tráiganme los pendientes de oro que llevan sus esposas,
hijos e hijas,'' respondió Aarón.

\bibverse{3} Así que todos se quitaron los pendientes de oro que
llevaban puestos y se los llevaron a Aarón. \bibverse{4} Él tomó lo que
le dieron y usando una herramienta moldeó un ídolo con forma de becerro.
Gritaron: ``Israel, estos son los dioses que te sacaron de la tierra de
Egipto''.

\bibverse{5} Al día siguiente, temprano, sacrificaron holocaustos y
presentaron ofrendas de paz. Luego se sentaron a celebrar con comida y
bebida. Luego se levantaron para bailar, y se convirtió en una orgía''.

\bibverse{6} Al día siguiente, temprano, sacrificaron ofrendas quemadas
y presentaron ofrendas de paz. Luego se sentaron a celebrar con comida y
bebida. Luego se levantaron para bailar, y se convirtió en una
orgía.\footnote{32.6 La palabra utilizada en este sentido, a veces
  traducida como ``juego'', no era una especie de juego de fiestas. Los
  matices sexuales están claros por su uso en el Génesis 26:8 donde se
  refiere a las ``caricias'' de la intimidad entre Isaac y su esposa
  Rebeca. Tal resultado final de un festival que incluía la indulgencia
  en la comida y la bebida era habitual en las ceremonias paganas.}

\bibverse{7} Entonces el Señor le dijo a Moisés, ``Baja, porque tu
pueblo, el que sacaste de Egipto está actuando inmoralmente.
\bibverse{8} Han abandonado rápidamente el camino que les ordené seguir.
Se han hecho un ídolo de metal con forma de becerro, inclinándose ante
él en adoración y ofreciéndole sacrificios. Dicen: ``Estos son los
dioses que los sacaron de la tierra de Egipto.'''

\bibverse{9} ``Sé cómo es este pueblo'', continuó diciendo el Señor a
Moisés. ``¡Son tan rebeldes!''\footnote{32.9 ``Rebeldes'' o
  ``perversos'': la imagen es de un caballo siendo tirado por las
  riendas en una dirección pero deliberadamente yendo en la dirección
  opuesta. Esto significa más que simplemente ser obstinado, sino que
  trata de hacer lo opuesto.} \bibverse{10} ¡Ahora déjame! Estoy
enfadado con ellos\ldots{} ¡déjame acabar con ellos! Te convertiré en
una gran nación''.

\bibverse{11} Pero Moisés suplicó al Señor su Dios, diciendo: ``¿Por qué
estás enojado con el pueblo que sacaste de la tierra de Egipto con
tremendo poder y gran fuerza? \bibverse{12} ¿Por qué permitirás que los
egipcios digan ``los sacó con el malvado propósito de matarlos en las
montañas, borrándolos de la faz de la tierra''? Apártate de tu feroz
ira. Por favor, arrepiéntete de esta amenaza contra tu pueblo.
\bibverse{13} Recuerda que juraste una promesa a tus siervos Abraham,
Isaac y Jacob,\footnote{32.13 ``Jacob'': literalmente, ``Israel.''}
diciéndoles: ``Haré que tu descendencia sea tan numerosa como las
estrellas del cielo, y te daré toda la tierra que les prometí, y la
poseerán para siempre.''

\bibverse{14} El Señor se arrepintió sobre el desastre que amenazó con
causar a su pueblo. \bibverse{15} Moisés se volvió y bajó del monte,
llevando las dos tablas de piedra de la Ley escritas a ambos lados.
\bibverse{16} Dios había hecho las tablas, y Dios mismo había grabado la
escritura.

\bibverse{17} Cuando Josué escuchó todos los gritos del campamento, le
dijo a Moisés: ``¡Suena como una pelea en el campamento!''

\bibverse{18} Pero Moisés respondió: ``Estos no son los gritos de la
victoria o de la derrota. ¡Lo que oigo es gente que está de fiesta!''
\bibverse{19} Al acercarse al campamento vio el ídolo del becerro y el
baile. Se enfadó tanto que tiró las tablas de piedra y las rompió allí
al pie de la montaña. \bibverse{20} Tomó el becerro, lo quemó y lo molió
en polvo. Luego mezcló esto con agua e hizo que los israelitas la
bebieran.

\bibverse{21} Entonces Moisés le preguntó a Aarón: ``¿Qué te hizo esta
gente para que los hicieras pecar tan mal?''

\bibverse{22} ``Por favor, no te enfades conmigo, mi señor,'' respondió
Aarón.``Tú mismo sabes cuánto mal es capaz de hacer este pueblo.
\bibverse{23} Me dijeron: ``Haznos unos dioses que nos guíen porque este
hombre, Moisés, que nos sacó de la tierra de Egipto, no sabemos qué le
ha pasado.'' \bibverse{24} Entonces les dije: ``El que tenga joyas de
oro, que se las quite y me las dé''. Eché el oro en el horno y salió
este becerro''.

\bibverse{25} Moisés vio al pueblo enloqueciendo completamente porque
Aarón lo había permitido, y que esto les había traído el ridículo de sus
enemigos. \bibverse{26} Así que fue y se paró a la entrada del
campamento, y gritó: ``¡Quien esté del lado del Señor, que venga y se
una a mí!''Y todos los levitas se reunieron a su alrededor.

\bibverse{27} Moisés les dijo: ``Esto es lo que dice el Señor, el Dios
de Israel: Cada uno amárrese su espada. Luego recorran todo el
campamento de un extremo a otro y maten a sus hermanos, amigos y
vecinos''.

\bibverse{28} Los levitas hicieron lo que Moisés les había dicho, y ese
día alrededor de 3.000 hombres fueron asesinados.

\bibverse{29} Moisés les dijo a los levitas: ``Hoy han sido dedicados al
Señor porque hanactuado contra sus hijos y hermanos. Hoy han ganado una
bendición para ustedes mismos''.

\bibverse{30} Al día siguiente Moisés habló al pueblo diciendo: ``Han
pecado muy mal. Pero ahora subiré al Señor. Tal vez pueda conseguir que
perdone su pecado''.

\bibverse{31} Así que Moisés volvió al Señor. Y dijo: ``Por favor, el
pueblo ha pecado muy mal al hacerse dioses de oro para sí mismos.
\bibverse{32} Pero ahora, si quieres, perdona sus pecados. Si no,
bórrame del pergamino en el que guardas tus registros''.

\bibverse{33} Pero el Señor respondió a Moisés: ``Los que pecaron contra
mí son los que serán borrados de mi pergamino. \bibverse{34} Ahora ve y
conduce al pueblo al lugar del que te hablé. Mi ángel irá delante de ti,
pero en el momento en que decida castigarlos, los castigaré por su
pecado''.

\bibverse{35} El Señor trajo una plaga sobre el pueblo porque hicieron
que Aarón hiciera el becerro.

\hypertarget{section-32}{%
\section{33}\label{section-32}}

\bibverse{1} Entonces el Señor le dijo a Moisés: ``Deja este lugar, tú y
el pueblo que sacaste de Egipto, y ve a la tierra que prometí con
juramento dar a Abraham, Isaac y Jacob, diciéndoles: `Daré esta tierra a
tu descendencia.' \bibverse{2} Enviaré un ángel delante de ti y
expulsaré a los cananeos, amorreos, hititas, ferezeos, heveos y
jebuseos. \bibverse{3} Entra en una tierra que fluye leche y miel, pero
no te acompañaré porque eres un pueblo rebelde. De lo contrario, te
destruiría en el camino''.

\bibverse{4} Cuando el pueblo escuchó estas palabras de crítica, se
pusieron de luto y no se pusieron sus joyas. \bibverse{5} Porque el
Señor ya le había dicho a Moisés: ``Dile al pueblo de Israel: ``Tú eres
un pueblo rebelde. Si estuviera contigo un momento, te aniquilaría.
Ahora quítate las joyas, y yo decidiré qué hacer contigo.'''
\bibverse{6} Así que los israelitas se quitaron las joyas desde que
dejaron el Monte Sinaí.\footnote{33.6 ``Monte Sinaí'': Literalmente,
  ``Monte Horeb,'' otro nombre para este mismo monte.}

\bibverse{7} Moisés solía montar el Tabernáculo de Reunión en las
afueras del campamento. Cualquiera que quisiera preguntarle algo al
Señor podía ir a el Tabernáculo de Reunión. \bibverse{8} Cada vez que
Moisés salía a la tienda, todo el pueblo iba y se paraba a la entrada de
sus tiendas. Lo observaban hasta que entraba. \bibverse{9} Tan pronto
como Moisés entraba en la tienda, la columna de nubes descendía y se
quedaba en la entrada mientras el Señor hablaba con Moisés.
\bibverse{10} Cuandoel pueblo veía la columna de nubes de pie en la
puerta de la tienda, todos se levantaban y se inclinaban en adoración a
la entrada de sus tiendas. \bibverse{11} Moisés hablaba con el Señor
cara a cara como si fuera un amigo, y luego regresaba al campamento. Sin
embargo, su joven ayudante Josué, hijo de Nun, se quedó en la Tienda.

\bibverse{12} Moisés le dijo al Señor: ``Mira, me has estado diciendo:
`Ve y dirige a estepueblo', pero no me has hecho saber a quién vas a
enviar conmigo. Y sin embargo has declarado: `Te conozco personalmente,
x y estoy feliz contigo.' \bibverse{13} Ahora bien, si es cierto que
eres feliz conmigo, por favor, enséñame tus caminos para que pueda
conocerte y seguir agradándote. Recuerda que la gente de esta nación es
tuya''.

\bibverse{14} El Señor respondió: ``Yo mismo iré contigo y te
apoyaré''.\footnote{33.14 ``Te apoyaré'': Literalmente, ``te daré
  descanso''.}

\bibverse{15} ``Si no vas con nosotros, por favor no nos saques de
aquí'', respondió Moisés. \bibverse{16} ``¿Cómo sabrán los demás que
eres feliz conmigo y con tu pueblo si no nos acompañas? ¿Cómo podría
alguien separarnos a mí y a tu pueblo de todos los demás pueblos que
viven en la tierra?''

\bibverse{17} El Señor le dijo a Moisés: ``Prometo hacer lo que me
pidas, porque soy feliz contigo y te conozco personalmente''.

\bibverse{18} ``Ahora, por favor, revélame tu gloria'', pidió Moisés.

\bibverse{19} ``Haré pasar toda la bondad de mi carácter delante de ti,
gritaré el nombre `Yahvé',\footnote{33.19 ``Yahvé'': esta es la palabra
  normalmente traducida como ``el Señor'', por lo que en los siguientes
  versos se observa que ``Yahvé'' y ``el Señor'' son lo mismo.}mostraré
gracia a los que les quiero mostrar gracia, y mostraré misericordia a
los que les quiero mostrar misericordia. \bibverse{20} Pero no podrás
ver mi rostro, porque nadie puede ver mi rostro y vivir''.

\bibverse{21} ``Ven aquí y quédate a mi lado en esta roca'', continuó el
Señor, \bibverse{22} ``y a medida que pase mi gloria te pondré en una
grieta de la roca y te cubriré con mi mano hasta que haya pasado.
\bibverse{23} Entonces quitaré mi mano y verás mi espalda; pero no verás
mi cara''.

\hypertarget{section-33}{%
\section{34}\label{section-33}}

\bibverse{1} El Señor le dijo a Moisés: ``Corta dos tablas de piedra
como las primeras, y escribiré en ellas de nuevo las mismas palabras que
estaban en las primeras tablas, las que tú rompiste. \bibverse{2}
Prepárate por la mañana, y luego sube al Monte Sinaí. Ponte delante de
mí en la cima de la montaña. \bibverse{3} Nadie más puede subir contigo.
No quiero ver a nadie en ningún lugar de la montaña, y ningún rebaño o
manada debe pastar al pie de la montaña''.

\bibverse{4} Entonces Moisés cortó dos tablas de piedra como las
anteriores y subió al monte Sinaí por la mañana temprano como el Señor
le había ordenado, llevando consigo las dos tablas de piedra.
\bibverse{5} El Señor descendió en una nube, se puso de pie con él, y
llamó el nombre ``Yahvé.''

\bibverse{6} El Señor pasó por delante de él, gritando: ``¡Yahvé! Yahvé!
¡Soy el Dios de la gracia y la misericordia! Soy lento para enojarme,
lleno de amor eterno, siempre fiel. \bibverse{7} Sigo mostrando mi amor
fiel a miles de personas, perdonando la culpa, la rebelión y el pecado.
Pero no dejaré a los culpables impunes, el impacto del pecado afectará
no sólo a los padres, sino también a sus hijos y nietos, hasta la
tercera y cuarta generación''.

\bibverse{8} Moisés se inclinó rápidamente hasta el suelo y adoró.
\bibverse{9} Dijo: ``Señor, si es verdad que eres feliz conmigo, por
favor acompáñanos. Es cierto que este es un pueblo rebelde, pero por
favor perdona nuestra culpa y nuestro pecado. Acéptanos como algo que te
pertenece especialmente''.

\bibverse{10} El Señor dijo: ``Verás que estoy haciendo un pacto
contigo. Frente a todos ustedes haré milagros que nunca se han hecho, ni
entre nadie en ningún lugar de la tierra. Todos aquí y los que están
alrededor verán al Señor trabajando, porque lo que voy a hacer por
ustedes será increíble. \bibverse{11} Pero deben seguir cuidadosamente
lo que les digo que hagan hoy. ¡Presten atención! Voy a expulsar delante
de ustedes a los amorreos, cananeos, hititas, ferezeos, heveos y
jebuseos. \bibverse{12} Asegúrense de no acordar un tratado de
paz\footnote{34.12 ``Acordar tratado de paz'': La palabra es la misma
  que ``pacto'' con Dios en el versículo 10. También ``acuerdo'' en el
  versículo 15.}con el pueblo que habite en la tierra a la que van. De
lo contrario, se convertirán en una trampa para ustedes. \bibverse{13}
Porque deben derribar sus altares, derribar sus pilares idólatras y
cortar sus postes de Asera, \bibverse{14} porque no debes adorar a
ningún otro dios que no sea el Señor. Su nombre significa exclusivo,+
34.14 ``Ser exclusivo'': Literalmente ``celoso.''Sin embargo, esto en
términos humanos se asocia con la envidia y el resentimiento. Dios es
``celoso'' al querer ser el único Dios que es adorado. porque es un Dios
que exige una relación exclusiva.

\bibverse{15} Asegúrense de no hacer un acuerdode paz con el pueblo que
habita enesa tierra, porque cuando se prostituyen adorando y
sacrificándose a sus dioses, los invitarán a unirse a ellos, y comerás
de sus sacrificios paganos. \bibverse{16} Cuando hagas que sus hijas se
casen con tus hijos y esas hijas se prostituyan con sus dioses, harán
que tus hijos adoren a sus dioses de la misma manera. \bibverse{17}
Nohagan ningún ídolo.

\bibverse{18} Guardarán el Festival de los Panes sin Levadura. Durante
siete días comerán panes sin levadura, como se los he ordenado. Lo harán
en el momento indicado en el mes de Abib, porque ese fue el mes en que
salieron de Egipto. \bibverse{19} Todo primogénito es mío. Eso incluye a
todos los primogénitos de su ganado, de sus manadas y rebaños.
\bibverse{20} Pueden redimir el primogénito de un asnoa cambio de un
cordero, pero si no lo hacen, deberán romperle el cuello. Todos tus
primogénitos deben ser redimidos. Nadie debe presentarse ante mí sin una
ofrenda. \bibverse{21} Trabajarás durante seis días, pero descansarás el
séptimo día. Incluso durante el tiempo de la siembra y la cosecha
descansarás.

\bibverse{22} Guarden el Festival de las Semanas cuando ofrezcan las
primicias de la cosecha de trigo, y el Festival de la Cosecha al final
del año agrícola. \bibverse{23} Tres veces al año todos tus varones
deben presentarse ante el Señor Yahvé, el Dios de Israel. \bibverse{24}
Expulsaré las naciones que están delante de ti y ampliaré tus fronteras,
y nadie vendrá a tomar tu tierra cuando vayas tres veces al año a
presentarte ante el Señor tu Dios. \bibverse{25} No ofrezcas pan hecho
con levadura cuando me presentes un sacrificio, ni guardes ningún
sacrificio de la fiesta de la Pascua hasta la mañana siguiente.
\bibverse{26} Cuandosiembrestus cosechas, llevalas primicias a la casa
del Señor tu Dios.

No cocines un cabrito joven en la leche de su madre.''

\bibverse{27} Entonces el Señor le dijo a Moisés: ``Escribe estas
palabras, porque son la base del acuerdo que he hecho contigo y con
Israel''.

\bibverse{28} Moisés pasó allí cuarenta días y cuarenta noches con el
Señor sin comer pan ni beber agua. Escribió en las tablas las palabras
del acuerdo, los Diez Mandamientos. \bibverse{29} Cuando Moisés bajó del
Monte Sinaí llevando las dos tablas de la Ley, no se dio cuenta de que
su rostro brillaba con fuerza porque había estado hablando con el Señor.
\bibverse{30} Cuando Aarón y los israelitas vieron a Moisés con su
rostro tan brillante que se asustaron al acercarse a él.

\bibverse{31} Pero Moisés los llamó, así que Aarón y todos los líderes
de la comunidad se acercaron a él y él habló con ellos. \bibverse{32}
Después todos los israelitas se acercaron y él les dio todas las
instrucciones del Señor que había recibido en el Monte Sinaí.
\bibverse{33} Cuando Moisés terminó de hablar con ellos, se puso un velo
sobre su rostro. \bibverse{34} Sin embargo, cada vez que Moisés entraba
a hablar con el Señor, se quitaba el velo hasta que volvía a salir.
Entonces les decía a los israelitas las instrucciones del Señor,
\bibverse{35} y los israelitas veían su rostro brillar con fuerza. Así
que se ponía el velo en la cara hasta la próxima vez que fuera a hablar
con el Señor.

\hypertarget{section-34}{%
\section{35}\label{section-34}}

\bibverse{1} Moisés convocó a todos los israelitas y les dijo: ``Esto es
lo que el Señor nos ha ordenado hacer: \bibverse{2} Seis días pueden
trabajar, pero el séptimo día debe ser un santo sábado de descanso para
el Señor. Cualquiera que haga cualquier trabajo en el día de reposo debe
ser asesinado. \bibverse{3} Noenciendan fuego en ninguna de sus casas en
el día de reposo''.

\bibverse{4} Moisés también les dijo a todos los israelitas:``Esto es lo
que el Señor ha ordenado: \bibverse{5} Recojan una ofrenda al Señor de
lo que poseen. Todo el que quiera debe traer una ofrenda al Señor: oro,
plata y bronce; \bibverse{6} hilos azules, púrpura y carmesí; lino y
pelo de cabra finamente tejidos; \bibverse{7} pieles de carnero curtidas
y cuero fino; madera de acacia; \bibverse{8} aceite de oliva para las
lámparas; especias para el aceite de la unción y para el incienso
aromático \bibverse{9} y piedras de ónice y gemas para hacer el efod y
el pectoral.

\bibverse{10} Todos tus artesanos vendrán a hacer todo lo que el Señor
ha ordenado: \bibverse{11} el Tabernáculo con su tienda y su cubierta,
sus pinzas y sus marcos, sus travesaños, postes y soportes;
\bibverse{12} el Arca con sus varas y su cubierta de expiación, y el
velo para colgarla; \bibverse{13} la mesa con sus varas, todo su equipo
y el Pan de la Presencia; \bibverse{14} el candelabro de luz con su
equipo y lámparas y aceite de oliva para alumbrar; \bibverse{15} el
altar de incienso con sus varas; el aceite de la unción y el incienso
aromático; la pantalla para la entrada del Tabernáculo y todos sus
accesorios; \bibverse{16} el altar del holocausto con su reja de bronce,
sus varas y todos sus utensilios; el lavabo más su soporte;
\bibverse{17} las cortinas del patio con sus postes y bases, y la
cortina para la entrada del patio; \bibverse{18} las estacas de la
tienda para el Tabernáculo y para el patio, así como sus cuerdas;
\bibverse{19} y las ropas tejidas para servir en el lugar santo: la ropa
sagrada para el sacerdote Aarón y para sus hijos para servir como
sacerdotes''.

\bibverse{20} Los israelitas se fueron y dejaron a Moisés. \bibverse{21}
Y todos aquellos que se sintieron movidos a hacerlo y que tenían un
espíritu dispuesto vinieron y trajeron una ofrenda al Señor por el
trabajo de hacer el Tabernáculo de Reunión, por todo lo que se requería
para sus servicios, y por las ropas sagradas.

\bibverse{22} Así que todos los que quisieron, tanto hombres como
mujeres, vinieron y presentaron su oro como ofrenda de agradecimiento al
Señor, incluyendo broches, pendientes, anillos y collares, todo tipo de
joyas de oro. \bibverse{23} Todos los que tenían hilos azules, púrpura y
carmesí, lino finamente tejido, pelo de cabra, pieles de carnero
curtidas y cuero fino, los trajeron. \bibverse{24} Los que podían
presentar una ofrenda de plata o bronce la traían como regalo al Señor.
Todos los que tenían madera de acacia para cualquier parte del trabajo,
la donaban.

\bibverse{25} Toda mujer hábil en el hilado con sus manos traía lo que
había hilado: hilo azul, púrpura o carmesí, o lino finamente tejido.
\bibverse{26} Todas las mujeres que estaban dispuestas a usar sus
habilidades hilaban el pelo de cabra. \bibverse{27} Los jefes trajeron
piedras de ónix y gemas para hacer el efod y el pectoral, \bibverse{28}
así como especias y aceite de oliva para el alumbrado, para el aceite de
la unción y para el incienso aromático. \bibverse{29} Todos los hombres
y mujeres israelitas que estaban dispuestos trajeron una ofrenda
voluntaria al Señor por todo el trabajo de hacer lo que el Señor, a
través de Moisés, les había ordenado hacer.

\bibverse{30} Entonces Moisés dijo a los israelitas: ``El Señor escogió
el nombre de Bezaleel, hijo de Uri, hijo de Hur, de la tribu de Judá.
\bibverse{31} Lo ha llenado del Espíritu de Dios dándole habilidad,
creatividad y experiencia en todo tipo de artesanía. \bibverse{32} Puede
producir diseños en oro, plata y bronce, \bibverse{33} puede tallar
piedras preciosas para colocarlas en los marcos, y puede tallar madera.
Es un maestro de todas las artesanías. \bibverse{34} El Señor también le
ha dado a él y a Aholiab, hijo de Ahisamac, de la tribu de Dan, la
habilidad de enseñar a otros. \bibverse{35} Los ha dotado de habilidad
para hacer todo tipo de trabajos como grabadores, diseñadores,
bordadores en hilo azul, púrpura y carmesí, y en lino finamente tejido,
y como tejedores, de hecho como hábiles diseñadores en todo tipo de
artesanía.

\hypertarget{section-35}{%
\section{36}\label{section-35}}

\bibverse{1} Así que Bezalel, Aholiab, y todos los demás artesanos con
la experiencia necesaria y con la habilidad y la capacidad dadas por el
Señor, deben trabajar para llevar a cabo todo el trabajo de construcción
del santuario como lo ordenó el Señor''.

\bibverse{2} Moisés convocó a Bezalel, a Aholiab y a todos los artesanos
a los que el Señor les había dado habilidades especiales, para que
vinieran a hacer el trabajo. \bibverse{3} Moisés les dio todo lo que los
israelitas habían contribuido para llevar a cabo el trabajo de
construcción del santuario. Mientras tanto el pueblo siguió trayendo
ofrendas voluntarias cada mañana, \bibverse{4} tanto que todos los
artesanos que trabajaban en el santuario dejaron lo que estaban haciendo
\bibverse{5} y fueron a decirle a Moisés: ``El pueblo ya ha traído lo
suficiente para completar el trabajo que el Señor nos ha ordenado
hacer''.

\bibverse{6} Moisés dio la orden, y se hizo un anuncio en todo el
campamento: ``Hombres y mujeres, no traigan nada más como ofrenda para
el santuario''. Así que se impidió que el pueblo trajera nada más,
\bibverse{7} puesto que ya había más que suficiente para hacer todo el
trabajo necesario.

\bibverse{8} Los hábiles artesanos entre los trabajadores hicieron las
diez cortinas para el Tabernáculo. Estaban hechas de lino finamente
hilado junto con hilos azules, púrpura y carmesí, bordadas con
querubines. \bibverse{9} Cada cortina tenía 28 codos de largo por 4
codos de ancho, y todas eran del mismo tamaño. \bibverse{10} Unieron
cinco de las cortinas como un conjunto, y las otras cinco las unió como
un segundo conjunto. \bibverse{11} Utilizaron material azul para hacer
lazos en el borde de la última cortina de ambos juegos. \bibverse{12}
Hicieron cincuenta lazos en una cortina y cincuenta lazos en la última
cortina del segundo juego, alineando los lazos entre sí. \bibverse{13}
También hicieron cincuenta ganchos de oro y unieron las cortinas con los
ganchos, de modo que el Tabernáculo era una sola estructura.

\bibverse{14} Hicieron once cortinas de pelo de cabra como una tienda de
campaña para cubrir el Tabernáculo. \bibverse{15} Cada una de las once
cortinas era del mismo tamaño, 30 codos de largo por 4 codos de ancho.
\bibverse{16} Unieron cinco de las cortinas como un conjunto y las otras
seis como otro conjunto. \bibverse{17} Confeccionaron cincuenta lazos en
el borde de la última cortina del primer juego, y cincuenta lazos a lo
largo del borde de la última cortina del segundo juego. \bibverse{18}
Hicieron cincuenta ganchos de bronce para unir la tienda como una sola
cubierta. \bibverse{19} Elaboraron una cubierta para la tienda con pelo
de cabra y pieles de carnero curtidas, y colocaron una cubierta extra de
cuero fino sobre ella.

\bibverse{20} Hicieron un marco vertical de madera de acacia para el
Tabernáculo. \bibverse{21} Cada marco tenía diez codos de largo por un
codo y medio de ancho. \bibverse{22} Cada marco tenía dos clavijas para
que los marcos pudieran conectarse entre sí. Hicieron todos los marcos
del Tabernáculo así. \bibverse{23} Hicieron veinte marcos para el lado
sur del Tabernáculo. \bibverse{24} Hicieron cuarenta soportes de plata
como apoyo para los veinte marcos usando dos soportes por marco, uno
debajo de cada clavija del marco. \bibverse{25} De manera similar para
el lado norte del Tabernáculo, hicieron veinte marcos \bibverse{26} y
cuarenta soportes de plata, dos soportes por marco. \bibverse{27}
Hicieron seis marcos para la parte trasera (lado oeste) del Tabernáculo,
\bibverse{28} junto con dos marcos para sus dos esquinas traseras.
\bibverse{29} Unieron estos marcos de las esquinas en la parte inferior
y en la parte superior cerca del primer anillo. Así es como hicieron los
dos marcos angulares. \bibverse{30} En total había ocho marcos y
dieciséis soportes de plata, dos debajo de cada marco.

\bibverse{31} Fabicaron cinco barras transversales de madera de acacia
para sostener los marcos en el lado sur del Tabernáculo, \bibverse{32}
cinco para los del norte y cinco para los de la parte trasera del
Tabernáculo, al oeste. \bibverse{33} Hicieron el travesaño central que
se colocó a la mitad de los marcos y corrió de un extremo al otro.
\bibverse{34} Cubrieron los marcos con oro, e hicieron anillos de oro
para sostener las barras transversales en su lugar. También cubrieron
los travesaños con oro.

\bibverse{35} Confeccionaron un velo de hilo azul, púrpura y carmesí, y
de lino finamente hilado, bordado con querubines por alguien que era
hábil en este arte. \bibverse{36} Fabricaron cuatro postes de madera de
acacia para ello y los cubrieron con oro. Hicieron ganchos de oro para
los postes y fundieron sus cuatro soportes de plata. \bibverse{37}
Hicieron un biombo para la entrada de la tienda usando hilos azules,
púrpura y carmesí, y lino finamente hilado, y lo hicieron bordar.
\bibverse{38} También hicieron cinco postes de madera de acacia con
ganchos para colgar el biombo. Cubrieron la parte superior de los postes
y sus bandas con oro, y sus cinco soportes eran de bronce.

\hypertarget{section-36}{%
\section{37}\label{section-36}}

\bibverse{1} Bezalel hizo el Arca de madera de acacia que mide dos codos
y medio de largo por un codo y medio de ancho por un codo y medio de
alto. \bibverse{2} La cubrió con oro puro por dentro y por fuera, e hizo
un adorno de oro para rodearla. \bibverse{3} Fundió cuatro anillos de
oro y los unió a sus cuatro pies, dos en un lado y dos en el otro.
\bibverse{4} Hizo palos de madera de acacia y los cubrió con oro.
\bibverse{5} Colocó las varas en los anillos de los lados del Arca, para
que pudiera ser transportada.

\bibverse{6} Hizo la tapa de expiación de oro puro, de dos codos y medio
de largo por un codo y medio de ancho. \bibverse{7} Hizo dos querubines
de oro martillado para los extremos de la tapa de expiación,
\bibverse{8} y puso un querubín en cada extremo. Todo esto fue hecho de
una sola pieza de oro. \bibverse{9} Los querubines fueron diseñados con
alas extendidas apuntando hacia arriba, cubriendo la cubierta de
expiación. Los querubines se colocaron uno frente al otro, mirando hacia
la cubierta de expiación.

\bibverse{10} Luego hizo la mesa de madera de acacia de dos codos de
largo por un codo de ancho por un codo y medio de alto. \bibverse{11} La
cubrió con oro puro e hizo un adorno de oro para rodearla. \bibverse{12}
Hizo un borde a su alrededor del ancho de una mano y puso un adorno de
oro en el borde. \bibverse{13} Fundió cuatro anillos de oro para la mesa
y los sujetó a las cuatro esquinas de la mesa por las patas.
\bibverse{14} Los anillos estaban cerca del borde para sujetar los palos
usados para llevar la mesa. \bibverse{15} Fabricó las varas de madera de
acacia para llevar la mesa y las cubrió con oro. \bibverse{16} Elaboró
utensilios para la mesa de oro puro: platos y fuentes, tazones y jarras
para verter las ofrendas de bebida.

\bibverse{17} Hizo el candelabro de oro puro, martillado. Todo el
conjunto estaba hecho de una sola pieza: su base, el fuste, las tazas,
los capullos y las flores. \bibverse{18} Tenía seis ramas que salían de
los lados del candelabro, tres en cada lado. Tenía tres tazas en forma
de flores de almendra en la primera rama, cada una con brotes y pétalos,
tres en la siguiente rama. \bibverse{19} Cada una de las seis ramas que
salían tenía tres copas en forma de flores de almendra, todas con brotes
y pétalos.

\bibverse{20} En el eje principal del candelabro hizo cuatro tazas en
forma de flores de almendra, con capullos y pétalos. \bibverse{21} En
las seis ramas que salían de él, colocó un brote bajo el primer par de
ramas, un brote bajo el segundo par, y un brote bajo el tercer par.
\bibverse{22} Los brotes y las ramas deben ser hechos con el candelabro
como una sola pieza, martillado en oro puro. \bibverse{23} Hizo siete
lámparas, así como pinzas de mecha y sus bandejas de oro puro.
\bibverse{24} El candelabro y todos estos utensilios requerían un
talento de oro puro.

\bibverse{25} Hizo el altar para quemar incienso de madera de acacia.
Era cuadrado, medía un codo por codo, por dos codo de alto, con cuernos
en sus esquinas que eran todos de una sola pieza con el altar.
\bibverse{26} Cubrió su parte superior, su costado y sus cuernos con oro
puro, e hizo un adorno de oro para rodearlo. \bibverse{27} Hizo dos
anillos de oro para el altar y los colocó debajo del adorno, dos a ambos
lados, para sostener los palos para llevarlo. \bibverse{28} Hizo las
varas de madera de acacia y las cubrió con oro. \bibverse{29} Hizo el
aceite de la santa unción y el incienso puro y aromático como el
producto de un experto perfumista.

\hypertarget{section-37}{%
\section{38}\label{section-37}}

\bibverse{1} Bezalelpresentó la ofrenda quemada en el altar hecho con
madera de acacia. Era cuadrado y medía cinco codos de largo por cinco de
ancho por tres de alto. \bibverse{2} Hizo cuernos para cada una de sus
esquinas, todos de una sola pieza con el altar, y cubrió todo el altar
con bronce.

\bibverse{3} Elaboró todos sus utensilios: cubos para quitar las
cenizas, palas, tazones para rociar, tenedores para la carne y
cacerolas. Todos sus utensilios los hizo de bronce. \bibverse{4}
Fabricóuna rejilla de malla de bronce para el altar y la colocó bajo el
saliente del altar, de modo que la malla llegara hasta la mitad del
altar. \bibverse{5} Fundió cuatro anillos de bronce para las cuatro
esquinas de la rejilla como soportes para los postes. \bibverse{6}
Elaboró postes de madera de acacia para el altar y los cubrió con
bronce. \bibverse{7} Puso las varas a través de los anillos a cada lado
del altar para que pudiera ser transportado. Hizo el altar hueco, usando
tablas.

\bibverse{8} Hizo la palangana de bronce con su soporte con bronce de
los espejos de las mujeres que servían en la entrada de la Carpa del
Encuentro.

\bibverse{9} Luegoconstruyó un patio. Para el lado sur del patio hizo
cortinas de lino finamente hilado, de cien codos de largo por un lado,
\bibverse{10} con veinte postes y veinte soportes de bronce, con ganchos
y bandas de plata en los postes. \bibverse{11} De manera similar hizo
cortinas colocadas en el lado norte en una disposición idéntica.
\bibverse{12} Confeccionó cortinas para el lado oeste del patio de
cincuenta codos de ancho, con diez postes y diez soportes. \bibverse{13}
El lado este del patio que da al amanecer tenía cincuenta codos de
ancho. \bibverse{14} Diseñó las cortinas de un lado de quince codos de
largo, con tres postes y tres soportes, \bibverse{15} y las cortinas del
otro lado de la misma manera. \bibverse{16} Todas las cortinas que
rodeaban el patio eran de lino finamente tejido. \bibverse{17} Las
gradas de los postes eran de bronce, los ganchos y las bandas eran de
plata, y la parte superior de los postes estaba cubierta de plata. Todos
los postes alrededor del patio tenían bandas de plata.

\bibverse{18} La cortina de la entrada al patio estaba bordada con hilos
azules, púrpura y carmesí, y con lino finamente hilado. Tenía 20 codos
de largo por 5 codos de alto, la misma altura que las cortinas del
patio. \bibverse{19} Estaba sostenido por cuatro postes y cuatro
soportes. Los postes tenían ganchos, tapas y bandas de plata.
\bibverse{20} Todas las estacas de la tienda para el Tabernáculo y para
el patio circundante eran de bronce.

\bibverse{21} Lo siguiente es lo que se usó para el Tabernáculo del
Testimonio, registrado bajo la dirección de Moisés por los levitas bajo
la supervisión de Itamar, hijo del sacerdote Aarón. \bibverse{22}
Bezaleel, hijo de Uri, hijo de Hur, de la tribu de Judá, hizo todo lo
que el Señor había ordenado a Moisés. \bibverse{23} Fue asistido por
Aholiab, hijo de Ahisamac, de la tribu de Dan, un grabador, diseñador y
bordador que usaba hilos azules, púrpura y carmesí y lino finamente
tejido.

\bibverse{24} La cantidad total de oro de la ofrenda que se utilizó para
el trabajo en el santuario fue de 29 talentos y 730 siclos, (usando el
estándar de siclos del santuario).

\bibverse{25} La cantidad total de plata de los que habían sido contados
en el censo era de 100 talentos y 1.775 siclos (usando el estándar del
siclo del santuario). \bibverse{26} Esto representa un beka por persona,
o medio siclo, (usando el estándar del siclo del santuario) de cada
persona de veinte años o más que había sido censada, un total de 603.550
hombres. \bibverse{27} Los cien talentos de plata se usaron para fundir
los soportes del santuario y los soportes de las cortinas, 100 bases de
los 100 talentos, o un talento por base. \bibverse{28} Bezalel usó los
1.775 siclos de plata para hacer los ganchos de los postes, cubrir sus
tapas y hacer bandas para ellos.

\bibverse{29} La cantidad total de bronce de la ofrenda fue de 70
talentos y 2.400 siclos. \bibverse{30} Bezalel lo usó para hacer las
gradas para la entrada a el Tabernáculo de Reunión, el altar de bronce y
su rejilla de bronce, todos los utensilios para el altar, \bibverse{31}
las gradas para el patio y su entrada, y todas las estacas de la tienda
para el Tabernáculo y el patio.

\hypertarget{section-38}{%
\section{39}\label{section-38}}

\bibverse{1} Estoshombres\footnote{39.1 Refiriéndose a los artesanos.}confeccionaron
ropa tejida con hilos azules, púrpura y carmesí para servir en el
santuario. También hicieron vestimentas sagradas para Aarón, como el
Señor le había ordenado a Moisés. \bibverse{2} Hicieron el efod de lino
finamente tejido bordado con oro, y con hilos azules, púrpura y carmesí.
\bibverse{3} Martillaron finas láminas de oro y cortaron hilos para
tejerlos con los hilos azul, púrpura y escarlata, junto con lino fino,
todo hábilmente trabajado. \bibverse{4} Dos piezas de hombro fueron
unidas a las piezas delanteras y traseras \bibverse{5} La cintura del
efod era una pieza hecha de la misma manera, usando hilo de oro, con
hilo azul, púrpura y carmesí, y con lino fino, como el Señor había
ordenado a Moisés. \bibverse{6} Colocaron las piedras de ónix en
engastes de oro ornamental, grabando los nombres de las tribus
israelitas de la misma manera que un joyero graba un sello personal.
\bibverse{7} Pusieron ambas piedras en los hombros del efod como
recordatorio para las tribus israelitas, como el Señor le había ordenado
a Moisés.

\bibverse{8} También hicieron un pectoral para las decisiones de la
misma manera hábil que el efod, para ser usado en la determinación de la
voluntad del Señor. Lo hicieron usando hilo de oro, con hilos azules,
púrpura y carmesí, y con lino finamente tejido. \bibverse{9} Era
cuadrado cuando se doblaba, midiendo alrededor de nueve
pulgadas\footnote{39.9 ``Nueve pulgadas'': Literalmente, ``un espacio,''
  la distancia entre el pulgar y el dedo meñique cuando la mano está
  estirada.} de largo y ancho. \bibverse{10} Adjuntaron un arreglo de
piedras preciosas en cuatro filas como sigue.+ 39.10 Ninguna de las
siguientes piedras ha sido identificada con certeza. En la primera fila
cornalina, peridoto y esmeralda. \bibverse{11} En la segunda fila
turquesa, lapislázuli y sardónice. \bibverse{12} En la tercera fila
jacinto, ágata y amatista. \bibverse{13} En la cuarta fila topacio,
berilo y jaspe. Todos ellos fueron colocados en un marco de oro
ornamental. \bibverse{14} Cada una de las doce piedras estaba grabada
como un sello personal con el nombre de una de las doce tribus
israelitas y las representaba.

\bibverse{15} Confeccionaron cordones de cadenas trenzadas de oro puro
para sujetar el pectoral. \bibverse{16} Hicieron dos ajustes de oro y
dos anillos de oro y sujetaron los anillos a las dos esquinas superiores
del pectoral. \bibverse{17} Fijaron las dos cadenas de oro a los dos
anillos de oro de las esquinas del pectoral, \bibverse{18} y luego
sujetaron los extremos opuestos de las dos cadenas a los adornos de oro
de los hombros de la parte delantera del efod. \bibverse{19} Hicieron
dos anillos de oro más y los fijaron a las dos esquinas inferiores del
pectoral, en el borde interior junto al efod. \bibverse{20} Hicieron dos
anillos de oro más y los fijaron en la parte inferior de las dos
hombreras de la parte delantera del efod, cerca de donde se une a su
cintura tejida. \bibverse{21} Ataron los anillos del pectoral a los
anillos del efod con un cordón de hilo azul, para que el pectoral no se
soltara del efod, como el Señor había ordenado a Moisés.

\bibverse{22} Hicieron la túnica que acompaña al efod exclusivamente de
tela azul tejida, \bibverse{23} con una abertura en el centro en la
parte superior. Cosieron un cuello tejido alrededor de la abertura para
reforzarla y que no se rompiera. They stitched a woven collar around the
opening to strengthen it so it wouldn't tear. \bibverse{24} Hicieron
granadas usando hilos azules, púrpura y carmesí y lino finamente tejido
y las unieron alrededor de su dobladillo. \bibverse{25} Hicieron
campanas de oro puro y las unieron entre las granadas alrededor de su
dobladillo, \bibverse{26} haciendo que las campanas y las granadas se
alternaran. La túnica debía ser usada para el servicio sacerdotal, como
el Señor le había ordenado a Moisés.

\bibverse{27} Confeccionaron túnicas con lino finamente hilado hechas
por un tejedor para Aarón y sus hijos. \bibverse{28} Tambiénelaboraron
turbantes, tocados y diademas de lino fino, y calzoncillos de lino
finamente tejidos, \bibverse{29} así como fajas de lino finamente
tejidas bordadas con hilos azules, púrpura y carmesí, como el Señor
había ordenado a Moisés.

\bibverse{30} Diseñaron la placa de la corona santa de oro puro y
escribieron en ella, grabada como un sello, ``Consagrado al Señor''.
\bibverse{31} Le ataron un cordón azul para atarlo a la parte delantera
del turbante, como el Señor le había ordenado a Moisés.

\bibverse{32} Así que todo el trabajo para el Tabernáculo,
elTarbernáculo de Reunión, estaba terminado. Los israelitas hicieron
todo lo que el Señor le había ordenado a Moisés. \bibverse{33} Luego
presentaron el Tabernáculo a Moisés: la tienda con todos sus muebles,
sus pinzas, sus marcos, sus travesaños y sus postes y soportes;
\bibverse{34} la cubierta de pieles de carnero curtidas, la cubierta de
cuero fino y el velo; \bibverse{35} el Arca del Testimonio con sus varas
y la cubierta de expiación; \bibverse{36} la mesa con todos sus equipos
y el Pan de la Presencia; \bibverse{37} el candelabro de oro puro con
sus lámparas puestas en fila, y todos sus equipos, así como el aceite de
oliva para las lámparas; \bibverse{38} el altar de oro, el aceite de la
unción, el incienso aromático y el biombo para la entrada de la tienda;
\bibverse{39} el altar de bronce con su reja de bronce, sus postes y
todos sus utensilios; la palangana más su soporte; \bibverse{40} las
cortinas del patio y sus postes y soportes; la cortina para la entrada
del patio, sus cuerdas y estacas de la tienda, y todo el equipo para los
servicios del Tabernáculo, el Tabernáculo de Reunión; \bibverse{41} y
las vestimentas tejidas para servir en el santuario, las ropas sagradas
para el sacerdote Aarón y para sus hijos para servir como sacerdotes.

\bibverse{42} Los israelitas hicieron todo el trabajo que el Señor había
ordenado a Moisés. \bibverse{43} Moisés inspeccionó todo el trabajo y se
aseguró de que lo habían hecho como el Señor se los había indicado.
Entonces Moisés los bendijo.

\hypertarget{section-39}{%
\section{40}\label{section-39}}

\bibverse{1} El Señor le dijo a Moisés: \bibverse{2} ``Levanta el
Tabernáculo de Reunión, el primer día del primer mes del año.
\bibverse{3} Coloca el Arca del Testimonio dentro de ella. Asegúrate de
que el Arca esté detrás del velo. \bibverse{4} Trae la mesa y pon sobre
ella lo que sea necesario. Trae también el candelabro y coloca sus
lámparas. \bibverse{5} Pon el altar de oro del incienso delante del Arca
del Testimonio, y pon el velo a la entrada del Tabernáculo. \bibverse{6}
Coloca el altar de los holocaustos frente a la entrada del Tabernáculo,
el Tabernáculo de Reunión. \bibverse{7} Coloca la palangana entre el
Tabernáculo de Reunión y el altar, y pon agua en ella.

\bibverse{8} Prepara el patio que lo rodea y pon la cortina para la
entrada del patio.

\bibverse{9} Usa el aceite de la unción para ungir el Tabernáculo y todo
lo que hay en él. Dedícalo y todos sus muebles para hacerlo sagrado.
\bibverse{10} Unge el altar de los holocaustos y todos sus utensilios.
Dedica el altar y será especialmente santo. \bibverse{11} Ungirás y
dedicarás la pila con su soporte.

\bibverse{12} Lleva a Aarón y a sus hijos a la entrada del Tabernáculo
de Reunión y lávalos allí con agua. \bibverse{13} Luegovistea Aarón con
los vestidos sagrados, úngelo y dedícalo, para que me sirva de
sacerdote. \bibverse{14} Que sus hijos se acerquen y los vistan con
túnicas. \bibverse{15} Úngelos de la misma manera que ungiste a su
padre, para que también me sirvan como sacerdotes. Su unción hará que su
linaje de sacerdotes sea eterno, para las generaciones futuras''.

\bibverse{16} Moisés llevó a cabo todas las instrucciones del Señor.
\bibverse{17} El tabernáculo se levantó el primer día del primer mes del
segundo año.\footnote{40.17 En otras palabras, hacía un año que habían
  salido de Egipto.} \bibverse{18} Cuando Moisés levantó el
tabernáculo,+ 40.18 Es evidente que Moisés no hizo todo este trabajo por
sí mismo, sino que lo supervisaba. colocó sus soportes, fijó sus marcos,
conectó sus travesaños y erigió sus postes. \bibverse{19} Luego extendió
la tienda sobre el tabernáculo y colocó la cubierta sobre la tienda,
como el Señor le había ordenado.

\bibverse{20} Moisés tomó el testimonio\footnote{40.20 Las dos tablas
  inscritas con los Diez Mandamientos.}y lo puso en el arca. Ató los
postes al Arca, y colocó la tapa de expiación en la parte superior del
Arca. \bibverse{21} Luego llevó el Arca al Tabernáculo. Levantó el velo
y se aseguró de que el Arca del Testimonio estuviera detrás de ella,
como el Señor le había ordenado.

\bibverse{22} Moisés colocó la mesa dentro delTabernáculo de Reunión en
el lado norte del Tabernáculo, fuera del velo. \bibverse{23} Puso el pan
sobre ella en presencia del Señor, como el Señor le había ordenado.
\bibverse{24} Colocó el candelabro en la tienda de la Reunión, frente a
la mesa, en el lado sur del Tabernáculo \bibverse{25} y levantó las
lámparas en presencia del Señor, como el Señor le había ordenado.

\bibverse{26} Moisés levantó el altar de oro en el Tabernáculo de
Reunión, frente al velo, \bibverse{27} y quemó incienso aromático en él,
como el Señor le había ordenado. \bibverse{28} Luego levantó el velo a
la entrada del Tabernáculo. \bibverse{29} Levantó el altar del
holocausto cerca de la entrada del Tabernáculo de Reunión, y presentó el
holocausto y la ofrenda de grano, como el Señor le había ordenado.
\bibverse{30} Puso la palangana entre el Tabernáculo de Reunión y el
altar y puso agua para lavar. \bibverse{31} Moisés, Aarón y sus hijos la
usaron para lavarse las manos y los pies. \bibverse{32} Se lavaban cada
vez que entraban en el Tabernáculo de Reunión o se acercaban al altar,
como el Señor le había ordenado a Moisés.

\bibverse{33} Moisés levantó el patio alrededor del Tabernáculo y del
altar, y puso la cortina para la entrada del patio. Esto marcó el final
del trabajo hecho por Moisés.

\bibverse{34} Entonces la nube cubrió la Tienda de la Reunión, y la
gloria del Señor llenó el Tabernáculo. \bibverse{35} Moisés no pudo
entrar en el Tabernáculo de Reunión porque la nube permaneció sobre
ella, y la gloria del Señor llenó el Tabernáculo. \bibverse{36} Cada vez
que la nube se levantaba del Tabernáculo, los israelitas se ponían en
marcha de nuevo en su viaje. \bibverse{37} Si la nube no se levantaba,
no se ponían en marcha hasta que la nube se levantara. \bibverse{38} La
nube del Señor permanecía sobre el Tabernáculo durante el día, y el
fuego ardía dentro de la nube durante la noche, de modo que podía ser
visto por todos los israelitas dondequiera que viajaran.
