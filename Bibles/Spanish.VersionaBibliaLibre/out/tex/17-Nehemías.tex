\hypertarget{section}{%
\section{1}\label{section}}

\bibverse{1} Este es el relato de Nehemías, hijo de Hacalías. En el mes
de Quisleu, en el vigésimo año del reinado de Artajerjes, yo estaba en
la fortaleza de Susa. \bibverse{2} Hanani, uno de mis hermanos, vino de
Judá con otros hombres. Les pregunté sobre el remanente de los exiliados
judíos que habían regresado del cautiverio, y también sobre Jerusalén.

\bibverse{3} Me dijeron: ``El remanente que quedó del exilio está allí
en la provincia, pero tiene muchos problemas y se siente humillado. Las
murallas de Jerusalén han sido derribadas y sus puertas incendiadas''.

\bibverse{4} Cuando me enteré de la noticia, me senté, llorando y
lamentándome durante días, ayunando y orando al Dios del cielo.

\bibverse{5} Entonces oré: ``Por favor, Señor Dios del cielo -- el Dios
grande y asombroso que mantiene su acuerdo de amor confiable con los que
lo aman y guardan sus mandamientos -- \bibverse{6} por favor escucha y
enfoca tu atención en la oración de tu siervo que te estoy orando ahora,
día y noche, en nombre de tus siervos, los israelitas. Confieso los
pecados que los israelitas hemos cometido contra ti, incluidos los míos
y los de mi familia. \bibverse{7} Hemos hecho cosas terribles para
ofenderte y no hemos cumplido los mandamientos, las leyes y los
reglamentos que le diste a tu siervo Moisés.

\bibverse{8} Por favor, recuerda lo que le dijiste a Moisés cuando le
dijiste: 'Si son infieles, los dispersaré entre las naciones,
\bibverse{9} pero si vuelven a mí y siguen mis mandamientos y los
obedecen, entonces, aunque sean exiliados hasta los confines de la
tierra, los reuniré y los llevaré al lugar que he elegido donde seré
honrado. \bibverse{10} Ellos son tus siervos y nuestro pueblo. Los has
salvado con tu gran poder y tu increíble fuerza. \bibverse{11} Señor,
por favor responde a mi oración y a las oraciones de los que aman
adorarte. Por favor, permíteme tener éxito hoy y haz que el rey
simpatice conmigo''. Yo era el copero del rey.

\hypertarget{section-1}{%
\section{2}\label{section-1}}

\bibverse{1} En el mes de Nisán, en el vigésimo año del reinado de
Artajerjes, cuando le trajeron el vino, lo cogí y se lo di al rey. Nunca
antes me había presentado ante él con aspecto triste, \bibverse{2} por
lo que el rey me preguntó: ``¿Por qué pareces tan triste, aunque no
pareces enfermo? Debes de estar muy disgustado''. Yo estaba
absolutamente aterrado, \bibverse{3} pero le respondí al rey: ``¡Viva el
rey! ¿Cómo puedo evitar estar triste? La ciudad donde están enterrados
mis antepasados está en ruinas, y sus puertas han sido incendiadas''.

\bibverse{4} ``¿Y qué quieres?'', me preguntó el rey.

Oré al Dios del cielo,\footnote{\textbf{2:4} Claramente una
  oraciónsilenciosa.}y le respondí al rey: \bibverse{5} ``Si le agrada a
Su Majestad, y si está contento conmigo, le pido que me envíe a Judá, a
la ciudad donde están enterrados mis antepasados, para que pueda
reconstruirla''.

\bibverse{6} El rey, con la reina sentada a su lado, me preguntó:
``¿Cuánto tiempo durará tu viaje y cuándo volverás?'' El rey aceptó
enviarme, y le dije cuánto tiempo estaría fuera.

\bibverse{7} También le pedí: ``Si le parece bien a Su Majestad, que se
me proporcionen cartas para entregar a los gobernadores al oeste del
Éufrates, para que me permitan pasar con seguridad hasta que llegue a
Judá. \bibverse{8} Que se me proporcione también una carta para Asaf,
guardián del bosque del rey, a fin de que me dé madera para hacer vigas
para las puertas de la fortaleza del Templo, para las murallas de la
ciudad y para la casa en que viviré.'' Como mi Dios bondadoso estaba
sobre mí, el rey me dio lo que le pedí.

\bibverse{9} Luego fui a los gobernadores de la provincia al oeste del
Éufrates y les entregué las cartas del rey. El rey también envió conmigo
una escolta militar de caballería. \bibverse{10} Pero cuando Sanbalat,
el horonita, y Tobías, el amonita, se enteraron de esto, se molestaron.
Para ellos esto era un desastre total: que alguien había llegado para
ayudar a los israelitas.

\bibverse{11} Llegué a Jerusalén y descansé durante tres días.
\bibverse{12} Luego me levanté durante la noche y salí con unos pocos
hombres. No le expliqué a nadie lo que mi Dios había puesto en mi mente
para hacer por Jerusalén. Sólo tomé un caballo para montar.\footnote{\textbf{2:12}
  En otras palabras, estaba manteniendo su misión lo más silenciosa
  posible y minimizando cualquier ruido de parte de su grupo.}
\bibverse{13} Así que cabalgué en la oscuridad a través de la Puerta del
Valle hacia el Manantial de la Serpiente y la Puerta del Desecho, e
inspeccioné los muros de Jerusalén que habían sido derribados y las
puertas que habían sido quemadas. \bibverse{14} Luego continué hacia la
Puerta de la Fuente y el Estanque del Rey, pero no pudimos pasar porque
no había suficiente espacio para hacerlo. \bibverse{15} Así que subí por
el valle en la oscuridad e inspeccioné la muralla. Luego regresé,
pasando de nuevo por la Puerta del Valle.

\bibverse{16} Los responsables de la ciudad no tenían idea de dónde
había ido ni de lo que estaba haciendo, porque todavía no les había
contado a los judíos, a los sacerdotes, a los nobles, a los funcionarios
ni a ningún otro sobre los planes de construcción.\footnote{\textbf{2:16}
  ``Planes de construcción'': literalmente, ``los trabajadores.''}

\bibverse{17} Entonces les dije: ``¡Miren el problema que tenemos!
Jerusalén es un montón de escombros, y sus puertas han sido quemadas.
Vamos, reconstruyamos la muralla de Jerusalén, para que ya no pasemos
tanta vergüenza''. \bibverse{18} Entonces les expliqué lo bueno que
había sido Dios conmigo y lo que me había dicho el rey.

``Pongámonos a reconstruir'', respondieron, y se pusieron a trabajar con
entusiasmo.\footnote{\textbf{2:18} ``Se pusieron a trabajar con
  entusiasmo'': literalmente, ``fortalecieron sus manos para el bien.''}

\bibverse{19} Pero cuando Sanbalat el horonita, Tobías el funcionario
amonita y Gesem el árabe se enteraron, se burlaron y se mofaron de
nosotros, preguntando: ``¿Qué traman? ¿Se están rebelando contra el
rey?''

\bibverse{20} Pero yo respondí, diciéndoles: ``El Dios del cielo se
encargará de que tengamos éxito. Nosotros, sus siervos, comenzaremos a
reconstruir, pero Jerusalén no les pertenece, y ustedes no tienen
autoridad ni derecho sobre ella.''

\hypertarget{section-2}{%
\section{3}\label{section-2}}

\bibverse{1} Eliasib, el sumo sacerdote, y otros sacerdotes con él,
comenzaron a reconstruir la Puerta de las Ovejas. La dedicaron y
colocaron sus puertas. Luego siguieron construyendo hasta la Torre de
los Cien y la Torre de Hananel y la dedicaron. \bibverse{2} Los hombres
de Jericó construyeron la parte contigua a Eliasib, y Zacur, hijo de
Imri, construyó la siguiente.

\bibverse{3} La Puerta del Pescado fue reconstruida por los hijos de
Senaa. Colocaron sus vigas y levantaron sus puertas, junto con sus
cerrojos y barras. \bibverse{4} La siguiente sección fue reparada por
Meremot, hijo de Urías, hijo de Cos; junto a él, estaba Mesulam, hijo de
Berequías, hijo de Mesezabeel; y junto a él, Sadoc, hijo de Baana.
\bibverse{5} A continuación estaban los tecoítas, pero sus nobles se
negaron a realizar trabajos manuales bajo un supervisor.

\bibverse{6} La Puerta Vieja fue reparada por Joiada, hijo de Paseah, y
Mesulam, hijo de Besodías. Colocaron sus vigas y levantaron sus puertas,
junto con sus cerrojos y barras. \bibverse{7} Luego fueron Melatiá el
gabaonita, Jadón el meronita y los hombres de Gabaón y Mizpa, que
estaban bajo la jurisdicción del gobernador de la provincia al oeste del
Éufrates.

\bibverse{8} A continuación estaba Uziel, hijo de Harhaiah, uno de los
orfebres; y junto a él, Hananías, hijo del perfumista. Ellos reforzaron
Jerusalén hasta el Muro Ancho. \bibverse{9} El siguiente fue Refaías,
hijo de Hur, gobernante de la mitad de Jerusalén.\footnote{\textbf{3:9}
  Se cree que es una zona fuera de la ciudad.} \bibverse{10} El
siguiente fue Jedaías, hijo de Harumap, que hizo reparaciones frente a
su casa. El siguiente fue Hattush, hijo de Hasabneías. \bibverse{11}
Malquías, hijo de Harim, y Hasub, hijo de Pahat-moab, trabajaron en otra
sección, así como en la Torre de los Hornos. \bibverse{12} A
continuación fue Salum, hijo de Halohes, gobernante de un medio distrito
de Jerusalén, ayudado por sus hijas.

\bibverse{13} La Puerta del Valle fue reparada por Hanún y la gente que
vivía en Zanoa. La reconstruyeron, colocaron sus puertas, junto con sus
cerrojos y barras, y repararon mil codos de la muralla hasta la Puerta
del Desecho. \bibverse{14} Malquías, hijo de Recab, jefe del distrito de
Bet-haquerem, reparó la Puerta de la Basura, la reconstruyó y colocó sus
puertas, junto con sus cerrojos y barras.

\bibverse{15} La Puerta de la Fuente fue reparada por Salún, hijo de
Col-hoze, jefe del distrito de Mizpa. La reconstruyó, le puso un techo y
levantó sus puertas, junto con sus cerrojos y barras. Reconstruyó el
muro del estanque de Selá, junto al jardín del rey, hasta las escaleras
que bajan de la ciudad de David. \bibverse{16} Después de él, Nehemías,
hijo de Azbuk, gobernante de un semidistrito de Bet-Zur, reparó hasta un
punto frente al cementerio de David, hasta el estanque artificial y la
Casa de los Guerreros Poderosos.

\bibverse{17} A continuación estaban los levitas bajo el mando de Rehum
hijo de Bani, y a continuación estaba Hasabías, gobernante de la mitad
de la región de Keila, que hizo las reparaciones en nombre de su
distrito. \bibverse{18} Junto a ellos estaban sus vecinos bajo
Binui,\footnote{\textbf{3:18} ``Binui'': El texto dice``Banai.''}hijo de
Henadad, gobernante de la otra mitad de la región de Keila.
\bibverse{19} El siguiente fue Ezer, hijo de Jesúa, gobernante de Mizpa,
quien reparó otra sección frente a la Colina de la Armería, donde la
muralla gira. \bibverse{20} El siguiente fue Baruc, hijo de Zabai, que
trabajó duro reparando otra sección, desde donde la muralla gira hasta
la entrada de la casa del sumo sacerdote Eliasib. \bibverse{21} A
continuación, Meremot, hijo de Urías, ton de Cos, que reparó otro tramo,
desde la entrada de la casa del sumo sacerdote Eliasib hasta el final.

\bibverse{22} A continuación, los sacerdotes de los alrededores hicieron
las reparaciones. \bibverse{23} Después de ellos estaban Benjamín y
Jasub, que hicieron reparaciones frente a su casa, y junto a ellos,
Azarías, hijo de Maasías, hijo de Ananías, hizo reparaciones junto a su
casa. \bibverse{24} El siguiente fue Binui, hijo de Henadad, que reparó
otra parte, desde la casa de Azarías hasta donde gira el muro y la
esquina. \bibverse{25} Palal, hijo de Uzai, trabajó frente a donde gira
el muro y la torre que se extiende desde el palacio superior, cerca del
patio de la guardia. Luego estaban Pedaías, hijo de Paros \bibverse{26}
y los servidores del Templo que vivían en la colina de Ofel, quienes
hacían reparaciones frente a la Puerta del Agua hacia el este y la torre
que se extiende. \bibverse{27} A continuación estaban los tecoítas que
repararon en otra sección frente a la gran torre que se extiende hasta
la muralla de Ofel.

\bibverse{28} Encima de la Puerta de los Caballos, cada uno de los
sacerdotes hizo reparaciones frente a su propia casa. \bibverse{29} El
siguiente era Sadoc, hijo de Imer, que trabajaba frente a su casa, y el
siguiente era Semaías, hijo de Secanías, el guardia de la Puerta
Oriental. \bibverse{30} A continuación estaban Hananías, hijo de
Selemías, y Hanún, sexto hijo de Zalaf, que se encargaba de la
reparación. El siguiente fue Mesulam, hijo de Berequías, que hizo
reparaciones frente a donde él vivía. \bibverse{31} El siguiente fue
Malquías, uno de los orfebres, que hizo reparaciones hasta la casa de
los sirvientes del Templo y de los mercaderes, frente a la Puerta de
Inspección, y hasta la sala que está sobre la esquina. \bibverse{32} Los
orfebres y los mercaderes hicieron reparaciones entre la sala que está
sobre la esquina y la Puerta de las Ovejas.

\hypertarget{section-3}{%
\section{4}\label{section-3}}

\bibverse{1} CuandoSanbalat se enteró de que estábamos reconstruyendo el
muro, se puso furioso, ¡muy furioso! Se burló de los judíos \bibverse{2}
delante de sus colegas y del ejército de Samaria, diciendo: ``¿Qué
pretenden estos judíos inútiles? ¿Creen que pueden reconstruir el muro?
¿Van a ofrecer sacrificios? ¿Van a terminarlo en un día? ¿Creen que
pueden reutilizar las piedras de los montones de escombros y de tierra,
sobre todo porque todas han sido quemadas?''

\bibverse{3} Tobías el amonita, que estaba a su lado, comentó: ``¡Hasta
una zorra que caminara sobre lo que están construyendo derribaría su
muro de piedras!''

\bibverse{4} Yo oré: ``Señor, por favor, escúchanos, porque nos están
tratando con desprecio. Haz que sus insultos caigan sobre sus propias
cabezas. Que se los lleven como un botín, prisioneros en tierra
extranjera. \bibverse{5} No perdones sus culpas ni borres sus pecados,
porque te han hecho enfadar delante de los constructores.''\footnote{\textbf{4:5}
  ``Porque te han hecho enfadar delante de los constructores'': o,
  ``porque han provocado a los constructores.''}

\bibverse{6} Así que reconstruimos la muralla hasta que quedó toda
unida, llegando a la mitad de su altura, porque el pueblo estaba deseoso
de trabajar.

\bibverse{7} CuandoSanbalat y Tobías, y los árabes, los amonitas y los
asdoditas, oyeron que la reparación de las murallas de Jerusalén
avanzaba y que se estaban rellenando los huecos, se pusieron furiosos.
\bibverse{8} Todos conspiraron juntos para venir a atacar Jerusalén y
para confundirlo todo. \bibverse{9} Así que oramos a nuestro Dios, y
tuvimos guardias preparados para defendernos de ellos día y noche.

\bibverse{10} Entonces la gente de Judá empezó a refunfuñar, diciendo:
``Los obreros están agotados. Hay demasiados escombros que limpiar.
Nunca podremos terminar el muro''.

\bibverse{11} Nuestros enemigos se decían: ``Antes de que se den cuenta,
antes de que se den cuenta de nada, apareceremos en medio de ellos, los
mataremos y pondremos fin a lo que están haciendo.''

\bibverse{12} Los judíos que vivían cerca venían y nos decían una y otra
vez: ``Nos van a atacar desde todas las direcciones!''\footnote{\textbf{4:12}
  Presunto significado, el hebreo no es claro en este aspecto.}
\bibverse{13} Así que posicioné a los defensores detrás de las secciones
más bajas y vulnerables de la muralla. Les hice tomar sus posiciones por
familias, armados con sus espadas, lanzas y arcos.

\bibverse{14} Después de inspeccionar nuestras defensas, me puse de pie
y me dirigí a los nobles, a los funcionarios y al resto del pueblo,
diciendo: ``¡No tengan miedo de ellos! ¡Recordad al Señor, que es
poderoso y formidable! Luchad por vuestros hermanos, vuestros hijos y
vuestras hijas, vuestras mujeres y vuestros hogares''.

\bibverse{15} Cuando nuestros enemigos descubrieron que conocíamos su
plan y que Dios lo había frustrado, todos volvimos a nuestro trabajo en
el muro. \bibverse{16} A partir de ese momento, la mitad de mis hombres
se dedicó a trabajar, mientras que la otra mitad estaba preparada para
luchar, con sus lanzas, escudos, arcos y armaduras. Los líderes se
colocaron detrás de todo el pueblo de Judá \bibverse{17} que estaba
construyendo el muro. Los que llevaban los materiales trabajaban con una
mano, y en la otra sostenían un arma. \bibverse{18} Todos los
constructores llevaban una espada atada al costado, y el trompetista
estaba a mi lado.\footnote{\textbf{4:18} Para avisar de un ataque.}

\bibverse{19} Luego les dije a los nobles, a los funcionarios y al resto
del pueblo: ``Tenemos mucho que hacer en todas partes, así que estamos
muy repartidos a lo largo de la muralla. \bibverse{20} Dondequiera que
estén y oigan el sonido de la trompeta, corranpara unirse a nosotros
allí. Nuestro Dios luchará por nosotros''.

\bibverse{21} Seguimos trabajando, con la mitad de los hombres
sosteniendo lanzas desde el amanecer hasta que salieron las estrellas.
\bibverse{22} También le dije a la gente: ``Todos, incluidos los
sirvientes, deben pasar la noche dentro de Jerusalén, para que puedan
hacer guardia por la noche y trabajar durante el día.'' \bibverse{23}
Durante ese tiempo ninguno de nosotros se cambió de ropa, ni yo, ni mis
hermanos, ni mis hombres, ni los guardias que estaban conmigo. Todos
llevaban sus armas en todo momento, incluso para ir a
buscaragua.\footnote{\textbf{4:23} ``Todos llevaban sus armas en todo
  momento, incluso para ir a buscar agua.''El hebreo no es claro.
  Literalmente se lee: ``cada uno su arma el agua.''}

\hypertarget{section-4}{%
\section{5}\label{section-4}}

\bibverse{1} Por aquel entonces, algunos hombres y sus esposas iniciaron
una tremenda discusión con los demás judíos. \bibverse{2} Se quejaban:
``Nuestras familias son tan numerosas que necesitamos más
comida\footnote{\textbf{5:2} ``Food'': literalmente, ``grano.''}para
comer y vivir''. \bibverse{3} Otros añadieron: ``Hemos tenido que
hipotecar nuestros campos, nuestros viñedos y nuestras casas para
comprar comida durante el tiempo de hambre.'' \bibverse{4} Otros más
dijeron: ``Hemos tenido que pedir prestado el dinero de nuestros campos
y viñedos para pagar el impuesto del rey. \bibverse{5} Aunque somos el
mismo pueblo que nuestros acreedores y aunque nuestros hijos son los
mismos que los suyos, vamos a tener que convertir a nuestros hijos e
hijas en esclavos. De hecho, algunas de nuestras hijas ya han sido
esclavizadas,\footnote{\textbf{5:5} Las niñas que se vendían podían ser
  tomadas como esposa por el comprador o por uno de sus hijos.}pero no
podemos hacer nada, porque nuestros campos y nuestras viñas son ahora
propiedad de otros''.

\bibverse{6} Me enfadé mucho cuando les oí protestar por sus quejas.
\bibverse{7} Me puse a pensar y luego fui a discutir con los nobles y
los funcionarios, diciéndoles: ``¡Están cobrándole intereses a sus
propios hermanos!'' Entonces convoqué una gran reunión para tratar con
ellos.

\bibverse{8} Allí les dije: ``Hemos hecho todo lo posible para comprar
de nuevo a nuestros hermanos judíos que fueron vendidos a los
extranjeros, pero ahora ustedes están vendiendo a sus propios hermanos
como esclavos. ¿Esperan venderlos de nuevo a nosotros?'' Se quedaron
callados porque no se les ocurría nada que decir.

\bibverse{9} ``Lo que ustedes están haciendo no está bien'', les dije.
``¿No creen que deberían respetar a nuestro Dios para que las naciones
enemigas no nos critiquen? \bibverse{10} Tanto yo como mis hermanos y
mis hombres hemos estado prestando al pueblo dinero y comida. Por favor,
¡dejemos este asunto de cobrar intereses! \bibverse{11} Devuélvanles
ahora mismo sus campos, viñedos, olivares y casas, junto con el uno por
ciento de interés sobre el dinero, el grano, el vino nuevo y el aceite
de oliva que les han estado cobrando.''

\bibverse{12} ``Lo devolveremos'', respondieron, ``y no les exigiremos
nada más. Haremos lo que tú digas''. Así que convoqué a los sacerdotes e
hice que los nobles y los funcionarios juraran que harían lo que habían
prometido.

\bibverse{13} Sacudí los pliegues de mi túnica y dije: ``¡Así es como mi
Dios los sacudirá de sus casas y de sus posesiones si no cumplen su
promesa! Si no lo hacen, serán sacudidos y acabarán sin nada''. Todos
los presentes dijeron: ``Amén'', y alabaron al Señor. El pueblo cumplió
lo que había prometido.

\bibverse{14} Además, desde el día en que el rey Artajerjes me nombró
gobernador en la tierra de Judá, que fue desde su vigésimo año hasta su
trigésimo segundo año, un total de doce años, ni yo ni mis hermanos
comimos la comida que se asignaba al gobernador. \bibverse{15} Pero los
gobernadores anteriores a mí habían impuesto una pesada carga al pueblo,
quitándole cuarenta siclos de plata, así como comida y vino. Sus
ayudantes también extorsionaban al pueblo. Pero por mi respeto a Dios no
actué así.

\bibverse{16} También hice de la reconstrucción de la muralla mi máxima
prioridad, y asigné a todos mis trabajadores para que ayudaran en ello.
No adquirimos ninguna tierra para nosotros. \bibverse{17} Tenía a 150
judíos y funcionarios comiendo en mi mesa, así como a visitantes de los
países cercanos. \bibverse{18} Cada día pagaba un buey, seis buenas
ovejas y aves de corral. Cada diez días pagaba una gran cantidad de vino
de todo tipo. Pero nunca exigí la asignación de alimentos del
gobernador, porque el pueblo ya llevaba una pesada carga. \bibverse{19}
Por favor, Dios mío, recuérdame positivamente por todo lo que he hecho
por este pueblo.

\hypertarget{section-5}{%
\section{6}\label{section-5}}

\bibverse{1} CuandoSanbalat, Tobías, Gesem el árabe y nuestros otros
enemigos se enteraron de que yo había reconstruido la muralla y que no
quedaban huecos -- aunqueen ese momento todavía no había colocado las
puertas en los portones -- \bibverse{2} me enviaron un mensaje,
diciendo: ``Vamos, reunámonos en una de las aldeas de la llanura de
Ono''. Pero tenían la intención de matarme.

\bibverse{3} Así que envié mensajeros para decirles: ``Estoy ocupado con
un trabajo importante y no puedo bajar. ¿Por qué voy a dejar lo que
estoy haciendo para venir a verlos a ustedes?''

\bibverse{4} Me enviaron el mismo mensaje cuatro veces, y cada vez mi
respuesta fue la misma.

\bibverse{5} Sanbalat me envió el mismo mensaje la quinta vez por medio
de su criado, que traía en su mano una carta abierta. \bibverse{6} La
carta decía: ``La gente de los alrededores dice, y Gesem lo confirma,
que tú y los judíos están planeando una rebelión, y que por eso estás
construyendo el muro. También planeas convertirte en su rey, según
dicen, \bibverse{7} e incluso has dispuesto que los profetas de
Jerusalén anuncien por ti: `Hay un rey en Judá'. El rey\footnote{\textbf{6:7}
  Refiriéndose al reypersa.}pronto se enterará de esto. Así que ven, y
hablemos de esto''.

\bibverse{8} Le contesté diciéndole: ``¡No pasa nada de lo que dices! De
hecho, ¡te lo estás inventando todo!''.

\bibverse{9} Todos intentaban asustarnos, diciéndose a sí mismos: ``No
tendrán fuerzas para trabajar, así que nunca se terminará''. Pero yo
rezaba, ¡hazme fuerte!

\bibverse{10} Más tarde, fui a casa de Semaías (era hijo de Delaías,
hijo de Mehetabel) que se había encerrado en su casa.\footnote{\textbf{6:10}
  ``Se había encerrado en su casa'': evidentemente las tácticas de miedo
  habían funcionado, o Semaiasaparentaba que así era.}Él dijo: ``Ven y
reúnete conmigo en la casa de Dios dentro del Templo. ¡Entonces podremos
cerrar las puertas del Templo porque vienen a matarte! Vienen a matarte
esta noche''.

\bibverse{11} Yo respondí: ``¿Debe alguien como yo huir? ¿Debería
alguien como yo ir y esconderse en el Templo para poder
sobrevivir?\footnote{\textbf{6:11} Nehemías no podía entrar
  legítimamente en el Templo porque no era sacerdote.}No voy a ir''.

\bibverse{12} Pensé en ello y vi que Dios no lo había enviado, sino que
había dicho esta profecía contra mí porque Tobías y Sanbalat lo habían
contratado. \bibverse{13} Lo habían contratado pensando que me asustaría
para que hiciera algo malo. Así podrían señalar con el dedo y arruinar
mi reputación. \bibverse{14} Dios mío, acuérdate de Tobías y de Sanbalat
por haber hecho esto, y también de la profetisa Noadías y de los otros
profetas que trataron de asustarme.

\bibverse{15} El muro fue terminado el día veinticinco del mes de Elul.
Se necesitaron cincuenta y dos días. 16 Cuando todos nuestros enemigos
se enteraron, se asustaron; todas las naciones de alrededor se
desanimaron mucho, pues reconocieron que esto había sido hecho por
nuestro Dios.

\bibverse{17} En aquel tiempo los nobles de Judá intercambiaban muchas
cartas con Tobías, \bibverse{18} porque mucha gente en Judá le había
hecho un juramento de lealtad, ya que era yerno de Secanías, hijo de
Ara, y su hijo Johanán estaba casado con la hija de Mesulam, hijo de
Berequías. \bibverse{19} No dejaban de contarme todas las cosas buenas
que hacía Tobías, y le informaban de lo que yo decía. Tobías también
envió cartas para tratar de asustarme.

\hypertarget{section-6}{%
\section{7}\label{section-6}}

\bibverse{1} Una vez reconstruida la muralla y levantadas las puertas,
nombré a los porteros, a los cantores y a los levitas. \bibverse{2} Puse
a mi hermano Hanani a cargo de Jerusalén, junto con Hananías, el
comandante de la fortaleza, porque era un hombre honesto que respetaba a
Dios más que muchos otros.

\bibverse{3} Les dije: ``No permitan que se abran las puertas de
Jerusalén hasta que el sol esté caliente,\footnote{\textbf{7:3} Esta
  frase también podría traducirse como ``No permitas que se abran las
  puertas de Jerusalén cuando el sol está caliente'', es decir, durante
  el tiempo después de la comida, cuando los guardias pueden ser
  negligentes en sus deberes.}y asegúrate de que los guardias cierren y
echen el cerrojo a las puertas mientras estén de servicio. Nombra a
algunos de los habitantes de Jerusalén como guardias, para que estén en
sus puestos, frente a sus propias casas''.

\bibverse{4} En aquellos tiempos la ciudad era grande y con mucho
espacio, pero no había mucha gente en ella, y las casas no habían sido
reconstruidas. \bibverse{5} Mi Dios me animó a que todos -los nobles,
los funcionarios y el pueblo- vinieran a registrarse según su genealogía
familiar. Encontré el registro genealógico de los que habían regresado
primero. Esto es lo que descubrí escrito allí.

\bibverse{6} Esta es una lista de la gente de la provincia que regresó
del cautiverio. Estos eran los exiliados que habían sido llevados a
Babilonia por el rey Nabucodonosor. Regresaron a Jerusalén y a Judá, a
sus ciudades de origen. \bibverse{7} Estaban dirigidos por Zorobabel,
Jesúa, Nehemías, Azarías, Raamías, Nahamani, Mardoqueo, Bilsán,
Misperet, Bigvai, Nehum y Baana.

Este es el número de hombres del pueblo de Israel\footnote{\textbf{7:7}
  Esta lista es similar a la que se encuentra en Esdras 2, con algunas
  diferencias de ortografía, orden y número.} \bibverse{8} Los hijos de
Paros, 2.172; \bibverse{9} los hijos de Sefatías, 372; \bibverse{10} los
hijos de Ara, 652; \bibverse{11} los hijos de Pahat-moab, (los hijos de
Jesúa y Joab), 2.818; \bibverse{12} los hijos de Elam, 1.254;
\bibverse{13} los hijos de Zatu, 845; \bibverse{14} los hijos de Zacai,
760; \bibverse{15} los hijos de Binui, 648; \bibverse{16} los hijos de
Bebai, 628; \bibverse{17} los hijos de Azgad, 2.322; \bibverse{18} los
hijos de Adonicam, 667; \bibverse{19} los hijos de Bigvai, 2.067.
\bibverse{20} Los hijos de Adin, 655. \bibverse{21} Los hijos de Ater,
(hijos de Ezequías), 98; \bibverse{22} los hijos de Hasum, 328;
\bibverse{23} los hijos de Bezai, 324; \bibverse{24} los hijos de Harif,
112; \bibverse{25} los hijos de Gabaón, 95; \bibverse{26} el pueblo de
Belén y Netofa, 188; \bibverse{27} el pueblo de Anatot, 128;
\bibverse{28} el pueblo de Bet-azmavet 42; \bibverse{29} el pueblo de
Quiriat-jearim, Cafira y Beerot, 743; \bibverse{30} el pueblo de Ramá y
Geba, 621; \bibverse{31} el pueblo de Micmas, 122; \bibverse{32} el
pueblo de Bet-el y Ai, 123; \bibverse{33} el pueblo del otro Nebo, 52;
\bibverse{34} los hijos del otro Elam, 1.254; \bibverse{35} los hijos de
Harim, 320; \bibverse{36} los hijos de Jericó, 345; \bibverse{37} los
hijos de Lod, Hadid y Ono, 721; \bibverse{38} los hijos de Senaa, 3.930.

\bibverse{39} Este es el número de los sacerdotes: los hijos de Jedaías
(por la familia de Jesúa), 973; \bibverse{40} los hijos de Imer, 1.052;
\bibverse{41} los hijos de Pasur, 1.247; \bibverse{42} los hijos de
Harim, 1.017.

\bibverse{43} Este es el número de los levitas: los hijos de Jesúa por
Cadmiel (hijos de Hodavías), 74; \bibverse{44} los cantores de los hijos
de Asaf, 148; \bibverse{45} los porteros de las familias de Salum, Ater,
Talmón, Acub, Hatita y Sobai, 138.

\bibverse{46} Los descendientes de estos servidores del Templo: Ziha,
Hasufa, Tabaot, \bibverse{47} Queros, Sia, Padón, \bibverse{48} Lebana,
Hagaba, Salmai, \bibverse{49} Hanán, Gidel, Gahar, \bibverse{50} Reaía,
Rezín, Necoda, \bibverse{51} Gazam, Uza, Paseah, \bibverse{52} Besai,
Mehunim, Nefusim, \bibverse{53} Bacbuc, Hacufa, Harhur, \bibverse{54}
Bazlut, Mehída, Harsa, \bibverse{55} Barcos, Sísara, Tema, \bibverse{56}
Nezía, y Hatifa.

\bibverse{57} Los descendientes de los siervos del rey Salomón: Sotai,
Soferet, Perida, \bibverse{58} Jaala, Darcón, Gidel, \bibverse{59}
Sefatías, Hatil, Poqueret-hazebaim y Amón. \bibverse{60} El total de los
siervos del Templo y de los descendientes de los siervos de Salomón era
de 392.

\bibverse{61} Los que procedían de las ciudades de Tel-mela, Tel-Harsa,
Querub, Addán e Imer no podían demostrar su genealogía familiar, ni
siquiera que eran descendientes de Israel. \bibverse{62} Entre ellos
estaban las familias de Delaía, Tobías y Necoda, 642 en total.
\bibverse{63} Además había tres familias sacerdotales, hijos de Habaía,
Cos y Barzilai. (Barzilai se había casado con una mujer descendiente de
Barzilai de Galaad, y se llamaba por ese nombre). \bibverse{64} Se buscó
un registro de ellos en las genealogías, pero no se encontraron sus
nombres, por lo que se les prohibió servir como sacerdotes.
\bibverse{65} El gobernador les ordenó que no comieran nada de los
sacrificios del santuario hasta que un sacerdote pudiera preguntar al
Señor sobre el asunto utilizando el Urim y el Tumim.

\bibverse{66} El total de personas que regresaron fue de 42.360.
\bibverse{67} Además había 7.337 sirvientes y 245 cantores y cantoras.
\bibverse{68} Tenían 736 caballos, 245 mulas,\footnote{\textbf{7:68} A
  la mayoría de los manuscritos hebreos les falta este versículo.}
\bibverse{69} 435 camellos y 6.720 burros.

\bibverse{70} Algunos de los jefes de familia hicieron contribuciones
voluntarias para el trabajo. El gobernador entregó a la tesorería 1.000
dáricos de oro, 50 cuencos y 530 conjuntos de ropa para los sacerdotes.
\bibverse{71} Algunos de los jefes de familia donaron al tesoro para la
obra 20.000 dáricos de oro y 2.200 minas de plata. \bibverse{72} El
resto del pueblo donó 20.000 dáricos de oro, 2.000 minas de plata y 67
conjuntos de ropa para los sacerdotes.

\bibverse{73} Los sacerdotes, los levitas, los porteros, los cantores y
los servidores del Templo, así como parte del pueblo y el resto de los
israelitas, volvieron a vivir en sus pueblos específicos. En el séptimo
mes los israelitas vivían en sus pueblos,

\hypertarget{section-7}{%
\section{8}\label{section-7}}

\bibverse{1} y el pueblo se reunió como uno solo en la plaza junto a la
Puerta del Agua. Le dijeron a Esdras el escriba\footnote{\textbf{8:1}
  ``Escriba'': como en otras partes de la Escritura, un escriba no es
  simplemente alguien que puede escribir, sino un maestro,
  particularmente de la Ley de Dios.}que sacara el Libro de la Ley de
Moisés, que el Señor había ordenado seguir a Israel.

\bibverse{2} El primer día del séptimo mes, el sacerdote Esdras llevó la
Ley ante la asamblea, hombres y mujeres, y todos los niños que podían
escuchar y entender. \bibverse{3} Leyó la Ley delante de la plaza de la
Puerta del Agua, desde la mañana hasta el mediodía, a todos los que
estaban allí, a los hombres y a las mujeres y a los que podían entender.
Todo el pueblo escuchaba atentamente el Libro de la Ley.

\bibverse{4} El escriba Esdras estaba de pie en un alto escenario de
madera construido para este evento. A su derecha estaban Matatías, Sema,
Anaías, Urías, Hilcías y Maasías, y a su izquierda Pedaías, Misael,
Malquías, Hasum, Hash-badana, Zacarías y Mesulam. \bibverse{5} Esdras
abrió el libro mientras todos lo miraban porque toda la multitud podía
verlo. Cuando lo abrió, todos se pusieron de pie.

\bibverse{6} Esdras alabó al Señor, el gran Dios, y todos respondieron:
``¡Amén! Amén!'' mientras levantaban las manos. Luego se inclinaron y
adoraron al Señor con el rostro en el suelo.

\bibverse{7} Jesúa, Baní, Serebías, Jamín, Acub, Sabetai, Hodías,
Maasías, Kelita, Azarías, Jozabad, Hanán y Pelaías, que eran los levitas
presentes, explicaron la Ley al pueblo mientras éste permanecía de pie.
8 Ellos leyeron del Libro de la Ley de Dios, aclarando el significado
para que el pueblo pudiera entender lo que decía.\footnote{\textbf{8:7}
  No sólo era necesaria una interpretación teológica, sino que, dado que
  muchos de los presentes se habían acostumbrado a hablar arameo en
  Babilonia, era necesaria una traducción del hebreo en el que estaba
  escrito el Libro.}

\bibverse{9} Entonces el gobernador Nehemías, el sacerdote y escriba
Esdras y los levitas que enseñaban al pueblo les dijeron a todos: ``Este
es un día santo para el Señor, vuestro Dios. No lloren ni se lamenten
``, porque todos lloraban al oír la lectura de la Ley.

\bibverse{10} Nehemías continuó diciendo: ``Vayan y disfruten de buena
comida y bebidas dulces, y compartan con los que no tienen nada
preparado, porque hoy es un día especial y santo para nuestro Señor. No
estén tristes, porque su fuerza viene del Señor, que los hace felices''.

\bibverse{11} También los levitas calmaban a todos, diciéndoles: ``¡No
lloren! Este es un día santo y no deben estar tristes''.

\bibverse{12} Entonces todos se fueron a comer y a beber, y a compartir
su comida. Celebraron con alegría porque ahora entendían la Ley tal como
se les había explicado.

\bibverse{13} Al día siguiente, los jefes de familia de todo el pueblo,
así como los sacerdotes y los levitas, se reunieron con el escriba
Esdras para estudiar la Ley con mayor profundidad. \bibverse{14}
Descubrieron que en la Ley que el Señor había ordenado observar por
medio de Moisés, estaba escrito que los israelitas debían vivir en
refugios durante la fiesta del séptimo mes. \bibverse{15} Debían hacer
un anuncio en todas sus ciudades y en Jerusalén, diciendo: ``Vayan al
monte y traigan ramas de olivo, de acebuche, de mirto, de palma y de
otros árboles frondosos, para hacer refugios para vivir, como lo exige
la Ley.''\footnote{\textbf{8:15} No se trata de una cita bíblica
  directa, sino de un resumen de los requisitos.}

\bibverse{16} Así que salieron y trajeron ramas y se hicieron refugios
en los tejados de sus casas, en sus patios, en los patios del Templo de
Dios y en las plazas cercanas a la Puerta del Agua y a la Puerta de
Efraín. \bibverse{17} Todos los que volvieron del exilio hicieron
refugios y se quedaron en ellos. No habían celebrado así desde los
tiempos de Josué, hijo de Nun. Todos estaban muy contentos.

\bibverse{18} Esdras leyó del Libro de la Ley de Dios todos los días,
desde el primero hasta el último. Los israelitas celebraron la fiesta
durante siete días, y el octavo día se reunieron en asamblea, como
exigía la Ley.

\hypertarget{section-8}{%
\section{9}\label{section-8}}

\bibverse{1} El día veinticuatro de este mismo mes, los israelitas se
reunieron, ayunando y vistiendo de cilicio, con polvo en la cabeza.
\bibverse{2} Los de ascendencia israelita se separaron de los
extranjeros y se pusieron de pie para confesar sus pecados y los de sus
antepasados. \bibverse{3} Pasaron tres horas\footnote{\textbf{9:3}
  ``Tres horas'': literalmente, ``un carto del día.''}de pie leyendo el
Libro de la Ley del Señor su Dios, y otras tres horas confesando sus
pecados y adorando al Señor su Dios. \bibverse{4} Los levitas se
pusieron de pie en el estrado y clamaron en voz alta al Señor su Dios.
(Sus nombres eran Jesúa, Baní, Cadmiel, Sebanías, Buní, Serebías, Baní y
Chenani).

\bibverse{5} Entonces los levitas anunciaron: ``Pónganse de pie y alaben
al Señor, su Dios, que vive eternamente: `Que sean bendecidos quienes
son y su gloria, y que sean elevados por encima de toda bendición y
alabanza'``. (Los nombres de los levitas eran Jesúa, Cadmiel, Bani,
Hasabneías, Serebías, Hodías, Sebanías y Petaías).

\bibverse{6} Ellos oraron: ``Sólo tú eres el Señor. Tú hiciste el cielo,
los cielos con todas sus estrellas, la tierra y todo lo que hay en ella,
los mares y todo lo que hay en ellos. Tú les das vida a todos ellos, y
todos los seres celestiales te adoran.

\bibverse{7} Tú eres el Señor, el Dios que eligió a Abram, que lo sacó
de Ur de los caldeos y le dio el nombre de Abraham. \bibverse{8} Tú
sabías que te sería fiel, e hiciste un acuerdo con él para darle a él y
a su descendencia la tierra de los cananeos, hititas, amorreos,
ferezeos, jebuseos y gergeseos. Cumpliste tu promesa, porque haces lo
que es justo.

\bibverse{9} Viste cuánto sufrían nuestros antepasados en Egipto. Oíste
sus gritos de auxilio en el Mar Rojo. \bibverse{10} Te manifestaste con
señales y milagros contra el Faraón, todos sus funcionarios y todo el
pueblo de su tierra, porque reconociste la arrogancia con que trataron a
nuestros antepasados. Te creaste una maravillosa fama que la gente sigue
reconociendo hasta el día de hoy. \bibverse{11} Dividiste el mar delante
de ellos para que pudieran atravesarlo en seco. Pero arrojaste a sus
perseguidores a las profundidades del mar, como piedras arrojadas a las
aguas embravecidas.

\bibverse{12} Los guiaste con una columna de nube durante el día y con
una columna de fuego durante la noche, mostrándoles el camino que debían
seguir. \bibverse{13} Descendiste en el monte Sinaí. Les hablaste desde
el cielo. Les diste caminos correctos para vivir, leyes verdaderas, y
buenos reglamentos y mandamientos. \bibverse{14} Les explicaste tu santo
sábado. Les diste mandamientos, reglamentos y leyes por medio de tu
siervo Moisés. \bibverse{15} Cuando tuvieron hambre les diste pan del
cielo, y cuando tuvieron sed les sacaste agua de la roca. Les dijiste
que fueran a tomar posesión de la tierra que habías jurado darles.

\bibverse{16} Peroellos\footnote{\textbf{9:16} ``Pero ellos'': La
  estructura de este pasaje se centra en la alternancia de las acciones
  del pueblo de Dios (pero ellos) y de Dios (pero tú). La presente
  traducción ha intentado conservar este formato, y asegurar que los
  párrafos pertinentes comiencen con estos elementos contrastantes.}y
nuestros antepasados actuaron con arrogancia y se volvieron obstinados,
y no prestaron atención a tus mandatos. \bibverse{17} Se negaron a
escucharte y se olvidaron de todos los milagros que hiciste por ellos.
Se obstinaron y decidieron elegir ellos mismos un líder que los llevara
de vuelta a la esclavitud en Egipto.\footnote{\textbf{9:17} ``En
  Egipto'': Tomado de la Septuaginta y de algunos manuscritos hebreos.
  La mayoría de los manuscritos hebreos leen ``en rebeldía''.
  VéaseNúmeros 14:4.}

Pero tú eres un Dios que perdona, clemente y misericordioso, lento para
enojarse y lleno de amor confiable. No los abandonaste, \bibverse{18} ni
siquiera cuando se hicieron un becerro de metal y dijeron: ``Este es su
dios que los sacó de Egipto'', y cometieron terribles blasfemias.

\bibverse{19} Pero tú, por ser tan misericordioso, no los abandonaste en
el desierto. La columna de nube no dejó de guiarlos durante el día, y la
columna de fuego siguió iluminando su camino por la noche. \bibverse{20}
Les diste tu buen Espíritu para enseñarles. No dejaste de alimentarlos
con tu maná, y les diste agua cuando tuvieron sed. \bibverse{21}
Cuidaste de ellos durante cuarenta años en el desierto. Sus ropas no se
desgastaron, no les faltó nada. Ni siquiera se les hincharon los pies.

\bibverse{22} Les diste reinos y naciones; les asignaste sus fronteras.
Se apoderaron de la tierra de Sehón, rey de Hesbón, y de Og, rey de
Basán. \bibverse{23} Hiciste que sus descendientes fueran tan
innumerables como las estrellas del cielo, y los condujiste a la tierra
que habías prometido a sus padres que entrarían y poseerían.
\bibverse{24} Sus descendientes entraron y se apoderaron de la tierra.
Delante de ellos conquistaste a los cananeos que vivían allí,
entregándoles sus reyes y su pueblo para que hicieran con ellos lo que
quisieran. \bibverse{25} Capturaron ciudades fortificadas y tierras
fértiles. Se apoderaron de casas llenas de cosas valiosas, cisternas de
agua, viñedos, olivares y muchos árboles frutales. Comieron hasta
saciarse y engordaron. Estaban muy contentos de lo buenos que eran con
ellos.

\bibverse{26} Pero se rebelaron por completo\footnote{\textbf{9:26}
  ``Pero se rebelaron por completo'': literalmente, ``Pero se rebelaron
  y fueron rebeldes''. La repetición del término intensifica el alcance
  de la rebelión.}contra ti. Arrojaron tu Ley tras sus espaldas. Mataron
a tus profetas que les advertían para que intentaran volver a ti, y
cometieron terribles blasfemias. \bibverse{27} Por eso los entregaste a
sus enemigos, que los trataron mal. En su sufrimiento clamaron a ti por
ayuda.

Pero tú oíste sus gritos desde el cielo y, como eres tan misericordioso,
les enviaste líderes\footnote{\textbf{9:27} ``Líderes'': literalmente,
  ``Salvadores.''}para salvarlos de sus enemigos.

\bibverse{28} Sin embargo, en cuanto tuvieron paz, volvieron a hacer el
mal ante tus ojos. Así que una vez más los entregaste a sus enemigos,
que los dominaron. Volvieron a ti, y te gritaron de nuevo.

Pero tú oíste desde el cielo una vez más, y los salvaste una y otra vez
porque eres muy misericordioso. \bibverse{29} Les advertiste que
volvieran a tu Ley, pero fueron arrogantes. Ignoraron tus mandatos y
pecaron contra tus reglas, que, como ya dijiste,\footnote{\textbf{9:29}
  ``Comola dijiste'': añadido para mayor claridad. Ver Leviticos 18:5;
  Deuteronomio4:1; Deuteronomio 30:16.} `Si la gente obedece vivirá por
ellos'. Se obstinaron en darte la espalda y se negaron a escuchar.
\bibverse{30} Tuviste paciencia con ellos durante muchos años. Les
advertiste con tu Espíritu por medio de tus profetas, pero no te
escucharon, así que los entregaste a las demás naciones. \bibverse{31}
Pero por tu maravillosa misericordia no terminaste con ellos y no los
abandonaste, porque eres un Dios clemente y misericordioso.

\bibverse{32} Así que ahora, nuestro Dios, el grande y poderoso y
asombroso Dios que mantiene su acuerdo de amor confiable, por favor no
ve como sin importancia todas las dificultades que nos han sucedido a
nosotros, y a nuestros reyes y líderes, a nuestros sacerdotes y
profetas, a nuestro antepasado y a todo tu pueblo, desde el tiempo de
los reyes asirios de Asiria hasta ahora.

\bibverse{33} Pero tú has hecho lo correcto con respecto a todo lo que
nos ha sucedido. Siempre has actuado con fidelidad, mientras que
nosotros hemos hecho tanto mal. \bibverse{34} Nuestros reyes, nuestros
dirigentes, nuestros sacerdotes y nuestros antepasados no siguieron tu
Ley, e ignoraron tus mandatos y reglamentos que les ordenaste cumplir.

\bibverse{35} Pero ellos, incluso durante el tiempo en que tuvieron su
propio reino, con tantas bendiciones que les habías dado en la tierra
amplia y fértil que les habías proporcionado, incluso entonces se
negaron a servirte y no se apartaron de sus malos caminos.

\bibverse{36} Míranos ahora, esclavos en la tierra que diste a nuestros
antepasados para disfrutar de sus frutos y de todas sus cosas buenas.
Míranos a nosotros, esclavos aquí. \bibverse{37} Las ricas cosechas de
esta tierra son para los reyes que has puesto sobre nosotros a causa de
nuestros pecados. Ellos gobiernan nuestros cuerpos y nuestro ganado,
haciendo lo que quieren. Estamos sufriendo mucho''.

\bibverse{38} En respuesta, el pueblo declaró,\footnote{\textbf{9:38}
  ``El respuesta, el pueblo declaró'': añadido para mayor claridad.}``Teniendo
en cuenta todo esto, estamos haciendo un acuerdo solemne, poniéndolo por
escrito. Está sellado por nuestros líderes, levitas y sacerdotes.''

\hypertarget{section-9}{%
\section{10}\label{section-9}}

\bibverse{1} El documento fue sellado por: Nehemías el gobernador, hijo
de Hacalías.

\bibverse{2} Seraías, Azarías, Jeremías, \bibverse{3} Pasur, Amarías,
Malquías, \bibverse{4} Hatús, Sebanías, Maluc, \bibverse{5} Harim,
Meremot, Obadías, \bibverse{6} Daniel, Ginetón, Baruc, \bibverse{7}
Mesulam, Abías, Mijamín, \bibverse{8} Maazías, Bilgai y Semaías; estos
eran sacerdotes.

\bibverse{9} Los levitas: Jesúa hijo de Azanías, Binúi, de los hijos de
Henadad, Cadmiel, \bibverse{10} y sus hermanos Sebanías, Hodías, Kelita,
Pelaías, Hanán, \bibverse{11} Micaía, Rehob, Hasabías, \bibverse{12}
Zacur, Serebías, Sebanías, \bibverse{13} Hodías, Bani y Beninu.

\bibverse{14} Los líderes del pueblo: Paros, Pahat-moab, Elam, Zatu,
Bani, \bibverse{15} Buni, Azgad, Bebai, \bibverse{16} Adonías, Bigvai,
Adin, \bibverse{17} Ater, Ezequías, Azur, \bibverse{18} Hodías, Hasum,
Bezai, \bibverse{19} Harif, Anatot, Nebai, \bibverse{20} Magpías,
Mesulam, Hezir, \bibverse{21} Mesezabeel, Sadoc, Jadúa, \bibverse{22}
Pelatías, Hanán, Anaías, \bibverse{23} Oseas, Ananías, Hasub,
\bibverse{24} Halohes, Pilha, Sobec, \bibverse{25} Rehum, Hasabna,
Maasías, \bibverse{26} Ahías, Hanán, Anán, \bibverse{27} Maluc, Harim y
Baana.

\bibverse{28} El resto del pueblo, incluidos los sacerdotes, los
levitas, los porteros, los cantores y los servidores del Templo, y todos
los que se habían separado del pueblo de la tierra\footnote{\textbf{10:28}
  ``El pueblo de la tierra'': refiriéndose principalmente a los pueblos
  cananeos que anteriormente habían gobernado la tierra.}para guardar la
Ley de Dios, así como sus esposas y todos sus hijos e hijas que tuvieran
edad suficiente para entender, \bibverse{29} se unieron a los líderes
para jurar seguir la Ley de Dios dada a través de Moisés, el siervo de
Dios, para prestar atención y llevar a cabo todos los mandatos del
Señor, nuestro Dios, sus normas y reglamentos.

\bibverse{30} ``Prometemos no permitir que nuestras hijas se casen con
el pueblo de la tierra, y no permitir que nuestros hijos se casen con
sus hijas. \bibverse{31} Cuando los pueblos de la tierra traigan
mercancías y toda clase de alimentos para venderlos en el santo día de
reposo, no les compraremos nada en el día de reposo ni en los demás días
sagrados. Cada siete años dejaremos que la tierradescanse,\footnote{\textbf{10:31}
  Literalmente, ``dejar el séptimo año,''siguiendo el reglamento del
  ``año sabático'' que cada siete años los campos utilizados para
  producir cosechas debían dejarse en ``descanso''. Éxodo 23:10-11.}y
anularemos todas las deudas.

\bibverse{32} Aceptamos la obligación de pagar un tercio de siclo para
el funcionamiento del Templo de Dios, \bibverse{33} para el pan de la
proposición, para las ofrendas regulares de grano y los holocaustos,
para las ofrendas del sábado, para la luna nueva y las fiestas anuales,
para las ofrendas sagradas, para las ofrendas por el pecado para hacer
expiación por Israel, en fin, todo lo que tiene lugar en el Templo de
nuestro Dios.

\bibverse{34} Hemos repartido por sorteo entre los sacerdotes, los
levitas y el pueblo, para determinar quiénes traerán leña al Templo de
nuestro Dios para quemarla en el altar del Señor, nuestro Dios, en
determinados momentos del año, como lo exige la Ley.

\bibverse{35} También prometemos traer cada año al Templo del Señor la
primera parte de los productos de nuestros campos y de todos los árboles
frutales. \bibverse{36} Llevaremos los primogénitos de nuestros hijos y
de nuestro ganado y de nuestras manadas y rebaños al Templo de nuestro
Dios, a los sacerdotes que allí ejercen su ministerio, como lo exige la
Ley. \bibverse{37} Llevaremos a los almacenes del Templo de nuestro
Dios, para los sacerdotes, la primera parte de nuestra harina molida, de
nuestras ofrendas de grano, del fruto de todos nuestros árboles, y de
nuestro vino nuevo y aceite de oliva. También llevaremos el diezmo de
nuestros productos a los levitas, porque los levitas son los que recogen
los diezmos en todas las ciudades agrícolas.

\bibverse{38} Un sacerdote descendiente de Aarón acompañará a los
levitas cuando recojan el diezmo, y los levitas deberán llevar un diezmo
de estos diezmos a las salas del almacén del Templo de nuestro Dios.

\bibverse{39} El pueblo de Israel y los levitas llevarán las ofrendas de
grano, vino nuevo y aceite de oliva a los almacenes donde se guardan los
objetos del santuario, donde están los sacerdotes ministrantes, los
porteros y los cantores. No olvidaremos el Templo de nuestro Dios.''

\hypertarget{section-10}{%
\section{11}\label{section-10}}

\bibverse{1} Los líderes del pueblo ya vivían en Jerusalén. El resto del
pueblo echó suertes para que uno de cada diez viniera a vivir a
Jerusalén, la ciudad santa, mientras que los otros nueve se quedarían en
sus propias ciudades. \bibverse{2} Todos alabaron a los que estaban
dispuestos a trasladarse a Jerusalén.

\bibverse{3} Esta es una lista de los líderes de la provincia que
vinieron a vivir a Jerusalén. (La mayoría de los israelitas vivían en su
propia propiedad en las ciudades de Judá. Esto incluía a los sacerdotes,
los levitas, los servidores del Templo y los descendientes de los
servidores de Salomón que vivían en sus ciudades de origen. \bibverse{4}
Sin embargo, algunos de los habitantes de Judá y Benjamín se trasladaron
a Jerusalén).

De la tribu de Judá Ataías, hijo de Uzías, hijo de Zacarías, hijo de
Amarías, hijo de Sefatías, hijo de Mahalalel, de los hijos de Fares;
\bibverse{5} y Maasías, hijo de Baruc, hijo de Colhoze, hijo de Hazaías,
hijo de Adaía, hijo de Joiarib, hijo de Zacarías, descendiente de Sela.
\bibverse{6} El total de los hijos de Fares que vivieron en Jerusalén
fue de 468 hombres hábiles.

\bibverse{7} De la tribu de Benjamín: Salú, hijo de Mesulam, hijo de
Joed, hijo de Pedaías, hijo de Colaías, hijo de Itiel, hijo de Jesaías,
\bibverse{8} y después de él Gabbai y Salai, en total 928. \bibverse{9}
Joel hijo de Zicri era el oficial a cargo de ellos, y Judá hijo de Senúa
era el segundo al mando de la ciudad.

\bibverse{10} De los sacerdotes Jedaías, hijo de Joiarib, Jaquín;
\bibverse{11} Seraías, hijo de Hilcías, hijo de Mesulam, hijo de Sadoc,
hijo de Meraiot, hijo de Ajitub, administrador principal del Templo de
Dios, \bibverse{12} y sus compañeros sacerdotes que servían en el
Templo, un total de 822; Adaía hijo de Jeroham, hijo de Pelaliah, hijo
de Amzi, hijo de Zacarías, hijo de Pasur, hijo de Malquías,
\bibverse{13} y los que trabajaban con él, jefes de familia, un total de
242; y Amasai, hijo de Azarel, hijo de Ahzai, hijo de Mesilemot, hijo de
Imer, \bibverse{14} y los que trabajaban con él, un total de 128
guerreros fuertes.\footnote{\textbf{11:14} ``Guerreros fuertes'': Para
  los oídos modernos puede sonar extraño ver que los sacerdotes figuran
  como guerreros. Sin embargo, en aquellos tiempos las necesidades
  defensivas eran una preocupación práctica, especialmente la defensa
  del Templo.}Zabdiel, hijo de Gedolim, estaba a cargo de ellos.

\bibverse{15} De los levitas Semaías, hijo de Hasub, hijo de Azricam,
hijo de Hasabías, hijo de Buni; \bibverse{16} y Sabetai y Jozabad,
líderes levitas que estaban a cargo de los trabajos exteriores del
Templo de Dios; \bibverse{17} Matanías, hijo de Mica, hijo de Zabdi,
hijo de Asaf, que dirigía la acción de gracias y la alabanza; y
Bacbuqías, que era el segundo; y Abda, hijo de Samúa, hijo de Galal,
hijo de Jedutún. \bibverse{18} El número total de sacerdotes en la
ciudad santa era de 284.

\bibverse{19} Los porteros: Acub, Talmón y sus compañeros, que
custodiaban las puertas: un total de 172.

\bibverse{20} Los demás israelitas, con el resto de los sacerdotes y
levitas, vivían en sus ciudades de origen en Judá, cada uno en su propia
propiedad.

\bibverse{21} Los servidores del Templo vivían en la colina de Ofel.
Ziha y Gispa estaban a cargo de ellos.

\bibverse{22} El que estaba a cargo de los levitas en Jerusalén era Uzi,
hijo de Bani, hijo de Hasabías, hijo de Matanías, hijo de Mica, uno de
los descendientes de Asaf, los cantores que dirigían el servicio en el
Templo de Dios. \bibverse{23} Tenían órdenes específicas del rey que les
había ordenado realizar un servicio diario.\footnote{\textbf{11:23}
  Probablemente se refiera al decreto emitido por el rey Ciro para que
  se recen oraciones por él y sus hijos. Véase Esdras 6:10.}

\bibverse{24} Petaías, hijo de Mesezabeel, descendiente de Zera, hijo de
Judá, era el consejero del rey en todo lo relacionado con los
israelitas.

\bibverse{25} En cuanto a las aldeas con sus campos cercanos algunos de
los habitantes de Judá vivían en Quiriat-arba, Dibón y Jekabzeel, y sus
asentamientos menores; \bibverse{26} en Jesúa, Molada y Bet-pelet;
\bibverse{27} en Hazar-sual, en Beerseba con sus asentamientos,
\bibverse{28} en Ziclag, en Mecona y sus asentamientos, \bibverse{29} en
En-rimón, en Zora, en Jarmut, \bibverse{30} Zanoa, Adulam y sus aldeas,
Laquis y sus campos, y Azeca y sus asentamientos. Vivían desde Beerseba
hasta el Valle de Hinom.

\bibverse{31} El pueblo de Benjamín, desde Geba, vivía en Micmas, Aija y
Bet-el y sus asentamientos, \bibverse{32} en Anatot, Nob, Ananías,
\bibverse{33} Hazor, Ramá, Gitaim, \bibverse{34} Hadid, Zeboim, Nebalat,
\bibverse{35} Lod, Ono y en el Valle de los Artesanos. \bibverse{36}
Algunas divisiones de los levitas de Judá también se establecieron en
Benjamín.

\hypertarget{section-11}{%
\section{12}\label{section-11}}

\bibverse{1} Esta es la lista de los sacerdotes y levitas que volvieron
con Zorobabel, hijo de Sealtiel, y Jesúa, el sumo sacerdote: Seraías,
Jeremías, Esdras, \bibverse{2} Amarías, Maluc, Hatús, \bibverse{3}
Secanías, Rehum, Meremot, \bibverse{4} Iddo, Gineto, Abías, \bibverse{5}
Miamín, Maadías, Bilga, \bibverse{6} Semaías, Joiarib, Jedaías,
\bibverse{7} Salú, Amoc, Hilcías y Jedaías. Estos eran los líderes de
los sacerdotes y sus parientes en el tiempo de Jesúa.

\bibverse{8} Los levitas eran Jesúa, Binúi, Cadmiel, Serebías, Judá y
Matanías, quien con sus compañeros levitas estaba a cargo de los cantos
de alabanza. \bibverse{9} Otros dos levitas, Bacbuquías y Uni, estaban
frente a ellos en el servicio.

\bibverse{10} El sumo sacerdote Jesúa era el padre de Joiacim, que era
el padre de Eliasib, que era el padre de Joiada, \bibverse{11} que era
el padre de Jonatán, que era el padre de Jadúa.

\bibverse{12} En tiempos de Joiacim, estos eran los jefes de familia de
los sacerdotes de la familia de Seraías, Meraías; de la de Jeremías,
Ananías; \bibverse{13} de la de Esdras, Mesulam; de la de Amarías,
Johanán; \bibverse{14} de la de Maluci, Jonatán; de la de Sebanías,
José; \bibverse{15} de la de Harim, Adna; de la de Meraiot, Helcai;
\bibverse{16} de la de Iddo, Zacarías; de la de Gineto, Mesulam
\bibverse{17} de Abías, Zicri; de Miiamín y de Maadías, Piltai;
\bibverse{18} de Bilga, Samúa; de Semaías, Jonatán; \bibverse{19} de
Joiarib, Matenai; de Jedaías, Uzi; \bibverse{20} de Salai, Calai; de
Amoc, Eber; \bibverse{21} de Hicíash, Hasabías; de Jedaías, Natanael.

\bibverse{22} Los jefes de familia de los levitas en la época de
Eliasib, Joiada, Johanán y Jaddua, además de los de los sacerdotes,
fueron registrados durante el reinado de Darío el Persa. \bibverse{23}
En cuanto a la genealogía de los hijos de Leví, los jefes de familia
hasta la época de Johanán, hijo de Eliasib, fueron inscritos en el Libro
de los Registros. \bibverse{24} Los jefes de los levitas eran Hasabías,
Serebías y Jesúa, hijo de Cadmiel, junto con sus compañeros levitas, que
estaban de pie frente a ellos, cada sección dando alabanzas y
agradecimientos y respondiendo unos a otros, según lo dispuesto por
David, el hombre de Dios. \bibverse{25} Se les unieron Matanías,
Bacbucías y Abdías.\footnote{\textbf{12:25} ``Matanías, Bacbucías y
  Abdías'': estos levitas son identificados como directores de coro en
  11:17, y no están incluidos en el grupo de guardianes. Abda es una
  varianteortográfica de Abdías.}

Mesulam, Talmón y Acub eran guardianes de los almacenes de las puertas.
\bibverse{26} Sirvieron en tiempos de Joiaquim, hijo de Jesúa, hijo de
Josadac, y en tiempos del gobernador Nehemías y del sacerdote y escriba
Esdras.

\bibverse{27} Para dedicar el muro de Jerusalén, se llamó a los levitas
de todos los lugares donde vivían para que vinieran a Jerusalén y
celebraran con alegría la dedicación con cantos de alabanza y
agradecimiento, acompañados de címbalos, arpas y liras. \bibverse{28}
También trajeron a los cantores de los alrededores de Jerusalén y de las
aldeas de los netofatitas, \bibverse{29} así como de Bet-gilgal y de la
zona de Geba y Azmavet, pues los cantores se habían construido aldeas
alrededor de Jerusalén. \bibverse{30} Una vez que los sacerdotes y los
levitas se purificaron, purificaron el pueblo, las puertas y la muralla.

\bibverse{31} Hice que los jefes de Judá subieran a la muralla y dispuse
dos grandes coros para dar gracias. Un grupo se dirigió a la derecha de
la muralla, hasta la Puerta del Desecho. \bibverse{32} Los siguieron
Oseas y la mitad de los dirigentes de Judá, \bibverse{33} con Azarías,
Esdras, Mesulam, \bibverse{34} Judá, Benjamín, Semaías y Jeremías,
\bibverse{35} así como algunos de los sacerdotes con trompetas, y
Zacarías, hijo de Jonatán, hijo de Semaías hijo de Matanías, hijo de
Micaías, hijo de Zaccur, hijo de Asaf, \bibverse{36} y sus compañeros
sacerdotes, Semaías, Azarel, Milalai, Gilalai, Maai, Netanel, Judá y
Hanani, junto con los instrumentos musicales, tal como lo exigió David,
el hombre de Dios. El escriba Esdras los dirigía. \bibverse{37} En la
Puerta de la Fuente se dirigieron hacia arriba por la escalera de la
ciudad de David, donde la muralla sube, por encima de la casa de David,
y hacia la Puerta del Agua, al este.

\bibverse{38} El otro grupo del coro fue en dirección contraria. Los
seguí, junto con la mitad del pueblo, por la muralla, pasando por la
Torre de los Hornos, hasta la Muralla Ancha, \bibverse{39} por la Puerta
de Efraín, la Puerta de Jesana, la Puerta del Pescado, la Torre de
Hananel y la Torre de los Cien, hasta la Puerta de las Ovejas,
deteniéndose en la Puerta de la Guardia.

\bibverse{40} Los dos coros de acción de gracias ocuparon entonces su
lugar en el Templo de Dios. Yo seguí con el grupo de líderes que me
acompañaba, \bibverse{41} junto con los sacerdotes que tocaban sus
trompetas: Eliaquim, Maasías, Miniamin, Micaías, Elioenai, Zacarías y
Ananías. \bibverse{42} Luego vinieron los cantores\footnote{\textbf{12:42}
  ``Cantantes'': añadido para mayor claridad. Ya han sido presentados en
  los versos 25 y 36 como acompañantes de Esdras como cantantes/músicos.}Maasías,
Semaías, Eleazar, Uzzi, Johanán, Malquías, Elam y Ezer, y los coros
cantaron, dirigidos por Izrahías.

\bibverse{43} Ese día se ofrecieron muchos sacrificios, celebrando que
Dios les había traído tanta felicidad, una felicidad tremenda. Las
mujeres y los niños también celebraron, y los sonidos de alegría en
Jerusalén se podían escuchar a lo lejos.

\bibverse{44} Ese mismo día se pusieron hombres a cargo de los almacenes
que contenían las ofrendas, la primera parte de las cosechas y los
diezmos. Lo que la Ley asignaba a los sacerdotes y a los levitas se
llevaba a estos almacenes desde los campos de los alrededores de las
ciudades, porque todo el pueblo de Judá se alegraba por el servicio de
los sacerdotes y de los levitas. \bibverse{45} Ellos eran responsables
del culto a su Dios y del servicio de purificación, junto con los
cantores y los porteros, siguiendo las instrucciones de David y de su
hijo Salomón. \bibverse{46} Porque hace mucho tiempo, en tiempos de
David y de Asaf, se habían designado directores para los cantores y para
los cantos de alabanza y de agradecimiento a Dios. \bibverse{47} Así que
en el tiempo de Zorobabel y de Nehemías, todos en Israel proveían las
dietas para los cantores y los porteros. También se aseguraban de
proveer para los otros levitas, y los levitas daban una parte de esto a
los descendientes de Aarón.

\hypertarget{section-12}{%
\section{13}\label{section-12}}

\bibverse{1} Un día,\footnote{\textbf{13:1} ``Un día'': la fecha no es
  específica, y ciertamente no es la misma que la del capítulo anterior.
  En el versículo 6, Nehemías señala que estaba ausente de Jerusalén en
  ese momento.}cuando se leía el Libro de Moisés al pueblo, encontraron
la sección en la que estaba escrito que no se debía permitir la entrada
de ningún amonita o moabita a la asamblea de Dios, \bibverse{2} porque
no habían traído comida ni agua cuando se encontraron con los
israelitas, sino que habían contratado a Balaam para que les echara una
maldición, aunque nuestro Dios convirtió esa maldición en una bendición.

\bibverse{3} Cuando el pueblo se enteró de esta ley, separó de Israel a
todos los que tenían ascendencia extranjera.

\bibverse{4} Antes de todo esto, el sacerdote Eliasib, que estaba
emparentado con Tobías,\footnote{\textbf{13:4} Tobías era un amonita:
  2:10.}había sido puesto a cargo de los almacenes del Templo de nuestro
Dios. \bibverse{5} Había puesto a disposición de Tobías una gran sala
que antes se utilizaba para almacenar las ofrendas de grano, el incienso
y los objetos del Templo, así como los diezmos de grano, vino nuevo y
aceite de oliva asignados a los levitas, cantores y porteros, además de
las ofrendas para los sacerdotes.

\bibverse{6} Cuando todo esto sucedió yo no estaba en Jerusalén porque
había regresado con el rey Artajerjes de Babilonia en el año treinta y
dos de su reinado. Algún tiempo después pedí permiso al rey para volver.
\bibverse{7} Cuando llegué de nuevo a Jerusalén, descubrí lo terrible
que había hecho Eliasib al proporcionarle a Tobías una habitación en el
patio del Templo de Dios. \bibverse{8} Me sentí sumamente molesto, y fui
a tirar todo lo que había en la habitación de Tobías. \bibverse{9}
Ordené que se purificaran las habitaciones, y volví a colocar los
objetos del Templo, las ofrendas de grano y el incienso.

\bibverse{10} También me enteré de que no se estaban suministrando las
asignaciones de alimentos para los levitas, por lo que los levitas
habían regresado a cuidar sus campos, junto con los cantantes que
dirigían los servicios de adoración. \bibverse{11} Fui y me enfrenté a
los dirigentes, preguntándoles: ``¿Por qué se descuida el Templo de
Dios?''. Llamé a los levitas\footnote{\textbf{13:11} ``Levitas'':
  implícito.}y se aseguró de que cumplieran con sus responsabilidades.
\bibverse{12} Todos los habitantes de Judá trajeron entonces los diezmos
de grano, vino nuevo y aceite de oliva a los almacenes.

\bibverse{13} Puse al sacerdote Selemías, al escriba Sadoc y a Pedaías,
uno de los levitas, a cargo de los almacenes, con Hanán, hijo de Zacur,
hijo de Matanías, para que los ayudaran, porque eran considerados
personas honestas. Su responsabilidad era distribuir las asignaciones a
sus compañeros levitas.

\bibverse{14} Dios mío, por favor, acuérdate de mí por esto. Por favor,
no olvides las buenas acciones que he hecho por el Templo de mi Dios y
sus servicios.

\bibverse{15} Por aquel entonces me di cuenta de que la gente pisaba el
lagar en sábado. Vi que otros recogían grano y lo cargaban en burros,
junto con vino, uvas, higos y toda clase de cargas, y lo llevaban todo a
Jerusalén en día de sábado. \bibverse{16} Yo los reprendí por vender sus
productos en ese día. \bibverse{17} Algunas personas de Tiro que vivían
en Jerusalén traían pescado y toda clase de cosas y las vendían en
sábado al pueblo de Judá en Jerusalén.

\bibverse{18} Me enfrenté a los dirigentes judíos y les pregunté: ``¿Por
qué están haciendo algo tan malo? ¡Ustedes están violando el día de
reposo! ¿No fue esto lo que hicieron vuestros antepasados, haciendo caer
a nuestro Dios sobre nosotros, causándonos a nosotros y a esta ciudad
tales desastres? ¡Ahora nos traes aún más problemas al violar el
sábado!''

\bibverse{19} Así que ordené que las puertas de Jerusalén se cerraran al
atardecer del día anterior al sábado, y que no se abrieran hasta después
de terminado el sábado. Asigné a algunos de mis hombres para que
vigilaran las puertas y se aseguraran de que no se introdujera ninguna
mercancía en el día de reposo.

\bibverse{20} Un par de veces los comerciantes y vendedores de toda
clase de mercancías pasaron la noche fuera de Jerusalén. \bibverse{21}
Yo les advertí diciendo: ``¿Por qué pasan la noche junto a la muralla?
Si vuelven a hacer eso, haré que los arresten''. Después de eso no
volvieron a venir en sábado.

\bibverse{22} Entonces les dije a los levitas que se purificaran y
vinieran a vigilar las puertas para santificar el día de reposo. Dios
mío, por favor, acuérdate también de mí por haber hecho esto, y sé
misericordioso conmigo a causa de tu amor digno de confianza.

\bibverse{23} Por esa misma época me di cuenta de que algunos judíos se
habían casado con mujeres de Asdod, Moab y Amón. \bibverse{24} La mitad
de sus hijos sólo sabían hablar la lengua de Asdod o la de otro pueblo,
y no sabían hablar la lengua de Judá. \bibverse{25} Así que me enfrenté
a ellos y les dije que estaban malditos. A algunos los golpeé y les
arranqué el pelo. Luego les hice prestar un juramento ante Dios,
diciendo: ``No deben permitir que sus hijas se casen con sus hijos, ni
permitir que sus hijos -- o ustedes mismos -- secasen con sus hijas.
\bibverse{26} ¿No fueron matrimonios como estos los que hicieron pecar
al rey Salomón de Israel? No hubo en ninguna nación un rey como él. Dios
lo amaba y lo hizo rey de todo Israel, pero incluso a él lo hicieron
pecar las mujeres extranjeras. \bibverse{27} ¿Acaso tenemos que oír que
tú cometes este terrible pecado, que le eres infiel a nuestro Dios
casándote con mujeres extranjeras?'' \bibverse{28} Incluso uno de los
hijos de Joiada, hijo del sumo sacerdote Eliasib, se había convertido en
yerno de Sanbalat el horonita. Así que lo expulsé.\footnote{\textbf{13:28}
  Literalmente, ``Lo alejé de mí.''Esto probablemente significa que fue
  exiliado. Haberse casado con una hija de Sanbalat, uno de los enemigos
  más importantes de Nehemías, debe haber sido un gran insulto para
  Nehemías.}

\bibverse{29} Dios mío, acuérdate de ellos y de lo que hicieron,
violando el sacerdocio y el acuerdo solemne de los sacerdotes y levitas.
\bibverse{30} Los purifiqué de todo lo ajeno, y me aseguré de que los
sacerdotes y los levitas cumplieran con sus responsabilidades asignadas.
\bibverse{31} También dispuse que se suministrara madera para el altar
en los tiempos especificados, y que se donara la primera parte de los
productos.

Dios mío, acuérdate de mí favorablemente.
