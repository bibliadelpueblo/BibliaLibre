\hypertarget{section}{%
\section{1}\label{section}}

\bibverse{1} Adán, Set, Enós,\footnote{\textbf{1:1} El libro comienza
  con la lista de nombres que puede parecer extraña para un lector
  moderno, pero al brindar esta línea genealógica el escritor de
  Crónicas está resumiendo la historia. En lugar de intentar
  proporcionar información sobre quiénes fueron todos estos individuos,
  se recomienda que la información relevante se encuentre en los libros
  históricos de la Biblia desde el Génesis en adelante.} \bibverse{2}
Quenán, Malalel, Jared, \bibverse{3} Enoc, Matusalén, Lamec, Noé.

\bibverse{4} Los hijos de Noé:\footnote{\textbf{1:4} Tomado de la
  Septuaginta: Esta línea está ausente en el texto hebreo.} Sem, Cam, y
Jafet.

\bibverse{5} Los hijos\footnote{\textbf{1:5} Como se ha señalado en
  otras partes, ``hijos'' puede significar ``descendientes.''} de Jafet:
Gómer, Magog, Madai, Javan, Tubal, Mésec, and Tirás.

\bibverse{6} Los hijos de Gomer: Asquenaz, Rifat,\footnote{\textbf{1:6}
  O Difat.} y Togarma.

\bibverse{7} Los hijos de Javán: Elisá, Tarsis, Quitín, Rodanín.

\bibverse{8} Los hijos de Cam: Cus,\footnote{\textbf{1:8} O
  ``Sudán/Etiopía.''} Mizrayin,\footnote{\textbf{1:8} O ``Egipto.''}
Fut, y Canaán.

\bibverse{9} Los hijos de Cus: Seba, Javilá, Sabta, Ragama y Sabteca.
Los hijos de Ragama: Sabá y Dedán.

\bibverse{10} Cus fue el padre de Nimrod, que se convirtió en el primer
tirano del mundo.

\bibverse{11} Mizrayin fue el padre de los ludeos, anameos, leabitas,
naftuitas, \bibverse{12} patruseos, caslujitas y los caftoritas (quienes
eran antepasados de los filisteos).

\bibverse{13} Canaán fue el padre Sidón, su primogénito, y de los
hititas, \bibverse{14} jebuseos, amorreos, gergeseos, \bibverse{15}
heveos, araceos, sineos, \bibverse{16} arvadeos, zemareos y jamatitas.

\bibverse{17} Los hijos de Sem: Elam, Asur, Arfaxad, Lud y Aram. Los
hijos de Aram:\footnote{\textbf{1:17} Algunos manuscritos de la
  Septuaginta: esta línea está ausente en la mayoría de los manuscritos
  hebreos. Véase Génesis 10:23.} Uz, Hul, Guéter, y Mésec.

\bibverse{18} Arfaxad fue el padre de Selá, y Selá el padre de Éber.

\bibverse{19} Éber tuvo dos hijos. Uno se llamaba Peleg,\footnote{\textbf{1:19}
  La palabra significa ``dividido.''} porque en su tiempo la tierra fue
dividida; el nombre de su hermano fue Joctán.

\bibverse{20} Joctán fue el padre de Almodad, Sélef, Jazar Mávet, Yeraj,
\bibverse{21} Adoram, Uzal, Diclá, \bibverse{22} Obal,\footnote{\textbf{1:22}
  La mayoría de los manuscritos lo llaman Ebal, pero véase Génesis
  10:28.} Abimael, Sabá, \bibverse{23} Ofir, Javilá y Jobab. Todos estos
fueron hijos de Joctán.

\bibverse{24} Sem, Arfaxad,\footnote{\textbf{1:24} Algunos manuscritos
  de la Septuaginta añaden aquí ``Cainán''.} Selá, \bibverse{25} Éber,
Peleg, Reú, \bibverse{26} Serug, Najor, Téraj, \bibverse{27} y Abram
(también llamado Abrahán).

\bibverse{28} Los hijos de Abrahán: Isaac e Ismael.

\bibverse{29} Estos fueron sus descendientes: Nebayot, quien fue el hijo
primogénito de Ismael, Cedar, Adbeel, Mibsam, 30 Mismá, Dumá, Masá,
Hadad, Temá, \bibverse{31} Jetur, Nafis y Cedema. Estos fueron los hijos
de Ismael.

\bibverse{32} Los hijos que le nacieron a Cetura, la concubina de
Abrahán. Ella dio a luz a: Zimrán, Jocsán, Medán, Madián, Isbac y Súah.
Los hijos de Jocsán: Sabá y Dedán.

\bibverse{33} Los hijos de Madián: Efá, Éfer, Janoc, Abidá y Eldá. Todos
ellos fueron descendientes de Cetura.

\bibverse{34} Abrahán fue el padre de Isaac. Los hijos de Isaac fueron
Esaú e Israel.

\bibverse{35} Los hijos de Esaú: Elifaz, Reuel, Jeús, Jalán y Coré.

\bibverse{36} Los hijos de Elifaz: Temán, Omar, Zefo, \footnote{\textbf{1:36}
  La mayoría de los manuscritos hebreos tienen ``Zefi'', pero véase
  Génesis 36:11.} Gatán y Quenaz; además Amalec por medio de
Timná.\footnote{\textbf{1:36} Según algunos manuscritos de la
  Septuaginta, Timna era la concubina de Elifaz (véase Génesis 36:12).}

\bibverse{37} Los hijos de Reuel: Najat, Zera, Sama y Mizá.

\bibverse{38} Los hijos de Seír: Lotán, Sobal, Zibeón, Aná, Disón, Ezer
y Disán.

\bibverse{39} Los hijos de Lotán: Horí y Homán. La hermana de Lotán era
Timná.

\bibverse{40} Los hijos de Sobal: Alván,\footnote{\textbf{1:40} En la
  mayoría de los manuscritos hebreos dice ``Alian'', pero algunos
  manuscritos hebreos y de la Septuaginta lo presentan como ``Alván''.
  Véase Génesis 36:23.} Manajat, Ebal, Sefó y Onam. Los hijos de Zibeón:
Aja y Aná.

\bibverse{41} El hijo de Aná fue Disón. Los hijos de Dishón fueron
Hemdán,\footnote{\textbf{1:41} En la mayoría de los manuscritos hebreos
  dice ``Hamran'', pero algunos manuscritos hebreos y de la Septuaginta
  dicen ``Hemdán''. Véase Génesis 36:26.} Esbán, Itrán y Querán.

\bibverse{42} Los hijos de Ezer: Bilán, Zaván y Acán.\footnote{\textbf{1:42}
  En la mayoría de los manuscritos hebreos aparece como ``Zaván'' o
  ``Jacán'', pero algunos manuscritos hebreos y de la Septuaginta lo
  tienen como ``Acán''. Véase Génesis 36:27.} Los hijos de
Disán:\footnote{\textbf{1:42} O ``Disón.''} Uz y Arán.

\bibverse{43} Estos fueron los reyes que reinaron sobre Edom antes de
que cualquier rey israelita reinara sobre ellos: Bela hijo de Beor, cuya
ciudad se llamaba Dinaba. \bibverse{44} Cuando murió Bela, Jobab hijo de
Zera, proveniente de Bosra, asumió el reinado. \bibverse{45} Tras la
muerte de Jobab, Husam asumió el reinado entonces, y era proveniente de
la tierra de los Temanitas.

\bibverse{46} Cuando murió Husam, Hadad, hijo de Bedad, asumió el
reinado. Él fue quien derrotó a Madián en el país de Moab. El nombre de
su ciudad era Avit. \bibverse{47} Cuando murió Hadad, Samá, de Masreca,
asumió el reinado. \bibverse{48} Cuando murió Samá, Saúl, proveniente de
Rehobot del río\footnote{\textbf{1:48} Probablemente el Río Eufrates.
  Véase Génesis 10:11.} asumió el reinado. \bibverse{49} Cuando murió
Saúl, Baal-Hanán, hijo de Acbor, asumió el reinado. \bibverse{50} Cuando
Baal-Hanán murió, Hadad reinó en su lugar. El nombre de su ciudad era
Pau.\footnote{\textbf{1:50} En la mayoría de los manuscritos hebreos
  ``Pai'', pero algunos manuscritos hebreos y de la Septuaginta tienen
  ``Pau''. Véase Génesis 36:39.} El nombre de su esposa era Mehetabel,
hija de Matred, nieta de Me-Zahab. \bibverse{51} Entonces murió Hadad.
Los jefes de Edom\footnote{\textbf{1:51} La lista de nombres cambia de
  reyes a jefes, ya que después de esta época Edom estaba bajo el
  dominio de Israel y por lo tanto no tenía su propio rey.} eran: Timná,
Alva, Jetet, \bibverse{52} Aholibama, Ela, Pinón, \bibverse{53} Quenaz,
Temán, Mibzar, \bibverse{54} Magdiel, e Iram. Estos eran los jefes de
Edom.

\hypertarget{section-1}{%
\section{2}\label{section-1}}

\bibverse{1} Estos fueron Los hijos de Israel: Rubén, Simeón, Leví,
Judá, Isacar, Zabulón, \bibverse{2} Dan, José, Benjamín, Neftalí, Gad y
Aser. \bibverse{3} Los hijos de Judá: Er, Onán y Selá: a estos tres los
dio a luz la hija de Súa, una mujer cananea. Er, el primogénito de Judá,
era malvado ante los ojos del Señor, por lo que le quitó la vida.
\bibverse{4} Tamar era la nuera de Judá, y le dio a luz a Fares y a
Zera. Judá tuvo un total de cinco hijos.

\bibverse{5} Los hijos de Fares: Hezrón y Hamul.

\bibverse{6} Los hijos de Zera: Zimri, Etán, Hemán, Calcol y
Darda\footnote{\textbf{2:6} En la mayoría de los manuscritos hebreos
  ``Dara'', pero algunos manuscritos de la Septuaginta tienen ``Darda''.
  Véase 1 Reyes 4:31.} para un total de cinco.

\bibverse{7} El hijo de Carmi: Acar,\footnote{\textbf{2:7} En el libro
  de Josué se le llama Acán. Véase Josué 7.} que le causó problemas a
Israel al ser infiel y tomar lo que estaba consagrado para el Señor.

8 El hijo de Etán: Azarías.

\bibverse{9} Los hijos que le nacieron a Hezrón: Jerameel, Ram y
Caleb.\footnote{\textbf{2:9} Literalmente, ``Quelubai.''}

\bibverse{10} Ram fue el padre de Aminadab, y Aminadab fue el padre de
Naasón, un líder de los descendientes de Judá. \bibverse{11} Naasón fue
el padre de Salmón, \footnote{\textbf{2:11} Lectura de la Septuaginta.
  El hebreo es ``Salma'', pero véase Rut 4:21.} Salmón fue el padre de
Booz, \bibverse{12} Booz fue el padre de Obed, y Obed fue el padre de
Isaí.

\bibverse{13} Isaí fue el padre de su hijo primogénito Eliab; el segundo
hijo fue Abinadab, el tercero Simea, \bibverse{14} el cuarto Netanel, el
quinto Raddai, \bibverse{15} el sexto Ozem y el séptimo David.
\bibverse{16} Sus hermanas fueron Zeruiah y Abigail. Los hijos de Sarvia
fueron Abisai, Joab y Asael, tres en total. \bibverse{17} Abigail dio a
luz a Amasa, y el padre de Amasa fue Jeter el ismaelita.

\bibverse{18} Caleb hijo de Hezrón tuvo hijos de su esposa Azuba, y
también de Jeriot. Estos fueron sus hijos Jeser, Sobab y Ardón.
\bibverse{19} Cuando Azuba murió, Caleb tomó a Efrat\footnote{\textbf{2:19}
  También llamada Efrata en 2:50, 4:4.} para que fuera su esposa, y ella
le dio a luz a Hur. \bibverse{20} Hur fue el padre de Uri, y Uri fue el
padre de Bezalel.

\bibverse{21} Más tarde, Hezrón se acostó con la hija de Maquir, padre
de Galaad, con quien se casó cuando tenía sesenta años, y ella le dio a
luz a Segub. \bibverse{22} Segub fue el padre de Jair, que tenía
veintitrés ciudades en Galaad. \bibverse{23} Pero Gesur y Aram les
quitaron las ciudades de Havvoth Jair, junto con Kenat y sus ciudades,
para un total de sesenta ciudades. Todos ellos eran descendientes de
Maquir, el padre de Galaad.

\bibverse{24} Después de la muerte de Hezrón en Caleb Efrata, su esposa
Abías dio a luz a Asur, padre de Tecoa.

\bibverse{25} Los hijos de Jerajmeel, primogénito de Hezrón: Ram
(primogénito), Bunah, Oren, Ozem y Ahías. \bibverse{26} Jerajmeel tuvo
otra esposa llamada Atara. Ella fue la madre de Onam.

\bibverse{27} Los hijos de Ram el primogénito de Jerajmeel: Maaz, Jamín
y Equer.

\bibverse{28} Los hijos de Onam: Samaiy Jada.

Los hijos de Samai: Nadab y Abisur. \bibverse{29} La mujer de Abisur se
llamaba Abihail, y dio a luz a Ahbán y Molid.

\bibverse{30} Los hijos de Nadab: Seled y Appaim. Seled murió sin tener
hijos.

\bibverse{31} El hijo de Apaim: Isi, el padre de Sesán. Sesán fue el
padre de Ahlai.

\bibverse{32} Los hijos de Jada, el hermano de Samai: Jeter y Jonathan.
Jeter murió sin tener hijos.

\bibverse{33} Los hijos de Jonatán: Pelet y Zaza. Estos son todos los
descendientes de Jerajmeel.

\bibverse{34} Sesán no tenía hijos, sino que sólo tenía hijas, pero
tenía un siervo egipcio llamada Jarha. \bibverse{35} Así que Sesán dio
su hija en matrimonio a su siervo Jarha, y ella le dio a luz a Atai.

\bibverse{36} Atai fue el padre de Natán. Natán fue el padre de Zabad,
37 Zabad fue el padre de Eflal, Eflal fue el padre de Obed,
\bibverse{38} Obed fue el padre de Jehú, Jehú fue el padre de Azarías,
\bibverse{39} Azarías fue el padre de Heles, Heles fue el padre de
Eleasá, \bibverse{40} Eleasá fue el padre de Sismai, Sismai fue el padre
de Salum, \bibverse{41} Salum fue el padre de Jecamías, y Jecamías fue
el padre de Elisama.

\bibverse{42} Los hijos de Caleb, hermano de Jerameel: Mesha, su
primogénito, que fue el padre de Zif, y su hijo Maresa, que fue el padre
de Hebrón.

\bibverse{43} Los hijos de Hebrón: Coré, Tapuá, Requem y Sema.
\bibverse{44} Sema fue el padre de Raham, y Raham el padre de Jorcoam.
Requem fue el padre de Samai. \bibverse{45} El hijo de Samai fue Maón, y
Maón fue el padre de Bet Sur.

\bibverse{46} Efá, concubina de Caleb, fue la madre de Harán, Mosa y
Gazez. Harán fue el padre de Gazez.

\bibverse{47} Los hijos de Jahdai: Regem, Jotam, Gesam, Pelet, Efá y
Saaf.

\bibverse{48} Maaca, concubina de Caleb, fue madre de Seber y de
Tirhana. \bibverse{49} También fue madre de Saaf, padre de Madmaná, y de
Seva, padre de Macbena y Gibea. La hija de Caleb fue Acsa. \bibverse{50}
Estos fueron todos los descendientes de Caleb.

Los hijos de Hur, primogénito de Efrata: Sobal, padre de Quiriat Jearim,
\bibverse{51} Salma, padre de Belén, y Haref, padre de Bet Gader.

\bibverse{52} Los descendientes de Sobal, padre de Quiriat Jearim,
fueron: Haroe, la mitad de los manahetitas, \bibverse{53} y las familias
de Quiriat Jearim: los itritas, los futitas, los sumatitas y los
misraítas. De ellos descendieron los zoratitas y los estaolitas.

\bibverse{54} Los descendientes de Salma: Belén, los netofatitas, Atrot
Bet Joab, la mitad de los manaítas, los zoritas, \bibverse{55} y las
familias de escribas que vivían en Jabes: los tirateos, los simeateos y
los sucateos. Estos fueron los ceneos que descendían de Hamat, el padre
de la casa de Recab.

\hypertarget{section-2}{%
\section{3}\label{section-2}}

\bibverse{1} Estos fueron los hijos de David que le nacieron en Hebrón:
El primogénito fue Amnón, cuya madre fue Ahinoam de Jezreel. El segundo
fue Daniel, cuya madre fue Abigail de Carmelo. \bibverse{2} El tercero
fue Absalón, cuya madre fue Maaca, hija de Talmai, rey de Gesur. El
cuarto fue Adonías, cuya madre fue Haguit. \bibverse{3} El quinto fue
Sefatías, cuya madre fue Abital. El sexto fue Itream, cuya madre fue
Egla, esposa de David.

\bibverse{4} Esos fueron los seis hijos que le nacieron a David en
Hebrón, donde reinó siete años y seis meses. David reinó en Jerusalén
treinta y tres años más, \bibverse{5} y estos fueron los hijos que le
nacieron allí:

Samúa,\footnote{\textbf{3:5} En realidad Simea, una escritura diferente
  de Samúa.} Sobab, Natán y Salomón. La madre de ellos fue
Betsabé,\footnote{\textbf{3:5} En realidad Batsúa, una escritura
  diferente de Betsabé.} hija de Ammiel. \bibverse{6} Además estaban
también Ibhar, Elisúa,\footnote{\textbf{3:6} En realidad Elisama, una
  escritura diferente de Elisúa.} Elifelet, \bibverse{7} Noga, Nefeg,
Jafía, \bibverse{8} Elisama, Eliada y Elifelet, un total de nueve.
\bibverse{9} Todos estos fueron los hijos de David, aparte de sus hijos
de sus concubinas. Su hermana era Tamar.

\bibverse{10} El linaje masculino\footnote{\textbf{3:10} ``Linaje
  masculino'': se utiliza este término en lugar de decir repetidamente
  ``su hijo.''} desde Salomón fue: Roboam, Abías, Asa, \bibverse{11}
Joram,\footnote{\textbf{3:11} De hecho dice Joram, una escritura
  diferente de Jehoram.} Ocozías, Joás, \bibverse{12} Amasías, Azarías,
Jotam, \bibverse{13} Acaz, Ezequías, Manasés, \bibverse{14} Amón,
Josías.

\bibverse{15} Los hijos de Josías: Johanán (primogénito), Joaquín (el
segundo), Sedequías (el tercero), y Salum (el cuarto).

\bibverse{16} Los hijos de Joacim: Joaquín\footnote{\textbf{3:16} De
  hecho dice Jeconías, una ortografía diferente de Joaquín.} y Sedecías.

\bibverse{17} Los hijos de Joaquín que fueron llevados al cautiverio:
Sealtiel, \bibverse{18} Malquiram, Pedaías, Senaza, Jecamías, Hosama y
Nedabías.

\bibverse{19} Los hijos de Pedaías: Zorobabel y Simei.

Los hijos de Zorobabel: Mesulam y Hananías. Su hermana era Selomit.
\bibverse{20} Otros cinco hijos fueron: Hasuba, Ohel, Berequías,
Hasadías y Jushab-Hesed.

\bibverse{21} Los hijos de Hananías: Pelatías y Jesaías, y Los hijos de
Refaías, Los hijos de Arnán, Los hijos de Abdías, y Los hijos de
Secanías.\footnote{\textbf{3:21} El texto tiene dificultades de
  interpretación.}

\bibverse{22} Los hijos de Secanías: Semaías y sus hijos: Hatús, Igal,
Barías, Nearías y Safat. En total eran seis.

\bibverse{23} Los hijos de Nearías: Elioenai, Ezequías y Azricam. Eran
tres en total.

\bibverse{24} Los hijos de Elioenai: Hodavías, Eliasib, Pelaías, Acub,
Johanán, Dalaías, y Anani. En total eran un total de siete.

\hypertarget{section-3}{%
\section{4}\label{section-3}}

\bibverse{1} Los hijos de Judá fueron Fares, Hezrón, Carmi, Hur y Sobal.
\bibverse{2} Reaía, hijo de Sobal, fue el padre de Jahath. Jahat fue el
padre de Ahumai y Lahad. Estas fueron las familias de los zoratitas.

\bibverse{3} Estos fueron los hijos\footnote{\textbf{4:3} ``Hijos'': el
  texto hebreo dice ``padre'', pero algunos manuscritos de la
  Septuaginta y la Vulgata dicen ``hijos''.} de Etam: Jezreel, Isma e
Ibdas. Su hermana se llamaba Haze-lelponi. \bibverse{4} Penuel fue el
padre de Gedor, y Ezer fue el padre de Husa. Estos fueron los
descendientes de Hur, primogénito de Efrata y padre\footnote{\textbf{4:4}
  ``Padre'': probablemente en el sentido de ``fundador.''} de Belén.

5 Asur fue el padre de Tecoa y tuvo dos esposas, Helá y Naara.

\bibverse{6} Naara fue la madre de Ahuzam, Hefer, Temeni y Ahastari.
Estos fueron los hijos de Naara.

\bibverse{7} Los hijos de Hela: Zeret, Zohar, Etnán, \bibverse{8} y Cos,
que fue el padre de Anub y Zobeba, y de las familias de Aharhel, hijo de
Harum.

\bibverse{9} Jabes fue más fiel a Dios\footnote{\textbf{4:9} ``Más fiel
  a Dios'': Literalmente, ``más honorable,'' pero esto no conlleva el
  significado de una mejor relación con Dios.} que sus hermanos. Su
madre le había puesto el nombre de Jabes, diciendo: ``Lo di a luz con
dolor''. \bibverse{10} Jabes suplicó al Dios de Israel: ``¡Por favor,
bendíceme y amplía mis fronteras!\footnote{\textbf{4:10} ``Amplía mis
  fronteras'': o, ``extiende mi territorio.'' Aunque esto puede ser
  visto como una simple petición de mayor propiedad de la tierra, es
  quizás mejor entender esta petición de que Dios expanda todo lo que
  Jabes tenía, incluyendo los aspectos espirituales.} Acompáñame y
mantenme a salvo de cualquier daño para que no tenga dolor.''\footnote{\textbf{4:10}
  ``Dolor'': parte de la oración es el deseo de que, a pesar del nombre
  que le dio su madre, no se le maldiga para que sufra dolor.} Y Dios le
dio lo que pidió.

\bibverse{11} Quelub, hermano de Súa, fue el padre de Mehir, quien a su
vez fue el padre de Estón. 12 Estón fue el padre de Bet-Rafa, Paseah y
Tehina, el padre\footnote{\textbf{4:11} ``Padre'': probablemente en el
  sentido de ``fundador''. Ir-Nahas significa ``ciudad de la serpiente''}
de Ir-Nahas. Estos fueron los hombres de Reca.\footnote{\textbf{4:11}
  ``Reca.'' En algunos manuscritos se lee ``Recab'', en cuyo caso se
  referiría a los mencionados en 2:55.}

\bibverse{13} Los hijos de Kenaz: Otoniel y Seraías.

Los hijos de Otoniel: Hatat y Meonothai.\footnote{\textbf{4:13}
  ``Meonothai'': algunos manuscritos de la Septuaginta y la Vulgata. El
  texto hebreo actual no tiene la palabra, probablemente perdida porque
  ocurre como la primera palabra del siguiente verso.} \bibverse{14}
Meonothai fue el padre de Ofra. Seraías fue el padre de Joab, el
padre\footnote{\textbf{4:14} ``Padre'': probablemente en el sentido de
  ``fundador''. Gue-Jarasim significa ``valle de los artesanos''.} de
Gue-Harashim, llamado así porque allí vivían artesanos.

\bibverse{15} Los hijos de Caleb hijo de Jefone: Iru, Ela y Naam.

El hijo de Elah: Kenaz.

\bibverse{16} Los hijos de Jehalelel: Zif, Zifa, Tirías y Asarel.

\bibverse{17} Los hijos de Esdras: Jeter, Mered, Efer y Jalón. Una de
las esposas de Mered\footnote{\textbf{4:17} ``Mered'': Se asume por el
  versículo siguiente.} fue la madre de Miriam, Samai e Ishbah el padre
de Estemoa.\footnote{\textbf{4:17} ``Padre'': en el sentido de
  ``fundador'' de la ciudad de ese nombre.} \bibverse{18} (Otra esposa
que vino de Judá fue la madre de Jered el padre de Gedor, Heber el padre
de Soco, y Jecuthiel el padre de Zanoa.\footnote{\textbf{4:18}
  ``Padre'': cada uno se refiere al ``fundador'' de las respectivas
  ciudades. Véase Josué 15.}) Estos eran Los hijos de Bitia, la hija del
Faraón, con quien Mered se había casado.\footnote{\textbf{4:18} Es de
  suponer que se refiere a los hijos mencionados en el verso anterior.}

\bibverse{19} Los hijos de la esposa de Hodías, hermana de Natán: un
hijo fue el padre de Keila la Garmita, y otro el padre de Estemoa la
Maacatea.

\bibverse{20} Los hijos de Simón: Amnón, Rinah, Ben-Hanan y Tilón.

Los hijos de Isi: Zohet y Ben-Zohet.

\bibverse{21} Los hijos de Selá hijo de Judá: Er, que fue el padre de
Leca; Laada, que fue el padre de Maresa; las familias de los
trabajadores del lino en Beth Asbea; \bibverse{22} Jaocim, los hombres
de Cozeba, y Joas y Saraf, que gobernaron sobre Moab y Jasubi-Lehem.
\bibverse{23} Eran alfareros, habitantes de Netaim y Gedera, que vivían
allí y trabajaban para el rey.

\bibverse{24} Los hijos de Simeón: Nemuel, Jamín, Jarib, Zera y Saúl.
\bibverse{25} Salum era hijo de Saúl, Mibsam su hijo y Mismá su hijo.

\bibverse{26} Los hijos de Mismá: Hamuel su hijo, Zacur su hijo y Simei
su hijo.

\bibverse{27} Simei tuvo dieciséis hijos y seis hijas, pero sus hermanos
no tuvieron muchos hijos, por lo que su tribu no fue tan numerosa como
la de Judá. \bibverse{28} Vivían en Beerseba, Molada, Hazar Sual,
\bibverse{29} Bilha, Ezem, Tolad, \bibverse{30} Betuel, Horma, Ziclag,
\bibverse{31} Bet Marcabot, Hazar Susim, Bet Birai y Saaraim. Estas
fueron sus ciudades hasta que David llegó a ser rey. \bibverse{32}
También vivían en Etam, Ain, Rimón, Toquén y Asán, un total de cinco
ciudades, \bibverse{33} junto con todas las aldeas de los alrededores
hasta Baal.\footnote{\textbf{4:33} Véase Josué 19:8.} Estos fueron los
lugares donde vivieron y registraron su genealogía.

\bibverse{34} Mesobab, Jamlec, Josá, hijo de Amasías, \bibverse{35}
Joel, Jehú, hijo de Josibías, hijo de Seraías, hijo de Asiel,
\bibverse{36} Elioenai, Jaacoba, Jesohaía, Asaías, Adiel, Jesimiel,
Benaía, \bibverse{37} y Ziza, hijo de Sifi, hijo de Alón, hijo de
Jedaiah, hijo de Shimri, hijo de Semaías.

\bibverse{38} Estos fueron los nombres de los jefes de sus familias,
cuyo linaje aumentó considerablemente. \bibverse{39} Llegaron hasta la
frontera de Gedor, en el lado oriental del valle, para buscar pastos
para sus rebaños. \bibverse{40} Allí encontraron buenos pastos, y la
zona era abierta, tranquila y apacible, pues los que vivían allí eran
descendientes de Cam.\footnote{\textbf{4:40} ``descendientes de Cam'':
  es decir, los antiguos habitantes cananeos.}

\bibverse{41} En la época de Ezequías, rey de Judá, los líderes
mencionados por su nombre vinieron y atacaron a estos descendientes de
Cam donde vivían, junto con los meunitas de allí y los destruyeron
totalmente, como está claro hasta el día de hoy. Luego se establecieron
allí, porque había pastizales para sus rebaños. \bibverse{42} Algunos de
estos simeonitas invadieron el monte de Seir: quinientos hombres
dirigidos por Pelatías, Nearías, Refaías y Uziel, los hijos de Isi.
\bibverse{43} Destruyeron al resto de los amalecitas que habían
escapado. Ellos han vivido allí hasta el día de hoy.

\hypertarget{section-4}{%
\section{5}\label{section-4}}

\bibverse{1} Los hijos de Rubén, el primogénito de Israel. (Aunque era
el primogénito, su primogenitura fue entregada a los hijos de José hijo
de Israel porque había profanado el lecho de su padre.\footnote{\textbf{5:1}
  Rubén se había acostado con Bilhá, la concubina de Jacob. Génesis
  35:22, Génesis 49:4.} Por eso Rubén no figura en la genealogía según
la primogenitura, \bibverse{2} y aunque Judá llegó a ser el más fuerte
de sus hermanos y de él salió un gobernante, la primogenitura le
pertenecía a José).

\bibverse{3} Los hijos de Rubén el primogénito de Israel: Hanoc, Falú,
Hezrón y Carmi.

\bibverse{4} Los hijos de Joel: Semaías su hijo, Gog su hijo, Simei su
hijo, \bibverse{5} Miqueas su hijo, Reaías su hijo, Baal su hijo,
\bibverse{6} y Beera su hijo, el que Tiglat-Pileser el rey de Asiria
llevó al exilio. Él (Beera) era un líder de los rubenitas.

\bibverse{7} Los parientes de Beera son, según sus registros
genealógicos: por familia Jeiel (el jefe de familia), Zacarías,
\bibverse{8} y Bela de Azaz, hijo de Sema, hijo de Joel. Ellos habitaban
en la zona que iba desde Aroer hasta Nebo y Baal Meón. \bibverse{9} Por
el lado oriental se extendieron por la tierra hasta el borde del
desierto que continúa hasta el río Éufrates, porque sus rebaños habían
crecido mucho en Galaad.

\bibverse{10} En los tiempos de Saúl fueron a la guerra contra los
agarenos, y los derrotaron. Se apoderaron de los lugares donde habían
vivido los agarenos en todas las regiones al este de Galaad.

\bibverse{11} Junto a ellos, los descendientes de Gad vivían en Basa,
hasta Salca. \bibverse{12} Joel (el jefe de familia), Safam (el
segundo), y Janai y Safat, en Basán.

\bibverse{13} Sus parientes, según la familia, eran: Miguel, Mesulán,
Sabá, Jorai, Jacán, Zía y Éber, siendo un total de siete.

\bibverse{14} Estos fueron Los hijos de Abihail, hijo de Huri, hijo de
Jaroa, hijo de Galaad, hijo de Miguel, hijo de Jesisai, hijo de Jahdo,
hijo de Buz.

\bibverse{15} Ahi hijo de Abdiel, hijo de Guni, era su jefe de familia.

\bibverse{16} Vivían en Galaad, en Basán y sus ciudades, y en los
pastizales de Sarón hasta sus fronteras. \bibverse{17} Todos ellos
fueron registrados en la genealogía durante el tiempo de Jotam, rey de
Judá, y de Jeroboam, rey de Israel.

\bibverse{18} La tribu de Rubén, la tribu de Gad y la media tribu de
Manasés tenían 44.760 guerreros fuertes y listos para la batalla,
capaces de usar escudos, espadas y arcos. \bibverse{19} Fueron a la
guerra contra los agarenos, los jetureos, los nafiseos y los nodabitas.
\bibverse{20} Recibieron ayuda en la lucha contra estos enemigos porque
invocaron a Dios durante las batallas. Así pudieron derrotar a los
agarenos y a todos los que estaban con ellos. Dios respondió a sus
oraciones porque confiaron en él. \bibverse{21} Capturaron el ganado de
sus enemigos: cincuenta mil camellos, doscientas cincuenta mil ovejas y
dos mil asnos. También capturaron a cien mil personas, \bibverse{22} y
muchas otras murieron porque la batalla era de Dios. Se apoderaron de la
tierra y vivieron allí hasta el exilio.

\bibverse{23} La media tribu de Manasés había crecido mucho. Vivían en
la tierra desde Basán hasta Baal Hermón, (también conocida como Senir y
Monte Hermón).

\bibverse{24} Estos eran los jefes de familia: Efer, Isi, Eliel, Azriel,
Jeremías, Hodavías y Jahdiel. Eran fuertes guerreros, hombres famosos,
jefes de sus familias. \bibverse{25} Pero fueron infieles al Dios de sus
antepasados. Se prostituyeron siguiendo a los dioses de los pueblos de
la tierra, los que Dios había destruido antes que ellos. \bibverse{26}
Así que el Dios de Israel animó a Pul, rey de Asiria, (también conocido
como Tiglat-Pileser, rey de Asiria), para que invadiera la tierra. Llevó
al exilio a los rubenitas, a los gaditas y a la media tribu de Manasés.
Los llevó a Halah, a Habor, a Hara y al río de Gozán, donde permanecen
hasta el día de hoy.

\hypertarget{section-5}{%
\section{6}\label{section-5}}

\bibverse{1} Los hijos de Levi: Gersón, Coat y Merari.

\bibverse{2} Los hijos de Coat: Amram, Izhar, Hebrón y Uziel.

\bibverse{3} Los hijos de Amram: Aarón, Moisés y Miriam.

Los hijos de Aarón: Nadab, Abiú, Eleazar e Itamar.

\bibverse{4} Eleazar fue el padre de Finehas. Finehas fue el padre de
Abisúa. \bibverse{5} Abisúa fue el padre de Buqui. Buqui fue el padre de
Uzi; \bibverse{6} Uzi fue el padre de Zeraías. Zeraías fue el padre de
Meraioth. \bibverse{7} Meraiot fue el padre de Amarías. Amarías fue el
padre de Ajitub. \bibverse{8} Ajitub fue el padre de Sadok. Sadok fue el
padre de Ahimaas. \bibverse{9} Ahimaas fue el padre de Azarías. Azarías
fue el padre de Johanán. \bibverse{10} Johanán fue el padre de Azarías
(quien sirvió como sacerdote cuando Salomón construyó el Templo en
Jerusalén). \bibverse{11} Azarías fue el padre de Amarías. Amarías fue
el padre de Ajitub. \bibverse{12} Ajitub fue el padre de Sadoc. Sadoc
fue el padre de Salum. \bibverse{13} Salum fue el padre de Hilcías.
Hilcías fue el padre de Azarías. \bibverse{14} Azarías fue el padre de
Seraías, y Seraías fue el padre de Josadac. \bibverse{15} Josadac fue
llevado al cautiverio cuando el Señor usó a Nabucodonosor para enviar a
Judá y a Jerusalén al exilio.

\bibverse{16} Los hijos de Leví: Gersón, Coat y Merari.

\bibverse{17} Estos son los nombres de los hijos de Gersón: Libni y
Simei.

\bibverse{18} Los hijos de Coat: Amram, Izhar, Hebrón y Uziel.

\bibverse{19} Los hijos de Merari: Mahli y Musi.

Estas son las familias de los levitas, que estaban ordenadas según sus
padres:

\bibverse{20} Los descendientes de Gersón: Libni su hijo, Jehat su hijo,
Zima su hijo, \bibverse{21} Joa su hijo, Iddo su hijo, Zera su hijo y
Jeatherai su hijo.

\bibverse{22} Los descendientes de Coat: Aminadab su hijo; Coré su hijo;
Asir su hijo, \bibverse{23} Elcana su hijo; Ebiasaf su hijo; Asir su
hijo; \bibverse{24} Tahat su hijo; Uriel su hijo; Uzías su hijo, y Saúl
su hijo.

\bibverse{25} Los descendientes de Elcana: Amasai, Ahimot, \bibverse{26}
Elcana su hijo; Zofai su hijo; Nahat su hijo; \bibverse{27} Eliab su
hijo, Jeroham su hijo; Elcana su hijo, y Samuel su hijo.\footnote{\textbf{6:27}
  ``Y Samuel, su hijo'': según algunos manuscritos de la Septuaginta. El
  texto hebreo omite estas palabras. Véase 1 Samuel 1:19-20 y 1 Crónicas
  6:33-34.}

\bibverse{28} Los hijos de Samuel: Joel\footnote{\textbf{6:28} ``Joel'':
  según algunos manuscritos de la Septuaginta. El texto hebreo omite
  esta palabra. Véase 1 Samuel 8:2 y 1 Crón. 6:33.} (primogénito) y
Abías (el segundo).

\bibverse{29} Los descendientes de Merari: Mahli, su hijo Libni, su hijo
Simei, su hijo Uza, \bibverse{30} su hijo Simea, su hijo Haguía y su
hijo Asaías.

\bibverse{31} Estos son los músicos que David designó para dirigir la
música en la casa del Señor una vez que el Arca fuera colocada allí.
\bibverse{32} Ellos dirigieron la música y el canto ante el Tabernáculo,
la Tienda de Reunión, hasta que Salomón construyó el Templo del Señor en
Jerusalén. Servían siguiendo el reglamento que se les había dado.

\bibverse{33} Estos son los hombres que servían, junto con sus hijos: De
los coatitas Hemán, el cantor, el hijo de Joel, el hijo de Samuel,
\bibverse{34} hijo de Elcana, hijo de Jeroham, hijo de Eliel, hijo de
Toa, \bibverse{35} hijo de Zuf, ehijo de Elcana, hijo de Mahat, hijo de
Amasai, \bibverse{36} hijo de Elcana, hijo de Joel, hijo de Azarías,
hijo de Sofonías, \bibverse{37} hijo de Tahat, hijo de Asir, hijo de
Ebiasaf, hijo de Coré, \bibverse{38} hijo de Izhar, hijo de Coat, hijo
de Leví, hijo de Israel. \bibverse{39} Asaf, pariente de Hemán, que
servía junto a él a la derecha: Asaf hijo de Berequías, hijo de Simea,
\bibverse{40} hijo de Miguel, hijo de Baasías, hijo de Malaquías,
\bibverse{41} hijo de Etni, hijo de Zera, hijo de Adaías, \bibverse{42}
hijo de Etán, hijo de Zima, hijo de Simei, \bibverse{43} hijo de Jahat,
hijo de Gersón, hijo de Leví. \bibverse{44} A la izquierda de Hemán
sirvieron los hijos de Merari: Etán hijo de Quisi, el hijo de Abdi,

hijo de Malluch, \bibverse{45} hijo de Hasabiah, hijo de Amasías, hijo
de Hilcías, \bibverse{46} hijo de Amzi, hijo de Bani, hijo de Semer,
\bibverse{47} hijo de Mahli, hijo de Musi, hijo de Merari, hijo de Levi.

\bibverse{48} Los demás levitas desempeñaban todas las demás funciones
en el Tabernáculo, la casa de Dios.

\bibverse{49} Sin embargo, eran Aarón y sus descendientes quienes daban
ofrendas en el altar de los holocaustos y en el altar del incienso y
hacían todo el trabajo en el Lugar Santísimo, haciendo la expiación por
Israel según todo lo que había ordenado Moisés, el siervo de Dios.

\bibverse{50} Los descendientes de Aarón fueron: Eleazar su hijo, Finees
su hijo, Abisúa su hijo, \bibverse{51} Buqui su hijo, Uzi su hijo,
Zeraías su hijo, \bibverse{52} Meraioth su hijo, Amariah su hijo, Ahitub
su hijo, 53 Zadok su hijo, y su hijo Ahimaas.

\bibverse{54} Estos fueron los lugares que se les dieron para vivir como
territorio asignado a los descendientes de Aarón, comenzando por el clan
coatita, porque el suyo fue el primer lote:

\bibverse{55} Recibieron Hebrón, en Judá, junto con los pastos que la
rodean. 56 Pero los campos y las aldeas cercanas a la ciudad fueron
entregados a Caleb hijo de Jefone.

\bibverse{57} Así, los descendientes de Aarón recibieron Hebrón, una
ciudad de refugio, Libna, Jatir, Estemoa, \bibverse{58} Hilén, Debir,
\bibverse{59} Asán, Juta\footnote{\textbf{6:59} Esta ciudad no aparece
  en esta lista, pero está incluida en Josué 21:16.} y Bet Semes, junto
con sus pastizales. \bibverse{60} De la tribu de Benjamín recibieron
Gabaón,\footnote{\textbf{6:60} Esta ciudad no aparece en esta lista,
  pero está incluida en Josué 21:17.} Geba, Alemeth y Anathoth, junto
con sus pastizales.

Tenían un total de trece ciudades entre sus familias.

\bibverse{61} Los demás descendientes de Coat recibieron por sorteo diez
ciudades de la media tribu de Manasés.

\bibverse{62} Los descendientes de Gersón, por familia, recibieron trece
ciudades de las tribus de Isacar, Aser y Neftalí, y de la media tribu de
Manasés en Basán.

\bibverse{63} Los descendientes de Merari, por familia, recibieron doce
ciudades de las tribus de Rubén, Gad y Zabulón.

\bibverse{64} El pueblo de Israel dio a los levitas estas ciudades y sus
pastizales. \bibverse{65} A las ciudades ya mencionadas les asignaron
por nombre las de las tribus de Judá, Simeón y Benjamín.

\bibverse{66} Algunas de las familias coatitas recibieron como
territorio ciudades de la tribu de Efraín. \bibverse{67} Se les dio
Siquem, una ciudad de refugio, en la región montañosa de Efraín,
Gezer,\footnote{\textbf{6:67} Aquí también se incluye Gezer como ciudad
  de refugio, pero véase Josué 21:21.} \bibverse{68} Jocmeán, Bet-Horon,
\bibverse{69} Aijalon y Gat Rimmon, junto con sus pastizales.

\bibverse{70} De la mitad de la tribu de Manasés, el pueblo de Israel
dio Aner y Bileam, junto con sus pastizales, al resto de las familias
coatitas.

\bibverse{71} Los descendientes de Gersón recibieron lo siguiente. De la
familia de la media tribu de Manasés Golán en Basán, y Astarot, junto
con sus pastizales;

\bibverse{72} de la tribu de Isacar: Cedes, Daberat, \bibverse{73} Ramot
y Anem, junto con sus pastizales;

\bibverse{74} de la tribu de Aser: Masal, Abdón, \bibverse{75} Hucoc y
Rehob, junto con sus pastizales;

\bibverse{76} y de la tribu de Neftalí: Cedes en Galilea, Hamón y
Quiriatím, con sus pastizales.

\bibverse{77} Los demás descendientes de Merari recibieron lo siguiente.
De la tribu de Zabulón Jocneam, Carta,\footnote{\textbf{6:77} Jocneam y
  Carta no están incluidos en la lista aquí, pero véase Josué 21:34.}
Rimón y Tabor, con sus pastizales;

\bibverse{78} de la tribu de Rubén, al este del Jordán, frente a Jericó:
Beser (en el desierto), Jahzá, \bibverse{79} Cedemot y Mefat, con sus
pastizales

\bibverse{80} y de la tribu de Gad: Ramot de Galaad, Mahanaim,
\bibverse{81} Hesbón y Jazer, con sus pastizales.

\hypertarget{section-6}{%
\section{7}\label{section-6}}

\bibverse{1} Los hijos de Isacar: Tola, Púa, Jasub y Simrón, un total de
cuatro.

\bibverse{2} Los hijos de Tola: Uzi, Refaías, Jeriel, Jahmai, Ibsam y
Samuel, quienes eran jefes de sus familias. En la época de David, los
descendientes de Tola enumeraban en su genealogía un total de 22.600
guerreros.

\bibverse{3} El hijo de Uzi: Israhías.

Los hijos de Israhías: Miguel, Obadías, Joel e Isías. Los cinco eran
jefes de familia. \bibverse{4} Tenían muchas esposas e hijos, por lo que
en su genealogía figuran 36.000 hombres de combate listos para la
batalla. \bibverse{5} Los parientes guerreros de todas las familias de
Isacar, según su genealogía, eran 87.000 en total.

\bibverse{6} Tres hijos de Benjamín: Bela, Bequer y Jediael.

\bibverse{7} Los hijos de Bela: Ezbón, Uzi, Uziel, Jerimot e Iri,
quienes eran jefes de sus familias, y eran un total de cinco. Tenían
22.034 combatientes según su genealogía.

\bibverse{8} Los hijos de Bequer: Zemira, Joás, Eliezer, Elioenai, Omrí,
Jerimot, Abías, Anatot y Alemet. Todos ellos fueron los hijos de Bequer.
\bibverse{9} Su genealogía incluía a los jefes de familia y a 20.200
combatientes.

\bibverse{10} El hijo de Jediael: Bilhán.

Los hijos de Bilhán: Jeús, Benjamín, Aod, Quenaana, Zetán, Tarsis y
Ahisahar. \bibverse{11} Todos estos hijos de Jediael eran jefes de sus
familias. Tenían 17.200 guerreros listos para la batalla.

\bibverse{12} Supim y Hupim eran los hijos de Ir, y Husim era hijo de
Aher.

\bibverse{13} Los hijos de Neftalí: Jahziel, Guni, Jezer y
Salum,\footnote{\textbf{7:13} ``Salum'': o ``Silem.''} quienes eran los
descendientes de Bilha.

\bibverse{14} Los hijos de Manasés: Asriel, cuya madre era su concubina
aramea. También fue la madre de Maquir, el padre de Galaad.
\bibverse{15} Maquir encontró una esposa para Hupim y otra para Suppim.
Su hermana se llamaba Maaca. La segunda se llamaba Zelofehad. Él solo
tuvo hijas.\footnote{\textbf{7:15} El texto hebreo de este verso es muy
  poco claro.}

\bibverse{16} Maaca, la esposa de Maquir, tuvo un hijo y lo llamó Peres.
Su hermano se llamaba Seres, y sus hijos fueron Ulam y Raquem.

\bibverse{17} El hijo de Ulam: Bedan.

Todos estos fueron los hijos de Galaad, hijo de Maquir, hijo de Manasés.
\bibverse{18} Su hermana Hamolequet fue la madre de Isod, Abiezer y
Mahala.

\bibverse{19} Los hijos de Semida fueron: Ahian, Siquem, Likhi y Aniam.

\bibverse{20} Los descendientes de Efraín fueron: Sutela, su hijo Bered,
su hijo Tahat, su hijo Elead, su hijo Tahat, \bibverse{21} su hijo Zabad
y su hijo Sutela. Ezer y Elead fueron asesinados por los hombres que
vivían en Gat cuando fueron allí a tratar de robar su ganado.
\bibverse{22} Su padre Efraín los lloró durante mucho tiempo, y sus
parientes fueron a consolarlo. \bibverse{23} Luego volvió a acostarse
con su mujer. Ella quedó embarazada y dio a luz un hijo, al que llamó
Bería por esta tragedia familiar. \bibverse{24} Seera, su hija, fundó la
parte baja y alta de Bet Horon junto con Uzen-Seera.

\bibverse{25} Sus desciendientes fueron: Refa su hioj, Resef su
hijo,\footnote{\textbf{7:25} ``Su hijo'': Lectura de la Septuaginta.
  Falta en el texto hebreo.} Telah su hijo, Tahan su hijo, \bibverse{26}
Ladan su hijo, Amiud su hijo, Elisama su hijo, \bibverse{27} Nun su hijo
y Josué su hijo. \bibverse{28} La tierra que poseían y los lugares donde
vivían incluían Betel y las ciudades cercanas, desde Naarán al este
hasta Gezer y sus ciudades al oeste, y Siquem y sus ciudades hasta Aya y
sus ciudades. \bibverse{29} En la frontera con Manasés estaban Bet-San,
Taanac, Meguido y Dor, junto con sus ciudades. Estas eran las ciudades
donde vivían los descendientes de José hijo de Israel.

\bibverse{30} Los hijos de Aser: Imna, Isúa, Isúi y Bería. Su hermana
era Sera.

\bibverse{31} Los hijos de Bería: Heber y Malquiel, el padre de
Birzavit.

\bibverse{32} Heber fue el padre de Jaflet, Somer y Hotam, y de su
hermana Súa.

\bibverse{33} Los hijos de Jaflet: Pasac, Bimhal y Asvat. Todos estos
fueron Los hijos de Jaflet.

\bibverse{34} Los hijos de Somer: Ahi,\footnote{\textbf{7:34} ``Los
  hijos de Somer: Ahí:'' o ``Los hijos de Somer, su hermano:''} Rohga,
Jeúba y Aram.

\bibverse{35} Los hijos de su hermano Helem: Zofa, Imna, Seles y Amal.

\bibverse{36} Los hijos de Zofa: Súa, Harnefer, Súal, Beri, Imra,
\bibverse{37} Beser, Hod, Sama, Silsa, Itrán y Beera.

\bibverse{38} Los hijos de Jeter fueron Jefone, Pispa y Ara.

\bibverse{39} Los hijos de Ula fueron Ara, Haniel y Rezia.

\bibverse{40} Todos ellos eran descendientes de los jefes de familia de
Aser, hombres selectos, fuertes guerreros y grandes líderes. Según su
genealogía, tenían 26.000 guerreros listos para la batalla.

\hypertarget{section-7}{%
\section{8}\label{section-7}}

\bibverse{1} Benjamín fue el padre de Bela (su primogénito), Asbel (el
segundo), Ahara (el tercero), \bibverse{2} Noé (el cuarto) y Rafa (el
quinto).

\bibverse{3} Los hijos de Bela fueron: Adar, Gera, Abiud, \bibverse{4}
Abisúa, Naamán, Ahoa, \bibverse{5} Gera, Sefufán y Huram.

\bibverse{6} Estos fueron los hijos de Aod, jefes de familia que vivían
en Geba, y fueron desterrados a Manahat: \bibverse{7} Naamán, Ahías y
Gera. Gera fue quien los exilió. Era el padre de Uzz y Ahiud.

\bibverse{8} Saharaim tuvo hijos en Moab después de divorciarse de sus
esposas Husim y Baara. \bibverse{9} Se casó con Hodes y tuvo a Jobab,
Zibia, Mesa, Malcam, \bibverse{10} Jeuz, Saquías y Mirma. Todos estos
fueron sus hijos, jefes de familia. \bibverse{11} También tuvo hijos con
Husim: Abitob y Elpal.

\bibverse{12} Los hijos de Elpal: Éber, Misham, Shemed (construyó Ono y
Lod con sus ciudades cercanas), \bibverse{13} y Bería y Sema, que eran
jefes de familia que vivían en Ajalón y que expulsaron a los que vivían
en Gat.

\bibverse{14} Ahío, Sasac, Jeremot, \bibverse{15} Zebadías, Arad, Eder,
\bibverse{16} Micael, Ispa y Joha eran los hijos de Bería.

\bibverse{17} Zebadías, Mesulám, Hizqui, Heber, 18 Ismerai, Jezlías y
Jobab fueron los hijos de Elpaal.

\bibverse{19} Jacim, Zicri, Zabdi, \bibverse{20} Elienai, Ziletai,
Eliel, \bibverse{21} Adaías, Beraías, y Simrat fueron los hijos de
Simei.

\bibverse{22} Ispán, Heber, Eliel, \bibverse{23} Abdón, Zicri, Hanán,
\bibverse{24} Hananías, Elam, Antotías, \bibverse{25} Ifdaías y Peniel
fueron los hijos de Sasac.

\bibverse{26} Samsherai, Seharías, Atalías, \bibverse{27} Jaresías,
Elías, y Zicri fueron los hijos de Jeroham. \bibverse{28} Todos ellos
eran jefes de familia, según su genealogía. Y vivían en Jerusalén.

\bibverse{29} Jeiel\footnote{\textbf{8:29} Según algunos manuscritos de
  la Septuaginta y también 1 Crónicas 9:35. Su nombre falta en el texto
  hebreo.} fundó Gabaón y vivió allí. Su mujer se llamaba Maaca.
\bibverse{30} Su hijo primogénito fue Abdón, luego Zur, Cis, Baal,
Ner,\footnote{\textbf{8:30} Según algunos manuscritos de la Septuaginta
  y también de 1 Crónicas 9:36. Su nombre falta en el texto hebreo.}
Nadab, \bibverse{31} Gedor, Ahío, Zequer, \bibverse{32} y Miclot. Miclot
fue el padre de Simea. También vivían cerca de sus parientes en
Jerusalén.

\bibverse{33} Ner fue el padre de Cis, Cis fue el padre de Saúl, y Saúl
fue el padre de Jonatán, Malquisúa, Abinadab y Es-Baal.\footnote{\textbf{8:33}
  En otros lugares se le conoce como ``Is-boset'', para evitar incluir
  en su nombre al dios pagano Baal. ``Boset'' significa ``vergüenza.''}

\bibverse{34} El hijo de Jonatán fue Merib-Baal, \footnote{\textbf{8:34}
  Del mismo modo, también se le conoce como Mefi-boset.} que fue el
padre de Miqueas.

\bibverse{35} Los hijos de Miqueas fueron Pitón, Melec, Tarea y Acaz.

\bibverse{36} Acaz fue el padre de Jada; Jada fue el padre de Alemet,
Azmavet y Zimri; y Zimri fue el padre de Mosa. \bibverse{37} Mosa fue el
padre de Bina. Su hijo fue Rafa, padre de Elasa, padre de Azel.

\bibverse{38} Azel tuvo seis hijos. Estos fueron sus nombres: Azricam,
Bocru, Ismael, Seraías, Obadías y Hanán. Todos ellos eran los hijos de
Azel.

\bibverse{39} Los hijos de su hermano Esec fueron Ulam (primogénito),
Jeús (el segundo) y Elifelet (el tercero). \bibverse{40} Los hijos de
Ulam eran fuertes guerreros y hábiles arqueros. Tuvieron muchos hijos y
nietos, un total de 150. Todos ellos fueron los hijos de Benjamín.

\hypertarget{section-8}{%
\section{9}\label{section-8}}

\bibverse{1} Todo el pueblo de Israel quedó registrado en los registros
de las genealogías del libro de los reyes de Israel.

Pero el pueblo de Judá fue llevado al cautiverio en Babilonia porque
había sido infiel.\footnote{\textbf{9:1} Claramente el autor de Crónicas
  está escribiendo después del cautiverio, y atribuye este evento al
  fracaso de la nación en seguir al verdadero Dios.}

\bibverse{2} Los primeros en regresar y reclamar sus propiedades y vivir
en sus ciudades fueron algunos israelitas, sacerdotes, levitas y
servidores del Templo. \bibverse{3} Algunos de los miembros de las
tribus de Judá, Benjamín, Efraín y Manasés volvieron a vivir en
Jerusalén. Entre ellos estaban:

\bibverse{4} Utaí hijo de Ammihud, hijo de Omrí, hijo de Imrí, hijo de
Baní, descendiente de Fares, hijo de Judá.

\bibverse{5} De los silonitas: Asaías (el primogénito) y sus hijos.

\bibverse{6} De los zeraítas: Jeuel y sus parientes, con un total de
690.

\bibverse{7} De los benjamitas: Salú, hijo de Mesulam, hijo de Hodavías,
hijo de Asenúa; \bibverse{8} Ibneías, hijo de Jeroham; Ela, hijo de Uzi,
hijo de Micrí; y Mesulam, hijo de Sefatías, hijo de Reuel, hijo de
Ibniás.

\bibverse{9} Todos ellos eran jefes de familia, según consta en sus
genealogías, y en total sumaban 956.

\bibverse{10} De los sacerdotes: Jedaías, Joiarib, Yacín, \bibverse{11}
Azarías hijo de Hilcías, hijo de Mesulam, hijo de Sadoc, hijo de
Meraiot, hijo de Ajitub. (Azarías era el funcionario principal a cargo
de la casa de Dios).

\bibverse{12} También Adaías hijo de Jeroham, hijo de Pasur, hijo de
Malquías, y Masai hijo de Adiel, hijo de Jazera, hijo de Mesulam, hijo
de Mesilemit, hijo de Imer. \bibverse{13} Los sacerdotes, que eran jefes
de familia, sumaban 1.760. Eran hombres fuertes y capaces que servían en
la casa de Dios.

\bibverse{14} De los levitas: Semaías, hijo de Jasub, hijo de Azricam,
hijo de Hasabías, descendiente de Merari; \bibverse{15} Bacbacar, Heres,
Galal y Matanías, hijo de Mica, hijo de Zicri, hijo de Asaf;
\bibverse{16} Abdías, hijo de Semaías, hijo de Galal, hijo de Jedutún; y
Berequías, hijo de Asa, hijo de Elcana, que vivían en las aldeas de los
netofatitas.

\bibverse{17} Los guardas de la puerta del templo:\footnote{\textbf{9:17}
  ``Del templo'': implícito.} Salum, Acub, Talmón, Ahimán y sus
parientes. Salum era el jefe de los guardianes de la puerta del templo.
\bibverse{18} Ellos tenían la responsabilidad hasta ahora de cuidar la
Puerta del Rey en el lado Este. Eran los guardianes de los campamentos
de los levitas. \bibverse{19} Salum era hijo de Coré, hijo de Ebiasaf,
hijo de Coré. Él y sus parientes, los coreítas, eran responsables de
vigilar las entradas del santuario\footnote{\textbf{9:19} ``Santuario'':
  Literalmente, ``tienda,'' aunque ahora se refiera al edificio del
  Templo.} de la misma manera que sus padres se habían encargado de
vigilar la entrada de la casa de campaña\footnote{\textbf{9:19} ``La
  casa de campaña,'' o ``Tabernáculo.''} del Señor. \bibverse{20}
Anteriormente, Finees hijo de Eleazar, había sido el líder de los
porteros. Y el Señor estaba con él. \bibverse{21} Posteriormente,
Zacarías hijo de Meselemías fue el portero de la entrada de la tienda de
reunión.

\bibverse{22} En total hubo 212 elegidos para ser porteros en las
entradas. Registraron sus genealogías en sus ciudades de origen. David y
el profeta Samuel habían seleccionado a sus antepasados por su
fidelidad. \bibverse{23} Ellos y sus descendientes eran responsables de
vigilar la entrada de la casa del Señor, también cuando era una tienda.

\bibverse{24} Los porteros estaban ubicados en cuatro lados: al este, al
oeste, al norte y al sur. \bibverse{25} Sus parientes en sus pueblos
venían cada siete días a ciertas horas para ayudarlos. \bibverse{26} Los
cuatro porteros principales, que eran levitas, tenían la responsabilidad
de cuidar las habitaciones y los tesoros de la casa de Dios.
\bibverse{27} Pasaban la noche alrededor de la casa de Dios porque
tenían que vigilarla y tenían la llave para abrirla por la mañana.
\bibverse{28} Algunos de los porteros eran responsables de los artículos
que se utilizaban en los servicios de culto. Llevaban un inventario de
lo que se traía y de lo que se sacaba. \bibverse{29} A otros se les
asignaba la tarea de cuidar el mobiliario y el equipo utilizado en el
santuario, así como la harina especial, el vino, el aceite de oliva, el
incienso y las especias. \bibverse{30} (Sin embargo, algunos de los
sacerdotes eran los encargados de mezclar las especias). \bibverse{31} A
Matatías, un levita, que era el hijo primogénito de Salum el coreíta, se
le dio la responsabilidad de hornear el pan usado en las ofrendas.
\bibverse{32} Otros coatitas se encargaban de preparar el pan que se
ponía en la mesa cada sábado. \bibverse{33} Los músicos, jefes de
familias de levitas, vivían en las habitaciones del Templo y no debían
realizar otras tareas porque estaban de servicio día y noche.
\bibverse{34} Todos ellos eran jefes de familias de levitas, líderes
según sus genealogías, y vivían en Jerusalén.

\bibverse{35} Jeiel\footnote{\textbf{9:35} Véase 8:29.} era el padre de
Gabaón y vivía en Gabaón. Su mujer se llamaba Maaca. \bibverse{36} Su
hijo primogénito fue Abdón, luego Zur, Cis, Baal, Ner, Nadab,
\bibverse{37} Gedor, Ahío, Zacarías y Miclot. \bibverse{38} Miclot fue
el padre de Simeam. Ellos también vivían cerca de sus parientes en
Jerusalén.

\bibverse{39} Ner fue el padre de Cis, Cis fue el padre de Saúl, y Saúl
fue el padre de Jonatán, Malquisúa, Abinadab y Es-Baal.\footnote{\textbf{9:39}
  Véase la nota a pie de página de 8:33.}

\bibverse{40} El hijo de Jonathan: Merib-Baal,\footnote{\textbf{9:40}
  Véase la nota a pie de página de 8:34.} que fue el padre de Miqueas.

\bibverse{41} Los hijos de Miqueas: Pitón, Melec, Tarea y
Acaz.\footnote{\textbf{9:41} ``Y Acaz'': Tomado de la Septuaginta y del
  verso 8:35. En el texto hebreo, el nombre no aparece aquí.}

\bibverse{42} Acaz fue el padre de Jada,\footnote{\textbf{9:42} Según la
  Septuaginta y el verso 8:36. En hebreo se escribe ``Jara.''} Jada fue
el padre de Alemet, Azmavet y Zimri, y Zimri fue el padre de Mosa.
\bibverse{43} Mosa fue el padre de Bina; Refaías fue su hijo, Elasa su
hijo, y Azel su hijo.

\bibverse{44} Azel tuvo seis hijos. Sus nombres eran: Azricam, Bocru,
Ismael, Searías, Obadías y Hanán. Estos fueron los hijos de Azel.

\hypertarget{section-9}{%
\section{10}\label{section-9}}

\bibverse{1} Los filisteos atacaron a Israel y todos los soldados
israelitas huyeron de ellos. Muchos israelitas fueron abatidos en el
monte Gilboa. 2 Los filisteos persiguieron a Saúl y a sus hijos. Mataron
a los hijos de Saúl: Jonatán, Abinadab y Malquisúa. \bibverse{3} La
batalla se desató intensamente alrededor de Saúl. Luego los arqueros
enemigos vieron dónde estaba y lo hirieron. \bibverse{4} Entonces Saúl
le dijo a su escudero: ``Saca tu espada y mátame antes de que estos
paganos vengan a torturarme''. Pero su escudero se negó, pues tenía
demasiado miedo de hacerlo. Así que Saúl tomó su propia espada y se hizo
caer sobre ella. \bibverse{5} Al ver que Saúl estaba muerto, su escudero
también se cayó sobre su espada y murió. \bibverse{6} Así que Saúl y
tres de sus hijos murieron allí, junto con su línaje real.\footnote{\textbf{10:6}
  Literalmente, ``todos los de su casa murieron juntos'', sin embargo,
  esto debe tomarse en el contexto de que ninguno de sus hijos le
  sucedió, pues su hijo Is-boset sí sobrevivió.} \bibverse{7} Cuando
todos los israelitas del valle vieron que su ejército había huido y que
Saúl y sus hijos habían muerto, abandonaron sus ciudades y también
huyeron. Entonces los filisteos llegaron y las ocuparon.

\bibverse{8} Al día siguiente, cuando los filisteos fueron a despojar a
los muertos, descubrieron los cuerpos de Saúl y de sus hijos en el monte
Gilboa. \bibverse{9} Lo desnudaron, le cortaron la cabeza y se llevaron
su armadura. Luego enviaron la noticia a toda la tierra de Filistea, a
sus ídolos y a su pueblo. \bibverse{10} Pusieron la armadura de Saúl en
el templo de sus ídolos y fijaron su cabeza en el templo de Dagón.
\bibverse{11} Sin embargo, cuando todos en Jabes de Galaad se enteraron
de todo lo que los filisteos habían hecho con Saúl, \bibverse{12} todos
sus combatientes fueron a recuperar los cuerpos de Saúl y de sus hijos.
Entonces los trajeron de vuelta y los enterraron bajo el gran árbol de
Jabes. Luego ayunaron durante siete días.

\bibverse{13} Saúl murió porque le fue infiel al Señor. No cumplió los
mandatos del Señor, y además fue a consultar a una médium. \bibverse{14}
No consultó al Señor, así que el Señor le dio muerte y le entregó el
reinado a David, hijo de Isaí.

\hypertarget{section-10}{%
\section{11}\label{section-10}}

\bibverse{1} Entonces todos los israelitas se reunieron con David en
Hebrón. Y le dijeron: ``Somos tu carne y tu sangre.\footnote{\textbf{11:1}
  ``Tu carne y tu sangre'': Literalmente, ``huesos y carne.''}
\bibverse{2} En los últimos tiempos, aunque Saúl era el rey, tú eras el
verdadero líder de Israel.\footnote{\textbf{11:2} ``Verdadero líder de
  Israel'': Literalmente, ``Tú guiaste y trajiste a Israel.''} El Señor,
tu Dios, te ha dicho: `Tú serás el pastor de mi pueblo, y tú serás el
jefe de mi pueblo Israel'``. \bibverse{3} Todos los ancianos de Israel
acudieron ante el rey en Hebrón, y David hizo un acuerdo
solemne\footnote{\textbf{11:3} ``Acuerdo solemne'': o ``pacto.''} con
ellos ante el Señor. Allí ungieron a David como rey de Israel, tal como
el Señor lo había prometido por medio de Samuel. \bibverse{4} Entonces
David y todos los israelitas fueron a Jerusalén (antes conocida como
Jebús), donde vivían los jebuseos. \bibverse{5} Pero los jebuseos le
dijeron a David: ``No entrarás aquí''. Sin embargo, David capturó la
fortaleza de Sión, ahora conocida como la Ciudad de David. \bibverse{6}
Y David había dicho: ``El primero que ataque a los jebuseos será mi
comandante en jefe''. Como Joab, hijo de Sarvia, fue el primero, se
convirtió en el comandante en jefe.

\bibverse{7} David decidió habitar en la fortaleza, por eso la llamaron
Ciudad de David. \bibverse{8} Entonces construyó la ciudad a su
alrededor, desde el Milo\footnote{\textbf{11:8} El significado de esta
  palabra es incierto.} e hizo un circuito alrededor, mientras que Joab
reparaba el resto de la ciudad. \bibverse{9} David se hizo cada vez más
poderoso, \footnote{\textbf{11:9} ``Más y más poderoso'': literalmente,
  ``aumentaba y aumentaba.''} porque el Señor Todopoderoso estaba con
él.

\bibverse{10} Estos fueron los líderes de los poderosos guerreros de
David que, junto con todos los israelitas, le dieron un fuerte apoyo
para que se convirtiera en rey, tal como el Señor había prometido que
sucedería con Israel. \bibverse{11} Esta es la lista de los principales
guerreros que apoyaron a David: Jasobeam, hijo de Hacmoni, líder de los
Tres. Con su lanza, una vez mató a 300 hombres en una sola batalla.
\bibverse{12} Después de él vino Eleazar, hijo de Dodo, descendiente de
Ahoha, uno de los Tres guerreros principales. \bibverse{13} Estaba con
David en Pasdamin cuando los filisteos se reunieron para la batalla que
tuvo lugar en un campo de cebada. El ejército israelita huyó cuando los
filisteos atacaron, \bibverse{14} pero David y Eleazar se apostaron en
medio del campo, defendiendo su terreno y matando a los filisteos. El
Señor los salvó dándoles una gran victoria.

\bibverse{15} En otra ocasión,\footnote{\textbf{11:15} Implícito.} los
Tres, que formaban parte de los Treinta guerreros principales, bajaron a
recibir a David cuando estaba en la cueva de Adulam. El ejército
filisteo estaba acampado en el valle de Refaim. \bibverse{16} En ese
momento David estaba en la fortaleza, y la guarnición filistea estaba en
Belén. \bibverse{17} David tenía mucha sed y dijo: ``¡Ojalá alguien
pudiera traerme un trago de agua del pozo que está junto a la puerta de
la entrada de Belén!'' \bibverse{18} Los Tres atravesaron las defensas
filisteas, tomaron un poco de agua del pozo de la puerta de Belén y se
la llevaron a David. Pero David se negó a beberla y la vertió como
ofrenda al Señor. \bibverse{19} ``¡Dios me libre de hacer esto!'', dijo.
``Sería como beber la sangre de esos hombres que arriesgaron sus vidas.
Ellos arriesgaron sus vidas para traerme una bebida''. Así que no la
bebió. Estas son algunas de las cosas que hicieron los tres guerreros
principales.

\bibverse{20} Abisai, hermano de Joab, era el líder de los segundos
Tres.\footnote{\textbf{11:20} Sin embargo, ya se ha mencionado a
  Jasobeam como líder de los Tres (11:11), y también se ha mencionado la
  muerte de 300 con su lanza. Algunos sugieren una confusión de nombres
  o una ortografía alternativa, o que esto se refiere a otra persona en
  conjunto como líder no de los Tres sino de los Treinta, o que había
  otro ``Tres''.''} Usando su lanza, una vez mató a 300 hombres, y se
hizo famoso entre los Tres. \bibverse{21} Era el más apreciado de los
Tres y era su comandante, aunque no fue uno de los primeros
Tres.\footnote{\textbf{11:21} Identificar un primer y un segundo Tres
  parece ser la solución más sencilla a lo que son versos confusos.}

\bibverse{22} Benaía, hijo de Joiada, un fuerte guerrero de Cabseel,
hizo muchas cosas sorprendentes. Mató a dos hijos de Ariel de
Moab.\footnote{\textbf{11:22} Entendido en la Septuaginta; puede
  referirse a dos campeones combatientes de Moab.} También fue tras un
león a un pozo en la nieve y lo mató. \bibverse{23} En otra ocasión mató
a un egipcio, un hombre enorme que medía dos metros y medio.\footnote{\textbf{11:23}
  Literalmente ``cinco codos.''} El egipcio tenía una lanza cuyo asta
era tan gruesa como la vara de un tejedor. Benaía lo atacó sólo con un
garrote, pero pudo agarrar la lanza de la mano del egipcio, y lo mató
con su propia lanza. \bibverse{24} Este fue el tipo de cosas que hizo
Benaía y que lo hicieron tan famoso como los tres guerreros principales.
\bibverse{25} Era el más apreciado de los Treinta, aunque no era uno de
los Tres. David lo puso a cargo de su guardia personal.

26 Otros guerreros principales eran: Asael, hermano de Joab; Elhanán,
hijo de Dodo, de Belén; 27 Sama el harodita; Heles el pelonita;
\bibverse{28} Ira, hijo de Iques, de Tecoa; Abiezer, de Anatot;
\bibverse{29} Sibecai el husatita; Ilai el ahohita; \bibverse{30}
Maharai, de Netofa; Heled, hijo de Baana, de Netofa; \bibverse{31} Itai,
hijo de Ribai, de Guibeá, de los benjamitas; Benaía el Piratonita;
\bibverse{32} Hurai de los valles de Gaas; Abiel de Arabá; \bibverse{33}
Azmavet de Bahurim; Eliaba el Saalbonita; \bibverse{34} Los hijos de
Jasén el Gizonita; Jonatán, hijo de Sage el Ararita; \bibverse{35}
Ahíam, hijo de Sacar el Ararita; Elifal, hijo de Ur; \bibverse{36} Hefer
de Mequer; Ahías el pelonita; \bibverse{37} Hezro el Carmelita; Naarai,
hijo de Ezbai; \bibverse{38} Joel, hermano de Natán; Mibhar, hijo de
Hagri; \bibverse{39} Zelek, el amonita; Naharai, de Beerot; el escudero
de Joab, hijo de Sarvia; \bibverse{40} Ira, de Jatir; Gareb, de Jatir;
\bibverse{41} Urías, el hitita; Zabad, hijo de Ahlai; \bibverse{42}
Adina, hijo de Siza, rubenita, jefe de los rubenitas, y los treinta que
estaban con él; \bibverse{43} Hanán, hijo de Maaca; Josafat mitnita;
\bibverse{44} Uzías de Astarot; Sama y Jeiel, los hijos de Hotam de
Aroer; \bibverse{45} Jediael, hijo de Simri, y su hermano, Joha el
tizita; \bibverse{46} Eliel de Mahava; Jerebai y Josavía, los hijos de
Elnaam; Itma el moabita; 47 Eliel; Obed y Jaasiel, todos ellos de Soba.

\hypertarget{section-11}{%
\section{12}\label{section-11}}

\bibverse{1} La siguiente es una lista de los hombres que se unieron a
David cuando estaba en Siclag, todavía escondiéndose de Saúl, hijo de
Cis. Eran algunos de los principales guerreros que lucharon del lado de
David. \bibverse{2} Todos ellos eran hábiles arqueros, y podían disparar
flechas o hondas con la mano derecha o con la izquierda. Eran parientes
de Saúl de la tribu de Benjamín. \bibverse{3} Ahiezer era su líder,
luego lo fue Joás; los hijos de Semaa de Gugibeá; Jeziel y Pelet los
hijos de Azmavet; Beraca; Jehú de Anatot; \bibverse{4} Ismaías de
Gabaón, (un fuerte guerrero entre los Treinta, y líder sobre los
Treinta); Jeremías; Jahaziel; Johanán; Jozabad de de Gedera;
\bibverse{5} Eluzai; Jerimot; Bealías; Semarías; Sefatías de Haruf.
\bibverse{6} Elcana, Isías, Azareel, Joezer y Jasobeam (quienes eran
eran coreítas); \bibverse{7} Joela y Zebadías, los hijos de Jeroham de
Gedor.

\bibverse{8} Algunos guerreros de la tribu de Gad se pasaron al lado de
David cuando éste estaba en la fortaleza del desierto. Eran guerreros
fuertes y experimentados, curtidos en la batalla, expertos en el uso de
escudos y lanzas. Sus rostros parecían tan fieros como los de los
leones, y corrían tan rápido como las gacelas en las montañas.
\bibverse{9} Ezer el era el líder, Obadías (el segundo), Eliab (el
tercero), \bibverse{10} Mismaná (el cuarto), Jeremías (el quinto),
\bibverse{11} Atai (el sexto), Eliel (el séptimo), \bibverse{12} Johanán
(el octavo), Elzabad (el noveno), \bibverse{13} Jeremías (el décimo),
Macbanai (el undécimo). \bibverse{14} Estos guerreros de Gad eran
oficiales del ejército. El menos hábil de ellos estaba a cargo de 100
hombres; y el mejor estaba a cargo de 1.000. \bibverse{15} Estos eran
los que cruzaban el río Jordán en el primer mes del año, cuando se
desborda. Expulsaron a todos los pueblos que vivían en el valle, tanto
al este como al oeste.

\bibverse{16} Otros de las tribus de Benjamín y Judá también vinieron a
unirse a David en la fortaleza. \bibverse{17} David salió a recibirlos y
les dijo: ``Si han venido en son de paz para ayudarme, podemos ser
amigos.\footnote{\textbf{12:17} ``Podemos ser amigos'': Literalmente,
  ``mi corazón estará junto a ustedes.''} Pero si han venido a
entregarme a mis enemigos, aunque no he hecho nada malo, que el Dios de
nuestros padres vea lo que hacen y los condene.''

\bibverse{18} Entonces el Espíritu vino sobre\footnote{\textbf{12:18}
  ``Vino sobre'': Literalmente, ``vistió''.} Amasai, el líder de los
Treinta. ``¡Somos tuyos, David, y estamos contigo, hijo de Isaí! Que la
paz, la prosperidad y el éxito\footnote{\textbf{12:18} ``Paz, la
  prosperidad y el éxito'': Literalmente, ``Shalom, shalom a ti, y
  shalom a quien te ayude.''} sean tuyos y de los que te ayuden, porque
Dios es el que te ayuda''. Así que David les permitió unirse a él, y los
puso al frente de su ejército.

\bibverse{19} Otros se pasaron al lado de David desde la tribu de
Manasés y se unieron a él cuando fue con los filisteos a atacar a Saúl.
Sin embargo, los jefes filisteos decidieron finalmente despedirlos,
diciéndose: ``Nos costará la cabeza si nos abandona y se une a su amo
Saúl.''

\bibverse{20} La siguiente es una lista de los hombres de Manasés que se
pasaron al lado de David cuando éste regresó a Siclag: Adnas, Jozabad,
Jediael, Micael, Jozabad, Eliú y Ziletai, jefes de millares en la tribu
de Manasés. \bibverse{21} Ellos ayudaron a David contra los asaltantes,
pues todos eran guerreros fuertes y experimentados y comandantes del
ejército. \bibverse{22} Cada día llegaban hombres para ayudar a David,
hasta que tuvo un gran ejército, como el ejército de Dios.

\bibverse{23} Esta es la lista del número de guerreros armados que
vinieron y se unieron a David en Hebrón para entregarle el reino de
Saúl, como había dicho el Señor. \bibverse{24} De la tribu de Judá,
6.800 guerreros con escudos y lanzas. \bibverse{25} De la tribu de
Simeón, 7.100 guerreros fuertes. \bibverse{26} De la tribu de Leví,
4.600, \bibverse{27} incluyendo a Joiada, jefe de la familia de Aarón, y
con él 3.700, \bibverse{28} y Sadoc, un joven guerrero fuerte, con 22
miembros de su familia, todos oficiales. \bibverse{29} De la tribu de
Benjamín, de entre los parientes de Saúl, 3.000, la mayoría de los
cuales habían permanecido leales a Saúl hasta ese momento. \bibverse{30}
De la tribu de Efraín, 20.800 guerreros fuertes, cada uno de ellos muy
apreciado en su propio clan. \bibverse{31} De la media tribu de Manasés,
18.000 hombres designados por su nombre para venir a hacer rey a David.
\bibverse{32} De la tribu de Isacar vinieron líderes que conocían y
podían entender los signos de los tiempos y lo que Israel debía hacer:
un total de 200 líderes de la tribu junto con sus parientes.
\bibverse{33} De la tribu de Zabulón, 50.000 guerreros. Estaban
completamente armados y listos para la batalla, y totalmente dedicados.
\bibverse{34} De la tribu de Neftalí, 1.000 oficiales y 37.000 guerreros
con escudos y lanzas. \bibverse{35} De la tribu de Dan, 28.600
guerreros, todos preparados para la batalla. \bibverse{36} De la tribu
de Aser, 40.000 guerreros experimentados, todos listos para la batalla.
\bibverse{37} De la parte oriental del río Jordán, de las tribus de
Rubén, Gad y la media tribu de Manasés, 120.000 guerreros con todo tipo
de armas.

\bibverse{38} Todos estos hombres llegaron a Hebrón vestidos para la
batalla, completamente decididos a hacer rey a David. Todo Israel estaba
de acuerdo en que David debía ser rey. \bibverse{39} Se quedaron allí
tres días, comiendo y bebiendo juntos, pues sus parientes les habían
proporcionado provisiones. \bibverse{40} Sus vecinos, incluso de lugares
tan lejanos como Isacar, Zabulón y Neftalí, llegaron trayendo comida en
burros, camellos, mulas y bueyes. Tenían mucha harina, tortas de higos,
racimos de pasas, vino, aceite de oliva, ganado y ovejas, porque Israel
estaba muy contento.

\hypertarget{section-12}{%
\section{13}\label{section-12}}

\bibverse{1} David tuvo discusiones con todos sus líderes, incluyendo
los comandantes del ejército de miles y cientos.\footnote{\textbf{13:1}
  ``Miles y cientos'': en referencia a la estructura del ejército, en el
  que unos estaban a cargo de 1000 hombres y otros de 100.} \bibverse{2}
Luego se dirigió a toda la asamblea de Israel, diciendo: ``Si están de
acuerdo, y si Dios lo aprueba, enviemos una invitación a todos los
israelitas de la tierra, incluidos los sacerdotes y levitas en sus
ciudades y pastos, para que vengan a unirse a nosotros. \bibverse{3}
Traigamos de vuelta el Arca de nuestro Dios\footnote{\textbf{13:3}
  ``Traigamos de vuelta'': Curiosamente la raíz del verbo tiene el
  significado básico de ``rodear.''} a nosotros, porque lo habíamos
olvidado en tiempos de Saúl''.

\bibverse{4} Toda la asamblea se alegró de la propuesta, y estuvo de
acuerdo en que sería una buena cosa. \bibverse{5} Así que David convocó
a todo Israel, desde el río Sihor de Egipto hasta Lebo-hamat, para que
ayudaran a traer el Arca desde Quiriat-jearim. \bibverse{6} Así pues,
David y todo Israel fueron a Baalá (llamada también Quiriat-jearim), en
Judá, para traer de vuelta el Arca de Dios el Señor, cuyo trono está
entre los querubines y que es llamado por el Nombre. \bibverse{7}
Cargaron el Arca de Dios en una carreta nueva y la trajeron desde la
casa de Adinadab, con Uza y Ahio dirigiéndola. \bibverse{8} David y todo
Israel estaban celebrando ante el Señor lo más alto posible, cantando
canciones y tocando música con liras, arpas, panderetas, címbalos y
trompetas.

\bibverse{9} Pero cuando llegaron a la era de Quidón, los bueyes
tropezaron y Uzza extendió la mano para evitar que el Arca se cayera.
\bibverse{10} El Señor se enfadó con Uza por atreverse a tocar el Arca
de esa manera, así que lo abatió, y Uza murió allí ante el Señor.

\bibverse{11} David se enfadó con el Señor por su violento arrebato
contra Uza. Llamó al lugar Fares-uza,\footnote{\textbf{13:11} Fares-uza
  significa ``arrebato contra Uzza.''} y aún hoy se le llama así.

\bibverse{12} Ese día, David tuvo miedo de Dios. ``¿Cómo podré devolver
el Arca de Dios a mi casa?'', se preguntó. \bibverse{13} Así que David
no trasladó el Arca de Dios para que estuviera con él en la Ciudad de
David. En lugar de eso, hizo que la llevaran a la casa de Obed-edom de
Gat. \bibverse{14} El Arca de Dios permaneció en la casa de Obed-edom
durante tres meses, y el Señor bendijo la casa de Obed-edom y todo lo
que tenía.

\hypertarget{section-13}{%
\section{14}\label{section-13}}

\bibverse{1} Entonces Hiram, rey de Tiro, envió mensajeros a David junto
con madera de cedro, canteros y carpinteros para que le construyeran un
palacio. 2 De esta manera David se dio cuenta de que el Señor lo había
colocado en el trono como rey de Israel y había bendecido apoyando su
reino por el bien del pueblo del Señor, Israel. \bibverse{3} David se
casó con más esposas en Jerusalén y tuvo más hijos e hijas. \bibverse{4}
Esta es una lista de los nombres de los hijos que tuvo en Jerusalén:
Samúa, Sobab, Natán, Salomón, \bibverse{5} Ibhar, Elisúa, Elpelet,
\bibverse{6} Noga, Nefeg, Jafía, \bibverse{7} Elisama, Beeliada y
Elifelet.

\bibverse{8} Cuando los filisteos se enteraron de que David había sido
ungido rey de todo Israel, reunieron todo su ejército para ir tras él.
Pero David oyó que venían y salió a enfrentarlos. 9 Los filisteos
llegaron y asaltaron el valle de Refaim.

\bibverse{10} David consultó a Dios y le preguntó: ``¿Debo ir a atacar a
los filisteos? ¿Me harás victorioso sobre ellos?''.

``Adelante'', le dijo el Señor, ``yo te haré victorioso sobre ellos''.

\bibverse{11} Así que David atacó y los derrotó allí en Baal-perazim.
``Dios me utilizó para derrotar a mis enemigos como un torrente de agua
que brota'', declaró. Por eso el lugar se llamó Baal-perazim.\footnote{\textbf{14:11}
  Baal-perazim significa ``el Señor irrumpe.''} \bibverse{12} Los
filisteos habían dejado sus dioses, así que David dio órdenes de que los
quemaran.

\bibverse{13} Sin embargo, los filisteos regresaron y realizaron otra
incursión en el valle. \bibverse{14} David volvió a consultar a Dios.
``No hagas un ataque frontal'', le dijo Dios. ``En lugar de eso, ve por
detrás de ellos y atácalos frente a los árboles de bálsamo.
\bibverse{15} En cuanto oigas el ruido de la marcha en las copas de los
bálsamos, ve y ataca, porque el Señor ha ido delante de ti para derribar
al ejército filisteo.'' \bibverse{16} Así que David hizo lo que Dios le
dijo, derribando al ejército filisteo desde Gabaón hasta Gezer.

\bibverse{17} Como resultado, la reputación de David se extendió por
todas partes, y el Señor hizo que todas las naciones tuvieran miedo de
David.

\hypertarget{section-14}{%
\section{15}\label{section-14}}

\bibverse{1} Una vez que David terminó de construirse casas en la Ciudad
de David, hizo un lugar para el Arca de Dios y levantó allí una tienda.
\bibverse{2} Luego dio órdenes: ``Nadie debe llevar el Arca de Dios,
excepto los levitas, porque el Señor mismo los eligió para llevar el
Arca del Señor y servirle para siempre''.

\bibverse{3} Entonces David convocó a todo Israel a Jerusalén para
llevar el Arca del Señor al lugar que había preparado para ella.
\bibverse{4} Esta es la lista de los levitas, Los hijos de Aarón, que
David convocó para asistir: \bibverse{5} De los hijos de Coat, Uriel (el
jefe de familia), y 120 de sus parientes; \bibverse{6} de los hijos de
Merari, Asaías (el jefe de familia), con 220 de sus parientes;
\bibverse{7} de los hijos de Gersón, Joel (el jefe de familia), con 130
de sus parientes; \bibverse{8} de los hijos de Elizafán, Semaías (el
jefe de familia), con 200 de sus parientes; \bibverse{9} de los hijos de
Hebrón, Eliel (el jefe de familia), con 80 de sus parientes;
\bibverse{10} de los hijos de Uziel, Aminadab (el jefe de familia), con
112 de sus parientes.

\bibverse{11} Entonces David convocó a los sacerdotes Sadoc y Abiatar, y
a los levitas Uriel, Asaías, Joel, Semaías, Eliel y Aminadab.
\bibverse{12} Les dijo: ``Ustedes son los jefes de las familias de los
levitas. Ustedes mismos y sus parientes deben estar ceremonialmente
limpios y puros\footnote{\textbf{15:12} ``Ceremonialmente limpio y
  puro'': seguir las normas y requisitos religiosos.} antes de que
traigas de vuelta el Arca de Dios, el Señor de Israel al lugar que he
hecho para ella. \bibverse{13} Por no haber estado allí la primera vez
para llevar el Arca, el Señor, nuestro Dios, estalló en violencia contra
nosotros. No la tratamos de acuerdo con sus instrucciones''.
\bibverse{14} Así que los sacerdotes y los levitas se purificaron para
poder traer de vuelta el Arca del Señor, el Dios de Israel.
\bibverse{15} Entonces los levitas llevaron el Arca de Dios de la manera
que Moisés había ordenado, según lo que Dios había dicho: sobre sus
hombros, usando las varas especiales para transportarla.

\bibverse{16} David también dio instrucciones a los jefes de los levitas
para que asignaran de entre sus parientes a cantores que cantaran con
alegría, acompañados por músicos que tocaran liras, arpas y címbalos.
\bibverse{17} Así que los levitas asignaron a Hemán, hijo de Joel, y de
sus parientes a Asaf, hijo de Berequías, y de los hijos de Merari, sus
parientes, a Etán, hijo de Cusaías. 18 El segundo grupo de levitas
estaba formado por Zacarías, Jaaziel, Semiramot, Jehiel, Uni, Eliab,
Benaía, Maaseías, Matatías, Elifelehu y Micnías; y los porteros
Obed-edom y Jeiel.

\bibverse{19} Los músicos Hemán, Asaf y Etán debían tocar los címbalos
de bronce; \bibverse{20} Zacarías, Aziel, Semiramot, Jehiel, Uni, Eliab,
Maasé y Benaía debían tocar las arpas ``según alamot'', \bibverse{21}
mientras que Matatías, Elifelehu, Micnías, Obed-edom, Jeiel y Azazías
debían dirigir la música con liras ``según seminit''. \bibverse{22}
Quenanías, el líder de los levitas en el canto, fue elegido para dirigir
la música debido a su habilidad. \bibverse{23} Berequías y Elcana fueron
designados para custodiar el Arca. \bibverse{24} Los sacerdotes
Sebanías, Josafat, Natanel, Amasai, Zacarías, Benaía y Eliezer debían
tocar las trompetas delante del Arca de Dios. Obed-edom y Jeías también
fueron designados para custodiar el Arca.

\bibverse{25} Luego David, los ancianos de Israel y los comandantes del
ejército de mayor rango,\footnote{\textbf{15:25} ``De mayor rango'':
  Literalmente, ``comandantes de miles.''} fueron con gran celebración a
traer el Arca del Pacto del Señor desde la casa de Obed-Edom. 26 Como
Dios ayudó a los levitas que llevaban el Arca del Pacto del Señor,
sacrificaron siete toros y siete carneros.

\bibverse{27} David se vistió con una túnica de lino fino, al igual que
todos los levitas que llevaban el Arca, y los cantores y Quenanías, el
líder de la música y los cantores. David también se puso un efod de
lino.\footnote{\textbf{15:27} ``Efod de lino'': ropa especial que llevan
  los sacerdotes.} \bibverse{28} Así que todo Israel trajo de vuelta el
Arca del Pacto del Señor con mucha gritería, acompañada de cuernos,
trompetas y címbalos, y música tocada con arpas y liras.

\bibverse{29} Pero cuando el Arca del Pacto del Señor entró en la Ciudad
de David, la hija de Saúl, Mical, miró desde una ventana. Al ver al rey
David saltando y bailando de alegría, se llenó de desprecio por él.

\hypertarget{section-15}{%
\section{16}\label{section-15}}

\bibverse{1} Trajeron el Arca de Dios y la colocaron en la tienda que
David había preparado para ella. Presentaron holocaustos y ofrendas de
amistad a Dios. \bibverse{2} Cuando David terminó de presentar los
holocaustos y las ofrendas de amistad, bendijo al pueblo en nombre del
Señor. \bibverse{3} Luego repartió a cada israelita, a cada hombre y a
cada mujer, una hogaza de pan, una torta de dátiles y una torta de
pasas.

\bibverse{4} David asignó a algunos de los levitas para que sirvieran de
ministros ante el Arca del Señor, para recordar, agradecer y alabar al
Señor, el Dios de Israel. \bibverse{5} Asaf era el encargado, Zacarías
era el segundo, luego Jeiel, Semiramot, Jehiel, Matatías, Eliab, Benaía,
Obed-edom y Jeiel. Tocaban arpas y liras, y Asaf tocaba los címbalos,
\bibverse{6} y los sacerdotes Benaía y Jahaziel tocaban continuamente
las trompetas delante del Arca del Pacto de Dios. \bibverse{7} Este fue
el día en que David instruyó por primera vez a Asaf y a sus parientes
para que dieran gracias al Señor de esta manera:\footnote{\textbf{16:7}
  ``De esta manera'': implícito. Lo que sigue es una selección de los
  salmos 105, 96, 107 y 106.}

\bibverse{8} Denle gracias al Señor, adoren su naturaleza maravillosa,
¡hagan saber lo que ha hecho!

\bibverse{9} Cántenle, canten sus alabanzas; cuéntenle a todos las
grandes cosas que ha hecho.

10 Enorgullézcanse de su carácter santo; ¡alégrense todos los que se
acercan al Señor!

\bibverse{11} Busquen al Señor y su fuerza; busquen siempre estar con
él.

\bibverse{12} Recuerden todas las maravillas que ha hecho, los milagros
que ha realizado y los juicios que ha pronunciado,

\bibverse{13} descendientes de Israel, hijos de Jacob, su pueblo
elegido.

\bibverse{14} Él es el Señor, nuestro Dios, sus juicios cubren toda la
tierra.

\bibverse{15} Él se acuerda de su acuerdo para siempre, la promesa que
hizo dura mil generaciones

\bibverse{16} el acuerdo que hizo con Abrahán, el voto que hizo a Isaac.

\bibverse{17} El Señor lo confirmó legalmente con Jacob, hizo este
acuerdo eterno con Israel

18 diciendo: ``Les daré la tierra de Canaán para que la posean''.

\bibverse{19} Dijo esto cuando sólo eran unos pocos, un pequeño grupo de
extranjeros en la tierra.

\bibverse{20} Iban de un país a otro, de un reino a otro.

\bibverse{21} No permitió que nadie los tratara mal; advirtió a los
reyes que los dejaran en paz:

\bibverse{22} ``¡No toquen a mi pueblo elegido, no hagan daño a mis
profetas!''

\bibverse{23} ¡Cántenle al Señor! ¡Que toda la tierra le cante al Señor!
¡Que cada día todos oigan de su salvación!

\bibverse{24} Anuncien sus actos gloriosos entre las naciones, las
maravillas que hace entre todos los pueblos.

\bibverse{25} Porque el Señor es grande y merece la mejor alabanza. Él
debe ser respetado con temor por encima de todos los dioses.

\bibverse{26} Porque todos los dioses de las demás naciones son ídolos,
pero el Señor hizo los cielos.

\bibverse{27} Suyos son el esplendor y la majestuosidad; en su santuario
hay poder y gloria.

\bibverse{28} Reconozcan al Señor, naciones del mundo, dénle la gloria y
el poder.

\bibverse{29} Dénle al Señor la gloria que se merece; traigan una
ofrenda y preséntense ante él. Adoren al Señor en su magnífica santidad.

\bibverse{30} Que todo el mundo en la tierra tiemble ante su presencia.
El mundo se mantiene unido con firmeza; no puede romperse.

\bibverse{31} Que los cielos canten de alegría, que la tierra se alegre.
Digan a las naciones: ``¡El Señor está al mando!''

\bibverse{32} ¡Que el mar y todo lo que hay en él griten de alabanza!
Que los campos y todo lo que hay en ellos celebren;

\bibverse{33} Que todos los árboles del bosque canten de alegría, porque
él viene a juzgar la tierra.

\bibverse{34} Denle gracias al Señor, porque es bueno. Su amor es
eterno.

\bibverse{35} Griten: ``¡Sálvanos, Señor, Dios nuestro! Reúnenos de
entre las naciones, rescátanos, para que podamos darte gracias y alabar
lo magnífico y santo que eres''.

\bibverse{36} ¡Qué maravilloso es el Señor, el Dios de Israel, que vive
por los siglos de los siglos! Entonces todo el pueblo dijo: ``¡Amén!'' y
``¡Alaben al Señor!''.

\bibverse{37} Entonces David se aseguró de que Asaf y sus hermanos
ministraran continuamente ante el Arca del Pacto del Señor, realizando
los servicios que fueran necesarios cada día, \bibverse{38} así como
Obed-edom y sus sesenta y ocho parientes. Obed-edom, hijo de Jedutún, y
Hosa, eran porteros. \bibverse{39} David puso al sacerdote Sadoc y a sus
compañeros sacerdotes a cargo del Arca del Señor en el lugar alto de
Gabaón \bibverse{40} para que presentaran holocaustos al Señor en el
altar de los holocaustos, por la mañana y por la tarde, según todo lo
que estaba escrito en la ley del Señor que él había ordenado seguir a
Israel. \bibverse{41} Los acompañaban Hemán, Jedutún y el resto de los
elegidos e identificados por su nombre para dar gracias al Señor, porque
``su amor confiable es eterno''. \bibverse{42} Hemán y Jedutún usaron
sus trompetas y címbalos para hacer música que acompañara los cantos de
Dios. Los hijos de Jedutún custodiaban la puerta. \bibverse{43} Luego
todo el pueblo se fue a su casa, y David fue a bendecir a su familia.

\hypertarget{section-16}{%
\section{17}\label{section-16}}

\bibverse{1} Una vez que David se instaló en su palacio, habló con el
profeta Natán. ---Mira---le dijo David---''¡Vivo en un palacio de cedro
mientras que el Arca del Pacto del Señor se guarda en una tienda!''.

\bibverse{2} ``Haz lo que creas que debes hacer, porque el Dios está
contigo'', respondió Natán.

\bibverse{3} Pero esa noche Dios le dijo a Natán: \bibverse{4} ``Ve y
habla con mi siervo David. Dile que esto es lo que dice el Señor: No
debes construir una casa para que yo viva en ella. \bibverse{5} No he
vivido en una casa desde que saqué a Israel de Egipto\footnote{\textbf{17:5}
  ``De Egipto'': implícito. Estas palabras no aparecen en el texto
  hebreo.} hasta ahora. He vivido en tiendas, moviéndome de un lugar a
otro. \bibverse{6} Pero en todos esos viajes con todo Israel nunca le
pregunté a ningún jefe israelita al que le hubiera ordenado cuidar de mi
pueblo: ``¿Por qué no has construido una casa de cedro para mí?''
\bibverse{7} Entonces, ve y dile a mi siervo David que esto es lo que
dice el Señor Todopoderoso. Fui yo quien te sacó del campo, del cuidado
de las ovejas, para convertirte en jefe de mi pueblo Israel.
\bibverse{8} He estado contigo dondequiera que hayas ido. He derribado a
todos tus enemigos delante de ti, y haré que tu reputación sea tan
grande como la de las personas más famosas de la tierra. \bibverse{9}
Elegiré un lugar para mi pueblo Israel. Allí los asentaré y ya no serán
molestados. Los malvados no los perseguirán como antes, \bibverse{10}
desde que puse jueces a cargo de mi pueblo. Derrotaré a todos sus
enemigos.

También quiero dejar claro que yo, el Señor, les construiré una
casa.\footnote{\textbf{17:10} En otras palabras, el Señor construiría
  una ``casa'' para David en el sentido de establecer una dinastía real.}
\bibverse{11} Porque cuando llegues al final de tu vida y te unas a tus
antepasados en la muerte, llevaré al poder a uno de tus descendientes, a
uno de tus hijos, y me aseguraré de que su reino tenga éxito.
\bibverse{12} Él será quien me construya una casa, y me aseguraré de que
su reino dure para siempre. \bibverse{13} Yo seré un padre para él, y él
será un hijo para mí. Nunca le quitaré mi bondad y mi amor, como hice
con el que gobernó antes que tú. \bibverse{14} Lo pondré al frente de mi
casa y de mi reino para siempre, y su dinastía durará para siempre''.
\bibverse{15} Esto es lo que Natán le explicó a David, todo lo que se le
dijo en esta revelación divina.

\bibverse{16} Entonces el rey David fue y se sentó en presencia del
Señor. Oró: ``¿Quién soy yo, Señor Dios, y qué importancia alguna tiene
mi familia, para que me hayas traído hasta este lugar? \bibverse{17}
Dios, hablas como si esto fuera poco a tus ojos, y también has hablado
del futuro de mi casa, de la dinastía de mi familia.\footnote{\textbf{17:17}
  ``La dinastía de mi familia'': explica el significado de ``casa'' en
  este contexto.} Tú también me ves como alguien muy importante, Señor
Dios.

\bibverse{18} ¿Qué más puedo decir yo, David, para que me honres de esta
manera? ¡Tú conoces muy bien a tu siervo! \bibverse{19} Señor, haces
todo esto por mí, tu siervo, y porque es lo que quieres: hacer todas
estas cosas increíbles y que la gente las conozca.

\bibverse{20} Señor, realmente no hay nadie como tú; no hay otro Dios,
sólo tú. Nunca hemos oído hablar de ningún otro. \bibverse{21} ¿Quién
más es tan afortunado como tu pueblo Israel? ¿A quién más en la tierra
fue Dios a redimir para hacer su propio pueblo? Te ganaste una
maravillosa reputación por todas las cosas tremendas y asombrosas que
hiciste al expulsar a otras naciones ante tu pueblo cuando lo redimiste
de Egipto. \bibverse{22} Hiciste tuyo a tu pueblo Israel para siempre, y
tú, Señor, te has convertido en su Dios.

\bibverse{23} Así que ahora, Señor, haz que lo que has dicho de mí y de
mi casa se cumpla y dure para siempre. Por favor, haz lo que has
prometido, \bibverse{24} y que tu verdadera naturaleza sea reconocida y
honrada para siempre, y que la gente declare: ``¡El Señor Todopoderoso,
el Dios de Israel, es el Dios de Israel! Que la casa de tu siervo David
siga estando en tu presencia.

\bibverse{25} Tú, Dios mío, me has explicado a mí, tu siervo, que me
construirás una casa. Por eso tu siervo ha tenido el valor de orar a ti.
\bibverse{26} Porque tú, Señor, eres Dios. Tú eres quien ha prometido
todas estas cosas buenas a tu siervo. \bibverse{27} Así que ahora, por
favor, bendice la casa de tu siervo para que continúe en tu presencia
para siempre. Porque cuando bendices, Señor, queda bendecida para
siempre.''

\hypertarget{section-17}{%
\section{18}\label{section-17}}

\bibverse{1} Tiempo después, David derrotó a los filisteos y los
sometió, y capturó Gat y sus ciudades cercanas a los filisteos.
\bibverse{2} David también derrotó a los moabitas, sometiéndolos y
obligándolos a pagar impuestos.

\bibverse{3} Luego David derrotó a Hadad-Ezer, rey de Soba, cerca de
Hamat, cuando intentaba imponer su control a lo largo del río Éufrates.
\bibverse{4} David le capturó 1.000 carros, 7.000 auriculares y 20.000
soldados de a pie. David ató a todos los caballos de los carros, pero
guardó lo suficiente para 100 carros.

\bibverse{5} Cuando los arameos de Damasco vinieron a ayudar a
Hadad-Ezer, rey de Soba, David mató a 22.000 de ellos. \bibverse{6}
David puso fuerzas\footnote{\textbf{18:6} En el texto hebreo no se
  especifica qué colocó David. Según el texto, parece tratarse de
  unidades del ejército o guarniciones, como sugieren las traducciones
  de la Septuaginta y la Vulgata, y se confirma en el pasaje paralelo de
  2 Samuel 8:6.} en la ciudad aramea de Damasco, y también los sometió y
les exigió el pago de impuestos. El Señor le dio a David victorias
dondequiera que fuera.

\bibverse{7} David tomó los escudos de oro que llevaban los oficiales de
Hadad-Ezer y los llevó a Jerusalén. \bibverse{8} David también tomó una
gran cantidad de bronce de Tibhat y de Cun, ciudades que habían
pertenecido a Hadad-Ezer. Salomón utilizó ese bronce para hacer el mar
de bronce, las columnas y los diversos objetos de bronce.\footnote{\textbf{18:8}
  Objetos utilizados en el Templo.}

\bibverse{9} Cuando Tou, rey de Hamat, se enteró de que David había
destruido todo el ejército de Hadad-Ezer, rey de Soba, \bibverse{10}
envió a su hijo Adoram donde David para que se hiciera amigo de él y lo
felicitara por su victoria en la batalla sobre Hadad-Ezer. Tou y
Hadad-Ezer habían estado a menudo en guerra. Adoram trajo regalos de
oro, plata y bronce. \bibverse{11} El rey David dedicó estos regalos al
Señor, junto con la plata y el oro que había tomado de todas las
naciones siguientes: Edom, Moab, los amonitas, los filisteos y los
amalecitas.

\bibverse{12} Abisai,\footnote{\textbf{18:12} En el pasaje paralelo de 2
  Samuel 8:13 se atribuye a David esta victoria.} hijo de Sarvia, mató a
18.000 edomitas en el Valle de la Sal. \bibverse{13} Estableció puestos
militares en Edom, y todos los edomitas se sometieron a David. El Señor
le dio a David victorias dondequiera que fuera. \bibverse{14} David
gobernó sobre todo Israel. Hizo lo que era justo y correcto para todo su
pueblo. \bibverse{15} Joab, hijo de Sarvia,\footnote{\textbf{18:15}
  Zeruiah era la hermana de David (2:16).} era el comandante del
ejército, mientras que Josafat, hijo de Ahilud, llevaba los registros
oficiales. \bibverse{16} Sadoc, hijo de Ahitob, y Ahimelec, hijo de
Abiatar, eran los sacerdotes, mientras que Savsha era el secretario.
\bibverse{17} Benaía, hijo de Joiada, estaba a cargo de los queretanos y
peletanos;\footnote{\textbf{18:17} ``Los queretanos y peletanos'': eran
  la guardia del rey (2 Samuel 15:18).} y los hijos de David estaban al
lado del rey, sirviendo como sus principales funcionarios.

\hypertarget{section-18}{%
\section{19}\label{section-18}}

\bibverse{1} Algún tiempo después, Nahas, rey de los amonitas, murió y
su hijo lo sucedió. \bibverse{2} Entonces David dijo: ``Seré bondadoso
con Hanún, hijo de Nahas, porque su padre fue bondadoso conmigo''. Así
que David envió mensajeros para consolarle por la muerte de su padre.
Los embajadores de David llegaron a la tierra de los amonitas y fueron a
consolar a Hanún. \bibverse{3} Pero los príncipes amonitas le dijeron a
Hanún: ``¿De verdad crees que David honra a tu padre enviándote a estos
hombres para consolarte? ¿Acaso no crees que han venido sólo a espiar la
tierra para encontrar la manera de conquistarla?'' \bibverse{4} Entonces
Hanún detuvo a los embajadores de David y los mandó a afeitar, y además
les cortó la túnica a la altura de las nalgas.\footnote{\textbf{19:4}
  Para humillarlos y avergonzarlos, y para enviar un mensaje de desafío
  a David.} Entonces los envió de vuelta.

\bibverse{5} Luego informaron a David de lo que había sucedido con estos
hombres. Entonces David envió mensajeros a los hombres para decirles:
``Quédense en Jericó hasta que les crezca la barba, y entonces podrán
regresar''.

\bibverse{6} Entonces los amonitas se dieron cuenta de que realmente
habían sido ofensivos con David. Así que Hanún y los amonitas enviaron
mil talentos de plata para contratar carros y sus conductores de
Aram-naharaim, Aram-maaca y Soba. \bibverse{7} También contrataron
32.000 carros y al rey de Maaca con su ejército. Vinieron a acampar
cerca de Medeba. Los amonitas también fueron llamados desde sus ciudades
y se prepararon para la batalla.

\bibverse{8} Cuando David se enteró de esto, envió a Joab y a todo el
ejército a enfrentarlos. \bibverse{9} Los amonitas establecieron sus
líneas de batalla cerca de la entrada de la ciudad, mientras que los
otros reyes que se les habían unido tomaron posiciones en los campos
abiertos.

\bibverse{10} Joab se dio cuenta de que tendría que luchar tanto delante
como detrás de él, así que escogió algunas de las mejores tropas de
Israel y se puso al frente de ellas para dirigir el ataque a los
arameos. \bibverse{11} Puso al resto del ejército bajo el mando de
Abisai, su hermano. Debían atacar a los amonitas. \bibverse{12} Joab le
dijo: ``Si los arameos son más fuertes que yo, ven a ayudarme. Si los
amonitas son más fuertes que tú, yo vendré a ayudarte. \bibverse{13} Sé
valiente y lucha lo mejor que puedas por nuestro pueblo y las ciudades
de nuestro Dios. Que el Señor haga lo que considere bueno''.

\bibverse{14} Joab atacó con sus fuerzas a los arameos y éstos huyeron
de él. \bibverse{15} Cuando los amonitas vieron que los arameos habían
huido, también huyeron de Abisai, el hermano de Joab, y se retiraron a
la ciudad. Entonces Joab regresó a Jerusalén.

\bibverse{16} En cuanto los arameos vieron que habían sido derrotados
por los israelitas, enviaron a buscar refuerzos del otro lado del río
Éufrates, bajo el mando de Sobac, comandante del ejército de Hadad-Ezer.

\bibverse{17} Cuando le informaron de esto a David, reunió a todo
Israel. Atravesó el Jordán y se acercó al ejército arameo, poniendo sus
fuerzas en línea de batalla contra ellos. Cuando David entró en combate
con ellos, ellos lucharon con él. \bibverse{18} Pero el ejército arameo
huyó de los israelitas, y David mató a 7.000 aurigas y 40.000 soldados
de infantería, así como a Sobac, su comandante. \bibverse{19} Cuando los
aliados de Hadad-Ezer se dieron cuenta de que habían sido derrotados por
Israel, hicieron la paz con David y se sometieron a él. Como resultado,
los arameos no quisieron ayudar más a los amonitas.

\hypertarget{section-19}{%
\section{20}\label{section-19}}

\bibverse{1} En primavera, en la época del año en que los reyes salen a
hacer la guerra, Joab dirigió el ejército israelita en los ataques
contra el país de los amonitas, asediando también Rabá. Sin embargo,
David se quedó en Jerusalén. Joab atacó Rabá y la destruyó.

\bibverse{2} David tomó la corona de la cabeza de su ídolo
Milcom.\footnote{\textbf{20:2} ``Milcom'': o ``su rey.''} Era de oro y
estaba engastado con gemas. Pesaba un talentob y fue colocado sobre la
cabeza de David. David también tomó una gran cantidad de botín de la
ciudad. \bibverse{3} David hizo trabajar a la gente de allí con sierras,
picos de hierro y hachas. También hizo lo mismo con todas las ciudades
amonitas. Luego David y todo su ejército regresaron a Jerusalén.

\bibverse{4} Algún tiempo después de esto estalló un conflicto con los
filisteos en Gezer. Pero entonces Sibecai de Husa mató a Sipai, un
descendiente de los refaítas,\footnote{\textbf{20:4} ``Refaim'': una
  raza de gigantes. Una palabra similar se utiliza en 20:8.} y los
filisteos se vieron obligados a someterse.

\bibverse{5} En otra batalla con los filisteos, Elhanán, hijo de Jair,
mató a Lahmi, hermano de Goliat de Gat. El asta de su lanza era tan
gruesa como una vara de tejedor.

\bibverse{6} En otra batalla en Gat, había un hombre gigantesco, que
tenía seis dedos en cada mano y seis dedos en cada pie, haciendo un
total de veinticuatro. También él descendía de los gigantes.
\bibverse{7} Pero cuando insultó a Israel, Jonatán, hijo de Simea,
hermano de David, lo mató. \bibverse{8} Estos eran los descendientes de
los gigantes en Gat, pero todos fueron muertos por David y sus hombres.

\hypertarget{section-20}{%
\section{21}\label{section-20}}

\bibverse{1} Satanás interfirió para causar problemas a Israel. Entonces
provocó a David para que hiciera un censo de Israel. \bibverse{2} David
les dijo a Joab y a los comandantes del ejército: ``Vayan a contar a los
israelitas desde Beerseba hasta Dan. Luego infórmenme para que tenga un
número total''.

\bibverse{3} Pero Joab respondió: ``Que el Señor multiplique su pueblo
cien veces. Su Majestad, ¿no son todos sus súbditos? ¿Por qué quieres
hacer esto? ¿Por qué culparás a Israel?''

\bibverse{4} Pero el rey se mostró inflexible, así que Joab se marchó y
recorrió todo Israel. Finalmente regresó a Jerusalén, \bibverse{5} y le
dio a David el número de personas censadas. En Israel había 1.100.000
hombres combatientes que podían manejar una espada, y 470.000 en Judá.
\bibverse{6} Sin embargo, Joab no incluyó a Leví y Benjamín en el total
del censo, porque no estaba de acuerdo con lo que el rey había ordenado.
\bibverse{7} El Señor consideró que el censo era algo malo y castigó a
Israel por ello.

\bibverse{8} Entonces David le dijo a Dios: ``He cometido un terrible
pecado al hacer esto. Por favor, quita la culpa de tu siervo, porque he
sido muy estúpido''.

\bibverse{9} El Señor le dijo a Gad, el vidente de David, \bibverse{10}
``Ve y dile a David que esto es lo que dice el Señor: `Te doy tres
opciones. Elige una de ellas, y eso es lo que te haré'\,''.

\bibverse{11} Gad fue y le dijo a David: ``Esto es lo que dice el Señor:
`Elige: \bibverse{12} o tres años de hambre; o tres meses de
devastación, huyendo de las espadas de tus enemigos; o tres días de la
espada del Señor, es decir, tres días de plaga en la tierra, con un
ángel del Señor causando la destrucción en todo Israel'. Ahora tienes
que decidir cómo debo responder al que me ha enviado''.

\bibverse{13} David respondió a Gad: ``¡Esta es una situación terrible
para mí! Por favor, deja que el Señor decida mi castigo,\footnote{\textbf{21:13}
  ``Deja que el Señor decida mi castigo'': Literalmente, ``déjame caer
  en las manos del Señor''. También al final del verso, ``no me dejes
  caer en manos humanas.''} porque es muy misericordioso. No permitas
que la gente me castigue''.

\bibverse{14} Entonces el Señor lanzó una plaga sobre Israel, y murieron
70.000 israelitas. \bibverse{15} Dios también envió un ángel para
destruir Jerusalén. Pero justo cuando el ángel estaba a punto de
destruirla, el Señor lo vio, y renunció a causar tal desastre. Le dijo
al ángel destructor: ``Es suficiente. Ya puedes parar''. Justo en ese
momento el ángel del Señor estaba junto a la era de Ornán el jebuseo.

\bibverse{16} Cuando David levantó la vista y vio al ángel del Señor de
pie entre la tierra y el cielo, con su espada desenvainada extendida
sobre Jerusalén, David y los ancianos, vestidos de saco, cayeron sobre
sus rostros. \bibverse{17} David le dijo a Dios: ``¿No fui yo quien
ordenó el censo del pueblo? Yo soy el que ha pecado y actuado con
maldad. Pero estas ovejas, ¿qué han hecho? Señor, Dios mío, por favor,
castígame a mí y a mi familia, pero no castigues a tu pueblo con esta
plaga''.

\bibverse{18} Entonces el ángel del Señor le dijo a Gad que le dijera a
David que fuera a construir un altar al Señor en la era de Ornán el
jebuseo. \bibverse{19} Así que David fue e hizo lo que Gad le había
dicho en nombre del Señor.

\bibverse{20} Ornán estaba ocupado trillando trigo. Se volvió y vio al
ángel; y sus cuatro hijos que estaban con él fueron a esconderse.
\bibverse{21} Cuando llegó David, Ornán se asomó y vio a David. Abandonó
la era y se inclinó ante David con el rostro en tierra.

\bibverse{22} David le dijo a Ornán: ``Por favor, déjame la era. La
compraré por su precio completo. Así podré construir aquí un altar al
Señor para que cese la plaga del pueblo''.

\bibverse{23} ``Tómala, y tu majestad podrá hacer lo que quiera con
ella'', le dijo Ornán a David. ``Puedes tener los bueyes para los
holocaustos, las tablas de trillar para la leña y el trigo para la
ofrenda de grano. Te lo daré todo''.

\bibverse{24} ``No, insisto, pagaré el precio completo'', respondió el
rey David. ``No tomaré para el Señor lo que es tuyo ni presentaré
holocaustos que no me costaron nada.''

\bibverse{25} Así que David pagó a Ornán seiscientos siclos de oro por
el lugar.

\bibverse{26} David construyó allí un altar al Señor y presentó
holocaustos y ofrendas de amistad. Invocó al Señor en oración, y el
Señor le respondió con fuego del cielo sobre el altar del holocausto.
\bibverse{27} Entonces el Señor le dijo al ángel que volviera a enfundar
su espada.

\bibverse{28} Cuando David vio que el Señor le había respondido en la
era de Ornán el jebuseo, ofreció allí sacrificios. \bibverse{29} En
aquel tiempo, la tienda del Señor que Moisés había hecho en el desierto
y el altar del holocausto estaban en el lugar alto de Gabaón.
\bibverse{30} Pero David no quiso ir allí a pedir la voluntad de
Dios,\footnote{\textbf{21:30} ``Pedir la voluntad de Dios'':
  Literalmente, ``preguntar a Dios.''} porque tenía miedo de la espada
del ángel del Señor.

\hypertarget{section-21}{%
\section{22}\label{section-21}}

\bibverse{1} Entonces David dijo: ``Aquí estará la casa del Señor Dios,
y este es el lugar para el altar de los holocaustos para Israel.''
\bibverse{2} Entonces David dio órdenes de convocar a los extranjeros
que vivían en la tierra de Israel, y asignó a canteros para que
prepararan piedras labradas para construir la casa de Dios. \bibverse{3}
David proporcionó mucho hierro para hacer los clavos de las puertas de
entrada y de los soportes, así como más bronce del que se podía pesar.
\bibverse{4} Proporcionó más troncos de cedro de los que se podían
contar, porque la gente de Sidón y Tiro le había traído a David una
enorme cantidad de troncos de cedro.

\bibverse{5} David se dijo: ``Mi hijo Salomón es todavía joven e
inexperto, y la casa que va a construir para el Señor debe ser realmente
magnífica, famosa y gloriosa en todo el mundo. Tengo que empezar a
prepararla''. Así que David se aseguró de tener listos muchos materiales
de construcción antes de morir.

\bibverse{6} Luego mandó llamar a su hijo Salomón y le encargó que
construyera una casa para el Señor, el Dios de Israel. \bibverse{7}
David le dijo a Salomón: ``Hijo mío, siempre había querido construir una
casa para honrar al Señor, mi Dios. \bibverse{8} Pero el Señor me dijo:
'Has derramado mucha sangre y has participado en muchas guerras. No
debes construir una casa para honrarme porque te he visto derramar mucha
sangre en la tierra. \bibverse{9} Pero tendrás un hijo que será un
hombre de paz. Le daré la paz de todos sus enemigos en las naciones de
alrededor. Salomón será su nombre, y concederé paz y tranquilidad a
Israel durante su reinado. \bibverse{10} Él es quien construirá una casa
para honrarme. Él será mi hijo, y yo seré su padre. Y me aseguraré de
que el trono de su reino sobre Israel dure para siempre''.

\bibverse{11} Ahora, hijo mío, que el Señor te acompañe para que logres
construir la casa del Señor, tu Dios, tal como él dijo que lo harías.
\bibverse{12} Que el Señor te dé inteligencia y entendimiento cuando te
ponga al frente de Israel, para que cumplas la ley del Señor, tu Dios.
\bibverse{13} Entonces tendrás éxito, siempre y cuando sigas las leyes y
los reglamentos que el Señor, a través de Moisés, le ordenó a Israel.
¡Sé fuerte y valiente! ¡No tengas miedo ni te desanimes!

\bibverse{14} Mira, me he tomado muchas molestias para proveer la casa
del Señor: 100.000 talentos de oro, 1.000.000 de talentos de plata, y
bronce y hierro, más de lo que se puede pesar. \bibverse{15} También he
proporcionado madera y piedra, pero tendrás que añadir más.
\bibverse{16} Tienes muchos trabajadores, como canteros, albañiles,
carpinteros y toda clase de artesanos del oro, la plata, el bronce y el
hierro, sin límite. Así que ponte en marcha, y que el Señor te
acompañe''.

\bibverse{17} David también ordenó a todos los dirigentes de Israel que
ayudaran a su hijo Salomón. \bibverse{18} ``¿No está el Señor Dios
contigo? ¿No te ha dado la paz en todas tus fronteras?'', preguntó.
``¿Por qué? Porque ha puesto a los habitantes de la tierra bajo mi
poder, y ahora están sometidos al Señor y a su pueblo. \bibverse{19}
Ahora, con toda tu mente y tu corazón, toma la decisión definitiva de
adorar siempre al Señor, tu Dios. Comienza a construir el santuario del
Señor Dios, Entonces podrás llevar el Arca del Pacto del Señor y las
cosas sagradas de Dios a la casa que se va a construir para honrar al
Señor.''

\hypertarget{section-22}{%
\section{23}\label{section-22}}

\bibverse{1} Cuando David envejeció, habiendo vivido una larga vida,
nombró a su hijo Salomón rey de Israel. \bibverse{2} También convocó a
todos los jefes de Israel, a los sacerdotes y a los levitas.
\bibverse{3} Se contaron los levitas mayores de treinta años, y fueron
38.000 en total. \bibverse{4} ``De ellos, 24.000 estarán a cargo de las
obras de la casa del Señor, mientras que 6.000 serán oficiales y
jueces'', instruyó David. \bibverse{5} ``Y 4.000 serán porteros,
mientras que 4.000 alabarán al Señor usando los instrumentos musicales
que he proporcionado para la alabanza''.

\bibverse{6} David los dividió en secciones correspondientes a Los hijos
de Levi: Gersón, Coat y Merari.

\bibverse{7} Los hijos de Gersón: Ladán y Simei. \bibverse{8} Los hijos
de Ladan: Jehiel (el jefe de familia), Zetham y Joel, tres en total.
\bibverse{9} Los hijos de Simei: Selomit, Haziel y Harán, tres en total.
Estos eran los jefes de las familias de Ladán. \bibverse{10} Los hijos
de Simei: Jahat, Ziza,\footnote{\textbf{23:10} ``Ziza'': según la
  Septuaginta y la Vulgata, en hebreo se lee ``Zina'' (pero nótese el
  siguiente verso).} Jeús y Bería, cuatro en total. \bibverse{11} Jahat
(el jefe de familia), y Ziza (el segundo); pero como Jeús y Bería no
tuvieron muchos hijos, fueron contados como una sola familia.

\bibverse{12} Los hijos de Coat: Amram, Izhar, Hebrón y Uziel-un total
de cuatro. \bibverse{13} Los hijos de Amram: Aarón y Moisés. Aarón
estaba dedicado al servicio con las cosas más sagradas, para que él y
sus hijos presentaran siempre ofrendas al Señor, y ministraran ante él,
y dieran bendiciones en su nombre para siempre. \bibverse{14} En cuanto
a Moisés, el hombre de Dios, sus hijos estaban incluidos en la tribu de
Leví. \bibverse{15} Los hijos de Moisés: Gersón y Eliezer. \bibverse{16}
Los hijos de Gersón: Sebuel (el jefe de familia). \bibverse{17} Los
hijos de Eliezer: Rehabías (el jefe de familia). Eliezer no tuvo más
hijos, pero Rehabías tuvo muchos hijos. \bibverse{18} Los hijos de
Izhar: Selomit (el jefe de familia). \bibverse{19} Los hijos de Hebrón:
Jeria (el jefe de familia), Amarías (el segundo), Jahaziel (el tercero)
y Jecamán (el cuarto). \bibverse{20} Los hijos de Uziel: Micaías (el
jefe de familia) e Isías (el segundo).

\bibverse{21} Los hijos de Merari: Mahli y Musi. Los hijos de Mahli:
Eleazar y Cis. \bibverse{22} Eleazar murió sin tener hijos, sólo hijas.
Sus primos, Los hijos de Cis, se casaron con ellas. \bibverse{23} Los
hijos de Musi: Mahli, Eder y Jeremot, tres en total.

\bibverse{24} Estos eran los descendientes de Leví por familias, los
jefes de familia enumerados individualmente por su nombre: los que
tenían veinte años o más y servían en la casa del Señor. \bibverse{25}
Porque David dijo: ``El Señor, el Dios de Israel, ha dado la paz a su
pueblo, y vivirá en Jerusalén para siempre. \bibverse{26} Así que los
levitas ya no necesitan llevar la Tienda ni nada necesario para su
servicio''. \bibverse{27} De acuerdo con las instrucciones finales de
David, se contaron los levitas de veinte años o más. \bibverse{28} Su
misión era ayudar a los descendientes de Aarón en el servicio de la casa
del Señor. Eran responsables de los patios y las habitaciones, de la
limpieza de todas las cosas sagradas y del trabajo del servicio de la
casa de Dios. \bibverse{29} También eran responsables del pan de la
proposición que se ponía sobre la mesa, de la harina especial para las
ofrendas de grano, de los panes sin levadura, de la cocción, de la
mezcla y del manejo de todas las cantidades y medidas. \bibverse{30}
También debían ponerse de pie todas las mañanas para dar gracias y
alabar al Señor, y hacer lo mismo al atardecer, \bibverse{31} y siempre
que se presentaran holocaustos al Señor, ya fuera en los sábados, lunas
nuevas y días festivos. Debían servir regularmente ante el Señor según
el número que se les exigiera. \bibverse{32} Así, los levitas debían
cumplir con la responsabilidad de cuidar la Tienda del Encuentro y el
santuario, y con sus hermanos los descendientes de Aarón, servían a la
casa del Señor.

\hypertarget{section-23}{%
\section{24}\label{section-23}}

\bibverse{1} Los hijos de Aarón fueron colocados en divisiones de la
siguiente manera. Los hijos de Aarón eran Nadab, Abiú, Eleazar e Itamar.
\bibverse{2} Pero Nadab y Abiú murieron antes que su padre, y no
tuvieron hijos. Sólo Eleazar e Itamar continuaron como sacerdotes.

\bibverse{3} Con la ayuda de Sadoc, descendiente de Eleazar, y de
Itamar, descendiente de Ahimelec, David los colocó en divisiones según
sus funciones asignadas. \bibverse{4} Como los descendientes de Eleazar
tenían más jefes que los de Itamar, se dividieron así: dieciséis jefes
de familia de los descendientes de Eleazar, y ocho de los descendientes
de Itamar.

\bibverse{5} Se dividieron echando suertes, sin preferencia, porque
había oficiales del santuario y oficiales de Dios tanto de los hijos de
Eleazar como de los hijos de Itamar. \bibverse{6} Semaías hijo de
Netanel, un levita, era el secretario. Anotó los nombres y las
asignaciones en presencia del rey, de los funcionarios, del sacerdote
Sadoc, de Ahimelec hijo de Abiatar y de los jefes de familia de los
sacerdotes y levitas. Una familia de Eleazar y otra de Itamar fueron
elegidas por turno.

\bibverse{7} La primera suerte recayó en Joiarib. El segundo a Jedaías.
\bibverse{8} La tercera a Harim. El cuarto a Seorim. \bibverse{9} La
quinta a Malquías. La sexta a Mijamín. \bibverse{10} La séptima a Cos.
La octava a Abías. \bibverse{11} La novena a Jesúa. La décima por
Secanías. \bibverse{12} La undécima por Eliasib. La duodécima a Jacim.
\bibverse{13} La decimotercera por Hupah. La decimocuarta por Jeshebeab.
\bibverse{14} la decimoquinta por Bilga. El decimosexto a Immer.
\bibverse{15} El decimoséptimo a Hezir. El decimoctavo a Afisés.
\bibverse{16} la decimonovena a Petaías. El vigésimo a Hezequiel.
\bibverse{17} el vigésimo primero a Jaquín. El vigésimo segundo a Gamul.
18 el vigésimo tercero a Delaía. El vigésimo cuarto a Maazías.

19 Este era el orden en que cada grupo debía servir cuando entraba en la
casa del Señor, siguiendo el procedimiento que les había definido su
antepasado Aarón, según las instrucciones del Señor, el Dios de Israel.
\bibverse{20} Estos fueron el resto de los hijos de Leví: de Los hijos
de Amram: Shubael; de Los hijos de Shubael: Jehdeiah. \bibverse{21} Para
Rehabía, de sus hijos Isías (el primogénito). \bibverse{22} De los
Izharitas: Shelomoth;

de Los hijos de Shelomoth: Jahat. \bibverse{23} Los hijos de Hebrón:
Jeriah (el mayor), Amariah (el segundo), Jahaziel (el tercero) y Jecamán
(el cuarto). \bibverse{24} El hijo de Uziel: Miqueas; de Los hijos de
Miqueas: Shamir. \bibverse{25} El hermano de Micaías: Isías; de Los
hijos de Isías: Zacarías. \bibverse{26} Los hijos de Merari: Mahli y
Musi. El hijo de Jaaziah: Beno. \bibverse{27} Los hijos de Merari: de
Jaaziah: Beno, Shoham, Zaccur e Ibri. \bibverse{28} De Mahli: Eleazar,
que no tuvo hijos. \bibverse{29} De Cis: el hijo de Cis, Jerajmeel. 30
Los hijos de Musi: Mahli, Eder y Jerimot.

Estos eran los hijos de los levitas, según sus familias. \bibverse{31}
También echaron suertes de la misma manera que sus parientes los
descendientes de Aarón. Lo hicieron en presencia del rey David, de
Sadoc, de Ahimelec y de los jefes de familia de los sacerdotes y de los
levitas, tanto de los jefes de familia como de sus hermanos menores.

\hypertarget{section-24}{%
\section{25}\label{section-24}}

\bibverse{1} David y los líderes de los levitas\footnote{\textbf{25:1}
  ``Líderes de los levitas'': Muchas traducciones lo traducen como
  ``comandantes del ejército'', lo que parece una función extraña para
  ellos aquí. Sin embargo, la palabra también se utiliza para designar a
  los líderes de una reunión de levitas (véase, por ejemplo, Números
  4:3; Números 8:24-25). Véase también 15:16 en este libro para una
  descripción similar.} eligió a hombres de las familias de Asaf, Hemán
y Jedutún para que sirvieran profetizando acompañados de liras, arpas y
címbalos. Esta es la lista de los que realizaron este servicio:
\bibverse{2} De los hijos de Asaf: Zaccur, José, Netanías y Asarela.
Estos hijos de Asaf estaban bajo la supervisión de Asaf, quien
profetizaba bajo la supervisión del rey. \bibverse{3} De los hijos de
Jedutún: Gedalías, Zeri, Jesaías, Simei, Hasabías y Matatías, seis en
total, bajo la supervisión de su padre Jedutún, que profetizaban
acompañados del arpa, dando gracias y alabando al Señor. \bibverse{4} De
los hijos de Hemán: Buquías, Matanías, Uziel, Sebuel, Jerimot, Hananías,
Hanani, Eliatá, Giddalti, Romamti-ezer, Josbecasa, Maloti, Hotir y
Mahaziot. \bibverse{5} Todos estos hijos de Hemán, el vidente del rey,
le fueron dados por las promesas de Dios de honrarlo, pues Dios le dio a
Hemán catorce hijos y tres hijas. \bibverse{6} Todos ellos estaban bajo
la supervisión de sus padres para la música de la casa del SEÑOR con
címbalos, arpas y liras, para el servicio de la casa de Dios. Asaf,
Jedutún y Hemán estaban bajo la supervisión del rey. \bibverse{7} Junto
con sus parientes, todos ellos entrenados y hábiles en el canto al
SEÑOR, sumaban 288.

\bibverse{8} Echaron suertes para cualquier responsabilidad que
tuvieran, el menos importante igual al más importante, el maestro al
alumno. \bibverse{9} La primera suerte, que era para Asaf, recayó en
José, sus hijos y su hermano, 12 en total. La segunda recayó en
Gedalías, sus hijos y sus hermanos, 12 en total. \bibverse{10} La
tercera cayó en manos de Zacur, sus hijos y sus hermanos, 12 en total.
\bibverse{11} La cuarta cayó en manos de Izri, sus hijos y sus hermanos,
12 en total. \bibverse{12} La quinta cayó en manos de Netanías, sus
hijos y sus hermanos, 12 en total. \bibverse{13} La sexta cayó en manos
de Buquías, sus hijos y sus hermanos, 12 en total. \bibverse{14} La
séptima cayó en manos de Jesarela, sus hijos y sus hermanos, 12 en
total. \bibverse{15} La octava cayó en manos de Jesaías, sus hijos y sus
hermanos, 12 en total. \bibverse{16} La novena cayó en manos de
Matanías, sus hijos y sus hermanos, 12 en total. \bibverse{17} La décima
cayó en manos de Simei, sus hijos y sus hermanos, 12 en total.
\bibverse{18} La undécima cayó en manos de Azarel, sus hijos y sus
hermanos, 12 en total. \bibverse{19} La duodécima cayó en manos de
Hasabías, sus hijos y sus hermanos, 12 en total. \bibverse{20} La
decimotercera cayó en manos de Subael, sus hijos y sus hermanos, 12 en
total. \bibverse{21} El decimocuarto cayó en manos de Matatías, sus
hijos y sus hermanos, 12 en total. \bibverse{22} La decimoquinta cayó en
manos de Jerimot, sus hijos y sus hermanos, 12 en total. \bibverse{23}
La decimosexta cayó en manos de Hananías, sus hijos y sus hermanos, 12
en total. \bibverse{24} La decimoséptima cayó en manos de Josbecasa, sus
hijos y sus hermanos, 12 en total. \bibverse{25} La decimoctava cayó en
manos de Hanani

sus hijos y sus hermanos, 12 en total. \bibverse{26} La decimonovena
cayó en manos de Maloti, sus hijos y sus hermanos, 12 en total.
\bibverse{27} La vigésima cayó en manos de Eliata, sus hijos y sus
hermanos, 12 en total. \bibverse{28} El vigésimo primero cayó en manos
de Hotir, sus hijos y sus hermanos, 12 en total. \bibverse{29} El
vigésimo segundo cayó en manos de Gidalti, sus hijos y sus hermanos, 12
en total. \bibverse{30} El vigésimo tercero cayó en manos de Mahaziot,
sus hijos y sus hermanos, 12 en total. \bibverse{31} El vigésimo cuarto
cayó en manos de Romanti-Ezer, sus hijos y sus hermanos, 12 en total.

\hypertarget{section-25}{%
\section{26}\label{section-25}}

\bibverse{1} Esta es una lista de las divisiones de los porteros. De los
corasitas: Meselemías hijo de Coré, uno de Los hijos de Asaf.
\bibverse{2} Los hijos de Meselemías: Zacarías (el mayor), Jediael (el
segundo), Zebadías (el tercero), Jatniel (el cuarto), \bibverse{3} Elam
(el quinto), Johanán (el sexto) y Elioenai (el séptimo).

\bibverse{4} Los hijos de Obed-edom: Semaías (el mayor), Jozabad (el
segundo), Joa (el tercero), Sacar (el cuarto), Natanel (el quinto),
\bibverse{5} Ammiel (el sexto), Isacar (el séptimo) y Peuletai (el
octavo), porque Dios había bendecido a Obed-edom.

\bibverse{6} Semaías, hijo de Obed-edom, tenía hijos que eran líderes
capaces y tenían gran autoridad en la familia de su padre. \bibverse{7}
Los hijos de Semaías: Othni, Refael, Obed y Elzabad. Sus parientes,
Elihú y Semaquías, también eran hombres capaces. \bibverse{8} Todos
estos descendientes de Obed-edom, más sus hijos y nietos, un total de
sesenta y dos, eran hombres capaces, bien calificados para su servicio.

\bibverse{9} Los dieciocho hijos y hermanos de Meselemías también eran
hombres capaces.

\bibverse{10} Hosa, uno de los hijos de Merari, puso a Simri como líder
entre sus hijos, aunque no era el primogénito. \bibverse{11} Entre sus
otros hijos estaban Hilcías (el segundo), Tebalías (el tercero) y
Zacarías (el cuarto). El total de los hijos y parientes de Osa era de
trece.

\bibverse{12} Estas divisiones de los porteros, a través de sus jefes de
familia, servían en la casa del Señor, al igual que sus hermanos.
\bibverse{13} Cada puerta fue asignada por sorteo a diferentes familias,
la menos importante igual a la más importante.

\bibverse{14} La suerte de la puerta oriental recayó en
Meselemías.\footnote{\textbf{26:14} Véase 26:2. El hebreo aquí se lee
  ``Selemías.''} Entonces echaron suertes sobre su hijo Zacarías,
consejero sabio y perspicaz, y la suerte de la puerta del norte le
correspondió a él. \bibverse{15} La suerte de la puerta del sur
correspondió a Obed-edom, y la del almacén a sus hijos. \bibverse{16}
Supim y Hosa recibieron la puerta del oeste y la puerta de Salet en el
camino que sube. Estaban siempre vigiladas.\footnote{\textbf{26:16}
  Literalmente, ``guardia de guardia''. El significado es incierto.}
\bibverse{17} Había seis levitas de servicio cada día en la puerta
oriental, cuatro en la puerta norte, cuatro en la puerta sur y dos a la
vez en el almacén. \bibverse{18} Seis estaban de servicio todos los días
en la puerta oeste, cuatro en el camino principal y dos en el patio.

\bibverse{19} Estas eran las divisiones de los porteros de Los hijos de
Coré y Los hijos de Merari.

\bibverse{20} Otros levitas bajo el mando de Ahías estaban a cargo de
los tesoros de la casa de Dios y de los tesoros de lo que había sido
dedicado a Dios. \bibverse{21} De Los hijos de Ladán, que eran los
descendientes de los gersonitas a través de Ladán, y eran los jefes de
familia de Ladán el gersonita: Jehieli. \bibverse{22} Los hijos de
Jehieli, Zetam y su hermano Joel, estaban a cargo de los tesoros de la
casa del Señor.

\bibverse{23} De los amramitas, los izaritas, los hebronitas y los
uzielitas: \bibverse{24} Sebuel, descendiente de Gersón, hijo de Moisés,
era el principal encargado de los tesoros. \bibverse{25} Sus parientes
por parte de Eliezer fueron Rehabías, Jesaías, Joram, Zicri y Selomot.

\bibverse{26} Selomot y sus parientes estaban a cargo de todos los
tesoros de todo lo que había sido dedicado por el rey David, por los
jefes de familia que eran los comandantes de millares y de centenas, y
por los comandantes del ejército. \bibverse{27} Dedicaron una parte del
botín que habían ganado en la batalla para ayudar a mantener la casa del
Señor. \bibverse{28} Selomot y sus parientes también se encargaron de
las ofrendas dedicadas al Señor por Samuel el vidente, Saúl hijo de Cis,
Abner hijo de Ner y Joab hijo de Sarvia. Todas las ofrendas dedicadas
eran responsabilidad de Selomot y sus parientes.

\bibverse{29} De los izaritas: Quenanías y sus hijos recibieron
funciones externas como funcionarios y jueces sobre Israel.

\bibverse{30} De los hebronitas: Hasabías y sus parientes, 1.700 hombres
capaces, fueron puestos a cargo del Israel al oeste del Jordán,
responsables de todo lo relacionado con la obra del Señor y el servicio
del rey.

\bibverse{31} También de los hebronitas salió Jericó, el líder de los
hebronitas según las genealogías familiares. En el año cuarenta del
reinado de David se examinaron los registros, y se descubrieron hombres
de gran capacidad en Jazer de Galaad. \bibverse{32} Entre los parientes
de Jericó había 2.700 hombres capaces que eran líderes familiares. El
rey David los puso a cargo de las tribus de Rubén y Gad y de la media
tribu de Manasés. Eran responsables de todo lo relacionado con la obra
del Señor y el servicio del rey.

\hypertarget{section-26}{%
\section{27}\label{section-26}}

\bibverse{1} Esta es una lista de los israelitas, de los jefes de
familia, de los comandantes de millares y de los comandantes de
centenas, y de sus oficiales que servían al rey en todo lo relacionado
con las divisiones que estaban de servicio cada mes durante el año.
Había 24.000 hombres en cada división.

\bibverse{2} Al mando de la primera división para el primer mes, estaba
Jashobeam, hijo de Zabdiel. Tenía 24.000 hombres en su división.
\bibverse{3} Era descendiente de Fares y estaba a cargo de todos los
oficiales del ejército durante el primer mes.

\bibverse{4} Al mando de la división para el segundo mes estaba Dodai el
ahohita. Miclot era su jefe de división. Tenía 24.000 hombres en su
división.

\bibverse{5} El tercer comandante del ejército para el tercer mes era
Benaía, hijo del sacerdote Joiada. Era el jefe y había 24.000 hombres en
su división. \bibverse{6} Este era el mismo Benaía que era un gran
guerrero entre los Treinta, y estaba a cargo de los Treinta. Su hijo
Amizabad era el jefe de su división.

\bibverse{7} El cuarto, para el cuarto mes, era Asael, hermano de Joab.
Su hijo Zebadías fue su sucesor. Tenía 24.000 hombres en su división.

\bibverse{8} El quinto, para el quinto mes, era el comandante del
ejército Shamhuth el Izrahita. Tenía 24.000 hombres en su división.

\bibverse{9} El sexto, para el sexto mes, era Ira, hijo de Iqués de
Tecoa. Tenía 24.000 hombres en su división.

\bibverse{10} El séptimo, para el séptimo mes, era Heles el pelonita de
la tribu de Efraín. Tenía 24.000 hombres en su división.

\bibverse{11} El octavo, para el octavo mes, era Sibecai de Husa, de la
tribu de Zera. Tenía 24.000 hombres en su división.

\bibverse{12} El noveno, para el noveno mes, era Abiezer, de Anatot, de
la tribu de Benjamín. Tenía 24.000 hombres en su división.

\bibverse{13} El décimo, para el décimo mes, era Maharai de Netofa, de
la tribu de Zera. Tenía 24.000 hombres en su división.

\bibverse{14} El undécimo, para el undécimo mes, era Benaía, de Piratón,
de la tribu de Efraín. Tenía 24.000 hombres en su división.

\bibverse{15} El duodécimo, para el duodécimo mes, era Heldai de Netofa,
de la familia de Otoniel. Tenía 24.000 hombres en su división.

\bibverse{16} Esta es la lista de los jefes para las tribus de Israel:
para los rubenitas Eliezer, hijo de Zicri; para los simeonitas:
Sefatías, hijo de Maaca; \bibverse{17} para Leví: Hasabías, hijo de
Quemuel; para Aarón: Sadoc; \bibverse{18} para Judá: Eliú, hermano de
David; por Isacar Omri, hijo de Miguel; \bibverse{19} para Zabulón
Ismaías, hijo de Abdías; por Neftalí: Jerimot, hijo de Azriel;
\bibverse{20} por los efraimitas: Oseas, hijo de Azazías; por la media
tribu de Manasés Joel, hijo de Pedaías; \bibverse{21} para la media
tribu de Manasés en Galaad Iddo, hijo de Zacarías; por Benjamín:
Jaasiel, hijo de Abner; \bibverse{22} por Dan: Azarel, hijo de Jeroham.
Estos fueron los oficiales para las tribus de Israel.

\bibverse{23} David no censó a los hombres menores de veinte años porque
el Señor había dicho que haría a Israel tan numeroso como las estrellas
del cielo. 24 Joab, hijo de Sarvia, había comenzado el censo, pero no lo
terminó. Israel fue castigado a causa de este censo, y los resultados no
fueron registrados en la cuenta oficial del rey David.\footnote{\textbf{27:23}
  Sin embargo, se registraron de forma resumida. Véase 21:5.}

\bibverse{25} Azmavet, hijo de Adiel, estaba a cargo de los almacenes
del rey, mientras que Jonatán, hijo de Uzías, estaba a cargo de los del
campo, las ciudades, las aldeas y las torres de vigilancia.
\bibverse{26} Ezri, hijo de Quelub, estaba a cargo de los campesinos que
trabajaban la tierra. \bibverse{27} Simei, el ramatita, estaba a cargo
de las viñas. Zabdi el sifmita estaba a cargo del producto de las viñas
para las bodegas. \bibverse{28} Baal-Hanán el gederita estaba a cargo de
los olivos y los sicómoros de las colinas. Joás estaba a cargo de los
almacenes de aceite de oliva. \bibverse{29} Sitrai de Sarón estaba a
cargo del ganado en los pastos de Sarón. Safat, hijo de Adlai, estaba a
cargo del ganado en los valles. \bibverse{30} Obil el ismaelita estaba a
cargo de los camellos. Jehedías de Meronot estaba a cargo de los asnos.
\bibverse{31} Jaziz el agareno estaba a cargo de las ovejas y las
cabras. Todos estos eran funcionarios a cargo de lo que pertenecía al
rey David.

\bibverse{32} Jonatán, tío de David, era un consejero, un hombre
perspicaz y un escriba. Jehiel, hijo de Hacmoni, cuidaba de los hijos
del rey. \bibverse{33} Ahitofel era el consejero del rey y Husai, el
arquita, era el amigo del rey. \bibverse{34} Después de Ahitofel vino
Joiada, hijo de Benaía y de Abiatar. Joab era el comandante del ejército
real.

\hypertarget{section-27}{%
\section{28}\label{section-27}}

\bibverse{1} David convocó a Jerusalén a todos los dirigentes de Israel:
los jefes de las tribus, los comandantes de las divisiones del ejército
al servicio del rey, los comandantes de millares y los comandantes de
centenas, y los funcionarios encargados de todas las propiedades y el
ganado del rey y de sus hijos, junto con los funcionarios de la corte,
los guerreros y todos los mejores combatientes. \bibverse{2} El rey
David se puso en pie y dijo: ``¡Escúchenme, hermanos míos y pueblo! Yo
quería construir una casa como lugar de descanso para el Arca del Pacto
del Señor, como escabel para nuestro Dios. Así que hice planes para
construirla. \bibverse{3} Pero Dios me dijo: `No debes construir una
casa para honrarme, porque eres un hombre de guerra que ha derramado
sangre'.

\bibverse{4} Sin embargo, el Señor, el Dios de Israel, me eligió de
entre toda la familia de mi padre para ser rey de Israel para siempre.
Porque eligió a Judá como tribu principal, y de entre las familias de
Judá eligió a la familia de mi padre. De entre los hijos de mi padre se
complació en elegirme rey de todo Israel. \bibverse{5} De entre todos
mis hijos (porque el Señor me dio muchos) el Señor ha elegido a mi hijo
Salomón para que se siente en el trono y gobierne el reino del Señor,
Israel. \bibverse{6} Me dijo: ``Tu hijo Salomón es el que construirá mi
casa y mis atrios, porque lo he elegido como hijo mío, y yo seré su
padre. \bibverse{7} Me aseguraré de que su reino sea eterno si cumple
con mis mandamientos y normas como lo hace hoy.

\bibverse{8} Así que ahora, a la vista de todo Israel, de la asamblea
del Señor, y mientras Dios te escucha, asegúrate de obedecer todos los
mandamientos del Señor, tu Dios, para que sigas poseyendo esta buena
tierra y puedas transmitirla como herencia a tus descendientes para
siempre.

\bibverse{9} Salomón, hijo mío, conoce al Dios de tu padre. Sírvele con
total dedicación y con una mente dispuesta, porque el Señor examina cada
motivación y entiende la intención de cada pensamiento. Si lo buscas, lo
encontrarás; pero si lo abandonas, te rechazará para siempre.
\bibverse{10} Presta atención ahora, porque el Señor te ha elegido para
construir una casa para el santuario. Sé fuerte y haz el trabajo''.

\bibverse{11} Entonces David le dio a su hijo Salomón los planos del
pórtico del Templo, de sus edificios, de los almacenes, de las salas
superiores, de las salas interiores y de la sala para el ``lugar de
expiación''. \bibverse{12} También le dio todo lo que había planeado
para los atrios de la casa del Señor, para todas las habitaciones
circundantes, para los tesoros de la casa de Dios y de las cosas que
habían sido dedicadas. \bibverse{13} Además, le dio instrucciones sobre
las divisiones de los sacerdotes y de los levitas, para todo el trabajo
de servicio de la casa del Señor y para todo lo que se utilizaba para el
culto en la casa del Señor.

\bibverse{14} También estableció la cantidad de oro y plata que debía
emplearse en la fabricación de los diferentes objetos utilizados en todo
tipo de servicio,\footnote{\textbf{28:14} En los siguientes versos hay
  muchas repeticiones, por lo que la traducción se ha simplificado para
  mayor claridad.} \bibverse{15} el peso de los candelabros de oro y de
plata y de sus lámparas, según el uso de cada candelabro; \bibverse{16}
el peso del oro para cada mesa de los panes de la proposición, y el peso
de la plata para las mesas de plata, \bibverse{17} el peso del oro puro
para los tenedores, las jofainas y las copas; el peso de cada plato de
oro; el peso de cada cuenco de plata; \bibverse{18} el peso del oro
refinado para el altar del incienso; y, por último, los planos de un
carro de oro con querubines que despliegan sus alas, cubriendo el Arca
del Pacto del Señor. \bibverse{19} ``Todo esto está por escrito de la
mano del Señor, que me ha sido dado como instrucciones: cada detalle de
este plan'', dijo David.

\bibverse{20} Entonces David también le dijo a Salomón: ``¡Sé fuerte, sé
valiente y actúa! No tengas miedo ni te desanimes, porque el Señor, mi
Dios, está contigo. Él no te dejará ni te abandonará. Él se encargará de
que todo el trabajo para el servicio de la casa del Señor esté
terminado. \bibverse{21} Las divisiones de los sacerdotes y los levitas
están preparadas para todo el servicio de la casa de Dios. La gente
estará dispuesta a usar sus diferentes habilidades para ayudarte en todo
el trabajo; los funcionarios y todo el pueblo harán lo que tú les
digas.''

\hypertarget{section-28}{%
\section{29}\label{section-28}}

\bibverse{1} Entonces el rey David dijo a todos los allí reunidos: ``Mi
hijo Salomón, elegido sólo por Dios, es joven e inexperto, y el trabajo
a realizar es grande porque este Templo\footnote{\textbf{29:1}
  ``Templo'': la palabra también puede traducirse como ``palacio'' o
  ``fortaleza''} no será para el hombre, sino para el Señor Dios.
\bibverse{2} Con todos mis medios he provisto para la casa de mi Dios:
oro para los artículos de oro, plata para la plata, bronce para el
bronce, hierro para el hierro y madera para la madera; piedras de ónice
y piedras para los engastes: turquesa, piedras de diferentes colores,
toda clase de piedras preciosas; y mucho mármol.

\bibverse{3} Además, por mi devoción a la casa de mi Dios, ahora doy mi
fortuna personal de oro y plata, además de todo lo que he provisto para
esta santa casa. \bibverse{4} 3.000 talentos de oro -el oro de Ofir- y
7.000 talentos de plata refinada irán a cubrir las paredes de los
edificios, \bibverse{5} el oro para la orfebrería, y la plata para la
platería, y para todo el trabajo de los artesanos. ¿Quién quiere
comprometerse de buen grado a dar hoy al Señor?''

\bibverse{6} Dieron de buena gana: los jefes de familia, los
responsables de las tribus de Israel, los comandantes de millares y de
centenas, y los funcionarios encargados de la obra del rey. \bibverse{7}
Dieron para el servicio de la casa de Dios 5.000 talentos y 10.000
dáricos\footnote{\textbf{29:7} Un dárico era una moneda persa.} de oro,
10.000 talentos de plata, 18.000 talentos de bronce y 100.000 talentos
de hierro. \bibverse{8} Los que tenían piedras preciosas las entregaron
al tesoro de la casa del Señor, bajo la supervisión de Jehiel el
gersonita. \bibverse{9} El pueblo celebró porque sus líderes habían
estado tan dispuestos a dar al Señor, libremente y de todo corazón. El
rey David también se alegró mucho.

\bibverse{10} Entonces David alabó al Señor ante toda la asamblea:

``¡Alabado seas, Señor, Dios de Israel, nuestro padre, por los siglos de
los siglos! \bibverse{11} Señor, tuyos son la grandeza, el poder, la
gloria, el esplendor y la majestad, porque todo lo que hay en el cielo y
en la tierra es tuyo. Señor, tuyo es el reino, y eres admirado como
gobernante de todo. \bibverse{12} Las riquezas y el honor provienen de
ti y tú reinas de forma suprema. Tú posees el poder y la fuerza, y
tienes la capacidad de engrandecer a las personas y de dar fuerza a
todos.

\bibverse{13} Ahora, nuestro Dios, te damos gracias y te alabamos a ti y
a tu glorioso carácter. \bibverse{14} Pero, ¿quién soy yo y quién es mi
pueblo, para que seamos capaces de dar de tan buena gana? Porque todo lo
que tenemos viene de ti; sólo te devolvemos lo que tú nos has dado.
\bibverse{15} A tus ojos somos extranjeros y forasteros, como nuestros
antepasados. Nuestro tiempo aquí en la tierra pasa como una sombra, no
tenemos esperanza de quedarnos mucho tiempo.

\bibverse{16} Señor, Dios nuestro, incluso toda esta riqueza que hemos
proporcionado para construirte una casa para tu santo nombre proviene de
lo que tú das, y todo te pertenece. \bibverse{17} Yo sé, Dios mío, que
tú miras por dentro y te alegras cuando vivimos bien. Todo lo he dado de
buena gana y con un corazón honesto, y ahora he visto a tu pueblo aquí
dando felizmente y de buena gana para ti. \bibverse{18} Señor, el Dios
de Abrahán, de Isaac, de Israel y de nuestros antepasados, por favor,
mantén estos pensamientos y compromisos en la mente de tu pueblo para
siempre, y haz que permanezcan leales\footnote{\textbf{29:18} ``Haz que
  permanezcan leales ``: Literalmente, ``de corazón''} a ti.
\bibverse{19} Por favor, dale también a mi hijo Salomón el deseo de
cumplir de todo corazón tus mandamientos, decretos y estatutos, y de
hacer todo lo posible para construir tu Templo que yo he dispuesto.''

\bibverse{20} Entonces David dijo a todos los presentes: ``¡Alaben al
Señor, su Dios!'' Y todos alabaron al Señor, el Dios de sus padres. Se
inclinaron en reverencia ante el Señor y ante el rey.

\bibverse{21} Al día siguiente presentaron sacrificios y holocaustos al
Señor: mil toros, mil carneros y mil corderos, con sus libaciones y
abundantes sacrificios para todo Israel. \bibverse{22} Entonces comieron
y bebieron en presencia del Señor con gran alegría aquel día. Hicieron
rey por segunda vez a Salomón, hijo de David, y lo ungieron como
gobernante del Señor, y ungieron a Sadoc como sacerdote.

\bibverse{23} Así Salomón ocupó el trono del Señor como rey en lugar de
David, su padre. Tuvo éxito, y todos los israelitas le obedecieron.
\bibverse{24} Todos los funcionarios y guerreros, así como todos los
hijos del rey David, hicieron una promesa solemne de lealtad al rey
Salomón. \bibverse{25} El Señor hizo que Salomón fuera muy respetado en
todo Israel, y le dio mayor majestad real que la que se le había dado a
cualquier otro rey de Israel antes de él.

\bibverse{26} Así, David, hijo de Isaí, gobernó sobre todo Israel.
\bibverse{27} Gobernó sobre Israel cuarenta años: siete en Hebrón y
treinta y tres en Jerusalén. \bibverse{28} David murió a una buena edad,
habiendo vivido una larga vida bendecida con riquezas y honor. Entonces
su hijo Salomón tomó el relevo y gobernó en su lugar. \bibverse{29} Todo
lo que hizo el rey David, desde el principio hasta el final, está
escrito en las Actas de Samuel el Vidente, las Actas de Natán el Profeta
y las Actas de Gad el Vidente. \bibverse{30} Estos incluyen todos los
detalles de su reinado, su poder y lo que le sucedió a él, a Israel y a
todos los reinos de los países vecinos.
