\hypertarget{section}{%
\section{1}\label{section}}

\bibverse{1} Esta carta viene de parte de Simón Pedro, siervo y apóstol
de Jesucristo, quien la envía a los que participan con nosotros de la
preciosa fe en nuestro Dios y Salvador Jesucristo, el único que es
verdaderamente justo y bueno. \bibverse{2} Reciban todavía más gracia y
paz a medida que crecen en el conocimiento de Dios y de Jesús nuestro
Señor. \bibverse{3} Por su poder divino hemos recibido todas las cosas
necesarias para una vida cuyo centro es Dios. Esto sucede al conocerlo a
él, quien nos llamó a sí mismo por su propia gloria y bondad.

De este modo él nos ha entregado promesas maravillosas y preciosas.
\bibverse{4} Por medio de estas promesas podemos participar de la
naturaleza divina, deshacernos de la corrupción que producen los deseos
malos de este mundo. \bibverse{5} Por esa misma razón, ¡hagan todo lo
que puedan! A su fe en Dios agréguenle bondad; a la bondad,
conocimiento; \bibverse{6} al conocimiento, dominio propio; al dominio
propio, paciencia; a la paciencia, reverencia; \bibverse{7} a la
reverencia, aprecio por los hermanos creyentes; y a este aprecio, amor.

\bibverse{8} Cuanto más desarrollen estas cualidades, tanto más
productivos y útiles serán en su conocimiento de nuestro Señor
Jesucristo. \bibverse{9} Porque quien no tenga estas cualidades, es como
si estuviera mal de la vista, o ciego. Olvidan que han sido limpiados de
sus pecados pasados. \bibverse{10} Así que, hermanos y hermanas, estén
todos cada vez más determinados a ser verdaderamente los ``llamados y
escogidos.'' Y si hacen esto, nunca caerán\footnote{\textbf{1:10} Se ha
  debatido mucho sobre este versículo. La idea es que debemos hacer todo
  lo que podamos por alcanzar la salvación. No necesariamente nos lleva
  a la conclusión de que no podemos perder la salvación.}. \bibverse{11}
Recibirán una gran bienvenida al reino eterno de nuestro Señor y
Salvador Jesucristo.

\bibverse{12} Por eso siempre les recuerdo estas cosas, aunque ya
ustedes las saben, y están firmes en la verdad que tienen. \bibverse{13}
Pero aun así yo creo que es bueno animarlos y recordarles estas cosas
mientras viva. \bibverse{14} Sé que se acerca la hora en que tendré que
partir de esta vida, pues nuestro Señor Jesucristo me lo ha dicho.
\bibverse{15} Así que haré mi mejor esfuerzo para que aunque me vaya,
ustedes puedan siempre recordar estas cosas.

\bibverse{16} Nosotros no seguimos mitos inventados cuando les hablamos
sobre la venida poderosa de nuestro Señor Jesucristo, pues nosotros
mismos vimos su majestad\footnote{\textbf{1:16} Este texto también hace
  referencia a la Transfiguración.}. \bibverse{17} Él recibió honra y
gloria de Dios el Padre, cuando la voz de majestuosa gloria le habló y
anunció: ``Este es mi Hijo, al que amo, y que verdaderamente me
complace.'' \bibverse{18} Nosotros mismos oímos esta voz que habló desde
el cielo cuando estábamos con él en el monte santo.

\bibverse{19} También tenemos la palabra de confirmación de la profecía
que es completamente fiel, y será bueno para ustedes que le presten
atención. Porque es como una lámpara que brilla en la oscuridad, hasta
que el día termina, y se levanta la estrella de la mañana en sus
corazones. \bibverse{20} Sobre todas las cosas, deben reconocer que
ninguna profecía de la Escritura está sujeta a una interpretación basada
en los caprichos de un individuo, \bibverse{21} pues ninguna profecía
tuvo su origen en las ideas humanas, sino que los profetas hablaron por
Dios, siendo movidos por el Espíritu Santo.

\hypertarget{section-1}{%
\section{2}\label{section-1}}

\bibverse{1} Pero así como había falsos profetas entre el pueblo en ese
entonces, habrá falsos maestros entre ustedes. Y sutilmente introducirán
enseñanzas destructivas, incluso negando al Señor que los redimió, y
trayendo rápida destrucción sobre sí mismos. \bibverse{2} Muchos
seguirán sus perversiones inmorales, y por causa de ellos la gente
condenará el camino de la verdad. \bibverse{3} Pues con avaricia los
explotarán a ustedes con historias falsas. Sin embargo, ellos ya están
condenados: su sentencia ha estado colgando de sus cuerpos hace mucho
tiempo, y su destrucción no tardará. \bibverse{4} Porque Dios no perdonó
ni siquiera a los ángeles cuando pecaron. Sino que los lanzó al
Tártaro\footnote{\textbf{2:4} ``Tártaro.'' A menudo traducido como
  ``infierno,'' pero esta palabra también está asociada a mitologías. Se
  cree que ``Tártaro'' se usaba para representar la palabra ``Seol'' del
  Antiguo Testamento, o el lugar de los muertos.}, manteniéndolos en
pozos de oscuridad, listos para el juicio. \bibverse{5} Dios tampoco
perdonó al mundo antiguo, pero protegió a Noé, quien le predicó a la
gente sobre el Dios justo. Él fue una de las ocho personas que se
salvaron cuando Dios envió un diluvio sobre un mundo de personas
malvadas.

\bibverse{6} Dios condenó a las ciudades de Sodoma y Gomorra a la
destrucción total, quemándolas hasta las cenizas, como un ejemplo de lo
que sucederá a los que llevan vidas de maldad. \bibverse{7} Pero Dios
rescató a Lot, porque era un buen hombre, indignado por la abominable
inmoralidad de sus vecinos. \bibverse{8} (Lot vivía entre ellos, pero
hacía lo bueno y lo recto. Ese día vio y escuchó lo que ellos hicieron,
y la maldad de ellos lo atormentaba).

\bibverse{9} Como pueden ver, el Señor puede rescatar de las
dificultades a quienes lo respetan, y puede mantener a los malvados
hasta el día del juicio, cuando complete su castigo. \bibverse{10} Esto
también aplica a los que siguen los deseos humanos corruptos, y que con
desprecio ignoran la autoridad. Son arrogantes y orgullosos, y no temen
difamar a los seres celestiales. \bibverse{11} En cambio, los ángeles,
aunque son más fuertes y poderosos, no se atreven a difamarlos ante el
Señor.

\bibverse{12} Estas personas son como bestias sin razón, que nacen para
ser capturadas y destruidas. Condenan cosas que no conocen, y serán
destruidos como animales. \bibverse{13} Recibirán su pago por el daño
que han hecho. Se divierten al satisfacer sus deseos perversos a plena
luz del día. Son como manchas y defectos en su comunidad. Pues ellos se
complacen en sus placeres engañosos incluso cuando comparten la comida
con ustedes. \bibverse{14} Siempre están en búsqueda de relaciones
adúlteras, y no pueden dejar de pecar. Seducen a quienes son
vulnerables, y se han entrenado en la codicia; son una descendencia
maldita. \bibverse{15} Han abandonado el camino recto y se han
descarriado, siguiendo el camino de Balaam, el hijo de Beor, a quien le
gustaba recibir pago por hacer lo malo. \bibverse{16} Pero se le
reprendió por sus acciones malvadas, y hasta un asno mudo le habló con
voz humana para detener la necedad de este profeta.

\bibverse{17} Las personas así son como fuentes secas, nieblas llevadas
por el viento. Están destinadas para siempre a la más negra oscuridad.
\bibverse{18} Se jactan de sí mismos con alardes sin sentido, incitan a
los deseos sexuales pervertidos, y así atraen a la inmoralidad a los que
apenas acaban de escapar de una vida de error. \bibverse{19} Les
prometen libertad, aunque ellos mismos son esclavos de la depravación.
Pues somos esclavos de todo lo que nos domina. \bibverse{20} Si las
personas logran escapar de la influencia malvada del mundo al conocer al
Señor y Salvador Jesucristo, y luego quedan atrapadas nuevamente en el
pecado y son vencidas por él, son peor de lo que eran al principio.
\bibverse{21} Mejor sería que nunca hubieran conocido el camino recto de
la verdad, que haberlo conocido y luego apartarse de las sagradas
instrucciones que se les dieron. \bibverse{22} Este proverbio aplica
justamente a ellos: ``El perro ha vuelto a su propio vómito, y el cerdo
recién bañado ha vuelto a arrastrarse en el barro.''

\hypertarget{section-2}{%
\section{3}\label{section-2}}

\bibverse{1} Amigos míos, esta es mi segunda carta para ustedes. En
ambas he tratado de despertarlos y recordarles que deben tener un
pensamiento limpio y puro. \bibverse{2} No olviden las palabras que los
profetas dijeron en el pasado, y lo que el Señor y Salvador ordenó por
medio de los apóstoles. \bibverse{3} Y sobre todo, sepan que en los
últimos días habrá personas burlonas, que se mofarán y seguirán sus
propios deseos malvados. \bibverse{4} ``¿Qué sucedió entonces con la
venida que prometió?'' preguntan. ``Desde que murieron nuestros
ancestros, todo ha seguido igual, desde la creación del mundo.''
\bibverse{5} Pero ignoran deliberadamente el hecho de que por orden de
Dios fueron creados los cielos hace mucho tiempo atrás. La tierra llegó
a existir a partir del agua, y estaba toda rodeada de ella. \bibverse{6}
Por el agua, el mundo que existía en ese entonces fue destruido,
inundado por ella. \bibverse{7} Pero por medio de esa misma orden
divina, los cielos y la tierra que existen ahora están reservados para
la destrucción con fuego\footnote{\textbf{3:7} ``Destrucción con
  fuego'': literalmente ``en fuego''.} en el día del juicio, cuando sean
destruidos los malvados.

\bibverse{8} Sin embargo, amigos míos, no olviden esto: Que para el
Señor un día es como mil años, y mil años es como un día. \bibverse{9}
El Señor no demora el cumplimiento de su promesa, como algunos definen
la demora, sino que está siendo muy paciente con ustedes. Pues no quiere
que ninguno se pierda, sino que todos se arrepientan.

\bibverse{10} Sin embargo, el día del Señor vendrá, y será
inesperadamente, como la venida de un ladrón. Los cielos explotarán con
un rugido atronador, y los elementos\footnote{\textbf{3:10} No está
  claro a qué elementos exactos se refiere.} se destruirán al ser
consumidos. La tierra y todo lo que hay en ella se
desvanecerá\footnote{\textbf{3:10} O ``será visto por lo que es.''}.
\bibverse{11} Y como todo quedará destruido de esta manera, ¿qué clase
de gente debemos ser? Debemos vivir de manera pura, consagrados a Dios,
\bibverse{12} esperando con ilusión y deseo la venida del día del Señor.
Ese día los cielos arderán en llamas, y los elementos se fundirán.
\bibverse{13} Pero en lo que a nosotros concierne, buscamos nuevos
cielos y nueva tierra que Dios ha prometido, y donde hay
justicia\footnote{\textbf{3:13} O ``donde mora la justicia.''}.

\bibverse{14} Así que, amigos míos, puesto que ustedes esperan estas
cosas, asegúrense de estar puros e irreprochables, y en paz con Dios.
\bibverse{15} Recuerden que esta es la paciencia de nuestro Señor, que
nos da oportunidad para la salvación. Eso es lo que nuestro querido
hermano Pablo les estaba explicando en todas sus cartas, con la
sabiduría que Dios le dio. \bibverse{16} Él habló sobre estas cosas,
aunque algunas de las que escribió son difíciles de entender. Algunas
personas ignorantes y desequilibradas han tergiversado lo que él
escribió según su conveniencia, como lo hacen con otros escritos.
\bibverse{17} Mis amigos, puesto que ya saben esto, asegúrense de que
estos errores de los malvados no los descarríen, y no tropiecen de su
firme posición. \bibverse{18} Deseo que crezcan en la gracia y en el
conocimiento de nuestro Señor y Salvador Jesucristo. ¡A él sea la
gloria, ahora y por siempre! Amén.
