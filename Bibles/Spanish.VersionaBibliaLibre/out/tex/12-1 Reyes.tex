\hypertarget{section}{%
\section{1}\label{section}}

\bibverse{1} El rey David había envejecido, y por muchas mantas que
usaran para cubrirlo, no lograba entrar en calor. \bibverse{2} Por eso
sus siervos sugirieron: ``Busquemos en nombre de Su Majestad a una joven
virgen que lo sirva y lo cuide. Ella podrá acostarse a su lado y darle
calor''.

\bibverse{3} Así que buscaron por todo el país de Israel una joven
hermosa y encontraron a Abisag, de la ciudad de Sunem, y la llevaron
ante el rey. \bibverse{4} Era muy hermosa, y cuidaba al rey, atendiendo
sus necesidades, pero él no tenía relaciones sexuales con ella.

\bibverse{5} Adonías, hijo de Jaguit, comenzó aautoproclamarse,
diciendo: ``¡Voy a ser rey!''. Y dispuso carros y jinetes para él, y
cincuenta hombres para que corrieran delante de él. \bibverse{6} (Su
padre nunca lo había corregido ni había cuestionado su comportamiento.
Además, era muy atractivo y había nacido después de Absalón).

\bibverse{7} Adonías discutió su plan con Joab, hijo de Sarvia, y con el
sacerdote Abiatar, quienes estuvieron de acuerdo en apoyarlo.
\bibverse{8} Pero el sacerdote Sadoc, Benaía, hijo de Joyadá, el profeta
Natán, Simí, Rei y la guardia de David no estaban de parte de Adonías.

\bibverse{9} Adonías invitó a todos sus hermanos -- loshijos del rey --
ya los funcionarios del rey de Judá, para que fueran a la piedra de
Zojélet, que está cerca de En-rogel, donde sacrificó ovejas, vacas y
terneros engordados. \bibverse{10} Pero no invitó al profeta Natán, ni a
Benaía, el guardaespaldas de David, ni a su hermano Salomón.

\bibverse{11} Natán se dirigió a Betsabé, la madre de Salomón, y le
preguntó: ``¿No te has enterado de que Adonías, hijo de Jaguit, se ha
convertido en rey, y su majestad el rey David ni siquiera lo sabe?
\bibverse{12} Déjame darte un consejo para que puedas salvar tu vida y
la de tu hijo Salomón. \bibverse{13} Ve inmediatamente dondeel rey David
y pregúntale: ``¿No me juró Su Majestad a mí, su sierva, que su hijo
Salomón sería definitivamente rey después de mí y se sentaría en mi
trono? Entonces, ¿por qué dice Adonías que es rey?'

\bibverse{14} Entonces entraré mientras aún estás allí hablando con el
rey y confirmaré lo que dices.''

\bibverse{15} Así que Betsabé fue a ver al rey a su dormitorio. Era muy
viejo y Abisag lo cuidaba. \bibverse{16} Betsabé se inclinó en señal de
respeto. Él le preguntó: ``¿Qué es lo que quieres?''.

\bibverse{17} Ella respondió: ``Su Majestad, me juraste a mí, tu sierva,
por el Señor, tu Dios diciendo:`Tu hijo Salomón serásin duda rey después
de mi y se sentará en mitrono.' \bibverse{18} Pero ahora Adonías se ha
convertido en rey y Su Majestad ni siquiera lo sabe. \bibverse{19} Ha
sacrificado mucho ganado, terneros cebados y ovejas, y ha invitado a
todos los hijos del rey, al sacerdote Abiatar y al comandante del
ejército Joab. Pero no ha invitado a tu siervo Salomón. \bibverse{20}
Ahora, Su Majestad, todos en Israel están pendientes de ver quién va a
decir que será el próximo rey. \bibverse{21} Si no haces
nada,\footnote{\textbf{1:21} ``Si no haces nada'': literalmente,
  ``Sucederá.''}tan pronto como muera Su Majestad, mi hijo y yo seremos
vistos como traidores\ldots{}''\footnote{\textbf{1:21} Literalmente,
  ``pecadores.''}

\bibverse{22} En ese momento, mientras aún hablaba con el rey, llegó el
profeta Natán. \bibverse{23} Entonces le dijeron al rey: ``El profeta
Natán está aquí''. EntoncesNatán entró a ver al rey, y se inclinó con el
rostro hacia el suelo.

\bibverse{24} Natán le preguntó al rey: ``Su Majestad, debió haber
anunciado: `Adonías será rey después de mí y se sentará en mi trono'.
\bibverse{25} Pues hoy ha ido a sacrificar muchas reses, terneros
cebados y ovejas, y ha invitado a todos los hijos del rey, a los jefes
del ejército y al sacerdote Abiatar. En este momento están comiendo y
bebiendo con él, gritando: ``¡Viva el rey Adonías!'' \bibverse{26} Pero
no me invitó a mí, tu siervo, ni al sacerdote Sadoc, ni a Benaía, hijo
de Joyadá, ni a tu hijo Salomón. \bibverse{27} Si Su Majestad hizo esto,
ciertamente no informó a sus funcionarios sobre quién debe sentarse en
su trono como próximo rey''.

\bibverse{28} El rey David respondió: ``Llama a Betsabé de mi parte''.
Entonces Betsabé entró y se presentó ante el rey.

\bibverse{29} Y el rey hizo un juramento, diciendo: ``Vive el Señor, que
me ha salvado de toda clase de problemas, tal como te juré anteriormente
por el Señor, el Dios de Israel, \bibverse{30} diciéndote que tu hijo
Salomón será el próximo rey y que se sentará en mi trono en lugar de mí,
juro que lo haré hoy''.

\bibverse{31} Betsabé se inclinó con el rostro hacia el suelo, honrando
al rey, y dijo: ``Que tu majestad el rey David viva para siempre''.

\bibverse{32} Entonces el rey David dijo: ``Llama por mí al sacerdote
Sadoc, al profeta Natán y a Benaía, hijo de Joyadá''. Cuando llegaron,
\bibverse{33} el rey les dijo: ``Lleven a los funcionarios del rey con
ustedes, y hagan que Salomón se monte en mi propia mula y que lo lleven
hasta el manantial de Gihón. \bibverse{34} Allí hagan que el sacerdote
Sadoc y el profeta Natán lo unjan como rey de Israel. Toquen la trompeta
y griten: ``¡Viva el rey Salomón!''. \bibverse{35} Luego síganlo y hagan
que venga a sentarse en mi trono. Él será rey en mi lugar. Lo pongo al
frente como gobernante de Israel y de Judá''.

\bibverse{36} ``¡Amén!'', respondió Benaía, hijo de Joyadá. ``¡Que el
Señor Dios de mi señor el rey lo confirme! \bibverse{37} De la misma
manera que el Señor estuvo con mi señor el rey, que esté con Salomón, y
que haga su reinado aún más grande que el de mi señor el rey David.''

\bibverse{38} Entonces el sacerdote Sadoc, el profeta Natán y Benaía,
hijo de Joyadá, junto con los cereteos y los peleteos,\footnote{\textbf{1:38}
  ``Los cereteosy los peleteos'': La guardia personal del rey.}fue y
colocó a Salomón en la mula del rey David, y lo condujo hasta el
manantial de Gihón. \bibverse{39} El sacerdote Sadoc tomó el cuerno que
contenía aceite de oliva de la Tienda y ungió a Salomón. Luego tocaron
la trompeta, y todo el pueblo gritó: ``¡Viva el rey Salomón!''.
\bibverse{40} Todos lo siguieron, tocando flautas y celebrando con tanta
alegría que el sonido hizo temblar la tierra.

\bibverse{41} Adonías y todos sus invitados oyeron el ruido mientras
terminaban de comer. Cuando Joab oyó el sonido de la trompeta, preguntó:
``¿Qué es todo ese ruido que viene de la ciudad?''.

\bibverse{42} Mientras hablaba, llegó de repente Jonatán, hijo del
sacerdote Abiatar. ``Entra'', dijo Adonías. ``Un hombre bueno como tú
debe traer buenas noticias''.

\bibverse{43} ``¡Claro que no!'' respondió Jonatán. ``¡Nuestro señor el
rey David ha hecho rey a Salomón! \bibverse{44} Envió a Salomón a ser
ungido\footnote{\textbf{1:44} ``A ser ungido'': implícito.}con el
sacerdote Sadoc, con el profeta Natán y con Benaía, hijo de Joyadá, y
con los cereteos y los peleteos. Le hicieron montar en la mula del rey.
\bibverse{45} El sacerdote Sadoc y el profeta Natán lo ungieron como rey
en la fuente de Gihón. Ahora han regresado, celebrando con gritos que
resuenan por toda la ciudad. Ese es el ruido que se oye. \bibverse{46}
Además, Salomón está sentado en el trono real, \bibverse{47} y los
funcionarios reales también han ido a expresar su aprobación a nuestro
señor, el rey David, diciendo: `Que tu Dios haga que la reputación de
Salomón sea aún más famosa que la tuya, y que haga que su reinado sea
más grande que el tuyo'. El rey, en su lecho, inclinó la cabeza,
\bibverse{48} y dijo: ``¡Alabado sea el Señor, el Dios de Israel! Hoy me
ha proporcionado un sucesor para que se siente en mi trono, y he tenido
el privilegio de verlo''.

\bibverse{49} Al oír esto, todos los invitados que apoyaban a Adonías se
estremecieron de miedo. Se levantaron y salieron corriendo en distintas
direcciones. \bibverse{50} Adonías estaba aterrorizado por Salomón, así
que salió corriendo. Fue y se agarró a los cuernos del altar.

\bibverse{51} Le dijeron a Salomón: ``Adonías está aterrorizado de Su
Majestad. Se ha aferrado a los cuernos del altar, diciendo: '¡Que el rey
Salomón jure hoy que no me matará a mí, su siervo!''

\bibverse{52} Salomón contestó: ``Si es un hombre honorable, no se le
caerá ni un pelo. Pero si se muestra malvado, morirá''.

\bibverse{53} El rey Salomón hizo bajar a Adonías del altar, y éste vino
y se inclinó ante el rey Salomón, quien le dijo: ``Vete a casa.''

\hypertarget{section-1}{%
\section{2}\label{section-1}}

\bibverse{1} Se acercaba el momento de la muerte de David, por lo que le
dio a su hijo Salomón estas últimas instrucciones:

\bibverse{2} ``Estoy a punto de recorrer el camino que todo el mundo
debe recorrer en la tierra. Sé valiente y actúa como un hombre.
\bibverse{3} Haz lo que Dios te ordena, sigue sus caminos. Guarda sus
normas, sus mandatos, sus leyes y reglamentos, tal como están escritos
en la Ley de Moisés, para que tengas éxito en todo lo que hagas y en
todo lo que te propongas. \bibverse{4} Silo haces, el Señor cumplirá la
promesa que me hizo cuando dijo: `Si tus descendientes viven bien
delante mí, con fidelidad y con total compromiso, siempre tendrás a uno
de ellos en el trono de Israel'.

\bibverse{5} Además, ya sabes lo que me hizo Joab, hijo de Sarvia, y lo
que les hizo a Abner, hijo de Ner, y a Amasa, hijo de Jeter, los dos
comandantes del ejército de Israel. Los asesinó, derramando la sangre de
la guerra en tiempos de paz. Untó la sangre de la guerra en su cinturón
y en sus sandalias.\footnote{\textbf{2:5} Considerado por algunos como
  un acto simbólico que mostraba la completa destrucción de la víctima,
  poniendo fin a su movimiento y a su capacidad de engendrar hijos.}
\bibverse{6} Haz lo que creas conveniente, pero no dejes que su cabeza
con canas baje tranquilamente a la tumba.

\bibverse{7} Pero sé amable con los hijos de Barzilai de Galaad. Tráelos
a tu corte real,\footnote{\textbf{2:7} Literalmente, ``déjalos comer en
  tu mesa.''}pues me ayudaron cuando huía de tu hermano Absalón.

\bibverse{8} No olvides a Simí, hijo de Gera, el benjamita de Bahurim
que me maldijo con palabras dolorosas cuando fui a Majanayin. Cuando me
encontró en el Jordán, le juré por el Señor: `No te mataré a espada'.
\bibverse{9} Así que no lo dejes impune. Tú eres un hombre sabio y sabes
lo que tienes que hacer con él: enviarlo a la tumba con sangre en su
cabeza llena de canas''.

\bibverse{10} Entonces David murió y fue enterrado en la Ciudad de
David. \bibverse{11} Su reinado sobre Israel duró cuarenta años; siete
años en Hebrón y treinta y tres en Jerusalén. \bibverse{12} Salomón
asumió el cargo de rey, sentándose en el trono de su padre David, y el
dominio de su reino estaba asegurado.

\bibverse{13} Adonías, hijo de Jaguit, fue a ver a Betsabé, la madre de
Salomón. Entonces ella le preguntó: ``¿Has venido con buenas
intenciones?''\footnote{\textbf{2:13} ``Con buenas intenciones'':
  Literalmente, ``en paz.'' Conociendo la historia anterior, Betsabé
  tenía razón al hacer esa pregunta. Sin embargo, su aceptación de la
  petición de Adonías es sorprendente, a no ser que viera a lo que
  podría conducir.}Él respondió: ``Sí, vengo con buenas intenciones''.

\bibverse{14} ``Tengo algo que me gustaría pedirte'', continuó.

``Continúa'', dijo ella.

\bibverse{15} ``Sabes que el reino era mío -declaró-, y todos en Israel
esperaban que yo fuera su próximo rey. Pero todo se ha vuelto del revés
y el reino ha pasado a mi hermano, porque así lo ha querido el Señor.
\bibverse{16} Ahora sólo tengo que pedirte una cosa: por favor, no digas
que no''.

``Dime'', dijo ella.

\bibverse{17} Entonces continuó: ``Por favor, habla con el rey Salomón
de mi parte, porque él no te rechazará. Pídele que me dé a Abisag de
Sunem como esposa''.

\bibverse{18} ``Muy bien'', respondió Betsabé. ``Hablaré con el rey por
ti''.

\bibverse{19} Así que Betsabé fue a hablar con el rey Salomón de parte
de Adonías. El rey se levantó de su trono para recibirla, y se inclinó
ante ella. Luego se volvió a sentar y ordenó que trajeran otro trono
para su madre. Ella se sentó a su derecha.

\bibverse{20} ``Sólo quiero pedirte una pequeña cosa'', dijo ella. ``Por
favor, no digas que no''.

El rey respondió: ``Pídelo, querida madre. No te diré que no''.

\bibverse{21} ``Por favor, dale a Abisag de Sunem a tu hermano Adonías
como esposa'', respondió ella.

\bibverse{22} El rey Salomón respondió a su madre: ``¿Por qué diablos me
pides que le dé a Abisag a Adonías? Es como si me pidieras que le diera
el reino a mi hermano. Él es mi hermano mayor, y el sacerdote Abiatar y
Joab, hijo de Sarvia, están de su lado''.

\bibverse{23} Entonces el rey Salomón juró ante el Señor: ``Que Dios me
castigue, me castigue de verdad, si lo que Adonías ha pedido no le
cuesta la vida. \bibverse{24} Así que juro que, vive el Señor, que me
afirmó como rey y me colocó en el trono de mi padre David, haciéndome
cabeza de una dinastía como lo había prometido, Adonías será ejecutado
hoy.''

\bibverse{25} El rey Salomón envió a Benaía, hijo de Joyadá, quien
cumplió las órdenes del rey y ejecutó a Adonías.

\bibverse{26} En el caso de Abiatar, el sumo sacerdote, el rey le dijo:
``Vete a casa y cuida tus campos. Deberías ser condenado a muerte, pero
no te mataré ahora porque llevaste el Arca del Señor Dios por delante de
mi padre David y pasaste con él por todos sus momentos difíciles''.
\bibverse{27} Así que Salomón despidió a Abiatar de su cargo de
sacerdote del Señor, y así se cumplió lo que el Señor había dicho en
Silo con respecto a los descendientes de Elí.\footnote{\textbf{2:27}
  Véase 1 Samuel 2:30-35; 1 Samuel 3:11-14.}

\bibverse{28} CuandoJoab se enteró de la noticia, corrió a la Tienda del
Señor y se aferró a los cuernos del altar. (Él no había apoyado la
rebelión de Absalón, pero sí la de Adonías). \bibverse{29} Cuando el rey
Salomón fue informado de que Joab buscaba santuario\footnote{\textbf{2:29}
  Santuario: esto sólo se aplicaba si el asesinato de otro era
  accidental. Este claramente no fue el caso de los asesinatos
  deliberados de Joab.}junto al altar, envió a Benaía, hijo de Joyadá,
para que lo ejecutara.

\bibverse{30} Benaía fue a la Tienda del Señor y le dijo a Joab: ``¡El
rey te ordena que salgas!''

``¡No! ¡Moriré aquí!'' respondió Joab.

Benaía volvió a ver al rey y le contó lo que había dicho Joab.
\bibverse{31} ``Haz lo que dice'', le dijo el rey a Benaía. ``Derríbalo
y entiérralo. Así quitarás de mí y de mi familia la culpa de la sangre
inocente que derramó Joab. \bibverse{32} El Señor le pagará la sangre
que derramó, pues sin que mi padre David lo supiera, mató a dos hombres
buenos que eran mejores que él. Con su espada mató a Abner, hijo de Ner,
comandante del ejército de Israel, y a Amasa, hijo de Jeter, comandante
del ejército de Judá. \bibverse{33} Que la responsabilidad de haber
derramado su sangre recaiga para siempre sobre Joab y sus descendientes;
pero que el Señor dé paz y prosperidad\footnote{\textbf{2:33} ``Paz y
  prosperidad'': la palabra ``shalom'' incluye estos dos conceptos.}a
David, a sus descendientes, a su familia y a su dinastía para siempre''.

\bibverse{34} EntoncesBenaía hijo de Joyadá regresó y mató a Joab. Lo
enterraron en su casa del desierto.

\bibverse{35} El rey nombró a Benaía, hijo de Joyadá, para que asumiera
el papel de Joab como comandante del ejército, y sustituyó a Abiatar por
el sacerdote Sadoc.

\bibverse{36} Entonces el rey llamó a Simí y le dijo: ``Ve y construye
una casa en Jerusalén y quédate allí, pero no salgas ni vayas a ninguna
otra parte. \bibverse{37} Debes saber con certeza que el día que salgas
y cruces el Valle de Cedrón morirás. Tu muerte será tu propia
responsabilidad''.

\bibverse{38} ``Lo que dice Su Majestad es justo'', respondió Simí. ``Su
servidor hará lo que mi señor el rey ha ordenado''. Simí vivió en
Jerusalén durante mucho tiempo.

\bibverse{39} Pero tres años después, dos esclavos de Simí se escaparon
a Aquis, hijo de Macá, rey de Gat. Le dijeron a Simí: ``Mira, tus
esclavos están en Gat''. \bibverse{40} Así que Simí ensilló su asno y
fue a Aquis en Gat para buscar a sus esclavos. Los encontró y los trajo
de Gat.

\bibverse{41} Salomón fue informado de que Simí había salido de
Jerusalén para ir a Gat, y que luego había regresado.

\bibverse{42} El rey convocó a Simí y le preguntó: ``¿No te juré por el
Señor, no te advertí que el día que te fueras a otro lugar debías saber
con certeza que ibas a morir? ¿No me respondiste: `Lo que dice Su
Majestad es justo; haré lo que me has ordenado'? \bibverse{43} Entonces,
¿por qué no has cumplido tu voto al Señor y has obedecido mis órdenes?''

\bibverse{44} El rey también le dijo a Simí: ``En el fondo sabes todo el
mal que le hiciste a mi padre David. Por eso el Señor te pagará tu
maldad. \bibverse{45} Pero yo, el rey Salomón, seré bendecido y la
dinastía de David se mantendrá a salvo en la presencia del Señor para
siempre.''

\bibverse{46} Entnoces el rey ordenó a Benaía, hijo de Joyadá, que
ejecutara a Simí, así que éste fue y mató a Simí. De esta manera, el
dominio de Salomón sobre el reino quedó asegurado.

\hypertarget{section-2}{%
\section{3}\label{section-2}}

\bibverse{1} Aconteció que Salomón hizo una alianza
matrimonial\footnote{\textbf{3:1} ``Alianza matrimonial'': La palabra
  usada aquí significa literalmente ``hacerse esposo de una hija.''}con
el Faraón de Egipto. Se casó con la hija del faraón y la llevó a vivir a
la Ciudad de David hasta que terminó de construir su palacio, el Templo
del Señor y las murallas que rodeaban Jerusalén. \bibverse{2} Sin
embargo, en aquellos días el pueblo seguía sacrificando en los altares
porque aún no se había construido un Templo para honrar al Señor.

\bibverse{3} Salomón demostró que amaba al Señor siguiendo las
instrucciones de su padre David, excepto que sacrificaba y quemaba
ofrendas en los altares. \bibverse{4} El rey fue a Gabaón para
sacrificar allí, pues era el lugar alto más importante. Salomón presentó
mil holocaustos en el altar de ese lugar.

\bibverse{5} El Señor se le apareció a Salomón en un sueño en Gabaón.
Dios le dijo: ``Pide lo que quieras''.

\bibverse{6} Entonces Salomón respondió: ``Le mostraste a tu siervo
David, mi padre, un gran amor fiel porque vivió su vida ante ti con
fidelidad, haciendo lo correcto y comprometido con los principios. Has
seguido mostrando este gran amor incondicional dándole un hijo para que
se sentara en su trono hasta el día de hoy.

\bibverse{7} Ahora, Señor Dios, me has hecho rey en lugar de mi padre
David. Pero soy como un joven inexperto que no sabe qué
hacer.\footnote{\textbf{3:7} ``Qué hacer'': Literalmente, ``Ni entrar,
  ni salir.''} \bibverse{8} Yo, tu siervo, estoy aquí en medio de tu
pueblo elegido, un gran pueblo que es tan numeroso que no se puede
contar. \bibverse{9} Así que, por favor, dame una mente receptiva para
poder gobernar \footnote{\textbf{3:9} Gobernar no tanto en términos de
  imponer el control, sino en gobernar sabiamente.}bien a tu pueblo,
entendiendo la diferencia entre el bien y el mal, porque ¿quién puede
gobernar a este difícil pueblo tuyo?''.

\bibverse{10} El Señor consideró que lo que Salomón pedía era bueno.
\bibverse{11} Así que Dios le dijo: ``Como pediste esto, y no pediste
una larga vida, ni riquezas, ni la muerte de tus enemigos, sino que
pediste entendimiento para saber lo que es correcto, \bibverse{12} te
estoy dando lo que pediste. Te estoy dando una mente sabia, con un
entendimiento de lo que es correcto, más que cualquier otro antes de ti
o después de ti. \bibverse{13} También te estoy dando lo que no pediste,
riqueza y estatus, tanto que ningún rey se comparará contigo durante
toda tu vida. \bibverse{14} Y si sigues mis caminos, guardando mis leyes
y mis mandatos, como hizo tu padre David, te daré una larga vida''.

\bibverse{15} EntoncesSaalomón se despertó y se dio cuenta de que había
tenido un sueño. Volvió a Jerusalén y se puso delante del Arca del Pacto
del Señor y presentó holocaustos y ofrendas de paz, y celebró un
banquete para todos sus funcionarios.

\bibverse{16} Más tarde, dos prostitutas vinieron al rey y se
presentaron ante él para ser juzgadas.\footnote{\textbf{3:16} ``Para ser
  juzgadas'': implícito por el hecho de que se presentaron delante de
  él.} \bibverse{17} Una de las mujeres tomó la palabra y dijo: ``Si le
place a Su Majestad: Comparto casa con esta mujer. Tuve un bebé mientras
ella vivía en la casa. \bibverse{18} Tres días después del nacimiento de
mi bebé, esta mujer también tuvo un bebé. Estábamos juntos; no había
nadie más en la casa, sólo nosotros dos. \bibverse{19} Durante la noche,
el hijo de esta mujer murió porque ella se volcó sobre él. \bibverse{20}
Ella se levantó en medio de la noche y tomó a mi hijo de mi lado
mientras yo dormía. Lo puso junto a ella para abrazarlo, y puso a su
hijo muerto junto a mí. Cuando me levanté por la mañana para amamantar a
mi hijo, vi que estaba muerto. Cuando lo miré de cerca a la luz me di
cuenta de que no era mi hijo''.

\bibverse{22} La otra mujer argumentó: ``¡No! Mi hijo es el que está
vivo. Tu hijo es el que está muerto''. La primera mujer objetó: ``¡No!
Tu hijo es el que está muerto. Mi hijo es el que está vivo''. Siguieron
discutiendo delante del rey.

\bibverse{23} El rey intervino diciendo: ``Así que esta mujer dice: ``Mi
hijo es el que está vivo. Tu hijo es el que está muerto', mientras que
la otra mujer dice: `No, tu hijo es el que está muerto. Mi hijo es el
que está vivo'``.

\bibverse{24} ``Tráiganme una espada'', ordenó el rey. Y le trajeron una
espada. \bibverse{25} ``Corten al niño que está vivo en dos, y denle la
mitad a una mujer y la otra mitad a la otra'', ordenó.

\bibverse{26} Pero la mujer cuyo hijo estaba vivo le tenía tanto amor de
madre que le gritó al rey: ``¡Por favor, Su Majestad, dale el niño! No
lo mates''. Pero la otra mujer dijo: ``¡No será mío ni tuyo, córtalo en
dos!''.

\bibverse{27} El rey dio su veredicto. ``Dale el niño que está vivo a la
primera mujer'', ordenó. ``Por ningún motivo lo mates, pues ella es su
verdadera madre''.

\bibverse{28} Cuando todos en Israel se enteraron del veredicto que
había dado el rey, sintieron un gran respeto por él, porque reconocieron
la sabiduría que Dios le había dado para juzgar correctamente.

\hypertarget{section-3}{%
\section{4}\label{section-3}}

\bibverse{1} El rey Salomón gobernó sobre todo Israel. \bibverse{2}
Estos eran sus funcionarios: Azarías, hijo de Sadoc, era el sacerdote;
\bibverse{3} Elijoref y Ahías, hijos de Sisá, eran los secretarios del
rey. Josafat, hijo de Ajilud, llevaba los registros reales. \bibverse{4}
Benaía, hijo de Joyadá, era el comandante del ejército. Sadoc y Abiatar
eran sacerdotes. \bibverse{5} Azarías, hijo de Natán, estaba a cargo de
los gobernadores. Zabud, hijo de Natán, era sacerdote y consejero del
rey. \bibverse{6} Ajisar era el administrador del palacio. Adoniram,
hijo de Abdá, estaba a cargo de los obligados a trabajar para el rey.

\bibverse{7} Salomón tenía doce gobernadores de zona cuyas
responsabilidades abarcaban todo Israel, proporcionando alimentos para
el rey y su casa. Cada uno, a su vez, organizaba los suministros para un
mes del año.

\bibverse{8} Sus nombres eran: Ben-Jur, en la región montañosa de
Efraín;

\bibverse{9} BenDecar en Macaz, Salbín, Bet-semes y Elon-bet-Janán;

\bibverse{10} BenJésed en Arubot (Soco y toda la tierra de Héferle
pertenecían);

\bibverse{11} BenAbinadab, en todo Nafat-dor (Tafat, la hija de Salomón,
era su esposa);

\bibverse{12} Baná, hijo de Ajilud, en Tanac y Meguido, en toda la zona
de BetSeán, cerca de Saretán, debajo de Jezrel, y desde Bet-seán hasta
Abel-Mejolá y a través de Jocmeán;

\bibverse{13} Ben-guéber, en Ramot de Galaad (las ciudades de Jair, hijo
de Manasés, en Galaad le pertenecían, así como la región de Argob en
Basán, con sesenta grandes ciudades con murallas y barras de bronce);

\bibverse{14} Ajinadab, hijo de Idó, en Majanayin;

\bibverse{15} Ajimaz, en Neftalí (se había casado con Basemat, hija de
Salomón);

\bibverse{16} Baaná, hijo de Husay, en Aser y en Alot;

\bibverse{17} Josafat, hijo de Parúa, en Isacar;

\bibverse{18} Simí, hijo de Elá, en Benjamín;

\bibverse{19} Geber hijo de Uri, en la tierra de Galaad, (la antigua
tierra de Sehón, rey de los amorreos, y de Og, rey de Basán. También
había un gobernador que estaba sobre la tierra de Judá).\footnote{\textbf{4:19}
  El texto hebreo no es claro. Aquí se sigue la lectura de la
  Septuaginta, que indica que también había un gobernador sobre Judá.
  Sin embargo, algunos consideran que se refiere al gobernador anterior
  mencionado, que fue el único gobernador durante este reinado.}

\bibverse{20} Judá e Israel habían llegado a ser tan numerosos como la
arena en la orilla de la fuente de metal. Comían, bebían y eran felices.
\bibverse{21} Salomón dominaba todos los reinos desde el río Éufrates
hasta el país de los filisteos, hasta la frontera de Egipto. Presentaron
tributo a Salomón y le sirvieron durante su vida. \bibverse{22} La
comida que necesitaba cada día la corte de Salomón era de treinta coros
de la mejor harina y sesenta coros de harina; \bibverse{23} diez reses
engordadas, veinte reses de pasto, cien ovejas, así como ciervos,
gacelas, corzos y aves de corral engordadas. \bibverse{24} Porque
Salomón gobernaba toda la región al oeste del Éufrates, desde Tifa hasta
Gaza, sobre todos los reinos al oeste del Éufrates. Y tenía paz por
todos lados a su alrededor. Tuvo paz en todas las fronteras.
\bibverse{25} Durante la vida de Salomón, todos los habitantes de Judá e
Israel vivían con seguridad, desde Dan hasta Beerseba. Cada uno tenía su
propia vid e higuera. \bibverse{26} Salomón tenía 40.000 establos para
los caballos de sus carros y 12.000 auriculares. \bibverse{27} Cada mes,
los gobernadores de la zona proporcionaban por turnos comida al rey
Salomón y a todos los que comían en su mesa. Se aseguraban de que no
faltara nada. \bibverse{28} También entregaban cebada y paja donde se
necesitaban para los caballos de los carros y las carretas.

\bibverse{29} Dios le dio a Salomón sabiduría, un discernimiento muy
grande y un entendimiento tan extenso como la arena de la orilla de la
fuente de metal. \bibverse{30} La sabiduría de Salomón era mayor que la
de todos los sabios de Oriente, mayor que toda la sabiduría de Egipto.
\bibverse{31} Era más sabio que cualquiera, más sabio que Etán el
ezraíta, más sabio que Hemán, Calcol y Darda, hijos de Mahol. Su fama se
extendió por las naciones que le rodeaban. \bibverse{32} Salomón compuso
tres mil proverbios y mil cinco canciones. \bibverse{33} Podía hablar
del conocimiento de los árboles, desde el cedro del Líbano hasta el
hisopo que crece en los muros. Enseñó sobre los animales, las aves, los
reptiles y los peces. \bibverse{34} Gente de todas las naciones acudía a
escuchar la sabiduría de Salomón. Eran enviadas por todos los reyes de
la tierra, que habían oído hablar de su sabiduría.

\hypertarget{section-4}{%
\section{5}\label{section-4}}

\bibverse{1} Cuando Hiram, rey de Tiro, se enteró de que Salomón había
sido ungido rey para suceder a su padre, envió embajadores a Salomón
porque Hiram siempre había sido amigo de David. \bibverse{2} Salomón
envió este mensaje a Hiram: \bibverse{3} ``Como sabes, mi padre David no
pudo construir un templo para honrar al Señor su Dios debido a las
guerras que se libraron contra él desde todas las direcciones, hasta que
el Señor conquistó a sus enemigos.\footnote{\textbf{5:3} ``Conquistó a
  sus enemigos'': Literalmente, ``los puso bajo suspies.''} \bibverse{4}
Pero ahora el Señor, mi Dios, me ha dado paz por todas partes: no hay
enemigos ni suceden cosas malas.

\bibverse{5} Así que pienso construir un Templo para honrar al Señor, mi
Dios, como el Señor le dijo a mi padre David. Le dijo: `Tu hijo, al que
pondré en tu trono para que te suceda, construirá el Templo para
honrarme'. \bibverse{6} Ordena, pues, que se corten algunos cedros del
Líbano para mí. Mis obreros ayudarán a los tuyos, y yo les pagaré a los
tuyos con la tarifa que tú decidas, pues sabes que no tenemos a nadie
que sepa cortar madera como los sidonios.''

\bibverse{7} Cuando Hiram escuchó el mensaje de Salomón, se alegró mucho
y dijo: ``¡Alabado sea hoy el Señor, porque le ha dado a David un hijo
sabio para dirigir esta gran nación!'' \bibverse{8} Hiram envió esta
respuesta a Salomón:

``Gracias por tu mensaje. En cuanto a la madera de cedro y ciprés, haré
todo lo que quieras. \bibverse{9} Mis obreros bajarán los troncos desde
el Líbano hasta el mar, y los haré flotar en balsas por mar hasta donde
tú decidas. Allí haré romper las balsas, y tú podrás llevarte los
troncos. A cambio, me gustaría que me proporcionaras comida para mi
casa''.

\bibverse{10} Así que Hiram le proporcionó a Salomón toda la madera de
cedro y ciprés que quiso, \bibverse{11} Salomón le dio a Hiram 20.000
coros de trigo para la comida y 20.000 coros de aceite de oliva para su
casa. Salomón le proporcionaba esto a Hiram cada año. \bibverse{12} El
Señor le dio a Salomón la sabiduría que le había prometido. Hiram y
Salomón mantuvieron una buena relación e hicieron un tratado de paz
entre ellos.

\bibverse{13} El rey Salomón reclutó una fuerza de trabajo de 30.000
personas de todo Israel. \bibverse{14} Los enviaba en turnos de 10.000
cada mes al Líbano, de modo que estaban un mes en el Líbano y dos meses
en casa, mientrasAdoniram estaba a cargo de todo el personal que
trabajaba en la obra. \bibverse{15} Salomón tenía 70.000 hombres para
transportar piedras, y 80.000 canteros en la región montañosa,
\bibverse{16} así como 3.300 capataces a quienes puso a cargo de los
trabajadores. \bibverse{17} Siguiendo las órdenes del rey, extrajeron
grandes bloques de piedra cuya producción era muy costosa, y colocaron
estas piedras labradas como cimientos del Templo. \bibverse{18} Entonces
los constructores de Salomón y de Hiram, junto con los hombres de
Guebal, cortaron la piedra. Prepararon la madera y la piedra para
construir el Templo.

\hypertarget{section-5}{%
\section{6}\label{section-5}}

\bibverse{1} Cuatrocientos ochenta años después de que los israelitas
salieran de Egipto, en el cuarto año del reinado del rey Salomón, en el
mes de Zif, Salomón comenzó a construir el Templo del Señor.
\bibverse{2} El Templo que el rey Salomón construyó para el Señor medía
sesenta codos de largo por veinte de ancho y treinta de alto.
\bibverse{3} La sala de entrada en la parte delantera del Templo tenía
veinte codos de ancho. Tenía todo el ancho del Templo y sobresalía diez
codos por delante del mismo. \bibverse{4} Mandó hacer ventanas enrejadas
para colocarlas en lo alto del Templo.

\bibverse{5} También mandó construir una estructura contra el muro
exterior del Templo, tanto en el santuario exterior como en el interior,
para disponer de una serie de habitaciones laterales. \bibverse{6} La
planta baja medía cinco codos de ancho, el primer piso seis codos y el
segundo siete codos. Además, mandó colocar repisas a lo largo de todo el
exterior, para no tener que introducir vigas en los muros del Templo.
\bibverse{7} Cuando se construyó el Templo, las piedras se terminaron en
la cantera para que no se oyera en el Templo el ruido de ningún
martillo, hacha o cualquier herramienta de hierro durante la
construcción. \bibverse{8} La entrada a la planta baja estaba en el lado
sur del Templo. Las escaleras conducían al primer piso, y luego al
segundo. \bibverse{9} Salomón terminó de construir el Templo,
cubriéndolo con un techo de vigas y tablas de cedro. \bibverse{10}
Construyó la estructura externa a lo largo de todo el Templo. Tenía una
altura de cinco codos, unida al Templo con vigas de madera de cedro.

\bibverse{11} El Señor envió este mensaje a Salomón, diciéndole:
\bibverse{12} ``Sobre este Templo que estás construyendo: si sigues lo
que te he dicho, cumpliendo mis requisitos y guardando mis mandamientos
en lo que haces, cumpliré la promesa que le hice a tu padre David por
medio de ti. \bibverse{13} Viviré entre los israelitas y no abandonaré a
Israel, mi pueblo''.

\bibverse{14} Salomón terminó de construir el Templo. \bibverse{15}
Forró las paredes con paneles de cedro desde el suelo hasta el techo y
cubrió el suelo del Templo con tablas de ciprés. \bibverse{16} En la
parte de atrás del Templo, separó veinte codos con tablas de cedro desde
el suelo hasta el techo, haciendo un santuario interior como el Lugar
Santísimo. \bibverse{17} El Templo principal, frente a esta sala, medía
cuarenta codos de largo. \bibverse{18} Los paneles de cedro del interior
del Templo estaban decorados con tallas de calabazas y flores abiertas.
Todo estaba revestido de cedro; no se veía nada de la piedra.

\bibverse{19} También mandó hacer el santuario interior dentro del
Templo, donde se colocaría el Arca del Pacto del Señor. 20El santuario
interior medía veinte codos de largo, veinte de ancho y veinte de alto.
Mandó a cubrir el interior con un revestimiento de oro puro, así como el
altar de cedro. \bibverse{21} Salomón hizo cubrir todo el interior del
Templo con oro puro, Hizo que se extendieran cadenas de oro por la parte
delantera del santuario interior, que también estaba cubierta de oro.
\bibverse{22} Cubrió todo el interior del Templo con una capa de oro
hasta que estuvo todo terminado. También cubrió de oro todo el altar del
santuario interior.

\bibverse{23} Mandó hacer dos querubines de madera de olivo para el
santuario interior, de diez codos de altura cada uno. \bibverse{24} Las
dos alas del querubín medían cinco codos, lo que hacía una envergadura
total de diez codos. \bibverse{25} El otro querubín también medía diez
codos, ya que eran idénticos en tamaño y forma. \bibverse{26} Ambos
querubines medían diez codos. \bibverse{27} Hizo colocar los querubines
dentro del santuario interior del Templo. Como sus alas estaban
completamente extendidas, un ala del primer querubín tocaba una pared,
un ala del segundo querubín tocaba la otra pared, y en el centro del
Templo sus alas se tocaban. 28 Los querubines también estaban cubiertos
con una capa de oro. \bibverse{29} Hizo tallar todas las paredes del
Templo, tanto las del santuario interior como las del exterior, con
diseños de querubines, palmeras y flores abiertas. \bibverse{30} También
mandó cubrir de oro el suelo del Templo, tanto el de los santuarios
interiores como el de los exteriores.

\bibverse{31} Mandó hacer puertas de madera de olivo para la entrada del
santuario interior, con un dintel y jambas de cinco lados. \bibverse{32}
Estas puertas dobles de madera de olivo tenían grabados diseños de
querubines, palmeras y flores abiertas. Las tallas de los querubines y
las palmeras estaban cubiertas de oro batido. \bibverse{33} De igual
manera, mandó a hacer postes de madera de olivo de cuatro lados para la
entrada del santuario. \bibverse{34} Sus puertas eran de madera de
ciprés, cada una con dos paneles abatibles. \bibverse{35} Las hizo
tallar con diseños de querubines, palmeras y flores abiertas, y las
cubrió con oro batido uniformemente sobre las tallas. 36 Salomón hizo
construir el patio interior con tres hileras de piedra labrada y una de
vigas de cedro. \bibverse{37} Los cimientos del Templo del Señor fueron
puestos en el cuarto año del reinado de Salomón, en el mes de Zif.
\bibverse{38} El Templo se terminó exactamente como estaba previsto y
especificado en el undécimo año de Salomón, en el octavo mes, el mes de
Bul. Así que tardó siete años en construir el Templo.

\hypertarget{section-6}{%
\section{7}\label{section-6}}

\bibverse{1} Sin embargo, Salomón tardó trece años en terminar de
construir todo su palacio. \bibverse{2} Construyó la Casa del Bosque del
Líbano, de cien codos de largo, cincuenta de ancho y treinta de alto.
Había cuatro filas de pilares de cedro que sostenían vigas que también
eran de cedro. \bibverse{3} El techo de cedro de la casa estaba encima
de las vigas que se apoyaban en los pilares. Había cuarenta y cinco
vigas, quince en cada fila. \bibverse{4} Las ventanas estaban colocadas
en lo alto, en tres filas, cada una frente a la otra. \bibverse{5} Todos
los portales y las cubiertas de las puertas tenían marcos rectangulares,
con las aberturas una frente a la otra, de tres en tres.

\bibverse{6} También mandó hacer la Sala de las Columnas, de cuarenta
codos de largo y treinta de ancho. Tenía un pórtico delante, cuyo dosel
también estaba sostenido por columnas. \bibverse{7} La sala del trono
donde se sentaba como juez se llamaba Sala de la Justicia, y estaba
revestida de paneles de cedro desde el suelo hasta el techo.

\bibverse{8} El palacio de Salomón, donde vivía, estaba en un patio
detrás del pórtico, hecho de manera similar al Templo.\footnote{\textbf{7:8}
  ``Al Templo'': implícito.}También mandó hacer un palacio para la hija
del faraón, con la que se había casado.

\bibverse{9} Todos estos edificios se construyeron con bloques de
piedra, cuya producción era muy costosa. Se cortaban a medida y se
recortaban con sierras por dentro y por fuera. Estas piedras se
utilizaban desde los cimientos hasta los aleros, desde el exterior del
edificio hasta el gran patio. \bibverse{10} Los cimientos se colocaron
con piedras muy grandes de primera calidad, de entre ocho y diez codos
de largo. \bibverse{11} Sobre ellas se colocaron piedras de primera
calidad, cortadas a medida, junto con madera de cedro. \bibverse{12}
Alrededor del gran patio, del patio interior y del pórtico del Templo
del Señor había tres hileras de piedra labrada y una hilera de vigas de
cedro.

\bibverse{13} Entonces el rey Salomón mandó a traer a
Hiram\footnote{\textbf{7:13} ``Hiram,'' o ``Huram.'' No el rey de Tiro
  que llevaba el mismo nombre.}desde Tiro. \bibverse{14} Este era hijo
de una viuda de la tribu de Neftalí, y su padre era de Tiro, un artesano
que trabajaba el bronce. Hiram tenía una gran experiencia, entendiendo y
conociendo toda clase de trabajos en bronce. Acudió al rey Salomón y
llevó a cabo todo lo que el rey le pidió.

\bibverse{15} Hiram fundió dos columnas de bronce. Ambas tenían
dieciocho codos de altura y doce codos de circunferencia. \bibverse{16}
También fundió dos capiteles de bronce para colocarlos encima de las
columnas. Cada capitel tenía una altura de cinco codos. \bibverse{17}
Para ambos capiteles hizo una red de cadenas trenzadas, siete para cada
uno. \bibverse{18} Alrededor de la red de cadenas trenzadas hizo dos
hileras de granadas ornamentales para cubrir los capiteles en la parte
superior de ambas columnas. \bibverse{19} Los capiteles colocados en la
parte superior de las columnas del pórtico tenían forma de lirios, de
cuatro codos de altura. \bibverse{20} En los capiteles de ambas columnas
estaban las doscientas hileras de granadas que las rodeaban, justo
encima de la parte redonda que estaba junto a la red de cadenas
trenzadas. \bibverse{21} Erigió las columnas en el pórtico de entrada
del Templo. A la columna del sur le puso el nombre de Jaquín, y a la del
norte, el de Booz. \bibverse{22} Los capiteles de las columnas tenían
forma de lirios. Así quedó terminada la obra de las columnas.

\bibverse{23} Luego hizo la fuente de metal fundido.\footnote{\textbf{7:23}
  El ``mar''era un tazón metálico muy grande que contenía agua.}Su forma
era circular, y medía diez codos de borde a borde, cinco codos de altura
y treinta codos de circunferencia. \bibverse{24} Debajo del borde estaba
decorado con calabazas ornamentales que lo rodeaban, diez por cada codo
en todo el contorno. Estaban en dos filas fundidas como una sola pieza
con la fuente de metal. \bibverse{25} La fuente de metal se apoyaba en
doce bueyes de metal. Tres miraban al norte, tres al oeste, tres al sur
y tres al este. La fuente de metal estaba colocada sobre ellos, con las
espaldas hacia el centro. \bibverse{26} Era tan gruesas como el ancho de
una mano, y su borde era como el borde acampanado de una copa o de una
flor de lis. En ella cabían dos mil baños.

\bibverse{27} También hizo diez carros para transportar los lavamanos.
Las carretillas medían cuatro codos de largo, cuatro de ancho y tres de
alto. \bibverse{28} El montaje de las carretillas era el siguiente: los
paneles laterales estaban unidos a los postes. \bibverse{29} Tanto los
paneles laterales como los postes estaban decorados con leones, bueyes y
querubines. Encima y debajo de los leones y los bueyes había guirnaldas
decorativas.

\bibverse{30} Y cada carro tenía cuatro ruedas de bronce con ejes de
bronce. Un lavamanos descansaba sobre cuatro soportes que tenían coronas
decorativas a cada lado. \bibverse{31} En la parte superior de cada
carro había una abertura redonda a modo de pedestal para sostener el
lavamanos.\footnote{\textbf{7:31} ``Para sostener el lavamanos'':
  implícito.}La abertura tenía un codo de profundidad y un codo y medio
de ancho. La abertura tenía tallas a su alrededor. Los paneles del carro
eran cuadrados, no redondos. \bibverse{32} Las cuatro ruedas estaban
debajo de los paneles, y los ejes de las ruedas estaban unidos al carro.
Cada rueda medía un codo y medio de diámetro. \bibverse{33} Las ruedas
estaban hechas de la misma manera que las ruedas de las carretillas; sus
ejes, llantas, radios y cubos estaban hechos de fundición.

\bibverse{34} Cada carretilla tenía cuatro asas, una en cada esquina,
hechas como parte del soporte. \bibverse{35} En la parte superior de la
carretilla había un anillo de medio codo de ancho. Los soportes y los
paneles estaban fundidos como una sola pieza con la parte superior de la
carretilla. \bibverse{36} Hizo grabar diseños de querubines, leones y
palmeras en los paneles, los soportes y el armazón, donde hubiera
espacio, con coronas decorativas alrededor. \bibverse{37} Así hizo las
diez carretillas, con los mismos moldes, tamaño y forma. \bibverse{38}
Luego hizo diez pilas de bronce. Cada una tenía capacidad para cuarenta
baños y medía cuatro codos de ancho, una pila para cada una de las diez
carretillas. \bibverse{39} Colocó cinco carretillas en el lado sur del
Templo y cinco en el lado norte. Colocó la fuente de metal en el lado
sur, junto a la esquina sureste del Templo. \bibverse{40} También hizo
las ollas, las tenazas y los aspersorios.

Así, Hiram terminó de hacer todo lo que el rey Salomón le había pedido
para el Templo del Señor: \bibverse{41} las dos columnas; los dos
capiteles con forma de cuenco en la parte superior de las columnas; las
dos redes de cadenas trenzadas que decoraban las cuencas de los
capiteles en la parte superior de las columnas; \bibverse{42} las
cuatrocientas granadas ornamentales para las redes de cadenas (en dos
filas para las redes de cadenas que cubrían los capiteles en la parte
superior de las columnas); \bibverse{43} las diez carretillas; los diez
lavamanossobre las carretillas; \bibverse{44} la fuente de metal; los
doce bueyes bajo la fuente de metal; \bibverse{45} y las ollas, las
tenazas y los aspersorios. Todo lo que Hiram hizo para el rey Salomón en
el Templo del Señor era de bronce pulido. \bibverse{46} El rey las hizo
fundir en moldes hechos de arcilla en el valle del Jordán, entre Sucot y
Saretán. \bibverse{47} Salomón no pesó nada de lo que se había hecho,
porque era mucho: no se podía medir el peso del bronce utilizado.
\bibverse{48} Salomón también había hecho todos los elementos para el
Templo del Señor el altar de oro; la mesa de oro donde se colocaba el
Pan de la Presencia; \bibverse{49} los candelabros de oro puro que
estaban delante del santuario interior, cinco a la derecha y cinco a la
izquierda; las flores, las lámparas y las tenazas, que eran todas de oro
puro; \bibverse{50} las copas, los adornos para las mechas, los
aspersorios, la vajilla y los incensarios que también eran todos de oro
puro; y las bisagras de oro para las puertas del santuario interior, el
Lugar Santísimo, además de las puertas de la sala principal del Templo.
\bibverse{51} De esta manera se completó toda la obra del rey Salomón
para el Templo del Señor. Luego Salomón trajo los objetos que su padre
David había dedicado, los objetos especiales de plata, el oro y el
mobiliario del Templo, y los colocó en la tesorería del Templo del
Señor.

\hypertarget{section-7}{%
\section{8}\label{section-7}}

\bibverse{1} Entonces Salomón convocó ante él, en Jerusalén, a los
ancianos de Israel, incluidos todos los jefes de las tribus y los jefes
de familia de los israelitas. Les ordenó que subieran con ellos el Arca
del Pacto del Señor desde Sión, la Ciudad de David. \bibverse{2} Todos
los hombres de Israel se reunieron ante el rey Salomón en la
fiesta\footnote{\textbf{8:2} La fiesta de los tabernáculos.}que se
celebra en el séptimo mes, el mes de Etanim.

\bibverse{3} Cuando todos los ancianos de Israel se reunieron, los
sacerdotes recogieron el Arca y trajeron el Arca del Señor y la Tienda
del Encuentro con todos sus objetos sagrados. \bibverse{4} Los
sacerdotes y los levitas los subieron. \bibverse{5} Delante del Arca, el
rey Salomón y toda la congregación de Israel que se había reunido allí
con él sacrificaron muchísimas ovejas y bueyes, ¡tan numerosos que no se
podían contar! \bibverse{6} Luego los sacerdotes llevaron el Arca del
Pacto del Señor a su lugar en el santuario interior del Templo, el Lugar
Santísimo, debajo de las alas de los querubines. \bibverse{7} Los
querubines desplegaban sus alas sobre el lugar donde estaba el Arca,
cubriendo el Arca y las varas para transportarla. \bibverse{8} Las varas
eran tan largas que los extremos podían verse desde el Lugar Santo,
frente al santuario interior, pero no desde fuera. Allí están hasta el
día de hoy.

\bibverse{9} No había nada en el Arca, aparte de las dos tablas de
piedra que Moisés había colocado en ella en Horeb,\footnote{\textbf{8:9}
  ``Horeb'': Otro nombre asignado al Monte Sinaí.}donde el Señor había
hecho un acuerdo con los israelitas después de salir de la tierra de
Egipto. \bibverse{10} Cuando los sacerdotes salieron del Lugar Santo, la
nube llenó el Templo del Señor. \bibverse{11} Por causa de la nube, los
sacerdotes no pudieron quedarse allí para realizar su servicio, pues la
gloria del Señor había llenado el Templo del Señor. \bibverse{12}
Entonces Salomón dijo: ``Señor, tú\footnote{\textbf{8:12} ``Tú'':
  Literalmente, ``él,'' pero cambiado a la segunda persona para ser
  coherente con el resto de la oración.}dijiste que vivirías en la
espesa nube. \bibverse{13} Ahora he construido para ti un templo
majestuoso, un lugar donde podrás vivir para siempre''.

\bibverse{14} El rey se volvió hacia toda la asamblea de Israel que
estaba de pie y los bendijo, \bibverse{15} diciendo: ``Alabado sea el
Señor, el Dios de Israel, que con su propio poder ha cumplido la promesa
que hizo a mi padre David cuando dijo: \bibverse{16} `Desde el día en
que saqué a mi pueblo Israel de Egipto no he elegido ninguna ciudad de
las tribus de Israel como lugar para construir un Templo en mi honor.
Pero he elegido a David como rey de mi pueblo Israel'.

\bibverse{17} Mi padre David quería realmente construir un templo para
honrar al Señor, el Dios de Israel. \bibverse{18} Pero el Señor le dijo
a mi padre David: `Tuviste el deseo de construir un Templo para
honrarme, y fue bueno que realmente quisieras hacerlo. \bibverse{19}
Pero no serás tú quien construya este Templo, sino tu hijo que te va a
nacer: él construirá el Templo para honrarme'.

\bibverse{20} Ahora el Señor ha cumplido la promesa que hizo. He
sucedido a mi padre David y me he sentado en el trono de Israel, tal
como el Señor había prometido. He construido el Templo para honrar al
Señor, el Dios de Israel. \bibverse{21} He dispuesto allí un lugar para
el Arca, que contiene el pacto del Señor que hizo con nuestros
antepasados cuando los sacó de la tierra de Egipto.''

\bibverse{22} Entonces Salomón se paró frente al altar del Señor, ante
toda la asamblea de Israel, y extendió sus manos hacia el cielo.
\bibverse{23} Y dijo: ``Señor, Dios de Israel, no hay Dios como tú en el
cielo de arriba ni en la tierra de abajo. Tú mantienes tu palabra de
amor fielcon tus siervos, los que te siguen de todo corazón.
\bibverse{24} Has cumplido la promesa que le hiciste a tu siervo David,
mi padre. Tú mismo lo prometiste, y por tu propio poder lo has cumplido
hoy. \bibverse{25} Así que ahora, Señor, Dios de Israel, te ruego que
también cumplas la promesa que hiciste a tu siervo, mi padre David,
cuando le dijiste: `Nunca dejará de haber un descendiente que se siente
en mi presencia en el trono de Israel, siempre que se asegure de
seguirme como tú lo has hecho'. \bibverse{26} Ahora, Dios de Israel,
cumple la promesa que hiciste a nuestro siervo, mi padre David.

\bibverse{27} ¿Pero vivirá Dios realmente aquí en la tierra? Los cielos,
incluso los más altos, no pueden contenerte, ¡y mucho menos este Templo
que he construido! \bibverse{28} Por favor, escucha la oración de tu
siervo y su petición, Señor Dios mío. Por favor, escucha las peticiones
y las oraciones que tu siervo presenta hoy ante ti. \bibverse{29} Que
vigiles este Templo día y noche, cuidando el lugar donde dijiste que
serías honrado. Que escuches la oración que tu siervo eleva hacia este
lugar, \bibverse{30} y que escuches la petición de tu siervo y de tu
pueblo Israel cuando oran hacia este lugar. Por favor, escucha desde el
cielo donde vives. Que escuches y perdones.

\bibverse{31} Cuando alguien peca contra otro y se le exige un juramento
ante tu altar en este Templo, \bibverse{32} escucha desde el cielo:
actúa y juzga a tus siervos. Devuelve la culpa a los culpables;
reivindica y recompensa a los que hacen el bien.

\bibverse{33} Cuando tu pueblo Israel sea derrotado por un enemigo
porque ha pecado contra ti, y si vuelve arrepentido a ti, orando por el
perdón en este Templo, \bibverse{34} entonces escucha desde el cielo y
perdona el pecado de tu pueblo Israel, y haz que vuelva a la tierra que
le diste a él y a sus antepasados.

\bibverse{35} Si los cielos se cierran y no llueve porque tu pueblo ha
pecado contra ti, si orancon su mirada hacia este lugar y si vuelven
arrepentidos a ti, apartándose de su pecado porque los has castigado,
\bibverse{36} entonces escucha desde el cielo y perdona el pecado de tus
siervos, tu pueblo Israel. Enséñales el buen camino para que puedan
andar por él, y envía la lluvia sobre la tierra que has dado a tu pueblo
como posesión.

\bibverse{37} Si hay hambre en la tierra, o enfermedad, o tizón o moho
en las cosechas, o si hay langostas u orugas, o si un enemigo viene a
sitiar las ciudades de la tierra -- sea cualquier tipo de plaga o
cualquier tipo de enfermedad -- \bibverse{38} entonces cualquieroración
o apelación que haga un israelita o todo tu pueblo Israel -- incluso
cualquiera que, reconociendo sus problemas y dolores -- oremirando hacia
este Templo, \bibverse{39} entonces escucha desde el cielo, el lugar
donde vives, y perdona. Contéstales según la forma en que viven sus
vidas, porque tú sabes cómo son realmente las personas por dentro, y
sólo tú conoces el verdadero carácter de las personas. \bibverse{40}
Entonces te respetarán y seguirán tus caminos todo el tiempo que vivan
en la tierra que les diste a nuestros antepasados.

\bibverse{41} En cuanto a los extranjeros que no pertenecen a tu pueblo
Israel, sino que vienen de una tierra lejana, \bibverse{42} --
habiendooído hablar de tu gran naturaleza y poder para ayudarlos, -- que
cuando vengan y oren hacia este Templo, \bibverse{43} tú escúchalos
desde el cielo, el lugar donde vives, y dales lo que piden. De esta
manera, todos los habitantes de la tierra llegarán a conocerte y
respetarte, al igual que tu propio pueblo Israel. También sabrán que
este Templo que he construido está dedicado a ti.

\bibverse{44} Cuando tu pueblo vaya a luchar contra sus enemigos,
dondequiera que lo envíes, y cuando orenmirando hacia la ciudad que has
elegido y la casa que he construido para honrarte, \bibverse{45}
entonces escucha desde el cielo lo que están orando y pidiendo, y apoya
su causa.

\bibverse{46} Si pecan contra ti -- puesno hay nadie que no peque
--podrás enojarte con ellos y entregarlos a un enemigo que los lleve
como prisioneros a una tierra extranjera, cercana o lejana.
\bibverse{47} Pero si recapacitan en su tierra de cautiverio y se
arrepienten y te piden misericordia, diciendo: ``Hemos pecado, hemos
hecho el mal, hemos actuado con maldad'', \bibverse{48} y vuelven a ti
con total sinceridad en sus pensamientos y actitudes allí en su tierra
de cautiverio; y oran mirando hacia la tierra que les diste a sus
antepasados, la ciudad que elegiste y el Templo que he construido para
honrarte, \bibverse{49} entonces escucha desde el cielo, el lugar donde
vives, responde y apoya su causa. \bibverse{50} Perdona a tu pueblo que
ha pecado contra ti, todos los actos de rebeldía que ha cometido contra
ti. Haz que quienes los han capturado se apiaden de ellos. \bibverse{51}
Porque ellos son tu pueblo, te pertenecen. Tú los sacaste de Egipto, de
en medio de un horno donde se fundeel hierro.

\bibverse{52} Prestaatención a las peticiones de tu siervo, y a las de
tu pueblo Israel, y responde siempre que te invoquen. \bibverse{53}
Porque los apartaste de todas las naciones del mundo como un pueblo que
te pertenece, tal como lo declaraste por medio de tu siervo Moisés
cuando sacaste a nuestros padres de Egipto.''

\bibverse{54} Cuando Salomón terminó de hacer todas estas oraciones y
peticiones al Señor, se levantó ante el altar del Señor, donde había
estado arrodillado con las manos extendidas hacia el cielo.

\bibverse{55} EntoncesSalomón se puso de pie, y en voz alta bendijo a
toda la asamblea de Israel, diciendo: \bibverse{56} ``Alabado sea el
Señor, que ha dado descanso a su pueblo Israel según todo lo que
prometió. Ni una sola palabra ha fallado entre todas las buenas promesas
que hizo por medio de su siervo Moisés. \bibverse{57} Que el Señor,
nuestro Dios, esté con nosotros como lo estuvo con nuestros antepasados.
Que nunca nos deje ni nos abandone. \bibverse{58} Que nos ayude a
acercarnos a él, a seguir todos sus caminos y a cumplir los
mandamientos, estatutos y reglamentos que ordenó a nuestros antepasados.
\bibverse{59} Que estas palabras mías con las que he hecho mi petición
en presencia del Señor estén ante el Señor, nuestro Dios, día y noche.
Así él podrá apoyar la causa de su siervo y de su pueblo Israel como es
necesario cada día, \bibverse{60} para que todos en la tierra sepan que
el Señor es Dios y que no hay otro. \bibverse{61} Así que asegúrense de
estar completamente comprometidos con el Señor, nuestro Dios, tal como
lo están hoy, y sean diligentes en seguir sus estatutos y cumplir sus
mandamientos.''

\bibverse{62} Entonces el rey y todo Israel ofrecieron sacrificios ante
el Señor. \bibverse{63} Salomón presentó como ofrendas de paz al Señor
22.000 bueyes y 120.000 ovejas. De este modo, el rey y todo el pueblo de
Israel dedicaron el Templo del Señor.

\bibverse{64} Ese mismo día, el rey dedicó el centro del patio frente al
Templo del Señor. Allí presentó los holocaustos, las ofrendas de grano y
la grasa de las ofrendas de paz, ya que el altar de bronce que estaba en
la presencia del Señor era demasiado pequeño para contener todas estas
ofrendas.

\bibverse{65} Entonces Salomón, junto con todo Israel, observó la fiesta
ante el Señor, nuestro Dios, durante siete días, y luego otros siete
días; en total, catorce días. Era una gran asamblea de gente, que venía
desde tan lejos como Lebó-jamat hasta el Wadi de Egipto. \bibverse{66}
Un día después\footnote{\textbf{8:66} ``Un día después'': Literalmente,
  ``el octavo día,'' contando desde el inicio de la segunda semana.}Salomón
envió al pueblo a casa. Bendijeron al rey y se fueron a casa, llenos de
alegría y felices por todas las cosas buenas que el Señor había hecho
por su siervo David y por su pueblo Israel.

\hypertarget{section-8}{%
\section{9}\label{section-8}}

\bibverse{1} Cuando Salomón terminó el Templo del Señor y el palacio
real, y habiendo logrado todo lo que había querido hacer, \bibverse{2}
el Señor se le apareció por segunda vez, como se le había aparecido en
Gabaón. \bibverse{3} Y el Señor le dijo: ``He escuchado tu oración y tu
petición a mí. He dedicado este Templo que has construido poniendo mi
nombre en él para siempre; siempre velaré por él y lo cuidaré.

\bibverse{4} En cuanto a ti, si sigues mis caminos como lo hizo tu padre
David, haciendo todo lo que te he dicho que hagas, y si guardas mis
leyes y reglamentos, \bibverse{5} entonces aseguraré tu trono para
siempre. Yo hice este pacto con tu padre David, diciéndole: `Siempre
tendrás un descendiente que gobierne sobre Israel'.

\bibverse{6} Pero si tú o tus descendientes se apartan y no guardan las
leyes y los mandamientos que les he dado, y si van a servir y adorar a
otros dioses, \bibverse{7} entonces sacaré a Israel de la tierra que les
he dado. Desterraré de mi presencia este Templo que he dedicado a mi
honor, y lo convertiré en objeto de burla entre las naciones.
\bibverse{8} Este Templo se convertirá en un montón de escombros. Todos
los que pasen junto a él se horrorizarán y silbarán diciendo: ``¿Por qué
ha actuado el Señor de esta manera con esta tierra y este Templo?''
\bibverse{9} La respuesta será: ``Porque han abandonado al Señor, su
Dios, que sacó a sus antepasados de Egipto, y han abrazado a otros
dioses, adorándolos y sirviéndolos. Por eso el Señor ha traído sobre
ellos todo este problema'''.

\bibverse{10} Salomón tardó veinte años en construir los dos edificios:
el Templo del Señor y su propio palacio. Después de esto, \bibverse{11}
el rey Salomón dio veinte ciudades en Galilea a Hiram, rey de Tiro,
porque Hiram le había proporcionado todo el cedro y el enebro y el oro
que quería. \bibverse{12} Pero cuando Hiram fue desde Tiro a ver las
ciudades que Salomón le había dado, no quedó contento con ellas.
\bibverse{13} ``¿Qué son estas ciudades que me has dado, hermano mío?'',
le reclamó Hiram. Y las llamó la tierra de Cabul,\footnote{\textbf{9:13}
  ``La tierra de Cabul:'' sugiriendo que estas ciudades no tenían ningún
  valor.}el nombre con el que se les conoce hasta hoy. \bibverse{14} Aun
así, Hiram le envió al rey 120 talentos de oro como pago.

\bibverse{15} Este es el relato de los trabajos forzados que el rey
Salomón impuso para construir el Templo del Señor, su propio palacio,
las terrazas y la muralla de Jerusalén, así como Hazor, Meguido y
Guézer. \bibverse{16} El faraón, rey de Egipto, había atacado y
capturado Guézer. La había incendiado y había matado a los cananeos que
vivían en la ciudad. Luego se la había dado como dote de boda a su hija,
la esposa de Salomón. \bibverse{17} Salomón reconstruyó Guézer y la
parte baja de Bet-horón, \bibverse{18} Baalat y Tamar en el desierto, en
la tierra de Judá, \bibverse{19} y todas las ciudades de Salomón para
almacenamiento, y las ciudades para sus carros y para sus jinetes,
además de todo lo que Salomón quería construir en Jerusalén, en el
Líbano y en todo su reino.

\bibverse{20} Los descendientes de los amorreos, hititas, ferezeos,
heveos y jebuseos (pueblos que no eran israelitas) \bibverse{21} que
permanecieron en la tierra -- losque los israelitas no pudieron destruir
por completo -- fueronreclutados por Salomón para trabajar como mano de
obra forzosa, como lo siguen haciendo hasta hoy. \bibverse{22} Pero
Salomón no esclavizó a ningún israelita. Ellos eran sus soldados,
oficiales, comandantes, capitanes, jefes de carros y jinetes.
\bibverse{23} También eran los principales oficiales a cargo de los
programas de Salomón: 550 al mando de la gente que realizaba las obras.

\bibverse{24} Una vez que la hija del faraón se trasladó de la Ciudad de
David al palacio que Salomón había construido para ella, éste construyó
las terrazas de la ciudad.

\bibverse{25} Tres veces al año Salomón sacrificaba holocaustos y
ofrendas de paz en el altar que había construido para el Señor, quemando
incienso ante el Señor con ellos, y así cumplía con lo que se exigía en
el Templo.\footnote{\textbf{9:25} ``Así cumplía con lo que se exigía en
  el Templo:'' Algunos interpretan esto como ``así completó la
  construcción del Templo'', sin embargoyase ha aclarado que este verso
  comienza afirmando que Salomón sacrificaba tres veces al año, actos
  que claramente eran regulares mucho después de la finalización del
  Templo.}

\bibverse{26} El rey Salomón construyó una flota de barcos en
Ezion-guéber, que está cerca de Elot, a orillas del Mar Rojo, en la
tierra de Edom. \bibverse{27} Hiram envió a sus marineros, que conocían
el mar, a servir en la flota con los hombres de Salomón. \bibverse{28}
Navegaron hasta Ofir y trajeron de allí 420 talentos de oro y se los
entregaron a Salomón.

\hypertarget{section-9}{%
\section{10}\label{section-9}}

\bibverse{1} Cuando la reina de Saba se enteró de la fama de Salomón,
vino a Jerusalén para ponerlo a prueba con preguntas difíciles.
\bibverse{2} Trajo consigo un séquito muy numeroso, con camellos
cargados de especias, grandes cantidades de oro y piedras preciosas. Se
acercó a Salomón y le preguntó todo lo que tenía en mente. \bibverse{3}
Salomón respondió a todas sus preguntas. No había nada que no pudiera
explicarle. \bibverse{4} Cuando la reina de Saba vio la sabiduría de
Salomón y el palacio que había construido, \bibverse{5} la comida que
había en la mesa, cómo vivían sus funcionarios, cómo funcionaban sus
sirvientes y cómo estaban vestidos, la ropa de los camareros y los
holocaustos que presentaba en el Templo del Señor, quedó tan
asombrada\footnote{\textbf{10:5} ``Quedó tan asombrada'': implícito por
  la frase (Literalmente) ``que apenas podía respirar.''}que apenas
podía respirar.

\bibverse{6} Le dijo al rey: ``¡Es cierto lo que he oído en mi país
sobre tus proverbios\footnote{\textbf{10:6} ``Proverbios'':
  Literalmente, ``palabras.''}y tu sabiduría! \bibverse{7} Pero no creí
lo que me dijeron hasta que vine y lo vi con mis propios ojos. De hecho,
no me contaron ni la mitad: ¡el alcance de tu sabiduría supera con
creces lo que he oído! \bibverse{8} ¡Qué feliz debe ser tu pueblo! ¡Qué
felices los que trabajan para ti, los que están aquí cada día escuchando
tu sabiduría! \bibverse{9} Alabado sea el Señor, tu Dios, que tanto se
complace en ti, que te puso en su trono como rey para gobernar en su
nombre. Por el amor de tu Dios a Israel los ha asegurado para siempre, y
te ha hecho rey sobre ellos para que hagas lo justo y lo correcto.''

\bibverse{10} Presentó al rey ciento veinte talentos de oro, enormes
cantidades de especias y piedras preciosas. Nunca había habido especias
como las que la reina de Sabale regaló al rey Salomón.

\bibverse{11} (La flota de barcos de Hiram trajo oro de Ofir, y también
llevó madera de sándalo y piedras preciosas. \bibverse{12} El rey
utilizó la madera de sándalo para hacer escalones\footnote{\textbf{10:12}
  ``Escalones'': o ``barandillas.''}para el Templo y para el palacio
real, y en liras y arpas para los músicos. Nunca se había visto nada
igual en la tierra de Judá).

\bibverse{13} El rey Salomón le dio a la reina de Saba todo lo que
quiso, todo lo que pidió. Esto se sumó a los regalos habituales que le
había dado generosamente. Luego, ella y sus acompañantes regresaron a su
país.

\bibverse{14} El peso del oro que Salomón recibía cada año era de 666
talentos, 15 sin contar el que recibía de los comerciantes y mercaderes,
y de todos los reyes de Arabia y gobernadores del país.

\bibverse{16} El rey Salomón hizo doscientos escudos de oro martillado.
Cada escudo requería seiscientos siclos de oro martillado. \bibverse{17}
También hizo trescientos escudos pequeños de oro martillado. Cada uno de
estos escudos requería tres minas de oro.\footnote{\textbf{10:17} Una
  mina equivalía a 50 siclos, aproximadamente.}El rey los colocó en el
Palacio del Bosque del Líbano.

\bibverse{18} El rey también hizo un gran trono de marfil y lo cubrió de
oro puro. \bibverse{19} El trono tenía seis peldaños, con la parte
superior redondeada\footnote{\textbf{10:19} ``Redondeada'': la
  Septuaginta dice ``terneros'', es decir, un grabado que representaba
  terneros.}en el respaldo. A ambos lados del asiento había
reposabrazos, junto a los cuales había leones. \bibverse{20} En los seis
escalones había doce leones, uno en los extremos opuestos de cada
escalón. Nunca se había hecho nada parecido para ningún reino.

\bibverse{21} Todas las copas del rey Salomón eran de oro, y todos los
utensilios del Palacio del Bosque del Líbano eran de oro puro. No se usó
plata, porque no era valorada en los días de Salomón.

\bibverse{22} El rey tenía una flota de barcos de Tarsis tripulada por
los marineros de Hiram. Una vez cada tres años los barcos de Tarsis
llegaban con un cargamento de oro, plata, marfil, monos y pavos reales.

\bibverse{23} El rey Salomón era más grande que cualquier otro rey de la
tierra en riqueza y sabiduría. \bibverse{24} El mundo entero quería
conocer a Salomón para escuchar la sabiduría que Dios había puesto en su
mente. \bibverse{25} Año tras año, todos los visitantes traían regalos:
objetos de plata y oro, ropa, armas, especias, caballos y mulas.

\bibverse{26} Salomón acumuló 1.400 carros y 12.000 jinetes. Los tenía
en las ciudades de los carros, y también con él en Jerusalén.
\bibverse{27} El rey hizo que en Jerusalén abundara la plata como las
piedras, y la madera de cedro como los sicómoros en las estribaciones.
\bibverse{28} Los caballos de Salomón eran importados de
Egipto\footnote{\textbf{10:28} ``Egipto'': o Musri (Capadocia).}y de
Coa, que era donde los mercaderes reales los compraban. \bibverse{29} Un
carro importado de Egipto costaba seiscientos siclos de plata, y un
caballo ciento cincuenta. También los exportaban a todos los reyes
hititas, y a los reyes arameos.

\hypertarget{section-10}{%
\section{11}\label{section-10}}

\bibverse{1} El rey Salomón amaba a muchas mujeres extranjeras. Además
de la hija del faraón, tenía mujeres moabitas, amonitas, edomitas,
sidonias e hititas. \bibverse{2} Eran de las naciones de las que el
Señor había advertido a los israelitas: ``No deben casarse con ellas,
porque sin duda los convencerán de que adoren a sus dioses.'' Sin
embargo, Salomón, debido a su amor por las mujeres, se aferró a ellas.
\bibverse{3} Tuvo setecientas esposas de noble cuna y trescientas
concubinas. Y sus esposas lo convencieron de alejarse del Señor.

\bibverse{4} Cuando Salomón envejeció, sus esposas lo llevaron a seguir
a otros dioses, y no se comprometió de todo corazón con el Señor como lo
había hecho su padre David. \bibverse{5} Salomón adoró a Astoret, diosa
de los sidonios, y a Moloc, dios vil\footnote{\textbf{11:5} La palabra
  ``dios'' se sustituye en el texto por la palabra ``suciedad'', que
  significa algo vil y detestable. También en el verso 7.}de los
amonitas. \bibverse{6} Así fue como Salomón hizo el mal ante los ojos
del Señor, y no se entregó completamente al Señor como lo hizo su padre
David.

\bibverse{7} Fue entonces cuando Salomón construyó un alto lugar de
culto para Quemos, el vil dios del pueblo de Moab, y para Moloc, el vil
dios de los amonitas, en una colina al este de Jerusalén. \bibverse{8}
Construyó lugares de culto para todas sus esposas extranjeras, donde
quemaban incienso y sacrificaban a sus dioses.

\bibverse{9} Entonces el Señor se enojó con Salomón porque se había
alejado de él, el Dios de Israel, que se le había aparecido dos veces.
\bibverse{10} El Señor le había advertido a Salomón que no debía adorar
a otros dioses. Pero Salomón no escuchó la advertencia del Señor.
\bibverse{11} Así que el Señor le dijo: ``Ya que esto es lo que has
hecho, y ya que no has guardado mi pacto y mis leyes que ordené,
definitivamente te quitaré\footnote{\textbf{11:11} La palabra utilizada
  aquí es arrancar o despedazar. También en el versículo 12.}el reino y
se lo daré a tu siervo. \bibverse{12} Sin embargo, por el bien de tu
padre David, no haré esto en tu vida: se lo quitaré a tu hijo.
\bibverse{13} Ni siquiera entonces le quitaré todo el reino. Dejaré a tu
hijo con una tribu por amor a mi siervo David, y por amor a mi ciudad
elegida, Jerusalén.''

\bibverse{14} Entonces el Señor animó a Hadad el edomita de la familia
real de Edom a oponerse a Salomón. \bibverse{15} Anteriormente, cuando
David estaba en Edom, Joab, el comandante del ejército israelita, había
ido a enterrar a algunos de sus soldados que habían muerto, y había
matado a todos los varones de Edom. \bibverse{16} Joab y todo el
ejército israelita habían pasado seis meses allí destruyéndolos a todos.

\bibverse{17} Pero Hadad y algunos edomitas que habían sido funcionarios
de su padre habían huido a Egipto; Hadad era sólo un niño en ese
momento. \bibverse{18} Se macharon deMadián y se fueron a Parán. Luego,
junto con algunas personas de Parán, fueron a Egipto, al Faraón, rey de
Egipto. Éste proporcionó a Hadad una casa y comida, y también le asignó
tierras como regalo. \bibverse{19} El faraón se hizo muy amigo de Hadad,
y le dio a la hermana de su propia esposa para que se casara con ella,
la hermana de la reina Tapenés. \bibverse{20} Ella dio a luz a su hijo
llamado Genubat. Y Tapenés lo educó en el palacio del faraón con los
propios hijos del faraón.

\bibverse{21} Sin embargo, cuando llegó la noticia a Hadad en Egipto de
que tanto David como Joab, el comandante del ejército, habían muerto,
Hadad le dijo al faraón: ``Déjame ir y regresar a mi país.''

\bibverse{22} El faraón le preguntó: ``¿Hay algo que te haya faltado
aquí conmigo para que ahora quieras volver a tu país?''

``No, no hay nada'', respondió Hadad, ``pero por favor, déjame volver a
mi país''.

\bibverse{23} Dios también animó a Rezón, hijo de Eliada, a oponerse a
Salomón. Había huido de su amo Hadad, rey de Soba. Después de que David
destruyó el ejército de Soba, \bibverse{24} Rezón reunió a su alrededor
una banda rebelde y se convirtió en su líder. Fueron y se establecieron
en Damasco, donde tomaron el control. \bibverse{25} Rezón fue el enemigo
de Israel durante toda la vida de Salomón, lo que se sumó a los
problemas que causó Hadad. Rezón realmente odiaba a Israel, y era el
gobernante de Aram.

\bibverse{26} Además, Jeroboam, hijo de Nabat, se rebeló contra el rey.
Uno de los funcionarios de Salomón, era un efraimita de Seredá. Su madre
era una viuda llamada Serúa.

\bibverse{27} Por eso se rebeló contra el rey: Salomón había construido
las terrazas y había cerrado la brecha en la muralla de la ciudad de su
padre David. \bibverse{28} Jeroboam era un hombre hábil, y cuando
Salomón se dio cuenta del éxito que tenía en lo que hacía, lo puso al
mando de todos los trabajos forzados de las tribus de José.

\bibverse{29} Por aquel entonces, el profeta Ahías, el silonita, se
encontró con Jeroboam en el camino cuando salía de Jerusalén.
\bibverse{30} Ahías se había envuelto en un manto nuevo, y los dos
estaban solos en el campo. Ahías tomó el manto nuevo que llevaba puesto
y lo rompió en doce pedazos. \bibverse{31} Dijo: ``Jeroboam, toma diez
pedazos. Esto es lo que dice el Señor Dios de Israel. 'Jeroboam, yo soy
el Señor, el Dios de Israel, y voy a quitarle el reino a Salomón y te
voy a dar diez de las tribus. \bibverse{32} Una tribu quedará por amor a
mi siervo David y por amor a Jerusalén, la ciudad que elegí de entre
todas las tribus de Israel. \bibverse{33} Esto se debe a que me han
abandonado y se han inclinado a adorar a Astoret, diosa de los sidonios,
a Quemos, dios de los moabitas, y a Moloc, dios de los amonitas. No han
seguido mis caminos; no han hecho lo que es justo a mis ojos; no han
guardado mis mandamientos y mis leyes como lo hizo David, el padre de
Salomón.

\bibverse{34} Aun así, no voy a quitarle todo el reino a Salomón, porque
lo hice gobernar durante toda su vida por amor a mi siervo David. Lo
elegí porque guardó mis mandamientos y mis leyes. \bibverse{35} Pero
tomaré del reino de su hijo diez tribus y te las daré a ti.
\bibverse{36} A su hijo le daré una tribu, para que mi siervo David
tenga siempre un descendiente como\footnote{\textbf{11:36} ``Un
  descendiente como'': implícito.}una lámpara ante mí en Jerusalén, la
ciudad que elegí para ser honrada. \bibverse{37} Te llevaré, y reinarás
sobre todo lo que quieras. Serás rey sobre Israel. \bibverse{38} Si
aceptas todo lo que te mando, si sigues mis caminos, si haces lo que es
justo a mis ojos, guardando mis leyes y mis mandamientos como lo hizo mi
siervo David, entonces estaré contigo. Estableceré para ti una dinastía
duradera, como lo hice con David, y te entregaré Israel. \bibverse{39}
Castigaré a los descendientes de David por esto, pero no para siempre''.

\bibverse{40} Entonces Salomón trató de matar a Jeroboam. Pero Jeroboam
huyó a Egipto, a Sisac, rey de Egipto. Allí permaneció hasta la muerte
de Salomón.

\bibverse{41} El registro del resto de los actos de Salomón, incluyendo
todo lo que hizo, y su sabiduría, están escritos en el Libro de los
Hechos de Salomón. \bibverse{42} Salomón reinó en Jerusalén sobre todo
Israel durante un total de cuarenta años. \bibverse{43} Salomón murió y
fue enterrado en la ciudad de su padre David. Su hijo Roboam le sucedió
como rey.

\hypertarget{section-11}{%
\section{12}\label{section-11}}

\bibverse{1} Roboam fue a Siquem porque allí había ido todo Israel para
hacerle rey. \bibverse{2} Jeroboam, hijo de Nabat, todavía estaba en
Egipto cuando se enteró de esto. (Había huido a Egipto para escapar del
rey Salomón y estaba viviendo allí). \bibverse{3} Los líderes israelitas
enviaron a buscarlo. Jeroboam y toda la asamblea de israelitas fueron a
hablar con Roboam. \bibverse{4} ``Tu padre nos impuso una pesada
carga'', le dijeron. ``Pero ahora, si aligeras la carga cuando servimos
a tu padre y las pesadas exigencias que nos impuso, te serviremos''.

\bibverse{5} Roboam respondió: ``Vayan y vuelvan dentro de tres días''.
Así que el pueblo se fue.

\bibverse{6} El rey Roboam pidió consejo a los ancianos que habían
servido a su padre Salomón en vida. ``¿Cómo me aconsejan ustedes que le
responda a este pueblo sobre esto?'', preguntó.

\bibverse{7} Ellos le respondieron: ``Si eres un servidor de este pueblo
hoy, si les sirves y les respondes, hablándoles con amabilidad, ellos
siempre te servirán a ti.''

\bibverse{8} PeroRoboam desestimó el consejo de los ancianos. En cambio,
pidió consejo a los jóvenes con los que había crecido y que estaban
cerca de él. \bibverse{9} Les preguntó: ``¿Qué respuesta aconsejan
ustedes que enviemos a este pueblo que me ha dicho: ``Aligera la carga
que tu padre puso sobre nosotros''?''

\bibverse{10} Los jóvenes con los que se había criado le dijeron: ``Esto
es lo que tienes que decirles a estas personas que te han dicho: `Tu
padre nos ha hecho pesada la carga, pero tú deberías aligerarla'. Esto
es lo que debes responderles: Mi dedo meñique es más grueso que la
cintura de mi padre. \bibverse{11} Mi padre les puso una carga pesada, y
yo la haré aún más pesada. Mi padre los castigaba con látigos; pero yo
los castigaré con escorpiones'\,``.

\bibverse{12} Tres días después, Jeroboam y todo el pueblo regresaron a
Roboam, porque el rey les había dicho: ``Vuelvan dentro de tres días.''

\bibverse{13} El rey respondió bruscamente al pueblo. Desechando el
consejo de los ancianos, \bibverse{14} respondió utilizando el consejo
de los jóvenes. Dijo: ``Mi padre lesimpuso una pesada carga, y yo la
haré aún más pesada. Mi padre los castigaba con látigos; pero yo los
castigaré con escorpiones''.

\bibverse{15} El rey no escuchó lo que el pueblo decía, pues este cambio
de circunstancias venía del Señor, para cumplir lo que el Señor le había
dicho a Jeroboam hijo de Nabat por medio de Ahías el silonita.

\bibverse{16} Cuando todos los israelitas vieron que el rey no los
escuchaba, le dijeron al rey ``¿Qué parte tenemos en David, y qué parte
tenemos en el hijo de Isaí? ¡Vete a casa, Israel! Estás solo, casa de
David''.

Así que todos los israelitas se fueron a casa. \bibverse{17} Sin
embargo, Roboam seguía gobernando sobre los israelitas que vivían en
Judá.

\bibverse{18} Entonces el rey Roboam envió a Hadoram, encargado de los
trabajos forzados,\footnote{\textbf{12:18} Fue enviado a sofocar la
  rebelión.}pero los israelitas lo apedrearon hasta la muerte. El rey
Roboam se subió rápidamente a su carro y corrió de regreso a Jerusalén.

\bibverse{19} Como resultado, Israel se rebeló contra la casa de David
hasta el día de hoy.

\bibverse{20} Cuando todos los israelitas se enteraron de que Jeroboam
había regresado, enviaron a buscarlo, lo convocaron a la asamblea y lo
nombraron rey de todo Israel. Sólo la tribu de Judá quedó en manos de la
casa de David.

\bibverse{21} CuandoRoboam llegó a Jerusalén, reunió a los hombres de
las familias de Judá y Benjamín -- 180.000 guerreros elegidos-- parair a
luchar contra Israel y devolver el reino a Roboam, hijo de Salomón.
\bibverse{22} Pero llegó un mensaje del Señor a Semaías, el hombre de
Dios, que decía: \bibverse{23} ``Dile a Roboam, hijo de Salomón, rey de
Judá, a Judá y a Benjamín, y al resto del pueblo: \bibverse{24} `Esto es
lo que dice el Señor. No luchen contra sus parientes israelitas. Cada
uno de ustedes, váyase a su casa. Porque he sido yo quien ha hecho que
esto ocurra'``. Así que obedecieron lo que el Señor les dijo y se fueron
a sus casas, como el Señor había dicho.

\bibverse{25} Jeroboam fortaleció\footnote{\textbf{12:25}
  ``Fortaleció'': Literalmente, ``edificó,'' pero Siquem existía mucho
  antes (véase, por ejemplo, Génesis 12:6).}la ciudad de Siquem en la
región montañosa de Efraín y vivió allí. Desde allí fue y construyó
Penuel.

\bibverse{26} Jeroboam se dijo a sí mismo: ``El reino podría volver
fácilmente a la casa de David. \bibverse{27} Cuando la gente de aquí
vaya a ofrecer sacrificios al Templo del Señor en Jerusalén, volverá a
transferir su lealtad a Roboam, rey de Judá. Entonces me matarán y
volverán al rey Roboam''.

\bibverse{28} Así que, después de asesorarse, el rey mandó hacer dos
becerros de oro y le dijo al pueblo: ``No se molesten más en ir a
Jerusalén. Mira, Israel, aquí están tus dioses que te sacaron de la
tierra de Egipto''. \bibverse{29} Colocó uno en Betel y el otro en Dan.
\bibverse{30} Esta acción trajo consigo el pecado, porque el pueblo fue
hasta el norte de Dan para adorar al ídolo de allí.

\bibverse{31} Además, Jeroboam hizo construir santuarios en altares y
nombró como sacerdotes a toda clase de personas que no eran levitas.
\bibverse{32} Jeroboam inició una fiesta el día quince del octavo mes,
como la que se celebraba en Judá, y ofreció sacrificios en el altar.
Hizo esta ofrenda en Betel, sacrificando a los becerros que había hecho,
y nombró sacerdotes en Betel para los altares que había construido.
\bibverse{33} Así, el día quince del octavo mes, mes que él mismo había
elegido, Jeroboam ofreció sacrificios sobre el altar que había levantado
en Betel. Así instituyó una fiesta para los israelitas, ofreciendo
sacrificios en el altar y quemando incienso.

\hypertarget{section-12}{%
\section{13}\label{section-12}}

\bibverse{1} El Señor ordenó a un hombre de Dios proveniente de Judá
para que fuera a Betel. Llegó justo cuando Jeroboam estaba de pie junto
al altar a punto de presentar un holocausto. \bibverse{2} Gritó la
condena del Señor al altar: ``Altar, altar, esto es lo que dice el
Señor. A la casa de David le nacerá un hijo. Se llamará Josías, y sobre
ti sacrificarán los sacerdotes de los altares que queman ofrendas sobre
ti, y sobre ti se quemarán huesos humanos.'' \bibverse{3} Ese mismo día
el hombre de Dios dio una señal, diciendo: ``Esta es la señal que prueba
que el Señor ha hablado. ¡Miren! El altar se partirá, y las cenizas que
hay sobre él se derramarán''.

\bibverse{4} Cuando el rey Jeroboam oyó la condena que el hombre de Dios
había gritado contra el altar de Betel, le señaló con la mano y dijo:
``¡Arréstenlo!'' Pero la mano con la que el rey lo había apuntado se
paralizó y no podía retirarla. \bibverse{5} El altar se partió y las
cenizas se derramaron de él, cumpliendo la señal que el hombre de Dios
había dado de parte del Señor.

\bibverse{6} Entonces el rey le dijo al hombre de Dios: ``Por favor,
ruega al Señor, tu Dios, que me devuelva la mano''. El hombre de Dios
suplicó al Señor, y el rey recuperó el uso de su mano como antes.

\bibverse{7} Entonces el rey le dijo al hombre de Dios: ``Ven a mi casa
y come para que pueda darte un regalo''.

\bibverse{8} Pero el hombre de Dios le dijo al rey: ``Aunque me dieras
la mitad de todo lo que tienes, no iría contigo. De hecho, me niego a
comer o beber nada en este lugar. \bibverse{9} El Señor me ha ordenado
que no coma ni beba nada, y que no regrese por el camino que vine''.
\bibverse{10} Así que se fue por otro camino y no regresó por donde
había venido a Betel.

\bibverse{11} Sucedió que en Betel vivía un viejo profeta. Sus
hijos\footnote{\textbf{13:11} ``Hijos'': el texto hebreo
  presenta``hijo'' aquí, pero en vista de más adelante se utiliza el
  plural, parece mejor usarlo aquí también.}vinieron y le contaron todo
lo que el hombre de Dios había hecho ese día en Betel. También le
contaron a su padre lo que el hombre le había dicho al rey.
\bibverse{12} ``¿Por dónde se fue?'', les preguntó su padre. Entonces
sus hijos le mostraron el camino que había tomado el hombre de Dios
desde Judá. \bibverse{13} ``Ensillen un asno para mí'', les dijo a sus
hijos. Ellos ensillaron un asno y él subió.

\bibverse{14} Entonces cabalgó tras el hombre de Dios y lo encontró
sentado bajo una encina. ``¿Eres tú el hombre de Dios que vino de
Judá?'', le preguntó. ``Sí, lo soy'', respondió el hombre.

\bibverse{15} ``Ven conmigo a casa y come algo'', le dijo.

\bibverse{16} ``No puedo dar la vuelta e ir contigo, y no comeré ni
beberé contigo en este lugar'', respondió el hombre de Dios.
\bibverse{17} ``El Señor me ha ordenado:`no comas ni bebas nada allí, ni
te vuelvas por donde has venido'\,''.

\bibverse{18} Pero el viejo profeta le dijo: ``Yo también soy profeta,
como tú. Un ángel me dijo que Dios había dicho: `Llévalo a casa contigo
para que tenga algo que comer y beber'\,''. Pero le estaba mintiendo.

\bibverse{19} Así que el hombre de Dios volvió con él y comió y bebió en
su casa. \bibverse{20} Mientras estaban sentados a la mesa, llegó un
mensaje del Señor al profeta que lo había traído de vuelta.
\bibverse{21} Este llamó al hombre de Dios que había venido de Judá:
``Esto es lo que dice el Señor: Por haber desobedecido la palabra del
Señor y no haber seguido las órdenes que el Señor, tu Dios, te dio,
\bibverse{22} y en su lugar volviste y comiste y bebiste en el lugar
donde él te dijo que no lo hicieras, tu cuerpo no será enterrado en la
tumba de tus padres.''

\bibverse{23} Cuando el hombre de Dios terminó de comer y beber, el
profeta que lo había traído de vuelta le ensilló su propio asno.
\bibverse{24} Pero mientras seguía su camino, un león se le cruzó en el
camino y lo mató. Su cuerpo quedó tendido en el camino, con el asno y el
león de pie junto a él. \bibverse{25} Algunos transeúntes vieron el
cuerpo tirado en el camino con el león parado al lado, así que fueron a
avisar a la gente del pueblo donde vivía el viejo profeta.

\bibverse{26} Cuando el viejo profeta que había desviado al otro se
enteró de lo sucedido, dijo: ``Es el hombre de Dios que desobedeció las
órdenes del Señor. Por eso el Señor lo puso en el camino del león, y
éste lo ha mutilado y lo ha matado, tal como el Señor le dijo que
sucedería.''

\bibverse{27} Entonces el profeta dijo a sus hijos: ``Ensillen un asno
para mí''. Así que ensillaron un asno, \bibverse{28} y fue a buscar el
cadáver. Todavía estaba tirado en el camino, con el burro y el león de
pie junto a él. El león no se había comido el cuerpo ni había atacado al
burro. \bibverse{29} El profeta recogió el cuerpo del hombre de Dios, lo
puso en el burro y lo llevó a su ciudad para llorar por él y enterrarlo.
\bibverse{30} Puso el cuerpo en su propia tumba, y lo lloraron,
gritando: ``¡Pobre hermano mío!''.

\bibverse{31} Después de enterrarlo, dijo a sus hijos: ``Cuando muera,
entiérrenme en la tumba donde está enterrado el hombre de Dios. Coloquen
mis huesos junto a los suyos. \bibverse{32} Porque el mensaje del Señor
que dio en condena contra el altar de Betel y contra todos los
santuarios de los altares de las ciudades de Samaria, se cumplirá
definitivamente.''

\bibverse{33} Pero aun después de todo esto, Jeroboam no cambió sus
malos caminos. Siguió eligiendo sacerdotes de toda clase de personas.
Nombró a cualquiera que quisiera ser sacerdote de los altares.
\bibverse{34} A causa de este pecado, la casa de Jeroboam fue borrada,
destruida por completo de la faz de la tierra.

\hypertarget{section-13}{%
\section{14}\label{section-13}}

\bibverse{1} Fue en ese momento cuando Abías, el hijo de Jeroboam, cayó
enfermo. \bibverse{2} Entonces Jeroboam le dijo a su esposa: ``Por
favor, ve y disfrázate para que nadie sepa que eres la esposa de
Jeroboam. Luego ve a Silo y busca al profeta Ahías. Él fue quien me dijo
que sería rey de este pueblo. \bibverse{3} Lleva contigo diez panes,
algunas tortas y un tarro de miel para él.\footnote{\textbf{14:3} ``Para
  él'': implícito.}Él te explicará lo que le sucederá al niño''.

\bibverse{4} La esposa de Jeroboam hizo lo que le dijeron. Se levantó y
fue a la casa de Ahías en Silo. Ahías no podía ver: se había quedado
ciego a causa de su edad. \bibverse{5} Pero el Señor le dijo a Ahías:
``Mira, la mujer de Jeroboam viene a preguntarte por su hijo, porque
está enfermo. Esto es lo que debes decirle, porque vendrá disfrazada''.

\bibverse{6} Así que, en cuanto Ahías oyó sus pasos en la puerta, gritó:
``¡Entra, mujer de Jeroboam! ¿Por qué te molestas en venir disfrazada?
Me han dado una mala noticia para ti. \bibverse{7} Ve y dile a Jeroboam
que esto es lo que dice el Señor, el Dios de Israel: Te escogí de entre
las masas y te hice gobernante de mi pueblo Israel. \bibverse{8} Le
quité el reino a la casa de David y te lo di a ti. Pero no fuiste como
mi siervo David, que guardaba mis mandamientos y se comprometía
totalmente a seguirme, haciendo sólo lo que era justo a mis ojos. 9 Has
hecho más mal que todos aquellos\footnote{\textbf{14:8} Probablemente se
  refiere a los reyes anteriores.}que vivieron antes que tú. Has ido y
te has hecho otros dioses, ídolos de metal fundido que me han hecho
enfadar. Me has desechado. \bibverse{10} Ahora presta atención, porque
como resultado de esto voy a traer el desastre a la casa de Jeroboam.
Exterminaré totalmente a cada uno de tus descendientes en Israel, ya sea
esclavo o libre. Quemaré la casa de Jeroboam como un hombre que quema
desechos hasta que todo desaparezca. \bibverse{11} Los de la familia de
Jeroboam que mueran en la ciudad serán devorados por los perros, y los
que mueran en el campo serán devorados por las aves. Porque el Señor lo
ha dicho.

\bibverse{12} En cuanto a ti, levántate y vuelve a casa. En cuanto
llegues a la ciudad, el niño morirá. \bibverse{13} Todo Israel lo
llorará y lo enterrará. Sólo él, de la familia de Jeroboam, será
enterrado en una tumba\footnote{\textbf{14:13} Recibir un entierro
  adecuado se consideraba muy importante en la sociedad israelita.}porque
sólo en él ha encontrado algo bueno el Señor, el Dios de Israel, de toda
la familia de Jeroboam. \bibverse{14} El Señor elegirá para sí un rey
que gobierne sobre Israel y que destruya la casa de Jeroboam. ¡Esto está
comenzando a suceder incluso ahora! \bibverse{15} El Señor golpeará a
Israel como una caña sacudida por el agua. Arrancará a Israel de raíz de
esta buena tierra que les dio a sus antepasados y los dispersará más
allá del Éufrates, porque han hecho postes de Asera paganos, haciendo
enojar al Señor. \bibverse{16} Abandonará a Israel a causa de los
pecados de Jeroboam, los que él mismo cometió y los que hizo cometer a
Israel.''

\bibverse{17} La esposa de Jeroboam se levantó y se fue a Tirsa. En
cuanto atravesó la puerta de su casa, el niño murió. \bibverse{18} Todo
Israel lo enterró y lo lloró, tal como el Señor había dicho por medio de
su siervo el profeta Ahías. \bibverse{19} El resto de lo que hizo
Jeroboam, cómo se dedicó a la guerra y cómo reinó, están registrados en
el Libro de las Crónicas de los Reyes de Israel.

\bibverse{20} Jeroboam reinó durante veintidós años y luego murió. Su
hijo Nadab lo sucedió como rey.

\bibverse{21} Roboam, hijo de Salomón, reinó en Judá. Tenía cuarenta y
un años cuando llegó a ser rey, y reinó diecisiete años en Jerusalén, la
ciudad que el Señor había elegido de entre todas las tribus de Israel,
donde sería honrado. El nombre de su madre era Noamá la amonita.

\bibverse{22} Judá hizo el mal ante los ojos del Señor, y por los
pecados que cometieron hicieron que su ferviente ira fuera aún mayor que
la de todos sus padres. \bibverse{23} También se erigieron altares,
pilares sagrados y postes de Asera en toda colina alta y bajo todo árbol
verde. \bibverse{24} Incluso había prostitutas de culto\footnote{\textbf{14:24}
  Refiriéndose tanto a los hombres como a las mujeres.}en la tierra.
Siguieron todas las prácticas repugnantes de las naciones que el Señor
había expulsado ante los israelitas.

\bibverse{25} En el quinto año del reinado de Roboam, Sisac, rey de
Egipto, atacó Jerusalén. \bibverse{26} Tomó los tesoros del Templo del
Señor y del palacio real. Se llevó todo, incluyendo todos los escudos de
oro que Salomón había hecho. \bibverse{27} Entonces el rey Roboam hizo
escudos de bronce para reemplazarlos y se los entregó a los capitanes de
la guardia para que los cuidaran. Ellos estaban de guardia a la entrada
del palacio real. \bibverse{28} Cada vez que el rey iba al Templo del
Señor, los guardias llevaban los escudos. Después los devolvían a la
sala de guardia.

\bibverse{29} El resto de lo que sucedió en el reinado de Roboam y todo
lo que hizo están registrados en el Libro de las Crónicas de los Reyes
de Judá. \bibverse{30} Roboam y Jeroboam estuvieron siempre en guerra
entre sí. \bibverse{31} EntoncesRoboam murió y fue enterrado con sus
antepasados en la Ciudad de David. El nombre de su madre era Noamá la
amonita. Y su hijo Abías le sucedió como rey.

\hypertarget{section-14}{%
\section{15}\label{section-14}}

\bibverse{1} Abías llegó a ser rey de Judá en el año dieciocho del
reinado de Jeroboam, hijo de Nabat. \bibverse{2} Reinó en Jerusalén
durante tres años. Su madre se llamaba Macá, hija de Abisalón.

\bibverse{3} Abías cometió todos los pecados que su padre había cometido
antes que él. No estaba totalmente dedicado al Señor su Dios como lo
había estado su antepasado David. \bibverse{4} Aun así, por amor a
David, el Señor su Dios permitió que sus descendientes siguieran
gobernando como una lámpara,\footnote{\textbf{15:4} Ver 11:36.}un hijo
que gobernara después de él y que hiciera fuerte a Jerusalén.
\bibverse{5} Porque David había hecho lo que era justo ante los ojos del
Señor, y no se había desviado de nada de lo que el Señor había ordenado
durante toda su vida, excepto en el caso de Urías el hitita.

\bibverse{6} (Roboam y Jeroboam siempre estuvieron en guerra entre
sí).\footnote{\textbf{15:6} Este verso parece repetirse desde 14:30 y no
  encaja aquí en la descripción de Abías. En el versículo siguiente se
  indica que Abías y Jeroboam también estaban siempre en guerra. Tal vez
  por esta razón este verso se omite en algunos manuscritos de la
  Septuaginta.} \bibverse{7} El resto de lo que sucedió en el reinado de
Abías y todo lo que hizo está registrado en el Libro de las Crónicas de
los Reyes de Judá. Abías y Jeroboam siempre estuvieron en guerra entre
sí. \bibverse{8} Abías murió y fue enterrado en la Ciudad de David. Su
hijo Asa lo sucedió como rey.

\bibverse{9} Asa llegó a ser rey de Judá en el vigésimo año del reinado
de Jeroboam, rey de Israel. \bibverse{10} Reinó en Jerusalén cuarenta y
un años. Su abuela se llamaba Macá, hija de Abisalón.

\bibverse{11} Asá hizo lo justo a los ojos del Señor, como lo había
hecho su antepasado David. \bibverse{12} Expulsó del país a las
prostitutas del culto y se deshizo de todos los ídolos que habían
fabricado sus antepasados. \bibverse{13} Incluso despidió a su abuela
Macá como reina madre, porque había hecho un ídolo repugnante. Asa hizo
cortar el ídolo y lo quemó en el valle del Cedrón. \bibverse{14} Aunque
los altares paganos no fueron removidos, Asá se comprometió
completamente con el Señor durante toda su vida. \bibverse{15} Llevó a
la casa del Señor la plata y el oro y los demás objetos que él y su
padre habían dedicado.

\bibverse{16} Asá y Basá, rey de Israel, estaban siempre en guerra entre
sí. 17 Basá, rey de Israel, atacó a Judá y fortificó Ramá para impedir
que la gente viniera o fuera a Asa, rey de Judá.\footnote{\textbf{15:16}
  Como algunos de los que estaban en el reino de Basá se dieron cuenta
  de que Asa seguía al verdadero Dios, quisieron pasarse a él (ver 2
  Crónicas 15:9).}

\bibverse{18} Entonces Asa tomó toda la plata y el oro que quedaba en
los tesoros del Templo del Señor y del palacio real. Lo entregó a sus
siervos y los envió a Ben-hadad, hijo de Tabrimón, hijo de Hezión, rey
de Aram, que vivía en Damasco, junto con este mensaje: \bibverse{19}
``Hagamos un tratado entre nosotros, como lo hubo entre mi padre y el
tuyo. Mira, te he enviado un regalo de plata y oro. Ve y rompe el
tratado con Basá, rey de Israel, para que se retire y me deje en paz''.

\bibverse{20} Ben Adad aceptó la propuesta de Asá y envió a su ejército
con sus comandantes a atacar las ciudades de Israel. Capturaron las
ciudades de Ijón, Dan, Abel-Bet-macá y todo Quinéret, incluyendo toda la
tierra de Neftalí. \bibverse{21} CuandoBasá se enteró de esto, dejó de
fortificar Ramá y se retiró a Tirsa. \bibverse{22} Entonces el rey Asá
emitió un anuncio público en toda Judá, sin excepción. El pueblo
obedeció y se llevó las piedras y los maderos que Basá había usado para
construir Ramá. El rey Asa utilizó estos materiales de construcción para
fortalecer Geba de Benjamín, así como Mizpa.

\bibverse{23} El resto de lo que sucedió en el reinado de Asa, todos sus
logros, todo lo que hizo y las ciudades que construyó, están registrados
en el Libro de las Crónicas de los Reyes de Judá. Pero cuando envejeció
tuvo una enfermedad en los pies.\footnote{\textbf{15:23} Este aspecto se
  asocia con problemas en la vida posterior de Asa. Véase 2 Crónicas 16.}
\bibverse{24} Asa murió y fue enterrado con sus antepasados en la Ciudad
de David. Su hijo Josafat lo sucedió como rey.

\bibverse{25} Nadab, hijo de Jeroboam, se convirtió en rey de Israel en
el segundo año del reinado de Asá de Judá. Reinó en Israel durante dos
años. \bibverse{26} Sus hechos fueron malos a los ojos del Señor. Siguió
los caminos de su padre y cometió los mismos pecados que su padre había
hecho cometer a Israel.

\bibverse{27} Basá, hijo de Ahías, de la tribu de Isacar, tramó una
rebelión contra él. Basá asesinó a Nadab en la ciudad filistea de
Guibetón, mientras Nadab y todo el ejército israelita la asediaban.
\bibverse{28} Basá mató a Nadab y asumió como rey en el tercer año del
reinado del rey Asa de Judá.

\bibverse{29} Tan pronto como llegó a ser rey, mató a todo el resto de
la familia de Jeroboam. No dejó con vida a ninguno de los descendientes
de Jeroboam: los destruyó a todos, como el Señor había dicho por medio
de su siervo Ahías el silonita. \bibverse{30} Esto sucedió por los
pecados que Jeroboam había cometido y había hecho cometer a Israel, y
porque había hecho enojar al Señor, el Dios de Israel.

\bibverse{31} El resto de lo que sucedió en el reinado de Nadab y todo
lo que hizo están registrados en el Libro de las Crónicas de los Reyes
de Israel.

\bibverse{32} Asá y Basá, rey de Israel, estaban siempre en guerra entre
sí. \bibverse{33} Basá, hijo de Ahías, llegó a ser rey de todo Israel en
el tercer año del reinado del rey Asá en Judá. Basá reinó en Tirsa
durante veinticuatro años. \bibverse{34} Perolos hechos de Basáfueron
malosa los ojos del Señor y siguió el camino de Jeroboam y su pecado,
que él había hecho cometer a Israel.

\hypertarget{section-15}{%
\section{16}\label{section-15}}

\bibverse{1} Entonces llegó este mensaje del Señor al profeta Jehú, hijo
de Jananí, condenando a Basá. \bibverse{2} ``Aunque te levanté del polvo
para hacerte gobernante de mi pueblo Israel, has seguido el camino de
Jeroboam y has hecho pecar a mi pueblo Israel, enojándome con sus
pecados. \bibverse{3} Ahora voy a destruir a Basá y a su familia. Basá,
haré que tu familia sea como la de Jeroboam, hijo de Nabat. \bibverse{4}
Los de la familia de Basá que mueran en la ciudad serán devorados por
los perros, y los que mueran en el campo serán devorados por las aves.''

\bibverse{5} El resto de los acontecimientos del reinado de Basá, todo
lo que hizo y lo que logró, están registrados en el Libro de las
Crónicas de los Reyes de Israel. \bibverse{6} Basá murió y fue enterrado
en Tirsa. Su hijo Elá le sucedió como rey.

\bibverse{7} El mensaje del Señor que condenaba a Basá y a su familia
llegó al profeta Jehú, hijo de Hanani. Llegó porque Basá había hecho el
mal a los ojos del Señor, de la misma manera que lo había hecho la
familia de Jeroboam, y también porque Basá había matado a la familia de
Jeroboam. El Señor estaba enojado por los pecados de Basá.

\bibverse{8} Elá, hijo de Basá, llegó a ser rey de Israel en el año
veintiséis del reinado del rey Asá de Judá. Reinó en Tirsa durante dos
años.

\bibverse{9} Uno de los funcionarios de Elá, llamado Zimri, que estaba a
cargo de la mitad de sus carros, tramó una rebelión contra él. Una vez
Elá estaba en Tirsa, emborrachándose en la casa de Arza, el
administrador del palacio de Tirsa. \bibverse{10} Zimri se acercó a él,
lo atacó y lo mató. Esto ocurrió en el año veintisiete del reinado de
Asá, rey de Judá. Luego lo sustituyó como rey.

\bibverse{11} Tan pronto como llegó a ser rey y se instaló en su trono,
mató a toda la familia de Basá. No dejó ni un solo varón vivo, ni de sus
parientes ni de sus amigos. \bibverse{12} Así que Zimri destruyó a toda
la familia de Basá, como había dicho el Señor en su condena de Basá por
medio del profeta Jehú. \bibverse{13} Esto se debió a todos los pecados
que Basá y su hijo Elá habían cometido y habían hecho cometer a Israel.
Su adoración de sus ídolos inútiles había enojado al Señor, el Dios de
Israel.

\bibverse{14} El resto de lo que sucedió en el reinado de Elá y todo lo
que hizo están registrados en el Libro de las Crónicas de los Reyes de
Israel.

\bibverse{15} Zimri llegó a ser rey de Israel en el año veintisiete del
reinado del rey Asá de Judá. Reinó en Tirsa siete días. En ese tiempo el
ejército israelita estaba atacando la ciudad filistea de Guibetón.
\bibverse{16} Cuando las tropas que estaban acampadas allí se enteraron
de que Zimri había tramado una rebelión contra el rey y lo había
asesinado, nombraron a Omri, el comandante del ejército, rey de Israel
ese mismo día en el campamento del ejército. \bibverse{17} Omri y todo
el ejército israelita salieron de Guibetón y fueron a sitiar Tirsa.
\bibverse{18} CuandoZimri vio que la ciudad había sido tomada, entró en
la fortaleza del palacio real y le prendió fuego a su alrededor, y murió
por los pecados que había cometido. \bibverse{19} Sus hechos fueron
malos a los ojos del Señor y siguió el camino de Jeroboam y su pecado
que había hecho cometer a Israel.

\bibverse{20} El resto de lo que sucedió en el reinado de Zimri y su
rebelión está registrado en el Libro de las Crónicas de los Reyes de
Israel.

\bibverse{21} Después de esto el pueblo de Israel se dividió. La mitad
apoyaba a Tibni, hijo de Ginat, como rey, mientras que la otra mitad
apoyaba a Omri. \bibverse{22} Sin embargo, los que estaban del lado de
Omri derrotaron a los partidarios de Tibni. Entonces Tibni fue asesinado
y Omri se convirtió en rey.

\bibverse{23} Omri se convirtió en rey de Israel en el año treinta y uno
del reinado del rey Asa de Judá. Reinó durante un total de doce años,
(seis de ellos en Tirsa). \bibverse{24} Compró la colina de Samaria a
Semer por dos talentos de plata. Fortificó la colina y llamó a la ciudad
que construyó Samaria, en honor a Semer, el anterior dueño de la colina.

\bibverse{25} Ylos hechos de Omri fueron malos a los ojos del Señor; de
hecho, hizo más mal que aquellos\footnote{\textbf{16:25} Probablemente
  se refiere a los reyes anteriores. También el versículo 30.}que
vivieron antes que él. \bibverse{26} Porque siguió todos los caminos de
Jeroboam, hijo de Nabat, y en sus pecados que hizo cometer a Israel,
adorando a sus ídolos inútiles que enojaban al Señor, el Dios de Israel.

\bibverse{27} El resto de lo que sucedió en el reinado de Omri, lo que
hizo y sus logros están registrados en el Libro de las Crónicas de los
Reyes de Israel. \bibverse{28} Omri murió y fue enterrado en Samaria. Su
hijo Acab lo sucedió como rey.

\bibverse{29} Acab, hijo de Omri, se convirtió en rey de Israel en el
año treinta y ocho del reinado del rey Asa de Judá. Reinó en Samaria
durante veintidós años. \bibverse{30} Acab, hijo de Omri, hizo el mal a
los ojos del Señor, más que los que vivieron antes que él. \bibverse{31}
No vio nada de qué preocuparse al seguir los pecados de Jeroboam, hijo
de Nabat, e incluso se casó con Jezabel, hija de Etbaal, rey de los
sidonios, y comenzó a servir y adorar a Baal. \bibverse{32} Acab hizo un
altar para Baal en el templo de Baal que había construido en Samaria.
\bibverse{33} Luego colocó un poste de Asera. Fue así como Acabhizo más
para enojar al Señor, el Dios de Israel, que todos los reyes anteriores
de Israel.

\bibverse{34} Durante el reinado de Acab, Hiel de Betel reconstruyó
Jericó. Sacrificó a Abiram, su hijo primogénito, cuando puso sus
cimientos, y sacrificó a Segub, su hijo menor, cuando construyó sus
puertas.\footnote{\textbf{16:34} La práctica de sacrificar niños al
  construir un edificio era un rito que llevaban a cabo los cananeos
  paganos.}Esto cumplió el mensaje que el Señor había dado a través de
Josué, hijo de Nun.

\hypertarget{section-16}{%
\section{17}\label{section-16}}

\bibverse{1} Elías el tisbita, (de Tisbe en Galaad), le dijo a Acab:
``¡Vive el Señor, el Dios de Israel, al que sirvo, que en los años
venideros no habrá rocío ni lluvia si yo no lo digo!''

\bibverse{2} Entonces el Señor le dijo a Elías: \bibverse{3} ``Sal de
aquí y vete al este. Escóndete en el valle del arroyo de Querit, donde
se une con el Jordán. \bibverse{4} Podrás beber del arroyo, y he
ordenado que los cuervos te lleven comida allí.''

\bibverse{5} Así que Elías hizo lo que el Señor le dijo. Fue al valle
del arroyo de Querit, donde se une con el Jordán, y se quedó allí.
\bibverse{6} Los cuervos le llevaban pan y carne por la mañana y por la
tarde, y él bebía del arroyo. \bibverse{7} Un tiempo después, el arroyo
se secó porque no había llovido en la tierra.

\bibverse{8} Entonces el Señor le dijo a Elías: \bibverse{9} ``Vete de
aquí y vete a Sarepta, cerca de Sidón, y quédate allí. He dado
instrucciones a una viuda de allí para que te proporcione comida''.

\bibverse{10} Así que partió hacia Sarepta. Cuando llegó a la entrada de
la ciudad, vio a una mujer, una viuda, que recogía palos. La llamó y le
preguntó: ``¿Podrías traerme un poco de agua en un vaso para que pueda
beber?''. \bibverse{11} Mientras ella iba a buscarla, él la llamó y le
dijo: ``Ah, y por favor, tráeme un pedazo de pan''.

\bibverse{12} Ella le contestó: ``Vive el Señor, tu Dios, que no tengo
pan, sólo me queda un puñado de harina en una tinaja y un poco de aceite
de oliva en una jarra. Ahora mismo estoy recogiendo unos cuantos palos
para ir a cocinar lo que queda para mí y para mi hijo y así poder
comerlo, y luego moriremos.''

\bibverse{13} Elías le dijo: ``No tengas miedo. Vete a casa y haz lo que
has dicho. Pero primero hazme una pequeña hogaza de pan de lo que tienes
y tráemela. Luego prepara algo para ti y para tu hijo. \bibverse{14}
Porque esto es lo que dice el Señor, el Dios de Israel: ``La vasija de
harina no se vaciará y la jarra de aceite de oliva no se agotará hasta
el día en que el Señor envíe la lluvia para regar la tierra.''

\bibverse{15} Ella fue e hizo lo que Elías le había dicho, y Elías, la
viuda y su familia pudieron comer durante muchos días. \bibverse{16} La
vasija de harina no se vació y la jarra de aceite de oliva no se agotó,
tal como el Señor había dicho por medio de Elías.

\bibverse{17} Más tarde, el hijo de la mujer cayó enfermo. (Ella era la
dueña de la casa.) Fue de mal en peor, y finalmente murió.

\bibverse{18} ``¿Qué me estáshaciendo,\footnote{\textbf{17:18} La frase
  literal en el hebreo es ``¿Qué hay de mi y de ti?''. A veces se
  traduce como ``¿Qué tengo que ver contigo?'', pero aquí se utiliza
  claramente como una pregunta relativa a la muerte del hijo de la
  viuda.}hombre de Dios?'', le preguntó la mujer a Elías. ``¿Has venido
a recordarme mis pecados y a provocar la muerte de mi hijo?''

\bibverse{19} ``Dame a tu hijo'', respondió Elías. Lo cogió de los
brazos de la mujer, lo subió a la habitación donde se alojaba y lo
acostó en su cama. \bibverse{20} Entonces clamó al Señor, diciendo:
``Señor, Dios mío, ¿por qué has permitido que le suceda esto a esta
viuda que me ha abierto su casa, esta terrible tragedia de hacer morir a
su hijo?''

\bibverse{21} Se tendió sobre el muchacho tres veces y clamó al Señor:
``¡Señor Dios mío, por favor, haz que la vida de este muchacho vuelva a
él!'' \bibverse{22} El Señor respondió al clamor de Elías. La vida del
muchacho volvió a él, y vivió.

\bibverse{23} Elías tomó al niño, lo bajó de la habitación a la casa y
se lo entregó a su madre. ``Mira, tu hijo está vivo'', le dijo Elías.

\bibverse{24} ``Ahora estoy convencida de que eres un hombre de Dios, y
de que lo que el Señor habla a través de ti es la verdad'', respondió la
mujer.

\hypertarget{section-17}{%
\section{18}\label{section-17}}

\bibverse{1} Algún tiempo después, durante el tercer año, un mensaje del
Señor llegó a Elías: ``Ve y preséntate ante Acab, y enviaré lluvia sobre
la tierra''. \bibverse{2} Así que Elías fue a presentarse ante Acab.
Mientras tanto, el hambre se había agravado en Samaria. \bibverse{3}
Acab convocó a Abdías, el administrador de su palacio (Abdías era un
creyente muy sincero en el Señor. \bibverse{4} Mientras Jezabel estaba
ocupada matando a los profetas del Señor, Abdías había tomado a cien
profetas y los había escondido, cincuenta en cada una de dos cuevas, y
les había proporcionado comida y agua). \bibverse{5} Acab le dijo a
Abdías: ``Recorre el país y revisa todos los manantiales y valles. Tal
vez podamos encontrar algo de hierba para mantener vivos a los caballos
y a las mulas, y así no perderemos a ninguno de los animales''.
\bibverse{6} Así que se repartieron la tierra. Acab fue en una
dirección, y Abdías en la otra.

\bibverse{7} Mientras Abdías seguía su camino, Elías salió a su
encuentro. Abdías lo reconoció, se inclinó hasta el suelo y dijo:
``¿Eres tú, mi señor Elías?''

\bibverse{8} ``Soy yo'', respondió Elías. ``Ve y dile a tu señor: `Elías
está aquí'\,''.

\bibverse{9} ``¿Cómo he pecado para que me entregues a mí, tu siervo, a
Acab para que me mate? \bibverse{10} Vive el Señor, tu Dios, que no hay
nación ni reino donde mi amo no haya enviado a alguien a buscarte.
Cuando una nación o reino dijo que no estabas, él les hizo jurar que no
podían encontrarte. \bibverse{11} Y ahora me dices que vaya a mi amo y
le anuncie: ``Elías está aquí''. \bibverse{12} No tengo idea de adónde
te llevará el Espíritu del Señor después de que te deje. Si voy y se lo
digo a Acab y luego no te encuentra, me va a matar, aunque yo, tu
siervo, he adorado al Señor desde que era joven. \bibverse{13} ¿No
oíste, mi señor, lo que hice cuando Jezabel se ocupaba de matar a los
profetas del Señor? Escondí a cien de los profetas del Señor, cincuenta
en cada una de las dos cuevas, y les di comida y agua. \bibverse{14} Y
ahora me dice que vaya a mi amo y le anuncie: `Elías está aquí'. Me va a
matar''.

\bibverse{15} Elías respondió: ``Vive el Señor Todopoderoso, a quien
sirvo, que hoy me presentaré definitivamente ante Acab''.

\bibverse{16} Así que Abdías fue a reunirse con Acab y le contó, y Acab
fue a reunirse con Elías. \bibverse{17} CuandoAcab vio a Elías, le dijo:
``¿Eres tú el que está causando problemas a Israel?''

\bibverse{18} ``No estoy causando problemas a Israel'', respondió Elías.
``¡Eres tú y la familia de tu padre! Han rechazado los mandatos del
Señor y están adorando a los baales. \bibverse{19} Ahora convoca a todo
Israel y reúnete conmigo en el monte Carmelo, junto con los
cuatrocientos cincuenta profetas de Baal y los cuatrocientos profetas de
Asera, que son apoyados por Jezabel.''

\bibverse{20} Así que Acab convocó a todo Israel y reunió también a los
profetas en el monte Carmelo. \bibverse{21} Elías se acercó al pueblo y
les preguntó: ``¿Hasta cuándo van a andar cojeando, dudando entre dos
creencias opuestas? Si el Señor es Dios, entonces síganlo. Pero si Baal
es Dios, entonces síganlo''. Pero la gente no respondió.

\bibverse{22} Entonces Elías les dijo: ``Yo soy el único que queda de
los profetas del Señor -sólo yo-, pero Baal tiene cuatrocientos
cincuenta profetas. \bibverse{23} Proporciónanos dos bueyes. Que los
profetas de Baal escojan el que quieran, y que lo corten en pedazos y lo
pongan sobre la leña. Pero no le prendan fuego. Yo prepararé el otro
buey y lo pondré sobre la leña, pero no le prenderé fuego. \bibverse{24}
Entonces tú invocarás a tu dios por su nombre, y yo invocaré al Señor
por su nombre. El dios que responde enviando fuego es Dios''. Entonces
todo el pueblo dijo: ``Estamos de acuerdo con lo que dices.''\footnote{\textbf{18:24}
  ``Estamos de acuerdo con lo que dices'': Literalmente ``la palabra es
  buena.''Quizás un coloquialismo más moderno sería ``buena idea.''}

\bibverse{25} Elías dijo a los profetas de Baal: ``Elijan uno de los
bueyes y prepárenlo primero, porque sonnumerosos. Invoquen a su dios por
su nombre, pero no enciendan el fuego''. \bibverse{26} Así que tomaron
el buey provisto y lo prepararon. Luego invocaron a Baal por su nombre
desde la mañana hasta el mediodía. ``¡Baal, respóndenos!'', suplicaron.
Pero no se oyó ninguna voz, ni nadie respondió. Cojeaban\footnote{\textbf{18:26}
  ``Cojeaban'': La palabra es la misma que se utiliza en el versículo 21
  para su vacilación entre dos creencias. Se utiliza aquí para describir
  el baile errante y tropezado de estos sacerdotes paganos que se
  desorientaban cada vez más al tratar de hacer que su ``dios'' les
  respondiera.}en una danza alrededor del altar que habían hecho.

\bibverse{27} Al mediodía, Elías comenzó a burlarse de ellos. ``¡Griten
muy fuerte!'', dijo. ``¿No se supone que es un dios? Quizá esté
meditando, o haya ido al baño, o esté de viaje. Tal vez esté dormido y
haya que despertarlo''.

\bibverse{28} Entonces gritaron aún más fuerte y se cortaron con espadas
y lanzas hasta sangrar. Esta era su forma habitual de adorar.
\bibverse{29} Llegó el mediodía y siguieron con sus maníacas
``profecías'' hasta la hora\footnote{\textbf{18:29} Cerca de las 3 de la
  tarde.}del sacrificio vespertino. Pero no se oía ninguna voz, nadie
respondía, nadie escuchaba.

\bibverse{30} LuegoElías les dijo a todos: ``Vengan hacia mí''. Se
acercaron a él, y reparó el altar del Señor que había sido derribado.
\bibverse{31} Elías tomó doce piedras que representaban las tribus de
los hijos de Jacob. (Jacob fue el que recibió el mensaje del Señor que
decía: ``Israel será tu nombre''). \bibverse{32} Con las piedras
construyó un altar en nombre del Señor. Cavó una zanja a su alrededor en
la que cabían dos seahs de semillas. \bibverse{33} Colocó la madera en
su lugar, cortó el buey en pedazos y lo puso sobre la madera. Luego les
dijo: ``Llenen de agua cuatro tinajas grandes y viértanla sobre la
ofrenda y la madera''.

\bibverse{34} ``Vuelvan a hacerlo'', les dijo. Así lo hicieron.
``Háganlo por tercera vez'', les dijo. Y lo hicieron por tercera vez.
\bibverse{35} El agua corrió por todo el altar y llenó la zanja.

\bibverse{36} A la hora del sacrificio vespertino, el profeta Elías se
acercó al altar y oró: ``Señor, Dios de Abrahán, de Isaac y de Israel,
demuestra hoy que eres Dios en Israel, que yo soy tu siervo y que todo
lo que he hecho ha sido por orden tuya. \bibverse{37} ¡Respóndeme,
Señor! Respóndeme, para que este pueblo sepa que tú, Señor, eres Dios, y
que los estás devolviendo a ti''.

\bibverse{38} Entonces el fuego del Señor bajó y quemó el sacrificio, la
madera, las piedras y la tierra; incluso lamió el agua de la zanja.

\bibverse{39} Al ver esto, todo el pueblo se postró en el suelo y gritó:
``¡El Señor es Dios! El Señor es Dios''.

\bibverse{40} Entonces Elías les ordenó: ``Agarren a los profetas de
Baal. No dejen escapar a ninguno''. Los agarraron, y Elías los bajó al
valle de Cisón y los mató allí.

\bibverse{41} Elías le dijo a Acab: ``Ve a comer y a beber, porque oigo
que viene una lluvia fuerte''. \bibverse{42} Así que Acab fue a comer y
a beber, pero Elías fue a la cima del Carmelo. Allí se inclinó hacia el
suelo, poniendo su rostro entre las rodillas.

\bibverse{43} ``Ve y mira hacia el mar'', le dijo a su siervo. El hombre
fue y miró. ``Allí no hay nada'', dijo. Siete veces le dijo Elías: ``Ve
y mira otra vez''.

\bibverse{44} La séptima vez el siervo regresó y dijo: ``He visto una
pequeña nube del tamaño de la mano de un hombre que subía del mar''.
Entonces Elías le dijo: ``Corre a ver a Acab y dile: ``Prepara tu carro
y baja antes de que la lluvia te detenga''.

\bibverse{45} Rápidamente el cielo se oscureció con nubes, sopló el
viento, comenzó a caer una fuerte lluvia y Acab bajó a caballo hasta
Jezrel. \bibverse{46} El Señor le dio su poder a Elías: se metió el
manto en el cinturón y corrió delante de Acab hasta Jezrel.

\hypertarget{section-18}{%
\section{19}\label{section-18}}

\bibverse{1} Acab le contó a Jezabel todo lo que había hecho Elías y que
había matado a espada a todos los profetas de Baal. \bibverse{2}
EntoncesJezabel le envió un mensajero a Elías para decirle: ``¡Que los
dioses me hagan tanto y más si para mañana no he hecho que tu vida sea
como la de los que mataste!''

\bibverse{3} Elías tuvo miedo y corrió por su vida. Cuando llegó a
Beerseba, en Judá, dejó allí a su criado \bibverse{4} y se adentró un
día más en el desierto. Se sentó bajo un arbusto y pidió morir. ``Ya
estoy harto, Señor'', dijo. ``¡Toma mi vida! No soy mejor que mis
antepasados''.

\bibverse{5} Se acostó y se durmió bajo el arbusto. De repente, un ángel
le tocó y le dijo: ``Levántate y come''. 6 Miró a su alrededor, y allí,
junto a su cabeza, había un poco de pan cocido sobre brasas y una jarra
de agua. Comió y bebió y se acostó de nuevo.

\bibverse{7} El ángel del Señor volvió por segunda vez, lo tocó y le
dijo: ``Levántate y come, porque si no el viaje será demasiado para
ti.''

\bibverse{8} Así que se levantó, comió y bebió, y con la fuerza que le
dio la comida pudo caminar cuarenta días y cuarenta noches hasta el
monte Horeb,\footnote{\textbf{19:8} Otro de los nombres del Monte Sinaí.}la
montaña de Dios. \bibverse{9} Allí entró en una cueva y pasó la noche.

El Señor habló a Elías y le preguntó: ``¿Qué haces aquí, Elías?''.

\bibverse{10} ``He trabajado apasionadamente para el Señor Dios
Todopoderoso'', respondió. ``Pero los israelitas han abandonado tu
pacto, han derribado tus altares y han matado a tus profetas a espada.
Soy el único que queda, y también intentan matarme a mí''.

\bibverse{11} Entonces el Señor le dijo: ``Sal y ponte en el monte ante
el Señor''. Justo en ese momento el Señor pasó por allí. Un viento
tremendamente poderoso arrasó las montañas y destrozó las rocas ante el
Señor, pero el Señor no estaba en el viento. Después del viento vino un
terremoto, pero el Señor no estaba en el terremoto. \bibverse{12}
Después del terremoto vino el fuego, pero el Señor no estaba en el
fuego. Y después del fuego vino una voz que hablaba en un suave susurro.
\bibverse{13} Al oírla, Elías se envolvió el rostro con su manto y salió
y se puso a la entrada de la cueva. Inmediatamente una voz le habló y le
preguntó: ``¿Qué haces aquí, Elías?''.

\bibverse{14} ``He trabajado apasionadamente para el Señor Dios
Todopoderoso'', respondió. ``Pero los israelitas han abandonado tu
pacto, han derribado tus altares y han matado a tus profetas a espada.
Soy el único que queda, y también intentan matarme a mí''.

\bibverse{15} El Señor le dijo: ``Vuelve por donde has venido al
desierto de Damasco. Cuando llegues allí, ve y unge a Jazael, rey de
Aram. \bibverse{16} Unge también a Jehú, hijo de Nimsí, rey de Israel, y
a Eliseo, hijo de Safat, de Abel-Mejolá, para que te sustituya como
profeta.

\bibverse{17} Jehú ejecutará a quien escape de la espada de Jazael, y
Eliseo ejecutará a quien escape de la espada de Jehú. \bibverse{18}
Todavía me quedan siete mil en Israel, todos los que no han doblado sus
rodillas para adorar y cuyas bocas no lo han besado.''

\bibverse{19} Entonces Elías se fue y encontró a Eliseo, hijo de Safat.
Estaba arando con doce pares de bueyes, y él estaba con el duodécimo
par. Elías se acercó a él y le echó su manto. \bibverse{20} Eliseo dejó
los bueyes, corrió tras Elías y le dijo: ``Por favor, déjame ir y
despedirme de mi padre y de mi madre, y luego te seguiré''. ``Vete a
casa'', respondió Elías. ``Nunca he hecho nada por ti.''\footnote{\textbf{19:20}
  Lo que significa que no había ningún beneficio material en seguir a
  Elías.}

\bibverse{21} Eliseo lo dejó, tomó su par de bueyes y los sacrificó.
Utilizando la madera del yugo de los bueyes como combustible, cocinó la
carne y se la dio al pueblo, y ellos la comieron.\footnote{\textbf{19:21}
  Al tomar estas medidas, Eliseo indicó a todos que no volvería a
  utilizar los bueyes y el arado.}Luego se fue para seguir y servir a
Elías.

\hypertarget{section-19}{%
\section{20}\label{section-19}}

\bibverse{1} Ben Adad, rey de Aram, convocó a todo su ejército. Junto
con treinta y dos reyes y sus caballos y carros reunidos, marchó para
sitiar Samaria, para luchar contra ella. \bibverse{2} Envió mensajeros a
Acab, rey de Israel, a la ciudad para decirle: ``Esto es lo que dice Ben
Adad: \bibverse{3} ¡Tu plata y tu oro me pertenecen ahora, y tus mejores
esposas e hijos también me pertenecen!''

\bibverse{4} ``Es como dices, mi señor el rey'', respondió el rey de
Israel. ``Soy tuyo, así como todo lo que me pertenece''.

\bibverse{5} Los mensajeros regresaron y dijeron: ``Esto es lo que dice
Ben Adad: te he enviado un mensaje exigiendo que me des tu plata, tu
oro, tus esposas y tus hijos. \bibverse{6} Pero mañana a esta hora voy a
enviar a mis hombres a registrar tu palacio y las casas de tus
funcionarios. Tomarán y se llevarán todo lo que consideres valioso''.

\bibverse{7} El rey de Israel llamó a todos los ancianos del país y les
dijo: ``¡Miren cómo este hombre trata de causar problemas! Cuando exigió
mis esposas y mis hijos, mi plata y mi oro, no dije que no''.

\bibverse{8} Todos los ancianos y todo el pueblo presente respondieron:
``No lo escuchen. No aceptes sus exigencias''.

\bibverse{9} Entonces el rey dijo a los mensajeros de Ben Adad: ``Dile a
mi señor el rey: Todo lo que exigiste al principio lo hará tu servidor,
pero no puedo acceder a esta última exigencia''. Los mensajeros le
llevaron la respuesta.

\bibverse{10} Ben Adad le respondió: ``¡Que los dioses me hagan tanto y
más si queda suficiente polvo en Samaria para dar a mis súbditos un
puñado a cada uno!''

\bibverse{11} El rey de Israel le respondió: ``Dile esto: Un hombre que
se pone la armadura no debe presumir como quien se la
quita.''\footnote{\textbf{20:11} En otras palabras, sólo se debe
  presumir cuando se obtiene la victoria.}

\bibverse{12} Ben Adad recibió este mensaje mientras él y los reyes
estaban bebiendo en sus tiendas. Inmediatamente dio la orden a sus
oficiales: ``¡Prepárense para atacar!''. Así que se prepararon para
atacar la ciudad.

\bibverse{13} Al mismo tiempo, un profeta se acercó a Acab, rey de
Israel, y le dijo: ``Esto es lo que dice el Señor: ¿Ves este enorme
ejército? Sólo mira, porque hoy te haré victorioso, y te
convencerás\footnote{\textbf{20:13} ``Convencerás'': Literalmente,
  ``sabrás,'' pero esto es más que simplemente ser consciente de algo,
  más bien una creencia motivadora.}que yo soy el Señor.''

\bibverse{14} ``Pero ¿quién va a hacer esto?'' preguntó Acab. El profeta
respondió: ``Esto es lo que dice el Señor: serán los oficiales jóvenes
bajo los comandantes de distrito''.

``¿Y quién va a iniciar la batalla?'', preguntó. El profeta respondió:
``¡Tú!''.

\bibverse{15} Así que Acab convocó a los 232 oficiales jóvenes de los
comandantes de distrito y reunió a los 7.000 soldados que formaban el
ejército de Israel. \bibverse{16} Partieron al mediodía, mientras Ben
Adad y los treinta y dos reyes que lo acompañaban estaban ocupados
emborrachándose en sus tiendas. \bibverse{17} Los jóvenes oficiales de
los comandantes de distrito tomaron la delantera. Los exploradores que
Ben-hadad había enviado vinieron y le informaron: ``Los soldados
enemigos avanzan desde Samaria''.

\bibverse{18} ``Si vienen en son de paz, tómenlos vivos'', ordenó. ``Si
vienen a atacar, tómenlos vivos''.

\bibverse{19} Los jóvenes oficiales de los comandantes de distrito
avanzaron desde la ciudad, seguidos por el ejército. \bibverse{20} Cada
hombre mató a su oponente, y los arameos huyeron. Los israelitas los
persiguieron, pero Ben Adad, rey de Aram, escapó a caballo con su
caballería. \bibverse{21} Entonces el rey de Israel salió y atacó a los
caballos y a los carros. Infligió una gran derrota a los arameos.

\bibverse{22} Más tarde el profeta se presentó ante el rey de Israel y
le dijo: ``Ve a reforzar tus defensas y revisa lo que debes hacer,
porque en la primavera el rey de Aram vendrá a atacarte de nuevo.''

\bibverse{23} Mientras tanto, los oficiales del rey de Aram le dijeron:
``Sus dioses son dioses de las montañas. Por eso pudieron derrotarnos.
Pero si luchamos contra ellos en las tierras bajas, podremos vencerlos.
\bibverse{24} Debes hacer lo siguiente: destituir a cada uno de los
reyes de sus cargos y sustituirlos por comandantes. \bibverse{25}
También tienes que levantar otro ejército para reemplazar el que
perdiste: caballo por caballo, carro por carro. Entonces podremos luchar
contra ellos en las tierras bajas y los venceremos definitivamente''.
Ben Adad escuchó sus consejos e hizo lo que le dijeron.

\bibverse{26} Cuando llegó la primavera, Ben Adad convocó al ejército
arameo y fue a atacar a Israel en Afec. \bibverse{27} El ejército
israelita también fue convocado y aprovisionado. Fueron a enfrentar a
los arameos. Pero cuando los israelitas instalaron su campamento frente
al enemigo, parecían un par de rebaños de cabras en comparación con el
ejército arameo que llenaba toda la tierra.

\bibverse{28} Entonces el hombre de Dios se acercó al rey de Israel y le
dijo: ``Esto es lo que dice el Señor: Como los arameos han dicho: `El
Señor es sólo un dios de las montañas y no de los valles', yo te haré
victorioso sobre todo este enorme ejército. Entonces se convencerán de
que yo soy el Señor''.

\bibverse{29} Los ejércitos acamparon uno frente al otro durante siete
días. Al séptimo día tuvo lugar la batalla. Los israelitas mataron a
100.000 de la infantería aramea en un solo día. \bibverse{30} El resto
huyó a la ciudad de Afec, donde un muro se derrumbó sobre 27.000 de los
que quedaron. Ben Adad también corrió a la ciudad y se escondió en una
habitación interior.

\bibverse{31} Los oficiales de Ben Adad le dijeron: ``Mira, hemos oído
que los reyes israelitas son misericordiosos. Vamos a rendirnos ante el
rey de Israel, llevando sacos en la cintura y cuerdas en la cabeza.
Quizá os deje vivir''.

\bibverse{32} Así que, llevando cilicio en la cintura y cuerdas en la
cabeza, fueron y se rindieron al rey de Israel, y le dijeron: ``Tu
siervo Ben Adad te pide: ``Por favor, déjame vivir''. El rey respondió:
``¿Sigue vivo? Lo considero mi hermano''.

\bibverse{33} Los hombres pensaron que esto era una buena señal e
inmediatamente le tomaron la palabra al rey, diciendo: ``Sí, Ben-Adad es
tu hermano.''

``¡Vayan a buscarlo!'', dijo el rey. Así que Ben-hadad salió de su
escondite y se entregó a Acab, quien lo subió a su carro.

\bibverse{34} Ben-hadad le dijo: ``Te devolveré las ciudades que mi
padre tomó de tu padre,\footnote{\textbf{20:34} Véase 15:20.}y podrás
organizar tus propios lugares de comercio en Damasco, como hizo mi padre
en Samaria''.

``Al hacer este pacto te libero'', respondió Acab. Hizo un tratado con
Ben Adad y lo dejó ir.

\bibverse{35} A raíz de un mensaje que recibió del Señor, uno de los
hijos de los profetas\footnote{\textbf{20:35} ``Hijos de los profetas:''
  también llamada ``la escuela de los profetas'' era una especie de
  institución de educación religiosa y un centro del don profético.}le
dijo a su colega: ``Por favor, pégame''. Pero el hombre se negó a
pegarle. \bibverse{36} Entonces el profeta le dijo: ``Como no has hecho
lo que dijo el Señor, en cuanto me dejes un león te va a matar''. Cuando
el hombre se fue, vino un león y lo mató.

\bibverse{37} El profeta encontró a otro hombre y le dijo: ``Por favor,
pégame''. Entonces el hombre lo golpeó, hiriéndolo.

\bibverse{38} Entonces el profeta fue y se quedó junto al camino,
esperando al rey. Se había disfrazado con una venda sobre los ojos.

\bibverse{39} Al pasar el rey, le gritó ``Tu siervo había salido a
luchar en medio de la batalla, cuando de repente se acercó un hombre con
un prisionero y me dijo: `¡Guarda a este hombre! Si por alguna razón se
escapa, pagarás su vida con la tuya, o serás multado con un talento de
plata'. \bibverse{40} Pero mientras tu siervo estaba ocupado en otras
cosas, el hombre se escapó''.

``Así que ese será tu castigo,'' le dijo el rey de Israel. ``Tú mismo te
has condenado''.

\bibverse{41} Entonces el profeta se quitó rápidamente la venda de los
ojos, y el rey de Israel reconoció que era uno de los profetas.
\bibverse{42} Le dijo al rey: ``Esto es lo que dice el Señor: Has dejado
ir a un hombre que yo había decidido que muriera. Por lo tanto, pagarás
su vida con tu vida, tu pueblo por su pueblo''.

\bibverse{43} El rey de Israel regresó a su casa en Samaria, enfadado y
furioso.

\hypertarget{section-20}{%
\section{21}\label{section-20}}

\bibverse{1} Algún tiempo después sucedió esto: Había un hombre llamado
Nabot, de Jezrel, que tenía una viña en Jezrel, cerca del palacio del
rey Acab en Samaria. \bibverse{2} Acab fue a ver a Nabot y le dijo:
``Dame tu viña para que la convierta en un huerto, porque está cerca de
mi palacio. A cambio te daré una viña mejor, o si quieres te la pagaré
al contado''.

\bibverse{3} Pero Nabot respondió: ``Que el Señor me maldiga si te doy
la herencia de mis antepasados.''\footnote{\textbf{21:3} Según la ley
  levítica, la herencia debía mantenerse según las asignaciones tribales
  originales. Véase Números 36:7-9.}

\bibverse{4} Acab se fue a casa malhumorado y enfurecido porque Nabot de
Jezrel le había dicho: ``No te daré la herencia de mis antepasados''. Se
acostó, no quiso mirar a nadie y se negó a comer.

\bibverse{5} Su esposa Jezabel entró y le preguntó: ``¿Por qué estás tan
molesto? ¿Que no quieres comer?''

\bibverse{6} Acab respondió: ``Es porque hablé con Nabot de Jezrel y le
pedí: `Dame tu viña por dinero, o si quieres, te daré otra viña en su
lugar'. Pero él dijo: `No te daré mi viña'``.

\bibverse{7} ``¿No eres tú el rey de Israel?'', le respondió su mujer
Jezabel. ``Levántate, come algo y anímate. Te conseguiré la viña de
Nabot de Jezrel''.

\bibverse{8} Entonces ella escribió unas cartas en nombre de Acab y las
selló con su sello. Envió las cartas a los ancianos y a los dirigentes
de la ciudad donde vivía Nabot. \bibverse{9} En las cartas les decía:
``Anuncien un ayuno religioso y den a Nabot un asiento de honor.
\bibverse{10} Pero sentar a dos hombres malos\footnote{\textbf{21:10}
  ``Hombres malos'': Literalmente, ``hijos de maldad.''}frente a él y
haz que lo acusen, diciendo: ``¡Has maldecido a Dios y al rey! Entonces
sáquenlo y mátenlo a pedradas''.

\bibverse{11} Así que los ancianos y los líderes que vivían en la ciudad
de Nabot hicieron lo que Jezabel había dicho en las cartas que les había
escrito y enviado. \bibverse{12} Anunciaron un ayuno religioso y le
dieron a Nabot un asiento de honor. \bibverse{13} Vinieron dos hombres
malos, se sentaron frente a él y lo acusaron delante del pueblo,
diciendo: ``Nabot ha maldecido a Dios y al rey.'' Así que lo llevaron
fuera de la ciudad y lo apedrearon hasta que murió.

\bibverse{14} Luego enviaron un mensaje a Jezabel, diciendo: ``Nabot ha
sido apedreado. Ha muerto''.

\bibverse{15} En cuanto Jezabel se enteró de que Nabot había sido
apedreado y que estaba muerto, le dijo a Acab: ``Levántate, ve y reclama
a Jezrel la propiedad de la viña de Nabot, que se negó a venderte,
porque Nabot ya no vive, sino que está muerto.''

\bibverse{16} CuandoAcab supo que Nabot había muerto, se levantó y fue a
reclamar la propiedad de la viña de Nabot.

\bibverse{17} Entonces el Señor envió un mensaje a Elías el tisbita:
\bibverse{18} ``Ve a encontrarte con Acab, rey de Israel, en Samaria.
Ahora mismo está en la viña de Nabot, donde ha ido a reclamar su
propiedad. \bibverse{19} Dile: ``Esto es lo que dice el Señor: `¿Has
asesinado a un hombre y le has robado?'\,'' Entonces dile: ``Esto es lo
que dice el Señor: `En el mismo lugar donde los perros lamieron la
sangre de Nabot, los perros lamerán tu propia sangre.'\,''

\bibverse{20} ``¿Así que has venido a buscarme, mi enemigo?'' preguntó
Acab a Elías. ``Te he encontrado, porque te has vendido a hacer lo que
es malo a los ojos del Señor'', respondió Elías.

\bibverse{21} Dijo: ``¡Cuidado! Voy a traer el desastre sobre ti y
destruiré a tu descendencia. Mataré a todo varón del linaje de Acab,
tanto esclavo como libre, en todo Israel. \bibverse{22} Haré que tu casa
sea como la de Jeroboam, hijo de Nabat, y como la de Basá, hijo de
Ahías, porque me has irritado y has hecho pecar a Israel. \bibverse{23}
En cuanto a Jezabel, el Señor dice: ``Los perros se comerán a Jezabel
junto al muro de Jezrel''. \bibverse{24} Los de la familia de Acab que
mueran en la ciudad serán comidos por los perros, y los que mueran en el
campo serán comidos por las aves.''

\bibverse{25} (Nadie fue tan malo como Acab, que se vendió para hacer lo
que es malo a los ojos del Señor, porque su esposa Jezabel lo animó.
\bibverse{26} Hizo las cosas más despreciables, adorando a los ídolos
como los amorreos que el Señor había expulsado delante de Israel).

\bibverse{27} En cuanto Acab escuchó este mensaje, se rasgó las
vestiduras, se vistió de cilicio y ayunó. Incluso se acostó en tela de
silicio, y caminabaarrepentido.\footnote{\textbf{21:27} ``Arrepentido'':
  Literalmente, ``caminaba con cuidado.''}

\bibverse{28} Entonces el Señor envió un mensaje a Elías tisbita:
\bibverse{29} ``¿Has visto cómo se ha humillado Acab ante mí? Porque se
ha humillado ante mí, no traeré el desastre durante su vida, sino que
haré caer el desastre sobre su familia en vida de su hijo.''

\hypertarget{section-21}{%
\section{22}\label{section-21}}

\bibverse{1} Durante tres años Aram e Israel no estuvieron en guerra.
\bibverse{2} Pero al tercer año Josafat, rey de Judá, fue a visitar al
rey de Israel. \bibverse{3} El rey de Israel había dicho a sus
oficiales: ``¿No se dan cuenta de que Ramot de Galaad nos pertenece
realmente y sin embargo no hemos hecho nada para recuperarla del rey de
Aram?''

\bibverse{4} Entonces le preguntó a Josafat: ``¿Te unirás a mí en un
ataque para reconquistar Ramot de Galaad?''

Josafat respondió al rey de Israel: ``Tú y yo somos como uno, mis
hombres y tus hombres son como uno, y mis caballos y tus caballos son
como uno.'' \bibverse{5} Entonces Josafat dijo al rey de Israel: ``Pero
antes, por favor, averigua lo que dice el Señor''.

\bibverse{6} Entonces el rey de Israel sacó a los profetas
-cuatrocientos- y les preguntó: ``¿Debo subir a atacar Ramot de Galaad,
o no?''

``Sí, adelante'', le respondieron, ``porque el Señor la entregará al
rey''.

\bibverse{7} Pero Josafat preguntó: ``¿No hay aquí otro profeta del
Señor al que podamos preguntar?''

\bibverse{8} ``Sí, hay otro hombre que podría consultar al Señor'',
respondió el rey de Israel, ``pero no me gusta porque nunca profetiza
nada bueno para mí, ¡siempre es malo! Se llama Micaías, hijo de Imá''.

``No deberías hablar así'', dijo Josafat.

\bibverse{9} El rey de Israel llamó a uno de sus funcionarios y le dijo:
``Tráeme enseguida a Micaías, hijo de Imá''.

\bibverse{10} Vestidos con sus ropas reales, el rey de Israel y el rey
Josafat de Judá, estaban sentados en sus tronos en la era junto a la
puerta de Samaria, con todos los profetas profetizando frente a ellos.
\bibverse{11} Uno de ellos, Sedequías, hijo de Quená, se había hecho
unos cuernos de hierro. Anunció: ``Esto es lo que dice el Señor: '¡Con
estos cuernos vas a corromper a los arameos hasta matarlos!''

\bibverse{12} Y todos los profetas profetizaban lo mismo, diciendo:
``Adelante, ataquen Ramot de Galaad; tendrán éxito, porque el Señor se
la entregará al rey.''

\bibverse{13} Entonces el mensajero que fue a llamar a Micaías le dijo:
``Mira, todos los profetas son unánimes en profetizar positivamente al
rey. Así que asegúrate de hablar positivamente como ellos''.

\bibverse{14} PeroMicaías respondió: ``Vive el Señor, yo sólo puedo
decir lo que mi Dios me dice''.

\bibverse{15} Cuando llegó ante el rey, éste le preguntó: ``¿Subimos a
atacar Ramot de Galaad, o no?''

``Sí, suban y salgan victoriosos'', respondió Micaías, ``porque el Señor
entregará la ciudad en manos del rey.''\footnote{\textbf{22:15} Tal vez
  Micaías está utilizando una repetición sarcástica de los otros
  profetas, lo que lleva a Acab a responder como lo hace en el siguiente
  verso.}

\bibverse{16} Pero el rey le dijo: ``¿Cuántas veces tengo que hacerte
jurar que sólo me dirás la verdad en nombre del Señor?''

\bibverse{17} EntoncesMicaías respondió: ``Vi a todo Israel disperso por
los montes como ovejas sin pastor. El Señor dijo: 'Este pueblo no tiene
dueño;\footnote{\textbf{22:17} ``No tiene dueño'': dando a entender que
  su amo está muerto.}que cada uno se vaya a su casa en paz''.

\bibverse{18} El rey de Israel le dijo a Josafat: ``¿No te he dicho que
él nunca me profetiza nada bueno, sino sólo malo?''

\bibverse{19} Micaías continuó diciendo: ``Escucha, pues, lo que dice el
Señor. Vi al Señor sentado en su trono, rodeado de todo el ejército del
cielo que estaba a su derecha y a su izquierda. \bibverse{20} El Señor
preguntó: ``¿Quién engañará a Acab, rey de Israel, para que ataque a
Ramot de Galaad y lo mate allí?

Uno dijo esto, otro dijo aquello, y otro dijo otra cosa. \bibverse{21}
Finalmente vino un espíritu y se acercó al Señor y dijo: `Yo lo
engañaré'.

\bibverse{22} ``¿Cómo vas a hacerlo?'', preguntó el Señor.

Iré y seré un espíritu mentiroso y haré que todos sus profetas digan
mentiras'', respondió el espíritu.

El Señor respondió: ``Eso funcionará''. Ve y hazlo'.

\bibverse{23} Como ves, el Señor ha puesto un espíritu mentiroso en
estos profetas tuyos, y el Señor ha dictado tu sentencia de muerte.''

\bibverse{24} EntoncesSedequías, hijo de Quená, fue y abofeteó a Micaías
en la cara, y le preguntó: ``¿A dónde se fue el Espíritu del Señor
cuando me dejó hablar contigo?''

\bibverse{25} ``¡Pronto lo descubrirás cuando intentes encontrar algún
lugar secreto para esconderte!'' respondió Micaías.

\bibverse{26} El rey de Israel ordenó: ``Pongan a Micaías bajo arresto y
llévenlo a Amón, el gobernador de la ciudad, y a mi hijo Joás.
\bibverse{27} Diles que estas son las instrucciones del rey: `Pongan a
este hombre en la cárcel. Denle sólo pan y agua hasta mi regreso
seguro'\,''.

\bibverse{28} ``Si de hecho regresas sano y salvo, entonces el Señor no
ha hablado a través de mí'', declaró Micaías. ``¡Presten atención todos
a todo lo que he dicho!''

\bibverse{29} El rey de Israel y Josafat, rey de Judá, fueron a atacar
Ramot de Galaad. \bibverse{30} El rey de Israel le dijo a Josafat:
``Cuando yo vaya a la batalla me disfrazaré, pero tú debes llevar tus
ropas reales''. Así que el rey de Israel se disfrazó y fue a la batalla.

\bibverse{31} El rey de Aram ya había dado estas órdenes a sus
comandantes de carros ``Diríjanse directamente hacia el rey de Israel
solo. No luchen con nadie más, sea quien sea''.

\bibverse{32} Así que cuando los comandantes de los carros vieron a
Josafat, gritaron: ``¡Este debe ser el rey de Israel!'' Así que se
volvieron para atacarlo, pero cuando Josafat pidió ayuda, \bibverse{33}
los comandantes de los carros vieron que no era el rey de Israel y
dejaron de perseguirlo.

\bibverse{34} Sin embargo, un arquero enemigo disparó una flecha al
azar, hiriendo al rey de Israel entre las junturas de su armadura, junto
al peto. El rey le dijo a su auriga: ``¡Da la vuelta y sácame del
combate, porque me han herido!''.

\bibverse{35} La batalla duró todo el día. El rey de Israel se apuntaló
en su carro para enfrentarse a los arameos, pero al anochecer murió. La
sangre se había derramado de su herida sobre el piso del carro.
\bibverse{36} Al atardecer, un grito salió de las filas: ``¡Retírense!
Cada uno vuelva a su ciudad, cada uno vuelva a su país''.

\bibverse{37} Así murió el rey. Lo llevaron de vuelta a Samaria, donde
lo enterraron. \bibverse{38} Lavaron su carro en un estanque de Samaria
donde las prostitutas venían a bañarse, y los perros lamieron su sangre,
tal como el Señor había dicho.

\bibverse{39} El resto de lo que sucedió en el reinado de Acab, todo lo
que hizo, el palacio de marfil que construyó y todas las ciudades que
edificó, están registrados en el Libro de las Crónicas de los Reyes de
Israel. \bibverse{40} Acab murió y su hijo Ocozías lo sucedió como rey.

\bibverse{41} Josafat, hijo de Asá, llegó a ser rey de Judá en el cuarto
año del reinado de Acab, rey de Israel. \bibverse{42} Josafat tenía
treinta y cinco años cuando llegó a ser rey, y reinó en Jerusalén
durante veinticinco años. Su madre se llamaba Azuba, hija de Silhi.
\bibverse{43} Siguió todos los caminos de su padre; no se apartó de
ellos, e hizo lo correcto a los ojos del Señor. Sin embargo, los altares
paganos no fueron destruidos y el pueblo siguió sacrificando y
presentando ofrendas allí. \bibverse{44} Josafat también hizo la paz con
el rey de Israel.

\bibverse{45} El resto de lo que sucedió en el reinado de Josafat, sus
grandes logros y las guerras que libró están registrados en el Libro de
las Crónicas de los Reyes de Judá. \bibverse{46} Expulsó del país a las
prostitutas del culto que quedaban de la época de su padre Asa.
\bibverse{47} (En esa época no había rey en Edom; sino que había un
diputado que hacía las veces de rey). \bibverse{48} Josafat construyó
barcos marítimos\footnote{\textbf{22:48} 22:48 ``barcos marítimos'':
  literalmente ``los barcos de Tarsis'' para indicar que estaban
  construidos para largas distancias. Véase 2 Crónicas 20:35-37.}para ir
a Ofir en busca de oro, pero se fueron porque naufragaron en
Ezión-guéber. \bibverse{49} En ese tiempo Ocozías, hijo de Acab, le
pidió a Josafat: ``Deja que mis hombres naveguen con los tuyos'', pero
Josafat se negó.

\bibverse{50} Josafat murió y fue enterrado con sus antepasados en la
Ciudad de David. Su hijo Jehoram lo sucedió como rey.

\bibverse{51} Ocozías, hijo de Acab, se convirtió en rey de Israel en
Samaria en el año diecisiete de Josafat, rey de Judá, y reinó sobre
Israel durante dos años. \bibverse{52} Sus hechos fueron malos a los
ojos del Señor y siguió los caminos de su padre y de su madre, y de
Jeroboam, hijo de Nabat, que había hecho pecar a Israel. \bibverse{53}
Sirvió a Baal y lo adoró, y enfureció al Señor, el Dios de Israel, tal
como lo había hecho su padre.
