\hypertarget{section}{%
\section{1}\label{section}}

\bibverse{1} El Señor envió un mensaje a Oseas, hijo de Berí, en el
tiempo en que Uzías, Jotam, y Acaz eran los reyes de Judá. Y Jeroboam,
hijo de Joás\footnote{\textbf{1:1} Deletreado aquí como ``Joash.''} era
el rey de Israel. \bibverse{2} El Señor comenzó hablando a través de
Oseas diciéndole: ``Ve y busca una prostituta para casarte con ella, y
ten hijos de una prostituta,\footnote{\textbf{1:2} Literalmente ``hijos
  de prostitución.'' Existe gran debate sobre esta frase. Algunos
  piensan que esta frase indica que Gomer había tenido hijos como
  resultado de su prostitución antes de casarse con Oseas. Otros creen
  que esto ocurrió después cuando ella volvió a sus antiguos caminos
  (nótese que solo el primer hijo, Jezreel, está identificado como hijo
  de Oseas). No obstante, otros creen que esto se refiere a que
  cualquier hijo que Oseas tuviera con Gomer sería manchado por su
  reputación de prostituta.} porque los habitantes de esta tierra se han
prostituido terriblemente al alejarse del Señor.''

\bibverse{3} Así que Oseas fue y se casó con Gómer, la hija de Diblaim.
Ella quedó embarazada y le dio un hijo a Oseas. \bibverse{4} Entonces el
Señor le dijo a Oseas: ``Ponle por nombre Jezreel,\footnote{\textbf{1:4}
  Jezreel significa ``el Señor sembrará'' (o esparcirá, ya que esta era
  la forma como se sembraban las semillas). Jezreel fue históricamente
  un sitio de mucha violencia y derramamiento de sangre.} porque yo
castigaré la casa de Jehú por la sangre que derramó sobre Jezreel; y yo
pondré fin al reino de Israel. \bibverse{5} Ese día yo quebrantaré al
ejército de Israel en el valle de Jezreel.''

\bibverse{6} Gómer volvió a quedar embarazada y esta vez tuvo una hija.
El Señor le dijo entonces a Oseas: ``Ponle por nombre
Lo-ruhama,\footnote{\textbf{1:6} El nombre significa ``no amada.''}
porque no amaré más a la casa de Israel y de seguro no los perdonaré.
\bibverse{7} Pero tendré piedad de los de la casa de Judá y los salvaré.
Pero no será con arco ni espada, ni tampoco con guerreros de a
caballo.''

\bibverse{8} Después que ya Gómer dejó de amamantar a Lo-ruhamah, volvió
a quedar embarazada y tuvo un hijo. \bibverse{9} Entonces el Señor le
dijo a Oseas: ``Ponle por nombre Lo-ammi,\footnote{\textbf{1:9} El
  nombre significa ``no mi pueblo.''} porque ustedes ya no son mi pueblo
y yo ya no soy su Dios.\footnote{\textbf{1:9} Literalmente, ``Yo no soy
  de ustedes.''} \bibverse{10} Sin embargo,\footnote{\textbf{1:10} ``Sin
  embargo'' es una transición añadida. El hebreo simplemente dice:
  ``y''.} el número de habitantes del pueblo de Israel será como la
arena del mar, que no podrá medirse ni contarse. Entonces, justo en el
lugar donde se les dijo `Ya no son mi pueblo' se les llamará `los hijos
del Dios vivo.' \bibverse{11} El pueblo de Israel y el pueblo de Judá se
reunirán y ellos mismos elegirán un líder, y tomarán posesión de la
tierra. Y el día de Jezreel será grande.

\hypertarget{section-1}{%
\section{2}\label{section-1}}

\bibverse{1} Ese día llamarás a tus hermanos Ammi, y a tus hermanas
Ruhama.\footnote{\textbf{2:1} Significa: 'mi pueblo,'' y ``apiadado de
  ella'' o ``amada.''} \bibverse{2} Condenen a su madre, condénenla
porque no es mi esposa y yo no soy su esposo. Pídanle que deje de lucir
como una prostituta, y que se quite el maquillaje y las vestiduras
provocativas.\footnote{\textbf{2:2} Literalmente, ``Que se quite la
  fornicación de su rostro y el adulterio de entre sus senos.'' El
  significado exacto no está claro. Algunos significados posibles son:
  la apariencia sugestiva de una prostituta; el maquillaje y las prendas
  de una prostituta; vestimenta reveladora; o las acciones de una
  prostituta con sus amantes. Sea cual sea el caso, el sentido principal
  es claro: Israel debe dejar de prostituirse con otros dioses.}

\bibverse{3} De lo contrario la dejaré desnuda, tal como el día en que
nació, y la convertiré en un desierto, en un terreno árido, y la dejaré
morir de sed. \bibverse{4} No tendré piedad de sus hijos porque son
hijos de prostitución. \bibverse{5} Pues su madre era un prostituta que
los concibió de manera vergonzosa. Ella dijo: `Buscaré a mis amantes que
me proveen comida y agua, así como la lana, el lino, el aceite de oliva
y me dan de beber.'

\bibverse{6} Por eso voy a obstaculizar su camino con arbustos de
espinas, y construiré un muro de piedra para detenerla y que no
encuentre forma de continuar. \bibverse{7} Cuando vaya en busca de sus
amantes no podrá hallarlos; los buscará pero no los encontrará. Entonces
dirá: `Volveré a mi ex esposo, porque estaba mejor con él que ahora.'

\bibverse{8} A ella se le olvidó que yo fui quien le dio grano, vino
nuevo y aceite de oliva, así como la plata y el oro que le di en
abundancia y que usaron para hacerle un ídolo a Baal. \bibverse{9} Así
que recuperaré mi grano maduro y mi nuevo vino que di en tiempo de
cosecha. Recuperaré mi lana y el lino que le di para cubrir su desnudez.
\bibverse{10} La dejaré desnuda ante la mirada de sus amantes, y ninguno
podrá rescatarla de mi. \bibverse{11} Pondré fin a sus festividades: sus
fiestas, celebraciones de luna nueva, sus días de reposo, y todos sus
festivales.\footnote{\textbf{2:11} Aunque las celebraciones mencionadas
  aquí son parte del calendario religioso, parece que estas se habían
  convertido en una excusa para festejar más que en una observancia
  sincera.} \bibverse{12} Destruiré sus viñedos e higueras que decía
haber recibido como pago por ser una prostituta. Los convertiré en
matorrales y solo los animales salvajes comerán de los frutos que
queden. \bibverse{13} Yo la castigaré por todas las veces que le ofreció
incienso a Baal, vestida, usando anillos y joyas, buscando a sus
amantes, y olvidándose de mi,'' dice el Señor.

\bibverse{14} Miren lo que voy a hacer: Haré que vuelva a mi, la llevaré
al desierto, y allí hablaré a su corazón. \bibverse{15} Le devolveré los
viñedos y convertiré el Valle de Acor\footnote{\textbf{2:15} ``Acor''
  significa ``aflicción.'' Ver Josué 7:26.} en una puerta de esperanza.
Ella me responderá de la misma manera que lo hizo cuando era joven, como
cuando salió de Egipto. \bibverse{16} Ese día, dice el Señor, tú me
volverás a llamar `mi esposo' y no `mi Baal.'\footnote{\textbf{2:16}
  ``Mi Baal'' normalmente se traduciría como ``mi Señor,'' pero debido a
  que el contexto se refiere a Israel siguiendo religiones paganas que
  llamaban a su(s) dios(es) ``Baal'' entonces es más apropiado llamarlo
  así aquí. Además, Dios desea una relación más cercana con Israel como
  de un esposo, y no simplemente como su Señor.} \bibverse{17} Haré que
deje de invocar a los baales, y sus nombres no serán mencionados nunca
más.

\bibverse{18} En ese momento haré un acuerdo solemne\footnote{\textbf{2:18}
  O ``pacto.'' Aquí está implícito que estas criaturas no le harían daño
  al pueblo de Israel.} con los animales salvajes y las aves del cielo,
así como todo lo que se arrastra sobre el suelo. Voy a deshacerme del
arco y la espada; aboliré la guerra de la tierra para que puedas
acostarte en paz. \bibverse{19} Serás mi esposa para siempre. Te haré mi
esposa en bondad y justicia. \bibverse{20} Seré fiel a ti, mi esposa, y
tú me reconocerás como el Señor. \bibverse{21} Ese día, declara el
Señor, yo le responderé a los cielos y ellos le responderán a la
tierra.\footnote{\textbf{2:21} Esto a menudo se dice para indicar que
  las nubes le proveerían la lluvia a la tierra.} \bibverse{22} La
tierra responderá al grano, y al nuevo vino, y al aceite de oliva y
ellos responderán `Jezreel' (Dios muestra). \bibverse{23} Yo la
`sembraré' para mi mismo en la tierra. Amaré a Lo-ruhamah (no amada) y a
Lo-ammi (no mi pueblo) le diré: `tú eres mi pueblo,' y me dirá: `Tú eres
mi Dios.'''

\hypertarget{section-2}{%
\section{3}\label{section-2}}

\bibverse{1} El Señor me dijo: ``Ve una vez más y ama a una
mujer\footnote{\textbf{3:1} ``Una mujer'' -- Se entiende que se refiere
  a Gomer, considerando lo que sigue en el texto. Es posible que no se
  identifique específicamente a Gomer, porque ella ya no puede reclamar
  que es la esposa de Oseas debido a su adulterio.} que es amada por
otro y es adúltera, así como el Señor ama a los hijos de Israel, aunque
ellos buscan a otros dioses y se deleitan en su adoración
sensual.''\footnote{\textbf{3:1} Literalmente ``tortas de uva pasa,'' lo
  cual era parte de los rituales paganos de adoración.} \bibverse{2} Así
que la compré de nuevo por quince siclos de plata y un homer y medio de
cebada.\footnote{\textbf{3:2} Aproximadamente el precio de un esclavo
  común. No está claro por qué Oseas debía comprarla. De alguna manera
  ella se había convertido en ``propiedad'' de otra persona.}
\bibverse{3} Y le dije: ``Debes quedarte conmigo por muchos días, y
abandonarás la prostitución. No tendrás intimidad con ningún hombre, y
entonces seré tuyo.'' \bibverse{4} Porque el pueblo de Israel durará
muchos días sin rey ni príncipe, sin altar de piedras, sin imágenes
paganas ni santuarios idólatras.\footnote{\textbf{3:4} La frase ``sin
  altar de piedras, sin imágenes paganas ni santuarios idólatras'' son
  todos los aspectos del adulterio de Israel al seguir otros dioses.
  Incluso las palabras ``rey'' y ``príncipe'' eran símbolos del rechazo
  de Israel hacia Dios como su líder.} \bibverse{5} Después de esto, el
pueblo de Israel volverá y se dedicará al Señor su Dios y al linaje de
David, su rey. En los últimos días vendrán con asombro y reverencia por
el Señor y su bondad.

\hypertarget{section-3}{%
\section{4}\label{section-3}}

\bibverse{1} ¡Escucha la palabra del Señor, pueblo de Israel, porque el
Señor tiene una acusación contra los habitantes de la tierra! ``No hay
fidelidad, lealtad ni conocimiento de Dios en la tierra. \bibverse{2}
Por el contrario, solo hay maldiciones, mentiras, asesinatos, robos y
adulterio. Cometen actos violentos y hay gran derramamiento de sangre.
\bibverse{3} Por causa de esto, la tierra se seca,\footnote{\textbf{4:3}
  O ``está de luto.''} y todos los que viven en ella se consumen, así
como los animales del campo y las aves del cielo, y los peces del mar.
Todos están muriendo. \bibverse{4} Pero nadie debe culpar a otro. Mi
disputa es con ustedes, los sacerdotes, pues ustedes son los
responsables.\footnote{\textbf{4:4} Parece que el significado aquí es
  que los sacerdotes han hecho un trabajo tan mediocre al representar a
  Dios y al guiar al pueblo de manera correcta, que no tiene sentido que
  la gente común se culpe unos a otros.} \bibverse{5} Por ello ustedes
tropezarán a plena luz del día, y el profeta\footnote{\textbf{4:5}
  Presumiblemente los falsos profetas.} tropezará igual que ustedes por
la noche, y yo destruiré a la madre de todos ustedes.\footnote{\textbf{4:5}
  Madre: Es decir, la nación de Israel.}

\bibverse{6} Mi pueblo está muriendo porque no me conoce. Y como se
niegan a conocerme, yo los aborrezco a ustedes como mis sacerdotes. Así
como han olvidado mis enseñanzas, yo me olvidaré de sus hijos.
\bibverse{7} Cuantos más había,\footnote{\textbf{4:7} Refiriéndose a los
  sacerdotes.} tanto más pecaban contra mí. Por lo tanto yo transformaré
su gloria en vergüenza. \bibverse{8} Ellos se alimentaban del
pecado\footnote{\textbf{4:8} O ``ofrendas de pecado.'' Ya que los
  sacerdotes recibían parte de las ofrendas, les convenía aumentar la
  necesidad de hacer sacrificios.} de mi pueblo, y estaban hambrientos
de su inmoralidad. \bibverse{9} Por lo tanto, al pueblo le ocurrirá lo
mismo que a los sacerdotes: Los castigaré por lo que han hecho, y les
pagaré por sus acciones. \bibverse{10} Comerán, pero no se sentirán
saciados; se entregarán a la prostitución,\footnote{\textbf{4:10} La
  prostitución aquí puede tener varios significados: el acto básico de
  adulterio, visitar a las sacerdotisas/prostitutas del templo como
  parte de la ``adoración'' pagana, y la prostitución espiritual al
  romper los votos hacia el Dios verdadero, al seguir a los dioses
  paganos.} pero no prosperarán porque han abandonado al Señor para
prostituirse con otros dioses. \bibverse{11} Ellosdestruyen sus mentes
con el vino viejo y el vino nuevo.

\bibverse{12} Mi pueblo pide consejo a sus ídolos de madera y sus cetros
de adivinación les dan respuestas, ya que un espíritu de prostitución
los lleva a la perdición. Así, prostituyéndose, han abandonado a su
Dios. \bibverse{13} Ofrecen sacrificios en lo alto de las montañas;
queman incienso en las colinas bajo la apacible sombra de los árboles de
roble, álamo y pistacho. Por eso tus hijas se han prostituido y tus
nueras se han vuelto adúlteras. \bibverse{14} No castigaré a tus hijas
por prostituirse, ni a tus nueras por su adulterio, porque ustedes los
hombres visitan rameras y hacen sacrificios con las prostitutas del
templo. Un pueblo con falta de entendimiento termina en el desastre.

\bibverse{15} Aunque tú, Israel, te has vuelto una prostituta, espero
que Judá, no cometa la misma ofensa, y entre a Guilgal, ni vaya a
Bet-Aven,\footnote{\textbf{4:15} Un lugar llamado Guilgal era donde los
  israelitas cruzaban el Jordán y levantaban un monumento de piedra
  (véase Josué 4). Puede ser que Guilgal en realidad se refiera a un
  círculo de piedras, por lo que puede haber más de un lugar llamado
  así. Ciertamente, la condena de Oseas indica que este Guilgal en
  particular se había asociado con la adoración pagana. Bet-Aven es un
  error deliberado de Bet-el. Bet-el significa ``casa de Dios'', pero
  Oseas elige llamar al lugar Bet-Aven, que significa ``casa de la
  nada.''} ni jure diciendo: `el Señor vive.' \bibverse{16} Si Israel se
comporta como una vaca rebelde, ¿cómo podría el Señor cuidar de Israel
como un cordero en una gran pradera? \bibverse{17} Efraín\footnote{\textbf{4:17}
  Efraín era la tribu líder del reino del norte de Israel y, por lo
  tanto, a menudo se usaba para describir todo el reino.} está
encantado\footnote{\textbf{4:17} La palabra utilizada puede significar
  ``unido a'' o ``encantado, bajo un hechizo''.} por ídolos, así que
déjenlo en paz! \bibverse{18} Cuando los líderes dejan de beber, se van
y buscan prostitutas para acostarse con ellas. Se complacen más en sus
actos inmorales que en el honor. \bibverse{19} Pero un viento se los
llevará, y serán avergonzados por su adoración pagana.

\hypertarget{section-4}{%
\section{5}\label{section-4}}

\bibverse{1} ¡Escuchen esto, sacerdotes! ¡Presta atención, casa de
Israel! ¡Escuchen, miembros de la failia real! \bibverse{2} El juicio es
de ustedes\footnote{\textbf{5:2} La frase es literalmente ``para ti el
  juicio'' y por eso es ambigua. Simplemente podría significar que el
  juicio de Dios está en contra de estos líderes; pero también podría
  significar que el poder del juicio les pertenece y que no han ejercido
  esta autoridad sabiamente.} porque ustedes han sido como una trampa
tendida en Mizpa, y como una red lanzada en Tabor.\footnote{\textbf{5:2}
  Tanto Mizpa como el monte Tabor tuvieron un significado histórico
  particular para Israel, pero ahora son sitios de degradación.} Ustedes
cavaron una trampa profunda en Sitín,\footnote{\textbf{5:2} Sitín fue el
  último lugar donde acamparon los israelitas antes de cruzar el Jordán
  (Números 25).} pero yo los castigaré a ustedes por todo esto que han
hecho. \bibverse{3} Yo conozco muy bien a Efraín, e Israel no puede
esconderse de mi, porque ahora Efraín es una prostituta e Israel se ha
contaminado. \bibverse{4} Sus acciones les impiden volver a Dios porque
en ustedes hay un espíritu de prostitución y ya no conocen al Señor.
\bibverse{5} El orgullo del pueblo de Israel habla contra ellos en su
propia cara. Israel y Efraín tropezarán por su culpabilidad; y Judá
tropezará junto con ellos también. \bibverse{6} Ellos buscarán al Señor
con sus manadas y rebaños,\footnote{\textbf{5:6} La mención de manadas y
  rebaños indica que la gente estaba usando muchos sacrificios y
  ofrendas, pensando que Dios estaría complacido. Sin embargo, la suya
  no es una verdadera adoración, sino más bien una adoración pagana que
  intenta apaciguar a la deidad.} pero no lo encontrarán, porque él ya
se ha olvidado de ellos. \bibverse{7} Ellos han sido infieles al Señor y
han tenido hijos que no son de él.\footnote{\textbf{5:7} La palabra
  usada aquí respecto a los hijos es que son ``extranjeros'', lo que
  significa que son ilegítimos y también descendientes de dioses
  ``extranjeros''.} Ahora la luna nueva los destruirá junto con sus
campos.\footnote{\textbf{5:7} Se han dado varias explicaciones de esta
  oración. La celebración de los festivales de la luna nueva era parte
  del culto israelita, pero se había corrompido (véase, por ejemplo,
  Isaías 1:13), por lo que ahora podría tomarse como un símbolo del
  culto pagano. Además, el reino del norte bajo Jeroboam había
  instituido diferentes festivales que no fueron ordenados por Dios (ver
  1 Reyes 12:33). El punto principal es la influencia corruptora de las
  creencias paganas en la adoración genuina del Dios verdadero.}

\bibverse{8} ¡Hagan sonar la trompeta en Guibeá! ¡Toquen la tompreta en
Ramá! ¡Que suenen las alarmas en Bet-Aven! ¡Benjamín, ve al
frente!\footnote{\textbf{5:8} Los tres lugares mencionados están en la
  frontera norte entre Judá e Israel, en el territorio de la tribu
  sureña de Benjamín.} \bibverse{9} Efraín quedará desolado en el día
del castigo. Yo revelaré la verdad entre las tribus de Israel.
\bibverse{10} Los gobernantes de Judá se han convertido en ladrones que
mueven las fronteras de forma ilegal. Derramaré mi enojo como agua sobre
ellos.\footnote{\textbf{5:10} Véase una imagen similar en Isaías 8: 5-10
  que describe el fin del reino del norte a manos de los asirios.}
\bibverse{11} El pueblo de Efraín está aplastado y hecho pedazos a causa
del juicio, porque ellos decidieron seguir leyes humanas.\footnote{\textbf{5:11}
  Este versículo se ha vinculado con la decisión del rey Menahem sobre
  la decisión de Israel de aceptar un gran pago en plata al rey asirio
  como un medio para evitar conflictos (ver 2 Reyes 15:19, 20). Otros
  han pensado que los ``mandamientos humanos'' son la institución de
  Jeroboam de usar terneros como imágenes para adorar (1 Reyes 12).
  Alternativamente, el final de este versículo también podría traducirse
  ``determinados a seguir la idolatría''.} \bibverse{12} Yo soy como un
gusano en Efraín, y como un carcoma para el pueblo de Judá.

\bibverse{13} Cuando Efraín vio cuán enfermo estaba, y Judá notó sus
propias heridas, Efraín se volvió al gran rey de Asiria para pedir su
ayuda; pero él no los pudo sanar ni curar sus heridas. \bibverse{14} Yo
seré como un león con Efraín, y como un feroz león con el pueblo de
Judá. Vendré y los desmenuzaré, los llevaré donde nadie podrá ir a
rescatarlos. \bibverse{15} Entonces me iré y volveré de donde vine,
hasta que reconozcan sus faltas, y en su desesperación busquen mi rostro
y supliquen mi ayuda.''

\hypertarget{section-5}{%
\section{6}\label{section-5}}

\bibverse{1} ``¡Vamos! Volvamos al Señor. Él nos ha hecho pedazos, pero
ahora nos sanará; nos ha derribado, pero pondrá vendas en nuestras
heridas. \bibverse{2} En dos días nos sanará, y después de tres días nos
levantará para que podamos vivir en su presencia. \bibverse{3}
Conozcamos al Señor; procuremos conocerlo y él se aparecerá frente a
nosotros como el sol brillante. Él vendrá a nosotros tan ciertamente
como la lluvia de la primavera que riega la tierra. \bibverse{4} ¿Qué
hare con Efraín,\footnote{\textbf{6:4} Efraín representaba al rey del
  norte de Israel, y Judá representaba al sur.} y Judá? El amor que me
profesan desaparece como la niebla al amanecer, y se desvanece como el
rocío de la mañana. \bibverse{5} Por eso los he reducido a través de los
profetas y los destruí con mis palabras. Mi juicio resplandece como la
luz. \bibverse{6} Quiero que me ofrezcan amor verdadero y no
sacrificios; quiero que me conozcan, y no que me traigan holocaustos.

\bibverse{7} Pero ustedes, como Adán, quebrantaron nuestro
acuerdo,\footnote{\textbf{6:7} Literalmente, ``pacto.''} y me fueron
infieles. \bibverse{8} Gilead es una ciudad de gente malvada, donde se
pueden rastrear las huellas de sangre. \bibverse{9} Los sacerdotes son
como una cuadrilla de bandidos, esperando a un lado del camino a que
pasen los viajeros para tenderles una emboscada. Cometen asesinatos en
Siquem, y cometen grandes crímenes. \bibverse{10} He visto algo
aborrecible en la casa de Israel: Efraín se ha prostituido e Israel se
ha corrompido sexualmente.\footnote{\textbf{6:10} Las imágenes de
  prostitución e inmoralidad sexual se usan para describir el adulterio
  espiritual de Israel al seguir a otros dioses.} \bibverse{11} Y en lo
que tiene que ver contigo, Judá, ha llegado tu tiempo de cosechar lo que
has sembrado. Cuando restaure la fortuna de mi pueblo,\footnote{\textbf{6:11}
  Esta última frase se toma mejor con el comienzo del siguiente
  capítulo.}

\hypertarget{section-6}{%
\section{7}\label{section-6}}

\bibverse{1} cuando sane a Israel, entonces quedará expuesto el pecado
de Efraín, así como los actos malvados de Samaria. Ellos practican la
mentira y son como ladrones que entran a robar a las casas y asaltan a
la gente en las calles. \bibverse{2} Pero no se dan cuenta de que yo me
acuerdo de toda su maldad. Sus pecados los rodean y siempre están
delante de mi. \bibverse{3} Alegran a su rey con su maldad, y a los
príncipes con mentiras. \bibverse{4} Todos son adúlteros, y arden de
lujuria como un horno cuyo fuego no se apaga, aunque no lo atice el
panadero; son como la masa ha preparado y que crece después de que se
fermenta.\footnote{\textbf{7:4} Esta es la imagen de una fogata lista
  para cocinar el pan. La masa que crece se compara con el tiempo de
  Israel antes de su fin, cuando fue llevado a la cautividad. También
  parece que es una referencia al rey que no hace nada por controlar el
  fuego de la apostasía.} \bibverse{5} En el cumpleaños del
rey\footnote{\textbf{7:5} Literalmente, ``el día del rey.''} los
príncipes beben hasta enfermar, mientras él se junta con los que se
ríen. \bibverse{6} Sus corazones están encendidos con fuego como un
horno, y van a él con sus conspiraciones. Su enojo arde toda la noche y
en la mañana es una llamarada sin control. \bibverse{7} Todos arden como
un horno ardiente y devoran a sus líderes. Todos sus reyes han caído, y
ninguno me invoca.

\bibverse{8} Efraín se mezcla con otras naciones. ¡Es tan inútil como el
pan cocido a medias!\footnote{\textbf{7:8} Literalmente, ``un pan plano
  que solo se ha cocido de un lado.''} \bibverse{9} Los extranjeros le
quitan toda su fuerza y él ni siquiera se da cuenta. Su cabello se
vuelve gris y no lo nota. \bibverse{10} El orgullo de Israel testifica
contra él, pero a pesar de todo esto Efraín no vuelve al Señor su Dios
ni lo busca. \bibverse{11} Efraín es como una paloma, ingenua y sin
razón, que clama a Egipto y luego va a Asiria. \bibverse{12} Cuando se
vaya, lanzaré mi red sobre ellos y los atraparé como aves silvestres.
Cuando los oiga volar en bandada, los castigaré.

\bibverse{13} ¡Grande es el desastre que viene sobre ellos por haberse
alejado de mi! ¡Serán destruidos por haberse rebelado contra mi! Yo
quisiera poder redimirlos, pero ellos me calumnian. \bibverse{14} No
claman a mi con sinceridad en sus corazones, sino que mienten mientras
se lamentan en sus camas. Se reúnen y se laceran\footnote{\textbf{7:14}
  Una práctica del culto pagano. Ver 1 Reyes 18:28.} para obtener grano
y vino nuevo, pero se alejan de mi. \bibverse{15} Yo mismo los entrené y
los hice fuertes; y ahora conspiran contra mi. \bibverse{16} Se vuelven,
pero no al Altísimo. Son como un arco defectuoso. Sus líderes morirán a
espada por causa de sus maledicencias.\footnote{\textbf{7:16} A menudo
  se refiere a decir maldiciones contra Dios.} Por eso serán
ridiculizados en Egipto.

\hypertarget{section-7}{%
\section{8}\label{section-7}}

\bibverse{1} ¡Pon trompeta en tus labios! Un águila\footnote{\textbf{8:1}
  Símbolo de un enemigo que viene a invadir.} se precipita sobre la casa
del Señor porque han quebrantado mi acuerdo y se han rebelado contra mi
ley. \bibverse{2} Israel invoca a mi: `¡Nuestro Dios, te conocemos!'
\bibverse{3} Pero Israel ha rechazado lo que es bueno. Un enemigo los
perseguirá. \bibverse{4} Nombraron reyes sin mi aprobación y eligieron
príncipes sin hacérmelo saber. Elaboraron ídolos con su oro y su plata
para su propia destrucción. \bibverse{5} ¡Samaria, yo aborrezco el ídolo
con figura de becerro que has hecho! ¡Mi ira arde contra ellos! ¿Hasta
cuándo serán incapaces de ser buenos? \bibverse{6} Este ídolo surgió de
Israel y fue hecho por un artesano. ¡No es Dios! ¡El becerro de Samaria
quedará hecho pedazos!

\bibverse{7} Los que siembran viento, cosecharán tempestad. Los tallos
no tienen granos, y no producirán harina. Incluso si produjera grano,
los extranjeros lo devorarían. \bibverse{8} Israel ha sido devorado.
Entre las naciones son como cosa despreciable. \bibverse{9} Han acudido
a Asiria como un asno errante y solitario. Efarín ha contratado amantes.
\bibverse{10} Aunque han contratado aliados entre las naciones, yo las
reuniré\footnote{\textbf{8:10} Puede referirse a los israelitas o a las
  naciones. Aunque Israel trató de contratar aliados, con el tiempo
  estos se volvieron contra Israel.}. Etonces se retorcerán bajo el
agobio del gran rey.\footnote{\textbf{8:10} Refiriéndose al agobio de
  los impuestos que debían pagar a los invasores extranjeros, en
  especial a los asirios.} \bibverse{11} ¡Aún cuando Efarín construyó
altares para presentar ofrendas por el pecado, se volvieron altares de
pecado! \bibverse{12} Yo les escribe muchos aspectos de mi ley, pero la
consideraron como si fueran extranjeros.\footnote{\textbf{8:12} En otras
  palabras, consideraban que la ley no se aplicaba a ellos.}
\bibverse{13} Vienen a presentarme sus sacrificios y se comen la carne,
pero yo, el Señor, no los acepto. Ahora él recordará su maldad y los
castigará por sus pecados. Ellos volverán a Egipto. \bibverse{14} Israel
se ha olvidado de su Hacedor y ha construido palacios. Judá ha
construido ciudades fortificdas. Pero yo haré caer fuego sobre sus
ciudades y consumiré sus castillos.

\hypertarget{section-8}{%
\section{9}\label{section-8}}

\bibverse{1} ¡No te alegres, Israel! ¡No celebres como las demás
naciones! Porque te has prostituido y has recibido salario de prostitute
en cada era donde se trilla el trigo.\footnote{\textbf{9:1} La era donde
  se procesaba el grano después de la cosecha era el lugar donde se
  adoraba a los dioses de la fertilidad por una buena cosecha.}
\bibverse{2} Tus eras y lagares no te alimentarán. La tierra no podrá
producir el nuevo vino. \bibverse{3} No permanecerás en la tierra del
Señor; en cambio Efraín volverá a Egipto y comerá alimentos inmundos en
Asiria. \bibverse{4} Tú no podrás traer inguna ofrenda de vino al Señor.
Ninguno de tus sacrificios le complacerá. Tus sacrificios serán como la
comida de quien guarda luto:\footnote{\textbf{9:4} Una persona que
  guardaba luto probablemente había tocado un cuerpo muerto, por lo que
  estaban impuros (ver Levítico 21:11; Números 19:11, etc.)} todos los
que comen estarán impuros. Ustedes mismos comerán esta comida, pero no
entrarán en la casa del Señor. \bibverse{5} ¿Qué herán en los días de
fiestas religiosas, y de otras festividades del Señor?\footnote{\textbf{9:5}
  Esto puede referirse específicamente al día especial instituido por
  Jeroboam I (1 Reyes 12:32) como una observancia religiosa sustituta en
  los lugares sagrados del norte, contrario a los verdaderos días dados
  por Dios y que eran celebrados en el reino del sur.} \bibverse{6}
Miren, se han ido por causa de la destrucción: Egito los reunirá y
Menfis los sepultará. Gracias a su plata\footnote{\textbf{9:6}
  Evidentemente hay cierto grado de sarcasmo aquí: Al huir a Egipto, lo
  único que recibieron como recompensa fue ruina y muerte.} tienen un
`tesoro valioso'. La maleza los poseerá, y los espinos crecerán sobre
sus tiendas.

\bibverse{7} ¡Ha llegado la hora del castigo! ¡El día de la paga ha
llegado! ¡Que lo sepa Israel! Ustedes dicen\footnote{\textbf{9:7}
  Implícito. Claramente esta es la visión del pueblo.} que el profeta es
un tonto, que el hombre del Espíritu ha enloquecido, porque su maldad y
hostilidad es grande. \bibverse{8} El Centinela de Efraín está con mi
Dios. El profeta es como una trampa de aves en los caminos.\footnote{\textbf{9:8}
  Algunos han visto esto como la obra de un falso profeta, pero teniendo
  en cuenta el pecado de Israel, un verdadero profeta habría sido
  considerado como uno que tendía trampas contra el pueblo, siguiendo el
  razonamiento del versículo anterior.} Hay odio en la casa de su Dios
\bibverse{9} porque se han corrompido hasta el límite, como en los
tiempos de Guibeá.\footnote{\textbf{9:9} Esto hace alusión a la
  violación y asesinato de la concubina levita en Jueces 19-21.} Él se
acordará de su pecado, y castigará su maldad.

\bibverse{10} Como encontrar uvas en el desierto, así fue como encontré
a Israel. Cuando vi a sus antepasados, fue como ver los primeros frutos
de la higuera. Pero cuando fueron a Baal Peor, se entregaron a ese ídolo
abominable, y se volvieron tan inmundos como lo que aman.\footnote{\textbf{9:10}
  Esto se refiere al incidente que se registra en Números 25 cuando el
  pueblo de Israel se dejó seducir por las mujeres de Moab para
  practicar ritos de inmoralidad sexual como adoración a su dios.}
\bibverse{11} ¡Escucha Efraín! Así como el ave que escapa y vuela lejos,
así será tu gloria: no habrá nacimientos, embarazos ni concepciones.
\bibverse{12} Incluso si lograsen concebir hijos, yo me encargaré de que
no sobrevivan. ¡Grande es el desastre que vendrá sobre ti cuando yo me
aleje! \bibverse{13} ¡Escucha Efraín! Así como vi a Tiro plantado en una
pradera, del mismo modo Efraín entregará a sus hijos al
asesino.\footnote{\textbf{9:13} Tanto Israel como Tiro practicaban el
  sacrificio de niños. Ambas ciudades fueron conquistadas por los
  asirios en el año 722 AC.} \bibverse{14} Dales, Señor\ldots{} ¿Qué
podrás darles? Dales vientres que aborten y senos secos.\footnote{\textbf{9:14}
  Debido a que la adoración a Baal estaba enfocada en la fertilidad, la
  infertilidad era un castigo contundente, contrario a lo que los dioses
  de la fertilidad afirmaban.} \bibverse{15} Toda su maldad comenzó en
Guilgal, y desde entonces comencé a aborrecerlos. Los expulsaré de mi
casa por su maldad. No los amaré más, porque todos sus líderes son
rebeldes. \bibverse{16} Efraín, estás en ruinas, seco desde la raíz. No
darás ningún fruto. Incluso si tienes hijos, yo destruiré a tu amada
descendencia con masacre. \bibverse{17} Mi Dios te rechazará porque no
lo has escuchado, y serás un pueblo de vagabundos sin hogar entre las
naciones.

\hypertarget{section-9}{%
\section{10}\label{section-9}}

\bibverse{1} Israel es como una viña frondosa\footnote{\textbf{10:1} La
  palabra que se usa aquí generalmente significa ``tirar basura''. Si
  bien muchas traducciones usan términos como ``exuberante'', el punto
  que se hace aquí es que esta vid no está podada y está demasiado
  cubierta, y además solo produce fruto ``por sí misma'', lo que no es
  una buena descripción de una vid productiva para el jardinero.} y que
produce fruto por sí sola. Cuanto más fruto producía, tantos más altares
construía\footnote{\textbf{10:1} Altares usados para el culto pagano.}.
Cuando más productiva era la tierra, tanto más hermosos eran los
pilares\footnote{\textbf{10:1} Una vez más, símbolos paganos usados para
  adorar a los dioses de la fertilidad.} sagrados que hacían.
\bibverse{2} El pueblo tiene corazones engañosos, y ahora deben asumir
la responsabilidad de su culpa. El Señor romperá sus altares y destruirá
sus pilares sagrados. \bibverse{3} Entonces dirán: `No tenemos rey,
porque no tememos al Señor, ¿y acaso qué hará un rey por nosotros?'
\bibverse{4} Hablan con palabras vacías, juran y hacen falsas promesas
para lograr un pacto.\footnote{\textbf{10:4} El contexto se refiere a
  hacer falsas promesas a Dios o acuerdos que no piensan cumplir.} Su
`justicia' florece como hierba venenosa en el campo. \bibverse{5} Los
que viven en Samaria tiemblan asombrados ante el becerro de
Bet-Aven.\footnote{\textbf{10:5} El verdadero nombre del lugar era
  Bethel, casa de Dios, pero debido a las prácticas paganas celebradas
  allí, los profetas posteriores se refirieron a ella como Bet-Aven, que
  significa casa de la nada (ídolos).} Su pueblo se lamenta por ello en
sus rituales paganos, mientras sus sacerdotes idólatras celebran su
gloria. Pero tal gloria les será quitada.\footnote{\textbf{10:5} Los
  verbos utilizados en este versículo describen la adoración pagana de
  Baal, en la que su muerte se lamenta con la automutilación y luego su
  regreso se celebra con ritos orgiásticos. Sin embargo, eventualmente
  los asirios se llevarían el ídolo durante la invasión.} Esa gloria se
le dará a Asiria como tributo por el gran rey. \bibverse{6} Efraín
sufrirá desgracia, e Israel será avergonzado por sus propias
decisiones.\footnote{\textbf{10:6} Decisiones: De confiar en ídolos, y
  buscar ayuda de las naciones paganas.} \bibverse{7} Samaria y su rey
serán arrastrados como una pequeña rama en la superficie del agua.
\bibverse{8} Los altares de Aven,\footnote{\textbf{10:8} Donde estaban
  ubicados los lugares sagrados paganos.} donde Israel pecó, serán
demolidos, y crecerán cardos con espinas sobre sus altares. Entonces
clamarán a las montañas y a las colinas: `¡Caigan sobre nosotros!'

\bibverse{9} Desde los días de Guibeá, oh Israel, has estado pecando y
no has cambiado. ¿Acaso el pueblo de Guibeá cree que la guerra no los
alcanzará? \bibverse{10} Cuando yo lo elija, castigaré a los malvados.
Las naciones se reunirán contra ellos cuando sean castigados por su
doble crimen.

\bibverse{11} Efraín es como una novilla adiestrada, a quien le gustaba
trillar el grano, pero yo le pondré un yugo en su cuello. Le pondré un
arnés a Efraín, y Judá tendrá que surcar del arado; y Jacob debe romper
la tierra por si mismo. \bibverse{12} Siembren ustedes mismos lo bueno y
cosecharán amor incondicional. Rompan la tierra sin arar. Es hora de ir
al Señor hasta que venga y haga llover bondad sobre ustedes.
\bibverse{13} Pero por el contrario han sembrado maldad y han cosechado
maldad. Han comido el fruto de las mentiras, porque confiaron en su
propia fuerza y en sus muchos guerreros. \bibverse{14} El ruido terrible
de la batalla se levantará contra su pueblo, y sus castillos serán
destruidos, así como Salmán azotó a Bet-Arbel en tiempos de guerra.
Hasta las madres junto a sus hijos fueron estrellados contra el suelo
hasta quedar en pedazos. \bibverse{15} Esto mismo te pasará a ti, Betel,
por tu gran maldad. Al amanecer, el rey de Israel será destruido por
completo.

\hypertarget{section-10}{%
\section{11}\label{section-10}}

\bibverse{1} Yo amé a Israel desde que era un niño. Es mi hijo a quien
saqué de Egipto. \bibverse{2} Así como los llamaban, iban :\footnote{\textbf{11:2}
  A veces esto se traduce: ``Cuanto más los llamaba, más se alejaban de
  mí'', pero esto requiere cambios significativos en el texto original.
  Lo que parece decir el texto hebreo, en el contexto del éxodo de
  Egipto, es que como ellos (Israel) los llamaban (Egipto) del mismo
  modo ellos (Israel) se alejaron de ellos (Egipto). En otras palabras,
  incluso en el Éxodo, Israel anhelaba las cosas de Egipto y solo se
  fueron bajo presión. Muchos habrían preferido quedarse, y Oseas
  compara la apostasía con la que está lidiando con el espíritu reacio y
  rebelde de algunos, incluso en el momento del Éxodo. Esto se confirma
  en la segunda parte del versículo.} presentaron sacrificios a Baal y
ofrecieron incienso a los ídolos. \bibverse{3} Yo mismo enseñé a Efraín
a caminar, llevándolo de la mano,\footnote{\textbf{11:3} Literalmente,
  ``brazo.''} pero no reconocieron que yo era su sanador.\footnote{\textbf{11:3}
  En el contexto del Éxodo. Ver Éxodo 15:26.} \bibverse{4} Los conduje
con cuerdas de bondad, con lazos de amor. Yo era quien aliviaba su carga
y me agachaba para alimentarlos.\footnote{\textbf{11:4} La imagen da un
  giro que indica el cuidado de un animal de granja. El hebreo dice
  literalmente: ``Me volví como aquellos que levantan un yugo que estaba
  en sus mandíbulas''. La carga no se elimina, pero se hace más fácil de
  soportar.} \bibverse{5} Sin embargo, como mi pueblo se niega a
regresar a mi, tampoco volverán a la tierra de Egipto\footnote{\textbf{11:5}
  Aunque no son llevados cautivos a Egipto, son llevados en cautiverio,
  esta vez a Asiria.} sino que Asiria será su rey. \bibverse{6}
Habráguerra\footnote{\textbf{11:6} Literalmente, ``la espada.''} en sus
ciudades que porndrá fin a su jactancia y destruirá sus planes.
\bibverse{7} Mi pueblo insiste en su apostasía. Lo llaman ``dios en lo
alto''\footnote{\textbf{11:7} Israel llamó a su ídolo ``El Al,'' o
  ``dios en lo alto,'' pero esto era para causar confusión deliberada
  con un título que mezclara la adoración a Yahweh y Baal.}pero él no
los levnatará.

\bibverse{8} ¿Cómo podría abandonarte, Efraín? ¿Cómo podría dejarte ir,
Israel? ¿Cómo podría hacer contigo lo mismo que con Adamá? ¿Cómo podría
tratarte como a Seboín?\footnote{\textbf{11:8} Adamá y Seboín eran las
  ciudades gemelas de Sodoma y Gomorra (véase Génesis 14: 2).} Mi
corazón se hace pedazos, y reboso en compasión. \bibverse{9} No cederé
al ardor de mi ira, no destruiré a Efraín de nuevo. Porque yo soy Dios,
y no un ser humano. Yo soy el Santo que vive entre ustedes. No entraré
en sus ciudades.\footnote{\textbf{11:9} Significa que Dios no los
  aniquilaría totalmente como lo hizo con las ciudades mencionadas
  anteriormente.}

\bibverse{10} El pueblo me segirá a mi, al Señor. El Señor rugirá como
león y entonces sus hijos vendrán temblando desde el oeste.
\bibverse{11} Como una bandada de aves, vendrán desde Egipto. Como
palomas vendrán de Asiria, y yo los traeré de nuevo a casa, dice el
Señor.

\bibverse{12} Efraín me rodea con sus mentiras e Israel con engaño. Judá
aún anda errante con algún dios, fiel a algún ``Santo.''\footnote{\textbf{11:12}
  Parece que Judá estaba fusionando conceptos de la adoración pagana con
  la adoración del Dios verdadero, y usando el término ``el'', que era
  el nombre del dios cananeo más alto, pero que también podía aplicarse
  a Yahweh. Entonces, lo que se dice aquí parece ser que Judá también
  está vacilando en su lealtad al Dios verdadero.}

\hypertarget{section-11}{%
\section{12}\label{section-11}}

\bibverse{1} Efraín trata de guiar al viento, yendo tras el viento del
este todo el día. Sus mentiras y su violencia siguen en aumento.
Hicieron un tratado con Asiria, y envían el aceite de oliva a
Egipto.\footnote{\textbf{12:1} En otras palabras, al hacer un tratado
  con Asiria, están cubriendo sus opciones al tratar de obtener el apoyo
  de Egipto y enviando suministros de aceite de oliva.} \bibverse{2} El
Señor tiene además una acusación en contra de Judá, y castigará a Jacob
por las acciones del pueblo. Les pagará conforme a lo que han hecho.
\bibverse{3} Incluso desde el vientre Jacob luchó con su
hermano;\footnote{\textbf{12:3} Literalmente, ``él agarró el talón de su
  hermano.''} y cuando se hizo hombre luchó con Dios. \bibverse{4} Peleó
con un ángel y ganó. Lloró y le rogó por una bendición. Jacob encontró a
Dios en Betel, y allí habló con él, \bibverse{5} el Señor Dios
Todopoderoso, ¡el Señor es su gran nombre! \bibverse{6} Ustedes deben
volver a su Dios. Actúen con amor y hagan lo recto, y siempre esperen en
Dios.

\bibverse{7} El mercader que sostiene en sus manos una balanza alterada
ama la estafa. \bibverse{8} Efraín dice: `Soy rico! ¡Me he enriquecido!
He recibido mucho fruto de mi trabajo y nadie puede demostrar que soy
culpable de ningún mal.'

\bibverse{9} Pero yo soy el Señor tu Dios, que te saqué de la tierra de
Egipto. Yo te hare vivir de nuevo en tiendas como lo haces en tiempos de
fiesta.\footnote{\textbf{12:9} Esto se refiere al Festival de los
  Tabernáculos donde la gente habitaba afuera, en tiendas de campaña o
  refugios durante una semana, para recordar su viaje por el desierto}

\bibverse{10} Yo hablé a través de los profetas. Yo mismo di muchas
visiones y parábolas a través de los profetas.

\bibverse{11} Si Galaad es idólatra,\footnote{\textbf{12:11} La palabra
  utilizada aquí es la misma que en Bet-Aven, el nombre sustituto de
  Betel, que significa la adoración de ídolos que realmente no valen
  nada, no son nada.} sin duda se volverán nada. Sacrifican toros en
Guilgal. Incluso sus altares son como pilas de rocas en los surcos del
campo.\footnote{\textbf{12:11} Este versículo sugiere adoración mixta
  que mezclaba ídolos con el Dios verdadero. Como rocas en un campo
  arado, este era un obstáculo para conocer al Dios verdadero.}

\bibverse{12} Jacob huyó a la tierra de Aram; Israel\footnote{\textbf{12:12}
  Israel fue el nombre nuevo que Dios le dio a Jacob.} trabajó allí para
ganarse una esposa. Se ganó una esposa cuidando ovejas. \bibverse{13} A
través de un profeta\footnote{\textbf{12:13} El profeta al que hace
  referencia aquí es Moisés.} el Señor sacó a Israel de Egipto, y fueron
cuidados por un profeta también.\footnote{\textbf{12:13} Se usa la misma
  para hablar de Jacob cuidando ovejas y del Señor cuidando a Israel por
  medio de Moisés.}

\bibverse{14} Efraín ha hecho enojar al Señor, y el Señor los dejará
sufrir las consecuencias de su derramamiento de sangre, y les pagará por
su desprecio.

\hypertarget{section-12}{%
\section{13}\label{section-12}}

\bibverse{1} Cuando habló Efraín, se asustaron porque eran la tribu
líder en Israel. Pero cuando fueron culpables de adoración a Baal,
murieron. \bibverse{2} Ahora pecan constantemente, y se forjan ídolos de
metal fundido. Todos esos ídolos fueron hábilmente forjados con plata
por los artesanos. `Ofrezcan sacrificios a estos ídolos,' dice el
pueblo. `Besen al ídolo con forma de becerro.' \bibverse{3} Por ello,
serán como la niebla de la mañana, como el rocío de la madrugada, como
la paja de la era que se lleva el viento, como el humo de una chimenea.

\bibverse{4} Pero yo soy el Señor que te sacó de la tierra de Egipto. No
conocerás\footnote{\textbf{13:4} Una variación de los Diez Mandamientos
  donde ``conocer'' que reemplaza ``tener'' (Éxodo 20: 3). Este cambio
  es significativo en el sentido de que la palabra ``conocer'' conlleva
  matices de relaciones íntimas y puede vincularse con la naturaleza
  erótica del culto a Baal. Israel no debería estar ``conociendo'' a
  Baal sino el Dios verdadero.} a otros dioses, sino solo a mi. Nadie
puede salvarte si no yo. \bibverse{5} Te cuidé en el desierto. Allí la
tierra fue como pasto para ellos\footnote{\textbf{13:5} ``Fue como pasto
  para ellos'' o ``Yo los alimenté.''} \bibverse{6} y se saciaban. Pero
cuando quedaron saciados, se volvieron arrogantes y se olvidaron de
mi.\footnote{\textbf{13:6} Véase la advertencia de que esto podría pasar
  en Deuteronomio 8:11-14.} \bibverse{7} Así que yo seré para ellos como
un león, como un leopardo acecharé junto al camino. \bibverse{8} Seré
como la madre oso a quien le han robado sus crías, y desgarraré sus
entrañas. Yo los devoraré como un león, como una bestia salvaje los
destrozaré. \bibverse{9} Se han destruido ustedes mismos, oh Israel,
porque tu única esperanza está en mi. \bibverse{10} ¿Dónde está tu rey?
¡Que venga y salve todas tus ciudades! ¿Dónde están tus líderes que me
exigían un rey y un príncipe? \bibverse{11} En mi ira te di un rey, y en
mi furia te lo quitaré.\footnote{\textbf{13:11} Las formas verbales
  utilizadas aquí no están en tiempo pasado, por lo que la traducción
  habitual que se refiere a la provisión de Dios del rey Saúl en
  respuesta a las demandas del pueblo es confusa. Como Oseas está
  lidiando con la situación actual del reino del norte, una
  interpretación es que el rey que Dios está proveyendo es el rey de
  Asiria, y el que está tomando es el rey de Israel. Aunque en el
  versículo anterior, Dios menciona la demanda previa de un rey por
  parte del pueblo, comienza ese versículo con la pregunta actual:
  ``¿Dónde está entonces tu rey?''}

\bibverse{12} La culpa de Efraín ha sido anotada, y su pecado será
erradicado.\footnote{\textbf{13:12} ``Erradicado:'': Literalmente,
  ``ocultado.'' Lo que Oseas parece estar diciendo aquí es que los
  pecados de Israel han llegado al punto de que Dios tiene que tomar
  medidas al eliminar el problema permitiendo la invasión y el exilio.
  El culto a Baal tiene que terminar (``desaparecer'', pero no en el
  sentido de que simplemente quede oculto y continúe).} \bibverse{13}
Sufren dolor de parto, tratando de dar a luz un hijo que no es `sabio'
porque no estará en la posición correcta cuando llegue su
tiempo.\footnote{\textbf{13:13} Esto podría referirse a los problemas de
  un parto de nalgas en el que tanto la madre como el bebé podrían
  morir.}

\bibverse{14} Yo los redimiré del poder del Seol. Los libraré de la
muerte. ¿Dónde, oh muerte, están tus plagas? ¿Dónde está, oh Seol, tu
destrucción? La compasión se ha ocultado de mis ojos.

\bibverse{15} Aunque prospere entre los juncos,\footnote{\textbf{13:15}
  O ``hermanos.''} un viento del este vendrá, un viento del Señor que se
origina en el desierto secará sus fuentes y sus pozos se romperán. Yo
robaré de su tesorería todo lo que tenga valor.

\bibverse{16} Samaria tendrá que acarrear las consecuencias de su culpa,
por haberse rebelado contra su Dios. Ellos serán destruidos con espada,
sus hijos serán estrellados contra el piso, y las mujeres embarazadas
quedarán desgarradas.

\hypertarget{section-13}{%
\section{14}\label{section-13}}

\bibverse{1} Vuelve, Israel, al Señor tu Dios, porque tus pecados te han
hecho caer. \bibverse{2} Toma estas palabras y vuélvete al Señor, y
dile: `Por favor, toma toda nuestra culpa, acepta lo bueno que hay, y
nosotros te pagaremos con alabanza en nuestros labios. \bibverse{3}
Asiria no puede salvarnos, y no escaparemos con nuestros caballos de
guerra, ni volveremos a decir: ``ustedes son nuestros dioses'' a los
ídolos que hemos hecho. Porque los huérfanos hallan misericordia en ti.'

\bibverse{4} Yo sanaré su falta de fe. Los amaré generosamente, porque
ya no estoy enojado con ellos. \bibverse{5} Yo seré como el rocío para
Israel, y ellos florecerán como los lirios, y sus raíces crecerán
fuertes como los cedros del Líbano. \bibverse{6} Sus retoños se
extenderán, y su esplendor será como el árbol de olivo, su fragancia
será como los cedros del Líbano. \bibverse{7} Los que habitan bajo su
sombra regresarán, y florecerán como el grano; florecerán como el vino,
y serán recordados como el vino del Líbano.

\bibverse{8} Efraín, ¿hasta cuándo tendré que advertirte sobre la
idolatría?\footnote{\textbf{14:8} En otras palabras, Dios ya se can
  cansado del tema.} Ya he contestado y ahora espero.\footnote{\textbf{14:8}
  Espera ver la respuesta del pueblo.} Soy como el árbol siempre verde,
y de mí nace tu fruto.

\bibverse{9} ¿Quién es sabio para entender todo esto? ¿Quién tiene el
discernimiento para entender? Los caminos del Señor son rectos, pero los
rebeldes se tropiezan y caen.
