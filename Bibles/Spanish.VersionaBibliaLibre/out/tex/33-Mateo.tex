\hypertarget{section}{%
\section{1}\label{section}}

\bibverse{1} Este libro es el registro de Jesús el Mesías,\footnote{\textbf{1:1}
  O ``Cristo.'' Cristo es el término griego para decir ``Mesías'' en
  hebreo.} Hijo de David, Hijo de Abraham, comenzando con el linaje de
su familia:

\bibverse{2} Abraham fue el padre\footnote{\textbf{1:2} O ``engendró.''}
de Isaac; e Isaac el padre de Jacob; y Jacob el padre de Judá y de sus
hermanos; \bibverse{3} y Judá fue el Padre de Fares y Zarah (su madre
fue Tamar); y Fares fue el padre de Esrom; y Esrom el padre de Ram;
\bibverse{4} y Ram fue el padre de Aminadab; y Aminadab el padre de
Nasón; y Nasón el padre de Salmón; \bibverse{5} y Salmón el padre de
Booz (su madre fue Rahab); y Booz el padre de Obed (su madre fue Rut); y
Obed el padre de Isaí; \bibverse{6} e Isaí el padre del Rey David. David
fue el padre de Salomón (su madre había sido la esposa de Urías);
\bibverse{7} y Salomón el padre de Roboam; y Roboam el padre de Abías; y
Abías el padre de Asa; \bibverse{8} y Asa fue el padre de Josafat; y
Josafat el padre de Yoram; y Yoram el padre de Uzías; \bibverse{9} y
Uzías fue el padre de Jotam; y Jotam el padre de Acaz; y Acaz el padre
de Ezequías; \bibverse{10} y Ezequías el padre de Manasés; y Manasés el
padre de Amón; y Amón el padre de Josías; \bibverse{11} y Josías el
padre de Joaquín y de sus hermanos, durante el tiempo del exilio a
Babilonia. \bibverse{12} Después del exilio a Babilonia, Joacím fue el
padre de Salatiel; y Salatiel el padre de Zorobabel; \bibverse{13} y
Zorobabel el padre de Abiud; y Abiud fue el padre de Eliaquim; y
Eliaquim el padre de Azor; \bibverse{14} y Azor el padre de Sadoc; y
Sadoc el padre de Aquim; y Aquim el padre de Eliud; \bibverse{15} y
Eliud fue el padre de Eleazar; y Eleazar el padre de Matán; y Matán el
padre de Jacob;

\bibverse{16} y Jacob fue el padre de José, quien fue el esposo de
María, de quien nació Jesus, el que es llamado el Mesías.

\bibverse{17} Así que todas las generaciones desde Abraham hasta David
suman catorce; desde David hasta el exilio de babilonia, catorce; y
desde el exilio de Babilonia hasta el Mesías, catorce.

\bibverse{18} Así fue como ocurrió el nacimiento de Jesús el Mesías: su
madre, María, estaba comprometida con José, pero antes de que durmieran
juntos ella quedó embarazada por obra del Espíritu Santo. \bibverse{19}
José, su prometido, era un buen hombre y no quería avergonzarla
públicamente, de modo que decidió romper el compromiso de manera
discreta.

\bibverse{20} Mientras José pensaba en todo esto, un ángel del Señor se
le apareció en un sueño y le dijo: ``José, hijo de David, no temas
casarte con María porque ella está embarazada por obra del Espíritu
Santo. \bibverse{21} Ella tendrá un hijo y tú le llamarás Jesús, porque
él salvará a su pueblo de sus pecados.'' \bibverse{22} Y todo esto
ocurrió para cumplir lo que el Señor dijo a través del profeta:
\bibverse{23} `Una virgen quedará embarazada y tendrá un hijo. Y le
llamarán Emanuel,' que significa `Dios con nosotros.'''\footnote{\textbf{1:23}
  Isaías 7:14.} \bibverse{24} José se despertó e hizo lo que el ángel
del Señor le dijo que hiciera. \bibverse{25} José se casó con María,
pero no durmió con ella hasta después que tuvo un hijo, a quien llamó
Jesús.

\hypertarget{section-1}{%
\section{2}\label{section-1}}

\bibverse{1} Después que Jesús nació en Belén de Judea, durante el
reinado del rey Herodes, unos hombres sabios\footnote{\textbf{2:1} O
  ``Magos.'' Se creía que estos eran sacerdotes gobernantes de Persia,
  quienes estudiaban las estrellas.} vinieron desde el oriente hasta
Jerusalén.

\bibverse{2} ``¿Dónde está el Rey de los judíos que ha nacido?''
preguntaron. ``Vimos su estrella en el oriente y hemos venido a
adorarlo.''

\bibverse{3} Cuando el rey Herodes escuchó esto, se preocupó mucho, y
toda Jerusalén con él. \bibverse{4} Entonces Herodes llamó a todos los
jefes de los sacerdotes y a los maestros religiosos del pueblo, y les
preguntó dónde se suponía que nacería el Mesías.

\bibverse{5} ``En Belén de Judea,'' le dijeron ellos, ``pues eso fue lo
que escribió el profeta: \bibverse{6} `y tu, Belén, en la tierra de
Judea, no eres la menor entre las ciudades reinantes de
Judea\footnote{\textbf{2:6} ``Ciudades'' está implícito.}, porque de ti
saldrá un gobernante que será el pastor de mi pueblo Israel.'\,''

\bibverse{7} Entonces Herodes llamó a los sabios y se reunió con ellos
en privado, y logró saber por medio de ellos el momento exacto en que
había aparecido la estrella. \bibverse{8} Los envió a Belén,
diciéndoles: ``cuando lleguen allí, busquen al niño, y cuando lo
encuentren, háganmelo saber para yo ir a adorarlo también.''

\bibverse{9} Después que los sabios escucharon lo que el rey iba a
decirles, siguieron su camino, y la estrella que habían visto en el
oriente los guió hasta que se detuvo justo sobre el lugar donde estaba
el niño. \bibverse{10} Cuando los sabios vieron la estrella,\footnote{\textbf{2:10}
  Claramente indica que fue cuando vieron que la estrella se detuvo,
  puesto que ellos ya habían visto la estrella y la habían seguido
  durante todo el camino desde su hogar en el oriente.} no pudieron
contener la felicidad. \bibverse{11} Entraron a la casa y vieron al niño
con María, su madre. Se arrodillaron y lo adoraron. Entonces abrieron
sus bolsas de tesoros y le obsequiaron regalos de oro, incienso y mirra.
\bibverse{12} Advertidos por un sueño de no regresar ante Herodes, se
marcharon a su país tomando otro camino.

\bibverse{13} Después que se fueron los sabios, un ángel del Señor se le
apareció a José en un sueño, y le dijo: ``Levántate, toma al niño y a su
madre, y huyan a Egipto. Quédense allí hasta que yo se los diga, porque
Herodes está intentando buscar al niño para matarlo.''

\bibverse{14} Entonces José se levantó y tomó al niño y a su madre, y
partió hacia Egipto en medio de la noche. \bibverse{15} Permanecieron
allí hasta que Herodes murió. Esto cumplió lo que el Señor dijo a través
del profeta: ``De Egipto llamé a mi hijo.''\footnote{\textbf{2:15} Oseas
  11:1.}

\bibverse{16} Cuando Herodes se dio cuenta que había sido engañado por
los sabios, se enojó mucho. Entonces envió hombres para que matasen a
todos los niños de Belén y de las regiones cercanas, que tuvieran menos
de dos años de edad. Esto se basaba en el marco de tiempo que escuchó de
los sabios\footnote{\textbf{2:16} En otras palabras, hacía dos años que
  la estrella ya se les había aparecido previamente a los sabios.}.
\bibverse{17} Así se cumplió la profecía del profeta Jeremías:
\bibverse{18} ``En Ramá se oyó una voz, llanto y gran lamento. Raquel
llora por sus hijos, se niega a que la consuelen, porque están
muertos.''

\bibverse{19} Después que Herodes murió, el ángel del Señor se le
apareció en un sueño a José en Egipto, y le dijo: \bibverse{20}
``¡Levántate! Toma al niño y a su madre y regresa a la tierra de Israel,
porque los que trataban de matar al niño están muertos.''

\bibverse{21} Entonces José se levantó y tomó al niño y a su madre, y
regresó a la tierra de Israel. \bibverse{22} Pero José tenía miedo de ir
allá después que supo que Arquelao había sucedido a su padre, el rey
Herodes, como rey de Judá. Habiendo sido advertido por medio de un
sueño, José se fue a Galilea, \bibverse{23} y se estableció en Nazaret.
Esto cumplió lo que los profetas habían dicho: ``Él será llamado
Nazareno.''\footnote{\textbf{2:23} Refiriéndose a Jesús.}

\hypertarget{section-2}{%
\section{3}\label{section-2}}

\bibverse{1} Tiempo después, apareció en escena Juan el Bautista,
predicando en el desierto de Judea: \bibverse{2} ``Arrepiéntanse, porque
el reino de los cielos está cerca.'' \bibverse{3} Él era de quien
hablaba el profeta Isaías cuando dijo: ``Se oye una voz que clama en el
desierto: `preparen el camino del Señor. Enderecen la senda para
él.'\,''

\bibverse{4} Juan tenía ropas hechas con pelo de camello, con un
cinturón de cuero puesto en su cintura. Su alimento era
langostas\footnote{\textbf{3:4} Probablemente, algarrobas.} y miel
silvestre. \bibverse{5} La gente venía a él desde Jerusalén, de toda
Judea y de toda la región del Jordán, \bibverse{6} y eran bautizados en
el río Jordán, reconociendo públicamente sus pecados.

\bibverse{7} Pero cuando Juan vio que muchos de los Fariseos y Saduceos
venían a ser bautizados, les dijo: ``¡Camada de víboras! ¿Quién les
advirtió que huyeran del juicio que vendrá\footnote{\textbf{3:7}
  Literalmente, ``ira.''}? \bibverse{8} Muestren a través de sus actos
que están verdaderamente arrepentidos\footnote{\textbf{3:8}
  Literalmente, ``Produzcan fruto equivalente al arrepentimiento.''},
\bibverse{9} y no se jacten de decirse a ustedes mismos: `Abrahán es
nuestro padre.' Les digo que Dios podría convertir estas piedras en
hijos de Abrahán. \bibverse{10} De hecho, el hacha está lista para
derribar los árboles. Todo árbol que no produce buen fruto, será
derribado y lanzado al fuego.

\bibverse{11} ``Sí, yo los bautizo en agua para mostrar arrepentimiento,
pero después de mi viene uno que es más grande que yo. Yo no soy
siquiera digno de quitar sus sandalias. Él los bautizará con el Espíritu
Santo y con fuego. \bibverse{12} Él tiene el aventador\footnote{\textbf{3:12}
  Usada después de la cosecha para separar el trigo de la paja.} lista
en su mano. Limpiará la era y almacenará el trigo en el granero, pero
quemará la paja en el fuego que no puede apagarse.''

\bibverse{13} Luego Jesús vino desde Galilea hasta el Río Jordán para
ser bautizado por Juan. \bibverse{14} Pero Juan trató de hacerlo cambiar
de opinión, diciendo, ``Yo necesito ser bautizado por ti, ¿y tu vienes a
mí para que yo te bautice?''

\bibverse{15} ``Por favor, hazlo, porque es bueno que hagamos lo que
Dios dice que es correcto,'' le dijo Jesús. Entonces Juan estuvo de
acuerdo en hacerlo.

\bibverse{16} Justo después de haber sido bautizado, Jesús salió del
agua. Los cielos se abrieron y él vio al Espíritu de Dios como una
paloma que descendía, posándose sobre él. \bibverse{17} Entonces una voz
desde el cielo dijo: ``este es mi hijo a quien amo, el cual me
complace.''

\hypertarget{section-3}{%
\section{4}\label{section-3}}

\bibverse{1} Entonces Jesús fue guiado por el Espíritu hasta el desierto
para ser tentado por el diablo. \bibverse{2} Después de haber ayunado
por cuarenta días y cuarenta noches, tenía hambre. \bibverse{3} El
tentador vino y le dijo: ``Si realmente eres el hijo de Dios, ordena a
estas piedras que se conviertan en pan.''

\bibverse{4} Jesús respondió: ``Como dicen las Escrituras, 'los seres
humanos no viven solo de comer pan, sino de cada palabra que sale de la
boca de Dios.''

\bibverse{5} Entonces el diablo lo llevó hasta la ciudad
santa\footnote{\textbf{4:5} Refiriéndose a Jerusalén.} y lo puso en la
parte más alta del Templo.

\bibverse{6} ``Si realmente eres el hijo de Dios, tírate,'' le dijo a
Jesús. ``Tal como dicen las Escrituras: `Él mandará a sus ángeles para
que te guarden del peligro. Te atraparán para que no caigas al
tropezarte con una roca.'\,''

\bibverse{7} Jesús respondió: ``Tal como dicen también las Escrituras,
`No tentarás al Señor tu Dios.'\,''

\bibverse{8} Entonces el diablo llevó a Jesús a una montaña muy alta y
le mostró todos los reinos del mundo en toda su gloria. \bibverse{9} Le
dijo a Jesús: ``Te daré todos estos reinos si te arrodillas y me
adoras.''

\bibverse{10} ``¡Vete de aquí Satanás!'' dijo Jesús. ``Tal como dicen
las Escrituras: `Adorarás al Señor tu Dios y le servirás solo a Él.'\,''
\bibverse{11} Entonces el diablo lo dejó, y los ángeles vinieron a
cuidar de él.

\bibverse{12} Cuando Jesús escuchó que Juan había sido arrestado,
regresó a Galilea. \bibverse{13} Después de salir de Nazaret, se quedó
en Capernaúm, a orillas del mar, en las regiones de Zabulón y Neftalí.
\bibverse{14} Esto cumplió lo que el profeta Isaías dijo: \bibverse{15}
``En la tierra de Zabulón y en la tierra de Neftalí, camino al mar, más
allá del Jordán, en Galilea, donde viven los gentiles: \bibverse{16} La
gente que vive en la oscuridad vio una gran luz; la luz de la mañana ha
brillado sobre aquellos que viven en la tierra de la oscuridad y la
muerte.'' \bibverse{17} Desde ese momento, Jesús comenzó a declarar su
mensaje, diciendo: ``Arrepiéntanse, porque el reino de los cielos está
cerca.''

\bibverse{18} Mientras caminaba por el mar de Galilea, Jesús vio a dos
hermanos: Simón, también llamado Pedro, y su hermano Andrés, que estaban
lanzando una red en el mar. Ellos vivían de la pesca.

\bibverse{19} ``Vengan y síganme, y yo les enseñaré cómo pescar
personas,'' les dijo. \bibverse{20} Ellos dejaron sus redes de inmediato
y lo siguieron. \bibverse{21} De camino, vio nuevamente a otros dos
hermanos: Santiago y Juan. Ellos estaban en un bote con su padre
Zebedeo, reparando sus redes. Él los llamó para que lo
siguieran\footnote{\textbf{4:21} ``Para que lo siguieran,'' está
  implícito.}. \bibverse{22} Ellos inmediatamente dejaron el bote y a su
padre, y lo siguieron.

\bibverse{23} Jesús viajó por toda Galilea, enseñando en las sinagogas,
contando las buena nueva del reino, y sanando todas las enfermedades que
tenían las personas. \bibverse{24} Entonces comenzó a difundirse la
noticia acerca de él por toda la provincia de Siria\footnote{\textbf{4:24}
  El área del norte de Galilea.}. La gente traía delante de él a todos
los que estaban enfermos: personas afligidas por todo tipo de
enfermedades, personas poseídas por demonios, enfermos mentales,
paralíticos, y él los sanaba a todos. \bibverse{25} Grandes multitudes
le siguieron desde Galilea, Decápolis, Jerusalén, Judea y la región que
estaba al otro lado del Jordán.

\hypertarget{section-4}{%
\section{5}\label{section-4}}

\bibverse{1} Cuando Jesús vio que las multitudes le seguían, subió a una
montaña. Allí se sentó junto con sus discípulos. \bibverse{2} Y comenzó
a enseñarles, diciendo:

\bibverse{3} ``Benditos son los que reconocen que son pobres
espiritualmente, porque de ellos es el reino de los cielos. \bibverse{4}
Benditos son los que lloran, porque ellos serán consolados. \bibverse{5}
Benditos son los que son amables\footnote{\textbf{5:5} Queriendo decir
  mansos, de temperamento afable.}, porque ellos poseerán el mundo
entero. \bibverse{6} Benditos son aquellos cuyo mayor deseo\footnote{\textbf{5:6}
  Literalmente, ``aquellos que están hambrientos y sedientos.''} es
hacer lo justo, porque su deseo será saciado. \bibverse{7} Benditos
aquellos que son misericordiosos, porque a ellos se les mostrará
misericordia. \bibverse{8} Benditos son; los corazón puro, porque ellos
verán a Dios. \bibverse{9} Benditos aquellos que trabajan por traer la
paz, porque ellos serán llamados hijos de Dios. \bibverse{10} Benditos
aquellos que son perseguidos por lo que es justo, porque de ellos es el
reino de los cielos. \bibverse{11} Benditos ustedes cuando las personas
los insulten y los persigan, y los acusen de todo tipo de males por mi
causa. \bibverse{12} Estén felices, muy felices, porque recibirán una
gran recompensa en el cielo---pues ellos persiguieron de esa misma
manera a los profetas que vinieron antes de ustedes.

\bibverse{13} ``Ustedes son la sal de la tierra, pero si la sal pierde
su sabor\footnote{\textbf{5:13} O ``inútil.''}, ¿cómo podrán hacer que
sea salada nuevamente? No sirve para nada, sino que se bota y es
pisoteada. \bibverse{14} Ustedes son la luz del mundo. Una ciudad que
está construida sobre lo alto de una montaña no puede ocultarse.
\bibverse{15} Nadie enciende una lámpara para luego ocultarla bajo una
cesta. No, se le coloca sobre un candelero y así da luz a todos los que
están en la casa. \bibverse{16} De la misma manera, ustedes deben dejar
que su luz brille delante de todos a fin de que ellos puedan ver las
cosas buenas que ustedes hacen y alaben a su Padre celestial.

\bibverse{17} ``No piensen que vine a abolir la ley o los escritos de
los profetas. No vine a abolirlos, sino a cumplirlos. \bibverse{18} Les
aseguro que hasta que el cielo y la tierra lleguen a su fin, ni una sola
letra, ni un solo punto que está en la ley quedarán descontinuados antes
de que todo se haya cumplido. \bibverse{19} De manera que cualquiera que
desprecia\footnote{\textbf{5:19} O ``invalida''} el mandamiento menos
importante, y enseña a las personas a hacer lo mismo, será considerado
como el menos importante en el reino de los cielos; pero cualquiera que
practica y enseña los mandamientos será considerado grande en el reino
de los cielos. \bibverse{20} Les digo que a menos que la justicia de
ustedes no sea mayor que la justicia de los maestros religiosos y de los
Fariseos, no podrán entrar nunca al reino de los cielos.

\bibverse{21} ``Ustedes han escuchado que la ley dijo\footnote{\textbf{5:21}
  Literalmente, ``Ustedes han escuchado que fue dicho.'' Esta frase se
  usa a menudo en este pasaje del texto por parte de Jesús para
  referirse a las leyes que se encuentran en el Antiguo Testamento.} al
pueblo de hace mucho tiempo: `No matarás, y cualquiera que cometa
asesinato será condenado como culpable\footnote{\textbf{5:21} O,
  ``responsable de juicio.''}.' \bibverse{22} Pero yo les digo:
cualquiera que está enojado con su hermano será condenado como culpable.
Cualquiera que llama a su hermano ``idiota'' tiene que dar cuenta ante
el concilio\footnote{\textbf{5:22} Probablemente, el concilio del
  Sanedrín.}, y cualquiera que insulta a la gente, de seguro irá al
fuego del juicio\footnote{\textbf{5:22} La palabra aquí, literalmente,
  es ``Gehenna,'' que a menudo se traduce como ``infierno'' o ``fuego
  infernal.'' Gehenna era el lugar situado a las afueras de Jerusalén
  donde se encendían fogatas para quemar la basura. ``Infierno'' es un
  concepto derivado de la mitología nórdica y anglosajona y no tiene
  paralelo con la idea de la cual se habla aquí.}.''

\bibverse{23} ``Si estás delante del altar presentando una ofrenda, y
recuerdas que tu hermano tiene algo contra ti, \bibverse{24} deja tu
ofrenda sobre el altar y ve y haz las paces con él primero, y luego
regresa y presenta tu ofrenda. \bibverse{25} Cuando vayas camino a la
corte con tu adversario, asegúrate de arreglar las cosas rápidamente. De
lo contrario, tu acusador podría entregarte ante el juez, y el juez te
entregará a la corte oficial, y serás llevado a la cárcel. \bibverse{26}
En verdad te digo: no saldrás de allí hasta que hayas pagado hasta el
último centavo.

\bibverse{27} ``Ustedes han escuchado que la ley dijo: `No cometerás
adulterio.' \bibverse{28} Pero yo les digo que todo el que mira con
lujuria a una mujer ya ha cometido adulterio en su corazón.
\bibverse{29} Si tu ojo derecho te lleva a pecar, entonces sácalo y
bótalo, porque es mejor perder una parte de tu cuerpo y no que todo tu
cuerpo sea lanzado en el fuego del juicio. \bibverse{30} Si tu mano
derecha te lleva a pecar, entonces córtala y bótala, porque es mejor que
pierdas uno de tus miembros y no que todo tu cuerpo vaya al fuego del
juicio.

\bibverse{31} ``La ley también dijo: `Si un hombre se divorcia de su
esposa, debe darle un certificado de divorcio.' \bibverse{32} Pero yo
les digo que cualquier hombre que se divorcia de su esposa, a menos que
sea por inmoralidad sexual, la hace cometer adulterio, y cualquiera que
se case con una mujer divorciada, comete adulterio.

\bibverse{33} ``Y una vez más, ustedes han escuchado que la ley dijo al
pueblo de hace mucho tiempo: `No jurarás en falso. En lugar de ello,
asegúrese de cumplir sus juramentos al Señor.' \bibverse{34} Pero yo les
digo: no juren nada. No juren por el cielo, porque ese es el trono de
Dios. \bibverse{35} No juren por la tierra, porque es allí donde
descansan sus pies. No juren por Jerusalén, por que es la ciudad del
gran Rey. \bibverse{36} Ni siquiera juren por su cabeza, porque ustedes
no tienen el poder de hacer que uno solo de sus cabellos sea blanco o
negro. \bibverse{37} Solamente digan sí o no---cualquier cosa aparte de
esto viene del Maligno.

\bibverse{38} ``Ustedes han escuchado que la ley dijo: `Ojo por ojo,
diente por diente.' \bibverse{39} Pero yo les digo, no pongan
resistencia a alguien que es malvado. Si alguien te da una bofetada, pon
la otra mejilla también. \bibverse{40} Si alguien quiere demandarte en
una corte y toma tu camisa, dale tu abrigo también\footnote{\textbf{5:40}
  El abrigo era una prenda de vestir mucho más valiosa.}. \bibverse{41}
Si alguien te pide que le acompañes una milla, acompáñale dos
millas\footnote{\textbf{5:41} Probablemente refiriéndose a un soldado
  romano que pedía que otra persona le llevara sus pertenencias.}.
\bibverse{42} Da a quienes te pidan, y no rechaces a quienes vengan a
pedirte algo prestado.

\bibverse{43} ``Ustedes han escuchado que la ley dijo: `Ama a tu prójimo
y odia a tu enemigo.' \bibverse{44} Pero yo les digo: amen a sus
enemigos y oren por los que los persiguen, \bibverse{45} a fin de que
ustedes lleguen a ser hijos del Padre celestial. Porque su sol sale
sobre buenos y malos; y él hace que la lluvia caiga sobre aquellos que
hacen el bien y también sobre los que hacen el mal. \bibverse{46} Porque
si ustedes solamente aman a quienes los aman, ¿qué recompensa tienen por
eso? ¿No hacen eso incluso los recaudadores de impuestos? \bibverse{47}
Si ustedes solo hablan de manera amable con su familia, ¿qué estarán
haciendo que no hagan todos los demás? ¡Incluso los paganos\footnote{\textbf{5:47}
  5:47 Literalmente, ``naciones,'' o ``gentiles.'' Es un término
  comúnmente utilizado en al Nuevo Testamento para identificar a quienes
  no eran judíos, a aquellos quienes se consideraba que no seguían al
  verdadero Dios.} hacen eso! \bibverse{48} Crezcan y sean completamente
fieles\footnote{\textbf{5:48} Literalmente, ``perfectos, completos, sin
  división, integrales, maduros.'' El concepto aquí se refiere a un
  estilo de vida totalmente dedicado a Dios más que a un concepto
  abstracto de perfección. El enfoque está en la madurez spiritual que
  se traduce en el hecho de que se pueda depender de esa persona,
  alguien en quien se puede confiar.}, así como su Padre que está en el
cielo es fiel.

\hypertarget{section-5}{%
\section{6}\label{section-5}}

\bibverse{1} ``Asegúrense de que sus buenas obras no sean hechas delante
de la gente, solo para que los vean. De lo contrario, no tendrán ninguna
recompensa de su Padre que está en el cielo. \bibverse{2} Cuando den a
los pobres, no sean como los hipócritas\footnote{\textbf{6:2} Esta es
  una palabra tomada del griego que literalmente significa
  ``actuación.''} que se jactan anunciando en las sinagogas y en las
calles lo que hacen para que la gente los alabe. Yo les digo la verdad:
ellos ya tienen su recompensa. \bibverse{3} Cuando den a los pobres, que
su mano izquierda no sepa lo que está haciendo su mano derecha.
\bibverse{4} De esta manera, lo que den será secreto, y su Padre que ve
lo que ocurre en secreto, los recompensará.

\bibverse{5} ``Cuando oren, no sean como los hipócritas, porque a ellos
les encanta ponerse en pie y orar en las sinagogas y en las esquinas de
las calles para que la gente los vea. Yo les prometo que ellos ya tienen
su recompensa. \bibverse{6} Pero ustedes, cuando oren, entren a su casa
y cierren la puerta, y oren a su Padre en privado, y su Padre que ve lo
que ocurre en privado, los recompensará. \bibverse{7} Cuando oren, no
usen palabrerías incoherentes como hacen los gentiles, que piensan que
serán escuchados por todas las palabras que repiten. \bibverse{8} No
sean como ellos, porque su Padre sabe lo que ustedes necesitan incluso
antes de que ustedes se lo pidan. \bibverse{9} Así que oren de esta
manera:

``Nuestro Padre celestial, que tu nombre sean honrado. \bibverse{10}
Venga tu reino. Que tu voluntad sea hecha en la tierra como se hace en
el cielo. \bibverse{11} Por favor, danos hoy el alimento que
necesitamos. \bibverse{12} Perdona nuestros pecados, así como nosotros
hemos perdonado a quienes han pecado contra nosotros. \bibverse{13} No
dejes que seamos tentados a hacer el mal\footnote{\textbf{6:13} O, ``Por
  favor, ayúdanos a no rendirnos ante la tentación.''}, y sálvanos del
Maligno.

\bibverse{14} ``Porque si perdonan a quienes pecan contra ustedes, su
Padre celestial también los perdonará. \bibverse{15} Pero si no perdonan
a quienes pecan contra ustedes, entonces su Padre celestial no les
perdonará sus pecados.

\bibverse{16} ``Cuando ayunen, no sean como los hipócritas que ponen
caras tristes y un semblante espantoso para que todos vean que están
ayunando. \bibverse{17} En lugar de eso, cuando ayunen, laven sus
rostros y luzcan elegantes, \bibverse{18} a fin de que las personas no
vean que ustedes están ayunando, y su Padre que es invisible y que ve lo
que ocurre en privado, los recompensará.

\bibverse{19} ``No acumulen riquezas aquí en la tierra donde la polilla
y el óxido las dañan, y donde los ladrones entran y las roban.
\bibverse{20} En lugar de ello, ustedes deben acumular sus riquezas en
el cielo, donde la polilla y el óxido no las dañan, y donde los ladrones
no entran ni las roban. \bibverse{21} Porque donde acumulen su riqueza,
allí es donde estará su corazón también.

\bibverse{22} ``El ojo es como una lámpara que ilumina el cuerpo. De
manera que si tu ojo es sano\footnote{\textbf{6:22} O, ``bueno,
  inocente.''}, entonces todo tu cuerpo tendrá luz. \bibverse{23} Pero
si tu ojo es perverso, entonces todo tu cuerpo estará en tinieblas. Si
la luz dentro de ustedes está en tinieblas, ¡cuán oscuro es eso!
\bibverse{24} Nadie puede servir a dos amos. Odiarán a uno y amarán al
otro, o serán devotos a uno y despreciarán al otro. Ustedes no pueden
servir a Dios y al dinero a la vez\footnote{\textbf{6:24} Literalmente,
  ``Mammón,'' una transliteración de la palabra aramea que se usa para
  referirse al dios sirio del dinero y la riqueza.}.

\bibverse{25} ``Por eso les digo que no se preocupen por sus vidas. No
se preocupen por lo que van a comer, o por lo que van a beber, o por la
ropa con la que van a vestir. ¿Acaso no es la vida más importante que la
comida, y el cuerpo más que la ropa? \bibverse{26} Miren las
aves\footnote{\textbf{6:26} Literalmente, ``aves del cielo,''
  refiriéndose a las aves silvestres más que a las aves domésticas.}---ellas
no siembran ni cosechan, ni guardan alimento en los graneros, porque el
Padre celestial las alimenta. ¿No son ustedes más que las aves?
\bibverse{27} ¿Quién de ustedes puede, por mucho que se afane, añadir un
minuto a su vida? \bibverse{28} ¿Y por qué se preocupan por la ropa?
Miren las hermosas flores del campo. Miren cómo crecen: No trabajan ni
hilan. \bibverse{29} Pero les digo que ni siquiera Salomón en todo su
esplendor se vistió como una de esas flores. \bibverse{30} De modo que
si Dios decora los campos así, la hierba que está hoy aquí y que mañana
es lanzada al fuego, ¿no hará mucho más por ustedes que son personas que
creen tan poco? \bibverse{31} Así que no se preocupen diciendo `¿Qué
comeremos?' o `¿Qué beberemos?' o `¿Qué vestiremos?' \bibverse{32} Todas
estas son las cosas que los paganos persiguen, pero el Padre celestial
sabe todo lo que ustedes necesitan. \bibverse{33} Busquen su reino en
primer lugar, y su senda de justicia, y todo se les dará. \bibverse{34}
Así que no se preocupen por el día de mañana, porque el mañana puede
preocuparse por sí mismo. Hay ya suficiente mal en cada día.

\hypertarget{section-6}{%
\section{7}\label{section-6}}

\bibverse{1} ``No juzguen a otros, para que ustedes no sean juzgados.
\bibverse{2} Porque cualquiera que sea el criterio que usen para juzgar
a otros, será usado para juzgarlos a ustedes, y cualquiera que sea la
medida que ustedes usen para medir a otros, será usada para medirlos a
ustedes. \bibverse{3} ¿Por qué miras la astilla que está en el ojo de tu
hermano? ¿No te das cuenta de la viga que está en tu propio ojo?
\bibverse{4} ¿Cómo puedes decirle a tu hermano: `Déjame sacarte esa
astilla de tu ojo' mientras tu tienes una viga en tu propio ojo?
\bibverse{5} ¡Estás siendo un hipócrita! Primero saca la viga que tienes
en tu propio ojo. Entonces podrás ver con claridad y sacar la astilla
del ojo de tu hermano.

\bibverse{6} ``No den a los perros lo que es santo. No tiren sus perlas
a los cerdos. Así los cerdos no las pisotearán, y los perros no vendrán
a atacarlos a ustedes.

\bibverse{7} ``Pidan y se les dará, busquen y encontrarán, toquen a la
puerta y la puerta se abrirá para ustedes\footnote{\textbf{7:7} En el
  texto original, estos son presentes imperativos, y podría traducirse
  como ``sigan pidiendo'' etc.}. \bibverse{8} Todo el que pide, recibe;
todo el que busca, encuentra; a todo el que toca, se le abre la puerta.
\bibverse{9} ¿Acaso alguno de ustedes le daría una piedra a su hijo si
este le pide un pan? \bibverse{10} ¿O si le pidiera un pez, le daría una
serpiente? \bibverse{11} De modo que si incluso ustedes que son malos
saben dar cosas buenas a sus hijos, cuánto más el Padre celestial dará
cosas buenas a quienes le piden.

\bibverse{12} ``Traten a los demás como quieren que los traten a
ustedes. Esto resume la ley y los profetas. \bibverse{13} Entren por la
puerta estrecha. Porque es amplia la puerta y espacioso el camino que
lleva a la destrucción, y muchos andan por él. \bibverse{14} Pero
estrecha es la puerta y angosto el camino que llevan a la vida, y solo
unos pocos lo encuentran.

\bibverse{15} ``Tengan cuidado con los falsos profetas que vienen
vestidos de ovejas, pero por dentro son lobos feroces. \bibverse{16}
Pueden reconocerlos por sus frutos\footnote{\textbf{7:16} En otras
  palabras, ustedes pueden reconocerlos por los resultados de lo que
  hacen.}. ¿Acaso las personas cosechan uvas de los matorrales de
espinos, o higos de los cardos? \bibverse{17} De modo que todo árbol
bueno produce frutos buenos, mientras que un árbol malo produce frutos
malos. \bibverse{18} Un buen árbol no puede producir frutos malos, y un
árbol malo no puede producir frutos buenos. \bibverse{19} Todo árbol que
no produce frutos buenos, se corta y se lanza al fuego. \bibverse{20}
Así que por sus frutos los conocerán.

\bibverse{21} ``No todo el que me dice `Señor, Señor' entrará al reino
de los cielos---sino solo aquellos que hacen la voluntad de mi Padre que
está en el cielo. \bibverse{22} Muchos me dirán el día del juicio:
`Señor, Señor, ¿acaso no profetizamos, nos sacamos demonios e hicimos
muchos milagros en tu nombre?' \bibverse{23} Entonces yo les diré: `Yo
nunca los conocí a ustedes. ¡Apártense de mi, practicantes de la
maldad!'\footnote{\textbf{7:23} Ver Salmos 6:8.} \bibverse{24} Todo
aquél que escucha las palabras que yo digo, y las sigue, es como el
hombre sabio que construyó su casa sobre la roca sólida. \bibverse{25}
La lluvia cayó, hubo inundación y los vientos soplaron fuertemente
contra aquella casa, pero no se cayó porque su fundamento estaba sobre
la roca sólida. \bibverse{26} Pero todo aquél que escucha las palabras
que yo digo y no las sigue, es como el hombre necio que construyó su
casa sobre la arena. \bibverse{27} La lluvia cayó, hubo inundación y los
vientos soplaron fuertemente contra aquella casa, y se cayó. Colapsó por
completo.''

\bibverse{28} Cuando Jesús terminó de explicar estas cosas, las
multitudes se maravillaban de su enseñanza, \bibverse{29} porque él
enseñaba como alguien que tenía autoridad, y no como sus maestros
religiosos.

\hypertarget{section-7}{%
\section{8}\label{section-7}}

\bibverse{1} Grandes multitudes siguieron a Jesús cuando bajó de la
montaña. \bibverse{2} Un leproso se acercó a él, y se arrodilló,
adorándolo, y le dijo: ``Señor, si quieres, por favor sáname.''
\bibverse{3} Jesús se extendió hacia él y lo tocó con su mano.
``Quiero,'' le dijo. ``Queda sano.'' Inmediatamente este hombre fue
sanado de su lepra.

\bibverse{4} ``Asegúrate de no contárselo a nadie,'' le dijo Jesús. ``Ve
y preséntate ante el sacerdote y da la ofrenda que Moisés ordenó, como
evidencia pública\footnote{\textbf{8:4} Como prueba de que había sido
  sanado y de que estaba ceremonialmente limpio.}.''

\bibverse{5} Cuando Jesús entró a Capernaúm, un centurión se le acercó,
suplicándole su ayuda, \bibverse{6} ``Señor, mi siervo está en casa,
acostado y sin poder moverse. Está sufriendo una terrible agonía.''

\bibverse{7} ``Iré y lo sanaré,'' respondió Jesús.

\bibverse{8} El centurión respondió: ``Señor, no merezco una visita a mi
casa. Solo di la palabra y mi siervo quedará sano. \bibverse{9} Porque
yo mismo estoy bajo la autoridad de mis superiores, y a la vez yo
también tengo soldados bajo mi mando. Yo le ordeno a uno: `¡Ve!' y él
va. Mando a otro: `¡Ven!' y él viene. Digo a mi siervo: `¡Haz esto!' y
él lo hace.''

\bibverse{10} Cuando Jesús escuchó lo que este hombre dijo, se quedó
asombrado. Entonces le dijo a los que le seguían: ``En verdad les digo
que no he encontrado este tipo de confianza en ninguna parte de Israel.
\bibverse{11} Les digo que muchos vendrán del este y del oeste, y se
sentarán con Abraham e Isaac en el reino de los cielos. \bibverse{12}
Pero los herederos\footnote{\textbf{8:12} Refiriéndose a los
  descendientes de Abraham e Isaac que confiaron en su ascendencia para
  la salvación.} del reino serán lanzados a la oscuridad absoluta, donde
habrá lamento y crujir de dientes.''

\bibverse{13} Entonces Jesús le dijo al centurión, ``Ve a casa. Lo que
pediste ya fue hecho, como creíste que pasaría.'' Y el siervo fue sanado
inmediatamente.

\bibverse{14} Cuando Jesús llegó a la casa de Pedro, vio que la suegra
de Pedro estaba enferma en cama y tenía una fiebre muy alta.
\bibverse{15} Entonces Jesús tocó su mano y se le quitó la fiebre. Ella
se levantó y comenzó a prepararle comida a Jesús. \bibverse{16} Cuando
llegó la noche, trajeron ante Jesús a un hombre endemoniado. Con solo
una orden, Jesús hizo que los espíritus salieran de él, y sanó a todos
los que estaban enfermos. \bibverse{17} Esto cumplió lo que el profeta
Isaías dijo: ``Él sanó nuestras enfermedades y nos libertó de nuestras
dolencias.''

\bibverse{18} Cuando Jesús vio las multitudes que lo rodeaban, dio
instrucciones de que debían\footnote{\textbf{8:18} ``debían'' se refiere
  a Jesús y los discípulos.} ir al otro lado del lago. \bibverse{19}
Entonces uno de los maestros religiosos se acercó a él y le dijo:
``Maestro, te seguiré adonde vayas.''

\bibverse{20} ``Los zorros tienen guaridas y las aves silvestres tienen
nidos, pero el Hijo del hombre no tiene dónde recostarse y
descansar\footnote{\textbf{8:20} Literalmente, ``recostar su cabeza.''},''
le dijo Jesús.

\bibverse{21} Otro discípulo le dijo a Jesús: ``Señor, primero déjame ir
y sepultar a mi padre.''

\bibverse{22} ``Sígueme. Deja que los muertos sepulten a sus propios
muertos,'' le respondió Jesús.

\bibverse{23} Entonces Jesús subió a una barca y sus discípulos se
fueron con él. \bibverse{24} De repente, sopló una fuerte tormenta, y
las olas golpeaban fuertemente contra la barca, pero Jesús seguía
durmiendo. \bibverse{25} Los discípulos se acercaron a él y lo
despertaron gritándole: ``¡Sálvanos, Señor! ¡Vamos a hundirnos!''

\bibverse{26} ``¿Por qué tienen tanto miedo? ¿Por qué tienen tan poca
confianza?'' les preguntó Jesús. Entonces se levantó y ordenó a los
vientos y las olas que se detuvieran. Y todo quedó completamente en
calma. \bibverse{27} Los discípulos estaban asombrados y decían:
``¿Quién es este? ¿Incluso los vientos y las olas le obedecen?''

\bibverse{28} Cuando llegaron al otro lado, a la región de los
gadarenos, dos hombres endemoniados salieron del cementerio para
encontrarse con Jesús. Estos hombres eran tan peligrosos que nadie se
atrevía a pasar por ese camino. \bibverse{29} Y ellos gritaban: ``¿Qué
tienes que ver con nosotros, Hijo de Dios? ¿Has venido a torturarnos
antes de tiempo?''

\bibverse{30} A lo lejos, había un gran hato de cerdos comiendo.
\bibverse{31} Los demonios le suplicaron a Jesús: ``Si vas a sacarnos de
aquí, envíanos a ese hato de cerdos.''

\bibverse{32} ``¡Vayan!'' les dijo Jesús. Los demonios salieron de los
dos hombres y huyeron hacia el hato de cerdos. Todo el hato de cerdos
corrió, descendiendo por una pendiente, hasta que cayeron al mar y se
ahogaron. \bibverse{33} Los que cuidaban el rebaño de cerdos, salieron
corriendo. Entonces se fueron hacia la ciudad y le contaron a la gente
que estaba allí todo lo que había sucedido y lo que había ocurrido con
los dos hombres endemoniados. \bibverse{34} Y toda la ciudad salió para
encontrarse con Jesús. Cuando lo encontraron, le suplicaron que
abandonara su ciudad.

\hypertarget{section-8}{%
\section{9}\label{section-8}}

\bibverse{1} Entonces Jesús tomó una barca para cruzar nuevamente el
lago hacia la ciudad donde él vivía. \bibverse{2} Allí le trajeron a un
hombre paralítico acostado en una estera. Cuando Jesús vio cuánto
confiaban en él, le dijo al paralítico: ``¡Anímate, amigo
mío\footnote{\textbf{9:2} Literalmente, ``hijo.''}! Tus pecados están
perdonados.''

\bibverse{3} En respuesta a esto, algunos de los maestros religiosos
decían para sí mismos: ``¡Está blasfemando!''

\bibverse{4} Pero Jesús sabía lo que ellos estaban pensando. Entonces
les preguntó: ``¿Por qué tienen pensamientos malvados en sus corazones?
\bibverse{5} ¿Qué es más fácil decir, `tus pecados están perdonados,' o
`levántate y camina'? \bibverse{6} Pero ahora, para convencerlos de que
el Hijo del hombre tiene autoridad para perdonar pecados\ldots{}''
Dirigiéndose al hombre paralítico, le dijo: ``¡Levántate, toma tu estera
y vete a casa!'' \bibverse{7} El hombre se levantó y se fue a su casa.
\bibverse{8} Cuando las multitudes vieron lo que había sucedido, estaban
atemorizados. Entonces alabaron a Dios por haber dado a los seres
humanos semejante poder.

\bibverse{9} Cuando Jesús se fue de allí, vio a un hombre llamado Mateo
que estaba sentado en su cabina de cobro de impuestos. Jesús lo llamó
diciéndole ``Sígueme.'' Entonces él se levantó y siguió a Jesús.
\bibverse{10} Mientras Jesús comía en la casa de Mateo, muchos
recaudadores de impuestos vinieron y se sentaron en la mesa con él y sus
discípulos. \bibverse{11} Y cuando los Fariseos vieron esto, le
preguntaron a los discípulos de Jesús: ``¿Por qué el Maestro de ustedes
come con los recaudadores de impuestos y pecadores?''

\bibverse{12} Cuando Jesús escuchó la pregunta, respondió: ``Los que
están sanos no necesitan de un médico, pero los que están enfermos, sí.
\bibverse{13} Vayan y descubran lo que esto significa: `quiero
misericordia, no sacrificio. Porque no vine a llamar a los que hacen el
bien---Vine a llamar a los pecadores.'\,''\footnote{\textbf{9:13} Oseas
  6:6.}

\bibverse{14} Entonces los discípulos de Juan vinieron y le preguntaron:
``¿Por qué nosotros y los Fariseos ayunamos a menudo y tus discípulos no
lo hacen?''

\bibverse{15} ¿Acaso los invitados a la boda lloran cuando el novio está
con ellos?'' respondió Jesús. ``Pero viene el tiempo cuando el novio ya
no estará y entonces ayunarán. \bibverse{16} Nadie pone un parche nuevo
en ropas viejas, de lo contrario, se encogerá y hará que la rotura luzca
peor. \bibverse{17} Nadie echa tampoco el vino nuevo en odres viejos, de
lo contrario los odres podrían romperse, derramando así el vino y
dañando los odres. No, el vino nuevo se coloca en odres nuevos, y así
ambos perduran.

\bibverse{18} Mientras él les decía esto, uno de los oficiales
principales llegó y se postró delante de él. ``Mi hija acaba de morir,''
le dijo el hombre a Jesús. ``Pero sé que si tú vas y colocas tu mano
sobre ella, volverá a vivir.''

\bibverse{19} Jesús y sus discípulos se levantaron y lo siguieron.
\bibverse{20} En ese momento, una mujer que había estado enferma con
sangrado durante doce años, venía detrás de él y tocó el dobladillo de
su manto. \bibverse{21} Ella había pensado para sí: ``Si tan solo puedo
llegar a tocar su manto, seré sanada.''

\bibverse{22} Jesús se dio vuelta y la vio. ``Alégrate hija, pues tu
confianza en mi te ha sanado,'' le dijo. Y la mujer fue sanada de
inmediato.

\bibverse{23} Jesús llegó a la casa del oficial. Vio a los que tocaban
las flautas y escuchó a la multitud que lloraba a gritos. \bibverse{24}
``Por favor, salgan'' -- les dijo -- ``porque esta niña no está muerta,
sino que simplemente está dormida.'' Pero ellos se rieron y se burlaron
de él. \bibverse{25} Sin embargo, cuando la multitud había sido
despedida, Jesús entró y tomó a la niña por la mano y esta se levantó.
\bibverse{26} Y la noticia sobre lo que había ocurrido se esparció por
toda esa región.

\bibverse{27} Al seguir Jesús su camino, dos hombres ciegos lo seguían y
le gritaban: ``¡Hijo de David, ten misericordia de nosotros!''
\bibverse{28} Y cuando Jesús entró a la casa donde se alojaba, los
hombres ciegos entraron también.

``¿Están convencidos de que yo puedo hacer esto?'' les preguntó.

``Sí, Señor,'' respondieron ellos.

\bibverse{29} Entonces Jesús tocó los ojos de ellos, y dijo: ``¡Por la
confianza que tienen en mí, así será!'' \bibverse{30} Y ellos pudieron
ver. Jesús les advirtió: ``Asegúrense de que nadie sepa esto.''
\bibverse{31} Pero ellos se fueron y dieron a conocer acerca de Jesús
por todas partes.

\bibverse{32} Cuando Jesús y sus discípulos ya se marchaban, trajeron
ante Jesús a un hombre que estaba mudo y endemoniado. \bibverse{33}
Cuando el demonio fue expulsado de él, el hombre habló, y las multitudes
estaban maravilladas. ``Nunca antes había ocurrido algo como esto en
Israel,'' decían. \bibverse{34} Pero los Fariseos comentaban diciendo:
``el echa fuera los demonios con el poder del jefe de los demonios.''

\bibverse{35} Jesús iba a todas partes, visitando ciudades y aldeas.
Enseñaba en sus sinagogas, les enseñaba acerca de la buena noticia del
reino, y sanaba todo tipo de enfermedades. \bibverse{36} Cuando veía las
multitudes, Jesús sentía gran compasión por ellos, porque estaban
atribulados y desamparados, como ovejas sin pastor. \bibverse{37}
Entonces le dijo a sus discípulos, ``la cosecha es grande, pero hay
apenas unos pocos trabajadores. \bibverse{38} Oren al Señor de la
cosecha, y pídanle que envíe más trabajadores para su cosecha.''

\hypertarget{section-9}{%
\section{10}\label{section-9}}

\bibverse{1} Jesús llamó y reunió a sus doce discípulos y les dio poder
para echar fuera espíritus malos y para sanar todo tipo de enfermedades.

\bibverse{2} Estos son los nombres de los doce apóstoles: primero, Simón
(también llamado Pedro), su hermano Andrés, Santiago el hijo de Zebedeo,
su hermano Juan, \bibverse{3} Felipe, Bartolomé, Tomás, Mateo el
recaudador de impuestos, Santiago el hijo de Alfeo, Tadeo, \bibverse{4}
Simón el revolucionario y Judas Iscariote, quien entregó a Jesús.

\bibverse{5} A estos doce envió Jesús, diciéndoles: ``no vayan a los
gentiles, ni a ninguna ciudad samaritana. \bibverse{6} Ustedes deben ir
a las ovejas perdidas de la casa de Israel. \bibverse{7} Donde vayan,
díganle a la gente: `el reino de los cielos está cerca.' \bibverse{8}
Sanen a los que estén enfermos. Resuciten a los muertos. Sanen a los
leprosos. Echen fuera demonios. ¡Ustedes recibieron gratuitamente,
entonces den gratuitamente! \bibverse{9} No lleven oro, plata, ni
monedas de cobre en sus bolsillos, \bibverse{10} ni lleven una bolsa de
provisiones para el camino, ni dos abrigos, o sandalias, ni un bastón
para caminar, porque todo trabajador merece su sustento\footnote{\textbf{10:10}
  O ``alimento.''}. \bibverse{11} Donde vayan, cualquiera sea la ciudad
o aldea, pregunten por alguien que viva conforme a buenos principios, y
quédense allí hasta que se marchen. \bibverse{12} Cuando lleguen a una
casa, dejen bendición en ella. \bibverse{13} Si esa casa la merece,
dejen su paz\footnote{\textbf{10:13} ``Paz, refiriéndose a bendición.}
en ella, pero si no la merece, la paz regresará a ustedes.

\bibverse{14} ``Si alguien no los recibe bien, y se niega a escuchar el
mensaje que ustedes tienen que decir, entonces váyanse de esa casa o de
esa ciudad, sacudiendo el polvo de sus pies mientras se marchan.
\bibverse{15} Les digo la verdad: ¡Mejor será el Día del Juicio para
Sodoma y Gomorra que para esa ciudad!

\bibverse{16} ``Miren que los estoy enviando como ovejas entre lobos.
Así que sean astutos como serpientes y mansos como palomas.
\bibverse{17} Cuídense de aquellos que los entregarán para ser juzgados
en los concilios de las ciudades\footnote{\textbf{10:17} Literalmente,
  ``sanedrines,'' que eran cortes religiosas locales.} y que los
azotarán en sus sinagogas. \bibverse{18} Ustedes serán arrastrados ante
gobernantes y reyes por mi causa, para dar testimonio a ellos y a los
gentiles. \bibverse{19} Pero cuando ellos los lleven a juicio, no se
preocupen por la manera como deben hablar o por lo que deben decir,
porque a ustedes se les dirá lo que deben decir en el momento correcto.
\bibverse{20} Porque no serán ustedes los que hablarán, sino el espíritu
del Padre quien hablará por medio de ustedes. \bibverse{21} El hermano
entregará a su hermano y lo mandará a matar, y el padre hará lo mismo
con su hijo. Los hijos se rebelarán contra sus padres, y los entregarán
a la muerte. \bibverse{22} Todo el mundo los odiará a ustedes porque
ustedes me siguen a mi, pero todo aquél que persevere hasta el fin, será
salvo.

\bibverse{23} ``Cuando ustedes sean perseguidos en una ciudad, huyan a
otra. Les digo la verdad: no terminarán de ir a las ciudades de Israel
antes de que venga el Hijo del hombre. \bibverse{24} Los discípulos no
son más importantes que su maestro; \bibverse{25} ellos deben estar
satisfechos con llegar a ser como su maestro, y los siervos como su amo.
Si a quien es la cabeza del hogar le han llamado demonio
Belcebú,\footnote{\textbf{10:25} Belcebú, refiriéndose a Satanás.} ¡aún
más llamarán demonios a los demás miembros de esta casa! \bibverse{26}
Así que no les tengan miedo, porque no hay nada encubierto que no salga
a la luz, ni hay nada oculto que no se llegue a saber. \bibverse{27} Lo
que yo les digo aquí en la oscuridad, díganlo a la luz del día, y lo que
han oído como un susurro en sus oídos, grítenlo desde las azoteas.
\bibverse{28} No tengan miedo de aquellos que pueden matarlos
físicamente, pero que no pueden matarlos espiritualmente. En lugar de
ello, tengan miedo de Aquel que puede destruirlos física y
espiritualmente en el fuego de la destrucción\footnote{\textbf{10:28}
  Literalmente, ``Gehenna.'' Ver la nota del versículo 5:22.}.
\bibverse{29} ¿No se venden dos gorriones por el precio de un solo
centavo? Pero ninguno de ellos cae al suelo sin que el Padre lo sepa.
\bibverse{30} Incluso los cabellos que ustedes tienen en sus cabezas han
sido contados. \bibverse{31} Así que no se preocupen. ¡Ustedes valen más
que muchos gorriones!

\bibverse{32} ``Si alguno declara públicamente su compromiso\footnote{\textbf{10:32}
  Literalmente, ``confiesa.''} conmigo, yo también declararé mi
compromiso con él ante mi Padre que está en el cielo. \bibverse{33} Pero
si alguno me niega públicamente, yo también lo negaré ante mi Padre en
el cielo. \bibverse{34} No piensen que he venido a traer paz a la
tierra. No he venido a traer paz sino espada. \bibverse{35} He venido a
poner al hombre contra su padre, a la hija contra su madre, y a la nuera
contra su suegra. \bibverse{36} ¡Sus enemigos serán los de su propia
familia! \bibverse{37} Si ustedes aman a su padre o su madre más que a
mi, no merecen ser míos; y si aman a su hijo o hija más que a mi, no
merecen ser míos. \bibverse{38} Si no cargan su cruz y me siguen, no
merecen ser míos. \bibverse{39} Si tratan de salvar su vida, la
perderán\footnote{\textbf{10:39} En otras palabras, si tratas de
  aferrarte a la vida por medio de tus propios esfuerzos humanos, no lo
  lograrás.}, pero si pierden su vida por causa de mí, la salvarán.
\bibverse{40} Aquellos que los reciban a ustedes me reciben a mi, y
aquellos que me reciben a mi, reciben al que me envió. \bibverse{41}
Aquellos que reciben al profeta por ser profeta, recibirán recompensa de
un profeta. Los que reciben a quien hace el bien, recibirán la misma
recompensa como quien hace el bien. \bibverse{42} Les digo la verdad:
los que den una bebida de agua fresca al menos importante de mis
discípulos, no perderán su recompensa.''

\hypertarget{section-10}{%
\section{11}\label{section-10}}

\bibverse{1} Cuando Jesús hubo terminado de darles instrucciones a sus
doce discípulos, se fue de allí para ir a enseñar y predicar
públicamente en las ciudades cercanas. \bibverse{2} Estando Juan en
prisión, escuchó sobre lo que el Mesías estaba haciendo, así que envió a
sus discípulos \bibverse{3} para que preguntaran en su nombre, ``¿Eres
tú el que estábamos esperando, o debemos seguir esperando a alguien
más?''

\bibverse{4} Jesús les respondió: ``regresen y díganle a Juan lo que
ustedes oyen y lo que ven. \bibverse{5} Los ciegos pueden ver, los
paralíticos pueden caminar, los leprosos son sanados, los sordos pueden
oír, los muertos han vuelto a vivir y los pobres escuchan la buena
noticia. \bibverse{6} ¡Benditos son los que no me desprecian!''

\bibverse{7} Cuando los discípulos de Juan se fueron, Jesús comenzó a
hablarles a las multitudes sobre Juan. ¿Qué esperaban ver cuando
salieron al desierto? ¿Una caña zarandeada por el viento? \bibverse{8}
¿Entonces qué salieron a ver? ¿A un hombre vestido con ropas finas? Las
personas que visten así viven en los palacios de los reyes. \bibverse{9}
¿Qué salieron a ver, entonces? ¿A un profeta? Sí, ¡Y les digo que él es
mucho más que un profeta! \bibverse{10} Él es de quien habló la
Escritura: `Yo envío a mi mensajero por anticipado. Él preparará el
camino para ti.' \bibverse{11} Les digo la verdad, y es que entre la
humanidad,\footnote{\textbf{11:11} Literalmente, ``entre aquellos que
  son nacidos de mujer.''} no hay ninguno más grande que Juan el
Bautista, pero incluso el menos importante en el reino de los cielos es
más grande que él. \bibverse{12} Desde los tiempos de Juan el Bautista
hasta ahora el reino de los cielos sigue estando bajo ataque y personas
violentas están tratando de apoderarse de él a la fuerza. \bibverse{13}
Pues todos los profetas y la ley\footnote{\textbf{11:13} Refiriéndose al
  mensaje del Antiguo Testamento.} hablaron por Dios hasta que vino
Juan. \bibverse{14} Si ustedes están listos para creerlo, él es Elías,
el que debía venir. \bibverse{15} ¡Todo el que tenga oídos, oiga!

\bibverse{16} ``¿Con qué compararé esta generación? Son como unos niños
que están en la plaza del mercado y se gritan unos a otros diciendo:
\bibverse{17} `tocamos la flauta para ustedes y no danzaron; cantamos
canciones tristes y no lloraron.' \bibverse{18} Juan no vino para
festejar o beber, entonces la gente dice: `él está endemoniado'
\bibverse{19} Pero el Hijo del hombre, por el contrario, vino y festejó
y bebió, y la gente dice: `¡Miren, es un glotón y bebe mucho; es amigo
de los recaudadores de impuestos y de los pecadores!' Pero la sabiduría
demuestra ser correcta por los resultados de lo que hace\ldots{}''

\bibverse{20} Entonces Jesús comenzó a reprender a las ciudades donde
había hecho muchos de sus milagros porque no se habían arrepentido.
\bibverse{21} ``¡Qué tristeza por ustedes Corazín y Betsaida! Si los
milagros que hice entre ustedes se hubieran hecho en Tiro y Sidón, hace
mucho tiempo ellos se habrían arrepentido en silicio y cenizas.
\bibverse{22} ¡Pero les digo que el Día del Juicio será mejor para Tiro
y Sidón que para ustedes! \bibverse{23} Y ¿qué decir de ti, Capernaúm?
¿Serás exaltada hasta el cielo? No, ¡Tú irás al Hades! Si los milagros
que hice entre ustedes hubieran sido hechos en Sodoma, aún hoy existiría
Sodoma. \bibverse{24} ¡Pero te digo que a Sodoma le irá mejor en el Día
del Juicio que a ti!''

\bibverse{25} Entonces Jesús oró: ``Te alabo, Padre, Señor del cielo y
de la tierra, porque has ocultado estas cosas de las mentes de los
inteligentes y sabios. Por el contrario, las has revelado a personas
comunes\footnote{\textbf{11:25} Literalmente, a ``infantes.''}.
\bibverse{26} ¡Sí, Padre, te complaciste en hacerlo así! \bibverse{27}
El Padre lo ha confiado todo en mis manos, y ninguno entiende
verdaderamente al Hijo, excepto el Padre, y nadie entiende
verdaderamente al Padre, excepto el Hijo, y aquellos a quienes el Hijo
elige para mostrarles al Padre. \bibverse{28} Vengan a mí todos ustedes
que luchan y están cargados. Yo les daré descanso. \bibverse{29} Acepten
mi yugo, y aprendan de mí. Porque yo soy manso y tengo un corazón
humilde, y en mí encontrarán el descanso que necesitan. \bibverse{30}
Pues mi yugo es suave, y mi carga es ligera.''

\hypertarget{section-11}{%
\section{12}\label{section-11}}

\bibverse{1} En esos días, Jesús caminaba por los campos de grano en el
día Sábado. Sus discípulos tenían hambre, así que comenzaron a recoger
espigas y a comérselas. \bibverse{2} Cuando los Fariseos vieron esto, le
dijeron a Jesús: ``¡Mira a tus discípulos---están haciendo lo que no se
debe hacer en Sábado!''

\bibverse{3} Pero Jesús les dijo: ``¿No han leído lo que hizo David
cuando él y sus hombres tuvieron hambre? \bibverse{4} Él entró a la casa
de Dios, y allí él y sus hombres comieron del pan sagrado que no debían
comer pues este pan estaba reservado solo para los sacerdotes.
\bibverse{5} ¿No han leído en la ley que los sacerdotes que están en el
templo quebrantan el sábado pero no son considerados como culpables?
\bibverse{6} Sin embargo yo les digo a ustedes: ¡Aquí hay alguien que es
aún más grande que el templo! \bibverse{7} Si ustedes conocieran el
significado de lo que dice la Escritura: `misericordia quiero y no
sacrificio,'\footnote{\textbf{12:7} Oseas 6:6.} no habrían condenado a
un hombre inocente. \bibverse{8} Porque el Hijo del hombre es Señor del
Sábado.''

\bibverse{9} Entonces Jesús se fue de allí y entró a la sinagoga de
ellos. \bibverse{10} Allí había un hombre que tenía la mano tullida.
``¿Acaso permite la ley sanar en Sábado?'' le preguntaron ellos,
buscando así un motivo para acusarlo.

\bibverse{11} ``Supongan que tienen una oveja y ésta se cae en un hueco,
en Sábado. ¿Acaso no la agarran y tratan de sacarla?'' les preguntó
Jesús. \bibverse{12} ``¿No creen ustedes que un ser humano vale mucho
más que una oveja? De modo que sí, es permitido hacer el bien en
Sábado.'' \bibverse{13} Entonces le dijo al hombre: ``Extiende tu
mano.'' El hombre entonces extendió su mano y fue sanada, quedando como
la otra mano que estaba sana.

\bibverse{14} Pero los Fariseos salieron y conspiraban sobre cómo matar
a Jesús. \bibverse{15} Sabiendo esto, Jesús salió de allí, con una
multitud que le seguía. Y Jesús los sanaba a todos, \bibverse{16} pero
les decía que no dijeran quién era él. \bibverse{17} Esto cumplía lo que
dijo el profeta Isaías:

\bibverse{18} ``Este es mi siervo a quien Yo he escogido,

mi siervo a quien amo, el cual me complace.

Yo pondré mi Espíritu sobre él,

y él le dirá a los extranjeros lo que es correcto.

\bibverse{19} Él no peleará, no gritará,

y ninguno oirá su voz por las calles.

\bibverse{20} Él no quebrará ni una caña dañada,

y no apagará una mecha que titila,

hasta que haya demostrado que su juicio es justo\footnote{\textbf{12:20}
  O ``haya dado la victoria a la justicia.''},

\bibverse{21} y los gentiles pondrán su confianza en él.\footnote{\textbf{12:21}
  Literalmente, ``esperanza en su nombre.''}''

\bibverse{22} Entonces trajeron delante de Jesús a un hombre que estaba
endemoniado, ciego y mudo. Jesús lo sanó, y el hombre mudo pudo hablar y
ver. \bibverse{23} Todas las multitudes estaban asombradas, y
preguntaban, ``¿Será que este es el hijo de David?''\footnote{\textbf{12:23}
  Queriendo decir, el Mesías que vendría.}

\bibverse{24} Pero cuando los Fariseos escucharon esto, respondieron:
``¡Este hombre solo puede echar fuera demonios mediante el poder de
Belcebú, el jefe de los demonios!''

\bibverse{25} Pero sabiendo lo que ellos estaban pensando, Jesús les
dijo: ``Cualquier reino que está dividido contra sí mismo, será
destruido. Ninguna ciudad que está dividida contra sí misma puede
permanecer. \bibverse{26} Si Satanás echa fuera a Satanás, entonces está
dividido contra sí mismo, ¿cómo podría entonces permanecer su reino?
\bibverse{27} Si yo estoy echando fuera los demonios en el nombre de
Belcebú, entonces, ¿en nombre de quién echan fuera demonios los
exorcistas de ustedes? ¡Ellos mismos son prueba de que ustedes están
equivocados! \bibverse{28} ¡Pero si yo echo fuera demonios mediante el
poder del Espíritu de Dios, entonces el reino de Dios ha venido a
ustedes!

\bibverse{29} ``¿Puede alguien entrar a la casa de un hombre fuerte y
robar sus pertenencias sin atarlo primero? Si haces esto, entonces
puedes robar todo lo que hay en su casa. \bibverse{30} Los que no están
conmigo, están contra mí, y los que no se reúnen conmigo hacen lo
contrario: están dispersos. \bibverse{31} Esa es la razón por la que les
digo que cada pecado y blasfemia que ustedes cometan será perdonada,
excepto la blasfemia contra el Espíritu Santo, la cual no será
perdonada. \bibverse{32} Aquellos que digan algo en contra del Hijo del
hombre serán perdonados, pero aquellos que digan algo contra el Espíritu
Santo no serán perdonados, ni en esta vida ni en la siguiente.
\bibverse{33} Un árbol bueno se conoce porque su fruto es bueno, y un
árbol malo se conoce porque su fruto es malo, pues un árbol se conoce
por su fruto. \bibverse{34} ¡Cría de víboras! ¿Cómo pueden ustedes decir
algo bueno siendo malos? Pues la boca de ustedes solo dice lo que pasa
por sus mentes. \bibverse{35} Una buena persona saca cosas buenas de las
cosas buenas que tiene guardadas, y una persona mala saca cosas malas de
las cosas malas que tiene guardadas. \bibverse{36} Yo les digo, ustedes
tendrán que dar cuenta en el Día del Juicio de cada cosa que hayan dicho
de manera descuidada. \bibverse{37} Porque lo que ustedes digan los
vindicará o los condenará.''

\bibverse{38} Entonces algunos de los maestros religiosos y Fariseos que
estaban allí le dijeron: ``Maestro, queremos que nos muestres una señal
milagrosa.''

\bibverse{39} ``Las personas malvadas que no creen en Dios son las que
buscan una señal milagrosa. A esas personas no se les dará ninguna señal
sino la señal del profeta Jonás,'' les dijo Jesús. \bibverse{40} ``De la
misma manera que Jonás estuvo en el vientre de un gran pez durante tres
días y tres noches, el Hijo del hombre estará en el corazón de la tierra
por tres días y tres noches. \bibverse{41} El pueblo de Nínive se
levantará en el juicio junto con esta generación y la condenarán, porque
ellos se arrepintieron como respuesta al mensaje de Jonás--- ¡Y como
pueden ver, aquí hay alguien más grande que Jonás! \bibverse{42} La
reina del Sur se levantará en el juicio junto con esta generación y la
condenará, porque ella vino desde los fines de la tierra para escuchar
la sabiduría de Salomón--- ¡Y como pueden ver, aquí hay alguien más
grande que Salomón! \bibverse{43} Cuando un espíritu maligno sale de una
persona, anda por lugares desiertos buscando descanso, y no encuentra
dónde quedarse. \bibverse{44} Entonces dice: `regresaré al lugar de
donde salí,' y cuando regresa, encuentra el lugar vacío, limpio y
organizado. \bibverse{45} Entonces va y trae consigo otros siete
espíritus mucho peores que él, y entra y vive allí. De modo que entonces
la persona termina siendo peor de lo que era al comienzo. Así sucederá
con esta generación malvada.''

\bibverse{46} Mientras Jesús hablaba a las multitudes, su madre y sus
hermanos llegaron y lo esperaban fuera, y querían hablar con él.
\bibverse{47} Entonces alguien vino y le dijo: ``mira, tu madre y tus
hermanos están afuera y quieren hablar contigo.''

\bibverse{48} ``¿Quién es mi madre? ¿Quiénes son mis hermanos?''
preguntó Jesús. \bibverse{49} Entonces Jesús señaló a sus discípulos y
dijo: ``¡Miren, ellos son mi madre y mis hermanos! \bibverse{50} Porque
los que hacen la voluntad de mi Padre celestial, ¡ellos son mi hermano,
mi hermana y mi madre!''

\hypertarget{section-12}{%
\section{13}\label{section-12}}

\bibverse{1} Más tarde, ese día, Jesús se fue de la casa y se sentó a
enseñar\footnote{\textbf{13:1} Está implícito. Los maestros religiosos
  se sentaban cuando querían instruir a sus discípulos.} junto al lago.
\bibverse{2} Pero se reunieron a su alrededor tantas personas, que tuvo
que subirse a una barca y allí se sentó a enseñar, mientras que todas
las multitudes se quedaron de pie en la playa. \bibverse{3} Él les
enseñaba muchas cosas, usando relatos para ilustrarlas\footnote{\textbf{13:3}
  ``Relatos en forma de ilustraciones,'' literalmente, ``parábolas.''}.

``El sembrador salió a sembrar,'' comenzó él. \bibverse{4} ``Mientras
sembraba, algunas de las semillas cayeron por el camino. Entonces las
aves vinieron y se las comieron. \bibverse{5} Otras semillas cayeron en
suelo rocoso y porque no habia mucha tierra, germinaron pronto.''
\bibverse{6} El sol salió y las chamuscó y se murieron porque no tenían
raíces. \bibverse{7} Otras semillas cayeron entre espinos que crecieron
y las sofocaron. \bibverse{8} No obstante, otras semillas cayeron en
buen suelo. Esas semillas produjeron una cosecha---algunas cien, otras
sesenta, y otras treinta veces lo que se había plantado. \bibverse{9}
¡Todo el que tenga oídos, escuche!

\bibverse{10} Los discípulos vinieron a Jesús y le preguntaron, ``¿Por
qué usas ilustraciones cuando hablas a la gente?''

\bibverse{11} ``Ustedes son privilegiados porque a ustedes se les han
revelado los misterios del reino de los cielos, pero ellos no tienen ese
conocimiento,'' respondió Jesús. \bibverse{12} ``Aquellos que ya
tienen\footnote{\textbf{13:12} Probablemente queriendo decir que
  ``tienen entendimiento.''} recibirán más, más que suficiente. Pero
aquellos que no tienen, lo que lleguen a tener se les quitará.
\bibverse{13} Esa es la razón por la que les hablo a ellos a través de
ilustraciones. Porque aunque ellos pueden ver, no ven; y aunque pueden
oír, no oyen; ni entienden tampoco.

\bibverse{14} ``La profecía de Isaías se cumple en ellos: `aunque
ustedes oigan, no entenderán, y aunque vean, no percibirán.
\bibverse{15} Ellos tienen un corazón duro, no quieren escuchar y han
cerrado sus ojos. Si no fuera así, entonces podrían ver con sus ojos,
oír con sus oídos y entender con sus mentes. Entonces podrían regresar a
mí y yo los sanaría.'\footnote{\textbf{13:15} Isaías 6:9, 10.}

\bibverse{16} ``Benditos los ojos de ustedes, porque pueden ver. También
sus oídos, porque pueden oír. \bibverse{17} Les digo que muchos profetas
y personas buenas anhelaron ver lo que ustedes están viendo ahora, pero
no lo vieron. Ellos anhelaban escuchar lo que ustedes están escuchando,
pero no lo escucharon.

\bibverse{18} ``Así que escuchen el relato del sembrador: \bibverse{19}
Cuando las personas oyen el mensaje del reino, y no lo entienden, el
maligno viene y arranca lo que fue sembrado en sus corazones. Esto es lo
que ocurre con las semillas que cayeron en el camino. \bibverse{20} Las
semillas sembradas en el suelo rocoso son las personas que escuchan el
mensaje e inmediatamente lo aceptan con alegría. \bibverse{21} De esta
manera permanecen por un tiempo, pero como no tienen raíces, cuando los
problemas llegan, se apartan rápidamente. \bibverse{22} Las semillas que
fueron sembradas entre los espinos son las personas que escuchan el
mensaje, pero luego las preocupaciones de la vida y la tentación por el
dinero ahogan el mensaje y éste no produce fruto. \bibverse{23} Las
semillas sembradas en buen suelo son las personas que escuchan el
mensaje, lo entienden, y producen buena cosecha---algunos cien, otros
sesenta, y otros treinta veces lo que fue sembrado.''

\bibverse{24} Entonces les contó otro relato ilustrado: ``El reino de
los cielos es como un granjero que sembró buena semilla en su campo.
\bibverse{25} Pero mientras sus trabajadores dormían, llegó un enemigo y
sembró maleza\footnote{\textbf{13:25} De hecho, se refiere a ``cizaña,''
  o ``trigo falso,'' una maleza que se parecía mucho al trigo.} encima
del trigo. Y se fueron. \bibverse{26} Cuando el trigo creció y produjo
espigas, la maleza también creció. \bibverse{27} Los trabajadores del
granjero vinieron a preguntarle: `Señor, ¿acaso no sembraste buena
semilla en tu campo? ¿De dónde salió esta maleza?'

\bibverse{28} ``\,`Algún enemigo hizo esto,' respondió él. `¿Quieres que
vayamos y arranquemos la maleza?' le preguntaron. \bibverse{29} `No,'
respondió él, `al arrancar la maleza podrían arrancar de raíz el trigo
también. \bibverse{30} Dejen que crezcan juntos hasta la cosecha, y
entonces le diré a los segadores: ``reúnan primero la maleza, átenla en
bultos y quémenlos. Luego reúnan el trigo y almacénenlo en mi
granero.''\,'\,''

\bibverse{31} Les dio otra ilustración: ``El reino de los cielos es como
una semilla de mostaza que sembró un granjero en su campo. \bibverse{32}
Aunque es la semilla más pequeña de todas, ésta crece y llega a ser
mucho más grande que las otras plantas. De hecho, se convierte en un
árbol tan grande, que las aves pueden posarse en sus ramas.''

\bibverse{33} Y les contó otro relato ilustrado: ``El reino de los
cielos es como la levadura que una mujer mezcló con una gran cantidad
de\footnote{\textbf{13:33} Aproximadamente, 50 libras, o 23 kilogramos.}
harina, hasta que toda la masa creció.'' \bibverse{34} Y Jesús le
enseñaba todas estas cosas a las multitudes por medio de relatos
ilustrados---de hecho, él no les hablaba sin usar relatos. \bibverse{35}
Esto cumplía las palabras del profeta: ``Hablaré por medio de relatos, y
enseñaré cosas ocultas desde la creación del mundo.''

\bibverse{36} Jesús se fue de donde estaba la multitude a una casa. Sus
discípulos vinieron donde él estaba y le dijeron: ``Por favor,
explícanos el relato de la maleza en el campo.''

\bibverse{37} ``El que siembra la buena semilla es el Hijo del hombre,''
les explicó Jesús. \bibverse{38} ``El campo es el mundo. Las semillas
buenas son los hijos del reino. Las semillas de maleza son los hijos del
maligno. \bibverse{39} El enemigo que sembró las semillas de maleza es
el diablo. La cosecha es el fin del mundo. Los segadores son ángeles.
\bibverse{40} Así como la maleza se recoge y se quema, así será en el
fin del mundo. \bibverse{41} El Hijo del hombre enviará ángeles, y ellos
recogerán todo lo que es pecaminoso y a todos los que hacen el mal,
\bibverse{42} y los lanzarán en el horno abrasador, donde habrá llanto y
crujir de dientes. \bibverse{43} Entonces aquellos que viven justamente
brillarán como el sol en el reino de su padre. ¡Todo el que tiene oídos,
oiga!

\bibverse{44} ``El reino de los cielos es como un tesoro escondido en un
campo. Un hombre lo encontró, lo volvió a enterrar, y lleno de alegría
se fue y vendió todo lo que tenía y entonces compró ese campo.
\bibverse{45} El reino de los cielos es también como un mercader que
busca perlas preciosas. \bibverse{46} Cuando encontró la perla más
costosa que alguna vez conociera, se fue y vendió todo lo que tenía y la
compró. \bibverse{47} Una vez más, el reino de los cielos es como una
red de pescar que fue lanzada al mar y atrapó todo tipo de peces.
\bibverse{48} Cuando estaba llena, fue sacada a la orilla. Los buenos
peces fueron colocados en las canastas, mientras que los malos peces
fueron echados a la basura.

\bibverse{49} ``Así serán las cosas cuando llegue el fin del mundo. Los
ángeles saldrán y separarán a las personas malas de las personas buenas,
\bibverse{50} y las lanzarán en el horno abrasador, donde habrá llanto y
crujir de dientes.

\bibverse{51} ``¿Ahora lo entienden todo?'' ``Sí,'' respondieron ellos.

\bibverse{52} ``Todo maestro religioso que haya aprendido acerca del
reino de los cielos es como el propietario de una casa que saca de su
despensa tesoros nuevos y viejos,'' respondió Jesús.

\bibverse{53} Después que Jesús terminó de contar estos relatos, se fue
de allí. \bibverse{54} Entonces regresó a la ciudad donde se había
criado\footnote{\textbf{13:54} Nazaret.} y allí enseñaba en la sinagoga.
Las personas estaban asombradas, y preguntaban: ``¿De dónde obtiene su
sabiduría y sus milagros? \bibverse{55} ¿No es este el hijo del
carpintero? ¿No es este el hijo de María, y hermano de Santiago, José,
Simón y Judas? \bibverse{56} ¿No viven sus hermanas entre nosotros? ¿De
dónde, entonces recibe todo esto?'' \bibverse{57} Y por esta razón se
negaban a creer en él.

``Un profeta es honrado en todas partes, excepto en su propia tierra y
entre su familia,'' les dijo Jesús. \bibverse{58} Como ellos no lograron
creer en él, Jesús no hizo muchos milagros allí.

\hypertarget{section-13}{%
\section{14}\label{section-13}}

\bibverse{1} En ese tiempo, Herodes el tetrarca\footnote{\textbf{14:1}
  ``Tetrarca'' quiere decir que era gobernante de una cuarta parte. En
  este caso, de la región de Galilea.} escuchó lo que Jesús hacía
\bibverse{2} y le dijo a sus siervos: ``¡Él debe ser Juan el Bautista
que resucitó de entre los muertos, y por eso tiene tales poderes!''
\bibverse{3} Herodes había arrestado a Juan, lo había encadenado y lo
había puesto en prisión por petición de Herodías, la esposa de Felipe,
su hermano. \bibverse{4} Esto lo hicieron porque Juan le había dicho:
``No es legal que te cases con ella.'' \bibverse{5} Herodes quería matar
a Juan pero tenía miedo de la reacción del pueblo, pues ellos
consideraban que él era un profeta.

\bibverse{6} Sin embargo, el día del cumpleaños de Herodes, la hija de
Herodías\footnote{\textbf{14:6} Comúnmente se le identifica como Salomé.}
danzó en la fiesta, y Herodes estaba contento con ella. \bibverse{7} Así
que le prometió con juramento darle cualquier cosa que ella deseara.
\bibverse{8} Impulsada por su madre, Herodías dijo: ``Dame aquí en un
plato la cabeza de Juan el Bautista.'' \bibverse{9} Entonces el rey se
arrepintió de la promesa que había hecho, pero por los juramentos que
había hecho frente a todos los invitados a su cena, dio la orden de
hacerlo. \bibverse{10} La orden fue enviada y Juan fue decapitado en la
cárcel. \bibverse{11} Trajeron la cabeza de Juan en un plato y le fue
entregado a la joven, quien lo entregó a su madre. \bibverse{12}
Entonces los discípulos de Juan vinieron y se llevaron el cuerpo y lo
sepultaron. Luego fueron a decírselo a Jesús.

\bibverse{13} Cuando Jesús escuchó la noticia, se fue lejos en una barca
a un lugar tranquilo para estar solo, pero cuando la multitud supo dónde
estaba, lo siguieron a pie desde las ciudades. \bibverse{14} Cuando
Jesús salió de la barca y vio a la gran multitud, se llenó de simpatía
por ellos, y sanó a los enfermos que había entre ellos. \bibverse{15} Al
llegar la noche, los discípulos se le acercaron y le dijeron, ``Este
lugar está a millas de distancia de cualquier parte y se está haciendo
tarde. Despide la multitud para que puedan irse a las aldeas y comprar
comida para ellos.''

\bibverse{16} Pero Jesús les dijo: ``Ellos no necesitan irse. ¡Denles
ustedes de comer!''

\bibverse{17} ``Lo único que tenemos son cinco panes y un par de
peces,'' respondieron ellos.

\bibverse{18} ``Tráiganmelos,'' dijo Jesús. \bibverse{19} Entonces les
dijo a las multitudes que se sentaran en la hierba. Luego tomó los cinco
panes y los dos peces, miró al cielo y los bendijo. Después de esto,
partió los panes en pedazos y dio el pan a los discípulos, y los
discípulos lo daban a las multitudes. \bibverse{20} Todos comieron hasta
que quedaron saciados. Entonces los discípulos recogieron las sobras y
llenaron doce canastas. \bibverse{21} Aproximadamente cinco mil hombres
comieron de aquella comida, sin contar las mujeres y los niños.

\bibverse{22} Justo después de esto, Jesús llamó a los discípulos a que
subieran a la barca para cruzar al otro lado del lago, mientras despedía
a la multitud. \bibverse{23} Después que los despidió a todos, subió a
la montaña para orar. Llegó la noche y él estaba allí solo.
\bibverse{24} En ese momento, ya la barca estaba lejos del suelo firme,
las olas la arrastraban porque el viento soplaba contra ella.
\bibverse{25} Cerca de las 3 a.m.\footnote{\textbf{14:25} Literalmente,
  ``la cuarta vigilia de la noche.''} Jesús los alcanzó, caminando sobre
el mar. \bibverse{26} Cuando los discípulos lo vieron caminando sobre el
mar, se asustaron. Entonces gritaron con terror: ``¡Es un fantasma!''

\bibverse{27} Pero inmediatamente Jesús les dijo: ``¡No se preocupen,
soy yo! ¡No tengan miedo!''

\bibverse{28} ``Señor, si eres tú realmente, haz que yo llegue donde tu
estás, caminando también sobre el agua,'' respondió Pedro.

\bibverse{29} ``Ven,'' le dijo Jesús. Entonces Pedro salió de la barca y
caminó sobre el agua hacia Jesús. \bibverse{30} Pero cuando vio cuán
fuerte soplaba el viento, se asustó y comenzó a hundirse. ``¡Señor!
¡Sálvame!'' gritaba. \bibverse{31} De inmediato Jesús se extendió y lo
tomó, y le dijo: ``Tienes tan poca confianza en mi. ¿Por qué dudaste?''
\bibverse{32} Y cuando entraron a la barca, el viento dejó de soplar.
\bibverse{33} Y los que estaban en la barca lo adoraban, diciendo:
``¡Realmente eres el Hijo de Dios!''

\bibverse{34} Después de cruzar el lago, llegaron a Genesaret.
\bibverse{35} Cuando la gente de allí se dio cuenta de que era Jesús, lo
hicieron saber a todos en la región. Entonces trajeron ante Jesús a
todos los que estaban enfermos, \bibverse{36} y le imploraban que dejara
que los enfermos tan solo tocasen su manto. Todos los que lo tocaban
eran sanados.

\hypertarget{section-14}{%
\section{15}\label{section-14}}

\bibverse{1} Entonces algunos Fariseos y maestros religiosos de
Jerusalén vinieron a Jesús y le preguntaron: \bibverse{2} ``¿Por qué tus
discípulos quebrantan la tradición de nuestros antepasados al no lavar
sus manos antes de comer?''

\bibverse{3} ``¿Por qué ustedes quebrantan el mandamiento por causa de
su tradición?'' respondió Jesús. \bibverse{4} ``Pues Dios dijo: `Honra a
tu padre y a tu madre,' y `Cualquiera que maldice a su padre o a su
madre debe ser condenado a muerte.' \bibverse{5} Pero ustedes dicen que
si alguno le dice su padre o a su madre `todo lo que yo deba darles a
ustedes ahora lo doy como ofrenda a Dios,' entonces \bibverse{6} no
tiene que honrar a su padre. De esta manera ustedes han anulado la
palabra de Dios por causa de sus tradiciones. \bibverse{7} ¡Ustedes son
unos hipócritas! Bien los describió Isaías cuando profetizó:
\bibverse{8} `Este pueblo dice que me honra pero en sus mentes no hay
interés hacia mí.\footnote{\textbf{15:8} O, ``Esas personas me honran
  con sus labios, pero sus corazones están lejos de mi.''} \bibverse{9}
Su adoración hacia mi es inútil. Lo que enseñan son solo exigencias
humanas.'\,''

\bibverse{10} Entonces Jesús llamó a la multitud y les dijo: ``Escuchen
y entiendan esto: \bibverse{11} No es lo que entra por la boca lo que
los contamina, sino lo que sale de ella.''

\bibverse{12} Entonces los discípulos de Jesús vinieron a él y le
dijeron: ``Ciertamente te das cuenta de que los Fariseos se ofendieron
por lo que dijiste.''

\bibverse{13} ``Toda planta que no haya sembrado mi Padre será
arrancada,'' respondió Jesús. \bibverse{14} ``Olvídense de ellos---ellos
son guías ciegos\footnote{\textbf{15:14} Refiriéndose a los Fariseos.}.
Si un hombre ciego guía a otro hombre ciego, los dos caerán en una
zanja.''

\bibverse{15} Entonces Pedro dijo: ``Por favor, dinos lo que quieres
decir con esta ilustración.''

\bibverse{16} ``¿Aún no lo han entendido?'' respondió Jesús.
\bibverse{17} ``¿No ven que todo lo que entra a la boca pasa por el
estómago y luego sale del cuerpo como un desperdicio\footnote{\textbf{15:17}
  Literalmente, ``botadas en el alcantarillado.''}? \bibverse{18} Pero
lo que sale de la boca viene de la mente, y eso es lo que los contamina.
\bibverse{19} Porque lo que sale de la mente son pensamientos malos,
asesinatos, adulterio, inmoralidad sexual, hurto, falso testimonio, y
blasfemia, \bibverse{20} y esas son las cosas que los contaminan a
ustedes. Comer sin lavarse las manos no los contamina.''

\bibverse{21} Jesús se fue de allí y se dirigió hacia la región de Tiro
y Sidón. \bibverse{22} Una mujer cananea de ese lugar vino gritando:
``¡Señor, Hijo de David! ¡Por favor, ten misericordia de mi, pues mi
hija sufre grandemente porque está poseída por un demonio!''
\bibverse{23} Pero Jesús no respondió en absoluto. Sus discípulos
vinieron y le dijeron: ``Dile que deje de seguirnos. ¡Sus gritos son muy
molestos!''

\bibverse{24} ``Yo fui enviado únicamente a las ovejas perdidas de la
casa de Israel,'' le dijo Jesús a la mujer. \bibverse{25} Pero la mujer
vino y se arrodilló delante de él, y le dijo: ``¡Señor, por favor,
ayúdame!''

\bibverse{26} ``No es correcto tomar el alimento de los hijos para
dárselo a los perros\footnote{\textbf{15:26} La palabra usada para
  ``perros'' aquí se refiere a perros domésticos, o cachorros.},'' le
dijo Jesús.

\bibverse{27} ``Sí, Señor, pero aun así, a los perros se les deja comer
las migajas que caen de la mesa de su amo,'' respondió ella.

\bibverse{28} ``Tu confías en mí grandemente,'' le respondió Jesús.
``¡Tu deseo está concedido!'' Y su hija fue sanada de inmediato.

\bibverse{29} Entonces Jesús regresó, pasando por el mar de Galilea. Se
fue hacia las montañas cercanas y allí se sentó. \bibverse{30} Grandes
multitudes vinieron a él, trayéndole a aquellos que estaban cojos,
ciegos, paralíticos, mudos y también muchos otros que estaban enfermos.
Los ponían en el piso, a sus pies, y él los sanaba. \bibverse{31} La
multitud estaba asombrada ante lo que ocurría: los sordos podían hablar,
los paralíticos eran sanados, los cojos podían caminar, y los ciegos
podían ver. Y alababan al Dios de Israel.

\bibverse{32} Entonces Jesús llamó a sus discípulos y les dijo: ``Siento
pesar por estas personas, porque han estado conmigo por tres días y no
tienen nada que comer. No quiero que se vayan con hambre, no sea que se
desmayen por el camino.''

\bibverse{33} ¿Dónde podríamos encontrar suficiente pan en este desierto
para alimentar a semejante multitud tan grande?'' respondieron los
discípulos.

\bibverse{34} ``¿Cuántos panes tienen ustedes allí?'' preguntó Jesús.

``Siete, y unos cuantos peces pequeños,'' respondieron ellos.

\bibverse{35} Jesús dijo a la multitud que se sentara en el suelo.
\bibverse{36} Entonces tomó los siete panes y los peces, y después de
bendecir la comida, la partió en trozos y la dio a los discípulos, y los
discípulos la daban a la multitud. \bibverse{37} Todos comieron hasta
que estuvieron saciados, y entonces recogieron las sobras, llenando así
siete canastas. \bibverse{38} Cuatro mil hombres comieron de esta
comida, sin contar a las mujeres y a los niños. \bibverse{39} Entonces
Jesús despidió a la multitud, subió a la barca, y se fue a la región de
Magadán.

\hypertarget{section-15}{%
\section{16}\label{section-15}}

\bibverse{1} Los Fariseos y los Saduceos vinieron para probar\footnote{\textbf{16:1}
  Puesto que la prueba era una tentativa para desacreditar a Jesús, esto
  también podría traducirse como ``vinieron a ponerle una trampa a
  Jesús.''} a Jesús, exigiéndole que les mostrara una señal del cielo.

\bibverse{2} Jesús les dijo: ``Por la noche, ustedes dicen, `mañana
habrá buen tiempo, porque el cielo se ve rojo,' \bibverse{3} pero por la
mañana dicen: `habrá mal tiempo hoy, porque el cielo está rojo y
nublado.' ¡Ustedes saben predecir el clima por cómo se ve el cielo, pero
no son capaces de reconocer las señales de los tiempos! \bibverse{4} La
gente mala que no confía en Dios es la que espera una señal milagrosa, y
a esas personas no se les dará ninguna señal excepto la señal de
Jonás.'' Y entonces se fue de allí.

\bibverse{5} Cuando iban hacia el otro lado del lago, los discípulos
olvidaron llevar pan. \bibverse{6} ``Cuídense de la levadura de los
Fariseos y los Saduceos,'' les dijo Jesús.

\bibverse{7} Los discípulos comenzaron a discutir entre ellos. ``Está
diciendo eso\footnote{\textbf{16:7} Está implícito en el texto.} porque
no trajimos pan,'' concluyeron. \bibverse{8} Pero Jesús sabía lo que
ellos estaban diciendo y les dijo: ``¡Ustedes confían muy poco en mi!
¿Por qué están discutiendo entre ustedes por no tener pan? \bibverse{9}
¿Acaso aún no lo han entendido? ¿No recuerdan los cinco panes que
alimentaron cinco mil personas? ¿Cuántas canastas sobraron?
\bibverse{10} ¿Y qué hay de los siete panes que alimentaron a los cuatro
mil? ¿Cuántas canastas sobraron? \bibverse{11} ¿No se han dado cuenta
aún de que yo no hablaba sobre el pan? ¡Cuídense de la levadura de los
Fariseos y los Saduceos!'' \bibverse{12} Entonces se dieron cuenta de
que él no les estaba advirtiendo sobre levadura de pan, sino sobre las
enseñanzas de los Fariseos y los Saduceos.

\bibverse{13} Cuando llegó a la región de Cesarea de Filipo, Jesús le
preguntó a sus discípulos: ``¿Quién dice la gente que es el Hijo del
hombre?''

\bibverse{14} ``Algunos dicen que Juan el Bautista, otros dicen que
Elías, y otros dicen que Jeremías o uno de los otros profetas,''
respondieron ellos.

\bibverse{15} ``¿Y ustedes?'' preguntó él. ``¿Quién dicen ustedes que
soy yo?''

\bibverse{16} ``Tú eres el Mesías, el Hijo del Dios viviente,''
respondió Simón Pedro.

\bibverse{17} ``Verdaderamente eres bendito, Simón hijo de Juan,'' le
dijo Jesús. ``Porque esto no te fue revelado por carne ni sangre humana,
sino por mi Padre que está en el cielo. \bibverse{18} También te digo
que tú eres Pedro\footnote{\textbf{16:18} Pedro significa ``piedra,'' en
  contraste con la palabra que se usa para roca sólida en este
  versículo.}, y sobre esta roca edificaré mi iglesia, y los poderes de
la muerte\footnote{\textbf{16:18} Literalmente, ``las puertas del
  Hades.''} no la destruirán. \bibverse{19} Te daré las llaves del reino
de los cielos, y todo lo que prohíbas en la tierra, será prohibido en
los cielos; y todo lo que permitas en la tierra, será permitido en los
cielos.'' \bibverse{20} Entonces le advirtió a sus discípulos de no
decirle a nadie que él era el Mesías.

\bibverse{21} A partir de entonces Jesús comenzó a explicarle a sus
discípulos que él tendría que ir a Jerusalén, y que sufriría
terriblemente en manos de los ancianos, de los jefes de los sacerdotes y
de los maestros religiosos, y que lo matarían, pero que él se levantaría
otra vez al tercer día.

\bibverse{22} Pedro levó a Jesús con él aparte y comenzó a decirle que
no era bueno que hablara así. ``¡Dios no permita, Señor, que algo así
llegue a ocurrirte!'' le dijo.

\bibverse{23} Jesús se volvió hacia Pedro y le dijo: ``¡Aléjate de mi,
Satanás! ¡Eres una trampa para hacerme tropezar\footnote{\textbf{16:23}
  Literalmente, ``una piedra de tropiezo'' o una ``trampa.''}, porque
estás pensando humanamente, y no como Dios piensa!''

\bibverse{24} Entonces Jesús le dijo a sus discípulos: ``El que quiera
seguirme, debe negarse a sí mismo, tomar su cruz y seguirme.
\bibverse{25} Porque el que quiera salvar su vida, la perderá, y el que
pierda la vida por mi causa, la ganará. \bibverse{26} ¿Qué beneficio
tiene ganar el mundo entero y perder la vida? ¿Qué darán ustedes a
cambio de su vida? \bibverse{27} Porque el Hijo del hombre vendrá en la
gloria de su Padre, junto con sus ángeles. Entonces le dará a cada uno
lo que merece conforme a lo que haya hecho. \bibverse{28} Les digo la
verdad: hay algunos aquí que no morirán\footnote{\textbf{16:28}
  Literalmente, ``probarán la muerte.''} antes de que vean al Hijo del
hombre venir en su reino.''

\hypertarget{section-16}{%
\section{17}\label{section-16}}

\bibverse{1} Seis días después Jesús llevó consigo a Pedro, a Santiago,
y a su hermano Juan hacia una montaña alta para estar solos allí.
\bibverse{2} Entonces Jesús se transformó frente a ellos. Su rostro
brillaba como el sol. Sus vestiduras se volvieron blancas como la luz.
\bibverse{3} De repente, aparecieron Moisés y Elías delante de ellos, y
estos dos estaban hablando con Jesús.

\bibverse{4} Pedro los interrumpió\footnote{\textbf{17:4} Está
  implícito. En el original dice: ``Pero respondiendo, Pedro dijo.''}
diciéndole a Jesús: ``Señor, qué bien se siente estar aquí. Si tú
quieres haré tres enramadas---una para ti, una para Moisés, y una para
Elías.''

\bibverse{5} Mientras Pedro aún hablaba, una nube brillante los cubrió.
Entonces se escuchó una voz que salía desde la nube, que decía: ``Este
es mi hijo a quien amo, el cual me complace. Escúchenlo.'' \bibverse{6}
Cuando oyeron esto, los discípulos cayeron sobre sus rostros,
completamente aterrorizados. \bibverse{7} Jesús se acercó a ellos y los
tocó. ``Levántense,'' les dijo. ``No tengan miedo.'' \bibverse{8} Cuando
levantaron la vista, no vieron a nadie más allí, excepto a Jesús.

\bibverse{9} Cuando descendieron de la montaña, Jesús les dio
instrucciones precisas: ``No le digan a nadie lo que vieron hasta que el
Hijo del hombre se haya levantado de entre los muertos,'' les dijo.

\bibverse{10} ``¿Por qué, entonces, los maestros religiosos dicen que
Elías debe venir primero?'' preguntaron sus discípulos.

\bibverse{11} ``Es cierto que Elías viene a poner cada cosa en su lugar,
\bibverse{12} pero déjenme decirles que Elías ya vino y sin embargo la
gente no reconoció quién era él. Hicieron con él todo lo que quisieron.
De la misma manera, el Hijo del hombre también sufrirá en manos de
ellos.'' \bibverse{13} Entonces los discípulos se dieron cuenta de que
Jesús se estaba refiriendo a Juan el Bautista.

\bibverse{14} Cuando se aproximaban a la multitud, un hombre llegó y se
arrodilló delante de Jesús. \bibverse{15} ``Señor, por favor, ten
misericordia de mi hijo,'' le dijo. ``Él se vuelve loco\footnote{\textbf{17:15}
  Literalmente, ``que está loco.'' Este término es paralelo al término
  ``lunático,'' del latín ``lunaticus.''} y sufre ataques tan terribles
que a veces hasta se lanza al fuego o al agua. \bibverse{16} Lo traje
ante tus discípulos pero ellos no pudieron sanarlo.''

\bibverse{17} ``¡Este pueblo\footnote{\textbf{17:17} Literalmente,
  ``generación.''} se niega a confiar en mi, y todos están corruptos!''
respondió Jesús. ``¿Cuánto tiempo más tengo que permanecer aquí con
ustedes? ¿Cuánto tiempo más tendré que aguantarlos? ¡Tráiganmelo aquí!''
\bibverse{18} Jesús confrontó al demonio y éste salió del joven, y quedó
sano de inmediato.

\bibverse{19} Después de esto, los discípulos vinieron a Jesús en
privado y le preguntaron: ``¿Por qué nosotros no pudimos sacarlo?''

\bibverse{20} hol``Porque ustedes no creen lo suficiente,'' les dijo
Jesús. ``Les digo que aún si la confianza de ustedes fuera tan pequeña
como una semilla de mostaza, ustedes podrían decir a esta montaña
`muévete de aquí para allá,' y esta se movería. Nada sería imposible
para ustedes.'' \bibverse{21} \footnote{\textbf{17:21} El versículo 21
  no está en los primeros manuscritos.}

\bibverse{22} Mientras caminaban por Galilea, Jesús les dijo: ``El Hijo
del hombre será traicionado y la gente tendrá poder\footnote{\textbf{17:22}
  Literalmente, ``entregado en manos de hombres.''} sobre él.
\bibverse{23} Lo matarán, pero el tercer día, él se levantará de
nuevo.'' Los discípulos se entristecieron.

\bibverse{24} Cuando llegaron a Capernaúm, los que estaban encargados de
recolectar el impuesto de medio siclo en el templo, vinieron donde
estaba Pedro y le preguntaron: ``Tu maestro paga el medio siclo, ¿no es
así?''

\bibverse{25} ``Si, por supuesto,'' respondió Pedro.

Cuando regresó donde estaban todos, Jesús se anticipó al hecho. ``¿Qué
piensas tu, Simón?'' le preguntó Jesús. ``¿Acaso los reyes de este mundo
obtienen los impuestos de parte sus propios hijos o de parte de los
otros?''

\bibverse{26} ``De los otros,'' respondió Pedro. Entonces Jesús le dijo:
``En ese caso, los hijos están exentos. \bibverse{27} Pero para no
ofender a nadie, ve al lago y saca un pez con un anzuelo. Saca el primer
pez que atrapes, y cuando abras su boca encontrarás una moneda de
stater\footnote{\textbf{17:27} Equivalente a un siclo. El impuesto del
  templo en esa época era medio siclo por persona.}. Toma la moneda y
entrégaselas de parte tuya y mía.''

\hypertarget{section-17}{%
\section{18}\label{section-17}}

\bibverse{1} Por ese tiempo los discípulos vinieron a Jesús, y le
preguntaron: ``¿Quién es el más grande en el reino de los cielos?''

\bibverse{2} Jesús llamó a un niño pequeño. Puso al niño de pie frente a
ellos. \bibverse{3} ``Les digo la verdad: a menos que cambien su manera
de pensar y se vuelvan como niños pequeños, nunca entrarán en el reino
de los cielos. \bibverse{4} Pero cualquiera que se humilla y se vuelve
como este niño, ese es el más grande en el reino de los cielos.
\bibverse{5} Cualquiera que acepta a un niño como este en mi nombre, me
acepta a mí. \bibverse{6} Pero cualquiera que hace pecar a uno de estos
niños que cree en mí, sería mejor que atase a su cuello una piedra de
moler\footnote{\textbf{18:6} Literalmente, ``un molino de asno,''
  refiriéndose a los molinos que eran girados por un asno, y no a los
  molinos que se manejaban manualmente.} y se lance a las profundidades
del mar.

\bibverse{7} ``¡Cuán grande es el desastre que sobrevendrá en el mundo
por todas sus tentaciones a pecar! ¡Las tentaciones ciertamente vendrán,
pero será un desastre para la persona por quien viene la tentación!
\bibverse{8} Si tu mano o tu pie te hacen pecar, córtalo y bótalo. Es
mejor que entres a la vida eterna siendo paralítico o cojo, que tener
dos manos o dos pies y ser lanzado al fuego eterno. \bibverse{9} Si tu
ojo te hace pecar, sácalo y bótalo. Es mejor que entres a la vida eterna
con un solo ojo que tener dos ojos y ser lanzado al fuego del juicio.
\bibverse{10} Asegúrense de no menospreciar a estos pequeños. Yo les
digo que en el cielo sus ángeles siempre están con\footnote{\textbf{18:10}
  Literalmente, ``ven el rostro de.''} mi Padre celestial. \bibverse{11}
\footnote{\textbf{18:11} El versículo 11 no está en los primeros
  manuscritos.} \bibverse{12} ¿Qué piensan ustedes? Si un hombre tiene
cien ovejas y una de ellas se pierde, ¿acaso no dejará él las noventa y
nueve en la colina e irá en búsqueda de la que está perdida?
\bibverse{13} Y si la encuentra, yo les digo que ese hombre se regocija
más por esa oveja que por las noventa y nueve que no se perdieron.
\bibverse{14} De la misma manera, mi Padre celestial no quiere que
ninguno de estos pequeños se pierda.

\bibverse{15} ``Si un hermano\footnote{\textbf{18:15} O ``hermano en la
  fe.''} peca contra ti, ve y muéstrale el error a él, solo entre
ustedes dos. Si te escucha, habrás convencido a tu hermano.
\bibverse{16} Pero si no escucha, entonces lleva contigo a una o dos
personas, para que con dos o tres testigos pueda confirmarse la verdad.
\bibverse{17} Si aun así él se niega a escucharte, entonces dilo a la
iglesia. Si se niega a escuchar a la iglesia, entonces trátalo como a un
extranjero\footnote{\textbf{18:17} Literalmente, un ``gentil,'' un
  incrédulo.} y recaudador de impuestos. \bibverse{18} Les digo la
verdad: todo lo que prohíban en la tierra será prohibido en el cielo, y
todo lo que permitan en la tierra, será permitido en el cielo.

\bibverse{19} ``También les digo que si dos de ustedes se ponen de
acuerdo aquí en la tierra acerca de algo por lo que están orando,
entonces mi Padre celestial lo hará por ustedes. \bibverse{20} Porque
donde dos o tres se reúnen en mi nombre, allí estoy con ellos.''

\bibverse{21} Entonces Pedro vino donde estaba Jesús y le preguntó:
``¿Cuántas veces debo perdonar a mi hermano por pecar contra mi? ¿Siete
veces?''

\bibverse{22} ``No, siete veces no. ¡Yo diría hasta setenta veces
siete!'' le dijo Jesús. \bibverse{23} ``Por eso el reino de los cielos
es como un rey que quería saldar cuentas con los siervos que le debían
dinero. \bibverse{24} Cuando comenzó a saldar cuentas, fue presentado
delante de él un siervo que le debía diez mil talentos\footnote{\textbf{18:24}
  Una cantidad astronómica.}. \bibverse{25} Como este hombre no tenía
dinero para pagar, su amo dio la orden de venderlo, junto con su esposa,
sus hijos y todas sus posesiones para poder pagar la deuda.
\bibverse{26} El siervo se arrodilló y le dijo a su amo: `¡Por favor,
ten paciencia conmigo! ¡Yo lo pagaré todo!' \bibverse{27} El amo tuvo
misericordia del siervo, lo liberó y canceló la deuda. \bibverse{28}
Pero cuando ese mismo siervo salió de allí, se encontró con uno de sus
consiervos que le debía apenas cien denarios\footnote{\textbf{18:28} Un
  denario era una moneda pequeña. Se hace contraste entre la gran
  cantidad que se le perdonó al primer siervo y la pequeña cantidad que
  le debía a éste el segundo siervo.}. Lo tomó por el cuello y
ahorcándolo, le decía: `¡Págame lo que me debes!' \bibverse{29} Su
consiervo se lanzó a los pies de este hombre y le rogó: `¡Por favor, sé
paciente conmigo! ¡Yo te pagaré!' \bibverse{30} Pero el hombre se negó,
y fue y puso a su consiervo en prisión hasta que le pagase lo que le
debía.

\bibverse{31} ``Cuando los otros siervos vieron lo que había pasado, se
aturdieron y estaban molestos. Fueron a decirle a su amo todo lo que
había ocurrido. \bibverse{32} Entonces el amo volvió a llamar a aquél
hombre y le dijo: ¡Siervo malo! Te perdoné toda la deuda porque me
rogaste que te perdonara. \bibverse{33} ¿Acaso no deberías haber sido
misericordioso con tu consiervo también, así como yo lo fui contigo?'
\bibverse{34} Su amo se enojó y lo entregó a los carceleros hasta que
pagase toda la deuda. \bibverse{35} Esto es lo que mi Padre celestial
hará con cada uno de ustedes a menos que con sinceridad\footnote{\textbf{18:35}
  Literalmente, ``de corazón.''} ustedes perdonen a sus hermanos.''

\hypertarget{section-18}{%
\section{19}\label{section-18}}

\bibverse{1} Cuando Jesús terminó de hablar se fue de Galilea y se
dirigió a la región de Judea, al otro lado del Jordán. \bibverse{2}
Grandes multitudes le seguían, y él sanaba a los que allí estaban
enfermos.

\bibverse{3} Entonces ciertos Fariseos vinieron para probarlo. ``¿Se le
permite a un hombre divorciarse de su esposa por cualquier razón?'' le
preguntaron.

\bibverse{4} Jesús respondió: ``¿No han leído que Dios, quien creó a las
personas en el principio, los creó hombre y mujer? \bibverse{5} Entonces
dijo: 'Esta es la razón por la cual el hombre se irá de donde su padre y
su madre y se unirá a su esposa, y entonces los dos se convertirán en
uno\footnote{\textbf{19:5} Literalmente, ``una carne.''}. \bibverse{6}
Ahora no son dos, sino uno. Lo que Dios ha unido, nadie debe
separarlo.''

\bibverse{7} ``¿Entonces por qué Moisés entregó una ley que dice que un
hombre puede divorciarse de su esposa entregándole un certificado de
divorcio escrito y despidiéndola?'' le preguntaron.

\bibverse{8} ``Por la actitud insensible de ustedes, Moisés les permitió
divorciarse de sus esposas, pero no era así al comienzo,'' respondió
Jesús. \bibverse{9} ``Les digo que cualquiera que se divorcia de su
esposa -- a menos que sea por inmoralidad sexual --, y luego se casa con
otra mujer, comete adulterio.''

\bibverse{10} ``¡Si esa es la situación entre el esposo y la esposa, es
mejor no casarse!'' respondieron sus discípulos.

\bibverse{11} ``No cualquiera puede aceptar esta instrucción\footnote{\textbf{19:11}
  Literalmente, ``palabra.''}, solo aquellos a quienes se les da,'' les
dijo Jesús. \bibverse{12} ``Algunos nacen siendo eunucos, algunos se
vuelven eunucos porque otros hombres los hacen eunucos, y otros deciden
ser eunucos por causa del reino de los cielos. Los que aceptan hacerlo,
deben aceptar tal enseñanza.''

\bibverse{13} Entonces la gente traía niños pequeños delante de él para
que los bendijera y orara por ellos. Pero los discípulos les decían que
no lo hicieran.

\bibverse{14} Pero Jesús dijo: ``Dejen que los niños vengan a mi. No se
lo impidan. ¡El reino de los cielos pertenece a quienes son como
ellos!'' \bibverse{15} Entonces él puso sus manos sobre ellos para
bendecirlos y luego se fue.

\bibverse{16} Un hombre vino a Jesús y le dijo: ``Maestro, ¿qué cosas
buenas debo hacer para recibir vida eterna?''

\bibverse{17} ``¿Por qué me preguntas a mi lo que es bueno?'' respondió
Jesús. ``Solo hay uno que es bueno. Pero si quieres tener vida
eterna\footnote{\textbf{19:17} Literalmente, ``entrar a la vida.''},
entonces guarda los mandamientos.''

\bibverse{18} ``¿Cuáles?'' preguntó el hombre.

``No mates, no cometas adulterio, no robes, no des falso testimonio,
\bibverse{19} honra a tu padre y a tu madre, y ama a tu prójimo como a
ti mismo,'' respondió Jesús.

\bibverse{20} ``Yo he guardado todos estos mandamientos,'' dijo el
joven. ``¿Qué más debo hacer?''

\bibverse{21} Jesús le dijo: ``Si quieres ser perfecto\footnote{\textbf{19:21}
  ``Perfecto'' aquí conlleva la idea de algo realizado o completo.},
entonces ve y vende todas tus posesiones, da el dinero a los pobres, y
tendrás tesoro en el cielo. Entonces ven y sígueme.''

\bibverse{22} Cuando el joven escuchó la respuesta de Jesús, se fue muy
triste, porque tenía muchas posesiones.

\bibverse{23} ``Les digo la verdad,'' dijo Jesús a sus discípulos, ``a
la gente rica se le hace difícil entrar al reino de los cielos.
\bibverse{24} También les digo esto: es más fácil que un camello pase a
través del ojo de una aguja que un rico entre al reino de los cielos.''

\bibverse{25} Cuando los discípulos oyeron esto, se sorprendieron, y
preguntaron: ``¿Quién puede salvarse entonces?''

\bibverse{26} Jesús los miró y dijo: ``Desde un punto de vista humano,
es imposible, pero con Dios todas las cosas son posibles.''

\bibverse{27} Pedro le respondió: ``Mira, hemos dejado todo y te hemos
seguido. ¿Qué recompensa tendremos?''

\bibverse{28} Jesús respondió: ``Les digo la verdad: cuando todo sea
hecho de nuevo y el Hijo del hombre se siente en su trono glorioso,
ustedes que me han seguido también se sentarán en tronos, y serán jueces
de las doce tribus de Israel. \bibverse{29} Todos los que dejan su
hogar, a sus hermanos, a sus hermanas, a sus padres, a sus madres, a sus
hijos y sus campos por mi causa, recibirán cien veces más y recibirán la
vida eterna. \bibverse{30} Porque muchos que son los primeros serán
dejados de ultimo, y muchos que son los últimos, serán los primeros.

\hypertarget{section-19}{%
\section{20}\label{section-19}}

\bibverse{1} ``Porque el reino de los cielos es como un terrateniente
que salió temprano por la mañana para contratar trabajadores para su
viña. \bibverse{2} Él decidió pagar un denario por día a los
trabajadores, y los envió a trabajar en ella. \bibverse{3} Cerca de las
9 a.m. salió y vio a otros que estaban sin trabajar en la plaza del
mercado.

\bibverse{4} ```Vayan y trabajen en la viña también, y yo les pagaré lo
justo,' les dijo. Entonces ellos se fueron a trabajar. \bibverse{5}
Entre el medio día y las 3 p.m. salió e hizo lo mismo. \bibverse{6} A
las 5 p.m. salió y encontró a otros que estaban allí. `¿Por qué están
por ahí todo el día sin hacer nada?' les preguntó. \bibverse{7} `Porque
nadie nos ha contratado,' respondieron ellos. `Vayan y trabajen en la
viña también,' les dijo.

\bibverse{8} ``Cuando llegó la noche, el propietario de la viña le dijo
a su administrador: `Llama a los trabajadores y págales sus salarios.
Comienza con los trabajadores que fueron contratados al final y luego
continúa con los que fueron contratados al principio.' \bibverse{9}
Cuando entraron los que fueron contratados a las 5 p.m., cada uno
recibió un denario. \bibverse{10} Así que cuando entraron los que fueron
contratados al principio, ellos pensaron que recibirían más, pero
también recibieron un denario. \bibverse{11} Cuando recibieron su pago,
se quejaron del propietario. \bibverse{12} `Los que fueron contratados
al final solo trabajaron una hora, y les pagaste lo mismo que a nosotros
que trabajamos todo el día en medio del calor abrasante,' refunfuñaban.

\bibverse{13} ``El propietario le respondió a uno de ellos: `Amigo, no
he sido injusto contigo. ¿No estuviste de acuerdo conmigo en trabajar
por un denario? \bibverse{14} Toma tu pago y vete. Lo mismo que te pagué
a ti, lo quiero pagar a los que fueron contratados al final.
\bibverse{15} ¿Acaso no puedo decidir qué hacer con mi propio dinero?
¿Por qué deberías mirarme con desprecio por querer hacer un bien?'
\bibverse{16} De esta manera, los últimos serán los primeros, y los
primeros serán los últimos.''

\bibverse{17} Cuando iba de camino hacia Jerusalén, Jesús llevó consigo
a los doce discípulos aparte mientras caminaban y les dijo:
\bibverse{18} ``Miren, vamos hacia Jerusalén, y el Hijo del hombre será
entregado a los jefes de los sacerdotes y los maestros religiosos. Ellos
lo condenarán a muerte \bibverse{19} y lo entregarán a los
gentiles\footnote{\textbf{20:19} Aquí se está refiriendo a los romanos.}
para que se burlen de él, lo azoten y lo crucifiquen. Pero el tercer día
será levantado de entre los muertos.''

\bibverse{20} Entonces la madre de los hijos de Zebedeo vino a Jesús con
sus dos hijos. Se arrodilló delante de él para hacerle una petición.

\bibverse{21} ``¿Qué es lo que me pides?'' le dijo Jesús.

``Por favor, aparta a mis hijos para que se sienten a tu lado en tu
reino, uno a tu derecha y el otro a tu izquierda,'' le pidió ella.

\bibverse{22} ``No sabes lo que estás pidiendo,'' le dijo Jesús.
``¿Pueden ustedes beber la copa\footnote{\textbf{20:22} Refiriéndose a
  la copa de sufrimiento.} que yo estoy a punto de beber?''

``Sí podemos,'' le dijeron.

\bibverse{23} ``Sin duda alguna ustedes beberán de mi copa,'' les dijo,
``pero el privilegio de sentarse a mi derecha y a mi izquierda no me
corresponde darlo a mi. Mi Padre es el que decide quién
será\footnote{\textbf{20:23} O, ``es para aquellos para quienes ha sido
  preparado por mi Padre.''}.''

\bibverse{24} Cuando los otros diez discípulos escucharon lo que ellos
habían pedido, se molestaron con los dos hermanos. \bibverse{25} Jesús
los llamó y les dijo: ``Ustedes saben que los gobernantes extranjeros se
enseñorean sobre sus pueblos, y los líderes poderosos los oprimen.
\bibverse{26} No será así para ustedes. Cualquiera entre ustedes que
quiera ser el más importante, será siervo de todos. \bibverse{27}
Cualquiera entre ustedes que quiera ser el primero, será como un
esclavo. \bibverse{28} De la misma manera, el Hijo del hombre no vino a
que le sirvan, sino a servir, y a dar su vida como rescate para
muchos.''

\bibverse{29} Cuando se fueron de Jericó, una gran multitud siguió a
Jesús. \bibverse{30} Dos hombres ciegos estaban sentados junto al
camino. Y cuando escucharon que Jesús iba pasando por allí, clamaron:
``¡Ten misericordia de nosotros, Señor, hijo de David!'' \bibverse{31} Y
la multitud les decía que se callaran, pero ellos gritaban aún más
fuerte: ``¡Ten misericordia de nosotros, Señor, hijo de David!''

\bibverse{32} Entonces Jesús se detuvo. Los llamó, preguntándoles:
``¿Qué quieren que haga por ustedes?''

\bibverse{33} ``Señor, por favor, haz que podamos ver,'' respondieron
ellos.

\bibverse{34} Jesús tuvo compasión de ellos y tocó sus ojos. Ellos
pudieron ver de inmediato, y le siguieron.

\hypertarget{section-20}{%
\section{21}\label{section-20}}

\bibverse{1} Entonces Jesús y sus discípulos fueron a Jerusalén. Cuando
se acercaban, llegaron a la aldea de Betfagé sobre el Monte de los
Olivos. Entonces Jesús envió a dos discípulos para que se adelantaran,
\bibverse{2} y les dijo: ``Vayan a la aldea. Apenas lleguen, encontrarán
allí un asno amarrado junto a un pollino Desamárrenlos y tráiganmelos.
\bibverse{3} Si alguien les pregunta qué hacen, solo díganle: 'El Señor
los necesita,'' y ellos los enviarán de inmediato.''

\bibverse{4} Esto cumplía lo que el profeta dijo: \bibverse{5} ``Di a la
hija de Sión: 'Mira, tu rey viene hacia ti. Es humilde, y monta un asno
y un pollino la cría de un asno.''\footnote{\textbf{21:5} Isaías 62:11,
  Zacarías 9:9.}

\bibverse{6} Los discípulos fueron e hicieron lo que Jesús les había
dicho. \bibverse{7} Trajeron el asno y el pollino. Colocaron sus mantos
sobre ellos y Jesús se sentó encima. \bibverse{8} Muchas personas que
estaban entre la multitud extendían sus mantos en el camino, mientras
que otros cortaban ramas de los árboles y las colocaban en el camino.
\bibverse{9} Las multitudes que iban delante de él y las que lo seguían
gritaban: ``¡Hosanna\footnote{\textbf{21:9} Una palabra aramea que
  significa ``por favor, sálvanos,'' y era usada como una exclamación de
  alabanza.} al hijo de David! ¡Bendito es el que viene en el nombre del
Señor! ¡Hosanna en las alturas!''

\bibverse{10} Cuando Jesús llegó a Jerusalén, toda la ciudad estaba
alborotada. ``¿Quién es este?'' preguntaban.

\bibverse{11} ``Este es Jesús, el profeta de Nazaret, en Galilea,''
respondieron las multitudes.

\bibverse{12} Jesús entró al Templo, y sacó de allí a todas las personas
que estaban comprando y vendiendo. Volteó las mesas de los cambistas y
las sillas de los vendedores de palomas. \bibverse{13} Entonces les
dijo: ``La Escritura dice: `Mi casa será llamada casa de
oración,'\footnote{\textbf{21:13} Isaías 56:7.} pero ustedes la han
convertido en una guarida de ladrones.''

\bibverse{14} Los ciegos y los paralíticos venían a Jesús al Templo, y
él los sanaba. \bibverse{15} Pero cuando el jefe de los sacerdotes y los
maestros religiosos vieron los milagros asombrosos que él hacía, y a los
niños que gritaban en el Templo, ``Hosanna al hijo de David,'' se
sintieron ofendidos.

``¿Escuchas lo que dicen estos niños?'' le preguntaron.

\bibverse{16} ``Sí,'' respondió Jesús. ``¿Acaso no han leído que la
Escritura dice `Preparaste a los niños y a los bebés para ofrecerte
alabanza perfecta'?''\footnote{\textbf{21:16} Salmos 8:2.} \bibverse{17}
Y dejándolos allí, se fue entonces a las afueras de la ciudad para
quedarse en Betania.

\bibverse{18} A la mañana siguiente, mientras caminaba de regreso a la
ciudad, Jesús sintió hambre. \bibverse{19} Entonces vio una higuera
junto al camino, y se dirigió hacia ella pero no encontró ningún fruto,
sino solamente hojas.

Entonces le dijo a la higuera: ``¡Ojalá que nunca más puedas producir
fruto!'' E inmediatamente la higuera se marchitó.

\bibverse{20} Los discípulos se asombraron al ver esto. ``¿Cómo pudo
marchitarse la higuera así de repente?'' preguntaban.

\bibverse{21} ``Les digo la verdad,'' respondió Jesús, ``Si ustedes
realmente creen en Dios, y no dudan de él, no solo podrían hacer lo que
acaba de suceder con la higuera, sino mucho más. Si ustedes dijeran a
esta montaña, `levántate y lánzate al mar,' ¡así sucedería!
\bibverse{22} Ustedes recibirán todo lo que pidan en oración, siempre
que crean en Dios.''

\bibverse{23} Entonces Jesús entró al Templo. Los jefes de los
sacerdotes y los ancianos del pueblo vinieron a él mientras enseñaba y
le preguntaron, ``¿Con qué autoridad haces estas cosas? ¿Quién te dio
esta autoridad?''

\bibverse{24} ``Yo también les haré una pregunta,'' respondió Jesús.
``Si me responden, yo les diré con qué autoridad hago estas cosas.
\bibverse{25} ¿Con qué autoridad bautizaba Juan? ¿Acaso su autoridad
venía del cielo, o de los hombres?''

Entonces ellos discutían unos con otros: ``Si decimos que venía del
cielo, entonces nos preguntará por qué no creímos en él. \bibverse{26}
Pero si decimos que venía de los hombres, entonces la multitud se
volverá contra nosotros\footnote{\textbf{21:26} Literalmente, ``tenemos
  miedo de la multitud.''}, porque todos ellos consideran a Juan como un
profeta.'' \bibverse{27} Entonces le respondieron a Jesús: ``No
sabemos.''

``Entonces yo no les diré con qué autoridad hago estas cosas,''
respondió Jesús. \bibverse{28} ``Pero ¿qué piensan de esta ilustración?
Había una vez un hombre que tenía dos hijos. Entonces fue donde el
primer hijo y le dijo: `Hijo, ve y trabaja en la viña hoy.'
\bibverse{29} Y el hijo le respondió, `No iré,' pero después se
arrepintió de lo que dijo y fue. \bibverse{30} Luego el hombre fue donde
el segundo hijo y le dijo lo mismo. Y él le dijo: `Iré,' pero no lo
hizo. \bibverse{31} ¿Cuál de los dos hijos hizo lo que su padre
quería?''

``El primero,'' respondieron ellos.

``Les digo la verdad: los recaudadores de impuestos y las prostitutas
están entrando al reino de los cielos antes que ustedes,'' les dijo
Jesús. \bibverse{32} ``Juan vino para mostrarles a ustedes la manera
correcta de vivir con Dios, y ustedes no creyeron en él, pero los
recaudadores de impuestos y las prostitutas creyeron en él. Después,
cuando vieron lo que sucedió, ustedes tampoco se arrepintieron ni
creyeron en él.

\bibverse{33} ``Esta es otra ilustración: había una vez un hombre, un
terrateniente, que plantó una viña. Puso una cerca alrededor de ella,
hizo un lagar y construyó una torre de vigilancia. La alquiló a unos
granjeros, y luego se fue a otro país. \bibverse{34} Cuando llegó el
tiempo de la cosecha, el hombre envió a sus siervos donde los granjeros
para recoger el fruto que le pertenecía. \bibverse{35} Pero los
granjeros atacaron a sus siervos. Golpearon a uno, mataron a otro y a
otro también lo apedrearon. \bibverse{36} Entonces el terrateniente
envió más siervos, pero los granjeros hicieron lo mismo con ellos.
\bibverse{37} Entonces el terrateniente envió a su hijo. `A mi hijo lo
respetarán,' pensó para sí. \bibverse{38} Pero los granjeros, cuando
vieron al hijo, se dijeron unos a otros, `¡Aquí viene el heredero!
¡Vamos! ¡Matémoslo para quedarnos con su herencia!' \bibverse{39} Lo
agarraron, lo sacaron de la viña y lo mataron. \bibverse{40} Entonces,
cuando el dueño de la viña regrese, ¿qué hará con esos granjeros?''

\bibverse{41} Entonces los jefes de los sacerdotes le dijeron a Jesús:
``Mandará a matar a esos hombres malvados de la manera más atroz, y
alquilará la viña a otros granjeros que de seguro sí le darán su fruto
en tiempo de la cosecha.''

\bibverse{42} ``¿Acaso no han leído las Escrituras?'' les preguntó
Jesús. ``\,`La piedra que rechazaron los constructores se ha convertido
en la piedra angular. El Señor ha hecho esto, y es maravilloso ante
nuestros ojos.' \bibverse{43} Por eso les digo que a ustedes se les
quitará el reino de Dios. Será entregado a un pueblo que producirá el
fruto apropiado. \bibverse{44} Cualquiera que tropiece con esta piedra,
será destruido, pero esta aplastará por completo a quien le caiga
encima.''

\bibverse{45} Cuando los jefes de los sacerdotes y los Fariseos
escucharon sus ilustraciones, se dieron cuenta de que Jesús estaba
hablando de ellos. \bibverse{46} Querían arrestarlo, pero tenían miedo
de lo que el pueblo pudiera hacer porque la gente creía que él era un
profeta.

\hypertarget{section-21}{%
\section{22}\label{section-21}}

\bibverse{1} Jesús les habló usando más relatos ilustrados. \bibverse{2}
``El reino de los cielos es como un rey que organizó una celebración de
boda para su hijo,'' explicó Jesús. \bibverse{3} ``Envió a sus siervos
donde todos los que estaban invitados a la boda para decirles que
vinieran, pero ellos se negaron a ir. \bibverse{4} Entonces envió más
siervos con las siguientes instrucciones: `Díganles a los que están
invitados, ``he preparado un banquete de bodas. Se han matado toros y
becerros---todo está listo. ¡Así que vengan a la boda!''\,'\,''

\bibverse{5} ``Pero ellos ignoraron la invitación y se fueron. Uno se
fue a sus campos; otro fue a ocuparse de su negocio. \bibverse{6} El
resto tomó a los siervos del rey, los maltrataron, y los mataron.
\bibverse{7} El rey se puso furioso. Entonces envió a sus soldados para
destruir a esos asesinos y quemar su ciudad.

\bibverse{8} ``Entonces el rey le dijo a sus siervos, `el banquete de la
boda está listo, pero los que estaban invitados no merecían asistir.
\bibverse{9} Vayan a las calles e inviten a todos los que encuentren
para que vengan a la boda.' \bibverse{10} Así que los siervos salieron a
las calles y trajeron a todos los que pudieron encontrar, tanto buenos
como malos. El salón de la boda estaba lleno.

\bibverse{11} ``Pero cuando el rey llegó a ver a los invitados, se dio
cuenta de que había un hombre que no tenía puesto el vestido adecuado
para la boda. \bibverse{12} Entonces le preguntó: `amigo, ¿cómo entraste
aquí sin vestido de bodas?' El hombre no sabía qué decir. \bibverse{13}
Entonces el rey dijo a sus siervos: `Aten sus manos y pies y láncenlo a
la oscuridad, donde habrá llanto y crujir de dientes.' \bibverse{14}
``Porque muchos son invitados, pero pocos son escogidos.''

\bibverse{15} Entonces los Fariseos se fueron de allí y se reunieron
para conspirar la manera en que podrían atraparlo por las cosas que
decía. \bibverse{16} Y enviaron a algunos de sus propios discípulos
donde él junto con algunos de los seguidores de Herodes.

``Maestro, sabemos que eres un hombre veraz, y que el camino de Dios que
enseñas es el verdadero,'' comenzaron ellos. ``Tú no te dejas influir
por ningún otro, y no te preocupa el rango o la posición social.
\bibverse{17} Así que déjanos saber lo que opinas. ¿Es correcto pagar
los impuestos del César, o no?''

\bibverse{18} Jesús sabía que sus intenciones eran malvadas. Entonces
les preguntó: ``¿Por qué están tratando de ponerme una trampa,
hipócritas? \bibverse{19} Muéstrenme la moneda que se usa para pagar el
impuesto.'' Entonces le trajeron una moneda de denario.\footnote{\textbf{22:19}
  Una moneda romana de plata que se usaba para pagar el impuesto exigido
  por los romanos.} \bibverse{20} ``¿De quién es la imagen y el título
que está inscrito en ella?'' les preguntó.

\bibverse{21} ``Es del césar,'' respondieron ellos.

``Ustedes deben dar al César lo es del César, y a Dios lo que es de
Dios,'' les dijo. \bibverse{22} Cuando escucharon la respuesta de Jesús,
se quedaron asombrados. Entonces se marcharon y lo dejaron allí.

\bibverse{23} Más tarde, ese mismo día, vinieron unos Saduceos a verlo.
(Los saduceos son los que dicen que no hay resurrección.) \bibverse{24}
Entonces le preguntaron: ``Maestro, Moisés dijo que si un hombre
casado\footnote{\textbf{22:24} Implícito.} muere sin haber tenido hijos,
su hermano debe casarse con la viuda y así tener hijos en representación
de su hermano\footnote{\textbf{22:24} Ver Deuteronomio. 25:5, 6.}.
\bibverse{25} Pues bien, supongamos que había siete hermanos. El primero
se casó y murió, y como no había tenido hijos, dejó la viuda a su
hermano. \bibverse{26} Lo mismo ocurrió con el segundo y el tercer
esposo, hasta que llegaron al séptimo. \bibverse{27} Al final, la mujer
también murió. \bibverse{28} Así que cuando ocurra la resurrección,
¿cuál de todos ellos será su esposo si ella se casó con todos?''

\bibverse{29} Jesús respondió: ``El error de ustedes es que no conocen
la Escritura ni lo que Dios puede hacer. \bibverse{30} Porque en la
resurrección las personas no se casarán ni serán entregados en
matrimonio tampoco---en el cielo son como ángeles. \bibverse{31} En
cuanto a la resurrección de los muertos, ¿no han leído lo que Dios les
dijo a ustedes: \bibverse{32} `Yo soy el Dios de Abraham, el Dios de
Isaac, y el Dios de Jacob'? Él no es Dios de los muertos, sino de los
que viven.'' \bibverse{33} Cuando las multitudes oyeron lo que dijo, se
quedaron asombrados de su enseñanza.

\bibverse{34} Cuando los Fariseos oyeron que Jesús había dejado sin
palabras a los Saduceos, se reunieron y fueron a hacerle más preguntas.
\bibverse{35} Uno de ellos, quien era un experto en la ley, le hizo una
pregunta para probarlo: \bibverse{36} ``Maestro, ¿cuál es el mandamiento
más importante de la ley?''

\bibverse{37} Jesús les dijo: ``\,`Ama al Señor tu Dios en todo lo que
piensas, en todo lo que sientes, y en todo lo que haces.'\footnote{\textbf{22:37}
  Deuteronomio 6:5.} \bibverse{38} Este es el mandamiento más
importante, el primer mandamiento. \bibverse{39} El segundo es similar:
`Ama a tu prójimo como a ti mismo.' \bibverse{40} Toda la ley bíblica y
los escritos de los profetas dependen de estos dos mandamientos.''

\bibverse{41} Mientras los Fariseos estaban allí reunidos, Jesús les
hizo una pregunta: \bibverse{42} ``¿Qué piensan ustedes del Mesías?''
les preguntó. ``¿De quién es hijo?''

``El hijo de David,'' respondieron ellos.

\bibverse{43} ``¿Cómo pudo David, bajo inspiración, llamarlo `Señor'?''
les preguntó Jesús. ``Él dice: \bibverse{44} `El Señor le dijo a mi
Señor: siéntate a mi diestra hasta que derrote a todos tus
enemigos\footnote{\textbf{22:44} Literalmente, ``coloque a todos tus
  enemigos debajo de tus pies.''}.' \bibverse{45} Si David lo llamó
Señor, ¿cómo puede el Mesías ser su hijo?'' \bibverse{46} Ninguno pudo
responderle, y desde entonces ninguno se atrevió a hacerle más
preguntas.

\hypertarget{section-22}{%
\section{23}\label{section-22}}

\bibverse{1} Entonces Jesús le habló a la multitud y a sus discípulos:
\bibverse{2} ``Los maestros religiosos y los Fariseos tienen la
responsabilidad de ser intérpretes de la ley de Moisés\footnote{\textbf{23:2}
  Literalmente, ``se sientan en la silla de Moisés.''}, \bibverse{3} así
que obedezcan y hagan lo que ellos les digan. Pero no imiten lo que
ellos hacen, porque ellos no practican lo que predican. \bibverse{4}
Ellos colocan cargas pesadas en los hombros del pueblo, pero ellos
mismos no mueven ni un dedo para ayudarles. \bibverse{5} Todo lo que
hacen es con el fin de hacerse notar. Ellos se alistan grandes cajas de
oraciones\footnote{\textbf{23:5} O ``filacterias.'' Estas eran cajas
  hechas con cuero que se ataban en la frente y los brazos y contenían
  textos escritos: Éxodo 13:1-6 y Deuteronomio 6:4-9; 11:13-21.} para
usarlas y colocan largas borlas en sus vestidos\footnote{\textbf{23:5}
  Estas borlas servían para mostrar su devoción a Dios. Ver Números
  15:37-41.}. \bibverse{6} Les gusta tener lugares de honor en los
banquetes y tener los mejores asientos en las sinagogas. \bibverse{7} A
ellos les gusta que los saluden con respeto en las plazas del mercado, y
que la gente les llame

`Rabí.'\footnote{\textbf{23:7} 23:7 Esta es una palabra Hebrea que
  significa ``mi gran {[}señor{]} ,'' y se usaba como un término que
  denotaba respeto hacia los maestros religiosos.}

\bibverse{8} ``No dejen que la gente los llame `Rabí.' El Gran Maestro
de ustedes es solo uno, y ustedes son todos hermanos. \bibverse{9} No
llamen a nadie con el título de `Padre' aquí en la tierra. El Padre de
ustedes es solo uno, y está en el cielo. \bibverse{10} No dejen que la
gente los llame `Maestro.' El Maestro de ustedes es solo uno, el Mesías.
\bibverse{11} El más importante entre ustedes tendrá que ser siervo
entre ustedes. \bibverse{12} Cualquiera que se enaltezca a sí mismo,
será humillado, y cualquiera que se humille, será enaltecido.

\bibverse{13} ``¡Pero qué desastre viene sobre ustedes, maestros
religiosos y Fariseos hipócritas! Ustedes cierran de golpe las puertas
del reino de los cielos en el rostro de la gente. No entran ustedes
mismos, ni dejan entrar a quien está tratando de hacerlo. \bibverse{14}
\footnote{\textbf{23:14} El versículo 14 no aparece en los primeros
  manuscritos más auténticos.} \bibverse{15} ¡Qué desastre viene sobre
ustedes, maestros religiosos y Fariseos hipócritas! Porque ustedes
viajan por tierra y mar para convertir a un solo individuo, y cuando lo
convierten, lo convierten dos veces más en un hijo de las
tinieblas\footnote{\textbf{23:15} Literalmente ``Gehenna'' (ver 5:22).
  El énfasis aquí está en el destino de los malvados.} como lo son
ustedes. \bibverse{16} ¡Qué desastre viene sobre ustedes los que dicen:
`si juras por el Templo, no tiene importancia, pero si juras por el oro
del Templo, entonces debes cumplir tu juramento!' ¡Cuán necios y ciegos
están ustedes! \bibverse{17} ¿Qué es más importante: el oro o el Templo
que santifica el oro? \bibverse{18} Ustedes dicen: `si juras sobre el
altar, no tiene importancia, pero si juras sobre el sacrificio que está
sobre el altar, entonces debes cumplir tu juramento.' \bibverse{19}
¡Cuán ciegos están ustedes! ¿Qué es más importante: el sacrificio, o el
altar que santifica el sacrificio? \bibverse{20} Si ustedes juran por el
altar, están jurando por el altar y por todo lo que está sobre él.
\bibverse{21} Si juran por el templo, están jurando por el Templo y por
Aquél que vive allí. \bibverse{22} Si juran por el cielo, están jurando
por el trono de Dios y por Aquél que se sienta en él.

\bibverse{23} ``¡Qué desastre viene sobre ustedes, maestros religiosos y
Fariseos hipócritas! Pagan el diezmo de la menta, de la semilla de anís
y del comino, pero son negligentes en los aspectos vitales de la ley:
hacer lo correcto, mostrar misericordia, ejercer la fe. Sí, es cierto
que deben pagar sus diezmos, pero no olviden estas otras cosas.
\bibverse{24} ¡Ustedes son guías ciegos que cuelan la bebida para no
dejar pasar una mosca, pero se tragan un camello!

\bibverse{25} ``¡Qué desastre viene sobre ustedes, maestros religiosos y
fariseos hipócritas! Limpian el exterior de la taza y del plato, pero
por dentro ustedes están llenos de glotonería y autocomplacencia.
\bibverse{26} ¡Fariseos ciegos! Limpien primero el interior de la taza y
del plato, para que entonces el exterior esté limpio también.

\bibverse{27} ``¡Qué desastre viene sobre ustedes, maestros religiosos y
Fariseos hipócritas! Son como sepulcros blanqueados, que se ven bien por
fuera, pero por dentro están llenos de esqueletos y todo tipo de
putrefacción\footnote{\textbf{23:27} Literalmente, ``inmundicia.''}.
\bibverse{28} Ustedes son simplemente una vergüenza. Por fuera parecen
buenas personas, pero por dentro están llenos de hipocresía y maldad.

\bibverse{29} ``¡Qué desastre viene sobre ustedes, maestros religiosos y
Fariseos hipócritas! Construyen sepulcros en memoria de los profetas, y
decoran las tumbas de los buenos, \bibverse{30} y dicen: `si hubiéramos
vivido en los tiempos de nuestros ancestros, no habríamos participado en
el derramamiento de la sangre de los profetas.' \bibverse{31} ¡Pero al
decir esto testifican contra ustedes mismos, demostrando que hacen parte
de esos que mataron a los profetas! \bibverse{32} ¡Entonces sigan y
acaben la obra de una vez por todas usando los métodos de sus
antepasados! \bibverse{33} Serpientes, camada de víboras, ¿cómo
escaparán del juicio de la condenación?\footnote{\textbf{23:33}
  Literalmente ``Gehenna'' (ver la nota del versículo 5:22). Hace
  referencia al juicio del fin de los tiempos.}

\bibverse{34} ``Por eso yo les envío profetas, hombres sabios y
maestros. A algunos los matarán, a otros los crucificarán, y a otros los
azotarán en las sinagogas, y los perseguirán de ciudad en ciudad.
\bibverse{35} Como consecuencia de ello, ustedes tendrán que dar cuenta
de la sangre de todas las personas buenas que se ha derramado sobre la
tierra: desde la sangre de Abel, que hizo lo correcto, hasta la sangre
de Zacarías, el hijo de Berequías, a quien ustedes mataron entre el
Templo y el altar.

\bibverse{36} ``Yo les digo que las consecuencias de todo esto caerán
sobre esta generación. \bibverse{37} ¡Oh Jerusalén, Jerusalén, tu matas
a los profetas y apedreas a los que se te envían! Tantas veces he
querido reunir a tus hijos así como una gallina reúne a sus polluelos
bajo sus alas, pero no me dejaste. \bibverse{38} Ahora mira, tu
casa\footnote{\textbf{23:38} La palabra ``Casa'' puede referirse al
  Templo.} ha sido abandonada, y está completamente vacía. \bibverse{39}
Yo te digo esto: no me volverás a ver hasta que digas `Bendito es el que
viene en el nombre del Señor.'\,''

\hypertarget{section-23}{%
\section{24}\label{section-23}}

\bibverse{1} Cuando Jesús iba saliendo del Templo, sus discípulos venían
hacia donde él estaba y mostraban con orgullo los edificios del Templo.
\bibverse{2} Pero Jesús respondió: ``¿Ven todos estos edificios? Les
digo la verdad: no quedará piedra sobre piedra. ¡Cada una de las piedras
que queden serán derribadas!''

\bibverse{3} Cuando Jesús se sentó en el Monte de los Olivos, los
discípulos vinieron donde él estaba y en privado le preguntaron: ``Por
favor, dinos cuándo ocurrirá esto. ¿Cuál será la señal de tu venida y
del fin del mundo?''

\bibverse{4} ``Asegúrense de que nadie los confunda,'' respondió Jesús.
\bibverse{5} ``Muchos vendrán diciendo que soy yo, y dirán `yo soy el
Mesías,' y engañarán a muchas personas. \bibverse{6} Escucharán de
guerras y rumores de guerras, pero ustedes no estén ansiosos. Estas
cosas tienen que pasar, pero este no es el fin. \bibverse{7} Habrá
naciones que atacarán a otras naciones, y reinos que pelearán contra
otros reinos. Habrá hambrunas y terremotos en diferentes lugares,
\bibverse{8} pero todas estas cosas son solo el principio de los dolores
del parto.

\bibverse{9} ``Entonces a ustedes los arrestarán, los perseguirán y los
matarán. Todas las personas los odiarán por mi causa. \bibverse{10} En
ese tiempo muchos que eran creyentes dejarán de creer. Se entregarán
unos a otros con traición y se odiarán unos a otros. \bibverse{11}
Muchos falsos profetas vendrán y engañarán a muchas personas.
\bibverse{12} El aumento del mal hará que el amor de muchos se enfríe,
\bibverse{13} pero aquellos que se mantengan firmes hasta el fin serán
salvos. \bibverse{14} La buena noticia del reino será proclamada en todo
el mundo de tal modo que todos la escucharán, y entonces vendrá el fin.
\bibverse{15} Así que cuando vean el `mal que profana'\footnote{\textbf{24:15}
  O, ``sacrilegio desolador,'' refiriéndose a Daniel 9:27, 11:31, 12:11.}
en el lugar santo del cual habló el profeta Daniel (los que leen esto,
por favor, examínenlo cuidadosamente), \bibverse{16} entonces las
personas que viven en Judea, deben huir a las montañas. \bibverse{17}
Todo el que esté en el tejado de la casa no debe descender para buscar
lo que hay en ella. \bibverse{18} El que esté en los campos, no debe
regresar a buscar el abrigo. \bibverse{19} ¡Cuán terrible será para
aquellas que estén embarazadas y para las que estén amamantando a sus
bebés en esos días! \bibverse{20} Oren para que no tengan que huir en
invierno, o en día Sábado. \bibverse{21} Porque en ese tiempo, habrá una
persecución terrible, más terrible que cualquier cosa que haya ocurrido
desde el principio del mundo hasta ahora, ni ocurrirá jamás.
\bibverse{22} A menos que esos días sean acortados, nadie será salvo,
pero por el bien de los elegidos, esos días serán acortados.

\bibverse{23} ``Así que si alguien les dice: `miren, este es el Mesías,'
o, `allá está,' no lo crean. \bibverse{24} Porque aparecerán falsos
mesías y falsos profetas también, y harán señales y milagros increíbles
para engañar a los escogidos, si fuera posible. \bibverse{25} Noten que
les he dicho esto antes de que siquiera ocurra. \bibverse{26} De modo
que si les dicen: `miren, está en el desierto,' no vayan a verlo allá; o
si dicen: `miren, está oculto aquí,' no lo crean. \bibverse{27} Porque
la venida del Hijo del hombre será como el relámpago que ilumina desde
el oriente hasta el occidente. \bibverse{28} `Los buitres se amontonan
donde está el cadáver.'

\bibverse{29} ``Pero justo después de estos días de persecución, el sol
se oscurecerá, la luna no brillará, las estrellas caerán del cielo, y
las potencias del cielo se conmoverán. \bibverse{30} Entonces aparecerá
en el cielo la señal del Hijo del hombre, y todos los pueblos de la
tierra se lamentarán. Verán al Hijo del hombre viniendo sobre las nubes
del cielo con poder y gran gloria. \bibverse{31} Con el toque de una
trompeta él enviará a sus ángeles para reunir a sus escogidos de todas
partes, desde un confín del cielo y de la tierra hasta el
otro\footnote{\textbf{24:31} Literalmente, ``de los cuatro vientos,
  desde un extremo del cielo hasta el otro.''}.

\bibverse{32} ``Aprendan una ilustración de la higuera. Cuando sus
brotes se vuelven más blandos y comienzan a salir las hojas, ya ustedes
saben que se acerca el verano. \bibverse{33} De la misma manera, cuando
vean que están ocurriendo todas estas cosas, ya sabrán que su venida
está cerca, ¡de hecho, está justo en la puerta! \bibverse{34} Les digo
la verdad: esta generación no morirá hasta que todas estas cosas hayan
pasado. \bibverse{35} El cielo y la tierra podrán perecer, pero mis
palabras no morirán.

\bibverse{36} ``Pero nadie sabe el día ni la hora en que esto ocurrirá,
ni siquiera los ángeles en el cielo, ni el Hijo. Solo el Padre sabe.
\bibverse{37} Cuando el Hijo del hombre venga, será como en los días de
Noé. \bibverse{38} Será como en los días antes del diluvio, donde todos
comían y bebían y se casaban y se entregaban en matrimonio, hasta el día
que Noé entró al arca. \bibverse{39} Ellos no se dieron cuenta de lo que
estaba ocurriendo hasta que el diluvio vino y se los llevó a todos. Así
será la venida del Hijo del hombre.

\bibverse{40} ``Dos hombres estarán trabajando en los campos. Se tomará
a uno y se dejará al otro. \bibverse{41} Dos mujeres estarán moliendo
grano en un molino. Se tomará a una y se dejará a la otra. \bibverse{42}
Así que estén prevenidos, porque ustedes no saben qué día viene el
Señor. \bibverse{43} Pero consideren esto: si el dueño de la casa
supiera a qué hora vendrá el ladrón, permanecería vigilando. No dejaría
que entre y robe en su casa. \bibverse{44} Ustedes también necesitan
estar listos, porque el Hijo del hombre viene en un momento en que
ustedes no lo esperan.

\bibverse{45} ``Pues ¿quién es el siervo fiel y considerado? Es el que
su amo pone a cargo de la familia para que provea el alimento en el
momento adecuado. \bibverse{46} ¡Cuán bueno es que el siervo se
encuentre haciendo esto cuando su amo regrese! \bibverse{47} Les digo la
verdad: el amo pondrá a ese siervo a cargo de todas sus posesiones.
\bibverse{48} Pero si fuese un siervo malo, diría para sí mismo: `mi
señor se está demorando,' \bibverse{49} y comenzaría a golpear a los
otros siervos, a festejar y a beber con los borrachos. \bibverse{50}
Entonces el amo de ese siervo regresará cuando este no lo espera, en un
momento que no sabe. \bibverse{51} Entonces el amo lo hará pedazos, y lo
tratará como a los hipócritas\footnote{\textbf{24:51} Los que dicen que
  siguen a su Señor pero en realidad no lo hacen.}, enviándolo a un
lugar donde hay lamento y crujir de dientes.

\hypertarget{section-24}{%
\section{25}\label{section-24}}

\bibverse{1} ``El reino de los cielos es como diez jovencitas, que
llevaron sus lámparas para ir al encuentro del novio. \bibverse{2} Cinco
de ellas eran necias, y cinco eran sabias. \bibverse{3} Las jóvenes
necias llevaron sus lámparas pero no llevaron aceite, \bibverse{4}
mientras que las sabias llevaron frascos de aceite junto con sus
lámparas. \bibverse{5} El novio se demoró mucho y todas las jóvenes
comenzaron a sentirse somnolientas y se durmieron. \bibverse{6} A la
media noche se escuchó el grito: `¡Miren aquí está el novio! ¡Vengan a
su encuentro!' \bibverse{7} Todas las jovencitas se levantaron y
cortaron la mecha de sus lámparas. Las jóvenes necias le dijeron a las
jóvenes sabias: \bibverse{8} `Dénnos un poco de su aceite porque
nuestras lámparas se están apagando.' Pero las jovencitas sabias
respondieron: \bibverse{9} `No, porque así no habrá suficiente aceite
para ustedes ni para nosotras. Vayan a las tiendas y compren aceite para
ustedes.' \bibverse{10} Mientras fueron a comprar el aceite, llegó el
novio y los que estaban listos entraron con él a la boda, y la puerta se
cerró con llave. \bibverse{11} Las otras jóvenes llegaron más tarde.
`Señor, Señor,' llamaron, `¡ábrenos la puerta!' \bibverse{12} Pero él
respondió: `En verdad les digo que no las conozco.' \bibverse{13} Así
que estén alerta, porque ustedes no saben el día ni la hora.

\bibverse{14} ``Es como un hombre que se fue de viaje. Llamó a sus
siervos y los dejó a cargo de sus posesiones. \bibverse{15} A uno de
ellos le entregó cinco talentos\footnote{\textbf{25:15} Refiriéndose
  probablemente a talentos de plata, una gran cantidad de dinero.}, a
otro le dio dos, y a otro le dio uno, conforme a sus capacidades. Luego
se fue. \bibverse{16} De inmediato, el que tenía cinco talentos fue y
los invirtió en un negocio, y obtuvo otros cinco talentos. \bibverse{17}
De la misma manera, el que tenía dos talentos obtuvo otros dos.
\bibverse{18} Pero el que recibió un talento se fue y cavó un hoyo y
escondió allí el dinero de su amo. \bibverse{19} Mucho tiempo después,
el amo de estos siervos regresó y se dispuso a ajustar cuentas con
ellos. \bibverse{20} El que recibió cinco talentos vino y presentó otros
cinco talentos. `Mi señor,' le dijo, `me diste cinco talentos. Mira,
obtuve ganancia de cinco talentos más.' \bibverse{21} Su amo le dijo:
`has hecho bien, eres un siervo bueno y fiel. Has demostrado que eres
fiel en cosas pequeñas, así que ahora te colocaré a cargo de muchas
cosas. ¡Alégrate porque estoy muy complacido de ti! \bibverse{22} El
siervo que recibió dos talentos también vino. `Mi señor,' le dijo, `me
entregaste dos talentos. Mira, he obtenido ganancia de dos talentos
más.' \bibverse{23} Su amo le dijo: `has hecho bien, eres un siervo
bueno y fiel. Has demostrado que eres fiel en cosas pequeñas, así que
ahora te pondré a cargo de muchas cosas. ¡Alégrate porque estoy muy
complacido de ti!'

\bibverse{24} ``Entonces vino el hombre que recibió un talento. ``Mi
señor,' le dijo, `sé que eres un hombre duro. Siegas donde no sembraste
y recoges cosechas que no plantaste. \bibverse{25} Así que como tuve
miedo de ti fui y enterré tu talento. Mira, aquí tienes lo que te
pertenece.' \bibverse{26} Pero su amo le respondió: `¡Eres un siervo
malo y perezoso! Si crees que siego donde no sembré, y que recojo
cosechas que no planté, \bibverse{27} entonces debiste depositar en el
banco la plata que me pertenece y así yo habría recibido mi dinero con
intereses al regresar. \bibverse{28} Quítenle el talento que tiene y
dénselo al que tiene diez talentos. \bibverse{29} Porque al que tiene se
le dará aún más; y al que no tiene nada, incluso lo que tenga se le
quitará. \bibverse{30} Ahora lancen a este siervo inútil en la oscuridad
donde habrá llanto y crujir de dientes.'

\bibverse{31} ``Pero cuando el Hijo del hombre venga en su Gloria, y
todos los ángeles con él, se sentará en su trono majestuoso.
\bibverse{32} Traerán a todos delante de él. Entonces él separará a los
unos de los otros, así como un pastor separa a las ovejas de los
cabritos. \bibverse{33} Entonces colocará a las ovejas a su derecha, y a
los cabritos en su mano izquierda. \bibverse{34} Entonces el rey dirá a
los de su derecha: `vengan ustedes, benditos de mi Padre, hereden el
reino que ha sido preparado para ustedes desde el principio del mundo.
\bibverse{35} Porque tuve hambre y me dieron alimento para comer. Tuve
sed, y me dieron de beber. Fui forastero y me hospedaron. \bibverse{36}
Estuve desnudo y me vistieron. Estuve enfermo y cuidaron de mí. Estuve
en la cárcel y me visitaron.' \bibverse{37} Entonces los de la derecha
responderán: `Señor, ¿cuándo te vimos con hambre y te alimentamos, o
sediento y te dimos de beber? \bibverse{38} ¿Cuándo te vimos como
forastero y te hospedamos, o desnudo y te vestimos? \bibverse{39}
¿Cuándo te vimos enfermo o en la cárcel y te visitamos?' \bibverse{40}
El rey les dirá: `en verdad les digo que todo lo que hicieron por uno de
estos de menor importancia, lo hicieron por mi.'

\bibverse{41} ``También dirá a los de su izquierda: `¡apártense de mi,
ustedes malditos, vayan al fuego eterno preparado para el diablo y sus
ángeles! \bibverse{42} Porque tuve hambre y no me dieron nada de comer.
Tuve sed y no me dieron de beber. \bibverse{43} Fui forastero y no me
hospedaron. Estuve desnudo y no me vistieron. Estuve enfermo y en la
cárcel y no me visitaron.' \bibverse{44} Entonces ellos también
responderán: `Señor, ¿cuándo te vimos con hambre, con sed, o como
forastero, o desnudo, o enfermo, o en la cárcel y no cuidamos de ti?'
\bibverse{45} Entonces él les dirá: `en verdad les digo que todo lo que
no hicieron por uno de estos de menor importancia, no lo hicieron por
mi.' \bibverse{46} Ellos se irán a la condenación eterna, pero los
justos entrarán a la vida eterna.''

\hypertarget{section-25}{%
\section{26}\label{section-25}}

\bibverse{1} Después que hubo dicho todo esto, Jesús le dijo a los
discípulos: \bibverse{2} ``Ustedes saben que en dos días es la Pascua, y
el Hijo del hombre será entregado y crucificado.''

\bibverse{3} Entonces los jefes de los sacerdotes y los ancianos del
pueblo se reunieron en el patio de Caifás, el sumo sacerdote.
\bibverse{4} Allí conspiraron para arrestar a Jesús bajo algún pretexto
engañoso\footnote{\textbf{26:4} Literalmente, ``con una artimaña.''} y
matarlo. \bibverse{5} Pero dijeron: ``no hagamos esto durante el
festival para que no haya disturbios en el pueblo.''

\bibverse{6} Mientras Jesús estaba en la casa de Simón el leproso, en
Betania, \bibverse{7} vino una mujer que traía un frasco de alabastro
que contenía un perfume muy costoso. Ella lo derramó en la cabeza de
Jesús mientras él estaba sentado y comía. Pero cuando los discípulos
vieron lo que ella hizo, se incomodaron por ello.

\bibverse{8} ``¡Qué gran desperdicio!'' objetaron. \bibverse{9} ``¡Este
perfume pudo haberse vendido por mucho dinero y lo habríamos regalado a
los pobres!''

\bibverse{10} Jesús sabía lo que estaba pasando y les dijo: ``¿Por qué
están enojados con esta mujer? ¡Ella ha hecho algo maravilloso por mí!
\bibverse{11} Los pobres siempre estarán entre ustedes, pero no siempre
me tendrán a mí. \bibverse{12} Al derramar este perfume en mi cuerpo,
ella me ha preparado para mi sepultura. \bibverse{13} Les digo la
verdad: dondequiera que se difunda esta buena noticia, se contará lo que
esta mujer ha hecho, en memoria de ella.''

\bibverse{14} Entonces Judas Iscariote, uno de los doce discípulos, fue
donde estaban los jefes de los sacerdotes \bibverse{15} y les preguntó:
``¿Cuánto me pagarán por entregarles a Jesús?'' Y Ellos le pagaron
treinta monedas de plata. \bibverse{16} A partir de ese momento, Judas
buscaba una oportunidad para entregar a Jesús.

\bibverse{17} El primer día del festival del pan sin levadura, los
discípulos vinieron donde Jesús y le preguntaron: ``¿Dónde quieres que
preparemos la cena de la Pascua para ti?''

\bibverse{18} Jesús les dijo: ``vayan a la ciudad y busquen a cierto
hombre que está ahí y díganle que el Maestro dice: `Se acerca mi hora.
Voy a celebrar la Pascua con mis discípulos en tu casa.'\,''
\bibverse{19} Entonces los discípulos hicieron lo que Jesús les dijo, y
prepararon allí la cena de la Pascua.

\bibverse{20} Cuando llegó la noche, Jesús se sentó allí a comer con los
doce. \bibverse{21} Mientras comían, les dijo: ``En verdad les digo que
uno de ustedes va a entregarme.''

\bibverse{22} Ellos estaban extremadamente incómodos. Uno por uno le
preguntaban: ``Señor, no soy yo, ¿cierto?''

\bibverse{23} ``El que ha metido su mano conmigo en el plato, me
entregará,'' respondió Jesús. \bibverse{24} ``El Hijo del hombre morirá
tal como fue profetizado acerca de él, pero ¡qué desgracia vendrá sobre
el hombre que entregue al Hijo del hombre! ¡Habría sido mejor que nunca
hubiera nacido!''

\bibverse{25} Judas, el que lo iba a entregar, preguntó ``¿Seré yo,
Rabí?''

``Tu lo has dicho,'' respondió Jesús.

\bibverse{26} Mientras comían, Jesús tomó del pan y lo bendijo. Entonces
lo partió y lo repartió entre los discípulos. ``Tomen este pan y cómanlo
porque este es mi cuerpo,'' dijo Jesús. \bibverse{27} Entonces cogió la
copa, la bendijo y se la entregó a ellos. ``Tomen todos de esta copa,''
les dijo. \bibverse{28} ``Porque esta es mi sangre del pacto, derramada
por muchos para el perdón de pecados. \bibverse{29} Sin embargo, les
digo, yo no beberé más de este fruto de la vid hasta el día en que
vuelva a beberlo nuevamente con ustedes en el reino de mi Padre.''
\bibverse{30} Después que terminaron de cantar, se fueron al Monte de
los Olivos.

\bibverse{31} ``Todos ustedes me abandonarán esta noche,'' les dijo
Jesús. ``Como dice la Escritura: `Yo golpearé al pastor, y el rebaño
estará completamente disperso.'\footnote{\textbf{26:31} Zacarías 13:7.}
\bibverse{32} Pero después que me haya levantado, yo iré delante de
ustedes a Galilea.''

\bibverse{33} Pero Pedro objetó: ``incluso si todos los demás te
abandonan, yo nunca te abandonaré.''

\bibverse{34} ``Te digo la verdad,'' le dijo Jesús, ``esta misma noche,
antes de que el gallo cante, me negarás tres veces.''

\bibverse{35} ``¡Aun si tengo que morir contigo, nunca te negaré!''
insistió Pedro. Y todos los discípulos dijeron lo mismo.

\bibverse{36} Entonces Jesús se fue con sus discípulos a un lugar
llamado Getsemaní. Les dijo: ``Siéntense aquí mientras yo voy allá a
orar.'' \bibverse{37} Entonces llevó consigo a Pedro y a los dos hijos
de Zebedeo, y comenzó a sufrir tristeza y aflicción agonizantes.
\bibverse{38} Entonces les dijo: ``Estoy tan inundado de tristeza, que
siento morir. Esperen aquí y estén en vigilia conmigo.'' \bibverse{39}
Entonces se fue un poco más lejos, se postró sobre su rostro y oró.

``Padre mío, por favor, si es posible, quítame esta copa de
sufrimiento,'' pidió Jesús. ``Aun así, que no sea lo que yo quiero sino
lo que tu quieres.''

\bibverse{40} Entonces regresó donde estaban los discípulos y los
encontró dormidos. Le dijo entonces a Pedro: ``¿Cómo es que no pudieron
estar despiertos conmigo apenas una hora? \bibverse{41} Estén despiertos
y oren, para que no caigan en tentación. Sí, el espíritu está dispuesto,
pero el cuerpo es débil.'' \bibverse{42} Entonces se fue por segunda vez
y oró.

``Padre mío, si no puedes quitarme esta copa sin que yo la beba,
entonces se hará tu voluntad,'' dijo. \bibverse{43} Regresó entonces y
encontró a los discípulos durmiendo, porque no pudieron mantenerse
despiertos\footnote{\textbf{26:43} Literalmente, ``sus ojos estaban
  pesados.''}. \bibverse{44} Entonces los dejó allí una vez más y se fue
y oró por tercera vez, repitiendo las mismas cosas. \bibverse{45}
Entonces regresó donde estaban sus discípulos, y les dijo: ``¿Cómo es
posible que aún estén durmiendo y descansando? Miren, el momento ha
llegado. ¡El Hijo del hombre está a punto de ser entregado en manos de
pecadores! \bibverse{46} ¡Levántense, vámonos! Miren, acaba de llegar el
que me entrega.''

\bibverse{47} Cuando dijo esto, Judas, uno de los doce, llegó con una
gran turba que estaba armada con espadas y palos, y habían sido enviados
por los jefes de los sacerdotes y por los ancianos del pueblo.
\bibverse{48} El traidor había acordado que les daría una señal: ``Al
que yo bese, ese es---arréstenlo,'' les dijo.

\bibverse{49} Judas llegó inmediatamente donde estaba Jesús y dijo:
``Hola, Rabí,'' y lo besó.

\bibverse{50} ``Amigo mío, haz lo que viniste a hacer,'' le dijo Jesús a
Judas. Entonces vinieron y tomaron a Jesús y lo arrestaron.

\bibverse{51} Uno de los que estaban con Jesús alcanzó su espada y la
sacó. Atacó con ella al siervo del sumo sacerdote, cortándole la oreja.
\bibverse{52} Pero Jesús le dijo: ``Guarda tu espada. Todo el que pelea
con una espada, morirá a espada. \bibverse{53} ¿Acaso no crees que yo
podría rogar a mi Padre, y él enviaría más de doce legiones de ángeles
de inmediato? \bibverse{54} Pero entonces ¿cómo podría cumplirse la
Escritura que dice que esto debe ocurrir?''

\bibverse{55} Entonces Jesús le dijo a la turba: ``¿Han venido con
espadas y palos para arrestarme como si yo fuese algún criminal? Todos
los días me sentaba en el templo a enseñarles y en ese momento no me
arrestaron. \bibverse{56} Pero todo esto está ocurriendo para que se
cumpla lo que escribieron los profetas.'' Entonces todos los discípulos
lo abandonaron y huyeron.

\bibverse{57} Los que habían arrestado a Jesús lo llevaron a la casa de
Caifás, el sumo sacerdote, donde se habían reunido los maestros
religiosos y los ancianos. \bibverse{58} Pedro los seguía a la
distancia, y entró al patio de los sumos sacerdotes. Se sentó allí con
los guardias para ver cómo terminaban las cosas.

\bibverse{59} Los jefes de los sacerdotes y todo el concilio estaban
tratando de encontrar alguna prueba falsa contra Jesús para mandarlo a
matar. \bibverse{60} Pero no podían encontrar nada, aun cuando habían
venido muchos testigos falsos. Finalmente, llegaron dos \bibverse{61} e
informaron: ``Este hombre dijo: `yo puedo destruir el templo de Dios, y
volver a construirlo en tres días.'\,''

\bibverse{62} El sumo sacerdote se levantó y le preguntó a Jesús: ``¿No
tienes nada que responder? ¿Qué tienes para decir en tu defensa?''
\bibverse{63} Pero Jesús se quedó en silencio.

El sumo sacerdote le dijo a Jesús: ``En nombre del Dios vivo, te coloco
bajo juramento. Dinos si eres el Mesías, el Hijo de Dios.''

\bibverse{64} ``Tu lo has dicho,'' respondió Jesús. ``Y también te digo
que en el futuro verás al Hijo de Dios sentado a la diestra del
Todopoderoso, y viniendo en las nubes de los cielos.''

\bibverse{65} Entonces el sumo sacerdote rasgó su ropa, y dijo: ``¡Está
diciendo blasfemia! ¿Para qué necesitamos testigos? ¡Miren, ustedes
mismos han escuchado su blasfemia! \bibverse{66} ¿Qué veredicto dan
ustedes?''

``¡Culpable! ¡Merece morir!'' respondieron ellos. \bibverse{67} Entonces
escupieron su rostro y lo golpearon. Algunos de ellos lo abofetearon con
sus manos, \bibverse{68} y dijeron: ``¡Profetízanos, `Mesías'! ¿Quién es
el que te acaba de golpear?''

\bibverse{69} Mientras tanto, Pedro estaba sentado afuera en el patio.
Una joven criada vino donde él estaba y dijo: ``¡Tu también estabas con
Jesús el galileo!'' \bibverse{70} Pero él lo negó delante de todos. ``No
sé de qué hablas,'' dijo él.

\bibverse{71} Entonces regresó a la entrada de la casa, donde otra
persona lo vio y le dijo a las personas que estaban allí: ``Este hombre
estaba con Jesús de Nazaret.'' \bibverse{72} Una vez más, Pedro lo negó,
diciendo con juramento: ``Yo no lo conozco.'' \bibverse{73} Un poco más
tarde, las personas que estaban allí vinieron donde estaba Pedro y
dijeron: ``Definitivamente tu eres uno de ellos. Tu acento te delata.''
\bibverse{74} Entonces comenzó a jurar: ``¡Que me caiga una maldición si
estoy mintiendo!\footnote{\textbf{26:74} O, ``invocó maldiciones sobre
  sí mismo.''} ¡No conozco al hombre!'' E inmediatamente el gallo cantó.

\bibverse{75} Entonces Pedro recordó lo que Jesús le había dicho:
``Antes de que el gallo cante, negarás tres veces que me conoces.''
Entonces salió y lloró amargamente.

\hypertarget{section-26}{%
\section{27}\label{section-26}}

\bibverse{1} Temprano en la mañana, todos los jefes de los sacerdotes y
los ancianos del pueblo se reunieron a consultar y decidieron mandar a
matar a Jesús. \bibverse{2} Lo ataron, se lo llevaron y se lo enviaron a
Pilato, el gobernador.

\bibverse{3} Cuando Judas, el que había entregado a Jesús, vio que Jesús
había sido condenado a muerte, se arrepintió de lo que había hecho y
devolvió las treinta monedas de plata a los jefes de los sacerdotes y a
los ancianos. \bibverse{4} ``¡He pecado! ¡He entregado sangre
inocente!'' les dijo.

``¿A nosotros qué nos importa eso?'' respondieron ellos. ``¡Ese es tu
problema!'' \bibverse{5} Judas lanzó las monedas de plata en el
santuario y se fue. Huyó y se ahorcó. \bibverse{6} Los jefes de los
sacerdotes tomaron las monedas de plata y dijeron: ``Este es dinero de
sangre, es contra la ley poner este dinero en la tesorería del templo.''
\bibverse{7} Entonces se pusieron de acuerdo para comprar el campo del
alfarero para usarlo como el lugar donde sepultarían a los extranjeros.
\bibverse{8} Por eso hasta hoy a ese campo se le llama el ``Campo de
Sangre''.'' \bibverse{9} Esto cumplió la profecía dicha por el profeta
Jeremías: ``Tomaron treinta monedas de plata --- el `valor' de aquel que
fue comprado por el precio que le pusieron unos hijos de Israel---
\bibverse{10} y las usaron para pagar el campo del alfarero, como el
Señor me mandó a hacerlo.''\footnote{\textbf{27:10} Zacarías11:12, 13,
  haciendo referencia a Jeremías 32:6-15.}

\bibverse{11} Jesús fue llevado delante de Pilato el gobernador, quien
le preguntó: ``¿Eres tu el Rey de los Judíos?''

``Tú lo has dicho,'' respondió Jesús. \bibverse{12} Pero cuando el jefe
de los sacerdotes y los ancianos presentaron cargos contra él, Jesús no
respondió.

\bibverse{13} ¿No escuchas todos los cargos que ellos están presentando
contra ti?'' le preguntó Pilato. \bibverse{14} Pero Jesús no dijo nada,
ni una sola palabra. Esto sorprendió en gran manera al gobernador.

\bibverse{15} Y era costumbre del gobernador, durante la fiesta, liberar
delante de la multitud a cualquier prisionero que ellos quisieran.
\bibverse{16} En esa época, estaba preso un hombre llamado Barrabás.
\bibverse{17} Así que Pilato le preguntó a las multitudes que se habían
reunido: ``¿A quién quieren que libere: a Barrabás, o a Jesús, llamado
el Mesías?'' \bibverse{18} (Él se había dado cuenta que ellos habían
arrestado a Jesús por celos para juzgarlo). \bibverse{19} Mientras
estaba sentado en la silla de juez, su esposa le envió un mensaje que
decía: ``No le hagas nada a este hombre inocente, porque he sufrido
terriblemente en el día de hoy por un sueño que tuve sobre él.''

\bibverse{20} Pero los jefes de los sacerdotes y los ancianos
convencieron a las multitudes de pedir a Barrabás, y mandar a matar a
Jesús. \bibverse{21} Cuando el gobernador les preguntó: ``¿A cuál de los
dos quieren que les libere entonces?'' ellos respondieron: ``Barrabás.''

\bibverse{22} ``¿Entonces qué hare con Jesús, el Mesías?'' les preguntó.

Todos gritaron: ``¡Que lo crucifiquen!''

\bibverse{23} ``¿Por qué? ¿Qué crimen ha cometido él?'' preguntó Pilato.
Pero ellos gritaban aún más fuerte: ``¡Crucifícalo!''

\bibverse{24} Cuando Pilato vio que la causa estaba perdida, y que se
estaba formando un motín, trajo agua y lavó sus manos frente a la
multitud. ``Soy inocente de la sangre de este hombre. ¡Su sangre estará
sobre sus cabezas!\footnote{\textbf{27:24} Literalmente, ``ustedes
  mismos sean responsables de ello.''}'' \bibverse{25} Todo el pueblo
respondió: ``¡Que su sangre sea sobre nuestras cabezas y las de nuestros
hijos!'' \bibverse{26} Entonces Pilato liberó a Barrabás, pero mandó a
azotar a Jesús y a crucificarlo.

\bibverse{27} Los soldados del gobernador llevaron a Jesús hasta el
Pretorio\footnote{\textbf{27:27} El cuartel militar.} y toda la tropa de
soldados lo rodeaba. \bibverse{28} Entonces lo desnudaron y pusieron un
manto de color escarlata sobre él. \bibverse{29} Hicieron una corona de
espinas y la colocaron sobre su cabeza, y le pusieron un palo en su mano
derecha. Y se arrodillaban frente a él y se burlaban diciendo: ``¡Salve,
Rey de los judíos!'' \bibverse{30} Luego lo escupieron, y tomando el
palo que tenía, le golpeaban la cabeza con él. \bibverse{31} Cuando
terminaron de burlarse de él, le quitaron el manto y volvieron a ponerle
su ropa. Entonces se lo llevaron para crucificarlo. \bibverse{32} En el
camino, se encontraron a un hombre llamado Simón, de Cirene, y lo
obligaron a llevar la cruz de Jesús.

\bibverse{33} Cuando llegaron a Gólgota, que significa ``Lugar de la
Calavera,'' \bibverse{34} le dieron vino mezclado con hiel. Pero después
de probarlo, se negó a beberlo. \bibverse{35} Después de haberlo
crucificado, lanzaron unos dados para dividir su ropa entre ellos.
\bibverse{36} Entonces se sentaron y se quedaron allí vigilándolo.
\bibverse{37} Colocaron una señal sobre su cabeza con el cargo que fue
presentado contra él. Decía: ``Este es Jesús, el Rey de los judíos.''
\bibverse{38} Entonces crucificaron a dos criminales con él, uno a su
derecha, y el otro a su izquierda. \bibverse{39} Los que pasaban por ahí
le gritaban insultos, sacudiendo sus cabezas, \bibverse{40} y decían:
``¡Tú que prometiste destruir el templo y reconstruirlo en tres días,
por qué no te salvas a ti mismo! Si realmente eres el Hijo de Dios,
entonces bájate de la cruz.'' \bibverse{41} Los jefes de los sacerdotes
se burlaban de él de la misma manera, igual que los maestros religiosos
y los ancianos. \bibverse{42} ``¡Salvó a otros pero no puede salvarse a
sí mismo!'' decían. ``¡Si realmente él es el rey de Israel, que se baje
de la cruz y le creeremos! \bibverse{43} Él cree en Dios con tanta
seguridad, ---pues entonces que Dios lo rescate si lo quiere, pues él
decía `yo soy el Hijo de Dios.'\,'' \bibverse{44} Y los criminales que
estaban crucificados con él también lo insultaban de la misma manera.

\bibverse{45} Desde el medio día hasta las tres de la tarde hubo
tinieblas en todo el país. \bibverse{46} Aproximadamente a las tres de
la tarde, Jesús gritó fuertemente diciendo: ``Eli, Eli, lama
sabachthani?'' que significa: ``Dios mío, Dios mío, ¿por qué me has
abandonado?'' \bibverse{47} Cuando algunos de los que estaban allí lo
escucharon, dijeron: ``¡Está llamando a Elías!'' \bibverse{48} E
inmediatamente uno de ellos tomó una esponja, la sumergió en vinagre y
se lo dio a beber a Jesús. \bibverse{49} Pero los otros decían: ``Déjalo
solo. Veamos si Elías viene y lo salva.''

\bibverse{50} Jesus gritó otra vez a gran voz, y dio su último
respiro\footnote{\textbf{27:50} Esta expresión es hebrea y quiere decir
  que murió.}. \bibverse{51} Justo en ese momento, el velo del templo se
rasgó de arriba a abajo. La tierra tembló, las rocas se partieron,
\bibverse{52} y las tumbas se abrieron. Muchos de los que habían vivido
de manera justa y habían muerto, fueron levantados a la vida.
\bibverse{53} Y después de la resurrección de Jesús, estos salieron de
los cementerios y entraron a la ciudad santa\footnote{\textbf{27:53}
  Refiriéndose a Jerusalén.} donde muchos los vieron.

\bibverse{54} Cuando el centurión y los que estaban con él vigilando a
Jesús vieron el terremoto y lo que había ocurrido, se atemorizaron y
dijeron: ``¡Este era realmente el Hijo de Dios!'' \bibverse{55} Muchas
mujeres también miraban a la distancia, las que habían seguido a Jesús
desde Galilea y lo habían apoyado. \bibverse{56} Entre estas estaba
María Magdalena, María la madre de Jesús, María la madre de Santiago y
José, y la madre de los hijos de Zebedeo.

\bibverse{57} Cuando llegó la noche, un hombre rico llamado José, de
Arimatea, (quien también era discípulo de Jesús), \bibverse{58} fue
donde Pilato y pidió que le entregaran el cuerpo de Jesús. Entonces
Pilato ordenó que se le entregara. \bibverse{59} José tomó el cuerpo y
lo envolvió en un paño nuevo de lino, \bibverse{60} y lo puso en su
propia tumba que estaba nueva, hecha de roca sólida. Entonces rodó una
gran piedra que estaba puesta a la entrada de la tumba, y se fue.
\bibverse{61} María Magdalena y la otra mujer llamada María, estaban
allí sentadas al otro lado de la tumba.

\bibverse{62} Al día siguiente\footnote{\textbf{27:62} Refiriéndose al
  Sábado.}, después del día de la Preparación, los jefes de los
sacerdotes fueron juntos a ver a Pilato. \bibverse{63} Y le dijeron:
``Señor, recordamos que el impostor cuando estaba vivo dijo: `Después de
tres días me levantaré de nuevo.' \bibverse{64} Da la orden para vigilar
la tumba hasta el tercer día. Así sus discípulos no pueden llegar y
robar el cuerpo y decir al pueblo que él se levantó de entre los
muertos, y que la decepción al final llegue a ser peor que lo que era al
principio.''

\bibverse{65} ``Les daré una guardia de soldados,'' les dijo Pilato.
``Ahora vayan y aseguren la tumba tanto como puedan.'' \bibverse{66}
Entonces ellos fueron y aseguraron la tumba, sellando la entrada con una
piedra y colocando soldados como guardas de ella.

\hypertarget{section-27}{%
\section{28}\label{section-27}}

\bibverse{1} El domingo temprano, durante el alba, María Magdalena y la
otra mujer llamada María, fueron a ver la tumba. \bibverse{2} De
repente, hubo un gran terremoto, pues un ángel del Señor bajó del cielo,
rodó la piedra, y se sentó sobre ella. \bibverse{3} Su rostro
resplandecía como un relámpago, y sus ropas eran blancas como la nieve.
\bibverse{4} Los guardias temblaban de miedo, y cayeron como si
estuvieran muertos.

\bibverse{5} El ángel dijo a las mujeres: ``¡No tengan miedo! Yo sé que
ustedes buscan a Jesús, el que fue crucificado. \bibverse{6} Él no está
aquí. Se ha levantado de entre los muertos, tal como dijo que lo haría.
Vengan y vean donde estuvo puesto el Señor. \bibverse{7} Ahora vayan
rápidamente y digan a sus discípulos que Jesús se ha levantado de entre
los muertos y que va delante de ustedes hacia Galilea. ¡Les prometo que
allí lo verán!''

\bibverse{8} Con miedo y a la vez muy felices, las mujeres se fueron
rápidamente de la tumba, e iban corriendo para decírselo a los
discípulos. \bibverse{9} De repente, Jesús llegó a su encuentro, y las
saludó. Ellas se lanzaron hacia él, se aferraron a sus pies y lo
adoraron. \bibverse{10} Entonces Jesús les dijo: ``¡No tengan miedo!
Vayan y díganle a mis hermanos que vayan a Galilea, y allí me verán.''

\bibverse{11} Cuando se fueron, algunos de los guardias fueron a la
ciudad y le contaron a los jefes de los sacerdotes todo lo que había
ocurrido. \bibverse{12} Después que los jefes de los sacerdotes se
hubieron reunido con los ancianos y hubieron elaborado un plan,
sobornaron a los soldados con una gran cantidad de dinero.

\bibverse{13} ``Digan así: `Sus discípulos vinieron por la noche y
robaron el cuerpo mientras dormíamos,'\,'' dijeron a los soldados.
\bibverse{14} ``Y si el gobernador llega a saber de esto, nosotros
hablaremos con él y ustedes no tendrán que preocuparse.''

\bibverse{15} Así que los soldados tomaron el dinero e hicieron lo que
les habían dicho. Esta historia se ha difundido entre el pueblo judío
hasta el día de hoy.

\bibverse{16} Pero los once discípulos fueron a Galilea, a la montaña
donde Jesús les había dicho que fueran. \bibverse{17} Cuando lo vieron,
lo adoraron, aunque algunos dudaban. \bibverse{18} Jesús vino donde
ellos estaban y les dijo: ``Se me ha entregado todo el poder del cielo y
de la tierra. \bibverse{19} Así que vayan y hagan discípulos entre las
personas de todas las naciones, bautizándolos en el nombre del Padre,
del Hijo y del Espíritu Santo. \bibverse{20} Enséñenles a seguir todos
los mandamientos que yo les he dado a ustedes. Recuerden, yo estoy
siempre con ustedes hasta el fin del mundo.''
