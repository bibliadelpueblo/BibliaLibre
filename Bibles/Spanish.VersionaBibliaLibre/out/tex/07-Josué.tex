\hypertarget{section}{%
\section{1}\label{section}}

\bibverse{1} Tras la muerte de Moisés, el siervo del Señor, el Señor
habló con Josué, el hijo de Nun, que había sido asistente de Moisés. Y
le dijo: \bibverse{2} ``Mi siervo Moisés ha muerto. Así que ve y cruza
el Jordán, tú y todo el pueblo, y entren en el país que yo le entrego a
los israelitas. \bibverse{3} Como se lo prometí a Moisés, dondequiera
que pongas un pie, será tierra que yo te daré,\footnote{\textbf{1:3}
  Refiriéndose no a Josué, sino a todo el pueblo.} \bibverse{4} desde el
desierto hasta el Líbano, y hasta el río Éufrates; toda la tierra de los
hititas, y hasta la costa oeste del Mar Mediterráneo. Este será su
territorio. \bibverse{5} Nadie podrá enfrentarse a ti mientras vivas.
Tal como lo hice con Moisés, estaré contigo. Nunca te dejaré y nunca te
abandonaré.

\bibverse{6} ¡Sé fuerte! ¡Sé valiente! Serás el líder del pueblo
mientras habiten la tierra que le prometí a sus antepasados.
\bibverse{7} Sé fuerte y muy valiente, y asegúrate de obedecer toda la
ley que mi siervo Moisés te ha ordenado seguir. No te apartes de ella,
ni a la derecha ni a la izquierda, para que tengas éxito en todo lo que
hagas. \bibverse{8} Sigue recordándole al pueblo la
ley.\footnote{\textbf{1:8} Literalmente: ``El rollo de la ley no se
  apartará de tu boca.''}Mediten en ella de día y de noche, para estés
seguro de hacer lo que es debido. Entonces tendrás éxito y prosperarás
en lo que hagas. \bibverse{9} No te olvides lo que te dije: ¡Sé fuerte!
¡Sé valiente! ¡No tengas miedo! ¡No te desanimes! Porque el Señor tu
Dios está contigo dondequiera que vayas.''

\bibverse{10} Entonces Josué le dio una orden a los líderes del pueblo:
\bibverse{11} ``Vayan por todo el campamento y díganle al pueblo:
``Preparen todo, porque en tres días cruzaremos el Jordán, para ir a
tomar la tierra que Dios les da.'''

\bibverse{12} Pero a las tribus de Rubén y Gad, y a la mitad de la tribu
de Manasés, Josué les dijo: \bibverse{13} ``Recuerden lo que Moisés, el
siervo del Señor, les ordenó hacer: ``El Señor su Dios les está dando
descanso, y les dará esta tierra.' \bibverse{14} Sus esposas, hijos, y
su ganado se quedarán aquí en la tierra que Moisés les
asignó\footnote{\textbf{1:14} Parece que, en general, las divisiones de
  la tierra se decidían echando suertes, por lo que esta parece ser la
  mejor palabra en este caso.}cuando estaban al oriente del Jordán. Pero
todos sus hombres armados, listos para la batalla, irán delante y
cruzarán primero para ayudarles, \bibverse{15} hasta que el Señor les de
descanso, como los ha dejado descansar a ustedes, y cuando hayan tomado
posesión de la tierra que el Señor les entrega. Entonces podrán regresar
y ocupar la tierra que Moisés les asignó al oriente del Jordán.''

\bibverse{16} Entonces ellos le dijeron a Josué:``Haremos todo lo que
nos has dicho, e iremos a donde nos envíes. \bibverse{17} Te
obedeceremos como obedecimos a Moisés en todo. Que el Señor Dios esté
contigo como estuvo con Moisés. \bibverse{18} Cualquiera que se rebele
contra lo que dices y no siga tus órdenes, y quienquieraque no obedezca
todo lo que dices, será ejecutado. ¡Sé fuerte! Sé valiente!''

\hypertarget{section-1}{%
\section{2}\label{section-1}}

\bibverse{1} Entonces Josué, hijo de Nun, envió en secreto a dos espías
de Sitín.\footnote{\textbf{2:1} El sitio donde acampaban los israelitas
  en ese momento.}Y les dijo: Vayan y exploren\footnote{\textbf{2:1}
  Literalmente, ``vayan a pie.''}la tierra, especialmente el territorio
de Jericó''. Entonces ellos se fueron, y se hospedaron en la casa de una
mujer llamada Rahab, que era una prostituta. Allí pasaron la noche.
\bibverse{2} Pero al rey de Jericó le informaron: ``Mira, unos
israelitas han venido aquí esta noche para espiar el territorio''.
\bibverse{3} Así que el rey de Jericó envió órdenes a Rahab, diciéndole:
``Entrega a los hombres que vinieron a visitarte y quédate en tu casa,
porque han venido a espiar todo nuestro país''.

\bibverse{4} La mujer se había llevado a los dos hombres y los había
escondido. Y le dijo a los mensajeros del rey:\footnote{\textbf{2:4}
  ``Mensajeros del rey'': implícito.}``Sí, es verdad, los hombres
vinieron a visitarme, pero no sabía de dónde eran. \bibverse{5} Se
fueron al atardecer, justo cuando se cerraba la puerta de la ciudad. No
tengo ni idea de adónde fueron. Si son rápidos, pueden ir tras ellos y
quizás los alcancen.'' \bibverse{6} (Ella los había llevado hasta el
tejado y los había escondido debajo de unos fardos de lino que tenía
allí.)

\bibverse{7} Los mensajeros del rey fueron tras los hombres por el
camino que lleva a la orilla del río Jordán. Tan pronto como los
perseguidores se fueron, la puerta de la ciudad se cerró tras de ellos.

\bibverse{8} Antes de que los espías se durmieran, Rahabsubió al tejado
para hablar con ellos. \bibverse{9} Les dijo: ``Sé que el Señor les ha
dado esta tierra. Todos estamos aterrorizados de ustedes. Todos los que
viven aquí temen en gran manera desde que ustedes llegaron.
\bibverse{10} Hemos oído cómo el Señor secó las aguas del Mar Rojo
delante ustedes cuando salieron de Egipto, y lo que le hicieron a los
dos reyes de los amorreos al Este del Jordán, Sijón y Og, a quienes
destruyeron por completo. \bibverse{11} Al oír esto, nuestro ánimo
decayó. A nadie le quedó ningún espíritu de lucha por causa de ustedes.
Porque el Señor su Dios es Dios arriba en el cielo y abajo en la tierra.
\bibverse{12} Así que ahora prométanme en el nombre del Señor que como
he actuado de buena fe con ustedes, entonces ustedes harán lo mismo por
mi familia. Denme una señal de que puedo confiar en ustedes,
\bibverse{13} y que apartarán a mi padre y a mi madre, así como a mis
hermanos y hermanas, y a todos los que forman parte de sus familias, y
que los salvarán de la muerte.''

\bibverse{14} ``¡Nuestras vidas por las vidas de ellos!'' le
respondieron los hombres. ``Si no le dices a nadie sobre esto, te
trataremos justa y fielmente cuando el Señor nos entregue la tierra.''

\bibverse{15} Entonces ella los hizo descender con una cuerda por la
ventana, pues la casa donde vivía estaba construida en el exterior de la
muralla de la ciudad.

\bibverse{16} ``Corran hacia las colinas'', les dijo. ``Así quienes los
persiguen no los encontrarán. Quédense allí tres días hasta que ellos se
hayan ido a casa, y entonces podrán seguir su camino.''

\bibverse{17} Los hombres le habían dicho: ``Seremos liberados de la
promesa que nos hiciste jurar, \bibverse{18} a menos que cuando entremos
en esta tierra, cuelgues un cordón escarlata en la ventana por la que
nos bajaste. Debes reunir en la casa a tu padre, tu madre y tus
hermanos, y a toda la familia. \bibverse{19} Si alguien sale de tu casa
y es asesinado, es su culpa y no somos responsables de su muerte. Pero
si alguien pone una mano sobre alguien que está dentro de su casa,
asumimos toda la responsabilidad de su muerte. \bibverse{20} Pero si le
dices a alguien sobre esto entonces seremos liberados de la promesa que
nos hiciste jurar.''

\bibverse{21} ``Estoy de acuerdo, que sea como ustedes lo han dicho,''
respondió. Y así los envió, y colgó un cordón escarlata en su ventana.

\bibverse{22} Ellos subieron a las colinas y se quedaron allí tres días.
Los hombres que los perseguían buscaron por todo elcamino pero no
pudieron encontrarlos, así que volvieron a casa. \bibverse{23} Entonces
los dos hombres regresaron. Bajaron de las colinas y cruzaron el Jordán.
Fueron a ver a Josué y le explicaron todo lo que les había pasado.

\bibverse{24} ``El Señor ha puesto esta tierra en nuestras manos'', le
aseguraron. ``¡Toda la gente que vive allí se muere de espanto por causa
de nosotros!''

\hypertarget{section-2}{%
\section{3}\label{section-2}}

\bibverse{1} A la mañana siguiente, Josué y los israelitas salieron de
Sitín y llegaron a la orilla del Jordán. Allí pasaron la noche antes de
cruzar.

\bibverse{2} Tres días después, los líderes del pueblo pasaron por el
campamento \bibverse{3} diciéndoles: ``Cuando vean el Arca del Pacto del
Señor su Dios siendo llevada por los sacerdotes, los levitas, deben
salir del lugar donde estén y seguirla. \bibverse{4} Así sabrán qué
camino tomar, ya que no han estado aquí antes. Mantengan una distancia
de 3.000 pies entre ustedes y el Arca. ¡No se acerquen!''

\bibverse{5} Entonces Josué le dijo al pueblo:``Asegúrense de estar
puros\footnote{\textbf{3:5} ``Puros'': un concepto de pureza religiosa
  logrado a través de rituales específicos.}, porque mañana el Señor
hará cosas asombrosas entre ustedes.''

\bibverse{6} Josuéles habló a los sacerdotes:\footnote{\textbf{3:6} Se
  presume que se refiere al día siguiente.}``Tomen el Arca del Pacto y
vayan delante del pueblo''. Entonces ellos levantaron el Arca del Pacto
y marcharon delante del pueblo.

\bibverse{7} El Señor le dijo a Josué: ``Lo que hago hoy te confirmará
como gran líder a la vista de todos los israelitas, para que se den
cuenta de que así como estuve con Moisés, estaré contigo. \bibverse{8}
Dile a los sacerdotes que llevan el Arca del Pacto: `Cuando lleguen a
cruce del Jordán, den unos pasos hacia el agua y luego deténganse
allí.'\,''

\bibverse{9} Entonces Josué les dijo a los israelitas: ``Vengan aquí y
escuchen lo que el Señor su Dios les quiere decir. \bibverse{10} Así
sabrán que el Dios vivo está aquí con ustedes'', les dijo. ``Tengan la
seguridad de que Él expulsará delante de ustedes a los cananeos, a los
hititas, a los heveos, a los ferezeos, a los guirgaseos, a los amorreos
y a los jebuseos. \bibverse{11} Sólo recuerden: el Arca del Pacto del
Señor de toda la tierra cruzará el Jordánantes que ustedes.
\bibverse{12} Elijan doce hombres de las tribus de Israel, uno por
tribu.\footnote{\textbf{3:12} Este verso parece fuera de lugar y se
  repite en 4:2.} \bibverse{13} En el momento en que los sacerdotes que
llevan el Arca pisen el agua, el río dejará de fluir y el agua se
amontonará.'' \bibverse{14} Así que el pueblo recogió el campamento y se
dispuso a cruzar el Jordán, y los sacerdotes que llevaban el Arca iban
adelante.

\bibverse{15} Como era la temporada de la cosecha, el Jordán estaba
lleno de agua, y sus orillas se desbordaban. Pero tan ponto como los
sacerdotes que llevaban el Arca entraron en el agua, el río dejó de
fluir. \bibverse{16} El agua se amontonó mucho más arriba, en la ciudad
de Adán, cerca de Zaretán, mientras que río abajo ya no fluía más agua
hacia el Mar Muerto. De este modo el pueblo pudo cruzar el río, frente a
Jericó. \bibverse{17} Los sacerdotes que llevaban el Arca se mantuvieron
firmes y en pie en el lecho seco del río Jordán mientras todo el pueblo
pasaba. Y se quedaron allí hasta que todos hubieron cruzado y se
encontraban en tierra seca.

\hypertarget{section-3}{%
\section{4}\label{section-3}}

\bibverse{1} Cuando toda la nación terminó de cruzar el Jordán, el Señor
le dijo a Josué: \bibverse{2} ``Escoge doce hombres del pueblo, uno por
tribu, \bibverse{3} y diles:`Recojan doce piedras del medio del Jordán,
donde están los sacerdotes. Luego llévenlas y déjenlas en el lugar donde
acamparán esta noche.'\,'' \bibverse{4} Entonces Josué mandó llamar a
los doce hombres que había elegido, uno de cada tribu, \bibverse{5} y
les dijo: ``Vuelvan y entren hasta la mitad del Jordán, justo delante
del Arca del Pacto del Señor su Dios, y cada uno de ustedes, elija una
piedra y llévenla sobre sus hombros, uno por cada una de las tribus de
Israel. \bibverse{6} Este será un monumento entre ustedes, para que un
día, cuando sus hijos pregunten: ``¿Qué significan estas piedras?'
\bibverse{7} ustedes puedan responderles: `Son un recordatorio de cuando
el río Jordán dejó de fluir mientras el Arca del Pacto del Señor cruzaba
el río''. Cuando el arca cruzó, el agua se detuvo. Estas piedras son un
recordatorio eterno para el pueblo de Israel.'\,''

\bibverse{8} Entonces el pueblo de Israel hizo lo que Josué les dijo.
Los hombres recogieron doce piedras del medio del Jordán como el Señor
le había ordenado a Josué. Las llevaron al lugar donde acamparon durante
la noche y las colocaron allí, una por cada una de las tribus de Israel.
\bibverse{9} Josué también colocó doce piedras en medio del Jordán justo
donde los sacerdotes que llevaban el Arca del Pacto se habían quedado en
pie, y siguen allí hasta el día de hoy.

\bibverse{10} Los sacerdotes que llevaban el Arca permanecieron de pie
en medio del Jordán hasta que todo se hizo tal como el Señor le había
dicho al pueblo, todo lo que Moisés le había dicho a Josué.\footnote{\textbf{4:10}
  Esta repetición y la mención adicional de Moisés llevan a algunos
  comentaristas a creer que el autor estaba utilizando múltiples fuentes
  para su relato.} Y el pueblo cruzó rápidamente. \bibverse{11} Después
que todo el pueblo cruzó, pudieron ver a los sacerdotes llevando el Arca
del Señor.\footnote{\textbf{4:11} O ``Una vez que todo el pueblo hubo
  cruzado, el Arca del Señor fue llevada por los sacerdotes e iban
  delante del pueblo.''}

\bibverse{12} Los hombres armados de las tribus de Rubén y Gad, y la
mitad de la tribu de Manasés cruzaron, yendo a la cabeza del pueblo de
Israel, como Moisés lo había estipulado. \bibverse{13} Fueronunos 40.000
hombres, armados y listos para la batalla, los que cruzaron en presencia
del Señor a las llanuras de Jericó.

\bibverse{14} Ese día el Señor confirmó a Josué como gran líder a la
vista de todos los israelitas, y ellos se maravillaron de él como lo
habían hecho con Moisés.

\bibverse{15} El Señor le había dicho a Josué: \bibverse{16} ``Dile a
los sacerdotes que llevan el Arca del Testimonio\footnote{\textbf{4:16}
  El Arca también recibió esta designación, ya que contenía los diez
  mandamientos, el testimonio o el testimonio de Dios para su pueblo.}
que salgan del Jordán.'' \bibverse{17} Así que Josué les dijo a los
sacerdotes, ``Salgan del Jordán.'' \bibverse{18} Los sacerdotes salieron
del Jordán llevando el Arca del Testimonio, y tan pronto como sus pies
tocaron tierra seca, las aguas del Jordán volvieron a donde habían
estado, desbordando sus orillas como antes.

\bibverse{19} El pueblo salió del Jordán y acampó en Gilgal, al oriente
de Jericó, el décimo día del primer mes.\footnote{\textbf{4:19} Finales
  de marzo o principios de abril.} \bibverse{20} Josué colocó en Gilgal
las doce piedras que habían sido tomadas del Jordán. \bibverse{21} Y
entonces les dijo a los israelitas: ``El día que sus hijos les pregunten
a sus padres: `¿Qué significan estas piedras?' \bibverse{22} ustedes
podrán explicarles: `Aquí es donde los israelitas cruzaron el Jordán en
tierra seca.' \bibverse{23} Porque el Señor su Dios hizo que el río
Jordán se secara delante de ustedes para que todos pudieran cruzarlo,
así como lo hizo en el Mar Rojo, que se secó para que todos pudiéramos
cruzarlo. \bibverse{24} Y lo hizo para que todos en la tierra supieran
cuán poderoso es el Señor, y para que ustedes se maravillaran del Señor
su Dios para siempre.''

\hypertarget{section-4}{%
\section{5}\label{section-4}}

\bibverse{1} Cuando todos los reyes amorreos al Oeste del Jordán y todos
los reyes cananeos de la costa mediterránea oyeron cómo el Señor había
secado las aguas del río Jordán para que los israelitas pudieran
cruzarlo, su ánimo decayó y ya no tenían ningún espíritu de lucha para
enfrentarse a los israelitas.

\bibverse{2} En ese momento, el Señor le dijo a Josué:``Haz cuchillos de
piedra y circuncida a la nueva generación\footnote{\textbf{5:2}
  ``Circuncida a la nueva generación'': literalmente, ``circuncidar de
  nuevo o por segunda vez''. No se trataba de una orden de repetir la
  circuncisión en los ya circuncidados, sino de circuncidar a la nueva
  generación que había nacido durante el tiempo en el desierto.} de
israelitas.'' \bibverse{3} Josué mandó a hacer cuchillos de piedra y
todos los israelitas varones fueron circuncidados en el lugar que más
adelante se conoció como ``la colina de los prepucios.'' \bibverse{4} Y
esta es la razón por la que Josué los hizo circuncidar a todos: todos
los que salieron de Egipto, los hombres en edad de luchar, habían muerto
en el viaje, en medio del desierto, después del Éxodo. \bibverse{5}
Todos habían sido circuncidados cuando salieron de Egipto, pero los
nacidos en el viaje desde entonces no lo habían hecho. \bibverse{6}
Durante cuarenta años los israelitas viajaron por el desierto hasta que
todos los hombres en edad de luchar cuando salieron de Egipto ya habían
muerto, porque no habían hecho lo que el Señor les había dicho que
hicieran. Así que el Señor había prometido que no les dejaría ver la
tierra que había prometido a sus antepasados que nos daría, una tierra
que fluye leche y miel. \bibverse{7} El Señor los reemplazó con sus
hijos, y estos fueron los que Josué circuncidó. No estaban circuncidados
porque no habían sido circuncidados en el camino. \bibverse{8} Una vez
que todos fueron circuncidados, se quedaron en el campo hasta que se
recuperaron.

\bibverse{9} Entonces el Señor le dijo a Josué: ``Hoy he quitado de
todos ustedes la desgracia de Egipto''\footnote{\textbf{5:9} No se
  define explícitamente en qué consistía esta desgracia. Algunos lo
  relacionan con la esclavitud en Egipto, pero lo más probable es que
  esté relacionada con la rebelión de los israelitas en Cades (Números
  14) y la decisión de Dios de no permitir que esa generación entre en
  la Tierra Prometida. Inicialmente había amenazado con destruirlos,
  pero Moisés intervino, mencionando lo mucho que esto complacería a los
  egipcios (Números 14:13). La desgracia sería entonces que los
  israelitas habían fallado a Dios al rebelarse contra él, y Dios sería
  percibido por los egipcios y otros como incapaz de cumplir su promesa.
  El acto de la circuncisión (una señal del favor de Dios) cerró el
  círculo de la situación para volver al Éxodo, y ahora a la entrada en
  la Tierra Prometida.}. Así que ese lugar se ha llamado Gilgal hasta
hoy. \bibverse{10} Los israelitas acamparon en Gilgal y celebraron allí
la Pascua en la tarde del día 14 del primer mes. \bibverse{11} A partir
del día siguiente, comenzaron a comer los productos de la tierra: pan
sin levadura y grano asado. \bibverse{12} El mismo día en que comenzaron
a comer el producto de la tierra no hubo más maná. Desde ese momento,
los israelitas no volvieron a comer maná, y en cambio comíanlo que la
tierra de Canaán producía.

\bibverse{13} Un día, cuando Josué estaba cerca de Jericó, levantó la
vista y vio a un hombre parado frente a él con una espada desenvainada
en su mano. Josué se acercó a él y le preguntó: ``¿Estás a favor o en
contra de nosotros?''

``Ninguna de las dos cosas'', dijo el hombre. ``Soy el comandante del
ejército del Señor. ¡Ahora estoy aquí!''

\bibverse{14} Josué cayó al suelo con el rostro en alto. Y entonces
dijo: ``¿Qué órdenes tiene mi señor para su siervo?''

\bibverse{15} El comandante del ejército del Señor le dijo a Josué:
``Quítate las sandalias, porque el lugar donde estás es tierra santa''.
Y Josué lo hizo.

\hypertarget{section-5}{%
\section{6}\label{section-5}}

\bibverse{1} Las puertas de Jericó se cerraron y se prohibieron por
culpa de los israelitas. No se le permitía a nadie entrar o salir.
\bibverse{2} Pero el Señor le dijo a Josué: ``Te entrego la ciudad de
Jericó, junto con su rey y su ejército de guerreros. \bibverse{3} Marcha
alrededor de la ciudad con tus hombres armados una vez al día durante
seis días. \bibverse{4} Siete sacerdotes irán delante del Arca, cada uno
con un cuerno de carnero. El séptimo día, marchen siete veces alrededor
de la ciudad, con los sacerdotes soplando sus cuernos de carnero.
\bibverse{5} Cuando escuchen un largo golpe en los cuernos de los
carneros, todos darán un grito muy fuerte. Las murallas de la ciudad se
derrumbarán, y todo el mundo podrá entrar.''

\bibverse{6} Así que Josué, hijo de Nun, mandó a llamar a los
sacerdotes, y les dijo: ``Levanten el Arca del Pacto, y que siete
sacerdotes lleven siete cuernos de carnero y vayan delante del Arca del
Señor.'' \bibverse{7} Luego le dijo al pueblo: ``¡Muévanse! ¡Marchen
alrededor de la ciudad con los hombres armados delante del Arca del
Señor!'' \bibverse{8} Cuando Josué terminó de hablarle al pueblo, los
siete sacerdotes que llevaban los siete cuernos de carnero en presencia
del Señor, comenzaron a soplar los cuernos, con el Arca tras ellos.
\bibverse{9} Algunos de los hombres armados marchaban delante de los
sacerdotes haciendo sonar los cuernos, mientras que otros seguían el
Arca, haciendo sonar los cuernos continuamente. \bibverse{10} Sin
embargo, Josué les ordenó: ``No griten, ni hablen. No digan nada hasta
que yo les de la orden de gritar, ¡solo entonces griten!'' \bibverse{11}
Así que el Arca del Señor fue llevada por todo alrededor de la ciudad, y
dieron una vuelta. Luego regresaron al campamento y pasaron la noche
allí.

\bibverse{12} Josué se levantó temprano en la mañana, y los sacerdotes
recogieron el Arca del Señor. \bibverse{13} Los siete sacerdotes con los
siete cuernos de carnero iban delante del Arca del Señor, haciendo sonar
los cuernos. Los hombres armados iban adelante ellos y detrás del Arca
del Señor, haciendo sonar continuamente los cuernos. \bibverse{14} Así
que el segundo día marcharon alrededor de la ciudad, dándole una vuelta,
y volvieron al campamento. E hicieron esto por un total de seis días.

\bibverse{15} El séptimo día, se levantaron al amanecer y marcharon
alrededor de la ciudad de la manera habitual, salvo que este día le
dieron siete vueltas a la ciudad. \bibverse{16} La séptima vez, cuando
los sacerdotes soplaron los cuernos, Josué le dijo al pueblo: ``¡Griten,
porque hoy el Señor nos ha dado la ciudad! \bibverse{17} La ciudad y
todo lo que hay en ella será apartado para el Señor y
destruido.\footnote{\textbf{6:17} El término utilizado significa que
  está ``apartado'', ``consagrado'' o ``dedicado'' al Señor, lo que en
  este caso significaba que nadie debía beneficiarse de nada en Jericó:
  todo debía ser destruido. Esto es similar a la idea de lo ``sagrado''
  y lo ``santo'' -- dedicado únicamente a Dios. En cierto modo, la
  ``separación'' de Jericó era similar a una prohibición: pertenecía
  sólo a Dios.} Sólo Rahab, la prostituta, y todos los que estén con
ella en su casa se salvarán, porque ella escondió a los espías que
enviamos. \bibverse{18} Pero no se acerquen a ninguna de las cosas que
se han apartado para el Señor, porque si se llevan algo, ustedes también
serán destruidos, y además provocarán un desastre en el campamento de
Israel. \bibverse{19} Así que toda la plata y el oro, y todo lo que sea
de bronce y hierro, son sagrados para el Señor y deben ser puestos en el
tesoro del Señor.''

\bibverse{20} Tan pronto como oyeron el sonido de las bocinas, el pueblo
dio un fuerte grito,\footnote{\textbf{6:20} El texto hebreo dice que el
  pueblo gritó y que los cuernos sonaron, y que cuando oyeron los
  cuernos, el pueblo gritó. Como se considera que hubo un solo
  acontecimiento, la repetición de los cuernos y el grito parece
  superflua.} y las murallas de la ciudad se derrumbaron. Los hombres
entraron de inmediato y capturaron la ciudad. \bibverse{21} Destruyeron
todo lo que había en la ciudad: hombres y mujeres, jóvenes y ancianos,
ganado, ovejas y burros, todos fueron asesinados con espada.

\bibverse{22} Josué lehabía dicho a los dos hombres que habían ido a
explorar la tierra: ``Vayan a la casa de la prostituta Rahab y sáquenla
junto con toda su familia, tal como se lo prometieron'' \bibverse{23}
Así que los espías fueron y sacaron a Rahab, a su padre y a su madre, y
a todos los que estaban con ella. Sacaron a toda la familia y los
llevaron a un lugar cerca del campamento israelita.

\bibverse{24} Los israelitas quemaron la ciudad y todo lo que había en
ella, excepto la plata y el oro, y todo lo que estaba hecho de bronce y
hierro, lo cual pusieron en el tesoro de la casa del Señor.
\bibverse{25} Josué salvó a Rahab, la prostituta, y a su familia porque
escondió a los hombres que Josué había enviado a espiar a Jericó. Y ella
vive entre los israelitas hasta el día de hoy.

\bibverse{26} En ese momento Josué declaró una maldición, diciendo:
``Maldito sea ante el Señor todo aquel que intente reconstruir esta
ciudad de Jericó. Él pone sus cimientos a costa de su hijo primogénito;
él pone sus puertas a costa de su hijo menor.''

\bibverse{27} Y el Señor estaba con Josué, y su fama se extendió por
toda la tierra.

\hypertarget{section-6}{%
\section{7}\label{section-6}}

\bibverse{1} Sin embargo, los israelitas no habían sido fieles respecto
a las cosas apartadas para el Señor. Acán había tomado algunas de ellas,
lo cual hizo que el Señor se enojara mucho con los israelitas. Acán era
hijo de Carmi, hijo de Zabdi, hijo de Zera, de la tribu de Judá.
\bibverse{2} Josué envió hombres desde el campamento cerca\footnote{\textbf{7:2}
  ``El campamento cercano'': implícito. Evidentemente, no se trata de
  hombres de Jericó.} de Jericó a la ciudad de Hai, que está cerca de
Bet-avén, al este de Bet-el. ``Vayan y exploren la tierra'', les dijo.
Así que fueron y exploraron alrededor de Hai. \bibverse{3} Cuando
regresaron, le dijeron a Josué: ``No necesitamos a todo el ejército. Dos
o tres mil hombres serán suficientes para atacar la ciudad deHai. No te
molestes en enviarlos a todos, pues sólo hay unos pocos.'' \bibverse{4}
Así que alrededor de tres mil hombres fueron a luchar, pero fueron
golpeados por los hombres de Hai y tuvieron que irse huyendo.
\bibverse{5} Los hombres de Hai mataron a unos treinta y seis de ellos,
persiguiendo a los israelitas desde la puerta del pueblo hasta que
fueron derrotados,\footnote{\textbf{7:5} ``Hasta que fueron
  derrotados'': o, ``a las canteras.''} matándolos mientras descendían.
Los israelitas se asustaron y perdieron todo su espíritu para luchar.

\bibverse{6} Entonces Josué rasgó sus ropas\footnote{\textbf{7:6} Un
  símbolo de dolor.}y cayó de bruces al suelo delante del Arca del Señor
hasta la noche. Los ancianos hicieron lo mismo, y él y los ancianos se
echaron polvo en la cabeza. \bibverse{7} Josué gritó: ``¿Por qué, oh por
qué, Señor Dios, nos trajiste al otro lado del río Jordán sólo para
entregarnos a los amorreos para que nos destruyan? ¡Deberíamos habernos
conformado con quedarnos al otro lado del Jordán! \bibverse{8} Perdona,
Señor, pero ¿qué puedo decir ahora que Israel ha dado la espalda y ha
huido de sus enemigos? \bibverse{9} Los cananeos y todos los que viven
en la tierra vendrán y nos rodearán y nos aniquilarán tan completamente
que incluso nuestro nombre será olvidado. ¿Y qué pasará con tu gran
nombre?''

\bibverse{10} Pero el Señor le respondió a Josué: ``¡Levántate! ¿Qué
crees que haces acostado sobre tu rostro de esa manera? \bibverse{11}
Israel ha pecado y ha quebrantado\footnote{\textbf{7:11} Literalmente,
  ``transgredieron'', en el sentido de salirse de lo prometido. La
  palabra real significa ``pasar por encima'', o ``cruzar'', y es
  exactamente la misma palabra que usa Josué cuando pregunta por qué el
  Señor los llevó al otro lado del Jordán. Así que, en un paralelismo
  del lenguaje moderno, Josué pregunta por qué el Señor se molestó en
  ayudar a los israelitas a ``cruzar'' el río, y el Señor le responde
  que lo han ``cruzado'' (o incluso ``traicionado'').}mi acuerdo, el
cual les ordené cumplir. Se han llevado algunas de las cosas que me
habían apartado; han actuado con deshonestidad; han escondido los
objetos robados junto con sus propias pertenencias. \bibverse{12} Por
eso los israelitas no pueden hacer frente a sus enemigos. Por eso dan la
espalda y huyen de sus enemigos, y han sido apartados para la
destrucción.\footnote{\textbf{7:12} Se hicieron susceptibles de ser
  destruidos porque habían tomado cosas que debían ser destruidas. Véase
  6:18.} No podrán hacer frente a sus enemigos hasta que no hayan
quitado de entre ustedes lascosas apartadas para la destrucción.
\bibverse{13} Levántate y asegúrate de que el pueblo esté puro. Diles:
``Purifíquense para mañana, porque así lo dice el Señor, el Dios de
Israel: `Hay cosas reservadas para mí que están escondidas entre
ustedes, pueblo de Israel. No podrán enfrentarse a sus enemigos hasta
que tales cosas sean eliminadas por completo. \bibverse{14} Por la
mañana, te presentarás, tribu por tribu. La tribu que elija\footnote{\textbf{7:14}
  Parece que la decisión se tomó echando suertes.} el Señor se
presentará clan por clan. El clan que el Señor elija se presentará
familia por familia. La familia que el Señor elija se presentará hombre
por hombre. \bibverse{15} El que sea sorprendido en posesión de lo que
fue apartado para la destrucción, será quemado con fuego,\footnote{\textbf{7:15}
  Esto no significa que el culpable fuera quemado vivo, como queda claro
  en los versículos siguientes.} junto con todo lo que es suyo, porque
rompió el acuerdo del Señor y cometió un acto terrible en Israel.'\,''

\bibverse{16} Josué se levantó temprano a la mañana siguiente y llamó a
Israel al frente, tribu por tribu. \bibverse{17} La tribu de Judá fue
elegida. Los clanes de Judá se presentaron y los zerahitas fueron
elegidos. El clan de los zerahitas se presentó y se eligió a la familia
de Zabdi.\footnote{\textbf{7:17} ``La familia de Zabdi'' (algunos
  manuscritos hebreos). Otros van directamente al individuo elegido, e
  identifican a Zimri, aunque el versículo siguiente lo hace
  problemático.} \bibverse{18} La familia de Zabdi se presentó, y Acán,
hijo de Carmi, hijo de Zabdi, hijo de Zera, de la tribu de Judá, fue
elegido.

\bibverse{19} Entonces Josué le dijo a Acán: ``Hijo mío, honra a Jehová,
el Dios de Israel, y confiesa. Dime lo que has hecho. No me lo
ocultes.''

\bibverse{20} ``¡Es verdad! respondió Acán. ``He pecado contra el Señor,
el Dios de Israel. Yo lo he hecho. \bibverse{21} A Entre el botín vi un
hermoso manto de Babilonia, doscientos siclos de plata y un lingote de
oro que pesaba cincuenta siclos.\footnote{\textbf{7:21} Unas cinco
  libras de plata y más de una libra de oro.} En realidad deseaba
tenerlos, así que los tomé. Están escondidos en el suelo dentro de mi
tienda, con la plata enterrada más profundamente.'' \bibverse{22} Josué
envió hombres que corrieron a revisar la tienda. Encontraron lo que
estaba escondido, con la plata enterrada aún más profundamente.
\bibverse{23} Los hombres sacaron las cosas de la tienda y se las
llevaron a Josué y a todos los israelitas. Allí las extendieron ante el
Señor.

\bibverse{24} Entonces Josué, con todos los israelitas, tomó a Acán,
hijo de Zera, la plata, el manto y el lingote de oro, junto con sus
hijos e hijas, su ganado, sus asnos, sus ovejas y su tienda -- todo lo
que tenía -- ylos llevó al valle de Acor.\footnote{\textbf{7:24}
  Significa ``valle de los problemas.''} \bibverse{25} Entonces Josué
dijo a Acán: ``¿Por qué nos has traído tantos problemas? Hoy el Señor te
traerá problemas''. Todos los israelitas apedrearon a Acán. Luego,
cuando apedrearon a los demás, quemaron sus cuerpos. \bibverse{26}
Colocaron sobre él un gran montón de piedras que aún perdura. El Señor
ya no estaba enojado. Por eso el lugar fue llamado el Valle de Acor
desde entonces.

\hypertarget{section-7}{%
\section{8}\label{section-7}}

\bibverse{1} El Señor le dijo a Josué: ``¡No tengas miedo ni te
desanimes! Toma a todos los combatientes contigo y ataca a Hai, porque
te voy a entregar al rey de Hai, a su pueblo, a su ciudad y a su tierra.
\bibverse{2} Harás con Hai y su rey lo mismo que hiciste con Jericó y su
rey. Sin embargo, esta vez podrán quedarse con el botín y el ganado.
Preparen una emboscada detrás de la ciudad.''

\bibverse{3} Así que Josué y todo el pueblo se prepararon para atacar a
Hai. Escogió a treinta mil de sus mejores combatientes y los envió de
noche. \bibverse{4} Les ordenó: ``Pongan una emboscada detrás de la
ciudad, no muy lejos. Todos deben estar preparados. \bibverse{5} Cuando
yo y el resto de los hombres que me acompañan nos acerquemos a la
ciudad, los defensores saldrán corriendo a atacarnos como antes, y
nosotros huiremos de ellos. \bibverse{6} Nos perseguirán mientras los
alejamos de la ciudad, porque se dirán unos a otros: ``Están huyendo de
nosotros igual que antes''. \bibverse{7} Mientras nosotros huimos de
ellos, ustedes se levantarán de sus posiciones de emboscada y tomarán la
ciudad, pues el Señor Dios se las entregará. \bibverse{8} Una vez que
hayan capturado la ciudad, préndanle fuego, como lo ha ordenado el
Señor. Sigan sus órdenes.''

\bibverse{9} Josué los envió, y fueron a tender una emboscada entre
Betel y el lado occidental de Hai. Pero esa noche Josué se quedó con el
pueblo en el campamento. \bibverse{10} A la mañana siguiente, Josué se
levantó temprano, reunió al pueblo y subió a atacar a Hai, acompañado
por los ancianos de Israel. \bibverse{11} Todos los combatientes que
estaban con él se acercaron al frente de la ciudad y acamparon allí, en
el lado norte, con un valle entre ellos y Hai. \bibverse{12} Tomó unos
cinco mil hombres y los puso en emboscada entre Betel y Hai, al oeste de
la ciudad. \bibverse{13} Así que el ejército tomó sus posiciones: el
ejército principal al norte de la ciudad, y la emboscada al oeste. Josué
fue esa noche al valle.

\bibverse{14} En cuanto el rey de Hai se percató de la situación, salió
de madrugada con todos sus hombres para atacar a los israelitas en el
mismo lugar donde lo habían hecho antes, en un lugar que daba al valle
del Jordán.\footnote{\textbf{8:14} Al Este de la ciudad.}Él no sabía de
la emboscada que les esperaba al otro lado de la ciudad. \bibverse{15}
Josué y los israelitas se dejaron llevar y huyeron en dirección al
desierto. \bibverse{16} Todos los hombres de la ciudad fueron llamados a
salir a perseguirlos, y mientras perseguían aJosué,se alejaron de la
ciudad. \bibverse{17} No quedó un solo hombre en Hai y Betel\footnote{\textbf{8:17}
  Es de suponer que los hombres de la cercana ciudad de Betel se unieron
  a lo que creían que era la derrota de los israelitas. La Septuaginta
  omite la mención de Betel.} que no saliera a perseguir a los
israelitas. Así dejaron la ciudad indefensa mientras perseguían a los
israelitas. \bibverse{18} Entonces el Señor le dijo a Josué: ``Levanta
la lanza que tienes en la mano y apunta a Hai, porque te la voy a
entregar''. Así que Josué levantó la lanza en su mano y apuntó a la
ciudad.

\bibverse{19} En cuanto vieron esta señal, los hombres que estaban
emboscados se levantaron y entraron corriendo en la ciudad. La
capturaron y rápidamente le prendieron fuego. \bibverse{20} Cuando los
hombres de Hai miraron hacia atrás, vieron el humo que se elevaba hacia
el cielo desde la ciudad. No tenían adonde huir, porque los israelitas
que habían estado huyendo hacia el desierto se volvieron ahora contra
sus perseguidores. \bibverse{21} Cuando Josué y los israelitas vieron
que el grupo de la emboscada había capturado la ciudad y que de ella
salía humo, se volvieron y atacaron a los hombres de Hai. \bibverse{22}
Los hombres de la emboscada también salieron de la ciudad y los
atacaron, por lo que quedaron atrapados en la mitad, con los israelitas
a ambos lados. Los israelitas los redujeron, y ni un solo hombre
sobrevivió o pudo escapar. \bibverse{23} Sólo el rey de Hai fue
capturado vivo, y fue llevado ante Josué.

\bibverse{24} Cuando los israelitas terminaron de matar a los hombres de
Hai que los habían perseguido hacia el desierto -- unavez que todos
habían sido pasados por la espada --, todo el ejército israelita regresó
a la ciudad y mató a todos los que vivían allí. \bibverse{25} Los
muertos de aquel día, contando hombres y mujeres, fueron doce mil,
quienes eran todos los habitantes de Hai. \bibverse{26} Porque Josué
había continuado con su lanza hasta que todo el pueblo de Hai había sido
destruido.\footnote{\textbf{8:26} ``Destruido'': La palabra para
  destrucción aquí es la misma que se usa para la destrucción de Jericó:
  ``consagrado al Señor''. Véase la nota a pie de página de 6:17.}
\bibverse{27} Los israelitas sólo se llevaron el ganado y el botín de la
ciudad, como el Señor le lo había ordenado a Josué. \bibverse{28} Así
que Josué quemó la ciudad de Hai, convirtiéndola definitivamente en un
montón de ruinas donde nadie vive hasta el día de hoy. \bibverse{29}
Mató al rey de Hai y colgó su cuerpo en un árbol hasta la noche. Cuando
se puso el sol, Josué ordenó que bajaran el cuerpo. Lo arrojaron frente
a la entrada de la puerta de la ciudad y amontonaron sobre él un montón
de piedras que todavía está allí.

\bibverse{30} Luego Josué construyó un altar en el monte Ebal para el
Señor, el Dios de Israel. \bibverse{31} Hizo lo que Moisés, el siervo
del Señor, le había dicho a los israelitas que hicieran, según consta en
el libro de la Ley de Moisés: un altar de piedras sin cortar que nadie
hubiera trabajado con herramientas de hierro. Sobre el altar hicieron
holocaustos y sacrificios de comunión al Señor. \bibverse{32} Allí, en
presencia de los israelitas, Josué inscribió en las piedras una copia de
la Ley de Moisés. \bibverse{33} Todos los israelitas, los ancianos, los
oficiales y los jueces se colocaron en dos grupos uno frente al otro,
con los sacerdotes, los levitas y el Arca del Acuerdo del Señor entre
ellos. La mitad de ellos se colocó frente al monte Gerizim, y la otra
mitad frente al monte Ebal, tal como Moisés había ordenado, para la
bendición del pueblo esta primera vez.\footnote{\textbf{8:33} La primera
  bendición al entrar en la Tierra Prometida.} \bibverse{34} Entonces
Josué leyó en voz alta toda la Ley: todas las bendiciones y maldiciones
registradas en el libro de la Ley. \bibverse{35} Josué leyó cada palabra
de las instrucciones de Moisés a toda la asamblea israelita, incluidas
las mujeres, los niños y los extranjeros que vivían entre ellos.

\hypertarget{section-8}{%
\section{9}\label{section-8}}

\bibverse{1} Todos los reyes al oeste del Jordán se enteraron de lo
sucedido. Entre ellos estaban los reyes de los hititas, amorreos,
cananeos, ferezeos, heveos y jebuseos que vivían en la región de las
colinas, en las tierras bajas y a lo largo de la costa hasta el Líbano.
\bibverse{2} Así que se reunieron para luchar juntos como un ejército
unido contra Josué y los israelitas.

\bibverse{3} Pero cuando el pueblo de Gabaón se enteró de lo que Josué
le había hecho a Jericó y a Hai, \bibverse{4} decidieron un plan astuto.
Enviaron mensajeros a Josué, con sus burros que llevaban monturas
desgastadas y cargaban odres viejos que estaban rotos y remendados.
\bibverse{5} Se pusieron sandalias gastadas que habían sido remendadas y
llevaban ropas viejas. Todo su pan estaba seco y enmohecido.\footnote{\textbf{9:5}
  O ``se desmoronó.''} \bibverse{6} Se dirigieron a Josué en el
campamento de Gilgal y le dijeron a él y a los hombres de Israel:
``Hemos venido de una tierra lejana, así que por favor hagan un
tratado\footnote{\textbf{9:6} ``Hacer un tratado'': literalmente,
  ``cortar un pacto''. A los israelitas se les permitía hacer tratados
  con pueblos de tierras lejanas, pero no con los cercanos. Véase
  Deuteronomio 7:1-2 Deuteronomio 20:10-15.}con nosotros.''

\bibverse{7} Pero los israelitas dijeron a los heveos: ``Tal vez ustedes
vivan cerca. Si es así, no podemos hacer un tratado con ustedes.''

\bibverse{8} ``Somos sus siervos'', respondieron.

``Pero ¿quiénes son ustedes? ¿De dónde vienen?'' preguntó Josué.

\bibverse{9} ``Nosotros tus siervos hemos venido de una tierra lejana'',
respondieron. ``Porque hemos oído hablar de la reputación del Señor, su
Dios, y de todo lo que hizo en Egipto, \bibverse{10} y de lo que hizo a
los dos reyes amorreos al este del Jordán: a Sejón, rey de Hesbón, y a
Og, rey de Basán, que gobernaba en Astarot.\footnote{\textbf{9:10} Es
  interesante que omitan deliberadamente toda mención de Jericó y Hai
  porque no habrían sabido de estas victorias recientes si hubieran
  venido de un país lejano.} \bibverse{11} Así que nuestros
jefes\footnote{\textbf{9:11} Al parecer, los gabaonitas no tenían rey.}nos
djieron a nosotros y a todos los habitantes de nuestra tierra:
``Llévense lo que necesiten para el viaje. Únanse a ellos y díganles:
`Somos tus siervos. Por favor, hagan un tratado con nosotros.'
\bibverse{12} 'Miren este pan. Estaba caliente cuando lo sacamos de
nuestras casas el día que salimos para venir aquí. Pero ahora está seco
y mohoso, como pueden ver. \bibverse{13} Estos odres eran nuevos cuando
los llenamos, pero mírenlos ahora: están rotos y dañados. Estas ropas
nuestras y nuestras sandalias están desgastadas porque el viaje ha sido
muy largo.''' \bibverse{14} Los israelitas probaron algunos de los
alimentos. Sin embargo, no consultaron al Señor. \bibverse{15} Entonces
Josué hizo un tratado con ellos, prometiendo perdonarles la vida, y los
líderes de la asamblea hicieron un juramento para garantizarlo.

\bibverse{16} Tres días después de haber hecho el tratado, los
israelitas se enteraron de que los gabaonitas vivían cerca, ¡justo en
medio de ellos! \bibverse{17} Entonces los israelitas partieron para ir
a las ciudades gabaonitas, y llegaron allí al tercer día. Las ciudades
eran Gabaón, Cafira, Berot y QuiriatYearín. \bibverse{18} Pero los
israelitas no los atacaron debido al tratado que habían jurado los
líderes de la asamblea en nombre del Señor, el Dios de Israel. Ante
esto, todos los israelitas protestaron contra los líderes. \bibverse{19}
Pero los líderes respondieron al pueblo: ``Les hemos jurado por el
Señor, el Dios de Israel, así que ahora no podemos ponerles la mano
encima. \bibverse{20} Así que esto es lo que vamos a hacer con ellos.
Los dejaremos vivir, para que no seamos castigados por romper el
juramento que les hicimos.'' \bibverse{21} Los líderes concluyeron:
``Déjenlos vivir''. Así que los gabaonitas se convirtieron en leñadores
y aguadores al servicio de toda la asamblea, como habían ordenado los
líderes israelitas.

\bibverse{22} Entonces Josué convocó a los gabaonitas y les preguntó:
``¿Por qué nos han engañado? ¡Nos dijeron que vivíamos lejos de ustedes,
pero resulta que ustedes viven al lado de nosotros! \bibverse{23} Ahora,
como consecuencia, ustedes estarán bajo una maldición. Desde ahora serán
para siempre siervos, leñadores y aguadores de la casa de mi Dios.''

\bibverse{24} Entonces ellos le respondieron a Josué: ``A nosotros, tus
siervos, se nos dijo muy claramente que el Señor, tu Dios, le había
ordenado a Moisés que te diera toda esta tierra, y que todos sus
habitantes debían ser exterminados ante ustedes. Así que temimos mucho
por nuestras vidas a causa de ustedes. Por eso hicimos lo que hicimos.
\bibverse{25} Ahora estamos en tus manos. Haz con nosotros lo que
consideres justo y correcto.''

\bibverse{26} Josué hizo lo que había dicho. Los salvó de los
israelitas, para que no los mataran. \bibverse{27} Aquel día Josué los
nombró leñadores y aguadores al servicio de toda la asamblea y para el
altar del Señor dondequieraque el Señor quisiera. Eso es lo que hacen
hasta el día de hoy.

\hypertarget{section-9}{%
\section{10}\label{section-9}}

\bibverse{1} Adonisedec, rey de Jerusalén, se enteró de que Josué había
capturado a Hai y había destruido la ciudad por completo, como también
lo había hecho con Jericó, y que había matado a su rey, al igual que al
rey de Jericó. También se enteró de que los gabaonitas habían hecho la
paz con los israelitas y se habían aliado con ellos. \bibverse{2} El
pueblo de Jerusalén se asustó mucho por esto, porque Gabaón era una
ciudad grande, tan grande como cualquier ciudad gobernada por el rey,
aún más grande que Hai, y sus hombres eran fuertes luchadores.

\bibverse{3} Así que Adoni-zedek, rey de Jerusalén, envió un mensaje a
Hoham, rey de Hebrón, a Piram, rey de Jarmut, a Jafía, rey de Laquis, y
a Debir, rey de Eglón, diciendo: \bibverse{4} ``Vengan y ayúdenme a
atacar a Gabaón porque han hecho la paz con Josué y los israelitas.''
\bibverse{5} Así que estos cinco reyes amorreos (los reyes de Jerusalén,
Hebrón, Jarmut, Laquis y Eglón) y sus ejércitos se reunieron y
partieron. Rodearon a Gabaón y comenzaron su ataque.

\bibverse{6} Los gabaonitas enviaron un mensaje a Josué en el campamento
de Gilgal, diciendo: ``¡Por favor, no nos abandones, tus siervos! ¡Ven
rápido y sálvanos! Necesitamos tu ayuda, pues todos los reyes amorreos
de la región montañosa se han unido para atacarnos.''

\bibverse{7} Así que Josué, con todos sus hombres de combate y sus
mejores combatientes, partió de Gilgal. \bibverse{8} El Señor le dijo a
Josué: ``No tengas miedo de ellos, porque los vencerás. Ni uno solo
podrá enfrentarse a ti.''

\bibverse{9} Al marchar toda la noche desde Gilgal, Josué llegó sin
avisar. \bibverse{10} El Señor hizo entrar en pánico a los ejércitos
amorreos cuando vieron a los israelitas. Los abatió con un gran golpe en
Gabaón; los persiguió hasta Bet-horón, y los redujo en el camino hacia
Azeca y Maceda. \bibverse{11} Mientras huían de los israelitas por la
ladera de BetJorón, el Señor les arrojó grandes piedras de granizo desde
el cielo hasta Azeca. Fueron más los muertos por las piedras de granizo
que los muertos por las espadas de los israelitas.

\bibverse{12} El día en que el Señor entregó a los amorreos a los
israelitas, Josué habló por\footnote{\textbf{10:12} ``Por'': o ``a causa
  de''. Aunque a menudo se traduce como ``para'', el sentido aquí indica
  la aprobación divina más que una conversación.} el Señor en presencia
de los israelitas, diciendo: ``¡Sol, detente sobre Gabaón! ¡Luna,
detente sobre el Valle de Ajalón!'' \bibverse{13} El sol dejó de moverse
y la luna se quedó quieta, hasta que la nación de Israel infligió la
derrota a sus enemigos. (Esto está registrado en el Libro de
Jashar\footnote{\textbf{10:13} ``Libro de Jasher'': O ``Libro de los
  Justos''. Este libro ya no se conoce. También se menciona en 2 Samuel
  1:18.}). El sol se detuvo en medio del cielo y no se puso durante un
día entero. \bibverse{14} Nunca antes ni después hubo un día así en el
que el Señor escuchara una voz humana de tal manera. Era porque el Señor
estaba luchando por Israel. \bibverse{15} Entonces Josué y todo el
ejército regresaron al campamento de Gilgal.

\bibverse{16} Los cinco reyes habían huido y se habían escondido en una
cueva en Maceda. \bibverse{17} Cuando Josué se enteró de que los cinco
reyes se habían escondido en una cueva en Maceda, \bibverse{18} dio esta
orden: ``Haz rodar algunas piedras grandes para bloquear la entrada de
la cueva y haz que algunos hombres la vigilen. \bibverse{19} Pero no te
quedes allí. Persigue al enemigo y atácalo por la retaguardia. No dejes
que escapen a sus ciudades, porque el Señor te los ha entregado para que
los derrotes.''\footnote{\textbf{10:19} ``Para que los derrotes'':
  literalmente, ``en tu mano.''} \bibverse{20} Así, Josué y los
israelitas los derrotaron totalmente, abatiéndolos y matándolos. Sólo
unos pocos sobrevivieron y escaparon a sus ciudades. \bibverse{21} El
ejército regresó con Josué al campamento de Maceda, y nadie se atrevió
siquiera a amenazar a los israelitas.\footnote{\textbf{10:21} En otras
  palabras, los pueblos circundantes se sintieron tan intimidados por
  este éxito que ni siquiera se atrevieron a hablar contra los
  israelitas, y mucho menos a atacarlos.}

\bibverse{22} Entonces Josué dijo: ``Abran la entrada de la cueva y
saquen de ella a los cinco reyes.'' \bibverse{23} Así lo hicieron,
sacando a los cinco reyes de la cueva: los reyes de Jerusalén, Hebrón,
Jarmut, Laquis y Eglón. \bibverse{24} Cuando trajeron a los reyes a
Josué, éste convocó a todos los combatientes y dijo a los comandantes
que habían ido con él: ``Vengan aquí y pongan sus pies sobre los cuellos
de estos reyes.'' Así que se acercaron y les pusieron los pies en el
cuello. \bibverse{25} Josué les dijo: ``¡Nuncantengan miedo ni se
desanimen! ¡Sean fuertes y valientes! Porque el Señor va a hacer lo
mismo con todos los enemigos que ustedes van a combatir'' \bibverse{26}
Entonces Josué mató a los reyes y colgó sus cuerpos en cinco árboles y
los dejó allí colgados hasta el atardecer.\footnote{\textbf{10:26} Véase
  Deuteronomio 21:22-23.} \bibverse{27} Al ponerse el sol, Josué dio la
orden de bajar sus cuerpos de los árboles y arrojarlos a la cueva donde
se habían escondido. Entonces los israelitas amontonaron piedras sobre
la entrada de la cueva, y allí permanecen hasta el día de hoy.

\bibverse{28} Ese día Josué capturó a la ciudad de Maceda, matando a
todos sus habitantes, incluido el rey. La apartó y la destruyó por
completo, así como a todos los que estaban en ella,\footnote{\textbf{10:28}
  Véase la explicación en 6:17.} sin dejar supervivientes. Mató al rey
de Macedatal como había matado al rey de Jericó.

\bibverse{29} Entonces Josué y el ejército israelita salieron de Maceda
y fueron a atacar Libna, \bibverse{30} y el Señor le entregó la ciudad y
a su rey a los israelitas. Josué mandó matar a todos los que estaban en
ella, sin dejar supervivientes. Mató a su rey como había matado al rey
de Jericó.

\bibverse{31} Entonces Josué y el ejército israelita pasaron de Libna a
Laquis, rodearon la ciudad y la atacaron. \bibverse{32} El Señor entregó
la ciudad a los israelitas, que la capturaron al segundo día. Josué hizo
matar a todos los que estaban en ella, tal como había hecho en Libna.
\bibverse{33} EntoncesHoram, rey de Gezer, vino con su ejército a ayudar
a Laquis, pero Josué y sus hombres los mataron, sin dejar
sobrevivientes.

\bibverse{34} Josué y el ejército israelita avanzaron desde Laquis hasta
Eglón, rodeando la ciudad y atacándola. \bibverse{35} Ese mismo día la
capturaron. Josué hizo matar a todos los que estaban en ella ese mismo
día. La apartó y la destruyó por completo, tal como había hecho en
Laquis.

\bibverse{36} Josué y el ejército israelita dejaron Eglón y fueron a
atacar Hebrón. \bibverse{37} Capturaron la ciudad, así como los pueblos
cercanos. Josué hizo matar a todos los habitantes, sin dejar
supervivientes. Al igual que había hecho en Eglón, la apartó y la
destruyó por completo con todos sus habitantes.

\bibverse{38} Entonces Josué y el ejército israelita se volvieron y
fueron a atacar Debir. \bibverse{39} La capturó, así como a su rey y a
todas las ciudades cercanas. Josué hizo matar a todos los habitantes,
sin dejar supervivientes. Al igual que había hecho en Hebrón, la apartó
y la destruyó por completo, así como a todos sus habitantes. Mató al rey
de Debir como había matado al rey de Libna.

\bibverse{40} Así, Josué conquistó toda la tierra -- la región
montañosa, el Néguev, las estribaciones y las laderas- y a todos sus
reyes. No dejó ni un solo sobreviviente. Mató a todos,\footnote{\textbf{10:40}
  Literalmente, ``todo lo que respiraba''. Sin embargo, esto no se
  refería a los animales.} tal como el Señor, el Dios de Israel, había
ordenado. \bibverse{41} Josué los destruyó desde Cades-barnea hasta Gaza
y toda la tierra desde Gosén\footnote{\textbf{10:41} No el Gosén de
  Egipto.} hasta Gabaón. \bibverse{42} Todos los reyes y sus tierras
fueron conquistados por Josué en una sola campaña porque el Señor, el
Dios de Israel, luchaba por los israelitas. \bibverse{43} Josué y el
ejército israelita regresaron entonces al campamento de Gilgal.

\hypertarget{section-10}{%
\section{11}\label{section-10}}

\bibverse{1} Cuando Jabín, rey de Hazor, se enteró de lo sucedido, envió
un mensaje\footnote{\textbf{11:1} Una llamada a las armas.} a Jobab, rey
de Madón, a los reyes de Simrón y Ajsaf, \bibverse{2} y a los reyes de
la región montañosa del norte, del valle del Jordán al sur de
Quinéret\footnote{\textbf{11:2} ``Quinéret'': el mar de Galilea.}, de
las estribaciones del oeste y las estribaciones de Dor al oeste,
\bibverse{3} a los reyes de los cananeos, tanto al este como al oeste, a
los amorreos, a los hititas, a los ferezeos, a los jebuseos en la región
montañosa, y a los heveos que viven cerca del monte Hermón en la tierra
de Mizpa. \bibverse{4} Todos sus ejércitos se reunieron, una vasta horda
tan numerosa como la arena de la orilla del mar, junto con muchísimos
caballos y carros. \bibverse{5} Todos estos reyes unieron sus fuerzas y
acamparon junto a las aguas de Merón para luchar contra Israel.

\bibverse{6} El Señor le dijo a Josué: ``No temas por ellos, porque
mañana a esta hora yo mismo los entregaré a todos a Israel, muertos.
Inutiliza sus caballos y quema sus carros.''

\bibverse{7} Josué y el ejército israelita fueron y los atacaron sin
previo aviso en las aguas de Merón. \bibverse{8} El Señor los entregó a
los israelitas, que los redujeron y los persiguieron hasta la Gran Sidón
y MisrefotMaim, y al este hasta el valle de Mizpa, matándolos hasta que
no quedó ninguno. \bibverse{9} Después Josué hizo lo que el Señor le
había ordenado: quebró las patas de los caballos y quemó los carros.

\bibverse{10} Entonces Josué se volvió contra Hazor. La capturó y mató a
su rey, pues Hazor era en ese momento el principal de todos estos
reinos.\footnote{\textbf{11:10} El rey de Hazor había sido el líder de
  la coalición contra Israel-véase 10:1.} \bibverse{11} Josué hizo matar
a todos los habitantes, sin dejar supervivientes. La apartó y la
destruyó por completo: no quedó nadie vivo. Luego incendió Hazor.

\bibverse{12} Josué capturó todas estas ciudades y mató a sus reyes. Las
apartó y las destruyó por completo, como lo había ordenado Moisés, el
siervo del Señor. \bibverse{13} Sin embargo, Israel no quemó ninguna de
las ciudades construidas sobre montículos, excepto Hazor, que Josué sí
quemó. \bibverse{14} Los israelitas sí tomaron para sí todo el botín y
el ganado de estas ciudades. Pero mataron a todos los habitantes,
destruyéndolos a todos para que no quedara ninguno vivo. \bibverse{15}
Como el Señor había instruido a Moisés, así Moisés había instruido a
Josué. Josué hizo lo que le habían dicho: hizo todo lo que el Señor le
había indicado a Moisés.

\bibverse{16} Así, Josué conquistó toda la tierra: la región montañosa,
el Néguev, toda la tierra de Gosén, las estribaciones occidentales, el
valle del Jordán, las montañas de Israel y las estribaciones orientales.
\bibverse{17} Esto abarcaba la tierra desde el monte Halac que lleva a
Seír en el sur, hasta Baal Gad en el norte, debajo del monte Hermón en
el valle del Líbano. Josué capturó y mató a todos sus reyes.
\bibverse{18} Josuélibró una larga guerra con todos estos reyes.
\bibverse{19} Ni un solo pueblo hizo la paz con los israelitas, excepto
los heveos, los habitantes de Gabaón. Todos los demás fueron
conquistados en batalla. \bibverse{20} Porque el Señor los hizo
obstinados, queriendo combatir a los israelitas para que fueran
apartados y destruidos por completo, aniquilados sin piedad, como el
Señor había instruido a Moisés.

\bibverse{21} Durante este tiempo Josué también aniquiló a los
descendientes de Anaki que vivían en la región montañosa de Hebrón,
Debir y Anab, y toda la región montañosa de Judá e Israel. Josué apartó
y destruyó completamente sus ciudades, \bibverse{22} y no quedaron
descendientes de Anac en la tierra de Israel, sólo algunos en Gaza, Gat
y Asdod.\footnote{\textbf{11:22} Ciudades filisteas en la llanura
  costera.}

\bibverse{23} Así que Josué tomó toda la tierra de acuerdo con lo que el
Señor le había ordenado a Moisés, dándosela a Israel para que la
poseyera tal como estaba repartida entre las tribus. Entonces la tierra
quedó en paz.\footnote{\textbf{11:23} ``En paz'': literalmente,
  ``descansado de la guerra.''}

\hypertarget{section-11}{%
\section{12}\label{section-11}}

\bibverse{1} Estos son los reyes que los israelitas derrotaron cuando
tomaron posesión de su tierra al este del Jordán, desde el valle de
Arnón en el sur hasta el monte Hermón en el norte, incluyendo toda la
tierra del lado oriental del Jordán. \bibverse{2} Sehón, rey de los
amorreos, que vivía en Hesbón, gobernaba desde Aroer, en el borde del
valle de Arnón, todo el camino desde la mitad del valle hasta el río
Jaboc, la frontera con los amonitas (e incluía la mitad de Galaad).
\bibverse{3} Su territorio también incluía el valle del Jordán hasta el
mar de Cineret y la tierra al este, y todo el camino hasta el Mar
Salado, al este hacia Beth-jeshimoth y al sur hasta las laderas de
Pisga. \bibverse{4} El rey Og de Basán, uno de los últimos de los
refaítas, que vivía en Astarot y Edrei, \bibverse{5} gobernaba en el
norte, desde el monte Hermón hasta Sacalé, y todo Basán al este, y al
oeste hasta las fronteras de los guesuritas y los maacatitas, junto con
la mitad de Galaad hasta la frontera de Sehón, rey de Hesbón.

\bibverse{6} Moisés, el siervo del Señor, y los israelitas los habían
derrotado, y Moisés había asignado la tierra a las tribus de Rubén, Gad
y la media tribu de Manasés.

\bibverse{7} Estos son los reyes de la tierra que Josué y los israelitas
derrotaron al oeste del Jordán, desde Baal Gad, en el valle del Líbano,
hasta el monte Halak que conduce a Seír. Josué la entregó a las tribus
de Israel para que la poseyeran tal y como les fue asignada.
\bibverse{8} La tierra incluía la región montañosa, las estribaciones,
el valle del Jordán, las laderas y el desierto del Néguev: la tierra de
los hititas, los amorreos, los cananeos, los ferezeos, los heveos y los
jebuseos.\footnote{\textbf{12:8} Véase la lista en Deuteronomio 7:1.}

\bibverse{9} El rey de Jericó. El rey de Hai, cerca de Betel.
\bibverse{10} El rey de Jerusalén. El rey de Hebrón. \bibverse{11} TEl
rey de Jarmut. El rey de Laquis. \bibverse{12} El rey de Eglón. El rey
de Gezer. \bibverse{13} El rey de Debir. El rey de Geder. \bibverse{14}
El rey de Horma. El rey de Arad. \bibverse{15} El rey de Libna. El rey
de Adulam. \bibverse{16} El rey de Maceda. El rey de Betel.
\bibverse{17} El rey de Tappúajh. El rey de Hefer. \bibverse{18} El rey
de Afec. El rey de Lasharon. \bibverse{19} El rey de Madón. El rey de
Hazor. \bibverse{20} El rey de Simrón-merón. El rey de Acsaf.
\bibverse{21} El rey de Taanac. El rey de Meguido. \bibverse{22} El rey
de Cedes. El rey de Jocneam en el Carmelo. \bibverse{23} El rey de
DorenNafat-dor. El rey de Goim enGilgal.\footnote{\textbf{12:23} La
  Septuaginta dice ``Galilea.''} \bibverse{24} El rey de Tirsa. El total
de todos los reyes es de 31.

\hypertarget{section-12}{%
\section{13}\label{section-12}}

\bibverse{1} Habían pasado muchos años y Josué había envejecido. El
Señor le habló diciendo: ``Ya eres un anciano, pero aún te queda mucha
tierra por conquistar.

\bibverse{2} Esta es la tierra que queda: el territorio de todos los
filisteos y de todos los gesureos, \bibverse{3} desde el río Sihor, en
la frontera con Egipto, hacia el norte, hasta la frontera de Ecrón; todo
esto se cuenta como cananeo, pero está bajo los cinco señores filisteos
de Gaza, Asdod, Ascalón, Gat y Ecrón. Además está la tierra de los
avvitas \bibverse{4} en el sur, toda la tierra de los cananeos, y Meará
que pertenece a los sidonios, hasta Afec en la frontera con los
amorreos, \bibverse{5} así como la tierra de los giblitas y la zona del
Líbano desde la ciudad de Baal-gad hasta las laderas del monte Hermón
hasta Lebo-hamat, \bibverse{6} Y todos los que viven en la región
montañosa desde el Líbano hasta MisrefotMaim, incluyendo toda la tierra
de los sidonios.

Yo mismo los expulsaré delante de los israelitas. Sólo asigna la tierra
a Israel para que la posea,\footnote{\textbf{13:6} ``Posser'' o
  ``comoherencia.''} como te he ordenado. \bibverse{7} Así que reparte
esta tierra entre las nueve tribus y la media tribu de Manasés para que
la posean.''

\bibverse{8} La otra mitad de la tribu de Manasés, y las tribus de Rubén
y Gad, ya habían recibido su concesión de tierras en el lado oriental
del Jordán, tal como les fue asignada por Moisés, el siervo del Señor.
\bibverse{9} Se extendía desde Aroer, al borde del valle de Arnón, desde
la ciudad situada en medio del valle, y toda la meseta de Medeba, hasta
Dibón; \bibverse{10} y todas las ciudades que pertenecían a Sehón, rey
de los amorreos, que gobernaba en Hesbón, hasta la frontera con los
amonitas. \bibverse{11} Además, incluía Galaad, la tierra de los
guesuritas y de los maacatitas, todo el monte Hermón y todo Basán hasta
Salecá, \bibverse{12} así como toda la tierra del reino de Og de Basán,
que había gobernado en Astarot y Edrei. Era uno de los últimos de los
refaítas. Moisés los había derrotado y expulsado. \bibverse{13} Pero los
israelitas no habían expulsado a los gueshuritas ni a los maacateos, que
aún viven entre ellos hasta el día de hoy. \bibverse{14} Moisés no
asignó ninguna tierra a los levitas. En cambio, les asignó las ofrendas
hechas con fuego al Señor, el Dios de Israel, como el Señor les había
prometido.

\bibverse{15} Esta fue la tierra que Moisés asignó a la tribu de Rubén,
por familias: \bibverse{16} Su territorio se extendía desde Aroer, al
borde del valle de Arnón, desde la ciudad en medio del valle, y toda la
meseta de Medeba; \bibverse{17} Hesbón y todas las ciudades en la meseta
-Dibón, Bamot Baal, Bet Baal Meón, \bibverse{18} Jahaza, Cedemot,
Mefaat, \bibverse{19} Quiriatáim, Sibma, Zeret-Sahar, en una colina del
valle, \bibverse{20} Bet Peor, las laderas de Pisga, BetJesimot,
\bibverse{21} todas las ciudades de la meseta y todo el reino de Sehón,
el rey amorreo, que gobernaba en Hesbón. Fue derrotado por Moisés, así
como por los líderes madianitas Evi, Rekem, Zur, Hur y Reba, príncipes
que vivían en el reino y que eran aliados de Sejón. \bibverse{22} Al
mismo tiempo, los israelitas mataron a Balaam, hijo de Beor, el adivino,
junto con los demás que fueron sacrificados. \bibverse{23} El Jordán era
el límite de la tribu de Rubén. Esta era la tierra, las ciudades y las
aldeas, asignadas a la tribu de Rubén, por familias.

\bibverse{24} Esta fue la tierra que Moisés asignó a la tribu de Gad,
por familias: \bibverse{25} Su territorio era Jazer, todas las ciudades
de Galaad y la mitad de la tierra de los amonitas hasta
Aroer,\footnote{\textbf{13:25} No es la misma ciudad que se menciona en
  13:16.} cerca de Rabá\footnote{\textbf{13:25} La actual Ammán.};
\bibverse{26} que se extendía desde Hesbón hasta Ramat-mizpa y Betonim,
y desde Mahanaim hasta la región de Debir. \bibverse{27} I En el valle
del Jordán estaban Bet-haram, Bet-nimra, Sucot y Zafón, el resto del
reino de Sehón, rey de Hesbón. La frontera corría a lo largo del Jordán
hasta el extremo inferior del mar de Cineret y luego corría hacia el
este. \bibverse{28} Esta era la tierra, las ciudades y las aldeas,
asignadas a la tribu de Gad, por familias.

\bibverse{29} Esta fue la tierra que Moisés asignó a la media tribu de
Manasés, es decir, a la mitad de la tribu de los descendientes de
Manasés, por familias: \bibverse{30} Su territorio se extendía desde
Manahaim por todo Basán, todo el reino de Og y todas las ciudades de
Jair en Basán: sesenta en total. \bibverse{31} También incluía Galaad,
Astarot y Edrei, las ciudades del rey Og en Basán. Esta fue la tierra
asignada a los descendientes de Maquir, hijo de Manasés, para la mitad
de ellos, por familias. \bibverse{32} Estas fueron las asignaciones que
hizo Moisés cuando estuvo en las llanuras de Moab, al otro lado del
Jordán, al este de Jericó. \bibverse{33} Sin embargo, Moisés no asignó
ninguna tierra a los levitas, porque el Señor, el Dios de Israel, les
había prometido que él sería su asignación.\footnote{\textbf{13:33}
  Véase 13:14.}

\hypertarget{section-13}{%
\section{14}\label{section-13}}

\bibverse{1} Esta fue la tierra que se asignó a los israelitas para que
la poseyeran en la tierra de Canaán por el sacerdote Eleazar, Josué,
hijo de Nun, y los jefes de las tribus. \bibverse{2} La decisión sobre
el reparto de la tierra entre las nueve tribus y media se hizo echando
suertes, como el Señor había instruido a Moisés. \bibverse{3} Moisés
había asignado tierras a las dos tribus y media al este del Jordán, pero
no había hecho ninguna asignación a los levitas entre ellas.
\bibverse{4} Los descendientes de José se habían convertido en dos
tribus, Manasés y Efraín. A los levitas no se les dio ninguna tierra,
sólo ciudades para vivir y pastos para sus rebaños y manadas.\footnote{\textbf{14:4}
  ``Rebaños'': Literalmente, ``posesiones,'' es decir, bienes muebles.}
\bibverse{5} Así que los israelitas siguieron las instrucciones que el
Señor le había dado a Moisés al repartir la tierra.

\bibverse{6} Los líderes de la tribu de Judá se acercaron a Josué en
Gilgal, y Caleb, hijo de Jefone el cenecista, le dijo: ``Recuerdas lo
que el Señor le dijo a Moisés, el siervo de Dios, en Cades-barnea acerca
de mí y de ti. \bibverse{7} Yo tenía cuarenta años cuando Moisés, el
siervo de Dios, me envió desde Cades-barnea a espiar la tierra. Cuando
regresé, dije la verdad en mi informe. \bibverse{8} Pero los que iban
conmigo hicieron que nuestro pueblo tuviera miedo. Sin embargo, he
seguido fielmente al Señor, mi Dios. \bibverse{9} En ese momento Moisés
me hizo una promesa solemne, diciéndome: `La tierra por la que has
caminado te pertenecerá a ti y a tus hijos para siempre, porque has
seguido fielmente al Señor mi Dios'. \bibverse{10} Mira: el Señor me ha
mantenido con vida estos últimos cuarenta y cinco años, tal como lo
prometió, desde que el Señor le dijo esto a Moisés mientras Israel
vagaba por el desierto. Ahora tengo ochenta y cinco años, \bibverse{11}
pero sigo siendo tan fuerte como cuando Moisés me envió. Soy tan fuerte
y estoy listo para la batalla o para lo que pueda venir como lo era
entonces. \bibverse{12} Así que dame la región montañosa de la que habló
el Señor. Ya oíste hablar entonces de los descendientes de Anac que
vivían allí en grandes ciudades fortificadas. Si el Señor está conmigo,
los expulsaré como el Señor prometió''.

\bibverse{13} Así que Josué bendijo a Caleb y le concedió la propiedad
de Hebrón. \bibverse{14} Así que Hebrón perteneció a Caleb, hijo de
Jefone el cenecista, desde aquel día hasta hoy, porque había seguido
fielmente al Señor, el Dios de Israel. \bibverse{15} (Hebrón se llamaba
antes Quiriat-arba, en honor a un gran jefe de los descendientes de
Anac). Y la tierra estaba en paz.

\hypertarget{section-14}{%
\section{15}\label{section-14}}

\bibverse{1} Esta fue la tierra asignada a la tribu de Judá, por
familias: se extendía hacia el sur hasta la frontera de Edom, hasta el
desierto de Zin en el extremo sur. \bibverse{2} Su frontera
comenzaba\footnote{\textbf{15:2} Muchas de las indicaciones dadas en el
  texto están en tiempo presente. Sin embargo, como ahora miramos hacia
  atrás históricamente, aquí se utiliza el tiempo pasado.} en el extremo
del Mar Salado -- la bahía que mira hacia el sur --- \bibverse{3} Hai
luego iba hacia el sur del Paso de los escorpiones\footnote{\textbf{15:3}
  También se menciona por su nombre en Jueces 1:36 y Números 34:4.}, a
través del desierto de Zin, para luego dirigirse al sur de Cades-barnea
hasta Hezrón. Desde allí subía hasta Adar y luego giraba hacia Carca,
\bibverse{4} pasando por Azmon y saliendo al Wadi de Egipto, terminando
en el mar.\footnote{\textbf{15:4} El mediterráneo.} Esta era su frontera
sur.

\bibverse{5} La frontera oriental de Judá era el Mar Salado, hasta donde
termina el río Jordán.

El límite septentrional iba desde la bahía septentrional del mar donde
termina el Jordán,

\bibverse{6} hasta el límite de Bet-Joglá, y luego al norte de Bet-arabá
hasta la Piedra de Bohán (hijo de Rubén). \bibverse{7} Desde allí iba
hasta el límite de Debir por el valle de Acor, y giraba al norte hacia
Gilgal,\footnote{\textbf{15:7} No el Gilgal cerca de Jericó.} frente a
las alturas de Adumín, al sur del valle. Luego el límite continuaba
hasta las aguas de En-semes y hasta En-rogel. \bibverse{8} El límite
pasaba entonces por el valle de Ben-Hinón, a lo largo de la ladera sur
de los jebuseos, (es decir, Jerusalén), y luego subía a la cima de la
montaña que domina el valle de Hinón hasta el extremo norte del valle de
Refayín. \bibverse{9} Desde allí, el límite iba desde la cima de la
montaña hasta el manantial de agua de Neftoa y hasta las ciudades del
monte Efrón. Luego se doblaba hacia Balá (Quiriath-Yearín).
\bibverse{10} Luego el límite daba la vuelta al oeste de Baalá hasta el
monte Seir y pasaba por la ladera norte del monte Yearín hasta la ciudad
de Kesalón, bajaba a BetSemes y seguía hasta Timná. \bibverse{11} El
límite seguía hasta la ladera norte de Ecrón y se doblaba hacia
Siquerón, pasando por el monte Balá, hasta Jabneel y terminaba en el
mar.

\bibverse{12} El límite occidental era la costa del Gran Mar.~

Estos eran los límites alrededor de la tribu de Judá, por familias.

\bibverse{13} El Señor le había ordenado a Josué que asignara algunas
tierras en el territorio de Judá a Caleb, hijo de Jefone, y así se le
dio la ciudad de Quiriat-arba, o Hebrón. (Arba era el padre de Anac.)
\bibverse{14} Caleb expulsó a tres grupos familiares: Sesay, Ajimán y
Talmai, descendientes de Anac.\footnote{\textbf{15:14} Véase Números
  13:22. Dado que estos nombres se mencionan más de cuarenta años antes,
  parece que se trata de nombres para los grupos familiares más grandes.}
\bibverse{15} Desde allí fue a atacar a los habitantes de Debir (antes
conocida como Quiriat-sefer). \bibverse{16} Caleb anunció: ``Al que
ataque a Quiriat-sefer y lo capture, le daré a mi hija Acsa para que se
case con él''. \bibverse{17} Otniel, hijo de Quenaz, hermano de Caleb,
capturó la ciudad, por lo que Caleb le dio a su hija Acsa para que se
casara.

\bibverse{18} Cuando ella se acercó,\footnote{\textbf{15:18} Algunos
  creen que esto se refiere al día de la boda.}él
laconvenció\footnote{\textbf{15:18} Algunos manuscritos griegos tienen
  ``la animó.''}para que le pidiera un campo a su padre.Y cuando ella se
bajó del asno, Caleb le preguntó: ``¿Qué quieres?''. \bibverse{19} Ella
respondió: ``Por favor, dame una bendición. Ya que me has dado una
tierra que es como el desierto, por favor, te pido que también me des
manantiales de agua''. Entonces él le dio tanto el manantial superior
como el inferior.

\bibverse{20} Esta fue la tierra asignada a la tribu de Judá, por
familias.

\bibverse{21} Las ciudades para la tribu de Judá en el extremo sur, en
la frontera con Edom: Cabzeel, Edar, Jagur, \bibverse{22} Quiná,
Dimoná,Adadá, \bibverse{23} Cedes, Jazor, Itnán, \bibverse{24} Zif,
Telén, Bealot, \bibverse{25} Jazor-jadatá, Queriot-Jezrón (o Jazor),
\bibverse{26} Amán, Semá, Moladá, \bibverse{27} Jazar-Gadá, Hesmón,
Bet-pelet, \bibverse{28} Jazar-súal, Beerseba, Biziotiá, \bibverse{29}
Balá, Iyín,Esen, \bibverse{30} Eltolad, Quesil, Jormá, \bibverse{31}
Siclag, Madmana, Sansaná, \bibverse{32} Lebaot, Sijín, Ayín, y Rimón, es
decir, veintinueve ciudades con sus aldeas.

\bibverse{33} Las ciudades de las estribaciones occidentales: Estaol,
Zora, Asena, \bibverse{34} Zanoa, Enganín, Tapúaj, Enam, \bibverse{35}
Jarmut, Adulán, Soco, Azeca, \bibverse{36} Sajarayin, Aditatin, Guederá,
y Guederotayin, es decir, diez ciudades con sus aldeas.

\bibverse{37} También: Zenán, Jadasá, Migdal-gad, \bibverse{38} Dileán,
Mizpa, Joctel, \bibverse{39} Laquis, Bocat, Eglón, \bibverse{40} Cabón,
Lajmás,Quitlís, \bibverse{41} Guederot, Bet-dagón, Noamá, y Maceda, es
deicr, diez ciudades con sus aldeas.

\bibverse{42} Además: Libná, Éter, Asán, \bibverse{43} Jifta, Asena,
Nezib, \bibverse{44} Queilá, Aczib y Maresá. Es decir, nueve ciudades
con sus aldeas.

\bibverse{45} Ecrón,con sus ciudades y aldeas, \bibverse{46} desde Ecrón
hasta el mar las ciudades cercanas a Asdod y sus aldeas, \bibverse{47}
Asdod y sus ciudades con sus aldeas, y Gaza con sus ciudades y aldeas,
hasta el Wadi de Egipto, y a lo largo de la costa del mar.

\bibverse{48} En la región de las colinas: Samir, Jatir, Soco,
\bibverse{49} Daná, Quiriat Saná (o Debir), \bibverse{50} Anab, Estemoa,
Anín, \bibverse{51} Gosén, Holón y Guiló. Es decir, once ciudades con
sus aldeas.

\bibverse{52} También: Arab, Dumá, Esán, \bibverse{53} Yanún,
Bet-tapúaj, Afecá, \bibverse{54} Humtá, Quiriat-arba (o Hebrón), y Sior.
Es decir, nueve ciudades con sus aldeas.

\bibverse{55} Además: Maón, Carmelo, Zif, Yutá, \bibverse{56} Jezrel,
Jocdeán, Zanoa, \bibverse{57} Caín, Guibeá y Timná. Es deicr, diez
ciudades con sus aldeas.

\bibverse{58} También: Jaljul, Betsur, Guedor, \bibverse{59} Marat,
BethAnot y Eltecón. Es decir, seis ciudades con sus aldeas.

\bibverse{60} Además: Quiriat Baal (o Quiriat-Yearín) y Rabá. Es decir,
dos ciudades con sus aldeas.

\bibverse{61} En el desierto: Bet-arabá, Midín, Secacá, \bibverse{62}
Nibsán, la Ciudad de la Sal, y Engadi. Es decir, seis ciudades con sus
aldea.

\bibverse{63} Sin embargo, la tribu de Judá no pudo expulsar a los
jebuseos, los habitantes de Jerusalén, por lo que los jebuseos viven
entre la tribu de Judá en Jerusalén hasta el día de hoy.

\hypertarget{section-15}{%
\section{16}\label{section-15}}

\bibverse{1} La frontera para el reparto a los descendientes de José iba
desde el Jordán cerca de Jericó, luego al este de las fuentes de Jericó
y a través del desierto desde Jericó hasta la región montañosa de Betel.
\bibverse{2} Desde Betel (o Luz) continuaba hasta el límite de Atarot el
arquita. \bibverse{3} Luego descendía hacia el oeste hasta el límite de
los jafletitas y el límite de la parte baja de Bet-horón, hasta Gezer, y
luego hasta el mar. \bibverse{4} Esta fue la asignación que recibieron
los descendientes de José, Efraín y Manasés.

\bibverse{5} Este fue el territorio asignado a la tribu de Efraín, por
familias. El límite de su asignación iba desde Atarot-addar, en el este,
hasta la parte alta de Bet-horón \bibverse{6} y luego hasta el mar.
Desde Micmetat, en el norte, el límite giraba hacia el este, pasando por
Tanat-Siló, al este de Janoa. \bibverse{7} DesdeJanoa bajaba hasta
Atarot y Nará, luego tocaba Jericó y terminaba en el Jordán.
\bibverse{8} DesdeTapúaj, el límite corría hacia el oeste hasta el
arroyo de Caná y luego salía al mar. Esta era la tierra asignada a la
tribu de Efraín, por familias. \bibverse{9} También se le asignaron a la
tribu de Efraín algunas ciudades con sus aldeas que estaban en la tierra
asignada a la tribu de Manasés. \bibverse{10} Sin embargo, no expulsaron
a los cananeos que vivían en Gezer, por lo que los cananeos viven en
medio de la tribu de Efraín hasta el día de hoy, pero sometidos a
trabajos forzados.

\hypertarget{section-16}{%
\section{17}\label{section-16}}

\bibverse{1} Esta fue la asignación a la tribu de Manasés, el hijo
primogénito de José. Maquir era el hijo primogénito de Manasés, que era
el padre de Galaad. Como Maquir había sido un excelente combatiente,
Galaad y Basán ya le habían sido asignados. \bibverse{2} La asignación
fue para el resto de la tribu de Manasés, a las familias de Abiezer,
Jelec, Asriel, Siquén, Héfer y Semidá. Estos eran los descendientes
varones de Manasés, hijo de José, por familias.

\bibverse{3} PeroZelofehad, hijo de Héfer, hijo de Galaad, hijo de
Maquir, hijo de Manasés, no tuvo hijos. Sólo tuvo hijas, cuyos nombres
eran Majlá, Noa, Joglá, Milca y Tirsá. \bibverse{4} Se acercaron al
sacerdote Eleazar, a Josué, hijo de Nun, y a los dirigentes, y les
dijeron: ``El Señor ordenó a Moisés que nos diera una asignación de
tierras junto con nuestros hermanos.'' Así que Josué les asignó tierras
junto con sus hermanos, como el Señor había ordenado. \bibverse{5} En
consecuencia, Manasés recibió diez cuotas de tierra junto a la tierra de
Galaad y Basán, al otro lado del Jordán, \bibverse{6} porque las hijas
de la tribu de Manasés recibieron una asignación junto con los hijos.
(La tierra de Galaad había sido asignada al resto de los descendientes
de Manasés).

\bibverse{7} Elterritorio de la tribu de Manasés iba desde Aser hasta
Micmetat, cerca de Siquem, y luego hacia el sur hasta el manantial de
Tappúaj. \bibverse{8} La tierra alrededor de Tapúajle fue asignada a
Manasés, pero la ciudad de Tappúaj, que estaba en la frontera de la
tierra de Manasés, fue asignada a Efraín. \bibverse{9} Desde allí la
frontera bajaba hasta el valle de Caná. Al sur del valle, algunos
pueblos pertenecían a Efraín entre los pueblos de Manasés. El límite
corría a lo largo del lado norte del valle y terminaba en el mar.
\bibverse{10} Al sur, la tierra pertenecía a Efraín, y al norte, a
Manasés. El mar era el límite. Al norte limitaba con Aser, y al oriente
con Isacar. \bibverse{11} Las siguientes ciudades con sus aldeas le
fueron asignadas a Manasés, pero se encontraban en la tierra de Isacar y
Aser: Bet-san, Ibleam, Dor (en la costa), Endor, Taanac y
Meguido.\footnote{\textbf{17:11} El hebreo al final del verso es difícil
  de entender. Dice literalmente ``tres de las alturas''. Una posible
  solución es que se trata de una referencia a la tercera ciudad
  nombrada, Dor, que ahora se identifica específicamente como ``la de la
  costa'', o Nafat-dor. Véase 12:23.} \bibverse{12} Pero los
descendientes de Manasés no pudieron tomar posesión de estas ciudades
porque los cananeos estaban decididos a seguir ocupando la tierra.
\bibverse{13} Sin embargo, más tarde, cuando los israelitas se hicieron
suficientemente fuertes, obligaron a los cananeos a realizar trabajos
forzados, pero no los expulsaron.

\bibverse{14} Entonces los descendientes de José se acercaron a Josué y
le preguntaron: ``¿Por qué nos has dado sólo una asignación -una parte
de la tierra- cuando somos tantos porque el Señor nos ha bendecido
mucho?''

\bibverse{15} Josué les dijo: ``Si son tantos, si la región montañosa de
Efraín es demasiado pequeña para ustedes, vayan a despejar la tierra del
bosque en el país de los ferezeos y de los refaítas.''

\bibverse{16} Los descendientes de Josué respondieron: ``La región
montañosa no es lo suficientemente grande para nosotros, pero los
cananeos que viven en las tierras bajas tienen carros de hierro, tanto
los de Bet-sán y sus aldeas como los del valle de Jezreel.''

\bibverse{17} Josué ledijo a las tribus de Efraín y Manasés, los
descendientes de José: ``Como son tantos y tan fuertes, se les dará más
que una parte. \bibverse{18} Seles asignará además la región montañosa.
Aunque sea un bosque, lo despejarán y lo poseerán, de un extremo a otro.
Expulsarán a los cananeos, aunque tengan carros de hierro, y aunque sean
fuertes.''

\hypertarget{section-17}{%
\section{18}\label{section-17}}

\bibverse{1} La tierra había sido sometida\footnote{\textbf{18:1} Aunque
  sometida, no había sido conquistada del todo, como demuestran los
  acontecimientos de la época y posteriores.} y estaba ante ellos. Los
israelitas se reunieron en Silo\footnote{\textbf{18:1} ``Silo''
  significa ``lugar de descanso.''} e instalaron el Tabernáculo de
Reunión.\footnote{\textbf{18:1} El Tabernáculo.} \bibverse{2} Sin
embargo, siete de las tribus israelitas no habían recibido sus
asignaciones de tierras.\footnote{\textbf{18:2} De lo que sigue se
  desprende que el problema no era la asignación de la tierra, sino la
  falta de deseo de ir a tomar posesión de ella.} \bibverse{3} Entonces
Josué les preguntó a los israelitas: ``¿Hasta cuándo seguirán siendo
reacios a ir a tomar posesión de la tierra que el Señor le dio a sus
antepasados? \bibverse{4} Elijan a tres hombres de cada tribu y los
enviaré a explorar la tierra. Luego podrán escribir una descripción
sobre la distribución de la tierra y traérmela. \bibverse{5} Deben
dividir la tierra en siete partes, hasta el límite de la tierra de Judá
en el sur y la de José\footnote{\textbf{18:5} Refiriéndose a Efraín y
  Manasés.} en el norte. \bibverse{6} Una vez que hayas escrito la
descripción de la tierra, dividiéndola en siete partes, me la traerás
aquí y yo te echaré suertes en presencia del Señor, nuestro Dios.

\bibverse{7} Pero los levitas no reciben una parte, pues su función como
sacerdotes del Señor es su asignación. Además, Gad, Rubén y la media
tribu de Manasés ya han recibido su asignación que Moisés, el siervo del
Señor, les dio en el lado oriental del Jordán.''

\bibverse{8} Cuando los hombres se pusieron en camino para explorar la
tierra Josué les dijo: ``Recorran la tierra y escriban una descripción
de lo que encuentren. Luego vuelvan a mí y yo les echaré suertes en
presencia del Señor, aquí en Silo''. \bibverse{9} Así que los hombres
fueron y exploraron la tierra y escribieron en un pergamino una
descripción de las siete partes, registrando los pueblos de cada parte.
Luego regresaron con Josué al campamento de Silo \bibverse{10} w donde
Josué les echó suertes en presencia del Señor. Allí Josué dividió la
tierra y asignó las diferentes partes a las tribus israelitas que
quedaban.\footnote{\textbf{18:10} ``Que quedaban'': implícito.}

\bibverse{11} La primera suerte echada fue para la tribu de Benjamín,
por familias. La tierra asignada estaba entre la de la tribu de Judá y
la de la tribu de José. \bibverse{12} Su límite comenzaba en el Jordán,
iba al norte de la ladera de Jericó, al oeste a través de la región
montañosa, y salía al desierto de Bet-aven. \bibverse{13} Luego el
límite iba hacia el sur hasta Luz (o Betel) y bajaba hasta Atarot-adar
en la montaña al sur de la parte baja de Bet-horón. \bibverse{14} Aquí
el límite giraba hacia el sur a lo largo del lado occidental de la
montaña frente a Bet-horón, terminando en Quiriat-baal (o
Quiriat-Yearín), una ciudad de la tribu de Judá. Este era el límite
occidental.

\bibverse{15} El límite sur comenzaba en el límite de Quiriat-Yearín.
Corría\footnote{\textbf{18:15} El texto dice ``oeste'' pero esta es la
  dirección equivocada.} hasta el manantial de Neftoa, \bibverse{16} y
luego bajaba hasta el pie de la montaña que da al valle de Ben-hinom, en
el extremo norte del valle de Refayín. Luego bajaba por el valle de
Hinom, por la ladera cercana a la ciudad jebusea,\footnote{\textbf{18:16}
  Jerusalén.} al sur, hacia En-rogel. \bibverse{17} Desde allí se
dirigía hacia el norte, hacia En-semes y hacia Gelilot, frente a las
alturas de Adumín, y luego bajó hasta la Piedra de Bohán (hijo de
Rubén). \bibverse{18} Luego recorría la cordillera frente al valle del
Jordán, hacia el norte, y después bajaban al mismo valle del Jordán.
\bibverse{19} Desde allí corría a lo largo de la ladera norte de
Bet-hogá, terminando en la bahía norte del Mar Salado, el extremo sur
del Jordán. Este era el límite sur.

\bibverse{20} El límite oriental era el Jordán.

Estos eran los límites alrededor de la tierra de la tribu de Benjamín,
por familias.

\bibverse{21} Estas eran las ciudades de la tribu de Benjamín, por
familias: Jericó, Bet-hogá, Emec-casis, \bibverse{22} Bet-arabá,
Zemaryin, Betel, \bibverse{23} Avīn, Pará, Ofra, \bibverse{24}
Quefar-amoní, Ofni y Gueba. En total, doce ciudades con sus
correspondientes aldeas. \bibverse{25} Además: Gabaón, Ramá, Beerot,
\bibverse{26} Mizpa, Kefira, Moza, \bibverse{27} Rekem, Irpeel, Taralá,
\bibverse{28} Zela, Haelef, Jebús (o Jerusalén), Guibeá y
Quiriat-Yearín, es decir, catorce ciudades con sus aldeas
correspondientes. Esta fue la tierra asignada a la tribu de Benjamín,
por familias.

\hypertarget{section-18}{%
\section{19}\label{section-18}}

\bibverse{1} Lasuerte para la segunda asignación cayó sobre la tribu de
Simeón, por familias. El territorio estaba dentro de la tierra asignada
a la tribu de Judá. \bibverse{2} Su asignación incluía Beerseba,
Seba,\footnote{\textbf{19:2} ``Saba'': Probablemente se trata de una
  repetición de la palabra anterior, y debería suprimirse, ya que esto
  haría el número catorce y no trece como se indica en el versículo 6.}Moladá,
\bibverse{3} Jazar-súal, Balá, Esen, \bibverse{4} Eltolad, Betul, Jormá,
\bibverse{5} Siclag, Bet-marcabot, Jazar-Susá, \bibverse{6} Bet-lebaot,
y Sarujén. Es decir, trece ciudades con sus aldeas. \bibverse{7}
También: Ayin, Rimón, Éter y Asán, es decir, cuatro ciudades con sus
aldeas, \bibverse{8} así como todas las aldeas alrededor de estas
ciudades hasta Baalath-beer (o Ramá del Néguev). Esta fue la tierra
asignada a la tribu de Simeón, por familias. \bibverse{9} La asignación
de la tribu de Simeón fue parte de la que se le dio a la tribu de Judá,
ya que lo que había recibido la tribu de Judá era demasiado grande para
ellos.

\bibverse{10} Lasuerte para la tercera asignación cayó sobre la tribu de
Zabulón, por familias. El límite de su asignación comenzaba en Sarid,
\bibverse{11} y luego iba hacia el oeste pasando por Maralá, tocaba
Dabesé y luego el arroyo cerca de Jocneán. \bibverse{12} Siguiendo el
otro camino\footnote{\textbf{19:12} ``Que va por el otro lado'':
  implícito.}desde Sarid, el límite se dirigía hacia el este hasta el
límite de Kislot-tabor, pasando por Daberat, y luego hasta Japhia.
\bibverse{13} Desde allí corría hacia el este hasta Gath-hepher,
Eth-kazin, y hasta Rimmon, y giraba hacia Neah. \bibverse{14} Allí el
límite giraba hacia el norte, hacia Hannatón, y terminaba en el valle de
Iphtah-el. \bibverse{15} T Las ciudades eran: Kattath, Nahalal, Shimron,
Idalah, y Bethlehem\footnote{\textbf{19:15} No el Belén cercano a
  Jerusalén.} -- doce ciudades con sus aldeas. \bibverse{16} Este fue el
reparto de tierra, de ciudades y de aldeas que se le dieron a la tribu
de Zabulón, por familias.

\bibverse{17} La suerte para la cuarta asignación cayó sobre la tribu de
Isacar, por familias. \bibverse{18} Su territorio incluía estas
ciudades: Jezreel, Quesulot, Sunén, \bibverse{19} Jafarayín, Sijón,
Anajarat, \bibverse{20} Rabit, Cisón, Abez, \bibverse{21} Rémet,
Enganín, Enadá y BetPasés. \bibverse{22} El límite también llegaba a las
ciudades de Tabor, Sajazimá y Beth-Semes, y terminaba en el río Jordán.
En total eran dieciséis ciudades con sus aldeas. \bibverse{23} Este fue
el reparto de territorio, ciudades y aldeas que se le dio a la tribu de
Zabulón, por familias.

\bibverse{24} La suerte para la quinta asignación cayó sobre la tribu de
Aser, por familias. \bibverse{25} Su asignación incluía las ciudades de
Jelkat, Jalí, Betén, Acsaf, \bibverse{26} Alamélec, Amad y Miseal. Su
límite llegaba hasta el Carmelo y Sijor-libnat en el oeste.
\bibverse{27} Luego giraba hacia el este, hacia Bet-dagón, llegando a la
tierra de Zabulón y al valle de Iphtah-el. Desde allí se dirigía al
norte hacia Bet-Emec y Neiel, y continuaba hacia el norte hasta Cabul, y
seguía hasta \bibverse{28} Ebrón,\footnote{\textbf{19:28} ``Ebrón'':
  Algunos creen que este lugar debe ser ``Abdón''}Rejob, Hamón, y Caná,
y tocaba Gran Sidón. \bibverse{29} El límite giraba entonces hacia Ramá
y luego hacia la ciudad fortificada de Tiro, girando hacia Josá y
terminaba en el mar. Las ciudades incluían Mehebel, Aczib, \bibverse{30}
Uma, Afec y Rejob-- veintidósciudades con sus aldeas. \bibverse{31} Este
fue el reparto -la tierra, las ciudades y las aldeas --que se le asignó
a la tribu de Aser, por familias.

\bibverse{32} Lasuerte para la sexta asignación cayó sobre la tribu de
Neftalí, por familias. \bibverse{33} Su límite comenzaba en Jélef, junto
a la encina de Sananín, y seguía hasta Adaminéqueb, Jabnel, y continuaba
hasta Lacún, terminando en el Jordán. \bibverse{34} Luego se dirigía
hacia el oeste hasta Aznot-tabor, y seguía hasta Hucoc. Llegaba a la
tierra de Zabulón por el sur, a la tierra de Aser por el oeste y al
Jordán por el este. \bibverse{35} Las ciudades fortificadas eran: Sidín,
Ser, Jamat, Racat, Quinéret, \bibverse{36} Adamá, Ramá, Jazor,
\bibverse{37} Cedes, Edrey, Enjazor, \bibverse{38} Irón, Migdal El,
Jorén, BetAnat, y BetSemes. En total eran diecinueve ciudades con sus
aldeas. \bibverse{39} Este fue el territorio -- latierra, las ciudades y
las aldeas -- que se le dieron a la tribu de Neftalí, por familias.

\bibverse{40} La suerte para la séptima asignación cayó sobre la tribu
de Dan, por familias. \bibverse{41} Su asignación incluía las ciudades
de Zora, Estaol, Ir-semes, \bibverse{42} Sagalbīn, Ayalón, Jetlá,
\bibverse{43} Elón, Timná, Ecrón, \bibverse{44} Eltequé, Guibetón,
Balat, \bibverse{45} Jehúd, BenéBerac, Gat-rimón, \bibverse{46}
Mejarcón, Racón, junto con el territorio frente a Jope \bibverse{47} Sin
embargo, la tribu de Dan no pudo conservar la tierra que le había sido
asignada, así que fue a atacar Lesén y la capturó. Mataron a sus
habitantes y se apoderaron de la ciudad, estableciéndose en ella.
Cambiaron el nombre de Lesénpor el de Dan, en honor a su antepasado.
\bibverse{48} Este fue el territorio -- latierra, las ciudades y las
aldeas -- quese le dio a la tribu de Dan, por familias.

\bibverse{49} Cuando terminaron de asignar la tierra y establecer sus
fronteras, los israelitas le dieron a Josué, hijo de Nun, una asignación
entre ellos. \bibverse{50} Siguiendo la orden del Señor, le dieron la
ciudad que pidió: TimnatSera, en la región montañosa de Efraín. Él
reconstruyó la ciudad y se estableció allí.

\bibverse{51} Estas fueron las asignaciones distribuidas por el
sacerdote Eleazar, por Josué, hijo de Nun, y por los jefes de las tribus
israelitas. Se hicieron echando suertes en Silo, en presencia del Señor,
a la entrada de la Tienda del Encuentro. Así terminaron de repartir la
tierra.

\hypertarget{section-19}{%
\section{20}\label{section-19}}

\bibverse{1} Entonces el Señor le dijo a Josué: \bibverse{2} ``Dile a
los israelitas: 'Asigna ciudades santuario, como te lo ordené por medio
de Moisés. \bibverse{3} Así, cualquier hombre que mate a alguien por
accidente, sin intención, podrá correr hacia allí y será protegido de
los que quieran vengarse.\footnote{\textbf{20:3} ``Venganza'':
  literalmente ``vengador de la sangre.''} \bibverse{4} Cuando llegue a
una de estas ciudades, expondrá su caso a los ancianos a las puertas de
la ciudad. Ellos deberán permitirle la entrada, y también le prepararán
un lugar para alojarse. \bibverse{5} Si el que busca venganza viene a
buscar al hombre, no deben entregarle al que cometió el homicidio,
porque mató a alguien sin intención y sin odio deliberado. \bibverse{6}
Permanecerá en esa ciudad hasta que se le celebre un juicio público y se
emita un veredicto, y hasta la muerte del sumo sacerdote de turno.
Entonces será libre de volver a su casa, a la ciudad de la que huyó''.

\bibverse{7} Así que asignaron las siguientes ciudades santuario Cedes
de Galilea, en la región montañosa de Neftalí; Siquem, en la región
montañosa de Efraín; y Quiriat-arba (o Hebrón), en la región montañosa
de Judá. \bibverse{8} Al otro lado del Jordán, al este de Jericó,
asignaron: Bezer, en el desierto de la meseta, de la tribu de Rubén;
Ramot en Galaad, de la tribu de Gad; y Golán en Basán, de la tribu de
Manasés.

\bibverse{9} Estasfueron las ciudades asignadas para todos los
israelitas, así como para los extranjeros que vivían entre ellos.
Cualquiera que matara a alguien involuntariamente podía ir allí para no
ser asesinado por quienes quisieran vengarse antes de que se le hiciera
un juicio público y se le diera un veredicto de culpabilidad.\footnote{\textbf{20:9}
  ``Y se le diera un veredicto de culpabilidad.}

\hypertarget{section-20}{%
\section{21}\label{section-20}}

\bibverse{1} Los jefes de la tribu de Leví se acercaron al sacerdote
Eleazar, a Josué hijo de Nun y a los jefes de las tribus israelitas.
\bibverse{2} Les hablaron en Silo, en Canaán, diciendo: ``El Señor dio
instrucciones por medio de Moisés de darnos ciudades para vivir y pastos
para nuestros rebaños.''

\bibverse{3} Así que, siguiendo las instrucciones del Señor, los
israelitas dieron ciudades y pastos a los levitas de sus propias
asignaciones.

\bibverse{4} Se echó la suerte a las familias de los ceutíes. A estos
levitas, descendientes de Aarón, se les asignaron trece ciudades de las
tribus de Judá, Simeón y Benjamín.\footnote{\textbf{21:4}
  Ciudadespreviamenteasignadas.} \bibverse{5} A las familias restantes
de los descendientes de Cota se les asignaron diez ciudades de las
tribus de Efraín, Dan y la media tribu de Manasés.

\bibverse{6} A las familias de los descendientes de Gersón se les
asignaron trece ciudades de las tribus de Isacar, Aser, Neftalí y la
media tribu de Manasés que vivían en Basán.

\bibverse{7} A las familias de los descendientes de Merari se les
asignaron doce ciudades de las tribus de Rubén, Gad y Zabulón.

\bibverse{8} Así los israelitas dieron a los levitas por sorteo estas
ciudades y pastos, tal como el Señor lo había ordenado por medio de
Moisés.

\bibverse{9} Dieron de la tribu de Judá y de la tribu de Simeón las
siguientes ciudades, específicamente nombradas, \bibverse{10} a las
familias de los coatitas, descendientes de Aarón, de la tribu de Leví,
ya que la primera suerte les correspondió a ellos: \bibverse{11}
Quiriat-arba (o Hebrón), en la región montañosa de Judá, junto con los
pastos que la rodean. (Arba era el antepasado de Anac.) \bibverse{12}
Pero los campos más alejados de la ciudad y las aldeas habían sido dados
en propiedad a Caleb hijo de Jefone.

\bibverse{13} Dieron a los descendientes del sacerdote Aarón las
siguientes ciudades y sus pastos Hebrón (una ciudad santuario para los
que accidentalmente cometieran un asesinato), Libna, \bibverse{14}
Jatir, Estemoa, \bibverse{15} Holón, Debir, \bibverse{16} Ain, Yutá y
BetSemes: nueve ciudades de estas dos tribus. \bibverse{17} De la tribu
de Benjamín, las siguientes cuatro ciudades y sus pastos Gabaón, Geba,
\bibverse{18} Anatot y Almón. \bibverse{19} En total, trece ciudades y
sus pastos fueron entregados a los sacerdotes, los descendientes de
Aarón.

\bibverse{20} En cuanto a las demás familias de los hijos de Clota de la
tribu de Leví, se les dio por sorteo cuatro ciudades con sus pastos de
la tribu de Efraín: \bibverse{21} Siquem en la región montañosa de
Efraín (una ciudad santuario para los que cometieran un asesinato
accidental), Gezer, \bibverse{22} Quibsayín y Bet-Jorón.

\bibverse{23} De la tribu de Dan, las siguientes cuatro ciudades y sus
pastos Eltequé, Guibetón, \bibverse{24} Ayalón y Gath-Rimón.

\bibverse{25} De la media tribu de Manasés, las siguientes dos ciudades
con sus pastos Tanac y Gat-rimón. \bibverse{26} Así que en total se
dieron diez ciudades y sus pastos a las familias restantes de los
descendientes de Koat.

\bibverse{27} Las familias de los descendientes de Gersón de la tribu de
Leví recibieron las siguientes dos ciudades y sus pastos de la media
tribu de Manasés Golán en Basán (una ciudad santuario para los que
accidentalmente cometieron un asesinato), y Besterá.

\bibverse{28} De la tribu de Isacar las siguientes cuatro ciudades y sus
pastos: Cisón, Daberat, \bibverse{29} Jarmut y Enganín.

\bibverse{30} De la tribu de Aser, las siguientes cuatro ciudades con
sus pastos Miseal, Abdón, \bibverse{31} Jelcat y Rejob.

\bibverse{32} De la tribu de Neftalí, las siguientes tres ciudades con
sus pastos Cedes en Galilea (una ciudad santuario para los que
accidentalmente cometieron un asesinato), Jamot-Dor y Cartán.
\bibverse{33} En total, trece ciudades y sus pastos fueron asignados a
las familias de Gersón.

\bibverse{34} Las familias de los descendientes de Merari, los que
quedaron de la tribu de Leví, recibieron las siguientes cuatro ciudades
y sus pastos de la tribu de Zabulón: Jocneán, Caráa, \bibverse{35} Dimná
y Nalal.

\bibverse{36} De la tribu de Rubén, las siguientes cuatro ciudades con
sus pastos Béser, Yahaza, \bibverse{37} Cedemot y Mefat.

\bibverse{38} De la tribu de Gad, las siguientes cuatro ciudades con sus
pastos Ramot de Galaad (ciudad santuario para los que cometieron un
asesinato accidental), Mahanaim, \bibverse{39} Hesbón y Jazer.
\bibverse{40} Así que en total se asignaron doce ciudades a las familias
de Merari, las que quedaban de la tribu de Leví.

\bibverse{41} Los levitas recibieron un total de cuarenta y ocho
ciudades y pastos dentro de la tierra de los israelitas. \bibverse{42}
Cada una de estas ciudades tenía pastos a su alrededor.

\bibverse{43} Así, el Señor dio a los israelitas toda la tierra que
había prometido a sus antepasados. Ellos se apoderaron de ella y se
establecieronallí. \bibverse{44} El Señor les dio la paz\footnote{\textbf{21:44}
  ``Paz'': literalmente, descanso.}por todas partes, como había
prometido a sus antepasados. Ni uno solo de sus enemigos pudo
enfrentarse a ellos, porque el Señor les había entregado a sus enemigos
para que los derrotaran. \bibverse{45} No faltó ni una sola de las cosas
buenas que el Señor le había prometido a Israel.Todo se había hecho
realidad.\footnote{\textbf{21:45} Es evidente que se trata de una
  hipérbole; sin embargo, si Israel hubiera seguido más de cerca los
  mandatos del Señor, esto habría sido indudablemente cierto.}

\hypertarget{section-21}{%
\section{22}\label{section-21}}

\bibverse{1} Entonces Josué convocó a las tribus de Rubén, Gad y la
media tribu de Manasés. \bibverse{2} Les dijo: ``Ustedes han hecho todo
lo que Moisés, el siervo del Señor, les dijo que hicieran, y han seguido
todos los mandatos que les di. \bibverse{3} En todo este tiempo, y hasta
el día de hoy nunca han abandonado a sus hermanos. Han seguido
cuidadosamente lo que el Señor, su Dios, les ordenó hacer. \bibverse{4}
Ahora que el Señor su Dios le ha dado la paz a sus hermanos, como lo
prometió, ustedes deben regresar a su tierra, la que Moisés el siervo
del Señor, les dio al otro lado del Jordán. \bibverse{5} Pero asegúrense
de cumplir los mandamientos y la ley, tal como se los instruyó Moisés.
Amen al Señor, su Dios, sigan todos sus caminos, guarden sus
mandamientos, permanezcan junto a él y sírvanle con todo su ser''.
\bibverse{6} Josué los bendijo, los despidió y se fueron a casa.

\bibverse{7} Moisés había dado a la media tribu de Manasés la tierra de
Basán, y a la otra mitad de la tribu Josué le había dado tierras al
oeste del Jordán. Así que Josué los bendijo y los envió a casa.

\bibverse{8} Les dijo: ``Llévense a casa todas las riquezas que se han
ganado: los grandes rebaños de ganado, los objetos de oro, plata, cobre
y hierro, la gran cantidad de ropa. Compartan todo este botín con sus
hermanos''.

\bibverse{9} Así que las tribus de Rubén y Gad y la media tribu de
Manasés dejaron al resto de los israelitas en Silo, en la tierra de
Canaán, y regresaron a su tierra en Galaad, la cual habían recibido por
orden del Señor a través de Moisés.

\bibverse{10} Cuando se acercaron a la región del Jordán, todavía en la
tierra de Canaán, las tribus de Rubén y Gad, y la media tribu de Manasés
construyeron un altar grande e impresionante\footnote{\textbf{22:10}
  Literalmente, ``por apariencia''. Esto puede indicar también que el
  altar no debía funcionar como lugar para el sacrificio, sino
  simplemente que parecía uno.} junto al río Jordán.

\bibverse{11} Se les dijo a los israelitas: ``Miren, las tribus de Rubén
y Gad, y la media tribu de Manasés han construido un altar en la región
del Jordán de la tierra de Canaán, del lado que pertenece a los
israelitas.''

\bibverse{12} Los israelitas se reunieron en Silo para ir a la guerra
contra ellos. \bibverse{13} Antes de hacerlo, enviaron a Finés, hijo del
sacerdote Eleazar, a las tribus de Rubén y Gad, y a la media tribu de
Manasés en la tierra de Galaad. \bibverse{14} Con él iban diez jefes,
uno de cada una de las diez tribus de Israel, y cada uno el jefe de su
familia. \bibverse{15} Cuando llegaron, les dijeron a las tribus de
Rubén y Gad, y a la media tribu de Manasés: \bibverse{16} ``Esto es lo
que dice todo el pueblo del Señor: ``¿Qué acto desleal es este que han
cometido contra el Dios de Israel al construiros un altar? ¿Cómo
pudieron apartarse de él ahora con tanta rebeldía? \bibverse{17} ¿No fue
suficiente nuestro pecado en Peors?\footnote{\textbf{22:17} Números
  25:1-9. Es probable que hablara Finés, como líder de la delegación, y
  que fuera él quien hubiera tomado la acción decisiva, como se registra
  en Números 25:7-8.} Ni siquiera ahora estamos limpios de la plaga que
atacó al pueblo del Señor.\footnote{\textbf{22:17} Esto puede significar
  que la enfermedad todavía estaba presente, pero los efectos de los
  parientes perdidos todavía tenían un efecto. Además, puede significar
  que la causa de la plaga -la adoración de dioses falsos- seguía siendo
  un problema, como lo indica la advertencia de Dios contra ellos en
  24:14-23.} \bibverse{18} Entonces, ¿por qué se alejan ahora del Señor?
Si hoy se rebelan contra el Señor, mañana se enfadará con todos
nosotros.

\bibverse{19} Pero si creen que su tierra está contaminada\footnote{\textbf{22:19}
  AUna posible razón para construir un altar podría ser que la tierra se
  consideraba ``impura'' y necesitaba ser ``purificada.''}, entonces
vengan a la tierra del Señor, donde se encuentra el Tabernáculo del
Señor, y compartan parte de nuestra tierra con nosotros. Pero no se
rebelen contra el Señor, ni contra nosotros\footnote{\textbf{22:19} ``Ni
  contra nosotros'': o, ``ni nos conviertan en rebeldes también.''},
construyendo para ustedes un altar distinto del altar del Señor, nuestro
Dios. \bibverse{20} Cuando Acán, hijo de Zéraj, actuó deslealmente al
tomar cosas consagradas,\footnote{\textbf{22:20} Véase 7:1-26.}¿no
sufrió el castigo todo Israel? No fue el único que murió a causa de su
pecado''.

\bibverse{21} Entonces las tribus de Rubén y Gad, y la media tribu de
Manasés, respondieron a los dirigentes israelitas:

\bibverse{22} ``El Señor es Dios de dioses,\footnote{\textbf{22:22} O
  ``¡El Señor, el Poderoso, es Dios!''}el Señor es Dios de dioses y él
lo sabe! ¡Que lo sepa Israel también!\footnote{\textbf{22:22} En otras
  palabras, Dios conoce sus motivos para construir el altar, y los
  líderes israelitas también deberían conocer sus motivos.}Si nos
rebelamos contra Dios o le somos desleales, ¡mátennos ahora mismo!
\bibverse{23} Si nuestra acción de construir un altar fue para alejarnos
del Señor, o para usar el altar para hacer holocaustos u ofrendas de
grano o de comunión, entonces que el Señor nos castigue.

\bibverse{24} Lo hicimos porque nos preocupaba que en el futuro tus
descendientes dijeran a los nuestros: ``¿Qué tienes que ver con el
Señor, el Dios de Israel? \bibverse{25} El Señor puso una frontera --
elrío Jordán- entre nosotros y ustedes, descendientes de Rubén y Gad.
Ustedes no pertenecen al Señor'. Así tus descendientes podrían impedir
que los nuestros adorasen al Señor.

\bibverse{26} Así que dijimos: `Construyamos un altar, no para
holocaustos ni para sacrificios, \bibverse{27} sino como testimonio
entre nosotros y ustedes, y para las generaciones que vengan después de
nosotros, de que vendremos a adorar al Señor en su presencia con
nuestros holocaustos, sacrificios y ofrendas de comunión.' Entonces tus
descendientes no podrán decir a los nuestros en el futuro: `Tú no
perteneces al Señor'.

\bibverse{28} Si lo hicieran en el futuro, nuestros descendientes
podrían responder: `Mira esta réplica del altar del Señor que hicieron
nuestros antepasados, no para los holocaustos ni para los sacrificios,
sino como testimonio entre nosotros y tú.'

\bibverse{29} Jamás se nos ocurriría rebelarnos contra el Señor o
alejarnos de él construyendo un altar para hacer holocaustos o ofrendas
de grano o sacrificios. El único altar del Señor, nuestro Dios, es el
que está frente a su Tabernáculo''.

\bibverse{30} Cuando Finés y los jefes israelitas oyeron esto de las
tribus de Rubén y Gad y de la media tribu de Manasés, se
alegraron.\footnote{\textbf{22:30} ``Se alegraron'': literalmente, ``fue
  bueno a sus ojos.''} \bibverse{31} Finés respondió a las tribus de
Rubén y Gad y a la media tribu de Manasés: ``Hoy sabemos que el Señor
está con nosotros porque ustedes no han actuado deslealmente al hacer
esto. Ahora han salvado a los israelitas de ser castigados por el
Señor.''\footnote{\textbf{22:31} Si los israelitas hubieran ido
  erróneamente a la guerra contra las otras tribus, esto seguramente
  habría traído sobre ellos el juicio divino.}

\bibverse{32} EntoncesFinés y los líderes israelitas dejaron a las
tribus de Rubén y Gad y a la media tribu de Manasés en la tierra de
Galaad y regresaron a la tierra de Canaán para explicar la situación a
los israelitas.

\bibverse{33} Los israelitas se alegraron del informe y Dios los
bendijo. Ya no hablaron de ir a la guerra para destruir la tierra donde
vivían las tribus de Rubén y Gad. \bibverse{34} Las tribus de Rubén y
Gad llamaron al altar ``Testigo'', porque dijeron: ``Es un testigo entre
nosotros de que el Señor es también nuestro Dios.''

\hypertarget{section-22}{%
\section{23}\label{section-22}}

\bibverse{1} Mucho tiempo después, una vez que el Señor había dado la
paz a los israelitas del conflicto con los enemigos que los rodeaban,
Josué, ya siendo muy anciano, \bibverse{2} convocó a todos los
israelitas -- los ancianos, los líderes, los jueces y los funcionarios
-- y les dijo: ``Yo estoy viejo, y me estoy envejeciendo aún más.
\bibverse{3} Ustedes han visto todo lo que el Señor, su Dios, ha hecho
por ustedes ante todas las naciones.El Señor, su Dios, ha luchado por
ustedes.

\bibverse{4} Les he asignado la tierra de las naciones restantes para
que la posean, así como las naciones ya conquistadas, desde el Jordán
hasta el Mar Mediterráneo. \bibverse{5} El Señor, tu Dios, las hará
retroceder ante ti. Los expulsará ante ti y tomarás posesión de su
tierra, como el Señor, tu Dios, te ha prometido.

\bibverse{6} Asegúrate de observar todo lo que está escrito en el libro
de la Ley de Moisés. No te desvíes de ella, ni a la izquierda ni a la
derecha. \bibverse{7} No te asocies\footnote{\textbf{23:7} Especialmente
  en lo que respecta a los matrimonios mixtos. Véase el versículo 12.}on
las naciones que quedan. No menciones los nombres de sus dioses, ni
jures por ellos, ni los adores, ni te inclines ante ellos. \bibverse{8}
Mantente cerca del Señor, tu Dios, como has hecho hasta ahora.
\bibverse{9} El Señor ha expulsado ante ti a naciones fuertes y
poderosas. Nadie ha podido enfrentarse a ti hasta el día de hoy.
\bibverse{10} Uno solo de ustedes puede ahuyentar a mil enemigos, porque
el Señor, su Dios, lucha por ustedes, como se los ha prometido.
\bibverse{11} Procurenamar al Señor, su Dios. \bibverse{12} Porque si se
apartan de él y siguen los caminos de las naciones que quedan, si se
unen en matrimonio, mezclándose unos con otros, \bibverse{13} pueden
estar absolutamente seguros de que el Señor, su Dios, no expulsará
definitivamente a estas naciones delante de ustedes. Por el
contrario,\footnote{\textbf{23:13} Implícito.}serán una trampa y un
lazo, un látigo en su espalda y como espinas en sus ojos hasta que
desaparezcan completamente de esta buena tierra que el Señor su Dios les
ha dado.

\bibverse{14} Ahora estoy a punto de morir, el destino de todo ser
viviente en la tierra. En el fondo sabes que no ha fallado ni una sola
de las buenas promesas del Señor. Todo se ha cumplido. Ni una sola ha
fallado. \bibverse{15} Pero de la misma manera que recibiste todas las
cosas buenas que el Señor, tu Dios, te prometió, el Señor traerá sobre
ti todas las cosas malas con las que te ha amenazado, hasta que seas
completamente eliminado de esta buena tierra que el Señor, tu Dios, te
ha dado. \bibverse{16} Si rompes el acuerdo que el Señor tu Dios hizo
contigo y vas a adorar a otros dioses, inclinándote ante ellos, entonces
el Señor se enojará contigo y serás rápidamente borrado de la buena
tierra que te ha dado.''

\hypertarget{section-23}{%
\section{24}\label{section-23}}

\bibverse{1} Josué convocó a todas las tribus de Israel en Siquem. Luego
llamó a los ancianos, a los líderes, a los jueces y a los funcionarios,
y vinieron y se pusieron de pie ante el Tabernáculo de Dios.
\bibverse{2} Josué dijo a todo el pueblo: ``El Señor, el Dios de Israel,
dice esto:'Hace mucho, mucho tiempo, tus antepasados, incluso Taré,
padre de Abraham y de Nacor, vivían más allá del río Éufrates, y
adoraban a otros dioses. \bibverse{3} Yo traje a tu padre Abraham desde
el otro lado del Éufrates y lo conduje por toda la tierra de Canaán y le
di muchos descendientes. Le di a Isaac. \bibverse{4} A Isaac le di Jacob
y Esaú. A Esaú le di en propiedad la región montañosa de Seír, pero
Jacob y sus hijos bajaron a Egipto.

\bibverse{5} Envié a Moisés y a Aarón, e hice caer plagas sobre el
pueblo de Egipto, y te saqué a ti\footnote{\textbf{24:5} El relato
  utiliza indistintamente ``a tus antepasados'' y ``a ti''.}.
\bibverse{6} Sí, saqué a tus antepasados, pero cuando llegaste al Mar
Rojo los egipcios perseguían a tus antepasados con carros y jinetes.
\bibverse{7} Tus antepasados le pidieron ayuda al Señor, y él puso la
oscuridad entre ustedes y los egipcios. Luego hizo que el mar volviera
sobre ellos y se ahogaron. Viste con tus propios ojos lo que hizo en
Egipto. Luego vivieron muchos años en el desierto.

\bibverse{8} Después te llevé a la tierra de los amorreos que vivían al
otro lado del Jordán. Ellos lucharon contra ti, pero te los entregué
para que los derrotaras y te apoderaras de su tierra. Yo los destruí
delante de ti.

\bibverse{9} Cuando Balac, hijo de Zipor, el rey de Moab, quiso luchar
contra Israel, mandó a llamar a Balaam, hijo de Beor, para que viniera a
maldecirte. \bibverse{10} Pero no estaba dispuesto a escuchar a Balaam,
así que en su lugar te bendijo repetidamente y te salvó de Balac.

\bibverse{11} Cruzaste el Jordán y llegaste a Jericó, donde los hombres
de Jericó lucharon contra ti. También lo hicieron los amorreos, los
ferezeos, los cananeos, los hititas, los gergeseos, los heveos y los
jebuseos. \bibverse{12} Pero te los entregué para que los derrotaras. Y
envié al hornete\footnote{\textbf{24:12} O``pánico''. Véanse
  declaraciones similares en Éxodo 23:28 y Deuteronomio 7:20.} delante
de ti para que expulsara a los dos reyes de los amorreos. ¡No ganaron
con sus propias espadas y arcos! \bibverse{13} Les di una tierra por la
que no trabajaron y ciudades que no construyeron. Ahora viven en ellas y
comen de viñas y olivares que no plantaron.'

\bibverse{14} Así que respeten al Señor y adórenlo, sincera y fielmente.
Desháganse de los dioses que sus antepasados adoraron más allá del
Éufrates y en Egipto, y adoren al Señor. \bibverse{15} Pero si no
quieren adorar al Señor, ¡elijan hoy a quién quieren adorar! ¿Adorarána
los dioses que adoraron sus antepasados más allá del Éufrates? ¿O a los
dioses de los amorreos en cuya tierra viven ahora? Pero yo y mi familia
adoraremos al Señor''.

\bibverse{16} El pueblo respondió: ``¡Nunca abandonaremos al Señor ni
adoraremos a otros dioses! \bibverse{17} Porque el Señor, nuestro Dios,
nos sacó a nosotros y a nuestros antepasados de la esclavitud en Egipto.
Él fue quien hizo grandes milagros ante nuestros ojos. Él cuidó de
nosotros en el camino mientras viajábamos por las tierras de muchas
naciones. \bibverse{18} El Señor expulsó ante nosotros a los amorreos y
a todas las demás naciones que habitaban la tierra. Así que adoraremos
al Señor, porque es nuestro Dios''.

\bibverse{19} Josué le dijo al pueblo: ``Recuerden que el Señor es un
Dios santo y celoso. No podrán adorarle, ni perdonará su rebeldía nisus
pecados \bibverse{20} si renuncian a él y adoran a dioses extranjeros.
Se volverá contra ustedes y los destruirá a pesar de todo el bien que ha
hecho por ustedes''. \bibverse{21} ``¡No digas
eso!''\footnote{\textbf{24:21} ``¡No digas eso!'': literalmente,
  ``¡No!''}respondió el pueblo. ``¡Adoraremos al Señor!''

\bibverse{22} Entonces Josué advirtió al pueblo: ``Hoy se han convertido
en testigos contra ustedes mismos al decir que han elegido adorar al
Señor.''\footnote{\textbf{24:22} En otras palabras, nunca podrán decir
  que no eran conscientes de esta elección, ya que la habían reconocido
  públicamente.}

``Sí, somos testigos,''respondió el pueblo.

\bibverse{23} ``Entonces desháganse de esos dioses extranjeros que
tienen y prometan ser leales sólo al Señor, el Dios de Israel'', les
dijo Josué.

\bibverse{24} El pueblo respondió a Josué: ``Adoraremos al Señor,
nuestro Dios, y le obedeceremos''.

\bibverse{25} Así que Josué hizo un acuerdo solemne entre el pueblo y el
Señor ese día en Siquem, obligándolos a seguir todas las leyes e
instrucciones del Señor. \bibverse{26} Josué lo anotó en el Libro de la
Ley de Dios, y colocó una gran piedra bajo la encina, cerca del
santuario del Señor.

\bibverse{27} Josué dijo al pueblo: ``Miren esta piedra. Está aquí como
testigo contra nosotros, pues ha oído todo lo que el Señor nos ha dicho,
y será testigo contra ustedes si alguna vez niegan lo que le han
prometido a su Dios.'' \bibverse{28} Entonces Josué despidió al pueblo,
enviándolo a sus tierras asignadas.

\bibverse{29} Más tarde, después de todo esto, Josué, hijo de Nun,
siervo del Señor, murió a la edad de ciento diez años. \bibverse{30} Lo
enterraron en Timnat-serah, en la región montañosa de Efraín, al norte
del monte Gaas, la tierra que le había sido asignada.

\bibverse{31} Los israelitas siguieron adorando al Señor durante toda la
vida de Josué y durante toda la vida de los ancianos que le
sobrevivieron, los que habían visto todo lo que el Señor había hecho por
Israel. \bibverse{32} Los huesos de José, que los israelitas habían
traído consigo desde Egipto, los enterraron en Siquem, en el pedazo de
tierra que Jacob había comprado a los hijos de Jamor, el padre de
Siquem, por cien piezas de plata. Esta tierra fue heredada por los hijos
de José. \bibverse{33} Eleazar, hijo de Aarón, murió y lo enterraron en
Guibeá, en la región montañosa de Efraín, tierra que había sido dada a
su hijo Finés.
