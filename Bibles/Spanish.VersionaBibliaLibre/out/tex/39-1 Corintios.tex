\hypertarget{section}{%
\section{1}\label{section}}

\bibverse{1} Esta carta viene de parte de Pablo, llamado para ser un
apóstol de Jesucristo, conforme a la voluntad de Dios, y de parte de
nuestro hermano Sóstenes. \bibverse{2} Es enviada a la iglesia de Dios
en Corinto, a aquellos que han sido justificados en Cristo Jesús,
llamados para vivir en santidad, y a todos los que adoran al Señor Jesús
en todas partes, el Señor de ellos y de nosotros.

\bibverse{3} Reciban gracia y paz de parte de Dios, nuestro Padre, y del
Señor Jesucristo. \bibverse{4} Siempre le doy gracias a Dios por
ustedes, y por la gracia que Dios les ha dado en Jesucristo.
\bibverse{5} Por medio de él ustedes han recibido riqueza en todas las
cosas, en todo lo que dicen y en cada aspecto de lo que saben.
\bibverse{6} De hecho, el testimonio de Cristo ha demostrado ser válido
mediante la experiencia de ustedes, \bibverse{7} a fin de que no pierdan
ningún don espiritual mientras esperan la venida de nuestro Señor
Jesucristo. \bibverse{8} Él también les dará fortaleza hasta el final, a
fin de que se mantengan rectos hasta el día del Señor Jesucristo.
\bibverse{9} Dios es fiel, y fue quien los llamó a compartir en
hermandad con su Hijo Jesucristo, nuestro Señor.

\bibverse{10} Hermanos y hermanas, les ruego en el nombre de nuestro
Señor Jesucristo que estén en armonía y no divididos. Por el contrario,
desarrollen una conducta y propósito de estar unidos. \bibverse{11}
Porque de parte de Cloé, algunos me han dicho cosas de ustedes, mis
hermanos y hermanas, me han dicho que hay discusiones entre ustedes.
\bibverse{12} Permítanme explicarles lo que quiero decir. Todos ustedes
andan diciendo: ``Yo sigo a Pablo,'' o ``Yo sigo a Apolo,'' o ``Yo sigo
a Pedro,'' o ``Yo sigo a Cristo.'' \bibverse{13} ¿Acaso Cristo está
dividido? ¿Acaso murió Pablo en una cruz por ustedes? ¿Acaso ustedes
fueron bautizados en el nombre de Pablo?

\bibverse{14} Estoy agradecido con Dios porque yo no bauticé a ninguno
de ustedes, excepto a Crispo y a Gayo, \bibverse{15} así que nadie puede
decir que fue bautizado en mi nombre. \bibverse{16} (Oh, y también
bauticé a la familia de Estéfanas, y aparte de ellos no recuerdo a
ningún otro). \bibverse{17} Pues Cristo no me envió a bautizar, sino a
esparcir la buena noticia, y ni siquiera con sabiduría y elocuencia
humana, de lo contrario la cruz de Cristo no tendría validez.\footnote{\textbf{1:17}
  O ``ineficaz.'' Literalmente, ``vacía.''}

\bibverse{18} Porque el mensaje de la cruz no tiene sentido para los que
están perdidos, pero es poder de Dios para nosotros, los que somos
salvos. \bibverse{19} Como dice la Escritura: ``Yo destruiré la
sabiduría del sabio, y desecharé el entendimiento de los
inteligentes.''\footnote{\textbf{1:19} Probablemente haciendo referencia
  a Isaías 29:14.}

\bibverse{20} ¿Qué decir entonces de los sabios, de los escritores, de
los filósofos de esta era? \bibverse{21} ¿Acaso Dios ha convertido la
sabiduría de este mundo en necedad? Puesto que Dios en su sabiduría no
permitió que el mundo lo conociera por medio de su propia sabiduría,
sino que su plan de gracia fue que por la necedad de la buena noticia
fueran salvados los que creyeran en él. \bibverse{22} Los judíos piden
señales milagrosas, y los griegos buscan la sabiduría, \bibverse{23}
pero nuestro mensaje es Cristo crucificado, lo cual es ofensivo para los
judíos y necedad para los extranjeros. \bibverse{24} Sin embargo, para
los que son llamados por Dios, tanto judíos como extranjeros, Cristo es
el poder y la sabiduría de Dios. \bibverse{25} Pues la necedad de Dios
es más sabia que nosotros; y la debilidad de Dios es más fuerte.

\bibverse{26} Hermanos y hermanas, recuerden su llamado, y recuerden que
este llamado no incluyó a muchos que son sabios, humanamente hablando,
ni a muchos que son poderosos, así como tampoco a muchos que son
importantes. \bibverse{27} Por el contrario, Dios eligió las cosas que
el mundo considera necedad para humillar a los que creen que son sabios.
Escogió las cosas que el mundo considera débiles, para humillar a los
que creen que son fuertes. \bibverse{28} Escogió cosas que son
irrelevantes y despreciadas por el mundo, incluso cosas que no son, para
deshacer las cosas que son,\footnote{\textbf{1:28} Lo que este
  versículo, que es complejo, quiere decir realmente es que Dios usa
  cosas y personas que no son consideradas importantes por este mundo
  para demostrar lo que es realmente importante.} \bibverse{29} a fin de
que nadie pueda jactarse en la presencia de Dios.

\bibverse{30} Es por él que ustedes viven en Jesucristo, a quien Dios
puso como sabiduría para nosotros. Él nos hace justos y nos hace libres.
\bibverse{31} Así como dice la Escritura: ``Quien quiera jactarse, que
se jacte en el Señor.''\footnote{\textbf{1:31} Haciendo referencia a
  Jeremías 9:23.}

\hypertarget{section-1}{%
\section{2}\label{section-1}}

\bibverse{1} Hermanos y hermanas, cuando vine a ustedes no traté de
impresionarlos con palabras excepcionales, o con gran sabiduría, cuando
les dije lo que Dios quería decirles. \bibverse{2} Decidí que mientras
estaba con ustedes no deseaba concentrarme en nada más, excepto en
Jesucristo, y en su crucifixión. \bibverse{3} Vine a ustedes estando
débil, temeroso y con temblor. \bibverse{4} Yo no les hablé
persuadiéndolos con palabras de sabiduría para convencerlos. Solo les
expliqué todo mediante la evidencia y el poder del Espíritu.
\bibverse{5} De este modo su confianza en Dios no estaría fundada en la
sabiduría humana, sino en el poder de Dios.

\bibverse{6} Sin embargo, usamos palabras de sabiduría para hablar con
los que son espiritualmente maduros, pero esta no es una sabiduría que
viene de este mundo, o de los gobernantes de este mundo que rápidamente
van desapareciendo. \bibverse{7} Por el contrario, explicamos la
sabiduría de Dios en términos de un misterio revelado\footnote{\textbf{2:7}
  Cuando se usa la palabra misterio en el Nuevo Testamento, normalmente
  se refiere a un misterio revelado, particularmente en referencia a
  Dios volviéndose humano en la persona de Jesucristo.} que fue oculto
anteriormente y que Dios planeó para nuestra gloria antes de la creación
de los mundos.

\bibverse{8} Ninguno de los gobernantes de este mundo comprendió cosa
alguna sobre esto, porque si así hubiera sido, no hubieran crucificado
al Señor de gloria. \bibverse{9} Pero como dice la Escritura: ``Nadie ha
visto, nadie ha escuchado, y nadie ha imaginado lo que Dios ha preparado
para los que lo aman.''\footnote{\textbf{2:9} Tomado de Isaías 64 y 65.}
\bibverse{10} Pero Dios nos ha revelado esto por medio del Espíritu,
porque el Espíritu ahonda en las profundidades de Dios. \bibverse{11}
¿Quién conoce los pensamientos de alguien si no es la misma persona que
los tiene?\footnote{\textbf{2:11} Literalmente, ``¿quién entre los
  hombres conoce las cosas de un hombre si no es el mismo espíritu que
  está en él?''} Del mismo modo, nadie conoce los pensamientos de Dios
excepto el Espíritu de Dios. \bibverse{12} Porque hemos recibido el
Espíritu de Dios, no espíritu de este mundo, a fin de que pudiéramos
entender lo que Dios nos dio tan generosamente. \bibverse{13} De eso
hablamos, no usando palabras enseñadas por la sabiduría humana, sino lo
que el Espíritu enseña. Nosotros explicamos lo que es espiritual usando
términos espirituales. \bibverse{14} Por supuesto, las personas que no
son espirituales no aceptan lo que viene del Espíritu de Dios. Para
ellos solo es necedad, y no pueden entenderlo, porque lo que es
espiritual necesita examinarse de manera apropiada. \bibverse{15} Las
personas que son espirituales lo investigan todo, pero ellos mismos no
son objeto de investigación.\footnote{\textbf{2:15}
  ``Investigar/investigación.'' En el original se usa la misma palabra,
  y puede significar también examinar o juzgar. También se relaciona con
  la palabra traducida como ``examinarse'' en el versículo 14. Los
  idiomas inglés y español no alcanzan a capturar la sutileza del
  término original.} \bibverse{16} Pues ``¿Quién entiende la mente del
Señor, y quién consideraría instruirlo?''\footnote{\textbf{2:16} Isaías
  40:13.} ¡Pero nosotros sí tenemos la mente de Cristo!

\hypertarget{section-2}{%
\section{3}\label{section-2}}

\bibverse{1} Mis hermanos y hermanas, no pude hablar\footnote{\textbf{3:1}
  Probablemente durante su visita anterior.} con ustedes como si hablara
con creyentes espirituales, sino como con personas del mundo, como si
hablara con cristianos recién nacidos. \bibverse{2} Les di a beber
leche, y no pude darles alimento sólido porque no estaban listos para
ello. \bibverse{3} Incluso ahora no están listos para ello, porque
todavía son del mundo. Si ustedes aún son envidiosos y andan en
discusiones, ¿no demuestra eso que todavía son mundanos? ¿No demuestran
que se comportan como lo hacen las personas comunes? \bibverse{4} Cuando
alguno de ustedes dice: ``Yo sigo a Pablo,'' mientras que otro dice:
``Yo sigo a Apolos,'' ¿no es eso prueba de que son como los del mundo?

\bibverse{5} ¿Quién es Apolos, después de todo? ¿Y quién es Pablo?
Nosotros solo somos siervos por medio de los cuales ustedes llegaron a
creer. Cada uno de nosotros hace la obra que Dios nos asignó.
\bibverse{6} Yo sembré, Apolo regó la tierra, pero fue Dios quien los
hizo crecer.

\bibverse{7} De modo que el que siembra no cuenta en absoluto más que el
que riega la tierra. El único que importa es Dios, quien los hace
crecer. \bibverse{8} Y el que siembra, tanto como el que riega la
tierra, tienen un mismo fin, y ambos serán recompensados conforme a lo
que hayan hecho.

\bibverse{9} Nosotros somos obreros, junto con Dios, y ustedes son el
campo de cultivo de Dios, su edificación. \bibverse{10} Por medio de la
gracia que Dios me dio, yo puse el fundamento como un supervisor
calificado en obras de edificación. Ahora alguien más construye sobre
ese fundamento. Quien hace la construcción debe vigilar lo que esté
haciendo. \bibverse{11} Porque nadie puede poner un fundamento distinto
al que ya se puso en principio, es decir, Jesucristo. \bibverse{12} Los
que construyen sobre ese fundamento pueden usar oro, plata, piedras
preciosas, madera, heno, o paja; \bibverse{13} pero cualquiera sea el
material usado para construir, saldrá a la luz. Porque en el Día del
Juicio, el fuego lo revelará y lo probará. La obra de cada uno será
mostrada tal como es. \bibverse{14} Aquellos cuya edificación se
mantenga en pie, serán recompensados. \bibverse{15} Aquellos cuya
edificación se queme, habrán perdido. ¡Ellos también serán salvos, pero
será como pasar por fuego! \bibverse{16} ¿Acaso no saben que ustedes son
templo de Dios y que el Espíritu vive en ustedes? \bibverse{17} Todo el
que destruye el tempo de Dios será destruido por Dios, porque el templo
de Dios es santo, y ustedes son el templo.

\bibverse{18} No se engañen. Si hay alguno de ustedes que piensa que es
sabio para el mundo, debe volverse necio para que pueda llegar a ser
realmente sabio. \bibverse{19} La sabiduría de este mundo es completa
necedad para Dios. Como dice la Escritura: ``Él usa la inteligencia de
los sabios para atraparlos en su propia astucia,''\footnote{\textbf{3:19}
  Job 5:13.} \bibverse{20} y ``El Señor sabe que los argumentos de los
sabios son vanos.''\footnote{\textbf{3:20} Salmos 94:11.} \bibverse{21}
Así que no se jacten de la gente. Porque lo tienen todo, \bibverse{22}
ya sea a Pablo, o a Apolos, o a Pedro---o al mundo, o la vida, la
muerte, o el presente, o el futuro. Ustedes lo tienen todo---
\bibverse{23} y son de Cristo, y Cristo es de Dios.

\hypertarget{section-3}{%
\section{4}\label{section-3}}

\bibverse{1} Así que piensen en nosotros como siervos de Cristo que
tienen por responsabilidad los ``misterios de Dios.''\footnote{\textbf{4:1}
  Una vez más, en el Nuevo Testamento los misterios son verdades
  reveladas acerca de Dios.} \bibverse{2} Más que esto, los que tienen
tales responsabilidades necesitan ser fieles. \bibverse{3} En lo
personal, muy poco me importa si alguien más me juzga. De hecho, ni
siquiera yo mismo me juzgo. \bibverse{4} No sé de nada que haya hecho
mal, pero eso no me hace justo. Es el Señor quien me juzga. \bibverse{5}
Así que no juzguen a nadie antes del tiempo correcto: cuando el Señor
venga. Él traerá a la luz los secretos más oscuros que están ocultos, y
revelará los motivos de las personas. Dios le dará a cada quien la
alabanza que le corresponda.

\bibverse{6} Ahora, hermanos y hermanas, he hecho esta aplicación para
mí y para Apolos como un ejemplo para ustedes. De esta manera aprenderán
a no ir más allá de lo que ha sido escrito, y no preferirán a uno más
que al otro con arrogancia.\footnote{\textbf{4:6} Se debate el
  significado del original. Esto se puede evidenciar en la cantidad de
  diferencias que hay en las distintas traducciones.} \bibverse{7}
¿Quién los hizo tan especiales? ¿Qué poseen que no les haya sido dado? Y
si lo recibieron, ¿por qué dicen con orgullo que no les fue dado?
\bibverse{8} Piensan que tienen todo lo que necesitan. Piensan que son
muy ricos. Ustedes creen que ya son reyes, y que no nos
necesitan.\footnote{\textbf{4:8} Literalmente, ``sin nosotros.''} Yo
desearía que en realidad ustedes estuvieran gobernando como reyes, para
que nosotros pudiéramos gobernar con ustedes. \bibverse{9} A mi modo de
ver, Dios nos ha puesto como apóstoles en primera fila, condenados a
morir. Nos hemos convertido en espectáculo ante todo el universo, para
los ángeles y los seres humanos. \bibverse{10} Nosotros somos necios en
Cristo, ¡pero ustedes son tan sabios en Cristo! ¡Nosotros somos los
débiles, pero ustedes son tan fuertes! ¡Ustedes tienen la gloria, pero
nosotros somos rechazados!\footnote{\textbf{4:10} Partiendo del
  contexto, parece que Pablo está hablando más bien de manera irónica.}
\bibverse{11} Hasta el momento presente estamos hambrientos y sedientos.
No tenemos ropa que ponernos. Somos maltratados y no tenemos donde ir.
\bibverse{12} Trabajamos duro con nuestras propias manos. Cuando la
gente nos maldice, nosotros les bendecimos. Cuando nos persiguen, lo
soportamos. \bibverse{13} Cuando nos insultan, respondemos con bondad.
Incluso ahora somos tratados como deshecho, como la peor basura que hay
en todo el mundo.

\bibverse{14} No escribo de esta manera para hacerlos sentir
avergonzados, sino para advertirlos como hijos a quienes amo en gran
manera. \bibverse{15} Aunque ustedes tengan miles de instructores
cristianos, no tendrán muchos padres. Y fue en Cristo Jesús que yo me
convertí en padre al compartir la buena noticia con ustedes.
\bibverse{16} Así que les ruego que imiten lo que yo hago.

\bibverse{17} Por eso les envié a Timoteo, mi hijo fiel en el Señor y a
quien amo. Él les recordará la manera como yo sigo a Cristo, así como
siempre lo enseño en cada iglesia que visito. \bibverse{18} Algunos
entre ustedes se han vuelto arrogantes y piensan que no me preocuparé
por irlos a visitar. \bibverse{19} Pero pronto iré a visitarlos, si es
la voluntad del Señor. Entonces podré darme cuenta de qué cosas están
diciendo estas personas arrogantes, y qué tipo de poder tienen.
\bibverse{20} Porque el reino de Dios no solo se trata de palabras, sino
de poder. \bibverse{21} Entonces, ¿qué quieren ustedes? ¿Acaso iré con
una vara a golpearlos, o iré con amor y espíritu de mansedumbre?

\hypertarget{section-4}{%
\section{5}\label{section-4}}

\bibverse{1} Escucho informes de que hay inmoralidad sexual entre
ustedes, un tipo de inmoralidad que ni siquiera los extranjeros
practican. ¡Un hombre viviendo con la esposa de su padre! \bibverse{2}
¡Y se sienten tan orgullosos de sí mismos! ¿Acaso no deberían haber
llorado de tristeza ante esto y expulsar a este hombre? \bibverse{3}
Aunque no esté allí físicamente, estoy allí en espíritu y tal como si
estuviera allí ya di mi juicio respecto a este hombre. \bibverse{4}
Cuando se reúnan en el nombre del Señor Jesús, estaré allí con ustedes
en espíritu y con el poder de nuestro Señor Jesús. \bibverse{5}
Entreguen a este hombre en manos de Satanás a fin de que su naturaleza
pecaminosa sea destruida y él mismo pueda ser salvo en el día del
Señor\footnote{\textbf{5:5} Aquí no se intenta sugerir que Satanás
  ``coopera'' en el proceso de salvación. Esta ``entrega en manos de
  Satanás'' es lenguaje figurado que tiene como fin indicar que a la
  persona implicada se le permite experimentar las consecuencias de su
  pecado para que pueda tomar la decisión de volver y salvarse.}.

\bibverse{6} No deberían estar orgullosos de esto. ¿Acaso no saben que
apenas se necesita un poco de levadura para que crezca toda la
masa?\footnote{\textbf{5:6} En otras palabras, apenas se necesita una
  pequeña porción de pecado para infectar a toda la iglesia.}
\bibverse{7} Desháganse de esta vieja levadura para que puedan ser una
nueva masa y hagan pan sin levadura. Cristo, nuestro Cordero de Pascua,
fue crucificado. \bibverse{8} Celebremos este festival\footnote{\textbf{5:8}
  Durante la temporada de la Pascua, los judíos comían pan sin levadura,
  y botaban toda la levadura que hubiera en sus casas. Pablo usa esta
  imagen para decir que la levadura del pecado debe ser eliminada, así
  como el símbolo del pecado (levadura) fue eliminado durante el
  sacrificio de la Pascua.}, no con la vieja levadura del mal y de
maldad, sino con el pan hecho sin levadura, el pan de la sinceridad y la
verdad.

\bibverse{9} En mi carta anterior les dije que no deberían juntarse con
personas inmorales. \bibverse{10} Y no me refería a la gente inmoral de
este mundo que tiene codicia y engaña a otros, o a los que son
idólatras, pues de ser así tendrían que irse de este mundo.
\bibverse{11} Lo que quise decir cuando les escribí es que no deben
juntarse con cualquiera que se haga llamar cristiano y sea inmoral,
codicioso, o idólatra; o que sea abusador, borrachón o engañador. ¡Ni
siquiera se sienten a comer con alguien así! \bibverse{12} No estoy en
autoridad de juzgar a los que están fuera de la iglesia. Pero, ¿no
deberíamos juzgar a los que están dentro de ella? \bibverse{13} Dios
juzga a los que están fuera de la iglesia. ``Expulsen al malvado de
entre ustedes.''\footnote{\textbf{5:13} Esta es una cita del libro de
  Deuteronomio, que se repite en varias partes: Deuteronomio 13:5, 17:7,
  19:19, 22:24, 24:7.}

\hypertarget{section-5}{%
\section{6}\label{section-5}}

\bibverse{1} ¡Cómo se atreven ustedes a interponer una demanda ante
jueces paganos cuando tienen una disputa con su prójimo! Por el
contrario, ustedes deberían llevar este caso ante otros creyentes.
\bibverse{2} ¿Acaso no saben que los creyentes cristianos juzgarán al
mundo? Si ustedes van a juzgar al mundo, ¿no estarán aptos para juzgar
en casos más pequeños? \bibverse{3} ¿Acaso no saben que nosotros
juzgaremos a los ángeles? ¡Cuánto más estas cosas que tienen que ver con
esta vida! \bibverse{4} De modo que si tienen que juzgar cosas que
tienen que ver con esta vida, ¿cómo es que pueden ir ante los jueces que
no son respetados por la iglesia?\footnote{\textbf{6:4} O, ``¿por qué no
  elegir jueces de entre los miembros menos respetados de la iglesia?''}
\bibverse{5} Y al decirles esto quiero que se sientan avergonzados.
¿Qué? ¿Acaso no pueden encontrar a una persona sabia entre ustedes que
pueda arreglar la disputa que tienen? \bibverse{6} ¡En lugar de ello, un
creyente lleva a otro creyente a la corte y presenta el caso ante
quienes no son creyentes! \bibverse{7} El hecho mismo de que ustedes
tienen demandas interpuestas contra otros ya es un completo desastre.
¿No sería mejor aceptar la injusticia? ¿Por qué no aceptan que otros los
defrauden? \bibverse{8} Pero sí prefieren mejor hacer juicio injusto y
defraudar incluso a sus hermanos creyentes de la iglesia.

\bibverse{9} ¿Acaso ustedes no saben que los injustos no heredarán el
reino de Dios? ¡No se dejen engañar! Las personas que son inmorales,
idólatras, adúlteros, pervertidos sexuales, homosexuales, \bibverse{10}
ladrones, codiciosos, bebedores, abusadores, o engañadores, no heredarán
el reino de Dios. \bibverse{11} Algunos de ustedes eran así, pero han
sido limpiados y santificados. Han sido justificados en el nombre del
Señor Jesucristo, y en el Espíritu de nuestro Dios. \bibverse{12} La
gente dice: ``Yo soy libre de hacer cualquier cosa,'' ¡pero no todo es
apropiado! ``Yo soy libre de hacer cualquier cosa,'' ¡Pero no permitiré
que eso tenga control sobre mí! La gente dice: \bibverse{13} ``La comida
es para el estómago y el estómago es para la comida,'' pero Dios
destruirá a ambos. Además, el cuerpo no debe ser usado para la
inmoralidad, sino para el Señor, y el Señor para el cuerpo.
\bibverse{14} Por su poder, Dios levantó al Señor de los muertos, y de
la misma manera nos levantará a nosotros\footnote{\textbf{6:14}
  Refiriéndose a la resurrección del cuerpo, siguiendo con el tema de la
  discusión.}. \bibverse{15} ¿No saben que sus cuerpos son parte del
cuerpo de Cristo? ¿Debería tomar las partes del cuerpo de Cristo y
unirlas con una prostituta? ¡Por supuesto que no! \bibverse{16} ¿No se
dan cuenta de que cualquiera que tiene sexo con una prostituta viene a
ser ``un cuerpo'' con ella? Recuerden que la Escritura dice: ``Los dos
serán un cuerpo.''\footnote{\textbf{6:16} Génesis 2:24.} \bibverse{17}
¡Pero todo el que se une al Señor es uno con él en espíritu!
\bibverse{18} ¡Manténganse lejos de la inmoralidad sexual! Todos los
demás pecados que la gente comete ocurren fuera del cuerpo, pero la
inmoralidad sexual es un pecado contra sus propios cuerpos.
\bibverse{19} ¿Acaso no saben que sus cuerpos son templo del Espíritu
Santo que está dentro de ustedes, y que recibieron de Dios?
\bibverse{20} Ustedes no se pertenecen. ¡Alguien pagó un precio por
ustedes! ¡Así que glorifiquen a Dios en sus cuerpos!

\hypertarget{section-6}{%
\section{7}\label{section-6}}

\bibverse{1} Hablaré en cuanto a lo que me escribieron, diciendo: ``No
es bueno casarse.''\footnote{\textbf{7:1} Parece que algunos en Corinto
  estaban solteros y la iglesia estaba escribiendo para preguntar si
  esto era permisible.} \bibverse{2} Sin embargo, por causa de la
tentación hacia la inmoralidad sexual, es mejor que cada hombre tenga su
propia esposa, y cada mujer su propio esposo. \bibverse{3} El esposo
debe satisfacer las necesidades sexuales de su esposa, y la esposa las
de su esposo. \bibverse{4} El cuerpo de la esposa no solo le pertenece a
ella, sino también a su esposo; y de la misma manera el cuerpo del
esposo no solo le pertenece a él sino también a su esposa. \bibverse{5}
De manera que no se priven el uno del otro, excepto por mutuo acuerdo,
por un tiempo, por ejemplo, si quieren dedicar un tiempo a la oración.
Después, vuelvan a estar juntos para que Satanás no los tiente a pecar
por causa de su falta de dominio propio. \bibverse{6} No les digo esto
como un mandamiento, sino como una concesión. \bibverse{7} No obstante,
desearía que todos fueran como yo, pero cada persona tiene su propio don
de Dios. Una persona tiene uno, mientras otra persona tiene otro.
\bibverse{8} A los que aún no están casados, o a los que han enviudado,
yo les diría que es mejor que permanezcan como yo. \bibverse{9} Pero si
carecen de dominio propio, entonces deben casarse, porque es mejor
casarse que estarse quemando de deseo.

\bibverse{10} Estos son mis consejos para los que están casados, de
hecho, no son míos sino del Señor: La esposa no debe abandonar a su
esposo \bibverse{11} (o si lo hace, no debe volver a casarse, o debe
regresar con él); y el esposo no debe abandonar a su esposa\footnote{\textbf{7:11}
  Un asunto particular en la iglesia primitiva era el de una persona que
  se convertía en Cristiana, y luego la manera como debía relacionarse
  con su pareja que no era cristiana. Este parece ser el asunto que se
  aborda aquí.}. \bibverse{12} Ahora, al resto de ustedes (y en esto
hablo yo, y no el Señor), yo les diría que si un hombre cristiano tiene
una esposa que no es cristiana y ella está dispuesta a permanecer con
él, entonces él no debe dejarla. \bibverse{13} Y si una mujer cristiana
tiene un esposo que no es cristiano, y él está dispuesto a permanecer
con ella, entonces ella no debe dejarlo.

\bibverse{14} Para un hombre que no es cristiano, su relación
matrimonial es santificada por la esposa que sí es cristiana, y para la
esposa que no es cristiana, la relación matrimonial es santificada por
el esposo que sí es cristiano\footnote{\textbf{7:14} Pablo no quiere
  decir que por el hecho de casarse con una pareja cristiana, la persona
  no cristiana se convierte en cristiana también o que por este hecho
  experimenta la salvación. Su interés está en abordar el asunto de que
  el hecho de estar casado/a con una pareja que no es cristiana, de
  alguna manera ``contamina'' el matrimonio o a la pareja cristiana de
  la relación. El verdadero asunto que se aclara aquí es con respecto a
  los hijos de tal matrimonio: que ellos tampoco son ``impuros'' sino
  que son ``santos'' y esto no supone referencia alguna en cuanto al
  estado espiritual real de los hijos.}. De otro modo significaría que
sus hijos serían impuros, pero ahora son santos. \bibverse{15} Sin
embargo, si la esposa que no es cristiana se va, que se vaya. En tales
casos el hombre o la mujer que sí son cristianos no tienen ataduras
esclavizantes, pues Dios nos ha llamado a vivir en paz. \bibverse{16} A
las esposas les digo: ¿quién sabe? ¡Puede ser que tú salves a tu esposo!
Y a los esposos también les digo: ¿quién sabe? ¡Puede ser que tú salves
a tu esposa!

\bibverse{17} Aparte de tales casos, cada uno de ustedes debería
mantenerse en la situación que el Señor le asignó, y seguir viviendo la
vida a la que Dios los ha llamado. Ese es mi consejo a todas las
iglesias. \bibverse{18} ¿Estaban ustedes circuncidados cuando se
convirtieron? No se vuelvan incircuncisos. ¿Estaban incircuncisos cuando
se convirtieron? No se circunciden. \bibverse{19} La circuncisión no
significa nada, y la incircuncisión tampoco. Lo que realmente importa es
guardar los mandamientos de Dios. \bibverse{20} Todos deberían
permanecer en la condición en que estaban cuando fueron
llamados\footnote{\textbf{7:20} ``Llamados''---en otras palabras,
  conversión.}. \bibverse{21} Si cuando fuiste llamado eras un esclavo,
no te preocupes, aunque si tienes la oportunidad de ser libre, tómala.
\bibverse{22} Si eras un esclavo cuando el Señor te llamó, ahora eres
libre, trabajando para el Señor. De la misma manera, si fuiste llamado
cuando eras libre, ¡ahora eres esclavo de Cristo! \bibverse{23} Por
ustedes se pagó un precio, así que ya no sean esclavos de nadie.
\bibverse{24} Hermanos y hermanas, permanezcan en la condición que
estaban cuando fueron llamados, pero viviendo con Dios.

\bibverse{25} Ahora, en cuanto a las ``personas que no están
casadas,''\footnote{\textbf{7:25} Literalmente, ``vírgenes.'' Aquí Pablo
  sigue debatiendo algunos asuntos que la iglesia de corinto había
  planteado. Ver 7:1.} no tengo una instrucción específica del Señor,
así que permítanme darles mi opinión como alguien que mediante la
misericordia del Señor es considerado digno de confianza. \bibverse{26}
Por la difícil situación en la que estamos en este momento, pienso que
es mejor que simplemente permanezcan como están. \bibverse{27} ¿Están ya
casados? No traten de divorciarse. ¿No están casados? No traten de
casarse. \bibverse{28} Si no se casan, no es pecado. Si una mujer que no
está casada se casa, no es pecado. Pero tendrán muchas dificultades en
este mundo y quisiera que las evitaran. \bibverse{29} Les digo, hermanos
y hermanas, que el tiempo es corto, y de ahora en adelante, para los que
están casados puede que parezca como si no estuvieran casados,
\bibverse{30} y los que lloran como si no lloraran, y los que
celebraban, como si no hubieran celebrado, y los que compraron, como si
no hubieran poseído nada, \bibverse{31} y los que andaban en cosas del
mundo, como si no los satisficiera. Porque el orden actual del mundo
está pasando\footnote{\textbf{7:31} En esta oración extensa Pablo indica
  que incluso el matrimonio puede estar relacionado con eventos
  temporales (``el tiempo es corto''). El vivir bajo persecución,
  esperando el fin de todas las cosas, significa que incluso el
  matrimonio es visto de manera distinta, igual que todo lo demás.}.

\bibverse{32} Yo preferiría que se mantuvieran libres de tales
preocupaciones. Un hombre que no está casado está más atento a las cosas
que son importantes para el Señor, y cómo puede agradarle. \bibverse{33}
Pero un hombre que está casado presta atención a lo que es importante en
este mundo y cómo puede agradar a su esposa. \bibverse{34} En
consecuencia, su lealtad está dividida. De la misma manera, una mujer o
jovencita está atenta a lo que es importante para el Señor para así
vivir una vida dedicada tanto en cuerpo como en espíritu. Pero una mujer
casada está atenta a lo que es importante en el mundo, y cómo puede
agradar a su esposo. \bibverse{35} Les digo esto para su bien. No
intento poner lazo en sus cuellos, sino mostrarles lo correcto a fin de
que puedan servir al Señor sin distracciones.

\bibverse{36} Pero si un hombre piensa que se está comportando de manera
inapropiada con la mujer que está comprometido, y si piensa que podría
ceder ante sus deseos sexuales, y cree que debe casarse, no será pecado
si se casa. \bibverse{37} Pero si un hombre se mantiene fiel a sus
principios, y no tiene obligación de casarse, y tiene el poder para
mantener sus sentimientos bajo control y permanecer comprometido con
ella, hace bien en no casarse. \bibverse{38} De modo que el hombre que
se casa con la mujer con quien está comprometido, hace bien, aunque el
que no se casa hace mejor.

\bibverse{39} Una mujer está atada a su esposo mientras él viva. Pero si
su esposo muere\footnote{\textbf{7:39} La palabra usada aquí significa
  ``dormirse'', que es la expresión usual en el Nuevo Testamento para
  referirse a la muerte.}, ella queda libre para casarse con quien ella
quiera en el Señor\footnote{\textbf{7:39} Queriendo decir que debe ser
  un matrimonio entre dos cristianos.}. \bibverse{40} Pero en mi
opinión, ella sería más feliz si no se volviera a casar, y creo que
cuando digo esto también tengo el Espíritu de Dios.

\hypertarget{section-7}{%
\section{8}\label{section-7}}

\bibverse{1} Ahora, en cuanto a la ``comida sacrificada a
ídolos.''\footnote{\textbf{8:1} Pablo sigue respondiendo las inquietudes
  que han mencionado los corintios.} Ya ``todos tenemos conocimiento''
sobre este tema. El conocimiento nos hace orgullosos, pero el amor nos
fortalece. \bibverse{2} ¡Si alguno piensa que sabe cosa alguna, no sabe
como realmente debería saber! \bibverse{3} Pero todo el que ama a Dios
es conocido por él\ldots{}

\bibverse{4} De modo que en cuanto a comer los alimentos sacrificados a
ídolos: sabemos que no existe tal cosa como los ídolos en el mundo, y
que hay solo un Dios verdadero. \bibverse{5} Aunque hay lo que llaman
``dioses,'' ya sea en el cielo o en la tierra, hay en realidad muchos
``dioses'' y ``señores.'' \bibverse{6} Pero para nosotros solo hay un
Dios, el Padre, a partir del cual fueron hechas todas las cosas, y él es
el propósito de nuestra existencia; y un Señor, Jesús, por medio de
quien todas las cosas fueron hechas, y él es el mediador de nuestra
existencia\footnote{\textbf{8:6} Este es un versículo complejo y se ha
  debatido mucho sobre su significado. Es considerado como un ``credo''
  primitivo, o una declaración que identifica a Dios como Creador y
  Re-creador, como el centro de nuestras vidas. Literalmente dice:
  ``Pero para nosotros, Dios el Padre, a quien pertenece todo y nosotros
  en él; y uno, el Señor Jesucristo, a través de quien todo y nosotros
  somos, por medio de él.''}.

\bibverse{7} Pero no todo el mundo tiene este
``conocimiento.''\footnote{\textbf{8:7} Pablo contradice este
  conocimiento que se está aplicando erróneamente, como vemos en el
  versículo 10, donde puede parecernos que está siendo orgulloso y
  arrogante.} Algunos hasta ahora se han acostumbrado tanto a los ídolos
como realidad, que cuando comen alimentos sacrificados a un ídolo, su
conciencia (que es débil) les dice que se han contaminado a sí mismos.
\bibverse{8} ¡Pero la comida no nos hace ganar la aprobación de Dios! Si
no comemos esta comida, no somos malos, y si la comemos, no somos
buenos. \bibverse{9} Simplemente cuídense de no usar esta libertad que
tienen para comer alimentos sacrificado a ídolos para ofender a los que
tienen una actitud más débil. \bibverse{10} Si otro creyente te ve a ti,
que tienes un ``mejor conocimiento,''\footnote{\textbf{8:10} Ver en el
  versículo 8:7.} comiendo alimentos en un templo donde hay ídolos, ¿no
se convencerá, esta débil conciencia, de comer alimentos sacrificados a
ídolos?\footnote{\textbf{8:10} En otras palabras, decidir seguir el
  ejemplo de otro aun creyendo que es un pecado.} \bibverse{11} Por tu
``mejor conocimiento'' el creyente más débil se destruye. Un creyente
por el que Cristo murió. \bibverse{12} De esta manera, pecas contra
otros creyentes, hiriendo sus conciencias que son más débiles, y pecas
contra Cristo. \bibverse{13} De modo que si comer alimentos sacrificados
a ídolos hará caer a mi hermano, no volveré a comer esa carne de nuevo,
para no ofender a ningún creyente.

\hypertarget{section-8}{%
\section{9}\label{section-8}}

\bibverse{1} ¿No soy libre? ¿No soy un apóstol? ¿No he visto a Jesús,
nuestro Señor? ¿Acaso no son ustedes fruto de mi obra en el Señor?
\bibverse{2} Incluso si no fuera apóstol para los demás, al menos soy
apóstol para ustedes. ¡Ustedes son la prueba de que soy apóstol del
Señor!

\bibverse{3} Esta es mi respuesta a los que me cuestionan sobre esto:
\bibverse{4} ¿Acaso no tenemos el derecho a que se nos provea alimento y
bebida? \bibverse{5} ¿No tenemos el derecho a que nos acompañe una
esposa cristiana, como el resto de los apóstoles, los hermanos del
Señor, y Pedro? \bibverse{6} ¿Acaso somos Bernabé y yo los únicos que
tenemos que trabajar para mantenernos?\footnote{\textbf{9:6} El original
  es presentado en términos de una doble negación. Lo que se sugiere es
  que Pablo y Bernabé eran los únicos que no tenían el privilegio de no
  tener que trabajar.} \bibverse{7} ¿Acaso qué soldado alguna vez tuvo
que pagar su propio salario? ¿Quién planta una viña y no come de sus
frutos? ¿Quién alimenta un rebaño y no consume su leche?

\bibverse{8} ¿Acaso hablo solo desde un punto de vista humano? ¿No dice
la ley lo mismo? \bibverse{9} En la ley de Moisés está escrito: ``No le
pongan bozal al buey cuando está desgranando el trigo.''\footnote{\textbf{9:9}
  Deuteronomio 25:4.} ¿Acaso pensaba Dios solo en los bueyes?
\bibverse{10} ¿No se dirigía a nosotros? Sin duda alguna esto fue
escrito para nosotros, porque todo el que ara debe arar con esperanza, y
todo el que trilla debe hacerlo con la esperanza de tener parte en la
cosecha. \bibverse{11} Si nosotros sembramos cosas espirituales en
ustedes, ¿es importante si cosechamos algún beneficio material?
\bibverse{12} Si otros ejercen este derecho sobre ustedes, ¿no lo
merecemos nosotros mucho más? Aun así, nosotros no ejercimos este
derecho. Por el contrario, estaríamos dispuestos a soportar cualquier
cosa antes que retener el evangelio de Cristo.

\bibverse{13} ¿No saben que los que trabajan en los templos reciben sus
alimentos de las ofrendas del templo, y los que sirven en el altar
reciben su porción del sacrificio que está sobre él? \bibverse{14} De la
misma manera, Dios ordenó que los que anuncian la buena noticia deben
vivir de las provisiones que dan los seguidores de la buena noticia.
\bibverse{15} Pero yo no he hecho uso de ninguna de estas provisiones, y
no escribo esto para insinuar que se haga en mi caso. Preferiría morir
antes que alguien me quite la honra de no haber recibido ningún
beneficio.

\bibverse{16} No tengo nada por lo cual jactarme en predicar la buena
noticia, porque es algo que hago como deber. ¡De hecho, para mí es
terrible si no comparto la buena noticia! \bibverse{17} Si hago esta
obra por mi propia elección, entonces tengo mi recompensa. Pero si no
fuera mi elección, y se me impusiera una obligación, \bibverse{18} ¿qué
recompensa tendría? Es la oportunidad de compartir la buena nueva sin
cobrar por ello, sin exigir mis derechos como trabajador en favor de la
buena nueva.

\bibverse{19} Aunque soy libre y no soy siervo de nadie, me he puesto a
servicio de todos para ganar más. \bibverse{20} Para los judíos me
comporto como judío para ganarme a los judíos. Para los que están bajo
la ley, me comporto como si estuviera bajo la ley (aunque no estoy
obligado a estar bajo la ley), para poder ganar a esos que están bajo la
ley. \bibverse{21} Para los que no obran conforme a la ley,\footnote{\textbf{9:21}
  Refiriéndose a los que no son judíos, que no observan la ley de
  Moisés.} me comporto como ellos, (aunque sin ignorar la ley de Dios,
sino obrando bajo la ley de Cristo), para poder ganar a los que no
observan la ley.

\bibverse{22} Con los que son débiles,\footnote{\textbf{9:22}
  Refiriéndose probablemente al tema del ``creyente más débil'' que se
  menciona desde el versículo 8:7 en adelante.} comparto en su debilidad
para ganar a los débiles. ¡He terminado siendo ``como todos'' para todos
a fin de que, usando todos los medios posibles, pueda ganar a algunos!
\bibverse{23} ¡Hago esto por causa de la buena noticia para yo también
ser partícipe de sus bendiciones!

\bibverse{24} ¿Acaso no concuerdan conmigo en que hay muchos corredores
en una carrera, pero solo uno recibe el premio? ¡Entonces corran de la
mejor manera posible, para que puedan ganar! \bibverse{25} Todo
competidor que participa en los juegos mantiene una disciplina estricta
de entrenamiento. Por supuesto, lo hacen para ganar una corona que no
perdura. ¡Pero nuestras coronas durarán para siempre!

\bibverse{26} Es por eso que me apresuro a correr en la dirección
correcta. Peleo teniendo un blanco, no golpeando al aire. \bibverse{27}
Y también soy severo con mi cuerpo para tenerlo bajo control, porque no
quiero de ninguna manera estar descalificado después de haber compartido
la buena noticia con todos los demás.

\hypertarget{section-9}{%
\section{10}\label{section-9}}

\bibverse{1} Ahora quiero explicarles algo, mis hermanos y hermanas.
Nuestros antepasados vivieron bajo la nube, y todos pasaron por el
mar\footnote{\textbf{10:1} La nube de la presencia de Dios, el pase a
  través del Mar rojo.}. \bibverse{2} De manera simbólica fueron
bautizados ``en Moisés,'' en la nube y en el mar. \bibverse{3} Todos
comieron de la misma comida espiritual \bibverse{4} y bebieron de la
misma bebida espiritual, porque ``bebieron de la roca espiritual'' que
los acompañaba. Esa roca era Cristo. \bibverse{5} Sin embargo, Dios no
estaba agradado con muchos de ellos, y perecieron en el desierto.

\bibverse{6} Ahora, estas experiencias sirven como ejemplo para
nosotros, para demostrarnos que no debemos desear lo malo, como lo
hicieron ellos. \bibverse{7} No deben adorar ídolos, como algunos de
ellos lo hicieron, tal como se registra en la Escritura: ``El pueblo
festejó y bebió, y se gozaron en culto pagano.''\footnote{\textbf{10:7}
  Ver Éxodo 32:6.} \bibverse{8} No debemos cometer pecados sexuales,
como lo hicieron algunos de ellos, y en consecuencia 23:000 murieron en
un día. \bibverse{9} Tampoco debemos presionar a Dios hasta el límite,
como algunos de ellos hicieron, y fueron muertos por serpientes.
\bibverse{10} No se quejen de Dios, como algunos lo hicieron, y murieron
en manos del ángel destructor.

\bibverse{11} Todas las cosas que les sucedieron a ellos son ejemplo
para nosotros y fueron escritas para advertirnos a nosotros que vivimos
cerca del fin del tiempo. \bibverse{12} De modo que si ustedes creen que
son lo suficientemente fuertes para mantenerse firmes, ¡cuídense de no
caer! \bibverse{13} No experimentarán ninguna tentación más grande que
la de ningún otro, y Dios es fiel. Él no permitirá que sean tentados más
allá de lo que pueden soportar. Y cuando sean tentados, él les
proporcionará una salida, a fin de que puedan mantenerse fuertes.
\bibverse{14} Así que, mis amigos, manténganse lejos del culto idólatra.

\bibverse{15} Hablo a personas sensatas, para que disciernan si estoy
diciendo la verdad. \bibverse{16} Cuando damos gracias a Dios por la
copa que usamos en la Cena del Señor, ¿acaso no participamos de la
sangre de Cristo? Y cuando partimos el pan de la comunión, ¿acaso no
participamos del cuerpo de Cristo? \bibverse{17} Al comer de un mismo
pan, demostramos que aunque somos muchos, somos un solo cuerpo.
\bibverse{18} Miren al pueblo de Israel. ¿Acaso los que comen los
sacrificios hechos en el altar no lo hacen juntos? \bibverse{19} ¿Qué
es, entonces, lo que quiero decir? Que ninguna cosa sacrificada a ídolos
tiene significado alguno, ¿o acaso un ídolo existe realmente? ¡Por
supuesto que no! \bibverse{20} Los paganos hacen sacrificios a demonios,
y no a Dios. ¡No quisiera que ustedes tengan nada que ver con demonios!
\bibverse{21} No pueden beber la copa del Señor y también la copa de los
demonios; así como no pueden comer en la mesa del Señor y también en la
mesa de los demonios. \bibverse{22} ¿Acaso intentamos provocarle celos
del Señor? ¿Somos más fuertes que él?

\bibverse{23} Algunos dicen: ``Yo soy libre de hacer cualquier cosa''---
¡pero no todo es apropiado! ``Soy libre de hacer cualquier cosa'' ¡pero
no todo edifica!\footnote{\textbf{10:23} Ver 6:12.} \bibverse{24} No
deberían estar preocupados por ustedes mismos, sino por su prójimo.
\bibverse{25} Coman todo lo que se venda en el mercado, sin hacer
preguntas, por razones de consciencia\footnote{\textbf{10:25} Esto una
  vez más hace referencia al tema de las comidas sacrificadas a ídolos.},
\bibverse{26} porque ``la tierra y todo lo que hay en ella le pertenece
a Dios.''\footnote{\textbf{10:26} Salmos 24:1}

\bibverse{27} Si una persona que no es cristiana te invita a comer y
sientes ganas de ir, come lo que te sirvan, sin hacer preguntas, por
razones de conciencia. \bibverse{28} Pero si alguien te dice: ``Esta
comida fue sacrificada a ídolos,'' no la comas, por causa de quien te lo
dijo, y por razones de conciencia. \bibverse{29} Razones de su
conciencia, no tuya. Pues, ¿por qué mi libertad debería estar
determinada por la conciencia de otra persona?\footnote{\textbf{10:29}
  Este asunto parece estar en desacuerdo con el versículo anterior.
  Pablo está debatiendo respecto a la tolerancia, tanto con quien se
  ofende por el consumo de carne sacrificada a ídolos, como con quien no
  ve ningún problema con este hecho, pues los ``dioses'' de ídolos no
  existen.} \bibverse{30} Si yo elijo comer con agradecimiento, ¿por qué
sería criticado por comer algo por lo cual estoy agradecido a Dios?
\bibverse{31} De modo que ya sea que comas o bebas, o cualquier cosa que
hagas, asegúrate de hacerlo para la gloria de Dios. \bibverse{32} No
causen ofensas, no importa si es a judíos, griegos o a la iglesia de
Dios, \bibverse{33} tal como yo mismo trato de agradar a todos en todo
lo que hago. No pienso en lo que me beneficia, sino en lo que beneficia
a otros, para que puedan ser salvos.

\hypertarget{section-10}{%
\section{11}\label{section-10}}

\bibverse{1} Deberían imitarme a mí, así como yo imito a Cristo.
\bibverse{2} Estoy agradecido de que ustedes siempre me recuerden y que
estén manteniendo las enseñanzas tal como se las impartí. \bibverse{3}
Quiero que entiendan que Cristo es la cabeza de todo hombre, que el
hombre es la cabeza de la mujer, y que Dios es la cabeza de
Cristo\footnote{\textbf{11:3} El significado de ``cabeza'' en este
  contexto es tema de gran debate. En la Escritura la cabeza puede
  guardar relación tanto con el ``origen'' como con la ``autoridad'' y
  en este caso pueden aplicarse ambos aspectos.}. \bibverse{4} La cabeza
de un hombre es deshonrada si ora o profetiza con su cabeza cubierta.
\bibverse{5} La cabeza de una mujer es deshonrada si ora o profetiza con
su cabeza descubierta, es como si tuviera su cabello rapado.
\bibverse{6} Si la cabeza de una mujer no está cubierta, entonces debe
afeitarla. Si cortar su cabello o afeitarlo es causa de escándalo,
entonces debe cubrir su cabeza. \bibverse{7} Un hombre no debe cubrir su
cabeza, porque él es la imagen y la gloria de Dios, mientras que la
mujer es la gloria del hombre. \bibverse{8} El hombre no fue hecho a
partir de la mujer, sino que la mujer fue hecha del hombre; \bibverse{9}
y el hombre no fue creado para la mujer, sino que la mujer fue creada
para el hombre. \bibverse{10} Es por eso que la mujer debe tener esta
señal de autoridad sobre su cabeza, por respeto a los ángeles que
vigilan\footnote{\textbf{11:10} Tal como lo mencionan algunos
  comentaristas, este es uno de los versículos más difíciles de traducir
  y comprender en el Nuevo Testamento. Algunos comprenden por
  ``autoridad'' el acto de cubrirse la cabeza, demostrando que la mujer
  es respetable y tiene una posición en lo que tiene que ver con la
  relación con el hombre. Otros ven esto como una ``autoridad'' para
  hablar y profetizar, pues este no era un rol común para una mujer en
  esta sociedad. Existen muchas otras interpretaciones de este
  versículo, así como de la frase literalmente traducida ``por causa de
  los ángeles.''}. \bibverse{11} Aún así, desde el punto de vista del
Señor, la mujer es tan esencial como el hombre, y el hombre es tan
esencial como la mujer\footnote{\textbf{11:11} Literalmente: ``sin
  embargo, ni la mujer sin el hombre ni el hombre sin la mujer en el
  Señor.''}. \bibverse{12} Como la mujer fue hecha del hombre, entonces
el hombre viene de la mujer\footnote{\textbf{11:12} Haciendo referencia
  a la creación, donde Eva es formada a partir de Adán, pero de allí en
  adelante la mujer dio a luz a los hombres.}---pero más importante es
el hecho de que todo viene de Dios. \bibverse{13} Juzguen ustedes
mismos: ¿Es apropiado que una mujer ore a Dios con su cabeza
descubierta? \bibverse{14} ¿Acaso la naturaleza misma indica que un
hombre con cabello largo se deshonra a sí mismo? \bibverse{15} Sin
embargo, una mujer con cabello largo se añade gloria a sí misma, porque
su cabello le es dado para cubrirse. \bibverse{16} Pero si alguno quiere
discutir sobre esto, no tenemos ninguna otra costumbre aparte de esta,
así como tampoco la tienen las otras iglesias de Dios\footnote{\textbf{11:16}
  Al usar la palabra ``costumbre'' o ``hábito'' en lugar de ``norma'' o
  ``mandamiento'' Pablo explica claramente que esta es sencillamente la
  manera como funcionan las cosas en la práctica dentro de la iglesia.}.

\bibverse{17} Ahora, al darles las instrucciones que presentaré a
continuación, no puedo alabarlos, ¡porque cuando se reúnen causan más
daño que bien! \bibverse{18} Primero que nada, he escuchado que cuando
tienen reuniones en la iglesia, están divididos en distintas facciones,
y creo que hay algo de verdad en esto. \bibverse{19} Por supuesto, tales
divisiones entre ustedes deben ocurrir para que los que son sinceros
puedan darse a conocer por medio de su testimonio. \bibverse{20} Cuando
ustedes se reúnen, realmente no están celebrando la Cena del Señor en
absoluto. \bibverse{21} Algunos quieren comer antes que todos los demás,
y dejarlos con hambre. Y todavía hay quienes se emborrachan.
\bibverse{22} ¿Acaso no tienen sus propias casas donde pueden comer y
beber? ¿Menosprecian la casa de Dios, y humillan a los que son pobres?
¿Acaso podría decirles que están haciendo bien? ¡No tengo nada bueno que
decirles por hacer esto!

\bibverse{23} Pues yo he recibido del Señor lo que les enseñé: el Señor
Jesús, en la noche que fue entregado, tomó pan. \bibverse{24} Después de
dar gracias, partió el pan en pedazos y dijo: ``Este pan es mi cuerpo,
el cual es dado para ustedes. Acuérdense de mí al hacer esto.''
\bibverse{25} De la misma manera tomó la copa, y dijo: ``Esta copa es el
nuevo acuerdo,\footnote{\textbf{11:25} Esto traduce la palabra a menudo
  usada como ``pacto,'' la cual tiene un uso limitado en nuestro idioma
  actualmente. El concepto es el de un acuerdo entre dos partes. En este
  caso, se refiere a la relación entre Dios y los seres humanos.}
sellado con mi sangre. Acuérdense de mí cuando la beban. \bibverse{26} Y
cada vez que coman este pan y beban esta copa, ustedes anuncian la
muerte del Señor, hasta su regreso.''

\bibverse{27} De modo que cualquiera que come del pan o bebe de la copa
del Señor con deshonra, será culpable de hacer mal contra el cuerpo y la
sangre del Señor. \bibverse{28} Que cada uno se examine así mismo y
entonces déjenlo comer del pan y beber de la copa. \bibverse{29} Los que
comen y beben traen juicio sobre sí mismos si no reconocen su relación
con el cuerpo del Señor. \bibverse{30} Esa es la razón por la que muchos
de ustedes están débiles y enfermos, e incluso algunos han muerto.
\bibverse{31} Sin embargo, si realmente nos examinamos nosotros mismos,
no seríamos juzgados de esta manera. \bibverse{32} Pero cuando somos
juzgados, estamos siendo disciplinados por el Señor, a fin de que no
seamos condenados junto con el mundo. \bibverse{33} Así que, mis
hermanos y hermanas, cuando se reúnan a comer la Cena del Señor,
espérense unos a otros. \bibverse{34} Si alguno tiene hambre, es mejor
que coma en su casa para que cuando se reúnan no traiga condenación
sobre sí. Les daré más instrucciones cuando vaya a visitarlos.

\hypertarget{section-11}{%
\section{12}\label{section-11}}

\bibverse{1} En cuanto a los ``dones espirituales.''\footnote{\textbf{12:1}
  Pablo retoma otro asunto sobre el cual le han preguntado los
  corintios.} Mis hermanos y hermanas, quiero explicarles esto:
\bibverse{2} Ustedes saben que cuando eran paganos, estaban engañados,
estaban descarriados en la adoración a ídolos que ni siquiera podían
hablar. \bibverse{3} Permítanme ser claro con ustedes: ninguno que habla
en el Espíritu de Dios dice: ``¡Maldigan a Jesús!'' y ninguno puede
decir: ``¡Jesús es el Señor!'' excepto por el Espíritu Santo.
\bibverse{4} Ahora, hay diferentes tipos de dones espirituales, pero
provienen del mismo Espíritu. \bibverse{5} Hay diferentes tipos de
ministerios\footnote{\textbf{12:5} O ``servicio.''}, pero provienen del
mismo Señor. \bibverse{6} Hay diferentes formas de trabajar, pero
provienen del mismo Señor, quien obra en todos ellos. \bibverse{7} El
Espíritu es enviado a cada uno de nosotros y se revela para bien de
todos. \bibverse{8} A una persona el Espíritu le da la capacidad de
hablar palabras de sabiduría. A otra, el mismo espíritu le da mensaje de
conocimiento. \bibverse{9} Otra persona recibe de ese mismo Espíritu el
don de la fe en Dios; alguna otra persona recibe dones de sanidad de
parte de ese mismo Espíritu. \bibverse{10} Otra persona recibe el don de
realizar milagros. Otra, recibe el don de profecía. Otra, recibe el don
del discernimiento espiritual. Otra persona recibe la capacidad de
hablar en diferentes idiomas, mientras que otra recibe el don de
interpretar los idiomas. \bibverse{11} Pero todos estos dones son obra
del único y del mismo Espíritu, haciendo partícipe a cada persona, según
su elección.

\bibverse{12} Así como el cuerpo humano es una unidad pero tiene muchas
partes. Y todas las partes del cuerpo, aunque son muchas, conforman un
cuerpo. Así es Cristo. \bibverse{13} Porque fue por medio de un Espíritu
que todos fuimos bautizados en un cuerpo. No importa si somos judíos o
griegos, esclavos o libres. A todos se nos dio a beber del mismo
Espíritu. \bibverse{14} El cuerpo no está conformado por una sola parte
sino por muchas. \bibverse{15} Si el pie dijera: ``Como no soy mano, no
soy parte del cuerpo,'' ¿dejaría de ser parte del cuerpo? \bibverse{16}
O si el oído dijera: ``Como no soy un ojo, no soy parte del cuerpo,''
¿dejaría de ser parte del cuerpo? \bibverse{17} Y si todo el cuerpo
fuera un ojo, ¿cómo podríamos escuchar? O si todo el cuerpo fuera un
oído, ¿cómo podríamos oler?

\bibverse{18} Pero Dios ha dispuesto cuidadosamente cada parte en el
cuerpo, hasta la más pequeña, y las ubicó tal como quiso hacerlo.
\bibverse{19} Si todas fueran la misma parte, ¿qué ocurriría con el
cuerpo? \bibverse{20} Sin embargo, como hay muchas partes, asi se
conforma el cuerpo. \bibverse{21} El ojo no puede decirle a la mano:
``No te necesito,'' o la cabeza decirle al pie: ``no te necesito.''
\bibverse{22} Muy por el contrario: algunas de esas partes que parecen
ser las menos importantes son las más esenciales. \bibverse{23} De
hecho, las partes del cuerpo que consideramos indignas de ser mostradas,
las ``honramos'' cubriéndolas. ¡Es decir que lo indecente lo tratamos
con mayor modestia! \bibverse{24} Lo que es presentable no necesita
cubrirse de esa manera. Dios ha dispuesto el cuerpo de tal manera que se
le dé mayor honra a las partes que son menos presentables. \bibverse{25}
Esto con el fin de que no haya ningún conflicto en el cuerpo, es decir,
las distintas partes deben considerar igualmente de todas las demás.
\bibverse{26} De modo que cuando una parte del cuerpo sufre, todas las
demás partes sufren con ella, y cuando una parte del cuerpo es bien
tratada, entonces todas las demás partes del cuerpo están felices
también.\footnote{\textbf{12:26} Aquí Pablo parece estar pensando más en
  el cuerpo de la iglesia que en un cuerpo físico.}

\bibverse{27} Ahora bien, ustedes son el cuerpo de Cristo, y cada uno
forma parte de él. \bibverse{28} En la iglesia, Dios ha asignado primero
que algunos sean apóstoles, en segundo lugar, que otros sean profetas, y
en tercer lugar, que algunos sean maestros. Luego están los que hacen
milagros, los que tienen dones de sanidad, los que pueden ayudar a
otros, los que son buenos en la administración, y los que pueden hablar
distintos idiomas. \bibverse{29} No todos son apóstoles, o profetas, o
maestros, o capaces de hacer milagros. \bibverse{30} No todos tienen
dones de sanidad, o la capacidad de hablar distintos idiomas, o de
interpretarlos. \bibverse{31} Pero ustedes deben poner sus corazones en
los dones más importantes\footnote{\textbf{12:31} Después de haber
  debatido sobre los distintos dones espirituales, Pablo afirma que los
  creyentes deben anhelar los dones más importantes. Por supuesto, sería
  un asunto de gran debate determinar cuáles con los más importantes.
  Pero lo que Pablo realmente está haciendo aquí es establecer el
  escenario para el siguiente capítulo, ya que sin amor ninguno de estos
  dones -- incluso los que se consideran más importantes -- no valen de
  nada.}. Así que ahora les mostraré un mejor camino.

\hypertarget{section-12}{%
\section{13}\label{section-12}}

\bibverse{1} Si yo tuviera elocuencia en lenguas humanas---incluso en
lenguas angelicales---pero no tengo amor, sería solo como un metal
ruidoso o címbalo que resuena. \bibverse{2} Si profetizara, si conociera
todos los misterios y tuviera todo conocimiento, y si pudiera tener una
fe tal que pudiera mover montañas, pero no tengo amor, entonces nada
soy. \bibverse{3} Si pudiera donar todo lo que poseo a los pobres, o si
me sacrificara para ser quemado como mártir, y no tengo amor, entonces
no habría logrado nada.

\bibverse{4} El amor es paciente y amable. El amor no es celoso. El amor
no es jactancioso. El amor no es orgulloso. \bibverse{5} El amor no
actúa de manera inapropiada ni insiste en salirse con la suya. El amor
no es contencioso ni guarda registro de los errores. \bibverse{6} El
amor no se deleita en el mal, sino que se alegra en la verdad.
\bibverse{7} El amor nunca se rinde, sigue creyendo, mantiene la
confianza, y espera con paciencia en todas las circunstancias.

\bibverse{8} El amor nunca falla. Las profecías se acabarán. Las lenguas
se callarán. El conocimiento se volverá inútil. \bibverse{9} Porque
nuestro conocimiento y nuestra comprensión profética están incompletos.
\bibverse{10} Pero cuando esté completo, entonces lo que está incompleto
desaparecerá. \bibverse{11} Cuando era un niño, hablaba como niño,
pensaba como niño y razonaba como niño. Pero cuando crecí dejé atrás las
cosas de niño. \bibverse{12} Ahora vemos como en un espejo con un
reflejo borroso, pero entonces veremos cara a cara. Porque ahora solo
tengo un conocimiento parcial, pero entonces conoceré por completo, tal
como soy completamente conocido. \bibverse{13} La confianza, la
esperanza, y el amor duran para siempre, pero el más importante es el
amor.

\hypertarget{section-13}{%
\section{14}\label{section-13}}

\bibverse{1} ¡Hagan del amor su objetivo más importante! Pero también
hagan su mejor esfuerzo para lograr los dones espirituales,
especialmente la capacidad de predicar el mensaje de Dios\footnote{\textbf{14:1}
  Literalmente ``profetizar,'' pero en el sentido de contar la buena
  noticia, más que predecir el futuro. Aquí se usan los términos
  ``hablar el mensaje de Dios,'' ``el mensaje profético de Dios,'' o
  ``hablar en lugar de Dios.''}. \bibverse{2} Los que hablan en
lengua\footnote{\textbf{14:2} Claramente esto no se refiere al uso del
  lenguaje humano normal. Existe mucho debate sobre este fenómeno. Sin
  duda, la iglesia primitiva recibió el don de hablar y ser entendida en
  distintas lenguas humanas, como queda claro en Hechos 2. Sin embargo,
  parece que aquí está considerándose un ``habla extática.'' Era una
  práctica de la cual se estaba abusando en Corinto, y por ello pablo
  tiene que contrarrestar este problema.} no están hablando con las
personas, sino con Dios, porque nadie puede entenderles, pues habla
misterios en el Espíritu. \bibverse{3} No obstante, las palabras de los
que hablan por Dios, edifican a la gente, proporcionan ánimo y consuelo.
\bibverse{4} Los que hablan en una lengua se edifican solo a sí mismos,
pero los que hablan el mensaje de Dios edifican a toda la iglesia.
Desearía que todos ustedes hablaran en lenguas, pero preferiría que
pudieran predicar el mensaje de Dios. \bibverse{5} Los que predican a
Dios son más importantes que los que hablan en lenguas, a menos que
interpreten lo que se ha dicho, a fin de que la iglesia sea edificada.

\bibverse{6} Hermanos y hermanas, si yo vengo a ustedes hablando en
lenguas, ¿qué beneficio les aportaría si no les traigo una revelación,
un conocimiento, o un mensaje profético, o una enseñanza? \bibverse{7}
Incluso cuando se trata de objetos sin vida, tal como los instrumentos
musicales, como la flauta o el harpa: si no producen notas claras ¿cómo
sabremos qué melodía se está tocando? \bibverse{8} Del mismo modo, si la
trompeta no emite un sonido claro, ¿quién se alistará para la batalla?
\bibverse{9} Lo mismo ocurre con ustedes: a menos que hablen con
palabras que sean fáciles de entender, ¿quién podrá saber lo que están
diciendo? Lo que dicen se perderá en el viento. \bibverse{10} Sin duda
alguna, hay muchos idiomas en este mundo, y cada uno tiene su
significado. \bibverse{11} Pero si yo no comprendo el idioma, los que
hablan no tienen sentido para mí, ni yo tengo sentido para
ellos\footnote{\textbf{14:11} Literalmente, soy un bárbaro para el que
  habla, y el que habla es un bárbaro para mí. La misma palabra
  ``bárbaro'' nace de la idea de que los sonidos producidos no tienen
  sentido---``baa-baa'' etc.}. \bibverse{12} Lo mismo ocurre con
ustedes: si están ansiosos por tener dones espirituales, traten de tener
muchos de los que edifican a la iglesia. \bibverse{13} Todo el que habla
en una lengua debe orar para que se le dé la capacidad de traducir lo
que dice. \bibverse{14} Porque si yo oro en voz alta en una lengua, mi
espíritu está orando, ¡pero no aporta nada a mi comprensión!

\bibverse{15} Entonces ¿qué debo hacer? Oraré ``en el Espíritu,'' pero
oraré con mi mente también. Cantaré ``en el Espíritu,'' pero cantaré con
mi mente también\footnote{\textbf{14:15} Aquí Pablo parece estar usando
  la fraseología de algunos en Corinto que estaban orgullosos de estar
  ``en el Espíritu'' como si eso fuera superior a cualquier otra cosa.
  Pablo señala que estar ``en el Espíritu'' no sirve de nada a menos que
  produzca entendimiento.}. \bibverse{16} Pues si ustedes solo oran ``en
el Espíritu,'' ¿Cómo podrán decir ``amén'' las personas comunes, después
de tu oración de agradecimiento, si no entendieron lo que dijiste?
\bibverse{17} Puede que hayas hecho una oración de agradecimiento
maravillosa, ¡pero no ayudó a los demás! \bibverse{18} Doy gracias a
Dios que puedo hablar en lenguas más que todos ustedes. \bibverse{19}
Pero en la iglesia, preferiría pronunciar cinco palabras que sean
entendidas por los demás, que diez mil palabras en una lengua que nadie
entiende. \bibverse{20} Hermanos y hermanas, no piensen como niños. Sean
inocentes como niños pequeños en lo que se refiere al mal, pero sean
adultos en su comprensión. \bibverse{21} Como registra la Escritura:
``\,`Hablaré a mi pueblo por medio de otros idiomas y labios de
extranjeros, pero incluso así no me escucharán,' dice el
Señor.''\footnote{\textbf{14:21} Isaías 28:11, 12.}

\bibverse{22} Hablar en lenguas es una señal, no para los creyentes,
sino para los que no creen. Hablar el mensaje profético de Dios es lo
contrario: no es para los que no creen, sino para los que creen.
\bibverse{23} Si toda la iglesia se reuniera y todos hablaran en
lenguas, y llegaran allí ciertas personas que no entienden, o si llegan
personas que no creen, ¿no pensarán que todos ustedes están locos?
\bibverse{24} Pero si todos hablan el mensaje de Dios, y alguno que no
es creyente llega allí, o alguien que no entiende, se convencerá y
sentirá el llamado por las palabras de todos. \bibverse{25} Los secretos
de su corazón quedarán descubiertos, y caerán de rodillas\footnote{\textbf{14:25}
  Literalmente, ``caerán ante su rostro.''} y adorarán a Dios, afirmando
que Dios está entre ustedes.

\bibverse{26} Entonces, hermanos y hermanas, ¿qué deben hacer? Cuando se
reúnan, que distintas personas canten, o enseñen, o prediquen un mensaje
especial, o hablen en lenguas, o den una interpretación. Pero todo debe
hacerse para edificar y animar a la iglesia. \bibverse{27} Si alguno
quiere hablar en una lengua, que sean solo dos, o máximo tres personas,
tomando turnos, y que alguno interprete lo que se dice. \bibverse{28} Si
no hay quien interprete, entonces los que hablan en lenguas deben
guardar silencio y solo hablar para sí mismos y para Dios. \bibverse{29}
Del mismo modo, permitan que hablen dos o tres de las personas que
predican el mensaje profético de Dios, y dejen que todos los demás
reflexionen sobre lo que se dijo. \bibverse{30} Sin embargo, si alguno
de los que están sentados recibe una revelación, entonces quien estaba
predicando debe darle la oportunidad de hablar. \bibverse{31} Todos
ustedes pueden predicar acerca de Dios, uno a la vez, para que todos
puedan aprender y animarse. \bibverse{32} Quienes predican acerca de
Dios deben controlar su inspiración profética, \bibverse{33} porque Dios
no es un Dios de desorden, sino de paz y calma\footnote{\textbf{14:33}
  Esta afirmación es para confrontar una situación donde las personas
  presumían de una revelación/inspiración especial, y por ello exigían
  tener prioridad, acompañada de las obvias discusiones que esto traería
  como resultado.}. Y así es como deben hacerse las cosas en las
iglesias del pueblo de Dios. \bibverse{34} Las mujeres deben permanecer
en silencio en las iglesias. No deberían hablar. Deben tener respeto por
su situación, como lo dicen las leyes. \bibverse{35} Si ellas quieren
aprender, pueden hacerlo en casa, preguntando a sus esposos. No es
apropiado que una mujer hable en la iglesia\footnote{\textbf{14:35} ``No
  es apropiado.'' Al usar este término (que también denota algo
  vergonzoso o deshonroso) Pablo revela que esto está relacionado con el
  contexto cultural. El hecho de que tres capítulos antes, en 11:5,
  Pablo haga referencia a mujeres orando y profetizando, indica que esta
  afirmación no puede leerse como una prohibición general a que las
  mujeres hablen en la iglesia. Según el contexto, parece ser posible
  que las mujeres en la iglesia de Corinto estaban debatiendo y
  cuestionando, formando parte del desorden que él menciona, y esto
  puede ser a lo que él se está refiriendo aquí. Algunos otros han
  sugerido que 14:34-35 define la posición de algunos en la iglesia de
  Corinto, y Pablo los está citando antes de refutar su argumento.}.
\bibverse{36} ¿Acaso la palabra de Dios comenzó con ustedes? ¿Fueron
ustedes los únicos que la recibieron? \bibverse{37} Todo el que crea que
es profeta, o que tiene algún don espiritual, debe saber que lo que les
escribo es un mandato del Señor. \bibverse{38} Aquellos que ignoran esto
serán ignorados también. \bibverse{39} Así que, mis hermanos y hermanas,
que su objetivo sea predicar acerca de Dios. No prohíban el hablar en
lenguas. \bibverse{40} Solo asegúrense de que todo sea hecho en orden y
de manera apropiada.

\hypertarget{section-14}{%
\section{15}\label{section-14}}

\bibverse{1} Ahora quiero recordarles sobre la buena nueva que les
anuncié. Ustedes la aceptaron y se han mantenido firmes en ella.
\bibverse{2} Por medio de esta buena noticia es que ustedes son salvos,
si se aferran al mensaje que les di. ¡De lo contrario, habrán creído sin
propósito alguno! \bibverse{3} Yo les di lo que yo mismo también recibí,
un mensaje de vital importancia: que Cristo murió por nuestros pecados,
conforme dice la Escritura; \bibverse{4} fue sepultado y resucitó de los
muertos el tercer día, conforme dice la Escritura también. \bibverse{5}
Se le apareció a Pedro, y después a los doce. \bibverse{6} Después de
eso, se le apareció a más de cinco mil hermanos y hermanas al mismo
tiempo, muchos de los cuales aún viven, aunque algunos murieron ya.
\bibverse{7} Se le apareció a Santiago, luego a todos los apóstoles.
\bibverse{8} Al final, se me apareció a mí también, que nací como en el
tiempo equivocado. \bibverse{9} Porque soy el menos importante de todos
los apóstoles, ni siquiera adecuado para ser llamado apóstol, siendo que
perseguí a la iglesia de Dios. \bibverse{10} Pero por la gracia de Dios
soy lo que soy, y su gracia por mí no fue desperdiciada. Por el
contrario, he trabajado con más esfuerzo que todos ellos, aunque no fui
yo, sino la gracia de Dios obrando en mí. \bibverse{11} Así que no
importa si soy yo o son ellos, este es el mensaje que compartimos con
ustedes y que los llevó a creer en Dios.

\bibverse{12} Ahora, si el mensaje declara que Cristo resucitó de los
muertos, ¿cómo es que algunos de ustedes dicen que no hay resurrección
de los muertos? \bibverse{13} Si no hubiera resurrección de los muertos,
entonces Cristo tampoco ha resucitado. \bibverse{14} Y si Cristo no
resucitó, entonces nuestro mensaje es en vano, y su fe en Dios también
lo es. \bibverse{15} Además, seríamos falsos testigos de Dios al decir
que Dios levantó a Cristo de los muertos. Pero si es cierto que no hay
resurrección, entonces Dios no levantó a Cristo de los muertos.
\bibverse{16} Y si los muertos no resucitan, entonces Cristo no resucitó
tampoco, \bibverse{17} y si Cristo no fue resucitado, la fe de ustedes
en Dios es inútil, y todavía siguen en sus pecados. \bibverse{18} Esto
también significa que los que murieron en Cristo están perdidos.
\bibverse{19} Y si nuestra esperanza en Cristo solo es para esta vida,
nadie es más digno de lástima que nosotros.

\bibverse{20} Pero Cristo fue levantado de los muertos, las
primicias\footnote{\textbf{15:20} La palabra ``primicia'' se refiere a
  la primera muestra de una cosecha que era dada como ofrenda a Dios, y
  que también era garantía de una cosecha exitosa. Por ello, Cristo,
  como primicia, se refiere a la cosecha inicial de los muertos y la
  garantía de su resurrección.} de la cosecha de los que han muerto.
\bibverse{21} Así como la muerte vino por un hombre, la resurrección de
los muertos también vino por un hombre. \bibverse{22} Así como en Adán
todos mueren, también en Cristo todos serán resucitados. \bibverse{23}
Pero cada uno a su tiempo: Cristo como las primicias, y luego los que
pertenecen a Cristo, cuando él venga. \bibverse{24} Después de esto
vendrá el fin, cuando Cristo entregue el reino al Padre, después de
haber destruido\footnote{\textbf{15:24} Destruidos, en el sentido de
  acabar con su poder.} a todos los gobernantes, autoridades y
potencias. \bibverse{25} Cristo tiene que gobernar hasta que haya puesto
a todos sus enemigos bajo sus pies\footnote{\textbf{15:25} Queriendo
  decir que han sido conquistados y humillados.}. \bibverse{26} El
último enemigo que será destruido es la muerte. \bibverse{27} Como dice
la Escritura: ``Él puso todo bajo sus pies.'' (Por supuesto, cuando dice
que ``todo'' está bajo sus pies, es obvio que no se refiere a Dios,
quien puso todo bajo la autoridad de Cristo.) \bibverse{28} Cuando todo
haya sido puesto bajo la autoridad de Cristo, entonces el Hijo también
se pondrá bajo la autoridad de Dios, para que Dios, quien le dio
autoridad al Hijo sobre todas las cosas, pueda ser todo en todas las
cosas\footnote{\textbf{15:28} ``Todo en todos.'' Se han dado diversas
  explicaciones para esta frase. Obviamente se está refiriendo a la
  plenitud del gobierno de Dios en el universo, y probablemente se
  traduce mejor de manera literal del texto original como ``todo en
  todo.''}.

\bibverse{29} De otro modo, ¿qué harán las personas que son bautizadas
por los muertos? Si los muertos no resucitan, ¿por qué bautizar a la
gente por ellos?\footnote{\textbf{15:29} El significado teológico de
  este versículo es tema de gran debate. Sin embargo, las palabras
  reales traducidas son suficientes.} \bibverse{30} En cuanto a
nosotros, ¿por qué nos exponemos al peligro en todo momento?
\bibverse{31} Déjenme decirles claramente, mis hermanos y hermanas: Yo
muero cada día. Y esto es tan seguro como el orgullo que tengo por lo
que Cristo ha hecho en ustedes. \bibverse{32} Humanamente hablando, ¿qué
ganaría yo discutiendo con las personas que están en Éfeso, que son como
bestias salvajes, si los muertos no resucitan? Si es así, entonces
``¡comamos y bebamos, que mañana moriremos''!

\bibverse{33} No se dejen engañar: ``las malas compañías dañan el buen
carácter.'' \bibverse{34} ¡Recobren la razón y dejen de pecar! Algunos
entre ustedes no conocen a Dios. Y les digo esto para avergonzarlos.

\bibverse{35} Por supuesto, alguno preguntará: ¿cómo resucitan los
muertos? ¿Qué tipo de cuerpo tendrán?'' \bibverse{36} ¡Cuán necia es
esta pregunta! Lo que sembramos no germina a menos que muera.
\bibverse{37} Cuando ustedes siembran, no siembran la planta como esta
es al crecer, sino la semilla solamente, ya sea trigo o cualquier otra
semilla que estén sembrando. \bibverse{38} Dios hace que la planta
crezca de la manera que él lo ha determinado, y cada semilla produce
plantas distintas, con diferentes formas. \bibverse{39} Los seres vivos
están hechos de diferentes formas. Los seres humanos tienen un tipo de
tejido en sus cuerpos, mientras que los animales tienen otro, las aves
otro, y los peces, otro. \bibverse{40} Hay cuerpos celestiales y cuerpos
terrenales. Los cuerpos celestiales tienen un tipo de belleza, mientras
que los cuerpos terrenales tienen otro tipo. \bibverse{41} El sol brilla
de una manera, la luna de otra manera, mientras que las estrellas
también son diferentes, cada una brillando de manera distinta.

\bibverse{42} Lo mismo ocurre con la resurrección de los muertos. El
cuerpo es enterrado en descomposición, pero es resucitado para perdurar
eternamente. \bibverse{43} Es sembrado con tristeza, pero es levantado
en gloria. Es enterrado en debilidad, pero es levantado en poder.
\bibverse{44} Es enterrado como un cuerpo natural, pero es levantado
como un cuerpo espiritual. Pues así como hay cuerpos naturales, también
hay cuerpos espirituales. \bibverse{45} Como dice la Escritura: ``El
primer hombre, Adán, se convirtió en un ser vivo;''\footnote{\textbf{15:45}
  Génesis 2:7.} pero el último Adán, en un espíritu que da vida.
\bibverse{46} El Adán espiritual no vino primero, sino el natural. El
Adán espiritual vino después. \bibverse{47} El primer hombre vino del
polvo de la tierra; el segundo vino del cielo. \bibverse{48} Las
personas terrenales son como el hombre hecho de la tierra; las personas
celestiales son como el hombre que vino del cielo. \bibverse{49} Así
como heredamos la semejanza del hombre terrenal, también heredaremos la
semejanza del hombre celestial. \bibverse{50} No obstante, les digo, mis
hermanos y hermanas: nuestros cuerpos presentes\footnote{\textbf{15:50}
  Literalmente, ``de carne y sangre.''} no pueden heredar el reino de
Dios. Estos cuerpos mortales no pueden heredar lo eterno.

\bibverse{51} Escuchen, voy a revelarles un misterio: No todos
moriremos, pero todos seremos transformados, \bibverse{52} en un
momento, en un abrir y cerrar de un ojo, al sonido de la última
trompeta. Esta sonará, y los muertos serán levantados para no morir más,
y nosotros seremos transformados. \bibverse{53} Porque este cuerpo
corruptible debe vestirse de un cuerpo incorruptible. Esta vida mortal
debe vestirse de inmortalidad. \bibverse{54} Cuando este cuerpo
corruptible se haya vestido de un cuerpo incorruptible, y esta vida
mortal se haya vestido de inmortalidad, entonces se cumplirá lo que dice
la Escritura: ``La muerte ha sido completamente conquistada y destruida.
\bibverse{55} Muerte, ¿dónde está tu victoria? ¿Dónde está tu
aguijón?\footnote{\textbf{15:55} Isaías 25:8; Oseas 13:14.}
\bibverse{56} El aguijón que causa la muerte es el pecado; y el poder
del pecado es la ley; \bibverse{57} pero alabemos a Dios, quien nos da
la victoria por medio de nuestro Señor Jesucristo. \bibverse{58} Así
que, mis queridos hermanos y hermanas: sean fuertes, permanezcan firmes,
haciendo todo lo que puedan por la obra del Señor, pues saben que
ninguna cosa que hagan por él es en vano.

\hypertarget{section-15}{%
\section{16}\label{section-15}}

\bibverse{1} En cuanto a ``recoger dinero para los hermanos creyentes,''
les doy las mismas instrucciones que les di a las iglesias de Galacia.
\bibverse{2} El primer día de la semana, todos deben apartar dinero del
que han ganado. No quisiera que se recogiera dinero cuando estoy con
ustedes. \bibverse{3} Cuando llegue, escribiré cartas de recomendación
para la persona que escojan, y esa persona llevará sus donativos a
Jerusalén. \bibverse{4} Si resulta que yo puedo ir también, entonces
ellos pueden ir conmigo.

\bibverse{5} Después de haber ido a Macedonia, tengo planes de ir a
visitarlos. Debo pasar por allí de camino a Macedonia \bibverse{6} y
puedo quedarme con ustedes por un tiempo, quizás por la temporada de
invierno, y después podrán enviarme nuevamente de camino hacia donde
voy. \bibverse{7} Esta vez no quiero ir a verlos por poco tiempo. Espero
poder quedarme más tiempo con ustedes, si el Señor lo permite.
\bibverse{8} Sin embargo me quedaré en Éfeso hasta el Pentecostés,
\bibverse{9} porque se me ha presentado una gran oportunidad allí,
aunque tengo también muchos opositores.

\bibverse{10} Ahora, si Timoteo llega, asegúrense de que no tenga ningún
temor de estar con ustedes, porque él está trabajando por el Señor tal
como yo lo hago. \bibverse{11} No permitan que nadie lo menosprecie.
Envíenlo con alegría en su viaje para que pueda venir a verme. Los
hermanos, hermanas y yo estamos esperándolo. \bibverse{12} En cuanto a
nuestro hermano Apolo: le insistí en que fuera a verlos junto con los
otros creyentes, pero no tenía disposición de ir en el momento. Él irá a
visitarlos cuando tenga la oportunidad de hacerlo.

\bibverse{13} Estén alerta. Manténganse firmes en su confianza en Dios.
Tengan valor. Sean fuertes. \bibverse{14} Todo lo que hagan, háganlo con
amor. \bibverse{15} Ustedes saben que Estéfanas y su familia estaban
entre los primeros conversos de Acaya, y se dedicaron a ayudar al pueblo
de Dios. Les ruego, hermanos y hermanas, \bibverse{16} que respeten su
liderazgo, y así mismo a todos los que ayudan en la obra con tanta
dedicación. \bibverse{17} Me alegro de que Estéfanas, Fortunata y Acaico
hayan llegado, porque lograron lo que ustedes no pudieron hacer.
\bibverse{18} Ellos han sido fuente de mucho ánimo para mí, y para
ustedes. Las personas como ellos merecen el reconocimiento de ustedes.

\bibverse{19} Las iglesias de Asia+ 16.19 Refiriéndose a la provincia
romana de Asia Menor. envían su saludo. Aquila y Priscila, junto con la
iglesia que se congrega en su casa, envían sus mejores deseos.
\bibverse{20} Todos los hermanos y hermanas aquí envían su saludo.
Salúdense unos a otros con afecto. \bibverse{21} Yo, Pablo, escribo este
saludo con mi propia mano. \bibverse{22} Cualquiera que no ama al Señor
debe ser excluido de la iglesia+ 16.22 Literalmente, ``sea maldito.''.
¡Ven Señor! \bibverse{23} Que la gracia de nuestro Señor Jesucristo esté
con ustedes. \bibverse{24} Reciban mi amor para todos ustedes en Cristo
Jesús. Amén.
