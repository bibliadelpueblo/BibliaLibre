\hypertarget{el-nacimiento-y-ordenaciuxf3n-de-samuel-como-siervo-del-seuxf1or-en-silo-canciuxf3n-de-alabanza-de-hanna}{%
\subsection{El nacimiento y ordenación de Samuel como siervo del Señor
en Silo; Canción de alabanza de
Hanna}\label{el-nacimiento-y-ordenaciuxf3n-de-samuel-como-siervo-del-seuxf1or-en-silo-canciuxf3n-de-alabanza-de-hanna}}

\hypertarget{section-09-1}{%
\section{1}\label{section-09-1}}

\bibleverse{1} Hubo un varón de Ramathaim de Sophim, del monte de
Ephraim, que se llamaba Elcana, hijo de Jeroham, hijo de Eliú, hijo de
Thohu, hijo de Suph, Ephrateo. \bibleverse{2} Y tenía él dos mujeres; el
nombre de la una era Anna, y el nombre de la otra Peninna. Y Peninna
tenía hijos, mas Anna no los tenía. \bibleverse{3} Y subía aquel varón
todos los años de su ciudad, á adorar y sacrificar á Jehová de los
ejércitos en Silo, donde estaban dos hijos de Eli, Ophni y Phinees,
sacerdotes de Jehová. \bibleverse{4} Y cuando venía el día, Elcana
sacrificaba, y daba á Peninna su mujer, y á todos sus hijos y á todas
sus hijas, á cada uno su parte. \bibleverse{5} Mas á Anna daba una parte
escogida; porque amaba á Anna, aunque Jehová había cerrado su matriz.
\bibleverse{6} Y su competidora la irritaba, enojándola y
entristeciéndola, porque Jehová había cerrado su matriz. \bibleverse{7}
Y así hacía cada año: cuando subía á la casa de Jehová, enojaba así á la
otra; por lo cual ella lloraba, y no comía. \bibleverse{8} Y Elcana su
marido le dijo: Anna, ¿por qué lloras? ¿y por qué no comes? ¿y por qué
está afligido tu corazón? ¿No te soy yo mejor que diez hijos?

\hypertarget{los-votos-de-hanna-en-silo-y-su-conversaciuxf3n-con-eli}{%
\subsection{Los votos de Hanna en Silo y su conversación con
Eli}\label{los-votos-de-hanna-en-silo-y-su-conversaciuxf3n-con-eli}}

\bibleverse{9} Y levantóse Anna después que hubo comido y bebido en
Silo; y mientras el sacerdote Eli estaba sentado en una silla junto á un
pilar del templo de Jehová, \bibleverse{10} Ella con amargura de alma
oró á Jehová, y lloró abundantemente. \bibleverse{11} E hizo voto,
diciendo: Jehová de los ejércitos, si te dignares mirar la aflicción de
tu sierva, y te acordares de mí, y no te olvidares de tu sierva, mas
dieres á tu sierva un hijo varón, yo lo dedicaré á Jehová todos los días
de su vida, y no subirá navaja sobre su cabeza.

\bibleverse{12} Y fué que como ella orase largamente delante de Jehová,
Eli estaba observando la boca de ella. \bibleverse{13} Mas Anna hablaba
en su corazón, y solamente se movían sus labios, y su voz no se oía; y
túvola Eli por borracha. \bibleverse{14} Entonces le dijo Eli: ¿Hasta
cuándo estarás borracha?; digiere tu vino.

\bibleverse{15} Y Anna le respondió, diciendo: No, señor mío: mas yo soy
una mujer trabajada de espíritu: no he bebido vino ni sidra, sino que he
derramado mi alma delante de Jehová. \bibleverse{16} No tengas á tu
sierva por una mujer impía: porque por la magnitud de mis congojas y de
mi aflicción he hablado hasta ahora.

\bibleverse{17} Y Eli respondió, y dijo: Ve en paz, y el Dios de Israel
te otorgue la petición que le has hecho.

\bibleverse{18} Y ella dijo: Halle tu sierva gracia delante de tus ojos.
Y fuése la mujer su camino, y comió, y no estuvo más triste.

\hypertarget{nacimiento-de-samuel-primera-infancia-y-consagraciuxf3n-en-silo}{%
\subsection{Nacimiento de Samuel, primera infancia y consagración en
Silo}\label{nacimiento-de-samuel-primera-infancia-y-consagraciuxf3n-en-silo}}

\bibleverse{19} Y levantándose de mañana, adoraron delante de Jehová, y
volviéronse, y vinieron á su casa en Ramatha. Y Elcana conoció á Anna su
mujer, y Jehová se acordó de ella.

\bibleverse{20} Y fué que corrido el tiempo, después de haber concebido
Anna, parió un hijo, y púsole por nombre Samuel, diciendo: Por cuanto lo
demandé á Jehová.

\bibleverse{21} Después subió el varón Elcana, con toda su familia, á
sacrificar á Jehová el sacrificio acostumbrado, y su voto.
\bibleverse{22} Mas Anna no subió, sino dijo á su marido: Yo no subiré
hasta que el niño sea destetado; para que lo lleve y sea presentado
delante de Jehová, y se quede allá para siempre.

\bibleverse{23} Y Elcana su marido le respondió: Haz lo que bien te
pareciere; quédate hasta que lo destetes: solamente Jehová cumpla su
palabra. Y quedóse la mujer, y crió su hijo hasta que lo destetó.

\bibleverse{24} Y después que lo hubo destetado, llevólo consigo, con
tres becerros, y un epha de harina, y una vasija de vino, y trájolo á la
casa de Jehová en Silo: y el niño era pequeño. \bibleverse{25} Y matando
el becerro, trajeron el niño á Eli. \bibleverse{26} Y ella dijo: ¡Oh,
señor mío! vive tu alma, señor mío, yo soy aquella mujer que estuvo aquí
junto á ti orando á Jehová. \bibleverse{27} Por este niño oraba, y
Jehová me dió lo que le pedí. \bibleverse{28} Yo pues le vuelvo también
á Jehová: todos los días que viviere, será de Jehová. Y adoró allí á
Jehová.

\hypertarget{himno-de-alabanza-a-hanna-inicio-de-servicio-de-samuel-en-silo}{%
\subsection{Himno de alabanza a Hanna; Inicio de servicio de Samuel en
Silo}\label{himno-de-alabanza-a-hanna-inicio-de-servicio-de-samuel-en-silo}}

\hypertarget{section-09-2}{%
\section{2}\label{section-09-2}}

\bibleverse{1} Y Anna oró y dijo: mi corazón se regocija en Jehová, mi
cuerno es ensalzado en Jehová; mi boca se ensanchó sobre mis enemigos,
por cuanto me alegré en tu salud. \bibleverse{2} No hay santo como
Jehová: porque no hay ninguno fuera de ti; y no hay refugio como el Dios
nuestro. \bibleverse{3} No multipliquéis hablando grandezas, altanerías;
cesen las palabras arrogantes de vuestra boca; porque el Dios de todo
saber es Jehová, y á él toca el pesar las acciones. \bibleverse{4} Los
arcos de los fuertes fueron quebrados, y los flacos se ciñeron de
fortaleza. \bibleverse{5} Los hartos se alquilaron por pan: y cesaron
los hambrientos: hasta parir siete la estéril, y la que tenía muchos
hijos enfermó. \bibleverse{6} Jehová mata, y él da vida: él hace
descender al sepulcro, y hace subir. \bibleverse{7} Jehová empobrece, y
él enriquece: abate, y ensalza. \bibleverse{8} El levanta del polvo al
pobre, y al menesteroso ensalza del estiércol, para asentarlo con los
príncipes; y hace que tengan por heredad asiento de honra: porque de
Jehová son las columnas de la tierra, y él asentó sobre ellas el mundo.
\bibleverse{9} El guarda los pies de sus santos, mas los impíos perecen
en tinieblas; porque nadie será fuerte por su fuerza. \bibleverse{10}
Delante de Jehová serán quebrantados sus adversarios, y sobre ellos
tronará desde los cielos: Jehová juzgará los términos de la tierra, y
dará fortaleza á su Rey, y ensalzará el cuerno de su Mesías.

\bibleverse{11} Y Elcana se volvió á su casa en Ramatha; y el niño
ministraba á Jehová delante del sacerdote Eli.

\hypertarget{la-maldad-de-los-hijos-de-eluxed-anuncio-del-juicio-divino}{%
\subsection{La maldad de los hijos de Elí; Anuncio del juicio
divino}\label{la-maldad-de-los-hijos-de-eluxed-anuncio-del-juicio-divino}}

\bibleverse{12} Mas los hijos de Eli eran hombres impíos, y no tenían
conocimiento de Jehová. \bibleverse{13} Y la costumbre de los sacerdotes
con el pueblo era que, cuando alguno ofrecía sacrificio, venía el criado
del sacerdote mientras la carne estaba á cocer, trayendo en su mano un
garfio de tres ganchos; \bibleverse{14} Y hería con él en la caldera, ó
en la olla, ó en el caldero, ó en el pote; y todo lo que sacaba el
garfio, el sacerdote lo tomaba para si. De esta manera hacían á todo
Israelita que venía á Silo. \bibleverse{15} Asimismo, antes de quemar el
sebo, venía el criado del sacerdote, y decía al que sacrificaba: Da
carne que ase para el sacerdote; porque no tomará de ti carne cocida,
sino cruda.

\bibleverse{16} Y si le respondía el varón, Quemen luego el sebo hoy, y
después toma tanta como quisieres; él respondía: No, sino ahora la has
de dar: de otra manera yo la tomaré por fuerza. \bibleverse{17} Era pues
el pecado de los mozos muy grande delante de Jehová; porque los hombres
menospreciaban los sacrificios de Jehová.

\hypertarget{hanna-y-el-niuxf1o-del-coro-samuel}{%
\subsection{Hanna y el niño del coro
Samuel}\label{hanna-y-el-niuxf1o-del-coro-samuel}}

\bibleverse{18} Y el joven Samuel ministraba delante de Jehová, vestido
de un ephod de lino. \bibleverse{19} Y hacíale su madre una túnica
pequeña, y traíasela cada año, cuando subía con su marido á ofrecer el
sacrificio acostumbrado. \bibleverse{20} Y Eli bendijo á Elcana y á su
mujer, diciendo: Jehová te dé simiente de esta mujer en lugar de esta
petición que hizo á Jehová. Y volviéronse á su casa. \bibleverse{21} Y
visitó Jehová á Anna, y concibió, y parió tres hijos, y dos hijas. Y el
joven Samuel crecía delante de Jehová.

\hypertarget{las-suaves-amonestaciones-de-eluxed-a-sus-hijos-degenerados}{%
\subsection{Las suaves amonestaciones de Elí a sus hijos
degenerados}\label{las-suaves-amonestaciones-de-eluxed-a-sus-hijos-degenerados}}

\bibleverse{22} Eli empero era muy viejo, y oyó todo lo que sus hijos
hacían á todo Israel, y como dormían con las mujeres que velaban á la
puerta del tabernáculo del testimonio. \bibleverse{23} Y díjoles: ¿Por
qué hacéis cosas semejantes? Porque yo oigo de todo este pueblo vuestros
malos procederes. \bibleverse{24} No, hijos míos; porque no es buena
fama la que yo oigo: que hacéis pecar al pueblo de Jehová.
\bibleverse{25} Si pecare el hombre contra el hombre, los jueces le
juzgarán; mas si alguno pecare contra Jehová, ¿quién rogará por él? Mas
ellos no oyeron la voz de su padre, porque Jehová los quería matar.

\bibleverse{26} Y el joven Samuel iba creciendo, y adelantando delante
de Dios y delante de los hombres.

\hypertarget{refruxe1n-del-profeta-anuncio-de-la-cauxedda-de-eli-y-su-casa}{%
\subsection{Refrán del Profeta: Anuncio de la caída de Eli y su
casa}\label{refruxe1n-del-profeta-anuncio-de-la-cauxedda-de-eli-y-su-casa}}

\bibleverse{27} Y vino un varón de Dios á Eli, y díjole: Así ha dicho
Jehová: ¿No me manifesté yo claramente á la casa de tu padre, cuando
estaban en Egipto en casa de Faraón? \bibleverse{28} Y yo le escogí por
mi sacerdote entre todas las tribus de Israel, para que ofreciese sobre
mi altar, y quemase perfume, y trajese ephod delante de mí; y dí á la
casa de tu padre todas las ofrendas de los hijos de Israel.
\bibleverse{29} ¿Por qué habéis hollado mis sacrificios y mis presentes,
que yo mandé ofrecer en el tabernáculo; y has honrado á tus hijos más
que á mí, engordándoos de lo principal de todas las ofrendas de mi
pueblo Israel? \bibleverse{30} Por tanto, Jehová el Dios de Israel dice:
Yo había dicho que tu casa y la casa de tu padre andarían delante de mí
perpetuamente; mas ahora ha dicho Jehová: Nunca yo tal haga, porque yo
honraré á los que me honran, y los que me tuvieren en poco, serán viles.
\bibleverse{31} He aquí vienen días, en que cortaré tu brazo, y el brazo
de la casa de tu padre, que no haya viejo en tu casa. \bibleverse{32} Y
verás competidor en el tabernáculo, en todas las cosas en que hiciere
bien á Israel; y en ningún tiempo habrá viejo en tu casa.
\bibleverse{33} Y no te cortaré del todo varón de mi altar, para hacerte
marchitar tus ojos, y henchir tu ánimo de dolor; mas toda la cría de tu
casa morirá en la edad varonil. \bibleverse{34} Y te será por señal esto
que acontecerá á tus dos hijos, Ophni y Phinees: ambos morirán en un
día. \bibleverse{35} Y yo me suscitaré un sacerdote fiel, que haga
conforme á mi corazón y á mi alma; y yo le edificaré casa firme, y
andará delante de mi ungido todos los días. \bibleverse{36} Y será que
el que hubiere quedado en tu casa, vendrá á postrársele por un dinero de
plata y un bocado de pan, diciéndole: Ruégote que me constituyas en
algún ministerio, para que coma un bocado de pan.

\hypertarget{dios-se-revela-a-samuel-y-anuncia-la-cauxedda-de-la-casa-de-eluxed}{%
\subsection{Dios se revela a Samuel y anuncia la caída de la casa de
Elí}\label{dios-se-revela-a-samuel-y-anuncia-la-cauxedda-de-la-casa-de-eluxed}}

\hypertarget{section-09-3}{%
\section{3}\label{section-09-3}}

\bibleverse{1} Y el joven Samuel ministraba á Jehová delante de Eli: y
la palabra de Jehová era de estima en aquellos días; no había visión
manifiesta. \bibleverse{2} Y aconteció un día, que estando Eli acostado
en su aposento, cuando sus ojos comenzaban á oscurecerse, que no podía
ver, \bibleverse{3} Samuel estaba durmiendo en el templo de Jehová,
donde el arca de Dios estaba: y antes que la lámpara de Dios fuese
apagada, \bibleverse{4} Jehová llamó á Samuel; y él respondió: Heme
aquí.

\bibleverse{5} Y corriendo luego á Eli, dijo: Heme aquí; ¿para qué me
llamaste? Y Eli le dijo: Yo no he llamado; vuélvete á acostar. Y él se
volvió, y acostóse.

\bibleverse{6} Y Jehová volvió á llamar otra vez á Samuel. Y
levantándose Samuel vino á Eli, y dijo: Heme aquí; ¿para qué me has
llamado? Y él dijo: Hijo mío, yo no he llamado; vuelve, y acuéstate.

\bibleverse{7} Y Samuel no había conocido aún á Jehová, ni la palabra de
Jehová le había sido revelada. \bibleverse{8} Jehová pues llamó la
tercera vez á Samuel. Y él levantándose vino á Eli, y dijo: Heme aquí;
¿para qué me has llamado? Entonces entendió Eli que Jehová llamaba al
joven.

\bibleverse{9} Y dijo Eli á Samuel: Ve, y acuéstate: y si te llamare,
dirás: Habla, Jehová, que tu siervo oye. Así se fué Samuel, y acostóse
en su lugar. \bibleverse{10} Y vino Jehová, y paróse, y llamó como las
otras veces: ¡Samuel, Samuel! Entonces Samuel dijo: Habla, que tu siervo
oye.

\bibleverse{11} Y Jehová dijo á Samuel: He aquí haré yo una cosa en
Israel, que á quien la oyere, le retiñirán ambos oídos. \bibleverse{12}
Aquel día yo despertaré contra Eli todas las cosas que he dicho sobre su
casa. En comenzando, acabaré también. \bibleverse{13} Y mostraréle que
yo juzgaré su casa para siempre, por la iniquidad que él sabe; porque
sus hijos se han envilecido, y él no los ha estorbado. \bibleverse{14} Y
por tanto yo he jurado á la casa de Eli, que la iniquidad de la casa de
Eli no será expiada jamás, ni con sacrificios ni con presentes.

\hypertarget{samuel-comparte-la-revelaciuxf3n-con-eluxed-y-comienza-su-trabajo-como-profeta-para-todo-israel}{%
\subsection{Samuel comparte la revelación con Elí y comienza su trabajo
como profeta para todo
Israel}\label{samuel-comparte-la-revelaciuxf3n-con-eluxed-y-comienza-su-trabajo-como-profeta-para-todo-israel}}

\bibleverse{15} Y Samuel estuvo acostado hasta la mañana, y abrió las
puertas de la casa de Jehová. Y Samuel temía descubrir la visión á Eli.
\bibleverse{16} Llamando pues Eli á Samuel, díjole: Hijo mío, Samuel. Y
él respondió: Heme aquí.

\bibleverse{17} Y dijo: ¿Qué es la palabra que te habló Jehová?; ruégote
que no me la encubras: así te haga Dios y así te añada, si me
encubrieres palabra de todo lo que habló contigo.

\bibleverse{18} Y Samuel se lo manifestó todo, sin encubrirle nada.
Entonces él dijo: Jehová es; haga lo que bien le pareciere.

\bibleverse{19} Y Samuel creció, y Jehová fué con él, y no dejó caer á
tierra ninguna de sus palabras. \bibleverse{20} Y conoció todo Israel
desde Dan hasta Beer-sebah, que Samuel era fiel profeta de Jehová.
\bibleverse{21} Así tornó Jehová á aparecer en Silo: porque Jehová se
manifestó á Samuel en Silo con palabra de Jehová.

\hypertarget{die-bundeslade-ins-lager-der-israeliten-geholt}{%
\subsection{Die Bundeslade ins Lager der Israeliten
geholt}\label{die-bundeslade-ins-lager-der-israeliten-geholt}}

\hypertarget{section-09-4}{%
\section{4}\label{section-09-4}}

\bibleverse{1} Y Samuel habló á todo Israel. Por aquel tiempo salió
Israel á encontrar en batalla á los Filisteos, y asentó campo junto á
Eben-ezer, y los Filisteos asentaron el suyo en Aphec.

\bibleverse{2} Y los Filisteos presentaron la batalla á Israel; y
trabándose el combate, Israel fué vencido delante de los Filisteos, los
cuales hirieron en la batalla por el campo como cuatro mil hombres.
\bibleverse{3} Y vuelto que hubo el pueblo al campamento, los ancianos
de Israel dijeron: ¿Por qué nos ha herido hoy Jehová delante de los
Filisteos? Traigamos á nosotros de Silo el arca del pacto de Jehová,
para que viniendo entre nosotros nos salve de la mano de nuestros
enemigos.

\bibleverse{4} Y envió el pueblo á Silo, y trajeron de allá el arca del
pacto de Jehová de los ejércitos, que estaba asentado entre los
querubines; y los dos hijos de Eli, Ophni y Phinees, estaban allí con el
arca del pacto de Dios.

\hypertarget{el-efecto-de-este-evento-en-las-partes-en-conflicto-derrota-de-los-israelitas-y-puxe9rdida-del-arca}{%
\subsection{El efecto de este evento en las partes en conflicto; Derrota
de los israelitas y pérdida del
arca}\label{el-efecto-de-este-evento-en-las-partes-en-conflicto-derrota-de-los-israelitas-y-puxe9rdida-del-arca}}

\bibleverse{5} Y aconteció que, como el arca del pacto de Jehová vino al
campo, todo Israel dió grita con tan gran júbilo, que la tierra tembló.
\bibleverse{6} Y cuando los Filisteos oyeron la voz de júbilo, dijeron:
¿Qué voz de gran júbilo es esta en el campo de los Hebreos? Y supieron
que el arca de Jehová había venido al campo. \bibleverse{7} Y los
Filisteos tuvieron miedo, porque decían: Ha venido Dios al campo. Y
dijeron: ¡Ay de nosotros! pues antes de ahora no fué así. \bibleverse{8}
¡Ay de nosotros! ¿Quién nos librará de las manos de estos dioses
fuertes? Estos son los dioses que hirieron á Egipto con toda plaga en el
desierto. \bibleverse{9} Esforzaos, oh Filisteos, y sed hombres, porque
no sirváis á los Hebreos, como ellos os han servido á vosotros: sed
hombres, y pelead. \bibleverse{10} Pelearon pues los Filisteos, é Israel
fué vencido, y huyeron cada cual á sus tiendas; y fué hecha muy grande
mortandad, pues cayeron de Israel treinta mil hombres de á pie.
\bibleverse{11} Y el arca de Dios fué tomada, y muertos los dos hijos de
Eli, Ophni y Phinees.

\hypertarget{los-tristes-efectos-del-mensaje-en-shiloh-la-muerte-de-eli-y-su-nuera}{%
\subsection{Los tristes efectos del mensaje en Shiloh; la muerte de Eli
y su
nuera}\label{los-tristes-efectos-del-mensaje-en-shiloh-la-muerte-de-eli-y-su-nuera}}

\bibleverse{12} Y corriendo de la batalla un hombre de Benjamín, vino
aquel día á Silo, rotos sus vestidos y tierra sobre su cabeza:
\bibleverse{13} Y cuando llegó, he aquí Eli que estaba sentado en una
silla atalayando junto al camino; porque su corazón estaba temblando por
causa del arca de Dios. Llegado pues aquel hombre á la ciudad, y dadas
las nuevas, toda la ciudad gritó. \bibleverse{14} Y como Eli oyó el
estruendo de la gritería, dijo: ¿Qué estruendo de alboroto es éste? Y
aquel hombre vino apriesa, y dió las nuevas á Eli.

\bibleverse{15} Era ya Eli de edad de noventa y ocho años, y sus ojos se
habían entenebrecido, de modo que no podía ver. \bibleverse{16} Dijo
pues aquel hombre á Eli: Yo vengo de la batalla, yo he escapado hoy del
combate. Y él dijo: ¿Qué ha acontecido, hijo mío?

\bibleverse{17} Y el mensajero respondió, y dijo: Israel huyó delante de
los Filisteos, y también fué hecha gran mortandad en el pueblo; y
también tus dos hijos, Ophni y Phinees, son muertos, y el arca de Dios
fué tomada.

\bibleverse{18} Y aconteció que como él hizo mención del arca de Dios,
Eli cayó hacia atrás de la silla al lado de la puerta, y quebrósele la
cerviz, y murió: porque era hombre viejo y pesado. Y había juzgado á
Israel cuarenta años.

\bibleverse{19} Y su nuera, la mujer de Phinees, que estaba preñada,
cercana al parto, oyendo el rumor que el arca de Dios era tomada, y
muertos su suegro y su marido, encorvóse y parió; porque sus dolores se
habían ya derramado por ella. \bibleverse{20} Y al tiempo que se moría,
decíanle las que estaban junto á ella: No tengas temor, porque has
parido un hijo. Mas ella no respondió, ni paró mientes. \bibleverse{21}
Y llamó al niño Ichâbod, diciendo: ¡Traspasada es la gloria de Israel!
por el arca de Dios que fué tomada, y porque era muerto su suegro, y su
marido. \bibleverse{22} Dijo pues: Traspasada es la gloria de Israel:
porque el arca de Dios fué tomada.

\hypertarget{en-la-tierra-de-los-filisteos-el-arca-estuxe1-causando-estragos-en-varias-ciudades}{%
\subsection{En la tierra de los filisteos, el arca está causando
estragos en varias
ciudades}\label{en-la-tierra-de-los-filisteos-el-arca-estuxe1-causando-estragos-en-varias-ciudades}}

\hypertarget{section-09-5}{%
\section{5}\label{section-09-5}}

\bibleverse{1} Y los Filisteos, tomada el arca de Dios, trajéronla desde
Eben-ezer á Asdod. \bibleverse{2} Y tomaron los Filisteos el arca de
Dios, y metiéronla en la casa de Dagón, y pusiéronla junto á Dagón.
\bibleverse{3} Y el siguiente día los de Asdod se levantaron de mañana,
y he aquí Dagón postrado en tierra delante del arca de Jehová: y tomaron
á Dagón, y volviéronlo á su lugar. \bibleverse{4} Y tornándose á
levantar de mañana el siguiente día, he aquí que Dagón había caído
postrado en tierra delante del arca de Jehová; y la cabeza de Dagón y
las dos palmas de sus manos estaban cortadas sobre el umbral, habiéndole
quedado á Dagón el tronco solamente. \bibleverse{5} Por esta causa los
sacerdotes de Dagón, y todos los que en el templo de Dagón entran, no
pisan el umbral de Dagón en Asdod, hasta hoy. \bibleverse{6} Empero
agravóse la mano de Jehová sobre los de Asdod, y destruyólos, é hiriólos
con hemorroides en Asdod y en todos sus términos.

\bibleverse{7} Y viendo esto los de Asdod, dijeron: No quede con
nosotros el arca del Dios de Israel, porque su mano es dura sobre
nosotros, y sobre nuestro dios Dagón. \bibleverse{8} Enviaron pues á
juntar á sí todos los príncipes de los Filisteos, y dijeron: ¿Qué
haremos del arca del Dios de Israel? Y ellos respondieron: Pásese el
arca del Dios de Israel á Gath. Y pasaron allá el arca del Dios de
Israel.

\bibleverse{9} Y aconteció que como la hubieron pasado, la mano de
Jehová fué contra la ciudad con grande quebrantamiento; é hirió los
hombres de aquella ciudad desde el chico hasta el grande, que se
llenaron de hemorroides. \bibleverse{10} Entonces enviaron el arca de
Dios á Ecrón. Y como el arca de Dios vino á Ecrón, los Ecronitas dieron
voces diciendo: Han pasado á mí el arca del Dios de Israel por matarme á
mí y á mi pueblo.

\bibleverse{11} Y enviaron á juntar todos los príncipes de los
Filisteos, diciendo: Despachad el arca del Dios de Israel, y tórnese á
su lugar, y no mate á mí ni á mi pueblo: porque había quebrantamiento de
muerte en toda la ciudad, y la mano de Dios se había allí agravado.
\bibleverse{12} Y los que no morían, eran heridos de hemorroides; y el
clamor de la ciudad subía al cielo.

\hypertarget{resoluciuxf3n-de-los-filisteos-sobre-el-regreso-del-arca}{%
\subsection{Resolución de los filisteos sobre el regreso del
arca}\label{resoluciuxf3n-de-los-filisteos-sobre-el-regreso-del-arca}}

\hypertarget{section-09-6}{%
\section{6}\label{section-09-6}}

\bibleverse{1} Y estuvo el arca de Jehová en la tierra de los Filisteos
siete meses. \bibleverse{2} Entonces los Filisteos, llamando los
sacerdotes y adivinos, preguntaron: ¿Qué haremos del arca de Jehová?
Declaradnos cómo la hemos de tornar á enviar á su lugar.

\bibleverse{3} Y ellos dijeron: Si enviáis el arca del Dios de Israel,
no la enviéis vacía; mas le pagaréis la expiación: y entonces seréis
sanos, y conoceréis por qué no se apartó de vosotros su mano.

\bibleverse{4} Y ellos dijeron: ¿Y qué será la expiación que le
pagaremos? Y ellos respondieron: Conforme al número de los príncipes de
los Filisteos, cinco hemorroides de oro, y cinco ratones de oro, porque
la misma plaga que todos tienen, tienen también vuestros príncipes.

\bibleverse{5} Haréis pues las formas de vuestras hemorroides, y las
formas de vuestros ratones que destruyen la tierra, y daréis gloria al
Dios de Israel: quizá aliviará su mano de sobre vosotros, y de sobre
vuestros dioses, y de sobre vuestra tierra. \bibleverse{6} Mas ¿por qué
endurecéis vuestro corazón, como los Egipcios y Faraón endurecieron su
corazón? Después que los hubo así tratado, ¿no los dejaron que se
fuesen, y se fueron?

\bibleverse{7} Haced pues ahora un carro nuevo, y tomad luego dos vacas
que críen, á las cuales no haya sido puesto yugo, y uncid las vacas al
carro, y haced tornar de detrás de ellas sus becerros á casa.
\bibleverse{8} Tomaréis luego el arca de Jehová, y la pondréis sobre el
carro; y poned en una caja al lado de ella las alhajas de oro que le
pagáis en expiación: y la dejaréis que se vaya. \bibleverse{9} Y mirad:
si sube por el camino de su término á Beth-semes, él nos ha hecho este
mal tan grande; y si no, seremos ciertos que su mano no nos hirió, nos
ha sido accidente.

\hypertarget{ejecuciuxf3n-de-la-resoluciuxf3n-llegada-y-recepciuxf3n-del-arca-en-bet-semes}{%
\subsection{Ejecución de la resolución; Llegada y recepción del arca en
Bet-semes}\label{ejecuciuxf3n-de-la-resoluciuxf3n-llegada-y-recepciuxf3n-del-arca-en-bet-semes}}

\bibleverse{10} Y aquellos hombres lo hicieron así; pues tomando dos
vacas que criaban, unciéronlas al carro, y encerraron en casa sus
becerros. \bibleverse{11} Luego pusieron el arca de Jehová sobre el
carro, y la caja con los ratones de oro y con las formas de sus
hemorroides. \bibleverse{12} Y las vacas se encaminaron por el camino de
Beth-semes, é iban por un mismo camino andando y bramando, sin apartarse
ni á diestra ni á siniestra: y los príncipes de los Filisteos fueron
tras ellas hasta el término de Beth-semes. \bibleverse{13} Y los de
Beth-semes segaban el trigo en el valle; y alzando sus ojos vieron el
arca, y holgáronse cuando la vieron. \bibleverse{14} Y el carro vino al
campo de Josué Beth-semita, y paró allí: porque allí había una gran
piedra: y ellos cortaron la madera del carro, y ofrecieron las vacas en
holocausto á Jehová. \bibleverse{15} Y los Levitas bajaron el arca de
Jehová, y la caja que estaba junto á ella, en la cual estaban las
alhajas de oro, y pusiéronlas sobre aquella gran piedra: y los hombres
de Beth-semes sacrificaron holocaustos y mataron víctimas á Jehová en
aquel día. \bibleverse{16} Lo cual viendo los cinco príncipes de los
Filisteos, volviéronse á Ecrón el mismo día. \bibleverse{17} Estas pues
son las hemorroides de oro que pagaron los Filisteos á Jehová en
expiación: por Asdod una, por Gaza una, por Ascalón una, por Gath una,
por Ecrón una; \bibleverse{18} Y ratones de oro conforme al número de
todas las ciudades de los Filisteos pertenecientes á los cinco
príncipes, desde las ciudades fuertes hasta las aldeas sin muro; y hasta
la gran piedra sobre la cual pusieron el arca de Jehová, piedra que está
en el campo de Josué Beth-semita hasta hoy.

\hypertarget{se-instala-el-arca-en-quiriat-jearim}{%
\subsection{Se instala el arca en
Quiriat-Jearim}\label{se-instala-el-arca-en-quiriat-jearim}}

\bibleverse{19} Entonces hirió Dios á los de Beth-semes, porque habían
mirado en el arca de Jehová; hirió en el pueblo cincuenta mil y setenta
hombres. Y el pueblo puso luto, porque Jehová le había herido de tan
gran plaga. \bibleverse{20} Y dijeron los de Beth-semes: ¿Quién podrá
estar delante de Jehová el Dios santo? ¿y á quién subirá desde nosotros?

\bibleverse{21} Y enviaron mensajeros á los de Chîriath-jearim,
diciendo: Los Filisteos han vuelto el arca de Jehová: descended pues, y
llevadla á vosotros.

\hypertarget{section-09-7}{%
\section{7}\label{section-09-7}}

\bibleverse{1} Y vinieron los de Chîriath-jearim, y llevaron el arca de
Jehová, y metiéronla en casa de Abinadab, situada en el collado; y
santificaron á Eleazar su hijo, para que guardase el arca de Jehová.

\hypertarget{los-israelitas-se-vuelven-arrepentidos-a-dios}{%
\subsection{Los israelitas se vuelven arrepentidos a
Dios}\label{los-israelitas-se-vuelven-arrepentidos-a-dios}}

\bibleverse{2} Y aconteció que desde el día que llegó el arca á
Chîriath-jearim pasaron muchos días, veinte años; y toda la casa de
Israel lamentaba en pos de Jehová. \bibleverse{3} Y habló Samuel á toda
la casa de Israel, diciendo: Si de todo vuestro corazón os volvéis á
Jehová, quitad los dioses ajenos y á Astaroth de entre vosotros, y
preparad vuestro corazón á Jehová, y á sólo él servid, y os librará de
mano de los Filisteos. \bibleverse{4} Entonces los hijos de Israel
quitaron á los Baales y á Astaroth, y sirvieron á solo Jehová.

\hypertarget{la-intercesiuxf3n-y-el-sacrificio-de-samuel-por-israel-en-mizpa-derrota-de-los-filisteos-la-piedra-eben-eser}{%
\subsection{La intercesión y el sacrificio de Samuel por Israel en
Mizpa; Derrota de los filisteos; la piedra
Eben-Eser}\label{la-intercesiuxf3n-y-el-sacrificio-de-samuel-por-israel-en-mizpa-derrota-de-los-filisteos-la-piedra-eben-eser}}

\bibleverse{5} Y Samuel dijo: Juntad á todo Israel en Mizpa, y yo oraré
por vosotros á Jehová. \bibleverse{6} Y juntándose en Mizpa, sacaron
agua, y derramáronla delante de Jehová, y ayunaron aquel día, y dijeron
allí: Contra Jehová hemos pecado. Y juzgó Samuel á los hijos de Israel
en Mizpa.

\bibleverse{7} Y oyendo los Filisteos que los hijos de Israel estaban
reunidos en Mizpa, subieron los príncipes de los Filisteos contra
Israel: lo cual como hubieron oído los hijos de Israel, tuvieron temor
de los Filisteos. \bibleverse{8} Y dijeron los hijos de Israel á Samuel:
No ceses de clamar por nosotros á Jehová nuestro Dios, que nos guarde de
mano de los filisteos. \bibleverse{9} Y Samuel tomó un cordero de leche,
y sacrificólo entero á Jehová en holocausto: y clamó Samuel á Jehová por
Israel, y Jehová le oyó. \bibleverse{10} Y aconteció que estando Samuel
sacrificando el holocausto, los Filisteos llegaron para pelear con los
hijos de Israel. Mas Jehová tronó aquel día con grande estruendo sobre
los Filisteos, y desbaratólos, y fueron vencidos delante de Israel.
\bibleverse{11} Y saliendo los hijos de Israel de Mizpa, siguieron á los
Filisteos, hiriéndolos hasta abajo de Beth-car.

\bibleverse{12} Tomó luego Samuel una piedra, y púsola entre Mizpa y
Sen, y púsole por nombre Eben-ezer, diciendo: Hasta aquí nos ayudó
Jehová.

\hypertarget{estado-de-paz-en-el-pauxeds-la-eficacia-de-samuel-como-juez}{%
\subsection{Estado de paz en el país; La eficacia de Samuel como
juez}\label{estado-de-paz-en-el-pauxeds-la-eficacia-de-samuel-como-juez}}

\bibleverse{13} Fueron pues los Filisteos humillados, que no vinieron
más al término de Israel; y la mano de Jehová fué contra los Filisteos
todo el tiempo de Samuel.

\bibleverse{14} Y fueron restituídas á los hijos de Israel las ciudades
que los Filisteos habían tomado á los Israelitas, desde Ecrón hasta
Gath, con sus términos: é Israel las libró de mano de los Filisteos. Y
hubo paz entre Israel y el Amorrheo.

\bibleverse{15} Y juzgó Samuel á Israel todo el tiempo que vivió.
\bibleverse{16} Y todos los años iba y daba vuelta á Beth-el, y á
Gilgal, y á Mizpa, y juzgaba á Israel en todos estos lugares.
\bibleverse{17} Volvíase después á Rama, porque allí estaba su casa, y
allí juzgaba á Israel; y edificó allí altar á Jehová.

\hypertarget{el-deseo-de-israel-por-un-rey-la-demanda-del-pueblo-despierta-el-disgusto-de-samuel-pero-encuentra-la-aprobaciuxf3n-de-dios}{%
\subsection{El deseo de Israel por un rey; La demanda del pueblo
despierta el disgusto de Samuel, pero encuentra la aprobación de
Dios}\label{el-deseo-de-israel-por-un-rey-la-demanda-del-pueblo-despierta-el-disgusto-de-samuel-pero-encuentra-la-aprobaciuxf3n-de-dios}}

\hypertarget{section-09-8}{%
\section{8}\label{section-09-8}}

\bibleverse{1} Y aconteció que habiendo Samuel envejecido, puso sus
hijos por jueces sobre Israel. \bibleverse{2} Y el nombre de su hijo
primogénito fué Joel, y el nombre del segundo, Abia: fueron jueces en
Beer-sebah. \bibleverse{3} Mas no anduvieron los hijos por los caminos
de su padre, antes se ladearon tras la avaricia, recibiendo cohecho y
pervirtiendo el derecho.

\bibleverse{4} Entonces todos los ancianos de Israel se juntaron, y
vinieron á Samuel en Rama, \bibleverse{5} Y dijéronle: He aquí tú has
envejecido, y tus hijos no van por tus caminos: por tanto, constitúyenos
ahora un rey que nos juzgue, como todas las gentes. \bibleverse{6} Y
descontentó á Samuel esta palabra que dijeron: Danos rey que nos juzgue.
Y Samuel oró á Jehová.

\bibleverse{7} Y dijo Jehová á Samuel: Oye la voz del pueblo en todo lo
que te dijeren: porque no te han desechado á ti, sino á mí me han
desechado, para que no reine sobre ellos. \bibleverse{8} Conforme á
todas las obras que han hecho desde el día que los saqué de Egipto hasta
hoy, que me han dejado y han servido á dioses ajenos, así hacen también
contigo. \bibleverse{9} Ahora pues, oye su voz: mas protesta contra
ellos declarándoles el derecho del rey que ha de reinar sobre ellos.

\hypertarget{samuel-le-dice-a-la-gente-los-derechos-de-un-rey}{%
\subsection{Samuel le dice a la gente los derechos de un
rey}\label{samuel-le-dice-a-la-gente-los-derechos-de-un-rey}}

\bibleverse{10} Y dijo Samuel todas las palabras de Jehová al pueblo que
le había pedido rey. \bibleverse{11} Dijo pues: Este será el derecho del
rey que hubiere de reinar sobre vosotros: tomará vuestros hijos, y
pondrálos en sus carros, y en su gente de á caballo, para que corran
delante de su carro: \bibleverse{12} Y se elegirá capitanes de mil, y
capitanes de cincuenta: pondrálos asimismo á que aren sus campos, y
sieguen sus mieses, y á que hagan sus armas de guerra, y los pertrechos
de sus carros: \bibleverse{13} Tomará también vuestras hijas para que
sean perfumadoras, cocineras, y amasadoras. \bibleverse{14} Asimismo
tomará vuestras tierras, vuestras viñas, y vuestros buenos olivares, y
los dará á sus siervos. \bibleverse{15} El diezmará vuestras simientes y
vuestras viñas, para dar á sus eunucos y á sus siervos. \bibleverse{16}
El tomará vuestros siervos, y vuestras siervas, y vuestros buenos
mancebos, y vuestros asnos, y con ellos hará sus obras. \bibleverse{17}
Diezmará también vuestro rebaño, y seréis sus siervos. \bibleverse{18} Y
clamaréis aquel día á causa de vuestro rey que os habréis elegido, mas
Jehová no os oirá en aquel día.

\hypertarget{la-gente-persiste-en-su-demanda-la-aprobaciuxf3n-de-dios}{%
\subsection{La gente persiste en su demanda; La aprobación de
dios}\label{la-gente-persiste-en-su-demanda-la-aprobaciuxf3n-de-dios}}

\bibleverse{19} Empero el pueblo no quiso oir la voz de Samuel; antes
dijeron: No, sino que habrá rey sobre nosotros: \bibleverse{20} Y
nosotros seremos también como todas las gentes, y nuestro rey nos
gobernará, y saldrá delante de nosotros, y hará nuestras guerras.

\bibleverse{21} Y oyó Samuel todas las palabras del pueblo, y refiriólas
en oídos de Jehová. \bibleverse{22} Y Jehová dijo á Samuel: Oye su voz,
y pon rey sobre ellos. Entonces dijo Samuel á los varones de Israel:
Idos cada uno á su ciudad.

\hypertarget{sauxfal-llega-a-la-casa-de-samuel-en-busca-de-los-asnos-de-su-padre}{%
\subsection{Saúl llega a la casa de Samuel en busca de los asnos de su
padre}\label{sauxfal-llega-a-la-casa-de-samuel-en-busca-de-los-asnos-de-su-padre}}

\hypertarget{section-09-9}{%
\section{9}\label{section-09-9}}

\bibleverse{1} Y había un varón de Benjamín, hombre valeroso, el cual se
llamaba Cis, hijo de Abiel, hijo de Seor, hijo de Bechôra, hijo de
Aphia, hijo de un hombre de Benjamín. \bibleverse{2} Y tenía él un hijo
que se llamaba Saúl, mancebo y hermoso, que entre los hijos de Israel no
había otro más hermoso que él; del hombro arriba sobrepujaba á
cualquiera del pueblo.

\bibleverse{3} Y habíanse perdido las asnas de Cis, padre de Saúl; por
lo que dijo Cis á Saúl su hijo: Toma ahora contigo alguno de los
criados, y levántate, y ve á buscar las asnas. \bibleverse{4} Y él pasó
al monte de Ephraim, y de allí á la tierra de Salisa, y no las hallaron.
Pasaron luego por la tierra de Saalim, y tampoco. Después pasaron por la
tierra de Benjamín, y no las encontraron.

\bibleverse{5} Y cuando vinieron á la tierra de Suph, Saúl dijo á su
criado que tenía consigo: Ven, volvámonos; porque quizá mi padre, dejado
el cuidado de las asnas, estará congojado por nosotros.

\bibleverse{6} Y él le respondió: He aquí ahora hay en esta ciudad un
hombre de Dios, que es varón insigne: todas las cosas que él dijere, sin
duda vendrán. Vamos pues allá: quizá nos enseñará nuestro camino por
donde hayamos de ir.

\bibleverse{7} Y Saúl respondió á su criado: Vamos ahora: ¿mas qué
llevaremos al varón? Porque el pan de nuestras alforjas se ha acabado, y
no tenemos qué presentar al varón de Dios: ¿qué tenemos?

\bibleverse{8} Entonces tornó el criado á responder á Saúl, diciendo: He
aquí se halla en mi mano la cuarta parte de un siclo de plata: esto daré
al varón de Dios, porque nos declare nuestro camino. \bibleverse{9}
(Antiguamente en Israel cualquiera que iba á consultar á Dios, decía
así: Venid y vamos hasta el vidente: porque el que ahora se llama
profeta, antiguamente era llamado vidente).

\hypertarget{la-cuxe1lida-bienvenida-de-sauxfal-y-el-trato-honorable-de-parte-de-samuel}{%
\subsection{La cálida bienvenida de Saúl y el trato honorable de parte
de
Samuel}\label{la-cuxe1lida-bienvenida-de-sauxfal-y-el-trato-honorable-de-parte-de-samuel}}

\bibleverse{10} Dijo entonces Saúl á su criado: Bien dices; ea pues,
vamos. Y fueron á la ciudad donde estaba el varón de Dios.
\bibleverse{11} Y cuando subían por la cuesta de la ciudad, hallaron
unas mozas que salían por agua, á las cuales dijeron: ¿Está en este
lugar el vidente?

\bibleverse{12} Y ellas respondiéndoles, dijeron: Sí; helo aquí delante
de ti: date pues priesa, porque hoy ha venido á la ciudad en atención á
que el pueblo tiene hoy sacrificio en el alto. \bibleverse{13} Y cuando
entrareis en la ciudad, le encontraréis luego, antes que suba al alto á
comer; pues el pueblo no comerá hasta que él haya venido, por cuanto él
haya de bendecir el sacrificio, y después comerán los convidados. Subid
pues ahora, porque ahora le hallaréis.

\bibleverse{14} Ellos entonces subieron á la ciudad; y cuando en medio
de la ciudad estuvieron, he aquí Samuel que delante de ellos salía para
subir al alto.

\bibleverse{15} Y un día antes que Saúl viniese, Jehová había revelado
al oído de Samuel, diciendo: \bibleverse{16} Mañana á esta misma hora yo
enviaré á ti un varón de la tierra de Benjamín, al cual ungirás por
príncipe sobre mi pueblo Israel, y salvará mi pueblo de mano de los
Filisteos: pues yo he mirado á mi pueblo, porque su clamor ha llegado
hasta mí.

\bibleverse{17} Y luego que Samuel vió á Saúl, Jehová le dijo: He aquí
éste es el varón del cual te hablé; éste señoreará á mi pueblo.

\bibleverse{18} Y llegando Saúl á Samuel en medio de la puerta, díjole:
Ruégote que me enseñes dónde está la casa del vidente.

\bibleverse{19} Y Samuel respondió á Saúl, y dijo: Yo soy el vidente:
sube delante de mí al alto, y comed hoy conmigo, y por la mañana te
despacharé, y te descubriré todo lo que está en tu corazón.
\bibleverse{20} Y de las asnas que se te perdieron hoy ha tres días,
pierde cuidado de ellas, porque se han hallado. Mas ¿por quién es todo
el deseo de Israel, sino por ti y por toda la casa de tu padre?

\bibleverse{21} Y Saúl respondió, y dijo: ¿No soy yo hijo de Benjamín,
de las más pequeñas tribus de Israel? Y mi familia ¿no es la más pequeña
de todas las familias de la tribu de Benjamín? ¿por qué pues me has
dicho cosa semejante?

\bibleverse{22} Y trabando Samuel de Saúl y de su criado, metiólos en la
sala, y dióles lugar á la cabecera de los convidados, que eran como unos
treinta hombres. \bibleverse{23} Y dijo Samuel al cocinero: Trae acá la
porción que te dí, la cual te dije que guardases aparte. \bibleverse{24}
Entonces alzó el cocinero una espaldilla, con lo que estaba sobre ella,
y púsola delante de Saúl. Y Samuel dijo: He aquí lo que estaba
reservado: ponlo delante de ti, y come; porque de industria se guardó
para ti, cuando dije: Yo he convidado al pueblo. Y Saúl comió aquel día
con Samuel.

\bibleverse{25} Y cuando hubieron descendido del alto á la ciudad, él
habló con Saúl en el terrado.

\hypertarget{sauxfal-ungido-rey-por-samuel-su-regreso-a-guibeuxe1}{%
\subsection{Saúl ungido rey por Samuel; su regreso a
Guibeá}\label{sauxfal-ungido-rey-por-samuel-su-regreso-a-guibeuxe1}}

\bibleverse{26} Y al otro día madrugaron: y como al apuntar del alba,
Samuel llamó á Saúl, que estaba en el terrado; y dijo: Levántate, para
que te despache. Levantóse luego Saúl, y salieron fuera ambos, él y
Samuel. \bibleverse{27} Y descendiendo ellos al cabo de la ciudad, dijo
Samuel á Saúl: Di al mozo que vaya delante, (y adelantóse el mozo); mas
espera tú un poco para que te declare palabra de Dios.

\hypertarget{section-09-10}{%
\section{10}\label{section-09-10}}

\bibleverse{1} Tomando entonces Samuel una ampolla de aceite, derramóla
sobre su cabeza, y besólo, y díjole: ¿No te ha ungido Jehová por capitán
sobre su heredad?

\hypertarget{samuel-profetiza-tres-seuxf1ales-que-sauxfal-recibiruxe1-de-camino-a-casa-y-lo-envuxeda-a-gilgal}{%
\subsection{Samuel profetiza tres señales que Saúl recibirá de camino a
casa y lo envía a
Gilgal}\label{samuel-profetiza-tres-seuxf1ales-que-sauxfal-recibiruxe1-de-camino-a-casa-y-lo-envuxeda-a-gilgal}}

\bibleverse{2} Hoy, después que te hayas apartado de mí, hallarás dos
hombres junto al sepulcro de Rachêl, en el término de Benjamín, en
Selsah, los cuales te dirán: Las asnas que habías ido á buscar, se han
hallado; tu padre pues ha dejado ya el negocio de las asnas, si bien
está angustioso por vosotros, diciendo: ¿Qué haré acerca de mi hijo?

\bibleverse{3} Y como de allí te fueres más adelante, y llegares á la
campiña de Tabor, te saldrán al encuentro tres hombres que suben á Dios
en Beth-el, llevando el uno tres cabritos, y el otro tres tortas de pan,
y el tercero una vasija de vino: \bibleverse{4} Los cuales, luego que te
hayan saludado, te darán dos panes, los que tomarás de manos de ellos.

\bibleverse{5} De allí vendrás al collado de Dios donde está la
guarnición de los Filisteos; y cuando entrares allá en la ciudad
encontrarás una compañía de profetas que descienden del alto, y delante
de ellos salterio, y adufe, y flauta, y arpa, y ellos profetizando:
\bibleverse{6} Y el espíritu de Jehová te arrebatará, y profetizarás con
ellos, y serás mudado en otro hombre. \bibleverse{7} Y cuando te
hubieren sobrevenido estas señales, haz lo que te viniere á la mano,
porque Dios es contigo.

\bibleverse{8} Y bajarás delante de mí á Gilgal; y luego descenderé yo á
ti para sacrificar holocaustos, é inmolar víctimas pacíficas. Espera
siete días, hasta que yo venga á ti, y te enseñe lo que has de hacer.

\hypertarget{la-llegada-de-los-carteles-anunciados-saulo-entre-los-profetas}{%
\subsection{La llegada de los carteles anunciados; Saulo entre los
profetas}\label{la-llegada-de-los-carteles-anunciados-saulo-entre-los-profetas}}

\bibleverse{9} Y fué que así como tornó él su hombro para partirse de
Samuel, mudóle Dios su corazón; y todas estas señales acaecieron en
aquel día. \bibleverse{10} Y cuando llegaron allá al collado, he aquí la
compañía de los profetas que venía á encontrarse con él, y el espíritu
de Dios lo arrebató, y profetizó entre ellos. \bibleverse{11} Y
aconteció que, cuando todos los que le conocían de ayer y de antes,
vieron como profetizaba con los profetas, el pueblo decía el uno al
otro: ¿Qué ha sucedido al hijo de Cis? ¿Saúl también entre los profetas?

\bibleverse{12} Y alguno de allí respondió, y dijo: ¿Y quién es el padre
de ellos? Por esta causa se tornó en proverbio: ¿También Saúl entre los
profetas?

\hypertarget{saul-de-regreso-a-casa-su-conversaciuxf3n-reservada-con-su-prima}{%
\subsection{Saul de regreso a casa; su conversación reservada con su
prima}\label{saul-de-regreso-a-casa-su-conversaciuxf3n-reservada-con-su-prima}}

\bibleverse{13} Y cesó de profetizar, y llegó al alto. \bibleverse{14} Y
un tío de Saúl dijo á él y á su criado: ¿Dónde fuisteis? Y él respondió:
A buscar las asnas; y como vimos que no parecían, fuimos á Samuel.

\bibleverse{15} Y dijo el tío de Saúl: Yo te ruego me declares qué os
dijo Samuel.

\bibleverse{16} Y Saúl respondió á su tío: Declarónos expresamente que
las asnas habían parecido. Mas del negocio del reino, de que Samuel le
había hablado, no le descubrió nada.

\hypertarget{sauxfal-estuxe1-decidido-a-ser-rey-en-mizpa-por-la-santa-suerte}{%
\subsection{Saúl está decidido a ser rey en Mizpa por la santa
suerte}\label{sauxfal-estuxe1-decidido-a-ser-rey-en-mizpa-por-la-santa-suerte}}

\bibleverse{17} Y Samuel convocó el pueblo á Jehová en Mizpa;
\bibleverse{18} Y dijo á los hijos de Israel: Así ha dicho Jehová el
Dios de Israel: Yo saqué á Israel de Egipto, y os libré de mano de los
Egipcios, y de mano de todos los reinos que os afligieron:
\bibleverse{19} Mas vosotros habéis desechado hoy á vuestro Dios, que os
guarda de todas vuestras aflicciones y angustias, y dijisteis: No, sino
pon rey sobre nosotros. Ahora pues, poneos delante de Jehová por
vuestras tribus y por vuestros millares.

\bibleverse{20} Y haciendo allegar Samuel todas las tribus de Israel,
fué tomada la tribu de Benjamín. \bibleverse{21} E hizo llegar la tribu
de Benjamín por sus linajes, y fué tomada la familia de Matri; y de ella
fué tomado Saúl hijo de Cis. Y le buscaron, mas no fué hallado.
\bibleverse{22} Preguntaron pues otra vez á Jehová, si había aún de
venir allí aquel varón. Y respondió Jehová: He aquí que él está
escondido entre el bagaje.

\bibleverse{23} Entonces corrieron, y tomáronlo de allí, y puesto en
medio del pueblo, desde el hombro arriba era más alto que todo el
pueblo. \bibleverse{24} Y Samuel dijo á todo el pueblo: ¿Habéis visto al
que ha elegido Jehová, que no hay semejante á él en todo el pueblo?
Entonces el pueblo clamó con alegría, diciendo: Viva el rey.

\bibleverse{25} Samuel recitó luego al pueblo el derecho del reino, y
escribiólo en un libro, el cual guardó delante de Jehová.
\bibleverse{26} Y envió Samuel á todo el pueblo cada uno á su casa. Y
Saúl también se fué á su casa en Gabaa, y fueron con él el ejército, el
corazón de los cuales Dios había tocado. \bibleverse{27} Pero los impíos
dijeron: ¿Cómo nos ha de salvar éste? Y tuviéronle en poco, y no le
trajeron presente: mas él disimuló.

\hypertarget{la-ciudad-de-jabuxe9s-que-estuxe1-en-apuros-por-los-amonitas-nahas-pide-la-ayuda-de-los-israelitas}{%
\subsection{La ciudad de Jabés, que está en apuros por los amonitas
Nahas, pide la ayuda de los
israelitas}\label{la-ciudad-de-jabuxe9s-que-estuxe1-en-apuros-por-los-amonitas-nahas-pide-la-ayuda-de-los-israelitas}}

\hypertarget{section-09-11}{%
\section{11}\label{section-09-11}}

\bibleverse{1} Y subió Naas Ammonita, y asentó campo contra Jabes de
Galaad. Y todos los de Jabes dijeron á Naas: Haz alianza con nosotros, y
te serviremos.

\bibleverse{2} Y Naas Ammonita les respondió: Con esta condición haré
alianza con vosotros, que á cada uno de todos vosotros saque el ojo
derecho, y ponga esta afrenta sobre todo Israel.

\bibleverse{3} Entonces los ancianos de Jabes le dijeron: Danos siete
días, para que enviemos mensajeros á todos los términos de Israel; y si
nadie hubiere que nos defienda, saldremos á ti.

\hypertarget{la-conducta-decidida-de-sauxfal-y-su-espluxe9ndida-victoria}{%
\subsection{La conducta decidida de Saúl y su espléndida
victoria}\label{la-conducta-decidida-de-sauxfal-y-su-espluxe9ndida-victoria}}

\bibleverse{4} Y llegando los mensajeros á Gabaa de Saúl, dijeron estas
palabras en oídos del pueblo; y todo el pueblo lloró á voz en grito.

\bibleverse{5} Y he aquí Saúl que venía del campo, tras los bueyes; y
dijo Saúl: ¿Qué tiene el pueblo, que lloran? Y contáronle las palabras
de los hombres de Jabes. \bibleverse{6} Y el espíritu de Dios arrebató á
Saúl en oyendo estas palabras, y encendióse en ira en gran manera.
\bibleverse{7} Y tomando un par de bueyes, cortólos en piezas, y
enviólas por todos los términos de Israel por mano de mensajeros,
diciendo: Cualquiera que no saliere en pos de Saúl y en pos de Samuel,
así será hecho á sus bueyes. Y cayó temor de Jehová sobre el pueblo, y
salieron como un solo hombre. \bibleverse{8} Y contólos en Bezec; y
fueron los hijos de Israel trescientos mil, y treinta mil los hombres de
Judá. \bibleverse{9} Y respondieron á los mensajeros que habían venido:
Así diréis á los de Jabes de Galaad: Mañana en calentando el sol,
tendréis salvamento. Y vinieron los mensajeros, y declaráronlo á los de
Jabes, los cuales se holgaron. \bibleverse{10} Y los de Jabes dijeron:
Mañana saldremos á vosotros, para que hagáis con nosotros todo lo que
bien os pareciere. \bibleverse{11} Y el día siguiente dispuso Saúl el
pueblo en tres escuadrones, y entraron en medio del real á la vela de la
mañana, é hirieron á los Ammonitas hasta que el día calentaba: y los que
quedaron fueron dispersos, tal que no quedaron dos de ellos juntos.

\hypertarget{la-generosidad-de-sauxfal-hacia-sus-despreciadores-celebraciuxf3n-de-la-alegruxeda-en-gilgal}{%
\subsection{La generosidad de Saúl hacia sus despreciadores; Celebración
de la alegría en
Gilgal}\label{la-generosidad-de-sauxfal-hacia-sus-despreciadores-celebraciuxf3n-de-la-alegruxeda-en-gilgal}}

\bibleverse{12} El pueblo entonces dijo á Samuel: ¿Quiénes son lo que
decían: Reinará Saúl sobre nosotros? Dadnos esos hombres, y los
mataremos.

\bibleverse{13} Y Saúl dijo: No morirá hoy ninguno, porque hoy ha obrado
Jehová salud en Israel.

\bibleverse{14} Mas Samuel dijo al pueblo: Venid, vamos á Gilgal para
que renovemos allí el reino. \bibleverse{15} Y fué todo el pueblo á
Gilgal, é invistieron allí á Saúl por rey delante de Jehová en Gilgal. Y
sacrificaron allí víctimas pacíficas delante de Jehová; y alegráronse
mucho allí Saúl y todos los de Israel.

\hypertarget{la-renuncia-voluntaria-de-samuel-y-la-solemne-despedida-del-pueblo}{%
\subsection{La renuncia voluntaria de Samuel y la solemne despedida del
pueblo}\label{la-renuncia-voluntaria-de-samuel-y-la-solemne-despedida-del-pueblo}}

\hypertarget{section-09-12}{%
\section{12}\label{section-09-12}}

\bibleverse{1} Y dijo Samuel á todo Israel: He aquí, yo he oído vuestra
voz en todas las cosas que me habéis dicho, y os he puesto rey.
\bibleverse{2} Ahora pues, he aquí vuestro rey va delante de vosotros.
Yo soy ya viejo y cano: mas mis hijos están con vosotros, y yo he andado
delante de vosotros desde mi mocedad hasta este día. \bibleverse{3} Aquí
estoy; atestiguad contra mí delante de Jehová y delante de su ungido, si
he tomado el buey de alguno, ó si he tomado el asno de alguno, ó si he
calumniado á alguien, ó si he agraviado á alguno, ó si de alguien he
tomado cohecho por el cual haya cubierto mis ojos: y os satisfaré.

\bibleverse{4} Entonces dijeron: Nunca nos has calumniado, ni agraviado,
ni has tomado algo de mano de ningún hombre.

\bibleverse{5} Y él les dijo: Jehová es testigo contra vosotros, y su
ungido también es testigo en este día, que no habéis hallado en mi mano
cosa ninguna. Y ellos respondieron: Así es.

\hypertarget{samuel-le-recuerda-al-pueblo-los-muchos-beneficios-de-dios}{%
\subsection{Samuel le recuerda al pueblo los muchos beneficios de
Dios}\label{samuel-le-recuerda-al-pueblo-los-muchos-beneficios-de-dios}}

\bibleverse{6} Entonces Samuel dijo al pueblo: Jehová es quien hizo á
Moisés y á Aarón, y que sacó á vuestros padres de la tierra de Egipto.
\bibleverse{7} Ahora pues, aguardad, y yo os haré cargo delante de
Jehová de todas las justicias de Jehová, que ha hecho con vosotros y con
vuestros padres.

\bibleverse{8} Después que Jacob hubo entrado en Egipto y vuestros
padres clamaron á Jehová, Jehová envió á Moisés y á Aarón, los cuales
sacaron á vuestros padres de Egipto, y los hicieron habitar en este
lugar. \bibleverse{9} Y olvidaron á Jehová su Dios, y él los vendió en
la mano de Sísara capitán del ejército de Asor, y en la mano de los
Filisteos, y en la mano del rey de Moab, los cuales les hicieron guerra.
\bibleverse{10} Y ellos clamaron á Jehová, y dijeron: Pecamos, que hemos
dejado á Jehová, y hemos servido á los Baales y á Astaroth: líbranos
pues ahora de la mano de nuestros enemigos, y te serviremos.
\bibleverse{11} Entonces Jehová envió á Jerobaal, y á Bedán, y á Jephté,
y á Samuel, y os libró de mano de vuestros enemigos alrededor, y
habitasteis seguros.

\hypertarget{samuel-demuestra-al-pueblo-a-travuxe9s-de-una-maravillosa-seuxf1al-divina-que-han-pecado-al-elegir-un-rey}{%
\subsection{Samuel demuestra al pueblo a través de una maravillosa señal
divina que han pecado al elegir un
rey}\label{samuel-demuestra-al-pueblo-a-travuxe9s-de-una-maravillosa-seuxf1al-divina-que-han-pecado-al-elegir-un-rey}}

\bibleverse{12} Y habiendo visto que Naas rey de lo hijos de Ammón venía
contra vosotros, me dijisteis: No, sino rey reinará sobre nosotros;
siendo vuestro rey Jehová vuestro Dios. \bibleverse{13} Ahora pues, ved
aquí vuestro rey que habéis elegido, el cual pedisteis; ya veis que
Jehová ha puesto sobre vosotros rey. \bibleverse{14} Si temiereis á
Jehová y le sirviereis, y oyereis su voz, y no fuereis rebeldes á la
palabra de Jehová, así vosotros como el rey que reina sobre vosotros,
seréis en pos de Jehová vuestro Dios. \bibleverse{15} Mas si no oyereis
la voz de Jehová, y si fuereis rebeldes á las palabras de Jehová, la
mano de Jehová será contra vosotros como contra vuestros padres.

\bibleverse{16} Esperad aún ahora, y mirad esta gran cosa que Jehová
hará delante de vuestros ojos. \bibleverse{17} ¿No es ahora la siega de
los trigos? Yo clamaré á Jehová, y él dará truenos y aguas; para que
conozcáis y veáis que es grande vuestra maldad que habéis hecho en los
ojos de Jehová, pidiéndoos rey.

\bibleverse{18} Y Samuel clamó á Jehová; y Jehová dió truenos y aguas en
aquel día; y todo el pueblo temió en gran manera á Jehová y á Samuel.

\hypertarget{samuel-anima-al-pueblo-les-exhorta-a-temer-a-dios-y-les-manda-recibir-bendiciones-divinas}{%
\subsection{Samuel anima al pueblo, les exhorta a temer a Dios y les
manda recibir bendiciones
divinas}\label{samuel-anima-al-pueblo-les-exhorta-a-temer-a-dios-y-les-manda-recibir-bendiciones-divinas}}

\bibleverse{19} Entonces dijo todo el pueblo á Samuel: Ruega por tus
siervos á Jehová tu Dios, que no muramos: porque á todos nuestros
pecados hemos añadido este mal de pedir rey para nosotros.

\bibleverse{20} Y Samuel respondió al pueblo: No temáis: vosotros habéis
cometido todo este mal; mas con todo eso no os apartéis de en pos de
Jehová, sino servid á Jehová con todo vuestro corazón: \bibleverse{21}
No os apartéis en pos de las vanidades, que no aprovechan ni libran,
porque son vanidades. \bibleverse{22} Pues Jehová no desamparará á su
pueblo por su grande nombre; porque Jehová ha querido haceros pueblo
suyo. \bibleverse{23} Así que, lejos sea de mí que peque yo contra
Jehová cesando de rogar por vosotros; antes yo os enseñaré por el camino
bueno y derecho. \bibleverse{24} Solamente temed á Jehová, y servidle de
verdad con todo vuestro corazón, porque considerad cuán grandes cosas ha
hecho con vosotros. \bibleverse{25} Mas si perseverareis en hacer mal,
vosotros y vuestro rey pereceréis.

\hypertarget{estallido-de-la-guerra-filistea-la-primera-desobediencia-de-sauxfal-mediante-un-sacrificio-apresurado}{%
\subsection{Estallido de la guerra filistea; La primera desobediencia de
Saúl mediante un sacrificio
apresurado}\label{estallido-de-la-guerra-filistea-la-primera-desobediencia-de-sauxfal-mediante-un-sacrificio-apresurado}}

\hypertarget{section-09-13}{%
\section{13}\label{section-09-13}}

\bibleverse{1} Había ya Saúl reinado un año; y reinado que hubo dos años
sobre Israel,

\bibleverse{2} Escogióse luego tres mil de Israel: los dos mil
estuvieron con Saúl en Michmas y en el monte de Beth-el, y los mil
estuvieron con Jonathán en Gabaa de Benjamín; y envió á todo el otro
pueblo cada uno á sus tiendas. \bibleverse{3} Y Jonathán hirió la
guarnición de los Filisteos que había en el collado, y oyéronlo los
Filisteos. E hizo Saúl tocar trompetas por toda la tierra, diciendo:
Oigan los Hebreos. \bibleverse{4} Y todo Israel oyó que se decía: Saúl
ha herido la guarnición de los Filisteos; y también que Israel olía mal
á los Filisteos. Y juntóse el pueblo en pos de Saúl en Gilgal.
\bibleverse{5} Entonces los Filisteos se juntaron para pelear con
Israel, treinta mil carros, y seis mil caballos, y pueblo como la arena
que está á la orilla de la mar en multitud; y subieron, y asentaron
campo en Michmas, al oriente de Beth-aven. \bibleverse{6} Mas los
hombres de Israel, viéndose puestos en estrecho, (porque el pueblo
estaba en aprieto), escondióse el pueblo en cuevas, en fosos, en
peñascos, en rocas y en cisternas. \bibleverse{7} Y algunos de los
Hebreos pasaron el Jordán á la tierra de Gad y de Galaad: y Saúl se
estaba aún en Gilgal, y todo el pueblo iba tras él temblando.

\hypertarget{el-sacrificio-apresurado-y-arbitrario-de-sauxfal-en-gilgal-rompe-entre-samuel-y-el-rey-el-rechazo-de-sauxfal}{%
\subsection{El sacrificio apresurado y arbitrario de Saúl en Gilgal;
Rompe entre Samuel y el rey; El rechazo de
Saúl}\label{el-sacrificio-apresurado-y-arbitrario-de-sauxfal-en-gilgal-rompe-entre-samuel-y-el-rey-el-rechazo-de-sauxfal}}

\bibleverse{8} Y él esperó siete días, conforme al plazo que Samuel
había dicho; pero Samuel no venía á Gilgal, y el pueblo se le desertaba.
\bibleverse{9} Entonces dijo Saúl: Traedme holocausto y sacrificios
pacíficos. Y ofreció el holocausto.

\bibleverse{10} Y como él acababa de hacer el holocausto, he aquí Samuel
que venía; y Saúl le salió á recibir para saludarle. \bibleverse{11}
Entonces Samuel dijo: ¿Qué has hecho? Y Saúl respondió: Porque vi que el
pueblo se me iba, y que tú no venías al plazo de los días, y que los
Filisteos estaban juntos en Michmas,

\bibleverse{12} Me dije: Los Filisteos descenderán ahora contra mí á
Gilgal, y yo no he implorado el favor de Jehová. Esforcéme pues, y
ofrecí holocausto.

\bibleverse{13} Entonces Samuel dijo á Saúl: Locamente has hecho; no
guardaste el mandamiento de Jehová tu Dios, que él te había intimado;
porque ahora Jehová hubiera confirmado tu reino sobre Israel para
siempre. \bibleverse{14} Mas ahora tu reino no será durable: Jehová se
ha buscado varón según su corazón, al cual Jehová ha mandado que sea
capitán sobre su pueblo, por cuanto tú no has guardado lo que Jehová te
mandó.

\hypertarget{el-pequeuxf1o-ejuxe9rcito-de-sauxfal-el-pillaje-de-los-filisteos-indefensiuxf3n-de-los-israelitas}{%
\subsection{El pequeño ejército de Saúl; el pillaje de los filisteos;
Indefensión de los
israelitas}\label{el-pequeuxf1o-ejuxe9rcito-de-sauxfal-el-pillaje-de-los-filisteos-indefensiuxf3n-de-los-israelitas}}

\bibleverse{15} Y levantándose Samuel, subió de Gilgal á Gabaa de
Benjamín. Y Saúl contó la gente que se hallaba con él, como seiscientos
hombres. \bibleverse{16} Saúl pues y Jonathán su hijo, y el pueblo que
con ellos se hallaba, quedáronse en Gabaa de Benjamín: mas los Filisteos
habían puesto su campo en Michmas. \bibleverse{17} Y salieron del campo
de los Filisteos en correría tres escuadrones. El un escuadrón tiró por
el camino de Ophra hacia la tierra de Sual. \bibleverse{18} El otro
escuadrón marchó hacia Beth-oron, y el tercer escuadrón marchó hacia la
región que mira al valle de Seboim hacia el desierto. \bibleverse{19} Y
en toda la tierra de Israel no se hallaba herrero; porque los Filisteos
habían dicho: Para que los Hebreos no hagan espada ó lanza.
\bibleverse{20} Y todos los de Israel descendían á los Filisteos cada
cual á amolar su reja, su azadón, su hacha, ó su sacho, \bibleverse{21}
Y cuando se hacían bocas en las rejas, ó en los azadones, ó en las
horquillas, ó en las hachas; hasta para una ahijada que se hubiera de
componer. \bibleverse{22} Así aconteció que el día de la batalla no se
halló espada ni lanza en mano de alguno de todo el pueblo que estaba con
Saúl y con Jonathán, excepto Saúl y Jonathán su hijo, que las tenían.

\bibleverse{23} Y la guarnición de los Filisteos salió al paso de
Michmas.

\hypertarget{el-herouxedsmo-de-jonathan-la-victoria-de-sauxfal-sobre-los-filisteos}{%
\subsection{El heroísmo de Jonathan; La victoria de Saúl sobre los
filisteos}\label{el-herouxedsmo-de-jonathan-la-victoria-de-sauxfal-sobre-los-filisteos}}

\hypertarget{section-09-14}{%
\section{14}\label{section-09-14}}

\bibleverse{1} Y un día aconteció, que Jonathán hijo de Saúl dijo á su
criado que le traía las armas: Ven, y pasemos á la guarnición de los
Filisteos, que está á aquel lado. Y no lo hizo saber á su padre.
\bibleverse{2} Y Saúl estaba en el término de Gabaa, debajo de un
granado que hay en Migrón, y el pueblo que estaba con él era como
seiscientos hombres. \bibleverse{3} Y Achîas hijo de Achîtob, hermano de
Ichâbod, hijo de Phinees, hijo de Eli, sacerdote de Jehová en Silo,
llevaba el ephod; y no sabía el pueblo que Jonathán se hubiese ido.

\bibleverse{4} Y entre los pasos por donde Jonathán procuraba pasar á la
guarnición de los Filisteos, había un peñasco agudo de la una parte, y
otro de la otra parte; el uno se llamaba Boses y el otro Sene:
\bibleverse{5} El un peñasco situado al norte hacia Michmas, y el otro
al mediodía hacia Gabaa. \bibleverse{6} Dijo pues Jonathán á su criado
que le traía las armas: Ven, pasemos á la guarnición de estos
incircuncisos: quizá hará Jehová por nosotros; que no es difícil á
Jehová salvar con multitud ó con poco número.

\bibleverse{7} Y su paje de armas le respondió: Haz todo lo que tienes
en tu corazón: ve, que aquí estoy contigo á tu voluntad.

\bibleverse{8} Y Jonathán dijo: He aquí, nosotros pasaremos á los
hombres, y nos mostraremos á ellos. \bibleverse{9} Si nos dijeren así:
Esperad hasta que lleguemos á vosotros; entonces nos estaremos en
nuestro lugar, y no subiremos á ellos. \bibleverse{10} Mas si nos
dijeren así: Subid á nosotros: entonces subiremos, porque Jehová los ha
entregado en nuestras manos: y esto nos será por señal.

\bibleverse{11} Mostráronse pues ambos á la guarnición de los Filisteos,
y los Filisteos dijeron: He aquí los Hebreos, que salen de las cavernas
en que se habían escondido. \bibleverse{12} Y los hombres de la
guarnición respondieron á Jonathán y á su paje de armas, y dijeron:
Subid á nosotros, y os haremos saber una cosa. Entonces Jonathán dijo á
su paje de armas: Sube tras mí, que Jehová los ha entregado en la mano
de Israel.

\bibleverse{13} Y subió Jonathán trepando con sus manos y sus pies, y
tras él su paje de armas; y los que caían delante de Jonathán, su paje
de armas que iba tras él, los mataba. \bibleverse{14} Esta fué la
primera rota, en la cual Jonathán con su paje de armas, mataron como
unos veinte hombres en el espacio de una media yugada.

\bibleverse{15} Y hubo temblor en el real y por el campo, y entre toda
la gente de la guarnición; y los que habían ido á hacer correrías,
también ellos temblaron, y alborotóse la tierra: hubo pues gran
consternación.

\hypertarget{sauxfal-interviene-y-obtiene-una-brillante-victoria}{%
\subsection{Saúl interviene y obtiene una brillante
victoria}\label{sauxfal-interviene-y-obtiene-una-brillante-victoria}}

\bibleverse{16} Y las centinelas de Saúl vieron desde Gabaa de Benjamín
cómo la multitud estaba turbada, é iba de una parte á otra, y era
deshecha. \bibleverse{17} Entonces Saúl dijo al pueblo que tenía
consigo: Reconoced luego, y mirad quién haya ido de los nuestros. Y
reconocido que hubieron, hallaron que faltaban Jonathán y su paje de
armas.

\bibleverse{18} Y Saúl dijo á Achîas: Trae el arca de Dios. Porque el
arca de Dios estaba entonces con los hijos de Israel. \bibleverse{19} Y
aconteció que estando aún hablando Saúl con el sacerdote, el alboroto
que había en el campo de los Filisteos se aumentaba, é iba creciendo en
gran manera. Entonces dijo Saúl al sacerdote: Detén tu mano.

\bibleverse{20} Y juntando Saúl todo el pueblo que con él estaba,
vinieron hasta el lugar de la batalla: y he aquí que la espada de cada
uno era vuelta contra su compañero, y la mortandad era grande.
\bibleverse{21} Y los Hebreos que habían estado con los Filisteos de
tiempo antes, y habían venido con ellos de los alrededores al campo,
también éstos se volvieron para ser con los Israelitas que estaban con
Saúl y con Jonathán. \bibleverse{22} Asimismo todos los Israelitas que
se habían escondido en el monte de Ephraim, oyendo que los Filisteos
huían, ellos también los persiguieron en aquella batalla.
\bibleverse{23} Así salvó Jehová á Israel aquel día. Y llegó el alcance
hasta Beth-aven.

\hypertarget{el-celo-intempestivo-de-sauxfal-jonathan-estuxe1-amenazado-de-muerte-las-guerras-de-sauxfal-y-su-familia}{%
\subsection{El celo intempestivo de Saúl; Jonathan está amenazado de
muerte; Las guerras de Saúl y su
familia}\label{el-celo-intempestivo-de-sauxfal-jonathan-estuxe1-amenazado-de-muerte-las-guerras-de-sauxfal-y-su-familia}}

\bibleverse{24} Pero los hombres de Israel fueron puestos en apuro aquel
día; porque Saúl había conjurado al pueblo, diciendo: Cualquiera que
comiere pan hasta la tarde, hasta que haya tomado venganza de mis
enemigos, sea maldito. Y todo el pueblo no había gustado pan.

\bibleverse{25} Y todo el pueblo del país llegó á un bosque donde había
miel en la superficie del campo. \bibleverse{26} Entró pues el pueblo en
el bosque, y he aquí que la miel corría; mas ninguno hubo que llegase la
mano á su boca: porque el pueblo temía el juramento. \bibleverse{27}
Empero Jonathán no había oído cuando su padre conjuró al pueblo, y
alargó la punta de una vara que traía en su mano, y mojóla en un panal
de miel, y llegó su mano á su boca; y sus ojos fueron aclarados.
\bibleverse{28} Entonces habló uno del pueblo, diciendo: Tu padre ha
conjurado expresamente al pueblo, diciendo: Maldito sea el hombre que
comiere hoy manjar. Y el pueblo desfallecía.

\bibleverse{29} Y respondió Jonathán: Mi padre ha turbado el país. Ved
ahora cómo han sido aclarados mis ojos, por haber gustado un poco de
esta miel: \bibleverse{30} ¿Cuánto más si el pueblo hubiera hoy comido
del despojo de sus enemigos que halló? ¿no se habría hecho ahora mayor
estrago en los Filisteos? \bibleverse{31} E hirieron aquel día á los
Filisteos desde Michmas hasta Ajalón: mas el pueblo se cansó mucho.
\bibleverse{32} Tornóse por tanto el pueblo al despojo, y tomaron ovejas
y vacas y becerros, y matáronlos en tierra, y el pueblo comió con
sangre. \bibleverse{33} Y dándole de ello aviso á Saúl, dijéronle: El
pueblo peca contra Jehová comiendo con sangre. Y él dijo: Vosotros
habéis prevaricado; rodadme ahora acá una grande piedra.

\bibleverse{34} Y Saúl tornó á decir: Esparcíos por el pueblo, y
decidles que me traigan cada uno su vaca, y cada cual su oveja, y
degolladlos aquí, y comed; y no pecaréis contra Jehová comiendo con
sangre. Y trajo todo el pueblo cada cual por su mano su vaca aquella
noche, y degollaron allí.

\bibleverse{35} Y edificó Saúl altar á Jehová, el cual altar fué el
primero que edificó á Jehová.

\hypertarget{jonatuxe1n-amenazado-de-muerte-por-el-celo-ciego-de-sauxfal-es-salvado-por-el-ejuxe9rcito}{%
\subsection{Jonatán, amenazado de muerte por el celo ciego de Saúl, es
salvado por el
ejército}\label{jonatuxe1n-amenazado-de-muerte-por-el-celo-ciego-de-sauxfal-es-salvado-por-el-ejuxe9rcito}}

\bibleverse{36} Y dijo Saúl: Descendamos de noche contra los Filisteos,
y los saquearemos hasta la mañana, y no dejaremos de ellos ninguno. Y
ellos dijeron: Haz lo que bien te pareciere. Dijo luego el sacerdote:
Lleguémonos aquí á Dios.

\bibleverse{37} Y Saúl consultó á Dios: ¿Descenderé tras los Filisteos?
¿los entregarás en mano de Israel? Mas Jehová no le dió respuesta aquel
día. \bibleverse{38} Entonces dijo Saúl: Llegaos acá todos los
principales del pueblo; y sabed y mirad por quién ha sido hoy este
pecado; \bibleverse{39} Porque vive Jehová, que salva á Israel, que si
fuere en mi hijo Jonathán, el morirá de cierto. Y no hubo en todo el
pueblo quien le respondiese. \bibleverse{40} Dijo luego á todo Israel:
Vosotros estaréis á un lado, y yo y Jonathán mi hijo estaremos á otro
lado. Y el pueblo respondió á Saúl: Haz lo que bien te pareciere.

\bibleverse{41} Entonces dijo Saúl á Jehová Dios de Israel: Da
perfección. Y fueron tomados Jonathán y Saúl, y el pueblo salió libre.

\bibleverse{42} Y Saúl dijo: Echad suerte entre mí y Jonathán mi hijo. Y
fué tomado Jonathán.

\bibleverse{43} Entonces Saúl dijo á Jonathán: Declárame qué has hecho.
Y Jonathán se lo declaró, y dijo: Cierto que gusté con la punta de la
vara que traía en mi mano, un poco de miel: ¿y he aquí he de morir?

\bibleverse{44} Y Saúl respondió: Así me haga Dios y así me añada, que
sin duda morirás, Jonathán.

\bibleverse{45} Mas el pueblo dijo á Saúl: ¿Ha pues de morir Jonathán,
el que ha hecho esta salud grande en Israel? No será así. Vive Jehová,
que no ha de caer un cabello de su cabeza en tierra, pues que ha obrado
hoy con Dios. Así libró el pueblo á Jonathán, para que no muriese.
\bibleverse{46} Y Saúl dejó de seguir á los Filisteos; y los Filisteos
se fueron á su lugar.

\hypertarget{los-otros-actos-de-guerra-de-sauxfal-y-su-familia}{%
\subsection{Los otros actos de guerra de Saúl y su
familia}\label{los-otros-actos-de-guerra-de-sauxfal-y-su-familia}}

\bibleverse{47} Y ocupando Saúl el reino sobre Israel, hizo guerra á
todos sus enemigos alrededor: contra Moab, contra los hijos de Ammón,
contra Edom, contra los reyes de Soba, y contra los Filisteos: y á donde
quiera que se tornaba era vencedor. \bibleverse{48} Y reunió un
ejército, é hirió á Amalec, y libró á Israel de mano de los que le
robaban. \bibleverse{49} Y los hijos de Saúl fueron Jonathán, Isui, y
Melchi-sua. Y los nombres de sus dos hijas eran, el nombre de la mayor,
Merab, y el de la menor, Michâl. \bibleverse{50} Y el nombre de la mujer
de Saúl era Ahinoam, hija de Aimaas. Y el nombre del general de su
ejército era Abner, hijo de Ner tío de Saúl. \bibleverse{51} Porque Cis
padre de Saúl, y Ner padre de Abner, fueron hijos de Abiel.

\bibleverse{52} Y la guerra fué fuerte contra los Filisteos todo el
tiempo de Saúl; y á cualquiera que Saúl veía hombre valiente y hombre de
esfuerzo, juntábale consigo.

\hypertarget{la-campauxf1a-de-sauxfal-contra-los-amalecitas-su-desobediencia-a-dios-y-su-rechazo}{%
\subsection{La campaña de Saúl contra los amalecitas; su desobediencia a
Dios y su
rechazo}\label{la-campauxf1a-de-sauxfal-contra-los-amalecitas-su-desobediencia-a-dios-y-su-rechazo}}

\hypertarget{section-09-15}{%
\section{15}\label{section-09-15}}

\bibleverse{1} Y Samuel dijo á Saúl: Jehová me envió á que te ungiese
por rey sobre su pueblo Israel: oye pues la voz de las palabras de
Jehová. \bibleverse{2} Así ha dicho Jehová de los ejércitos: Acuérdome
de lo que hizo Amalec á Israel; que se le opuso en el camino, cuando
subía de Egipto. \bibleverse{3} Ve pues, y hiere á Amalec, y destruiréis
en él todo lo que tuviere: y no te apiades de él: mata hombres y
mujeres, niños y mamantes, vacas y ovejas, camellos y asnos.

\bibleverse{4} Saúl pues juntó el pueblo, y reconociólos en Telaim,
doscientos mil de á pie, y diez mil hombres de Judá. \bibleverse{5} Y
viniendo Saúl á la ciudad de Amalec, puso emboscada en el valle.
\bibleverse{6} Y dijo Saúl al Cineo: Idos, apartaos, y salid de entre
los de Amalec, para que no te destruya juntamente con él: pues que tú
hiciste misericordia con todos los hijos de Israel, cuando subían de
Egipto. Apartóse pues el Cineo de entre los de Amalec.

\bibleverse{7} Y Saúl hirió á Amalec, desde Havila hasta llegar á Shur,
que está á la frontera de Egipto. \bibleverse{8} Y tomó vivo á Agag rey
de Amalec, mas á todo el pueblo mató á filo de espada. \bibleverse{9} Y
Saúl y el pueblo perdonaron á Agag, y á lo mejor de las ovejas, y al
ganado mayor, á los gruesos y á los carneros, y á todo lo bueno: que no
lo quisieron destruir: mas todo lo que era vil y flaco destruyeron.

\hypertarget{saulo-rechazado-por-dios-a-causa-de-su-desobediencia-el-discurso-de-samuel-y-la-admisiuxf3n-de-culpabilidad-de-sauxfal}{%
\subsection{Saulo rechazado por Dios a causa de su desobediencia; El
discurso de Samuel y la admisión de culpabilidad de
Saúl}\label{saulo-rechazado-por-dios-a-causa-de-su-desobediencia-el-discurso-de-samuel-y-la-admisiuxf3n-de-culpabilidad-de-sauxfal}}

\bibleverse{10} Y fué palabra de Jehová á Samuel, diciendo:
\bibleverse{11} Pésame de haber puesto por rey á Saúl, porque se ha
vuelto de en pos de mí, y no ha cumplido mis palabras. Y apesadumbróse
Samuel, y clamó á Jehová toda aquella noche.

\bibleverse{12} Madrugó luego Samuel para ir á encontrar á Saúl por la
mañana; y fué dado aviso á Samuel, diciendo: Saúl ha venido al Carmel, y
he aquí él se ha levantado un trofeo, y después volviendo, ha pasado y
descendido á Gilgal.

\bibleverse{13} Vino pues Samuel á Saúl, y Saúl le dijo: Bendito seas tú
de Jehová; yo he cumplido la palabra de Jehová.

\bibleverse{14} Samuel entonces dijo: ¿Pues qué balido de ganados y
bramido de bueyes es este que yo oigo con mis oídos?

\bibleverse{15} Y Saúl respondió: De Amalec los han traído; porque el
pueblo perdonó á lo mejor de las ovejas y de las vacas, para
sacrificarlas á Jehová tu Dios; pero lo demás lo destruimos.

\bibleverse{16} Entonces dijo Samuel á Saúl: Déjame declararte lo que
Jehová me ha dicho esta noche. Y él le respondió: Di.

\bibleverse{17} Y dijo Samuel: Siendo tú pequeño en tus ojos ¿no has
sido hecho cabeza á las tribus de Israel, y Jehová te ha ungido por rey
sobre Israel? \bibleverse{18} Y envióte Jehová en jornada, y dijo: Ve, y
destruye los pecadores de Amalec, y hazles guerra hasta que los acabes.
\bibleverse{19} ¿Por qué pues no has oído la voz de Jehová, sino que
vuelto al despojo, has hecho lo malo en los ojos de Jehová?

\bibleverse{20} Y Saúl respondió á Samuel: Antes he oído la voz de
Jehová, y fuí á la jornada que Jehová me envió, y he traído á Agag rey
de Amalec, y he destruído á los Amalecitas: \bibleverse{21} Mas el
pueblo tomó del despojo ovejas y vacas, las primicias del anatema, para
sacrificarlas á Jehová tu Dios en Gilgal.

\bibleverse{22} Y Samuel dijo: ¿Tiene Jehová tanto contentamiento con
los holocaustos y víctimas, como en obedecer á las palabras de Jehová?
Ciertamente el obedecer es mejor que los sacrificios; y el prestar
atención que el sebo de los carneros: \bibleverse{23} Porque como pecado
de adivinación es la rebelión, y como ídolos é idolatría el infringir.
Por cuanto tú desechaste la palabra de Jehová, él también te ha
desechado para que no seas rey.

\bibleverse{24} Entonces Saúl dijo á Samuel: Yo he pecado; que he
quebrantado el dicho de Jehová y tus palabras: porque temí al pueblo,
consentí á la voz de ellos. Perdona pues ahora mi pecado,
\bibleverse{25} Y vuelve conmigo para que adore á Jehová.

\bibleverse{26} Y Samuel respondió á Saúl: No volveré contigo; porque
desechaste la palabra de Jehová, y Jehová te ha desechado para que no
seas rey sobre Israel. \bibleverse{27} Y volviéndose Samuel para irse,
él echó mano de la orla de su capa, y desgarróse. \bibleverse{28}
Entonces Samuel le dijo: Jehová ha desgarrado hoy de ti el reino de
Israel, y lo ha dado á tu prójimo mejor que tú. \bibleverse{29} Y
también el Vencedor de Israel no mentirá, ni se arrepentirá: porque no
es hombre para que se arrepienta.

\bibleverse{30} Y él dijo: Yo he pecado: mas ruégote que me honres
delante de los ancianos de mi pueblo, y delante de Israel; y vuelve
conmigo para que adore á Jehová tu Dios.

\bibleverse{31} Y volvió Samuel tras Saúl, y adoró Saúl á Jehová.

\hypertarget{samuel-realiza-la-proscripciuxf3n-del-rey-agag-y-se-separa-de-sauxfal-para-no-volver-a-ser-visto}{%
\subsection{Samuel realiza la proscripción del rey Agag y se separa de
Saúl para no volver a ser
visto}\label{samuel-realiza-la-proscripciuxf3n-del-rey-agag-y-se-separa-de-sauxfal-para-no-volver-a-ser-visto}}

\bibleverse{32} Después dijo Samuel: Traedme á Agag rey de Amalec. Y
Agag vino á él delicadamente. Y dijo Agag: Ciertamente se pasó la
amargura de la muerte.

\bibleverse{33} Y Samuel dijo: Como tu espada dejó las mujeres sin
hijos, así tu madre será sin hijo entre las mujeres. Entonces Samuel
cortó en pedazos á Agag delante de Jehová en Gilgal.

\bibleverse{34} Fuése luego Samuel á Rama, y Saúl subió á su casa en
Gabaa de Saúl. \bibleverse{35} Y nunca después vió Samuel á Saúl en toda
su vida: y Samuel lloraba á Saúl: mas Jehová se había arrepentido de
haber puesto á Saúl por rey sobre Israel.

\hypertarget{el-llamado-y-unciuxf3n-de-david-por-samuel}{%
\subsection{El llamado y unción de David por
Samuel}\label{el-llamado-y-unciuxf3n-de-david-por-samuel}}

\hypertarget{section-09-16}{%
\section{16}\label{section-09-16}}

\bibleverse{1} Y dijo Jehová á Samuel: ¿Hasta cuándo has tú de llorar á
Saúl, habiéndolo yo desechado para que no reine sobre Israel? Hinche tu
cuerno de aceite, y ven, te enviaré á Isaí de Beth-lehem: porque de sus
hijos me he provisto de rey.

\bibleverse{2} Y dijo Samuel: ¿Cómo iré? Si Saúl lo entendiere, me
matará. Jehová respondió: Toma contigo una becerra de la vacada, y di: A
sacrificar á Jehová he venido.

\bibleverse{3} Y llama á Isaí al sacrificio, y yo te enseñaré lo que has
de hacer; y ungirme has al que yo te dijere.

\bibleverse{4} Hizo pues Samuel como le dijo Jehová: y luego que él
llegó á Beth-lehem, los ancianos de la ciudad le salieron á recibir con
miedo, y dijeron: ¿Es pacífica tu venida?

\bibleverse{5} Y él respondió: Sí, vengo á sacrificar á Jehová;
santificaos, y venid conmigo al sacrificio. Y santificando él á Isaí y á
sus hijos, llamólos al sacrificio.

\hypertarget{samuel-unge-como-rey-al-hijo-menor-de-isauxed-david}{%
\subsection{Samuel unge como rey al hijo menor de Isaí,
David}\label{samuel-unge-como-rey-al-hijo-menor-de-isauxed-david}}

\bibleverse{6} Y aconteció que como ellos vinieron, él vió á Eliab, y
dijo: De cierto delante de Jehová está su ungido.

\bibleverse{7} Y Jehová respondió á Samuel: No mires á su parecer, ni á
lo grande de su estatura, porque yo lo desecho; porque Jehová mira no lo
que el hombre mira; pues que el hombre mira lo que está delante de sus
ojos, mas Jehová mira el corazón.

\bibleverse{8} Entonces llamó Isaí á Abinadab, é hízole pasar delante de
Samuel, el cual dijo: Ni á éste ha elegido Jehová. \bibleverse{9} Hizo
luego pasar Isaí á Samma. Y él dijo: Tampoco á éste ha elegido Jehová.
\bibleverse{10} E hizo pasar Isaí sus siete hijos delante de Samuel; mas
Samuel dijo á Isaí: Jehová no ha elegido á éstos. \bibleverse{11}
Entonces dijo Samuel á Isaí: ¿Hanse acabado los mozos? Y él respondió:
Aun queda el menor, que apacienta las ovejas. Y dijo Samuel á Isaí:
Envía por él, porque no nos asentaremos á la mesa hasta que él venga
aquí.

\bibleverse{12} Envió pues por él, é introdújolo; el cual era rubio, de
hermoso parecer y de bello aspecto. Entonces Jehová dijo: Levántate y
úngelo, que éste es.

\bibleverse{13} Y Samuel tomó el cuerno del aceite, y ungiólo de entre
sus hermanos: y desde aquel día en adelante el espíritu de Jehová tomó á
David. Levantóse luego Samuel, y volvióse á Rama.

\hypertarget{david-es-llamado-a-tocar-el-arpa-en-la-corte-de-sauxfal-y-entra-al-servicio-real}{%
\subsection{David es llamado a tocar el arpa en la corte de Saúl y entra
al servicio
real}\label{david-es-llamado-a-tocar-el-arpa-en-la-corte-de-sauxfal-y-entra-al-servicio-real}}

\bibleverse{14} Y el espíritu de Jehová se apartó de Saúl, y
atormentábale el espíritu malo de parte de Jehová. \bibleverse{15} Y los
criados de Saúl le dijeron: He aquí ahora, que el espíritu malo de parte
de Dios te atormenta. \bibleverse{16} Diga pues nuestro señor á tus
siervos que están delante de ti, que busquen alguno que sepa tocar el
arpa; para que cuando fuere sobre ti el espíritu malo de parte de Dios,
él taña con su mano, y tengas alivio.

\bibleverse{17} Y Saúl respondió á sus criados: Buscadme pues ahora
alguno que taña bien, y traédmelo.

\bibleverse{18} Entonces uno de los criados respondió, diciendo: He aquí
yo he visto á un hijo de Isaí de Beth-lehem, que sabe tocar, y es
valiente y vigoroso, y hombre de guerra, prudente en sus palabras, y
hermoso, y Jehová es con él.

\bibleverse{19} Y Saúl envió mensajeros á Isaí, diciendo: Envíame á
David tu hijo, el que está con las ovejas.

\bibleverse{20} Y tomó Isaí un asno cargado de pan, y una vasija de vino
y un cabrito, y enviólo á Saúl por mano de David su hijo.
\bibleverse{21} Y viniendo David á Saúl, estuvo delante de él: y amólo
él mucho, y fué hecho su escudero. \bibleverse{22} Y Saúl envió á decir
á Isaí: Yo te ruego que esté David conmigo; porque ha hallado gracia en
mis ojos. \bibleverse{23} Y cuando el espíritu malo de parte de Dios era
sobre Saúl, David tomaba el arpa, y tañía con su mano; y Saúl tenía
refrigerio, y estaba mejor, y el espíritu malo se apartaba de él.

\hypertarget{david-y-el-campeuxf3n-enemigo-goliat}{%
\subsection{David y el campeón enemigo
Goliat}\label{david-y-el-campeuxf3n-enemigo-goliat}}

\hypertarget{section-09-17}{%
\section{17}\label{section-09-17}}

\bibleverse{1} Y los Filisteos juntaron sus ejércitos para la guerra, y
congregáronse en Sochô, que es de Judá, y asentaron el campo entre Sochô
y Azeca, en Ephes-dammim. \bibleverse{2} Y también Saúl y los hombres de
Israel se juntaron, y asentaron el campo en el valle del Alcornoque, y
ordenaron la batalla contra los Filisteos. \bibleverse{3} Y los
Filisteos estaban sobre el un monte de la una parte, é Israel estaba
sobre el otro monte de la otra parte, y el valle entre ellos:
\bibleverse{4} Salió entonces un varón del campo de los Filisteos que se
puso entre los dos campos, el cual se llamaba Goliath, de Gath, y tenía
de altura seis codos y un palmo. \bibleverse{5} Y traía un almete de
acero en su cabeza, é iba vestido con corazas de planchas: y era el peso
de la coraza cinco mil siclos de metal: \bibleverse{6} Y sobre sus
piernas traía grebas de hierro, y escudo de acero á sus hombros.
\bibleverse{7} El asta de su lanza era como un enjullo de telar, y tenía
el hierro de su lanza seiscientos siclos de hierro: é iba su escudero
delante de él. \bibleverse{8} Y paróse, y dió voces á los escuadrones de
Israel, diciéndoles: ¿Para qué salís á dar batalla? ¿no soy yo el
Filisteo, y vosotros los siervos de Saúl? Escoged de entre vosotros un
hombre que venga contra mí: \bibleverse{9} Si él pudiere pelear conmigo,
y me venciere, nosotros seremos vuestros siervos: y si yo pudiere más
que él, y lo venciere, vosotros seréis nuestros siervos y nos serviréis.
\bibleverse{10} Y añadió el Filisteo: Hoy yo he desafiado el campo de
Israel; dadme un hombre que pelee conmigo.

\bibleverse{11} Y oyendo Saúl y todo Israel estas palabras del Filisteo,
conturbáronse, y tuvieron gran miedo.

\hypertarget{david-enviado-por-su-padre-a-sus-hermanos-en-el-campamento-estuxe1-indignado-por-la-arrogancia-de-goliat-y-se-siente-llamado-a-pelear-con-uxe9l}{%
\subsection{David, enviado por su padre a sus hermanos en el campamento,
está indignado por la arrogancia de Goliat y se siente llamado a pelear
con
él}\label{david-enviado-por-su-padre-a-sus-hermanos-en-el-campamento-estuxe1-indignado-por-la-arrogancia-de-goliat-y-se-siente-llamado-a-pelear-con-uxe9l}}

\bibleverse{12} Y David era hijo de aquel hombre Ephrateo de Beth-lehem
de Judá, cuyo nombre era Isaí, el cual tenía ocho hijos; y era este
hombre en el tiempo de Saúl, viejo, y de grande edad entre los hombres.
\bibleverse{13} Y los tres hijos mayores de Isaí habían ido á seguir á
Saúl en la guerra. Y los nombres de sus tres hijos que habían ido á la
guerra, eran, Eliab el primogénito, el segundo Abinadab, y el tercero
Samma. \bibleverse{14} Y David era el menor. Siguieron pues los tres
mayores á Saúl. \bibleverse{15} Empero David había ido y vuelto de con
Saúl, para apacentar las ovejas de su padre en Beth-lehem.

\bibleverse{16} Venía pues aquel Filisteo por la mañana y á la tarde, y
presentóse por cuarenta días.

\bibleverse{17} Y dijo Isaí á David su hijo: Toma ahora para tus
hermanos un epha de este grano tostado, y estos diez panes, y llévalo
presto al campamento á tus hermanos. \bibleverse{18} Llevarás asimismo
estos diez quesos de leche al capitán, y cuida de ver si tus hermanos
están buenos, y toma prendas de ellos. \bibleverse{19} Y Saúl y ellos y
todos lo de Israel, estaban en el valle del Alcornoque, peleando con los
Filisteos.

\bibleverse{20} Levantóse pues David de mañana, y dejando las ovejas al
cuidado de un guarda, fuése con su carga, como Isaí le había mandado; y
llegó al atrincheramiento del ejército, el cual había salido en
ordenanza, y tocaba alarma para la pelea. \bibleverse{21} Porque así los
Israelitas como los Filisteos estaban en ordenanza, escuadrón contra
escuadrón. \bibleverse{22} Y David dejó de sobre sí la carga en mano del
que guardaba el bagaje, y corrió al escuadrón; y llegado que hubo,
preguntaba por sus hermanos, si estaban buenos. \bibleverse{23} Y
estando él hablando con ellos, he aquí aquel varón que se ponía en medio
de los dos campos, que se llamaba Goliath, el Filisteo de Gath, salió de
los escuadrones de los Filisteos, y habló las mismas palabras; las
cuales oyó David. \bibleverse{24} Y todos los varones de Israel que
veían aquel hombre, huían de su presencia, y tenían gran temor.
\bibleverse{25} Y cada uno de los de Israel decía: ¿No habéis visto
aquel hombre que ha salido? él se adelanta para provocar á Israel. Al
que le venciere, el rey le enriquecerá con grandes riquezas, y le dará
su hija, y hará franca la casa de su padre en Israel.

\bibleverse{26} Entonces habló David á los que junto á él estaban,
diciendo: ¿Qué harán al hombre que venciere á este Filisteo, y quitare
el oprobio de Israel? Porque ¿quién es este Filisteo incircunciso, para
que provoque á los escuadrones del Dios viviente?

\bibleverse{27} Y el pueblo le respondió las mismas palabras, diciendo:
Así se hará al hombre que lo venciere.

\bibleverse{28} Y oyéndole hablar Eliab su hermano mayor con aquellos
hombres, Eliab se encendió en ira contra David, y dijo: ¿Para qué has
descendido acá? ¿y á quién has dejado aquellas pocas ovejas en el
desierto? Yo conozco tu soberbia y la malicia de tu corazón, que para
ver la batalla has venido.

\bibleverse{29} Y David respondió: ¿Qué he hecho yo ahora? Estas, ¿no
son palabras? \bibleverse{30} Y apartándose de él hacia otros, habló lo
mismo; y respondiéronle los del pueblo como primero.

\hypertarget{david-se-ofrece-a-duelo-pero-rechaza-la-armadura-de-sauxfal-y-solo-usa-su-honda-como-arma}{%
\subsection{David se ofrece a duelo, pero rechaza la armadura de Saúl y
solo usa su honda como
arma}\label{david-se-ofrece-a-duelo-pero-rechaza-la-armadura-de-sauxfal-y-solo-usa-su-honda-como-arma}}

\bibleverse{31} Y fueron oídas las palabras que David había dicho, las
cuales como refiriesen delante de Saúl, él lo hizo venir.
\bibleverse{32} Y dijo David á Saúl: No desmaye ninguno á causa de él;
tu siervo irá y peleará con este Filisteo.

\bibleverse{33} Y dijo Saúl á David: No podrás tú ir contra aquel
Filisteo, para pelear con él; porque tú eres mozo, y él un hombre de
guerra desde su juventud.

\bibleverse{34} Y David respondió á Saúl: Tu siervo era pastor en las
ovejas de su padre, y venía un león, ó un oso, y tomaba algún cordero de
la manada, \bibleverse{35} Y salía yo tras él, y heríalo, y librábale de
su boca: y si se levantaba contra mí, yo le echaba mano de la quijada, y
lo hería y mataba. \bibleverse{36} Fuese león, fuese oso, tu siervo lo
mataba; pues este Filisteo incircunciso será como uno de ellos, porque
ha provocado al ejército del Dios viviente. \bibleverse{37} Y añadió
David: Jehová que me ha librado de las garras del león y de las garras
del oso, él también me librará de la mano de este Filisteo. Y dijo Saúl
á David: Ve, y Jehová sea contigo.

\bibleverse{38} Y Saúl vistió á David de sus ropas, y puso sobre su
cabeza un almete de acero, y armóle de coraza. \bibleverse{39} Y ciñó
David su espada sobre sus vestidos, y probó á andar, porque nunca había
probado. Y dijo David á Saúl: Yo no puedo andar con esto, porque nunca
lo practiqué. Y echando de sí David aquellas cosas,

\bibleverse{40} Tomó su cayado en su mano, y escogióse cinco piedras
lisas del arroyo, y púsolas en el saco pastoril y en el zurrón que
traía, y con su honda en su mano vase hacia el Filisteo.

\hypertarget{la-lucha-victoriosa-de-david-con-goliat}{%
\subsection{La lucha victoriosa de David con
Goliat}\label{la-lucha-victoriosa-de-david-con-goliat}}

\bibleverse{41} Y el Filisteo venía andando y acercándose á David, y su
escudero delante de él. \bibleverse{42} Y como el Filisteo miró y vió á
David túvole en poco; porque era mancebo, y rubio, y de hermoso parecer.
\bibleverse{43} Y dijo el Filisteo á David: ¿Soy yo perro para que
vengas á mí con palos? Y maldijo á David por sus dioses. \bibleverse{44}
Dijo luego el Filisteo á David: Ven á mí, y daré tu carne á las aves del
cielo, y á las bestias del campo.

\bibleverse{45} Entonces dijo David al Filisteo: Tú vienes á mí con
espada y lanza y escudo; mas yo vengo á ti en el nombre de Jehová de los
ejércitos, el Dios de los escuadrones de Israel, que tú has provocado.
\bibleverse{46} Jehová te entregará hoy en mi mano, y yo te venceré, y
quitaré tu cabeza de ti: y daré hoy los cuerpos de los Filisteos á las
aves del cielo y á las bestias de la tierra: y sabrá la tierra toda que
hay Dios en Israel. \bibleverse{47} Y sabrá toda esta congregación que
Jehová no salva con espada y lanza; porque de Jehová es la guerra, y él
os entregará en nuestras manos.

\bibleverse{48} Y aconteció que, como el Filisteo se levantó para ir y
llegarse contra David, David se dió priesa, y corrió al combate contra
el Filisteo. \bibleverse{49} Y metiendo David su mano en el saco, tomó
de allí una piedra, y tirósela con la honda, é hirió al Filisteo en la
frente: y la piedra quedó hincada en la frente, y cayó en tierra sobre
su rostro. \bibleverse{50} Así venció David al Filisteo con honda y
piedra; é hirió al Filisteo y matólo, sin tener David espada en su mano.
\bibleverse{51} Mas corrió David y púsose sobre el Filisteo, y tomando
la espada de él, sacándola de su vaina, matólo, y cortóle con ella la
cabeza. Y como los Filisteos vieron su gigante muerto, huyeron.

\bibleverse{52} Y levantándose los de Israel y de Judá, dieron grita, y
siguieron á los Filisteos hasta llegar al valle, y hasta las puertas de
Ecrón. Y cayeron heridos de los Filisteos por el camino de Saraim, hasta
Gath y Ecrón. \bibleverse{53} Tornando luego los hijos de Israel de
seguir los Filisteos, despojaron su campamento. \bibleverse{54} Y David
tomó la cabeza del Filisteo, y trájola á Jerusalem, mas puso sus armas
en su tienda.

\hypertarget{sauxfal-pregunta-por-david}{%
\subsection{Saúl pregunta por David}\label{sauxfal-pregunta-por-david}}

\bibleverse{55} Y cuando Saúl vió á David que salía á encontrarse con el
Filisteo, dijo á Abner general del ejército: Abner, ¿de quién es hijo
aquel mancebo? Y Abner respondió:

\bibleverse{56} Vive tu alma, oh rey, que no lo sé. Y el rey dijo:
Pregunta pues de quién es hijo aquel mancebo.

\bibleverse{57} Y cuando David volvía de matar al Filisteo, Abner lo
tomó, y llevólo delante de Saúl, teniendo la cabeza del Filisteo en su
mano. \bibleverse{58} Y díjole Saúl: Mancebo, ¿de quién eres hijo? Y
David respondió: Yo soy hijo de tu siervo Isaí de Beth-lehem.

\hypertarget{david-llega-a-la-corte-de-sauxfal-su-relaciuxf3n-con-sauxfal-y-jonatuxe1n}{%
\subsection{David llega a la corte de Saúl; su relación con Saúl y
Jonatán}\label{david-llega-a-la-corte-de-sauxfal-su-relaciuxf3n-con-sauxfal-y-jonatuxe1n}}

\hypertarget{section-09-18}{%
\section{18}\label{section-09-18}}

\bibleverse{1} Y así que él hubo acabado de hablar con Saúl, el alma de
Jonathán fué ligada con la de David, y amólo Jonathán como á su alma.
\bibleverse{2} Y Saúl le tomó aquel día, y no le dejó volver á casa de
su padre. \bibleverse{3} E hicieron alianza Jonathán y David, porque él
le amaba como á su alma. \bibleverse{4} Y Jonathán se desnudó la ropa
que tenía sobre sí, y dióla á David, y otras ropas suyas, hasta su
espada, y su arco, y su talabarte.

\bibleverse{5} Y salía David á donde quiera que Saúl le enviaba, y
portábase prudentemente. Hízolo por tanto Saúl capitán de gente de
guerra, y era acepto en los ojos de todo el pueblo, y en los ojos de los
criados de Saúl.

\hypertarget{regreso-festivo-de-los-guerreros-david-fue-celebrado-como-el-vencedor-por-la-gente}{%
\subsection{Regreso festivo de los guerreros; David fue celebrado como
el vencedor por la
gente}\label{regreso-festivo-de-los-guerreros-david-fue-celebrado-como-el-vencedor-por-la-gente}}

\bibleverse{6} Y aconteció que como volvían ellos, cuando David tornó de
matar al Filisteo, salieron las mujeres de todas las ciudades de Israel
cantando, y con danzas, con tamboriles, y con alegrías y sonajas, á
recibir al rey Saúl. \bibleverse{7} Y cantaban las mujeres que danzaba,
y decían: Saúl hirió sus miles, y David sus diez miles.

\bibleverse{8} Y enojóse Saúl en gran manera, y desagradó esta palabra
en sus ojos, y dijo: A David dieron diez miles, y á mí miles; no le
falta más que el reino. \bibleverse{9} Y desde aquel día Saúl miró de
través á David.

\hypertarget{david-odiado-mortalmente-por-sauxfal-demuestra-ser-un-huxe9roe-de-guerra}{%
\subsection{David, odiado mortalmente por Saúl, demuestra ser un héroe
de
guerra}\label{david-odiado-mortalmente-por-sauxfal-demuestra-ser-un-huxe9roe-de-guerra}}

\bibleverse{10} Otro día aconteció que el espíritu malo de parte de Dios
tomó á Saúl, y mostrábase en su casa con trasportes de profeta: y David
tañía con su mano como los otros días; y estaba una lanza á mano de
Saúl. \bibleverse{11} Y arrojó Saúl la lanza, diciendo: Enclavaré á
David en la pared. Y dos veces se apartó de él David. \bibleverse{12}
Mas Saúl se temía de David, por cuanto Jehová era con él, y se había
apartado de Saúl. \bibleverse{13} Apartólo pues Saúl de sí, é hízole
capitán de mil; y salía y entraba delante del pueblo.

\bibleverse{14} Y David se conducía prudentemente en todos sus negocios,
y Jehová era con él. \bibleverse{15} Y viendo Saúl que se portaba tan
prudentemente, temíase de él. \bibleverse{16} Mas todo Israel y Judá
amaba á David, porque él salía y entraba delante de ellos.

\hypertarget{david-engauxf1ado-para-casarse-con-la-hija-mayor-de-sauxfal-se-casuxf3-con-su-hermana-menor-michal}{%
\subsection{David, engañado para casarse con la hija mayor de Saúl, se
casó con su hermana menor,
Michal}\label{david-engauxf1ado-para-casarse-con-la-hija-mayor-de-sauxfal-se-casuxf3-con-su-hermana-menor-michal}}

\bibleverse{17} Y dijo Saúl á David: He aquí yo te daré á Merab mi hija
mayor por mujer: solamente que me seas hombre valiente, y hagas las
guerras de Jehová. Mas Saúl decía: No será mi mano contra él, mas la
mano de los Filisteos será contra él.

\bibleverse{18} Y David respondió á Saúl: ¿Quién soy yo, ó qué es mi
vida, ó la familia de mi padre en Israel, para ser yerno del rey?

\bibleverse{19} Y venido el tiempo en que Merab, hija de Saúl, se había
de dar á David, fué dada por mujer á Adriel Meholatita.

\hypertarget{el-servicio-militar-de-david-para-la-novia}{%
\subsection{El servicio militar de David para la
novia}\label{el-servicio-militar-de-david-para-la-novia}}

\bibleverse{20} Mas Michâl la otra hija de Saúl amaba á David; y fué
dicho á Saúl, lo cual plugo en sus ojos. \bibleverse{21} Y Saúl dijo: Yo
se la daré, para que le sea por lazo, y para que la mano de los
Filisteos sea contra él. Dijo pues Saúl á David: Con la otra serás mi
yerno hoy.

\bibleverse{22} Y mandó Saúl á sus criados: Hablad en secreto á David,
diciéndole: He aquí, el rey te ama, y todos sus criados te quieren bien;
sé pues yerno del rey.

\bibleverse{23} Y los criados de Saúl hablaron estas palabras á los
oídos de David. Y David dijo: ¿Paréceos á vosotros que es poco ser yerno
del rey, siendo yo un hombre pobre y de ninguna estima?

\bibleverse{24} Y los criados de Saúl le dieron la respuesta diciendo:
Tales palabras ha dicho David.

\bibleverse{25} Y Saúl dijo: Decid así á David: No está el
contentamiento del rey en el dote, sino en cien prepucios de Filisteos,
para que sea tomada venganza de los enemigos del rey. Mas Saúl pensaba
echar á David en manos de los Filisteos. \bibleverse{26} Y como sus
criados declararon á David estas palabras, plugo la cosa en los ojos de
David, para ser yerno del rey. Y como el plazo no era aún cumplido,
\bibleverse{27} Levantóse David, y partióse con su gente, é hirió
doscientos hombres de los Filisteos; y trajo David los prepucios de
ellos, y entregáronlos todos al rey, para que él fuese hecho yerno del
rey. Y Saúl le dió á su hija Michâl por mujer. \bibleverse{28} Pero
Saúl, viendo y considerando que Jehová era con David, y que su hija
Michâl lo amaba, \bibleverse{29} Temióse más de David; y fué Saúl
enemigo de David todos los días.

\bibleverse{30} Y salían los príncipes de los Filisteos; y como ellos
salían, portábase David más prudentemente que todos los siervos de Saúl:
y era su nombre muy ilustre.

\hypertarget{la-reconciliaciuxf3n-de-sauxfal-con-david-como-resultado-de-la-intercesiuxf3n-de-jonatuxe1n-despuuxe9s-de-los-repetidos-asesinatos-de-sauxfal-david-huye-a-samuel}{%
\subsection{La reconciliación de Saúl con David como resultado de la
intercesión de Jonatán; Después de los repetidos asesinatos de Saúl,
David huye a
Samuel}\label{la-reconciliaciuxf3n-de-sauxfal-con-david-como-resultado-de-la-intercesiuxf3n-de-jonatuxe1n-despuuxe9s-de-los-repetidos-asesinatos-de-sauxfal-david-huye-a-samuel}}

\hypertarget{section-09-19}{%
\section{19}\label{section-09-19}}

\bibleverse{1} Y habló Saúl á Jonathán su hijo, y á todos sus criados,
para que matasen á David; mas Jonathán hijo de Saúl amaba á David en
gran manera. \bibleverse{2} Y dió aviso á David, diciendo: Saúl mi padre
procura matarte; por tanto mira ahora por ti hasta la mañana, y estáte
en paraje oculto, y escóndete: \bibleverse{3} Y yo saldré y estaré junto
á mi padre en el campo donde estuvieres: y hablaré de ti á mi padre, y
te haré saber lo que notare.

\bibleverse{4} Y Jonathán habló bien de David á Saúl su padre, y díjole:
No peque el rey contra su siervo David, pues que ninguna cosa ha
cometido contra ti: antes sus obras te han sido muy buenas;
\bibleverse{5} Porque él puso su alma en su palma, é hirió al Filisteo,
y Jehová hizo una gran salud á todo Israel. Tú lo viste, y te holgaste:
¿por qué pues pecarás contra la sangre inocente, matando á David sin
causa?

\bibleverse{6} Y oyendo Saúl la voz de Jonathán, juró: Vive Jehová, que
no morirá.

\bibleverse{7} Llamando entonces Jonathán á David, declaróle todas estas
palabras; y él mismo presentó á David á Saúl, y estuvo delante de él
como antes.

\hypertarget{la-nueva-fortuna-de-david-en-la-guerra-el-repetido-intento-de-asesinato-de-sauxfal}{%
\subsection{La nueva fortuna de David en la guerra; El repetido intento
de asesinato de
Saúl}\label{la-nueva-fortuna-de-david-en-la-guerra-el-repetido-intento-de-asesinato-de-sauxfal}}

\bibleverse{8} Y tornó á hacerse guerra: y salió David y peleó contra
los Filisteos, é hiriólos con grande estrago, y huyeron delante de él.

\bibleverse{9} Y el espíritu malo de parte de Jehová fué sobre Saúl: y
estando sentado en su casa tenía una lanza á mano, mientras David estaba
tañendo con su mano. \bibleverse{10} Y Saúl procuró enclavar á David con
la lanza en la pared; mas él se apartó de delante de Saúl, el cual hirió
con la lanza en la pared; y David huyó, y escapóse aquella noche.

\hypertarget{el-escape-de-david-a-su-hogar-y-su-salvaciuxf3n-a-travuxe9s-de-la-astucia-de-michal}{%
\subsection{El escape de David a su hogar y su salvación a través de la
astucia de
Michal}\label{el-escape-de-david-a-su-hogar-y-su-salvaciuxf3n-a-travuxe9s-de-la-astucia-de-michal}}

\bibleverse{11} Saúl envió luego mensajeros á casa de David para que lo
guardasen, y lo matasen á la mañana. Mas Michâl su mujer lo descubrió á
David, diciendo: Si no salvares tu vida esta noche, mañana serás muerto.
\bibleverse{12} Y descolgó Michâl á David por una ventana; y él se fué,
y huyó, y escapóse. \bibleverse{13} Tomó luego Michâl una estatua, y
púsola sobre la cama, y acomodóle por cabecera una almohada de pelos de
cabra, y cubrióla con una ropa. \bibleverse{14} Y cuando Saúl envió
mensajeros que tomasen á David, ella respondió: Está enfermo.

\bibleverse{15} Y tornó Saúl á enviar mensajeros para que viesen á
David, diciendo: Traédmelo en la cama para que lo mate. \bibleverse{16}
Y como los mensajeros entraron, he aquí la estatua estaba en la cama, y
una almohada de pelos de cabra por cabecera.

\bibleverse{17} Entonces Saúl dijo á Michâl: ¿Por qué me has así
engañado, y has dejado escapar á mi enemigo? Y Michâl respondió á Saúl:
Porque él me dijo: Déjame ir; si no, yo te mataré.

\hypertarget{david-con-samuel-en-rama-el-rapto-profuxe9tico-de-sauxfal-en-la-casa-profuxe9tica-alluxed}{%
\subsection{David con Samuel en Rama; El rapto profético de Saúl en la
casa profética
allí}\label{david-con-samuel-en-rama-el-rapto-profuxe9tico-de-sauxfal-en-la-casa-profuxe9tica-alluxed}}

\bibleverse{18} Huyó pues David, y escapóse, y vino á Samuel en Rama, y
díjole todo lo que Saúl había hecho con él. Y fuéronse él y Samuel, y
moraron en Najoth. \bibleverse{19} Y fué dado aviso á Saúl, diciendo: He
aquí que David está en Najoth en Rama.

\bibleverse{20} Y envió Saúl mensajeros que trajesen á David, los cuales
vieron una compañía de profetas que profetizaban, y á Samuel que estaba
allí, y los presidía. Y fué el espíritu de Dios sobre los mensajeros de
Saúl, y ellos también profetizaron. \bibleverse{21} Y hecho que fué
saber á Saúl, él envió otros mensajeros, los cuales también
profetizaron. Y Saúl volvió á enviar por tercera vez mensajeros, y ellos
también profetizaron. \bibleverse{22} Entonces él mismo vino á Rama; y
llegando al pozo grande que está en Sochô, preguntó diciendo: ¿Dónde
están Samuel y David? Y fuéle respondido: He aquí están en Najoth en
Rama.

\bibleverse{23} Y fué allá á Najoth en Rama; y también vino sobre él el
espíritu de Dios, é iba profetizando, hasta que llegó á Najoth en Rama.
\bibleverse{24} Y él también se desnudó sus vestidos, y profetizó
igualmente delante de Samuel, y cayó desnudo todo aquel día y toda
aquella noche. De aquí se dijo: ¿También Saúl entre los profetas?

\hypertarget{la-reuniuxf3n-de-david-y-la-discusiuxf3n-con-jonatuxe1n-renovaciuxf3n-de-su-alianza-de-amistad}{%
\subsection{La reunión de David y la discusión con Jonatán; Renovación
de su alianza de
amistad}\label{la-reuniuxf3n-de-david-y-la-discusiuxf3n-con-jonatuxe1n-renovaciuxf3n-de-su-alianza-de-amistad}}

\hypertarget{section-09-20}{%
\section{20}\label{section-09-20}}

\bibleverse{1} Y David huyó de Najoth que es en Rama, y vínose delante
de Jonathán, y dijo: ¿Qué he hecho yo? ¿cuál es mi maldad, ó cuál mi
pecado contra tu padre, que él busca mi vida?

\bibleverse{2} Y él le dijo: En ninguna manera; no morirás. He aquí que
mi padre ninguna cosa hará, grande ni pequeña, que no me la descubra:
¿por qué pues me encubrirá mi padre este negocio? No será así.

\bibleverse{3} Y David volvió á jurar, diciendo: Tu padre sabe
claramente que yo he hallado gracia delante de tus ojos, y dirá: No sepa
esto Jonathán, porque no tenga pesar: y ciertamente, vive Jehová y vive
tu alma, que apenas hay un paso entre mí y la muerte.

\bibleverse{4} Y Jonathán dijo á David: ¿Qué discurre tu alma, y harélo
por ti?

\hypertarget{la-sugerencia-de-david}{%
\subsection{La sugerencia de David}\label{la-sugerencia-de-david}}

\bibleverse{5} Y David respondió á Jonathán: He aquí que mañana será
nueva luna, y yo acostumbro sentarme con el rey á comer: mas tú dejarás
que me esconda en el campo hasta la tarde del tercer día. \bibleverse{6}
Si tu padre hiciere mención de mí, dirás: Rogóme mucho que lo dejase ir
presto á Beth-lehem su ciudad, porque todos los de su linaje tienen allá
sacrificio aniversario. \bibleverse{7} Si él dijere, Bien está, paz
tendrá tu siervo; mas si se enojare, sabe que la malicia es en él
consumada. \bibleverse{8} Harás pues misericordia con tu siervo, ya que
has traído tu siervo á alianza de Jehová contigo: y si maldad hay en mí
mátame tú, que no hay necesidad de llevarme hasta tu padre.

\bibleverse{9} Y Jonathán le dijo: Nunca tal te suceda; antes bien, si
yo entendiera ser consumada la malicia de mi padre, para venir sobre ti,
¿no había yo de descubrírtelo?

\bibleverse{10} Dijo entonces David á Jonathán: ¿Quién me dará aviso? ó
¿qué si tu padre te respondiere ásperamente?

\bibleverse{11} Y Jonathán dijo á David: Ven, salgamos al campo. Y
salieron ambos al campo.

\hypertarget{el-juramento-mutuo}{%
\subsection{El juramento mutuo}\label{el-juramento-mutuo}}

\bibleverse{12} Entonces dijo Jonathán á David: Oh Jehová Dios de
Israel, cuando habré yo preguntado á mi padre mañana á esta hora, ó
después de mañana, y él apareciere bien para con David, si entonces no
enviare á ti, y te lo descubriere, \bibleverse{13} Jehová haga así á
Jonathán, y esto añada. Mas si á mi padre pareciere bien hacerte mal,
también te lo descubriré, y te enviaré, y te irás en paz: y sea Jehová
contigo, como fué con mi padre. \bibleverse{14} Y si yo viviere, harás
conmigo misericordia de Jehová; mas si fuere muerto, \bibleverse{15} No
quitarás perpetuamente tu misericordia de mi casa. Cuando desarraigare
Jehová uno por uno los enemigos de David de la tierra, aun á Jonathán
quite de su casa, si te faltare. \bibleverse{16} Así hizo Jonathán
alianza con la casa de David, diciendo: Requiéralo Jehová de la mano de
los enemigos de David.

\bibleverse{17} Y tornó Jonathán á jurar á David, porque le amaba,
porque le amaba como á su alma.

\hypertarget{acordar-el-procedimiento-a-seguir-para-la-comunicaciuxf3n-de-la-informaciuxf3n}{%
\subsection{Acordar el procedimiento a seguir para la comunicación de la
información}\label{acordar-el-procedimiento-a-seguir-para-la-comunicaciuxf3n-de-la-informaciuxf3n}}

\bibleverse{18} Díjole luego Jonathán: Mañana es nueva luna, y tú serás
echado de menos, porque tu asiento estará vacío. \bibleverse{19} Estarás
pues tres días, y luego descenderás, y vendrás al lugar donde estabas
escondido el día de trabajo, y esperarás junto á la piedra de Ezel;
\bibleverse{20} Y yo tiraré tres saetas hacia aquel lado, como
ejercitándome al blanco. \bibleverse{21} Y luego enviaré el criado
diciéndole: Ve, busca las saetas. Y si dijere al mozo: He allí las
saetas más acá de ti, tómalas: tú vendrás, porque paz tienes, y nada hay
de mal, vive Jehová. \bibleverse{22} Mas si yo dijere al mozo así: He
allí las saetas más allá de ti: vete, porque Jehová te ha enviado.
\bibleverse{23} Y cuanto á las palabras que yo y tú hemos hablado, sea
Jehová entre mí y ti para siempre.

\hypertarget{curso-de-las-dos-comidas-del-medioduxeda-en-casa-de-sauxfal-en-la-luna-nueva-y-al-duxeda-siguiente}{%
\subsection{Curso de las dos comidas del mediodía en casa de Saúl en la
luna nueva y al día
siguiente}\label{curso-de-las-dos-comidas-del-medioduxeda-en-casa-de-sauxfal-en-la-luna-nueva-y-al-duxeda-siguiente}}

\bibleverse{24} David pues se escondió en el campo, y venida que fué la
nueva luna, sentóse el rey á comer pan. \bibleverse{25} Y el rey se
sentó en su silla, como solía, en el asiento junto á la pared, y
Jonathán se levantó, y sentóse Abner al lado de Saúl, y el lugar de
David estaba vacío. \bibleverse{26} Mas aquel día Saúl no dijo nada,
porque se decía: Habrále acontecido algo, y no está limpio; no estará
purificado.

\bibleverse{27} El día siguiente, el segundo día de la nueva luna,
aconteció también que el asiento de David estaba vacío. Y Saúl dijo á
Jonathán su hijo: ¿Por qué no ha venido á comer el hijo de Isaí hoy ni
ayer?

\bibleverse{28} Y Jonathán respondió á Saúl: David me pidió
encarecidamente le dejase ir hasta Beth-lehem. \bibleverse{29} Y dijo:
Ruégote que me dejes ir, porque tenemos sacrificio los de nuestro linaje
en la ciudad, y mi hermano mismo me lo ha mandado; por tanto, si he
hallado gracia en tus ojos, haré una escapada ahora, y visitaré á mis
hermanos. Por esto pues no ha venido á la mesa del rey.

\bibleverse{30} Entonces Saúl se enardeció contra Jonathán, y díjole:
Hijo de la perversa y rebelde, ¿no sé yo que tú has elegido al hijo de
Isaí para confusión tuya, y para confusión de la vergüenza de tu madre?
\bibleverse{31} Porque todo el tiempo que el hijo de Isaí viviere sobre
la tierra, ni tú serás firme, ni tu reino. Envía pues ahora, y tráemelo,
porque ha de morir.

\bibleverse{32} Y Jonathán respondió á su padre Saúl, y díjole: ¿Por qué
morirá? ¿qué ha hecho?

\bibleverse{33} Entonces Saúl le arrojó una lanza por herirlo: de donde
entendió Jonathán que su padre estaba determinado á matar á David.
\bibleverse{34} Y levantóse Jonathán de la mesa con exaltada ira, y no
comió pan el segundo día de la nueva luna: porque tenía dolor á causa de
David, porque su padre le había afrentado.

\hypertarget{jonatuxe1n-informa-a-david-de-la-situaciuxf3n-desfavorable-y-se-despide-de-uxe9l}{%
\subsection{Jonatán informa a David de la situación desfavorable y se
despide de
él}\label{jonatuxe1n-informa-a-david-de-la-situaciuxf3n-desfavorable-y-se-despide-de-uxe9l}}

\bibleverse{35} Al otro día de mañana, salió Jonathán al campo, al
tiempo aplazado con David, y un mozo pequeño con él. \bibleverse{36} Y
dijo á su mozo: Corre y busca las saetas que yo tirare. Y como el
muchacho iba corriendo, él tiraba la saeta que pasara más allá de él.
\bibleverse{37} Y llegando el muchacho adonde estaba la saeta que
Jonathán había tirado, Jonathán dió voces tras el muchacho, diciendo:
¿No está la saeta más allá de ti? \bibleverse{38} Y tornó á gritar
Jonathán tras el muchacho: Date priesa, aligera, no te pares. Y el
muchacho de Jonathán cogió las saetas, y vínose á su señor.
\bibleverse{39} Empero ninguna cosa entendió el muchacho: solamente
Jonathán y David entendían el negocio. \bibleverse{40} Luego dió
Jonathán sus armas á su muchacho, y díjole: Vete y llévalas á la ciudad.

\bibleverse{41} Y luego que el muchacho se hubo ido, se levantó David de
la parte del mediodía, é inclinóse tres veces postrándose hasta la
tierra: y besándose el uno al otro, lloraron el uno con el otro, aunque
David lloró más. \bibleverse{42} Y Jonathán dijo á David: Vete en paz,
que ambos hemos jurado por el nombre de Jehová, diciendo: Jehová sea
entre mí y ti, entre mi simiente y la simiente tuya, para siempre. Y él
se levantó y fuése: y Jonathán se entró en la ciudad.

\hypertarget{david-como-refugiado-en-nob-y-gat-el-asesinato-del-sacerdote-por-parte-de-sauxfal}{%
\subsection{David como refugiado en Nob y Gat; El asesinato del
sacerdote por parte de
Saúl}\label{david-como-refugiado-en-nob-y-gat-el-asesinato-del-sacerdote-por-parte-de-sauxfal}}

\hypertarget{section-09-21}{%
\section{21}\label{section-09-21}}

\bibleverse{1} Y vino David á Nob, á Ahimelech sacerdote: y sorprendióse
Ahimelech de su encuentro, y díjole: ¿Cómo tú solo, y nadie contigo?
\bibleverse{2} Y respondió David al sacerdote Ahimelech: El rey me
encomendó un negocio, y me dijo: Nadie sepa cosa alguna de este negocio
á que yo te envío, y que yo te he mandado; y yo señalé á los criados un
cierto lugar. \bibleverse{3} Ahora pues, ¿qué tienes á mano? dame cinco
panes, ó lo que se hallare.

\bibleverse{4} Y el sacerdote respondió á David, y dijo: No tengo pan
común á la mano; solamente tengo pan sagrado: mas lo daré si los criados
se han guardado mayormente de mujeres.

\bibleverse{5} Y David respondió al sacerdote, y díjole: Cierto las
mujeres nos han sido reservadas desde anteayer cuando salí, y los vasos
de los mozos fueron santos, aunque el camino es profano: cuanto más que
hoy habrá otro pan santificado en los vasos. \bibleverse{6} Así el
sacerdote le dió el pan sagrado, porque allí no había otro pan que los
panes de la proposición, los cuales habían sido quitados de delante de
Jehová, para que se pusiesen panes calientes el día que los otros fueron
quitados.

\bibleverse{7} Aquel día estaba allí uno de los siervos de Saúl detenido
delante de Jehová, el nombre del cual era Doeg, Idumeo, principal de los
pastores de Saúl.

\bibleverse{8} Y David dijo á Ahimelech: ¿No tienes aquí á mano lanza ó
espada? porque no tomé en mi mano mi espada ni mis armas, por cuanto el
mandamiento del rey era apremiante.

\bibleverse{9} Y el sacerdote respondió: La espada de Goliath el
Filisteo, que tú venciste en el valle del Alcornoque, está aquí envuelta
en un velo detrás del ephod: si tú quieres tomarla, tómala: porque aquí
no hay otra sino esa. Y dijo David: Ninguna como ella: dámela.

\hypertarget{david-se-vuelve-loco-con-el-rey-achis-en-gat}{%
\subsection{David se vuelve loco con el rey Achis en
Gat}\label{david-se-vuelve-loco-con-el-rey-achis-en-gat}}

\bibleverse{10} Y levantándose David aquel día, huyó de la presencia de
Saúl, y vínose á Achîs rey de Gath. \bibleverse{11} Y los siervos de
Achîs le dijeron: ¿No es éste David, el rey de la tierra? ¿no es éste á
quien cantaban en corros, diciendo: Hirió Saúl sus miles, y David sus
diez miles?

\bibleverse{12} Y David puso en su corazón estas palabras, y tuvo gran
temor de Achîs rey de Gath. \bibleverse{13} Y mudó su habla delante de
ellos, y fingióse loco entre sus manos, y escribía en las portadas de
las puertas, dejando correr su saliva por su barba. \bibleverse{14} Y
dijo Achîs á sus siervos: He aquí estáis viendo un hombre demente; ¿por
qué lo habéis traído á mí? \bibleverse{15} ¿Fáltanme á mí locos, para
que hayáis traído éste que hiciese del loco delante de mí? ¿había de
venir éste á mi casa?

\hypertarget{la-posterior-huida-de-david-a-adullam-mizpe-en-moab-y-jaar-hereth-en-juduxe1-su-cuidado-por-sus-padres}{%
\subsection{La posterior huida de David a Adullam, Mizpe en Moab y
Jaar-Hereth en Judá; su cuidado por sus
padres}\label{la-posterior-huida-de-david-a-adullam-mizpe-en-moab-y-jaar-hereth-en-juduxe1-su-cuidado-por-sus-padres}}

\hypertarget{section-09-22}{%
\section{22}\label{section-09-22}}

\bibleverse{1} Y yéndose David de allí escapóse á la cueva de Adullam;
lo cual como oyeron sus hermanos y toda la casa de su padre, vinieron
allí á él. \bibleverse{2} Y juntáronse con él todos los afligidos, y
todo el que estaba adeudado, y todos los que se hallaban en amargura de
espíritu, y fué hecho capitán de ellos: y tuvo consigo como
cuatrocientos hombres. \bibleverse{3} Y fuése David de allí á Mizpa de
Moab, y dijo al rey de Moab: Yo te ruego que mi padre y mi madre estén
con vosotros, hasta que sepa lo que Dios hará de mí. \bibleverse{4}
Trájolos pues á la presencia del rey de Moab, y habitaron con él todo el
tiempo que David estuvo en la fortaleza. \bibleverse{5} Y Gad profeta
dijo á David: No te estés en esta fortaleza, pártete, y vete á tierra de
Judá. Y David se partió, y vino al bosque de Hareth.

\hypertarget{la-queja-de-sauxfal-a-los-que-lo-rodeaban-en-guibeuxe1-traiciuxf3n-del-edomita-doeg-la-sangrienta-venganza-de-sauxfal-contra-los-sacerdotes-de-nob}{%
\subsection{La queja de Saúl a los que lo rodeaban en Guibeá; Traición
del edomita Doeg; La sangrienta venganza de Saúl contra los sacerdotes
de
Nob}\label{la-queja-de-sauxfal-a-los-que-lo-rodeaban-en-guibeuxe1-traiciuxf3n-del-edomita-doeg-la-sangrienta-venganza-de-sauxfal-contra-los-sacerdotes-de-nob}}

\bibleverse{6} Y oyó Saúl como había parecido David, y los que estaban
con él. Estaba entonces Saúl en Gabaa debajo de un árbol en Rama, y
tenía su lanza en su mano, y todos sus criados estaban en derredor de
él. \bibleverse{7} Y dijo Saúl á sus criados que estaban en derredor de
él: Oid ahora, hijos de Benjamín: ¿Os dará también á todos vosotros el
hijo de Isaí tierras y viñas, y os hará á todos tribunos y centuriones;
\bibleverse{8} Que todos vosotros habéis conspirado contra mí, y no hay
quien me descubra al oído como mi hijo ha hecho alianza con el hijo de
Isaí, ni alguno de vosotros que se duela de mí, y me descubra como mi
hijo ha levantado á mi siervo contra mí, para que me aceche, según hace
hoy día?

\bibleverse{9} Entonces Doeg Idumeo, que era superior entre los siervos
de Saúl, respondió y dijo: Yo vi al hijo de Isaí que vino á Nob, á
Ahimelech hijo de Ahitob; \bibleverse{10} El cual consultó por él á
Jehová, y dióle provisión, y también le dió la espada de Goliath el
Filisteo.

\hypertarget{el-plato-de-sangre-en-guibeuxe1}{%
\subsection{El plato de sangre en
Guibeá}\label{el-plato-de-sangre-en-guibeuxe1}}

\bibleverse{11} Y el rey envió por el sacerdote Ahimelech hijo de
Ahitob, y por toda la casa de su padre, los sacerdotes que estaban en
Nob: y todos vinieron al rey. \bibleverse{12} Y Saúl le dijo: Oye ahora,
hijo de Ahitob. Y él dijo: Heme aquí, señor mío.

\bibleverse{13} Y díjole Saúl: ¿Por qué habéis conspirado contra mí, tú
y el hijo de Isaí, cuando tú le diste pan y espada, y consultaste por él
á Dios, para que se levantase contra mí y me acechase, como lo hace hoy
día?

\bibleverse{14} Entonces Ahimelech respondió al rey, y dijo: ¿Y quién
entre todos tus siervos es tan fiel como David, yerno además del rey, y
que va por tu mandado, y es ilustre en tu casa? \bibleverse{15} ¿He
comenzado yo desde hoy á consultar por él á Dios? lejos sea de mí: no
impute el rey cosa alguna á su siervo, ni á toda la casa de mi padre;
porque tu siervo ninguna cosa sabe de este negocio, grande ni chica.

\bibleverse{16} Y el rey dijo: Sin duda morirás, Ahimelech, tú y toda la
casa de tu padre. \bibleverse{17} Entonces dijo el rey á la gente de su
guardia que estaba alrededor de él: Cercad y matad á los sacerdotes de
Jehová; porque también la mano de ellos es con David, pues sabiendo
ellos que huía, no me lo descubrieron. Mas los siervos del rey no
quisieron extender sus manos para matar á los sacerdotes de Jehová.

\bibleverse{18} Entonces dijo el rey á Doeg: Vuelve tú, y arremete
contra los sacerdotes. Y revolviéndose Doeg Idumeo, arremetió contra los
sacerdotes, y mató en aquel día ochenta y cinco varones que vestían
ephod de lino.

\bibleverse{19} Y á Nob, ciudad de los sacerdotes, puso á cuchillo: así
á hombres como á mujeres, niños y mamantes, bueyes y asnos y ovejas,
todo á cuchillo.

\hypertarget{el-sacerdote-fugitivo-abjathar-encuentra-una-recepciuxf3n-amistosa-con-david}{%
\subsection{El sacerdote fugitivo Abjathar encuentra una recepción
amistosa con
David}\label{el-sacerdote-fugitivo-abjathar-encuentra-una-recepciuxf3n-amistosa-con-david}}

\bibleverse{20} Mas uno de los hijos de Ahimelech hijo de Ahitob, que se
llamaba Abiathar, escapó, y huyóse á David. \bibleverse{21} Y Abiathar
notició á David como Saúl había muerto los sacerdotes de Jehová.

\bibleverse{22} Y dijo David á Abiathar: Yo sabía que estando allí aquel
día Doeg el Idumeo, él lo había de hacer saber á Saúl. Yo he dado
ocasión contra todas las personas de la casa de tu padre.
\bibleverse{23} Quédate conmigo, no temas: quien buscare mi vida,
buscará también la tuya: bien que tú estarás conmigo guardado.

\hypertarget{david-en-el-desierto-de-juduxe1-en-kegila-y-maon-su-uxfaltimo-encuentro-con-jonathan-traiciuxf3n-de-los-sifitas}{%
\subsection{David en el desierto de Judá (en Kegila y Maon); su último
encuentro con Jonathan; Traición de los
sifitas}\label{david-en-el-desierto-de-juduxe1-en-kegila-y-maon-su-uxfaltimo-encuentro-con-jonathan-traiciuxf3n-de-los-sifitas}}

\hypertarget{section-09-23}{%
\section{23}\label{section-09-23}}

\bibleverse{1} Y dieron aviso á David, diciendo: He aquí que los
Filisteos combaten á Keila, y roban las eras.

\bibleverse{2} Y David consultó á Jehová, diciendo: ¿Iré á herir á estos
Filisteos? Y Jehová respondió á David: Ve, hiere á los Filisteos, y
libra á Keila.

\bibleverse{3} Mas los que estaban con David le dijeron: He aquí que
nosotros aquí en Judá estamos con miedo; ¿cuánto más si fuéremos á Keila
contra el ejército de los Filisteos?

\bibleverse{4} Entonces David volvió á consultar á Jehová. Y Jehová le
respondió, y dijo: Levántate, desciende á Keila, que yo entregaré en tus
manos á los Filisteos.

\bibleverse{5} Partióse pues David con sus hombres á Keila, y peleó
contra los Filisteos, y trajo antecogidos sus ganados, é hiriólos con
grande estrago: y libró David á los de Keila.

\bibleverse{6} Y aconteció que, huyendo Abiathar hijo de Ahimelech á
David á Keila, vino también con él el ephod.

\bibleverse{7} Y fué dicho á Saúl que David había venido á Keila.
Entonces dijo Saúl: Dios lo ha traído á mis manos; porque él está
encerrado, habiéndose metido en ciudad con puertas y cerraduras.
\bibleverse{8} Y convocó Saúl todo el pueblo á la batalla, para
descender á Keila, y poner cerco á David y á los suyos. \bibleverse{9}
Mas entendiendo David que Saúl ideaba el mal contra él, dijo á Abiathar
sacerdote: Trae el ephod. \bibleverse{10} Y dijo David: Jehová Dios de
Israel, tu siervo tiene entendido que Saúl trata de venir contra Keila,
á destruir la ciudad por causa mía. \bibleverse{11} ¿Me entregarán los
vecinos de Keila en sus manos? ¿descenderá Saúl, como tu siervo tiene
oído? Jehová Dios de Israel, ruégote que lo declares á tu siervo. Y
Jehová dijo: Sí, descenderá.

\bibleverse{12} Dijo luego David: ¿Me entregarán los vecinos de Keila á
mí y á mis hombres en manos de Saúl? Y Jehová respondió: Te entregarán.

\bibleverse{13} David entonces se levantó con sus hombres, que eran como
seiscientos, y saliéronse de Keila, y fuéronse de una parte á otra. Y
vino la nueva á Saúl de como David se había escapado de Keila; y dejó de
salir.

\hypertarget{david-perseguido-por-sauxfal-en-el-desierto-de-siph-su-entrevista-con-jonathan-en-horesa}{%
\subsection{David perseguido por Saúl en el desierto de Siph; su
entrevista con Jonathan en
Horesa}\label{david-perseguido-por-sauxfal-en-el-desierto-de-siph-su-entrevista-con-jonathan-en-horesa}}

\bibleverse{14} Y David se estaba en el desierto en peñas, y habitaba en
un monte en el desierto de Ziph; y buscábalo Saúl todos los días, mas
Dios no lo entregó en sus manos. \bibleverse{15} Viendo pues David que
Saúl había salido en busca de su alma, estábase él en el bosque en el
desierto de Ziph.

\bibleverse{16} Entonces se levantó Jonathán hijo de Saúl, y vino á
David en el bosque, y confortó su mano en Dios. \bibleverse{17} Y
díjole: No temas, que no te hallará la mano de Saúl mi padre, y tú
reinarás sobre Israel, y yo seré segundo después de ti; y aun Saúl mi
padre así lo sabe. \bibleverse{18} Y entrambos hicieron alianza delante
de Jehová: y David se quedó en el bosque, y Jonathán se volvió á su
casa.

\hypertarget{david-traicionado-por-los-sifitas-y-maravillosamente-salvado-de-sauxfal-en-el-desierto-de-mauxf3n}{%
\subsection{David traicionado por los sifitas y maravillosamente salvado
de Saúl en el desierto de
Maón}\label{david-traicionado-por-los-sifitas-y-maravillosamente-salvado-de-sauxfal-en-el-desierto-de-mauxf3n}}

\bibleverse{19} Y subieron los de Ziph á decir á Saúl en Gabaa: ¿No está
David escondido en nuestra tierra en las peñas del bosque, en el collado
de Hachîla que está á la mano derecha del desierto? \bibleverse{20} Por
tanto, rey, desciende ahora presto, según todo el deseo de tu alma, y
nosotros lo entregaremos en la mano del rey.

\bibleverse{21} Y Saúl dijo: Benditos seáis vosotros de Jehová, que
habéis tenido compasión de mí: \bibleverse{22} Id pues ahora, apercibid
aún, considerad y ved su lugar donde tiene el pie, y quién lo haya visto
allí; porque se me ha dicho que él es en gran manera astuto.
\bibleverse{23} Considerad pues, y ved todos los escondrijos donde se
oculta, y volved á mí con la certidumbre, y yo iré con vosotros: que si
él estuviere en la tierra, yo le buscaré entre todos los millares de
Judá.

\bibleverse{24} Y ellos se levantaron, y se fueron á Ziph delante de
Saúl. Mas David y su gente estaban en el desierto de Maón, en la llanura
que está á la diestra del desierto. \bibleverse{25} Y partióse Saúl con
su gente á buscarlo; pero fué dado aviso á David, y descendió á la peña,
y quedóse en el desierto de Maón. Lo cual como Saúl oyó, siguió á David
al desierto de Maón. \bibleverse{26} Y Saúl iba por el un lado del
monte, y David con los suyos por el otro lado del monte: y dábase priesa
David para ir delante de Saúl; mas Saúl y los suyos habían encerrado á
David y á su gente para tomarlos. \bibleverse{27} Entonces vino un
mensajero á Saúl, diciendo: Ven luego, porque los Filisteos han hecho
una irrupción en el país. \bibleverse{28} Volvióse por tanto Saúl de
perseguir á David, y partió contra los Filisteos. Por esta causa
pusieron á aquel lugar por nombre Sela-hammah-lecoth.

\hypertarget{la-generosidad-de-david-hacia-sauxfal-en-la-cueva-cerca-de-engedi}{%
\subsection{La generosidad de David hacia Saúl en la cueva cerca de
Engedi}\label{la-generosidad-de-david-hacia-sauxfal-en-la-cueva-cerca-de-engedi}}

\hypertarget{section-09-24}{%
\section{24}\label{section-09-24}}

\bibleverse{1} Entonces David subió de allí, y habitó en los parajes
fuertes en Engaddi. \bibleverse{2} Y como Saúl volvió de los Filisteos,
diéronle aviso diciendo: He aquí que David está en el desierto de
Engaddi. \bibleverse{3} Y tomando Saúl tres mil hombres escogidos de
todo Israel, fué en busca de David y de los suyos, por las cumbres de
los peñascos de las cabras monteses. \bibleverse{4} Y como llegó á una
majada de ovejas en el camino, donde había una cueva, entró Saúl en ella
á cubrir sus pies: y David y los suyos estaban á los lados de la cueva.
\bibleverse{5} Entonces los de David le dijeron: He aquí el día de que
te ha dicho Jehová: He aquí que entrego tu enemigo en tus manos, y harás
con él como te pareciere. Y levantóse David, y calladamente cortó la
orilla del manto de Saúl. \bibleverse{6} Después de lo cual el corazón
de David le golpeaba, porque había cortado la orilla del manto de Saúl.
\bibleverse{7} Y dijo á los suyos: Jehová me guarde de hacer tal cosa
contra mi señor, el ungido de Jehová, que yo extienda mi mano contra él;
porque es el ungido de Jehová. \bibleverse{8} Así quebrantó David á los
suyos con palabras, y no les permitió que se levantasen contra Saúl. Y
Saúl, saliendo de la cueva, fuése su camino.

\hypertarget{los-discursos-intercambiados-entre-sauxfal-y-david-su-despedida}{%
\subsection{Los discursos intercambiados entre Saúl y David; su
despedida}\label{los-discursos-intercambiados-entre-sauxfal-y-david-su-despedida}}

\bibleverse{9} También David se levantó después, y saliendo de la cueva
dió voces á las espaldas de Saúl, diciendo: ¡Mi señor el rey! Y como
Saúl miró atrás, David inclinó su rostro á tierra, é hizo reverencia.
\bibleverse{10} Y dijo David á Saúl: ¿Por qué oyes las palabras de los
que dicen: Mira que David procura tu mal? \bibleverse{11} He aquí han
visto hoy tus ojos como Jehová te ha puesto hoy en mis manos en la
cueva: y dijeron que te matase, mas te perdoné, porque dije: No
extenderé mi mano contra mi señor, porque ungido es de Jehová.
\bibleverse{12} Y mira, padre mío, mira aún la orilla de tu manto en mi
mano: porque yo corté la orilla de tu manto, y no te maté. Conoce pues,
y ve que no hay mal ni traición en mi mano, ni he pecado contra ti; con
todo, tú andas á caza de mi vida para quitármela. \bibleverse{13} Juzgue
Jehová entre mí y ti, y véngueme de ti Jehová: empero mi mano no será
contra ti. \bibleverse{14} Como dice el proverbio de los antiguos: De
los impíos saldrá la impiedad: así que mi mano no será contra ti.
\bibleverse{15} ¿Tras quién ha salido el rey de Israel? ¿á quién
persigues? ¿á un perro muerto? ¿á una pulga?

\bibleverse{16} Jehová pues será juez, y él juzgará entre mí y ti. El
vea, y sustente mi causa, y me defienda de tu mano. \bibleverse{17} Y
aconteció que, como David acabó de decir estas palabras á Saúl, Saúl
dijo: ¿No es esta la voz tuya, hijo mío David? Y alzando Saúl su voz
lloró. \bibleverse{18} Y dijo á David: Más justo eres tú que yo, que me
has pagado con bien, habiéndote yo pagado con mal. \bibleverse{19} Tú
has mostrado hoy que has hecho conmigo bien; pues no me has muerto,
habiéndome Jehová puesto en tus manos. \bibleverse{20} Porque ¿quién
hallará á su enemigo, y lo dejará ir sano y salvo? Jehová te pague con
bien por lo que en este día has hecho conmigo. \bibleverse{21} Y ahora,
como yo entiendo que tú has de reinar, y que el reino de Israel ha de
ser en tu mano firme y estable,

\bibleverse{22} Júrame pues ahora por Jehová, que no cortarás mi
simiente después de mí, ni raerás mi nombre de la casa de mi padre.
Entonces David juró á Saúl. Y fuése Saúl á su casa, y David y los suyos
se subieron al sitio fuerte.

\hypertarget{la-muerte-de-samuel-la-locura-de-nabal-david-y-abigail}{%
\subsection{La muerte de Samuel; La locura de Nabal; David y
Abigail}\label{la-muerte-de-samuel-la-locura-de-nabal-david-y-abigail}}

\hypertarget{section-09-25}{%
\section{25}\label{section-09-25}}

\bibleverse{1} Y murió Samuel, y juntóse todo Israel, y lo lloraron, y
lo sepultaron en su casa en Rama. Y levantóse David, y se fué al
desierto de Parán.

\hypertarget{el-comportamiento-necio-de-nabal-hacia-la-peticiuxf3n-de-david}{%
\subsection{El comportamiento necio de Nabal hacia la petición de
David}\label{el-comportamiento-necio-de-nabal-hacia-la-peticiuxf3n-de-david}}

\bibleverse{2} Y en Maón había un hombre que tenía su hacienda en el
Carmelo, el cual era muy rico, que tenía tres mil ovejas y mil cabras. Y
aconteció hallarse esquilando sus ovejas en el Carmelo. \bibleverse{3}
El nombre de aquel varón era Nabal, y el nombre de su mujer, Abigail. Y
era aquella mujer de buen entendimiento y de buena gracia; mas el hombre
era duro y de malos hechos; y era del linaje de Caleb. \bibleverse{4} Y
oyó David en el desierto que Nabal esquilaba sus ovejas. \bibleverse{5}
Entonces envió David diez criados, y díjoles: Subid al Carmelo, é id á
Nabal, y saludadle en mi nombre, \bibleverse{6} Y decidle así: Que vivas
y sea paz á ti, y paz á tu familia, y paz á todo cuanto tienes.
\bibleverse{7} Ha poco supe que tienes esquiladores. Ahora, á los
pastores tuyos que han estado con nosotros, nunca les hicimos fuerza, ni
les faltó algo en todo el tiempo que han estado en el Carmelo.
\bibleverse{8} Pregunta á tus criados, que ellos te lo dirán. Hallen por
tanto estos criados gracia en tus ojos, pues que venimos en buen día:
ruégote que des lo que tuvieres á mano á tus siervos, y á tu hijo David.

\bibleverse{9} Y como llegaron los criados de David, dijeron á Nabal
todas estas palabras en nombre de David, y callaron.

\bibleverse{10} Y Nabal respondió á los criados de David, y dijo: ¿Quién
es David? ¿y quién es el hijo de Isaí? Muchos siervos hay hoy que se
huyen de sus señores. \bibleverse{11} ¿He de tomar yo ahora mi pan, mi
agua, y mi víctima que he preparado para mis esquiladores, y la daré á
hombres que no sé de dónde son?

\bibleverse{12} Y tornándose los criados de David, volviéronse por su
camino, y vinieron y dijeron á David todas estas palabras.

\hypertarget{david-se-lanza-a-la-venganza-abigail-se-entera-de-la-erupciuxf3n-de-su-marido}{%
\subsection{David se lanza a la venganza; Abigail se entera de la
erupción de su
marido}\label{david-se-lanza-a-la-venganza-abigail-se-entera-de-la-erupciuxf3n-de-su-marido}}

\bibleverse{13} Entonces David dijo á sus hombres: Cíñase cada uno su
espada. Y ciñóse cada uno su espada: también David ciñó su espada; y
subieron tras David como cuatrocientos hombres, y dejaron doscientos con
el bagaje.

\bibleverse{14} Y uno de los criados dió aviso á Abigail mujer de Nabal,
diciendo: He aquí David envió mensajeros del desierto que saludasen á
nuestro amo, y él los ha zaherido. \bibleverse{15} Mas aquellos hombres
nos han sido muy buenos, y nunca nos han hecho fuerza, ni ninguna cosa
nos ha faltado en todo el tiempo que hemos conversado con ellos,
mientras hemos estado en el campo. \bibleverse{16} Hannos sido por muro
de día y de noche, todos los días que hemos estado con ellos apacentando
las ovejas. \bibleverse{17} Ahora pues, entiende y mira lo que has de
hacer, porque el mal está del todo resuelto contra nuestro amo y contra
toda su casa: pues él es un hombre tan malo, que no hay quien pueda
hablarle.

\hypertarget{abigail-usa-muxe9todos-inteligentes-para-evitar-que-david-tome-su-venganza}{%
\subsection{Abigail usa métodos inteligentes para evitar que David tome
su
venganza}\label{abigail-usa-muxe9todos-inteligentes-para-evitar-que-david-tome-su-venganza}}

\bibleverse{18} Entonces Abigail tomó luego doscientos panes, y dos
cueros de vino, y cinco ovejas guisadas, y cinco medidas de grano
tostado, y cien hilos de uvas pasas, y doscientos panes de higos secos,
y cargólo en asnos; \bibleverse{19} Y dijo á sus criados: Id delante de
mí, que yo os seguiré luego. Y nada declaró á su marido Nabal.
\bibleverse{20} Y sentándose sobre un asno descendió por una parte
secreta del monte, y he aquí David y los suyos que venían frente á ella,
y ella les fué al encuentro.

\bibleverse{21} Y David había dicho: Ciertamente en vano he guardado
todo lo que éste tiene en el desierto, sin que nada le haya faltado de
todo cuanto es suyo; y él me ha vuelto mal por bien. \bibleverse{22} Así
haga Dios, y así añada á los enemigos de David, que de aquí á mañana no
tengo de dejar de todo lo que fuere suyo ni aun meante á la pared.

\bibleverse{23} Y como Abigail vió á David, apeóse prestamente del asno,
y postrándose delante de David sobre su rostro, inclinóse á tierra;
\bibleverse{24} Y echóse á sus pies, y dijo: Señor mío, sobre mí sea el
pecado; mas ruégote hable tu sierva en tus oídos, y oye las palabras de
tu sierva. \bibleverse{25} No ponga ahora mi señor su corazón á aquel
hombre brusco, á Nabal; porque conforme á su nombre, así es. El se llama
Nabal, y la locura está con él: mas yo tu sierva no vi los criados de mi
señor, los cuales tú enviaste. \bibleverse{26} Ahora pues, señor mío,
vive Jehová y vive tu alma, que Jehová te ha estorbado que vinieses á
derramar sangre, y vengarte por tu propia mano. Sean pues como Nabal tus
enemigos, y todos los que procuran mal contra mi señor. \bibleverse{27}
Y ahora esta bendición que tu sierva ha traído á mi señor, dése á los
criados que siguen á mi señor. \bibleverse{28} Y yo te ruego que
perdones á tu sierva esta ofensa; pues Jehová de cierto hará casa firme
á mi señor, por cuanto mi señor hace las guerras de Jehová, y mal no se
ha hallado en ti en tus días. \bibleverse{29} Bien que alguien se haya
levantado á perseguirte y atentar á tu vida, con todo, el alma de mi
señor será ligada en el haz de los que viven con Jehová Dios tuyo, y él
arrojará el alma de tus enemigos como de en medio de la palma de una
honda. \bibleverse{30} Y acontecerá que cuando Jehová hiciere con mi
señor conforme á todo el bien que ha hablado de ti, y te mandare que
seas caudillo sobre Israel, \bibleverse{31} Entonces, señor mío, no te
será esto en tropiezo y turbación de corazón, el que hayas derramado
sangre sin causa, ó que mi señor se haya vengado por sí mismo. Guárdese
pues mi señor, y cuando Jehová hiciere bien á mi señor, acuérdate de tu
sierva.

\bibleverse{32} Y dijo David á Abigail: Bendito sea Jehová Dios de
Israel, que te envió para que hoy me encontrases; \bibleverse{33} Y
bendito sea tu razonamiento, y bendita tú, que me has estorbado hoy el
ir á derramar sangre, y á vengarme por mi propia mano: \bibleverse{34}
Porque, vive Jehová Dios de Israel que me ha defendido de hacerte mal,
que si no te hubieras dado priesa en venirme al encuentro, de aquí á
mañana no le quedara á Nabal meante á la pared.

\bibleverse{35} Y recibió David de su mano lo que le había traído, y
díjole: Sube en paz á tu casa, y mira que he oído tu voz, y tenídote
respeto.

\hypertarget{la-muerte-repentina-de-nabal-el-matrimonio-de-david-con-abigail-y-ahinoam}{%
\subsection{La muerte repentina de Nabal; El matrimonio de David con
Abigail (y
Ahinoam)}\label{la-muerte-repentina-de-nabal-el-matrimonio-de-david-con-abigail-y-ahinoam}}

\bibleverse{36} Y Abigail se vino á Nabal, y he aquí que él tenía
banquete en su casa como banquete de rey: y el corazón de Nabal estaba
alegre en él, y estaba muy borracho; por lo que ella no le declaró poco
ni mucho, hasta que vino el día siguiente. \bibleverse{37} Pero á la
mañana, cuando el vino había salido de Nabal, refirióle su mujer
aquestas cosas; y se le amorteció el corazón, y quedóse como una piedra.
\bibleverse{38} Y pasados diez días Jehová hirió á Nabal, y murió.
\bibleverse{39} Y luego que David oyó que Nabal era muerto, dijo:
Bendito sea Jehová que juzgó la causa de mi afrenta recibida de la mano
de Nabal, y ha preservado del mal á su siervo; y Jehová ha tornado la
malicia de Nabal sobre su propia cabeza. Después envió David á hablar á
Abigail, para tomarla por su mujer.

\bibleverse{40} Y los criados de David vinieron á Abigail en el Carmelo,
y hablaron con ella, diciendo: David nos ha enviado á ti, para tomarte
por su mujer.

\bibleverse{41} Y ella se levantó, é inclinó su rostro á tierra,
diciendo: He aquí tu sierva, para que sea sierva que lave los pies de
los siervos de mi señor. \bibleverse{42} Y levantándose luego Abigail
con cinco mozas que la seguían, montóse en un asno, y siguió los
mensajeros de David, y fué su mujer. \bibleverse{43} También tomó David
á Ahinoam de Jezreel, y ambas á dos fueron sus mujeres.

\bibleverse{44} Porque Saúl había dado su hija Michâl mujer de David, á
Palti hijo de Lais, que era de Gallim.

\hypertarget{la-renovada-generosidad-de-david-hacia-sauxfal-en-el-desierto-de-siph}{%
\subsection{La renovada generosidad de David hacia Saúl en el desierto
de
Siph}\label{la-renovada-generosidad-de-david-hacia-sauxfal-en-el-desierto-de-siph}}

\hypertarget{section-09-26}{%
\section{26}\label{section-09-26}}

\bibleverse{1} Y vinieron los Zipheos á Saúl en Gabaa, diciendo: ¿No
está David escondido en el collado de Hachîla delante del desierto?
\bibleverse{2} Saúl entonces se levantó, y descendió al desierto de
Ziph, llevando consigo tres mil hombres escogidos de Israel, para buscar
á David en el desierto de Ziph. \bibleverse{3} Y asentó Saúl el campo en
el collado de Hachîla, que está delante del desierto junto al camino. Y
estaba David en el desierto, y entendió que Saúl le seguía en el
desierto. \bibleverse{4} David por tanto envió espías, y entendió por
cierto que Saúl había venido. \bibleverse{5} Y levantóse David, y vino
al sitio donde Saúl había asentado el campo; y miró David el lugar donde
dormía Saúl, y Abner hijo de Ner, general de su ejército. Y estaba Saúl
durmiendo en la trinchera, y el pueblo por el campo en derredor de él.

\bibleverse{6} Entonces habló David, y requirió á Ahimelech Hetheo, y á
Abisai hijo de Sarvia, hermano de Joab, diciendo: ¿Quién descenderá
conmigo á Saúl al campo? Y dijo Abisai: Yo descenderé contigo.

\bibleverse{7} David pues y Abisai vinieron al pueblo de noche: y he
aquí Saúl que estaba tendido durmiendo en la trinchera, y su lanza
hincada en tierra á su cabecera; y Abner y el pueblo estaban alrededor
de él tendidos. \bibleverse{8} Entonces dijo Abisai á David: Hoy ha Dios
entregado á tu enemigo en tus manos: ahora pues, herirélo luego con la
lanza, cosiéndole con la tierra de un golpe, y no segundaré.

\bibleverse{9} Y David respondió á Abisai: No le mates: porque ¿quién
extenderá su mano contra el ungido de Jehová, y será inocente?
\bibleverse{10} Dijo además David: Vive Jehová, que si Jehová no lo
hiriere, ó que su día llegue para que muera, ó que descendiendo en
batalla perezca, \bibleverse{11} Guárdeme Jehová de extender mi mano
contra el ungido de Jehová; empero toma ahora la lanza que está á su
cabecera, y la botija del agua, y vámonos.

\bibleverse{12} Llevóse pues David la lanza y la botija de agua de la
cabecera de Saúl, y fuéronse; que no hubo nadie que viese, ni
entendiese, ni velase, pues todos dormían: porque un profundo sueño
enviado de Jehová había caído sobre ellos.

\hypertarget{la-aclamaciuxf3n-burlona-de-david-a-abner}{%
\subsection{La aclamación burlona de David a
Abner}\label{la-aclamaciuxf3n-burlona-de-david-a-abner}}

\bibleverse{13} Y pasando David de la otra parte, púsose desviado en la
cumbre del monte, habiendo grande distancia entre ellos; \bibleverse{14}
Y dió voces David al pueblo, y á Abner hijo de Ner, diciendo: ¿No
respondes, Abner? Entonces Abner respondió y dijo: ¿Quién eres tú que
das voces al rey?

\bibleverse{15} Y dijo David á Abner: ¿No eres varón tú? ¿y quién hay
como tú en Israel? ¿por qué pues no has guardado al rey tu señor? que ha
entrado uno del pueblo á matar á tu señor el rey. \bibleverse{16} Esto
que has hecho, no está bien. Vive Jehová, que sois dignos de muerte, que
no habéis guardado á vuestro señor, al ungido de Jehová. Mira pues ahora
dónde está la lanza del rey, y la botija del agua que estaba á su
cabecera.

\hypertarget{los-discursos-intercambiados-entre-sauxfal-y-david-la-divergencia-de-ambos}{%
\subsection{Los discursos intercambiados entre Saúl y David; la
divergencia de
ambos}\label{los-discursos-intercambiados-entre-sauxfal-y-david-la-divergencia-de-ambos}}

\bibleverse{17} Y conociendo Saúl la voz de David, dijo: ¿No es esta tu
voz, hijo mío David? Y David respondió: Mi voz es, rey señor mío.

\bibleverse{18} Y dijo: ¿Por qué persigue así mi señor á su siervo? ¿qué
he hecho? ¿qué mal hay en mi mano? \bibleverse{19} Ruego pues, que el
rey mi señor oiga ahora las palabras de su siervo. Si Jehová te incita
contra mí, acepte un sacrificio: mas si fueren hijos de hombres,
malditos ellos en presencia de Jehová, que me han echado hoy para que no
me junte en la heredad de Jehová, diciendo: Ve y sirve á dioses ajenos.
\bibleverse{20} No caiga pues ahora mi sangre en tierra delante de
Jehová: porque ha salido el rey de Israel á buscar una pulga, así como
quien persigue una perdiz por los montes.

\bibleverse{21} Entonces dijo Saúl: He pecado: vuélvete, hijo mío David,
que ningún mal te haré más, pues que mi vida ha sido estimada hoy en tus
ojos. He aquí, yo he hecho neciamente, y he errado en gran manera.

\bibleverse{22} Y David respondió, y dijo: He aquí la lanza del rey;
pase acá uno de los criados, y tómela. \bibleverse{23} Y Jehová pague á
cada uno su justicia y su lealtad: que Jehová te había entregado hoy en
mi mano, mas yo no quise extender mi mano sobre el ungido de Jehová.
\bibleverse{24} Y he aquí, como tu vida ha sido estimada hoy en mis
ojos, así sea mi vida estimada en los ojos de Jehová, y me libre de toda
aflicción.

\bibleverse{25} Y Saúl dijo á David: Bendito eres tú, hijo mío David;
sin duda ejecutarás tú grandes empresas, y prevalecerás. Entonces David
se fué su camino, y Saúl se volvió á su lugar.

\hypertarget{la-conversiuxf3n-de-david-a-los-filisteos-su-estancia-con-el-pruxedncipe-filisteo-achis-en-gat-y-en-siclag}{%
\subsection{La conversión de David a los filisteos; su estancia con el
príncipe filisteo Achis en Gat y en
Siclag}\label{la-conversiuxf3n-de-david-a-los-filisteos-su-estancia-con-el-pruxedncipe-filisteo-achis-en-gat-y-en-siclag}}

\hypertarget{section-09-27}{%
\section{27}\label{section-09-27}}

\bibleverse{1} Y dijo David en su corazón: Al fin seré muerto algún día
por la mano de Saúl: nada por tanto me será mejor que fugarme á la
tierra de los Filisteos, para que Saúl se deje de mí, y no me ande
buscando más por todos los términos de Israel, y así me escaparé de sus
manos. \bibleverse{2} Levantóse pues David, y con los seiscientos
hombres que tenía consigo pasóse á Achîs hijo de Maoch, rey de Gath.
\bibleverse{3} Y moró David con Achîs en Gath, él y los suyos, cada uno
con su familia: David con sus dos mujeres, Ahinoam Jezreelita, y
Abigail, la que fué mujer de Nabal el del Carmelo. \bibleverse{4} Y vino
la nueva á Saúl que David se había huído á Gath, y no lo buscó más.

\bibleverse{5} Y David dijo á Achîs: Si he hallado ahora gracia en tus
ojos, séame dado lugar en algunas de las ciudades de la tierra, donde
habite: porque ¿ha de morar tu siervo contigo en la ciudad real?
\bibleverse{6} Y Achîs le dió aquel día á Siclag. De aquí fué Siclag de
los reyes de Judá hasta hoy. \bibleverse{7} Y fué el número de los días
que David habitó en la tierra de los Filisteos, cuatro meses y algunos
días.

\hypertarget{la-vida-privada-de-david-su-engauxf1o-a-los-filisteos}{%
\subsection{La vida privada de David; su engaño a los
filisteos}\label{la-vida-privada-de-david-su-engauxf1o-a-los-filisteos}}

\bibleverse{8} Y subía David con los suyos, y hacían entradas en los
Gesureos, y en los Gerzeos, y en los Amalecitas: porque estos habitaban
de largo tiempo la tierra, desde como se va á Shur hasta la tierra de
Egipto. \bibleverse{9} Y hería David el país, y no dejaba á vida hombre
ni mujer: y llevábase las ovejas y las vacas y los asnos y los camellos
y las ropas; y volvía, y veníase á Achîs.

\bibleverse{10} Y decía Achîs: ¿Dónde habéis corrido hoy? Y David decía:
Al mediodía de Judá, y al mediodía de Jerameel, ó contra el mediodía de
Ceni.

\bibleverse{11} Ni hombre ni mujer dejaba á vida David, que viniese á
Gath; diciendo: Porque no den aviso de nosotros, diciendo, Esto hizo
David. Y esta era su costumbre todo el tiempo que moró en tierra de los
Filisteos.

\bibleverse{12} Y Achîs creía á David, diciendo así: El se hace
abominable en su pueblo de Israel, y será siempre mi siervo.

\hypertarget{la-guerra-con-los-filisteos-saul-con-el-nigromante-en-endor}{%
\subsection{La guerra con los filisteos; Saul con el nigromante en
Endor}\label{la-guerra-con-los-filisteos-saul-con-el-nigromante-en-endor}}

\hypertarget{section-09-28}{%
\section{28}\label{section-09-28}}

\bibleverse{1} Y aconteció que en aquellos días los Filisteos juntaron
sus campos para pelear contra Israel. Y dijo Achîs á David: Sabe de
cierto que has de salir conmigo á campaña, tú y los tuyos.

\bibleverse{2} Y David respondió á Achîs: Sabrás pues lo que hará tu
siervo. Y Achîs dijo á David: Por tanto te haré guarda de mi cabeza
todos los días.

\hypertarget{comienzo-de-la-guerra-en-su-perplejidad-sauxfal-decide-cuestionar-un-oruxe1culo-de-los-muertos}{%
\subsection{Comienzo de la guerra; En su perplejidad, Saúl decide
cuestionar un oráculo de los
muertos}\label{comienzo-de-la-guerra-en-su-perplejidad-sauxfal-decide-cuestionar-un-oruxe1culo-de-los-muertos}}

\bibleverse{3} Ya Samuel era muerto, y todo Israel lo había lamentado, y
habíanle sepultado en Rama, en su ciudad. Y Saúl había echado de la
tierra los encantadores y adivinos.

\bibleverse{4} Pues como los Filisteos se juntaron, vinieron y asentaron
campo en Sunam: y Saúl juntó á todo Israel, y asentaron campo en Gilboa.
\bibleverse{5} Y cuando vió Saúl el campo de los Filisteos, temió, y
turbóse su corazón en gran manera. \bibleverse{6} Y consultó Saúl á
Jehová; pero Jehová no le respondió, ni por sueños, ni por Urim, ni por
profetas. \bibleverse{7} Entonces Saúl dijo á sus criados: Buscadme una
mujer que tenga espíritu de pythón, para que yo vaya á ella, y por medio
de ella pregunte. Y sus criados le respondieron: He aquí hay una mujer
en Endor que tiene espíritu de pythón.

\hypertarget{saul-con-el-nigromante-en-endor-la-apariciuxf3n-y-profecuxeda-de-la-desgracia-del-espuxedritu-de-samuel}{%
\subsection{Saul con el nigromante en Endor; la aparición y profecía de
la desgracia del espíritu de
Samuel}\label{saul-con-el-nigromante-en-endor-la-apariciuxf3n-y-profecuxeda-de-la-desgracia-del-espuxedritu-de-samuel}}

\bibleverse{8} Y disfrazóse Saúl, y púsose otros vestidos, y fuése con
dos hombres, y vinieron á aquella mujer de noche; y él dijo: Yo te ruego
que me adivines por el espíritu de pythón, y me hagas subir á quien yo
te dijere.

\bibleverse{9} Y la mujer le dijo: He aquí tú sabes lo que Saúl ha
hecho, cómo ha separado de la tierra los pythones y los adivinos: ¿por
qué pues pones tropiezo á mi vida, para hacerme matar?

\bibleverse{10} Entoces Saúl le juró por Jehová, diciendo: Vive Jehová,
que ningún mal te vendrá por esto.

\bibleverse{11} La mujer entonces dijo: ¿A quién te haré venir? Y él
respondió: Hazme venir á Samuel.

\bibleverse{12} Y viendo la mujer á Samuel, clamó en alta voz, y habló
aquella mujer á Saúl, diciendo:

\bibleverse{13} ¿Por qué me has engañado? que tú eres Saúl. Y el rey le
dijo: No temas: ¿qué has visto? Y la mujer respondió á Saúl: He visto
dioses que suben de la tierra.

\bibleverse{14} Y él le dijo: ¿Cuál es su forma? Y ella respondió: Un
hombre anciano viene, cubierto de un manto. Saúl entonces entendió que
era Samuel, y humillando el rostro á tierra, hizo gran reverencia.

\bibleverse{15} Y Samuel dijo á Saúl: ¿Por qué me has inquietado
haciéndome venir? Y Saúl respondió: Estoy muy congojado; pues los
Filisteos pelean contra mí, y Dios se ha apartado de mí, y no me
responde más, ni por mano de profetas, ni por sueños: por esto te he
llamado, para que me declares qué tengo de hacer.

\bibleverse{16} Entonces Samuel dijo: ¿Y para qué me preguntas á mí,
habiéndose apartado de ti Jehová, y es tu enemigo? \bibleverse{17}
Jehová pues ha hecho como habló por medio de mí; pues ha cortado Jehová
el reino de tu mano, y lo ha dado á tu compañero David. \bibleverse{18}
Como tú no obedeciste á la voz de Jehová, ni cumpliste el furor de su
ira sobre Amalec, por eso Jehová te ha hecho esto hoy. \bibleverse{19} Y
Jehová entregará á Israel también contigo en manos de los Filisteos: y
mañana seréis conmigo, tú y tus hijos: y aun el campo de Israel
entregará Jehová en manos de los Filisteos.

\hypertarget{efecto-de-la-profecuxeda-sobre-saulo}{%
\subsection{Efecto de la profecía sobre
Saulo}\label{efecto-de-la-profecuxeda-sobre-saulo}}

\bibleverse{20} En aquel punto cayó Saúl en tierra cuan grande era, y
tuvo gran temor por las palabras de Samuel; que no quedó en él esfuerzo
ninguno, porque en todo aquel día y aquella noche no había comido pan.

\bibleverse{21} Entonces la mujer vino á Saúl, y viéndole en grande
manera turbado, díjole: He aquí que tu criada ha obedecido á tu voz, y
he puesto mi vida en mi mano, y he oído las palabras que tú me has
dicho. \bibleverse{22} Ruégote pues, que tú también oigas la voz de tu
sierva: pondré yo delante de ti un bocado de pan que comas, para que te
corrobores, y vayas tu camino.

\bibleverse{23} Y él lo rehusó, diciendo: No comeré. Mas sus criados
juntamente con la mujer le constriñeron, y él los obedeció. Levantóse
pues del suelo, y sentóse sobre una cama. \bibleverse{24} Y aquella
mujer tenía en su casa un ternero grueso, el cual mató luego; y tomó
harina y amasóla, y coció de ella panes sin levadura. \bibleverse{25} Y
lo trajo delante de Saúl y de sus criados; y luego que hubieron comido,
se levantaron, y partieron aquella noche.

\hypertarget{el-envuxedo-de-david-a-casa-a-instancias-de-los-sospechosos-pruxedncipes-filisteos}{%
\subsection{El envío de David a casa a instancias de los sospechosos
príncipes
filisteos}\label{el-envuxedo-de-david-a-casa-a-instancias-de-los-sospechosos-pruxedncipes-filisteos}}

\hypertarget{section-09-29}{%
\section{29}\label{section-09-29}}

\bibleverse{1} Y los Filisteos juntaron todos sus campos en Aphec; é
Israel puso su campo junto á la fuente que está en Jezreel.
\bibleverse{2} Y reconociendo los príncipes de los Filisteos sus
compañías de á ciento y de á mil hombres, David y los suyos iban en los
postreros con Achîs.

\bibleverse{3} Y dijeron los príncipes de los Filisteos: ¿Qué hacen aquí
estos Hebreos? Y Achîs respondió á los príncipes de los Filisteos: ¿No
es éste David, el siervo de Saúl rey de Israel, que ha estado conmigo
algunos días ó algunos años, y no he hallado cosa en él desde el día que
se pasó á mí hasta hoy?

\bibleverse{4} Entonces los príncipes de los Filisteos se enojaron
contra él, y dijéronle: Envía á este hombre, que se vuelva al lugar que
le señalaste, y no venga con nosotros á la batalla, no sea que en la
batalla se nos vuelva enemigo: porque ¿con qué cosa volvería mejor á la
gracia de su señor que con las cabezas de estos hombres? \bibleverse{5}
¿No es este David de quien cantaban en los corros, diciendo: Saúl hirió
sus miles, y David sus diez miles?

\bibleverse{6} Y Achîs llamó á David, y díjole: Vive Jehová, que tú has
sido recto, y que me ha parecido bien tu salida y entrada en el campo
conmigo, y que ninguna cosa mala he hallado en ti desde el día que
viniste á mí hasta hoy: mas en los ojos de los príncipes no agradas.
\bibleverse{7} Vuélvete pues, y vete en paz; y no hagas lo malo en los
ojos de los príncipes de los Filisteos.

\bibleverse{8} Y David respondió á Achîs: ¿Qué he hecho? ¿qué has
hallado en tu siervo desde el día que estoy contigo hasta hoy, para que
yo no vaya y pelee contra los enemigos de mi señor el rey?

\bibleverse{9} Y Achîs respondió á David, y dijo: Yo sé que tú eres
bueno en mis ojos, como un ángel de Dios; mas los príncipes de los
Filisteos han dicho: No venga con nosotros á la batalla. \bibleverse{10}
Levántate pues de mañana, tú y los siervos de tu señor que han venido
contigo; y levantándoos de mañana, luego al amanecer partíos.

\bibleverse{11} Y levantóse David de mañana, él y los suyos, para irse y
volverse á la tierra de los Filisteos; y los Filisteos fueron á Jezreel.

\hypertarget{david-encuentra-siclag-devastada-por-los-amalecitas-su-consternaciuxf3n-y-aliento}{%
\subsection{David encuentra Siclag devastada por los amalecitas; su
consternación y
aliento}\label{david-encuentra-siclag-devastada-por-los-amalecitas-su-consternaciuxf3n-y-aliento}}

\hypertarget{section-09-30}{%
\section{30}\label{section-09-30}}

\bibleverse{1} Y cuando David y los suyos vinieron á Siclag al tercer
día, los de Amalec habían invadido el mediodía y á Siclag, y habían
desolado á Siclag, y puéstola á fuego. \bibleverse{2} Y habíanse llevado
cautivas á las mujeres que estaban en ella, desde el menor hasta el
mayor; mas á nadie habían muerto, sino llevado, é ídose su camino.
\bibleverse{3} Vino pues David con los suyos á la ciudad, y he aquí que
estaba quemada á fuego, y sus mujeres y sus hijos é hijas llevadas
cautivas. \bibleverse{4} Entonces David y la gente que con él estaba,
alzaron su voz y lloraron, hasta que les faltaron las fuerzas para
llorar. \bibleverse{5} Las dos mujeres de David, Ahinoam Jezreelita y
Abigail la que fué mujer de Nabal del Carmelo, también eran cautivas.
\bibleverse{6} Y David fué muy angustiado, porque el pueblo hablaba de
apedrearlo; porque todo el pueblo estaba con ánimo amargo, cada uno por
sus hijos y por sus hijas: mas David se esforzó en Jehová su Dios.
\bibleverse{7} Y dijo David al sacerdote Abiathar hijo de Ahimelech: Yo
te ruego que me acerques el ephod. Y Abiathar acercó el ephod á David.

\bibleverse{8} Y David consultó á Jehová, diciendo: ¿Seguiré esta tropa?
¿podréla alcanzar? Y él le dijo: Síguela, que de cierto la alcanzarás, y
sin falta librarás la presa.

\bibleverse{9} Partióse pues David, él y los seiscientos hombres que con
él estaban, y vinieron hasta el torrente de Besor, donde se quedaron
algunos. \bibleverse{10} Y David siguió el alcance con cuatrocientos
hombres; porque se quedaron atrás doscientos, que cansados no pudieron
pasar el torrente de Besor.

\hypertarget{la-persecuciuxf3n-y-destrucciuxf3n-de-david-de-la-banda-de-ladrones-de-amalecita}{%
\subsection{La persecución y destrucción de David de la banda de
ladrones de
Amalecita}\label{la-persecuciuxf3n-y-destrucciuxf3n-de-david-de-la-banda-de-ladrones-de-amalecita}}

\bibleverse{11} Y hallaron en el campo un hombre Egipcio, el cual
trajeron á David, y diéronle pan que comiese, y á beber agua;
\bibleverse{12} Diéronle también un pedazo de masa de higos secos, y dos
hilos de pasas. Y luego que comió, volvió en él su espíritu; porque no
había comido pan ni bebido agua en tres días y tres noches.
\bibleverse{13} Y díjole David: ¿De quién eres tú? ¿y de dónde eres? Y
respondió el mozo Egipcio: Yo soy siervo de un Amalecita, y dejóme mi
amo hoy ha tres días, porque estaba enfermo;

\bibleverse{14} Pues hicimos una incursión á la parte del mediodía de
Cerethi, y á Judá, y al mediodía de Caleb; y pusimos fuego á Siclag.

\bibleverse{15} Y díjole David: ¿Me llevarás tú á esa tropa? Y él dijo:
Hazme juramento por Dios que no me matarás, ni me entregarás en las
manos de mi amo, y yo te llevaré á esa gente.

\bibleverse{16} Llevólo pues: y he aquí que estaban derramados sobre la
haz de toda aquella tierra, comiendo y bebiendo y haciendo fiesta, por
toda aquella gran presa que habían tomado de la tierra de los Filisteos,
y de la tierra de Judá. \bibleverse{17} E hiriólos David desde aquella
mañana hasta la tarde del día siguiente: y no escapó de ellos ninguno,
sino cuatrocientos mancebos, que habían subido en camellos y huyeron.
\bibleverse{18} Y libró David todo lo que los Amalecitas habían tomado:
y asimismo libertó David á sus dos mujeres. \bibleverse{19} Y no les
faltó cosa chica ni grande, así de hijos como de hijas, del robo, y de
todas las cosas que les habían tomado: todo lo recobró David.
\bibleverse{20} Tomó también David todas las ovejas y ganados mayores; y
trayéndolo todo delante, decían: Esta es la presa de David.

\hypertarget{david-hace-que-su-pueblo-lleve-ante-la-justicia-a-sus-camaradas}{%
\subsection{David hace que su pueblo lleve ante la justicia a sus
camaradas}\label{david-hace-que-su-pueblo-lleve-ante-la-justicia-a-sus-camaradas}}

\bibleverse{21} Y vino David á los doscientos hombres que habían quedado
cansados y no habían podido seguir á David, á los cuales habían hecho
quedar en el torrente de Besor; y ellos salieron á recibir á David, y al
pueblo que con él estaba. Y como David llegó á la gente, saludóles con
paz. \bibleverse{22} Entonces todos los malos y perversos de entre los
que habían ido con David, respondieron y dijeron: Pues que no fueron con
nosotros, no les daremos de la presa que hemos quitado, sino á cada uno
su mujer y sus hijos; los cuales tomen y se vayan.

\bibleverse{23} Y David dijo: No hagáis eso, hermanos míos, de lo que
nos ha dado Jehová; el cual nos ha guardado, y ha entregado en nuestras
manos la caterva que vino sobre nosotros. \bibleverse{24} ¿Y quién os
escuchará en este caso? porque igual parte ha de ser la de los que
vienen á la batalla, y la de los que quedan con el bagaje: que partan
juntamente. \bibleverse{25} Y desde aquel día en adelante fué esto
puesto por ley y ordenanza en Israel, hasta hoy.

\hypertarget{david-envuxeda-regalos-a-los-ancianos-en-numerosas-ciudades-de-juduxe1}{%
\subsection{David envía regalos a los ancianos en numerosas ciudades de
Judá}\label{david-envuxeda-regalos-a-los-ancianos-en-numerosas-ciudades-de-juduxe1}}

\bibleverse{26} Y como David llegó á Siclag, envió de la presa á los
ancianos de Judá, sus amigos, diciendo: He aquí una bendición para
vosotros, de la presa de los enemigos de Jehová. \bibleverse{27} A los
que estaban en Beth-el, y en Ramoth al mediodía, y á los que estaban en
Jattir; \bibleverse{28} Y á los que estaban en Aroer, y en Siphmoth, y á
los que estaban en Esthemoa; \bibleverse{29} Y á los que estaban en
Rachâl, y á los que estaban en las ciudades de Jerameel, y á los que
estaban en las ciudades del Cineo; \bibleverse{30} Y á los que estaban
en Horma, y á los que estaban en Chôrasán, y á los que estaban en
Athach; \bibleverse{31} Y á los que estaban en Hebrón, y en todos los
lugares donde David había estado con los suyos.

\hypertarget{la-derrota-de-israel-y-el-desastre-de-sauxfal-y-su-casa}{%
\subsection{La derrota de Israel y el desastre de Saúl y su
casa}\label{la-derrota-de-israel-y-el-desastre-de-sauxfal-y-su-casa}}

\hypertarget{section-09-31}{%
\section{31}\label{section-09-31}}

\bibleverse{1} Los Filisteos pues pelearon con Israel, y los de Israel
huyeron delante de los Filisteos, y cayeron muertos en el monte de
Gilboa. \bibleverse{2} Y siguiendo los Filisteos á Saúl y á sus hijos,
mataron á Jonathán, y á Abinadab, y á Melchîsua, hijos de Saúl.
\bibleverse{3} Y agravóse la batalla sobre Saúl, y le alcanzaron los
flecheros; y tuvo gran temor de los flecheros. \bibleverse{4} Entonces
dijo Saúl á su escudero: Saca tu espada, y pásame con ella, porque no
vengan estos incircuncisos, y me pasen, y me escarnezcan. Mas su
escudero no quería, porque tenía gran temor. Entonces tomó Saúl la
espada, y echóse sobre ella. \bibleverse{5} Y viendo su escudero á Saúl
muerto, él también se echó sobre su espada, y murió con él.
\bibleverse{6} Así murió Saúl en aquel día, juntamente con sus tres
hijos, y su escudero, y todos sus varones.

\bibleverse{7} Y los de Israel que eran de la otra parte del valle, y de
la otra parte del Jordán, viendo que Israel había huído, y que Saúl y
sus hijos eran muertos, dejaron las ciudades y huyeron; y los Filisteos
vinieron y habitaron en ellas.

\hypertarget{el-destino-de-los-caduxe1veres-de-sauxfal-y-sus-hijos}{%
\subsection{El destino de los cadáveres de Saúl y sus
hijos}\label{el-destino-de-los-caduxe1veres-de-sauxfal-y-sus-hijos}}

\bibleverse{8} Y aconteció el siguiente día, que viniendo los Filisteos
á despojar los muertos, hallaron á Saúl y á sus tres hijos tendidos en
el monte de Gilboa; \bibleverse{9} Y cortáronle la cabeza, y
desnudáronle las armas; y enviaron á tierra de los Filisteos al
contorno, para que lo noticiaran en el templo de sus ídolos, y por el
pueblo. \bibleverse{10} Y pusieron sus armas en el templo de Astaroth, y
colgaron su cuerpo en el muro de Beth-san. \bibleverse{11} Mas oyendo
los de Jabes de Galaad esto que los Filisteos hicieron á Saúl,
\bibleverse{12} Todos los hombres valientes se levantaron, y anduvieron
toda aquella noche, y quitaron el cuerpo de Saúl y los cuerpos de sus
hijos del muro de Beth-san; y viniendo á Jabes, quemáronlos allí.
\bibleverse{13} Y tomando sus huesos, sepultáronlos debajo de un árbol
en Jabes, y ayunaron siete días.
