\hypertarget{section}{%
\section{1}\label{section}}

\bibverse{1} Paulus, ein Apostel Jesu Christi durch den Willen Gottes
nach der Verheißung des Lebens in Christo Jesu, \bibverse{2} meinem
lieben Sohn Timotheus: Gnade, Barmherzigkeit, Friede von Gott, dem
Vater, und Christo Jesu, unserem Herrn!

\bibverse{3} Ich danke Gott, dem ich diene von meinen Voreltern her in
reinem Gewissen, dass ich ohne Unterlass dein gedenke in meinem Gebet
Tag und Nacht; \footnote{\textbf{1:3} Apg 23,1; Apg 24,16; Phil 3,5}
\bibverse{4} und mich verlangt, dich zu sehen, wenn ich denke an deine
Tränen, auf dass ich mit Freude erfüllt würde; \footnote{\textbf{1:4}
  2Tim 4,9} \bibverse{5} und wenn ich mich erinnere des ungefärbten
Glaubens in dir, welcher zuvor gewohnt hat in deiner Großmutter Lois und
in deiner Mutter Eunike; ich bin aber gewiss, auch in dir. \footnote{\textbf{1:5}
  Apg 16,1-3}

\bibverse{6} Um solcher Ursache willen erinnere ich dich, dass du
erweckest die Gabe Gottes, die in dir ist durch die Auflegung meiner
Hände. \footnote{\textbf{1:6} 1Tim 4,14} \bibverse{7} Denn Gott hat uns
nicht gegeben den Geist der Furcht, sondern der Kraft und der Liebe und
der Zucht. \footnote{\textbf{1:7} Röm 8,15} \bibverse{8} Darum so schäme
dich nicht des Zeugnisses unseres Herrn noch meiner, der ich sein
Gebundener bin, sondern leide mit für das Evangelium wie ich, nach der
Kraft Gottes, \footnote{\textbf{1:8} Röm 1,16} \bibverse{9} der uns hat
selig gemacht und berufen mit einem heiligen Ruf, nicht nach unseren
Werken, sondern nach seinem Vorsatz und der Gnade, die uns gegeben ist
in Christo Jesu vor der Zeit der Welt, \footnote{\textbf{1:9} Tit 3,5}
\bibverse{10} jetzt aber offenbart durch die Erscheinung unseres
Heilandes Jesu Christi, der dem Tode die Macht hat genommen und das
Leben und ein unvergänglich Wesen ans Licht gebracht durch das
Evangelium, \footnote{\textbf{1:10} 1Kor 15,55; 1Kor 15,57; Hebr 2,14}
\bibverse{11} für welches ich gesetzt bin als Prediger und Apostel und
Lehrer der Heiden. \footnote{\textbf{1:11} 1Tim 2,7} \bibverse{12} Um
dieser Ursache willen leide ich auch solches; aber ich schäme mich
dessen nicht; denn ich weiß, an wen ich glaube, und bin gewiss, er kann
mir bewahren, was mir beigelegt ist, bis an jenen Tag.

\bibverse{13} Halte an dem Vorbilde der heilsamen Worte, die du von mir
gehört hast, im Glauben und in der Liebe in Christo Jesu.

\bibverse{14} Dies beigelegte Gut bewahre durch den heiligen Geist, der
in uns wohnt. \footnote{\textbf{1:14} 1Tim 6,20}

\bibverse{15} Das weißt du, dass sich von mir gewandt haben alle, die in
Asien sind, unter welchen ist Phygellus und Hermogenes. \footnote{\textbf{1:15}
  2Tim 4,16} \bibverse{16} Der Herr gebe Barmherzigkeit dem Hause des
Onesiphorus; denn er hat mich oft erquickt und hat sich meiner Kette
nicht geschämt, \footnote{\textbf{1:16} 2Tim 4,20} \bibverse{17} sondern
da er zu Rom war, suchte er mich aufs fleißigste und fand mich.
\bibverse{18} Der Herr gebe ihm, dass er finde Barmherzigkeit bei dem
Herrn an jenem Tage. Und wieviel er zu Ephesus gedient hat, weißt du am
besten. \# 2 \bibverse{1} So sei nun stark, mein Sohn, durch die Gnade
in Christo Jesu. \bibverse{2} Und was du von mir gehört hast durch viele
Zeugen, das befiehl treuen Menschen, die da tüchtig sind, auch andere zu
lehren. \bibverse{3} Leide mit als ein guter Streiter Jesu Christi.
\footnote{\textbf{2:3} 2Tim 1,8; 2Tim 4,5} \bibverse{4} Kein Kriegsmann
flicht sich in Händel der Nahrung, auf dass er gefalle dem, der ihn
angenommen hat. \bibverse{5} Und wenn jemand auch kämpft, wird er doch
nicht gekrönt, er kämpfe denn recht. \bibverse{6} Es soll aber der
Ackermann, der den Acker baut, die Früchte am ersten genießen. Merke,
was ich sage! \bibverse{7} Der Herr aber wird dir in allen Dingen
Verstand geben.

\bibverse{8} Halt im Gedächtnis Jesum Christum, der auferstanden ist von
den Toten, aus dem Samen Davids, nach meinem Evangelium, \footnote{\textbf{2:8}
  1Kor 15,4; 1Kor 15,20; Röm 1,3} \bibverse{9} für welches ich leide bis
zu den Banden wie ein Übeltäter; aber Gottes Wort ist nicht gebunden.
\footnote{\textbf{2:9} Phil 1,12-14} \bibverse{10} Darum erdulde ich
alles um der Auserwählten willen, auf dass auch sie die Seligkeit
erlangen in Christo Jesu mit ewiger Herrlichkeit. \footnote{\textbf{2:10}
  Kol 1,24} \bibverse{11} Das ist gewisslich wahr: Sterben wir mit, so
werden wir mitleben; \footnote{\textbf{2:11} 2Kor 4,11} \bibverse{12}
dulden wir, so werden wir mitherrschen; verleugnen wir, so wird er uns
auch verleugnen; \footnote{\textbf{2:12} Mt 10,33} \bibverse{13} glauben
wir nicht, so bleibt er treu; er kann sich selbst nicht verleugnen.
\footnote{\textbf{2:13} 4Mo 23,19; Ps 89,31-34; Röm 3,2-3; Tit 1,2}

\bibverse{14} Solches erinnere sie und bezeuge vor dem Herrn, dass sie
nicht um Worte zanken, welches nichts nütze ist denn zu verkehren, die
da zuhören. \footnote{\textbf{2:14} 1Tim 6,4; Tit 3,9}

\bibverse{15} Befleißige dich, Gott dich zu erzeigen als einen
rechtschaffenen und unsträflichen Arbeiter, der da recht teile das Wort
der Wahrheit. \footnote{\textbf{2:15} 1Tim 4,6; Tit 2,7; Tit 1,2-8}
\bibverse{16} Des ungeistlichen, losen Geschwätzes entschlage dich; denn
es hilft viel zum ungöttlichen Wesen, \footnote{\textbf{2:16} 1Tim 4,7}
\bibverse{17} und ihr Wort frisst um sich wie der Krebs; unter welchen
ist Hymenäus und Philetus, \footnote{\textbf{2:17} 1Tim 1,20}
\bibverse{18} welche von der Wahrheit irregegangen sind und sagen, die
Auferstehung sei schon geschehen, und haben etlicher Glauben verkehrt.
\bibverse{19} Aber der feste Grund Gottes besteht und hat dieses Siegel:
Der Herr kennt die seinen; und: Es trete ab von Ungerechtigkeit, wer den
Namen Christi nennt. \footnote{\textbf{2:19} Mt 7,22-23; Joh 10,14; Joh
  10,27; 4Mo 16,5}

\bibverse{20} In einem großen Hause aber sind nicht allein goldene und
silberne Gefäße, sondern auch hölzerne und irdene, und etliche zu Ehren,
etliche aber zu Unehren. \bibverse{21} So nun jemand sich reinigt von
solchen Leuten, der wird ein geheiligtes Gefäß sein zu Ehren, dem
Hausherrn bräuchlich und zu allem guten Werk bereitet.

\bibverse{22} Fliehe die Lüste der Jugend; jage aber nach -- der
Gerechtigkeit, dem Glauben, der Liebe, dem Frieden mit allen, die den
Herrn anrufen von reinem Herzen. \bibverse{23} Aber der törichten und
unnützen Fragen entschlage dich; denn du weißt, dass sie nur Zank
gebären. \footnote{\textbf{2:23} 1Tim 4,7} \bibverse{24} Ein Knecht aber
des Herrn soll nicht zänkisch sein, sondern freundlich gegen jedermann,
lehrhaft, der die Bösen tragen kann \footnote{\textbf{2:24} Tit 1,7}
\bibverse{25} und mit Sanftmut strafe die Widerspenstigen, ob ihnen Gott
dermaleinst Buße gebe, die Wahrheit zu erkennen, \bibverse{26} und sie
wieder nüchtern würden aus des Teufels Strick, von dem sie gefangen sind
zu seinem Willen. \# 3 \bibverse{1} Das sollst du aber wissen, dass in
den letzten Tagen werden gräuliche Zeiten kommen. \footnote{\textbf{3:1}
  1Tim 4,1} \bibverse{2} Denn es werden Menschen sein, die viel von sich
halten, geizig, ruhmredig, hoffärtig, Lästerer, den Eltern ungehorsam,
undankbar, ungeistlich, \bibverse{3} lieblos, unversöhnlich, Verleumder,
unkeusch, wild, ungütig, \bibverse{4} Verräter, Frevler, aufgeblasen,
die mehr lieben Wollust denn Gott, \bibverse{5} die da haben den Schein
eines gottseligen Wesens, aber seine Kraft verleugnen sie; und solche
meide. \bibverse{6} Aus denselben sind, die hin und her in die Häuser
schleichen und führen die Weiblein gefangen, die mit Sünden beladen sind
und von mancherlei Lüsten umgetrieben, \bibverse{7} lernen immerdar, und
können nimmer zur Erkenntnis der Wahrheit kommen. \bibverse{8}
Gleicherweise aber, wie Jannes und Jambres dem Mose widerstanden, also
widerstehen auch diese der Wahrheit; es sind Menschen von zerrütteten
Sinnen, untüchtig zum Glauben. \footnote{\textbf{3:8} 2Mo 7,11; 2Mo 7,22}
\bibverse{9} Aber sie werden's in die Länge nicht treiben; denn ihre
Torheit wird offenbar werden jedermann, gleichwie auch jener Torheit
offenbar ward.

\bibverse{10} Du aber bist nachgefolgt meiner Lehre, meiner Weise,
meiner Meinung, meinem Glauben, meiner Langmut, meiner Liebe, meiner
Geduld, \bibverse{11} meinen Verfolgungen, meinen Leiden, welche mir
widerfahren sind zu Antiochien, zu Ikonien, zu Lystra. Welche
Verfolgungen ich da ertrug! Und aus allen hat mich der Herr erlöst.
\bibverse{12} Und alle, die gottselig leben wollen in Christo Jesu,
müssen Verfolgung leiden. \footnote{\textbf{3:12} Mt 16,24; Apg 14,22;
  1Thes 3,3} \bibverse{13} Mit den bösen Menschen aber und
verführerischen wird's je länger, je ärger: sie verführen und werden
verführt. \footnote{\textbf{3:13} 1Tim 4,1} \bibverse{14} Du aber bleibe
in dem, was du gelernt hast und dir vertrauet ist, sintemal du weißt,
von wem du gelernt hast. \bibverse{15} Und weil du von Kind auf die
heilige Schrift weißt, kann dich dieselbe unterweisen zur Seligkeit
durch den Glauben an Christum Jesum. \footnote{\textbf{3:15} Joh 5,39}
\bibverse{16} Denn alle Schrift, von Gott eingegeben, ist nütze zur
Lehre, zur Strafe, zur Besserung, zur Züchtigung in der Gerechtigkeit,
\footnote{\textbf{3:16} 2Petr 1,19-21} \bibverse{17} dass ein Mensch
Gottes sei vollkommen, zu allem guten Werk geschickt. \footnote{\textbf{3:17}
  1Tim 6,11}

\hypertarget{section-1}{%
\section{4}\label{section-1}}

\bibverse{1} So bezeuge ich nun vor Gott und dem Herrn Jesus Christus,
der da zukünftig ist, zu richten die Lebendigen und die Toten mit seiner
Erscheinung und mit seinem Reich: \footnote{\textbf{4:1} 1Petr 4,5}
\bibverse{2} Predige das Wort, halte an, es sei zu rechter Zeit oder zur
Unzeit; strafe, drohe, ermahne mit aller Geduld und Lehre. \footnote{\textbf{4:2}
  Apg 20,20; Apg 20,31} \bibverse{3} Denn es wird eine Zeit sein, da sie
die heilsame Lehre nicht leiden werden; sondern nach ihren eigenen
Lüsten werden sie sich selbst Lehrer aufladen, nach dem ihnen die Ohren
jucken, \footnote{\textbf{4:3} 1Tim 4,1} \bibverse{4} und werden die
Ohren von der Wahrheit wenden und sich zu Fabeln kehren. \footnote{\textbf{4:4}
  1Tim 4,7; 2Thes 2,11} \bibverse{5} Du aber sei nüchtern allenthalben,
sei willig, zu leiden, tue das Werk eines evangelischen Predigers,
richte dein Amt redlich aus. \footnote{\textbf{4:5} 2Tim 2,3}

\bibverse{6} Denn ich werde schon geopfert, und die Zeit meines
Abscheidens ist vorhanden. \footnote{\textbf{4:6} Phil 2,17}
\bibverse{7} Ich habe einen guten Kampf gekämpft, ich habe den Lauf
vollendet, ich habe Glauben gehalten; \footnote{\textbf{4:7} Apg 20,24;
  1Kor 9,25; Phil 3,14; 1Tim 6,12} \bibverse{8} hinfort ist mir
beigelegt die Krone der Gerechtigkeit, welche mir der Herr an jenem
Tage, der gerechte Richter, geben wird, nicht mir aber allein, sondern
auch allen, die seine Erscheinung liebhaben. \footnote{\textbf{4:8} 2Tim
  2,5; Mt 25,21; 1Petr 5,4; Jak 1,12; Offb 2,10}

\bibverse{9} Befleißige dich, dass du bald zu mir kommst. \footnote{\textbf{4:9}
  2Tim 1,4} \bibverse{10} Denn Demas hat mich verlassen und diese Welt
liebgewonnen und ist gen Thessalonich gezogen, Kreszens nach Galatien,
Titus nach Dalmatien. \footnote{\textbf{4:10} Kol 4,7; Kol 4,10; Kol
  4,14} \bibverse{11} Lukas allein ist bei mir. Markus nimm zu dir und
bringe ihn mit dir; denn er ist mir nützlich zum Dienst. \footnote{\textbf{4:11}
  Apg 15,37; Kol 4,10} \bibverse{12} Tychikus habe ich gen Ephesus
gesandt. \footnote{\textbf{4:12} Eph 6,21} \bibverse{13} Den Mantel, den
ich zu Troas ließ bei Karpus, bringe mit, wenn du kommst, und die
Bücher, sonderlich die Pergamente. \bibverse{14} Alexander, der Schmied,
hat mir viel Böses bewiesen; der Herr bezahle ihm nach seinen Werken.
\bibverse{15} Vor dem hüte du dich auch; denn er hat unseren Worten sehr
widerstanden.

\bibverse{16} In meiner ersten Verantwortung stand mir niemand bei,
sondern sie verließen mich alle. Es sei ihnen nicht zugerechnet.
\footnote{\textbf{4:16} 2Tim 1,15} \bibverse{17} Der Herr aber stand mir
bei und stärkte mich, auf dass durch mich die Predigt bestätigt würde
und alle Heiden sie hörten; und ich ward erlöst von des Löwen Rachen.
\footnote{\textbf{4:17} Apg 23,11; Apg 27,23} \bibverse{18} Der Herr
aber wird mich erlösen von allem Übel und mir aushelfen zu seinem
himmlischen Reich; welchem sei Ehre von Ewigkeit zu Ewigkeit! Amen.

\bibverse{19} Grüße Priska und Aquila und das Haus des Onesiphorus.
\footnote{\textbf{4:19} Apg 18,2; 2Tim 1,16} \bibverse{20} Erastus blieb
zu Korinth; Trophimus aber ließ ich zu Milet krank. \footnote{\textbf{4:20}
  Apg 19,22; Apg 20,4; 2Tim 1,16} \bibverse{21} Tue Fleiß, dass du vor
dem Winter kommst. Es grüßt dich Eubulus und Pudens und Linus und
Klaudia und alle Brüder.

\bibverse{22} Der Herr Jesus Christus sei mit deinem Geiste! Die Gnade
sei mit euch! Amen.
