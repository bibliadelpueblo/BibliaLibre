\hypertarget{section}{%
\section{1}\label{section}}

\bibverse{1} Zu den Zeiten Ahasveros, der da König war von Indien bis an
die Mohren, über hundertundsiebenundzwanzig Länder, \bibverse{2} und da
er auf seinem königlichen Stuhl saß zu Schloß Susan, \bibverse{3} im
dritten Jahr seines Königreichs, machte er bei ihm ein Mahl allen seinen
Fürsten und Knechten, nämlich den Gewaltigen in Persien und Medien, den
Landpflegern und Obersten in seinen Ländern, \bibverse{4} daß er sehen
ließe den herrlichen Reichtum seines Königreichs und die köstliche
Pracht seiner Majestät viel Tage lang, nämlich hundertundachtzig Tage.
\bibverse{5} Und da die Tage aus waren, machte der König ein Mahl allem
Volk, das zu Schloß Susan war, beide Großen und Kleinen, sieben Tage
lang im Hofe des Gartens am Hause des Königs. \bibverse{6} Da hingen
weiße, rote und gelbe Tücher, mit leinenen und scharlakenen Seilen
gefasset in silbernen Ringen auf Marmelsäulen. Die Bänke waren gülden
und silbern, auf Pflaster von grünen, weißen, gelben und schwarzen
Marmeln gemacht. \bibverse{7} Und das Getränk trug man in güldenen
Gefäßen, und immer andern und andern Gefäßen, und königlichen Wein die
Menge, wie denn der König vermochte. \bibverse{8} Und man setzte
niemand, was er trinken sollte; denn der König hatte allen Vorstehern in
seinem Hause befohlen, daß ein jeglicher sollte tun, wie es ihm
wohlgefiele. \bibverse{9} Und die Königin Vasthi machte auch ein Mahl
für die Weiber im königlichen Hause des Königs Ahasveros. \bibverse{10}
Und am siebenten Tage, da der König gutes Muts war vom Wein, hieß er
Mehuman, Bistha, Harbona, Bigtha, Abagtha, Sethar und Charkas, die
sieben Kämmerer, die vor dem Könige Ahasveros dieneten, \bibverse{11}
daß sie die Königin Vasthi holeten vor den König mit der königlichen
Krone, daß er den Völkern und Fürsten zeigete ihre Schöne; denn sie war
schön. \bibverse{12} Aber die Königin Vasthi wollte nicht kommen nach
dem Wort des Königs durch seine Kämmerer. Da ward der König sehr zornig,
und sein Grimm entbrannte in ihm. \bibverse{13} Und der König sprach zu
den Weisen, die sich auf Landes Sitten verstunden (denn des Königs
Sachen mußten geschehen vor allen Verständigen auf Recht und Händel;
\bibverse{14} die Nächsten aber bei ihm waren Charsena, Sethar, Admatha,
Tharsis, Meres, Marsena und Memuchan, die sieben Fürsten der Perser und
Meder, die das Angesicht des Königs sahen und saßen obenan im
Königreich), \bibverse{15} was für ein Recht man an der Königin Vasthi
tun sollte, darum daß sie nicht getan hatte nach dem Wort des Königs
durch seine Kämmerer. \bibverse{16} Da sprach Memuchan vor dem Könige
und Fürsten: Die Königin Vasthi hat nicht allein an dem Könige übel
getan, sondern auch an allen Fürsten und an allen Völkern in allen
Landen des Königs Ahasveros. \bibverse{17} Denn es wird solche Tat der
Königin auskommen zu allen Weibern, daß sie ihre Männer verachten vor
ihren Augen und werden sagen: Der König Ahasveros hieß die Königin
Vasthi vor sich kommen; aber sie wollte nicht. \bibverse{18} So werden
nun die Fürstinnen in Persien und Medien auch so sagen zu allen Fürsten
des Königs, wenn sie solche Tat der Königin hören; so wird sich
Verachtens und Zorns genug heben. \bibverse{19} Gefällt es dem Könige,
so lasse man ein königlich Gebot von ihm ausgehen und schreiben nach der
Perser und Meder Gesetz, welches man nicht darf übertreten, daß Vasthi
nicht mehr vor den König Ahasveros komme; und der König gebe ihr
Königreich ihrer Nächsten, die besser ist denn sie; \bibverse{20} und
daß dieser Brief des Königs, der gemacht wird, in sein ganz Reich
(welches groß ist) erschalle, daß alle Weiber ihre Männer in Ehren
halten, beide unter Großen und Kleinen. \bibverse{21} Das gefiel dem
Könige und den Fürsten; und der König tat nach dem Wort Memuchans.
\bibverse{22} Da wurden Briefe ausgesandt in alle Länder des Königs, in
ein jeglich Land nach seiner Schrift und zu jeglichem Volk nach seiner
Sprache, daß ein jeglicher Mann der Oberherr in seinem Hause sei; und
ließ reden nach der Sprache seines Volks

\hypertarget{section-1}{%
\section{2}\label{section-1}}

\bibverse{1} Nach diesen Geschichten, da der Grimm des Königs Ahasveros
sich gelegt hatte, gedachte er an Vasthi, was sie getan hatte, und was
über sie beschlossen wäre. \bibverse{2} Da sprachen die Knaben des
Königs, die ihm dieneten: Man suche dem Könige junge schöne Jungfrauen;
\bibverse{3} und der König bestelle Schauer in allen Landen seines
Königreichs, daß sie allerlei junge schöne Jungfrauen zusammenbringen
gen Schloß Susan, ins Frauenzimmer, unter die Hand Hegais, des Königs
Kämmerers, der der Weiber wartet, und gebe ihnen ihren Schmuck;
\bibverse{4} und welche Dirne dem Könige gefällt, die werde Königin an
Vasthis Statt. Das gefiel dem Könige und tat also. \bibverse{5} Es war
aber ein jüdischer Mann zu Schloß Susan, der hieß Mardachai, ein Sohn
Jairs, des Sohns Simeis, des Sohns Kis, des Sohns Jeminis, \bibverse{6}
der mit weggeführet war von Jerusalem, da Jechanja, der König Judas,
weggeführet ward, welchen Nebukadnezar, der König zu Babel, wegführete.
\bibverse{7} Und er war ein Vormund Hadassas, die ist Esther, eine
Tochter seines Vetters; denn sie hatte weder Vater noch Mutter. Und sie
war eine schöne und feine Dirne. Und da ihr Vater und Mutter starb, nahm
sie Mardachai auf zur Tochter. \bibverse{8} Da nun das Gebot und Gesetz
des Königs laut ward, und viel Dirnen zu Hause gebracht wurden gen
Schloß Susan unter die Hand Hegais, ward Esther auch genommen zu des
Königs Haus unter die Hand Hegais, des Hüters der Weiber. \bibverse{9}
Und die Dirne gefiel ihm, und sie fand Barmherzigkeit vor ihm. Und er
eilete mit ihrem Schmuck, daß er ihr ihren Teil gäbe, und sieben feine
Dirnen von des Königs Hause dazu. Und er tat sie mit ihren Dirnen an den
besten Ort im Frauenzimmer. \bibverse{10} Und Esther sagte ihm nicht an
ihr Volk und ihre Freundschaft; denn Mardachai hatte ihr geboten, sie
sollte es nicht ansagen. \bibverse{11} Und Mardachai wandelte alle Tage
vor dem Hofe am Frauenzimmer, daß er erführe, ob es Esther wohlginge,
und was ihr geschehen würde. \bibverse{12} Wenn aber die bestimmte Zeit
einer jeglichen Dirne kam, daß sie zum Könige Ahasveros kommen sollte,
nachdem sie zwölf Monden im Frauenschmücken gewesen war (denn ihr
Schmücken mußte so viel Zeit haben, nämlich sechs Monden mit Balsam und
Myrrhen und sechs Monden mit guter Spezerei, so waren denn die Weiber
geschmückt), \bibverse{13} alsdann ging eine Dirne zum Könige, und
welche sie wollte, mußte man ihr geben, die mit ihr vom Frauenzimmer zu
des Königs Hause ginge. \bibverse{14} Und wenn eine des Abends
hineinkam, die ging des Morgens von ihm in das andere Frauenzimmer unter
die Hand Saasgas, des Königs Kämmerers, der Kebsweiber Hüters. Und sie
mußte nicht wieder zum Könige kommen, es lüstete denn den König und
ließe sie mit Namen rufen. \bibverse{15} Da nun die Zeit Esthers
herzukam, der Tochter Abihails, des Vetters Mardachais (die er zur
Tochter hatte aufgenommen), daß sie zum Könige kommen sollte, begehrete
sie nichts, denn was Hegai, des Königs Kämmerer, der Weiber Hüter,
sprach. Und Esther fand Gnade vor allen, die sie ansahen. \bibverse{16}
Es ward aber Esther genommen zum Könige Ahasveros ins königliche Haus im
zehnten Monden, der da heißet Tebeth, im siebenten Jahr seines
Königreichs. \bibverse{17} Und der König gewann Esther lieb über alle
Weiber, und sie fand Gnade und Barmherzigkeit vor ihm vor allen
Jungfrauen. Und er setzte die königliche Krone auf ihr Haupt und machte
sie zur Königin an Vasthis Statt. \bibverse{18} Und der König machte ein
groß Mahl allen seinen Fürsten und Knechten (das war ein Mahl um Esthers
willen) und ließ die Länder ruhen und gab königliche Geschenke aus.
\bibverse{19} Und da man das andere Mal Jungfrauen versammelte, saß
Mardachai im Tor des Königs. \bibverse{20} Und Esther hatte noch nicht
angesagt ihre Freundschaft noch ihr Volk, wie ihr denn Mardachai geboten
hatte. Denn Esther tat nach dem Wort Mardachais, gleich als da er ihr
Vormund war. \bibverse{21} Zur selbigen Zeit, da Mardachai im Tor des
Königs saß, wurden zween Kämmerer des Königs, Bigthan und Theres, die
der Tür hüteten, zornig und trachteten, ihre Hände an den König
Ahasveros zu legen. \bibverse{22} Das ward Mardachai kund, und er sagte
es an der Königin Esther, und Esther sagte es dem Könige in Mardachais
Namen. \bibverse{23} Und da man solches forschete, ward es funden. Und
sie wurden beide an Bäume gehängt, und ward geschrieben in die Chronik
vor dem Könige.

\hypertarget{section-2}{%
\section{3}\label{section-2}}

\bibverse{1} Nach diesen Geschichten machte der König Ahasveros Haman
groß, den Sohn Medathas, den Agagiter, und erhöhete ihn und setzte
seinen Stuhl über alle Fürsten, die bei ihm waren. \bibverse{2} Und alle
Knechte des Königs, die im Tor des Königs waren, beugten die Kniee und
beteten Haman an; denn der König hatte es also geboten. Aber Mardachai
beugete die Kniee nicht und betete nicht an. \bibverse{3} Da sprachen
des Königs Knechte, die im Tor des Königs waren, zu Mardachai: Warum
übertrittst du des Königs Gebot? \bibverse{4} Und da sie solches täglich
zu ihm sagten, und er ihnen nicht gehorchte, sagten sie es Haman an, daß
sie sähen, ob solch Tun Mardachais bestehen würde; denn er hatte ihnen
gesagt, daß er ein Jude wäre. \bibverse{5} Und da Haman sah, daß
Mardachai ihm nicht die Kniee beugete noch ihn anbetete, ward er voll
Grimms. \bibverse{6} Und verachtete es, daß er an Mardachai allein
sollte die Hand legen, denn sie hatten ihm das Volk Mardachais angesagt;
sondern er trachtete, das Volk Mardachais, alle Juden, so im ganzen
Königreich Ahasveros waren, zu vertilgen. \bibverse{7} Im ersten Monden,
das ist der Mond Nisan, im zwölften Jahr des Königs Ahasveros, ward das
Los geworfen vor Haman, von einem Tage auf den andern und vom Monden bis
auf den zwölften Monden, das ist der Mond Adar. \bibverse{8} Und Haman
sprach zum Könige Ahasveros: Es ist ein Volk zerstreuet und teilet sich
unter alle Völker in allen Landen deines Königreichs, und ihr Gesetz ist
anders denn aller Völker, und tun nicht nach des Königs Gesetzen; und
ist dem Könige nicht zu leiden, sie also zu lassen. \bibverse{9} Gefällt
es dem Könige, so schreibe er, daß man es umbringe; so will ich
zehntausend Zentner Silbers darwägen unter die Hand der Amtleute, daß
man es bringe in die Kammer des Königs. \bibverse{10} Da tat der König
seinen Ring von der Hand und gab ihn Haman, dem Sohne Medathas, dem
Agagiter, der Juden Feind. \bibverse{11} Und der König sprach zu Haman:
Das Silber sei dir gegeben, dazu das Volk, daß du damit tust, was dir
gefällt. \bibverse{12} Da rief man den Schreibern des Königs am
dreizehnten Tage des ersten Monden; und ward geschrieben, wie Haman
befahl, an die Fürsten des Königs und zu den Landpflegern hin und her in
den Ländern und zu den Hauptleuten eines jeglichen Volks in den Ländern
hin und her nach der Schrift eines jeglichen Volks und nach ihrer
Sprache, im Namen des Königs Ahasveros und mit des Königs Ringe
versiegelt. \bibverse{13} Und die Briefe wurden gesandt durch die Läufer
in alle Länder des Königs, zu vertilgen, zu erwürgen und umzubringen
alle Juden, beide jung und alt, Kinder und Weiber, auf einen Tag,
nämlich auf den dreizehnten Tag des zwölften Monden, das ist der Mond
Adar, und ihr Gut zu rauben. \bibverse{14} Also war der Inhalt der
Schrift, daß ein Gebot gegeben wäre in allen Ländern, allen Völkern zu
eröffnen, daß sie auf denselben Tag geschickt wären. \bibverse{15} Und
die Läufer gingen aus eilend nach des Königs Gebot. Und zu Schloß Susan
ward angeschlagen ein Gebot. Und der König und Haman aßen und tranken;
aber die Stadt Susan ward irre.

\hypertarget{section-3}{%
\section{4}\label{section-3}}

\bibverse{1} Da Mardachai erfuhr alles, was geschehen war, zerriß er
seine Kleider und legte einen Sack an und Asche; und ging hinaus mitten
in die Stadt und schrie laut und kläglich. \bibverse{2} Und kam vor das
Tor des Königs. Denn es mußte niemand zu des Königs Tor eingehen, der
einen Sack anhätte. \bibverse{3} Und in allen Ländern, an welchen Ort
des Königs Wort und Gebot gelangete, war ein groß Klagen unter den
Juden, und viele fasteten, weineten, trugen Leid und lagen in Säcken und
in der Asche. \bibverse{4} Da kamen die Dirnen Esthers und ihre Kämmerer
und sagten's ihr an. Da erschrak die Königin sehr. Und sie sandte
Kleider, daß Mardachai anzöge und den Sack von ihm ablegte; er aber nahm
sie nicht. \bibverse{5} Da rief Esther Hathach unter des Königs
Kämmerern, der vor ihr stund, und befahl ihm an Mardachai, daß sie
erführe, was das wäre, und warum er so täte. \bibverse{6} Da ging
Hathach hinaus zu Mardachai an die Gasse in der Stadt, die vor dem Tor
des Königs war. \bibverse{7} Und Mardachai sagte ihm alles, was ihm
begegnet wäre, und die Summa des Silbers, das Haman geredet hatte in des
Königs Kammer darzuwägen um der Juden willen, sie zu vertilgen.
\bibverse{8} Und gab ihm die Abschrift des Gebots, das zu Susan
angeschlagen war, sie zu vertilgen, daß er es Esther zeigete und ihr
ansagete und geböte ihr, daß sie zum Könige hineinginge und täte eine
Bitte an ihn um ihr Volk. \bibverse{9} Und da Hathach hineinkam und
sagte Esther die Worte Mardachais, \bibverse{10} sprach Esther zu
Hathach und gebot ihm an Mardachai: \bibverse{11} Es wissen alle Knechte
des Königs und das Volk in den Landen des Königs, daß, wer zum Könige
hineingehet inwendig in den Hof, er sei Mann oder Weib, der nicht
gerufen ist, der soll stracks Gebots sterben, es sei denn, daß der König
den güldenen Zepter gegen ihn reiche, damit er lebendig bleibe. Ich aber
bin nun in dreißig Tagen nicht gerufen, zum Könige hineinzukommen.
\bibverse{12} Und da die Worte der Esther wurden Mardachai angesagt,
\bibverse{13} hieß Mardachai Esther wieder sagen: Gedenke nicht, daß du
dein Leben errettest, weil du im Hause des Königs bist, vor allen Juden;
\bibverse{14} denn wo du wirst zu dieser Zeit schweigen, so wird eine
Hilfe und Errettung aus einem andern Ort den Juden entstehen, und du und
deines Vaters Haus werdet umkommen. Und wer weiß, ob du um dieser Zeit
willen zum Königreich kommen bist? \bibverse{15} Esther hieß Mardachai
antworten: \bibverse{16} So gehe hin und versammle alle Juden, die zu
Susan vorhanden sind, und fastet für mich, daß ihr nicht esset und
trinket in dreien Tagen weder Tag noch Nacht; ich und meine Dirnen
wollen auch also fasten. Und also will ich zum Könige hineingehen wider
das Gebot; komme ich um, so komme ich um. \bibverse{17} Mardachai ging
hin und tat alles, was ihm Esther geboten hatte.

\hypertarget{section-4}{%
\section{5}\label{section-4}}

\bibverse{1} Und am dritten Tage zog sich Esther königlich an und trat
in den Hof am Hause des Königs inwendig gegen dem Hause des Königs. Und
der König saß auf seinem königlichen Stuhl im königlichen Hause, gegen
der Tür des Hauses. \bibverse{2} Und da der König sah Esther, die
Königin, stehen im Hofe, fand sie Gnade vor seinen Augen. Und der König
reckte den güldenen Zepter in seiner Hand gegen Esther. Da trat Esther
herzu und rührete die Spitze des Zepters an. \bibverse{3} Da sprach der
König zu ihr: Was ist dir, Esther, Königin? und was forderst du? Auch
die Hälfte des Königreichs soll dir gegeben werden. \bibverse{4} Esther
sprach: Gefällt es dem Könige, so komme der König und Haman heute zu dem
Mahl, das ich zugerichtet habe. \bibverse{5} Der König sprach: Eilet,
daß Haman tue, was Esther gesagt hat! Da nun der König und Haman zu dem
Mahl kamen, das Esther zugerichtet hatte, \bibverse{6} sprach der König
zu Esther, da er Wein getrunken hatte: Was bittest du, Esther? Es soll
dir gegeben werden. Und was forderst du? Auch die Hälfte des
Königreichs, es soll geschehen. \bibverse{7} Da antwortete Esther und
sprach: Meine Bitte und Begehr ist: \bibverse{8} Habe ich Gnade gefunden
vor dem Könige, und so es dem Könige gefällt, mir zu geben meine Bitte
und zu tun mein Begehr, so komme der König und Haman zu dem Mahl, das
ich für sie zurichten will, so will ich morgen tun, was der König gesagt
hat. \bibverse{9} Da ging Haman des Tages hinaus fröhlich und gutes
Muts. Und da er sah Mardachai im Tor des Königs, daß er nicht aufstund,
noch sich vor ihm bewegte, ward er voll Zorns über Mardachai.
\bibverse{10} Aber er enthielt sich. Und da er heim kam, sandte er hin
und ließ holen seine Freunde und sein Weib Seres. \bibverse{11} Und
erzählete ihnen die Herrlichkeit seines Reichtums und die Menge seiner
Kinder und alles, wie ihn der König so groß gemacht hätte, und daß er
über die Fürsten und Knechte des Königs erhaben wäre. \bibverse{12} Auch
sprach Haman: Und die Königin Esther hat niemand lassen kommen mit dem
Könige zum Mahl, das sie zugerichtet hat, ohne mich; und bin auch morgen
zu ihr geladen mit dem Könige. \bibverse{13} Aber an dem allem habe ich
keine Genüge, solange ich sehe den Juden Mardachai am Königstor sitzen.
\bibverse{14} Da sprach zu ihm sein Weib Seres und alle seine Freunde:
Man mache einen Baum fünfzig Ellen hoch und sage morgen dem Könige, daß
man Mardachai daran hänge; so kommst du mit dem Könige fröhlich zum
Mahl. Das gefiel Haman wohl und ließ einen Baum zurichten.

\hypertarget{section-5}{%
\section{6}\label{section-5}}

\bibverse{1} In derselben Nacht konnte der König nicht schlafen und hieß
die Chronik und die Historien bringen. Da die wurden vor dem Könige
gelesen, \bibverse{2} traf sich's, da geschrieben war, wie Mardachai
hatte angesagt, daß die zween Kämmerer des Königs, Bigthan und Theres,
die an der Schwelle hüteten, getrachtet hätten, die Hand an den König
Ahasveros zu legen. \bibverse{3} Und der König sprach: Was haben wir
Mardachai Ehre und Gutes dafür getan? Da sprachen die Knaben des Königs,
die ihm dieneten: Es ist ihm nichts geschehen. \bibverse{4} Und der
König sprach: Wer ist im Hofe? (Denn Haman war in den Hof gegangen,
draußen vor des Königs Hause, daß er dem Könige sagte, Mardachai zu
hängen an den Baum, den er ihm zubereitet hatte.) \bibverse{5} Und des
Königs Knaben sprachen zu ihm: Siehe, Haman stehet im Hofe. Der König
sprach: Laßt ihn hereingehen! \bibverse{6} Und da Haman hineinkam,
sprach der König zu ihm: Was soll man dem Manne tun, den der König gerne
wollte ehren? Haman aber gedachte in seinem Herzen: Wem sollte der König
anders gerne wollen Ehre tun denn mir? \bibverse{7} Und Haman sprach zum
Könige: Den Mann, den der König gerne wollte ehren, \bibverse{8} soll
man herbringen, daß man ihm königliche Kleider anziehe, die der König
pflegt zu tragen, und das Roß, da der König auf reitet, und daß man die
königliche Krone auf sein Haupt setze. \bibverse{9} Und man soll solch
Kleid und Roß geben in die Hand eines Fürsten des Königs, daß derselbe
den Mann anziehe, den der König gerne ehren wollte, und führe ihn auf
dem Roß in der Stadt Gassen und lasse rufen vor ihm her: So wird man tun
dem Manne, den der König gerne ehren wollte. \bibverse{10} Der König
sprach zu Haman: Eile und nimm das Kleid und Roß, wie du gesagt hast,
und tue also mit Mardachai, dem Juden, der vor dem Tor des Königs
sitzet; und laß nichts fehlen an allem, das du geredet hast.
\bibverse{11} Da nahm Haman das Kleid und Roß, und zog Mardachai an und
führete ihn auf der Stadt Gassen und rief vor ihm her: So wird man tun
dem Manne, den der König gerne ehren wollte. \bibverse{12} Und Mardachai
kam wieder an das Tor des Königs. Haman aber eilete nach Hause, trug
Leid mit verhülletem Kopfe \bibverse{13} und erzählete seinem Weibe
Seres und seinen Freunden allen alles, was ihm begegnet war. Da sprachen
zu ihm seine Weisen und sein Weib Seres: Ist Mardachai vom Samen der
Juden, vor dem du zu fallen angehoben hast, so vermagst du nichts an
ihm, sondern du wirst vor ihm fallen. \bibverse{14} Da sie aber noch mit
ihm redeten, kamen herbei des Königs Kämmerer und trieben Haman, zum
Mahl zu kommen, das Esther zugerichtet hatte.

\hypertarget{section-6}{%
\section{7}\label{section-6}}

\bibverse{1} Und da der König mit Haman kam zum Mahl, das die Königin
Esther zugerichtet hatte, \bibverse{2} sprach der König zu Esther des
andern Tages, da er Wein getrunken hatte: Was bittest du, Königin
Esther, daß man dir's gebe? Und was forderst du? Auch das halbe
Königreich, es soll geschehen. \bibverse{3} Esther, die Königin,
antwortete und sprach: Habe ich Gnade vor dir funden, o König, und
gefällt es dem Könige, so gib mir mein Leben um meiner Bitte willen und
mein Volk um meines Begehrens willen. \bibverse{4} Denn wir sind
verkauft, ich und mein Volk, daß wir vertilget, erwürget und umgebracht
werden; und wollte GOtt, wir würden doch zu Knechten und Mägden
verkauft, so wollte ich schweigen, so würde der Feind doch dem Könige
nicht schaden. \bibverse{5} Der König Ahasveros redete und sprach zu der
Königin Esther: Wer ist der? Oder wo ist der, der solches in seinen Sinn
nehmen dürfte, also zu tun? \bibverse{6} Esther sprach: Der Feind und
Widersacher ist dieser böse Haman. Haman entsetzte sich vor dem Könige
und der Königin. \bibverse{7} Und der König stund auf vom Mahl und vom
Wein in seinem Grimm und ging in den Garten am Hause. Und Haman stund
auf und bat die Königin Esther um sein Leben; denn er sah, daß ihm ein
Unglück vom Könige schon bereitet war. \bibverse{8} Und da der König
wieder aus dem Garten am Hause in den Saal, da man gegessen hatte, kam,
lag Haman an der Bank, da Esther auf saß. Da sprach der König: Will er
die Königin würgen bei mir im Hause? Da das Wort aus des Königs Munde
ging, verhülleten sie Haman das Antlitz. \bibverse{9} Und Harbona, der
Kämmerer einer vor dem Könige, sprach: Siehe, es stehet ein Baum im
Hause Hamans fünfzig Ellen hoch, den er Mardachai gemacht hatte, der
Gutes für den König geredet hat. Der König sprach: Laßt ihn daran
hängen! \bibverse{10} Also hängte man Haman an den Baum, den er
Mardachai gemacht hatte. Da legte sich des Königs Zorn.

\hypertarget{section-7}{%
\section{8}\label{section-7}}

\bibverse{1} An dem Tage gab der König Ahasveros der Königin Esther das
Haus Hamans, des Judenfeindes. Und Mardachai kam vor den König; denn
Esther sagte an, wie er ihr zugehörete. \bibverse{2} Und der König tat
ab seinen Fingerreif, den er von Haman hatte genommen; und gab ihn
Mardachai. Und Esther setzte Mardachai über das Haus Hamans.
\bibverse{3} Und Esther redete weiter vor dem König und fiel ihm zu den
Füßen und flehete ihn, daß er wegtäte die Bosheit Hamans, des Agagiters,
und seine Anschläge, die er wider die Juden erdacht hatte. \bibverse{4}
Und der König reckte das güldene Zepter zu Esther. Da stund Esther auf
und trat vor den König \bibverse{5} und sprach: Gefällt es dem Könige,
und habe ich Gnade funden vor ihm, und ist's gelegen dem Könige, und ich
ihm gefalle, so schreibe man, daß die Briefe der Anschläge Hamans, des
Sohns Medathas, des Agagiters, widerrufen werden, die er geschrieben
hat, die Juden umzubringen in allen Landen des Königs. \bibverse{6} Denn
wie kann ich zusehen dem Übel, das mein Volk treffen würde? Und wie kann
ich zusehen, daß mein Geschlecht umkomme? \bibverse{7} Da sprach der
König Ahasveros zur Königin Esther und zu Mardachai, dem Juden: Siehe,
ich habe Esther das Haus Hamans gegeben, und ihn hat man an einen Baum
gehänget, darum daß er seine Hand hat an die Juden gelegt. \bibverse{8}
So schreibet nun ihr für die Juden, wie es euch gefällt, in des Königs
Namen und versiegelt es mit des Königs Ringe. Denn die Schrift, die in
des Königs Namen geschrieben und mit des Königs Ringe versiegelt worden,
mußte niemand widerrufen. \bibverse{9} Da wurden gerufen des Königs
Schreiber zu der Zeit im dritten Monden, das ist der Mond Sivan, am
dreiundzwanzigsten Tage; und wurde geschrieben, wie Mardachai gebot zu
den Juden und zu den Fürsten, Landpflegern und Hauptleuten in Landen von
Indien an bis an die Mohren, nämlich hundertundsiebenundzwanzig Länder,
einem jeglichen Lande nach seinen Schriften, einem jeglichen Volk nach
seiner Sprache und den Juden nach ihrer Schrift und Sprache.
\bibverse{10} Und es ward geschrieben in des Königs Ahasveros Namen und
mit des Königs Ringe versiegelt. Und er sandte die Briefe durch die
reitenden Boten auf jungen Mäulern, \bibverse{11} darinnen der König den
Juden gab, wo sie in Städten waren, sich zu versammeln und zu stehen für
ihr Leben und zu vertilgen, zu erwürgen und umzubringen alle Macht des
Volks und Landes, die sie ängsteten, samt den Kindern und Weibern, und
ihr Gut zu rauben, \bibverse{12} auf einen Tag in allen Ländern des
Königs Ahasveros, nämlich am dreizehnten Tage des zwölften Monden, das
ist der Mond Adar. \bibverse{13} Der Inhalt aber der Schrift war, daß
ein Gebot gegeben wäre in allen Landen, zu öffnen allen Völkern, daß die
Juden auf den Tag geschickt sein sollten, sich zu rächen an ihren
Feinden. \bibverse{14} Und die reitenden Boten auf den Mäulern ritten
aus schnell und eilend nach dem Wort des Königs, und das Gebot ward zu
Schloß Susan angeschlagen. \bibverse{15} Mardachai aber ging aus, von
dem Könige in königlichen Kleidern, gelb und weiß, und mit einer großen
güldenen Krone, angetan mit einem Leinen- und Purpurmantel; und die
Stadt Susan jauchzete und war fröhlich. \bibverse{16} Den Juden aber war
ein Licht und Freude und Wonne und Ehre kommen. \bibverse{17} Und in
allen Landen und Städten, an welchen Ort des Königs Wort und Gebot
gelangete, da ward Freude und Wonne unter den Juden, Wohlleben und gute
Tage, daß viele der Völker im Lande Juden wurden; denn die Furcht der
Juden kam über sie.

\hypertarget{section-8}{%
\section{9}\label{section-8}}

\bibverse{1} Im zwölften Monden, das ist der Mond Adar, am dreizehnten
Tage, den des Königs Wort und Gebot bestimmt hatte, daß man's tun
sollte, eben desselben Tages, da die Feinde der Juden hofften, sie zu
überwältigen, wandte sich's, daß die Juden ihre Feinde überwältigen
sollten. \bibverse{2} Da versammelten sich die Juden in ihren Städten,
in allen Landen des Königs Ahasveros, daß sie die Hand legten an die, so
ihnen übel wollten. Und niemand konnte ihnen widerstehen; denn Furcht
war über alle Völker kommen. \bibverse{3} Auch alle Obersten in Landen
und Fürsten und Landpfleger und Amtleute des Königs erhuben die Juden;
denn die Furcht Mardachais kam über sie. \bibverse{4} Denn Mardachai war
groß im Hause des Königs, und sein Gerücht erscholl in allen Ländern,
wie er zunähme und groß würde. \bibverse{5} Also schlugen die Juden an
allen ihren Feinden mit der Schwertschlacht und würgeten und brachten um
und taten nach ihrem Willen an denen, die ihnen feind waren.
\bibverse{6} Und zu Schloß Susan erwürgeten die Juden und brachten um
fünfhundert Mann. \bibverse{7} Dazu erwürgeten sie Parsandatha, Dalphon,
Aspatha, \bibverse{8} Poratha, Adalja, Aridatha, \bibverse{9} Parmastha,
Arisai, Aridai, Vajesatha, \bibverse{10} die zehn Söhne Hamans, des
Sohns Medathas, des Judenfeindes; aber an seine Güter legten sie ihre
Hände nicht. \bibverse{11} Zu derselbigen Zeit kam die Zahl der
Erwürgten gen Schloß Susan vor den König. \bibverse{12} Und der König
sprach zu der Königin Esther: Die Juden haben zu Schloß Susan
fünfhundert Mann erwürget und umgebracht und die zehn Söhne Hamans; was
werden sie tun in den andern Ländern des Königs? Was bittest du, daß man
dir gebe? und was forderst du mehr, daß man tue? \bibverse{13} Esther
sprach: Gefällt es dem Könige, so lasse er auch morgen die Juden zu
Susan tun nach dem heutigen Gebot, daß sie die zehn Söhne Hamans an den
Baum hängen. \bibverse{14} Und der König hieß also tun. Und das Gebot
ward zu Susan angeschlagen, und die zehn Söhne Hamans wurden gehänget.
\bibverse{15} Und die Juden versammelten sich zu Susan am vierzehnten
Tage des Monden Adar und erwürgeten zu Susan dreihundert Mann; aber an
ihre Güter legten sie ihre Hände nicht. \bibverse{16} Aber die andern
Juden in den Ländern des Königs kamen zusammen und stunden für ihr
Leben, daß sie Ruhe schaffeten vor ihren Feinden; und erwürgeten ihrer
Feinde fünfundsiebenzigtausend; aber an ihre Güter legten sie ihre Hände
nicht. \bibverse{17} Das geschah am dreizehnten Tage des Monden Adar,
und ruheten am vierzehnten Tage desselben Monden; den machte man zum
Tage des Wohllebens und Freuden. \bibverse{18} Aber die Juden zu Susan
waren zusammenkommen, beide am dreizehnten und vierzehnten Tage, und
ruheten am fünfzehnten Tage; und den Tag machte man zum Tage des
Wohllebens und Freuden. \bibverse{19} Darum machten die Juden, die auf
den Dörfern und Flecken wohneten, den vierzehnten Tag des Monden Adar
zum Tage des Wohllebens und Freuden, und sandte einer dem andern
Geschenke. \bibverse{20} Und Mardachai beschrieb diese Geschichte und
sandte die Briefe zu allen Juden, die in allen Ländern des Königs
Ahasveros waren, beide nahen und fernen, \bibverse{21} daß sie annähmen
und hielten den vierzehnten und fünfzehnten Tag des Monden Adar
jährlich, \bibverse{22} nach den Tagen, darinnen die Juden zur Ruhe
kommen waren von ihren Feinden, und nach dem Monden, darinnen ihre
Schmerzen in Freude und ihr Leid in gute Tage verkehret war, daß sie
dieselben halten sollten für Tage des Wohllebens und Freuden, und einer
dem andern Geschenke schicken und den Armen mitteilen. \bibverse{23} Und
die Juden nahmen's an, das sie angefangen hatten zu tun, und das
Mardachai zu ihnen schrieb: \bibverse{24} wie Haman, der Sohn Medathas,
der Agagiter, aller Juden Feind, gedacht hatte, alle Juden umzubringen,
und das Los werfen lassen, sie zu schrecken und umzubringen;
\bibverse{25} und wie Esther zum Könige gegangen war und geredet, daß
durch Briefe seine bösen Anschläge, die er wider die Juden gedacht, auf
seinen Kopf gekehret würden; und wie man ihn und seine Söhne an den Baum
gehänget hätte. \bibverse{26} Daher sie diese Tage Purim nannten nach
dem Namen des Loses, nach allen Worten dieses Briefes, und was sie
selbst gesehen hatten, und was an sie gelanget war. \bibverse{27} Und
die Juden richteten es auf und nahmen es auf sich und auf ihren Samen
und auf alle, die sich zu ihnen taten, daß sie nicht übergehen wollten,
zu halten diese zween Tage jährlich, wie die beschrieben und bestimmt
wurden, \bibverse{28} daß diese Tage nicht zu vergessen, sondern zu
halten seien bei Kindeskindern, bei allen Geschlechtern in allen Ländern
und Städten. Es sind die Tage Purim, welche nicht sollen übergangen
werden unter den Juden, und ihr Gedächtnis nicht umkommen bei ihrem
Samen. \bibverse{29} Und die Königin Esther, die Tochter Abihails, und
Mardachai, der Jude, schrieben mit ganzer Gewalt, zu bestätigen diesen
andern Brief von Purim. \bibverse{30} Und sandte die Briefe zu allen
Juden in den hundertundsiebenundzwanzig Ländern des Königreichs
Ahasveros mit freundlichen und treuen Worten: \bibverse{31} daß sie
bestätigten diese Tage Purim auf ihre bestimmte Zeit, wie Mardachai, der
Jude, über sie bestätiget hatte, und die Königin Esther; wie sie auf
ihre Seele und auf ihren Samen bestätiget hatten die Geschichte der
Fasten und ihres Schreiens. \bibverse{32} Und Esther befahl, die
Geschichte dieser Purim zu bestätigen und in ein Buch zu schreiben.

\hypertarget{section-9}{%
\section{10}\label{section-9}}

\bibverse{1} Und der König Ahasveros legte Zins auf das Land und auf die
Inseln im Meer. \bibverse{2} Aber alle Werke seiner Gewalt und Macht und
die große Herrlichkeit Mardachais, die ihm der König gab, siehe, das ist
geschrieben in der Chronik der Könige in Medien und Persien.
\bibverse{3} Denn Mardachai, der Jude, war der andere nach dem Könige
Ahasveros und groß unter den Juden und angenehm unter der Menge seiner
Brüder, der für sein Volk Gutes suchte und redete das Beste für allen
seinen Samen.
