\hypertarget{section}{%
\section{1}\label{section}}

\bibverse{1} Dies ist die Last, welche der Prophet Habakuk gesehen hat.
\bibverse{2} HErr, wie lange soll ich schreien, und du willst nicht
hören? Wie lange soll ich zu dir rufen über Frevel, und du willst nicht
helfen? \bibverse{3} Warum lässest du mich sehen Mühe und Arbeit? Warum
zeigest du mir Raub und Frevel um mich? Es gehet Gewalt über Recht.
\bibverse{4} Darum gehet es gar anders denn recht und kann keine rechte
Sache gewinnen; denn der Gottlose übervorteilt den Gerechten, darum
gehen verkehrte Urteile. \bibverse{5} Schauet unter den Heiden, sehet
und verwundert euch; denn ich will etwas tun zu euren Zeiten, welches
ihr nicht glauben werdet, wenn man davon sagen wird. \bibverse{6} Denn
siehe, ich will die Chaldäer erwecken, ein bitter und schnell Volk,
welches ziehen wird, soweit das Land ist, Wohnungen einzunehmen, die
nicht sein sind, \bibverse{7} und wird grausam und schrecklich sein, das
da gebeut und zwinget, wie es will. \bibverse{8} Ihre Rosse sind
schneller denn die Parden; so sind sie auch beißiger denn die Wölfe des
Abends. Ihre Reiter ziehen mit großen Haufen von ferne daher, als flögen
sie, wie die Adler eilen zum Aas. \bibverse{9} Sie kommen allesamt, daß
sie Schaden tun; wo sie hin wollen, reißen sie hindurch wie ein Ostwind
und werden Gefangene zusammenraffen wie Sand. \bibverse{10} Sie werden
der Könige spotten und der Fürsten werden sie lachen. Alle Festungen
werden ihnen ein Scherz sein; denn sie werden Schutt machen und sie doch
gewinnen. \bibverse{11} Alsdann werden sie einen neuen Mut nehmen,
werden fortfahren und sich versündigen; dann muß ihr Sieg ihres Gottes
sein. \bibverse{12} Aber du, HErr, mein GOtt, mein Heiliger, der du von
Ewigkeit her bist, laß uns nicht sterben, sondern laß sie uns, o HErr,
nur eine Strafe sein und laß sie, o unser Hort, uns nur züchtigen!
\bibverse{13} Deine Augen sind rein, daß du Übels nicht sehen magst, und
dem Jammer kannst du nicht zusehen. Warum siehest du denn zu den
Verächtern und schweigest, daß der Gottlose verschlinget den, der
frömmer denn er ist, \bibverse{14} und lässet die Menschen gehen wie
Fische im Meer, wie Gewürm, das keinen, Herrn hat? \bibverse{15} Sie
ziehen's alles mit dem Hamen und fahen's mit ihrem Netze und sammeln's
mit ihrem Garn; des freuen sie sich und sind fröhlich. \bibverse{16}
Darum opfern sie ihrem Netze und räuchern ihrem Garn, weil durch
dieselbigen ihr Teil so fett und ihre Speise so völlig worden ist.
\bibverse{17} Derhalben werfen sie ihr Netz noch immer aus und wollen
nicht aufhören, Leute zu erwürgen.

\hypertarget{section-1}{%
\section{2}\label{section-1}}

\bibverse{1} Hie stehe ich auf meiner Hut und trete auf meine Feste und
schaue und sehe zu, was mir gesagt werde, und was ich antworten solle
dem, der mich schilt. \bibverse{2} Der HErr aber antwortet mir und
spricht: Schreibe das Gesicht und male es auf eine Tafel, daß es lesen
könne, wer vorüberläuft (nämlich also): \bibverse{3} Die Weissagung wird
ja noch erfüllet werden zu seiner Zeit und wird endlich frei an Tag
kommen und nicht außen bleiben. Ob sie aber verzeucht, so harre ihrer;
sie wird gewißlich kommen und nicht verziehen. \bibverse{4} Siehe, wer
halsstarrig ist, der wird keine Ruhe in seinem Herzen haben; denn der
Gerechte lebet seines Glaubens. \bibverse{5} Aber der Wein betrügt den
stolzen Mann, daß er nicht bleiben kann, welcher seine Seele aufsperret
wie die Hölle, und ist gerade wie der Tod, der nicht zu sättigen ist,
sondern rafft zu sich alle Heiden und sammelt zu sich alle Völker.
\bibverse{6} Was gilt's aber? Dieselbigen alle werden einen Spruch von
ihm machen und eine Sage und Sprichwort und werden sagen: Wehe dem, der
sein Gut mehret mit fremdem Gut! Wie lange wird's währen? und ladet nur
viel Schlammes auf sich. \bibverse{7} O wie plötzlich werden aufwachen,
die dich beißen, und erwachen, die dich wegstoßen! Und du mußt ihnen
zuteil werden. \bibverse{8} Denn du hast viel Heiden geraubt; so werden
dich wieder rauben alle übrigen von den Völkern um der Menschen Bluts
willen und um des Frevels willen, im Lande und in der Stadt und an
allen, die drinnen wohnen, begangen. \bibverse{9} Wehe dem, der da
geizet zum Unglück seines Hauses, auf daß er sein Nest in die Höhe lege,
daß er dem Unfall entrinne! \bibverse{10} Aber dein Ratschlag wird zur
Schande deines Hauses geraten; denn du hast zu viel Völker zerschlagen
und hast mit allem Mutwillen gesündiget. \bibverse{11} Denn auch die
Steine in der Mauer werden schreien, und die Balken am Gesperre werden
ihnen antworten. \bibverse{12} Wehe dem, der die Stadt mit Blut bauet
und zurichtet die Stadt mit Unrecht! \bibverse{13} Ist's nicht also, daß
vom HErrn Zebaoth geschehen wird? Was dir die Völker gearbeitet haben,
muß mit Feuer verbrennen, und daran die Leute müde worden sind, muß
verloren sein. \bibverse{14} Denn die Erde wird voll werden von
Erkenntnis der Ehre des HErrn, wie Wasser, das das Meer bedeckt.
\bibverse{15} Wehe dir, der du deinem Nächsten einschenkest und mischest
deinen Grimm darunter und trunken machest, daß du seine Scham sehest!
\bibverse{16} Man wird dich auch sättigen mit Schande für Ehre. So saufe
du nun auch, daß du taumelst; denn dich wird umgeben der Kelch in der
Rechten des HErrn, und mußt schändlich speien für deine Herrlichkeit.
\bibverse{17} Denn der Frevel, am Libanon begangen, wird dich
überfallen, und die verstörten Tiere werden dich schrecken um der
Menschen Bluts willen und um des Frevels willen, im Lande und in der
Stadt und an allen, die drinnen wohnen, begangen. \bibverse{18} Was wird
dann helfen das Bild, das sein Meister gebildet hat, und das falsche
gegossene Bild, darauf sich verläßt sein Meister, daß er stumme Götzen
machte? \bibverse{19} Wehe dem, der zum Holz spricht: Wache auf! und zum
stummen Stein: Stehe auf! Wie sollt es lehren? Siehe, es ist mit Gold
und Silber überzogen, und ist kein Odem in ihm. \bibverse{20} Aber der
HErr ist in seinem heiligen Tempel. Es sei vor ihm stille alle Welt!

\hypertarget{section-2}{%
\section{3}\label{section-2}}

\bibverse{1} Dies ist das Gebet des Propheten Habakuk für die
Unschuldigen: \bibverse{2} HErr, ich habe dein Gerücht gehöret, daß ich
mich entsetze. HErr, du machst dein Werk lebendig mitten in den Jahren
und lässest es kund werden mitten in den Jahren. Wenn Trübsal da ist, so
denkest du der Barmherzigkeit. \bibverse{3} GOtt kam vom Mittage und der
Heilige vom Gebirge Paran. Sela. Seines Lobes war der Himmel voll und
seiner Ehre war die Erde voll. \bibverse{4} Sein Glanz war wie Licht;
Glänze gingen von seinen Händen; daselbst war heimlich seine Macht.
\bibverse{5} Vor ihm her ging Pestilenz, und Plage ging aus, wo er
hintrat. \bibverse{6} Er stund und maß das Land; er schauete und
zertrennete die Heiden, daß der Welt Berge zerschmettert wurden und sich
bücken mußten die Hügel in der Welt, da er ging in der Welt.
\bibverse{7} Ich sah der Mohren Hütten in Mühe und der Midianiter
Gezelte betrübt. \bibverse{8} Warest du nicht zornig, HErr, in der Flut
und dein Grimm in den Wassern und dein Zorn im Meer, da du auf deinen
Rossen rittest und deine Wagen den Sieg behielten? \bibverse{9} Du
zogest den Bogen hervor, wie du geschworen hattest den Stämmen, Sela,
und teiltest die Ströme ins Land. \bibverse{10} Die Berge sahen dich,
und ihnen ward bange; der Wasserstrom fuhr dahin, die Tiefe ließ sich
hören, die Höhe hub die Hände auf. \bibverse{11} Sonne und Mond stunden
still. Deine Pfeile fuhren mit Glänzen dahin und deine Speere mit
Blicken des Blitzes. \bibverse{12} Du zertratest das Land im Zorn und
zerdroschest die Heiden im Grimm. \bibverse{13} Du zogest aus, deinem
Volk zu helfen, zu helfen deinem Gesalbten. Du zerschmissest das Haupt
im Hause des Gottlosen und entblößetest die Grundfeste bis an den Hals.
Sela. \bibverse{14} Du wolltest fluchen dem Zepter des Haupts samt
seinen Flecken, die wie ein Wetter kommen, mich zu zerstreuen, und
freuen sich, als fräßen sie den Elenden verborgen. \bibverse{15} Deine
Pferde gehen im Meer, im Schlamm großer Wasser. \bibverse{16} Weil ich
solches höre, ist mein Bauch betrübt, meine Lippen zittern von dem
Geschrei; Eiter gehet in meine Gebeine, ich bin bei mir betrübt. O daß
ich ruhen möchte zur Zeit der Trübsal, da wir hinaufziehen zum Volk, das
uns bestreitet. \bibverse{17} Denn der Feigenbaum wird nicht grünen, und
wird kein Gewächs sein an den Weinstöcken; die Arbeit am Ölbaum fehlet
und die Äcker bringen keine Nahrung, und Schafe werden aus den Hürden
gerissen, und werden keine Rinder in den Ställen sein. \bibverse{18}
Aber ich will mich freuen des HErrn und fröhlich sein in GOtt, meinem
Heil. \bibverse{19} Denn der HErr HErr ist meine Kraft und wird meine
Füße machen wie Hirschfüße und wird mich in der Höhe führen, daß ich
singe auf meinem Saitenspiel.
