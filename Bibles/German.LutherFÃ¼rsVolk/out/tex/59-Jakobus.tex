\hypertarget{section}{%
\section{1}\label{section}}

\bibverse{1} Jakobus, ein Knecht Gottes und des Herrn Jesu Christi, den
zwölf Geschlechtern, die da sind hin und her, Freude zuvor!

\bibverse{2} Meine lieben Brüder, achtet es für eitel Freude, wenn ihr
in mancherlei Anfechtungen fallet, \footnote{\textbf{1:2} Röm 5,3-5;
  1Petr 4,13} \bibverse{3} und wisset, dass euer Glaube, wenn er
rechtschaffen ist, Geduld wirkt. \bibverse{4} Die Geduld aber soll
festbleiben bis ans Ende, auf dass ihr seid vollkommen und ganz und
keinen Mangel habet.

\bibverse{5} So aber jemand unter euch Weisheit mangelt, der bitte Gott,
der da gibt einfältig jedermann und rücket's niemand auf, so wird sie
ihm gegeben werden. \bibverse{6} Er bitte aber im Glauben und zweifle
nicht; denn wer da zweifelt, der ist gleich wie die Meereswoge, die vom
Winde getrieben und gewebt wird. \footnote{\textbf{1:6} Mk 11,24; 1Tim
  2,8} \bibverse{7} Solcher Mensch denke nicht, dass er etwas von dem
Herrn empfangen werde. \bibverse{8} Ein Zweifler ist unbeständig in
allen seinen Wegen.

\bibverse{9} Ein Bruder aber, der niedrig ist, rühme sich seiner Höhe;
\bibverse{10} und der da reich ist, rühme sich seiner Niedrigkeit, denn
wie eine Blume des Grases wird er vergehen. \footnote{\textbf{1:10}
  1Petr 1,24; 1Tim 6,17} \bibverse{11} Die Sonne geht auf mit der Hitze,
und das Gras verwelkt, und seine Blume fällt ab, und seine schöne
Gestalt verdirbt: also wird der Reiche in seinen Wegen verwelken.
\footnote{\textbf{1:11} Jes 40,6-7}

\bibverse{12} Selig ist der Mann, der die Anfechtung erduldet; denn
nachdem er bewährt ist, wird er die Krone des Lebens empfangen, welche
Gott verheißen hat denen, die ihn liebhaben. \footnote{\textbf{1:12}
  2Tim 4,8}

\bibverse{13} Niemand sage, wenn er versucht wird, dass er von Gott
versucht werde. Denn Gott kann nicht versucht werden zum Bösen, und er
selbst versucht niemand. \bibverse{14} Sondern ein jeglicher wird
versucht, wenn er von seiner eigenen Lust gereizt und gelockt wird.
\bibverse{15} Darnach, wenn die Lust empfangen hat, gebiert sie die
Sünde; die Sünde aber, wenn sie vollendet ist, gebiert sie den Tod.
\footnote{\textbf{1:15} Röm 7,10} \bibverse{16} Irret nicht, liebe
Brüder. \bibverse{17} Alle gute Gabe und alle vollkommene Gabe kommt von
obenherab, von dem Vater des Lichts, bei welchem ist keine Veränderung
noch Wechsel des Lichtes und der Finsternis. \bibverse{18} Er hat uns
gezeugt nach seinem Willen durch das Wort der Wahrheit, auf dass wir
wären Erstlinge seiner Kreaturen. \footnote{\textbf{1:18} Joh 1,13;
  1Petr 1,23}

\bibverse{19} Darum, liebe Brüder, ein jeglicher Mensch sei schnell, zu
hören, langsam aber, zu reden, und langsam zum Zorn. \footnote{\textbf{1:19}
  Spr 29,20; Pred 5,1-2; Pred 7,9} \bibverse{20} Denn des Menschen Zorn
tut nicht, was vor Gott recht ist. \footnote{\textbf{1:20} Spr 29,22;
  Eph 4,26} \bibverse{21} Darum so leget ab alle Unsauberkeit und alle
Bosheit und nehmet das Wort an mit Sanftmut, das in euch gepflanzt ist,
welches kann eure Seelen selig machen. \footnote{\textbf{1:21} 1Petr 2,1}

\bibverse{22} Seid aber Täter des Worts und nicht Hörer allein, wodurch
ihr euch selbst betrüget. \footnote{\textbf{1:22} Mt 7,26; Röm 2,13}
\bibverse{23} Denn wenn jemand ist ein Hörer des Worts und nicht ein
Täter, der ist gleich einem Mann, der sein leiblich Angesicht im Spiegel
beschaut. \bibverse{24} Denn nachdem er sich beschaut hat, geht er davon
und vergisst von Stund an, wie er gestaltet war. \bibverse{25} Wer aber
durchschaut in das vollkommene Gesetz der Freiheit und darin beharrt und
ist nicht ein vergesslicher Hörer, sondern ein Täter, der wird selig
sein in seiner Tat.

\bibverse{26} Wenn sich jemand unter euch lässt dünken, er diene Gott,
und hält seine Zunge nicht im Zaum, sondern täuscht sein Herz, des
Gottesdienst ist eitel. \footnote{\textbf{1:26} 1Petr 3,10}
\bibverse{27} Ein reiner und unbefleckter Gottesdienst vor Gott dem
Vater ist der: Die Waisen und Witwen in ihrer Trübsal besuchen und sich
von der Welt unbefleckt erhalten. \# 2 \bibverse{1} Liebe Brüder, haltet
nicht dafür, dass der Glaube an Jesum Christum, unseren Herrn der
Herrlichkeit, Ansehung der Person leide. \bibverse{2} Denn wenn in eure
Versammlung käme ein Mann mit einem goldenen Ringe und mit einem
herrlichen Kleide, es käme aber auch ein Armer in einem unsauberen
Kleide, \bibverse{3} und ihr sähet auf den, der das herrliche Kleid
trägt, und sprächet zu ihm: Setze du dich her aufs beste! und sprächet
zu dem Armen: Stehe du dort! oder: Setze dich her zu meinen Füßen!
\bibverse{4} ist's recht, dass ihr solchen Unterschied bei euch selbst
macht und richtet nach argen Gedanken? \bibverse{5} Höret zu, meine
lieben Brüder! Hat nicht Gott erwählt die Armen auf dieser Welt, die am
Glauben reich sind und Erben des Reichs, welches er verheißen hat denen,
die ihn liebhaben? \bibverse{6} Ihr aber habt dem Armen Unehre getan.
Sind nicht die Reichen die, die Gewalt an euch üben und ziehen euch vor
Gericht? \bibverse{7} Verlästern sie nicht den guten Namen, nach dem ihr
genannt seid? \footnote{\textbf{2:7} 1Petr 4,14}

\bibverse{8} So ihr das königliche Gesetz erfüllet nach der Schrift:
„Liebe deinen Nächsten wie dich selbst,`` so tut ihr wohl; \bibverse{9}
so ihr aber die Person ansehet, tut ihr Sünde und werdet überführt vom
Gesetz als Übertreter. \bibverse{10} Denn wenn jemand das ganze Gesetz
hält und sündigt an einem, der ist's ganz schuldig. \footnote{\textbf{2:10}
  Mt 5,19} \bibverse{11} Denn der da gesagt hat: „Du sollst nicht
ehebrechen,`` der hat auch gesagt: „Du sollst nicht töten.`` Wenn du nun
nicht ehebrichst, tötest aber, bist du ein Übertreter des Gesetzes.
\bibverse{12} Also redet und also tut, als die da sollen durchs Gesetz
der Freiheit gerichtet werden. \bibverse{13} Es wird aber ein
unbarmherziges Gericht über den ergehen, der nicht Barmherzigkeit getan
hat; und die Barmherzigkeit rühmt sich wider das Gericht.

\bibverse{14} Was hilft's, liebe Brüder, wenn jemand sagt, er habe den
Glauben, und hat doch die Werke nicht? Kann auch der Glaube ihn selig
machen? \footnote{\textbf{2:14} Mt 7,21} \bibverse{15} Wenn aber ein
Bruder oder eine Schwester bloß wäre und Mangel hätte der täglichen
Nahrung, \bibverse{16} und jemand unter euch spräche zu ihnen: Gott
berate euch, wärmet euch und sättiget euch! ihr gäbet ihnen aber nicht,
was des Leibes Notdurft ist: was hülfe ihnen das? \bibverse{17} Also
auch der Glaube, wenn er nicht Werke hat, ist er tot an ihm selber.
\bibverse{18} Aber es möchte jemand sagen: Du hast den Glauben, und ich
habe die Werke; zeige mir deinen Glauben ohne die Werke, so will ich dir
meinen Glauben zeigen aus meinen Werken. \footnote{\textbf{2:18} Gal 5,6}

\bibverse{19} Du glaubst, dass ein einiger Gott ist? Du tust wohl daran;
die Teufel glauben's auch und -- zittern. \bibverse{20} Willst du aber
erkennen, du eitler Mensch, dass der Glaube ohne Werke tot sei?
\bibverse{21} Ist nicht Abraham, unser Vater, durch die Werke gerecht
geworden, da er seinen Sohn Isaak auf dem Altar opferte? \bibverse{22}
Da siehst du, dass der Glaube mitgewirkt hat an seinen Werken, und durch
die Werke ist der Glaube vollkommen geworden; \bibverse{23} und ist die
Schrift erfüllt, die da spricht: „Abraham hat Gott geglaubt, und das ist
ihm zur Gerechtigkeit gerechnet,`` und er ward ein Freund Gottes
geheißen. \bibverse{24} So sehet ihr nun, dass der Mensch durch die
Werke gerecht wird, nicht durch den Glauben allein. \bibverse{25}
Desgleichen die Hure Rahab, ist sie nicht durch die Werke gerecht
geworden, da sie die Boten aufnahm und ließ sie einen anderen Weg
hinaus? \footnote{\textbf{2:25} Jos 2,-1; Hebr 11,31} \bibverse{26} Denn
gleichwie der Leib ohne Geist tot ist, also ist auch der Glaube ohne
Werke tot. \# 3 \bibverse{1} Liebe Brüder, unterwinde sich nicht
jedermann, Lehrer zu sein, und wisset, dass wir desto mehr Urteil
empfangen werden. \bibverse{2} Denn wir fehlen alle mannigfaltig. Wer
aber auch in keinem Wort fehlt, der ist ein vollkommener Mann und kann
auch den ganzen Leib im Zaum halten. \bibverse{3} Siehe, die Pferde
halten wir in Zäumen, dass sie uns gehorchen, und wir lenken ihren
ganzen Leib. \bibverse{4} Siehe, die Schiffe, ob sie wohl so groß sind
und von starken Winden getrieben werden, werden sie doch gelenkt mit
einem kleinen Ruder, wo der hin will, der es regiert. \bibverse{5} Also
ist auch die Zunge ein kleines Glied und richtet große Dinge an. Siehe,
ein kleines Feuer, welch einen Wald zündet's an! \bibverse{6} Und die
Zunge ist auch ein Feuer, eine Welt voll Ungerechtigkeit. Also ist die
Zunge unter unseren Gliedern und befleckt den ganzen Leib und zündet an
allen unseren Wandel, wenn sie von der Hölle entzündet ist. \bibverse{7}
Denn alle Natur der Tiere und der Vögel und der Schlangen und der
Meerwunder wird gezähmt und ist gezähmt von der menschlichen Natur;
\bibverse{8} aber die Zunge kann kein Mensch zähmen, das unruhige Übel,
voll tödlichen Giftes. \bibverse{9} Durch sie loben wir Gott, den Vater,
und durch sie fluchen wir den Menschen, die nach dem Bilde Gottes
gemacht sind. \footnote{\textbf{3:9} 1Mo 1,27} \bibverse{10} Aus einem
Munde geht Loben und Fluchen. Es soll nicht, liebe Brüder, also sein.
\footnote{\textbf{3:10} Eph 4,29} \bibverse{11} Quillt auch ein Brunnen
aus einem Loch süß und bitter? \bibverse{12} Kann auch, liebe Brüder,
ein Feigenbaum Ölbeeren oder ein Weinstock Feigen tragen? Also kann auch
ein Brunnen nicht salziges und süßes Wasser geben.

\bibverse{13} Wer ist weise und klug unter euch? Der erzeige mit seinem
guten Wandel seine Werke in der Sanftmut und Weisheit. \bibverse{14}
Habt ihr aber bitteren Neid und Zank in eurem Herzen, so rühmet euch
nicht und lüget nicht wider die Wahrheit. \bibverse{15} Das ist nicht
die Weisheit, die von obenherab kommt, sondern irdisch, menschlich und
teuflisch. \footnote{\textbf{3:15} Jak 1,5} \bibverse{16} Denn wo Neid
und Zank ist, da ist Unordnung und eitel böses Ding. \bibverse{17} Die
Weisheit aber von obenher ist aufs erste keusch, darnach friedsam,
gelinde, lässt sich sagen, voll Barmherzigkeit und guter Früchte,
unparteiisch, ohne Heuchelei. \bibverse{18} Die Frucht aber der
Gerechtigkeit wird gesät im Frieden denen, die den Frieden halten. \# 4
\bibverse{1} Woher kommt Streit und Krieg unter euch? Kommt's nicht
daher: aus euren Wollüsten, die da streiten in euren Gliedern?
\bibverse{2} Ihr seid begierig, und erlanget's damit nicht; ihr hasset
und neidet, und gewinnet damit nichts; ihr streitet und krieget. Ihr
habt nicht, darum dass ihr nicht bittet; \footnote{\textbf{4:2} Gal 5,15}
\bibverse{3} ihr bittet, und nehmet nicht, darum dass ihr übel bittet,
nämlich dahin, dass ihr's mit euren Wollüsten verzehret. \bibverse{4}
Ihr Ehebrecher und Ehebrecherinnen, wisset ihr nicht, dass der Welt
Freundschaft Gottes Feindschaft ist? Wer der Welt Freund sein will, der
wird Gottes Feind sein. \bibverse{5} Oder lasset ihr euch dünken, die
Schrift sage umsonst: Der Geist, der in euch wohnt, begehrt und eifert?
\footnote{\textbf{4:5} 2Mo 20,3; 2Mo 20,5} \bibverse{6} Er gibt aber
desto reichlicher Gnade. Darum sagt sie: „Gott widerstehet den
Hoffärtigen, aber den Demütigen gibt er Gnade.`` \footnote{\textbf{4:6}
  Hi 22,29; Mt 23,12; 1Petr 5,5} \bibverse{7} So seid nun Gott
untertänig. Widerstehet dem Teufel, so fliehet er von euch; \footnote{\textbf{4:7}
  1Petr 5,8-9} \bibverse{8} nahet euch zu Gott, so naht er sich zu euch.
Reiniget die Hände, ihr Sünder, und machet eure Herzen keusch, ihr
Wankelmütigen. \footnote{\textbf{4:8} Sach 1,3; Jes 1,16} \bibverse{9}
Seid elend und traget Leid und weinet; euer Lachen verkehre sich in
Weinen und eure Freude in Traurigkeit. \bibverse{10} Demütiget euch vor
Gott, so wir er euch erhöhen. \footnote{\textbf{4:10} 1Petr 5,6}

\bibverse{11} Afterredet nicht untereinander, liebe Brüder. Wer seinem
Bruder afterredet und richtet seinen Bruder, der afterredet dem Gesetz
und richtet das Gesetz. Richtest du aber das Gesetz, so bist du nicht
ein Täter des Gesetzes, sondern ein Richter. \bibverse{12} Es ist ein
einiger Gesetzgeber, der kann selig machen und verdammen. Wer bist du,
der du einen anderen richtest?

\bibverse{13} Wohlan nun, die ihr sagt: Heute oder morgen wollen wir
gehen in die oder die Stadt und wollen ein Jahr da liegen und Handel
treiben und gewinnen; \footnote{\textbf{4:13} Spr 27,1} \bibverse{14}
die ihr nicht wisset, was morgen sein wird. Denn was ist euer Leben? Ein
Dampf ist's, der eine kleine Zeit währt, darnach aber verschwindet er.
\footnote{\textbf{4:14} Lk 12,20} \bibverse{15} Dafür ihr sagen solltet:
So der Herr will und wir leben, wollen wir dies oder das tun.
\footnote{\textbf{4:15} Apg 18,21; 1Kor 4,19} \bibverse{16} Nun aber
rühmet ihr euch in eurem Hochmut. Aller solcher Ruhm ist böse.
\bibverse{17} Denn wer da weiß Gutes zu tun, und tut's nicht, dem ist's
Sünde. \# 5 \bibverse{1} Wohlan nun, ihr Reichen, weinet und heulet über
euer Elend, das über euch kommen wird! \footnote{\textbf{5:1} Lk 6,24-25}
\bibverse{2} Euer Reichtum ist verfault, eure Kleider sind mottenfräßig
geworden. \footnote{\textbf{5:2} Mt 6,19} \bibverse{3} Euer Gold und
Silber ist verrostet, und sein Rost wird euch zum Zeugnis sein und wird
euer Fleisch fressen wie ein Feuer. Ihr habt euch Schätze gesammelt in
den letzten Tagen. \bibverse{4} Siehe, der Arbeiter Lohn, die euer Land
eingeerntet haben, der von euch abgebrochen ist, der schreit, und das
Rufen der Ernter ist gekommen vor die Ohren des Herrn Zebaoth.
\footnote{\textbf{5:4} 5Mo 24,14-15} \bibverse{5} Ihr habt wohlgelebt
auf Erden und eure Wollust gehabt und eure Herzen geweidet am
Schlachttag. \footnote{\textbf{5:5} Lk 16,19; Lk 16,25; Jer 12,3; Jer
  25,34} \bibverse{6} Ihr habt verurteilt den Gerechten und getötet, und
er hat euch nicht widerstanden. \footnote{\textbf{5:6} Jak 2,6}

\bibverse{7} So seid nun geduldig, liebe Brüder, bis auf die Zukunft des
Herrn. Siehe, ein Ackermann wartet auf die köstliche Frucht der Erde und
ist geduldig darüber, bis sie empfange den Frühregen und den Spätregen.
\footnote{\textbf{5:7} Lk 21,19; Hebr 10,36} \bibverse{8} Seid ihr auch
geduldig und stärket eure Herzen; denn die Zukunft des Herrn ist nahe.

\bibverse{9} Seufzet nicht widereinander, liebe Brüder, auf dass ihr
nicht verdammt werdet. Siehe, der Richter ist vor der Tür. \bibverse{10}
Nehmet, meine lieben Brüder, zum Exempel des Leidens und der Geduld die
Propheten, die geredet haben in dem Namen des Herrn. \footnote{\textbf{5:10}
  Mt 5,12} \bibverse{11} Siehe, wir preisen selig, die erduldet haben.
Die Geduld Hiobs habt ihr gehört, und das Ende des Herrn habt ihr
gesehen; denn der Herr ist barmherzig und ein Erbarmer. \footnote{\textbf{5:11}
  Hi 1,21; Hi 42,10-16}

\bibverse{12} Vor allen Dingen aber, meine Brüder, schwöret nicht, weder
bei dem Himmel noch bei der Erde noch mit einem anderen Eid. Es sei aber
euer Wort: Ja, das Ja ist; und: Nein, das Nein ist, auf dass ihr nicht
unter ein Gericht fallet. \footnote{\textbf{5:12} Mt 5,34-37}

\bibverse{13} Leidet jemand unter euch, der bete; ist jemand gutes Muts,
der singe Psalmen. \footnote{\textbf{5:13} Ps 50,15; Kol 3,16}
\bibverse{14} Ist jemand krank, der rufe zu sich die Ältesten von der
Gemeinde, dass sie über ihm beten und ihn salben mit Öl in dem Namen des
Herrn. \footnote{\textbf{5:14} Mk 6,13} \bibverse{15} Und das Gebet des
Glaubens wird dem Kranken helfen, und der Herr wird ihn aufrichten; und
wenn er hat Sünden getan, werden sie ihm vergeben sein. \footnote{\textbf{5:15}
  Mk 16,18} \bibverse{16} Bekenne einer dem anderen seine Sünden und
betet füreinander, dass ihr gesund werdet. Des Gerechten Gebet vermag
viel, wenn es ernstlich ist. \footnote{\textbf{5:16} Apg 12,5}
\bibverse{17} Elia war ein Mensch gleich wie wir; und er betete ein
Gebet, dass es nicht regnen sollte, und es regnete nicht auf Erden drei
Jahre und sechs Monate. \footnote{\textbf{5:17} 1Kö 17,1; Lk 4,25}
\bibverse{18} Und er betete abermals, und der Himmel gab den Regen, und
die Erde brachte ihre Frucht. \footnote{\textbf{5:18} 1Kö 18,41-45}

\bibverse{19} Liebe Brüder, wenn jemand unter euch irren würde von der
Wahrheit, und jemand bekehrte ihn, \^{}\^{} \bibverse{20} der soll
wissen, dass, wer den Sünder bekehrt hat von dem Irrtum seines Weges,
der hat einer Seele vom Tode geholfen und wird bedecken die Menge der
Sünden.
