\hypertarget{section}{%
\section{1}\label{section}}

\bibverse{1} Und der HErr rief Mose und redete mit ihm von der Hütte des
Stifts und sprach: \bibverse{2} Rede mit den Kindern Israel und sprich
zu ihnen: Welcher unter euch dem HErrn ein Opfer tun will, der tue es
von dem Vieh, von Rindern und Schafen. \bibverse{3} Will er ein
Brandopfer tun von Rindern, so opfere er ein Männlein, das ohne Wandel
sei, vor der Tür der Hütte des Stifts, daß es dem HErrn angenehm sei von
ihm, \bibverse{4} und lege seine Hand auf des Brandopfers Haupt, so wird
es angenehm sein und ihn versöhnen. \bibverse{5} Und soll das junge Rind
schlachten vor dem HErrn; und die Priester, Aarons Söhne, sollen das
Blut herzubringen und auf den Altar umher sprengen, der vor der Tür der
Hütte des Stifts ist. \bibverse{6} Und man soll dem Brandopfer die Haut
abziehen; und es soll in Stücke zerhauen werden. \bibverse{7} Und die
Söhne Aarons, des Priesters, sollen ein Feuer auf den Altar machen und
Holz oben drauf legen; \bibverse{8} Und sollen die Stücke, nämlich den
Kopf und das Fett, auf das Holz legen, das auf dem Feuer auf dem Altar
liegt. \bibverse{9} Das Eingeweide aber und die Schenkel soll man mit
Wasser waschen, und der Priester soll das alles anzünden auf dem Altar
zum Brandopfer. Das ist ein Feuer zum süßen Geruch dem HErrn.
\bibverse{10} Will er aber von Schafen oder Ziegen ein Brandopfer tun,
so opfere er ein Männlein, das ohne Wandel sei. \bibverse{11} Und soll
es schlachten zur Seite des Altars, gegen Mitternacht, vor dem HErrn.
Und die Priester, Aarons Söhne, sollen sein Blut auf den Altar umher
sprengen. \bibverse{12} Und man soll es in Stücke zerhauen. Und der
Priester soll den Kopf und das Fett auf das Holz und Feuer, das auf dem
Altar ist, legen. \bibverse{13} Aber das Eingeweide und die Schenkel
soll man mit Wasser waschen. Und der Priester soll es alles opfern und
anzünden auf dem Altar zum Brandopfer. Das ist ein Feuer zum süßen
Geruch dem HErrn. \bibverse{14} Will er aber von Vögeln dem HErrn ein
Brandopfer tun, so tue er's von Turteltauben oder von jungen Tauben.
\bibverse{15} Und der Priester soll's zum Altar bringen und ihm den Kopf
abkneipen, daß es auf dem Altar angezündet werde, und sein Blut
ausbluten lassen an der Wand des Altars. \bibverse{16} Und seinen Kropf
mit seinen Federn soll man neben dem Altar gegen den Morgen auf den
Aschenhaufen werfen. \bibverse{17} Und soll seine Flügel spalten, aber
nicht abbrechen. Und also soll es der Priester auf dem Altar anzünden,
auf dem Holz auf dem Feuer, zum Brandopfer. Das ist ein Feuer zum süßen
Geruch dem HErrn.

\hypertarget{section-1}{%
\section{2}\label{section-1}}

\bibverse{1} Wenn eine SeeLev dem HErrn ein Speisopfer tun will, so soll
es von Semmelmehl sein; und soll Öl drauf gießen und Weihrauch drauf
legen \bibverse{2} und also bringen zu den Priestern, Aarons Söhnen. Da
soll der Priester seine Hand voll nehmen von demselben Semmelmehl und Öl
samt dem ganzen Weihrauch und anzünden zum Gedächtnis auf dem Altar. Das
ist ein Feuer zum süßen Geruch dem HErrn. \bibverse{3} Das Übrige aber
vom Speisopfer soll Aarons und seiner Söhne sein. Das soll das
Allerheiligste sein von den Feuern des HErrn. \bibverse{4} Will er aber
kein Speisopfer tun vom Gebackenen im Ofen, so nehme er Kuchen von
Semmelmehl ungesäuert mit Öl gemenget, und ungesäuerte Fladen, mit Öl
bestrichen. \bibverse{5} Ist aber dein Speisopfer etwas vom Gebackenen
in der Pfanne, so soll's von ungesäuertem Semmelmehl, mit Öl gemenget,
sein. \bibverse{6} Und sollst es in Stücke zerteilen und Öl drauf
gießen, so ist's ein Speisopfer. \bibverse{7} Ist aber dein Speisopfer
etwas auf dem Rost geröstet, so sollst du es von Semmelmehl mit Öl
machen. \bibverse{8} Und sollst das Speisopfer, das du von solcherlei
machen willst dem HErrn, zu dem Priester bringen; der soll's zu dem
Altar bringen \bibverse{9} und desselben Speisopfer heben zum Gedächtnis
und anzünden auf dem Altar. Das ist ein Feuer zum süßen Geruch dem
HErrn. \bibverse{10} Das Übrige aber soll Aarons und seiner Söhne sein.
Das soll das Allerheiligste sein von den Feuern des HErrn. \bibverse{11}
AlLev Speisopfer, die ihr dem HErrn opfern wollt, sollt ihr ohne
Sauerteig machen; denn kein Sauerteig noch Honig soll darunter dem HErrn
zum Feuer angezündet werden. \bibverse{12} Aber zum Erstling sollt ihr
sie dem HErrn bringen; aber auf keinen Altar sollen sie kommen zum süßen
Geruch. \bibverse{13} AlLev deine Speisopfer sollst du salzen, und dein
Speisopfer soll nimmer ohne Salz des Bundes deines GOttes sein; denn in
alLev deinem Opfer sollst du Salz opfern. \bibverse{14} Willst du aber
ein Speisopfer dem HErrn tun von den ersten Früchten, sollst du die
Sangen, am Feuer gedörret, klein zerstoßen und also das Speisopfer
deiner ersten Früchte opfern; \bibverse{15} und sollst Öl drauf tun und
Weihrauch drauf legen, so ist's ein Speisopfer. \bibverse{16} Und der
Priester soll von dem Zerstoßenen und vom Öl mit dem ganzen Weihrauch
anzünden zum Gedächtnis. Das ist ein Feuer dem HErrn.

\hypertarget{section-2}{%
\section{3}\label{section-2}}

\bibverse{1} Ist aber sein Opfer ein Dankopfer von Rindern, es sei ein
Ochse oder Kuh, soll er's opfern vor dem HErrn, das ohne Wandel sei.
\bibverse{2} Und soll seine Hand auf desselben Haupt legen und
schlachten vor der Tür der Hütte des Stifts. Und die Priester, Aarons
Söhne, sollen das Blut auf den Altar umher sprengen. \bibverse{3} Und
soll von dem Dankopfer dem HErrn opfern, nämlich alles Fett am
Eingeweide \bibverse{4} und die Nieren mit dem Fett, das dran ist, an
den Lenden, und das Netz um die Leber, an den Nieren abgerissen.
\bibverse{5} Und Aarons Söhne sollen's anzünden auf dem Altar zum
Brandopfer, auf dem Holz, das auf dem Feuer liegt. Das ist ein Feuer zum
süßen Geruch dem HErrn. \bibverse{6} Will er aber dem HErrn ein
Dankopfer von kleinem Vieh tun, es sei ein Schöps oder Schaf, so soll's
ohne Wandel sein. \bibverse{7} Ist's ein Lämmlein soll er's vor den
HErrn bringen \bibverse{8} und soll seine Hand auf desselben Haupt legen
und schlachten vor der Hütte des Stifts. Und die Söhne Aarons sollen
sein Blut auf den Altar umher sprengen. \bibverse{9} Und soll also von
dem Dankopfer dem HErrn opfern zum Feuer, nämlich sein Fett, den ganzen
Schwanz, von dem Rücken abgerissen, und alles Fett am Eingeweide,
\bibverse{10} die zwo Nieren mit dem Fett, das dran ist, an den Lenden,
und das Netz um die Leber, an den Nieren abgerissen. \bibverse{11} Und
der Priester soll's anzünden auf dem Altar zur Speise des Feuers dem
HErrn. \bibverse{12} Ist aber sein Opfer eine Ziege, und bringet es vor
den HErrn, \bibverse{13} soll er seine Hand auf ihr Haupt legen und sie
schlachten vor der Hütte des Stifts. Und die Söhne Aarons sollen das
Blut auf den Altar umher sprengen. \bibverse{14} Und soll davon opfern
ein Opfer dem HErrn, nämlich das Fett am Eingeweide, \bibverse{15} die
Nieren mit dem Fett, das dran ist, an den Lenden, und das Netz über der
Leber, an den Nieren abgerissen. \bibverse{16} Und der Priester soll's
anzünden auf dem Altar zur Speise des Feuers, zum süßen Geruch. Alles
Fett ist des HErrn. \bibverse{17} Das sei eine ewige Sitte bei euren
Nachkommen in allen euren Wohnungen, daß ihr kein Fett noch Blut esset.

\hypertarget{section-3}{%
\section{4}\label{section-3}}

\bibverse{1} Und der HErr redete mit Mose und sprach: \bibverse{2} Rede
mit den Kindern Israel und sprich: Wenn eine SeeLev sündigen würde aus
Versehen an irgend einem Gebot des HErrn, das sie nicht tun sollte,
\bibverse{3} nämlich so ein Priester, der gesalbet ist, sündigen würde,
daß er das Volk ärgerte: der soll für seine Sünde, die er getan hat,
einen jungen Farren bringen, der ohne Wandel sei, dem HErrn zum
Sündopfer. \bibverse{4} Und soll den Farren vor die Tür der Hütte des
Stifts bringen, vor den HErrn, und seine Hand auf desselben Haupt legen
und schlachten vor dem HErrn. \bibverse{5} Und der Priester, der
gesalbet ist, soll des Farren Bluts nehmen und in die Hütte des Stifts
bringen. \bibverse{6} Und soll seinen Finger in das Blut tunken und
damit siebenmal sprengen vor dem HErrn, vor dem Vorhang im Heiligen.
\bibverse{7} Und soll desselben Bluts tun auf die Hörner des
Räuchaltars, der vor dem HErrn in der Hütte des Stifts stehet, und alles
Blut gießen an den Boden des Brandopferaltars, der vor der Tür der Hütte
des Stifts stehet. \bibverse{8} Und alles Fett des Sündopfers soll er
heben, nämlich das Fett am Eingeweide, \bibverse{9} die zwo Nieren mit
dem Fett, das dran ist, an den Lenden, und das Netz über der Leber, an
den Nieren abgerissen, \bibverse{10} gleichwie er's hebet vom Ochsen im
Dankopfer; und soll's anzünden auf dem Brandopferaltar. \bibverse{11}
Aber das Fell des Farren mit allem Fleisch, samt dem Kopf, und Schenkel
und das Eingeweide und den Mist, \bibverse{12} das soll er alles
hinausführen außer dem Lager an eine reine Stätte, da man die Asche
hinschüttet, und soll's verbrennen auf dem Holz mit Feuer. \bibverse{13}
Wenn es eine ganze Gemeine in Israel versehen würde, und die Tat vor
ihren Augen verborgen wäre, daß sie irgend wider ein Gebot des HErrn
getan hätten, das sie nicht tun sollten, und sich also verschuldeten,
\bibverse{14} und danach ihrer Sünde inne würden, die sie getan hätten:
sollen sie einen jungen Farren darbringen zum Sündopfer und vor die Tür
der Hütte des Stifts stellen. \bibverse{15} Und die Ältesten von der
Gemeine sollen ihre Hände auf sein Haupt legen vor dem HErrn und den
Farren schlachten vor dem HErrn. \bibverse{16} Und der Priester, der
gesalbet ist, soll des Bluts vom Farren in die Hütte des Stifts bringen
\bibverse{17} und mit seinem Finger drein tunken und siebenmal sprengen
vor dem HErrn, vor dem Vorhang. \bibverse{18} Und soll des Bluts auf die
Hörner des Altars tun, der vor dem HErrn stehet in der Hütte des Stifts,
und alles andere Blut an den Boden des Brandopferaltars gießen, der vor
der Tür der Hütte des Stifts stehet. \bibverse{19} Alles sein Fett aber
soll er heben und auf dem Altar anzünden. \bibverse{20} Und soll mit dem
Farren tun, wie er mit dem Farren des Sündopfers getan hat. Und soll
also der Priester sie versöhnen, so wird's ihnen vergeben. \bibverse{21}
Und soll den Farren außer dem Lager führen und verbrennen wie er den
vorigen Farren verbrannt hat. Das soll das Sündopfer der Gemeine sein.
\bibverse{22} Wenn aber ein Fürst sündiget und irgend wider des HErrn,
seines GOttes, Gebot tut, das er nicht tun sollte, und versiehet es, daß
er sich verschuldet, \bibverse{23} und wird seiner Sünde inne, die er
getan hat: der soll zum Opfer bringen einen Ziegenbock ohne Wandel;
\bibverse{24} und seine Hand auf des Bocks Haupt legen und ihn
schlachten an der Stätte, da man die Brandopfer schlachtet vor dem
HErrn. Das sei ein Sündopfer. \bibverse{25} Da soll denn der Priester
des Bluts von dem Sündopfer nehmen mit seinem Finger und auf die Hörner
des Brandopferaltars tun und das andere Blut an den Boden des
Brandopferaltars gießen. \bibverse{26} Aber alles sein Fett soll er auf
dem Altar anzünden, gleichwie das Fett des Dankopfers. Und soll also der
Priester seine Sünde versöhnen, so wird's ihm vergeben. \bibverse{27}
Wenn es aber eine SeeLev vom gemeinen Volk versiehet und sündiget, daß
sie irgend wider der Gebote des HErrn eines tut, das sie nicht tun
sollte, und sich also verschuldet, \bibverse{28} und ihrer Sünde inne
wird, die sie getan hat: die soll zum Opfer eine Ziege bringen ohne
Wandel für die Sünde, die sie getan hat; \bibverse{29} und soll ihre
Hand auf des Sündopfers Haupt legen und schlachten an der Stätte des
Brandopfers. \bibverse{30} Und der Priester soll des Bluts mit seinem
Finger nehmen und auf die Hörner des Altars des Brandopfers tun und
alles Blut an des Altars Boden gießen. \bibverse{31} All sein Fett aber
soll er abreißen, wie er das Fett des Dankopfers abgerissen hat, und
soll's anzünden auf dem Altar zum süßen Geruch dem HErrn. Und soll also
der Priester sie versöhnen, so wird's ihr vergeben. \bibverse{32} Wird
er aber ein Schaf zum Sündopfer bringen, so bringe er, das eine Sie ist,
ohne Wandel, \bibverse{33} und lege seine Hand auf des Sündopfers Haupt
und schlachte es zum Sündopfer an der Stätte, da man die Brandopfer
schlachtet. \bibverse{34} Und der Priester soll des Bluts mit seinem
Finger nehmen und auf die Hörner des Brandopferaltars tun und alles Blut
an den Boden des Altars gießen. \bibverse{35} Aber all sein Fett soll er
abreißen, wie er das Fett vom Schaf des Dankopfers abgerissen hat, und
soll's auf dem Altar anzünden zum Feuer dem HErrn. Und soll also der
Priester versöhnen seine Sünde, die er getan hat, so wird's ihm
vergeben.

\hypertarget{section-4}{%
\section{5}\label{section-4}}

\bibverse{1} Wenn eine SeeLev sündigen würde, daß er einen Fluch höret,
und er des Zeuge ist, oder gesehen oder erfahren hat und nicht angesagt,
der ist einer Missetat schuldig. \bibverse{2} Oder wenn eine SeeLev
etwas Unreines anrühret, es sei ein Aas eines unreinen Tieres oder
Viehes oder Gewürmes, und wüßte es nicht, der ist unrein und hat sich
verschuldet. \bibverse{3} Oder wenn er einen unreinen Menschen anrühret,
in waserlei Unreinigkeit der Mensch unrein werden kann, und wüßte es
nicht, und wird's inne, der hat sich verschuldet. \bibverse{4} Oder wenn
eine SeeLev schwöret, daß ihm aus dem Munde entfähret, Schaden oder
Gutes zu tun (wie denn einem Menschen ein Schwur entfahren mag, ehe er's
bedacht), und wird's inne, der hat sich an der einem verschuldet.
\bibverse{5} Wenn es nun geschiehet, daß er sich der eines verschuldet
und erkennet sich, daß er daran gesündigt hat, \bibverse{6} so soll er
für seine Schuld dieser seiner Sünde, die er getan hat, dem HErrn
bringen von der Herde eine Schaf- oder Ziegenmutter zum Sündopfer; so
soll ihm der Priester seine Sünde versöhnen. \bibverse{7} Vermag er aber
nicht ein Schaf, so bringe er dem HErrn für seine Schuld, die er getan
hat, zwo Turteltauben oder zwo junge Tauben, die erste zum Sündopfer,
die andere zum Brandopfer. \bibverse{8} Und bringe sie dem Priester. Der
soll die erste zum Sündopfer machen und ihr den Kopf abkneipen hinter
dem Genick, und nicht abbrechen. \bibverse{9} Und sprenge mit dem Blut
des Sündopfers an die Seite des Altars und lasse das übrige Blut
ausbluten an des Altars Boden. Das ist das Sündopfer. \bibverse{10} Die
andere aber soll er zum Brandopfer machen nach seinem Recht. Und soll
also der Priester ihm seine Sünde versöhnen, die er getan hat; so wird's
ihm vergeben. \bibverse{11} Vermag er aber nicht zwo Turteltauben oder
zwo junge Tauben, so bringe er für seine Sünde sein Opfer, einen zehnten
Teil Epha Semmelmehl zum Sündopfer. Er soll aber kein Öl drauf legen
noch Weihrauch drauf tun: denn es ist ein Sündopfer. \bibverse{12} Und
soll's zum Priester bringen. Der Priester aber soll eine Handvoll davon
nehmen zum Gedächtnis und anzünden auf dem Altar zum Feuer dem HErrn.
Das ist ein Sündopfer. \bibverse{13} Und der Priester soll also seine
Sünde, die er getan hat, ihm versöhnen, so wird's ihm vergeben. Und soll
des Priesters sein, wie ein Speisopfer. \bibverse{14} Und der HErr
redete mit Mose und sprach: \bibverse{15} Wenn sich eine SeeLev
vergreift, daß sie es versiehet, und sich versündiget an dem, das dem
HErrn geweihet ist, soll sie ihr Schuldopfer dem HErrn bringen, einen
Widder ohne Wandel von der Herde, der zween Sekel Silbers wert sei nach
dem Sekel des Heiligtums, zum Schuldopfer. \bibverse{16} Dazu, was er
gesündiget hat an dem Geweiheten, soll er wiedergeben und das fünfte
Teil darüber geben, und soll's dem Priester geben; der soll ihn
versöhnen mit dem Widder des Schuldopfers, so wird's ihm vergeben.
\bibverse{17} Wenn eine SeeLev sündiget und tut wider irgend ein Gebot
des HErrn, das sie nicht tun sollte, und hat es nicht gewußt, die hat
sich verschuldet und ist einer Missetat schuldig. \bibverse{18} Und soll
bringen einen Widder von der Herde ohne Wandel, der eines Schuldopfers
wert ist, zum Priester; der soll ihm seine Unwissenheit versöhnen, die
er getan hat und wußte es nicht; so wird's ihm vergeben. \bibverse{19}
Das ist das Schuldopfer, das er dem HErrn verfallen ist.

\hypertarget{section-5}{%
\section{6}\label{section-5}}

\bibverse{1} Und der HErr redete mit Mose und sprach: \bibverse{2} Wenn
eine SeeLev sündigen würde und sich an dem HErrn vergreifen, daß er
seinem Nebenmenschen verleugnet, was er ihm befohlen hat, oder das ihm
zu treuer Hand getan ist, oder das er mit Gewalt genommen, oder mit
Unrecht zu sich gebracht, \bibverse{3} oder, das verloren ist, funden
hat, und leugnet solches mit einem falschen Eide, wie es der eines ist,
darin ein Mensch wider seinen Nächsten Sünde tut: \bibverse{4} wenn's
nun geschiehet, daß er also sündiget und sich verschuldet, soll er
wiedergeben, was er mit Gewalt genommen, oder mit Unrecht zu sich
gebracht, oder was ihm befohlen ist, oder was er funden hat,
\bibverse{5} oder worüber er den falschen Eid getan hat; das soll er
alles ganz wiedergeben, dazu das fünfte Teil drüber geben dem, des es
gewesen ist, des Tages, wenn er sein Schuldopfer gibt. \bibverse{6} Aber
für seine Schuld soll er dem HErrn zu dem Priester einen Widder von der
Herde ohne Wandel bringen, der eines Schuldopfers wert ist. \bibverse{7}
So soll ihn der Priester versöhnen vor dem HErrn; so wird ihm vergeben
alles, was er getan hat, daran er sich verschuldet hat. \bibverse{8} Und
der HErr redete mit Mose und sprach: \bibverse{9} Gebeut Aaron und
seinen Söhnen und sprich: Dies ist das Gesetz des Brandopfers. Das
Brandopfer soll brennen auf dem Altar die ganze Nacht bis an den Morgen.
Es soll aber allein des Altars Feuer drauf brennen. \bibverse{10} Und
der Priester soll seinen leinenen Rock anziehen und die leinene
Niederwand an seinen Leib; und soll die Asche aufheben, die das Feuer
des Brandopfers auf dem Altar gemacht hat, und soll sie neben den Altar
schütten. \bibverse{11} Und soll seine Kleider danach ausziehen und
andere Kleider anziehen und die Asche hinaustragen außer dem Lager an
eine reine Stätte. \bibverse{12} Das Feuer auf dem Altar soll brennen
und nimmer verlöschen; der Priester soll da alLev Morgen Holz drauf
anzünden und oben drauf das Brandopfer zurichten und das Fett der
Dankopfer drauf anzünden. \bibverse{13} Ewig soll das Feuer auf dem
Altar brennen und nimmer verlöschen. \bibverse{14} Und das ist das
Gesetz des Speisopfers, das Aarons Söhne opfern sollen vor dem HErrn auf
dem Altar. \bibverse{15} Es soll einer heben seine Hand voll Semmelmehls
vom Speisopfer und des Öles und den ganzen Weihrauch, der auf dem
Speisopfer liegt und soll's anzünden auf dem Altar zum süßen Geruch, ein
Gedächtnis dem HErrn. \bibverse{16} Das übrige aber sollen Aaron und
seine Söhne verzehren; und sollen es ungesäuert essen an heiliger
Stätte, im Vorhof der Hütte des Stifts. \bibverse{17} Sie sollen es
nicht mit Sauerteig backen; denn es ist ihr Teil, das ich ihnen gegeben
habe von meinem Opfer. Es soll ihnen das Allerheiligste sein, gleich wie
das Sündopfer und Schuldopfer. \bibverse{18} Was männlich ist unter den
Kindern Aarons, sollen es essen. Das sei ein ewiges Recht euren
Nachkommen an den Opfern des HErrn: Es soll sie niemand anrühren, er sei
denn geweihet. \bibverse{19} Und der HErr redete mit Mose und sprach:
\bibverse{20} Das soll das Opfer sein Aarons und seiner Söhne, das sie
dem HErrn opfern sollen am Tage seiner Salbung: das zehnte Teil Epha vom
Semmelmehl des täglichen Speisopfers, eine Hälfte des Morgens, die
andere Hälfte des Abends. \bibverse{21} In der Pfanne mit Öl sollst du
es machen und geröstet darbringen; und in Stücken gebacken sollst du
solches opfern zum süßen Geruch dem HErrn. \bibverse{22} Und der
Priester, der unter seinen Söhnen an seiner Statt gesalbet wird, soll
solches tun. Das ist ein ewiges Recht dem HErrn; es soll ganz verbrannt
werden. \bibverse{23} Denn alles Speisopfer eines Priesters soll ganz
verbrannt und nicht gegessen werden. \bibverse{24} Und der HErr redete
mit Mose und sprach: \bibverse{25} Säge Aaron und seinen Söhnen und
sprich: Dies ist das Gesetz des Sündopfers: An der Stätte, da du das
Brandopfer schlachtest, sollst du auch das Sündopfer schlachten vor dem
HErrn; das ist das Allerheiligste. \bibverse{26} Der Priester, der das
Sündopfer tut, soll's essen an heiliger Stätte, im Vorhof der Hütte des
Stifts. \bibverse{27} Niemand soll seines Fleisches anrühren, er sei
denn geweihet. Und wer von seinem Blut ein Kleid besprenget, der soll
das besprengte Stück waschen an heiliger Stätte. \bibverse{28} Und den
Topf, darin es gekocht ist, soll man zerbrechen. Ist's aber ein eherner
Topf, so soll man ihn scheuern und mit Wasser spülen. \bibverse{29} Was
männlich ist unter den Priestern, sollen davon essen; denn es ist das
Allerheiligste. \bibverse{30} Aber all das Sündopfer, des Blut in die
Hütte des Stifts gebracht wird, zu versöhnen im Heiligen, soll man nicht
essen, sondern mit Feuer verbrennen.

\hypertarget{section-6}{%
\section{7}\label{section-6}}

\bibverse{1} Und dies ist das Gesetz des Schuldopfers; und das ist das
Allerheiligste. \bibverse{2} An der Stätte, da man das Brandopfer
schlachtet, soll man auch das Schuldopfer schlachten und seines Bluts
auf den Altar umhersprengen. \bibverse{3} Und all sein Fett soll man
opfern, den Schwanz und das Fett am Eingeweide, \bibverse{4} die zwo
Nieren mit dem Fett, das dran ist, an den Lenden, und das Netz über der
Leber, an den Nieren abgerissen. \bibverse{5} Und der Priester soll's
auf dem Altar anzünden zum Feuer dem HErrn. Das ist ein Schuldopfer.
\bibverse{6} Was männlich ist unter den Priestern, sollen das essen an
heiliger Stätte; denn es ist das Allerheiligste. \bibverse{7} Wie das
Sündopfer, also soll auch das Schuldopfer sein; aller beider soll
einerlei Gesetz sein; und soll des Priesters sein, der dadurch
versöhnet. \bibverse{8} Welcher Priester jemandes Brandopfer opfert, des
soll desselben Brandopfers Fell sein, das er geopfert hat. \bibverse{9}
Und alles Speisopfer, das im Ofen oder auf dem Rost oder in der Pfanne
gebacken ist, soll des Priesters sein, der es opfert. \bibverse{10} Und
alles Speisopfer, das mit Öl gemenget oder trocken ist, soll aller
Aarons Kinder sein, eines wie des andern. \bibverse{11} Und dies ist das
Gesetz des Dankopfers, das man dem HErrn opfert. \bibverse{12} Wollen
sie ein Lobopfer tun, so sollen sie ungesäuerte Kuchen opfern, mit Öl
gemenget, und ungesäuerte Fladen, mit Öl bestrichen, und geröstete
Semmelkuchen, mit Öl gemenget. \bibverse{13} Sie sollen aber solches
Opfer tun auf einem Kuchen von gesäuertem Brot zum Lobopfer seines
Dankopfers. \bibverse{14} Und soll einen von den allen dem HErrn zur
Hebe opfern; und soll des Priesters sein, der das Blut des Dankopfers
sprenget. \bibverse{15} Und das Fleisch des Lobopfers in seinem
Dankopfer soll desselben Tages gegessen werden, da es geopfert ist, und
nichts übergelassen werden bis an den Morgen. \bibverse{16} Und es sei
ein Gelübde oder freiwillig Opfer, so soll es desselben Tages, da es
geopfert ist, gegessen werden; so aber etwas überbleibet auf den andern
Tag, soll man's doch essen. \bibverse{17} Aber was von geopfertem
Fleisch überbleibet am dritten Tag, soll mit Feuer verbrannt werden.
\bibverse{18} Und wo jemand am dritten Tage wird essen von dem
geopferten Fleisch seines Dankopfers, so wird der nicht angenehm sein,
der es geopfert hat; es wird ihm auch nicht zugerechnet werden, sondern
es wird ein Greuel sein; und welche SeeLev davon essen wird, die ist
einer Missetat schuldig. \bibverse{19} Und das Fleisch, das etwas
Unreines anrühret, soll nicht gegessen, sondern mit Feuer verbrannt
werden. Wer reines Leibes ist, soll des Fleisches essen. \bibverse{20}
Und welche SeeLev essen wird von dem Fleisch des Dankopfers, das dem
HErrn zugehöret, derselben Unreinigkeit sei auf ihr, und sie wird
ausgerottet werden von ihrem Volk. \bibverse{21} Und wenn eine SeeLev
etwas Unreines anrühret, es sei ein unreiner Mensch, Vieh, oder was
sonst greulich ist, und vom Fleisch des Dankopfers isset, das dem HErrn
zugehöret, die wird ausgerottet werden von ihrem Volk. \bibverse{22} Und
der HErr redete mit Mose und sprach: \bibverse{23} Rede mit den Kindern
Israel und sprich: Ihr sollt kein Fett essen von Ochsen, Lämmern und
Ziegen. \bibverse{24} Aber das Fett vom Aas, und was vom Wild zerrissen
ist, macht euch zu allerlei Nutz aber essen sollt ihr's nicht.
\bibverse{25} Denn wer das Fett isset vom Vieh, das dem HErrn zum Opfer
gegeben ist, dieselbe SeeLev soll ausgerottet werden von ihrem Volk.
\bibverse{26} Ihr sollt auch kein Blut essen, weder vom Vieh noch von
Vögeln, wo ihr wohnet. \bibverse{27} Welche SeeLev würde irgend ein Blut
essen, die soll ausgerottet werden von ihrem Volk. \bibverse{28} Und der
HErr redete mit Mose und sprach: \bibverse{29} Rede mit den Kindern
Israel und sprich: Wer dem HErrn sein Dankopfer tun will, der soll auch
mitbringen, was zum Dankopfer dem HErrn gehöret. \bibverse{30} Er soll's
aber mit seiner Hand herzubringen zum Opfer des HErrn; nämlich das Fett
an der Brust soll er bringen samt der Brust, daß sie eine Webe werden
vor dem HErrn. \bibverse{31} Und der Priester soll das Fett anzünden auf
dem Altar; und die Brust soll Aarons und seiner Söhne sein.
\bibverse{32} Und die rechte Schulter sollen sie dem Priester geben zur
Hebe von ihren Dankopfern. \bibverse{33} Und welcher unter Aarons Söhnen
das Blut der Dankopfer opfert und das Fett, des soll die rechte Schulter
sein zu seinem Teil. \bibverse{34} Denn die Webebrust und die
Hebeschulter habe ich genommen von den Kindern Israel von ihren
Dankopfern und habe sie dem Priester Aaron und seinen Söhnen gegeben zum
ewigen Recht. \bibverse{35} Dies ist die Salbung Aarons und seiner Söhne
von den Opfern des HErrn des Tages, da sie überantwortet wurden,
Priester zu sein dem HErrn, \bibverse{36} da der HErr gebot am Tage, da
er sie salbete, daß ihm gegeben werden sollte von den Kindern Israel zum
ewigen Recht allen ihren Nachkommen. \bibverse{37} Und dies ist das
Gesetz des Brandopfers, des Speisopfers, des Sündopfers, des
Schuldopfers, der Füllopfer und der Dankopfer, \bibverse{38} das der
HErr Mose gebot auf dem Berge Sinai des Tages, da er ihm gebot an die
Kinder Israel, zu opfern ihre Opfer dem HErrn in der Wüste Sinai.

\hypertarget{section-7}{%
\section{8}\label{section-7}}

\bibverse{1} Und der HErr redete mit Mose und sprach: \bibverse{2} Nimm
Aaron und seine Söhne mit ihm samt ihren Kleidern und das Salböl und
einen Farren zum Sündopfer, zween Widder und einen Korb mit ungesäuertem
Brot. \bibverse{3} Und versammLev die ganze Gemeine vor die Tür der
Hütte des Stifts. \bibverse{4} Mose tat, wie ihm der HErr gebot, und
versammelte die Gemeine vor die Tür der Hütte des Stifts \bibverse{5}
und sprach zu ihnen: Das ist's, das der HErr geboten hat zu tun.
\bibverse{6} Und nahm Aaron und seine Söhne und wusch sie mit Wasser.
\bibverse{7} Und legte ihm den leinenen Rock an und gürtete ihn mit dem
Gürtel und zog ihm den Seidenrock an und tat ihm den Leibrock an und
gürtete ihn über den Leibrock her. \bibverse{8} Und tat ihm das
Schildlein an und in das Schildlein Licht und Recht. \bibverse{9} Und
setzte ihm den Hut auf sein Haupt und setzte an den Hut, oben an seiner
Stirn, das güldene Blatt der heiligen Krone, wie der HErr Mose geboten
hatte. \bibverse{10} Und Mose nahm das Salböl und salbete die Wohnung
und alles, was drinnen war, und weihete es. \bibverse{11} Und sprengete
damit siebenmal auf den Altar und salbete den Altar mit alLev seinem
Geräte, das Handfaß mit seinem Fuß, daß es geweihet würde. \bibverse{12}
Und goß des Salböles auf Aarons Haupt und salbete ihn, daß er geweihet
würde. \bibverse{13} Und brachte herzu Aarons Söhne und zog ihnen
leinene Röcke an und gürtete sie mit dem Gürtel und band ihnen Hauben
auf, wie ihm der HErr geboten hatte. \bibverse{14} Und ließ herzuführen
einen Farren zum Sündopfer. Und Aaron mit seinen Söhnen legten ihre
Hände auf sein Haupt. \bibverse{15} Da schlachtete man es. Und Mose nahm
des Bluts und tat's auf die Hörner des Altars umher mit seinem Finger
und entsündigte den Altar; und goß das Blut an des Altars Boden und
weihete ihn, daß er ihn versöhnete. \bibverse{16} Und nahm alles Fett am
Eingeweide, das Netz über der Leber und die zwo Nieren mit dem Fett
daran und zündete es an auf dem Altar. \bibverse{17} Aber den Farren mit
seinem Fell, Fleisch und Mist verbrannte er mit Feuer außer dem Lager,
wie ihm der HErr geboten hatte. \bibverse{18} Und brachte herzu einen
Widder zum Brandopfer. Und Aaron mit seinen Söhnen legten ihre Hände auf
sein Haupt. \bibverse{19} Da schlachtete man ihn. Und Mose sprengete des
Bluts auf den Altar umher, \bibverse{20} zerhieb den Widder in Stücke
und zündete an das Haupt, die Stücke und den Stumpf. \bibverse{21} Und
wusch die Eingeweide und Schenkel mit Wasser und zündete also den ganzen
Widder an auf dem Altar. Das war ein Brandopfer zum süßen Geruch, ein
Feuer dem HErrn, wie ihm der HErr geboten hatte. \bibverse{22} Er
brachte auch herzu den andern Widder des Füllopfers. Und Aaron mit
seinen Söhnen legten ihre Hände auf sein Haupt. \bibverse{23} Da
schlachtete man ihn. Und Mose nahm seines Bluts und tat's Aaron auf den
Knorpel seines rechten Ohrs und auf den Daumen seiner rechten Hand und
auf den großen Zehen seines rechten Fußes. \bibverse{24} Und brachte
herzu Aarons Söhne; und tat des Bluts auf den Knorpel ihres rechten Ohrs
und auf den Daumen ihrer rechten Hand und auf den großen Zehen ihres
rechten Fußes; und sprengete das Blut auf den Altar umher. \bibverse{25}
Und nahm das Fett und den Schwanz und alles Fett am Eingeweide und das
Netz über der Leber, die zwo Nieren mit dem Fett daran und die rechte
Schulter. \bibverse{26} Dazu nahm er von dem Korbe des ungesäuerten
Brots vor dem HErrn einen ungesäuerten Kuchen und einen Kuchen geölten
Brots und einen Fladen; und legte es auf das Fett und auf die rechte
Schulter. \bibverse{27} Und gab das allesamt auf die Hände Aarons und
seiner Söhne und webete es zur Webe vor dem HErrn. \bibverse{28} Und
nahm's alles wieder von ihren Händen und zündete es an auf dem Altar,
oben auf dem Brandopfer; denn es ist ein Füllopfer zum süßen Geruch, ein
Feuer dem HErrn. \bibverse{29} Und Mose nahm die Brust und webete eine
Webe vor dem HErrn von dem Widder des Füllopfers. Die ward Mose zu einem
Teil, wie ihm der HErr geboten hatte. \bibverse{30} Und Mose nahm des
Salböles und des Bluts auf dem Altar und sprengete auf Aaron und seine
Kleider, auf seine Söhne und auf ihre Kleider; und weihete also Aaron
und seine Kleider, seine Söhne und ihre Kleider mit ihm. \bibverse{31}
Und sprach zu Aaron und seinen Söhnen: Kochet das Fleisch vor der Tür
der Hütte des Stifts und esset es daselbst, dazu, auch das Brot im Korbe
des Füllopfers, wie mir geboten ist und gesagt, daß Aaron und seine
Söhne sollen's essen. \bibverse{32} Was aber überbleibt vom Fleisch und
Brot, das sollt ihr mit Feuer verbrennen. \bibverse{33} Und sollt in
sieben Tagen nicht ausgehen von der Tür der Hütte des Stifts bis an den
Tag, da die Tage eures Füllopfers aus sind; denn sieben Tage sind eure
Hände gefüllet, \bibverse{34} wie es an diesem Tage geschehen ist. Der
HErr hat's geboten zu tun, auf daß ihr versöhnet seiet. \bibverse{35}
Und sollt vor der Tür der Hütte des Stifts Tag und Nacht bleiben, sieben
Tage lang, und sollt auf die Hut des HErrn warten, daß ihr nicht
sterbet; denn also ist mir's geboten. \bibverse{36} Und Aaron mit seinen
Söhnen taten alles, was der HErr geboten hatte durch Mose.

\hypertarget{section-8}{%
\section{9}\label{section-8}}

\bibverse{1} Und am achten Tage rief Mose Aaron und seinen Söhnen und
den Ältesten in Israel \bibverse{2} und sprach zu Aaron: Nimm zu dir ein
jung Kalb zum Sündopfer und einen Widder zum Brandopfer, beide ohne
Wandel, und bringe sie vor den HErrn. \bibverse{3} Und rede mit den
Kindern Israel und sprich: Nehmet einen Ziegenbock zum Sündopfer und ein
Kalb und ein Schaf, beide eines Jahrs alt und ohne Wandel, zum
Brandopfer; \bibverse{4} und einen Ochsen und einen Widder zum
Dankopfer, daß wir vor dem HErrn opfern; und ein Speisopfer, mit Öl
gemenget. Denn heute wird euch der HErr erscheinen. \bibverse{5} Und sie
nahmen, was Mose geboten hatte, vor der Tür der Hütte des Stifts. Und
trat herzu die ganze Gemeine und stund vor dem HErrn. \bibverse{6} Da
sprach Mose: Das ist's, das der HErr geboten hat, daß ihr tun sollt, so
wird euch des HErrn Herrlichkeit erscheinen. \bibverse{7} Und Mose
sprach zu Aaron: Tritt zum Altar und mache dein Sündopfer und dein
Brandopfer und versöhne dich und das Volk; danach mache des Volks Opfer
und versöhne sie auch, wie der HErr geboten hat. \bibverse{8} Und Aaron
trat zum Altar und schlachtete das Kalb zu seinem Sündopfer.
\bibverse{9} Und seine Söhne brachten das Blut zu ihm; und er tunkte mit
seinem Finger ins Blut und tat's auf die Hörner des Altars und goß das
Blut an des Altars Boden. \bibverse{10} Aber das Fett und die Nieren und
das Netz von der Leber am Sündopfer zündete er an auf dem Altar, wie der
HErr Mose geboten hatte. \bibverse{11} Und das Fleisch und das Fell
verbrannte er mit Feuer außer dem Lager. \bibverse{12} Danach
schlachtete er das Brandopfer; und Aarons Söhne brachten das Blut zu
ihm, und er sprengete es auf den Altar umher. \bibverse{13} Und sie
brachten das Brandopfer zu ihm zerstücket, und den Kopf; und er zündete
es an auf dem Altar. \bibverse{14} Und er wusch das Eingeweide und die
Schenkel und zündete es an oben auf dem Brandopfer, auf dem Altar.
\bibverse{15} Danach brachte er herzu des Volks Opfer; und nahm den
Bock, das Sündopfer des Volks, und schlachtete ihn und machte ein
Sündopfer draus, wie das vorige. \bibverse{16} Und brachte das
Brandopfer herzu und tat ihm sein Recht. \bibverse{17} Und brachte herzu
das Speisopfer und nahm seine Hand voll und zündete es an auf dem Altar,
außer des Morgens Brandopfer. \bibverse{18} Danach schlachtete er den
Ochsen und Widder zum Dankopfer des Volks. Und seine Söhne brachten ihm
das Blut; das sprengete er auf den Altar umher. \bibverse{19} Aber das
Fett vom Ochsen und vom Widder, den Schwanz und das Fett am Eingeweide
und die Nieren und das Netz über der Leber, \bibverse{20} alles solches
Fett legten sie auf die Brust; und er zündete das Fett an auf dem Altar.
\bibverse{21} Aber die Brust und die rechte Schulter webete Aaron zur
Webe vor dem HErrn, wie der HErr Mose geboten hatte. \bibverse{22} Und
Aaron hub seine Hand auf zum Volk und segnete sie; und stieg herab, da
er das Sündopfer, Brandopfer und Dankopfer gemacht hatte. \bibverse{23}
Und Mose und Aaron gingen in die Hütte des Stifts; und da sie wieder
herausgingen, segneten sie das Volk. Da erschien die Herrlichkeit des
HErrn allem Volk. \bibverse{24} Denn das Feuer kam aus von dem HErrn und
verzehrete auf dem Altar das Brandopfer und das Fett. Da das alles Volk
sah, frohlockten sie und fielen auf ihr Antlitz.

\hypertarget{section-9}{%
\section{10}\label{section-9}}

\bibverse{1} Und die Söhne Aarons, Nadab und Abihu, nahmen ein jeglicher
seinen Napf und taten Feuer drein und legten Räuchwerk drauf und
brachten das fremde Feuer vor den HErrn, das er ihnen nicht geboten
hatte. \bibverse{2} Da fuhr ein Feuer aus von dem HErrn und verzehrete
sie, daß sie starben vor dem HErrn. \bibverse{3} Da sprach Mose zu
Aaron: Das ist's, das der HErr gesagt hat: Ich werde geheiliget werden
an denen, die zu mir nahen, und vor allem Volk werde ich herrlich
werden. Und Aaron schwieg stille. \bibverse{4} Mose aber rief Misael und
Elzaphan, den Söhnen Usiels, Aarons Vettern, und sprach zu ihnen: Tretet
hinzu und traget eure Brüder von dem Heiligtum hinaus vor das Lager.
\bibverse{5} Und sie traten hinzu und trugen sie hinaus mit ihren
leinenen Röcken vor das Lager, wie Mose gesagt hatte. \bibverse{6} Da
sprach Mose zu Aaron und seinen Söhnen, Eleazar und Ithamar: Ihr sollt
eure Häupter nicht blößen, noch eure Kleider zerreißen, daß ihr nicht
sterbet, und der Zorn über die ganze Gemeine komme. Lasset eure Brüder
des ganzen Hauses Israel weinen über diesen Brand, den der HErr getan
hat. \bibverse{7} Ihr aber sollt nicht ausgehen von der Tür der Hütte
des Stifts; ihr möchtet sterben. Denn das Salböl des HErrn ist auf euch.
Und sie taten, wie Mose sagte. \bibverse{8} Der HErr aber redete mit
Aaron und sprach: \bibverse{9} Du und deine Söhne mit dir, sollt keinen
Wein noch stark Getränke trinken, wenn ihr in die Hütte des Stifts
gehet, auf daß ihr nicht sterbet. Das sei ein ewiges Recht allen euren
Nachkommen, \bibverse{10} auf daß ihr könnet unterscheiden, was heilig
und unheilig, was unrein und rein ist, \bibverse{11} und daß ihr die
Kinder Israel lehret alLev Rechte, die der HErr zu euch geredet hat
durch Mose. \bibverse{12} Und Mose redete mit Aaron und mit seinen
übrigen Söhnen, Eleazar und Ithamar: Nehmet, das überblieben ist vom
Speisopfer an den Opfern des HErrn und esset es ungesäuert bei dem
Altar; denn es ist das Allerheiligste. \bibverse{13} Ihr sollt es aber
an heiliger Stätte essen; denn das ist dein Recht und deiner Söhne Recht
an den Opfern des HErrn; denn so ist mir's geboten. \bibverse{14} Aber
die Webebrust und die Hebeschulter sollst du und deine Söhne und deine
Töchter mit dir essen an reiner Stätte; denn solch Recht ist dir und
deinen Kindern gegeben an den Dankopfern der Kinder Israel.
\bibverse{15} Denn die Hebeschulter und die Webebrust zu den Opfern des
Fettes werden gebracht, daß sie zur Webe gewebet werden vor dem HErrn:
darum ist's dein und deiner Kinder zum ewigen Recht, wie der HErr
geboten hat. \bibverse{16} Und Mose suchte den Bock des Sündopfers und
fand ihn verbrannt. Und er ward zornig über Eleazar und Ithamar, Aarons
Söhne, die noch übrig waren, und sprach: \bibverse{17} Warum habt ihr
das Sündopfer nicht gegessen an heiliger Stätte? denn es das
Allerheiligste ist, und er hat's euch gegeben, daß ihr die Missetat der
Gemeine tragen sollt, daß ihr sie versöhnet vor dem HErrn. \bibverse{18}
Siehe, sein Blut ist nicht kommen in das Heilige hinein. Ihr solltet es
im Heiligen gegessen haben, wie mir geboten ist. \bibverse{19} Aaron
aber sprach zu Mose: Siehe, heute haben sie ihr Sündopfer und ihr
Brandopfer vor dem HErrn geopfert, und es ist mir also gegangen, wie du
siehest; und ich sollte essen heute vom Sündopfer? Sollte das dem HErrn
gefallen? \bibverse{20} Da das Mose hörete, ließ er's ihm gefallen.

\hypertarget{section-10}{%
\section{11}\label{section-10}}

\bibverse{1} Und der HErr redete mit Mose und Aaron und sprach zu ihnen:
\bibverse{2} Redet mit den Kindern Israel und sprechet: Das sind die
Tiere, die ihr essen sollt unter allen Tieren auf Erden. \bibverse{3}
Alles, was die Klauen spaltet und wiederkäuet unter den Tieren, das
sollt ihr essen. \bibverse{4} Was aber wiederkäuet und hat Klauen und
spaltet sie doch nicht, als das Kamel, das ist euch unrein, und sollt es
nicht essen. \bibverse{5} Die Kaninchen wiederkäuen wohl, aber sie
spalten die Klauen nicht; darum sind sie unrein. \bibverse{6} Der Hase
wiederkäuet auch, aber er spaltet die Klauen nicht; darum ist er euch
unrein. \bibverse{7} Und ein Schwein spaltet wohl die Klauen, aber es
wiederkäuet nicht; darum soll es euch unrein sein. \bibverse{8} Von
diesem Fleisch sollt ihr nicht essen, noch ihr Aas anrühren; denn sie
sind euch unrein. \bibverse{9} Dies sollt ihr essen unter dem, das in
Wassern ist: Alles, was Floßfedern und Schuppen hat in Wassern, im Meer
und Bächen, sollt ihr essen. \bibverse{10} Alles aber, was nicht
Floßfedern und Schuppen hat im Meer und Bächen, unter allem, das sich
reget in Wassern, und allem, was lebet im Wasser, soll euch eine Scheu
sein, \bibverse{11} daß ihr von ihrem Fleisch nicht esset und vor ihrem
Aas euch scheuet. \bibverse{12} Denn alles, was nicht Floßfedern und
Schuppen hat in Wassern, sollt ihr scheuen. \bibverse{13} Und dies sollt
ihr scheuen unter den Vögeln, daß ihr's nicht esset: den Adler, den
Habicht, den Fischaar, \bibverse{14} den Geier, den Weihe und was seiner
Art ist, \bibverse{15} und alLev Raben mit ihrer Art, \bibverse{16} den
Strauß, die Nachteule, den Kuckuck, den Sperber mit seiner Art,
\bibverse{17} das Käuzlein, den Schwan, den Huhu, \bibverse{18} die
Fledermaus, die Rohrdommel, \bibverse{19} den Storch, den Reiher, den
Heher mit seiner Art, den Wiedehopf und die Schwalbe. \bibverse{20}
Alles auch, was sich reget unter den Vögeln und gehet auf vier Füßen,
das soll euch eine Scheu sein. \bibverse{21} Doch das sollt ihr essen
von Vögeln, das sich reget und gehet auf vier Füßen und nicht mit zweien
Beinen auf Erden hüpfet. \bibverse{22} Von denselben möget ihr essen,
als da ist: Arbe mit seiner Art und Selaam mit seiner Art und Hargol mit
seiner Art und Hagab mit ihrer Art. \bibverse{23} Alles aber, was sonst
vier Füße hat unter den Vögeln, soll euch eine Scheu sein, \bibverse{24}
und sollt sie unrein achten. Wer solcher Aas anrühret, der wird unrein
sein bis auf den Abend. \bibverse{25} Und wer dieser Aas eines tragen
wird, der soll seine Kleider waschen und wird unrein sein bis auf den
Abend. \bibverse{26} Darum alles Tier, das Klauen hat und spaltet sie
nicht und wiederkäuet nicht, das soll euch unrein sein; wer es anrühret,
wird unrein sein. \bibverse{27} Und alles, was auf Tappen gehet unter
den Tieren, die auf vier Füßen gehen, soll euch unrein sein; wer ihr Aas
anrühret, wird unrein sein bis auf den Abend. \bibverse{28} Und wer ihr
Aas trägt, soll seine Kleider waschen und unrein sein bis auf den Abend;
denn solche sind euch unrein. \bibverse{29} Diese sollen euch auch
unrein sein unter den Tieren, die auf Erden kriechen: das Wiesel, die
Maus, die Kröte, ein jegliches mit seiner Art; \bibverse{30} der Igel,
der Molch, die Eidechse, die Blindschleiche und der Maulwurf.
\bibverse{31} Die sind euch unrein unter allem, das da kreucht wer ihr
Aas anrühret, der wird unrein sein bis an den Abend. \bibverse{32} Und
alles, worauf ein solch tot Aas fällt, das wird unrein, es sei allerlei
hölzern Gefäß, oder Kleider, oder Fell, oder Sack; und alles Geräte,
damit man etwas schaffet, soll man ins Wasser tun, und ist unrein bis
auf den Abend; alsdann wird's rein. \bibverse{33} Allerlei irden Gefäß,
wo solcher Aas eines drein fällt, wird alles unrein, was drinnen ist;
und sollt es zerbrechen. \bibverse{34} AlLev Speise, die man isset, so
solches Wasser drein kommt, ist unrein; und aller Trank, den man
trinket, in allerlei solchem Gefäß, ist unrein. \bibverse{35} Und alles,
worauf ein solch Aas fällt, wird unrein, es sei Ofen oder Kessel, so
soll man's zerbrechen; denn es ist unrein, und soll euch unrein sein.
\bibverse{36} Doch die Brunnen und Kölke und Teiche sind rein. Wer aber
ihr Aas anrühret, ist unrein. \bibverse{37} Und ob ein solch Aas fieLev
auf Samen, den man gesät hat, so ist er doch rein. \bibverse{38} Wenn
man aber Wasser über den Samen gösse, und fieLev danach ein solch Aas
darauf, so würde er euch unrein. \bibverse{39} Wenn ein Tier stirbt, das
ihr essen möget: wer das Aas anrühret, der ist unrein bis an den Abend.
\bibverse{40} Wer von solchem Aas isset, der soll sein Kleid waschen und
wird unrein sein bis an den Abend. Also wer auch trägt ein solch Aas,
soll sein Kleid waschen und wird unrein sein bis an den Abend.
\bibverse{41} Was auf Erden schleicht, das soll euch eine Scheu sein,
und man soll's nicht essen. \bibverse{42} Und alles, was auf dem Bauch
kreucht, und alles, was auf vier oder mehr Füßen gehet, unter allem, das
auf Erden schleicht, sollt ihr nicht essen; denn es soll euch eine Scheu
sein. \bibverse{43} Machet eure SeeLev nicht zum Scheusal und
verunreiniget euch nicht an ihnen, daß ihr euch besudelt. \bibverse{44}
Denn ich bin der HErr, euer GOtt. Darum sollt ihr euch heiligen, daß ihr
heilig seid, denn ich bin heilig; und sollt nicht eure SeeLev
verunreinigen an irgend einem kriechenden Tier, das auf Erden schleicht.
\bibverse{45} Denn ich bin der HErr, der euch aus Ägyptenland geführet
hat, daß ich euer GOtt sei. Darum sollt ihr heilig sein, denn ich bin
heilig. \bibverse{46} Dies ist das Gesetz von den Tieren und Vögeln und
allerlei kriechenden Tieren im Wasser und allerlei Tieren, die auf Erden
schleichen, \bibverse{47} daß ihr unterscheiden könntet, was unrein und
rein ist, und welches Tier man essen und welches man nicht essen soll.

\hypertarget{section-11}{%
\section{12}\label{section-11}}

\bibverse{1} Und der HErr redete mit Mose und sprach: \bibverse{2} Rede
mit den Kindern Israel und sprich: Wenn ein Weib besamet wird und
gebiert ein Knäblein, so soll sie sieben Tage unrein sein, solange sie
ihre Krankheit leidet. \bibverse{3} Und am achten Tage soll man das
Fleisch seiner Vorhaut beschneiden. \bibverse{4} Und sie soll daheim
bleiben dreiunddreißig Tage im Blut ihrer Reinigung. Kein Heiliges soll
sie anrühren und zum Heiligtum soll sie nicht kommen, bis daß die Tage
ihrer Reinigung aus sind. \bibverse{5} Gebiert sie aber ein Mägdlein, so
soll sie zwo Wochen unrein sein, solange sie ihre Krankheit leidet, und
soll sechsundsechzig Tage daheim bleiben in dem Blut ihrer Reinigung.
\bibverse{6} Und wenn die Tage ihrer Reinigung aus sind für den Sohn
oder für die Tochter, soll sie ein jährig Lamm bringen zum Brandopfer
und eine junge Taube oder Turteltaube zum Sündopfer dem Priester vor die
Tür der Hütte des Stifts. \bibverse{7} Der soll es opfern vor dem HErrn
und sie versöhnen; so wird sie rein von ihrem Blutgang. Das ist das
Gesetz für die, so ein Knäblein oder Mägdlein gebiert. \bibverse{8}
Vermag aber ihre Hand nicht ein Schaf, so nehme sie zwo Turteltauben
oder zwo junge Tauben, eine zum Brandopfer, die andere zum Sündopfer; so
soll sie der Priester versöhnen, daß sie rein werde.

\hypertarget{section-12}{%
\section{13}\label{section-12}}

\bibverse{1} Und der HErr redete mit Mose und Aaron und sprach:
\bibverse{2} Wenn einem Menschen an der Haut seines Fleisches etwas
auffähret, oder schäbicht oder eiterweiß wird, als wollte ein Aussatz
werden an der Haut seines Fleisches, soll man ihn zum Priester Aaron
führen oder zu seiner Söhne einem unter den Priestern. \bibverse{3} Und
wenn der Priester das Mal an der Haut des Fleisches siehet, daß die
Haare in Weiß verwandelt sind, und das Ansehen an dem Ort tiefer ist
denn die andere Haut seines Fleisches, so ist's gewiß der Aussatz. Darum
soll ihn der Priester besehen und für unrein urteilen. \bibverse{4} Wenn
aber etwas eiterweiß ist an der Haut seines Fleisches, und doch das
Ansehen nicht tiefer denn die andere Haut des Fleisches, und die Haare
nicht in Weiß verwandelt sind, so soll der Priester denselben
verschließen sieben Tage \bibverse{5} und am siebenten Tage besehen.
Ist's, daß das Mal bleibt, wie er's zuvor gesehen hat, und hat nicht
weiter gefressen an der Haut, \bibverse{6} so soll ihn der Priester
abermal sieben Tage verschließen. Und wenn er ihn zum andernmal am
siebenten Tage besiehet und findet, daß das Mal verschwunden ist und
nicht weiter gefressen hat an der Haut, so soll er ihn rein urteilen,
denn es ist Grind. Und er soll seine Kleider waschen, so ist er rein.
\bibverse{7} Wenn aber der Grind weiter frißt in der Haut, nachdem er
vom Priester besehen und rein gesprochen ist, und wird nun zum andernmal
vom Priester besehen; \bibverse{8} wenn denn da der Priester siehet, daß
der Grind weiter gefressen hat in der Haut, soll er ihn unrein urteilen,
denn es ist gewiß Aussatz. \bibverse{9} Wenn ein Mal des Aussatzes am
Menschen sein wird, den soll man zum Priester bringen. \bibverse{10}
Wenn derselbe siehet und findet, daß weiß aufgefahren ist an der Haut,
und die Haare in Weiß verwandelt, und roh Fleisch im Geschwür ist,
\bibverse{11} so ist's gewiß ein alter Aussatz in der Haut seines
Fleisches. Darum soll ihn der Priester unrein urteilen und nicht
verschließen; denn er ist schon unrein. \bibverse{12} Wenn aber der
Aussatz blühet in der Haut und bedeckt die ganze Haut, von dem Haupt an
bis auf die Füße, alles, was dem Priester vor Augen sein mag;
\bibverse{13} wenn dann der Priester besiehet und findet, daß der
Aussatz das ganze Fleisch bedeckt hat, so soll er denselben rein
urteilen, dieweil es alles an ihm in Weiß verwandelt ist; denn er ist
rein. \bibverse{14} Ist aber roh Fleisch da des Tages, wenn er besehen
wird, so ist er unrein. \bibverse{15} Und wenn der Priester das rohe
Fleisch besiehet, soll er ihn unrein urteilen; denn er ist unrein, und
es ist gewiß Aussatz. \bibverse{16} Verkehret sich aber das rohe Fleisch
wieder und verwandelt sich in Weiß, so soll er zum Priester kommen.
\bibverse{17} Und wenn der Priester besiehet und findet, daß das Mal ist
in Weiß verwandelt, soll er ihn rein urteilen, denn er ist rein.
\bibverse{18} Wenn in jemandes Fleisch an der Haut eine Drüse wird, und
wieder heilet, \bibverse{19} danach an demselben Ort etwas weiß
auffähret oder rötlich Eiterweiß wird, soll er vom Priester besehen
werden. \bibverse{20} Wenn dann der Priester siehet, daß das Ansehen
tiefer ist denn die andere Haut und das Haar in Weiß verwandelt; so soll
er ihn unrein urteilen; denn es ist gewiß ein Aussatzmal aus der Drüse
worden. \bibverse{21} Siehet aber der Priester und findet, daß die Haare
nicht weiß sind, und ist nicht tiefer denn die andere Haut, und ist
verschwunden, so soll er ihn sieben Tage verschließen. \bibverse{22}
Frißt es weiter in der Haut, so soll er ihn unrein urteilen; denn es ist
gewiß ein Aussatzmal. \bibverse{23} Bleibt aber das Eiterweiß also
stehen und frißt nicht weiter, so ist's die Narbe von der Drüse, und der
Priester soll ihn rein urteilen. \bibverse{24} Wenn sich jemand an der
Haut am Feuer brennet, und das Brandmal rötlich oder weiß, ist,
\bibverse{25} und der Priester ihn besiehet und findet das Haar in Weiß
verwandelt an dem Brandmal und das Ansehen tiefer denn die andere Haut,
so ist gewiß Aussatz aus dem Brandmal worden. Darum soll ihn der
Priester unrein urteilen; denn es ist ein Aussatzmal. \bibverse{26}
Siehet aber der Priester und findet, daß die Haare am Brandmal nicht in
Weiß verwandelt, und nicht tiefer ist denn die andere Haut, und ist dazu
verschwunden, soll er ihn sieben Tage verschließen. \bibverse{27} Und am
siebenten Tage soll er ihn besehen. Hat's weiter gefressen an der Haut,
so soll er ihn unrein urteilen; denn es ist Aussatz. \bibverse{28} Ist's
aber gestanden an dem Brandmal und nicht weiter gefressen an der Haut
und ist dazu verschwunden, so ist's ein Geschwür des Brandmals. Und der
Priester soll ihn rein urteilen; denn es ist eine Narbe des Brandmals.
\bibverse{29} Wenn ein Mann oder Weib auf dem Haupt oder am Bart
schäbicht wird, \bibverse{30} und der Priester das Mal besiehet und
findet, daß das Ansehen tiefer ist denn die andere Haut, und das Haar
daselbst gülden und dünne, so soll er ihn unrein urteilen; denn es ist
aussätziger Grind des Haupts oder des Barts. \bibverse{31} Siehet aber
der Priester, daß der Grind nicht tiefer anzusehen ist denn die Haut,
und das Haar nicht falb ist, soll er denselben sieben Tage verschließen.
\bibverse{32} Und wenn er am siebenten Tage besiehet und findet, daß der
Grind nicht weiter gefressen hat, und kein gülden Haar da ist, und das
Ansehen des Grindes nicht tiefer ist denn die andere Haut, \bibverse{33}
soll er sich bescheren, doch daß er den Grind nicht beschere. Und soll
ihn der Priester abermal sieben Tage verschließen. \bibverse{34} Und
wenn er ihn am siebenten Tage besiehet und findet, daß der Grind nicht
weiter gefressen hat in der Haut, und das Ansehen ist nicht tiefer denn
die andere Haut, so soll ihn der Priester rein sprechen; und er soll
seine Kleider waschen, denn er ist rein. \bibverse{35} Frißt aber der
Grind weiter an der Haut, nachdem er rein gesprochen ist, \bibverse{36}
und der Priester besiehet und findet, daß der Grind also weiter
gefressen hat an der Haut, so soll er nicht mehr danach fragen, ob die
Haare gülden sind; denn er ist unrein. \bibverse{37} Ist aber vor Augen
der Grind still gestanden, und falb Haar daselbst aufgegangen, so ist
der Grind heil und er rein. Darum soll ihn der Priester rein sprechen.
\bibverse{38} Wenn einem Mann oder Weib an der Haut ihres Fleisches
etwas eiterweiß ist, \bibverse{39} und der Priester siehet daselbst, daß
das Eiterweiß schwindet, das ist ein weißer Grind, in der Haut
aufgegangen, und er ist rein. \bibverse{40} Wenn einem Manne die
Haupthaare ausfallen, daß er kahl wird, der ist rein. \bibverse{41}
Fallen sie ihm vorne am Haupt aus, und wird eine Glatze, so ist er rein.
\bibverse{42} Wird aber an der Glatze, oder da er kahl ist, ein weiß
oder rötlich Mal, so ist ihm Aussatz an der Glatze oder am Kahlkopf
aufgegangen. \bibverse{43} Darum soll ihn der Priester besehen. Und wenn
er findet, daß ein weiß oder rötlich Mal aufgelaufen an seiner Glatze
oder Kahlkopf, daß es siehet, wie sonst der Aussatz an der Haut,
\bibverse{44} so ist er aussätzig und unrein; und der Priester soll ihn
unrein sprechen solches Mals halben auf seinem Haupt. \bibverse{45} Wer
nun aussätzig ist, des Kleider sollen zerrissen sein und das Haupt bloß
und die Lippen verhüllet und soll allerdinge unrein genannt werden:
\bibverse{46} Und solange das Mal an ihm ist, soll er unrein sein,
alleine wohnen, und seine Wohnung soll außer dem Lager sein.
\bibverse{47} Wenn an einem Kleide eines Aussatzes Mal sein wird, es sei
wollen oder leinen, \bibverse{48} am Werft oder am Eintracht, es sei
leinen oder wollen, oder an einem Fell, oder an allem, das aus Fellen
gemacht wird; \bibverse{49} und wenn das Mal bleich oder rötlich ist am
Kleid, oder am Fell, oder am Werft, oder am Eintracht, oder an
einigerlei Ding, das von Fellen gemacht ist: das ist gewiß ein Mal des
Aussatzes; darum soll's der Priester besehen. \bibverse{50} Und wenn er
das Mal siehet, soll er's einschließen sieben Tage. \bibverse{51} Und
wenn er am siebenten Tage siehet, daß das Mal hat weiter gefressen am
Kleid, am Werft oder am Eintracht, am Fell oder an allem, das man aus
Fellen macht, so ist's ein fressend Mal des Aussatzes und ist unrein.
\bibverse{52} Und soll das Kleid verbrennen, oder den Werft, oder den
Eintracht, es sei wollen oder leinen, oder allerlei Fellwerk, darin
solch Mal ist; denn es ist ein Mal des Aussatzes; und soll es mit Feuer
verbrennen. \bibverse{53} Wird aber der Priester sehen, daß das Mal
nicht weiter gefressen hat am Kleid, oder am Werft, oder am Eintracht,
oder an allerlei Fellwerk, \bibverse{54} so soll er gebieten, daß man's
wasche, darin das Mal ist; und soll es einschließen andere sieben Tage.
\bibverse{55} Und wenn der Priester sehen wird, nachdem das Mal
gewaschen ist, daß das Mal nicht verwandelt ist vor seinen Augen und
auch nicht weiter gefressen hat, so ist's unrein, und sollst es mit
Feuer verbrennen; denn es ist tief eingefressen und hat es beschabt
gemacht. \bibverse{56} Wenn aber der Priester siehet, daß das Mal
verschwunden ist nach seinem Waschen, so soll er's abreißen vom Kleid,
vom Fell, vom Werft oder vom Eintracht. \bibverse{57} Wird's aber noch
gesehen am Kleid, am Werft, am Eintracht oder allerlei Fellwerk, so
ist's ein Fleck, und sollst es mit Feuer verbrennen, darin solch Mal
ist. \bibverse{58} Das Kleid aber, oder Werft, oder Eintracht, oder
allerlei Fellwerk, das gewaschen ist und das Mal von ihm gelassen hat,
soll man zum andernmal waschen, so ist's rein. \bibverse{59} Das ist das
Gesetz über die MaLev des Aussatzes an Kleidern, sie seien wollen oder
leinen, am Werft und am Eintracht und an allerlei Fellwerk, rein oder
unrein zu sprechen.

\hypertarget{section-13}{%
\section{14}\label{section-13}}

\bibverse{1} Und der HErr redete mit Mose und sprach: \bibverse{2} Das
ist das Gesetz über den Aussätzigen, wenn er soll gereinigt werden. Er
soll zum Priester kommen, \bibverse{3} und der Priester soll aus dem
Lager gehen und besehen, wie das Mal des Aussatzes am Aussätzigen heil
worden ist; \bibverse{4} und soll gebieten dem, der zu reinigen ist, daß
er zween lebendige Vögel nehme, die da rein sind, und Zedernholz und
rosinfarbene WolLev und Ysop. \bibverse{5} Und soll gebieten, den einen
Vogel zu schlachten in einem irdenen Gefäß am fließenden Wasser.
\bibverse{6} Und soll den lebendigen Vogel nehmen mit dem Zedernholz,
rosinfarbener WolLev und Ysop und in des geschlachteten Vogels Blut
tunken am fließenden Wasser \bibverse{7} und besprengen den, der vom
Aussatz zu reinigen ist, siebenmal; und reinige ihn also und lasse den
lebendigen Vogel ins freie Feld fliegen. \bibverse{8} Der Gereinigte
aber soll seine Kleider waschen und alLev seine Haare abscheren und sich
mit Wasser baden, so ist er rein. Danach gehe er ins Lager; doch soll er
außer seiner Hütte sieben Tage bleiben. \bibverse{9} Und am siebenten
Tage soll er alLev seine Haare abscheren auf dem Haupt, am Barte, an den
Augenbrauen, daß alLev Haare abgeschoren seien, und soll seine Kleider
waschen und sein Fleisch im Wasser baden, so ist er rein. \bibverse{10}
Und am achten Tage soll er zwei Lämmer nehmen ohne Wandel und ein jährig
Schaf ohne Wandel und drei Zehnten Semmelmehl zum Speisopfer, mit Öl
gemenget, und ein Log Öls. \bibverse{11} Da soll der Priester denselben
Gereinigten und diese Dinge stellen vor den HErrn vor der Tür der Hütte
des Stifts. \bibverse{12} Und soll das eine Lamm nehmen und zum
Schuldopfer opfern mit dem Log Öl; und soll solches vor dem HErrn weben;
\bibverse{13} und danach das Lamm schlachten, da man das Sündopfer und
Brandopfer schlachtet, nämlich an heiliger Stätte; denn wie das
Sündopfer, also ist auch das Schuldopfer des Priesters; denn es ist das
Allerheiligste. \bibverse{14} Und der Priester soll des Bluts nehmen vom
Schuldopfer und dem Gereinigten auf den Knorpel des rechten Ohrs tun und
auf den Daumen seiner rechten Hand und auf den großen Zehen seines
rechten Fußes. \bibverse{15} Danach soll er des Öls aus dem Log nehmen
und in seine (des Priesters) linke Hand gießen, \bibverse{16} und mit
seinem rechten Finger in das Öl tunken, das in seiner linken Hand ist,
und sprengen mit seinem Finger das Öl siebenmal vor dem HErrn.
\bibverse{17} Das übrige Öl aber in seiner Hand soll er dem Gereinigten
auf den Knorpel des rechten Ohrs tun und auf den rechten Daumen und auf
den großen Zehen seines rechten Fußes, oben auf das Blut des
Schuldopfers. \bibverse{18} Das übrige Öl aber in seiner Hand soll er
auf des Gereinigten Haupt tun und ihn versöhnen vor dem HErrn.
\bibverse{19} Und soll das Sündopfer machen und den Gereinigten
versöhnen seiner Unreinigkeit halben; und soll danach das Brandopfer
schlachten \bibverse{20} und soll es auf dem Altar opfern samt dem
Speisopfer und ihn versöhnen, so ist er rein. \bibverse{21} Ist er aber
arm und mit seiner Hand nicht so viel erwirbt, so nehme er ein Lamm zum
Schuldopfer zu weben, ihn zu versöhnen, und einen Zehnten Semmelmehl,
mit Öl gemenget, zum Speisopfer und ein Log Öl \bibverse{22} und zwo
Turteltauben oder zwo junge Tauben, die er mit seiner Hand erwerben
kann, daß eine sei ein Sündopfer, die andere ein Brandopfer;
\bibverse{23} und bringe sie am achten Tage seiner Reinigung zum
Priester vor der Tür der Hütte des Stifts, vor dem HErrn. \bibverse{24}
Da soll der Priester das Lamm zum Schuldopfer nehmen und das Log Öl und
soll's alles weben vor dem HErrn; \bibverse{25} und das Lamm des
Schuldopfers schlachten und des Bluts nehmen von demselben Schuldopfer
und dem Gereinigten tun auf den Knorpel seines rechten Ohrs und auf den
Daumen seiner rechten Hand und auf den großen Zehen seines rechten
Fußes; \bibverse{26} und des Öls in seine (des Priesters) linke Hand
gießen \bibverse{27} und mit seinem rechten Finger das Öl, das in seiner
linken Hand ist, siebenmal sprengen vor dem HErrn. \bibverse{28} Des
übrigen aber in seiner Hand soll er dem Gereinigten auf den Knorpel
seines rechten Ohrs und auf den Daumen seiner rechten Hand und auf den
großen Zehen seines rechten Fußes tun, oben auf das Blut des
Schuldopfers. \bibverse{29} Das übrige Öl aber in seiner Hand soll er
dem Gereinigten auf das Haupt tun, ihn zu versöhnen vor dem HErrn,
\bibverse{30} und danach aus der einen Turteltaube oder jungen Taube,
wie seine Hand hat mögen erwerben, \bibverse{31} ein Sündopfer, aus der
andern ein Brandopfer machen samt dem Speisopfer. Und soll der Priester
den Gereinigten also versöhnen vor dem HErrn. \bibverse{32} Das sei das
Gesetz für den Aussätzigen, der mit seiner Hand nicht erwerben kann, was
zu seiner Reinigung gehört. \bibverse{33} Und der HErr redete mit Mose
und Aaron und sprach: \bibverse{34} Wenn ihr ins Land Kanaan kommt, das
ich euch zur Besitzung gebe, und werde irgend in einem Hause eurer
Besitzung ein Aussatzmal geben, \bibverse{35} so soll der kommen, des
das Haus ist, dem Priester ansagen und sprechen: Es siehet mich an, als
sei ein Aussatzmal an meinem Hause. \bibverse{36} Da soll der Priester
heißen, daß sie das Haus ausräumen, ehe denn der Priester hineingehet,
das Mal zu besehen, auf daß nicht unrein werde alles, was im Hause ist;
danach soll der Priester hineingehen, das Haus zu besehen. \bibverse{37}
Wenn er nun das Mal besiehet und findet, daß an der Wand des Hauses
gelbe oder rötliche Grüblein sind und ihr Ansehen tiefer, denn sonst die
Wand ist, \bibverse{38} so soll er zum Hause zur Tür herausgehen und das
Haus sieben Tage verschließen. \bibverse{39} Und wenn er am siebenten
Tage wiederkommt und siehet, daß das Mal weiter gefressen hat an des
Hauses Wand, \bibverse{40} so soll er die Steine heißen ausbrechen,
darin das Mal ist, und hinaus vor die Stadt an einen unreinen Ort
werfen. \bibverse{41} Und das Haus soll man inwendig ringsherum schaben,
und soll den abgeschabten Leimen hinaus vor die Stadt an einen unreinen
Ort schütten \bibverse{42} und andere Steine nehmen und an jener Statt
tun und andern Leimen nehmen und das Haus bewerfen. \bibverse{43} Wenn
dann das Mal wiederkommt und ausbricht am Hause, nachdem man die Steine
ausgerissen und das Haus anders beworfen hat, \bibverse{44} so soll der
Priester hineingehen. Und wenn er siehet, daß das Mal weiter gefressen
hat am Hause, so ist's gewiß ein fressender Aussatz am Hause und ist
unrein. \bibverse{45} Darum soll man das Haus abbrechen, Stein und Holz,
und allen Leimen am Hause, und soll's hinausführen vor die Stadt an
einen unreinen Ort. \bibverse{46} Und wer in das Haus gehet, solange es
verschlossen ist, der ist unrein bis an den Abend. \bibverse{47} Und wer
drinnen liegt oder drinnen isset, der soll seine Kleider waschen.
\bibverse{48} Wo aber der Priester, wenn er hineingehet, siehet, daß
dies Mal nicht weiter am Hause gefressen hat, nachdem das Haus beworfen
ist, so soll er's reinsprechen, denn das Mal ist heil worden.
\bibverse{49} Und soll zum Sündopfer für das Haus nehmen zween Vögel,
Zedernholz und rosinfarbne WolLev und Ysop \bibverse{50} und den einen
Vogel schlachten in einem irdenen Gefäß an einem fließenden Wasser.
\bibverse{51} Und soll nehmen das Zedernholz, die rosinfarbne Wolle, den
Ysop und den lebendigen Vogel und in des geschlachteten Vogels Blut
tunken an dem fließenden Wasser und das Haus siebenmal besprengen.
\bibverse{52} Und soll also das Haus entsündigen mit dem Blut des Vogels
und mit fließendem Wasser, mit dem lebendigen Vogel, mit dem Zedernholz,
mit Ysop und mit rosinfarbner Wolle. \bibverse{53} Und soll den
lebendigen Vogel lassen hinaus vor die Stadt ins freie Feld fliegen und
das Haus versöhnen, so ist's rein. \bibverse{54} Das ist das Gesetz über
allerlei Mal des Aussatzes und Grindes: \bibverse{55} über den Aussatz
der Kleider und der Häuser, \bibverse{56} über die Beulen, Gnätze und
Eiterweiß, \bibverse{57} auf daß man wisse, wenn etwas unrein oder rein
ist. Das ist das Gesetz vom Aussatz.

\hypertarget{section-14}{%
\section{15}\label{section-14}}

\bibverse{1} Und der HErr redete mit Mose und Aaron und sprach:
\bibverse{2} Redet mit den Kindern Israel und sprecht zu ihnen: Wenn ein
Mann an seinem Fleisch einen Fluß hat, derselbe ist unrein. \bibverse{3}
Dann aber ist er unrein an diesem Fluß, wenn sein Fleisch vom Fluß
eitert oder verstopft ist. \bibverse{4} Alles Lager, darauf er liegt,
und alles, darauf er sitzet, wird unrein werden. \bibverse{5} Und wer
sein Lager anrühret; der soll seine Kleider waschen und sich mit Wasser
baden und unrein sein bis auf den Abend. \bibverse{6} Und wer sich
setzt, da er gesessen hat, der soll seine Kleider waschen und sich mit
Wasser baden und unrein sein bis auf den Abend. \bibverse{7} Wer sein
Fleisch anrühret, der soll seine Kleider waschen und sich mit Wasser
baden und unrein sein bis auf den Abend. \bibverse{8} Wenn er seinen
Speichel wirft auf den der rein ist, der soll seine Kleider waschen und
sich mit Wasser baden und unrein sein bis auf den Abend. \bibverse{9}
Und der Sattel, darauf er reitet, wird unrein werden. \bibverse{10} Und
wer anrühret irgend etwas, das er unter sich gehabt hat, der wird unrein
sein bis auf den Abend. Und wer solches trägt, der soll seine Kleider
waschen und sich mit Wasser baden und unrein sein bis auf den Abend.
\bibverse{11} Und welchen er anrühret, ehe er die Hände wäschet, der
soll seine Kleider waschen und sich mit Wasser baden und unrein sein bis
auf den Abend. \bibverse{12} Wenn er ein irden Gefäß anrühret, das soll
man zerbrechen; aber das hölzerne Gefäß soll man mit Wasser spülen.
\bibverse{13} Und wenn er rein wird von seinem Fluß, so soll er sieben
Tage zählen, nach dem er rein worden ist, und seine Kleider waschen und
sein Fleisch mit fließendem Wasser baden, so ist er rein. \bibverse{14}
Und am achten Tage soll er zwo Turteltauben oder zwo junge Tauben nehmen
und vor den HErrn bringen vor der Tür der Hütte des Stifts und dem
Priester geben. \bibverse{15} Und der Priester soll aus einer ein
Sündopfer, aus der andern ein Brandopfer machen und ihn versöhnen vor
dem HErrn seines Flusses halben. \bibverse{16} Wenn einem Mann im Schlaf
der Same entgehet, der soll sein ganzes Fleisch mit Wasser baden und
unrein sein bis auf den Abend. \bibverse{17} Und das Kleid und alles
Fell, das mit solchem Samen beflecket ist, soll er waschen mit Wasser
und unrein sein bis auf den Abend. \bibverse{18} Ein Weib, bei welchem
ein solcher liegt, die soll sich mit Wasser baden und unrein sein bis
auf den Abend. \bibverse{19} Wenn ein Weib ihres Leibes Blutfluß hat,
die soll sieben Tage beiseit getan werden; wer sie anrühret, der wird
unrein sein bis auf den Abend. \bibverse{20} Und alles, worauf sie
liegt, solange sie ihre Zeit hat, wird unrein sein; und worauf sie
sitzt, wird unrein sein. \bibverse{21} Und wer ihr Lager anrühret, der
soll seine Kleider waschen und sich mit Wasser baden und unrein sein bis
auf den Abend. \bibverse{22} Und wer anrühret irgendwas, darauf sie
gesessen hat, soll seine Kleider waschen und sich mit Wasser baden und
unrein sein bis auf den Abend. \bibverse{23} Und wer etwas anrühret, das
auf ihrem Lager, oder wo sie gesessen, gelegen oder gestanden, soll
unrein sein bis auf den Abend. \bibverse{24} Und wenn ein Mann bei ihr
liegt, und es kommt sie ihre Zeit an bei ihm; der wird sieben Tage
unrein sein, und das Lager, darauf er gelegen ist, wird unrein sein.
\bibverse{25} Wenn aber ein Weib ihren Blutfluß eine lange Zeit hat,
nicht allein zur gewöhnlichen Zeit, sondern auch über die gewöhnliche
Zeit, so wird sie unrein sein, solange sie fleußt; wie zur Zeit ihrer
Absonderung, so soll sie auch hie unrein sein. \bibverse{26} Alles
Lager, darauf sie liegt, die ganze Zeit ihres Flusses, soll sein wie das
Lager ihrer Absonderung. Und alles, worauf sie sitzt, wird unrein sein,
gleich der Unreinigkeit ihrer Absonderung. \bibverse{27} Wer der etwas
anrühret, der wird unrein sein und soll seine Kleider waschen und sich
mit Wasser baden und unrein sein bis auf den Abend. \bibverse{28} Wird
sie aber rein von ihrem Fluß, so soll sie sieben Tage zählen; danach
soll sie rein sein. \bibverse{29} Und am achten Tage soll sie zwo
Turteltauben oder zwo junge Tauben nehmen und zum Priester bringen vor
die Tür der Hütte des Stifts. \bibverse{30} Und der Priester soll aus
einer machen ein Sündopfer, aus der andern ein Brandopfer und sie
versöhnen vor dem HErrn über dem Fluß ihrer Unreinigkeit. \bibverse{31}
So sollt ihr die Kinder Israel warnen vor ihrer Unreinigkeit, daß sie
nicht sterben in ihrer Unreinigkeit, wenn sie meine Wohnung
verunreinigen, die unter euch ist. \bibverse{32} Das ist das Gesetz über
den, der einen Fluß hat, und dem der Same im Schlaf entgehet, daß er
unrein davon wird, \bibverse{33} und über die, die ihren Blutfluß hat,
und wer einen Fluß hat, es sei Mann oder Weib, und wenn ein Mann bei
einer Unreinen liegt.

\hypertarget{section-15}{%
\section{16}\label{section-15}}

\bibverse{1} Und der HErr redete mit Mose (nachdem die zween Söhne
Aarons gestorben waren, da sie vor dem HErrn opferten) \bibverse{2} und
sprach: Sage deinem Bruder Aaron, daß er nicht allerlei Zeit in das
inwendige Heiligtum gehe hinter dem Vorhang vor dem Gnadenstuhl, der auf
der Lade ist, daß er nicht sterbe; denn ich will in einer Wolke
erscheinen auf dem Gnadenstuhl. \bibverse{3} Sondern damit soll er
hineingehen: mit einem jungen Farren zum Sündopfer und mit einem Widder
zum Brandopfer. \bibverse{4} Und soll den heiligen leinenen Rock anlegen
und leinen Niederwand an seinem Fleisch haben und sich mit einem
leinenen Gürtel gürten und den leinenen Hut aufhaben; denn das sind die
heiligen Kleider; und soll sein Fleisch mit Wasser baden und sie
anlegen. \bibverse{5} Und soll von der Gemeine der Kinder Israel zween
Ziegenböcke nehmen zum Sündopfer und einen Widder zum Brandopfer.
\bibverse{6} Und Aaron soll den Farren, sein Sündopfer, herzubringen und
sich und sein Haus versöhnen; \bibverse{7} und danach die zween Böcke
nehmen und vor den HErrn stellen vor der Tür der Hütte des Stifts.
\bibverse{8} Und soll das Los werfen über die zween Böcke, ein Los dem
HErrn und das andere dem ledigen Bock. \bibverse{9} Und soll den Bock,
auf welchen des HErrn Los fällt, opfern zum Sündopfer. \bibverse{10}
Aber den Bock, auf welchen das Los des ledigen fällt, soll er lebendig
vor den HErrn stellen, daß er ihn versöhne, und lasse den ledigen Bock
in die Wüste. \bibverse{11} Und also soll er denn den Farren seines
Sündopfers herzubringen und sich und sein Haus versöhnen; und soll ihn
schlachten. \bibverse{12} Und soll einen Napf voll Glut vom Altar
nehmen, der vor dem HErrn stehet, und die Hand voll zerstoßenes
Räuchwerks und hinein hinter den Vorhang bringen; \bibverse{13} und das
Räuchwerk aufs Feuer tun vor dem HErrn, daß der Nebel vom Räuchwerk den
Gnadenstuhl bedecke, der auf dem Zeugnis ist, daß er nicht sterbe.
\bibverse{14} Und soll des Bluts vom Farren nehmen und mit seinem Finger
gegen den Gnadenstuhl sprengen vorne an; siebenmal soll er also vor dem
Gnadenstuhl mit seinem Finger vom Blut sprengen. \bibverse{15} Danach
soll er den Bock, des Volks Sündopfer, schlachten und seines Bluts
hineinbringen hinter den Vorhang; und soll mit seinem Blut tun, wie er
mit des Farren Blut getan hat, und damit auch sprengen vorne gegen den
Gnadenstuhl; \bibverse{16} und soll also versöhnen das Heiligtum von der
Unreinigkeit der Kinder Israel und von ihrer Übertretung in allen ihren
Sünden. Also soll er tun der Hütte des Stifts; denn sie sind unrein, die
umher liegen. \bibverse{17} Kein Mensch soll in der Hütte des Stifts
sein, wenn er hineingehet, zu versöhnen im Heiligtum, bis er herausgehe;
und soll also versöhnen sich und sein Haus und die ganze Gemeine Israel.
\bibverse{18} Und wenn er herausgehet zum Altar, der vor dem HErrn
stehet, soll er ihn versöhnen und soll des Bluts vom Farren und des
Bluts vom Bock nehmen und auf des Altars Hörner umher tun. \bibverse{19}
Und soll mit seinem Finger vom Blut darauf sprengen siebenmal und ihn
reinigen und heiligen von der Unreinigkeit der Kinder Israel.
\bibverse{20} Und wenn er vollbracht hat das Versöhnen des Heiligtums
und der Hütte des Stifts und des Altars, so soll er den lebendigen Bock
herzubringen. \bibverse{21} Da soll denn Aaron seine beiden Hände auf
sein Haupt legen und bekennen auf ihn alLev Missetat der Kinder Israel
und alLev ihre Übertretung in allen ihren Sünden; und soll sie dem Bock
auf das Haupt legen und ihn durch einen Mann, der vorhanden ist, in die
Wüste laufen lassen, \bibverse{22} daß also der Bock alLev ihre Missetat
auf ihm in eine Wildnis trage; und lasse ihn in die Wüste. \bibverse{23}
Und Aaron soll in die Hütte des Stifts gehen und ausziehen die leinenen
Kleider, die er anzog, da er in das Heiligtum ging, und soll sie
daselbst lassen. \bibverse{24} Und soll sein Fleisch mit Wasser baden an
heiliger Stätte und seine eigenen Kleider antun; und herausgehen und
sein Brandopfer und des Volks Brandopfer machen und beide sich und das
Volk versöhnen, \bibverse{25} und das Fett vom Sündopfer auf dem Altar
anzünden. \bibverse{26} Der aber den ledigen Bock hat ausgeführet, soll
seine Kleider waschen und sein Fleisch mit Wasser baden und danach ins
Lager kommen. \bibverse{27} Den Farren des Sündopfers und den Bock des
Sündopfers, welcher Blut in das Heiligtum zu versöhnen gebracht wird,
soll man hinausführen vor das Lager und mit Feuer verbrennen, beide ihre
Haut, Fleisch und Mist. \bibverse{28} Und der sie verbrennet, soll seine
Kleider waschen und sein Fleisch mit Wasser baden und danach ins Lager
kommen. \bibverse{29} Auch soll euch das ein ewiges Recht sein: Am
zehnten Tage des siebenten Monden sollt ihr euren Leib kasteien und kein
Werk tun, er sei einheimisch oder fremd unter euch. \bibverse{30} Denn
an diesem Tage geschieht eure Versöhnung, daß ihr gereiniget werdet; von
allen euren Sünden werdet ihr gereiniget vor dem Herrn. \bibverse{31}
Darum soll's euch der größte Sabbat sein, und ihr sollt euren Leib
demütigen. Ein ewig Recht sei das! \bibverse{32} Es soll aber solche
Versöhnung tun ein Priester, den man geweihet, und des Hand man gefüllet
hat zum Priester an seines Vaters Statt; und soll die leinenen Kleider
antun, nämlich die heiligen Kleider. \bibverse{33} Und soll also
versöhnen das heilige Heiligtum und die Hütte des Stifts und den Altar
und die Priester und alles Volk der Gemeine. \bibverse{34} Das soll euch
ein ewiges Recht sein, daß ihr die Kinder Israel versöhnet von allen
ihren Sünden im Jahr einmal Und Mose tat, wie ihm der HErr geboten
hatte.

\hypertarget{section-16}{%
\section{17}\label{section-16}}

\bibverse{1} Und der HErr redete mit Mose und sprach: \bibverse{2} Sage
Aaron und seinen Söhnen und allen Kindern Israel und sprich zu ihnen:
Das ist's, das der HErr geboten hat: \bibverse{3} Welcher aus dem Hause
Israel einen Ochsen oder Lamm oder Ziege schlachtet in dem Lager oder
außen vor dem Lager \bibverse{4} und nicht vor die Tür der Hütte des
Stifts bringet, daß es dem HErrn zum Opfer gebracht werde vor der
Wohnung des HErrn, der soll des Bluts schuldig sein, als der Blut
vergossen hat, und solcher Mensch soll ausgerottet werden aus seinem
Volk. \bibverse{5} Darum sollen die Kinder Israel ihre Opfer, die sie
auf dem freien Felde opfern wollen, vor den HErrn bringen, vor die Türe
der Hütte des Stifts, zum Priester, und allda ihre Dankopfer dem HErrn
opfern. \bibverse{6} Und der Priester soll das Blut auf den Altar des
HErrn sprengen vor der Tür der Hütte des Stifts und das Fett anzünden
zum süßen Geruch dem HErrn; \bibverse{7} und mitnichten ihre Opfer
hinfort den Feldteufeln opfern, mit denen sie huren. Das soll ihnen ein
ewiges Recht sein bei ihren Nachkommen. \bibverse{8} Darum sollst du zu
ihnen sagen: Welcher Mensch aus dem Hause Israel, oder auch ein
Fremdling, der unter euch ist, der ein Opfer oder Brandopfer tut,
\bibverse{9} und bringt's nicht vor die Tür der Hütte des Stifts, daß
er's dem HErrn tue, der soll ausgerottet werden von seinem Volk.
\bibverse{10} Und welcher Mensch, er sei vom Hause Israel, oder ein
Fremdling unter euch, irgend Blut isset, wider den will ich mein Antlitz
setzen und will ihn mitten aus seinem Volk rotten. \bibverse{11} Denn
des Leibes Leben ist im Blut, und ich hab's euch zum Altar gegeben, daß
eure Seelen damit versöhnet werden. Denn das Blut ist die Versöhnung für
das Leben. \bibverse{12} Darum hab ich gesagt den Kindern Israel: Keine
SeeLev unter euch soll Blut essen, auch kein Fremdling, der unter euch
wohnet. \bibverse{13} Und welcher Mensch, er sei vom Hause Israel oder
ein Fremdling unter euch, der ein Tier oder Vogel fähet auf der Jagd,
das man isset, der soll desselben Blut vergießen und mit Erde
zuscharren. \bibverse{14} Denn des Leibes Leben ist in seinem Blut,
solange es lebet; und ich habe den Kindern Israel gesagt: Ihr sollt
keines Leibes Blut essen. Denn des Leibes Leben ist in seinem Blut. Wer
es isset, der soll ausgerottet werden. \bibverse{15} Und welche SeeLev
ein Aas, oder was vom Wilde zerrissen ist, isset, er sei ein
Einheimischer oder Fremdling, der soll sein Kleid waschen und sich mit
Wasser baden und unrein sein bis auf den Abend, so wird er rein.
\bibverse{16} Wo er seine Kleider nicht waschen noch sich baden wird, so
soll er seiner Missetat schuldig sein.

\hypertarget{section-17}{%
\section{18}\label{section-17}}

\bibverse{1} Und der HErr redete mit Mose und sprach: \bibverse{2} Rede
mit den Kindern Israel und sprich zu ihnen: Ich bin der HErr, euer GOtt.
\bibverse{3} Ihr sollt nicht tun nach den Werken des Landes Ägypten,
darinnen ihr gewohnet habt, auch nicht nach den Werken des Landes
Kanaan, darein ich euch führen will; ihr sollt auch euch nach ihrer
Weise nicht halten. \bibverse{4} Sondern nach einen Rechten sollt ihr
tun und meine Satzungen sollt ihr halten, daß ihr darinnen wandelt; denn
ich bin der HErr, euer GOtt. \bibverse{5} Darum sollt ihr meine
Satzungen halten und meine Rechte. Denn welcher Mensch dieselben tut,
der wird dadurch leben; denn ich bin der HErr. \bibverse{6} Niemand soll
sich zu seiner nächsten Blutsfreundin tun, ihre Scham zu blößen; denn
ich bin der HErr. \bibverse{7} Du sollst deines Vaters und deiner Mutter
Scham nicht blößen; es ist deine Mutter, darum sollst du ihre Scham
nicht blößen. \bibverse{8} Du sollst deines Vaters Weibes Scham nicht
blößen; denn es ist deines Vaters Scham. \bibverse{9} Du sollst deiner
Schwester Scham, die deines Vaters oder deiner Mutter Tochter ist,
daheim oder draußen geboren, nicht blößen. \bibverse{10} Du sollst
deines Sohns oder deiner Tochter Tochter Scham nicht blößen; denn es ist
deine Scham. \bibverse{11} Du sollst der Tochter deines Vaters Weibes,
die deinem Vater geboren ist und deine Schwester ist, Scham nicht
blößen. \bibverse{12} Du sollst deines Vaters Schwester Scham nicht
blößen; denn es ist deines Vaters nächste Blutsfreundin. \bibverse{13}
Du sollst deiner Mutter Schwester Scham nicht blößen denn es ist deiner
Mutter nächste Blutsfreundin. \bibverse{14} Du sollst deines Vaters
Bruders Scham nicht blößen, daß du sein Weib nehmest; denn sie ist deine
Base. \bibverse{15} Du sollst deiner Schnur Scham nicht blößen; denn es
ist deines Sohns Weib, darum sollst du ihre Scham nicht blößen.
\bibverse{16} Du sollst deines Bruders Weibes Scham nicht blößen; denn
sie ist deines Bruders Scham. \bibverse{17} Du sollst deines Weibes samt
ihrer Tochter Scham nicht blößen, noch ihres Sohns Tochter oder Tochter
Tochter nehmen, ihre Scham zu blößen; denn es ist ihre nächste
Blutsfreundin, und ist ein Laster. \bibverse{18} Du sollst auch deines
Weibes Schwester nicht nehmen neben ihr, ihre Scham zu blößen, ihr
zuwider, weil sie noch lebet. \bibverse{19} Du sollst nicht zum Weibe
gehen, weil sie ihre Krankheit hat, in ihrer Unreinigkeit ihre Scham zu
blößen. \bibverse{20} Du sollst auch nicht bei deines nächsten Weib
liegen, sie zu besamen, damit du dich an ihr verunreinigest.
\bibverse{21} Du sollst auch deines Samens nicht geben, daß es dem
Molech verbrannt werde, daß du nicht entheiligest den Namen deines
GOttes; denn ich bin der HErr. \bibverse{22} Du sollst nicht bei Knaben
liegen wie beim Weibe; denn es ist ein Greuel. \bibverse{23} Du sollst
auch bei keinem Tier liegen, daß du mit ihm verunreiniget werdest. Und
kein Weib soll mit einem Tier zu schaffen haben; denn es ist ein Greuel.
\bibverse{24} Ihr sollt euch in dieser keinem verunreinigen; denn in
diesem allem haben sich verunreiniget die Heiden, die ich vor euch her
will ausstoßen, \bibverse{25} und das Land dadurch verunreiniget ist.
Und ich will ihre Missetat an ihnen heimsuchen, daß das Land seine
Einwohner ausspeie. \bibverse{26} Darum haltet meine Satzungen und
Rechte und tut dieser Greuel keine, weder der Einheimische noch der
Fremdling unter euch; \bibverse{27} denn alLev solche Greuel haben die
Leute dieses Landes getan, die vor euch waren, und haben das Land
verunreiniget; \bibverse{28} auf daß euch nicht auch das Land ausspeie,
wenn ihr es verunreiniget, gleichwie es die Heiden hat ausgespeiet, die
vor euch waren. \bibverse{29} Denn welche diese Greuel tun, deren Seelen
sollen ausgerottet werden von ihrem Volk. \bibverse{30} Darum haltet
meine Satzung, daß ihr nicht tut nach den greulichen Sitten, die vor
euch waren, daß ihr nicht damit verunreiniget werdet; denn ich bin der
HErr, euer GOtt.

\hypertarget{section-18}{%
\section{19}\label{section-18}}

\bibverse{1} Und der HErr redete mit Mose und sprach: \bibverse{2} Rede
mit der ganzen Gemeine der Kinder Israel und sprich zu ihnen: Ihr sollt
heilig sein; denn ich bin heilig, der HErr, euer GOtt. \bibverse{3} Ein
jeglicher fürchte seine Mutter und seinen Vater. Haltet meine Feiertage;
denn ich bin der HErr, euer GOtt. \bibverse{4} Ihr sollt euch nicht zu
den Götzen wenden und sollt euch keine gegossenen Götter machen; denn
ich bin der HErr, euer GOtt. \bibverse{5} Und wenn ihr dem HErrn wollt
Dankopfer tun, so sollt ihr opfern, das ihm gefallen könnte.
\bibverse{6} Aber ihr sollt es desselben Tages essen, da ihr's opfert,
und des andern Tages; was aber auf den dritten Tag überbleibet, soll man
mit Feuer verbrennen. \bibverse{7} Wird aber jemand am dritten Tage
davon essen, so ist er ein Greuel und wird nicht angenehm sein.
\bibverse{8} Und derselbe Esser wird seine Missetat tragen, daß er das
Heiligtum des HErrn entheiligte, und solche SeeLev wird ausgerottet
werden von ihrem Volk. \bibverse{9} Wenn, du dein Land einerntest,
sollst du es nicht an den Enden umher abschneiden, auch nicht alles
genau aufsammeln. \bibverse{10} Also auch sollst du deinen Weinberg
nicht genau lesen noch die abgefallenen Beeren auflesen, sondern dem
Armen und Fremdling sollst du es lassen; denn ich bin der HErr, euer
GOtt. \bibverse{11} Ihr sollt nicht stehlen, noch lügen, noch fälschlich
handeln, einer mit dem andern. \bibverse{12} Ihr sollt nicht falsch
schwören bei meinem Namen und entheiligen den Namen deines GOttes; denn
ich bin der HErr. \bibverse{13} Du sollst deinem Nächsten nicht unrecht
tun noch berauben. Es soll des Taglöhners Lohn nicht bei dir bleiben bis
an den Morgen. \bibverse{14} Du sollst dem Tauben nicht fluchen. Du
sollst vor dem Blinden keinen Anstoß setzen; denn du sollst dich vor
deinem GOtt fürchten; denn ich bin der HErr. \bibverse{15} Ihr sollt
nicht unrecht handeln am Gericht, und sollst nicht vorziehen den
Geringen, noch den Großen ehren, sondern du sollst deinen Nächsten recht
richten. \bibverse{16} Du sollst kein Verleumder sein unter deinem Volk.
Du sollst auch nicht stehen wider deines Nächsten Blut; denn ich bin der
HErr. \bibverse{17} Du sollst deinen Brüder nicht hassen in deinem
Herzen sondern du sollst deinen Nächsten strafen, auf daß du nicht
seinethalben Schuld tragen müssest. \bibverse{18} Du sollst nicht
rachgierig sein, noch Zorn halten gegen die Kinder deines Volks. Du
sollst deinen Nächsten lieben wie dich selbst; denn ich bin der HErr.
\bibverse{19} Meine Satzungen sollt ihr halten, daß du dein Vieh nicht
lassest mit anderlei Tier zu schaffen haben und dein Feld nicht besäest
mit mancherlei Samen, und kein Kleid an dich komme, das mit WolLev und
Leinen gemenget ist. \bibverse{20} Wenn ein Mann bei einem Weibe liegt
und sie beschläft, die eine leibeigene Magd und von dem Manne
verschmähet ist, doch nicht erlöset, noch Freiheit erlanget hat: das
soll gestraft werden; aber sie sollen nicht sterben, denn sie ist nicht
frei gewesen. \bibverse{21} Er soll aber für seine Schuld dem HErrn vor
die Tür der Hütte des Stifts einen Widder zum Schuldopfer bringen.
\bibverse{22} Und der Priester soll ihn versöhnen mit dem Schuldopfer
vor dem HErrn über der Sünde, die er getan hat, so wird ihm GOtt gnädig
sein über seine Sünde, die er getan hat. \bibverse{23} Wenn ihr ins Land
kommt und allerlei Bäume pflanzet, davon man isset, sollt ihr derselben
Vorhaut beschneiden und ihre Früchte. Drei Jahre sollt ihr sie
unbeschnitten achten, daß ihr nicht esset. \bibverse{24} Im vierten Jahr
aber sollen alLev ihre Früchte heilig, und gepreiset sein dem HErrn.
\bibverse{25} Im fünften Jahr aber sollt ihr die Früchte essen und sie
einsammeln; denn ich bin der HErr, euer GOtt. \bibverse{26} Ihr sollt
nichts mit Blut essen. Ihr sollt nicht auf Vogelgeschrei achten noch
Tage wählen. \bibverse{27} Ihr sollt euer Haar am Haupt nicht rund umher
abschneiden, noch euren Bart gar abscheren. \bibverse{28} Ihr sollt kein
Mal um eines Toten willen an eurem Leibe reißen, noch Buchstaben an euch
pfetzen; denn ich bin der HErr. \bibverse{29} Du sollst deine Tochter
nicht zur Hurerei halten, daß nicht das Land Hurerei treibe und werde
voll Lasters. \bibverse{30} Meine Feier haltet und fürchtet euch vor
meinem Heiligtum; denn ich bin der HErr. \bibverse{31} Ihr sollt euch
nicht wenden zu den Wahrsagern und forschet nicht von den
Zeichendeutern, daß ihr nicht an ihnen verunreiniget werdet; denn ich
bin der HErr, euer GOtt. \bibverse{32} Vor einem grauen Haupt sollst du
aufstehen und die Alten ehren; denn du sollst dich fürchten vor deinem
GOtt; denn ich bin der HErr. \bibverse{33} Wenn ein Fremdling bei dir in
eurem Lande wohnen wird, den sollt ihr nicht schinden. \bibverse{34} Er
soll bei euch wohnen wie ein Einheimischer unter euch, und sollst ihn
lieben wie dich selbst; denn ihr seid auch Fremdlinge gewesen in
Ägyptenland. Ich bin der HErr, euer GOtt. \bibverse{35} Ihr sollt nicht
ungleich handeln am Gericht mit der Elle, mit Gewicht, mit Maß.
\bibverse{36} Rechte Waage, rechte Pfunde, rechte Scheffel, rechte
Kannen sollen bei euch sein; denn ich bin der HErr, euer GOtt, der euch
aus Ägyptenland geführet hat, \bibverse{37} daß ihr alLev meine
Satzungen und alLev meine Rechte haltet und tut; denn ich bin der HErr.

\hypertarget{section-19}{%
\section{20}\label{section-19}}

\bibverse{1} Und der HErr redete mit Mose und sprach: \bibverse{2} Sage
den Kindern Israel: Welcher unter den Kindern Israel oder ein Fremdling,
der in Israel wohnet, seines Samens dem Molech gibt, der soll des Todes
sterben; das Volk im Lande soll ihn steinigen. \bibverse{3} Und ich will
mein Antlitz setzen wider solchen Menschen und will ihn aus seinem Volk
rotten, daß er dem Molech seines Samens gegeben und mein Heiligtum
verunreiniget und meinen heiligen Namen entheiliget hat. \bibverse{4}
Und wo das Volk im Lande durch die Finger sehen würde dem Menschen, der
seines Samens dem Molech gegeben hat, daß es ihn nicht tötet,
\bibverse{5} so will doch ich mein Antlitz wider denselben Menschen
Setzen und wider sein Geschlecht; und will ihn und alle, die ihm
nachgehuret haben mit dem Molech, aus ihrem Volk rotten. \bibverse{6}
Wenn eine SeeLev sich zu den Wahrsagern und Zeichendeutern wenden wird,
daß sie ihnen nachhuret, so will ich mein Antlitz wider dieselbe SeeLev
setzen und will sie aus ihrem Volk rotten. \bibverse{7} Darum heiliget
euch und seid heilig; denn ich bin der HErr, euer GOtt. \bibverse{8} Und
haltet meine Satzungen und tut sie; denn ich bin der HErr, der euch
heiliget. \bibverse{9} Wer seinem Vater oder seiner Mutter fluchet, der
soll des Todes sterben. Sein Blut sei auf ihm, daß er seinem Vater oder
Mutter gefluchet hat! \bibverse{10} Wer die Ehe bricht mit jemandes
Weibe, der soll des Todes sterben, beide Ehebrecher und Ehebrecherin,
darum daß er mit seines Nächsten Weibe die Ehe gebrochen hat.
\bibverse{11} Wenn jemand bei seines Vaters Weibe schläft, daß er seines
Vaters Scham geblößet hat, die sollen beide des Todes sterben. Ihr Blut
sei auf ihnen! \bibverse{12} Wenn jemand bei seiner Schnur schläft, so
sollen sie beide des Todes sterben; denn sie haben eine Schande
begangen. Ihr Blut sei auf ihnen! \bibverse{13} Wenn jemand beim Knaben
schläft wie beim Weibe, die haben einen Greuel getan und sollen beide
des Todes sterben. Ihr Blut sei auf ihnen! \bibverse{14} Wenn jemand ein
Weib nimmt und ihre Mutter dazu, der hat ein Laster verwirkt; man soll
ihn mit Feuer verbrennen, und sie beide auch, daß kein Laster sei unter
euch. \bibverse{15} Wenn jemand beim Vieh liegt, der soll des Todes
sterben, und das Vieh soll man erwürgen. \bibverse{16} Wenn ein Weib
sich irgend zu einem Vieh tut, daß sie mit ihm zu schaffen hat, die
sollst du töten, und das Vieh auch; des Todes sollen sie sterben. Ihr
Blut sei auf ihnen! \bibverse{17} Wenn jemand seine Schwester nimmt,
seines Vaters Tochter oder seiner Mutter Tochter, und ihre Scham
beschauet, und sie wieder seine Scham: das ist eine Blutschande; die
sollen ausgerottet wer den vor den Leuten ihres Volks, denn er hat
seiner Schwester Scham entblößet. Er soll seine Missetat tragen.
\bibverse{18} Wenn ein Mann beim Weibe schläft zur Zeit ihrer Krankheit
und entblößet ihre Scham und decket ihren Brunnen auf, und sie entblößet
den Brunnen ihres Bluts, die sollen beide aus ihrem Volk gerottet
werden. \bibverse{19} Deiner Mutter Schwester Scham und deines Vaters
Schwester Scham sollst du nicht blößen; denn ein solcher hat seine
nächste Blutsfreundin aufgedecket, und sie sollen ihre Missetat tragen.
\bibverse{20} Wenn jemand bei seines Vaters Bruders Weibe schläft, der
hat seines Vetters Scham geblößet; sie sollen ihre Sünde tragen: ohne
Kinder sollen sie sterben. \bibverse{21} Wenn jemand seines Bruders Weib
nimmt, das ist eine schändliche Tat; sie sollen ohne Kinder sein, darum
daß er hat seines Bruders Scham geblößet. \bibverse{22} So haltet nun
alLev meine Satzungen und meine Rechte und tut danach, auf daß euch
nicht das Land ausspeie, darein ich euch führe, daß ihr drinnen wohnet.
\bibverse{23} Und wandelt nicht in den Satzungen der Heiden, die ich vor
euch her werde ausstoßen. Denn solches alles haben sie getan, und ich
habe einen Greuel an ihnen gehabt. \bibverse{24} Euch aber sage ich: Ihr
sollt jener Land besitzen; denn ich will euch ein Land zum Erbe geben,
darin Milch und Honig fleußt. Ich bin der HErr, euer GOtt, der euch von
den Völkern abgesondert hat, \bibverse{25} daß ihr auch absondern sollt
das reine Vieh vom unreinen und unreine Vögel von den reinen, und eure
Seelen nicht verunreiniget am Vieh, an Vögeln und an allem, das auf
Erden kreucht, das ich euch abgesondert habe, daß es unrein sei.
\bibverse{26} Darum sollt ihr mir heilig sein; denn ich, der HErr, bin
heilig, der euch abgesondert hat von den Völkern, daß ihr mein wäret.
\bibverse{27} Wenn ein Mann oder Weib ein Wahrsager oder Zeichendeuter
sein wird, die sollen des Todes sterben, man soll sie steinigen. Ihr
Blut sei auf ihnen!

\hypertarget{section-20}{%
\section{21}\label{section-20}}

\bibverse{1} Und der HErr sprach zu Mose: Sage den Priestern, Aarons
Söhnen, und sprich zu ihnen: Ein Priester soll sich an keinem Toten
seines Volks verunreinigen, \bibverse{2} ohne an seinem Blutsfreunde,
der ihm am nächsten angehöret, als an seiner Mutter, an seinem Vater, an
seinem Sohne, an seiner Tochter, an seinem Bruder \bibverse{3} und an
seiner Schwester, die noch eine Jungfrau und noch bei ihm ist und keines
Mannes Weib gewesen ist; an der mag er sich verunreinigen. \bibverse{4}
Sonst soll er sich nicht verunreinigen an irgend einem, der ihm
zugehöret unter seinem Volk, daß er sich entheilige. \bibverse{5} Sie
sollen auch keine Platte machen auf ihrem Haupte, noch ihren Bart
abscheren, und an ihrem Leibe kein Mal pfetzen. \bibverse{6} Sie sollen
ihrem GOtt heilig sein und nicht entheiligen den Namen ihres GOttes.
Denn sie opfern des HErrn Opfer, das Brot ihres GOttes; darum sollen sie
heilig sein. \bibverse{7} Sie sollen keine Hure nehmen, noch keine
Geschwächte, oder die von ihrem Manne verstoßen ist; denn er ist heilig
seinem GOtt. \bibverse{8} Darum sollst du ihn heilig halten, denn er
opfert das Brot deines GOttes; er soll dir heilig sein, denn ich bin
heilig, der HErr, der euch heiliget. \bibverse{9} Wenn eines Priesters
Tochter anfähet zu huren, die soll man mit Feuer verbrennen; denn sie
hat ihren Vater geschändet. \bibverse{10} Welcher Hoherpriester ist
unter seinen Brüdern, auf des Haupt das Salböl gegossen und seine Hand
gefüllet ist, daß er angezogen würde mit den Kleidern, der soll sein
Haupt nicht blößen und seine Kleider nicht zerschneiden; \bibverse{11}
und soll zu keinem Toten kommen und soll sich weder über Vater noch über
Mutter verunreinigen. \bibverse{12} Aus dem Heiligtum soll er nicht
gehen, daß er nicht entheilige das Heiligtum seines GOttes; denn die
heilige Krone, das Salböl seines GOttes, ist auf ihm. Ich bin der HErr.
\bibverse{13} Eine Jungfrau soll er zum Weibe nehmen. \bibverse{14} Aber
keine Witwe noch Verstoßene, noch Geschwächte, noch Hure, sondern eine
Jungfrau seines Volks soll er zum Weibe nehmen, \bibverse{15} auf daß er
nicht seinen Samen entheilige unter seinem Volk; denn ich bin der HErr,
der ihn heiliget. \bibverse{16} Und der HErr redete mit Mose und sprach:
\bibverse{17} Rede mit Aaron und sprich: Wenn an jemand deines Samens in
euren Geschlechtern ein Fehl ist, der soll nicht herzutreten, daß er das
Brot seines GOttes opfere. \bibverse{18} Denn keiner, an dem ein Fehl
ist, soll herzutreten. Er sei blind, lahm, mit einer seltsamen Nase, mit
ungewöhnlichem Gliede, \bibverse{19} oder der an einem Fuß oder Hand
gebrechlich ist, \bibverse{20} oder höckericht ist, oder ein Fell auf
dem Auge hat, oder scheel ist, oder grindicht, oder schäbicht, oder der
gebrochen ist. \bibverse{21} Welcher nun von Aarons, des Priesters Samen
einen Fehl an ihm hat, der soll nicht herzutreten, zu opfern die Opfer
des HErrn; denn er hat einen Fehl, darum soll er zu den Broten seines
GOttes nicht nahen, daß er sie opfere. \bibverse{22} Doch soll er das
Brot seines GOttes essen, beide von dem heiligen und vom
allerheiligsten. \bibverse{23} Aber doch zum Vorhang soll er nicht
kommen, noch zum Altar nahen, weil der Fehl an ihm ist, daß er nicht
entheilige mein Heiligtum; denn ich bin der HErr, der sie heiliget.
\bibverse{24} Und Mose redete solches zu Aaron und zu seinen Söhnen und
zu allen Kindern Israel.

\hypertarget{section-21}{%
\section{22}\label{section-21}}

\bibverse{1} Und der HErr redete mit Mose und sprach: \bibverse{2} Sage
Aaron und seinen Söhnen, daß sie sich enthalten von dem Heiligen der
Kinder Israel, welches sie mir heiligen, und meinen heiligen Namen nicht
entheiligen; denn ich bin der HErr. \bibverse{3} So sage nun ihnen auf
ihre Nachkommen: Welcher eures Samens herzutritt zu dem Heiligen, das
die Kinder Israel dem HErrn heiligen, und verunreiniget sich also über
demselben, des SeeLev soll ausgerottet werden von meinem Antlitz; denn
ich bin der HErr. \bibverse{4} Welcher des Samens Aarons aussätzig ist
oder einen Fluß hat, der soll nicht essen von dem Heiligen, bis er rein
werde. Wer etwa einen unreinen Leib anrühret, oder welchem der Same
entgehet im Schlaf, \bibverse{5} und welcher irgend ein Gewürm anrühret,
das ihm unrein ist, oder einen Menschen, der ihm unrein ist, und alles,
was ihn verunreiniget: \bibverse{6} welche SeeLev der eines anrühret,
die ist unrein bis auf den Abend und soll von dem Heiligen nicht essen,
sondern soll zuvor seinen Leib mit Wasserbaden. \bibverse{7} Und wenn
die Sonne untergegangen, und er rein worden ist, dann mag er davon
essen; denn es ist seine Nahrung. \bibverse{8} Ein Aas, und was von
wilden Tieren zerrissen ist, soll er nicht essen, auf daß er nicht
unrein daran werde denn ich bin der HErr. \bibverse{9} Darum sollen sie
meine Sätze halten, daß sie nicht Sünde auf sich laden und daran
sterben, wenn sie sich entheiligen; denn ich bin der HErr, der sie
heiliget. \bibverse{10} Kein anderer soll von dem Heiligen essen, noch
des Priesters Hausgenoß, noch Taglöhner. \bibverse{11} Wenn aber der
Priester eine SeeLev um sein Geld kaufet, der mag davon essen; und was
ihm in seinem Hause geboren wird, das mag auch von seinem Brot essen.
\bibverse{12} Wenn aber des Priesters Tochter eines Fremden Weib wird,
die soll nicht von der heiligen Hebe essen. \bibverse{13} Wird sie aber
eine Witwe, oder ausgestoßen, und hat keinen Samen und kommt wieder zu
ihres Vaters Hause, so soll sie essen von ihres Vaters Brot, als da sie
noch eine Magd war. Aber kein Fremdling soll davon essen. \bibverse{14}
Wer's versiehet und sonst von dem Heiligen isset, der soll das fünfte
Teil dazu tun und dem Priester geben samt dem Heiligen, \bibverse{15}
auf daß sie nicht entheiligen das Heilige der Kinder Israel, das sie dem
HErrn heben, \bibverse{16} auf daß sie sich nicht mit Missetat und
Schuld beladen, wenn sie ihr Geheiligtes essen; denn ich bin der HErr,
der sie heiliget. \bibverse{17} Und der HErr redete mit Mose und sprach:
\bibverse{18} Sage Aaron und seinen Söhnen und allen Kindern Israel:
Welcher Israeliter oder Fremdling in Israel sein Opfer tun will, es sei
irgend ihr Gelübde oder von freiem Willen, daß sie dem HErrn ein
Brandopfer tun wollen, das ihm von euch angenehm sei, \bibverse{19} das
soll ein Männlein und ohne Wandel sein, von Rindern oder Lämmern oder
Ziegen. \bibverse{20} Alles, was einen Fehl hat, sollt ihr nicht opfern;
denn es wird für euch nicht angenehm sein. \bibverse{21} Und wer ein
Dankopfer dem HErrn tun will, ein sonderlich Gelübde oder von freiem
Willen, von Rindern oder Schafen, das soll ohne Wandel sein, daß es
angenehm sei; es soll keinen Fehl haben. \bibverse{22} Ist's blind, oder
gebrechlich, oder geschlagen, oder dürre, oder räudicht, oder schäbicht,
so sollt ihr solches dem HErrn nicht opfern und davon kein Opfer geben
auf den Altar des HErrn. \bibverse{23} Einen Ochsen oder Schaf, das
ungewöhnliche Glieder oder wandelbare Glieder hat, magst du von freiem
Willen opfern; aber angenehm mag's nicht sein zum Gelübde. \bibverse{24}
Du sollst auch dem HErrn kein Zerstoßenes, oder Zerriebenes, oder
Zerrissenes, oder das verwundet ist, opfern, und sollt in eurem Lande
solches nicht tun. \bibverse{25} Du sollst auch solcher keins von eines
Fremdlings Hand neben dem Brot eures GOttes opfern; denn es taugt nicht
und hat einen Fehl, darum wird es nicht angenehm sein für euch.
\bibverse{26} Und der HErr redete mit Mose und sprach: \bibverse{27}
Wenn ein Ochse, oder Lamm, oder Ziege geboren ist, so soll es sieben
Tage bei seiner Mutter sein, und am achten Tage und danach mag man's dem
HErrn opfern, so ist's angenehm. \bibverse{28} Es sei ein Ochse oder
Lamm, so soll man's nicht mit seinem Jungen auf einen Tag schlachten.
\bibverse{29} Wenn ihr aber wollt dem HErrn ein Lobopfer tun, das für
euch angenehm sei, \bibverse{30} so sollt ihr's desselben Tages essen
und sollt nichts übrig bis auf den Morgen behalten; denn ich bin der
HErr. \bibverse{31} Darum haltet meine Gebote und tut danach; denn ich
bin der HErr. \bibverse{32} Daß ihr meinen heiligen Namen nicht
entheiliget, und ich geheiliget werde unter den Kindern Israel; denn ich
bin der HErr, der euch heiliget, \bibverse{33} der euch aus Ägyptenland
geführet hat, daß ich euer GOtt wäre, ich der HErr.

\hypertarget{section-22}{%
\section{23}\label{section-22}}

\bibverse{1} Und der HErr redete mit Mose und sprach: \bibverse{2} Sage
den Kindern Israel und sprich zu ihnen: Dies sind die Feste des HErrn,
die ihr heilig und meine Feste heißen sollt, da ihr zusammenkommt.
\bibverse{3} Sechs Tage sollst du arbeiten; der siebente Tag aber ist
der große heilige Sabbat, da ihr zusammenkommt. Keine Arbeit sollt ihr
drinnen tun; denn es ist der Sabbat des HErrn in allen euren Wohnungen.
\bibverse{4} Dies sind aber die Feste des HErrn, die ihr heilige Feste
heißen sollt, da ihr zusammenkommt: \bibverse{5} Am vierzehnten Tage des
ersten Monden zwischen Abend, ist des HErrn Passah. \bibverse{6} Und am
fünfzehnten desselben Monden ist das Fest der ungesäuerten Brote des
HErrn; da sollt ihr sieben Tage ungesäuert Brot essen. \bibverse{7} Der
erste Tag soll heilig unter euch heißen, da ihr zusammenkommt; da sollt
ihr keine Dienstarbeit tun \bibverse{8} und dem HErrn opfern sieben
Tage. Der siebente Tag soll auch heilig heißen, da ihr zusammenkommt; da
sollt ihr auch keine Dienstarbeit tun. \bibverse{9} Und der HErr redete
mit Mose und sprach: \bibverse{10} Sage den Kindern Israel und sprich zu
ihnen: Wenn ihr ins Land kommt, das ich euch geben werde, und werdet es
ernten, so sollt ihr eine Garbe der Erstlinge eurer Ernte zu dem
Priester bringen. \bibverse{11} Da soll die Garbe gewebet werden vor dem
HErrn, daß es von euch angenehm sei; solches soll aber der Priester tun
des andern Tages nach dem Sabbat. \bibverse{12} Und sollt des Tages, da
eure Garbe gewebet wird, ein Brandopfer dem HErrn tun von einem Lamm,
das ohne Wandel und jährig sei, \bibverse{13} samt dem Speisopfer, zwo
Zehnten Semmelmehl mit Öl gemenget, zum Opfer dem HErrn eines süßen
Geruchs; dazu das Trankopfer, ein Viertel Hin Weins. \bibverse{14} Und
sollt kein neu Brot, noch Sangen, noch Korn zuvor essen, bis auf den
Tag, da ihr eurem GOtt Opfer bringet. Das soll ein Recht sein euren
Nachkommen in allen euren Wohnungen. \bibverse{15} Danach sollt ihr
zählen vom andern Tage des Sabbats, da ihr die Webegarbe brachtet,
sieben ganzer Sabbate; \bibverse{16} bis an den andern Tag des siebenten
Sabbats, nämlich fünfzig Tage sollt ihr zählen, und neu Speisopfer dem
HErrn opfern. \bibverse{17} Und sollt es aus allen euren Wohnungen
opfern, nämlich zwei Webebrote von zwo Zehnten Semmelmehl, gesäuert und
gebacken, zu Erstlingen dem HErrn. \bibverse{18} Und sollt herzubringen
neben eurem Brot sieben jährige Lämmer ohne Wandel und einen jungen
Farren und zween Widder. Das soll des HErrn Brandopfer, Speisopfer und
Trankopfer sein; das ist ein Opfer eines süßen Geruchs dem HErrn.
\bibverse{19} Dazu sollt ihr machen einen Ziegenbock zum Sündopfer und
zwei jährige Lämmer zum Dankopfer. \bibverse{20} Und der Priester soll's
weben samt dem Brot der Erstlinge vor dem HErrn und den zweien Lämmern;
und soll dem HErrn heilig und des Priesters sein. \bibverse{21} Und
sollt diesen Tag ausrufen, denn er soll unter euch heilig heißen, da ihr
zusammenkommt; keine Dienstarbeit sollt ihr tun. Ein ewiges Recht soll
das sein bei euren Nachkommen in allen euren Wohnungen. \bibverse{22}
Wenn ihr aber euer Land erntet, sollt ihr's nicht gar auf dem Felde
einschneiden, auch nicht alles genau auflesen, sondern sollt es den
Armen und Fremdlingen lassen. Ich bin der HErr, euer GOtt. \bibverse{23}
Und der HErr redete mit Mose und sprach: \bibverse{24} Rede mit den
Kindern Israel und sprich: Am ersten Tage des siebenten Monden sollt ihr
den heiligen Sabbat des Blasens zum Gedächtnis halten, da ihr
zusammenkommt. \bibverse{25} Da sollt ihr keine Dienstarbeit tun und
sollt dem HErrn opfern. \bibverse{26} Und der HErr redete mit Mose und
sprach: \bibverse{27} Des zehnten Tages in diesem siebenten Monden ist
der Versöhnetag. Der soll bei euch heilig heißen, daß ihr zusammen
kommt; da sollt ihr euren Leib kasteien und dem HErrn opfern.
\bibverse{28} Und sollt keine Arbeit tun an diesem Tage; denn es, ist
der Versöhnetag, daß ihr versöhnet werdet vor dem HErrn, eurem GOtt.
\bibverse{29} Denn wer seinen Leib nicht kasteiet an diesem Tage, der
soll aus seinem Volk gerottet werden. \bibverse{30} Und wer dieses Tages
irgend eine Arbeit tut, den will ich vertilgen aus seinem Volk.
\bibverse{31} Darum sollt ihr keine Arbeit tun. Das soll ein ewiges
Recht sein euren Nachkommen in allen euren Wohnungen. \bibverse{32} Es
ist euer großer Sabbat, daß ihr eure Leiber kasteiet. Am neunten Tage
des Monden, zu Abend, sollt ihr diesen Sabbat halten, von Abend an bis
wieder zu Abend. \bibverse{33} Und der HErr redete mit Mose und sprach:
\bibverse{34} Rede mit den Kindern Israel und sprich: Am fünfzehnten
Tage dieses siebenten Monden ist das Fest der Laubhütten sieben Tage dem
HErrn. \bibverse{35} Der erste Tag soll heilig heißen, daß ihr
zusammenkommt; keine Dienstarbeit sollt ihr tun. \bibverse{36} Sieben
Tage sollt ihr dem HErrn opfern. Der achte Tag soll auch heilig heißen,
daß ihr zusammenkommt, und sollt euer Opfer dem HErrn tun; denn es ist
der Versammlungstag; keine Dienstarbeit sollt ihr tun. \bibverse{37} Das
sind die Feste des HErrn, die ihr sollt für heilig halten, daß ihr
zusammenkommt und dem HErrn Opfer tut, Brandopfer, Speisopfer,
Trankopfer und andere Opfer, ein jegliches nach seinem Tage,
\bibverse{38} ohne was der Sabbat des HErrn und eure Gaben und Gelübde
und freiwillige Gaben sind, die ihr dem HErrn gebet. \bibverse{39} So
sollt ihr nun am fünfzehnten Tage des siebenten Monden, wenn ihr das
Einkommen vom Lande eingebracht habt, das Fest des HErrn halten sieben
Tage lang. Am ersten Tage ist es Sabbat, und am achten Tage ist es auch
Sabbat. \bibverse{40} Und sollt am ersten Tage Früchte nehmen von
schönen Bäumen, Palmenzweige und Maien von dichten Bäumen und Bachweiden
und sieben Tage fröhlich sein vor dem HErrn, eurem GOtt. \bibverse{41}
Und sollt also dem HErrn des Jahrs das Fest halten sieben Tage. Das soll
ein ewiges Recht sein bei euren Nachkommen, daß sie im siebenten Monden
also feiern. \bibverse{42} Sieben Tage sollt ihr in Laubhütten wohnen;
wer einheimisch ist in Israel, der soll in Laubhütten wohnen,
\bibverse{43} daß eure Nachkommen wissen, wie ich die Kinder Israel habe
lassen in Hütten wohnen, da ich sie aus Ägyptenland führete. Ich bin der
HErr, euer GOtt. \bibverse{44} Und Mose sagte den Kindern Israel solche
Feste des HErrn.

\hypertarget{section-23}{%
\section{24}\label{section-23}}

\bibverse{1} Und der HErr redete mit Mose und sprach: \bibverse{2}
Gebeut den Kindern Israel, daß sie zu dir bringen gestoßen lauter Baumöl
zu Lichtern, das oben in die Lampen täglich getan werde, \bibverse{3}
außen vor dem Vorhang des Zeugnisses in der Hütte des Stifts. Und Aaron
soll's zurichten des Abends und des Morgens vor dem HErrn täglich. Das
sei ein ewiges Recht euren Nachkommen. \bibverse{4} Er soll aber die
Lampen auf dem feinen Leuchter zurichten vor dem HErrn täglich.
\bibverse{5} Und sollst Semmelmehl nehmen und davon zwölf Kuchen backen;
zwo Zehnten soll ein Kuchen haben. \bibverse{6} Und sollst sie legen je
sechs auf eine Schicht auf den feinen Tisch vor dem HErrn. \bibverse{7}
Und sollst auf dieselben legen reinen Weihrauch, daß es seien Denkbrote
zum Feuer dem HErrn. \bibverse{8} AlLev Sabbate für und für soll er sie
zurichten vor dem HErrn, von den Kindern Israel, zum ewigen Bunde.
\bibverse{9} Und sollen Aarons und seiner Söhne sein, die sollen sie
essen an heiliger Stätte; denn das ist sein Allerheiligstes von den
Opfern des HErrn zum ewigen Recht. \bibverse{10} Es ging aber aus eines
israelitischen Weibes Sohn, der eines ägyptischen Mannes Kind war, unter
den Kindern Israel und zankte sich im Lager mit einem israelitischen
Manne \bibverse{11} und lästerte den Namen und fluchte. Da brachten sie
ihn zu Mose (seine Mutter aber hieß Selomith, eine Tochter Dibris, vom
Stamm Dan) \bibverse{12} und legten ihn gefangen, bis ihnen klare
Antwort würde durch den Mund des HErrn. \bibverse{13} Und der HErr
redete mit Mose und sprach: \bibverse{14} Führe den Flucher hinaus vor
das Lager und laß alle, die es gehört haben, ihre Hände auf sein Haupt
legen und laß ihn die ganze Gemeine steinigen. \bibverse{15} Und sage
den Kindern Israel: Welcher seinem GOtt fluchet, der soll seine Sünde
tragen. \bibverse{16} Welcher des HErrn Namen lästert, der soll des
Todes sterben; die ganze Gemeine soll ihn steinigen. Wie der Fremdling,
so soll auch der Einheimische sein: wenn er den Namen lästert, so soll
er sterben. \bibverse{17} Wer irgend einen Menschen erschlägt, der soll
des Todes sterben. \bibverse{18} Wer aber ein Vieh erschlägt, der soll's
bezahlen, Leib um Leib. \bibverse{19} Und wer seinen Nächsten verletzet,
dem soll man tun wie er getan hat: \bibverse{20} Schade um Schade, Auge
um Auge, Zahn um Zahn; wie er hat einen Menschen verletzet, so soll man
ihm wieder tun. \bibverse{21} Also daß, wer ein Vieh erschlägt, der
soll's bezahlen; wer aber einen Menschen erschlägt, der soll sterben.
\bibverse{22} Es soll einerlei Recht unter euch sein, dem Fremdling wie
dem Einheimischen; denn ich bin der HErr, euer GOtt. \bibverse{23} Mose
aber sagte es den Kindern Israel; und führeten den Flucher aus vor das
Lager und steinigten ihn. Also taten die Kinder Israel, wie der HErr
Mose geboten hatte.

\hypertarget{section-24}{%
\section{25}\label{section-24}}

\bibverse{1} Und der HErr redete mit Mose auf dem Berge Sinai und
sprach: \bibverse{2} Rede mit den Kindern Israel und sprich zu ihnen:
Wenn ihr ins Land kommt, das ich euch geben werde, so soll das Land
seine Feier dem HErrn feiern, \bibverse{3} daß du sechs Jahre dein Feld
besäest und sechs Jahre deinen Weinberg beschneidest und sammelst die
Früchte ein. \bibverse{4} Aber im siebenten Jahr soll das Land seine
große Feier dem HErrn feiern, darin du dein Feld nicht besäen noch
deinen Weinberg beschneiden sollst. \bibverse{5} Was aber von ihm selber
nach deiner Ernte wächst, sollst du nicht ernten, und die Trauben, so
ohne deine Arbeit wachsen, sollst du nicht lesen, dieweil es ein
Feierjahr ist des Landes. \bibverse{6} Sondern die Feier des Landes
sollt ihr darum halten, daß du davon essest, dein Knecht, deine Magd,
dein Taglöhner, dein Hausgenoß, dein Fremdling bei dir, \bibverse{7}
dein Vieh und die Tiere in deinem Lande. AlLev Früchte sollen Speise
sein. \bibverse{8} Und du sollst zählen solcher Feierjahre sieben, daß
sieben Jahre siebenmal gezählet werden und die Zeit der sieben
Feierjahre mache neunundvierzig Jahre. \bibverse{9} Da sollst du die
Posaune lassen blasen durch all euer Land am zehnten Tage des siebenten
Monden, eben am Tage der Versöhnung. \bibverse{10} Und ihr sollt das
fünfzigste Jahr heiligen und sollt es ein Erlaßjahr heißen im Lande
allen, die drinnen wohnen; denn es ist euer Halljahr, da soll ein
jeglicher bei euch wieder zu seiner Habe und zu seinem Geschlecht
kommen. \bibverse{11} Denn das fünfzigste Jahr ist euer Halljahr; ihr
sollt nicht säen, auch, was von ihm selber wächst, nicht ernten, auch
was ohne Arbeit wächst im Weinberge, nicht lesen. \bibverse{12} Denn das
Halljahr soll unter euch heilig sein. Ihr sollt aber essen, was das Feld
trägt. \bibverse{13} Das ist das Halljahr, da jedermann wieder zu dem
Seinen kommen soll. \bibverse{14} Wenn du nun etwas deinem Nächsten
verkaufst oder ihm etwas abkaufst, soll keiner seinen Bruder
übervorteilen, \bibverse{15} sondern nach der Zahl vom Halljahr an
sollst du es von ihm kaufen; und was die Jahre hernach tragen mögen, so
hoch soll er dir's verkaufen. \bibverse{16} Nach der Menge der Jahre
sollst du den Kauf steigern und nach der Wenige der Jahre sollst du den
Kauf ringern; denn er soll dir's, nachdem es tragen mag, verkaufen.
\bibverse{17} So übervorteiLev nun keiner seinen Nächsten, sondern
fürchte dich vor deinem GOtt; denn ich bin der HErr, euer GOtt.
\bibverse{18} Darum tut nach meinen Satzungen und haltet meine Rechte,
daß ihr danach tut, auf daß ihr im Lande sicher wohnen möget.
\bibverse{19} Denn das Land soll euch seine Früchte geben, daß ihr zu
essen genug habet und sicher darinnen wohnet. \bibverse{20} Und ob du
würdest sagen: Was sollen wir essen im siebenten Jahr? denn wir säen
nicht, so sammeln wir auch kein Getreide ein: \bibverse{21} da will ich
meinem Segen über euch im sechsten Jahr gebieten, daß er soll dreier
Jahre Getreide machen, \bibverse{22} daß ihr säet im achten Jahr und von
dem alten Getreide esset bis in das neunte Jahr, daß ihr vom alten
esset, bis wieder neu Getreide kommt. \bibverse{23} Darum sollt ihr das
Land nicht verkaufen ewiglich; denn das Land ist mein, und ihr seid
Fremdlinge und Gäste vor mir. \bibverse{24} Und sollt in all eurem Lande
das Land zu lösen geben. \bibverse{25} Wenn dein Bruder verarmet und
verkauft dir seine Habe, und sein nächster Freund kommt zu ihm, daß er's
löse, so soll er's lösen, was sein Bruder verkauft hat. \bibverse{26}
Wenn aber jemand keinen Löser hat und kann mit seiner Hand so viel
zuwege bringen, daß er's ein Teil löse, \bibverse{27} so soll man
rechnen von dem Jahr, da er's hat verkauft, und dem Verkäufer die
übrigen Jahre wieder einräumen, daß er wieder zu seiner Habe komme.
\bibverse{28} Kann aber seine Hand nicht so viel finden, daß eines Teils
ihm wieder werde, so soll, daß er verkauft hat, in der Hand des Käufers
sein bis zum Halljahr; in demselben soll es ausgehen, und er wieder zu
seiner Habe kommen. \bibverse{29} Wer ein Wohnhaus verkauft inner der
Stadtmauer, der hat ein ganz Jahr Frist, dasselbe wieder zu lösen; das
soll die Zeit sein, darinnen er's lösen mag. \bibverse{30} Wo er's aber
nicht löset, ehe denn das ganze Jahr um ist, so soll's der Käufer
ewiglich behalten und seine Nachkommen, und soll nicht los ausgehen im
Halljahr. \bibverse{31} Ist's aber ein Haus auf dem Dorfe, da keine
Mauer um ist, das soll man dem Felde des Landes gleich rechnen und soll
los werden und im Halljahr ledig ausgehen. \bibverse{32} Die Städte der
Leviten und die Häuser in den Städten, da ihre Habe innen ist, mögen
immerdar gelöset werden. \bibverse{33} Wer etwas von den Leviten löset,
der soll's verlassen im Halljahr, es sei Haus oder Stadt, das er
besessen hat; denn die Häuser in Städten der Leviten sind ihre Habe
unter den Kindern Israel. \bibverse{34} Aber das Feld vor ihren Städten
soll man nicht verkaufen; denn das ist ihr Eigentum ewiglich.
\bibverse{35} Wenn dein Bruder verarmet und neben dir abnimmt, so sollst
du ihn aufnehmen als einen Fremdling oder Gast, daß er lebe neben dir.
\bibverse{36} Und sollst nicht Wucher von ihm nehmen noch Übersatz,
sondern sollst dich vor deinem GOtt fürchten, auf daß dein Bruder neben
dir leben könne. \bibverse{37} Denn du sollst ihm dein Geld nicht auf
Wucher tun, noch deine Speise auf Übersatz austun. \bibverse{38} Denn
ich bin der HErr, euer GOtt, der euch aus Ägyptenland geführet hat, daß
ich euch das Land Kanaan gäbe und euer GOtt wäre. \bibverse{39} Wenn
dein Bruder verarmet neben dir und verkauft sich dir, so sollst du ihn
nicht lassen dienen als einen Leibeigenen, \bibverse{40} sondern wie ein
Taglöhner und Gast soll er bei dir sein und bis an das Halljahr bei dir
dienen. \bibverse{41} Dann soll er von dir los ausgehen und seine Kinder
mit ihm; und soll wiederkommen zu seinem Geschlecht und zu seiner Väter
Habe. \bibverse{42} Denn sie sind meine Knechte, die ich aus Ägyptenland
geführet habe; darum soll man sie nicht auf leibeigene Weise verkaufen.
\bibverse{43} Und sollst nicht mit der Strenge über sie herrschen,
sondern dich fürchten vor deinem GOtt. \bibverse{44} Willst du aber
leibeigene Knechte und Mägde haben, so sollst du sie kaufen von den
Heiden, die um euch her sind, \bibverse{45} von den Gästen, die
Fremdlinge unter euch sind, und von ihren Nachkommen, die sie bei euch
in eurem Lande zeugen: dieselben sollt ihr zu eigen haben \bibverse{46}
und sollt sie besitzen, und eure Kinder nach euch, zum Eigentum für und
für; die sollt ihr leibeigene Knechte sein lassen. Aber über eure
Brüder, die Kinder Israel, soll keiner des andern herrschen mit der
Strenge. \bibverse{47} Wenn irgend ein Fremdling oder Gast bei dir
zunimmt, und dein Bruder neben ihm verarmet und sich dem Fremdling oder
Gast bei dir oder jemand von seinem Stamm verkauft, \bibverse{48} so
soll er nach seinem Verkaufen Recht haben, wieder los zu werden, und es
mag ihn jemand unter seinen Brüdern lösen, \bibverse{49} oder sein
Vetter oder Vetters Sohn, oder sonst sein nächster Blutsfreund seines
Geschlechts; oder so seine selbst Hand so viel erwirbt, so soll er sich
lösen. \bibverse{50} Und soll mit seinem Käufer rechnen vom Jahr an, da
er sich verkauft hatte, bis aufs Halljahr; und das Geld soll nach der
Zahl der Jahre seines Verkaufens gerechnet werden; und soll sein Taglohn
der ganzen Zeit mit einrechnen. \bibverse{51} Sind noch viel Jahre bis
an das Halljahr, so soll er nach denselben desto mehr zu lösen geben,
danach er gekauft ist. \bibverse{52} Sind aber wenig Jahre übrig bis an
das Halljahr, so soll er auch danach wieder geben zu seiner Lösung und
soll sein Taglohn von Jahr zu Jahr mit einrechnen. \bibverse{53} Und
sollst nicht lassen mit der Strenge über ihn herrschen vor deinen Augen.
\bibverse{54} Wird er aber auf diese Weise sich nicht lösen, so soll er
im Halljahr los ausgehen und seine Kinder mit ihm. \bibverse{55} Denn
die Kinder Israel sind meine Knechte, die ich aus Ägyptenland geführet
habe. Ich bin der HErr, euer GOtt.

\hypertarget{section-25}{%
\section{26}\label{section-25}}

\bibverse{1} Ihr sollt euch keinen Götzen machen noch Bild, und sollt
euch keine SäuLev aufrichten, noch keinen Malstein setzen in eurem
Lande, daß ihr davor anbetet; denn ich bin der HErr, euer GOtt.
\bibverse{2} Haltet meine Sabbate und fürchtet euch vor meinem
Heiligtum! Ich bin der HErr. \bibverse{3} Werdet ihr in meinen Satzungen
wandeln und meine Gebote halten und tun, \bibverse{4} so will, ich euch
Regen geben zu seiner Zeit, und das Land soll sein Gewächs geben und die
Bäume auf dem Felde ihre Früchte bringen. \bibverse{5} Und die
Dreschzeit soll reichen bis zur Weinernte, und die Weinernte soll
reichen bis zur Zeit der Saat. Und sollt Brots die FülLev haben und
sollt sicher in eurem Lande wohnen. \bibverse{6} Ich will Frieden geben
in eurem Lande, daß ihr schlafet, und euch niemand schrecke. Ich will
die bösen Tiere aus eurem Lande tun, und soll kein Schwert durch euer
Land gehen. \bibverse{7} Ihr sollt eure Feinde jagen, und sie sollen vor
euch her ins Schwert fallen. \bibverse{8} Euer fünf sollen hundert
jagen, und euer hundert sollen zehntausend jagen; denn eure Feinde
sollen vor euch her fallen ins Schwert. \bibverse{9} Und ich will mich
zu euch wenden und will euch wachsen und mehren lassen und will meinen
Bund euch halten. \bibverse{10} Und sollt von dem Firnen essen, und wenn
das Neue kommt, das Firne wegtun. \bibverse{11} Ich will meine Wohnung
unter euch haben, und meine SeeLev soll euch nicht verwerfen.
\bibverse{12} Und will unter euch wandeln und will euer GOtt sein; so
sollt ihr mein Volk sein. \bibverse{13} Denn ich bin der HErr, euer
GOtt, der euch aus Ägyptenland geführet hat, daß ihr nicht ihre Knechte
wäret, und habe euer Joch zerbrochen und habe euch aufgerichtet wandeln
lassen. \bibverse{14} Werdet ihr aber mir nicht gehorchen und nicht tun
diese Gebote alle, \bibverse{15} und werdet meine Satzungen verachten,
und eure SeeLev meine Rechte verwerfen, daß ihr nicht tut alLev meine
Gebote, und werdet meinen Bund lassen anstehen, \bibverse{16} so will
ich euch auch solches tun: Ich will euch heimsuchen mit Schrecken,
Schwulst und Fieber, daß euch die Angesichte verfallen und der Leib
verschmachte; ihr sollt umsonst euren Samen säen, und eure Feinde sollen
ihn fressen. \bibverse{17} und ich will mein Antlitz wider euch stellen,
und sollt geschlagen werden vor euren Feinden; und die euch hassen,
sollen über euch herrschen; und sollt fliehen, da euch niemand jaget.
\bibverse{18} So ihr aber über das noch nicht mir gehorchet, so will
ich's noch siebenmal mehr machen, euch zu strafen um eure Sünde,
\bibverse{19} daß ich euren Stolz und Halsstarrigkeit breche; und will
euren Himmel wie Eisen und eure Erde wie Erz machen. \bibverse{20} Und
eure Mühe und Arbeit soll verloren sein, daß euer Land sein Gewächs
nicht gebe und die Bäume im Lande ihre Früchte nicht bringen.
\bibverse{21} Und wo ihr mir entgegen wandelt und mich nicht hören
wollt, so will ich's noch siebenmal mehr machen, auf euch zu schlagen um
eurer Sünde willen. \bibverse{22} Und will wilde Tiere unter euch
senden, die sollen eure Kinder fressen und euer Vieh zerreißen und euer
weniger machen; und eure Straßen sollen wüste werden. \bibverse{23}
Werdet ihr euch aber damit noch nicht von mir züchtigen lassen und mir
entgegen wandeln, \bibverse{24} will ich euch auch entgegen wandeln und
will euch noch siebenmal mehr schlagen um eurer Sünde willen.
\bibverse{25} Und will ein Racheschwert über euch bringen, das meinen
Bund rächen soll. Und ob ihr euch in eure Städte versammelt, will ich
doch die Pestilenz unter euch senden und will euch in eurer Feinde Hände
geben. \bibverse{26} Dann will ich euch den Vorrat des Brots verderben,
daß zehn Weiber sollen euer Brot in einem Ofen backen, und euer Brot
soll man mit Gewicht auswägen, und wenn ihr esset, sollt ihr nicht satt
werden. \bibverse{27} Werdet ihr aber dadurch mir noch nicht gehorchen
und mir entgegen wandeln, \bibverse{28} so will ich auch euch im Grimm
entgegen wandeln und will euch siebenmal mehr strafen um eure Sünde,
\bibverse{29} daß ihr sollt eurer Söhne und Töchter Fleisch fressen.
\bibverse{30} Und will eure Höhen vertilgen und eure Bilder ausrotten;
und will eure Leichname auf eure Götzen werfen; und meine SeeLev wird an
euch Ekel haben. \bibverse{31} Und will eure Städte wüste machen und
eures Heiligtums Kirchen einreißen; und will euren süßen Geruch nicht
riechen. \bibverse{32} Also will ich das Land wüste machen, daß eure
Feinde, so drinnen wohnen, sich davor entsetzen werden. \bibverse{33}
Euch aber will ich unter die Heiden streuen und das Schwert ausziehen
hinter euch her, daß euer Land soll wüste sein und eure Städte
verstöret. \bibverse{34} Alsdann wird das Land ihm seine Feier gefallen
lassen, solange es wüste liegt, und ihr in der Feinde Land seid; ja,
dann wird das Land feiern und ihm seine Feier gefallen lassen,
\bibverse{35} solange es wüste liegt, darum daß es nicht feiern konnte,
da ihr's solltet feiern lassen, da ihr drinnen wohnetet. \bibverse{36}
Und denen; die von euch überbleiben, will ich ein feig Herz machen in
ihrer Feinde Land, daß sie soll ein rauschend Blatt jagen; und sollen
fliehen davor, als jagte sie ein Schwert, und fallen, da sie niemand
jaget. \bibverse{37} Und soll einer über den andern hinfallen, gleich
als vor dem Schwert, und doch sie niemand jaget; und ihr sollt euch
nicht auflehnen dürfen wider eure Feinde. \bibverse{38} Und ihr sollt
umkommen unter den Heiden, und eurer Feinde Land soll euch fressen.
\bibverse{39} Welche aber von euch überbleiben, die sollen in ihrer
Missetat verschmachten in der Feinde Land; auch in ihrer Väter Missetat
sollen sie verschmachten. \bibverse{40} Da werden sie denn bekennen ihre
Missetat und ihrer Väter Missetat, damit sie sich an mir versündiget und
mir entgegen gewandelt haben. \bibverse{41} Darum will ich auch ihnen
entgegen wandeln und will sie in ihrer Feinde Land wegtreiben. Da wird
sich ja ihr unbeschnittenes Herz demütigen, und dann werden sie ihnen
die Strafe ihrer Missetat gefallen lassen. \bibverse{42} Und ich werde
gedenken an meinen Bund mit Jakob und an meinen Bund mit Isaak und an
meinen Bund mit Abraham und werde an das Land gedenken, \bibverse{43}
das von ihnen verlassen ist und ihm seine Feier gefallen lässet, dieweil
es wüste von ihnen liegt, und sie ihnen die Strafe ihrer Missetat
gefallen lassen, darum daß sie meine Rechte verachtet, und ihre SeeLev
an meinen Satzungen Ekel gehabt hat. \bibverse{44} Auch wenn sie schon
in der Feinde Land sind, habe ich sie gleichwohl nicht verworfen, und
ekelt mich ihrer nicht also, daß es mit ihnen aus sein sollte, und mein
Bund mit ihnen sollte nicht mehr gelten; denn ich bin der HErr, ihr
GOtt. \bibverse{45} Und will über sie an meinen ersten Bund gedenken, da
ich sie aus Ägyptenland führete vor den Augen der Heiden, daß ich ihr
GOtt wäre, ich der HErr. \bibverse{46} Dies sind die Satzungen und
Rechte und Gesetze, die der HErr zwischen ihm und den Kindern Israel
gestellet hat auf dem Berge Sinai durch die Hand Moses.

\hypertarget{section-26}{%
\section{27}\label{section-26}}

\bibverse{1} Und der HErr redete mit Mose und sprach: \bibverse{2} Rede
mit den Kindern Israel und sprich zu ihnen: Wenn jemand dem HErrn ein
besonder Gelübde tut, daß er seinen Leib schätzet, \bibverse{3} so soll
das die Schätzung sein: Ein Mannsbild, zwanzig Jahre alt, bis ins
sechzigste Jahr, sollst du schätzen auf fünfzig silberne Sekel nach dem
Sekel des Heiligtums; \bibverse{4} ein Weibsbild auf dreißig Sekel.
\bibverse{5} Von fünf Jahren bis auf zwanzig Jahre sollst du ihn
schätzen auf zwanzig Sekel, wenn's ein Mannsbild ist; ein Weibsbild aber
auf zehn Sekel. \bibverse{6} Von einem Monden an bis auf fünf Jahre
sollst du ihn schätzen auf fünf silberne Sekel, wenn's ein Mannsbild
ist; ein Weibsbild aber auf drei silberne Sekel. \bibverse{7} Ist er
aber sechzig Jahre alt und drüber, so sollst du ihn schätzen auf
fünfzehn Sekel, wenn's ein Mannsbild ist; ein Weibsbild aber auf zehn
Sekel. \bibverse{8} Ist er aber zu arm zu solcher Schätzung, so soll er
sich vor den Priester stellen, und der Priester soll ihn schätzen; er
soll ihn aber schätzen, nachdem seine Hand, des, der gelobet hat,
erwerben kann. \bibverse{9} Ist's aber ein Vieh, das man dem HErrn
opfern kann: alles, was man des dem HErrn gibt, ist heilig.
\bibverse{10} Man soll's nicht wechseln noch wandeln, ein gutes um ein
böses, oder ein böses um ein gutes. Wird's aber jemand wechseln; ein
Vieh um das andere, so sollen sie beide dem HErrn heilig sein.
\bibverse{11} Ist aber das Tier unrein, daß man's dem HErrn nicht opfern
darf, so soll man's vor den Priester stellen. \bibverse{12} Und der
Priester soll es schätzen, ob's gut oder böse sei; und es soll bei des
Priesters Schätzen bleiben. \bibverse{13} Will's aber jemand lösen, der
soll den Fünften über die Schätzung geben. \bibverse{14} Wenn jemand
sein Haus heiliget, daß es dem HErrn heilig sei, das soll der Priester
schätzen, ob's gut oder böse sei; und danach es der Priester schätzet,
so soll's bleiben. \bibverse{15} So es aber der, so es geheiliget hat,
will lösen, so soll er den fünften Teil des Geldes, über das es
geschätzet ist, drauf geben, so soll's sein werden. \bibverse{16} Wenn
jemand ein Stück Ackers von seinem Erbgut dem HErrn heiliget, so soll er
geschätzet werden, nachdem er trägt. Trägt er ein Homor Gerste, so soll
er fünfzig Sekel Silbers gelten. \bibverse{17} Heiliget er aber seinen
Acker vom Halljahr an, so soll er nach seiner Würde gelten.
\bibverse{18} Hat er ihn aber nach dem Halljahr geheiliget, so soll ihn
der Priester rechnen nach den übrigen Jahren zum Halljahr und danach
geringer schätzen; \bibverse{19} Will aber der, so ihn geheiliget hat,
den Acker lösen, so soll er den fünften Teil des Geldes, über das er
geschätzet ist, drauf geben, so soll er sein werden. \bibverse{20} Will
er ihn aber nicht lösen, sondern verkauft ihn einem andern, so soll er
ihn nicht mehr lösen, \bibverse{21} sondern derselbe Acker, wenn er im
Halljahr los ausgehet, soll dem HErrn heilig sein, wie ein verbannter
Acker; und soll des Priesters Erbgut sein. \bibverse{22} Wenn aber
jemand einen Acker dem HErrn heiliget, den er gekauft hat, und nicht
sein Erbgut ist, \bibverse{23} so soll ihn der Priester rechnen, was er
gilt, bis an das Halljahr; und er soll desselben Tages solche Schätzung
geben, daß er dem HErrn heilig sei. \bibverse{24} Aber im Halljahr soll
er wieder gelangen an denselben, von dem er ihn gekauft hat, daß er sein
Erbgut im Lande sei. \bibverse{25} AlLev Würderung soll geschehen nach
dem Sekel des Heiligtums. Ein Sekel aber macht zwanzig Gera.
\bibverse{26} Die Erstgeburt unter dem Vieh, die dem HErrn sonst
gebühret, soll niemand dem HErrn heiligen, es sei ein Ochse oder Schaf;
denn es ist des HErrn. \bibverse{27} Ist aber an dem Vieh etwas
Unreines, so soll man's lösen nach seiner Würde und drüber geben den
Fünften. Will er's nicht lösen, so verkaufe man's nach seiner Würde.
\bibverse{28} Man soll kein Verbanntes verkaufen, noch lösen, das jemand
dem HErrn verbannet von allem, das sein ist, es seien Menschen Vieh oder
Erbacker; denn alles Verbannte ist das Allerheiligste dem HErrn.
\bibverse{29} Man soll auch keinen verbannten Menschen lösen, sondern er
soll des Todes sterben. \bibverse{30} AlLev Zehnten im Lande, beide von
Samen des Landes und von Früchten der Bäume, sind des HErrn und sollen
dem HErrn heilig sein. \bibverse{31} Will aber jemand seinen Zehnten
lösen, der soll den Fünften drüber geben. \bibverse{32} Und alLev
Zehnten von Rindern und Schafen, und was unter der Rute gehet, das ist
ein heiliger Zehnte dem HErrn. \bibverse{33} Man soll nicht fragen, ob's
gut oder böse sei; man soll's auch nicht wechseln. Wird es aber jemand
wechseln, so soll beides heilig sein und nicht gelöset werden.
\bibverse{34} Dies sind die Gebote, die der HErr gebot an die Kinder
Israel auf dem Berge Sinai.
