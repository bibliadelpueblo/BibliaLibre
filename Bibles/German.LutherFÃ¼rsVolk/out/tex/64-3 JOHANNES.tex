\hypertarget{section}{%
\section{1}\label{section}}

\bibverse{1} Der Älteste: Gajus, dem Lieben, den ich liebhabe in der
Wahrheit. \bibverse{2} Mein Lieber, ich wünsche in allen Stücken, daß
dir's wohl gehe und du gesund seist, wie es denn deiner Seele wohl geht.
\bibverse{3} Ich bin aber sehr erfreut worden, da die Brüder kamen und
zeugten von deiner Wahrheit, wie denn du wandelst in der Wahrheit.
\bibverse{4} Ich habe keine größere Freude denn die, daß ich höre, wie
meine Kinder in der Wahrheit wandeln. \bibverse{5} Mein Lieber, du tust
treulich, was du tust an den Brüdern und Gästen, \bibverse{6} die von
deiner Liebe gezeugt haben vor der Gemeinde; und du wirst wohl tun, wenn
du sie abfertigst würdig vor Gott. \bibverse{7} Denn um seines Namens
willen sind sie ausgezogen und nehmen von den Heiden nichts.
\bibverse{8} So sollen wir nun solche aufnehmen, auf daß wir der
Wahrheit Gehilfen werden. \bibverse{9} Ich habe der Gemeinde
geschrieben, aber Diotrephes, der unter ihnen hochgehalten sein will,
nimmt uns nicht an. \bibverse{10} Darum, wenn ich komme, will ich ihn
erinnern seiner Werke, die er tut; denn er plaudert mit bösen Worten
wider uns und läßt sich an dem nicht genügen; er selbst nimmt die Brüder
nicht an und wehrt denen, die es tun wollen, und stößt sie aus der
Gemeinde. \bibverse{11} Mein Lieber, folge nicht nach dem Bösen, sondern
dem Guten. Wer Gutes tut, der ist von Gott; wer Böses tut, der sieht
Gott nicht. \bibverse{12} Demetrius hat Zeugnis von jedermann und von
der Wahrheit selbst; und wir zeugen auch, und ihr wisset, das unser
Zeugnis wahr ist. \bibverse{13} Ich hatte viel zu schreiben; aber ich
will nicht mit der Tinte und der Feder an dich schreiben. \bibverse{14}
Ich hoffe aber, dich bald zu sehen; so wollen wir mündlich miteinander
reden. Friede sei mit dir! Es grüßen dich die Freunde. Grüße die Freunde
bei Namen.
