\hypertarget{section}{%
\section{1}\label{section}}

\bibverse{1} Im dritten Jahr des Reichs Jojakims, des Königs Judas, kam
Nebukadnezar, der König zu Babel, vor Jerusalem und belagerte sie.
\bibverse{2} Und der HErr übergab ihm Jojakim, den König Judas, und
etliche Gefäße aus dem Hause GOttes; die ließ er führen ins Land Sinear,
in seines Gottes Haus, und tat die Gefäße in seines Gottes Schatzkasten.
\bibverse{3} Und der König sprach zu Aspenas, seinem obersten Kämmerer,
er sollte aus den Kindern Israel vom königlichen Stamm und Herrenkindern
wählen \bibverse{4} Knaben, die nicht gebrechlich wären, sondern schöne,
vernünftige, weise, kluge und verständige, die da geschickt wären, zu
dienen in des Königs Hofe und zu lernen chaldäische Schrift und Sprache.
\bibverse{5} Solchen verschaffte der König, was man ihnen täglich geben
sollte von seiner Speise und von dem Wein, den er selbst trank, daß sie,
also drei Jahre auferzogen, danach vor dem Könige dienen sollten.
\bibverse{6} Unter welchen waren Daniel, Hananja, Misael und Asarja von
den Kindern Judas. \bibverse{7} Und der oberste Kämmerer gab ihnen Namen
und nannte Daniel Beltsazar und Hananja Sadrach und Misael Mesach und
Asarja Abed-Nego. \bibverse{8} Aber Daniel setzte ihm vor in seinem
Herzen, daß er sich mit des Königs Speise und mit dem Wein, den er
selbst trank, nicht verunreinigen wollte, und bat den obersten Kämmerer,
daß er sich nicht müßte verunreinigen. \bibverse{9} Und GOtt gab Daniel,
daß ihm der oberste Kämmerer günstig und gnädig ward. \bibverse{10}
Derselbe sprach zu ihm: Ich fürchte mich vor meinem Herrn, dem Könige,
der euch eure Speise und Trank verschaffet hat; wo er würde sehen, daß
eure Angesichte jämmerlicher wären denn der andern Knaben eures Alters,
so brächtet ihr mich bei dem Könige um mein Leben. \bibverse{11} Da
sprach Daniel zu Melzar, welchem der oberste Kämmerer Daniel, Hananja,
Misael und Asarja befohlen hatte: \bibverse{12} Versuch es doch mit
deinen Knechten zehn Tage und laß uns geben Gemüse zu essen und Wasser
zu trinken! \bibverse{13} Und laß dann vor dir unsere Gestalt und der
Knaben, so von des Königs Speise essen, besehen; und danach du sehen
wirst, danach schaffe mit deinen Knechten. \bibverse{14} Und er
gehorchte ihnen darin und versuchte es mit ihnen zehn Tage.
\bibverse{15} Und nach den zehn Tagen waren sie schöner und baß bei
Leibe denn alle Knaben, so von des Königs Speise aßen. \bibverse{16} Da
tat Melzar ihre verordnete Speise und Trank weg und gab ihnen Gemüse.
\bibverse{17} Aber der GOtt dieser vier gab ihnen Kunst und Verstand in
allerlei Schrift und Weisheit; Daniel aber gab er Verstand in allen
Gesichten und Träumen. \bibverse{18} Und da die Zeit um war, die der
König bestimmt hatte, daß sie sollten hineingebracht werden, brachte sie
der oberste Kämmerer hinein vor Nebukadnezar. \bibverse{19} Und der
König redete mit ihnen, und ward unter allen niemand erfunden, der
Daniel, Hananja, Misael und Asarja gleich wäre. Und sie wurden des
Königs Diener. \bibverse{20} Und der König fand sie in allen Sachen, die
er sie fragte, zehnmal klüger und verständiger denn alle Sternseher und
Weisen in seinem ganzen Reich. \bibverse{21} Und Daniel lebte bis ins
erste Jahr des Königs Kores.

\hypertarget{section-1}{%
\section{2}\label{section-1}}

\bibverse{1} Im andern Jahr des Reichs Nebukadnezars hatte Nebukadnezar
einen Traum, davon er erschrak, daß er aufwachte. \bibverse{2} Und er
hieß alle Sternseher, und Weisen und Zauberer und Chaldäer
zusammenfordern, daß sie dem Könige seinen Traum sagen sollten. Und sie
kamen und traten vor den König. \bibverse{3} Und der König sprach zu
ihnen: Ich habe einen Traum gehabt, der hat mich erschreckt; und ich
wollte gerne wissen, was es für ein Traum gewesen sei. \bibverse{4} Da
sprachen die Chaldäer zum Könige auf chaldäisch: Herr König, GOtt
verleihe dir langes Leben! Sage deinen Knechten den Traum, so wollen wir
ihn deuten. \bibverse{5} Der König antwortete und sprach zu den
Chaldäern: Es ist mir entfallen. Werdet ihr mir den Traum nicht anzeigen
und ihn deuten, so werdet ihr gar umkommen und eure Häuser schändlich
verstöret werden. \bibverse{6} Werdet ihr mir aber den Traum anzeigen
und deuten, so sollt ihr Geschenke, Gaben und große Ehre von mir haben.
Darum so sagt mir den Traum und seine Deutung! \bibverse{7} Sie
antworteten wiederum und sprachen: Der König sage seinen Knechten den
Traum, so wollen wir ihn deuten. \bibverse{8} Der König antwortete und
sprach: Wahrlich, ich merke es, daß ihr Frist suchet, weil ihr sehet,
daß mir's entfallen ist. \bibverse{9} Aber werdet ihr mir nicht den
Traum sagen, so gehet das Recht über euch, als die ihr Lügen und
Gedichte vor mir zu reden vorgenommen habt, bis die Zeit vorübergehe.
Darum so sagt mir den Traum, so kann ich merken, daß ihr auch die
Deutung treffet. \bibverse{10} Da antworteten die Chaldäer vor dem
Könige und sprachen zu ihm: Es ist kein Mensch auf Erden, der sagen
könne, das der König fordert. So ist auch kein König, wie groß oder
mächtig er sei, der solches von irgendeinem Sternseher, Weisen oder
Chaldäer fordere. \bibverse{11} Denn das der König fordert, ist zu hoch,
und ist auch sonst niemand, der es vor dem Könige sagen könne,
ausgenommen die Götter, die bei den Menschen nicht wohnen. \bibverse{12}
Da ward der König sehr zornig und befahl, alle Weisen zu Babel
umzubringen. \bibverse{13} Und das Urteil ging aus, daß man die Weisen
töten sollte. Und Daniel samt seinen Gesellen ward auch gesucht, daß man
sie tötete. \bibverse{14} Da vernahm Daniel solch Urteil und Befehl von
dem obersten Richter des Königs, welcher auszog, zu töten die Weisen zu
Babel. \bibverse{15} Und er fing an und sprach zu des Königs Vogt
Arioch: Warum ist so ein streng Urteil vom Könige ausgegangen? Und
Arioch zeigte es dem Daniel an. \bibverse{16} Da ging Daniel hinauf und
bat den König, daß er ihm Frist gäbe, damit er die Deutung dem Könige
sagen möchte. \bibverse{17} Und Daniel ging heim und zeigte solches an
seinen Gesellen, Hananja, Misael und Asarja, \bibverse{18} daß sie GOtt
vom Himmel um Gnade bäten solches verborgenen Dings halben, damit Daniel
und seine Gesellen nicht samt den andern Weisen zu Babel umkämen.
\bibverse{19} Da ward Daniel solch verborgen Ding durch ein Gesicht des
Nachts offenbaret. \bibverse{20} Darüber lobte Daniel den GOtt vom
Himmel, fing an und sprach: Gelobet sei der Name GOttes von Ewigkeit zu
Ewigkeit; denn sein ist beides, Weisheit und Stärke! \bibverse{21} Er
ändert Zeit und Stunde; er setzt Könige ab und setzt Könige ein; er gibt
den Weisen ihre Weisheit und den Verständigen ihren Verstand;
\bibverse{22} er offenbaret, was tief und verborgen ist; er weiß, was in
Finsternis liegt; denn bei ihm ist eitel Licht. \bibverse{23} Ich danke
dir und lobe dich, GOtt meiner Väter, daß du mir Weisheit und Stärke
verleihest und jetzt offenbaret hast, darum wir dich gebeten haben;
nämlich du hast uns des Königs Sache offenbaret. \bibverse{24} Da ging
Daniel hinauf zu Arioch, der vom Könige Befehl hatte, die Weisen zu
Babel umzubringen, und sprach zu ihm also: Du sollst die Weisen zu Babel
nicht umbringen, sondern führe mich hinauf zum Könige, ich will dem
Könige die Deutung sagen. \bibverse{25} Arioch brachte Daniel eilends
hinauf vor den König und sprach zu ihm also: Es ist einer funden unter
den Gefangenen aus Juda, der dem Könige die Deutung sagen kann.
\bibverse{26} Der König antwortete und sprach zu Daniel, den sie
Beltsazar hießen: Bist du, der mir den Traum, den ich gesehen habe und
seine Deutung zeigen kann? \bibverse{27} Daniel fing an vor dem Könige
und sprach: Das verborgene Ding, das der König fordert von den Weisen,
Gelehrten, Sternsehern und Wahrsagern, stehet in ihrem Vermögen nicht,
dem Könige zu sagen, \bibverse{28} sondern GOtt vom Himmel, der kann
verborgene Dinge offenbaren; der hat dem Könige Nebukadnezar angezeiget,
was in künftigen Zeiten geschehen soll. \bibverse{29} Dein Traum und
dein Gesicht, da du schliefest, kam daher: Du, König, dachtest auf
deinem Bette, wie es doch hernach gehen würde; und der, so verborgene
Dinge offenbaret, hat dir angezeiget, wie es gehen werde. \bibverse{30}
So ist mir solch verborgen Ding offenbaret, nicht durch meine Weisheit,
als wäre sie größer denn aller, die da leben, sondern darum, daß dem
Könige die Deutung angezeiget würde, und du deines Herzens Gedanken
erführest. \bibverse{31} Du, König, sahst, und siehe, ein sehr groß und
hoch Bild stund vor dir, das war schrecklich anzusehen. \bibverse{32}
Desselben Bildes Haupt war von feinem Golde; seine Brust und Arme waren
von Silber; sein Bauch und Lenden waren von Erz; \bibverse{33} seine
Schenkel waren Eisen; seine Füße waren eines Teils Eisen und eines Teils
Ton. \bibverse{34} Solches sahst du, bis daß ein Stein herabgerissen
ward ohne Hände; der schlug das Bild an seine Füße, die Eisen und Ton
waren, und zermalmete sie. \bibverse{35} Da wurden miteinander zermalmet
das Eisen, Ton, Erz, Silber und Gold und wurden wie Spreu auf der
Sommertenne; und der Wind verwebte sie, daß man sie nirgends mehr finden
konnte. Der Stein aber, der das Bild schlug, ward ein großer Berg, daß
er die ganze Welt füllete. \bibverse{36} Das ist der Traum. Nun wollen
wir die Deutung vor dem Könige sagen. \bibverse{37} Du, König, bist ein
König aller Könige, dem GOtt vom Himmel Königreich, Macht, Stärke und
Ehre gegeben hat \bibverse{38} und alles da Leute wohnen, dazu die Tiere
auf dem Felde und die Vögel unter dem Himmel in deine Hände gegeben und
dir über alles Gewalt verliehen hat. Du bist das güldene Haupt.
\bibverse{39} Nach dir wird ein ander Königreich aufkommen, geringer
denn deines. Danach das dritte Königreich, das ehern ist, welches wird
über alle Lande herrschen. \bibverse{40} Das vierte wird hart sein wie
Eisen. Denn gleichwie Eisen alles zermalmet und zerschlägt, ja, wie
Eisen alles zerbricht, also wird es auch alles zermalmen und zerbrechen.
\bibverse{41} Daß du aber gesehen hast die Füße und Zehen eines Teils
Ton und eines Teils Eisen, das wird ein zerteilt Königreich sein; doch
wird von des Eisens Pflanze drinnen bleiben, wie du denn gesehen hast
Eisen mit Ton vermenget. \bibverse{42} Und daß die Zehen an seinen Füßen
eines Teils Eisen und eines Teils Ton sind, wird es zum Teil ein stark
und zum Teil ein schwach Reich sein. \bibverse{43} Und daß du gesehen
hast Eisen mit Ton vermenget, werden sie sich wohl nach Menschengeblüt
untereinander mengen, aber sie werden doch nicht aneinander halten,
gleichwie sich Eisen mit Ton nicht mengen läßt. \bibverse{44} Aber zur
Zeit solcher Königreiche wird GOtt vom Himmel ein Königreich aufrichten,
das nimmermehr zerstöret wird; und sein Königreich wird auf kein ander
Volk kommen. Es wird alle diese Königreiche zermalmen und verstören,
aber es wird ewiglich bleiben. \bibverse{45} Wie du denn gesehen hast,
einen Stein ohne Hände vom Berge herabgerissen, der das Eisen, Erz, Ton,
Silber und Gold zermalmet. Also hat der große GOtt dem Könige gezeiget,
wie es hernach gehen werde; und das ist gewiß der Traum, und die Deutung
ist recht. \bibverse{46} Da fiel der König Nebukadnezar auf sein
Angesicht und betete an vor dem Daniel und befahl, man sollte ihm
Speisopfer und Räuchopfer tun. \bibverse{47} Und der König antwortete
Daniel und sprach: Es ist kein Zweifel, euer GOtt ist ein GOtt über alle
Götter und ein HErr über alle Könige, der da kann verborgene Dinge
offenbaren, weil du dies verborgene Ding hast können offenbaren.
\bibverse{48} Und der König erhöhete Daniel und gab ihm große und viele
Geschenke und machte ihn zum Fürsten über das ganze Land zu Babel und
setzte ihn zum Obersten über alle Weisen zu Babel. \bibverse{49} Und
Daniel bat vom Könige, daß er über die Landschaften zu Babel setzen
möchte Sadrach, Mesach, Abed-Nego; und er, Daniel, blieb bei dem Könige
zu Hofe.

\hypertarget{section-2}{%
\section{3}\label{section-2}}

\bibverse{1} Der König Nebukadnezar ließ ein gülden Bild machen, sechzig
Ellen hoch und sechs Ellen breit, und ließ es setzen im Lande zu Babel
im Tal Dura. \bibverse{2} Und der König Nebukadnezar sandte nach den
Fürsten, Herren, Landpflegern, Richtern, Vögten, Räten, Amtleuten und
allen Gewaltigen im Lande, daß sie zusammenkommen sollten, das Bild zu
weihen, das der König Nebukadnezar hatte setzen lassen. \bibverse{3} Da
kamen zusammen die Fürsten, Herren, Landpfleger, Richter, Vögte, Räte,
Amtleute und alle Gewaltigen im Lande, das Bild zu weihen, das der König
Nebukadnezar hatte setzen lassen. Und sie mußten vor das Bild treten,
das Nebukadnezar hatte setzen lassen. \bibverse{4} Und der Ehrenhold
rief überlaut: Das laßt euch gesagt sein, ihr Völker, Leute und Zungen:
\bibverse{5} Wenn ihr hören werdet den Schall der Posaunen, Trommeten,
Harfen, Geigen, Psalter, Lauten und allerlei Saitenspiel, so sollt ihr
niederfallen und das güldene Bild anbeten, das der König Nebukadnezar
hat setzen lassen. \bibverse{6} Wer aber alsdann nicht niederfällt und
anbetet, der soll von Stund an in den glühenden Ofen geworfen werden.
\bibverse{7} Da sie nun höreten den Schall der Posaunen, Trommeten,
Harfen, Geigen, Psalter und allerlei Saitenspiel, fielen nieder alle
Völker, Leute und Zungen und beteten an das güldene Bild, das der König
Nebukadnezar hatte setzen lassen. \bibverse{8} Von Stund an traten hinzu
etliche chaldäische Männer und verklagten die Juden, \bibverse{9} fingen
an und sprachen zum Könige Nebukadnezar: Herr König, GOtt verleihe dir
langes Leben! \bibverse{10} Du hast ein Gebot lassen ausgehen, daß alle
Menschen, wenn sie hören würden den Schall der Posaunen, Trommeten,
Harfen, Geigen, Psalter, Lauten und allerlei Saitenspiel, sollten sie
niederfallen und das güldene Bild anbeten; \bibverse{11} wer aber nicht
niederfiele und anbetete, sollte in einen glühenden Ofen geworfen
werden. \bibverse{12} Nun sind da jüdische Männer, welche du über die
Ämter im Lande zu Babel gesetzet hast: Sadrach, Mesach und Abed-Nego;
dieselbigen verachten dein Gebot und ehren deine Götter nicht und beten
nicht an das güldene Bild, das du hast setzen lassen. \bibverse{13} Da
befahl Nebukadnezar mit Grimm und Zorn, daß man vor ihn stellete
Sadrach, Mesach und Abed-Nego. Und die Männer wurden vor den König
gestellet. \bibverse{14} Da fing Nebukadnezar an und sprach zu ihnen:
Wie? wollt ihr, Sadrach, Mesach, Abed-Nego, meinen Gott nicht ehren und
das güldene Bild nicht anbeten, das ich habe setzen lassen?
\bibverse{15} Wohlan, schicket euch! Sobald ihr hören werdet den Schall
der Posaunen, Trommeten, Harfen, Geigen, Psalter, Lauten und allerlei
Saitenspiel, so fallet nieder und betet das Bild an, das ich habe machen
lassen! Werdet ihr's nicht anbeten, so sollt ihr von Stund an in den
glühenden Ofen geworfen werden. Laßt sehen, wer der GOtt sei, der euch
aus meiner Hand erretten werde! \bibverse{16} Da fingen an Sadrach,
Mesach; Abed-Nego und sprachen zum Könige Nebukadnezar: Es ist nicht
not, daß wir dir darauf antworten. \bibverse{17} Siehe, unser GOtt, den
wir ehren, kann uns wohl erretten aus dem glühenden Ofen, dazu auch von
deiner Hand erretten. \bibverse{18} Und wo er's nicht tun will, so
sollst du dennoch wissen, daß wir deine Götter nicht ehren, noch das
güldene Bild, das du hast setzen lassen, anbeten wollen. \bibverse{19}
Da ward Nebukadnezar voll Grimms und stellete sich scheußlich wider
Sadrach, Mesach und Abed-Nego und befahl, man sollte den Ofen siebenmal
heißer machen, denn man sonst zu tun pflegte. \bibverse{20} Und befahl
den besten Kriegsleuten, die in seinem Heer waren, daß sie Sadrach,
Mesach und Abed-Nego bänden und in den glühenden Ofen würfen.
\bibverse{21} Also wurden diese Männer in ihren Mänteln, Schuhen, Hüten
und andern Kleidern gebunden und in den glühenden Ofen geworfen.
\bibverse{22} Denn des Königs Gebot mußte man eilend tun. Und man
schürete das Feuer im Ofen so sehr, daß die Männer, so den Sadrach,
Mesach und Abed-Nego verbrennen sollten, verdarben von des Feuers
Flammen. \bibverse{23} Aber die drei Männer Sadrach, Mesach und
Abed-Nego, fielen hinab in den glühenden Ofen, wie sie gebunden waren.
\bibverse{24} Da entsetzte sich der König Nebukadnezar und fuhr eilends
auf und sprach zu seinen Räten: Haben wir nicht drei Männer gebunden in
das Feuer lassen werfen? Sie antworteten und sprachen zum Könige: Ja,
Herr König! \bibverse{25} Er antwortete und sprach: Sehe ich doch vier
Männer los im Feuer gehen, und sind unversehrt; und der vierte ist
gleich, als wäre er ein Sohn der Götter. \bibverse{26} Und Nebukadnezar
trat hinzu vor das Loch des glühenden Ofens und sprach: Sadrach, Mesach,
Abed-Nego, ihr Knechte GOttes des Höchsten, gehet heraus und kommt her!
Da gingen Sadrach, Mesach und Abed-Nego heraus aus dem Feuer.
\bibverse{27} Und die Fürsten, Herren, Vögte und Räte des Königs kamen
zusammen und sahen, daß das Feuer keine Macht am Leibe dieser Männer
beweiset hatte, und ihr Haupthaar nicht versenget und ihre Mäntel nicht
versehrt waren; ja, man konnte keinen Brand an ihnen riechen.
\bibverse{28} Da fing an Nebukadnezar und sprach: Gelobet sei der GOtt
Sadrachs, Mesachs und Abed-Negos, der seinen Engel gesandt und seine
Knechte errettet hat, die ihm vertrauet und des Königs Gebot nicht
gehalten, sondern ihren Leib dargegeben haben, daß sie keinen Gott ehren
noch anbeten wollten ohne allein ihren GOtt. \bibverse{29} So sei nun
dies mein Gebot: Welcher unter allen Völkern, Leuten und Zungen den GOtt
Sadrachs, Mesachs und Abed-Negos lästert, der soll umkommen, und sein
Haus schändlich verstöret werden. Denn es ist kein anderer GOtt, der
also erretten kann als dieser. \bibverse{30} Und der König gab Sadrach,
Mesach und Abed-Nego große Gewalt im Lande zu Babel.

\hypertarget{section-3}{%
\section{4}\label{section-3}}

\bibverse{1} König Nebukadnezar allen Völkern, Leuten und Zungen: GOtt
gebe euch viel Friede! \bibverse{2} Ich sehe es für gut an, daß ich
verkündige die Zeichen und Wunder, so GOtt der Höchste an mir getan hat.
\bibverse{3} Denn seine Zeichen sind groß, und seine Wunder sind
mächtig; und sein Reich ist ein ewiges Reich, und seine Herrschaft
währet für und für. \bibverse{4} Ich, Nebukadnezar, da ich gute Ruhe
hatte in meinem Hause, und es wohl stund auf meiner Burg, \bibverse{5}
sah ich einen Traum und erschrak, und die Gedanken, die ich auf meinem
Bette hatte über dem Gesichte, so ich gesehen hatte, betrübten mich.
\bibverse{6} Und ich befahl, daß alle Weisen zu Babel vor mich
heraufgebracht würden, daß sie mir sageten, was der Traum bedeutete.
\bibverse{7} Da brachte man herauf die Sternseher, Weisen, Chaldäer und
Wahrsager, und ich erzählte den Traum vor ihnen; aber sie konnten mir
nicht sagen, was er bedeutete, \bibverse{8} bis zuletzt Daniel vor mich
kam, welcher Beltsazar heißt, nach dem Namen meines Gottes, der den
Geist der heiligen Götter hat. Und ich erzählte vor ihm den Traum:
\bibverse{9} Beltsazar, du Oberster unter den Sternsehern, welchen ich
weiß, daß du den Geist der heiligen Götter hast und dir nichts verborgen
ist, sage das Gesicht meines Traums, den ich gesehen habe, und was er
bedeutet. \bibverse{10} Dies ist aber das Gesicht, das ich gesehen habe
auf meinem Bette: Siehe, es stund ein Baum mitten im Lande, der war sehr
hoch, \bibverse{11} groß und dick; seine Höhe reichte bis in Himmel und
breitete sich aus bis ans Ende des ganzen Landes. \bibverse{12} Seine
Äste waren schön und trugen viel Früchte, davon alles zu essen hatte.
Alle Tiere auf dem Felde fanden Schatten unter ihm, und die Vögel unter
dem Himmel saßen auf seinen Ästen, und alles Fleisch nährete sich von
ihm. \bibverse{13} Und ich sah ein Gesicht auf meinem Bette, und siehe,
ein heiliger Wächter fuhr vom Himmel herab, \bibverse{14} der rief
überlaut und sprach also: Hauet den Baum um und behauet ihm die Äste und
streifet ihm das Laub ab und zerstreuet seine Früchte, daß die Tiere, so
unter ihm liegen, weglaufen, und die Vögel von seinen Zweigen fliehen.
\bibverse{15} Doch laß den Stock mit seinen Wurzeln in der Erde bleiben;
er aber soll in eisernen und ehernen Ketten auf dem Felde im Grase
gehen; er soll unter dem Tau des Himmels liegen und naß werden und soll
sich weiden mit den Tieren von den Kräutern der Erde. \bibverse{16} Und
das menschliche Herz soll von ihm genommen und ein viehisch Herz ihm
gegeben werden, bis daß sieben Zeiten über ihm um sind. \bibverse{17}
Solches ist im Rat der Wächter beschlossen und im Gespräch der Heiligen
beratschlaget, auf daß die Lebendigen erkennen, daß der Höchste Gewalt
hat über der Menschen Königreiche und gibt sie, wem er will, und erhöhet
die Niedrigen zu denselbigen. \bibverse{18} Solchen Traum habe ich,
König Nebukadnezar, gesehen. Du aber, Beltsazar, sage, was er bedeute;
denn alle Weisen in meinem Königreich können mir nicht anzeigen, was er
bedeute; du aber kannst es wohl, denn der Geist der heiligen Götter ist
bei dir. \bibverse{19} Da entsetzte sich Daniel, der sonst Beltsazar
heißt, bei einer Stunde lang, und seine Gedanken betrübten ihn. Aber der
König sprach: Beltsazar, laß dich den Traum und seine Deutung nicht
betrüben! Beltsazar fing an und sprach: Ach, mein Herr, daß der Traum
deinen Feinden und seine Deutung deinen Widerwärtigen gälte!
\bibverse{20} Der Baum, den du gesehen hast, daß er groß und dick war
und seine Höhe an den Himmel reichte und breitete sich über das ganze
Land, \bibverse{21} und seine Äste schön und seiner Früchte viel, davon
alles zu essen hatte, und die Tiere auf dem Felde unter ihm wohneten,
und die Vögel des Himmels auf seinen Ästen saßen: \bibverse{22} das bist
du, König der du so groß und mächtig bist; denn deine Macht ist groß und
reichet an den Himmel, und deine Gewalt langet bis an der Welt Ende.
\bibverse{23} Daß aber der König einen heiligen Wächter gesehen hat vom
Himmel herabfahren und sagen: Hauet den Baum um und verderbet ihn, doch
den Stock mit seinen Wurzeln laßt in der Erde bleiben; er aber soll in
eisernen und ehernen Ketten auf dem Felde im Grase gehen und unter dem
Tau des Himmels liegen und naß werden und sich mit den Tieren auf dem
Felde weiden, bis über ihm sieben Zeiten um sind: \bibverse{24} das ist
die Deutung, Herr König, und solcher Rat des Höchsten gehet über meinen
Herrn König. \bibverse{25} Man wird dich von den Leuten verstoßen, und
mußt bei den Tieren auf dem Felde bleiben; und man wird dich Gras essen
lassen wie die Ochsen; und wirst unter dem Tau des Himmels liegen und
naß werden, bis über dir sieben Zeiten um sind, auf daß du erkennest,
daß der Höchste Gewalt hat über der Menschen Königreiche und gibt sie,
wem er will. \bibverse{26} Daß aber gesagt ist, man solle dennoch den
Stock mit seinen Wurzeln des Baums bleiben lassen: dein Königreich soll
dir bleiben, wenn du erkannt hast die Gewalt im Himmel. \bibverse{27}
Darum, Herr König, laß dir meinen Rat gefallen und mache dich los von
deinen Sünden durch Gerechtigkeit und ledig von deiner Missetat durch
Wohltat an den Armen, so wird er Geduld haben mit deinen Sünden.
\bibverse{28} Dies alles widerfuhr dem Könige Nebukadnezar.
\bibverse{29} Denn nach zwölf Monden, da der König auf der königlichen
Burg zu Babel ging, \bibverse{30} hub er an und sprach: Das ist die
große Babel, die ich erbauet habe zum königlichen Hause durch meine
große Macht, zu Ehren meiner Herrlichkeit. \bibverse{31} Ehe der König
diese Worte ausgeredet hatte, fiel eine Stimme vom Himmel: Dir, König
Nebukadnezar, wird gesagt: Dein Königreich soll dir genommen werden,
\bibverse{32} und man wird dich von den Leuten verstoßen, und sollst bei
den Tieren, so auf dem Felde gehen, bleiben; Gras wird man dich essen
lassen, wie Ochsen, bis daß über dir sieben Zeiten um sind, auf daß du
erkennest, daß der Höchste Gewalt hat über der Menschen Königreiche und
gibt sie, wem er will. \bibverse{33} Von Stund an ward das Wort
vollbracht über Nebukadnezar, und er ward von den Leuten verstoßen und
er aß Gras wie Ochsen, und sein Leib lag unter dem Tau des Himmels und
ward naß, bis sein Haar wuchs, so groß als Adlersfedern, und seine Nägel
wie Vogelklauen wurden. \bibverse{34} Nach dieser Zeit hub ich,
Nebukadnezar, meine Augen auf gen Himmel und kam wieder zur Vernunft und
lobte den Höchsten. Ich preisete und ehrete den, so ewiglich lebet, des
Gewalt ewig ist und sein Reich für und für währet, \bibverse{35} gegen
welchen alle, so auf Erden wohnen, als nichts zu rechnen sind. Er macht
es, wie er will, beide, mit den Kräften im Himmel und mit denen, so auf
Erden wohnen; und niemand kann seiner Hand wehren noch zu ihm sagen: Was
machst du? \bibverse{36} Zur selbigen Zeit kam ich wieder zur Vernunft,
auch zu meinen königlichen Ehren, zu meiner Herrlichkeit und zu meiner
Gestalt. Und meine Räte und Gewaltigen suchten mich; und ward wieder in
mein Königreich gesetzt; und ich überkam noch größere Herrlichkeit.
\bibverse{37} Darum lobe ich, Nebukadnezar, und ehre und preise den
König vom Himmel. Denn all sein Tun ist Wahrheit, und seine Wege sind
recht; und wer stolz ist, den kann er demütigen.

\hypertarget{section-4}{%
\section{5}\label{section-4}}

\bibverse{1} König Belsazer machte ein herrlich Mahl tausend seinen
Gewaltigen und Hauptleuten und soff sich voll mit ihnen. \bibverse{2}
Und da er trunken war, hieß er die güldenen und silbernen Gefäße
herbringen, die sein Vater Nebukadnezar aus dem Tempel zu Jerusalem
weggenommen hatte, daß der König mit seinen Gewaltigen, mit seinen
Weibern und mit seinen Kebsweibern daraus tränken. \bibverse{3} Also
wurden hergebracht die güldenen Gefäße, die aus dem Tempel, aus dem
Hause GOttes zu Jerusalem, genommen wären; und der König, seine
Gewaltigen, seine Weiber und Kebsweiber tranken daraus. \bibverse{4} Und
da sie so soffen, lobten sie die güldenen, silbernen, ehernen, eisernen,
hölzernen und steinernen Götter. \bibverse{5} Eben zur selbigen Stunde
gingen hervor Finger, als einer Menschenhand, die schrieben, gegenüber
dem Leuchter, auf die getünchte Wand in dem königlichen Saal. Und der
König ward gewahr der Hand, die da schrieb. \bibverse{6} Da entfärbte
sich der König, und seine Gedanken erschreckten ihn, daß ihm die Lenden
schütterten und die Beine zitterten. \bibverse{7} Und der König rief
überlaut, daß man die Weisen, Chaldäer und Wahrsager heraufbringen
sollte Und ließ den Weisen zu Babel sagen: Welcher Mensch diese Schrift
lieset und sagen kann, was sie bedeute, der soll mit Purpur gekleidet
werden und güldene Ketten am Halse tragen und der dritte Herr sein in
meinem Königreiche. \bibverse{8} Da wurden alle Weisen des Königs
heraufgebracht; aber sie konnten weder die Schrift lesen noch die
Deutung dem Könige anzeigen. \bibverse{9} Des erschrak der König
Belsazer noch härter und verlor ganz seine Gestalt, und seinen
Gewaltigen ward bange. \bibverse{10} Da ging die Königin um solcher
Sache willen des Königs und seiner Gewaltigen hinauf in den Saal und
sprach: Herr König, GOtt verleihe dir langes Leben! Laß dich deine
Gedanken nicht so erschrecken und entfärbe dich nicht also!
\bibverse{11} Es ist ein Mann in deinem Königreich, der den Geist der
heiligen Götter hat. Denn zu deines Vaters Zeit ward bei ihm Erleuchtung
erfunden, Klugheit und Weisheit, wie der Götter Weisheit ist; und dein
Vater, König Nebukadnezar, setzte ihn über die Sternseher, Weisen,
Chaldäer und Wahrsager, \bibverse{12} darum daß ein hoher Geist bei ihm
funden ward, dazu Verstand und Klugheit, Sprüche zu deuten, dunkle
Sprüche zu erraten und verborgene Sachen zu offenbaren, nämlich Daniel,
den der König ließ Beltsazar nennen. So rufe man nun Daniel; der wird
sagen, was es bedeute. \bibverse{13} Da ward Daniel hinauf vor den König
gebracht. Und der König sprach zu Daniel: Bist du der Daniel, der
Gefangenen einer aus Juda, die der König, mein Vater, aus Juda
hergebracht hat? \bibverse{14} Ich habe von dir hören sagen, daß du den
Geist der heiligen Götter habest, und Erleuchtung, Verstand und hohe
Weisheit bei dir funden sei. \bibverse{15} Nun hab ich vor mich fordern
lassen die Klugen und Weisen, daß sie mir diese Schrift lesen und
anzeigen sollen, was sie bedeute; und sie können mir nicht sagen, was
solches bedeute. \bibverse{16} Von dir aber höre ich, daß du könnest die
Deutung geben und das Verborgene offenbaren. Kannst du nun die Schrift
lesen und mir anzeigen, was sie bedeutet, so sollst du mit Purpur
gekleidet werden und güldene Ketten an deinem Halse tragen und der
dritte Herr sein in meinem Königreiche. \bibverse{17} Da fing Daniel an
und redete vor dem Könige: Behalte deine Gaben selbst und gib dein
Geschenk einem andern; ich will dennoch die Schrift dem Könige lesen und
anzeigen, was sie bedeute. \bibverse{18} Herr König, GOtt der Höchste
hat deinem Vater, Nebukadnezar, Königreich, Macht, Ehre und Herrlichkeit
gegeben. \bibverse{19} Und vor solcher Macht, die ihm gegeben war,
fürchteten und scheueten sich vor ihm alle Völker, Leute und Zungen. Er
tötete, wen er wollte; er schlug, wen er wollte; er erhöhete, wen er
wollte; er demütigte, wen er wollte. \bibverse{20} Da sich aber sein
Herz erhub und er stolz und hochmütig ward, ward er vom königlichen
Stuhl gestoßen und verlor seine Ehre; \bibverse{21} und ward verstoßen
von den Leuten, und sein Herz ward gleich den Tieren, und mußte bei dem
Wild laufen und fraß Gras wie Ochsen, und sein Leib lag unter dem Tau
des Himmels und ward naß, bis daß er lernete, daß GOtt der Höchste
Gewalt hat über der Menschen Königreiche und gibt sie, wem er will.
\bibverse{22} Und du, Belsazer, sein Sohn, hast dein Herz nicht
gedemütiget, ob du wohl solches alles weißt, \bibverse{23} sondern hast
dich wider den HErrn des Himmels erhoben, und die Gefäße seines Hauses
hat man vor dich bringen müssen; und du, deine Gewaltigen, deine Weiber
und deine Kebsweiber habt daraus gesoffen, dazu die silbernen, güldenen,
ehernen, eisernen, hölzernen, steinernen Götter gelobet, die weder
sehen, noch hören, noch fühlen; den GOtt aber, der deinen Odem und alle
deine Wege in seiner Hand hat, hast du nicht geehret. \bibverse{24}
Darum ist von ihm gesandt diese Hand und diese Schrift, die da
verzeichnet stehen. \bibverse{25} Das ist aber die Schrift allda
verzeichnet: Mene, mene, tekel, upharsin. \bibverse{26} Und sie bedeutet
dies: Mene, das ist, GOtt hat dein Königreich gezählet und vollendet.
\bibverse{27} Tekel, das ist, man hat dich in einer Waage gewogen und zu
leicht funden. \bibverse{28} Peres, das ist, dein Königreich ist
zerteilet und den Medern und Persern gegeben. \bibverse{29} Da befahl
Belsazer, daß man Daniel mit Purpur kleiden sollte und güldene Ketten an
den Hals geben; und ließ von ihm verkündigen, daß er der dritte Herr sei
im Königreich. \bibverse{30} Aber des Nachts ward der Chaldäer König
Belsazer getötet. \bibverse{31} Und Darius aus Medien nahm das Reich
ein, da er zweiundsechzig Jahre alt war.

\hypertarget{section-5}{%
\section{6}\label{section-5}}

\bibverse{1} Und Darius sah es für gut an, daß er über das ganze
Königreich setzte hundertundzwanzig Landvögte. \bibverse{2} Über diese
setzte er drei Fürsten, deren einer war Daniel, welchen die Landvögte
sollten Rechnung tun, und der König der Mühe überhoben wäre.
\bibverse{3} Daniel aber übertraf die Fürsten und Landvögte alle, denn
es war ein hoher Geist in ihm; darum gedachte der König ihn über das
ganze Königreich zu setzen. \bibverse{4} Derhalben trachteten die
Fürsten und Landvögte danach, wie sie eine Sache zu Daniel fänden, die
wider das Königreich wäre; aber sie konnten keine Sache noch Übeltat
finden, denn er war treu, daß man keine Schuld noch Übeltat an ihm
finden mochte. \bibverse{5} Da sprachen die Männer: Wir werden keine
Sache zu Daniel finden ohne über seinem Gottesdienst. \bibverse{6} Da
kamen die Fürsten und Landvögte häufig vor den König und sprachen zu ihm
also: Herr König Darius, GOtt verleihe dir langes Leben! \bibverse{7} Es
haben die Fürsten des Königreichs, die Herren, die Landvögte, die Räte
und Hauptleute alle gedacht, daß man einen königlichen Befehl solle
ausgehen lassen und ein streng Gebot stellen, daß, wer in dreißig Tagen
etwas bitten wird von irgendeinem Gott oder Menschen ohne von dir,
König, alleine, solle zu den Löwen in den Graben geworfen werden.
\bibverse{8} Darum, lieber König, sollst du solch Gebot bestätigen und
dich unterschreiben, auf daß nicht wieder geändert werde, nach dem Recht
der Meder und Perser, welches niemand übertreten darf. \bibverse{9} Also
unterschrieb sich der König Darius. \bibverse{10} Als nun Daniel erfuhr,
daß solch Gebot unterschrieben wäre, ging er hinauf in sein Haus (er
hatte aber an seinem Sommerhause offene Fenster gegen Jerusalem). Und er
fiel des Tages dreimal auf seine Kniee, betete, lobte und dankte seinem
GOtt, wie er denn vorhin zu tun pflegte. \bibverse{11} Da kamen diese
Männer häufig und fanden Daniel beten und flehen vor seinem GOtt.
\bibverse{12} Und traten hinzu und redeten mit dem Könige von dem
königlichen Gebot: Herr König, hast du nicht ein Gebot unterschrieben,
daß, wer in dreißig Tagen etwas bitten würde von irgendeinem Gott oder
Menschen ohne von dir, König, alleine, solle zu den Löwen in den Graben
geworfen werden? Der König antwortete und sprach: Es ist wahr, und das
Recht der Meder und Perser soll niemand übertreten. \bibverse{13} Sie
antworteten und sprachen vor dem Könige: Daniel, der Gefangenen aus Juda
einer, der achtet weder dich noch dein Gebot, das du verzeichnet hast;
denn er betet des Tages dreimal. \bibverse{14} Da der König solches
hörete, ward er sehr betrübt und tat großen Fleiß, daß er Daniel
erlösete, und mühete sich, bis die Sonne unterging, daß er ihn
errettete. \bibverse{15} Aber die Männer kamen häufig zu dem Könige und
sprachen zu ihm: Du weißt, Herr König, daß der Meder und Perser Recht
ist, daß alle Gebote und Befehle, so der König beschlossen hat, sollen
unverändert bleiben. \bibverse{16} Da befahl der König, daß man Daniel
herbrächte; und warfen ihn zu den Löwen in den Graben. Der König aber
sprach zu Daniel: Dein GOtt, dem du ohne Unterlaß dienest, der helfe
dir! \bibverse{17} Und sie brachten einen Stein, den legten sie vor die
Tür am Graben; den versiegelte der König mit seinem eigenen Ringe und
mit dem Ringe seiner Gewaltigen, auf daß sonst niemand an Daniel
Mutwillen übete. \bibverse{18} Und der König ging weg in seine Burg und
blieb ungegessen und ließ kein Essen vor sich bringen, konnte auch nicht
schlafen. \bibverse{19} Des Morgens früh, da der Tag anbrach, stund der
König auf und ging eilend zum Graben, da die Löwen waren. \bibverse{20}
Und als er zum Graben kam, rief er Daniel mit kläglicher Stimme. Und der
König sprach zu Daniel: Daniel, du Knecht des lebendigen GOttes, hat
dich auch dein GOtt, dem du ohn Unterlaß dienest, mögen von den Löwen
erlösen? \bibverse{21} Daniel aber redete mit dem Könige: Herr König,
GOtt verleihe dir langes Leben! \bibverse{22} Mein GOtt hat seinen Engel
gesandt, der den Löwen den Rachen zugehalten hat, daß sie mir kein Leid
getan haben. Denn vor ihm bin ich unschuldig erfunden, so habe ich auch
wider dich, Herr König, nichts getan. \bibverse{23} Da ward der König
sehr froh und ließ Daniel aus dem Graben ziehen. Und sie zogen Daniel
aus dem Graben, und man spürete keinen Schaden an ihm; denn er hatte
seinem GOtt vertrauet. \bibverse{24} Da hieß der König die Männer, so
Daniel verklagt hatten, herbringen und zu den Löwen in den Graben werfen
samt ihren Kindern und Weibern. Und ehe sie auf den Boden hinab kamen,
ergriffen sie die Löwen und zermalmeten auch ihre Gebeine. \bibverse{25}
Da ließ der König Darius schreiben allen Völkern, Leuten und Zungen:
GOtt gebe euch viel Frieden! \bibverse{26} Das ist mein Befehl, daß man
in der ganzen Herrschaft meines Königreichs den GOtt Daniels fürchten
und scheuen soll. Denn er ist der lebendige GOtt, der ewiglich bleibet;
und sein Königreich ist unvergänglich, und seine Herrschaft hat kein
Ende. \bibverse{27} Er ist ein Erlöser und Nothelfer, und er tut Zeichen
und Wunder, beide, im Himmel und auf Erden. Der hat Daniel von den Löwen
erlöset. \bibverse{28} Und Daniel ward gewaltig im Königreich Darius und
auch im Königreich Kores, der Perser.

\hypertarget{section-6}{%
\section{7}\label{section-6}}

\bibverse{1} Im ersten Jahr Belsazers, des Königs zu Babel, hatte Daniel
einen Traum und Gesicht auf seinem Bette; und er schrieb denselbigen
Traum und verfaßte ihn also: \bibverse{2} Ich, Daniel, sah ein Gesicht
in der Nacht, und siehe, die vier Winde unter dem Himmel stürmeten
widereinander auf dem großen Meer. \bibverse{3} Und vier große Tiere
stiegen herauf aus dem Meer, eins je anders denn das andere.
\bibverse{4} Das erste wie ein Löwe und hatte Flügel wie ein Adler. Ich
sah zu, bis daß ihm die Flügel ausgerauft wurden; und es ward von der
Erde genommen und es stund auf seinen Füßen wie ein Mensch, und ihm ward
ein menschlich Herz gegeben. \bibverse{5} Und siehe, das andere Tier
hernach war gleich einem Bären und stund auf der einen Seite und hatte
in seinem Maul unter seinen Zähnen drei große lange Zähne. Und man
sprach zu ihm: Stehe auf und friß viel Fleisch! \bibverse{6} Nach diesem
sah ich, und siehe, ein ander Tier, gleich einem Parden, das hatte vier
Flügel, wie ein Vogel, auf seinem Rücken; und dasselbige Tier hatte vier
Köpfe, und ihm ward Gewalt gegeben. \bibverse{7} Nach diesem sah ich in
diesem Gesicht in der Nacht, und siehe, das vierte Tier war greulich und
schrecklich und sehr stark und hatte große eiserne Zähne, fraß um sich
und zermalmete, und das übrige zertrat es mit seinen Füßen; es war auch
viel anders denn die vorigen und hatte zehn Hörner. \bibverse{8} Da ich
aber die Hörner schauete, siehe, da brach hervor zwischen denselbigen
ein ander klein Horn, vor welchem der vordersten Hörner drei ausgerissen
wurden; und siehe, dasselbige Horn hatte Augen wie Menschenaugen und ein
Maul, das redete große Dinge. \bibverse{9} Solches sah ich, bis daß
Stühle gesetzt wurden; und der Alte setzte sich, des Kleid war
schneeweiß und das Haar auf seinem Haupt wie reine Wolle; sein Stuhl war
eitel Feuerflammen, und desselbigen Räder brannten mit Feuer.
\bibverse{10} Und von demselbigen ging aus ein langer feuriger Strahl.
Tausendmal tausend dieneten ihm, und zehntausendmal zehntausend stunden
vor ihm. Das Gericht ward gehalten, und die Bücher wurden aufgetan.
\bibverse{11} Ich sah zu um der großen Rede willen, so das Horn redete;
ich sah zu, bis das Tier getötet ward und sein Leib umkam und ins Feuer
geworfen ward, \bibverse{12} und der andern Tiere Gewalt auch aus war;
denn es war ihnen Zeit und Stunde bestimmt, wie lange ein jegliches
währen sollte. \bibverse{13} Ich sah in diesem Gesichte des Nachts, und
siehe, es kam einer in des Himmels Wolken wie eines Menschen Sohn bis zu
dem Alten und ward vor denselbigen gebracht. \bibverse{14} Der gab ihm
Gewalt, Ehre und Reich, daß ihm alle Völker, Leute und Zungen dienen
sollten. Seine Gewalt ist ewig, die nicht vergehet, und sein Königreich
hat kein Ende. \bibverse{15} Ich, Daniel, entsetzte mich davor, und
solch Gesicht erschreckte mich. \bibverse{16} Und ich ging zu deren
einem, die da stunden, und bat ihn, daß er mir von dem allem gewissen
Bericht gäbe. Und er redete mit mir und zeigte mir, was es bedeutete.
\bibverse{17} Diese vier großen Tiere sind vier Reiche, so auf Erden
kommen werden. \bibverse{18} Aber die Heiligen des Höchsten werden das
Reich einnehmen und werden es immer und ewiglich besitzen. \bibverse{19}
Danach hätte ich gerne gewußt gewissen Bericht von dem vierten Tier,
welches gar anders war denn die andern alle, sehr greulich, das eiserne
Zähne und eherne Klauen hatte, das um sich fraß und zermalmete und das
übrige mit seinen Füßen zertrat, \bibverse{20} und von den zehn Hörnern
auf seinem Haupt und von dem andern, das hervorbrach, vor welchem drei
abfielen, und von demselbigen Horn, das Augen hatte und ein Maul, das
große Dinge redete und größer war, denn die neben ihm waren.
\bibverse{21} Und ich sah dasselbige Horn streiten wider die Heiligen
und behielt den Sieg wider sie, \bibverse{22} bis der Alte kam und
Gericht hielt für die Heiligen des Höchsten; und die Zeit kam, daß die
Heiligen das Reich einnahmen. \bibverse{23} Er sprach also: Das vierte
Tier wird das vierte Reich auf Erden sein, welches wird mächtiger sein
denn alle Reiche; es wird alle Lande fressen, zertreten und zermalmen.
\bibverse{24} Die zehn Hörner bedeuten zehn Könige, so aus demselbigen
Reich entstehen werden. Nach demselben aber wird ein anderer aufkommen,
der wird mächtiger sein denn der vorigen keiner und wird drei Könige
demütigen. \bibverse{25} Er wird den Höchsten lästern und die Heiligen
des Höchsten verstören und wird sich unterstehen, Zeit und Gesetz zu
ändern. Sie werden aber in seine Hand gegeben werden eine Zeit und
etliche Zeiten und eine halbe Zeit. \bibverse{26} Danach wird das
Gericht gehalten werden; da wird dann seine Gewalt weggenommen werden,
daß er zugrunde vertilget und umgebracht werde. \bibverse{27} Aber das
Reich, Gewalt und Macht unter dem ganzen Himmel wird dem heiligen Volk
des Höchsten gegeben werden, des Reich ewig ist, und alle Gewalt wird
ihm dienen und gehorchen. \bibverse{28} Das war der Rede Ende. Aber ich,
Daniel, ward sehr betrübt in meinen Gedanken, und meine Gestalt verfiel;
doch behielt ich die Rede in meinem Herzen.

\hypertarget{section-7}{%
\section{8}\label{section-7}}

\bibverse{1} Im dritten Jahr des Königreichs des Königs Belsazer
erschien mir, Daniel, ein Gesicht nach dem, so mir am ersten erschienen
war. \bibverse{2} Ich war aber, da ich solch Gesicht sah, zu Schloß
Susan im Lande Elam am Wasser Ulai. \bibverse{3} Und ich hub meine Augen
auf und sah, und siehe, ein Widder stund vor dem Wasser, der hatte zwei
hohe Hörner, doch eins höher denn das andere, und das höchste wuchs am
letzten. \bibverse{4} Ich sah, daß der Widder mit den Hörnern stieß
gegen Abend, gegen Mitternacht und gegen Mittag, und kein Tier konnte
vor ihm bestehen noch von seiner Hand errettet werden, sondern er tat,
was er wollte, und ward groß. \bibverse{5} Und indem ich darauf merkte,
siehe, so kommt ein Ziegenbock vom Abend her über die ganze Erde, daß er
die Erde nicht rührete; und der Bock hatte ein ansehnlich Horn zwischen
seinen Augen. \bibverse{6} Und er kam bis zu dem Widder, der zwei Hörner
hatte, den ich stehen sah vor dem Wasser; und er lief in seinem Zorn
gewaltiglich zu ihm zu. \bibverse{7} Und ich sah ihm zu, daß er hart an
den Widder kam, und ergrimmete über ihn und stieß den Widder und
zerbrach ihm seine zwei Hörner. Und der Widder hatte keine Kraft, daß er
vor ihm hätte mögen bestehen, sondern er warf ihn zu Boden und zertrat
ihn; und niemand konnte den Widder von seiner Hand erretten.
\bibverse{8} Und der Ziegenbock ward sehr groß. Und da er aufs stärkste
worden war, zerbrach das große Horn; und wuchsen an des Statt
ansehnliche vier gegen die vier Winde des Himmels. \bibverse{9} Und aus
derselbigen einem wuchs ein klein Horn, das ward sehr groß gegen Mittag,
gegen Morgen und gegen das werte Land. \bibverse{10} Und es wuchs bis an
des Himmels Heer und warf etliche davon und von den Sternen zur Erde und
zertrat sie. \bibverse{11} Ja, es wuchs bis an den Fürsten des Heers und
nahm von ihm weg das tägliche Opfer und verwüstete die Wohnung seines
Heiligtums. \bibverse{12} Es ward ihm aber solche Macht gegeben wider
das tägliche Opfer um der Sünde willen, daß er die Wahrheit zu Boden
schlüge und, was er tat, ihm gelingen mußte. \bibverse{13} Ich hörete
aber einen Heiligen reden; und derselbige Heilige sprach zu einem, der
da redete: Wie lange soll doch währen solch Gesicht vom täglichen Opfer
und von der Sünde, um welcher willen diese Verwüstung geschieht, daß
beide, das Heiligtum und das Heer, zertreten werden? \bibverse{14} Und
er antwortete mir: Es sind zweitausend und dreihundert Tage, von Abend
gegen Morgen zu rechnen, so wird das Heiligtum wieder geweihet werden.
\bibverse{15} Und da ich, Daniel, solch Gesicht sah und hätte es gerne
verstanden, siehe, da stund es vor mir wie ein Mann. \bibverse{16} Und
ich hörete zwischen Ulai eines Menschen Stimme, der rief und sprach:
Gabriel, lege diesem das Gesicht aus, daß er's verstehe! \bibverse{17}
Und er kam hart zu mir. Ich erschrak aber, da er kam, und fiel auf mein
Angesicht. Er aber sprach zu mir: Merke auf, du Menschenkind; denn dies
Gesicht gehört in die Zeit des Endes. \bibverse{18} Und da er mit mir
redete, sank ich in eine Ohnmacht zur Erde auf mein Angesicht. Er aber
rührete mich an und richtete mich auf, daß ich stund. \bibverse{19} Und
er sprach: Siehe, ich will dir zeigen, wie es gehen wird zur Zeit des
letzten Zorns; denn das Ende hat seine bestimmte Zeit. \bibverse{20} Der
Widder mit den zweien Hörnern, den du gesehen hast, sind die Könige in
Medien und Persien. \bibverse{21} Der Ziegenbock aber ist der König in
Griechenland. Das große Horn zwischen seinen Augen ist der erste König.
\bibverse{22} Daß aber vier an seiner Statt stunden, da es zerbrochen
war, bedeutet, daß vier Königreiche aus dem Volk entstehen werden, aber
nicht so mächtig, als er war. \bibverse{23} Nach diesen Königreichen,
wenn die Übertreter überhandnehmen, wird aufkommen ein frecher und
tückischer König. \bibverse{24} Der wird mächtig sein, doch nicht durch
seine Kraft. Er wird's wunderlich verwüsten; und wird ihm gelingen, daß
er's ausrichte. Er wird die Starken samt dem heiligen Volk verstören.
\bibverse{25} Und durch seine Klugheit wird ihm der Betrug geraten. Und
wird sich in seinem Herzen erheben und durch Wohlfahrt wird er viele
verderben und wird sich auflehnen wider den Fürsten aller Fürsten. Aber
er wird ohne Hand zerbrochen werden. \bibverse{26} Dies Gesicht vom
Abend und Morgen, das dir gesagt ist, das ist wahr; aber du sollst das
Gesicht heimlich halten, denn es ist noch eine lange Zeit dahin.
\bibverse{27} Und ich, Daniel, ward schwach und lag etliche Tage krank.
Danach stand ich auf und richtete aus des Königs Geschäfte. Und
verwunderte mich des Gesichts; und niemand war, der mir's berichtete.

\hypertarget{section-8}{%
\section{9}\label{section-8}}

\bibverse{1} Im ersten Jahr Darius, des Sohnes Ahasveros, aus der Meder
Stamm, der über das Königreich der Chaldäer König ward, \bibverse{2} in
demselbigen ersten Jahr seines Königreichs merkte ich, Daniel, in den
Büchern auf die Zahl der Jahre, davon der HErr geredet hatte zum
Propheten Jeremia, daß Jerusalem sollte siebenzig Jahre wüste liegen.
\bibverse{3} Und ich kehrete mich zu GOtt dem HErrn, zu beten und zu
flehen, mit Fasten, im Sack und in der Asche. \bibverse{4} Ich betete
aber zu dem HErrn, meinem GOtt, bekannte und sprach: Ach, lieber HErr,
du großer und schrecklicher GOtt, der du Bund und Gnade hältst denen,
die dich lieben und deine Gebote halten: \bibverse{5} wir haben
gesündigt, unrecht getan, sind gottlos gewesen und abtrünnig worden; wir
sind von deinen Geboten und Rechten gewichen. \bibverse{6} Wir
gehorchten nicht deinen Knechten, den Propheten, die in deinem Namen
unsern Königen, Fürsten, Vätern und allem Volk im Lande predigten.
\bibverse{7} Du, HErr, bist gerecht, wir aber müssen uns schämen, wie es
denn jetzt gehet denen von Juda und denen von Jerusalem und dem ganzen
Israel, beide, denen, die nahe und ferne sind, in allen Landen, dahin du
uns verstoßen hast um ihrer Missetat willen, die sie an dir begangen
haben. \bibverse{8} Ja, HErr, wir, unsere Könige, unsere Fürsten und
unsere Väter müssen uns schämen, daß wir uns an dir versündiget haben.
\bibverse{9} Dein aber, HErr, unser GOtt, ist die Barmherzigkeit und
Vergebung. Denn wir sind abtrünnig worden \bibverse{10} und gehorchten
nicht der Stimme des HErrn, unsers GOttes, daß wir gewandelt hätten in
seinem Gesetz welches er uns vorlegte durch seine Knechte, die
Propheten, \bibverse{11} sondern das ganze Israel übertrat dein Gesetz
und wichen ab, daß sie deiner Stimme nicht gehorchten. Daher trifft uns
auch der Fluch und Schwur, der geschrieben stehet im Gesetz Mose, des
Knechtes GOttes, daß wir an ihm gesündiget haben. \bibverse{12} Und er
hat seine Worte gehalten, die er geredet hat wider uns und unsere
Richter, die uns richten sollten, daß er solch groß Unglück über uns hat
gehen lassen, daß desgleichen unter allem Himmel nicht geschehen ist,
wie über Jerusalem geschehen ist. \bibverse{13} Gleichwie es geschrieben
stehet im Gesetz Mose, so ist all dies große Unglück über uns gegangen.
So beteten wir auch nicht vor dem HErrn, unserm GOtt, daß wir uns von
den Sünden bekehreten und deine Wahrheit vernähmen. \bibverse{14} Darum
ist der HErr auch wacker gewesen mit diesem Unglück und hat es über uns
gehen lassen. Denn der HErr, unser GOtt, ist gerecht in allen seinen
Werken, die er tut; denn wir gehorchten seiner Stimme nicht.
\bibverse{15} Und nun, HErr, unser GOtt, der du dein Volk aus
Ägyptenland geführet hast mit starker Hand und hast dir einen Namen
gemacht, wie er jetzt ist: wir haben ja gesündiget und sind leider
gottlos gewesen. \bibverse{16} Ach HErr, um aller deiner Gerechtigkeit
willen wende ab deinen Zorn und Grimm von deiner Stadt Jerusalem und
deinem heiligen Berge! Denn um unserer Sünde willen und um unserer Väter
Missetat willen trägt Jerusalem und dein Volk Schmach bei allen, die um
uns her sind. \bibverse{17} Und nun, unser GOtt, höre das Gebet deines
Knechts und sein Flehen und siehe gnädiglich an dein Heiligtum, das
verstöret ist, um des HErrn willen! \bibverse{18} Neige deine Ohren,
mein GOtt, und höre, tue deine Augen auf und siehe, wie wir verstört
sind, und die Stadt, die nach deinem Namen genannt ist! Denn wir liegen
vor dir mit unserm Gebet, nicht auf unsere Gerechtigkeit, sondern auf
deine große Barmherzigkeit. \bibverse{19} Ach HErr, höre, ach HErr, sei
gnädig, ach HErr, merke auf und tue es und verzeuch nicht um dein selbst
willen, mein GOtt! Denn deine Stadt und dein Volk ist nach deinem Namen
genannt. \bibverse{20} Als ich noch so redete und betete und meine und
meines Volks Israel Sünde bekannte und lag mit meinem Gebet vor dem
HErrn, meinem GOtt, um den heiligen Berg meines GOttes, \bibverse{21}
eben da ich so redete in meinem Gebet, flog daher der Mann Gabriel, den
ich vorhin gesehen hatte im Gesicht, und rührete mich an um die Zeit des
Abendopfers. \bibverse{22} Und er berichtete mir und redete mit mir und
sprach: Daniel, jetzt bin ich ausgegangen, dir zu berichten.
\bibverse{23} Denn da du anfingest zu beten, ging dieser Befehl aus, und
ich komme darum, daß ich dir's anzeige; denn du bist lieb und wert. So
merke nun darauf, daß du das Gesicht verstehest! \bibverse{24} Siebenzig
Wochen sind bestimmt über dein Volk und über deine heilige Stadt, so
wird dem Übertreten gewehret und die Sünde zugesiegelt und die Missetat
versöhnet und die ewige Gerechtigkeit gebracht und die Gesichte und
Weissagung zugesiegelt und der Allerheiligste gesalbet werden.
\bibverse{25} So wisse nun und merke: Von der Zeit an, so ausgehet der
Befehl, daß Jerusalem soll wiederum gebauet werden, bis auf Christum,
den Fürsten, sind sieben Wochen und zweiundsechzig Wochen, so werden die
Gassen und Mauern wieder gebauet werden, wiewohl in kümmerlicher Zeit.
\bibverse{26} Und nach den zweiundsechzig Wochen wird Christus
ausgerottet werden und nichts mehr sein. Und ein Volk des Fürsten wird
kommen und die Stadt und das Heiligtum verstören, daß es ein Ende nehmen
wird wie durch eine Flut; und bis zum Ende des Streits wird's wüst
bleiben. \bibverse{27} Er wird aber vielen den Bund stärken eine Woche
lang. Und mitten in der Woche wird das Opfer und Speisopfer aufhören.
Und bei den Flügeln werden stehen Greuel der Verwüstung; und ist
beschlossen, daß bis ans Ende über die Verwüstung triefen wird.

\hypertarget{section-9}{%
\section{10}\label{section-9}}

\bibverse{1} Im dritten Jahr des Königs Kores aus Persien ward dem
Daniel, der Beltsazar heißt, etwas offenbaret, das gewiß ist und von
großen Sachen; und er merkte darauf und verstund das Gesicht wohl.
\bibverse{2} Zur selbigen Zeit war ich, Daniel, traurig drei Wochen
lang. \bibverse{3} Ich aß keine niedliche Speise, Fleisch und Wein kam
in meinen Mund nicht; und salbete mich auch nie, bis die drei Wochen um
waren. \bibverse{4} Am vierundzwanzigsten Tage des ersten Monden war
ich, bei dem großen Wasser Hiddekel \bibverse{5} und hub meine Augen auf
und sah, und siehe, da stund ein Mann in Leinwand und hatte einen
güldenen Gürtel um seine Lenden. \bibverse{6} Sein Leib war wie ein
Türkis, sein Antlitz sah wie ein Blitz, seine Augen wie eine feurige
Fackel, seine Arme und Füße wie ein glühend Erz, und seine Rede war wie
ein groß Getön. \bibverse{7} Ich, Daniel, aber sah solch Gesicht
alleine, und die Männer, so bei mir waren, sahen's nicht; doch fiel ein
groß Schrecken über sie, daß sie flohen und sich verkrochen.
\bibverse{8} Und ich blieb alleine und sah dies große Gesicht. Es blieb
aber keine Kraft in mir, und ich ward sehr ungestalt und hatte keine
Kraft mehr. \bibverse{9} Und ich hörete seine Rede; und indem ich sie
hörete, sank ich nieder auf mein Angesicht zur Erde. \bibverse{10} Und
siehe, eine Hand rührete mich an und half mir auf die Kniee und auf die
Hände \bibverse{11} und sprach zu mir: Du lieber Daniel, merke auf die
Worte, die ich mit dir rede, und richte dich auf; denn ich bin jetzt zu
dir gesandt. Und da er solches mit mir redete, richtete ich mich auf und
zitterte. \bibverse{12} Und er sprach zu mir: Fürchte dich nicht,
Daniel; denn von dem ersten Tage an, da du von Herzen begehretest zu
verstehen, und dich kasteietest vor deinem GOtt, sind deine, Worte
erhöret; und ich bin kommen um deinetwillen. \bibverse{13} Aber der
Fürst des Königreichs in Persienland hat mir einundzwanzig Tage
widerstanden; und siehe, Michael, der vornehmsten Fürsten einer, kam mir
zu Hilfe; da behielt ich den Sieg bei den Königen in Persien.
\bibverse{14} Nun aber komme ich, daß ich dir berichte, wie es deinem
Volk hernach gehen wird; denn das Gesicht wird nach etlicher Zeit
geschehen. \bibverse{15} Und als er solches mit mir redete, schlug ich
mein Angesicht nieder zur Erde und schwieg stille. \bibverse{16} Und
siehe, einer, gleich einem Menschen, rührete meine Lippen an. Da tat ich
meinen Mund auf und redete und sprach zu dem, der vor mir stund: Mein
Herr, meine Gelenke beben mir über dem Gesicht, und ich habe keine Kraft
mehr. \bibverse{17} Und wie kann der Knecht meines Herrn mit meinem
Herrn reden, weil nun keine Kraft mehr in mir ist, und habe auch keinen
Odem mehr? \bibverse{18} Da rührete mich abermal an einer, gleichwie ein
Mensch gestaltet, und stärkte mich \bibverse{19} und sprach: Fürchte
dich nicht, du lieber Mann! Friede sei mit dir; und sei getrost, sei
getrost! Und als er mit mir redete, ermannete ich mich und sprach: Mein
Herr, rede; denn du hast mich gestärkt. \bibverse{20} Und er sprach:
Weißt du auch, warum ich zu dir kommen bin? Jetzt will ich wieder hin
und mit dem Fürsten in Persienland streiten; aber wenn ich wegziehe,
siehe, so wird der Fürst aus Griechenland kommen. \bibverse{21} Doch
will ich dir anzeigen, was geschrieben ist, das gewißlich geschehen
wird. Und ist keiner, der mir hilft wider jene denn euer Fürst Michael.

\hypertarget{section-10}{%
\section{11}\label{section-10}}

\bibverse{1} Denn ich stund auch bei ihm im ersten Jahr Darius, des
Meders, daß ich ihm hülfe und ihn stärkete. \bibverse{2} Und nun will
ich dir anzeigen, was gewiß geschehen soll. Siehe, es werden noch drei
Könige in Persien stehen; der vierte aber wird größern Reichtum haben
denn alle andern; und wenn er in seinem Reichtum am mächtigsten ist,
wird er alles wider das Königreich in Griechenland erregen. \bibverse{3}
Danach wird ein mächtiger König aufstehen und mit großer Macht
herrschen, und was er will, wird er ausrichten. \bibverse{4} Und wenn er
aufs höchste kommen ist, wird sein Reich zerbrechen und sich in die vier
Winde des Himmels zerteilen, nicht auf seine Nachkommen, auch nicht mit
solcher Macht, wie seine gewesen ist; denn sein Reich wird ausgerottet
und Fremden zuteil werden. \bibverse{5} Und der König gegen Mittag,
welcher ist seiner Fürsten einer, wird mächtig werden; aber gegen ihn
wird einer auch mächtig sein und herrschen, welches Herrschaft wird groß
sein. \bibverse{6} Nach etlichen Jahren aber werden sie sich miteinander
befreunden; und die Tochter des Königs gegen Mittag wird kommen zum
Könige gegen Mitternacht, Einigkeit zu machen. Aber sie wird nicht
bleiben bei der Macht des Arms, dazu ihr Same auch nicht stehen bleiben,
sondern sie wird übergeben samt denen, die sie gebracht haben, und mit
dem Kinde und dem, der sie eine Weile mächtig gemacht hatte.
\bibverse{7} Es wird aber der Zweige einer von ihrem Stamm aufkommen,
der wird kommen mit Heereskraft und dem Könige gegen Mitternacht in
seine Feste fallen; und wird's, ausrichten und siegen. \bibverse{8} Auch
wird er ihre Götter und Bilder samt den köstlichen Kleinoden, beide,
silbernen und güldenen wegführen nach Ägypten und etliche Jahre vor dem
Könige gegen Mitternacht wohl stehen bleiben. \bibverse{9} Und wenn er
durch desselbigen Königreich gezogen ist, wird er wiederum in sein Land
ziehen. \bibverse{10} Aber seine Söhne werden erzürnen und große Heere
zusammenbringen; und der eine wird kommen und wie eine Flut daherfahren
und jenen wiederum vor seinen Festen reizen. \bibverse{11} Da wird der
König gegen Mittag ergrimmen und ausziehen und mit dem Könige gegen
Mitternacht streiten und wird solchen großen Haufen zusammenbringen, daß
ihm jener Haufe wird in seine Hand gegeben. \bibverse{12} Und wird
denselbigen Haufen wegführen. Des wird sich sein Herz erheben, daß er so
viel tausend daniedergelegt hat; aber damit wird er sein nicht mächtig
werden. \bibverse{13} Denn der König gegen Mitternacht wird wiederum
einen größern Haufen zusammenbringen, denn der vorige war; und nach
etlichen Jahren wird er daherziehen mit großer Heereskraft und mit
großem Gut. \bibverse{14} Und zur selbigen Zeit werden sich viele wider
den König gegen Mittag setzen; auch werden sich etliche Abtrünnige aus
deinem Volk erheben und die Weissagung erfüllen und werden fallen.
\bibverse{15} Also wird der König gegen Mitternacht daherziehen und
Schütte machen und feste Städte gewinnen; und die Mittagsarme werden's
nicht können wehren, und sein bestes Volk werden nicht können
widerstehen, \bibverse{16} sondern er wird, wenn er an ihn kommt, seinen
Willen schaffen; und niemand wird ihm widerstehen mögen. Er wird auch in
das werte Land kommen und wird's vollenden durch seine Hand.
\bibverse{17} Und wird sein Angesicht richten, daß er mit Macht seines
ganzen Königreichs komme. Aber er wird sich mit ihm vertragen und wird
ihm seine Tochter zum Weibe geben, daß er ihn verderbe; aber es wird ihm
nicht geraten, und wird nichts daraus werden. \bibverse{18} Danach wird
er sich kehren wider die Inseln und derselbigen viele gewinnen. Aber ein
Fürst wird ihn lehren aufhören mit Schmähen, daß er ihn nicht mehr
schmähe. \bibverse{19} Also wird er sich wiederum kehren zu den Festen
seines Landes und wird sich stoßen und fallen, daß man ihn nirgend
finden wird. \bibverse{20} Und an seiner Statt wird einer aufkommen, der
wird in königlichen Ehren sitzen wie ein Scherge. Aber nach wenig Tagen
wird er brechen, doch weder durch Zorn noch durch Streit. \bibverse{21}
An des Statt wird aufkommen ein Ungeachteter, welchem die Ehre des
Königreichs nicht bedacht war; der wird kommen, und wird ihm gelingen
und das Königreich mit süßen Worten einnehmen. \bibverse{22} Und die
Arme, die wie eine Flut daherfahren, werden vor ihm wie mit einer Flut
überfallen und zerbrochen werden, dazu auch der Fürst, mit dem der Bund
gemacht war. \bibverse{23} Denn nachdem er mit ihm befreundet ist, wird
er listiglich gegen ihn handeln; und wird heraufziehen und mit geringem
Volk ihn überwältigen. \bibverse{24} Und wird ihm gelingen, daß er in
die besten Städte des Landes kommen wird; und wird's also ausrichten,
das seine Väter noch seine Voreltern nicht tun konnten mit Rauben,
Plündern und Ausbeuten; und wird nach den allerfestesten Städten
trachten, und das eine Zeitlang. \bibverse{25} Und er wird seine Macht
und sein Herz wider den König gegen Mittag erregen mit großer
Heereskraft. Da wird der König gegen Mittag gereizet werden zum Streit
mit einer großen, mächtigen Heereskraft. Aber er wird nicht bestehen;
denn es werden Verrätereien wider ihn gemacht. \bibverse{26} Und eben,
die sein Brot essen, die werden ihn helfen verderben und sein Heer
unterdrücken, daß gar viele erschlagen werden. \bibverse{27} Und beider
Könige Herz wird denken, wie sie einander Schaden tun, und werden doch
über einem Tisch fälschlich miteinander reden. Es wird ihnen aber
fehlen; denn das Ende ist noch auf eine andere Zeit bestimmt.
\bibverse{28} Danach wird er wiederum heimziehen mit großem Gut und sein
Herz richten wider den heiligen Bund; da wird er etwas ausrichten und
also heim in sein Land ziehen. \bibverse{29} Danach wird er zu gelegener
Zeit wieder gegen Mittag ziehen; aber es wird ihm zum andernmal nicht
geraten wie zum erstenmal. \bibverse{30} Denn es werden Schiffe aus
Chittim wider ihn kommen, daß er verzagen wird und umkehren muß. Da wird
er wider den heiligen Bund ergrimmen und wird's ausrichten; und wird
sich umsehen und an sich ziehen, die den heiligen Bund verlassen.
\bibverse{31} Und es werden seine Arme daselbst stehen; die werden das
Heiligtum in der Feste entweihen und das tägliche Opfer abtun und einen
Greuel der Wüstung aufrichten. \bibverse{32} Und er wird heucheln und
gute Worte geben den Gottlosen, so den Bund übertreten. Aber das Volk,
so ihren GOtt kennen, werden sich ermannen und es ausrichten.
\bibverse{33} Und die Verständigen im Volk werden viel andere lehren;
darüber werden sie fallen durch Schwert, Feuer, Gefängnis und Raub eine
Zeitlang. \bibverse{34} Und wenn sie so fallen, wird ihnen dennoch eine
kleine Hilfe geschehen. Aber viele werden sich zu ihnen tun betrüglich.
\bibverse{35} Und der Verständigen werden etliche fallen, auf daß sie
bewähret, rein und lauter werden, bis daß es ein Ende habe; denn es ist
noch eine andere Zeit vorhanden. \bibverse{36} Und der König wird tun,
was er will, und wird sich erheben und aufwerfen wider alles, das GOtt
ist; und wider den GOtt aller Götter wird er greulich reden; und wird
ihm gelingen, bis der Zorn aus sei; denn es ist beschlossen, wie lange
es währen soll. \bibverse{37} Und seiner Väter GOtt wird er nicht
achten; er wird weder Frauenliebe noch einiges Gottes achten, denn er
wird sich wider alles aufwerfen. \bibverse{38} Aber an des Statt wird er
seinen Gott Maußim ehren; denn er wird einen Gott, davon seine Väter
nichts gewußt haben, ehren mit Gold, Silber, Edelstein und Kleinoden.
\bibverse{39} Und wird denen, so ihm helfen stärken Maußim mit dem
fremden Gott, den er erwählet hat, große Ehre tun und sie zu Herren
machen über große Güter und ihnen das Land zu Lohn austeilen.
\bibverse{40} Und am Ende wird sich der König gegen Mittag mit ihm
stoßen; und der König gegen Mitternacht wird sich gegen ihn sträuben mit
Wagen, Reitern und viel Schiffen; und wird in die Länder fallen und
verderben und durchziehen. \bibverse{41} Und wird in das werte Land
fallen, und viele werden umkommen. Diese aber werden seiner Hand
entrinnen: Edom, Moab und die Erstlinge der Kinder Ammon. \bibverse{42}
Und er wird seine Macht in die Länder schicken, und Ägypten wird ihm
nicht entrinnen, \bibverse{43} sondern er wird durch seinen Zug
herrschen über die güldenen und silbernen Schätze und über alle Kleinode
Ägyptens, Libyens und der Mohren. \bibverse{44} Es wird ihn aber ein
Geschrei erschrecken von Morgen und Mitternacht; und er wird mit großem
Grimm ausziehen, willens, viele zu vertilgen und zu verderben.
\bibverse{45} Und er wird das Gezelt seines Palasts aufschlagen zwischen
zweien Meeren um den werten heiligen Berg, bis mit ihm ein Ende werde;
und niemand wird ihm helfen.

\hypertarget{section-11}{%
\section{12}\label{section-11}}

\bibverse{1} Zur selbigen Zeit wird der große Fürst Michael, der für
dein Volk stehet, sich aufmachen. Denn es wird eine solche trübselige
Zeit sein, als sie nicht gewesen ist, seit daß Leute gewesen sind, bis
auf dieselbige Zeit. Zur selbigen Zeit wird dein Volk errettet werden,
alle, die im Buch geschrieben stehen. \bibverse{2} Und viele, so unter
der Erde schlafen liegen, werden aufwachen, etliche zum ewigen Leben,
etliche zur ewigen Schmach und Schande. \bibverse{3} Die Lehrer aber
werden leuchten wie des Himmels Glanz und die, so viele zur
Gerechtigkeit weisen, wie die Sterne immer und ewiglich. \bibverse{4}
Und nun, Daniel, verbirg diese Worte und versiegele diese Schrift bis
auf die letzte Zeit, so werden viele drüber kommen und großen Verstand
finden. \bibverse{5} Und ich, Daniel, sah, und siehe, es stunden zween
andere da, einer an diesem Ufer des Wassers, der andere an jenem Ufer.
\bibverse{6} Und er sprach zu dem in leinenen Kleidern, der oben am
Wasser stund: Wann will's denn ein Ende sein mit solchen Wundern?
\bibverse{7} Und ich hörete zu dem in leinenen Kleidern, der oben am
Wasser stund; und er hub seine rechte und linke Hand auf gen Himmel und
schwur bei dem, so ewiglich lebet, daß es eine Zeit und etliche Zeiten
und eine halbe Zeit währen soll; und wenn die Zerstreuung des heiligen
Volks ein Ende hat, soll solches alles geschehen. \bibverse{8} Und ich
hörete es: aber ich verstund es nicht und sprach: Mein Herr, was wird
danach werden? \bibverse{9} Er aber sprach: Gehe hin, Daniel; denn es
ist verborgen und versiegelt bis auf die letzte Zeit. \bibverse{10}
Viele werden gereiniget, geläutert und bewähret werden; und die
Gottlosen werden gottlos Wesen führen, und die Gottlosen werden's nicht
achten; aber die Verständigen werden's achten. \bibverse{11} Und von der
Zeit an, wenn das tägliche Opfer abgetan und ein Greuel der Wüstung
dargesetzt wird, sind tausend zweihundert und neunzig Tage:
\bibverse{12} Wohl dem, der da erwartet und erreichet tausend
dreihundert und fünfunddreißig Tage! \bibverse{13} Du aber, Daniel, gehe
hin, bis das Ende komme, und ruhe, daß du aufstehest in deinem Teil am
Ende der Tage!
