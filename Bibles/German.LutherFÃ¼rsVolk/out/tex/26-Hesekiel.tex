\hypertarget{section}{%
\section{1}\label{section}}

\bibverse{1} Im dreißigsten Jahr, am fünften Tage des vierten Monden, da
ich war unter den Gefangenen am Wasser Chebar, tat sich der Himmel auf,
und GOtt zeigte mir Gesichte. \bibverse{2} Derselbe fünfte Tag des
Monden war eben im fünften Jahr, nachdem Jojachin, der König Judas, war
gefangen weggeführet. \bibverse{3} Da geschah des HErrn Wort zu
Hesekiel, dem Sohne Busis, des Priesters, im Lande der Chaldäer, am
Wasser Chebar; daselbst kam die Hand des HErrn über ihn. \bibverse{4}
Und ich sah, und siehe, es kam ein ungestümer Wind von Mitternacht her
mit einer großen Wolke voll Feuers, das allenthalben umher glänzte; und
mitten in demselben Feuer war es wie lichthell. \bibverse{5} Und drinnen
war es gestaltet wie vier Tiere, und unter ihnen eins gestaltet wie ein
Mensch. \bibverse{6} Und ein jegliches hatte vier Angesichte und vier
Flügel. \bibverse{7} Und ihre Beine stunden gerade, aber ihre Füße waren
gleichwie runde Füße und glänzten wie ein hell, glatt Erz. \bibverse{8}
Und hatten Menschenhände unter ihren Flügeln an ihren vier Orten; denn
sie hatten alle vier ihre Angesichte und ihre Flügel. \bibverse{9} Und
derselbigen Flügel war je einer an dem andern. Und wenn sie gingen,
durften sie sich nicht herumlenken, sondern wo sie hingingen, gingen sie
stracks vor sich. \bibverse{10} Ihre Angesichte zur rechten Seite der
viere waren gleich einem Menschen und Löwen; aber zur linken Seite der
viere waren Ihre Angesichte gleich einem Ochsen und Adler. \bibverse{11}
Und ihre Angesichte und Flügel waren obenher zerteilet, daß je zween
Flügel zusammenschlugen und mit zween Flügeln ihren Leib bedeckten.
\bibverse{12} Wo sie hingingen, da gingen sie stracks vor sich; sie
gingen aber, wohin der Wind stund; und durften sich nicht herumlenken,
wenn sie gingen. \bibverse{13} Und die Tiere waren anzusehen wie feurige
Kohlen, die da brennen, und wie Fackeln, die zwischen den Tieren gingen.
Das Feuer aber gab einen Glanz von sich, und aus dem Feuer ging ein
Blitz. \bibverse{14} Die Tiere aber liefen hin und her wie ein Blitz.
\bibverse{15} Als ich die Tiere so sah, siehe, da stund ein Rad auf der
Erde bei den vier Tieren und war anzusehen wie vier Räder. \bibverse{16}
Und dieselbigen Räder waren wie ein Türkis und waren alle vier eins wie
das andere; und sie waren anzusehen, als wäre ein Rad im andern.
\bibverse{17} Wenn sie gehen sollten, konnten sie in alle ihre vier Orte
gehen und durften sich nicht herumlenken, wenn sie gingen. \bibverse{18}
Ihre Felgen und Höhe waren schrecklich; und ihre Felgen waren voller
Augen um und um an allen vier Rädern. \bibverse{19} Und wenn die Tiere
gingen, so gingen die Räder auch neben ihnen; und wenn die Tiere sich
von der Erde emporhuben, so huben sich die Räder auch empor.
\bibverse{20} Wo der Wind hinging, da gingen sie auch hin; und die Räder
huben sich neben ihnen empor; denn es war ein lebendiger Wind in den
Rädern. \bibverse{21} Wenn sie gingen, so gingen diese auch; wenn sie
stunden, so stunden diese auch; und wenn sie sich emporhuben von der
Erde, so huben sich auch die Räder neben ihnen empor; denn es war ein
lebendiger Wind in den Rädern. \bibverse{22} Oben aber über den Tieren
war es gleich gestaltet wie der Himmel, als ein Kristall, schrecklich,
gerade oben über ihnen ausgebreitet, \bibverse{23} daß unter dem Himmel
ihre Flügel einer stracks gegen den andern stund, und eines jeglichen
Leib bedeckten zween Flügel. \bibverse{24} Und ich hörete die Flügel
rauschen wie große Wasser und wie ein Getön des Allmächtigen, wenn sie
gingen, und wie ein Getümmel in einem Heer. Wenn sie aber stille
stunden, so ließen sie die Flügel nieder. \bibverse{25} Und wenn sie
stille stunden und die Flügel niederließen, so donnerte es im Himmel
oben über ihnen. \bibverse{26} Und über dem Himmel, so oben über ihnen
war, war es gestaltet wie ein Saphir, gleichwie ein Stuhl; und auf
demselbigen Stuhl saß einer, gleichwie ein Mensch gestaltet.
\bibverse{27} Und ich sah, und es war wie lichthell, und inwendig war es
gestaltet wie ein Feuer um und um. Von seinen Lenden über sich und unter
sich, sah ich's wie Feuer glänzen um und um. \bibverse{28} Gleichwie der
Regenbogen stehet in den Wolken, wenn es geregnet hat, also glänzte es
um und um. Dies war das Ansehen der Herrlichkeit des HErrn. Und da ich's
gesehen hatte, fiel ich auf mein Angesicht und hörete einen reden.

\hypertarget{section-1}{%
\section{2}\label{section-1}}

\bibverse{1} Und er sprach zu mir: Du Menschenkind, tritt auf deine
Füße, so will ich mit dir reden. \bibverse{2} Und da er so mit mir
redete, ward ich wieder erquickt und trat auf meine Füße und hörete dem
zu, der mit mir redete. \bibverse{3} Und er sprach zu mir: Du
Menschenkind, ich sende dich zu den Kindern Israel, zu dem abtrünnigen
Volk, so von mir abtrünnig worden sind. Sie samt ihren Vätern haben bis
auf diesen heutigen Tag wider mich getan. \bibverse{4} Aber die Kinder,
zu welchen ich dich sende, haben harte Köpfe und verstockte Herzen. Zu
denen sollst du sagen: So spricht der HErr HErr! \bibverse{5} Sie
gehorchen oder lassen's. Es ist wohl ein ungehorsam Haus; dennoch sollen
sie wissen, daß ein Prophet unter ihnen ist. \bibverse{6} Und du,
Menschenkind, sollst dich vor ihnen nicht fürchten noch vor ihren Worten
fürchten. Es sind wohl widerspenstige und stachlige Dornen bei dir, und
du wohnest unter den Skorpionen; aber du sollst dich nicht fürchten vor
ihren Worten noch vor ihrem Angesicht dich entsetzen, ob sie wohl ein
ungehorsam Haus sind, \bibverse{7} sondern du sollst ihnen mein Wort
sagen, sie gehorchen oder lassen's; denn es ist ein ungehorsam Volk.
\bibverse{8} Aber du, Menschenkind, höre du, was ich dir sage, und sei
nicht ungehorsam, wie das ungehorsame Haus ist. Tu deinen Mund auf und
iß, was ich dir geben werde. \bibverse{9} Und ich sah, und siehe, da war
eine Hand gegen mich ausgereckt, die hatte einen zusammengelegten Brief.
\bibverse{10} Den breitete sie aus vor mir, und er war beschrieben
auswendig und inwendig; und stund drinnen geschrieben: Klage, Ach und
Wehe.

\hypertarget{section-2}{%
\section{3}\label{section-2}}

\bibverse{1} Und er sprach zu mir: Du Menschenkind, iß, was vor dir ist,
nämlich diesen Brief, und gehe hin und predige dem Hause Israel!
\bibverse{2} Da tat ich meinen Mund auf, und er gab mir den Brief zu
essen \bibverse{3} und sprach zu mir: Du Menschenkind, du mußt diesen
Brief, den ich dir gebe, in deinen Leib essen und deinen Bauch damit
füllen. Da aß ich ihn, und er war in meinem Mund so süß als Honig.
\bibverse{4} Und er sprach zu mir: Du Menschenkind, gehe hin zum Hause
Israel und predige ihnen mein Wort! \bibverse{5} Denn ich sende dich ja
nicht zum Volk, das eine fremde Rede und unbekannte Sprache habe,
sondern zum Hause Israel; \bibverse{6} ja freilich nicht zu großen
Völkern, die fremde Rede und unbekannte Sprache haben, welcher Worte du
nicht vernehmen könntest. Und wenn ich dich gleich zu denselbigen
sendete, würden sie dich doch gerne hören. \bibverse{7} Aber das Haus
Israel will dich nicht hören, denn sie wollen mich selbst nicht hören,
denn das ganze Haus Israel hat harte Stirnen und verstockte Herzen.
\bibverse{8} Aber doch habe ich dein Angesicht hart gemacht gegen ihr
Angesicht und deine Stirn gegen ihre Stirn. \bibverse{9} Ja, ich habe
deine Stirn so hart als einen Demant, der härter ist denn ein Fels,
gemacht. Darum fürchte dich nicht, entsetze dich auch nicht vor ihnen,
daß sie so ein ungehorsam Haus sind. \bibverse{10} Und er sprach zu mir:
Du Menschenkind, alle meine Worte, die ich dir sage, die fasse mit
Herzen und nimm sie zu Ohren. \bibverse{11} Und gehe hin zu den
Gefangenen deines Volks und predige ihnen und sprich zu ihnen: So
spricht der HErr HErr; sie hören's oder lassen's. \bibverse{12} Und ein
Wind hub mich auf, und ich hörete hinter mir ein Getön wie eines großen
Erdbebens: Gelobet sei die Herrlichkeit des HErrn an ihrem Ort!
\bibverse{13} Und war ein Rauschen von den Flügeln der Tiere, die sich
aneinander küsseten, und auch das Rasseln der Räder, so hart bei ihnen
waren, und das Getön eines großen Erdbebens. \bibverse{14} Da hub mich
der Wind auf und führete mich weg. Und ich fuhr dahin und erschrak sehr;
aber des HErrn Hand hielt mich fest. \bibverse{15} Und ich kam zu den
Gefangenen, die am Wasser Chebar wohneten, da die Mandeln stunden, im
Monden Abib, und setzte mich zu ihnen, die da saßen, und blieb daselbst
unter ihnen sieben Tage ganz traurig. \bibverse{16} Und da die sieben
Tage um waren, geschah des HErrn Wort zu mir und sprach: \bibverse{17}
Du Menschenkind, ich habe dich zum Wächter gesetzt über das Haus Israel;
du sollst aus meinem Munde das Wort hören und sie von meinetwegen
warnen. \bibverse{18} Wenn ich dem Gottlosen sage: Du mußt des Todes
sterben, und du warnest ihn nicht und sagst es ihm nicht, damit sich der
Gottlose vor seinem gottlosen Wesen hüte, auf daß er lebendig bleibe, so
wird der Gottlose um seiner Sünde willen sterben, aber sein Blut will
ich von deiner Hand fordern. \bibverse{19} Wo du aber den Gottlosen
warnest, und er sich nicht bekehret von seinem gottlosen Wesen und Wege,
so wird er um seiner Sünde willen sterben, aber du hast deine Seele
errettet. \bibverse{20} Und wenn sich ein Gerechter von seiner
Gerechtigkeit wendet und tut Böses, so werde ich ihn lassen anlaufen,
daß er muß sterben. Denn weil du ihn nicht gewarnet hast, wird er um
seiner Sünde willen sterben müssen, und seine Gerechtigkeit, die er
getan hat, wird nicht angesehen werden; aber sein Blut will ich von
deiner Hand fordern. \bibverse{21} Wo du aber den Gerechten warnest, daß
er nicht sündigen soll, und er sündiget auch nicht, so soll er leben,
denn er hat sich warnen lassen; und du hast deine Seele errettet.
\bibverse{22} Und daselbst kam des HErrn Hand über mich und sprach zu
mir: Mache dich auf und gehe hinaus ins Feld; da will ich mit dir reden.
\bibverse{23} Und ich machte mich auf und ging hinaus ins Feld; und
siehe, da stund die Herrlichkeit des HErrn daselbst, gleichwie ich sie
am Wasser Chebar gesehen hatte. Und ich fiel nieder auf mein Angesicht.
\bibverse{24} Und ich ward erquicket und trat auf meine Füße. Und er
redete mit mir und sprach zu mir: Gehe hin und verschleuß dich in deinem
Hause. \bibverse{25} Und du, Menschenkind, siehe, man wird dir Stricke
anlegen und dich damit binden, daß du ihnen nicht entgehen sollst.
\bibverse{26} Und ich will dir die Zunge an deinem Gaumen kleben lassen,
daß du erstummen sollst und nicht mehr sie strafen mögest; denn es ist
ein ungehorsam Haus. \bibverse{27} Wenn ich aber mit dir reden werde,
will ich dir den Mund auftun, daß du zu ihnen sagen sollst: So spricht
der HErr HErr! Wer es höret, der höre es; wer es läßt, der lasse es;
denn es ist ein ungehorsam Haus.

\hypertarget{section-3}{%
\section{4}\label{section-3}}

\bibverse{1} Und du, Menschenkind, nimm einen Ziegel, den lege vor dich
und entwirf darauf die Stadt Jerusalem. \bibverse{2} Und mache eine
Belagerung darum und baue ein Bollwerk darum und grabe einen Schutt
darum und mache ein Heer darum und stelle Böcke rings um sie her.
\bibverse{3} Für dich aber nimm eine eiserne Pfanne, die laß eine
eiserne Mauer sein zwischen dir und der Stadt; und richte dein Angesicht
gegen sie und belagere sie. Das sei ein Zeichen dem Hause Israel.
\bibverse{4} Du sollst dich auch auf deine linke Seite legen und die
Missetat des Hauses Israel auf dieselbige legen; so viel Tage du darauf
liegest, so lange sollst du auch ihre Missetat tragen. \bibverse{5} Ich
will dir aber die Jahre ihrer Missetat zur Anzahl der Tage machen,
nämlich dreihundertundneunzig Tage; so lange sollst du die Missetat des
Hauses Israel tragen. \bibverse{6} Und wenn du solches ausgerichtet
hast, sollst du danach dich auf deine rechte Seite legen und sollst
tragen die Missetat des Hauses Juda vierzig Tage lang, denn ich gebe dir
hie auch je einen Tag für ein Jahr. \bibverse{7} Und richte dein
Angesicht und deinen bloßen Arm wider das belagerte Jerusalem und
weissage wider sie. \bibverse{8} Und siehe, ich will dir Stricke
anlegen, daß du dich nicht wenden mögest von einer Seite zur andern, bis
du die Tage deiner Belagerung vollendet hast. \bibverse{9} So nimm nun
zu dir Weizen, Gerste, Bohnen, Linsen, Hirse und Spelt und tu es alles
in ein Faß und mache dir so viel Brote daraus, so viele Tage du auf
deiner Seite liegest, daß du dreihundertundneunzig Tage daran zu essen
habest, \bibverse{10} also daß deine Speise, die du täglich essen mußt,
sei zwanzig Sekel schwer. Solches sollst du von einer Zeit zur andern
essen. \bibverse{11} Das Wasser sollst du auch nach dem Maß trinken,
nämlich das sechste Teil vom Hin; und sollst solches auch von einer Zeit
zur andern trinken. \bibverse{12} Gerstenkuchen sollst du essen, die du
vor ihren Augen mit Menschenmist backen sollst. \bibverse{13} Und der
HErr sprach: Also müssen die Kinder Israel ihr unrein Brot essen unter
den Heiden, dahin ich sie verstoßen habe. \bibverse{14} Ich aber sprach:
Ach, HErr HErr, siehe, meine Seele ist noch nie unrein worden; denn ich
habe von meiner Jugend auf bis auf diese Zeit kein Aas noch Zerrissenes
gegessen, und ist nie kein unrein Fleisch in meinen Mund kommen.
\bibverse{15} Er aber sprach zu mir: Siehe, ich will dir Kuhmist für
Menschenmist zulassen, damit du dein Brot machen sollst. \bibverse{16}
Und sprach zu mir: Du Menschenkind, siehe, ich will den Vorrat des Brots
zu Jerusalem wegnehmen, daß sie das Brot essen müssen nach dem Gewicht
und mit Kummer und das Wasser nach dem Maß mit Kummer trinken,
\bibverse{17} darum daß an Brot und Wasser mangeln wird, und einer mit
dem andern trauern und in ihrer Missetat verschmachten sollen.

\hypertarget{section-4}{%
\section{5}\label{section-4}}

\bibverse{1} Und du, Menschenkind, nimm ein Schwert, scharf wie ein
Schermesser, und fahre damit über dein Haupt und Bart und nimm eine
Waage und teile sie damit. \bibverse{2} Das eine dritte Teil sollst du
mit Feuer verbrennen mitten in der Stadt, wenn die Tage der Belagerung
um sind; das andere dritte Teil nimm und schlage es mit dem Schwert
ringsumher; das letzte dritte Teil streue in den Wind, daß ich das
Schwert hinter ihnen her ausziehe. \bibverse{3} Nimm aber ein klein
wenig davon und binde es in deinen Mantelzipfel. \bibverse{4} Und nimm
wiederum etliches davon und wirf's in ein Feuer und verbrenne es mit
Feuer; von dem soll ein Feuer auskommen über das ganze Haus Israel.
\bibverse{5} So spricht der HErr HErr: Das ist Jerusalem, die ich unter
die Heiden gesetzt habe, und rings um sie her Länder. \bibverse{6} Sie
aber hat mein Gesetz verwandelt in gottlose Lehre mehr denn die Heiden
und meine Rechte mehr denn die Länder, so rings um sie her liegen. Denn
sie verwerfen mein Gesetz und wollen nicht nach meinen Rechten leben.
\bibverse{7} Darum spricht der HErr HErr also: Weil ihr's mehr machet
denn die Heiden, so um euch her sind, und nach meinen Geboten nicht
lebet und nach meinen Rechten nicht tut, sondern nach der Heiden Weise
tut, die um euch her sind, \bibverse{8} so spricht der HErr HErr also:
Siehe, ich will auch an dich und will Recht über dich gehen lassen, daß
die Heiden zusehen sollen. \bibverse{9} Und will also mit dir umgehen,
als ich nie getan und hinfort nicht tun werde, um aller deiner Greuel
willen, \bibverse{10} daß in dir die Väter ihre Kinder und die Kinder
ihre Väter fressen sollen; und will solch Recht über dich gehen lassen,
daß alle deine Übrigen sollen in alle Winde zerstreuet werden.
\bibverse{11} Darum, so wahr als ich lebe, spricht der HErr HErr, weil
du mein Heiligtum mit deinen allerlei Greueln und Götzen verunreiniget
hast, will ich dich auch zerschlagen, und mein Auge soll dein nicht
schonen und will nicht gnädig sein. \bibverse{12} Es soll das dritte
Teil von dir an der Pestilenz sterben und durch Hunger alle werden, und
das andere dritte Teil durchs Schwert fallen rings um dich her, und das
letzte dritte Teil will ich in alle Winde zerstreuen und das Schwert
hinter ihnen her ausziehen. \bibverse{13} Also soll mein Zorn vollendet
und mein Grimm über ihnen ausgerichtet werden, daß ich meinen Mut kühle;
und sie sollen erfahren, daß ich, der HErr, in meinem Eifer geredet
habe, wenn ich meinen Grimm an ihnen ausgerichtet habe. \bibverse{14}
Ich will dich zur Wüste und zur Schmach setzen vor den Heiden, so um
dich her sind, vor den Augen aller, die vorübergehen. \bibverse{15} Und
sollst eine Schmach, Hohn, Exempel und Wunder sein allen Heiden, die um
dich her sind, wenn ich über dich das Recht gehen lasse mit Zorn, Grimm
und zornigem Schelten (das sage ich, der HErr), \bibverse{16} und wenn
ich böse Pfeile des Hungers unter sie schießen werde, die da schädlich
sein sollen, und ich sie ausschießen werde, euch zu verderben, und den
Hunger über euch immer größer werden lasse und den Vorrat des Brots
wegnehme. \bibverse{17} Ja, Hunger und böse wilde Tiere will ich unter
euch schicken, die sollen euch ohne Kinder machen; und soll Pestilenz
und Flut unter dir umgehen, und will das Schwert über dich bringen. Ich,
der HErr, habe es gesagt.

\hypertarget{section-5}{%
\section{6}\label{section-5}}

\bibverse{1} Und des HErrn Wort geschah zu mir und sprach: \bibverse{2}
Du Menschenkind, kehre dein Angesicht wider die Berge Israels und
weissage wider sie \bibverse{3} und sprich: Ihr Berge Israels, höret das
Wort des HErrn HErrn! So spricht der HErr HErr beide, zu den Bergen und
Hügeln, beide, zu den Bächen und Tälern: Siehe, ich will das Schwert
über euch bringen und eure Höhen umbringen, \bibverse{4} daß eure Altäre
verwüstet und eure Götzen zerbrochen sollen werden; und will eure
Leichname vor den Bildern totschlagen lassen. \bibverse{5} Ja, ich will
die Leichname der Kinder Israel vor euren Bildern fällen und will eure
Gebeine um eure Altäre her zerstreuen. \bibverse{6} Wo ihr wohnet, da
sollen die Städte wüste und die Höhen zur Einöde werden. Denn man wird
eure Altäre wüste und zur Einöde machen und eure Götzen zerbrechen und
zunichte machen und eure Bilder zerschlagen und eure Stifte vertilgen,
\bibverse{7} und sollen Erschlagene unter euch daliegen, daß ihr
erfahret, ich sei der HErr. \bibverse{8} Ich will aber etliche von euch
überbleiben lassen, die dem Schwert entgehen unter den Heiden, wenn ich
euch in die Länder zerstreuet habe. \bibverse{9} Dieselbigen eure
Übrigen werden dann an mich gedenken unter den Heiden, da sie gefangen
sein müssen, wenn ich ihr hurisch Herz, so von mir gewichen, und ihre
hurischen Augen, so nach ihren Götzen gesehen, zerschlagen habe; und
wird sie gereuen die Bosheit, die sie durch allerlei ihre Greuel
begangen haben. \bibverse{10} Und sollen erfahren, daß ich der HErr sei
und nicht umsonst geredet habe, solches Unglück ihnen zu tun.
\bibverse{11} So spricht der HErr HErr: Schlage deine Hände zusammen und
strampel mit deinen Füßen und sprich: Wehe über alle Greuel der Bosheit
im Hause Israels, darum sie durch Schwert, Hunger und Pestilenz fallen
müssen! \bibverse{12} Wer ferne ist, wird an der Pestilenz sterben, und
wer nahe ist, wird durchs Schwert fallen; wer aber überbleibet und davor
behütet ist, wird Hungers sterben. Also will ich meinen Grimm unter
ihnen vollenden, \bibverse{13} daß ihr erfahren sollt, ich sei der HErr,
wenn ihre Erschlagenen unter ihren Götzen liegen werden um ihre Altäre
her, oben auf allen Hügeln und oben auf allen Bergen und unter allen
grünen Bäumen und unter allen dicken Eichen, an welchen Orten sie
allerlei Götzen süßes Räuchopfer taten. \bibverse{14} Ich will meine
Hand wider sie ausstrecken und das Land wüste und öde machen, von der
Wüste an bis gen Diblath, wo sie wohnen, und sollen erfahren, daß ich
der HErr sei.

\hypertarget{section-6}{%
\section{7}\label{section-6}}

\bibverse{1} Und des HErrn Wort geschah zu mir und sprach: \bibverse{2}
Du Menschenkind, so spricht der HErr HErr vom Lande Israel: Das Ende
kommt, das Ende über alle vier Örter des Landes. \bibverse{3} Nun kommt
das Ende über dich; denn ich will meinen Grimm über dich senden und will
dich richten, wie du verdienet hast, und will dir geben, was allen
deinen Greueln gebührt. \bibverse{4} Mein Auge soll dein nicht schonen,
noch übersehen, sondern ich will dir geben, wie du verdienet hast und
deine Greuel sollen unter dich kommen, daß ihr erfahren sollt, ich sei
der HErr. \bibverse{5} So spricht der HErr HErr: Siehe, es kommt ein
Unglück über das andere! \bibverse{6} Das Ende kommt, es kommt das Ende,
es ist erwacht über dich; siehe, es kommt! \bibverse{7} Es gehet schon
auf und bricht daher über dich, du Einwohner des Landes; die Zeit kommt,
der Tag des Jammers ist nahe, da kein Singen auf den Bergen sein wird.
\bibverse{8} Nun will ich bald meinen Grimm über dich schütten und
meinen Zorn an dir vollenden; und will dich richten, wie du verdienet
hast, und dir geben, was deinen Greueln allen gebührt. \bibverse{9} Mein
Auge soll dein nicht schonen und will nicht gnädig sein, sondern ich
will dir geben, wie du verdienet hast, und deine Greuel sollen unter
dich kommen, daß ihr erfahren sollt, ich sei der HErr, der euch schlägt.
\bibverse{10} Siehe, der Tag, siehe, er kommt daher, er bricht an! Die
Rute blühet, und der Stolze grünet. \bibverse{11} Der Tyrann hat sich
aufgemacht zur Rute über die Gottlosen, daß nichts von ihnen, noch von
ihrem Volk, noch von ihrem Haufen Trost haben wird. \bibverse{12} Darum
kommt die Zeit, der Tag nahet herzu. Der Käufer freue sich nicht, und
der Verkäufer traure nicht; denn es kommt der Zorn über all ihren
Haufen. \bibverse{13} Darum soll der Verkäufer zu seinem verkauften Gut
nicht wieder trachten; denn wer da lebet, der wird's haben. Denn die
Weissagung über all ihren Haufen wird nicht zurückkehren; keiner wird
sein Leben erhalten um seiner Missetat willen. \bibverse{14} Laßt sie
die Posaune nur blasen und alles zurüsten! Es wird doch niemand in den
Krieg ziehen; denn mein Grimm gehet über all ihren Haufen. \bibverse{15}
Auf den Gassen gehet das Schwert, in den Häusern gehet Pestilenz und
Hunger. Wer auf dem Felde ist, der wird vom Schwert sterben; wer aber in
der Stadt ist, den wird die Pestilenz und Hunger fressen. \bibverse{16}
Und welche unter ihnen entrinnen, die müssen auf den Gebirgen sein und
wie die Tauben in Gründen, die alle untereinander girren, ein jeglicher
um seiner Missetat willen. \bibverse{17} Aller Hände werden dahinsinken
und aller Kniee werden so ungewiß stehen wie Wasser. \bibverse{18} Und
werden Säcke um sich gürten und mit Furcht überschüttet sein, und aller
Angesicht jämmerlich sehen, und aller Häupter werden kahl sein.
\bibverse{19} Sie werden ihr Silber hinauf auf die Gassen werfen und ihr
Gold als einen Unflat achten; denn ihr Silber und Gold wird sie nicht
erretten am Tage des Zorns des HErrn. Und werden doch ihre Seele davon
nicht sättigen noch ihren Bauch davon füllen; denn es ist ihnen gewesen
ein Ärgernis zu ihrer Missetat. \bibverse{20} Sie haben aus ihren edlen
Kleinoden, damit sie Hoffart trieben, Bilder ihrer Greuel und Scheuel
gemacht; darum will ich's ihnen zum Unflat machen \bibverse{21} und
will's Fremden in die Hände geben, daß sie es rauben, und den Gottlosen
auf Erden zur Ausbeute, daß sie es entheiligen sollen. \bibverse{22} Ich
will mein Angesicht davon kehren, daß sie meinen Schatz ja wohl
entheiligen; ja, Räuber sollen darüber kommen und es entheiligen.
\bibverse{23} Mache Ketten; denn das Land ist voll Blutschulden und die
Stadt voll Frevels. \bibverse{24} So will ich die Ärgsten unter den
Heiden kommen lassen, daß sie sollen ihre Häuser einnehmen, und will der
Gewaltigen Hoffart ein Ende machen und ihre Kirchen entheiligen.
\bibverse{25} Der Ausrotter kommt; da werden sie Frieden suchen, und
wird nicht da sein. \bibverse{26} Ein Unfall wird über den andern
kommen, ein Gerücht über das andere. So werden sie dann ein Gesicht bei
den Propheten suchen; aber es wird weder Gesetz bei den Priestern noch
Rat bei den Alten mehr sein. \bibverse{27} Der König wird betrübt sein,
und die Fürsten werden traurig gekleidet sein, und die Hände des Volks
im Lande werden verzagt sein. Ich will mit ihnen umgehen, wie sie
gelebet haben, und will sie richten, wie sie es verdienet haben, daß sie
erfahren sollen, ich sei der HErr.

\hypertarget{section-7}{%
\section{8}\label{section-7}}

\bibverse{1} Und es begab sich im sechsten Jahr, am fünften Tage des
sechsten Monden, daß ich saß in meinem Hause, und die Alten aus Juda
saßen vor mir; daselbst fiel die Hand des HErrn HErrn auf mich.
\bibverse{2} Und siehe, ich sah, daß von seinen Lenden herunterwärts war
gleich wie Feuer; aber oben über seinen Lenden war es lichthelle.
\bibverse{3} Und reckte aus gleichwie eine Hand und ergriff mich bei dem
Haar meines Haupts. Da führete mich ein Wind zwischen Himmel und Erde
und brachte mich gen Jerusalem in einem göttlichen Gesichte zu dem
innern Tor, das gegen Mitternacht stehet, da denn saß ein Bild zu
Verdrieß dem Hausherrn. \bibverse{4} Und siehe, da war die Herrlichkeit
des GOttes Israels, wie ich sie zuvor gesehen hatte im Felde.
\bibverse{5} Und er sprach zu mir: Du Menschenkind, hebe deine Augen auf
gegen Mitternacht! Und da ich meine Augen aufhub gegen Mitternacht,
siehe, da saß gegen Mitternacht das verdrießliche Bild am Tor des
Altars, eben da man hineingehet. \bibverse{6} Und er sprach zu mir: Du
Menschenkind, siehest du auch, was diese tun, nämlich große Greuel, die
das Haus Israel hie tut, daß sie mich ja ferne von meinem Heiligtum
treiben? Aber du wirst noch mehr größere Greuel sehen. \bibverse{7} Und
er führete mich zur Tür des Vorhofs; da sah ich, und siehe, da war ein
Loch in der Wand. \bibverse{8} Und er sprach zu mir: Du Menschenkind,
grabe durch die Wand! Und da ich durch die Wand grub, siehe, da war eine
Tür. \bibverse{9} Und er sprach zu mir: Gehe hinein und schaue die bösen
Greuel, die sie allhie tun. \bibverse{10} Und da ich hineinkam und sah,
siehe, da waren allerlei Bildnisse der Würmer und Tiere, eitel Scheuel
und allerlei Götzen des Hauses Israel, allenthalben umher an der Wand
gemacht, \bibverse{11} vor welchen stunden siebenzig Männer aus den
Ältesten des Hauses Israel; und Jasanja, der Sohn Saphans, stund auch
unter ihnen; und ein jeglicher hatte sein Räuchwerk in der Hand. Und
ging ein dicker Nebel auf vom Räuchwerk. \bibverse{12} Und er sprach zu
mir: Menschenkind, siehest du, was die Ältesten des Hauses Israel tun in
der Finsternis, ein jeglicher in seiner schönsten Kammer? Denn sie
sagen: Der HErr siehet uns nicht, sondern der HErr hat das Land
verlassen. \bibverse{13} Und er sprach zu mir: Du sollst noch mehr
größere Greuel sehen, die sie tun. \bibverse{14} Und er führete mich
hinein zum Tor an des HErrn Hause, das gegen Mitternacht stehet; und
siehe, daselbst saßen Weiber, die weineten über den Thamus.
\bibverse{15} Und er sprach zu mir: Menschenkind, siehest du das? Aber
du sollst noch größere Greuel sehen, denn diese sind. \bibverse{16} Und
er führete mich in den innern Hof am Hause des HErrn; und siehe, vor der
Tür am Tempel des HErrn, zwischen der Halle und dem Altar, da waren bei
fünfundzwanzig Männer, die ihren Rücken gegen den Tempel des HErrn und
ihr Angesicht gegen den Morgen gekehret hatten, und beteten gegen der
Sonnen Aufgang. \bibverse{17} Und er sprach zu mir: Menschenkind,
siehest du das? Ist's dem Hause Juda zu wenig, daß sie alle solche
Greuel hie tun? so sie doch sonst im ganzen Lande eitel Gewalt und
Unrecht treiben und fahren zu und reizen mich auch; und siehe, sie
halten die Weinreben an die Nasen. \bibverse{18} Darum will ich auch
wider sie mit Grimm handeln, und mein Auge soll ihrer nicht verschonen,
und will nicht gnädig sein. Und wenn sie gleich mit lauter Stimme vor
meinen Ohren schreien, will ich sie doch nicht hören.

\hypertarget{section-8}{%
\section{9}\label{section-8}}

\bibverse{1} Und er rief mit lauter Stimme vor meinen Ohren und sprach:
Laßt herzukommen die Heimsuchung der Stadt, und ein jeglicher habe eine
mördliche Waffe in seiner Hand. \bibverse{2} Und siehe, es kamen sechs
Männer auf dem Wege vom Obertor her, das gegen Mitternacht stehet, und
ein jeglicher hatte eine schädliche Waffe in seiner Hand. Aber es war
einer unter ihnen, der hatte Leinwand an und ein Schreibzeug an seiner
Seite. Und sie gingen hinein und traten neben den ehernen Altar.
\bibverse{3} Und die Herrlichkeit des GOttes Israels erhub sich von dem
Cherub, über dem sie war, zu der Schwelle am Hause und rief dem, der die
Leinwand anhatte und das Schreibzeug an seiner Seite. \bibverse{4} Und
der HErr sprach zu ihm: Gehe durch die Stadt Jerusalem und zeichne mit
einem Zeichen an die Stirn die Leute, so da seufzen und jammern über
alle Greuel, so drinnen geschehen. \bibverse{5} Zu jenen aber sprach er,
daß ich's hörete: Gehet diesem nach durch die Stadt und schlaget drein;
eure Augen sollen nicht schonen noch übersehen. \bibverse{6} Erwürget
beide, Alte, Jünglinge, Jungfrauen, Kinder und Weiber, alles tot; aber
die das Zeichen an sich haben, derer sollt ihr keinen anrühren. Fanget
aber an an meinem Heiligtum! Und sie fingen an an den alten Leuten, so
vor dem Hause waren. \bibverse{7} Und er sprach zu ihnen: Verunreiniget
das Haus und machet die Vorhöfe voll toter Leichname; gehet heraus! Und
sie gingen heraus und schlugen in der Stadt. \bibverse{8} Und da sie
ausgeschlagen hatten, war ich noch übrig. Und ich fiel auf mein
Angesicht, schrie und sprach: Ach, HErr HErr, willst du denn alle
Übrigen in Israel verderben, daß du deinen Zorn so ausschüttest über
Jerusalem? \bibverse{9} Und er sprach zu mir: Es ist die Missetat des
Hauses Israel und Juda allzusehr groß; es ist eitel Gewalt im Lande und
Unrecht in der Stadt. Denn sie sprechen: Der HErr hat das Land verlassen
und der HErr siehet uns nicht. \bibverse{10} Darum soll mein Auge auch
nicht schonen, will auch nicht gnädig sein, sondern ich will ihr Tun auf
ihren Kopf werfen. \bibverse{11} Und siehe, der Mann, der die Leinwand
anhatte und das Schreibzeug an seiner Seite, antwortete und sprach: Ich
habe getan, wie du mir geboten hast.

\hypertarget{section-9}{%
\section{10}\label{section-9}}

\bibverse{1} Und ich sah, und siehe, am Himmel über dem Haupt der
Cherubim war es gestaltet wie ein Saphir, und über denselbigen war es
gleich anzusehen wie ein Thron. \bibverse{2} Und er sprach zu dem Manne
in Leinwand: Gehe hinein zwischen die Räder unter den Cherub und fasse
die Hände voll glühender Kohlen, so zwischen den Cherubim sind, und
streue sie über die Stadt. Und er ging hinein, daß ich's sah, da
derselbige hineinging. \bibverse{3} Die Cherubim aber stunden zur
Rechten am Hause, und der Vorhof ward inwendig voll Nebels. \bibverse{4}
Und die Herrlichkeit des HErrn erhub sich von dem Cherub zur Schwelle am
Hause; und das Haus ward voll Nebels und der Vorhof voll Glanzes von der
Herrlichkeit des HErrn. \bibverse{5} Und man hörete die Flügel der
Cherubim rauschen bis heraus vor den Vorhof, wie eine Stimme des
allmächtigen GOttes, wenn er redet. \bibverse{6} Und da er dem Manne in
Leinwand geboten hatte und gesagt: Nimm Feuer zwischen den Rädern unter
den Cherubim, ging derselbige hinein und trat neben das Rad.
\bibverse{7} Und der Cherub streckte seine Hand heraus zwischen den
Cherubim zum Feuer, das zwischen den Cherubim war, nahm davon und gab's
dem Manne in Leinwand in die Hände; der empfing's und ging hinaus.
\bibverse{8} Und erschien an den Cherubim gleichwie eines Menschen Hand
unter ihren Flügeln. \bibverse{9} Und ich sah, und siehe, vier Räder
stunden bei den Cherubim, bei einem jeglichen Cherub ein Rad; und die
Räder waren anzusehen gleichwie ein Türkis. \bibverse{10} Und waren alle
vier eins wie das andere, als wäre ein Rad im andern. \bibverse{11} Wenn
sie gehen sollten, so konnten sie in alle ihre vier Örter gehen und
durften sich nicht herumlenken, wenn sie gingen, sondern wohin das erste
ging, da gingen sie hinnach, und durften sich nicht herumlenken,
\bibverse{12} samt ihrem ganzen Leibe, Rücken, Händen und Flügeln. Und
die Räder waren voll Augen um und um an allen vier Rädern. \bibverse{13}
Und es rief zu den Rädern: Galgal! daß ich's hörete. \bibverse{14} Ein
jegliches hatte vier Angesichte. Das erste Angesicht war ein Cherub, das
andere ein Mensch, das dritte ein Löwe, das vierte ein Adler.
\bibverse{15} Und die Cherubim schwebten empor. Es ist eben das Tier,
das ich sah am Wasser Chebar. \bibverse{16} Wenn die Cherubim gingen, so
gingen die Räder auch neben ihnen; und wenn die Cherubim ihre Flügel
schwangen, daß sie sich von der Erde erhuben, so lenkten sich die Räder
auch nicht von ihnen. \bibverse{17} Wenn jene stunden, so stunden diese
auch; erhuben sie sich, so erhuben sich diese auch: denn es war ein
lebendiger Wind in ihnen. \bibverse{18} Und die Herrlichkeit des HErrn
ging wieder aus von der Schwelle am Hause und stellete sich über die
Cherubim. \bibverse{19} Da schwangen die Cherubim ihre Flügel und
erhuben sich von der Erde vor meinen Augen; und da sie ausgingen, gingen
die Räder neben ihnen. Und sie traten in das Tor am Hause des HErrn
gegen Morgen, und die Herrlichkeit des GOttes Israels war oben über
ihnen. \bibverse{20} Das ist das Tier, das ich unter dem GOtt Israels
sah am Wasser Chebar, und merkte, daß es Cherubim wären, \bibverse{21}
da ein jegliches vier Angesichte hatte und vier Flügel und unter den
Flügeln gleichwie Menschenhände. \bibverse{22} Es waren ihre Angesichte
gestaltet, wie ich sie am Wasser Chebar sah, und gingen stracks vor
sich.

\hypertarget{section-10}{%
\section{11}\label{section-10}}

\bibverse{1} Und mich hub ein Wind auf und brachte mich zum Tor am Hause
des HErrn, das gegen Morgen stehet; und siehe, unter dem Tor waren
fünfundzwanzig Männer. Und ich sah unter ihnen Jasanja, den Sohn Assurs,
und Platja, den Sohn Benajas, die Fürsten im Volk. \bibverse{2} Und er
sprach zu mir: Menschenkind diese Leute haben unselige Gedanken und
schädliche Ratschläge in dieser Stadt. \bibverse{3} Denn sie sprechen:
Es ist nicht so nahe, laßt uns nur Häuser bauen; sie ist der Topf, so
sind wir das Fleisch. \bibverse{4} Darum sollst du, Menschenkind, wider
sie weissagen. \bibverse{5} Und der Geist des HErrn fiel auf mich und
sprach zu mir: Sprich: So sagt der HErr: Ihr habt also geredet, ihr vom
Hause Israel; und eures Geistes Gedanken kenne ich wohl. \bibverse{6}
Ihr habt viele erschlagen in dieser Stadt, und ihre Gassen liegen voller
Toten. \bibverse{7} Darum spricht der HErr HErr also: Die ihr drinnen
getötet habt, die sind das Fleisch, und sie ist der Topf; aber ihr
müsset hinaus. \bibverse{8} Das Schwert, das ihr fürchtet, das will ich
über euch kommen lassen, spricht der HErr HErr. \bibverse{9} Ich will
euch von dannen herausstoßen und den Fremden in die Hand geben und will
euch euer Recht tun. \bibverse{10} Ihr sollt durchs Schwert fallen, in
den Grenzen Israels will ich euch richten; und sollt erfahren, daß ich
der HErr bin. \bibverse{11} Die Stadt aber soll nicht euer Topf sein,
noch ihr das Fleisch drinnen, sondern in den Grenzen Israels will ich
euch richten. \bibverse{12} Und sollt erfahren, daß ich der HErr bin;
denn ihr habt nach meinen Geboten nicht gewandelt und meine Rechte nicht
gehalten, sondern getan nach der Heiden Weise, die um euch her sind.
\bibverse{13} Und da ich so weissagte, starb Platja, der Sohn Benajas.
Da fiel ich auf mein Angesicht und schrie mit lauter Stimme und sprach:
Ach, HErr HErr, du wirst's mit den Übrigen Israels gar ausmachen!
\bibverse{14} Da geschah des HErrn Wort zu mir und sprach: \bibverse{15}
Du Menschenkind, deine Brüder und nahen Freunde und das ganze Haus
Israel, so noch zu Jerusalem wohnen, sprechen wohl untereinander; jene
sind vom HErrn ferne weggeflohen, aber wir haben das Land inne.
\bibverse{16} Darum sprich du: So spricht der HErr HErr: Ja, ich habe
sie ferne weg unter die Heiden lassen treiben und in die Länder
zerstreuet; doch will ich bald ihr Heiland sein in den Ländern, dahin
sie kommen sind. \bibverse{17} Darum sprich: So sagt der HErr HErr: Ich
will euch sammeln aus den Völkern und will euch sammeln aus den Ländern,
dahin ihr zerstreuet seid, und will euch das Land Israel geben.
\bibverse{18} Da sollen sie kommen und alle Scheuel und Greuel daraus
wegtun. \bibverse{19} Ich will euch ein einträchtig Herz geben und einen
neuen Geist in euch geben; und will das steinerne Herz wegnehmen aus
eurem Leibe und ein fleischern Herz geben, \bibverse{20} auf daß sie in
meinen Sitten wandeln und meine Rechte halten und danach tun. Und sie
sollen mein Volk sein, so will ich ihr GOtt sein. \bibverse{21} Denen
aber, so nach ihres Herzens Scheueln und Greueln wandeln, will ich ihr
Tun auf ihren Kopf werfen, spricht der HErr HErr. \bibverse{22} Da
schwangen die Cherubim ihre Flügel, und die Räder gingen neben ihnen,
und die Herrlichkeit des GOttes Israels war oben über ihnen.
\bibverse{23} Und die Herrlichkeit des HErrn erhub sich aus der Stadt
und stellete sich auf den Berg, der gegen Morgen vor der Stadt liegt.
\bibverse{24} Und ein Wind hub mich auf und brachte mich im Gesicht und
im Geist GOttes nach Chaldäa zu den Gefangenen. Und das Gesicht, so ich
gesehen hatte, verschwand vor mir. \bibverse{25} Und ich sagte den
Gefangenen alle Worte des HErrn, die er mir gezeiget hatte.

\hypertarget{section-11}{%
\section{12}\label{section-11}}

\bibverse{1} Und des HErrn Wort geschah zu mir und sprach: \bibverse{2}
Du Menschenkind, du wohnest unter einem ungehorsamen Hause, welches hat
wohl Augen, daß sie sehen könnten, und wollen nicht sehen, Ohren, daß
sie hören könnten, und wollen nicht hören, sondern es ist ein ungehorsam
Haus. \bibverse{3} Darum, du Menschenkind, nimm dein Wandergerät und
zeuch am lichten Tage davon vor ihren Augen. Von deinem Ort sollst du
ziehen an einen andern Ort vor ihren Augen, ob sie vielleicht merken
wollten, daß sie ein ungehorsam Haus sind. \bibverse{4} Und sollst dein
Gerät heraustun, wie Wandergerät, bei lichtem Tage vor ihren Augen; und
du sollst ausziehen des Abends vor ihren Augen, gleichwie man auszieht,
wenn man wandern will. \bibverse{5} Und du sollst durch die Wand brechen
vor ihren Augen und daselbst durch ausziehen. \bibverse{6} Und du sollst
es auf deine Schulter nehmen vor ihren Augen und wenn es dunkel worden
ist, heraustragen; dein Angesicht sollst du verhüllen, daß du das Land
nicht sehest. Denn ich habe dich dem Hause Israel zum Wunderzeichen
gesetzt. \bibverse{7} Und ich tat, wie mir befohlen war, und trug mein
Gerät heraus, wie Wandergerät, bei lichtem Tage; und am Abend brach ich
mit der Hand durch die Wand; und da es dunkel worden war, nahm ich's auf
die Schulter und trug's heraus vor ihren Augen. \bibverse{8} Und
frühmorgens geschah des HErrn Wort zu mir und sprach: \bibverse{9}
Menschenkind, hat das Haus Israel, das ungehorsame Haus, nicht zu dir
gesagt: Was machst du? \bibverse{10} So sprich zu ihnen: So spricht der
HErr HErr: Diese Last betrifft den Fürsten zu Jerusalem und das ganze
Haus Israel, das drinnen ist. \bibverse{11} Sprich: Ich bin euer
Wunderzeichen; wie ich getan habe, also soll euch geschehen, daß ihr
wandern müsset und gefangen geführet werden. \bibverse{12} Ihr Fürst
wird auf der Schulter tragen im Dunkel und muß ausziehen durch die Wand,
so sie brechen werden, daß sie dadurch ausziehen; sein Angesicht wird
verhüllet werden, daß er mit keinem Auge das Land sehe. \bibverse{13}
Ich will auch mein Netz über ihn werfen, daß er in meiner Jagd gefangen
werde, und will ihn gen Babel bringen, in der Chaldäer Land, das er doch
nicht sehen wird, und soll daselbst sterben. \bibverse{14} Und alle, die
um ihn her sind, seine Gehilfen und all seinen Anhang will ich unter
alle Winde zerstreuen und das Schwert hinter ihnen her ausziehen.
\bibverse{15} Also sollen sie erfahren, daß ich der HErr sei, wenn ich
sie unter die Heiden verstoße und in die Länder zerstreue. \bibverse{16}
Aber ich will ihrer etliche wenige überbleiben lassen vor dem Schwert,
Hunger und Pestilenz; die sollen jener Greuel erzählen unter den Heiden,
dahin sie kommen werden, und sollen erfahren, daß ich der HErr sei.
\bibverse{17} Und des HErrn Wort geschah zu mir und sprach:
\bibverse{18} Du Menschenkind, du sollst dein Brot essen mit Beben und
dein Wasser trinken mit Zittern und Sorgen. \bibverse{19} Und sprich zum
Volk im Lande: So spricht der HErr HErr von den Einwohnern zu Jerusalem
im Lande Israel: Sie müssen ihr Brot essen in Sorgen und ihr Wasser
trinken im Elend; denn das Land soll wüst werden von allem, das drinnen
ist, um des Frevels willen aller Einwohner. \bibverse{20} Und die
Städte, so wohl bewohnet sind, sollen verwüstet und das Land öde werden.
Also sollt ihr erfahren, daß ich der HErr sei. \bibverse{21} Und des
HErrn Wort geschah zu mir und sprach: \bibverse{22} Du Menschenkind, was
habt ihr für ein Sprichwort im Lande Israel und sprechet: Weil sich's so
lange verzeucht, so wird nun fort nichts aus der Weissagung?
\bibverse{23} Darum sprich zu ihnen: So spricht der HErr HErr: Ich will
das Sprichwort aufheben, daß man es nicht mehr führen soll in Israel.
Und rede zu ihnen: Die Zeit ist nahe und alles, was geweissaget ist.
\bibverse{24} Denn ihr sollt nun fort inne werden, daß kein Gesicht
fehlen und keine Weissagung lügen wird wider das Haus Israel.
\bibverse{25} Denn ich bin der HErr; was ich rede, das soll geschehen
und nicht länger verzogen werden, sondern bei eurer Zeit, ihr
ungehorsames Haus, will ich tun, was ich rede, spricht der HErr HErr.
\bibverse{26} Und des HErrn Wort geschah zu mir und sprach:
\bibverse{27} Du Menschenkind, siehe, das Haus Israel spricht: Das
Gesicht, das dieser siehet, da ist noch lange hin, und weissaget auf die
Zeit, so noch ferne ist. \bibverse{28} Darum sprich zu ihnen: So spricht
der HErr HErr: Was ich rede, soll nicht länger verzogen werden, sondern
soll geschehen, spricht der HErr HErr.

\hypertarget{section-12}{%
\section{13}\label{section-12}}

\bibverse{1} Und des HErrn Wort geschah zu mir und sprach: \bibverse{2}
Du Menschenkind, weissage wider die Propheten Israels und sprich zu
denen, so aus ihrem eigenen Herzen weissagen: Höret des HErrn Wort!
\bibverse{3} So spricht der HErr HErr: Wehe den tollen Propheten, die
ihrem eigenen Geist folgen und haben doch nicht Gesichte! \bibverse{4} O
Israel, deine Propheten sind wie die Füchse in den Wüsten! \bibverse{5}
Sie treten nicht vor die Lücken und machen sich nicht zur Hürde um das
Haus Israel und stehen nicht im Streit am Tage des HErrn. \bibverse{6}
Ihr Gesicht ist nichts, und ihr Weissagen ist eitel Lügen. Sie sprechen:
Der HErr hat's gesagt, so sie doch der HErr nicht gesandt hat, und mühen
sich, daß sie ihre Dinge erhalten. \bibverse{7} Ist's nicht also, daß
euer Gesicht ist nichts, und euer Weissagen ist eitel Lügen? Und
sprechet doch: Der HErr hat's geredet, so ich's doch nicht geredet habe.
\bibverse{8} Darum spricht der HErr HErr also: Weil ihr das prediget, da
nichts aus wird, und Lügen weissaget, so will ich an euch, spricht der
HErr HErr. \bibverse{9} Und meine Hand soll kommen über die Propheten,
so das predigen, da nichts aus wird, und Lügen weissagen. Sie sollen in
der Versammlung meines Volks nicht sein und in die Zahl des Hauses
Israel nicht geschrieben werden noch ins Land Israel kommen; und ihr
sollt erfahren, daß ich der HErr HErr bin, \bibverse{10} darum daß sie
mein Volk verführen und sagen: Friede! so doch kein Friede ist. Das Volk
bauet die Wand, so tünchen sie dieselbe mit losem Kalk. \bibverse{11}
Sprich zu den Tünchern, die mit losem Kalk tünchen, daß es abfallen
wird; denn es wird ein Platzregen kommen, und werden große Hagel fallen,
die es fällen, und ein Windwirbel wird es zerreißen. \bibverse{12}
Siehe, so wird die Wand einfallen. Was gilt's, dann wird man zu euch
sagen: Wo ist nun, das Getünchte, das ihr getüncht habt? \bibverse{13}
So spricht der HErr HErr: Ich will einen Windwirbel reißen lassen in
meinem Grimm und einen Platzregen in meinem Zorn und große Hagelsteine
im Grimm, die sollen es alles umstoßen. \bibverse{14} Also will ich die
Wand umwerfen, die ihr mit losem Kalk getüncht habt, und will sie zu
Boden stoßen, daß man ihren Grund sehen soll, daß sie da liege; und ihr
sollt drinnen auch umkommen und erfahren, daß ich der HErr sei.
\bibverse{15} Also will ich meinen Grimm vollenden an der Wand und an
denen, die sie mit losem Kalk tünchen und zu euch sagen: Hie ist weder
Wand noch Tüncher. \bibverse{16} Das sind die Propheten Israels, die
Jerusalem weissagen und predigen von Friede, so doch kein Friede ist,
spricht der HErr HErr. \bibverse{17} Und, du Menschenkind, richte dein
Angesicht wider die Töchter in deinem Volk, welche weissagen aus ihrem
Herzen, und weissage wider sie \bibverse{18} und sprich: So spricht der
HErr HErr: Wehe euch, die ihr Kissen machet den Leuten unter die Arme
und Pfühle zu den Häupten, beide, Jungen und Alten, die Seelen zu fahen!
Wenn ihr nun die Seelen gefangen habt unter meinem Volk, verheißet ihr
denselbigen das Leben \bibverse{19} und entheiliget mich in meinem Volk
um einer Hand voll Gerste und Bissen Brots willen, damit daß ihr die
Seelen zum Tode verurteilet, die doch nicht sollten sterben, und
urteilet die zum Leben; die doch nicht leben sollten, durch eure Lügen
unter meinem Volk, welches gerne Lügen höret. \bibverse{20} Darum
spricht der HErr HErr: Siehe, ich will an eure Kissen, damit ihr die
Seelen fahet und vertröstet, und will sie von euren Armen wegreißen und
die Seelen, so ihr fahet und vertröstet, losmachen. \bibverse{21} Und
will eure Pfühle zerreißen und mein Volk aus eurer Hand erretten, daß
ihr sie nicht mehr fahen sollt; und sollt erfahren, daß ich der HErr
sei, \bibverse{22} darum daß ihr das Herz der Gerechten fälschlich
betrübet, die ich nicht betrübet habe, und habt gestärket die Hände der
Gottlosen, daß sie sich von ihrem bösen Wesen nicht bekehren, damit sie
lebendig möchten bleiben. \bibverse{23} Darum sollt ihr nicht mehr
unnütze Lehre predigen noch weissagen, sondern ich will mein Volk aus
euren Händen erretten, und ihr sollt erfahren, daß ich der HErr bin.

\hypertarget{section-13}{%
\section{14}\label{section-13}}

\bibverse{1} Und es kamen etliche von den Ältesten Israels zu mir und
setzten sich vor mir. \bibverse{2} Da geschah des HErrn Wort zu mir und
sprach: \bibverse{3} Menschenkind, diese Leute hangen mit ihrem Herzen
an ihren Götzen und halten ob dem Ärgernis ihrer Missetat. Sollt ich
denn ihnen antworten, wenn sie mich fragen? \bibverse{4} Darum rede mit
ihnen und sage zu ihnen: So spricht der HErr HErr: Welcher Mensch vom
Hause Israel mit dem Herzen an seinen Götzen hanget und hält ob dem
Ärgernis seiner Missetat und kommt zum Propheten, so will ich, der HErr,
demselbigen antworten, wie er verdienet hat mit seiner großen
Abgötterei, \bibverse{5} auf daß das Haus Israel betrogen werde in ihrem
Herzen, darum daß sie alle von mir gewichen sind durch Abgötterei.
\bibverse{6} Darum sollst du zum Hause Israel sagen: So spricht der HErr
HErr: Kehret und wendet euch von eurer Abgötterei und wendet euer
Angesicht von allen euren Greueln! \bibverse{7} Denn welcher Mensch vom
Hause Israel oder Fremdling, so in Israel wohnet, von mir weichet und
mit seinem Herzen an seinen Götzen hanget und ob dem Ärgernis seiner
Abgötterei hält und zum Propheten kommt, daß er durch ihn mich frage,
dem will ich, der HErr, selbst antworten. \bibverse{8} Und will mein
Angesicht wider denselbigen setzen, daß sie sollen wüst und zum Zeichen
und Sprichwort werden; und will sie aus meinem Volk rotten, daß ihr
erfahren sollt, ich sei der HErr. \bibverse{9} Wo aber ein betrogener
Prophet etwas redet, den will ich, der HErr, wiederum lassen betrogen
werden und will meine Hand über ihn ausstrecken und ihn aus meinem Volk
Israel rotten. \bibverse{10} Also sollen sie beide ihre Missetat tragen;
wie die Missetat des Fragers, also soll auch sein die Missetat des
Propheten, \bibverse{11} auf daß sie nicht mehr das Haus Israel
verführen von mir und sich nicht mehr verunreinigen in allerlei ihrer
Übertretung, sondern sie sollen mein Volk sein, und ich will ihr GOtt
sein, spricht der HErr HErr. \bibverse{12} Und des HErrn Wort geschah zu
mir und sprach: \bibverse{13} Du Menschenkind, wenn ein Land an mir
sündiget und dazu mich verschmähet, so will ich meine Hand über dasselbe
ausstrecken und den Vorrat des Brots wegnehmen und will Teurung
hineinschicken, daß ich beide, Menschen und Vieh, drinnen ausrotte.
\bibverse{14} Und wenn dann gleich die drei Männer, Noah, Daniel und
Hiob, drinnen wären, so würden sie allein ihre eigene Seele erretten
durch ihre Gerechtigkeit, spricht der HErr HErr. \bibverse{15} Und wenn
ich böse Tiere in das Land bringen würde, die die Leute aufräumeten und
dasselbige verwüsteten, daß niemand drinnen wandeln könnte vor den
Tieren, \bibverse{16} und diese drei Männer wären auch drinnen: so wahr
ich lebe, spricht der HErr HErr, sie würden weder Söhne noch Töchter
erretten, sondern allein sich selbst, und das Land müßte öde werden.
\bibverse{17} Oder wo ich das Schwert kommen ließe über das Land und
spräche: Schwert, fahre durchs Land und würde also beide, Menschen und
Vieh, ausrotten, \bibverse{18} und die drei Männer wären drinnen: so
wahr ich lebe, spricht der HErr HErr, sie würden weder Söhne noch
Töchter erretten, sondern sie allein würden errettet sein. \bibverse{19}
Oder so ich Pestilenz in das Land schicken und meinen Grimm über
dasselbige ausschütten würde und Blut stürzen, also daß ich beide,
Menschen und Vieh, ausrottete, \bibverse{20} und Noah, Daniel und Hiob
wären drinnen: so wahr ich lebe, spricht der HErr HErr, würden sie weder
Söhne noch Töchter, sondern allein ihre eigene Seele durch ihre
Gerechtigkeit erretten. \bibverse{21} Denn so spricht der HErr HErr: So
ich meine vier bösen Strafen, als Schwert, Hunger, böse Tiere und
Pestilenz, über Jerusalem schicken würde, daß ich drinnen ausrottete
beide, Menschen und Vieh, \bibverse{22} siehe, so sollen etliche Übrige
drinnen davonkommen, die Söhne und Töchter herausbringen werden, und zu
euch anherkommen, daß ihr sehen werdet, wie es ihnen gehet, und euch
trösten über dem Unglück, das ich über Jerusalem habe kommen lassen,
samt allem andern, das ich über sie habe kommen lassen. \bibverse{23}
Sie werden euer Trost sein, wenn ihr sehen werdet, wie es ihnen gehet,
und werdet erfahren, daß ich nicht ohne Ursache getan habe, was ich
drinnen getan habe, spricht der HErr HErr.

\hypertarget{section-14}{%
\section{15}\label{section-14}}

\bibverse{1} Und des HErrn Wort geschah zu mir und sprach: \bibverse{2}
Du Menschenkind, was ist das Holz vom Weinstock vor anderm Holz oder
eine Rebe vor anderm Holz im Walde? \bibverse{3} Nimmt man es auch und
macht etwas daraus? Oder macht man auch einen Nagel daraus, daran man
etwas möge hängen? \bibverse{4} Siehe, man wirft es ins Feuer, daß es
verzehret wird, daß seine beiden Orte das Feuer verzehret und sein
Mittelstes verbrennet. Wozu sollt es nun taugen? Taugt es denn auch zu
etwas? \bibverse{5} Siehe, da es noch ganz war, konnte man nichts daraus
machen; wieviel weniger kann nun fort mehr etwas daraus gemacht werden,
so es das Feuer verzehret und verbrannt hat? \bibverse{6} Darum spricht
der HErr HErr: Gleichwie ich das Holz vom Weinstock vor anderm Holz im
Walde dem Feuer zu verzehren gebe, also will ich mit den Einwohnern zu
Jerusalem auch umgehen \bibverse{7} und will mein Angesicht wider sie
setzen, daß sie dem Feuer nicht entgehen sollen, sondern das Feuer soll
sie fressen. Und ihr sollt es erfahren, daß ich der HErr bin, wenn ich
mein Angesicht wider sie setze \bibverse{8} und das Land wüste mache,
darum daß sie mich verschmähen, spricht der HErr HErr.

\hypertarget{section-15}{%
\section{16}\label{section-15}}

\bibverse{1} Und des HErrn Wort geschah zu mir und sprach: \bibverse{2}
Du Menschenkind, offenbare der Stadt Jerusalem ihre Greuel und sprich:
\bibverse{3} So spricht der HErr HErr zu Jerusalem: Dein Geschlecht und
deine Geburt ist aus der Kanaaniter Lande, dein Vater aus den Amoritern
und deine Mutter aus den Hethitern. \bibverse{4} Deine Geburt ist also
gewesen: Dein Nabel, da du geboren wurdest, ist nicht verschnitten; so
hat man dich auch mit Wasser nicht gebadet, daß du sauber würdest, noch
mit Salz gerieben, noch in Windeln gewickelt. \bibverse{5} Denn niemand
jammerte dein, daß er sich über dich hätte erbarmet und der Stücke eins
dir erzeiget, sondern du wurdest aufs Feld geworfen. Also verachtet war
deine Seele, da du geboren warest. \bibverse{6} Ich aber ging vor dir
über und sah dich in deinem Blut liegen und sprach zu dir, da du so in
deinem Blut lagest: Du sollst leben! Ja, zu dir sprach ich, da du so in
deinem Blut lagest: Du sollst leben! \bibverse{7} Und habe dich erzogen
und lassen groß werden wie ein Gewächs auf dem Felde; und warest nun
gewachsen und groß und schön worden. Deine Brüste waren gewachsen und
hattest schon lange Haare gekriegt; aber du warest noch bloß und
beschamet. \bibverse{8} Und ich ging vor dir über und sah dich an; und
siehe, es war die Zeit, um dich zu werben. Da breitete ich meinen Geren
über dich und bedeckte deine Scham. Und ich gelobte dir's und begab mich
mit dir in einen Bund, spricht der HErr HErr, daß du solltest mein sein.
\bibverse{9} Und ich badete dich mit Wasser und wusch dich von deinem
Blut und salbete dich mit Balsam \bibverse{10} und kleidete dich mit
gestickten Kleidern und zog dir sämische Schuhe an; ich gab dir feine
leinene Kleider und seidene Schleier \bibverse{11} und zierte dich mit
Kleinoden und legte Geschmeide an deine Arme und Kettlein an deinen Hals
\bibverse{12} und gab dir Haarband an deine Stirn und Ohrringe an deine
Ohren und eine schöne Krone auf dein Haupt. \bibverse{13} Summa, du
warest geziert mit eitel Gold und Silber und gekleidet mit eitel
Leinwand, Seiden und Gesticktem. Du aßest auch eitel Semmel, Honig und
Öl und warest überaus schön und bekamest das Königreich. \bibverse{14}
Und dein Ruhm erscholl unter die Heiden deiner Schöne halben, welche
ganz vollkommen war durch den Schmuck, so ich an dich gehänget hatte,
spricht der HErr HErr. \bibverse{15} Aber du verließest dich auf deine
Schöne; und weil du so gerühmet warest, triebest du Hurerei, also daß du
dich einem jeglichen, wer vorüberging, gemein machtest und tatest seinen
Willen. \bibverse{16} Und nahmest von deinen Kleidern und machtest dir
bunte Altäre daraus und triebest deine Hurerei darauf, als nie geschehen
ist noch geschehen wird. \bibverse{17} Du nahmest auch dein schön Gerät,
das ich dir von meinem Gold und Silber gegeben hatte, und machtest dir
Mannsbilder daraus und triebest deine Hurerei mit denselben.
\bibverse{18} Und nahmest deine gestickten Kleider und bedecktest sie
damit; und mein Öl und Räuchwerk legtest du ihnen vor. \bibverse{19}
Meine Speise, die ich dir zu essen gab, Semmel, Öl, Honig, legtest du
ihnen vor zum süßen Geruch. Ja, es kam dahin, spricht der HErr HErr,
\bibverse{20} daß du nahmest deine Söhne und Töchter, die du mir
gezeuget hattest, und opfertest sie denselben zu fressen. Meinest du
denn, daß ein Geringes sei um deine Hurerei, \bibverse{21} daß du mir
meine Kinder schlachtest und lässest sie denselben verbrennen?
\bibverse{22} Noch hast du in allen deinen Greueln und Hurerei nie
gedacht an die Zeit deiner Jugend, wie bloß und nackend du warest und in
deinem Blut lagest. \bibverse{23} Über alle diese deine Bosheit (ach,
wehe, wehe dir!), spricht der HErr HErr, \bibverse{24} bauetest du dir
Bergkirchen und machtest dir Bergaltäre auf allen Gassen. \bibverse{25}
Und vornean auf allen Straßen bauetest du deine Bergaltäre und machtest
deine Schöne zu eitel Greuel. Du gretetest mit deinen Beinen gegen alle,
so vorübergingen, und triebest große Hurerei. \bibverse{26} Erstlich
triebest du Hurerei mit den Kindern Ägyptens, deinen Nachbarn, die groß
Fleisch hatten, und triebest große Hurerei, mich zu reizen.
\bibverse{27} Ich aber streckte meine Hand aus wider dich und steuerte
solcher deiner Weise und übergab dich in den Willen deiner Feinde, den
Töchtern der Philister, welche sich schämten vor deinem verruchten
Wesen. \bibverse{28} Danach triebest du Hurerei mit den Kindern Assur
und konntest des nicht satt werden; ja, da du mit ihnen Hurerei
getrieben hattest, und des nicht satt werden konntest, \bibverse{29}
machtest du der Hurerei noch mehr im Lande Kanaan bis nach Chaldäa; noch
konntest du damit auch nicht satt werden. \bibverse{30} Wie soll ich dir
doch dein Herz beschneiden, spricht der HErr HErr, weil du solche Werke
tust einer großen Erzhure, \bibverse{31} damit daß du deine Bergkirchen
bauetest vornean auf allen Straßen und deine Altäre machtest auf allen
Gassen? Dazu warest du nicht wie eine andere Hure, die man muß mit Geld
kaufen, \bibverse{32} noch wie die Ehebrecherin, die anstatt ihres
Mannes andere zuläßt. \bibverse{33} Denn allen andern Huren gibt man
Geld; du aber gibst allen deinen Buhlern Geld zu und schenkest ihnen,
daß sie zu dir kommen allenthalben und mit dir Hurerei treiben.
\bibverse{34} Und findet sich an dir das Widerspiel vor andern Weibern
mit deiner Hurerei, weil man dir nicht nachläuft, sondern du Geld
zugibst und man dir nicht Geld zugibt. Also treibest du das Widerspiel.
\bibverse{35} Darum, du Hure, höre des HErrn Wort! \bibverse{36} So
spricht der HErr HErr: Weil du denn so milde Geld zugibst und deine
Scham durch deine Hurerei gegen deine Buhlen entblößest und gegen alle
Götzen deiner Greuel und vergeußest das Blut deiner Kinder, welche du
ihnen opferst, \bibverse{37} darum, siehe, will ich sammeln alle deine
Buhlen, mit welchen du Wollust getrieben hast, samt allen, die du für
Freunde hieltest, zu deinen Feinden; und will sie beide wider dich
sammeln allenthalben und will ihnen deine Scham blößen, daß sie deine
Scham gar sehen sollen. \bibverse{38} Und will das Recht der
Ehebrecherinnen und Blutvergießerinnen über dich gehen lassen und will
dein Blut stürzen mit Grimm und Eifer. \bibverse{39} Und will dich in
ihre Hände geben, daß sie deine Bergkirchen abbrechen und deine
Bergaltäre umreißen und dir deine Kleider ausziehen und dein schön Gerät
dir nehmen und dich nackend und bloß sitzen lassen. \bibverse{40} Und
sollen Haufen Leute über dich bringen, die dich steinigen und mit ihren
Schwertern zerhauen \bibverse{41} und deine Häuser mit Feuer verbrennen
und dir dein Recht tun vor den Augen vieler Weiber. Also will ich deiner
Hurerei ein Ende machen, daß du nicht mehr sollst Geld noch zugeben.
\bibverse{42} Und will meinen Mut an dir kühlen und meinen Eifer an dir
sättigen, daß ich ruhe und nicht mehr zürnen dürfe. \bibverse{43} Darum
daß du nicht gedacht hast an die Zeit deiner Jugend, sondern mich mit
diesem allem gereizet, darum will ich auch dir all dein Tun auf den Kopf
legen, spricht der HErr HErr; wiewohl ich damit nicht getan habe nach
dem Laster in deinen Greueln. \bibverse{44} Siehe, alle die, so
Sprichwort pflegen zu üben, werden von dir dies Sprichwort sagen: Die
Tochter ist wie die Mutter. \bibverse{45} Du bist deiner Mutter Tochter,
welche ihren Mann und Kinder verstößt, und bist eine Schwester deiner
Schwestern, die ihre Männer und Kinder verstoßen. Eure Mutter ist eine
von den Hethitern und euer Vater ein Amoriter. \bibverse{46} Samaria ist
deine große Schwester mit ihren Töchtern, die dir zur Linken wohnet, und
Sodom ist deine kleine Schwester mit ihren Töchtern, die zu deiner
Rechten wohnet; \bibverse{47} wiewohl du dennoch nicht gelebet hast nach
ihrem Wesen noch getan nach ihren Greueln. Es fehlet nicht weit, daß du
es ärger gemacht hast denn sie in all deinem Wesen. \bibverse{48} So
wahr ich lebe, spricht der HErr HErr, Sodom, deine Schwester, samt ihren
Töchtern hat nicht so getan wie du und deine Töchter. \bibverse{49}
Siehe, das war deiner Schwester Sodom Missetat: Hoffart und alles
vollauf und guter Friede, den sie und ihre Töchter hatten; aber dem
Armen und Dürftigen halfen sie nicht, \bibverse{50} sondern waren stolz
und taten Greuel vor mir; darum ich sie auch weggetan habe, da ich
begann dareinzusehen. \bibverse{51} So hat auch Samaria nicht die Hälfte
deiner Sünden getan, sondern du hast deiner Greuel so viel mehr über sie
getan, daß du deine Schwester gleich fromm gemacht hast gegen alle deine
Greuel, die du getan hast. \bibverse{52} So trage auch nun deine
Schande, die du deine Schwester fromm machst durch deine Sünden, in
welchen du größere Greuel denn sie getan hast, und machst sie frömmer,
denn du bist. So sei nun auch du schamrot und trage deine Schande, daß
du deine Schwester fromm gemacht hast. \bibverse{53} Ich will aber ihr
Gefängnis wenden, nämlich das Gefängnis dieser Sodom und ihrer Töchter
und das Gefängnis dieser Samaria und ihrer Töchter und die Gefangenen
deines jetzigen Gefängnisses samt ihnen, \bibverse{54} daß du tragen
müssest deine Schande und Hohn für alles, das du getan hast, und dennoch
ihr getröstet werdet. \bibverse{55} Und deine Schwester, diese Sodom und
ihre Töchter sollen bekehret werden, wie sie vor gewesen sind, und
Samaria und ihre Töchter sollen bekehret werden, wie sie vor gewesen
sind, dazu du auch und deine Töchter sollet bekehret werden, wie ihr vor
gewesen seid. \bibverse{56} Und wirst nicht mehr dieselbige Sodom, deine
Schwester, rühmen wie zur Zeit deines Hochmuts, \bibverse{57} da deine
Bosheit noch nicht entdeckt war, als zur Zeit, da dich die Töchter
Syriens und die Töchter der Philister allenthalben schändeten und
verachteten dich um und um, \bibverse{58} da ihr mußtet eure Laster und
Greuel tragen, spricht der HErr HErr. \bibverse{59} Denn also spricht
der HErr HErr: Ich will dir tun, wie du getan hast, daß du den Eid
verachtest und brichst den Bund. \bibverse{60} Ich will aber gedenken an
meinen Bund, den ich mit dir gemacht habe zur Zeit deiner Jugend, und
will mit dir einen ewigen Bund aufrichten. \bibverse{61} Da wirst du an
deine Wege gedenken und dich schämen, wenn du deine großen und kleinen
Schwestern zu dir nehmen wirst, die ich dir zu Töchtern geben werde,
aber nicht aus deinem Bunde, \bibverse{62} sondern will meinen Bund mit
dir aufrichten, daß du erfahren sollst, daß ich der HErr sei,
\bibverse{63} auf daß du daran gedenkest und dich schämest und vor
Schanden nicht mehr deinen Mund auftun dürfest, wenn ich dir alles
vergeben werde, was du getan hast, spricht der HErr HErr.

\hypertarget{section-16}{%
\section{17}\label{section-16}}

\bibverse{1} Und des HErrn Wort geschah zu mir und sprach: \bibverse{2}
Du Menschenkind, lege dem Hause Israel ein Rätsel vor und ein Gleichnis
\bibverse{3} und sprich: So spricht der HErr HErr: Ein großer Adler mit
großen Flügeln und langen Fittichen und voll Federn, die bunt waren, kam
auf Libanon und nahm den Wipfel von der Zeder \bibverse{4} und brach das
oberste Reis ab und führete es ins Krämerland und setzte es in die
Kaufmannsstadt. \bibverse{5} Er nahm auch Samen aus demselbigen Lande
und säete ihn in dasselbige gute Land, da viel Wassers ist, und setzte
es lose hin. \bibverse{6} Und es wuchs und ward ein ausgebreiteter
Weinstock und niedriges Stammes; denn seine Reben bogen sich zu ihm, und
seine Wurzeln waren unter ihm; und war also ein Weinstock, der Reben
kriegte und Zweige. \bibverse{7} Und da war ein anderer großer Adler mit
großen Flügeln und vielen Federn; und siehe, der Weinstock hatte
Verlangen an seinen Wurzeln zu diesem Adler und streckte seine Reben aus
gegen ihn, daß er gewässert würde vom Platz seiner Pflanzen.
\bibverse{8} Und war doch auf einem guten Boden an viel Wasser
gepflanzet, daß er wohl hätte können Zweige bringen, Frucht tragen und
ein herrlicher Weinstock werden. \bibverse{9} So sprich nun: Also sagt
der HErr HErr: Sollte der geraten? Ja, man wird seine Wurzel ausrotten
und seine Frucht abreißen, und wird verdorren, daß all seines Gewächses
Blätter verdorren werden, und wird nicht geschehen durch großen Arm noch
viel Volks, auf daß man ihn von seinen Wurzeln wegführe. \bibverse{10}
Siehe, er ist zwar gepflanzet, aber sollt er geraten? Ja, sobald ihn der
Ostwind rühren wird, wird er verdorren auf dem Platz seines Gewächses.
\bibverse{11} Und des HErrn Wort geschah zu mir und sprach:
\bibverse{12} Lieber, sprich zu dem ungehorsamen Hause: Wisset ihr
nicht, was das ist? Und sprich: Siehe, es kam der König zu Babel gen
Jerusalem und nahm ihren König und ihre Fürsten und führete sie weg zu
sich gen Babel \bibverse{13} und nahm von dem königlichen Samen und
machte einen Bund mit ihm und nahm einen Eid von ihm; aber die
Gewaltigen im Lande nahm er weg, \bibverse{14} damit das Königreich
demütig bliebe und sich nicht erhübe, auf daß sein Bund gehalten würde
und bestünde. \bibverse{15} Aber derselbe (Same) fiel von ihm ab und
sandte seine Botschaft nach Ägypten, daß man ihm Rosse und viel Volks
schicken sollte. Sollt es dem geraten? Sollt er davonkommen, der solches
tut? Und sollte der, so den Bund bricht, davonkommen? \bibverse{16} So
wahr ich lebe, spricht der HErr HErr, an dem Ort des Königs, der ihn zum
Könige gesetzt hat, welches Eid er verachtet, und welches Bund er
gebrochen hat, da soll er sterben, nämlich zu Babel. \bibverse{17} Auch
wird ihm Pharao nicht beistehen im Kriege mit großem Heer und viel
Volks, wenn man die Schütte aufwerfen wird und die Bollwerke bauen, daß
viel Leute umgebracht werden. \bibverse{18} Denn weil er den Eid
verachtet und den Bund gebrochen hat, darauf er seine Hand gegeben hat,
und solches alles tut, wird er nicht davonkommen. \bibverse{19} Darum
spricht der HErr HErr also: So wahr als ich lebe, so will ich meinen
Eid, den er verachtet hat, und meinen Bund, den er gebrochen hat, auf
seinen Kopf bringen. \bibverse{20} Denn ich will mein Netz über ihn
werfen, und muß in meiner Jagd gefangen werden; und will ihn gen Babel
bringen und will daselbst mit ihm rechten über dem, daß er sich also an
mir vergriffen hat. \bibverse{21} Und alle seine Flüchtigen, die ihm
anhingen, sollen durchs Schwert fallen, und ihre Übrigen sollen in alle
Winde zerstreuet werden, und sollt es erfahren, daß ich's, der HErr,
geredet habe. \bibverse{22} So spricht der HErr HErr: Ich will auch von
dem Wipfel des hohen Zedernbaums nehmen und oben von seinen Zweigen ein
zartes Reis brechen und will's auf einen hohen gehäuften Berg pflanzen,
\bibverse{23} nämlich auf den hohen Berg Israel will ich's pflanzen, daß
es Zweige gewinne und Früchte bringe und ein herrlicher Zedernbaum
werde, also daß allerlei Vögel unter ihm wohnen und allerlei Fliegendes
unter dem Schatten seiner Zweige bleiben möge. \bibverse{24} Und sollen
alle Feldbäume erfahren, daß ich, der HErr, den hohen Baum geniedriget
und den niedrigen Baum erhöhet habe und den grünen Baum ausgedorret und
den dürren Baum grünend gemacht habe. Ich, der HErr, rede es und tue es
auch.

\hypertarget{section-17}{%
\section{18}\label{section-17}}

\bibverse{1} Und des HErrn Wort geschah zu mir und sprach: \bibverse{2}
Was treibet ihr unter euch im Lande Israel dies Sprichwort und sprechet:
Die Väter haben Herlinge gegessen, aber den Kindern sind die Zähne davon
stumpf worden? \bibverse{3} So wahr als ich lebe, spricht der HErr HErr,
solch Sprichwort soll nicht mehr unter euch gehen in Israel.
\bibverse{4} Denn siehe, alle Seelen sind mein; des Vaters Seele ist
sowohl mein als des Sohnes Seele. Welche Seele sündiget, die soll
sterben. \bibverse{5} Wenn nun einer fromm ist, der recht und wohl tut;
\bibverse{6} der auf den Bergen nicht isset; der seine Augen nicht
aufhebet zu den Götzen des Hauses Israel und seines Nächsten Weib nicht
befleckt und liegt nicht bei der Frau in ihrer Krankheit; \bibverse{7}
der niemand beschädiget; der dem Schuldner sein Pfand wiedergibt; der
niemand etwas mit Gewalt nimmt; der dem Hungrigen sein Brot mitteilet
und den Nackenden kleidet; \bibverse{8} der nicht wuchert; der niemand
übersetzet; der seine Hand vom Unrechten kehret; der zwischen den Leuten
recht urteilet; \bibverse{9} der nach meinen Rechten wandelt und meine
Gebote hält, daß er ernstlich danach tue: das ist ein frommer Mann; der
soll das Leben haben, spricht der HErr HErr. \bibverse{10} Wenn er aber
einen Sohn zeuget, und derselbige wird ein Mörder, der Blut vergeußt
oder dieser Stücke eins tut \bibverse{11} und der andern Stücke keins
nicht tut, sondern isset auf den Bergen und beflecket seines Nächsten
Weib, \bibverse{12} beschädiget die Armen und Elenden, mit Gewalt etwas
nimmt, das Pfand nicht wiedergibt, seine Augen zu den Götzen aufhebet,
damit er einen Greuel begehet, \bibverse{13} gibt auf Wucher, übersetzt:
sollte der leben? Er soll nicht leben, sondern weil er solche Greuel
alle getan hat, soll er des Todes sterben; sein Blut soll auf ihm sein.
\bibverse{14} Wo er aber einen Sohn zeuget, der alle solche Sünde
siehet, so sein Vater tut, und sich fürchtet und nicht also tut:
\bibverse{15} isset nicht auf den Bergen, hebet seine Augen nicht auf zu
den Götzen des Hauses Israel, beflecket nicht seines Nächsten Weib,
\bibverse{16} beschädiget niemand, behält das Pfand nicht, nicht mit
Gewalt etwas nimmt, teilet sein Brot mit dem Hungrigen und kleidet den
Nackenden; \bibverse{17} der seine Hand vom Unrechten kehret, keinen
Wucher noch Übersatz nimmt, sondern meine Gebote hält und nach meinen
Rechten lebet: der soll nicht sterben um seines Vaters Missetat willen,
sondern leben. \bibverse{18} Aber sein Vater, der Gewalt und Unrecht
geübet und unter seinem Volk getan hat, das nicht taugt, siehe,
derselbige soll sterben um seiner Missetat willen. \bibverse{19} So
sprechet ihr: Warum soll denn ein Sohn nicht tragen seines Vaters
Missetat? Darum, daß er recht und wohl getan und alle meine Rechte
gehalten und getan hat, soll er leben. \bibverse{20} Denn welche Seele
sündiget, die soll sterben. Der Sohn soll nicht tragen die Missetat des
Vaters, und der Vater soll nicht tragen die Missetat des Sohnes, sondern
des Gerechten Gerechtigkeit soll über ihm sein, und des Ungerechten
Ungerechtigkeit soll über ihm sein. \bibverse{21} Wo sich aber der
Gottlose bekehret von allen seinen Sünden, die er getan hat, und hält
alle meine Rechte und tut recht und wohl, so soll er leben und nicht
sterben. \bibverse{22} Es soll aller seiner Übertretung, so er begangen
hat, nicht gedacht werden, sondern soll leben um der Gerechtigkeit
willen, die er tut. \bibverse{23} Meinest du, daß ich Gefallen habe am
Tode des Gottlosen, spricht der HErr HErr, und nicht vielmehr, daß er
sich bekehre von seinem Wesen und lebe? \bibverse{24} Und wo sich der
Gerechte kehret von seiner Gerechtigkeit und tut Böses und lebet nach
allen Greueln, die ein Gottloser tut, sollte der leben? Ja, aller seiner
Gerechtigkeit, die er getan hat, soll nicht gedacht werden, sondern in
seiner Übertretung und Sünden, die er getan hat, soll er sterben.
\bibverse{25} Noch sprechet ihr: Der HErr handelt nicht recht. So höret
nun, ihr vom Hause Israel: Ist's nicht also, daß ich recht habe und ihr
unrecht habt? \bibverse{26} Denn wenn der Gerechte sich kehret von
seiner Gerechtigkeit und tut Böses, so muß er sterben; er muß aber um
seiner Bosheit willen, die er getan hat, sterben. \bibverse{27}
Wiederum, wenn sich der Gottlose kehret von seiner Ungerechtigkeit, die
er getan hat, und tut nun recht und wohl, der wird seine Seele lebendig
behalten. \bibverse{28} Denn weil er siehet und bekehret sich von aller
seiner Bosheit, die er getan hat, so soll er leben und nicht sterben.
\bibverse{29} Noch sprechen die vom Hause Israel: Der HErr handelt nicht
recht. Sollt ich unrecht haben? Ihr vom Hause Israel habt unrecht.
\bibverse{30} Darum will ich euch richten, ihr vom Hause Israel, einen
jeglichen nach seinem Wesen, spricht der HErr HErr. Darum so bekehret
euch von aller eurer Übertretung, auf daß ihr nicht fallen müsset um der
Missetat willen. \bibverse{31} Werfet von euch alle eure Übertretung,
damit ihr übertreten habt, und machet euch ein neu Herz und neuen Geist.
Denn warum willst du also sterben, du Haus Israel? \bibverse{32} Denn
ich habe keinen Gefallen am Tode des Sterbenden, spricht der HErr HErr.
Darum bekehret euch, so werdet ihr leben!

\hypertarget{section-18}{%
\section{19}\label{section-18}}

\bibverse{1} Du aber mache eine Wehklage über die Fürsten Israels
\bibverse{2} und sprich: Warum liegt deine Mutter, die Löwin, unter den
Löwinnen und erzeucht ihre Jungen unter den jungen Löwen? \bibverse{3}
Derselbigen eins zog sie auf, und ward ein junger Löwe daraus; der
gewöhnte sich, die Leute zu reißen und fressen. \bibverse{4} Da das die
Heiden von ihm höreten, fingen sie ihn in ihren Gruben und führeten ihn
an Ketten nach Ägyptenland. \bibverse{5} Da nun die Mutter sah, daß ihre
Hoffnung verloren war, da sie lange gehoffet hatte, nahm sie ein anderes
aus ihren Jungen und machte einen jungen Löwen daraus. \bibverse{6} Da
der unter den Löwinnen wandelte, ward er ein junger Löwe; der gewohnte
auch, die Leute zu reißen und fressen. \bibverse{7} Er lernte ihre
Witwen kennen und verwüstete ihre Städte, daß das Land, und was drinnen
ist, vor der Stimme seines Brüllens sich entsetzte. \bibverse{8} Da
legten sich die Heiden aus allen Ländern ringsumher und warfen ein Netz
über ihn und fingen ihn in ihren Gruben \bibverse{9} und stießen ihn
gebunden in ein Gatter und führeten ihn zum Könige zu Babel; und man
ließ ihn verwahren, daß seine Stimme nicht mehr gehöret würde auf den
Bergen Israels. \bibverse{10} Deine Mutter war wie ein Weinstock,
gleichwie du, am Wasser gepflanzet, und ihre Frucht und Reben wuchsen
von dem großen Wasser, \bibverse{11} daß seine Reben so stark wurden,
daß sie zu Herrenzeptern gut waren, und ward hoch unter den Reben. Und
da man sah, daß er so hoch und viel Reben hatte, \bibverse{12} ward er
im Grimm zu Boden gerissen und verworfen; der Ostwind verdorrete seine
Frucht, und seine starken Reben wurden zerbrochen, daß sie verdorreten
und verbrannt wurden. \bibverse{13} Nun aber ist sie gepflanzet in der
Wüste, in einem dürren, durstigen Lande, \bibverse{14} und ist ein Feuer
ausgegangen von ihren starken Reben, das verzehret ihre Frucht, daß in
ihr keine starke Rebe mehr ist zu eines Herrn Zepter. Das ist ein
kläglich und jämmerlich Ding.

\hypertarget{section-19}{%
\section{20}\label{section-19}}

\bibverse{1} Und es begab sich im siebenten Jahr, am zehnten Tage des
fünften Monden, kamen etliche aus den Ältesten Israels, den HErrn zu
fragen, und setzten sich vor mir nieder. \bibverse{2} Da geschah des
HErrn Wort zu mir und sprach: \bibverse{3} Du Menschenkind, sage den
Ältesten Israels und sprich zu ihnen: So spricht der HErr HErr: Seid ihr
kommen, mich zu fragen? So wahr ich lebe, ich will von euch ungefragt
sein, spricht der HErr HErr. \bibverse{4} Aber willst du sie strafen, du
Menschenkind, so magst du sie also strafen. Zeige ihnen an die Greuel
ihrer Väter \bibverse{5} und sprich zu ihnen: So spricht der HErr HErr:
Zu der Zeit, da ich Israel erwählete, erhub ich meine Hand zu dem Samen
des Hauses Jakob und gab mich ihnen zu erkennen in Ägyptenland. Ja, ich
erhub meine Hand zu ihnen und sprach: Ich bin der HErr, euer GOtt.
\bibverse{6} Ich erhub aber zur selbigen Zeit meine Hand, daß ich sie
führete aus Ägyptenland in ein Land, das ich ihnen versehen hatte, das
mit Milch und Honig fleußt, ein edel Land vor allen Ländern,
\bibverse{7} und sprach zu ihnen: Ein jeglicher werfe weg die Greuel vor
seinen Augen und verunreiniget euch nicht an den Götzen Ägyptens; denn
ich bin der HErr, euer GOtt. \bibverse{8} Sie aber waren mir ungehorsam
und wollten mir nicht gehorchen, und warf ihrer keiner weg die Greuel
vor seinen Augen und verließen die Götzen Ägyptens nicht. Da dachte ich
meinen Grimm über sie auszuschütten und allen meinen Zorn über sie gehen
zu lassen noch in Ägyptenland. \bibverse{9} Aber ich ließ es um meines
Namens willen, daß er nicht entheiliget würde vor den Heiden, unter
denen sie waren, und vor denen ich mich ihnen hatte zu erkennen gegeben,
daß ich sie aus Ägyptenland führen wollte. \bibverse{10} Und da ich sie
aus Ägyptenland geführet hatte und in die Wüste gebracht, \bibverse{11}
gab ich ihnen meine Gebote und lehrete sie meine Rechte, durch welche
lebet der Mensch, der sie hält. \bibverse{12} Ich gab ihnen auch meine
Sabbate zum Zeichen zwischen mir und ihnen damit sie lerneten, daß ich
der HErr sei, der sie heiliget. \bibverse{13} Aber das Haus Israel war
mir ungehorsam auch in der Wüste und lebten nicht nach meinen Geboten
und verachteten meine Rechte, durch welche der Mensch lebet, der sie
hält, und entheiligten meine Sabbate sehr. Da gedachte ich meinen Grimm
über sie auszuschütten in der Wüste und sie gar umzubringen.
\bibverse{14} Aber ich ließ es um meines Namens willen, auf daß er nicht
entheiliget würde vor den Heiden, vor welchen ich sie hatte ausgeführet.
\bibverse{15} Und hub auch meine Hand auf wider sie in der Wüste, daß
ich sie nicht wollte bringen in das Land, so ich ihnen gegeben hatte,
das mit Milch und Honig fleußt, ein edel Land vor allen Ländern,
\bibverse{16} darum daß sie meine Rechte verachtet und nach meinen
Geboten nicht gelebet und meine Sabbate entheiliget hatten; denn sie
wandelten nach den Götzen ihres Herzens. \bibverse{17} Aber mein Auge
verschonete ihrer, daß ich sie nicht verderbete noch gar umbrächte in
der Wüste. \bibverse{18} Und ich sprach zu ihren Kindern in der Wüste:
Ihr sollt nach eurer Väter Geboten nicht leben und ihre Rechte nicht
halten und an ihren Götzen euch nicht verunreinigen. \bibverse{19} Denn
ich hin der HErr, euer GOtt; nach meinen Geboten sollt ihr leben und
meine Rechte sollt ihr halten und danach tun \bibverse{20} und meine
Sabbate sollt ihr heiligen, daß sie seien ein Zeichen zwischen mir und
euch, damit ihr wisset, daß ich der HErr, euer GOtt, bin. \bibverse{21}
Aber die Kinder waren mir auch ungehorsam, lebten nach meinen Geboten
nicht, hielten auch meine Rechte nicht, daß sie danach täten, durch
welche der Mensch lebet, der sie hält, und entheiligten meine Sabbate.
Da gedachte ich meinen Grimm über sie auszuschütten und all meinen Zorn
über sie gehen zu lassen in der Wüste. \bibverse{22} Ich wandte aber
meine Hand und ließ es um meines Namens willen, auf daß er nicht
entheiliget würde vor den Heiden, vor welchen ich sie hatte ausgeführet.
\bibverse{23} Ich hub auch meine Hand auf wider sie in der Wüste, daß
ich sie zerstreuete unter die Heiden und zerstäubete in die Länder,
\bibverse{24} darum daß sie meine Gebote nicht gehalten und meine Rechte
verachtet und meine Sabbate entheiliget hatten und nach den Götzen ihrer
Väter sahen. \bibverse{25} Darum übergab ich sie in die Lehre, so nicht
gut ist, und in Rechte, darin sie kein Leben konnten haben,
\bibverse{26} und verwarf sie mit ihrem Opfer, da sie alle Erstgeburt
durchs Feuer verbrannten, damit ich sie verstörte und sie lernen mußten,
daß ich der HErr sei. \bibverse{27} Darum rede, du Menschenkind, mit dem
Hause Israel und sprich zu ihnen: So spricht der HErr HErr: Eure Väter
haben mich noch weiter gelästert und getrotzet. \bibverse{28} Denn da
ich sie in das Land gebracht hatte, über welches ich meine Hand
aufgehoben hatte, daß ich's ihnen gäbe: wo sie einen hohen Hügel oder
dicken Baum ersahen, daselbst opferten sie ihre Opfer und brachten dahin
ihre feindseligen Gaben und räucherten daselbst ihren süßen Geruch und
gossen daselbst ihre Trankopfer. \bibverse{29} Ich aber sprach zu ihnen:
Was soll doch die Höhe, dahin ihr gehet? Und also heißt sie bis auf
diesen Tag die Höhe. \bibverse{30} Darum sprich zum Hause Israel: So
spricht der HErr HErr: Ihr verunreiniget euch in dem Wesen eurer Väter
und treibet Hurerei mit ihren Greueln \bibverse{31} und verunreiniget
euch an euren Götzen, welchen ihr eure Gaben opfert und eure Söhne und
Töchter durchs Feuer verbrennet, bis auf den heutigen Tag; und ich
sollte mich euch vom Hause Israel fragen lassen? So wahr ich lebe,
spricht der HErr HErr, ich will von euch ungefragt sein! \bibverse{32}
Dazu, daß ihr gedenket, wir wollen tun wie die Heiden und wie andere
Leute in Ländern, Holz und Stein anbeten, das soll euch fehlen!
\bibverse{33} So wahr ich lebe, spricht der HErr HErr, ich will über
euch herrschen mit starker Hand und ausgestrecktem Arm und mit
ausgeschüttetem Grimm; \bibverse{34} und will euch aus den Völkern
führen und aus den Ländern, dahin ihr verstreuet seid, sammeln mit
starker Hand, mit ausgestrecktem Arm und mit ausgeschüttetem Grimm;
\bibverse{35} und will euch bringen in die Wüste der Völker und daselbst
mit euch rechten von Angesicht zu Angesicht. \bibverse{36} Wie ich mit
euren Vätern in der Wüste bei Ägypten gerechtet habe, ebenso will ich
auch mit euch rechten, spricht der HErr HErr. \bibverse{37} Ich will
euch wohl unter die Rute bringen und euch in die Bande des Bundes
zwingen. \bibverse{38} Und will die Abtrünnigen, und so wider mich
übertreten, unter euch ausfegen; ja, aus dem Lande, da ihr jetzt wohnet,
will ich sie führen und ins Land Israel nicht kommen lassen, daß ihr
lernen sollt, ich sei der HErr. \bibverse{39} Darum, ihr vom Hause
Israel, so spricht der HErr HErr: Weil ihr denn mir ja nicht wollt
gehorchen, so fahret hin und diene ein jeglicher seinem Götzen; aber
meinen heiligen Namen laßt hinfort ungeschändet mit euren Opfern und
Götzen! \bibverse{40} Denn so spricht der HErr HErr: Auf meinem heiligen
Berge, auf dem hohen Berge Israel, daselbst wird mir das ganze Haus
Israel und alle, die im Lande sind, dienen; daselbst werden sie mir
angenehm sein, und daselbst will ich eure Hebopfer und Erstlinge eurer
Opfer fordern mit allem, das ihr mir heiliget. \bibverse{41} Ihr werdet
mir angenehm sein mit dem süßen Geruch, wenn ich euch aus den Völkern
bringen und aus den Ländern sammeln werde, dahin ihr verstreuet seid,
und werde in euch geheiliget werden vor den Heiden. \bibverse{42} Und
ihr werdet erfahren, daß ich der HErr bin, wenn ich euch ins Land Israel
gebracht habe, in das Land, darüber ich meine Hand aufhub, daß ich's
euren Vätern gäbe. \bibverse{43} Daselbst werdet ihr gedenken an euer
Wesen und an all euer Tun, darinnen ihr verunreiniget seid, und werdet
Mißfallen haben über alle eure Bosheit, die ihr getan habt,
\bibverse{44} und werdet erfahren, daß ich der HErr bin, wenn ich mit
euch tue um meines Namens willen und nicht nach eurem bösen Wesen und
schädlichem Tun, du Haus Israel, spricht der HErr HErr. \bibverse{45}
Und des HErrn Wort geschah zu mir und sprach: \bibverse{46} Du
Menschenkind, richte dein Angesicht gegen den Südwind zu und träufe
gegen den Mittag und weissage wider den Wald im Felde gegen Mittag.
\bibverse{47} Und sprich zum Walde gegen Mittag: Höre des HErrn Wort! So
spricht der HErr HErr: Siehe, ich will in dir ein Feuer anzünden, das
soll beide, grüne und dürre Bäume, verzehren, daß man seine Flamme nicht
wird löschen können, sondern es soll verbrannt werden alles, was vom
Mittage gegen Mitternacht stehet. \bibverse{48} Und alles Fleisch soll
sehen, daß ich, der HErr, es angezündet habe und niemand löschen möge.
\bibverse{49} Und ich sprach: Ach, HErr HErr, sie sagen von mir: Dieser
redet eitel verdeckte Worte.

\hypertarget{section-20}{%
\section{21}\label{section-20}}

\bibverse{1} Und des HErrn Wort geschah zu mir und sprach: \bibverse{2}
Du Menschenkind, richte dein Angesicht wider Jerusalem und träufe wider
die Heiligtümer und weissage wider das Land Israel \bibverse{3} und
sprich zum Lande Israel: So spricht der HErr HErr: Siehe, ich will an
dich; ich will mein Schwert aus der Scheide ziehen und will in dir
ausrotten beide, Gerechte und Ungerechte. \bibverse{4} Weil ich denn in
dir beide, Gerechte und Ungerechte, ausrotte, so wird mein Schwert aus
der Scheide fahren über alles Fleisch vom Mittage her bis gen
Mitternacht. \bibverse{5} Und soll alles Fleisch erfahren, daß ich, der
HErr, mein Schwert hab' aus seiner Scheide gezogen; und soll nicht
wieder eingesteckt werden. \bibverse{6} Und du, Menschenkind, sollst
seufzen, bis dir die Lenden weh tun; ja, bitterlich sollst du seufzen,
daß sie es sehen. \bibverse{7} Und wenn sie zu dir sagen werden: Warum
seufzest du? sollst du sagen: Um des Geschreies willen, das da kommt,
vor welchem alle Herzen verzagen und alle Hände sinken, aller Mut
fallen, und alle Kniee wie Wasser gehen werden. Siehe, es kommt und wird
geschehen, spricht der HErr HErr. \bibverse{8} Und des HErrn Wort
geschah zu mir und sprach: \bibverse{9} Du Menschenkind, weissage und
sprich: So spricht der HErr: Sprich: Das Schwert, ja, das Schwert ist
geschärft und gefegt. \bibverse{10} Es ist geschärft, daß es schlachten
soll; es ist gefegt, daß es blinken soll. O wie froh wollten wir sein,
wenn er gleich alle Bäume zu Ruten machte über die bösen Kinder!
\bibverse{11} Aber er hat ein Schwert zu fegen gegeben, daß man es
fassen soll; es ist geschärft und gefegt, daß man's dem Totschläger in
die Hand gebe. \bibverse{12} Schreie und heule, du Menschenkind; denn es
gehet über mein Volk und über alle Regenten in Israel, die zum Schwert
samt meinem Volk versammelt sind. Darum schlage auf deine Lenden:
\bibverse{13} Denn er hat sie oft gezüchtiget; was hat's geholfen? Es
will der bösen Kinder Rute nicht helfen, spricht der HErr HErr.
\bibverse{14} Und du, Menschenkind, weissage und schlage deine Hände
zusammen. Denn das Schwert wird zwiefach, ja dreifach kommen, ein
Würgeschwert, ein Schwert großer Schlacht, das sie auch treffen wird in
den Kammern, da sie hinfliehen. \bibverse{15} Ich will das Schwert
lassen klingen, daß die Herzen verzagen und viele fallen sollen an allen
ihren Toren. Ach, wie glänzet es und hauet daher zur Schlacht!
\bibverse{16} Und sprechen: Haue drein, beide, zur Rechten und Linken,
was vor dir ist! \bibverse{17} Da will ich dann mit meinen Händen darob
frohlocken und meinen Zorn gehen lassen. Ich, der HErr, hab es gesagt.
\bibverse{18} Und des HErrn Wort geschah zu mir und sprach:
\bibverse{19} Du Menschenkind, mache zween Wege, durch welche kommen
soll das Schwert des Königs zu Babel; sie sollen aber alle beide aus
einem Lande gehen. \bibverse{20} Und stelle ein Zeichen vorne an den Weg
zur Stadt, dahin es weisen soll; und mache den Weg, daß das Schwert
komme gen Rabbath der Kinder Ammon und nach Juda, zu der festen Stadt
Jerusalem. \bibverse{21} Denn der König zu Babel wird sich an die
Wegscheide stellen, vorne an den zween Wegen, daß er ihm wahrsagen
lasse, mit den Pfeilen um das Los schieße, seinen Abgott frage und
schaue die Leber an. \bibverse{22} Und die Wahrsagung wird auf die
rechte Seite gen Jerusalem deuten, daß er solle Böcke hinanführen lassen
und Löcher machen und mit großem Geschrei sie überfalle und morde, und
daß er Böcke führen solle wider die Tore und da Wall schütte und
Bollwerk baue. \bibverse{23} Aber es wird sie solch Wahrsagen falsch
dünken, er schwöre, wie teuer er will. Er aber wird denken an die
Missetat, daß er sie gewinne. \bibverse{24} Darum spricht der HErr HErr
also: Darum daß euer gedacht wird um eurer Missetat und euer Ungehorsam
offenbart ist, daß man eure Sünde siehet in all eurem Tun, ja darum daß
euer gedacht wird, werdet ihr mit Gewalt gefangen werden. \bibverse{25}
Und du, Fürst in Israel, der du verdammt und verurteilet bist, des Tag
daherkommen wird, wenn die Missetat zum Ende kommen ist, \bibverse{26}
so spricht der HErr HErr: Tu weg den Hut und heb ab die Krone! Denn es
wird weder der Hut noch die Krone bleiben, sondern der sich erhöhet hat,
soll geniedriget werden, und der sich niedriget, soll erhöhet werden.
\bibverse{27} Ich will die Krone zunichte, zunichte, zunichte machen,
bis der komme, der sie haben soll; dem will ich sie geben. \bibverse{28}
Und du, Menschenkind, weissage und sprich: So spricht der HErr HErr von
den Kindern Ammon und von ihrer Schmach und sprich: Das Schwert, das
Schwert ist gezückt, daß es schlachten soll; es ist gefegt, daß es
würgen soll, und soll blinken, \bibverse{29} darum daß du falsche
Gesichte dir sagen lässest und Lügen weissagen, damit du auch übergeben
werdest unter den erschlagenen Gottlosen, welchen ihr Tag kam, da die
Missetat zum Ende kommen war. \bibverse{30} Und ob es schon wieder in
die Scheide gesteckt würde, so will ich dich doch richten an dem Ort, da
du geschaffen, und im Lande, da du geboren bist. \bibverse{31} Und will
meinen Zorn über dich schütten, ich will das Feuer meines Grimms über
dich aufblasen und will dich Leuten, die brennen und verderben können,
überantworten. \bibverse{32} Du mußt dem Feuer zur Speise werden, und
dein Blut muß im Lande vergossen werden; und man wird dein nicht mehr
gedenken. Denn ich, der HErr, hab es geredet.

\hypertarget{section-21}{%
\section{22}\label{section-21}}

\bibverse{1} Und des HErrn Wort geschah zu mir und sprach: \bibverse{2}
Du Menschenkind, willst du nicht strafen die mörderische Stadt und ihr
anzeigen alle ihre Greuel? \bibverse{3} Sprich: So spricht der HErr
HErr: O Stadt, die du der Deinen Blut vergeußest, auf daß deine Zeit
komme, und die du Götzen bei dir machst, damit du dich verunreinigest!
\bibverse{4} Du verschuldest dich an dem Blut, das Du vergeußest, und
verunreinigest dich an den Götzen, die du machst; damit bringest du
deine Tage herzu und machst, daß deine Jahre kommen müssen. Darum will
ich dich zum Spott unter den Heiden und zum Hohn in allen Ländern
machen. \bibverse{5} Beide, in der Nähe und in der Ferne, sollen sie
dein spotten, daß du ein schändlich Gerücht haben und großen Jammer
leiden müssest. \bibverse{6} Siehe die Fürsten in Israel! Ein jeglicher
ist mächtig bei dir, Blut zu vergießen. \bibverse{7} Vater und Mutter
verachten sie; den Fremdlingen tun sie Gewalt und Unrecht; die Witwen
und Waisen schinden sie. \bibverse{8} Du verachtest meine Heiligtümer
und entheiligest meine Sabbate. \bibverse{9} Verräter sind in dir, auf
daß sie Blut vergießen. Sie essen auf den Bergen und handeln
mutwilliglich in dir; \bibverse{10} sie blößen die Scham der Väter und
nötigen die Weiber in ihrer Krankheit \bibverse{11} und treiben
untereinander, Freund mit Freundes Weibe, Greuel; sie schänden ihre
eigene Schnur mit allem Mutwillen; sie notzüchtigen ihre eigenen
Schwestern, ihres Vaters Töchter; \bibverse{12} sie nehmen Geschenke,
auf daß sie Blut vergießen; sie wuchern und übersetzen einander und
treiben ihren Geiz wider ihren Nächsten und tun einander Gewalt und
vergessen mein also, spricht der HErr HErr. \bibverse{13} Siehe, ich
schlage meine Hände zusammen über den Geiz, den du treibest, und über
das Blut, so in dir vergossen ist. \bibverse{14} Meinest du aber, dein
Herz möge es erleiden oder deine Hände ertragen zu der Zeit, wenn ich's
mit dir machen werde? Ich, der HErr, hab es geredet und will's auch tun
\bibverse{15} und will dich zerstreuen unter die Heiden und dich
verstoßen in die Länder und will deines Unflats ein Ende machen,
\bibverse{16} daß du bei den Heiden mußt verflucht geachtet werden und
erfahren, daß ich der HErr sei. \bibverse{17} Und des HErrn Wort geschah
zu mir und sprach: \bibverse{18} Du Menschenkind, das Haus Israel ist
mir zu Schaum worden; all ihr Erz, Zinn, Eisen und Blei ist im Ofen zu
Silberschaum worden. \bibverse{19} Darum spricht der HErr HErr also:
Weil ihr denn alle Schaum worden seid, siehe, so will ich euch alle gen
Jerusalem zusammentun. \bibverse{20} Wie man Silber, Erz, Eisen, Blei
und Zinn zusammentut im Ofen, daß man ein Feuer darunter aufblase und
zerschmelze es, also will ich euch auch in meinem Zorn und Grimm
zusammentun, einlegen und schmelzen. \bibverse{21} Ja, ich will euch
sammeln und das Feuer meines Zorns unter euch aufblasen, daß ihr drinnen
zerschmelzen müsset. \bibverse{22} Wie das Silber zerschmilzet im Ofen,
so sollt ihr auch drinnen zerschmelzen und erfahren, daß ich, der HErr,
meinen Grimm über euch ausgeschüttet habe. \bibverse{23} Und des HErrn
Wort geschah zu mir und sprach: \bibverse{24} Du Menschenkind, sprich zu
ihnen: Du bist ein Land, das nicht zu reinigen ist, wie eins, das nicht
beregnet wird zur Zeit des Zorns. \bibverse{25} Die Propheten, so
drinnen sind, haben sich gerottet, die Seelen zu fressen, wie ein
brüllender Löwe, wenn er raubet; sie reißen Gut und Geld zu sich und
machen der Witwen viel drinnen. \bibverse{26} Ihre Priester verkehren
mein Gesetz freventlich und entheiligen mein Heiligtum; sie halten unter
dem Heiligen und Unheiligen keinen Unterschied und lehren nicht, was
rein oder unrein sei, und warten meiner Sabbate nicht; und ich werde
unter ihnen entheiliget. \bibverse{27} Ihre Fürsten sind drinnen wie die
reißenden Wölfe, Blut zu vergießen und Seelen umzubringen, um ihres
Geizes willen. \bibverse{28} Und ihre Propheten tünchen sie mit losem
Kalk, predigen lose Teidinge und weissagen ihnen Lügen und sagen: So
spricht der HErr HErr! so es doch der HErr nicht geredet hat.
\bibverse{29} Das Volk im Lande übet Gewalt und rauben getrost und
schinden die Armen und Elenden und tun den Fremdlingen Gewalt und
Unrecht. \bibverse{30} Ich suchte unter ihnen, ob jemand sich eine Mauer
machte und wider den Riß stünde vor mir für das Land, daß ich's nicht
verderbete; aber ich fand keinen. \bibverse{31} Darum schüttete ich
meinen Zorn über sie und mit dem Feuer meines Grimms machte ich ihrer
ein Ende und gab ihnen also ihren Verdienst auf ihren Kopf, spricht der
HErr HErr.

\hypertarget{section-22}{%
\section{23}\label{section-22}}

\bibverse{1} Und des HErrn Wort geschah zu mir und sprach: \bibverse{2}
Du Menschenkind, es waren zwei Weiber, einer Mutter Töchter.
\bibverse{3} Die trieben Hurerei in Ägypten in ihrer Jugend; daselbst
ließen sie ihre Brüste begreifen und die Zitzen ihrer Jungfrauschaft
betasten. \bibverse{4} Die große heißt Ahala und ihre Schwester Ahaliba.
Und ich nahm sie zur Ehe, und sie zeugeten mir Söhne und Töchter. Und
Ahala heißt Samaria und Ahaliba Jerusalem. \bibverse{5} Ahala trieb
Hurerei, da ich sie genommen hatte, und brannte gegen ihre Buhlen,
nämlich gegen die Assyrier, die zu ihr kamen, \bibverse{6} gegen die
Fürsten und Herren, die mit Seide gekleidet waren, und alle junge
liebliche Gesellen, nämlich gegen die Reiter und Wagen. \bibverse{7} Und
buhlete mit allen schönen Gesellen in Assyrien und verunreinigte sich
mit allen ihren Götzen, wo sie auf einen entbrannte. \bibverse{8} Dazu
verließ sie auch nicht ihre Hurerei mit Ägypten, die bei ihr gelegen
waren von ihrer Jugend auf und die Brüste ihrer Jungfrauschaft betastet
und große Hurerei mit ihr getrieben hatten. \bibverse{9} Da übergab ich
sie in die Hand ihrer Buhlen, den Kindern Assur, gegen welche sie
brannte vor Lust. \bibverse{10} Die deckten ihre Scham auf und nahmen
ihre Söhne und Töchter weg; sie aber töteten sie mit dem Schwert. Und es
kam aus, daß diese Weiber gestraft wären. \bibverse{11} Da es aber ihre
Schwester Ahaliba sah, entbrannte sie noch viel ärger denn jene und
trieb der Hurerei mehr denn ihre Schwester \bibverse{12} und entbrannte
gegen die Kinder Assur, nämlich die Fürsten und Herren, die zu ihr kamen
wohl gekleidet, Reiter und Wagen, und alle junge liebliche Gesellen.
\bibverse{13} Da sah ich, daß sie alle beide gleicherweise verunreiniget
waren. \bibverse{14} Aber diese trieb ihre Hurerei mehr. Denn da sie sah
gemalte Männer an der Wand in roter Farbe, die Bilder der Chaldäer,
\bibverse{15} um ihre Lenden gegürtet und bunte Kogel auf ihren Köpfen,
und alle gleich anzusehen wie gewaltige Leute, wie denn die Kinder Babel
und die Chaldäer tragen in ihrem Vaterlande, \bibverse{16} entbrannte
sie gegen sie, sobald sie ihrer gewahr ward, und schickte Botschaft zu
ihnen nach Chaldäa. \bibverse{17} Als nun die Kinder Babel zu ihr kamen,
bei ihr zu schlafen nach der Liebe, verunreinigten sie dieselbe mit
ihrer Hurerei, und sie verunreinigte sich mit ihnen, daß sie ihrer müde
ward. \bibverse{18} Und da beide ihre Hurerei und Scham so gar offenbar
war, ward ich ihrer auch überdrüssig, wie ich ihrer Schwester auch war
müde worden. \bibverse{19} Sie aber trieb ihre Hurerei immer mehr und
gedachte an die Zeit ihrer Jugend, da sie in Ägyptenland Hurerei
getrieben hatte, \bibverse{20} und entbrannte gegen ihre Buhlen, welcher
Brunst war wie der Esel und der Hengste Brunst. \bibverse{21} Und
bestelletest deine Unzucht wie in deiner Jugend, da die in Ägypten deine
Brüste begriffen und deine Zitzen betastet wurden. \bibverse{22} Darum,
Ahaliba, so spricht der HErr HErr: Siehe, ich will deine Buhlen, deren
du müde bist worden, wider dich erwecken und will sie ringsumher wider
dich bringen, \bibverse{23} nämlich die Kinder Babel und alle Chaldäer
mit Hauptleuten, Fürsten und Herren, und alle Assyrier mit ihnen, die
schöne junge Mannschaft, alle Fürsten und Herren, Ritter und Edle und
allerlei Reiter. \bibverse{24} Und werden über dich kommen, gerüstet mit
Wagen und Rädern und mit großem Haufen Volks, und werden dich belagern
mit Tartschen, Schilden und Helmen um und um. Denen will ich das Recht
befehlen, daß sie dich richten sollen nach ihrem Recht. \bibverse{25}
Ich will meinen Eifer über dich gehen lassen, daß sie unbarmherzig mit
dir handeln sollen. Sie sollen dir Nasen und Ohren abschneiden; und was
übrig bleibt, soll durchs Schwert fallen. Sie sollen deine Söhne und
Töchter wegnehmen und das Übrige mit Feuer verbrennen. \bibverse{26} Sie
sollen dir deine Kleider ausziehen und deinen Schmuck wegnehmen.
\bibverse{27} Also will ich deiner Unzucht und deiner Hurerei mit
Ägyptenland ein Ende machen, daß du deine Augen nicht mehr nach ihnen
aufheben und Ägyptens nicht mehr gedenken sollst. \bibverse{28} Denn so
spricht der HErr HErr: Siehe, ich will dich überantworten, denen du
feind worden und deren du müde bist. \bibverse{29} Die sollen als Feinde
mit dir umgehen und alles nehmen, was du erworben hast, und dich nackend
und bloß lassen, daß deine Scham aufgedeckt werde samt deiner Unzucht
und Hurerei. \bibverse{30} Solches wird dir geschehen um deiner Hurerei
willen, so du mit den Heiden getrieben, an welcher Götzen du dich
verunreiniget hast. \bibverse{31} Du bist auf dem Wege deiner Schwester
gegangen; darum gebe ich dir auch derselbigen Kelch in deine Hand.
\bibverse{32} So spricht der HErr HErr: Du mußt den Kelch deiner
Schwester trinken, so tief und weit er ist; du sollst so zu großem Spott
und Hohn werden, daß es unerträglich sein wird. \bibverse{33} Du mußt
dich des starken Tranks und Jammers voll saufen; denn der Kelch deiner
Schwester Samaria ist ein Kelch des Jammers und Trauerns. \bibverse{34}
Denselben mußt du rein austrinken, danach die Scherben zerwerfen und
deine Brüste zerreißen; denn ich hab es geredet, spricht der HErr HErr.
\bibverse{35} Darum so spricht der HErr HErr: Darum daß du mein
vergessen und mich hinter deinen Rücken geworfen hast, so trage auch nun
deine Unzucht und deine Hurerei. \bibverse{36} Und der Herr sprach zu
mir: Du Menschenkind, willst du Ahala und Ahaliba strafen, so zeige
ihnen an ihre Greuel, \bibverse{37} wie sie Ehebrecherei getrieben und
Blut vergossen und die Ehe gebrochen haben mit den Götzen; dazu ihre
Kinder, die sie mir gezeuget hatten, verbrannten sie denselben zum
Opfer. \bibverse{38} Über das haben sie mir das getan: sie haben meine
Heiligtümer verunreiniget dazumal und meine Sabbate entheiliget.
\bibverse{39} Denn da sie ihre Kinder den Götzen geschlachtet hatten,
gingen sie desselbigen Tages in mein Heiligtum, dasselbige zu
entheiligen. Siehe, solches haben sie in meinem Hause begangen.
\bibverse{40} Sie haben auch Boten geschickt nach Leuten, die aus fernen
Landen kommen sollten; und siehe, da sie kamen, badetest du dich und
schminktest dich und schmücktest dich mit Geschmeide, ihnen zu Ehren,
\bibverse{41} und saßest auf einem herrlichen Bette, vor welchem stund
ein Tisch zugerichtet; darauf räuchertest du und opfertest mein Öl
darauf. \bibverse{42} Daselbst hub sich ein groß Freudengeschrei; und
sie gaben den Leuten, so allenthalben aus großem Volk und aus der Wüste
kommen waren, Geschmeide an ihre Arme und schöne Kronen auf ihre
Häupter. \bibverse{43} Ich aber gedachte: Sie ist der Ehebrecherei
gewohnt von alters her, sie kann von der Hurerei nicht lassen.
\bibverse{44} Denn man geht zu ihr ein, wie man zu einer Hure eingeht;
ebenso geht man zu Ahala und Ahaliba, den unzüchtigen Weibern.
\bibverse{45} Darum werden sie die Männer strafen, die das Recht
vollbringen, wie man die Ehebrecherinnen und Blutvergießerinnen strafen
soll; denn sie sind Ehebrecherinnen, und ihre Hände sind voll Blut.
\bibverse{46} Also spricht der HErr HErr: Führe einen großen Haufen über
sie herauf und gib sie in die Rapuse und Raub, \bibverse{47} die sie
steinigen und mit ihren Schwertern erstechen und ihre Söhne und Töchter
erwürgen und ihre Häuser mit Feuer verbrennen. \bibverse{48} Also will
ich der Unzucht im Lande ein Ende machen, daß sich alle Weiber daran
stoßen sollen und nicht nach solcher Unzucht tun. \bibverse{49} Und man
soll eure Unzucht auf euch legen, und sollt eurer Götzen Sünde tragen,
auf daß ihr erfahret, daß ich der HErr HErr bin.

\hypertarget{section-23}{%
\section{24}\label{section-23}}

\bibverse{1} Und es geschah das Wort des HErrn zu mir im neunten Jahr,
am zehnten Tage des zehnten Monden, und sprach: \bibverse{2} Du
Menschenkind, schreibe diesen Tag an, ja eben diesen Tag; denn der König
zu Babel hat sich eben an diesem Tage wider Jerusalem gerüstet.
\bibverse{3} Und gib dem ungehorsamen Volk ein Gleichnis und sprich zu
ihnen: So spricht der HErr HErr: Setze einen Topf zu; setze zu und gieß
Wasser darein! \bibverse{4} Tue die Stücke zusammen darein, die hinein
sollen, und die besten Stücke, die Lenden und Schultern, und fülle ihn
mit den besten Markstücken. \bibverse{5} Nimm das Beste von der Herde
und mache ein Feuer darunter, Markstücke zu kochen, und laß es getrost
sieden und die Markstücke drinnen wohl kochen. \bibverse{6} Darum
spricht der HErr HErr: O der mörderischen Stadt, die ein solcher Topf
ist, da das Angebrannte drinnen klebet, und nicht abgehen will! Tue ein
Stück nach dem andern heraus, und darfst nicht darum losen, welches erst
heraus solle. \bibverse{7} Denn ihr Blut ist drinnen, das sie auf einen
bloßen Felsen und nicht auf die Erde verschüttet hat, da man's doch
hätte mit Erde können zuscharren. \bibverse{8} Und ich hab auch darum
sie lassen dasselbige Blut auf einen bloßen Felsen schütten, daß es
nicht zugescharret würde, auf daß der Grimm über sie käme und gerochen
würde. \bibverse{9} Darum spricht der HErr HErr also: O du mörderische
Stadt, welche ich will zu einem großen Feuer machen! \bibverse{10} Trage
nur viel Holz her, zünde das Feuer an, daß das Fleisch gar werde, und
würze es wohl, daß die Markstücke anbrennen! \bibverse{11} Lege auch den
Topf leer auf die Glut, auf daß er heiß werde und sein Erz entbrenne, ob
seine Unreinigkeit zerschmelzen und sein Angebranntes abgehen wollte.
\bibverse{12} Aber das Angebrannte, wie fast es brennet, will nicht
abgehen, denn es ist zu sehr angebrannt; es muß im Feuer verschmelzen.
\bibverse{13} Deine Unreinigkeit ist so verhärtet, daß, ob ich dich
gleich gerne reinigen wollte, dennoch du nicht willst dich reinigen
lassen von deiner Unreinigkeit. Darum kannst du fort nicht wieder rein
werden, bis mein Grimm sich an dir gekühlet habe. \bibverse{14} Ich, der
HErr, hab es geredet; es soll kommen. Ich will's tun und nicht säumen;
ich will nicht schonen noch mich's reuen lassen, sondern sie sollen dich
richten, wie du gelebt und getan hast, spricht der HErr HErr.
\bibverse{15} Und des HErrn Wort geschah zu mir und sprach:
\bibverse{16} Du Menschenkind, siehe, ich will dir deiner Augen Lust
nehmen durch eine Plage. Aber du sollst nicht klagen noch weinen, noch
eine Träne lassen. \bibverse{17} Heimlich magst du seufzen, aber keine
Totenklage führen, sondern du sollst deinen Schmuck anlegen und deine
Schuhe anziehen. Du sollst deinen Mund nicht verhüllen und nicht das
Trauerbrot essen. \bibverse{18} Und da ich des Morgens früh zum Volk
redete, starb mir zu Abend mein Weib. Und ich tat des andern Morgens,
wie mir befohlen war. \bibverse{19} Und das Volk sprach zu mir: Willst
du uns denn nicht anzeigen, was uns das bedeute, das du tust?
\bibverse{20} Und ich sprach zu ihnen: Der HErr hat mit mir geredet und
gesagt: \bibverse{21} Sage dem Hause Israel, daß der HErr HErr spricht
also: Siehe, ich will mein Heiligtum, euren höchsten Trost, die Lust
eurer Augen und eures Herzens Wunsch, entheiligen; und eure Söhne und
Töchter, die ihr verlassen müsset, werden durchs Schwert fallen;
\bibverse{22} und müsset tun, wie ich getan habe: euren Mund müsset ihr
nicht verhüllen und das Trauerbrot nicht essen, \bibverse{23} sondern
müsset euren Schmuck auf euer Haupt setzen und eure Schuhe anziehen. Ihr
werdet nicht klagen noch weinen, sondern über euren Sünden verschmachten
und untereinander seufzen. \bibverse{24} Und soll also Hesekiel euch ein
Wunder sein, daß ihr tun müsset, wie er getan hat, wenn es nun kommen
wird, damit ihr erfahret, daß ich der HErr HErr bin. \bibverse{25} Und
du, Menschenkind, zu der Zeit, wenn ich wegnehmen werde von ihnen ihre
Macht und Trost, die Lust ihrer Augen und ihres Herzens Wunsch, ihre
Söhne und Töchter, \bibverse{26} ja, zur selbigen Zeit wird einer, so
entronnen ist, zu dir kommen und dir's kundtun. \bibverse{27} Zur
selbigen Zeit wird dein Mund aufgetan werden samt dem, der entronnen
ist, daß du reden sollst und nicht mehr schweigen; denn du mußt ihr
Wunder sein, daß sie erfahren, ich sei der HErr.

\hypertarget{section-24}{%
\section{25}\label{section-24}}

\bibverse{1} Und des HErrn Wort geschah zu mir und sprach: \bibverse{2}
Du Menschenkind, richte dein Angesicht gegen die Kinder Ammon und
weissage wider sie. \bibverse{3} Und sprich zu den Kindern Ammon: Höret
des HErrn HErrn Wort! So spricht der HErr HErr: Darum daß ihr über mein
Heiligtum sprechet: Heh, es ist entheiliget! und über das Land Israel:
Es ist verwüstet! und über das Haus Juda: Es ist gefangen weggeführt!
\bibverse{4} darum siehe, ich will dich den Kindern gegen Morgen
übergeben, daß sie ihre Schlösser drinnen bauen und ihre Wohnung drinnen
machen sollen; sie sollen deine Früchte essen und deine Milch trinken.
\bibverse{5} Und will Rabbath zum Kamelstall machen und die Kinder Ammon
zur Schafhürde machen; und sollet erfahren, daß ich der HErr bin.
\bibverse{6} Denn so spricht der HErr HErr: Darum daß du mit deinen
Händen geklatschet und mit den Füßen gescharret und über das Land Israel
von ganzem Herzen so höhnisch dich gefreuet hast, \bibverse{7} darum
siehe, ich will meine Hand über dich ausstrecken und dich den Heiden zur
Beute geben und dich aus den Völkern ausrotten und aus den Ländern
umbringen und dich vertilgen; und sollst erfahren, daß ich der HErr bin.
\bibverse{8} So spricht der HErr HErr: Darum daß Moab und Seir sprechen:
Siehe, das Haus Juda ist eben wie alle andern Heiden, \bibverse{9}
siehe, so will ich Moab zur Seite öffnen in seinen Städten und in seinen
Grenzen des edlen Landes, nämlich Beth-Jesimoth, Baal-Meon und
Kiriathaim, \bibverse{10} den Kindern gegen Morgen samt den Kindern
Ammon, und will sie ihnen zum Erbe geben, daß man der Kinder Ammon nicht
mehr gedenken soll unter den Heiden. \bibverse{11} Und will das Recht
gehen lassen über Moab, und sollen erfahren, daß ich der HErr bin.
\bibverse{12} So spricht der HErr HErr: Darum daß sich Edom am Hause
Juda gerächt hat und damit sich verschuldet mit ihrem Rächen,
\bibverse{13} darum spricht der HErr HErr also: Ich will meine Hand
ausstrecken über Edom und will ausrotten von ihm beide, Menschen und
Vieh; und will sie wüste machen von Theman bis gen Dedan und durchs
Schwert fällen. \bibverse{14} Und will mich an Edom rächen durch mein
Volk Israel, und sollen mit Edom umgehen nach meinem Zorn und Grimm, daß
sie meine Rache erfahren sollen, spricht der HErr HErr. \bibverse{15} So
spricht der HErr HErr: darum daß die Philister sich gerächt haben und
den alten Haß gebüßet nach all ihrem Willen am Schaden (meines Volks),
\bibverse{16} darum spricht der HErr HErr also: Siehe, ich will meine
Hand ausstrecken über die Philister und die Krieger ausrotten und will
die Übrigen am Hafen des Meers umbringen \bibverse{17} und will große
Rache an ihnen üben und mit Grimm sie strafen, daß sie erfahren sollen,
ich sei der HErr, wenn ich meine Rache an ihnen geübet habe.

\hypertarget{section-25}{%
\section{26}\label{section-25}}

\bibverse{1} Und es begab sich im elften Jahr, am ersten Tage des ersten
Monden, geschah des HErrn Wort zu mir und sprach: \bibverse{2} Du
Menschenkind, darum daß Tyrus spricht über Jerusalem: Heh, die Pforten
der Völker sind zerbrochen, es ist zu mir gewandt; ich werde nun voll
werden, weil sie wüste ist: \bibverse{3} darum spricht der HErr HErr
also: Siehe, ich will an dich, Tyrus, und will viel Heiden über dich
heraufbringen, gleichwie sich ein Meer erhebt mit seinen Wellen.
\bibverse{4} Die sollen die Mauern zu Tyrus verderben und ihre Türme
abbrechen; ja, ich will auch den Staub vor ihr wegfegen und will einen
bloßen Fels aus ihr machen \bibverse{5} und zu einem Werd im Meer,
darauf man die Fischgarne ausspannet; denn ich hab es geredet, spricht
der HErr HErr: und sie soll den Heiden zum Raub werden. \bibverse{6} Und
ihre Töchter, so auf dem Felde liegen, sollen durchs Schwert erwürget
werden; und sollen erfahren, daß ich der HErr bin. \bibverse{7} Denn so
spricht der HErr HErr: Siehe, ich will über Tyrus kommen lassen
Nebukadnezar, den König zu Babel, von Mitternacht her, der ein König
aller Könige ist, mit Rossen, Wagen, Reitern und mit großem Haufen
Volks. \bibverse{8} Der soll deine Töchter, so auf dem Felde liegen, mit
dem Schwert erwürgen; aber wider dich wird er Bollwerk aufschlagen und
einen Schutt machen und Schilde wider dich rüsten. \bibverse{9} Er wird
mit Böcken deine Mauern zerstoßen und deine Türme mit seinen Waffen
umreißen. \bibverse{10} Der Staub von der Menge seiner Pferde wird dich
bedecken, so werden auch deine Mauern erbeben vor dem Getümmel seiner
Rosse, Räder und Reiter, wenn er zu deinen Toren einziehen wird, wie man
pflegt in eine zerrissene Stadt einzuziehen. \bibverse{11} Er wird mit
den Füßen seiner Rosse alle deine Gassen zertreten. Dein Volk wird er
mit dem Schwert erwürgen und deine starken Säulen zu Boden reißen.
\bibverse{12} Sie werden dein Gut rauben und deinen Handel plündern.
Deine Mauern werden sie abbrechen und deine feinen Häuser umreißen und
werden deine Steine, Holz und Staub ins Wasser werfen. \bibverse{13}
Also will ich mit dem Getöne deines Gesangs ein Ende machen, daß man den
Klang deiner Harfen nicht mehr hören soll. \bibverse{14} Und ich will
einen bloßen Fels aus dir machen und einen Werd, darauf man die
Fischgarne ausspannet, daß du nicht mehr gebauet werdest; denn ich bin
der HErr, der solches redet, spricht der HErr HErr. \bibverse{15} So
spricht der HErr HErr wider Tyrus: Was gilt's, die Inseln werden
erbeben, wenn du so scheußlich zerfallen wirst und deine Verwundeten
seufzen werden, so in dir sollen ermordet werden. \bibverse{16} Alle
Fürsten am Meer werden herab von ihren Stühlen sitzen und ihre Röcke von
sich tun und ihre gestickten Kleider ausziehen und werden in
Trauerkleidern gehen und auf der Erde sitzen und werden erschrecken und
sich entsetzen deines plötzlichen Falls. \bibverse{17} Sie werden dich
wehklagen und von dir sagen: Ach, wie bist du so gar wüste worden, du
berühmte Stadt, die du am Meer lagest und so mächtig warest auf dem Meer
samt deinen Einwohnern, daß sich das ganze Land vor dir fürchten mußte!
\bibverse{18} Ach, wie entsetzen sich die Inseln über deinen Fall! Ja,
die Inseln im Meer erschrecken über deinen Untergang. \bibverse{19} So
spricht der HErr HErr: Ich will dich zu einer wüsten Stadt machen, wie
andere Städte, da niemand innen wohnet, und eine große Flut über dich
kommen lassen, daß dich große Wasser bedecken. \bibverse{20} Und will
dich hinunterstoßen zu denen, die in die Grube fahren, nämlich zu den
Toten. Ich will dich unter die Erde hinabstoßen und wie eine ewige Wüste
machen mit denen, die in die Grube fahren, auf daß niemand in dir wohne.
Ich will dich, du Zarte, im Lande der Lebendigen machen, \bibverse{21}
ja, zum Schrecken will ich dich machen, daß du nichts mehr seiest, und
wenn man nach dir fraget, daß man dich ewiglich nimmer finden könne,
spricht der HErr HErr.

\hypertarget{section-26}{%
\section{27}\label{section-26}}

\bibverse{1} Und des HErrn Wort geschah zu mir und sprach: \bibverse{2}
Du Menschenkind, mache eine Wehklage über Tyrus \bibverse{3} und sprich
zu Tyrus, die da liegt vorne am Meer und mit vielen Inseln der Völker
handelt: So spricht der HErr HErr: O Tyrus, du sprichst: Ich bin die
allerschönste. \bibverse{4} Deine Grenzen sind mitten im Meer, und deine
Bauleute haben dich aufs allerschönste zugerichtet. \bibverse{5} Sie
haben all dein Tafelwerk aus Zypressenholz vom Senir gemacht und die
Zedern von dem Libanon führen lassen und deine Mastbäume daraus gemacht
\bibverse{6} und deine Ruder von Eichen aus Basan und deine Bänke von
Elfenbein und die köstlichen Gestühle aus den Inseln Chittim.
\bibverse{7} Dein Segel war von gestickter Seide aus Ägypten, daß es
dein Panier wäre, und deine Decken von gelber Seide und Purpur aus den
Inseln Elisa. \bibverse{8} Die von Zidon und Arvad waren deine
Ruderknechte, und hattest geschickte Leute zu Tyrus zu schiffen.
\bibverse{9} Die Ältesten und Klugen von Gebal mußten deine Schiffe
zimmern. Alle Schiffe im Meer und Schiffsleute fand man bei dir, die
hatten ihren Handel in dir. \bibverse{10} Die aus Persien, Lydien und
Libyen waren dein Kriegsvolk, die ihren Schild und Helm in dir
aufhingen, und haben dich so schön gemacht. \bibverse{11} Die von Arvad
waren unter deinem Heer rings um deine Mauern und Wächter auf deinen
Türmen; die haben ihre Schilde allenthalben von deinen Mauern
herabgehänget und dich so schön gemacht. \bibverse{12} Du hast deinen
Handel auf dem Meer gehabt und allerlei Ware, Silber, Eisen, Zinn und
Blei, auf deine Märkte gebracht. \bibverse{13} Javan, Thubal und Mesech
haben mit dir gehandelt und haben dir leibeigene Leute und Erz auf deine
Märkte gebracht. \bibverse{14} Die von Thogarma haben dir Pferde und
Wagen und Maulesel auf deine Märkte gebracht. \bibverse{15} Die von
Dedan sind deine Kaufleute gewesen, und hast allenthalben in den Inseln
gehandelt; die haben dir Elfenbein und Ebenholz verkauft. \bibverse{16}
Die Syrer haben bei dir geholet deine Arbeit, was du gemacht hast; und
Rubin, Purpur, Tapet, Seide und Sammet und Kristalle auf deine Märkte
gebracht. \bibverse{17} Juda und das Land Israel haben auch mit dir
gehandelt und haben dir Weizen von Minnith und Balsam und Honig und Öl
und Mastix auf deine Märkte gebracht; \bibverse{18} Dazu hat auch
Damaskus bei dir geholet deine Arbeit und allerlei Ware um starken Wein
und köstliche Wolle. \bibverse{19} Dan und Javan und Mehusal haben auch
auf deine Märkte gebracht Eisenwerk, Kasia und Kalmus, daß du damit
handeltest. \bibverse{20} Dedan hat mit dir gehandelt mit Decken, darauf
man sitzet. \bibverse{21} Arabien und alle Fürsten von Kedar haben mit
dir gehandelt mit Schafen Widdern und Böcken. \bibverse{22} Die
Kaufleute aus Saba und Raema haben mit dir gehandelt und allerlei
köstliche Spezerei und Edelstein und Gold auf deine Märkte gebracht.
\bibverse{23} Haran und Kanne und Eden samt den Kaufleuten aus Seba,
Assur und Kilmad sind auch deine Kaufleute gewesen. \bibverse{24} Die
haben alle mit dir gehandelt mit köstlichem Gewand, mit seidenen und
gestickten Tüchern, welche sie in köstlichen Kasten, von Zedern gemacht
und wohlverwahrt, auf deine Märkte geführet haben. \bibverse{25} Aber
die Meerschiffe sind die vornehmsten auf deinen Märkten gewesen. Also
bist du sehr reich und prächtig worden mitten im Meer. \bibverse{26} Und
deine Schiffsleute haben dir auf großen Wassern zugeführet. Aber ein
Ostwind wird dich mitten auf dem Meer zerbrechen, \bibverse{27} also daß
deine Ware, Kaufleute, Händler, Fergen, Schiffsherren und die, so die
Schiffe machen, und deine Hantierer und alle deine Kriegsleute und alles
Volk in dir mitten auf dem Meer umkommen werden zur Zeit, wenn du
untergehest, \bibverse{28} daß auch die Anfurten erbeben werden vor dem
Geschrei deiner Schiffsherren. \bibverse{29} Und alle, die an den Rudern
ziehen, samt den Schiffsknechten und Meistern, werden aus den Schiffen
ans Land treten \bibverse{30} und laut über dich schreien, bitterlich
klagen und werden Staub auf ihre Häupter werfen, und sich in der Asche
wälzen. \bibverse{31} Sie werden sich kahl bescheren über dir und Säcke
um sich gürten und von Herzen bitterlich um dich weinen und trauern.
\bibverse{32} Es werden auch ihre Kinder über dich klagen: Ach, wer ist
jemals auf dem Meere so stille worden wie du, Tyrus? \bibverse{33} Da du
deinen Handel auf dem Meer triebest, da machtest du viel Länder reich;
ja, mit der Menge deiner Ware und deiner Kaufmannschaft machtest du
reich die Könige auf Erden. \bibverse{34} Nun aber bist du vom Meer in
die recht tiefen Wasser gestürzt, daß dein Handel und all dein Volk in
dir umkommen ist. \bibverse{35} Alle, die in Inseln wohnen, erschrecken
über dir, und ihre Könige entsetzen sich und sehen jämmerlich.
\bibverse{36} Die Kaufleute in Ländern pfeifen dich an, daß du so
plötzlich untergegangen bist und nicht mehr aufkommen kannst.

\hypertarget{section-27}{%
\section{28}\label{section-27}}

\bibverse{1} Und des HErrn Wort geschah zu mir und sprach: \bibverse{2}
Du Menschenkind, sage dem Fürsten zu Tyrus: So spricht der HErr HErr:
Darum daß sich dein Herz erhebt und spricht: Ich bin GOtt, ich sitze im
Thron GOttes, mitten auf dem Meer, so du doch ein Mensch und nicht GOtt
bist; noch erhebt sich dein Herz als ein Herz GOttes; \bibverse{3}
siehe, du hältst dich für klüger denn Daniel, daß dir nichts verborgen
sei, \bibverse{4} und habest durch deine Klugheit und Verstand solche
Macht zuwegegebracht und Schätze von Gold und Silber gesammelt
\bibverse{5} und habest durch deine große Weisheit und Hantierung so
große Macht überkommen, davon bist du so stolz worden, daß du so mächtig
bist: \bibverse{6} darum spricht der HErr HErr also: Weil sich denn dein
Herz erhebt als ein Herz GOttes, \bibverse{7} darum siehe, ich will
Fremde über dich schicken, nämlich die Tyrannen der Heiden; die sollen
ihr Schwert zücken über deine schöne Weisheit und deine große Ehre
zuschanden machen. \bibverse{8} Sie sollen dich hinunter in die Grube
stoßen, daß du mitten auf dem Meer sterbest, wie die Erschlagenen.
\bibverse{9} Was gilt's, ob du dann vor deinem Totschläger werdest
sagen: Ich bin GOtt, so du doch nicht GOtt, sondern ein Mensch und in
deiner Totschläger Hand bist? \bibverse{10} Du sollst sterben wie die
Unbeschnittenen, von der Hand der Fremden; denn ich hab es geredet,
spricht der HErr HErr. \bibverse{11} Und des HErrn Wort geschah zu mir
und sprach: \bibverse{12} Du Menschenkind, mache eine Wehklage über den
König zu Tyrus und sprich von ihm: So spricht der HErr HErr: Du bist ein
reinlich Siegel voller Weisheit und aus der Maßen schön. \bibverse{13}
Du bist im Lustgarten GOttes und mit allerlei Edelsteinen geschmückt,
nämlich mit Sarder, Topaser, Demanten, Türkis, Onyxen, Jaspis, Saphir,
Amethyst, Smaragden und Gold. Am Tage, da du geschaffen wurdest, mußten
da bereit sein bei dir dein Paukenwerk und Pfeifen. \bibverse{14} Du
bist wie ein Cherub, der sich weit ausbreitet und decket; und ich habe
dich auf den heiligen Berg GOttes gesetzt, daß du unter den feurigen
Steinen wandelst, \bibverse{15} und warest ohne Wandel in deinem Tun des
Tages, da du geschaffen warest, so lange, bis sich deine Missetat funden
hat. \bibverse{16} Denn du bist inwendig voll Frevels worden vor deiner
großen Hantierung und hast dich versündiget. Darum will ich dich
entheiligen von dem Berge GOttes und will dich ausgebreiteten Cherub aus
den feurigen Steinen verstoßen. \bibverse{17} Und weil sich dein Herz
erhebt, daß du so schön bist, und hast dich deine Klugheit lassen
betrügen in deiner Pracht, darum will ich dich zu Boden stürzen und ein
Schauspiel aus dir machen vor den Königen. \bibverse{18} Denn du hast
dein Heiligtum verderbet mit deiner großen Missetat und unrechtem
Handel. Darum will ich ein Feuer aus dir angehen lassen, das dich soll
verzehren, und will dich zu Asche machen auf der Erde, daß alle Welt
zusehen soll. \bibverse{19} Alle, die dich kennen unter den Heiden,
werden sich über dir entsetzen, daß du so plötzlich bist untergegangen
und nimmermehr aufkommen kannst. \bibverse{20} Und des HErrn Wort
geschah zu mir und sprach: \bibverse{21} Du Menschenkind, richte dein
Angesicht wider Zidon und weissage wider sie \bibverse{22} und sprich:
So spricht der HErr HErr: Siehe, ich will an dich, Zidon, und will an
dir Ehre einlegen, daß man erfahren soll, daß ich der HErr bin, wenn ich
das Recht über sie gehen lasse und an ihr erzeige, daß ich heilig sei.
\bibverse{23} Und ich will Pestilenz und Blutvergießen unter sie
schicken auf ihren Gassen, und sollen tödlich verwundet drinnen fallen
durchs Schwert, welches allenthalben über sie gehen wird; und sollen
erfahren, daß ich der HErr bin. \bibverse{24} Und soll forthin
allenthalben um das Haus Israel, da ihre Feinde sind, kein Dorn, der da
sticht, noch Stachel, der da weh tut, bleiben, daß sie erfahren sollen,
daß ich der HErr HErr bin. \bibverse{25} So spricht der HErr HErr: Wenn
ich das Haus Israel wieder versammeln werde von den Völkern, dahin sie
zerstreuet sind, so will ich vor den Heiden an ihnen erzeigen, daß ich
heilig bin. Und sie sollen wohnen in ihrem Lande, das ich meinem Knechte
Jakob gegeben habe; \bibverse{26} und sollen sicher darin wohnen und
Häuser bauen und Weinberge pflanzen; ja, sicher sollen sie wohnen, wenn
ich das Recht gehen lasse über alle ihre Feinde um und um; und sollen
erfahren, daß ich der HErr, ihr GOtt, bin.

\hypertarget{section-28}{%
\section{29}\label{section-28}}

\bibverse{1} Im zehnten Jahr, am zehnten Tage des zwölften Monden,
geschah des HErrn Wort zu mir und sprach: \bibverse{2} Du Menschenkind,
richte dein Angesicht wider Pharao, den König in Ägypten, und weissage
wider ihn und wider ganz Ägyptenland. \bibverse{3} Predige und sprich:
So spricht der HErr HErr: Siehe, ich will an dich, Pharao, du König in
Ägypten, du großer Drache, der du in deinem Wasser liegst und sprichst:
Der Strom ist mein, und ich habe ihn mir gemacht. \bibverse{4} Aber ich
will dir ein Gebiß ins Maul legen und die Fische in deinen Wassern an
deine Schuppen hängen und will dich aus deinem Strom herausziehen samt
allen Fischen in deinen Wassern, die an deinen Schuppen hangen.
\bibverse{5} Ich will dich mit den Fischen aus deinen Wassern in die
Wüste wegwerfen: du wirst aufs Land fallen und nicht wieder aufgelesen
noch gesammelt werden, sondern den Tieren auf dem Lande und den Vögeln
des Himmels zum Aas werden. \bibverse{6} Und alle, die in Ägypten
wohnen, sollen erfahren, daß ich der HErr bin, darum daß sie dem Hause
Israel ein Rohrstab gewesen sind, \bibverse{7} welcher, wenn sie ihn in
die Hand faßeten, so brach er und stach sie durch die Seiten, wenn sie
sich aber darauf lehneten, so zerbrach er und stach sie in die Lenden.
\bibverse{8} Darum spricht der HErr HErr also: Siehe, ich will das
Schwert über dich kommen lassen und beide, Leute und Vieh, in dir
ausrotten. \bibverse{9} Und Ägyptenland soll zur Wüste und öde werden,
und sollen erfahren, daß ich der HErr sei, darum daß er spricht: Der
Wasserstrom ist mein, und ich bin's, der es tut. \bibverse{10} Darum
siehe, ich will an dich und an deine Wasserströme und will Ägyptenland
wüst und öde machen von dem Turm zu Syene an bis an die Grenze des
Mohrenlandes, \bibverse{11} daß weder Vieh noch Leute darin gehen oder
da wohnen sollen vierzig Jahre lang. \bibverse{12} Denn ich will
Ägyptenland wüst machen, will ihre wüste Grenze und ihre Städte wüst
liegen lassen, wie andere wüste Städte, vierzig Jahre lang; und will die
Ägypter zerstreuen unter die Heiden und in die Länder will ich sie
verjagen. \bibverse{13} Doch, so spricht der HErr HErr: Wenn die vierzig
Jahre aus sein werden, will ich die Ägypter wieder sammeln aus den
Völkern, darunter sie zerstreuet sollen werden, \bibverse{14} und will
das Gefängnis Ägyptens wenden und sie wiederum ins Land Pathros bringen,
welches ihr Vaterland ist, und sollen daselbst ein klein Königreich
sein. \bibverse{15} Denn sie sollen klein sein gegen andere Königreiche
und nicht mehr herrschen über die Heiden; und ich will sie gering
machen, daß sie nicht mehr über die Heiden herrschen sollen,
\bibverse{16} daß sich das Haus Israel nicht mehr auf sie verlasse und
sich damit versündige, wenn sie sich an sie hängen; und sollen erfahren,
daß ich der HErr HErr hin. \bibverse{17} Und es begab sich im
siebenundzwanzigsten Jahr, am ersten Tage des ersten Monden, geschah des
HErrn Wort zu mir und sprach: \bibverse{18} Du Menschenkind,
Nebukadnezar, der König zu Babel, hat sein Heer mit großer Mühe vor
Tyrus geführet, daß alle Häupter kahl und alle Seiten wund gerieben
waren, und ist doch weder ihm noch seinem Heer seine Arbeit vor Tyrus
belohnet worden. \bibverse{19} Darum spricht der HErr HErr also: Siehe,
ich will Nebukadnezar, dem Könige zu Babel, Ägyptenland geben, daß er
all ihr Gut wegnehmen und sie berauben und plündern soll, daß er seinem
Heer den Sold gebe. \bibverse{20} Aber das Land Ägypten will ich ihm
geben für seine Arbeit, die er daran getan hat; denn sie haben mir
gedienet, spricht der HErr HErr. \bibverse{21} Zur selbigen Zeit will
ich das Horn des Hauses Israel wachsen lassen und will deinen Mund unter
ihnen auftun, daß sie erfahren, daß ich der HErr bin.

\hypertarget{section-29}{%
\section{30}\label{section-29}}

\bibverse{1} Und des HErrn Wort geschah zu mir und sprach: \bibverse{2}
Du Menschenkind, weissage und sprich: So spricht der HErr HErr: Heulet
(und sprechet): O weh des Tages! \bibverse{3} Denn der Tag ist nahe, ja,
des HErrn Tag ist nahe, ein finsterer Tag; die Zeit ist da, daß die
Heiden kommen sollen. \bibverse{4} Und das Schwert soll über Ägypten
kommen, und Mohrenland muß erschrecken, wenn die Erschlagenen in Ägypten
fallen werden, und ihr Volk weggeführet und ihre Grundfesten umgerissen
werden. \bibverse{5} Mohrenland und Libyen und Lydien mit allerlei Pöbel
und Chub und die aus dem Lande des Bundes sind, sollen samt ihnen durchs
Schwert fallen. \bibverse{6} So spricht der HErr: Die Schutzherren
Ägyptens müssen fallen, und die Hoffart ihrer Macht muß herunter; von
dem Turm zu Syene an sollen sie durchs Schwert fallen, spricht der HErr
HErr, \bibverse{7} und sollen, wie ihre wüste Grenze, wüst werden, und
ihre Städte unter andern wüsten Städten wüst liegen, \bibverse{8} daß
sie erfahren, daß ich der HErr sei, wenn ich ein Feuer in Ägypten mache,
daß alle, die ihnen helfen, zerstöret werden. \bibverse{9} Zur selben
Zeit werden Boten von mir ausziehen in Schiffen, Mohrenland zu
schrecken, das jetzt so sicher ist, und wird ein Schrecken unter ihnen
sein, gleichwie es Ägypten ging, da ihre Zeit kam; denn siehe, es kommt
gewißlich. \bibverse{10} So spricht der HErr HErr: Ich will die Menge in
Ägypten wegräumen durch Nebukadnezar, den König zu Babel. \bibverse{11}
Denn er und sein Volk mit ihm, samt den Tyrannen der Heiden, sind
herzugebracht, das Land zu verderben, und werden ihre Schwerter
ausziehen wider Ägypten, daß das Land allenthalben voll Erschlagener
liege. \bibverse{12} Und ich will die Wasserströme trocken machen und
das Land bösen Leuten verkaufen und will das Land, und was drinnen ist,
durch Fremde verwüsten. Ich, der HErr, hab es geredet. \bibverse{13} So
spricht der HErr HErr: Ich will die Götzen zu Noph ausrotten und die
Abgötter vertilgen, und Ägypten soll keinen Fürsten mehr haben; und will
ein Schrecken in Ägyptenland schicken. \bibverse{14} Ich will Pathros
wüst machen und ein Feuer zu Zoan anzünden und das Recht über No gehen
lassen. \bibverse{15} Ich will meinen Grimm ausschütten über Sin, welche
ist eine Festung Ägyptens, und will die Menge zu No ausrotten.
\bibverse{16} Ich will ein Feuer in Ägypten anzünden, und Sin soll angst
und bange werden; und No soll zerrissen und Noph täglich geängstet
werden. \bibverse{17} Die junge Mannschaft zu On und Bubasto sollen
durchs Schwert fallen, und die Weiber gefangen weggeführet werden.
\bibverse{18} Tachpanhes wird einen finstern Tag haben, wenn ich das
Joch Ägyptens schlagen werde, daß die Hoffart ihrer Macht darinnen ein
Ende habe; sie wird mit Wolken bedeckt werden, und ihre Töchter werden
gefangen weggeführet werden. \bibverse{19} Und ich will das Recht über
Ägypten gehen lassen, daß sie erfahren, daß ich der HErr sei.
\bibverse{20} Und es begab sich im elften Jahr, am siebenten Tage des
ersten Monden, geschah des HErrn Wort zu mir und sprach: \bibverse{21}
Du Menschenkind, ich will den Arm Pharaos, des Königs in Ägypten,
zerbrechen; und siehe, er soll nicht verbunden werden, daß er heilen
möge, noch mit Binden zugebunden werden, daß er stark werde und ein
Schwert fassen könne. \bibverse{22} Darum spricht der HErr HErr also:
Siehe, ich will an Pharao, den König in Ägypten, und will seine Arme
zerbrechen, beide, den starken und den schwachen, daß ihm das Schwert
aus seiner Hand entfallen muß. \bibverse{23} Und will die Ägypter unter
die Heiden zerstreuen und in die Länder verjagen. \bibverse{24} Aber die
Arme des Königs zu Babel will ich stärken und ihm mein Schwert in seine
Hand geben; und will die Arme Pharaos zerbrechen, daß er vor ihm winseln
soll wie ein tödlich Verwundeter. \bibverse{25} Ja, ich will die Arme
des Königs zu Babel stärken, daß die Arme Pharaos dahinfallen, auf daß
sie erfahren, daß ich der HErr sei, wenn ich mein Schwert dem Könige zu
Babel in die Hand gebe, daß er's über Ägyptenland zücke, \bibverse{26}
und ich die Ägypter unter die Heiden zerstreue und in die Länder
verjage, daß sie erfahren, daß ich der HErr bin.

\hypertarget{section-30}{%
\section{31}\label{section-30}}

\bibverse{1} Und es begab sich im elften Jahr, am ersten Tage des
dritten Monden, geschah des HErrn Wort zu mir und sprach: \bibverse{2}
Du Menschenkind, sage zu Pharao, dem Könige zu Ägypten, und zu all
seinem Volk: Wem meinest du denn, daß du gleich seiest in deiner
Herrlichkeit? \bibverse{3} Siehe, Assur war wie ein Zedernbaum auf dem
Libanon, von schönen Ästen und dick von Laub und sehr hoch, daß sein
Wipfel hoch stund unter großen, dicken Zweigen. \bibverse{4} Die Wasser
machten, daß er groß ward, und die Tiefe, daß er hoch wuchs. Seine
Ströme gingen rings um seinen Stamm her und seine Bäche zu allen Bäumen
im Felde: \bibverse{5} Darum ist er höher worden denn alle Bäume im
Felde und kriegte viel Äste und lange Zweige; denn er hatte Wassers
genug, sich auszubreiten. \bibverse{6} Alle Vögel des Himmels nisteten
auf seinen Ästen, und alle Tiere im Felde hatten Junge unter seinen
Zweigen; und unter seinem Schatten wohneten alle großen Völker.
\bibverse{7} Er hatte schöne große und lange Äste; denn seine Wurzeln
hatten viel Wassers; \bibverse{8} und war ihm kein Zedernbaum gleich in
GOttes Garten, und die Tannenbäume waren seinen Ästen nicht zu gleichen,
und die Kastanienbäume waren nichts gegen seine Zweige. Ja, er war so
schön als kein Baum im Garten GOttes. \bibverse{9} Ich hab ihn so schön
gemacht, daß er so viel Äste kriegte, daß ihn alle lustigen Bäume im
Garten GOttes neideten. \bibverse{10} Darum spricht der HErr HErr also:
Weil er so hoch worden ist, daß sein Wipfel stund unter großen, hohen,
dicken Zweigen, und sein Herz sich erhub, daß er so hoch war,
\bibverse{11} darum gab ich ihn dem Mächtigsten unter den Heiden in die
Hände, der mit ihm umginge und ihn vertriebe, wie er verdienet hat mit
seinem gottlosen Wesen, \bibverse{12} daß Fremde ihn ausrotten sollten,
nämlich die Tyrannen der Heiden, und ihn zerstreuen, und seine Äste auf
den Bergen und in allen Tälern liegen mußten, und seine Zweige
zerbrachen an allen Bächen im Lande, daß alle Völker auf Erden von
seinem Schatten wegziehen mußten und ihn verlassen; \bibverse{13} und
alle Vögel des Himmels auf seinem umgefallenen Stamm saßen, und alle
Tiere im Felde legten sich auf seine Äste, \bibverse{14} auf daß sich
forthin kein Baum am Wasser seiner Höhe erhebe, daß sein Wipfel unter
großen, dicken Zweigen stehe, und kein Baum am Wasser sich erhebe über
die andern; denn sie müssen alle unter die Erde und dem Tode übergeben
werden, wie andere Menschen, die in die Grube fahren. \bibverse{15} So
spricht der HErr HErr: Zu der Zeit, da er hinunter in die Hölle fuhr, da
machte ich ein Trauern, daß ihn die Tiefe bedeckte, und seine Ströme
stillstehen mußten; und die großen Wasser nicht laufen konnten, und
machte, daß der Libanon um ihn trauerte und alle Feldbäume verdorreten
über ihm. \bibverse{16} Ich erschreckte die Heiden, da sie ihn höreten
fallen, da ich ihn hinunterstieß zur Hölle mit denen, so in die Grube
fahren. Und alle lustigen Bäume unter der Erde, die edelsten und besten
auf dem Libanon, und alle, die am Wasser gestanden waren, gönneten es
ihm wohl. \bibverse{17} Denn sie mußten auch mit ihm hinunter zur Hölle,
zu den Erschlagenen mit dem Schwert, weil sie unter dem Schatten seines
Arms gewohnet hatten unter den Heiden. \bibverse{18} Wie groß meinest du
denn, daß du (Pharao) seiest mit deiner Pracht und Herrlichkeit unter
den lustigen Bäumen? Denn du mußt mit den lustigen Bäumen unter die Erde
hinabfahren und unter den Unbeschnittenen liegen, so mit dem Schwert
erschlagen sind. Also soll es Pharao gehen samt all seinem Volk, spricht
der HErr HErr.

\hypertarget{section-31}{%
\section{32}\label{section-31}}

\bibverse{1} Und es begab sich im zwölften Jahr, am ersten Tage des
zwölften Monden, geschah des HErrn Wort zu mir und sprach: \bibverse{2}
Du Menschenkind, mache eine Wehklage über Pharao, den König zu Ägypten,
und sprich zu ihm: Du bist gleichwie ein Löwe unter den Heiden und wie
ein Meerdrache und springest in deinen Strömen und trübest das Wasser
mit deinen Füßen und machst seine Ströme trübe. \bibverse{3} So spricht
der HErr HErr: Ich will mein Netz über dich auswerfen durch einen großen
Haufen Volks, die dich sollen in mein Garn jagen. \bibverse{4} Und will
dich aufs Land ziehen und aufs Feld werfen, daß alle Vögel des Himmels
auf dir sitzen sollen, und alle Tiere auf Erden von dir satt werden.
\bibverse{5} Und will dein Aas auf die Berge werfen und mit deiner Höhe
die Täler ausfüllen. \bibverse{6} Das Land, darin du schwimmest, will
ich von deinem Blut rot machen bis an die Berge hinan, daß die Bäche von
dir voll werden. \bibverse{7} Und wenn du nun gar dahin bist, so will
ich den Himmel verhüllen und seine Sterne verfinstern und die Sonne mit
Wolken überziehen, und der Mond soll nicht scheinen. \bibverse{8} Alle
Lichter am Himmel will ich über dir lassen dunkel werden und will eine
Finsternis in deinem Lande machen, spricht der HErr HErr. \bibverse{9}
Dazu will ich vieler Völker Herz erschreckt machen, wenn ich die Heiden
deine Plage erfahren lasse und viel Länder, die du nicht kennest.
\bibverse{10} Viel Völker sollen sich über dir entsetzen, und ihren
Königen soll vor dir grauen, wenn ich mein Schwert wider sie blinken
lasse, und sollen plötzlich erschrecken, daß ihnen das Herz entfallen
wird über deinem Fall. \bibverse{11} Denn so spricht der HErr HErr: Das
Schwert des Königs zu Babel soll dich treffen. \bibverse{12} Und ich
will dein Volk fällen durch das Schwert der Helden und durch allerlei
Tyrannen der Heiden; die sollen die Herrlichkeit Ägyptens verheeren, daß
all ihr Volk vertilget werde. \bibverse{13} Und ich will alle ihre Tiere
umbringen an den großen Wassern, daß sie keines Menschen Fuß und keines
Tieres Klauen trübe machen soll. \bibverse{14} Alsdann will ich ihre
Wasser lauter machen, daß ihre Ströme fließen wie Öl spricht der HErr
HErr, \bibverse{15} wenn ich das Land Ägypten verwüstet und alles, was
im Lande ist, öde gemacht und alle, so drinnen wohnen, erschlagen habe,
daß sie erfahren, daß ich der HErr sei. \bibverse{16} Das wird der
Jammer sein, den man wohl mag klagen; ja, viel Töchter der Heiden werden
solche Klage führen über Ägypten und all ihr Volk wird man klagen,
spricht der HErr HErr. \bibverse{17} Und im zwölften Jahr, am
fünfzehnten Tage desselbigen Monden, geschah des HErrn Wort zu mir und
sprach: \bibverse{18} Du Menschenkind, beweine das Volk in Ägypten und
stoße es mit den Töchtern der starken Heiden hinab unter die Erde zu
denen, die in die Grube fahren. \bibverse{19} Wo ist nun deine Wollust?
Hinunter, und lege dich zu den Unbeschnittenen! \bibverse{20} Sie werden
fallen unter den Erschlagenen mit dem Schwert. Das Schwert ist schon
gefaßt und gezückt über ihr ganzes Volk. \bibverse{21} Davon werden
sagen in der Hölle die starken Helden mit ihren Gehilfen, die alle
hinuntergefahren sind und liegen da unter den Unbeschnittenen und
Erschlagenen vom Schwert. \bibverse{22} Daselbst liegt Assur mit all
seinem Volk umher begraben, die alle erschlagen und durchs Schwert
gefallen sind. \bibverse{23} Ihre Gräber sind tief in der Grube, und
sein Volk liegt allenthalben umher begraben, die alle erschlagen und
durchs Schwert gefallen sind, da sich die ganze Welt vor fürchtete.
\bibverse{24} Da liegt auch Elam mit all seinem Haufen umher begraben,
die alle erschlagen und durchs Schwert gefallen sind und
hinuntergefahren als die Unbeschnittenen unter die Erde, davor sich auch
alle Welt fürchtete; und müssen ihre Schande tragen mit denen, die in
die Grube fahren. \bibverse{25} Man hat sie unter die Erschlagenen
gelegt samt all ihrem Haufen, und liegen umher begraben; und sind alle
wie die Unbeschnittenen und die Erschlagenen vom Schwert, vor denen sich
auch alle Welt fürchten mußte; und müssen ihre Schande tragen mit denen,
die in die Grube fahren, und unter den Erschlagenen bleiben.
\bibverse{26} Da liegt Mesech und Thubal mit all ihrem Haufen umher
begraben, die alle unbeschnitten und mit dem Schwert erschlagen sind,
vor denen sich auch die ganze Welt fürchten mußte, \bibverse{27} und
alle andern Helden, die unter den Unbeschnittenen gefallen sind und mit
ihrer Kriegswehre zur Hölle gefahren und ihre Schwerter unter ihre
Häupter haben müssen legen, und ihre Missetat über ihre Gebeine kommen
ist, die doch auch gefürchtete Helden waren in der ganzen Welt; also
müssen sie liegen. \bibverse{28} So mußt du freilich auch unter den
Unbeschnittenen zerschmettert werden und unter denen, die mit dem
Schwert erschlagen sind, liegen. \bibverse{29} Da liegt Edom mit seinen
Königen und all seinen Fürsten unter den Erschlagenen mit dem Schwert
und unter den Unbeschnittenen, samt andern, so in die Grube fahren, die
doch mächtig gewesen sind. \bibverse{30} Ja, es müssen alle Fürsten von
Mitternacht dahin und alle Zidonier, die mit den Erschlagenen
hinabgefahren sind, und ihre schreckliche Gewalt ist zuschanden worden,
und müssen liegen unter den Unbeschnittenen und denen, so mit dem
Schwert erschlagen sind, und ihre Schande tragen samt denen, so in die
Grube fahren. \bibverse{31} Diese wird Pharao sehen und sich trösten mit
all seinem Volk, die unter ihm mit dem Schwert erschlagen sind, und mit
seinem ganzen Heer, spricht der HErr HErr. \bibverse{32} Denn es soll
sich auch einmal alle Welt vor mir fürchten, daß Pharao und alle seine
Menge soll liegen unter den Unbeschnittenen und mit dem Schwert
Erschlagenen, spricht der HErr HErr.

\hypertarget{section-32}{%
\section{33}\label{section-32}}

\bibverse{1} Und des HErrn Wort geschah zu mir und sprach: \bibverse{2}
Du Menschenkind, predige wider dein Volk und sprich zu ihnen: Wenn ich
ein Schwert über das Land führen würde, und das Volk im Lande nähme
einen Mann unter ihnen und machten ihn zu ihrem Wächter, \bibverse{3}
und er sähe das Schwert kommen über das Land und bliese die Trommete und
warnete das Volk: \bibverse{4} wer nun der Trommeten Hall hörete und
wollte sich nicht warnen lassen, und das Schwert käme und nähme ihn weg,
desselben Blut sei auf seinem Kopf; \bibverse{5} denn er hat der
Trommeten Hall gehöret und hat sich dennoch nicht warnen lassen; darum
sei sein Blut auf ihm. Wer sich aber warnen läßt, der wird sein Leben
davonbringen. \bibverse{6} Wo aber der Wächter sähe das Schwert kommen
und die Trommete nicht bliese noch sein Volk warnete, und das Schwert
käme und nähme etliche weg: dieselben würden wohl um ihrer Sünde willen
weggenommen, aber ihr Blut will ich von des Wächters Hand fordern.
\bibverse{7} Und nun, du Menschenkind, ich habe dich zu einem Wächter
gesetzt über das Haus Israel, wenn du etwas aus meinem Munde hörest, daß
du sie von meinetwegen warnen sollst. \bibverse{8} Wenn ich nun zu dem
Gottlosen sage: Du Gottloser mußt des Todes sterben, und du sagst ihm
solches nicht, daß sich der Gottlose warnen lasse vor seinem Wesen, so
wird wohl der Gottlose um seines gottlosen Wesens willen sterben, aber
sein Blut will ich von deiner Hand fordern. \bibverse{9} Warnest du aber
den Gottlosen vor seinem Wesen, daß er sich davon bekehre, und er sich
nicht will von seinem Wesen bekehren, so wird er um seiner Sünde willen
sterben, und du hast deine Seele errettet. \bibverse{10} Darum, du
Menschenkind, sage dem Hause Israel: Ihr sprechet also: Unsere Sünden
und Missetaten liegen auf uns, daß wir darunter vergehen; wie können wir
denn leben? \bibverse{11} So sprich zu ihnen: So wahr als ich lebe;
spricht der HErr HErr, ich habe keinen Gefallen am Tode des Gottlosen,
sondern daß sich der Gottlose bekehre von seinem Wesen und lebe. So
bekehret euch doch nun von eurem bösen Wesen! Warum wollt ihr sterben,
ihr vom Hause Israel? \bibverse{12} Und du, Menschenkind, sprich zu
deinem Volk: Wenn ein Gerechter Böses tut, so wird's ihm nicht helfen,
daß er fromm gewesen ist; und wenn ein Gottloser fromm wird, so soll's
ihm nicht schaden, daß er gottlos gewesen ist. So kann auch der Gerechte
nicht leben, wenn er sündiget. \bibverse{13} Denn wo ich zu dem
Gerechten spreche, er soll leben, und er verläßt sich auf seine
Gerechtigkeit und tut Böses, so soll all seiner Frömmigkeit nicht
gedacht werden, sondern er soll sterben in seiner Bosheit, die er tut.
\bibverse{14} Und wenn ich zum Gottlosen spreche, er soll sterben, und
er bekehret sich von seiner Sünde und tut, was recht und gut ist,
\bibverse{15} also daß der Gottlose das Pfand wiedergibt und bezahlet,
was er geraubet hat, und nach dem Wort des Lebens wandelt, daß er kein
Böses tut, so soll er leben und nicht sterben, \bibverse{16} und aller
seiner Sünden; die er getan hat, soll nicht gedacht werden; denn er tut
nun, was recht und gut ist; darum soll er leben. \bibverse{17} Noch
spricht dein Volk: Der HErr urteilet nicht recht, so sie doch unrecht
haben. \bibverse{18} Denn wo der Gerechte sich kehret von seiner
Gerechtigkeit und tut Böses, so stirbt er ja billig darum; \bibverse{19}
und wo sich der Gottlose bekehret von seinem gottlosen Wesen und tut,
was recht und gut ist, so soll er ja billig leben. \bibverse{20} Noch
sprechet ihr: Der HErr urteilet nicht recht, so ich doch euch vom Hause
Israel einen jeglichen nach seinem Wesen urteile. \bibverse{21} Und es
begab sich im zwölften Jahr unsers Gefängnisses, am fünften Tage des
zehnten Monden, kam zu mir ein Entronnener von Jerusalem und sprach: Die
Stadt ist geschlagen. \bibverse{22} Und die Hand des HErrn war über mir
des Abends, ehe der Entronnene kam, und tat mir meinen Mund auf, bis er
zu mir kam des Morgens, und tat mir meinen Mund auf, also daß ich nicht
mehr schweigen konnte. \bibverse{23} Und des HErrn Wort geschah zu mir
und sprach: \bibverse{24} Du Menschenkind, die Einwohner dieser Wüste im
Lande Israel sprechen also: Abraham war ein einiger Mann und erbte dies
Land; unser aber ist viel, so haben wir ja das Land billiger.
\bibverse{25} Darum sprich zu ihnen: So spricht der HErr HErr: Ihr habt
Blut gefressen und eure Augen zu den Götzen aufgehoben und Blut
vergossen; und ihr meinet, ihr wollet das Land besitzen? \bibverse{26}
Ja, ihr fahret immer fort mit Morden und übet Greuel, und einer schändet
dem andern sein Weib, und meinet, ihr wollet das Land besitzen?
\bibverse{27} So sprich zu ihnen: So spricht der HErr HErr: So wahr ich
lebe, sollen alle, so in den Wüsten wohnen, durchs Schwert fallen, und
was auf dem Felde ist, will ich den Tieren zu fressen geben; und die in
Festungen und Höhlen sind, sollen an der Pestilenz sterben.
\bibverse{28} Denn ich will das Land gar verwüsten und seiner Hoffart
und Macht ein Ende machen, daß das Gebirge Israel so wüst werde, daß
niemand dadurchgehe. \bibverse{29} Und sollen erfahren, daß ich der HErr
bin, wenn ich das Land gar verwüstet habe um aller ihrer Greuel willen,
die sie üben. \bibverse{30} Und du, Menschenkind, dein Volk redet wider
dich an den Wänden und unter den Haustüren und spricht je einer zum
andern: Lieber, kommt und laßt uns hören, was der HErr sage!
\bibverse{31} Und sie werden zu dir kommen in die Versammlung und vor
dir sitzen als mein Volk und werden deine Worte hören, aber nichts
danach tun, sondern werden dich anpfeifen und gleichwohl fortleben nach
ihrem Geiz. \bibverse{32} Und siehe, du mußt ihr Liedlein sein, das sie
gerne singen und spielen werden. Also werden sie deine Worte hören und
nichts danach tun. \bibverse{33} Wenn es aber kommt, was kommen soll,
siehe, so werden sie erfahren, daß ein Prophet unter ihnen gewesen sei.

\hypertarget{section-33}{%
\section{34}\label{section-33}}

\bibverse{1} Und des HErrn Wort geschah zu mir und sprach: \bibverse{2}
Du Menschenkind, weissage wider die Hirten Israels, weissage und sprich
zu ihnen: So spricht der HErr HErr: Wehe den Hirten Israels, die sich
selbst weiden! Sollen nicht die Hirten die Herde weiden? \bibverse{3}
Aber ihr fresset das Fette und kleidet euch mit der Wolle und schlachtet
das Gemästete; aber die Schafe wollet ihr nicht weiden. \bibverse{4} Der
Schwachen wartet ihr nicht und die Kranken heilet ihr nicht, das
Verwundete verbindet ihr nicht, das Verirrete holet ihr nicht und das
Verlorene suchet ihr nicht, sondern streng und hart herrschet ihr über
sie. \bibverse{5} Und meine Schafe sind zerstreuet, als die keinen
Hirten haben, und allen wilden Tieren zur Speise worden und gar
zerstreuet \bibverse{6} und gehen irre hin und wieder auf den Bergen und
auf den hohen Hügeln und sind auf dem ganzen Lande zerstreuet, und ist
niemand, der nach ihnen frage, oder ihrer achte. \bibverse{7} Darum
höret, ihr Hirten, des HErrn Wort! \bibverse{8} So wahr ich lebe,
spricht der HErr HErr, weil ihr meine Schafe lasset zum Raube und meine
Herde allen wilden Tieren zur Speise werden, weil sie keinen Hirten
haben, und meine Hirten nach meiner Herde nicht fragen, sondern sind
solche Hirten, die sich selbst weiden, aber meine Schafe wollen sie
nicht weiden, \bibverse{9} darum, ihr Hirten, höret des HErrn Wort!
\bibverse{10} So spricht der HErr HErr: Siehe, ich will an die Hirten
und will meine Herde von ihren Händen fordern; und will's mit ihnen ein
Ende machen, daß sie nicht mehr sollen Hirten sein und sollen sich nicht
mehr selbst weiden. Ich will meine Schafe erretten aus ihrem Maul, daß
sie sie forthin nicht mehr fressen sollen. \bibverse{11} Denn so spricht
der HErr HErr: Siehe, ich will mich meiner Herde selbst annehmen und sie
suchen. \bibverse{12} Wie ein Hirte seine Schafe suchet, wenn sie von
seiner Herde verirret sind, also will ich meine Schafe suchen und will
sie erretten von allen Örtern, dahin sie zerstreuet waren, zur Zeit, da
es trübe und finster war. \bibverse{13} Ich will sie von allen Völkern
ausführen und aus allen Ländern versammeln und will sie in ihr Land
führen; und will sie weiden auf den Bergen Israels und in allen Auen und
auf allen Angern des Landes. \bibverse{14} Ich will sie auf die beste
Weide führen, und ihre Hürden werden auf den hohen Bergen in Israel
stehen; daselbst werden sie in sanften Hürden liegen und fette Weide
haben auf den Bergen Israels. \bibverse{15} Ich will selbst meine Schafe
weiden und ich will sie lagern, spricht der HErr HErr. \bibverse{16} Ich
will das Verlorne wieder suchen und das Verirrete wiederbringen und das
Verwundete verbinden und des Schwachen warten; und was fett und stark
ist, will ich behüten, und will ihrer pflegen, wie es recht ist.
\bibverse{17} Aber zu euch, meine Herde, spricht der HErr HErr also:
Siehe, ich will richten zwischen Schaf und Schaf und zwischen Widdern
und Böcken. \bibverse{18} Ist's nicht genug, daß ihr so gute Weide habt
und so überflüssig, daß ihr's mit Füßen tretet, und so schöne Borne zu
trinken, so überflüssig, daß ihr dareintretet und sie trübe machet,
\bibverse{19} daß meine Schafe essen müssen, was ihr mit euren Füßen
zertreten habt, und trinken, was ihr mit euren Füßen trübe gemacht habt?
\bibverse{20} Darum so spricht der HErr HErr zu ihnen: Siehe, ich will
richten zwischen den fetten und magern Schafen, \bibverse{21} darum daß
ihr lecket mit den Füßen und die Schwachen von euch stoßet mit euren
Hörnern, bis ihr sie alle von euch zerstreuet. \bibverse{22} Und ich
will meiner Herde helfen, daß sie nicht mehr sollen zum Raube werden,
und will richten zwischen Schaf und Schaf. \bibverse{23} Und ich will
ihnen einen einigen Hirten erwecken, der sie weiden soll, nämlich meinen
Knecht David. Der wird sie weiden und soll ihr Hirte sein. \bibverse{24}
Und ich, der HErr, will ihr GOtt sein; aber mein Knecht David soll der
Fürst unter ihnen sein. Das sage ich, der HErr. \bibverse{25} Und ich
will einen Bund des Friedens mit ihnen machen und alle bösen Tiere aus
dem Lande ausrotten, daß sie sicher wohnen sollen in der Wüste und in
den Wäldern schlafen. \bibverse{26} Ich will sie und alle meine Hügel
umher segnen und auf sie regnen lassen zu rechter Zeit. Das sollen
gnädige Regen sein, \bibverse{27} daß die Bäume auf dem Felde ihre
Früchte bringen, und das Land sein Gewächs geben wird; und sie sollen
sicher auf dem Lande wohnen und sollen erfahren, daß ich der HErr bin,
wenn ich ihr Joch zerbrochen und sie errettet habe von der Hand derer,
denen sie dienen mußten. \bibverse{28} Und sie sollen nicht mehr den
Heiden zum Raube werden, und kein Tier auf Erden soll sie mehr fressen,
sondern sollen sicher wohnen ohne alle Furcht. \bibverse{29} Und ich
will ihnen eine berühmte Pflanze erwecken, daß sie nicht mehr sollen
Hunger leiden im Lande und ihre Schmach unter den Heiden nicht mehr
tragen sollen. \bibverse{30} Und sollen erfahren, daß ich, der HErr, ihr
GOtt, bei ihnen bin, und daß sie vom Hause Israel mein Volk seien,
spricht der HErr HErr. \bibverse{31} Ja, ihr Menschen sollt die Herde
meiner Weide sein, und ich will euer GOtt sein, spricht der HErr HErr.

\hypertarget{section-34}{%
\section{35}\label{section-34}}

\bibverse{1} Und des HErrn Wort geschah zu mir und sprach: \bibverse{2}
Du Menschenkind, richte dein Angesicht wider das Gebirge Seir und
weissage dawider \bibverse{3} und sprich zum selbigen: So spricht der
HErr HErr: Siehe, ich will an dich, du Berg Seir, und meine Hand wider
dich ausstrecken und will dich gar wüst machen. \bibverse{4} Ich will
deine Städte öde machen, daß du sollst zur Wüste werden und erfahren,
daß ich der HErr bin, \bibverse{5} darum daß ihr ewige Feindschaft
traget wider die Kinder Israel und triebet sie ins Schwert, da es ihnen
übel ging und ihre Sünde ein Ende hatte. \bibverse{6} Darum, so wahr ich
lebe, spricht der HErr HErr, will ich dich auch blutend machen, und
sollst dem Blute nicht entrinnen; weil du Lust zu Blut hast, sollst du
dem Blute nicht entrinnen. \bibverse{7} Und ich will den Berg Seir wüst
und öde machen, daß niemand darauf wandeln noch gehen soll. \bibverse{8}
Und will sein Gebirge und alle Hügel, Täler und alle Gründe voll Toter
machen, die durchs Schwert sollen erschlagen daliegen. \bibverse{9} Ja,
zu einer ewigen Wüste will ich dich machen, daß niemand in deinen
Städten wohnen soll, und sollt erfahren daß ich der HErr bin.
\bibverse{10} Und darum daß du sprichst: Diese beiden Völker mit beiden
Ländern müssen mein werden und wir wollen sie einnehmen, obgleich der
HErr da wohnet, \bibverse{11} darum, so wahr ich lebe, spricht der HErr
HErr, will ich nach deinem Zorn und Haß mit dir umgehen, wie du mit
ihnen umgegangen bist aus lauterm Haß und will bei ihnen bekannt werden,
wenn ich dich gestraft habe, \bibverse{12} und sollst erfahren, daß ich,
der HErr, all dein Lästern gehöret habe, so du geredet hast wider das
Gebirge Israel und gesagt: Sie sind verwüstet und uns zu verderben
gegeben. \bibverse{13} Und habt euch wider mich gerühmet und heftig
wider mich geredet; das hab' ich gehöret. \bibverse{14} So spricht nun
der HErr HErr: Ich will dich zur Wüste machen, daß sich alles Land
freuen soll. \bibverse{15} Und wie du dich gefreuet hast über dem Erbe
des Hauses Israel, darum daß es war wüst worden, ebenso will ich mit dir
tun, daß der Berg Seir wüst sein muß samt dem ganzen Edom; und sollen
erfahren, daß ich der HErr bin.

\hypertarget{section-35}{%
\section{36}\label{section-35}}

\bibverse{1} Und du, Menschenkind, weissage den Bergen Israels und
sprich: Höret des HErrn Wort, ihr Berge Israels! \bibverse{2} So spricht
der HErr HErr: Darum daß der Feind über euch rühmet: Heh, die ewigen
Höhen sind nun unser Erbe worden! \bibverse{3} darum weissage und
sprich: So spricht der HErr HErr: Weil man euch allenthalben verwüstet
und vertilget, und seid den übrigen Heiden zuteil worden und seid den
Leuten ins Maul kommen und ein böses Geschrei worden, \bibverse{4} darum
höret, ihr Berge Israels, das Wort des HErrn HErrn! So spricht der HErr
HErr beide, zu den Bergen und Hügeln, zu den Bächen und Tälern, zu den
öden Wüsten und verlassenen Städten, welche den übrigen Heiden
ringsumher zum Raub und Spott worden sind; \bibverse{5} ja, so spricht
der HErr HErr: Ich hab' in meinem feurigen Eifer geredet wider die
übrigen Heiden und wider das ganze Edom, welche mein Land eingenommen
haben mit Freuden von ganzem Herzen und mit Hohnlachen, dasselbige zu
verheeren und plündern. \bibverse{6} Darum weissage von dem Lande Israel
und sprich zu den Bergen und Hügeln, zu den Bächen und Tälern: So
spricht der HErr HErr: Siehe, ich hab' in meinem Eifer und Grimm
geredet, weil ihr (solche) Schmach von den Heiden tragen müsset.
\bibverse{7} Darum spricht der HErr HErr also: Ich hebe meine Hand auf,
daß eure Nachbarn, die Heiden umher, ihre Schande wieder tragen sollen.
\bibverse{8} Aber ihr Berge Israels sollt wieder grünen und eure Frucht
bringen meinem Volk Israel; und soll in kurzem geschehen. \bibverse{9}
Denn siehe, ich will mich wieder zu euch wenden und euch ansehen, daß
ihr gebauet und besäet werdet, \bibverse{10} und will bei euch der Leute
viel machen, das ganze Israel allzumal; und die Städte sollen wieder
bewohnet und die Wüsten erbauet werden. \bibverse{11} Ja, ich will bei
euch der Leute und des Viehes viel machen, daß ihr euch mehren und
wachsen sollet. Und ich will euch wieder einsetzen, da ihr vorhin
wohnetet; und will euch mehr Gutes tun denn zuvor je; und sollet
erfahren, daß ich der HErr sei. \bibverse{12} Ich will euch Leute
herzubringen, die mein Volk Israel sollen sein, die werden dich
besitzen; und sollst ihr Erbteil sein und sollst nicht mehr ohne Erben
sein. \bibverse{13} So spricht der HErr HErr: Weil man das von euch
sagt: Du hast Leute gefressen und hast dein Volk ohne Erben gemacht,
\bibverse{14} darum sollst du (nun) nicht mehr Leute fressen, noch dein
Volk ohne Erben machen, spricht der HErr HErr. \bibverse{15} Und ich
will dich nicht mehr lassen hören die Schmach der Heiden; und sollst den
Spott der Heiden nicht mehr tragen und sollst dein Volk nicht mehr ohne
Erben machen, spricht der HErr HErr. \bibverse{16} Und des HErrn Wort
geschah weiter zu mir: \bibverse{17} Du Menschenkind, da das Haus Israel
in ihrem Lande wohneten und dasselbige verunreinigten mit ihrem Wesen
und Tun, daß ihr Wesen vor mir war wie die Unreinigkeit eines Weibes in
ihrer Krankheit, \bibverse{18} da schüttete ich meinen Grimm über sie
aus um des Bluts willen, das sie im Lande vergossen und dasselbe
verunreiniget hatten durch ihre Götzen. \bibverse{19} Und ich
zerstreuete sie unter die Heiden und zerstäubte sie in die Länder und
richtete sie nach ihrem Wesen und Tun. \bibverse{20} Und hielten sich
wie die Heiden, dahin sie kamen, und entheiligten meinen heiligen Namen,
daß man von ihnen sagte: Ist das des HErrn Volk, das aus seinem Lande
hat müssen ziehen? \bibverse{21} Aber ich verschonete um meines heiligen
Namens willen, welchen das Haus Israel entheiligte unter den Heiden,
dahin sie kamen. \bibverse{22} Darum sollst du zum Hause Israel sagen:
So spricht der HErr HErr: Ich tue es nicht um euretwillen, ihr vom Hause
Israel, sondern um meines heiligen Namens willen, welchen ihr
entheiliget habt unter den Heiden, zu welchen ihr kommen seid.
\bibverse{23} Denn ich will meinen großen Namen, der durch euch vor den
Heiden entheiliget ist, den ihr unter denselbigen entheiliget habt,
heilig machen. Und die Heiden sollen erfahren, daß ich der HErr sei,
spricht der HErr HErr, wenn ich mich vor ihnen an euch erzeige, daß ich
heilig sei. \bibverse{24} Denn ich will euch aus den Heiden holen und
euch aus allen Landen versammeln und wieder in euer Land führen.
\bibverse{25} Und will rein Wasser über euch sprengen, daß ihr rein
werdet von aller eurer Unreinigkeit, und von allen euren Götzen will ich
euch reinigen. \bibverse{26} Und ich will euch ein neu Herz und einen
neuen Geist in euch geben; und will das steinerne Herz aus eurem Fleisch
wegnehmen und euch ein fleischern Herz geben. \bibverse{27} Ich will
meinen Geist in euch geben und will solche Leute aus euch machen, die in
meinen Geboten wandeln und meine Rechte halten und danach tun.
\bibverse{28} Und ihr sollet wohnen im Lande, das ich euren Vätern
gegeben habe, und sollet mein Volk sein, und ich will euer GOtt sein.
\bibverse{29} Ich will euch von aller eurer Unreinigkeit losmachen und
will dem Korn rufen und will es mehren und will euch keine Teurung
kommen lassen. \bibverse{30} Ich will die Früchte auf den Bäumen und das
Gewächs auf dem Felde mehren, daß euch die Heiden nicht mehr spotten mit
der Teurung. \bibverse{31} Alsdann werdet ihr an euer böses Wesen
gedenken und eures Tuns, das nicht gut war, und wird euch eure Sünde und
Abgötterei gereuen. \bibverse{32} Solches will ich tun, nicht um
euretwillen, spricht der HErr HErr, daß ihr es wisset, sondern ihr
werdet euch müssen schämen und schamrot werden, ihr vom Hause Israel,
über eurem Wesen. \bibverse{33} So spricht der HErr HErr: Zu der Zeit,
wenn ich euch reinigen werde von allen euren Sünden, so will ich die
Städte wieder besetzen, und die Wüsten sollen wieder gebauet werden.
\bibverse{34} Das verwüstete Land soll wieder gepflüget werden, dafür,
daß es verheeret war, daß es sehen sollen alle, die dadurch gehen,
\bibverse{35} und sagen: Dies Land war verheeret, und jetzt ist's wie
ein Lustgarten, und diese Städte waren zerstöret, öde und zerrissen und
stehen nun fest gebauet. \bibverse{36} Und die übrigen Heiden um euch
her sollen erfahren, daß ich der HErr bin, der da bauet, was zerrissen
ist, und pflanzet, was verheeret war. Ich, der HErr, sage es und tue es
auch. \bibverse{37} So spricht der HErr HErr: Ich will mich wieder
fragen lassen vom Hause Israel, daß ich mich an ihnen erzeige; und ich
will sie mehren wie eine Menschenherde. \bibverse{38} Wie eine heilige
Herde, wie eine Herde zu Jerusalem auf ihren Festen; so sollen die
verheerten Städte voll Menschenherden werden; und sollen erfahren, daß
ich der HErr bin.

\hypertarget{section-36}{%
\section{37}\label{section-36}}

\bibverse{1} Und des HErrn Hand kam über mich und führete mich hinaus im
Geist des HErrn und stellete mich auf ein weit Feld, das voller Beine
lag. \bibverse{2} Und er führete mich allenthalben dadurch. Und siehe,
(des Gebeins) lag sehr viel auf dem Felde; und siehe, sie waren sehr
verdorret. \bibverse{3} Und er sprach zu mir: Du Menschenkind, meinest
du auch, daß diese Beine wieder lebendig werden? Und ich sprach: HErr
HErr, das weißt du wohl. \bibverse{4} Und er sprach zu mir: Weissage von
diesen Beinen und sprich zu ihnen: Ihr verdorreten Beine, höret des
HErrn Wort! \bibverse{5} So spricht der HErr HErr von diesen Gebeinen:
Siehe, ich will einen Odem in euch bringen, daß ihr sollt lebendig
werden. \bibverse{6} Ich will euch Adern geben und Fleisch lassen über
euch wachsen und mit Haut überziehen; und will euch Odem geben, daß ihr
wieder lebendig werdet; und sollt erfahren, daß ich der HErr bin.
\bibverse{7} Und ich weissagte, wie mir befohlen war; und siehe, da
rauschte es, als ich weissagte; und siehe, es regte sich! Und die
Gebeine kamen wieder zusammen, ein jegliches zu seinem Gebein.
\bibverse{8} Und ich sah, und siehe, es wuchsen Adern und Fleisch
darauf, und er überzog sie mit Haut; es war aber noch kein Odem in
ihnen. \bibverse{9} Und er sprach zu mir: Weissage zum Winde; weissage,
du Menschenkind, und sprich zum Winde: So spricht der HErr HErr: Wind,
komm herzu aus den vier Winden und blase diese Getöteten an, daß sie
wieder lebendig werden! \bibverse{10} Und ich weissagte, wie er mir
befohlen hatte. Da kam Odem in sie, und sie wurden wieder lebendig und
richteten sich auf ihre Füße. Und ihrer war ein sehr groß Heer.
\bibverse{11} Und er sprach zu mir: Du Menschenkind, diese Beine sind
das ganze Haus Israel. Siehe, jetzt sprechen sie: Unsere Beine sind
verdorret, und unsere Hoffnung ist verloren, und ist aus mit uns.
\bibverse{12} Darum weissage und sprich zu ihnen: So spricht der HErr
HErr: Siehe, ich will eure Gräber auftun und will euch, mein Volk, aus
denselben herausholen und euch ins Land Israel bringen; \bibverse{13}
und sollt erfahren, daß ich der HErr bin, wenn ich eure Gräber geöffnet
und euch, mein Volk, aus denselben, gebracht habe. \bibverse{14} Und ich
will meinen Geist in euch geben, daß ihr wieder leben sollt; und will
euch in euer Land setzen, und sollt erfahren, daß ich der HErr bin. Ich
rede es und tue es auch, spricht der HErr. \bibverse{15} Und des HErrn
Wort geschah zu mir und sprach: \bibverse{16} Du Menschenkind, nimm dir
ein Holz und schreibe darauf: Des Juda und der Kinder Israel samt ihren
Zugetanen. Und nimm noch ein Holz und schreibe darauf: Des Joseph,
nämlich das Holz Ephraim, und des ganzen Hauses Israel samt ihren
Zugetanen, \bibverse{17} und tue eins zum andern zusammen, daß ein Holz
werde in deiner Hand. \bibverse{18} So nun dein Volk zu dir wird sagen
und sprechen: Willst du uns nicht zeigen, was du damit meinest?
\bibverse{19} so sprich zu ihnen: So spricht der HErr HErr: Siehe, ich
will das Holz Josephs, welches ist in Ephraims Hand, nehmen, samt ihren
Zugetanen, den Stämmen Israels, und will sie zu dem Holz Judas tun und
ein Holz daraus machen, und sollen eins in meiner Hand sein.
\bibverse{20} Und sollst also die Hölzer, darauf du geschrieben hast, in
deiner Hand halten, daß sie zusehen. \bibverse{21} Und sollst zu ihnen
sagen: So spricht der HErr HErr: Siehe, ich will die Kinder Israel holen
aus den Heiden, dahin sie gezogen sind, und will sie allenthalben
sammeln und will sie wieder in ihr Land bringen. \bibverse{22} Und will
ein einig Volk aus ihnen machen im Lande auf dem Gebirge Israel, und sie
sollen allesamt einen einigen König haben; und sollen nicht mehr zwei
Völker noch in zwei Königreiche zerteilet sein, \bibverse{23} sollen
sich auch nicht mehr verunreinigen mit ihren Götzen und Greueln und
allerlei Sünden. Ich will ihnen heraushelfen aus allen Orten, da sie
gesündiget haben, und will sie reinigen, und sollen mein Volk sein, und
ich will ihr GOtt sein. \bibverse{24} Und mein Knecht David soll ihr
König und ihrer aller einiger Hirte sein. Und sollen wandeln in meinen
Rechten und meine Gebote halten und danach tun. \bibverse{25} Und sie
sollen wieder im Lande wohnen, das ich meinem Knechte Jakob gegeben
habe, darinnen eure Väter gewohnet haben. Sie und ihre Kinder und
Kindeskinder sollen darin wohnen ewiglich; und mein Knecht David soll
ewiglich ihr Fürst sein. \bibverse{26} Und ich will mit ihnen einen Bund
des Friedens machen, das soll ein ewiger Bund sein mit ihnen; und will
sie erhalten und mehren, und mein Heiligtum soll unter ihnen sein
ewiglich. \bibverse{27} Und ich will unter ihnen wohnen und will ihr
GOtt sein, und sie sollen mein Volk sein, \bibverse{28} daß auch die
Heiden sollen erfahren, daß ich der HErr bin, der Israel heilig macht,
wenn mein Heiligtum ewiglich unter ihnen sein wird.

\hypertarget{section-37}{%
\section{38}\label{section-37}}

\bibverse{1} Und des HErrn Wort geschah zu mir und sprach: \bibverse{2}
Du Menschenkind, wende dich gegen Gog, der im Lande Magog ist und der
oberste Fürst ist in Mesech und Thubal, und weissage von ihm
\bibverse{3} und sprich: So spricht der HErr HErr: Siehe, ich will an
dich, Gog, der du der oberste Fürst bist aus den Herren in Mesech und
Thubal. \bibverse{4} Siehe, ich will dich herumlenken und will dir einen
Zaum ins Maul legen und will dich herausführen mit all deinem Heer, Roß
und Mann, die alle wohlgekleidet sind, und ist ihrer ein großer Haufe,
die alle Tartschen und Schild und Schwert führen. \bibverse{5} Du
führest mit dir Perser, Mohren und Libyer, die alle Schild und Helme
führen, \bibverse{6} dazu Gomer und all sein Heer samt dem Hause
Thogarma, so gegen Mitternacht liegt, mit all seinem Heer; ja, du
führest ein groß Volk mit dir. \bibverse{7} Wohlan, rüste dich wohl, du
und alle deine Haufen, so bei dir sind; und sei du ihr Hauptmann!
\bibverse{8} Nach langer Zeit sollst du heimgesucht werden. Zur letzten
Zeit wirst du kommen in das Land, das vom Schwert wiedergebracht und aus
vielen Völkern zusammenkommen ist, nämlich auf die Berge Israels, welche
lange Zeit wüst gewesen sind und nun ausgeführet aus vielen Völkern und
alle sicher wohnen. \bibverse{9} Du wirst heraufziehen und daherkommen
mit großem Ungestüm und wirst sein wie eine Wolke, das Land zu bedecken,
du und all dein Heer und das große Volk mit dir. \bibverse{10} So
spricht der HErr HErr: Zu der Zeit wirst du dir solches vornehmen und
wirst's böse im Sinn haben \bibverse{11} und gedenken: Ich will das Land
ohne Mauern überfallen und über die kommen, so still und sicher wohnen,
als die alle ohne Mauern dasitzen und haben weder Riegel noch Tor,
\bibverse{12} auf daß du rauben und plündern mögest und deine Hand
lassen gehen über die Verstörten, so wieder bewohnet sind, und über das
Volk, so aus den Heiden zusammengerafft ist und sich in die Nahrung und
Güter geschickt hat und mitten im Lande wohnet. \bibverse{13} Das
Reicharabien, Dedan und die Kaufleute auf dem Meer und alle Gewaltigen,
die daselbst sind, werden zu dir sagen: Ich meine ja, du seiest recht
kommen zu rauben, und hast deine Haufen versammelt zu plündern, auf daß
du wegnehmest Silber und Gold und sammelst Vieh und Güter und großen
Raub treibest. \bibverse{14} Darum so weissage, du Menschenkind, und
sprich zu Gog: So spricht der HErr HErr: Ist's nicht also, daß du wirst
merken, wenn mein Volk Israel sicher wohnen wird? \bibverse{15} So wirst
du kommen aus deinem Ort, nämlich von den Enden gegen Mitternacht, du
und ein groß Volk mit dir, alle zu Roß, ein großer Haufe und ein
mächtiges Heer. \bibverse{16} Und wirst heraufziehen über mein Volk
Israel wie eine Wolke, das Land zu bedecken. Solches wird zur letzten
Zeit geschehen. Ich will dich aber darum in mein Land kommen lassen, auf
daß die Heiden mich erkennen, wie ich an dir, o Gog, geheiliget werde
vor ihren Augen. \bibverse{17} So spricht der HErr HErr: Du bist's, von
dem ich vorzeiten gesagt habe durch meine Diener, die Propheten in
Israel, die zur selbigen Zeit weissagten, daß ich dich über sie kommen
lassen wollte. \bibverse{18} Und es wird geschehen zur Zeit, wenn Gog
kommen wird über das Land Israel, spricht der HErr HErr, wird
heraufziehen mein Zorn in meinem Grimm. \bibverse{19} Und ich rede
solches in meinem Eifer und im Feuer meines Zorns. Denn zur selbigen
Zeit wird groß Zittern sein im Lande Israel, \bibverse{20} daß vor
meinem Angesicht zittern sollen die Fische im Meer, die Vögel unter dem
Himmel, das Vieh auf dem Felde und alles, was sich regt und wegt auf dem
Lande, und alle Menschen, so auf der Erde sind; und sollen die Berge
umgekehret werden, und die Wände und alle Mauern zu Boden fallen.
\bibverse{21} Ich will aber über ihn rufen das Schwert auf allen meinen
Bergen, spricht der HErr HErr, daß eines jeglichen Schwert soll wider
den andern sein. \bibverse{22} Und ich will ihn richten mit Pestilenz
und Blut und will regnen lassen Platzregen mit Schloßen, Feuer und
Schwefel über ihn und sein Heer und über das große Volk, das mit ihm
ist. \bibverse{23} Also will ich denn herrlich, heilig und bekannt
werden vor vielen Heiden, daß sie erfahren sollen, daß ich der HErr bin.

\hypertarget{section-38}{%
\section{39}\label{section-38}}

\bibverse{1} Und du, Menschenkind, weissage wider Gog und sprich: Also
spricht der HErr HErr: Siehe, ich will an dich, Gog, der du der oberste
Fürst bist in Mesech und Thubal. \bibverse{2} Siehe, ich will dich
herumlenken und locken und aus den Enden von Mitternacht bringen und auf
die Berge Israels kommen lassen. \bibverse{3} Und will dir den Bogen aus
deiner linken Hand schlagen und deine Pfeile aus deiner rechten Hand
werfen. \bibverse{4} Auf den Bergen Israels sollst du niedergelegt
werden, du mit all deinem Heer und mit dem Volk, das bei dir ist. Ich
will dich den Vögeln, woher sie fliegen, und den Tieren auf dem Felde zu
fressen geben. \bibverse{5} Du sollst auf dem Felde daniederliegen; denn
ich, der HErr HErr, hab es gesagt. \bibverse{6} Und ich will Feuer
werfen über Magog und über die, so in den Inseln sicher wohnen; und
sollen's erfahren, daß ich der HErr bin. \bibverse{7} Denn ich will
meinen heiligen Namen kundmachen unter meinem Volk Israel und will
meinen heiligen Namen nicht länger schänden lassen, sondern die Heiden
sollen erfahren, daß ich der HErr bin, der Heilige in Israel.
\bibverse{8} Siehe, es ist schon kommen und geschehen, spricht der HErr
HErr; das ist der Tag, davon ich geredet habe. \bibverse{9} Und die
Bürger in den Städten Israels werden herausgehen und Feuer machen und
verbrennen die Waffen, Schilde, Tartschen, Bogen, Pfeile, Fauststangen
und langen Spieße und werden sieben Jahre lang Feuerwerk damit halten,
\bibverse{10} daß sie nicht dürfen Holz auf dem Felde holen noch im
Walde hauen, sondern von den Waffen werden sie Feuer halten; und sollen
rauben, von denen sie beraubt sind, und plündern, von denen sie
geplündert sind, spricht der HErr HErr. \bibverse{11} Und soll zu der
Zeit geschehen, da will ich Gog einen Ort geben zum Begräbnis in Israel,
nämlich das Tal, da man gehet am Meer gegen Morgen, also daß die, so
vorübergehen, sich davor scheuen werden, weil man daselbst Gog mit
seiner Menge begraben hat; und soll heißen Gogs Haufental. \bibverse{12}
Es wird sie aber das Haus Israel begraben sieben Monden lang, damit das
Land gereiniget werde. \bibverse{13} Ja, alles Volk im Lande wird an
ihnen zu begraben haben, und werden Ruhm davon haben, daß ich des Tages
meine Herrlichkeit erzeiget habe, spricht der HErr HErr. \bibverse{14}
Und sie werden Leute aussondern, die stets im Lande umhergehen und mit
denselbigen die Totengräber, zu begraben die Übrigen auf dem Lande, auf
daß es gereiniget werde; nach sieben Monden werden sie forschen.
\bibverse{15} Und die, so im Lande umhergehen und etwa eines Menschen
Bein sehen, werden dabei ein Mal aufrichten, bis es die Totengräber auch
in Gogs Haufental begraben. \bibverse{16} So soll auch die Stadt heißen
Hamona. Also werden sie das Land reinigen. \bibverse{17} Nun, du
Menschenkind, so spricht der HErr HErr: Sage allen Vögeln, woher sie
fliegen, und allen Tieren auf dem Felde: Sammelt euch und kommt her!
Findet euch allenthalben her zuhauf zu meinem Schlachtopfer, das ich
euch schlachte, ein groß Schlachtopfer auf den Bergen Israels, und
fresset Fleisch und saufet Blut! \bibverse{18} Fleisch der Starken sollt
ihr fressen und Blut der Fürsten auf Erden sollt ihr saufen, der Widder,
der Hammel, der Böcke, der Ochsen, die allzumal feist und wohlgemästet
sind. \bibverse{19} Und sollt das Fette fressen, daß ihr voll werdet,
und das Blut saufen, daß ihr trunken werdet von dem Schlachtopfer, das
ich euch schlachte. \bibverse{20} Sättiget euch nun über meinem Tisch
von Rossen und Reitern, von Starken und allerlei Kriegsleuten, spricht
der HErr HErr. \bibverse{21} Und ich will meine Herrlichkeit unter die
Heiden bringen, daß alle Heiden sehen sollen mein Urteil, das ich habe
ergehen lassen, und meine Hand, die ich an sie gelegt habe,
\bibverse{22} und also das Haus Israel erfahre, daß ich, der HErr, ihr
GOtt bin, von dem Tage und hinfürder, \bibverse{23} und die Heiden
erfahren, wie das Haus Israel um seiner Missetat willen sei weggeführet,
und daß sie sich an mir versündiget hatten. Darum habe ich mein
Angesicht vor ihnen verborgen und habe sie übergeben in die Hände ihrer
Widersacher, daß sie allzumal durchs Schwert fallen mußten.
\bibverse{24} Ich habe ihnen getan, wie ihre Sünde und Übertreten
verdienet haben, und also mein Angesicht vor ihnen verborgen.
\bibverse{25} Darum so spricht der HErr HErr: Nun will ich das Gefängnis
Jakobs wenden und mich des ganzen Hauses Israel erbarmen und um meinen
heiligen Namen eifern. \bibverse{26} Sie aber werden ihre Schmach und
alle ihre Sünde, damit sie sich an mir versündiget haben, tragen, wenn
sie nun sicher in ihrem Lande wohnen, daß sie niemand schrecke,
\bibverse{27} und ich sie wieder aus den Völkern gebracht und aus den
Landen ihrer Feinde versammelt habe und ich in ihnen geheiliget worden
bin vor den Augen vieler Heiden. \bibverse{28} Also werden sie erfahren,
daß ich, der HErr, ihr GOtt bin, der ich sie habe lassen unter die
Heiden wegführen und wiederum in ihr Land versammeln und nicht einen von
ihnen dort gelassen habe. \bibverse{29} Und will mein Angesicht nicht
mehr vor ihnen verbergen; denn ich habe meinen Geist über das Haus
Israel ausgegossen, spricht der HErr HErr.

\hypertarget{section-39}{%
\section{40}\label{section-39}}

\bibverse{1} Im fünfundzwanzigsten Jahr unsers Gefängnisses, im Anfang
des Jahres, am zehnten Tage des Monden, das ist das vierzehnte Jahr,
nachdem die Stadt geschlagen war, eben am selbigen Tage kam des HErrn
Hand über mich und führete mich daselbst hin \bibverse{2} durch
göttliche Gesichte, nämlich ins Land Israel, und stellete mich auf einen
sehr hohen Berg, darauf war es wie eine gebauete Stadt vom Mittag
herwärts. \bibverse{3} Und da er mich daselbst hingebracht hatte, siehe,
da war ein Mann, des Gestalt war wie Erz; der hatte eine leinene Schnur
und eine Meßrute in seiner Hand und stund unter dem Tor. \bibverse{4}
Und er sprach zu mir: Du Menschenkind, siehe und höre fleißig zu und
merke eben darauf, was ich dir zeigen will! Denn darum bist du
hergebracht, daß ich dir solches zeige, auf daß du solches alles, was du
hie siehest, verkündigest dem Hause Israel. \bibverse{5} Und siehe, es
ging eine Mauer auswendig am Hause ringsumher. Und der Mann hatte die
Meßrute in der Hand, die war sechs Ellen lang; eine jegliche Elle war
eine Handbreit länger denn eine gemeine Elle. Und er maß das Gebäude in
die Breite eine Rute und in die Höhe auch eine Rute. \bibverse{6} Und er
kam zum Tor, das gegen Morgen lag, und ging hinauf auf seinen Stufen und
maß die Schwellen am Tor, eine jegliche Schwelle eine Rute breit.
\bibverse{7} Und die Gemächer, so beiderseits neben dem Tor waren, maß
er auch, nach der Länge eine Rute und nach der Breite eine Rute; und der
Raum zwischen den Gemächern war fünf Ellen weit. Und er maß auch die
Schwellen am Tor neben der Halle von inwendig eine Rute. \bibverse{8}
Und er maß die Halle am Tor von inwendig eine Rute. \bibverse{9} Und maß
die Halle am Tor acht Ellen und seine Erker zwo Ellen und die Halle von
inwendig des Tors. \bibverse{10} Und der Gemächer waren auf jeglicher
Seite drei am Tor gegen Morgen, je eins so weit als das andere; und
stunden auf beiden Seiten Erker, die waren gleich groß. \bibverse{11}
Danach maß er die Weite der Tür im Tor, nämlich zehn Ellen, und die
Länge des Tors dreizehn Ellen. \bibverse{12} Und vorne an den Gemächern
war Raum auf beiden Seiten, je einer Elle; aber die Gemächer waren je
sechs Ellen auf beiden Seiten. \bibverse{13} Dazu maß er das Tor vom
Dache des Gemachs bis zu des Tors Dach, fünfundzwanzig Ellen breit; und
eine Tür stund gegen der andern. \bibverse{14} Er machte auch Erker
sechzig Ellen und vor jeglichem Erker einen Vorhof am Tor ringsherum.
\bibverse{15} Und bis an die Halle am innern Tor, da man hineingehet,
waren fünfzig Ellen. \bibverse{16} Und es waren enge Fensterlein an den
Gemächern und Erkern hineinwärts, am Tor ringsumher. Also waren auch
Fenster inwendig an den Hallen herum, und an den Erkern umher war
Palmlaubwerk. \bibverse{17} Und er führete mich weiter zum äußern
Vorhof; und siehe, da waren Kammern und ein Pflaster gemacht im Vorhof
herum und dreißig Kammern auf dem Pflaster. \bibverse{18} Und es war das
höhere Pflaster an den Toren, so lang die Tore waren, am niedrigen
Pflaster. \bibverse{19} Und er maß die Breite des untern Tors vor dem
innern Hofe, auswendig hundert Ellen, beide, gegen Morgen und
Mitternacht. \bibverse{20} Also maß er auch das Tor, so gegen
Mitternacht lag, am äußern Vorhofe, nach der Länge und Breite.
\bibverse{21} Das hatte auch auf jeder Seite drei Gemächer und hatte
auch seine Erker und Hallen, gleich so groß wie am vorigen Tor, fünfzig
Ellen die Länge und fünfundzwanzig Ellen die Breite. \bibverse{22} Und
hatte auch seine Fenster und seine Hallen und sein Palmlaubwerk,
gleichwie das Tor gegen Morgen; und hatte sieben Stufen, da man
hinaufging, und hatte seine Halle davor. \bibverse{23} Und es war das
Tor am innern Vorhof gegen das Tor, so gegen Mitternacht und Morgen
stund; und maß hundert Ellen von einem Tor zum andern. \bibverse{24}
Danach führete er mich gegen Mittag, und siehe, da war auch ein Tor
gegen Mittag; und er maß seine Erker und Hallen, gleich als die andern.
\bibverse{25} Die hatten auch Fenster und Hallen umher, gleichwie jene
Fenster, fünfzig Ellen lang und fünfundzwanzig Ellen breit.
\bibverse{26} Und waren auch sieben Stufen hinauf und eine Halle davor
und Palmlaubwerk an seinen Erkern auf jeglicher Seite. \bibverse{27} Und
er maß auch das Tor am innern Vorhof gegen Mittag, nämlich hundert Ellen
von dem einen Mittagstor zum andern. \bibverse{28} Und er führete mich
weiter durch das Mittagstor in den innern Vorhof; und maß dasselbe Tor
gegen Mittag, gleich so groß wie die andern, \bibverse{29} mit seinen
Gemächern, Erkern und Hallen und mit Fenstern und Hallen daran, ebenso
groß wie jene umher, fünfzig Ellen lang und fünfundzwanzig Ellen breit.
\bibverse{30} Und es ging eine Halle herum, fünfundzwanzig Ellen lang
und fünf Ellen breit. \bibverse{31} Dieselbige stund vorne gegen den
äußern Vorhof und hatte auch Palmlaubwerk an den Erkern; es waren aber
acht Stufen hinaufzugehen. \bibverse{32} Danach führete er mich zum
innern Tor gegen Morgen und maß dasselbige, gleich so groß wie die
andern, \bibverse{33} mit seinen Gemächern, Erkern und Hallen und ihren
Fenstern und Hallen umher, gleich so groß wie die andern, fünfzig Ellen
lang und fünfundzwanzig Ellen breit. \bibverse{34} Und hatte auch eine
Halle gegen den äußern Vorhof und Palmlaubwerk an den Erkern zu beiden
Seiten und acht Stufen hinauf. \bibverse{35} Danach führete er mich zum
Tor gegen Mitternacht; das maß er, gleich so groß wie die andern,
\bibverse{36} mit seinen Gemächern, Erkern und Hallen und ihren Fenstern
und Hallen umher, fünfzig Ellen lang und fünfundzwanzig Ellen breit.
\bibverse{37} Und hatte auch eine Halle gegen den äußern Vorhof und
Palmlaubwerk an den Erkern zu beiden Seiten und acht Stufen hinauf.
\bibverse{38} Und unten an den Erkern an jedem Tor war eine Kammer mit
einer Tür, darin man das Brandopfer wusch. \bibverse{39} Aber in der
Halle vor dem Tor stunden auf jeglicher Seite zween Tische, darauf man
die Brandopfer, Sündopfer und Schuldopfer schlachten sollte.
\bibverse{40} Und herauswärts zur Seite, da man hinaufgehet zum Tor,
gegen Mitternacht, stunden auch zween Tische und an der andern Seite
unter der Halle des Tors auch zween Tische. \bibverse{41} Also stunden
auf jeder Seite vor dem Tor vier Tische; das sind acht Tische zuhauf,
darauf man schlachtete. \bibverse{42} Und die vier Tische, zum
Brandopfer gemacht, waren aus gehauenen Steinen, je anderthalb Ellen
lang und breit und einer Elle hoch, darauf man legte allerlei Geräte,
damit man Brandopfer und andere Opfer schlachtete. \bibverse{43} Und es
gingen Leisten herum, hineinwärts gebogen, einer Querhand hoch. Und auf
die Tische sollte man das Opferfleisch legen. \bibverse{44} Und außen
vor dem innern Tor waren Kammern für die Sänger im innern Vorhofe: eine
an der Seite, neben dem Tor zur Mitternacht, die sah gegen Mittag; die
andere zur Seite gegen Morgen, die sah gegen Mitternacht. \bibverse{45}
Und er sprach zu mir: Die Kammer gegen Mittag gehört den Priestern, die
im Hause dienen sollen; \bibverse{46} aber die Kammer gegen Mitternacht
gehört den Priestern, so auf dem Altar dienen. Dies sind die Kinder
Zadoks, welche allein unter den Kindern Levi vor den HErrn treten
sollen, ihm zu dienen. \bibverse{47} Und er maß den Platz im Hause,
nämlich hundert Ellen lang und hundert Ellen breit ins Gevierte; und der
Altar stund eben vorne vor dem Tempel. \bibverse{48} Und er führete mich
hinein zur Halle des Tempels und maß die Halle, fünf Ellen auf jeder
Seite, und das Tor drei Ellen weit auf jeder Seite. \bibverse{49} Aber
die Halle war zwanzig Ellen lang und elf Ellen weit und hatte Stufen, da
man hinaufging; und Pfeiler stunden unten an den Erkern, auf jeder Seite
eine.

\hypertarget{section-40}{%
\section{41}\label{section-40}}

\bibverse{1} Und er führete mich hinein in den Tempel und maß die Erker
an den Wänden; die waren zu jeder Seite sechs Ellen weit, so weit das
Haus war. \bibverse{2} Und die Tür war zehn Ellen weit, aber die Wände
zu beiden Seiten an der Tür waren jede fünf Ellen breit. Und er maß den
Raum im Tempel; der hatte vierzig Ellen in die Länge und zwanzig Ellen
in die Breite. \bibverse{3} Und er ging inwendig hinein und maß die Tür,
zwo Ellen; und die Tür hatte sechs Ellen und die Weite der Tür sieben
Ellen. \bibverse{4} Und er maß zwanzig Ellen in die Länge und zwanzig
Ellen in die Breite am Tempel. Und er sprach zu mir: Dies ist das
Allerheiligste. \bibverse{5} Und er maß die Wand des Hauses, sechs Ellen
hoch; darauf waren Gänge allenthalben herum, geteilt in Gemächer, die
waren allenthalben vier Ellen weit. \bibverse{6} Und derselben Gemächer
waren auf jeder Seite dreiunddreißig, je eins an dem andern; und stunden
Pfeiler unten bei den Wänden am Hause allenthalben herum, die sie
trugen. \bibverse{7} Und über diesen waren noch mehr Gänge umher, und
oben waren die Gänge weiter, daß man aus den untern in die mittlern und
aus den mittlern in die obersten ging. \bibverse{8} Und stund je einer
sechs Ellen über dem andern. \bibverse{9} Und die Weite der obern Gänge
war fünf Ellen, und die Pfeiler trugen die Gänge am Hause. \bibverse{10}
Und es war je von einer Wand am Hause zu der andern zwanzig Ellen.
\bibverse{11} Und es waren zwo Türen an der Schnecke hinauf, eine gegen
Mitternacht, die andere gegen Mittag; und die Schnecke war fünf Ellen
weit. \bibverse{12} Und die Mauer gegen Abend war fünfundsiebenzig Ellen
breit und neunzig Ellen lang. \bibverse{13} Und er maß die Länge des
Hauses, die hatte durchaus hundert Ellen, die Mauer und was daran war.
\bibverse{14} Und die Weite vorne am Hause gegen Morgen mit dem, was
daran hing, war auch hundert Ellen. \bibverse{15} Und er maß die Länge
des Gebäudes mit allem, was daran hing, von einer Ecke bis zur andern;
das war auf jeder Seite hundert Ellen mit dem innern Tempel und Hallen
im Vorhofe \bibverse{16} samt den Türen, Fenstern, Ecken und den dreien
Gängen und Tafelwerk allenthalben herum. \bibverse{17} Er maß auch, wie
hoch von der Erde bis zu den Fenstern war, und wie breit die Fenster
sein sollten; und maß vom Tor bis zum Allerheiligsten, auswendig und
inwendig herum. \bibverse{18} Und am ganzen Hause herum, von unten an
bis oben hinauf an der Tür und an den Wänden, waren Cherubim und
Palmlaubwerk unter die Cherubim gemacht. \bibverse{19} Und ein jeder
Cherub hatte zween Köpfe, auf einer Seite wie ein Menschenkopf, auf der
andern Seite wie ein Löwenkopf. \bibverse{20} Vom Boden an bis hinauf
über die Tür waren die Cherubim und die Palmen geschnitzet, desgleichen
an der Wand des Tempels. \bibverse{21} Und die Tür im Tempel war
viereckig, und war alles artig ineinandergefüget. \bibverse{22} Und der
hölzerne Altar war drei Ellen hoch und zwo Ellen lang und breit; und
seine Ecken und alle seine Seiten waren hölzern. Und er sprach zu mir:
Das ist der Tisch, der vor dem HErrn stehen soll. \bibverse{23} Und die
Tür, beide, am Tempel und am Allerheiligsten, \bibverse{24} hatte zwei
Blätter, die man auf und zu tat. \bibverse{25} Und waren auch Cherubim
und Palmlaubwerk daran, wie an den Wänden. Und davor waren starke
Riegel, gegen der Halle. \bibverse{26} Und waren enge Fenster und viel
Palmlaubwerks herum an der Halle und an den Wänden.

\hypertarget{section-41}{%
\section{42}\label{section-41}}

\bibverse{1} Und er führete mich hinaus zum äußern Vorhof gegen
Mitternacht unter die Kammern, so gegen dem Gebäude, das am Tempel hing,
und gegen dem Tempel zu Mitternacht lagen, \bibverse{2} welcher Platz
hundert Ellen lang war von dem Tor an gegen Mitternacht und fünfzig
Ellen breit. \bibverse{3} Zwanzig Ellen waren gegen dem innern Vorhof
und gegen dem Pflaster im äußern Vorhof und dreißig Ellen von einer Ecke
zur andern. \bibverse{4} Und inwendig vor den Kammern war ein Platz zehn
Ellen breit vor den Türen der Kammern, das lag alles gegen Mitternacht.
\bibverse{5} Und über diesen Kammern waren andere, engere Kammern; denn
der Raum auf den untern und mittlern Kammern war nicht groß.
\bibverse{6} Denn es war drei Gemächer hoch, und hatten doch keine
Pfeiler, wie die Vorhöfe Pfeiler hatten, sondern sie waren schlecht
aufeinandergesetzt. \bibverse{7} Und der äußere Vorhof war umfangen mit
einer Mauer, daran die Kammern stunden; die war fünfzig Ellen lang.
\bibverse{8} Und die Kammern stunden nacheinander, auch fünfzig Ellen
lang, am äußern Vorhofe; aber der Raum vor dem Tempel war hundert Ellen
lang. \bibverse{9} Und unten vor den Kammern war ein Platz gegen Morgen,
da man aus dem äußern Vorhof ging. \bibverse{10} Und an der Mauer von
Morgen an waren auch Kammern. \bibverse{11} Und war auch ein Platz
davor, wie vor jenen Kammern, gegen Mitternacht; und war alles gleich
mit der Länge, Breite und allem, was daran war, wie droben an jenen.
\bibverse{12} Und gegen Mittag waren auch eben solche Kammern mit ihren
Türen; und vor dem Platz war die Tür gegen Mittag, dazu man kommt von
der Mauer, die gegen Morgen liegt. \bibverse{13} Und er sprach zu mir:
Die Kammern gegen Mitternacht und die Kammern gegen Mittag gegen dem
Tempel, die gehören zum Heiligtum, darin die Priester essen, wenn sie
dem HErrn opfern das allerheiligste Opfer. Und sollen die
allerheiligsten Opfer, nämlich Speisopfer, Sündopfer und Schuldopfer,
daselbst hineinlegen; denn es ist eine heilige Stätte. \bibverse{14} Und
wenn die Priester hineingehen, sollen sie nicht wieder aus dem Heiligtum
gehen in den äußern Vorhof, sondern sollen zuvor ihre Kleider, darin sie
gedienet haben, in denselbigen Kammern weglegen, denn sie sind heilig;
und sollen ihre andern Kleider anlegen und alsdann heraus unter das Volk
gehen. \bibverse{15} Und da er das Haus inwendig gar gemessen hatte,
führete er mich heraus zum Tor gegen Morgen und maß von demselbigen
allenthalben herum. \bibverse{16} Gegen Morgen maß er fünfhundert Ruten
lang \bibverse{17} und gegen Mitternacht maß er auch fünfhundert Ruten
lang, \bibverse{18} desgleichen gegen Mittag auch fünfhundert Ruten.
\bibverse{19} Und da er kam gegen Abend, maß er auch fünfhundert Ruten
lang. \bibverse{20} Also hatte die Mauer, die er gemessen, ins Gevierte
auf jeder Seite herum fünfhundert Ruten, damit das Heilige von dem
Unheiligen unterschieden wäre.

\hypertarget{section-42}{%
\section{43}\label{section-42}}

\bibverse{1} Und er führete mich wieder zum Tor gegen Morgen.
\bibverse{2} Und siehe, die Herrlichkeit des GOttes Israels kam von
Morgen und brausete, wie ein groß Wasser brauset; und es ward sehr licht
auf der Erde von seiner Herrlichkeit. \bibverse{3} Und war eben wie das
Gesicht, das ich gesehen hatte am Wasser Chebar, da ich kam, daß die
Stadt sollte zerstöret werden. Da fiel ich nieder auf mein Angesicht.
\bibverse{4} Und die Herrlichkeit des HErrn kam hinein zum Hause durchs
Tor gegen Morgen. \bibverse{5} Da hub mich ein Wind auf und brachte mich
in den innern Vorhof; und siehe, die Herrlichkeit des HErrn erfüllete
das Haus. \bibverse{6} Und ich hörete einen mit mir reden vom Hause
heraus. Und ein Mann stund neben mir, \bibverse{7} der sprach zu mir: Du
Menschenkind, das ist der Ort meines Throns und die Stätte meiner
Fußsohlen, darin ich ewiglich will wohnen unter den Kindern Israel. Und
das Haus Israel soll nicht mehr meinen heiligen Namen verunreinigen,
weder sie noch ihre Könige, durch ihre Hurerei und durch die Leichen
ihrer Könige in ihren Höhen, \bibverse{8} welche ihre Schwelle an meine
Schwelle und ihre Pfosten an meine Pfosten gesetzt haben, daß nur eine
Wand zwischen mir und ihnen war, und haben also meinen heiligen Namen
verunreiniget durch ihre Greuel, die sie taten, darum ich sie auch in
meinem Zorn verzehret habe. \bibverse{9} Nun aber sollen sie ihre
Hurerei und die Leichen ihrer Könige ferne von mir wegtun; und ich will
ewiglich unter ihnen wohnen. \bibverse{10} Und du, Menschenkind, zeige
dem Hause Israel den Tempel an, daß sie sich schämen ihrer Missetat, und
laß sie ein reinlich Muster davon nehmen. \bibverse{11} Und wenn sie
sich nun alles ihres Tuns schämen, so zeige ihnen die Weise und Muster
des Hauses und seinen Ausgang und Eingang und alle seine Weise und alle
seine Sitten und alle seine Weise und alle seine Gesetze und schreibe es
ihnen vor, daß sie alle seine Weise und alle seine Sitten halten und
danach tun. \bibverse{12} Das soll aber das Gesetz des Hauses sein: Auf
der Höhe des Berges, soweit es umfangen hat, soll es das Allerheiligste
sein; das ist das Gesetz des Hauses. \bibverse{13} Dies ist aber das Maß
des Altars nach der Elle, welche einer Hand breit länger ist denn eine
gemeine Elle: Sein Fuß ist eine Elle hoch und eine Elle breit; und der
Altar reicht hinauf bis an den Rand, der ist eine Spanne breit umher;
und das ist seine Höhe. \bibverse{14} Und von dem Fuß auf der Erde bis
an den untern Absatz sind zwo Ellen hoch und eine Elle breit; aber von
demselben kleinem Absatz bis an den größern Absatz sind's vier Ellen
hoch und eine Elle breit. \bibverse{15} Und der Harel vier Ellen hoch
und vom Ariel überwärts vier Hörner. \bibverse{16} Der Ariel war aber
zwölf Ellen lang und zwölf Ellen breit ins Gevierte. \bibverse{17} Und
der oberste Absatz war vierzehn Ellen lang und vierzehn Ellen breit ins
Gevierte; und ein Rand ging allenthalben umher, einer halben Elle breit;
und sein Fuß war eine Elle hoch, und seine Stufen waren gegen Morgen.
\bibverse{18} Und er sprach zu mir: Du Menschenkind, so spricht der HErr
HErr: Dies sollen die Sitten des Altars sein des Tages, da er gemacht
ist, daß man Brandopfer darauf lege und das Blut darauf sprenge.
\bibverse{19} Und den Priestern von Levi aus dem Samen Zadok, die da vor
mich treten, daß sie mir dienen, spricht der HErr HErr, sollst du geben
einen jungen Farren zum Sündopfer. \bibverse{20} Und von desselben Blut
sollst du nehmen und seine vier Hörner damit besprengen und die vier
Ecken an dem obersten Absatz und um die Leisten herum; damit sollst du
ihn entsündigen und versöhnen. \bibverse{21} Und sollst den Farren des
Sündopfers nehmen und ihn verbrennen an einem Ort im Hause, das dazu
verordnet ist, außer dem Heiligtum. \bibverse{22} Aber am andern Tage
sollst du einen Ziegenbock opfern, der ohne Wandel sei, zu einem
Sündopfer und den Altar damit entsündigen, wie er mit dem Farren
entsündiget ist. \bibverse{23} Und wenn das Entsündigen vollendet ist,
sollst du einen jungen Farren opfern, der ohne Wandel sei, und einen
Widder von der Herde ohne Wandel. \bibverse{24} Und sollst sie beide vor
dem HErrn opfern; und die Priester sollen Salz darauf streuen und sollen
sie also opfern dem HErrn zum Brandopfer. \bibverse{25} Also sollst du
sieben Tage nacheinander täglich einen Bock zum Sündopfer opfern; und
sie sollen einen jungen Farren und einen Widder von der Herde, die beide
ohne Wandel sind, opfern. \bibverse{26} Und sollen also sieben Tage lang
den Altar versöhnen und ihn reinigen und seine Hände füllen.
\bibverse{27} Und nach denselben Tagen sollen die Priester am achten
Tage und hernach für und für auf dem Altar opfern eure Brandopfer und
eure Dankopfer, so will ich euch gnädig sein, spricht der HErr HErr.

\hypertarget{section-43}{%
\section{44}\label{section-43}}

\bibverse{1} Und er führete mich wiederum zu dem Tor des äußern
Heiligtums gegen Morgen; es war aber zugeschlossen. \bibverse{2} Und der
HErr sprach zu mir: Dies Tor soll zugeschlossen bleiben und nicht
aufgetan werden; und soll niemand da durch gehen, ohne allein der HErr,
der GOtt Israels, soll dadurchgehen; und soll zugeschlossen bleiben.
\bibverse{3} Doch den Fürsten ausgenommen; denn der Fürst soll darunter
sitzen, das Brot zu essen vor dem HErrn; durch die Halle soll er
hineingehen und durch dieselbige wieder herausgehen. \bibverse{4} Danach
führete er mich zum Tor gegen Mitternacht vor das Haus; und ich sah, und
siehe, des HErrn Haus ward voll der Herrlichkeit des HErrn; und ich fiel
auf mein Angesicht. \bibverse{5} Und der HErr sprach zu mir: Du
Menschenkind, merke eben darauf und siehe und höre fleißig auf alles,
was ich dir sagen will von allen Sitten und Gesetzen im Hause des HErrn;
und merke eben, wie man hineingehen soll, und auf alle Ausgänge des
Heiligtums. \bibverse{6} Und sage dem ungehorsamen Hause Israel: So
spricht der HErr HErr: Ihr macht's zu viel, ihr vom Hause Israel, mit
allen euren Greueln; \bibverse{7} denn ihr führet fremde Leute, eines
unbeschnittenen Herzens und unbeschnittenen Fleisches, in mein
Heiligtum, dadurch ihr mein Haus entheiliget, wenn ihr mein Brot, Fettes
und Blut opfert, und brechet also meinen Bund mit allen euren Greueln
\bibverse{8} und haltet die Sitten meines Heiligtums nicht, sondern
macht euch selbst neue Sitten in meinem Heiligtum. \bibverse{9} Darum
spricht der HErr HErr also: Es soll kein Fremder eines unbeschnittenen
Herzens und unbeschnittenen Fleisches in mein Heiligtum kommen aus allen
Fremdlingen, so unter den Kindern Israel sind, \bibverse{10} ja, auch
nicht die Leviten, die von mir gewichen sind und samt Israel von mir
irregegangen nach ihren Götzen. Darum sollen sie ihre Sünde tragen.
\bibverse{11} Sie sollen aber in meinem Heiligtum an den Ämtern, den
Türen des Hauses und dem Hause dienen und sollen nur das Brandopfer und
andere Opfer, so das Volk herzubringet, schlachten und vor den Priestern
stehen, daß sie ihnen dienen. \bibverse{12} Darum daß sie jenen gedienet
vor ihren Götzen und dem Hause Israel ein Ärgernis zur Sünde gegeben
haben, darum hab ich meine Hand über sie ausgestreckt, spricht der HErr
HErr, daß sie müssen ihre Sünde tragen. \bibverse{13} Und sollen nicht
zu mir nahen, Priesteramt zu führen, noch kommen zu einigem meinem
Heiligtum, zu dem Allerheiligsten, sondern sollen ihre Schande tragen
und ihre Greuel, die sie geübt haben. \bibverse{14} Darum hab ich sie zu
Hütern gemacht an allem Dienst des Hauses und zu allem, das man drin tun
soll. \bibverse{15} Aber die Priester aus den Leviten, die Kinder
Zadoks, so die Sitten meines Heiligtums gehalten haben, da die Kinder
Israel von mir abfielen, die sollen vor mich treten und mir dienen und
vor mir stehen, daß sie mir das Fette und Blut opfern, spricht der HErr
HErr. \bibverse{16} Und sie sollen hineingehen in mein Heiligtum und vor
meinen Tisch treten, mir zu dienen und meine Sitten zu halten.
\bibverse{17} Und wenn sie durch die Tore des innern Vorhofs gehen
wollen, sollen sie leinene Kleider anziehen und nichts Wollenes anhaben,
weil sie in den Toren im innern Vorhofe dienen. \bibverse{18} Und sollen
leinenen Schmuck auf ihrem Haupt haben und leinen Niederkleid um ihre
Lenden; und sollen sich nicht im Schweiß gürten. \bibverse{19} Und wenn
sie etwa zu einem äußern Vorhof zum Volk herausgehen, sollen sie die
Kleider, darin sie gedienet haben, ausziehen und dieselben in die
Kammern des Heiligtums legen und andere Kleider anziehen und das Volk
nicht heiligen in ihren eigenen Kleidern. \bibverse{20} Ihr Haupt sollen
sie nicht bescheren und sollen auch nicht die Haare frei wachsen lassen,
sondern sollen die Haare umher verschneiden. \bibverse{21} Und soll auch
kein Priester keinen Wein trinken, wenn sie in den innern Vorhof gehen
sollen. \bibverse{22} Und sollen keine Witwe noch Verstoßene zur Ehe
nehmen, sondern Jungfrauen vom Samen des Hauses Israel, oder eines
Priesters nachgelassene Witwe. \bibverse{23} Und sie sollen mein Volk
lehren, daß sie wissen Unterschied zu halten zwischen Heiligem und
Unheiligem und zwischen Reinem und Unreinem. \bibverse{24} Und wo eine
Sache vor sie kommt, sollen sie stehen und richten und nach meinen
Rechten sprechen und meine Gebote und Sitten halten und alle meine Feste
halten und meine Sabbate heiligen. \bibverse{25} Und sollen zu keinem
Toten gehen und sich verunreinigen, ohne allein zu Vater und Mutter,
Sohn oder Tochter, Bruder oder Schwester, die noch keinen Mann gehabt
habe; über denen mögen sie sich verunreinigen. \bibverse{26} Und nach
seiner Reinigung soll man ihm zählen sieben Tage. \bibverse{27} Und wenn
er wieder hinein zum Heiligtum gehet in den innern Vorhof, daß er im
Heiligtum diene, so soll er sein Sündopfer opfern, spricht der HErr
HErr. \bibverse{28} Aber das Erbteil, das sie haben sollen, das will ich
selbst sein. Darum sollt ihr ihnen kein eigen Land geben in Israel; denn
ich bin ihr Erbteil. \bibverse{29} Sie sollen ihre Nahrung haben vom
Speisopfer, Sündopfer und Schuldopfer, und alles Verbannte in Israel
soll ihr sein. \bibverse{30} Und alle ersten Früchte und Erstgeburt von
allen Hebopfern sollen der Priester sein. Ihr sollt auch den Priestern
die Erstlinge geben von allem, das man isset, damit der Segen in deinem
Hause bleibe. \bibverse{31} Was aber ein Aas oder zerrissen ist, es sei
von Vögeln oder Tieren, das sollen die Priester nicht essen.

\hypertarget{section-44}{%
\section{45}\label{section-44}}

\bibverse{1} Wenn ihr nun das Land durchs Los austeilet, so sollt ihr
ein Hebopfer vom Lande absondern, das dem HErrn heilig sein soll,
fünfundzwanzigtausend (Ruten) lang und zehntausend breit. Der Platz soll
heilig sein, soweit er reicht. \bibverse{2} Und von diesem sollen zum
Heiligtum kommen je fünfhundert (Ruten) ins Gevierte und dazu ein freier
Raum umher fünfzig Ellen. \bibverse{3} Und auf demselben Platz, der
fünfundzwanzigtausend Ruten lang und zehntausend breit ist, soll das
Heiligtum stehen und das Allerheiligste. \bibverse{4} Das übrige aber
vom geheiligten Lande soll den Priestern gehören, die im Heiligtum
dienen und vor den HErrn treten, ihm zu dienen, daß sie Raum zu Häusern
haben, und soll auch heilig sein. \bibverse{5} Aber die Leviten, so vor
dem Hause dienen, sollen auch fünfundzwanzigtausend (Ruten) lang und
zehntausend breit haben zu ihrem Teil zu zwanzig Kammern. \bibverse{6}
Und der Stadt sollt ihr auch einen Platz lassen für das ganze Haus
Israel, fünftausend (Ruten) breit und fünfundzwanzigtausend lang, neben
dem abgesonderten Platz des Heiligtums. \bibverse{7} Dem Fürsten aber
sollt ihr auch einen Platz geben zu beiden Seiten zwischen dem Platz der
Priester und zwischen dem Platz der Stadt, gegen Abend und gegen Morgen;
und sollen beide gegen Morgen und gegen Abend gleich lang sein.
\bibverse{8} Das soll sein eigen Teil sein in Israel, damit meine
Fürsten nicht mehr meinem Volk das Ihre nehmen, sondern sollen das Land
dem Hause Israel lassen für ihre Stämme. \bibverse{9} Denn so spricht
der HErr HErr: Ihr habt es lange genug gemacht, ihr Fürsten Israels;
laßt ab vom Frevel und Gewalt und tut, was recht und gut ist, und tut ab
von meinem Volk euer Austreiben, spricht der HErr HErr. \bibverse{10}
Ihr sollt recht Gewicht und rechte Scheffel und recht Maß haben.
\bibverse{11} Epha und Bath sollen gleich sein, daß ein Bath das zehnte
Teil vom Homer habe und das Epha auch das zehnte Teil vom Homer; denn
nach dem Homer soll man sie beide messen. \bibverse{12} Aber ein Sekel
soll zwanzig Gera haben; und eine Mina macht zwanzig Sekel,
fünfundzwanzig Sekel und fünfzehn Sekel. \bibverse{13} Das soll nun das
Hebopfer sein, das ihr heben sollt, nämlich das sechste Teil eines Epha
von einem Homer Weizen und das sechste Teil eines Epha von einem Homer
Gerste. \bibverse{14} Und vom Öl sollt ihr geben einen Bath, nämlich je
den zehnten Bath vom Kor und den zehnten vom Homer; denn zehn Bath
machen einen Homer. \bibverse{15} Und je ein Lamm von zweihundert
Schafen aus der Herde auf der Weide Israels zum Speisopfer und
Brandopfer und Dankopfer, zur Versöhnung für sie, spricht der HErr HErr.
\bibverse{16} Alles Volk im Lande soll solch Hebopfer zum Fürsten in
Israel bringen. \bibverse{17} Und der Fürst soll sein Brandopfer,
Speisopfer und Trankopfer opfern auf die Feste, Neumonden und Sabbate
und auf alle hohen Feste des Hauses Israel, dazu Sündopfer und
Speisopfer, Brandopfer und Dankopfer tun zur Versöhnung für das Haus
Israel. \bibverse{18} So spricht der HErr HErr: Am ersten Tage des
ersten Monden sollst du nehmen einen jungen Farren, der ohne Wandel sei,
und das Heiligtum entsündigen. \bibverse{19} Und der Priester soll von
dem Blut des Sündopfers nehmen und die Pfosten am Hause damit besprengen
und die vier Ecken des Absatzes am Altar samt den Pfosten am Tor des
innern Vorhofs. \bibverse{20} Also sollst du auch tun am siebenten Tage
des Monden, wo jemand geirret hat oder verführet ist, daß ihr das Haus
entsündiget. \bibverse{21} Am vierzehnten Tage des ersten Monden sollt
ihr das Passah halten und sieben Tage feiern und ungesäuert Brot essen.
\bibverse{22} Und am selbigen Tage soll der Fürst für sich und für alles
Volk im Lande einen Farren zum Sündopfer opfern. \bibverse{23} Aber die
sieben Tage des Festes soll er dem HErrn täglich ein Brandopfer tun, je
sieben Farren und sieben Widder, die ohne Wandel seien, und je einen
Ziegenbock zum Sündopfer. \bibverse{24} Zum Speisopfer aber soll er je
ein Epha zu einem Farren und ein Epha einem Widder opfern und je ein Hin
Öls zu einem Epha. \bibverse{25} Am fünfzehnten Tage des siebenten
Monden soll er sieben Tage nacheinander feiern, gleichwie jene sieben
Tage, und ebenso halten mit Sündopfer, Brandopfer, Speisopfer samt dem
Öl.

\hypertarget{section-45}{%
\section{46}\label{section-45}}

\bibverse{1} So spricht der HErr HErr: Das Tor am innern Vorhofe gegen
morgenwärts soll die sechs Werktage zugeschlossen sein; aber am
Sabbattage und am Neumonden soll man's auftun. \bibverse{2} Und der
Fürst soll auswendig unter die Halle des Tors treten und draußen bei den
Pfosten am Tor stehenbleiben. Und die Priester sollen sein Brandopfer
und Dankopfer opfern; er aber soll auf der Schwelle des Tors anbeten und
danach wieder hinausgehen. Das Tor aber soll offen bleiben bis an den
Abend. \bibverse{3} Desgleichen das Volk im Lande sollen in der Tür
desselben Tors anbeten vor dem HErrn an den Sabbaten und Neumonden.
\bibverse{4} Das Brandopfer aber, so der Fürst vor dem HErrn opfern soll
am Sabbattage, soll sein sechs Lämmer, die ohne Wandel seien, und ein
Widder ohne Wandel \bibverse{5} und je ein Epha Speisopfer zu einem
Widder zum Speisopfer. Zu den Lämmern aber, soviel seine Hand gibt, zum
Speisopfer und je ein Hin Öls zu einem Epha. \bibverse{6} Am Neumonden
aber soll er einen jungen Farren opfern, der ohne Wandel sei, und sechs
Lämmer und einen Widder, auch ohne Wandel, \bibverse{7} und je ein Epha
zum Farren und je ein Epha zum Widder zum Speisopfer. Aber zu den
Lämmern so viel, als er greift; und je ein Hin Öls zu einem Epha.
\bibverse{8} Und wenn der Fürst hineingehet, soll er durch die Halle des
Tors hineingehen und desselben Weges wieder herausgehen. \bibverse{9}
Aber das Volk im Lande, so vor den HErrn kommt auf die hohen Feste und
zum Tor gegen Mitternacht hineingehet anzubeten, das soll durch das Tor
gegen Mittag wieder herausgehen; und welche zum Tor gegen Mittag
hineingehen, die sollen zum Tor gegen Mitternacht wieder herausgehen und
sollen nicht wieder zu dem Tor hinausgehen, dadurch sie hinein sind
gegangen, sondern stracks vor sich hinausgehen. \bibverse{10} Der Fürst
aber soll mit ihnen beide hinein und herausgehen. \bibverse{11} Aber an
den Feiertagen und hohen Festen soll man zum Speisopfer je zu einem
Farren ein Epha und je zu einem Widder ein Epha opfern und zu den
Lämmern, soviel seine Hand gibt, und je ein Hin Öls zu einem Epha.
\bibverse{12} Wenn aber der Fürst ein freiwillig Brandopfer oder
Dankopfer dem HErrn tun wollte, so soll man ihm das Tor gegen
morgenwärts auftun, daß er sein Brandopfer und Dankopfer opfere, wie er
sonst am Sabbat pflegt zu opfern; und wenn er wieder herausgehet, soll
man das Tor nach ihm zuschließen. \bibverse{13} Und er soll dem HErrn
täglich ein Brandopfer tun, nämlich ein jähriges Lamm ohne Wandel;
dasselbe soll er alle Morgen opfern. \bibverse{14} Und soll alle Morgen
das sechste Teil von einem Epha zum Speisopfer darauf tun und ein
drittes Teil von einem Hin Öls, auf das Semmelmehl zu träufen, dem HErrn
zum Speisopfer. Das soll ein ewiges Recht sein, vom täglichen Opfer.
\bibverse{15} Und also sollen sie das Lamm samt dem Speisopfer und Öl
alle Morgen opfern zum täglichen Brandopfer. \bibverse{16} So spricht
der HErr HErr: Wenn der Fürst seiner Söhne einem ein Geschenk gibt von
seinem Erbe, dasselbe soll seinen Söhnen bleiben, und sollen es erblich
besitzen. \bibverse{17} Wo er aber seiner Knechte einem von seinem
Erbteil etwas schenket, das sollen sie besitzen bis aufs Freijahr, und
soll alsdann dem Fürsten wieder heimfallen; denn sein Teil soll allein
auf seine Söhne erben. \bibverse{18} Es soll auch der Fürst dem Volk
nichts nehmen von seinem Erbteil, noch sie aus ihren eigenen Gütern
stoßen, sondern soll sein eigen Gut auf seine Kinder erben, auf daß
meines Volks nicht jemand von seinem Eigentum zerstreuet werde.
\bibverse{19} Und er führete mich unter den Eingang an der Seite des
Tors gegen Mitternacht zu den Kammern des Heiligtums, so den Priestern
gehörten, und siehe, daselbst war ein Raum in einer Ecke gegen Abend.
\bibverse{20} Und er sprach zu mir: Dies ist der Ort, da die Priester
kochen sollen das Schuldopfer und Sündopfer und das Speisopfer backen,
daß sie es nicht hinaus in den äußern Vorhof tragen dürfen, das Volk zu
heiligen. \bibverse{21} Danach führete er mich hinaus in den äußern
Vorhof und hieß mich gehen in die vier Ecken des Vorhofs. \bibverse{22}
Und siehe, da war ein jeglicher der vier Ecken ein ander Vorhöflein zu
räuchern, vierzig Ellen lang und dreißig Ellen breit, alle vier einerlei
Maß. \bibverse{23} Und es ging ein Mäuerlein um ein jegliches der vier;
da waren Herde herum gemacht unten an den Mauern. \bibverse{24} Und er
sprach zu mir: Dies ist die Küche, darin die Diener im Hause kochen
sollen, was das Volk opfert.

\hypertarget{section-46}{%
\section{47}\label{section-46}}

\bibverse{1} Und er führete mich wieder zu der Tür des Tempels. Und
siehe, da floß ein Wasser heraus unter der Schwelle des Tempels gegen
Morgen; denn die Tür des Tempels war auch gegen Morgen. Und das Wasser
lief an der rechten Seite des Tempels neben dem Altar hin gegen Mittag.
\bibverse{2} Und er führete mich auswendig zum Tor gegen Mitternacht vom
äußern Tor gegen Morgen; und siehe, das Wasser sprang heraus von der
rechten Seite. \bibverse{3} Und der Mann ging heraus gegen Morgen und
hatte die Meßschnur in der Hand; und er maß tausend Ellen und führete
mich durchs Wasser, bis mir's an die Knöchel ging. \bibverse{4} Und maß
abermal tausend Ellen und führete mich durchs Wasser, bis mir's an die
Kniee ging. Und maß noch tausend Ellen und ließ mich dadurch gehen, bis
es mir an die Lenden ging. \bibverse{5} Da maß er noch tausend Ellen,
und es war so tief, daß ich nicht mehr gründen konnte; denn das Wasser
war zu hoch, daß man darüber schwimmen mußte und konnte es nicht
gründen. \bibverse{6} Und er sprach zu mir: Du Menschenkind, das hast du
ja gesehen. Und er führete mich wieder zurück am Ufer des Bachs.
\bibverse{7} Und siehe, da stunden sehr viel Bäume am Ufer auf beiden
Seiten. \bibverse{8} Und er sprach zu mir: Dies Wasser das da gegen
Morgen herausfleußt, wird durchs Blachfeld fließen ins Meer und von
einem Meer ins andere, und wenn es dahin ins Meer kommt, da sollen
dieselbigen Wasser gesund werden. \bibverse{9} Ja alles, was darin lebt
und webt, dahin diese Ströme kommen, das soll leben, und soll sehr viel
Fische haben; und soll alles gesund werden und leben, wo dieser Strom
hinkommt. \bibverse{10} Und es werden die Fischer an demselben stehen;
von Engeddi bis zu En- Eglaim wird man die Fischgarne aufspannen; denn
es werden daselbst sehr viel Fische sein, gleichwie im großen Meer.
\bibverse{11} Aber die Teiche und Lachen daneben werden nicht gesund
werden, sondern gesalzen bleiben. \bibverse{12} Und an demselben Strom,
am Ufer auf beiden Seiten, werden allerlei fruchtbare Bäume wachsen; und
ihre Blätter werden nicht verwelken noch ihre Früchte verfaulen; und
werden alle Monden neue Früchte bringen, denn ihr Wasser fleußt aus dem
Heiligtum. Ihre Frucht wird zur Speise dienen und ihre Blätter zur
Arznei. \bibverse{13} So spricht der HErr HErr: Dies ist die Grenze,
nach der ihr das Land sollt austeilen den zwölf Stämmen Israels; denn
zwei Teile gehören dem Stamm Joseph. \bibverse{14} Und ihr sollt es
gleich austeilen, einem wie dem andern; denn ich habe meine Hand
aufgehoben, das Land euren Vätern und euch zum Erbteil zu geben.
\bibverse{15} Dies ist nun die Grenze des Landes gegen Mitternacht von
dem großen Meer an, von Hethlon bis gen Zedad: \bibverse{16} nämlich
Hemath, Berotha, Sibraim, die mit Damaskus und Hemath grenzen; und
Hazar-Tichon, die mit Haveran grenzet. \bibverse{17} Das soll die Grenze
sein vom Meer an bis gen Hazar-Enon; und Damaskus und Hemath sollen das
Ende sein gegen Mitternacht. \bibverse{18} Aber die Grenze gegen Morgen
sollt ihr messen zwischen Haveran und Damaskus und zwischen Gilead und
zwischen dem Lande Israel, am Jordan hinab bis ans Meer gegen Morgen.
Das soll die Grenze gegen Morgen sein. \bibverse{19} Aber die Grenze
gegen Mittag ist von Thamar bis ans Haderwasser zu Kades und gegen das
Wasser am großen Meer. Das soll die Grenze gegen Mittag sein.
\bibverse{20} Und die Grenze gegen Abend ist vom großen Meer an stracks
bis gen Hemath. Das sei die Grenze gegen Abend. \bibverse{21} Also sollt
ihr das Land austeilen unter die Stämme Israels. \bibverse{22} Und wenn
ihr das Los werfet, das Land unter euch zu teilen, so sollt ihr die
Fremdlinge, die bei euch wohnen und Kinder unter euch zeugen, halten
gleich wie die Einheimischen unter den Kindern Israel; \bibverse{23} und
sollen auch ihren Teil am Lande haben, ein jeglicher unter dem Stamm,
dabei er wohnet, spricht der HErr HErr.

\hypertarget{section-47}{%
\section{48}\label{section-47}}

\bibverse{1} Dies sind die Namen der Stämme. Von Mitternacht, von
Hethlon gegen Hemath und Hazar-Enon und von Damaskus gegen Hemath; das
soll Dan für seinen Teil haben von Morgen bis gen Abend. \bibverse{2}
Neben Dan soll Asser seinen Teil haben von Morgen bis gen Abend.
\bibverse{3} Neben Asser soll Naphthali seinen Teil haben von Morgen bis
gen Abend. \bibverse{4} Neben Naphthali soll Manasse seinen Teil haben
von Morgen bis gen Abend. \bibverse{5} Neben Manasse soll Ephraim seinen
Teil haben von Morgen bis gen Abend. \bibverse{6} Neben Ephraim soll
Ruben seinen Teil haben von Morgen bis gen Abend. \bibverse{7} Neben
Ruben soll Juda seinen Teil haben von Morgen bis gen Abend. \bibverse{8}
Neben Juda aber sollt ihr einen Teil absondern vom Morgen bis gegen
Abend, der fünfundzwanzigtausend Ruten breit und lang sei, ein Stück von
den Teilen, so von Morgen bis gen Abend reichen; darin soll das
Heiligtum stehen. \bibverse{9} Und davon sollt ihr dem HErrn einen Teil
absondern, fünfundzwanzigtausend Ruten lang und zehntausend Ruten breit.
\bibverse{10} Und dasselbige heilige Teil soll der Priester sein,
nämlich fünfundzwanzigtausend Ruten lang gegen Mitternacht und gegen
Mittag und zehntausend breit gegen Morgen und gegen Abend. Und das
Heiligtum des HErrn soll mitten drinnen stehen. \bibverse{11} Das soll
geheiliget sein den Priestern, den Kindern Zadoks, welche meine Sitten
gehalten haben und sind nicht abgefallen mit den Kindern Israel, wie die
Leviten abgefallen sind. \bibverse{12} Und soll also dies abgesonderte
Teil des Landes ihr eigen sein, darin das Allerheiligste ist neben der
Leviten Grenze. \bibverse{13} Die Leviten aber sollen neben der Priester
Grenze auch fünfundzwanzigtausend Ruten in die Länge und zehntausend in
die Breite haben; denn alle Länge soll fünfundzwanzigtausend und die
Breite zehntausend Ruten haben. \bibverse{14} Und sollen nichts davon
verkaufen noch verändern, damit des Landes Erstling nicht wegkomme; denn
es ist dem HErrn geheiliget. \bibverse{15} Aber die übrigen fünftausend
Ruten in die Breite gegen die fünfundzwanzigtausend Ruten in die Länge,
das soll unheilig sein zur Stadt, drinnen zu wohnen, und zu Vorstädten;
und die Stadt soll mitten drinnen stehen. \bibverse{16} Und das soll ihr
Maß sein: viertausend und fünfhundert Ruten gegen Mitternacht und gegen
Mittag; desgleichen gegen Morgen und gegen Abend auch viertausend und
fünfhundert. \bibverse{17} Die Vorstadt aber soll haben
zweihundertundfünfzig Ruten gegen Mitternacht und gegen Mittag;
desgleichen auch gegen Morgen und gegen Abend zweihundertundfünfzig
Ruten. \bibverse{18} Aber das übrige an der Länge desselben neben dem
Abgesonderten und Geheiligten, nämlich zehntausend Ruten gegen Morgen
und gegen Abend, das gehört zur Unterhaltung derer, die in der Stadt
arbeiten. \bibverse{19} Und die Arbeiter sollen aus allen Stämmen
Israels der Stadt arbeiten, \bibverse{20} daß die ganze Absonderung der
fünfundzwanzigtausend Ruten ins Gevierte eine geheiligte Absonderung sei
zu eigen der Stadt. \bibverse{21} Was aber noch übrig ist auf beiden
Seiten neben dem abgesonderten heiligen Teil und neben der Stadt Teil,
nämlich fünfundzwanzigtausend Ruten gegen Morgen und gegen Abend, das
soll alles des Fürsten sein. Aber das abgesonderte heilige Teil und das
Haus des Heiligtums soll mitten innen sein. \bibverse{22} Was aber
dazwischen liegt, zwischen der Leviten Teil und zwischen der Stadt Teil
und zwischen der Grenze Judas und der Grenze Benjamins, das soll des
Fürsten sein. \bibverse{23} Danach sollen die andern Stämme sein:
Benjamin soll seinen Teil haben von Morgen bis gen Abend. \bibverse{24}
Aber neben der Grenze Benjamins soll Simeon seinen Teil haben von Morgen
bis gen Abend. \bibverse{25} Neben der Grenze Simeons soll Isaschar
seinen Teil haben von Morgen bis gen Abend. \bibverse{26} Neben der
Grenze Isaschars soll Sebulon seinen Teil haben von Morgen bis gen
Abend. \bibverse{27} Neben der Grenze Sebulons soll Gad seinen Teil
haben von Morgen bis gen Abend. \bibverse{28} Aber neben Gad ist die
Grenze gegen Mittag, von Thamar bis an das Haderwasser zu Kades und
gegen dem Wasser am großen Meer. \bibverse{29} Also soll das Land
ausgeteilet werden zum Erbteil unter die Stämme Israels; und das soll
ihr Erbteil sein, spricht der HErr HErr. \bibverse{30} Und so weit soll
die Stadt sein: viertausend und fünfhundert Ruten gegen Mitternacht.
\bibverse{31} Und die Tore der Stadt sollen nach den Namen der Stämme
Israels genannt werden, drei Tore gegen Mitternacht: das erste Tor
Ruben, das andere Juda, das dritte Levi. \bibverse{32} Also auch gegen
Morgen viertausend und fünfhundert Ruten und auch drei Tore: nämlich das
erste Tor Joseph, das andere Benjamin, das dritte Dan. \bibverse{33}
Gegen Mittag auch also: viertausend und fünfhundert Ruten und auch drei
Tore: das erste Tor Simeon, das andere Isaschar, das dritte Sebulon.
\bibverse{34} Also auch gegen Abend viertausend und fünfhundert Ruten
und drei Tore: ein Tor Gad, das andere Assur, das dritte Naphthali.
\bibverse{35} Also soll es um und um achtzehntausend Ruten haben. Und
alsdann soll die Stadt genannt werden: Hie ist der HErr!
