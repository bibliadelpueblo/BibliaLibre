\hypertarget{section}{%
\section{1}\label{section}}

\bibverse{1} Und der HErr redete mit Mose in der Wüste Sinai in der
Hütte des Stifts am ersten Tage des zweiten Monats im zweiten Jahr, da
sie aus Ägyptenland gegangen waren, und sprach: \bibverse{2} Nehmet die
Summe der ganzen Gemeinde der Kinder Israel nach ihren Geschlechtern und
Vaterhäusern und Namen, alles, was männlich ist, von Haupt zu Haupt,
\bibverse{3} von 20 Jahren an und darüber, was ins Heer zu ziehen taugt
in Israel; ihr sollt sie zählen nach ihren Heeren, du und Aaron.
\bibverse{4} Und sollt zu euch nehmen je vom Stamm einen Hauptmann über
sein Vaterhaus. \bibverse{5} Dies sind aber die Namen der Hauptleute,
die neben euch stehen sollen: von Ruben sei Elizur, der Sohn Sedeurs;

\bibverse{6} von Simeon sei Selumiel, der Sohn Zuri-Saddais;

\bibverse{7} von Juda sei Nahesson, der Sohn Amminadabs; \footnote{\textbf{1:7}
  2Mo 6,23}

\bibverse{8} von Isaschar sei Nathanael, der Sohn Zuars;

\bibverse{9} von Sebulon sei Eliab, der Sohn Helons;

\bibverse{10} von den Kindern Josephs: von Ephraim sei Elisama, der Sohn
Ammihuds; von Manasse sei Gamliel, der Sohn Pedazurs;

\bibverse{11} von Benjamin sei Abidan, der Sohn des Gideoni;

\bibverse{12} von Dan sei Ahieser, der Sohn Ammi-Saddais;

\bibverse{13} von Asser sei Pagiel, der Sohn Ochrans;

\bibverse{14} von Gad sei Eljasaph, der Sohn Deguels;

\bibverse{15} von Naphthali sei Ahira, der Sohn Enans.

\bibverse{16} Das sind die Vornehmsten der Gemeinde, die Fürsten unter
den Stämmen ihrer Väter, die da Häupter über die Tausende in Israel
waren.

\bibverse{17} Und Mose und Aaron nahmen sie zu sich, wie sie da mit
Namen genannt sind, \bibverse{18} und sammelten auch die ganze Gemeinde
am ersten Tage des zweiten Monats und rechneten sie nach ihrer Geburt,
nach ihren Geschlechtern und Vaterhäusern und Namen, von 20 Jahren an
und darüber, von Haupt zu Haupt, \bibverse{19} wie der HErr dem Mose
geboten hatte, und zählten sie in der Wüste Sinai.

\bibverse{20} Der Kinder Ruben, des ersten Sohnes Israels, nach ihrer
Geburt und Geschlecht, ihren Vaterhäusern und Namen, von Haupt zu Haupt,
alles, was männlich war, von 20 Jahren und darüber, und ins Heer zu
ziehen taugte, \bibverse{21} wurden gezählt vom Stamm Ruben 46.500.

\bibverse{22} Der Kinder Simeon nach ihrer Geburt und Geschlecht, ihren
Vaterhäusern, Zahl und Namen, von Haupt zu Haupt, alles, was männlich
war, von 20 Jahren und darüber, und ins Heer zu ziehen taugte,
\bibverse{23} wurden gezählt zum Stamm Simeon 59.300.

\bibverse{24} Der Kinder Gad nach ihrer Geburt und Geschlecht, ihren
Vaterhäusern und Namen, von 20 Jahren und darüber, und ins Heer zu
ziehen taugte, \bibverse{25} wurden gezählt zum Stamm Gad 45.650.

\bibverse{26} Der Kinder Juda nach ihrer Geburt und Geschlecht, ihren
Vaterhäusern und Namen, von 20 Jahren und darüber, was ins Heer zu
ziehen taugte, \bibverse{27} wurden gezählt zum Stamm Juda 74.600.

\bibverse{28} Der Kinder Isaschar nach ihrer Geburt und Geschlecht,
ihren Vaterhäusern und Namen, von 20 Jahren und darüber, was ins Heer zu
ziehen taugte, \bibverse{29} wurden gezählt zum Stamm Isaschar 54.400.

\bibverse{30} Der Kinder Sebulon nach ihrer Geburt und Geschlecht, ihren
Vaterhäusern und Namen, von 20 Jahren und darüber, was ins Heer zu
ziehen taugte, \bibverse{31} wurden gezählt zum Stamm Sebulon 57.400.

\bibverse{32} Der Kinder Joseph von Ephraim nach ihrer Geburt und
Geschlecht, ihren Vaterhäusern und Namen, von 20 Jahren und darüber, was
ins Heer zu ziehen taugte, \bibverse{33} wurden gezählt zum Stamm
Ephraim 40.500.

\bibverse{34} Der Kinder Manasse nach ihrer Geburt und Geschlecht, ihren
Vaterhäusern und Namen, von 20 Jahren und darüber, was ins Heer zu
ziehen taugte, \bibverse{35} wurden zum Stamm Manasse gezählt 32.200.

\bibverse{36} Der Kinder Benjamin nach ihrer Geburt und Geschlecht,
ihren Vaterhäusern und Namen, von 20 Jahren und darüber, was ins Heer zu
ziehen taugte, \bibverse{37} wurden zum Stamm Benjamin gezählt 35.400.

\bibverse{38} Der Kinder Dan nach ihrer Geburt und Geschlecht, ihren
Vaterhäusern und Namen, von 20 Jahren und darüber, was ins Heer zu
ziehen taugte, \bibverse{39} wurden gezählt zum Stamme Dan 62.700.

\bibverse{40} Der Kinder Asser nach ihrer Geburt und Geschlecht, ihren
Vaterhäusern und Namen, von 20 Jahren und darüber, was ins Heer zu
ziehen taugte, \bibverse{41} wurden gezählt zum Stamm Asser 41.500.

\bibverse{42} Der Kinder Naphthali nach ihrer Geburt und Geschlecht,
ihren Vaterhäusern und Namen, von 20 Jahren und darüber, was ins Heer zu
ziehen taugte, \bibverse{43} wurden zum Stamm Naphthali gezählt 53.400.

\bibverse{44} Dies sind, die Mose und Aaron zählten samt den zwölf
Fürsten Israels, deren je einer über ein Vaterhaus war. \bibverse{45}
Und die Summe der Kinder Israel nach ihren Vaterhäusern, von 20 Jahren
und darüber, was ins Heer zu ziehen taugte in Israel, \bibverse{46} war
603.550. \footnote{\textbf{1:46} 4Mo 2,32; 2Mo 12,37} \bibverse{47} Aber
die Leviten nach ihrer Väter Stamm wurden nicht mit darunter gezählt.
\bibverse{48} Und der HErr redete mit Mose und sprach: \bibverse{49} Den
Stamm Levi sollst du nicht zählen noch ihre Summe nehmen unter den
Kindern Israel, \bibverse{50} sondern du sollst sie ordnen zur Wohnung
des Zeugnisses und zu allem Geräte und allem, was dazu gehört. Und sie
sollen die Wohnung tragen und alles Gerät und sollen sein pflegen und um
die Wohnung her sich lagern. \footnote{\textbf{1:50} 4Mo 4,-1; 4Mo
  3,23-38} \bibverse{51} Und wenn man reisen soll, so sollen die Leviten
die Wohnung abnehmen. Wenn aber das Heer zu lagern ist, sollen sie die
Wohnung aufschlagen. Und wo ein Fremder sich dazumacht, der soll
sterben. \footnote{\textbf{1:51} 4Mo 3,10; 4Mo 3,38} \bibverse{52} Die
Kinder Israel sollen sich lagern, ein jeglicher in sein Lager und zu dem
Panier seiner Schar. \bibverse{53} Aber die Leviten sollen sich um die
Wohnung des Zeugnisses her lagern, auf dass nicht ein Zorn über die
Gemeinde der Kinder Israel komme; darum sollen die Leviten des Dienstes
warten an der Wohnung des Zeugnisses.

\bibverse{54} Und die Kinder Israel taten alles, wie der HErr dem Mose
geboten hatte. \# 2 \bibverse{1} Und der HErr redete mit Mose und Aaron
und sprach: \bibverse{2} Die Kinder Israel sollen vor der Hütte des
Stifts umher sich lagern, ein jeglicher unter seinem Panier und Zeichen
nach ihren Vaterhäusern. \footnote{\textbf{2:2} 4Mo 1,1}

\bibverse{3} Gegen Morgen soll sich lagern Juda mit seinem Panier und
Heer; ihr Hauptmann Nahesson, der Sohn Amminadabs, \bibverse{4} und sein
Heer, zusammen 74.600.

\bibverse{5} Neben ihm soll sich lagern der Stamm Isaschar; ihr
Hauptmann Nathanael, der Sohn Zuars, \bibverse{6} und sein Heer,
zusammen 54.400.

\bibverse{7} Dazu der Stamm Sebulon; ihr Hauptmann Eliab, der Sohn
Helons, \bibverse{8} sein Heer, zusammen 57.400.

\bibverse{9} Dass alle, die ins Lager Judas gehören, seien zusammen
186.400, die zu ihrem Heer gehören; und sie sollen vornean ziehen.

\bibverse{10} Gegen Mittag soll liegen das Gezelt und Panier Rubens mit
ihrem Heer; ihr Hauptmann Elizur, der Sohn Sedeurs, \bibverse{11} und
sein Heer, zusammen 46.500.

\bibverse{12} Neben ihm soll sich lagern der Stamm Simeon; ihr Hauptmann
Selumiel, der Sohn Zuri-Saddais, \bibverse{13} und sein Heer, zusammen
59.300.

\bibverse{14} Dazu der Stamm Gad; ihr Hauptmann Eljasaph, der Sohn
Reguels, \bibverse{15} und sein Heer, zusammen 45.650.

\bibverse{16} Dass alle, die ins Lager Rubens gehören, seien zusammen
151.450, die zu ihrem Heer gehören; und sie sollen die zweiten im
Ausziehen sein.

\bibverse{17} Darnach soll die Hütte des Stifts ziehen mit dem Lager der
Leviten, mitten unter den Lagern; und wie sie sich lagern, so sollen sie
auch ziehen, ein jeglicher an seinem Ort unter seinem Panier.

\bibverse{18} Gegen Abend soll liegen das Gezelt und Panier Ephraims mit
ihrem Heer; ihr Hauptmann soll sein Elisama, der Sohn Ammihuds,
\bibverse{19} und sein Heer, zusammen 40.500.

\bibverse{20} Neben ihm soll sich lagern der Stamm Manasse; ihr
Hauptmann Gamliel, der Sohn Pedazurs, \bibverse{21} und sein Heer,
zusammen 32.200.

\bibverse{22} Dazu der Stamm Benjamin; ihr Hauptmann Abidan, der Sohn
des Gideoni, \bibverse{23} und sein Heer, zusammen 35.400.

\bibverse{24} Dass alle, die ins Lager Ephraims gehören, seien zusammen
108.100, die zu seinem Heer gehören; und sie sollen die dritten im
Ausziehen sein.

\bibverse{25} Gegen Mitternacht soll liegen das Gezelt und Panier Dans
mit ihrem Heer; ihr Hauptmann Ahieser, der Sohn Ammi-Saddais,
\bibverse{26} und sein Heer, zusammen 62.700.

\bibverse{27} Neben ihm soll sich lagern der Stamm Asser; ihr Hauptmann
Pagiel, der Sohn Ochrans, \bibverse{28} und sein Heer, zusammen 41.500.

\bibverse{29} Dazu der Stamm Naphthali; ihr Hauptmann Ahira, der Sohn
Enans, \bibverse{30} und sein Heer, zusammen 53.400.

\bibverse{31} Dass alle, die ins Lager Dans gehören, seien zusammen
157.600; und sie sollen die letzten sein im Ausziehen mit ihrem Panier.

\bibverse{32} Dies ist die Summe der Kinder Israel nach ihren
Vaterhäusern und Lagern mit ihren Heeren: 603.550. \bibverse{33} Aber
die Leviten wurden nicht in die Summe unter die Kinder Israel gezählt,
wie der HErr dem Mose geboten hatte. \footnote{\textbf{2:33} 4Mo 1,48-49}

\bibverse{34} Und die Kinder Israel taten alles, wie der HErr dem Mose
geboten hatte, und lagerten sich unter ihre Paniere und zogen aus, ein
jeglicher in seinem Geschlecht nach seinem Vaterhaus. \footnote{\textbf{2:34}
  4Mo 2,2}

\hypertarget{section-1}{%
\section{3}\label{section-1}}

\bibverse{1} Dies ist das Geschlecht Aarons und Moses zu der Zeit, da
der HErr mit Mose redete auf dem Berge Sinai. \footnote{\textbf{3:1} 2Mo
  6,23} \bibverse{2} Und dies sind die Namen der Söhne Aarons: der
Erstgeborene Nadab, darnach Abihu, Eleasar und Ithamar.

\bibverse{3} Das sind die Namen der Söhne Aarons, die zu Priestern
gesalbt waren und deren Hände gefüllt wurden zum Priestertum.
\bibverse{4} Aber Nadab und Abihu starben vor dem HErrn, da sie fremdes
Feuer opferten vor dem HErrn in der Wüste Sinai, und hatten keine Söhne.
Eleasar aber und Ithamar pflegten des Priesteramtes unter ihrem Vater
Aaron.

\bibverse{5} Und der HErr redete mit Mose und sprach: \bibverse{6}
Bringe den Stamm Levi herzu und stelle sie vor den Priester Aaron, dass
sie ihm dienen \footnote{\textbf{3:6} 2Mo 32,29} \bibverse{7} und seiner
und der ganzen Gemeinde Hut warten vor der Hütte des Stifts und dienen
am Dienst der Wohnung \bibverse{8} und warten alles Gerätes der Hütte
des Stifts und der Hut der Kinder Israel, zu dienen am Dienst der
Wohnung. \bibverse{9} Und sollst die Leviten Aaron und seinen Söhnen
zuordnen zum Geschenk von den Kindern Israel. \bibverse{10} Aaron aber
und seine Söhne sollst du setzen, dass sie ihres Priestertums warten. Wo
ein Fremder sich herzutut, der soll sterben. \footnote{\textbf{3:10} 4Mo
  1,51}

\bibverse{11} Und der HErr redete mit Mose und sprach: \bibverse{12}
Siehe, ich habe die Leviten genommen unter den Kindern Israel für alle
Erstgeburt, welche die Mutter bricht, unter den Kindern Israel, also
dass die Leviten sollen mein sein. \bibverse{13} Denn die Erstgeburten
sind mein seit der Zeit, da ich alle Erstgeburt schlug in Ägyptenland;
da heiligte ich mir alle Erstgeburt in Israel, vom Menschen an bis auf
das Vieh, dass sie mein sein sollen, ich, der HErr.

\bibverse{14} Und der HErr redete mit Mose in der Wüste Sinai und
sprach: \bibverse{15} Zähle die Kinder Levi nach ihren Vaterhäusern und
Geschlechtern, alles, was männlich ist, einen Monat alt und darüber.

\bibverse{16} Also zählte sie Mose nach dem Wort des HErrn, wie er
geboten hatte.

\bibverse{17} Und dies waren die Kinder Levis mit Namen: Gerson, Kahath,
Merari. \footnote{\textbf{3:17} 2Mo 6,16-19; 4Mo 26,57-64}

\bibverse{18} Die Namen aber der Kinder Gersons nach ihren Geschlechtern
waren: Libni und Simei.

\bibverse{19} Die Kinder Kahaths nach ihren Geschlechtern waren: Amram,
Jizhar, Hebron und Usiel.

\bibverse{20} Die Kinder Meraris nach ihren Geschlechtern waren; Maheli
und Musi. Dies sind die Geschlechter Levis nach ihren Vaterhäusern.

\bibverse{21} Dies sind die Geschlechter von Gerson: die Libniter und
Simeiter.

\bibverse{22} Deren Summe war an der Zahl gefunden 7500, alles, was
männlich war, einen Monat alt und darüber.

\bibverse{23} Und dieselben Geschlechter der Gersoniter sollen sich
lagern hinter der Wohnung gegen Abend.

\bibverse{24} Ihr Oberster sei Eljasaph, der Sohn Laels.

\bibverse{25} Und sie sollen an der Hütte des Stifts warten der Wohnung
und der Hütte und ihrer Decken und des Tuchs in der Tür der Hütte des
Stifts, \bibverse{26} des Umhangs am Vorhof und des Tuchs in der Tür des
Vorhofs, welcher um die Wohnung und um den Altar her geht, und ihre
Seile und alles dessen, was zu ihrem Dienst gehört.

\bibverse{27} Dies sind die Geschlechter von Kahath: die Amramiten, die
Jizhariten, die Hebroniten und Usieliten, \bibverse{28} was männlich
war, einen Monat alt und darüber, an der Zahl achttausendsechshundert,
die der Sorge für das Heiligtum warten.

\bibverse{29} und sie sollen sich lagern an die Seite der Wohnung gegen
Mittag. \bibverse{30} Ihr Oberster sei Elizaphan, der Sohn Usiels.
\bibverse{31} Und sie sollen warten der Lade, des Tisches, des
Leuchters, der Altäre und alles Gerätes des Heiligtums, daran sie dienen
und des Tuchs und was sonst zu ihrem Dienst gehört. \footnote{\textbf{3:31}
  4Mo 7,9} \bibverse{32} Aber der Oberste über alle Obersten der Leviten
soll Eleasar sein, Aarons Sohn, des Priesters, über die, die verordnet
sind, zu warten der Sorge für das Heiligtum.

\bibverse{33} Dies sind die Geschlechter Meraris: die Maheliter und
Musiter, \bibverse{34} die an der Zahl waren 6200, alles was männlich
war, einen Monat alt und darüber.

\bibverse{35} Ihr Oberster sei Zuriel, der Sohn Abihails. Und sollen
sich lagern an die Seite der Wohnung gegen Mitternacht. \bibverse{36}
Und ihr Amt soll sein, zu warten der Bretter und Riegel und Säulen und
Füße der Wohnung und alles ihres Gerätes und ihres Dienstes,
\bibverse{37} dazu der Säulen um den Vorhof her mit den Füßen und Nägeln
und Seilen.

\bibverse{38} Aber vor der Wohnung und vor der Hütte des Stifts gegen
Morgen sollen sich lagern Mose und Aaron und seine Söhne, dass sie des
Heiligtums warten für die Kinder Israel. Wenn sich ein Fremder herzutut,
der soll sterben. \bibverse{39} Alle Leviten zusammen, die Mose und
Aaron zählten nach ihren Geschlechtern nach dem Wort des HErrn, eitel
Mannsbilder einen Monat alt und darüber, waren 22.000.

\bibverse{40} Und der HErr sprach zu Mose: Zähle alle Erstgeburt, was
männlich ist unter den Kindern Israel, einen Monat alt und darüber, und
nimm die Zahl ihrer Namen. \bibverse{41} Und sollst die Leviten mir, dem
HErrn, aussondern für alle Erstgeburt der Kinder Israel und der Leviten
Vieh für alle Erstgeburt unter dem Vieh der Kinder Israel.

\bibverse{42} Und Mose zählte, wie ihm der HErr geboten hatte, alle
Erstgeburt unter den Kindern Israel; \bibverse{43} und fand sich die
Zahl der Namen aller Erstgeburt, was männlich war, einen Monat alt und
darüber, in ihrer Summe 22.273.

\bibverse{44} Und der HErr redete mit Mose und sprach: \bibverse{45}
Nimm die Leviten für alle Erstgeburt unter den Kindern Israel und das
Vieh der Leviten für ihr Vieh, dass die Leviten mein, des HErrn, seien.
\footnote{\textbf{3:45} 4Mo 3,12} \bibverse{46} Aber als Lösegeld von
den 273 Erstgeburten der Kinder Israel, die über der Leviten Zahl sind,
\footnote{\textbf{3:46} 4Mo 3,39; 4Mo 3,43} \bibverse{47} sollst du je
fünf Silberlinge nehmen von Haupt zu Haupt nach dem Lot des Heiligtums
(20 Gera hat ein Lot) \bibverse{48} und sollst das Geld für die, die
überzählig sind unter ihnen, geben Aaron und seinen Söhnen.

\bibverse{49} Da nahm Mose das Lösegeld von denen, die über der Leviten
Zahl waren, \bibverse{50} von den Erstgeburten der Kinder Israel, 1365
Silberlinge nach dem Lot des Heiligtums, \bibverse{51} und gab's Aaron
und seinen Söhnen nach dem Wort des HErrn, wie der HErr dem Mose geboten
hatte. \# 4 \bibverse{1} Und der HErr redete mit Mose und Aaron und
sprach: \bibverse{2} Nimm die Summe der Kinder Kahath aus den Kindern
Levi nach ihren Geschlechtern und Vaterhäusern, \bibverse{3} von 30
Jahren an und darüber bis ins 50. Jahr, alle, die zum Dienst taugen,
dass sie tun die Werke in der Hütte des Stifts. \footnote{\textbf{4:3}
  4Mo 8,24}

\bibverse{4} Das soll aber das Amt der Kinder Kahath in der Hütte des
Stifts sein: was das Hochheilige ist. \bibverse{5} Wenn das Heer
aufbricht, so sollen Aaron und seine Söhne hineingehen und den Vorhang
abnehmen und die Lade des Zeugnisses darein winden \bibverse{6} und
darauf tun die Decke von Dachsfellen und obendrauf eine ganz blaue Decke
breiten und ihre Stangen daran legen

\bibverse{7} und über den Schaubrottisch auch eine blaue Decke breiten
und darauf legen die Schüsseln, Löffel, die Schalen und Kannen zum
Trankopfer, und das beständige Brot soll darauf liegen. \bibverse{8} Und
sollen darüber breiten eine scharlachrote Decke und dieselbe bedecken
mit einer Decke von Dachsfellen und seine Stangen daran legen.

\bibverse{9} Und sollen eine blaue Decke nehmen und darein winden den
Leuchter des Lichts und seine Lampen mit seinen Schneuzen und Näpfen und
alle Ölgefäße, die zum Amt gehören. \bibverse{10} Und sollen um das
alles tun eine Decke von Dachsfellen und sollen es auf Stangen legen.

\bibverse{11} Also sollen sie auch über den goldenen Altar eine blaue
Decke breiten und sie bedecken mit der Decke von Dachsfellen und seine
Stangen daran tun.

\bibverse{12} Alle Gerät, womit sie schaffen im Heiligtum, sollen sie
nehmen und blaue Decken darüber tun und mit einer Decke von Dachsfellen
bedecken und auf Stangen legen.

\bibverse{13} Sie sollen auch die Asche vom Altar fegen und eine Decke
von rotem Purpur über ihn breiten \bibverse{14} und alle seine Geräte
darauf tun, womit sie darauf schaffen, Kohlenpfannen, Gabeln, Schaufeln,
Becken mit allem Geräte des Altars; und sollen darüber breiten eine
Decke von Dachsfellen und seine Stangen daran tun.

\bibverse{15} Wenn nun Aaron und seine Söhne solches ausgerichtet und
das Heiligtum und all sein Gerät bedeckt haben, wenn das Heer aufbricht,
darnach sollen die Kinder Kahath hineingehen, dass sie es tragen; und
sollen das Heiligtum nicht anrühren, dass sie nicht sterben. Dies sind
die Lasten der Kinder Kahath an der Hütte des Stifts. \footnote{\textbf{4:15}
  4Mo 7,9; 2Sam 6,6-7}

\bibverse{16} Und Eleasar, Aarons, des Priesters, Sohn, soll das Amt
haben, dass er ordne das Öl zum Licht und die Spezerei zum Räuchwerk und
das tägliche Speisopfer und das Salböl, dass er beschicke die ganze
Wohnung und alles, was darin ist, im Heiligtum und seinem Geräte.

\bibverse{17} Und der HErr redete mit Mose und Aaron und sprach:
\bibverse{18} Ihr sollt den Stamm der Geschlechter der Kahathiter nicht
lassen sich verderben unter den Leviten; \bibverse{19} sondern das sollt
ihr mit ihnen tun, dass sie leben und nicht sterben, wo sie würden
anrühren das Hochheilige: Aaron und seine Söhne sollen hineingehen und
einen jeglichen stellen zu seinem Amt und seiner Last. \bibverse{20} Sie
sollen aber nicht hineingehen, zu schauen das Heiligtum auch nur einen
Augenblick, dass sie nicht sterben.

\bibverse{21} Und der HErr redete mit Mose und sprach: \bibverse{22}
Nimm die Summe der Kinder Gerson auch nach ihren Vaterhäusern und
Geschlechtern, \bibverse{23} von 30 Jahren an und darüber bis ins 50.
Jahr, und ordne sie alle, die da zum Dienst tüchtig sind, dass sie ein
Amt haben in der Hütte des Stifts.

\bibverse{24} Das soll aber der Geschlechter der Gersoniter Amt sein,
dass sie schaffen und tragen: \bibverse{25} sie sollen die Teppiche der
Wohnung und der Hütte des Stifts tragen und ihre Decke und die Decke von
Dachsfellen, die obendrüber ist, und das Tuch in der Hütte des Stifts
\bibverse{26} und die Umhänge des Vorhofs und das Tuch in der Tür des
Tors am Vorhof, welcher um die Wohnung und den Altar her geht, und ihre
Seile und alle Geräte ihres Amts und alles, was zu ihrem Amt gehört.
\bibverse{27} Nach dem Wort Aarons und seiner Söhne soll alles Amt der
Kinder Gerson geschehen, alles, was sie tragen und schaffen sollen, und
ihr sollt zusehen, dass sie aller ihrer Last warten. \bibverse{28} Das
soll das Amt der Geschlechter der Kinder der Gersoniter sein in der
Hütte des Stifts; und ihr Dienst soll unter der Hand Ithamars sein, des
Sohnes Aarons, des Priesters.

\bibverse{29} Die Kinder Merari nach ihren Geschlechtern und
Vaterhäusern sollst du auch ordnen, \bibverse{30} von 30 Jahren an und
darüber bis ins 50. Jahr, alle, die zum Dienst taugen, dass sie ein Amt
haben in der Hütte des Stifts. \bibverse{31} Dieser Last aber sollen sie
warten nach allem ihrem Amt in der Hütte des Stifts, dass sie tragen die
Bretter der Wohnung und Riegel und Säulen und Füße, \bibverse{32} dazu
die Säulen des Vorhofs umher und Füße und Nägel und Seile mit allem
ihrem Geräte, nach allem ihrem Amt; einem jeglichen sollt ihr seinen
Teil der Last am Geräte zu warten verordnen. \bibverse{33} Das sei das
Amt der Geschlechter der Kinder Merari, alles, was sie schaffen sollen
in der Hütte des Stifts unter der Hand Ithamars, des Priesters, des
Sohnes Aarons.

\bibverse{34} Und Mose und Aaron samt den Hauptleuten der Gemeinde
zählten die Kinder der Kahathiter nach ihren Geschlechtern und
Vaterhäusern, \bibverse{35} von 30 Jahren und darüber bis ins 50., alle,
die zum Dienst taugten, dass sie Amt in der Hütte des Stifts hätten.
\bibverse{36} Und die Summe war 2750. \bibverse{37} Das ist die Summe
der Geschlechter der Kahathiter, die alle zu schaffen hatten in der
Hütte des Stifts, die Mose und Aaron zählten nach dem Wort des HErrn
durch Mose.

\bibverse{38} Die Kinder Gerson wurden auch gezählt in ihren
Geschlechtern und Vaterhäusern, \bibverse{39} von 30 Jahren und darüber
bis ins 50., alle, die zum Dienst taugten, dass sie Amt in der Hütte des
Stifts hätten. \bibverse{40} Und die Summe war 2630. \bibverse{41} Das
ist die Summe der Geschlechter der Kinder Gerson, die alle zu schaffen
hatten in der Hütte des Stifts, welche Mose und Aaron zählten nach dem
Wort des HErrn.

\bibverse{42} Die Kinder Merari wurden auch gezählt nach ihren
Geschlechtern und Vaterhäusern, \bibverse{43} von 30 Jahren und darüber
bis ins 50., alle, die zum Dienst taugten, dass sie Amt in der Hütte des
Stifts hätten. \bibverse{44} Und die Summe war 3200. \bibverse{45} Das
ist die Summe der Geschlechter der Kinder Merari, die Mose und Aaron
zählten nach dem Wort des HErrn durch Mose.

\bibverse{46} Die Summe aller Leviten, die Mose und Aaron samt den
Hauptleuten Israels zählten nach ihren Geschlechtern und Vaterhäusern,
\bibverse{47} von 30 Jahren und darüber bis ins 50., aller, die
eingingen, zu schaffen ein jeglicher sein Amt und zu tragen die Last in
der Hütte des Stifts, \bibverse{48} war 8580, \bibverse{49} die gezählt
wurden nach dem Wort des HErrn durch Mose, ein jeglicher zu seinem Amt
und seiner Last, wie der HErr dem Mose geboten hatte. \# 5 \bibverse{1}
Und der HErr redete mit Mose und sprach: \bibverse{2} Gebiete den
Kindern Israel, dass sie aus dem Lager tun alle Aussätzigen und alle,
die Eiterflüsse haben und die an Toten unrein geworden sind. \footnote{\textbf{5:2}
  3Mo 13,46; 3Mo 15,2} \bibverse{3} Beide, Mann und Weib, sollt ihr
hinaustun vor das Lager, dass sie nicht ihr Lager verunreinigen, darin
ich unter ihnen wohne. \footnote{\textbf{5:3} 4Mo 12,14; 4Mo 35,34}

\bibverse{4} Und die Kinder Israel taten also und taten sie hinaus vor
das Lager, wie der HErr zu Mose geredet hatte.

\bibverse{5} Und der HErr redete mit Mose und sprach: \bibverse{6} Sage
den Kindern Israel und sprich zu ihnen: Wenn ein Mann oder Weib
irgendeine Sünde wider einen Menschen tut und sich an dem HErrn damit
versündigt, so hat die Seele eine Schuld auf sich; \footnote{\textbf{5:6}
  3Mo 5,21-26} \bibverse{7} und sie sollen ihre Sünde bekennen, die sie
getan haben, und sollen ihre Schuld versöhnen mit der Hauptsumme und
darüber den fünften Teil dazutun und dem geben, an dem sie sich
verschuldigt haben. \bibverse{8} Ist aber niemand da, dem man's bezahlen
sollte, so soll man's dem HErrn geben für den Priester außer dem Widder
der Versöhnung, dadurch er versöhnt wird. \bibverse{9} Desgleichen soll
alle Hebe von allem, was die Kinder Israel heiligen und dem Priester
opfern, sein sein. \bibverse{10} Und wer etwas heiligt, das soll auch
sein sein; und wer etwas dem Priester gibt, das soll auch sein sein.

\bibverse{11} Und der HErr redete mit Mose und sprach: \bibverse{12}
Sage den Kindern Israel und sprich zu ihnen: Wenn irgendeines Mannes
Weib untreu würde und sich an ihm versündigte \bibverse{13} und jemand
bei ihr liegt, und es würde doch dem Manne verborgen vor seinen Augen
und würde verdeckt, dass sie unrein geworden ist, und er kann sie nicht
überführen, denn sie ist nicht dabei ergriffen, \bibverse{14} und der
Eifergeist entzündet ihn, dass er um sein Weib eifert, sie sei unrein
oder nicht unrein, \bibverse{15} so soll er sie zum Priester bringen und
ein Opfer über sie bringen, ein zehntel Epha Gerstenmehl, und soll kein
Öl darauf gießen noch Weihrauch darauf tun. Denn es ist ein Eiferopfer
und Rügeopfer, das Missetat rügt. \bibverse{16} Da soll sie der Priester
herzuführen und vor den HErrn stellen \bibverse{17} und heiliges Wasser
nehmen in ein irdenes Gefäß und Staub vom Boden der Wohnung ins Wasser
tun. \footnote{\textbf{5:17} 2Mo 30,18} \bibverse{18} Und soll das Weib
vor den HErrn stellen und ihr Haupt entblößen und das Rügeopfer, das ein
Eiferopfer ist, auf ihre Hand legen; und der Priester soll in seiner
Hand bitteres verfluchtes Wasser haben \bibverse{19} und soll das Weib
beschwören und zu ihr sagen: Hat kein Mann bei dir gelegen, und bist du
deinem Mann nicht untreu geworden, dass du dich verunreinigt hast, so
sollen dir diese bitteren verfluchten Wasser nicht schaden.
\bibverse{20} Wo du aber deinem Mann untreu geworden bist, dass du
unrein wurdest, und hat jemand bei dir gelegen außer deinem Mann,
\bibverse{21} so soll der Priester das Weib beschwören mit solchem Fluch
und soll zu ihr sagen: Der HErr setze dich zum Fluch und zum Schwur
unter deinem Volk, dass der HErr deine Hüfte schwinden und deinen Bauch
schwellen lasse! \bibverse{22} So gehe nun das verfluchte Wasser in
deinen Leib, dass dein Bauch schwelle und deine Hüfte schwinde! Und das
Weib soll sagen: Amen, amen.

\bibverse{23} Also soll der Priester diese Flüche auf einen Zettel
schreiben und mit dem bitteren Wasser abwaschen \bibverse{24} und soll
dem Weibe von dem bitteren verfluchten Wasser zu trinken geben, dass das
verfluchte bittere Wasser in sie gehe. \bibverse{25} Es soll aber der
Priester von ihrer Hand das Eiferopfer nehmen und zum Speisopfer vor dem
HErrn weben und auf dem Altar opfern, nämlich: \bibverse{26} er soll
eine Handvoll des Speisopfers nehmen und auf dem Altar anzünden zum
Gedächtnis und darnach dem Weibe das Wasser zu trinken geben.
\bibverse{27} Und wenn sie das Wasser getrunken hat: ist sie unrein und
hat sich an ihrem Mann versündigt, so wird das verfluchte Wasser in sie
gehen und ihr bitter sein, dass ihr der Bauch schwellen und die Hüfte
schwinden wird, und wird das Weib ein Fluch sein unter ihrem Volk;
\bibverse{28} ist aber ein solch Weib nicht verunreinigt, sondern rein,
so wird's ihr nicht schaden, dass sie kann schwanger werden.

\bibverse{29} Dies ist das Eifergesetz, wenn ein Weib ihrem Mann untreu
ist und unrein wird, \bibverse{30} oder wenn einen Mann der Eifergeist
entzündet, dass er um sein Weib eifert, dass er's stelle vor den HErrn
und der Priester mit ihr tue alles nach diesem Gesetz. \bibverse{31} Und
der Mann soll unschuldig sein an der Missetat; aber das Weib soll ihre
Missetat tragen. \# 6 \bibverse{1} Und der HErr redete mit Mose und
sprach: \bibverse{2} Sage den Kindern Israel und sprich zu ihnen: Wenn
ein Mann oder Weib ein besonderes Gelübde tut, dem HErrn sich zu
enthalten, \bibverse{3} der soll sich Weins und starken Getränks
enthalten; Weinessig oder Essig von starkem Getränk soll er auch nicht
trinken, auch nichts, das aus Weinbeeren gemacht wird; er soll weder
frische noch dürre Weinbeeren essen. \footnote{\textbf{6:3} Lk 1,15}
\bibverse{4} Solange solch ein Gelübde währt, soll er nichts essen, das
man vom Weinstock macht, vom Weinkern bis zu den Hülsen.

\bibverse{5} Solange die Zeit solches seines Gelübdes währt, soll kein
Schermesser über sein Haupt fahren, bis dass die Zeit aus sei, die er
dem HErrn gelobt hat; denn er ist heilig und soll das Haar auf seinem
Haupt lassen frei wachsen.

\bibverse{6} Die ganze Zeit über, die er dem HErrn gelobt hat, soll er
zu keinem Toten gehen. \bibverse{7} Er soll sich auch nicht
verunreinigen an dem Tod seines Vaters, seiner Mutter, seines Bruders
oder seiner Schwester; denn das Gelübde seines Gottes ist auf seinem
Haupt. \footnote{\textbf{6:7} 3Mo 21,11} \bibverse{8} Die ganze Zeit
seines Gelübdes soll er dem HErrn heilig sein.

\bibverse{9} Und wo jemand vor ihm unversehens plötzlich stirbt, da wird
das Haupt seines Gelübdes verunreinigt; darum soll er sein Haupt scheren
am Tage seiner Reinigung, das ist am siebenten Tage. \bibverse{10} Und
am achten Tage soll er zwei Turteltauben bringen oder zwei junge Tauben
zum Priester vor die Tür der Hütte des Stifts. \footnote{\textbf{6:10}
  3Mo 5,7} \bibverse{11} Und der Priester soll eine zum Sündopfer und
die andere zum Brandopfer machen und ihn versöhnen, darum dass er sich
an einem Toten versündigt hat, und also sein Haupt desselben Tages
heiligen, \bibverse{12} dass er dem HErrn die Zeit seines Gelübdes
aushalte. Und soll ein jähriges Lamm bringen zum Schuldopfer. Aber die
vorigen Tage sollen umsonst sein, darum dass sein Gelübde verunreinigt
ist.

\bibverse{13} Dies ist das Gesetz des Gottgeweihten: wenn die Zeit
seines Gelübdes aus ist, so soll man ihn bringen vor die Tür der Hütte
des Stifts. \bibverse{14} Und er soll bringen sein Opfer dem HErrn, ein
jähriges Lamm ohne Fehl zum Brandopfer und ein jähriges Schaf ohne Fehl
zum Sündopfer und einen Widder ohne Fehl zum Dankopfer \bibverse{15} und
einen Korb mit ungesäuerten Kuchen von Semmelmehl, mit Öl gemengt, und
ungesäuerte Fladen, mit Öl bestrichen, und ihre Speisopfer und
Trankopfer. \bibverse{16} Und der Priester soll's vor den HErrn bringen
und soll sein Sündopfer und sein Brandopfer machen. \bibverse{17} Und
den Widder soll er zum Dankopfer machen dem HErrn samt dem Korbe mit dem
ungesäuerten Brot; und soll auch sein Speisopfer und sein Trankopfer
machen. \bibverse{18} Und der Geweihte soll das Haupt seines Gelübdes
scheren vor der Tür der Hütte des Stifts und soll das Haupthaar seines
Gelübdes nehmen und aufs Feuer werfen, das unter dem Dankopfer ist.
\bibverse{19} Und der Priester soll den gekochten Bug nehmen von dem
Widder und einen ungesäuerten Kuchen aus dem Korbe und einen
ungesäuerten Fladen und soll's dem Geweihten auf seine Hände legen,
nachdem er sein Gelübde abgeschoren hat, \bibverse{20} und der Priester
soll's vor dem HErrn weben. Das ist heilig dem Priester samt der
Webebrust und der Hebeschulter. Darnach mag der Geweihte Wein trinken.
\footnote{\textbf{6:20} 3Mo 7,29-34}

\bibverse{21} Das ist das Gesetz des Gottgeweihten, der sein Opfer dem
HErrn gelobt wegen seines Gelübdes, außer dem, was er sonst vermag; wie
er gelobt hat, soll er tun nach dem Gesetz seines Gelübdes.

\bibverse{22} Und der HErr redete mit Mose und sprach: \bibverse{23}
Sage Aaron und seinen Söhnen und sprich: Also sollt ihr sagen zu den
Kindern Israel, wenn ihr sie segnet: \bibverse{24} Der HErr segne dich
und behüte dich; \footnote{\textbf{6:24} Ps 121,-1} \bibverse{25} der
HErr lasse sein Angesicht leuchten über dir und sei dir gnädig;
\footnote{\textbf{6:25} Ps 80,4} \bibverse{26} der HErr hebe sein
Angesicht über dich und gebe dir Frieden. \footnote{\textbf{6:26} Ps
  69,17-18}

\bibverse{27} Denn ihr sollt meinen Namen auf die Kinder Israel legen,
dass ich sie segne. \# 7 \bibverse{1} Und da Mose die Wohnung
aufgerichtet hatte und sie gesalbt und geheiligt mit allem ihrem Geräte,
dazu auch den Altar mit allem seinem Geräte gesalbt und geheiligt,
\bibverse{2} da opferten die Fürsten Israels, die Häupter waren in ihren
Vaterhäusern; denn sie waren die Obersten unter den Stämmen und standen
obenan unter denen, die gezählt waren. \bibverse{3} Und sie brachten
ihre Opfer vor den HErrn, sechs bedeckte Wagen und zwölf Rinder, je
einen Wagen für zwei Fürsten und einen Ochsen für einen, und brachten
sie vor die Wohnung. \bibverse{4} Und der HErr sprach zu Mose:
\bibverse{5} Nimm's von ihnen, dass es diene zum Dienst der Hütte des
Stifts, und gib's den Leviten, einem jeglichen nach seinem Amt.

\bibverse{6} Da nahm Mose die Wagen und Rinder und gab sie den Leviten.
\bibverse{7} Zwei Wagen und vier Rinder gab er den Kindern Gerson nach
ihrem Amt; \bibverse{8} und vier Wagen und acht Ochsen gab er den
Kindern Merari nach ihrem Amt unter der Hand Ithamars, des Sohnes
Aarons, des Priesters; \footnote{\textbf{7:8} 2Mo 38,21; 4Mo 4,28; 4Mo
  4,33} \bibverse{9} den Kindern Kahath aber gab er nichts, darum dass
sie ein heiliges Amt auf sich hatten und auf ihren Achseln tragen
mussten.

\bibverse{10} Und die Fürsten opferten zur Einweihung das Altars an dem
Tage, da er gesalbt ward, und opferten ihre Gabe vor dem Altar.

\bibverse{11} Und der HErr sprach zu Mose: Lass einen jeglichen Fürsten
an seinem Tage sein Opfer bringen zur Einweihung des Altars. \footnote{\textbf{7:11}
  4Mo 1,4-16; 4Mo 2,3-29}

\bibverse{12} Am ersten Tage opferte seine Gabe Nahesson, der Sohn
Amminadabs, des Stammes Juda. \bibverse{13} Und seine Gabe war eine
silberne Schüssel, 130 Lot schwer, eine silberne Schale 70 Lot schwer
nach dem Lot des Heiligtums, beide voll Semmelmehl, mit Öl gemengt, zum
Speisopfer;

\bibverse{14} dazu einen goldenen Löffel, zehn Lot schwer, voll
Räuchwerk,

\bibverse{15} einen jungen Farren, einen Widder, ein jähriges Lamm zum
Brandopfer;

\bibverse{16} einen Ziegenbock zum Sündopfer;

\bibverse{17} und zum Dankopfer zwei Rinder, fünf Widder, fünf Böcke und
fünf jährige Lämmer. Das ist die Gabe Nahessons, des Sohnes Amminadabs.

\bibverse{18} Am zweiten Tage opferte Nathanael, der Sohn Zuars, der
Fürst Isaschars.

\bibverse{19} Seine Gabe war eine silberne Schüssel, 130 Lot schwer,
eine silberne Schale, 70 Lot schwer nach dem Lot des Heiligtums, beide
voll Semmelmehl, mit Öl gemengt, zum Speisopfer;

\bibverse{20} dazu einen goldenen Löffel, zehn Lot schwer, voll
Räuchwerk,

\bibverse{21} einen jungen Farren, einen Widder, ein jähriges Lamm zum
Brandopfer;

\bibverse{22} einen Ziegenbock zum Sündopfer;

\bibverse{23} und zum Dankopfer zwei Rinder, fünf Widder, fünf Böcke und
fünf jährige Lämmer. Das ist die Gabe Nathanaels, des Sohnes Zuars.

\bibverse{24} Am dritten Tage der Fürst der Kinder Sebulon, Eliab, der
Sohn Helons.

\bibverse{25} Seine Gabe war eine silberne Schüssel, 130 Lot schwer,
eine silberne Schale, 70 Lot schwer nach dem Lot des Heiligtums, beide
voll Semmelmehl, mit Öl gemengt, zum Speisopfer;

\bibverse{26} dazu einen goldenen Löffel, zehn Lot schwer, voll
Räuchwerk,

\bibverse{27} einen jungen Farren, einen Widder, ein jähriges Lamm zum
Brandopfer;

\bibverse{28} einen Ziegenbock zum Sündopfer;

\bibverse{29} und zum Dankopfer zwei Rinder, fünf Widder, fünf Böcke und
fünf jährige Lämmer. Das ist die Gabe Eliabs, des Sohnes Helons.

\bibverse{30} Am vierten Tage der Fürst der Kinder Ruben, Elizur, der
Sohn Sedeurs.

\bibverse{31} Seine Gabe war eine silberne Schüssel, 130 Lot schwer,
eine silberne Schale, 70 Lot schwer nach dem Lot des Heiligtums, beide
voll Semmelmehl, mit Öl gemengt, zum Speisopfer;

\bibverse{32} dazu einen goldenen Löffel, zehn Lot schwer, voll
Räuchwerk,

\bibverse{33} einen jungen Farren, einen Widder, ein jähriges Lamm zum
Brandopfer;

\bibverse{34} einen Ziegenbock zum Sündopfer;

\bibverse{35} und zum Dankopfer zwei Rinder, fünf Widder, fünf Böcke und
fünf jährige Lämmer. Das ist die Gabe Elizurs, des Sohnes Sedeurs.

\bibverse{36} Am fünften Tage der Fürst der Kinder Simeon, Selumiel, der
Sohn Zuri-Saddais.

\bibverse{37} Seine Gabe war eine silberne Schüssel, 130 Lot schwer,
eine silberne Schale, 70 Lot schwer nach dem Lot des Heiligtums, beide
voll Semmelmehl, mit Öl gemengt, zum Speisopfer;

\bibverse{38} dazu einen goldenen Löffel, zehn Lot schwer, voll
Räuchwerk,

\bibverse{39} einen jungen Farren, einen Widder, ein jähriges Lamm zum
Brandopfer;

\bibverse{40} einen Ziegenbock zum Sündopfer;

\bibverse{41} und zum Dankopfer zwei Rinder, fünf Widder, fünf Böcke und
fünf jährige Lämmer. Das ist die Gabe Selumiels, des Sohnes
Zuri-Saddais.

\bibverse{42} Am sechsten Tage der Fürst der Kinder Gad, Eljasaph, der
Sohn Deguels.

\bibverse{43} Seine Gabe war eine silberne Schüssel, 130 Lot schwer,
eine silberne Schale, 70 Lot schwer nach dem Lot des Heiligtums, beide
voll Semmelmehl, mit Öl gemengt, zum Speisopfer;

\bibverse{44} dazu einen goldenen Löffel, zehn Lot schwer, voll
Räuchwerk,

\bibverse{45} einen jungen Farren, einen Widder, ein jähriges Lamm zum
Brandopfer;

\bibverse{46} einen Ziegenbock zum Sündopfer;

\bibverse{47} und zum Dankopfer zwei Rinder, fünf Widder, fünf Böcke und
fünf jährige Lämmer. Das ist die Gabe Eljasaphs, des Sohnes Deguels.

\bibverse{48} Am siebenten Tage der Fürst der Kinder Ephraim, Elisama,
der Sohn Ammihuds.

\bibverse{49} Seine Gabe war eine silberne Schüssel, 130 Lot schwer,
eine silberne Schale, 70 Lot schwer nach dem Lot des Heiligtums, beide
voll Semmelmehl, mit Öl gemengt, zum Speisopfer;

\bibverse{50} dazu einen goldenen Löffel, zehn Lot schwer, voll
Räuchwerk,

\bibverse{51} einen jungen Farren, einen Widder, ein jähriges Lamm zum
Brandopfer;

\bibverse{52} einen Ziegenbock zum Sündopfer;

\bibverse{53} und zum Dankopfer zwei Rinder, fünf Widder, fünf Böcke und
fünf jährige Lämmer. Das ist die Gabe Elisamas, des Sohnes Ammihuds.

\bibverse{54} Am achten Tage der Fürst der Kinder Manasse, Gamliel, der
Sohn Pedazurs.

\bibverse{55} Seine Gabe war eine silberne Schüssel, 130 Lot schwer,
eine silberne Schale, 70 Lot schwer nach dem Lot des Heiligtums, beide
voll Semmelmehl, mit Öl gemengt, zum Speisopfer;

\bibverse{56} dazu einen goldenen Löffel, zehn Lot schwer, voll
Räuchwerk,

\bibverse{57} einen jungen Farren, einen Widder, ein jähriges Lamm zum
Brandopfer;

\bibverse{58} einen Ziegenbock zum Sündopfer;

\bibverse{59} und zum Dankopfer zwei Rinder, fünf Widder, fünf Böcke und
fünf jährige Lämmer. Das ist die Gabe Gamliels, des Sohnes Pedazurs.

\bibverse{60} Am neunten Tage der Fürst der Kinder Benjamin, Abidan, der
Sohn des Gideoni.

\bibverse{61} Seine Gabe war eine silberne Schüssel, 130 Lot schwer,
eine silberne Schale, 70 Lot schwer nach dem Lot des Heiligtums, beide
voll Semmelmehl, mit Öl gemengt, zum Speisopfer;

\bibverse{62} dazu einen goldenen Löffel, zehn Lot schwer, voll
Räuchwerk,

\bibverse{63} einen jungen Farren, einen Widder, ein jähriges Lamm zum
Brandopfer;

\bibverse{64} einen Ziegenbock zum Sündopfer;

\bibverse{65} und zum Dankopfer zwei Rinder, fünf Widder, fünf Böcke und
fünf jährige Lämmer. Das ist die Gabe Abidans, des Sohnes Gideonis.

\bibverse{66} Am zehnten Tage der Fürst der Kinder Dan, Ahi-Eser, der
Sohn Ammi-Saddais.

\bibverse{67} Seine Gabe war eine silberne Schüssel, 130 Lot schwer,
eine silberne Schale, 70 Lot schwer nach dem Lot des Heiligtums, beide
voll Semmelmehl, mit Öl gemengt, zum Speisopfer;

\bibverse{68} dazu einen goldenen Löffel, zehn Lot schwer, voll
Räuchwerk,

\bibverse{69} einen jungen Farren, einen Widder, ein jähriges Lamm zum
Brandopfer;

\bibverse{70} einen Ziegenbock zum Sündopfer;

\bibverse{71} und zum Dankopfer zwei Rinder, fünf Widder, fünf Böcke und
fünf jährige Lämmer. Das ist die Gabe Ahi-Esers, des Sohnes
Ammi-Saddais.

\bibverse{72} Am elften Tage der Fürst der Kinder Asser, Pagiel, der
Sohn Ochrans.

\bibverse{73} Seine Gabe war eine silberne Schüssel, 130 Lot schwer,
eine silberne Schale, 70 Lot schwer nach dem Lot des Heiligtums, beide
voll Semmelmehl, mit Öl gemengt, zum Speisopfer;

\bibverse{74} dazu einen goldenen Löffel, zehn Lot schwer, voll
Räuchwerk,

\bibverse{75} einen jungen Farren, einen Widder, ein jähriges Lamm zum
Brandopfer;

\bibverse{76} einen Ziegenbock zum Sündopfer;

\bibverse{77} und zum Dankopfer zwei Rinder, fünf Widder, fünf Böcke und
fünf jährige Lämmer. Das ist die Gabe Pagiels, des Sohnes Ochrans.

\bibverse{78} Am zwölften Tage der Fürst der Kinder Naphthali, Ahira,
der Sohn Enans.

\bibverse{79} Seine Gabe war eine silberne Schüssel, 130 Lot schwer,
eine silberne Schale, 70 Lot schwer nach dem Lot des Heiligtums, beide
voll Semmelmehl, mit Öl gemengt, zum Speisopfer;

\bibverse{80} dazu einen goldenen Löffel, zehn Lot schwer, voll
Räuchwerk,

\bibverse{81} einen jungen Farren, einen Widder, ein jähriges Lamm zum
Brandopfer;

\bibverse{82} einen Ziegenbock zum Sündopfer;

\bibverse{83} und zum Dankopfer zwei Rinder, fünf Widder, fünf Böcke und
fünf jährige Lämmer. Das ist die Gabe Ahiras, des Sohnes Enans.

\bibverse{84} Das ist die Einweihung des Altars zur Zeit, da er gesalbt
ward, dazu die Fürsten Israels opferten diese zwölf silbernen Schüsseln,
zwölf silbernen Schalen, zwölf goldenen Löffel,

\bibverse{85} also dass je eine Schüssel 130 Lot Silber und je eine
Schale 70 Lot hatte, dass die Summe alles Silbers am Gefäß betrug 2400
Lot nach dem Lot des Heiligtums.

\bibverse{86} Und der zwölf goldenen Löffel voll Räuchwerk hatte je
einer zehn Lot nach dem Lot des Heiligtums, dass die Summe Goldes an den
Löffeln betrug 120 Lot.

\bibverse{87} Die Summe der Rinder zum Brandopfer waren zwölf Farren,
zwölf Widder, zwölf jährige Lämmer samt ihren Speisopfern und zwölf
Ziegenböcke zum Sündopfer.

\bibverse{88} Und die Summe der Rinder zum Dankopfer war 24 Farren, 60
Widder, 60 Böcke, 60 jährige Lämmer. Das ist die Einweihung des Altars,
da er gesalbt ward.

\bibverse{89} Und wenn Mose in die Hütte des Stifts ging, dass mit ihm
geredet würde, so hörte er die Stimme mit ihm reden von dem Gnadenstuhl,
der auf der Lade des Zeugnisses war, zwischen den zwei Cherubim; dort
ward mit ihm geredet. \# 8 \bibverse{1} Und der HErr redete mit Mose und
sprach: \bibverse{2} Rede mit Aaron und sprich zu ihm: Wenn du die
Lampen aufsetzest, sollst du sie also setzen, dass sie alle sieben
vorwärts von dem Leuchter scheinen. \footnote{\textbf{8:2} 2Mo 25,31-40}
\bibverse{3} Und Aaron tat also und setzte die Lampen auf, vorwärts von
dem Leuchter zu scheinen, wie der HErr dem Mose geboten hatte.
\bibverse{4} Der Leuchter aber war getriebenes Gold, beide, sein Schaft
und seine Blumen; nach dem Gesicht, das der HErr dem Mose gezeigt hatte,
also machte er den Leuchter. \bibverse{5} Und der HErr redete mit Mose
und sprach: \bibverse{6} Nimm die Leviten aus den Kindern Israel und
reinige sie. \bibverse{7} Also sollst du aber mit ihnen tun, dass du sie
reinigst: du sollst Sündwasser auf sie sprengen, und sie sollen alle
ihre Haare rein abscheren und ihre Kleider waschen, so sind sie rein.
\footnote{\textbf{8:7} 4Mo 5,17; 4Mo 19,9; 4Mo 19,17; 3Mo 14,8}
\bibverse{8} Dann sollen sie nehmen einen jungen Farren und sein
Speisopfer, Semmelmehl, mit Öl gemengt; und einen anderen jungen Farren
sollst du zum Sündopfer nehmen. \bibverse{9} Und sollst die Leviten vor
die Hütte des Stifts bringen und die ganze Gemeinde der Kinder Israel
versammeln \bibverse{10} und die Leviten vor den HErrn bringen; und die
Kinder Israel sollen ihre Hände auf die Leviten legen, \bibverse{11} und
Aaron soll die Leviten vor dem HErrn weben als Webeopfer von den Kindern
Israel, auf dass sie dienen mögen in dem Amt des HErrn. \bibverse{12}
Und die Leviten sollen ihre Hände aufs Haupt der Farren legen, und einer
soll zum Sündopfer, der andere zum Brandopfer dem HErrn gemacht werden,
die Leviten zu versöhnen. \bibverse{13} Und sollst die Leviten vor Aaron
und seine Söhne stellen und vor dem HErrn weben, \bibverse{14} und
sollst sie also aussondern von den Kindern Israel, dass sie mein seien.
\footnote{\textbf{8:14} 4Mo 3,45} \bibverse{15} Darnach sollen sie
hineingehen, dass sie dienen in der Hütte des Stifts. Also sollst du sie
reinigen und weben; \bibverse{16} denn sie sind mein Geschenk von den
Kindern Israel, und ich habe sie mir genommen für alles, was die Mutter
bricht, nämlich für die Erstgeburt aller Kinder Israel. \bibverse{17}
Denn alle Erstgeburt unter den Kindern Israel ist mein, der Menschen und
des Viehes, seit der Zeit ich alle Erstgeburt in Ägyptenland schlug und
heiligte sie mir \footnote{\textbf{8:17} 2Mo 13,2} \bibverse{18} und
nahm die Leviten an für alle Erstgeburt unter den Kindern Israel
\bibverse{19} und gab sie zum Geschenk Aaron und seinen Söhnen aus den
Kindern Israel, dass sie dienen im Amt der Kinder Israel in der Hütte
des Stifts, die Kinder Israel zu versöhnen, auf dass nicht unter den
Kindern Israel sei eine Plage, wenn sie sich nahen wollten zum
Heiligtum. \bibverse{20} Und Mose mit Aaron samt der ganzen Gemeinde der
Kinder Israel taten mit den Leviten alles, wie der HErr dem Mose geboten
hatte. \bibverse{21} Und die Leviten entsündigten sich und wuschen ihre
Kleider, und Aaron webte sie vor dem HErrn und versöhnte sie, dass sie
rein wurden. \footnote{\textbf{8:21} 4Mo 8,11} \bibverse{22} Darnach
gingen sie hinein, dass sie ihr Amt täten in der Hütte des Stifts vor
Aaron und seinen Söhnen. Wie der HErr dem Mose geboten hatte über die
Leviten, also taten sie mit ihnen. \bibverse{23} Und der HErr redete mit
Mose und sprach: \bibverse{24} Das ist's, was den Leviten gebührt: von
25 Jahren und darüber taugen sie zum Amt und Dienst in der Hütte des
Stifts; \bibverse{25} aber von dem 50. Jahr an sollen sie ledig sein vom
Amt des Dienstes und sollen nicht mehr dienen, \bibverse{26} sondern
ihren Brüdern helfen des Dienstes warten an der Hütte des Stifts; des
Amts aber sollen sie nicht pflegen. Also sollst du mit den Leviten tun,
dass ein jeglicher seines Dienstes warte. \# 9 \bibverse{1} Und der HErr
redete mit Mose in der Wüste Sinai im zweiten Jahr, nachdem sie aus
Ägyptenland gezogen waren, im ersten Monat, und sprach: \bibverse{2}
Lass die Kinder Israel Passah halten zu seiner Zeit, \footnote{\textbf{9:2}
  2Mo 12,3; 3Mo 23,5} \bibverse{3} am 14. Tage dieses Monats gegen
Abend; zu seiner Zeit sollen sie es halten nach aller seiner Satzung und
seinem Recht. \bibverse{4} Und Mose redete mit den Kindern Israel, dass
sie das Passah hielten. \bibverse{5} Und sie hielten Passah am 14. Tage
des ersten Monats gegen Abend in der Wüste Sinai; alles, wie der HErr
dem Mose geboten hatte, so taten die Kinder Israel. \bibverse{6} Da
waren etliche Männer unrein geworden an einem toten Menschen, dass sie
nicht konnten Passah halten des Tages. Die traten vor Mose und Aaron
desselben Tages \bibverse{7} und sprachen zu ihm: Wir sind unrein
geworden an einem toten Menschen; warum sollen wir geringer sein, dass
wir unsere Gabe dem HErrn nicht bringen dürfen zu seiner Zeit unter den
Kindern Israel? \bibverse{8} Mose sprach zu ihnen: Harret, ich will
hören, was euch der HErr gebietet. \bibverse{9} Und der HErr redete mit
Mose und sprach: \bibverse{10} Sage den Kindern Israel und sprich: Wenn
jemand unrein an einem Toten oder ferne über Feld ist, unter euch oder
unter euren Nachkommen, der soll dennoch dem HErrn Passah halten,
\bibverse{11} aber im zweiten Monat, am 14. Tage gegen Abend, und soll's
neben ungesäuertem Brot und bitteren Kräutern essen, \bibverse{12} und
sie sollen nichts davon übriglassen bis morgen, auch kein Bein daran
zerbrechen, und sollen's nach aller Weise des Passah halten.
\bibverse{13} Wer aber rein und nicht über Feld ist und lässt es
anstehen, das Passah zu halten, des Seele soll ausgerottet werden von
seinem Volk, darum dass er seine Gabe dem HErrn nicht gebracht hat zu
seiner Zeit; er soll seine Sünde tragen. \bibverse{14} Und wenn ein
Fremdling bei euch wohnt und auch dem HErrn Passah hält, der soll's
halten nach der Satzung und dem Recht des Passah. Diese Satzung soll
euch gleich sein, dem Fremden wie des Landes Einheimischen.
\bibverse{15} Und des Tages, da die Wohnung aufgerichtet ward, bedeckte
sie eine Wolke auf der Hütte des Zeugnisses; und des Abends bis an den
Morgen war über der Wohnung eine Gestalt des Feuers. \footnote{\textbf{9:15}
  2Mo 40,34-38} \bibverse{16} Also geschah's immerdar, dass die Wolke
sie bedeckte, und des Nachts die Gestalt des Feuers. \bibverse{17} Und
so oft sich die Wolke aufhob von der Hütte, so zogen die Kinder Israel;
und an welchem Ort die Wolke blieb, da lagerten sich die Kinder Israel.
\bibverse{18} Nach dem Wort des HErrn zogen die Kinder Israel, und nach
seinem Wort lagerten sie sich. Solange die Wolke auf der Wohnung blieb,
so lange lagen sie still. \bibverse{19} Und wenn die Wolke viele Tage
verzog auf der Wohnung, so taten die Kinder Israel nach dem Gebot des
HErrn und zogen nicht. \bibverse{20} Und wenn's war, dass die Wolke auf
der Wohnung nur etliche Tage blieb, so lagerten sie sich nach dem Wort
des HErrn und zogen nach dem Wort des HErrn. \bibverse{21} Wenn die
Wolke da war von Abend bis an den Morgen und sich dann erhob, so zogen
sie; oder wenn sie sich des Tages oder des Nachts erhob, so zogen sie
auch. \bibverse{22} Wenn sie aber zwei Tage oder einen Monat oder länger
auf der Wohnung blieb, so lagen die Kinder Israel und zogen nicht; und
wenn sie sich dann erhob, so zogen sie. \bibverse{23} Denn nach des
HErrn Mund lagen sie, und nach des HErrn Mund zogen sie, dass sie täten,
wie der HErr gebot nach des HErrn Wort durch Mose. \# 10 \bibverse{1}
Und der HErr redete mit Mose und sprach: \bibverse{2} Mache dir zwei
Drommeten von getriebenem Silber, dass du sie brauchest, die Gemeinde zu
berufen und wenn das Heer aufbrechen soll. \bibverse{3} Wenn man mit
beiden schlicht bläst, soll sich zu dir versammeln die ganze Gemeinde
vor die Tür der Hütte des Stifts. \bibverse{4} Wenn man nur mit einer
schlicht bläst, so sollen sich zu dir versammeln die Fürsten, die
Obersten über die Tausende in Israel. \bibverse{5} Wenn ihr aber
drommetet, so sollen die Lager aufbrechen, die gegen Morgen liegen.
\bibverse{6} Und wenn ihr zum andernmal drommetet, so sollen die Lager
aufbrechen, die gegen Mittag liegen. Denn wenn sie reisen sollen, so
sollt ihr drommeten. \bibverse{7} Wenn aber die Gemeinde zu versammeln
ist, sollt ihr schlicht blasen und nicht drommeten. \bibverse{8} Es
sollen aber solch Blasen mit den Drommeten die Söhne Aarons, die
Priester, tun; und das soll euer Recht sein ewiglich bei euren
Nachkommen. \bibverse{9} Wenn ihr in einen Streit ziehet in eurem Lande
wider eure Feinde, die euch bedrängen, so sollt ihr drommeten mit den
Drommeten, dass euer gedacht werde vor dem HErrn, eurem Gott, und ihr
erlöst werdet von euren Feinden. \bibverse{10} Desgleichen, wenn ihr
fröhlich seid, und an euren Festen und an euren Neumonden sollt ihr mit
den Drommeten blasen über eure Brandopfer und Dankopfer, dass es euch
sei zum Gedächtnis vor eurem Gott. Ich bin der HErr, euer Gott.
\footnote{\textbf{10:10} 3Mo 23,24; 2Kö 11,14; 2Chr 7,6} \bibverse{11}
Am 20. Tage im zweiten Monat des zweiten Jahres erhob sich die Wolke von
der Wohnung des Zeugnisses. \bibverse{12} Und die Kinder Israel brachen
auf und zogen aus der Wüste Sinai, und die Wolke blieb in der Wüste
Pharan. \bibverse{13} Es brachen aber auf die ersten nach dem Wort des
HErrn durch Mose; \bibverse{14} nämlich das Panier des Lagers der Kinder
Juda zog am ersten mit ihrem Heer, und über ihr Heer war Nahesson, der
Sohn Amminadabs; \bibverse{15} und über das Heer des Stammes der Kinder
Isaschar war Nathanael, der Sohn Zuars; \bibverse{16} und über das Heer
des Stammes der Kinder Sebulon war Eliab, der Sohn Helons. \bibverse{17}
Da zerlegte man die Wohnung, und zogen die Kinder Gerson und Merari und
trugen die Wohnung. \bibverse{18} Darnach zog das Panier des Lagers
Rubens mit ihrem Heer, und über ihr Heer war Elizur, der Sohn Sedeurs;
\bibverse{19} und über das Heer des Stammes der Kinder Simeon war
Selumiel, der Sohn Zuri-Saddais; \bibverse{20} und Eljasaph, der Sohn
Deguels, über das Heer des Stammes der Kinder Gad. \bibverse{21} Da
zogen auch die Kahathiten und trugen das Heiligtum; und jene richteten
die Wohnung auf, bis diese nachkamen. \bibverse{22} Darnach zog das
Panier des Lagers der Kinder Ephraim mit ihrem Heer, und über ihr Heer
war Elisama, der Sohn Ammihuds; \bibverse{23} und Gamliel, der Sohn
Pedazurs, über das Heer des Stammes der Kinder Manasse; \bibverse{24}
und Abidan, der Sohn des Gideoni, über das Heer des Stammes der Kinder
Benjamin. \bibverse{25} Darnach zog das Panier des Lagers der Kinder Dan
mit ihrem Heer; und so waren die Lager alle auf. Und Ahi-Eser, der Sohn
Ammi-Saddais, war über ihr Heer; \bibverse{26} und Pagiel, der Sohn
Ochrans, über das Heer des Stammes der Kinder Asser; \bibverse{27} und
Ahira, der Sohn Enans, über das Heer des Stammes der Kinder Naphthali.
\bibverse{28} So zogen die Kinder Israel mit ihrem Heer. \bibverse{29}
Und Mose sprach zu seinem Schwager Hobab, dem Sohn Reguels, aus Midian:
Wir ziehen dahin an die Stätte, davon der HErr gesagt hat: Ich will sie
euch geben; so komm nun mit uns, so wollen wir das Beste an dir tun;
denn der HErr hat Israel Gutes zugesagt. \footnote{\textbf{10:29} Ri
  1,16; 2Mo 2,18} \bibverse{30} Er aber antwortete: Ich will nicht mit
euch, sondern in mein Land zu meiner Freundschaft ziehen. \bibverse{31}
Er sprach: Verlass uns doch nicht; denn du weißt, wo wir in der Wüste
uns lagern sollen, und sollst unser Auge sein. \bibverse{32} Und wenn du
mit uns ziehst: was der HErr Gutes an uns tut, das wollen wir an dir
tun. \bibverse{33} Also zogen sie von dem Berge des HErrn drei
Tagereisen, und die Lade des Bundes des HErrn zog vor ihnen her die drei
Tagereisen, ihnen zu weisen, wo sie ruhen sollten. \bibverse{34} Und die
Wolke des HErrn war des Tages über ihnen, wenn sie aus dem Lager zogen.
\bibverse{35} Und wenn die Lade zog, so sprach Mose: HErr, stehe auf!
lass deine Feinde zerstreut und die dich hassen, flüchtig werden vor
dir! \footnote{\textbf{10:35} Ps 68,2; Ps 132,8} \bibverse{36} Und wenn
sie ruhte, so sprach er: Komm wieder, HErr, zu der Menge der Tausende
Israels! \# 11 \bibverse{1} Und da sich das Volk ungeduldig machte,
gefiel es übel vor den Ohren des HErrn. Und als es der HErr hörte,
ergrimmte sein Zorn, und zündete das Feuer des HErrn unter ihnen an; das
verzehrte die äußersten Lager. \bibverse{2} Da schrie das Volk zu Mose,
und Mose bat den HErrn; da verschwand das Feuer. \bibverse{3} Und man
hieß die Stätte Thabeera, darum dass sich unter ihnen des HErrn Feuer
angezündet hatte. \bibverse{4} Das Pöbelvolk aber unter ihnen war
lüstern geworden, und sie saßen und weinten samt den Kindern Israel und
sprachen: Wer will uns Fleisch zu essen geben? \footnote{\textbf{11:4}
  2Mo 16,3} \bibverse{5} Wir gedenken der Fische, die wir in Ägypten
umsonst aßen, und der Kürbisse, der Melonen, des Lauchs, der Zwiebeln
und des Knoblauchs. \bibverse{6} Nun aber ist unsere Seele matt; denn
unsere Augen sehen nichts als das Man. \bibverse{7} Es war aber das Man
wie Koriandersamen und anzusehen wie Bedellion. \bibverse{8} Und das
Volk lief hin und her und sammelte und zerrieb es mit Mühlen und stieß
es in Mörsern und kochte es in Töpfen und machte sich Aschenkuchen
daraus; und es hatte einen Geschmack wie ein Ölkuchen. \bibverse{9} Und
wenn des Nachts der Tau über die Lager fiel, so fiel das Man mit darauf.
\bibverse{10} Da nun Mose das Volk hörte weinen unter ihren
Geschlechtern, einen jeglichen in seiner Hütte Tür, da ergrimmte der
Zorn des HErrn sehr, und Mose ward auch bange. \bibverse{11} Und Mose
sprach zu dem HErrn: Warum bekümmerst du deinen Knecht? und warum finde
ich nicht Gnade vor deinen Augen, dass du die Last dieses ganzen Volks
auf mich legst? \bibverse{12} Habe ich nun all das Volk empfangen oder
geboren, dass du zu mir sagen magst: Trag es in deinen Armen, wie eine
Amme ein Kind trägt, in das Land, das du ihren Vätern geschworen hast?
\bibverse{13} Woher soll ich Fleisch nehmen, dass ich allem diesem Volk
gebe? Sie weinen vor mir und sprechen: Gib uns Fleisch, dass wir essen.
\bibverse{14} Ich vermag alles das Volk nicht allein zu ertragen; denn
es ist mir zu schwer. \bibverse{15} Und willst du also mit mir tun, so
erwürge mich lieber, habe ich anders Gnade vor deinen Augen gefunden,
dass ich nicht mein Unglück so sehen müsse. \footnote{\textbf{11:15} 2Mo
  32,32} \bibverse{16} Und der HErr sprach zu Mose: Sammle mir 70 Männer
unter den Ältesten Israels, von denen du weißt, dass sie Älteste im Volk
und seine Amtleute sind, und nimm sie vor die Hütte des Stifts und
stelle sie daselbst vor dich, \footnote{\textbf{11:16} 2Mo 18,21; 2Mo
  24,1} \bibverse{17} so will ich herniederkommen und mit dir daselbst
reden und von deinem Geist, der auf dir ist, nehmen und auf sie legen,
dass sie mit dir die Last des Volkes tragen, dass du nicht allein
tragest. \bibverse{18} Und zum Volk sollst du sagen: Heiliget euch auf
morgen, dass ihr Fleisch esset; denn euer Weinen ist vor die Ohren des
HErrn gekommen, die ihr sprecht: Wer gibt uns Fleisch zu essen? denn es
ging uns wohl in Ägypten. Darum wird euch der HErr Fleisch geben, dass
ihr esset, \footnote{\textbf{11:18} 2Mo 19,10} \bibverse{19} nicht einen
Tag, nicht zwei, nicht fünf, nicht zehn, nicht zwanzig Tage lang,
\bibverse{20} sondern einen Monat lang, bis dass es euch zur Nase
ausgehe und euch ein Ekel sei; darum dass ihr den HErrn verworfen habt,
der unter euch ist, und vor ihm geweint und gesagt: Warum sind wir aus
Ägypten gegangen? \bibverse{21} Und Mose sprach: 600.000 Mann Fußvolk
ist es, darunter ich bin, und du sprichst: Ich will euch Fleisch geben,
dass ihr esset einen Monat lang! \bibverse{22} Soll man Schafe und
Rinder schlachten, dass es ihnen genug sei? Oder werden sich alle Fische
des Meeres herzu versammeln, dass es ihnen genug sei? \bibverse{23} Der
HErr aber sprach zu Mose: Ist denn die Hand des HErrn verkürzt? Aber du
sollst jetzt sehen, ob meine Worte können dir etwas gelten oder nicht.
\footnote{\textbf{11:23} Jes 50,2; Jes 59,1} \bibverse{24} Und Mose ging
heraus und sagte dem Volk des HErrn Worte und versammelte 70 Männer
unter den Ältesten des Volks und stellte sie um die Hütte her.
\bibverse{25} Da kam der HErr hernieder in der Wolke und redete mit ihm
und nahm von dem Geist, der auf ihm war, und legte ihn auf die 70
ältesten Männer. Und da der Geist auf ihnen ruhte, weissagten sie und
hörten nicht auf. \bibverse{26} Es waren aber noch zwei Männer im Lager
geblieben; der eine hieß Eldad, der andere Medad, und der Geist ruhte
auf ihnen; denn sie waren auch angeschrieben und doch nicht
hinausgegangen zu der Hütte, und sie weissagten im Lager. \bibverse{27}
Da lief ein Knabe hin und sagte es Mose an und sprach: Eldad und Medad
weissagen im Lager. \bibverse{28} Da antwortete Josua, der Sohn Nuns,
Moses Diener, den er erwählt hatte, und sprach: Mein Herr Mose, wehre
ihnen. \bibverse{29} Aber Mose sprach zu ihm: Bist du der Eiferer für
mich? Wollte Gott, dass all das Volk des HErrn weissagte und der HErr
seinen Geist über sie gäbe! \footnote{\textbf{11:29} Mk 9,39; Joe 3,1}
\bibverse{30} Also sammelte sich Mose zum Lager mit den Ältesten
Israels. \bibverse{31} Da fuhr aus der Wind von dem HErrn und ließ
Wachteln kommen vom Meer und streute sie über das Lager, hier eine
Tagereise lang, da eine Tagereise lang um das Lager her, zwei Ellen hoch
über der Erde. \bibverse{32} Da machte sich das Volk auf denselben
ganzen Tag und die ganze Nacht und den ganzen anderen Tag und sammelten
Wachteln; und welcher am wenigsten sammelte, der sammelte zehn Homer.
Und sie hängten sie auf um das Lager her. \bibverse{33} Da aber das
Fleisch noch unter ihren Zähnen war und ehe es aufgezehrt war, da
ergrimmte der Zorn des HErrn unter dem Volk, und schlug sie mit einer
sehr großen Plage. \bibverse{34} Daher heißt diese Stätte Lustgräber,
darum dass man daselbst begrub das lüsterne Volk. \footnote{\textbf{11:34}
  1Kor 10,6} \bibverse{35} Von den Lustgräbern aber zog das Volk aus gen
Hazeroth, und sie blieben zu Hazeroth. \# 12 \bibverse{1} Und Mirjam und
Aaron redeten wider Mose um seines Weibes willen, der Mohrin, die er
genommen hatte, darum dass er eine Mohrin zum Weibe genommen hatte,
\bibverse{2} und sprachen: Redet denn der HErr allein durch Mose? Redet
er nicht auch durch uns? Und der HErr hörte es. \bibverse{3} Aber Mose
war ein sehr geplagter Mensch über alle Menschen auf Erden. \bibverse{4}
Und plötzlich sprach der HErr zu Mose und zu Aaron und zu Mirjam: Gehet
heraus, ihr drei, zu der Hütte des Stifts. Und sie gingen alle drei
heraus. \bibverse{5} Da kam der HErr hernieder in der Wolkensäule und
trat in der Hütte Tür und rief Aaron und Mirjam; und die gingen beide
hinaus. \footnote{\textbf{12:5} 2Mo 16,10} \bibverse{6} Und er sprach:
Höret meine Worte: Ist jemand unter euch ein Prophet des HErrn, dem will
ich mich kundmachen in einem Gesicht oder will mit ihm reden in einem
Traum. \bibverse{7} Aber nicht also mein Knecht Mose, der in meinem
ganzen Hause treu ist. \bibverse{8} Mündlich rede ich mit ihm, und er
sieht den HErrn in seiner Gestalt, nicht durch dunkle Worte oder
Gleichnisse. Warum habt ihr euch denn nicht gefürchtet, wider meinen
Knecht Mose zu reden? \footnote{\textbf{12:8} 2Mo 33,11; 2Mo 33,23}
\bibverse{9} Und der Zorn des HErrn ergrimmte über sie, und er wandte
sich weg; \bibverse{10} dazu die Wolke wich auch von der Hütte. Und
siehe, da war Mirjam aussätzig wie der Schnee. Und Aaron wandte sich zu
Mirjam und wird gewahr, dass sie aussätzig ist, \bibverse{11} Und sprach
zu Mose: Ach, mein Herr, lass die Sünde nicht auf uns bleiben, mit der
wir töricht getan und uns versündigt haben, \bibverse{12} dass diese
nicht sei wie ein Totes, das von seiner Mutter Leibe kommt und ist schon
die Hälfte seines Fleisches gefressen. \bibverse{13} Mose aber schrie zu
dem HErrn und sprach: Ach Gott, heile sie! \footnote{\textbf{12:13} 2Mo
  15,26} \bibverse{14} Der HErr sprach zu Mose: Wenn ihr Vater ihr ins
Angesicht gespien hätte, sollte sie nicht sieben Tage sich schämen? Lass
sie verschließen sieben Tage außerhalb des Lagers; darnach lass sie
wieder aufnehmen. \footnote{\textbf{12:14} 3Mo 13,46} \bibverse{15} Also
ward Mirjam sieben Tage verschlossen außerhalb des Lagers. Und das Volk
zog nicht weiter, bis Mirjam aufgenommen ward. \bibverse{16} Darnach zog
das Volk von Hazeroth und lagerte sich in die Wüste Pharan. \# 13
\bibverse{1} Und der HErr redet mit Mose und sprach: \bibverse{2} Sende
Männer aus, die das Land Kanaan erkunden, das ich den Kindern Israel
geben will, aus jeglichem Stamm ihrer Väter einen vornehmen Mann.
\bibverse{3} Mose, der sandte sie aus der Wüste Pharan nach dem Wort des
HErrn, die alle vornehme Männer waren unter den Kindern Israel.
\bibverse{4} Und hießen also: Sammua, der Sohn Sakkurs, des Stammes
Ruben; \bibverse{5} Saphat, der Sohn Horis, des Stammes Simeon;
\bibverse{6} Kaleb, der Sohn Jephunnes, des Stammes Juda; \footnote{\textbf{13:6}
  Jos 14,7} \bibverse{7} Jigeal, der Sohn Josephs, des Stammes Isaschar;
\bibverse{8} Hosea, der Sohn Nuns, des Stammes Ephraim; \bibverse{9}
Palti, der Sohn Raphus, des Stammes Benjamin; \bibverse{10} Gaddiel, der
Sohn Sodis, des Stammes Sebulon; \bibverse{11} Gaddi, der Sohn Susis,
des Stammes Joseph von Manasse; \bibverse{12} Ammiel, der Sohn Gemallis,
des Stammes Dan; \bibverse{13} Sethur, der Sohn Michaels, des Stammes
Asser; \bibverse{14} Nahebi, der Sohn Vaphsis, des Stammes Naphthali;
\bibverse{15} Guel, der Sohn Machis, des Stammes Gad. \bibverse{16} Das
sind die Namen der Männer, die Mose aussandte, zu erkunden das Land.
Aber den Hosea, den Sohn Nuns, nannte Mose Josua. \footnote{\textbf{13:16}
  4Mo 11,28} \bibverse{17} Da sie nun Mose sandte, das Land Kanaan zu
erkunden, sprach er zu ihnen: Ziehet hinauf ins Mittagsland und geht auf
das Gebirge \bibverse{18} und besehet das Land, wie es ist, und das
Volk, das darin wohnt, ob's stark oder schwach, wenig oder viel ist;
\bibverse{19} und was es für ein Land ist, darin sie wohnen, ob's gut
oder böse sei; und was es für Städte sind, darin sie wohnen, ob sie in
Gezelten oder Festungen wohnen; \bibverse{20} und was es für Land sei,
ob's fett oder mager sei und ob Bäume darin sind oder nicht. Seid
getrost und nehmet die Früchte des Landes. Es war aber eben um die Zeit
der ersten Weintrauben. \bibverse{21} Sie gingen hinauf und erkundeten
das Land von der Wüste Zin bis gen Rehob, da man gen Hamath geht.
\bibverse{22} Sie gingen auch hinauf ins Mittagsland und kamen bis gen
Hebron; da waren Ahiman, Sesai und Thalmai, die Kinder Enaks. Hebron
aber war sieben Jahre gebaut vor Zoan in Ägypten. \bibverse{23} Und sie
kamen bis an den Bach Eskol und schnitten daselbst eine Rebe ab mit
einer Weintraube und ließen sie zwei auf einem Stecken tragen, dazu auch
Granatäpfel und Feigen. \bibverse{24} Der Ort heißt Bach Eskol um der
Traube willen, die die Kinder Israel daselbst abschnitten. \bibverse{25}
Und sie kehrten um, als sie das Land erkundet hatten, nach 40 Tagen,
\bibverse{26} gingen hin und kamen zu Mose und Aaron und zu der ganzen
Gemeinde der Kinder Israel in die Wüste Pharan gen Kades und sagten
ihnen wieder und der ganzen Gemeinde, wie es stände; und ließen sie die
Früchte des Landes sehen. \bibverse{27} Und erzählten ihnen und
sprachen: Wir sind in das Land gekommen, dahin ihr uns sandtet, darin
Milch und Honig fließt, und dies ist seine Frucht; \bibverse{28} nur,
dass starkes Volk darin wohnt und sehr große und feste Städte sind; und
wir sahen auch Enaks Kinder daselbst. \bibverse{29} So wohnen die
Amalekiter im Lande gegen Mittag, die Hethiter und Jebusiter und
Amoriter wohnen auf dem Gebirge, die Kanaaniter aber wohnen am Meer und
um den Jordan. \bibverse{30} Kaleb aber stillte das Volk gegen Mose und
sprach: Lasst uns hinaufziehen und das Land einnehmen; denn wir können
es überwältigen. \footnote{\textbf{13:30} 4Mo 13,6} \bibverse{31} Aber
die Männer, die mit ihm waren hinaufgezogen, sprachen: Wir vermögen
nicht hinaufzuziehen gegen das Volk; denn sie sind uns zu stark, --
\bibverse{32} und machten dem Lande, das sie erkundet hatten, ein böses
Geschrei unter den Kindern Israel und sprachen: Das Land, dadurch wir
gegangen sind, es zu erkunden, frisst seine Einwohner, und alles Volk,
das wir darin sahen, sind Leute von großer Länge. \bibverse{33} Wir
sahen auch Riesen daselbst, Enaks Kinder von den Riesen; und wir waren
vor unseren Augen wie Heuschrecken, und also waren wir auch vor ihren
Augen. \# 14 \bibverse{1} Da fuhr die ganze Gemeinde auf und schrie, und
das Volk weinte die Nacht. \bibverse{2} Und alle Kinder Israel murrten
wider Mose und Aaron, und die ganze Gemeinde sprach zu ihnen: Ach, dass
wir in Ägyptenland gestorben wären oder noch stürben in dieser Wüste!
\footnote{\textbf{14:2} 2Mo 16,3} \bibverse{3} Warum führt uns der HErr
in dieses Land, dass wir durchs Schwert fallen und unsere Weiber und
unsere Kinder ein Raub werden? Ist's nicht besser, wir ziehen wieder
nach Ägypten? \footnote{\textbf{14:3} Ps 106,24} \bibverse{4} Und einer
sprach zu dem anderen: Lasst uns einen Hauptmann aufwerfen und wieder
nach Ägypten ziehen! \bibverse{5} Mose aber und Aaron fielen auf ihr
Angesicht vor der ganzen Versammlung der Gemeinde der Kinder Israel.
\footnote{\textbf{14:5} 4Mo 16,4} \bibverse{6} Und Josua, der Sohn Nuns,
und Kaleb, der Sohn Jephunnes, die auch das Land erkundet hatten,
zerrissen ihre Kleider \footnote{\textbf{14:6} 4Mo 13,16; 4Mo 13,30}
\bibverse{7} und sprachen zu der ganzen Gemeinde der Kinder Israel: Das
Land, das wir durchwandelt haben, es zu erkunden, ist sehr gut.
\bibverse{8} Wenn der HErr uns gnädig ist, so wird er uns in das Land
bringen und es uns geben, ein Land, darin Milch und Honig fließt.
\footnote{\textbf{14:8} 4Mo 13,27} \bibverse{9} Fallet nur nicht ab vom
HErrn und fürchtet euch vor dem Volk dieses Landes nicht; denn wir
wollen sie wie Brot fressen. Es ist ihr Schutz von ihnen gewichen; der
HErr aber ist mit uns. Fürchtet euch nicht vor ihnen. \bibverse{10} Da
sprach das ganze Volk, man sollte sie steinigen. Da erschien die
Herrlichkeit des HErrn in der Hütte des Stifts allen Kindern Israel.
\bibverse{11} Und der HErr sprach zu Mose: Wie lange lästert mich dieses
Volk? und wie lange wollen sie nicht an mich glauben durch allerlei
Zeichen, die ich unter ihnen getan habe? \bibverse{12} So will ich sie
mit Pestilenz schlagen und vertilgen und dich zu einem größeren und
mächtigeren Volk machen, denn dies ist. \footnote{\textbf{14:12} 2Mo
  32,10-14} \bibverse{13} Mose aber sprach zu dem HErrn: So werden's die
Ägypter hören; denn du hast dieses Volk mit deiner Kraft mitten aus
ihnen geführt. \bibverse{14} Und man wird es sagen zu den Einwohnern
dieses Landes, die da gehört haben, dass du, HErr, unter diesem Volk
seist, dass du von Angesicht gesehen werdest und deine Wolke stehe über
ihnen und du, HErr, gehest vor ihnen her in der Wolkensäule des Tages
und Feuersäule des Nachts. \bibverse{15} Würdest du nun dieses Volk
töten, wie einen Mann, so würden die Heiden sagen, die solch Gerücht von
dir hörten, und sprechen: \bibverse{16} Der HErr konnte mitnichten
dieses Volk in das Land bringen, das er ihnen geschworen hatte; darum
hat er sie geschlachtet in der Wüste. \bibverse{17} So lass nun die
Kraft des HErrn groß werden, wie du gesagt hast und gesprochen:
\bibverse{18} Der HErr ist geduldig und von großer Barmherzigkeit und
vergibt Missetat und Übertretung und lässt niemand ungestraft, sondern
sucht heim die Missetat der Väter über die Kinder ins dritte und vierte
Glied. \footnote{\textbf{14:18} 2Mo 34,6-7} \bibverse{19} So sei nun
gnädig der Missetat dieses Volks nach deiner großen Barmherzigkeit, wie
du auch vergeben hast diesem Volk aus Ägypten bis hierher. \bibverse{20}
Und der HErr sprach: Ich habe es vergeben, wie du gesagt hast.
\bibverse{21} Aber so wahr als ich lebe, so soll alle Welt der
Herrlichkeit des HErrn voll werden. \bibverse{22} Denn alle die Männer,
die meine Herrlichkeit und meine Zeichen gesehen haben, die ich getan
habe in Ägypten und in der Wüste, und mich nun zehnmal versucht und
meiner Stimme nicht gehorcht haben, \bibverse{23} deren soll keiner das
Land sehen, das ich ihren Vätern geschworen habe; auch keiner soll es
sehen, der mich verlästert hat. \footnote{\textbf{14:23} Ps 95,11; Hebr
  3,17-19} \bibverse{24} Aber meinen Knecht Kaleb, darum dass ein
anderer Geist mit ihm ist und er mir treulich nachgefolgt ist, den will
ich in das Land bringen, darein er gekommen ist, und sein Same soll es
einnehmen, \footnote{\textbf{14:24} Jos 14,6; Jos 14,9} \bibverse{25}
dazu die Amalekiter und Kanaaniter, die im Tale wohnen. Morgen wendet
euch und ziehet in die Wüste auf dem Wege zum Schilfmeer. \bibverse{26}
Und der HErr redete mit Mose und Aaron und sprach: \bibverse{27} Wie
lange murrt diese böse Gemeinde wider mich? Denn ich habe das Murren der
Kinder Israel, das sie wider mich gemurrt haben, gehört. \bibverse{28}
Darum sprich zu ihnen: So wahr ich lebe, spricht der HErr, ich will euch
tun, wie ihr vor meinen Ohren gesagt habt. \bibverse{29} Eure Leiber
sollen in dieser Wüste verfallen; und alle, die ihr gezählt seid von 20
Jahren und darüber, die ihr wider mich gemurrt habt, \bibverse{30} sollt
nicht in das Land kommen, darüber ich meine Hand gehoben habe, dass ich
euch darin wohnen ließe, außer Kaleb, dem Sohn Jephunnes, und Josua, dem
Sohn Nuns. \bibverse{31} Eure Kinder, von denen ihr sagtet: Sie werden
ein Raub sein, die will ich hineinbringen, dass sie erkennen sollen das
Land, das ihr verwerft. \bibverse{32} Aber ihr samt euren Leibern sollt
in dieser Wüste verfallen. \bibverse{33} Und eure Kinder sollen Hirten
sein in dieser Wüste 40 Jahre und eure Untreue tragen, bis dass eure
Leiber aufgerieben werden in der Wüste, \bibverse{34} nach der Zahl der
40 Tage, darin ihr das Land erkundet habt; je ein Tag soll ein Jahr
gelten, dass ihr 40 Jahre eure Missetaten tragt; auf dass ihr
innewerdet, was es sei, wenn ich die Hand abziehe. \footnote{\textbf{14:34}
  Jer 2,19} \bibverse{35} Ich, der HErr, habe es gesagt; das will ich
auch tun aller dieser bösen Gemeinde, die sich wider mich empört hat. In
dieser Wüste sollen sie aufgerieben werden und daselbst sterben.
\bibverse{36} Also starben durch die Plage vor dem HErrn alle die
Männer, die Mose gesandt hatte, das Land zu erkunden, und wiedergekommen
waren und wider ihn murren machten die ganze Gemeinde, \bibverse{37}
damit dass sie dem Lande ein Geschrei machten, dass es böse wäre.
\bibverse{38} Aber Josua, der Sohn Nuns, und Kaleb, der Sohn Jephunnes,
blieben lebendig aus den Männern, die gegangen waren, das Land zu
erkunden. \footnote{\textbf{14:38} 4Mo 14,30} \bibverse{39} Und Mose
redete diese Worte zu allen Kindern Israel. Da trauerte das Volk sehr,
\bibverse{40} und sie machten sich des Morgens früh auf und zogen auf
die Höhe des Gebirges und sprachen: Hier sind wir und wollen
hinaufziehen an die Stätte, davon der HErr gesagt hat; denn wir haben
gesündigt. \bibverse{41} Mose aber sprach: Warum übertretet ihr also das
Wort des HErrn? Es wird euch nicht gelingen. \bibverse{42} Ziehet nicht
hinauf -- denn der HErr ist nicht unter Euch --, dass ihr nicht
geschlagen werdet vor euren Feinden. \bibverse{43} Denn die Amalekiter
und Kanaaniter sind vor euch daselbst, und ihr werdet durchs Schwert
fallen, darum dass ihr euch vom HErrn gekehrt habt, und der HErr wird
nicht mit euch sein. \bibverse{44} Aber sie waren störrig,
hinaufzuziehen auf die Höhe des Gebirges; aber die Lade des Bundes des
HErrn und Mose kamen nicht aus dem Lager. \bibverse{45} Da kamen die
Amalekiter und Kanaaniter, die auf dem Gebirge wohnten, herab und
schlugen und zersprengten sie bis gen Horma. \footnote{\textbf{14:45}
  4Mo 31,3}

\hypertarget{section-2}{%
\section{15}\label{section-2}}

\bibverse{1} Und der HErr redete mit Mose und sprach: \bibverse{2} Rede
mit den Kindern Israel und sprich zu ihnen: Wenn ihr in das Land eurer
Wohnung kommt, das ich euch geben werde, \bibverse{3} und wollt dem
HErrn Opfer tun, es sei ein Brandopfer oder ein Opfer zum besonderen
Gelübde oder ein freiwilliges Opfer oder euer Festopfer, auf dass ihr
dem HErrn einen süßen Geruch machet von Rindern oder von Schafen:
\bibverse{4} wer nun seine Gabe dem HErrn opfern will, der soll das
Speisopfer tun, ein Zehntel Semmelmehl, mit einem viertel Hin Öl;
\bibverse{5} und Wein zum Trankopfer, auch ein viertel Hin, zu dem
Brandopfer oder sonst zu dem Opfer, da ein Lamm geopfert wird.
\footnote{\textbf{15:5} 4Mo 28,7} \bibverse{6} Wenn aber ein Widder
geopfert wird, sollst du das Speisopfer machen aus zwei Zehntel
Semmelmehl, mit einem drittel Hin Öl gemengt, \bibverse{7} und Wein zum
Trankopfer, auch ein drittel Hin; das sollst du dem HErrn zum süßen
Geruch opfern. \bibverse{8} Willst du aber ein Rind zum Brandopfer oder
zum besonderen Gelübdeopfer oder zum Dankopfer dem HErrn machen,
\bibverse{9} so sollst du zu dem Rind ein Speisopfer tun, drei Zehntel
Semmelmehl, mit einem halben Hin Öl gemengt, \bibverse{10} und Wein zum
Trankopfer, auch ein halbes Hin; das ist ein Opfer dem HErrn zum süßen
Geruch. \bibverse{11} Also sollst du tun mit einem Ochsen, mit einem
Widder, mit einem Schaf oder mit einer Ziege. \bibverse{12} Darnach die
Zahl dieser Opfer ist, darnach soll auch die Zahl der Speisopfer und
Trankopfer sein. \bibverse{13} Wer ein Einheimischer ist, der soll
solches tun, dass er dem HErrn opfere ein Opfer zum süßen Geruch.
\bibverse{14} Und wenn ein Fremdling bei euch wohnt oder unter euch bei
euren Nachkommen ist, und will dem HErrn ein Opfer zum süßen Geruch tun,
der soll tun, wie ihr tut. \bibverse{15} Der ganzen Gemeinde sei eine
Satzung, euch sowohl als den Fremdlingen; eine ewige Satzung soll das
sein euren Nachkommen, dass vor dem HErrn der Fremdling sei wie ihr.
\bibverse{16} Ein Gesetz, ein Recht soll euch und dem Fremdling sein,
der bei euch wohnt. \bibverse{17} Und der HErr redete mit Mose und
sprach: \bibverse{18} Rede mit den Kindern Israel und sprich zu ihnen:
Wenn ihr in das Land kommt, darein ich euch bringen werde, \bibverse{19}
dass ihr esset von dem Brot im Lande, sollt ihr dem HErrn eine Hebe
geben: \footnote{\textbf{15:19} 2Mo 23,16; 2Mo 23,19} \bibverse{20} als
eures Teiges Erstling sollt ihr einen Kuchen zur Hebe geben; wie die
Hebe von der Scheune, \bibverse{21} also sollt ihr auch dem HErrn eures
Teiges Erstling zur Hebe geben bei euren Nachkommen. \bibverse{22} Und
wenn ihr aus Versehen dieser Gebote irgendeins nicht tut, die der HErr
zu Mose geredet hat, \bibverse{23} alles, was der HErr euch durch Mose
geboten hat, von dem Tage an, da er anfing zu gebieten auf eure
Nachkommen; \bibverse{24} wenn nun ohne Wissen der Gemeinde etwas
versehen würde, so soll die ganze Gemeinde einen jungen Farren aus den
Rindern zum Brandopfer machen, zum süßen Geruch dem HErrn, samt seinem
Speisopfer und Trankopfer, wie es recht ist, und einen Ziegenbock zum
Sündopfer. \bibverse{25} Und der Priester soll also die ganze Gemeinde
der Kinder Israel versöhnen, so wird's ihnen vergeben sein; denn es ist
ein Versehen. Und sie sollen bringen solch ihre Gabe zum Opfer dem HErrn
und ihr Sündopfer vor den HErrn über ihr Versehen, \bibverse{26} so
wird's vergeben der ganzen Gemeinde der Kinder Israel, dazu auch dem
Fremdling, der unter euch wohnt, weil das ganze Volk an solchem versehen
teilhat. \bibverse{27} Wenn aber eine Seele aus Versehen sündigen wird,
die soll eine jährige Ziege zum Sündopfer bringen. \footnote{\textbf{15:27}
  3Mo 4,27-28} \bibverse{28} Und der Priester soll versöhnen solche
Seele, die aus Versehen gesündigt hat, vor dem HErrn, dass er sie
versöhne und ihr vergeben werde. \bibverse{29} Und es soll ein Gesetz
sein für die, die ein Versehen begehen, für den Einheimischen unter den
Kindern Israel und für den Fremdling, der unter ihnen wohnt.
\bibverse{30} Wenn aber eine Seele aus Frevel etwas tut, es sei ein
Einheimischer oder Fremdling, der hat den HErrn geschmäht. Solche Seele
soll ausgerottet werden aus ihrem Volk; \bibverse{31} denn sie hat des
HErrn Wort verachtet und sein Gebot lassen fahren. Ja, sie soll
ausgerottet werden; die Schuld sei ihr. \bibverse{32} Als nun die Kinder
Israel in der Wüste waren, fanden sie einen Mann Holz lesen am
Sabbattage. \bibverse{33} Und die ihn darob gefunden hatten, da er Holz
las, brachten ihn zu Mose und Aaron und vor die ganze Gemeinde.
\bibverse{34} Und sie legten ihn gefangen; denn es war nicht klar
ausgedrückt, was man mit ihm tun sollte. \footnote{\textbf{15:34} 3Mo
  24,12; 2Mo 31,14; 2Mo 35,2} \bibverse{35} Der HErr aber sprach zu
Mose: Der Mann soll des Todes sterben; die ganze Gemeinde soll ihn
steinigen draußen vor dem Lager. \bibverse{36} Da führte die ganze
Gemeinde ihn hinaus vor das Lager und steinigten ihn, dass er starb, wie
der HErr dem Mose geboten hatte. \bibverse{37} Und der HErr sprach zu
Mose: \bibverse{38} Rede mit den Kindern Israel und sprich zu ihnen,
dass sie sich Quasten machen an den Zipfeln ihrer Kleider samt allen
ihren Nachkommen, und blaue Schnüre auf die Quasten an die Zipfel tun;
\bibverse{39} und sollen euch die Quasten dazu dienen, dass ihr sie
ansehet und gedenket aller Gebote des HErrn und tut sie, dass ihr nicht
von eures Herzens Dünken noch von euren Augen euch umtreiben lasset und
abgöttisch werdet. \bibverse{40} Darum sollt ihr gedenken und tun alle
meine Gebote und heilig sein eurem Gott. \bibverse{41} Ich bin der HErr,
euer Gott, der euch aus Ägyptenland geführt hat, dass ich euer Gott
wäre, ich, der HErr, euer Gott. \# 16 \bibverse{1} Und Korah, der Sohn
Jizhars, des Sohnes Kahaths, des Sohnes Levis, samt Dathan und Abiram,
den Söhnen Eliabs, und On, dem Sohn Peleths, den Söhnen Rubens,
\footnote{\textbf{16:1} 2Mo 6,18; 2Mo 6,21; 4Mo 26,9; Jud 1,11}
\bibverse{2} die empörten sich wider Mose samt etlichen Männern unter
den Kindern Israel, 250, Vornehmste in der Gemeinde, Ratsherren und
namhafte Leute. \footnote{\textbf{16:2} 4Mo 12,1-2} \bibverse{3} Und sie
versammelten sich wider Mose und Aaron und sprachen zu ihnen: Ihr
macht's zu viel. Denn die ganze Gemeinde ist überall heilig, und der
HErr ist unter ihnen; warum erhebt ihr euch über die Gemeinde des HErrn?
\bibverse{4} Da das Mose hörte, fiel er auf sein Angesicht \footnote{\textbf{16:4}
  4Mo 14,5} \bibverse{5} und sprach zu Korah und zu seiner ganzen Rotte:
Morgen wird der HErr kundtun, wer sein sei, wer heilig sei und zu ihm
nahen soll; welchen er erwählt, der soll zu ihm nahen. \footnote{\textbf{16:5}
  2Tim 2,19} \bibverse{6} Das tut: nehmet euch Pfannen, Korah und seine
ganze Rotte, \bibverse{7} und legt Feuer darein und tut Räuchwerk darauf
vor dem HErrn morgen. Welchen der HErr erwählt, der sei heilig. Ihr
macht's zu viel, ihr Kinder Levi. \bibverse{8} Und Mose sprach zu Korah:
Höret doch, ihr Kinder Levi! \bibverse{9} Ist's euch zu wenig, dass euch
der Gott Israels ausgesondert hat von der Gemeinde Israel, dass ihr zu
ihm nahen sollt, dass ihr dienet im Amt der Wohnung des HErrn und vor
die Gemeinde tretet, ihr zu dienen? \footnote{\textbf{16:9} 4Mo 3,6-13;
  4Mo 4,4-20} \bibverse{10} Er hat dich und alle deine Brüder, die
Kinder Levi, samt dir zu sich genommen; und ihr sucht nun auch das
Priestertum? \bibverse{11} Du und deine ganze Rotte macht einen Aufruhr
wider den HErrn. Was ist Aaron, dass ihr wider ihn murret? \bibverse{12}
Und Mose schickte hin und ließ Dathan und Abiram rufen, die Söhne
Eliabs. Sie aber sprachen: Wir kommen nicht hinauf. \bibverse{13} Ist's
zu wenig, dass du uns aus dem Lande geführt hast, darin Milch und Honig
fließt, dass du uns tötest in der Wüste? Du musst auch noch über uns
herrschen? \bibverse{14} Wie fein hast du uns gebracht in ein Land,
darin Milch und Honig fließt, und hast uns Äcker und Weinberge zum
Erbteil gegeben! Willst du den Leuten auch die Augen ausreißen? Wir
kommen nicht hinauf. \footnote{\textbf{16:14} 2Mo 3,8; 2Mo 3,17}
\bibverse{15} Da ergrimmte Mose sehr und sprach zu dem HErrn: Wende dich
nicht zu ihrem Speisopfer! Ich habe nicht einen Esel von ihnen genommen
und habe ihrer keinem nie ein Leid getan. \footnote{\textbf{16:15} 1Sam
  12,3; Apg 20,33} \bibverse{16} Und er sprach zu Korah: Du und deine
ganze Rotte sollt morgen vor dem HErrn sein; du, sie auch und Aaron.
\bibverse{17} Und ein jeglicher nehme seine Pfanne und lege Räuchwerk
darauf, und tretet herzu vor den HErrn, ein jeglicher mit seiner Pfanne,
das sind 250 Pfannen; auch du und Aaron, ein jeglicher mit seiner
Pfanne. \bibverse{18} Und ein jeglicher nahm seine Pfanne und legte
Feuer darein und tat Räuchwerk darauf; und sie traten vor die Tür der
Hütte des Stifts, und Mose und Aaron auch. \bibverse{19} Und Korah
versammelte wider sie die ganze Gemeinde vor der Tür der Hütte des
Stifts. Aber die Herrlichkeit des HErrn erschien vor der ganzen
Gemeinde. \footnote{\textbf{16:19} 4Mo 14,10} \bibverse{20} Und der HErr
redete mit Mose und Aaron und sprach: \bibverse{21} Scheidet euch von
dieser Gemeinde, dass ich sie plötzlich vertilge. \bibverse{22} Sie
fielen aber auf ihr Angesicht und sprachen: Ach Gott, der du bist ein
Gott der Geister alles Fleisches, wenn ein Mann gesündigt hat, willst du
darum über die ganze Gemeinde wüten? \bibverse{23} Und der HErr redete
mit Mose und sprach: \bibverse{24} Sage der Gemeinde und sprich: Weichet
ringsherum von der Wohnung Korahs und Dathans und Abirams. \bibverse{25}
Und Mose stand auf und ging zu Dathan und Abiram, und die Ältesten
Israels folgten ihm nach, \bibverse{26} und er redete mit der Gemeinde
und sprach: Weichet von den Hütten dieser gottlosen Menschen und rührt
nichts an, was ihr ist, dass ihr nicht vielleicht umkommt in irgendeiner
ihrer Sünden. \bibverse{27} Und sie gingen hinweg von der Wohnung
Korahs, Dathans und Abirams. Dathan aber und Abiram gingen heraus und
traten an die Tür ihrer Hütten mit ihren Weibern und Söhnen und Kindern.
\bibverse{28} Und Mose sprach: Dabei sollt ihr merken, dass mich der
HErr gesandt hat, dass ich alle diese Werke täte, und nicht aus meinem
Herzen: \bibverse{29} werden sie sterben, wie alle Menschen sterben,
oder heimgesucht, wie alle Menschen heimgesucht werden, so hat mich der
HErr nicht gesandt; \bibverse{30} wird aber der HErr etwas Neues
schaffen, dass die Erde ihren Mund auftut und verschlingt sie mit allem,
was sie haben, dass sie lebendig hinunter in die Hölle fahren, so werdet
ihr erkennen, dass diese Leute den HErrn gelästert haben. \bibverse{31}
Und als er diese Worte hatte alle ausgeredet, zerriss die Erde unter
ihnen \footnote{\textbf{16:31} 5Mo 11,6} \bibverse{32} und tat ihren
Mund auf und verschlang sie mit ihren Häusern, mit allen Menschen, die
bei Korah waren, und mit aller ihrer Habe; \bibverse{33} und sie fuhren
hinunter lebendig in die Hölle mit allem, was sie hatten, und die Erde
deckte sie zu, und kamen um aus der Gemeinde. \bibverse{34} Und ganz
Israel, das um sie her war, floh vor ihrem Geschrei; denn sie sprachen:
dass uns die Erde nicht auch verschlinge! \bibverse{35} Dazu fuhr das
Feuer aus von dem HErrn und fraß die 250 Männer, die das Räuchwerk
opferten. \# 17 \bibverse{1} Und der HErr redete mit Mose und sprach:
\bibverse{2} Sage Eleasar, dem Sohn Aarons, des Priesters, dass er die
Pfannen aufhebe aus dem Brand und streue das Feuer hin und her;
\bibverse{3} denn die Pfannen solcher Sünder sind dem Heiligtum
verfallen durch ihre Seelen. Man schlage sie zu breiten Blechen, dass
man den Altar damit überziehe; denn sie sind geopfert vor dem HErrn und
geheiligt und sollen den Kindern Israel zum Zeichen sein. \bibverse{4}
Und Eleasar, der Priester, nahm die ehernen Pfannen, die die Verbrannten
geopfert hatten, und schlug sie zu Blechen, den Altar zu überziehen,
\bibverse{5} zum Gedächtnis der Kinder Israel, dass nicht jemand Fremdes
sich herzumache, der nicht ist des Samens Aarons, zu opfern Räuchwerk
vor dem HErrn, auf dass es ihm nicht gehe wie Korah und seiner Rotte,
wie der HErr ihm geredet hatte durch Mose. \footnote{\textbf{17:5} 4Mo
  1,51} \bibverse{6} Des anderen Morgens aber murrte die ganze Gemeinde
der Kinder Israel wider Mose und Aaron, und sprachen: Ihr habt des HErrn
Volk getötet. \bibverse{7} Und da sich die Gemeinde versammelte wider
Mose und Aaron, wandten sie sich zu der Hütte des Stifts. Und siehe, da
bedeckte es die Wolke, und die Herrlichkeit des HErrn erschien.
\bibverse{8} Und Mose und Aaron gingen herzu vor die Hütte des Stifts.
\bibverse{9} Und der HErr redete mit Mose und sprach: \bibverse{10} Hebt
euch aus dieser Gemeinde; ich will sie plötzlich vertilgen! Und sie
fielen auf ihr Angesicht. \footnote{\textbf{17:10} 4Mo 16,4; 4Mo 16,22}
\bibverse{11} Und Mose sprach zu Aaron: Nimm die Pfanne und tue Feuer
darein vom Altar und lege Räuchwerk darauf und gehe eilend zu der
Gemeinde und versöhne sie; denn das Wüten ist von dem HErrn ausgegangen,
und die Plage ist angegangen. \footnote{\textbf{17:11} 2Mo 28,38; 3Mo
  16,13} \bibverse{12} Und Aaron nahm, wie ihm Mose gesagt hatte, und
lief mitten unter die Gemeinde (und siehe, die Plage war angegangen
unter dem Volk) und räucherte und versöhnte das Volk \bibverse{13} und
stand zwischen den Toten und den Lebendigen. Da ward der Plage gewehrt.
\bibverse{14} Derer aber, die an der Plage gestorben waren, waren
14.700, ohne die, die mit Korah starben. \bibverse{15} Und Aaron kam
wieder zu Mose vor die Tür der Hütte des Stifts, und der Plage ward
gewehrt. \bibverse{16} Und der HErr redete mit Mose und sprach:
\bibverse{17} Sage den Kindern Israel und nimm von ihnen zwölf Stecken,
von jeglichem Fürsten seines Vaterhauses einen, und schreib eines
jeglichen Namen auf seinen Stecken. \bibverse{18} Aber den Namen Aarons
sollst du schreiben auf den Stecken Levis. Denn je für ein Haupt ihrer
Vaterhäuser soll ein Stecken sein. \bibverse{19} Und lege sie in die
Hütte des Stifts vor dem Zeugnis, da ich mich euch bezeuge. \footnote{\textbf{17:19}
  2Mo 25,22} \bibverse{20} Und welchen ich erwählen werde, des Stecken
wird grünen, dass ich das Murren der Kinder Israel, das sie wider euch
murren, stille. \bibverse{21} Mose redete mit den Kindern Israel, und
alle ihre Fürsten gaben ihm zwölf Stecken, ein jeglicher Fürst einen
Stecken, nach ihren Vaterhäusern; und der Stecken Aarons war auch unter
ihren Stecken. \bibverse{22} Und Mose legte die Stecken vor den HErrn in
der Hütte des Zeugnisses. \bibverse{23} Des Morgens aber, da Mose in die
Hütte des Zeugnisses ging, fand er den Stecken Aarons des Hauses Levi
grünen und die Blüte aufgegangen und Mandeln tragen. \bibverse{24} Und
Mose trug die Stecken alle heraus von dem HErrn vor alle Kinder Israel,
dass sie es sahen; und ein jeglicher nahm seinen Stecken. \bibverse{25}
Der HErr sprach aber zu Mose: Trage den Stecken Aarons wieder vor das
Zeugnis, dass er verwahrt werde zum Zeichen den ungehorsamen Kindern,
dass ihr Murren von mir aufhöre, dass sie nicht sterben. \bibverse{26}
Mose tat wie ihm der HErr geboten hatte. \bibverse{27} Und die Kinder
Israel sprachen zu Mose: Siehe, wir verderben und kommen um; wir werden
alle vertilgt und kommen um. \bibverse{28} Wer sich naht zu der Wohnung
des HErrn, der stirbt. Sollen wir denn ganz und gar untergehen?
\footnote{\textbf{17:28} 4Mo 17,5}

\hypertarget{section-3}{%
\section{18}\label{section-3}}

\bibverse{1} Und der HErr sprach zu Aaron: Du und deine Söhne und deines
Vaters Haus mit dir sollt die Missetat des Heiligtums tragen; und du und
deine Söhne mit dir sollt die Missetat eures Priestertums tragen.
\footnote{\textbf{18:1} 2Mo 28,38; 3Mo 16,32-33} \bibverse{2} Aber deine
Brüder des Stammes Levis, deines Vaters, sollst du zu dir nehmen, dass
sie bei dir seien und dir dienen; du aber und deine Söhne mit dir vor
der Hütte des Zeugnisses. \footnote{\textbf{18:2} 4Mo 3,6-10}
\bibverse{3} Und sie sollen deines Dienstes und des Dienstes der ganzen
Hütte warten. Doch zu dem Gerät des Heiligtums und zu dem Altar sollen
sie sich nicht nahen, dass nicht beide, sie und ihr, sterbet;
\bibverse{4} sondern sie sollen bei dir sein, dass sie des Dienstes
warten an der Hütte des Stifts in allem Amt der Hütte; und kein Fremder
soll sich zu euch tun. \bibverse{5} So wartet nun des Dienstes des
Heiligtums und des Dienstes des Altars, dass hinfort nicht mehr ein
Wüten komme über die Kinder Israel. \bibverse{6} Denn siehe, ich habe
die Leviten, eure Brüder, genommen aus den Kindern Israel, dem HErrn zum
Geschenk, und euch gegeben, dass sie des Amts pflegen an der Hütte des
Stifts. \footnote{\textbf{18:6} 4Mo 3,12; 4Mo 3,45} \bibverse{7} Du aber
und deine Söhne mit dir sollt eures Priestertums warten, dass ihr dienet
in allerlei Geschäft des Altars und inwendig hinter dem Vorhang; denn
euer Priestertum gebe ich euch zum Amt, zum Geschenk. Wenn ein Fremder
sich herzutut, der soll sterben. \footnote{\textbf{18:7} 4Mo 1,51}
\bibverse{8} Und der HErr sagte zu Aaron: Siehe, ich habe dir gegeben
meine Hebopfer von allem, was die Kinder Israel heiligen, als Gebühr dir
und deinen Söhnen zum ewigen Recht. \footnote{\textbf{18:8} 3Mo 2,3; 3Mo
  2,10; 3Mo 6,9-11; 3Mo 6,19-22; 3Mo 7,6-10} \bibverse{9} Das sollst du
haben von dem Hochheiligen: was nicht angezündet wird von allen ihren
Gaben an allen ihren Speisopfern und an allen ihren Sündopfern und an
allen ihren Schuldopfern, die sie mir geben, das soll dir und deinen
Söhnen ein Hochheiliges sein. \bibverse{10} An einem hochheiligen Ort
sollst du es essen. Was männlich ist, soll davon essen; denn es soll dir
heilig sein. \bibverse{11} Ich habe auch das Hebopfer ihrer Gabe an
allen Webeopfern der Kinder Israel dir gegeben und deinen Söhnen und
Töchtern samt dir zum ewigen Recht; wer rein ist in deinem Hause, soll
davon essen. \bibverse{12} Alles beste Öl und alles Beste vom Most und
Korn, nämlich ihre Erstlinge, die sie dem HErrn geben, habe ich dir
gegeben. \bibverse{13} Die erste Frucht, die sie dem HErrn bringen von
allem, was in ihrem Lande ist, soll dein sein; wer rein ist in deinem
Hause, soll davon essen. \footnote{\textbf{18:13} 2Mo 23,19; 5Mo 18,4}
\bibverse{14} Alles Verbannte in Israel soll dein sein. \footnote{\textbf{18:14}
  3Mo 27,28} \bibverse{15} Alles, was die Mutter bricht unter allem
Fleisch, das sie dem HErrn bringen, es sei ein Mensch oder Vieh, soll
dein sein; doch dass du die erste Menschenfrucht lösen lassest und die
erste Frucht eines unreinen Viehs auch lösen lassest. \footnote{\textbf{18:15}
  2Mo 13,12-13; 2Mo 34,19-20} \bibverse{16} Sie sollen's aber lösen,
wenn's einen Monat alt ist; und sollst es zu lösen geben um Geld, um
fünf Silberlinge nach dem Lot des Heiligtums, das hat 20 Gera.
\bibverse{17} Aber die erste Frucht eines Rindes oder Schafes oder einer
Ziege sollst du nicht zu lösen geben, denn sie sind heilig; ihr Blut
sollst du sprengen auf den Altar, und ihr Fett sollst du anzünden zum
Opfer des süßen Geruchs dem HErrn. \bibverse{18} Ihr Fleisch soll dein
sein, wie auch die Webebrust und die rechte Schulter dein ist.
\bibverse{19} Alle Hebopfer, die die Kinder Israel heiligen dem HErrn,
habe ich dir gegeben und deinen Söhnen und deinen Töchtern samt dir zum
ewigen Recht. Das soll ein unverweslicher Bund sein ewig vor dem HErrn,
dir und deinem Samen samt dir. \bibverse{20} Und der HErr sprach zu
Aaron: Du sollst in ihrem Lande nichts besitzen, auch kein Teil unter
ihnen haben; denn ich bin dein Teil und dein Erbgut unter den Kindern
Israel. \bibverse{21} Den Kindern Levi aber habe ich alle Zehnten
gegeben in Israel zum Erbgut für ihr Amt, das sie mir tun an der Hütte
des Stifts. \footnote{\textbf{18:21} 3Mo 27,30} \bibverse{22} Dass
hinfort die Kinder Israel nicht zur Hütte des Stifts sich tun, Sünde auf
sich zu laden, und sterben; \bibverse{23} sondern die Leviten sollen des
Amts pflegen an der Hütte des Stifts, und sie sollen jener Missetat
tragen zu ewigem Recht bei euren Nachkommen. Und sie sollen unter den
Kindern Israel kein Erbgut besitzen; \bibverse{24} Denn den Zehnten der
Kinder Israel, den sie dem HErrn heben, habe ich den Leviten zum Erbgut
gegeben. Darum habe ich zu ihnen gesagt, dass sie unter den Kindern
Israel kein Erbgut besitzen sollen. \bibverse{25} Und der HErr redete
mit Mose und sprach: \bibverse{26} Sage den Leviten und sprich zu ihnen:
Wenn ihr den Zehnten nehmt von den Kindern Israel, den ich euch von
ihnen gegeben habe zu eurem Erbgut, so sollt ihr davon ein Hebopfer dem
HErrn tun, je den Zehnten von dem Zehnten; \bibverse{27} und sollt solch
euer Hebopfer achten, als gäbet ihr Korn aus der Scheune und Fülle aus
der Kelter. \bibverse{28} Also sollt auch ihr das Hebopfer dem HErrn
geben von allen euren Zehnten, die ihr nehmt von den Kindern Israel,
dass ihr solches Hebopfer des HErrn dem Priester Aaron gebet.
\bibverse{29} Von allem, was euch gegeben wird, sollt ihr dem HErrn
allerlei Hebopfer geben, von allem Besten das, was davon geheiligt wird.
\bibverse{30} Und sprich zu ihnen: Wenn ihr also das Beste davon hebt,
so soll's den Leviten gerechnet werden wie ein Einkommen der Scheune und
wie ein Einkommen der Kelter. \bibverse{31} Ihr möget's essen an allen
Stätten, ihr und eure Kinder; denn es ist euer Lohn für euer Amt in der
Hütte des Stifts. \bibverse{32} So werdet ihr nicht Sünde auf euch laden
an demselben, wenn ihr das Beste davon hebt, und nicht entweihen das
Geheiligte der Kinder Israel und nicht sterben. \# 19 \bibverse{1} Und
der HErr redete mit Mose und Aaron und sprach: \bibverse{2} Diese Weise
soll ein Gesetz sein, das der HErr geboten hat und gesagt: Sage den
Kindern Israel, dass sie zu dir führen eine rötliche Kuh ohne Gebrechen,
an der kein Fehl sei und auf die noch nie ein Joch gekommen ist.
\footnote{\textbf{19:2} Hebr 9,13; 3Mo 22,20} \bibverse{3} Und gebt sie
dem Priester Eleasar; der soll sie hinaus vor das Lager führen und
daselbst vor ihm schlachten lassen. \bibverse{4} Und Eleasar, der
Priester, soll von ihrem Blut mit seinem Finger nehmen und stracks gegen
die Hütte des Stifts siebenmal sprengen \bibverse{5} und die Kuh vor ihm
verbrennen lassen, beides, ihr Fell und ihr Fleisch, dazu ihr Blut samt
ihrem Mist. \bibverse{6} Und der Priester soll Zedernholz und Isop und
scharlachrote Wolle nehmen und auf die brennende Kuh werfen \footnote{\textbf{19:6}
  3Mo 14,6} \bibverse{7} und soll seine Kleider waschen und seinen Leib
mit Wasser baden und darnach ins Lager gehen und unrein sein bis an den
Abend. \footnote{\textbf{19:7} 3Mo 16,28} \bibverse{8} Und der sie
verbrannt hat, soll auch seine Kleider mit Wasser waschen und seinen
Leib in Wasser baden und unrein sein bis an den Abend. \bibverse{9} Und
ein reiner Mann soll die Asche von der Kuh aufraffen und sie schütten
draußen vor dem Lager an eine reine Stätte, dass sie daselbst verwahrt
werde für die Gemeinde der Kinder Israel zum Sprengwasser; denn es ist
ein Sündopfer. \bibverse{10} Und derselbe, der die Asche der Kuh
aufgerafft hat, soll seine Kleider waschen und unrein sein bis an den
Abend. Dies soll ein ewiges Recht sein den Kindern Israel und den
Fremdlingen, die unter euch wohnen: \bibverse{11} Wer nun irgendeinen
toten Menschen anrührt, der wird sieben Tage unrein sein. \bibverse{12}
Der soll sich hiermit entsündigen am dritten Tage und am siebenten Tage,
so wird er rein; und wo er sich nicht am dritten Tage und am siebenten
Tage entsündigt, so wird er nicht rein werden. \bibverse{13} Wenn aber
jemand irgendeinen toten Menschen anrührt und sich nicht entsündigen
wollte, der verunreinigt die Wohnung des HErrn, und solche Seele soll
ausgerottet werden aus Israel. Darum dass das Sprengwasser nicht über
ihn gesprengt ist, so ist er unrein; seine Unreinigkeit bleibt an ihm.
\footnote{\textbf{19:13} 3Mo 15,31} \bibverse{14} Das ist das Gesetz:
Wenn ein Mensch in der Hütte stirbt, soll jeder, der in die Hütte geht
und wer in der Hütte ist, unrein sein sieben Tage. \bibverse{15} Und
alles offene Gerät, das keinen Deckel noch Band hat, ist unrein.
\bibverse{16} Auch wer anrührt auf dem Felde einen, der erschlagen ist
mit dem Schwert, oder einen Toten oder eines Menschen Gebein oder ein
Grab, der ist unrein sieben Tage. \bibverse{17} So sollen sie nun für
den Unreinen nehmen Asche von diesem verbrannten Sündopfer und
fließendes Wasser darauf tun in ein Gefäß. \bibverse{18} Und ein reiner
Mann soll Isop nehmen und ins Wasser tauchen und die Hütte besprengen
und alle Geräte und alle Seelen, die darin sind; also auch den, der
eines Toten Gebein oder einen Erschlagenen oder Toten oder ein Grab
angerührt hat. \bibverse{19} Es soll aber der Reine den Unreinen am
dritten Tage und am siebenten Tage besprengen und ihn am siebenten Tage
entsündigen; und er soll seine Kleider waschen und sich im Wasser baden,
so wird er am Abend rein. \bibverse{20} Welcher aber unrein sein wird
und sich nicht entsündigen will, des Seele soll ausgerottet werden aus
der Gemeinde; denn er hat das Heiligtum des HErrn verunreinigt und ist
mit Sprengwasser nicht besprengt; darum ist er unrein. \bibverse{21} Und
dies soll ihnen ein ewiges Recht sein. Und der auch, der mit dem
Sprengwasser gesprengt hat, soll seine Kleider waschen; und wer das
Sprengwasser anrührt, der soll unrein sein bis an den Abend.
\bibverse{22} Und alles, was der Unreine anrührt, wird unrein werden;
und welche Seele ihn anrühren wird, soll unrein sein bis an den Abend.
\# 20 \bibverse{1} Und die Kinder Israel kamen mit der ganzen Gemeinde
in die Wüste Zin im ersten Monat, und das Volk lag zu Kades. Und Mirjam
starb daselbst und ward daselbst begraben. \bibverse{2} Und die Gemeinde
hatte kein Wasser, und sie versammelten sich wider Mose und Aaron.
\footnote{\textbf{20:2} 2Mo 17,1-7} \bibverse{3} Und das Volk haderte
mit Mose und sprach: Ach, dass wir umgekommen wären, da unsere Brüder
umkamen vor dem HErrn! \bibverse{4} Warum habt ihr die Gemeinde des
HErrn in diese Wüste gebracht, dass wir hier sterben mit unserem Vieh?
\bibverse{5} Und warum habt ihr uns aus Ägypten geführt an diesen bösen
Ort, da man nicht säen kann, da weder Feigen noch Weinstöcke noch
Granatäpfel sind und dazu kein Wasser zu trinken? \bibverse{6} Mose und
Aaron gingen von der Gemeinde zur Tür der Hütte des Stifts und fielen
auf ihr Angesicht, und die Herrlichkeit des HErrn erschien ihnen.
\bibverse{7} Und der HErr redete mit Mose und sprach: \bibverse{8} Nimm
den Stab und versammle die Gemeinde, du und dein Bruder Aaron, und redet
mit dem Fels vor ihren Augen; der wird sein Wasser geben. Also sollst du
ihnen Wasser aus dem Fels bringen und die Gemeinde tränken und ihr Vieh.
\bibverse{9} Da nahm Mose den Stab vor dem HErrn, wie er ihm geboten
hatte. \bibverse{10} Und Mose und Aaron versammelten die Gemeinde vor
den Fels, und er sprach zu ihnen: Höret, ihr Ungehorsamen, werden wir
euch auch Wasser bringen aus diesem Fels? \footnote{\textbf{20:10} Ps
  106,33} \bibverse{11} Und Mose hob seine Hand auf und schlug den Fels
mit dem Stab zweimal. Da ging viel Wasser heraus, dass die Gemeinde
trank und ihr Vieh. \bibverse{12} Der HErr aber sprach zu Mose und
Aaron: Darum dass ihr nicht an mich geglaubt habt, mich zu heiligen vor
den Kindern Israel, sollt ihr diese Gemeinde nicht in das Land bringen,
das ich ihnen geben werde. \bibverse{13} Das ist das Haderwasser,
darüber die Kinder Israel mit dem HErrn haderten und er geheiligt ward
an ihnen. \footnote{\textbf{20:13} Ps 81,8} \bibverse{14} Und Mose
sandte Botschaft aus Kades zu dem König der Edomiter: Also lässt dir
dein Bruder Israel sagen: Du weißt alle die Mühsal, die uns betroffen
hat, \footnote{\textbf{20:14} 1Mo 32,4; Ri 11,17; 5Mo 23,8}
\bibverse{15} dass unsere Väter nach Ägypten hinabgezogen sind und wir
lange Zeit in Ägypten gewohnt haben, und die Ägypter behandelten uns und
unsere Väter übel. \bibverse{16} Und wir schrien zu dem HErrn; der hat
unsere Stimme erhört und einen Engel gesandt und uns aus Ägypten
geführt. Und siehe, wir sind zu Kades, in der Stadt an deinen Grenzen.
\footnote{\textbf{20:16} 2Mo 23,20} \bibverse{17} Lass uns durch dein
Land ziehen. Wir wollen nicht durch Äcker noch Weinberge gehen, auch
nicht Wasser aus den Brunnen trinken; die Landstraße wollen wir ziehen,
weder zur Rechten noch zur Linken weichen, bis wir durch deine Grenze
kommen. \footnote{\textbf{20:17} 4Mo 21,22} \bibverse{18} Edom aber
sprach zu ihnen: Du sollst nicht durch mich ziehen, oder ich will dir
mit dem Schwert entgegenziehen. \bibverse{19} Die Kinder Israel sprachen
zu ihm: Wir wollen auf der gebahnten Straße ziehen, und so wir von
deinem Wasser trinken, wir und unser Vieh, so wollen wir's bezahlen; wir
wollen nichts denn nur zu Fuße hindurchziehen. \bibverse{20} Er aber
sprach: Du sollst nicht herdurchziehen. Und die Edomiter zogen aus,
ihnen entgegen, mit mächtigem Volk und starker Hand. \bibverse{21} Also
weigerten sich die Edomiter, Israel zu vergönnen, durch ihr Gebiet zu
ziehen. Und Israel wich von ihnen. \bibverse{22} Und die Kinder Israel
brachen auf von Kades und kamen mit der ganzen Gemeinde an den Berg Hor.
\bibverse{23} Und der HErr redete mit Mose und Aaron am Berge Hor, an
den Grenzen des Landes der Edomiter, und sprach: \bibverse{24} Lass sich
Aaron sammeln zu seinem Volk; denn er soll nicht in das Land kommen, das
ich den Kindern Israel gegeben habe, darum dass ihr meinem Munde
ungehorsam gewesen seid bei dem Haderwasser. \bibverse{25} Nimm aber
Aaron und seinen Sohn Eleasar und führe sie auf den Berg Hor
\bibverse{26} und zieh Aaron seine Kleider aus und zieh sie Eleasar an,
seinem Sohne. Und Aaron soll sich daselbst sammeln und sterben.
\footnote{\textbf{20:26} 3Mo 21,10} \bibverse{27} Da tat Mose, wie ihm
der HErr geboten hatte, und sie stiegen auf den Berg Hor vor der ganzen
Gemeinde. \bibverse{28} Und Mose zog Aaron seine Kleider aus und zog sie
Eleasar an, seinem Sohne. Und Aaron starb daselbst oben auf dem Berge.
Mose aber und Eleasar stiegen herab vom Berge. \bibverse{29} Und da die
ganze Gemeinde sah, dass Aaron dahin war, beweinten sie ihn 30 Tage, das
ganze Haus Israel. \# 21 \bibverse{1} Und da der Kanaaniter, der König
von Arad, der gegen Mittag wohnte, hörte, dass Israel hereinkommt durch
den Weg der Kundschafter, stritt er wider Israel und führte etliche
gefangen. \bibverse{2} Da gelobte Israel dem HErrn ein Gelübde und
sprach: Wenn du dieses Volk unter meine Hand gibst, so will ich ihre
Städte verbannen. \footnote{\textbf{21:2} 5Mo 13,16; Jos 6,17; Ri 1,17;
  1Sam 15,3} \bibverse{3} Und der HErr erhörte die Stimme Israels und
gab die Kanaaniter, und sie verbannten sie samt ihren Städten und hießen
die Stätte Horma. \bibverse{4} Da zogen sie von dem Berge Hor auf dem
Wege gegen das Schilfmeer, dass sie um der Edomiter Land hinzögen. Und
das Volk ward verdrossen auf dem Wege \bibverse{5} und redete wider Gott
und wider Mose: Warum hast du uns aus Ägypten geführt, dass wir sterben
in der Wüste? Denn es ist kein Brot noch Wasser hier, und unsere Seele
ekelt vor dieser mageren Speise. \bibverse{6} Da sandte der HErr feurige
Schlangen unter das Volk; die bissen das Volk, dass viel Volks in Israel
starb. \footnote{\textbf{21:6} 1Kor 10,9} \bibverse{7} Da kamen sie zu
Mose und sprachen: Wir haben gesündigt, dass wir wider den HErrn und
wider dich geredet haben; bitte den HErrn, dass er die Schlangen von uns
nehme. Mose bat für das Volk. \bibverse{8} Da sprach der HErr zu Mose:
Mache dir eine eherne Schlange und richte sie zum Zeichen auf; wer
gebissen ist und sieht sie an, der soll leben. \bibverse{9} Da machte
Mose eine eherne Schlange und richtete sie auf zum Zeichen; und wenn
jemanden eine Schlange biss, so sah er die eherne Schlange an und blieb
leben. \bibverse{10} Und die Kinder Israel zogen aus und lagerten sich
in Oboth. \bibverse{11} Und von Oboth zogen sie aus und lagerten sich in
Ije-Abarim, in der Wüste Moab gegenüber gegen der Sonne Aufgang.
\bibverse{12} Und von da zogen sie und lagerten sich am Bach Sered.
\bibverse{13} Von da zogen sie und lagerten sich diesseits am Arnon, der
in der Wüste ist und herauskommt von der Grenze der Amoriter; denn der
Arnon ist die Grenze Moabs zwischen Moab und den Amoritern.
\bibverse{14} Daher heißt es in dem Buch von den Kriegen des HErrn: „Das
Vaheb in Supha und die Bäche Arnon \bibverse{15} und die Quelle der
Bäche, welche reicht hinan zur Stadt Ar und lenkt sich und ist die
Grenze Moabs.`` \bibverse{16} Und von da zogen sie zum Brunnen. Das ist
der Brunnen, davon der HErr zu Mose sagte: Sammle das Volk, ich will
ihnen Wasser geben. \bibverse{17} Da sang Israel das Lied: „Brunnen,
steige auf! Singet von ihm! \bibverse{18} Das ist der Brunnen, den die
Fürsten gegraben haben; die Edlen im Volk haben ihn gegraben mit dem
Zepter, mit ihren Stäben.`` Und von dieser Wüste zogen sie gen Matthana;
\bibverse{19} und von Matthana gen Nahaliel; und von Nahaliel gen
Bamoth; \bibverse{20} und von Bamoth in das Tal, das im Felde Moab
liegt, zu dem hohen Berge Pisga, der gegen die Wüste sieht.
\bibverse{21} Und Israel sandte Boten zu Sihon, dem König der Amoriter,
und ließ ihm sagen: \footnote{\textbf{21:21} 5Mo 2,26-37} \bibverse{22}
Lass mich durch dein Land ziehen. Wir wollen nicht weichen in die Äcker
noch in die Weingärten, wollen auch Brunnenwasser nicht trinken; die
Landstraße wollen wir ziehen, bis wir durch deine Grenze kommen.
\bibverse{23} Aber Sihon gestattete den Kindern Israel nicht den Zug
durch sein Gebiet, sondern sammelte all sein Volk und zog aus, Israel
entgegen in die Wüste; und als er gen Jahza kam, stritt er wider Israel.
\bibverse{24} Israel aber schlug ihn mit der Schärfe des Schwerts und
nahm sein Land ein vom Arnon an bis an den Jabbok und bis an die Kinder
Ammon; denn die Grenzen der Kinder Ammon waren fest. \bibverse{25} Also
nahm Israel alle diese Städte und wohnte in allen Städten der Amoriter,
zu Hesbon und in allen seinen Ortschaften. \bibverse{26} Denn Hesbon war
die Stadt Sihons, des Königs der Amoriter, und er hatte zuvor mit dem
König der Moabiter gestritten und ihm all sein Land abgewonnen bis zum
Arnon. \bibverse{27} Daher sagt man im Lied: „Kommt gen Hesbon, dass man
die Stadt Sihons baue und aufrichte; \bibverse{28} denn Feuer ist aus
Hesbon gefahren, eine Flamme von der Stadt Sihons, die hat gefressen Ar
der Moabiter und die Bürger der Höhen am Arnon. \bibverse{29} Weh dir,
Moab! Du Volk des Kamos bist verloren; man hat seine Söhne in die Flucht
geschlagen und seine Töchter gefangen geführt Sihon, dem König der
Amoriter. \bibverse{30} Ihre Herrlichkeit ist zunichte worden von Hesbon
bis gen Dibon; sie ist verstört bis gen Nophah, die da langt bis gen
Medeba.`` \bibverse{31} Also wohnte Israel im Lande der Amoriter.
\bibverse{32} Und Mose sandte aus Kundschafter gen Jaser, und sie
gewannen seine Ortschaften und nahmen die Amoriter ein, die darin waren,
\bibverse{33} und wandten sich und zogen hinauf den Weg nach Basan. Da
zog aus, ihnen entgegen, Og, der König von Basan, mit allem seinem Volk,
zu streiten in Edrei. \bibverse{34} Und der HErr sprach zu Mose: Fürchte
dich nicht vor ihm; denn ich habe ihn in deine Hand gegeben mit Land und
Leuten, und du sollst mit ihm tun, wie du mit Sihon, dem König der
Amoriter, getan hast, der zu Hesbon wohnte. \footnote{\textbf{21:34} Ps
  136,17-22} \bibverse{35} Und sie schlugen ihn und seine Söhne und all
sein Volk, bis dass keiner übrigblieb, und nahmen das Land ein. \# 22
\bibverse{1} Darnach zogen die Kinder Israel und lagerten sich in das
Gefilde Moab jenseits des Jordans, gegenüber Jericho. \bibverse{2} Und
Balak, der Sohn Zippors, sah alles, was Israel getan hatte den
Amoritern; \bibverse{3} und die Moabiter fürchteten sich sehr vor dem
Volk, das so groß war, und den Moabitern graute vor den Kindern Israel;
\bibverse{4} und sie sprachen zu den Ältesten der Midianiter: Nun wird
dieser Haufe auffressen, was um uns ist, wie ein Ochse Kraut auf dem
Felde auffrisst. Balak aber, der Sohn Zippors, war zu der Zeit König der
Moabiter. \bibverse{5} Und er sandte Boten aus zu Bileam, dem Sohn
Beors, gen Pethor, der wohnte an dem Strom im Lande der Kinder seines
Volks, dass sie ihn forderten, und ließ ihm sagen: Siehe, es ist ein
Volk aus Ägypten gezogen, das bedeckt das Angesicht der Erde und liegt
mir gegenüber. \bibverse{6} So komm nun und verfluche mir das Volk (denn
es ist mir zu mächtig), ob ich's schlagen möchte und aus dem Lande
vertreiben; denn ich weiß, dass, welchen du segnest, der ist gesegnet,
und welchen du verfluchst, der ist verflucht. \bibverse{7} Und die
Ältesten der Moabiter gingen hin mit den Ältesten der Midianiter und
hatten den Lohn des Wahrsagers in ihren Händen und kamen zu Bileam und
sagten ihm die Worte Balaks. \footnote{\textbf{22:7} 2Petr 2,15}
\bibverse{8} Und er sprach zu ihnen: Bleibt hier über Nacht, so will ich
euch wieder sagen, wie mir der HErr sagen wird. Also blieben die Fürsten
der Moabiter bei Bileam. \bibverse{9} Und Gott kam zu Bileam und sprach:
Wer sind die Leute, die bei dir sind? \bibverse{10} Bileam sprach zu
Gott: Balak, der Sohn Zippors, der Moabiter König, hat zu mir gesandt:
\bibverse{11} Siehe, ein Volk ist aus Ägypten gezogen und bedeckt das
Angesicht der Erde; so komm nun und fluche ihm, ob ich mit ihm streiten
möge und sie vertreiben. \bibverse{12} Gott aber sprach zu Bileam: Gehe
nicht mit ihnen, verfluche das Volk auch nicht; denn es ist gesegnet.
\bibverse{13} Da stand Bileam des Morgens auf und sprach zu den Fürsten
Balaks: Gehet hin in euer Land; denn der HErr will's nicht gestatten,
dass ich mit euch ziehe. \bibverse{14} Und die Fürsten der Moabiter
machten sich auf, kamen zu Balak und sprachen: Bileam weigert sich, mit
uns zu ziehen. \bibverse{15} Da sandte Balak noch größere und
herrlichere Fürsten, denn jene waren. \bibverse{16} Da die zu Bileam
kamen, sprachen sie zu ihm: Also lässt dir sagen Balak, der Sohn
Zippors: Wehre dich doch nicht, zu mir zu ziehen; \bibverse{17} denn ich
will dich hoch ehren, und was du mir sagst, das will ich tun; komm doch
und fluche mir diesem Volk. \bibverse{18} Bileam antwortete und sprach
zu den Dienern Balaks: Wenn mir Balak sein Haus voll Silber und Gold
gäbe, so könnte ich doch nicht übertreten das Wort des HErrn, meines
Gottes, Kleines oder Großes zu tun. \bibverse{19} So bleibt doch nur
hier auch ihr diese Nacht, dass ich erfahre, was der HErr weiter mit mir
reden werde. \bibverse{20} Da kam Gott des Nachts zu Bileam und sprach
zu ihm: Sind die Männer gekommen, dich zu rufen, so mache dich auf und
zieh mit ihnen; doch was ich dir sagen werde, sollst du tun.
\bibverse{21} Da stand Bileam des Morgens auf und sattelte seine Eselin
und zog mit den Fürsten der Moabiter. \bibverse{22} Aber der Zorn Gottes
ergrimmte, dass er hinzog. Und der Engel des HErrn trat in den Weg, dass
er ihm widerstünde. Er aber ritt auf seiner Eselin, und zwei Knechte
waren mit ihm. \bibverse{23} Und die Eselin sah den Engel des HErrn im
Wege stehen und ein bloßes Schwert in seiner Hand. Und die Eselin wich
aus dem Wege und ging auf dem Felde; Bileam aber schlug sie, dass sie in
den Weg sollte gehen. \footnote{\textbf{22:23} 1Mo 3,24; Jos 5,13}
\bibverse{24} Da trat der Engel des HErrn in den Pfad bei den
Weinbergen, da auf beiden Seiten Wände waren. \bibverse{25} Und da die
Eselin den Engel des HErrn sah, drängte sie sich an die Wand und klemmte
Bileam den Fuß an der Wand; und er schlug sie noch mehr. \bibverse{26}
Da ging der Engel des HErrn weiter und trat an einen engen Ort, da kein
Weg war zu weichen, weder zur Rechten noch zur Linken. \bibverse{27} Und
da die Eselin den Engel des HErrn sah, fiel sie auf ihre Knie unter
Bileam. Da ergrimmte der Zorn Bileams, und er schlug die Eselin mit dem
Stabe. \bibverse{28} Da tat der HErr der Eselin den Mund auf, und sie
sprach zu Bileam: Was habe ich dir getan, dass du mich geschlagen hast
nun dreimal? \bibverse{29} Bileam sprach zur Eselin: dass du mich
höhnest! ach, dass ich jetzt ein Schwert in der Hand hätte, ich wollte
dich erwürgen! \bibverse{30} Die Eselin sprach zu Bileam: Bin ich nicht
deine Eselin, darauf du geritten bist zu deiner Zeit bis auf diesen Tag?
Habe ich auch je gepflegt, dir also zu tun? Er sprach: Nein.
\bibverse{31} Da öffnete der HErr dem Bileam die Augen, dass er den
Engel des HErrn sah im Wege stehen und ein bloßes Schwert in seiner
Hand, und er neigte und bückte sich mit seinem Angesicht. \bibverse{32}
Und der Engel des HErrn sprach zu ihm: Warum hast du deine Eselin
geschlagen nun dreimal? Siehe, ich bin ausgegangen, dass ich dir
widerstehe; denn dein Weg ist vor mir verkehrt. \bibverse{33} Und die
Eselin hat mich gesehen und ist mir dreimal gewichen; sonst, wo sie
nicht vor mir gewichen wäre, so wollte ich dich auch jetzt erwürgt und
die Eselin lebendig erhalten haben. \bibverse{34} Da sprach Bileam zu
dem Engel des HErrn: Ich habe gesündigt; denn ich habe es nicht gewusst,
dass du mir entgegenstandest im Wege. Und nun, so dir's nicht gefällt,
will ich wieder umkehren. \bibverse{35} Der Engel des HErrn sprach zu
ihm: Zieh hin mit den Männern; aber nichts anderes, denn was ich dir
sagen werde, sollst du reden. Also zog Bileam mit den Fürsten Balaks.
\bibverse{36} Da Balak hörte, dass Bileam kam, zog er aus ihm entgegen
in die Stadt der Moabiter, die da liegt an der Grenze des Arnon, welcher
ist an der äußersten Grenze, \bibverse{37} und sprach zu ihm: Habe ich
nicht zu dir gesandt und dich fordern lassen? Warum bist du denn nicht
zu mir gekommen? Meinst du, ich könnte dich nicht ehren? \bibverse{38}
Bileam antwortete ihm: Siehe, ich bin gekommen zu dir; aber wie kann ich
etwas anderes reden, als was mir Gott in den Mund gibt? Das muss ich
reden. \bibverse{39} Also zog Bileam mit Balak, und sie kamen in die
Gassenstadt. \bibverse{40} Und Balak opferte Rinder und Schafe und
sandte davon an Bileam und an die Fürsten, die bei ihm waren.
\bibverse{41} Und des Morgens nahm Balak den Bileam und führte ihn hin
auf die Höhe Baals, dass er von da sehen konnte das Ende des Volks.
\footnote{\textbf{22:41} 4Mo 23,28}

\hypertarget{section-4}{%
\section{23}\label{section-4}}

\bibverse{1} Und Bileam sprach zu Balak: Baue mir hier sieben Altäre und
schaffe mir her sieben Farren und sieben Widder. \bibverse{2} Balak tat,
wie ihm Bileam sagte; und beide, Balak und Bileam, opferten je auf einem
Altar einen Farren und einen Widder. \bibverse{3} Und Bileam sprach zu
Balak: Tritt zu deinem Brandopfer; ich will hingehen, ob vielleicht mir
der HErr begegne, dass ich dir ansage, was er mir zeigt. Und ging hin
eilend. \bibverse{4} Und Gott begegnete Bileam; er aber sprach zu ihm:
Sieben Altäre habe ich zugerichtet und je auf einem Altar einen Farren
und einen Widder geopfert. \bibverse{5} Der HErr aber gab das Wort dem
Bileam in den Mund und sprach: Gehe wieder zu Balak und rede also.
\bibverse{6} Und da er wieder zu ihm kam, siehe, da stand er bei seinem
Brandopfer samt allen Fürsten der Moabiter. \bibverse{7} Da hob er an
seinen Spruch und sprach: Aus Syrien hat mich Balak, der Moabiter König,
holen lassen von dem Gebirge gegen Aufgang: Komm, verfluche mir Jakob!
komm schilt Israel! \bibverse{8} Wie soll ich fluchen, dem Gott nicht
flucht? Wie soll ich schelten, den der HErr nicht schilt? \bibverse{9}
Denn von der Höhe der Felsen sehe ich ihn wohl, und von den Hügeln
schaue ich ihn. Siehe, das Volk wird besonders wohnen und nicht unter
die Heiden gerechnet werden. \bibverse{10} Wer kann zählen den Staub
Jakobs und die Zahl des vierten Teils Israels? Meine Seele müsse sterben
des Todes der Gerechten, und mein Ende werde wie dieser Ende!
\bibverse{11} Da sprach Balak zu Bileam: Was tust du an mir? Ich habe
dich holen lassen, zu fluchen meinen Feinden; und siehe, du segnest.
\bibverse{12} Er antwortete und sprach: Muss ich das nicht halten und
reden, was mir der HErr in den Mund gibt? \footnote{\textbf{23:12} 4Mo
  22,38} \bibverse{13} Balak sprach zu ihm: Komm doch mit mir an einen
anderen Ort, von wo du nur sein Ende sehest und es nicht ganz sehest,
und fluche mir ihm daselbst. \bibverse{14} Und er führte ihn auf einen
freien Platz auf der Höhe Pisga und baute sieben Altäre und opferte je
auf einem Altar einen Farren und einen Widder. \bibverse{15} Und
(Bileam) sprach zu Balak: Tritt her zu deinem Brandopfer; ich will dort
warten. \bibverse{16} Und der HErr begegnete Bileam und gab ihm das Wort
in seinen Mund und sprach: Gehe wieder zu Balak und rede also.
\bibverse{17} Und da er wieder zu ihm kam, siehe, da stand er bei seinem
Brandopfer samt den Fürsten der Moabiter. Und Balak sprach zu ihm: Was
hat der HErr gesagt? \bibverse{18} Und er hob an seinen Spruch und
sprach: Stehe auf, Balak, und höre! nimm zu Ohren was ich sage, du Sohn
Zippors! \bibverse{19} Gott ist nicht ein Mensch, dass er lüge, noch ein
Menschenkind, dass ihn etwas gereue. Sollte er etwas sagen und nicht
tun? Sollte er etwas reden und nicht halten? \bibverse{20} Siehe, zu
segnen bin ich hergebracht; er segnet, und ich kann's nicht wenden.
\bibverse{21} Man sieht keine Mühe in Jakob und keine Arbeit in Israel.
Der HErr, sein Gott, ist bei ihm und das Drommeten des Königs unter ihm.
\bibverse{22} Gott hat sie aus Ägypten geführt; seine Freudigkeit ist
wie eines Einhorns. \bibverse{23} Denn es ist kein Zauberer in Jakob und
kein Wahrsager in Israel. Zu seiner Zeit wird Jakob gesagt und Israel,
was Gott tut. \bibverse{24} Siehe, das Volk wird aufstehen, wie ein
junger Löwe und wird sich erheben wie ein Löwe; es wird sich nicht
legen, bis es den Raub fresse und das Blut der Erschlagenen saufe.
\footnote{\textbf{23:24} 4Mo 24,9} \bibverse{25} Da sprach Balak zu
Bileam: Du sollst ihm weder fluchen noch es segnen. \bibverse{26} Bileam
antwortete und sprach zu Balak: Habe ich dir nicht gesagt, alles, was
der HErr reden würde, das würde ich tun? \bibverse{27} Balak sprach zu
ihm: Komm doch, ich will dich an einen anderen Ort führen, ob's
vielleicht Gott gefalle, dass du daselbst mir sie verfluchest.
\bibverse{28} Und er führte ihn auf die Höhe des Berges Peor, welcher
gegen die Wüste sieht. \footnote{\textbf{23:28} 4Mo 25,3} \bibverse{29}
Und Bileam sprach zu Balak: Baue mir hier sieben Altäre und schaffe mir
sieben Farren und sieben Widder. \footnote{\textbf{23:29} 4Mo 23,1}
\bibverse{30} Balak tat, wie Bileam sagte, und opferte je auf einem
Altar einen Farren und einen Widder. \# 24 \bibverse{1} Da nun Bileam
sah, dass es dem HErrn gefiel, dass er Israel segnete, ging er nicht
aus, wie vormals, nach Zauberei, sondern richtete sein Angesicht stracks
zu der Wüste, \bibverse{2} hob auf seine Augen und sah Israel, wie sie
lagen nach ihren Stämmen. Und der Geist Gottes kam auf ihn, \bibverse{3}
und er hob an seinen Spruch und sprach: Es sagt Bileam, der Sohn Beors,
es sagt der Mann, dem die Augen geöffnet sind, \footnote{\textbf{24:3}
  1Sam 9,9} \bibverse{4} es sagt der Hörer göttlicher Rede, der des
Allmächtigen Offenbarung sieht, dem die Augen geöffnet werden, wenn er
niederkniet: \footnote{\textbf{24:4} Jes 50,4} \bibverse{5} Wie fein
sind deine Hütten, Jakob, und deine Wohnungen, Israel! \bibverse{6} Wie
die Täler, die sich ausbreiten, wie die Gärten an den Wassern, wie die
Aloebäume, die der HErr pflanzt, wie die Zedern an den Wassern.
\bibverse{7} Es wird Wasser aus seinem Eimer fließen, und sein Same wird
ein großes Wasser werden; sein König wird höher werden denn Agag, und
sein Reich wird sich erheben. \bibverse{8} Gott hat ihn aus Ägypten
geführt; seine Freudigkeit ist wie eines Einhorns. Er wird die Heiden,
seine Verfolger, fressen und ihre Gebeine zermalmen und mit seinen
Pfeilen zerschmettern. \bibverse{9} Er hat sich niedergelegt wie ein
Löwe und wie ein junger Löwe; wer will sich wider ihn auflehnen?
Gesegnet sei, der dich segnet, und verflucht, der dir flucht!
\footnote{\textbf{24:9} 4Mo 23,24; 1Mo 49,9; 1Mo 12,3} \bibverse{10} Da
ergrimmte Balak im Zorn wider Bileam und schlug die Hände zusammen und
sprach zu ihm: Ich habe dich gefordert, dass du meinen Feinden fluchen
solltest; und siehe, du hast sie nun dreimal gesegnet. \bibverse{11} Und
nun hebe dich an deinen Ort! Ich gedachte, ich wollte dich ehren; aber
der HErr hat dir die Ehre verwehrt. \bibverse{12} Bileam antwortete ihm:
Habe ich nicht auch zu deinen Boten gesagt, die du zu mir sandtest, und
gesprochen: \bibverse{13} Wenn mir Balak sein Haus voll Silber und Gold
gäbe, so könnte ich doch an des HErrn Wort nicht vorüber, Böses oder
Gutes zu tun nach meinem Herzen; sondern was der HErr reden würde, das
würde ich auch reden? \bibverse{14} Und nun siehe, ich ziehe zu meinem
Volk. So komm, ich will dir verkündigen, was dieses Volk deinem Volk tun
wird zur letzten Zeit. \bibverse{15} Und er hob an seinen Spruch und
sprach: Es sagt Bileam, der Sohn Beors, es sagt der Mann, dem die Augen
geöffnet sind, \footnote{\textbf{24:15} 4Mo 24,3-4} \bibverse{16} es
sagt der Hörer göttlicher Rede und der die Erkenntnis hat des Höchsten,
der die Offenbarung des Allmächtigen sieht und dem die Augen geöffnet
werden, wenn er niederkniet: \bibverse{17} Ich sehe ihn, aber nicht
jetzt; ich schaue ihn, aber nicht von nahe. Es wird ein Stern aus Jakob
aufgehen und ein Zepter aus Israel aufkommen und wird zerschmettern die
Fürsten der Moabiter und verstören alle Kinder des Getümmels.
\bibverse{18} Edom wird er einnehmen, und Seir wird seinen Feinden
unterworfen sein; Israel aber wird Sieg haben. \footnote{\textbf{24:18}
  2Sam 8,14; Am 9,11; Am 1,9-12} \bibverse{19} Aus Jakob wird der
Herrscher kommen und umbringen, was übrig ist von den Städten.
\footnote{\textbf{24:19} Mi 5,1; Mi 5,7-8} \bibverse{20} Und da er sah
die Amalekiter, hob er an seinen Spruch und sprach: Amalek, die Ersten
unter den Heiden; aber zuletzt wirst du gar umkommen. \footnote{\textbf{24:20}
  2Mo 17,14} \bibverse{21} Und da er die Keniter sah, hob er an seinen
Spruch und sprach: Fest ist deine Wohnung, und hast dein Nest in einen
Fels gelegt. \footnote{\textbf{24:21} 1Sam 15,6; Ob 1,3} \bibverse{22}
Aber, o Kain, du wirst verbrannt werden, wenn Assur dich gefangen
wegführen wird. \bibverse{23} Und er hob abermals an seinen Spruch und
sprach: Ach, wer wird leben, wenn Gott solches tun wird? \bibverse{24}
Und Schiffe aus Chittim werden verderben den Assur und Eber; er aber
wird auch umkommen. \footnote{\textbf{24:24} 1Makk 1,1} \bibverse{25}
Und Bileam machte sich auf und zog hin und kam wieder an seinen Ort, und
Balak zog seinen Weg. \# 25 \bibverse{1} Und Israel wohnte in Sittim.
Und das Volk hob an zu huren mit der Moabiter Töchtern, \bibverse{2}
welche luden das Volk zum Opfer ihrer Götter. Und das Volk aß und betete
ihre Götter an. \bibverse{3} Und Israel hängte sich an den Baal-Peor. Da
ergrimmte des HErrn Zorn über Israel, \footnote{\textbf{25:3} 5Mo 4,3}
\bibverse{4} und er sprach zu Mose: nimm alle Obersten des Volks und
hänge sie dem HErrn auf an der Sonne, auf dass der grimmige Zorn des
HErrn von Israel gewandt werde. \footnote{\textbf{25:4} 2Sam 21,6; 2Sam
  21,9; 5Mo 21,22-23} \bibverse{5} Und Mose sprach zu den Richtern
Israels: Erwürge ein jeglicher seine Leute, die sich an den Baal-Peor
gehängt haben. \bibverse{6} Und siehe, ein Mann aus den Kindern Israel
kam und brachte unter seine Brüder eine Midianitin vor den Augen Moses
und der ganzen Gemeinde der Kinder Israel, die da weinten vor der Tür
der Hütte des Stifts. \bibverse{7} Da das sah Pinehas, der Sohn
Eleasars, des Sohnes Aarons, des Priesters, stand er auf aus der
Gemeinde und nahm einen Spieß in seine Hand \bibverse{8} und ging dem
israelitischen Mann nach hinein in die Kammer und durchstach sie beide,
den israelitischen Mann und das Weib, durch ihren Bauch. Da hörte die
Plage auf von den Kindern Israel. \bibverse{9} Und es wurden getötet in
der Plage 24.000. \footnote{\textbf{25:9} 1Kor 10,8} \bibverse{10} Und
der HErr redete mit Mose und sprach: \bibverse{11} Pinehas, der Sohn
Eleasars, des Sohnes Aarons, des Priesters, hat meinen Grimm von den
Kindern Israel gewendet durch seinen Eifer um mich, dass ich nicht in
meinem Eifer die Kinder Israel vertilgte. \bibverse{12} Darum sage:
Siehe, ich gebe ihm meinen Bund des Friedens; \bibverse{13} und er soll
haben und sein Same nach ihm den Bund eines ewigen Priestertums, darum
dass er für seinen Gott geeifert und die Kinder Israel versöhnt hat.
\footnote{\textbf{25:13} Ps 106,30-31} \bibverse{14} Der israelitische
Mann aber, der erschlagen ward mit der Midianitin, hieß Simri, der Sohn
Salus, der Fürst eines Vaterhauses der Simeoniter. \bibverse{15} Das
midianitische Weib, das auch erschlagen ward, hieß Kosbi, eine Tochter
Zurs, der ein Fürst war eines Geschlechts unter den Midianitern.
\bibverse{16} Und der HErr redete mit Mose und sprach: \bibverse{17} Tut
den Midianitern Schaden und schlagt sie; \footnote{\textbf{25:17} 4Mo
  31,2-10} \bibverse{18} denn sie haben euch Schaden getan mit ihrer
List, die sie wider euch geübt haben durch den Peor und durch ihre
Schwester Kosbi, die Tochter des Fürsten der Midianiter, die erschlagen
ist am Tag der Plage um des Peor willen. \# 26 \bibverse{1} Und es
geschah, nach der Plage sprach der HErr zu Mose und Eleasar, dem Sohn
des Priesters Aaron: \bibverse{2} Nehmt die Summe der ganzen Gemeinde
der Kinder Israel, von 20 Jahren und darüber, nach ihren Vaterhäusern,
alle, die ins Heer zu ziehen taugen in Israel. \bibverse{3} Und Mose
redete mit ihnen samt Eleasar, dem Priester, in dem Gefilde der
Moabiter, an dem Jordan gegenüber Jericho, \bibverse{4} die 20 Jahre alt
waren und darüber, wie der HErr dem Mose geboten hatte und den Kindern
Israel, die aus Ägypten gezogen waren. \bibverse{5} Ruben, der
Erstgeborene Israels. Die Kinder Rubens aber waren: Henoch, von dem das
Geschlecht der Henochiter kommt; Pallu, von dem das Geschlecht der
Palluiter kommt; \footnote{\textbf{26:5} 1Mo 46,8-27; 1Chr 4,1-7}
\bibverse{6} Hezron, von dem das Geschlecht der Hezroniter kommt;
Charmi, von dem das Geschlecht der Charmiter kommt. \bibverse{7} Das
sind die Geschlechter von Ruben, und ihre Zahl war 43.730. \bibverse{8}
Aber die Kinder Pallus waren: Eliab. \bibverse{9} Und die Kinder Eliabs
waren: Nemuel und Dathan und Abiram, die Vornehmen in der Gemeinde, die
sich wider Mose und Aaron auflehnten in der Rotte Korahs, da sie sich
wider den HErrn auflehnten \bibverse{10} und die Erde ihren Mund auftat
und sie verschlang mit Korah, da die Rotte starb, da das Feuer 250
Männer fraß und sie ein Zeichen wurden. \bibverse{11} Aber die Kinder
Korahs starben nicht. \bibverse{12} Die Kinder Simeons in ihren
Geschlechtern waren: Nemuel, daher kommt das Geschlecht der Nemueliter;
Jamin, daher kommt das Geschlecht der Jaminiter; Jachin, daher das
Geschlecht der Jachiniter kommt; \bibverse{13} Serah, daher das
Geschlecht der Serahiter kommt; Saul, daher das Geschlecht der Sauliter
kommt. \bibverse{14} Das sind die Geschlechter von Simeon, 22.200.
\bibverse{15} Die Kinder Gads in ihren Geschlechtern waren: Ziphon,
daher das Geschlecht der Ziphoniter kommt; Haggi, daher das Geschlecht
der Haggiter kommt; Suni, daher das Geschlecht der Suniter kommt;
\bibverse{16} Osni, daher das Geschlecht der Osniter kommt; Eri, daher
das Geschlecht der Eriter kommt; \bibverse{17} Arod, daher das
Geschlecht der Aroditer kommt; Ariel, daher das Geschlecht der Arieliter
kommt. \bibverse{18} Das sind die Geschlechter der Kinder Gads, an ihrer
Zahl 40.500 \bibverse{19} Die Kinder Judas: Ger und Onan, welche beide
starben im Lande Kanaan. \footnote{\textbf{26:19} 1Mo 38,7; 1Mo 38,10}
\bibverse{20} Es waren aber die Kinder Judas in ihren Geschlechtern:
Sela, daher das Geschlecht der Selaniter kommt; Perez, daher das
Geschlecht der Pereziter kommt; Serah, daher das Geschlecht der
Serahiter kommt. \bibverse{21} Aber die Kinder des Perez waren: Hezron,
daher das Geschlecht der Hezroniter kommt; Hamul, daher das Geschlecht
der Hamuliter kommt. \bibverse{22} Das sind die Geschlechter Judas, an
ihrer Zahl 76.500. \bibverse{23} Die Kinder Isaschars in ihren
Geschlechtern waren: Thola, daher das Geschlecht der Tholaiter kommt;
Phuva, daher das Geschlecht der Phuvaniter kommt; \bibverse{24} Jasub,
daher das Geschlecht der Jasubiter kommt; Simron, daher das Geschlecht
der Simroniter kommt. \bibverse{25} Das sind die Geschlechter Isaschars,
an der Zahl 64.300. \bibverse{26} Die Kinder Sebulons in ihren
Geschlechtern waren: Sered, daher das Geschlecht der Serediter kommt;
Elon, daher das Geschlecht der Eloniter kommt; Jahleel, daher das
Geschlecht der Jahleeliter kommt. \bibverse{27} Das sind die
Geschlechter Sebulons, an ihrer Zahl 60.500. \bibverse{28} Die Kinder
Josephs in ihren Geschlechtern waren: Manasse und Ephraim. \bibverse{29}
Die Kinder aber Manasses waren: Machir, daher kommt das Geschlecht der
Machiriter; Machir aber zeugte Gilead, daher kommt das Geschlecht der
Gileaditer. \footnote{\textbf{26:29} Jos 17,1-3} \bibverse{30} Dies sind
aber die Kinder Gileads: Hieser, daher kommt das Geschlecht der
Hieseriter; Helek, daher kommt das Geschlecht der Helekiter;
\bibverse{31} Asriel, daher kommt das Geschlecht der Asrieliter; Sichem,
daher kommt das Geschlecht der Sichemiter; \bibverse{32} Semida, daher
kommt das Geschlecht der Semiditer; Hepher, daher kommt das Geschlecht
der Hepheriter. \bibverse{33} Zelophehad aber war Hephers Sohn und hatte
keine Söhne, sondern Töchter; die hießen Mahela, Noa, Hogla, Milka und
Thirza. \bibverse{34} Das sind die Geschlechter Manasses, an ihrer Zahl
52.700. \bibverse{35} Die Kinder Ephraims in ihren Geschlechtern waren:
Suthelah, daher kommt das Geschlecht der Suthelahiter; Becher, daher
kommt das Geschlecht der Becheriter; Thahan, daher kommt das Geschlecht
der Thahaniter. \bibverse{36} Die Kinder aber Suthelahs waren: Eran,
daher kommt das Geschlecht der Eraniter. \bibverse{37} Das sind die
Geschlechter der Kinder Ephraims, an ihrer Zahl 32.500. Das sind die
Kinder Josephs in ihren Geschlechtern. \bibverse{38} Die Kinder
Benjamins in ihren Geschlechtern waren: Bela, daher kommt das Geschlecht
der Belaiter; Asbel, daher kommt das Geschlecht der Asbeliter; Ahiram,
daher kommt das Geschlecht der Ahiramiter; \bibverse{39} Supham, daher
kommt das Geschlecht der Suphamiter; Hupham, daher kommt das Geschlecht
der Huphamiter. \bibverse{40} Die Kinder aber Belas waren: Ard und
Naeman, daher kommt das Geschlecht der Arditer und Naemaniter.
\bibverse{41} Das sind die Kinder Benjamins in ihren Geschlechtern, an
der Zahl 45.600. \bibverse{42} Die Kinder Dans in ihren Geschlechtern
waren: Suham, daher kommt das Geschlecht der Suhamiter. \bibverse{43}
Das sind die Geschlechter Dans in ihren Geschlechtern, allesamt an der
Zahl 64.400. \bibverse{44} Die Kinder Assers in ihren Geschlechtern
waren: Jimna, daher kommt das Geschlecht der Jimniter; Jiswi, daher
kommt das Geschlecht der Jiswiter; Beria, daher kommt das Geschlecht der
Beriiter. \bibverse{45} Aber die Kinder Berias waren: Heber, daher kommt
das Geschlecht der Hebriter; Melchiel, daher kommt das Geschlecht der
Melchieliter. \bibverse{46} Und die Tochter Assers hieß Sarah.
\bibverse{47} Das sind die Geschlechter der Kinder Assers, an ihrer Zahl
53.400. \bibverse{48} Die Kinder Naphthalis in ihren Geschlechtern
waren: Jahzeel, daher kommt das Geschlecht der Jahzeeliter; Guni, daher
kommt das Geschlecht der Guniter; \bibverse{49} Jezer, daher kommt das
Geschlecht der Jezeriter; Sillem, daher kommt das Geschlecht der
Sillemiter. \bibverse{50} Das sind die Geschlechter von Naphthali, an
ihrer Zahl 45.400. \bibverse{51} Das ist die Summe der Kinder Israel
601.730. \bibverse{52} Und der HErr redete mit Mose und sprach:
\bibverse{53} Diesen sollst du das Land austeilen zum Erbe nach der Zahl
der Namen. \bibverse{54} Vielen sollst du viel zum Erbe geben, und
wenigen wenig; jeglichen soll man geben nach ihrer Zahl. \bibverse{55}
Doch man soll das Land durchs Los teilen; nach den Namen der Stämme
ihrer Väter sollen sie Erbe nehmen. \footnote{\textbf{26:55} 4Mo 33,54;
  Jos 14,2} \bibverse{56} Denn nach dem Los sollst du ihr Erbe austeilen
zwischen den vielen und den wenigen. \bibverse{57} Und das ist die Summe
der Leviten in ihren Geschlechtern: Gerson, daher das Geschlecht der
Gersoniter; Kahath, daher das Geschlecht der Kahathiter; Merari, daher
das Geschlecht der Merariter. \bibverse{58} Dies sind die Geschlechter
Levis: das Geschlecht der Libniter, das Geschlecht der Hebroniter, das
Geschlecht der Maheliter, das Geschlecht der Musiter, das Geschlecht der
Korahiter. Kahath zeugte Amram. \bibverse{59} Und Amrams Weib hieß
Jochebed, eine Tochter Levis, die ihm geboren ward in Ägypten; und sie
gebar dem Amram Aaron und Mose und ihre Schwester Mirjam. \bibverse{60}
Dem Aaron aber ward geboren: Nadab, Abihu, Eleasar und Ithamar.
\bibverse{61} Nadab aber und Abihu starben, da sie fremdes Feuer
opferten vor dem HErrn. \footnote{\textbf{26:61} 3Mo 10,1-2}
\bibverse{62} Und ihre Summe war 23.000, alles Mannsbilder, von einem
Monat und darüber. Denn sie wurden nicht gezählt unter die Kinder
Israel; denn man gab ihnen kein Erbe unter den Kindern Israel.
\bibverse{63} Das ist die Summe der Kinder Israel, die Mose und Eleasar,
der Priester, zählten im Gefilde der Moabiter, an dem Jordan gegenüber
Jericho;

\bibverse{64} unter welchen war keiner aus der Summe, da Mose und Aaron,
der Priester, die Kinder Israel zählten in der Wüste Sinai.

\bibverse{65} Denn der HErr hatte ihnen gesagt, sie sollten des Todes
sterben in der Wüste. Und blieb keiner übrig als Kaleb, der Sohn
Jephunnes, und Josua, der Sohn Nuns. \footnote{\textbf{26:65} 4Mo
  14,22-38}

\hypertarget{section-5}{%
\section{27}\label{section-5}}

\bibverse{1} Und die Töchter Zelophehads, des Sohnes Hephers, des Sohnes
Gileads, des Sohnes Machirs, des Sohnes Manasses, unter den
Geschlechtern Manasses, des Sohnes Josephs, mit Namen Mahela, Noa,
Hogla, Milka und Thirza, kamen herzu \footnote{\textbf{27:1} 4Mo 26,33;
  4Mo 36,2; Jos 17,3-6} \bibverse{2} und traten vor Mose und vor
Eleasar, den Priester, und vor die Fürsten und die ganze Gemeinde vor
der Tür der Hütte des Stifts und sprachen: \bibverse{3} Unser Vater ist
gestorben in der Wüste und war nicht mit unter der Gemeinde, die sich
wider den HErrn empörte in der Rotte Korahs, sondern ist an seiner Sünde
gestorben, und hatte keine Söhne. \footnote{\textbf{27:3} 4Mo 16,2; 4Mo
  26,65} \bibverse{4} Warum soll denn unseres Vaters Name unter seinem
Geschlecht untergehen, weil er keinen Sohn hat? Gebt uns auch ein Gut
unter unseres Vaters Brüdern! \bibverse{5} Mose brachte ihre Sache vor
den HErrn. \bibverse{6} Und der HErr sprach zu ihm: \bibverse{7} Die
Töchter Zelophehads haben recht geredet; du sollst ihnen ein Erbgut
unter ihres Vaters Brüdern geben und sollst ihres Vaters Erbe ihnen
zuwenden. \bibverse{8} Und sage den Kindern Israel: Wenn jemand stirbt
und hat nicht Söhne, so sollt ihr sein Erbe seiner Tochter zuwenden.
\bibverse{9} Hat er keine Tochter, sollt ihr's seinen Brüdern geben.
\bibverse{10} Hat er keine Brüder, sollt ihr's seines Vaters Brüdern
geben. \bibverse{11} Hat er nicht Vatersbrüder, sollt ihr's seinen
nächsten Blutsfreunden geben, die ihm angehören in seinem Geschlecht,
dass sie es einnehmen. Das soll den Kindern Israel ein Gesetz und Recht
sein, wie der HErr dem Mose geboten hat. \bibverse{12} Und der HErr
sprach zu Mose: Steig auf dies Gebirge Abarim und besiehe das Land, das
ich den Kindern Israel gebe werde. \footnote{\textbf{27:12} 5Mo 32,48-49}
\bibverse{13} Und wenn du es gesehen hast, sollst du dich sammeln zu
deinem Volk, wie dein Bruder Aaron versammelt ist, \footnote{\textbf{27:13}
  4Mo 20,24; 4Mo 20,28} \bibverse{14} dieweil ihr meinem Wort ungehorsam
gewesen seid in der Wüste Zin bei dem Hader der Gemeinde, da ihr mich
heiligen solltet durch das Wasser vor ihnen. Das ist das Haderwasser zu
Kades in der Wüste Zin. \footnote{\textbf{27:14} 4Mo 20,12-13}
\bibverse{15} Und Mose redete mit dem HErrn und sprach: \bibverse{16}
Der HErr, der Gott der Geister alles Fleisches, wolle einen Mann setzen
über die Gemeinde, \bibverse{17} der vor ihnen her aus und ein gehe und
sie aus und ein führe, dass die Gemeinde des HErrn nicht sei wie die
Schafe ohne Hirten. \footnote{\textbf{27:17} Mt 9,36} \bibverse{18} Und
der HErr sprach zu Mose: Nimm Josua zu dir, den Sohn Nuns, einen Mann,
in dem der Geist ist, und lege deine Hände auf ihn \footnote{\textbf{27:18}
  5Mo 3,21; 5Mo 34,9} \bibverse{19} und stelle ihn vor den Priester
Eleasar und vor die ganze Gemeinde und gebiete ihm vor ihren Augen,
\bibverse{20} und lege von deiner Herrlichkeit auf ihn, dass ihm
gehorche die ganze Gemeinde der Kinder Israel. \footnote{\textbf{27:20}
  2Kö 2,9; 2Kö 2,15} \bibverse{21} Und er soll treten vor den Priester
Eleasar, der soll für ihn ratfragen durch die Weise des Lichts vor dem
HErrn. Nach desselben Mund sollen aus und ein ziehen er und alle Kinder
Israel mit ihm und die ganze Gemeinde. \footnote{\textbf{27:21} 2Mo
  28,30} \bibverse{22} Mose tat, wie ihm der HErr geboten hatte, und
nahm Josua und stellte ihn vor den Priester Eleasar und vor die ganze
Gemeinde \bibverse{23} und legte seine Hand auf ihn und gebot ihm, wie
der HErr mit Mose geredet hatte. \# 28 \bibverse{1} Und der HErr redete
mit Mose und sprach: \bibverse{2} Gebiete den Kindern Israel und sprich
zu ihnen: Die Opfer meines Brots, welches mein Opfer des süßen Geruchs
ist, sollt ihr halten zu seiner Zeit, dass ihr mir's opfert. \footnote{\textbf{28:2}
  3Mo 21,6} \bibverse{3} Und sprich zu ihnen: Das sind die Opfer, die
ihr dem HErrn opfern sollt: jährige Lämmer, die ohne Fehl sind, täglich
zwei zum täglichen Brandopfer, \footnote{\textbf{28:3} 2Mo 29,38-42}
\bibverse{4} Ein Lamm des Morgens, das andere gegen Abend; \bibverse{5}
dazu ein zehntel Epha Semmelmehl zum Speisopfer, mit Öl gemengt, das
gestoßen ist, ein viertel Hin. \footnote{\textbf{28:5} 3Mo 2,1}
\bibverse{6} Das ist das tägliche Brandopfer, das ihr am Berge Sinai
opfertet, zum süßen Geruch ein Feuer dem HErrn. \bibverse{7} Dazu sein
Trankopfer je zu einem Lamm ein viertel Hin. Im Heiligtum soll man den
Wein des Trankopfers opfern dem HErrn. \bibverse{8} Das andere Lamm
sollst du gegen Abend zurichten; mit dem Speisopfer wie am Morgen und
mit seinem Trankopfer sollst du es machen zum Opfer des süßen Geruchs
dem HErrn. \bibverse{9} Am Sabbattag aber zwei jährige Lämmer ohne Fehl
und zwei Zehntel Semmelmehl zum Speisopfer, mit Öl gemengt, und sein
Trankopfer. \bibverse{10} Das ist das Brandopfer eines jeglichen Sabbats
außer dem täglichen Brandopfer samt seinem Trankopfer. \bibverse{11}
Aber des ersten Tages eurer Monate sollt ihr dem HErrn ein Brandopfer
opfern: zwei junge Farren, einen Widder, sieben jährige Lämmer ohne
Fehl; \footnote{\textbf{28:11} 4Mo 10,10} \bibverse{12} und je drei
Zehntel Semmelmehl zum Speisopfer, mit Öl gemengt, zu einem Farren; und
zwei Zehntel Semmelmehl zum Speisopfer, mit Öl gemengt, zu dem einen
Widder; \footnote{\textbf{28:12} 4Mo 28,20; 4Mo 28,28; 4Mo 15,2-13}
\bibverse{13} und je ein Zehntel Semmelmehl zum Speisopfer, mit Öl
gemengt, zu einem Lamm. Das ist ein Brandopfer des süßen Geruchs, ein
Opfer dem HErrn. \bibverse{14} Und ihr Trankopfer soll sein ein halbes
Hin Wein zum Farren, ein drittel Hin zum Widder, ein viertel Hin zum
Lamm. Das ist das Brandopfer eines jeglichen Monats im Jahr.
\bibverse{15} Dazu soll man einen Ziegenbock zum Sündopfer dem HErrn
machen außer dem täglichen Brandopfer und seinem Trankopfer. \footnote{\textbf{28:15}
  4Mo 28,22} \bibverse{16} Aber am 14. Tage des ersten Monats ist das
Passah des HErrn. \footnote{\textbf{28:16} 3Mo 23,5-14} \bibverse{17}
Und am 15. Tage desselben Monats ist Fest. Sieben Tage soll man
ungesäuertes Brot essen. \bibverse{18} Der erste Tag soll heilig heißen,
dass ihr zusammenkommt; keine Dienstarbeit sollt ihr an dem tun
\footnote{\textbf{28:18} 4Mo 28,25-26} \bibverse{19} und sollt dem HErrn
Brandopfer tun: zwei junge Farren, einen Widder, sieben jährige Lämmer
ohne Fehl; \bibverse{20} samt ihren Speisopfern: drei Zehntel
Semmelmehl, mit Öl gemengt, zu einem Farren, und zwei Zehntel zu dem
Widder, \bibverse{21} und je ein Zehntel auf ein Lamm unter den sieben
Lämmern; \bibverse{22} dazu einen Bock zum Sündopfer, dass ihr versöhnt
werdet. \footnote{\textbf{28:22} 4Mo 28,15} \bibverse{23} Und sollt
solches tun außer dem Brandopfer am Morgen, welches das tägliche
Brandopfer ist. \bibverse{24} Nach dieser Weise sollt ihr alle Tage, die
sieben Tage lang, das Brot opfern zum Opfer des süßen Geruchs dem HErrn
außer dem täglichen Brandopfer, dazu sein Trankopfer. \bibverse{25} Und
der siebente Tag soll bei euch heilig heißen, dass ihr zusammenkommt;
keine Dienstarbeit sollt ihr da tun. \bibverse{26} Und der Tag der
Erstlinge, wenn ihr opfert das neue Speisopfer dem HErrn, wenn eure
Wochen um sind, soll heilig heißen, dass ihr zusammenkommt; keine
Dienstarbeit sollt ihr da tun \bibverse{27} und sollt dem HErrn
Brandopfer tun zum süßen Geruch: zwei junge Farren, einen Widder, sieben
jährige Lämmer; \bibverse{28} samt ihrem Speisopfer: drei Zehntel
Semmelmehl, mit Öl gemengt, zu einem Farren, zwei Zehntel zu dem Widder,
\bibverse{29} und je ein Zehntel zu einem Lamm der sieben Lämmer;
\bibverse{30} und einen Ziegenbock, euch zu versöhnen. \footnote{\textbf{28:30}
  4Mo 28,15} \bibverse{31} Dies sollt ihr tun außer dem täglichen
Brandopfer mit seinem Speisopfer. Ohne Fehl soll's sein, dazu ihre
Trankopfer. \# 29 \bibverse{1} Und der erste Tag des siebenten Monats
soll bei euch heilig heißen, dass ihr zusammenkommt; keine Dienstarbeit
sollt ihr da tun -- es ist euer Drommetentag -- \bibverse{2} und sollt
Brandopfer tun zum süßen Geruch dem HErrn: einen jungen Farren, einen
Widder, sieben jährige Lämmer ohne Fehl; \bibverse{3} dazu ihr
Speisopfer: drei Zehntel Semmelmehl, mit Öl gemengt, zu dem Farren, zwei
Zehntel zu dem Widder, \bibverse{4} und ein Zehntel auf ein jegliches
Lamm der sieben Lämmer; \bibverse{5} auch einen Ziegenbock zum
Sündopfer, euch zu versöhnen -- \bibverse{6} außer dem Brandopfer des
Monats und seinem Speisopfer und außer dem täglichen Brandopfer mit
seinem Speisopfer und mit ihrem Trankopfer, wie es recht ist --, zum
süßen Geruch. Das ist ein Opfer dem HErrn. \bibverse{7} Der zehnte Tag
dieses siebenten Monats soll bei euch auch heilig heißen, dass ihr
zusammenkommt; und sollt eure Leiber kasteien und keine Arbeit da tun,
\footnote{\textbf{29:7} 3Mo 23,27-32} \bibverse{8} sondern Brandopfer
dem HErrn zum süßen Geruch opfern: einen jungen Farren, einen Widder,
sieben jährige Lämmer ohne Fehl; \bibverse{9} mit ihren Speisopfern:
drei Zehntel Semmelmehl, mit Öl gemengt, zu dem Farren, zwei Zehntel zu
dem Widder, \bibverse{10} und ein Zehntel je zu einem der sieben Lämmer;
\bibverse{11} dazu einen Ziegenbock zum Sündopfer, -- außer dem
Sündopfer der Versöhnung und dem täglichen Brandopfer mit seinem
Speisopfer und mit ihrem Trankopfer. \bibverse{12} Der 15. Tag des
siebenten Monats soll bei euch heilig heißen, dass ihr zusammenkommt;
keine Dienstarbeit sollt ihr an dem tun und sollt dem HErrn sieben Tage
feiern \footnote{\textbf{29:12} 3Mo 23,34-43} \bibverse{13} und sollt
dem HErrn Brandopfer tun zum Opfer des süßen Geruchs dem HErrn: 13 junge
Farren, zwei Widder; 14 jährige Lämmer ohne Fehl; \bibverse{14} samt
ihrem Speisopfer: drei Zehntel Semmelmehl, mit Öl gemengt, je zu einem
der 13 Farren, zwei Zehntel je zu einem der zwei Widder, \bibverse{15}
und ein Zehntel je zu einem der 14 Lämmer; \bibverse{16} dazu einen
Ziegenbock zum Sündopfer, -- außer dem täglichen Brandopfer mit seinem
Speisopfer und seinem Trankopfer. \bibverse{17} Am zweiten Tage: zwölf
junge Farren, zwei Widder, 14 jährige Lämmer ohne Fehl; \bibverse{18}
mit ihrem Speisopfer und Trankopfer zu den Farren, zu den Widdern und zu
den Lämmern in ihrer Zahl, wie es recht ist; \bibverse{19} dazu einen
Ziegenbock zum Sündopfer, -- außer dem täglichen Brandopfer mit seinem
Speisopfer und mit ihrem Trankopfer. \bibverse{20} Am dritten Tage: elf
Farren, zwei Widder, 14 jährige Lämmer ohne Fehl; \bibverse{21} mit
ihren Speisopfern und Trankopfern zu den Farren, zu den Widdern und zu
den Lämmern in ihrer Zahl, wie es recht ist; \bibverse{22} dazu einen
Bock zum Sündopfer, -- außer dem täglichen Brandopfer mit seinem
Speisopfer und mit seinem Trankopfer. \bibverse{23} Am vierten Tage:
zehn Farren, zwei Widder, 14 jährige Lämmer ohne Fehl; \bibverse{24}
samt ihren Speisopfern und Trankopfern zu den Farren, zu den Widdern und
zu den Lämmern in ihrer Zahl, wie es recht ist; \bibverse{25} dazu einen
Ziegenbock zum Sündopfer, -- außer dem täglichen Brandopfer mit seinem
Speisopfer und seinem Trankopfer. \bibverse{26} Am fünften Tage: neun
Farren, zwei Widder, 14 jährige Lämmer ohne Fehl; \bibverse{27} samt
ihren Speisopfern und Trankopfern zu den Farren, zu den Widdern und zu
den Lämmern in ihrer Zahl, wie es recht ist; \bibverse{28} dazu einen
Bock zum Sündopfer, -- außer dem täglichen Brandopfer mit seinem
Speisopfer und seinem Trankopfer. \bibverse{29} Am sechsten Tage: acht
Farren, zwei Widder, 14 jährige Lämmer ohne Fehl; \bibverse{30} samt
ihren Speisopfern und Trankopfern zu den Farren, zu den Widdern und zu
den Lämmern in ihrer Zahl, wie es recht ist; \bibverse{31} dazu einen
Bock zum Sündopfer, -- außer dem täglichen Brandopfer mit seinem
Speisopfer und seinem Trankopfer. \bibverse{32} Am siebenten Tage:
sieben Farren, zwei Widder, 14 jährige Lämmer ohne Fehl; \bibverse{33}
samt ihren Speisopfern und Trankopfern zu den Farren, zu den Widdern und
zu den Lämmern in ihrer Zahl, wie es recht ist; \bibverse{34} dazu einen
Bock zum Sündopfer, -- außer dem täglichen Brandopfer mit seinem
Speisopfer und seinem Trankopfer. \bibverse{35} Am achten soll der Tag
der Versammlung sein; keine Dienstarbeit sollt ihr da tun \bibverse{36}
und sollt Brandopfer opfern zum Opfer des süßen Geruchs dem HErrn: einen
Farren, einen Widder, sieben jährige Lämmer ohne Fehl; \bibverse{37}
samt ihren Speisopfern und Trankopfern zu dem Farren, zu dem Widder und
zu den Lämmern in ihrer Zahl, wie es recht ist; \bibverse{38} dazu einen
Bock zum Sündopfer, -- außer dem täglichen Brandopfer mit seinem
Speisopfer und seinem Trankopfer. \bibverse{39} Solches sollt ihr dem
HErrn tun auf eure Feste, außerdem, was ihr gelobt und freiwillig gebt
zu Brandopfern, Speisopfern, Trankopfern und Dankopfern. \# 30
\bibverse{1} Und Mose sagte den Kindern Israel alles, was ihm der HErr
geboten hatte. \bibverse{2} Und Mose redete mit den Fürsten der Stämme
der Kinder Israel und sprach: das ist's, was der HErr geboten hat:
\bibverse{3} Wenn jemand dem HErrn ein Gelübde tut oder einen Eid
schwört, dass er seine Seele verbindet, der soll sein Wort nicht
aufheben, sondern alles tun, wie es zu seinem Munde ist ausgegangen.
\bibverse{4} Wenn ein Weib dem HErrn ein Gelübde tut und sich verbindet,
solange sie in ihres Vaters Hause und ledig ist, \bibverse{5} und ihr
Gelübde und Verbündnis, das sie nimmt auf ihre Seele, kommt vor ihren
Vater, und er schweigt dazu, so gilt all ihr Gelübde und all ihr
Verbündnis, das sie ihrer Seele aufgelegt hat. \bibverse{6} Wo aber ihr
Vater ihr wehrt des Tages, wenn er's hört, so gilt kein Gelübde noch
Verbündnis, das sie auf ihre Seele genommen hat; und der HErr wird ihr
gnädig sein, weil ihr Vater ihr gewehrt hat. \bibverse{7} Wird sie aber
eines Mannes und hat ein Gelübde auf sich oder ist ihr aus ihren Lippen
ein Verbündnis entfahren über ihre Seele, \bibverse{8} und der Mann hört
es, und schweigt desselben Tages, wenn er's hört, so gilt ihr Gelübde
und Verbündnis, das sie auf ihre Seele genommen hat. \bibverse{9} Wo
aber ihr Mann ihr wehrt des Tages, wenn er's hört, so ist ihr Gelübde
los, das sie auf sich hat, und das Verbündnis, das ihr aus ihren Lippen
entfahren ist über ihre Seele; und der HErr wird ihr gnädig sein.
\bibverse{10} Das Gelübde einer Witwe und Verstoßenen, alles Verbündnis,
das sie nimmt auf ihre Seele, das gilt auf ihr. \bibverse{11} Wenn eine
in ihres Mannes Hause gelobt oder sich mit einem Eide verbindet über
ihre Seele, \bibverse{12} und ihr Mann hört es, und schweigt dazu und
wehrt es nicht, so gilt all dasselbe Gelübde und alles Verbündnis, das
sie auflegt ihrer Seele. \bibverse{13} Macht's aber ihr Mann des Tages
los, wenn er's hört, so gilt das nichts, was aus ihren Lippen gegangen
ist, was sie gelobt oder wozu sie sich verbunden hat über ihre Seele;
denn ihr Mann hat's losgemacht, und der HErr wird ihr gnädig sein.
\bibverse{14} Alle Gelübde und Eide, die verbinden, den Leib zu
kasteien, mag ihr Mann bekräftigen oder aufheben also: \bibverse{15}
wenn er dazu schweigt von einem Tage zum anderen, so bekräftigt er alle
ihre Gelübde und Verbündnisse, die sie auf sich hat, darum dass er
geschwiegen hat des Tages, da er's hörte; \bibverse{16} wird er's aber
aufheben, nachdem er's gehört hat, so soll er ihre Missetat tragen.
\bibverse{17} Das sind die Satzungen, die der HErr dem Mose geboten hat
zwischen Mann und Weib, zwischen Vater und Tochter, solange sie noch
ledig ist in ihres Vaters Hause. \# 31 \bibverse{1} Und der HErr redete
mit Mose und sprach: \bibverse{2} Räche die Kinder Israel an den
Midianitern, dass du darnach dich sammelst zu deinem Volk. \footnote{\textbf{31:2}
  4Mo 25,17; 4Mo 27,13} \bibverse{3} Da redete Mose mit dem Volk und
sprach: Rüstet unter euch Leute zum Heer wider die Midianiter, dass sie
den HErrn rächen an den Midianitern, \bibverse{4} aus jeglichem Stamm
1000, dass ihr aus allen Stämmen Israels in das Heer schickt.
\bibverse{5} Und sie nahmen aus den Tausenden Israels je 1000 eines
Stammes, 12.000 gerüstet zum Heer. \bibverse{6} Und Mose schickte sie
mit Pinehas, dem Sohn Eleasars, des Priesters, ins Heer und die heiligen
Geräte und die Halldrommeten in seiner Hand. \bibverse{7} Und sie
führten das Heer wider die Midianiter, wie der HErr dem Mose geboten
hatte, und erwürgten alles, was männlich war. \footnote{\textbf{31:7}
  2Mo 20,13} \bibverse{8} Dazu die Könige der Midianiter erwürgten sie
samt ihren Erschlagenen, nämlich Evi, Rekem, Zur, Hur und Reba, die fünf
Könige der Midianiter. Bileam, den Sohn Beors, erwürgten sie auch mit
dem Schwert. \footnote{\textbf{31:8} Jos 13,21-22; 4Mo 22,5}
\bibverse{9} Und die Kinder Israel nahmen gefangen die Weiber der
Midianiter und ihre Kinder; all ihr Vieh, alle ihre Habe und alle ihre
Güter raubten sie, \bibverse{10} und verbrannten mit Feuer alle ihre
Städte ihrer Wohnung und alle Zeltdörfer. \bibverse{11} Und nahmen allen
Raub und alles, was zu nehmen war, Menschen und Vieh, \bibverse{12} und
brachten's zu Mose und zu Eleasar, dem Priester, und zu der Gemeinde der
Kinder Israel, nämlich die Gefangenen und das genommene Vieh und das
geraubte Gut ins Lager auf der Moabiter Gefilde, das am Jordan liegt
gegenüber Jericho. \bibverse{13} Und Mose und Eleasar, der Priester, und
alle Fürsten der Gemeinde gingen ihnen entgegen, hinaus vor das Lager.
\bibverse{14} Und Mose ward zornig über die Hauptleute des Heeres, die
Hauptleute über 1000 und über 100 waren, die aus dem Heer und Streit
kamen, \bibverse{15} und sprach zu ihnen: Warum habt ihr alle Weiber
leben lassen? \bibverse{16} Siehe, haben nicht dieselben die Kinder
Israel durch Bileams Rat abwendig gemacht, dass sie sich versündigten am
HErrn über dem Peor und eine Plage der Gemeinde des HErrn widerfuhr?
\footnote{\textbf{31:16} 4Mo 25,1; Offb 2,14} \bibverse{17} So erwürget
nun alles, was männlich ist unter den Kindern, und alle Weiber, die
Männer erkannt und beigelegen haben; \footnote{\textbf{31:17} Ri 21,11}
\bibverse{18} aber alle Kinder, die weiblich sind und nicht Männer
erkannt haben, die lasst für euch leben. \bibverse{19} Und lagert euch
draußen vor dem Lager sieben Tage, alle, die jemand erwürgt oder die
Erschlagene angerührt haben, dass ihr euch entsündigt am dritten und
siebenten Tage, samt denen, die ihr gefangen genommen habt. \footnote{\textbf{31:19}
  4Mo 19,11} \bibverse{20} Und alle Kleider und alles Gerät von Fellen
und alles Pelzwerk und alles hölzerne Gefäß sollt ihr entsündigen.
\bibverse{21} Und Eleasar, der Priester, sprach zu dem Kriegsvolk, das
in den Streit gezogen war: Das ist das Gesetz, welches der HErr dem Mose
geboten hat: \bibverse{22} Gold, Silber, Erz, Eisen, Zinn und Blei
\bibverse{23} und alles was das Feuer leidet, sollt ihr durchs Feuer
lassen gehen und reinigen; nur dass es mit dem Sprengwasser entsündigt
werde. Aber alles, was nicht Feuer leidet, sollt ihr durchs Wasser gehen
lassen. \bibverse{24} Und sollt eure Kleider waschen am siebenten Tage,
so werdet ihr rein; darnach sollt ihr ins Lager kommen. \bibverse{25}
Und der HErr redete mit Mose und sprach: \bibverse{26} Nimm die Summe
des Raubes der Gefangenen, an Menschen und an Vieh, du und Eleasar, der
Priester, und die obersten Väter der Gemeinde; \bibverse{27} und gib die
Hälfte denen, die ins Heer ausgezogen sind und die Schlacht getan haben,
und die andere Hälfte der Gemeinde. \bibverse{28} Du sollst aber dem
HErrn heben von den Kriegsleuten, die ins Heer gezogen sind, je von fünf
Hunderten eine Seele, an Menschen, Rindern, Eseln und Schafen.
\bibverse{29} Von ihrer Hälfte sollst du es nehmen und dem Priester
Eleasar geben zur Hebe dem HErrn. \bibverse{30} Aber von der Hälfte der
Kinder Israel sollst du je ein Stück von fünfzigen nehmen, an Menschen,
Rindern, Eseln und Schafen und von allem Vieh, und sollst es den Leviten
geben, die des Dienstes warten an der Wohnung des HErrn. \bibverse{31}
Und Mose und Eleasar, der Priester, taten, wie der HErr dem Mose geboten
hatte. \bibverse{32} Und es war die übrige Ausbeute, die das Kriegsvolk
geraubt hatte, 675.000 Schafe, \bibverse{33} 72.000 Rinder,
\bibverse{34} 61.000 Esel \bibverse{35} und der Mädchen, die nicht
Männer erkannt hatten, 32.000 Seelen. \bibverse{36} Und die Hälfte, die
denen, welche ins Heer gezogen waren, gehörte, war an der Zahl 337.500
Schafe; \bibverse{37} davon wurden dem HErrn 675 Schafe. \bibverse{38}
Desgleichen 36.000 Rinder; davon wurden dem HErrn 72. \bibverse{39}
Desgleichen 30.500 Esel; davon wurden dem HErrn 61. \bibverse{40}
Desgleichen Menschenseelen, 16.000 Seelen; davon wurden dem HErrn 32
Seelen. \bibverse{41} Und Mose gab solche Hebe des HErrn dem Priester
Eleasar, wie ihm der HErr geboten hatte. \bibverse{42} Aber die andere
Hälfte, die Mose den Kindern Israel zuteilte von den Kriegsleuten,
\bibverse{43} nämlich die Hälfte, der Gemeinde zuständig, war auch
337.500 Schafe, \bibverse{44} 36.000 Rinder, \bibverse{45} 30.500 Esel
\bibverse{46} und 16.000 Menschenseelen. \bibverse{47} Und Mose nahm von
dieser Hälfte der Kinder Israel je ein Stück von fünfzigen, sowohl des
Viehs als der Menschen, und gab's den Leviten, die des Dienstes warteten
an der Wohnung des HErrn, wie der HErr dem Mose geboten hatte.
\bibverse{48} Und es traten herzu die Hauptleute über die Tausende des
Kriegsvolks, nämlich die über 1000 und über 100 waren, zu Mose
\bibverse{49} und sprachen zu ihm: Deine Knechte haben die Summe
genommen der Kriegsleute, die unter unseren Händen gewesen sind, und
fehlt nicht einer. \bibverse{50} Darum bringen wir dem HErrn Geschenke,
was ein jeglicher gefunden hat von goldenem Geräte, Ketten,
Armgeschmeide, Ringe, Ohrenringe und Spangen, dass unsere Seelen
versöhnt werden vor dem HErrn. \bibverse{51} Und Mose samt dem Priester
Eleasar nahm von ihnen das Gold von allerlei Geräte. \bibverse{52} Und
alles Goldes Hebe, das sie dem HErrn hoben, war 16.750 Lot von den
Hauptleuten über 1000 und 100. \bibverse{53} Denn die Kriegsleute hatten
geraubt ein jeglicher für sich. \bibverse{54} Und Mose mit Eleasar, dem
Priester, nahm das Gold von den Hauptleuten über 1000 und 100, und
brachten es in die Hütte des Stifts zum Gedächtnis der Kinder Israel vor
dem HErrn. \# 32 \bibverse{1} Die Kinder Ruben und die Kinder Gad hatten
sehr viel Vieh und sahen das Land Jaser und Gilead an als gute Stätte
für ihr Vieh \bibverse{2} und kamen und sprachen zu Mose und zu dem
Priester Eleasar und zu den Fürsten der Gemeinde: \bibverse{3} Das Land
Ataroth, Dibon, Jaser, Nimra, Hesbon, Eleale, Sebam, Nebo und Beon,
\bibverse{4} das der HErr geschlagen hat vor der Gemeinde Israel, ist
gut zur Weide; und wir, deine Knechte, haben Vieh. \bibverse{5} Und
sprachen weiter: Haben wir Gnade vor dir gefunden, so gib dieses Land
deinen Knechten zu eigen, so wollen wir nicht über den Jordan ziehen.
\bibverse{6} Mose sprach zu ihnen: Eure Brüder sollen in den Streit
ziehen, und ihr wollt hier bleiben? \bibverse{7} Warum macht ihr der
Kinder Israel Herzen abwendig, dass sie nicht hinüberziehen in das Land,
das ihnen der HErr geben wird? \bibverse{8} Also taten auch eure Väter,
da ich sie aussandte von Kades-Barnea, das Land zu schauen; \footnote{\textbf{32:8}
  4Mo 13,-1} \bibverse{9} und da sie hinaufgekommen waren bis an den
Bach Eskol und sahen das Land, machten sie das Herz der Kinder Israel
abwendig, dass sie nicht in das Land wollten, das ihnen der HErr geben
wollte. \bibverse{10} Und des HErrn Zorn ergrimmte zur selben Zeit, und
er schwur und sprach: \bibverse{11} Diese Leute, die aus Ägypten gezogen
sind, von zwanzig Jahren und darüber sollen wahrlich das Land nicht
sehen, das ich Abraham, Isaak und Jakob geschworen habe, darum dass sie
mir nicht treulich nachgefolgt sind; \bibverse{12} ausgenommen Kaleb,
den Sohn Jephunnes, des Kenisiters, und Josua, den Sohn Nuns; denn sie
sind dem HErrn treulich nachgefolgt. \bibverse{13} Also ergrimmte des
HErrn Zorn über Israel, und er ließ sie hin und her in der Wüste ziehen
40 Jahre, bis dass ein Ende ward all des Geschlechts, das übel getan
hatte vor dem HErrn. \bibverse{14} Und siehe, ihr seid aufgetreten an
eurer Väter Statt, dass der Sünder desto mehr seien und ihr auch den
Zorn und Grimm des HErrn noch mehr macht wider Israel. \bibverse{15}
Denn wo ihr euch von ihm wendet, so wird er auch noch länger sie lassen
in der Wüste, und ihr werdet dies Volk alles verderben. \bibverse{16} Da
traten sie herzu und sprachen: Wir wollen nur Schafhürden hier bauen für
unser Vieh und Städte für unsere Kinder; \bibverse{17} wir aber wollen
uns rüsten vornan vor den Kindern Israel her, bis dass wir sie bringen
an ihren Ort. Unsere Kinder sollen in den verschlossenen Städten bleiben
um der Einwohner willen des Landes. \bibverse{18} Wir wollen nicht
heimkehren, bis die Kinder Israel einnehmen ein jeglicher sein Erbe.
\bibverse{19} Denn wir wollen nicht mit ihnen erben jenseits des
Jordans, sondern unser Erbe soll uns diesseits des Jordan gegen Morgen
gefallen sein. \bibverse{20} Mose sprach zu Ihnen: Wenn ihr das tun
wollt, dass ihr euch rüstet zum Streit vor dem HErrn, \footnote{\textbf{32:20}
  Jos 1,13-15} \bibverse{21} so ziehet über den Jordan vor dem HErrn,
wer unter euch gerüstet ist, bis dass er seine Feinde austreibe von
seinem Angesicht \bibverse{22} und das Land untertan werde dem HErrn;
darnach sollt ihr umwenden und unschuldig sein vor dem HErrn und vor
Israel und sollt dieses Land also haben zu eigen vor dem HErrn.
\bibverse{23} Wo ihr aber nicht also tun wollt, siehe, so werdet ihr
euch an dem HErrn versündigen und werdet eurer Sünde innewerden, wenn
sie euch finden wird. \bibverse{24} So bauet nun Städte für eure Kinder
und Hürden für euer Vieh und tut, was ihr geredet habt. \bibverse{25}
Die Kinder Gad und die Kinder Ruben sprachen zu Mose: Deine Knechte
sollen tun, wie mein Herr geboten hat. \bibverse{26} Unsere Kinder,
Weiber, Habe und all unser Vieh sollen in den Städten Gileads sein;
\bibverse{27} wir aber, deine Knechte, wollen alle gerüstet zum Heer in
den Streit ziehen vor dem HErrn, wie mein Herr geredet hat.
\bibverse{28} Da gebot Mose ihrethalben dem Priester Eleasar und Josua,
dem Sohn Nuns, und den obersten Vätern der Stämme der Kinder Israel
\bibverse{29} und sprach zu ihnen: Wenn die Kinder Gad und die Kinder
Ruben mit euch über den Jordan ziehen, alle gerüstet zum Streit vor dem
HErrn, und das Land euch untertan ist, so gebet ihnen das Land Gilead zu
eigen; \bibverse{30} ziehen sie aber nicht mit euch gerüstet, so sollen
sie unter euch erben im Lande Kanaan. \bibverse{31} Die Kinder Gad und
die Kinder Ruben antworteten und sprachen: Wie der HErr redete zu deinen
Knechten, so wollen wir tun. \bibverse{32} Wir wollen gerüstet ziehen
vor dem HErrn ins Land Kanaan und unser Erbgut besitzen diesseits des
Jordans. \bibverse{33} Also gab Mose den Kindern Gad und den Kindern
Ruben und dem halben Stamm Manasses, des Sohnes Josephs, das Königreich
Sihons, des Königs der Amoriter, und das Königreich Ogs, des Königs von
Basan, das Land samt den Städten in dem ganzen Gebiete umher.
\footnote{\textbf{32:33} Jos 13,8-31} \bibverse{34} Da bauten die Kinder
Gad Dibon, Ataroth, Aroer, \bibverse{35} Atroth-Sophan, Jaser, Jogbeha,
\bibverse{36} Beth-Nimra und Beth-Haran, verschlossene Städte und
Schafhürden. \bibverse{37} Die Kinder Ruben bauten Hesbon, Eleale,
Kirjathaim, \bibverse{38} Nebo, Baal-Meon, und änderten die Namen, und
Sibma, und gaben den Städten Namen, die sie bauten. \bibverse{39} Und
die Kinder Machirs, des Sohnes Manasses, gingen nach Gilead und
gewannen's und vertrieben die Amoriter, die darin waren. \bibverse{40}
Da gab Mose dem Machir, dem Sohn Manasses, Gilead; und er wohnte darin.
\bibverse{41} Jair aber, der Sohn Manasses, ging hin und gewann ihre
Dörfer und hieß sie Dörfer Jairs. \bibverse{42} Nobah ging hin und
gewann Knath mit seinen Ortschaften und hieß sie Nobah nach seinem
Namen. \# 33 \bibverse{1} Das sind die Reisen der Kinder Israel, da sie
aus Ägyptenland gezogen sind mit ihrem Heer durch Mose und Aaron.
\bibverse{2} Und Mose beschrieb ihren Auszug, wie sie zogen nach dem
Befehl des HErrn, und dies sind die Reisen ihres Zuges. \bibverse{3} Sie
zogen aus von Raemses am 15. Tag des ersten Monats, dem zweiten Tage der
Ostern, durch eine hohe Hand, dass es alle Ägypter sahen, \footnote{\textbf{33:3}
  2Mo 1,11; 2Mo 14,8} \bibverse{4} als sie eben die Erstgeburt begruben,
die der HErr unter ihnen geschlagen hatte; denn der HErr hatte auch an
ihren Göttern Gericht geübt. \footnote{\textbf{33:4} 2Mo 12,12}
\bibverse{5} Als sie von Raemses auszogen, lagerten sie sich in Sukkoth.
\footnote{\textbf{33:5} 2Mo 12,37} \bibverse{6} Und zogen aus von
Sukkoth und lagerten sich in Etham, welches liegt an dem Ende der Wüste.
\footnote{\textbf{33:6} 2Mo 13,20} \bibverse{7} Von Etham zogen sie aus
und blieben in Pihachiroth, welches liegt gegen Baal-Zephon, und
lagerten sich gegen Migdol. \footnote{\textbf{33:7} 2Mo 14,2}
\bibverse{8} Von Hachiroth zogen sie aus und gingen mitten durchs Meer
in die Wüste und reisten drei Tagereisen in der Wüste Etham und lagerten
sich in Mara. \footnote{\textbf{33:8} 2Mo 14,22; 2Mo 15,23} \bibverse{9}
Von Mara zogen sie aus und kamen gen Elim; da waren zwölf Wasserbrunnen
und 70 Palmen; und lagerten sich daselbst. \footnote{\textbf{33:9} 2Mo
  15,27} \bibverse{10} Von Elim zogen sie aus und lagerten sich an das
Schilfmeer. \bibverse{11} Von dem Schilfmeer zogen sie aus und lagerten
sich in der Wüste Sin. \bibverse{12} Von der Wüste Sin zogen sie aus und
lagerten sich in Dophka. \bibverse{13} Von Dophka zogen sie aus und
lagerten sich in Alus. \bibverse{14} Von Alus zogen sie aus und lagerten
sich in Raphidim, daselbst hatte das Volk kein Wasser zu trinken.
\footnote{\textbf{33:14} 2Mo 17,1} \bibverse{15} Von Raphidim zogen sie
aus und lagerten sich in der Wüste Sinai. \footnote{\textbf{33:15} 2Mo
  19,1} \bibverse{16} Von Sinai zogen sie aus und lagerten sich bei den
Lustgräbern. \footnote{\textbf{33:16} 4Mo 11,34} \bibverse{17} Von den
Lustgräbern zogen sie aus und lagerten sich in Hazeroth. \footnote{\textbf{33:17}
  4Mo 11,35} \bibverse{18} Von Hazeroth zogen sie aus und lagerten sich
in Rithma. \footnote{\textbf{33:18} 4Mo 12,16} \bibverse{19} Von Rithma
zogen sie aus und lagerten sich in Rimmon-Perez. \bibverse{20} Von
Rimmon-Perez zogen sie aus und lagerten sich in Libna. \bibverse{21} Von
Libna zogen sie aus und lagerten sich in Rissa. \bibverse{22} Von Rissa
zogen sie aus und lagerten sich in Kehelatha. \bibverse{23} Von
Kehelatha zogen sie aus und lagerten sich im Gebirge Sepher.
\bibverse{24} Vom Gebirge Sepher zogen sie aus und lagerten sich in
Harada. \bibverse{25} Von Harada zogen sie aus und lagerten sich in
Makheloth. \bibverse{26} Von Makheloth zogen sie aus und lagerten sich
in Thahath. \bibverse{27} Von Thahath zogen sie aus und lagerten sich in
Tharah. \bibverse{28} Von Tharah zogen sie aus und lagerten sich in
Mithka. \bibverse{29} Von Mithka zogen sie aus und lagerten sich in
Hasmona. \bibverse{30} Von Hasmona zogen sie aus und lagerten sich in
Moseroth. \bibverse{31} Von Moseroth zogen sie aus und lagerten sich in
Bne-Jaakan. \bibverse{32} Von Bne-Jaakan zogen sie aus und lagerten sich
in Horgidgad. \bibverse{33} Von Horgidgad zogen sie aus und lagerten
sich in Jotbatha. \footnote{\textbf{33:33} 5Mo 10,7} \bibverse{34} Von
Jotbatha zogen sie aus und lagerten sich in Abrona. \bibverse{35} Von
Abrona zogen sie aus und lagerten sich in Ezeon-Geber. \bibverse{36} Von
Ezeon-Geber zogen sie aus und lagerten sich in der Wüste Zin, das ist
Kades. \bibverse{37} Von Kades zogen sie aus und lagerten sich an dem
Berge Hor, an der Grenze des Landes Edom. \footnote{\textbf{33:37} 4Mo
  20,22-29} \bibverse{38} Da ging der Priester Aaron auf den Berg Hor
nach dem Befehl des HErrn und starb daselbst im 40. Jahr des Auszugs der
Kinder Israel aus Ägyptenland am ersten Tage des fünften Monats,
\bibverse{39} da er 123 Jahre alt war. \bibverse{40} Und der König der
Kanaaniter zu Arad, der da wohnte gegen Mittag des Landes Kanaan, hörte,
dass die Kinder Israel kamen. \bibverse{41} Und von dem Berge Hor zogen
sie aus und lagerten sich in Zalmona. \bibverse{42} Von Zalmona zogen
sie aus und lagerten sich in Phunon. \bibverse{43} Von Phunon zogen sie
aus und lagerten sich in Oboth. \footnote{\textbf{33:43} 4Mo 21,10}
\bibverse{44} Von Oboth zogen sie aus und lagerten sich in Ije-Abarim,
in der Moabiter Gebiet. \footnote{\textbf{33:44} 4Mo 21,11}
\bibverse{45} Von Ijim zogen sie aus und lagerten sich in Dibon-Gad.
\bibverse{46} Von Dibon-Gad zogen sie aus und lagerten sich in
Almon-Diblathaim. \bibverse{47} Von Almon-Diblathaim zogen sie aus und
lagerten sich in dem Gebirge Abarim vor dem Nebo. \footnote{\textbf{33:47}
  4Mo 21,20} \bibverse{48} Von dem Gebirge Abarim zogen sie aus und
lagerten sich in das Gefilde der Moabiter an dem Jordan gegenüber
Jericho. \footnote{\textbf{33:48} 4Mo 22,1; 5Mo 32,49} \bibverse{49} Sie
lagerten sich aber am Jordan von Beth-Jesimoth an bis an Abel-Sittim, im
Gefilde der Moabiter. \footnote{\textbf{33:49} 4Mo 25,1} \bibverse{50}
Und der HErr redete mit Mose in dem Gefilde der Moabiter an dem Jordan
gegenüber Jericho und sprach: \bibverse{51} Rede mit den Kindern Israel
und sprich zu ihnen: Wenn ihr über den Jordan gegangen seid in das Land
Kanaan, \bibverse{52} so sollt ihr alle Einwohner vertreiben vor eurem
Angesicht und alle ihre Säulen und alle ihre gegossenen Bilder zerstören
und alle ihre Höhen vertilgen, \bibverse{53} dass ihr also das Land
einnehmet und darin wohnet; denn euch habe ich das Land gegeben, dass
ihr's einnehmet. \bibverse{54} Und sollt das Land austeilen durchs Los
unter eure Geschlechter. Denen, deren viele sind, sollt ihr desto mehr
zuteilen, und denen, deren wenige sind, sollt ihr desto weniger
zuteilen. Wie das Los einem jeglichen daselbst fällt, so soll er's
haben; nach den Stämmen eurer Väter sollt ihr's austeilen. \bibverse{55}
Werdet ihr aber die Einwohner des Landes nicht vertreiben vor eurem
Angesicht, so werden euch die, die ihr überbleiben lasst, zu Dornen
werden in euren Augen und zu Stacheln in euren Seiten und werden euch
drängen in dem Lande, darin ihr wohnet. \footnote{\textbf{33:55} Jos
  23,13} \bibverse{56} So wird's dann gehen, dass ich euch gleich tun
werde, wie ich gedachte ihnen zu tun. \# 34 \bibverse{1} Und der HErr
redete mit Mose und sprach: \bibverse{2} Gebiete den Kindern Israel und
sprich zu ihnen: Wenn ihr ins Land Kanaan kommt, so soll dies das Land
sein, das euch zum Erbteil fällt, das Land Kanaan nach seinen Grenzen.
\bibverse{3} Die Ecke gegen Mittag soll anfangen an der Wüste Zin bei
Edom, dass eure Grenze gegen Mittag sei vom Ende des Salzmeeres, das
gegen Morgen liegt, \footnote{\textbf{34:3} Jos 15,1} \bibverse{4} und
das die Grenze sich lenke mittagwärts von der Steige Akrabbim und gehe
durch Zin, und ihr Ausgang sei mittagwärts von Kades-Barnea und gelange
zum Dorf Adar und gehe durch Azmon \bibverse{5} und lenke sich von Azmon
an den Bach Ägyptens, und ihr Ende sei an dem Meer. \bibverse{6} Aber
die Grenze gegen Abend soll diese sein, nämlich das große Meer. Das sei
eure Grenze gegen Abend. \bibverse{7} Die Grenze gegen Mitternacht soll
diese sein: ihr sollt messen von dem großen Meer bis an den Berg Hor,
\bibverse{8} und von dem Berge Hor messen, bis man kommt gen Hamath,
dass der Ausgang der Grenze sei gen Zedad \bibverse{9} und die Grenze
ausgehe gen Siphron und ihr Ende sei am Dorf Enan. Das sei eure Grenze
gegen Mitternacht. \bibverse{10} Und sollt euch messen die Grenze gegen
Morgen vom Dorf Enan gen Sepham, \bibverse{11} und die Grenze gehe herab
von Sepham gen Ribla morgenwärts von Ain; darnach gehe sie herab und
lenke sich an die Seite des Meers Kinneret gegen Morgen \bibverse{12}
und komme herab an den Jordan, dass ihr Ende sei das Salzmeer. Das sei
euer Land mit seiner Grenze umher. \bibverse{13} Und Mose gebot den
Kindern Israel und sprach: Das ist das Land, das ihr durchs Los unter
euch teilen sollt, das der HErr geboten hat den neun Stämmen und dem
halben Stamm zu geben. \bibverse{14} Denn der Stamm der Kinder Ruben
nach ihren Vaterhäusern und der Stamm der Kinder Gad nach ihren
Vaterhäusern und der halbe Stamm Manasse haben ihr Teil genommen.
\footnote{\textbf{34:14} 4Mo 32,33} \bibverse{15} Also haben die zwei
Stämme und der halbe Stamm ihr Erbteil dahin, diesseits des Jordans
gegenüber Jericho gegen Morgen. \bibverse{16} Und der HErr redete mit
Mose und sprach: \bibverse{17} Das sind die Namen der Männer, die das
Land unter euch teilen sollen: der Priester Eleasar und Josua, der Sohn
Nuns. \bibverse{18} Dazu sollt ihr nehmen von einem jeglichen Stamm
einen Fürsten, das Land auszuteilen. \bibverse{19} Und das sind der
Männer Namen: Kaleb, der Sohn Jephunnes, des Stammes Juda; \footnote{\textbf{34:19}
  4Mo 13,6; 4Mo 13,30} \bibverse{20} Samuel, der Sohn Ammihuds, des
Stammes Simeon; \bibverse{21} Elidad, der Sohn Chislons, des Stammes
Benjamin; \bibverse{22} Bukki, der Sohn Joglis, Fürst des Stammes der
Kinder Dan; \bibverse{23} Hanniel, der Sohn Ephods, Fürst des Stammes
der Kinder Manasse, von den Kindern Joseph; \bibverse{24} Kemuel, der
Sohn Siphtans, Fürst des Stammes der Kinder Ephraim; \bibverse{25}
Elizaphan, der Sohn Parnachs, Fürst des Stammes der Kinder Sebulon;
\bibverse{26} Paltiel, der Sohn Assans, Fürst des Stammes der Kinder
Isaschar; \bibverse{27} Ahihud, der Sohn Selomis, Fürst des Stammes der
Kinder Asser; \bibverse{28} Pedahel, der Sohn Ammihuds, Fürst des
Stammes der Kinder Naphthali. \bibverse{29} Dies sind die, denen der
HErr gebot, dass sie den Kindern Israel Erbe austeilten im Lande Kanaan.
\# 35 \bibverse{1} Und der HErr redete mit Mose auf den Gefilde der
Moabiter am Jordan gegenüber Jericho und sprach: \bibverse{2} Gebiete
den Kindern Israel, dass sie den Leviten Städte geben von ihren
Erbgütern zur Wohnung; \bibverse{3} dazu die Vorstädte um die Städte her
sollt ihr den Leviten auch geben, dass sie in den Städten wohnen und in
den Vorstädten ihr Vieh und Gut und allerlei Tiere haben. \bibverse{4}
Die Weite aber der Vorstädte, die ihr den Leviten gebt, soll 1000 Ellen
draußen vor der Stadtmauer umher haben. \bibverse{5} So sollt ihr nun
messen außen an der Stadt von der Ecke gegen Morgen 2000 Ellen und von
der Ecke gegen Mittag 2000 Ellen und von der Ecke gegen Abend 2000 Ellen
und von der Ecke gegen Mitternacht 2000 Ellen, dass die Stadt in der
Mitte sei. Das sollen ihre Vorstädte sein. \bibverse{6} Und unter den
Städten, die ihr den Leviten geben werdet, sollt ihr sechs Freistädte
geben, dass dahinein fliehe, wer einen Totschlag getan hat. Über
dieselben sollt ihr noch 42 Städte geben, \footnote{\textbf{35:6} 2Mo
  21,13; 5Mo 4,41; 5Mo 19,2; 5Mo 19,9; Jos 20,-1} \bibverse{7} dass alle
Städte, die ihr den Leviten gebt, seien 48 mit ihren Vorstädten.
\bibverse{8} Und sollt derselben desto mehr geben von denen, die viel
besitzen unter den Kindern Israel, und desto weniger von denen, die
wenig besitzen; ein jeglicher nach seinem Erbteil, das ihm zugeteilt
wird, soll Städte den Leviten geben. \bibverse{9} Und der HErr redete
mit Mose und sprach: \bibverse{10} Rede mit den Kindern Israel und
sprich zu ihnen: Wenn ihr über den Jordan ins Land Kanaan kommt,
\bibverse{11} sollt ihr Städte auswählen, dass sie Freistädte seien,
wohin fliehe, wer einen Totschlag unversehens tut. \bibverse{12} Und
sollen unter euch solche Freistädte sein vor dem Bluträcher, dass der
nicht sterben müsse, der einen Totschlag getan hat, bis dass er vor der
Gemeinde vor Gericht gestanden sei. \bibverse{13} Und der Städte, die
ihr geben werdet zu Freistädten, sollen sechs sein. \bibverse{14} Drei
sollt ihr geben diesseits des Jordans und drei im Lande Kanaan.
\bibverse{15} Das sind die sechs Freistädte, den Kindern Israel und den
Fremdlingen und den Beisassen unter euch, dass dahin fliehe, wer einen
Totschlag getan hat unversehens. \bibverse{16} Wer jemand mit einem
Eisen schlägt, dass er stirbt, der ist ein Totschläger und soll des
Todes sterben. \bibverse{17} Wirft er ihn mit einem Stein, mit dem
jemand mag getötet werden, dass er davon stirbt, so ist er ein
Totschläger und soll des Todes sterben. \bibverse{18} Schlägt er ihn
aber mit einem Holz, mit dem jemand mag totgeschlagen werden, dass er
stirbt, so ist er ein Totschläger und soll des Todes sterben.
\bibverse{19} Der Rächer des Bluts soll den Totschläger zum Tode
bringen; wo er ihm begegnet, soll er ihn töten. \bibverse{20} Stößt er
ihn aus Hass oder wirft etwas auf ihn aus List, dass er stirbt,
\bibverse{21} oder schlägt ihn aus Feindschaft mit seiner Hand, dass er
stirbt, so soll er des Todes sterben, der ihn geschlagen hat; denn er
ist ein Totschläger. Der Rächer des Bluts soll ihn zum Tode bringen, wo
er ihm begegnet. \bibverse{22} Wenn er ihn aber ungefähr stößt, ohne
Feindschaft, oder wirft irgend etwas auf ihn unversehens \bibverse{23}
oder wirft irgendeinen Stein auf ihn, davon man sterben mag, und er
hat's nicht gesehen, also dass er stirbt, und er ist nicht sein Feind,
hat ihm auch kein Übles gewollt, \bibverse{24} so soll die Gemeinde
richten zwischen dem, der geschlagen hat, und dem Rächer des Bluts nach
diesen Rechten. \bibverse{25} Und die Gemeinde soll den Totschläger
erretten von der Hand des Bluträchers und soll ihn wiederkommen lassen
zu der Freistadt, dahin er geflohen war; und er soll daselbst bleiben,
bis dass der Hohepriester sterbe, den man mit dem heiligen Öl gesalbt
hat. \footnote{\textbf{35:25} 3Mo 21,10} \bibverse{26} Wird aber der
Totschläger aus seiner Freistadt Grenze gehen, dahin er geflohen ist,
\bibverse{27} und der Bluträcher findet ihn außerhalb der Grenze seiner
Freistadt und schlägt ihn tot, so soll er des Bluts nicht schuldig sein.
\bibverse{28} Denn er sollte in seiner Freistadt bleiben bis an den Tod
des Hohenpriesters, und nach des Hohenpriesters Tod wieder zum Lande
seines Erbguts kommen. \bibverse{29} Das soll euch ein Recht sein bei
euren Nachkommen, überall, wo ihr wohnt. \bibverse{30} Den Totschläger
soll man töten nach dem Mund zweier Zeugen. Ein Zeuge soll nicht
aussagen über eine Seele zum Tode. \bibverse{31} Und ihr sollt keine
Versühnung nehmen für die Seele des Totschlägers; denn er ist des Todes
schuldig, und er soll des Todes sterben. \bibverse{32} Und sollt keine
Versühnung nehmen für den, der zur Freistadt geflohen ist, dass er
wiederkomme, zu wohnen im Lande, bis der Priester sterbe. \bibverse{33}
Und schändet das Land nicht, darin ihr wohnet; denn wer blutschuldig
ist, der schändet das Land, und das Land kann vom Blut nicht versöhnt
werden, das darin vergossen wird, außer durch das Blut des, der es
vergossen hat. \footnote{\textbf{35:33} 1Mo 9,6} \bibverse{34}
Verunreinigt das Land nicht, darin ihr wohnet, darin ich auch wohne;
denn ich bin der HErr, der unter den Kindern Israel wohnt. \footnote{\textbf{35:34}
  2Mo 29,45}

\hypertarget{section-6}{%
\section{36}\label{section-6}}

\bibverse{1} Und die obersten Väter des Geschlechts der Kinder Gileads,
des Sohnes Machirs, der Manasses Sohn war, von den Geschlechtern der
Kinder Joseph, traten herzu und redeten vor Mose und vor den Fürsten,
den obersten Vätern der Kinder Israel, \bibverse{2} und sprachen: Meinem
Herrn hat der HErr geboten, dass man das Land zum Erbteil geben sollte
durchs Los den Kindern Israel; auch ward meinem Herrn geboten von dem
HErrn, dass man das Erbteil Zelophehads, unseres Bruders, seinen
Töchtern geben soll. \^{}\^{} \bibverse{3} Wenn sie jemand aus den
Stämmen der Kinder Israel zu Weibern nimmt, so wird unserer Väter
Erbteil weniger werden, und so viel sie haben, wird zu dem Erbteil
kommen des Stammes, dahin sie kommen; also wird das Los unseres Erbteils
verringert. \bibverse{4} Wenn nun das Halljahr der Kinder Israel kommt,
so wird ihr Erbteil zu dem Erbteil des Stammes kommen, da sie sind; also
wird das Erbteil des Stammes unserer Väter verringert, so viel sie
haben. \bibverse{5} Mose gebot den Kindern Israel nach dem Befehl des
HErrn und sprach: Der Stamm der Kinder Joseph hat recht geredet.
\bibverse{6} Das ist's, was der HErr gebietet den Töchtern Zelophehads
und spricht: Lass sie freien, wie es ihnen gefällt; allein dass sie
freien unter dem Geschlecht des Stammes ihres Vaters, \bibverse{7} auf
dass nicht die Erbteile der Kinder Israel fallen von einem Stamm zum
anderen; denn ein jeglicher unter den Kindern Israel soll anhangen an
dem Erbe des Stammes seiner Väter. \bibverse{8} Und alle Töchter, die
Erbteil besitzen unter den Stämmen der Kinder Israel, sollen freien
einen von dem Geschlecht des Stammes ihres Vaters, auf dass ein
jeglicher unter den Kindern Israel seiner Väter Erbe behalte
\bibverse{9} und nicht ein Erbteil von einem Stamm falle auf den
anderen, sondern ein jeglicher hange an seinem Erbe unter den Stämmen
der Kinder Israel. \bibverse{10} Wie der HErr dem Mose geboten hatte, so
taten die Töchter Zelophehads, \bibverse{11} Mahela, Thirza, Hogla,
Milka und Noa, und freiten die Kinder ihrer Vettern, \bibverse{12} des
Geschlechts der Kinder Manasses, des Sohnes Josephs. Also blieb ihr
Erbteil an dem Stamm des Geschlechts ihres Vaters. \bibverse{13} Das
sind die Gebote und Rechte, die der HErr gebot durch Mose den Kindern
Israel auf dem Gefilde der Moabiter am Jordan gegenüber Jericho.
