\hypertarget{section}{%
\section{1}\label{section}}

\bibverse{1} Paulus, ein Apostel Jesu Christi nach dem Befehl Gottes,
unseres Heilandes, und des Herrn Jesu Christi, der unsere Hoffnung ist,
\bibverse{2} dem Timotheus, meinem rechtschaffenen Sohn im Glauben:
Gnade, Barmherzigkeit, Friede von Gott, unserem Vater, und unserem Herrn
Jesus Christus! \footnote{\textbf{1:2} Apg 16,1-2; Tit 1,4}

\bibverse{3} Wie ich dich ermahnt habe, dass du zu Ephesus bliebest, da
ich nach Mazedonien zog, und gebötest etlichen, dass sie nicht anders
lehrten, \footnote{\textbf{1:3} Apg 20,1} \bibverse{4} und nicht
achthätten auf die Fabeln und Geschlechtsregister, die kein Ende haben
und Fragen aufbringen mehr denn Besserung zu Gott im Glauben;
\footnote{\textbf{1:4} 1Tim 4,7} \bibverse{5} denn die Hauptsumme des
Gebotes ist Liebe von reinem Herzen und von gutem Gewissen und von
ungefärbtem Glauben; \footnote{\textbf{1:5} Mt 22,37-40; Röm 13,10; Gal
  5,6} \bibverse{6} wovon etliche sind abgeirrt und haben sich umgewandt
zu unnützem Geschwätz, \footnote{\textbf{1:6} 1Tim 6,4; 1Tim 6,20}
\bibverse{7} wollen der Schrift Meister sein, und verstehen nicht, was
sie sagen oder was sie setzen.

\bibverse{8} Wir wissen aber, dass das Gesetz gut ist, so es jemand
recht braucht \bibverse{9} und weiß solches, dass dem Gerechten kein
Gesetz gegeben ist, sondern den Ungerechten und Ungehorsamen, den
Gottlosen und Sündern, den Unheiligen und Ungeistlichen, den
Vatermördern und Muttermördern, den Totschlägern \footnote{\textbf{1:9}
  1Kor 6,9-11} \bibverse{10} den Hurern, den Knabenschändern, den
Menschendieben, den Lügnern, den Meineidigen und wenn etwas mehr der
heilsamen Lehre zuwider ist, \bibverse{11} nach dem herrlichen
Evangelium des seligen Gottes, welches mir vertrauet ist.

\bibverse{12} Ich danke unserem Herrn Christus Jesus, der mich stark
gemacht und treu geachtet hat und gesetzt in das Amt, \bibverse{13} der
ich zuvor war ein Lästerer und ein Verfolger und ein Schmäher; aber mir
ist Barmherzigkeit widerfahren, denn ich habe es unwissend getan im
Unglauben. \bibverse{14} Es ist aber desto reicher gewesen die Gnade
unseres Herrn samt dem Glauben und der Liebe, die in Christo Jesu ist.
\bibverse{15} Das ist gewisslich wahr und ein teuer wertes Wort, dass
Christus Jesus gekommen ist in die Welt, die Sünder selig zu machen,
unter welchen ich der vornehmste bin. \footnote{\textbf{1:15} Lk 19,10}
\bibverse{16} Aber darum ist mir Barmherzigkeit widerfahren, auf dass an
mir vornehmlich Jesus Christus erzeigte alle Geduld, zum Vorbild denen,
die an ihn glauben sollten zum ewigen Leben. \bibverse{17} Aber Gott,
dem ewigen König, dem Unvergänglichen und Unsichtbaren und allein
Weisen, sei Ehre und Preis in Ewigkeit! Amen.

\bibverse{18} Dies Gebot befehle ich dir, mein Sohn Timotheus, nach den
vorigen Weissagungen über dich, dass du in ihnen eine gute Ritterschaft
übest \bibverse{19} und habest den Glauben und gutes Gewissen, welches
etliche von sich gestoßen und am Glauben Schiffbruch erlitten haben;
\footnote{\textbf{1:19} 1Tim 3,9; 1Tim 6,10} \bibverse{20} unter welchen
ist Hymenäus und Alexander, welche ich habe dem Satan übergeben, dass
sie gezüchtigt werden, nicht mehr zu lästern. \footnote{\textbf{1:20}
  1Kor 5,5; 2Tim 2,17}

\hypertarget{section-1}{%
\section{2}\label{section-1}}

\bibverse{1} So ermahne ich nun, dass man vor allen Dingen zuerst tue
Bitte, Gebet, Fürbitte und Danksagung für alle Menschen, \bibverse{2}
für die Könige und für alle Obrigkeit, auf dass wir ein ruhiges und
stilles Leben führen mögen in aller Gottseligkeit und Ehrbarkeit.
\bibverse{3} Denn solches ist gut und angenehm vor Gott, unserem
Heiland, \bibverse{4} welcher will, dass allen Menschen geholfen werde
und sie zur Erkenntnis der Wahrheit kommen. \footnote{\textbf{2:4} Hes
  18,23; Röm 11,32; 2Petr 3,9} \bibverse{5} Denn es ist ein Gott und ein
Mittler zwischen Gott und den Menschen, nämlich der Mensch Christus
Jesus, \footnote{\textbf{2:5} Hebr 9,15} \bibverse{6} der sich selbst
gegeben hat für alle zur Erlösung, dass solches zu seiner Zeit gepredigt
würde; \footnote{\textbf{2:6} Gal 1,4; Gal 2,20; Tit 2,14} \bibverse{7}
dazu ich gesetzt bin als Prediger und Apostel (ich sage die Wahrheit in
Christo und lüge nicht), als Lehrer der Heiden im Glauben und in der
Wahrheit. \footnote{\textbf{2:7} 2Tim 1,11; Gal 2,7-8}

\bibverse{8} So will ich nun, dass die Männer beten an allen Orten und
aufheben heilige Hände ohne Zorn und Zweifel. \footnote{\textbf{2:8} Jak
  1,6} \bibverse{9} Desgleichen dass die Weiber in zierlichem Kleide mit
Scham und Zucht sich schmücken, nicht mit Zöpfen oder Gold oder Perlen
oder köstlichem Gewand, \footnote{\textbf{2:9} 1Petr 3,3-5}
\bibverse{10} sondern, wie sich's ziemt den Weibern, die da
Gottseligkeit beweisen wollen, durch gute Werke. \footnote{\textbf{2:10}
  1Tim 5,10} \bibverse{11} Ein Weib lerne in der Stille mit aller
Untertänigkeit. \footnote{\textbf{2:11} Eph 5,22} \bibverse{12} Einem
Weibe aber gestatte ich nicht, dass sie lehre, auch nicht, dass sie des
Mannes Herr sei, sondern stille sei. \footnote{\textbf{2:12} 1Kor 14,34;
  1Mo 3,16} \bibverse{13} Denn Adam ist am ersten gemacht, darnach Eva.
\bibverse{14} Und Adam ward nicht verführt; das Weib aber ward verführt
und hat die Übertretung eingeführt. \bibverse{15} Sie wird aber selig
werden durch Kinderzeugen, wenn sie bleiben im Glauben und in der Liebe
und in der Heiligung samt der Zucht. \footnote{\textbf{2:15} 1Tim 5,14;
  Tit 2,4; Tit 1,2-5}

\hypertarget{section-2}{%
\section{3}\label{section-2}}

\bibverse{1} Das ist gewisslich wahr: Wenn jemand ein Bischofsamt
begehrt, der begehrt ein köstlich Werk. \footnote{\textbf{3:1} Apg
  20,28; Phil 1,1; Tit 1,5-9} \bibverse{2} Es soll aber ein Bischof
unsträflich sein, eines Weibes Mann, nüchtern, mäßig, sittig, gastfrei,
lehrhaft, \bibverse{3} nicht ein Weinsäufer, nicht raufen, nicht
unehrliche Hantierung treiben, sondern gelinde, nicht zänkisch, nicht
geizig, \bibverse{4} der seinem eigenen Hause wohl vorstehe, der
gehorsame Kinder habe mit aller Ehrbarkeit, \footnote{\textbf{3:4} 1Sam
  2,12} \bibverse{5} (so aber jemand seinem eigenen Hause nicht weiß
vorzustehen, wie wird er die Gemeinde Gottes versorgen?); \bibverse{6}
nicht ein Neuling, auf dass er sich nicht aufblase und ins Urteil des
Lästerers falle. \bibverse{7} Er muss aber auch ein gutes Zeugnis haben
von denen, die draußen sind, auf dass er nicht falle dem Lästerer in
Schmach und Strick.

\bibverse{8} Desgleichen die Diener sollen ehrbar sein, nicht
zweizüngig, nicht Weinsäufer, nicht unehrliche Hantierung treiben;
\bibverse{9} die das Geheimnis des Glaubens in reinem Gewissen haben.
\footnote{\textbf{3:9} 1Tim 1,19} \bibverse{10} Und diese lasse man
zuvor versuchen; darnach lasse man sie dienen, wenn sie unsträflich
sind. \bibverse{11} Desgleichen ihre Weiber sollen ehrbar sein, nicht
Lästerinnen, nüchtern, treu in allen Dingen. \bibverse{12} Die Diener
lass einen jeglichen sein eines Weibes Mann, die ihren Kindern wohl
vorstehen und ihren eigenen Häusern. \bibverse{13} Welche aber wohl
dienen, die erwerben sich selbst eine gute Stufe und eine große
Freudigkeit im Glauben an Christum Jesum.

\bibverse{14} Solches schreibe ich dir und hoffe, bald zu dir zu kommen;
\bibverse{15} wenn ich aber verzöge, dass du wissest, wie du wandeln
sollst in dem Hause Gottes, welches ist die Gemeinde des lebendigen
Gottes, ein Pfeiler und eine Grundfeste der Wahrheit. \footnote{\textbf{3:15}
  Eph 2,19-22} \bibverse{16} Und kündlich groß ist das gottselige
Geheimnis: Gott ist offenbart im Fleisch, gerechtfertigt im Geist,
erschienen den Engeln, gepredigt den Heiden, geglaubt von der Welt,
aufgenommen in die Herrlichkeit. \footnote{\textbf{3:16} Joh 1,14; Röm
  1,4; Eph 1,20-21; Apg 28,28; Mk 16,19}

\hypertarget{section-3}{%
\section{4}\label{section-3}}

\bibverse{1} Der Geist aber sagt deutlich, dass in den letzten Zeiten
werden etliche von dem Glauben abtreten und anhangen den verführerischen
Geistern und Lehren der Teufel \footnote{\textbf{4:1} Mt 24,24; 2Thes
  2,3; 2Tim 3,1; 2Petr 3,3; 1Jo 2,18; Jud 1,18} \bibverse{2} durch die,
die in Gleisnerei Lügen reden und Brandmal in ihrem Gewissen haben,
\bibverse{3} die da gebieten, nicht ehelich zu werden und zu meiden die
Speisen, die Gott geschaffen hat zu nehmen mit Danksagung, den Gläubigen
und denen, die die Wahrheit erkennen. \bibverse{4} Denn alle Kreatur
Gottes ist gut, und nichts ist verwerflich, das mit Danksagung empfangen
wird; \footnote{\textbf{4:4} 1Mo 1,31; Mt 15,11; Apg 10,15} \bibverse{5}
denn es wird geheiligt durch das Wort Gottes und Gebet.

\bibverse{6} Wenn du den Brüdern solches vorhältst, so wirst du ein
guter Diener Jesu Christi sein, auferzogen in den Worten des Glaubens
und der guten Lehre, bei welcher du immerdar gewesen bist. \bibverse{7}
Aber der ungeistlichen Altweiberfabeln entschlage dich; übe dich selbst
aber in der Gottseligkeit. \footnote{\textbf{4:7} 1Tim 6,20; 2Tim 2,16;
  2Tim 2,23; 2Tim 4,4; Tit 1,14; Tit 3,9} \bibverse{8} Denn die
leibliche Übung ist wenig nütz; aber die Gottseligkeit ist zu allen
Dingen nütz und hat die Verheißung dieses und des zukünftigen Lebens.
\footnote{\textbf{4:8} 1Tim 6,6; Hebr 13,9} \bibverse{9} Das ist
gewisslich wahr und ein teuer wertes Wort. \bibverse{10} Denn dahin
arbeiten wir auch und werden geschmäht, dass wir auf den lebendigen Gott
gehofft haben, welcher ist der Heiland aller Menschen, sonderlich der
Gläubigen. \bibverse{11} Solches gebiete und lehre.

\bibverse{12} Niemand verachte deine Jugend; sondern sei ein Vorbild den
Gläubigen im Wort, im Wandel, in der Liebe, im Geist, im Glauben, in der
Keuschheit. \footnote{\textbf{4:12} Tit 2,15; 2Tim 2,22} \bibverse{13}
Halte an mit Lesen, mit Ermahnen, mit Lehren, bis ich komme.
\bibverse{14} Lass nicht aus der Acht die Gabe, die dir gegeben ist
durch die Weissagung mit Handauflegung der Ältesten. \bibverse{15}
Dessen warte, damit gehe um, auf dass dein Zunehmen in allen Dingen
offenbar sei. \bibverse{16} Habe acht auf dich selbst und auf die Lehre;
beharre in diesen Stücken. Denn wo du solches tust, wirst du dich selbst
selig machen und die dich hören. \footnote{\textbf{4:16} Röm 11,14}

\hypertarget{section-4}{%
\section{5}\label{section-4}}

\bibverse{1} Einen Alten schilt nicht, sondern ermahne ihn als einen
Vater, die Jungen als Brüder, \footnote{\textbf{5:1} 3Mo 19,32; Tit 2,2}
\bibverse{2} die alten Weiber als Mütter, die jungen als Schwestern mit
aller Keuschheit.

\bibverse{3} Ehre die Witwen, welche rechte Witwen sind. \bibverse{4} So
aber eine Witwe Enkel oder Kinder hat, solche lass zuvor lernen, ihre
eigenen Häuser göttlich regieren und den Eltern Gleiches vergelten; denn
das ist wohl getan und angenehm vor Gott. \bibverse{5} Das ist aber eine
rechte Witwe, die einsam ist, die ihre Hoffnung auf Gott stellt und
bleibt am Gebet und Flehen Tag und Nacht. \footnote{\textbf{5:5} Lk 2,37}
\bibverse{6} Welche aber in Wollüsten lebt, die ist lebendig tot.
\bibverse{7} Solches gebiete, auf dass sie untadelig seien. \bibverse{8}
So aber jemand die Seinen, sonderlich seine Hausgenossen, nicht
versorgt, der hat den Glauben verleugnet und ist ärger denn ein Heide.

\bibverse{9} Lass keine Witwe erwählt werden unter sechzig Jahren, und
die da gewesen sei eines Mannes Weib, \bibverse{10} und die ein Zeugnis
habe guter Werke: wenn sie Kinder aufgezogen hat, wenn sie gastfrei
gewesen ist, wenn sie der Heiligen Füße gewaschen hat, wenn sie den
Trübseligen Handreichung getan hat, wenn sie allem guten Werk
nachgekommen ist. \footnote{\textbf{5:10} Joh 13,14; Hebr 13,2}

\bibverse{11} Der jungen Witwen aber entschlage dich; denn wenn sie geil
geworden sind wider Christum, so wollen sie freien \bibverse{12} und
haben ihr Urteil, dass sie den ersten Glauben gebrochen haben.
\bibverse{13} Daneben sind sie faul und lernen umlaufen durch die
Häuser; nicht allein aber sind sie faul, sondern auch geschwätzig und
vorwitzig und reden, was nicht sein soll. \bibverse{14} So will ich nun,
dass die jungen Witwen freien, Kinder zeugen, haushalten, dem
Widersacher keine Ursache geben zu schelten. \bibverse{15} Denn es sind
schon etliche umgewandt dem Satan nach. \bibverse{16} Wenn aber ein
Gläubiger oder Gläubige Witwen hat, der versorge sie und lasse die
Gemeinde nicht beschwert werden, auf dass die, die rechte Witwen sind,
mögen genug haben. \footnote{\textbf{5:16} Apg 6,1}

\bibverse{17} Die Ältesten, die wohl vorstehen, die halte man zwiefacher
Ehre wert, sonderlich die da arbeiten im Wort und in der Lehre.
\footnote{\textbf{5:17} Apg 14,23; Röm 12,8} \bibverse{18} Denn es
spricht die Schrift: „Du sollst nicht dem Ochsen das Maul verbinden, der
da drischt;`` und: „Ein Arbeiter ist seines Lohnes wert.`` \footnote{\textbf{5:18}
  1Kor 9,9; Lk 10,7}

\bibverse{19} Wider einen Ältesten nimm keine Klage an ohne zwei oder
drei Zeugen. \footnote{\textbf{5:19} 5Mo 19,15; Mt 18,16} \bibverse{20}
Die da sündigen, die strafe vor allen, auf dass sich auch die anderen
fürchten. \footnote{\textbf{5:20} Gal 2,14} \bibverse{21} Ich bezeuge
vor Gott und dem Herrn Jesus Christus und den auserwählten Engeln, dass
du solches haltest ohne eigenes Gutdünken und nichts tust nach Gunst.
\bibverse{22} Die Hände lege niemand zu bald auf, mache dich auch nicht
teilhaftig fremder Sünden. Halte dich selber keusch.

\bibverse{23} Trinke nicht mehr Wasser, sondern brauche auch ein wenig
Wein um deines Magens willen und weil du oft krank bist.

\bibverse{24} Etlicher Menschen Sünden sind offenbar, dass man sie zuvor
richten kann; bei etlichen aber werden sie hernach offenbar.
\bibverse{25} Desgleichen auch etlicher gute Werke sind zuvor offenbar,
und die anderen bleiben auch nicht verborgen. \# 6 \bibverse{1} Die
Knechte, die unter dem Joch sind, sollen ihre Herren aller Ehre wert
halten, auf dass nicht der Name Gottes und die Lehre verlästert werde.
\footnote{\textbf{6:1} Eph 6,5; Tit 2,9; Tit 1,2-10} \bibverse{2} Welche
aber gläubige Herren haben, sollen sie nicht verachten, weil sie Brüder
sind, sondern sollen viel mehr dienstbar sein, dieweil sie gläubig und
geliebt und der Wohltat teilhaftig sind. Solches lehre und ermahne.
\footnote{\textbf{6:2} Eph 6,5-8; Phim 1,16}

\bibverse{3} Wenn jemand anders lehrt und bleibt nicht bei den heilsamen
Worten unseres Herrn Jesu Christi und bei der Lehre, die gemäß ist der
Gottseligkeit, \footnote{\textbf{6:3} Gal 1,6-9; 2Tim 1,13} \bibverse{4}
der ist aufgeblasen und weiß nichts, sondern hat die Seuche der Fragen
und Wortkriege, aus welchen entspringt Neid, Hader, Lästerung, böser
Argwohn. \footnote{\textbf{6:4} 2Tim 2,14; Tit 3,10; Tit 1,3-11}
\bibverse{5} Schulgezänke solcher Menschen, die zerrüttete Sinne haben
und der Wahrheit beraubt sind, die da meinen, Gottseligkeit sei ein
Gewerbe. Tue dich von solchen! \footnote{\textbf{6:5} 1Tim 4,8; Mt
  6,25-34; Phil 4,11-12; Hebr 13,5}

\bibverse{6} Es ist aber ein großer Gewinn, wer gottselig ist und lässet
sich genügen. \bibverse{7} Denn wir haben nichts in die Welt gebracht;
darum offenbar ist, wir werden auch nichts hinausbringen. \bibverse{8}
Wenn wir aber Nahrung und Kleider haben, so lasset uns genügen.
\footnote{\textbf{6:8} Spr 30,8} \bibverse{9} Denn die da reich werden
wollen, die fallen in Versuchung und Stricke und viel törichte und
schädliche Lüste, welche versenken die Menschen ins Verderben und
Verdammnis. \footnote{\textbf{6:9} Spr 28,22; Mt 13,22} \bibverse{10}
Denn Geiz ist eine Wurzel alles Übels; das hat etliche gelüstet und sind
vom Glauben irregegangen und machen sich selbst viel Schmerzen.
\footnote{\textbf{6:10} 1Tim 1,19; Eph 5,5}

\bibverse{11} Aber du, Gottesmensch, fliehe solches! Jage aber nach --
der Gerechtigkeit, der Gottseligkeit, dem Glauben, der Liebe, der
Geduld, der Sanftmut; \footnote{\textbf{6:11} 2Tim 2,22; 2Tim 3,17}
\bibverse{12} kämpfe den guten Kampf des Glaubens; ergreife das ewige
Leben, dazu du auch berufen bist und bekannt hast ein gutes Bekenntnis
vor vielen Zeugen. \footnote{\textbf{6:12} 1Tim 1,18; 1Tim 4,14; 1Kor
  9,25-26; 2Tim 4,7; Hebr 3,1} \bibverse{13} Ich gebiete dir vor Gott,
der alle Dinge lebendig macht, und vor Christo Jesu, der unter Pontius
Pilatus bezeugt hat ein gutes Bekenntnis, \footnote{\textbf{6:13} Joh
  18,36-37; Offb 1,5} \bibverse{14} dass du haltest das Gebot ohne
Flecken, untadelig, bis auf die Erscheinung unseres Herrn Jesu Christi,
\bibverse{15} welche wird zeigen zu seiner Zeit der Selige und allein
Gewaltige, der König aller Könige und Herr aller Herren, \footnote{\textbf{6:15}
  5Mo 10,17; Offb 17,14} \bibverse{16} der allein Unsterblichkeit hat,
der da wohnt in einem Licht, da niemand zukommen kann, welchen kein
Mensch gesehen hat noch sehen kann; dem sei Ehre und ewiges Reich! Amen.
\footnote{\textbf{6:16} 2Mo 33,20; Joh 1,18}

\bibverse{17} Den Reichen von dieser Welt gebiete, dass sie nicht stolz
seien, auch nicht hoffen auf den ungewissen Reichtum, sondern auf den
lebendigen Gott, der uns dargibt reichlich, allerlei zu genießen;
\footnote{\textbf{6:17} Ps 62,11; Lk 12,15-21} \bibverse{18} dass sie
Gutes tun, reich werden an guten Werken, gern geben, behilflich seien,
\bibverse{19} Schätze sammeln, sich selbst einen guten Grund aufs
Zukünftige, dass sie ergreifen das wahre Leben.

\bibverse{20} O Timotheus! bewahre, was dir vertraut ist, und meide die
ungeistlichen, losen Geschwätze und das Gezänke der falsch berühmten
Kunst, \^{}\^{} \bibverse{21} welche etliche vorgeben und gehen vom
Glauben irre. Die Gnade sei mit dir! Amen.
