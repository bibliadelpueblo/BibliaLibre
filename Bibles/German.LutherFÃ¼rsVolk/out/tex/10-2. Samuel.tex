\hypertarget{section}{%
\section{1}\label{section}}

\bibverse{1} Nach dem Tode Sauls, da David von der Amalekiter Schlacht
wiedergekommen und zwei Tage zu Ziklag geblieben war, \bibverse{2}
siehe, da kam am dritten Tage ein Mann aus dem Heer von Saul mit
zerrissenen Kleidern und Erde auf seinem Haupt. Und da er zu David kam,
fiel er zur Erde und beugte sich nieder.

\bibverse{3} David aber sprach zu ihm: Wo kommst du her? Er sprach zu
ihm: Aus dem Heer Israels bin ich entronnen.

\bibverse{4} David sprach zu ihm: Sage mir, wie geht es zu? Er sprach:
Das Volk ist geflohen vom Streit, und ist viel Volks gefallen; dazu ist
auch Saul tot und sein Sohn Jonathan.

\bibverse{5} David sprach zu dem Jüngling, der ihm solches sagte: Woher
weißt du, dass Saul und sein Sohn Jonathan tot sind?

\bibverse{6} Der Jüngling, der ihm solches sagte, sprach: Ich kam von
ungefähr aufs Gebirge Gilboa, und siehe, Saul lehnte sich auf seinen
Spieß, und die Wagen und Reiter jagten hinter ihm her.

\bibverse{7} Und er wandte sich um und sah mich und rief mich. Und ich
sprach: Hier bin ich.

\bibverse{8} Und er sprach zu mir: Wer bist du? Ich sprach zu ihm: Ich
bin ein Amalekiter. \bibverse{9} Und er sprach zu mir: Tritt zu mir und
töte mich; denn ich bin bedrängt umher, und mein Leben ist noch ganz in
mir. \bibverse{10} Da trat ich zu ihm und tötete ihn; denn ich wusste
wohl, dass er nicht leben konnte nach seinem Fall; und nahm die Krone
von seinem Haupt und das Armgeschmeide von seinem Arm und habe es
hergebracht zu dir, meinem Herrn.

\bibverse{11} Da fasste David seine Kleider und zerriss sie, und alle
Männer, die bei ihm waren, \footnote{\textbf{1:11} 1Mo 37,29}
\bibverse{12} und trugen Leid und weinten und fasteten bis an den Abend
über Saul und Jonathan, seinen Sohn, und über das Volk des HErrn und
über das Haus Israel, dass sie durchs Schwert gefallen waren.
\footnote{\textbf{1:12} 1Sam 31,13}

\bibverse{13} Und David sprach zu dem Jüngling, der es ihm ansagte: Wo
bist du her? Er sprach: Ich bin eines Fremdlings, eines Amalekiters,
Sohn.

\bibverse{14} David sprach zu ihm: Wie, dass du dich nicht gefürchtet
hast, deine Hand zu legen an den Gesalbten des HErrn, ihn zu verderben!
\footnote{\textbf{1:14} 1Sam 24,7}

\bibverse{15} Und David sprach zu seiner Jünglinge einem: Herzu, und
schlag ihn! Und er schlug ihn, dass er starb. \footnote{\textbf{1:15}
  2Sam 4,10; 2Sam 4,12} \bibverse{16} Da sprach David zu ihm: Dein Blut
sei über deinem Kopf; denn dein Mund hat wider dich selbst geredet und
gesprochen: Ich habe den Gesalbten des HErrn getötet. \footnote{\textbf{1:16}
  1Kö 2,23; 1Kö 2,33}

\bibverse{17} Und David klagte diese Klage über Saul und Jonathan,
seinen Sohn, \bibverse{18} und befahl, man sollte die Kinder Juda das
Bogenlied lehren. Siehe, es steht geschrieben im Buch der Redlichen:
\bibverse{19} „Die Edelsten in Israel sind auf deiner Höhe erschlagen.
Wie sind die Helden gefallen! \bibverse{20} Sagt's nicht an zu Gath,
verkündet's nicht auf den Gassen zu Askalon, dass sich nicht freuen die
Töchter der Philister, dass nicht frohlocken die Töchter der
Unbeschnittenen. \footnote{\textbf{1:20} Mi 1,10; 1Sam 18,6}
\bibverse{21} Ihr Berge zu Gilboa, es müsse weder tauen noch regnen auf
euch, noch Äcker sein, davon Hebopfer kommen; denn daselbst ist den
Helden ihr Schild abgeschlagen, der Schild Sauls, als wäre er nicht
gesalbt mit Öl. \footnote{\textbf{1:21} 4Mo 15,18-21} \bibverse{22} Der
Bogen Jonathans hat nie gefehlt, und das Schwert Sauls ist nie leer
wiedergekommen von dem Blut der Erschlagenen und vom Fett der Helden.
\bibverse{23} Saul und Jonathan, holdselig und lieblich in ihrem Leben,
sind auch im Tode nicht geschieden; schneller waren sie denn die Adler
und stärker denn die Löwen. \bibverse{24} Ihr Töchter Israels, weinet
über Saul, der euch kleidete mit Scharlach säuberlich und schmückte euch
mit goldenen Kleinoden an euren Kleidern. \bibverse{25} Wie sind die
Helden so gefallen im Streit! Jonathan ist auf deinen Höhen erschlagen.
\bibverse{26} Es ist mir leid um dich, mein Bruder Jonathan: ich habe
große Freude und Wonne an dir gehabt; deine Liebe ist mir sonderlicher
gewesen, denn Frauenliebe ist. \bibverse{27} Wie sind die Helden
gefallen und die Streitbaren umgekommen!{}`` \# 2 \bibverse{1} Nach
dieser Geschichte fragte David den HErrn und sprach: Soll ich hinauf in
der Städte Judas eine ziehen? Und der HErr sprach zu ihm: Zieh hinauf!
David sprach: Wohin? Er sprach: Gen Hebron. \footnote{\textbf{2:1} 1Sam
  30,8}

\bibverse{2} Also zog David dahin mit seinen zwei Weibern, Ahinoam, der
Jesreelitin, und Abigail, Nabals, des Karmeliten, Weib. \footnote{\textbf{2:2}
  1Sam 25,42-43}

\bibverse{3} Dazu die Männer, die bei ihm waren, führte David hinauf,
einen jeglichen mit seinem Hause, und sie wohnten in den Städten
Hebrons.

\bibverse{4} Und die Männer Judas kamen und salbten daselbst David zum
König über das Haus Juda. Und da es David ward angesagt, dass die von
Jabes in Gilead Saul begraben hatten, \footnote{\textbf{2:4} 2Sam 5,3;
  1Sam 16,13; 1Sam 31,12}

\bibverse{5} sandte er Boten zu ihnen und ließ ihnen sagen: Gesegnet
seid ihr dem HErrn, dass ihr solche Barmherzigkeit an eurem Herrn, Saul,
getan und ihn begraben habt. \bibverse{6} So tue nun an euch der HErr
Barmherzigkeit und Treue; und ich will euch auch Gutes tun, darum dass
ihr solches getan habt. \bibverse{7} So seien nun eure Hände getrost,
und seiet freudig; denn euer Herr, Saul, ist tot; so hat mich das Haus
Juda zum König gesalbt über sich.

\bibverse{8} Abner aber, der Sohn Ners, der Sauls Feldhauptmann war,
nahm Is-Boseth, Sauls Sohn, und führte ihn gen Mahanaim \bibverse{9} und
machte ihn zum König über Gilead, über die Asuriter, über Jesreel,
Ephraim, Benjamin und über ganz Israel. \bibverse{10} Und Is-Boseth,
Sauls Sohn, war 40 Jahre alt, da er König ward über Israel, und regierte
zwei Jahre. Aber das Haus Juda hielt es mit David. \bibverse{11} Die
Zeit aber, da David König war zu Hebron über das Haus Juda, war sieben
Jahre und sechs Monate.

\bibverse{12} Und Abner, der Sohn Ners, zog aus samt den Knechten
Is-Boseths, des Sohnes Sauls, von Mahanaim, gen Gibeon; \bibverse{13}
und Joab, der Zeruja Sohn, zog aus samt den Knechten Davids; und sie
stießen aufeinander am Teich zu Gibeon, und lagerten sich diese auf
dieser Seite des Teichs, jene auf jener Seite. \bibverse{14} Und Abner
sprach zu Joab: Lass sich die Leute aufmachen und vor uns spielen. Joab
sprach: Es gilt wohl.

\bibverse{15} Da machten sich auf und gingen hin an der Zahl zwölf aus
Benjamin auf Is-Boseths Teil, des Sohnes Sauls, und zwölf von den
Knechten Davids. \bibverse{16} Und ein jeglicher ergriff den anderen bei
dem Kopf und stieß ihm sein Schwert in seine Seite, und fielen
miteinander; daher der Ort genannt wird: Helkath-Hazzurim, der zu Gibeon
ist. \bibverse{17} Und es erhob sich ein sehr harter Streit des Tages.
Abner aber und die Männer Israels wurden geschlagen vor den Knechten
Davids. \bibverse{18} Es waren aber drei Söhne der Zeruja daselbst:
Joab, Abisai und Asahel. Asahel aber war von leichten Füßen wie ein Reh
auf dem Felde \footnote{\textbf{2:18} 1Chr 2,16} \bibverse{19} und jagte
Abner nach und wich nicht weder zur Rechten noch zur Linken von Abner.

\bibverse{20} Da wandte sich Abner um und sprach: Bist du Asahel? Er
sprach: Ja.

\bibverse{21} Abner sprach zu ihm: Hebe dich entweder zur Rechten oder
zur Linken und nimm für dich der Leute einen und nimm ihm seine Waffen.
Aber Asahel wollte nicht von ihm ablassen.

\bibverse{22} Da sprach Abner weiter zu Asahel: Hebe dich von mir! Warum
willst du, dass ich dich zu Boden schlage? Und wie dürfte ich mein
Antlitz aufheben vor deinem Bruder Joab? \bibverse{23} Aber er weigerte
sich zu weichen. Da stach ihn Abner mit dem Schaft des Spießes in seinen
Bauch, dass der Spieß hinten ausging; und er fiel daselbst und starb vor
ihm. Und wer an den Ort kam, da Asahel tot lag, der stand still.

\bibverse{24} Aber Joab und Abisai jagten Abner nach, bis die Sonne
unterging. Und da sie kamen auf den Hügel Amma, der vor Giah liegt auf
dem Wege zur Wüste Gibeon, \bibverse{25} versammelten sich die Kinder
Benjamin hinter Abner her und wurden ein Haufe und traten auf eines
Hügels Spitze. \bibverse{26} Und Abner rief zu Joab und sprach: Soll
denn das Schwert ohne Ende fressen? Weißt du nicht, dass hernach möchte
mehr Jammer werden? Wie lange willst du dem Volk nicht sagen, dass es
ablasse von seinen Brüdern?

\bibverse{27} Joab sprach: So wahr Gott lebt, hättest du heute morgen so
gesagt, das Volk hätte ein jeglicher von seinem Bruder abgelassen.
\bibverse{28} Und Joab blies die Posaune, und alles Volk stand still und
jagten nicht mehr Israel nach und stritten auch nicht mehr.
\bibverse{29} Abner aber und seine Männer gingen die ganze Nacht über
das Blachfeld und gingen über den Jordan und wandelten durchs ganze
Bithron und kamen gen Mahanaim.

\bibverse{30} Joab aber wandte sich von Abner und versammelte das ganze
Volk; und es fehlten an den Knechten Davids 19 Mann und Asahel.
\bibverse{31} Aber die Knechte Davids hatten geschlagen unter Benjamin
und den Männern Abners, dass 360 Mann waren tot geblieben. \bibverse{32}
Und sie hoben Asahel auf und begruben ihn in seines Vaters Grab zu
Bethlehem. Und Joab mit seinen Männern gingen die ganze Nacht, dass
ihnen das Licht anbrach zu Hebron. \# 3 \bibverse{1} Und es war ein
langer Streit zwischen dem Hause Sauls und dem Hause Davids. David aber
nahm immer mehr zu, und das Haus Saul nahm immer mehr ab. \footnote{\textbf{3:1}
  2Sam 5,10} \bibverse{2} Und es wurden David Kinder geboren zu Hebron:
sein erstgeborener Sohn: Amnon, von Ahinoam, der Jesreelitin;
\footnote{\textbf{3:2} 1Chr 3,1-4; 2Sam 13,1} \bibverse{3} der zweite:
Chileab, von Abigail, Nabals Weib, des Karmeliten; der dritte: Absalom,
der Sohn Maachas, der Tochter Thalmais, des Königs zu Gessur;
\bibverse{4} der vierte: Adonia, der Sohn der Haggith; der fünfte:
Sephatja, der Sohn der Abital; \footnote{\textbf{3:4} 1Kö 1,5}
\bibverse{5} der sechste: Jethream, von Egla, dem Weibe Davids. Diese
sind David geboren zu Hebron.

\bibverse{6} Als nun der Streit war zwischen dem Hause Sauls und dem
Hause Davids, stärkte Abner das Haus Sauls. \bibverse{7} Und Saul hatte
ein Kebsweib, die hieß Rizpa, eine Tochter Ajas. Und Is-Boseth sprach zu
Abner: Warum hast du dich getan zu meines Vaters Kebsweib?

\bibverse{8} Da ward Abner sehr zornig über diese Worte Is-Boseths und
sprach: Bin ich denn ein Hundskopf, der ich wider Juda an dem Hause
Sauls, deines Vaters, und an seinen Brüdern und Freunden Barmherzigkeit
tue und habe dich nicht in Davids Hände gegeben? Und du rechnest mir
heute eine Missetat zu um ein Weib? \bibverse{9} Gott tue Abner dies und
das, wenn ich nicht tue, wie der HErr dem David geschworen hat,
\bibverse{10} dass das Königreich vom Hause Sauls genommen werde und der
Stuhl Davids aufgerichtet werde über Israel und Juda von Dan bis gen
Beer-Seba.

\bibverse{11} Da konnte er fürder ihm kein Wort mehr antworten, so
fürchtete er sich vor ihm.

\bibverse{12} Und Abner sandte Boten zu David für sich und ließ ihm
sagen: Wes ist das Land? Und sprach: Mache deinen Bund mit mir; siehe,
meine Hand soll mit dir sein, dass ich zu dir kehre das ganze Israel.

\bibverse{13} Er sprach: Wohl, ich will einen Bund mit dir machen. Aber
eins bitte ich von dir, dass du mein Angesicht nicht sehest, du bringest
denn zuvor zu mir Michal, Sauls Tochter, wenn du kommst, mein Angesicht
zu sehen. \bibverse{14} Auch sandte David Boten zu Is-Boseth, dem Sohn
Sauls, und ließ ihm sagen: Gib mir mein Weib Michal, die ich mir verlobt
habe mit 100 Vorhäuten der Philister. \footnote{\textbf{3:14} 1Sam
  18,25-27}

\bibverse{15} Is-Boseth sandte hin und ließ sie nehmen von dem Mann
Paltiel, dem Sohn des Lais. \bibverse{16} Und ihr Mann ging mit ihr und
weinte hinter ihr bis gen Bahurim. Da sprach Abner zu ihm: Kehre um und
gehe hin! Und er kehrte um.

\bibverse{17} Und Abner hatte eine Rede mit den Ältesten in Israel und
sprach: Ihr habt schon längst nach David getrachtet, dass er König wäre
über euch. \bibverse{18} So tut's nun; denn der HErr hat von David
gesagt: Ich will mein Volk Israel erretten durch die Hand Davids, meines
Knechtes, von der Philister Hand und von aller ihrer Feinde Hand.

\bibverse{19} Auch redete Abner vor den Ohren Benjamins und ging auch
hin, zu reden vor den Ohren Davids zu Hebron alles, was Israel und dem
ganzen Hause Benjamin wohl gefiel. \bibverse{20} Da nun Abner gen Hebron
zu David kam und mit ihm 20 Mann, machte ihnen David ein Mahl.
\bibverse{21} Und Abner sprach zu David: Ich will mich aufmachen und
hingehen, dass ich das ganze Israel zu meinem Herrn, dem König, sammle
und dass sie einen Bund mit dir machen, auf dass du König seist, wie es
deine Seele begehrt. Also ließ David Abner von sich, dass er hinginge
mit Frieden.

\bibverse{22} Und siehe, die Knechte Davids und Joab kamen von einem
Streifzuge und brachten mit sich eine große Beute. Abner aber war nicht
mehr bei David zu Hebron, sondern er hatte ihn von sich gelassen, dass
er mit Frieden weggegangen war. \bibverse{23} Da aber Joab und das ganze
Heer mit ihm war gekommen, ward ihm angesagt, dass Abner, der Sohn Ners,
zum König gekommen war und er hatte ihn von sich gelassen, dass er mit
Frieden war weggegangen.

\bibverse{24} Da ging Joab zum König hinein und sprach: Was hast du
getan? Siehe, Abner ist zu dir gekommen; warum hast du ihn von dir
gelassen, dass er ist weggegangen? \bibverse{25} Kennst du Abner, den
Sohn Ners, nicht? Denn er ist gekommen, dich zu überreden, dass er
erkennte deinen Ausgang und Eingang und erführe alles, was du tust.

\bibverse{26} Und da Joab von David ausging, sandte er Boten Abner nach,
dass sie ihn wiederum holten von Bor-Hassira; und David wusste nichts
darum. \bibverse{27} Als nun Abner wieder gen Hebron kam, führte ihn
Joab mitten unter das Tor, dass er heimlich mit ihm redete, und stach
ihn daselbst in den Bauch, dass er starb, um seines Bruders Asahel Bluts
willen. \bibverse{28} Da das David hernach erfuhr, sprach er: Ich bin
unschuldig und mein Königreich vor dem HErrn ewiglich an dem Blut
Abners, des Sohnes Ners; \bibverse{29} es falle aber auf den Kopf Joabs
und auf seines Vaters ganzes Haus, und müsse nicht aufhören im Hause
Joabs, der einen Eiterfluss und Aussatz habe und am Stabe gehe und
durchs Schwert falle und an Brot Mangel habe. \bibverse{30} Also
erwürgten Joab und sein Bruder Abisai Abner, darum dass er ihren Bruder
Asahel getötet hatte im Streit zu Gibeon.

\bibverse{31} David aber sprach zu Joab und allem Volk, das mit ihm war:
Zerreißet eure Kleider und gürtet Säcke um euch und traget Leid um
Abner! Und der König ging dem Sarge nach. \bibverse{32} Und da sie Abner
begruben zu Hebron, hob der König seine Stimme auf und weinte bei dem
Grabe Abners, und weinte auch alles Volk. \footnote{\textbf{3:32} 1Sam
  30,4} \bibverse{33} Und der König klagte um Abner und sprach: Musste
Abner sterben, wie ein Ruchloser stirbt? \bibverse{34} Deine Hände waren
nicht gebunden, deine Füße waren nicht in Fesseln gesetzt; du bist
gefallen, wie man vor bösen Buben fällt. Da beweinte ihn alles Volk noch
mehr.

\bibverse{35} Da nun alles Volk hineinkam, mit David zu essen, da es
noch hoch am Tage war, schwur David und sprach: Gott tue mir dies und
das, wo ich Brot oder etwas koste, ehe die Sonne untergeht.

\bibverse{36} Und alles Volk erkannte es, und gefiel ihnen auch wohl,
wie alles, was der König tat, dem ganzen Volke wohl gefiel;
\bibverse{37} und alles Volk und ganz Israel merkten des Tages, dass es
nicht vom König war, dass Abner, der Sohn Ners, getötet ward.
\bibverse{38} Und der König sprach zu seinen Knechten: Wisset ihr nicht,
dass auf diesen Tag ein Fürst und Großer gefallen ist in Israel?
\bibverse{39} Ich aber bin noch zart und erst gesalbt zum König. Aber
die Männer, die Kinder der Zeruja, sind mir verdrießlich. Der HErr
vergelte dem, der Böses tut, nach seiner Bosheit. \# 4 \bibverse{1} Da
aber der Sohn Sauls hörte, dass Abner zu Hebron tot wäre, wurden seine
Hände lass, und ganz Israel erschrak. \bibverse{2} Es waren aber zwei
Männer, Hauptleute der streifenden Rotten unter dem Sohn Sauls; einer
hieß Baana, der andere Rechab, Söhne Rimmons, des Beerothiters, aus den
Kindern Benjamin. (Denn Beeroth ward auch unter Benjamin gerechnet;
\bibverse{3} und die Beerothiter waren geflohen gen Gitthaim und wohnten
daselbst gastweise bis auf den heutigen Tag.)

\bibverse{4} Auch hatte Jonathan, der Sohn Sauls, einen Sohn, der war
lahm an den Füßen, und war fünf Jahre alt, da das Geschrei von Saul und
Jonathan aus Jesreel kam und seine Amme ihn aufhob und floh; und indem
sie eilte und floh, fiel er und ward hinkend; und er hieß Mephiboseth.
\footnote{\textbf{4:4} 2Sam 9,3}

\bibverse{5} So gingen nun hin die Söhne Rimmons, des Beerothiters,
Rechab und Baana, und kamen zum Hause Is-Boseths, da der Tag am
heißesten war; und er lag auf seinem Lager am Mittag. \bibverse{6} Und
sie kamen ins Haus, Weizen zu holen, und stachen ihn in den Bauch und
entrannen. \bibverse{7} Denn da sie ins Haus kamen, lag er auf seinem
Bette in seiner Schlafkammer; und sie stachen ihn tot und hieben ihm den
Kopf ab und nahmen seinen Kopf und gingen hin des Weges auf dem
Blachfelde die ganze Nacht \bibverse{8} und brachten das Haupt
Is-Boseths zu David gen Hebron und sprachen zum König: Siehe, da ist das
Haupt Is-Boseths, Sauls Sohnes, deines Feindes, der nach deiner Seele
stand; der HErr hat heute meinen Herrn, den König, gerächt an Saul und
an seinem Samen.

\bibverse{9} Da antwortete ihnen David: So wahr der HErr lebt, der meine
Seele aus aller Trübsal erlöst hat, \bibverse{10} ich griff den, der mir
verkündigte und sprach: Saul ist tot! und meinte, er wäre ein guter
Bote, und erwürgte ihn zu Ziklag, dem ich sollte Botenlohn geben.
\bibverse{11} Und diese gottlosen Leute haben einen gerechten Mann in
seinem Hause auf seinem Lager erwürgt. Ja, sollte ich das Blut nicht
fordern von euren Händen und euch von der Erde tun? \bibverse{12} Und
David gebot seinen Jünglingen; die erwürgten sie und hieben ihre Hände
und Füße ab und hingen sie auf am Teich zu Hebron. Aber das Haupt
Is-Boseths nahmen sie und begruben's in Abners Grab zu Hebron. \# 5
\bibverse{1} Und es kamen alle Stämme Israels zu David gen Hebron und
sprachen: Siehe, wir sind deines Gebeins und deines Fleisches.
\footnote{\textbf{5:1} 2Sam 19,13} \bibverse{2} Dazu auch vormals, da
Saul über uns König war, führtest du Israel aus und ein. So hat der HErr
dir gesagt: Du sollst mein Volk Israel hüten und sollst ein Herzog sein
über Israel. \footnote{\textbf{5:2} 1Sam 13,14; 1Sam 25,30} \bibverse{3}
Und es kamen alle Ältesten in Israel zum König gen Hebron. Und der König
David machte mit ihnen einen Bund zu Hebron vor dem HErrn, und sie
salbten David zum König über Israel. \footnote{\textbf{5:3} 2Sam 2,4;
  1Sam 16,13}

\bibverse{4} Dreißig Jahre war David alt, da er König ward, und regierte
40 Jahre. \footnote{\textbf{5:4} 1Kö 2,11; 1Chr 29,27} \bibverse{5} Zu
Hebron regierte er sieben Jahre und sechs Monate über Juda; aber zu
Jerusalem regierte er 33 Jahre über ganz Israel und Juda.

\bibverse{6} Und der König zog hin mit seinen Männern gen Jerusalem
wider die Jebusiter, die im Lande wohnten. Sie aber sprachen zu David:
Du wirst nicht hier hereinkommen, sondern Blinde und Lahme werden dich
abtreiben. Damit meinten sie aber, dass David nicht würde dahinein
kommen. \bibverse{7} David aber gewann die Burg Zion, das ist Davids
Stadt. \bibverse{8} Da sprach David desselben Tages: Wer die Jebusiter
schlägt und erlangt die Dachrinnen, die Lahmen und Blinden, denen die
Seele Davids feind ist\ldots! Daher spricht man: Lass keinen Blinden und
Lahmen ins Haus kommen.

\bibverse{9} Also wohnte David auf der Burg und hieß sie Davids Stadt.
Und David baute ringsumher von Millo an einwärts. \bibverse{10} Und
David nahm immer mehr zu, und der HErr, der Gott Zebaoth, war mit ihm.
\footnote{\textbf{5:10} 2Sam 3,1} \bibverse{11} Und Hiram, der König zu
Tyrus, sandte Boten zu David und Zedernbäume und Zimmerleute und
Steinmetzen, dass sie David ein Haus bauten. \bibverse{12} Und David
merkte, dass ihn der HErr zum König über Israel bestätigt hatte und sein
Königreich erhöht um seines Volks Israel willen.

\bibverse{13} Und David nahm noch mehr Weiber und Kebsweiber zu
Jerusalem, nachdem er von Hebron gekommen war; und wurden ihm noch mehr
Söhne und Töchter geboren. \bibverse{14} Und das sind die Namen derer,
die ihm zu Jerusalem geboren sind: Sammua, Sobab, Nathan, Salomo,
\bibverse{15} Jibhar, Elisua, Nepheg, Japhia, \bibverse{16} Elisama,
Eljada, Eliphelet.

\bibverse{17} Und da die Philister hörten, dass man David zum König über
Israel gesalbt hatte, zogen sie alle herauf, David zu suchen. Da das
David erfuhr, zog er hinab in eine Burg. \bibverse{18} Aber die
Philister kamen und ließen sich nieder im Grunde Rephaim. \bibverse{19}
Und David fragte den HErrn und sprach: Soll ich hinaufziehen wider die
Philister? und willst du sie in meine Hand geben? Der HErr sprach zu
David: Zieh hinauf! ich will die Philister in deine Hände geben.
\footnote{\textbf{5:19} 1Sam 30,8}

\bibverse{20} Und David kam gen Baal-Perazim und schlug sie daselbst und
sprach: Der HErr hat meine Feinde vor mir voneinander gerissen, wie die
Wasser reißen. Daher hieß man den Ort Baal-Perazim.

\bibverse{21} Und sie ließen ihre Götzen daselbst; David aber und seine
Männer hoben sie auf.

\bibverse{22} Die Philister aber zogen abermals herauf und ließen sich
nieder im Grunde Rephaim. \bibverse{23} Und David fragte den HErrn; der
sprach: Du sollst nicht hinaufziehen, sondern komm von hinten zu ihnen,
dass du an sie kommest gegenüber den Maulbeerbäumen. \bibverse{24} Und
wenn du hören wirst das Rauschen auf den Wipfeln der Maulbeerbäume
einhergehen, so eile; denn der HErr ist dann ausgegangen vor dir her, zu
schlagen das Heer der Philister.

\bibverse{25} David tat, wie der HErr ihm geboten hatte, und schlug die
Philister von Geba an, bis man kommt gen Geser. \# 6 \bibverse{1} Und
David sammelte abermals alle junge Mannschaft in Israel, 30.000,
\bibverse{2} und machte sich auf und ging hin mit allem Volk, das bei
ihm war, gen Baal in Juda, dass er die Lade Gottes von da heraufholte,
deren Name heißt: Der Name des HErrn Zebaoth wohnt darauf über den
Cherubim. \footnote{\textbf{6:2} Jos 15,9; 2Mo 25,22} \bibverse{3} Und
sie ließen die Lade Gottes führen auf einem neuen Wagen und holten sie
aus dem Hause Abinadabs, der auf dem Hügel wohnte. Usa aber und Ahjo,
die Söhne Abinadabs, trieben den neuen Wagen. \footnote{\textbf{6:3}
  1Sam 7,1} \bibverse{4} Und da sie ihn mit der Lade Gottes aus dem
Hause Abinadabs führten, der auf einem Hügel wohnte, und Ahjo vor der
Lade her ging, \bibverse{5} spielte David und das ganze Haus Israel vor
dem HErrn her mit allerlei Saitenspiel von Tannenholz, mit Harfen und
Psaltern und Pauken und Schellen und Zimbeln.

\bibverse{6} Und da sie kamen zu Tenne Nachons, griff Usa zu und hielt
die Lade Gottes; denn die Rinder traten beiseit aus. \bibverse{7} Da
ergrimmte des HErrn Zorn über Usa, und Gott schlug ihn daselbst um
seines Frevels willen, dass er daselbst starb bei der Lade Gottes.
\footnote{\textbf{6:7} 4Mo 4,15; 1Sam 6,19} \bibverse{8} Da ward David
betrübt, dass der HErr den Usa so wegriss, und man hieß die Stätte
Perez-Usa bis auf diesen Tag. \bibverse{9} Und David fürchtete sich vor
dem HErrn des Tages und sprach: Wie soll die Lade des HErrn zu mir
kommen? \bibverse{10} Und wollte sie nicht lassen zu sich bringen in die
Stadt Davids, sondern ließ sie bringen ins Haus Obed-Edoms, des
Gathiters. \bibverse{11} Und da die Lade des HErrn drei Monate blieb im
Hause Obed-Edoms, des Gathiters, segnete ihn der HErr und sein ganzes
Haus. \bibverse{12} Und es ward dem König David angesagt, dass der HErr
das Haus Obed-Edoms segnete und alles, was er hatte, um der Lade Gottes
willen. Da ging er hin und holte die Lade Gottes aus dem Hause
Obed-Edoms herauf in die Stadt Davids mit Freuden.

\bibverse{13} Und da sie einhergegangen waren mit der Lade des HErrn
sechs Gänge, opferte man einen Ochsen und ein fettes Schaf. \footnote{\textbf{6:13}
  1Kö 8,5} \bibverse{14} Und David tanzte mit aller Macht vor dem HErrn
her und war begürtet mit einem leinenen Leibrock. \bibverse{15} Und
David samt dem ganzen Israel führten die Lade des HErrn herauf mit
Jauchzen und Posaunen.

\bibverse{16} Und da die Lade des HErrn in die Stadt Davids kam, guckte
Michal, die Tochter Sauls, durchs Fenster und sah den König David
springen und tanzen vor dem HErrn und verachtete ihn in ihrem Herzen.
\bibverse{17} Da sie aber die Lade des HErrn hineinbrachten, stellten
sie die an ihren Ort mitten in der Hütte, die David für sie hatte
aufgeschlagen. Und David opferte Brandopfer und Dankopfer vor dem HErrn.
\bibverse{18} Und da David hatte ausgeopfert die Brandopfer und
Dankopfer, segnete er das Volk in dem Namen des HErrn Zebaoth
\footnote{\textbf{6:18} 1Kö 8,55} \bibverse{19} und teilte aus allem
Volk, der ganzen Menge Israels, sowohl Mann als Weib, einem jeglichen
einen Brotkuchen und ein Stück Fleisch und ein halbes Maß Wein. Da
kehrte alles Volk heim, ein jeglicher in sein Haus.

\bibverse{20} Da aber David wiederkam, sein Haus zu grüßen, ging Michal,
die Tochter Sauls, heraus ihm entgegen und sprach: Wie herrlich ist
heute der König von Israel gewesen, der sich vor den Mägden seiner
Knechte entblößt hat, wie sich die losen Leute entblößen!

\bibverse{21} David aber sprach zu Michal: Ich will vor dem HErrn
spielen, der mich erwählt hat vor deinem Vater und vor allem seinem
Hause, dass er mir befohlen hat, ein Fürst zu sein über das Volk des
HErrn, über Israel, \bibverse{22} Und ich will noch geringer werden denn
also und will niedrig sein in meinen Augen, und mit den Mägden, von
denen du geredet hast, zu Ehren kommen.

\bibverse{23} Aber Michal, Sauls Tochter, hatte kein Kind bis an den Tag
ihres Todes. \# 7 \bibverse{1} Da nun der König in seinem Hause saß und
der HErr ihm Ruhe gegeben hatte von allen seinen Feinden umher,
\bibverse{2} sprach er zu dem Propheten Nathan: Siehe, ich wohne in
einem Zedernhause, und die Lade Gottes wohnt unter den Teppichen.
\footnote{\textbf{7:2} Ps 132,-1}

\bibverse{3} Nathan sprach zu dem König: Gehe hin; alles, was du in
deinem Herzen hast, das tue, denn der HErr ist mit dir.

\bibverse{4} Des Nachts aber kam das Wort des HErrn zu Nathan und
sprach: \bibverse{5} Gehe hin und sage zu meinem Knechte David: So
spricht der HErr: Solltest du mir ein Haus bauen, dass ich darin wohne?
\bibverse{6} Habe ich doch in keinem Hause gewohnt seit dem Tage, da ich
die Kinder Israel aus Ägypten führte, bis auf diesen Tag, sondern ich
habe gewandelt in der Hütte und Wohnung. \footnote{\textbf{7:6} 1Kö
  8,16; 1Kö 8,27; Jes 66,1} \bibverse{7} Wo ich mit allen Kindern Israel
hin wandelte, habe ich auch je geredet mit irgend der Stämme Israels
einem, dem ich befohlen habe, mein Volk Israel zu weiden, und gesagt:
Warum bauet ihr mir nicht ein Zedernhaus? \bibverse{8} So sollst du nun
so sagen meinem Knechte David: So spricht der HErr Zebaoth: Ich habe
dich genommen von den Schafhürden, dass du sein solltest ein Fürst über
mein Volk Israel, \bibverse{9} und bin mit dir gewesen, wo du hin
gegangen bist, und habe alle deine Feinde vor dir ausgerottet und habe
dir einen großen Namen gemacht wie der Name der Großen auf Erden.
\bibverse{10} Und ich will meinem Volk Israel einen Ort setzen und will
es pflanzen, dass es daselbst wohne und nicht mehr in der Irre gehe, und
es Kinder der Bosheit nicht mehr drängen wie vormals und seit der Zeit,
dass ich Richter über mein Volk Israel verordnet habe; \bibverse{11} und
will dir Ruhe geben von allen deinen Feinden. Und der HErr verkündigt
dir, dass der HErr dir ein Haus machen will. \bibverse{12} Wenn nun
deine Zeit hin ist, dass du mit deinen Vätern schlafen liegst, will ich
deinen Samen nach dir erwecken, der von deinem Leibe kommen soll; dem
will ich sein Reich bestätigen. \footnote{\textbf{7:12} 1Kö 8,20; Jes
  9,6} \bibverse{13} Der soll meinem Namen ein Haus bauen, und ich will
den Stuhl seines Königreichs bestätigen ewiglich. \footnote{\textbf{7:13}
  1Kö 5,19; 1Kö 6,12; Ps 89,4-5} \bibverse{14} Ich will sein Vater sein,
und er soll mein Sohn sein. Wenn er eine Missetat tut, will ich ihn mit
Menschenruten und mit der Menschenkinder Schlägen strafen; \footnote{\textbf{7:14}
  Ps 89,27; Hebr 1,5; Lk 1,32} \bibverse{15} aber meine Barmherzigkeit
soll nicht von ihm entwandt werden, wie ich sie entwandt habe von Saul,
den ich vor dir habe weggenommen. \footnote{\textbf{7:15} 1Sam 15,23;
  1Sam 15,26} \bibverse{16} Aber dein Haus und dein Königreich soll
beständig sein ewiglich vor dir, und dein Stuhl soll ewiglich bestehen.
\footnote{\textbf{7:16} Ps 72,17; Jes 55,3} \bibverse{17} Da Nathan alle
diese Worte und all dies Gesicht David gesagt hatte,

\bibverse{18} kam David, der König, und blieb vor dem HErrn und sprach:
Wer bin ich, Herr HErr, und was ist mein Haus, dass du mich bis hierher
gebracht hast? \bibverse{19} Dazu hast du das zu wenig geachtet, Herr
HErr, sondern hast dem Hause deines Knechtes noch von fernem Zukünftigem
geredet, und das nach Menschenweise, Herr HErr! \bibverse{20} Und was
soll David mehr reden mit dir? Du erkennst deinen Knecht, Herr HErr!
\bibverse{21} Um deines Wortes willen und nach deinem Herzen hast du
solche große Dinge alle getan, dass du sie deinem Knecht kundtätest.
\bibverse{22} Darum bist du auch groß geachtet, HErr, Gott; denn es ist
keiner wie du und ist kein Gott als du, nach allem, was wir mit unseren
Ohren gehört haben. \bibverse{23} Denn wo ist ein Volk auf Erden wie
dein Volk Israel, um welches willen Gott ist hingegangen, sich ein Volk
zu erlösen und sich einen Namen zu machen und solch große und
schreckliche Dinge zu tun in deinem Lande vor deinem Volk, welches du
dir erlöst hast von Ägypten, von den Heiden und ihren Göttern?
\footnote{\textbf{7:23} 5Mo 4,7} \bibverse{24} Und du hast dir dein Volk
Israel zubereitet, dir zum Volk in Ewigkeit; und du, HErr, bist ihr Gott
geworden.

\bibverse{25} So bekräftige nun, HErr, Gott, das Wort in Ewigkeit, das
du über deinen Knecht und über sein Haus geredet hast, und tue, wie du
geredet hast! \bibverse{26} So wird dein Name groß werden in Ewigkeit,
dass man wird sagen: Der HErr Zebaoth ist der Gott über Israel, und das
Haus deines Knechtes David wird bestehen vor dir. \bibverse{27} Denn du,
HErr Zebaoth, du Gott Israels, hast das Ohr deines Knechts geöffnet und
gesagt: Ich will dir ein Haus bauen. Darum hat dein Knecht sein Herz
gefunden, dass er dies Gebet zu dir betet.

\bibverse{28} Nun, Herr HErr, du bist Gott, und deine Worte werden
Wahrheit sein. Du hast solches Gute über deinen Knecht geredet.
\footnote{\textbf{7:28} 1Kö 8,26} \bibverse{29} So hebe nun an und segne
das Haus deines Knechtes, dass es ewiglich vor dir sei; denn du, Herr
HErr, hast's geredet, und mit deinem Segen wird deines Knechtes Haus
gesegnet werden ewiglich. \# 8 \bibverse{1} Und es begab sich darnach,
dass David die Philister schlug und schwächte sie und nahm den
Dienstzaum von der Philister Hand. \bibverse{2} Er schlug auch die
Moabiter also zu Boden, dass er zwei Teile zum Tod brachte und einen
Teil am Leben ließ. Also wurden die Moabiter David untertänig, dass sie
ihm Geschenke zutrugen.

\bibverse{3} David schlug auch Hadadeser, den Sohn Rehobs, König zu
Zoba, da er hinzog, seine Macht wieder zu holen an dem Wasser Euphrat.
\bibverse{4} Und David fing aus ihnen 1700 Reiter und 20.000 Mann
Fußvolk und verlähmte alle Rosse der Wagen und behielt übrig 100 Wagen.
\bibverse{5} Es kamen aber die Syrer von Damaskus, zu helfen Hadadeser,
dem König zu Zoba; und David schlug der Syrer 22.000 Mann \bibverse{6}
und legte Volk in das Syrien von Damaskus. Also ward Syrien David
untertänig, dass sie ihm Geschenke zutrugen. Denn der HErr half David,
wo er hin zog. \bibverse{7} Und David nahm die goldenen Schilde, die
Hadadesers Knechte gehabt hatten, und brachte sie gen Jerusalem.
\bibverse{8} Aber von Betah und Berothai, den Städten Hadadesers, nahm
der König David sehr viel Erz.

\bibverse{9} Da aber Thoi, der König zu Hamath, hörte, dass David hatte
alle Macht des Hadadeser geschlagen, \bibverse{10} sandte er Joram,
seinen Sohn, zu David, ihn freundlich zu grüßen und ihn zu segnen, dass
er wider Hadadeser gestritten und ihn geschlagen hatte (denn Thoi hatte
einen Streit mit Hadadeser): und er hatte mit sich silberne, goldene und
eherne Kleinode, \bibverse{11} welche der König David auch dem HErrn
heiligte samt dem Silber und Gold, das er heiligte von allen Heiden, die
er unter sich gebracht; \bibverse{12} von Syrien, von Moab, von den
Kindern Ammon, von den Philistern, von Amalek, von der Beute Hadadesers,
des Sohnes Rehobs, Königs zu Zoba.

\bibverse{13} Auch machte sich David einen Namen, da er wiederkam von
der Syrer Schlacht und schlug im Salztal 18.000 Mann, \footnote{\textbf{8:13}
  Ps 60,2} \bibverse{14} und legte Volk in ganz Edom, und ganz Edom war
David unterworfen; denn der HErr half David, wo er hin zog. \footnote{\textbf{8:14}
  1Mo 27,40}

\bibverse{15} Also war David König über ganz Israel, und er schaffte
Recht und Gerechtigkeit allem Volk. \bibverse{16} Joab, der Zeruja Sohn,
war über das Heer; Josaphat aber, der Sohn Ahiluds, war Kanzler;
\footnote{\textbf{8:16} 2Sam 20,23-26} \bibverse{17} Zadok, der Sohn
Ahitobs, und Ahimelech, der Sohn Abjathars, waren Priester; Seraja war
Schreiber; \bibverse{18} Benaja, der Sohn Jojadas, war über die Kreter
und Plether, und die Söhne Davids waren Priester. \# 9 \bibverse{1} Und
David sprach: Ist auch noch jemand übriggeblieben von dem Hause Sauls,
dass ich Barmherzigkeit an ihm tue um Jonathans willen? \bibverse{2} Es
war aber ein Knecht vom Hause Sauls, der hieß Ziba; den riefen sie zu
David. Und der König sprach zu ihm: Bist du Ziba? Er sprach: Ja, dein
Knecht. \footnote{\textbf{9:2} 2Sam 16,1}

\bibverse{3} Der König sprach: Ist noch jemand vom Hause Sauls, dass ich
Gottes Barmherzigkeit an ihm tue? Ziba sprach zum König: Es ist noch da
ein Sohn Jonathans, lahm an den Füßen. \footnote{\textbf{9:3} 2Sam 4,4}

\bibverse{4} Der König sprach zu ihm: Wo ist er? Ziba sprach zum König:
Siehe, er ist zu Lo-Dabar im Hause Machirs, des Sohnes Ammiels.
\footnote{\textbf{9:4} 2Sam 17,27}

\bibverse{5} Da sandte der König David hin und ließ ihn holen von
Lo-Dabar aus dem Hause Machirs, des Sohnes Ammiels.

\bibverse{6} Da nun Mephiboseth, der Sohn Jonathans, des Sohnes Sauls,
zu David kam, fiel er auf sein Angesicht und beugte sich nieder. David
aber sprach: Mephiboseth! Er sprach: Hier bin ich, dein Knecht.

\bibverse{7} David sprach zu ihm: Fürchte dich nicht; denn ich will
Barmherzigkeit an dir tun um Jonathans, deines Vaters, willen und will
dir allen Acker deines Vaters Saul wiedergeben; du aber sollst täglich
an meinem Tisch das Brot essen.

\bibverse{8} Er aber fiel nieder und sprach: Wer bin ich, dein Knecht,
dass du dich wendest zu einem toten Hunde, wie ich bin?

\bibverse{9} Da rief der König Ziba, den Diener Sauls, und sprach zu
ihm: Alles, was Saul gehört hat und seinem ganzen Hause, habe ich dem
Sohn deines Herrn gegeben.

\bibverse{10} So arbeite ihm nun seinen Acker, du und deine Kinder und
Knechte, und bringe es ein, dass es das Brot sei des Sohnes deines
Herrn, dass er sich nähre; aber Mephiboseth, deines Herrn Sohn, soll
täglich das Brot essen an meinem Tisch. Ziba aber hatte 15 Söhne und 20
Knechte.

\bibverse{11} Und Ziba sprach zum König: Alles, wie mein Herr, der
König, seinem Knecht geboten hat, so soll dein Knecht tun. Und
Mephiboseth (sprach David) esse an meinem Tische wie der Königskinder
eins. \footnote{\textbf{9:11} 2Sam 19,29}

\bibverse{12} Und Mephiboseth hatte einen kleinen Sohn, der hieß Micha.
Aber alles, was im Hause Zibas wohnte, das diente Mephiboseth.

\bibverse{13} Mephiboseth aber wohnte zu Jerusalem; denn er aß täglich
an des Königs Tisch, und er hinkte mit seinen beiden Füßen. \# 10
\bibverse{1} Und es begab sich darnach, dass der König der Kinder Ammon
starb, und sein Sohn Hanun ward König an seiner Statt. \bibverse{2} Da
sprach David: Ich will Barmherzigkeit tun an Hanun, dem Sohn des Nahas,
wie sein Vater an mir Barmherzigkeit getan hat. Und sandte hin und ließ
ihn trösten durch seine Knechte über seinen Vater. Da nun die Knechte
Davids ins Land der Kinder Ammon kamen,

\bibverse{3} sprachen die Gewaltigen der Kinder Ammon zu ihrem Herrn,
Hanun: Meinst du, dass David deinen Vater ehren wolle, dass er Tröster
zu dir gesandt hat? Meinst du nicht, dass er darum hat seine Knechte zu
dir gesandt, dass er die Stadt erforsche und erkunde und umkehre?

\bibverse{4} Da nahm Hanun die Knechte Davids und schor ihnen den Bart
halb und schnitt ihnen die Kleider halb ab bis an den Gürtel und ließ
sie gehen. \bibverse{5} Da das David ward angesagt, sandte er ihnen
entgegen; denn die Männer waren sehr geschändet. Und der König ließ
ihnen sagen: Bleibt zu Jericho, bis euer Bart gewachsen; so kommt dann
wieder.

\bibverse{6} Da aber die Kinder Ammon sahen, dass sie vor David stinkend
waren geworden, sandten sie hin und dingten die Syrer des Hauses Rehob
und die Syrer zu Zoba, 20.000 Mann Fußvolk, und von dem König Maachas
1000 Mann und von Is-Tob 12.000 Mann. \bibverse{7} Da das David hörte,
sandte er Joab mit dem ganzen Heer der Kriegsleute. \bibverse{8} Und die
Kinder Ammon zogen aus und rüsteten sich zum Streit vor dem Eingang des
Tors. Die Syrer aber von Zoba, von Rehob, von Is-Tob und von Maacha
waren allein im Felde. \bibverse{9} Da Joab nun sah, dass der Streit auf
ihn gestellt war vorn und hinten, erwählte er aus aller jungen
Mannschaft in Israel und stellte sich wider die Syrer. \bibverse{10} Und
das übrige Volk tat er unter die Hand seines Bruders Abisai, dass er
sich rüstete wider dir Kinder Ammon, \bibverse{11} und sprach: Werden
mir die Syrer überlegen sein, so komm mir zu Hilfe; werden aber die
Kinder Ammon dir überlegen sein, so will ich dir zu Hilfe kommen.
\bibverse{12} Sei getrost und lass uns stark sein für unser Volk und für
die Städte unseres Gottes; der HErr aber tue, was ihm gefällt.
\bibverse{13} Und Joab machte sich herzu mit dem Volk, das bei ihm war,
zu streiten wider die Syrer; und sie flohen vor ihm. \bibverse{14} Und
da die Kinder Ammon sahen, dass die Syrer flohen, flohen sie auch vor
Abisai und zogen in die Stadt. Also kehrte Joab um von den Kindern Ammon
und kam gen Jerusalem.

\bibverse{15} Und da die Syrer sahen, dass sie geschlagen waren vor
Israel, kamen sie zuhauf. \bibverse{16} Und Hadadeser sandte hin und
brachte heraus die Syrer jenseits des Stromes und führte herein ihre
Macht; und Sobach, der Feldhauptmann Hadadesers, zog vor ihnen her.
\bibverse{17} Da das David ward angesagt, sammelte er zuhauf das ganze
Israel und zog über den Jordan und kam gen Helam. Und die Syrer stellten
sich wider David, mit ihm zu streiten. \bibverse{18} Aber die Syrer
flohen vor Israel. Und David verderbte der Syrer 700 Wagen und 40.000
Reiter; dazu Sobach, den Feldhauptmann, schlug er, dass er daselbst
starb. \bibverse{19} Da aber die Könige, die unter Hadadeser waren,
sahen, dass sie geschlagen waren vor Israel, machten sie Frieden mit
Israel und wurden ihnen untertan. Und die Syrer fürchteten sich, den
Kindern Ammon mehr zu helfen. \# 11 \bibverse{1} Und da das Jahr um kam,
zur Zeit, wann die Könige pflegen auszuziehen, sandte David Joab und
seine Knechte mit ihm und das ganze Israel, dass sie die Kinder Ammon
verderbten und Rabba belagerten. David aber blieb zu Jerusalem.
\footnote{\textbf{11:1} 1Chr 20,1} \bibverse{2} Und es begab sich, dass
David um den Abend aufstand von seinem Lager und ging auf dem Dach des
Königshauses und sah vom Dach ein Weib sich waschen; und das Weib war
sehr schöner Gestalt. \footnote{\textbf{11:2} Mt 5,28-29} \bibverse{3}
Und David sandte hin und ließ nach dem Weibe fragen, und man sagte: Ist
das nicht Bath-Seba, die Tochter Eliams, das Weib Urias, des Hethiters?
\footnote{\textbf{11:3} 2Sam 23,29}

\bibverse{4} Und David sandte Boten hin und ließ sie holen. Und da sie
zu ihm hineinkam, schlief er bei ihr. Sie aber reinigte sich von ihrer
Unreinigkeit und kehrte wieder zu ihrem Hause. \footnote{\textbf{11:4}
  3Mo 15,18} \bibverse{5} Und das Weib ward schwanger und sandte hin und
ließ David verkündigen und sagen: Ich bin schwanger geworden.

\bibverse{6} David aber sandte zu Joab: Sende zu mir Uria, den Hethiter.
Und Joab sandte Uria zu David. \bibverse{7} Und da Uria zu ihm kam,
fragte David, ob es mit Joab und mit dem Volk und mit dem Streit wohl
stünde? \bibverse{8} Und David sprach zu Uria: Gehe hinab in dein Haus
und wasche deine Füße. Und da Uria zu des Königs Haus hinausging, folgte
ihm nach des Königs Geschenk. \bibverse{9} Aber Uria legte sich schlafen
vor der Tür des Königshauses, da alle Knechte seines Herrn lagen, und
ging nicht hinab in sein Haus. \bibverse{10} Da man aber David ansagte:
Uria ist nicht hinab in sein Haus gegangen, sprach David zu ihm: Bist du
nicht über Feld hergekommen? Warum bist du nicht hinab in dein Haus
gegangen?

\bibverse{11} Uria aber sprach zu David: Die Lade und Israel und Juda
bleiben in Zelten, und Joab, mein Herr, und meines Herrn Knechte liegen
zu Felde, und ich sollte in mein Haus gehen, dass ich äße und tränke und
bei meinem Weibe läge? So wahr du lebst und deine Seele lebt, ich tue
solches nicht. \footnote{\textbf{11:11} 1Sam 4,4}

\bibverse{12} David sprach zu Uria: So bleibe auch heute hier; morgen
will ich dich lassen gehen. So blieb Uria zu Jerusalem des Tages und des
anderen dazu. \bibverse{13} Und David lud ihn, dass er vor ihm aß und
trank, und machte ihn trunken. Aber des Abends ging er aus, dass er sich
schlafen legte auf sein Lager mit seines Herrn Knechten, und ging nicht
hinab in sein Haus. \bibverse{14} Des Morgens schrieb David einen Brief
an Joab und sandte ihn durch Uria. \bibverse{15} Er schrieb aber also in
den Brief: Stellet Uria an den Streit, da er am härtesten ist, und
wendet euch hinter ihm ab, dass er erschlagen werde und sterbe.

\bibverse{16} Als nun Joab um die Stadt lag, stellte er Uria an den Ort,
wo er wusste, dass streitbare Männer waren. \bibverse{17} Und da die
Männer der Stadt herausfielen und stritten wider Joab, fielen etliche
des Volks von den Knechten Davids, und Uria, der Hethiter, starb auch.
\bibverse{18} Da sandte Joab hin und ließ David ansagen allen Handel des
Streits \bibverse{19} und gebot dem Boten und sprach: Wenn du allen
Handel des Streits hast ausgeredet mit dem König \bibverse{20} und
siehst, dass der König sich erzürnt und zu dir spricht: Warum habt ihr
euch so nahe zur Stadt gemacht mit dem Streit? Wisset ihr nicht, wie man
pflegt von der Mauer zu schießen? \bibverse{21} Wer schlug Abimelech,
den Sohn Jerubbeseths? Warf nicht ein Weib einen Mühlstein auf ihn von
der Mauer, dass er starb zu Thebez? Warum habt ihr euch so nahe zur
Mauer gemacht? so sollst du sagen: Dein Knecht Uria, der Hethiter, ist
auch tot.

\bibverse{22} Der Bote ging hin und kam und sagte an David alles, darum
ihn Joab gesandt hatte. \bibverse{23} Und der Bote sprach zu David: Die
Männer nahmen überhand wider uns und fielen zu uns heraus aufs Feld; wir
aber waren an ihnen bis vor die Tür des Tors; \bibverse{24} und die
Schützen schossen von der Mauer auf deine Knechte und töteten etliche
von des Königs Knechten; dazu ist Uria, dein Knecht, der Hethiter, auch
tot.

\bibverse{25} David sprach zum Boten: So sollst du zu Joab sagen: Lass
dir das nicht übel gefallen; denn das Schwert frisst jetzt diesen, jetzt
jenen. Fahre fort mit dem Streit wider die Stadt, dass du sie
zerbrechest, und seid getrost.

\bibverse{26} Und da Urias Weib hörte, dass ihr Mann, Uria, tot war,
trug sie Leid um ihren Eheherrn. \bibverse{27} Da sie aber ausgetrauert
hatte, sandte David hin und ließ sie in sein Haus holen, und sie ward
sein Weib und gebar ihm einen Sohn. Aber die Tat gefiel dem HErrn übel,
die David tat. \footnote{\textbf{11:27} 2Mo 20,13-14}

\hypertarget{section-1}{%
\section{12}\label{section-1}}

\bibverse{1} Und der HErr sandte Nathan zu David. Da der zu ihm kam,
sprach er zu ihm: Es waren zwei Männer in einer Stadt, einer reich, der
andere arm. \bibverse{2} Der Reiche hatte sehr viele Schafe und Rinder;
\bibverse{3} aber der Arme hatte nichts denn ein einziges kleines
Schäflein, das er gekauft hatte. Und er nährte es, dass es groß ward bei
ihm und bei seinen Kindern zugleich: es aß von seinem Bissen und trank
von seinem Becher und schlief in seinem Schoß, und er hielt es wie eine
Tochter. \bibverse{4} Da aber zu dem reichen Mann ein Gast kam, schonte
er zu nehmen von seinen Schafen und Rindern, dass er dem Gast etwas
zurichtete, der zu ihm gekommen war, und nahm das Schaf des armen Mannes
und richtete es zu dem Mann, der zu ihm gekommen war.

\bibverse{5} Da ergrimmte David mit großem Zorn wider den Mann und
sprach zu Nathan: So wahr der HErr lebt, der Mann ist ein Kind des
Todes, der das getan hat! \bibverse{6} Dazu soll er das Schaf vierfältig
bezahlen, darum dass er solches getan hat und nicht geschont hat.

\bibverse{7} Da sprach Nathan zu David: Du bist der Mann! So spricht der
HErr, der Gott Israels: Ich habe dich zum König gesalbt über Israel und
habe dich errettet aus der Hand Sauls \footnote{\textbf{12:7} 1Kö 20,40}
\bibverse{8} und habe dir deines Herrn Haus gegeben, dazu seine Weiber
in deinen Schoß, und habe dir das Haus Israel und Juda gegeben; und ist
das zu wenig, will ich noch dies und das dazutun. \bibverse{9} Warum
hast du denn das Wort des HErrn verachtet, dass du solches Übel vor
seinen Augen tatest? Uria, den Hethiter, hast du erschlagen mit dem
Schwert; sein Weib hast du dir zum Weib genommen; ihn aber hast du
erwürgt mit dem Schwert der Kinder Ammon. \bibverse{10} Nun, so soll von
deinem Hause das Schwert nicht lassen ewiglich, darum dass du mich
verachtet hast und das Weib Urias, des Hethiters, genommen hast, dass
sie dein Weib sei. \footnote{\textbf{12:10} 2Sam 13,28-29; 2Sam 18,14;
  2Kö 25,7}

\bibverse{11} So spricht der HErr: Siehe, ich will Unglück über dich
erwecken aus deinem eigenen Hause und will deine Weiber nehmen vor
deinen Augen und will sie deinem Nächsten geben, dass er bei deinen
Weibern schlafen soll an der lichten Sonne. \bibverse{12} Denn du hast
es heimlich getan; ich aber will dies tun vor dem ganzen Israel und an
der Sonne.

\bibverse{13} Da sprach David zu Nathan: Ich habe gesündigt wider den
HErrn. Nathan sprach zu David: So hat auch der HErr deine Sünde
weggenommen; du wirst nicht sterben.

\bibverse{14} Aber weil du die Feinde des HErrn hast durch diese
Geschichte lästern gemacht, wird der Sohn, der dir geboren ist, des
Todes sterben. \footnote{\textbf{12:14} 2Sam 11,27} \bibverse{15} Und
Nathan ging heim. Und der HErr schlug das Kind, das Urias Weib David
geboren hatte, dass es todkrank ward.

\bibverse{16} Und David suchte Gott um des Knäbleins willen und fastete
und ging hinein und lag über Nacht auf der Erde. \bibverse{17} Da
standen auf die Ältesten seines Hauses und wollten ihn aufrichten von
der Erde; er wollte aber nicht und aß auch nicht mit ihnen.
\bibverse{18} Am siebenten Tage aber starb das Kind. Und die Knechte
Davids fürchteten sich ihm anzusagen, dass das Kind tot wäre; denn sie
gedachten: Siehe, da das Kind noch lebendig war, redeten wir mit ihm,
und er gehorchte unserer Stimme nicht; wie viel mehr wird er sich wehe
tun, so wir sagen: Das Kind ist tot.

\bibverse{19} Da aber David sah, dass seine Knechte leise redeten, und
merkte, dass das Kind tot wäre, sprach er zu seinen Knechten: Ist das
Kind tot? Sie sprachen: Ja.

\bibverse{20} Da stand David auf von der Erde und wusch sich und salbte
sich und tat andere Kleider an und ging in das Haus des HErrn und betete
an. Und da er wieder heimkam, hieß er ihm Brot auftragen und aß.

\bibverse{21} Da sprachen seine Knechte zu ihm: Was ist das für ein
Ding, das du tust? Da das Kind lebte, fastetest du und weintest; nun es
aber gestorben ist, stehst du auf und isst?

\bibverse{22} Er sprach: Um das Kind fastete ich und weinte, da es
lebte; denn ich gedachte: Wer weiß, ob mir der HErr nicht gnädig wird,
dass das Kind lebendig bleibe. \bibverse{23} Nun es aber tot ist, was
soll ich fasten? Kann ich es auch wiederum holen? Ich werde wohl zu ihm
fahren; es kommt aber nicht wieder zu mir.

\bibverse{24} Und da David sein Weib Bath-Seba getröstet hatte, ging er
zu ihr hinein und schlief bei ihr. Und sie gebar einen Sohn, den hieß er
Salomo. Und der HErr liebte ihn. \bibverse{25} Und er tat ihn unter die
Hand Nathans, des Propheten; der hieß ihn Jedidja, um des HErrn willen.

\bibverse{26} So stritt nun Joab wider Rabba der Kinder Ammon und gewann
die königliche Stadt \bibverse{27} und sandte Boten zu David und ließ
ihm sagen: Ich habe gestritten wider Rabba und habe auch gewonnen die
Wasserstadt. \bibverse{28} So nimm nun zuhauf das übrige Volk und
belagere die Stadt und gewinne sie, auf dass ich sie nicht gewinne und
ich den Namen davon habe.

\bibverse{29} Also nahm David alles Volk zuhauf und zog hin und stritt
wider Rabba und gewann es \bibverse{30} und nahm die Krone seines Königs
von seinem Haupt, die am Gewicht einen Zentner Gold hatte und
Edelgesteine, und sie ward David auf sein Haupt gesetzt; und er führte
aus der Stadt sehr viel Beute. \bibverse{31} Aber das Volk drinnen
führte er heraus und legte sie unter eiserne Sägen und Zacken und
eiserne Keile und verbrannte sie in Ziegelöfen. So tat er allen Städten
der Kinder Ammon. Da kehrte David und alles Volk wieder gen Jerusalem.
\# 13 \bibverse{1} Und es begab sich darnach, dass Absalom, der Sohn
Davids, hatte eine schöne Schwester, die hieß Thamar; und Amnon, der
Sohn Davids, gewann sie lieb. \footnote{\textbf{13:1} 2Sam 3,2-3}
\bibverse{2} Und dem Amnon ward wehe, als wollte er krank werden um
Thamars, seiner Schwester, willen. Denn sie war eine Jungfrau, und es
deuchte Amnon schwer sein, dass er ihr etwas sollte tun. \bibverse{3}
Amnon aber hatte einen Freund, der hieß Jonadab, ein Sohn Simeas, Davids
Bruders; und derselbe Jonadab war ein sehr weiser Mann. \bibverse{4} Der
sprach zu ihm: Warum wirst du so mager, du Königssohn, von Tage zu Tage?
Magst du mir's nicht ansagen? Da sprach Amnon zu ihm: Ich habe Thamar,
meines Bruders Absalom Schwester, liebgewonnen.

\bibverse{5} Jonadab sprach zu ihm: Lege dich auf dein Bett und stelle
dich krank. Wenn dann dein Vater kommt, dich zu besuchen, so sprich zu
ihm: Lass doch meine Schwester Thamar kommen, dass sie mir zu essen gebe
und mache vor mir das Essen, dass ich zusehe und von ihrer Hand esse.

\bibverse{6} Also legte sich Amnon und stellte sich krank. Da nun der
König kam, ihn zu besuchen, sprach Amnon zum König: Lass doch meine
Schwester Thamar kommen, dass sie vor mir einen Kuchen oder zwei mache
und ich von ihrer Hand esse.

\bibverse{7} Da sandte David nach Thamar ins Haus und ließ ihr sagen:
Gehe hin ins Haus deines Bruders Amnon und mache ihm eine Speise.

\bibverse{8} Thamar ging hin ins Haus ihres Bruders Amnon; er aber lag
zu Bett. Und sie nahm einen Teig und knetete und bereitete es vor seinen
Augen und buk die Kuchen. \bibverse{9} Und sie nahm die Pfanne und
schüttete es vor ihm aus; aber er weigerte sich zu essen. Und Amnon
sprach: Lasst jedermann von mir hinausgehen. Und es ging jedermann von
ihm hinaus. \bibverse{10} Da sprach Amnon zu Thamar: Bringe das Essen in
die Kammer, dass ich von deiner Hand esse. Da nahm Thamar die Kuchen,
die sie gemacht hatte, und brachte sie zu Amnon, ihrem Bruder, in die
Kammer. \bibverse{11} Und da sie es zu ihm brachte, dass er äße, ergriff
er sie und sprach zu ihr: Komm her, meine Schwester, schlaf bei mir!
\footnote{\textbf{13:11} 3Mo 18,9}

\bibverse{12} Sie aber sprach zu ihm: Nicht, mein Bruder, schwäche mich
nicht, denn so tut man nicht in Israel; tue nicht eine solche Torheit!
\footnote{\textbf{13:12} 5Mo 22,21} \bibverse{13} Wo will ich mit meiner
Schande hin? Und du wirst sein wie die Toren in Israel. Rede aber mit
dem König; der wird mich dir nicht versagen.

\bibverse{14} Aber er wollte ihr nicht gehorchen und überwältigte sie
und schwächte sie und schlief bei ihr. \bibverse{15} Und Amnon ward ihr
überaus gram, dass der Hass größer war, denn vorhin die Liebe war. Und
Amnon sprach zu ihr: Mache dich auf und hebe dich!

\bibverse{16} Sie aber sprach zu ihm: Das Übel ist größer denn das
andere, das du an mir getan hast, dass du mich ausstößest. Aber er
gehorchte ihrer Stimme nicht,

\bibverse{17} sondern rief seinen Knaben, der sein Diener war, und
sprach: Treibe diese von mir hinaus und schließe die Tür hinter ihr zu!

\bibverse{18} Und sie hatte einen bunten Rock an; denn solche Röcke
trugen des Königs Töchter, welche Jungfrauen waren. Und da sie sein
Diener hinausgetrieben und die Tür hinter ihr zugeschlossen hatte,
\bibverse{19} warf Thamar Asche auf ihr Haupt und zerriss den bunten
Rock, den sie anhatte, und legte ihre Hand auf das Haupt und ging daher
und schrie. \footnote{\textbf{13:19} Hi 2,12} \bibverse{20} Und ihr
Bruder Absalom sprach zu ihr: Ist dein Bruder Amnon bei dir gewesen?
Nun, meine Schwester, schweig still; es ist dein Bruder, und nimm die
Sache nicht so zu Herzen. Also blieb Thamar einsam in Absaloms, ihres
Bruders, Hause.

\bibverse{21} Und da der König David solches alles hörte, ward er sehr
zornig. Aber Absalom redete nicht mit Amnon, weder Böses noch Gutes;
\bibverse{22} denn Absalom war Amnon gram, darum dass er seine Schwester
Thamar geschwächt hatte.

\bibverse{23} Über zwei Jahre aber hatte Absalom Schafscherer zu
Baal-Hazor, das bei Ephraim liegt; und Absalom lud alle Kinder des
Königs \bibverse{24} und kam zum König und sprach: Siehe, dein Knecht
hat Schafscherer; der König wolle samt seinen Knechten mit seinem Knecht
gehen.

\bibverse{25} Der König aber sprach zu Absalom: Nicht, mein Sohn, lass
uns nicht alle gehen, dass wir dich nicht beschweren. Und da er ihn
nötigte, wollte er doch nicht gehen, sondern segnete ihn.

\bibverse{26} Absalom sprach: Soll denn nicht mein Bruder Amnon mit uns
gehen? Der König sprach zu ihm: Warum soll er mit dir gehen?

\bibverse{27} Da nötigte ihn Absalom, dass er mit ihm ließ Amnon und
alle Kinder des Königs.

\bibverse{28} Absalom aber gebot seinen Leuten und sprach: Sehet darauf,
wenn Amnon guter Dinge wird von dem Wein und ich zu euch spreche:
Schlagt Amnon! und tötet ihn, dass ihr euch nicht fürchtet; denn ich
hab's euch geheißen. Seid getrost und frisch daran!

\bibverse{29} Also taten die Leute Absaloms dem Amnon, wie ihnen Absalom
geboten hatte. Da standen alle Kinder des Königs auf, und ein jeglicher
setzte sich auf sein Maultier und flohen.

\bibverse{30} Und da sie noch auf dem Wege waren, kam das Gerücht vor
David, dass Absalom hätte alle Kinder des Königs erschlagen, dass nicht
einer von ihnen übrig wäre.

\bibverse{31} Da stand der König auf und zerriss seine Kleider und legte
sich auf die Erde; und alle seine Knechte, die um ihn her standen,
zerrissen ihre Kleider. \bibverse{32} Da hob Jonadab an, der Sohn
Simeas, des Bruders Davids, und sprach: Mein Herr denke nicht, dass alle
jungen Männer, die Kinder des Königs, tot sind, sondern Amnon ist allein
tot. Denn Absalom hat's bei sich behalten von dem Tage an, da er seine
Schwester Thamar schwächte. \bibverse{33} So nehme nun mein Herr, der
König, solches nicht zu Herzen, dass alle Kinder des Königs tot seien,
sondern Amnon ist allein tot. \bibverse{34} Absalom aber floh. Und der
Diener auf der Warte hob seine Augen auf und sah; und siehe, ein großes
Volk kam auf dem Wege nacheinander an der Seite des Berges.
\bibverse{35} Da sprach Jonadab zum König: Siehe, die Kinder des Königs
kommen; wie dein Knecht gesagt hat, so ist's ergangen. \bibverse{36} Und
da er hatte ausgeredet, siehe, da kamen die Kinder des Königs und hoben
ihre Stimme auf und weinten. Der König und alle seine Knechte weinten
auch gar sehr.

\bibverse{37} Absalom aber floh und zog zu Thalmai, dem Sohn Ammihuds,
dem König zu Gessur. Er aber trug Leid über seinen Sohn alle Tage.
\footnote{\textbf{13:37} 2Sam 3,3; 2Sam 14,23} \bibverse{38} Da aber
Absalom geflohen war und gen Gessur gezogen, blieb er daselbst drei
Jahre. \bibverse{39} Und der König David hörte auf, auszuziehen wider
Absalom; denn er hatte sich getröstet über Amnon, dass er tot war. \# 14
\bibverse{1} Joab aber, der Zeruja Sohn, merkte, dass des Königs Herz
war wider Absalom, \bibverse{2} und sandte hin gen Thekoa und ließ holen
von dort ein kluges Weib und sprach zu ihr: Trage Leid und zieh
Trauerkleider an und salbe dich nicht mit Öl, sondern stelle dich wie
ein Weib, das eine lange Zeit Leid getragen hat über einen Toten;
\bibverse{3} und sollst zum König hineingehen und mit ihm reden so und
so. Und Joab gab ihr ein, was sie reden sollte.

\bibverse{4} Und da das Weib von Thekoa mit dem König reden wollte, fiel
sie auf ihr Antlitz zur Erde und beugte sich nieder und sprach: Hilf
mir, König!

\bibverse{5} Der König sprach zu ihr: Was ist dir? Sie sprach: Ach, ich
bin eine Witwe, und mein Mann ist gestorben.

\bibverse{6} Und deine Magd hatte zwei Söhne, die zankten miteinander
auf dem Felde, und da kein Retter war, schlug einer den anderen und
tötete ihn. \bibverse{7} Und siehe, nun steht auf die ganze Freundschaft
wider deine Magd und sagen: Gib her den, der seinen Bruder erschlagen
hat, dass wir ihn töten für die Seele seines Bruders, den er erwürgt
hat, und auch den Erben vertilgen; und wollen meinen Funken auslöschen,
der noch übrig ist, dass meinem Mann kein Name und nichts Übriges bleibe
auf Erden.

\bibverse{8} Der König sprach zum Weibe: Gehe heim, ich will für dich
gebieten.

\bibverse{9} Und das Weib von Thekoa sprach zum König: Mein Herr König,
die Missetat sei auf mir und meines Vaters Hause; der König aber und
sein Stuhl sei unschuldig.

\bibverse{10} Der König sprach: Wer wider dich redet, den bringe zu mir,
so soll er nicht mehr dich antasten.

\bibverse{11} Sie sprach: Der König gedenke an den HErrn, deinen Gott,
dass der Bluträcher nicht noch mehr Verderben anrichte und sie meinen
Sohn nicht vertilgen. Er sprach: So wahr der HErr lebt, es soll kein
Haar von deinem Sohn auf die Erde fallen. \footnote{\textbf{14:11} 1Sam
  14,45; 1Kö 1,52}

\bibverse{12} Und das Weib sprach: Lass deine Magd meinem Herrn König
etwas sagen. Er sprach: Sage an!

\bibverse{13} Das Weib sprach: Warum bist du also gesinnt wider Gottes
Volk? Denn da der König solches geredet hat, ist er wie ein Schuldiger,
dieweil er seinen Verstoßenen nicht wieder holen lässt.

\bibverse{14} Denn wir sterben des Todes und sind wie Wasser, das in die
Erde verläuft, das man nicht aufhält; und Gott will nicht das Leben
wegnehmen, sondern bedenkt sich, dass nicht das Verstoßene auch von ihm
verstoßen werde.

\bibverse{15} So bin ich nun gekommen, mit meinem Herrn König solches zu
reden; denn das Volk macht mir bang. Denn deine Magd gedachte: Ich will
mit dem König reden; vielleicht wird er tun, was seine Magd sagt.
\bibverse{16} Denn er wird seine Magd erhören, dass er mich errette von
der Hand aller, die mich samt meinem Sohn vertilgen wollen vom Erbe
Gottes. \bibverse{17} Und deine Magd gedachte: Meines Herrn, des Königs,
Wort soll mir ein Trost sein; denn mein Herr, der König, ist wie ein
Engel Gottes, dass er Gutes und Böses hören kann. Darum wird der HErr,
dein Gott, mit dir sein. \footnote{\textbf{14:17} 2Sam 19,28}

\bibverse{18} Der König antwortete und sprach zum Weibe: Leugne mir
nicht, was ich dich frage. Das Weib sprach: Mein Herr, der König, rede.

\bibverse{19} Der König sprach: Ist nicht die Hand Joabs mit dir in
diesem allem? Das Weib antwortete und sprach: So wahr deine Seele lebt,
mein Herr König, es ist nicht anders, weder zur Rechten noch zur Linken,
denn wie mein Herr, der König, geredet hat. Denn dein Knecht Joab hat
mir's geboten, und er hat solches alles seiner Magd eingegeben;

\bibverse{20} dass ich diese Sache also wenden sollte, das hat dein
Knecht Joab gemacht. Aber mein Herr ist weise wie die Weisheit eines
Engels Gottes, dass er merkt alles auf Erden.

\bibverse{21} Da sprach der König zu Joab: Siehe, ich habe solches
getan; so gehe hin und bringe den Knaben Absalom wieder.

\bibverse{22} Da fiel Joab auf sein Antlitz zur Erde und beugte sich
nieder und dankte dem König und sprach: Heute merkt dein Knecht, dass
ich Gnade gefunden habe vor deinen Augen, mein Herr König, da der König
tut, was sein Knecht sagt.

\bibverse{23} Also machte sich Joab auf und zog gen Gessur und brachte
Absalom gen Jerusalem. \footnote{\textbf{14:23} 2Sam 13,37}

\bibverse{24} Aber der König sprach: Lasst ihn wieder in sein Haus gehen
und mein Angesicht nicht sehen. Also kam Absalom wieder in sein Haus und
sah des Königs Angesicht nicht. \bibverse{25} Es war aber in ganz Israel
kein Mann so schön wie Absalom, und er hatte dieses Lob vor allen; von
seiner Fußsohle an bis auf seinen Scheitel war nicht ein Fehl an ihm.
\bibverse{26} Und wenn man sein Haupt schor (das geschah gemeiniglich
alle Jahre; denn es war ihm zu schwer, dass man's abscheren musste), so
wog sein Haupthaar zweihundert Lot nach dem königlichen Gewicht.
\bibverse{27} Und Absalom wurden drei Söhne geboren und eine Tochter,
die hieß Thamar und war ein Weib schön von Gestalt. \bibverse{28} Also
blieb Absalom zwei Jahre zu Jerusalem, dass er des Königs Angesicht
nicht sah. \bibverse{29} Und Absalom sandte nach Joab, dass er ihn zum
König sendete; und er wollte nicht zu ihm kommen. Er aber sandte zum
andernmal; immer noch wollte er nicht kommen. \bibverse{30} Da sprach er
zu seinen Knechten: Sehet das Stück Acker Joabs neben meinem, und er hat
Gerste darauf; so geht hin und steckt's mit Feuer an. Da steckten die
Knechte Absaloms das Stück mit Feuer an.

\bibverse{31} Da machte sich Joab auf und kam zu Absalom ins Haus und
sprach zu ihm: Warum haben deine Knechte mein Stück mit Feuer
angesteckt?

\bibverse{32} Absalom sprach zu Joab: Siehe, ich sandte nach dir und
ließ dir sagen: Komm her, dass ich dich zum König sende und sagen lasse:
Warum bin ich von Gessur gekommen? Es wäre mir besser, dass ich noch da
wäre. So lass mich nun das Angesicht des Königs sehen; ist aber eine
Missetat an mir, so töte mich.

\bibverse{33} Und Joab ging hinein zum König und sagte es ihm an. Und er
rief Absalom, dass er hinein zum König kam; und er fiel nieder vor dem
König auf sein Antlitz zur Erde, und der König küsste Absalom. \# 15
\bibverse{1} Und es begab sich darnach, dass Absalom ließ sich machen
einen Wagen und Rosse und 50 Mann, die seine Trabanten waren.
\footnote{\textbf{15:1} 1Kö 1,5} \bibverse{2} Auch machte sich Absalom
des Morgens früh auf und trat an den Weg bei dem Tor. Und wenn jemand
einen Handel hatte, dass er zum König vor Gericht kommen sollte, rief
ihn Absalom zu sich und sprach: Aus welcher Stadt bist du? Wenn dann der
sprach: Dein Knecht ist aus der Stämme Israels einem,

\bibverse{3} so sprach Absalom zu ihm: Siehe, deine Sache ist recht und
schlecht; aber du hast keinen, der dich hört, beim König.

\bibverse{4} Und Absalom sprach: O, wer setzt mich zum Richter im Lande,
dass jedermann zu mir käme, der eine Sache und Gerichtshandel hat, dass
ich ihm zum Recht hülfe! \bibverse{5} Und wenn jemand sich zu ihm tat,
dass er wollte vor ihm niederfallen, so reckte er seine Hand aus und
ergriff ihn und küsste ihn. \bibverse{6} Auf diese Weise tat Absalom dem
ganzen Israel, wenn sie kamen vor Gericht zum König, und stahl also das
Herz der Männer Israels.

\bibverse{7} Nach 40 Jahren sprach Absalom zum König: Ich will hingehen
und mein Gelübde zu Hebron ausrichten, das ich dem HErrn gelobt habe.
\bibverse{8} Denn dein Knecht tat ein Gelübde, da ich zu Gessur in
Syrien wohnte, und sprach: Wenn mich der HErr wieder gen Jerusalem
bringt, so will ich dem HErrn einen Gottesdienst tun.

\bibverse{9} Der König sprach zu ihm: Gehe hin mit Frieden. Und er
machte sich auf und ging gen Hebron.

\bibverse{10} Absalom aber hatte Kundschafter ausgesandt in alle Stämme
Israels und lassen sagen: Wenn ihr der Posaune Schall hören werdet, so
sprecht: Absalom ist König geworden zu Hebron.

\bibverse{11} Es gingen aber mit Absalom 200 Mann von Jerusalem, die
geladen waren; aber sie gingen in ihrer Einfalt und wussten nichts um
die Sache. \bibverse{12} Absalom aber sandte auch nach Ahithophel, dem
Giloniten, Davids Rat, aus seiner Stadt Gilo. Da er nun die Opfer tat,
ward der Bund stark, und das Volk lief zu und mehrte sich mit Absalom.
\footnote{\textbf{15:12} 2Sam 23,34} \bibverse{13} Da kam einer, der
sagte es David an und sprach: Das Herz jedermanns in Israel folgt
Absalom nach.

\bibverse{14} David sprach aber zu allen seinen Knechten, die bei ihm
waren zu Jerusalem: Auf, lasst uns fliehen! denn hier wird kein
Entrinnen sein vor Absalom; eilet, dass wir gehen, dass er uns nicht
übereile und ergreife uns und treibe ein Unglück auf uns und schlage die
Stadt mit der Schärfe des Schwerts.

\bibverse{15} Da sprachen die Knechte des Königs zu ihm: Was mein Herr,
der König, erwählt, siehe, hier sind deine Knechte.

\bibverse{16} Und der König zog hinaus und sein ganzes Haus ihm nach. Er
ließ aber zehn Kebsweiber zurück, das Haus zu bewahren. \bibverse{17}
Und da der König und alles Volk, das ihm nachfolgte, hinauskamen,
blieben sie stehen am äußersten Hause. \bibverse{18} Und alle seine
Knechte gingen an ihm vorüber; dazu alle Kreter und Plether und alle
Gathiter, 600 Mann, die von Gath ihm nachgefolgt waren, gingen an dem
König vorüber.

\bibverse{19} Und der König sprach zu Itthai, dem Gathiter: Warum gehst
du auch mit uns? Kehre um und bleibe bei dem König; denn du bist hier
fremd und von deinem Ort gezogen hierher. \footnote{\textbf{15:19} 2Sam
  18,2} \bibverse{20} Gestern bist du gekommen, und heute sollte ich
dich mit uns hin und her ziehen lassen? Denn ich will gehen, wohin ich
gehen kann. Kehre um und deine Brüder mit dir; dir widerfahre
Barmherzigkeit und Treue.

\bibverse{21} Itthai antwortete und sprach: So wahr der HErr lebt, und
so wahr mein Herr König lebt, an welchem Ort mein Herr, der König, sein
wird, es gerate zum Tod oder zum Leben, da wird dein Knecht auch sein.

\bibverse{22} David sprach zu Itthai: So komm und gehe mit. Also ging
Itthai, der Gathiter, und alle seine Männer und der ganze Haufe Kinder,
die mit ihm waren. \bibverse{23} Und das ganze Land weinte mit lauter
Stimme, und alles Volk ging mit. Und der König ging über den Bach
Kidron, und alles Volk ging vor auf dem Wege, der zur Wüste geht.
\bibverse{24} Und siehe, Zadok war auch da und alle Leviten, die bei ihm
waren, und trugen die Lade des Bundes Gottes und stellten sie dahin. Und
Abjathar trat empor, bis dass alles Volk zur Stadt hinauskam.
\bibverse{25} Aber der König sprach zu Zadok: Bringe die Lade Gottes
wieder in die Stadt. Werde ich Gnade finden vor dem HErrn, so wird er
mich wieder holen und wird mich sie sehen lassen und sein Haus.
\bibverse{26} Spricht er aber also: Ich habe nicht Lust zu dir, --
siehe, hier bin ich. Er mache es mit mir, wie es ihm wohl gefällt.
\footnote{\textbf{15:26} 2Sam 10,12; 1Sam 3,18} \bibverse{27} Und der
König sprach zu dem Priester Zadok: O du Seher, kehre um wieder in die
Stadt mit Frieden und mit euch eure beiden Söhne, Ahimaaz, dein Sohn,
und Jonathan, der Sohn Abjathars! \footnote{\textbf{15:27} 1Kö 1,42}
\bibverse{28} Siehe, ich will verziehen auf dem blachen Felde in der
Wüste, bis dass Botschaft von euch komme und sage mir an. \bibverse{29}
Also brachten Zadok und Abjathar die Lade Gottes wieder gen Jerusalem
und blieben daselbst. \bibverse{30} David aber ging den Ölberg hinan und
weinte, und sein Haupt war verhüllt, und er ging barfuß. Dazu alles
Volk, das bei ihm war, hatte ein jeglicher sein Haupt verhüllt und
gingen hinan und weinten.

\bibverse{31} Und da es David angesagt ward, dass Ahithophel im Bund mit
Absalom war, sprach er: HErr, mache den Ratschlag Ahithophels zur
Narrheit!

\bibverse{32} Und da David auf die Höhe kam, da man Gott pflegte
anzubeten, siehe, da begegnete ihm Husai, der Arachiter, mit zerrissenem
Rock und Erde auf seinem Haupt.

\bibverse{33} Und David sprach zu ihm: Wenn du mit mir gehst, wirst du
mir eine Last sein. \bibverse{34} Wenn du aber wieder in die Stadt
gingest und sprächest zu Absalom: Ich bin dein Knecht, ich will des
Königs sein; der ich deines Vaters Knecht war zu der Zeit, will nun dein
Knecht sein: so würdest du mir zugut den Ratschlag Ahithophels zunichte
machen. \footnote{\textbf{15:34} 2Sam 17,7} \bibverse{35} Auch sind
Zadok und Abjathar, die Priester, mit dir. Alles, was du hörtest aus des
Königs Hause, würdest du ansagen den Priestern Zadok und Abjathar.
\bibverse{36} Siehe, es sind bei ihnen ihre zwei Söhne: Ahimaaz, Zadoks,
und Jonathan, Abjathars Sohn. Durch die kannst du mir entbieten, was du
hören wirst.

\bibverse{37} Also kam Husai, der Freund Davids, in die Stadt; und
Absalom kam gen Jerusalem. \footnote{\textbf{15:37} 1Chr 27,33}

\hypertarget{section-2}{%
\section{16}\label{section-2}}

\bibverse{1} Und da David ein wenig von der Höhe gegangen war, siehe, da
begegnete ihm Ziba, der Diener Mephiboseths, mit einem Paar Esel,
gesattelt, darauf waren 200 Brote und 100 Rosinenkuchen und 100
Feigenkuchen und ein Krug Wein. \footnote{\textbf{16:1} 2Sam 9,2}
\bibverse{2} Da sprach der König zu Ziba: Was willst du damit machen?
Ziba sprach: Die Esel sollen für das Haus des Königs sein, darauf zu
reiten, und die Brote und Feigenkuchen für die Diener, zu essen, und der
Wein, zu trinken, wenn sie müde werden in der Wüste.

\bibverse{3} Der König sprach: Wo ist der Sohn deines Herrn? Ziba sprach
zum König: Siehe, er blieb zu Jerusalem; denn er sprach: Heute wird mir
das Haus Israel meines Vaters Reich wiedergeben. \footnote{\textbf{16:3}
  2Sam 19,27}

\bibverse{4} Der König sprach zu Ziba: Siehe, es soll dein sein alles,
was Mephiboseth hat. Ziba sprach: Ich neige mich; lass mich Gnade finden
vor dir, mein Herr König.

\bibverse{5} Da aber der König bis gen Bahurim kam, siehe, da ging ein
Mann daselbst heraus, vom Geschlecht des Hauses Sauls, der hieß Simei,
der Sohn Geras; der ging heraus und fluchte

\bibverse{6} und warf David mit Steinen und alle Knechte des Königs
David. Denn alles Volk und alle Gewaltigen waren zu seiner Rechten und
zur Linken.

\bibverse{7} So sprach aber Simei, da er fluchte: Heraus, heraus, du
Bluthund, du heilloser Mann!

\bibverse{8} Der HErr hat dir vergolten alles Blut des Hauses Sauls,
dass du an seiner Statt bist König geworden. Nun hat der HErr das Reich
gegeben in die Hand deines Sohnes Absalom; und siehe, nun steckst du in
deinem Unglück; denn du bist ein Bluthund.

\bibverse{9} Aber Abisai, der Zeruja Sohn, sprach zu dem König: Sollte
dieser tote Hund meinem Herrn, dem König, fluchen? Ich will hingehen und
ihm den Kopf abreißen. \footnote{\textbf{16:9} 1Sam 26,8} \bibverse{10}
Der König sprach: Ihr Kinder der Zeruja, was habe ich mit euch zu
schaffen? Lasst ihn fluchen; denn der HErr hat's ihn geheißen: Fluche
David! Wer kann nun sagen: Warum tust du also?

\bibverse{11} Und David sprach zu Abisai und zu allen seinen Knechten:
Siehe, mein Sohn, der von meinem Leibe gekommen ist, steht mir nach
meinem Leben; warum nicht auch jetzt der Benjaminiter? Lasst ihn, dass
er fluche; denn der HErr hat's ihn geheißen. \bibverse{12} Vielleicht
wird der HErr mein Elend ansehen und mir mit Gutem vergelten sein
heutiges Fluchen. \bibverse{13} Also ging David mit seinen Leuten des
Weges; aber Simei ging an des Berges Seite her ihm gegenüber und fluchte
und warf mit Steinen nach ihm und besprengte ihn mit Erdklößen.
\bibverse{14} Und der König kam hinein mit allem Volk, das bei ihm war,
müde und erquickte sich daselbst.

\bibverse{15} Aber Absalom und alles Volk der Männer Israels kamen gen
Jerusalem und Ahithophel mit ihm. \bibverse{16} Da aber Husai, der
Arachiter, Davids Freund, zu Absalom hineinkam, sprach er zu Absalom:
Glück zu, Herr König! Glück zu, Herr König!

\bibverse{17} Absalom aber sprach zu Husai: Ist das deine Barmherzigkeit
an deinem Freunde? Warum bist du nicht mit deinem Freunde gezogen?

\bibverse{18} Husai aber sprach zu Absalom: Nicht also, sondern welchen
der HErr erwählt und dieses Volk und alle Männer in Israel, des will ich
sein und bei ihm bleiben. \bibverse{19} Zum anderen, wem sollte ich
dienen? Sollte ich nicht vor seinem Sohn dienen? Wie ich vor deinem
Vater gedient habe, so will ich auch vor dir sein.

\bibverse{20} Und Absalom sprach zu Ahithophel: Ratet zu, was sollen wir
tun?

\bibverse{21} Ahithophel sprach zu Absalom: Gehe hinein zu den
Kebsweibern deines Vaters, die er zurückgelassen hat, das Haus zu
bewahren, so wird das ganze Israel hören, dass du dich bei deinem Vater
hast stinkend gemacht, und wird aller Hand, die bei dir sind, desto
kühner werden. \footnote{\textbf{16:21} 2Sam 15,16}

\bibverse{22} Da machten sie Absalom eine Hütte auf dem Dache, und
Absalom ging hinein zu den Kebsweibern seines Vaters vor den Augen des
ganzen Israel. \footnote{\textbf{16:22} 2Sam 12,11; 3Mo 18,8}

\bibverse{23} Zu der Zeit, wenn Ahithophel einen Rat gab, das war, als
wenn man Gott um etwas hätte gefragt; also waren alle Ratschläge
Ahithophels bei David und bei Absalom. \# 17 \bibverse{1} Und Ahithophel
sprach zu Absalom: Ich will 12.000 Mann auslesen und mich aufmachen und
David nachjagen bei der Nacht \footnote{\textbf{17:1} Ps 71,11}
\bibverse{2} und will ihn überfallen, weil er matt und lass ist. Wenn
ich ihn dann erschrecke, dass alles Volk, das bei ihm ist, flieht, will
ich den König allein schlagen \bibverse{3} und alles Volk wieder zu dir
bringen. Wenn dann jedermann zu dir gebracht ist, wie du begehrst, so
bleibt alles Volk mit Frieden.

\bibverse{4} Das deuchte Absalom gut und alle Ältesten in Israel.
\bibverse{5} Aber Absalom sprach: Lasst doch Husai, den Arachiten, auch
rufen und hören, was er dazu sagt.

\bibverse{6} Und da Husai hinein zu Absalom kam, sprach Absalom zu ihm:
Solches hat Ahithophel geredet; sage du, sollen wir's tun oder nicht?

\bibverse{7} Da sprach Husai zu Absalom: Es ist nicht ein guter Rat, den
Ahithophel auf diesmal gegeben hat. \bibverse{8} Und Husai sprach
weiter: Du kennst deinen Vater wohl und seine Leute, dass sie stark sind
und zornigen Gemüts wie ein Bär auf dem Felde, dem die Jungen geraubt
sind; dazu ist dein Vater ein Kriegsmann und wird sich nicht säumen mit
dem Volk. \bibverse{9} Siehe, er hat sich jetzt vielleicht verkrochen
irgend in einer Grube oder sonst an einen Ort. Wenn's dann geschähe,
dass es das erstemal übel geriete und käme ein Geschrei und spräche: Es
ist das Volk, welches Absalom nachfolgt, geschlagen worden,
\bibverse{10} so würde jedermann verzagt werden, der auch sonst ein
Krieger ist und ein Herz hat wie ein Löwe. Denn es weiß ganz Israel,
dass dein Vater stark ist und Krieger, die bei ihm sind. \bibverse{11}
Aber das rate ich, dass du zu dir versammlest ganz Israel von Dan an bis
gen Beer-Seba, so viel als der Sand am Meer, und deine Person ziehe
unter ihnen. \bibverse{12} So wollen wir ihn überfallen, an welchem Ort
wir ihn finden, und wollen über ihn kommen, wie der Tau auf die Erde
fällt, dass wir von ihm und allen seinen Männern nicht einen
übriglassen. \bibverse{13} Wird er sich aber in eine Stadt versammeln,
so soll das ganze Israel Stricke an die Stadt werfen und sie in den Bach
reißen, dass man nicht ein Kieselein da finde.

\bibverse{14} Da sprach Absalom und jedermann in Israel: Der Rat Husais,
des Arachiten, ist besser denn Ahithophels Rat. Aber der HErr schickte
es also, dass der gute Rat Ahithophels verhindert wurde, auf dass der
HErr Unglück über Absalom brächte. \footnote{\textbf{17:14} 2Sam 15,31;
  2Sam 15,34}

\bibverse{15} Und Husai sprach zu Zadok und Abjathar, den Priestern: So
und so hat Ahithophel Absalom und den Ältesten in Israel geraten; ich
aber habe so und so geraten. \bibverse{16} So sendet nun eilend hin und
lasset David ansagen und sprecht: Bleibe nicht über Nacht auf dem
blachen Felde der Wüste, sondern mache dich hinüber, dass der König
nicht verschlungen werde und alles Volk, das bei ihm ist.

\bibverse{17} Jonathan aber und Ahimaaz standen bei dem Brunnen Rogel,
und eine Magd ging hin und sagte es ihnen an. Sie aber gingen hin und
sagten's dem König David an; denn sie durften sich nicht sehen lassen,
dass sie in die Stadt kämen. \bibverse{18} Es sah sie aber ein Knabe und
sagte es Absalom an. Aber die beiden gingen eilend hin und kamen in
eines Mannes Haus zu Bahurim; der hatte einen Brunnen in seinem Hofe.
Dahinein stiegen sie, \bibverse{19} und das Weib nahm und breitete eine
Decke über des Brunnens Loch und breitete Grütze darüber, dass man es
nicht merkte. \bibverse{20} Da nun die Knechte Absaloms zum Weibe ins
Haus kamen, sprachen sie: Wo ist Ahimaaz und Jonathan? Das Weib sprach
zu ihnen: Sie gingen über das Wässerlein. Und da sie suchten, und nicht
fanden, gingen sie wieder gen Jerusalem.

\bibverse{21} Und da sie weg waren, stiegen jene aus dem Brunnen und
gingen hin und sagten's David, dem König, an und sprachen zu David:
Macht euch auf und gehet eilend über das Wasser; denn so und so hat
Ahithophel wider euch Rat gegeben.

\bibverse{22} Da machte sich David auf und alles Volk, das bei ihm war,
und gingen über den Jordan, bis es lichter Morgen ward, und fehlte nicht
an einem, der nicht über den Jordan gegangen wäre.

\bibverse{23} Als aber Ahithophel sah, dass sein Rat nicht ausgeführt
ward, sattelte er seinen Esel, machte sich auf und zog heim in seine
Stadt und beschickte sein Haus und erhängte sich und starb und ward
begraben in seines Vaters Grab. \footnote{\textbf{17:23} Mt 27,5}

\bibverse{24} Und David kam gen Mahanaim. Und Absalom zog über den
Jordan und alle Männer Israels mit ihm.

\bibverse{25} Und Absalom hatte Amasa an Joabs Statt gesetzt über das
Heer. Es war aber Amasa eines Mannes Sohn, der hieß Jethra, ein
Israeliter, welcher einging zu Abigail, der Tochter des Nahas, der
Schwester der Zeruja, Joabs Mutter. \bibverse{26} Israel aber und
Absalom lagerten sich in Gilead.

\bibverse{27} Da David gen Mahanaim gekommen war, da brachten Sobi, der
Sohn des Nahas von Rabba der Kinder Ammon, und Machir, der Sohn Ammiels
von Lo-Dabar, und Barsillai, ein Gileaditer von Roglim, \footnote{\textbf{17:27}
  2Sam 9,4; 1Kö 2,7} \bibverse{28} Bettwerk, Becken, irdene Gefäße,
Weizen, Gerste, Mehl, geröstete Körner, Bohnen, Linsen, Grütze,
\bibverse{29} Honig, Butter, Schafe und Rinderkäse zu David und zu dem
Volk, das bei ihm war, zu essen. Denn sie gedachten: Das Volk wird
hungrig, müde und durstig sein in der Wüste. \# 18 \bibverse{1} Und
David ordnete das Volk, das bei ihm war, und setzte über sie Hauptleute,
über 1000 und über 100, \bibverse{2} und stellte des Volks einen dritten
Teil unter Joab und einen dritten Teil unter Abisai, den Sohn der
Zeruja, Joabs Bruder, und einen dritten Teil unter Itthai, den Gathiter.
Und der König sprach zum Volk: Ich will auch mit euch ausziehen.
\footnote{\textbf{18:2} 2Sam 15,19}

\bibverse{3} Aber das Volk sprach: Du sollst nicht ausziehen; denn ob
wir gleich fliehen oder die Hälfte sterben, so werden sie unser nicht
achten; denn du bist wie unser 10.000; so ist's nun besser, dass du uns
von der Stadt aus helfen mögest.

\bibverse{4} Der König sprach zu ihnen: Was euch gefällt, das will ich
tun. Und der König trat ans Tor, und alles Volk zog aus bei Hunderten
und bei Tausenden.

\bibverse{5} Und der König gebot Joab und Abisai und Itthai und sprach:
Fahret mir säuberlich mit dem Knaben Absalom! Und alles Volk hörte es,
da der König gebot allen Hauptleuten um Absalom.

\bibverse{6} Und da das Volk hinauskam aufs Feld, Israel entgegen, erhob
sich der Streit im Walde Ephraim. \bibverse{7} Und das Volk Israel ward
daselbst geschlagen vor den Knechten Davids, dass desselben Tages eine
große Schlacht geschah, 20.000 Mann. \bibverse{8} Und war daselbst der
Streit zerstreut auf allem Lande; und der Wald fraß viel mehr Volks des
Tages, denn das Schwert fraß.

\bibverse{9} Und Absalom begegnete den Knechten Davids und ritt auf
einem Maultier. Und da das Maultier unter eine große Eiche mit dichten
Zweigen kam, blieb sein Haupt an der Eiche hangen, und er schwebte
zwischen Himmel und Erde; aber sein Maultier lief unter ihm weg.
\bibverse{10} Da das ein Mann sah, sagte er's Joab an und sprach: Siehe,
ich sah Absalom an einer Eiche hangen.

\bibverse{11} Und Joab sprach zu dem Mann, der's ihm hatte angesagt:
Siehe, sahst du das, warum schlugst du ihn nicht daselbst zur Erde? so
wollte ich dir von meinetwegen zehn Silberlinge und einen Gürtel gegeben
haben.

\bibverse{12} Der Mann sprach zu Joab: Wenn du mir 1000 Silberlinge in
meine Hand gewogen hättest, so wollte ich dennoch meine Hand nicht an
des Königs Sohn gelegt haben; denn der König gebot dir und Abisai und
Itthai vor unseren Ohren und sprach: Hütet euch, dass nicht jemand dem
Knaben Absalom\ldots! \footnote{\textbf{18:12} 2Sam 18,5} \bibverse{13}
Oder wenn ich etwas Falsches getan hätte auf meiner Seele Gefahr, weil
dem König nichts verhohlen wird, würdest du selbst wider mich gestanden
sein.

\bibverse{14} Joab sprach: Ich kann nicht so lange bei dir verziehen. Da
nahm Joab drei Spieße in seine Hand und stieß sie Absalom ins Herz, da
er noch lebte an der Eiche. \bibverse{15} Und zehn Knappen, Joabs
Waffenträger, machten sich umher und schlugen ihn zu Tod. \bibverse{16}
Da blies Joab die Posaune und brachte das Volk wieder, dass es nicht
weiter Israel nachjagte; denn Joab wollte des Volks schonen.
\bibverse{17} Und sie nahmen Absalom und warfen ihn in dem Wald in eine
große Grube und legten einen sehr großen Haufen Steine auf ihn. Und das
ganze Israel floh, ein jeglicher in seine Hütte.

\bibverse{18} Absalom aber hatte sich eine Säule aufgerichtet, da er
noch lebte; die steht im Königsgrunde. Denn er sprach: Ich habe keinen
Sohn, darum soll dies meines Namens Gedächtnis sein; und er hieß die
Säule nach seinem Namen, und sie heißt auch bis auf diesen Tag Absaloms
Mal.

\bibverse{19} Ahimaaz, der Sohn Zadoks, sprach: Lass mich doch laufen
und dem König verkündigen, dass der HErr ihm Recht verschafft hat von
seiner Feinde Händen.

\bibverse{20} Joab aber sprach zu ihm: Du bringst heute keine gute
Botschaft. Einen anderen Tag sollst du Botschaft bringen, und heute
nicht; denn des Königs Sohn ist tot.

\bibverse{21} Aber zu Chusi sprach Joab: Gehe hin und sage dem König an,
was du gesehen hast. Und Chusi neigte sich vor Joab und lief.

\bibverse{22} Ahimaaz aber, der Sohn Zadoks, sprach abermals zu Joab:
Wie, wenn ich auch liefe dem Chusi nach? Joab sprach: Was willst du
laufen, mein Sohn? Komm her, die Botschaft wird dir nichts einbringen.

\bibverse{23} Wie wenn ich liefe? Er sprach zu ihm: So laufe doch! Also
lief Ahimaaz geradeswegs und kam Chusi vor.

\bibverse{24} David aber saß zwischen den beiden Toren. Und der Wächter
ging aufs Dach des Tors an der Mauer und hob seine Augen auf und sah
einen Mann laufen allein

\bibverse{25} und rief und sagte es dem König an. Der König aber sprach:
Ist er allein, so ist eine gute Botschaft in seinem Munde. Und da
derselbe immer näher kam,

\bibverse{26} sah der Wächter einen anderen Mann laufen, und rief in das
Tor und sprach: Siehe, ein Mann läuft allein. Der König aber sprach: Der
ist auch ein guter Bote.

\bibverse{27} Der Wächter sprach: Ich sehe des ersten Lauf wie den Lauf
des Ahimaaz, des Sohnes Zadoks. Und der König sprach: Es ist ein guter
Mann und bringt eine gute Botschaft.

\bibverse{28} Ahimaaz aber rief und sprach zum König: Friede! Und fiel
nieder vor dem König auf sein Antlitz zur Erde und sprach: Gelobt sei
der HErr, dein Gott, der die Leute, die ihre Hand wider meinen Herrn,
den König, aufhoben, übergeben hat.

\bibverse{29} Der König aber sprach: Geht es auch wohl dem Knaben
Absalom? Ahimaaz sprach: Ich sah ein großes Getümmel, da des Königs
Knecht Joab mich, deinen Knecht, sandte, und weiß nicht, was es war.

\bibverse{30} Der König sprach: Gehe herum und tritt daher. Und er ging
herum und stand allda.

\bibverse{31} Siehe, da kam Chusi und sprach: Hier gute Botschaft, mein
Herr König! Der HErr hat dir heute Recht verschafft von der Hand aller,
die sich wider dich auflehnten.

\bibverse{32} Der König aber sprach zu Chusi: Geht es dem Knaben Absalom
auch wohl? Chusi sprach: Es müsse allen Feinden meines Herrn Königs
gehen, wie es dem Knaben geht, und allen, die sich wider dich auflehnen,
übel zu tun. \# 19 \bibverse{1} Da ward der König traurig und ging
hinauf auf den Söller im Tor und weinte, und im Gehen sprach er also:
Mein Sohn Absalom! mein Sohn, mein Sohn Absalom! Wollte Gott, ich wäre
für dich gestorben! O Absalom, mein Sohn, mein Sohn! \bibverse{2} Und es
ward Joab angesagt: Siehe, der König weint und trägt Leid um Absalom.
\bibverse{3} Und es ward aus dem Sieg des Tages ein Leid unter dem
ganzen Volk; denn das Volk hatte gehört des Tages, dass sich der König
um seinen Sohn bekümmerte. \bibverse{4} Und das Volk stahl sich weg an
dem Tage in die Stadt, wie sich ein Volk wegstiehlt, das zu Schanden
geworden ist, wenn's im Streit geflohen ist. \bibverse{5} Der König aber
hatte sein Angesicht verhüllt und schrie laut: Ach, mein Sohn Absalom!
Absalom, mein Sohn, mein Sohn! \bibverse{6} Joab aber kam zum König ins
Haus und sprach: Du hast heute schamrot gemacht alle deine Knechte, die
heute deine, deiner Söhne, deiner Töchter, deiner Weiber und deiner
Kebsweiber Seele errettet haben, \bibverse{7} dass du liebhast, die dich
hassen, und hassest, die dich liebhaben. Denn du lässt heute merken,
dass dir's nicht gelegen ist an den Hauptleuten und Knechten. Denn ich
merke heute wohl: wenn dir nur Absalom lebte und wir heute alle tot
wären, das wäre dir recht. \bibverse{8} So mache dich nun auf und gehe
heraus und rede mit deinen Knechten freundlich. Denn ich schwöre dir bei
dem HErrn: Wirst du nicht herausgehen, es wird kein Mann bei dir bleiben
diese Nacht über. Das wird dir ärger sein denn alles Übel, das über dich
gekommen ist von deiner Jugend auf bis hierher. \bibverse{9} Da machte
sich der König auf und setzte sich ins Tor. Und man sagte es allem Volk:
Siehe, der König sitzt im Tor. Da kam alles Volk vor den König. Aber
Israel war geflohen, ein jeglicher in seine Hütte. \bibverse{10} Und es
zankte sich alles Volk in allen Stämmen Israels und sprachen: Der König
hat uns errettet von der Hand unserer Feinde und erlöste uns von der
Philister Hand und hat müssen aus dem Lande fliehen vor Absalom.
\bibverse{11} So ist Absalom, den wir über uns gesalbt hatten, gestorben
im Streit. Warum seid ihr nun so still, dass ihr den König nicht wieder
holet? \bibverse{12} Der König aber sandte zu Zadok und Abjathar, den
Priestern, und ließ ihnen sagen: Redet mit den Ältesten in Juda und
sprecht: Warum wollt ihr die letzten sein, den König wieder zu holen in
sein Haus? (Denn die Rede des ganzen Israel war vor den König gekommen
in sein Haus.) \bibverse{13} Ihr seid meine Brüder, mein Bein und mein
Fleisch; warum wollt ihr denn die letzten sein, den König wieder zu
holen? \bibverse{14} Und zu Amasa sprecht: Bist du nicht mein Bein und
mein Fleisch? Gott tue mir dies und das, wo du nicht sollst sein
Feldhauptmann vor mir dein Leben lang an Joabs Statt. \footnote{\textbf{19:14}
  2Sam 17,25; 1Chr 2,16-17} \bibverse{15} Und er neigte das Herz aller
Männer Judas wie eines Mannes; und sie sandten hin zum König: Komm
wieder, du und alle deine Knechte! \bibverse{16} Also kam der König
wieder. Und da er an den Jordan kam, waren die Männer Judas gen Gilgal
gekommen, hinabzuziehen dem König entgegen, dass sie den König über den
Jordan führten. \bibverse{17} Und Simei, der Sohn Geras, der
Benjaminiter, der zu Bahurim wohnte, eilte und zog mit den Männern Judas
hinab, dem König David entgegen; \bibverse{18} und waren 1000 Mann mit
ihm von Benjamin, dazu auch Ziba, der Diener des Hauses Sauls, mit
seinen 15 Söhnen und 20 Knechten; und sie gingen durch den Jordan vor
dem König hin; \footnote{\textbf{19:18} 2Sam 16,1-4; 2Sam 9,2; 2Sam 9,10}
\bibverse{19} und die Fähre war hinübergegangen, dass sie das Gesinde
des Königs hinüberführten und täten, was ihm gefiele. Simei aber, der
Sohn Geras, fiel vor dem König nieder, da er über den Jordan fuhr,
\bibverse{20} und sprach zum König: Mein Herr, rechne mir nicht zu die
Missetat und gedenke nicht, dass dein Knecht dich beleidigte des Tages,
da mein Herr, der König, aus Jerusalem ging, und der König nehme es
nicht zu Herzen. \bibverse{21} Denn dein Knecht erkennt, dass ich
gesündigt habe. Und siehe, ich bin heute zuerst gekommen unter dem
ganzen Hause Joseph, dass ich meinem Herrn, dem König, entgegen
herabzöge. \bibverse{22} Aber Abisai, der Zeruja Sohn, antwortete und
sprach: Und Simei sollte darum nicht sterben, obwohl er doch dem
Gesalbten des HErrn geflucht hat? \bibverse{23} David aber sprach: Was
habe ich mit euch zu schaffen, ihr Kinder der Zeruja, dass ihr mir heute
wollt zum Satan werden? Sollte heute jemand sterben in Israel? Meinst
du, ich wisse nicht, dass ich heute König bin geworden über Israel?
\footnote{\textbf{19:23} 2Sam 16,10} \bibverse{24} Und der König sprach
zu Simei: Du sollst nicht sterben. Und der König schwur ihm.
\bibverse{25} Mephiboseth, der Sohn Sauls, kam auch herab, dem König
entgegen. Und er hatte seine Füße und seinen Bart nicht gereinigt und
seine Kleider nicht gewaschen von dem Tage an, da der König weggegangen
war, bis an den Tag, da er mit Frieden kam. \bibverse{26} Da er nun von
Jerusalem kam, dem König zu begegnen, sprach der König zu ihm: Warum
bist du nicht mit mir gezogen, Mephiboseth? \bibverse{27} Und er sprach:
Mein Herr König, mein Knecht hat mich betrogen. Denn dein Knecht
gedachte, ich will einen Esel satteln und darauf reiten und zum König
ziehen; denn dein Knecht ist lahm. \bibverse{28} Dazu hat er deinen
Knecht angegeben vor meinem Herrn, dem König. Aber mein Herr, der König,
ist wie ein Engel Gottes; tue, was dir wohl gefällt. \footnote{\textbf{19:28}
  2Sam 16,3; 2Sam 14,17} \bibverse{29} Denn all meines Vaters Haus ist
nichts gewesen als Leute des Todes vor meinem Herrn, dem König; so hast
du deinen Knecht gesetzt unter die, die an deinem Tisch essen. Was habe
ich weiter Gerechtigkeit oder weiter zu schreien zu dem König?
\footnote{\textbf{19:29} 2Sam 9,11}

\bibverse{30} Der König sprach zu ihm: Was redest du noch weiter von
deinem Dinge? Ich habe es gesagt: Du und Ziba teilet den Acker
miteinander. \footnote{\textbf{19:30} 2Sam 9,9-10; 2Sam 16,4}

\bibverse{31} Mephiboseth sprach zum König: Er nehme ihn auch ganz
dahin, nachdem mein Herr König mit Frieden heimgekommen ist.

\bibverse{32} Und Barsillai, der Gileaditer, kam herab von Roglim und
führte den König über den Jordan, dass er ihn über den Jordan geleitete.

\bibverse{33} Und Barsillai war sehr alt, wohl 80 Jahre, der hatte den
König versorgt, als er zu Mahanaim war; denn er war ein Mann von großem
Vermögen. \footnote{\textbf{19:33} 2Sam 17,27}

\bibverse{34} Und der König sprach zu Barsillai: Du sollst mit mir
hinüberziehen; ich will dich versorgen bei mir zu Jerusalem.

\bibverse{35} Aber Barsillai sprach zum König: was ist's noch, das ich
zu leben habe, dass ich mit dem König sollte hinauf gen Jerusalem
ziehen?

\bibverse{36} Ich bin heute 80 Jahre alt. Wie sollte ich kennen, was gut
oder böse ist, oder schmecken, was ich esse oder trinke, oder hören, was
die Sänger oder Sängerinnen singen? Warum sollte dein Knecht meinen
Herrn König fürder beschweren?

\bibverse{37} Dein Knecht soll ein wenig gehen mit dem König über den
Jordan. Warum will mir der König eine solche Vergeltung tun?

\bibverse{38} Lass deinen Knecht umkehren, dass ich sterbe in meiner
Stadt bei meines Vaters und meiner Mutter Grab. Siehe, da ist dein
Knecht Chimham; den lass mit meinem Herrn König hinüberziehen, und tue
ihm, was dir wohl gefällt.

\bibverse{39} Der König sprach: Chimham soll mit mir hinüberziehen, und
ich will ihm tun, was dir wohl gefällt; auch alles, was du von mir
begehrst, will ich dir tun.

\bibverse{40} Und da alles Volk über den Jordan war gegangen und der
König auch, küsste der König den Barsillai und segnete ihn; und er
kehrte wieder an seinen Ort.

\bibverse{41} Und der König zog hinüber gen Gilgal, und Chimham zog mit
ihm. Und alles Volk Juda hatte den König hinübergeführt; aber des Volkes
Israel war nur die Hälfte da.

\bibverse{42} Und siehe, da kamen alle Männer Israels zum König und
sprachen zu ihm: Warum haben dich unsere Brüder, die Männer Judas,
gestohlen und haben den König und sein Haus über den Jordan geführt und
alle Männer Davids mit ihm?

\bibverse{43} Da antworteten die von Juda denen von Israel: Der König
gehört uns nahe zu; was zürnet ihr darum? Meinet ihr, dass wir von dem
König Nahrung oder Geschenke empfangen haben?

\bibverse{44} So antworteten dann die von Israel denen von Juda und
sprachen: Wir haben zehnmal mehr beim König, dazu auch bei David, denn
ihr. Warum hast du mich denn so gering geachtet? Und haben wir nicht
zuerst davon geredet, uns unseren König zu holen? Aber die von Juda
redeten härter denn die von Israel. \# 20 \bibverse{1} Es traf sich
aber, dass daselbst ein heilloser Mann war, der hieß Seba, ein Sohn
Bichris, ein Benjaminiter; der blies die Posaune und sprach: Wir haben
keinen Teil an David noch Erbe am Sohn Isais. Ein jeglicher hebe sich zu
seiner Hütte, o Israel! \bibverse{2} Da fiel von David jedermann in
Israel, und sie folgten Seba, dem Sohn Bichris. Aber die Männer Judas
hingen an ihrem König vom Jordan an bis gen Jerusalem. \bibverse{3} Da
aber der König David heimkam gen Jerusalem, nahm er die zehn Kebsweiber,
die er hatte zurückgelassen, das Haus zu bewahren, und tat sie in eine
Verwahrung und versorgte sie; aber er ging nicht zu ihnen ein. Und sie
waren also verschlossen bis an ihren Tod und lebten als Witwen.
\footnote{\textbf{20:3} 2Sam 16,21} \bibverse{4} Und der König sprach zu
Amasa: Berufe mir alle Männer in Juda auf den dritten Tag, und du sollst
auch hier stehen! \bibverse{5} Und Amasa ging hin, Juda zu berufen; aber
er verzog die Zeit, die er ihm bestimmt hatte. \bibverse{6} Da sprach
David zu Abisai: Nun wird uns Seba, der Sohn Bichris, mehr Leides tun
denn Absalom. Nimm du die Knechte deines Herrn und jage ihm nach, dass
er nicht etwa für sich feste Städte finde und entrinne aus unseren
Augen. \bibverse{7} Da zogen aus, ihm nach, die Männer Joabs, dazu die
Kreter und Plether und alle Starken. Sie zogen aber aus von Jerusalem,
nachzujagen Seba, dem Sohn Bichris. \bibverse{8} Da sie aber bei dem
großen Stein waren zu Gibeon, kam Amasa vor ihnen her. Joab aber war
gegürtet über seinem Kleide, das er anhatte, und hatte darüber ein
Schwert gegürtet, das hing an seiner Hüfte in der Scheide; das ging
gerne aus und ein. \bibverse{9} Und Joab sprach zu Amasa: Friede mit
dir, mein Bruder! Und Joab fasste mit seiner rechten Hand Amasa bei dem
Bart, dass er ihn küsste. \bibverse{10} Und Amasa hatte nicht Acht auf
das Schwert in der Hand Joabs; und er stach ihn damit in den Bauch, dass
sein Eingeweide sich auf die Erde schüttete, und gab ihm keinen Stich
mehr und er starb. Joab aber und sein Bruder Abisai jagten nach Seba,
dem Sohn Bichris. \footnote{\textbf{20:10} 1Kö 2,5} \bibverse{11} Und es
trat ein Mann von den Leuten Joabs neben ihn und sprach: Wer's mit Joab
hält und für David ist, der folge Joab nach! \bibverse{12} Amasa aber
lag im Blut gewälzt mitten auf der Straße. Da aber der Mann sah, dass
alles Volk da stehenblieb, wandte er Amasa von der Straße auf den Acker
und warf Kleider auf ihn, weil er sah, dass, wer an ihn kam,
stehenblieb. \bibverse{13} Da er nun aus der Straße getan war, folgte
jedermann Joab nach, Seba, dem Sohn Bichris, nachzujagen. \bibverse{14}
Und er zog durch alle Stämme Israels gen Abel und Beth-Maacha und ganz
Habberim; und sie versammelten sich und folgten ihm nach \bibverse{15}
und kamen und belagerten ihn zu Abel-Beth-Maacha und schütteten einen
Wall gegen die Stadt hin, dass er bis an die Vormauer langte; und alles
Volk, das mit Joab war, stürmte und wollte die Mauer niederwerfen.
\bibverse{16} Da rief eine weise Frau aus der Stadt: Höret! höret!
Sprecht zu Joab, dass er hieherzu komme; ich will mit ihm reden.
\bibverse{17} Und da er zu ihr kam, sprach die Frau: Bist du Joab? Er
sprach: Ja. Sie sprach zu ihm: Höre die Rede deiner Magd. Er sprach: Ich
höre. \bibverse{18} Sie sprach: Vorzeiten sprach man: Wer fragen will,
der frage zu Abel; und so ging's wohl aus. \bibverse{19} Ich bin eine
von den friedsamen und treuen Städten in Israel; und du willst die Stadt
und Mutter in Israel töten? Warum willst du das Erbteil des HErrn
verschlingen? \bibverse{20} Joab antwortete und sprach: Das sei ferne,
das sei ferne von mir, dass ich verschlingen und verderben sollte! Es
steht nicht also; \bibverse{21} sondern ein Mann vom Gebirge Ephraim mit
Namen Seba, der Sohn Bichris, hat sich empört wider den König David.
Gebt ihn allein her, so will ich von der Stadt ziehen. Die Frau sprach
zu Joab: Siehe, sein Haupt soll zu dir über die Mauer geworfen werden.
\bibverse{22} Und die Frau kam hinein zu allem Volk mit ihrer Weisheit.
Und sie hieben Seba, dem Sohn Bichris, den Kopf ab und warfen ihn zu
Joab. Da blies er die Posaune, und sie zerstreuten sich von der Stadt,
ein jeglicher in seine Hütte. Joab aber kam wieder gen Jerusalem zum
König. \bibverse{23} Joab aber war über das ganze Heer Israels. Benaja,
der Sohn Jojadas, war über die Kreter und Plether. \bibverse{24} Adoram
war Rentmeister. Josaphat, der Sohn Ahiluds, war Kanzler. \bibverse{25}
Seja war Schreiber. Zadok und Abjathar waren Priester; \bibverse{26}
dazu war Ira, der Jairiter, Davids Priester. \# 21 \bibverse{1} Es war
auch eine Teuerung zu Davids Zeiten drei Jahre aneinander. Und David
suchte das Angesicht des HErrn; und der HErr sprach: Um Sauls willen und
um des Bluthauses willen, dass er die Gibeoniter getötet hat.
\bibverse{2} Da ließ der König die Gibeoniter rufen und sprach zu ihnen.
(Die Gibeoniter aber waren nicht von den Kindern Israel, sondern übrig
von den Amoritern; aber die Kinder Israel hatten ihnen geschworen, und
Saul suchte sie zu schlagen in seinem Eifer für die Kinder Israel und
Juda.) \footnote{\textbf{21:2} Jos 9,15; Jos 9,19} \bibverse{3} So
sprach nun David zu den Gibeonitern: Was soll ich euch tun? und womit
soll ich sühnen, dass ihr das Erbteil des HErrn segnet? \bibverse{4} Die
Gibeoniter sprachen zu ihm: Es ist uns nicht um Gold noch Silber zu tun
an Saul und seinem Hause und steht uns nicht zu, jemand zu töten in
Israel. Er sprach: Was sprecht ihr denn, dass ich euch tun soll?
\bibverse{5} Sie sprachen zum König: Den Mann, der uns verderbt und
zunichte gemacht hat, sollen wir vertilgen, dass ihm nichts bleibe in
allen Grenzen Israels. \bibverse{6} Gebet uns sieben Männer aus seinem
Hause, dass wir sie aufhängen dem HErrn zu Gibea Sauls, des Erwählten
des HErrn. Der König sprach: Ich will sie geben. \bibverse{7} Aber der
König verschonte Mephiboseth, den Sohn Jonathans, des Sohnes Sauls, um
des Eides willen des HErrn, der zwischen ihnen war, zwischen David und
Jonathan, dem Sohn Sauls. \footnote{\textbf{21:7} 1Sam 20,15-17}
\bibverse{8} Aber die zwei Söhne Rizpas, der Tochter Ajas, die sie Saul
geboren hatte, Armoni und Mephiboseth, dazu die fünf Söhne Merabs, der
Tochter Sauls, die sie dem Adriel geboren hatte, dem Sohn Barsillais,
des Meholathiters, nahm der König \footnote{\textbf{21:8} 2Sam 3,7; 1Sam
  18,19} \bibverse{9} und gab sie in die Hand der Gibeoniter; die hingen
sie auf dem Berge vor dem HErrn. Also fielen diese sieben auf einmal und
starben zur Zeit der ersten Ernte, wann die Gerstenernte angeht.
\bibverse{10} Da nahm Rizpa, die Tochter Ajas, einen Sack und breitete
ihn auf den Fels am Anfang der Ernte, bis dass Wasser vom Himmel über
sie troff, und ließ des Tages die Vögel des Himmels nicht auf ihnen
ruhen noch des Nachts die Tiere des Feldes. \bibverse{11} Und es ward
David angesagt, was Rizpa, die Tochter Ajas, Sauls Kebsweib, getan
hatte. \bibverse{12} Und David ging hin und nahm die Gebeine Sauls und
die Gebeine Jonathans, seines Sohnes, von den Bürgern zu Jabes in Gilead
(die sie vom Platz am Tor Beth-Seans gestohlen hatten, dahin sie die
Philister gehängt hatten zu der Zeit, da die Philister Saul schlugen auf
dem Berge Gilboa), \footnote{\textbf{21:12} 1Sam 31,12} \bibverse{13}
und brachte sie von da herauf; und sie sammelten sie zuhauf mit den
Gebeinen der Gehängten \bibverse{14} und begruben die Gebeine Sauls und
seines Sohnes Jonathan im Lande Benjamin zu Zela im Grabe seines Vaters
Kis und taten alles, wie der König geboten hatte. Also ward Gott nach
diesem dem Lande wieder versöhnt. \bibverse{15} Es erhob sich aber
wieder ein Krieg von den Philistern wider Israel; und David zog hinab
und seine Knechte mit ihm und stritten wider die Philister. Und David
ward müde. \bibverse{16} Und Jesbi zu Nob (welcher war der Kinder Raphas
einer, und das Gewicht seines Speers war 300 Gewicht Erzes, und er hatte
neue Waffen), der gedachte David zu schlagen. \bibverse{17} Aber Abisai,
der Zeruja Sohn, half ihm und schlug den Philister tot. Da schwuren ihm
die Männer Davids und sprachen: Du sollst nicht mehr mit uns ausziehen
in den Streit, dass nicht die Leuchte in Israel verlösche. \footnote{\textbf{21:17}
  2Sam 23,18} \bibverse{18} Darnach erhob sich noch ein Krieg zu Gob mit
den Philistern. Da schlug Sibbechai, der Husathiter, den Saph, welcher
auch der Kinder Raphas einer war. \footnote{\textbf{21:18} 1Chr 20,4-8}
\bibverse{19} Und es erhob sich noch ein Krieg zu Gob mit den
Philistern. Da schlug Elhanan, der Sohn Jaere-Orgims, ein Bethlehemiter,
den Goliath, den Gathiter, welcher hatte einen Spieß, des Stange war wie
ein Weberbaum. \footnote{\textbf{21:19} 1Sam 17,7} \bibverse{20} Und es
erhob sich noch ein Krieg zu Gath. Da war ein langer Mann, der hatte
sechs Finger an seinen Händen und sechs Zehen an seinen Füßen, das ist
24 an der Zahl; und er war auch geboren dem Rapha. \bibverse{21} Und da
er Israel Hohn sprach, schlug ihn Jonathan, der Sohn Simeas, des Bruders
Davids. \bibverse{22} Diese vier waren geboren dem Rapha zu Gath und
fielen durch die Hand Davids und seiner Knechte. \# 22 \bibverse{1} Und
David redete vor dem HErrn die Worte dieses Liedes zur Zeit, da ihn der
HErr errettet hatte von der Hand aller seiner Feinde und von der Hand
Sauls, und sprach: \bibverse{2} Der HErr ist mein Fels und meine Burg
und mein Erretter. \bibverse{3} Gott ist mein Hort, auf den ich traue,
mein Schild und Horn meines Heils, mein Schutz und meine Zuflucht, mein
Heiland, der du mir hilfst vor dem Frevel. \bibverse{4} Ich rufe an den
HErrn, den Hochgelobten, so werde ich von meinen Feinden erlöst.
\bibverse{5} Es hatten mich umfangen die Schmerzen des Todes, und die
Bäche des Verderbens erschreckten mich. \bibverse{6} Der Hölle Bande
umfingen mich, und des Todes Stricke überwältigten mich. \bibverse{7} Da
mir angst war, rief ich den HErrn an und schrie zu meinem Gott; da
erhörte er meine Stimme von seinem Tempel, und mein Schreien kam vor ihn
zu seinen Ohren. \bibverse{8} Die Erde bebte und ward bewegt; die
Grundfesten des Himmels regten sich und bebten, da er zornig war.
\bibverse{9} Dampf ging auf von seiner Nase und verzehrend Feuer von
seinem Munde, dass es davon blitzte. \bibverse{10} Er neigte den Himmel
und fuhr herab, und Dunkel war unter seinen Füßen. \bibverse{11} Und er
fuhr auf dem Cherub und flog daher, und er schwebte auf den Fittichen
des Windes. \bibverse{12} Sein Gezelt um ihn her war finster und
schwarze, dicke Wolken. \bibverse{13} Von dem Glanz vor ihm brannte es
mit Blitzen. \bibverse{14} Der HErr donnerte vom Himmel, und der Höchste
ließ seinen Donner aus. \bibverse{15} Er schoss seine Strahlen und
zerstreute sie; er ließ blitzen und schreckte sie. \bibverse{16} Da sah
man das Bett der Wasser, und des Erdbodens Grund ward aufgedeckt von dem
Schelten des HErrn, von dem Odem und Schnauben seiner Nase.
\bibverse{17} Er streckte seine Hand aus von der Höhe und holte mich und
zog mich aus großen Wassern. \bibverse{18} Er errettete mich von meinen
starken Feinden, von meinen Hassern, die mir zu mächtig waren,
\bibverse{19} die mich überwältigten zur Zeit meines Unglücks; und der
HErr ward meine Zuversicht. \bibverse{20} Und er führte mich aus in das
Weite, er riss mich heraus; denn er hatte Lust zu mir. \bibverse{21} Der
HErr tut wohl an mir nach meiner Gerechtigkeit; er vergilt mir nach der
Reinigkeit meiner Hände. \bibverse{22} Denn ich halte die Wege des HErrn
und bin nicht gottlos wider meinen Gott. \bibverse{23} Denn alle seine
Rechte habe ich vor Augen, und seine Gebote werfe ich nicht von mir;
\bibverse{24} sondern ich bin ohne Tadel vor ihm und hüte mich vor
Sünden. \bibverse{25} Darum vergilt mir der HErr nach meiner
Gerechtigkeit, nach meiner Reinigkeit vor seinen Augen. \bibverse{26}
Bei den Heiligen bist du heilig, bei den Frommen bist du fromm,
\bibverse{27} bei den Reinen bist du rein, und bei den Verkehrten bist
du verkehrt. \bibverse{28} Denn du hilfst dem elenden Volk, und mit
deinen Augen erniedrigst du die Hohen. \bibverse{29} Denn du, HErr, bist
meine Leuchte; der HErr macht meine Finsternis licht.

\bibverse{30} Denn mit dir kann ich Kriegsvolk zerschlagen und mit
meinem Gott über die Mauer springen.

\bibverse{31} Gottes Wege sind vollkommen; des HErrn Reden sind
durchläutert. Er ist ein Schild allen, die ihm vertrauen. \bibverse{32}
Denn wo ist ein Gott außer dem HErrn, und wo ist ein Hort außer unserem
Gott? \bibverse{33} Gott stärkt mich mit Kraft und weist mir einen Weg
ohne Tadel.

\bibverse{34} Er macht meine Füße gleich den Hirschen und stellt mich
auf meine Höhen. \bibverse{35} Er lehrt meine Hände streiten und lehrt
meinen Arm den ehernen Bogen spannen. \bibverse{36} Du gibst mir den
Schild deines Heils; und wenn du mich demütigst, machst du mich groß.
\bibverse{37} Du machst unter mir Raum zu gehen, dass meine Knöchel
nicht wanken.

\bibverse{38} Ich will meinen Feinden nachjagen und sie vertilgen und
will nicht umkehren, bis ich sie umgebracht habe.

\bibverse{39} Ich will sie umbringen und zerschmettern; sie sollen mir
nicht widerstehen und müssen unter meine Füße fallen. \bibverse{40} Du
kannst mich rüsten mit Stärke zum Streit; du kannst unter mich werfen,
die sich wider mich setzen. \bibverse{41} Du gibst mir meine Feinde in
die Flucht, dass ich verstöre, die mich hassen.

\bibverse{42} Sie sehen sich um -- aber da ist kein Helfer -- nach dem
HErrn; aber er antwortet ihnen nicht.

\bibverse{43} Ich will sie zerstoßen wie Staub auf der Erde; wie Kot auf
der Gasse will ich sie verstäuben und zerstreuen.

\bibverse{44} Du hilfst mir von dem zänkischen Volk und behütest mich,
dass ich ein Haupt sei unter den Heiden; ein Volk, das ich nicht kannte,
dient mir.

\bibverse{45} Den Kindern der Fremde hat's wider mich gefehlt; sie
gehorchen mir mit gehorsamen Ohren.

\bibverse{46} Die Kinder der Fremde sind verschmachtet und kommen mit
Zittern aus ihren Burgen.

\bibverse{47} Der HErr lebt, und gelobt sei mein Hort; und Gott, der
Hort meines Heils, werde erhoben,

\bibverse{48} der Gott, der mir Rache gibt und wirft die Völker unter
mich.

\bibverse{49} Er hilft mir aus von meinen Feinden. Du erhöhest mich aus
denen, die sich wider mich setzen; du hilfst mir von den Frevlern.

\bibverse{50} Darum will ich dir danken, HErr, unter den Heiden und
deinem Namen lobsingen,

\bibverse{51} der seinem König großes Heil beweist und wohltut seinem
Gesalbten, David und seinem Samen ewiglich. \# 23 \bibverse{1} Dies sind
die letzten Worte Davids: Es sprach David, der Sohn Isais, es sprach der
Mann, der hoch erhoben ist, der Gesalbte des Gottes Jakobs, lieblich mit
Psalmen Israels. \bibverse{2} Der Geist des HErrn hat durch mich
geredet, und seine Rede ist auf meiner Zunge. \bibverse{3} Es hat der
Gott Israels zu mir gesprochen, der Hort Israels hat geredet: Ein
Gerechter herrscht unter den Menschen, er herrscht in der Furcht Gottes
\bibverse{4} und ist wie das Licht des Morgens, wenn die Sonne aufgeht
am Morgen ohne Wolken, da vom Glanz nach dem Regen das Gras aus der Erde
wächst. \bibverse{5} Denn ist mein Haus nicht also bei Gott? Denn er hat
mir einen ewigen Bund gesetzt, der in allem wohl geordnet und gehalten
wird. All mein Heil und all mein Begehren, das wird er wachsen lassen.
\bibverse{6} Aber die heillosen Leute sind allesamt wie die
ausgeworfenen Disteln, die man nicht mit Händen fassen kann;
\bibverse{7} sondern wer sie angreifen soll, muss Eisen und Spießstange
in der Hand haben; sie werden mit Feuer verbrannt an ihrem Ort.
\bibverse{8} Dies sind die Namen der Helden Davids: Jasobeam, der Sohn
Hachmonis, ein Vornehmster unter den Rittern; er hob seinen Spieß auf
und schlug 800 auf einmal. \bibverse{9} Nach ihm war Eleasar, der Sohn
Dodos, des Sohnes Ahohis, unter den drei Helden mit David. Da sie Hohn
sprachen den Philistern und daselbst versammelt waren zum Streit und die
Männer Israels hinaufzogen, \bibverse{10} da stand er und schlug die
Philister, bis dass seine Hand müde am Schwert erstarrte. Und der HErr
gab ein großes Heil zu der Zeit, dass das Volk umwandte ihm nach, zu
rauben. \bibverse{11} Nach ihm war Samma, der Sohn Ages, des Harariters.
Da die Philister sich versammelten in eine Rotte -- und war daselbst ein
Stück Acker voll Linsen, und das Volk floh vor den Philistern --,
\bibverse{12} da trat er mitten auf das Stück und errettete es und
schlug die Philister; und Gott gab ein großes Heil. \bibverse{13} Und
diese drei Vornehmsten unter dreißigen kamen hinab in der Ernte zu David
in der Höhle Adullam, und die Rotte der Philister lag im Grunde Rephaim.
\bibverse{14} David aber war dazumal an sicherem Ort; aber der Philister
Volk lag zu Bethlehem. \bibverse{15} Und David ward lüstern und sprach:
Wer will mir Wasser zu trinken holen aus dem Brunnen zu Bethlehem unter
dem Tor? \bibverse{16} Da brachen die drei Helden ins Lager der
Philister und schöpften Wasser aus dem Brunnen zu Bethlehem unter dem
Tor und trugen's und brachten's zu David. Aber er wollte nicht trinken,
sondern goss es aus dem HErrn \bibverse{17} und sprach: Das lasse der
HErr fern von mir sein, dass ich das tue! Ist's nicht das Blut der
Männer, die ihr Leben gewagt haben und dahin gegangen sind? Und wollte
es nicht trinken. Das taten die drei Helden.

\bibverse{18} Abisai, Joabs Bruder, der Zeruja Sohn, war auch ein
Vornehmster unter den Rittern: er hob seinen Spieß auf und schlug 300,
und war auch berühmt unter dreien \footnote{\textbf{23:18} 2Sam 21,17}

\bibverse{19} und der Herrlichste unter dreien und war ihr Oberster;
aber er kam nicht bis an jene drei.

\bibverse{20} Und Benaja, der Sohn Jojadas, des Sohnes Is-Hails, von
großen Taten, von Kabzeel, der schlug zwei Helden der Moabiter und ging
hinab und schlug einen Löwen im Brunnen zur Schneezeit. \bibverse{21}
Und schlug auch einen ägyptischen ansehnlichen Mann, der hatte einen
Spieß in seiner Hand. Er aber ging zu ihm hinab mit einem Stecken und
riss dem Ägypter den Spieß aus der Hand und erwürgte ihn mit seinem
eigenen Spieß.

\bibverse{22} Das tat Benaja, der Sohn Jojadas, und war berühmt unter
den drei Helden

\bibverse{23} und herrlicher denn die 30; aber er kam nicht bis an jene
drei. Und David machte ihn zum heimlichen Rat.

\bibverse{24} Asahel, der Bruder Joabs, war unter den 30; Elhanan, der
Sohn Dodos, zu Bethlehem; \footnote{\textbf{23:24} 2Sam 2,18}

\bibverse{25} Samma, der Haroditer; Elika, der Haroditer;

\bibverse{26} Helez, der Paltiter; Ira, der Sohn des Ikkes, des
Thekoiters;

\bibverse{27} Abieser, der Anathothiter; Mebunnai, der Husathiter;

\bibverse{28} Zalmon, der Ahohiter; Maherai, der Netophathiter;

\bibverse{29} Heleb, der Sohn Baanas, der Netophathiter; Itthai, der
Sohn Ribais, von Gibea der Kinder Benjamin;

\bibverse{30} Benaja, der Pirathoniter; Hiddai, von Nahale-Gaas;

\bibverse{31} Abi-Albon, der Arbathiter; Asmaveth, der Barhumiter;

\bibverse{32} Eljahba, der Saalboniter; die Kinder Jasen und Jonathan;

\bibverse{33} Samma, der Harariter; Ahiam, der Sohn Sarars, der
Harariter;

\bibverse{34} Eliphelet, der Sohn Ahasbais, des Maachathiters; Eliam,
der Sohn Ahithophels, des Giloniters; \footnote{\textbf{23:34} 2Sam
  15,12}

\bibverse{35} Hezrai, der Karmeliter; Paerai, der Arbiter;

\bibverse{36} Jigeal, der Sohn Nathans, von Zoba; Bani, der Gaditer;

\bibverse{37} Zelek, der Ammoniter; Naharai, der Beerothiter, der
Waffenträger Joabs, des Sohnes der Zeruja;

\bibverse{38} Ira, der Jethriter; Gareb, der Jethriter;

\bibverse{39} Uria, der Hethiter. Das sind allesamt 37. \# 24
\bibverse{1} Und der Zorn des HErrn ergrimmte abermals wider Israel, und
er reizte David wider sie, dass er sprach: Gehe hin, zähle Israel und
Juda! \footnote{\textbf{24:1} 2Sam 21,1} \bibverse{2} Und der König
sprach zu Joab, seinem Feldhauptmann: Gehe umher in allen Stämmen
Israels von Dan an bis gen Beer-Seba und zähle das Volk, dass ich wisse,
wieviel sein ist!

\bibverse{3} Joab sprach zu dem König: Der HErr, dein Gott, tue zu
diesem Volk, wie es jetzt ist, noch hundertmal soviel, dass mein Herr,
der König, seiner Augen Lust daran sehe; aber was hat mein Herr König zu
dieser Sache Lust?

\bibverse{4} Aber des Königs Wort stand fest wider Joab und die
Hauptleute des Heeres. Also zog Joab aus und die Hauptleute des Heeres
von dem König, dass sie das Volk Israel zählten. \bibverse{5} Und sie
gingen über den Jordan und lagerten sich zu Aroer, zur Rechten der
Stadt, die am Bach Gad liegt, und gen Jaser hin, \bibverse{6} und kamen
gen Gilead und ins Niederland Hodsi, und kamen gen Dan-Jaan und um Sidon
her, \bibverse{7} und kamen zu der festen Stadt Tyrus und allen Städten
der Heviter und Kanaaniter, und kamen hinaus an den Mittag Judas gen
Beer-Seba, \bibverse{8} und durchzogen das ganze Land, und kamen nach
neun Monaten und 20 Tagen gen Jerusalem. \bibverse{9} Und Joab gab dem
König die Summe des Volks, das gezählt war. Und es waren in Israel
800.000 starke Männer, die das Schwert auszogen, und in Juda 500.000
Mann.

\bibverse{10} Und das Herz schlug David, nachdem das Volk gezählt war.
Und David sprach zum HErrn: Ich habe schwer gesündigt, dass ich das
getan habe; und nun, HErr, nimm weg die Missetat deines Knechtes; denn
ich habe sehr töricht getan.

\bibverse{11} Und da David des Morgens aufstand, kam des HErrn Wort zu
Gad, dem Propheten, Davids Schauer, und sprach: \bibverse{12} Gehe hin
und rede mit David: So spricht der HErr: Dreierlei bringe ich zu dir;
erwähle dir deren eins, dass ich es dir tue.

\bibverse{13} Gad kam zu David und sagte es ihm an und sprach zu ihm:
Willst du, dass sieben Jahre Teuerung in dein Land komme? oder dass du
drei Monate vor deinen Widersachern fliehen müssest und sie dich
verfolgen? oder dass drei Tage Pestilenz in deinem Lande sei? So merke
nun und siehe, was ich wieder sagen soll dem, der mich gesandt hat.
\footnote{\textbf{24:13} Jer 24,10; Jer 29,17; Hes 6,12}

\bibverse{14} David sprach zu Gad: Es ist mir sehr angst; aber lass uns
in die Hand des HErrn fallen, denn seine Barmherzigkeit ist groß; ich
will nicht in der Menschen Hand fallen.

\bibverse{15} Also ließ der HErr Pestilenz in Israel kommen vom Morgen
an bis zur bestimmten Zeit, dass des Volks starb von Dan an bis gen
Beer-Seba 70.000 Mann. \bibverse{16} Und da der Engel seine Hand
ausstreckte über Jerusalem, dass er es verderbte, reute den HErrn das
Übel, und er sprach zum Engel, zu dem Verderber im Volk: Es ist genug;
lass nun deine Hand ab! Der Engel aber des HErrn war bei der Tenne
Aravnas, des Jebusiters.

\bibverse{17} Da aber David den Engel sah, der das Volk schlug, sprach
er zum HErrn: Siehe, ich habe gesündigt, ich habe die Missetat getan;
was haben diese Schafe getan? Lass deine Hand wider mich und meines
Vaters Haus sein!

\bibverse{18} Und Gad kam zu David zur selben Zeit und sprach zu ihm:
Gehe hinauf und richte dem HErrn einen Altar auf in der Tenne Aravnas,
des Jebusiters!

\bibverse{19} Also ging David hinauf, wie Gad gesagt und der HErr
geboten hatte. \bibverse{20} Und da Aravna sich wandte, sah er den König
mit seinen Knechten zu ihm herüberkommen und fiel nieder auf sein
Angesicht zur Erde \bibverse{21} und sprach: Warum kommt mein Herr, der
König, zu seinem Knecht? David sprach: Zu kaufen von dir die Tenne und
zu bauen dem HErrn einen Altar, dass die Plage vom Volk aufhöre.

\bibverse{22} Aber Aravna sprach zu David: Mein Herr, der König, nehme
und opfere, wie es ihm gefällt: siehe, da ist ein Rind zum Brandopfer
und Schleifen und Geschirr vom Ochsen zu Holz.

\bibverse{23} Das alles gab Aravna, der König, dem König. Und Aravna
sprach zum König: Der HErr, dein Gott, lasse dich ihm angenehm sein.

\bibverse{24} Aber der König sprach zu Aravna: Nicht also, sondern ich
will dir's abkaufen um seinen Preis; denn ich will dem HErrn, meinem
Gott, nicht Brandopfer tun, das ich umsonst habe. Also kaufte David die
Tenne und das Rind um 50 Silberlinge \bibverse{25} und baute daselbst
dem HErrn einen Altar und opferte Brandopfer und Dankopfer. Und der HErr
ward dem Land versöhnt, und die Plage hörte auf von dem Volk Israel.
