\hypertarget{section}{%
\section{1}\label{section}}

\bibverse{1} Im Anfang war das Wort, und das Wort war bei Gott, und Gott
war das Wort. \footnote{\textbf{1:1} 1Mo 1,1; 1Jo 1,1-2; Joh 17,5; Offb
  19,13} \bibverse{2} Dasselbe war im Anfang bei Gott. \bibverse{3} Alle
Dinge sind durch dasselbe gemacht, und ohne dasselbe ist nichts gemacht,
was gemacht ist. \bibverse{4} In ihm war das Leben, und das Leben war
das Licht der Menschen. \footnote{\textbf{1:4} Joh 8,12} \bibverse{5}
Und das Licht scheint in der Finsternis, und die Finsternis hat's nicht
begriffen. \footnote{\textbf{1:5} Joh 3,19}

\bibverse{6} Es ward ein Mensch, von Gott gesandt, der hieß Johannes.
\footnote{\textbf{1:6} Mt 3,1; Mk 1,4} \bibverse{7} Dieser kam zum
Zeugnis, dass er von dem Licht zeugte, auf dass sie alle durch ihn
glaubten. \footnote{\textbf{1:7} Apg 19,4} \bibverse{8} Er war nicht das
Licht, sondern dass er zeugte von dem Licht. \bibverse{9} Das war das
wahrhaftige Licht, welches alle Menschen erleuchtet, die in diese Welt
kommen.

\bibverse{10} Es war in der Welt, und die Welt ist durch dasselbe
gemacht; und die Welt kannte es nicht. \bibverse{11} Er kam in sein
Eigentum; und die Seinen nahmen ihn nicht auf. \bibverse{12} Wie viele
ihn aber aufnahmen, denen gab er Macht, Kinder Gottes zu werden, die an
seinen Namen glauben; \footnote{\textbf{1:12} Gal 3,26} \bibverse{13}
welche nicht von dem Geblüt noch von dem Willen des Fleisches noch von
dem Willen eines Mannes, sondern von Gott geboren sind. \footnote{\textbf{1:13}
  Joh 3,5-6}

\bibverse{14} Und das Wort ward Fleisch und wohnte unter uns, und wir
sahen seine Herrlichkeit, eine Herrlichkeit als des eingeborenen Sohnes
vom Vater, voller Gnade und Wahrheit. \footnote{\textbf{1:14} Jes 7,14;
  Jes 60,1; 2Petr 1,16-17} \bibverse{15} Johannes zeugt von ihm, ruft
und spricht: Dieser war es, von dem ich gesagt habe: Nach mir wird
kommen, der vor mir gewesen ist; denn er war eher als ich. \bibverse{16}
Und von seiner Fülle haben wir alle genommen Gnade um Gnade.
\bibverse{17} Denn das Gesetz ist durch Moses gegeben; die Gnade und
Wahrheit ist durch Jesum Christum geworden. \footnote{\textbf{1:17} Röm
  10,4} \bibverse{18} Niemand hat Gott je gesehen; der eingeborene Sohn,
der in des Vaters Schoß ist, der hat es uns verkündigt. \footnote{\textbf{1:18}
  Joh 6,46; Mt 11,27}

\bibverse{19} Und dies ist das Zeugnis des Johannes, da die Juden
sandten von Jerusalem Priester und Leviten, dass sie ihn fragten: Wer
bist du?

\bibverse{20} Und er bekannte und leugnete nicht; und er bekannte: Ich
bin nicht Christus.

\bibverse{21} Und sie fragten ihn: Was denn? Bist du Elia? Er sprach:
Ich bin's nicht. -- Bist du der Prophet? Und er antwortete: Nein!
\footnote{\textbf{1:21} Mt 17,10-13; 5Mo 18,15; Mal 3,23}

\bibverse{22} Da sprachen sie zu ihm: Was bist du denn? dass wir Antwort
geben denen, die uns gesandt haben. Was sagst du von dir selbst?

\bibverse{23} Er sprach: Ich bin eine Stimme eines Predigers in der
Wüste: Richtet den Weg des Herrn! wie der Prophet Jesaja gesagt hat.

\bibverse{24} Und die gesandt waren, die waren von den Pharisäern.

\bibverse{25} Und sie fragten ihn und sprachen zu ihm: Warum taufst du
denn, so du nicht Christus bist noch Elia noch der Prophet?

\bibverse{26} Johannes antwortete ihnen und sprach: Ich taufe mit
Wasser; aber er ist mitten unter euch getreten, den ihr nicht kennet.

\bibverse{27} Der ist's, der nach mir kommen wird, welcher vor mir
gewesen ist, des ich nicht wert bin, dass ich seine Schuhriemen auflöse.

\bibverse{28} Dies geschah zu Bethabara jenseits des Jordans, wo
Johannes taufte.

\bibverse{29} Des anderen Tages sieht Johannes Jesum zu ihm kommen und
spricht: Siehe, das ist Gottes Lamm, welches der Welt Sünde trägt!
\footnote{\textbf{1:29} Jes 53,7} \bibverse{30} Dieser ist's, von dem
ich gesagt habe: Nach mir kommt ein Mann, welcher vor mir gewesen ist;
denn er war eher denn ich. \bibverse{31} Und ich kannte ihn nicht;
sondern auf dass er offenbar würde in Israel, darum bin ich gekommen, zu
taufen mit Wasser. \bibverse{32} Und Johannes zeugte und sprach: Ich
sah, dass der Geist herabfuhr wie eine Taube vom Himmel und blieb auf
ihm. \bibverse{33} Und ich kannte ihn nicht; aber der mich sandte, zu
taufen mit Wasser, der sprach zu mir: Auf welchen du sehen wirst den
Geist herabfahren und auf ihm bleiben, der ist's, der mit dem heiligen
Geist tauft. \bibverse{34} Und ich sah es und zeugte, dass dieser ist
Gottes Sohn.

\bibverse{35} Des anderen Tages stand abermals Johannes und zwei seiner
Jünger. \bibverse{36} Und als er sah Jesum wandeln, sprach er: Siehe,
das ist Gottes Lamm! \bibverse{37} Und die zwei Jünger hörten ihn reden
und folgten Jesu nach. \bibverse{38} Jesus aber wandte sich um und sah
sie nachfolgen und sprach zu ihnen: Was suchet ihr? Sie aber sprachen zu
ihm: Rabbi (das ist verdolmetscht: Meister), wo bist du zur Herberge?

\bibverse{39} Er sprach zu ihnen: Kommt und sehet's! Sie kamen und
sahen's und blieben den Tag bei ihm. Es war aber um die zehnte Stunde.

\bibverse{40} Einer aus den zweien, die von Johannes hörten und Jesus
nachfolgten, war Andreas, der Bruder des Simon Petrus.

\bibverse{41} Der findet am ersten seinen Bruder Simon und spricht zu
ihm: Wir haben den Messias gefunden (welches ist verdolmetscht: der
Gesalbte), \bibverse{42} und führte ihn zu Jesu. Da ihn Jesus sah,
sprach er: Du bist Simon, Jona's Sohn; du sollst Kephas heißen (das ist
verdolmetscht: ein Fels). \footnote{\textbf{1:42} Mt 16,18}

\bibverse{43} Des anderen Tages wollte Jesus wieder nach Galiläa ziehen
und findet Philippus und spricht zu ihm: Folge mir nach! \bibverse{44}
Philippus aber war von Bethsaida, aus der Stadt des Andreas und Petrus.
\bibverse{45} Philippus findet Nathanael und spricht zu ihm: Wir haben
den gefunden, von welchem Mose im Gesetz und die Propheten geschrieben
haben, Jesum, Josephs Sohn von Nazareth.

\bibverse{46} Und Nathanael sprach zu ihm: Was kann von Nazareth Gutes
kommen? Philippus spricht zu ihm: Komm und sieh es! \footnote{\textbf{1:46}
  Joh 7,41}

\bibverse{47} Jesus sah Nathanael zu sich kommen und spricht von ihm:
Siehe, ein rechter Israeliter, in welchem kein Falsch ist.

\bibverse{48} Nathanael spricht zu ihm: Woher kennst du mich? Jesus
antwortete und sprach zu ihm: Ehe denn dich Philippus rief, da du unter
dem Feigenbaum warst, sah ich dich.

\bibverse{49} Nathanael antwortete und spricht zu ihm: Rabbi, du bist
Gottes Sohn, du bist der König von Israel!

\bibverse{50} Jesus antwortete und sprach zu ihm: Du glaubst, weil ich
dir gesagt habe, dass ich dich gesehen habe unter dem Feigenbaum; du
wirst noch Größeres denn das sehen.

\bibverse{51} Und spricht zu ihm: Wahrlich, wahrlich ich sage euch: Von
nun an werdet ihr den Himmel offen sehen und die Engel Gottes hinauf und
herab fahren auf des Menschen Sohn. \footnote{\textbf{1:51} 1Mo 28,12;
  Mt 4,11}

\hypertarget{section-1}{%
\section{2}\label{section-1}}

\bibverse{1} Und am dritten Tage ward eine Hochzeit zu Kana in Galiläa;
und die Mutter Jesu war da. \bibverse{2} Jesus aber und seine Jünger
wurden auch auf die Hochzeit geladen. \bibverse{3} Und da es an Wein
gebrach, spricht die Mutter Jesu zu ihm: Sie haben nicht Wein.

\bibverse{4} Jesus spricht zu ihr: Weib, was habe ich mit dir zu
schaffen? Meine Stunde ist noch nicht gekommen.

\bibverse{5} Seine Mutter spricht zu den Dienern: Was er euch sagt, das
tut.

\bibverse{6} Es waren aber allda sechs steinerne Wasserkrüge gesetzt
nach der Weise der jüdischen Reinigung, und ging in je einen zwei oder
drei Maß. \footnote{\textbf{2:6} Mk 7,3-4} \bibverse{7} Jesus spricht zu
ihnen: Füllet die Wasserkrüge mit Wasser! Und sie füllten sie bis
obenan. \bibverse{8} Und er spricht zu ihnen: Schöpfet nun und bringet's
dem Speisemeister! Und sie brachten's. \bibverse{9} Als aber der
Speisemeister kostete den Wein, der Wasser gewesen war, und wusste
nicht, woher er kam (die Diener aber wussten's, die das Wasser geschöpft
hatten), ruft der Speisemeister den Bräutigam \bibverse{10} und spricht
zu ihm: Jedermann gibt zum ersten guten Wein, und wenn sie trunken
geworden sind, alsdann den geringeren; du hast den guten Wein bisher
behalten. \bibverse{11} Das ist das erste Zeichen, das Jesus tat,
geschehen zu Kana in Galiläa, und offenbarte seine Herrlichkeit. Und
seine Jünger glaubten an ihn.

\bibverse{12} Darnach zog er hinab gen Kapernaum, er, seine Mutter,
seine Brüder und seine Jünger; und sie blieben nicht lange daselbst.
\footnote{\textbf{2:12} Joh 7,3; Mt 13,55}

\bibverse{13} Und der Juden Ostern war nahe, und Jesus zog hinauf gen
Jerusalem. \footnote{\textbf{2:13} Mt 20,18; Mk 11,1; Lk 19,28; Joh 5,1}
\bibverse{14} Und er fand im Tempel sitzen, die da Ochsen, Schafe und
Tauben feil hatten, und die Wechsler. \bibverse{15} Und er machte eine
Geißel aus Stricken und trieb sie alle zum Tempel hinaus samt den
Schafen und Ochsen und verschüttete den Wechslern das Geld und stieß die
Tische um \bibverse{16} und sprach zu denen, die die Tauben feil hatten:
Traget das von dannen und machet nicht meines Vaters Haus zum Kaufhause!
\bibverse{17} Seine Jünger aber gedachten daran, dass geschrieben steht:
„Der Eifer um dein Haus hat mich gefressen.``

\bibverse{18} Da antworteten nun die Juden und sprachen zu ihm: Was
zeigst du uns für ein Zeichen, dass du solches tun mögest? \footnote{\textbf{2:18}
  Mt 21,3}

\bibverse{19} Jesus antwortete und sprach zu ihnen: Brechet diesen
Tempel, und am dritten Tage will ich ihn aufrichten. \footnote{\textbf{2:19}
  Mt 26,61; Mt 27,40}

\bibverse{20} Da sprachen die Juden: Dieser Tempel ist in
sechsundvierzig Jahren erbaut; und du willst ihn in drei Tagen
aufrichten? \bibverse{21} (Er aber redete von dem Tempel seines Leibes.
\footnote{\textbf{2:21} 1Kor 6,19} \bibverse{22} Da er nun auferstanden
war von den Toten, gedachten seine Jünger daran, dass er dies gesagt
hatte, und glaubten der Schrift und der Rede, die Jesus gesagt hatte.)
\footnote{\textbf{2:22} Hos 6,2}

\bibverse{23} Als er aber zu Jerusalem war am Osterfest, glaubten viele
an seinen Namen, da sie die Zeichen sahen, die er tat. \bibverse{24}
Aber Jesus vertraute sich ihnen nicht; denn er kannte sie alle
\bibverse{25} und bedurfte nicht, dass jemand Zeugnis gäbe von einem
Menschen; denn er wusste wohl, was im Menschen war. \footnote{\textbf{2:25}
  Mk 2,8}

\hypertarget{section-2}{%
\section{3}\label{section-2}}

\bibverse{1} Es war aber ein Mensch unter den Pharisäern mit Namen
Nikodemus, ein Oberster unter den Juden. \footnote{\textbf{3:1} Joh
  7,50; Joh 19,39} \bibverse{2} Der kam zu Jesu bei der Nacht und sprach
zu ihm: Meister, wir wissen, dass du bist ein Lehrer von Gott gekommen;
denn niemand kann die Zeichen tun, die du tust, es sei denn Gott mit
ihm.

\bibverse{3} Jesus antwortete und sprach zu ihm: Wahrlich, wahrlich, ich
sage dir: Es sei denn, dass jemand von neuem geboren werde, so kann er
das Reich Gottes nicht sehen. \footnote{\textbf{3:3} 1Petr 1,23}

\bibverse{4} Nikodemus spricht zu ihm: Wie kann ein Mensch geboren
werden wenn er alt ist? Kann er auch wiederum in seiner Mutter Leib
gehen und geboren werden?

\bibverse{5} Jesus antwortete: Wahrlich, wahrlich ich sage dir: Es sei
denn, dass jemand geboren werde aus Wasser und Geist, so kann er nicht
in das Reich Gottes kommen. \bibverse{6} Was vom Fleisch geboren wird,
das ist Fleisch; und was vom Geist geboren wird, das ist Geist.
\footnote{\textbf{3:6} Joh 1,13; Röm 8,5-9} \bibverse{7} Lass dich's
nicht wundern, dass ich dir gesagt habe: Ihr müsset von neuem geboren
werden. \bibverse{8} Der Wind bläst, wo er will, und du hörst sein
Sausen wohl; aber du weißt nicht, woher er kommt und wohin er fährt.
Also ist ein jeglicher, der aus dem Geist geboren ist.

\bibverse{9} Nikodemus antwortete und sprach zu ihm: Wie mag solches
zugehen?

\bibverse{10} Jesus antwortete und sprach zu ihm: Bist du ein Meister in
Israel und weißt das nicht? \bibverse{11} Wahrlich, wahrlich ich sage
dir: Wir reden, was wir wissen, und zeugen, was wir gesehen haben; und
ihr nehmt unser Zeugnis nicht an. \bibverse{12} Glaubet ihr nicht, wenn
ich euch von irdischen Dingen sage, wie würdet ihr glauben, wenn ich
euch von himmlischen Dingen sagen würde? \bibverse{13} Und niemand fährt
gen Himmel, denn der vom Himmel herniedergekommen ist, nämlich des
Menschen Sohn, der im Himmel ist. \bibverse{14} Und wie Mose in der
Wüste eine Schlange erhöht hat, also muss des Menschen Sohn erhöht
werden, \bibverse{15} auf dass alle, die an ihn glauben, nicht verloren
werden, sondern das ewige Leben haben. \bibverse{16} Also hat Gott die
Welt geliebt, dass er seinen eingeborenen Sohn gab, auf dass alle, die
an ihn glauben, nicht verloren werden, sondern das ewige Leben haben.
\footnote{\textbf{3:16} Röm 5,8; Röm 8,32; 1Jo 4,9} \bibverse{17} Denn
Gott hat seinen Sohn nicht gesandt in die Welt, dass er die Welt richte,
sondern dass die Welt durch ihn selig werde. \footnote{\textbf{3:17} Lk
  19,10} \bibverse{18} Wer an ihn glaubt, der wird nicht gerichtet; wer
aber nicht glaubt, der ist schon gerichtet, denn er glaubt nicht an den
Namen des eingeborenen Sohnes Gottes. \footnote{\textbf{3:18} Joh 5,24}
\bibverse{19} Das ist aber das Gericht, dass das Licht in die Welt
gekommen ist, und die Menschen liebten die Finsternis mehr als das
Licht; denn ihre Werke waren böse. \footnote{\textbf{3:19} Joh 1,5; Joh
  1,9-11} \bibverse{20} Wer Arges tut, der hasst das Licht und kommt
nicht an das Licht, auf dass seine Werke nicht gestraft werden.
\footnote{\textbf{3:20} Eph 5,13} \bibverse{21} Wer aber die Wahrheit
tut, der kommt an das Licht, dass seine Werke offenbar werden; denn sie
sind in Gott getan. \footnote{\textbf{3:21} 1Jo 1,6}

\bibverse{22} Darnach kam Jesus und seine Jünger in das jüdische Land
und hatte daselbst sein Wesen mit ihnen und taufte. \footnote{\textbf{3:22}
  Joh 4,1-2} \bibverse{23} Johannes aber taufte auch noch zu Enon, nahe
bei Salim, denn es war viel Wasser daselbst; und sie kamen dahin und
ließen sich taufen. \bibverse{24} Denn Johannes war noch nicht ins
Gefängnis gelegt. \bibverse{25} Da erhob sich eine Frage unter den
Jüngern des Johannes mit den Juden über die Reinigung. \bibverse{26} Und
sie kamen zu Johannes und sprachen zu ihm: Meister, der bei dir war
jenseits des Jordans, von dem du zeugtest, siehe, der tauft, und
jedermann kommt zu ihm. \footnote{\textbf{3:26} Joh 1,26-34}

\bibverse{27} Johannes antwortete und sprach: Ein Mensch kann nichts
nehmen, es werde ihm denn gegeben vom Himmel. \footnote{\textbf{3:27}
  Hebr 5,4} \bibverse{28} Ihr selbst seid meine Zeugen, dass ich gesagt
habe, ich sei nicht Christus, sondern vor ihm her gesandt. \footnote{\textbf{3:28}
  Joh 1,20; Joh 1,23; Joh 1,27} \bibverse{29} Wer die Braut hat, der ist
der Bräutigam; der Freund aber des Bräutigams steht und hört ihm zu und
freut sich hoch über des Bräutigams Stimme. Diese meine Freude ist nun
erfüllt. \footnote{\textbf{3:29} Mt 9,15} \bibverse{30} Er muss wachsen,
ich aber muss abnehmen.

\bibverse{31} Der von obenher kommt, ist über alle. Wer von der Erde
ist, der ist von der Erde und redet von der Erde. Der vom Himmel kommt,
der ist über alle \footnote{\textbf{3:31} Joh 8,23} \bibverse{32} und
zeugt, was er gesehen und gehört hat; und -- sein Zeugnis nimmt niemand
an. \bibverse{33} Wer es aber annimmt, der besiegelt's, dass Gott
wahrhaftig sei. \bibverse{34} Denn welchen Gott gesandt hat, der redet
Gottes Worte; denn Gott gibt den Geist nicht nach dem Maß. \bibverse{35}
Der Vater hat den Sohn lieb und hat ihm alles in seine Hand gegeben.
\footnote{\textbf{3:35} Joh 5,20; Mt 11,27} \bibverse{36} Wer an den
Sohn glaubt, der hat das ewige Leben. Wer dem Sohn nicht glaubt, der
wird das Leben nicht sehen, sondern der Zorn Gottes bleibt über ihm. \#
4 \bibverse{1} Da nun der Herr inneward, dass vor die Pharisäer gekommen
war, wie Jesus mehr Jünger machte und taufte denn Johannes \bibverse{2}
(wiewohl Jesus selber nicht taufte, sondern seine Jünger), \bibverse{3}
verließ er das Land Judäa und zog wieder nach Galiläa. \bibverse{4} Er
musste aber durch Samaria reisen. \bibverse{5} Da kam er in eine Stadt
Samarias, die heißt Sichar, nahe bei dem Feld, das Jakob seinem Sohn
Joseph gab. \footnote{\textbf{4:5} 1Mo 48,22; Jos 24,32} \bibverse{6} Es
war aber daselbst Jakobs Brunnen. Da nun Jesus müde war von der Reise,
setzte er sich also auf den Brunnen; und es war um die sechste Stunde.

\bibverse{7} Da kommt ein Weib aus Samaria, Wasser zu schöpfen. Jesus
spricht zu ihr: Gib mir zu trinken! \bibverse{8} (Denn seine Jünger
waren in die Stadt gegangen, dass sie Speise kauften.)

\bibverse{9} Spricht nun das samaritische Weib zu ihm: Wie bittest du
von mir zu trinken, so du ein Jude bist, und ich ein samaritisch Weib?
(Denn die Juden haben keine Gemeinschaft mit den Samaritern.)

\bibverse{10} Jesus antwortete und sprach zu ihr: Wenn du erkenntest die
Gabe Gottes und wer der ist, der zu dir sagt: „Gib mir zu trinken!{}``,
du bätest ihn, und er gäbe dir lebendiges Wasser. \footnote{\textbf{4:10}
  Joh 7,38-39}

\bibverse{11} Spricht zu ihm das Weib: Herr, hast du doch nichts, womit
du schöpfest, und der Brunnen ist tief; woher hast du denn lebendiges
Wasser? \bibverse{12} Bist du mehr denn unser Vater Jakob, der uns
diesen Brunnen gegeben hat? Und er hat daraus getrunken und seine Kinder
und sein Vieh.

\bibverse{13} Jesus antwortete und sprach zu ihr: Wer von diesem Wasser
trinkt, den wird wieder dürsten; \bibverse{14} wer aber von dem Wasser
trinken wird, das ich ihm gebe, den wird ewiglich nicht dürsten; sondern
das Wasser, das ich ihm geben werde, das wird in ihm ein Brunnen des
Wassers werden, das in das ewige Leben quillt. \footnote{\textbf{4:14}
  Joh 6,35; Joh 7,38-39}

\bibverse{15} Spricht das Weib zu ihm: Herr, gib mir dieses Wasser, auf
dass mich nicht dürste und ich nicht herkommen müsse, zu schöpfen!

\bibverse{16} Jesus spricht zu ihr: Gehe hin, rufe deinen Mann und komm
her!

\bibverse{17} Das Weib antwortete und sprach zu ihm: Ich habe keinen
Mann. Jesus spricht zu ihr: Du hast recht gesagt: Ich habe keinen Mann.

\bibverse{18} Fünf Männer hast du gehabt, und den du nun hast, der ist
nicht dein Mann; da hast du recht gesagt.

\bibverse{19} Das Weib spricht zu ihm: Herr, ich sehe, dass du ein
Prophet bist. \bibverse{20} Unsere Väter haben auf diesem Berge
angebetet, und ihr sagt, zu Jerusalem sei die Stätte, da man anbeten
solle.

\bibverse{21} Jesus spricht zu ihr: Weib, glaube mir, es kommt die Zeit,
dass ihr weder auf diesem Berge noch zu Jerusalem werdet den Vater
anbeten. \bibverse{22} Ihr wisset nicht, was ihr anbetet; wir wissen
aber, was wir anbeten, denn das Heil kommt von den Juden. \footnote{\textbf{4:22}
  2Kö 17,29-41; Jes 2,3} \bibverse{23} Aber es kommt die Zeit und ist
schon jetzt, dass die wahrhaftigen Anbeter werden den Vater anbeten im
Geist und in der Wahrheit; denn der Vater will haben, die ihn also
anbeten. \bibverse{24} Gott ist Geist, und die ihn anbeten, die müssen
ihn im Geist und in der Wahrheit anbeten.

\bibverse{25} Spricht das Weib zu ihm: Ich weiß, dass der Messias kommt,
der da Christus heißt. Wenn derselbe kommen wird, so wird er's uns alles
verkündigen. \footnote{\textbf{4:25} Joh 1,41}

\bibverse{26} Jesus spricht zu ihr: Ich bin's, der mit dir redet.

\bibverse{27} Und über dem kamen seine Jünger, und es nahm sie wunder,
dass er mit dem Weib redete. Doch sprach niemand: Was fragst du? oder:
Was redest du mit ihr? \bibverse{28} Da ließ das Weib ihren Krug stehen
und ging hin in die Stadt und spricht zu den Leuten: \bibverse{29}
Kommt, sehet einen Menschen, der mir gesagt hat alles, was ich getan
habe, ob er nicht Christus sei! \bibverse{30} Da gingen sie aus der
Stadt und kamen zu ihm.

\bibverse{31} Indes aber ermahnten ihn die Jünger und sprachen: Rabbi,
iss!

\bibverse{32} Er aber sprach zu ihnen: Ich habe eine Speise zu essen,
von der ihr nicht wisset.

\bibverse{33} Da sprachen die Jünger untereinander: Hat ihm jemand zu
essen gebracht?

\bibverse{34} Jesus spricht zu ihnen: Meine Speise ist die, dass ich tue
den Willen des, der mich gesandt hat, und vollende sein Werk.
\bibverse{35} Saget ihr nicht: Es sind noch vier Monate, so kommt die
Ernte? Siehe, ich sage euch: Hebet eure Augen auf und sehet in das Feld;
denn es ist schon weiß zur Ernte. \footnote{\textbf{4:35} Mt 9,37}
\bibverse{36} Und wer da schneidet, der empfängt Lohn und sammelt Frucht
zum ewigen Leben, auf dass sich miteinander freuen, der da sät und der
da schneidet. \bibverse{37} Denn hier ist der Spruch wahr: Dieser sät,
der andere schneidet. \bibverse{38} Ich habe euch gesandt, zu schneiden,
was ihr nicht gearbeitet habt; andere haben gearbeitet, und ihr seid in
ihre Arbeit gekommen.

\bibverse{39} Es glaubten aber an ihn viele der Samariter aus der Stadt
um des Weibes Rede willen, welches da zeugte: Er hat mir gesagt alles,
was ich getan habe. \bibverse{40} Als nun die Samariter zu ihm kamen,
baten sie ihn, dass er bei ihnen bliebe; und er blieb zwei Tage da.
\bibverse{41} Und viel mehr glaubten um seines Wortes willen
\bibverse{42} und sprachen zum Weibe: Wir glauben nun hinfort nicht um
deiner Rede willen; wir haben selber gehört und erkannt, dass dieser ist
wahrlich Christus, der Welt Heiland.

\bibverse{43} Aber nach zwei Tagen zog er aus von dannen und zog nach
Galiläa. \footnote{\textbf{4:43} Mt 4,12} \bibverse{44} Denn er selber,
Jesus, zeugte, dass ein Prophet daheim nichts gilt. \footnote{\textbf{4:44}
  Mt 13,57} \bibverse{45} Da er nun nach Galiläa kam, nahmen ihn die
Galiläer auf, die gesehen hatten alles, was er zu Jerusalem auf dem Fest
getan hatte; denn sie waren auch zum Fest gekommen. \footnote{\textbf{4:45}
  Joh 2,23} \bibverse{46} Und Jesus kam abermals gen Kana in Galiläa, da
er das Wasser hatte zu Wein gemacht. \footnote{\textbf{4:46} Joh 2,1;
  Joh 2,9} \bibverse{47} Und es war ein Königischer, des Sohn lag krank
zu Kapernaum. Dieser hörte, dass Jesus kam aus Judäa nach Galiläa, und
ging hin zu ihm und bat ihn, dass er hinabkäme und hülfe seinem Sohn;
denn er war todkrank. \bibverse{48} Und Jesus sprach zu ihm: Wenn ihr
nicht Zeichen und Wunder sehet, so glaubet ihr nicht. \footnote{\textbf{4:48}
  Joh 2,18; 1Kor 1,22}

\bibverse{49} Der Königische sprach zu ihm: Herr, komm hinab, ehe denn
mein Kind stirbt!

\bibverse{50} Jesus spricht zu ihm: Gehe hin, dein Sohn lebt! Der Mensch
glaubte dem Wort, das Jesus zu ihm sagte, und ging hin. \bibverse{51}
Und indem er hinabging, begegneten ihm seine Knechte, verkündigten ihm
und sprachen: Dein Kind lebt. \bibverse{52} Da forschte er von ihnen die
Stunde, in welcher es besser mit ihm geworden war. Und sie sprachen zu
ihm: Gestern um die siebente Stunde verließ ihn das Fieber.
\bibverse{53} Da merkte der Vater, dass es um die Stunde wäre, in
welcher Jesus zu ihm gesagt hatte: Dein Sohn lebt. Und er glaubte mit
seinem ganzen Hause. \bibverse{54} Das ist nun das andere Zeichen, das
Jesus tat, da er aus Judäa nach Galiläa kam. \# 5 \bibverse{1} Darnach
war ein Fest der Juden, und Jesus zog hinauf gen Jerusalem. \footnote{\textbf{5:1}
  Joh 2,13} \bibverse{2} Es ist aber zu Jerusalem bei dem Schaftor ein
Teich, der heißt auf hebräisch Bethesda und hat fünf Hallen, \footnote{\textbf{5:2}
  Neh 3,1} \bibverse{3} in welchem lagen viele Kranke, Blinde, Lahme,
Verdorrte, die warteten, wann sich das Wasser bewegte. \bibverse{4}
(Denn ein Engel fuhr herab zu seiner Zeit in den Teich und bewegte das
Wasser.) Welcher nun zuerst, nachdem das Wasser bewegt war, hineinstieg,
der ward gesund, mit welcherlei Seuche er behaftet war. \bibverse{5} Es
war aber ein Mensch daselbst, achtunddreißig Jahre lang krank gelegen.
\bibverse{6} Da Jesus ihn sah liegen und vernahm, dass er so lange
gelegen hatte, spricht er zu ihm: Willst du gesund werden?

\bibverse{7} Der Kranke antwortete ihm: Herr, ich habe keinen Menschen,
wenn das Wasser sich bewegt, der mich in den Teich lasse; und wenn ich
komme, so steigt ein anderer vor mir hinein.

\bibverse{8} Jesus spricht zu ihm: Stehe auf, nimm dein Bett und gehe
hin!

\bibverse{9} Und alsbald ward der Mensch gesund und nahm sein Bett und
ging hin. Es war aber desselben Tages der Sabbat.

\bibverse{10} Da sprachen die Juden zu dem, der geheilt worden war: Es
ist heute Sabbat; es ziemt dir nicht, das Bett zu tragen. \footnote{\textbf{5:10}
  Jer 17,21-22}

\bibverse{11} Er antwortete ihnen: Der mich gesund machte, der sprach zu
mir: „Nimm dein Bett und gehe hin!{}``

\bibverse{12} Da fragten sie ihn: Wer ist der Mensch, der zu dir gesagt
hat: „Nimm dein Bett und gehe hin!{}``?

\bibverse{13} Der aber geheilt worden war, wusste nicht, wer es war;
denn Jesus war gewichen, da so viel Volks an dem Ort war.

\bibverse{14} Darnach fand ihn Jesus im Tempel und sprach zu ihm: Siehe
zu, du bist gesund geworden; sündige hinfort nicht mehr, dass dir nicht
etwas Ärgeres widerfahre.

\bibverse{15} Der Mensch ging hin und verkündete es den Juden, es sei
Jesus, der ihn gesund gemacht habe. \bibverse{16} Darum verfolgten die
Juden Jesum und suchten ihn zu töten, dass er solches getan hatte am
Sabbat. \footnote{\textbf{5:16} Mt 12,14} \bibverse{17} Jesus aber
antwortete ihnen: Mein Vater wirkt bisher, und ich wirke auch.
\footnote{\textbf{5:17} Joh 9,4}

\bibverse{18} Darum trachteten ihm die Juden nun viel mehr nach, dass
sie ihn töteten, dass er nicht allein den Sabbat brach, sondern sagte
auch, Gott sei sein Vater, und machte sich selbst Gott gleich.
\footnote{\textbf{5:18} Joh 7,30; Joh 10,33} \bibverse{19} Da antwortete
Jesus und sprach zu ihnen: Wahrlich, wahrlich ich sage euch: Der Sohn
kann nichts von sich selber tun, sondern was er sieht den Vater tun;
denn was dieser tut, das tut gleicherweise auch der Sohn. \footnote{\textbf{5:19}
  Joh 3,11; Joh 3,32} \bibverse{20} Der Vater aber hat den Sohn lieb und
zeigt ihm alles, was er tut, und wird ihm noch größere Werke zeigen,
dass ihr euch verwundern werdet. \footnote{\textbf{5:20} Joh 3,35}
\bibverse{21} Denn wie der Vater die Toten auferweckt und macht sie
lebendig, also auch der Sohn macht lebendig, welche er will.
\bibverse{22} Denn der Vater richtet niemand; sondern alles Gericht hat
er dem Sohn gegeben, \bibverse{23} auf dass sie alle den Sohn ehren, wie
sie den Vater ehren. Wer den Sohn nicht ehrt, der ehrt den Vater nicht,
der ihn gesandt hat. \footnote{\textbf{5:23} Phil 2,10-11; 1Jo 2,23}

\bibverse{24} Wahrlich, wahrlich ich sage euch: Wer mein Wort hört und
glaubt dem, der mich gesandt hat, der hat das ewige Leben und kommt
nicht in das Gericht, sondern er ist vom Tode zum Leben
hindurchgedrungen. \footnote{\textbf{5:24} Joh 3,16; Joh 3,18}
\bibverse{25} Wahrlich, wahrlich ich sage euch: Es kommt die Stunde und
ist schon jetzt, dass die Toten werden die Stimme des Sohnes Gottes
hören; und die sie hören werden, die werden leben. \footnote{\textbf{5:25}
  Eph 2,5-6} \bibverse{26} Denn wie der Vater hat das Leben in ihm
selber, also hat er dem Sohn gegeben, das Leben zu haben in ihm selber,
\footnote{\textbf{5:26} Joh 1,1-4} \bibverse{27} und hat ihm Macht
gegeben, auch das Gericht zu halten, darum dass er des Menschen Sohn
ist. \footnote{\textbf{5:27} Dan 7,13-14} \bibverse{28} Verwundert euch
des nicht. Denn es kommt die Stunde, in welcher alle, die in den Gräbern
sind, werden seine Stimme hören, \bibverse{29} und werden hervorgehen,
die da Gutes getan haben, zur Auferstehung des Lebens, die aber Übles
getan haben, zur Auferstehung des Gerichts. \bibverse{30} Ich kann
nichts von mir selber tun. Wie ich höre, so richte ich, und mein Gericht
ist recht; denn ich suche nicht meinen Willen, sondern des Vaters
Willen, der mich gesandt hat. \footnote{\textbf{5:30} Joh 6,38}

\bibverse{31} So ich von mir selbst zeuge, so ist mein Zeugnis nicht
wahr. \bibverse{32} Ein anderer ist's, der von mir zeugt; und ich weiß,
dass das Zeugnis wahr ist, das er von mir zeugt. \bibverse{33} Ihr
schicktet zu Johannes, und er zeugte von der Wahrheit. \bibverse{34} Ich
aber nehme nicht Zeugnis von Menschen; sondern solches sage ich, auf
dass ihr selig werdet. \bibverse{35} Er war ein brennend und scheinend
Licht; ihr aber wolltet eine kleine Weile fröhlich sein in seinem
Lichte. \bibverse{36} Ich aber habe ein größeres Zeugnis; denn des
Johannes Zeugnis; denn die Werke, die mir der Vater gegeben hat, dass
ich sie vollende, eben diese Werke, die ich tue, zeugen von mir, dass
mich der Vater gesandt habe. \footnote{\textbf{5:36} Joh 3,2; Joh 10,25;
  Joh 10,38} \bibverse{37} Und der Vater, der mich gesandt hat, derselbe
hat von mir gezeugt. Ihr habt nie weder seine Stimme gehört noch seine
Gestalt gesehen, \footnote{\textbf{5:37} Mt 3,17} \bibverse{38} und sein
Wort habt ihr nicht in euch wohnend; denn ihr glaubet dem nicht, den er
gesandt hat.

\bibverse{39} Suchet in der Schrift; denn ihr meinet, ihr habet das
ewige Leben darin; und sie ist's, die von mir zeuget; \footnote{\textbf{5:39}
  Lk 24,27; Lk 24,44; 2Tim 3,15-17} \bibverse{40} und ihr wollt nicht zu
mir kommen, dass ihr das Leben haben möchtet. \bibverse{41} Ich nehme
nicht Ehre von Menschen; \bibverse{42} aber ich kenne euch, dass ihr
nicht Gottes Liebe in euch habt. \bibverse{43} Ich bin gekommen in
meines Vaters Namen, und ihr nehmet mich nicht an. So ein anderer wird
in seinem eigenen Namen kommen, den werdet ihr annehmen. \bibverse{44}
Wie könnet ihr glauben, die ihr Ehre voneinander nehmet? und die Ehre,
die von Gott allein ist, suchet ihr nicht. \footnote{\textbf{5:44} Joh
  12,42-43; 1Thes 2,6}

\bibverse{45} Ihr sollt nicht meinen, dass ich euch vor dem Vater
verklagen werde; es ist einer, der euch verklagt, der Mose, auf welchen
ihr hoffet. \footnote{\textbf{5:45} 5Mo 31,26-27} \bibverse{46} Wenn ihr
Mose glaubtet, so glaubtet ihr auch mir; denn er hat von mir
geschrieben. \footnote{\textbf{5:46} 1Mo 3,15; 1Mo 49,10; 5Mo 18,15}
\bibverse{47} So ihr aber seinen Schriften nicht glaubet, wie werdet ihr
meinen Worten glauben? \footnote{\textbf{5:47} Lk 16,31}

\hypertarget{section-3}{%
\section{6}\label{section-3}}

\bibverse{1} Darnach fuhr Jesus weg über das Meer an der Stadt Tiberias
in Galiläa. \bibverse{2} Und es zog ihm viel Volks nach, darum dass sie
die Zeichen sahen, die er an den Kranken tat. \bibverse{3} Jesus aber
ging hinauf auf einen Berg und setzte sich daselbst mit seinen Jüngern.
\bibverse{4} Es war aber nahe Ostern, der Juden Fest. \footnote{\textbf{6:4}
  Joh 2,13; Joh 11,55} \bibverse{5} Da hob Jesus seine Augen auf und
sieht, dass viel Volks zu ihm kommt, und spricht zu Philippus: Wo kaufen
wir Brot, dass diese essen? \bibverse{6} (Das sagte er aber, ihn zu
versuchen; denn er wusste wohl, was er tun wollte.)

\bibverse{7} Philippus antwortete ihm: Für zweihundert Groschen Brot ist
nicht genug unter sie, dass ein jeglicher unter ihnen ein wenig nehme.

\bibverse{8} Spricht zu ihm einer seiner Jünger, Andreas, der Bruder des
Simon Petrus: \bibverse{9} Es ist ein Knabe hier, der hat fünf
Gerstenbrote und zwei Fische; aber was ist das unter so viele?

\bibverse{10} Jesus aber sprach: Schaffet, dass sich das Volk lagere. Es
war aber viel Gras an dem Ort. Da lagerten sich bei fünftausend Mann.
\bibverse{11} Jesus aber nahm die Brote, dankte und gab sie den Jüngern,
die Jünger aber denen, die sich gelagert hatten; desgleichen auch von
den Fischen, wieviel sie wollten. \bibverse{12} Da sie aber satt waren,
sprach er zu seinen Jüngern: Sammelt die übrigen Brocken, dass nichts
umkomme. \bibverse{13} Da sammelten sie und füllten zwölf Körbe mit
Brocken von den fünf Gerstenbroten, die übrig blieben denen, die
gespeist worden. \bibverse{14} Da nun die Menschen das Zeichen sahen,
das Jesus tat, sprachen sie: Das ist wahrlich der Prophet, der in die
Welt kommen soll. \bibverse{15} Da Jesus nun merkte, dass sie kommen
würden und ihn haschen, dass sie ihn zum König machten, entwich er
abermals auf den Berg, er selbst allein. \footnote{\textbf{6:15} Joh
  18,36}

\bibverse{16} Am Abend aber gingen die Jünger hinab an das Meer
\bibverse{17} und traten in das Schiff und kamen über das Meer gen
Kapernaum. Und es war schon finster geworden, und Jesus war nicht zu
ihnen gekommen. \bibverse{18} Und das Meer erhob sich von einem großen
Winde. \bibverse{19} Da sie nun gerudert hatten bei fünfundzwanzig oder
dreißig Feld Wegs, sahen sie Jesum auf dem Meere dahergehen und nahe zum
Schiff kommen; und sie fürchteten sich. \bibverse{20} Er aber sprach zu
ihnen: Ich bin's; fürchtet euch nicht! \bibverse{21} Da wollten sie ihn
in das Schiff nehmen; und alsbald war das Schiff am Lande, da sie hin
fuhren.

\bibverse{22} Des anderen Tages sah das Volk, das diesseits des Meers
stand, dass kein anderes Schiff daselbst war denn das eine, darein seine
Jünger getreten waren, und dass Jesus nicht mit seinen Jüngern in das
Schiff getreten war, sondern allein seine Jünger waren weggefahren.
\bibverse{23} Es kamen aber andere Schiffe von Tiberias nahe zur Stätte,
da sie das Brot gegessen hatten durch des Herrn Danksagung.
\bibverse{24} Da nun das Volk sah, dass Jesus nicht da war noch seine
Jünger, traten sie auch in die Schiffe und kamen gen Kapernaum und
suchten Jesum. \bibverse{25} Und da sie ihn fanden jenseits des Meers,
sprachen sie zu ihm: Rabbi, wann bist du hergekommen?

\bibverse{26} Jesus antwortete ihnen und sprach: Wahrlich, wahrlich ich
sage euch: Ihr suchet mich nicht darum, dass ihr Zeichen gesehen habt,
sondern dass ihr von dem Brot gegessen habt und seid satt geworden.
\bibverse{27} Wirket Speise, nicht, die vergänglich ist, sondern die da
bleibt in das ewige Leben, welche euch des Menschen Sohn geben wird;
denn den hat Gott der Vater versiegelt.

\bibverse{28} Da sprachen sie zu ihm: Was sollen wir tun, dass wir
Gottes Werke wirken?

\bibverse{29} Jesus antwortete und sprach zu ihnen: Das ist Gottes Werk,
dass ihr an den glaubet, den er gesandt hat.

\bibverse{30} Da sprachen sie zu ihm: Was tust du denn für ein Zeichen,
auf dass wir sehen und glauben dir? Was wirkst du? \bibverse{31} Unsere
Väter haben Manna gegessen in der Wüste, wie geschrieben steht: „Er gab
ihnen Brot vom Himmel zu essen.`` \footnote{\textbf{6:31} 2Mo 16,13-14}

\bibverse{32} Da sprach Jesus zu ihnen: Wahrlich, wahrlich ich sage
euch: Mose hat euch nicht das Brot vom Himmel gegeben, sondern mein
Vater gibt euch das rechte Brot vom Himmel. \bibverse{33} Denn dies ist
das Brot Gottes, das vom Himmel kommt und gibt der Welt das Leben.

\bibverse{34} Da sprachen sie zu ihm: Herr, gib uns allewege solch Brot.

\bibverse{35} Jesus aber sprach zu ihnen: Ich bin das Brot des Lebens.
Wer zu mir kommt, den wird nicht hungern; und wer an mich glaubt, den
wird nimmermehr dürsten. \bibverse{36} Aber ich habe es euch gesagt,
dass ihr mich gesehen habt, und glaubet doch nicht. \bibverse{37} Alles,
was mir mein Vater gibt, das kommt zu mir; und wer zu mir kommt, den
werde ich nicht hinausstoßen. \footnote{\textbf{6:37} Mt 11,28}
\bibverse{38} Denn ich bin vom Himmel gekommen, nicht dass ich meinen
Willen tue, sondern den Willen des, der mich gesandt hat. \footnote{\textbf{6:38}
  Joh 4,34} \bibverse{39} Das ist aber der Wille des Vaters, der mich
gesandt hat, dass ich nichts verliere von allem, was er mir gegeben hat,
sondern dass ich's auferwecke am Jüngsten Tage. \footnote{\textbf{6:39}
  Joh 10,28-29; Joh 17,12} \bibverse{40} Denn das ist der Wille des, der
mich gesandt hat, dass, wer den Sohn sieht und glaubt an ihn, habe das
ewige Leben; und ich werde ihn auferwecken am Jüngsten Tage. \footnote{\textbf{6:40}
  Joh 5,29; Joh 11,24}

\bibverse{41} Da murrten die Juden darüber, dass er sagte: Ich bin das
Brot, dass vom Himmel gekommen ist, \bibverse{42} und sprachen: Ist
dieser nicht Jesus, Josephs Sohn, des Vater und Mutter wir kennen? Wie
spricht er denn: Ich bin vom Himmel gekommen? \footnote{\textbf{6:42} Lk
  4,22}

\bibverse{43} Jesus antwortete und sprach zu ihnen: Murret nicht
untereinander. \bibverse{44} Es kann niemand zu mir kommen, es sei denn,
dass ihn ziehe der Vater, der mich gesandt hat; und ich werde ihn
auferwecken am Jüngsten Tage. \bibverse{45} Es steht geschrieben in den
Propheten: „Sie werden alle von Gott gelehrt sein.`` Wer es nun hört vom
Vater und lernt es, der kommt zu mir. \bibverse{46} Nicht dass jemand
den Vater habe gesehen, außer dem, der vom Vater ist; der hat den Vater
gesehen. \bibverse{47} Wahrlich, wahrlich ich sage euch: Wer an mich
glaubt, der hat das ewige Leben. \footnote{\textbf{6:47} Joh 3,16}
\bibverse{48} Ich bin das Brot des Lebens. \footnote{\textbf{6:48} Joh
  6,35} \bibverse{49} Eure Väter haben Manna gegessen in der Wüste und
sind gestorben. \footnote{\textbf{6:49} 1Kor 10,3-5} \bibverse{50} Dies
ist das Brot, das vom Himmel kommt, auf dass, wer davon isset, nicht
sterbe. \bibverse{51} Ich bin das lebendige Brot, vom Himmel gekommen.
Wer von diesem Brot essen wird, der wird leben in Ewigkeit. Und das
Brot, dass ich geben werde, ist mein Fleisch, welches ich geben werde
für das Leben der Welt.

\bibverse{52} Da zankten die Juden untereinander und sprachen: Wie kann
dieser uns sein Fleisch zu essen geben?

\bibverse{53} Jesus sprach zu ihnen: Wahrlich, wahrlich ich sage euch:
Werdet ihr nicht essen das Fleisch des Menschensohnes und trinken sein
Blut, so habt ihr kein Leben in euch. \bibverse{54} Wer mein Fleisch
isset und trinket mein Blut, der hat das ewige Leben, und ich werde ihn
am Jüngsten Tage auferwecken. \bibverse{55} Denn mein Fleisch ist die
rechte Speise, und mein Blut ist der rechte Trank. \bibverse{56} Wer
mein Fleisch isset und trinket mein Blut, der bleibt in mir und ich in
ihm. \footnote{\textbf{6:56} Joh 15,4; 1Jo 3,24} \bibverse{57} Wie mich
gesandt hat der lebendige Vater und ich lebe um des Vaters willen, also,
wer mich isset, der wird auch leben um meinetwillen. \bibverse{58} Dies
ist das Brot, das vom Himmel gekommen ist; nicht, wie eure Väter haben
Manna gegessen und sind gestorben: wer dies Brot isset, der wird leben
in Ewigkeit. \bibverse{59} Solches sagte er in der Schule, da er lehrte
zu Kapernaum.

\bibverse{60} Viele nun seiner Jünger, die das hörten, sprachen: Das ist
eine harte Rede; wer kann sie hören?

\bibverse{61} Da Jesus aber bei sich selbst merkte, dass seine Jünger
darüber murrten, sprach er zu ihnen: Ärgert euch das? \bibverse{62} Wie,
wenn ihr denn sehen werdet des Menschen Sohn auffahren dahin, da er
zuvor war? \bibverse{63} Der Geist ist's, der da lebendig macht; das
Fleisch ist nichts nütze. Die Worte, die ich rede, die sind Geist und
sind Leben. \footnote{\textbf{6:63} 2Kor 3,6} \bibverse{64} Aber es sind
etliche unter euch, die glauben nicht. (Denn Jesus wusste von Anfang
wohl, welche nicht glaubend waren und welcher ihn verraten würde.)
\bibverse{65} Und er sprach: Darum habe ich euch gesagt: Niemand kann zu
mir kommen, es sei ihm denn von meinem Vater gegeben.

\bibverse{66} Von dem an gingen seiner Jünger viele hinter sich und
wandelten hinfort nicht mehr mit ihm. \bibverse{67} Da sprach Jesus zu
den Zwölfen: Wollt ihr auch weggehen?

\bibverse{68} Da antwortete ihm Simon Petrus: Herr, wohin sollen wir
gehen? Du hast Worte des ewigen Lebens; \bibverse{69} und wir haben
geglaubt und erkannt, dass du bist Christus, der Sohn des lebendigen
Gottes.

\bibverse{70} Jesus antwortete ihnen: Habe ich nicht euch zwölf erwählt?
und -- euer einer ist ein Teufel! \bibverse{71} Er redete aber von dem
Judas, Simons Sohn, Ischariot; der verriet ihn hernach, und war der
Zwölfe einer. \# 7 \bibverse{1} Darnach zog Jesus umher in Galiläa; denn
er wollte nicht in Judäa umherziehen, darum dass ihm die Juden nach dem
Leben stellten. \footnote{\textbf{7:1} Joh 4,43} \bibverse{2} Es war
aber nahe der Juden Fest, die Laubhütten. \footnote{\textbf{7:2} 3Mo
  23,34-36} \bibverse{3} Da sprachen seine Brüder zu ihm: Mache dich auf
von dannen und gehe nach Judäa, auf dass auch deine Jünger sehen, die
Werke die du tust. \footnote{\textbf{7:3} Joh 2,12; Mt 12,46; Apg 1,14}
\bibverse{4} Niemand tut etwas im Verborgenen und will doch frei
offenbar sein. Tust du solches, so offenbare dich vor der Welt.
\bibverse{5} Denn auch seine Brüder glaubten nicht an ihn.

\bibverse{6} Da spricht Jesus zu ihnen: Meine Zeit ist noch nicht hier;
eure Zeit aber ist allewege. \bibverse{7} Die Welt kann euch nicht
hassen; mich aber hasst sie, denn ich zeuge von ihr, dass ihre Werke
böse sind. \footnote{\textbf{7:7} Joh 15,18} \bibverse{8} Gehet ihr
hinauf auf dieses Fest; ich will noch nicht hinaufgehen auf dieses Fest,
denn meine Zeit ist noch nicht erfüllt.

\bibverse{9} Da er aber das zu ihnen gesagt, blieb er in Galiläa.
\bibverse{10} Als aber seine Brüder waren hinaufgegangen, da ging er
auch hinauf zu dem Fest, nicht offenbar, sondern wie heimlich.
\bibverse{11} Da suchten ihn die Juden am Fest und sprachen: Wo ist der?
\bibverse{12} Und es war ein großes Gemurmel von ihm unter dem Volk.
Etliche sprachen: Er ist fromm; die anderen aber sprachen: Nein, sondern
er verführt das Volk. \bibverse{13} Niemand aber redete frei von ihm um
der Furcht willen vor den Juden. \footnote{\textbf{7:13} Joh 9,22; Joh
  12,42; Joh 19,38} \bibverse{14} Aber mitten im Fest ging Jesus hinauf
in den Tempel und lehrte. \bibverse{15} Und die Juden verwunderten sich
und sprachen: Wie kann dieser die Schrift, obwohl er sie doch nicht
gelernt hat?

\bibverse{16} Jesus antwortete ihnen und sprach: Meine Lehre ist nicht
mein, sondern des, der mich gesandt hat. \bibverse{17} Wenn jemand will
des Willen tun, der wird innewerden, ob diese Lehre von Gott sei, oder
ob ich von mir selbst rede. \bibverse{18} Wer von sich selbst redet, der
sucht seine eigene Ehre; wer aber sucht die Ehre des, der ihn gesandt
hat, der ist wahrhaftig, und ist keine Ungerechtigkeit an ihm.
\footnote{\textbf{7:18} Joh 5,41; Joh 5,44} \bibverse{19} Hat euch nicht
Mose das Gesetz gegeben? und niemand unter euch tut das Gesetz. Warum
sucht ihr mich zu töten? \footnote{\textbf{7:19} Joh 5,16; Joh 5,18; Röm
  2,17-24}

\bibverse{20} Das Volk antwortete und sprach: Du hast den Teufel; wer
sucht dich zu töten? \footnote{\textbf{7:20} Joh 10,20}

\bibverse{21} Jesus antwortete und sprach: Ein einziges Werk habe ich
getan, und es wundert euch alle. \footnote{\textbf{7:21} Joh 5,16}
\bibverse{22} Mose hat euch darum gegeben die Beschneidung -- nicht dass
sie von Mose kommt, sondern von den Vätern --, und ihr beschneidet den
Menschen am Sabbat. \footnote{\textbf{7:22} 1Mo 17,10-12; 3Mo 12,3}
\bibverse{23} So ein Mensch die Beschneidung annimmt am Sabbat, auf dass
nicht das Gesetz Moses gebrochen werde, zürnet ihr denn über mich, dass
ich den ganzen Menschen habe am Sabbat gesund gemacht? \bibverse{24}
Richtet nicht nach dem Ansehen, sondern richtet ein rechtes Gericht.

\bibverse{25} Da sprachen etliche aus Jerusalem: Ist das nicht der, den
sie suchten zu töten? \bibverse{26} Und siehe zu, er redet frei, und sie
sagen ihm nichts. Erkennen unsere Obersten nun gewiss, dass er gewiss
Christus sei? \bibverse{27} Doch wir wissen, woher dieser ist; wenn aber
Christus kommen wird, so wird niemand wissen, woher er ist.

\bibverse{28} Da rief Jesus im Tempel, lehrte und sprach: Ja, ihr kennet
mich und wisset, woher ich bin; und von mir selbst bin ich nicht
gekommen, sondern es ist ein Wahrhaftiger, der mich gesandt hat, welchen
ihr nicht kennet. \bibverse{29} Ich kenne ihn aber; denn ich bin von
ihm, und er hat mich gesandt. \footnote{\textbf{7:29} Mt 11,27}

\bibverse{30} Da suchten sie ihn zu greifen; aber niemand legte die Hand
an ihn, denn seine Stunde war noch nicht gekommen. \footnote{\textbf{7:30}
  Joh 8,20; Lk 22,53} \bibverse{31} Aber viele vom Volk glaubten an ihn
und sprachen: Wenn Christus kommen wird, wird er auch mehr Zeichen tun,
denn dieser tut? \bibverse{32} Und es kam vor die Pharisäer, dass das
Volk solches von ihm murmelte. Da sandten die Pharisäer und
Hohenpriester Knechte aus, das sie ihn griffen.

\bibverse{33} Da sprach Jesus zu ihnen: Ich bin noch eine kleine Zeit
bei euch, und dann gehe ich hin zu dem, der mich gesandt hat.
\footnote{\textbf{7:33} Joh 13,33} \bibverse{34} Ihr werdet mich suchen,
und nicht finden; und wo ich bin, könnet ihr nicht hin kommen.
\footnote{\textbf{7:34} Joh 8,21}

\bibverse{35} Da sprachen die Juden untereinander: Wo will dieser hin
gehen, dass wir ihn nicht finden sollen? Will er zu den Zerstreuten
unter den Griechen gehen und die Griechen lehren? \bibverse{36} Was ist
das für eine Rede, dass er sagte: „Ihr werdet mich suchen, und nicht
finden; und wo ich bin, da könnet ihr nicht hin kommen``?

\bibverse{37} Aber am letzten Tage des Festes, der am herrlichsten war,
trat Jesus auf, rief und sprach: Wen da dürstet, der komme zu mir und
trinke! \footnote{\textbf{7:37} 3Mo 23,36; Joh 4,10; Jes 55,1; Offb
  22,17} \bibverse{38} Wer an mich glaubt, wie die Schrift sagt, von des
Leibe werden Ströme des lebendigen Wassers fließen. \footnote{\textbf{7:38}
  Jes 58,11} \bibverse{39} Das sagte er aber von dem Geist, welchen
empfangen sollten, die an ihn glaubten; denn der Heilige Geist war noch
nicht da, denn Jesus war noch nicht verklärt. \footnote{\textbf{7:39}
  Joh 16,7}

\bibverse{40} Viele nun vom Volk, die diese Rede hörten, sprachen:
Dieser ist wahrlich der Prophet. \footnote{\textbf{7:40} Joh 6,14}
\bibverse{41} Andere sprachen: Er ist Christus. Etliche aber sprachen:
Soll Christus aus Galiläa kommen? \footnote{\textbf{7:41} Joh 1,46}
\bibverse{42} Spricht nicht die Schrift: von dem Samen Davids und aus
dem Flecken Bethlehem, da David war, solle Christus kommen? \footnote{\textbf{7:42}
  Mi 5,1; Mt 2,5-6; Mt 22,42} \bibverse{43} Also ward eine Zwietracht
unter dem Volk über ihn. \footnote{\textbf{7:43} Joh 9,16} \bibverse{44}
Es wollten aber etliche ihn greifen; aber niemand legte die Hand an ihn.
\bibverse{45} Die Knechte kamen zu den Hohenpriestern und Pharisäern;
und sie sprachen zu ihnen: Warum habt ihr ihn nicht gebracht?

\bibverse{46} Die Knechte antworteten: Es hat nie ein Mensch also
geredet wie dieser Mensch.

\bibverse{47} Da antworteten ihnen die Pharisäer: Seid ihr auch
verführt? \bibverse{48} Glaubt auch irgendein Oberster oder Pharisäer an
ihn? \bibverse{49} sondern das Volk, das nichts vom Gesetz weiß, ist
verflucht.

\bibverse{50} Spricht zu ihnen Nikodemus, der bei der Nacht zu ihm kam,
welcher einer unter ihnen war: \footnote{\textbf{7:50} Joh 3,1-2}
\bibverse{51} Richtet unser Gesetz auch einen Menschen, ehe man ihn
verhört und erkennt, was er tut? \footnote{\textbf{7:51} 5Mo 1,16-17}

\bibverse{52} Sie antworteten und sprachen zu ihm: Bist du auch ein
Galiläer? Forsche und siehe, aus Galiläa steht kein Prophet auf.

\bibverse{53} Und ein jeglicher ging also heim. \# 8 \bibverse{1} Jesus
aber ging an den Ölberg.

\bibverse{2} Und frühmorgens kam er wieder in den Tempel, und alles Volk
kam zu ihm; und er setzte sich und lehrte sie. \bibverse{3} Aber die
Schriftgelehrten und Pharisäer brachten ein Weib zu ihm, im Ehebruch
ergriffen, und stellten sie in die Mitte dar \bibverse{4} und sprachen
zu ihm: Meister, dies Weib ist ergriffen auf frischer Tat im Ehebruch.
\bibverse{5} Mose aber hat uns im Gesetz geboten, solche zu steinigen;
was sagst du? \footnote{\textbf{8:5} 3Mo 20,10} \bibverse{6} Das
sprachen sie aber, ihn zu versuchen, auf dass sie eine Sache wider ihn
hätten. Aber Jesus bückte sich nieder und schrieb mit dem Finger auf die
Erde.

\bibverse{7} Als sie nun anhielten, ihn zu fragen, richtete er sich auf
und sprach zu ihnen: Wer unter euch ohne Sünde ist, der werfe den ersten
Stein auf sie. \bibverse{8} Und bückte sich wieder nieder und schrieb
auf die Erde.

\bibverse{9} Da sie aber das hörten, gingen sie hinaus (von ihrem
Gewissen überführt), einer nach dem anderen, von den Ältesten an bis zu
den Geringsten; und Jesus ward gelassen allein und das Weib in der Mitte
stehend. \bibverse{10} Jesus aber richtete sich auf; und da er niemand
sah denn das Weib, sprach er zu ihr: Weib, wo sind sie, deine Verkläger?
Hat dich niemand verdammt?

\bibverse{11} Sie aber sprach: Herr, niemand. Jesus aber sprach: So
verdamme ich dich auch nicht; gehe hin und sündige hinfort nicht mehr!
\footnote{\textbf{8:11} Joh 5,14}

\bibverse{12} Da redete Jesus abermals zu ihnen und sprach: Ich bin das
Licht der Welt; wer mir nachfolgt, der wird nicht wandeln in der
Finsternis, sondern wird das Licht des Lebens haben. \footnote{\textbf{8:12}
  Jes 49,6; Joh 1,5; Joh 1,9; Mt 5,14-16}

\bibverse{13} Da sprachen die Pharisäer zu ihm: Du zeugst von dir
selbst; dein Zeugnis ist nicht wahr.

\bibverse{14} Jesus antwortete und sprach zu ihnen: So ich von mir
selbst zeugen würde, so ist mein Zeugnis wahr; denn ich weiß, woher ich
gekommen bin und wo ich hin gehe; ihr aber wisset nicht, woher ich komme
und wo ich hin gehe. \footnote{\textbf{8:14} Joh 5,31; Joh 7,28}

\bibverse{15} Ihr richtet nach dem Fleisch; ich richte niemand.
\footnote{\textbf{8:15} Joh 3,17} \bibverse{16} So ich aber richte, so
ist mein Gericht recht; denn ich bin nicht allein, sondern ich und der
Vater, der mich gesandt hat. \bibverse{17} Auch steht in eurem Gesetz
geschrieben, dass zweier Menschen Zeugnis wahr sei. \footnote{\textbf{8:17}
  5Mo 19,15} \bibverse{18} Ich bin's, der ich von mir selbst zeuge; und
der Vater, der mich gesandt hat, zeugt auch von mir.

\bibverse{19} Da sprachen sie zu ihm: Wo ist dein Vater? Jesus
antwortete: Ihr kennet weder mich noch meinen Vater; wenn ihr mich
kenntet, so kenntet ihr auch meinen Vater.

\bibverse{20} Diese Worte redete Jesus an dem Gotteskasten, da er lehrte
im Tempel; und niemand griff ihn, denn seine Stunde war noch nicht
gekommen. \footnote{\textbf{8:20} Joh 7,30} \bibverse{21} Da sprach
Jesus abermals zu ihnen: Ich gehe hinweg, und ihr werdet mich suchen und
in eurer Sünde sterben. Wo ich hin gehe, da könnet ihr nicht hin kommen.
\footnote{\textbf{8:21} Joh 7,34-35; Joh 13,33}

\bibverse{22} Da sprachen die Juden: Will er sich denn selbst töten,
dass er spricht: „Wohin ich gehe, da könnet ihr nicht hin kommen``?

\bibverse{23} Und er sprach zu ihnen: Ihr seid von untenher, ich bin von
obenher; ihr seid von dieser Welt, ich bin nicht von dieser Welt.
\footnote{\textbf{8:23} Joh 3,31} \bibverse{24} So habe ich euch gesagt,
dass ihr sterben werdet in euren Sünden; denn wenn ihr nicht glaubet,
dass ich es sei, so werdet ihr sterben in euren Sünden.

\bibverse{25} Da sprachen sie zu ihm: Wer bist du denn? Und Jesus sprach
zu ihnen: Erstlich der, der ich mit euch rede.

\bibverse{26} Ich habe viel von euch zu reden und zu richten; aber der
mich gesandt hat, ist wahrhaftig, und was ich von ihm gehört habe, das
rede ich vor der Welt.

\bibverse{27} Sie verstanden aber nicht, dass er ihnen von dem Vater
sagte. \bibverse{28} Da sprach Jesus zu ihnen: Wenn ihr des Menschen
Sohn erhöhen werdet, dann werdet ihr erkennen, dass ich es sei und
nichts von mir selber tue, sondern, wie mich mein Vater gelehrt hat, so
rede ich. \bibverse{29} Und der mich gesandt hat, ist mit mir. Der Vater
lässt mich nicht allein; denn ich tue allezeit, was ihm gefällt.

\bibverse{30} Da er solches redete, glaubten viele an ihn. \bibverse{31}
Da sprach nun Jesus zu den Juden, die an ihn glaubten: So ihr bleiben
werdet an meiner Rede, so seid ihr meine rechten Jünger \footnote{\textbf{8:31}
  Joh 15,7} \bibverse{32} und werdet die Wahrheit erkennen, und die
Wahrheit wird euch frei machen.

\bibverse{33} Da antworteten sie ihm: Wir sind Abrahams Samen, sind
niemals jemandes Knechte gewesen; wie sprichst du denn: „Ihr sollt frei
werden``?

\bibverse{34} Jesus antwortete ihnen und sprach: Wahrlich, wahrlich ich
sage euch: Wer Sünde tut, der ist der Sünde Knecht. \bibverse{35} Der
Knecht aber bleibt nicht ewiglich im Hause; der Sohn bleibt ewiglich.
\bibverse{36} So euch nun der Sohn frei macht, so seid ihr recht frei.
\footnote{\textbf{8:36} Röm 6,16; Röm 6,18; Röm 6,22} \bibverse{37} Ich
weiß wohl, dass ihr Abrahams Samen seid; aber ihr sucht mich zu töten,
denn meine Rede fängt nicht bei euch. \bibverse{38} Ich rede, was ich
von meinem Vater gesehen habe; so tut ihr, was ihr von eurem Vater
gesehen habt.

\bibverse{39} Sie antworteten und sprachen zu ihm: Abraham ist unser
Vater. Spricht Jesus zu ihnen: Wenn ihr Abrahams Kinder wäret, so tätet
ihr Abrahams Werke.

\bibverse{40} Nun aber sucht ihr mich zu töten, einen solchen Menschen,
der ich euch die Wahrheit gesagt habe, die ich von Gott gehört habe. Das
hat Abraham nicht getan. \bibverse{41} Ihr tut eures Vaters Werke. Da
sprachen sie zu ihm: Wir sind nicht unehelich geboren; wir haben einen
Vater, Gott.

\bibverse{42} Jesus sprach zu ihnen: Wäre Gott euer Vater, so liebtet
ihr mich; denn ich bin ausgegangen und komme von Gott; denn ich bin
nicht von mir selber gekommen, sondern er hat mich gesandt.

\bibverse{43} Warum kennet ihr denn meine Sprache nicht? Denn ihr könnt
ja mein Wort nicht hören. \bibverse{44} Ihr seid von dem Vater, dem
Teufel, und nach eures Vaters Lust wollt ihr tun. Der ist ein Mörder von
Anfang und ist nicht bestanden in der Wahrheit; denn die Wahrheit ist
nicht in ihm. Wenn er die Lüge redet, so redet er von seinem Eigenen;
denn er ist ein Lügner und ein Vater derselben. \footnote{\textbf{8:44}
  1Jo 3,8-10; 1Mo 3,4; 1Mo 3,19} \bibverse{45} Ich aber, weil ich die
Wahrheit sage, so glaubet ihr mir nicht. \bibverse{46} Welcher unter
euch kann mich einer Sünde zeihen? Wenn ich euch aber die Wahrheit sage,
warum glaubet ihr mir nicht? \bibverse{47} Wer von Gott ist, der hört
Gottes Worte; darum hört ihr nicht, denn ihr seid nicht von Gott.
\footnote{\textbf{8:47} Joh 18,37}

\bibverse{48} Da antworteten die Juden und sprachen zu ihm: Sagen wir
nicht recht, dass du ein Samariter bist und hast den Teufel? \footnote{\textbf{8:48}
  Joh 7,20}

\bibverse{49} Jesus antwortete: Ich habe keinen Teufel, sondern ich ehre
meinen Vater, und ihr unehret mich. \bibverse{50} Ich suche nicht meine
Ehre; es ist aber einer, der sie sucht, und richtet. \bibverse{51}
Wahrlich, wahrlich ich sage euch: Wenn jemand mein Wort wird halten, der
wird den Tod nicht sehen ewiglich. \footnote{\textbf{8:51} Joh 6,40; Joh
  6,47}

\bibverse{52} Da sprachen die Juden zu ihm: Nun erkennen wir, dass du
den Teufel hast. Abraham ist gestorben und die Propheten, und du
sprichst: „Wenn jemand mein Wort hält, der wird den Tod nicht schmecken
ewiglich.`` \bibverse{53} Bist du denn mehr als unser Vater Abraham,
welcher gestorben ist? Und die Propheten sind gestorben. Was machst du
aus dir selbst?

\bibverse{54} Jesus antwortete: Wenn ich mich selber ehre, so ist meine
Ehre nichts. Es ist aber mein Vater, der mich ehrt, von welchem ihr
sprecht, er sei euer Gott; \bibverse{55} und kennet ihn nicht, ich aber
kenne ihn. Und wenn ich würde sagen: Ich kenne ihn nicht, so würde ich
ein Lügner, gleichwie ihr seid. Aber ich kenne ihn und halte sein Wort.
\bibverse{56} Abraham, euer Vater, ward froh, dass er meinen Tag sehen
sollte; und er sah ihn und freute sich.

\bibverse{57} Da sprachen die Juden zu ihm: Du bist noch nicht fünfzig
Jahre alt und hast Abraham gesehen?

\bibverse{58} Jesus sprach zu ihnen: Wahrlich, wahrlich ich sage euch:
Ehe denn Abraham ward, bin ich. \footnote{\textbf{8:58} Joh 1,1-2}

\bibverse{59} Da hoben sie Steine auf, dass sie auf ihn würfen. Aber
Jesus verbarg sich und ging zum Tempel hinaus. \footnote{\textbf{8:59}
  Joh 10,31}

\hypertarget{section-4}{%
\section{9}\label{section-4}}

\bibverse{1} Und Jesus ging vorüber und sah einen, der blind geboren
war. \bibverse{2} Und seine Jünger fragten ihn und sprachen: Meister,
wer hat gesündigt, dieser oder seine Eltern, dass er ist blind geboren?
\footnote{\textbf{9:2} Lk 13,2}

\bibverse{3} Jesus antwortete: Es hat weder dieser gesündigt noch seine
Eltern, sondern dass die Werke Gottes offenbar würden an ihm.
\footnote{\textbf{9:3} Joh 11,4} \bibverse{4} Ich muss wirken die Werke
des, der mich gesandt hat, solange es Tag ist; es kommt die Nacht, da
niemand wirken kann. \footnote{\textbf{9:4} Joh 5,17; Jer 13,16}
\bibverse{5} Dieweil ich bin in der Welt, bin ich das Licht der Welt.
\footnote{\textbf{9:5} Joh 12,35; Joh 8,12} \bibverse{6} Da er solches
gesagt, spützte er auf die Erde und machte einen Kot aus dem Speichel
und schmierte den Kot auf des Blinden Augen \footnote{\textbf{9:6} Mk
  8,23} \bibverse{7} und sprach zu ihm: Gehe hin zu dem Teich Siloah
(das ist verdolmetscht: gesandt) und wasche dich! Da ging er hin und
wusch sich und kam sehend.

\bibverse{8} Die Nachbarn und die ihn zuvor gesehen hatten, dass er ein
Bettler war, sprachen: Ist dieser nicht, der dasaß und bettelte?
\bibverse{9} Etliche sprachen: Er ist's, etliche aber: Er ist ihm
ähnlich. Er selbst aber sprach: Ich bin's.

\bibverse{10} Da sprachen sie zu ihm: Wie sind deine Augen aufgetan
worden?

\bibverse{11} Er antwortete und sprach: Der Mensch, der Jesus heißt,
machte einen Kot und schmierte meine Augen und sprach: „Gehe hin zu dem
Teich Siloah und wasche dich!{}`` Ich ging hin und wusch mich und ward
sehend.

\bibverse{12} Da sprachen sie zu ihm: Wo ist er? Er sprach: Ich weiß
nicht.

\bibverse{13} Da führten sie ihn zu den Pharisäern, der vordem blind
war.

\bibverse{14} (Es war aber Sabbat, da Jesus den Kot machte und seine
Augen öffnete.)

\bibverse{15} Da fragten ihn abermals auch die Pharisäer, wie er wäre
sehend geworden. Er aber sprach zu ihnen: Kot legte er mir auf die
Augen, und ich wusch mich und bin nun sehend.

\bibverse{16} Da sprachen etliche der Pharisäer: Der Mensch ist nicht
von Gott, dieweil er den Sabbat nicht hält. Die anderen aber sprachen:
Wie kann ein sündiger Mensch solche Zeichen tun? Und es ward eine
Zwietracht unter ihnen.

\bibverse{17} Sie sprachen wieder zu dem Blinden: Was sagst du von ihm,
dass er hat deine Augen aufgetan? Er aber sprach: Er ist ein Prophet.

\bibverse{18} Die Juden glaubten nicht von ihm, dass er blind gewesen
und sehend geworden wäre, bis dass sie riefen die Eltern des, der sehend
war geworden,

\bibverse{19} fragten sie und sprachen: Ist das euer Sohn, von welchem
ihr sagt, er sei blind geboren? Wie ist er denn nun sehend?

\bibverse{20} Seine Eltern antworteten ihnen und sprachen: Wir wissen,
dass dieser unser Sohn ist und dass er blind geboren ist;

\bibverse{21} wie er aber nun sehend ist, wissen wir nicht; oder wer ihm
hat seine Augen aufgetan, wissen wir auch nicht. Er ist alt genug,
fraget ihn, lasst ihn selbst für sich reden. \bibverse{22} Solches
sagten seine Eltern; denn sie fürchteten sich vor den Juden. Denn die
Juden hatten sich schon vereinigt, wenn jemand ihn für Christus
bekennte, dass er in den Bann getan würde. \bibverse{23} Darum sprachen
seine Eltern: Er ist alt genug, fraget ihn selbst.

\bibverse{24} Da riefen sie zum andernmal den Menschen, der blind
gewesen war, und sprachen zu ihm: Gib Gott die Ehre! wir wissen, dass
dieser Mensch ein Sünder ist.

\bibverse{25} Er antwortete und sprach: Ist er ein Sünder, das weiß ich
nicht; eines weiß ich wohl, dass ich blind war und bin nun sehend.

\bibverse{26} Da sprachen sie wieder zu ihm: Was tat er dir? Wie tat er
deine Augen auf?

\bibverse{27} Er antwortete ihnen: Ich habe es euch jetzt gesagt; habt
ihr's nicht gehört? Was wollt ihr's abermals hören? Wollt ihr auch seine
Jünger werden?

\bibverse{28} Da schalten sie ihn und sprachen: Du bist sein Jünger; wir
aber sind Moses Jünger. \bibverse{29} Wir wissen, dass Gott mit Mose
geredet hat; woher aber dieser ist, wissen wir nicht.

\bibverse{30} Der Mensch antwortete und sprach zu ihnen: Das ist ein
wunderlich Ding, dass ihr nicht wisset, woher er sei, und er hat meine
Augen aufgetan. \bibverse{31} Wir wissen aber, dass Gott die Sünder
nicht hört; sondern wenn jemand gottesfürchtig ist und tut seinen
Willen, den hört er. \footnote{\textbf{9:31} Ps 66,18; Spr 15,29; Jes
  1,15} \bibverse{32} Von der Welt an ist's nicht erhört, dass jemand
einem geborenen Blinden die Augen aufgetan habe. \bibverse{33} Wäre
dieser nicht von Gott, er könnte nichts tun.

\bibverse{34} Sie antworteten und sprachen zu ihm: Du bist ganz in
Sünden geboren, und lehrst uns? Und stießen ihn hinaus.

\bibverse{35} Es kam vor Jesum, dass sie ihn ausgestoßen hatten. Und da
er ihn fand, sprach er zu ihm: Glaubst du an den Sohn Gottes?

\bibverse{36} Er antwortete und sprach: Herr, welcher ist's? auf dass
ich an ihn glaube.

\bibverse{37} Jesus sprach zu ihm: Du hast ihn gesehen, und der mit dir
redet, der ist's.

\bibverse{38} Er aber sprach: Herr, ich glaube, und betete ihn an.

\bibverse{39} Und Jesus sprach: Ich bin zum Gericht auf diese Welt
gekommen, auf dass, die da nicht sehen, sehend werden, und die da sehen,
blind werden. \footnote{\textbf{9:39} Mt 13,11-15}

\bibverse{40} Und solches hörten etliche der Pharisäer, die bei ihm
waren, und sprachen zu ihm: Sind wir denn auch blind?

\bibverse{41} Jesus sprach zu ihnen: Wäret ihr blind, so hättet ihr
keine Sünde; nun ihr aber sprecht: „Wir sind sehend``, bleibt eure
Sünde. \# 10 \bibverse{1} Wahrlich, wahrlich ich sage euch: Wer nicht
zur Tür hineingeht in den Schafstall, sondern steigt anderswo hinein,
der ist ein Dieb und ein Mörder. \bibverse{2} Der aber zur Tür
hineingeht, der ist ein Hirte der Schafe. \bibverse{3} Dem tut der
Türhüter auf, und die Schafe hören seine Stimme; und er ruft seine
Schafe mit Namen und führt sie aus. \bibverse{4} Und wenn er seine
Schafe hat ausgelassen, geht er vor ihnen hin, und die Schafe folgen ihm
nach; denn sie kennen seine Stimme. \bibverse{5} Einem Fremden aber
folgen sie nicht nach, sondern fliehen von ihm; denn sie kennen der
Fremden Stimme nicht. \bibverse{6} Diesen Spruch sagte Jesus zu ihnen;
sie verstanden aber nicht, was es war, das er zu ihnen sagte.

\bibverse{7} Da sprach Jesus wieder zu ihnen: Wahrlich, wahrlich ich
sage euch: Ich bin die Tür zu den Schafen. \bibverse{8} Alle, die vor
mir gekommen sind, die sind Diebe und Mörder; aber die Schafe haben
ihnen nicht gehorcht. \bibverse{9} Ich bin die Tür; wenn jemand durch
mich eingeht, der wird selig werden und wird ein und aus gehen und Weide
finden. \footnote{\textbf{10:9} Joh 14,6} \bibverse{10} Ein Dieb kommt
nur, dass er stehle, würge und umbringe. Ich bin gekommen, dass sie das
Leben und volle Genüge haben sollen.

\bibverse{11} Ich bin der gute Hirte. Der gute Hirte lässt sein Leben
für die Schafe. \bibverse{12} Der Mietling aber, der nicht Hirte ist,
des die Schafe nicht eigen sind, sieht den Wolf kommen und verlässt die
Schafe und flieht; und der Wolf erhascht und zerstreut die Schafe.
\bibverse{13} Der Mietling aber flieht; denn er ist ein Mietling und
achtet der Schafe nicht. \bibverse{14} Ich bin der gute Hirte und
erkenne die Meinen und bin bekannt den Meinen, \footnote{\textbf{10:14}
  2Tim 2,19} \bibverse{15} wie mich mein Vater kennt und ich kenne den
Vater. Und ich lasse mein Leben für die Schafe. \bibverse{16} Und ich
habe noch andere Schafe, die sind nicht aus diesem Stalle; und dieselben
muss ich herführen, und sie werden meine Stimme hören, und wird eine
Herde und ein Hirte werden. \bibverse{17} Darum liebt mich mein Vater,
dass ich mein Leben lasse, auf dass ich's wiedernehme. \bibverse{18}
Niemand nimmt es von mir, sondern ich lasse es von mir selber. Ich habe
Macht, es zu lassen, und habe Macht, es wiederzunehmen. Solch Gebot habe
ich empfangen von meinem Vater. \footnote{\textbf{10:18} Joh 5,26}

\bibverse{19} Da ward abermals eine Zwietracht unter den Juden über
diese Worte. \footnote{\textbf{10:19} Joh 7,43; Joh 9,16} \bibverse{20}
Viele unter ihnen sprachen: Er hat den Teufel und ist unsinnig; was
höret ihr ihm zu? \footnote{\textbf{10:20} Joh 7,20; Mk 3,21}
\bibverse{21} Die anderen sprachen: Das sind nicht Worte eines
Besessenen; kann der Teufel auch der Blinden Augen auftun?

\bibverse{22} Es ward aber Kirchweihe zu Jerusalem und war Winter.
\bibverse{23} Und Jesus wandelte im Tempel in der Halle Salomos.
\bibverse{24} Da umringten ihn die Juden und sprachen zu ihm: Wie lange
hältst du unsere Seele auf? Bist du Christus, so sage es uns frei
heraus.

\bibverse{25} Jesus antwortete ihnen: Ich habe es euch gesagt, und ihr
glaubet nicht. Die Werke, die ich tue in meines Vaters Namen, die zeugen
von mir. \footnote{\textbf{10:25} Joh 5,36} \bibverse{26} Aber ihr
glaubet nicht; denn ihr seid von meinen Schafen nicht, wie ich euch
gesagt habe. \footnote{\textbf{10:26} Joh 8,45; Joh 8,47} \bibverse{27}
Denn meine Schafe hören meine Stimme, und ich kenne sie; und sie folgen
mir, \bibverse{28} und ich gebe ihnen das ewige Leben; und sie werden
nimmermehr umkommen, und niemand wird sie mir aus meiner Hand reißen.
\bibverse{29} Der Vater, der mir sie gegeben hat, ist größer denn alles;
und niemand kann sie aus meines Vaters Hand reißen. \bibverse{30} Ich
und der Vater sind eins.

\bibverse{31} Da hoben die Juden abermals Steine auf, dass sie ihn
steinigten. \footnote{\textbf{10:31} Joh 8,59} \bibverse{32} Jesus
antwortete ihnen: Viel gute Werke habe ich euch erzeigt von meinem
Vater; um welches Werk unter ihnen steiniget ihr mich?

\bibverse{33} Die Juden antworteten ihm und sprachen: Um des guten Werks
willen steinigen wir dich nicht, sondern um der Gotteslästerung willen
und dass du ein Mensch bist und machst dich selbst zu Gott.

\bibverse{34} Jesus antwortete ihnen: Steht nicht geschrieben in eurem
Gesetz: „Ich habe gesagt: Ihr seid Götter``? \bibverse{35} So er die
Götter nennt, zu welchen das Wort geschah -- und die Schrift kann doch
nicht gebrochen werden --, \bibverse{36} sprecht ihr denn zu dem, den
der Vater geheiligt und in die Welt gesandt hat: „Du lästerst Gott``,
darum dass ich sage: Ich bin Gottes Sohn? \footnote{\textbf{10:36} Joh
  5,17-20} \bibverse{37} Tue ich nicht die Werke meines Vaters, so
glaubet mir nicht; \bibverse{38} tue ich sie aber, glaubet doch den
Werken, wollt ihr mir nicht glauben, auf dass ihr erkennet und glaubet,
dass der Vater in mir ist und ich in ihm.

\bibverse{39} Sie suchten abermals ihn zu greifen; aber er entging ihnen
aus ihren Händen \bibverse{40} und zog hin wieder jenseits des Jordans
an den Ort, da Johannes zuvor getauft hatte, und blieb allda.
\footnote{\textbf{10:40} Joh 1,28} \bibverse{41} Und viele kamen zu ihm
und sprachen: Johannes tat kein Zeichen; aber alles, was Johannes von
diesem gesagt hat, das ist wahr. \bibverse{42} Und glaubten allda viele
an ihn. \# 11 \bibverse{1} Es lag aber einer krank mit Namen Lazarus,
von Bethanien, in dem Flecken Marias und ihrer Schwester Martha.
\bibverse{2} (Maria aber war, die den Herrn gesalbt hat mit Salbe und
seine Füße getrocknet mit ihrem Haar; deren Bruder, Lazarus, war krank.)
\footnote{\textbf{11:2} Joh 12,3} \bibverse{3} Da sandten seine
Schwestern zu ihm und ließen ihm sagen: Herr, siehe, den du liebhast,
der liegt krank.

\bibverse{4} Da Jesus das hörte, sprach er: Die Krankheit ist nicht zum
Tode, sondern zur Ehre Gottes, dass der Sohn Gottes dadurch geehrt
werde. \bibverse{5} Jesus aber hatte Martha lieb und ihre Schwester und
Lazarus. \bibverse{6} Als er nun hörte, dass er krank war, blieb er zwei
Tage an dem Ort, da er war. \bibverse{7} Darnach spricht er zu seinen
Jüngern: Lasst uns wieder nach Judäa ziehen!

\bibverse{8} Seine Jünger sprachen zu ihm: Meister, jenes Mal wollten
die Juden dich steinigen, und du willst wieder dahin ziehen? \footnote{\textbf{11:8}
  Joh 10,31}

\bibverse{9} Jesus antwortete: Sind nicht des Tages zwölf Stunden? Wer
des Tages wandelt, der stößt sich nicht; denn er sieht das Licht dieser
Welt. \footnote{\textbf{11:9} Joh 9,4-5} \bibverse{10} Wer aber des
Nachts wandelt, der stößt sich; denn es ist kein Licht in ihm.
\footnote{\textbf{11:10} Joh 12,35} \bibverse{11} Solches sagte er, und
darnach spricht er zu ihnen: Lazarus, unser Freund, schläft; aber ich
gehe hin, dass ich ihn aufwecke. \footnote{\textbf{11:11} Mt 9,24}

\bibverse{12} Da sprachen seine Jünger: Herr, schläft er, so wird's
besser mit ihm.

\bibverse{13} Jesus aber sagte von seinem Tode; sie meinten aber, er
redete vom leiblichen Schlaf. \bibverse{14} Da sagte es ihnen Jesus frei
heraus: Lazarus ist gestorben; \bibverse{15} und ich bin froh um
euretwillen, dass ich nicht dagewesen bin, auf dass ihr glaubet. Aber
lasset uns zu ihm ziehen!

\bibverse{16} Da sprach Thomas, der genannt ist Zwilling, zu den
Jüngern: Lasst uns mitziehen, dass wir mit ihm sterben! \footnote{\textbf{11:16}
  Joh 20,24-28}

\bibverse{17} Da kam Jesus und fand ihn, dass er schon vier Tage im
Grabe gelegen hatte. \bibverse{18} Bethanien aber war nahe bei
Jerusalem, bei fünfzehn Feld Weges; \bibverse{19} und viele Juden waren
zu Martha und Maria gekommen, sie zu trösten über ihren Bruder.
\bibverse{20} Als Martha nun hörte, dass Jesus kommt, geht sie ihm
entgegen; Maria aber blieb daheim sitzen. \bibverse{21} Da sprach Martha
zu Jesus: Herr, wärest du hier gewesen, mein Bruder wäre nicht
gestorben! \bibverse{22} Aber ich weiß auch noch, dass, was du bittest
von Gott, das wird dir Gott geben.

\bibverse{23} Jesus spricht zu ihr: Dein Bruder soll auferstehen.

\bibverse{24} Martha spricht zu ihm: Ich weiß wohl, dass er auferstehen
wird in der Auferstehung am Jüngsten Tage.

\bibverse{25} Jesus spricht zu ihr: Ich bin die Auferstehung und das
Leben. Wer an mich glaubet, der wird leben, ob er gleich stürbe;
\bibverse{26} und wer da lebet und glaubet an mich, der wird nimmermehr
sterben. Glaubst du das? \footnote{\textbf{11:26} Joh 8,51}

\bibverse{27} Sie spricht zu ihm: Herr, ja, ich glaube, dass du bist
Christus, der Sohn Gottes, der in die Welt gekommen ist. \footnote{\textbf{11:27}
  Mt 16,16}

\bibverse{28} Und da sie das gesagt hatte, ging sie hin und rief ihre
Schwester Maria heimlich und sprach: Der Meister ist da und ruft dich.

\bibverse{29} Dieselbe, als sie das hörte, stand sie eilend auf und kam
zu ihm. \bibverse{30} (Denn Jesus war noch nicht in den Flecken
gekommen, sondern war noch an dem Ort, da ihm Martha war
entgegengekommen.) \bibverse{31} Die Juden, die bei ihr im Haus waren
und sie trösteten, da sie sahen Maria, dass sie eilend aufstand und
hinausging, folgten sie ihr nach und sprachen: Sie geht hin zum Grabe,
dass sie daselbst weine.

\bibverse{32} Als nun Maria kam, da Jesus war, und sah ihn, fiel sie zu
seinen Füßen und sprach zu ihm: Herr, wärest du hier gewesen, mein
Bruder wäre nicht gestorben!

\bibverse{33} Als Jesus sie sah weinen und die Juden auch weinen, die
mit ihr kamen, ergrimmte er im Geist und betrübte sich selbst
\footnote{\textbf{11:33} Joh 13,21} \bibverse{34} und sprach: Wo habt
ihr ihn hingelegt? Sie sprachen zu ihm: Herr, komm und sieh es!

\bibverse{35} Und Jesu gingen die Augen über.

\bibverse{36} Da sprachen die Juden: Siehe, wie hat er ihn so
liebgehabt!

\bibverse{37} Etliche aber unter ihnen sprachen: Konnte, der dem Blinden
die Augen aufgetan hat, nicht verschaffen, dass auch dieser nicht
stürbe?

\bibverse{38} Da ergrimmte Jesus abermals in sich selbst und kam zum
Grabe. Es war aber eine Kluft, und ein Stein daraufgelegt. \footnote{\textbf{11:38}
  Mt 27,60} \bibverse{39} Jesus sprach: Hebt den Stein ab! Spricht zu
ihm Martha, die Schwester des Verstorbenen: Herr, er stinkt schon; denn
er ist vier Tage gelegen.

\bibverse{40} Jesus spricht zu ihr: Habe ich dir nicht gesagt, wenn du
glauben würdest, du solltest die Herrlichkeit Gottes sehen?

\bibverse{41} Da hoben sie den Stein ab, da der Verstorbene lag. Jesus
aber hob seine Augen empor und sprach: Vater, ich danke dir, dass du
mich erhört hast.

\bibverse{42} Doch ich weiß, dass du mich allezeit hörst; aber um des
Volks willen, das umhersteht, sage ich's, dass sie glauben, du habest
mich gesandt. \bibverse{43} Da er das gesagt hatte, rief er mit lauter
Stimme: Lazarus, komm heraus!

\bibverse{44} Und der Verstorbene kam heraus, gebunden mit Grabtüchern
an Füßen und Händen und sein Angesicht verhüllt mit einem Schweißtuch.
Jesus spricht zu ihnen: Löset ihn auf und lasset ihn gehen!

\bibverse{45} Viele nun der Juden, die zu Maria gekommen waren und
sahen, was Jesus tat, glaubten an ihn.

\bibverse{46} Etliche aber von ihnen gingen hin zu den Pharisäern und
sagten ihnen, was Jesus getan hatte. \bibverse{47} Da versammelten die
Hohenpriester und die Pharisäer einen Rat und sprachen: Was tun wir?
Dieser Mensch tut viele Zeichen. \footnote{\textbf{11:47} Mt 26,3-4}
\bibverse{48} Lassen wir ihn also, so werden sie alle an ihn glauben; so
kommen dann die Römer und nehmen uns Land und Leute.

\bibverse{49} Einer aber unter ihnen, Kaiphas, der desselben Jahres
Hoherpriester war, sprach zu ihnen: Ihr wisset nichts, \bibverse{50}
bedenket auch nichts; es ist uns besser, ein Mensch sterbe für das Volk,
denn dass das ganze Volk verderbe. \bibverse{51} (Solches aber redete er
nicht von sich selbst, sondern weil er desselben Jahres Hoherpriester
war, weissagte er. Denn Jesus sollte sterben für das Volk; \footnote{\textbf{11:51}
  2Mo 28,30; 4Mo 27,21} \bibverse{52} und nicht für das Volk allein,
sondern dass er auch die Kinder Gottes, die zerstreut waren,
zusammenbrächte.) \footnote{\textbf{11:52} Joh 7,35; Joh 10,16; 1Jo 2,2}
\bibverse{53} Von dem Tage an ratschlagten sie, wie sie ihn töteten.
\bibverse{54} Jesus aber wandelte nicht mehr frei unter den Juden,
sondern ging von dannen in eine Gegend nahe bei der Wüste, in eine
Stadt, genannt Ephrem, und hatte sein Wesen daselbst mit seinen Jüngern.

\bibverse{55} Es war aber nahe das Ostern der Juden; und es gingen viele
aus der Gegend hinauf gen Jerusalem vor Ostern, dass sie sich reinigten.
\footnote{\textbf{11:55} 2Chr 30,17-18} \bibverse{56} Da standen sie und
fragten nach Jesus und redeten miteinander im Tempel: Was dünkt euch,
dass er nicht kommt auf das Fest? \bibverse{57} Es hatten aber die
Hohenpriester und Pharisäer lassen ein Gebot ausgehen, wenn jemand
wüsste, wo er wäre, dass er's anzeige, dass sie ihn griffen. \# 12
\bibverse{1} Sechs Tage vor Ostern kam Jesus gen Bethanien, da Lazarus
war, der Verstorbene, welchen Jesus auferweckt hatte von den Toten.
\bibverse{2} Daselbst machten sie ihm ein Abendmahl, und Martha diente;
Lazarus aber war deren einer, die mit ihm zu Tische saßen. \bibverse{3}
Da nahm Maria ein Pfund Salbe von ungefälschter, köstlicher Narde und
salbte die Füße Jesu und trocknete mit ihrem Haar seine Füße; das Haus
aber ward voll vom Geruch der Salbe. \footnote{\textbf{12:3} Lk 7,38}

\bibverse{4} Da sprach seiner Jünger einer, Judas, Simons Sohn,
Ischariot, der ihn hernach verriet: \bibverse{5} Warum ist diese Salbe
nicht verkauft um dreihundert Groschen und den Armen gegeben?
\bibverse{6} Das sagte er aber nicht, dass er nach den Armen fragte;
sondern er war ein Dieb und hatte den Beutel und trug, was gegeben ward.

\bibverse{7} Da sprach Jesus: Lass sie mit Frieden! Solches hat sie
behalten zum Tage meines Begräbnisses. \bibverse{8} Denn Arme habt ihr
allezeit bei euch; mich aber habt ihr nicht allezeit. \footnote{\textbf{12:8}
  5Mo 15,11}

\bibverse{9} Da erfuhr viel Volks der Juden, dass er daselbst war; und
sie kamen nicht um Jesu willen allein, sondern dass sie auch Lazarus
sähen, welchen er von den Toten auferweckt hatte. \bibverse{10} Aber die
Hohenpriester trachteten darnach, dass sie auch Lazarus töteten;
\bibverse{11} denn um seinetwillen gingen viele Juden hin und glaubten
an Jesum.

\bibverse{12} Des anderen Tages, da viel Volks, das aufs Fest gekommen
war, hörte, dass Jesus käme gen Jerusalem, \bibverse{13} nahmen sie
Palmenzweige und gingen hinaus ihm entgegen und schrien: Hosianna!
Gelobt sei, der da kommt in dem Namen des Herrn, der König von Israel!

\bibverse{14} Jesus aber fand ein Eselein und ritt darauf; wie denn
geschrieben steht: \bibverse{15} „Fürchte dich nicht du Tochter Zion!
Siehe, dein König kommt, reitend auf einem Eselsfüllen.`` \bibverse{16}
Solches aber verstanden seine Jünger zuvor nicht; sondern da Jesus
verklärt ward, da dachten sie daran, dass solches von ihm geschrieben
war und sie solches ihm getan hatten. \bibverse{17} Das Volk aber, das
mit ihm war, da er Lazarus aus dem Grabe rief und von den Toten
auferweckte, rühmte die Tat. \bibverse{18} Darum ging ihm auch das Volk
entgegen, da sie hörten, er hätte solches Zeichen getan. \bibverse{19}
Die Pharisäer aber sprachen untereinander: Ihr sehet, dass ihr nichts
ausrichtet; siehe, alle Welt läuft ihm nach! \footnote{\textbf{12:19}
  Joh 11,48}

\bibverse{20} Es waren aber etliche Griechen unter denen, die
hinaufgekommen waren, dass sie anbeteten auf dem Fest. \bibverse{21} Die
traten zu Philippus, der von Bethsaida aus Galiläa war, baten ihn und
sprachen: Herr, wir wollten Jesum gerne sehen. \bibverse{22} Philippus
kommt und sagt es Andreas, und Philippus und Andreas sagten's weiter
Jesu.

\bibverse{23} Jesus aber antwortete ihnen und sprach: Die Zeit ist
gekommen, dass des Menschen Sohn verklärt werde. \bibverse{24} Wahrlich,
wahrlich ich sage euch: Es sei denn, dass das Weizenkorn in die Erde
falle und ersterbe, so bleibt's allein; wo es aber erstirbt, so bringt
es viele Früchte. \footnote{\textbf{12:24} Röm 14,9; 1Kor 15,36}
\bibverse{25} Wer sein Leben liebhat, der wird's verlieren; und wer sein
Leben auf dieser Welt hasst, der wird's erhalten zum ewigen Leben.
\footnote{\textbf{12:25} Mt 10,39; Mt 16,25; Lk 17,33} \bibverse{26} Wer
mir dienen will, der folge mir nach; und wo ich bin, da soll mein Diener
auch sein. Und wer mir dienen wird, den wird mein Vater ehren.
\footnote{\textbf{12:26} Joh 17,24}

\bibverse{27} Jetzt ist meine Seele betrübt. Und was soll ich sagen?
Vater, hilf mir aus dieser Stunde! Doch darum bin ich in diese Stunde
gekommen. \footnote{\textbf{12:27} Mt 26,38} \bibverse{28} Vater
verkläre deinen Namen! Da kam eine Stimme vom Himmel: Ich habe ihn
verklärt und will ihn abermals verklären. \footnote{\textbf{12:28} Mt
  3,17; Mt 17,5; Joh 13,31}

\bibverse{29} Da sprach das Volk, das dabeistand und zuhörte: Es
donnerte. Die anderen sprachen: Es redete ein Engel mit ihm.

\bibverse{30} Jesus antwortete und sprach: Diese Stimme ist nicht um
meinetwillen geschehen, sondern um euretwillen.

\bibverse{31} Jetzt geht das Gericht über die Welt; nun wird der Fürst
dieser Welt ausgestoßen werden. \footnote{\textbf{12:31} Joh 14,30; Joh
  16,11; Lk 10,18} \bibverse{32} Und ich, wenn ich erhöht werde von der
Erde, so will ich sie alle zu mir ziehen. \footnote{\textbf{12:32} Joh
  8,28} \bibverse{33} (Das sagte er aber, zu deuten, welches Todes er
sterben würde.)

\bibverse{34} Da antwortete ihm das Volk: Wir haben gehört im Gesetz,
dass Christus ewiglich bleibe; und wie sagst du denn: „Des Menschen Sohn
muss erhöht werden``? Wer ist dieser Menschensohn? \footnote{\textbf{12:34}
  Ps 110,4; Dan 7,14}

\bibverse{35} Da sprach Jesus zu ihnen: Es ist das Licht noch eine
kleine Zeit bei euch. Wandelt, dieweil ihr das Licht habt, dass euch die
Finsternis nicht überfalle. Wer in der Finsternis wandelt, der weiß
nicht, wo er hingeht. \footnote{\textbf{12:35} Joh 11,10} \bibverse{36}
Glaubet an das Licht, dieweil ihr's habt, auf dass ihr des Lichtes
Kinder seid. \footnote{\textbf{12:36} Eph 5,9} \bibverse{37} Solches
redete Jesus und ging weg und verbarg sich vor ihnen. Und ob er wohl
solche Zeichen vor ihnen getan hatte, glaubten sie doch nicht an ihn,
\bibverse{38} auf dass erfüllet würde der Spruch des Propheten Jesaja,
den er sagte: „Herr, wer glaubt unserem Predigen? Und wem ist der Arm
des Herrn offenbart?{}``

\bibverse{39} Darum konnten sie nicht glauben, denn Jesaja sagte
abermals: \bibverse{40} „Er hat ihre Augen verblendet und ihr Herz
verstockt, dass sie mit den Augen nicht sehen noch mit dem Herzen
vernehmen und sich bekehren und ich ihnen hülfe.``

\bibverse{41} Solches sagte Jesaja, da er seine Herrlichkeit sah und
redete von ihm. \footnote{\textbf{12:41} Jes 6,1} \bibverse{42} Doch
auch der Obersten glaubten viele an ihn; aber um der Pharisäer willen
bekannten sie es nicht, dass sie nicht in den Bann getan würden.
\footnote{\textbf{12:42} Joh 9,22} \bibverse{43} Denn sie hatten lieber
die Ehre bei den Menschen als die Ehre bei Gott. \footnote{\textbf{12:43}
  Joh 5,44}

\bibverse{44} Jesus aber rief und sprach: Wer an mich glaubt, der glaubt
nicht an mich, sondern an den, der mich gesandt hat. \bibverse{45} Und
wer mich sieht, der sieht den, der mich gesandt hat. \bibverse{46} Ich
bin gekommen in die Welt ein Licht, auf dass, wer an mich glaubt, nicht
in der Finsternis bleibe. \bibverse{47} Und wer meine Worte hört, und
glaubt nicht, den werde ich nicht richten; denn ich bin nicht gekommen,
dass ich die Welt richte, sondern dass ich die Welt selig mache.
\footnote{\textbf{12:47} Joh 3,17; Lk 9,56} \bibverse{48} Wer mich
verachtet und nimmt meine Worte nicht auf, der hat schon seinen Richter
das Wort, welches ich geredet habe, das wird ihn richten am Jüngsten
Tage. \bibverse{49} Denn ich habe nicht von mir selber geredet; sondern
der Vater, der mich gesandt hat, der hat mir ein Gebot gegeben, was ich
tun und reden soll. \bibverse{50} Und ich weiß, dass sein Gebot ist das
ewige Leben. Darum, was ich rede, das rede ich also, wie mir der Vater
gesagt hat. \# 13 \bibverse{1} Vor dem Fest aber der Ostern, da Jesus
erkannte, dass seine Zeit gekommen war, dass er aus dieser Welt ginge
zum Vater: wie er hatte geliebt die Seinen, die in der Welt waren, so
liebte er sie bis ans Ende. \bibverse{2} Und bei dem Abendessen, da
schon der Teufel hatte dem Judas, Simons Sohn, dem Ischariot, ins Herz
gegeben, dass er ihn verriete, \footnote{\textbf{13:2} Lk 22,3}
\bibverse{3} und Jesus wusste, dass ihm der Vater hatte alles in seine
Hände gegeben und dass er von Gott gekommen war und zu Gott ging:
\footnote{\textbf{13:3} Joh 3,35; Joh 16,28} \bibverse{4} stand er vom
Abendmahl auf, legte seine Kleider ab und nahm einen Schurz und
umgürtete sich. \bibverse{5} Darnach goss er Wasser in ein Becken, hob
an, den Jüngern die Füße zu waschen, und trocknete sie mit dem Schurz,
damit er umgürtet war. \bibverse{6} Da kam er zu Simon Petrus; und der
sprach zu ihm: Herr, solltest du mir meine Füße waschen?

\bibverse{7} Jesus antwortete und sprach zu ihm: Was ich tue, das weißt
du jetzt nicht; du wirst es aber hernach erfahren.

\bibverse{8} Da sprach Petrus zu ihm: Nimmermehr sollst du mir die Füße
waschen! Jesus antwortete ihm: Werde ich dich nicht waschen, so hast du
kein Teil mit mir.

\bibverse{9} Spricht zu ihm Simon Petrus: Herr, nicht die Füße allein,
sondern auch die Hände und das Haupt!

\bibverse{10} Spricht Jesus zu ihm: Wer gewaschen ist, der bedarf nichts
denn die Füße waschen, sondern er ist ganz rein. Und ihr seid rein, aber
nicht alle. \footnote{\textbf{13:10} Joh 15,3}

\bibverse{11} (Denn er wusste seinen Verräter wohl; darum sprach er: Ihr
seid nicht alle rein.) \bibverse{12} Da er nun ihre Füße gewaschen
hatte, nahm er seine Kleider und setzte sich wieder nieder und sprach
abermals zu ihnen: Wisset ihr, was ich euch getan habe? \bibverse{13}
Ihr heißet mich Meister und Herr und saget recht daran, denn ich bin es
auch. \bibverse{14} So nun ich, euer Herr und Meister, euch die Füße
gewaschen habe, so sollt ihr auch euch untereinander die Füße waschen.
\footnote{\textbf{13:14} Lk 22,27} \bibverse{15} Ein Beispiel habe ich
euch gegeben, dass ihr tut, wie ich euch getan habe. \footnote{\textbf{13:15}
  Phil 2,5; 1Petr 2,21} \bibverse{16} Wahrlich, wahrlich ich sage euch:
Der Knecht ist nicht größer denn sein Herr, noch der Apostel größer denn
der ihn gesandt hat. \footnote{\textbf{13:16} Mt 10,24} \bibverse{17} So
ihr solches wisset, selig seid ihr, so ihr's tut. \footnote{\textbf{13:17}
  Mt 7,24} \bibverse{18} Nicht sage ich von euch allen; ich weiß, welche
ich erwählt habe. Aber es muss die Schrift erfüllt werden: „Der mein
Brot isset, der tritt mich mit Füßen.`` \bibverse{19} Jetzt sage ich's
euch, ehe denn es geschieht, auf dass, wenn es geschehen ist, ihr
glaubet, dass ich es bin. \bibverse{20} Wahrlich, wahrlich ich sage
euch: Wer aufnimmt, wenn ich jemand senden werde, der nimmt mich auf;
wer aber mich aufnimmt, der nimmt den auf, der mich gesandt hat.
\footnote{\textbf{13:20} Mt 10,40}

\bibverse{21} Da Jesus solches gesagt hatte, ward er betrübt im Geist
und zeugte und sprach: Wahrlich, wahrlich ich sage euch: Einer unter
euch wird mich verraten. \footnote{\textbf{13:21} Joh 12,27}

\bibverse{22} Da sahen sich die Jünger untereinander an, und ward ihnen
bange, von welchem er redete. \bibverse{23} Es war aber einer unter
seinen Jüngern, der zu Tische saß an der Brust Jesu, welchen Jesus
liebhatte. \footnote{\textbf{13:23} Joh 19,26; Joh 20,2; Joh 21,20}
\bibverse{24} Dem winkte Simon Petrus, dass er forschen sollte, wer es
wäre, von dem er sagte.

\bibverse{25} Denn derselbe lag an der Brust Jesu, und er sprach zu ihm:
Herr, wer ist's?

\bibverse{26} Jesus antwortete: Der ist's, dem ich den Bissen eintauche
und gebe. Und er tauchte den Bissen ein und gab ihn Judas, Simons Sohn,
dem Ischariot. \bibverse{27} Und nach dem Bissen fuhr der Satan in ihn.
Da sprach Jesus zu ihm: Was du tust, das tue bald!

\bibverse{28} Das aber wusste niemand am Tische, wozu er's ihm sagte.

\bibverse{29} Etliche meinten, dieweil Judas den Beutel hatte, Jesus
spräche zu ihm: Kaufe, was uns not ist auf das Fest! oder dass er den
Armen etwas gäbe. \bibverse{30} Da er nun den Bissen genommen hatte,
ging er alsbald hinaus. Und es war Nacht.

\bibverse{31} Da er aber hinausgegangen war, spricht Jesus: Nun ist des
Menschen Sohn verklärt, und Gott ist verklärt in ihm. \bibverse{32} Ist
Gott verklärt in ihm, so wird ihn Gott auch verklären in sich selbst und
wird ihn bald verklären. \footnote{\textbf{13:32} Joh 17,1-5}
\bibverse{33} Liebe Kindlein, ich bin noch eine kleine Weile bei euch.
Ihr werdet mich suchen; und wie ich zu den Juden sagte: „Wo ich hin
gehe, da könnet ihr nicht hin kommen``, sage ich jetzt auch euch.
\footnote{\textbf{13:33} Joh 9,21} \bibverse{34} Ein neu Gebot gebe ich
euch, dass ihr euch untereinander liebet, wie ich euch geliebt habe, auf
dass auch ihr einander liebhabet. \footnote{\textbf{13:34} Joh 15,12-13;
  Joh 15,17} \bibverse{35} Dabei wird jedermann erkennen, dass ihr meine
Jünger seid, so ihr Liebe untereinander habt.

\bibverse{36} Spricht Simon Petrus zu ihm: Herr, wo gehst du hin? Jesus
antwortete ihm: Wo ich hin gehe, kannst du mir diesmal nicht folgen;
aber du wirst mir nachmals folgen.

\bibverse{37} Petrus spricht zu ihm: Herr, warum kann ich dir diesmal
nicht folgen? Ich will mein Leben für dich lassen.

\bibverse{38} Jesus antwortete ihm: Solltest du dein Leben für mich
lassen? Wahrlich, wahrlich ich sage dir: Der Hahn wird nicht krähen, bis
du mich dreimal habest verleugnet. \# 14 \bibverse{1} Und er sprach zu
seinen Jüngern: Euer Herz erschrecke nicht! Glaubet an Gott und glaubet
an mich! \bibverse{2} In meines Vaters Hause sind viele Wohnungen.
Wenn's nicht so wäre, so wollte ich zu euch sagen: Ich gehe hin, euch
die Stätte zu bereiten. \footnote{\textbf{14:2} Mt 25,34} \bibverse{3}
Und wenn ich hingehe, euch die Stätte zu bereiten, so will ich
wiederkommen und euch zu mir nehmen, auf dass ihr seid, wo ich bin.
\footnote{\textbf{14:3} Joh 12,26; Joh 17,24} \bibverse{4} Und wo ich
hin gehe, das wisset ihr, und den Weg wisset ihr auch.

\bibverse{5} Spricht zu ihm Thomas: Herr, wir wissen nicht, wo du hin
gehst; und wie können wir den Weg wissen?

\bibverse{6} Jesus spricht zu ihm: Ich bin der Weg und die Wahrheit und
das Leben; niemand kommt zum Vater denn durch mich. \footnote{\textbf{14:6}
  Hebr 10,20; Mt 11,27; Joh 10,9; Röm 5,1-2} \bibverse{7} Wenn ihr mich
kenntet, so kenntet ihr auch meinen Vater. Und von nun an kennet ihr ihn
und habt ihn gesehen.

\bibverse{8} Spricht zu ihm Philippus: Herr, zeige uns den Vater, so
genügt uns.

\bibverse{9} Jesus spricht zu ihm: So lange bin ich bei euch, und du
kennst mich nicht, Philippus? Wer mich sieht, der sieht den Vater; wie
sprichst du denn: Zeige uns den Vater? \bibverse{10} Glaubst du nicht,
dass ich im Vater bin und der Vater in mir ist? Die Worte, die ich zu
euch rede, die rede ich nicht von mir selbst. Der Vater aber, der in mir
wohnt, der tut die Werke. \footnote{\textbf{14:10} Joh 12,49}
\bibverse{11} Glaubet mir, dass ich im Vater und der Vater in mir ist;
wo nicht, so glaubet mir doch um der Werke willen. \footnote{\textbf{14:11}
  Joh 10,25; Joh 10,38} \bibverse{12} Wahrlich, wahrlich ich sage euch:
Wer an mich glaubt, der wird die Werke auch tun, die ich tue, und wird
größere als diese tun; denn ich gehe zum Vater. \footnote{\textbf{14:12}
  Mt 28,19} \bibverse{13} Und was ihr bitten werdet in meinem Namen, das
will ich tun, auf dass der Vater geehrt werde in dem Sohne. \footnote{\textbf{14:13}
  Joh 15,7; Joh 16,24; Mk 11,24; 1Jo 5,14; 1Jo 1,5-15} \bibverse{14} Was
ihr bitten werdet in meinem Namen, das will ich tun. \bibverse{15}
Liebet ihr mich, so haltet meine Gebote. \footnote{\textbf{14:15} Joh
  15,10; 1Jo 5,3} \bibverse{16} Und ich will den Vater bitten, und er
soll euch einen anderen Tröster geben, dass er bei euch bleibe ewiglich:
\footnote{\textbf{14:16} Joh 15,26; Joh 16,7} \bibverse{17} den Geist
der Wahrheit, welchen die Welt nicht kann empfangen; denn sie sieht ihn
nicht und kennt ihn nicht. Ihr aber kennet ihn; denn er bleibt bei euch
und wird in euch sein. \footnote{\textbf{14:17} Joh 16,13} \bibverse{18}
Ich will euch nicht Waisen lassen; ich komme zu euch. \bibverse{19} Es
ist noch um ein kleines, so wird mich die Welt nicht mehr sehen; ihr
aber sollt mich sehen; denn ich lebe, und ihr sollt auch leben.
\bibverse{20} An dem Tage werdet ihr erkennen, dass ich in meinem Vater
bin und ihr in mir und ich in euch. \bibverse{21} Wer meine Gebote hat
und hält sie, der ist es, der mich liebt. Wer mich aber liebt, der wird
von meinem Vater geliebt werden, und ich werde ihn lieben und mich ihm
offenbaren. \footnote{\textbf{14:21} Joh 16,27; 1Jo 5,3}

\bibverse{22} Spricht zu ihm Judas, nicht der Ischariot: Herr, was
ist's, dass du dich uns willst offenbaren und nicht der Welt?
\footnote{\textbf{14:22} Apg 10,40-41}

\bibverse{23} Jesus antwortete und sprach zu ihm: Wer mich liebt, der
wird mein Wort halten; und mein Vater wird ihn lieben, und wir werden zu
ihm kommen und Wohnung bei ihm machen. \footnote{\textbf{14:23} Spr
  8,17; Eph 3,17} \bibverse{24} Wer mich aber nicht liebt, der hält
meine Worte nicht. Und das Wort, das ihr höret, ist nicht mein, sondern
des Vaters, der mich gesandt hat. \footnote{\textbf{14:24} Joh 7,16-17}

\bibverse{25} Solches habe ich zu euch geredet, solange ich bei euch
gewesen bin. \bibverse{26} Aber der Tröster, der Heilige Geist, welchen
mein Vater senden wird in meinem Namen, der wird euch alles lehren und
euch erinnern alles des, das ich euch gesagt habe. \bibverse{27} Den
Frieden lasse ich euch, meinen Frieden gebe ich euch. Nicht gebe ich
euch, wie die Welt gibt. Euer Herz erschrecke nicht und fürchte sich
nicht. \footnote{\textbf{14:27} Joh 16,33; Phil 4,7} \bibverse{28} Ihr
habt gehört, dass ich euch gesagt habe: Ich gehe hin, und komme wieder
zu euch. Hättet ihr mich lieb, so würdet ihr euch freuen, dass ich
gesagt habe: „Ich gehe zum Vater``; denn der Vater ist größer als ich.
\bibverse{29} Und nun habe ich es euch gesagt habe, ehe denn es
geschieht, auf dass, wenn es nun geschehen wird, ihr glaubet.
\bibverse{30} Ich werde nicht mehr viel mit euch reden; denn es kommt
der Fürst dieser Welt, und hat nichts an mir. \bibverse{31} Aber auf
dass die Welt erkenne, dass ich den Vater liebe und ich also tue, wie
mir der Vater geboten hat: stehet auf und lasset uns von hinnen gehen.
\footnote{\textbf{14:31} Joh 10,18}

\hypertarget{section-5}{%
\section{15}\label{section-5}}

\bibverse{1} Ich bin der rechte Weinstock, und mein Vater der
Weingärtner. \bibverse{2} Eine jegliche Rebe an mir, die nicht Frucht
bringt, wird er wegnehmen; und eine jegliche, die da Frucht bringt, wird
er reinigen, dass sie mehr Frucht bringe. \bibverse{3} Ihr seid schon
rein um des Wortes willen, das ich zu euch geredet habe. \bibverse{4}
Bleibet in mir, und ich in euch. Gleichwie die Rebe kann keine Frucht
bringen von ihr selber, sie bleibe denn am Weinstock, also auch ihr
nicht, ihr bleibet denn in mir. \bibverse{5} Ich bin der Weinstock, ihr
seid die Reben. Wer in mir bleibt und ich in ihm, der bringt viele
Frucht, denn ohne mich könnt ihr nichts tun. \footnote{\textbf{15:5}
  2Kor 3,5-6} \bibverse{6} Wer nicht in mir bleibt, der wird weggeworfen
wie eine Rebe und verdorrt, und man sammelt sie und wirft sie ins Feuer,
und müssen brennen. \bibverse{7} So ihr in mir bleibet und meine Worte
in euch bleiben, werdet ihr bitten, was ihr wollt, und es wird euch
widerfahren.

\bibverse{8} Darin wird mein Vater geehrt, dass ihr viel Frucht bringet
und werdet meine Jünger. \footnote{\textbf{15:8} Mt 5,16} \bibverse{9}
Gleichwie mich mein Vater liebt, also liebe ich euch auch. Bleibet in
meiner Liebe! \bibverse{10} So ihr meine Gebote haltet, so bleibet ihr
in meiner Liebe, gleichwie ich meines Vaters Gebote halte und bleibe in
seiner Liebe. \bibverse{11} Solches rede ich zu euch, auf dass meine
Freude in euch bleibe und eure Freude vollkommen werde.

\bibverse{12} Das ist mein Gebot, dass ihr euch untereinander liebet,
gleichwie ich euch liebe. \footnote{\textbf{15:12} Joh 13,34}
\bibverse{13} Niemand hat größere Liebe denn die, dass er sein Leben
lässt für seine Freunde. \footnote{\textbf{15:13} Joh 10,12; 1Jo 3,16}
\bibverse{14} Ihr seid meine Freunde, wenn ihr tut, was ich euch
gebiete. \footnote{\textbf{15:14} Joh 8,31; Mt 12,50} \bibverse{15} Ich
sage hinfort nicht, dass ihr Knechte seid; denn ein Knecht weiß nicht,
was sein Herr tut. Euch aber habe ich gesagt, dass ihr Freunde seid;
denn alles, was ich habe von meinem Vater gehört, habe ich euch
kundgetan. \bibverse{16} Ihr habt mich nicht erwählt; sondern ich habe
euch erwählt und gesetzt, dass ihr hingehet und Frucht bringet und eure
Frucht bleibe, auf dass, so ihr den Vater bittet in meinem Namen, er's
euch gebe.

\bibverse{17} Das gebiete ich euch, dass ihr euch untereinander liebet.
\bibverse{18} So euch die Welt hasst, so wisset, dass sie mich vor euch
gehasst hat. \bibverse{19} Wäret ihr von der Welt, so hätte die Welt das
Ihre lieb; weil ihr aber nicht von der Welt seid, sondern ich habe euch
von der Welt erwählt, darum hasst euch die Welt. \footnote{\textbf{15:19}
  1Jo 4,4; 1Jo 1,4-5; Joh 17,14} \bibverse{20} Gedenket an mein Wort,
das ich euch gesagt habe: „Der Knecht ist nicht größer als sein Herr.``
Haben sie mich verfolgt, sie werden euch auch verfolgen; haben sie mein
Wort gehalten, so werden sie eures auch halten. \footnote{\textbf{15:20}
  Joh 13,16; Mt 10,24-25} \bibverse{21} Aber das alles werden sie euch
tun um meines Namens willen; denn sie kennen den nicht, der mich gesandt
hat. \footnote{\textbf{15:21} Joh 16,3} \bibverse{22} Wenn ich nicht
gekommen wäre und hätte es ihnen gesagt, so hätten sie keine Sünde; nun
aber können sie nichts vorwenden, ihre Sünde zu entschuldigen.
\footnote{\textbf{15:22} Joh 9,41} \bibverse{23} Wer mich hasst, der
hasst auch meinen Vater. \footnote{\textbf{15:23} Lk 10,16}
\bibverse{24} Hätte ich nicht die Werke getan unter ihnen, die kein
anderer getan hat, so hätten sie keine Sünde; nun aber haben sie es
gesehen und hassen doch beide, mich und meinen Vater. \bibverse{25} Doch
dass erfüllet werde der Spruch, in ihrem Gesetz geschrieben: „Sie hassen
mich ohne Ursache.``

\bibverse{26} Wenn aber der Tröster kommen wird, welchen ich euch senden
werde vom Vater, der Geist der Wahrheit, der vom Vater ausgeht, der wird
zeugen von mir. \bibverse{27} Und ihr werdet auch zeugen; denn ihr seid
von Anfang bei mir gewesen. \footnote{\textbf{15:27} Apg 1,8; Apg
  1,21-22; Apg 5,32}

\hypertarget{section-6}{%
\section{16}\label{section-6}}

\bibverse{1} Solches habe ich zu euch geredet, dass ihr euch nicht
ärgert. \bibverse{2} Sie werden euch in den Bann tun. Es kommt aber die
Zeit, dass wer euch tötet, wird meinen, er tue Gott einen Dienst daran.
\bibverse{3} Und solches werden sie euch darum tun, dass sie weder
meinen Vater noch mich erkennen. \footnote{\textbf{16:3} Joh 15,21}
\bibverse{4} Aber solches habe ich zu euch geredet, auf dass, wenn die
Zeit kommen wird, ihr daran gedenket, dass ich's euch gesagt habe.
Solches aber habe ich von Anfang nicht gesagt; denn ich war bei euch.
\bibverse{5} Nun aber gehe ich hin zu dem, der mich gesandt hat; und
niemand unter euch fragt mich: Wo gehst du hin? \bibverse{6} Sondern
weil ich solches zu euch geredet habe, ist euer Herz voll Trauerns
geworden. \bibverse{7} Aber ich sage euch die Wahrheit: es ist euch gut,
dass ich hingehe. Denn wenn ich nicht hingehe, so kommt der Tröster
nicht zu euch; wenn ich aber gehe, will ich ihn zu euch senden.
\bibverse{8} Und wenn derselbe kommt, wird er die Welt strafen um die
Sünde und um die Gerechtigkeit und um das Gericht: \bibverse{9} um die
Sünde, dass sie nicht glauben an mich; \footnote{\textbf{16:9} Joh
  15,22; Joh 15,24} \bibverse{10} um die Gerechtigkeit aber, dass ich
zum Vater gehe und ihr mich hinfort nicht sehet; \footnote{\textbf{16:10}
  Apg 5,31; Röm 4,25} \bibverse{11} um das Gericht, dass der Fürst
dieser Welt gerichtet ist. \footnote{\textbf{16:11} Joh 12,31}

\bibverse{12} Ich habe euch noch viel zu sagen; aber ihr könnt es jetzt
nicht tragen. \footnote{\textbf{16:12} 1Kor 3,1} \bibverse{13} Wenn aber
jener, der Geist der Wahrheit, kommen wird, der wird euch in alle
Wahrheit leiten. Denn er wird nicht von sich selber reden; sondern was
er hören wird, das wird er reden, und was zukünftig ist, wird er euch
verkündigen. \footnote{\textbf{16:13} Joh 14,26; 1Jo 2,27} \bibverse{14}
Derselbe wird mich verklären; denn von dem Meinen wird er's nehmen und
euch verkündigen. \bibverse{15} Alles, was der Vater hat, das ist mein.
Darum habe ich euch gesagt: Er wird's von dem Meinen nehmen und euch
verkündigen.

\bibverse{16} Über ein kleines, so werdet ihr mich nicht sehen; und aber
über ein kleines, so werdet ihr mich sehen, denn ich gehe zum Vater.
\footnote{\textbf{16:16} Joh 14,19}

\bibverse{17} Da sprachen etliche unter seinen Jüngern untereinander:
Was ist das, was er sagt zu uns: Über ein kleines, so werdet ihr mich
nicht sehen; und aber über ein kleines, so werdet ihr mich sehen, und:
Ich gehe zum Vater? \bibverse{18} Da sprachen sie: Was ist das, was er
sagt: Über ein kleines? Wir wissen nicht, was er redet.

\bibverse{19} Da merkte Jesus, dass sie ihn fragen wollten, und sprach
zu ihnen: Davon fraget ihr untereinander, dass ich gesagt habe: Über ein
kleines, so werdet ihr mich nicht sehen; und aber über ein kleines, so
werdet ihr mich sehen. \bibverse{20} Wahrlich, wahrlich ich sage euch:
Ihr werdet weinen und heulen, aber die Welt wird sich freuen; ihr aber
werdet traurig sein; doch eure Traurigkeit soll in Freude verkehrt
werden. \bibverse{21} Ein Weib, wenn sie gebiert, so hat sie
Traurigkeit; denn ihre Stunde ist gekommen. Wenn sie aber das Kind
geboren hat, denkt sie nicht mehr an die Angst um der Freude willen,
dass der Mensch zur Welt geboren ist. \footnote{\textbf{16:21} Jes 26,17}
\bibverse{22} Und ihr habt auch nun Traurigkeit; aber ich will euch
wiedersehen, und euer Herz soll sich freuen, und eure Freude soll
niemand von euch nehmen.

\bibverse{23} Und an dem Tage werdet ihr mich nichts fragen. Wahrlich,
wahrlich ich sage euch: So ihr den Vater etwas bitten werdet in meinem
Namen, so wird er's euch geben. \bibverse{24} Bisher habt ihr nichts
gebeten in meinem Namen. Bittet, so werdet ihr nehmen, dass eure Freude
vollkommen sei. \footnote{\textbf{16:24} Joh 15,11}

\bibverse{25} Solches habe ich zu euch durch Sprichwörter geredet. Es
kommt aber die Zeit, dass ich nicht mehr durch Sprichwörter mit euch
reden werde, sondern euch frei heraus verkündigen von meinem Vater.
\bibverse{26} An dem Tage werdet ihr bitten in meinem Namen. Und ich
sage euch nicht, dass ich den Vater für euch bitten will; \bibverse{27}
denn er selbst, der Vater, hat euch lieb, darum dass ihr mich liebet und
glaubet, dass ich von Gott ausgegangen bin. \bibverse{28} Ich bin vom
Vater ausgegangen und gekommen in die Welt; wiederum verlasse ich die
Welt und gehe zum Vater.

\bibverse{29} Sprechen zu ihm seine Jünger: Siehe, nun redest du frei
heraus und sagst kein Sprichwort. \bibverse{30} Nun wissen wir, dass du
alle Dinge weißt und bedarfst nicht, dass dich jemand frage; darum
glauben wir, dass du von Gott ausgegangen bist.

\bibverse{31} Jesus antwortete ihnen: Jetzt glaubet ihr? \bibverse{32}
Siehe, es kommt die Stunde und ist schon gekommen, dass ihr zerstreut
werdet, ein jeglicher in das Seine, und mich allein lasset. Aber ich bin
nicht allein; denn der Vater ist bei mir. \footnote{\textbf{16:32} Sach
  13,7; Mt 26,31} \bibverse{33} Solches habe ich mit euch geredet, dass
ihr in mir Frieden habet. In der Welt habt ihr Angst; aber seid getrost,
ich habe die Welt überwunden. \footnote{\textbf{16:33} Joh 14,27; Röm
  5,1; 1Jo 5,4}

\hypertarget{section-7}{%
\section{17}\label{section-7}}

\bibverse{1} Solches redete Jesus, und hob seine Augen auf gen Himmel
und sprach: Vater, die Stunde ist da, dass du deinen Sohn verklärest,
auf dass dich dein Sohn auch verkläre; \bibverse{2} gleichwie du ihm
Macht hast gegeben über alles Fleisch, auf dass er das ewige Leben gebe
allen, die du ihm gegeben hast. \footnote{\textbf{17:2} Mt 11,27}
\bibverse{3} Das ist aber das ewige Leben, dass sie dich, der du allein
wahrer Gott bist, und den du gesandt hast, Jesum Christum, erkennen.
\footnote{\textbf{17:3} 1Jo 5,20} \bibverse{4} Ich habe dich verklärt
auf Erden und vollendet das Werk, das du mir gegeben hast, dass ich es
tun sollte. \bibverse{5} Und nun verkläre mich du, Vater, bei dir selbst
mit der Klarheit, die ich bei dir hatte, ehe die Welt war. \footnote{\textbf{17:5}
  Joh 1,1; Phil 2,6}

\bibverse{6} Ich habe deinen Namen offenbart den Menschen, die du mir
von der Welt gegeben hast. Sie waren dein, und du hast sie mir gegeben,
und sie haben dein Wort behalten. \bibverse{7} Nun wissen sie, dass
alles, was du mir gegeben hast, sei von dir. \bibverse{8} Denn die
Worte, die du mir gegeben hast, habe ich ihnen gegeben; und sie haben's
angenommen und erkannt wahrhaftig, dass ich von dir ausgegangen bin, und
glauben, dass du mich gesandt hast. \bibverse{9} Ich bitte für sie und
bitte nicht für die Welt, sondern für die, die du mir gegeben hast; denn
sie sind dein. \footnote{\textbf{17:9} Joh 6,37; Joh 6,44} \bibverse{10}
Und alles, was mein ist, das ist dein, und was dein ist, das ist mein;
und ich bin in ihnen verklärt. \footnote{\textbf{17:10} Joh 16,15}
\bibverse{11} Und ich bin nicht mehr in der Welt; sie aber sind in der
Welt, und ich komme zu dir. Heiliger Vater, erhalte sie in deinem Namen,
die du mir gegeben hast, dass sie eins seien gleichwie wir.
\bibverse{12} Dieweil ich bei ihnen war in der Welt, erhielt ich sie in
deinem Namen. Die du mir gegeben hast, die habe ich bewahrt, und ist
keiner von ihnen verloren, als das verlorene Kind, dass die Schrift
erfüllet würde. \footnote{\textbf{17:12} Joh 6,39; Ps 41,10}
\bibverse{13} Nun aber komme ich zu dir und rede solches in der Welt,
auf dass sie in ihnen haben meine Freude vollkommen. \footnote{\textbf{17:13}
  Joh 15,11} \bibverse{14} Ich habe ihnen gegeben dein Wort, und die
Welt hasste sie; denn sie sind nicht von der Welt, wie denn auch ich
nicht von der Welt bin. \footnote{\textbf{17:14} Joh 15,19}
\bibverse{15} Ich bitte nicht, dass du sie von der Welt nehmest, sondern
dass du sie bewahrest vor dem Übel. \footnote{\textbf{17:15} Mt 6,13;
  2Thes 3,3} \bibverse{16} Sie sind nicht von der Welt, gleichwie ich
auch nicht von der Welt bin. \bibverse{17} Heilige sie in deiner
Wahrheit; dein Wort ist die Wahrheit. \footnote{\textbf{17:17} Ps
  119,160} \bibverse{18} Gleichwie du mich gesandt hast in die Welt, so
sende ich sie auch in die Welt. \footnote{\textbf{17:18} Joh 20,21}
\bibverse{19} Ich heilige mich selbst für sie, auf dass auch sie
geheiligt seien in der Wahrheit. \footnote{\textbf{17:19} Hebr 10,10}

\bibverse{20} Ich bitte aber nicht allein für sie, sondern auch für die,
die durch ihr Wort an mich glauben werden, \footnote{\textbf{17:20} Röm
  10,17} \bibverse{21} auf dass sie alle eins seien, gleichwie du,
Vater, in mir und ich in dir; dass auch sie in uns eins seien, auf dass
die Welt glaube, du habest mich gesandt. \footnote{\textbf{17:21} Gal
  3,28} \bibverse{22} Und ich habe ihnen gegeben die Herrlichkeit, die
du mir gegeben hast, dass sie eins seien, gleichwie wir eins sind,
\footnote{\textbf{17:22} Apg 4,32} \bibverse{23} ich in ihnen und du in
mir, auf dass sie vollkommen seien in eins und die Welt erkenne, dass du
mich gesandt hast und liebest sie, gleichwie du mich liebst. \footnote{\textbf{17:23}
  1Kor 6,17} \bibverse{24} Vater, ich will, dass, wo ich bin, auch die
bei mir seien, die du mir gegeben hast, dass sie meine Herrlichkeit
sehen, die du mir gegeben hast; denn du hast mich geliebt, ehe denn die
Welt gegründet ward. \footnote{\textbf{17:24} Joh 12,26} \bibverse{25}
Gerechter Vater, die Welt kennt dich nicht; ich aber kenne dich, und
diese erkennen, dass du mich gesandt hast. \bibverse{26} Und ich habe
ihnen deinen Namen kundgetan und will ihn kundtun, auf dass die Liebe,
damit du mich liebst, sei in ihnen und ich in ihnen. \# 18 \bibverse{1}
Da Jesus solches geredet hatte, ging er hinaus mit seinen Jüngern über
den Bach Kidron; da war ein Garten, darein ging Jesus und seine Jünger.
\bibverse{2} Judas aber, der ihn verriet, wusste den Ort auch; denn
Jesus versammelte sich oft daselbst mit seinen Jüngern. \footnote{\textbf{18:2}
  Lk 21,37} \bibverse{3} Da nun Judas zu sich hatte genommen die Schar
und der Hohenpriester und Pharisäer Diener, kommt er dahin mit Fackeln,
Lampen und mit Waffen. \bibverse{4} Wie nun Jesus wusste alles, was ihm
begegnen sollte, ging er hinaus und sprach zu ihnen: Wen suchet ihr?

\bibverse{5} Sie antworteten ihm: Jesum von Nazareth. Jesus spricht zu
ihnen: Ich bin's! Judas aber, der ihn verriet, stand auch bei ihnen.

\bibverse{6} Als nun Jesus zu ihnen sprach: Ich bin's! wichen sie zurück
und fielen zu Boden.

\bibverse{7} Da fragte er sie abermals: Wen suchet ihr? Sie aber
sprachen: Jesum von Nazareth.

\bibverse{8} Jesus antwortete: Ich habe es euch gesagt, dass ich es sei.
Suchet ihr denn mich, so lasset diese gehen!

\bibverse{9} (Auf dass das Wort erfüllet würde, welches er sagte: Ich
habe der keinen verloren, die du mir gegeben hast.)

\bibverse{10} Da hatte Simon Petrus ein Schwert und zog es aus und
schlug nach des Hohenpriesters Knecht und hieb ihm sein rechtes Ohr ab.
Und der Knecht hieß Malchus.

\bibverse{11} Da sprach Jesus zu Petrus: Stecke dein Schwert in die
Scheide! Soll ich den Kelch nicht trinken, den mir mein Vater gegeben
hat?

\bibverse{12} Die Schar aber und der Oberhauptmann und die Diener der
Juden nahmen Jesum und banden ihn \bibverse{13} und führten ihn zuerst
zu Hannas; der war des Kaiphas Schwiegervater, welcher des Jahrs
Hoherpriester war. \bibverse{14} Es war aber Kaiphas, der den Juden
riet, es wäre gut, dass ein Mensch würde umgebracht für das Volk.
\footnote{\textbf{18:14} Joh 11,49-50; Lk 3,1-2}

\bibverse{15} Simon Petrus aber folgte Jesu nach und ein anderer Jünger.
Dieser Jünger war dem Hohenpriester bekannt und ging mit Jesu hinein in
des Hohenpriesters Palast. \bibverse{16} Petrus aber stand draußen vor
der Tür. Da ging der andere Jünger, der dem Hohenpriester bekannt war,
hinaus und redete mit der Türhüterin und führte Petrus hinein.
\bibverse{17} Da sprach die Magd, die Türhüterin, zu Petrus: Bist du
nicht auch dieses Menschen Jünger einer? Er sprach: Ich bin's nicht.

\bibverse{18} Es standen aber die Knechte und Diener und hatten ein
Kohlenfeuer gemacht, denn es war kalt, und wärmten sich. Petrus aber
stand bei ihnen und wärmte sich.

\bibverse{19} Aber der Hohepriester fragte Jesum um seine Jünger und um
seine Lehre.

\bibverse{20} Jesus antwortete ihm: Ich habe frei öffentlich geredet vor
der Welt; ich habe allezeit gelehrt in der Schule und in dem Tempel, da
alle Juden zusammenkommen, und habe nichts im Verborgenen geredet.

\bibverse{21} Was fragst du mich darum? Frage die darum, die gehört
haben, was ich zu ihnen geredet habe; siehe, diese wissen, was ich
gesagt habe.

\bibverse{22} Als er aber solches redete, gab der Diener einer, die
dabeistanden, Jesu einen Backenstreich und sprach: Sollst du dem
Hohenpriester also antworten?

\bibverse{23} Jesus antwortete: Habe ich übel geredet, so beweise es,
dass es böse sei; habe ich aber recht geredet, was schlägst du mich?

\bibverse{24} Und Hannas sandte ihn gebunden zu dem Hohenpriester
Kaiphas.

\bibverse{25} Simon Petrus aber stand und wärmte sich. Da sprachen sie
zu ihm: Bist du nicht seiner Jünger einer? Er leugnete aber und sprach:
Ich bin's nicht!

\bibverse{26} Spricht einer von des Hohenpriesters Knechten, ein
Gefreunder des, dem Petrus das Ohr abgehauen hatte: Sah ich dich nicht
im Garten bei ihm?

\bibverse{27} Da leugnete Petrus abermals, und alsbald krähte der Hahn.

\bibverse{28} Da führten sie Jesum von Kaiphas vor das Richthaus. Und es
war früh; und sie gingen nicht in das Richthaus, auf dass sie nicht
unrein würden, sondern Ostern essen möchten.

\bibverse{29} Da ging Pilatus zu ihnen heraus und sprach: Was bringet
ihr für Klage wider diesen Menschen?

\bibverse{30} Sie antworteten und sprachen zu ihm: Wäre dieser nicht ein
Übeltäter, wir hätten dir ihn nicht überantwortet.

\bibverse{31} Da sprach Pilatus zu ihnen: So nehmet ihr ihn hin und
richtet ihn nach eurem Gesetz. Da sprachen die Juden zu ihm: Wir dürfen
niemand töten. \footnote{\textbf{18:31} Joh 19,6-7}

\bibverse{32} (Auf das erfüllet würde das Wort Jesu, welches er sagte,
da er deutete, welches Todes er sterben würde.) \footnote{\textbf{18:32}
  Joh 12,32-33; Mt 20,19}

\bibverse{33} Da ging Pilatus wieder hinein ins Richthaus und rief Jesum
und sprach zu ihm: Bist du der Juden König?

\bibverse{34} Jesus antwortete: Redest du das von dir selbst, oder
haben's dir andere von mir gesagt?

\bibverse{35} Pilatus antwortete: Bin ich ein Jude? Dein Volk und die
Hohenpriester haben dich mir überantwortet. Was hast du getan?

\bibverse{36} Jesus antwortete: Mein Reich ist nicht von dieser Welt.
Wäre mein Reich von dieser Welt, meine Diener würden kämpfen, dass ich
den Juden nicht überantwortet würde; aber nun ist mein Reich nicht von
dannen.

\bibverse{37} Da sprach Pilatus zu ihm: So bist du dennoch ein König?
Jesus antwortete: Du sagst es, ich bin ein König. Ich bin dazu geboren
und in die Welt gekommen, dass ich für die Wahrheit zeugen soll. Wer aus
der Wahrheit ist, der höret meine Stimme. \footnote{\textbf{18:37} 1Tim
  6,13}

\bibverse{38} Spricht Pilatus zu ihm: Was ist Wahrheit? Und da er das
gesagt, ging er wieder hinaus zu den Juden und spricht zu ihnen: Ich
finde keine Schuld an ihm.

\bibverse{39} Ihr habt aber eine Gewohnheit, dass ich euch einen auf
Ostern losgebe; wollt ihr nun, dass ich euch der Juden König losgebe?

\bibverse{40} Da schrien sie wieder allesamt und sprachen: Nicht diesen,
sondern Barabbas! Barabbas aber war ein Mörder. \# 19 \bibverse{1} Da
nahm Pilatus Jesum und geißelte ihn. \bibverse{2} Und die Kriegsknechte
flochten eine Krone von Dornen und setzten sie auf sein Haupt und legten
ihm ein Purpurkleid an \bibverse{3} und sprachen: Sei gegrüßt, lieber
Judenkönig! und gaben ihm Backenstreiche.

\bibverse{4} Da ging Pilatus wieder heraus und sprach zu ihnen: Sehet,
ich führe ihn heraus zu euch, dass ihr erkennet, dass ich keine Schuld
an ihm finde.

\bibverse{5} Also ging Jesus heraus und trug eine Dornenkrone und ein
Purpurkleid. Und er spricht zu ihnen: Sehet, welch ein Mensch!

\bibverse{6} Da ihn die Hohenpriester und die Diener sahen, schrien sie
und sprachen: Kreuzige! kreuzige! Pilatus spricht zu ihnen: Nehmt ihr
ihn hin und kreuzigt ihn; denn ich finde keine Schuld an ihm.

\bibverse{7} Die Juden antworteten ihm: Wir haben ein Gesetz, und nach
dem Gesetz soll er sterben; denn er hat sich selbst zu Gottes Sohn
gemacht.

\bibverse{8} Da Pilatus das Wort hörte, fürchtete er sich noch mehr

\bibverse{9} und ging wieder hinein in das Richthaus und spricht zu
Jesus: Woher bist du? Aber Jesus gab ihm keine Antwort. \bibverse{10} Da
sprach Pilatus zu ihm: Redest du nicht mit mir? Weißt du nicht, dass ich
Macht habe, dich zu kreuzigen, und Macht habe, dich loszugeben?

\bibverse{11} Jesus antwortete: Du hättest keine Macht über mich, wenn
sie dir nicht wäre von obenherab gegeben; darum, der mich dir
überantwortet hat, der hat größere Sünde.

\bibverse{12} Von da an trachtete Pilatus, wie er ihn losließe. Die
Juden aber schrien und sprachen: Lässt du diesen los, so bist du des
Kaisers Freund nicht; denn wer sich zum König macht, der ist wider den
Kaiser. \footnote{\textbf{19:12} Apg 17,7}

\bibverse{13} Da Pilatus das Wort hörte, führte er Jesum heraus und
setzte sich auf den Richtstuhl an der Stätte, die da heißt Hochpflaster,
auf hebräisch aber Gabbatha. \bibverse{14} Es war aber der Rüsttag auf
Ostern, um die sechste Stunde. Und er spricht zu den Juden: Sehet, das
ist euer König!

\bibverse{15} Sie schrien aber: Weg, weg mit dem! kreuzige ihn! Spricht
Pilatus zu ihnen: Soll ich euren König kreuzigen? Die Hohenpriester
antworteten: Wir haben keinen König denn den Kaiser.

\bibverse{16} Da überantwortete er ihn, dass er gekreuzigt würde. Sie
nahmen aber Jesum und führten ihn hin.

\bibverse{17} Und er trug sein Kreuz und ging hinaus zur Stätte, die da
heißt Schädelstätte, welche heißt auf hebräisch Golgatha.

\bibverse{18} Allda kreuzigten sie ihn und mit ihm zwei andere zu beiden
Seiten, Jesum aber mitteninne. \bibverse{19} Pilatus aber schrieb eine
Überschrift und setzte sie auf das Kreuz; und war geschrieben: Jesus von
Nazareth, der Juden König. \bibverse{20} Diese Überschrift lasen viele
Juden; denn die Stätte war nahe bei der Stadt, da Jesus gekreuzigt ward.
Und es war geschrieben in hebräischer, griechischer und lateinischer
Sprache. \bibverse{21} Da sprachen die Hohenpriester der Juden zu
Pilatus: Schreibe nicht: „Der Juden König``, sondern dass er gesagt
habe: Ich bin der Juden König.

\bibverse{22} Pilatus antwortete: Was ich geschrieben habe, das habe ich
geschrieben.

\bibverse{23} Die Kriegsknechte aber, da sie Jesum gekreuzigt hatten,
nahmen sie seine Kleider und machten vier Teile, einem jeglichen
Kriegsknechte ein Teil, dazu auch den Rock. Der Rock aber war ungenäht,
von obenan gewirkt durch und durch. \bibverse{24} Da sprachen sie
untereinander: Lasset uns den nicht zerteilen, sondern darum losen, wes
er sein soll. (Auf dass erfüllet würde die Schrift, die da sagt: „Sie
haben meine Kleider unter sich geteilt und haben über meinen Rock das
Los geworfen.``) Solches taten die Kriegsknechte.

\bibverse{25} Es stand aber bei dem Kreuze Jesu seine Mutter und seiner
Mutter Schwester, Maria, des Kleophas Weib, und Maria Magdalena.
\bibverse{26} Da nun Jesus seine Mutter sah und den Jünger dabeistehen,
den er liebhatte, spricht er zu seiner Mutter: Weib, siehe, das ist dein
Sohn! \footnote{\textbf{19:26} Joh 13,23} \bibverse{27} Darnach spricht
er zu dem Jünger: Siehe, das ist deine Mutter! Und von der Stunde an
nahm sie der Jünger zu sich.

\bibverse{28} Darnach, da Jesus wusste, dass schon alles vollbracht war,
dass die Schrift erfüllt würde, spricht er: Mich dürstet! \bibverse{29}
Da stand ein Gefäß voll Essig. Sie aber füllten einen Schwamm mit Essig
und legten ihn um einen Isop und hielten es ihm dar zum Munde.
\footnote{\textbf{19:29} Ps 69,22} \bibverse{30} Da nun Jesus den Essig
genommen hatte, sprach er: Es ist vollbracht! und neigte das Haupt und
verschied.

\bibverse{31} Die Juden aber, dieweil es der Rüsttag war, dass nicht die
Leichname am Kreuze blieben den Sabbat über (denn desselben Sabbats Tag
war groß), baten sie Pilatus, dass ihre Beine gebrochen und sie
abgenommen würden. \bibverse{32} Da kamen die Kriegsknechte und brachen
dem ersten die Beine und dem anderen, der mit ihm gekreuzigt war.
\bibverse{33} Als sie aber zu Jesu kamen und sahen, dass er schon
gestorben war, brachen sie ihm die Beine nicht; \bibverse{34} sondern
der Kriegsknechte einer öffnete seine Seite mit einem Speer, und alsbald
ging Blut und Wasser heraus. \bibverse{35} Und der das gesehen hat, der
hat es bezeugt, und sein Zeugnis ist wahr; und dieser weiß, dass er die
Wahrheit sagt, auf dass auch ihr glaubet. \bibverse{36} Denn solches ist
geschehen, dass die Schrift erfüllet würde: „Ihr sollt ihm kein Bein
zerbrechen.`` \bibverse{37} Und abermals spricht eine andere Schrift:
„Sie werden sehen, in welchen sie gestochen haben.`` \footnote{\textbf{19:37}
  Offb 1,7}

\bibverse{38} Darnach bat den Pilatus Joseph von Arimathia, der ein
Jünger Jesu war, doch heimlich aus Furcht vor den Juden, dass er möchte
abnehmen den Leichnam Jesu. Und Pilatus erlaubte es. Da kam er und nahm
den Leichnam Jesu herab. \footnote{\textbf{19:38} Joh 7,13}
\bibverse{39} Es kam aber auch Nikodemus, der vormals bei der Nacht zu
Jesu gekommen war, und brachte Myrrhe und Aloe untereinander bei hundert
Pfunden. \footnote{\textbf{19:39} Joh 3,2} \bibverse{40} Da nahmen sie
den Leichnam Jesu und banden ihn in leinene Tücher mit den Spezereien,
wie die Juden pflegen zu begraben. \bibverse{41} Es war aber an der
Stätte, da er gekreuzigt ward, ein Garten, und im Garten ein neues Grab,
in welches niemand je gelegt war. \bibverse{42} Dahin legten sie Jesum
um des Rüsttages willen der Juden, dieweil das Grab nahe war. \# 20
\bibverse{1} An dem ersten Tage der Woche kommt Maria Magdalena früh, da
es noch finster war, zum Grabe und sieht, dass der Stein vom Grabe
hinweg war. \bibverse{2} Da läuft sie und kommt zu Simon Petrus und zu
dem anderen Jünger, welchen Jesus liebhatte, und spricht zu ihnen: Sie
haben den Herrn weggenommen aus dem Grabe, und wir wissen nicht, wo sie
ihn hin gelegt haben.

\bibverse{3} Da ging Petrus und der andere Jünger hinaus und kamen zum
Grabe. \bibverse{4} Es liefen aber die zwei miteinander, und der andere
Jünger lief zuvor, schneller denn Petrus, und kam am ersten zum Grabe,
\bibverse{5} guckt hinein und sieht die Leinen gelegt; er ging aber
nicht hinein. \bibverse{6} Da kam Simon Petrus ihm nach und ging hinein
in das Grab und sieht die Leinen gelegt, \bibverse{7} und das
Schweißtuch, das Jesus um das Haupt gebunden war, nicht zu den Leinen
gelegt, sondern beiseite, zusammengewickelt, an einen besonderen Ort.
\footnote{\textbf{20:7} Joh 11,44} \bibverse{8} Da ging auch der andere
Jünger hinein, der am ersten zum Grabe kam, und sah und glaubte es.
\bibverse{9} Denn sie wussten die Schrift noch nicht, dass er von den
Toten auferstehen müsste. \bibverse{10} Da gingen die Jünger wieder
heim.

\bibverse{11} Maria aber stand vor dem Grabe und weinte draußen. Als sie
nun weinte, guckte sie ins Grab \bibverse{12} und sieht zwei Engel in
weißen Kleidern sitzen, einen zu den Häupten und den anderen zu den
Füßen, da sie den Leichnam Jesu hin gelegt hatten. \bibverse{13} Und
diese sprachen zu ihr: Weib, was weinest du? Sie spricht zu ihnen: Sie
haben meinen Herrn weggenommen, und ich weiß nicht, wo sie ihn hin
gelegt haben.

\bibverse{14} Und als sie das sagte, wandte sie sich zurück und sieht
Jesum stehen und weiß nicht, dass es Jesus ist.

\bibverse{15} Spricht Jesus zu ihr: Weib, was weinest du? Wen suchest
du? Sie meint, es sei der Gärtner, und spricht zu ihm: Herr, hast du ihn
weggetragen, so sage mir, wo hast du ihn hin gelegt, so will ich ihn
holen.

\bibverse{16} Spricht Jesus zu ihr: Maria! Da wandte sie sich um und
spricht zu ihm: Rabbuni (das heißt: Meister)!

\bibverse{17} Spricht Jesus zu ihr: Rühre mich nicht an! denn ich bin
noch nicht aufgefahren zu meinem Vater. Gehe aber hin zu meinen Brüdern
und sage ihnen: Ich fahre auf zu meinem Vater und zu eurem Vater, zu
meinem Gott und zu eurem Gott. \footnote{\textbf{20:17} Hebr 2,11-12}

\bibverse{18} Maria Magdalena kommt und verkündigt den Jüngern: Ich habe
den Herrn gesehen, und solches hat er zu mir gesagt.

\bibverse{19} Am Abend aber desselben ersten Tages der Woche, da die
Jünger versammelt und die Türen verschlossen waren aus Furcht vor den
Juden, kam Jesus und trat mitten ein und spricht zu ihnen: Friede sei
mit euch!

\bibverse{20} Und als er das gesagt hatte, zeigte er ihnen die Hände und
seine Seite. Da wurden die Jünger froh, dass sie den Herrn sahen.

\bibverse{21} Da sprach Jesus abermals zu ihnen: Friede sei mit euch!
Gleichwie mich der Vater gesandt hat, so sende ich euch. \footnote{\textbf{20:21}
  Joh 17,18} \bibverse{22} Und da er das gesagt hatte, blies er sie an
und spricht zu ihnen: Nehmet hin den Heiligen Geist! \bibverse{23}
Welchen ihr die Sünden erlasset, denen sind sie erlassen; und welchen
ihr sie behaltet, denen sind sie behalten.

\bibverse{24} Thomas aber, der Zwölf einer, der da heißt Zwilling, war
nicht bei ihnen, da Jesus kam. \footnote{\textbf{20:24} Joh 11,16; Joh
  14,5; Joh 21,2} \bibverse{25} Da sagten die anderen Jünger zu ihm: Wir
haben den Herrn gesehen. Er aber sprach zu ihnen: Es sei denn, dass ich
in seinen Händen sehe die Nägelmale und lege meinen Finger in die
Nägelmale und lege meine Hand in seine Seite, will ich's nicht glauben.
\footnote{\textbf{20:25} Joh 19,34}

\bibverse{26} Und über acht Tage waren abermals seine Jünger drinnen und
Thomas mit ihnen. Kommt Jesus, da die Türen verschlossen waren, und
tritt mitten ein und spricht: Friede sei mit euch!

\bibverse{27} Darnach spricht er zu Thomas: Reiche deinen Finger her und
siehe meine Hände, und reiche dein Hand her und lege sie in meine Seite,
und sei nicht ungläubig, sondern gläubig!

\bibverse{28} Thomas antwortete und sprach zu ihm: Mein Herr und mein
Gott! \footnote{\textbf{20:28} Joh 1,1}

\bibverse{29} Spricht Jesus zu ihm: Dieweil du mich gesehen hast,
Thomas, so glaubst du. Selig sind, die nicht sehen und doch glauben!
\footnote{\textbf{20:29} 1Petr 1,8; Hebr 11,1}

\bibverse{30} Auch viele andere Zeichen tat Jesus vor seinen Jüngern,
die nicht geschrieben sind in diesem Buch. \footnote{\textbf{20:30} Joh
  21,24-25} \bibverse{31} Diese aber sind geschrieben, dass ihr glaubet,
Jesus sei Christus, der Sohn Gottes, und dass ihr durch den Glauben das
Leben habet in seinem Namen. \footnote{\textbf{20:31} 1Jo 5,13}

\hypertarget{section-8}{%
\section{21}\label{section-8}}

\bibverse{1} Darnach offenbarte sich Jesus abermals den Jüngern an dem
Meer bei Tiberias. Er offenbarte sich aber also: \bibverse{2} Es waren
beieinander Simon Petrus und Thomas, der da heißt Zwilling, und
Nathanael von Kana in Galiläa und die Söhne des Zebedäus und andere zwei
seiner Jünger. \footnote{\textbf{21:2} Joh 1,45} \bibverse{3} Spricht
Simon Petrus zu ihnen: Ich will hin fischen gehen. Sie sprechen zu ihm:
So wollen wir mit dir gehen. Sie gingen hinaus und traten in das Schiff
alsobald; und in derselben Nacht fingen sie nichts.

\bibverse{4} Da aber jetzt Morgen war, stand Jesus am Ufer; aber die
Jünger wussten nicht, dass es Jesus war. \bibverse{5} Spricht Jesus zu
ihnen: Kinder, habt ihr nichts zu essen? Sie antworteten ihm: Nein.
\footnote{\textbf{21:5} Lk 24,41}

\bibverse{6} Er aber sprach zu ihnen: Werfet das Netz zur Rechten des
Schiffs, so werdet ihr finden. Da warfen sie, und konnten's nicht mehr
ziehen vor der Menge der Fische. \footnote{\textbf{21:6} Lk 5,4-7}

\bibverse{7} Da spricht der Jünger, welchen Jesus liebhatte, zu Petrus:
Es ist der Herr! Da Simon Petrus hörte, dass es der Herr war, gürtete er
das Hemd um sich (denn er war nackt) und warf sich ins Meer. \footnote{\textbf{21:7}
  Joh 13,23}

\bibverse{8} Die anderen Jünger aber kamen auf dem Schiff (denn sie
waren nicht ferne vom Lande, sondern bei zweihundert Ellen) und zogen
das Netz mit den Fischen.

\bibverse{9} Als sie nun austraten auf das Land, sahen sie Kohlen gelegt
und Fische darauf und Brot. \bibverse{10} Spricht Jesus zu ihnen:
Bringet her von den Fischen, die ihr jetzt gefangen habt!

\bibverse{11} Simon Petrus stieg hinein und zog das Netz auf das Land
voll großer Fische, hundertdreiundfünfzig. Und wiewohl ihrer so viel
waren, zerriss doch das Netz nicht.

\bibverse{12} Spricht Jesus zu ihnen: Kommt und haltet das Mahl! Niemand
aber unter den Jüngern wagte, ihn zu fragen: Wer bist du? denn sie
wussten, dass es der Herr war.

\bibverse{13} Da kommt Jesus und nimmt das Brot und gibt es ihnen,
desgleichen auch die Fische.

\bibverse{14} Das ist nun das drittemal, dass Jesus offenbart ward
seinen Jüngern, nachdem er von den Toten auferstanden war. \bibverse{15}
Da sie nun das Mahl gehalten hatten, spricht Jesus zu Simon Petrus:
Simon Jona, hast du mich lieber, denn mich diese haben? Er spricht zu
ihm: Ja, Herr, du weißt, dass ich dich liebhabe. Spricht er zu ihm:
Weide meine Lämmer! \footnote{\textbf{21:15} Joh 1,42}

\bibverse{16} Spricht er wieder zum andernmal zu ihm: Simon Jona, hast
du mich lieb? Er spricht zu ihm: Ja, Herr, du weißt, dass ich dich
liebhabe. Spricht Jesus zu ihm: Weide meine Schafe! \footnote{\textbf{21:16}
  1Petr 5,2; 1Petr 5,4}

\bibverse{17} Spricht er zum drittenmal zu ihm: Simon Jona, hast du mich
lieb? Petrus ward traurig, dass er zum drittenmal zu ihm sagte: Hast du
mich lieb? und sprach zu ihm: Herr, du weißt alle Dinge, du weißt, dass
ich dich liebhabe. Spricht Jesus zu ihm: Weide meine Schafe! \footnote{\textbf{21:17}
  Joh 13,38; Joh 16,30}

\bibverse{18} Wahrlich, wahrlich ich sage dir: Da du jünger warst,
gürtetest du dich selbst und wandeltest, wohin du wolltest; wenn du aber
alt wirst, wirst du deine Hände ausstrecken, und ein anderer wird dich
gürten und führen, wohin du nicht willst.

\bibverse{19} Das sagte er aber, zu deuten, mit welchem Tode er Gott
preisen würde. Und da er das gesagt, spricht er zu ihm: Folge mir nach!

\bibverse{20} Petrus aber wandte sich um und sah den Jünger folgen,
welchen Jesus liebhatte, der auch an seiner Brust beim Abendessen
gelegen war und gesagt hatte: Herr, wer ist's, der dich verrät?
\footnote{\textbf{21:20} Joh 13,23; Joh 13,25}

\bibverse{21} Da Petrus diesen sah, spricht er zu Jesu: Herr, was soll
aber dieser?

\bibverse{22} Jesus spricht zu ihm: So ich will, dass er bleibe, bis ich
komme, was geht es dich an? Folge du mir nach!

\bibverse{23} Da ging eine Rede aus unter den Brüdern: Dieser Jünger
stirbt nicht. Und Jesus sprach nicht zu ihm: „Er stirbt nicht``,
sondern: „So ich will, dass er bleibe, bis ich komme, was geht es dich
an?{}``

\bibverse{24} Dies ist der Jünger, der von diesen Dingen zeugt und dies
geschrieben hat; und wir wissen, dass sein Zeugnis wahrhaftig ist.

\bibverse{25} Es sind auch viele andere Dinge, die Jesus getan hat; wenn
sie aber sollten eins nach dem anderen geschrieben werden, achte ich,
die Welt würde die Bücher nicht fassen, die zu schreiben wären.
