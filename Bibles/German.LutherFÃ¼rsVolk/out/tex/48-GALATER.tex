\hypertarget{section}{%
\section{1}\label{section}}

\bibverse{1} Paulus, ein Apostel (nicht von Menschen, auch nicht durch
Menschen, sondern durch JEsum Christum und GOtt den Vater, der ihn
auferwecket hat von den Toten),\bibverse{1} und alle Brüder, die bei mir
sind: Den Gemeinden in Galatien. \bibverse{2} Gnade sei mit euch und
Friede von GOtt dem Vater und unserm HErrn JEsu Christo, \bibverse{3}
der sich selbst für unsere Sünden gegeben hat, daß er uns errettete von
dieser gegenwärtigen argen Welt nach dem Willen GOttes und unsers
Vaters, \bibverse{4} welchem sei Ehre von Ewigkeit zu Ewigkeit! Amen.
\bibverse{5} Mich wundert, daß ihr euch so bald abwenden lasset von dem,
der euch berufen hat in die Gnade Christi, auf ein ander Evangelium,
\bibverse{6} so doch kein anderes ist; ohne daß etliche sind, die euch
verwirren und wollen das Evangelium Christi verkehren. \bibverse{7} Aber
so auch wir oder ein Engel vom Himmel euch würde Evangelium predigen
anders, denn das wir euch geprediget haben, der sei verflucht!
\bibverse{8} Wie wir jetzt gesagt haben, so sagen wir auch abermal: So
jemand euch Evangelium prediget anders, denn das ihr empfangen habt, der
sei verflucht! \bibverse{9} Predige ich denn jetzt Menschen oder GOtt zu
Dienst? Oder gedenke ich, Menschen gefällig zu sein? Wenn ich den
Menschen noch gefällig wäre, so wäre ich Christi Knecht nicht.
\bibverse{10} Ich tue euch aber kund, liebe Brüder, daß das Evangelium,
das von mir geprediget ist, nicht menschlich ist. \bibverse{11} Denn ich
hab' es von keinem Menschen empfangen noch gelernet, sondern durch die
Offenbarung JEsu Christi. \bibverse{12} Denn ihr habt je wohl gehöret
meinen Wandel weiland im Judentum, wie ich über die Maßen die Gemeinde
GOttes verfolgte und verstörete sie \bibverse{13} und nahm zu im
Judentum über viele meinesgleichen in meinem Geschlecht und eiferte über
die Maßen um das väterliche Gesetz. \bibverse{14} Da es aber GOtt
wohlgefiel, der mich von meiner Mutter Leibe hat ausgesondert und
berufen durch seine Gnade, \bibverse{15} daß er seinen Sohn offenbarete
in mir, daß ich ihn durchs Evangelium verkündigen sollte unter den
Heiden, alsobald fuhr ich zu und besprach mich nicht darüber mit Fleisch
und Blut, \bibverse{16} kam auch nicht gen Jerusalem zu denen, die vor
mir Apostel waren, sondern zog hin nach Arabien und kam wiederum gen
Damaskus. \bibverse{17} Danach über drei Jahre kam ich gen Jerusalem,
Petrus zu schauen, und blieb fünfzehn Tage bei ihm. \bibverse{18} Der
andern Apostel aber sah ich keinen ohne Jakobus, des HErrn Bruder.
\bibverse{19} Was ich euch aber schreibe, siehe, GOtt weiß, ich lüge
nicht. \bibverse{20} Danach kam ich in die Länder Syrien und Zilizien.
\bibverse{21} Ich war aber unbekannt von Angesicht den christlichen
Gemeinden in Judäa. \bibverse{22} Sie hatten aber allein gehöret, daß,
der uns weiland verfolgte, der prediget jetzt den Glauben, welchen er
weiland verstörete; \bibverse{23} und preiseten GOtt über mir.

\hypertarget{section-1}{%
\section{2}\label{section-1}}

\bibverse{1} Danach über vierzehn Jahre zog ich abermal hinauf gen
Jerusalem mit Barnabas und nahm Titus auch mit mir. \bibverse{2} Ich zog
aber hinauf aus einer Offenbarung und besprach mich mit ihnen über dem
Evangelium, das ich predige unter den Heiden, besonders aber mit denen,
die das Ansehen hatten, auf daß ich nicht vergeblich liefe oder gelaufen
hätte. \bibverse{3} Aber es ward auch Titus nicht gezwungen, sich zu
beschneiden, der mit mir war, ob er wohl ein Grieche war. \bibverse{4}
Denn da etliche falsche Brüder sich mit eingedrungen und neben
eingeschlichen waren, zu verkundschaften unsere Freiheit, die wir haben
in Christo JEsu, daß sie uns gefangennähmen, \bibverse{5} wichen wir
denselbigen nicht eine Stunde, untertan zu sein, auf daß die Wahrheit
des Evangeliums bei euch bestünde. \bibverse{6} Von denen aber, die das
Ansehen hatten, welcherlei sie weiland gewesen sind, da liegt mir nichts
an; denn GOtt achtet das Ansehen der Menschen nicht. Mich aber haben
die, so das Ansehen hatten, nichts anderes gelehret, \bibverse{7}
sondern wiederum, da sie sahen, daß mir vertrauet war das Evangelium an
die Vorhaut, gleichwie Petrus das Evangelium an die Beschneidung
\bibverse{8} (denn der mit Petrus kräftig ist gewesen zum Apostelamt
unter die Beschneidung, der ist mit mir auch kräftig gewesen unter die
Heiden), \bibverse{9} und erkannten die Gnade, die mir gegeben war,
Jakobus und Kephas und Johannes, die für Säulen angesehen waren, gaben
sie mir und Barnabas die rechte Hand und wurden mit uns eins, daß wir
unter die Heiden, sie aber unter der Beschneidung predigten;
\bibverse{10} allein daß wir der Armen gedächten, welches ich auch
fleißig bin gewesen zu tun. \bibverse{11} Da aber Petrus gen Antiochien
kam, widerstund ich ihm unter Augen; denn es war Klage über ihn kommen.
\bibverse{12} Denn zuvor, ehe etliche von Jakobus kamen, aß er mit den
Heiden; da sie aber kamen, entzog er sich und sonderte sich, darum daß
er die von der Beschneidung fürchtete. \bibverse{13} Und heuchelten mit
ihm die andern Juden, also daß auch Barnabas verführet ward, mit ihnen
zu heucheln. \bibverse{14} Aber da ich sah, daß sie nicht richtig
wandelten nach der Wahrheit des Evangeliums, sprach ich zu Petrus vor
allen öffentlich: So du, der du ein Jude bist, heidnisch lebest und
nicht jüdisch, warum zwingest du denn die Heiden, jüdisch zu leben?
\bibverse{15} Wiewohl wir von Natur Juden und nicht Sünder aus den
Heiden sind, \bibverse{16} doch, weil wir wissen, daß der Mensch durch
des Gesetzes Werke nicht gerecht wird, sondern durch den Glauben an
JEsum Christum, so glauben wir auch an Christum JEsum, auf daß wir
gerecht werden durch den Glauben an Christum und nicht durch des
Gesetzes Werke; denn durch des Gesetzes Werke wird kein Fleisch gerecht.
\bibverse{17} Sollten wir aber, die da suchen durch Christum gerecht zu
werden, auch noch selbst als Sünder erfunden werden, so wäre Christus
ein Sündendiener. Das sei ferne! \bibverse{18} Wenn ich aber das, so ich
zerbrochen habe, wiederum baue, so mache ich mich selbst zu einem
Übertreter. \bibverse{19} Ich bin aber durchs Gesetz dem Gesetz
gestorben auf daß ich GOtt lebe; ich bin mit Christo gekreuziget.
\bibverse{20} Ich lebe aber, doch nun nicht ich, sondern Christus lebet
in mir. Denn was ich jetzt lebe im Fleisch, das lebe ich in dem Glauben
des Sohnes GOttes, der mich geliebet hat und sich selbst für mich
dargegeben. \bibverse{21} Ich werfe nicht weg die Gnade GOttes; denn so
durch das Gesetz die Gerechtigkeit kommt, so ist Christus vergeblich
gestorben.

\hypertarget{section-2}{%
\section{3}\label{section-2}}

\bibverse{1} O ihr unverständigen Galater, wer hat euch bezaubert, daß
ihr der Wahrheit nicht gehorchet? welchen Christus JEsus vor die Augen
gemalet war, und jetzt unter euch gekreuziget ist! \bibverse{2} Das will
ich allein von euch lernen: Habt ihr den Geist empfangen durch des
Gesetzes Werke oder durch die Predigt vom Glauben? \bibverse{3} Seid ihr
so unverständig? Im Geist habt ihr angefangen, wollt ihr's denn nun im
Fleisch vollenden? \bibverse{4} Habt ihr denn so viel umsonst erlitten?
Ist's anders umsonst. \bibverse{5} Der euch nun den Geist reichet und
tut solche Taten unter euch, tut er's durch des Gesetzes Werke oder
durch die Predigt vom Glauben? \bibverse{6} Gleichwie Abraham hat GOtt
geglaubet, und ist ihm gerechnet zur Gerechtigkeit. \bibverse{7} So
erkennet ihr ja nun, daß, die des Glaubens sind, das sind Abrahams
Kinder. \bibverse{8} Die Schrift aber hat es zuvor ersehen, daß GOtt die
Heiden durch den Glauben gerecht macht. Darum verkündigte sie dem
Abraham: In dir sollen alle Heiden gesegnet werden. \bibverse{9} Also
werden nun, die des Glaubens sind, gesegnet mit dem gläubigen Abraham.
\bibverse{10} Denn die mit des Gesetzes Werken umgehen, die sind unter
dem Fluch; denn es stehet geschrieben: Verflucht sei jedermann, der
nicht bleibet in alledem, das geschrieben stehet in dem Buch des
Gesetzes, daß er's tue! \bibverse{11} Daß aber durchs Gesetz niemand
gerecht wird vor GOtt, ist offenbar; denn der Gerechte wird seines
Glaubens leben. \bibverse{12} Das Gesetz aber ist nicht des Glaubens,
sondern der Mensch, der es tut, wird dadurch leben. \bibverse{13}
Christus aber hat uns erlöset von dem Fluch des Gesetzes, da er ward ein
Fluch für uns (denn es stehet geschrieben: Verflucht sei jedermann, der
am Holz hänget!), \bibverse{14} auf daß der Segen Abrahams unter die
Heiden käme in Christo JEsu, und wir also den verheißenen Geist
empfingen durch den Glauben. \bibverse{15} Liebe Brüder, ich will nach
menschlicher Weise reden: Verachtet man doch eines Menschen Testament
nicht, wenn es bestätiget ist, und tut auch nichts dazu. \bibverse{16}
Nun ist je die Verheißung Abraham und seinem Samen zugesagt. Er spricht
nicht: durch die Samen, als durch viele, sondern als durch einen, durch
deinen Samen, welcher ist Christus. \bibverse{17} Ich sage aber davon:
Das Testament, das von GOtt zuvor bestätiget ist auf Christum, wird
nicht aufgehoben, daß die Verheißung sollte durchs Gesetz aufhören,
welches gegeben ist über vierhundertunddreißig Jahre hernach.
\bibverse{18} Denn so das Erbe durch das Gesetz erworben würde, so würde
es nicht durch Verheißung gegeben. GOtt aber hat es Abraham durch
Verheißung frei geschenkt. \bibverse{19} Was soll denn das Gesetz? Es
ist dazukommen um der Sünde willen, bis der Same käme, dem die
Verheißung geschehen ist, und ist gestellet von den Engeln durch die
Hand des Mittlers. \bibverse{20} Ein Mittler aber ist nicht eines
einigen Mittler; GOtt aber ist einig. \bibverse{21} Wie? Ist denn das
Gesetz wider GOttes Verheißungen? Das sei ferne! Wenn aber ein Gesetz
gegeben wäre, das da könnte lebendig machen, so käme die Gerechtigkeit
wahrhaftig aus dem Gesetze. \bibverse{22} Aber die Schrift hat es alles
beschlossen unter die Sünde, auf daß die Verheißung käme durch den
Glauben an JEsum Christum, gegeben denen, die da glauben. \bibverse{23}
Ehe denn aber der Glaube kam, wurden wir unter dem Gesetz verwahret und
verschlossen auf den Glauben, der da sollte offenbart werden.
\bibverse{24} Also ist das Gesetz unser Zuchtmeister gewesen auf
Christum, daß wir durch den Glauben gerecht würden. \bibverse{25} Nun
aber der Glaube kommen ist, sind wir nicht mehr unter dem Zuchtmeister.
\bibverse{26} Denn ihr seid alle GOttes Kinder durch den Glauben an
Christum JEsum. \bibverse{27} Denn wieviel euer getauft sind, die haben
Christum angezogen. \bibverse{28} Hier ist kein Jude noch Grieche, hier
ist kein Knecht noch Freier, hier ist kein Mann noch Weib; denn ihr seid
allzumal einer in Christo JEsu. \bibverse{29} Seid ihr aber Christi, so
seid ihr ja Abrahams Samen und nach der Verheißung Erben.

\hypertarget{section-3}{%
\section{4}\label{section-3}}

\bibverse{1} Ich sage aber, solange der Erbe ein Kind ist, so ist unter
ihm und einem Knechte kein Unterschied, ob er wohl ein Herr ist aller
Güter, \bibverse{2} sondern er ist unter den Vormündern und Pflegern bis
auf die bestimmte Zeit vom Vater. \bibverse{3} Also auch wir, da wir
Kinder waren, waren wir gefangen unter den äußerlichen Satzungen.
\bibverse{4} Da aber die Zeit erfüllet ward, sandte GOtt seinen Sohn,
geboren von einem Weibe und unter das Gesetz getan, \bibverse{5} auf daß
er die, so unter dem Gesetz waren, erlösete, daß wir die Kindschaft
empfingen. \bibverse{6} Weil ihr denn Kinder seid, hat GOtt gesandt den
Geist seines Sohnes in eure Herzen, der schreiet: Abba, lieber Vater!
\bibverse{7} Also ist nun hier kein Knecht mehr, sondern eitel Kinder.
Sind's aber Kinder, so sind's auch Erben GOttes durch Christum.
\bibverse{8} Aber zu der Zeit, da ihr GOtt nicht erkanntet, dientet ihr
denen, die von Natur nicht Götter sind. \bibverse{9} Nun ihr aber GOtt
erkannt habt, ja vielmehr von GOtt erkannt seid, wie wendet ihr euch
denn um wieder zu den schwachen und dürftigen Satzungen, welchen ihr von
neuem an dienen wollt? \bibverse{10} Ihr haltet Tage und Monden und
Feste und Jahrzeiten. \bibverse{11} Ich fürchte für euch, daß ich nicht
vielleicht umsonst habe an euch gearbeitet. \bibverse{12} Seid doch wie
ich, denn ich bin wie ihr. Liebe Brüder, ich bitte euch, ihr habt mir
kein Leid getan. \bibverse{13} Denn ihr wisset, daß ich euch in
Schwachheit nach dem Fleisch das Evangelium geprediget habe zum
erstenmal. \bibverse{14} Und meine Anfechtungen, die ich leide nach dem
Fleisch, habt ihr nicht verachtet noch verschmähet, sondern als einen
Engel GOttes nahmet ihr mich auf, ja als Christum JEsum. \bibverse{15}
Wie waret ihr dazumal so selig! Ich bin euer Zeuge, daß, wenn es möglich
gewesen wäre, ihr hättet eure Augen ausgerissen und mir gegeben.
\bibverse{16} Bin ich denn also euer Feind worden, daß ich euch die
Wahrheit vorhalte? \bibverse{17} Sie eifern um euch nicht fein, sondern
sie wollen euch von mir abfällig machen, daß ihr um sie sollt eifern.
\bibverse{18} Eifern ist gut, wenn's immerdar geschiehet um das Gute und
nicht allein, wenn ich gegenwärtig bei euch bin. \bibverse{19} Meine
lieben Kinder, welche ich abermal mit Ängsten gebäre, bis daß Christus
in euch eine Gestalt gewinne. \bibverse{20} Ich wollte aber, daß ich
jetzt bei euch wäre, und meine Stimme wandeln könnte, denn ich bin irre
an euch. \bibverse{21} Saget mir, die ihr unter dem Gesetz sein wollt:
Habt ihr das Gesetz nicht gehöret? \bibverse{22} Denn es stehet
geschrieben, daß Abraham zween Söhne hatte, einen von der Magd, den
andern von der Freien. \bibverse{23} Aber der von der Magd war, ist nach
dem Fleisch geboren; der aber von der Freien, ist durch die Verheißung
geboren. \bibverse{24} Die Worte bedeuten etwas. Denn das sind die zwei
Testamente, eines von dem Berge Sinai, das zur Knechtschaft gebieret,
welches ist die Hagar. \bibverse{25} Denn Hagar heißet in Arabien der
Berg Sinai und langet bis gen Jerusalem, das zu dieser Zeit ist, und ist
dienstbar mit seinen Kindern. \bibverse{26} Aber das Jerusalem, das
droben ist, das ist die Freie, die ist unser aller Mutter. \bibverse{27}
Denn es stehet geschrieben: Sei fröhlich, du Unfruchtbare, die du nicht
gebierest, und brich hervor und rufe, die du nicht schwanger bist! Denn
die Einsame hat viel mehr Kinder, denn die den Mann hat. \bibverse{28}
Wir aber, liebe Brüder, sind Isaak nach der Verheißung Kinder.
\bibverse{29} Aber gleichwie zu der Zeit, der nach dem Fleisch geboren
war, verfolgete den, der nach dem Geist geboren war, also gehet es jetzt
auch. \bibverse{30} Aber was spricht die Schrift? Stoß die Magd hinaus
mit ihrem Sohn! Denn der Magd Sohn soll nicht erben mit dem Sohn der
Freien. \bibverse{31} So sind wir nun, liebe Brüder, nicht der Magd
Kinder, sondern der Freien.

\hypertarget{section-4}{%
\section{5}\label{section-4}}

\bibverse{1} So bestehet nun in der Freiheit, damit uns Christus
befreiet hat, und lasset euch nicht wiederum in das knechtische Joch
fangen! \bibverse{2} Siehe, ich, Paulus, sage euch: Wo ihr euch
beschneiden lasset, so ist euch Christus kein nütze. \bibverse{3} Ich
zeuge abermal einem jeden, der sich beschneiden läßt, daß er noch das
ganze Gesetz schuldig ist zu tun. \bibverse{4} Ihr habt Christum
verloren, die ihr durch das Gesetz gerecht werden wollt, und seid von
der Gnade gefallen. \bibverse{5} Wir aber warten im Geist durch den
Glauben der Gerechtigkeit, der man hoffen muß. \bibverse{6} Denn in
Christo JEsu gilt weder Beschneidung noch Vorhaut etwas, sondern der
Glaube, der durch die Liebe tätig ist. \bibverse{7} Ihr liefet fein. Wer
hat euch aufgehalten, der Wahrheit nicht zu gehorchen? \bibverse{8}
Solch Überreden ist nicht von dem, der euch berufen hat. \bibverse{9}
Ein wenig Sauerteig versäuert den ganzen Teig. \bibverse{10} Ich versehe
mich zu euch in dem HErrn, ihr werdet nicht anders gesinnet sein. Wer
euch aber irremacht, der wird sein Urteil tragen, er sei, wer er wolle.
\bibverse{11} Ich aber, liebe Brüder, so ich die Beschneidung noch
predige, warum leide ich denn Verfolgung? So hätte das Ärgernis des
Kreuzes aufgehöret. \bibverse{12} Wollte GOtt, daß sie auch ausgerottet
würden, die euch verstören! \bibverse{13} Ihr aber, liebe Brüder, seid
zur Freiheit berufen Allein sehet zu, daß ihr durch die Freiheit dem
Fleisch nicht Raum gebet, sondern durch die Liebe diene einer dem
andern. \bibverse{14} Denn alle Gesetze werden in einem Wort erfüllet,
in dem: Liebe deinen Nächsten wie dich selbst. \bibverse{15} So ihr euch
aber untereinander beißet und fresset, so sehet zu, daß ihr nicht
untereinander verzehret werdet. \bibverse{16} Ich sage aber: Wandelt im
Geist, so werdet ihr die Lüste des Fleisches nicht vollbringen.
\bibverse{17} Denn das Fleisch gelüstet wider den Geist und den Geist
wider das Fleisch. Dieselbigen sind wider einander, daß ihr nicht tut,
was ihr wollt. \bibverse{18} Regieret euch aber der Geist, so seid ihr
nicht unter dem Gesetze. \bibverse{19} Offenbar sind aber die Werke des
Fleisches, als da sind: Ehebruch, Hurerei, Unreinigkeit, Unzucht,
\bibverse{20} Abgötterei, Zauberei, Feindschaft, Hader, Neid, Zorn,
Zank, Zwietracht, Rotten, Haß, Mord, \bibverse{21} Saufen, Fressen und
dergleichen; von welchen ich euch habe zuvor gesagt und sage noch zuvor,
daß, die solches tun, werden das Reich GOttes nicht erben. \bibverse{22}
Die Frucht aber des Geistes ist: Liebe, Freude, Friede, Geduld,
Freundlichkeit, Gütigkeit, Glaube, Sanftmut, Keuschheit. \bibverse{23}
Wider solche ist das Gesetz nicht. \bibverse{24} Welche aber Christo
angehören, die kreuzigen ihr Fleisch samt den Lüsten und Begierden.
\bibverse{25} So wir im Geist leben, so lasset uns auch im Geist
wandeln. \bibverse{26} Lasset uns nicht eitler Ehre geizig sein,
untereinander zu entrüsten und zu hassen!

\hypertarget{section-5}{%
\section{6}\label{section-5}}

\bibverse{1} Liebe Brüder, so ein Mensch etwa von einem Fehl übereilet
würde, so helfet ihm wieder zurecht mit sanftmütigem Geist, die ihr
geistlich seid. Und siehe auf dich selbst, daß du nicht auch versucht
werdest! \bibverse{2} Einer trage des andern Last, so werdet ihr das
Gesetz Christi erfüllen. \bibverse{3} So aber sich jemand lässet dünken,
er sei etwas, so er doch nichts ist, der betrüget sich selbst.
\bibverse{4} Ein jeglicher aber prüfe sein selbst Werk, und alsdann wird
er an sich selber Ruhm haben und nicht an einem andern. \bibverse{5}
Denn ein jeglicher wird seine Last tragen. \bibverse{6} Der aber
unterrichtet wird mit dem Wort, der teile mit allerlei Gutes dem, der
ihn unterrichtet. \bibverse{7} Irret euch nicht; GOtt läßt sich nicht
spotten! Denn was der Mensch säet, das wird er ernten. \bibverse{8} Wer
auf sein Fleisch säet, der wird von dem Fleisch das Verderben ernten;
wer aber auf den Geist säet, der wird von dem Geist das ewige Leben
ernten. \bibverse{9} Lasset uns aber Gutes tun und nicht müde werden;
denn zu seiner Zeit werden wir auch ernten ohne Aufhören. \bibverse{10}
Als wir denn nun Zeit haben, so lasset uns Gutes tun an jedermann,
allermeist aber an des Glaubens Genossen. \bibverse{11} Sehet, mit wie
vielen Worten hab' ich euch geschrieben mit eigener Hand! \bibverse{12}
Die sich wollen angenehm machen nach dem Fleisch, die zwingen euch zu
beschneiden, allein daß sie nicht mit dem Kreuz Christi verfolget
werden. \bibverse{13} Denn auch sie selbst, die sich beschneiden lassen,
halten das Gesetz nicht, sondern sie wollen, daß ihr euch beschneiden
lasset, auf daß sie sich von eurem Fleisch rühmen mögen. \bibverse{14}
Es sei aber ferne von mir rühmen denn allein von dem Kreuz unsers HErrn
JEsu Christi, durch welchen mir die Welt gekreuziget ist und ich der
Welt. \bibverse{15} Denn in Christo JEsu gilt weder Beschneidung noch
Vorhaut etwas, sondern eine neue Kreatur. \bibverse{16} Und wieviel nach
dieser Regel einhergehen, über die sei Friede und Barmherzigkeit und
über den Israel GOttes! \bibverse{17} Hinfort mache mir niemand weiter
Mühe; denn ich trage die Malzeichen des HErrn JEsu an meinem Leibe.
\bibverse{18} Die Gnade unsers HErrn JEsu Christi sei mit eurem Geist,
liebe Brüder! Amen.
