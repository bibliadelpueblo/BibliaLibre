\hypertarget{section}{%
\section{1}\label{section}}

\bibleverse{1} Paulus, ein Apostel (nicht von Menschen, auch nicht durch
Menschen, sondern durch Jesum Christum und Gott, den Vater, der ihn
auferweckt hat von den Toten), \footnote{\textbf{1:1} Gal 1,11-12}
\bibleverse{2} und alle Brüder, die bei mir sind, den Gemeinden in
Galatien: \bibleverse{3} Gnade sei mit euch und Friede von Gott, dem
Vater, und unserem Herrn Jesus Christus, \bibleverse{4} der sich selbst
für unsere Sünden gegeben hat, dass er uns errettete von dieser
gegenwärtigen argen Welt nach dem Willen Gottes und unseres Vaters,
\bibleverse{5} welchem sei Ehre von Ewigkeit zu Ewigkeit! Amen.

\bibleverse{6} Mich wundert, dass ihr euch so bald abwenden lasset von
dem, der euch berufen hat in die Gnade Christi, zu einem anderen
Evangelium, \bibleverse{7} so doch kein anderes ist, außer, dass etliche
sind, die euch verwirren und wollen das Evangelium Christi verkehren.
\footnote{\textbf{1:7} Apg 15,1; Apg 15,24} \bibleverse{8} Aber so auch
wir oder ein Engel vom Himmel euch würde Evangelium predigen anders,
denn das wir euch gepredigt haben, der sei verflucht! \bibleverse{9} Wie
wir jetzt gesagt haben, so sagen wir auch abermals: Wenn jemand euch
Evangelium predigt anders, denn das ihr empfangen habt, der sei
verflucht!

\bibleverse{10} Predige ich denn jetzt Menschen oder Gott zu Dienst?
Oder gedenke ich, Menschen gefällig zu sein? Wenn ich den Menschen noch
gefällig wäre, so wäre ich Christi Knecht nicht. \footnote{\textbf{1:10}
  1Thes 2,4-6}

\bibleverse{11} Ich tue euch aber kund, liebe Brüder, dass das
Evangelium, das von mir gepredigt ist, nicht menschlich ist.
\bibleverse{12} Denn ich habe es von keinem Menschen empfangen noch
gelernt, sondern durch die Offenbarung Jesu Christi. \bibleverse{13}
Denn ihr habt ja wohl gehört meinen Wandel vordem im Judentum, wie ich
über die Maßen die Gemeinde Gottes verfolgte und sie verstörte
\bibleverse{14} und nahm zu im Judentum über viele meinesgleichen in
meinem Geschlecht und eiferte über die Maßen um das väterliche Gesetz.
\footnote{\textbf{1:14} Apg 23,6; Apg 26,5} \bibleverse{15} Da es aber
Gott wohl gefiel, der mich von meiner Mutter Leibe an hat ausgesondert
und berufen durch seine Gnade, \footnote{\textbf{1:15} Röm 1,1; Jer 1,5}
\bibleverse{16} dass er seinen Sohn offenbarte in mir, dass ich ihn
durchs Evangelium verkündigen sollte unter den Heiden: alsobald fuhr ich
zu und besprach mich nicht darüber mit Fleisch und Blut, \footnote{\textbf{1:16}
  Mt 16,17; Gal 2,7} \bibleverse{17} kam auch nicht gen Jerusalem zu
denen, die vor mir Apostel waren, sondern zog hin nach Arabien und kam
wiederum gen Damaskus.

\bibleverse{18} Darnach über drei Jahre kam ich gen Jerusalem, Petrus zu
schauen, und blieb fünfzehn Tage bei ihm. \bibleverse{19} Der anderen
Apostel aber sah ich keinen außer Jakobus, des Herrn Bruder. \footnote{\textbf{1:19}
  Mt 13,55} \bibleverse{20} Was ich euch aber schreibe, siehe, Gott
weiß, ich lüge nicht! \bibleverse{21} Darnach kam ich in die Länder
Syrien und Zilizien. \bibleverse{22} Ich war aber unbekannt von
Angesicht den christlichen Gemeinden in Judäa. \bibleverse{23} Sie
hatten aber allein gehört, dass, der uns vordem verfolgte, der predigt
jetzt den Glauben, welchen er vordem verstörte, \bibleverse{24} und
priesen Gott über mir. \# 2 \bibleverse{1} Darnach über vierzehn Jahre
zog ich abermals hinauf gen Jerusalem mit Barnabas und nahm Titus auch
mit mir. \footnote{\textbf{2:1} Apg 15,1-29; Apg 4,36} \bibleverse{2}
Ich zog aber hinauf aus einer Offenbarung und besprach mich mit ihnen
über das Evangelium, das ich predige unter den Heiden, besonders aber
mit denen, die das Ansehen hatten, auf dass ich nicht vergeblich liefe
oder gelaufen wäre. \bibleverse{3} Aber es ward auch Titus nicht
gezwungen, sich beschneiden zu lassen, der mit mir war, obwohl er ein
Grieche war. \bibleverse{4} Denn da etliche falsche Brüder sich mit
eingedrängt hatten und neben eingeschlichen waren, auszukundschaften
unsere Freiheit, die wir haben in Christo Jesu, dass sie uns
gefangennähmen, \bibleverse{5} wichen wir denselben nicht eine Stunde,
ihnen untertan zu sein, auf dass die Wahrheit des Evangeliums bei euch
bestünde. \footnote{\textbf{2:5} Gal 3,1} \bibleverse{6} Von denen aber,
die das Ansehen hatten -- welcherlei sie vordem gewesen sind, daran
liegt mir nichts; denn Gott achtet das Ansehen der Menschen nicht --,
mich haben die, die das Ansehen hatten, nichts anderes gelehrt;
\footnote{\textbf{2:6} 2Kor 11,5; 2Kor 11,23} \bibleverse{7} sondern
dagegen, da sie sahen, dass mir vertraut war das Evangelium an die
Heiden, gleichwie dem Petrus das Evangelium an die Juden \footnote{\textbf{2:7}
  Eph 3,1-2} \bibleverse{8} (denn der mit Petrus kräftig gewesen ist zum
Apostelamt unter den Juden, der ist mit mir auch kräftig gewesen unter
den Heiden), \bibleverse{9} und da sie erkannten die Gnade, die mir
gegeben war, Jakobus und Kephas und Johannes, die für Säulen angesehen
waren, gaben sie mir und Barnabas die rechte Hand und wurden mit uns
eins, dass wir unter die Heiden, sie aber unter die Juden gingen,
\bibleverse{10} allein dass wir der Armen gedächten, welches ich auch
fleißig bin gewesen zu tun. \footnote{\textbf{2:10} Apg 11,29-30; 2Kor
  8,9}

\bibleverse{11} Da aber Petrus gen Antiochien kam, widerstand ich ihm
unter Augen; denn es war Klage über ihn gekommen. \bibleverse{12} Denn
zuvor, ehe etliche von Jakobus kamen, aß er mit den Heiden; da sie aber
kamen, entzog er sich und sonderte sich ab, darum dass er die aus den
Juden fürchtete. \bibleverse{13} Und mit ihm heuchelten die anderen
Juden, also dass auch Barnabas verführt ward, mit ihnen zu heucheln.
\bibleverse{14} Aber da ich sah, dass sie nicht richtig wandelten nach
der Wahrheit des Evangeliums, sprach ich zu Petrus vor allen öffentlich:
So du, der du ein Jude bist, heidnisch lebst und nicht jüdisch, warum
zwingst du denn die Heiden, jüdisch zu leben?

\bibleverse{15} Wir sind von Natur Juden und nicht Sünder aus den
Heiden; \bibleverse{16} doch weil wir wissen, dass der Mensch durch des
Gesetzes Werke nicht gerecht wird, sondern durch den Glauben an Jesum
Christum, so glauben wir auch an Christum Jesum, auf dass wir gerecht
werden durch den Glauben an Christum und nicht durch des Gesetzes Werke;
denn durch des Gesetzeswerke wird kein Fleisch gerecht. \footnote{\textbf{2:16}
  Röm 3,20; Röm 3,28; Eph 2,8} \bibleverse{17} Sollten wir aber, die da
suchen, durch Christum gerecht zu werden, auch selbst als Sünder
erfunden werden, so wäre Christus ja ein Sündendiener. Das sei ferne!
\bibleverse{18} Wenn ich aber das, was ich zerbrochen habe, wiederum
baue, so mache ich mich selbst zu einem Übertreter. \bibleverse{19} Ich
bin aber durchs Gesetz dem Gesetz gestorben, auf dass ich Gott lebe; ich
bin mit Christo gekreuzigt. \bibleverse{20} Ich lebe aber; doch nun
nicht ich, sondern Christus lebt in mir. Denn was ich jetzt lebe im
Fleisch, das lebe ich in dem Glauben des Sohnes Gottes, der mich geliebt
hat und sich selbst für mich dargegeben. \footnote{\textbf{2:20} Gal
  1,4; Joh 17,23} \bibleverse{21} Ich werfe nicht weg die Gnade Gottes;
denn wenn durch das Gesetz die Gerechtigkeit kommt, so ist Christus
vergeblich gestorben. \# 3 \bibleverse{1} O ihr unverständigen Galater,
wer hat euch bezaubert, dass ihr der Wahrheit nicht gehorchet, welchen
Christus Jesus vor die Augen gemalt war, als wäre er unter euch
gekreuzigt? \bibleverse{2} Das will ich allein von euch lernen: Habt ihr
den Geist empfangen durch des Gesetzes Werke oder durch die Predigt vom
Glauben? \bibleverse{3} Seid ihr so unverständig? Im Geist habt ihr
angefangen, wollt ihr's denn nun im Fleisch vollenden? \bibleverse{4}
Habt ihr denn so viel umsonst erlitten? Ist's anders umsonst!
\bibleverse{5} Der euch nun den Geist reicht und tut solche Taten unter
euch, tut er's durch des Gesetzes Werke oder durch die Predigt vom
Glauben? \bibleverse{6} Gleichwie Abraham hat Gott geglaubt und es ist
ihm gerechnet zur Gerechtigkeit. \bibleverse{7} So erkennet ihr ja,
dass, die des Glaubens sind, das sind Abrahams Kinder. \bibleverse{8}
Die Schrift aber hat es zuvor gesehen, dass Gott die Heiden durch den
Glauben gerecht macht; darum verkündigte sie dem Abraham: „In dir sollen
alle Heiden gesegnet werden.`` \bibleverse{9} Also werden nun, die des
Glaubens sind, gesegnet mit dem gläubigen Abraham. \footnote{\textbf{3:9}
  Röm 5,20; Apg 7,38; Apg 7,53; 5Mo 5,5; Hebr 2,2}

\bibleverse{10} Denn die mit des Gesetzes Werken umgehen, die sind unter
dem Fluch. Denn es steht geschrieben: „Verflucht sei jedermann, der
nicht bleibt in alle dem, was geschrieben steht in dem Buch des
Gesetzes, dass er's tue.`` \bibleverse{11} dass aber durchs Gesetz
niemand gerecht wird vor Gott, ist offenbar; denn „der Gerechte wird
seines Glaubens leben.`` \bibleverse{12} Das Gesetz aber ist nicht des
Glaubens; sondern „der Mensch, der es tut, wird dadurch leben.``

\bibleverse{13} Christus aber hat uns erlöst von dem Fluch des Gesetzes,
da er ward ein Fluch für uns (denn es steht geschrieben: „Verflucht ist
jedermann, der am Holz hängt!{}``), \bibleverse{14} auf dass der Segen
Abrahams unter die Heiden käme in Christo Jesu und wir also den
verheißenen Geist empfingen durch den Glauben.

\bibleverse{15} Liebe Brüder, ich will nach menschlicher Weise reden:
Verwirft man doch eines Menschen Testament nicht, wenn es bestätigt ist,
und tut auch nichts dazu. \bibleverse{16} Nun ist ja die Verheißung
Abraham und seinem Samen zugesagt. Er spricht nicht: „durch die Samen``,
als durch viele, sondern als durch einen: „durch deinen Samen``, welcher
ist Christus. \bibleverse{17} Ich sage aber davon: Das Testament, das
von Gott zuvor bestätigt ist auf Christum, wird nicht aufgehoben, dass
die Verheißung sollte durchs Gesetz aufhören, welches gegeben ist
vierhundertdreißig Jahre hernach. \footnote{\textbf{3:17} 2Mo 12,40}
\bibleverse{18} Denn so das Erbe durch das Gesetz erworben würde, so
würde es nicht durch Verheißung gegeben; Gott aber hat's Abraham durch
Verheißung frei geschenkt.

\bibleverse{19} Was soll denn das Gesetz? Es ist hinzugekommen um der
Sünden willen, bis der Same käme, dem die Verheißung geschehen ist, und
ist gestellt von den Engeln durch die Hand des Mittlers. \bibleverse{20}
Ein Mittler aber ist nicht eines Mittler; Gott aber ist einer.

\bibleverse{21} Wie? Ist denn das Gesetz wider Gottes Verheißungen? Das
sei ferne! Wenn aber ein Gesetz gegeben wäre, das da könnte lebendig
machen, so käme die Gerechtigkeit wahrhaftig aus dem Gesetz. \footnote{\textbf{3:21}
  Röm 8,2-4} \bibleverse{22} Aber die Schrift hat alles beschlossen
unter die Sünde, auf dass die Verheißung käme durch den Glauben an Jesum
Christum, gegeben denen, die da glauben. \footnote{\textbf{3:22} Röm
  3,9-20; Röm 11,32}

\bibleverse{23} Ehe denn aber der Glaube kam, wurden wir unter dem
Gesetz verwahrt und verschlossen auf den Glauben, der da sollte
offenbart werden. \footnote{\textbf{3:23} Gal 4,3} \bibleverse{24} Also
ist das Gesetz unser Zuchtmeister gewesen auf Christum, dass wir durch
den Glauben gerecht würden. \bibleverse{25} Nun aber der Glaube gekommen
ist, sind wir nicht mehr unter dem Zuchtmeister. \bibleverse{26} Denn
ihr seid alle Gottes Kinder durch den Glauben an Christum Jesum.
\footnote{\textbf{3:26} Joh 1,12; Röm 8,17} \bibleverse{27} Denn wieviel
euer auf Christum getauft sind, die haben Christum angezogen.
\footnote{\textbf{3:27} Röm 6,3; Röm 13,14} \bibleverse{28} Hier ist
kein Jude noch Grieche, hier ist kein Knecht noch Freier, hier ist kein
Mann noch Weib; denn ihr seid allzumal einer in Christo Jesu.
\footnote{\textbf{3:28} Röm 10,12; 1Kor 12,13} \bibleverse{29} Seid ihr
aber Christi, so seid ihr ja Abrahams Same und nach der Verheißung
Erben. \# 4 \bibleverse{1} Ich sage aber: Solange der Erbe unmündig ist,
so ist zwischen ihm und einem Knecht kein Unterschied, ob er wohl ein
Herr ist aller Güter; \bibleverse{2} sondern er ist unter den Vormündern
und Pflegern bis auf die Zeit, die der Vater bestimmt hat.
\bibleverse{3} Also auch wir, da wir unmündig waren, waren wir gefangen
unter den äußerlichen Satzungen. \bibleverse{4} Da aber die Zeit
erfüllet ward, sandte Gott seinen Sohn, geboren von einem Weibe und
unter das Gesetz getan, \footnote{\textbf{4:4} Mk 1,15; Eph 1,10}
\bibleverse{5} auf dass er die, die unter dem Gesetz waren, erlöste,
dass wir die Kindschaft empfingen. \bibleverse{6} Weil ihr denn Kinder
seid, hat Gott gesandt den Geist seines Sohnes in eure Herzen, der
schreit: Abba, lieber Vater! \bibleverse{7} Also ist nun hier kein
Knecht mehr, sondern eitel Kinder; sind's aber Kinder, so sind's auch
Erben Gottes durch Christum. \footnote{\textbf{4:7} Röm 8,16-17}

\bibleverse{8} Aber zu der Zeit, da ihr Gott nicht erkanntet, dientet
ihr denen, die von Natur nicht Götter sind. \bibleverse{9} Nun ihr aber
Gott erkannt habt, ja vielmehr von Gott erkannt seid, wie wendet ihr
euch denn wiederum zu den schwachen und dürftigen Satzungen, welchen ihr
von neuem an dienen wollt? \bibleverse{10} Ihr haltet Tage und Monate
und Feste und Jahre. \bibleverse{11} Ich fürchte für euch, dass ich
vielleicht umsonst an euch gearbeitet habe. \footnote{\textbf{4:11} 2Jo
  1,-1}

\bibleverse{12} Seid doch wie ich; denn ich bin wie ihr. Liebe Brüder,
ich bitte euch. Ihr habt mir kein Leid getan. \bibleverse{13} Denn ihr
wisset, dass ich euch in Schwachheit nach dem Fleisch das Evangelium
gepredigt habe zum erstenmal. \bibleverse{14} Und meine Anfechtungen,
die ich leide nach dem Fleisch, habt ihr nicht verachtet noch
verschmäht; sondern wie ein Engel Gottes nahmet ihr mich auf, ja wie
Christum Jesum.

\bibleverse{15} Wie wart ihr dazumal so selig! ich bin euer Zeuge, dass,
wenn es möglich gewesen wäre, ihr hättet eure Augen ausgerissen und mir
gegeben. \bibleverse{16} Bin ich denn damit euer Feind geworden, dass
ich euch die Wahrheit vorhalte? \bibleverse{17} Sie eifern um euch nicht
fein; sondern sie wollen euch von mir abfällig machen, dass ihr um sie
eifern sollt. \footnote{\textbf{4:17} Gal 1,7} \bibleverse{18} Eifern
ist gut, wenn's immerdar geschieht um das Gute, und nicht allein, wenn
ich gegenwärtig bei euch bin.

\bibleverse{19} Meine lieben Kinder, welche ich abermals mit Ängsten
gebäre, bis dass Christus in euch eine Gestalt gewinne, \bibleverse{20}
ich wollte, dass ich jetzt bei euch wäre und meine Stimme wandeln
könnte; denn ich bin irre an euch.

\bibleverse{21} Saget mir, die ihr unter dem Gesetz sein wollt: Habt ihr
das Gesetz nicht gehört? \bibleverse{22} Denn es steht geschrieben, dass
Abraham zwei Söhne hatte: einen von der Magd, den anderen von der
Freien. \footnote{\textbf{4:22} 1Mo 16,15; 1Mo 21,2} \bibleverse{23}
Aber der von der Magd war, ist nach dem Fleisch geboren; der aber von
der Freien ist durch die Verheißung geboren. \footnote{\textbf{4:23} Röm
  9,7-9} \bibleverse{24} Die Worte bedeuten etwas. Denn das sind zwei
Testamente: eins von dem Berge Sinai, dass zur Knechtschaft gebiert,
welches ist die Hagar; \footnote{\textbf{4:24} Gal 5,1; Röm 8,15}
\bibleverse{25} denn Hagar heißt in Arabien der Berg Sinai und kommt
überein mit Jerusalem, das zu dieser Zeit ist und dienstbar ist mit
seinen Kindern. \bibleverse{26} Aber das Jerusalem, das droben ist, das
ist die Freie; die ist unser aller Mutter. \bibleverse{27} Denn es steht
geschrieben: „Sei fröhlich, du Unfruchtbare, die du nicht gebierst! Und
brich hervor und rufe, die du nicht schwanger bist! Denn die Einsame hat
viel mehr Kinder, denn die den Mann hat.``

\bibleverse{28} Wir aber, liebe Brüder, sind, Isaak nach, der Verheißung
Kinder. \bibleverse{29} Aber gleichwie zu der Zeit, der nach dem Fleisch
geboren war, verfolgte den, der nach dem Geist geboren war, also geht es
auch jetzt. \footnote{\textbf{4:29} 1Mo 21,9} \bibleverse{30} Aber was
spricht die Schrift? „Stoß die Magd hinaus mit ihrem Sohn; denn der Magd
Sohn soll nicht erben mit dem Sohn der Freien.`` \bibleverse{31} So sind
wir nun, liebe Brüder, nicht der Magd Kinder, sondern der Freien. \# 5
\bibleverse{1} So bestehet nun in der Freiheit, zu der uns Christus
befreit hat, und lasset euch nicht wiederum in das knechtische Joch
fangen.

\bibleverse{2} Siehe, ich, Paulus, sage euch: Wo ihr euch beschneiden
lasset, so nützt euch Christus nichts. \bibleverse{3} Ich bezeuge
abermals einem jeden, der sich beschneiden lässt, dass er das ganze
Gesetz schuldig ist zu tun. \bibleverse{4} Ihr habt Christum verloren,
die ihr durch das Gesetz gerecht werden wollt, und seid von der Gnade
gefallen. \bibleverse{5} Wir aber warten im Geist durch den Glauben der
Gerechtigkeit, auf die man hoffen muss. \bibleverse{6} Denn in Christo
Jesu gilt weder Beschneidung noch unbeschnitten sein etwas, sondern der
Glaube, der durch die Liebe tätig ist. \footnote{\textbf{5:6} Gal 6,15;
  Röm 2,26; 1Kor 7,19}

\bibleverse{7} Ihr liefet fein. Wer hat euch aufgehalten, der Wahrheit
nicht zu gehorchen? \bibleverse{8} Solch Überreden ist nicht von dem,
der euch berufen hat. \bibleverse{9} Ein wenig Sauerteig versäuert den
ganzen Teig. \bibleverse{10} Ich versehe mich zu euch in dem Herrn, ihr
werdet nicht anders gesinnt sein. Wer euch aber irremacht, der wird sein
Urteil tragen, er sei, wer er wolle. \footnote{\textbf{5:10} Gal 1,7}

\bibleverse{11} Ich aber, liebe Brüder, wenn ich die Beschneidung noch
predige, warum leide ich denn Verfolgung? So hätte ja das Ärgernis des
Kreuzes aufgehört. \footnote{\textbf{5:11} Gal 6,12; 1Kor 1,23; 1Kor
  15,30} \bibleverse{12} Wollte Gott, dass sie auch ausgerottet würden,
die euch verstören!

\bibleverse{13} Ihr aber, liebe Brüder, seid zur Freiheit berufen!
Allein sehet zu, dass ihr durch die Freiheit dem Fleisch nicht Raum
gebet; sondern durch die Liebe diene einer dem anderen. \footnote{\textbf{5:13}
  1Petr 2,16; 2Petr 2,19} \bibleverse{14} Denn alle Gesetze werden in
einem Wort erfüllt, in dem: „Liebe deinen Nächsten wie dich selbst.``
\bibleverse{15} So ihr euch aber untereinander beißet und fresset, so
seht zu, dass ihr nicht untereinander verzehrt werdet.

\bibleverse{16} Ich sage aber: Wandelt im Geist, so werdet ihr die Lüste
des Fleisches nicht vollbringen. \bibleverse{17} Denn das Fleisch
gelüstet wider den Geist, und der Geist wider das Fleisch; dieselben
sind widereinander, dass ihr nicht tut, was ihr wollt. \footnote{\textbf{5:17}
  Röm 7,15; Röm 7,23} \bibleverse{18} Regiert euch aber der Geist, so
seid ihr nicht unter dem Gesetz. \bibleverse{19} Offenbar sind aber die
Werke des Fleisches, als da sind: Ehebruch, Hurerei, Unreinigkeit,
Unzucht, \bibleverse{20} Abgötterei, Zauberei, Feindschaft, Hader, Neid,
Zorn, Zank, Zwietracht, Rotten, Hass, Mord, \bibleverse{21} Saufen,
Fressen und dergleichen, von welchen ich euch zuvor gesagt und sage noch
zuvor, dass, die solches tun, werden das Reich Gottes nicht erben.
\footnote{\textbf{5:21} Eph 5,5; Offb 22,15}

\bibleverse{22} Die Frucht aber des Geistes ist Liebe, Freude, Friede,
Geduld, Freundlichkeit, Gütigkeit, Glaube, Sanftmut, Keuschheit.
\footnote{\textbf{5:22} Eph 5,9} \bibleverse{23} Wider solche ist das
Gesetz nicht. \footnote{\textbf{5:23} 1Tim 1,9} \bibleverse{24} Welche
aber Christo angehören, die kreuzigen ihr Fleisch samt den Lüsten und
Begierden. \footnote{\textbf{5:24} Röm 6,6}

\bibleverse{25} So wir im Geist leben, so lasset uns auch im Geist
wandeln. \footnote{\textbf{5:25} Röm 8,4} \bibleverse{26} Lasset uns
nicht eitler Ehre geizig sein, einander zu entrüsten und zu hassen.
\footnote{\textbf{5:26} Phil 2,3}

\hypertarget{section-1}{%
\section{6}\label{section-1}}

\bibleverse{1} Liebe Brüder, so ein Mensch etwa von einem Fehler
übereilt würde, so helfet ihm wieder zurecht mit sanftmütigem Geist ihr,
die ihr geistlich seid; und sieh auf dich selbst, dass du nicht auch
versucht werdest. \footnote{\textbf{6:1} Mt 18,15; Jak 5,19}
\bibleverse{2} Einer trage des anderen Last, so werdet ihr das Gesetz
Christi erfüllen. \footnote{\textbf{6:2} 2Kor 11,29} \bibleverse{3} So
aber jemand sich lässt dünken, er sei etwas, wenn er doch nichts ist,
der betrügt sich selbst. \bibleverse{4} Ein jeglicher aber prüfe sein
eigen Werk; und alsdann wird er an sich selber Ruhm haben und nicht an
einem anderen. \footnote{\textbf{6:4} 2Kor 13,5} \bibleverse{5} Denn ein
jeglicher wird seine Last tragen. \footnote{\textbf{6:5} Röm 14,12}

\bibleverse{6} Der aber unterrichtet wird mit dem Wort, der teile mit
allerlei Gutes dem, der ihn unterrichtet. \footnote{\textbf{6:6} 1Kor
  9,14}

\bibleverse{7} Irret euch nicht! Gott lässt sich nicht spotten. Denn was
der Mensch sät, das wird er ernten. \bibleverse{8} Wer auf sein Fleisch
sät, der wird von dem Fleisch das Verderben ernten; wer aber auf den
Geist sät, der wird von dem Geist das ewige Leben ernten. \bibleverse{9}
Lasset uns aber Gutes tun und nicht müde werden; denn zu seiner Zeit
werden wir auch ernten ohne Aufhören. \footnote{\textbf{6:9} 2Thes 3,13}
\bibleverse{10} Als wir denn nun Zeit haben, so lasset uns Gutes tun an
jedermann, allermeist aber an des Glaubens Genossen. \footnote{\textbf{6:10}
  2Petr 1,7}

\bibleverse{11} Sehet, mit wie vielen Worten habe ich euch geschrieben
mit eigener Hand! \bibleverse{12} Die sich wollen angenehm machen nach
dem Fleisch, die zwingen euch zur Beschneidung, nur damit sie nicht mit
dem Kreuz Christi verfolgt werden. \footnote{\textbf{6:12} Gal 5,11}
\bibleverse{13} Denn auch sie selbst, die sich beschneiden lassen,
halten das Gesetz nicht; sondern sie wollen, dass ihr euch beschneiden
lasset, auf dass sie sich von eurem Fleisch rühmen mögen.
\bibleverse{14} Es sei aber ferne von mir, mich zu rühmen, denn allein
von dem Kreuz unseres Herrn Jesu Christi, durch welchen mir die Welt
gekreuzigt ist und ich der Welt. \bibleverse{15} Denn in Christo Jesu
gilt weder Beschneidung noch unbeschnitten sein etwas, sondern eine neue
Kreatur. \footnote{\textbf{6:15} Gal 5,6; 1Kor 7,19; 2Kor 5,17}
\bibleverse{16} Und wie viele nach dieser Regel einhergehen, über die
sei Friede und Barmherzigkeit und über das Israel Gottes. \footnote{\textbf{6:16}
  Ps 125,5; Phil 3,16}

\bibleverse{17} Hinfort mache mir niemand weiter Mühe; denn ich trage
die Malzeichen des Herrn Jesu an meinem Leibe. \footnote{\textbf{6:17}
  2Kor 4,10}

\bibleverse{18} Die Gnade unseres Herrn Jesu Christi sei mit eurem
Geist, liebe Brüder! Amen.
