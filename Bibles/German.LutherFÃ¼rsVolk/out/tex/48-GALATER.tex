\hypertarget{section}{%
\section{1}\label{section}}

\bibverse{1} Paulus, ein Apostel (nicht von Menschen, auch nicht durch
Menschen, sondern durch Jesum Christum und Gott, den Vater, der ihn
auferweckt hat von den Toten), \bibverse{2} und alle Brüder, die bei mir
sind, den Gemeinden in Galatien: \bibverse{3} Gnade sei mit euch und
Friede von Gott, dem Vater, und unserm HERRN Jesus Christus,
\bibverse{4} der sich selbst für unsere Sünden gegeben hat, daß er uns
errettete von dieser gegenwärtigen, argen Welt nach dem Willen Gottes
und unseres Vaters, \bibverse{5} welchem sei Ehre von Ewigkeit zu
Ewigkeit! Amen. \bibverse{6} Mich wundert, daß ihr euch so bald abwenden
lasset von dem, der euch berufen hat in die Gnade Christi, zu einem
anderen Evangelium, \bibverse{7} so doch kein anderes ist, außer, daß
etliche sind, die euch verwirren und wollen das Evangelium Christi
verkehren. \bibverse{8} Aber so auch wir oder ein Engel vom Himmel euch
würde Evangelium predigen anders, denn das wir euch gepredigt haben, der
sei verflucht! \bibverse{9} Wie wir jetzt gesagt haben, so sagen wir
abermals: So jemand euch Evangelium predigt anders, denn das ihr
empfangen habt, der sei verflucht! \bibverse{10} Predige ich denn jetzt
Menschen oder Gott zu Dienst? Oder gedenke ich, Menschen gefällig zu
sein? Wenn ich den Menschen noch gefällig wäre, so wäre ich Christi
Knecht nicht. \bibverse{11} Ich tue euch aber kund, liebe Brüder, daß
das Evangelium, das von mir gepredigt ist, nicht menschlich ist.
\bibverse{12} Denn ich habe es von keinem Menschen empfangen noch
gelernt, sondern durch die Offenbarung Jesu Christi. \bibverse{13} Denn
ihr habt ja wohl gehört meinen Wandel weiland im Judentum, wie ich über
die Maßen die Gemeinde Gottes verfolgte und verstörte \bibverse{14} und
nahm zu im Judentum über viele meinesgleichen in meinem Geschlecht und
eiferte über die Maßen um das väterliche Gesetz. \bibverse{15} Da es
aber Gott wohl gefiel, der mich von meiner Mutter Leibe an hat
ausgesondert und berufen durch seine Gnade, \bibverse{16} daß er seinen
Sohn offenbarte in mir, daß ich ihn durchs Evangelium verkündigen sollte
unter den Heiden: alsobald fuhr ich zu und besprach mich nicht darüber
mit Fleisch und Blut, \bibverse{17} kam auch nicht gen Jerusalem zu
denen, die vor mir Apostel waren, sondern zog hin nach Arabien und kam
wiederum gen Damaskus. \bibverse{18} Darnach über drei Jahre kam ich
nach Jerusalem, Petrus zu schauen, und blieb fünfzehn Tage bei ihm.
\bibverse{19} Der andern Apostel aber sah ich keinen außer Jakobus, des
HERRN Bruder. \bibverse{20} Was ich euch aber schreibe, siehe, Gott
weiß, ich lüge nicht! \bibverse{21} Darnach kam ich in die Länder Syrien
und Zilizien. \bibverse{22} Ich war aber unbekannt von Angesicht den
christlichen Gemeinden in Judäa. \bibverse{23} Sie hatten aber allein
gehört, daß, der uns weiland verfolgte, der predigt jetzt den Glauben,
welchen er weiland verstörte, \bibverse{24} und priesen Gott über mir.

\hypertarget{section-1}{%
\section{2}\label{section-1}}

\bibverse{1} Darnach über vierzehn Jahre zog ich abermals hinauf gen
Jerusalem mit Barnabas und nahm Titus auch mit mir. \bibverse{2} Ich zog
aber hinauf aus einer Offenbarung und besprach mich mit ihnen über das
Evangelium, das ich predige unter den Heiden, besonders aber mit denen,
die das Ansehen hatten, auf daß ich nicht vergeblich liefe oder gelaufen
wäre. \bibverse{3} Aber es ward auch Titus nicht gezwungen, sich
beschneiden zu lassen, der mit mir war, obwohl er ein Grieche war.
\bibverse{4} Denn da etliche falsche Brüder sich mit eingedrängt hatten
und neben eingeschlichen waren, auszukundschaften unsre Freiheit, die
wir haben in Christo Jesu, daß sie uns gefangennähmen, \bibverse{5}
wichen wir denselben nicht eine Stunde, ihnen untertan zu sein, auf daß
die Wahrheit des Evangeliums bei euch bestünde. \bibverse{6} Von denen
aber, die das Ansehen hatten, welcherlei sie weiland gewesen sind, daran
liegt mir nichts; denn Gott achtet das Ansehen der Menschen nicht, mich
haben die, so das Ansehen hatten, nichts anderes gelehrt; \bibverse{7}
sondern dagegen, da sie sahen, daß mir vertraut war das Evangelium an
die Heiden, gleichwie dem Petrus das Evangelium an die Juden
\bibverse{8} (denn der mit Petrus kräftig gewesen ist zum Apostelamt
unter den Juden, der ist mit mir auch kräftig gewesen unter den Heiden),
\bibverse{9} und da sie erkannten die Gnade, die mir gegeben war,
Jakobus und Kephas und Johannes, die für Säulen angesehen waren, gaben
sie mir und Barnabas die rechte Hand und wurden mit uns eins, daß wir
unter die Heiden, sie aber unter die Juden gingen, \bibverse{10} allein
daß wir der Armen gedächten, welches ich auch fleißig bin gewesen zu
tun. \bibverse{11} Da aber Petrus gen Antiochien kam, widerstand ich ihm
unter Augen; denn es war Klage über ihn gekommen. \bibverse{12} Denn
zuvor, ehe etliche von Jakobus kamen, aß er mit den Heiden; da sie aber
kamen, entzog er sich und sonderte sich ab, darum daß er die aus den
Juden fürchtete. \bibverse{13} Und mit ihm heuchelten die andern Juden,
also daß auch Barnabas verführt ward, mit ihnen zu heucheln.
\bibverse{14} Aber da ich sah, daß sie nicht richtig wandelten nach der
Wahrheit des Evangeliums, sprach ich zu Petrus vor allen öffentlich: So
du, der du ein Jude bist, heidnisch lebst und nicht jüdisch, warum
zwingst du denn die Heiden, jüdisch zu leben? \bibverse{15} Wir sind von
Natur Juden und nicht Sünder aus den Heiden; \bibverse{16} doch weil wir
wissen, daß der Mensch durch des Gesetzes Werke nicht gerecht wird,
sondern durch den Glauben an Jesum Christum, so glauben wir auch an
Christum Jesum, auf daß wir gerecht werden durch den Glauben an Christum
und nicht durch des Gesetzes Werke; denn durch des Gesetzeswerke wird
kein Fleisch gerecht. \bibverse{17} Sollten wir aber, die da suchen,
durch Christum gerecht zu werden, auch selbst als Sünder erfunden
werden, so wäre Christus ja ein Sündendiener. Das sei ferne!
\bibverse{18} Wenn ich aber das, was ich zerbrochen habe, wiederum baue,
so mache ich mich selbst zu einem Übertreter. \bibverse{19} Ich bin aber
durchs Gesetz dem Gesetz gestorben, auf daß ich Gott lebe; ich bin mit
Christo gekreuzigt. \bibverse{20} Ich lebe aber; doch nun nicht ich,
sondern Christus lebt in mir. Denn was ich jetzt lebe im Fleisch, das
lebe ich in dem Glauben des Sohnes Gottes, der mich geliebt hat und sich
selbst für mich dargegeben. \bibverse{21} Ich werfe nicht weg die Gnade
Gottes; denn so durch das Gesetz die Gerechtigkeit kommt, so ist
Christus vergeblich gestorben.

\hypertarget{section-2}{%
\section{3}\label{section-2}}

\bibverse{1} O ihr unverständigen Galater, wer hat euch bezaubert, daß
ihr der Wahrheit nicht gehorchet, welchen Christus Jesus vor die Augen
gemalt war, als wäre er unter euch gekreuzigt? \bibverse{2} Das will ich
allein von euch lernen: Habt ihr den Geist empfangen durch des Gesetzes
Werke oder durch die Predigt vom Glauben? \bibverse{3} Seid ihr so
unverständig? Im Geist habt ihr angefangen, wollt ihr's denn nun im
Fleisch vollenden? \bibverse{4} Habt ihr denn so viel umsonst erlitten?
Ist's anders umsonst! \bibverse{5} Der euch nun den Geist reicht und tut
solche Taten unter euch, tut er's durch des Gesetzes Werke oder durch
die Predigt vom Glauben? \bibverse{6} Gleichwie Abraham hat Gott
geglaubt und es ist ihm gerechnet zur Gerechtigkeit. \bibverse{7} So
erkennet ihr ja, daß, die des Glaubens sind, das sind Abrahams Kinder.
\bibverse{8} Die Schrift aber hat es zuvor gesehen, daß Gott die Heiden
durch den Glauben gerecht macht; darum verkündigte sie dem Abraham: ``In
dir sollen alle Heiden gesegnet werden.'' \bibverse{9} Also werden nun,
die des Glaubens sind, gesegnet mit dem gläubigen Abraham. \bibverse{10}
Denn die mit des Gesetzes Werken umgehen, die sind unter dem Fluch. Denn
es steht geschrieben: ``Verflucht sei jedermann, der nicht bleibt in
alle dem, was geschrieben steht in dem Buch des Gesetzes, daß er's
tue.'' \bibverse{11} Daß aber durchs Gesetz niemand gerecht wird vor
Gott, ist offenbar; denn ``der Gerechte wird seines Glaubens leben.''
\bibverse{12} Das Gesetz aber ist nicht des Glaubens; sondern ``der
Mensch, der es tut, wird dadurch leben.'' \bibverse{13} Christus aber
hat uns erlöst von dem Fluch des Gesetzes, da er ward ein Fluch für uns
(denn es steht geschrieben: ``Verflucht ist jedermann, der am Holz
hängt!''), \bibverse{14} auf daß der Segen Abrahams unter die Heiden
käme in Christo Jesu und wir also den verheißenen Geist empfingen durch
den Glauben. \bibverse{15} Liebe Brüder, ich will nach menschlicher
Weise reden: Verwirft man doch eines Menschen Testament nicht, wenn es
bestätigt ist, und tut auch nichts dazu. \bibverse{16} Nun ist ja die
Verheißung Abraham und seinem Samen zugesagt. Er spricht nicht: ``durch
die Samen'', als durch viele, sondern als durch einen: ``durch deinen
Samen'', welcher ist Christus. \bibverse{17} Ich sage aber davon: Das
Testament, das von Gott zuvor bestätigt ist auf Christum, wird nicht
aufgehoben, daß die Verheißung sollte durchs Gesetz aufhören, welches
gegeben ist vierhundertdreißig Jahre hernach. \bibverse{18} Denn so das
Erbe durch das Gesetz erworben würde, so würde es nicht durch Verheißung
gegeben; Gott aber hat's Abraham durch Verheißung frei geschenkt.
\bibverse{19} Was soll denn das Gesetz? Es ist hinzugekommen um der
Sünden willen, bis der Same käme, dem die Verheißung geschehen ist, und
ist gestellt von den Engeln durch die Hand des Mittlers. \bibverse{20}
Ein Mittler aber ist nicht eines Mittler; Gott aber ist einer.
\bibverse{21} Wie? Ist denn das Gesetz wider Gottes Verheißungen? Das
sei ferne! Wenn aber ein Gesetz gegeben wäre, das da könnte lebendig
machen, so käme die Gerechtigkeit wahrhaftig aus dem Gesetz.
\bibverse{22} Aber die Schrift hat alles beschlossen unter die Sünde,
auf daß die Verheißung käme durch den Glauben an Jesum Christum, gegeben
denen, die da glauben. \bibverse{23} Ehe denn aber der Glaube kam,
wurden wir unter dem Gesetz verwahrt und verschlossen auf den Glauben,
der da sollte offenbart werden. \bibverse{24} Also ist das Gesetz unser
Zuchtmeister gewesen auf Christum, daß wir durch den Glauben gerecht
würden. \bibverse{25} Nun aber der Glaube gekommen ist, sind wir nicht
mehr unter dem Zuchtmeister. \bibverse{26} Denn ihr seid alle Gottes
Kinder durch den Glauben an Christum Jesum. \bibverse{27} Denn wieviel
euer auf Christum getauft sind, die haben Christum angezogen.
\bibverse{28} Hier ist kein Jude noch Grieche, hier ist kein Knecht noch
Freier, hier ist kein Mann noch Weib; denn ihr seid allzumal einer in
Christo Jesu. \bibverse{29} Seid ihr aber Christi, so seid ihr ja
Abrahams Same und nach der Verheißung Erben.

\hypertarget{section-3}{%
\section{4}\label{section-3}}

\bibverse{1} Ich sage aber: Solange der Erbe unmündig ist, so ist
zwischen ihm und einem Knecht kein Unterschied, ob er wohl ein Herr ist
aller Güter; \bibverse{2} sondern er ist unter den Vormündern und
Pflegern bis auf die Zeit, die der Vater bestimmt hat. \bibverse{3} Also
auch wir, da wir unmündig waren, waren wir gefangen unter den
äußerlichen Satzungen. \bibverse{4} Da aber die Zeit erfüllet ward,
sandte Gott seinen Sohn, geboren von einem Weibe und unter das Gesetz
getan, \bibverse{5} auf daß er die, so unter dem Gesetz waren, erlöste,
daß wir die Kindschaft empfingen. \bibverse{6} Weil ihr denn Kinder
seid, hat Gott gesandt den Geist seines Sohnes in eure Herzen, der
schreit: Abba, lieber Vater! \bibverse{7} Also ist nun hier kein Knecht
mehr, sondern eitel Kinder; sind's aber Kinder, so sind's auch Erben
Gottes durch Christum. \bibverse{8} Aber zu der Zeit, da ihr Gott nicht
erkanntet, dientet ihr denen, die von Natur nicht Götter sind.
\bibverse{9} Nun ihr aber Gott erkannt habt, ja vielmehr von Gott
erkannt seid, wie wendet ihr euch denn wiederum zu den schwachen und
dürftigen Satzungen, welchen ihr von neuem an dienen wollt?
\bibverse{10} Ihr haltet Tage und Monate und Feste und Jahre.
\bibverse{11} Ich fürchte für euch, daß ich vielleicht umsonst an euch
gearbeitet habe. \bibverse{12} Seid doch wie ich; denn ich bin wie ihr.
Liebe Brüder, ich bitte euch. Ihr habt mir kein Leid getan.
\bibverse{13} Denn ihr wisset, daß ich euch in Schwachheit nach dem
Fleisch das Evangelium gepredigt habe zum erstenmal. \bibverse{14} Und
meine Anfechtungen, die ich leide nach dem Fleisch, habt ihr nicht
verachtet noch verschmäht; sondern wie ein Engel Gottes nahmet ihr mich
auf, ja wie Christum Jesum. \bibverse{15} Wie wart ihr dazumal so selig!
ich bin euer Zeuge, daß, wenn es möglich gewesen wäre, ihr hättet eure
Augen ausgerissen und mir gegeben. \bibverse{16} Bin ich denn damit euer
Feind geworden, daß ich euch die Wahrheit vorhalte? \bibverse{17} Sie
eifern um euch nicht fein; sondern sie wollen euch von mir abfällig
machen, daß ihr um sie eifern sollt. \bibverse{18} Eifern ist gut,
wenn's immerdar geschieht um das Gute, und nicht allein, wenn ich
gegenwärtig bei euch bin. \bibverse{19} Meine lieben Kinder, welche ich
abermals mit Ängsten gebäre, bis daß Christus in euch eine Gestalt
gewinne, \bibverse{20} ich wollte, daß ich jetzt bei euch wäre und meine
Stimme wandeln könnte; denn ich bin irre an euch. \bibverse{21} Saget
mir, die ihr unter dem Gesetz sein wollt: Habt ihr das Gesetz nicht
gehört? \bibverse{22} Denn es steht geschrieben, daß Abraham zwei Söhne
hatte: einen von der Magd, den andern von der Freien. \bibverse{23} Aber
der von der Magd war, ist nach dem Fleisch geboren; der aber von der
Freien ist durch die Verheißung geboren. \bibverse{24} Die Worte
bedeuten etwas. Denn das sind zwei Testamente: eins von dem Berge Sinai,
daß zur Knechtschaft gebiert, welches ist die Hagar; \bibverse{25} denn
Hagar heißt in Arabien der Berg Sinai und kommt überein mit Jerusalem,
das zu dieser Zeit ist und dienstbar ist mit seinen Kindern.
\bibverse{26} Aber das Jerusalem, das droben ist, das ist die Freie; die
ist unser aller Mutter. \bibverse{27} Denn es steht geschrieben: ``Sei
fröhlich, du Unfruchtbare, die du nicht gebierst! Und brich hervor und
rufe, die du nicht schwanger bist! Denn die Einsame hat viel mehr
Kinder, denn die den Mann hat.'' \bibverse{28} Wir aber, liebe Brüder,
sind, Isaak nach, der Verheißung Kinder. \bibverse{29} Aber gleichwie zu
der Zeit, der nach dem Fleisch geboren war, verfolgte den, der nach dem
Geist geboren war, also geht es auch jetzt. \bibverse{30} Aber was
spricht die Schrift? ``Stoß die Magd hinaus mit ihrem Sohn; denn der
Magd Sohn soll nicht erben mit dem Sohn der Freien.'' \bibverse{31} So
sind wir nun, liebe Brüder, nicht der Magd Kinder, sondern der Freien.

\hypertarget{section-4}{%
\section{5}\label{section-4}}

\bibverse{1} So bestehet nun in der Freiheit, zu der uns Christus
befreit hat, und lasset euch nicht wiederum in das knechtische Joch
fangen. \bibverse{2} Siehe, ich, Paulus, sage euch: Wo ihr euch
beschneiden lasset, so nützt euch Christus nichts. \bibverse{3} Ich
bezeuge abermals einem jeden, der sich beschneiden läßt, daß er das
ganze Gesetz schuldig ist zu tun. \bibverse{4} Ihr habt Christum
verloren, die ihr durch das Gesetz gerecht werden wollt, und seid von
der Gnade gefallen. \bibverse{5} Wir aber warten im Geist durch den
Glauben der Gerechtigkeit, auf die man hoffen muß. \bibverse{6} Denn in
Christo Jesu gilt weder Beschneidung noch unbeschnitten sein etwas,
sondern der Glaube, der durch die Liebe tätig ist. \bibverse{7} Ihr
liefet fein. Wer hat euch aufgehalten, der Wahrheit nicht zu gehorchen?
\bibverse{8} Solch Überreden ist nicht von dem, der euch berufen hat.
\bibverse{9} Ein wenig Sauerteig versäuert den ganzen Teig.
\bibverse{10} Ich versehe mich zu euch in dem HERRN, ihr werdet nicht
anders gesinnt sein. Wer euch aber irremacht, der wird sein Urteil
tragen, er sei, wer er wolle. \bibverse{11} Ich aber, liebe Brüder, so
ich die Beschneidung noch predige, warum leide ich denn Verfolgung? So
hätte ja das Ärgernis des Kreuzes aufgehört. \bibverse{12} Wollte Gott,
daß sie auch ausgerottet würden, die euch verstören! \bibverse{13} Ihr
aber, liebe Brüder, seid zur Freiheit berufen! Allein sehet zu, daß ihr
durch die Freiheit dem Fleisch nicht Raum gebet; sondern durch die Liebe
diene einer dem andern. \bibverse{14} Denn alle Gesetze werden in einem
Wort erfüllt, in dem: ``Liebe deinen Nächsten wie dich selbst.''
\bibverse{15} So ihr euch aber untereinander beißet und fresset, so seht
zu, daß ihr nicht untereinander verzehrt werdet. \bibverse{16} Ich sage
aber: Wandelt im Geist, so werdet ihr die Lüste des Fleisches nicht
vollbringen. \bibverse{17} Denn das Fleisch gelüstet wider den Geist,
und der Geist wider das Fleisch; dieselben sind widereinander, daß ihr
nicht tut, was ihr wollt. \bibverse{18} Regiert euch aber der Geist, so
seid ihr nicht unter dem Gesetz. \bibverse{19} Offenbar sind aber die
Werke des Fleisches, als da sind: Ehebruch, Hurerei, Unreinigkeit,
Unzucht, \bibverse{20} Abgötterei, Zauberei, Feindschaft, Hader, Neid,
Zorn, Zank, Zwietracht, Rotten, Haß, Mord, \bibverse{21} Saufen, Fressen
und dergleichen, von welchen ich euch zuvor gesagt und sage noch zuvor,
daß, die solches tun, werden das Reich Gottes nicht erben. \bibverse{22}
Die Frucht aber des Geistes ist Liebe, Freude, Friede, Geduld,
Freundlichkeit, Gütigkeit, Glaube, Sanftmut, Keuschheit. \bibverse{23}
Wider solche ist das Gesetz nicht. \bibverse{24} Welche aber Christo
angehören, die kreuzigen ihr Fleisch samt den Lüsten und Begierden.
\bibverse{25} So wir im Geist leben, so lasset uns auch im Geist
wandeln. \bibverse{26} Lasset uns nicht eitler Ehre geizig sein,
einander zu entrüsten und zu hassen.

\hypertarget{section-5}{%
\section{6}\label{section-5}}

\bibverse{1} Liebe Brüder, so ein Mensch etwa von einem Fehler übereilt
würde, so helfet ihm wieder zurecht mit sanftmütigem Geist ihr, die ihr
geistlich seid; und sieh auf dich selbst, daß du nicht auch versucht
werdest. \bibverse{2} Einer trage des andern Last, so werdet ihr das
Gesetz Christi erfüllen. \bibverse{3} So aber jemand sich läßt dünken,
er sei etwas, so er doch nichts ist, der betrügt sich selbst.
\bibverse{4} Ein jeglicher aber prüfe sein eigen Werk; und alsdann wird
er an sich selber Ruhm haben und nicht an einem andern. \bibverse{5}
Denn ein jeglicher wird seine Last tragen. \bibverse{6} Der aber
unterrichtet wird mit dem Wort, der teile mit allerlei Gutes dem, der
ihn unterrichtet. \bibverse{7} Irrt euch nicht! Gott läßt sich nicht
spotten. Denn was der Mensch sät, das wird er ernten. \bibverse{8} Wer
auf sein Fleisch sät, der wird von dem Fleisch das Verderben ernten; wer
aber auf den Geist sät, der wird von dem Geist das ewige Leben ernten.
\bibverse{9} Lasset uns aber Gutes tun und nicht müde werden; denn zu
seiner Zeit werden wir auch ernten ohne Aufhören. \bibverse{10} Als wir
denn nun Zeit haben, so lasset uns Gutes tun an jedermann, allermeist
aber an des Glaubens Genossen. \bibverse{11} Sehet, mit wie vielen
Worten habe ich euch geschrieben mit eigener Hand! \bibverse{12} Die
sich wollen angenehm machen nach dem Fleisch, die zwingen euch zur
Beschneidung, nur damit sie nicht mit dem Kreuz Christi verfolgt werden.
\bibverse{13} Denn auch sie selbst, die sich beschneiden lassen, halten
das Gesetz nicht; sondern sie wollen, daß ihr euch beschneiden lasset,
auf daß sie sich von eurem Fleisch rühmen mögen. \bibverse{14} Es sei
aber ferne von mir, mich zu rühmen, denn allein von dem Kreuz unsers
HERRN Jesu Christi, durch welchen mir die Welt gekreuzigt ist und ich
der Welt. \bibverse{15} Denn in Christo Jesu gilt weder Beschneidung
noch unbeschnitten sein etwas, sondern eine neue Kreatur. \bibverse{16}
Und wie viele nach dieser Regel einhergehen, über die sei Friede und
Barmherzigkeit und über das Israel Gottes. \bibverse{17} Hinfort mache
mir niemand weiter Mühe; denn ich trage die Malzeichen des HERRN Jesu an
meinem Leibe. \bibverse{18} Die Gnade unsers HERRN Jesu Christi sei mit
eurem Geist, liebe Brüder! Amen.
