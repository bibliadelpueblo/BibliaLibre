\hypertarget{section}{%
\section{1}\label{section}}

\bibverse{1} Dies sind die Reden des Predigers, des Sohnes Davids, des
Königs zu Jerusalem. \bibverse{2} Es ist alles ganz eitel, sprach der
Prediger, es ist alles ganz eitel. \bibverse{3} Was hat der Mensch für
Gewinn von aller seiner Mühe, die er hat unter der Sonne? \bibverse{4}
Ein Geschlecht vergeht, das andere kommt; die Erde aber bleibt ewiglich.
\bibverse{5} Die Sonne geht auf und geht unter und läuft an ihren Ort,
daß sie wieder daselbst aufgehe. \bibverse{6} Der Wind geht gen Mittag
und kommt herum zur Mitternacht und wieder herum an den Ort, da er
anfing. \bibverse{7} Alle Wasser laufen ins Meer, doch wird das Meer
nicht voller; an den Ort, da sie her fließen, fließen sie wieder hin.
\bibverse{8} Es sind alle Dinge so voll Mühe, daß es niemand ausreden
kann. Das Auge sieht sich nimmer satt, und das Ohr hört sich nimmer
satt. \bibverse{9} Was ist's, das geschehen ist? Eben das hernach
geschehen wird. Was ist's, das man getan hat? Eben das man hernach tun
wird; und geschieht nichts Neues unter der Sonne. \bibverse{10}
Geschieht auch etwas, davon man sagen möchte: Siehe, das ist neu? Es ist
zuvor auch geschehen in den langen Zeiten, die vor uns gewesen sind.
\bibverse{11} Man gedenkt nicht derer, die zuvor gewesen sind; also auch
derer, so hernach kommen, wird man nicht gedenken bei denen, die darnach
sein werden. \bibverse{12} Ich, der Prediger, war König zu Jerusalem
\bibverse{13} und richtete mein Herz zu suchen und zu forschen weislich
alles, was man unter dem Himmel tut. Solche unselige Mühe hat Gott den
Menschenkindern gegeben, daß sie sich darin müssen quälen. \bibverse{14}
Ich sah an alles Tun, das unter der Sonne geschieht; und siehe, es war
alles eitel und Haschen nach dem Wind. \bibverse{15} Krumm kann nicht
schlicht werden noch, was fehlt, gezählt werden. \bibverse{16} Ich
sprach in meinem Herzen: Siehe, ich bin herrlich geworden und habe mehr
Weisheit denn alle, die vor mir gewesen sind zu Jerusalem, und mein Herz
hat viel gelernt und erfahren. \bibverse{17} Und richtete auch mein Herz
darauf, daß ich erkennte Weisheit und erkennte Tollheit und Torheit. Ich
ward aber gewahr, daß solches auch Mühe um Wind ist. \bibverse{18} Denn
wo viel Weisheit ist, da ist viel Grämens; und wer viel lernt, der muß
viel leiden.

\hypertarget{section-1}{%
\section{2}\label{section-1}}

\bibverse{1} Ich sprach in meinem Herzen: Wohlan, ich will wohl leben
und gute Tage haben! Aber siehe, das war auch eitel. \bibverse{2} Ich
sprach zum Lachen: Du bist toll! und zur Freude: Was machst du?
\bibverse{3} Da dachte ich in meinem Herzen, meinen Leib mit Wein zu
pflegen, doch also, daß mein Herz mich mit Weisheit leitete, und zu
ergreifen, was Torheit ist, bis ich lernte, was dem Menschen gut wäre,
daß sie tun sollten, solange sie unter dem Himmel leben. \bibverse{4}
Ich tat große Dinge: ich baute Häuser, pflanzte Weinberge; \bibverse{5}
ich machte mir Gärten und Lustgärten und pflanzte allerlei fruchtbare
Bäume darein; \bibverse{6} ich machte mir Teiche, daraus zu wässern den
Wald der grünenden Bäume; \bibverse{7} ich hatte Knechte und Mägde und
auch Gesinde, im Hause geboren; ich hatte eine größere Habe an Rindern
und Schafen denn alle, die vor mir zu Jerusalem gewesen waren;
\bibverse{8} ich sammelte mir auch Silber und Gold und von den Königen
und Ländern einen Schatz; ich schaffte mir Sänger und Sängerinnen und
die Wonne der Menschen, allerlei Saitenspiel; \bibverse{9} `01431' und
nahm zu über alle, die vor mir zu Jerusalem gewesen waren; auch blieb
meine Weisheit bei mir; \bibverse{10} und alles, was meine Augen
wünschten, das ließ ich ihnen und wehrte meinem Herzen keine Freude, daß
es fröhlich war von aller meiner Arbeit; und das hielt ich für mein Teil
von aller meiner Arbeit. \bibverse{11} Da ich aber ansah alle meine
Werke, die meine Hand gemacht hatte, und die Mühe, die ich gehabt hatte,
siehe, da war es alles eitel und Haschen nach dem Wind und kein Gewinn
unter der Sonne. \bibverse{12} Da wandte ich mich, zu sehen die Weisheit
und die Tollheit und Torheit. Denn wer weiß, was der für ein Mensch
werden wird nach dem König, den sie schon bereit gemacht haben?
\bibverse{13} Da ich aber sah, daß die Weisheit die Torheit übertraf wie
das Licht die Finsternis; \bibverse{14} daß dem Weisen seine Augen im
Haupt stehen, aber die Narren in der Finsternis gehen; und merkte doch,
daß es einem geht wie dem andern. \bibverse{15} Da dachte ich in meinem
Herzen: Weil es denn mir geht wie dem Narren, warum habe ich denn nach
Weisheit getrachtet? Da dachte ich in meinem Herzen, daß solches auch
eitel sei. \bibverse{16} Denn man gedenkt des Weisen nicht immerdar,
ebenso wenig wie des Narren, und die künftigen Tage vergessen alles; und
wie der Narr stirbt, also auch der Weise. \bibverse{17} Darum verdroß
mich zu leben; denn es gefiel mir übel, was unter der Sonne geschieht,
daß alles eitel ist und Haschen nach dem Wind. \bibverse{18} Und mich
verdroß alle meine Arbeit, die ich unter der Sonne hatte, daß ich
dieselbe einem Menschen lassen müßte, der nach mir sein sollte.
\bibverse{19} Denn wer weiß, ob er weise oder toll sein wird? und soll
doch herrschen in aller meiner Arbeit, die ich weislich getan habe unter
der Sonne. Das ist auch eitel. \bibverse{20} Darum wandte ich mich, daß
mein Herz abließe von aller Arbeit, die ich tat unter der Sonne.
\bibverse{21} Denn es muß ein Mensch, der seine Arbeit mit Weisheit,
Vernunft und Geschicklichkeit getan hat, sie einem andern zum Erbteil
lassen, der nicht daran gearbeitet hat. Das ist auch eitel und ein
großes Unglück. \bibverse{22} Denn was kriegt der Mensch von aller
seiner Arbeit und Mühe seines Herzens, die er hat unter der Sonne?
\bibverse{23} Denn alle seine Lebtage hat er Schmerzen mit Grämen und
Leid, daß auch sein Herz des Nachts nicht ruht. Das ist auch eitel.
\bibverse{24} Ist's nun nicht besser dem Menschen, daß er esse und
trinke und seine Seele guter Dinge sei in seiner Arbeit? Aber solches
sah ich auch, daß es von Gottes Hand kommt. \bibverse{25} Denn wer kann
fröhlich essen und sich ergötzen ohne ihn? \bibverse{26} Denn dem
Menschen, der ihm gefällt, gibt er Weisheit, Vernunft und Freude; aber
dem Sünder gibt er Mühe, daß er sammle und häufe, und es doch dem
gegeben werde, der Gott gefällt. Darum ist das auch eitel und Haschen
nach dem Wind.

\hypertarget{section-2}{%
\section{3}\label{section-2}}

\bibverse{1} Ein jegliches hat seine Zeit, und alles Vornehmen unter dem
Himmel hat seine Stunde. \bibverse{2} Geboren werden und sterben,
pflanzen und ausrotten, was gepflanzt ist, \bibverse{3} würgen und
heilen, brechen und bauen, \bibverse{4} weinen und lachen, klagen und
tanzen, \bibverse{5} Stein zerstreuen und Steine sammeln, herzen und
ferne sein von Herzen, \bibverse{6} suchen und verlieren, behalten und
wegwerfen, \bibverse{7} zerreißen und zunähen, schweigen und reden,
\bibverse{8} lieben und hassen, Streit und Friede hat seine Zeit.
\bibverse{9} Man arbeite, wie man will, so hat man doch keinen Gewinn
davon. \bibverse{10} Ich sah die Mühe, die Gott den Menschen gegeben
hat, daß sie darin geplagt werden. \bibverse{11} Er aber tut alles fein
zu seiner Zeit und läßt ihr Herz sich ängsten, wie es gehen solle in der
Welt; denn der Mensch kann doch nicht treffen das Werk, das Gott tut,
weder Anfang noch Ende. \bibverse{12} Darum merkte ich, daß nichts
Besseres darin ist denn fröhlich sein und sich gütlich tun in seinem
Leben. \bibverse{13} Denn ein jeglicher Mensch, der da ißt und trinkt
und hat guten Mut in aller seiner Arbeit, das ist eine Gabe Gottes.
\bibverse{14} Ich merkte, daß alles, was Gott tut, das besteht immer:
man kann nichts dazutun noch abtun; und solches tut Gott, daß man sich
vor ihm fürchten soll. \bibverse{15} Was geschieht, das ist zuvor
geschehen, und was geschehen wird, ist auch zuvor geschehen; und Gott
sucht wieder auf, was vergangen ist. \bibverse{16} Weiter sah ich unter
der Sonne Stätten des Gerichts, da war ein gottlos Wesen, und Stätten
der Gerechtigkeit, da waren Gottlose. \bibverse{17} Da dachte ich in
meinem Herzen: Gott muß richten den Gerechten und den Gottlosen; denn es
hat alles Vornehmen seine Zeit und alle Werke. \bibverse{18} Ich sprach
in meinem Herzen: Es geschieht wegen der Menschenkinder, auf daß Gott
sie prüfe und sie sehen, daß sie an sich selbst sind wie das Vieh.
\bibverse{19} Denn es geht dem Menschen wie dem Vieh: wie dies stirbt,
so stirbt er auch, und haben alle einerlei Odem, und der Mensch hat
nichts mehr als das Vieh; denn es ist alles eitel. \bibverse{20} Es
fährt alles an einen Ort; es ist alles von Staub gemacht und wird wieder
zu Staub. \bibverse{21} Wer weiß, ob der Odem der Menschen aufwärts
fahre und der Odem des Viehes abwärts unter die Erde fahre?
\bibverse{22} So sah ich denn, daß nichts Besseres ist, als daß ein
Mensch fröhlich sei in seiner Arbeit; denn das ist sein Teil. Denn wer
will ihn dahin bringen, daß er sehe, was nach ihm geschehen wird?

\hypertarget{section-3}{%
\section{4}\label{section-3}}

\bibverse{1} Ich wandte mich um und sah an alles Unrecht, das geschah
unter der Sonne; und siehe, da waren die Tränen derer, so Unrecht litten
und hatten keinen Tröster; und die ihnen Unrecht taten, waren zu
mächtig, daß sie keinen Tröster haben konnten. \bibverse{2} Da lobte ich
die Toten, die schon gestorben waren, mehr denn die Lebendigen, die noch
das Leben hatten; \bibverse{3} und besser als alle beide ist, der noch
nicht ist und des Bösen nicht innewird, das unter der Sonne geschieht.
\bibverse{4} Ich sah an Arbeit und Geschicklichkeit in allen Sachen; da
neidet einer den andern. Das ist auch eitel und Haschen nach dem Wind.
\bibverse{5} Ein Narr schlägt die Finger ineinander und verzehrt sich
selbst. \bibverse{6} Es ist besser eine Handvoll mit Ruhe denn beide
Fäuste voll mit Mühe und Haschen nach Wind. \bibverse{7} Ich wandte mich
um und sah die Eitelkeit unter der Sonne. \bibverse{8} Es ist ein
einzelner, und nicht selbander, und hat weder Kind noch Bruder; doch ist
seines Arbeitens kein Ende, und seine Augen werden Reichtums nicht satt.
Wem arbeite ich doch und breche meiner Seele ab? Das ist auch eitel und
eine böse Mühe. \bibverse{9} So ist's ja besser zwei als eins; denn sie
genießen doch ihrer Arbeit wohl. \bibverse{10} Fällt ihrer einer so
hilft ihm sein Gesell auf. Weh dem, der allein ist! Wenn er fällt, so
ist keiner da, der ihm aufhelfe. \bibverse{11} Auch wenn zwei
beieinander liegen, wärmen sie sich; wie kann ein einzelner warm werden?
\bibverse{12} Einer mag überwältigt werden, aber zwei mögen widerstehen;
und eine dreifältige Schnur reißt nicht leicht entzwei. \bibverse{13}
Ein armes Kind, das weise ist, ist besser denn ein alter König, der ein
Narr ist und weiß nicht sich zu hüten. \bibverse{14} Es kommt einer aus
dem Gefängnis zum Königreich; und einer, der in seinem Königreich
geboren ist, verarmt. \bibverse{15} Und ich sah, daß alle Lebendigen
unter der Sonne wandelten bei dem andern, dem Kinde, das an jenes Statt
sollte aufkommen. \bibverse{16} Und des Volks, das vor ihm ging, war
kein Ende und des, das ihm nachging; und wurden sein doch nicht froh.
Das ist auch eitel und Mühe um Wind.

\hypertarget{section-4}{%
\section{5}\label{section-4}}

\bibverse{1} {[}4:17{]} Bewahre deinen Fuß, wenn du zum Hause Gottes
gehst, und komme, daß du hörst. Das ist besser als der Narren Opfer;
denn sie wissen nicht, was sie Böses tun. \bibverse{2} Sei nicht schnell
mit deinem Munde und laß dein Herz nicht eilen, was zu reden vor Gott;
denn Gott ist im Himmel, und du auf Erden; darum laß deiner Worte wenig
sein. \bibverse{3} Denn wo viel Sorgen ist, da kommen Träume; und wo
viel Worte sind, da hört man den Narren. \bibverse{4} Wenn du Gott ein
Gelübde tust, so verzieh nicht, es zu halten; denn er hat kein Gefallen
an den Narren. Was du gelobst, das halte. \bibverse{5} Es ist besser, du
gelobst nichts, denn daß du nicht hältst, was du gelobst. \bibverse{6}
Laß deinem Mund nicht zu, daß er dein Fleisch verführe; und sprich vor
dem Engel nicht: Es ist ein Versehen. Gott möchte erzürnen über deine
Stimme und verderben alle Werke deiner Hände. \bibverse{7} Wo viel
Träume sind, da ist Eitelkeit und viel Worte; aber fürchte du Gott.
\bibverse{8} Siehst du dem Armen Unrecht tun und Recht und Gerechtigkeit
im Lande wegreißen, wundere dich des Vornehmens nicht; denn es ist ein
hoher Hüter über den Hohen und sind noch Höhere über die beiden.
\bibverse{9} Und immer ist's Gewinn für ein Land, wenn ein König da ist
für das Feld, das man baut. \bibverse{10} Wer Geld liebt, wird Geldes
nimmer satt; und wer Reichtum liebt, wird keinen Nutzen davon haben. Das
ist auch eitel. \bibverse{11} Denn wo viel Guts ist, da sind viele, die
es essen; und was genießt davon, der es hat, außer daß er's mit Augen
ansieht? \bibverse{12} Wer arbeitet, dem ist der Schaf süß, er habe
wenig oder viel gegessen; aber die Fülle des Reichen läßt ihn nicht
schlafen. \bibverse{13} Es ist ein böses Übel, das ich sah unter der
Sonne: Reichtum, behalten zum Schaden dem, der ihn hat. \bibverse{14}
Denn der Reiche kommt um mit großem Jammer; und so er einen Sohn gezeugt
hat, dem bleibt nichts in der Hand. \bibverse{15} Wie er nackt ist von
seine Mutter Leibe gekommen, so fährt er wieder hin, wie er gekommen
ist, und nimmt nichts mit sich von seiner Arbeit in seiner Hand, wenn er
hinfährt. \bibverse{16} Das ist ein böses Übel, daß er hinfährt, wie er
gekommen ist. Was hilft's ihm denn, daß er in den Wind gearbeitet hat?
\bibverse{17} Sein Leben lang hat er im Finstern gegessen und in großem
Grämen und Krankheit und Verdruß. \bibverse{18} So sehe ich nun das für
gut an, daß es fein sei, wenn man ißt und trinkt und gutes Muts ist in
aller Arbeit, die einer tut unter der Sonne sein Leben lang, das Gott
ihm gibt; denn das ist sein Teil. \bibverse{19} Denn welchem Menschen
Gott Reichtum und Güter gibt und die Gewalt, daß er davon ißt und trinkt
für sein Teil und fröhlich ist in seiner Arbeit, das ist eine
Gottesgabe. \bibverse{20} Denn er denkt nicht viel an die Tage seines
Lebens, weil Gott sein Herz erfreut.

\hypertarget{section-5}{%
\section{6}\label{section-5}}

\bibverse{1} Es ist ein Unglück, das ich sah unter der Sonne, und ist
gemein bei den Menschen: \bibverse{2} einer, dem Gott Reichtum, Güter
und Ehre gegeben hat und mangelt ihm keins, das sein Herz begehrt; und
Gott gibt doch ihm nicht Macht, es zu genießen, sondern ein anderer
verzehrt es; das ist eitel und ein böses Übel. \bibverse{3} Wenn einer
gleich hundert Kinder zeugte und hätte langes Leben, daß er viele Jahre
überlebte, und seine Seele sättigte sich des Guten nicht und bliebe ohne
Grab, von dem spreche ich, daß eine unzeitige Geburt besser sei denn er.
\bibverse{4} Denn in Nichtigkeit kommt sie, und in Finsternis fährt sie
dahin, und ihr Name bleibt in Finsternis bedeckt, \bibverse{5} auch hat
sie die Sonne nicht gesehen noch gekannt; so hat sie mehr Ruhe denn
jener. \bibverse{6} Ob er auch zweitausend Jahre lebte, und genösse
keines Guten: kommt's nicht alles an einen Ort? \bibverse{7} Alle Arbeit
des Menschen ist für seinen Mund; aber doch wird die Seele nicht davon
satt. \bibverse{8} Denn was hat ein Weiser mehr als ein Narr? Was
hilft's den Armen, daß er weiß zu wandeln vor den Lebendigen?
\bibverse{9} Es ist besser, das gegenwärtige Gut gebrauchen, denn nach
anderm gedenken. Das ist auch Eitelkeit und Haschen nach Wind.
\bibverse{10} Was da ist, des Name ist zuvor genannt, und es ist
bestimmt, was ein Mensch sein wird; und er kann nicht hadern mit dem,
der ihm zu mächtig ist. \bibverse{11} Denn es ist des eitlen Dinges
zuviel; was hat ein Mensch davon? \bibverse{12} Denn wer weiß, was dem
Menschen nütze ist im Leben, solange er lebt in seiner Eitelkeit,
welches dahinfährt wie ein Schatten? Oder wer will dem Menschen sagen,
was nach ihm kommen wird unter der Sonne?

\hypertarget{section-6}{%
\section{7}\label{section-6}}

\bibverse{1} Ein guter Ruf ist besser denn gute Salbe, und der Tag des
Todes denn der Tag der Geburt. \bibverse{2} Es ist besser in das
Klagehaus gehen, denn in ein Trinkhaus; in jenem ist das Ende aller
Menschen, und der Lebendige nimmt's zu Herzen. \bibverse{3} Es ist
Trauern besser als Lachen; denn durch Trauern wird das Herz gebessert.
\bibverse{4} Das Herz der Weisen ist im Klagehause, und das Herz der
Narren im Hause der Freude. \bibverse{5} Es ist besser hören das
Schelten der Weisen, denn hören den Gesang der Narren. \bibverse{6} Denn
das Lachen der Narren ist wie das Krachen der Dornen unter den Töpfen;
und das ist auch eitel. \bibverse{7} Ein Widerspenstiger macht einen
Weisen unwillig und verderbt ein mildtätiges Herz. \bibverse{8} Das Ende
eines Dinges ist besser denn sein Anfang. Ein geduldiger Geist ist
besser denn ein hoher Geist. \bibverse{9} Sei nicht schnellen Gemütes zu
zürnen; denn Zorn ruht im Herzen eines Narren. \bibverse{10} Sprich
nicht: Was ist's, daß die vorigen Tage besser waren als diese? denn du
fragst solches nicht weislich. \bibverse{11} Weisheit ist gut mit einem
Erbgut und hilft, daß sich einer der Sonne freuen kann. \bibverse{12}
Denn die Weisheit beschirmt, so beschirmt Geld auch; aber die Weisheit
gibt das Leben dem, der sie hat. \bibverse{13} Siehe an die Werke
Gottes; denn wer kann das schlicht machen, was er krümmt? \bibverse{14}
Am guten Tage sei guter Dinge, und den bösen Tag nimm auch für gut; denn
diesen schafft Gott neben jenem, daß der Mensch nicht wissen soll, was
künftig ist. \bibverse{15} Allerlei habe ich gesehen in den Tagen meiner
Eitelkeit. Da ist ein Gerechter, und geht unter mit seiner
Gerechtigkeit; und ein Gottloser, der lange lebt in seiner Bosheit.
\bibverse{16} Sei nicht allzu gerecht und nicht allzu weise, daß du dich
nicht verderbest. \bibverse{17} Sei nicht allzu gottlos und narre nicht,
daß du nicht sterbest zur Unzeit. \bibverse{18} Es ist gut, daß du dies
fassest und jenes auch nicht aus deiner Hand lässest; denn wer Gott
fürchtet, der entgeht dem allem. \bibverse{19} Die Weisheit stärkt den
Weisen mehr denn zehn Gewaltige, die in der Stadt sind. \bibverse{20}
Denn es ist kein Mensch so gerecht auf Erden, daß er Gutes tue und nicht
sündige. \bibverse{21} Gib auch nicht acht auf alles, was man sagt, daß
du nicht hören müssest deinen Knecht dir fluchen. \bibverse{22} Denn
dein Herz weiß, daß du andern oftmals geflucht hast. \bibverse{23}
Solches alles habe ich versucht mit Weisheit. Ich gedachte, ich will
weise sein; sie blieb aber ferne von mir. \bibverse{24} Alles, was da
ist, das ist ferne und sehr tief; wer will's finden? \bibverse{25} Ich
kehrte mein Herz, zu erfahren und erforschen und zu suchen Weisheit und
Kunst, zu erfahren der Gottlosen Torheit und Irrtum der Tollen,
\bibverse{26} und fand, daß bitterer sei denn der Tod ein solches Weib,
dessen Herz Netz und Strick ist und deren Hände Bande sind. Wer Gott
gefällt, der wird ihr entrinnen; aber der Sünder wird durch sie
gefangen. \bibverse{27} Schau, das habe ich gefunden, spricht der
Prediger, eins nach dem andern, daß ich Erkenntnis fände. \bibverse{28}
Und meine Seele sucht noch und hat's nicht gefunden: unter tausend habe
ich einen Mann gefunden; aber ein Weib habe ich unter den allen nicht
gefunden. \bibverse{29} Allein schaue das: ich habe gefunden, daß Gott
den Menschen hat aufrichtig gemacht; aber sie suchen viele Künste.

\hypertarget{section-7}{%
\section{8}\label{section-7}}

\bibverse{1} Wer ist wie der Weise, und wer kann die Dinge auslegen? Die
Weisheit des Menschen erleuchtet sein Angesicht; aber ein freches
Angesicht wird gehaßt. \bibverse{2} Halte das Wort des Königs und den
Eid Gottes. \bibverse{3} Eile nicht zu gehen von seinem Angesicht, und
bleibe nicht in böser Sache; denn er tut, was er will. \bibverse{4} In
des Königs Wort ist Gewalt; und wer mag zu ihm sagen: Was machst du?
\bibverse{5} Wer das Gebot hält, der wird nichts Böses erfahren; aber
eines Weisen Herz weiß Zeit und Weise. \bibverse{6} Denn ein jeglich
Vornehmen hat seine Zeit und Weise; denn des Unglücks des Menschen ist
viel bei ihm. \bibverse{7} Denn er weiß nicht, was geschehen wird; und
wer soll ihm sagen, wie es werden soll? \bibverse{8} Ein Mensch hat
nicht Macht über den Geist, den Geist zurückzuhalten, und hat nicht
Macht über den Tag des Todes, und keiner wird losgelassen im Streit; und
das gottlose Wesen errettet den Gottlosen nicht. \bibverse{9} Das habe
ich alles gesehen, und richtete mein Herz auf alle Werke, die unter der
Sonne geschehen. Ein Mensch herrscht zuzeiten über den andern zu seinem
Unglück. \bibverse{10} Und da sah ich Gottlose, die begraben wurden und
zur Ruhe kamen; aber es wandelten hinweg von heiliger Stätte und wurden
vergessen in der Stadt die, so recht getan hatten. Das ist auch eitel.
\bibverse{11} Weil nicht alsbald geschieht ein Urteil über die bösen
Werke, dadurch wird das Herz der Menschen voll, Böses zu tun.
\bibverse{12} Ob ein Sünder hundertmal Böses tut und lange lebt, so weiß
ich doch, daß es wohl gehen wird denen, die Gott fürchten, die sein
Angesicht scheuen. \bibverse{13} Aber dem Gottlosen wird es nicht wohl
gehen; und wie ein Schatten werden nicht lange leben, die sich vor Gott
nicht fürchten. \bibverse{14} Es ist eine Eitelkeit, die auf Erden
geschieht: es sind Gerechte, denen geht es als hätten sie Werke der
Gottlosen, und sind Gottlose, denen geht es als hätten sie Werke der
Gerechten. Ich sprach: Das ist auch eitel. \bibverse{15} Darum lobte ich
die Freude, daß der Mensch nichts Besseres hat unter der Sonne denn
essen und trinken und fröhlich sein; und solches werde ihm von der
Arbeit sein Leben lang, das ihm Gott gibt unter der Sonne. \bibverse{16}
Ich gab mein Herz, zu wissen die Weisheit und zu schauen die Mühe, die
auf Erden geschieht, daß auch einer weder Tag noch Nacht den Schlaf
sieht mit seinen Augen. \bibverse{17} Und ich sah alle Werke Gottes, daß
ein Mensch das Werk nicht finden kann, das unter der Sonne geschieht;
und je mehr der Mensch arbeitet, zu suchen, je weniger er findet. Wenn
er gleich spricht: ``Ich bin weise und weiß es'', so kann er's doch
nicht finden.

\hypertarget{section-8}{%
\section{9}\label{section-8}}

\bibverse{1} Denn ich habe solches alles zu Herzen genommen, zu forschen
das alles, daß Gerechte und Weise und ihre Werke sind in Gottes Hand;
kein Mensch kennt weder die Liebe noch den Haß irgend eines, den er vor
sich hat. \bibverse{2} Es begegnet dasselbe einem wie dem andern: dem
Gerechten wie dem Gottlosen, dem Guten und Reinen wie dem Unreinen, dem,
der opfert, wie dem, der nicht opfert; wie es dem Guten geht, so geht's
auch dem Sünder; wie es dem, der schwört, geht, so geht's auch dem, der
den Eid fürchtet. \bibverse{3} Das ist ein böses Ding unter allem, was
unter der Sonne geschieht, daß es einem geht wie dem andern; daher auch
das Herz der Menschen voll Arges wird, und Torheit ist in ihrem Herzen,
dieweil sie leben; darnach müssen sie sterben. \bibverse{4} Denn bei
allen Lebendigen ist, was man wünscht: Hoffnung; denn ein lebendiger
Hund ist besser denn ein toter Löwe. \bibverse{5} Denn die Lebendigen
wissen, daß sie sterben werden; die Toten aber wissen nichts, sie haben
auch keinen Lohn mehr, denn ihr Gedächtnis ist vergessen, \bibverse{6}
daß man sie nicht mehr liebt noch haßt noch neidet, und haben kein Teil
mehr auf dieser Welt an allem, was unter der Sonne geschieht.
\bibverse{7} So gehe hin und iß dein Brot mit Freuden, trink deinen Wein
mit gutem Mut; denn dein Werk gefällt Gott. \bibverse{8} Laß deine
Kleider immer weiß sein und laß deinem Haupt Salbe nicht mangeln.
\bibverse{9} Brauche das Leben mit deinem Weibe, das du liebhast,
solange du das eitle Leben hast, das dir Gott unter der Sonne gegeben
hat, solange dein eitel Leben währt; denn das ist dein Teil im Leben und
in deiner Arbeit, die du tust unter der Sonne. \bibverse{10} Alles, was
dir vor Handen kommt, zu tun, das tue frisch; denn bei den Toten, dahin
du fährst, ist weder Werk, Kunst, Vernunft noch Weisheit. \bibverse{11}
Ich wandte mich und sah, wie es unter der Sonne zugeht, daß zum Laufen
nicht hilft schnell zu sein, zum Streit hilft nicht stark sein, zur
Nahrung hilft nicht geschickt sein, zum Reichtum hilft nicht klug sein;
daß einer angenehm sei, dazu hilft nicht, daß er ein Ding wohl kann;
sondern alles liegt an Zeit und Glück. \bibverse{12} Auch weiß der
Mensch seine Zeit nicht; sondern, wie die Fische gefangen werden mit
einem verderblichen Haken, und wie die Vögel mit einem Strick gefangen
werden, so werden auch die Menschen berückt zur bösen Zeit, wenn sie
plötzlich über sie fällt. \bibverse{13} Ich habe auch diese Weisheit
gesehen unter der Sonne, die mich groß deuchte: \bibverse{14} daß eine
kleine Stadt war und wenig Leute darin, und kam ein großer König und
belagerte sie und baute große Bollwerke darum, \bibverse{15} und ward
darin gefunden ein armer, weiser Mann, der errettete dieselbe Stadt
durch seine Weisheit; und kein Mensch gedachte desselben armen Mannes.
\bibverse{16} Da sprach ich: ``Weisheit ist ja besser den Stärke; doch
wird des Armen Weisheit verachtet und seinen Worten nicht gehorcht.''
\bibverse{17} Der Weisen Worte, in Stille vernommen, sind besser denn
der Herren Schreien unter den Narren. \bibverse{18} Weisheit ist besser
denn Harnisch; aber eine einziger Bube verderbt viel Gutes.

\hypertarget{section-9}{%
\section{10}\label{section-9}}

\bibverse{1} Schädliche Fliegen verderben gute Salben; also wiegt ein
wenig Torheit schwerer denn Weisheit und Ehre. \bibverse{2} Des Weisen
Herz ist zu seiner Rechten; aber des Narren Herz ist zu seiner Linken.
\bibverse{3} Auch ob der Narr selbst närrisch ist in seinem Tun, doch
hält er jedermann für einen Narren. \bibverse{4} Wenn eines Gewaltigen
Zorn wider dich ergeht, so laß dich nicht entrüsten; denn Nachlassen
stillt großes Unglück. \bibverse{5} Es ist ein Unglück, das ich sah
unter der Sonne, gleich einem Versehen, das vom Gewaltigen ausgeht:
\bibverse{6} daß ein Narr sitzt in großer Würde, und die Reichen in
Niedrigkeit sitzen. \bibverse{7} Ich sah Knechte auf Rossen, und Fürsten
zu Fuß gehen wie Knechte. \bibverse{8} Aber wer eine Grube macht, der
wird selbst hineinfallen; und wer den Zaun zerreißt, den wird eine
Schlange stechen. \bibverse{9} Wer Steine wegwälzt, der wird Mühe damit
haben; und wer Holz spaltet, der wird davon verletzt werden.
\bibverse{10} Wenn ein Eisen stumpf wird und an der Schneide
ungeschliffen bleibt, muß man's mit Macht wieder schärfen; also folgt
auch Weisheit dem Fleiß. \bibverse{11} Ein Schwätzer ist nichts Besseres
als eine Schlange, die ohne Beschwörung sticht. \bibverse{12} Die Worte
aus dem Mund eines Weisen sind holdselig; aber des Narren Lippen
verschlingen ihn selbst. \bibverse{13} Der Anfang seiner Worte ist
Narrheit, und das Ende ist schädliche Torheit. \bibverse{14} Ein Narr
macht viele Worte; aber der Mensch weiß nicht, was gewesen ist, und wer
will ihm sagen, was nach ihm werden wird? \bibverse{15} Die Arbeit der
Narren wird ihnen sauer, weil sie nicht wissen in die Stadt zu gehen.
\bibverse{16} Weh dir, Land, dessen König ein Kind ist, und dessen
Fürsten in der Frühe speisen! \bibverse{17} Wohl dir, Land, dessen König
edel ist, und dessen Fürsten zu rechter Zeit speisen, zur Stärke und
nicht zur Lust! \bibverse{18} Denn durch Faulheit sinken die Balken, und
durch lässige Hände wird das Haus triefend. \bibverse{19} Das macht, sie
halten Mahlzeiten, um zu lachen, und der Wein muß die Lebendigen
erfreuen, und das Geld muß ihnen alles zuwege bringen. \bibverse{20}
Fluche dem König nicht in deinem Herzen und fluche dem Reichen nicht in
deiner Schlafkammer; denn die Vögel des Himmels führen die Stimme fort,
und die Fittiche haben, sagen's weiter.

\hypertarget{section-10}{%
\section{11}\label{section-10}}

\bibverse{1} Laß dein Brot über das Wasser fahren, so wirst du es finden
nach langer Zeit. \bibverse{2} `05414' Teile aus unter sieben und unter
acht; denn du weißt nicht, was für Unglück auf Erden kommen wird.
\bibverse{3} Wenn die Wolken voll sind, so geben sie Regen auf die Erde;
und wenn der Baum fällt, er falle gegen Mittag oder Mitternacht, auf
welchen Ort er fällt, da wird er liegen. \bibverse{4} Wer auf den Wind
achtet, der sät nicht; und wer auf die Wolken sieht, der erntet nicht.
\bibverse{5} Gleichwie du nicht weißt den Weg des Windes und wie die
Gebeine in Mutterleibe bereitet werden, also kannst du auch Gottes Werk
nicht wissen, das er tut überall. \bibverse{6} Frühe säe deinen Samen
und laß deine Hand des Abends nicht ab; denn du weißt nicht, ob dies
oder das geraten wird; und ob beides geriete, so wäre es desto besser.
\bibverse{7} Es ist das Licht süß, und den Augen lieblich, die Sonne zu
sehen. \bibverse{8} Wenn ein Mensch viele Jahre lebt, so sei er fröhlich
in ihnen allen und gedenke der finstern Tage, daß ihrer viel sein
werden; denn alles, was kommt, ist eitel. \bibverse{9} So freue dich,
Jüngling, in deiner Jugend und laß dein Herz guter Dinge sein in deiner
Jugend. Tue, was dein Herz gelüstet und deinen Augen gefällt, und wisse,
daß dich Gott um dies alles wird vor Gericht führen. \bibverse{10} Laß
die Traurigkeit in deinem Herzen und tue das Übel von deinem Leibe; denn
Kindheit und Jugend ist eitel.

\hypertarget{section-11}{%
\section{12}\label{section-11}}

\bibverse{1} Gedenke an deinen Schöpfer in deiner Jugend, ehe denn die
bösen Tage kommen und die Jahre herzutreten, da du wirst sagen: Sie
gefallen mir nicht; \bibverse{2} ehe denn die Sonne und das Licht, Mond
und Sterne finster werden und Wolken wieder kommen nach dem Regen;
\bibverse{3} zur Zeit, wenn die Hüter im Hause zittern, und sich krümmen
die Starken, und müßig stehen die Müller, weil ihrer so wenig geworden
sind, und finster werden, die durch die Fenster sehen, \bibverse{4} und
die Türen an der Gasse geschlossen werden, daß die Stimme der Mühle
leise wird, und man erwacht, wenn der Vogel singt, und gedämpft sind
alle Töchter des Gesangs; \bibverse{5} wenn man auch vor Höhen sich
fürchtet und sich scheut auf dem Wege; wenn der Mandelbaum blüht, und
die Heuschrecke beladen wird, und alle Lust vergeht (denn der Mensch
fährt hin, da er ewig bleibt, und die Klageleute gehen umher auf der
Gasse); \bibverse{6} ehe denn der silberne Strick wegkomme, und die
goldene Schale zerbreche, und der Eimer zerfalle an der Quelle, und das
Rad zerbrochen werde am Born. \bibverse{7} Denn der Staub muß wieder zu
der Erde kommen, wie er gewesen ist, und der Geist wieder zu Gott, der
ihn gegeben hat. \bibverse{8} Es ist alles ganz eitel, sprach der
Prediger, ganz eitel. \bibverse{9} Derselbe Prediger war nicht allein
weise, sondern lehrte auch das Volk gute Lehre und merkte und forschte
und stellte viel Sprüche. \bibverse{10} Er suchte, daß er fände
angenehme Worte, und schrieb recht die Worte der Wahrheit. \bibverse{11}
Die Worte der Weisen sind Stacheln und Nägel; sie sind geschrieben durch
die Meister der Versammlungen und von einem Hirten gegeben.
\bibverse{12} Hüte dich, mein Sohn, vor andern mehr; denn viel
Büchermachens ist kein Ende, und viel studieren macht den Leib müde.
\bibverse{13} Laßt uns die Hauptsumme alle Lehre hören: Fürchte Gott und
halte seine Gebote; denn das gehört allen Menschen zu. \bibverse{14}
Denn Gott wird alle Werke vor Gericht bringen, alles, was verborgen ist,
es sei gut oder böse.
