\hypertarget{section}{%
\section{1}\label{section}}

\bibverse{1} Im dritten Jahr des Reichs Jojakims, des Königs Judas, kam
Nebukadnezar,der König zu Babel, vor Jerusalem und belagerte sie.
\bibverse{2} Und der HErr übergab ihm Jojakim, den König Judas, und
etliche Gefäße ausdem Hause GOttes; die ließ er führen ins Land Sinear,
in seines Gottes Haus, und tat dieGefäße in seines Gottes Schatzkasten.
\bibverse{3} Und der König sprach zu Aspenas, seinem obersten Kämmerer,
er sollte ausden Kindern Israel vom königlichen Stamm und Herrenkindern
wählen \bibverse{4} Knaben, die nicht gebrechlich wären, sondern schöne,
vernünftige, weise,kluge und verständige, die da geschickt wären, zu
dienen in des Königs Hofe und zulernen chaldäische Schrift und Sprache.
\bibverse{5} Solchen verschaffte der König, was man ihnen täglich geben
sollte von seinerSpeise und von dem Wein, den er selbst trank, daß sie,
also drei Jahre auferzogen,danach vor dem Könige dienen sollten.
\bibverse{6} Unter welchen waren Daniel, Hananja, Misael und Asarja von
den KindernJudas. \bibverse{7} Und der oberste Kämmerer gab ihnen Namen
und nannte Daniel Beltsazarund Hananja Sadrach und Misael Mesach und
Asarja Abed-Nego. \bibverse{8} Aber Daniel setzte ihm vor in seinem
Herzen, daß er sich mit des KönigsSpeise und mit dem Wein, den er selbst
trank, nicht verunreinigen wollte, und bat denobersten Kämmerer, daß er
sich nicht müßte verunreinigen. \bibverse{9} Und GOtt gab Daniel, daß
ihm der oberste Kämmerer günstig und gnädigward. \bibverse{10} Derselbe
sprach zu ihm: Ich fürchte mich vor meinem Herrn, dem Könige,der euch
eure Speise und Trank verschaffet hat; wo er würde sehen, daß
eureAngesichte jämmerlicher wären denn der andern Knaben eures Alters,
so brächtet ihrmich bei dem Könige um mein Leben. \bibverse{11} Da
sprach Daniel zu Melzar, welchem der oberste Kämmerer Daniel,Hananja,
Misael und Asarja befohlen hatte: \bibverse{12} Versuch es doch mit
deinen Knechten zehn Tage und laß uns gebenGemüse zu essen und Wasser zu
trinken! \bibverse{13} Und laß dann vor dir unsere Gestalt und der
Knaben, so von des KönigsSpeise essen, besehen; und danach du sehen
wirst, danach schaffe mit deinenKnechten. \bibverse{14} Und er gehorchte
ihnen darin und versuchte es mit ihnen zehn Tage. \bibverse{15} Und nach
den zehn Tagen waren sie schöner und baß bei Leibe denn alleKnaben, so
von des Königs Speise aßen. \bibverse{16} Da tat Melzar ihre verordnete
Speise und Trank weg und gab ihnenGemüse. \bibverse{17} Aber der GOtt
dieser vier gab ihnen Kunst und Verstand in allerlei Schriftund
Weisheit; Daniel aber gab er Verstand in allen Gesichten und Träumen.
\bibverse{18} Und da die Zeit um war, die der König bestimmt hatte, daß
sie solltenhineingebracht werden, brachte sie der oberste Kämmerer
hinein vor Nebukadnezar. \bibverse{19} Und der König redete mit ihnen,
und ward unter allen niemand erfunden,der Daniel, Hananja, Misael und
Asarja gleich wäre. Und sie wurden des Königs Diener. \bibverse{20} Und
der König fand sie in allen Sachen, die er sie fragte, zehnmal klügerund
verständiger denn alle Sternseher und Weisen in seinem ganzen Reich.
\bibverse{21} Und Daniel lebte bis ins erste Jahr des Königs Kores.

\hypertarget{section-1}{%
\section{2}\label{section-1}}

\bibverse{1} Im andern Jahr des Reichs Nebukadnezars hatte Nebukadnezar
einen Traum,davon er erschrak, daß er aufwachte. \bibverse{2} Und er
hieß alle Sternseher, und Weisen und Zauberer und
Chaldäerzusammenfordern, daß sie dem Könige seinen Traum sagen sollten.
Und sie kamen undtraten vor den König. \bibverse{3} Und der König sprach
zu ihnen: Ich habe einen Traum gehabt, der hat micherschreckt; und ich
wollte gerne wissen, was es für ein Traum gewesen sei. \bibverse{4} Da
sprachen die Chaldäer zum Könige auf chaldäisch: Herr König,
GOttverleihe dir langes Leben! Sage deinen Knechten den Traum, so wollen
wir ihn deuten. \bibverse{5} Der König antwortete und sprach zu den
Chaldäern: Es ist mir entfallen.Werdet ihr mir den Traum nicht anzeigen
und ihn deuten, so werdet ihr gar umkommenund eure Häuser schändlich
verstöret werden. \bibverse{6} Werdet ihr mir aber den Traum anzeigen
und deuten, so sollt ihr Geschenke,Gaben und große Ehre von mir haben.
Darum so sagt mir den Traum und seineDeutung! \bibverse{7} Sie
antworteten wiederum und sprachen: Der König sage seinen Knechtenden
Traum, so wollen wir ihn deuten. \bibverse{8} Der König antwortete und
sprach: Wahrlich, ich merke es, daß ihr Fristsuchet, weil ihr sehet, daß
mir's entfallen ist. \bibverse{9} Aber werdet ihr mir nicht den Traum
sagen, so gehet das Recht über euch,als die ihr Lügen und Gedichte vor
mir zu reden vorgenommen habt, bis die Zeitvorübergehe. Darum so sagt
mir den Traum, so kann ich merken, daß ihr auch dieDeutung treffet.
\bibverse{10} Da antworteten die Chaldäer vor dem Könige und sprachen zu
ihm: Es istkein Mensch auf Erden, der sagen könne, das der König
fordert. So ist auch kein König,wie groß oder mächtig er sei, der
solches von irgendeinem Sternseher, Weisen oderChaldäer fordere.
\bibverse{11} Denn das der König fordert, ist zu hoch, und ist auch
sonst niemand, der esvor dem Könige sagen könne, ausgenommen die Götter,
die bei den Menschen nichtwohnen. \bibverse{12} Da ward der König sehr
zornig und befahl, alle Weisen zu Babelumzubringen. \bibverse{13} Und
das Urteil ging aus, daß man die Weisen töten sollte. Und Daniel
samtseinen Gesellen ward auch gesucht, daß man sie tötete. \bibverse{14}
Da vernahm Daniel solch Urteil und Befehl von dem obersten Richter
desKönigs, welcher auszog, zu töten die Weisen zu Babel. \bibverse{15}
Und er fing an und sprach zu des Königs Vogt Arioch: Warum ist so
einstreng Urteil vom Könige ausgegangen? Und Arioch zeigte es dem Daniel
an. \bibverse{16} Da ging Daniel hinauf und bat den König, daß er ihm
Frist gäbe, damit erdie Deutung dem Könige sagen möchte. \bibverse{17}
Und Daniel ging heim und zeigte solches an seinen Gesellen,
Hananja,Misael und Asarja, \bibverse{18} daß sie GOtt vom Himmel um
Gnade bäten solches verborgenen Dingshalben, damit Daniel und seine
Gesellen nicht samt den andern Weisen zu Babelumkämen. \bibverse{19} Da
ward Daniel solch verborgen Ding durch ein Gesicht des Nachtsoffenbaret.
\bibverse{20} Darüber lobte Daniel den GOtt vom Himmel, fing an und
sprach: Gelobetsei der Name GOttes von Ewigkeit zu Ewigkeit; denn sein
ist beides, Weisheit undStärke! \bibverse{21} Er ändert Zeit und Stunde;
er setzt Könige ab und setzt Könige ein; er gibtden Weisen ihre Weisheit
und den Verständigen ihren Verstand; \bibverse{22} er offenbaret, was
tief und verborgen ist; er weiß, was in Finsternis liegt;denn bei ihm
ist eitel Licht. \bibverse{23} Ich danke dir und lobe dich, GOtt meiner
Väter, daß du mir Weisheit undStärke verleihest und jetzt offenbaret
hast, darum wir dich gebeten haben; nämlich duhast uns des Königs Sache
offenbaret. \bibverse{24} Da ging Daniel hinauf zu Arioch, der vom
Könige Befehl hatte, die Weisenzu Babel umzubringen, und sprach zu ihm
also: Du sollst die Weisen zu Babel nichtumbringen, sondern führe mich
hinauf zum Könige, ich will dem Könige die Deutungsagen. \bibverse{25}
Arioch brachte Daniel eilends hinauf vor den König und sprach zu ihm
also:Es ist einer funden unter den Gefangenen aus Juda, der dem Könige
die Deutung sagenkann. \bibverse{26} Der König antwortete und sprach zu
Daniel, den sie Beltsazar hießen: Bistdu, der mir den Traum, den ich
gesehen habe und seine Deutung zeigen kann? \bibverse{27} Daniel fing an
vor dem Könige und sprach: Das verborgene Ding, das derKönig fordert von
den Weisen, Gelehrten, Sternsehern und Wahrsagern, stehet in
ihremVermögen nicht, dem Könige zu sagen, \bibverse{28} sondern GOtt vom
Himmel, der kann verborgene Dinge offenbaren; der hatdem Könige
Nebukadnezar angezeiget, was in künftigen Zeiten geschehen soll.
\bibverse{29} Dein Traum und dein Gesicht, da du schliefest, kam daher:
Du, König,dachtest auf deinem Bette, wie es doch hernach gehen würde;
und der, so verborgeneDinge offenbaret, hat dir angezeiget, wie es gehen
werde. \bibverse{30} So ist mir solch verborgen Ding offenbaret, nicht
durch meine Weisheit, alswäre sie größer denn aller, die da leben,
sondern darum, daß dem Könige die Deutungangezeiget würde, und du deines
Herzens Gedanken erführest. \bibverse{31} Du, König, sahst, und siehe,
ein sehr groß und hoch Bild stund vor dir, daswar schrecklich anzusehen.
\bibverse{32} Desselben Bildes Haupt war von feinem Golde; seine Brust
und Arme warenvon Silber; sein Bauch und Lenden waren von Erz;
\bibverse{33} seine Schenkel waren Eisen; seine Füße waren eines Teils
Eisen und einesTeils Ton. \bibverse{34} Solches sahst du, bis daß ein
Stein herabgerissen ward ohne Hände; derschlug das Bild an seine Füße,
die Eisen und Ton waren, und zermalmete sie. \bibverse{35} Da wurden
miteinander zermalmet das Eisen, Ton, Erz, Silber und Gold undwurden wie
Spreu auf der Sommertenne; und der Wind verwebte sie, daß man
sienirgends mehr finden konnte. Der Stein aber, der das Bild schlug,
ward ein großer Berg,daß er die ganze Welt füllete. \bibverse{36} Das
ist der Traum. Nun wollen wir die Deutung vor dem Könige sagen.
\bibverse{37} Du, König, bist ein König aller Könige, dem GOtt vom
Himmel Königreich,Macht, Stärke und Ehre gegeben hat \bibverse{38} und
alles da Leute wohnen, dazu die Tiere auf dem Felde und die Vögelunter
dem Himmel in deine Hände gegeben und dir über alles Gewalt verliehen
hat. Dubist das güldene Haupt. \bibverse{39} Nach dir wird ein ander
Königreich aufkommen, geringer denn deines.Danach das dritte Königreich,
das ehern ist, welches wird über alle Lande herrschen. \bibverse{40} Das
vierte wird hart sein wie Eisen. Denn gleichwie Eisen alles zermalmetund
zerschlägt, ja, wie Eisen alles zerbricht, also wird es auch alles
zermalmen undzerbrechen. \bibverse{41} Daß du aber gesehen hast die Füße
und Zehen eines Teils Ton und einesTeils Eisen, das wird ein zerteilt
Königreich sein; doch wird von des Eisens Pflanzedrinnen bleiben, wie du
denn gesehen hast Eisen mit Ton vermenget. \bibverse{42} Und daß die
Zehen an seinen Füßen eines Teils Eisen und eines Teils Tonsind, wird es
zum Teil ein stark und zum Teil ein schwach Reich sein. \bibverse{43}
Und daß du gesehen hast Eisen mit Ton vermenget, werden sie sich
wohlnach Menschengeblüt untereinander mengen, aber sie werden doch nicht
aneinanderhalten, gleichwie sich Eisen mit Ton nicht mengen läßt.
\bibverse{44} Aber zur Zeit solcher Königreiche wird GOtt vom Himmel ein
Königreichaufrichten, das nimmermehr zerstöret wird; und sein Königreich
wird auf kein ander Volkkommen. Es wird alle diese Königreiche zermalmen
und verstören, aber es wird ewiglichbleiben. \bibverse{45} Wie du denn
gesehen hast, einen Stein ohne Hände vom Bergeherabgerissen, der das
Eisen, Erz, Ton, Silber und Gold zermalmet. Also hat der großeGOtt dem
Könige gezeiget, wie es hernach gehen werde; und das ist gewiß der
Traum,und die Deutung ist recht. \bibverse{46} Da fiel der König
Nebukadnezar auf sein Angesicht und betete an vor demDaniel und befahl,
man sollte ihm Speisopfer und Räuchopfer tun. \bibverse{47} Und der
König antwortete Daniel und sprach: Es ist kein Zweifel, euer GOttist
ein GOtt über alle Götter und ein HErr über alle Könige, der da kann
verborgeneDinge offenbaren, weil du dies verborgene Ding hast können
offenbaren. \bibverse{48} Und der König erhöhete Daniel und gab ihm
große und viele Geschenke undmachte ihn zum Fürsten über das ganze Land
zu Babel und setzte ihn zum Oberstenüber alle Weisen zu Babel.
\bibverse{49} Und Daniel bat vom Könige, daß er über die Landschaften zu
Babel setzenmöchte Sadrach, Mesach, Abed-Nego; und er, Daniel, blieb bei
dem Könige zu Hofe.

\hypertarget{section-2}{%
\section{3}\label{section-2}}

\bibverse{1} Der König Nebukadnezar ließ ein gülden Bild machen, sechzig
Ellen hoch undsechs Ellen breit, und ließ es setzen im Lande zu Babel im
Tal Dura. \bibverse{2} Und der König Nebukadnezar sandte nach den
Fürsten, Herren,Landpflegern, Richtern, Vögten, Räten, Amtleuten und
allen Gewaltigen im Lande, daßsie zusammenkommen sollten, das Bild zu
weihen, das der König Nebukadnezar hattesetzen lassen. \bibverse{3} Da
kamen zusammen die Fürsten, Herren, Landpfleger, Richter, Vögte,
Räte,Amtleute und alle Gewaltigen im Lande, das Bild zu weihen, das der
König Nebukadnezarhatte setzen lassen. Und sie mußten vor das Bild
treten, das Nebukadnezar hatte setzenlassen. \bibverse{4} Und der
Ehrenhold rief überlaut: Das laßt euch gesagt sein, ihr Völker, Leuteund
Zungen: \bibverse{5} Wenn ihr hören werdet den Schall der Posaunen,
Trommeten, Harfen,Geigen, Psalter, Lauten und allerlei Saitenspiel, so
sollt ihr niederfallen und das güldeneBild anbeten, das der König
Nebukadnezar hat setzen lassen. \bibverse{6} Wer aber alsdann nicht
niederfällt und anbetet, der soll von Stund an in denglühenden Ofen
geworfen werden. \bibverse{7} Da sie nun höreten den Schall der
Posaunen, Trommeten, Harfen, Geigen,Psalter und allerlei Saitenspiel,
fielen nieder alle Völker, Leute und Zungen und betetenan das güldene
Bild, das der König Nebukadnezar hatte setzen lassen. \bibverse{8} Von
Stund an traten hinzu etliche chaldäische Männer und verklagten
dieJuden, \bibverse{9} fingen an und sprachen zum Könige Nebukadnezar:
Herr König, GOttverleihe dir langes Leben! \bibverse{10} Du hast ein
Gebot lassen ausgehen, daß alle Menschen, wenn sie hörenwürden den
Schall der Posaunen, Trommeten, Harfen, Geigen, Psalter, Lauten
undallerlei Saitenspiel, sollten sie niederfallen und das güldene Bild
anbeten; \bibverse{11} wer aber nicht niederfiele und anbetete, sollte
in einen glühenden Ofengeworfen werden. \bibverse{12} Nun sind da
jüdische Männer, welche du über die Ämter im Lande zu Babelgesetzet
hast: Sadrach, Mesach und Abed-Nego; dieselbigen verachten dein Gebot
undehren deine Götter nicht und beten nicht an das güldene Bild, das du
hast setzen lassen. \bibverse{13} Da befahl Nebukadnezar mit Grimm und
Zorn, daß man vor ihn stelleteSadrach, Mesach und Abed-Nego. Und die
Männer wurden vor den König gestellet. \bibverse{14} Da fing
Nebukadnezar an und sprach zu ihnen: Wie? wollt ihr, Sadrach,Mesach,
Abed-Nego, meinen Gott nicht ehren und das güldene Bild nicht anbeten,
dasich habe setzen lassen? \bibverse{15} Wohlan, schicket euch! Sobald
ihr hören werdet den Schall der Posaunen,Trommeten, Harfen, Geigen,
Psalter, Lauten und allerlei Saitenspiel, so fallet nieder undbetet das
Bild an, das ich habe machen lassen! Werdet ihr's nicht anbeten, so
sollt ihrvon Stund an in den glühenden Ofen geworfen werden. Laßt sehen,
wer der GOtt sei,der euch aus meiner Hand erretten werde! \bibverse{16}
Da fingen an Sadrach, Mesach; Abed-Nego und sprachen zum
KönigeNebukadnezar: Es ist nicht not, daß wir dir darauf antworten.
\bibverse{17} Siehe, unser GOtt, den wir ehren, kann uns wohl erretten
aus demglühenden Ofen, dazu auch von deiner Hand erretten. \bibverse{18}
Und wo er's nicht tun will, so sollst du dennoch wissen, daß wir
deineGötter nicht ehren, noch das güldene Bild, das du hast setzen
lassen, anbeten wollen. \bibverse{19} Da ward Nebukadnezar voll Grimms
und stellete sich scheußlich widerSadrach, Mesach und Abed-Nego und
befahl, man sollte den Ofen siebenmal heißermachen, denn man sonst zu
tun pflegte. \bibverse{20} Und befahl den besten Kriegsleuten, die in
seinem Heer waren, daß sieSadrach, Mesach und Abed-Nego bänden und in
den glühenden Ofen würfen. \bibverse{21} Also wurden diese Männer in
ihren Mänteln, Schuhen, Hüten und andernKleidern gebunden und in den
glühenden Ofen geworfen. \bibverse{22} Denn des Königs Gebot mußte man
eilend tun. Und man schürete das Feuerim Ofen so sehr, daß die Männer,
so den Sadrach, Mesach und Abed-Nego verbrennensollten, verdarben von
des Feuers Flammen. \bibverse{23} Aber die drei Männer Sadrach, Mesach
und Abed-Nego, fielen hinab in denglühenden Ofen, wie sie gebunden
waren. \bibverse{24} Da entsetzte sich der König Nebukadnezar und fuhr
eilends auf und sprachzu seinen Räten: Haben wir nicht drei Männer
gebunden in das Feuer lassen werfen? Sieantworteten und sprachen zum
Könige: Ja, Herr König! \bibverse{25} Er antwortete und sprach: Sehe ich
doch vier Männer los im Feuer gehen,und sind unversehrt; und der vierte
ist gleich, als wäre er ein Sohn der Götter. \bibverse{26} Und
Nebukadnezar trat hinzu vor das Loch des glühenden Ofens undsprach:
Sadrach, Mesach, Abed-Nego, ihr Knechte GOttes des Höchsten, gehet
herausund kommt her! Da gingen Sadrach, Mesach und Abed-Nego heraus aus
dem Feuer. \bibverse{27} Und die Fürsten, Herren, Vögte und Räte des
Königs kamen zusammen undsahen, daß das Feuer keine Macht am Leibe
dieser Männer beweiset hatte, und ihrHaupthaar nicht versenget und ihre
Mäntel nicht versehrt waren; ja, man konnte keinenBrand an ihnen
riechen. \bibverse{28} Da fing an Nebukadnezar und sprach: Gelobet sei
der GOtt Sadrachs,Mesachs und Abed-Negos, der seinen Engel gesandt und
seine Knechte errettet hat, dieihm vertrauet und des Königs Gebot nicht
gehalten, sondern ihren Leib dargegebenhaben, daß sie keinen Gott ehren
noch anbeten wollten ohne allein ihren GOtt. \bibverse{29} So sei nun
dies mein Gebot: Welcher unter allen Völkern, Leuten undZungen den GOtt
Sadrachs, Mesachs und Abed-Negos lästert, der soll umkommen, undsein
Haus schändlich verstöret werden. Denn es ist kein anderer GOtt, der
also errettenkann als dieser. \bibverse{30} Und der König gab Sadrach,
Mesach und Abed-Nego große Gewalt im Landezu Babel.

\hypertarget{section-3}{%
\section{4}\label{section-3}}

\bibverse{1} König Nebukadnezar allen Völkern, Leuten und Zungen: GOtt
gebe euch vielFriede! \bibverse{2} Ich sehe es für gut an, daß ich
verkündige die Zeichen und Wunder, so GOttder Höchste an mir getan hat.
\bibverse{3} Denn seine Zeichen sind groß, und seine Wunder sind
mächtig; und seinReich ist ein ewiges Reich, und seine Herrschaft währet
für und für. \bibverse{4} Ich, Nebukadnezar, da ich gute Ruhe hatte in
meinem Hause, und es wohlstund auf meiner Burg, \bibverse{5} sah ich
einen Traum und erschrak, und die Gedanken, die ich auf meinemBette
hatte über dem Gesichte, so ich gesehen hatte, betrübten mich.
\bibverse{6} Und ich befahl, daß alle Weisen zu Babel vor mich
heraufgebracht würden,daß sie mir sageten, was der Traum bedeutete.
\bibverse{7} Da brachte man herauf die Sternseher, Weisen, Chaldäer und
Wahrsager,und ich erzählte den Traum vor ihnen; aber sie konnten mir
nicht sagen, was erbedeutete, \bibverse{8} bis zuletzt Daniel vor mich
kam, welcher Beltsazar heißt, nach dem Namenmeines Gottes, der den Geist
der heiligen Götter hat. Und ich erzählte vor ihm denTraum: \bibverse{9}
Beltsazar, du Oberster unter den Sternsehern, welchen ich weiß, daß du
denGeist der heiligen Götter hast und dir nichts verborgen ist, sage das
Gesicht meinesTraums, den ich gesehen habe, und was er bedeutet.
\bibverse{10} Dies ist aber das Gesicht, das ich gesehen habe auf meinem
Bette: Siehe,es stund ein Baum mitten im Lande, der war sehr hoch,
\bibverse{11} groß und dick; seine Höhe reichte bis in Himmel und
breitete sich aus bisans Ende des ganzen Landes. \bibverse{12} Seine
Äste waren schön und trugen viel Früchte, davon alles zu essenhatte.
Alle Tiere auf dem Felde fanden Schatten unter ihm, und die Vögel unter
demHimmel saßen auf seinen Ästen, und alles Fleisch nährete sich von
ihm. \bibverse{13} Und ich sah ein Gesicht auf meinem Bette, und siehe,
ein heiliger Wächterfuhr vom Himmel herab, \bibverse{14} der rief
überlaut und sprach also: Hauet den Baum um und behauet ihm dieÄste und
streifet ihm das Laub ab und zerstreuet seine Früchte, daß die Tiere, so
unterihm liegen, weglaufen, und die Vögel von seinen Zweigen fliehen.
\bibverse{15} Doch laß den Stock mit seinen Wurzeln in der Erde bleiben;
er aber soll ineisernen und ehernen Ketten auf dem Felde im Grase gehen;
er soll unter dem Tau desHimmels liegen und naß werden und soll sich
weiden mit den Tieren von den Kräuternder Erde. \bibverse{16} Und das
menschliche Herz soll von ihm genommen und ein viehisch Herzihm gegeben
werden, bis daß sieben Zeiten über ihm um sind. \bibverse{17} Solches
ist im Rat der Wächter beschlossen und im Gespräch der
Heiligenberatschlaget, auf daß die Lebendigen erkennen, daß der Höchste
Gewalt hat über derMenschen Königreiche und gibt sie, wem er will, und
erhöhet die Niedrigen zudenselbigen. \bibverse{18} Solchen Traum habe
ich, König Nebukadnezar, gesehen. Du aber,Beltsazar, sage, was er
bedeute; denn alle Weisen in meinem Königreich können mirnicht anzeigen,
was er bedeute; du aber kannst es wohl, denn der Geist der
heiligenGötter ist bei dir. \bibverse{19} Da entsetzte sich Daniel, der
sonst Beltsazar heißt, bei einer Stunde lang,und seine Gedanken
betrübten ihn. Aber der König sprach: Beltsazar, laß dich denTraum und
seine Deutung nicht betrüben! Beltsazar fing an und sprach: Ach, mein
Herr,daß der Traum deinen Feinden und seine Deutung deinen Widerwärtigen
gälte! \bibverse{20} Der Baum, den du gesehen hast, daß er groß und dick
war und seine Höhean den Himmel reichte und breitete sich über das ganze
Land, \bibverse{21} und seine Äste schön und seiner Früchte viel, davon
alles zu essen hatte,und die Tiere auf dem Felde unter ihm wohneten, und
die Vögel des Himmels auf seinenÄsten saßen: \bibverse{22} das bist du,
König der du so groß und mächtig bist; denn deine Macht istgroß und
reichet an den Himmel, und deine Gewalt langet bis an der Welt Ende.
\bibverse{23} Daß aber der König einen heiligen Wächter gesehen hat vom
Himmelherabfahren und sagen: Hauet den Baum um und verderbet ihn, doch
den Stock mitseinen Wurzeln laßt in der Erde bleiben; er aber soll in
eisernen und ehernen Ketten aufdem Felde im Grase gehen und unter dem
Tau des Himmels liegen und naß werden undsich mit den Tieren auf dem
Felde weiden, bis über ihm sieben Zeiten um sind: \bibverse{24} das ist
die Deutung, Herr König, und solcher Rat des Höchsten gehet übermeinen
Herrn König. \bibverse{25} Man wird dich von den Leuten verstoßen, und
mußt bei den Tieren auf demFelde bleiben; und man wird dich Gras essen
lassen wie die Ochsen; und wirst unterdem Tau des Himmels liegen und naß
werden, bis über dir sieben Zeiten um sind, aufdaß du erkennest, daß der
Höchste Gewalt hat über der Menschen Königreiche und gibtsie, wem er
will. \bibverse{26} Daß aber gesagt ist, man solle dennoch den Stock mit
seinen Wurzeln desBaums bleiben lassen: dein Königreich soll dir
bleiben, wenn du erkannt hast die Gewaltim Himmel. \bibverse{27} Darum,
Herr König, laß dir meinen Rat gefallen und mache dich los vondeinen
Sünden durch Gerechtigkeit und ledig von deiner Missetat durch Wohltat
an denArmen, so wird er Geduld haben mit deinen Sünden. \bibverse{28}
Dies alles widerfuhr dem Könige Nebukadnezar. \bibverse{29} Denn nach
zwölf Monden, da der König auf der königlichen Burg zu Babelging,
\bibverse{30} hub er an und sprach: Das ist die große Babel, die ich
erbauet habe zumköniglichen Hause durch meine große Macht, zu Ehren
meiner Herrlichkeit. \bibverse{31} Ehe der König diese Worte ausgeredet
hatte, fiel eine Stimme vom Himmel:Dir, König Nebukadnezar, wird gesagt:
Dein Königreich soll dir genommen werden, \bibverse{32} und man wird
dich von den Leuten verstoßen, und sollst bei den Tieren, soauf dem
Felde gehen, bleiben; Gras wird man dich essen lassen, wie Ochsen, bis
daßüber dir sieben Zeiten um sind, auf daß du erkennest, daß der Höchste
Gewalt hat überder Menschen Königreiche und gibt sie, wem er will.
\bibverse{33} Von Stund an ward das Wort vollbracht über Nebukadnezar,
und er wardvon den Leuten verstoßen und er aß Gras wie Ochsen, und sein
Leib lag unter dem Taudes Himmels und ward naß, bis sein Haar wuchs, so
groß als Adlersfedern, und seineNägel wie Vogelklauen wurden.
\bibverse{34} Nach dieser Zeit hub ich, Nebukadnezar, meine Augen auf
gen Himmel undkam wieder zur Vernunft und lobte den Höchsten. Ich
preisete und ehrete den, soewiglich lebet, des Gewalt ewig ist und sein
Reich für und für währet, \bibverse{35} gegen welchen alle, so auf Erden
wohnen, als nichts zu rechnen sind. Ermacht es, wie er will, beide, mit
den Kräften im Himmel und mit denen, so auf Erdenwohnen; und niemand
kann seiner Hand wehren noch zu ihm sagen: Was machst du? \bibverse{36}
Zur selbigen Zeit kam ich wieder zur Vernunft, auch zu meinen
königlichenEhren, zu meiner Herrlichkeit und zu meiner Gestalt. Und
meine Räte und Gewaltigensuchten mich; und ward wieder in mein
Königreich gesetzt; und ich überkam nochgrößere Herrlichkeit.
\bibverse{37} Darum lobe ich, Nebukadnezar, und ehre und preise den
König vomHimmel. Denn all sein Tun ist Wahrheit, und seine Wege sind
recht; und wer stolz ist,den kann er demütigen.

\hypertarget{section-4}{%
\section{5}\label{section-4}}

\bibverse{1} König Belsazer machte ein herrlich Mahl tausend seinen
Gewaltigen undHauptleuten und soff sich voll mit ihnen. \bibverse{2} Und
da er trunken war, hieß er die güldenen und silbernen Gefäßeherbringen,
die sein Vater Nebukadnezar aus dem Tempel zu Jerusalem
weggenommenhatte, daß der König mit seinen Gewaltigen, mit seinen
Weibern und mit seinenKebsweibern daraus tränken. \bibverse{3} Also
wurden hergebracht die güldenen Gefäße, die aus dem Tempel, ausdem Hause
GOttes zu Jerusalem, genommen wären; und der König, seine
Gewaltigen,seine Weiber und Kebsweiber tranken daraus. \bibverse{4} Und
da sie so soffen, lobten sie die güldenen, silbernen, ehernen,
eisernen,hölzernen und steinernen Götter. \bibverse{5} Eben zur selbigen
Stunde gingen hervor Finger, als einer Menschenhand, dieschrieben,
gegenüber dem Leuchter, auf die getünchte Wand in dem königlichen
Saal.Und der König ward gewahr der Hand, die da schrieb. \bibverse{6} Da
entfärbte sich der König, und seine Gedanken erschreckten ihn, daß
ihmdie Lenden schütterten und die Beine zitterten. \bibverse{7} Und der
König rief überlaut, daß man die Weisen, Chaldäer und
Wahrsagerheraufbringen sollte Und ließ den Weisen zu Babel sagen:
Welcher Mensch diese Schriftlieset und sagen kann, was sie bedeute, der
soll mit Purpur gekleidet werden undgüldene Ketten am Halse tragen und
der dritte Herr sein in meinem Königreiche. \bibverse{8} Da wurden alle
Weisen des Königs heraufgebracht; aber sie konnten wederdie Schrift
lesen noch die Deutung dem Könige anzeigen. \bibverse{9} Des erschrak
der König Belsazer noch härter und verlor ganz seine Gestalt,und seinen
Gewaltigen ward bange. \bibverse{10} Da ging die Königin um solcher
Sache willen des Königs und seinerGewaltigen hinauf in den Saal und
sprach: Herr König, GOtt verleihe dir langes Leben!Laß dich deine
Gedanken nicht so erschrecken und entfärbe dich nicht also!
\bibverse{11} Es ist ein Mann in deinem Königreich, der den Geist der
heiligen Götter hat.Denn zu deines Vaters Zeit ward bei ihm Erleuchtung
erfunden, Klugheit und Weisheit,wie der Götter Weisheit ist; und dein
Vater, König Nebukadnezar, setzte ihn über dieSternseher, Weisen,
Chaldäer und Wahrsager, \bibverse{12} darum daß ein hoher Geist bei ihm
funden ward, dazu Verstand undKlugheit, Sprüche zu deuten, dunkle
Sprüche zu erraten und verborgene Sachen zuoffenbaren, nämlich Daniel,
den der König ließ Beltsazar nennen. So rufe man nunDaniel; der wird
sagen, was es bedeute. \bibverse{13} Da ward Daniel hinauf vor den König
gebracht. Und der König sprach zuDaniel: Bist du der Daniel, der
Gefangenen einer aus Juda, die der König, mein Vater,aus Juda
hergebracht hat? \bibverse{14} Ich habe von dir hören sagen, daß du den
Geist der heiligen Götter habest,und Erleuchtung, Verstand und hohe
Weisheit bei dir funden sei. \bibverse{15} Nun hab ich vor mich fordern
lassen die Klugen und Weisen, daß sie mirdiese Schrift lesen und
anzeigen sollen, was sie bedeute; und sie können mir nichtsagen, was
solches bedeute. \bibverse{16} Von dir aber höre ich, daß du könnest die
Deutung geben und dasVerborgene offenbaren. Kannst du nun die Schrift
lesen und mir anzeigen, was siebedeutet, so sollst du mit Purpur
gekleidet werden und güldene Ketten an deinem Halsetragen und der dritte
Herr sein in meinem Königreiche. \bibverse{17} Da fing Daniel an und
redete vor dem Könige: Behalte deine Gaben selbstund gib dein Geschenk
einem andern; ich will dennoch die Schrift dem Könige lesen undanzeigen,
was sie bedeute. \bibverse{18} Herr König, GOtt der Höchste hat deinem
Vater, Nebukadnezar, Königreich,Macht, Ehre und Herrlichkeit gegeben.
\bibverse{19} Und vor solcher Macht, die ihm gegeben war, fürchteten und
scheueten sichvor ihm alle Völker, Leute und Zungen. Er tötete, wen er
wollte; er schlug, wen erwollte; er erhöhete, wen er wollte; er
demütigte, wen er wollte. \bibverse{20} Da sich aber sein Herz erhub und
er stolz und hochmütig ward, ward ervom königlichen Stuhl gestoßen und
verlor seine Ehre; \bibverse{21} und ward verstoßen von den Leuten, und
sein Herz ward gleich den Tieren,und mußte bei dem Wild laufen und fraß
Gras wie Ochsen, und sein Leib lag unter demTau des Himmels und ward
naß, bis daß er lernete, daß GOtt der Höchste Gewalt hatüber der
Menschen Königreiche und gibt sie, wem er will. \bibverse{22} Und du,
Belsazer, sein Sohn, hast dein Herz nicht gedemütiget, ob du wohlsolches
alles weißt, \bibverse{23} sondern hast dich wider den HErrn des Himmels
erhoben, und die Gefäßeseines Hauses hat man vor dich bringen müssen;
und du, deine Gewaltigen, deineWeiber und deine Kebsweiber habt daraus
gesoffen, dazu die silbernen, güldenen,ehernen, eisernen, hölzernen,
steinernen Götter gelobet, die weder sehen, noch hören,noch fühlen; den
GOtt aber, der deinen Odem und alle deine Wege in seiner Hand hat,hast
du nicht geehret. \bibverse{24} Darum ist von ihm gesandt diese Hand und
diese Schrift, die da verzeichnetstehen. \bibverse{25} Das ist aber die
Schrift allda verzeichnet: Mene, mene, tekel, upharsin. \bibverse{26}
Und sie bedeutet dies: Mene, das ist, GOtt hat dein Königreich gezählet
undvollendet. \bibverse{27} Tekel, das ist, man hat dich in einer Waage
gewogen und zu leicht funden. \bibverse{28} Peres, das ist, dein
Königreich ist zerteilet und den Medern und Perserngegeben.
\bibverse{29} Da befahl Belsazer, daß man Daniel mit Purpur kleiden
sollte und güldeneKetten an den Hals geben; und ließ von ihm
verkündigen, daß er der dritte Herr sei imKönigreich. \bibverse{30} Aber
des Nachts ward der Chaldäer König Belsazer getötet. \bibverse{31} Und
Darius aus Medien nahm das Reich ein, da er zweiundsechzig Jahre altwar.

\hypertarget{section-5}{%
\section{6}\label{section-5}}

\bibverse{1} Und Darius sah es für gut an, daß er über das ganze
Königreich setztehundertundzwanzig Landvögte. \bibverse{2} Über diese
setzte er drei Fürsten, deren einer war Daniel, welchen dieLandvögte
sollten Rechnung tun, und der König der Mühe überhoben wäre.
\bibverse{3} Daniel aber übertraf die Fürsten und Landvögte alle, denn
es war ein hoherGeist in ihm; darum gedachte der König ihn über das
ganze Königreich zu setzen. \bibverse{4} Derhalben trachteten die
Fürsten und Landvögte danach, wie sie eine Sachezu Daniel fänden, die
wider das Königreich wäre; aber sie konnten keine Sache nochÜbeltat
finden, denn er war treu, daß man keine Schuld noch Übeltat an ihm
findenmochte. \bibverse{5} Da sprachen die Männer: Wir werden keine
Sache zu Daniel finden ohne überseinem Gottesdienst. \bibverse{6} Da
kamen die Fürsten und Landvögte häufig vor den König und sprachen zuihm
also: Herr König Darius, GOtt verleihe dir langes Leben! \bibverse{7} Es
haben die Fürsten des Königreichs, die Herren, die Landvögte, die
Räteund Hauptleute alle gedacht, daß man einen königlichen Befehl solle
ausgehen lassenund ein streng Gebot stellen, daß, wer in dreißig Tagen
etwas bitten wird vonirgendeinem Gott oder Menschen ohne von dir, König,
alleine, solle zu den Löwen in denGraben geworfen werden. \bibverse{8}
Darum, lieber König, sollst du solch Gebot bestätigen und
dichunterschreiben, auf daß nicht wieder geändert werde, nach dem Recht
der Meder undPerser, welches niemand übertreten darf. \bibverse{9} Also
unterschrieb sich der König Darius. \bibverse{10} Als nun Daniel erfuhr,
daß solch Gebot unterschrieben wäre, ging er hinaufin sein Haus (er
hatte aber an seinem Sommerhause offene Fenster gegen Jerusalem).Und er
fiel des Tages dreimal auf seine Kniee, betete, lobte und dankte seinem
GOtt, wieer denn vorhin zu tun pflegte. \bibverse{11} Da kamen diese
Männer häufig und fanden Daniel beten und flehen vorseinem GOtt.
\bibverse{12} Und traten hinzu und redeten mit dem Könige von dem
königlichen Gebot:Herr König, hast du nicht ein Gebot unterschrieben,
daß, wer in dreißig Tagen etwasbitten würde von irgendeinem Gott oder
Menschen ohne von dir, König, alleine, solle zuden Löwen in den Graben
geworfen werden? Der König antwortete und sprach: Es istwahr, und das
Recht der Meder und Perser soll niemand übertreten. \bibverse{13} Sie
antworteten und sprachen vor dem Könige: Daniel, der Gefangenen ausJuda
einer, der achtet weder dich noch dein Gebot, das du verzeichnet hast;
denn erbetet des Tages dreimal. \bibverse{14} Da der König solches
hörete, ward er sehr betrübt und tat großen Fleiß, daßer Daniel
erlösete, und mühete sich, bis die Sonne unterging, daß er ihn
errettete. \bibverse{15} Aber die Männer kamen häufig zu dem Könige und
sprachen zu ihm: Duweißt, Herr König, daß der Meder und Perser Recht
ist, daß alle Gebote und Befehle, soder König beschlossen hat, sollen
unverändert bleiben. \bibverse{16} Da befahl der König, daß man Daniel
herbrächte; und warfen ihn zu denLöwen in den Graben. Der König aber
sprach zu Daniel: Dein GOtt, dem du ohneUnterlaß dienest, der helfe dir!
\bibverse{17} Und sie brachten einen Stein, den legten sie vor die Tür
am Graben; denversiegelte der König mit seinem eigenen Ringe und mit dem
Ringe seiner Gewaltigen,auf daß sonst niemand an Daniel Mutwillen übete.
\bibverse{18} Und der König ging weg in seine Burg und blieb ungegessen
und ließ keinEssen vor sich bringen, konnte auch nicht schlafen.
\bibverse{19} Des Morgens früh, da der Tag anbrach, stund der König auf
und ging eilendzum Graben, da die Löwen waren. \bibverse{20} Und als er
zum Graben kam, rief er Daniel mit kläglicher Stimme. Und derKönig
sprach zu Daniel: Daniel, du Knecht des lebendigen GOttes, hat dich auch
deinGOtt, dem du ohn Unterlaß dienest, mögen von den Löwen erlösen?
\bibverse{21} Daniel aber redete mit dem Könige: Herr König, GOtt
verleihe dir langesLeben! \bibverse{22} Mein GOtt hat seinen Engel
gesandt, der den Löwen den Rachen zugehaltenhat, daß sie mir kein Leid
getan haben. Denn vor ihm bin ich unschuldig erfunden, sohabe ich auch
wider dich, Herr König, nichts getan. \bibverse{23} Da ward der König
sehr froh und ließ Daniel aus dem Graben ziehen. Undsie zogen Daniel aus
dem Graben, und man spürete keinen Schaden an ihm; denn erhatte seinem
GOtt vertrauet. \bibverse{24} Da hieß der König die Männer, so Daniel
verklagt hatten, herbringen und zuden Löwen in den Graben werfen samt
ihren Kindern und Weibern. Und ehe sie auf denBoden hinab kamen,
ergriffen sie die Löwen und zermalmeten auch ihre Gebeine. \bibverse{25}
Da ließ der König Darius schreiben allen Völkern, Leuten und Zungen:
GOttgebe euch viel Frieden! \bibverse{26} Das ist mein Befehl, daß man
in der ganzen Herrschaft meines Königreichsden GOtt Daniels fürchten und
scheuen soll. Denn er ist der lebendige GOtt, der ewiglichbleibet; und
sein Königreich ist unvergänglich, und seine Herrschaft hat kein Ende.
\bibverse{27} Er ist ein Erlöser und Nothelfer, und er tut Zeichen und
Wunder, beide, imHimmel und auf Erden. Der hat Daniel von den Löwen
erlöset. \bibverse{28} Und Daniel ward gewaltig im Königreich Darius und
auch im KönigreichKores, der Perser.

\hypertarget{section-6}{%
\section{7}\label{section-6}}

\bibverse{1} Im ersten Jahr Belsazers, des Königs zu Babel, hatte Daniel
einen Traum undGesicht auf seinem Bette; und er schrieb denselbigen
Traum und verfaßte ihn also: \bibverse{2} Ich, Daniel, sah ein Gesicht
in der Nacht, und siehe, die vier Winde unterdem Himmel stürmeten
widereinander auf dem großen Meer. \bibverse{3} Und vier große Tiere
stiegen herauf aus dem Meer, eins je anders denn dasandere. \bibverse{4}
Das erste wie ein Löwe und hatte Flügel wie ein Adler. Ich sah zu, bis
daßihm die Flügel ausgerauft wurden; und es ward von der Erde genommen
und es stundauf seinen Füßen wie ein Mensch, und ihm ward ein menschlich
Herz gegeben. \bibverse{5} Und siehe, das andere Tier hernach war gleich
einem Bären und stund aufder einen Seite und hatte in seinem Maul unter
seinen Zähnen drei große lange Zähne.Und man sprach zu ihm: Stehe auf
und friß viel Fleisch! \bibverse{6} Nach diesem sah ich, und siehe, ein
ander Tier, gleich einem Parden, dashatte vier Flügel, wie ein Vogel,
auf seinem Rücken; und dasselbige Tier hatte vierKöpfe, und ihm ward
Gewalt gegeben. \bibverse{7} Nach diesem sah ich in diesem Gesicht in
der Nacht, und siehe, das vierteTier war greulich und schrecklich und
sehr stark und hatte große eiserne Zähne, fraß umsich und zermalmete,
und das übrige zertrat es mit seinen Füßen; es war auch vielanders denn
die vorigen und hatte zehn Hörner. \bibverse{8} Da ich aber die Hörner
schauete, siehe, da brach hervor zwischendenselbigen ein ander klein
Horn, vor welchem der vordersten Hörner drei ausgerissenwurden; und
siehe, dasselbige Horn hatte Augen wie Menschenaugen und ein Maul,
dasredete große Dinge. \bibverse{9} Solches sah ich, bis daß Stühle
gesetzt wurden; und der Alte setzte sich, desKleid war schneeweiß und
das Haar auf seinem Haupt wie reine Wolle; sein Stuhl wareitel
Feuerflammen, und desselbigen Räder brannten mit Feuer. \bibverse{10}
Und von demselbigen ging aus ein langer feuriger Strahl.
Tausendmaltausend dieneten ihm, und zehntausendmal zehntausend stunden
vor ihm. Das Gerichtward gehalten, und die Bücher wurden aufgetan.
\bibverse{11} Ich sah zu um der großen Rede willen, so das Horn redete;
ich sah zu, bisdas Tier getötet ward und sein Leib umkam und ins Feuer
geworfen ward, \bibverse{12} und der andern Tiere Gewalt auch aus war;
denn es war ihnen Zeit undStunde bestimmt, wie lange ein jegliches
währen sollte. \bibverse{13} Ich sah in diesem Gesichte des Nachts, und
siehe, es kam einer in desHimmels Wolken wie eines Menschen Sohn bis zu
dem Alten und ward vor denselbigengebracht. \bibverse{14} Der gab ihm
Gewalt, Ehre und Reich, daß ihm alle Völker, Leute undZungen dienen
sollten. Seine Gewalt ist ewig, die nicht vergehet, und sein
Königreichhat kein Ende. \bibverse{15} Ich, Daniel, entsetzte mich
davor, und solch Gesicht erschreckte mich. \bibverse{16} Und ich ging zu
deren einem, die da stunden, und bat ihn, daß er mir vondem allem
gewissen Bericht gäbe. Und er redete mit mir und zeigte mir, was
esbedeutete. \bibverse{17} Diese vier großen Tiere sind vier Reiche, so
auf Erden kommen werden. \bibverse{18} Aber die Heiligen des Höchsten
werden das Reich einnehmen und werdenes immer und ewiglich besitzen.
\bibverse{19} Danach hätte ich gerne gewußt gewissen Bericht von dem
vierten Tier,welches gar anders war denn die andern alle, sehr greulich,
das eiserne Zähne undeherne Klauen hatte, das um sich fraß und
zermalmete und das übrige mit seinen Füßenzertrat, \bibverse{20} und von
den zehn Hörnern auf seinem Haupt und von dem andern, dashervorbrach,
vor welchem drei abfielen, und von demselbigen Horn, das Augen hatteund
ein Maul, das große Dinge redete und größer war, denn die neben ihm
waren. \bibverse{21} Und ich sah dasselbige Horn streiten wider die
Heiligen und behielt den Siegwider sie, \bibverse{22} bis der Alte kam
und Gericht hielt für die Heiligen des Höchsten; und dieZeit kam, daß
die Heiligen das Reich einnahmen. \bibverse{23} Er sprach also: Das
vierte Tier wird das vierte Reich auf Erden sein, welcheswird mächtiger
sein denn alle Reiche; es wird alle Lande fressen, zertreten
undzermalmen. \bibverse{24} Die zehn Hörner bedeuten zehn Könige, so aus
demselbigen Reichentstehen werden. Nach demselben aber wird ein anderer
aufkommen, der wirdmächtiger sein denn der vorigen keiner und wird drei
Könige demütigen. \bibverse{25} Er wird den Höchsten lästern und die
Heiligen des Höchsten verstören undwird sich unterstehen, Zeit und
Gesetz zu ändern. Sie werden aber in seine Handgegeben werden eine Zeit
und etliche Zeiten und eine halbe Zeit. \bibverse{26} Danach wird das
Gericht gehalten werden; da wird dann seine Gewaltweggenommen werden,
daß er zugrunde vertilget und umgebracht werde. \bibverse{27} Aber das
Reich, Gewalt und Macht unter dem ganzen Himmel wird demheiligen Volk
des Höchsten gegeben werden, des Reich ewig ist, und alle Gewalt wirdihm
dienen und gehorchen. \bibverse{28} Das war der Rede Ende. Aber ich,
Daniel, ward sehr betrübt in meinenGedanken, und meine Gestalt verfiel;
doch behielt ich die Rede in meinem Herzen.

\hypertarget{section-7}{%
\section{8}\label{section-7}}

\bibverse{1} Im dritten Jahr des Königreichs des Königs Belsazer
erschien mir, Daniel, einGesicht nach dem, so mir am ersten erschienen
war. \bibverse{2} Ich war aber, da ich solch Gesicht sah, zu Schloß
Susan im Lande Elam amWasser Ulai. \bibverse{3} Und ich hub meine Augen
auf und sah, und siehe, ein Widder stund vor demWasser, der hatte zwei
hohe Hörner, doch eins höher denn das andere, und das höchstewuchs am
letzten. \bibverse{4} Ich sah, daß der Widder mit den Hörnern stieß
gegen Abend, gegenMitternacht und gegen Mittag, und kein Tier konnte vor
ihm bestehen noch von seinerHand errettet werden, sondern er tat, was er
wollte, und ward groß. \bibverse{5} Und indem ich darauf merkte, siehe,
so kommt ein Ziegenbock vom Abendher über die ganze Erde, daß er die
Erde nicht rührete; und der Bock hatte einansehnlich Horn zwischen
seinen Augen. \bibverse{6} Und er kam bis zu dem Widder, der zwei Hörner
hatte, den ich stehen sahvor dem Wasser; und er lief in seinem Zorn
gewaltiglich zu ihm zu. \bibverse{7} Und ich sah ihm zu, daß er hart an
den Widder kam, und ergrimmete überihn und stieß den Widder und zerbrach
ihm seine zwei Hörner. Und der Widder hattekeine Kraft, daß er vor ihm
hätte mögen bestehen, sondern er warf ihn zu Boden undzertrat ihn; und
niemand konnte den Widder von seiner Hand erretten. \bibverse{8} Und der
Ziegenbock ward sehr groß. Und da er aufs stärkste worden war,zerbrach
das große Horn; und wuchsen an des Statt ansehnliche vier gegen die
vierWinde des Himmels. \bibverse{9} Und aus derselbigen einem wuchs ein
klein Horn, das ward sehr groß gegenMittag, gegen Morgen und gegen das
werte Land. \bibverse{10} Und es wuchs bis an des Himmels Heer und warf
etliche davon und von denSternen zur Erde und zertrat sie. \bibverse{11}
Ja, es wuchs bis an den Fürsten des Heers und nahm von ihm weg
dastägliche Opfer und verwüstete die Wohnung seines Heiligtums.
\bibverse{12} Es ward ihm aber solche Macht gegeben wider das tägliche
Opfer um derSünde willen, daß er die Wahrheit zu Boden schlüge und, was
er tat, ihm gelingenmußte. \bibverse{13} Ich hörete aber einen Heiligen
reden; und derselbige Heilige sprach zueinem, der da redete: Wie lange
soll doch währen solch Gesicht vom täglichen Opfer undvon der Sünde, um
welcher willen diese Verwüstung geschieht, daß beide, das Heiligtumund
das Heer, zertreten werden? \bibverse{14} Und er antwortete mir: Es sind
zweitausend und dreihundert Tage, vonAbend gegen Morgen zu rechnen, so
wird das Heiligtum wieder geweihet werden. \bibverse{15} Und da ich,
Daniel, solch Gesicht sah und hätte es gerne verstanden, siehe,da stund
es vor mir wie ein Mann. \bibverse{16} Und ich hörete zwischen Ulai
eines Menschen Stimme, der rief und sprach:Gabriel, lege diesem das
Gesicht aus, daß er's verstehe! \bibverse{17} Und er kam hart zu mir.
Ich erschrak aber, da er kam, und fiel auf meinAngesicht. Er aber sprach
zu mir: Merke auf, du Menschenkind; denn dies Gesicht gehörtin die Zeit
des Endes. \bibverse{18} Und da er mit mir redete, sank ich in eine
Ohnmacht zur Erde auf meinAngesicht. Er aber rührete mich an und
richtete mich auf, daß ich stund. \bibverse{19} Und er sprach: Siehe,
ich will dir zeigen, wie es gehen wird zur Zeit desletzten Zorns; denn
das Ende hat seine bestimmte Zeit. \bibverse{20} Der Widder mit den
zweien Hörnern, den du gesehen hast, sind die Königein Medien und
Persien. \bibverse{21} Der Ziegenbock aber ist der König in
Griechenland. Das große Hornzwischen seinen Augen ist der erste König.
\bibverse{22} Daß aber vier an seiner Statt stunden, da es zerbrochen
war, bedeutet, daßvier Königreiche aus dem Volk entstehen werden, aber
nicht so mächtig, als er war. \bibverse{23} Nach diesen Königreichen,
wenn die Übertreter überhandnehmen, wirdaufkommen ein frecher und
tückischer König. \bibverse{24} Der wird mächtig sein, doch nicht durch
seine Kraft. Er wird's wunderlichverwüsten; und wird ihm gelingen, daß
er's ausrichte. Er wird die Starken samt demheiligen Volk verstören.
\bibverse{25} Und durch seine Klugheit wird ihm der Betrug geraten. Und
wird sich inseinem Herzen erheben und durch Wohlfahrt wird er viele
verderben und wird sichauflehnen wider den Fürsten aller Fürsten. Aber
er wird ohne Hand zerbrochen werden. \bibverse{26} Dies Gesicht vom
Abend und Morgen, das dir gesagt ist, das ist wahr; aberdu sollst das
Gesicht heimlich halten, denn es ist noch eine lange Zeit dahin.
\bibverse{27} Und ich, Daniel, ward schwach und lag etliche Tage krank.
Danach standich auf und richtete aus des Königs Geschäfte. Und
verwunderte mich des Gesichts; undniemand war, der mir's berichtete.

\hypertarget{section-8}{%
\section{9}\label{section-8}}

\bibverse{1} Im ersten Jahr Darius, des Sohnes Ahasveros, aus der Meder
Stamm, derüber das Königreich der Chaldäer König ward, \bibverse{2} in
demselbigen ersten Jahr seines Königreichs merkte ich, Daniel, in
denBüchern auf die Zahl der Jahre, davon der HErr geredet hatte zum
Propheten Jeremia,daß Jerusalem sollte siebenzig Jahre wüste liegen.
\bibverse{3} Und ich kehrete mich zu GOtt dem HErrn, zu beten und zu
flehen, mitFasten, im Sack und in der Asche. \bibverse{4} Ich betete
aber zu dem HErrn, meinem GOtt, bekannte und sprach: Ach,lieber HErr, du
großer und schrecklicher GOtt, der du Bund und Gnade hältst denen,
diedich lieben und deine Gebote halten: \bibverse{5} wir haben
gesündigt, unrecht getan, sind gottlos gewesen und abtrünnigworden; wir
sind von deinen Geboten und Rechten gewichen. \bibverse{6} Wir
gehorchten nicht deinen Knechten, den Propheten, die in deinem
Namenunsern Königen, Fürsten, Vätern und allem Volk im Lande predigten.
\bibverse{7} Du, HErr, bist gerecht, wir aber müssen uns schämen, wie es
denn jetztgehet denen von Juda und denen von Jerusalem und dem ganzen
Israel, beide, denen,die nahe und ferne sind, in allen Landen, dahin du
uns verstoßen hast um ihrer Missetatwillen, die sie an dir begangen
haben. \bibverse{8} Ja, HErr, wir, unsere Könige, unsere Fürsten und
unsere Väter müssen unsschämen, daß wir uns an dir versündiget haben.
\bibverse{9} Dein aber, HErr, unser GOtt, ist die Barmherzigkeit und
Vergebung. Denn wirsind abtrünnig worden \bibverse{10} und gehorchten
nicht der Stimme des HErrn, unsers GOttes, daß wirgewandelt hätten in
seinem Gesetz welches er uns vorlegte durch seine Knechte, diePropheten,
\bibverse{11} sondern das ganze Israel übertrat dein Gesetz und wichen
ab, daß siedeiner Stimme nicht gehorchten. Daher trifft uns auch der
Fluch und Schwur, dergeschrieben stehet im Gesetz Mose, des Knechtes
GOttes, daß wir an ihm gesündigethaben. \bibverse{12} Und er hat seine
Worte gehalten, die er geredet hat wider uns und unsereRichter, die uns
richten sollten, daß er solch groß Unglück über uns hat gehen lassen,daß
desgleichen unter allem Himmel nicht geschehen ist, wie über Jerusalem
geschehenist. \bibverse{13} Gleichwie es geschrieben stehet im Gesetz
Mose, so ist all dies großeUnglück über uns gegangen. So beteten wir
auch nicht vor dem HErrn, unserm GOtt,daß wir uns von den Sünden
bekehreten und deine Wahrheit vernähmen. \bibverse{14} Darum ist der
HErr auch wacker gewesen mit diesem Unglück und hat esüber uns gehen
lassen. Denn der HErr, unser GOtt, ist gerecht in allen seinen
Werken,die er tut; denn wir gehorchten seiner Stimme nicht.
\bibverse{15} Und nun, HErr, unser GOtt, der du dein Volk aus
Ägyptenland geführet hastmit starker Hand und hast dir einen Namen
gemacht, wie er jetzt ist: wir haben jagesündiget und sind leider
gottlos gewesen. \bibverse{16} Ach HErr, um aller deiner Gerechtigkeit
willen wende ab deinen Zorn undGrimm von deiner Stadt Jerusalem und
deinem heiligen Berge! Denn um unserer Sündewillen und um unserer Väter
Missetat willen trägt Jerusalem und dein Volk Schmach beiallen, die um
uns her sind. \bibverse{17} Und nun, unser GOtt, höre das Gebet deines
Knechts und sein Flehen undsiehe gnädiglich an dein Heiligtum, das
verstöret ist, um des HErrn willen! \bibverse{18} Neige deine Ohren,
mein GOtt, und höre, tue deine Augen auf und siehe,wie wir verstört
sind, und die Stadt, die nach deinem Namen genannt ist! Denn wirliegen
vor dir mit unserm Gebet, nicht auf unsere Gerechtigkeit, sondern auf
deine großeBarmherzigkeit. \bibverse{19} Ach HErr, höre, ach HErr, sei
gnädig, ach HErr, merke auf und tue es undverzeuch nicht um dein selbst
willen, mein GOtt! Denn deine Stadt und dein Volk istnach deinem Namen
genannt. \bibverse{20} Als ich noch so redete und betete und meine und
meines Volks Israel Sündebekannte und lag mit meinem Gebet vor dem
HErrn, meinem GOtt, um den heiligenBerg meines GOttes, \bibverse{21}
eben da ich so redete in meinem Gebet, flog daher der Mann Gabriel,
denich vorhin gesehen hatte im Gesicht, und rührete mich an um die Zeit
des Abendopfers. \bibverse{22} Und er berichtete mir und redete mit mir
und sprach: Daniel, jetzt bin ichausgegangen, dir zu berichten.
\bibverse{23} Denn da du anfingest zu beten, ging dieser Befehl aus, und
ich kommedarum, daß ich dir's anzeige; denn du bist lieb und wert. So
merke nun darauf, daß dudas Gesicht verstehest! \bibverse{24} Siebenzig
Wochen sind bestimmt über dein Volk und über deine heiligeStadt, so wird
dem Übertreten gewehret und die Sünde zugesiegelt und die
Missetatversöhnet und die ewige Gerechtigkeit gebracht und die Gesichte
und Weissagungzugesiegelt und der Allerheiligste gesalbet werden.
\bibverse{25} So wisse nun und merke: Von der Zeit an, so ausgehet der
Befehl, daßJerusalem soll wiederum gebauet werden, bis auf Christum, den
Fürsten, sind siebenWochen und zweiundsechzig Wochen, so werden die
Gassen und Mauern wieder gebauetwerden, wiewohl in kümmerlicher Zeit.
\bibverse{26} Und nach den zweiundsechzig Wochen wird Christus
ausgerottet werdenund nichts mehr sein. Und ein Volk des Fürsten wird
kommen und die Stadt und dasHeiligtum verstören, daß es ein Ende nehmen
wird wie durch eine Flut; und bis zumEnde des Streits wird's wüst
bleiben. \bibverse{27} Er wird aber vielen den Bund stärken eine Woche
lang. Und mitten in derWoche wird das Opfer und Speisopfer aufhören. Und
bei den Flügeln werden stehenGreuel der Verwüstung; und ist beschlossen,
daß bis ans Ende über die Verwüstungtriefen wird.

\hypertarget{section-9}{%
\section{10}\label{section-9}}

\bibverse{1} Im dritten Jahr des Königs Kores aus Persien ward dem
Daniel, derBeltsazar heißt, etwas offenbaret, das gewiß ist und von
großen Sachen; und er merktedarauf und verstund das Gesicht wohl.
\bibverse{2} Zur selbigen Zeit war ich, Daniel, traurig drei Wochen
lang. \bibverse{3} Ich aß keine niedliche Speise, Fleisch und Wein kam
in meinen Mund nicht;und salbete mich auch nie, bis die drei Wochen um
waren. \bibverse{4} Am vierundzwanzigsten Tage des ersten Monden war
ich, bei dem großenWasser Hiddekel \bibverse{5} und hub meine Augen auf
und sah, und siehe, da stund ein Mann inLeinwand und hatte einen
güldenen Gürtel um seine Lenden. \bibverse{6} Sein Leib war wie ein
Türkis, sein Antlitz sah wie ein Blitz, seine Augen wieeine feurige
Fackel, seine Arme und Füße wie ein glühend Erz, und seine Rede war
wieein groß Getön. \bibverse{7} Ich, Daniel, aber sah solch Gesicht
alleine, und die Männer, so bei mirwaren, sahen's nicht; doch fiel ein
groß Schrecken über sie, daß sie flohen und sichverkrochen. \bibverse{8}
Und ich blieb alleine und sah dies große Gesicht. Es blieb aber keine
Kraft inmir, und ich ward sehr ungestalt und hatte keine Kraft mehr.
\bibverse{9} Und ich hörete seine Rede; und indem ich sie hörete, sank
ich nieder aufmein Angesicht zur Erde. \bibverse{10} Und siehe, eine
Hand rührete mich an und half mir auf die Kniee und aufdie Hände
\bibverse{11} und sprach zu mir: Du lieber Daniel, merke auf die Worte,
die ich mit dirrede, und richte dich auf; denn ich bin jetzt zu dir
gesandt. Und da er solches mit mirredete, richtete ich mich auf und
zitterte. \bibverse{12} Und er sprach zu mir: Fürchte dich nicht,
Daniel; denn von dem erstenTage an, da du von Herzen begehretest zu
verstehen, und dich kasteietest vor deinemGOtt, sind deine, Worte
erhöret; und ich bin kommen um deinetwillen. \bibverse{13} Aber der
Fürst des Königreichs in Persienland hat mir einundzwanzig
Tagewiderstanden; und siehe, Michael, der vornehmsten Fürsten einer, kam
mir zu Hilfe; dabehielt ich den Sieg bei den Königen in Persien.
\bibverse{14} Nun aber komme ich, daß ich dir berichte, wie es deinem
Volk hernachgehen wird; denn das Gesicht wird nach etlicher Zeit
geschehen. \bibverse{15} Und als er solches mit mir redete, schlug ich
mein Angesicht nieder zurErde und schwieg stille. \bibverse{16} Und
siehe, einer, gleich einem Menschen, rührete meine Lippen an. Da tatich
meinen Mund auf und redete und sprach zu dem, der vor mir stund: Mein
Herr,meine Gelenke beben mir über dem Gesicht, und ich habe keine Kraft
mehr. \bibverse{17} Und wie kann der Knecht meines Herrn mit meinem
Herrn reden, weil nunkeine Kraft mehr in mir ist, und habe auch keinen
Odem mehr? \bibverse{18} Da rührete mich abermal an einer, gleichwie ein
Mensch gestaltet, undstärkte mich \bibverse{19} und sprach: Fürchte dich
nicht, du lieber Mann! Friede sei mit dir; und seigetrost, sei getrost!
Und als er mit mir redete, ermannete ich mich und sprach: MeinHerr,
rede; denn du hast mich gestärkt. \bibverse{20} Und er sprach: Weißt du
auch, warum ich zu dir kommen bin? Jetzt will ichwieder hin und mit dem
Fürsten in Persienland streiten; aber wenn ich wegziehe, siehe,so wird
der Fürst aus Griechenland kommen. \bibverse{21} Doch will ich dir
anzeigen, was geschrieben ist, das gewißlich geschehenwird. Und ist
keiner, der mir hilft wider jene denn euer Fürst Michael.

\hypertarget{section-10}{%
\section{11}\label{section-10}}

\bibverse{1} Denn ich stund auch bei ihm im ersten Jahr Darius, des
Meders, daß ichihm hülfe und ihn stärkete. \bibverse{2} Und nun will ich
dir anzeigen, was gewiß geschehen soll. Siehe, es werdennoch drei Könige
in Persien stehen; der vierte aber wird größern Reichtum haben dennalle
andern; und wenn er in seinem Reichtum am mächtigsten ist, wird er alles
wider dasKönigreich in Griechenland erregen. \bibverse{3} Danach wird
ein mächtiger König aufstehen und mit großer Machtherrschen, und was er
will, wird er ausrichten. \bibverse{4} Und wenn er aufs höchste kommen
ist, wird sein Reich zerbrechen und sichin die vier Winde des Himmels
zerteilen, nicht auf seine Nachkommen, auch nicht mitsolcher Macht, wie
seine gewesen ist; denn sein Reich wird ausgerottet und Fremdenzuteil
werden. \bibverse{5} Und der König gegen Mittag, welcher ist seiner
Fürsten einer, wird mächtigwerden; aber gegen ihn wird einer auch
mächtig sein und herrschen, welches Herrschaftwird groß sein.
\bibverse{6} Nach etlichen Jahren aber werden sie sich miteinander
befreunden; und dieTochter des Königs gegen Mittag wird kommen zum
Könige gegen Mitternacht, Einigkeitzu machen. Aber sie wird nicht
bleiben bei der Macht des Arms, dazu ihr Same auchnicht stehen bleiben,
sondern sie wird übergeben samt denen, die sie gebracht haben,und mit
dem Kinde und dem, der sie eine Weile mächtig gemacht hatte.
\bibverse{7} Es wird aber der Zweige einer von ihrem Stamm aufkommen,
der wirdkommen mit Heereskraft und dem Könige gegen Mitternacht in seine
Feste fallen; undwird's, ausrichten und siegen. \bibverse{8} Auch wird
er ihre Götter und Bilder samt den köstlichen Kleinoden, beide,silbernen
und güldenen wegführen nach Ägypten und etliche Jahre vor dem
Königegegen Mitternacht wohl stehen bleiben. \bibverse{9} Und wenn er
durch desselbigen Königreich gezogen ist, wird er wiederum insein Land
ziehen. \bibverse{10} Aber seine Söhne werden erzürnen und große Heere
zusammenbringen;und der eine wird kommen und wie eine Flut daherfahren
und jenen wiederum vorseinen Festen reizen. \bibverse{11} Da wird der
König gegen Mittag ergrimmen und ausziehen und mit demKönige gegen
Mitternacht streiten und wird solchen großen Haufen zusammenbringen,daß
ihm jener Haufe wird in seine Hand gegeben. \bibverse{12} Und wird
denselbigen Haufen wegführen. Des wird sich sein Herz erheben,daß er so
viel tausend daniedergelegt hat; aber damit wird er sein nicht mächtig
werden. \bibverse{13} Denn der König gegen Mitternacht wird wiederum
einen größern Haufenzusammenbringen, denn der vorige war; und nach
etlichen Jahren wird er daherziehenmit großer Heereskraft und mit großem
Gut. \bibverse{14} Und zur selbigen Zeit werden sich viele wider den
König gegen Mittagsetzen; auch werden sich etliche Abtrünnige aus deinem
Volk erheben und dieWeissagung erfüllen und werden fallen. \bibverse{15}
Also wird der König gegen Mitternacht daherziehen und Schütte machenund
feste Städte gewinnen; und die Mittagsarme werden's nicht können wehren,
undsein bestes Volk werden nicht können widerstehen, \bibverse{16}
sondern er wird, wenn er an ihn kommt, seinen Willen schaffen;
undniemand wird ihm widerstehen mögen. Er wird auch in das werte Land
kommen undwird's vollenden durch seine Hand. \bibverse{17} Und wird sein
Angesicht richten, daß er mit Macht seines ganzenKönigreichs komme. Aber
er wird sich mit ihm vertragen und wird ihm seine Tochterzum Weibe
geben, daß er ihn verderbe; aber es wird ihm nicht geraten, und wird
nichtsdaraus werden. \bibverse{18} Danach wird er sich kehren wider die
Inseln und derselbigen vielegewinnen. Aber ein Fürst wird ihn lehren
aufhören mit Schmähen, daß er ihn nicht mehrschmähe. \bibverse{19} Also
wird er sich wiederum kehren zu den Festen seines Landes und wirdsich
stoßen und fallen, daß man ihn nirgend finden wird. \bibverse{20} Und an
seiner Statt wird einer aufkommen, der wird in königlichen Ehrensitzen
wie ein Scherge. Aber nach wenig Tagen wird er brechen, doch weder durch
Zornnoch durch Streit. \bibverse{21} An des Statt wird aufkommen ein
Ungeachteter, welchem die Ehre desKönigreichs nicht bedacht war; der
wird kommen, und wird ihm gelingen und dasKönigreich mit süßen Worten
einnehmen. \bibverse{22} Und die Arme, die wie eine Flut daherfahren,
werden vor ihm wie mit einerFlut überfallen und zerbrochen werden, dazu
auch der Fürst, mit dem der Bund gemachtwar. \bibverse{23} Denn nachdem
er mit ihm befreundet ist, wird er listiglich gegen ihnhandeln; und wird
heraufziehen und mit geringem Volk ihn überwältigen. \bibverse{24} Und
wird ihm gelingen, daß er in die besten Städte des Landes kommenwird;
und wird's also ausrichten, das seine Väter noch seine Voreltern nicht
tun konntenmit Rauben, Plündern und Ausbeuten; und wird nach den
allerfestesten Städtentrachten, und das eine Zeitlang. \bibverse{25} Und
er wird seine Macht und sein Herz wider den König gegen Mittagerregen
mit großer Heereskraft. Da wird der König gegen Mittag gereizet werden
zumStreit mit einer großen, mächtigen Heereskraft. Aber er wird nicht
bestehen; denn eswerden Verrätereien wider ihn gemacht. \bibverse{26}
Und eben, die sein Brot essen, die werden ihn helfen verderben und
seinHeer unterdrücken, daß gar viele erschlagen werden. \bibverse{27}
Und beider Könige Herz wird denken, wie sie einander Schaden tun,
undwerden doch über einem Tisch fälschlich miteinander reden. Es wird
ihnen aber fehlen;denn das Ende ist noch auf eine andere Zeit bestimmt.
\bibverse{28} Danach wird er wiederum heimziehen mit großem Gut und sein
Herzrichten wider den heiligen Bund; da wird er etwas ausrichten und
also heim in sein Landziehen. \bibverse{29} Danach wird er zu gelegener
Zeit wieder gegen Mittag ziehen; aber eswird ihm zum andernmal nicht
geraten wie zum erstenmal. \bibverse{30} Denn es werden Schiffe aus
Chittim wider ihn kommen, daß er verzagenwird und umkehren muß. Da wird
er wider den heiligen Bund ergrimmen und wird'sausrichten; und wird sich
umsehen und an sich ziehen, die den heiligen Bund verlassen.
\bibverse{31} Und es werden seine Arme daselbst stehen; die werden das
Heiligtum inder Feste entweihen und das tägliche Opfer abtun und einen
Greuel der Wüstungaufrichten. \bibverse{32} Und er wird heucheln und
gute Worte geben den Gottlosen, so den Bundübertreten. Aber das Volk, so
ihren GOtt kennen, werden sich ermannen und esausrichten. \bibverse{33}
Und die Verständigen im Volk werden viel andere lehren; darüber
werdensie fallen durch Schwert, Feuer, Gefängnis und Raub eine Zeitlang.
\bibverse{34} Und wenn sie so fallen, wird ihnen dennoch eine kleine
Hilfe geschehen.Aber viele werden sich zu ihnen tun betrüglich.
\bibverse{35} Und der Verständigen werden etliche fallen, auf daß sie
bewähret, reinund lauter werden, bis daß es ein Ende habe; denn es ist
noch eine andere Zeitvorhanden. \bibverse{36} Und der König wird tun,
was er will, und wird sich erheben und aufwerfenwider alles, das GOtt
ist; und wider den GOtt aller Götter wird er greulich reden; undwird ihm
gelingen, bis der Zorn aus sei; denn es ist beschlossen, wie lange es
währensoll. \bibverse{37} Und seiner Väter GOtt wird er nicht achten; er
wird weder Frauenliebenoch einiges Gottes achten, denn er wird sich
wider alles aufwerfen. \bibverse{38} Aber an des Statt wird er seinen
Gott Maußim ehren; denn er wird einenGott, davon seine Väter nichts
gewußt haben, ehren mit Gold, Silber, Edelstein undKleinoden.
\bibverse{39} Und wird denen, so ihm helfen stärken Maußim mit dem
fremden Gott,den er erwählet hat, große Ehre tun und sie zu Herren
machen über große Güter undihnen das Land zu Lohn austeilen.
\bibverse{40} Und am Ende wird sich der König gegen Mittag mit ihm
stoßen; und derKönig gegen Mitternacht wird sich gegen ihn sträuben mit
Wagen, Reitern und vielSchiffen; und wird in die Länder fallen und
verderben und durchziehen. \bibverse{41} Und wird in das werte Land
fallen, und viele werden umkommen. Dieseaber werden seiner Hand
entrinnen: Edom, Moab und die Erstlinge der Kinder Ammon. \bibverse{42}
Und er wird seine Macht in die Länder schicken, und Ägypten wird
ihmnicht entrinnen, \bibverse{43} sondern er wird durch seinen Zug
herrschen über die güldenen undsilbernen Schätze und über alle Kleinode
Ägyptens, Libyens und der Mohren. \bibverse{44} Es wird ihn aber ein
Geschrei erschrecken von Morgen und Mitternacht;und er wird mit großem
Grimm ausziehen, willens, viele zu vertilgen und zu verderben.
\bibverse{45} Und er wird das Gezelt seines Palasts aufschlagen zwischen
zweienMeeren um den werten heiligen Berg, bis mit ihm ein Ende werde;
und niemand wirdihm helfen.

\hypertarget{section-11}{%
\section{12}\label{section-11}}

\bibverse{1} Zur selbigen Zeit wird der große Fürst Michael, der für
dein Volk stehet,sich aufmachen. Denn es wird eine solche trübselige
Zeit sein, als sie nicht gewesen ist,seit daß Leute gewesen sind, bis
auf dieselbige Zeit. Zur selbigen Zeit wird dein Volkerrettet werden,
alle, die im Buch geschrieben stehen. \bibverse{2} Und viele, so unter
der Erde schlafen liegen, werden aufwachen, etlichezum ewigen Leben,
etliche zur ewigen Schmach und Schande. \bibverse{3} Die Lehrer aber
werden leuchten wie des Himmels Glanz und die, so vielezur Gerechtigkeit
weisen, wie die Sterne immer und ewiglich. \bibverse{4} Und nun, Daniel,
verbirg diese Worte und versiegele diese Schrift bis aufdie letzte Zeit,
so werden viele drüber kommen und großen Verstand finden. \bibverse{5}
Und ich, Daniel, sah, und siehe, es stunden zween andere da, einer
andiesem Ufer des Wassers, der andere an jenem Ufer. \bibverse{6} Und er
sprach zu dem in leinenen Kleidern, der oben am Wasser stund:Wann will's
denn ein Ende sein mit solchen Wundern? \bibverse{7} Und ich hörete zu
dem in leinenen Kleidern, der oben am Wasser stund;und er hub seine
rechte und linke Hand auf gen Himmel und schwur bei dem, so
ewiglichlebet, daß es eine Zeit und etliche Zeiten und eine halbe Zeit
währen soll; und wenn dieZerstreuung des heiligen Volks ein Ende hat,
soll solches alles geschehen. \bibverse{8} Und ich hörete es: aber ich
verstund es nicht und sprach: Mein Herr, waswird danach werden?
\bibverse{9} Er aber sprach: Gehe hin, Daniel; denn es ist verborgen und
versiegelt bisauf die letzte Zeit. \bibverse{10} Viele werden
gereiniget, geläutert und bewähret werden; und dieGottlosen werden
gottlos Wesen führen, und die Gottlosen werden's nicht achten; aberdie
Verständigen werden's achten. \bibverse{11} Und von der Zeit an, wenn
das tägliche Opfer abgetan und ein Greuel derWüstung dargesetzt wird,
sind tausend zweihundert und neunzig Tage: \bibverse{12} Wohl dem, der
da erwartet und erreichet tausend dreihundert undfünfunddreißig Tage!
\bibverse{13} Du aber, Daniel, gehe hin, bis das Ende komme, und ruhe,
daß duaufstehest in deinem Teil am Ende der Tage!
