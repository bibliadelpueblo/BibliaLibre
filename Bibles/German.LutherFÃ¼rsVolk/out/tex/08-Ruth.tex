\hypertarget{section}{%
\section{1}\label{section}}

\bibverse{1} Zu der Zeit, da die Richter regierten, ward eine Teurung im
Lande. Und ein Mann von Bethlehem-Juda zog wallen in der Moabiter Land
mit seinem Weibe und zween Söhnen. \bibverse{2} Der hieß Elimelech und
sein Weib Naemi, und seine zween Söhne Mahlon und Chiljon, die waren
Ephrather, von Bethlehem-Juda. Und da sie kamen ins Land der Moabiter,
blieben sie daselbst. \bibverse{3} Und Elimelech, der Naemi Mann, starb,
und sie blieb übrig mit ihren zween Söhnen. \bibverse{4} Die nahmen
moabitische Weiber. Eine hieß Arpa, die andere Ruth. Und da sie daselbst
gewohnet hatten bei zehn Jahren, \bibverse{5} starben sie alle beide,
Mahlon und Chiljon, daß das Weib überblieb beiden Söhnen und ihrem
Manne. \bibverse{6} Da machte sie sich auf mit ihren zwo Schnüren und
zog wieder aus der Moabiter Lande; denn sie hatte erfahren im Moabiter
Lande, daß der HErr sein Volk hatte heimgesucht und ihnen Brot gegeben.
\bibverse{7} Und ging aus von dem Ort, da sie gewesen war, und ihre
beiden Schnüre mit ihr. Und da sie ging auf dem Wege, daß sie wiederkäme
ins Land Juda, \bibverse{8} sprach sie zu ihren beiden Schnüren: Gehet
hin und kehret um, eine jegliche zu ihrer Mutter Haus; der HErr tue an
euch Barmherzigkeit, wie ihr an den Toten und an mir getan habt!
\bibverse{9} Der HErr gebe euch, daß ihr Ruhe findet, eine jegliche in
ihres Mannes Hause! Und küssete sie. Da huben sie ihre Stimme auf und
weineten. \bibverse{10} Und sprachen zu ihr: Wir wollen mit dir zu
deinem Volk gehen. \bibverse{11} Aber Naemi sprach: Kehret um, meine
Töchter; warum wollt ihr mit mir gehen? Wie kann ich fürder Kinder in
meinem Leibe haben, die eure Männer sein möchten? \bibverse{12} Kehret
um, meine Töchter, und gehet hin; denn ich bin nun zu alt, daß ich einen
Mann nehme. Und wenn ich spräche: Es ist zu hoffen, daß ich diese Nacht
einen Mann nehme und Kinder gebäre, \bibverse{13} wie könnet ihr doch
harren, bis sie groß würden? Wie wollt ihr verziehen, daß ihr nicht
Männer solltet nehmen? Nicht, meine Töchter; denn mich jammert euer
sehr, denn des HErrn Hand ist über mich ausgegangen. \bibverse{14} Da
huben sie ihre Stimme auf und weineten noch mehr. Und Arpa küssete ihre
Schwieger; Ruth aber blieb bei ihr. \bibverse{15} Sie aber sprach:
Siehe, deine Schwägerin ist umgewandt zu ihrem Volk und zu ihrem Gott;
kehre du auch um, deiner Schwägerin nach. \bibverse{16} Ruth antwortete:
Rede mir nicht darein, daß ich dich verlassen sollte und von dir
umkehren. Wo du hingehest, da will ich auch hingehen; wo du bleibest, da
bleibe ich auch. Dein Volk ist mein Volk und dein GOtt ist mein GOtt.
\bibverse{17} Wo du stirbst, da sterbe ich auch; da will ich auch
begraben werden. Der HErr tue mir dies und das: der Tod muß mich und
dich scheiden. \bibverse{18} Als sie nun sah, daß sie fest im Sinne war,
mit ihr zu gehen, ließ sie ab, mit ihr davon zu reden. \bibverse{19}
Also gingen die beiden miteinander bis sie gen Bethlehem kamen. Und da
sie zu Bethlehem einkamen, regte sich die ganze Stadt über ihnen und
sprach: Ist das die Naemi? \bibverse{20} Sie aber sprach zu ihnen:
Heißet mich nicht Naemi, sondern Mara; denn der Allmächtige hat mich
sehr betrübet. \bibverse{21} Voll zog ich aus, aber leer hat mich der
HErr wieder heimgebracht. Warum heißet ihr mich denn Naemi, so mich doch
der HErr gedemütiget und der Allmächtige betrübet hat? \bibverse{22} Es
war aber um die Zeit, daß die Gerstenernte anging, da Naemi und ihre
Schnur Ruth, die Moabitin, wiederkamen vom Moabiter Lande gen Bethlehem.

\hypertarget{section-1}{%
\section{2}\label{section-1}}

\bibverse{1} Es war auch ein Mann, der Naemi Mannes Freund, von dem
Geschlecht Elimelechs, mit Namen Boas, der war ein weidlicher Mann.
\bibverse{2} Und Ruth, die Moabitin, sprach zu Naemi: Laß mich aufs Feld
gehen und Ähren auflesen, dem nach, vor dem ich Gnade finde. Sie aber
sprach zu ihr: Gehe hin, meine Tochter! \bibverse{3} Sie ging hin, kam
und las auf, den Schnittern nach, auf dem Felde. Und es begab sich eben,
daß dasselbe Feld war des Boas, der von dem Geschlecht Elimelechs war.
\bibverse{4} Und siehe, Boas kam eben von Bethlehem und sprach zu den
Schnittern: Der HErr mit euch! Sie antworteten: Der HErr segne dich!
\bibverse{5} Und Boas sprach zu seinem Knaben, der über die Schnitter
gestellet war: Wes ist die Dirne? \bibverse{6} Der Knabe, der über die
Schnitter gestellet war, antwortete und sprach: Es ist die Dirne, die
Moabitin, die mit Naemi wiederkommen ist von der Moabiter Lande.
\bibverse{7} Denn sie sprach: Lieber, laß mich auflesen und sammeln
unter den Garben, den Schnittern nach; und ist also kommen und da
gestanden von Morgen an bis her und bleibt wenig daheim. \bibverse{8} Da
sprach Boas zu Ruth: Hörest du es, meine Tochter? Du sollst nicht gehen
auf einen andern Acker aufzulesen; und gehe auch nicht von hinnen,
sondern halte dich zu meinen Dirnen; \bibverse{9} und siehe, wo sie
schneiden im Felde, da gehe ihnen nach. Ich habe meinem Knaben geboten,
daß dich niemand antaste. Und so dich dürstet, so gehe hin zu dem Gefäß
und trinke, da meine Knaben schöpfen. \bibverse{10} Da fiel sie auf ihr
Angesicht und betete an zur Erde und sprach zu ihm: Womit habe ich die
Gnade funden vor deinen Augen, daß du mich erkennest, die ich doch fremd
bin? \bibverse{11} Boas antwortete und sprach zu ihr: Es ist mir
angesagt alles, was du getan hast an deiner Schwieger nach deines Mannes
Tode: daß du verlassen hast deinen Vater und deine Mutter und dein
Vaterland und bist zu einem Volk gezogen, das du zuvor nicht kanntest.
\bibverse{12} Der HErr vergelte dir deine Tat; und müsse dein Lohn
vollkommen sein bei dem HErrn, dem GOtt Israels, zu welchem du kommen
bist, daß du unter seinen Flügeln Zuversicht hättest. \bibverse{13} Sie
sprach: Laß mich Gnade vor deinen Augen finden, mein HErr; denn du hast
mich getröstet und deine Magd freundlich angesprochen, so ich doch nicht
bin als deiner Mägde eine. \bibverse{14} Boas sprach zu ihr: Wenn's
Essenszeit ist, so mache dich hie herzu und iß des Brots und tunke
deinen Bissen in den Essig. Und sie setzte sich zur Seite der Schnitter.
Er aber legte ihr Sangen vor; und sie aß und ward satt und ließ über.
\bibverse{15} Und da sie sich aufmachte zu lesen, gebot Boas seinen
Knaben und sprach: Lasset sie auch zwischen den Garben lesen und
beschämet sie nicht! \bibverse{16} Auch von den Haufen lasset
überbleiben und lasset liegen, daß sie es auflese; und niemand schelte
sie drum. \bibverse{17} Also las sie auf dem Felde bis zum Abend und
schlug es aus, was sie aufgelesen hatte; und es war bei einem Epha
Gerste. \bibverse{18} Und sie hub es auf und kam in die Stadt. Und ihre
Schwieger sah es, was sie gelesen hatte. Da zog sie hervor und gab ihr,
was ihr übriggeblieben war, davon sie satt war worden. \bibverse{19} Da
sprach ihre Schwieger zu ihr: Wo hast du heute gelesen, und wo hast du
gearbeitet? Gesegnet sei, der dich erkannt hat! Sie aber sagte es ihrer
Schwieger, bei wem sie gearbeitet hätte, und sprach: Der Mann, bei dem
ich heute gearbeitet habe, heißt Boas. \bibverse{20} Naemi aber sprach
zu ihrer Schnur: Gesegnet sei er dem HErrn, denn er hat seine
Barmherzigkeit nicht gelassen, beide an den Lebendigen und an den Toten.
Und Naemi sprach zu ihr: Der Mann gehöret uns zu und ist unser Erbe.
\bibverse{21} Ruth, die Moabitin, sprach: Er sprach auch das zu mir: Du
sollst dich zu meinen Knaben halten, bis sie mir alles eingeerntet
haben. \bibverse{22} Naemi sprach zu Ruth, ihrer Schnur: Es ist besser,
meine Tochter, daß du mit seinen Dirnen ausgehest, auf daß nicht jemand
dir dreinrede auf einem andern Acker. \bibverse{23} Also hielt sie sich
zu den Dirnen Boas, daß sie las, bis daß die Gerstenernte und
Weizenernte aus war. Und kam wieder zu ihrer Schwieger.

\hypertarget{section-2}{%
\section{3}\label{section-2}}

\bibverse{1} Und Naemi, ihre Schwieger, sprach zu ihr: Meine Tochter,
ich will dir Ruhe schaffen, daß dir's wohlgehe. \bibverse{2} Nun, der
Boas, unser Freund, bei des Dirnen du gewesen bist, worfelt diese Nacht
Gerste auf seiner Tenne. \bibverse{3} So bade dich und salbe dich und
lege dein Kleid an und gehe hinab auf die Tenne, daß dich niemand kenne,
bis man ganz gegessen und getrunken hat. \bibverse{4} Wenn er sich dann
leget, so merke den Ort, da er sich hinlegt; und komm und decke auf zu
seinen Füßen und lege dich, so wird er dir wohl sagen, was du tun
sollst. \bibverse{5} Sie sprach zu ihr: Alles was du mir sagest, will
ich tun. \bibverse{6} Sie ging hinab zur Tenne und tat alles, wie ihre
Schwieger geboten hatte. \bibverse{7} Und da Boas gegessen und getrunken
hatte, ward sein Herz guter Dinge; und kam und legte sich hinter eine
Mandel. Und sie kam leise und deckte auf zu seinen Füßen und legte sich.
\bibverse{8} Da es nun Mitternacht ward, erschrak der Mann und
erschütterte; und siehe, ein Weib lag zu seinen Füßen. \bibverse{9} Und
er sprach: Wer bist du? Sie antwortete: Ich bin Ruth, deine Magd. Breite
deinen Flügel über deine Magd, denn du bist der Erbe. \bibverse{10} Er
aber sprach: Gesegnet seiest du dem HErrn, meine Tochter! Du hast eine
bessere Barmherzigkeit hernach getan denn vorhin, daß du nicht bist den
Jünglingen nachgegangen, weder reich noch arm. \bibverse{11} Nun, meine
Tochter, fürchte dich nicht! Alles, was du sagst, will ich dir tun; denn
die ganze Stadt meines Volks weiß, daß du ein tugendsam Weib bist.
\bibverse{12} Nun, es ist wahr, daß ich der Erbe bin; aber es ist einer
näher denn ich. \bibverse{13} Bleib über Nacht. Morgen, so er dich
nimmt, wohl; gelüstet's ihn aber nicht, dich zu nehmen, so will ich dich
nehmen, so wahr der HErr lebt. Schlaf bis morgen. \bibverse{14} Und sie
schlief bis morgen zu seinen Füßen. Und sie stund auf, ehe denn einer
den andern, kennen mochte; und er gedachte, daß nur niemand inne werde,
daß ein Weib in die Tenne kommen sei! \bibverse{15} Und sprach: Lange
her den Mantel, den du anhast, und halt ihn zu. Und sie hielt ihn zu.
Und er maß sechs Maß Gerste und legte es auf sie. Und er kam in die
Stadt. \bibverse{16} Sie aber kam zu ihrer Schwieger, die sprach: Wie
steht es mit dir, meine Tochter? Und sie sagte ihr alles, was ihr der
Mann getan hatte, \bibverse{17} und sprach: Diese sechs Maß Gerste gab
er mir, denn er sprach: Du sollst nicht leer zu deiner Schwieger kommen.
\bibverse{18} Sie aber sprach: Sei stille, meine Tochter, bis du
erfährest, wo es hinaus will; denn der Mann wird nicht ruhen, er bringe
es denn heute zu Ende.

\hypertarget{section-3}{%
\section{4}\label{section-3}}

\bibverse{1} Boas ging hinauf ins Tor und setzte sich daselbst. Und
siehe, da der Erbe vorüberging, redete Boas mit ihm und sprach: Komm und
setze dich etwa hie oder da her. Und er kam und setzte sich.
\bibverse{2} Und er nahm zehn Männer von den Ältesten der Stadt und
sprach: Setzet euch her! Und sie setzten sich. \bibverse{3} Da sprach er
zu dem Erben: Naemi, die vom Land der Moabiter wiederkommen ist, beut
feil das Stück Feld, das unsers Bruders war, Elimelechs. \bibverse{4}
Darum gedachte ich's vor deine Ohren zu bringen und zu sagen: Willst du
es beerben, so kaufe es vor den Bürgern und vor den Ältesten meines
Volks; willst du es aber nicht beerben, so sage mir's, daß ich's wisse;
denn es ist kein Erbe ohne du und ich nach dir. Er sprach: Ich will's
beerben. \bibverse{5} Boas sprach: Welches Tages du das Feld kaufst von
der Hand Naemis, so mußt du auch Ruth, die Moabitin, des Verstorbenen
Weib, nehmen, daß du dem Verstorbenen einen Namen erweckest auf sein
Erbteil. \bibverse{6} Da sprach er: Ich mag es nicht beerben, daß ich
nicht vielleicht mein Erbteil verderbe. Beerbe du, was ich beerben soll;
denn ich mag es nicht beerben. \bibverse{7} Es war aber von alters her
eine solche Gewohnheit in Israel: Wenn einer ein Gut nicht beerben noch
erkaufen wollte, auf daß allerlei Sache bestünde, so zog er seinen Schuh
aus und gab ihn dem andern; das war das Zeugnis in Israel. \bibverse{8}
Und der Erbe sprach zu Boas: Kaufe du es; und zog seinen Schuh aus.
\bibverse{9} Und Boas sprach zu den Ältesten und zu allem Volk: Ihr seid
heute Zeugen, daß ich alles gekauft habe, was Elimelechs gewesen ist,
und alles, was Chiljons und Mahlons, von der Hand Naemis. \bibverse{10}
Dazu auch Ruth, die Moabitin, Mahlons Weib, nehme ich zum Weibe, daß ich
dem Verstorbenen einen Namen erwecke auf sein Erbteil, und sein Name
nicht ausgerottet werde unter seinen Brüdern und aus dem Tor seines
Orts; Zeugen seid ihr des heute. \bibverse{11} Und alles Volk, das im
Tor war, samt den Ältesten sprachen: Wir sind Zeugen. Der HErr mache das
Weib, das in dein Haus kommt, wie Rahel und Lea, die beide das Haus
Israel gebauet haben; und wachse sehr in Ephratha und werde gepreiset zu
Bethlehem! \bibverse{12} Und dein Haus werde wie das Haus Perez, den
Thamar Juda gebar, von dem Samen, den dir der HErr geben wird von dieser
Dirne. \bibverse{13} Also nahm Boas die Ruth, daß sie sein Weib ward.
Und da er bei ihr lag, gab ihr der HErr, daß sie schwanger ward, und
gebar einen Sohn. \bibverse{14} Da sprachen die Weiber zu Naemi: Gelobet
sei der HErr, der dir nicht hat lassen abgehen einen Erben zu dieser
Zeit, daß sein Name in Israel bliebe. \bibverse{15} Der wird dich
erquicken und dein Alter versorgen. Denn deine Schnur, die dich geliebet
hat, hat ihn geboren, welche dir besser ist denn sieben Söhne.
\bibverse{16} Und Naemi nahm das Kind und legte es auf ihren Schoß und
ward seine Wärterin. \bibverse{17} Und ihre Nachbarinnen gaben ihm einen
Namen und sprachen: Naemi ist ein Kind geboren; und hießen ihn Obed, der
ist der Vater Isais, welcher ist Davids Vater. \bibverse{18} Dies ist
das Geschlecht Perez: Perez zeugete Hezron; \bibverse{19} Hezron zeugete
Ram; Ram zeugete Amminadab; \bibverse{20} Amminadab zeugete Nahesson;
Nahesson zeugete Salma; \bibverse{21} Salma zeugete Boas; Boas zeugete
Obed; \bibverse{22} Obed zeugete Isai; Isai zeugete David.
