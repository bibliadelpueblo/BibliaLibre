\hypertarget{section}{%
\section{1}\label{section}}

\bibverse{1} Die erste Rede habe ich getan, lieber Theophilus, von alle
dem, das Jesus anfing, beides, zu tun und zu lehren, \footnote{\textbf{1:1}
  Lk 1,3} \bibverse{2} bis an den Tag, da er aufgenommen ward, nachdem
er den Aposteln, welche er hatte erwählt, durch den heiligen Geist
Befehl getan hatte, \footnote{\textbf{1:2} Mt 28,19-20} \bibverse{3}
welchen er sich nach seinem Leiden lebendig erzeigt hatte durch
mancherlei Erweisungen, und ließ sich sehen unter ihnen vierzig Tage
lang und redete mit ihnen vom Reich Gottes. \bibverse{4} Und als er sie
versammelt hatte, befahl er ihnen, dass sie nicht von Jerusalem wichen,
sondern warteten auf die Verheißung des Vaters, welche ihr habt gehört
(sprach er) von mir; \bibverse{5} denn Johannes hat mit Wasser getauft,
ihr aber sollt mit dem Heiligen Geist getauft werden nicht lange nach
diesen Tagen. \footnote{\textbf{1:5} Mt 3,11}

\bibverse{6} Die aber, die zusammengekommen waren, fragten ihn und
sprachen: Herr, wirst du auf diese Zeit wieder aufrichten das Reich
Israel? \footnote{\textbf{1:6} Lk 19,11; Lk 24,21}

\bibverse{7} Er aber sprach zu ihnen: Es gebührt euch nicht, zu wissen
Zeit oder Stunde, welche der Vater seiner Macht vorbehalten hat;
\footnote{\textbf{1:7} Mt 24,36} \bibverse{8} sondern ihr werdet die
Kraft des Heiligen Geistes empfangen, welcher auf euch kommen wird, und
werdet meine Zeugen sein zu Jerusalem und in ganz Judäa und Samarien und
bis an das Ende der Erde. \footnote{\textbf{1:8} Apg 8,11; Lk 24,48}

\bibverse{9} Und da er solches gesagt, ward er aufgehoben zusehends, und
eine Wolke nahm ihn auf vor ihren Augen weg. \footnote{\textbf{1:9} Mk
  16,19; Lk 24,51} \bibverse{10} Und als sie ihm nachsahen, wie er gen
Himmel fuhr, siehe, da standen bei ihnen zwei Männer in weißen Kleidern,
\footnote{\textbf{1:10} Lk 24,4} \bibverse{11} welche auch sagten: Ihr
Männer von Galiläa, was stehet ihr und sehet gen Himmel? Dieser Jesus,
welcher von euch ist aufgenommen gen Himmel, wird kommen, wie ihr ihn
gesehen habt gen Himmel fahren. \footnote{\textbf{1:11} Lk 21,27}

\bibverse{12} Da wandten sie um gen Jerusalem von dem Berge, der da
heißt der Ölberg, welcher ist nahe bei Jerusalem und liegt einen
Sabbatweg davon. \footnote{\textbf{1:12} Lk 24,50; Lk 24,52-53}
\bibverse{13} Und als sie hineinkamen, stiegen sie auf den Söller, da
denn sich aufhielten Petrus und Jakobus, Johannes und Andreas, Philippus
und Thomas, Bartholomäus und Matthäus, Jakobus, des Alphäus Sohn, und
Simon Zelotes und Judas, des Jakobus Sohn. \footnote{\textbf{1:13} Lk
  6,13-16} \bibverse{14} Diese alle waren stets beieinander einmütig mit
Beten und Flehen samt den Weibern und Maria, der Mutter Jesu, und seinen
Brüdern. \footnote{\textbf{1:14} Joh 7,3}

\bibverse{15} Und in den Tagen trat auf Petrus unter die Jünger und
sprach (es war aber eine Schar zuhauf bei hundertzwanzig Namen):
\footnote{\textbf{1:15} Joh 21,15-19} \bibverse{16} Ihr Männer und
Brüder, es musste die Schrift erfüllet werden, welche zuvor gesagt hat
der Heilige Geist durch den Mund Davids von Judas, der ein Führer war
derer, die Jesus fingen; \footnote{\textbf{1:16} Ps 41,10} \bibverse{17}
denn er war zu uns gezählt und hatte dies Amt mit uns überkommen.
\bibverse{18} Dieser hat erworben den Acker um den ungerechten Lohn und
ist abgestürzt und mitten entzweigeborsten, und all sein Eingeweide
ausgeschüttet. \footnote{\textbf{1:18} Mt 27,3-10} \bibverse{19} Und es
ist kund geworden allen, die zu Jerusalem wohnen, also dass dieser Acker
genannt wird auf ihre Sprache: Hakeldama (das ist: ein Blutacker).
\bibverse{20} Denn es steht geschrieben im Psalmbuch: „Seine Behausung
müsse wüst werden, und sei niemand, der darin wohne``, und: „Sein Bistum
empfange ein anderer.``

\bibverse{21} So muss nun einer unter diesen Männern, die bei uns
gewesen sind die ganze Zeit über, welche der Herr Jesus unter uns ist
aus und ein gegangen,

\bibverse{22} von der Taufe des Johannes an bis auf den Tag, da er von
uns genommen ist, ein Zeuge seiner Auferstehung mit uns werden.

\bibverse{23} Und sie stellten zwei, Joseph, genannt Barsabas, mit dem
Zunamen Just, und Matthias, \bibverse{24} beteten und sprachen: Herr,
aller Herzen Kündiger, zeige an, welchen du erwählt hast unter diesen
zweien, \footnote{\textbf{1:24} Apg 6,6} \bibverse{25} dass einer
empfange diesen Dienst und Apostelamt, davon Judas abgewichen ist, dass
er hinginge an seinen Ort. \bibverse{26} Und sie warfen das Los über
sie, und das Los fiel auf Matthias; und er ward zugeordnet zu den elf
Aposteln. \# 2 \bibverse{1} Und als der Tag der Pfingsten erfüllt war,
waren sie alle einmütig beieinander. \footnote{\textbf{2:1} 3Mo 23,15-21}
\bibverse{2} Und es geschah schnell ein Brausen vom Himmel wie eines
gewaltigen Windes und erfüllte das ganze Haus, da sie saßen.
\bibverse{3} Und es erschienen ihnen Zungen, zerteilt, wie vom Feuer;
und er setzte sich auf einen jeglichen unter ihnen; \bibverse{4} und sie
wurden alle voll des Heiligen Geistes und fingen an, zu predigen mit
anderen Zungen, nach dem der Geist ihnen gab auszusprechen. \footnote{\textbf{2:4}
  Apg 4,31; Apg 10,44-46}

\bibverse{5} Es waren aber Juden zu Jerusalem wohnend, die waren
gottesfürchtige Männer aus allerlei Volk, das unter dem Himmel ist.
\footnote{\textbf{2:5} Apg 13,26} \bibverse{6} Da nun diese Stimme
geschah, kam die Menge zusammen und wurden bestürzt; denn es hörte ein
jeglicher, dass sie mit seiner Sprache redeten. \bibverse{7} Sie
entsetzten sich aber alle, verwunderten sich und sprachen untereinander:
Siehe, sind nicht diese alle, die da reden, aus Galiläa? \bibverse{8}
Wie hören wir denn ein jeglicher seine Sprache, darin wir geboren sind?
\bibverse{9} Parther und Meder und Elamiter, und die wir wohnen in
Mesopotamien und in Judäa und Kappadozien, Pontus und Asien,
\bibverse{10} Phrygien und Pamphylien, Ägypten und an den Enden von
Libyen bei Kyrene und Ausländer von Rom, \bibverse{11} Juden und
Judengenossen, Kreter und Araber: wir hören sie mit unseren Zungen die
großen Taten Gottes reden. \bibverse{12} Sie entsetzten sich aber alle
und wurden irre und sprachen einer zu dem anderen: Was will das werden?
\bibverse{13} Die anderen aber hatten ihren Spott und sprachen: Sie sind
voll süßen Weins.

\bibverse{14} Da trat Petrus auf mit den Elfen, erhob seine Stimme und
redete zu ihnen: Ihr Juden, liebe Männer, und alle, die ihr zu Jerusalem
wohnet, das sei euch kundgetan, und lasset meine Worte zu euren Ohren
eingehen. \bibverse{15} Denn diese sind nicht trunken, wie ihr wähnet --
sintemal es ist die dritte Stunde am Tage --; \bibverse{16} sondern das
ist's, was durch den Propheten Joel zuvor gesagt ist: \bibverse{17} „Und
es soll geschehen in den letzten Tagen, spricht Gott, ich will ausgießen
von meinem Geist auf alles Fleisch; und eure Söhne und eure Töchter
sollen weissagen, und eure Jünglinge sollen Gesichte sehen, und eure
Ältesten sollen Träume haben; \bibverse{18} und auf meine Knechte und
auf meine Mägde will ich in denselben Tagen von meinem Geist ausgießen,
und sie sollen weissagen. \bibverse{19} Und ich will Wunder tun oben im
Himmel und Zeichen unten auf Erden: Blut und Feuer und Rauchdampf;
\bibverse{20} die Sonne soll sich verkehren in Finsternis und der Mond
in Blut, ehe denn der große und offenbare Tag des Herrn kommt.
\bibverse{21} Und soll geschehen, wer den Namen des Herrn anrufen wird,
soll selig werden.``

\bibverse{22} Ihr Männer von Israel, höret diese Worte: Jesum von
Nazareth, den Mann, von Gott unter euch mit Taten und Wundern und
Zeichen erwiesen, welche Gott durch ihn tat unter euch (wie denn auch
ihr selbst wisset), \bibverse{23} denselben (nachdem er aus bedachtem
Rat und Vorsehung Gottes übergeben war) habt ihr genommen durch die
Hände der Ungerechten und ihn angeheftet und erwürgt. \footnote{\textbf{2:23}
  Apg 4,28} \bibverse{24} Den hat Gott auferweckt, und aufgelöst die
Schmerzen des Todes, wie es denn unmöglich war, dass er sollte von ihm
gehalten werden. \bibverse{25} Denn David spricht von ihm: „Ich habe den
Herrn allezeit vorgesetzt vor mein Angesicht; denn er ist an meiner
Rechten, auf dass ich nicht bewegt werde. \bibverse{26} Darum ist mein
Herz fröhlich, und meine Zunge freuet sich; denn auch mein Fleisch wird
ruhen in der Hoffnung. \bibverse{27} Denn du wirst meine Seele nicht dem
Tode lassen, auch nicht zugeben, dass dein Heiliger die Verwesung sehe.
\bibverse{28} Du hast mir kundgetan die Wege des Lebens; du wirst mich
erfüllen mit Freuden vor deinem Angesicht.``

\bibverse{29} Ihr Männer, liebe Brüder, lasset mich frei reden zu euch
von dem Erzvater David. Er ist gestorben und begraben, und sein Grab ist
bei uns bis auf diesen Tag. \bibverse{30} Da er nun ein Prophet war und
wusste, dass ihm Gott verheißen hatte mit einem Eide, dass die Frucht
seiner Lenden sollte auf seinem Stuhl sitzen, \footnote{\textbf{2:30} Ps
  89,4-5; 2Sam 7,12-13} \bibverse{31} hat er's zuvor gesehen und geredet
von der Auferstehung Christi, dass seine Seele nicht dem Tode gelassen
ist und sein Fleisch die Verwesung nicht gesehen hat. \bibverse{32}
Diesen Jesus hat Gott auferweckt; des sind wir alle Zeugen.
\bibverse{33} Nun er durch die Rechte Gottes erhöht ist und empfangen
hat die Verheißung des heiligen Geistes vom Vater, hat er ausgegossen
dies, das ihr sehet und höret. \bibverse{34} Denn David ist nicht gen
Himmel gefahren. Er spricht aber: „Der Herr hat gesagt zu meinem Herrn:
Setze dich zu meiner Rechten, \bibverse{35} bis dass ich deine Feinde
lege zum Schemel deiner Füße.``

\bibverse{36} So wisse nun das ganze Haus Israel gewiss, dass Gott
diesen Jesus, den ihr gekreuzigt habt, zu einem Herrn und Christus
gemacht hat. \footnote{\textbf{2:36} Apg 5,31}

\bibverse{37} Da sie aber das hörten, ging's ihnen durchs Herz, und
sprachen zu Petrus und zu den anderen Aposteln: Ihr Männer, liebe
Brüder, was sollen wir tun? \footnote{\textbf{2:37} Apg 16,30; Lk 3,10}

\bibverse{38} Petrus sprach zu ihnen: Tut Buße und lasse sich ein
jeglicher taufen auf den Namen Jesu Christi zur Vergebung der Sünden, so
werdet ihr empfangen die Gabe des heiligen Geistes. \footnote{\textbf{2:38}
  Apg 3,17-19; Lk 24,47} \bibverse{39} Denn euer und eurer Kinder ist
diese Verheißung und aller, die ferne sind, welche Gott, unser Herr,
herzurufen wird. \footnote{\textbf{2:39} Joe 3,5} \bibverse{40} Auch mit
vielen anderen Worten bezeugte er und ermahnte und sprach: Lasset euch
erretten aus diesem verkehrten Geschlecht! \footnote{\textbf{2:40} Mt
  17,17; Phil 2,15}

\bibverse{41} Die nun sein Wort gern annahmen, ließen sich taufen; und
wurden hinzugetan an dem Tage bei dreitausend Seelen. \bibverse{42} Sie
blieben aber beständig in der Apostel Lehre und in der Gemeinschaft und
im Brotbrechen und im Gebet. \bibverse{43} Es kam auch alle Seelen
Furcht an, und geschahen viel Wunder und Zeichen durch die Apostel.
\bibverse{44} Alle aber, die gläubig waren geworden, waren beieinander
und hielten alle Dinge gemein. \footnote{\textbf{2:44} Apg 4,32-35}
\bibverse{45} Ihre Güter und Habe verkauften sie und teilten sie aus
unter alle, nach dem jedermann not war. \bibverse{46} Und sie waren
täglich und stets beieinander einmütig im Tempel und brachen das Brot
hin und her in Häusern, \bibverse{47} nahmen die Speise und lobten Gott
mit Freuden und einfältigem Herzen und hatten Gnade bei dem ganzen Volk.
Der Herr aber tat hinzu täglich, die da selig wurden, zu der Gemeinde.
\# 3 \bibverse{1} Petrus aber und Johannes gingen miteinander hinauf in
den Tempel um die neunte Stunde, da man pflegt zu beten. \bibverse{2}
Und es war ein Mann, lahm von Mutterleibe, der ließ sich tragen; und sie
setzten ihn täglich vor des Tempels Tür, die da heißt „die schöne``,
dass er bettelte das Almosen von denen, die in den Tempel gingen.
\bibverse{3} Da er nun sah Petrus und Johannes, dass sie wollten zum
Tempel hineingehen, bat er um ein Almosen. \bibverse{4} Petrus aber sah
ihn an mit Johannes und sprach: Sieh uns an! \bibverse{5} Und er sah sie
an, wartete, dass er etwas von ihnen empfinge. \bibverse{6} Petrus aber
sprach: Silber und Gold habe ich nicht; was ich aber habe, das gebe ich
dir: Im Namen Jesu Christi von Nazareth stehe auf und wandle!
\bibverse{7} Und griff ihn bei der rechten Hand und richtete ihn auf.
Alsobald standen seine Schenkel und Knöchel fest; \bibverse{8} sprang
auf, konnte gehen und stehen und ging mit ihnen in den Tempel, wandelte
und sprang und lobte Gott. \bibverse{9} Und es sah ihn alles Volk
wandeln und Gott loben. \bibverse{10} Sie kannten ihn auch, dass er's
war, der um das Almosen gesessen hatte vor der schönen Tür des Tempels;
und sie wurden voll Wunderns und Entsetzens über das, was ihm
widerfahren war. \bibverse{11} Als aber dieser Lahme, der nun gesund
war, sich zu Petrus und Johannes hielt, lief alles Volk zu ihnen in die
Halle, die da heißt Salomos, und wunderten sich. \footnote{\textbf{3:11}
  Apg 5,12; Joh 10,23}

\bibverse{12} Als Petrus das sah, antwortete er dem Volk: Ihr Männer von
Israel, was wundert ihr euch darüber, oder was sehet ihr auf uns, als
hätten wir diesen wandeln gemacht durch unsere eigene Kraft oder
Verdienst? \bibverse{13} Der Gott Abrahams und Isaaks und Jakobs, der
Gott unserer Väter, hat seinen Knecht Jesus verklärt, welchen ihr
überantwortet und verleugnet habt vor Pilatus, da der urteilte, ihn
loszulassen. \bibverse{14} Ihr aber verleugnetet den Heiligen und
Gerechten und batet, dass man euch den Mörder schenkte; \bibverse{15}
aber den Fürsten des Lebens habt ihr getötet. Den hat Gott auferweckt
von den Toten; des sind wir Zeugen. \bibverse{16} Und durch den Glauben
an seinen Namen hat diesen, den ihr sehet und kennet, sein Name stark
gemacht; und der Glaube durch ihn hat diesem gegeben diese Gesundheit
vor euren Augen.

\bibverse{17} Nun, liebe Brüder, ich weiß, dass ihr's durch Unwissenheit
getan habt wie auch eure Obersten. \footnote{\textbf{3:17} Lk 23,34}
\bibverse{18} Gott aber, was er durch den Mund aller seiner Propheten
zuvor verkündigt hat, wie Christus leiden sollte, hat's also erfüllet.
\footnote{\textbf{3:18} Lk 24,44}

\bibverse{19} So tut nun Buße und bekehret euch, dass eure Sünden
vertilgt werden; \footnote{\textbf{3:19} Apg 2,38} \bibverse{20} auf
dass da komme die Zeit der Erquickung von dem Angesichte des Herrn, wenn
er senden wird den, der euch jetzt zuvor gepredigt wird, Jesus Christus,
\bibverse{21} welcher muss den Himmel einnehmen bis auf die Zeit, da
herwiedergebracht werde alles, was Gott geredet hat durch den Mund aller
seiner heiligen Propheten von der Welt an. \bibverse{22} Denn Mose hat
gesagt zu den Vätern: „Einen Propheten wird euch der Herr, euer Gott,
erwecken aus euren Brüdern gleich wie mich; den sollt ihr hören in
allem, was er zu euch sagen wird. \bibverse{23} Und es wird geschehen,
welche Seele denselben Propheten nicht hören wird, die soll vertilgt
werden aus dem Volk.`` \bibverse{24} Und alle Propheten von Samuel an
und hernach, wieviel ihrer geredet haben, die haben von diesen Tagen
verkündigt. \bibverse{25} Ihr seid der Propheten und des Bundes Kinder,
welchen Gott gemacht hat mit euren Vätern, da er sprach zu Abraham:
„Durch deinen Samen sollen gesegnet werden alle Völker auf Erden.``
\bibverse{26} Euch zuvörderst hat Gott auferweckt seinen Knecht Jesus
und hat ihn zu euch gesandt, euch zu segnen, dass ein jeglicher sich
bekehre von seiner Bosheit. \footnote{\textbf{3:26} Apg 13,46}

\hypertarget{section-1}{%
\section{4}\label{section-1}}

\bibverse{1} Als sie aber zum Volk redeten, traten zu ihnen die Priester
und der Hauptmann des Tempels und die Sadduzäer \footnote{\textbf{4:1}
  Lk 22,4; Lk 22,52} \bibverse{2} (sie verdross, dass sie das Volk
lehrten und verkündigten an Jesu die Auferstehung von den Toten)
\footnote{\textbf{4:2} Apg 23,8} \bibverse{3} und legten die Hände an
sie und setzten sie ein bis auf morgen; denn es war jetzt Abend.
\bibverse{4} Aber viele unter denen, die dem Wort zuhörten, wurden
gläubig; und ward die Zahl der Männer bei fünftausend.

\bibverse{5} Als es nun kam auf den Morgen, versammelten sich ihre
Obersten und Ältesten und Schriftgelehrten gen Jerusalem, \bibverse{6}
Hannas, der Hohepriester, und Kaiphas und Johannes und Alexander und wie
viel ihrer waren vom Hohenpriestergeschlecht; \footnote{\textbf{4:6} Lk
  3,1} \bibverse{7} und stellten sie vor sich und fragten sie: Aus
welcher Gewalt oder in welchem Namen habt ihr das getan? \footnote{\textbf{4:7}
  Mt 21,33}

\bibverse{8} Petrus, voll des heiligen Geistes, sprach zu ihnen: Ihr
Obersten des Volkes und ihr Ältesten von Israel, \footnote{\textbf{4:8}
  Mt 10,19-20} \bibverse{9} so wir heute werden gerichtet über dieser
Wohltat an dem kranken Menschen, durch welche er ist geheilt worden,
\bibverse{10} so sei euch und allem Volk von Israel kundgetan, dass in
dem Namen Jesu Christi von Nazareth, welchen ihr gekreuzigt habt, den
Gott von den Toten auferweckt hat, steht dieser allhier vor euch gesund.
\bibverse{11} Das ist der Stein, von euch Bauleuten verworfen, der zum
Eckstein geworden ist. \footnote{\textbf{4:11} Mt 21,42} \bibverse{12}
Und ist in keinem anderen Heil, ist auch kein anderer Name unter dem
Himmel den Menschen gegeben, darin wir sollen selig werden. \footnote{\textbf{4:12}
  Apg 10,43; Mt 1,21}

\bibverse{13} Sie sahen aber an die Freudigkeit des Petrus und Johannes
und verwunderten sich; denn sie waren gewiss, dass es ungelehrte Leute
und Laien waren, und kannten sie auch wohl, dass sie mit Jesu gewesen
waren. \bibverse{14} Sie sahen aber den Menschen, der geheilt worden
war, bei ihnen stehen und hatten nichts dawider zu reden. \footnote{\textbf{4:14}
  Apg 3,8-9} \bibverse{15} Da hießen sie sie hinausgehen aus dem Rat und
handelten miteinander und sprachen: \bibverse{16} Was wollen wir diesen
Menschen tun? Denn das offenbare Zeichen, durch sie geschehen, ist kund
allen, die zu Jerusalem wohnen, und wir können's nicht leugnen.
\bibverse{17} Aber auf dass es nicht weiter einreiße unter das Volk,
lasset uns ernstlich sie bedrohen, dass sie hinfort keinem Menschen von
diesem Namen sagen. \bibverse{18} Und riefen sie und geboten ihnen, dass
sie sich allerdinge nicht hören ließen noch lehrten in dem Namen Jesu.

\bibverse{19} Petrus aber und Johannes antworteten und sprachen zu
ihnen: Richtet ihr selbst, ob es vor Gott recht sei, dass wir euch mehr
gehorchen denn Gott. \footnote{\textbf{4:19} Apg 5,28-29} \bibverse{20}
Wir können's ja nicht lassen, dass wir nicht reden sollten, was wir
gesehen und gehört haben.

\bibverse{21} Aber sie drohten ihnen und ließen sie gehen und fanden
nicht, wie sie sie peinigten, um des Volkes willen; denn sie lobten alle
Gott über das, was geschehen war. \bibverse{22} Denn der Mensch war über
vierzig Jahre alt, an welchem dies Zeichen der Gesundheit geschehen war.

\bibverse{23} Und als man sie hatte gehen lassen, kamen sie zu den Ihren
und verkündigten ihnen, was die Hohenpriester und Ältesten zu ihnen
gesagt hatten. \bibverse{24} Da sie das hörten, hoben sie ihre Stimme
auf einmütig zu Gott und sprachen: Herr, der du bist der Gott, der
Himmel und Erde und das Meer und alles, was darinnen ist, gemacht hat;
\bibverse{25} der du durch den Mund Davids, deines Knechtes, gesagt
hast: „Warum empören sich die Heiden, und die Völker nehmen vor, was
umsonst ist? \bibverse{26} Die Könige der Erde treten zusammen, und die
Fürsten versammeln sich zuhauf wider den Herrn und wider seinen
Christus``:

\bibverse{27} wahrlich ja, sie haben sich versammelt über deinen
heiligen Knecht Jesus, welchen du gesalbt hast, Herodes und Pontius
Pilatus mit den Heiden und dem Volk Israel, \bibverse{28} zu tun, was
deine Hand und dein Rat zuvor bedacht hat, dass es geschehen sollte.
\footnote{\textbf{4:28} Apg 2,23} \bibverse{29} Und nun, Herr, siehe an
ihr Drohen und gib deinen Knechten, mit aller Freudigkeit zu reden dein
Wort, \footnote{\textbf{4:29} Eph 6,19} \bibverse{30} und strecke deine
Hand aus, dass Gesundheit und Zeichen und Wunder geschehen durch den
Namen deines heiligen Knechtes Jesus.

\bibverse{31} Und da sie gebetet hatten, bewegte sich die Stätte, da sie
versammelt waren; und sie wurden alle des heiligen Geistes voll und
redeten das Wort Gottes mit Freudigkeit.

\bibverse{32} Die Menge aber der Gläubigen war ein Herz und eine Seele;
auch keiner sagte von seinen Gütern, dass sie sein wären, sondern es war
ihnen alles gemein. \footnote{\textbf{4:32} Apg 2,44} \bibverse{33} Und
mit großer Kraft gaben die Apostel Zeugnis von der Auferstehung des
Herrn Jesu, und war große Gnade bei ihnen allen. \footnote{\textbf{4:33}
  Apg 2,47} \bibverse{34} Es war auch keiner unter ihnen, der Mangel
hatte; denn wie viel ihrer waren, die da Äcker oder Häuser hatten, die
verkauften sie und brachten das Geld des verkauften Guts \footnote{\textbf{4:34}
  Apg 2,45} \bibverse{35} und legten es zu der Apostel Füßen; und man
gab einem jeglichen, was ihm not war.

\bibverse{36} Joses aber, mit dem Zunamen von den Aposteln genannt
Barnabas (das heißt: ein Sohn des Trostes), von Geschlecht ein Levit aus
Zypern, \bibverse{37} der hatte einen Acker und verkaufte ihn und
brachte das Geld und legte es zu der Apostel Füßen. \# 5 \bibverse{1}
Ein Mann aber, mit Namen Ananias samt seinem Weibe Saphira verkaufte
sein Gut \bibverse{2} und entwandte etwas vom Gelde mit Wissen seines
Weibes und brachte einen Teil und legte ihn zu der Apostel Füßen.
\footnote{\textbf{5:2} Apg 4,34-37} \bibverse{3} Petrus aber sprach:
Ananias, warum hat der Satan dein Herz erfüllt, dass du dem heiligen
Geist lögest und entwendetest etwas vom Gelde des Ackers? \bibverse{4}
Hättest du ihn doch wohl mögen behalten, da du ihn hattest; und da er
verkauft war, war es auch in deiner Gewalt. Warum hast du denn solches
in deinem Herzen vorgenommen? Du hast nicht Menschen, sondern Gott
gelogen.

\bibverse{5} Da Ananias aber diese Worte hörte, fiel er nieder und gab
den Geist auf. Und es kam eine große Furcht über alle, die dies hörten.
\bibverse{6} Es standen aber die Jünglinge auf und taten ihn beiseite
und trugen ihn hinaus und begruben ihn. \bibverse{7} Und es begab sich
über eine Weile, bei drei Stunden, dass sein Weib hineinkam und wusste
nicht, was geschehen war. \bibverse{8} Aber Petrus antwortete ihr: Sage
mir: Habt ihr den Acker so teuer verkauft? Sie sprach: Ja, so teuer.

\bibverse{9} Petrus aber sprach zu ihr: Warum seid ihr denn eins
geworden, zu versuchen den Geist des Herrn? Siehe, die Füße derer, die
deinen Mann begraben haben, sind vor der Tür und werden dich
hinaustragen.

\bibverse{10} Und alsbald fiel sie zu seinen Füßen und gab den Geist
auf. Da kamen die Jünglinge und fanden sie tot, trugen sie hinaus und
begruben sie neben ihren Mann.

\bibverse{11} Und es kam eine große Furcht über die ganze Gemeinde und
über alle, die solches hörten.

\bibverse{12} Es geschahen aber viel Zeichen und Wunder im Volk durch
der Apostel Hände; und sie waren alle in der Halle Salomos einmütig.
\bibverse{13} Der anderen aber wagte keiner, sich zu ihnen zu tun,
sondern das Volk hielt groß von ihnen. \bibverse{14} Es wurden aber
immer mehr hinzugetan, die da glaubten an den Herrn, eine Menge Männer
und Weiber, \footnote{\textbf{5:14} Apg 2,47} \bibverse{15} also dass
sie die Kranken auf die Gassen heraustrugen und legten sie auf Betten
und Bahren, auf dass, wenn Petrus käme, sein Schatten ihrer etliche
überschattete. \footnote{\textbf{5:15} Apg 19,11-12} \bibverse{16} Es
kamen auch herzu viele von den umliegenden Städten gen Jerusalem und
brachten die Kranken und die von unsauberen Geistern gepeinigt waren;
und wurden alle gesund.

\bibverse{17} Es stand aber auf der Hohepriester und alle, die mit ihm
waren, welches ist die Sekte der Sadduzäer, und wurden voll Eifers
\footnote{\textbf{5:17} Apg 4,1; Apg 4,6} \bibverse{18} und legten die
Hände an die Apostel und warfen sie in das gemeine Gefängnis.
\bibverse{19} Aber der Engel des Herrn tat in der Nacht die Türen des
Gefängnisses auf und führte sie heraus und sprach: \bibverse{20} Gehet
hin und tretet auf und redet im Tempel zum Volk alle Worte dieses
Lebens.

\bibverse{21} Da sie das gehört hatten, gingen sie früh in den Tempel
und lehrten. Der Hohepriester aber kam und die mit ihm waren und riefen
zusammen den Rat und alle Ältesten der Kinder Israel und sandten hin zum
Gefängnis, sie zu holen. \bibverse{22} Die Diener aber kamen hin und
fanden sie nicht im Gefängnis, kamen wieder und verkündigten
\bibverse{23} und sprachen: Das Gefängnis fanden wir verschlossen mit
allem Fleiß und die Hüter außen stehen vor den Türen; aber da wir
auftaten, fanden wir niemand darin.

\bibverse{24} Da diese Rede hörten der Hohenpriester und der Hauptmann
des Tempels und andere Hohepriester, wurden sie darüber betreten, was
doch das werden wollte. \bibverse{25} Da kam einer, der verkündigte
ihnen: Siehe, die Männer, die ihr ins Gefängnis geworfen habt, sind im
Tempel, stehen und lehren das Volk. \bibverse{26} Da ging hin der
Hauptmann mit den Dienern und holten sie, nicht mit Gewalt; denn sie
fürchteten sich vor dem Volk, dass sie gesteinigt würden.

\bibverse{27} Und als sie sie brachten, stellten sie sie vor den Rat.
Und der Hohepriester fragte sie \bibverse{28} und sprach: Haben wir euch
nicht mit Ernst geboten, dass ihr nicht solltet lehren in diesem Namen?
Und sehet, ihr habt Jerusalem erfüllt mit eurer Lehre und wollt dieses
Menschen Blut über uns führen. \footnote{\textbf{5:28} Apg 4,18; Mt
  27,25}

\bibverse{29} Petrus aber antwortete und die Apostel und sprachen: Man
muss Gott mehr gehorchen denn den Menschen. \footnote{\textbf{5:29} Apg
  4,19; Dan 3,16-18} \bibverse{30} Der Gott unserer Väter hat Jesum
auferweckt, welchen ihr erwürgt habt und an das Holz gehängt.
\footnote{\textbf{5:30} Apg 3,15} \bibverse{31} Den hat Gott durch seine
rechte Hand erhöht zu einem Fürsten und Heiland, zu geben Israel Buße
und Vergebung der Sünden. \footnote{\textbf{5:31} Apg 2,33}
\bibverse{32} Und wir sind seine Zeugen über diese Worte und der heilige
Geist, welchen Gott gegeben hat denen, die ihm gehorchen. \footnote{\textbf{5:32}
  Lk 24,48; Joh 15,26-27}

\bibverse{33} Da sie das hörten, ging's ihnen durchs Herz, und dachten,
sie zu töten. \bibverse{34} Da stand aber auf im Rat ein Pharisäer mit
Namen Gamaliel, ein Schriftgelehrter, in Ehren gehalten vor allem Volk,
und hieß die Apostel ein wenig hinaustun \bibverse{35} und sprach zu
ihnen: Ihr Männer von Israel, nehmet euer selbst wahr an diesen
Menschen, was ihr tun sollt. \bibverse{36} Vor diesen Tagen stand auf
Theudas und gab vor, er wäre etwas, und hingen an ihm eine Zahl Männer,
bei vierhundert; der ist erschlagen, und alle, die ihm zufielen, sind
zerstreut und zunichte geworden. \bibverse{37} Darnach stand auf Judas
aus Galiläa in den Tagen der Schätzung und machte viel Volks abfällig
ihm nach; und der ist auch umgekommen, und alle, die ihm zufielen, sind
zerstreut. \bibverse{38} Und nun sage ich euch: Lasset ab von diesen
Menschen und lasset sie fahren! Ist der Rat oder das Werk aus den
Menschen, so wird's untergehen; \footnote{\textbf{5:38} Mt 15,13}
\bibverse{39} ist's aber aus Gott, so könnet ihr's nicht dämpfen; auf
dass ihr nicht erfunden werdet als die wider Gott streiten wollen.

\bibverse{40} Da fielen sie ihm zu und riefen die Apostel, stäupten sie
und geboten ihnen, sie sollten nicht Reden in dem Namen Jesu, und ließen
sie gehen. \bibverse{41} Sie gingen aber fröhlich von des Rats
Angesicht, dass sie würdig gewesen waren, um seines Namens willen
Schmach zu leiden, \footnote{\textbf{5:41} Mt 5,10-12; 1Petr 4,13}

\bibverse{42} und hörten nicht auf, alle Tage im Tempel und hin und her
in Häusern zu lehren und zu predigen das Evangelium von Jesu Christo. \#
6 \bibverse{1} In den Tagen aber, da der Jünger viele wurden, erhob sich
ein Murmeln unter den Griechen wider die Hebräer, darum dass ihre Witwen
übersehen wurden in der täglichen Handreichung. \bibverse{2} Da riefen
die Zwölf die Menge der Jünger zusammen und sprachen: Es taugt nicht,
dass wir das Wort Gottes unterlassen und zu Tische dienen. \bibverse{3}
Darum, ihr lieben Brüder, sehet unter euch nach sieben Männern, die ein
gut Gerücht haben und voll heiligen Geistes und Weisheit sind, welche
wir bestellen mögen zu dieser Notdurft. \footnote{\textbf{6:3} 1Tim
  3,8-10} \bibverse{4} Wir aber wollen anhalten am Gebet und am Amt des
Wortes.

\bibverse{5} Und die Rede gefiel der ganzen Menge wohl; und sie
erwählten Stephanus, einen Mann voll Glaubens und heiligen Geistes, und
Philippus und Prochorus und Nikanor und Timon und Parmenas und Nikolaus,
den Judengenossen von Antiochien. \bibverse{6} Diese stellten sie vor
die Apostel und beteten und legten die Hände auf sie. \footnote{\textbf{6:6}
  Apg 1,24; Apg 13,3; Apg 14,23}

\bibverse{7} Und das Wort Gottes nahm zu, und die Zahl der Jünger ward
sehr groß zu Jerusalem. Es wurden auch viele Priester dem Glauben
gehorsam. \footnote{\textbf{6:7} Apg 2,47; Apg 19,20}

\bibverse{8} Stephanus aber, voll Glaubens und Kräfte, tat Wunder und
große Zeichen unter dem Volk. \bibverse{9} Da standen etliche auf von
der Schule, die da heißt der Libertiner und der Kyrener und der
Alexanderer, und derer, die aus Zilizien und Asien waren, und befragten
sich mit Stephanus. \bibverse{10} Und sie vermochten nicht, zu
widerstehen der Weisheit und dem Geiste, aus welchem er redete.
\footnote{\textbf{6:10} Lk 21,15} \bibverse{11} Da richteten sie zu
etliche Männer, die sprachen: Wir haben ihn gehört Lästerworte reden
wider Mose und wider Gott. \footnote{\textbf{6:11} Mt 26,60-66}
\bibverse{12} Und sie bewegten das Volk und die Ältesten und die
Schriftgelehrten und traten herzu und rissen ihn hin und führten ihn vor
den Rat \bibverse{13} und stellten falsche Zeugen dar, die sprachen:
Dieser Mensch hört nicht auf, zu reden Lästerworte wider diese heilige
Stätte und das Gesetz. \footnote{\textbf{6:13} Jer 26,11} \bibverse{14}
Denn wir haben ihn hören sagen: Jesus von Nazareth wird diese Stätte
zerstören und ändern die Sitten, die uns Mose gegeben hat. \footnote{\textbf{6:14}
  Joh 2,19} \bibverse{15} Und sie sahen auf ihn alle, die im Rat saßen,
und sahen sein Angesicht wie eines Engels Angesicht. \# 7 \bibverse{1}
Da sprach der Hohepriester: Ist dem also?

\bibverse{2} Er aber sprach: Liebe Brüder und Väter, höret zu. Der Gott
der Herrlichkeit erschien unserem Vater Abraham, da er noch in
Mesopotamien war, ehe er wohnte in Haran, \footnote{\textbf{7:2} 1Mo
  11,1-50; Jos 24,32} \bibverse{3} und sprach zu ihm: Gehe aus deinem
Lande und von deiner Freundschaft und zieh in ein Land, das ich dir
zeigen will. \bibverse{4} Da ging er aus der Chaldäer Lande und wohnte
in Haran. Und von dort, da sein Vater gestorben war, brachte er ihn
herüber in dieses Land, darin ihr nun wohnet, \bibverse{5} und gab ihm
kein Erbteil darin, auch nicht einen Fuß breit, und verhieß ihm, er
wollte es geben ihm zu besitzen und seinem Samen nach ihm, da er noch
kein Kind hatte. \bibverse{6} Aber Gott sprach also: Dein Same wird ein
Fremdling sein in einem fremden Lande, und sie werden ihn dienstbar
machen und übel behandeln vierhundert Jahre; \bibverse{7} und das Volk,
dem sie dienen werden, will ich richten, sprach Gott; und darnach werden
sie ausziehen und mir dienen an dieser Stätte. \bibverse{8} Und gab ihm
den Bund der Beschneidung. Und er zeugte Isaak und beschnitt ihn am
achten Tage, und Isaak den Jakob, und Jakob die zwölf Erzväter.

\bibverse{9} Und die Erzväter neideten Joseph und verkauften ihn nach
Ägypten; aber Gott war mit ihm \bibverse{10} und errettete ihn aus aller
seiner Trübsal und gab ihm Gnade und Weisheit vor Pharao, dem König in
Ägypten; der setzte ihn zum Fürsten über Ägypten und über sein ganzes
Haus. \bibverse{11} Es kam aber eine teure Zeit über das ganze Land
Ägypten und Kanaan und eine große Trübsal, und unsere Väter fanden nicht
Nahrung. \bibverse{12} Jakob aber hörte, dass in Ägypten Getreide wäre,
und sandte unsere Väter aus aufs erstemal. \bibverse{13} Und zum
andernmal ward Joseph erkannt von seinen Brüdern, und ward dem Pharao
Josephs Geschlecht offenbar. \bibverse{14} Joseph aber sandte aus und
ließ holen seinen Vater Jakob und seine ganze Freundschaft,
fünfundsiebzig Seelen. \bibverse{15} Und Jakob zog hinab nach Ägypten
und starb, er und unsere Väter. \bibverse{16} Und sie sind
herübergebracht nach Sichem und gelegt in das Grab, das Abraham gekauft
hatte ums Geld von den Kindern Hemor zu Sichem.

\bibverse{17} Da nun sich die Zeit der Verheißung nahte, die Gott
Abraham geschworen hatte, wuchs das Volk und mehrte sich in Ägypten,
\footnote{\textbf{7:17} 2Mo 1,1-3} \bibverse{18} bis dass ein anderer
König aufkam, der nichts wusste von Joseph. \bibverse{19} Dieser trieb
Hinterlist mit unserem Geschlecht und behandelte unsere Väter übel und
schaffte, dass man die jungen Kindlein aussetzen musste, dass sie nicht
lebendig blieben. \bibverse{20} Zu der Zeit ward Mose geboren, und war
ein feines Kind vor Gott und ward drei Monate ernährt in seines Vaters
Hause. \bibverse{21} Als er aber ausgesetzt ward, nahm ihn die Tochter
Pharaos auf und zog ihn auf, ihr selbst zu einem Sohn. \bibverse{22} Und
Mose ward gelehrt in aller Weisheit der Ägypter und war mächtig in
Werken und Worten. \bibverse{23} Da er aber vierzig Jahre alt ward,
gedachte er zu sehen nach seinen Brüdern, den Kindern von Israel.
\bibverse{24} Und sah einen Unrecht leiden; da stand er bei und rächte
den, dem Leid geschah, und erschlug den Ägypter. \bibverse{25} Er meinte
aber, seine Brüder sollten's verstehen, dass Gott durch seine Hand ihnen
Heil gäbe; aber sie verstanden's nicht.

\bibverse{26} Und am anderen Tage kam er zu ihnen, da sie miteinander
haderten, und handelte mit ihnen, dass sie Frieden hätten, und sprach:
Liebe Männer, ihr seid Brüder, warum tut einer dem anderen Unrecht?
\bibverse{27} Der aber seinem Nächsten Unrecht tat, stieß ihn von sich
und sprach: Wer hat dich über uns gesetzt zum Obersten und Richter?
\bibverse{28} Willst du mich auch töten, wie du gestern den Ägypter
getötet hast? \bibverse{29} Mose aber floh wegen dieser Rede und ward
ein Fremdling im Lande Midian; daselbst zeugte er zwei Söhne.

\bibverse{30} Und über vierzig Jahre erschien ihm in der Wüste an dem
Berge Sinai der Engel des Herrn in einer Feuerflamme im Busch.
\bibverse{31} Da es aber Mose sah, wunderte er sich des Gesichtes. Als
er aber hinzuging zu schauen, geschah die Stimme des Herrn zu ihm:
\bibverse{32} Ich bin der Gott deiner Väter, der Gott Abrahams und der
Gott Isaaks und der Gott Jakobs. Mose aber ward zitternd und wagte nicht
anzuschauen. \bibverse{33} Aber der Herr sprach zu ihm: Zieh die Schuhe
aus von deinen Füßen; denn die Stätte, da du stehst, ist heilig Land!
\bibverse{34} Ich habe wohl gesehen das Leiden meines Volkes, das in
Ägypten ist, und habe ihr Seufzen gehört und bin herabgekommen, sie zu
erretten. Und nun komm her, ich will dich nach Ägypten senden.

\bibverse{35} Diesen Mose, welchen sie verleugneten, da sie sprachen:
Wer hat dich zum Obersten und Richter gesetzt? den sandte Gott zu einem
Obersten und Erlöser durch die Hand des Engels, der ihm erschien im
Busch. \bibverse{36} Dieser führte sie aus und tat Wunder und Zeichen in
Ägypten, im Roten Meer und in der Wüste vierzig Jahre. \footnote{\textbf{7:36}
  2Mo 7,10; 2Mo 14,21} \bibverse{37} Dies ist der Mose, der zu den
Kindern Israel gesagt hat: „Einen Propheten wird euch der Herr, euer
Gott, erwecken aus euren Brüdern gleichwie mich; den sollt ihr hören.``
\bibverse{38} Dieser ist's, der in der Gemeinde in der Wüste mit dem
Engel war, der mit ihm redete auf dem Berge Sinai und mit unseren
Vätern; dieser empfing lebendige Worte, uns zu geben; \bibverse{39}
welchem nicht wollten gehorsam werden eure Väter, sondern stießen ihn
von sich und wandten sich um mit ihren Herzen nach Ägypten \bibverse{40}
und sprachen zu Aaron: Mache uns Götter, die vor uns hin gehen; denn wir
wissen nicht, was diesem Mose, der uns aus dem Lande Ägypten geführt
hat, widerfahren ist. \bibverse{41} Und sie machten ein Kalb zu der Zeit
und brachten dem Götzen Opfer und freuten sich der Werke ihrer Hände.
\bibverse{42} Aber Gott wandte sich und gab sie dahin, das sie dienten
des Himmels Heer; wie denn geschrieben steht in dem Buch der Propheten:
„Habt ihr vom Hause Israel die vierzig Jahre in der Wüste mir auch je
Opfer und Vieh geopfert? \bibverse{43} Und ihr nahmet die Hütte Molochs
an und das Gestirn eures Gottes Remphan, die Bilder, die ihr gemacht
hattet, sie anzubeten. Und ich will euch wegwerfen jenseits Babylon.``

\bibverse{44} Es hatten unsere Väter die Hütte des Zeugnisses in der
Wüste, wie ihnen das verordnet hatte, der zu Mose redete, dass er sie
machen sollte nach dem Vorbilde, das er gesehen hatte; \footnote{\textbf{7:44}
  2Mo 25,-1} \bibverse{45} welche unsere Väter auch annahmen und mit
Josua in das Land brachten, das die Heiden innehatten, welche Gott
ausstieß vor dem Angesicht unserer Väter bis zur Zeit Davids.
\footnote{\textbf{7:45} Jos 3,14; Jos 18,1} \bibverse{46} Der fand Gnade
bei Gott und bat, dass er eine Wohnung finden möchte für den Gott
Jakobs. \footnote{\textbf{7:46} 2Sam 7,-1; Ps 132,1-5} \bibverse{47}
Salomo aber baute ihm ein Haus. \footnote{\textbf{7:47} 1Kö 6,-1}
\bibverse{48} Aber der Allerhöchste wohnt nicht in Tempeln, die mit
Händen gemacht sind, wie der Prophet spricht: \bibverse{49} „Der Himmel
ist mein Stuhl und die Erde meiner Füße Schemel; was wollt ihr mir denn
für ein Haus bauen? spricht der Herr, oder welches ist die Stätte meiner
Ruhe? \bibverse{50} Hat nicht meine Hand das alles gemacht?{}``

\bibverse{51} Ihr Halsstarrigen und Unbeschnittenen an Herzen und Ohren,
ihr widerstrebt allezeit dem heiligen Geist, wie eure Väter also auch
ihr. \footnote{\textbf{7:51} 2Mo 32,9; 3Mo 26,41; Röm 2,28-29}
\bibverse{52} Welchen Propheten haben eure Väter nicht verfolgt? Und sie
haben getötet, die da zuvor verkündigten die Zukunft dieses Gerechten,
dessen Verräter und Mörder ihr nun geworden seid. \footnote{\textbf{7:52}
  2Chr 36,16; Mt 23,31} \bibverse{53} Ihr habt das Gesetz empfangen
durch der Engel Geschäfte, und habt's nicht gehalten. \footnote{\textbf{7:53}
  2Mo 20,-1; Gal 3,19; Hebr 2,2}

\bibverse{54} Da sie solches hörten, ging's ihnen durchs Herz, und
bissen die Zähne zusammen über ihn. \bibverse{55} Wie er aber voll
heiligen Geistes war, sah er auf gen Himmel und sah die Herrlichkeit
Gottes und Jesum stehen zur Rechten Gottes \bibverse{56} und sprach:
Siehe, ich sehe den Himmel offen und des Menschen Sohn zur Rechten
Gottes stehen.

\bibverse{57} Sie schrien aber laut und hielten ihre Ohren zu und
stürmten einmütig auf ihn ein, stießen ihn zur Stadt hinaus und
steinigten ihn. \bibverse{58} Und die Zeugen legten ihre Kleider ab zu
den Füßen eines Jünglings, der hieß Saulus, \footnote{\textbf{7:58} Apg
  22,20; 3Mo 24,16} \bibverse{59} und steinigten Stephanus, der anrief
und sprach: Herr Jesu, nimm meinen Geist auf! \footnote{\textbf{7:59} Lk
  23,46} \bibverse{60} Er kniete aber nieder und schrie laut: Herr,
behalte ihnen diese Sünde nicht! Und als er das gesagt, entschlief er.
\# 8 \bibverse{1} Saulus aber hatte Wohlgefallen an seinem Tode. Es
erhob sich aber zu der Zeit eine große Verfolgung über die Gemeinde zu
Jerusalem; und sie zerstreuten sich alle in die Länder Judäa und
Samarien, außer den Aposteln. \footnote{\textbf{8:1} Apg 18,-1; Apg
  11,19} \bibverse{2} Es bestatteten aber Stephanus gottesfürchtige
Männer und hielten eine große Klage über ihn. \bibverse{3} Saulus aber
verstörte die Gemeinde, ging hin und her in die Häuser und zog hervor
Männer und Weiber und überantwortete sie ins Gefängnis. \bibverse{4} Die
nun zerstreut waren, gingen um und predigten das Wort. \bibverse{5}
Philippus aber kam hinab in eine Stadt in Samarien und predigte ihnen
von Christo. \footnote{\textbf{8:5} Apg 6,5} \bibverse{6} Das Volk aber
hörte einmütig und fleißig zu, was Philippus sagte, und sah die Zeichen,
die er tat. \bibverse{7} Denn die unsauberen Geister fuhren aus vielen
Besessenen mit großem Geschrei; auch viele Gichtbrüchige und Lahme
wurden gesund gemacht. \bibverse{8} Und es ward eine große Freude in
derselben Stadt.

\bibverse{9} Es war aber ein Mann, mit Namen Simon, der zuvor in der
Stadt Zauberei trieb und bezauberte das samaritische Volk und gab vor,
er wäre etwas Großes. \bibverse{10} Und sie sahen alle auf ihn, beide,
klein und groß, und sprachen: Der ist die Kraft Gottes, die da groß ist.
\bibverse{11} Sie sahen aber darum auf ihn, dass er sie lange Zeit mit
seiner Zauberei bezaubert hatte. \bibverse{12} Da sie aber den Predigten
des Philippus glaubten vom Reich Gottes und von dem Namen Jesu Christi,
ließen sich taufen Männer und Weiber. \bibverse{13} Da ward auch Simon
gläubig und ließ sich taufen und hielt sich zu Philippus. Und als er sah
die Zeichen und Taten, die da geschahen, verwunderte er sich.

\bibverse{14} Da aber die Apostel hörten zu Jerusalem, dass Samarien das
Wort Gottes angenommen hatte, sandten sie zu ihnen Petrus und Johannes,
\bibverse{15} welche, da sie hinabkamen, beteten sie über sie, dass sie
den heiligen Geist empfingen. \bibverse{16} (Denn er war noch auf keinen
gefallen, sondern sie waren allein getauft auf den Namen Christi Jesu.)
\bibverse{17} Da legten sie die Hände auf sie, und sie empfingen den
heiligen Geist. \bibverse{18} Da aber Simon sah, dass der heilige Geist
gegeben ward, wenn die Apostel die Hände auflegten, bot er ihnen Geld an
\bibverse{19} und sprach: Gebt mir auch die Macht, dass, wenn ich jemand
die Hände auflege, derselbe den heiligen Geist empfange. \bibverse{20}
Petrus aber sprach zu ihm: Dass du verdammt werdest mit deinem Gelde,
darum dass du meinst, Gottes Gabe werde durch Geld erlangt!
\bibverse{21} Du wirst weder Teil noch Anfall haben an diesem Wort; denn
dein Herz ist nicht rechtschaffen vor Gott. \bibverse{22} Darum tue Buße
für diese deine Bosheit und bitte Gott, ob dir vergeben werden möchte
die Tücke deines Herzens. \bibverse{23} Denn ich sehe, dass du bist voll
bitterer Galle und verknüpft mit Ungerechtigkeit.

\bibverse{24} Da antwortete Simon und sprach: Bittet ihr den Herrn für
mich, dass der keines über mich komme, davon ihr gesagt habt.

\bibverse{25} Sie aber, da sie bezeugt und geredet hatten das Wort des
Herrn, wandten sich wieder um gen Jerusalem und predigten das Evangelium
vielen samaritischen Flecken.

\bibverse{26} Aber der Engel des Herrn redete zu Philippus und sprach:
Stehe auf und gehe gegen Mittag auf die Straße, die von Jerusalem geht
hinab gen Gaza, die da wüst ist.

\bibverse{27} Und er stand auf und ging hin. Und siehe, ein Mann aus
Mohrenland, ein Kämmerer und Gewaltiger der Königin Kandaze in
Mohrenland, welcher war über ihre ganze Schatzkammer, der war gekommen
gen Jerusalem, anzubeten, \bibverse{28} und zog wieder heim und saß auf
seinem Wagen und las den Propheten Jesaja.

\bibverse{29} Der Geist aber sprach zu Philippus: Gehe hinzu und halte
dich zu diesem Wagen!

\bibverse{30} Da lief Philippus hinzu und hörte, dass er den Propheten
Jesaja las, und sprach: Verstehst du auch, was du liesest?

\bibverse{31} Er aber sprach: Wie kann ich, so mich nicht jemand
anleitet? Und ermahnte Philippus, dass er aufträte und setzte sich zu
ihm. \bibverse{32} Der Inhalt aber der Schrift, die er las, war dieser:
„Er ist wie ein Schaf zur Schlachtung geführt; und still wie ein Lamm
vor seinem Scherer, also hat er nicht aufgetan seinen Mund.
\bibverse{33} In seiner Niedrigkeit ist sein Gericht aufgehoben. Wer
wird aber seines Lebens Länge ausreden? denn sein Leben ist von der Erde
weggenommen.``

\bibverse{34} Da antwortete der Kämmerer dem Philippus und sprach: Ich
bitte dich, von wem redet der Prophet solches? von sich selber oder von
jemand anders?

\bibverse{35} Philippus aber tat seinen Mund auf und fing von dieser
Schrift an und predigte ihm das Evangelium von Jesu. \bibverse{36} Und
als sie zogen der Straße nach, kamen sie an ein Wasser. Und der Kämmerer
sprach: Siehe, da ist Wasser; was hindert's, dass ich mich taufen lasse?

\bibverse{37} Philippus aber sprach: Glaubst du von ganzem Herzen, so
mag's wohl sein. Er antwortete und sprach: Ich glaube, dass Jesus
Christus Gottes Sohn ist. \footnote{\textbf{8:37} Mt 16,16}
\bibverse{38} Und er hieß den Wagen halten, und stiegen hinab in das
Wasser beide, Philippus und der Kämmerer, und er taufte ihn.

\bibverse{39} Da sie aber heraufstiegen aus dem Wasser, rückte der Geist
des Herrn Philippus hinweg, und der Kämmerer sah ihn nicht mehr; er zog
aber seine Straße fröhlich. \bibverse{40} Philippus aber ward gefunden
zu Asdod und wandelte umher und predigte allen Städten das Evangelium,
bis dass er kam gen Cäsarea. \footnote{\textbf{8:40} Apg 21,8-9}

\hypertarget{section-2}{%
\section{9}\label{section-2}}

\bibverse{1} Saulus aber schnaubte noch mit Drohen und Morden wider die
Jünger des Herrn und ging zum Hohenpriester \footnote{\textbf{9:1} Apg
  8,3; Apg 22,3-16; Apg 26,9-18} \bibverse{2} und bat ihn um Briefe gen
Damaskus an die Schulen, auf dass, wenn er etliche dieses Weges fände,
Männer und Weiber, er sie gebunden führte gen Jerusalem. \bibverse{3}
Und da er auf dem Wege war und nahe an Damaskus kam, umleuchtete ihn
plötzlich ein Licht vom Himmel; \footnote{\textbf{9:3} 1Kor 15,8}
\bibverse{4} und er fiel auf die Erde und hörte eine Stimme, die sprach
zu ihm: Saul, Saul, was verfolgst du mich?

\bibverse{5} Er aber sprach: Herr, wer bist du? Der Herr sprach: Ich bin
Jesus, den du verfolgst. Es wird dir schwer werden, wider den Stachel zu
lecken.

\bibverse{6} Und er sprach mit Zittern und Zagen: Herr, was willst du,
dass ich tun soll? Der Herr sprach zu ihm: Stehe auf und gehe in die
Stadt; da wird man dir sagen, was du tun sollst.

\bibverse{7} Die Männer aber, die seine Gefährten waren, standen und
waren erstarrt; denn sie hörten die Stimme, und sahen niemand.
\bibverse{8} Saulus aber richtete sich auf von der Erde; und als er
seine Augen auftat, sah er niemand. Sie nahmen ihn aber bei der Hand und
führten ihn gen Damaskus; \bibverse{9} und er war drei Tage nicht sehend
und aß nicht und trank nicht.

\bibverse{10} Es war aber ein Jünger zu Damaskus mit Namen Ananias; zu
dem sprach der Herr im Gesicht: Ananias! Und er sprach: Hier bin ich,
Herr.

\bibverse{11} Der Herr sprach zu ihm: Stehe auf und gehe in die Gasse,
die da heißt „die gerade``, und frage im Hause des Judas nach einem
namens Saul von Tarsus; denn siehe, er betet --

\bibverse{12} und hat gesehen im Gesicht einen Mann mit Namen Ananias zu
ihm hineinkommen und die Hand auf ihn legen, dass er wieder sehend
werde.

\bibverse{13} Ananias aber antwortete: Herr, ich habe von vielen gehört
von diesem Manne, wieviel Übles er deinen Heiligen getan hat zu
Jerusalem; \bibverse{14} und er hat allhier Macht von den
Hohenpriestern, zu binden alle, die deinen Namen anrufen.

\bibverse{15} Der Herr sprach zu ihm: Gehe hin; denn dieser ist mir ein
auserwähltes Rüstzeug, dass er meinen Namen trage vor den Heiden und vor
den Königen und vor den Kindern von Israel. \bibverse{16} Ich will ihm
zeigen, wieviel er leiden muss um meines Namens willen. \footnote{\textbf{9:16}
  2Kor 11,23-28}

\bibverse{17} Und Ananias ging hin und kam in das Haus und legte die
Hände auf ihn und sprach: Lieber Bruder Saul, der Herr hat mich gesandt
(der dir erschienen ist auf dem Wege, da du her kamst), dass du wieder
sehend und mit dem heiligen Geist erfüllt werdest. \bibverse{18} Und
alsobald fiel es von seinen Augen wie Schuppen, und er ward wieder
sehend \bibverse{19} und stand auf, ließ sich taufen und nahm Speise zu
sich und stärkte sich. Saulus aber war eine Zeitlang bei den Jüngern zu
Damaskus.

\bibverse{20} Und alsbald predigte er Christum in den Schulen, dass
derselbe Gottes Sohn sei. \bibverse{21} Sie entsetzten sich aber alle,
die es hörten, und sprachen: Ist das nicht, der zu Jerusalem verstörte
alle, die diesen Namen anrufen, und darum hergekommen, dass er sie
gebunden führe zu den Hohenpriestern?

\bibverse{22} Saulus aber ward immer kräftiger und trieb die Juden in
die Enge, die zu Damaskus wohnten, und bewährte es, dass dieser ist der
Christus. \footnote{\textbf{9:22} Apg 18,28} \bibverse{23} Und nach
vielen Tagen hielten die Juden einen Rat zusammen, dass sie ihn töteten.
\bibverse{24} Aber es ward Saulus kundgetan, dass sie ihm nachstellten.
Sie hüteten aber Tag und Nacht an den Toren, dass sie ihn töteten.
\bibverse{25} Da nahmen ihn die Jünger bei der Nacht und taten ihn durch
die Mauer und ließen ihn in einem Korbe hinab.

\bibverse{26} Da aber Saulus gen Jerusalem kam, versuchte er, sich zu
den Jüngern zu tun; und sie fürchteten sich alle vor ihm und glaubten
nicht, dass er ein Jünger wäre. \footnote{\textbf{9:26} Gal 1,17-19}
\bibverse{27} Barnabas aber nahm ihn zu sich und führte ihn zu den
Aposteln und erzählte ihnen, wie er auf der Straße den Herrn gesehen und
er mit ihm geredet und wie er zu Damaskus den Namen Jesus frei gepredigt
hätte. \bibverse{28} Und er war bei ihnen und ging aus und ein zu
Jerusalem und predigte den Namen des Herrn Jesu frei. \bibverse{29} Er
redete auch und befragte sich mit den Griechen; aber sie stellten ihm
nach, dass sie ihn töteten. \bibverse{30} Da das die Brüder erfuhren,
geleiteten sie ihn gen Cäsarea und schickten ihn gen Tarsus.

\bibverse{31} So hatte nun die Gemeinde Frieden durch ganz Judäa und
Galiläa und Samarien und baute sich und wandelte in der Furcht des Herrn
und ward erfüllt mit Trost des heiligen Geistes.

\bibverse{32} Es geschah aber, da Petrus durchzog allenthalben, dass er
auch zu den Heiligen kam, die zu Lydda wohnten. \bibverse{33} Daselbst
fand er einen Mann mit Namen Äneas, acht Jahre lang auf dem Bette
gelegen, der war gichtbrüchig. \bibverse{34} Und Petrus sprach zu ihm:
Äneas, Jesus Christus macht dich gesund; stehe auf und bette dir selber!
Und alsobald stand er auf. \bibverse{35} Und es sahen ihn alle, die zu
Lydda und in Saron wohnten; die bekehrten sich zu dem Herrn.

\bibverse{36} Zu Joppe aber war eine Jüngerin mit Namen Tabea (welches
verdolmetscht heißt: Rehe), die war voll guter Werke und Almosen, die
sie tat. \bibverse{37} Es begab sich aber zu der Zeit, dass sie krank
ward und starb. Da wuschen sie dieselbe und legten sie auf den Söller.
\bibverse{38} Nun aber Lydda nahe bei Joppe ist, da die Jünger hörten,
dass Petrus daselbst war, sandten sie zwei Männer zu ihm und ermahnten
ihn, dass er sich's nicht ließe verdrießen, zu ihnen zu kommen.
\bibverse{39} Petrus aber stand auf und kam mit ihnen. Und als er
hingekommen war, führten sie ihn hinauf auf den Söller, und traten um
ihn alle Witwen, weinten und zeigten ihm die Röcke und Kleider, welche
die Rehe machte, als sie noch bei ihnen war. \bibverse{40} Und da Petrus
sie alle hinausgetrieben hatte, kniete er nieder, betete und wandte sich
zu dem Leichnam und sprach: Tabea, stehe auf! Und sie tat ihre Augen
auf; und da sie Petrus sah, setzte sie sich wieder. \footnote{\textbf{9:40}
  Mk 5,41} \bibverse{41} Er aber gab ihr die Hand und richtete sie auf
und rief die Heiligen und die Witwen und stellte sie lebendig dar.
\bibverse{42} Und es ward kund durch ganz Joppe, und viele wurden
gläubig an den Herrn. \bibverse{43} Und es geschah, dass er lange Zeit
zu Joppe blieb bei einem Simon, der ein Gerber war. \# 10 \bibverse{1}
Es war aber ein Mann zu Cäsarea, mit Namen Kornelius, ein Hauptmann von
der Schar, die da heißt die italische, \bibverse{2} gottselig und
gottesfürchtig samt seinem ganzen Hause, und gab dem Volk viel Almosen
und betete immer zu Gott. \bibverse{3} Der sah in einem Gesicht
offenbarlich um die neunte Stunde am Tage einen Engel Gottes zu sich
eingehen, der sprach zu ihm: Kornelius!

\bibverse{4} Er aber sah ihn an, erschrak und sprach: Herr, was ist's?
Er aber sprach zu ihm: Deine Gebete und deine Almosen sind
hinaufgekommen ins Gedächtnis vor Gott.

\bibverse{5} Und nun sende Männer gen Joppe und lass fordern Simon, mit
dem Zunamen Petrus, \bibverse{6} welcher ist zur Herberge bei einem
Gerber Simon, des Haus am Meer liegt; der wird dir sagen, was du tun
sollst.

\bibverse{7} Und da der Engel, der mit Kornelius redete, hinweggegangen
war, rief er zwei seiner Hausknechte und einen gottesfürchtigen
Kriegsknecht von denen, die ihm aufwarteten, \bibverse{8} und erzählte
es ihnen alles und sandte sie gen Joppe. \footnote{\textbf{10:8} Apg
  11,5-17}

\bibverse{9} Des anderen Tages, da diese auf dem Wege waren, und nahe
zur Stadt kamen, stieg Petrus hinauf auf den Söller, zu beten, um die
sechste Stunde. \bibverse{10} Und als er hungrig ward, wollte er essen.
Da sie ihm aber zubereiteten, ward er entzückt \bibverse{11} und sah den
Himmel aufgetan und herniederfahren zu ihm ein Gefäß wie ein großes
leinenes Tuch, an vier Zipfeln gebunden, und es ward niedergelassen auf
die Erde. \bibverse{12} Darin waren allerlei vierfüßige Tiere der Erde
und wilde Tiere und Gewürm und Vögel des Himmels. \bibverse{13} Und es
geschah eine Stimme zu ihm: Stehe auf, Petrus, schlachte und iss!

\bibverse{14} Petrus aber sprach: O nein, Herr; denn ich habe noch nie
etwas Gemeines oder Unreines gegessen.

\bibverse{15} Und die Stimme sprach zum andernmal zu ihm: Was Gott
gereinigt hat, das mache du nicht gemein. \footnote{\textbf{10:15} Röm
  14,14} \bibverse{16} Und das geschah zu drei Malen; und das Gefäß ward
wieder aufgenommen gen Himmel.

\bibverse{17} Als aber Petrus sich in sich selbst bekümmerte, was das
Gesicht wäre, das er gesehen hatte, siehe, da fragten die Männer, von
Kornelius gesandt, nach dem Hause Simons und standen an der Tür,
\bibverse{18} riefen und forschten, ob Simon, mit dem Zunamen Petrus,
allda zur Herberge wäre. \bibverse{19} Indem aber Petrus nachsann über
das Gesicht, sprach der Geist zu ihm: Siehe, drei Männer suchen dich;
\bibverse{20} aber stehe auf, steig hinab und zieh mit ihnen und zweifle
nicht; denn ich habe sie gesandt.

\bibverse{21} Da stieg Petrus hinab zu den Männern, die von Kornelius zu
ihm gesandt waren, und sprach: Siehe, ich bin's, den ihr suchet; was ist
die Sache, darum ihr hier seid?

\bibverse{22} Sie aber sprachen: Kornelius, der Hauptmann, ein frommer
und gottesfürchtiger Mann und gutes Gerüchts bei dem ganzen Volk der
Juden, hat Befehl empfangen von einem heiligen Engel, dass er dich
sollte fordern lassen in sein Haus und Worte von dir hören.
\bibverse{23} Da rief er sie hinein und herbergte sie. Des anderen Tages
zog Petrus aus mit ihnen, und etliche Brüder von Joppe gingen mit ihm.

\bibverse{24} Und des anderen Tages kamen sie gen Cäsarea. Kornelius
aber wartete auf sie und hatte zusammengerufen seine Verwandten und
Freunde. \bibverse{25} Und als Petrus hineinkam, ging ihm Kornelius
entgegen und fiel zu seinen Füßen und betete ihn an. \bibverse{26}
Petrus aber richtete ihn auf und sprach: Stehe auf, ich bin auch ein
Mensch. \bibverse{27} Und als er sich mit ihm besprochen hatte, ging er
hinein und fand ihrer viele, die zusammengekommen waren. \bibverse{28}
Und er sprach zu ihnen: Ihr wisset, wie es ein unerlaubtes Ding ist
einem jüdischen Mann, sich zu tun oder zu kommen zu einem Fremdling;
aber Gott hat mir gezeigt, keinen Menschen gemein oder unrein zu heißen.
\bibverse{29} Darum habe ich mich nicht geweigert zu kommen, als ich
ward hergefordert. So frage ich euch nun, warum ihr mich habt lassen
fordern?

\bibverse{30} Kornelius sprach: Ich habe vier Tage gefastet, bis an
diese Stunde, und um die neunte Stunde betete ich in meinen Hause. Und
siehe, da stand ein Mann vor mir in einem hellen Kleid \bibverse{31} und
sprach: Kornelius, dein Gebet ist erhört, und deiner Almosen ist gedacht
worden vor Gott. \bibverse{32} So sende nun gen Joppe und lass herrufen
einen Simon, mit dem Zunamen Petrus, welcher ist zur Herberge in dem
Hause des Gerbers Simon an dem Meer; der wird, wenn er kommt, mit dir
reden. \bibverse{33} Da sandte ich von Stund an zu dir; und du hast wohl
getan, dass du gekommen bist. Nun sind wir alle hier gegenwärtig vor
Gott, zu hören alles, was dir von Gott befohlen ist.

\bibverse{34} Petrus aber tat seinen Mund auf und sprach: Nun erfahre
ich mit der Wahrheit, dass Gott die Person nicht ansieht; \footnote{\textbf{10:34}
  1Sam 16,7; Röm 2,11} \bibverse{35} sondern in allerlei Volk, wer ihn
fürchtet und recht tut, der ist ihm angenehm. \footnote{\textbf{10:35}
  Joh 10,16} \bibverse{36} Ihr wisset wohl von der Predigt, die Gott zu
den Kindern Israel gesandt hat, und dass er hat den Frieden verkündigen
lassen durch Jesum Christum (welcher ist ein Herr über alles),
\footnote{\textbf{10:36} Eph 2,17} \bibverse{37} die durchs ganze
jüdische Land geschehen ist und angegangen in Galiläa nach der Taufe,
die Johannes predigte: \footnote{\textbf{10:37} Mt 4,12-17}
\bibverse{38} wie Gott diesen Jesus von Nazareth gesalbt hat mit dem
heiligen Geist und Kraft; der umhergezogen ist und hat wohlgetan und
gesund gemacht alle, die vom Teufel überwältigt waren; denn Gott war mit
ihm. \footnote{\textbf{10:38} Mt 3,16} \bibverse{39} Und wir sind Zeugen
alles des, das er getan hat im jüdischen Lande und zu Jerusalem. Den
haben sie getötet und an ein Holz gehängt. \bibverse{40} Den hat Gott
auferweckt am dritten Tage und ihn lassen offenbar werden, \bibverse{41}
nicht allem Volk, sondern uns, den vorerwählten Zeugen von Gott, die wir
mit ihm gegessen und getrunken haben, nachdem er auferstanden war von
den Toten. \footnote{\textbf{10:41} Joh 14,19; Joh 14,22; Lk 24,30; Lk
  24,43} \bibverse{42} Und er hat uns geboten, zu predigen dem Volk und
zu zeugen, dass er ist verordnet von Gott zum Richter der Lebendigen und
der Toten. \footnote{\textbf{10:42} Joh 5,22} \bibverse{43} Von diesem
zeugen alle Propheten, dass durch seinen Namen alle, die an ihn glauben,
Vergebung der Sünden empfangen sollen. \footnote{\textbf{10:43} Jes
  53,5-6; Jer 31,34}

\bibverse{44} Da Petrus noch diese Worte redete, fiel der heilige Geist
auf alle, die dem Wort zuhörten. \bibverse{45} Und die Gläubigen aus den
Juden, die mit Petrus gekommen waren, entsetzten sich, dass auch auf die
Heiden die Gabe des heiligen Geistes ausgegossen ward; \bibverse{46}
denn sie hörten, dass sie mit Zungen redeten und Gott hoch priesen. Da
antwortete Petrus:

\bibverse{47} Mag auch jemand das Wasser wehren, dass diese nicht
getauft werden, die den heiligen Geist empfangen haben gleichwie auch
wir? \bibverse{48} Und befahl, sie zu taufen in dem Namen des Herrn. Da
baten sie ihn, dass er etliche Tage dabliebe. \# 11 \bibverse{1} Es kam
aber vor die Apostel und Brüder, die in dem jüdischen Lande waren, dass
auch die Heiden hätten Gottes Wort angenommen. \bibverse{2} Und da
Petrus hinaufkam gen Jerusalem, zankten mit ihm, die aus den Juden
waren, \bibverse{3} und sprachen: Du bist eingegangen zu den Männern,
die unbeschnitten sind, und hast mit ihnen gegessen. \footnote{\textbf{11:3}
  Gal 2,12}

\bibverse{4} Petrus aber hob an und erzählte es ihnen nacheinander her
und sprach: \bibverse{5} Ich war in der Stadt Joppe im Gebete und war
entzückt und sah ein Gesicht, nämlich ein Gefäß herniederfahren, wie ein
großes leinenes Tuch mit vier Zipfeln, und niedergelassen vom Himmel,
das kam bis zu mir. \bibverse{6} Darein sah ich und ward gewahr und sah
vierfüßige Tiere der Erde und wilde Tiere und Gewürm und Vögel des
Himmels. \bibverse{7} Ich hörte aber eine Stimme, die sprach zu mir:
Stehe auf, Petrus, schlachte und iss! \bibverse{8} Ich aber sprach: O
nein, Herr; denn es ist nie etwas Gemeines oder Unreines in meinen Mund
gegangen. \bibverse{9} Aber die Stimme antwortete mir zum andernmal vom
Himmel: Was Gott gereinigt hat, das mache du nicht gemein. \bibverse{10}
Das geschah aber dreimal; und alles ward wieder hinauf gen Himmel
gezogen. \bibverse{11} Und siehe, von Stund an standen drei Männer vor
dem Hause, darin ich war, gesandt von Cäsarea zu mir. \bibverse{12} Der
Geist aber sprach zu mir, ich sollte mit ihnen gehen und nicht zweifeln.
Es kamen aber mit mir diese sechs Brüder, und wir gingen in des Mannes
Haus. \bibverse{13} Und er verkündigte uns, wie er gesehen hätte einen
Engel in seinem Hause stehen, der zu ihm gesprochen hätte: Sende Männer
gen Joppe und lass fordern den Simon, mit dem Zunamen Petrus;
\bibverse{14} der wird dir Worte sagen, dadurch du selig werdest und
dein ganzes Haus. \bibverse{15} Indem aber ich anfing zu reden, fiel der
heilige Geist auf sie gleichwie auf uns am ersten Anfang. \footnote{\textbf{11:15}
  Apg 2,1-4} \bibverse{16} Da dachte ich an das Wort des Herrn, als er
sagte: „Johannes hat mit Wasser getauft; ihr aber sollt mit dem heiligen
Geist getauft werden.`` \footnote{\textbf{11:16} Apg 1,5} \bibverse{17}
So nun Gott ihnen die gleiche Gabe gegeben hat wie auch uns, die da
glauben an den Herrn Jesus Christus: wer war ich, dass ich konnte Gott
wehren?

\bibverse{18} Da sie das hörten, schwiegen sie still und lobten Gott und
sprachen: So hat Gott auch den Heiden Buße gegeben zum Leben!

\bibverse{19} Die aber zerstreut waren in der Trübsal, die sich über
Stephanus erhob, gingen umher bis gen Phönizien und Zypern und
Antiochien und redeten das Wort zu niemand denn allein zu den Juden.
\footnote{\textbf{11:19} Apg 8,1-4} \bibverse{20} Es waren aber etliche
unter ihnen, Männer von Zypern und Kyrene, die kamen gen Antiochien und
redeten auch zu den Griechen und predigten das Evangelium vom Herrn
Jesus. \bibverse{21} Und die Hand des Herrn war mit ihnen, und eine
große Zahl ward gläubig und bekehrte sich zu dem Herrn. \bibverse{22} Es
kam aber diese Rede von ihnen vor die Ohren der Gemeinde zu Jerusalem;
und sie sandten Barnabas, dass er hinginge bis gen Antiochien.
\footnote{\textbf{11:22} Apg 4,36} \bibverse{23} Dieser, da er
hingekommen war und sah die Gnade Gottes, ward er froh und ermahnte sie
alle, dass sie mit festem Herzen an dem Herrn bleiben wollten.
\bibverse{24} Denn er war ein frommer Mann, voll heiligen Geistes und
Glaubens. Und es ward ein großes Volk dem Herrn zugetan.

\bibverse{25} Barnabas aber zog aus gen Tarsus, Saulus wieder zu suchen;
\footnote{\textbf{11:25} Apg 9,30} \bibverse{26} und da er ihn fand,
führte er ihn gen Antiochien. Und sie blieben bei der Gemeinde ein
ganzes Jahr und lehrten viel Volks; daher die Jünger am ersten zu
Antiochien Christen genannt wurden. \footnote{\textbf{11:26} Gal 2,11}

\bibverse{27} In diesen Tagen kamen Propheten von Jerusalem gen
Antiochien. \footnote{\textbf{11:27} Apg 13,1; Apg 15,32} \bibverse{28}
Und einer unter ihnen mit Namen Agabus stand auf und deutete durch den
Geist eine große Teuerung, die da kommen sollte über den ganzen Kreis
der Erde; welche geschah unter dem Kaiser Klaudius. \footnote{\textbf{11:28}
  Apg 21,10} \bibverse{29} Aber unter den Jüngern beschloss ein
jeglicher, nach dem er vermochte, zu senden eine Handreichung den
Brüdern, die in Judäa wohnten; \bibverse{30} wie sie denn auch taten,
und schickten's zu den Ältesten durch die Hand des Barnabas und Saulus.
\footnote{\textbf{11:30} Apg 12,25; 1Kor 16,1-4}

\hypertarget{section-3}{%
\section{12}\label{section-3}}

\bibverse{1} Um diese Zeit legte der König Herodes die Hände an etliche
von der Gemeinde, sie zu peinigen. \bibverse{2} Er tötete aber Jakobus,
den Bruder des Johannes, mit dem Schwert. \bibverse{3} Und da er sah,
dass es den Juden gefiel, fuhr er fort und fing Petrus auch. Es waren
aber eben die Tage der süßen Brote. \bibverse{4} Da er ihn nun griff,
legte er ihn ins Gefängnis und überantwortete ihn vier Rotten, je von
vier Kriegsknechten, ihn zu bewahren, und gedachte, ihn nach Ostern dem
Volk vorzustellen. \bibverse{5} Und Petrus ward zwar im Gefängnis
gehalten; aber die Gemeinde betete ohne Aufhören für ihn zu Gott.
\bibverse{6} Und da ihn Herodes wollte vorstellen, in derselben Nacht
schlief Petrus zwischen zwei Kriegsknechten, gebunden mit zwei Ketten,
und die Hüter vor der Tür hüteten das Gefängnis.

\bibverse{7} Und siehe, der Engel des Herrn kam daher, und ein Licht
schien in dem Gemach; und er schlug Petrus an die Seite und weckte ihn
und sprach: Stehe behende auf! Und die Ketten fielen ihm von seinen
Händen. \footnote{\textbf{12:7} Apg 5,19} \bibverse{8} Und der Engel
sprach zu ihm: Gürte dich und tu deine Schuhe an! Und er tat also. Und
er sprach zu ihm: Wirf deinen Mantel um dich und folge mir nach!
\bibverse{9} Und er ging hinaus und folgte ihm und wusste nicht, dass
ihm wahrhaftig solches geschähe durch den Engel; sondern es deuchte ihn,
er sähe ein Gesicht. \bibverse{10} Sie gingen aber durch die erste und
andere Hut und kamen zu der eisernen Tür, welche zur Stadt führt; die
tat sich ihnen von selber auf. Und sie traten hinaus und gingen hin eine
Gasse lang; und alsobald schied der Engel von ihm.

\bibverse{11} Und da Petrus zu sich selber kam, sprach er: Nun weiß ich
wahrhaftig, dass der Herr seinen Engel gesandt hat und mich errettet aus
der Hand des Herodes und von allem Warten des jüdischen Volkes.
\bibverse{12} Und als er sich besann, kam er vor das Haus Marias, der
Mutter des Johannes, der mit dem Zunamen Markus hieß, da viele
beieinander waren und beteten. \bibverse{13} Als aber Petrus an die Tür
des Tores klopfte, trat hervor eine Magd, zu horchen, mit Namen Rhode.
\bibverse{14} Und als sie des Petrus Stimme erkannte, tat sie das Tor
nicht auf vor Freuden, lief aber hinein und verkündigte es ihnen, Petrus
stünde vor dem Tor.

\bibverse{15} Sie aber sprachen zu ihr: Du bist unsinnig. Sie aber
bestand darauf, es wäre also. Sie sprachen: Es ist sein Engel.
\bibverse{16} Petrus klopfte weiter an. Da sie aber auftaten, sahen sie
ihn und entsetzten sich. \bibverse{17} Er aber winkte ihnen mit der
Hand, zu schweigen, und erzählte ihnen, wie ihn der Herr hatte aus dem
Gefängnis geführt, und sprach: Verkündiget dies Jakobus und den Brüdern.
Und ging hinaus und zog an einen anderen Ort.

\bibverse{18} Da es aber Tag ward, war eine nicht kleine Bekümmernis
unter den Kriegsknechten, wie es doch mit Petrus gegangen wäre.
\footnote{\textbf{12:18} Apg 5,21-22} \bibverse{19} Herodes aber, da er
ihn forderte und nicht fand, ließ die Hüter verhören und hieß sie
wegführen; und zog von Judäa hinab gen Cäsarea und hielt allda sein
Wesen.

\bibverse{20} Denn er gedachte, wider die von Tyrus und Sidon zu
kriegen. Sie aber kamen einmütig zu ihm und überredeten des Königs
Kämmerer, Blastus, und baten um Frieden, darum dass ihre Lande sich
nähren mussten von des Königs Land. \bibverse{21} Aber auf einen
bestimmten Tag tat Herodes das königliche Kleid an, setzte sich auf den
Richtstuhl und tat eine Rede zu ihnen. \bibverse{22} Das Volk aber rief
zu: Das ist Gottes Stimme und nicht eines Menschen! \footnote{\textbf{12:22}
  Hes 28,2} \bibverse{23} Alsbald schlug ihn der Engel des Herrn, darum
dass er die Ehre nicht Gott gab; und ward gefressen von den Würmern und
gab den Geist auf. \footnote{\textbf{12:23} Dan 5,20}

\bibverse{24} Das Wort Gottes aber wuchs und mehrte sich. \footnote{\textbf{12:24}
  Apg 6,7; Jes 55,11} \bibverse{25} Barnabas aber und Saulus kehrten
wieder von Jerusalem, nachdem sie überantwortet hatten die Handreichung,
und nahmen mit sich Johannes, mit dem Zunamen Markus. \footnote{\textbf{12:25}
  Apg 11,29-30; Apg 13,5}

\hypertarget{section-4}{%
\section{13}\label{section-4}}

\bibverse{1} Es waren aber zu Antiochien in der Gemeinde Propheten und
Lehrer, nämlich Barnabas und Simon, genannt Niger, und Luzius von Kyrene
und Manahen, der mit Herodes dem Vierfürsten erzogen war, und Saulus.
\footnote{\textbf{13:1} Apg 11,27; 1Kor 12,28} \bibverse{2} Da sie aber
dem Herrn dienten und fasteten, sprach der heilige Geist: Sondert mir
aus Barnabas und Saulus zu dem Werk, dazu ich sie berufen habe.
\footnote{\textbf{13:2} Apg 9,15}

\bibverse{3} Da fasteten sie und beteten und legten die Hände auf sie
und ließen sie gehen. \footnote{\textbf{13:3} Apg 6,6} \bibverse{4}
Diese nun, wie sie ausgesandt waren vom heiligen Geist, kamen sie gen
Seleucia, und von da schifften sie gen Zypern. \bibverse{5} Und da sie
in die Stadt Salamis kamen, verkündigten sie das Wort Gottes in der
Juden Schulen; sie hatten aber auch Johannes zum Diener. \bibverse{6}
Und da sie die Insel durchzogen bis zu der Stadt Paphos, fanden sie
einen Zauberer und falschen Propheten, einen Juden, der hieß Bar-Jesus;
\bibverse{7} der war bei Sergius Paulus, dem Landvogt, einem
verständigen Mann. Der rief zu sich Barnabas und Saulus und begehrte,
das Wort Gottes zu hören. \bibverse{8} Da widerstand ihnen der Zauberer
Elymas (denn also wird sein Name gedeutet) und trachtete, dass er den
Landvogt vom Glauben wendete. \bibverse{9} Saulus aber, der auch Paulus
heißt, voll heiligen Geistes, sah ihn an \bibverse{10} und sprach: O du
Kind des Teufels, voll aller List und aller Schalkheit, und Feind aller
Gerechtigkeit, du hörst nicht auf, abzuwenden die rechten Wege des
Herrn; \bibverse{11} und nun siehe, die Hand des Herrn kommt über dich,
und du sollst blind sein und die Sonne eine Zeitlang nicht sehen! Und
von Stund an fiel auf ihn Dunkelheit und Finsternis, und er ging umher
und suchte Handleiter.

\bibverse{12} Als der Landvogt die Geschichte sah, glaubte er und
verwunderte sich der Lehre des Herrn.

\bibverse{13} Da aber Paulus und die um ihn waren, von Paphos schifften,
kamen sie gen Perge im Lande Pamphylien. Johannes aber wich von ihnen
und zog wieder gen Jerusalem. \footnote{\textbf{13:13} Apg 15,38}
\bibverse{14} Sie aber zogen weiter von Perge und kamen gen Antiochien
im Lande Pisidien und gingen in die Schule am Sabbattage und setzten
sich. \bibverse{15} Nach der Lektion aber des Gesetzes und der Propheten
sandten die Obersten der Schule zu ihnen und ließen ihnen sagen: Liebe
Brüder, wollt ihr etwas reden und das Volk ermahnen, so saget an.

\bibverse{16} Da stand Paulus auf und winkte mit der Hand und sprach:
Ihr Männer von Israel und die ihr Gott fürchtet, höret zu! \bibverse{17}
Der Gott dieses Volkes hat erwählt unsere Väter und erhöht das Volk, da
sie Fremdlinge waren im Lande Ägypten, und mit einem hohen Arm führte er
sie aus demselben. \footnote{\textbf{13:17} 2Mo 12,37; 2Mo 12,41; 2Mo
  14,8} \bibverse{18} Und vierzig Jahre lang duldete er ihre Weise in
der Wüste, \footnote{\textbf{13:18} 2Mo 16,35} \bibverse{19} und
vertilgte sieben Völker in dem Lande Kanaan und teilte unter sie nach
dem Los deren Lande. \footnote{\textbf{13:19} 5Mo 7,1; Jos 14,2}
\bibverse{20} Darnach gab er ihnen Richter vierhundertfünfzig Jahre lang
bis auf den Propheten Samuel. \footnote{\textbf{13:20} Ri 2,16; 1Sam
  3,20} \bibverse{21} Und von da an baten sie um einen König; und Gott
gab ihnen Saul, den Sohn des Kis, einen Mann aus dem Geschlechte
Benjamin, vierzig Jahre lang. \footnote{\textbf{13:21} 1Sam 8,5; 1Sam
  10,21; 1Sam 10,24} \bibverse{22} Und da er denselben wegtat, richtete
er auf über sie David zum König, von welchem er zeugte: „Ich habe
gefunden David, den Sohn Jesses, einen Mann nach meinem Herzen, der soll
tun allen meinen Willen.`` \bibverse{23} Aus dieses Samen hat Gott, wie
er verheißen hat, kommen lassen Jesum, dem Volk Israel zum Heiland;
\bibverse{24} wie denn Johannes zuvor dem Volk Israel predigte die Taufe
der Buße, ehe denn er anfing. \footnote{\textbf{13:24} Lk 3,3}
\bibverse{25} Da aber Johannes seinen Lauf erfüllte, sprach er: „Ich bin
nicht der, für den ihr mich haltet; aber siehe, er kommt nach mir, des
ich nicht wert bin, dass ich ihm die Schuhe seiner Füße auflöse.``
\footnote{\textbf{13:25} Joh 1,20; Joh 1,27; Lk 3,16; Mk 1,7}

\bibverse{26} Ihr Männer, liebe Brüder, ihr Kinder des Geschlechts
Abraham und die unter euch Gott fürchten, euch ist das Wort dieses Heils
gesandt. \bibverse{27} Denn die zu Jerusalem wohnen und ihre Obersten,
dieweil sie diesen nicht kannten noch die Stimme der Propheten (welche
an allen Sabbaten gelesen werden), haben sie dieselben mit ihrem
Urteilen erfüllt. \footnote{\textbf{13:27} Joh 16,3} \bibverse{28} Und
wiewohl sie keine Ursache des Todes an ihm fanden, baten sie doch
Pilatus, ihn zu töten. \footnote{\textbf{13:28} Mt 27,22-23}
\bibverse{29} Und als sie alles vollendet hatten, was von ihm
geschrieben ist, nahmen sie ihn von dem Holz und legten ihn in ein Grab.
\footnote{\textbf{13:29} Mt 27,59-60} \bibverse{30} Aber Gott hat ihn
auferweckt von den Toten; \footnote{\textbf{13:30} Apg 3,15}
\bibverse{31} und er ist erschienen viele Tage denen, die mit ihm hinauf
von Galiläa gen Jerusalem gegangen waren, welche sind seine Zeugen an
das Volk. \footnote{\textbf{13:31} Apg 1,3} \bibverse{32} Und wir
verkündigen euch die Verheißung, die zu unseren Vätern geschehen ist,
\bibverse{33} dass sie Gott uns, ihren Kindern, erfüllt hat in dem, dass
er Jesum auferweckte; wie denn im zweiten Psalm geschrieben steht: „Du
bist mein Sohn, heute habe ich dich gezeuget.``

\bibverse{34} dass er ihn aber hat von den Toten auferweckt, dass er
hinfort nicht soll verwesen, spricht er also: „Ich will euch die Gnade,
David verheißen, treulich halten.`` \bibverse{35} Darum spricht er auch
an einem anderen Ort: „Du wirst es nicht zugeben, dass dein Heiliger die
Verwesung sehe.`` \bibverse{36} Denn David, da er zu seiner Zeit gedient
hatte dem Willen Gottes, ist entschlafen und zu seinen Vätern getan und
hat die Verwesung gesehen. \bibverse{37} Den aber Gott auferweckt hat,
der hat die Verwesung nicht gesehen. \bibverse{38} So sei es nun euch
kund, liebe Brüder, dass euch verkündigt wird Vergebung der Sünden durch
diesen und von dem allem, wovon ihr nicht konntet im Gesetz Moses
gerecht werden. \bibverse{39} Wer aber an diesen glaubt, der ist
gerecht. \bibverse{40} Sehet nun zu, dass nicht über euch komme, was in
den Propheten gesagt ist: \bibverse{41} „Sehet, ihr Verächter, und
verwundert euch und werdet zunichte! denn ich tue ein Werk zu euren
Zeiten, welches ihr nicht glauben werdet, so es euch jemand erzählen
wird.``

\bibverse{42} Da aber die Juden aus der Schule gingen, baten die Heiden,
dass sie am nächsten Sabbat ihnen die Worte sagten. \bibverse{43} Und
als die Gemeinde der Schule voneinander ging, folgten Paulus und
Barnabas nach viele Juden und gottesfürchtige Judengenossen. Sie aber
sagten ihnen und ermahnten sie, dass sie bleiben sollten in der Gnade
Gottes.

\bibverse{44} Am folgenden Sabbat aber kam zusammen fast die ganze
Stadt, das Wort Gottes zu hören. \bibverse{45} Da aber die Juden das
Volk sahen, wurden sie voll Neides und widersprachen dem, was von Paulus
gesagt ward, widersprachen und lästerten.

\bibverse{46} Paulus aber und Barnabas sprachen frei und öffentlich:
Euch musste zuerst das Wort Gottes gesagt werden; nun ihr es aber von
euch stoßet und achtet euch selbst nicht wert des ewigen Lebens, siehe,
so wenden wir uns zu den Heiden. \footnote{\textbf{13:46} Apg 3,25-26;
  Mt 10,5-6} \bibverse{47} Denn also hat uns der Herr geboten: „Ich habe
dich den Heiden zum Licht gesetzt, dass du das Heil seist bis an das
Ende der Erde.``

\bibverse{48} Da es aber die Heiden hörten, wurden sie froh und priesen
das Wort des Herrn und wurden gläubig, wie viele ihrer zum ewigen Leben
verordnet waren. \bibverse{49} Und das Wort des Herrn ward ausgebreitet
durch die ganze Gegend. \bibverse{50} Aber die Juden bewegten die
andächtigen und ehrbaren Weiber und der Stadt Oberste und erweckten eine
Verfolgung über Paulus und Barnabas und stießen sie zu ihren Grenzen
hinaus. \bibverse{51} Sie aber schüttelten den Staub von ihren Füßen
über sie und kamen gen Ikonion. \footnote{\textbf{13:51} Apg 18,6; Mt
  10,14} \bibverse{52} Die Jünger aber wurden voll Freude und heiligen
Geistes. \# 14 \bibverse{1} Es geschah aber zu Ikonion, dass sie
zusammenkamen und predigten in der Juden Schule, also dass eine große
Menge der Juden und Griechen gläubig ward. \bibverse{2} Die ungläubigen
Juden aber erweckten und entrüsteten die Seelen der Heiden wider die
Brüder. \bibverse{3} So hatten sie nun ihr Wesen daselbst eine lange
Zeit und lehrten frei im Herrn, welcher bezeugte das Wort seiner Gnade
und ließ Zeichen und Wunder geschehen durch ihre Hände. \bibverse{4} Die
Menge aber der Stadt spaltete sich; etliche hielten's mit den Juden und
etliche mit den Aposteln. \bibverse{5} Da sich aber ein Sturm erhob der
Heiden und der Juden und ihrer Obersten, sie zu schmähen und zu
steinigen, \footnote{\textbf{14:5} 2Tim 3,11} \bibverse{6} wurden sie
des inne und entflohen in die Städte des Landes Lykaonien, gen Lystra
und Derbe, und in die Gegend umher \bibverse{7} und predigten daselbst
das Evangelium.

\bibverse{8} Und es war ein Mann zu Lystra, der musste sitzen; denn er
hatte schwache Füße und war lahm von Mutterleibe, der noch nie gewandelt
hatte. \bibverse{9} Der hörte Paulus reden. Und als dieser ihn ansah und
merkte, dass er glaubte, ihm möchte geholfen werden, \bibverse{10}
sprach er mit lauter Stimme: Stehe aufrecht auf deine Füße! Und er
sprang auf und wandelte. \bibverse{11} Da aber das Volk sah, was Paulus
getan hatte, hoben sie ihre Stimme auf und sprachen auf lykaonisch: Die
Götter sind den Menschen gleich geworden und zu uns herniedergekommen.
\footnote{\textbf{14:11} Apg 28,6} \bibverse{12} Und nannten Barnabas
Jupiter und Paulus Merkurius, dieweil er das Wort führte. \bibverse{13}
Der Priester aber Jupiters aus dem Tempel vor ihrer Stadt brachte Ochsen
und Kränze vor das Tor und wollte opfern samt dem Volk.

\bibverse{14} Da das die Apostel Barnabas und Paulus hörten, zerrissen
sie ihre Kleider und sprangen unter das Volk, schrien \bibverse{15} und
sprachen: Ihr Männer, was macht ihr da? Wir sind auch sterbliche
Menschen gleichwie ihr und predigen euch das Evangelium, dass ihr euch
bekehren sollt von diesen falschen zu dem lebendigen Gott, welcher
gemacht hat Himmel und Erde und das Meer und alles, was darinnen ist;
\bibverse{16} der in den vergangenen Zeiten hat lassen alle Heiden
wandeln ihre eigenen Wege; \footnote{\textbf{14:16} Apg 17,30}
\bibverse{17} und doch hat er sich selbst nicht unbezeugt gelassen, hat
uns viel Gutes getan und vom Himmel Regen und fruchtbare Zeiten gegeben,
unsere Herzen erfüllt mit Speise und Freude.

\bibverse{18} Und da sie das sagten, stillten sie kaum das Volk, dass
sie ihnen nicht opferten. \bibverse{19} Es kamen aber dahin Juden von
Antiochien und Ikonion und überredeten das Volk und steinigten Paulus
und schleiften ihn zur Stadt hinaus, meinten, er wäre gestorben.

\bibverse{20} Da ihn aber die Jünger umringten, stand er auf und ging in
die Stadt. Und den anderen Tag ging er aus mit Barnabas gen Derbe;

\bibverse{21} und sie predigten der Stadt das Evangelium und unterwiesen
ihrer viele und zogen wieder gen Lystra und Ikonion und Antiochien,
\bibverse{22} stärkten die Seelen der Jünger und ermahnten sie, dass sie
im Glauben blieben, und dass wir durch viel Trübsale müssen in das Reich
Gottes gehen. \footnote{\textbf{14:22} Röm 5,3-5; 1Thes 3,3}
\bibverse{23} Und sie ordneten ihnen hin und her Älteste in den
Gemeinden, beteten und fasteten und befahlen sie dem Herrn, an den sie
gläubig geworden waren. \footnote{\textbf{14:23} Apg 6,6}

\bibverse{24} Und zogen durch Pisidien und kamen nach Pamphylien
\bibverse{25} und redeten das Wort zu Perge und zogen hinab gen
Attalien. \bibverse{26} Und von da schifften sie gen Antiochien, woher
sie verordnet waren durch die Gnade Gottes zu dem Werk, das sie hatten
ausgerichtet. \footnote{\textbf{14:26} Apg 13,1-2} \bibverse{27} Da sie
aber hinkamen, versammelten sie die Gemeinde und verkündigten, wieviel
Gott mit ihnen getan hatte und wie er den Heiden hätte die Tür des
Glaubens aufgetan. \footnote{\textbf{14:27} 1Kor 16,9} \bibverse{28} Sie
hatten aber ihr Wesen allda eine nicht kleine Zeit bei den Jüngern. \#
15 \bibverse{1} Und etliche kamen herab von Judäa und lehrten die
Brüder: Wo ihr euch nicht beschneiden lasset nach der Weise Moses, so
könnt ihr nicht selig werden. \footnote{\textbf{15:1} Gal 5,2}
\bibverse{2} Da sich nun ein Aufruhr erhob und Paulus und Barnabas einen
nicht geringen Streit mit ihnen hatten, ordneten sie, dass Paulus und
Barnabas und etliche andere aus ihnen hinaufzögen gen Jerusalem zu den
Aposteln und Ältesten um dieser Frage willen. \footnote{\textbf{15:2}
  Gal 2,1} \bibverse{3} Und sie wurden von der Gemeinde geleitet und
zogen durch Phönizien und Samarien und erzählten die Bekehrung der
Heiden und machten große Freude allen Brüdern. \bibverse{4} Da sie aber
hinkamen gen Jerusalem, wurden sie empfangen von der Gemeinde und von
den Aposteln und von den Ältesten. Und sie verkündigten, wieviel Gott
mit ihnen getan hatte.

\bibverse{5} Da traten auf etliche von der Pharisäer Sekte, die gläubig
geworden waren, und sprachen: Man muss sie beschneiden und ihnen
gebieten, zu halten das Gesetz Moses.

\bibverse{6} Aber die Apostel und die Ältesten kamen zusammen, über
diese Rede sich zu beraten. \bibverse{7} Da man sich aber lange
gestritten hatte, stand Petrus auf und sprach zu ihnen: Ihr Männer,
liebe Brüder, ihr wisset, dass Gott lange vor dieser Zeit unter uns
erwählt hat, dass durch meinen Mund die Heiden das Wort des Evangeliums
hörten und glaubten. \footnote{\textbf{15:7} Apg 10,44; Apg 11,15}
\bibverse{8} Und Gott, der Herzenskündiger, zeugte über sie und gab
ihnen den heiligen Geist gleichwie auch uns \bibverse{9} und machte
keinen Unterschied zwischen uns und ihnen und reinigte ihre Herzen durch
den Glauben. \bibverse{10} Was versucht ihr denn nun Gott mit Auflegen
des Jochs auf der Jünger Hälse, welches weder unsere Väter noch wir
haben können tragen? \bibverse{11} Sondern wir glauben, durch die Gnade
des Herrn Jesu Christi selig zu werden, gleicherweise wie auch sie.
\footnote{\textbf{15:11} Gal 2,16; Eph 2,4-10}

\bibverse{12} Da schwieg die ganze Menge still und hörte zu Paulus und
Barnabas, die da erzählten, wie große Zeichen und Wunder Gott durch sie
getan hatte unter den Heiden. \bibverse{13} Darnach, als sie geschwiegen
hatten, antwortete Jakobus und sprach: Ihr Männer, liebe Brüder, höret
mir zu! \bibverse{14} Simon hat erzählt, wie aufs erste Gott heimgesucht
hat und angenommen ein Volk aus den Heiden zu seinem Namen.
\bibverse{15} Und damit stimmen der Propheten Reden, wie geschrieben
steht: \bibverse{16} „Darnach will ich wiederkommen und will wieder
bauen die Hütte Davids, die zerfallen ist, und ihre Lücken will ich
wieder bauen und will sie aufrichten, \bibverse{17} auf dass, was übrig
ist von Menschen, nach dem Herrn frage, dazu alle Heiden, über welche
mein Name genannt ist, spricht der Herr, der das alles tut.``

\bibverse{18} Gott sind alle seine Werke bewusst von der Welt her.
\bibverse{19} Darum urteile ich, dass man denen, die aus den Heiden zu
Gott sich bekehren, nicht Unruhe mache, \bibverse{20} sondern schreibe
ihnen, dass sie sich enthalten von Unsauberkeit der Abgötter und von
Hurerei und vom Erstickten und vom Blut. \footnote{\textbf{15:20} 1Mo
  9,4; 3Mo 17,10-14; 3Mo 19,4; 3Mo 19,29} \bibverse{21} Denn Mose hat
von langen Zeiten her in allen Städten, die ihn predigen, und wird alle
Sabbattage in den Schulen gelesen. \footnote{\textbf{15:21} Apg 13,15}

\bibverse{22} Und es deuchte gut die Apostel und Ältesten samt der
ganzen Gemeinde, aus ihnen Männer zu erwählen und zu senden gen
Antiochien mit Paulus und Barnabas, nämlich Judas, mit dem Zunamen
Barsabas, und Silas, welche Männer Lehrer waren unter den Brüdern.
\bibverse{23} Und sie gaben Schrift in ihre Hand, also: Wir, die Apostel
und Ältesten und Brüder, wünschen Heil den Brüdern aus den Heiden, die
zu Antiochien und Syrien und Zilizien sind.

\bibverse{24} Dieweil wir gehört haben, dass etliche von den Unseren
sind ausgegangen und haben euch mit Lehren irregemacht und eure Seelen
zerrüttet und sagen, ihr sollt euch beschneiden lassen und das Gesetz
halten, welchen wir nichts befohlen haben, \bibverse{25} hat es uns gut
gedeucht, einmütig versammelt, Männer zu erwählen und zu euch zu senden
mit unseren liebsten Barnabas und Paulus, \bibverse{26} welche Menschen
ihre Seele dargegeben haben für den Namen unseres Herrn Jesu Christi.
\bibverse{27} So haben wir gesandt Judas und Silas, welche auch mit
Worten dasselbe verkündigen werden. \bibverse{28} Denn es gefällt dem
heiligen Geiste und uns, euch keine Beschwerung mehr aufzulegen als nur
diese nötigen Stücke: \bibverse{29} dass ihr euch enthaltet vom
Götzenopfer und vom Blut und vom Erstickten und von Hurerei; so ihr euch
vor diesen bewahret, tut ihr recht. Gehabt euch wohl!

\bibverse{30} Da diese abgefertigt waren, kamen sie gen Antiochien und
versammelten die Menge und überantworteten den Brief. \bibverse{31} Da
sie den lasen, wurden sie des Trostes froh. \bibverse{32} Judas aber und
Silas, die auch Propheten waren, ermahnten die Brüder mit vielen Reden
und stärkten sie. \footnote{\textbf{15:32} Apg 11,27; Apg 13,1}
\bibverse{33} Und da sie verzogen hatten eine Zeitlang, wurden sie von
den Brüdern mit Frieden abgefertigt zu den Aposteln. \bibverse{34} Es
gefiel aber Silas, dass er dabliebe. \bibverse{35} Paulus aber und
Barnabas hatten ihr Wesen zu Antiochien, lehrten und predigten des Herrn
Wort samt vielen anderen.

\bibverse{36} Nach etlichen Tagen aber sprach Paulus zu Barnabas: Lass
uns wiederum ziehen und nach unseren Brüdern sehen durch alle Städte, in
welchen wir des Herrn Wort verkündigt haben, wie sie sich halten.
\bibverse{37} Barnabas aber gab Rat, dass sie mit sich nähmen Johannes,
mit dem Zunamen Markus. \bibverse{38} Paulus aber achtete es billig,
dass sie nicht mit sich nähmen einen solchen, der von ihnen gewichen war
in Pamphylien und war nicht mit ihnen gezogen zu dem Werk. \footnote{\textbf{15:38}
  Apg 13,13} \bibverse{39} Und sie kamen scharf aneinander, also dass
sie voneinander zogen und Barnabas zu sich nahm Markus und schiffte nach
Zypern. \bibverse{40} Paulus aber wählte Silas und zog hin, der Gnade
Gottes befohlen von den Brüdern. \bibverse{41} Er zog aber durch Syrien
und Zilizien und stärkte die Gemeinden. \# 16 \bibverse{1} Er kam aber
gen Derbe und Lystra; und siehe, ein Jünger war daselbst mit Namen
Timotheus, eines jüdischen Weibes Sohn, die war gläubig, aber eines
griechischen Vaters. \bibverse{2} Der hatte ein gut Gerücht bei den
Brüdern unter den Lystranern und zu Ikonion. \bibverse{3} Diesen wollte
Paulus mit sich ziehen lassen und nahm und beschnitt ihn um der Juden
willen, die an den Orten waren; denn sie wussten alle, dass sein Vater
war ein Grieche gewesen. \footnote{\textbf{16:3} Gal 2,3} \bibverse{4}
Wie sie aber durch die Städte zogen, überantworteten sie ihnen, zu
halten den Spruch, welcher von den Aposteln und den Ältesten zu
Jerusalem beschlossen war. \footnote{\textbf{16:4} Apg 15,23-29}
\bibverse{5} Da wurden die Gemeinden im Glauben befestigt und nahmen zu
an der Zahl täglich.

\bibverse{6} Da sie aber durch Phrygien und das Land Galatien zogen,
ward ihnen gewehrt von dem heiligen Geiste, zu reden das Wort in Asien.
\footnote{\textbf{16:6} Apg 18,23} \bibverse{7} Als sie aber kamen an
Mysien, versuchten sie, durch Bithynien zu reisen; und der Geist ließ es
ihnen nicht zu. \bibverse{8} Sie zogen aber an Mysien vorüber und kamen
hinab gen Troas. \bibverse{9} Und Paulus erschien ein Gesicht bei der
Nacht; das war ein Mann aus Mazedonien, der stand und bat ihn und
sprach: Komm herüber nach Mazedonien und hilf uns! \bibverse{10} Als er
aber das Gesicht gesehen hatte, da trachteten wir alsobald, zu reisen
nach Mazedonien, gewiss, dass uns der Herr dahin berufen hätte, ihnen
das Evangelium zu predigen. \bibverse{11} Da fuhren wir aus von Troas;
und geradewegs kamen wir gen Samothrazien, des anderen Tages gen
Neapolis \bibverse{12} und von da gen Philippi, welches ist die
Hauptstadt des Landes Mazedonien und eine Freistadt. Wir hatten aber in
dieser Stadt unser Wesen etliche Tage.

\bibverse{13} Am Tage des Sabbats gingen wir hinaus vor die Stadt an das
Wasser, da man pflegte zu beten, und setzten uns und redeten zu den
Weibern, die da zusammenkamen. \bibverse{14} Und ein gottesfürchtiges
Weib mit Namen Lydia, eine Purpurkrämerin aus der Stadt der Thyatirer,
hörte zu; dieser tat der Herr das Herz auf, dass sie darauf achthatte,
was von Paulus geredet ward. \bibverse{15} Als sie aber und ihr Haus
getauft ward, ermahnte sie uns und sprach: So ihr mich achtet, dass ich
gläubig bin an den Herrn, so kommt in mein Haus und bleibt allda. Und
sie nötigte uns.

\bibverse{16} Es geschah aber, da wir zu dem Gebet gingen, dass eine
Magd uns begegnete, die hatte einen Wahrsagergeist und trug ihren Herren
viel Gewinst zu mit Wahrsagen. \bibverse{17} Die folgte allenthalben
Paulus und uns nach, schrie und sprach: Diese Menschen sind die Knechte
Gottes des Allerhöchsten, die euch den Weg der Seligkeit verkündigen.
\bibverse{18} Solches tat sie manchen Tag. Paulus aber tat das wehe, und
er wandte sich um und sprach zu dem Geiste: Ich gebiete dir in dem Namen
Jesu Christi, dass du von ihr ausfahrest. Und er fuhr aus zu derselben
Stunde. \footnote{\textbf{16:18} Mk 16,17}

\bibverse{19} Da aber die Herren sahen, dass die Hoffnung ihres
Gewinstes war ausgefahren, nahmen sie Paulus und Silas, zogen sie auf
den Markt vor die Obersten \bibverse{20} und führten sie zu den
Hauptleuten und sprachen: Diese Menschen machen unsere Stadt irre; sie
sind Juden \bibverse{21} und verkündigen eine Weise, welche uns nicht
ziemt anzunehmen noch zu tun, weil wir Römer sind.

\bibverse{22} Und das Volk ward erregt wider sie; und die Hauptleute
ließen ihnen die Kleider abreißen und hießen sie stäupen. \footnote{\textbf{16:22}
  2Kor 11,25; Phil 1,30; 1Thes 2,2} \bibverse{23} Und da sie sie wohl
gestäupt hatten, warfen sie sie ins Gefängnis und geboten dem
Kerkermeister, dass er sie wohl verwahrte. \bibverse{24} Der, da er
solches Gebot empfangen hatte, warf sie in das innerste Gefängnis und
legte ihre Füße in den Stock.

\bibverse{25} Um die Mitternacht aber beteten Paulus und Silas und
lobten Gott. Und es hörten sie die Gefangenen. \bibverse{26} Schnell
aber ward ein großes Erdbeben, also dass sich bewegten die Grundfesten
des Gefängnisses. Und von Stund an wurden alle Türen aufgetan und aller
Bande los. \bibverse{27} Als aber der Kerkermeister aus dem Schlafe fuhr
und sah die Türen des Gefängnisses aufgetan, zog er das Schwert aus und
wollte sich selbst erwürgen; denn er meinte die Gefangenen wären
entflohen. \bibverse{28} Paulus rief aber laut und sprach: Tu dir nichts
Übles; denn wir sind alle hier!

\bibverse{29} Er forderte aber ein Licht und sprang hinein und ward
zitternd und fiel Paulus und Silas zu den Füßen \bibverse{30} und führte
sie heraus und sprach: Liebe Herren, was soll ich tun, dass ich selig
werde?

\bibverse{31} Sie sprachen: Glaube an den Herrn Jesus Christus, so wirst
du und dein Haus selig! \bibverse{32} Und sagten ihm das Wort des Herrn
und allen, die in seinem Hause waren.

\bibverse{33} Und er nahm sie zu sich in derselben Stunde der Nacht und
wusch ihnen die Striemen ab; und er ließ sich taufen und alle die Seinen
alsobald. \bibverse{34} Und führte sie in sein Haus und setzte ihnen
einen Tisch und freute sich mit seinem ganzen Hause, dass er an Gott
gläubig geworden war.

\bibverse{35} Und da es Tag ward, sandten die Hauptleute Stadtdiener und
sprachen: Lass die Menschen gehen!

\bibverse{36} Und der Kerkermeister verkündigte diese Rede Paulus: Die
Hauptleute haben hergesandt, dass ihr los sein sollt. Nun ziehet aus und
gehet hin mit Frieden!

\bibverse{37} Paulus aber sprach zu ihnen: Sie haben uns ohne Recht und
Urteil öffentlich gestäupt, die wir doch Römer sind, und uns in das
Gefängnis geworfen, und sollten uns nun heimlich ausstoßen? Nicht also;
sondern lasset sie selbst kommen und uns hinausführen! \footnote{\textbf{16:37}
  Apg 22,25}

\bibverse{38} Die Stadtdiener verkündigten diese Worte den Hauptleuten.
Und sie fürchteten sich, da sie hörten, dass sie Römer wären,
\bibverse{39} und kamen und redeten ihnen zu, führten sie heraus und
baten sie, dass sie auszögen aus der Stadt. \bibverse{40} Da gingen sie
aus dem Gefängnis und gingen zu der Lydia. Und da sie die Brüder gesehen
hatten und getröstet, zogen sie aus. \# 17 \bibverse{1} Nachdem sie aber
durch Amphipolis und Apollonia gereist waren, kamen sie gen
Thessalonich; da war eine Judenschule. \bibverse{2} Wie nun Paulus
gewohnt war, ging er zu ihnen hinein und redete mit ihnen an drei
Sabbaten aus der Schrift, \bibverse{3} tat sie ihnen auf und legte es
ihnen vor, dass Christus musste leiden und auferstehen von den Toten und
dass dieser Jesus, den ich (sprach er) euch verkündige, ist der
Christus. \footnote{\textbf{17:3} Lk 24,26-27; Lk 24,45-46}

\bibverse{4} Und etliche unter ihnen fielen ihm zu und gesellten sich zu
Paulus und Silas, auch der gottesfürchtigen Griechen eine große Menge,
dazu der vornehmsten Weiber nicht wenige. \footnote{\textbf{17:4} 1Thes
  1,1; 2Thes 1,1} \bibverse{5} Aber die halsstarrigen Juden neideten und
nahmen zu sich etliche boshafte Männer Pöbelvolks, machten eine Rotte
und richteten einen Aufruhr in der Stadt an und traten vor das Haus
Jasons und suchten sie zu führen vor das Volk. \bibverse{6} Da sie aber
sie nicht fanden, schleiften sie den Jason und etliche Brüder vor die
Obersten der Stadt und schrien: Diese, die den ganzen Weltkreis erregen,
sind auch hergekommen; \footnote{\textbf{17:6} Apg 16,20} \bibverse{7}
die herbergt Jason. Und diese alle handeln gegen des Kaisers Gebote,
sagen, ein anderer sei der König, nämlich Jesus. \footnote{\textbf{17:7}
  Lk 23,2} \bibverse{8} Sie bewegten aber das Volk und die Obersten der
Stadt, die solches hörten. \bibverse{9} Und da ihnen Genüge von Jason
und anderen geleistet war, ließen sie sie los.

\bibverse{10} Die Brüder aber fertigten alsobald ab bei der Nacht Paulus
und Silas gen Beröa. Da sie dahin kamen, gingen sie in die Judenschule.

\bibverse{11} Diese aber waren edler denn die zu Thessalonich; die
nahmen das Wort auf ganz willig und forschten täglich in der Schrift, ob
sich's also verhielte. \footnote{\textbf{17:11} Joh 5,39} \bibverse{12}
So glaubten nun viele aus ihnen, auch der griechischen ehrbaren Weiber
und Männer nicht wenige. \bibverse{13} Als aber die Juden von
Thessalonich erfuhren, dass auch zu Beröa das Wort Gottes von Paulus
verkündigt würde, kamen sie und bewegten auch allda das Volk.
\bibverse{14} Aber da fertigten die Brüder Paulus alsobald ab, dass er
ginge bis an das Meer; Silas aber und Timotheus blieben da.
\bibverse{15} Die aber Paulus geleiteten, führten ihn bis gen Athen. Und
nachdem sie Befehl empfangen an den Silas und Timotheus, dass sie aufs
schnellste zu ihm kämen, zogen sie hin.

\bibverse{16} Da aber Paulus ihrer zu Athen wartete, ergrimmte sein
Geist in ihm, da er sah die Stadt so gar abgöttisch. \bibverse{17} Und
er redete zu den Juden und Gottesfürchtigen in der Schule, auch auf dem
Markte alle Tage zu denen, die sich herzufanden. \bibverse{18} Etliche
aber der Epikurer und Stoiker Philosophen stritten mit ihm. Und etliche
sprachen: Was will dieser Lotterbube sagen? Etliche aber: Es sieht, als
wolle er neue Götter verkündigen. (Das machte, er hatte das Evangelium
von Jesu und von der Auferstehung ihnen verkündigt.) \footnote{\textbf{17:18}
  1Kor 4,12}

\bibverse{19} Sie nahmen ihn aber und führten ihn auf den Gerichtsplatz
und sprachen: Können wir auch erfahren, was das für eine neue Lehre sei,
die du lehrst?

\bibverse{20} Denn du bringst etwas Neues vor unsere Ohren; so wollten
wir gern wissen, was das sei. \bibverse{21} (Die Athener aber alle, auch
die Ausländer und Gäste, waren gerichtet auf nichts anderes, denn etwas
Neues zu sagen oder zu hören.)

\bibverse{22} Paulus aber stand mitten auf dem Gerichtsplatz und sprach:
Ihr Männer von Athen, ich sehe, dass ihr in allen Stücken gar sehr die
Götter fürchtet. \bibverse{23} Ich bin herdurchgegangen und habe gesehen
eure Gottesdienste und fand einen Altar, darauf war geschrieben: Dem
unbekannten Gott. Nun verkündige ich euch denselben, dem ihr unwissend
Gottesdienst tut. \bibverse{24} Gott, der die Welt gemacht hat und
alles, was darinnen ist, er, der ein Herr ist Himmels und der Erde,
wohnt nicht in Tempeln mit Händen gemacht; \bibverse{25} sein wird auch
nicht von Menschenhänden gepflegt, als der jemandes bedürfe, so er
selber jedermann Leben und Odem allenthalben gibt. \footnote{\textbf{17:25}
  Ps 50,9-12} \bibverse{26} Und er hat gemacht, dass von einem Blut
aller Menschen Geschlechter auf dem ganzen Erdboden wohnen, und hat Ziel
gesetzt und vorgesehen, wie lange und wie weit sie wohnen sollen;
\footnote{\textbf{17:26} 5Mo 32,8} \bibverse{27} dass sie den Herrn
suchen sollten, ob sie doch ihn fühlen und finden möchten; und fürwahr,
er ist nicht ferne von einem jeglichen unter uns. \footnote{\textbf{17:27}
  Jes 55,6} \bibverse{28} Denn in ihm leben, weben und sind wir; wie
auch etliche Poeten bei euch gesagt haben: „Wir sind seines
Geschlechts.`` \bibverse{29} So wir denn göttlichen Geschlechts sind,
sollen wir nicht meinen, die Gottheit sei gleich den goldenen, silbernen
und steinernen Bildern, durch menschliche Kunst und Gedanken gemacht.
\bibverse{30} Und zwar hat Gott die Zeit der Unwissenheit übersehen; nun
aber gebietet er allen Menschen an allen Enden, Buße zu tun, \footnote{\textbf{17:30}
  Apg 14,16; Lk 24,47} \bibverse{31} darum dass er einen Tag gesetzt
hat, an welchem er richten will den Kreis des Erdbodens mit
Gerechtigkeit durch einen Mann, in welchem er's beschlossen hat und
jedermann vorhält den Glauben, nachdem er ihn hat von den Toten
auferweckt. \footnote{\textbf{17:31} Apg 10,42; Mt 25,31-33}

\bibverse{32} Da sie hörten die Auferstehung der Toten, da hatten's
etliche ihren Spott; etliche aber sprachen: Wir wollen dich davon weiter
hören.

\bibverse{33} Also ging Paulus von ihnen. \bibverse{34} Etliche Männer
aber hingen ihm an und wurden gläubig, unter welchen war Dionysius,
einer aus dem Rat, und ein Weib mit Namen Damaris und andere mit ihnen.
\# 18 \bibverse{1} Darnach schied Paulus von Athen und kam gen Korinth
\bibverse{2} und fand einen Juden mit Namen Aquila, von Geburt aus
Pontus, welcher war neulich aus Italien gekommen samt seinem Weibe
Priscilla (darum dass der Kaiser Klaudius geboten hatte allen Juden, zu
weichen aus Rom). \footnote{\textbf{18:2} Röm 16,3} \bibverse{3} Zu
denen ging er ein; und dieweil er gleiches Handwerks war, blieb er bei
ihnen und arbeitete. (Sie waren aber des Handwerks Teppichmacher).
\footnote{\textbf{18:3} Apg 20,34; 1Kor 4,12} \bibverse{4} Und er lehrte
in der Schule an allen Sabbaten und beredete beide, Juden und Griechen.

\bibverse{5} Da aber Silas und Timotheus aus Mazedonien kamen, drang
Paulus der Geist, zu bezeugen den Juden Jesum, dass er der Christus sei.
\footnote{\textbf{18:5} Apg 17,14-15; 2Kor 1,19} \bibverse{6} Da sie
aber widerstrebten und lästerten, schüttelte er die Kleider aus und
sprach zu ihnen: Euer Blut sei über euer Haupt! Rein gehe ich von nun an
zu den Heiden. \footnote{\textbf{18:6} Apg 13,51; Apg 20,26}

\bibverse{7} Und machte sich von dannen und kam in ein Haus eines mit
Namen Just, der gottesfürchtig war; dessen Haus war zunächst an der
Schule. \bibverse{8} Krispus aber, der Oberste der Schule, glaubte an
den Herrn mit seinem ganzen Hause; und viele Korinther, die zuhörten,
wurden gläubig und ließen sich taufen. \footnote{\textbf{18:8} 1Kor
  1,14; 1Kor 1,19} \bibverse{9} Es sprach aber der Herr durch ein
Gesicht in der Nacht zu Paulus: Fürchte dich nicht, sondern rede, und
schweige nicht! \footnote{\textbf{18:9} 1Kor 2,3} \bibverse{10} denn ich
bin mit dir, und niemand soll sich unterstehen, dir zu schaden; denn ich
habe ein großes Volk in dieser Stadt. \footnote{\textbf{18:10} Jer 1,8;
  Joh 10,16}

\bibverse{11} Er saß aber daselbst ein Jahr und sechs Monate und lehrte
sie das Wort Gottes. \bibverse{12} Da aber Gallion Landvogt war in
Achaja, empörten sich die Juden einmütig wider Paulus und führten ihn
vor den Richtstuhl \bibverse{13} und sprachen: Dieser überredet die
Leute, Gott zu dienen dem Gesetz zuwider.

\bibverse{14} Da aber Paulus wollte den Mund auftun, sprach Gallion zu
den Juden: Wenn es ein Frevel oder eine Schalkheit wäre, liebe Juden, so
hörte ich euch billig; \bibverse{15} weil es aber eine Frage ist von der
Lehre und von den Worten und von dem Gesetz unter euch, so sehet ihr
selber zu; ich gedenke darüber nicht Richter zu sein. \footnote{\textbf{18:15}
  Joh 18,31} \bibverse{16} Und trieb sie von dem Richtstuhl.

\bibverse{17} Da ergriffen alle Griechen Sosthenes, den Obersten der
Schule, und schlugen ihn vor dem Richtstuhl; und Gallion nahm sich's
nicht an.

\bibverse{18} Paulus aber blieb noch lange daselbst; darnach machte er
seinen Abschied mit den Brüdern und wollte nach Syrien schiffen und mit
ihm Priscilla und Aquila. Und er schor sein Haupt zu Kenchreä, denn er
hatte ein Gelübde. \bibverse{19} Und er kam gen Ephesus und ließ sie
daselbst; er aber ging in die Schule und redete mit den Juden.
\bibverse{20} Sie baten ihn aber, dass er längere Zeit bei ihnen bliebe.
Und er willigte nicht ein, \bibverse{21} sondern machte seinen Abschied
mit ihnen und sprach: Ich muss durchaus das künftige Fest zu Jerusalem
halten; will's Gott, so will ich wieder zu euch kommen. Und fuhr weg von
Ephesus \footnote{\textbf{18:21} Jak 4,15}

\bibverse{22} und kam gen Cäsarea und ging hinauf (nach Jerusalem) und
grüßte die Gemeinde und zog hinab gen Antiochien. \footnote{\textbf{18:22}
  Apg 21,15} \bibverse{23} Und verzog etliche Zeit und reiste weiter und
durchwandelte nacheinander das galatische Land und Phrygien und stärkte
alle Jünger. \bibverse{24} Es kam aber gen Ephesus ein Jude mit namen
Apollos, von Geburt aus Alexandrien, ein beredter Mann und mächtig in
der Schrift. \footnote{\textbf{18:24} 1Kor 3,5-6} \bibverse{25} Dieser
war unterwiesen im Weg des Herrn und redete mit brünstigem Geist und
lehrte mit Fleiß von dem Herrn, wusste aber allein von der Taufe des
Johannes. \footnote{\textbf{18:25} Apg 19,3} \bibverse{26} Dieser fing
an, frei zu predigen in der Schule. Da ihn aber Aquila und Priscilla
hörten, nahmen sie ihn zu sich und legten ihm den Weg Gottes noch
fleißiger aus.

\bibverse{27} Da er aber wollte nach Achaja reisen, schrieben die Brüder
und vermahnten die Jünger, dass sie ihn aufnähmen. Und als er
dahingekommen war, half er viel denen, die gläubig waren geworden durch
die Gnade. \bibverse{28} Denn er überwand die Juden beständig und erwies
öffentlich durch die Schrift, dass Jesus der Christus sei. \footnote{\textbf{18:28}
  Apg 9,22; Apg 17,3}

\hypertarget{section-5}{%
\section{19}\label{section-5}}

\bibverse{1} Es geschah aber, da Apollos zu Korinth war, dass Paulus
durchwandelte die oberen Länder und kam gen Ephesus und fand etliche
Jünger; \bibverse{2} zu denen sprach er: Habt ihr den heiligen Geist
empfangen, da ihr gläubig wurdet? Sie sprachen zu ihm: Wir haben auch
nie gehört, ob ein heiliger Geist sei.

\bibverse{3} Und er sprach zu ihnen: Worauf seid ihr denn getauft? Sie
sprachen: Auf die Taufe des Johannes.

\bibverse{4} Paulus aber sprach: Johannes hat getauft mit der Taufe der
Buße und sagte dem Volk, dass sie glauben sollten an den, der nach ihm
kommen sollte, das ist an Jesum, dass der Christus sei. \footnote{\textbf{19:4}
  Mt 3,11}

\bibverse{5} Da sie das hörten, ließen sie sich taufen auf den Namen des
Herrn Jesu.

\bibverse{6} Und da Paulus die Hände auf sie legte, kam der heilige
Geist auf sie, und sie redeten mit Zungen und weissagten.

\bibverse{7} Und aller der Männer waren bei zwölf.

\bibverse{8} Er ging aber in die Schule und predigte frei drei Monate
lang, lehrte und beredete sie von dem Reich Gottes.

\bibverse{9} Da aber etliche verstockt waren und nicht glaubten und übel
redeten von dem Wege vor der Menge, wich er von ihnen und sonderte ab
die Jünger und redete täglich in der Schule eines, der hieß Tyrannus.
\bibverse{10} Und das geschah zwei Jahre lang, also dass alle, die in
Asien wohnten, das Wort des Herrn Jesu hörten, beide, Juden und
Griechen.

\bibverse{11} Und Gott wirkte nicht geringe Taten durch die Hände
Paulus, \footnote{\textbf{19:11} Apg 14,3; 2Kor 12,12} \bibverse{12}
also dass sie auch von seiner Haut die Schweißtüchlein und Binden über
die Kranken hielten und die Seuchen von ihnen wichen und die bösen
Geister von ihnen ausfuhren. \footnote{\textbf{19:12} Apg 5,15}
\bibverse{13} Es unterwanden sich aber etliche der umherziehenden Juden,
die da Beschwörer waren, den Namen des Herrn Jesu zu nennen über die da
böse Geister hatten, und sprachen: Wir beschwören euch bei dem Jesus,
den Paulus predigt. \footnote{\textbf{19:13} Lk 9,49} \bibverse{14} Es
waren ihrer aber sieben Söhne eines Juden Skevas, des Hohenpriesters,
die solches taten.

\bibverse{15} Aber der böse Geist antwortete und sprach: Jesum kenne ich
wohl, und von Paulus weiß ich wohl; wer seid ihr aber? \bibverse{16} Und
der Mensch, in dem der böse Geist war, sprang auf sie und ward ihrer
mächtig und warf sie unter sich, also dass sie nackt und verwundet aus
demselben Hause entflohen. \bibverse{17} Das aber ward kund allen, die
zu Ephesus wohnten, sowohl Juden als Griechen; und es fiel eine Furcht
über sie alle, und der Name des Herrn Jesus ward hoch gelobt.
\bibverse{18} Es kamen auch viele derer, die gläubig waren geworden, und
bekannten und verkündigten, was sie getrieben hatten. \bibverse{19}
Viele aber, die da vorwitzige Kunst getrieben hatten, brachten die
Bücher zusammen und verbrannten sie öffentlich und überrechneten, was
sie wert waren, und fanden des Geldes fünfzigtausend Groschen.
\bibverse{20} Also mächtig wuchs das Wort des Herrn und nahm überhand.

\bibverse{21} Da das ausgerichtet war, setzte sich Paulus vor im Geiste,
durch Mazedonien und Achaja zu ziehen und gen Jerusalem zu reisen, und
sprach: Nach dem, wenn ich daselbst gewesen bin, muss ich auch Rom
sehen. \footnote{\textbf{19:21} Apg 23,11}

\bibverse{22} Und sandte zwei, die ihm dienten, Timotheus und Erastus,
nach Mazedonien; er aber verzog eine Weile in Asien. \footnote{\textbf{19:22}
  2Tim 4,20} \bibverse{23} Es erhob sich aber um diese Zeit eine nicht
kleine Bewegung über diesem Wege. \footnote{\textbf{19:23} 2Kor 1,8-9}
\bibverse{24} Denn einer mit Namen Demetrius, ein Goldschmied, der
machte silberne Tempel der Diana und wandte denen vom Handwerk nicht
geringen Gewinst zu. \bibverse{25} Dieselben und die Beiarbeiter des
Handwerks versammelte er und sprach: Liebe Männer, ihr wisset, dass wir
großen Gewinn von diesem Gewerbe haben; \bibverse{26} und ihr sehet und
höret, dass nicht allein zu Ephesus, sondern auch fast in ganz Asien
dieser Paulus viel Volks abfällig macht, überredet und spricht: Es sind
nicht Götter, welche von Händen gemacht sind. \bibverse{27} Aber es will
nicht allein unserem Handel dahin geraten, dass er nichts gelte, sondern
auch der Tempel der großen Göttin Diana wird für nichts geachtet werden,
und wird dazu ihre Majestät untergehen, welcher doch ganz Asien und der
Weltkreis Gottesdienst erzeigt.

\bibverse{28} Als sie das hörten, wurden sie voll Zorns, schrien und
sprachen: Groß ist die Diana der Epheser! \bibverse{29} Und die ganze
Stadt ward voll Getümmels; sie stürmten aber einmütig zu dem Schauplatz
und ergriffen Gajus und Aristarchus aus Mazedonien, des Paulus
Gefährten. \bibverse{30} Da aber Paulus wollte unter das Volk gehen,
ließen's ihm die Jünger nicht zu. \bibverse{31} Auch etliche der
Obersten in Asien, die des Paulus gute Freunde waren, sandten zu ihm und
ermahnten ihn, dass er sich nicht begäbe auf den Schauplatz.
\bibverse{32} Etliche schrien so, etliche ein anderes, und die Gemeinde
war irre, und die meisten wussten nicht, warum sie zusammengekommen
waren. \bibverse{33} Etliche aber vom Volk zogen Alexander hervor, da
ihn die Juden hervorstießen. Alexander aber winkte mit der Hand und
wollte sich vor dem Volk verantworten. \bibverse{34} Da sie aber
innewurden, dass er ein Jude war, erhob sich eine Stimme von allen, und
schrien bei zwei Stunden: Groß ist die Diana der Epheser!

\bibverse{35} Da aber der Kanzler das Volk gestillt hatte, sprach er:
Ihr Männer von Ephesus, welcher Mensch ist, der nicht wisse, dass die
Stadt Ephesus sei eine Pflegerin der großen Göttin Diana und des
himmlischen Bildes? \bibverse{36} Weil nun das unwidersprechlich ist, so
sollt ihr ja stille sein und nichts Unbedächtiges handeln. \bibverse{37}
Ihr habt diese Menschen hergeführt, die weder Tempelräuber noch Lästerer
eurer Göttin sind. \bibverse{38} Hat aber Demetrius und die mit ihm sind
vom Handwerk, an jemand einen Anspruch, so hält man Gericht und sind
Landvögte da; lasset sie sich untereinander verklagen. \bibverse{39}
Wollt ihr aber etwas anderes handeln, so mag man es ausrichten in einer
ordentlichen Gemeinde. \bibverse{40} Denn wir stehen in der Gefahr, dass
wir um diese heutige Empörung verklagt möchten werden, da doch keine
Sache vorhanden ist, womit wir uns solches Aufruhrs entschuldigen
könnten. Und da er solches gesagt, ließ er die Gemeinde gehen. \# 20
\bibverse{1} Da nun die Empörung aufgehört, rief Paulus die Jünger zu
sich und segnete sie und ging aus, zu reisen nach Mazedonien.
\footnote{\textbf{20:1} 2Kor 2,13} \bibverse{2} Und da er diese Länder
durchzogen und sie ermahnt hatte mit vielen Worten, kam er nach
Griechenland und verzog allda drei Monate. \bibverse{3} Da aber ihm die
Juden nachstellten, als er nach Syrien wollte fahren, beschloss er
wieder umzuwenden durch Mazedonien. \bibverse{4} Es zogen aber mit ihm
bis nach Asien Sopater von Beröa, von Thessalonich aber Aristarchus und
Sekundus, und Gajus von Derbe und Timotheus, aus Asien aber Tychikus und
Trophimus. \bibverse{5} Diese gingen voran und harrten unser zu Troas.
\bibverse{6} Wir aber schifften nach den Ostertagen von Philippi an bis
an den fünften Tag und kamen zu ihnen gen Troas und hatten da unser
Wesen sieben Tage.

\bibverse{7} Am ersten Tage der Woche aber, da die Jünger zusammenkamen,
das Brot zu brechen, predigte ihnen Paulus, und wollte des anderen Tages
weiterreisen und zog die Rede hin bis zu Mitternacht. \footnote{\textbf{20:7}
  Apg 2,42; Apg 2,46; Mt 28,1} \bibverse{8} Und es waren viel Lampen auf
dem Söller, da sie versammelt waren. \bibverse{9} Es saß aber ein
Jüngling mit namen Eutychus in einem Fenster und sank in einen tiefen
Schlaf, dieweil Paulus so lange redete, und ward vom Schlaf überwältigt
und fiel hinunter vom dritten Söller und ward tot aufgehoben.
\bibverse{10} Paulus aber ging hinab und legte sich auf ihn, umfing ihn
und sprach: Machet kein Getümmel; denn seine Seele ist in ihm.

\bibverse{11} Da ging er hinauf und brach das Brot und aß und redete
viel mit ihnen, bis der Tag anbrach; und also zog er aus. \bibverse{12}
Sie brachten aber den Knaben lebendig und wurden nicht wenig getröstet.

\bibverse{13} Wir aber zogen voran auf dem Schiff und fuhren gen Assos
und wollten daselbst Paulus zu uns nehmen; denn er hatte es also
befohlen, und er wollte zu Fuße gehen. \bibverse{14} Als er nun zu uns
traf zu Assos, nahmen wir ihn zu uns und kamen gen Mitylene.
\bibverse{15} Und von da schifften wir und kamen des anderen Tages hin
gegen Chios; und des folgenden Tages stießen wir an Samos und blieben in
Trogyllion; und des nächsten Tages kamen wir gen Milet. \bibverse{16}
Denn Paulus hatte beschlossen, an Ephesus vorüberzuschiffen, dass er
nicht müsste in Asien Zeit zubringen; denn er eilte, auf den Pfingsttag
zu Jerusalem zu sein, so es ihm möglich wäre. \footnote{\textbf{20:16}
  Apg 18,21}

\bibverse{17} Aber von Milet sandte er gen Ephesus und ließ fordern die
Ältesten von der Gemeinde. \bibverse{18} Als aber die zu ihm kamen,
sprach er zu ihnen: Ihr wisset von dem ersten Tage an, da ich bin nach
Asien gekommen, wie ich allezeit bin bei euch gewesen \bibverse{19} und
dem Herrn gedient habe mit aller Demut und mit viel Tränen und
Anfechtungen, die mir sind widerfahren von den Juden, die mir
nachstellten; \bibverse{20} wie ich nichts verhalten habe, das da
nützlich ist, dass ich's euch nicht verkündigt hätte und euch gelehrt
öffentlich und sonderlich; \bibverse{21} und habe bezeugt, beiden, den
Juden und Griechen, die Buße zu Gott und den Glauben an unseren Herrn
Jesus Christus. \bibverse{22} Und nun siehe, ich, im Geiste gebunden,
fahre hin gen Jerusalem, weiß nicht, was mir daselbst begegnen wird,
\footnote{\textbf{20:22} Apg 19,21} \bibverse{23} nur dass der heilige
Geist in allen Städten bezeugt und spricht, Bande und Trübsale warten
mein daselbst. \footnote{\textbf{20:23} Apg 9,16; Apg 21,4; Apg 21,11}
\bibverse{24} Aber ich achte der keines, ich halte mein Leben auch nicht
selbst teuer, auf dass ich vollende meinen Lauf mit Freuden und das Amt,
das ich empfangen habe von dem Herrn Jesus, zu bezeugen das Evangelium
von der Gnade Gottes. \footnote{\textbf{20:24} Apg 21,13; 2Tim 4,7}

\bibverse{25} Und nun siehe, ich weiß, dass ihr mein Angesicht nicht
mehr sehen werdet, alle die, bei welchen ich durchgekommen bin und
gepredigt habe das Reich Gottes. \bibverse{26} Darum bezeuge ich euch an
diesem heutigen Tage, dass ich rein bin von aller Blut; \bibverse{27}
denn ich habe euch nichts verhalten, dass ich nicht verkündigt hätte all
den Rat Gottes. \bibverse{28} So habt nun Acht auf euch selbst und auf
die ganze Herde, unter welche euch der heilige Geist gesetzt hat zu
Bischöfen, zu weiden die Gemeinde Gottes, welche er durch sein eigen
Blut erworben hat. \footnote{\textbf{20:28} 1Tim 4,16; 1Petr 5,2-4}
\bibverse{29} Denn das weiß ich, dass nach meinem Abschied werden unter
euch kommen gräuliche Wölfe, die die Herde nicht verschonen werden.
\footnote{\textbf{20:29} Mt 7,15} \bibverse{30} Auch aus euch selbst
werden aufstehen Männer, die da verkehrte Lehren reden, die Jünger an
sich zu ziehen. \footnote{\textbf{20:30} 1Jo 2,18; 1Jo 1,2-19}
\bibverse{31} Darum seid wach und denket daran, dass ich nicht
abgelassen habe drei Jahre, Tag und Nacht, einen jeglichen mit Tränen zu
vermahnen. \bibverse{32} Und nun, liebe Brüder, ich befehle euch Gott
und dem Wort seiner Gnade, der da mächtig ist, euch zu erbauen und zu
geben das Erbe unter allen, die geheiligt werden. \bibverse{33} Ich habe
euer keines Silber noch Gold noch Kleid begehrt. \bibverse{34} Denn ihr
wisset selber, dass mir diese Hände zu meiner Notdurft und derer, die
mit mir gewesen sind, gedient haben. \bibverse{35} Ich habe es euch
alles gezeigt, dass man also arbeiten müsse und die Schwachen aufnehmen
und gedenken an das Wort des Herrn Jesus, dass er gesagt hat: „Geben ist
seliger denn Nehmen!{}``

\bibverse{36} Und als er solches gesagt, kniete er nieder und betete mit
ihnen allen. \footnote{\textbf{20:36} Apg 21,5} \bibverse{37} Es war
aber viel Weinen unter ihnen allen, und sie fielen Paulus um den Hals
und küssten ihn, \bibverse{38} am allermeisten betrübt über das Wort,
das er sagte, sie würden sein Angesicht nicht mehr sehen; und geleiteten
ihn in das Schiff. \# 21 \bibverse{1} Als nun geschah, dass wir, von
ihnen gewandt, dahinfuhren, kamen wir geradewegs gen Kos und am
folgenden Tage gen Rhodus und von da gen Patara. \bibverse{2} Und da wir
ein Schiff fanden, das nach Phönizien fuhr, traten wir hinein und fuhren
hin. \bibverse{3} Als wir aber Zypern ansichtig wurden, ließen wir es
zur linken Hand und schifften nach Syrien und kamen an zu Tyrus; denn
daselbst sollte das Schiff die Ware niederlegen. \bibverse{4} Und als
wir Jünger fanden, blieben wir daselbst sieben Tage. Die sagten Paulus
durch den Geist, er sollte nicht hinauf gen Jerusalem ziehen.
\bibverse{5} Und es geschah, da wir die Tage zugebracht hatten, zogen
wir aus und reisten weiter. Und sie geleiteten uns alle mit Weib und
Kindern bis hinaus vor die Stadt, und wir knieten nieder am Ufer und
beteten. \footnote{\textbf{21:5} Apg 20,36} \bibverse{6} Und als wir
einander gesegnet, traten wir ins Schiff; jene aber wandten sich wieder
zu dem Ihren.

\bibverse{7} Wir aber vollzogen die Schifffahrt von Tyrus und kamen gen
Ptolemais und grüßten die Brüder und blieben einen Tag bei ihnen.
\bibverse{8} Des anderen Tages zogen wir aus, die wir um Paulus waren,
und kamen gen Cäsarea und gingen in das Haus Philippus des Evangelisten,
der einer von den Sieben war, und blieben bei ihm.

\bibverse{9} Der hatte vier Töchter, die waren Jungfrauen und
weissagten. \bibverse{10} Und als wir mehrere Tage dablieben, reiste
herab ein Prophet aus Judäa, mit Namen Agabus, und kam zu uns.
\footnote{\textbf{21:10} Apg 11,28} \bibverse{11} Der nahm den Gürtel
des Paulus und band sich die Hände und Füße und sprach: Das sagt der
heilige Geist: Den Mann, des der Gürtel ist, werden die Juden also
binden zu Jerusalem und überantworten in der Heiden Hände. \footnote{\textbf{21:11}
  Apg 20,23}

\bibverse{12} Als wir aber solches hörten, baten wir und die desselben
Ortes waren, dass er nicht hinauf gen Jerusalem zöge. \footnote{\textbf{21:12}
  Mt 16,22} \bibverse{13} Paulus aber antwortete: Was macht ihr, dass
ihr weinet und brechet mir mein Herz? Denn ich bin bereit, nicht allein
mich binden zu lassen, sondern auch zu sterben zu Jerusalem um des
Namens willen des Herrn Jesu. \footnote{\textbf{21:13} Apg 20,24}

\bibverse{14} Da er aber sich nicht überreden ließ, schwiegen wir und
sprachen: Des Herrn Wille geschehe. \footnote{\textbf{21:14} Lk 22,42}

\bibverse{15} Und nach diesen Tagen machten wir uns fertig und zogen
hinauf gen Jerusalem. \bibverse{16} Es kamen aber mit uns auch etliche
Jünger von Cäsarea und führten uns zu einem mit Namen Mnason aus Zypern,
der ein alter Jünger war, bei dem wir herbergen sollten.

\bibverse{17} Da wir nun gen Jerusalem kamen, nahmen uns die Brüder gern
auf. \bibverse{18} Des anderen Tages aber ging Paulus mit uns ein zu
Jakobus, und es kamen die Ältesten alle dahin. \bibverse{19} Und als er
sie gegrüßt hatte, erzählte er eines nach dem anderen, was Gott getan
hatte unter den Heiden durch sein Amt. \bibverse{20} Da sie aber das
hörten, lobten sie den Herrn und sprachen zu ihm: Bruder, du siehst,
wieviel tausend Juden sind, die gläubig geworden sind, und alle sind
Eiferer für das Gesetz; \footnote{\textbf{21:20} Apg 15,1} \bibverse{21}
sie sind aber berichtet worden wider dich, dass du lehrest von Moses
abfallen alle Juden, die unter den Heiden sind, und sagest, sie sollen
ihre Kinder nicht beschneiden, auch nicht nach desselben Weise wandeln.
\footnote{\textbf{21:21} Apg 16,3} \bibverse{22} Was denn nun?
Jedenfalls muss die Menge zusammenkommen; denn sie werden's hören, dass
du gekommen bist. \bibverse{23} So tue nun dies, was wir dir sagen.
\bibverse{24} Wir haben vier Männer, die haben ein Gelübde auf sich; die
nimm zu dir und heilige dich mit ihnen und wage die Kosten an sie, dass
sie ihr Haupt scheren, so werden alle vernehmen, dass es nicht so sei,
wie sie wider dich berichtet sind, sondern dass du auch einhergehst und
hältst das Gesetz. \footnote{\textbf{21:24} Apg 18,18} \bibverse{25}
Denn den Gläubigen aus den Heiden haben wir geschrieben und beschlossen,
dass sie der keines halten sollen, sondern nur sich bewahren vor dem
Götzenopfer, vor Blut, vor Ersticktem und vor Hurerei. \footnote{\textbf{21:25}
  Apg 15,20; Apg 15,29}

\bibverse{26} Da nahm Paulus die Männer zu sich und heiligte sich des
anderen Tages mit ihnen und ging in den Tempel und ließ sich sehen, wie
er aushielte die Tage, auf welche er sich heiligte, bis dass für einen
jeglichen unter ihnen das Opfer gebracht ward. \footnote{\textbf{21:26}
  4Mo 6,1-20; 1Kor 9,20} \bibverse{27} Als aber die sieben Tage sollten
vollendet werden, sahen ihn die Juden aus Asien im Tempel und erregten
das ganze Volk, legten die Hände an ihn und schrien: \bibverse{28} Ihr
Männer von Israel, helft! Dies ist der Mensch, der alle Menschen an
allen Enden lehrt wider dieses Volk, wider das Gesetz und wider diese
Stätte; dazu hat er auch Griechen in den Tempel geführt und diese
heilige Stätte gemein gemacht. \bibverse{29} (Denn sie hatten mit ihm in
der Stadt Trophimus, den Epheser, gesehen; den, meinten sie, hätte
Paulus in den Tempel geführt.) \footnote{\textbf{21:29} Apg 20,4; 2Tim
  4,20}

\bibverse{30} Und die ganze Stadt ward bewegt, und ward ein Zulauf des
Volks. Sie griffen aber Paulus und zogen ihn zum Tempel hinaus; und
alsbald wurden die Türen zugeschlossen. \bibverse{31} Da sie ihn aber
töten wollten, kam das Geschrei hinauf vor den obersten Hauptmann der
Schar, wie das ganze Jerusalem sich empörte. \bibverse{32} Der nahm von
Stund an die Kriegsknechte und Hauptleute zu sich und lief unter sie. Da
sie aber den Hauptmann und die Kriegsknechte sahen, hörten sie auf,
Paulus zu schlagen. \bibverse{33} Als aber der Hauptmann nahe herzukam,
nahm er ihn an sich und hieß ihn binden mit zwei Ketten und fragte, wer
er wäre und was er getan hätte. \bibverse{34} Einer aber rief dies, der
andere das im Volk. Da er aber nichts Gewisses erfahren konnte um des
Getümmels willen, hieß er ihn in das Lager führen.

\bibverse{35} Und als er an die Stufen kam, mussten ihn die
Kriegsknechte tragen vor Gewalt des Volks; \bibverse{36} denn es folgte
viel Volks nach und schrie: Weg mit ihm! \footnote{\textbf{21:36} Apg
  22,22; Lk 23,18} \bibverse{37} Als aber Paulus jetzt zum Lager
eingeführt ward, sprach er zu dem Hauptmann: Darf ich mit dir reden? Er
aber sprach: Kannst du Griechisch?

\bibverse{38} Bist du nicht der Ägypter, der vor diesen Tagen einen
Aufruhr gemacht hat und führte in die Wüste hinaus viertausend
Meuchelmörder?

\bibverse{39} Paulus aber sprach: Ich bin ein jüdischer Mann von Tarsus,
ein Bürger einer namhaften Stadt in Zilizien. Ich bitte dich, erlaube
mir, zu reden zu dem Volk.

\bibverse{40} Als er aber es ihm erlaubte, trat Paulus auf die Stufen
und winkte dem Volk mit der Hand. Da nun eine große Stille ward, redete
er zu ihnen auf hebräisch und sprach: \# 22 \bibverse{1} Ihr Männer,
liebe Brüder und Väter, höret mein Verantworten an euch.

\bibverse{2} Da sie aber hörten, dass er auf hebräisch zu ihnen redete,
wurden sie noch stiller. Und er sprach:

\bibverse{3} Ich bin ein jüdischer Mann, geboren zu Tarsus in Zilizien
und erzogen in dieser Stadt zu den Füßen Gamaliels, gelehrt mit allem
Fleiß im väterlichen Gesetz, und war ein Eiferer um Gott, gleichwie ihr
heute alle seid, \footnote{\textbf{22:3} Apg 5,34; Apg 9,1-29; Apg
  26,9-20} \bibverse{4} und habe diesen Weg verfolgt bis an den Tod. Ich
band sie und überantwortete sie ins Gefängnis, Männer und Weiber;
\footnote{\textbf{22:4} Apg 8,3} \bibverse{5} wie mir auch der
Hohepriester und der ganze Haufe der Ältesten Zeugnis gibt, von welchen
ich Briefe nahm an die Brüder und reiste gen Damaskus; dass ich, die
daselbst waren, gebunden führte gen Jerusalem, dass sie bestraft würden.

\bibverse{6} Es geschah aber, da ich hinzog und nahe Damaskus kam, um
den Mittag, umleuchtete mich schnell ein großes Licht vom Himmel.
\bibverse{7} Und ich fiel zum Erdboden und hörte eine Stimme, die sprach
zu mir: Saul, Saul, was verfolgst du mich? \bibverse{8} Ich antwortete
aber: Herr, wer bist du? Und er sprach zu mir: Ich bin Jesus von
Nazareth, den du verfolgst.

\bibverse{9} Die aber mit mir waren, sahen das Licht und erschraken; die
Stimme aber des, der mit mir redete, hörten sie nicht. \bibverse{10} Ich
sprach aber: Herr, was soll ich tun? Der Herr aber sprach zu mir: Stehe
auf und gehe gen Damaskus; da wird man dir sagen von allem, was dir zu
tun verordnet ist. \bibverse{11} Als ich aber vor Klarheit dieses
Lichtes nicht sehen konnte, ward ich bei der Hand geleitet von denen,
die mit mir waren, und kam gen Damaskus.

\bibverse{12} Es war aber ein gottesfürchtiger Mann nach dem Gesetz,
Ananias, der ein gut Gerücht hatte bei allen Juden, die daselbst
wohnten; \bibverse{13} der kam zu mir und trat her und sprach zu mir:
Saul, lieber Bruder, siehe auf! Und ich sah ihn an zu derselben Stunde.
\bibverse{14} Er aber sprach: Der Gott unserer Väter hat dich verordnet,
dass du seinen Willen erkennen solltest und sehen den Gerechten und
hören die Stimme aus seinem Munde; \bibverse{15} denn du wirst sein
Zeuge zu allen Menschen sein von dem, das du gesehen und gehört hast.
\bibverse{16} Und nun, was verziehest du? Stehe auf und lass dich taufen
und abwaschen deine Sünden und rufe an den Namen des Herrn!

\bibverse{17} Es geschah aber, da ich wieder gen Jerusalem kam und
betete im Tempel, dass ich entzückt ward und sah ihn. \bibverse{18} Da
sprach er zu mir: Eile und mache dich behend von Jerusalem hinaus; denn
sie werden nicht aufnehmen dein Zeugnis von mir. \bibverse{19} Und ich
sprach: Herr, sie wissen selbst, dass ich gefangen legte und stäupte
die, die an dich glaubten, in den Schulen hin und her; \bibverse{20} und
da das Blut des Stephanus, deines Zeugen, vergossen ward, stand ich auch
dabei und hatte Wohlgefallen an seinem Tode und verwahrte denen die
Kleider, die ihn töteten. \footnote{\textbf{22:20} Apg 7,57; Apg 8,1}

\bibverse{21} Und er sprach zu mir: Gehe hin; denn ich will dich ferne
unter die Heiden senden! \footnote{\textbf{22:21} Apg 13,2}

\bibverse{22} Sie hörten aber ihm zu bis auf dies Wort und hoben ihre
Stimme auf und sprachen: Hinweg mit solchem von der Erde! denn es ist
nicht billig, dass er leben soll. \footnote{\textbf{22:22} Apg 21,36}

\bibverse{23} Da sie aber schrien und ihre Kleider abwarfen und den
Staub in die Luft warfen, \bibverse{24} hieß ihn der Hauptmann in das
Lager führen und sagte, dass man ihn stäupen und befragen sollte, dass
er erführe, um welcher Ursache willen sie also über ihn riefen.
\bibverse{25} Als man ihn aber mit Riemen anband, sprach Paulus zu dem
Unterhauptmann der dabeistand: Ist's auch recht bei euch, einen
römischen Menschen ohne Urteil und Recht zu geißeln?

\bibverse{26} Da das der Unterhauptmann hörte, ging er zu dem
Oberhauptmann und verkündigte ihm und sprach: Was willst du machen?
Dieser Mensch ist römisch.

\bibverse{27} Da kam zu ihm der Oberhauptmann und sprach zu ihm: Sage
mir, bist du römisch? Er aber sprach: Ja.

\bibverse{28} Und der Oberhauptmann antwortete: Ich habe dies
Bürgerrecht mit großer Summe zuwege gebracht. Paulus aber sprach: Ich
bin aber auch römisch geboren.

\bibverse{29} Da traten sie alsobald von ihm ab, die ihn befragen
sollten. Und der Oberhauptmann fürchtete sich, da er vernahm, dass er
römisch war, und er ihn gebunden hatte.

\bibverse{30} Des anderen Tages wollte er gewiss erkunden, warum er
verklagt würde von den Juden, und löste ihn von den Banden und hieß die
Hohenpriester und ihren ganzen Rat kommen und führte Paulus hervor und
stellte ihn unter sie. \# 23 \bibverse{1} Paulus aber sah den Rat an und
sprach: Ihr Männer, liebe Brüder, ich habe mit allem guten Gewissen
gewandelt vor Gott bis auf diesen Tag. \footnote{\textbf{23:1} Apg 24,16}

\bibverse{2} Der Hohepriester aber, Ananias, befahl denen, die um ihn
standen, dass sie ihn aufs Maul schlügen.

\bibverse{3} Da sprach Paulus zu ihm: Gott wird dich schlagen, du
getünchte Wand! Sitzest du, mich zu richten nach dem Gesetz, und heißest
mich schlagen wider das Gesetz?

\bibverse{4} Die aber umherstanden, sprachen: Schiltst du den
Hohenpriester Gottes?

\bibverse{5} Und Paulus sprach: Liebe Brüder, ich wusste nicht, dass er
der Hohepriester ist. Denn es steht geschrieben: „Dem Obersten deines
Volkes sollst du nicht fluchen.``

\bibverse{6} Da aber Paulus wusste, dass ein Teil Sadduzäer war und der
andere Teil Pharisäer, rief er im Rat: Ihr Männer, liebe Brüder, ich bin
ein Pharisäer und eines Pharisäers Sohn; ich werde angeklagt um der
Hoffnung und Auferstehung willen der Toten. \footnote{\textbf{23:6} Apg
  22,3; Apg 26,5; Gal 1,14}

\bibverse{7} Da er aber das sagte, ward ein Aufruhr unter den Pharisäern
und Sadduzäern, und die Menge zerspaltete sich. \bibverse{8} (Denn die
Sadduzäer sagen: Es sei keine Auferstehung noch Engel noch Geist; die
Pharisäer aber bekennen beides.) \bibverse{9} Es ward aber ein großes
Geschrei; und die Schriftgelehrten von der Pharisäer Teil standen auf,
stritten und sprachen: Wir finden nichts Arges an diesem Menschen; hat
aber ein Geist oder ein Engel mit ihm geredet, so können wir mit Gott
nicht streiten. \footnote{\textbf{23:9} Apg 25,25; Apg 5,39}

\bibverse{10} Da aber der Aufruhr groß ward, besorgte sich der oberste
Hauptmann, sie möchten Paulus zerreißen, und hieß das Kriegsvolk
hinabgehen und ihn von ihnen reißen und in das Lager führen.

\bibverse{11} Des anderen Tages aber in der Nacht stand der Herr bei ihm
und sprach: Sei getrost, Paulus! denn wie du von mir zu Jerusalem
gezeugt hast, also musst du auch zu Rom zeugen.

\bibverse{12} Da es aber Tag ward, schlugen sich etliche Juden zusammen
und verschworen sich, weder zu essen noch zu trinken, bis dass sie
Paulus getötet hätten. \bibverse{13} Ihrer aber waren mehr denn vierzig,
die solchen Bund machten. \bibverse{14} Die traten zu den Hohenpriestern
und Ältesten und sprachen: Wir haben uns hart verschworen, nichts zu
essen, bis wir Paulus getötet haben. \bibverse{15} So tut nun kund dem
Oberhauptmann und dem Rat, dass er ihn morgen zu euch führe, als wolltet
ihr ihn besser verhören; wir aber sind bereit, ihn zu töten, ehe er denn
vor euch kommt.

\bibverse{16} Da aber des Paulus Schwestersohn den Anschlag hörte, ging
er hin und kam in das Lager und verkündigte es Paulus. \bibverse{17}
Paulus aber rief zu sich einen von den Unterhauptleuten und sprach:
Diesen Jüngling führe hin zu dem Oberhauptmann; denn er hat ihm etwas zu
sagen.

\bibverse{18} Der nahm ihn und führte ihn zum Oberhauptmann und sprach:
Der gebundene Paulus rief mich zu sich und bat mich, diesen Jüngling zu
dir zu führen, der dir etwas zu sagen habe.

\bibverse{19} Da nahm ihn der Oberhauptmann bei der Hand und wich an
einen besonderen Ort und fragte ihn: Was ist's, das du mir zu sagen
hast?

\bibverse{20} Er aber sprach: Die Juden sind eins geworden, dich zu
bitten, dass du morgen Paulus vor den Rat bringen lassest, als wollten
sie ihn besser verhören. \bibverse{21} Du aber traue ihnen nicht; denn
es lauern auf ihn mehr als vierzig Männer unter ihnen, die haben sich
verschworen, weder zu essen noch zu trinken, bis sie Paulus töten; und
sind jetzt bereit und warten auf deine Verheißung.

\bibverse{22} Da ließ der Oberhauptmann den Jüngling von sich und gebot
ihm, dass niemand sagte, dass er ihm solches eröffnet hätte,

\bibverse{23} und rief zu sich zwei Unterhauptleute und sprach: Rüstet
zweihundert Kriegsknechte, dass sie gen Cäsarea ziehen, und siebzig
Reiter und zweihundert Schützen auf die dritte Stunde der Nacht;
\bibverse{24} und die Tiere richtet zu, dass sie Paulus draufsetzen und
bringen ihn bewahrt zu Felix, dem Landpfleger. \bibverse{25} Und schrieb
einen Brief, der lautete also:

\bibverse{26} Klaudius Lysias dem teuren Landpfleger Felix Freude zuvor!

\bibverse{27} Diesen Mann hatten die Juden gegriffen und wollten ihn
getötet haben. Da kam ich mit dem Kriegsvolk dazu und riss ihn von ihnen
und erfuhr, dass er ein Römer ist. \footnote{\textbf{23:27} Apg 21,33;
  Apg 22,25} \bibverse{28} Da ich aber erkunden wollte die Ursache,
darum sie ihn beschuldigten, führte ich ihn in ihren Rat. \footnote{\textbf{23:28}
  Apg 22,30} \bibverse{29} Da befand ich, dass er beschuldigt ward von
wegen Fragen ihres Gesetzes, aber keine Anklage hatte, des Todes oder
der Bande wert. \footnote{\textbf{23:29} Apg 18,14-15} \bibverse{30} Und
da vor mich kam, dass etliche Juden auf ihn lauerten, sandte ich ihn von
Stund an zu dir und entbot den Klägern auch, dass sie vor Dir sagten,
was sie wider ihn hätten. Gehab dich wohl! \footnote{\textbf{23:30} Apg
  24,8}

\bibverse{31} Die Kriegsknechte, wie ihnen befohlen war, nahmen Paulus
und führten ihn bei der Nacht gen Antipatris. \bibverse{32} Des anderen
Tages aber ließen sie die Reiter mit ihm ziehen und wandten wieder um
zum Lager. \bibverse{33} Da die gen Cäsarea kamen, überantworteten sie
den Brief dem Landpfleger und stellten ihm Paulus auch dar.
\bibverse{34} Da der Landpfleger den Brief las, fragte er, aus welchem
Lande er wäre. Und da er erkundet, dass er aus Zilizien wäre, sprach er:
\footnote{\textbf{23:34} Apg 22,3} \bibverse{35} Ich will dich verhören,
wenn deine Verkläger auch da sind. Und hieß ihn verwahren in dem
Richthause des Herodes. \# 24 \bibverse{1} Über fünf Tage zog hinab der
Hohepriester Ananias mit den Ältesten und mit dem Redner Tertullus; die
erschienen vor dem Landpfleger wider Paulus. \bibverse{2} Da er aber
berufen ward, fing an Tertullus zu verklagen und sprach: \bibverse{3}
Dass wir in großem Frieden leben unter dir und viel Wohltaten diesem
Volk widerfahren durch deine Fürsichtigkeit, allerteuerster Felix, das
nehmen wir an allewege und allenthalben mit aller Dankbarkeit.
\bibverse{4} Auf dass ich aber dich nicht zu lange aufhalte, bitte ich
dich, du wolltest uns kürzlich hören nach deiner Gelindigkeit.
\bibverse{5} Wir haben diesen Mann gefunden schädlich, und der Aufruhr
erregt allen Juden auf dem ganzen Erdboden, und einen vornehmsten der
Sekte der Nazarener, \bibverse{6} der auch versucht hat, den Tempel zu
entweihen; welchen wir auch griffen und wollten ihn gerichtet haben nach
unserem Gesetz. \footnote{\textbf{24:6} Apg 21,28-29} \bibverse{7} Aber
Lysias, der Hauptmann, kam dazu und führte ihn mit großer Gewalt aus
unseren Händen \bibverse{8} und hieß seine Verkläger zu dir kommen; von
welchem du kannst, so du es erforschen willst, das alles erkunden, um
was wir ihn verklagen.

\bibverse{9} Die Juden aber redeten auch dazu und sprachen, es verhielte
sich also.

\bibverse{10} Paulus aber, da ihm der Landpfleger winkte zu reden,
antwortete: Dieweil ich weiß, dass du in diesem Volk nun viele Jahre ein
Richter bist, will ich unerschrocken mich verantworten; \bibverse{11}
denn du kannst erkennen, dass es nicht mehr als zwölf Tage sind, dass
ich bin hinauf gen Jerusalem gekommen, anzubeten. \bibverse{12} Auch
haben sie mich nicht gefunden im Tempel mit jemanden reden oder einen
Aufruhr machen im Volk noch in den Schulen noch in der Stadt.
\bibverse{13} Sie können mir auch der keines beweisen, dessen sie mich
verklagen. \bibverse{14} Das bekenne ich aber dir, dass ich nach diesem
Wege, den sie eine Sekte heißen, diene also dem Gott meiner Väter, dass
ich glaube allem, was geschrieben steht im Gesetz und in den Propheten,
\bibverse{15} und habe die Hoffnung zu Gott, auf welche auch sie selbst
warten, nämlich, dass zukünftig sei die Auferstehung der Toten, der
Gerechten und der Ungerechten. \footnote{\textbf{24:15} Dan 12,2; Joh
  5,28-29} \bibverse{16} Dabei aber übe ich mich, zu haben ein
unverletzt Gewissen allenthalben, gegen Gott und die Menschen.
\footnote{\textbf{24:16} Apg 23,1} \bibverse{17} Aber nach vielen Jahren
bin ich gekommen und habe ein Almosen gebracht meinem Volk, und Opfer.
\footnote{\textbf{24:17} Röm 15,25-26; Gal 2,10} \bibverse{18} Darüber
fanden sie mich, dass ich mich geheiligt hatte im Tempel, ohne allen
Lärm und Getümmel. \footnote{\textbf{24:18} Apg 21,27} \bibverse{19} Das
waren aber etliche Juden aus Asien, welche sollten hier sein vor dir und
mich verklagen, wenn sie etwas wider mich hätten. \bibverse{20} Oder
lass diese selbst sagen, ob sie etwas Unrechtes an mir gefunden haben,
dieweil ich stand vor dem Rat, \bibverse{21} außer um des einzigen
Wortes willen, da ich unter ihnen stand und rief: Über die Auferstehung
der Toten werde ich von euch heute angeklagt. \footnote{\textbf{24:21}
  Apg 23,6}

\bibverse{22} Da aber Felix solches hörte, zog er sie hin; denn er
wusste gar wohl um diesen Weg und sprach: Wenn Lysias, der Hauptmann,
herabkommt, so will ich eure Sache erkunden. \footnote{\textbf{24:22}
  Apg 23,26} \bibverse{23} Er befahl aber dem Unterhauptmann, Paulus zu
behalten und lassen Ruhe haben und dass er niemand von den Seinen
wehrte, ihm zu dienen oder zu ihm zu kommen. \footnote{\textbf{24:23}
  Apg 27,3}

\bibverse{24} Nach etlichen Tagen aber kam Felix mit seinem Weibe
Drusilla, die eine Jüdin war, und forderte Paulus und hörte ihn von dem
Glauben an Christum. \bibverse{25} Da aber Paulus redete von der
Gerechtigkeit und von der Keuschheit und von dem zukünftigen Gericht,
erschrak Felix und antwortete: Gehe hin auf diesmal; wenn ich gelegene
Zeit habe, will ich dich herrufen lassen. \bibverse{26} Er hoffte aber
daneben, dass ihm von Paulus sollte Geld gegeben werden, dass er ihn
losgäbe; darum er ihn auch oft fordern ließ und besprach sich mit ihm.

\bibverse{27} Da aber zwei Jahre um waren, kam Porcius Festus an Felix
Statt. Felix aber wollte den Juden eine Gunst erzeigen und ließ Paulus
hinter sich gefangen. \# 25 \bibverse{1} Da nun Festus ins Land gekommen
war, zog er über drei Tage hinauf von Cäsarea gen Jerusalem.
\bibverse{2} Da erschienen vor ihm die Hohenpriester und die Vornehmsten
der Juden wider Paulus und ermahnten ihn \bibverse{3} und baten um Gunst
wider ihn, dass er ihn fordern ließe gen Jerusalem, und stellten ihm
nach, dass sie ihn unterwegs umbrächten. \footnote{\textbf{25:3} Apg
  23,15} \bibverse{4} Da antwortete Festus, Paulus würde ja behalten zu
Cäsarea; aber er würde in kurzem wieder dahin ziehen. \bibverse{5}
Welche nun unter euch (sprach er) können, die lasst mit hinabziehen und
den Mann verklagen, so etwas an ihm ist.

\bibverse{6} Da er aber bei ihnen mehr denn zehn Tage gewesen war, zog
er hinab gen Cäsarea; und des anderen Tages setzte er sich auf den
Richtstuhl und hieß Paulus holen. \bibverse{7} Da der aber vor ihn kam,
traten umher die Juden, die von Jerusalem herabgekommen waren, und
brachten auf viele und schwere Klagen wider Paulus, welche sie nicht
konnten beweisen, \bibverse{8} dieweil er sich verantwortete: Ich habe
weder an der Juden Gesetz noch an dem Tempel noch am Kaiser mich
versündigt.

\bibverse{9} Festus aber wollte den Juden eine Gunst erzeigen und
antwortete Paulus und sprach: Willst du hinauf gen Jerusalem und
daselbst über dieses dich vor mir richten lassen?

\bibverse{10} Paulus aber sprach: Ich stehe vor des Kaisers Gericht, da
soll ich mich lassen richten; den Juden habe ich kein Leid getan, wie
auch du aufs beste weißt. \bibverse{11} Habe ich aber jemand Leid getan
und des Todes wert gehandelt, so weigere ich mich nicht, zu sterben; ist
aber der keines nicht, dessen sie mich verklagen, so kann mich ihnen
niemand übergeben. Ich berufe mich auf den Kaiser!

\bibverse{12} Da besprach sich Festus mit dem Rat und antwortete: Auf
den Kaiser hast du dich berufen, zum Kaiser sollst du ziehen.

\bibverse{13} Aber nach etlichen Tagen kamen der König Agrippa und
Bernice gen Cäsarea, Festus zu begrüßen. \bibverse{14} Und da sie viele
Tage daselbst gewesen waren, legte Festus dem König den Handel von
Paulus vor und sprach: Es ist ein Mann von Felix hinterlassen gefangen,
\footnote{\textbf{25:14} Apg 24,27} \bibverse{15} um welches willen die
Hohenpriester und Ältesten der Juden vor mir erschienen, da ich zu
Jerusalem war, und baten, ich sollte ihn richten lassen; \bibverse{16}
denen antwortete ich: Es ist der Römer Weise nicht, dass ein Mensch
übergeben werde, ihn umzubringen, ehe denn der Verklagte seine Kläger
gegenwärtig habe und Raum empfange, sich auf die Anklage zu
verantworten. \bibverse{17} Da sie aber her zusammenkamen, machte ich
keinen Aufschub und hielt des anderen Tages Gericht und hieß den Mann
vorbringen; \bibverse{18} und da seine Verkläger auftraten, brachten sie
der Ursachen keine auf, deren ich mich versah. \bibverse{19} Sie hatten
aber etliche Fragen wider ihn von ihrem Aberglauben und von einem
verstorbenen Jesus, von welchem Paulus sagte, er lebe. \footnote{\textbf{25:19}
  Apg 18,15} \bibverse{20} Da ich aber mich auf die Frage nicht
verstand, sprach ich, ob er wollte gen Jerusalem reisen und daselbst
sich darüber lassen richten. \bibverse{21} Da aber Paulus sich berief,
dass er für des Kaisers Erkenntnis aufbehalten würde, hieß ich ihn
behalten, bis dass ich ihn zum Kaiser sende.

\bibverse{22} Agrippa aber sprach zu Festus: Ich möchte den Menschen
auch gerne hören. Er aber sprach: Morgen sollst du ihn hören.

\bibverse{23} Und am anderen Tage, da Agrippa und Bernice kamen mit
großem Gepränge und gingen in das Richthaus mit den Hauptleuten und
vornehmsten Männern der Stadt, und da es Festus hieß, ward Paulus
gebracht.

\bibverse{24} Und Festus sprach: Lieber König Agrippa und alle ihr
Männer, die ihr mit uns hier seid, da sehet ihr den, um welchen mich die
ganze Menge der Juden angegangen hat, zu Jerusalem und auch hier, und
schrien, er solle nicht länger leben. \footnote{\textbf{25:24} Apg 22,22}
\bibverse{25} Ich aber, da ich vernahm, dass er nichts getan hatte, das
des Todes wert sei, und er auch selber sich auf den Kaiser berief, habe
ich beschlossen, ihn zu senden. \bibverse{26} Etwas Gewisses aber habe
ich von ihm nicht, das ich dem Herrn schreibe. Darum habe ich ihn lassen
hervorbringen vor euch, allermeist aber vor dich, König Agrippa, auf
dass ich nach geschehener Erforschung haben möge, was ich schreibe.
\bibverse{27} Denn es dünkt mich ein ungeschicktes Ding zu sein, einen
Gefangenen schicken und keine Ursachen wider ihn anzuzeigen. \# 26
\bibverse{1} Agrippa aber sprach zu Paulus: Es ist dir erlaubt, für dich
zu reden. Da reckte Paulus die Hand aus und verantwortete sich:

\bibverse{2} Es ist mir sehr lieb, König Agrippa, dass ich mich heute
vor dir verantworten soll über alles, dessen ich von den Juden
beschuldigt werde; \bibverse{3} allermeist weil du weißt alle Sitten und
Fragen der Juden. Darum bitte ich dich, du wollest mich geduldig hören.

\bibverse{4} Zwar mein Leben von Jugend auf, wie das von Anfang unter
diesem Volk zu Jerusalem zugebracht ist, wissen alle Juden, \bibverse{5}
die mich von Anbeginn gekannt haben, wenn sie es wollten bezeugen. Denn
ich bin ein Pharisäer gewesen, welches ist die strengste Sekte unseres
Gottesdienstes. \bibverse{6} Und nun stehe ich und werde angeklagt über
die Hoffnung auf die Verheißung, die geschehen ist von Gott zu unseren
Vätern, \footnote{\textbf{26:6} Apg 28,20} \bibverse{7} zu welcher
hoffen die zwölf Geschlechter der Unseren zu kommen mit Gottesdienst
emsig Tag und Nacht. Dieser Hoffnung halben werde ich, König Agrippa,
von den Juden beschuldigt. \footnote{\textbf{26:7} Apg 24,15}
\bibverse{8} Warum wird das für unglaublich bei euch geachtet, das Gott
Tote auferweckt? \footnote{\textbf{26:8} Apg 23,8}

\bibverse{9} Zwar meinte ich auch bei mir selbst, ich müsste viel
zuwider tun dem Namen Jesu von Nazareth, \footnote{\textbf{26:9} Apg
  9,1-29; Apg 22,3-21} \bibverse{10} wie ich denn auch zu Jerusalem
getan habe, da ich viele Heilige in das Gefängnis verschloss, darüber
ich Macht von den Hohenpriestern empfing; und wenn sie erwürgt wurden,
half ich das Urteil sprechen. \bibverse{11} Und durch alle Schulen
peinigte ich sie oft und zwang sie zu lästern; und war überaus unsinnig
auf sie, verfolgte sie auch bis in die fremden Städte.

\bibverse{12} Über dem, da ich auch gen Damaskus reiste mit Macht und
Befehl von den Hohenpriestern, \bibverse{13} sah ich mitten am Tage, o
König, auf dem Wege ein Licht vom Himmel, heller denn der Sonne Glanz,
das mich und die mit mir reisten, umleuchtete. \bibverse{14} Da wir aber
alle zur Erde niederfielen, hörte ich eine Stimme reden zu mir, die
sprach auf hebräisch: Saul, Saul, was verfolgst du mich? Es wird dir
schwer sein, wider den Stachel zu lecken.

\bibverse{15} Ich aber sprach: Herr, wer bist du? Er sprach: Ich bin
Jesus, den du verfolgst; aber stehe auf und tritt auf deine Füße.

\bibverse{16} Denn dazu bin ich dir erschienen, dass ich dich ordne zum
Diener und Zeugen des, das du gesehen hast und das ich dir noch will
erscheinen lassen; \bibverse{17} und ich will dich erretten von dem Volk
und von den Heiden, unter welche ich dich jetzt sende, \bibverse{18}
aufzutun ihre Augen, dass sie sich bekehren von der Finsternis zu dem
Licht und von der Gewalt des Satans zu Gott, zu empfangen Vergebung der
Sünden und das Erbe samt denen, die geheiligt werden durch den Glauben
an mich. \footnote{\textbf{26:18} Apg 20,32}

\bibverse{19} Daher, König Agrippa, war ich der himmlischen Erscheinung
nicht ungläubig, \footnote{\textbf{26:19} Gal 1,16} \bibverse{20}
sondern verkündigte zuerst denen zu Damaskus und Jerusalem und in alle
Gegend des jüdischen Landes und auch den Heiden, dass sie Buße täten und
sich bekehrten zu Gott und täten rechtschaffene Werke der Buße.
\bibverse{21} Um deswillen haben mich die Juden im Tempel gegriffen und
versuchten, mich zu töten. \footnote{\textbf{26:21} Apg 21,30-31}
\bibverse{22} Aber durch Hilfe Gottes ist es mir gelungen und stehe ich
bis auf diesen Tag und zeuge beiden, dem Kleinen und Großen, und sage
nichts außer dem, was die Propheten gesagt haben, dass es geschehen
sollte, und Mose: \footnote{\textbf{26:22} Lk 24,44-47} \bibverse{23}
dass Christus sollte leiden und der erste sein aus der Auferstehung von
den Toten und verkündigen ein Licht dem Volk und den Heiden. \footnote{\textbf{26:23}
  1Kor 15,20}

\bibverse{24} Da er aber solches zur Verantwortung gab, sprach Festus
mit lauter Stimme: Paulus, du rasest! Die große Kunst macht dich rasend.

\bibverse{25} Er aber sprach: Mein teurer Festus, ich rase nicht,
sondern ich rede wahre und vernünftige Worte. \bibverse{26} Denn der
König weiß solches wohl, zu welchem ich freudig rede. Denn ich achte,
ihm sei der keines verborgen; denn solches ist nicht im Winkel
geschehen. \bibverse{27} Glaubst du, König Agrippa, den Propheten? Ich
weiß, dass du glaubst.

\bibverse{28} Agrippa aber sprach zu Paulus: Es fehlt nicht viel, du
überredest mich, dass ich ein Christ würde.

\bibverse{29} Paulus aber sprach: Ich wünschte vor Gott, es fehle nun an
viel oder an wenig, dass nicht allein du, sondern alle, die mich heute
hören, solche würden, wie ich bin, ausgenommen diese Bande.

\bibverse{30} Und da er das gesagt, stand der König auf und der
Landpfleger und Bernice und die die mit ihnen saßen, \bibverse{31} und
wichen beiseits, redeten miteinander und sprachen: Dieser Mensch hat
nichts getan, das des Todes oder der Bande wert sei. \bibverse{32}
Agrippa aber sprach zu Festus: Dieser Mensch hätte können losgegeben
werden, wenn er sich nicht auf den Kaiser berufen hätte. \footnote{\textbf{26:32}
  Apg 25,11}

\hypertarget{section-6}{%
\section{27}\label{section-6}}

\bibverse{1} Da es aber beschlossen war, dass wir nach Italien schiffen
sollten, übergaben sie Paulus und etliche andere Gefangene dem
Unterhauptmann mit Namen Julius, von der „kaiserlichen`` Schar.
\footnote{\textbf{27:1} Apg 25,12} \bibverse{2} Da wir aber in ein
adramyttisches Schiff traten, dass wir an Asien hin schiffen sollten,
fuhren wir vom Lande; und mit uns war Aristarchus aus Mazedonien, von
Thessalonich. \footnote{\textbf{27:2} Apg 20,4} \bibverse{3} Und des
anderen Tages kamen wir an zu Sidon; und Julius hielt sich freundlich
gegen Paulus, erlaubte ihm, zu seinen guten Freunden zu gehen und sich
zu pflegen. \footnote{\textbf{27:3} Apg 24,23; Apg 28,16} \bibverse{4}
Und von da stießen wir ab und schifften unter Zypern hin, darum dass uns
die Winde entgegen waren, \bibverse{5} und schifften durch das Meer bei
Zilizien und Pamphylien und kamen gen Myra in Lyzien. \bibverse{6} Und
daselbst fand der Unterhauptmann ein Schiff von Alexandrien, das
schiffte nach Italien, und ließ uns darauf übersteigen. \bibverse{7} Da
wir aber langsam schifften und in vielen Tagen kaum gegen Knidus kamen
(denn der Wind wehrte uns), schifften wir unter Kreta hin bei Salmone
\bibverse{8} und zogen mit Mühe vorüber; da kamen wir an eine Stätte,
die heißt Gutfurt, dabei war nahe die Stadt Lasäa.

\bibverse{9} Da nun viel Zeit vergangen war und nunmehr gefährlich war
zu schiffen, darum dass auch das Fasten schon vorüber war, vermahnte sie
Paulus \footnote{\textbf{27:9} 2Kor 11,25-26; 3Mo 16,29} \bibverse{10}
und sprach zu ihnen: Liebe Männer, ich sehe, dass die Schifffahrt will
mit Leid und großem Schaden ergehen, nicht allein der Last und des
Schiffes sondern auch unseres Lebens. \bibverse{11} Aber der
Unterhauptmann glaubte dem Steuermann und dem Schiffsherrn mehr denn
dem, was Paulus sagte. \bibverse{12} Und da die Anfurt ungelegen war, zu
überwintern, bestanden ihrer der mehrere Teil auf dem Rat, von dannen zu
fahren, ob sie könnten kommen gen Phönix, zu überwintern, welches ist
eine Anfurt an Kreta gegen Südwest und Nordwest.

\bibverse{13} Da aber der Südwind wehte und sie meinten, sie hätten nun
ihr Vornehmen, erhoben sie sich und fuhren näher an Kreta hin.
\bibverse{14} Nicht lange aber darnach erhob sich wider ihr Vornehmen
eine Windsbraut, die man nennt Nordost. \bibverse{15} Und da das Schiff
ergriffen ward und konnte sich nicht wider den Wind richten, gaben wir's
dahin und schwebten also. \bibverse{16} Wir kamen aber an eine Insel,
die heißt Klauda; da konnten wir kaum den Kahn ergreifen. \bibverse{17}
Den hoben wir auf und brauchten die Hilfe und unterbanden das Schiff;
denn wir fürchteten, es möchte in die Syrte fallen, und ließen die Segel
herunter und fuhren also. \bibverse{18} Und da wir großes Ungewitter
erlitten, taten sie des nächsten Tages einen Auswurf. \bibverse{19} Und
am dritten Tage warfen wir mit unseren Händen aus die Gerätschaft im
Schiffe. \bibverse{20} Da aber in vielen Tagen weder Sonne noch Gestirn
erschien und ein nicht kleines Ungewitter uns drängte, war alle Hoffnung
unseres Lebens dahin.

\bibverse{21} Und da man lange nicht gegessen hatte, trat Paulus mitten
unter sie und sprach: Liebe Männer, man sollte mir gehorcht haben und
nicht von Kreta aufgebrochen sein, und uns dieses Leides und Schadens
überhoben haben. \bibverse{22} Und nun ermahne ich euch, dass ihr
unverzagt seid; denn keines Leben aus uns wird umkommen, nur das Schiff.
\bibverse{23} Denn diese Nacht ist bei mir gestanden der Engel Gottes,
des ich bin und dem ich diene, \bibverse{24} und sprach: Fürchte dich
nicht, Paulus! du musst vor den Kaiser gestellt werden; und siehe, Gott
hat dir geschenkt alle, die mit dir schiffen. \bibverse{25} Darum, liebe
Männer, seid unverzagt; denn ich glaube Gott, es wird also geschehen,
wie mir gesagt ist. \bibverse{26} Wir müssen aber anfahren an eine
Insel. \footnote{\textbf{27:26} Apg 28,1}

\bibverse{27} Da aber die vierzehnte Nacht kam, dass wir im Adria-Meer
fuhren, um die Mitternacht, wähnten die Schiffsleute, sie kämen etwa an
ein Land. \bibverse{28} Und sie senkten den Bleiwurf ein und fanden
zwanzig Klafter tief; und über ein wenig davon senkten sie abermals und
fanden fünfzehn Klafter. \bibverse{29} Da fürchteten sie sich, sie
würden an harte Orte anstoßen, und warfen hinten vom Schiffe vier Anker
und wünschten, dass es Tag würde. \bibverse{30} Da aber die Schiffsleute
die Flucht suchten aus dem Schiffe und den Kahn niederließen in das Meer
und gaben vor, sie wollten die Anker vorn aus dem Schiffe lassen,
\bibverse{31} sprach Paulus zu dem Unterhauptmann und zu den
Kriegsknechten: Wenn diese nicht im Schiffe bleiben, so könnt ihr nicht
am Leben bleiben. \bibverse{32} Da hieben die Kriegsknechte die Stricke
ab von dem Kahn und ließen ihn fallen.

\bibverse{33} Und da es anfing licht zu werden, ermahnte sie Paulus
alle, dass sie Speise nähmen, und sprach: Es ist heute der vierzehnte
Tag, dass ihr wartet und ungegessen geblieben seid und habt nichts zu
euch genommen. \bibverse{34} Darum ermahne ich euch, Speise zu nehmen,
euch zu laben; denn es wird euer keinem ein Haar von dem Haupt
entfallen. \bibverse{35} Und da er das gesagt, nahm er das Brot, dankte
Gott vor ihnen allen und brach's und fing an zu essen. \footnote{\textbf{27:35}
  Joh 6,11} \bibverse{36} Da wurden sie alle gutes Muts und nahmen auch
Speise. \bibverse{37} Unser waren aber alle zusammen im Schiff
zweihundertsechsundsiebzig Seelen. \bibverse{38} Und da sie satt
geworden, erleichterten sie das Schiff und warfen das Getreide in das
Meer. \bibverse{39} Da es aber Tag ward, kannten sie das Land nicht;
einer Anfurt aber wurden sie gewahr, die hatte ein Ufer; dahinan wollten
sie das Schiff treiben, wenn es möglich wäre. \bibverse{40} Und sie
hieben die Anker ab und ließen sie dem Meer, lösten zugleich die Bande
der Steuerruder auf und richteten das Segel nach dem Winde und
trachteten nach dem Ufer. \bibverse{41} Und da wir fuhren an einen Ort,
der auf beiden Seiten Meer hatte, stieß sich das Schiff an, und das
Vorderteil blieb feststehen unbeweglich; aber das Hinterteil zerbrach
von der Gewalt der Wellen.

\bibverse{42} Die Kriegsknechte aber hatten einen Rat, die Gefangenen zu
töten, dass nicht jemand, wenn er ausschwömme, entflöhe. \bibverse{43}
Aber der Unterhauptmann wollte Paulus erhalten und wehrte ihrem
Vornehmen und hieß, die da schwimmen könnten, sich zuerst in das Meer
lassen und entrinnen an das Land, \bibverse{44} die anderen aber etliche
auf Brettern, etliche auf dem, das vom Schiff war. Und also geschah es,
dass sie alle gerettet zu Lande kamen. \# 28 \bibverse{1} Und da wir
gerettet waren, erfuhren wir, dass die Insel Melite hieß. \bibverse{2}
Die Leutlein aber erzeigten uns nicht geringe Freundschaft, zündeten ein
Feuer an und nahmen uns alle auf um des Regens, der über uns gekommen
war, und um der Kälte willen. \footnote{\textbf{28:2} 2Kor 11,27}
\bibverse{3} Da aber Paulus einen Haufen Reiser zusammenraffte, und
legte sie aufs Feuer, kam eine Otter von der Hitze hervor und fuhr
Paulus an seine Hand. \bibverse{4} Da aber die Leutlein sahen das Tier
an seiner Hand hangen, sprachen sie untereinander: Dieser Mensch muss
ein Mörder sein, den die Rache nicht leben lässt, ob er gleich dem Meer
entgangen ist. \bibverse{5} Er aber schlenkerte das Tier ins Feuer, und
ihm widerfuhr nichts Übles. \bibverse{6} Sie aber warteten, wenn er
schwellen würde oder tot niederfallen. Da sie aber lange warteten und
sahen, dass ihm nichts Ungeheures widerfuhr, wurden sie anderes Sinnes
und sprachen, er wäre ein Gott. \footnote{\textbf{28:6} Apg 14,11}

\bibverse{7} An diesen Örtern aber hatte der Oberste der Insel, mit
Namen Publius, ein Vorwerk; der nahm uns auf und herbergte uns drei Tage
freundlich. \bibverse{8} Es geschah aber, dass der Vater des Publius am
Fieber und an der Ruhr lag. Zu dem ging Paulus hinein und betete und
legte die Hand auf ihn und machte ihn gesund. \bibverse{9} Da das
geschah, kamen auch die anderen auf der Insel herzu, die Krankheiten
hatten, und ließen sich gesund machen. \bibverse{10} Und sie taten uns
große Ehre; und da wir auszogen, luden sie auf, was uns not war.

\bibverse{11} Nach drei Monaten aber fuhren wir aus in einem Schiffe von
Alexandrien, welches bei der Insel überwintert hatte und hatte ein
Panier der Zwillinge. \bibverse{12} Und da wir gen Syrakus kamen,
blieben wir drei Tage da. \bibverse{13} Und da wir umschifften, kamen
wir gen Rhegion; und nach einem Tage, da der Südwind sich erhob, kamen
wir des anderen Tages gen Puteoli. \bibverse{14} Da fanden wir Brüder
und wurden von ihnen gebeten, dass wir sieben Tage dablieben. Und also
kamen wir gen Rom. \bibverse{15} Und von dort, da die Brüder von uns
hörten, gingen sie aus, uns entgegen, bis gen Appifor und Tretabern. Da
die Paulus sah, dankte er Gott und gewann eine Zuversicht. \bibverse{16}
Da wir aber gen Rom kamen, überantwortete der Unterhauptmann die
Gefangenen dem obersten Hauptmann. Aber Paulus ward erlaubt zu bleiben,
wo er wollte, mit einem Kriegsknechte, der ihn hütete.

\bibverse{17} Es geschah aber nach drei Tagen, dass Paulus zusammenrief
die Vornehmsten der Juden. Da sie zusammenkamen, sprach er zu ihnen: Ihr
Männer, liebe Brüder, ich habe nichts getan wider unser Volk noch wider
väterliche Sitten, und bin doch gefangen aus Jerusalem übergeben in der
Römer Hände. \footnote{\textbf{28:17} Apg 23,1} \bibverse{18} Diese, da
sie mich verhört hatten, wollten sie mich losgeben, dieweil keine
Ursache des Todes an mir war. \bibverse{19} Da aber die Juden dawider
redeten, ward ich genötigt, mich auf den Kaiser zu berufen; nicht, als
hätte ich mein Volk um etwas zu verklagen. \bibverse{20} Um der Ursache
willen habe ich euch gebeten, dass ich euch sehen und ansprechen möchte;
denn um der Hoffnung willen Israels bin ich mit dieser Kette umgeben.
\footnote{\textbf{28:20} Apg 26,6-7}

\bibverse{21} Sie aber sprachen zu ihm: Wir haben weder Schrift
empfangen aus Judäa deinethalben, noch ist ein Bruder gekommen, der von
dir etwas Arges verkündigt oder gesagt habe. \bibverse{22} Doch wollen
wir von dir hören, was du hältst; denn von dieser Sekte ist uns kund,
dass ihr wird an allen Enden widersprochen.

\bibverse{23} Und da sie ihm einen Tag bestimmt hatten, kamen viele zu
ihm in die Herberge, welchen er auslegte und bezeugte das Reich Gottes;
und er predigte ihnen von Jesus aus dem Gesetz Moses und aus den
Propheten von frühmorgens an bis an den Abend. \bibverse{24} Und etliche
fielen dem zu, was er sagte; etliche aber glaubten nicht. \bibverse{25}
Da sie aber untereinander misshellig waren, gingen sie weg, als Paulus
das eine Wort redete: Wohl hat der heilige Geist gesagt durch den
Propheten Jesaja zu unseren Vätern \bibverse{26} und gesprochen: „Gehe
hin zu diesem Volk und sprich: Mit den Ohren werdet ihr's hören, und
nicht verstehen; und mit den Augen werdet ihr's sehen, und nicht
erkennen. \bibverse{27} Denn das Herz dieses Volks ist verstockt, und
sie hören schwer mit den Ohren und schlummern mit ihren Augen, auf dass
sie nicht dermaleinst sehen und mit den Augen und hören mit den Ohren
und verständig werden im Herzen und sich bekehren, dass ich ihnen
hülfe.``

\bibverse{28} So sei es euch kundgetan, dass den Heiden gesandt ist dies
Heil Gottes; und sie werden's hören. -- \footnote{\textbf{28:28} Apg
  13,46}

\bibverse{29} Und da er solches redete, gingen die Juden hin und hatten
viel Fragens unter sich selbst.

\bibverse{30} Paulus aber blieb zwei Jahre in seinem eigenen Gedinge und
nahm auf alle, die zu ihm kamen, \bibverse{31} predigte das Reich Gottes
und lehrte von dem Herrn Jesus mit aller Freudigkeit unverboten.
