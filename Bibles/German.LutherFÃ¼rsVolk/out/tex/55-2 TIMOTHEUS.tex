\hypertarget{section}{%
\section{1}\label{section}}

\bibverse{1} Paulus, ein Apostel Jesu Christi durch den Willen Gottes
nach der Verheißung des Lebens in Christo Jesu, \bibverse{2} meinem
lieben Sohn Timotheus: Gnade, Barmherzigkeit, Friede von Gott, dem
Vater, und Christo Jesu, unserm HERRN! \bibverse{3} Ich danke Gott, dem
ich diene von meinen Voreltern her in reinem Gewissen, daß ich ohne
Unterlaß dein gedenke in meinem Gebet Tag und Nacht; \bibverse{4} und
mich verlangt, dich zu sehen, wenn ich denke an deine Tränen, auf daß
ich mit Freude erfüllt würde; \bibverse{5} und wenn ich mich erinnere
des ungefärbten Glaubens in dir, welcher zuvor gewohnt hat in deiner
Großmutter Lois und deiner Mutter Eunike; ich bin aber gewiß, auch in
dir. \bibverse{6} Um solcher Ursache willen erinnere ich dich, daß du
erweckest die Gabe Gottes, die in dir ist durch die Auflegung meiner
Hände. \bibverse{7} Denn Gott hat uns nicht gegeben den Geist der
Furcht, sondern der Kraft und der Liebe und der Zucht. \bibverse{8}
Darum so schäme dich nicht des Zeugnisses unsers HERRN noch meiner, der
ich sein Gebundener bin, sondern leide mit für das Evangelium wie ich,
nach der Kraft Gottes, \bibverse{9} der uns hat selig gemacht und
berufen mit einem heiligen Ruf, nicht nach unsern Werken, sondern nach
dem Vorsatz und der Gnade, die uns gegeben ist in Christo Jesu vor der
Zeit der Welt, \bibverse{10} jetzt aber offenbart durch die Erscheinung
unsers Heilandes Jesu Christi, der dem Tode die Macht hat genommen und
das Leben und ein unvergänglich Wesen ans Licht gebracht durch das
Evangelium, \bibverse{11} für welches ich gesetzt bin als Prediger und
Apostel der Heiden. \bibverse{12} Um dieser Ursache willen leide ich
auch solches; aber ich schäme mich dessen nicht; denn ich weiß, an wen
ich glaube, und bin gewiß, er kann mir bewahren, was mir beigelegt ist,
bis an jenen Tag. \bibverse{13} Halte an dem Vorbilde der heilsamen
Worte, die du von mir gehört hast, im Glauben und in der Liebe in
Christo Jesu. \bibverse{14} Dies beigelegte Gut bewahre durch den
heiligen Geist, der in uns wohnt. \bibverse{15} Das weißt du, daß sich
von mir gewandt haben alle, die in Asien sind, unter welchen ist
Phygellus und Hermogenes. \bibverse{16} Der HERR gebe Barmherzigkeit dem
Hause Onesiphorus; denn er hat mich oft erquickt und hat sich meiner
Kette nicht geschämt, \bibverse{17} sondern da er zu Rom war, suchte er
mich aufs fleißigste und fand mich. \bibverse{18} Der HERR gebe ihm, daß
er finde Barmherzigkeit bei dem HERRN an jenem Tage. Und wieviel er zu
Ephesus gedient hat, weißt du am besten.

\hypertarget{section-1}{%
\section{2}\label{section-1}}

\bibverse{1} So sei nun stark, mein Sohn, durch die Gnade in Christo
Jesu. \bibverse{2} Und was du von mir gehört hast durch viele Zeugen,
das befiehl treuen Menschen, die da tüchtig sind, auch andere zu lehren.
\bibverse{3} `4771' Leide mit als ein guter Streiter Jesu Christi.
\bibverse{4} Kein Kriegsmann flicht sich in Händel der Nahrung, auf daß
er gefalle dem, der ihn angenommen hat. \bibverse{5} Und so jemand auch
kämpft, wird er doch nicht gekrönt, er kämpfe denn recht. \bibverse{6}
Es soll aber der Ackermann, der den Acker baut, die Früchte am ersten
genießen. Merke, was ich sage! \bibverse{7} Der HERR aber wird dir in
allen Dingen Verstand geben. \bibverse{8} Halt im Gedächtnis Jesum
Christum, der auferstanden ist von den Toten, aus dem Samen Davids, nach
meinem Evangelium, \bibverse{9} für welches ich leide bis zu den Banden
wie ein Übeltäter; aber Gottes Wort ist nicht gebunden. \bibverse{10}
Darum erdulde ich alles um der Auserwählten willen, auf daß auch sie die
Seligkeit erlangen in Christo Jesu mit ewiger Herrlichkeit.
\bibverse{11} Das ist gewißlich wahr: Sterben wir mit, so werden wir
mitleben; \bibverse{12} dulden wir, so werden wir mitherrschen;
verleugnen wir, so wird er uns auch verleugnen; \bibverse{13} glauben
wir nicht, so bleibt er treu; er kann sich selbst nicht verleugnen.
\bibverse{14} Solches erinnere sie und bezeuge vor dem HERRN, daß sie
nicht um Worte zanken, welches nichts nütze ist denn zu verkehren, die
da zuhören. \bibverse{15} Befleißige dich, Gott dich zu erzeigen als
einen rechtschaffenen und unsträflichen Arbeiter, der da recht teile das
Wort der Wahrheit. \bibverse{16} Des ungeistlichen, losen Geschwätzes
entschlage dich; denn es hilft viel zum ungöttlichen Wesen,
\bibverse{17} und ihr Wort frißt um sich wie der Krebs; unter welchen
ist Hymenäus und Philetus, \bibverse{18} welche von der Wahrheit
irregegangen sind und sagen, die Auferstehung sei schon geschehen, und
haben etlicher Glauben verkehrt. \bibverse{19} Aber der feste Grund
Gottes besteht und hat dieses Siegel: Der HERR kennt die seinen; und: Es
trete ab von Ungerechtigkeit, wer den Namen Christi nennt. \bibverse{20}
In einem großen Hause aber sind nicht allein goldene und silberne
Gefäße, sondern auch hölzerne und irdene, und etliche zu Ehren, etliche
aber zu Unehren. \bibverse{21} So nun jemand sich reinigt von solchen
Leuten, der wird ein geheiligtes Gefäß sein zu Ehren, dem Hausherrn
bräuchlich und zu allem guten Werk bereitet. \bibverse{22} Fliehe die
Lüste der Jugend; jage aber nach der Gerechtigkeit, dem Glauben, der
Liebe, dem Frieden mit allen, die den HERRN anrufen von reinem Herzen.
\bibverse{23} Aber der törichten und unnützen Fragen entschlage dich;
denn du weißt, daß sie nur Zank gebären. \bibverse{24} Ein Knecht aber
des HERRN soll nicht zänkisch sein, sondern freundlich gegen jedermann,
lehrhaft, der die Bösen tragen kann \bibverse{25} und mit Sanftmut
strafe die Widerspenstigen, ob ihnen Gott dermaleinst Buße gebe, die
Wahrheit zu erkennen, \bibverse{26} und sie wieder nüchtern würden aus
des Teufels Strick, von dem sie gefangen sind zu seinem Willen.

\hypertarget{section-2}{%
\section{3}\label{section-2}}

\bibverse{1} Das sollst du aber wissen, daß in den letzten Tagen werden
greuliche Zeiten kommen. \bibverse{2} Denn es werden Menschen sein, die
viel von sich halten, geizig, ruhmredig, hoffärtig, Lästerer, den Eltern
ungehorsam, undankbar, ungeistlich, \bibverse{3} lieblos, unversöhnlich,
Verleumder, unkeusch, wild, ungütig, \bibverse{4} Verräter, Frevler,
aufgeblasen, die mehr lieben Wollust denn Gott, \bibverse{5} die da
haben den Schein eines gottseligen Wesens, aber seine Kraft verleugnen
sie; und solche meide. \bibverse{6} Aus denselben sind, die hin und her
in die Häuser schleichen und führen die Weiblein gefangen, die mit
Sünden beladen sind und von mancherlei Lüsten umgetrieben, \bibverse{7}
lernen immerdar, und können nimmer zur Erkenntnis kommen. \bibverse{8}
Gleicherweise aber, wie Jannes und Jambres dem Mose widerstanden, also
widerstehen auch diese der Wahrheit; es sind Menschen von zerrütteten
Sinnen, untüchtig zum Glauben. \bibverse{9} Aber sie werden's in die
Länge nicht treiben; denn ihre Torheit wird offenbar werden jedermann,
gleichwie auch jener Torheit offenbar ward. \bibverse{10} Du aber bist
nachgefolgt meiner Lehre, meiner Weise, meiner Meinung, meinem Glauben,
meiner Langmut, meiner Liebe, meiner Geduld, \bibverse{11} meinen
Verfolgungen, meinen Leiden, welche mir widerfahren sind zu Antiochien,
zu Ikonien, zu Lystra. Welche Verfolgungen ich da ertrug! Und aus allen
hat mich der HERR erlöst. \bibverse{12} Und alle, die gottselig leben
wollen in Christo Jesu, müssen Verfolgung leiden. \bibverse{13} Mit den
bösen Menschen aber und verführerischen wird's je länger, je ärger: sie
verführen und werden verführt. \bibverse{14} Du aber bleibe in dem, was
du gelernt hast und dir vertrauet ist, sintemal du weißt, von wem du
gelernt hast. \bibverse{15} Und weil du von Kind auf die heilige Schrift
weißt, kann dich dieselbe unterweisen zur Seligkeit durch den Glauben an
Christum Jesum. \bibverse{16} Denn alle Schrift, von Gott eingegeben,
ist nütze zur Lehre, zur Strafe, zur Besserung, zur Züchtigung in der
Gerechtigkeit, \bibverse{17} daß ein Mensch Gottes sei vollkommen, zu
allem guten Werk geschickt.

\hypertarget{section-3}{%
\section{4}\label{section-3}}

\bibverse{1} So bezeuge ich nun vor Gott und dem HERRN Jesus Christus,
der da zukünftig ist, zu richten die Lebendigen und die Toten mit seiner
Erscheinung und mit seinem Reich: \bibverse{2} Predige das Wort, halte
an, es sei zu rechter Zeit oder zur Unzeit; strafe, drohe, ermahne mit
aller Geduld und Lehre. \bibverse{3} Denn es wird eine Zeit sein, da sie
die heilsame Lehre nicht leiden werden; sondern nach ihren eigenen
Lüsten werden sie sich selbst Lehrer aufladen, nach dem ihnen die Ohren
jucken, \bibverse{4} und werden die Ohren von der Wahrheit wenden und
sich zu Fabeln kehren. \bibverse{5} Du aber sei nüchtern allenthalben,
sei willig, zu leiden, tue das Werk eines evangelischen Predigers,
richte dein Amt redlich aus. \bibverse{6} Denn ich werde schon geopfert,
und die Zeit meines Abscheidens ist vorhanden. \bibverse{7} Ich habe
einen guten Kampf gekämpft, ich habe den Lauf vollendet, ich habe
Glauben gehalten; \bibverse{8} hinfort ist mir beigelegt die Krone der
Gerechtigkeit, welche mir der HERR an jenem Tage, der gerechte Richter,
geben wird, nicht aber mir allein, sondern auch allen, die seine
Erscheinung liebhaben. \bibverse{9} Befleißige dich, daß du bald zu mir
kommst. \bibverse{10} Denn Demas hat mich verlassen und hat diese Welt
liebgewonnen und ist gen Thessalonich gezogen, Kreszens nach Galatien,
Titus nach Dalmatien. \bibverse{11} Lukas allein ist bei mir. Markus
nimm zu dir und bringe ihn mit dir; denn er ist mir nützlich zum Dienst.
\bibverse{12} Tychikus habe ich gen Ephesus gesandt. \bibverse{13} Den
Mantel, den ich zu Troas ließ bei Karpus, bringe mit, wenn du kommst,
und die Bücher, sonderlich die Pergamente. \bibverse{14} Alexander, der
Schmied, hat mir viel Böses bewiesen; der HERR bezahle ihm nach seinen
Werken. \bibverse{15} Vor dem hüte du dich auch; denn er hat unsern
Worten sehr widerstanden. \bibverse{16} In meiner ersten Verantwortung
stand mir niemand bei, sondern sie verließen mich alle. Es sei ihnen
nicht zugerechnet. \bibverse{17} Der HERR aber stand mir bei und stärkte
mich, auf daß durch mich die Predigt bestätigt würde und alle Heiden sie
hörten; und ich ward erlöst von des Löwen Rachen. \bibverse{18} Der HERR
aber wird mich erlösen von allem Übel und mir aushelfen zu seinem
himmlischen Reich; welchem sei Ehre von Ewigkeit zu Ewigkeit! Amen.
\bibverse{19} Grüße Priska und Aquila und das Haus des Onesiphorus.
\bibverse{20} Erastus blieb zu Korinth; Trophimus aber ließ ich zu Milet
krank. \bibverse{21} Tue Fleiß, daß du vor dem Winter kommst. Es grüßt
dich Eubulus und Pudens und Linus und Klaudia und alle Brüder.
\bibverse{22} Der HERR Jesus Christus sei mit deinem Geiste! Die Gnade
sei mit euch! Amen.
