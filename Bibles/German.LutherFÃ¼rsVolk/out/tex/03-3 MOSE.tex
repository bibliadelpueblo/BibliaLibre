\hypertarget{section}{%
\section{1}\label{section}}

\bibverse{1} Und der HERR rief Mose und redete mit ihm aus der Hütte des
Stifts und sprach: \bibverse{2} Rede mit den Kindern Israel und sprich
zu ihnen: Welcher unter euch dem HERRN ein Opfer tun will, der tue es
von dem Vieh, von Rindern und Schafen. \bibverse{3} Will er ein
Brandopfer tun von Rindern, so opfere er ein Männlein, das ohne Fehl
sei, vor der Tür der Hütte des Stifts, daß es dem HERRN angenehm sei von
ihm, \bibverse{4} und lege seine Hand auf des Brandopfers Haupt, so wird
es angenehm sein und ihn versöhnen. \bibverse{5} Und er soll das junge
Rind schlachten vor dem HERRN; und die Priester, Aarons Söhne, sollen
das Blut herzubringen und auf den Altar umhersprengen, der vor der Tür
der Hütte des Stifts ist. \bibverse{6} Und man soll dem Brandopfer die
Haut abziehen, und es soll in Stücke zerhauen werden; \bibverse{7} und
die Söhne Aarons, des Priesters, sollen ein Feuer auf dem Altar machen
und Holz obendarauf legen \bibverse{8} und sollen die Stücke, den Kopf
und das Fett auf das Holz legen, das auf dem Feuer auf dem Altar liegt.
\bibverse{9} Das Eingeweide aber und die Schenkel soll man mit Wasser
waschen, und der Priester soll das alles anzünden auf dem Altar zum
Brandopfer. Das ist ein Feuer zum süßen Geruch dem HERRN. \bibverse{10}
Will er aber von Schafen oder Ziegen ein Brandopfer tun, so opfere er
ein Männlein, das ohne Fehl sei. \bibverse{11} Und soll es schlachten
zur Seite des Altars gegen Mitternacht vor dem HERRN. Und die Priester,
Aarons Söhne, sollen sein Blut auf den Altar umhersprengen.
\bibverse{12} Und man soll es in Stücke zerhauen, und der Priester soll
sie samt dem Kopf und dem Fett auf das Holz und Feuer, das auf dem Altar
ist, legen. \bibverse{13} Aber das Eingeweide und die Schenkel soll man
mit Wasser waschen, und der Priester soll es alles opfern und anzünden
auf dem Altar zum Brandopfer. Das ist ein Feuer zum süßen Geruch dem
HERRN. \bibverse{14} Will er aber von Vögeln dem HERRN ein Brandopfer
tun, so tue er's von Turteltauben oder von jungen Tauben. \bibverse{15}
Und der Priester soll's zum Altar bringen und ihm den Kopf abkneipen,
daß es auf dem Altar angezündet werde, und sein Blut ausbluten lassen an
der Wand des Altars. \bibverse{16} Und seinen Kropf mit seinen Federn
soll man neben den Altar gegen Morgen auf den Aschenhaufen werfen;
\bibverse{17} und soll seine Flügel spalten, aber nicht abbrechen. Und
also soll's der Priester auf dem Altar anzünden, auf dem Holz, auf dem
Feuer zum Brandopfer. Das ist ein Feuer zum süßen Geruch dem HERRN.

\hypertarget{section-1}{%
\section{2}\label{section-1}}

\bibverse{1} Wenn eine Seele dem HERRN ein Speisopfer tun will, so soll
es von Semmelmehl sein, und sie sollen Öl darauf gießen und Weihrauch
darauf legen \bibverse{2} und es also bringen zu den Priestern, Aarons
Söhnen. Da soll der Priester seine Hand voll nehmen von dem Semmelmehl
und Öl, samt dem ganzen Weihrauch und es anzünden zum Gedächtnis auf dem
Altar. Das ist ein Feuer zum süßen Geruch dem HERRN. \bibverse{3} Das
übrige aber vom Speisopfer soll Aarons und seiner Söhne sein. Das soll
ein Hochheiliges sein von den Feuern des HERRN. \bibverse{4} Will er
aber sein Speisopfer tun vom Gebackenen im Ofen, so nehme er Kuchen von
Semmelmehl, ungesäuert, mit Öl gemengt, oder ungesäuerte Fladen, mit Öl
bestrichen. \bibverse{5} Ist aber dein Speisopfer etwas vom Gebackenen
in der Pfanne, so soll's von ungesäuertem Semmelmehl mit Öl gemengt
sein; \bibverse{6} und sollst's in Stücke zerteilen und Öl darauf
gießen, so ist's ein Speisopfer. \bibverse{7} Ist aber dein Speisopfer
etwas auf dem Rost Geröstetes, so sollst du es von Semmelmehl mit Öl
machen \bibverse{8} und sollst das Speisopfer, das du von solcherlei
machen willst dem HERRN, zu dem Priester bringen; der soll es zu dem
Altar bringen \bibverse{9} und des Speisopfers einen Teil abzuheben zum
Gedächtnis und anzünden auf dem Altar. Das ist ein Feuer zum süßen
Geruch dem HERRN. \bibverse{10} Das übrige aber soll Aarons und seiner
Söhne sein. Das soll ein Hochheiliges sein von den Feuern des HERRN.
\bibverse{11} Alle Speisopfer, die ihr dem HERRN opfern wollt, sollt ihr
ohne Sauerteig machen; denn kein Sauerteig noch Honig soll dem HERRN zum
Feuer angezündet werden. \bibverse{12} Unter den Erstlingen sollt ihr
sie dem HERRN bringen; aber auf den Altar sollen sie nicht kommen zum
süßen Geruch. \bibverse{13} Alle deine Speisopfer sollst du salzen, und
dein Speisopfer soll nimmer ohne Salz des Bundes deines Gottes sein;
denn in allem deinem Opfer sollst du Salz opfern. \bibverse{14} Willst
du aber ein Speisopfer dem HERRN tun von den ersten Früchten, so sollst
du Ähren, am Feuer gedörrt, klein zerstoßen und also das Speisopfer
deiner ersten Früchte opfern; \bibverse{15} und sollst Öl darauf tun und
Weihrauch darauf legen, so ist's ein Speisopfer. \bibverse{16} Und der
Priester soll einen Teil von dem Zerstoßenen und vom Öl mit dem ganzen
Weihrauch anzünden zum Gedächtnis. Das ist ein Feuer dem HERRN.

\hypertarget{section-2}{%
\section{3}\label{section-2}}

\bibverse{1} Ist aber sein Opfer ein Dankopfer von Rindern, es sei ein
Ochse oder eine Kuh, soll er eins opfern vor dem HERRN, das ohne Fehl
sei. \bibverse{2} Und soll seine Hand auf desselben Haupt legen und es
schlachten vor der Tür der Hütte des Stifts. Und die Priester, Aarons
Söhne, sollen das Blut auf den Altar umhersprengen. \bibverse{3} Und er
soll von dem Dankopfer dem HERRN opfern, nämlich das Fett, welches die
Eingeweide bedeckt, und alles Fett am Eingeweide \bibverse{4} und die
zwei Nieren mit dem Fett, das daran ist, an den Lenden, und das Netz um
die Leber, an den Nieren abgerissen. \bibverse{5} Und Aarons Söhne
sollen's anzünden auf dem Altar zum Brandopfer, auf dem Holz, das auf
dem Feuer liegt. Das ist ein Feuer zum süßen Geruch dem HERRN.
\bibverse{6} Will er aber dem HERRN ein Dankopfer von kleinem Vieh tun,
es sei ein Widder oder Schaf, so soll's ohne Fehl sein. \bibverse{7}
Ist's ein Lämmlein, soll er's vor den HERRN bringen \bibverse{8} und
soll seine Hand auf desselben Haupt legen und es schlachten vor der
Hütte des Stifts. Und die Söhne Aarons sollen sein Blut auf dem Altar
umhersprengen. \bibverse{9} Und er soll also von dem Dankopfer dem HERRN
opfern zum Feuer, nämlich sein Fett, den ganzen Schwanz, von dem Rücken
abgerissen, dazu das Fett, welches das Eingeweide bedeckt, und alles
Fett am Eingeweide, \bibverse{10} die zwei Nieren mit dem Fett, das
daran ist, an den Lenden, und das Netz um die Leber, an den Nieren
abgerissen. \bibverse{11} Und der Priester soll es anzünden auf dem
Altar zur Speise des Feuers dem HERRN. \bibverse{12} Ist aber sein Opfer
eine Ziege und er bringt es vor den HERRN, \bibverse{13} soll er seine
Hand auf ihr Haupt legen und sie schlachten vor der Hütte des Stifts.
Und die Söhne Aarons sollen das Blut auf dem Altar umhersprengen,
\bibverse{14} und er soll davon opfern ein Opfer dem HERRN, nämlich das
Fett, welches die Eingeweide bedeckt, und alles Fett am Eingeweide,
\bibverse{15} die zwei Nieren mit dem Fett, das daran ist, an den
Lenden, und das Netz über der Leber, an den Nieren abgerissen.
\bibverse{16} Und der Priester soll's anzünden auf dem Altar zur Speise
des Feuers zum süßen Geruch. Alles Fett ist des HERRN. \bibverse{17} Das
sei eine ewige Sitte bei euren Nachkommen in allen Wohnungen, daß ihr
kein Fett noch Blut esset.

\hypertarget{section-3}{%
\section{4}\label{section-3}}

\bibverse{1} Und der HERR redete mit Mose und sprach: \bibverse{2} Rede
mit den Kindern Israel und sprich: Wenn eine Seele sündigen würde aus
Versehen an irgend einem Gebot des HERRN und täte, was sie nicht tun
sollte: \bibverse{3} nämlich so der Priester, der gesalbt ist, sündigen
würde, daß er eine Schuld auf das Volk brächte, der soll für seine
Sünde, die er getan hat, einen jungen Farren bringen, der ohne Fehl sei,
dem HERRN zum Sündopfer. \bibverse{4} Und soll den Farren vor die Tür
der Hütte des Stifts bringen vor den HERRN und seine Hand auf desselben
Haupt legen und ihn schlachten vor dem HERRN. \bibverse{5} Und der
Priester, der gesalbt ist, soll von des Farren Blut nehmen und es in die
Hütte des Stifts bringen \bibverse{6} und soll seinen Finger in das Blut
tauchen und damit siebenmal sprengen vor dem HERRN, vor dem Vorhang im
Heiligen. \bibverse{7} Und soll von dem Blut tun auf die Hörner des
Räucheraltars, der vor dem HERRN in der Hütte des Stifts steht, und
alles übrige Blut gießen an den Boden des Brandopferaltars, der vor der
Tür der Hütte des Stifts steht. \bibverse{8} Und alles Fett des
Sündopfers soll er abheben, nämlich das Fett, welches das Eingeweide
bedeckt, und alles Fett am Eingeweide, \bibverse{9} die zwei Nieren mit
dem Fett, das daran ist, an den Lenden, und das Netz über der Leber, an
den Nieren abgerissen, \bibverse{10} gleichwie er's abhebt vom Ochsen im
Dankopfer; und soll es anzünden auf dem Brandopferaltar. \bibverse{11}
Aber das Fell des Farren mit allem Fleisch samt Kopf und Schenkeln und
das Eingeweide und den Mist, \bibverse{12} das soll er alles
hinausführen aus dem Lager an eine reine Stätte, da man die Asche hin
schüttet, und soll's verbrennen auf dem Holz mit Feuer. \bibverse{13}
Wenn die ganze Gemeinde Israel etwas versehen würde und die Tat vor
ihren Augen verborgen wäre, daß sie wider irgend ein Gebot des HERRN
getan hätten, was sie nicht tun sollten, und also sich verschuldeten,
\bibverse{14} und darnach ihrer Sünde innewürden, die sie getan hätten,
sollen sie einen jungen Farren darbringen zum Sündopfer und vor die Tür
der Hütte des Stifts stellen. \bibverse{15} Und die Ältesten von der
Gemeinde sollen ihre Hände auf sein Haupt legen vor dem HERRN und den
Farren schlachten vor dem HERRN. \bibverse{16} Und der Priester, der
gesalbt ist, soll Blut vom Farren in die Hütte des Stifts bringen
\bibverse{17} und mit seinem Finger siebenmal sprengen vor dem HERRN vor
dem Vorhang. \bibverse{18} Und soll von dem Blut auf die Hörner des
Altars tun, der vor dem HERRN steht in der Hütte des Stifts, und alles
andere Blut an den Boden des Brandopferaltars gießen, der vor der Tür
der Hütte des Stifts steht. \bibverse{19} Alles sein Fett aber soll er
abheben und auf dem Altar anzünden. \bibverse{20} Und soll mit dem
Farren tun, wie er mit dem Farren des Sündopfers getan hat. Und soll
also der Priester sie versöhnen, so wird's ihnen vergeben. \bibverse{21}
Und soll den Farren hinaus vor das Lager tragen und verbrennen, wie er
den vorigen Farren verbrannt hat. Das soll das Sündopfer der Gemeinde
sein. \bibverse{22} Wenn aber ein Fürst sündigt und irgend etwas wider
des HERRN, seines Gottes, Gebote tut, was er nicht tun sollte, und
versieht etwas, daß er verschuldet, \bibverse{23} und er wird seiner
Sünde inne, die er getan hat, der soll zum Opfer bringen einen
Ziegenbock ohne Fehl, \bibverse{24} und seine Hand auf des Bockes Haupt
legen und ihn schlachten an der Stätte, da man die Brandopfer schlachtet
vor dem HERRN. Das sei sein Sündopfer. \bibverse{25} Da soll denn der
Priester von dem Blut des Sündopfers nehmen mit seinem Finger und es auf
die Hörner des Brandopferaltars tun und das andere Blut an den Boden des
Brandopferaltars gießen. \bibverse{26} Aber alles sein Fett soll er auf
dem Altar anzünden gleich wie das Fett des Dankopfers. Und soll also der
Priester seine Sünde versöhnen, so wird's ihm vergeben. \bibverse{27}
Wenn aber eine Seele vom gemeinen Volk etwas versieht und sündigt, daß
sie wider irgend eines der Gebote des HERRN tut, was sie nicht tun
sollte, und sich also verschuldet, \bibverse{28} und ihrer Sünde
innewird, die sie getan hat, die soll zum Opfer eine Ziege bringen ohne
Fehl für die Sünde, die sie getan hat, \bibverse{29} und soll ihre Hand
auf des Sündopfers Haupt legen und es schlachten an der Stätte des
Brandopfers. \bibverse{30} Und der Priester soll von dem Blut mit seinem
Finger nehmen und auf die Hörner des Altars des Brandopfers tun und
alles andere Blut an des Altars Boden gießen. \bibverse{31} All sein
Fett aber soll er abreißen, wie er das Fett des Dankopfers abgerissen
hat, und soll's anzünden auf dem Altar zum süßen Geruch dem HERRN. Und
soll also der Priester sie versöhnen, so wird's ihr vergeben.
\bibverse{32} Wird er aber ein Schaf zum Sündopfer bringen, so bringe er
ein weibliches, das ohne Fehl ist, \bibverse{33} und lege seine Hand auf
des Sündopfers Haupt und schlachte es zum Sündopfer an der Stätte, da
man die Brandopfer schlachtet. \bibverse{34} Und der Priester soll von
dem Blut mit seinem Finger nehmen und auf die Hörner des
Brandopferaltars tun und alles andere Blut an den Boden des Altars
gießen. \bibverse{35} Aber all sein Fett soll er abreißen, wie er das
Fett vom Schaf des Dankopfers abgerissen hat, und soll's auf dem Altar
anzünden zum Feuer dem HERRN. Und soll also der Priester versöhnen seine
Sünde, die er getan hat, so wird's ihm vergeben.

\hypertarget{section-4}{%
\section{5}\label{section-4}}

\bibverse{1} Wenn jemand also sündigen würde, daß er den Fluch
aussprechen hört und Zeuge ist, weil er's gesehen oder erfahren hat, es
aber nicht ansagt, der ist einer Missetat schuldig. \bibverse{2} Oder
wenn jemand etwas Unreines anrührt, es sei ein Aas eines unreinen Tieres
oder Viehs oder Gewürms, und wüßte es nicht, der ist unrein und hat sich
verschuldet. \bibverse{3} Oder wenn er einen unreinen Menschen anrührt,
in was für Unreinigkeit der Mensch unrein werden kann, und wüßte es
nicht und wird's inne, der hat sich verschuldet. \bibverse{4} Oder wenn
jemand schwört, daß ihm aus dem Mund entfährt, Schaden oder Gutes zu tun
(wie denn einem Menschen ein Schwur entfahren mag, ehe er's bedächte),
und wird's inne, der hat sich an der einem verschuldet. \bibverse{5}
Wenn's nun geschieht, daß er sich an einem verschuldet und bekennt, daß
er daran gesündigt hat, \bibverse{6} so soll er für seine Schuld dieser
seiner Sünde, die er getan hat, dem HERRN bringen von der Herde eine
Schaf-oder Ziegenmutter zum Sündopfer, so soll ihm der Priester seine
Sünden versöhnen. \bibverse{7} Vermag er aber nicht ein Schaf, so bringe
er dem HERRN für seine Schuld, die er getan hat, zwei Turteltauben oder
zwei junge Tauben, die erste zum Sündopfer, die andere zum Brandopfer,
\bibverse{8} und bringe sie dem Priester. Der soll die erste zum
Sündopfer machen, und ihr den Kopf abkneipen hinter dem Genick, und
nicht abbrechen; \bibverse{9} und sprenge mit dem Blut des Sündopfers an
die Seite des Altars, und lasse das übrige Blut ausbluten an des Altars
Boden. Das ist das Sündopfer, \bibverse{10} Die andere aber soll er zum
Brandopfer machen, so wie es recht ist. Und soll also der Priester ihm
seine Sünde versöhnen, die er getan hat, so wird's ihm vergeben.
\bibverse{11} Vermag er aber nicht zwei Turteltauben oder zwei junge
Tauben, so bringe er für seine Sünde als ein Opfer ein zehntel Epha
Semmelmehl zum Sündopfer. Er soll aber kein Öl darauf legen noch
Weihrauch darauf tun; denn es ist ein Sündopfer. \bibverse{12} Und
soll's zum Priester bringen. Der Priester aber soll eine Handvoll davon
nehmen zum Gedächtnis und anzünden auf dem Altar zum Feuer dem HERRN.
Das ist ein Sündopfer. \bibverse{13} Und der Priester soll also seine
Sünde, die er getan hat, ihm versöhnen, so wird's ihm vergeben. Und es
soll dem Priester gehören wie ein Speisopfer. \bibverse{14} Und der HERR
redete mit Mose und sprach: \bibverse{15} Wenn sich jemand vergreift,
daß er es versieht und sich versündigt an dem, das dem HERRN geweiht
ist, soll er ein Schuldopfer dem HERRN bringen, einen Widder ohne Fehl
von der Herde, der zwei Silberlinge wert sei nach dem Lot des
Heiligtums, zum Schuldopfer. \bibverse{16} Dazu was er gesündigt hat an
dem Geweihten, soll er wiedergeben und den fünften Teil darüber geben,
und soll's dem Priester geben; der soll ihn versöhnen mit dem Widder des
Schuldopfers, so wird's ihm vergeben. \bibverse{17} Wenn jemand sündigt
und tut wider irgend ein Gebot des HERRN, was er nicht tun sollte, und
hat's nicht gewußt, der hat sich verschuldet und ist einer Missetat
schuldig \bibverse{18} und soll bringen einen Widder von der Herde ohne
Fehl, der eines Schuldopfers wert ist, zum Priester; der soll ihm
versöhnen, was er versehen hat und wußte es nicht, so wird's ihm
vergeben. \bibverse{19} Das ist das Schuldopfer; verschuldet hat er sich
an dem HERRN.

\hypertarget{section-5}{%
\section{6}\label{section-5}}

\bibverse{1} {[}5:20{]} Und der HERR redete mit Mose und sprach:
\bibverse{2} {[}5:21{]} Wenn jemand sündigen würde und sich damit an dem
Herrn vergreifen, daß er seinem Nebenmenschen ableugnet, was ihm dieser
befohlen hat, oder was ihm zu treuer Hand getan ist, oder was er sich
mit Gewalt genommen oder mit Unrecht an sich gebracht, \bibverse{3}
{[}5:22{]} oder wenn er, was verloren ist, gefunden hat, und leugnet
solches und tut einen falschen Eid über irgend etwas, darin ein Mensch
wider seinen Nächsten Sünde tut; \bibverse{4} {[}5:23{]} wenn's nun
geschieht, daß er also sündigt und sich verschuldet, so soll er
wiedergeben, was er mit Gewalt genommen oder mit Unrecht an sich
gebracht, oder was ihm befohlen ist, oder was er gefunden hat,
\bibverse{5} {[}5:24{]} oder worüber er den falschen Eid getan hat; das
soll er alles ganz wiedergeben, dazu den fünften Teil darüber geben dem,
des es gewesen ist, des Tages, wenn er sein Schuldopfer gibt.
\bibverse{6} {[}5:25{]} Aber für seine Schuld soll er dem HERRN zu dem
Priester einen Widder von der Herde ohne Fehl bringen, der eines
Schuldopfers wert ist. \bibverse{7} {[}5:26{]} So soll ihn der Priester
versöhnen vor dem HERRN, so wird ihm vergeben alles, was er getan hat,
darum er sich verschuldet hat. \bibverse{8} {[}6:1{]} Und der HERR
redete mit Mose und sprach: \bibverse{9} {[}6:2{]} Gebiete Aaron und
seinen Söhnen und sprich: Dies ist das Gesetz des Brandopfers. Das
Brandopfer soll brennen auf dem Herd des Altars die ganze Nacht bis an
den Morgen, und es soll des Altars Feuer brennend darauf erhalten
werden. \bibverse{10} {[}6:3{]} Und der Priester soll seinen leinenen
Rock anziehen und die leinenen Beinkleider an seinen Leib und soll die
Asche aufheben, die das Feuer auf dem Altar gemacht hat, und soll sie
neben den Altar schütten. \bibverse{11} {[}6:4{]} und soll seine Kleider
darnach ausziehen und andere Kleider anziehen und die Asche hinaustragen
aus dem Lager an eine reine Stätte. \bibverse{12} {[}6:5{]} Das Feuer
auf dem Altar soll brennen und nimmer verlöschen; der Priester soll alle
Morgen Holz darauf anzünden und obendarauf das Brandopfer zurichten und
das Fett der Dankopfer darauf anzünden. \bibverse{13} {[}6:6{]} Ewig
soll das Feuer auf dem Altar brennen und nimmer verlöschen.
\bibverse{14} {[}6:7{]} Und das ist das Gesetz des Speisopfers, das
Aarons Söhne opfern sollen vor dem HERRN auf dem Altar. \bibverse{15}
{[}6:8{]} Es soll einer abheben eine Handvoll Semmelmehl vom Speisopfer
und vom Öl und den ganzen Weihrauch, der auf dem Speisopfer liegt, und
soll's anzünden auf dem Altar zum süßen Geruch, ein Gedächtnis dem
HERRN. \bibverse{16} {[}6:9{]} Das übrige aber sollen Aaron und seine
Söhne verzehren und sollen's ungesäuert essen an heiliger Stätte, im
Vorhof der Hütte des Stifts. \bibverse{17} {[}6:10{]} Sie sollen's nicht
mit Sauerteig backen; denn es ist ihr Teil, den ich ihnen gegeben habe
von meinem Opfer. Es soll ihnen ein Hochheiliges sein gleichwie das
Sündopfer und Schuldopfer. \bibverse{18} {[}6:11{]} Was männlich ist
unter den Kindern Aarons, die sollen's essen. Das sei ein ewiges Recht
euren Nachkommen an den Opfern des HERRN: es soll sie niemand anrühren,
er sei den geweiht. \bibverse{19} {[}6:12{]} Und der HERR redete mit
Mose und sprach: \bibverse{20} {[}6:13{]} Das soll das Opfer sein Aarons
und seiner Söhne, das sie dem HERRN opfern sollen am Tage der Salbung:
ein zehntel Epha Semmelmehl als tägliches Speisopfer, eine Hälfte des
Morgens, die andere Hälfte des Abends. \bibverse{21} {[}6:14{]} In der
Pfanne mit Öl sollst du es machen und geröstet darbringen; und in
Stücken gebacken sollst du solches opfern zum süßen Geruch dem HERRN.
\bibverse{22} {[}6:15{]} Und der Priester, der unter seinen Söhnen an
seiner Statt gesalbt wird, soll solches tun; das ist ein ewiges Recht.
Es soll dem HERRN ganz verbrannt werden; \bibverse{23} {[}6:16{]} denn
alles Speisopfer eines Priesters soll ganz verbrannt und nicht gegessen
werden. \bibverse{24} {[}6:17{]} Und der HERR redete mit Mose und
sprach: \bibverse{25} {[}6:18{]} Sage Aaron und seinen Söhnen und
sprich: Dies ist das Gesetz des Sündopfers. An der Stätte, da du das
Brandopfer schlachtest, sollst du auch das Sündopfer schlachten vor dem
HERRN; das ist ein Hochheiliges. \bibverse{26} {[}6:19{]} Der Priester,
der das Sündopfer tut, soll's essen an heiliger Stätte, im Vorhof der
Hütte des Stifts. \bibverse{27} {[}6:20{]} Niemand soll sein Fleisch
anrühren, er sei denn geweiht. Und wer von seinem Blut ein Kleid
besprengt, der soll das besprengte Stück waschen an heiliger Stätte.
\bibverse{28} {[}6:21{]} Und den Topf, darin es gekocht ist, soll man
zerbrechen. Ist's aber ein eherner Topf, so soll man ihn scheuern und
mit Wasser spülen. \bibverse{29} {[}6:22{]} Was männlich ist unter den
Priestern, die sollen davon essen; denn es ist ein Hochheiliges.
\bibverse{30} {[}6:23{]} Aber all das Sündopfer, des Blut in die Hütte
des Stifts gebracht wird, zu versöhnen im Heiligen, soll man nicht
essen, sondern mit Feuer verbrennen.

\hypertarget{section-6}{%
\section{7}\label{section-6}}

\bibverse{1} Und dies ist das Gesetz des Schuldopfers. Ein Hochheiliges
ist es. \bibverse{2} An der Stätte, da man das Brandopfer schlachtet,
soll man auch das Schuldopfer schlachten und sein Blut auf dem Altar
umhersprengen. \bibverse{3} Und all sein Fett soll man opfern, den
Schwanz und das Fett, welches das Eingeweide bedeckt, \bibverse{4} die
zwei Nieren mit dem Fett, das daran ist, an den Lenden, und das Netz
über der Leber, an den Nieren abgerissen. \bibverse{5} Und der Priester
soll's auf dem Altar anzünden zum Feuer dem HERRN. Das ist ein
Schuldopfer. \bibverse{6} Was männlich ist unter den Priestern, die
sollen das essen an heiliger Stätte; denn es ist ein Hochheiliges.
\bibverse{7} Wie das Sündopfer, also soll auch das Schuldopfer sein;
aller beider soll einerlei Gesetz sein; und sollen dem Priester gehören,
der dadurch versöhnt. \bibverse{8} Welcher Priester jemandes Brandopfer
opfert, des soll des Brandopfers Fell sein, das er geopfert hat.
\bibverse{9} Und alles Speisopfer, das im Ofen oder auf dem Rost oder in
der Pfanne gebacken ist, soll dem Priester gehören, der es opfert.
\bibverse{10} Und alles Speisopfer, das mit Öl gemengt oder trocken ist,
soll aller Kinder Aarons sein, eines wie des andern. \bibverse{11} Und
dies ist das Gesetz des Dankopfers, das man dem HERRN opfert.
\bibverse{12} Wollen sie ein Lobopfer tun, so sollen sie ungesäuerte
Kuchen opfern, mit Öl gemengt, oder ungesäuerte Fladen, mit Öl
bestrichen, oder geröstete Semmelkuchen, mit Öl gemengt. \bibverse{13}
Sie sollen aber solches Opfer tun auf Kuchen von gesäuerten Brot mit
ihrem Lob-`02077' und Dankopfer, \bibverse{14} und sollen einen von den
allen dem HERRN zur Hebe opfern, und es soll dem Priester gehören, der
das Blut des Dankopfers sprengt. \bibverse{15} Und das Fleisch ihres
Lob-`02077' und Dankopfers soll desselben Tages gegessen werden, da es
geopfert ist, und nichts übriggelassen werden bis an den Morgen.
\bibverse{16} Ist es aber ein Gelübde oder freiwilliges Opfer, so soll
es desselben Tages, da es geopfert ist, gegessen werden; so aber etwas
übrigbleibt auf den andern Tag, so soll man's doch essen. \bibverse{17}
Aber was vom geopferten Fleisch übrigbleibt am dritten Tage, soll mit
Feuer verbrannt werden. \bibverse{18} Und wo jemand am dritten Tage wird
essen von dem geopferten Fleisch seines Dankopfers, so wird er nicht
angenehm sein, der es geopfert hat; es wird ihm auch nicht zugerechnet
werden, sondern es wird ein Greuel sein; und welche Seele davon essen
wird, die ist einer Missetat schuldig. \bibverse{19} Und das Fleisch,
das von etwas Unreinem berührt wird, soll nicht gegessen, sondern mit
Feuer verbrannt werden. Wer reines Leibes ist, soll von dem Fleisch
essen. \bibverse{20} Und welche Seele essen wird von dem Fleisch des
Dankopfers, das dem HERRN zugehört, und hat eine Unreinigkeit an sich,
die wird ausgerottet werden von ihrem Volk. \bibverse{21} Und wenn eine
Seele etwas Unreines anrührt, es sei ein unreiner Mensch, ein unreines
Vieh oder sonst was greulich ist, und vom Fleisch des Dankopfers ißt,
das dem HERRN zugehört, die wird ausgerottet werden von ihrem Volk.
\bibverse{22} Und der HERR redete mit Mose und sprach: \bibverse{23}
Rede mit den Kindern Israel und sprich: Ihr sollt kein Fett essen von
Ochsen, Lämmern und Ziegen. \bibverse{24} Aber das Fett vom Aas, und was
vom Wild zerrissen ist, macht euch zu allerlei Nutz; aber essen sollt
ihr's nicht. \bibverse{25} Denn wer das Fett ißt von dem Vieh, davon man
dem HERRN Opfer bringt, dieselbe Seele soll ausgerottet werde von ihrem
Volk. \bibverse{26} Ihr sollt auch kein Blut essen, weder vom Vieh noch
von Vögeln, überall, wo ihr wohnt. \bibverse{27} Welche Seele würde
irgend ein Blut essen, die soll ausgerottet werden von ihrem Volk.
\bibverse{28} Und der HERR redete mit Mose und sprach: \bibverse{29}
Rede mit den Kindern Israel und sprich: Wer dem HERRN sein Dankopfer tun
will, der soll darbringen, was vom Dankopfer dem HERRN gehört.
\bibverse{30} Er soll's aber mit seiner Hand herzubringen zum Opfer des
HERRN; nämlich das Fett soll er bringen samt der Brust, daß sie ein
Webeopfer werden vor dem HERRN. \bibverse{31} Und der Priester soll das
Fett anzünden auf dem Altar, aber die Brust soll Aarons und seiner Söhne
sein. \bibverse{32} Und die rechte Schulter sollen sie dem Priester
geben zur Hebe von ihren Dankopfern. \bibverse{33} Und welcher unter
Aarons Söhnen das Blut der Dankopfer opfert und das Fett, des soll die
rechte Schulter sein zu seinem Teil. \bibverse{34} Denn die Webebrust
und die Hebeschulter habe ich genommen von den Kindern Israel von ihren
Dankopfern und habe sie dem Priester Aaron und seinen Söhnen gegeben zum
ewigen Recht. \bibverse{35} Dies ist die Gebühr Aarons und seiner Söhne
von den Opfern des HERRN, des Tages, da sie überantwortet wurden
Priester zu sein dem HERRN, \bibverse{36} die der HERR gebot am Tage, da
er sie salbte, daß sie ihnen gegeben werden sollte von den Kindern
Israel, zum ewigen Recht allen ihren Nachkommen. \bibverse{37} Dies ist
das Gesetz des Brandopfers, des Speisopfers, des Sündopfers, des
Schuldopfers, der Füllopfer und der Dankopfer, \bibverse{38} das der
HERR dem Mose gebot auf dem Berge Sinai des Tages, da er ihm gebot an
die Kinder Israel, zu opfern ihre Opfer dem HERRN in der Wüste Sinai.

\hypertarget{section-7}{%
\section{8}\label{section-7}}

\bibverse{1} Und der HERR redete mit Mose und sprach: \bibverse{2} Nimm
Aaron und seine Söhne mit ihm samt ihren Kleidern und das Salböl und
einen Farren zum Sündopfer, zwei Widder und einen Korb mit ungesäuertem
Brot, \bibverse{3} und versammle die ganze Gemeinde vor die Tür der
Hütte des Stifts. \bibverse{4} Mose tat, wie ihm der HERR gebot, und
versammelte die Gemeinde vor die Tür der Hütte des Stifts \bibverse{5}
und sprach zu ihnen: Das ist's, was der HERR geboten hat zu tun.
\bibverse{6} Und nahm Aaron und seine Söhne und wusch sie mit Wasser
\bibverse{7} und legte ihnen den leinenen Rock an und gürtete sie mit
dem Gürtel und zog ihnen den Purpurrock an und tat ihm den Leibrock an
und Gürtete ihn über den Leibrock her \bibverse{8} und tat ihm das
Amtschild an und das Schild Licht und Recht \bibverse{9} und setzte ihm
den Hut auf sein Haupt und setzte an den Hut oben an seiner Stirn das
goldene Blatt der heiligen Krone, wie der HERR dem Mose geboten hatte.
\bibverse{10} Und Mose nahm das Salböl und salbte die Wohnung und alles,
was darin war, und weihte es \bibverse{11} und sprengte damit siebenmal
auf den Altar und salbte den Altar mit allem seinem Geräte, das Handfaß
mit seinem Fuß, daß es geweiht würde, \bibverse{12} und goß von dem
Salböl auf Aarons Haupt und salbte ihn, daß er geweiht würde,
\bibverse{13} und brachte herzu Aarons Söhne und zog ihnen leinene Röcke
an und gürtete sie mit dem Gürtel und band ihnen Hauben auf, wie ihm der
HERR geboten hatte. \bibverse{14} Und ließ herzuführen einen Farren zum
Sündopfer. Und Aaron und seine Söhne legten ihre Hände auf sein Haupt.
\bibverse{15} Da schlachtete er ihn. Und Mose nahm das Blut und tat's
auf die Hörner des Altars umher mit seinem Finger und entsündigte den
Altar und goß das Blut an des Altars Boden und weihte ihn, daß er ihn
versöhnte. \bibverse{16} Und nahm alles Fett am Eingeweide, das Netz
über der Leber und die zwei Nieren mit dem Fett daran, und zündete es an
auf dem Altar. \bibverse{17} Aber den Farren mit seinem Fell, Fleisch
und Mist verbrannte er mit Feuer draußen vor dem Lager, wie ihm der HERR
geboten hatte. \bibverse{18} Und brachte herzu einen Widder zum
Brandopfer. Und Aaron und seine Söhne legten ihre Hände auf sein Haupt.
\bibverse{19} Da schlachtete er ihn. Und Mose sprengte das Blut auf den
Altar umher, \bibverse{20} zerhieb den Widder in Stücke und zündete an
das Haupt, die Stücke und das Fett \bibverse{21} und wusch die
Eingeweide und Schenkel mit Wasser und zündete also den ganzen Widder an
auf dem Altar. Das war ein Brandopfer zum süßen Geruch, ein Feuer dem
HERRN, wie ihm der HERR geboten hatte. \bibverse{22} Er brachte auch
herzu den andern Widder des Füllopfers. Und Aaron und seine Söhne legten
ihre Hände auf sein Haupt. \bibverse{23} Da schlachtete er ihn. Und Mose
nahm von seinem Blut und tat's Aaron auf den Knorpel seines rechten Ohrs
und auf den Daumen seiner rechten Hand und auf die große Zehe seines
rechten Fußes. \bibverse{24} Und brachte herzu Aarons Söhne und tat von
dem Blut auf den Knorpel des rechten Ohrs und auf den Daumen ihrer
rechten Hand und auf die große Zehe ihres rechten Fußes und sprengte das
Blut auf den Altar umher. \bibverse{25} Und nahm das Fett und den
Schwanz und alles Fett am Eingeweide und das Netz über der Leber, die
zwei Nieren mit dem Fett daran und die rechte Schulter; \bibverse{26}
dazu nahm er von dem Korb des ungesäuerten Brots vor dem HERRN einen
ungesäuerten Kuchen und einen Kuchen geölten Brots und einen Fladen und
legte es auf das Fett und auf die rechte Schulter. \bibverse{27} Und gab
das allesamt auf die Hände Aarons und seiner Söhne und webte es zum
Webeopfer vor dem HERRN. \bibverse{28} Und nahm alles wieder von ihren
Händen und zündete es an auf dem Altar oben auf dem Brandopfer. Ein
Füllopfer war es zum süßen Geruch, ein Feuer dem HERRN. \bibverse{29}
Und Mose nahm die Brust und webte ein Webeopfer vor dem HERRN von dem
Widder des Füllopfers; der ward Mose zu seinem Teil, wie ihm der HERR
geboten hatte. \bibverse{30} Und Mose nahm von dem Salböl und dem Blut
auf dem Altar und sprengte es auf Aaron und seine Kleider, auf seine
Söhne und ihre Kleider, und weihte also Aaron und seine Kleider, seine
Söhne und ihre Kleider mit ihm. \bibverse{31} Und sprach zu Aaron und
seinen Söhnen: Kochet das Fleisch vor der Tür der Hütte des Stifts, und
esset es daselbst, dazu auch das Brot im Korbe des Füllopfers, wie mir
geboten ist und gesagt, daß Aaron und seine Söhne es essen sollen.
\bibverse{32} Was aber übrigbleibt vom Fleisch und Brot, das sollt ihr
mit Feuer verbrennen. \bibverse{33} Und sollt in sieben Tagen nicht
ausgehen von der Tür der Hütte des Stifts bis an den Tag, da die Tage
eures Füllopfers aus sind; denn sieben Tage sind eure Hände gefüllt,
\bibverse{34} wie es an diesem Tage geschehen ist; der HERR hat's
geboten zu tun, auf daß ihr versöhnt seid. \bibverse{35} Und sollt vor
der Tür der Hütte des Stifts Tag und Nacht bleiben sieben Tage lang und
sollt nach dem Gebot des HERRN tun, daß ihr nicht sterbet; denn also ist
mir's geboten. \bibverse{36} Und Aaron und seine Söhne taten alles, was
ihnen der HERR geboten hatte durch Mose.

\hypertarget{section-8}{%
\section{9}\label{section-8}}

\bibverse{1} Und am achten Tage rief Mose Aaron und seine Söhne und die
Ältesten in Israel \bibverse{2} und sprach zu Aaron: Nimm zu dir ein
junges Kalb zum Sündopfer und einen Widder zum Brandopfer, beide ohne
Fehl, und bringe sie vor den Herrn. \bibverse{3} Und rede mit den
Kindern Israel und sprich: Nehmt einen Ziegenbock zum Sündopfer und ein
Kalb und ein Schaf, beide ein Jahr alt und ohne Fehl, zum Brandopfer
\bibverse{4} und einen Ochsen und einen Widder zum Dankopfer, daß wir
dem HERRN opfern, und ein Speisopfer, mit Öl gemengt. Denn heute wird
euch der HERR erscheinen. \bibverse{5} Und sie nahmen, was Mose geboten
hatte, vor der Tür der Hütte des Stifts; und es trat herzu die ganze
Gemeinde und stand vor dem HERRN. \bibverse{6} Da sprach Mose: Das
ist's, was der HERR geboten hat, daß ihr es tun sollt, so wird euch des
HERRN Herrlichkeit erscheinen. \bibverse{7} Und Mose sprach zu Aaron:
Tritt zum Altar und mache dein Sündopfer und dein Brandopfer und
versöhne dich und das Volk; darnach mache des Volkes Opfer und versöhne
sie auch, wie der HERR geboten hat. \bibverse{8} Und Aaron trat zum
Altar und schlachtete das Kalb zu seinem Sündopfer. \bibverse{9} Und
seine Söhne brachten das Blut zu ihm, und er tauchte mit seinem Finger
ins Blut und tat's auf die Hörner des Altars und goß das Blut an des
Altars Boden. \bibverse{10} Aber das Fett und die Nieren und das Netz
von der Leber am Sündopfer zündete er an auf dem Altar, wie der HERR dem
Mose geboten hatte. \bibverse{11} Und das Fleisch und das Fell
verbrannte er mit Feuer draußen vor dem Lager. \bibverse{12} Darnach
schlachtete er das Brandopfer; und Aarons Söhne brachten das Blut zu
ihm, und er sprengte es auf den Altar umher. \bibverse{13} Und sie
brachten das Brandopfer zu ihm zerstückt und den Kopf; und er zündete es
an auf dem Altar. \bibverse{14} Und er wusch das Eingeweide und die
Schenkel und zündete es an oben auf dem Brandopfer auf dem Altar.
\bibverse{15} Darnach brachte er herzu des Volks Opfer und nahm den
Bock, das Sündopfer des Volks, und schlachtete ihn und machte ein
Sündopfer daraus wie das vorige. \bibverse{16} Und brachte das
Brandopfer herzu und tat damit, wie es recht war. \bibverse{17} Und
brachte herzu das Speisopfer und nahm seine Hand voll und zündete es an
auf dem Altar, außer dem Morgenbrandopfer. \bibverse{18} Darnach
schlachtete er den Ochsen und den Widder zum Dankopfer des Volks; und
seine Söhne brachten ihm das Blut, das sprengte er auf dem Altar umher.
\bibverse{19} Aber das Fett vom Ochsen und vom Widder, den Schwanz und
das Fett am Eingeweide und die Nieren und das Netz über der Leber:
\bibverse{20} alles solches Fett legten sie auf die Brust; und er
zündete das Fett an auf dem Altar. \bibverse{21} Aber die Brust und die
rechte Schulter webte Aaron zum Webopfer vor dem HERRN, wie der HERR dem
Mose geboten hatte. \bibverse{22} Und Aaron hob seine Hand auf zum Volk
und segnete sie; und er stieg herab, da er das Sündopfer, Brandopfer und
Dankopfer gemacht hatte. \bibverse{23} Und Mose und Aaron gingen in die
Hütte des Stifts; und da sie wieder herausgingen, segneten sie das Volk.
Da erschien die Herrlichkeit des HERRN allem Volk. \bibverse{24} Und ein
Feuer ging aus von dem HERRN und verzehrte auf dem Altar das Brandopfer
und das Fett. Da das alles Volk sah, frohlockten sie und fielen auf ihr
Antlitz.

\hypertarget{section-9}{%
\section{10}\label{section-9}}

\bibverse{1} Und die Söhne Aarons Nadab und Abihu nahmen ein jeglicher
seinen Napf und taten Feuer darein und legten Räuchwerk darauf und
brachten das fremde Feuer vor den HERRN, das er ihnen nicht geboten
hatte. \bibverse{2} Da fuhr ein Feuer aus von dem HERRN und verzehrte
sie, daß sie starben vor dem HERRN. \bibverse{3} Da sprach Mose zu
Aaron: Das ist's, was der HERR gesagt hat: Ich erzeige mich heilig an
denen, die mir nahe sind, und vor allem Volk erweise ich mich herrlich.
Und Aaron schwieg still. \bibverse{4} Mose aber rief Misael und
Elzaphan, die Söhne Usiels, Aarons Vettern, und sprach zu ihnen: Tretet
hinzu und traget eure Brüder von dem Heiligtum hinaus vor das Lager.
\bibverse{5} Und sie traten hinzu und trugen sie hinaus mit ihren
leinenen Röcken vor das Lager, wie Mose gesagt hatte. \bibverse{6} Da
sprach Mose zu Aaron und seinen Söhnen Eleasar und Ithamar: Ihr sollt
eure Häupter nicht entblößen noch eure Kleider zerreißen, daß ihr nicht
sterbet und der Zorn über die ganze Gemeinde komme. Laßt eure Brüder,
das ganze Haus Israel, weinen über diesen Brand, den der HERR getan hat.
\bibverse{7} Ihr aber sollt nicht ausgehen von der Tür der Hütte des
Stifts, ihr möchtet sterben; denn das Salböl des HERRN ist auf euch. Und
sie taten, wie Mose sagte. \bibverse{8} Der HERR aber redete mit Aaron
und sprach: \bibverse{9} Du und deine Söhne mit dir sollt keinen Wein
noch starkes Getränk trinken, wenn ihr in die Hütte des Stifts geht, auf
daß ihr nicht sterbet. Das sei ein ewiges Recht allen euren Nachkommen,
\bibverse{10} auf daß ihr könnt unterscheiden, was heilig und unheilig,
was rein und unrein ist, \bibverse{11} und daß ihr die Kinder Israel
lehret alle Rechte, die der HERR zu ihnen geredet hat durch Mose.
\bibverse{12} Und Mose redete mit Aaron und mit seinen noch übrigen
Söhnen, Eleasar und Ithamar: Nehmet, was übriggeblieben ist vom
Speisopfer an den Opfern des HERRN, und esset's ungesäuert bei dem
Altar; denn es ist ein Hochheiliges. \bibverse{13} Ihr sollt's aber an
heiliger Stätte essen; denn das ist dein Recht und deiner Söhne Recht an
den Opfern des HERRN; denn so ist's mir geboten. \bibverse{14} Aber die
Webebrust und die Hebeschulter sollst du und deine Söhne und deine
Töchter mit dir essen an reiner Stätte; denn solch Recht ist dir und
deinen Kindern gegeben an den Dankopfern der Kinder Israel.
\bibverse{15} Denn die Hebeschulter und die Webebrust soll man zu den
Opfern des Fetts bringen, daß sie zum Webeopfer gewebt werden vor dem
HERRN; darum ist's dein und deiner Kinder zum ewigen Recht, wie der HERR
geboten hat. \bibverse{16} Und Mose suchte den Bock des Sündopfers, und
fand ihn verbrannt, Und er ward zornig über Eleasar und Ithamar, Aarons
Söhne, die noch übrig waren, und sprach: \bibverse{17} Warum habt ihr
das Sündopfer nicht gegessen an heiliger Stätte? denn es ist ein
Hochheiliges, und er hat's euch gegeben, daß ihr die Missetat der
Gemeinde tragen sollt, daß ihr sie versöhnet vor dem HERRN.
\bibverse{18} Siehe, sein Blut ist nicht gekommen in das Heilige hinein.
Ihr solltet es im Heiligen gegessen haben, wie mir geboten ist.
\bibverse{19} Aaron aber sprach zu Mose: Siehe, heute haben sie ihr
Sündopfer und ihr Brandopfer vor dem HERRN geopfert, und es ist mir also
gegangen, wie du siehst; und ich sollte essen heute vom Sündopfer?
Sollte das dem HERRN gefallen? \bibverse{20} Da das Mose hörte, ließ
er's sich gefallen.

\hypertarget{section-10}{%
\section{11}\label{section-10}}

\bibverse{1} Und der HERR redete mit Mose und Aaron und sprach zu ihnen:
\bibverse{2} Redet mit den Kindern Israel und sprecht: Das sind die
Tiere, die ihr essen sollt unter allen Tieren auf Erden. \bibverse{3}
Alles, was die Klauen spaltet und wiederkäut unter den Tieren, das sollt
ihr essen. \bibverse{4} Was aber wiederkäut und hat Klauen und spaltet
sie doch nicht, wie das Kamel, das ist euch unrein, und ihr sollt's
nicht essen. \bibverse{5} Die Kaninchen wiederkäuen wohl, aber sie
spalten die Klauen nicht; darum sind sie unrein. \bibverse{6} Der Hase
wiederkäut auch, aber er spaltet die Klauen nicht; darum ist er euch
unrein. \bibverse{7} Und ein Schwein spaltet wohl die Klauen, aber es
wiederkäut nicht; darum soll's euch unrein sein. \bibverse{8} Von dieser
Fleisch sollt ihr nicht essen noch ihr Aas anrühren; denn sie sind euch
unrein. \bibverse{9} Dies sollt ihr essen unter dem, was in Wassern ist:
alles, was Floßfedern und Schuppen hat in Wassern, im Meer und in
Bächen, sollt ihr essen. \bibverse{10} Alles aber, was nicht Floßfedern
und Schuppen hat im Meer und in Bächen, unter allem, was sich regt in
Wassern, und allem, was lebt im Wasser, soll euch eine Scheu sein,
\bibverse{11} daß ihr von ihrem Fleisch nicht eßt und vor ihrem Aas euch
scheut. \bibverse{12} Denn alles, was nicht Floßfedern und Schuppen hat
in Wassern, sollt ihr scheuen. \bibverse{13} Und dies sollt ihr scheuen
unter den Vögeln, daß ihr's nicht eßt: den Adler, den Habicht, den
Fischaar, \bibverse{14} den Geier, den Weih, und was seine Art ist,
\bibverse{15} und alle Raben mit ihrer Art, \bibverse{16} den Strauß,
die Nachteule, den Kuckuck, den Sperber mit seiner Art, \bibverse{17}
das Käuzlein, den Schwan, den Uhu, \bibverse{18} die Fledermaus, die
Rohrdommel, \bibverse{19} den Storch, den Reiher, den Häher mit seiner
Art, den Wiedehopf und die Schwalbe. \bibverse{20} Alles auch, was sich
regt und Flügel hat und geht auf vier Füßen, das soll euch eine Scheu
sein. \bibverse{21} Doch das sollt ihr essen von allem, was sich regt
und Flügel hat und geht auf vier Füßen: was noch zwei Beine hat, womit
es auf Erden hüpft; \bibverse{22} von demselben mögt ihr essen die
Heuschrecken, als da ist: Arbe mit seiner Art und Solam mit seiner Art
und Hargol mit seiner Art und Hagab mit seiner Art. \bibverse{23} Aber
alles, was sonst Flügel und vier Füße hat, soll euch eine Scheu sein,
\bibverse{24} und sollt sie unrein achten. Wer solcher Aas anrührt, der
wird unrein sein bis auf den Abend. \bibverse{25} Und wer dieser Aase
eines tragen wird, soll seine Kleider waschen und wird unrein sein bis
auf den Abend. \bibverse{26} Darum alles Getier, das Klauen hat und
spaltet sie nicht und wiederkäuet nicht, das soll euch unrein sein.
\bibverse{27} Und alles, was auf Tatzen geht unter den Tieren, die auf
vier Füßen gehen, soll euch unrein sein; wer ihr Aas anrührt, wird
unrein sein bis auf den Abend. \bibverse{28} Und wer ihr Aas trägt, soll
seine Kleider waschen und unrein sein bis auf den Abend; denn solche
sind euch unrein. \bibverse{29} Diese sollen euch auch unrein sein unter
den Tieren, die auf Erden kriechen: das Wiesel, die Maus, die Kröte, ein
jegliches mit seiner Art, \bibverse{30} der Igel, der Molch, die
Eidechse, die Blindschleiche und der Maulwurf; \bibverse{31} die sind
euch unrein unter allem, was da kriecht; wer ihr Aas anrührt, der wird
unrein sein bis auf den Abend. \bibverse{32} Und alles, worauf ein solch
totes Aas fällt, das wird unrein, es sei allerlei hölzernes Gefäß oder
Kleider oder Fell oder Sack; und alles Gerät, womit man etwas schafft,
soll man ins Wasser tun, und es ist unrein bis auf den Abend; alsdann
wird's rein. \bibverse{33} Allerlei irdenes Gefäß, wo solcher Aas
hineinfällt, wird alles unrein, was darin ist; und sollt's zerbrechen.
\bibverse{34} Alle Speise, die man ißt, so solch Wasser hineinkommt, ist
unrein; und aller Trank, den man trinkt in allerlei solchem Gefäß, ist
unrein. \bibverse{35} Und alles, worauf solches Aas fällt, wird unrein,
es sei ein Ofen oder Kessel, so soll man's zerbrechen; denn es ist
unrein und soll euch unrein sein. \bibverse{36} Doch die Brunnen und
Gruben und Teiche bleiben rein. Wer aber ihr Aas anrührt, ist unrein.
\bibverse{37} Und ob solch ein Aas fiele auf Samen, den man sät, so ist
er doch rein. \bibverse{38} Wenn man aber Wasser über den Samen gösse,
und fiele darnach ein solch Aas darauf, so würde er euch unrein.
\bibverse{39} Wenn ein Tier stirbt, das ihr essen mögt: wer das Aas
anrührt, der ist unrein bis an den Abend. \bibverse{40} Wer von solchem
Aas ißt, der soll sein Kleid waschen und wird unrein sein bis an den
Abend. Also wer auch trägt ein solch Aas, soll sein Kleid waschen, und
ist unrein bis an den Abend \bibverse{41} Was auf Erden schleicht, das
soll euch eine Scheu sein, und man soll's nicht essen. \bibverse{42}
Alles, was auf dem Bauch kriecht, und alles, was auf vier oder mehr
Füßen geht, unter allem, was auf Erden schleicht, sollt ihr nicht essen;
denn es soll euch eine Scheu sein. \bibverse{43} Macht eure Seelen nicht
zum Scheusal und verunreinigt euch nicht an ihnen, daß ihr euch
besudelt. \bibverse{44} Denn ich bin der HERR, euer Gott. Darum sollt
ihr euch heiligen, daß ihr heilig seid, denn ich bin heilig, und sollt
eure Seelen nicht verunreinigen an irgend einem kriechenden Tier, das
auf Erden schleicht. \bibverse{45} Denn ich bin der HERR, der euch aus
Ägyptenland geführt hat, daß ich euer Gott sei. Darum sollt ihr heilig
sein, denn ich bin heilig. \bibverse{46} Dies ist das Gesetz von den
Tieren und Vögeln und allerlei Tieren, die sich regen im Wasser, und
allerlei Tieren, die auf Erden schleichen, \bibverse{47} daß ihr
unterscheiden könnt, was unrein und rein ist, und welches Tier man essen
und welches man nicht essen soll.

\hypertarget{section-11}{%
\section{12}\label{section-11}}

\bibverse{1} Und der HERR redete mit Mose und sprach: \bibverse{2} Rede
mit den Kindern Israel und sprich: Wenn ein Weib empfängt und gebiert
ein Knäblein, so soll sie sieben Tage unrein sein, wie wenn sie ihre
Krankheit leidet. \bibverse{3} Und am achten Tage soll man das Fleisch
seiner Vorhaut beschneiden. \bibverse{4} Und sie soll daheimbleiben
dreiunddreißig Tage im Blut ihrer Reinigung. Kein Heiliges soll sie
anrühren, und zum Heiligtum soll sie nicht kommen, bis daß die Tage
ihrer Reinigung aus sind. \bibverse{5} Gebiert sie aber ein Mägdlein, so
soll sie zwei Wochen unrein sein, wie wenn sie ihre Krankheit leidet,
und soll sechsundsechzig Tage daheimbleiben in dem Blut ihrer Reinigung.
\bibverse{6} Und wenn die Tage ihrer Reinigung aus sind für den Sohn
oder für die Tochter, soll sie ein jähriges Lamm bringen zum Brandopfer
und eine junge Taube oder Turteltaube zum Sündopfer dem Priester vor die
Tür der Hütte des Stifts. \bibverse{7} Der soll es opfern vor dem HERRN
und sie versöhnen, so wird sie rein von ihrem Blutgang. Das ist das
Gesetz für die, so ein Knäblein oder Mägdlein gebiert. \bibverse{8}
Vermag aber ihre Hand nicht ein Schaf, so nehme sie zwei Turteltauben
oder zwei junge Tauben, eine zum Brandopfer, die andere zum Sündopfer;
so soll sie der Priester versöhnen, daß sie rein werde.

\hypertarget{section-12}{%
\section{13}\label{section-12}}

\bibverse{1} Und der HERR redete mit Mose und Aaron und sprach:
\bibverse{2} Wenn einem Menschen an der Haut seines Fleisches etwas
auffährt oder ausschlägt oder eiterweiß wird, als wollte ein Aussatz
werden an der Haut seines Fleisches, soll man ihn zum Priester Aaron
führen oder zu einem unter seinen Söhnen, den Priestern. \bibverse{3}
Und wenn der Priester das Mal an der Haut des Fleisches sieht, daß die
Haare in Weiß verwandelt sind und das Ansehen an dem Ort tiefer ist denn
die andere Haut seines Fleisches, so ist's gewiß der Aussatz. Darum soll
ihn der Priester besehen und für unrein urteilen. \bibverse{4} Wenn aber
etwas eiterweiß ist an der Haut des Fleisches, und doch das Ansehen der
Haut nicht tiefer denn die andere Haut des Fleisches und die Haare nicht
in Weiß verwandelt sind, so soll der Priester ihn verschließen sieben
Tage \bibverse{5} und am siebenten Tage besehen. Ist's, daß das Mal
bleibt, wie er's zuvor gesehen hat, und hat nicht weitergefressen an der
Haut, \bibverse{6} so soll ihn der Priester abermals sieben Tage
verschließen. Und wenn er ihn zum andermal am siebenten Tage besieht und
findet, daß das Mal verschwunden ist und nicht weitergefressen hat an
der Haut, so soll er ihn rein urteilen; denn es ist ein Grind. Und er
soll seine Kleider waschen, so ist er rein. \bibverse{7} Wenn aber der
Grind weiterfrißt in der Haut, nachdem er vom Priester besehen worden
ist, ob er rein sei, und wird nun zum andernmal vom Priester besehen,
\bibverse{8} wenn dann da der Priester sieht, daß der Grind
weitergefressen hat in der Haut, so soll er ihn unrein urteilen; denn es
ist gewiß Aussatz. \bibverse{9} Wenn ein Mal des Aussatzes an einem
Menschen sein wird, den soll man zum Priester bringen. \bibverse{10}
Wenn derselbe sieht und findet, daß Weißes aufgefahren ist an der Haut
und die Haare in Weiß verwandelt und rohes Fleisch im Geschwür ist,
\bibverse{11} so ist's gewiß ein alter Aussatz in der Haut des
Fleisches. Darum soll ihn der Priester unrein urteilen und nicht
verschließen; denn er ist schon unrein. \bibverse{12} Wenn aber der
Aussatz blüht in der Haut und bedeckt die ganze Haut, von dem Haupt bis
auf die Füße, alles, was dem Priester vor Augen sein mag, \bibverse{13}
wenn dann der Priester besieht und findet, daß der Aussatz das ganze
Fleisch bedeckt hat, so soll er denselben rein urteilen, dieweil es
alles an ihm in Weiß verwandelt ist; denn er ist rein. \bibverse{14} Ist
aber rohes Fleisch da des Tages, wenn er besehen wird, so ist er unrein.
\bibverse{15} Und wenn der Priester das rohe Fleisch sieht, soll er ihn
unrein urteilen; denn das rohe Fleisch ist unrein, und es ist gewiß
Aussatz. \bibverse{16} Verkehrt sich aber das rohe Fleisch wieder und
verwandelt sich in Weiß, so soll er zum Priester kommen. \bibverse{17}
Und wenn der Priester besieht und findet, daß das Mal ist in Weiß
verwandelt, soll er ihn rein urteilen; denn er ist rein. \bibverse{18}
Wenn jemandes Fleisch an der Haut eine Drüse wird und wieder heilt,
\bibverse{19} darnach an demselben Ort etwas Weißes auffährt oder
rötliches Eiterweiß wird, soll er vom Priester besehen werden.
\bibverse{20} Wenn dann der Priester sieht, daß das Ansehen tiefer ist
denn die andere Haut und das Haar in Weiß verwandelt, so soll er ihn
unrein urteilen; denn es ist gewiß ein Aussatzmal aus der Drüse
geworden. \bibverse{21} Sieht aber der Priester und findet, daß die
Haare nicht weiß sind und es ist nicht tiefer denn die andere Haut und
ist verschwunden, so soll er ihn sieben Tage verschließen. \bibverse{22}
Frißt es weiter in der Haut, so soll er unrein urteilen; denn es ist
gewiß ein Aussatzmal. \bibverse{23} Bleibt aber das Eiterweiß also
stehen und frißt nicht weiter, so ist's die Narbe von der Drüse, und der
Priester soll ihn rein urteilen. \bibverse{24} Wenn sich jemand an der
Haut am Feuer brennt und das Brandmal weißrötlich oder weiß ist
\bibverse{25} und der Priester ihn besieht und findet das Haar in Weiß
verwandelt an dem Brandmal und das Ansehen tiefer denn die andere Haut,
so ist's gewiß Aussatz, aus dem Brandmal geworden. Darum soll ihn der
Priester unrein urteilen; denn es ist ein Aussatzmal. \bibverse{26}
Sieht aber der Priester und findet, daß die Haare am Brandmal nicht in
Weiß verwandelt und es nicht tiefer ist denn die andere Haut und ist
dazu verschwunden, soll er ihn sieben Tage verschließen; \bibverse{27}
und am siebenten Tage soll er ihn besehen. Hat's weitergefressen an der
Haut, so soll er unrein urteilen; denn es ist Aussatz. \bibverse{28}
Ist's aber gestanden an dem Brandmal und hat nicht weitergefressen an
der Haut und ist dazu verschwunden, so ist's ein Geschwür des Brandmals.
Und der Priester soll ihn rein urteilen; denn es ist die Narbe des
Brandmals. \bibverse{29} Wenn ein Mann oder Weib auf dem Haupt oder am
Bart ein Mal hat \bibverse{30} und der Priester das Mal besieht und
findet, daß das Ansehen der Haut tiefer ist denn die andere Haut und das
Haar daselbst golden und dünn, so soll er ihn unrein urteilen; denn es
ist ein aussätziger Grind des Hauptes oder des Bartes. \bibverse{31}
Sieht aber der Priester, daß der Grind nicht tiefer anzusehen ist denn
die andere Haut und das Haar nicht dunkel ist, soll er denselben sieben
Tage verschließen. \bibverse{32} Und wenn er am siebenten Tage besieht
und findet, daß der Grind nicht weitergefressen hat und kein goldenes
Haar da ist und das Ansehen des Grindes nicht tiefer ist denn die andere
Haut, \bibverse{33} soll er sich scheren, doch daß er den Grind nicht
beschere; und soll ihn der Priester abermals sieben Tage verschließen.
\bibverse{34} Und wenn er ihn am siebenten Tage besieht und findet, daß
der Grind nicht weitergefressen hat in der Haut und das Ansehen ist
nicht tiefer als die andere Haut, so soll er ihn rein sprechen, und er
soll seine Kleider waschen; denn er ist rein. \bibverse{35} Frißt aber
der Grind weiter an der Haut, nachdem er rein gesprochen ist,
\bibverse{36} und der Priester besieht und findet, daß der Grind also
weitergefressen hat an der Haut, so soll er nicht mehr darnach fragen,
ob die Haare golden sind; denn er ist unrein. \bibverse{37} Ist aber vor
Augen der Grind stillgestanden und dunkles Haar daselbst aufgegangen, so
ist der Grind heil und er rein. Darum soll ihn der Priester rein
sprechen. \bibverse{38} Wenn einem Mann oder Weib an der Haut ihres
Fleisches etwas eiterweiß ist \bibverse{39} und der Priester sieht
daselbst, daß das Eiterweiß schwindet, das ist ein weißer Grind, in der
Haut aufgegangen, und er ist rein. \bibverse{40} Wenn einem Manne die
Haupthaare ausfallen, daß er kahl wird, der ist rein. \bibverse{41}
Fallen sie ihm vorn am Haupt aus und wird eine Glatze, so ist er rein.
\bibverse{42} Wird aber an der Glatze, oder wo er kahl ist, ein weißes
oder rötliches Mal, so ist ihm Aussatz an der Glatze oder am Kahlkopf
aufgegangen. \bibverse{43} Darum soll ihn der Priester besehen. Und wenn
er findet, daß ein weißes oder rötliches Mal aufgelaufen an seiner
Glatze oder am Kahlkopf, daß es sieht, wie sonst der Aussatz an der
Haut, \bibverse{44} so ist er aussätzig und unrein; und der Priester
soll ihn unrein sprechen solches Mals halben auf seinem Haupt.
\bibverse{45} Wer nun aussätzig ist, des Kleider sollen zerrissen sein
und das Haupt bloß und die Lippen verhüllt und er soll rufen: Unrein,
unrein! \bibverse{46} Und solange das Mal an ihm ist, soll er unrein
sein, allein wohnen und seine Wohnung soll außerhalb des Lagers sein.
\bibverse{47} Wenn an einem Kleid ein Aussatzmal sein wird, es sei
wollen oder leinen, \bibverse{48} am Aufzug oder am Eintrag, es sei
wollen oder leinen, oder an einem Fell oder an allem, was aus Fellen
gemacht wird, \bibverse{49} und wenn das Mal grünlich oder rötlich ist
am Kleid oder am Fell oder am Aufzug oder am Eintrag oder an irgend
einem Ding, das von Fellen gemacht ist, das ist gewiß ein Mal des
Aussatzes; darum soll's der Priester besehen. \bibverse{50} Und wenn er
das Mal sieht, soll er's einschließen sieben Tage. \bibverse{51} Und
wenn er am siebenten Tage sieht, daß das Mal hat weitergefressen am
Kleid, am Aufzug oder am Eintrag, am Fell oder an allem, was man aus
Fellen macht, so ist das Mal ein fressender Aussatz, und es ist unrein.
\bibverse{52} Und man soll das Kleid verbrennen oder den Aufzug oder den
Eintrag, es sei wollen oder leinen oder allerlei Fellwerk, darin solch
ein Mal ist; denn es ist fressender Aussatz, und man soll es mit Feuer
verbrennen. \bibverse{53} Wird aber der Priester sehen, daß das Mal
nicht weitergefressen hat am Kleid oder am Aufzug oder am Eintrag oder
an allerlei Fellwerk, \bibverse{54} so soll er gebieten, daß man solches
wasche, worin solches Mal ist, und soll's einschließen andere sieben
Tage. \bibverse{55} Und wenn der Priester sehen wird, nachdem das Mal
gewaschen ist, daß das Mal nicht verwandelt ist vor seinen Augen und
auch nicht weitergefressen hat, so ist's unrein, und sollst es mit Feuer
verbrennen; denn es ist tief eingefressen und hat's vorn oder hinten
schäbig gemacht. \bibverse{56} Wenn aber der Priester sieht, daß das Mal
verschwunden ist nach seinem Waschen, so soll er's abreißen vom Kleid,
vom Fell, von Aufzug oder vom Eintrag. \bibverse{57} Wird's aber noch
gesehen am Kleid, am Aufzug, am Eintrag oder allerlei Fellwerk, so ist's
ein Aussatzmal, und sollst das mit Feuer verbrennen, worin solch Mal
ist. \bibverse{58} Das Kleid aber oder der Aufzug oder Eintrag oder
allerlei Fellwerk, das gewaschen und von dem das Mal entfernt ist, soll
man zum andernmal waschen, so ist's rein. \bibverse{59} Das ist das
Gesetz über die Male des Aussatzes an Kleidern, sie seien wollen oder
leinen, am Aufzug und am Eintrag und allerlei Fellwerk, rein oder unrein
zu sprechen.

\hypertarget{section-13}{%
\section{14}\label{section-13}}

\bibverse{1} Und der HERR redete mit Mose und sprach: \bibverse{2} Das
ist das Gesetz über den Aussätzigen, wenn er soll gereinigt werden. Er
soll zum Priester kommen. \bibverse{3} Und der Priester soll aus dem
Lager gehen und besehen, wie das Mal des Aussatzes am Aussätzigen heil
geworden ist, \bibverse{4} und soll gebieten dem, der zu reinigen ist,
daß er zwei lebendige Vögel nehme, die da rein sind, und Zedernholz und
scharlachfarbene Wolle und Isop. \bibverse{5} Und soll gebieten, den
einen Vogel zu schlachten in ein irdenes Gefäß über frischem Wasser.
\bibverse{6} Und soll den lebendigen Vogel nehmen mit dem Zedernholz,
scharlachfarbener Wolle und Isop und in des Vogels Blut tauchen, der
über dem frischen Wasser geschlachtet ist, \bibverse{7} und besprengen
den, der vom Aussatz zu reinigen ist, siebenmal; und reinige ihn also
und lasse den lebendigen Vogel ins freie Feld fliegen. \bibverse{8} Der
Gereinigte aber soll seine Kleider waschen und alle seine Haare
abscheren und sich mit Wasser baden, so ist er rein. Darnach gehe er ins
Lager; doch soll er außerhalb seiner Hütte sieben Tage bleiben.
\bibverse{9} Und am siebenten Tage soll er alle seine Haare abscheren
auf dem Haupt, am Bart, an den Augenbrauen, daß alle Haare abgeschoren
seien, und soll seine Kleider waschen und sein Fleisch im Wasser baden,
so ist er rein. \bibverse{10} Und am achten Tage soll er zwei Lämmer
nehmen ohne Fehl und ein jähriges Schaf ohne Fehl und drei zehntel
Semmelmehl zum Speisopfer, mit Öl gemengt, und ein Log Öl. \bibverse{11}
Da soll der Priester den Gereinigten und diese Dinge stellen vor den
HERRN, vor der Tür der Hütte des Stifts. \bibverse{12} Und soll das eine
Lamm nehmen und zum Schuldopfer opfern mit dem Log Öl; und soll solches
vor dem HERRN weben \bibverse{13} und darnach das Lamm schlachten, wo
man das Sündopfer und Brandopfer schlachtet, nämlich an heiliger Stätte;
denn wie das Sündopfer, also ist auch das Schuldopfer des Priesters;
denn es ist ein Hochheiliges. \bibverse{14} Und der Priester soll von
dem Blut nehmen vom Schuldopfer und dem Gereinigten auf den Knorpel des
rechten Ohrs tun und auf den Daumen seiner rechten Hand und auf die
große Zehe seines rechten Fußes. \bibverse{15} Darnach soll er von dem
Log Öl nehmen und es in seine, des Priesters, linke Hand gießen
\bibverse{16} und mit seinem rechten Finger in das Öl tauchen, das in
seiner linken Hand ist, und sprengen vom Öl mit seinem Finger siebenmal
vor dem HERRN. \bibverse{17} Vom übrigen Öl aber in seiner Hand soll er
dem Gereinigten auf den Knorpel des rechten Ohrs tun und auf den rechten
Daumen und auf die große Zehe seines rechten Fußes, oben auf das Blut
des Schuldopfers. \bibverse{18} Das übrige Öl aber in seiner Hand soll
er auf des Gereinigten Haupt tun und ihn versöhnen vor dem HERRN.
\bibverse{19} Und soll das Sündopfer machen und den Gereinigten
versöhnen seiner Unreinigkeit halben; und soll darnach das Brandopfer
schlachten \bibverse{20} und soll es auf dem Altar opfern samt dem
Speisopfer und ihn versöhnen, so ist er rein. \bibverse{21} Ist er aber
arm und erwirbt mit seiner Hand nicht so viel, so nehme er ein Lamm zum
Schuldopfer, zu weben, zu seiner Versöhnung und ein zehntel Semmelmehl,
mit Öl gemengt, zum Speisopfer, und ein Log Öl \bibverse{22} und zwei
Turteltauben oder zwei junge Tauben, die er mit seiner Hand erwerben
kann, daß eine sei ein Sündopfer, die andere ein Brandopfer,
\bibverse{23} und bringe sie am achten Tage seiner Reinigung zum
Priester vor die Tür der Hütte des Stifts, vor den HERRN. \bibverse{24}
Da soll der Priester das Lamm zum Schuldopfer nehmen und das Log Öl und
soll's alles weben vor dem HERRN \bibverse{25} und das Lamm des
Schuldopfers schlachten und Blut nehmen von demselben Schuldopfer und es
dem Gereinigten tun auf den Knorpel seines rechten Ohrs und auf den
Daumen seiner rechten Hand und auf die große Zehe seines rechten Fußes,
\bibverse{26} und von dem Öl in seine, des Priesters, linke Hand gießen
\bibverse{27} und mit seinem rechten Finger vom Öl, das in seiner linken
Hand ist, siebenmal sprengen vor dem HERRN. \bibverse{28} Von dem
übrigen aber in seiner Hand soll er dem Gereinigten auf den Knorpel
seines rechten Ohrs und auf den Daumen seiner rechten Hand und auf die
große Zehe seines rechten Fußes tun, oben auf das Blut des Schuldopfers.
\bibverse{29} Das übrige Öl aber in seiner Hand soll er dem Gereinigten
auf das Haupt tun, ihn zu versöhnen vor dem HERRN; \bibverse{30} und
darnach aus der einen Turteltaube oder jungen Taube, wie seine Hand hat
mögen erwerben, \bibverse{31} ein Sündopfer, aus der andern ein
Brandopfer machen samt dem Speisopfer. Und soll der Priester den
Gereinigten also versöhnen vor dem HERRN. \bibverse{32} Das sei das
Gesetz für den Aussätzigen, der mit seiner Hand nicht erwerben kann, was
zur Reinigung gehört. \bibverse{33} Und der HERR redete mit Mose und
Aaron und sprach: \bibverse{34} Wenn ihr in das Land Kanaan kommt, das
ich euch zur Besitzung gebe, und ich werde irgend in einem Hause eurer
Besitzung ein Aussatzmal geben, \bibverse{35} so soll der kommen, des
das Haus ist, es dem Priester ansagen und sprechen: Es sieht mich an,
als sei ein Aussatzmal an meinem Hause. \bibverse{36} Da soll der
Priester heißen, daß sie das Haus ausräumen, ehe denn der Priester
hineingeht, das Mal zu besehen, auf daß nicht unrein werde alles, was im
Hause ist; darnach soll der Priester hineingehen, das Haus zu besehen.
\bibverse{37} Wenn er nun das Mal besieht und findet, daß an der Wand
des Hauses grünliche oder rötliche Grüblein sind und ihr Ansehen tiefer
denn sonst die Wand ist, \bibverse{38} so soll er aus dem Hause zur Tür
herausgehen und das Haus sieben Tage verschließen. \bibverse{39} Und
wenn er am siebenten Tage wiederkommt und sieht, daß das Mal
weitergefressen hat an des Hauses Wand, \bibverse{40} so soll er die
Steine heißen ausbrechen, darin das Mal ist, und hinaus vor die Stadt an
einen unreinen Ort werfen. \bibverse{41} Und das Haus soll man inwendig
ringsherum schaben und die abgeschabte Tünche hinaus vor die Stadt an
einen unreinen Ort schütten \bibverse{42} und andere Steine nehmen und
an jener Statt tun und andern Lehm nehmen und das Haus bewerfen.
\bibverse{43} Wenn das Mal wiederkommt und ausbricht am Hause, nachdem
man die Steine ausgerissen und das Haus anders beworfen hat,
\bibverse{44} so soll der Priester hineingehen. Und wenn er sieht, daß
das Mal weitergefressen hat am Hause, so ist's gewiß ein fressender
Aussatz am Hause, und es ist unrein. \bibverse{45} Darum soll man das
Haus abbrechen, Steine und Holz und alle Tünche am Hause, und soll's
hinausführen vor die Stadt an einen unreinen Ort. \bibverse{46} Und wer
in das Haus geht, solange es verschlossen ist, der ist unrein bis an den
Abend. \bibverse{47} Und wer darin liegt oder darin ißt, der soll seine
Kleider waschen. \bibverse{48} Wo aber der Priester, wenn er hineingeht,
sieht, daß dies Mal nicht weiter am Haus gefressen hat, nachdem das Haus
beworfen ist, so soll er's rein sprechen; denn das Mal ist heil
geworden. \bibverse{49} Und soll zum Sündopfer für das Haus nehmen zwei
Vögel, Zedernholz und scharlachfarbene Wolle und Isop, \bibverse{50} und
den einen Vogel schlachten in ein irdenes Gefäß über frischem Wasser.
\bibverse{51} Und soll nehmen das Zedernholz, die scharlachfarbene
Wolle, den Isop und den lebendigen Vogel, und in des geschlachteten
Vogels Blut und in das frische Wasser tauchen, und das Haus siebenmal
besprengen. \bibverse{52} Und soll also das Haus entsündigen mit dem
Blut des Vogels und mit dem frischen Wasser, mit dem lebendigen Vogel,
mit dem Zedernholz, mit Isop und mit scharlachfarbener Wolle.
\bibverse{53} Und soll den lebendigen Vogel lassen hinaus vor die Stadt
ins freie Feld fliegen, und das Haus versöhnen, so ist's rein.
\bibverse{54} Das ist das Gesetz über allerlei Mal des Aussatzes und
Grindes, \bibverse{55} über den Aussatz der Kleider und der Häuser,
\bibverse{56} über Beulen, Ausschlag und Eiterweiß, \bibverse{57} auf
daß man wisse, wann etwas unrein oder rein ist. Das ist das Gesetz vom
Aussatz.

\hypertarget{section-14}{%
\section{15}\label{section-14}}

\bibverse{1} Und der HERR redete mit Mose und Aaron und sprach:
\bibverse{2} Redet mit den Kindern Israel und sprecht zu ihnen: Wenn ein
Mann an seinem Fleisch einen Fluß hat, derselbe ist unrein. \bibverse{3}
Dann aber ist er unrein an diesem Fluß, wenn sein Fleisch eitert oder
verstopft ist. \bibverse{4} Alles Lager, darauf er liegt, und alles,
darauf er sitzt, wird unrein werden. \bibverse{5} Und wer sein Lager
anrührt, der soll seine Kleider waschen und sich mit Wasser baden und
unrein sein bis auf den Abend. \bibverse{6} Und wer sich setzt, wo er
gesessen hat, der soll seine Kleider waschen und sich mit Wasser baden
und unrein sein bis auf den Abend. \bibverse{7} Wer sein Fleisch
anrührt, der soll seine Kleider waschen und sich mit Wasser baden und
unrein sein bis auf den Abend. \bibverse{8} Wenn er seinen Speichel
wirft auf den, der rein ist, der soll seine Kleider waschen und sich mit
Wasser baden und unrein sein bis auf den Abend. \bibverse{9} Und der
Sattel, darauf er reitet, wird unrein werden. \bibverse{10} Und wer
anrührt irgend etwas, das er unter sich gehabt hat, der wird unrein sein
bis auf den Abend. Und wer solches trägt, der soll seine Kleider waschen
und sich mit Wasser baden und unrein sein bis auf den Abend.
\bibverse{11} Und welchen er anrührt, ehe er die Hände wäscht, der soll
seine Kleider waschen und sich mit Wasser baden und unrein sein bis auf
den Abend. \bibverse{12} Wenn er ein irdenes Gefäß anrührt, das soll man
zerbrechen; aber das hölzerne Gefäß soll man mit Wasser spülen.
\bibverse{13} Und wenn er rein wird von seinem Fluß, so soll er sieben
Tage zählen, nachdem er rein geworden ist, und seine Kleider waschen und
sein Fleisch mit fließendem Wasser baden, so ist er rein. \bibverse{14}
Und am achten Tage soll er zwei Turteltauben oder zwei junge Tauben
nehmen und vor den HERRN bringen vor die Tür der Hütte des Stifts und
dem Priester geben. \bibverse{15} Und der Priester soll aus einer ein
Sündopfer, aus der andern ein Brandopfer machen und ihn versöhnen vor
dem HERRN seines Flusses halben. \bibverse{16} Wenn einem Mann im Schlaf
der Same entgeht, der soll sein ganzes Fleisch mit Wasser baden und
unrein sein bis auf den Abend. \bibverse{17} Und alles Kleid und alles
Fell, das mit solchem Samen befleckt ist, soll er waschen mit Wasser und
unrein sein bis auf den Abend. \bibverse{18} Ein Weib, bei welchem ein
solcher liegt, die soll sich mit Wasser baden und unrein sein bis auf
den Abend. \bibverse{19} Wenn ein Weib ihres Leibes Blutfluß hat, die
soll sieben Tage unrein geachtet werden; wer sie anrührt, der wird
unrein sein bis auf den Abend. \bibverse{20} Und alles, worauf sie
liegt, solange sie ihre Zeit hat, und worauf sie sitzt, wird unrein
sein. \bibverse{21} Und wer ihr Lager anrührt, der soll seine Kleider
waschen und sich mit Wasser baden und unrein sein bis auf den Abend.
\bibverse{22} Und wer anrührt irgend etwas, darauf sie gesessen hat,
soll seine Kleider waschen und sich mit Wasser baden und unrein sein bis
auf den Abend. \bibverse{23} Und wer anrührt irgend etwas, das auf ihrem
Lager gewesen ist oder da, wo sie gesessen hat soll unrein sein bis auf
den Abend. \bibverse{24} Und wenn ein Mann bei ihr liegt und es kommt
sie ihre Zeit an bei ihm, der wird sieben Tage unrein sein, und das
Lager, auf dem er gelegen hat wird unrein sein. \bibverse{25} Wenn aber
ein Weib den Blutfluß eine lange Zeit hat, zu ungewöhnlicher Zeit oder
über die gewöhnliche Zeit, so wird sie unrein sein, solange sie ihn hat;
wie zu ihrer gewöhnlichen Zeit, so soll sie auch da unrein sein.
\bibverse{26} Alles Lager, darauf sie liegt die ganze Zeit ihres Flußes,
soll sein wie ihr Lager zu ihrer gewöhnlichen Zeit. Und alles, worauf
sie sitzt, wird unrein sein gleich der Unreinigkeit ihrer gewöhnlichen
Zeit. \bibverse{27} Wer deren etwas anrührt, der wird unrein sein und
soll seine Kleider waschen und sich mit Wasser baden und unrein sein bis
auf den Abend. \bibverse{28} Wird sie aber rein von ihrem Fluß, so soll
sie sieben Tage zählen; darnach soll sie rein sein. \bibverse{29} Und am
achten Tage soll sie zwei Turteltauben oder zwei junge Tauben nehmen und
zum Priester bringen vor die Tür der Hütte des Stifts. \bibverse{30} Und
der Priester soll aus einer machen ein Sündopfer, aus der andern ein
Brandopfer, und sie versöhnen vor dem HERRN über den Fluß ihrer
Unreinigkeit. \bibverse{31} So sollt ihr die Kinder Israel warnen vor
ihrer Unreinigkeit, daß sie nicht sterben in ihrer Unreinigkeit, wenn
sie meine Wohnung verunreinigen, die unter ihnen ist. \bibverse{32} Das
ist das Gesetz über den, der einen Fluß hat und dem der Same im Schlaf
entgeht, daß er unrein davon wird, \bibverse{33} und über die, die ihren
Blutfluß hat, und wer einen Fluß hat, es sei Mann oder Weib, und wenn
ein Mann bei einer Unreinen liegt.

\hypertarget{section-15}{%
\section{16}\label{section-15}}

\bibverse{1} Und der HERR redete mit Mose, nachdem die zwei Söhne Aarons
gestorben waren, da sie vor dem HERRN opferten, \bibverse{2} und sprach:
Sage deinem Bruder Aaron, daß er nicht zu aller Zeit in das inwendige
Heiligtum gehe hinter den Vorhang vor den Gnadenstuhl, der auf der Lade
ist, daß er nicht sterbe; denn ich will in einer Wolke erscheinen auf
dem Gnadenstuhl; \bibverse{3} sondern damit soll er hineingehen: mit
einem jungen Farren zum Sündopfer und mit einem Widder zum Brandopfer,
\bibverse{4} und soll den heiligen leinenen Rock anlegen und leinene
Beinkleider an seinem Fleisch haben und sich mit einem leinenen Gürtel
gürten und den leinenen Hut aufhaben, denn das sind die heiligen
Kleider, und soll sein Fleisch mit Wasser baden und sie anlegen.
\bibverse{5} Und soll von der Gemeinde der Kinder Israel zwei
Ziegenböcke nehmen zum Sündopfer und einen Widder zum Brandopfer.
\bibverse{6} Und Aaron soll den Farren, sein Sündopfer, herzubringen,
daß er sich und sein Haus versöhne, \bibverse{7} und darnach die zwei
Böcke nehmen und vor den HERRN stellen vor der Tür der Hütte des Stifts,
\bibverse{8} und soll das Los werfen über die zwei Böcke: ein Los dem
HERRN, das andere dem Asasel. \bibverse{9} Und soll den Bock, auf
welchen das Los des HERRN fällt, opfern zum Sündopfer. \bibverse{10}
Aber den Bock, auf welchen das Los für Asasel fällt, soll er lebendig
vor den HERRN stellen, daß er über ihm versöhne, und lasse den Bock für
Asasel in die Wüste. \bibverse{11} Und also soll er denn den Farren
seines Sündopfers herzubringen und sich und sein Haus versöhnen und soll
ihn schlachten \bibverse{12} und soll einen Napf voll Glut vom Altar
nehmen, der vor dem HERRN steht, und die Hand voll zerstoßenen
Räuchwerks und es hinein hinter den Vorhang bringen \bibverse{13} und
das Räuchwerk aufs Feuer tun vor dem HERRN, daß der Nebel vom Räuchwerk
den Gnadenstuhl bedecke, der auf dem Zeugnis ist, daß er nicht sterbe.
\bibverse{14} Und soll von dem Blut des Farren nehmen und es mit seinem
Finger auf den Gnadenstuhl sprengen vornean; vor den Gnadenstuhl aber
soll er siebenmal mit seinem Finger vom Blut sprengen. \bibverse{15}
Darnach soll er den Bock, des Volkes Sündopfer, schlachten und sein Blut
hineinbringen hinter den Vorhang und soll mit seinem Blut tun, wie er
mit des Farren Blut getan hat, und damit auch sprengen auf den
Gnadenstuhl und vor den Gnadenstuhl; \bibverse{16} und soll also
versöhnen das Heiligtum von der Unreinigkeit der Kinder Israel und von
ihrer Übertretung in allen ihren Sünden. Also soll er auch tun der Hütte
des Stifts; denn sie sind unrein, die umher lagern. \bibverse{17} Kein
Mensch soll in der Hütte des Stifts sein, wenn er hineingeht, zu
versöhnen im Heiligtum, bis er herausgehe; und soll also versöhnen sich
und sein Haus und die ganze Gemeinde Israel. \bibverse{18} Und wenn er
herausgeht zum Altar, der vor dem HERRN steht, soll er ihn versöhnen und
soll vom Blut des Farren und vom Blut des Bocks nehmen und es auf des
Altars Hörner umher tun; \bibverse{19} und soll mit seinem Finger vom
Blut darauf sprengen siebenmal und ihn reinigen und heiligen von der
Unreinigkeit der Kinder Israel. \bibverse{20} Und wenn er vollbracht hat
das Versöhnen des Heiligtums und der Hütte des Stifts und des Altars, so
soll er den lebendigen Bock herzubringen. \bibverse{21} Da soll Aaron
seine beiden Hände auf sein Haupt legen und bekennen auf ihn alle
Missetat der Kinder Israel und alle ihre Übertretung in allen ihren
Sünden, und soll sie dem Bock auf das Haupt legen und ihn durch einen
Mann, der bereit ist, in die Wüste laufen lassen, \bibverse{22} daß also
der Bock alle ihre Missetat auf sich in eine Wildnis trage; und er lasse
ihn in die Wüste. \bibverse{23} Und Aaron soll in die Hütte des Stifts
gehen und ausziehen die leinenen Kleider, die er anzog, da er in das
Heiligtum ging, und soll sie daselbst lassen. \bibverse{24} Und soll
sein Fleisch mit Wasser baden an heiliger Stätte und seine eigenen
Kleider antun und herausgehen und sein Brandopfer und des Volkes
Brandopfer machen und beide, sich und das Volk, versöhnen, \bibverse{25}
und das Fett vom Sündopfer auf dem Altar anzünden. \bibverse{26} Der
aber den Bock für Asasel hat ausgeführt, soll seine Kleider waschen und
sein Fleisch mit Wasser baden und darnach ins Lager kommen.
\bibverse{27} Den Farren des Sündopfers und den Bock des Sündopfers,
deren Blut in das Heiligtum zu versöhnen gebracht ward, soll man
hinausschaffen vor das Lager und mit Feuer verbrennen, Haut, Fleisch und
Mist. \bibverse{28} Und der sie verbrennt, soll seine Kleider waschen
und sein Fleisch mit Wasser baden und darnach ins Lager kommen.
\bibverse{29} Auch soll euch das ein ewiges Recht sein: am zehnten Tage
des siebenten Monats sollt ihr euren Leib kasteien und kein Werk tun,
weder ein Einheimischer noch ein Fremder unter euch. \bibverse{30} Denn
an diesem Tage geschieht eure Versöhnung, daß ihr gereinigt werdet; von
allen euren Sünden werdet ihr gereinigt vor dem HERRN. \bibverse{31}
Darum soll's euch ein großer Sabbat sein, und ihr sollt euren Leib
kasteien. Ein ewiges Recht sei das. \bibverse{32} Es soll aber solche
Versöhnung tun ein Priester, den man geweiht und des Hand man gefüllt
hat zum Priester an seines Vaters Statt; und er soll die leinenen
Kleider antun, die heiligen Kleider, \bibverse{33} und soll also
versöhnen das heiligste Heiligtum und die Hütte des Stifts und den Altar
und die Priester und alles Volk der Gemeinde. \bibverse{34} Das soll
euch ein ewiges Recht sein, daß ihr die Kinder Israel versöhnt von allen
ihren Sünden, im Jahr einmal. Und Aaron tat, wie der HERR dem Mose
geboten hatte.

\hypertarget{section-16}{%
\section{17}\label{section-16}}

\bibverse{1} Und der HERR redete mit Mose und sprach: \bibverse{2} Sage
Aaron und seinen Söhnen und allen Kindern Israel und sprich zu ihnen:
Das ist's, was der HERR geboten hat. \bibverse{3} Welcher aus dem Haus
Israel einen Ochsen oder Lamm oder Ziege schlachtet, in dem Lager oder
draußen vor dem Lager, \bibverse{4} und es nicht vor die Tür der Hütte
des Stifts bringt, daß es dem HERRN zum Opfer gebracht werde vor der
Wohnung des HERRN, der soll des Blutes schuldig sein als der Blut
vergossen hat, und solcher Mensch soll ausgerottet werden aus seinem
Volk. \bibverse{5} Darum sollen die Kinder Israel ihre Schlachttiere,
die sie auf dem freien Feld schlachten wollen, vor den HERRN bringen vor
die Tür der Hütte des Stifts zum Priester und allda ihre Dankopfer dem
HERRN opfern. \bibverse{6} Und der Priester soll das Blut auf den Altar
des HERRN sprengen vor der Tür der Hütte des Stifts und das Fett
anzünden zum süßen Geruch dem HERRN. \bibverse{7} Und mitnichten sollen
sie ihre Opfer hinfort den Feldteufeln opfern, mit denen sie Abgötterei
treiben. Das soll ihnen ein ewiges Recht sein bei ihren Nachkommen.
\bibverse{8} Darum sollst du zu ihnen sagen: Welcher Mensch aus dem
Hause Israel oder auch ein Fremdling, der unter euch ist, ein Opfer oder
Brandopfer tut \bibverse{9} und bringt's nicht vor die Tür der Hütte des
Stifts, daß er's dem HERRN tue, der soll ausgerottet werden von seinem
Volk. \bibverse{10} Und welcher Mensch, er sei vom Haus Israel oder ein
Fremdling unter euch, irgend Blut ißt, wider den will ich mein Antlitz
setzen und will ihn mitten aus seinem Volk ausrotten. \bibverse{11} Denn
des Leibes Leben ist im Blut, und ich habe es euch auf den Altar
gegeben, daß eure Seelen damit versöhnt werden. Denn das Blut ist die
Versöhnung, weil das Leben in ihm ist. \bibverse{12} Darum habe ich
gesagt den Kindern Israel: Keine Seele unter euch soll Blut essen, auch
kein Fremdling, der unter euch wohnt. \bibverse{13} Und welcher Mensch,
er sei vom Haus Israel oder ein Fremdling unter euch, ein Tier oder
einen Vogel fängt auf der Jagd, das man ißt, der soll desselben Blut
hingießen und mit Erde zuscharren. \bibverse{14} Denn des Leibes Leben
ist in seinem Blut, solange es lebt; und ich habe den Kindern Israel
gesagt: Ihr sollt keines Leibes Blut essen; denn des Leibes Leben ist in
seinem Blut; wer es ißt, der soll ausgerottet werden. \bibverse{15} Und
welche Seele ein Aas oder was vom Wild zerrissen ist, ißt, er sei ein
Einheimischer oder Fremdling, der soll sein Kleid waschen und sich mit
Wasser baden und unrein sein bis auf den Abend, so wird er rein.
\bibverse{16} Wo er seine Kleider nicht waschen noch sich baden wird, so
soll er seiner Missetat schuldig sein.

\hypertarget{section-17}{%
\section{18}\label{section-17}}

\bibverse{1} Und der HERR redete mit Mose und sprach: \bibverse{2} Rede
mit den Kindern Israel und sprich zu ihnen: Ich bin der HERR, euer Gott.
\bibverse{3} Ihr sollt nicht tun nach den Werken des Landes Ägypten,
darin ihr gewohnt habt, auch nicht nach den Werken des Landes Kanaan,
darein ich euch führen will; ihr sollt auch nach ihrer Weise nicht
halten; \bibverse{4} sondern nach meinen Rechten sollt ihr tun, und
meine Satzungen sollt ihr halten, daß ihr darin wandelt; denn ich bin
der HERR, euer Gott. \bibverse{5} Darum sollt ihr meine Satzungen halten
und meine Rechte. Denn welcher Mensch dieselben tut, der wird dadurch
leben; denn ich bin der HERR. \bibverse{6} Niemand soll sich zu seiner
nächsten Blutsfreundin tun, ihre Blöße aufzudecken; denn ich bin der
HERR. \bibverse{7} Du sollst deines Vaters und deiner Mutter Blöße nicht
aufdecken; es ist deine Mutter, darum sollst du ihre Blöße nicht
aufdecken. \bibverse{8} Du sollst deines Vaters Weibes Blöße nicht
aufdecken; denn sie ist deines Vaters Blöße. \bibverse{9} Du sollst
deiner Schwester Blöße, die deines Vaters oder deiner Mutter Tochter
ist, daheim oder draußen geboren, nicht aufdecken. \bibverse{10} Du
sollst die Blöße der Tochter deines Sohnes oder deiner Tochter nicht
aufdecken; denn es ist deine Blöße. \bibverse{11} Du sollst die Blöße
der Tochter deines Vaters Weibes, die deinem Vater geboren ist und deine
Schwester ist, nicht aufdecken. \bibverse{12} Do sollst die Blöße der
Schwester deines Vaters nicht aufdecken; denn es ist deines Vaters
nächste Blutsfreundin. \bibverse{13} Du sollst deiner Mutter Schwester
Blöße nicht aufdecken; denn es ist deiner Mutter nächste Blutsfreundin.
\bibverse{14} Du sollst deines Vaters Bruders Blöße nicht aufdecken, daß
du sein Weib nehmest; denn sie ist deine Base. \bibverse{15} Du sollst
deiner Schwiegertochter Blöße nicht aufdecken; denn es ist deines Sohnes
Weib, darum sollst du ihre Blöße nicht aufdecken. \bibverse{16} Du
sollst deines Bruders Weibes Blöße nicht aufdecken; denn sie ist deines
Bruders Blöße. \bibverse{17} Du sollst eines Weibes samt ihrer Tochter
Blöße nicht aufdecken noch ihres Sohnes Tochter oder ihrer Tochter
Tochter nehmen, ihre Blöße aufzudecken; denn sie sind ihre nächsten
Blutsfreundinnen, und es ist ein Frevel. \bibverse{18} Du sollst auch
deines Weibes Schwester nicht nehmen neben ihr, ihre Blöße aufzudecken,
ihr zuwider, solange sie noch lebt. \bibverse{19} Du sollst nicht zum
Weibe gehen, solange sie ihre Krankheit hat, in ihrer Unreinigkeit ihre
Blöße aufzudecken. \bibverse{20} Du sollst auch nicht bei deines
Nächsten Weibe liegen, dadurch du dich an ihr verunreinigst.
\bibverse{21} Du sollst auch nicht eines deiner Kinder dahingeben, daß
es dem Moloch verbrannt werde, daß du nicht entheiligst den Namen deines
Gottes; denn ich bin der HERR. \bibverse{22} Du sollst nicht beim Knaben
liegen wie beim Weibe; denn es ist ein Greuel. \bibverse{23} Du sollst
auch bei keinem Tier liegen, daß du mit ihm verunreinigt werdest. Und
kein Weib soll mit einem Tier zu schaffen haben; denn es ist ein Greuel.
\bibverse{24} Ihr sollt euch in dieser keinem verunreinigen; denn in
diesem allem haben sich verunreinigt die Heiden, die ich vor euch her
will ausstoßen, \bibverse{25} und das Land ist dadurch verunreinigt. Und
ich will ihre Missetat an ihnen heimsuchen, daß das Land seine Einwohner
ausspeie. \bibverse{26} Darum haltet meine Satzungen und Rechte, und tut
dieser Greuel keine, weder der Einheimische noch der Fremdling unter
euch; \bibverse{27} denn alle solche Greuel haben die Leute dieses
Landes getan, die vor euch waren, und haben das Land verunreinigt;
\bibverse{28} auf daß euch nicht auch das Land ausspeie, wenn ihr es
verunreinigt, gleich wie es die Heiden hat ausgespieen, die vor euch
waren. \bibverse{29} Denn welche diese Greuel tun, deren Seelen sollen
ausgerottet werden von ihrem Volk. \bibverse{30} Darum haltet meine
Satzungen, daß ihr nicht tut nach den greulichen Sitten, die vor euch
waren, daß ihr nicht damit verunreinigt werdet; denn ich bin der HERR,
euer Gott.

\hypertarget{section-18}{%
\section{19}\label{section-18}}

\bibverse{1} Und der HERR redete mit Mose und sprach: \bibverse{2} Rede
mit der ganzen Gemeinde der Kinder Israel und sprich zu ihnen: Ihr sollt
heilig sein; denn ich bin heilig, der HERR, euer Gott. \bibverse{3} Ein
jeglicher fürchte seine Mutter und seinen Vater. Haltet meine Feiertage;
denn ich bin der HERR, euer Gott. \bibverse{4} Ihr sollt euch nicht zu
den Götzen wenden und sollt euch keine gegossenen Götter machen; denn
ich bin der HERR, euer Gott. \bibverse{5} Und wenn ihr dem HERRN wollt
ein Dankopfer tun, so sollt ihr es opfern, daß es ihm gefallen könnte.
\bibverse{6} Ihr sollt es desselben Tages essen, da ihr's opfert, und
des andern Tages; was aber auf den dritten Tag übrigbleibt, soll man mit
Feuer verbrennen. \bibverse{7} Wird aber jemand am dritten Tage davon
essen, so ist er ein Greuel und wird nicht angenehm sein. \bibverse{8}
Und der Esser wird seine Missetat tragen, darum daß er das Heiligtum des
HERRN entheiligte, und solche Seele wird ausgerottet werden von ihrem
Volk. \bibverse{9} Wenn du dein Land einerntest, sollst du nicht alles
bis an die Enden umher abschneiden, auch nicht alles genau aufsammeln.
\bibverse{10} Also auch sollst du deinen Weinberg nicht genau lesen noch
die abgefallenen Beeren auflesen, sondern dem Armen und Fremdling sollst
du es lassen; denn ich bin der HERR euer Gott. \bibverse{11} Ihr sollt
nicht stehlen noch lügen noch fälschlich handeln einer mit dem andern.
\bibverse{12} Ihr sollt nicht falsch schwören bei meinem Namen und
entheiligen den Namen deines Gottes; denn ich bin der HERR.
\bibverse{13} Du sollst deinem Nächsten nicht unrecht tun noch ihn
berauben. Es soll des Tagelöhners Lohn nicht bei dir bleiben bis an den
Morgen. \bibverse{14} Du sollst dem Tauben nicht fluchen und sollst dem
Blinden keinen Anstoß setzen; denn du sollst dich vor deinem Gott
fürchten, denn ich bin der HERR. \bibverse{15} Ihr sollt nicht unrecht
handeln im Gericht, und sollst nicht vorziehen den Geringen noch den
Großen ehren; sondern du sollst deinen Nächsten recht richten.
\bibverse{16} Du sollst kein Verleumder sein unter deinem Volk. Du
sollst auch nicht stehen wider deines Nächsten Blut; denn ich bin der
HERR. \bibverse{17} Du sollst deinen Bruder nicht hassen in deinem
Herzen, sondern du sollst deinen Nächsten zurechtweisen, auf daß du
nicht seineshalben Schuld tragen müssest. \bibverse{18} Du sollst nicht
rachgierig sein noch Zorn halten gegen die Kinder deines Volks. Du
sollst deinen Nächsten lieben wie dich selbst; denn ich bin der HERR.
\bibverse{19} Meine Satzungen sollt ihr halten, daß du dein Vieh nicht
lassest mit anderlei Tier zu schaffen haben und dein Feld nicht besäest
mit mancherlei Samen und kein Kleid an dich komme, daß mit Wolle und
Leinen gemengt ist. \bibverse{20} Wenn ein Mann bei einem Weibe liegt,
die eine leibeigene Magd und von dem Mann verschmäht ist, doch nicht
erlöst noch Freiheit erlangt hat, das soll gestraft werden; aber sie
sollen nicht sterben, denn sie ist nicht frei gewesen. \bibverse{21} Er
soll aber für seine Schuld dem HERRN vor die Tür der Hütte des Stifts
einen Widder zum Schuldopfer bringen; \bibverse{22} und der Priester
soll ihn versöhnen mit dem Schuldopfer vor dem HERRN über die Sünde, die
er getan hat, so wird ihm Gott gnädig sein über seine Sünde, die er
getan hat. \bibverse{23} Wenn ihr in das Land kommt und allerlei Bäume
pflanzt, davon man ißt, sollt ihr mit seinen Früchten tun wie mit einer
Vorhaut. Drei Jahre sollt ihr sie unbeschnitten achten, daß ihr sie
nicht esset; \bibverse{24} im vierten Jahr aber sollen alle ihre Früchte
heilig sein, ein Preisopfer dem HERRN; \bibverse{25} im fünften Jahr
aber sollt ihr die Früchte essen und sie einsammeln; denn ich bin der
HERR, euer Gott. \bibverse{26} Ihr sollt nichts vom Blut essen. Ihr
sollt nicht auf Vogelgeschrei achten noch Tage wählen. \bibverse{27} Ihr
sollt euer Haar am Haupt nicht rundumher abschneiden noch euren Bart gar
abscheren. \bibverse{28} Ihr sollt kein Mal um eines Toten willen an
eurem Leibe reißen noch Buchstaben an euch ätzen; denn ich bin der HERR.
\bibverse{29} Du sollst deine Tochter nicht zur Hurerei halten, daß
nicht das Land Hurerei treibe und werde voll Lasters. \bibverse{30}
Meine Feiertage haltet, und fürchtet euch vor meinem Heiligtum; denn ich
bin der HERR. \bibverse{31} Ihr sollt euch nicht wenden zu den
Wahrsagern, und forscht nicht von den Zeichendeutern, daß ihr nicht an
ihnen verunreinigt werdet; denn ich bin der HERR, euer Gott.
\bibverse{32} Vor einem grauen Haupt sollst du aufstehen und die Alten
ehren; denn du sollst dich fürchten vor deinem Gott, denn ich bin der
HERR. \bibverse{33} Wenn ein Fremdling bei dir in eurem Lande wohnen
wird, den sollt ihr nicht schinden. \bibverse{34} Er soll bei euch
wohnen wie ein Einheimischer unter euch, und sollst ihn lieben wie dich
selbst; denn ihr seid auch Fremdlinge gewesen in Ägyptenland. Ich bin
der HERR, euer Gott. \bibverse{35} Ihr sollt nicht unrecht handeln im
Gericht mit der Elle, mit Gewicht, mit Maß. \bibverse{36} Rechte Waage,
rechte Pfunde, rechte Scheffel, rechte Kannen sollen bei euch sein; denn
ich bin der HERR, euer Gott, der euch aus Ägyptenland geführt hat,
\bibverse{37} daß ihr alle meine Satzungen und alle meine Rechte haltet
und tut; denn ich bin der HERR.

\hypertarget{section-19}{%
\section{20}\label{section-19}}

\bibverse{1} Und der HERR redete mit Mose und sprach: \bibverse{2} Sage
den Kindern Israel: Welcher unter den Kindern Israel oder ein Fremdling,
der in Israel wohnt, eines seiner Kinder dem Moloch gibt, der soll des
Todes sterben; das Volk im Lande soll ihn steinigen. \bibverse{3} Und
ich will mein Antlitz setzen wider solchen Menschen und will ihn aus
seinem Volk ausrotten, daß er dem Moloch eines seiner Kinder gegeben und
mein Heiligtum verunreinigt und meinen heiligen Namen entheiligt hat.
\bibverse{4} Und wo das Volk im Lande durch die Finger sehen würde dem
Menschen, der eines seiner Kinder dem Moloch gegeben hat, daß es ihn
nicht tötet, \bibverse{5} so will doch ich mein Antlitz wider denselben
Menschen setzen und wider sein Geschlecht und will ihn und alle, die mit
ihm mit dem Moloch Abgötterei getrieben haben, aus ihrem Volke
ausrotten. \bibverse{6} Wenn eine Seele sich zu den Wahrsagern und
Zeichendeutern wenden wird, daß sie ihnen nachfolgt, so will ich mein
Antlitz wider dieselbe Seele setzen und will sie aus ihrem Volk
ausrotten. \bibverse{7} Darum heiligt euch und seid heilig; denn ich bin
der HERR, euer Gott. \bibverse{8} Und haltet meine Satzungen und tut
sie; denn ich bin der HERR, der euch heiligt. \bibverse{9} Wer seinem
Vater oder seiner Mutter flucht, der soll des Todes sterben. Sein Blut
sei auf ihm, daß er seinem Vater oder seiner Mutter geflucht hat.
\bibverse{10} Wer die Ehe bricht mit jemandes Weibe, der soll des Todes
sterben, beide, Ehebrecher und Ehebrecherin, darum daß er mit seines
Nächsten Weibe die Ehe gebrochen hat. \bibverse{11} Wenn jemand bei
seines Vaters Weibe schläft, daß er seines Vater Blöße aufgedeckt hat,
die sollen beide des Todes sterben; ihr Blut sei auf ihnen.
\bibverse{12} Wenn jemand bei seiner Schwiegertochter schläft, so sollen
sie beide des Todes sterben; ihr Blut sei auf ihnen. \bibverse{13} Wenn
jemand beim Knaben schläft wie beim Weibe, die haben einen Greuel getan
und sollen beide des Todes sterben; ihr Blut sei auf ihnen.
\bibverse{14} Wenn jemand ein Weib nimmt und ihre Mutter dazu, der hat
einen Frevel verwirkt; man soll ihn mit Feuer verbrennen und sie beide
auch, daß kein Frevel sei unter euch. \bibverse{15} Wenn jemand beim
Vieh liegt, der soll des Todes sterben, und das Vieh soll man erwürgen.
\bibverse{16} Wenn ein Weib sich irgend zu einem Vieh tut, daß sie mit
ihm zu schaffen hat, die sollst du töten und das Vieh auch; des Todes
sollen sie sterben; ihr Blut sei auf ihnen. \bibverse{17} Wenn jemand
seine Schwester nimmt, seines Vaters Tochter oder seiner Mutter Tochter,
und ihre Blöße schaut und sie wieder seine Blöße, das ist Blutschande.
Die sollen ausgerottet werden vor den Leuten ihres Volks; denn er hat
seiner Schwester Blöße aufgedeckt; er soll seine Missetat tragen.
\bibverse{18} Wenn ein Mann beim Weibe schläft zur Zeit ihrer Krankheit
und entblößt ihre Scham und deckt ihren Brunnen auf, und entblößt den
Brunnen ihres Bluts, die sollen beide aus ihrem Volk ausgerottet werden.
\bibverse{19} Deiner Mutter Schwester Blöße und deines Vater Schwester
Blöße sollst du nicht aufdecken; denn ein solcher hat seine nächste
Blutsfreundin aufgedeckt, und sie sollen ihre Missetat tragen.
\bibverse{20} Wenn jemand bei seines Vaters Bruders Weibe schläft, der
hat seines Oheims Blöße aufgedeckt. Sie sollen ihre Sünde tragen; ohne
Kinder sollen sie sterben. \bibverse{21} Wenn jemand seines Bruders Weib
nimmt, das ist eine schändliche Tat; sie sollen ohne Kinder sein, darum
daß er seines Bruders Blöße aufgedeckt hat. \bibverse{22} So haltet nun
alle meine Satzungen und meine Rechte und tut darnach, auf daß euch das
Land nicht ausspeie, darein ich euch führe, daß ihr darin wohnt.
\bibverse{23} Und wandelt nicht in den Satzungen der Heiden, die ich vor
euch her werde ausstoßen. Denn solches alles haben sie getan, und ich
habe einen Greuel an ihnen gehabt. \bibverse{24} Euch aber sage ich: Ihr
sollt jener Land besitzen; denn ich will euch ein Land zum Erbe geben,
darin Milch und Honig fließt. Ich bin der HERR, euer Gott, der euch von
allen Völkern abgesondert hat, \bibverse{25} daß ihr auch absondern
sollt das reine Vieh vom unreinen und unreine Vögel von den reinen, und
eure Seelen nicht verunreinigt am Vieh, an Vögeln und an allem, was auf
Erden kriecht, das ich euch abgesondert habe, daß es euch unrein sei.
\bibverse{26} Darum sollt ihr mir heilig sein; denn ich, der HERR, bin
heilig, der euch abgesondert hat von den Völkern, daß ihr mein wäret.
\bibverse{27} Wenn ein Mann oder Weib ein Wahrsager oder Zeichendeuter
sein wird, die sollen des Todes sterben. Man soll sie steinigen; ihr
Blut sei auf ihnen.

\hypertarget{section-20}{%
\section{21}\label{section-20}}

\bibverse{1} Und der HERR sprach zu Mose: Sage den Priestern, Aarons
Söhnen, und sprich zu ihnen: Ein Priester soll sich an keinem Toten
seines Volkes verunreinigen, \bibverse{2} außer an seinem Blutsfreunde,
der ihm am nächsten angehört, als: an seiner Mutter, an seinem Vater, an
seinem Sohne, an seiner Tochter, an seinem Bruder \bibverse{3} und an
seiner Schwester, die noch eine Jungfrau und noch bei ihm ist und keines
Mannes Weib gewesen ist; an der mag er sich verunreinigen. \bibverse{4}
Sonst soll er sich nicht verunreinigen an irgend einem, der ihm zugehört
unter seinem Volk, daß er sich entheilige. \bibverse{5} Sie sollen auch
keine Platte machen auf ihrem Haupt noch ihren Bart abscheren und an
ihrem Leib kein Mal stechen. \bibverse{6} Sie sollen ihrem Gott heilig
sein und nicht entheiligen den Namen ihres Gottes. Denn sie opfern des
HERRN Opfer, das Brot ihres Gottes; darum sollen sie heilig sein.
\bibverse{7} Sie sollen keine Hure nehmen noch eine Geschwächte oder die
von ihrem Mann verstoßen ist; denn er ist heilig seinem Gott.
\bibverse{8} Darum sollst du ihn heilig halten, denn er opfert das Brot
deines Gottes; er soll dir heilig sein, denn ich bin heilig, der HERR,
der euch heiligt. \bibverse{9} Wenn eines Priesters Tochter anfängt zu
huren, die soll man mit Feuer verbrennen; denn sie hat ihren Vater
geschändet. \bibverse{10} Wer Hoherpriester ist unter seinen Brüdern,
auf dessen Haupt das Salböl gegossen und dessen Hand gefüllt ist, daß er
angezogen würde mit den Kleidern, der soll sein Haupt nicht entblößen
und seine Kleider nicht zerreißen \bibverse{11} und soll zu keinem Toten
kommen und soll sich weder über Vater noch über Mutter verunreinigen.
\bibverse{12} Aus dem Heiligtum soll er nicht gehen, daß er nicht
entheilige das Heiligtum seines Gottes; denn die Weihe des Salböls
seines Gottes ist auf ihm. Ich bin der HERR. \bibverse{13} Eine Jungfrau
soll er zum Weibe nehmen; \bibverse{14} aber keine Witwe noch Verstoßene
noch Geschwächte noch Hure, sondern eine Jungfrau seines Volks soll er
zum Weibe nehmen, \bibverse{15} auf daß er nicht seinen Samen entheilige
unter seinem Volk; denn ich bin der HERR, der ihn heiligt. \bibverse{16}
Und der HERR redete mit Mose und sprach: \bibverse{17} Rede mit Aaron
und sprich: Wenn an jemand deiner Nachkommen in euren Geschlechtern ein
Fehl ist, der soll nicht herzutreten, daß er das Brot seines Gottes
opfere. \bibverse{18} Denn keiner, an dem ein Fehl ist, soll
herzutreten; er sei blind, lahm, mit einer seltsamen Nase, mit
ungewöhnlichem Glied, \bibverse{19} oder der an einem Fuß oder einer
Hand gebrechlich ist \bibverse{20} oder höckerig ist oder ein Fell auf
dem Auge hat oder schielt oder den Grind oder Flechten hat oder der
gebrochen ist. \bibverse{21} Welcher nun von Aarons, des Priesters,
Nachkommen einen Fehl an sich hat, der soll nicht herzutreten, zu opfern
die Opfer des HERRN; denn er hat einen Fehl, darum soll er zu dem Brot
seines Gottes nicht nahen, daß er es opfere. \bibverse{22} Doch soll er
das Brot seines Gottes essen, von dem Heiligen und vom Hochheiligen.
\bibverse{23} Aber zum Vorhang soll er nicht kommen noch zum Altar
nahen, weil der Fehl an ihm ist, daß er nicht entheilige mein Heiligtum;
denn ich bin der HERR, der sie heiligt. \bibverse{24} Und Mose redete
solches zu Aaron und zu seinen Söhnen und zu allen Kindern Israel.

\hypertarget{section-21}{%
\section{22}\label{section-21}}

\bibverse{1} Und der HERR redete mit Mose und sprach: \bibverse{2} Sage
Aaron und seinen Söhnen, daß sie sich enthalten von dem Heiligen der
Kinder Israel, welches sie mir heiligen und meinen heiligen Namen nicht
entheiligen, denn ich bin der HERR. \bibverse{3} So sage nun ihnen auf
ihre Nachkommen: Welcher eurer Nachkommen herzutritt zum Heiligen, das
die Kinder Israel dem HERRN heiligen, und hat eine Unreinheit an sich,
des Seele soll ausgerottet werden von meinem Antlitz; denn ich bin der
HERR. \bibverse{4} Welcher der Nachkommen Aarons aussätzig ist oder
einen Fluß hat, der soll nicht essen vom Heiligen, bis er rein werde.
Wer etwa einen anrührt, der an einem Toten unrein geworden ist, oder
welchem der Same entgeht im Schlaf, \bibverse{5} und welcher irgend ein
Gewürm anrührt, dadurch er unrein wird, oder einen Menschen, durch den
er unrein wird, und alles, was ihn verunreinigt: \bibverse{6} welcher
der eins anrührt, der ist unrein bis auf den Abend und soll von dem
Heiligen nicht essen, sondern soll zuvor seinen Leib mit Wasser baden.
\bibverse{7} Und wenn die Sonne untergegangen und er rein geworden ist,
dann mag er davon essen; denn es ist seine Nahrung. \bibverse{8} Ein Aas
und was von wilden Tieren zerreißen ist, soll er nicht essen, auf daß er
nicht unrein daran werde; denn ich bin der HERR. \bibverse{9} Darum
sollen sie meine Sätze halten, daß sie nicht Sünde auf sich laden und
daran sterben, wenn sie sich entheiligen; denn ich bin der HERR, der sie
heiligt. \bibverse{10} Kein anderer soll von dem Heiligen essen noch des
Priesters Beisaß oder Tagelöhner. \bibverse{11} Wenn aber der Priester
eine Seele um sein Geld kauft, die mag davon essen; und was ihm in
seinem Hause geboren wird, das mag auch von seinem Brot essen.
\bibverse{12} Wenn aber des Priesters Tochter eines Fremden Weib wird,
die soll nicht von der heiligen Hebe essen. \bibverse{13} Wird sie aber
eine Witwe oder ausgestoßen und hat keine Kinder und kommt wieder zu
ihres Vaters Hause, so soll sie essen von ihres Vaters Brot, wie da sie
noch Jungfrau war. Aber kein Fremdling soll davon essen. \bibverse{14}
Wer sonst aus Versehen von dem Heiligen ißt der soll den fünften Teil
dazutun und dem Priester geben samt dem Heiligen, \bibverse{15} auf daß
sie nicht entheiligen das Heilige der Kinder Israel, das sie dem HERRN
heben, \bibverse{16} auf daß sie nicht mit Missetat und Schuld beladen,
wenn sie ihr Geheiligtes essen; denn ich bin der HERR, der sie heiligt.
\bibverse{17} Und der HERR redete mit Mose und sprach: \bibverse{18}
Sage Aaron und seinen Söhnen und allen Kindern Israel: Welcher
Israeliter oder Fremdling in Israel sein Opfer tun will, es sei ein
Gelübde oder von freiem Willen, daß sie dem HERRN ein Brandopfer tun
wollen, das ihm von euch angenehm sei, \bibverse{19} das soll ein
Männlein und ohne Fehl sein, von Rindern oder Lämmern oder Ziegen.
\bibverse{20} Alles, was ein Fehl hat, sollt ihr nicht opfern; denn es
wird von euch nicht angenehm sein. \bibverse{21} Und wer ein Dankopfer
dem HERRN tun will, ein besonderes Gelübde oder von freiem Willen, von
Rindern oder Schafen, das soll ohne Gebrechen sein, daß es angenehm sei;
es soll keinen Fehl haben. \bibverse{22} Ist's blind oder gebrechlich
oder geschlagen oder dürr oder räudig oder hat es Flechten, so sollt ihr
solches dem HERRN nicht opfern und davon kein Opfer geben auf den Altar
des HERRN. \bibverse{23} Einen Ochsen oder Schaf, die zu lange oder zu
kurze Glieder haben, magst du von freiem Willen opfern; aber angenehm
mag's nicht sein zum Gelübde. \bibverse{24} Du sollst auch dem HERRN
kein zerstoßenes oder zerriebenes oder zerrissenes oder das
ausgeschnitten ist, opfern, und sollt im Lande solches nicht tun.
\bibverse{25} Du sollst auch solcher keins von eines Fremdlings Hand als
Brot eures Gottes opfern; denn es taugt nicht und hat einen Fehl; darum
wird's nicht angenehm sein von euch. \bibverse{26} Und der HERR redete
mit Mose und sprach: \bibverse{27} Wenn ein Ochs oder Lamm oder Ziege
geboren ist, so soll es sieben Tage bei seiner Mutter sein, und am
achten Tage und darnach mag man's dem HERRN opfern, so ist's angenehm.
\bibverse{28} Es sei ein Ochs oder Schaf, so soll man's nicht mit seinem
Jungen auf einen Tag schlachten. \bibverse{29} Wenn ihr aber wollt dem
HERRN ein Lobopfer tun, das von euch angenehm sei, \bibverse{30} so
sollt ihr's desselben Tages essen und sollt nichts übrig bis auf den
Morgen behalten; denn ich bin der HERR. \bibverse{31} Darum haltet meine
Gebote und tut darnach; denn ich bin der HERR. \bibverse{32} Daß ihr
meinen heiligen Namen nicht entheiligt, und ich geheiligt werde unter
den Kindern Israel; denn ich bin der HERR, der euch heiligt,
\bibverse{33} der euch aus Ägyptenland geführt hat, daß ich euer Gott
wäre, ich, der HERR.

\hypertarget{section-22}{%
\section{23}\label{section-22}}

\bibverse{1} Und der HERR redete mit Mose und sprach: \bibverse{2} Sage
den Kindern Israel und sprich zu ihnen: Das sind die Feste des HERRN,
die ihr heilig und meine Feste heißen sollt, da ihr zusammenkommt.
\bibverse{3} Sechs Tage sollst du arbeiten; der siebente Tag aber ist
der große, heilige Sabbat, da ihr zusammenkommt. Keine Arbeit sollt ihr
an dem tun; denn es ist der Sabbat des HERRN in allen euren Wohnungen.
\bibverse{4} Dies sind aber die Feste des HERRN, die ihr die heiligen
Feste heißen sollt, da ihr zusammenkommt. \bibverse{5} Am vierzehnten
Tage des ersten Monats gegen Abend ist des HERRN Passah. \bibverse{6}
Und am fünfzehnten desselben Monats ist das Fest der ungesäuerten Brote
des HERRN; da sollt ihr sieben Tage ungesäuertes Brot essen.
\bibverse{7} Der erste Tag soll heilig unter euch heißen, da ihr
zusammenkommt; da sollt ihr keine Dienstarbeit tun. \bibverse{8} Und
sieben Tage sollt ihr dem HERRN opfern. Der siebente Tag soll auch
heilig heißen, da ihr zusammenkommt; da sollt ihr auch keine
Dienstarbeit tun. \bibverse{9} Und der HERR redete mit Mose und sprach:
\bibverse{10} Sage den Kindern Israel und sprich zu ihnen: Wenn ihr in
das Land kommt, das ich euch geben werde, und werdet's ernten, so sollt
ihr eine Garbe der Erstlinge eurer Ernte zu dem Priester bringen.
\bibverse{11} Da soll die Garbe gewebt werden vor dem HERRN, daß es von
euch angenehm sei; solches soll aber der Priester tun des Tages nach dem
Sabbat. \bibverse{12} Und ihr sollt des Tages, da eure Garbe gewebt
wird, ein Brandopfer dem HERRN tun von einem Lamm, das ohne Fehl und
jährig sei, \bibverse{13} samt dem Speisopfer: zwei Zehntel Semmelmehl,
mit Öl gemengt, als ein Opfer dem HERRN zum süßen Geruch; dazu das
Trankopfer: ein viertel Hin Wein. \bibverse{14} Und sollt kein neues
Brot noch geröstete oder frische Körner zuvor essen bis auf den Tag, da
ihr eurem Gott Opfer bringt. Das soll ein Recht sein euren Nachkommen in
allen euren Wohnungen. \bibverse{15} Darnach sollt ihr Zählen vom Tage
nach dem Sabbat, da ihr die Webegarbe brachtet, sieben ganze Wochen;
\bibverse{16} bis an den Tag nach dem siebenten Sabbat, nämlich fünfzig
Tage, sollt ihr zählen und neues Speisopfer dem HERRN opfern,
\bibverse{17} und sollt's aus euren Wohnungen opfern, nämlich zwei
Webebrote von zwei Zehntel Semmelmehl, gesäuert und gebacken, zu
Erstlingen dem HERRN. \bibverse{18} Und sollt herzubringen neben eurem
Brot sieben jährige Lämmer ohne Fehl und einen jungen Farren und zwei
Widder, die sollen des HERRN Brandopfer sein, mit ihrem Speisopfern und
Trankopfern, ein Opfer eines süßen Geruchs dem HERRN. \bibverse{19} Dazu
sollt ihr machen einen Ziegenbock zum Sündopfer und zwei jährige Lämmer
zum Dankopfer. \bibverse{20} Und der Priester soll's weben samt den
Erstlingsbroten vor dem HERRN; die sollen samt den zwei Lämmern dem
HERRN heilig sein und dem Priester gehören. \bibverse{21} Und sollt
diesen Tag ausrufen; denn er soll unter euch heilig heißen, da ihr
zusammenkommt; keine Dienstarbeit sollt ihr tun. Ein ewiges Recht soll
das sein bei euren Nachkommen in allen euren Wohnungen. \bibverse{22}
Wenn ihr aber euer Land erntet sollt ihr nicht alles bis an die Enden
des Feldes abschneiden, auch nicht alles genau auflesen, sondern sollt's
den Armen und Fremdlingen lassen. Ich bin der HERR, euer Gott.
\bibverse{23} Und der HERR redete mit Mose und sprach: \bibverse{24}
Rede mit den Kindern Israel und sprich: Am ersten Tage des siebenten
Monats sollt ihr den heiligen Sabbat des Blasens zum Gedächtnis halten,
da ihr zusammenkommt; \bibverse{25} da sollt ihr keine Dienstarbeit tun
und sollt dem HERRN opfern. \bibverse{26} Und der HERR redete mit Mose
und sprach: \bibverse{27} Des zehnten Tages in diesem siebenten Monat
ist der Versöhnungstag. Der soll bei euch heilig heißen, daß ihr
zusammenkommt; da sollt ihr euren Leib kasteien und dem HERRN opfern
\bibverse{28} und sollt keine Arbeit tun an diesem Tage; denn es ist der
Versöhnungstag, daß ihr versöhnt werdet vor dem HERRN, eurem Gott.
\bibverse{29} Denn wer seinen Leib nicht kasteit an diesem Tage, der
soll aus seinem Volk ausgerottet werden. \bibverse{30} Und wer dieses
Tages irgend eine Arbeit tut, den will ich vertilgen aus seinem Volk.
\bibverse{31} Darum sollt ihr keine Arbeit tun. Das soll ein ewiges
Recht sein euren Nachkommen in allen ihren Wohnungen. \bibverse{32} Es
ist euer großer Sabbat, daß ihr eure Leiber kasteit. Am neunten Tage des
Monats zu Abend sollt ihr diesen Sabbat halten, von Abend bis wieder zu
Abend. \bibverse{33} Und der HERR redete mit Mose und sprach:
\bibverse{34} Rede mit den Kindern Israel und sprich: Am fünfzehnten
Tage dieses siebenten Monats ist das Fest der Laubhütten sieben Tage dem
HERRN. \bibverse{35} Der erste Tag soll heilig heißen, daß ihr
zusammenkommt; keine Dienstarbeit sollt ihr tun. \bibverse{36} Sieben
Tage sollt ihr dem HERRN opfern. Der achte Tag soll auch heilig heißen,
daß ihr zusammenkommt, und sollt eure Opfer dem HERRN tun; denn es ist
der Tag der Versammlung; keine Dienstarbeit sollt ihr tun. \bibverse{37}
Das sind die Feste des HERRN, die ihr sollt für heilig halten, daß ihr
zusammenkommt und dem HERRN Opfer tut: Brandopfer, Speisopfer,
Trankopfer und andere Opfer, ein jegliches nach seinem Tage,
\bibverse{38} außer was die Sabbate des HERRN und eure Gaben und Gelübde
und freiwillige Gaben sind, die ihr dem HERRN gebt. \bibverse{39} So
sollt ihr nun am fünfzehnten Tage des siebenten Monats, wenn ihr die
Früchte des Landes eingebracht habt, das Fest des HERRN halten sieben
Tage lang. Am ersten Tage ist es Sabbat, und am achten Tage ist es auch
Sabbat. \bibverse{40} Und sollt am ersten Tage Früchte nehmen von
schönen Bäumen, Palmenzweige und Maien von dichten Bäumen und Bachweiden
und sieben Tage fröhlich sein vor dem HERRN, eurem Gott. \bibverse{41}
Und sollt also dem HERRN das Fest halten sieben Tage des Jahres. Das
soll ein ewiges Recht sein bei euren Nachkommen, daß sie im siebenten
Monat also feiern. \bibverse{42} Sieben Tage sollt ihr in Laubhütten
wohnen; wer einheimisch ist in Israel, der soll in Laubhütten wohnen,
\bibverse{43} daß eure Nachkommen wissen, wie ich die Kinder Israel habe
lassen in Hütten wohnen, da ich sie aus Ägyptenland führte. Ich bin der
HERR, euer Gott. \bibverse{44} Und Mose sagte den Kindern Israel solche
Feste des HERRN.

\hypertarget{section-23}{%
\section{24}\label{section-23}}

\bibverse{1} Und der HERR redete mit Mose und sprach: \bibverse{2}
Gebiete den Kindern Israel, daß sie zu dir bringen gestoßenes lauteres
Baumöl zur Leuchte, daß man täglich Lampen aufsetze \bibverse{3} außen
vor dem Vorhang des Zeugnisses in der Hütte des Stifts. Und Aaron soll's
zurichten des Abends und des Morgens vor dem HERRN täglich. Das sei ein
ewiges Recht euren Nachkommen. \bibverse{4} Er soll die Lampen auf dem
feinen Leuchter zurichten vor dem HERRN täglich. \bibverse{5} Und sollst
Semmelmehl nehmen und davon zwölf Kuchen backen; zwei Zehntel soll ein
Kuchen haben. \bibverse{6} Und sollst sie legen je sechs auf eine
Schicht auf den feinen Tisch vor dem HERRN. \bibverse{7} Und sollst auf
dieselben legen reinen Weihrauch, daß er sei bei den Broten zum
Gedächtnis, ein Feuer dem HERRN. \bibverse{8} Alle Sabbate für und für
soll er sie zurichten vor dem HERRN, von den Kindern Israel zum ewigen
Bund. \bibverse{9} Und sie sollen Aarons und seiner Söhne sein; die
sollen sie essen an heiliger Stätte; denn das ist ein Hochheiliges von
den Opfern des HERRN zum ewigen Recht. \bibverse{10} Es ging aber aus
eines israelitischen Weibes Sohn, der eines ägyptischen Mannes Kind war,
unter den Kindern Israel und zankte sich im Lager mit einem
israelitischen Mann \bibverse{11} und lästerte den Namen des HERRN und
fluchte. Da brachten sie ihn zu Mose (seine Mutter aber hieß Selomith,
eine Tochter Dibris vom Stamme Dan) \bibverse{12} und legten ihn
gefangen, bis ihnen klare Antwort würde durch den Mund des HERRN.
\bibverse{13} Und der HERR redete mit Mose und sprach: \bibverse{14}
Führe den Flucher hinaus vor das Lager und laß alle, die es gehört
haben, ihre Hände auf sein Haupt legen und laß ihn die ganze Gemeinde
steinigen. \bibverse{15} Und sage den Kindern Israel: Welcher seinem
Gott flucht, der soll seine Sünde tragen. \bibverse{16} Welcher des
HERRN Namen lästert, der soll des Todes sterben; die ganze Gemeinde soll
ihn steinigen. Wie der Fremdling, so soll auch der Einheimische sein;
wenn er den Namen lästert, so soll er sterben. \bibverse{17} Wer irgend
einen Menschen erschlägt, der soll des Todes sterben. \bibverse{18} Wer
aber ein Vieh erschlägt, der soll's bezahlen, Leib um Leib.
\bibverse{19} Und wer seinen Nächsten verletzt, dem soll man tun, wie er
getan hat, \bibverse{20} Schade um Schade, Auge um Auge, Zahn um Zahn;
wie er hat einen Menschen verletzt, so soll man ihm wieder tun.
\bibverse{21} Also daß, wer ein Vieh erschlägt, der soll's bezahlen; wer
aber einen Menschen erschlägt, der soll sterben. \bibverse{22} Es soll
einerlei Recht unter euch sein, dem Fremdling wie dem Einheimischen;
denn ich bin der HERR, euer Gott. \bibverse{23} Mose aber sagte es den
Kindern Israel; und sie führten den Flucher hinaus vor das Lager und
steinigten ihn. Also taten die Kinder Israel, wie der HERR dem Mose
geboten hatte.

\hypertarget{section-24}{%
\section{25}\label{section-24}}

\bibverse{1} Und der HERR redete mit Mose auf dem Berge Sinai und
sprach: \bibverse{2} Rede mit den Kindern Israel und sprich zu ihnen:
Wenn ihr in das Land kommt, das ich euch geben werde, so soll das Land
seinen Sabbat dem HERRN feiern, \bibverse{3} daß du sechs Jahre dein
Feld besäest und sechs Jahre deinen Weinberg beschneidest und sammelst
die Früchte ein; \bibverse{4} aber im siebenten Jahr soll das Land
seinen großen Sabbat dem HERRN feiern, darin du dein Feld nicht besäen
noch deinen Weinberg beschneiden sollst. \bibverse{5} Was aber von
selber nach deiner Ernte wächst, sollst du nicht ernten, und die
Trauben, so ohne deine Arbeit wachsen, sollst du nicht lesen, dieweil es
ein Sabbatjahr des Landes ist. \bibverse{6} Aber was das Land während
seines Sabbats trägt, davon sollt ihr essen, du und dein Knecht, deine
Magd, dein Tagelöhner, dein Beisaß, dein Fremdling bei dir, \bibverse{7}
dein Vieh und die Tiere in deinem Lande; alle Früchte sollen Speise
sein. \bibverse{8} Und du sollst zählen solcher Sabbatjahre sieben, daß
sieben Jahre siebenmal gezählt werden, und die Zeit der sieben
Sabbatjahre mache neunundvierzig Jahre. \bibverse{9} Da sollst du die
Posaune lassen blasen durch all euer Land am zehnten Tage des siebenten
Monats, eben am Tage der Versöhnung. \bibverse{10} Und ihr sollt das
fünfzigste Jahr heiligen und sollt ein Freijahr ausrufen im Lande allen,
die darin wohnen; denn es ist euer Halljahr. Da soll ein jeglicher bei
euch wieder zu seiner Habe und zu seinem Geschlecht kommen;
\bibverse{11} denn das fünfzigste Jahr ist euer Halljahr. Ihr sollt
nicht säen, auch was von selber wächst, nicht ernten, auch was ohne
Arbeit wächst im Weinberge, nicht lesen; \bibverse{12} denn das Halljahr
soll unter euch heilig sein. Ihr sollt aber essen, was das Feld trägt.
\bibverse{13} Das ist das Halljahr, da jedermann wieder zu dem Seinen
kommen soll. \bibverse{14} Wenn du nun etwas deinem Nächsten verkaufst
oder ihm etwas abkaufst, soll keiner seinen Bruder übervorteilen,
\bibverse{15} sondern nach der Zahl der Jahre vom Halljahr an sollst du
es von ihm kaufen; und was die Jahre hernach tragen mögen, so hoch soll
er dir's verkaufen. \bibverse{16} Nach der Menge der Jahre sollst du den
Kauf steigern, und nach der wenigen der Jahre sollst du den Kauf
verringern; denn er soll dir's, nach dem es tragen mag, verkaufen.
\bibverse{17} So übervorteile nun keiner seinen Nächsten, sondern
fürchte dich vor deinem Gott; denn ich bin der HERR, euer Gott.
\bibverse{18} Darum tut nach meinen Satzungen und haltet meine Rechte,
daß ihr darnach tut, auf daß ihr im Lande sicher wohnen möget.
\bibverse{19} Denn das Land soll euch seine Früchte geben, daß ihr zu
essen genug habet und sicher darin wohnt. \bibverse{20} Und ob du
würdest sagen: Was sollen wir essen im siebenten Jahr? denn wir säen
nicht, so sammeln wir auch kein Getreide ein: \bibverse{21} da will ich
meinem Segen über euch im sechsten Jahr gebieten, das er soll dreier
Jahr Getreide machen, \bibverse{22} daß ihr säet im achten Jahr und von
dem alten Getreide esset bis in das neunte Jahr, daß ihr vom alten
esset, bis wieder neues Getreide kommt. \bibverse{23} Darum sollt ihr
das Land nicht verkaufen für immer; denn das Land ist mein, und ihr seid
Fremdlinge und Gäste vor mir. \bibverse{24} Und sollt in all eurem Lande
das Land zu lösen geben. \bibverse{25} Wenn dein Bruder verarmt, und
verkauft dir seine Habe, und sein nächster Verwandter kommt zu ihm, daß
er's löse, so soll er's lösen, was sein Bruder verkauft hat.
\bibverse{26} Wenn aber jemand keinen Löser hat und kann mit seiner Hand
so viel zuwege bringen, daß er's löse, \bibverse{27} so soll er rechnen
von dem Jahr, da er's verkauft hat, und was noch übrig ist, dem Käufer
wiedergeben und also wieder zu seiner Habe kommen. \bibverse{28} Kann
aber seine Hand nicht so viel finden, daß er's ihm wiedergebe, so soll,
was er verkauft hat, in der Hand des Käufers bleiben bis zum Halljahr;
in demselben soll es frei werden und er wieder zu seiner Habe kommen.
\bibverse{29} Wer ein Wohnhaus verkauft in einer Stadt mit Mauern, der
hat ein ganzes Jahr Frist, dasselbe wieder zu lösen; das soll die Zeit
sein, darin er es lösen kann. \bibverse{30} Wo er's aber nicht löst, ehe
denn das ganze Jahr um ist, so soll's der Käufer für immer behalten und
seine Nachkommen, und es soll nicht frei werden im Halljahr.
\bibverse{31} Ist's aber ein Haus auf dem Dorfe, um das keine Mauer ist,
das soll man dem Feld des Landes gleich rechnen, und es soll können los
werden und im Halljahr frei werden. \bibverse{32} Die Städte der Leviten
aber, nämlich die Häuser in den Städten, darin ihre Habe ist, können
immerdar gelöst werden. \bibverse{33} Wer etwas von den Leviten löst,
der soll's verlassen im Halljahr, es sei Haus oder Stadt, das er
besessen hat; denn die Häuser in den Städten der Leviten sind ihre Habe
unter den Kindern Israel. \bibverse{34} Aber das Feld vor ihren Städten
soll man nicht verkaufen; denn das ist ihr Eigentum ewiglich.
\bibverse{35} Wenn dein Bruder verarmt und neben dir abnimmt, so sollst
du ihn aufnehmen als einen Fremdling oder Gast, daß er lebe neben dir,
\bibverse{36} und sollst nicht Zinsen von ihm nehmen noch Wucher,
sondern sollst dich vor deinem Gott fürchten, auf daß dein Bruder neben
dir leben könne. \bibverse{37} Denn du sollst ihm dein Geld nicht auf
Zinsen leihen noch deine Speise auf Wucher austun. \bibverse{38} Denn
ich bin der HERR, euer Gott, der euch aus Ägyptenland geführt hat, daß
ich euch das Land Kanaan gäbe und euer Gott wäre. \bibverse{39} Wenn
dein Bruder verarmt neben dir und verkauft sich dir, so sollst du ihn
nicht lassen dienen als einen Leibeigenen; \bibverse{40} sondern wie ein
Tagelöhner und Gast soll er bei dir sein und bis an das Halljahr bei dir
dienen. \bibverse{41} Dann soll er von dir frei ausgehen und seine
Kinder mit ihm und soll wiederkommen zu seinem Geschlecht und zu seiner
Väter Habe. \bibverse{42} Denn sie sind meine Knechte, die ich aus
Ägyptenland geführt habe; darum soll man sie nicht auf leibeigene Weise
verkaufen. \bibverse{43} Und sollst nicht mit Strenge über sie
herrschen, sondern dich fürchten vor deinem Gott. \bibverse{44} Willst
du aber leibeigene Knechte und Mägde haben, so sollst du sie kaufen von
den Heiden, die um euch her sind, \bibverse{45} und auch von den Kindern
der Gäste, die Fremdlinge unter euch sind, und von ihren Nachkommen, die
sie bei euch in eurem Land zeugen; dieselben mögt ihr zu eigen haben
\bibverse{46} und sollt sie besitzen und eure Kinder nach euch zum
Eigentum für und für; die sollt ihr leibeigene Knechte sein lassen. Aber
von euren Brüdern, den Kindern Israel, soll keiner über den andern
herrschen mit Strenge. \bibverse{47} Wenn irgend ein Fremdling oder Gast
bei dir zunimmt und dein Bruder neben ihm verarmt und sich dem Fremdling
oder Gast bei dir oder jemand von seinem Stamm verkauft, \bibverse{48}
so soll er nach seinem Verkaufen Recht haben, wieder frei zu werden, und
es mag ihn jemand unter seinen Brüdern lösen, \bibverse{49} oder sein
Vetter oder Vetters Sohn oder sonst ein Blutsfreund seines Geschlechts;
oder so seine Hand so viel erwirbt, so soll er selbst sich lösen.
\bibverse{50} Und soll mit seinem Käufer rechnen von dem Jahr an, da er
sich verkauft hatte, bis aufs Halljahr; und das Geld, darum er sich
verkauft hat, soll nach der Zahl der Jahre gerechnet werden, als wäre er
die ganze Zeit Tagelöhner bei ihm gewesen. \bibverse{51} Sind noch viele
Jahre bis an das Halljahr, so soll er nach denselben desto mehr zu
seiner Lösung wiedergeben von dem Gelde darum er gekauft ist.
\bibverse{52} Sind aber wenig Jahre übrig bis ans Halljahr, so soll er
auch darnach wiedergeben zu seiner Lösung. \bibverse{53} Als Tagelöhner
soll er von Jahr zu Jahr bei ihm sein, und sollst nicht lassen mit
Strenge über ihn herrschen vor deinen Augen. \bibverse{54} Wird er aber
auf diese Weise sich nicht lösen, so soll er im Halljahr frei ausgehen
und seine Kinder mit ihm. \bibverse{55} Denn die Kinder Israel sind
meine Knechte, die ich aus Ägyptenland geführt habe. Ich bin der HERR,
euer Gott.

\hypertarget{section-25}{%
\section{26}\label{section-25}}

\bibverse{1} Ihr sollt keine Götzen machen noch Bild und sollt euch
keine Säule aufrichten, auch keinen Malstein setzen in eurem Lande, daß
ihr davor anbetet; denn ich bin der HERR, euer Gott. \bibverse{2} Haltet
meine Sabbate und fürchtet euch vor meinem Heiligtum. Ich bin der HERR.
\bibverse{3} Werdet ihr in meinen Satzungen wandeln und meine Gebote
halten und tun, \bibverse{4} so will ich euch Regen geben zu seiner
Zeit, und das Land soll sein Gewächs geben und die Bäume auf dem Felde
ihre Früchte bringen, \bibverse{5} und die Dreschzeit soll reichen bis
zur Weinernte, und die Weinernte bis zur Zeit der Saat; und sollt Brots
die Fülle haben und sollt sicher in eurem Lande wohnen. \bibverse{6} Ich
will Frieden geben in eurem Lande, daß ihr schlafet und euch niemand
schrecke. Ich will die bösen Tiere aus eurem Land tun, und soll kein
Schwert durch euer Land gehen. \bibverse{7} Ihr sollt eure Feinde jagen,
und sie sollen vor euch her ins Schwert fallen. \bibverse{8} Euer fünf
sollen hundert jagen, und euer hundert sollen zehntausend jagen; denn
eure Feinde sollen vor euch her fallen ins Schwert. \bibverse{9} Und ich
will mich zu euch wenden und will euch wachsen und euch mehren lassen
und will meinen Bund euch halten. \bibverse{10} Und sollt von dem
Vorjährigen essen, und wenn das Neue kommt, das Vorjährige wegtun.
\bibverse{11} Ich will meine Wohnung unter euch haben, und meine Seele
soll euch nicht verwerfen. \bibverse{12} Und will unter euch wandeln und
will euer Gott sein; so sollt ihr mein Volk sein. \bibverse{13} Denn ich
bin der HERR, euer Gott, der euch aus Ägyptenland geführt hat, daß ihr
meine Knechte wäret, und habe euer Joch zerbrochen und habe euch
aufgerichtet wandeln lassen. \bibverse{14} Werdet ihr mir aber nicht
gehorchen und nicht tun diese Gebote alle \bibverse{15} und werdet meine
Satzungen verachten und eure Seele wird meine Rechte verwerfen, daß ihr
nicht tut alle meine Gebote, und werdet meinen Bund brechen,
\bibverse{16} so will ich euch auch solches tun: ich will euch
heimsuchen mit Schrecken, Darre und Fieber, daß euch die Angesichter
verfallen und der Leib verschmachte; ihr sollt umsonst euren Samen säen,
und eure Feinde sollen ihn essen; \bibverse{17} und ich will mein
Antlitz wider euch stellen, und sollt geschlagen werden vor euren
Feinden; und die euch hassen, sollen über euch herrschen, und sollt
fliehen, da euch niemand jagt. \bibverse{18} So ihr aber über das noch
nicht mir gehorcht, so will ich's noch siebenmal mehr machen, euch zu
strafen um eure Sünden, \bibverse{19} daß ich euren Stolz und eure
Halsstarrigkeit breche; und will euren Himmel wie Eisen und eure Erde
wie Erz machen. \bibverse{20} Und eure Mühe und Arbeit soll verloren
sein, daß euer Land sein Gewächs nicht gebe und die Bäume des Landes
ihre Früchte nicht bringen. \bibverse{21} Und wo ihr mir entgegen
wandelt und mich nicht hören wollt, so will ich's noch siebenmal mehr
machen, auf euch zu schlagen um eurer Sünden willen. \bibverse{22} Und
will wilde Tiere unter euch senden, die sollen eure Kinder fressen und
euer Vieh zerreißen und euer weniger machen, und eure Straßen sollen
wüst werden. \bibverse{23} Werdet ihr euch aber damit noch nicht von mir
züchtigen lassen und mir entgegen wandeln, \bibverse{24} so will ich
euch auch entgegen wandeln und will euch noch siebenmal mehr schlagen um
eurer Sünden willen \bibverse{25} und will ein Racheschwert über euch
bringen, das meinen Bund rächen soll. Und ob ihr euch in eure Städte
versammelt, will ich doch die Pestilenz unter euch senden und will euch
in eurer Feinde Hände geben. \bibverse{26} Dann will ich euch den Vorrat
des Brots verderben, daß zehn Weiber sollen in einem Ofen backen, und
euer Brot soll man mit Gewicht auswägen, und wenn ihr esset, sollt ihr
nicht satt werden. \bibverse{27} Werdet ihr aber dadurch mir noch nicht
gehorchen und mir entgegen wandeln, \bibverse{28} so will ich euch im
Grimm entgegen wandeln und will euch siebenmal mehr strafen um eure
Sünden, \bibverse{29} daß ihr sollt eurer Söhne und Töchter Fleisch
essen. \bibverse{30} Und will eure Höhen vertilgen und eure Sonnensäulen
ausrotten und will eure Leichname auf eure Götzen werfen, und meine
Seele wird an euch Ekel haben. \bibverse{31} Und will eure Städte
einreißen und will euren süßen Geruch nicht riechen. \bibverse{32} Also
will ich das Land wüst machen, daß eure Feinde, so darin wohnen, sich
davor entsetzen werden. \bibverse{33} Euch aber will ich unter die
Heiden streuen, und das Schwert ausziehen hinter euch her, daß euer Land
soll wüst sein und eure Städte verstört. \bibverse{34} Alsdann wird das
Land sich seine Sabbate gefallen lassen, solange es wüst liegt und ihr
in der Feinde Land seid; ja, dann wird das Land feiern und sich seine
Sabbate gefallen lassen. \bibverse{35} Solange es wüst liegt, wird es
feiern, darum daß es nicht feiern konnte, da ihr's solltet feiern
lassen, da ihr darin wohntet. \bibverse{36} Und denen, die von euch
übrigbleiben will ich ein feiges Herz machen in ihrer Feinde Land, daß
sie soll ein rauschend Blatt jagen, und soll fliehen davor, als jage sie
ein Schwert, und fallen, da sie niemand jagt. \bibverse{37} Und soll
einer über den andern hinfallen, gleich als vor dem Schwert, da sie doch
niemand jagt; und ihr sollt euch nicht auflehnen dürfen wider eure
Feinde. \bibverse{38} Und ihr sollt umkommen unter den Heiden, und eurer
Feinde Land soll euch fressen. \bibverse{39} Welche aber von euch
übrigbleiben, die sollen in ihrer Missetat verschmachten in der Feinde
Land; auch in ihrer Väter Missetat sollen sie mit ihnen verschmachten.
\bibverse{40} Da werden sie denn bekennen ihre Missetat und ihrer Väter
Missetat, womit sie sich an mir versündigt und mir entgegen gewandelt
haben. \bibverse{41} Darum will ich auch ihnen entgegen wandeln und will
sie in ihrer Feinde Land wegtreiben; da wird sich ja ihr unbeschnittenes
Herz demütigen, und dann werden sie sich die Strafe ihrer Missetat
gefallen lassen. \bibverse{42} Und ich werde gedenken an meinen Bund mit
Jakob und an meinen Bund mit Isaak und an meinen Bund mit Abraham und
werde an das Land gedenken, \bibverse{43} das von ihnen verlassen ist
und sich seine Sabbate gefallen läßt, dieweil es wüst von ihnen liegt,
und sie sich die Strafe ihrer Missetat gefallen lassen, darum daß sie
meine Rechte verachtet haben und ihre Seele an meinen Satzungen Ekel
gehabt hat. \bibverse{44} Auch wenn sie schon in der Feinde Land sind,
habe ich sie gleichwohl nicht verworfen und ekelt mich ihrer nicht also,
daß es mit ihnen aus sein sollte und mein Bund mit ihnen sollte nicht
mehr gelten; denn ich bin der HERR, ihr Gott. \bibverse{45} Und ich will
über sie an meinen ersten Bund gedenken, da ich sie aus Ägyptenland
führte vor den Augen der Heiden, daß ich ihr Gott wäre, ich, der HERR.
\bibverse{46} Dies sind die Satzungen und Rechte und Gesetze, die der
HERR zwischen ihm selbst und den Kindern Israel gestellt hat auf dem
Berge Sinai durch die Hand Mose's.

\hypertarget{section-26}{%
\section{27}\label{section-26}}

\bibverse{1} Und der HERR redete mit Mose und sprach: \bibverse{2} Rede
mit den Kindern Israel und sprich zu ihnen: Wenn jemand ein besonderes
Gelübde tut, also daß du seinen Leib schätzen mußt, \bibverse{3} so soll
dies eine Schätzung sein: ein Mannsbild, zwanzig Jahre alt bis ins
sechzigste Jahr, sollst du schätzen auf fünfzig Silberlinge nach dem Lot
des Heiligtums, \bibverse{4} ein Weibsbild auf dreißig Silberlinge.
\bibverse{5} Von fünf Jahren an bis auf zwanzig Jahre sollst du ihn
schätzen auf zwanzig Silberlinge, wenn's ein Mannsbild ist, ein
Weibsbild aber auf zehn Silberlinge. \bibverse{6} Von einem Monat an bis
auf fünf Jahre sollst du ihn schätzen auf fünf Silberlinge, wenn's ein
Mannsbild ist, ein Weibsbild aber auf drei Silberlinge. \bibverse{7} Ist
er aber sechzig Jahre alt und darüber, so sollst du ihn schätzen auf
fünfzehn Silberlinge, wenn's ein Mannsbild ist, ein Weibsbild aber auf
zehn Silberlinge. \bibverse{8} Ist er aber zu arm zu solcher Schätzung,
so soll er sich vor den Priester stellen, und der Priester soll ihn
schätzen; er soll ihn aber schätzen, nach dem die Hand des, der gelobt
hat, erwerben kann. \bibverse{9} Ist's aber ein Vieh, das man dem HERRN
opfern kann: alles, was man davon dem HERRN gibt ist heilig.
\bibverse{10} Man soll's nicht wechseln noch wandeln, ein gutes um ein
böses, oder ein böses um ein gutes. Wird's aber jemand wechseln, ein
Vieh um das andere, so sollen sie beide dem HERRN heilig sein.
\bibverse{11} Ist aber das Tier unrein, daß man's dem HERRN nicht opfern
darf, so soll man's vor den Priester stellen, \bibverse{12} und der
Priester soll's schätzen, ob es gut oder böse sei; und es soll bei des
Priesters Schätzung bleiben. \bibverse{13} Will's aber jemand lösen, der
soll den Fünften über die Schätzung geben. \bibverse{14} Wenn jemand
sein Haus heiligt, daß es dem HERRN heilig sei, das soll der Priester
schätzen, ob's gut oder böse sei; und darnach es der Priester schätzt,
so soll's bleiben. \bibverse{15} So es aber der, so es geheiligt hat,
will lösen, so soll er den fünften Teil des Geldes, zu dem es geschätzt
ist, draufgeben, so soll's sein werden. \bibverse{16} Wenn jemand ein
Stück Acker von seinem Erbgut dem HERRN heiligt, so soll es geschätzt
werden nach der Aussaat. Ist die Aussaat ein Homer Gerste, so soll es
fünfzig Silberlinge gelten. \bibverse{17} Heiligt er seinen Acker vom
Halljahr an, so soll er nach seinem Wert gelten. \bibverse{18} Hat er
ihn aber nach dem Halljahr geheiligt, so soll der Priester das Geld
berechnen nach den übrigen Jahren zum Halljahr und ihn darnach geringer
schätzen. \bibverse{19} Will aber der, so ihn geheiligt hat, den Acker
lösen, so soll er den fünften Teil des Geldes, zu dem er geschätzt ist,
draufgeben, so soll er sein werden. \bibverse{20} Will er ihn aber nicht
lösen, sondern verkauft ihn einem andern, so soll er ihn nicht mehr
lösen können; \bibverse{21} sondern derselbe Acker, wenn er im Halljahr
frei wird, soll dem HERRN heilig sein wie ein verbannter Acker und soll
des Priesters Erbgut sein. \bibverse{22} Wenn aber jemand dem HERRN
einen Acker heiligt, den er gekauft hat und der nicht sein Erbgut ist,
\bibverse{23} so soll der Priester berechnen, was er gilt bis an das
Halljahr; und soll desselben Tages solche Schätzung geben, daß sie dem
HERRN heilig sei. \bibverse{24} Aber im Halljahr soll er wieder gelangen
an den, von dem er ihn gekauft hat, daß sein Erbgut im Lande sei.
\bibverse{25} Alle Schätzung soll geschehen nach dem Lot des Heiligtums;
ein Lot aber hat zwanzig Gera. \bibverse{26} Die Erstgeburt unter dem
Vieh, die dem HERRN sonst gebührt, soll niemand dem HERRN heiligen, es
sei ein Ochs oder Schaf; denn es ist des HERRN. \bibverse{27} Ist es
aber unreines Vieh, so soll man's lösen nach seinem Werte, und
darübergeben den Fünften. Will er's aber nicht lösen, so verkaufe man's
nach seinem Werte. \bibverse{28} Man soll kein Verbanntes verkaufen noch
lösen, das jemand dem HERRN verbannt von allem, was sein ist, es seien
Menschen, Vieh oder Erbacker; denn alles verbannte ist ein Hochheiliges
dem HERRN. \bibverse{29} Man soll auch keinen verbannten Menschen lösen,
sondern er soll des Todes sterben. \bibverse{30} Alle Zehnten im Lande
von Samen des Landes und von Früchten der Bäume sind des HERRN und
sollen dem HERRN heilig sein. \bibverse{31} Will aber jemand seinen
Zehnten lösen, der soll den Fünften darübergeben. \bibverse{32} Und alle
Zehnten von Rindern und Schafen, von allem, was unter dem Hirtenstabe
geht, das ist ein heiliger Zehnt dem HERRN. \bibverse{33} Man soll nicht
fragen, ob's gut oder böse sei; man soll's auch nicht wechseln. Wird's
aber jemand wechseln, so soll's beides heilig sein und nicht gelöst
werden. \bibverse{34} Dies sind die Gebote, die der HERR dem Mose gebot
an die Kinder Israel auf dem Berge Sinai.
