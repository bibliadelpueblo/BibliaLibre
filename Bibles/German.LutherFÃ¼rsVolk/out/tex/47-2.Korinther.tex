\hypertarget{section}{%
\section{1}\label{section}}

\bibverse{1} Paulus, ein Apostel JEsu Christi durch den Willen GOttes,
und Bruder Timotheus: Der Gemeinde GOttes zu Korinth samt allen Heiligen
in ganz Achaja. \bibverse{2} Gnade sei mit euch und Friede von GOtt,
unserm Vater, und dem HErrn JEsu Christo! \bibverse{3} Gelobet sei GOtt
und der Vater unsers HErrn JEsu Christi, der Vater der Barmherzigkeit
und GOtt alles Trostes, \bibverse{4} der uns tröstet in aller unserer
Trübsal, daß wir auch trösten können, die da sind in allerlei Trübsal,
mit dem Trost, damit wir getröstet werden von GOtt. \bibverse{5} Denn
gleichwie wir des Leidens Christi viel haben, also werden wir auch
reichlich getröstet durch Christum. \bibverse{6} Wir haben aber Trübsal
oder Trost, so geschieht es euch zugut. Ist's Trübsal, so geschieht es
euch zu Trost und Heil; welches Heil beweiset sich, so ihr leidet mit
Geduld dermaßen, wie wir leiden. Ist's Trost, so geschieht es euch auch
zu Trost und Heil. \bibverse{7} Und stehet unsere Hoffnung fest für
euch, dieweil wir wissen, daß, wie ihr des Leidens teilhaftig seid, so
werdet ihr auch des Trostes teilhaftig sein. \bibverse{8} Denn wir
wollen euch nicht verhalten, liebe Brüder, unsere Trübsal, die uns in
Asien widerfahren ist da wir über die Maßen beschweret waren und über
Macht, also daß wir auch am Leben verzagten \bibverse{9} und bei uns
beschlossen hatten, wir müßten sterben. Das geschah aber darum, daß wir
unser Vertrauen nicht auf uns selbst stelleten, sondern auf GOtt, der
die Toten auferwecket, \bibverse{10} welcher uns von solchem Tode
erlöset hat und noch täglich erlöset; und hoffen auf ihn, er werde uns
auch hinfort erlösen \bibverse{11} durch Hilfe eurer Fürbitte für uns,
auf daß über uns für die Gabe, die uns gegeben ist, durch viel Personen
viel Danks geschehe. \bibverse{12} Denn unser Ruhm ist der, nämlich das
Zeugnis unsers Gewissens, daß wir in Einfältigkeit und göttlicher
Lauterkeit, nicht in fleischlicher Weisheit, sondern in der Gnade GOttes
auf der Welt gewandelt haben, allermeist aber bei euch. \bibverse{13}
Denn wir schreiben euch nichts anderes, denn was ihr leset und auch
befindet. Ich hoffe aber, ihr werdet uns auch bis ans Ende also
befinden, gleichwie ihr uns zum Teil befunden habt. \bibverse{14} Denn
wir sind euer Ruhm, gleichwie auch ihr unser Ruhm seid auf des HErrn
JEsu Tag. \bibverse{15} Und auf solch Vertrauen gedachte ich jenes Mal
zu euch zu kommen, auf daß ihr abermal eine Wohltat empfinget,
\bibverse{16} und ich durch euch nach Mazedonien reisete und wiederum
aus Mazedonien zu euch käme und von euch geleitet würde nach Judäa.
\bibverse{17} Hab' ich aber eine Leichtfertigkeit gebrauchet, da ich
solches gedachte, oder sind meine Anschläge fleischlich? Nicht also,
sondern bei mir ist Ja Ja, und Nein ist Nein. \bibverse{18} Aber, o ein
treuer GOtt, daß unser Wort an euch nicht Ja und Nein gewesen ist!
\bibverse{19} Denn der Sohn GOttes, JEsus Christus, der unter euch durch
uns geprediget ist, durch mich und Silvanus und Timotheus, der war nicht
Ja und Nein, sondern es war Ja in ihm. \bibverse{20} Denn alle
GOttesverheißungen sind Ja in ihm und sind Amen in ihm GOtt zu Lobe
durch uns. \bibverse{21} GOtt ist's aber, der uns befestiget samt euch
in Christum und uns gesalbet \bibverse{22} und versiegelt und in unsere
Herzen das Pfand, den Geist, gegeben hat. \bibverse{23} Ich rufe aber
GOtt an zum Zeugen auf meine Seele, daß ich euer verschonet habe in dem,
daß ich nicht wieder gen Korinth kommen bin. \bibverse{24} Nicht daß wir
Herren seien über euren Glauben, sondern wir sind Gehilfen eurer Freude;
denn ihr stehet im Glauben.

\hypertarget{section-1}{%
\section{2}\label{section-1}}

\bibverse{1} Ich dachte aber solches bei mir, daß ich nicht abermal in
Traurigkeit zu euch käme. \bibverse{2} Denn so ich euch traurig mache,
wer ist, der mich fröhlich mache, ohne der da von mir betrübet wird?
\bibverse{3} Und dasselbige habe ich euch geschrieben, daß ich nicht,
wenn ich käme, traurig sein müßte, über welche ich mich billig sollte
freuen, sintemal ich mich des zu euch allen versehe, daß meine Freude
euer aller Freude sei. \bibverse{4} Denn ich schrieb euch in großer
Trübsal und Angst des Herzens mit viel Tränen, nicht daß ihr solltet
betrübet werden, sondern auf, daß ihr die Liebe erkennetet, welche ich
habe sonderlich zu euch. \bibverse{5} So aber jemand eine Betrübnis hat
angerichtet, der hat nicht mich betrübet, ohne zum Teil, auf daß ich
nicht euch alle beschwere. \bibverse{6} Es ist aber genug, daß
derselbige von vielen also gestraft ist. \bibverse{7} daß ihr nun
hinfort ihm desto mehr vergebet und tröstet, auf daß er nicht in allzu
große Traurigkeit versinke. \bibverse{8} Darum ermahne ich euch, daß ihr
die Liebe an ihm beweiset. \bibverse{9} Denn darum habe ich euch auch
geschrieben, daß ich erkennete, ob ihr rechtschaffen seid, gehorsam zu
sein in allen Stücken. \bibverse{10} Welchem aber ihr etwas vergebet,
dem vergebe ich auch. Denn auch ich, so ich etwas vergebe jemandem, das
vergebe ich um euretwillen an Christi Statt, \bibverse{11} auf daß wir
nicht übervorteilt werden vom Satan; denn uns ist nicht unbewußt, was er
im Sinn hat. \bibverse{12} Da ich aber gen Troas kam, zu predigen das
Evangelium Christi, und mir eine Tür aufgetan war in dem HErrn,
\bibverse{13} hatte ich keine Ruhe in meinem Geist, da ich Titus, meinen
Bruder, nicht fand, sondern ich machte meinen Abschied mit ihnen und
fuhr aus nach Mazedonien. \bibverse{14} Aber GOtt sei gedankt, der uns
allezeit Sieg gibt in Christo und offenbaret den Geruch seiner
Erkenntnis durch uns an allen Orten. \bibverse{15} Denn wir sind GOtt
ein guter Geruch Christi, beide, unter denen, die selig werden, und
unter denen, die verloren werden: \bibverse{16} diesen ein Geruch des
Todes zum Tode, jenen aber ein Geruch des Lebens zum Leben. Und wer ist
hiezu tüchtig? \bibverse{17} Denn wir sind nicht wie etliche viele, die
das Wort GOttes verfälschen, sondern als aus Lauterkeit und als aus
GOtt, vor GOtt reden wir in Christo.

\hypertarget{section-2}{%
\section{3}\label{section-2}}

\bibverse{1} Heben wir denn abermal an, uns selbst zu preisen? Oder
bedürfen wir, wie etliche, der Lobebriefe an euch oder Lobebriefe von
euch? \bibverse{2} Ihr seid unser Brief, in unser Herz geschrieben, der
erkannt und gelesen wird von allen Menschen, \bibverse{3} die ihr
offenbar worden seid, daß ihr ein Brief Christi seid, durch unser
Predigtamt zubereitet und durch uns geschrieben, nicht mit Tinte,
sondern mit dem Geist des lebendigen GOttes, nicht in steinerne Tafeln,
sondern in fleischerne Tafeln des Herzens. \bibverse{4} Ein solch
Vertrauen aber haben wir durch Christum zu GOtt. \bibverse{5} Nicht daß
wir tüchtig sind von uns selber, etwas zu denken als von uns selber,
sondern daß wir tüchtig sind, ist von GOtt, \bibverse{6} welcher auch
uns tüchtig gemacht hat, das Amt zu führen des Neuen Testaments, nicht
des Buchstabens, sondern des Geistes. Denn der Buchstabe tötet, aber der
Geist macht lebendig. \bibverse{7} So aber das Amt, das durch die
Buchstaben tötet und in die Steine ist gebildet, Klarheit hatte, also
daß die Kinder Israel nicht konnten ansehen das Angesicht Mose's um der
Klarheit willen seines Angesichtes, die doch aufhöret, \bibverse{8} wie
sollte nicht viel mehr das Amt, das den Geist gibt, Klarheit haben?
\bibverse{9} Denn so das Amt, das die Verdammnis prediget, Klarheit hat,
viel mehr hat das Amt, das die Gerechtigkeit prediget, überschwengliche
Klarheit. \bibverse{10} Denn auch jenes Teil, das verkläret war, ist
nicht für Klarheit zu achten gegen diese überschwengliche Klarheit.
\bibverse{11} Denn so das Klarheit hatte, das da aufhöret, viel mehr
wird das Klarheit haben, das da bleibet. \bibverse{12} Dieweil wir nun
solche Hoffnung haben, brauchen wir große Freudigkeit \bibverse{13} und
tun nicht wie Mose, der die Decke vor sein Angesicht hing, daß die
Kinder Israel nicht ansehen konnten das Ende des, der aufhöret.
\bibverse{14} Sondern ihre Sinne sind verstockt; denn bis auf den
heutigen Tag bleibt dieselbige Decke unaufgedeckt über dem Alten
Testament, wenn sie es lesen, welche in Christo aufhöret. \bibverse{15}
Aber bis auf den heutigen Tag, wenn Mose gelesen wird, hängt die Decke
vor ihrem Herzen. \bibverse{16} Wenn es aber sich bekehrete zu dem
HErrn, so würde die Decke abgetan. \bibverse{17} Denn der HErr ist der
Geist. Wo aber der Geist des HErrn ist, da ist Freiheit. \bibverse{18}
Nun aber spiegelt sich in uns allen des HErrn Klarheit mit aufgedecktem
Angesichte; und wir werden verkläret in dasselbige Bild von einer
Klarheit zu der andern als vom Geist des HErrn.

\hypertarget{section-3}{%
\section{4}\label{section-3}}

\bibverse{1} Darum, dieweil wir ein solch Amt haben, nachdem uns
Barmherzigkeit widerfahren ist, so werden wir nicht müde \bibverse{2}
sondern meiden auch heimliche Schande und gehen nicht mit Schalkheit um,
fälschen auch nicht GOttes Wort, sondern mit Offenbarung der Wahrheit
und beweisen uns wohl gegen aller Menschen Gewissen vor GOtt.
\bibverse{3} Ist nun unser Evangelium verdeckt, so ist's in denen, die
verloren werden, verdeckt, \bibverse{4} bei welchen der GOtt dieser Welt
der Ungläubigen Sinn verblendet hat, daß sie nicht sehen das helle Licht
des Evangeliums von der Klarheit Christi, welcher ist das Ebenbild
GOttes. \bibverse{5} Denn wir predigen nicht uns selbst, sondern JEsum
Christ, daß er sei der HErr, wir aber eure Knechte um JEsu willen.
\bibverse{6} Denn GOtt, der da hieß das Licht aus der Finsternis
hervorleuchten, der hat einen hellen Schein in unsere Herzen gegeben,
daß (durch uns) entstünde die Erleuchtung von der Erkenntnis der
Klarheit GOttes in dem Angesichte JEsu Christi. \bibverse{7} Wir haben
aber solchen Schatz in irdischen Gefäßen, auf daß die überschwengliche
Kraft sei GOttes und nicht von uns. \bibverse{8} Wir haben allenthalben
Trübsal, aber wir ängsten uns nicht; uns ist bange, aber wir verzagen
nicht; \bibverse{9} wir leiden Verfolgung, aber wir werden nicht
verlassen; wir werden unterdrückt, aber wir kommen nicht um.
\bibverse{10} Und tragen um allezeit das Sterben des HErrn JEsu an
unserm Leibe, auf daß auch das Leben des HErrn JEsu an unserm Leibe
offenbar werde. \bibverse{11} Denn wir, die wir leben, werden immerdar
in den Tod gegeben um JEsu willen, auf daß auch das Leben JEsu offenbar
werde an unserm sterblichen Fleische. \bibverse{12} Darum so ist nun der
Tod mächtig in uns, aber das Leben in euch. \bibverse{13} Dieweil wir
aber denselbigen Geist des Glaubens haben (nachdem geschrieben stehet:
Ich glaube, darum rede ich), so glauben wir auch, darum so reden wir
auch \bibverse{14} und wissen, daß der, so den HErrn JEsum hat
auferweckt, wird uns auch auferwecken durch JEsum und wird uns
darstellen samt euch. \bibverse{15} Denn es geschiehet alles um
euretwillen, auf daß die überschwengliche Gnade durch vieler Danksagen
GOtt reichlich preise. \bibverse{16} Darum werden wir nicht müde,
sondern ob unser äußerlicher Mensch verweset, so wird doch der
innerliche von Tag zu Tag erneuert. \bibverse{17} Denn unsere Trübsal,
die zeitlich und leicht ist, schaffet eine ewige und über alle Maßen
wichtige Herrlichkeit \bibverse{18} uns, die wir nicht sehen auf das
Sichtbare, sondern auf das Unsichtbare. Denn was sichtbar ist, das ist
zeitlich; was aber unsichtbar ist, das ist ewig.

\hypertarget{section-4}{%
\section{5}\label{section-4}}

\bibverse{1} Wir wissen aber, so unser irdisch Haus dieser Hütte
zerbrochen wird, daß wir einen Bau haben, von GOtt erbauet, ein Haus,
nicht mit Händen gemacht, das ewig ist, im Himmel. \bibverse{2} Und über
demselbigen sehnen wir uns auch nach unserer Behausung, die vom Himmel
ist, und uns verlanget, daß wir damit überkleidet werden, \bibverse{3}
So doch, wo wir bekleidet und nicht bloß erfunden werden. \bibverse{4}
Denn dieweil wir in der Hütte sind, sehnen wir uns und sind beschweret,
sintemal wir wollten lieber nicht entkleidet, sondern überkleidet
werden, auf daß das Sterbliche würde verschlungen von dem Leben.
\bibverse{5} Der uns aber zu demselbigen bereitet, das ist GOtt, der uns
das Pfand, den Geist gegeben hat. \bibverse{6} Wir sind aber getrost
allezeit und wissen, daß, dieweil wir im Leibe wohnen, so wallen wir dem
HErrn. \bibverse{7} Denn wir wandeln im Glauben und nicht im Schauen.
\bibverse{8} Wir sind aber getrost und haben vielmehr Lust, außer dem
Leibe zu wallen und daheim zu sein bei dem HErrn. \bibverse{9} Darum
fleißigen wir uns auch, wir sind daheim oder wallen, daß wir ihm
wohlgefallen. \bibverse{10} Denn wir müssen alle offenbar werden vor dem
Richterstuhl Christi, auf daß ein jeglicher empfange, nachdem er
gehandelt hat bei Leibesleben, es sei gut oder böse. \bibverse{11}
Dieweil wir denn wissen, daß der HErr zu fürchten ist, fahren wir schön
mit den Leuten; aber GOtt sind wir offenbar. Ich hoffe aber, daß wir
auch in eurem Gewissen offenbar sind. \bibverse{12} Daß wir uns nicht
abermal loben, sondern euch eine Ursache geben, zu rühmen von uns, auf
daß ihr habet zu rühmen wider die, so sich nach dem Ansehen rühmen und
nicht nach dem Herzen. \bibverse{13} Denn tun wir zu viel, so tun wir's
GOtt; sind wir mäßig, so sind wir euch mäßig. \bibverse{14} Denn die
Liebe Christi dringet uns also, sintemal wir halten, daß, so einer für
alle gestorben ist, so sind sie alle gestorben. \bibverse{15} Und er ist
darum für sie alle gestorben, auf daß die, so da leben, hinfort nicht
ihnen selbst leben, sondern dem, der für sie gestorben und auferstanden
ist. \bibverse{16} Darum von nun an kennen wir niemand nach dem Fleisch;
und ob wir auch Christum gekannt haben nach dem Fleisch, so kennen wir
ihn doch jetzt nicht mehr. \bibverse{17} Darum, ist jemand in Christo,
so ist er eine neue Kreatur. Das Alte ist vergangen; siehe, es ist alles
neu worden. \bibverse{18} Aber das alles von GOtt, der uns mit ihm
selber versöhnet hat durch JEsum Christum und das Amt gegeben, das die
Versöhnung prediget. \bibverse{19} Denn GOtt war in Christo und
versöhnete die Welt mit ihm selber und rechnete ihnen ihre Sünden nicht
zu und hat unter uns aufgerichtet das Wort von der Versöhnung.
\bibverse{20} So sind wir nun Botschafter an Christi Statt; denn GOtt
vermahnet durch uns. So bitten wir nun an Christi Statt: Lasset euch
versöhnen mit GOtt! \bibverse{21} Denn er hat den, der von keiner Sünde
wußte, für uns zur Sünde gemacht, auf daß wir würden in ihm die
Gerechtigkeit, die vor GOtt gilt.

\hypertarget{section-5}{%
\section{6}\label{section-5}}

\bibverse{1} Wir ermahnen aber euch als Mithelfer, daß ihr nicht
vergeblich die Gnade GOttes empfanget. \bibverse{2} Denn er spricht: Ich
habe dich in der angenehmen Zeit erhöret und habe dir am Tage des Heils
geholfen. Sehet, jetzt ist die angenehme Zeit, jetzt ist der Tag des
Heils. \bibverse{3} Lasset uns aber niemand irgendein Ärgernis geben,
auf daß unser Amt nicht verlästert werde; \bibverse{4} sondern in allen
Dingen lasset uns beweisen als die Diener GOttes: in großer Geduld, in
Trübsalen, in Nöten, in Ängsten, \bibverse{5} in Schlägen, in
Gefängnissen, in Aufruhren, in Arbeit, in Wachen, in Fasten,
\bibverse{6} in Keuschheit, in Erkenntnis, in Langmut, in
Freundlichkeit, in dem Heiligen Geist, in ungefärbter Liebe,
\bibverse{7} in dem Wort der Wahrheit, in der Kraft GOttes, durch Waffen
der Gerechtigkeit zur Rechten und zur Linken; \bibverse{8} durch Ehre
und Schande, durch böse Gerüchte und gute Gerüchte; als die Verführer
und doch wahrhaftig; \bibverse{9} als die Unbekannten und doch bekannt;
als die Sterbenden, und siehe, wir leben; als die Gezüchtigten und doch
nicht ertötet; \bibverse{10} als die Traurigen, aber allezeit fröhlich;
als die Armen, aber die doch viele reich machen; als die nichts
innehaben und doch alles haben. \bibverse{11} O ihr Korinther, unser
Mund hat sich zu euch aufgetan; unser Herz ist getrost. \bibverse{12}
Unserthalben dürft ihr euch nicht ängsten. Daß ihr euch aber ängstet,
das tut ihr aus herzlicher Meinung. \bibverse{13} Ich rede mit euch als
mit meinen Kindern, daß ihr euch auch also gegen mich stellet und seiet
auch getrost. \bibverse{14} Ziehet nicht am fremden Joch mit den
Ungläubigen! Denn was hat die Gerechtigkeit für Genieß mit der
Ungerechtigkeit? Was hat das Licht für Gemeinschaft mit der Finsternis?
\bibverse{15} Wie stimmt Christus mit Belial? Oder was für ein Teil hat
der Gläubige mit dem Ungläubigen? \bibverse{16} Was hat der Tempel
GOttes für Gleichheit mit den Götzen? Ihr aber seid der Tempel des
lebendigen GOttes, wie denn GOtt spricht: Ich will in ihnen wohnen und
in ihnen wandeln und will ihr GOtt sein, und sie sollen mein Volk sein.
\bibverse{17} Darum gehet aus von ihnen und sondert euch ab, spricht der
HErr, und rühret kein Unreines an, so will ich euch annehmen.
\bibverse{18} und euer Vater sein, und ihr sollet meine Söhne und
Töchter sein, spricht der allmächtige HErr.

\hypertarget{section-6}{%
\section{7}\label{section-6}}

\bibverse{1} Dieweil wir nun solche Verheißung haben, meine Liebsten, so
lasset uns von aller Befleckung des Fleisches und des Geistes uns
reinigen und fortfahren mit der Heiligung in der Furcht GOttes.
\bibverse{2} Fasset uns! Wir haben niemand Leid getan; wir haben niemand
verletzt; wir haben niemand übervorteilet. \bibverse{3} Nicht sage ich
solches, euch zu verdammen; denn ich habe droben zuvor gesagt, daß ihr
in unserm Herzen seid, mitzusterben und mitzuleben. \bibverse{4} Ich
rede mit großer Freudigkeit zu euch; ich rühme viel von euch; ich bin
erfüllet mit Trost; ich bin überschwenglich in Freuden in aller unserer
Trübsal. \bibverse{5} Denn da wir nach Mazedonien kamen, hatte unser
Fleisch keine Ruhe, sondern allenthalben waren wir in Trübsal: auswendig
Streit, inwendig Furcht. \bibverse{6} Aber GOtt, der die Geringen
tröstet, der tröstete uns durch die Ankunft des Titus. \bibverse{7}
Nicht allein aber durch seine Ankunft, sondern auch durch den Trost,
damit er getröstet war an euch, und verkündigte uns euer Verlangen, euer
Weinen, euren Eifer um mich, also daß ich mich noch mehr freuete.
\bibverse{8} Denn daß ich euch durch den Brief habe traurig gemacht,
reuet mich nicht. Und ob's mich reuete, so ich aber sehe, daß der Brief
vielleicht eine Weile euch betrübt hat, \bibverse{9} so freue ich mich
doch nun, nicht darüber, daß ihr seid betrübt worden, sondern daß ihr
seid betrübt worden zur Reue. Denn ihr seid göttlich betrübt worden, daß
ihr von uns ja keinen Schaden irgend worinnen nehmet. \bibverse{10} Denn
die göttliche Traurigkeit wirket zur Seligkeit eine Reue, die niemand
gereuet; die Traurigkeit aber der Welt wirket den Tod. \bibverse{11}
Siehe, dasselbige, daß ihr göttlich seid betrübt worden, welchen Fleiß
hat es in euch gewirket, dazu Verantwortung, Zorn, Furcht, Verlangen,
Eifer, Rache! Ihr habt euch beweiset in allen Stücken, daß ihr rein seid
an der Tat. \bibverse{12} Darum, ob ich euch geschrieben habe, so ist's
doch nicht geschehen um deswillen, der beleidiget hat, auch nicht um
deswillen, der beleidiget ist, sondern um deswillen, daß euer Fleiß
gegen uns offenbar würde bei euch vor GOtt. \bibverse{13} Derhalben sind
wir getröstet worden, daß ihr getröstet seid. Überschwenglicher aber
haben wir uns noch mehr gefreuet über die Freude des Titus; denn sein
Geist ist erquicket an euch allen. \bibverse{14} Denn was ich vor ihm
von euch gerühmet habe, bin ich nicht zuschanden worden; sondern
gleichwie alles wahr ist, was ich mit euch geredet habe, also ist auch
unser Rühmen vor Titus wahr worden. \bibverse{15} Und er ist überaus
herzlich wohl an euch, wenn er gedenket an euer aller Gehorsam, wie ihr
ihn mit Furcht und Zittern habt aufgenommen. \bibverse{16} Ich freue
mich, daß ich mich zu euch alles versehen darf.

\hypertarget{section-7}{%
\section{8}\label{section-7}}

\bibverse{1} Ich tue euch kund, liebe Brüder, die Gnade GOttes, die in
den Gemeinden in Mazedonien gegeben ist. \bibverse{2} Denn ihre Freude
war da überschwenglich, da sie durch viel Trübsal arm waren, haben sie
doch reichlich gegeben in aller Einfältigkeit. \bibverse{3} Denn nach
allem Vermögen (das zeuge ich) und über Vermögen waren sie selbst willig
\bibverse{4} und fleheten uns mit vielem Ermahnen, daß wir aufnähmen die
Wohltat und Gemeinschaft der Handreichung, die da geschieht den
Heiligen. \bibverse{5} Und nicht, wie wir hofften, sondern ergaben sich
selbst zuerst dem HErrn und danach uns durch den Willen GOttes,
\bibverse{6} daß wir mußten Titus ermahnen, auf daß er, wie er zuvor
hatte angefangen, also auch unter euch solche Wohltat ausrichtete.
\bibverse{7} Aber gleichwie ihr in allen Stücken reich seid, im Glauben
und im Wort und in der Erkenntnis und in allerlei Fleiß und in eurer
Liebe zu uns, also schaffet, daß ihr auch in dieser Wohltat reich seid.
\bibverse{8} Nicht sage ich, daß ich etwas gebiete, sondern dieweil
andere so fleißig sind, versuche ich auch eure Liebe, ob sie rechter Art
sei. \bibverse{9} Denn ihr wisset die Gnade unsers HErrn JEsu Christi,
daß, ob er wohl reich ist, ward er doch arm um euretwillen, auf daß ihr
durch seine Armut reich würdet. \bibverse{10} Und mein Wohlmeinen
hierinnen gebe ich. Denn solches ist euch nützlich, die ihr angefangen
habt vor dem Jahre her, nicht allein das Tun, sondern auch das Wollen.
\bibverse{11} Nun aber vollbringet auch das Tun, auf daß, gleichwie da
ist ein geneigt Gemüt zu wollen, so sei auch da ein geneigt Gemüt zu tun
von dem, was ihr habt. \bibverse{12} Denn so einer willig ist, so ist er
angenehm, nachdem er hat, nicht nachdem er nicht hat. \bibverse{13}
Nicht geschieht das der Meinung, daß die andern Ruhe haben und ihr
Trübsal, sondern daß es gleich sei. \bibverse{14} So diene euer Überfluß
ihrem Mangel diese (teure) Zeit lang, auf daß auch ihr Überschwang
hernach diene eurem Mangel, und geschehe, was gleich ist. \bibverse{15}
Wie geschrieben stehet: Der viel sammelte, hatte nicht Überfluß, und der
wenig sammelte, hatte nicht Mangel. \bibverse{16} GOtt aber sei Dank,
der solchen Fleiß für euch gegeben hat in das Herz des Titus!
\bibverse{17} Denn er nahm zwar die Ermahnung an; aber dieweil er so
sehr fleißig war, ist er von selber zu euch gereiset. \bibverse{18} Wir
haben aber einen Bruder mit ihm gesandt; der das Lob hat am Evangelium
durch alle Gemeinden; \bibverse{19} nicht allein aber das, sondern er
ist auch verordnet von den Gemeinden zum Gefährten unserer Fahrt in
dieser Wohltat, welche durch uns ausgerichtet wird dem HErrn zu Ehren
und (zum Preis) eures guten Willens. \bibverse{20} Und verhüten das, daß
uns nicht jemand übel nachreden möge solcher reichen Steuer halben, die
durch uns ausgerichtet wird. \bibverse{21} und sehen darauf, daß es
redlich zugehe, nicht allein vor dem HErrn, sondern auch vor den
Menschen. \bibverse{22} Auch haben wir mit ihnen gesandt unsern Bruder,
den wir oft gespüret haben in vielen Stücken, daß er fleißig sei, nun
aber viel fleißiger. \bibverse{23} Und wir sind großer Zuversicht zu
euch, es sei des Titus halben, welcher mein Geselle und Gehilfe unter
euch ist, oder unserer Brüder halben, welche Apostel sind der Gemeinden
und eine Ehre Christi. \bibverse{24} Erzeiget nun die Beweisung eurer
Liebe und unsers Ruhms von euch an diesen, auch öffentlich vor den
Gemeinden.

\hypertarget{section-8}{%
\section{9}\label{section-8}}

\bibverse{1} Denn von solcher Steuer, die den Heiligen geschieht, ist
mir nicht not, euch zu schreiben. \bibverse{2} Denn ich weiß euren guten
Willen, davon ich rühme bei denen aus Mazedonien (und sage): Achaja ist
vor dem Jahr bereit gewesen. Und euer Exempel hat viele gereizet.
\bibverse{3} Ich habe aber diese Brüder darum gesandt, daß nicht unser
Ruhm von euch zunichte würde in dem Stücke, und daß ihr bereit seid,
gleichwie ich von euch gesagt habe, \bibverse{4} auf daß nicht, so die
aus Mazedonien mit mir kämen und euch unbereitet fänden, wir (will nicht
sagen ihr) zuschanden würden mit solchem Rühmen. \bibverse{5} Ich habe
es aber für nötig angesehen, die Brüder zu ermahnen, daß sie voranzögen
zu euch, zu verfertigen diesen zuvor verheißenen Segen, daß er bereitet
sei, also daß es sei ein Segen und nicht ein Geiz. \bibverse{6} Ich
meine aber das: Wer da kärglich säet, der wird auch kärglich ernten; und
wer da säet im Segen, der wird auch ernten im Segen. \bibverse{7} Ein
jeglicher nach seiner Willkür, nicht mit Unwillen oder aus Zwang; denn
einen fröhlichen Geber hat GOtt lieb. \bibverse{8} GOtt aber kann
machen, daß allerlei Gnade unter euch reichlich sei, daß ihr in allen
Dingen volle Genüge habet und reich seid zu allerlei guten Werken,
\bibverse{9} wie geschrieben stehet: Er hat ausgestreuet und gegeben den
Armen; seine Gerechtigkeit bleibet in Ewigkeit. \bibverse{10} Der aber
Samen reichet beim Säemann, der wird je auch das Brot reichen zur Speise
und wird vermehren euren Samen und wachsen lassen das Gewächs eurer
Gerechtigkeit, \bibverse{11} daß ihr reich seid in allen Dingen mit
aller Einfältigkeit, welche wirket durch uns Danksagung GOtt.
\bibverse{12} Denn die Handreichung dieser Steuer erfüllet nicht allein
den Mangel der Heiligen, sondern ist auch überschwenglich darinnen, daß
viele GOtt danken für diesen unsern treuen Dienst \bibverse{13} und
preisen GOtt über eurem untertänigen Bekenntnis des Evangeliums Christi
und über eurer einfältigen Steuer an sie und an alle \bibverse{14} und
über ihrem Gebet für euch, welche verlanget nach euch, um der
überschwenglichen Gnade GOttes willen in euch. \bibverse{15} GOtt aber
sei Dank für seine unaussprechliche Gabe!

\hypertarget{section-9}{%
\section{10}\label{section-9}}

\bibverse{1} Ich aber, Paulus, ermahne euch durch die Sanftmütigkeit und
Lindigkeit Christi, der ich gegenwärtig unter euch gering bin, im
Abwesen aber bin ich türstig gegen euch. \bibverse{2} Ich bitte aber,
daß mir nicht not sei, gegenwärtig türstig zu handeln und der Kühnheit
zu brauchen, die man mir zumisset, gegen etliche, die uns schätzen, als
wandelten wir fleischlicherweise. \bibverse{3} Denn ob wir wohl im
Fleisch wandeln, so streiten wir doch nicht fleischlicherweise.
\bibverse{4} Denn die Waffen unserer Ritterschaft sind nicht
fleischlich, sondern mächtig vor GOtt, zu zerstören die Befestungen,
\bibverse{5} damit wir zerstören die Anschläge und alle Höhe, die sich
erhebet wider die Erkenntnis GOttes und nehmen gefangen alle Vernunft
unter den Gehorsam Christi \bibverse{6} und sind bereit, zu rächen allen
Ungehorsam, wenn euer Gehorsam erfüllet ist. \bibverse{7} Richtet ihr
nach dem Ansehen? Verläßt sich jemand darauf, daß er Christo angehöre,
der denke solches, auch wieder um bei ihm, daß, gleichwie er Christo
angehöret, also gehören wir auch Christo an. \bibverse{8} Und so ich
auch etwas weiter, mich rühmete von unserer Gewalt, welche uns der HErr
gegeben hat, euch zu bessern und nicht zu verderben, wollte ich nicht
zuschanden werden. \bibverse{9} (Das sage ich aber,) daß ihr nicht euch
dünken lasset, als hätte ich euch wollen schrecken mit Briefen.
\bibverse{10} Denn die Briefe (sprechen, sie) sind schwer und stark;
aber die Gegenwärtigkeit des Leibes ist schwach und die Rede
verächtlich. \bibverse{11} Wer ein solcher ist, der denke, daß wie wir
sind mit Worten in den Briefen im Abwesen, so dürfen wir auch wohl sein
mit der Tat gegenwärtig. \bibverse{12} Denn wir dürfen uns nicht unter
die rechnen oder zählen, so sich selbst loben; aber dieweil sie sich bei
sich selbst messen und halten allein von sich selbst, verstehen sie
nichts. \bibverse{13} Wir aber rühmen uns nicht über das Ziel, sondern
nur nach dem Ziel der Regel, damit uns GOtt abgemessen hat das Ziel, zu
gelangen auch bis an euch. \bibverse{14} Denn wir fahren nicht zu weit,
als wären wir nicht gelanget bis zu euch; denn wir sind ja auch bis zu
euch kommen mit dem Evangelium Christi. \bibverse{15} Und rühmen uns
nicht über das Ziel in fremder Arbeit und haben Hoffnung, wenn nun euer
Glaube in euch gewachsen, daß wir unserer Regel nach wollen weiter
kommen \bibverse{16} und das Evangelium auch predigen denen, die jenseit
euch wohnen, und uns nicht rühmen in dem, was mit fremder Regel bereitet
ist. \bibverse{17} Wer sich aber rühmet, der rühme sich des HErrn.
\bibverse{18} Denn darum ist einer nicht tüchtig, daß er sich selbst
lobet, sondern daß ihn der HErr lobet.

\hypertarget{section-10}{%
\section{11}\label{section-10}}

\bibverse{1} Wollte GOtt, ihr hieltet mir ein wenig Torheit zugut! Doch
ihr haltet mir's wohl zugut. \bibverse{2} Denn ich eifere um euch mit
göttlichem Eifer. Denn ich habe euch vertrauet einem Manne, daß ich eine
reine Jungfrau Christo zubrächte. \bibverse{3} Ich fürchte aber, daß
nicht, wie die Schlange Eva verführete mit ihrer Schalkheit, also auch
eure Sinne verrücket werden von der Einfältigkeit in Christo.
\bibverse{4} Denn so, der da zu euch kommt, einen andern JEsum predigte,
den wir nicht geprediget haben, oder ihr einen andern Geist empfinget,
den ihr nicht empfangen habt, oder ein ander Evangelium, das ihr nicht
angenommen habt, so vertrüget ihr's. \bibverse{5} Denn ich achte, ich
sei nicht weniger, denn die hohen Apostel sind. \bibverse{6} Und ob ich
albern bin mit Reden, so bin ich doch nicht albern in der Erkenntnis.
Doch, ich bin bei euch allenthalben wohlbekannt. \bibverse{7} Oder habe
ich gesündiget, daß ich mich erniedriget habe, auf daß ihr erhöhet
würdet? Denn ich habe euch das Evangelium umsonst verkündigt
\bibverse{8} und habe andere Gemeinden beraubet und Sold von ihnen
genommen, daß ich euch predigte. \bibverse{9} Und da ich bei euch war
gegenwärtig und Mangel hatte, war ich niemand beschwerlich (denn meinen
Mangel erstatteten die Brüder, die aus Mazedonien kamen); und habe mich
in allen Stücken euch unbeschwerlich gehalten und will auch noch mich
also halten. \bibverse{10} So gewiß die Wahrheit Christi in mir ist, so
soll mir dieser Ruhm in den Ländern Achajas nicht gestopfet werden.
\bibverse{11} Warum das? Daß ich euch nicht sollte liebhaben? GOtt weiß
es. \bibverse{12} Was ich aber tue und tun will, das tue ich darum, daß
ich die Ursache abhaue denen, die Ursache suchen, daß sie rühmen
möchten, sie seien wie wir. \bibverse{13} Denn solche falsche Apostel
und trügliche Arbeiter verstellen sich zu Christi Aposteln.
\bibverse{14} Und das ist auch kein Wunder; denn er selbst, der Satan,
verstellet sich zum Engel des Lichts. \bibverse{15} Darum ist es nicht
ein Großes, ob sich auch seine Diener verstellen als Prediger der
Gerechtigkeit; welcher Ende sein wird nach ihren Werken. \bibverse{16}
Ich sage abermal, daß nicht jemand wähne, ich sei töricht; wo aber
nicht, so nehmet mich an als einen Törichten, daß ich mich auch ein
wenig rühme. \bibverse{17} Was ich jetzt rede, das rede ich nicht als im
HErrn, sondern als in der Torheit, dieweil wir in das Rühmen kommen
sind. \bibverse{18} Sintemal viele sich rühmen nach dem Fleisch, will
ich mich auch rühmen. \bibverse{19} Denn ihr vertraget gerne die Narren,
dieweil ihr klug seid. \bibverse{20} Ihr vertraget, so euch jemand zu
Knechten macht, so euch jemand schindet, so euch jemand nimmt, so jemand
euch trotzet, so euch jemand in das Angesicht streicht. \bibverse{21}
Das sage ich nach der Unehre, als wären wir schwach worden. Worauf nun
jemand kühn ist (ich rede in Torheit), darauf bin ich auch kühn.
\bibverse{22} Sie sind Hebräer, ich auch. Sie sind Israeliter, ich auch.
Sie sind Abrahams Same, ich auch. \bibverse{23} Sie sind Diener Christi;
(ich rede töricht) ich bin wohl mehr. Ich habe mehr gearbeitet, ich habe
mehr Schläge erlitten, ich bin öfter gefangen, oft in Todesnöten
gewesen. \bibverse{24} Von den Juden habe ich fünfmal empfangen vierzig
Streiche weniger eines. \bibverse{25} Ich bin dreimal gestäupet, einmal
gesteiniget, dreimal habe ich Schiffbruch erlitten, Tag und Nacht habe
ich zugebracht in der Tiefe (des Meers). \bibverse{26} Ich bin oft
gereiset; ich bin in Gefahr gewesen zu Wasser, in Gefahr unter den
Mördern, in Gefahr unter den Juden, in Gefahr unter den Heiden, in
Gefahr in den Städten, in Gefahr in der Wüste, in Gefahr auf dem Meer,
in Gefahr unter den falschen Brüdern, \bibverse{27} in Mühe und Arbeit,
in viel Wachen, in Hunger und Durst, in viel Fasten, in Frost und Blöße,
\bibverse{28} ohne was sich sonst zuträgt, nämlich daß ich täglich werde
angelaufen und trage Sorge für alle Gemeinden. \bibverse{29} Wer ist
schwach, und ich werde nicht schwach? Wer wird geärgert, und ich brenne
nicht? \bibverse{30} So ich mich je rühmen soll, will ich mich meiner
Schwachheit rühmen. \bibverse{31} GOtt und der Vater unsers HErrn JEsu
Christi, welcher sei gelobet in Ewigkeit, weiß, daß ich nicht lüge.
\bibverse{32} Zu Damaskus, der Landpfleger des Königs Aretas verwahrete
die Stadt der Damasker und wollte mich greifen; \bibverse{33} und ich
ward einem Korbe zum Fenster aus durch die Mauer niedergelassen und
entrann aus seinen Händen.

\hypertarget{section-11}{%
\section{12}\label{section-11}}

\bibverse{1} Es ist mir ja das Rühmen nichts nütze; doch will ich kommen
auf die Gesichte und Offenbarungen des HErrn. \bibverse{2} Ich kenne
einen Menschen in Christo vor vierzehn Jahren (ist er in dem Leibe
gewesen, so weiß ich's nicht, oder ist er außer dem Leibe gewesen, so
weiß ich's auch nicht; GOtt weiß es); derselbige ward entzückt bis in
den dritten Himmel. \bibverse{3} Und ich kenne denselbigen Menschen (ob
er in dem Leibe oder außer dem Leibe gewesen ist, weiß ich nicht; GOtt
weiß es). \bibverse{4} Er ward entzückt in das Paradies und hörete
unaussprechliche Worte, welche kein Mensch sagen kann. \bibverse{5}
Davon will ich mich rühmen; von mir selbst aber will ich mich nichts
rühmen ohne meiner Schwachheit. \bibverse{6} Und so ich mich rühmen
wollte, täte ich darum nicht töricht; denn ich wollte die Wahrheit
sagen. Ich enthalte mich aber des, auf daß nicht jemand mich höher
achte, denn er an mir siehet, oder von mir höret, \bibverse{7} Und auf
daß ich mich nicht der hoher Offenbarung überhebe, ist mir gegeben ein
Pfahl ins Fleisch, nämlich des Satanas Engel, der mich mit Fäusten
schlage, auf daß ich mich nicht überhebe. \bibverse{8} Dafür ich dreimal
zum HErrn geflehet habe, daß er von mir wiche; \bibverse{9} und er hat
zu mir gesagt: Laß dir an meiner Gnade genügen; denn meine Kraft ist in
den Schwachen mächtig. Darum will ich mich am allerliebsten rühmen
meiner Schwachheit, auf daß die Kraft Christi bei mir wohne.
\bibverse{10} Darum bin ich gutes Muts in Schwachheiten, in Schmachen,
in Nöten, in Verfolgungen, in Ängsten um Christi willen. Denn wenn ich
schwach bin, so bin ich stark. \bibverse{11} Ich bin ein Narr worden
über den Rühmen; dazu habt ihr mich gezwungen. Denn ich sollte von euch
gelobet werden, sintemal ich nichts weniger bin, denn die hohen Apostel
sind; wiewohl ich nichts bin. \bibverse{12} Denn es sind ja eines
Apostels Zeichen unter euch geschehen mit aller Geduld; mit Zeichen und
mit Wundern und mit Taten. \bibverse{13} Welches ist's, darinnen ihr
geringer seid denn die andern Gemeinden, ohne daß ich selbst euch nicht
habe beschweret? Vergebet mir diese Sünde! \bibverse{14} Siehe, ich bin
bereit, zum drittenmal zu euch zu kommen, und will euch nicht
beschweren; denn ich suche nicht das Eure, sondern euch. Denn es sollen
nicht die Kinder den Eltern Schätze sammeln, sondern die Eltern den
Kindern. \bibverse{15} Ich will aber fast gerne darlegen und dargelegt
werden für eure Seelen; wiewohl ich euch fast sehr liebe und doch wenig
geliebt werde. \bibverse{16} Aber laß also sein, daß ich euch nicht habe
beschweret, sondern dieweil ich tückisch war, habe ich euch mit
Hinterlist gefangen. \bibverse{17} Habe ich aber auch jemand
übervorteilet durch deren etliche, die ich zu euch gesandt habe?
\bibverse{18} Ich habe Titus ermahnet und mit ihm gesandt einen Bruder.
Hat euch auch Titus übervorteilet? Haben wir nicht in einem Geist
gewandelt? Sind wir nicht in einerlei Fußtapfen gegangen? \bibverse{19}
Lasset ihr euch abermal dünken, wir verantworten uns? Wir reden in
Christo vor GOtt; aber das alles geschieht, meine Liebsten, euch zur
Besserung. \bibverse{20} Denn ich fürchte, wenn ich komme, daß ich euch
nicht finde, wie ich will, und ihr mich auch nicht findet, wie ihr
wollet: daß nicht Hader, Neid, Zorn, Zank, Afterreden, Ohrenblasen,
Aufblähen, Aufruhr da sei; \bibverse{21} daß ich nicht abermal komme,
und mich mein GOtt demütige bei euch, und müsse Leid tragen über viele,
die zuvor gesündiget und nicht Buße getan haben für die Unreinigkeit und
Hurerei und Unzucht, die sie getrieben haben.

\hypertarget{section-12}{%
\section{13}\label{section-12}}

\bibverse{1} Komme ich zum drittenmal zu euch, so soll in zweier oder
dreier Zeugen Munde bestehen allerlei Sache. \bibverse{2} Ich hab es
euch zuvor gesagt und sage es euch zuvor als gegenwärtig zum andernmal
und schreibe es nun im Abwesen denen, die zuvor gesündiget haben, und
den andern allen: Wenn ich abermal komme, so will ich nicht schonen.
\bibverse{3} Sintemal ihr suchet, daß ihr einmal gewahr werdet des, der
in mir redet, nämlich Christi, welcher unter euch nicht schwach ist,
sondern ist mächtig unter euch. \bibverse{4} Und ob er wohl gekreuziget
ist in der Schwachheit, so lebet er doch in der Kraft GOttes. Und ob wir
auch schwach sind in ihm, so leben wir doch mit ihm in der Kraft GOttes
unter euch. \bibverse{5} Versuchet euch selbst, ob ihr im Glauben seid,
prüfet euch selbst! Oder erkennet ihr euch selbst nicht, daß JEsus
Christus in euch ist? Es sei denn, daß ihr untüchtig seid. \bibverse{6}
Ich hoffe aber, ihr erkennet, daß wir nicht untüchtig sind. \bibverse{7}
Ich bitte aber GOtt, daß ihr nichts Übles tut, nicht auf daß wir tüchtig
gesehen werden, sondern auf daß ihr das Gute tut, und wir wie die
Untüchtigen seien. \bibverse{8} Denn wir können nichts wider die
Wahrheit, sondern für die Wahrheit. \bibverse{9} Wir freuen uns aber,
wenn wir schwach sind, und ihr mächtig seid. Und dasselbige wünschen wir
auch nämlich eure Vollkommenheit. \bibverse{10} Derhalben ich auch
solches abwesend schreibe, auf daß ich nicht, wenn ich gegenwärtig bin,
Schärfe brauchen müsse nach der Macht, welche mir der HErr, zu bessern
und nicht zu verderben, gegeben hat. \bibverse{11} Zuletzt, liebe
Brüder, freuet euch, seid vollkommen, tröstet euch, habt einerlei Sinn,
seid friedsam, so wird GOtt der Liebe und des Friedens mit euch sein.
\bibverse{12} Grüßet euch untereinander mit dem heiligen Kuß.
\bibverse{13} Es grüßen euch alle Heiligen. \bibverse{14} Die Gnade
unsers HErrn JEsu Christi und die Liebe GOttes und die Gemeinschaft des
Heiligen Geistes sei mit euch allen! Amen.
