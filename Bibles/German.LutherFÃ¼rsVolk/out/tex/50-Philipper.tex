\hypertarget{section}{%
\section{1}\label{section}}

\bibverse{1} Paulus und Timotheus, Knechte Jesu Christi, allen Heiligen
in Christo Jesu zu Philippi samt den Bischöfen und Dienern: \bibverse{2}
Gnade sei mit euch und Friede von Gott, unserm Vater, und dem HERRN
Jesus Christus! \bibverse{3} Ich danke meinem Gott, so oft ich euer
gedenke \bibverse{4} (welches ich allezeit tue in allem meinem Gebet für
euch alle, und tue das Gebet mit Freuden), \bibverse{5} über eure
Gemeinschaft am Evangelium vom ersten Tage an bis her, \bibverse{6} und
bin desselben in guter Zuversicht, daß, der in euch angefangen hat das
gute Werk, der wird's auch vollführen bis an den Tag Jesu Christi.
\bibverse{7} Wie es denn mir billig ist, daß ich dermaßen von euch
halte, darum daß ich euch in meinem Herzen habe in diesem meinem
Gefängnis, darin ich das Evangelium verantworte und bekräftige, als die
ihr alle mit mir der Gnade teilhaftig seid. \bibverse{8} Denn Gott ist
mein Zeuge, wie mich nach euch allen verlangt von Herzensgrund in Jesu
Christo. \bibverse{9} Und darum bete ich, daß eure Liebe je mehr und
mehr reich werde in allerlei Erkenntnis und Erfahrung, \bibverse{10} daß
ihr prüfen möget, was das Beste sei, auf daß ihr seid lauter und
unanstößig auf den Tag Christi, \bibverse{11} erfüllt mit Früchten der
Gerechtigkeit, die durch Jesum Christum geschehen in euch zur Ehre und
Lobe Gottes. \bibverse{12} Ich lasse euch aber wissen, liebe Brüder,
daß, wie es um mich steht, das ist nur mehr zur Förderung des
Evangeliums geraten, \bibverse{13} also daß meine Bande offenbar
geworden sind in Christo in dem ganzen Richthause und bei den andern
allen, \bibverse{14} und viele Brüder in dem HERRN aus meinen Banden
Zuversicht gewonnen haben und desto kühner geworden sind, das Wort zu
reden ohne Scheu. \bibverse{15} Etliche zwar predigen Christum um des
Neides und Haders willen, etliche aber aus guter Meinung. \bibverse{16}
Jene verkündigen Christum aus Zank und nicht lauter; denn sie meinen,
sie wollen eine Trübsal zuwenden meinen Banden; \bibverse{17} diese aber
aus Liebe; denn sie wissen, daß ich zur Verantwortung des Evangeliums
hier liege. \bibverse{18} Was tut's aber? Daß nur Christus verkündigt
werde allerleiweise, es geschehe zum Vorwand oder in Wahrheit, so freue
ich mich doch darin und will mich auch freuen. \bibverse{19} Denn ich
weiß, daß mir dies gelingt zur Seligkeit durch euer Gebet und durch
Handreichung des Geistes Jesu Christi, \bibverse{20} wie ich sehnlich
warte und hoffe, daß ich in keinerlei Stück zu Schanden werde, sondern
daß mit aller Freudigkeit, gleichwie sonst allezeit also auch jetzt,
Christus hoch gepriesen werde an meinem Leibe, es sei durch Leben oder
durch Tod. \bibverse{21} Denn Christus ist mein Leben, und Sterben ist
mein Gewinn. \bibverse{22} Sintemal aber im Fleisch leben dient, mehr
Frucht zu schaffen, so weiß ich nicht, welches ich erwählen soll.
\bibverse{23} Denn es liegt mir beides hart an: ich habe Lust,
abzuscheiden und bei Christo zu sein, was auch viel besser wäre;
\bibverse{24} aber es ist nötiger, im Fleisch bleiben um euretwillen.
\bibverse{25} Und in guter Zuversicht weiß ich, daß ich bleiben und bei
euch allen sein werde, euch zur Förderung und Freude des Glaubens,
\bibverse{26} auf daß ihr euch sehr rühmen möget in Christo Jesu an mir,
wenn ich wieder zu euch komme. \bibverse{27} Wandelt nur würdig dem
Evangelium Christi, auf daß, ob ich komme und sehe euch oder abwesend
von euch höre, ihr steht in einem Geist und einer Seele und samt uns
kämpfet für den Glauben des Evangeliums \bibverse{28} und euch in keinem
Weg erschrecken lasset von den Widersachern, welches ist ein Anzeichen,
ihnen der Verdammnis, euch aber der Seligkeit, und das von Gott.
\bibverse{29} Denn euch ist gegeben, um Christi willen zu tun, daß ihr
nicht allein an ihn glaubet sondern auch um seinetwillen leidet;
\bibverse{30} und habet denselben Kampf, welchen ihr an mir gesehen habt
und nun von mir höret.

\hypertarget{section-1}{%
\section{2}\label{section-1}}

\bibverse{1} Ist nun bei euch Ermahnung in Christo, ist Trost der Liebe,
ist Gemeinschaft des Geistes, ist herzliche Liebe und Barmherzigkeit,
\bibverse{2} so erfüllet meine Freude, daß ihr eines Sinnes seid,
gleiche Liebe habt, einmütig und einhellig seid. \bibverse{3} Nichts tut
durch Zank oder eitle Ehre; sondern durch Demut achte einer den andern
höher denn sich selbst, \bibverse{4} und ein jeglicher sehe nicht auf
das Seine, sondern auch auf das, was des andern ist. \bibverse{5} `1063'
Ein jeglicher sei gesinnt, wie Jesus Christus auch war: \bibverse{6}
welcher, ob er wohl in göttlicher Gestalt war, hielt er's nicht für
einen Raub, Gott gleich sein, \bibverse{7} sondern entäußerte sich
selbst und nahm Knechtsgestalt an, ward gleich wie ein andrer Mensch und
an Gebärden als ein Mensch erfunden; \bibverse{8} er erniedrigte sich
selbst und ward gehorsam bis zum Tode, ja zum Tode am Kreuz.
\bibverse{9} Darum hat ihn auch Gott erhöht und hat ihm einen Namen
gegeben, der über alle Namen ist, \bibverse{10} daß in dem Namen Jesu
sich beugen aller derer Kniee, die im Himmel und auf Erden und unter der
Erde sind, \bibverse{11} und alle Zungen bekennen sollen, daß Jesus
Christus der HERR sei, zur Ehre Gottes, des Vaters. \bibverse{12} Also,
meine Liebsten, wie ihr allezeit seid gehorsam gewesen, nicht allein in
meiner Gegenwart sondern auch nun viel mehr in meiner Abwesenheit,
schaffet, daß ihr selig werdet, mit Furcht und Zittern. \bibverse{13}
Denn Gott ist's, der in euch wirkt beides, das Wollen und das
Vollbringen, nach seinem Wohlgefallen. \bibverse{14} Tut alles ohne
Murren und ohne Zweifel, \bibverse{15} auf daß ihr seid ohne Tadel und
lauter und Gottes Kinder, unsträflich mitten unter dem unschlachtigen
und verkehrten Geschlecht, unter welchem ihr scheinet als Lichter in der
Welt, \bibverse{16} damit daß ihr haltet an dem Wort des Lebens, mir zu
einem Ruhm an dem Tage Christi, als der ich nicht vergeblich gelaufen
noch vergeblich gearbeitet habe. \bibverse{17} Und ob ich geopfert werde
über dem Opfer und Gottesdienst eures Glaubens, so freue ich mich und
freue mich mit euch allen. \bibverse{18} Dessen sollt ihr euch auch
freuen und sollt euch mit mir freuen. \bibverse{19} Ich hoffe aber in
dem HERRN Jesus, daß ich Timotheus bald werde zu euch senden, daß ich
auch erquickt werde, wenn ich erfahre, wie es um euch steht.
\bibverse{20} Denn ich habe keinen, der so gar meines Sinnes sei, der so
herzlich für euch sorgt. \bibverse{21} Denn sie suchen alle das ihre,
nicht, das Christi Jesu ist. \bibverse{22} Ihr aber wisset, daß er
rechtschaffen ist; denn wie ein Kind dem Vater hat er mir gedient am
Evangelium. \bibverse{23} Ihn, hoffe ich, werde ich senden von Stund an,
wenn ich erfahren habe, wie es um mich steht. \bibverse{24} Ich vertraue
aber in dem HERRN, daß auch ich selbst bald kommen werde. \bibverse{25}
Ich habe es aber für nötig angesehen, den Bruder Epaphroditus zu euch zu
senden, der mein Gehilfe und Mitstreiter und euer Gesandter und meiner
Notdurft Diener ist; \bibverse{26} sintemal er nach euch allen Verlangen
hatte und war hoch bekümmert, darum daß ihr gehört hattet, daß er krank
war gewesen. \bibverse{27} Und er war todkrank, aber Gott hat sich über
ihn erbarmt; nicht allein aber über ihn, sondern auch über mich, auf daß
ich nicht eine Traurigkeit über die andern hätte. \bibverse{28} Ich habe
ihn aber desto eilender gesandt, auf daß ihr ihn seht und wieder
fröhlich werdet und ich auch der Traurigkeit weniger habe. \bibverse{29}
So nehmet ihn nun auf in dem HERRN mit allen Freuden und habt solche
Leute in Ehren. \bibverse{30} Denn um des Werkes Christi willen ist er
dem Tode so nahe gekommen, da er sein Leben gering bedachte, auf daß er
mir diente an eurer Statt.

\hypertarget{section-2}{%
\section{3}\label{section-2}}

\bibverse{1} Weiter, liebe Brüder, freuet euch in dem HERRN! Daß ich
euch immer einerlei schreibe, verdrießt mich nicht und macht euch desto
gewisser. \bibverse{2} Sehet auf die Hunde, sehet auf die bösen
Arbeiter, sehet auf die Zerschneidung! \bibverse{3} Denn wir sind die
Beschneidung, die wir Gott im Geiste dienen und rühmen uns von Christo
Jesu und verlassen uns nicht auf Fleisch, \bibverse{4} wiewohl ich auch
habe, daß ich mich Fleisches rühmen könnte. So ein anderer sich dünken
läßt, er könnte sich Fleisches rühmen, ich könnte es viel mehr:
\bibverse{5} der ich am achten Tag beschnitten bin, einer aus dem Volk
von Israel, des Geschlechts Benjamin, ein Hebräer von Hebräern und nach
dem Gesetz ein Pharisäer, \bibverse{6} nach dem Eifer ein Verfolger der
Gemeinde, nach der Gerechtigkeit im Gesetz gewesen unsträflich.
\bibverse{7} Aber was mir Gewinn war, das habe ich um Christi willen für
Schaden geachtet. \bibverse{8} Ja, ich achte es noch alles für Schaden
gegen die überschwengliche Erkenntnis Christi Jesu, meines HERRN, um
welches willen ich alles habe für Schaden gerechnet, und achte es für
Kot, auf daß ich Christum gewinne \bibverse{9} und in ihm erfunden
werde, daß ich nicht habe meine Gerechtigkeit, die aus dem Gesetz,
sondern die durch den Glauben an Christum kommt, nämlich die
Gerechtigkeit, die von Gott dem Glauben zugerechnet wird, \bibverse{10}
zu erkennen ihn und die Kraft seiner Auferstehung und die Gemeinschaft
seiner Leiden, daß ich seinem Tode ähnlich werde, \bibverse{11} damit
ich gelange zur Auferstehung der Toten. \bibverse{12} Nicht, daß ich's
schon ergriffen habe oder schon vollkommen sei; ich jage ihm aber nach,
ob ich's auch ergreifen möchte, nachdem ich von Christo Jesu ergriffen
bin. \bibverse{13} Meine Brüder, ich schätze mich selbst noch nicht, daß
ich's ergriffen habe. Eines aber sage ich: Ich vergesse, was dahinten
ist, und strecke mich zu dem, was da vorne ist, \bibverse{14} und jage
nach dem vorgesteckten Ziel, nach dem Kleinod, welches vorhält die
himmlische Berufung Gottes in Christo Jesu. \bibverse{15} Wie viele nun
unser vollkommen sind, die lasset uns also gesinnt sein. Und solltet ihr
sonst etwas halten, das lasset euch Gott offenbaren; \bibverse{16} doch
soferne, daß wir nach derselben Regel, darin wir gekommen sind, wandeln
und gleich gesinnt seien. \bibverse{17} Folget mir, liebe Brüder, und
sehet auf die, die also wandeln, wie ihr uns habt zum Vorbilde.
\bibverse{18} Denn viele wandeln, von welchen ich euch oft gesagt habe,
nun aber sage ich auch mit Weinen, daß sie sind die Feinde des Kreuzes
Christi, \bibverse{19} welcher Ende ist die Verdammnis, welchen der
Bauch ihr Gott ist, und deren Ehre zu Schanden wird, die irdisch gesinnt
sind. \bibverse{20} Unser Wandel aber ist im Himmel, von dannen wir auch
warten des Heilands Jesu Christi, des HERRN, \bibverse{21} welcher
unsern nichtigen Leib verklären wird, daß er ähnlich werde seinem
verklärten Leibe nach der Wirkung, mit der er kann auch alle Dinge sich
untertänig machen.

\hypertarget{section-3}{%
\section{4}\label{section-3}}

\bibverse{1} Also, meine lieben und ersehnten Brüder, meine Freude und
meine Krone, besteht also in dem HERRN, ihr Lieben. \bibverse{2} Die
Evodia ermahne ich, und die Syntyche ermahne ich, daß sie eines Sinnes
seien in dem HERRN. \bibverse{3} Ja ich bitte auch dich, mein treuer
Geselle, stehe ihnen bei, die samt mir für das Evangelium gekämpft
haben, mit Klemens und meinen andern Gehilfen, welcher Namen sind in dem
Buch des Lebens. \bibverse{4} Freuet euch in dem HERRN allewege! Und
abermals sage ich: Freuet euch! \bibverse{5} Eure Lindigkeit lasset kund
sein allen Menschen! der HERR ist nahe! \bibverse{6} Sorget nichts!
sondern in allen Dingen lasset eure Bitten im Gebet und Flehen mit
Danksagung vor Gott kund werden. \bibverse{7} Und der Friede Gottes,
welcher höher ist denn alle Vernunft, bewahre eure Herzen und Sinne in
Christo Jesu! \bibverse{8} Weiter, liebe Brüder, was wahrhaftig ist, was
ehrbar, was gerecht, was keusch, was lieblich, was wohllautet, ist etwa
eine Tugend, ist etwa ein Lob, dem denket nach! \bibverse{9} Welches ihr
auch gelernt und empfangen und gehört und gesehen habt an mir, das tut;
so wird der Gott des Friedens mit euch sein. \bibverse{10} Ich bin aber
höchlich erfreut in dem HERRN, daß ihr wieder wacker geworden seid, für
mich zu sorgen; wiewohl ihr allewege gesorgt habt, aber die Zeit hat's
nicht wollen leiden. \bibverse{11} Nicht sage ich das des Mangels
halben; denn ich habe gelernt, worin ich bin, mir genügen zu lassen.
\bibverse{12} Ich kann niedrig sein und kann hoch sein; ich bin in allen
Dingen und bei allen geschickt, beides, satt sein und hungern, beides,
übrighaben und Mangel leiden. \bibverse{13} Ich vermag alles durch den,
der mich mächtig macht, Christus. \bibverse{14} Doch ihr habt wohl
getan, daß ihr euch meiner Trübsal angenommen habt. \bibverse{15} Ihr
aber von Philippi wisset, daß von Anfang des Evangeliums, da ich auszog
aus Mazedonien, keine Gemeinde mit mir geteilt hat nach der Rechnung der
Ausgabe und Einnahme als ihr allein. \bibverse{16} Denn auch gen
Thessalonich sandtet ihr zu meiner Notdurft einmal und darnach noch
einmal. \bibverse{17} Nicht, daß ich das Geschenk suche; sondern ich
suche die Frucht, daß sie reichlich in eurer Rechnung sei. \bibverse{18}
Denn ich habe alles und habe überflüssig. Ich habe die Fülle, da ich
empfing durch Epaphroditus, was von euch kam: ein süßer Geruch, ein
angenehmes Opfer, Gott gefällig. \bibverse{19} Mein Gott aber fülle aus
alle eure Notdurft nach seinem Reichtum in der Herrlichkeit in Christo
Jesu. \bibverse{20} Gott aber, unserm Vater, sei Ehre von Ewigkeit zu
Ewigkeit! Amen. \bibverse{21} Grüßet alle Heiligen in Christo Jesu. Es
grüßen euch die Brüder, die bei mir sind. \bibverse{22} Es grüßen euch
alle Heiligen, sonderlich aber die von des Kaisers Hause. \bibverse{23}
Die Gnade unsers HERRN Jesu Christi sei mit euch allen! Amen.
