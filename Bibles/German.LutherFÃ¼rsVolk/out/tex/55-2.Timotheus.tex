\hypertarget{section}{%
\section{1}\label{section}}

\bibverse{1} Paulus, ein Apostel JEsu Christi durch den Willen GOttes
nach der Verheißung des Lebens in Christo JEsu: \bibverse{2} Meinem
lieben Sohn Timotheus Gnade, Barmherzigkeit, Friede von GOtt dem Vater
und Christo JEsu, unserm HErrn. \bibverse{3} Ich danke GOtt dem ich
diene von meinen Voreltern her in reinem Gewissen, daß ich ohne Unterlaß
dein gedenke in meinem Gebet Tag und Nacht. \bibverse{4} Und mich
verlanget, dich zu sehen, wenn ich denke an deine Tränen, auf daß ich
mit Freuden erfüllet werde. \bibverse{5} Und erinnere mich des
ungefärbten Glaubens in dir, welcher zuvor gewohnet hat in deiner
Großmutter Lois und in deiner Mutter Eunike, bin aber gewiß, daß auch in
dir. \bibverse{6} Um welcher Sache willen ich dich erinnere, daß du
erweckest die Gabe GOttes, die in dir ist durch die Auflegung meiner
Hände. \bibverse{7} Denn GOtt hat uns nicht gegeben den Geist der
Furcht, sondern der Kraft und der Liebe und der Zucht. \bibverse{8}
Darum so schäme dich nicht des Zeugnisses unsers HErrn noch meiner, der
ich sein Gebundener bin, sondern leide dich mit dem Evangelium wie ich
nach der Kraft GOttes, \bibverse{9} der uns hat selig gemacht und
berufen mit einem heiligen Ruf, nicht nach unsern Werken, sondern nach
seinem Vorsatz und Gnade, die uns gegeben ist in Christo JEsu vor der
Zeit der Welt, \bibverse{10} jetzt aber offenbart durch die Erscheinung
unsers Heilandes JEsu Christi, der dem Tode die Macht hat genommen und
das Leben und ein unvergänglich Wesen ans Licht gebracht durch das
Evangelium, \bibverse{11} zu welchem ich gesetzt bin ein Prediger und
Apostel und Lehrer der Heiden. \bibverse{12} Um welcher Sache willen ich
solches leide, aber ich schäme mich's nicht; denn ich weiß, an wen ich
glaube, und bin gewiß, daß er kann mir meine Beilage bewahren bis an
jenen Tag. \bibverse{13} Halt an dem Vorbilde der heilsamen Worte, die
du von mir gehört hast, vom Glauben und von der Liebe in Christo JEsu.
\bibverse{14} Diese gute Beilage bewahre durch den Heiligen Geist, der
in uns wohnet. \bibverse{15} Das weißt du, daß sich gewendet haben von
mir alle, die in Asien sind, unter welchen ist Phygellus und Hermogenes.
\bibverse{16} Der HErr gebe Barmherzigkeit dem Hause Onesiphorus; denn
er hat mich oft erquicket und hat sich meiner Ketten nicht geschämet,
\bibverse{17} sondern da er zu Rom war, suchte er mich aufs fleißigste
und fand mich. \bibverse{18} Der HErr gebe ihm, daß er finde
Barmherzigkeit bei dem HErrn an jenem Tage! Und wieviel er mir zu
Ephesus gedienet hat, weißt du am besten.

\hypertarget{section-1}{%
\section{2}\label{section-1}}

\bibverse{1} So sei nun stark, mein Sohn, durch die Gnade in Christo
JEsu! \bibverse{2} Und was du von mir gehöret hast durch viel Zeugen,
das befiehl treuen Menschen, die da tüchtig sind, auch andere zu lehren.
\bibverse{3} Leide dich als ein guter Streiter JEsu Christi!
\bibverse{4} Kein Kriegsmann flicht sich in Händel der Nahrung, auf daß
er gefalle dem, der ihn angenommen hat. \bibverse{5} Und so jemand auch
kämpfet, wird er doch nicht gekrönet, er kämpfe denn recht. \bibverse{6}
Es soll aber der Ackermann, der den Acker bauet, der Früchte am ersten
genießen. Merke, was ich sage! \bibverse{7} Der HErr aber wird dir in
allen Dingen Verstand geben. \bibverse{8} Halt im Gedächtnis JEsum
Christum, der auferstanden ist von den Toten, aus dem Samen Davids, nach
meinem Evangelium, \bibverse{9} über welchem ich leide bis an die Bande
als ein Übeltäter. Aber GOttes Wort ist nicht gebunden. \bibverse{10}
Darum dulde ich alles um der Auserwählten willen, auf daß auch sie die
Seligkeit erlangen in Christo JEsu mit ewiger Herrlichkeit.
\bibverse{11} Das ist je gewißlich wahr: Sterben wir mit, so werden wir
mitleben; \bibverse{12} dulden wir, so werden wir mitherrschen;
verleugnen wir, so wird er uns auch verleugnen. \bibverse{13} Glauben
wir nicht, so bleibet er treu; er kann sich selbst nicht leugnen.
\bibverse{14} Solches erinnere sie und bezeuge vor dem HErrn, daß sie
nicht um Worte zanken, welches nichts nütze ist, denn zu verkehren, die
da zuhören. \bibverse{15} Befleißige dich, GOtt zu erzeigen einen
rechtschaffenen, unsträflichen Arbeiter, der da recht teile das Wort der
Wahrheit. \bibverse{16} Des ungeistlichen; losen Geschwätzes entschlage
dich; denn es hilft viel zum ungöttlichen Wesen. \bibverse{17} Und ihr
Wort frißt um sich wie der Krebs, unter welchen ist Hymenäus und
Philetus, \bibverse{18} welche der Wahrheit gefehlet haben und sagen,
die Auferstehung sei schon geschehen, und haben etlicher Glauben
verkehret. \bibverse{19} Aber der feste Grund GOttes bestehet und hat
dieses Siegel: Der HErr kennet die Seinen, und: Es trete ab von
Ungerechtigkeit, wer den Namen Christi nennet. \bibverse{20} In einem
großen Hause aber sind nicht allein güldene und silberne Gefäße, sondern
auch hölzerne und irdene und etliche zu Ehren, etliche aber zu Unehren.
\bibverse{21} So nun jemand sich reiniget von solchen Leuten, der wird
ein geheiliget Faß sein zu Ehren, dem Hausherrn bräuchlich und zu allem
guten Werk bereitet. \bibverse{22} Flieh die Lüste der Jugend! Jage aber
nach der Gerechtigkeit, dem Glauben, der Liebe, dem Frieden mit allen,
die den HErrn anrufen von reinem Herzen. \bibverse{23} Aber der
törichten und unnützen Fragen entschlage dich; denn du weißt, daß sie
nur Zank gebären. \bibverse{24} Ein Knecht aber des HErrn soll nicht
zänkisch sein, sondern freundlich gegen jedermann, lehrhaftig, der die
Bösen tragen kann mit Sanftmut \bibverse{25} und strafe die
Widerspenstigen, ob ihnen GOtt dermaleinst Buße gäbe die Wahrheit zu
erkennen, \bibverse{26} und wieder nüchtern würden aus des Teufels
Strick, von dem sie gefangen sind zu seinem Willen.

\hypertarget{section-2}{%
\section{3}\label{section-2}}

\bibverse{1} Das sollst du aber wissen, daß in den letzten Tagen werden
greuliche Zeiten kommen. \bibverse{2} Denn es werden Menschen sein, die
von sich selbst halten, geizig, ruhmredig, hoffärtig, Lästerer, den
Eltern ungehorsam, undankbar, ungeistlich, \bibverse{3} störrig,
unversöhnlich, Schänder, unkeusch, wild, ungütig, \bibverse{4} Verräter,
Frevler, aufgeblasen, die mehr lieben Wollust denn GOtt, \bibverse{5}
die da haben den Schein eines gottseligen Wesens, aber seine Kraft
verleugnen sie. Und solche meide! \bibverse{6} Aus denselbigen sind, die
hin und her in die Häuser schleichen und führen die Weiblein gefangen,
die mit Sünden beladen sind und mit mancherlei Lüsten fahren,
\bibverse{7} lernen immerdar und können nimmer zur Erkenntnis der
Wahrheit kommen. \bibverse{8} Gleicherweise aber, wie Jannes und Jambres
dem Mose widerstunden, also widerstehen auch diese der Wahrheit; es sind
Menschen von zerrütteten Sinnen, untüchtig zum Glauben. \bibverse{9}
Aber sie werden's die Länge nicht treiben; denn ihre Torheit wird
offenbar werden jedermann, gleichwie auch jener war. \bibverse{10} Du
aber hast erfahren meine Lehre, meine Weise, meine Meinung, meinen
Glauben, meine Langmut, meine Liebe, meine Geduld, \bibverse{11} meine
Verfolgung, meine Leiden, welche mir widerfahren sind zu Antiochien, zu
Ikonien, zu Lystra, welche Verfolgung ich da ertrug: und aus allen hat
mich der HErr erlöset. \bibverse{12} Und alle, die gottselig leben
wollen in Christo JEsu, müssen Verfolgung leiden. \bibverse{13} Mit den
bösen Menschen aber und verführerischen wird's je länger, je ärger,
verführen und werden verführet. \bibverse{14} Du aber bleibe in dem, was
du gelernet hast und dir vertrauet ist, sintemal du weißt, von wem du
gelernet hast. \bibverse{15} Und weil du von Kind auf die Heilige
Schrift weißt, kann dich dieselbige unterweisen zur Seligkeit durch den
Glauben an Christum JEsum. \bibverse{16} Denn alle Schrift, von GOtt
eingegeben, ist nutze zur Lehre, zur Strafe, zur Besserung, zur
Züchtigung in der Gerechtigkeit, \bibverse{17} daß ein Mensch GOttes sei
vollkommen, zu allem guten Werk geschickt.

\hypertarget{section-3}{%
\section{4}\label{section-3}}

\bibverse{1} So bezeuge ich nun vor GOtt und dem HErrn JEsu Christo, der
da zukünftig ist, zu richten die Lebendigen und die Toten, mit seiner
Erscheinung und mit seinem Reich: \bibverse{2} Predige das Wort; halt
an, es sei zu rechter Zeit oder zur Unzeit; strafe, dräue, ermahne mit
aller Geduld und Lehre! \bibverse{3} Denn es wird eine Zeit sein, da sie
die heilsame Lehre nicht leiden werden, sondern nach ihren eigenen
Lüsten werden sie sich selbst Lehrer aufladen, nach dem ihnen die Ohren
jucken; \bibverse{4} und werden die Ohren von der Wahrheit wenden und
sich zu den Fabeln kehren. \bibverse{5} Du aber sei nüchtern
allenthalben. Leide dich, tu das Werk eines evangelischen Predigers,
richte dein Amt redlich aus. \bibverse{6} Denn ich werde schon geopfert,
und die Zeit meines Abscheidens ist vorhanden. \bibverse{7} Ich habe
einen guten Kampf gekämpfet; ich habe den Lauf vollendet; ich habe
Glauben gehalten. \bibverse{8} Hinfort ist mir beigelegt die Krone der
Gerechtigkeit, welche mir der HErr an jenem Tage, der gerechte Richter,
geben wird, nicht mir aber allein, sondern auch allen, die seine
Erscheinung liebhaben. \bibverse{9} Fleißige dich, daß du bald zu mir
kommest. \bibverse{10} Denn Demas hat mich verlassen und diese Welt
liebgewonnen und ist gen Thessalonich gezogen, Krescens nach Galatien,
Titus nach Dalmatien. \bibverse{11} Lukas ist allein bei mir. Markus
nimm zu dir und bringe ihn mit dir; denn er ist mir nützlich zum Dienst.
\bibverse{12} Tychikus habe ich gen Ephesus gesandt. \bibverse{13} Den
Mantel, den ich zu Troas ließ bei Karpo, bringe mit, wenn du kommst und
die Bücher, sonderlich aber das Pergament. \bibverse{14} Alexander, der
Schmied hat mir viel Böses beweiset; der HErr bezahle ihm nach seinen
Werken! \bibverse{15} Vor welchem hüte du dich auch; denn er hat unsern
Worten sehr widerstanden. \bibverse{16} In meiner ersten Verantwortung
stund niemand bei mir, sondern sie verließen mich alle. Es sei ihnen
nicht zugerechnet! \bibverse{17} Der HErr aber stund mir bei und stärkte
mich, auf daß durch mich die Predigt bestätiget würde, und alle Heiden
höreten. Und ich bin erlöset von des Löwen Rachen. \bibverse{18} Der
HErr aber wird mich erlösen von allem Übel und aushelfen zu seinem
himmlischen Reich; welchem sei Ehre von Ewigkeit zu Ewigkeit! Amen.
\bibverse{19} Grüße Priska und Aquila und das Haus Onesiphorus.
\bibverse{20} Erastus blieb zu Korinth; Trophimus aber ließ ich zu Milet
krank. \bibverse{21} Tu Fleiß, daß du vor dem Winter kommest. Es grüßet
dich Eubulus und Pudens und Linus und Klaudia und alle Brüder.
\bibverse{22} Der HErr Jesus Christus sei mit deinem Geiste! Die Gnade
sei mit euch! Amen.
