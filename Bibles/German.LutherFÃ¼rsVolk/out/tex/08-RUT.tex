\hypertarget{section}{%
\section{1}\label{section}}

\bibverse{1} Zu der Zeit, da die Richter regierten, ward eine Teuerung
im Lande. Und ein Mann von Bethlehem-Juda zog wallen in der Moabiter
Land mit seinem Weibe und seinen zwei Söhnen. \bibverse{2} Der hieß
Elimelech und sein Weib Naemi und seine zwei Söhne Mahlon und Chiljon;
die waren Ephrather von Bethlehem-Juda. Und da sie kamen ins Land der
Moabiter, blieben sie daselbst. \bibverse{3} Und Elimelech, der Naemi
Mann, starb, und sie blieb übrig mit ihren zwei Söhnen. \bibverse{4} Die
nahmen moabitische Weiber; eine hieß Orpa, die andere Ruth. Und da sie
daselbst gewohnt hatten ungefähr zehn Jahre, \bibverse{5} starben sie
alle beide, Mahlon und Chiljon, daß das Weib überlebte beide Söhne und
ihren Mann. \bibverse{6} Da machte sie sich auf mit ihren zwei
Schwiegertöchtern und zog wieder aus der Moabiter Lande; denn sie hatte
erfahren im Moabiterlande, daß der HERR sein Volk hatte heimgesucht und
ihnen Brot gegeben. \bibverse{7} Und sie ging aus von dem Ort, da sie
gewesen war, und ihre beiden Schwiegertöchter mit ihr. Und da sie ging
auf dem Wege, daß sie wiederkäme ins Land Juda, \bibverse{8} sprach sie
zu ihren beiden Schwiegertöchtern: Geht hin und kehrt um, eine jegliche
zu ihrer Mutter Haus. Der HERR tue an euch Barmherzigkeit, wie ihr an
den Toten und mir getan habt! \bibverse{9} Der HERR gebe euch, daß ihr
Ruhe findet, eine jegliche in ihres Mannes Hause! Und küßte sie. Da
hoben sie ihre Stimmen auf und weinten \bibverse{10} und sprachen zu
ihr: Wir wollen mit dir zu deinem Volk gehen. \bibverse{11} Aber Naemi
sprach: Kehrt um, meine Töchter! warum wollt ihr mit mir gehen? Wie kann
ich fürder Kinder in meinem Leibe haben, die eure Männer sein könnten?
\bibverse{12} Kehrt um, meine Töchter, und geht hin! denn ich bin nun zu
alt, daß ich einen Mann nehme. Und wenn ich spräche: Es ist zu hoffen,
daß ich diese Nacht einen Mann nehme und Kinder gebäre, \bibverse{13}
wie könntet ihr doch harren, bis sie groß würden? wie wolltet ihr
verziehen, daß ihr nicht Männer solltet nehmen? Nicht, meine Töchter!
denn mich jammert euer sehr; denn des HERRN Hand ist über mich
ausgegangen. \bibverse{14} Da hoben sie ihre Stimme auf und weinten noch
mehr. Und Opra küßte ihre Schwiegermutter; Ruth aber blieb bei ihr.
\bibverse{15} Sie aber sprach: Siehe, deine Schwägerin ist umgewandt zu
ihrem Volk und zu ihrem Gott; kehre auch du um, deiner Schwägerin nach.
\bibverse{16} Ruth antwortete: Rede mir nicht ein, daß ich dich
verlassen sollte und von dir umkehren. Wo du hin gehst, da will ich auch
hin gehen; wo du bleibst, da bleibe ich auch. Dein Volk ist mein Volk,
und dein Gott ist mein Gott. \bibverse{17} Wo du stirbst, da sterbe ich
auch, da will ich auch begraben werden. Der HERR tue mir dies und das,
der Tod muß mich und dich scheiden. \bibverse{18} Als sie nun sah, daß
sie fest im Sinn war, mit ihr zu gehen, ließ sie ab, mit ihr davon zu
reden. \bibverse{19} Also gingen die beiden miteinander, bis sie gen
Bethlehem kamen. Und da sie nach Bethlehem hineinkamen, regte sich die
ganze Stadt über ihnen und sprach: Ist das die Naemi? \bibverse{20} Sie
aber sprach: Heißt mich nicht Naemi, sondern Mara; denn der Allmächtige
hat mich sehr betrübt. \bibverse{21} Voll zog ich aus, aber leer hat
mich der HERR wieder heimgebracht. Warum heißt ihr mich denn Naemi, so
mich doch der HERR gedemütigt und der Allmächtige betrübt hat?
\bibverse{22} Es war aber um die Zeit, daß die Gerstenernte anging, da
Naemi mit ihrer Schwiegertochter Ruth, der Moabitin, wiederkam vom
Moabiterlande gen Bethlehem.

\hypertarget{section-1}{%
\section{2}\label{section-1}}

\bibverse{1} Es war auch ein Mann, ein Verwandter des Mannes der Naemi,
von dem Geschlecht Elimelechs, mit Namen Boas; der war ein wohlhabender
Mann. \bibverse{2} Und Ruth, die Moabitin, sprach zu Naemi: Laß mich
aufs Feld gehen und Ähren auflesen dem nach, vor dem ich Gnade finde.
Sie aber sprach zu ihr: Gehe hin, meine Tochter. \bibverse{3} Sie ging
hin, kam und las auf, den Schnittern nach, auf dem Felde. Und es begab
sich eben, daß dasselbe Feld war des Boas, der von dem Geschlecht
Elimelechs war. \bibverse{4} Und siehe, Boas kam eben von Bethlehem und
sprach zu den Schnittern: Der HERR mit euch! Sie antworteten: Der HERR
segne dich! \bibverse{5} Und Boas sprach zu seinem Knechte, der über die
Schnitter gestellt war: Wes ist die Dirne? \bibverse{6} Der Knecht, der
über die Schnitter gestellt war, antwortete und sprach: Es ist die
Dirne, die Moabitin, die mit Naemi wiedergekommen ist von der Moabiter
Lande. \bibverse{7} Denn sie sprach: Laßt mich doch auflesen und sammeln
unter den Garben, den Schnittern nach; und ist also gekommen und
dagestanden vom Morgen an bis her und bleibt wenig daheim. \bibverse{8}
Da sprach Boas zu Ruth: Hörst du es, meine Tochter? Du sollst nicht
gehen auf einen andern Acker, aufzulesen, und gehe auch nicht von
hinnen, sondern halte dich zu meinen Dirnen. \bibverse{9} Und siehe, wo
sie schneiden im Felde, da gehe ihnen nach. Ich habe meinen Knechten
geboten, daß dich niemand antaste. Und so dich dürstet, so gehe hin zu
dem Gefäß und trinke von dem, was meine Knechte schöpfen. \bibverse{10}
Da fiel sie auf ihr Angesicht und beugte sich nieder zur Erde und sprach
zu ihm: Womit habe ich die Gnade gefunden vor deinen Augen, daß du mich
ansiehst, die ich doch fremd bin? \bibverse{11} Boas antwortete und
sprach zu ihr: Es ist mir angesagt alles, was du hast getan an deiner
Schwiegermutter nach deines Mannes Tod: daß du verlassen hast deinen
Vater und deine Mutter und dein Vaterland und bist zu meinem Volk
gezogen, das du zuvor nicht kanntest. \bibverse{12} Der HERR vergelte
dir deine Tat, und dein Lohn müsse vollkommen sein bei dem HERRN, dem
Gott Israels, zu welchem du gekommen bist, daß du unter seinen Flügeln
Zuversicht hättest. \bibverse{13} Sie sprach: Laß mich Gnade vor deinen
Augen finden, mein Herr; denn du hast mich getröstet und deine Magd
freundlich angesprochen, so ich doch nicht bin wie deiner Mägde eine.
\bibverse{14} Boas sprach zu ihr, da Essenszeit war: Mache dich hier
herzu und iß vom Brot und tauche deinen Bissen in den Essig. Und sie
setzte sich zur Seite der Schnitter. Er aber legte ihr geröstete Körner
vor, und sie aß und ward satt und ließ übrig. \bibverse{15} Und da sie
sich aufmachte, zu lesen, gebot Boas seinen Knechten und sprach: Laßt
sie auch zwischen den Garben lesen und beschämt sie nicht; \bibverse{16}
Auch von den Haufen laßt übrigbleiben und laßt liegen, daß sie es
auflese, und niemand schelte sie darum. \bibverse{17} Also las sie auf
dem Felde bis zum Abend und schlug's aus, was sie aufgelesen hatte; und
es war bei einem Epha Gerste. \bibverse{18} Und sie hob's auf und kam in
die Stadt; und ihre Schwiegermutter sah es, was sie gelesen hatte. Da
zog sie hervor und gab ihr, was übriggeblieben war, davon sie satt war
geworden. \bibverse{19} Da sprach ihre Schwiegermutter zu ihr: Wo hast
du heute gelesen, und wo hast du gearbeitet? Gesegnet sei, der dich
angesehen hat! Sie aber sagte es ihrer Schwiegermutter, bei wem sie
gearbeitet hätte, und sprach: Der Mann, bei dem ich heute gearbeitet
habe, heißt Boas. \bibverse{20} Naemi aber sprach zu ihrer
Schwiegertochter: Gesegnet sei er dem HERRN! denn er hat seine
Barmherzigkeit nicht gelassen an den Lebendigen und an den Toten. Und
Naemi sprach zu ihr: Der Mann gehört zu uns und ist unser Erbe.
\bibverse{21} Ruth, die Moabitin, sprach: Er sprach auch das zu mir: Du
sollst dich zu meinen Leuten halten, bis sie mir alles eingeerntet
haben. \bibverse{22} Naemi sprach zu Ruth, ihrer Schwiegertochter: Es
ist gut, meine Tochter, daß du mit seinen Dirnen ausgehst, auf daß nicht
jemand dir dreinrede auf einem andern Acker. \bibverse{23} Also hielt
sie sich zu den Dirnen des Boas, daß sie las, bis daß die Gerstenernte
und Weizenernte aus war; und kam wieder zu ihrer Schwiegermutter.

\hypertarget{section-2}{%
\section{3}\label{section-2}}

\bibverse{1} Und Naemi, ihre Schwiegermutter, sprach zu ihr: Meine
Tochter, ich will dir Ruhe schaffen, daß dir's wohl gehe. \bibverse{2}
Nun, der Boas, unser Verwandter, bei des Dirnen du gewesen bist, worfelt
diese Nacht Gerste auf seiner Tenne. \bibverse{3} So bade dich und salbe
dich und lege dein Kleid an und gehe hinab auf die Tenne; gib dich dem
Manne nicht zu erkennen, bis er ganz gegessen und getrunken hat.
\bibverse{4} Wenn er sich dann legt, so merke den Ort, da er sich hin
legt, und komm und decke auf zu seinen Füßen und lege dich, so wird er
dir wohl sagen, was du tun sollst. \bibverse{5} Sie sprach zu ihr:
Alles, was du mir sagst, will ich tun. \bibverse{6} Sie ging hinab zur
Tenne und tat alles, wie ihre Schwiegermutter geboten hatte.
\bibverse{7} Und da Boas gegessen und getrunken hatte, ward sein Herz
guter Dinge, und er kam und legte sich hinter einen Kornhaufen; und sie
kam leise und deckte auf zu seinen Füßen und legte sich. \bibverse{8} Da
es nun Mitternacht ward, erschrak der Mann und beugte sich vor; und
siehe, ein Weib lag zu seinen Füßen. \bibverse{9} Und er sprach: Wer
bist du? Sie antwortete: Ich bin Ruth, deine Magd. Breite deine Decke
über deine Magd; denn du bist der Erbe. \bibverse{10} Er aber sprach:
Gesegnet seist du dem HERRN, meine Tochter! Du hast deine Liebe hernach
besser gezeigt den zuvor, daß du bist nicht den Jünglingen nachgegangen,
weder reich noch arm. \bibverse{11} Nun, meine Tochter, fürchte dich
nicht. Alles was du sagst, will ich dir tun; denn die ganze Stadt meines
Volkes weiß, daß du ein tugendsam Weib bist. \bibverse{12} Nun, es ist
wahr, daß ich der Erbe bin; aber es ist einer näher denn ich.
\bibverse{13} Bleibe über Nacht. Morgen, so er dich nimmt, wohl;
gelüstet's ihn aber nicht, dich zu nehmen, so will ich dich nehmen, so
wahr der HERR lebt. Schlaf bis zum Morgen. \bibverse{14} Und sie schlief
bis zum Morgen zu seinen Füßen. Und sie stand auf, ehe denn einer den
andern erkennen konnte; und er gedachte, daß nur niemand innewerde, daß
das Weib in die Tenne gekommen sei. \bibverse{15} Und sprach: Lange her
den Mantel, den du anhast, und halt ihn. Und sie hielt ihn. Und er maß
sechs Maß Gerste und legte es auf sie. Und er kam in die Stadt.
\bibverse{16} Sie aber kam zu ihrer Schwiegermutter; die sprach: Wie
steht's mit dir, meine Tochter? Und sie sagte ihr alles, was ihr der
Mann getan hatte, \bibverse{17} und sprach: Diese sechs Maß Gerste gab
er mir; denn er sprach: Du sollst nicht leer zu deiner Schwiegermutter
kommen. \bibverse{18} Sie aber sprach: Sei still, meine Tochter, bis du
erfährst, wo es hinaus will; denn der Mann wird nicht ruhen, er bringe
es denn heute zu Ende.

\hypertarget{section-3}{%
\section{4}\label{section-3}}

\bibverse{1} Boas ging hinauf ins Tor und setzte sich daselbst. Und
siehe, da der Erbe vorüberging, von welchem er geredet hatte, sprach
Boas: Komm und setze dich hierher! Und er kam und setzte sich.
\bibverse{2} Und er nahm zehn Männer von den Ältesten der Stadt und
sprach: Setzt euch her! Und sie setzten sich. \bibverse{3} Da sprach er
zu dem Erben: Naemi, die vom Lande der Moabiter wiedergekommen ist,
bietet feil das Stück Feld, das unsers Bruders war, Elimelechs.
\bibverse{4} Darum gedachte ich's vor deine Ohren zu bringen und zu
sagen: Willst du es beerben, so kaufe es vor den Bürgern und vor den
Ältesten meines Volkes; willst du es aber nicht beerben, so sage mir's,
daß ich's wisse. Denn es ist kein Erbe außer dir und ich nach dir. Er
sprach: Ich will's beerben. \bibverse{5} Boas sprach: Welches Tages du
das Feld kaufst von der Hand Naemis, so mußt du auch Ruth, die Moabitin,
des Verstorbenen Weib, nehmen, daß du dem Verstorbenen einen Namen
erweckst auf seinem Erbteil. \bibverse{6} Da sprach er: Ich vermag es
nicht zu beerben, daß ich nicht vielleicht mein Erbteil verderbe. Beerbe
du, was ich beerben soll; denn ich vermag es nicht zu beerben.
\bibverse{7} Und es war von alters her eine solche Gewohnheit in Israel:
wenn einer ein Gut nicht beerben noch erkaufen wollte, auf daß eine
Sache bestätigt würde, so zog er seinen Schuh aus und gab ihn dem
andern; das war das Zeugnis in Israel. \bibverse{8} Und der Erbe sprach
zu Boas: Kaufe du es! und zog seinen Schuh aus. \bibverse{9} Und Boas
sprach zu den Ältesten und zu allem Volk: Ihr seid heute Zeugen, daß ich
alles gekauft habe, was dem Elimelech, und alles, was Chiljon und Mahlon
gehört hat, von der Hand Naemis; \bibverse{10} dazu auch Ruth, die
Moabitin, Mahlons Weib, habe ich mir erworben zum Weibe, daß ich dem
Verstorbenen einen Namen erwecke auf sein Erbteil und sein Name nicht
ausgerottet werde unter seinen Brüdern und aus dem Tor seines Orts;
Zeugen seid ihr des heute. \bibverse{11} Und alles Volk, das im Tor war,
samt den Ältesten sprachen: Wir sind Zeugen. Der HERR mache das Weib,
das in dein Haus kommt, wie Rahel und Leah, die beide das Haus Israels
gebaut haben; und wachse sehr in Ephratha und werde gepriesen zu
Bethlehem. \bibverse{12} Und dein Haus werde wie das Haus des Perez, den
Thamar dem Juda gebar, von dem Samen, den dir der HERR geben wird von
dieser Dirne. \bibverse{13} Also nahm Boas die Ruth, daß sie sein Weib
ward. Und da er zu ihr einging, gab ihr der HERR, daß sie schwanger ward
und gebar einen Sohn. \bibverse{14} Da sprachen die Weiber zu Naemi:
Gelobt sei der HERR, der dir nicht hat lassen abgehen einen Erben zu
dieser Zeit, daß sein Name in Israel bliebe. \bibverse{15} Der wir dich
erquicken und dein Alter versorgen. Denn deine Schwiegertochter, die
dich geliebt hat, hat ihn geboren, welche dir besser ist als sieben
Söhne. \bibverse{16} Und Naemi nahm das Kind und legte es auf ihren
Schoß und ward seine Wärterin. \bibverse{17} Und ihre Nachbarinnen gaben
ihm einen Namen und sprachen: Naemi ist ein Kind geboren; und hießen ihn
Obed. Der ist der Vater Isais, welcher ist Davids Vater. \bibverse{18}
Dies ist das Geschlecht des Perez: Perez zeugte Hezron; \bibverse{19}
Hezron zeugte Ram; Ram zeugte Amminadab; \bibverse{20} Amminadab zeugte
Nahesson; Nahesson zeugte Salma; \bibverse{21} Salma zeugte Boas; Boas
zeugte Obed; \bibverse{22} Obed zeugte Isai; Isai zeugte David.
