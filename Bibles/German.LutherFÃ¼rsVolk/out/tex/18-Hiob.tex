\hypertarget{section}{%
\section{1}\label{section}}

\bibverse{1} Es war ein Mann im Lande Uz, der hieß Hiob. Derselbe war
schlecht und recht, gottesfürchtig und meidete das Böse. \bibverse{2}
Und zeugete sieben Söhne und drei Töchter. \bibverse{3} Und seines
Viehes waren siebentausend Schafe, dreitausend Kamele, fünfhundert Joch
Rinder und fünfhundert Eselinnen und sehr viel Gesindes; und er war
herrlicher denn alle, die gegen Morgen wohneten. \bibverse{4} Und seine
Söhne gingen hin und machten Wohlleben, ein jeglicher in seinem Hause
auf seinen Tag; und sandten hin und luden ihre drei Schwestern, mit
ihnen zu essen und zu trinken. \bibverse{5} Und wenn ein Tag des
Wohllebens um war, sandte Hiob hin und heiligte sie; und machte sich des
Morgens frühe auf und opferte Brandopfer nach ihrer aller Zahl. Denn
Hiob gedachte: Meine Söhne möchten gesündiget und GOtt gesegnet haben in
ihrem Herzen. Also tat Hiob alle Tage. \bibverse{6} Es begab sich aber
auf einen Tag, da die Kinder GOttes kamen und vor den HErrn traten, kam
der Satan auch unter ihnen. \bibverse{7} Der HErr aber sprach zu dem
Satan: Wo kommst du her? Satan antwortete dem HErrn und sprach: Ich habe
das Land umher durchzogen. \bibverse{8} Der HErr sprach zu Satan: Hast
du nicht achtgehabt auf meinen Knecht Hiob? Denn es ist seinesgleichen
nicht im Lande, schlecht und recht, gottesfürchtig und meidet das Böse.
\bibverse{9} Satan antwortete dem HErrn und sprach: Meinest du, daß Hiob
umsonst GOtt fürchtet? \bibverse{10} Hast du doch ihn, sein Haus und
alles, was er hat, rings umher verwahret. Du hast das Werk seiner Hände
gesegnet, und sein Gut hat sich ausgebreitet im Lande. \bibverse{11}
Aber recke deine Hand aus und taste an alles, was er hat; was gilt's, er
wird dich ins Angesicht segnen? \bibverse{12} Der HErr sprach zu Satan:
Siehe, alles, was er hat, sei in deiner Hand; ohne allein an ihn selbst
lege deine Hand nicht. Da ging Satan aus von dem HErrn. \bibverse{13}
Des Tages aber, da seine Söhne und Töchter aßen und tranken Wein in
ihres Bruders Hause, des Erstgeborenen, \bibverse{14} kam ein Bote zu
Hiob und sprach: Die Rinder pflügeten, und die Eselinnen gingen neben
ihnen an der Weide; \bibverse{15} da fielen die aus Reicharabien herein
und nahmen sie und schlugen die Knaben mit der Schärfe des Schwerts; und
ich bin allein entronnen, daß ich dir's ansagte. \bibverse{16} Da der
noch redete, kam ein anderer und sprach: Das Feuer GOttes fiel vom
Himmel und verbrannte Schafe und Knaben und verzehrete sie; und ich bin
allein entronnen, daß ich dir's ansagte. \bibverse{17} Da der noch
redete, kam einer und sprach: Die Chaldäer machten drei Spitzen und
überfielen die Kamele und nahmen sie und schlugen die Knaben mit der
Schärfe des Schwerts; und ich bin allein entronnen, daß ich dir's
ansagte. \bibverse{18} Da der noch redete, kam einer und sprach: Deine
Söhne und Töchter aßen und tranken im Hause ihres Bruders, des
Erstgebornen; \bibverse{19} und siehe, da kam ein großer Wind von der
Wüste her und stieß auf die vier Ecken des Hauses und warf's auf die
Knaben, daß sie starben; und ich bin allein entronnen, daß ich dir's
ansagte. \bibverse{20} Da stund Hiob auf und zerriß sein Kleid und
raufte sein Haupt; und fiel auf die Erde und betete an \bibverse{21} und
sprach: Ich bin nackend von meiner Mutter Leibe kommen, nackend werde
ich wieder dahinfahren. Der HErr hat's gegeben, der HErr hat's genommen;
der Name des HErrn sei gelobt! \bibverse{22} In diesem allem sündigte
Hiob nicht und tat nichts Törichtes wider GOtt.

\hypertarget{section-1}{%
\section{2}\label{section-1}}

\bibverse{1} Es begab sich aber des Tages, da die Kinder GOttes kamen
und traten vor den HErrn, daß Satan auch unter ihnen kam und vor den
HErrn trat. \bibverse{2} Da sprach der HErr zu dem Satan: Wo kommst du
her? Satan antwortete dem HErrn und sprach: Ich habe das Land umher
durchzogen. \bibverse{3} Der HErr sprach zu dem Satan: Hast du nicht
acht auf meinen Knecht Hiob gehabt? Denn es ist seinesgleichen im Lande
nicht, schlecht und recht, gottesfürchtig und meidet das Böse und hält
noch fest an seiner Frömmigkeit; du aber hast mich bewegt, daß ich ihn
ohne Ursache verderbet habe. \bibverse{4} Satan antwortete dem HErrn und
sprach: Haut für Haut; und alles, was ein Mann hat, läßt er für sein
Leben. \bibverse{5} Aber recke deine Hand aus und taste sein Gebein und
Fleisch an; was gilt's, er wird dich ins Angesicht segnen? \bibverse{6}
Der HErr sprach zu dem Satan: Siehe da, er sei in deiner Hand; doch
schone seines Lebens! \bibverse{7} Da fuhr der Satan aus vom Angesicht
des HErrn und schlug Hiob mit bösen Schwären von der Fußsohle an bis auf
seine Scheitel. \bibverse{8} Und er nahm einen Scherben und schabte sich
und saß in der Asche. \bibverse{9} Und sein Weib sprach zu ihm: Hältst
du noch fest an deiner Frömmigkeit? Ja, segne GOtt und stirb!
\bibverse{10} Er aber sprach zu ihr: Du redest, wie die närrischen
Weiber reden. Haben wir Gutes empfangen von GOtt und sollten das Böse
nicht auch annehmen? In diesem allem versündigte sich Hiob nicht mit
seinen Lippen. \bibverse{11} Da aber die drei Freunde Hiobs höreten all
das Unglück, das über ihn kommen war, kamen sie, ein jeglicher aus
seinem Ort: Eliphas von Theman, Bildad von Suah und Zophar von Naema.
Denn sie wurden eins, daß sie kämen, ihn zu klagen und zu trösten.
\bibverse{12} Und da sie ihre Augen aufhuben von ferne, kannten sie ihn
nicht und huben auf ihre Stimme und weineten; und ein jeglicher zerriß
sein Kleid und sprengeten Erde auf ihr Haupt gen Himmel. \bibverse{13}
Und saßen mit ihm auf der Erde sieben Tage und sieben Nächte und redeten
nichts mit ihm; denn sie sahen, daß der Schmerz sehr groß war.

\hypertarget{section-2}{%
\section{3}\label{section-2}}

\bibverse{1} Danach tat Hiob seinen Mund auf und verfluchte seinen Tag.
\bibverse{2} Und Hiob sprach: \bibverse{3} Der Tag müsse verloren sein,
darinnen ich geboren bin, und die Nacht, da man sprach: Es ist ein
Männlein empfangen. \bibverse{4} Derselbe Tag müsse finster sein, und
GOtt von oben herab müsse nicht nach ihm fragen; kein Glanz müsse über
ihn scheinen. \bibverse{5} Finsternis und Dunkel müssen ihn
überwältigen, und dicke Wolken müssen über ihm bleiben, und der Dampf am
Tage mache ihn gräßlich. \bibverse{6} Die Nacht müsse ein Dunkel
einnehmen, und müsse sich nicht unter den Tagen des Jahres freuen, noch
in die Zahl der Monden kommen. \bibverse{7} Siehe, die Nacht müsse
einsam sein und kein Jauchzen drinnen sein. \bibverse{8} Es verfluchen
sie die Verflucher des Tages, und die da bereit sind, zu erwecken den
Leviathan. \bibverse{9} Ihre Sterne müssen finster sein in ihrer
Dämmerung; sie hoffe aufs Licht und komme nicht und müsse nicht sehen
die Augenbrauen der Morgenröte, \bibverse{10} daß sie nicht verschlossen
hat die Tür meines Leibes und nicht verborgen das Unglück vor meinen
Augen. \bibverse{11} Warum bin ich nicht gestorben von Mutterleib an?
Warum bin ich nicht umkommen, da ich aus dem Leibe kam? \bibverse{12}
Warum hat man mich auf den Schoß gesetzet? Warum bin ich mit Brüsten
gesäuget? \bibverse{13} So läge ich doch nun und wäre stille, schliefe
und hätte Ruhe \bibverse{14} mit den Königen und Ratsherren auf Erden,
die das Wüste bauen; \bibverse{15} oder mit den Fürsten, die Gold haben
und ihre Häuser voll Silbers sind; \bibverse{16} oder wie eine unzeitige
Geburt verborgen und nichts wäre, wie die jungen Kinder, die das Licht
nie gesehen haben. \bibverse{17} Daselbst müssen doch aufhören die
Gottlosen mit Toben; daselbst ruhen doch, die viel Mühe gehabt haben.
\bibverse{18} Da haben doch miteinander Frieden die Gefangenen und hören
nicht die Stimme des Drängers. \bibverse{19} Da sind beide klein und
groß, Knecht und der von seinem Herrn frei gelassen ist. \bibverse{20}
Warum ist das Licht gegeben dem Mühseligen und das Leben den betrübten
Herzen, \bibverse{21} (die des Todes warten und kommt nicht, und grüben
ihn wohl aus dem Verborgenen, \bibverse{22} die sich fast freuen und
sind fröhlich, daß sie das Grab bekommen,) \bibverse{23} und dem Manne,
des Weg verborgen ist, und GOtt vor ihm denselben bedecket?
\bibverse{24} Denn wenn ich essen soll, muß ich seufzen, und mein Heulen
fähret heraus wie Wasser. \bibverse{25} Denn das ich gefürchtet habe,
ist über mich kommen, und das ich sorgte, hat mich getroffen.
\bibverse{26} War ich nicht glückselig? War ich nicht fein stille? Hatte
ich nicht gute Ruhe? Und kommt solche Unruhe!

\hypertarget{section-3}{%
\section{4}\label{section-3}}

\bibverse{1} Da antwortete Eliphas von Theman und sprach: \bibverse{2}
Du hast's vielleicht nicht gerne, so man versucht, mit dir zu reden;
aber wer kann sich's enthalten? \bibverse{3} Siehe, du hast viele
unterweiset und lasse Hände gestärkt; \bibverse{4} deine Rede hat die
Gefallenen aufgerichtet, und die bebenden Kniee hast du bekräftiget.
\bibverse{5} Nun es aber an dich kommt, wirst du weich; und nun es dich
trifft, erschrickst du. \bibverse{6} Ist das deine (GOttes-)Furcht, dein
Trost, deine Hoffnung und deine Frömmigkeit? \bibverse{7} Lieber,
gedenke, wo ist ein Unschuldiger umkommen, oder wo sind die Gerechten je
vertilget? \bibverse{8} Wie ich wohl gesehen habe, die da Mühe pflügten
und Unglück säeten und ernten sie auch ein, \bibverse{9} daß sie durch
den Odem GOttes sind umkommen und vom Geist seines Zorns vertilget.
\bibverse{10} Das Brüllen der Löwen und die Stimme der großen Löwen und
die Zähne der jungen Löwen sind zerbrochen. \bibverse{11} Der Löwe ist
umkommen, daß er nicht mehr raubet, und die Jungen der Löwin sind
zerstreuet. \bibverse{12} Und zu mir ist kommen ein heimlich Wort, und
mein Ohr hat ein Wörtlein aus demselben empfangen. \bibverse{13} Da ich
Gesichte betrachtete in der Nacht, wenn der Schlaf auf die Leute fällt,
\bibverse{14} da kam mich Furcht und Zittern an, und alle meine Gebeine
erschraken. \bibverse{15} Und da der Geist vor mir überging, stunden mir
die Haare zu Berge an meinem Leibe. \bibverse{16} Da stund ein Bild vor
meinen Augen, und ich kannte seine Gestalt nicht; es war stille, und ich
hörete eine Stimme: \bibverse{17} Wie mag ein Mensch gerechter sein denn
GOtt, oder ein Mann reiner sein, denn der ihn gemacht hat? \bibverse{18}
Siehe, unter seinen Knechten ist keiner ohne Tadel, und in seinen Boten
findet er Torheit. \bibverse{19} Wie viel mehr, die in den leimenen
Häusern wohnen und welche auf Erden gegründet sind, werden von den
Würmern gefressen werden. \bibverse{20} Es währet von Morgen bis an den
Abend, so werden sie ausgehauen; und ehe sie es gewahr werden, sind sie
gar dahin; \bibverse{21} und ihre Übrigen vergehen und sterben auch
unversehens.

\hypertarget{section-4}{%
\section{5}\label{section-4}}

\bibverse{1} Nenne mir einen; was gilt's, ob du einen findest? Und siehe
dich um irgend nach einem Heiligen. \bibverse{2} Einen Tollen aber
erwürget wohl der Zorn, und den Albernen tötet der Eifer. \bibverse{3}
Ich sah einen Tollen eingewurzelt, und ich fluchte plötzlich seinem
Hause. \bibverse{4} Seine Kinder werden ferne sein vom Heil und werden
zerschlagen werden im Tor, da kein Erretter sein wird. \bibverse{5}
Seine Ernte wird essen der Hungrige, und die Gewappneten werden ihn
holen, und sein Gut werden die Durstigen aussaufen. \bibverse{6} Denn
Mühe aus der Erde nicht gehet, und Unglück aus dem Acker nicht wächset,
\bibverse{7} sondern der Mensch wird zu Unglück geboren, wie die Vögel
schweben, emporzufliegen. \bibverse{8} Doch ich will jetzt von GOtt
reden und von ihm handeln, \bibverse{9} der große Dinge tut, die nicht
zu forschen sind, und Wunder, die nicht zu zählen sind; \bibverse{10}
der den Regen aufs Land gibt und lässet Wasser kommen auf die Straßen;
\bibverse{11} der die Niedrigen erhöhet und den Betrübten emporhilft.
\bibverse{12} Er macht zunichte die Anschläge der Listigen, daß es ihre
Hand nicht ausführen kann. \bibverse{13} Er fähet die Weisen in ihrer
Listigkeit und stürzet der Verkehrten Rat, \bibverse{14} daß sie des
Tages in Finsternis laufen und tappen im Mittag wie in der Nacht;
\bibverse{15} und hilft dem Armen von dem Schwert und von ihrem Munde
und von der Hand des Mächtigen; \bibverse{16} und ist des Armen
Hoffnung, daß die Bosheit wird ihren Mund müssen zuhalten. \bibverse{17}
Siehe, selig ist der Mensch, den GOtt strafet; darum weigere dich der
Züchtigung des Allmächtigen nicht! \bibverse{18} Denn er verletzet und
verbindet; er zerschmeißet, und seine Hand heilet. \bibverse{19} Aus
sechs Trübsalen wird er dich erretten, und in der siebenten wird dich
kein Übel rühren. \bibverse{20} In der Teurung wird er dich vom Tode
erlösen und im Kriege von des Schwerts Hand. \bibverse{21} Er wird dich
verbergen vor der Geißel der Zunge, daß du dich nicht fürchtest vor dem
Verderben, wenn es kommt. \bibverse{22} Im Verderben und Hunger wirst du
lachen und dich vor den wilden Tieren im Lande nicht fürchten,
\bibverse{23} sondern dein Bund wird sein mit den Steinen auf dem Felde,
und die wilden Tiere auf dem Lande werden Frieden mit dir halten;
\bibverse{24} und wirst erfahren, daß deine Hütte Frieden hat; und wirst
deine Behausung versorgen und nicht sündigen; \bibverse{25} und wirst
erfahren, daß deines Samens wird viel werden und deine Nachkommen wie
das Gras auf Erden; \bibverse{26} und wirst im Alter zu Grabe kommen,
wie Garben eingeführet werden zu seiner Zeit. \bibverse{27} Siehe, das
haben wir erforschet, und ist also; dem gehorche und merke du dir's!

\hypertarget{section-5}{%
\section{6}\label{section-5}}

\bibverse{1} Hiob antwortete und sprach: \bibverse{2} Wenn man meinen
Jammer wöge und mein Leiden zusammen in eine Waage legte, \bibverse{3}
so würde es schwerer sein denn Sand am Meer; darum ist's umsonst, was
ich rede. \bibverse{4} Denn die Pfeile des Allmächtigen stecken in mir,
derselben Grimm säuft aus meinen Geist, und die Schrecknisse GOttes sind
auf mich gerichtet. \bibverse{5} Das Wild schreiet nicht, wenn es Gras
hat; der Ochse blöket nicht, wenn er sein Futter hat. \bibverse{6} Kann
man auch essen, das ungesalzen ist? Oder wer mag kosten das Weiße um den
Dotter? \bibverse{7} Was meiner Seele widerte anzurühren, das ist meine
Speise vor Schmerzen. \bibverse{8} O daß meine Bitte geschähe, und GOtt
gäbe mir, wes ich hoffe! \bibverse{9} Daß GOtt anfinge und zerschlüge
mich und ließe seine Hand gehen und zerscheiterte mich! \bibverse{10} So
hätte ich noch Trost und wollte bitten in meiner Krankheit, daß er nur
nicht schonete. Habe ich doch nicht verleugnet die Rede des Heiligen.
\bibverse{11} Was ist meine Kraft, daß ich möge beharren? und welch ist
mein Ende, daß meine Seele geduldig sollte sein? \bibverse{12} Ist doch
meine Kraft nicht steinern, so ist mein Fleisch nicht ehern.
\bibverse{13} Habe ich doch nirgend keine Hilfe, und mein Vermögen ist
weg. \bibverse{14} Wer Barmherzigkeit seinem Nächsten weigert, der
verlässet des Allmächtigen Furcht. \bibverse{15} Meine Brüder gehen
verächtlich vor mir über, wie ein Bach, wie die Wasserströme
vorüberfließen. \bibverse{16} Doch, welche sich vor dem Reif scheuen,
über die wird der Schnee fallen. \bibverse{17} Zur Zeit, wenn sie die
Hitze drücken wird, werden sie verschmachten, und wenn es heiß wird,
werden sie vergehen von ihrer Stätte. \bibverse{18} Ihr Weg gehet
beiseit aus; sie treten auf das Ungebahnte und werden umkommen.
\bibverse{19} Sie sehen auf die Wege Themas; auf die Pfade Reicharabias
warten sie. \bibverse{20} Aber sie werden zuschanden werden, wenn's am
sichersten ist, und sich schämen müssen, wenn sie dahin kommen.
\bibverse{21} Denn ihr seid nun zu mir kommen; und weil ihr Jammer
sehet, fürchtet ihr euch. \bibverse{22} Habe ich auch gesagt: Bringet
her und von eurem Vermögen schenket mir \bibverse{23} und errettet mich
aus der Hand des Feindes und erlöset mich von der Hand der Tyrannen?
\bibverse{24} Lehret mich, ich will schweigen; und was ich nicht weiß,
das unterweiset mich. \bibverse{25} Warum tadelt ihr die rechte Rede?
Wer ist unter euch, der sie strafen könnte? \bibverse{26} Ihr erdenket
Worte, daß ihr nur strafet, und daß ihr nur paustet Worte, die mich
verzagt machen sollen. \bibverse{27} Ihr fallet über einen armen Waisen
und grabet eurem Nächsten Gruben. \bibverse{28} Doch weil ihr habt
angehoben, sehet auf mich, ob ich vor euch mit Lügen bestehen werde.
\bibverse{29} Antwortet, was recht ist; meine Antwort wird noch recht
bleiben. \bibverse{30} Was gilt's, ob meine Zunge unrecht habe und mein
Mund Böses vorgebe?

\hypertarget{section-6}{%
\section{7}\label{section-6}}

\bibverse{1} Muß nicht der Mensch immer im Streit sein auf Erden, und
seine Tage sind wie eines Taglöhners? \bibverse{2} Wie ein Knecht sich
sehnet nach dem Schatten und ein Taglöhner, daß seine Arbeit aus sei,
\bibverse{3} also habe ich wohl ganze Monden vergeblich gearbeitet, und
elende Nächte sind mir viel worden. \bibverse{4} Wenn ich mich legte,
sprach ich: Wann werde ich aufstehen? Und danach rechnete ich, wenn es
Abend wollte werden; denn ich war ganz ein Scheusal jedermann, bis es
finster ward. \bibverse{5} Mein Fleisch ist um und um wurmig und kotig:
meine Haut ist verschrumpft und zunichte worden. \bibverse{6} Meine Tage
sind leichter dahingeflogen denn eine Weberspule und sind vergangen, daß
kein Aufhalten dagewesen ist. \bibverse{7} Gedenke, daß mein Leben ein
Wind ist, und meine Augen nicht wiederkommen, zu sehen das Gute.
\bibverse{8} Und kein lebendig Auge wird mich mehr sehen. Deine Augen
sehen mich an; darüber vergehe ich. \bibverse{9} Eine Wolke vergehet und
fähret dahin; also, wer in die Hölle hinunterfährt, kommt nicht wieder
herauf \bibverse{10} und kommt nicht wieder in sein Haus, und sein Ort
kennet ihn nicht mehr. \bibverse{11} Darum will auch ich meinem Munde
nicht wehren; ich will reden von der Angst meines Herzens und will
heraussagen von der Betrübnis meiner Seele. \bibverse{12} Bin ich denn
ein Meer oder ein Walfisch, daß du mich so verwahrest? \bibverse{13}
Wenn ich gedachte, mein Bett soll mich trösten, mein Lager soll mir's
leichtern; \bibverse{14} wenn ich mit mir selbst rede, so erschreckst du
mich mit Träumen und machst mir Grauen, \bibverse{15} daß meine Seele
wünschet erhangen zu sein, und meine Gebeine den Tod. \bibverse{16} Ich
begehre nicht mehr zu leben. Höre auf von mir, denn meine Tage sind
vergeblich gewesen. \bibverse{17} Was ist ein Mensch, daß du ihn groß
achtest und bekümmerst dich mit ihm? \bibverse{18} Du suchest ihn
täglich heim und versuchest ihn alle Stunde. \bibverse{19} Warum tust du
dich nicht von mir und lässest nicht ab, bis ich meinen Speichel
schlinge? \bibverse{20} Habe ich gesündiget, was soll ich dir tun, o du
Menschenhüter? Warum machst du mich, daß ich auf dich stoße und bin mir
selbst eine Last? \bibverse{21} Und warum vergibst du mir meine Missetat
nicht und nimmst nicht weg meine Sünde? Denn nun werde ich mich in die
Erde legen; und wenn man mich morgen suchet, werde ich nicht da sein.

\hypertarget{section-7}{%
\section{8}\label{section-7}}

\bibverse{1} Da antwortete Bildad von Suah und sprach: \bibverse{2} Wie
lange willst du solches reden und die Rede deines Mundes so einen
stolzen Mut haben? \bibverse{3} Meinest du, daß GOtt unrecht richte,
oder der Allmächtige das Recht verkehre? \bibverse{4} Haben deine Söhne
vor ihm gesündiget, so hat er sie verstoßen um ihrer Missetat willen.
\bibverse{5} So du aber dich beizeiten zu GOtt tust und dem Allmächtigen
flehest, \bibverse{6} und du so rein und fromm bist, so wird er
aufwachen zu dir und wird wieder aufrichten die Wohnung um deiner
Gerechtigkeit willen; \bibverse{7} und was du zuerst wenig gehabt hast,
wird hernach fast zunehmen. \bibverse{8} Denn frage die vorigen
Geschlechter und nimm dir vor, zu forschen ihre Väter. \bibverse{9}
(Denn wir sind von gestern her und wissen nichts; unser Leben ist ein
Schatten auf Erden.) \bibverse{10} Sie werden dich's lehren und dir
sagen und ihre Rede aus ihrem Herzen hervorbringen. \bibverse{11} Kann
auch das Schilf aufwachsen, wo es nicht feucht stehet, oder Gras wachsen
ohne Wasser? \bibverse{12} Sonst wenn's noch in der Blüte ist, ehe es
abgehauen wird, verdorret es, ehe denn man Heu machet. \bibverse{13} So
geht es allen denen, die GOttes vergessen, und die Hoffnung der Heuchler
wird verloren sein. \bibverse{14} Denn seine Zuversicht vergehet, und
seine Hoffnung ist eine Spinnwebe. \bibverse{15} Er verlässet sich auf
sein Haus und wird doch nicht bestehen; er wird sich dran halten, aber
doch nicht stehen bleiben. \bibverse{16} Es hat wohl Früchte, ehe denn
die Sonne kommt; und Reiser wachsen hervor in seinem Garten.
\bibverse{17} Seine Saat stehet dicke bei den Quellen und sein Haus auf
Steinen. \bibverse{18} Wenn er ihn aber verschlinget von seinem Ort,
wird er sich gegen ihn stellen, als kennete er ihn nicht. \bibverse{19}
Siehe, das ist die Freude seines Wesens; und werden andere aus dem
Staube wachsen. \bibverse{20} Darum siehe, daß GOtt nicht verwirft die
Frommen und erhält nicht die Hand der Boshaftigen, \bibverse{21} bis daß
dein Mund voll Lachens werde und deine Lippen voll Jauchzens.
\bibverse{22} Die dich aber hassen; werden zuschanden werden, und der
Gottlosen Hütte wird nicht bestehen.

\hypertarget{section-8}{%
\section{9}\label{section-8}}

\bibverse{1} Hiob antwortete und sprach: \bibverse{2} Ja, ich weiß fast
wohl, daß also ist, daß ein Mensch nicht rechtfertig bestehen mag gegen
GOtt. \bibverse{3} Hat er Lust, mit ihm zu hadern, so kann er ihm auf
tausend nicht eins antworten. \bibverse{4} Er ist weise und mächtig wem
ist's je gelungen, der sich wider ihn gelegt hat? \bibverse{5} Er
versetzt Berge, ehe sie es inne werden, die er in seinem Zorn umkehret.
\bibverse{6} Er weget ein Land aus seinem Ort, daß seine Pfeiler
zittern. \bibverse{7} Er spricht zur Sonne, so gehet sie nicht auf, und
versiegelt die Sterne. \bibverse{8} Er breitet den Himmel aus allein und
gehet auf den Wogen des Meers. \bibverse{9} Er machet den Wagen am
Himmel und Orion und die Glucke und die Sterne gegen Mittag.
\bibverse{10} Er tut große Dinge, die nicht zu forschen sind, und
Wunder, deren keine Zahl ist. \bibverse{11} Siehe, er gehet vor mir
über, ehe ich's gewahr werde, und verwandelt sich, ehe ich's merke.
\bibverse{12} Siehe, wenn er geschwind hinfähret, wer will ihn
wiederholen? Wer will zu ihm sagen: Was machst du? \bibverse{13} Er ist
GOtt, seinen Zorn kann niemand stillen; unter ihm müssen sich beugen die
stolzen Herren. \bibverse{14} Wie sollt ich denn ihm antworten und Worte
finden gegen ihn? \bibverse{15} Wenn ich auch gleich recht habe, kann
ich ihm dennoch nicht antworten sondern ich müßte um mein Recht flehen.
\bibverse{16} Wenn ich ihn schon anrufe, und er mich erhöret, so glaube
ich doch nicht, daß er meine Stimme höre. \bibverse{17} Denn er fähret
über mich mit Ungestüm und macht mir der Wunden viel ohne Ursache.
\bibverse{18} Er läßt meinen Geist sich nicht erquicken, sondern macht
mich voll Betrübnis. \bibverse{19} Will man Macht; so ist er zu mächtig;
will man Recht, wer will mein Zeuge sein? \bibverse{20} Sage ich, daß
ich gerecht bin; so verdammet er mich doch; bin ich fromm, so macht er
mich doch zu Unrecht. \bibverse{21} Bin ich denn fromm, so darf sich's
meine Seele nicht annehmen. Ich begehre keines Lebens mehr.
\bibverse{22} Das ist das Eine, das ich gesagt habe: Er bringet um beide
den Frommen und Gottlosen. \bibverse{23} Wenn er anhebt zu geißeln, so
dringet er fort bald zum Tode und spottet der Anfechtung der
Unschuldigen. \bibverse{24} Das Land aber wird gegeben unter die Hand
des Gottlosen, daß er ihre Richter unterdrücke. Ist's nicht also? Wie
sollte es anders sein? \bibverse{25} Meine Tage sind schneller gewesen
denn ein Läufer; sie sind geflohen und haben nichts Gutes erlebt.
\bibverse{26} Sie sind vergangen wie die starken Schiffe, wie ein Adler
fleugt zur Speise. \bibverse{27} Wenn ich gedenke, ich will meiner Klage
vergessen und meine Gebärde lassen fahren und mich erquicken,
\bibverse{28} so fürchte ich alle meine Schmerzen, weil ich weiß, daß du
mich nicht unschuldig sein lässest. \bibverse{29} Bin ich denn gottlos,
warum leide ich denn solche vergebliche Plage? \bibverse{30} Wenn ich
mich gleich mit Schneewasser wünsche und reinigte meine Hände mit dem
Brunnen, \bibverse{31} so wirst du mich doch tunken in Kot, und werden
mir meine Kleider scheußlich anstehen. \bibverse{32} Denn er ist nicht
meinesgleichen, dem ich antworten möchte, daß wir vor Gericht
miteinander kämen. \bibverse{33} Es ist unter uns kein Schiedsmann, noch
der seine Hand zwischen uns beide lege. \bibverse{34} Er nehme von mir
seine Rute und lasse sein Schrecken von mir, \bibverse{35} daß ich möge
reden und mich nicht vor ihm fürchten dürfe; sonst kann ich nichts tun,
das für mich sei.

\hypertarget{section-9}{%
\section{10}\label{section-9}}

\bibverse{1} Meine Seele verdreußt mein Leben; ich will meine Klage bei
mir gehen lassen und reden von Betrübnis meiner Seele \bibverse{2} und
zu GOtt sagen: Verdamme mich nicht; laß mich wissen, warum du mit mir
haderst! \bibverse{3} Gefällt dir's, daß du Gewalt tust und mich
verwirfst, den deine Hände gemacht haben, und machest der Gottlosen
Vornehmen zu Ehren? \bibverse{4} Hast du denn auch fleischliche Augen,
oder siehest du, wie ein Mensch siehet? \bibverse{5} Oder ist deine Zeit
wie eines Menschen Zeit, oder deine Jahre wie eines Mannes Jahre,
\bibverse{6} daß du nach meiner Missetat fragest und suchest meine
Sünde? \bibverse{7} So du doch weißt, wie ich nicht gottlos sei; so doch
niemand ist, der aus deiner Hand erretten möge. \bibverse{8} Deine Hände
haben mich gearbeitet und gemacht alles, was ich um und um bin; und
versenkest mich sogar! \bibverse{9} Gedenke doch, daß du mich aus Leimen
gemacht hast, und wirst mich wieder zu Erden machen. \bibverse{10} Hast
du mich nicht wie Milch gemolken und wie Käse lassen gerinnen?
\bibverse{11} Du hast mir Haut und Fleisch angezogen, mit Beinen und
Adern hast du mich zusammengefüget. \bibverse{12} Leben und Wohltat hast
du an mir getan, und dein Aufsehen bewahret meinen Odem. \bibverse{13}
Und wiewohl du solches in deinem Herzen verbirgest, so weiß ich doch,
daß du des gedenkest. \bibverse{14} Wenn ich sündige, so merkest du es
bald und lässest meine Missetat nicht ungestraft. \bibverse{15} Bin ich
gottlos, so ist mir aber wehe; bin ich gerecht, so darf ich doch mein
Haupt nicht aufheben, als der ich voll Schmach bin und sehe mein Elend.
\bibverse{16} Und wie ein aufgereckter Löwe jagest du mich und handelst
wiederum greulich mit mir. \bibverse{17} Du erneuest deine Zeugen wider
mich und machst deines Zorns viel auf mich; es zerplagt mich eins über
das andere mit Haufen. \bibverse{18} Warum hast du mich aus Mutterleibe
kommen lassen? Ach, daß ich wäre umkommen, und mich nie kein Auge
gesehen hätte! \bibverse{19} So wäre ich, als die nie gewesen sind, von
Mutterleibe zum Grabe gebracht. \bibverse{20} Will denn nicht ein Ende
haben mein kurzes Leben, und von mir lassen, daß ich ein wenig erquickt
würde, \bibverse{21} ehe denn ich hingehe und komme nicht wieder,
nämlich ins Land der Finsternis und des Dunkels, \bibverse{22} ins Land,
da es stockdick finster ist, und da keine Ordnung ist, da es scheinet
wie das Dunkel?

\hypertarget{section-10}{%
\section{11}\label{section-10}}

\bibverse{1} Da antwortete Zophar von Naema und sprach: \bibverse{2}
Wenn einer lange geredet, muß er nicht auch hören? Muß denn ein Wäscher
immer recht haben? \bibverse{3} Müssen die Leute deinem großen Schwätzen
Schweigen, daß du spottest, und niemand dich beschäme? \bibverse{4} Du
sprichst: Meine Rede ist rein, und lauter bin ich vor deinen Augen.
\bibverse{5} Ach, daß GOtt mit dir redete und täte seine Lippen auf
\bibverse{6} und zeigete die heimliche Weisheit! Denn er hätte wohl noch
mehr an dir zu tun, auf daß du wissest, daß er deiner Sünden nicht aller
gedenkt. \bibverse{7} Meinest du, daß du so viel wissest, als GOtt weiß,
und wollest alles so vollkommen treffen als der Allmächtige?
\bibverse{8} Er ist höher denn der Himmel; was willst du tun? tiefer
denn die Hölle; was kannst du wissen? \bibverse{9} Länger denn die Erde
und breiter denn das Meer. \bibverse{10} So er sie umkehrete oder
verbürge oder in einen Haufen würfe, wer will's ihm wehren?
\bibverse{11} Denn er kennet die losen Leute, er siehet die Untugend,
und sollte es nicht merken? \bibverse{12} Ein unnützer Mann blähet sich;
und ein geborener Mensch will sein wie ein junges Wild. \bibverse{13}
Wenn du dein Herz hättest gerichtet und deine Hände zu ihm ausgebreitet;
\bibverse{14} wenn du die Untugend, die in deiner Hand ist, hättest
ferne von dir getan, daß in deiner Hütte kein Unrecht bliebe,
\bibverse{15} so möchtest du dein Antlitz aufheben ohne Tadel und
würdest fest sein und dich nicht fürchten. \bibverse{16} Dann würdest du
der Mühe vergessen und so wenig gedenken als des Wassers, das
vorübergehet. \bibverse{17} Und die Zeit deines Lebens würde aufgehen
wie der Mittag, und das Finstere würde ein lichter Morgen werden.
\bibverse{18} Und dürftest dich des trösten, daß Hoffnung da sei; du
würdest mit Ruhe ins Grab kommen. \bibverse{19} Und würdest dich legen,
und niemand würde dich aufschrecken; und viele würden vor dir flehen.
\bibverse{20} Aber die Augen der Gottlosen werden verschmachten, und
werden nicht entrinnen mögen; denn ihre Hoffnung wird ihrer Seele
fehlen.

\hypertarget{section-11}{%
\section{12}\label{section-11}}

\bibverse{1} Da antwortete Hiob und sprach: \bibverse{2} Ja, ihr seid
die Leute; mit euch wird die Weisheit sterben! \bibverse{3} Ich habe so
wohl ein Herz als ihr und bin nicht geringer denn ihr; und wer ist, der
solches nicht wisse? \bibverse{4} Wer von seinem Nächsten verlachet
wird, der wird GOtt anrufen, der wird ihn erhören. Der Gerechte und
Fromme muß verlachet sein \bibverse{5} und ist ein verachtet Lichtlein
vor den Gedanken der Stolzen, stehet aber, daß sie sich dran ärgern.
\bibverse{6} Der Verstörer Hütten haben die Fülle und toben wider GOtt
türstiglich, wiewohl es ihnen GOtt in ihre Hände gegeben hat.
\bibverse{7} Frage doch das Vieh, das wird dich's lehren, und die Vögel
unter dem Himmel, die werden dir's sagen. \bibverse{8} Oder rede mit der
Erde, die wird dich's lehren, und die Fische im Meer werden dir's
erzählen. \bibverse{9} Wer weiß solches alles nicht, daß des HErrn Hand
das gemacht hat, \bibverse{10} daß in seiner Hand ist die Seele alles
des, das da lebet, und der Geist alles Fleisches eines jeglichen?
\bibverse{11} Prüfet nicht das Ohr die Rede; und der Mund schmecket die
Speise? \bibverse{12} Ja, bei den Großvätern ist die Weisheit und der
Verstand bei den Alten. \bibverse{13} Bei ihm ist Weisheit und Gewalt,
Rat und Verstand. \bibverse{14} Siehe, wenn er zerbricht, so hilft kein
Bauen; wenn er jemand verschleußt, kann niemand aufmachen. \bibverse{15}
Siehe, wenn er das Wasser verschleußt, so wird's alles dürre; und wenn
er's ausläßt, so kehret es das Land um. \bibverse{16} Er ist stark und
führet es aus. Sein ist, der da irret, und der da verführet.
\bibverse{17} Er führet die Klugen wie einen Raub und machet die Richter
toll. \bibverse{18} Er löset auf der Könige Zwang und gürtet mit einem
Gürtel ihre Lenden. \bibverse{19} Er führet die Priester wie einen Raub
und lässet es fehlen den Festen. \bibverse{20} Er wendet weg die Lippen
der Wahrhaftigen und nimmt weg die Sitten der Alten. \bibverse{21} Er
schüttet Verachtung auf die Fürsten und macht den Bund der Gewaltigen
los. \bibverse{22} Er öffnet die finstern Gründe und bringet heraus das
Dunkel an das Licht. \bibverse{23} Er macht etliche zum großen Volk und
bringet sie wieder um. Er breitet ein Volk aus und treibet es wieder
weg. \bibverse{24} Er nimmt weg den Mut der Obersten des Volks im Lande
und macht sie irre auf einem Umwege, da kein Weg ist, \bibverse{25} daß
sie in der Finsternis tappen ohne Licht; und macht sie irre wie die
Trunkenen.

\hypertarget{section-12}{%
\section{13}\label{section-12}}

\bibverse{1} Siehe, das hat alles mein Auge gesehen und mein Ohr
gehöret, und habe es verstanden. \bibverse{2} Was ihr wisset, das weiß
ich auch, und bin nicht geringer denn ihr. \bibverse{3} Doch wollte ich
gerne wider den Allmächtigen reden und wollte gerne mit GOtt rechten.
\bibverse{4} Denn ihr deutet es fälschlich und seid alle unnütze Ärzte.
\bibverse{5} Wollte GOtt, ihr schwieget; so würdet ihr weise.
\bibverse{6} Höret doch meine Strafe und merket auf die Sache, davon ich
rede. \bibverse{7} Wollt ihr GOtt verteidigen mit Unrecht und für ihn
List brauchen? \bibverse{8} Wollt ihr seine Person ansehen? Wollt ihr
GOtt vertreten? \bibverse{9} Wird's euch auch wohlgehen, wenn er euch
richten wird? Meinet ihr, daß ihr ihn täuschen werdet, wie man einen
Menschen täuschet? \bibverse{10} Er wird euch strafen, wo ihr Person
ansehet heimlich. \bibverse{11} Wird er euch nicht erschrecken, wenn er
sich wird hervortun, und seine Furcht wird über euch fallen?
\bibverse{12} Euer Gedächtnis wird verglichen werden der Asche, und euer
Rücken wird wie ein Leimenhaufe sein. \bibverse{13} Schweiget mir, daß
ich rede; es soll mir nichts fehlen. \bibverse{14} Was soll ich mein
Fleisch mit meinen Zähnen beißen und meine Seele in meine Hände legen?
\bibverse{15} Siehe, er wird mich doch erwürgen, und ich kann's nicht
erwarten; doch will ich meine Wege vor ihm strafen. \bibverse{16} Er
wird ja mein Heil sein; denn es kommt kein Heuchler vor ihn.
\bibverse{17} Höret meine Rede und meine Auslegung vor euren Ohren!
\bibverse{18} Siehe, ich habe das Urteil schon gefället; ich weiß, daß
ich werde gerecht sein. \bibverse{19} Wer ist, der mit mir rechten will?
Aber nun muß ich schweigen und verderben. \bibverse{20} Zweierlei tu mir
nur nicht, so will ich mich vor dir nicht verbergen: \bibverse{21} Laß
deine Hand ferne von mir sein, und dein Schrecken erschrecke mich nicht.
\bibverse{22} Rufe mir, ich will dir antworten; oder ich will reden,
antworte du mir. \bibverse{23} Wie viel ist meiner Missetat und Sünden?
Laß mich wissen meine Übertretung und Sünde! \bibverse{24} Warum
verbirgest du dein Antlitz und hältst mich für deinen Feind?
\bibverse{25} Willst du wider ein fliegend Blatt so ernst sein und einen
dürren Halm verfolgen? \bibverse{26} Denn du schreibest mir an Betrübnis
und willst mich umbringen um der Sünden willen meiner Jugend.
\bibverse{27} Du hast meinen Fuß in Stock gelegt und hast acht auf alle
meine Pfade und siehest auf die Fußtapfen meiner Füße, \bibverse{28} der
ich doch wie ein faul Aas vergehe und wie ein Kleid, das die Motten
fressen.

\hypertarget{section-13}{%
\section{14}\label{section-13}}

\bibverse{1} Der Mensch, vom Weibe geboren, lebt kurze Zeit und ist voll
Unruhe, \bibverse{2} gehet auf wie eine Blume und fällt ab, fleucht wie
ein Schatten und bleibet nicht. \bibverse{3} Und du tust deine Augen
über solchem auf, daß du mich vor dir in das Gericht ziehest.
\bibverse{4} Wer will einen Reinen finden bei denen, da keiner rein ist?
\bibverse{5} Er hat seine bestimmte Zeit, die Zahl seiner Monden stehet
bei dir; du hast ein Ziel gesetzt, das wird er nicht übergehen.
\bibverse{6} Tue dich von ihm, daß er Ruhe habe, bis daß seine Zeit
komme, deren er wie ein Taglöhner wartet. \bibverse{7} Ein Baum hat
Hoffnung, wenn er schon abgehauen ist, daß er sich wieder verändere, und
seine Schößlinge hören nicht auf. \bibverse{8} Ob seine Wurzel in der
Erde veraltet und sein Stamm in dem Staube erstirbt, \bibverse{9} grünet
er doch wieder vom Geruch des Wassers und wächst daher, als wäre er
gepflanzet. \bibverse{10} Wo ist aber ein Mensch, wenn er tot und
umkommen und dahin ist? \bibverse{11} Wie ein Wasser ausläuft aus dem
See und wie ein Strom versieget und vertrocknet, \bibverse{12} so ist
ein Mensch, wenn er sich legt, und wird nicht aufstehen und wird nicht
aufwachen, solange der Himmel bleibt, noch von seinem Schlaf erweckt
werden. \bibverse{13} Ach, daß du mich in der Hölle verdecktest und
verbärgest, bis dein Zorn sich lege, und setztest mir ein Ziel, daß du
an mich denkest! \bibverse{14} Meinest du, ein toter Mensch werde wieder
leben? Ich harre täglich; dieweil ich streite, bis daß meine Veränderung
komme, \bibverse{15} daß du wollest mir rufen, und ich dir antworten,
und wollest das Werk deiner Hände nicht ausschlagen. \bibverse{16} Denn
du hast schon meine Gänge gezählet; aber du wollest ja nicht achthaben
auf meine Sünde. \bibverse{17} Du hast meine Übertretung in einem
Bündlein versiegelt und meine Missetat zusammengefasset. \bibverse{18}
Zerfällt doch ein Berg und vergehet, und ein Fels wird von seinem Ort
versetzt. \bibverse{19} Wasser wäschet Steine weg, und die Tropfen
flößen die Erde weg; aber des Menschen Hoffnung ist verloren.
\bibverse{20} Denn du stößest ihn gar um, daß er dahinfähret, veränderst
sein Wesen und lässest ihn fahren. \bibverse{21} Sind seine Kinder in
Ehren, das weiß er nicht; oder ob sie geringe sind, des wird er nicht
gewahr. \bibverse{22} Weil er das Fleisch an sich trägt, muß er
Schmerzen haben, und weil seine Seele noch bei ihm ist, muß er Leid
tragen.

\hypertarget{section-14}{%
\section{15}\label{section-14}}

\bibverse{1} Da antwortete Eliphas von Theman und sprach: \bibverse{2}
Soll ein weiser Mann so aufgeblasene Worte reden und seinen Bauch so
blähen mit losen Reden? \bibverse{3} Du strafest mit Worten, die nicht
taugen, und dein Reden ist kein nütze. \bibverse{4} Du hast die Furcht
fahren lassen und redest zu verächtlich vor GOtt. \bibverse{5} Denn
deine Missetat lehret deinen Mund also, und hast erwählet eine
schalkhafte Zunge. \bibverse{6} Dein Mund wird dich verdammen, und nicht
ich; deine Lippen sollen dir antworten. \bibverse{7} Bist du der erste
Mensch geboren? Bist du vor allen Hügeln empfangen? \bibverse{8} Hast du
GOttes heimlichen Rat gehöret? und ist die Weisheit selbst geringer denn
du? \bibverse{9} Was weißt du, das wir nicht wissen? Was verstehest du,
das nicht bei uns sei? \bibverse{10} Es sind Graue und Alte unter uns,
die länger gelebt haben denn deine Väter. \bibverse{11} Sollten GOttes
Tröstungen so geringe vor dir gelten? Aber du hast irgend noch ein
heimlich Stück bei dir. \bibverse{12} Was nimmt dein Herz vor? Was
siehest du so stolz? \bibverse{13} Was setzt sich dein Mut wider GOtt,
daß du solche Rede aus deinem Munde lässest? \bibverse{14} Was ist ein
Mensch, daß er sollte rein sein, und daß der sollte gerecht sein, der
vom Weibe geboren ist? \bibverse{15} Siehe, unter seinen Heiligen ist
keiner ohne Tadel, und die Himmel sind nicht rein vor ihm. \bibverse{16}
Wie viel mehr ein Mensch, der ein Greuel und schnöde ist, der Unrecht
säuft wie Wasser. \bibverse{17} Ich will dir's zeigen, höre mir zu; und
will dir erzählen was ich gesehen habe, \bibverse{18} was die Weisen
gesagt haben, und ihren Vätern nicht verhohlen gewesen ist,
\bibverse{19} welchen allein das Land gegeben ist, daß kein Fremder
durch sie gehen muß. \bibverse{20} Der Gottlose bebet sein Leben lang;
und dem Tyrannen ist die Zahl seiner Jahre verborgen. \bibverse{21} Was
er höret, das schrecket ihn; und wenn's gleich Friede ist, fürchtet er
sich, der Verderber komme; \bibverse{22} glaubt nicht, daß er möge dem
Unglück entrinnen, und versiehet sich immer des Schwerts. \bibverse{23}
Er zeucht hin und hernach Brot und dünket ihn immer, die Zeit seines
Unglücks sei vorhanden. \bibverse{24} Angst und Not schrecken ihn und
schlagen ihn nieder als ein König mit einem Heer. \bibverse{25} Denn er
hat seine Hand wider GOtt gestrecket und wider den Allmächtigen sich
gesträubet. \bibverse{26} Er läuft mit dem Kopf an ihn und ficht
halsstarriglich wider ihn. \bibverse{27} Er brüstet sich wie ein fetter
Wanst und macht sich fett und dick. \bibverse{28} Er wird aber wohnen in
verstörten Städten, da keine Häupter sind, sondern auf einem Haufen
liegen. \bibverse{29} Er wird nicht reich bleiben, und sein Gut wird
nicht bestehen, und sein Glück wird sich nicht ausbreiten im Lande.
\bibverse{30} Unfall wird nicht von ihm lassen. Die Flamme wird seine
Zweige verdorren und durch den Odem ihres Mundes ihn wegfressen.
\bibverse{31} Er wird nicht bestehen, denn er ist in seinem eiteln
Dünkel betrogen, und eitel wird sein Lohn werden. \bibverse{32} Er wird
ein Ende nehmen, wenn's ihm uneben ist, und sein Zweig wird nicht
grünen. \bibverse{33} Er wird abgerissen werden wie eine unzeitige
Traube vom Weinstock, und wie ein Ölbaum seine Blüte abwirft.
\bibverse{34} Denn der Heuchler Versammlung wird einsam bleiben, und das
Feuer wird die Hütten fressen, die Geschenke nehmen. \bibverse{35} Er
gehet schwanger mit Unglück und gebiert Mühe, und ihr Bauch bringet
Fehl.

\hypertarget{section-15}{%
\section{16}\label{section-15}}

\bibverse{1} Hiob antwortete und sprach: \bibverse{2} Ich habe solches
oft gehöret. Ihr seid allzumal leidige Tröster. \bibverse{3} Wollen die
losen Worte kein Ende haben? Oder was macht dich so frech, also zu
reden? \bibverse{4} Ich könnte auch wohl reden wie ihr. Wollte GOtt,
eure Seele wäre an meiner Seele Statt! Ich wollte auch mit Worten an
euch setzen und mein Haupt also über euch schütteln. \bibverse{5} Ich
wollte euch stärken mit dem Munde und mit meinen Lippen trösten.
\bibverse{6} Aber wenn ich schon rede, so schonet mein der Schmerz
nicht; lasse ich's anstehen, so gehet er nicht von mir. \bibverse{7} Nun
aber macht er mich müde und verstöret alles, was ich bin. \bibverse{8}
Er hat mich runzlicht gemacht und zeuget wider mich; und mein
Widersprecher lehnet sich wider mich auf und antwortet wider mich.
\bibverse{9} Sein Grimm reißet, und der mir gram ist, beißet die Zähne
über mich zusammen; mein Widersacher funkelt mit seinen Augen auf mich.
\bibverse{10} Sie haben ihren Mund aufgesperret wider mich und haben
mich schmählich auf meine Backen geschlagen; sie haben ihren Mut
miteinander an mir gekühlet. \bibverse{11} GOtt hat mich übergeben dem
Ungerechten und hat mich in der Gottlosen Hände lassen kommen.
\bibverse{12} Ich war reich, aber er hat mich zunichte gemacht; er hat
mich beim Hals genommen und zerstoßen und hat mich ihm zum Ziel
aufgerichtet. \bibverse{13} Er hat mich umgeben mit seinen Schützen; er
hat meine Nieren gespalten und nicht verschonet; er hat meine Galle auf
die Erde geschüttet; \bibverse{14} er hat mir eine Wunde über die andere
gemacht; er ist an mich gelaufen wie ein Gewaltiger. \bibverse{15} Ich
habe einen Sack um meine Haut genähet und habe mein Horn in den Staub
gelegt. \bibverse{16} Mein Antlitz ist geschwollen von Weinen, und meine
Augenlider sind verdunkelt, \bibverse{17} wiewohl kein Frevel in meiner
Hand ist, und mein Gebet ist rein. \bibverse{18} Ach, Erde, verdecke
mein Blut nicht! und mein Geschrei müsse nicht Raum finden!
\bibverse{19} Auch siehe da, mein Zeuge ist im Himmel; und der mich
kennet, ist in der Höhe. \bibverse{20} Meine Freunde sind meine Spötter;
aber mein Auge tränet zu GOtt. \bibverse{21} Wenn ein Mann könnte mit
GOtt rechten wie ein Menschenkind mit seinem Freunde! \bibverse{22} Aber
die bestimmten Jahre sind kommen, und ich gehe hin des Weges, den ich
nicht wiederkommen werde.

\hypertarget{section-16}{%
\section{17}\label{section-16}}

\bibverse{1} Mein Odem ist schwach, und meine Tage sind abgekürzt, das
Grab ist da. \bibverse{2} Niemand ist von mir getäuschet, noch muß mein
Auge darum bleiben in Betrübnis. \bibverse{3} Ob du gleich einen Bürgen
für mich wolltest, wer will für mich geloben? \bibverse{4} Du hast ihrem
Herzen den Verstand verborgen, darum wirst du sie nicht erhöhen.
\bibverse{5} Er rühmet wohl seinen Freunden die Ausbeute; aber seiner
Kinder Augen werden verschmachten. \bibverse{6} Er hat mich zum
Sprichwort unter den Leuten gesetzt, und muß ein Wunder unter ihnen
sein. \bibverse{7} Meine Gestalt ist dunkel worden vor Trauern, und alle
meine Glieder sind wie ein Schatten. \bibverse{8} Darüber werden die
Gerechten übel sehen, und die Unschuldigen werden sich setzen wider die
Heuchler. \bibverse{9} Der Gerechte wird seinen Weg behalten, und der
von reinen Händen wird stark bleiben. \bibverse{10} Wohlan, so kehret
euch alle her und kommt; ich werde doch keinen Weisen unter euch finden.
\bibverse{11} Meine Tage sind vergangen, meine Anschläge sind
zertrennet, die mein Herz besessen haben, \bibverse{12} und haben aus
der Nacht Tag gemacht und aus dem Tage Nacht. \bibverse{13} Wenn ich
gleich lange harre, so ist doch die Hölle mein Haus, und in Finsternis
ist mein Bett gemacht. \bibverse{14} Die Verwesung heiße ich meinen
Vater und die Würmer meine Mutter und meine Schwester. \bibverse{15} Was
soll ich harren? und wer achtet mein Hoffen? \bibverse{16} Hinunter in
die Hölle wird es fahren und wird mit mir im Staube liegen.

\hypertarget{section-17}{%
\section{18}\label{section-17}}

\bibverse{1} Da antwortete Bildad von Suah und sprach: \bibverse{2} Wann
wollt ihr der Rede ein Ende machen? Merket doch, danach wollen wir
reden. \bibverse{3} Warum werden wir geachtet wie Vieh und sind so
unrein vor euren Augen? \bibverse{4} Willst du vor Bosheit bersten?
Meinest du, daß um deinetwillen die Erde verlassen werde, und der Fels
von seinem Ort versetzt werde? \bibverse{5} Auch wird das Licht der
Gottlosen verlöschen, und der Funke seines Feuers wird nicht leuchten.
\bibverse{6} Das Licht wird finster werden in seiner Hütte und seine
Leuchte über ihm verlöschen. \bibverse{7} Die Zugänge seiner Habe werden
schmal werden, und sein Anschlag wird ihn fällen. \bibverse{8} Denn er
ist mit seinen Füßen in Strick gebracht und wandelt im Netze.
\bibverse{9} Der Strick wird seine Fersen halten, und die Türstigen
werden ihn erhaschen. \bibverse{10} Sein Strick ist gelegt in die Erde
und seine Falle auf seinen Gang. \bibverse{11} Um und um wird ihn
schrecken plötzliche Furcht, daß er nicht weiß, wo er hinaus soll.
\bibverse{12} Hunger wird seine Habe sein, und Unglück wird ihm bereitet
sein und anhangen. \bibverse{13} Die Stärke seiner Haut wird verzehret
werden, und seine Stärke wird verzehren der Fürst des Todes.
\bibverse{14} Seine Hoffnung wird aus seiner Hütte gerottet werden, und
sie werden ihn treiben zum Könige des Schreckens. \bibverse{15} In
seiner Hütte wird nichts bleiben; über seine Hütte wird Schwefel
gestreuet werden. \bibverse{16} Von unten werden verdorren seine Wurzeln
und von oben abgeschnitten seine Ernte. \bibverse{17} Sein Gedächtnis
wird vergehen im Lande, und wird keinen Namen haben auf der Gasse.
\bibverse{18} Er wird vom Licht in die Finsternis vertrieben werden und
vom Erdboden verstoßen werden. \bibverse{19} Er wird keine Kinder haben
und keine Neffen unter seinem Volk; es wird ihm keiner überbleiben in
seinen Gütern. \bibverse{20} Die nach ihm kommen, werden sich über
seinen Tag entsetzen; und die vor ihm sind, wird eine Furcht ankommen.
\bibverse{21} Das ist die Wohnung des Ungerechten, und dies ist die
Stätte des, der GOtt nicht achtet.

\hypertarget{section-18}{%
\section{19}\label{section-18}}

\bibverse{1} Hiob antwortete und sprach: \bibverse{2} Was plaget ihr
doch meine Seele und peiniget mich mit Worten? \bibverse{3} Ihr habt
mich nun zehnmal gehöhnet und schämet euch nicht, daß ihr mich also
umtreibet. \bibverse{4} Irre ich, so irre ich mir. \bibverse{5} Aber ihr
erhebet euch wahrlich wider mich und scheltet mich zu meiner Schmach.
\bibverse{6} Merket doch einst, daß mir GOtt unrecht tut und hat mich
mit seinem Jagestrick umgeben. \bibverse{7} Siehe, ob ich schon schreie
über Frevel, so werde ich doch nicht erhöret; ich rufe, und ist kein
Recht da. \bibverse{8} Er hat meinen Weg verzäunet, daß ich nicht kann
hinübergehen, und hat Finsternis auf meinen Steig gestellet.
\bibverse{9} Er hat meine Ehre mir ausgezogen und die Krone von meinem
Haupt genommen. \bibverse{10} Er hat mich zerbrochen um und um und läßt
mich gehen, und hat ausgerissen meine Hoffnung wie einen Baum.
\bibverse{11} Sein Zorn ist über mich ergrimmet, und er achtet mich für
seinen Feind. \bibverse{12} Seine Kriegsleute sind miteinander kommen
und haben ihren Weg über mich gepflastert und haben sich um meine Hütte
her gelagert. \bibverse{13} Er hat meine Brüder ferne von mir getan, und
meine Verwandten sind mir fremd worden. \bibverse{14} Meine Nächsten
haben sich entzogen, und meine Freunde haben mein vergessen.
\bibverse{15} Meine Hausgenossen und meine Mägde achten mich für fremd,
ich bin unbekannt worden vor ihren Augen. \bibverse{16} Ich rief meinem
Knecht, und er antwortete mir nicht; ich mußte ihm flehen mit eigenem
Munde. \bibverse{17} Mein Weib stellet sich fremd, wenn ich ihr rufe;
ich muß flehen den Kindern meines Leibes. \bibverse{18} Auch die jungen
Kinder geben nichts auf mich; wenn ich mich wider sie setze, so geben
sie mir böse Worte. \bibverse{19} Alle meine Getreuen haben Greuel an
mir; und die ich liebhatte, haben sich wider mich gekehret.
\bibverse{20} Mein Gebein hanget an meiner Haut und Fleisch, und kann
meine Zähne mit der Haut nicht bedecken. \bibverse{21} Erbarmet euch
mein, erbarmet euch mein, ihr, meine Freunde; denn die Hand GOttes hat
mich gerühret. \bibverse{22} Warum verfolget ihr mich gleich so wohl als
GOtt und könnet meines Fleisches nicht satt werden? \bibverse{23} Ach,
daß meine Reden geschrieben würden! Ach, daß sie in ein Buch gestellet
würden, \bibverse{24} mit einem eisernen Griffel auf Blei und zu ewigem
Gedächtnis in einen Fels gehauen würden! \bibverse{25} Aber ich weiß,
daß mein Erlöser lebet; und er wird mich hernach aus der Erde
auferwecken; \bibverse{26} und werde danach mit dieser meiner Haut
umgeben werden und werde in meinem Fleisch GOtt sehen. \bibverse{27}
Denselben werde ich mir sehen, und meine Augen werden ihn schauen, und
kein Fremder. Meine Nieren sind verzehret in meinem Schoß. \bibverse{28}
Denn ihr sprechet: Wie wollen wir ihn verfolgen und eine Sache zu ihm
finden? \bibverse{29} Fürchtet euch vor dem Schwert; denn das Schwert
ist der Zorn über die Missetat, auf daß ihr wisset, daß ein Gericht sei.

\hypertarget{section-19}{%
\section{20}\label{section-19}}

\bibverse{1} Da antwortete Zophar von Naema und sprach: \bibverse{2}
Darauf muß ich antworten und kann nicht harren. \bibverse{3} Und will
gerne hören, wer mir das soll strafen und tadeln; denn der Geist meines
Verstandes soll für mich antworten. \bibverse{4} Weißt du nicht, daß es
allezeit so gegangen ist, seit daß Menschen auf Erden gewesen sind;
\bibverse{5} daß der Ruhm der Gottlosen stehet nicht lange, und die
Freude des Heuchlers währet einen Augenblick? \bibverse{6} Wenn gleich
seine Höhe in den Himmel reichet und sein Haupt an die Wolken rühret,
\bibverse{7} so wird er doch zuletzt umkommen wie ein Kot, daß die, vor
denen er ist angesehen, werden sagen: Wo ist er? \bibverse{8} Wie ein
Traum vergehet, so wird er auch nicht funden werden, und wie ein Gesicht
in der Nacht verschwindet. \bibverse{9} Welch Auge ihn gesehen hat, wird
ihn nicht mehr sehen, und seine Stätte wird ihn nicht mehr schauen.
\bibverse{10} Seine Kinder werden betteln gehen, und seine Hand wird ihm
Mühe zu Lohn geben. \bibverse{11} Seine Beine werden seine heimliche
Sünde wohl bezahlen und werden sich mit ihm in die Erde legen.
\bibverse{12} Wenn ihm die Bosheit gleich in seinem Munde wohl schmeckt,
wird sie doch ihm in seiner Zunge fehlen. \bibverse{13} Sie wird
aufgehalten und ihm nicht gestattet, und wird ihm gewehret werden in
seinem Halse. \bibverse{14} Seine Speise inwendig im Leibe wird sich
verwandeln in Otterngalle. \bibverse{15} Die Güter, die er verschlungen
hat, muß er wieder ausspeien; und GOtt wird sie aus seinem Bauch stoßen.
\bibverse{16} Er wird der Ottern Galle saugen, und die Zunge der
Schlange wird ihn töten. \bibverse{17} Er wird nicht sehen die Ströme
noch die Wasserbäche, die mit Honig und Butter fließen. \bibverse{18} Er
wird arbeiten und des nicht genießen; und seine Güter werden andern, daß
er deren nicht froh wird. \bibverse{19} Denn er hat unterdrückt und
verlassen den Armen; er hat Häuser zu sich gerissen, die er nicht
erbauet hat. \bibverse{20} Denn sein Wanst konnte nicht voll werden, und
wird durch sein köstlich Gut nicht entrinnen. \bibverse{21} Es wird
seiner Speise nicht überbleiben; darum wird sein gut Leben keinen
Bestand haben. \bibverse{22} Wenn er gleich die Fülle und genug hat,
wird ihm doch angst werden; allerhand Mühe wird über ihn kommen.
\bibverse{23} Es wird ihm der Wanst einmal voll werden, und er wird den
Grimm seines Zornes über ihn senden; er wird über ihn regnen lassen
seinen Streit. \bibverse{24} Er wird fliehen vor dem eisernen Harnisch,
und der eherne Bogen wird ihn verjagen. \bibverse{25} Ein bloß Schwert
wird durch ihn ausgehen, und des Schwerts Blitz, der ihm bitter sein
wird, wird mit Schrecken über ihn fahren. \bibverse{26} Es ist keine
Finsternis da, die ihn verdecken möchte. Es wird ihn ein Feuer
verzehren, das nicht aufgeblasen ist; und wer übrig ist in seiner Hütte,
dem wird's übel gehen. \bibverse{27} Der Himmel wird seine Missetat
eröffnen, und die Erde wird sich wider ihn setzen. \bibverse{28} Das
Getreide in seinem Hause wird weggeführet werden, zerstreuet am Tage
seines Zorns. \bibverse{29} Das ist der Lohn eines gottlosen Menschen
bei GOtt und das Erbe seiner Rede bei GOtt.

\hypertarget{section-20}{%
\section{21}\label{section-20}}

\bibverse{1} Hiob antwortete und sprach: \bibverse{2} Höret doch zu
meiner Rede und lasset euch raten! \bibverse{3} Vertraget mich, daß ich
auch rede, und spottet danach mein. \bibverse{4} Handele ich denn mit
einem Menschen, daß mein Mut hierin nicht sollte unwillig sein?
\bibverse{5} Kehret euch her zu mir; ihr werdet sauer sehen und die Hand
aufs Maul legen müssen. \bibverse{6} Wenn ich daran gedenke, so
erschrecke ich, und Zittern kommt mein Fleisch an. \bibverse{7} Warum
leben denn die Gottlosen, werden alt und nehmen zu mit Gütern?
\bibverse{8} Ihr Same ist sicher um sie her, und ihre Nachkömmlinge sind
bei ihnen. \bibverse{9} Ihr Haus hat Frieden vor der Furcht, und GOttes
Rute ist nicht über ihnen. \bibverse{10} Seine Ochsen lässet man zu, und
mißrät ihm nicht; seine Kuh kalbet und ist nicht unfruchtbar.
\bibverse{11} Ihre jungen Kinder gehen aus wie eine Herde, und ihre
Kinder lecken. \bibverse{12} Sie jauchzen mit Pauken und Harfen und sind
fröhlich mit Pfeifen. \bibverse{13} Sie werden alt bei guten Tagen und
erschrecken kaum einen Augenblick vor der Hölle, \bibverse{14} die doch
sagen zu GOtt: Hebe dich von uns, wir wollen von deinen Wegen nicht
wissen. \bibverse{15} Wer ist der Allmächtige, daß wir ihm dienen
sollten, oder was sind wir gebessert, so wir ihn anrufen? \bibverse{16}
Aber siehe, ihr Gut stehet nicht in ihren Händen; darum soll der
Gottlosen Sinn ferne von mir sein. \bibverse{17} Wie wird die Leuchte
der Gottlosen verlöschen und ihr Unglück über sie kommen! Er wird
Herzeleid austeilen in seinem Zorn. \bibverse{18} Sie werden sein wie
Stoppeln vor dem Winde und wie Spreu, die der Sturmwind wegführet.
\bibverse{19} GOtt behält desselben Unglück auf seine Kinder. Wenn er's
ihm vergelten wird, so wird man's inne werden. \bibverse{20} Seine Augen
werden sein Verderben sehen, und vom Grimm des Allmächtigen wird er
trinken. \bibverse{21} Denn wer wird Gefallen haben an seinem Hause nach
ihm? Und die Zahl seiner Monden wird kaum halb bleiben. \bibverse{22}
Wer will GOtt lehren, der auch die Hohen richtet? \bibverse{23} Dieser
stirbt frisch und gesund in allem Reichtum und voller Genüge;
\bibverse{24} sein Melkfaß ist voll Milch, und seine Gebeine werden
gemästet mit Mark; \bibverse{25} jener aber stirbt mit betrübter Seele
und hat nie mit Freuden gegessen; \bibverse{26} und liegen gleich
miteinander in der Erde, und Würmer decken sie zu. \bibverse{27} Siehe,
ich kenne eure Gedanken wohl und euer frevel Vornehmen wider mich.
\bibverse{28} Denn ihr sprechet: Wo ist das Haus des Fürsten, und wo ist
die Hütte, da die Gottlosen wohneten? \bibverse{29} Redet ihr doch davon
wie der gemeine Pöbel und merket nicht, was jener Wesen bedeutet.
\bibverse{30} Denn der Böse wird behalten auf den Tag des Verderbens,
und auf den Tag des Grimms bleibet er. \bibverse{31} Wer will sagen, was
er verdienet, wenn man's äußerlich ansiehet? Wer will ihm vergelten, was
er tut? \bibverse{32} Aber er wird zum Grabe gerissen und muß bleiben
bei dem Haufen. \bibverse{33} Es gefiel ihm wohl der Schlamm des Bachs,
und alle Menschen werden ihm nachgezogen; und derer, die vor ihm gewesen
sind, ist keine Zahl. \bibverse{34} Wie tröstet ihr mich so vergeblich,
und eure Antwort findet sich unrecht.

\hypertarget{section-21}{%
\section{22}\label{section-21}}

\bibverse{1} Da antwortete Eliphas von Theman und sprach: \bibverse{2}
Was darf GOtt eines Starken, und was nützt ihm ein Kluger? \bibverse{3}
Meinest du, daß dem Allmächtigen gefalle, daß du dich so fromm machest?
Oder was hilft's ihm, ob du deine Wege gleich ohne Wandel achtest?
\bibverse{4} Meinest du, er wird sich vor dir fürchten, dich zu strafen,
und mit dir vor Gericht treten? \bibverse{5} Ja, deine Bosheit ist zu
groß, und deiner Missetat ist kein Ende. \bibverse{6} Du hast etwa
deinem Bruder ein Pfand genommen ohne Ursache, du hast den Nackenden die
Kleider ausgezogen; \bibverse{7} du hast die Müden nicht getränket mit
Wasser und hast dem Hungrigen dein Brot versagt; \bibverse{8} du hast
Gewalt im Lande geübet und prächtig drinnen gesessen; \bibverse{9} die
Witwen hast du leer lassen gehen und die Arme der Waisen zerbrochen.
\bibverse{10} Darum bist du mit Stricken umgeben, und Furcht hat dich
plötzlich erschrecket. \bibverse{11} Solltest du denn nicht die
Finsternis sehen, und die Wasserflut dich nicht bedecken? \bibverse{12}
Siehe, GOtt ist hoch droben im Himmel und siehet die Sterne droben in
der Höhe. \bibverse{13} Und du sprichst: Was weiß GOtt? Sollt er, das im
Dunkeln ist, richten können? \bibverse{14} Die Wolken sind seine
Vordecke, und siehet nicht, und wandelt im Umgang des Himmels.
\bibverse{15} Willst du der Welt Lauf achten, darinnen die Ungerechten
gegangen sind, \bibverse{16} die vergangen sind, ehe denn es Zeit war,
und das Wasser hat ihren Grund weggewaschen, \bibverse{17} die zu GOtt
sprachen: Heb dich von uns, was sollte der Allmächtige ihnen tun können,
\bibverse{18} so er doch ihr Haus mit Gütern füllete? Aber der Gottlosen
Rat sei ferne von mir! \bibverse{19} Die Gerechten werden sehen und sich
freuen, und der Unschuldige wird ihrer spotten. \bibverse{20} Was
gilt's, ihr Wesen wird verschwinden und ihr Übriges das Feuer verzehren!
\bibverse{21} So vertrage dich nun mit ihm und habe Frieden; daraus wird
dir viel Gutes kommen. \bibverse{22} Höre das Gesetz von seinem Munde
und fasse seine Rede in dein Herz. \bibverse{23} Wirst du dich bekehren
zu dem Allmächtigen, so wirst du gebauet werden und Unrecht ferne von
deiner Hütte tun, \bibverse{24} so wirst du für Erde Gold geben und für
die Felsen güldene Bäche; \bibverse{25} und der Allmächtige wird dein
Gold sein, und Silber wird dir zugehäuft werden. \bibverse{26} Dann
wirst du deine Lust haben an dem Allmächtigen und dein Antlitz zu GOtt
aufheben. \bibverse{27} So wirst du ihn bitten, und er wird dich hören;
und wirst deine Gelübde bezahlen. \bibverse{28} Was du wirst vornehmen,
wird er dir lassen gelingen; und das Licht wird auf deinem Wege
scheinen. \bibverse{29} Denn die sich demütigen, die erhöhet er; und wer
seine Augen niederschlägt, der wird genesen. \bibverse{30} Und der
Unschuldige wird errettet werden; er wird aber errettet um seiner Hände
Reinigkeit willen.

\hypertarget{section-22}{%
\section{23}\label{section-22}}

\bibverse{1} Hiob antwortete und sprach: \bibverse{2} Meine Rede bleibet
noch betrübt; meine Macht ist schwach über meinem Seufzen. \bibverse{3}
Ach, daß ich wüßte, wie ich ihn finden und zu seinem Stuhl kommen möchte
\bibverse{4} und das Recht vor ihm sollte vorlegen und den Mund voll
Strafe fassen \bibverse{5} und erfahren die Rede, die er mir antworten,
und vernehmen, was er mir sagen würde! \bibverse{6} Will er mit großer
Macht mit mir rechten? Er stelle sich nicht so gegen mich, \bibverse{7}
sondern lege mir's gleich vor, so will ich mein Recht wohl gewinnen.
\bibverse{8} Aber gehe ich nun stracks vor mich, so ist er nicht da;
gehe ich zurück, so spüre ich ihn nicht. \bibverse{9} Ist er zur Linken,
so ergreife ich ihn nicht; verbirget er sich zur Rechten, so sehe ich
ihn nicht. \bibverse{10} Er aber kennet meinen Weg wohl. Er versuche
mich, so will ich erfunden werden wie das Gold. \bibverse{11} Denn ich
setze meinen Fuß auf seine Bahn und halte seinen Weg und weiche nicht ab
\bibverse{12} und trete nicht von dem Gebot seiner Lippen; und bewahre
die Rede seines Mundes mehr, denn ich schuldig bin. \bibverse{13} Er ist
einig, wer will ihm antworten? Und er macht es, wie er will.
\bibverse{14} Und wenn er mir gleich vergilt, was ich verdienet habe, so
ist sein noch mehr dahinten. \bibverse{15} Darum erschrecke ich vor ihm;
und wenn ich's merke, so fürchte ich mich vor ihm. \bibverse{16} GOtt
hat mein Herz blöde gemacht, und der Allmächtige hat mich erschrecket.
\bibverse{17} Denn die Finsternis macht kein Ende mit mir, und das
Dunkel will vor mir nicht verdeckt werden.

\hypertarget{section-23}{%
\section{24}\label{section-23}}

\bibverse{1} Warum sollten die Zeiten dem Allmächtigen nicht verborgen
sein? Und die ihn kennen, sehen seine Tage nicht. \bibverse{2} Sie
treiben die Grenzen zurück; sie rauben die Herden und weiden sie.
\bibverse{3} Sie treiben der Waisen Esel weg und nehmen der Witwen
Ochsen zu Pfande. \bibverse{4} Die Armen müssen ihnen weichen, und die
Dürftigen im Lande müssen sich verkriechen. \bibverse{5} Siehe, das Wild
in der Wüste gehet heraus, wie sie pflegen, frühe zum Raub, daß sie
Speise bereiten für die Jungen. \bibverse{6} Sie ernten auf dem Acker
alles, was er trägt, und lesen den Weinberg, den sie mit Unrecht haben.
\bibverse{7} Die Nackenden lassen sie liegen und lassen ihnen keine
Decke im Frost, denen sie die Kleider genommen haben, \bibverse{8} daß
sie sich müssen zu den Felsen halten, wenn ein Platzregen von den Bergen
auf sie gießt, weil sie sonst keinen Trost haben. \bibverse{9} Sie
reißen das Kind von den Brüsten und machen's zum Waisen und machen die
Leute arm mit Pfänden. \bibverse{10} Den Nackenden lassen sie ohne
Kleider gehen und den Hungrigen nehmen sie die Garben. \bibverse{11} Sie
zwingen sie, Öl zu machen auf ihrer eigenen Mühle und ihre eigene Kelter
zu treten, und lassen sie doch Durst leiden. \bibverse{12} Sie machen
die Leute in der Stadt seufzend und die Seelen der Erschlagenen
schreiend; und GOtt stürzet sie nicht. \bibverse{13} Darum sind sie
abtrünnig worden vom Licht und kennen seinen Weg nicht und kehren nicht
wieder zu seiner Straße. \bibverse{14} Wenn der Tag anbricht, stehet auf
der Mörder und erwürget den Armen und Dürftigen; und des Nachts ist er
wie ein Dieb. \bibverse{15} Das Auge des Ehebrechers hat acht auf das
Dunkel und spricht: Mich siehet kein Auge; und verdecket sein Antlitz.
\bibverse{16} Im Finstern bricht er zu den Häusern ein. Des Tages
verbergen sie sich miteinander und scheuen das Licht. \bibverse{17} Denn
wo ihnen der Morgen kommt, ist's ihnen wie eine Finsternis; denn er
fühlet das Schrecken der Finsternis. \bibverse{18} Er fähret
leichtfertig wie auf einem Wasser dahin; seine Habe wird geringe im
Lande, und bauet seinen Weinberg nicht. \bibverse{19} Die Hölle nimmt
weg, die da sündigen, wie die Hitze und Dürre das Schneewasser
verzehret. \bibverse{20} Es werden sein vergessen die Barmherzigen;
seine Lust wird wurmig werden; sein wird nicht mehr gedacht; er wird
zerbrochen werden wie ein fauler Baum. \bibverse{21} Er hat beleidiget
die Einsame, die nicht gebiert, und hat der Witwe kein Gutes getan
\bibverse{22} und die Mächtigen unter sich gezogen mit seiner Kraft.
Wenn er stehet, wird er seines Lebens nicht gewiß sein. \bibverse{23} Er
macht ihm wohl selbst eine Sicherheit, darauf er sich verlasse; doch
sehen seine Augen auf ihr Tun. \bibverse{24} Sie sind eine kleine Zeit
erhaben und werden zunichte und unterdrückt und ganz und gar ausgetilget
werden, und wie die erste Blüte an den Ähren werden sie abgeschlagen
werden. \bibverse{25} Ist's nicht also? Wohlan, wer will mich Lügen
strafen und bewähren, daß meine Rede nichts sei?

\hypertarget{section-24}{%
\section{25}\label{section-24}}

\bibverse{1} Da antwortete Bildad von Suah und sprach: \bibverse{2} Ist
nicht die Herrschaft und Furcht bei ihm, der den Frieden macht unter
seinen Höchsten? \bibverse{3} Wer will seine Kriegsleute zählen? Und
über welchen gehet nicht auf sein Licht? \bibverse{4} Und wie mag ein
Mensch gerecht vor GOtt sein? Und wie mag rein sein eines Weibes Kind?
\bibverse{5} Siehe, der Mond scheinet noch nicht, und die Sterne sind
noch nicht rein vor seinen Augen; \bibverse{6} wieviel weniger ein
Mensch, die Made, und ein Menschenkind, der Wurm?

\hypertarget{section-25}{%
\section{26}\label{section-25}}

\bibverse{1} Hiob antwortete und sprach: \bibverse{2} Wem stehest du
bei? Dem, der keine Kraft hat? Hilfst du dem, der keine Stärke in Armen
hat? \bibverse{3} Wem gibst du Rat? Dem, der keine Weisheit hat? und
zeigest einem Mächtigen, wie er's ausführen soll? \bibverse{4} Für wen
redest du, und für wen gehet der Odem von dir? \bibverse{5} Die Riesen
ängsten sich unter den Wassern und die bei ihnen wohnen. \bibverse{6}
Die Hölle ist aufgedeckt vor ihm, und das Verderben hat keine Decke.
\bibverse{7} Er breitet aus die Mitternacht nirgend an und hänget die
Erde an nichts. \bibverse{8} Er fasset das Wasser zusammen in seine
Wolken, und die Wolken zerreißen drunter nicht. \bibverse{9} Er hält
seinen Stuhl und breitet seine Wolken davor. \bibverse{10} Er hat um das
Wasser ein Ziel gesetzt, bis das Licht samt der Finsternis vergehe.
\bibverse{11} Die Säulen des Himmels zittern und entsetzen sich vor
seinem Schelten. \bibverse{12} Vor seiner Kraft wird das Meer plötzlich
ungestüm, und vor seinem Verstand erhebet sich die Höhe des Meers.
\bibverse{13} Am Himmel wird's schön durch seinen Wind, und seine Hand
bereitet die gerade Schlange. \bibverse{14} Siehe, also gehet sein Tun,
aber davon haben wir ein gering Wörtlein vernommen. Wer will aber den
Donner seiner Macht verstehen?

\hypertarget{section-26}{%
\section{27}\label{section-26}}

\bibverse{1} Und Hiob fuhr fort und hub an seine Sprüche und sprach:
\bibverse{2} So wahr GOtt lebt, der mir mein Recht nicht gehen lässet,
und der Allmächtige, der meine Seele betrübet, \bibverse{3} solange mein
Odem in mir ist, und das Schnauben von GOtt in meiner Nase ist:
\bibverse{4} meine Lippen sollen nichts Unrechts reden, und meine Zunge
soll keinen Betrug sagen. \bibverse{5} Das sei ferne von mir, daß ich
euch recht gebe; bis daß mein Ende kommt, will ich nicht weichen von
meiner Frömmigkeit. \bibverse{6} Von meiner Gerechtigkeit, die ich habe,
will ich nicht lassen; mein Gewissen beißt mich nicht meines ganzen
Lebens halber. \bibverse{7} Aber mein Feind wird erfunden werden ein
Gottloser, und der sich wider mich auflehnet, ein Ungerechter.
\bibverse{8} Denn was ist die Hoffnung des Heuchlers, daß er so geizig
ist, und GOtt doch seine Seele hinreißet? \bibverse{9} Meinest du, daß
GOtt sein Schreien hören wird, wenn die Angst über ihn kommt?
\bibverse{10} Wie kann er an dem Allmächtigen Lust haben und GOtt etwa
anrufen? \bibverse{11} Ich will euch lehren von der Hand GOttes; und was
bei dem Allmächtigen gilt, will ich nicht verhehlen. \bibverse{12}
Siehe, ihr haltet euch alle für klug. Warum gebt ihr denn solch unnütze
Dinge vor? \bibverse{13} Das ist der Lohn eines gottlosen Menschen bei
GOtt und das Erbe der Tyrannen, das sie von dem Allmächtigen nehmen
werden. \bibverse{14} Wird er viel Kinder haben, so werden sie des
Schwerts sein; und seine Nachkömmlinge werden des Brots nicht satt
haben. \bibverse{15} Seine Übrigen werden im Tode begraben werden, und
seine Witwen werden nicht weinen. \bibverse{16} Wenn er Geld
zusammenbringet wie Erde und sammelt Kleider wie Leimen, \bibverse{17}
so wird er es wohl bereiten; aber der Gerechte wird es anziehen, und der
Unschuldige wird das Geld austeilen. \bibverse{18} Er bauet sein Haus
wie eine Spinne, und wie ein Hüter einen Schauer machet. \bibverse{19}
Der Reiche, wenn er sich legt, wird er's nicht mitraffen; er wird seine
Augen auftun, und da wird nichts sein. \bibverse{20} Es wird ihn
Schrecken überfallen wie Wasser; des Nachts wird ihn das Ungewitter
wegnehmen. \bibverse{21} Der Ostwind wird ihn wegführen, daß er
dahinfähret, und Ungestüm wird ihn von seinem Ort treiben. \bibverse{22}
Er wird solches über ihn führen und wird sein nicht schonen; es wird ihm
alles aus seinen Händen entfliehen. \bibverse{23} Man wird über ihn mit
den Händen klappen und über ihn zischen, da er gewesen ist.

\hypertarget{section-27}{%
\section{28}\label{section-27}}

\bibverse{1} Es hat das Silber seine Gänge und das Gold seinen Ort, da
man es schmelzt. \bibverse{2} Eisen bringet man aus der Erde, und aus
den Steinen schmelzt man Erz. \bibverse{3} Es wird je des Finstern etwa
ein Ende, und jemand findet ja zuletzt den Schiefer tief verborgen.
\bibverse{4} Es bricht ein solcher Bach hervor, daß, die darum wohnen,
den Weg daselbst verlieren; und fällt wieder und schießt dahin von den
Leuten. \bibverse{5} Man bringet auch Feuer unten aus der Erde, da doch
oben Speise auf wächst. \bibverse{6} Man findet Saphir an etlichen Orten
und Erdenklöße, da Gold ist. \bibverse{7} Den Steig kein Vogel erkannt
hat und kein Geiersauge gesehen. \bibverse{8} Es haben die stolzen
Kinder nicht drauf getreten, und ist kein Löwe drauf gegangen.
\bibverse{9} Auch legt man die Hand an die Felsen und gräbet die Berge
um. \bibverse{10} Man reißet Bäche aus den Felsen; und alles, was
köstlich ist, siehet das Auge. \bibverse{11} Man wehret dem Strom des
Wassers und bringet, das verborgen drinnen ist, ans Licht. \bibverse{12}
Wo will man aber Weisheit finden, und wo ist die Stätte des Verstandes?
\bibverse{13} Niemand weiß, wo sie liegt, und wird nicht funden im Lande
der Lebendigen. \bibverse{14} Der Abgrund spricht: Sie ist in mir nicht;
und das Meer spricht: Sie ist nicht bei mir. \bibverse{15} Man kann
nicht Gold um sie geben, noch Silber darwägen, sie zu bezahlen.
\bibverse{16} Es gilt ihr nicht gleich ophirisch Gold oder köstlicher
Onyx und Saphir. \bibverse{17} Gold und Demant mag ihr nicht gleichen,
noch um sie gülden Kleinod wechseln. \bibverse{18} Ramoth und Gabis
achtet man nicht. Die Weisheit ist höher zu wägen denn Perlen.
\bibverse{19} Topasius aus Mohrenland wird ihr nicht gleich geschätzt,
und das reinste Gold gilt ihr nicht gleich. \bibverse{20} Woher kommt
denn die Weisheit, und wo ist die Stätte des Verstandes? \bibverse{21}
Sie ist verhohlen vor den Augen aller Lebendigen, auch verborgen den
Vögeln unter dem Himmel. \bibverse{22} Die Verdammnis und der Tod
sprechen: Wir haben mit unsern Ohren ihr Gerücht gehöret. \bibverse{23}
GOtt weiß den Weg dazu und kennet ihre Stätte. \bibverse{24} Denn er
siehet die Enden der Erde und schauet alles, was unter dem Himmel ist.
\bibverse{25} Da er dem Winde sein Gewicht machte und setzte dem Wasser
sein gewisses Maß, \bibverse{26} da er dem Regen ein Ziel machte und dem
Blitz und Donner den Weg, \bibverse{27} da sah er sie und erzählete sie,
bereitete sie und erfand sie; \bibverse{28} und sprach zum Menschen:
Siehe, die Furcht des HErrn, das ist Weisheit, und meiden das Böse, das
ist Verstand.

\hypertarget{section-28}{%
\section{29}\label{section-28}}

\bibverse{1} Und Hiob hub abermal an seine Sprüche und sprach:
\bibverse{2} O daß ich wäre wie in den vorigen Monden, in den Tagen, da
mich GOtt behütete, \bibverse{3} da seine Leuchte über meinem Haupte
schien, und ich bei seinem Licht in der Finsternis ging; \bibverse{4}
wie ich war zur Zeit meiner Jugend, da GOttes Geheimnis über meiner
Hütte war; \bibverse{5} da der Allmächtige noch mit mir war und meine
Kinder um mich her; \bibverse{6} da ich meine Tritte wusch in Butter,
und die Felsen mir Ölbäche gossen; \bibverse{7} da ich ausging zum Tor
in der Stadt und ließ meinen Stuhl auf der Gasse bereiten; \bibverse{8}
da mich die Jungen sahen und sich versteckten, und die Alten vor mir
aufstunden; \bibverse{9} da die Obersten aufhöreten zu reden, und legten
ihre Hand auf ihren Mund; \bibverse{10} da die Stimme der Fürsten sich
verkroch, und ihre Zunge an ihrem Gaumen klebte. \bibverse{11} Denn
welches Ohr mich hörete, der preisete mich selig, und welches Auge mich
sah, der rühmte mich. \bibverse{12} Denn ich errettete den Armen, der da
schrie, und den Waisen, der keinen Helfer hatte. \bibverse{13} Der Segen
des, der verderben sollte, kam über mich; und ich erfreuete das Herz der
Witwe. \bibverse{14} Gerechtigkeit war mein Kleid, das ich anzog wie
einen Rock; und mein Recht war mein fürstlicher Hut. \bibverse{15} Ich
war des Blinden Auge und des Lahmen Füße. \bibverse{16} Ich war ein
Vater der Armen; und welche Sache ich nicht wußte, die erforschete ich.
\bibverse{17} Ich zerbrach die Backenzähne des Ungerechten und riß den
Raub aus seinen Zähnen. \bibverse{18} Ich gedachte: Ich will in meinem
Nest ersterben und meiner Tage viel machen wie Sand. \bibverse{19} Meine
Saat ging auf am Wasser; und der Tau blieb über meiner Ernte.
\bibverse{20} Meine Herrlichkeit erneuerte sich immer an mir; und mein
Bogen besserte sich in meiner Hand. \bibverse{21} Man hörete mir zu, und
schwiegen und warteten auf meinen Rat. \bibverse{22} Nach meinen Worten
redete niemand mehr; und meine Rede troff auf sie. \bibverse{23} Sie
warteten auf mich wie auf den Regen und sperreten ihren Mund auf als
nach dem Abendregen. \bibverse{24} Wenn ich sie anlachte, wurden sie
nicht zu kühn darauf, und das Licht meines Angesichts machte mich nicht
geringer. \bibverse{25} Wenn ich zu ihrem Geschäfte wollte kommen, so
mußte ich obenan sitzen und wohnete wie ein König unter Kriegsknechten,
da ich tröstete, die Leid trugen.

\hypertarget{section-29}{%
\section{30}\label{section-29}}

\bibverse{1} Nun aber lachen mein, die jünger sind denn ich, welcher
Väter ich verachtet hätte, zu stellen unter meine Schafhunde,
\bibverse{2} welcher Vermögen ich für nichts hielt, die nicht zum Alter
kommen konnten, \bibverse{3} die vor Hunger und Kummer einsam flohen in
die Einöde, neulich verdorben und elend worden, \bibverse{4} die da
Nesseln ausrauften um die Büsche, und Wacholderwurzel war ihre Speise;
\bibverse{5} und wenn sie die herausrissen, jauchzeten sie drüber wie
ein Dieb. \bibverse{6} An den grausamen Bächen wohneten sie, in den
Löchern der Erde und Steinritzen. \bibverse{7} Zwischen den Büschen
riefen sie und unter den Disteln sammelten sie, \bibverse{8} die Kinder
loser und verachteter Leute, die die Geringsten im Lande waren.
\bibverse{9} Nun bin ich ihr Saitenspiel worden und muß ihr Märlein
sein. \bibverse{10} Sie haben einen Greuel an mir und machen sich ferne
von mir und schonen nicht, vor meinem Angesicht zu speien. \bibverse{11}
Sie haben mein Seil ausgespannet und mich zunichte gemacht und das Meine
abgezäumet. \bibverse{12} Zur Rechten, da ich grünete, haben sie sich
wieder mich gesetzt und haben meinen Fuß ausgestoßen; und haben über
mich einen Weg gemacht, mich zu verderben. \bibverse{13} Sie haben meine
Steige zerbrochen; es war ihnen so leicht, mich zu beschädigen, daß sie
keiner Hilfe dazu bedurften. \bibverse{14} Sie sind kommen, wie zur
weiten Lücke herein, und sind ohne Ordnung dahergefallen. \bibverse{15}
Schrecken hat sich gegen mich gekehret und hat verfolget wie der Wind
meine Herrlichkeit und wie eine laufende Wolke meinen glückseligen
Stand. \bibverse{16} Nun aber gießt sich aus meine Seele über mich, und
mich hat ergriffen die elende Zeit. \bibverse{17} Des Nachts wird mein
Gebein durchbohret allenthalben, und die mich jagen, legen sich nicht
schlafen. \bibverse{18} Durch die Menge der Kraft werde ich anders und
anders gekleidet; und man gürtet mich damit wie mit dem Loch meines
Rocks. \bibverse{19} Man hat mich in Kot getreten und gleich geachtet
dem Staub und Asche. \bibverse{20} Schreie ich zu dir, so antwortest du
mir nicht; trete ich hervor, so achtest du nicht auf mich. \bibverse{21}
Du bist mir verwandelt in einen Grausamen und zeigest deinen Gram an mir
mit der Stärke deiner Hand. \bibverse{22} Du hebest mich auf und lässest
mich auf dem Winde fahren und zerschmelzest mich kräftiglich.
\bibverse{23} Denn ich weiß, du wirst mich dem Tode überantworten; da
ist das bestimmte Haus aller Lebendigen. \bibverse{24} Doch wird er
nicht die Hand ausstrecken ins Beinhaus, und werden nicht schreien vor
seinem Verderben. \bibverse{25} Ich weinete ja in der harten Zeit, und
meine Seele jammerte der Armen. \bibverse{26} Ich wartete des Guten, und
kommt das Böse; ich hoffte aufs Licht, und kommt Finsternis.
\bibverse{27} Meine Eingeweide sieden und hören nicht auf; mich hat
überfallen die elende Zeit. \bibverse{28} Ich gehe schwarz einher, und
brennet mich doch keine Sonne nicht; ich stehe auf in der Gemeine und
schreie. \bibverse{29} Ich bin ein Bruder der Schlangen und ein Geselle
der Straußen. \bibverse{30} Meine Haut über mir ist schwarz worden, und
meine Gebeine sind verdorret vor Hitze. \bibverse{31} Meine Harfe ist
eine Klage worden und meine Pfeife ein Weinen.

\hypertarget{section-30}{%
\section{31}\label{section-30}}

\bibverse{1} Ich habe einen Bund gemacht mit meinen Augen, daß ich nicht
achtete auf eine Jungfrau. \bibverse{2} Was gibt mir aber GOtt zu Lohn
von oben? und was für ein Erbe der Allmächtige von der Höhe?
\bibverse{3} Sollte nicht billiger der Ungerechte solch Unglück haben,
und ein Übeltäter so verstoßen werden? \bibverse{4} Siehet er nicht
meine Wege und zählet alle meine Gänge? \bibverse{5} Hab ich gewandelt
in Eitelkeit? oder hat mein Fuß geeilet zum Betrug? \bibverse{6} So wäge
man mich auf rechter Waage, so wird GOtt erfahren meine Frömmigkeit.
\bibverse{7} Hat mein Gang gewichen aus dem Wege und mein Herz meinen
Augen nachgefolget, und ist etwas in meinen Händen beklebet,
\bibverse{8} so müsse ich säen, und ein anderer fresse es, und mein
Geschlecht müsse ausgewurzelt werden. \bibverse{9} Hat sich mein Herz
lassen reizen zum Weibe, und habe an meines Nächsten Tür gelauert,
\bibverse{10} so müsse mein Weib von einem andern geschändet werden, und
andere müssen sie beschlafen. \bibverse{11} Denn das ist ein Laster und
eine Missetat für die Richter. \bibverse{12} Denn das wäre ein Feuer,
das bis ins Verderben verzehrete und all mein Einkommen auswurzelte.
\bibverse{13} Hab ich verachtet das Recht meines Knechts oder meiner
Magd, wenn sie eine Sache wider mich hatten, \bibverse{14} was wollte
ich tun, wenn GOtt sich aufmachte, und was würde ich antworten, wenn er
heimsuchte? \bibverse{15} Hat ihn nicht auch der gemacht, der mich in
Mutterleibe machte, und hat ihn im Leibe ebensowohl bereitet?
\bibverse{16} Hab ich den Dürftigen ihre Begierde versagt und die Augen
der Witwen lassen verschmachten? \bibverse{17} Hab ich meinen Bissen
allein gegessen, und nicht der Waise auch davon gegessen? \bibverse{18}
Denn ich habe mich von Jugend auf gehalten wie ein Vater; und von meiner
Mutter Leibe an hab ich gerne getröstet. \bibverse{19} Hab ich jemand
sehen umkommen, daß er kein Kleid hatte, und den Armen ohne Decke gehen
lassen? \bibverse{20} Haben mich nicht gesegnet seine Seiten, da er von
den Fellen meiner Lämmer erwärmet ward? \bibverse{21} Hab ich meine Hand
an den Waisen gelegt, weil ich mich sah im Tor Macht zu helfen haben,
\bibverse{22} so falle meine Schulter von der Achsel, und mein Arm
breche von der Röhre. \bibverse{23} Denn ich fürchte GOtt, wie einen
Unfall über mich, und könnte seine Last nicht ertragen. \bibverse{24}
Hab ich das Gold zu meiner Zuversicht gestellet und zu dem Goldklumpen
gesagt: Mein Trost? \bibverse{25} Hab ich mich gefreuet, daß ich groß
Gut hatte und meine Hand allerlei erworben hatte? \bibverse{26} Hab ich
das Licht angesehen, wenn es helle leuchtete, und den Mond, wenn er voll
ging? \bibverse{27} Hat sich mein Herz heimlich bereden lassen, daß
meine Hand meinen Mund küsse? \bibverse{28} Welches ist auch eine
Missetat für die Richter; denn damit hätte ich verleugnet GOtt von oben.
\bibverse{29} Hab ich mich gefreuet, wenn's meinem Feinde übel ging, und
habe mich erhoben, daß ihn Unglück betreten hatte? \bibverse{30} Denn
ich ließ meinen Mund nicht sündigen, daß er wünschte einen Fluch seiner
Seele. \bibverse{31} Haben nicht die Männer in meiner Hütte müssen
sagen: O wollte GOtt, daß wir von seinem Fleisch nicht gesättiget
würden! \bibverse{32} Draußen mußte der Gast nicht bleiben, sondern
meine Tür tat ich dem Wanderer auf. \bibverse{33} Hab ich meine
Schalkheit wie ein Mensch gedeckt, daß ich heimlich meine Missetat
verbärge? \bibverse{34} Hab ich mir grauen lassen vor der großen Menge,
und hat die Verachtung der Freundschaften mich abgeschreckt? Ich blieb
stille und ging nicht zur Tür aus. \bibverse{35} Wer gibt mir einen
Verhörer, daß meine Begierde der Allmächtige erhöre, daß jemand ein Buch
schriebe von meiner Sache? \bibverse{36} So wollt ich's auf meine
Achseln nehmen und mir wie eine Krone umbinden. \bibverse{37} Ich wollte
die Zahl meiner Gänge ansagen und wie ein Fürst wollte ich sie
darbringen. \bibverse{38} Wird mein Land wider mich schreien und
miteinander seine Furchen weinen; \bibverse{39} hab ich seine Früchte
unbezahlt gegessen und das Leben der Ackerleute sauer gemacht,
\bibverse{40} so wachsen mir Disteln für Weizen und Dornen für Gerste.
Die Worte Hiobs haben ein Ende.

\hypertarget{section-31}{%
\section{32}\label{section-31}}

\bibverse{1} Da höreten die drei Männer auf, Hiob zu antworten, weil er
sich für gerecht hielt. \bibverse{2} Aber Elihu, der Sohn Baracheels,
von Bus, des Geschlechts Ram, ward zornig über Hiob, daß er seine Seele
gerechter hielt denn GOtt. \bibverse{3} Auch ward er zornig über seine
drei Freunde, daß sie keine Antwort fanden und doch Hiob verdammeten.
\bibverse{4} Denn Elihu hatte geharret, bis daß sie mit Hiob geredet
hatten, weil sie älter waren denn er. \bibverse{5} Darum, da er sah, daß
keine Antwort war im Munde der drei Männer, ward er zornig. \bibverse{6}
Und so antwortete Elihu, der Sohn Baracheels, von Bus, und sprach: Ich
bin jung, ihr aber seid alt; darum hab ich mich gescheuet und
gefürchtet, meine Kunst an euch zu beweisen. \bibverse{7} Ich dachte:
Laß die Jahre reden, und die Menge des Alters laß Weisheit beweisen.
\bibverse{8} Aber der Geist ist in den Leuten, und der Odem des
Allmächtigen macht sie verständig. \bibverse{9} Die Großen sind nicht
die Weisesten, und die Alten verstehen nicht das Recht. \bibverse{10}
Darum will ich auch reden; höre mir zu! Ich will meine Kunst auch sehen
lassen. \bibverse{11} Siehe, ich habe geharret, daß ihr geredet habt;
ich habe aufgemerkt auf euren Verstand, bis ihr träfet die rechte Rede,
\bibverse{12} und habe achtgehabt auf euch; aber siehe, da ist keiner
unter euch, der Hiob strafe oder seiner Rede antworte. \bibverse{13} Ihr
werdet vielleicht sagen: Wir haben die Weisheit getroffen, daß GOtt ihn
verstoßen hat, und sonst niemand. \bibverse{14} Die Rede tut mir nicht
genug; ich will ihm nicht so nach eurer Rede antworten. \bibverse{15}
Ach! sie sind verzagt, können nicht mehr antworten, sie können nicht
mehr reden. \bibverse{16} Weil ich denn geharret habe, und sie konnten
nicht reden (denn sie stehen still und antworten nicht mehr),
\bibverse{17} will doch ich mein Teil antworten und will meine Kunst
beweisen. \bibverse{18} Denn ich bin der Rede so voll, daß mich der Odem
in meinem Bauche ängstet. \bibverse{19} Siehe, mein Bauch ist wie der
Most, der zugestopfet ist, der die neuen Fässer zerreißet. \bibverse{20}
Ich muß reden, daß ich Odem hole; ich muß meine Lippen auftun und
antworten. \bibverse{21} Ich will niemandes Person ansehen und will
keinen Menschen rühmen. \bibverse{22} Denn ich weiß nicht, wo ich's
täte, ob mich mein Schöpfer über ein kleines hinnehmen würde.

\hypertarget{section-32}{%
\section{33}\label{section-32}}

\bibverse{1} Höre doch, Hiob, meine Rede und merke auf alle meine Worte!
\bibverse{2} Siehe, ich tue meinen Mund auf, und meine Zunge redet in
meinem Munde. \bibverse{3} Mein Herz soll recht reden, und meine Lippen
sollen den reinen Verstand sagen. \bibverse{4} Der Geist GOttes hat mich
gemacht, und der Odem des Allmächtigen hat mir das Leben gegeben.
\bibverse{5} Kannst du, so antworte mir; schicke dich gegen mich und
stelle dich! \bibverse{6} Siehe ich bin GOttes ebensowohl als du, und
aus Leimen bin ich auch gemacht. \bibverse{7} Doch du darfst vor mir
nicht erschrecken, und meine Hand soll dir nicht zu schwer sein.
\bibverse{8} Du hast geredet vor meinen Ohren, die Stimme deiner Rede
mußte ich hören: \bibverse{9} Ich bin rein, ohne Missetat, unschuldig
und habe keine Sünde. \bibverse{10} Siehe, er hat eine Sache wider mich
funden, darum achtet er mich für seinen Feind. \bibverse{11} Er hat
meinen Fuß in Stock gelegt und hat alle meine Wege verwahret.
\bibverse{12} Siehe, eben daraus schließe ich wider dich, daß du nicht
recht bist; denn GOtt ist mehr weder ein Mensch. \bibverse{13} Warum
willst du mit ihm zanken, daß er dir nicht Rechenschaft gibt alles
seines Tuns? \bibverse{14} Denn wenn GOtt einmal etwas beschließt, so
bedenket er's nicht erst her nach. \bibverse{15} Im Traum des Gesichts
in der Nacht, wenn der Schlaf auf die Leute fällt, wenn sie schlafen auf
dem Bette, \bibverse{16} da öffnet er das Ohr der Leute und schreckt sie
und züchtiget sie, \bibverse{17} daß er den Menschen von seinem Vorhaben
wende und beschirme ihn vor Hoffart. \bibverse{18} Und verschonet seiner
Seele vor dem Verderben und seines Lebens, daß es nicht ins Schwert
falle. \bibverse{19} Er straft ihn mit Schmerzen auf seinem Bette und
alle seine Gebeine heftig; \bibverse{20} und richtet ihm sein Leben so
zu, daß ihm vor der Speise ekelt, und seine Seele, daß sie nicht Lust zu
essen hat. \bibverse{21} Sein Fleisch verschwindet, daß er nicht wohl
sehen mag, und seine Beine werden zerschlagen, daß man sie nicht gerne
ansiehet, \bibverse{22} daß seine Seele nahet zum Verderben und sein
Leben zu den Toten. \bibverse{23} So dann ein Engel, einer aus tausend,
mit ihm redet, zu verkündigen dem Menschen, wie er solle recht tun,
\bibverse{24} so wird er ihm gnädig sein und sagen: Er soll erlöset
werden, daß er nicht hinunterfahre ins Verderben; denn ich habe eine
Versöhnung funden. \bibverse{25} Sein Fleisch grüne wieder wie in der
Jugend, und laß ihn wieder jung werden. \bibverse{26} Er wird GOtt
bitten; der wird ihm Gnade erzeigen und wird sein Antlitz sehen lassen
mit Freuden und wird dem Menschen nach seiner Gerechtigkeit vergelten.
\bibverse{27} Er wird vor den Leuten bekennen und sagen: Ich wollte
gesündiget und das Recht verkehret haben, aber es hätte mir nichts
genützet. \bibverse{28} Er hat meine Seele erlöset, daß sie nicht führe
ins Verderben, sondern mein Leben das Licht sähe. \bibverse{29} Siehe,
das alles tut GOtt zwei oder dreimal mit einem jeglichen, \bibverse{30}
daß er seine Seele herumhole aus dem Verderben und erleuchte ihn mit dem
Licht der Lebendigen. \bibverse{31} Merke auf, Hiob, und höre mir zu,
und schweige, daß ich rede! \bibverse{32} Hast du aber was zu sagen, so
antworte mir; sage her, bist du recht, ich will's gerne hören.
\bibverse{33} Hast du aber nichts, so höre mir zu und schweige, ich will
dich die Weisheit lehren.

\hypertarget{section-33}{%
\section{34}\label{section-33}}

\bibverse{1} Und Elihu antwortete und sprach: \bibverse{2} Höret, ihr
Weisen, meine Rede, und ihr Verständigen, merket auf mich! \bibverse{3}
Denn das Ohr prüfet die Rede, und der Mund schmecket die Speise.
\bibverse{4} Laßt uns ein Urteil erwählen, daß wir erkennen unter uns,
was gut sei. \bibverse{5} Denn Hiob hat gesagt: Ich bin gerecht, und
GOtt weigert mir mein Recht. \bibverse{6} Ich muß lügen, ob ich wohl
recht habe, und bin gequälet von meinen Pfeilen, ob ich wohl nichts
verschuldet habe. \bibverse{7} Wer ist ein solcher wie Hiob, der da
Spötterei trinket wie Wasser \bibverse{8} und auf dem Wege gehet mit den
Übeltätern und wandelt mit den gottlosen Leuten? \bibverse{9} Denn er
hat gesagt: Wenn jemand schon fromm ist, so gilt er doch nichts bei
GOtt. \bibverse{10} Darum höret mir zu, ihr weisen Leute: Es sei ferne,
daß GOtt sollte gottlos sein und der Allmächtige ungerecht,
\bibverse{11} sondern er vergilt dem Menschen, danach er verdienet hat,
und trifft einen jeglichen nach seinem Tun. \bibverse{12} Ohne Zweifel,
GOtt verdammet niemand mit Unrecht, und der Allmächtige beuget das Recht
nicht. \bibverse{13} Wer hat, das auf Erden ist, verordnet, und wer hat
den ganzen Erdboden gesetzt? \bibverse{14} So er sich's würde
unterwinden, so würde er aller Geist und Odem zu sich sammeln.
\bibverse{15} Alles Fleisch würde miteinander vergehen, und der Mensch
würde wieder zu Asche werden. \bibverse{16} Hast du nun Verstand, so
höre das und merke auf die Stimme meiner Rede. \bibverse{17} Sollte
einer darum das Recht zwingen, daß er's hasset? Und daß du stolz bist,
solltest du darum den Gerechten verdammen? \bibverse{18} Sollt einer zum
Könige sagen: Du loser Mann! und zu den Fürsten: Ihr Gottlosen!?
\bibverse{19} Der doch nicht ansiehet die Person der Fürsten und kennet
den Herrlichen nicht mehr denn den Armen; denn sie sind alle seiner
Hände Werk. \bibverse{20} Plötzlich müssen die Leute sterben und zu
Mitternacht erschrecken und vergehen; die Mächtigen werden kraftlos
weggenommen. \bibverse{21} Denn seine Augen sehen auf eines jeglichen
Wege, und er schaut alle ihre Gänge. \bibverse{22} Es ist kein
Finsternis noch Dunkel, daß sich da möchten verbergen die Übeltäter.
\bibverse{23} Denn es wird niemand gestattet, daß er mit GOtt rechte.
\bibverse{24} Er bringet der Stolzen viel um, die nicht zu zählen sind,
und stellet andere an ihre Statt, \bibverse{25} darum daß er kennet ihre
Werke und kehret sie um des Nachts, daß sie zerschlagen werden.
\bibverse{26} Er wirft die Gottlosen über einen Haufen, da man's gerne
siehet, \bibverse{27} darum daß sie von ihm weggewichen sind und
verstunden seiner Wege keinen, \bibverse{28} daß das Schreien der Armen
mußte vor ihn kommen, und er das Schreien der Elenden hörete.
\bibverse{29} Wenn er Frieden gibt, wer will verdammen? und wenn er das
Antlitz verbirget, wer will ihn schauen unter den Völkern und Leuten?
\bibverse{30} Und läßt über sie regieren einen Heuchler, das Volk zu
drängen. \bibverse{31} Ich muß für GOtt reden und kann's nicht lassen.
\bibverse{32} Hab ich's nicht getroffen, so lehre du mich's besser; hab
ich unrecht gehandelt, ich will's nicht mehr tun. \bibverse{33} Man
wartet der Antwort von dir, denn du verwirfst alles; und du hast's
angefangen und nicht ich. Weißest du nun was, so sage an! \bibverse{34}
Weise Leute lasse ich mir sagen, und ein weiser Mann gehorchet mir.
\bibverse{35} Aber Hiob redete mit Unverstand, und seine Worte sind
nicht klug. \bibverse{36} Mein Vater! laß Hiob versucht werden bis ans
Ende, darum daß er sich zu unrechten Leuten kehret. \bibverse{37} Er hat
über seine Sünde dazu noch gelästert; darum laß Ihn zwischen uns
geschlagen werden und danach viel wider GOtt plaudern.

\hypertarget{section-34}{%
\section{35}\label{section-34}}

\bibverse{1} Und Elihu antwortete und sprach: \bibverse{2} Achtest du
das für recht, daß du sprichst: Ich bin gerechter denn GOtt?
\bibverse{3} Denn du sprichst: Wer gilt bei dir etwas? Was hilft's, ob
ich mich ohne Sünde mache? \bibverse{4} Ich will dir antworten ein Wort
und deinen Freunden mit dir. \bibverse{5} Schaue gen Himmel und siehe,
und schaue an die Wolken, daß sie dir zu hoch sind. \bibverse{6}
Sündigest du, was kannst du mit ihm machen? Und ob deiner Missetat viel
ist, was kannst du ihm tun? \bibverse{7} Und ob du gerecht seiest, was
kannst du ihm geben, oder was wird er von deinen Händen nehmen?
\bibverse{8} Einem Menschen, wie du bist, mag wohl etwas tun deine
Bosheit und einem Menschenkinde deine Gerechtigkeit. \bibverse{9}
Dieselbigen mögen schreien, wenn ihnen viel Gewalt geschieht, und rufen
über den Arm der Großen, \bibverse{10} die nicht danach fragen, wo ist
GOtt, mein Schöpfer, der das Gesänge macht in der Nacht, \bibverse{11}
der uns gelehrter macht denn das Vieh auf Erden und weiser denn die
Vögel unter dem Himmel? \bibverse{12} Aber sie werden da auch schreien
über den Hochmut der Bösen, und er wird sie nicht erhören. \bibverse{13}
Denn GOtt wird das Eitle nicht erhören, und der Allmächtige wird es
nicht ansehen. \bibverse{14} Dazu sprichst du, du werdest ihn nicht
sehen. Aber es ist ein Gericht vor ihm; harre sein nur, \bibverse{15} ob
sein Zorn bald nicht heimsucht, und sich nicht annimmt, daß so viel
Laster da sind. \bibverse{16} Darum hat Hiob seinen Mund umsonst
aufgesperrt und gibt stolze Teiding vor mit Unverstand.

\hypertarget{section-35}{%
\section{36}\label{section-35}}

\bibverse{1} Elihu redete weiter und sprach: \bibverse{2} Harre mir noch
ein wenig, ich will dir's zeigen; denn ich habe noch von GOttes wegen
was zu sagen. \bibverse{3} Ich will meinen Verstand weit holen und
meinen Schöpfer beweisen, daß er recht sei. \bibverse{4} Meine Reden
sollen ohne Zweifel nicht falsch sein, mein Verstand soll ohne Wandel
vor dir sein. \bibverse{5} Siehe, GOtt verwirft die Mächtigen nicht;
denn er ist auch mächtig von Kraft des Herzens. \bibverse{6} Den
Gottlosen erhält er nicht, sondern hilft dem Elenden zum Rechten.
\bibverse{7} Er wendet seine Augen nicht von dem Gerechten und die
Könige läßt er sitzen auf dem Thron immerdar, daß sie hoch bleiben.
\bibverse{8} Und wo Gefangene liegen in Stöcken und gebunden mit
Stricken elendiglich, \bibverse{9} so verkündiget er ihnen, was sie
getan haben, und ihre Untugend, daß sie mit Gewalt gefahren haben.
\bibverse{10} Und öffnet ihnen das Ohr zur Zucht und sagt ihnen, daß sie
sich von dem Unrechten bekehren sollen. \bibverse{11} Gehorchen sie und
dienen ihm, so werden sie bei guten Tagen alt werden und mit Lust leben.
\bibverse{12} Gehorchen sie nicht, so werden sie ins Schwert fallen und
vergehen, ehe sie es gewahr werden. \bibverse{13} Die Heuchler, wenn sie
der Zorn trifft, schreien sie nicht, wenn sie gefangen liegen;
\bibverse{14} so wird ihre Seele mit Qual sterben und ihr Leben unter
den Hurern. \bibverse{15} Aber den Elenden wird er aus seinem Elend
erretten und dem Armen das Ohr öffnen in Trübsal. \bibverse{16} Er wird
dich reißen aus dem weiten Rachen der Angst, die keinen Boden hat; und
dein Tisch wird Ruhe haben, voll alles Guten. \bibverse{17} Du aber
machst die Sache der Gottlosen gut, daß ihre Sache und Recht erhalten
wird. \bibverse{18} Siehe zu, daß dich nicht vielleicht Zorn beweget
habe, jemand zu plagen, oder groß Geschenk dich nicht gebeuget habe.
\bibverse{19} Meinest du, daß er deine Gewalt achte, oder Gold, oder
irgend eine Stärke oder Vermögen? \bibverse{20} Du darfst der Nacht
nicht begehren, die Leute an ihrem Ort zu überfallen. \bibverse{21} Hüte
dich und kehre dich nicht zum Unrecht, wie du denn vor Elend angefangen
hast. \bibverse{22} Siehe, GOtt ist zu hoch in seiner Kraft; wo ist ein
Lehrer, wie er ist? \bibverse{23} Wer will über ihn heimsuchen seinen
Weg, und wer will zu ihm sagen: Du tust unrecht? \bibverse{24} Gedenke,
daß du sein Werk nicht wissest, wie die Leute singen. \bibverse{25} Denn
alle Menschen sehen das, die Leute schauen's von ferne. \bibverse{26}
Siehe, GOtt ist groß und unbekannt; seiner Jahre Zahl kann niemand
forschen. \bibverse{27} Er macht das Wasser zu kleinen Tropfen und
treibt seine Wolken zusammen zum Regen, \bibverse{28} daß die Wolken
fließen und triefen sehr auf die Menschen. \bibverse{29} Wenn er
vornimmt, die Wolken auszubreiten, wie sein hoch Gezelt, \bibverse{30}
siehe, so breitet er aus seinen Blitz über dieselben und bedecket alle
Enden des Meers. \bibverse{31} Denn damit schreckt er die Leute und gibt
doch Speise die Fülle. \bibverse{32} Er decket den Blitz wie mit Händen
und heißt es doch wiederkommen. \bibverse{33} Davon zeuget sein Geselle,
nämlich des Donners Zorn in Wolken.

\hypertarget{section-36}{%
\section{37}\label{section-36}}

\bibverse{1} Des entsetzt sich mein Herz und bebet. \bibverse{2} Lieber,
höret doch, wie sein Donner zürnet, und was für Gespräch von seinem
Munde ausgehet! \bibverse{3} Er siehet unter allen Himmeln, und sein
Blitz scheinet auf die Enden der Erde. \bibverse{4} Dem nach brüllet der
Donner, und er donnert mit seinem großen Schall, und wenn sein Donner
gehöret wird, kann man's nicht aufhalten. \bibverse{5} GOtt donnert mit
seinem Donner greulich und tut große Dinge, und wird doch nicht erkannt.
\bibverse{6} Er spricht zum Schnee, so ist er bald auf Erden, und zum
Platzregen, so ist der Platzregen da mit Macht. \bibverse{7} Alle
Menschen hat er in der Hand als verschlossen, daß die Leute lernen, was
er tun kann. \bibverse{8} Das wilde Tier gehet in die Höhle und bleibt
an seinem Ort. \bibverse{9} Von Mittag her kommt Wetter und von
Mitternacht Kälte. \bibverse{10} Vom Odem GOttes kommt Frost, und große
Wasser, wenn er auftauen läßt. \bibverse{11} Die dicken Wolken scheiden
sich, daß es helle werde, und durch den Nebel bricht sein Licht.
\bibverse{12} Er kehret die Wolken, wo er hin will, daß sie schaffen
alles, was er ihnen gebeut, auf dem Erdboden, \bibverse{13} es sei über
ein Geschlecht oder über ein Land, so man ihn barmherzig findet.
\bibverse{14} Da merke auf, Hiob; stehe, und vernimm die Wunder GOttes!
\bibverse{15} Weißt du, wenn GOtt solches über sie bringt und wenn er
das Licht seiner Wolken läßt hervorbrechen? \bibverse{16} Weißt du, wie
sich die Wolken ausstreuen? Welche Wunder die Vollkommenen wissen.
\bibverse{17} Daß deine Kleider warm sind, wenn das Land stille ist vom
Mittagswind? \bibverse{18} Ja, du wirst mit ihm die Wolken ausbreiten,
die fest stehen wie ein gegossener Spiegel. \bibverse{19} Zeige uns, was
wir ihm sagen sollen; denn wir werden nicht dahin reichen vor
Finsternis. \bibverse{20} Wer wird ihm erzählen, daß ich rede? So jemand
redet, der wird verschlungen. \bibverse{21} Jetzt siehet man das Licht
nicht, das in den Wolken helle leuchtet; wenn aber der Wind wehet, so
wird's klar. \bibverse{22} Von Mitternacht kommt Gold zu Lob vor dem
schrecklichen GOtt. \bibverse{23} Den Allmächtigen aber mögen sie nicht
begreifen, der so groß ist von Kraft; denn er wird von seinem Recht und
guter Sache nicht Rechenschaft geben. \bibverse{24} Darum müssen ihn
fürchten die Leute; und er fürchtet sich vor keinem, wie weise sie sind.

\hypertarget{section-37}{%
\section{38}\label{section-37}}

\bibverse{1} Und der HErr antwortete Hiob aus einem Wetter und sprach:
\bibverse{2} Wer ist der, der so fehlet in der Weisheit und redet so mit
Unverstand? \bibverse{3} Gürte deine Lenden wie ein Mann; ich will dich
fragen, lehre mich! \bibverse{4} Wo warest du, da ich die Erde gründete?
Sage mir's, bist du so klug? \bibverse{5} Weißt du, wer ihr das Maß
gesetzt hat, oder wer über sie eine Richtschnur gezogen hat?
\bibverse{6} Oder worauf stehen ihre Füße versenket? Oder wer hat ihr
einen Eckstein gelegt, \bibverse{7} da mich die Morgensterne miteinander
lobeten, und jauchzeten alle Kinder GOttes? \bibverse{8} Wer hat das
Meer mit seinen Türen verschlossen, da es herausbrach wie aus
Mutterleibe, \bibverse{9} da ich's mit Wolken kleidete und in Dunkel
einwickelte, wie in Windeln, \bibverse{10} da ich ihm den Lauf brach mit
meinem Damm und setzte ihm Riegel und Tür \bibverse{11} und sprach: Bis
hieher sollst du kommen und nicht weiter; hie sollen sich legen deine
stolzen Wellen!? \bibverse{12} Hast du bei deiner Zeit dem Morgen
geboten und der Morgenröte ihren Ort gezeiget, \bibverse{13} daß die
Ecken der Erde gefasset und die Gottlosen herausgeschüttelt würden?
\bibverse{14} Das Siegel wird sich wandeln wie Leimen, und sie stehen
wie ein Kleid. \bibverse{15} Und den Gottlosen wird ihr Licht genommen
werden; und der Arm der Hoffärtigen wird zerbrochen werden.
\bibverse{16} Bist du in den Grund des Meers kommen und hast in den
Fußtapfen der Tiefen gewandelt? \bibverse{17} Haben sich dir des Todes
Tore je aufgetan? Oder hast du gesehen die Tore der Finsternis?
\bibverse{18} Hast du vernommen, wie breit die Erde sei? Sage an, weißt
du solches alles? \bibverse{19} Welches ist der Weg, da das Licht
wohnet, und welches sei der Finsternis Stätte, \bibverse{20} daß du
mögest abnehmen seine Grenze und merken den Pfad zu seinem Hause?
\bibverse{21} Wußtest du, daß du zu der Zeit solltest geboren werden und
wieviel deiner Tage sein würden? \bibverse{22} Bist du gewesen, da der
Schnee herkommt, oder hast du gesehen, wo der Hagel herkommt,
\bibverse{23} die ich habe verhalten bis auf die Zeit der Trübsal und
auf den Tag des Streits und Kriegs? \bibverse{24} Durch welchen Weg
teilet sich das Licht, und auffähret der Ostwind auf Erden?
\bibverse{25} Wer hat dem Platzregen seinen Lauf ausgeteilet und den Weg
dem Blitze und Donner, \bibverse{26} daß es regnet aufs Land, da niemand
ist, in der Wüste, da kein Mensch ist, \bibverse{27} daß er füllet die
Einöden und Wildnis und macht, daß Gras wächset? \bibverse{28} Wer ist
des Regens Vater? Wer hat die Tropfen des Taues gezeuget? \bibverse{29}
Aus wes Leibe ist das Eis gegangen? Und wer hat den Reif unter dem
Himmel gezeuget, \bibverse{30} daß das Wasser verborgen wird wie unter
Steinen und die Tiefe oben gestehet? \bibverse{31} Kannst du die Bande
der sieben Sterne zusammenbinden, oder das Band des Orion auflösen?
\bibverse{32} Kannst du den Morgenstern hervorbringen zu seiner Zeit,
oder den Wagen am Himmel über seine Kinder führen? \bibverse{33} Weißt
du, wie der Himmel zu regieren ist? Oder kannst du ihn meistern auf
Erden? \bibverse{34} Kannst du deinen Donner in der Wolke hoch
herführen? Oder wird dich die Menge des Wassers verdecken? \bibverse{35}
Kannst du die Blitze auslassen, daß sie hinfahren und sprechen: Hie sind
wir? \bibverse{36} Wer gibt die Weisheit ins Verborgene? Wer gibt
verständige Gedanken? \bibverse{37} Wer ist so weise, der die Wolken
erzählen könnte? Wer kann die Wasserschläuche am Himmel verstopfen,
\bibverse{38} wenn der Staub begossen wird, daß er zuhaufe läuft und die
Klöße aneinander kleben? \bibverse{39} Kannst du der Löwin ihren Raub zu
jagen geben und die jungen Löwen sättigen, \bibverse{40} daß sie sich
legen in ihre Stätte und ruhen in der Höhle, da sie lauern?
\bibverse{41} Wer bereitet dem Raben die Speise, wenn seine Jungen zu
GOtt rufen und fliegen irre, wenn sie nicht zu essen haben?

\hypertarget{section-38}{%
\section{39}\label{section-38}}

\bibverse{1} Weißt du die Zeit, wann die Gemsen auf den Felsen gebären?
Oder hast du gemerkt, wann die Hirsche schwanger gehen? \bibverse{2}
Hast du erzählet ihre Monden, wann sie voll werden? Oder weißt du die
Zeit, wann sie gebären? \bibverse{3} Sie beugen sich, wenn sie gebären,
und reißen sich und lassen aus ihre Jungen. \bibverse{4} Ihre Jungen
werden feist und mehren sich im Getreide; und gehen aus und kommen nicht
wieder zu ihnen. \bibverse{5} Wer hat das Wild so frei lassen gehen? Wer
hat die Bande des Wildes aufgelöset, \bibverse{6} dem ich das Feld zum
Hause gegeben habe und die Wüste zur Wohnung? \bibverse{7} Es verlacht
das Getümmel der Stadt; das Pochen des Treibers höret es nicht.
\bibverse{8} Es schauet nach den Bergen, da seine Weide ist, und suchet,
wo es grün ist. \bibverse{9} Meinest du, das Einhorn werde dir dienen
und werde bleiben an deiner Krippe? \bibverse{10} Kannst du ihm dein
Joch anknüpfen, die Furchen zu machen, daß es hinter dir brache in
Gründen? \bibverse{11} Magst du dich auf es verlassen, daß es so stark
ist, und wirst es dir lassen arbeiten? \bibverse{12} Magst du ihm
trauen, daß es deinen Samen dir wiederbringe und in deine Scheune
sammle? \bibverse{13} Die Federn des Pfauen sind schöner denn die Flügel
und Federn des Storchs, \bibverse{14} der seine Eier auf der Erde lässet
und läßt sie die heiße Erde ausbrüten. \bibverse{15} Er vergisset, daß
sie möchten zertreten werden und ein wild Tier sie zerbreche.
\bibverse{16} Er wird so hart gegen seine Jungen, als wären sie nicht
sein, achtet es nicht, daß er umsonst arbeitet. \bibverse{17} Denn GOtt
hat ihm die Weisheit genommen und hat ihm keinen Verstand mitgeteilet.
\bibverse{18} Zu der Zeit, wenn er hoch fähret, erhöhet er sich und
verlachet beide Roß und Mann. \bibverse{19} Kannst du dem Roß Kräfte
geben, oder seinen Hals zieren mit seinem Geschrei? \bibverse{20} Kannst
du es schrecken wie die Heuschrecken? Das ist Preis seiner Nase, was
schrecklich ist. \bibverse{21} Es stampfet auf den Boden und ist freudig
mit Kraft und zeucht aus den Geharnischten entgegen. \bibverse{22} Es
spottet der Furcht und erschrickt nicht und fleucht vor dem Schwert
nicht, \bibverse{23} wenngleich wider es klinget der Köcher und glänzet
beide Spieß und Lanze. \bibverse{24} Es zittert und tobet und scharret
in die Erde und achtet nicht der Trommeten Hall. \bibverse{25} Wenn die
Trommete fast klinget, spricht es: Hui! und riecht den Streit von ferne,
das Schreien der Fürsten und Jauchzen. \bibverse{26} Fleuget der Habicht
durch deinen Verstand und breitet seine Flügel gegen Mittag?
\bibverse{27} Fleuget der Adler auf deinen Befehl so hoch, daß er sein
Nest in der Höhe macht? \bibverse{28} In Felsen wohnet er und bleibt auf
den Klippen an Felsen und in festen Orten. \bibverse{29} Von dannen
schauet er nach der Speise, und seine Augen sehen ferne. \bibverse{30}
Seine Jungen saufen Blut; und wo ein Aas ist, da ist er.

\hypertarget{section-39}{%
\section{40}\label{section-39}}

\bibverse{1} Und der HErr antwortete Hiob und sprach: \bibverse{2} Wer
mit dem Allmächtigen hadern will, soll's ihm der nicht beibringen? Und
wer GOtt tadelt, soll's der nicht verantworten? \bibverse{3} Hiob aber
antwortete dem HErrn und sprach: \bibverse{4} Siehe, ich bin zu
leichtfertig gewesen, was soll ich antworten? Ich will meine Hand auf
meinen Mund legen. \bibverse{5} Ich habe einmal geredet, darum will ich
nicht mehr antworten; hernach will ich's nicht mehr tun. \bibverse{6}
Und der HErr antwortete Hiob aus einem Wetter und sprach: \bibverse{7}
Gürte wie ein Mann deine Lenden; ich will dich fragen, lehre mich!
\bibverse{8} Solltest du mein Urteil zunichte machen und mich verdammen,
daß du gerecht seiest? \bibverse{9} Hast du einen Arm wie GOtt und
kannst mit gleicher Stimme donnern, als er tut? \bibverse{10} Schmücke
dich mit Pracht und erhebe dich; zeuch dich löblich und herrlich an!
\bibverse{11} Streue aus den Zorn deines Grimms; schaue an die
Hochmütigen, wo sie sind, und demütige sie. \bibverse{12} Ja, schaue die
Hochmütigen, wo sie sind, und beuge sie und mache die Gottlosen dünne,
wo sie sind. \bibverse{13} Verscharre sie miteinander in der Erde und
versenke ihre Pracht ins Verborgene, \bibverse{14} so will ich dir auch
bekennen, daß dir deine rechte Hand helfen kann. \bibverse{15} Siehe,
der Behemoth, den ich neben dir gemacht habe, frißt Heu wie ein Ochse.
\bibverse{16} Siehe, seine Kraft ist in seinen Lenden und sein Vermögen
im Nabel seines Bauchs. \bibverse{17} Sein Schwanz strecket sich wie
eine Zeder, die Adern seiner Scham starren wie ein Ast. \bibverse{18}
Seine Knochen sind wie fest Erz, seine Gebeine sind wie eiserne Stäbe.
\bibverse{19} Er ist der Anfang der Wege GOttes; der ihn gemacht hat,
der greift ihn an mit seinem Schwert. \bibverse{20} Die Berge tragen ihm
Kräuter, und alle wilden Tiere spielen daselbst. \bibverse{21} Er liegt
gern im Schatten, im Rohr und im Schlamm verborgen. \bibverse{22} Das
Gebüsch bedeckt ihn mit seinem Schatten, und die Bachweiden bedecken
ihn. \bibverse{23} Siehe, er schluckt in sich den Strom und achtet es
nicht groß; läßt sich dünken, er wolle den Jordan mit seinem Munde
ausschöpfen. \bibverse{24} Noch fähet man ihn mit seinen eigenen Augen,
und durch Fallstricke durchbohret man ihm seine Nase.

\hypertarget{section-40}{%
\section{41}\label{section-40}}

\bibverse{1} Kannst du den Leviathan ziehen mit dem Hamen und seine
Zunge mit einem Strick fassen? \bibverse{2} Kannst du ihm eine Angel in
die Nase legen und mit einem Stachel ihm die Backen durchbohren?
\bibverse{3} Meinest du, er werde dir viel Flehens machen oder dir
heucheln? \bibverse{4} Meinest du, daß er einen Bund mit dir machen
werde, daß du ihn immer zum Knecht habest? \bibverse{5} Kannst du mit
ihm spielen wie mit einem Vogel, oder ihn deinen Dirnen binden?
\bibverse{6} Meinest du, die Gesellschaften werden ihn zerschneiden, daß
er unter die Kaufleute zerteilet wird? \bibverse{7} Kannst du das Netz
füllen mit seiner Haut und die Fischreusen mit seinem Kopf? \bibverse{8}
Wenn du deine Hand an ihn legst, so gedenke, daß ein Streit sei, den du
nicht ausführen wirst. \bibverse{9} Siehe, seine Hoffnung wird ihm
fehlen; und wenn er sein ansichtig wird, schwinget er sich dahin.
\bibverse{10} Niemand ist so kühn, der ihn reizen darf; wer ist denn,
der vor mir stehen könne? \bibverse{11} Wer hat mir was zuvor getan, daß
ich's ihm vergelte? Es ist mein, was unter allen Himmeln ist.
\bibverse{12} Dazu muß ich nun sagen, wie groß, wie mächtig und wohl
geschaffen er ist. \bibverse{13} Wer kann ihm sein Kleid aufdecken? Und
wer darf es wagen, ihm zwischen die Zähne zu greifen? \bibverse{14} Wer
kann die Kinnbacken seines Antlitzes auftun? Schrecklich stehen seine
Zähne umher. \bibverse{15} Seine stolzen Schuppen sind wie feste
Schilde, fest und enge ineinander. \bibverse{16} Eine rührt an die
andere, daß nicht ein Lüftlein dazwischengehet. \bibverse{17} Es hängt
eine an der andern, und halten sich zusammen, daß sie sich nicht
voneinander trennen. \bibverse{18} Sein Niesen glänzet wie ein Licht;
seine Augen sind wie die Augenlider der Morgenröte. \bibverse{19} Aus
seinem Munde fahren Fackeln, und feurige Funken schießen heraus.
\bibverse{20} Aus seiner Nase gehet Rauch wie von heißen Töpfen und
Kessel. \bibverse{21} Sein Odem ist wie lichte Lohe, und aus seinem
Munde gehen Flammen. \bibverse{22} Er hat einen starken Hals; und ist
seine Lust, wo er etwas verderbet. \bibverse{23} Die Gliedmaßen seines
Fleisches hangen aneinander und halten hart an ihm, daß er nicht
zerfallen kann. \bibverse{24} Sein Herz ist so hart wie ein Stein und so
fest wie ein Stück vom untersten Mühlstein. \bibverse{25} Wenn er sich
erhebt, so entsetzen sich die Starken; und wenn er daherbricht, so ist
keine Gnade da. \bibverse{26} Wenn man zu ihm will mit dem Schwert, so
regt er sich nicht; oder mit Spieß, Geschoß und Panzer. \bibverse{27} Er
achtet Eisen wie Stroh und Erz wie faul Holz. \bibverse{28} Kein Pfeil
wird ihn verjagen; die Schleudersteine sind wie Stoppeln. \bibverse{29}
Den Hammer achtet er wie Stoppeln; er spottet der bebenden Lanze.
\bibverse{30} Unter ihm liegen scharfe Steine und fährt über die
scharfen Felsen wie über Kot. \bibverse{31} Er macht, daß das tiefe Meer
siedet wie ein Topf, und rührt es ineinander, wie man eine Salbe menget.
\bibverse{32} Nach ihm leuchtet der Weg, er macht die Tiefe ganz grau.
\bibverse{33} Auf Erden ist ihm niemand zu gleichen; er ist gemacht ohne
Furcht zu sein. \bibverse{34} Er verachtet alles, was hoch ist; er ist
ein König über alle Stolzen.

\hypertarget{section-41}{%
\section{42}\label{section-41}}

\bibverse{1} Und Hiob antwortete dem HErrn und sprach: \bibverse{2} ich
erkenne, daß du alles vermagst, und kein Gedanke ist dir verborgen.
\bibverse{3} Es ist ein unbesonnener Mann, der seinen Rat meinet zu
verbergen. Darum bekenne ich, daß ich habe unweislich geredet, das mir
zu hoch ist und nicht verstehe. \bibverse{4} So erhöre nun, laß mich
reden; ich will dich fragen, lehre mich! \bibverse{5} Ich habe dich mit
den Ohren gehöret, und mein Auge siehet dich auch nun. \bibverse{6}
Darum schuldige ich mich und tue Buße in Staub und Asche. \bibverse{7}
Da nun der HErr diese Worte mit Hiob geredet hatte; sprach er zu Eliphas
von Theman: Mein Zorn ist ergrimmet über dich und über deine zween
Freunde; denn ihr habt nicht recht von mir geredet wie mein Knecht Hiob.
\bibverse{8} So nehmet nun sieben Farren und sieben Widder und gehet hin
zu meinem Knechte Hiob und opfert Brandopfer für euch und laßt meinen
Knecht Hiob für euch bitten. Denn ihn will ich ansehen, daß ich euch
nicht sehen lasse, wie ihr Torheit begangen habt; denn ihr habt nicht
recht von mir geredet wie mein Knecht Hiob. \bibverse{9} Da gingen hin
Eliphas von Theman, Bildad von Suah und Zophar von Naema und taten, wie
der HErr ihnen gesagt hatte. Und der HErr sah an Hiob. \bibverse{10} Und
der HErr wendete das Gefängnis Hiobs, da er bat für seine Freunde. Und
der HErr gab Hiob zwiefältig so viel, als er gehabt hatte. \bibverse{11}
Und es kamen zu ihm alle seine Brüder und alle seine Schwestern und
alle, die ihn vorhin kannten, und aßen mit ihm in seinem Hause und
kehreten sich zu ihm und trösteten ihn über allem Übel, das der HErr
über ihn hatte kommen lassen. Und ein jeglicher gab ihm einen schönen
Groschen und ein gülden Stirnband. \bibverse{12} Und der HErr segnete
hernach Hiob mehr denn vorhin, daß er kriegte vierzehntausend Schafe und
sechstausend Kamele und tausend Joch Rinder und tausend Esel.
\bibverse{13} Und kriegte sieben Söhne und drei Töchter. \bibverse{14}
Und hieß die erste Jemima, die andere Kezia und die dritte Keren-Hapuch.
\bibverse{15} Und wurden nicht so schöne Weiber funden in allen Landen
als die Töchter Hiobs. Und ihr Vater gab ihnen Erbteil unter ihren
Brüdern. \bibverse{16} Und Hiob lebte nach diesem hundertundvierzig
Jahre, daß er sah Kinder und Kindeskinder bis in das vierte Glied.
\bibverse{17} Und Hiob starb alt und lebenssatt.
