\bibverse{1} Nachdem vorzeiten GOtt manchmal und mancherlei Weise
geredet hat zu den Vätern durch die Propheten, \bibverse{2} hat er am
letzten in diesen Tagen zu uns geredet durch den Sohn, welchen er
gesetzt hat zum Erben über alles, durch welchen er auch die Welt gemacht
hat; \bibverse{3} welcher, sintemal er ist der Glanz seiner Herrlichkeit
und das Ebenbild seines Wesens und trägt alle Dinge mit seinem kräftigen
Wort und hat gemacht die Reinigung unserer Sünden durch sich selbst, hat
er sich gesetzt zu der Rechten der Majestät in der Höhe, \bibverse{4} so
viel besser worden denn die Engel, so gar viel einen höhern Namen er vor
ihnen ererbet hat. \bibverse{5} Denn zu welchem Engel hat er jemals
gesagt: Du bist mein Sohn, heute habe ich dich gezeuget? Und abermal:
Ich werde sein Vater sein, und er wird mein Sohn sein? \bibverse{6} Und
abermal, da er einführet den Erstgeborenen in die Welt, spricht er: Und
es sollen ihn alle Gottesengel anbeten. \bibverse{7} Von den Engeln
spricht er zwar: Er macht seine Engel Geister und seine Diener
Feuerflammen; \bibverse{8} aber von dem Sohn: GOtt, dein Stuhl währet
von Ewigkeit zu Ewigkeit; das Zepter deines Reichs ist ein richtiges
Zepter. \bibverse{9} Du hast geliebet die Gerechtigkeit und gehasset die
Ungerechtigkeit; darum hat dich, o GOtt, gesalbet dein GOtt mit dem Öle
der Freuden über deine Genossen; \bibverse{10} und: Du, HErr, hast von
Anfang die Erde gegründet, und die Himmel sind deiner Hände Werk.
\bibverse{11} Dieselbigen werden vergehen, du aber wirst bleiben; und
sie werden alle veralten wie ein Kleid, \bibverse{12} und wie ein Gewand
wirst du sie wandeln, und sie werden sich verwandeln. Du aber bist
derselbige, und deine Jahre werden nicht aufhören. \bibverse{13} Zu
welchem Engel aber hat er jemals gesagt: Setze dich zu meiner Rechten,
bis ich lege deine Feinde zum Schemel deiner Füße? \bibverse{14} Sind
sie nicht allzumal dienstbare Geister, ausgesandt zum Dienst um derer
willen, die ererben sollen die Seligkeit?

\hypertarget{section}{%
\section{2}\label{section}}

\bibverse{1} Darum sollen wir desto mehr wahrnehmen des Worts, das wir
hören, daß wir nicht dahinfahren. \bibverse{2} Denn so das Wort fest
worden ist, das durch die Engel geredet ist, und eine jegliche
Übertretung und Ungehorsam hat empfangen seinen rechten Lohn:
\bibverse{3} wie wollen wir entfliehen, so wir eine solche Seligkeit
nicht achten? welche, nachdem sie erstlich geprediget ist durch den
HErrn, ist sie auf uns kommen durch die, so es gehöret haben.
\bibverse{4} Und GOtt hat ihr Zeugnis gegeben mit Zeichen, Wundern und
mancherlei Kräften und mit Austeilung des Heiligen Geistes nach seinem
Willen. \bibverse{5} Denn er hat nicht den Engeln untertan die
zukünftige Welt, davon wir reden. \bibverse{6} Es bezeuget aber einer an
einem Ort und spricht: Was ist der Mensch, daß du sein gedenkest, und
des Menschen Sohn, daß du ihn heimsuchest? \bibverse{7} Du hast ihn eine
kleine Zeit der Engel mangeln lassen; mit Preis und Ehren hast du ihn
gekrönet und hast ihn gesetzt über die Werke deiner Hände; \bibverse{8}
alles hast du untertan zu seinen Füßen. In dem, daß er ihm alles hat
untertan, hat er nichts gelassen, das ihm nicht untertan sei; jetzt aber
sehen wir noch nicht, daß ihm alles untertan sei. \bibverse{9} Den aber,
der eine kleine Zeit der Engel gemangelt hat, sehen wir, daß es JEsus
ist, durch Leiden des Todes gekrönet mit Preis und Ehren, auf daß er von
GOttes Gnaden für alle den Tod schmeckete. \bibverse{10} Denn es ziemete
dem, um deswillen alle Dinge sind, und durch den alle Dinge sind, der da
viel Kinder hat zur Herrlichkeit geführet, daß er den Herzog ihrer
Seligkeit durch Leiden vollkommen machte. \bibverse{11} Sintemal sie
alle von einem kommen, beide, der da heiliget, und die da geheiliget
werden. Darum schämet er sich auch nicht, sie Brüder zu heißen,
\bibverse{12} und spricht: Ich will verkündigen deinen Namen meinen
Brüdern und mitten in der Gemeinde dir Lob singen. \bibverse{13} Und
abermal: Ich will mein Vertrauen auf ihn setzen. Und abermal: Siehe da,
ich und die Kinder, welche mir GOtt gegeben hat. \bibverse{14} Nachdem
nun die Kinder Fleisch und Blut haben, ist er's gleichermaßen teilhaftig
worden, auf daß er durch den Tod die Macht nähme dem, der des Todes
Gewalt hatte, das ist, dem Teufel, \bibverse{15} und erlösete die, so
durch Furcht des Todes im ganzen Leben Knechte sein mußten.
\bibverse{16} Denn er nimmt nirgend die Engel an sich, sondern den Samen
Abrahams nimmt er an sich. \bibverse{17} Daher mußte er allerdinge
seinen Brüdern gleich werden, auf daß er barmherzig würde und ein treuer
Hoherpriester vor GOtt, zu versöhnen die Sünde des Volks. \bibverse{18}
Denn darinnen er gelitten hat und versucht ist, kann er helfen denen,
die versucht werden.

\hypertarget{section-1}{%
\section{3}\label{section-1}}

\bibverse{1} Derhalben, ihr heiligen Brüder, die ihr mit berufen seid
durch die himmlische Berufung, nehmet wahr des Apostels und
Hohenpriesters, den wir bekennen, Christi JEsu, \bibverse{2} der da treu
ist dem, der ihn gemacht hat (wie auch Mose) in seinem ganzen Hause.
\bibverse{3} Dieser aber ist größerer Ehre wert denn Mose, nachdem der
eine größere Ehre am Hause hat, der es bereitet, denn das Haus.
\bibverse{4} Denn ein jeglich Haus wird von jemand bereitet; der aber
alles bereitet, das ist GOtt. \bibverse{5} Und Mose zwar war treu in
seinem ganzen Hause als ein Knecht zum Zeugnis des, das gesagt sollte
werden; \bibverse{6} Christus aber als ein Sohn über sein Haus; welches
Haus sind wir, so wir anders das Vertrauen und den Ruhm der Hoffnung bis
ans Ende fest behalten. \bibverse{7} Darum, wie der Heilige Geist
spricht: Heute, so ihr hören werdet seine Stimme, \bibverse{8} so
verstocket eure Herzen nicht, als geschah in der Verbitterung, am Tage
der Versuchung in der Wüste, \bibverse{9} da mich eure Väter versuchten;
sie prüften mich und sahen meine Werke vierzig Jahre lang; \bibverse{10}
darum ich entrüstet ward über dies Geschlecht und sprach: Immerdar irren
sie mit dem Herzen, aber sie wußten meine Wege nicht, \bibverse{11} daß
ich auch schwur in meinem Zorn, sie sollten zu meiner Ruhe nicht kommen.
\bibverse{12} Sehet zu, liebe Brüder, daß nicht jemand unter euch ein
arges, ungläubiges Herz habe, das da abtrete von dem lebendigen GOtt,
\bibverse{13} sondern ermahnet euch selbst alle Tage, solange es heute
heißt, daß nicht jemand unter euch verstocket werde durch Betrug der
Sünde. \bibverse{14} Denn wir sind Christi teilhaftig worden, so wir
anders das angefangene Wesen bis ans Ende fest behalten, \bibverse{15}
solange gesagt wird: Heute, so ihr seine Stimme hören werdet, so
verstocket eure Herzen nicht, wie in der Verbitterung geschah.
\bibverse{16} Denn etliche, da sie höreten, richteten eine Verbitterung
an, aber nicht alle, die von Ägypten ausgingen durch Mose. \bibverse{17}
Über welche aber ward er entrüstet vierzig Jahre lang? Ist's nicht also,
daß über die, so da sündigten, deren Leiber in der Wüste verfielen?
\bibverse{18} Welchen schwur er aber, daß sie nicht zu seiner Ruhe
kommen sollten, denn den Ungläubigen? \bibverse{19} Und wir sehen, daß
sie nicht haben können hineinkommen um des Unglaubens willen.

\hypertarget{section-2}{%
\section{4}\label{section-2}}

\bibverse{1} So lasset uns nun fürchten, daß wir die Verheißung,
einzukommen zu seiner Ruhe, nicht versäumen, und unser keiner
dahintenbleibe. \bibverse{2} Denn es ist uns auch verkündiget gleichwie
jenen; aber das Wort der Predigt half jenen nichts, da nicht glaubeten
die, so es höreten. \bibverse{3} Denn wir, die wir glauben, gehen in die
Ruhe, wie er spricht: Daß ich schwur in meinem Zorn, sie sollten zu
meiner Ruhe nicht kommen. Und zwar, da die Werke von Anbeginn der Welt
waren gemacht, \bibverse{4} sprach er an einem Ort von dem siebenten
Tage also: Und GOtt ruhete am siebenten Tage von allen seinen Werken.
\bibverse{5} Und hier an diesem Ort abermal: Sie sollen nicht kommen zu
meiner Ruhe. \bibverse{6} Nachdem es nun noch vorhanden ist, daß etliche
sollen zu derselbigen kommen, und die, denen es zuerst verkündiget ist,
sind nicht dazu kommen um des Unglaubens willen, \bibverse{7} bestimmte
er abermal einen Tag nach solcher langen Zeit und sagte durch David:
Heute, wie gesagt ist, heute, so ihr seine Stimme hören werdet, so
verstocket eure Herzen nicht! \bibverse{8} Denn so Josua sie hätte zur
Ruhe gebracht, würde er nicht hernach von einem andern Tage gesagt
haben. \bibverse{9} Darum ist noch eine Ruhe vorhanden dem Volk GOttes.
\bibverse{10} Denn wer zu seiner Ruhe kommen ist, der ruhet auch von
seinen Werken, gleich wie GOtt von seinen. \bibverse{11} So lasset uns
nun Fleiß tun, einzukommen zu dieser Ruhe, auf daß nicht jemand falle in
dasselbige Exempel des Unglaubens. \bibverse{12} Denn das Wort GOttes
ist lebendig und kräftig und schärfer denn kein zweischneidig Schwert
und durchdringet, bis daß es scheidet Seele und Geist, auch Mark und
Bein, und ist ein Richter der Gedanken und Sinne des Herzens.
\bibverse{13} Und ist keine Kreatur vor ihm unsichtbar; es ist aber
alles bloß und entdeckt vor seinen Augen; von dem reden wir.
\bibverse{14} Dieweil wir denn einen großen Hohenpriester haben, JEsum,
den Sohn GOttes, der gen Himmel gefahren ist, so lasset uns halten an
dem Bekenntnis. \bibverse{15} Denn wir haben nicht einen Hohenpriester,
der nicht könnte Mitleid haben mit unserer Schwachheit, sondern der
versucht ist allenthalben gleich wie wir, doch ohne Sünde. \bibverse{16}
Darum lasset uns hinzutreten mit Freudigkeit zu dem Gnadenstuhl, auf daß
wir Barmherzigkeit empfangen und Gnade finden auf die Zeit, wenn uns
Hilfe not sein wird.

\hypertarget{section-3}{%
\section{5}\label{section-3}}

\bibverse{1} Denn ein jeglicher Hoherpriester, der aus den Menschen
genommen wird, der wird gesetzt für die Menschen gegen GOtt, auf daß er
opfere Gaben und Opfer für die Sünden, \bibverse{2} der da könnte
mitleiden über die, so unwissend sind und irren, nachdem er auch selbst
umgeben ist mit Schwachheit. \bibverse{3} Darum muß er auch, gleichwie
für das Volk, also auch für sich selbst opfern für die Sünden.
\bibverse{4} Und niemand nimmt sich selbst die Ehre, sondern der auch
berufen sei von GOtt gleichwie Aaron. \bibverse{5} Also auch Christus
hat sich nicht selbst in die Ehre gesetzt, daß er Hoherpriester würde,
sondern der zu ihm gesagt hat: Du bist mein Sohn; heute habe ich dich
gezeuget. \bibverse{6} Wie er auch am andern Ort spricht: Du bist ein
Priester in Ewigkeit nach der Ordnung Melchisedeks. \bibverse{7} Und er
hat in den Tagen seines Fleisches Gebet und Flehen mit starkem Geschrei
und Tränen geopfert zu dem, der ihm von dem Tode konnte aushelfen; und
ist auch erhöret, darum daß er GOtt in Ehren hatte. \bibverse{8} Und
wiewohl er GOttes Sohn war, hat er doch an dem, was er litt, Gehorsam
gelernet. \bibverse{9} Und da er ist vollendet, ist er worden allen, die
ihm gehorsam sind, eine Ursache zur ewigen Seligkeit, \bibverse{10}
genannt von GOtt ein Hoherpriester nach der Ordnung Melchisedeks.
\bibverse{11} Davon hätten wir wohl viel zu reden; aber es ist schwer,
weil ihr so unverständig seid. \bibverse{12} Und die ihr solltet längst
Meister sein, bedürfet ihr wiederum, daß man euch die ersten Buchstaben
der göttlichen Worte lehre, und daß man euch Milch gebe und nicht starke
Speise. \bibverse{13} Denn wem man noch Milch geben muß, der ist
unerfahren in dem Wort der Gerechtigkeit; denn er ist ein junges Kind.
\bibverse{14} Den Vollkommenen aber gehört starke Speise, die durch
Gewohnheit haben geübte Sinne zum Unterschied des Guten und des Bösen.

\hypertarget{section-4}{%
\section{6}\label{section-4}}

\bibverse{1} Darum wollen wir die Lehre vom Anfang christliches Lebens
jetzt lassen und zur Vollkommenheit fahren, nicht abermal Grund legen
von Buße der toten Werke, vom Glauben an GOtt, \bibverse{2} von der
Taufe, von der Lehre, vom Händeauflegen, von der Toten Auferstehung und
vom ewigen Gerichte. \bibverse{3} Und das wollen wir tun, so es GOtt
anders zulässet. \bibverse{4} Denn es ist unmöglich, daß die, so einmal
erleuchtet sind und geschmeckt haben die himmlische Gabe und teilhaftig
worden sind des Heiligen Geistes \bibverse{5} und geschmeckt haben das
gütige Wort GOttes und die Kräfte der zukünftigen Welt, \bibverse{6} wo
sie abfallen und wiederum sich selbst den Sohn GOttes kreuzigen und für
Spott halten, daß sie sollten wiederum erneuert werden zur Buße.
\bibverse{7} Denn die Erde, die den Regen trinkt, der oft über sie
kommt, und bequem Kraut träget denen, die sie bauen, empfänget Segen von
GOtt. \bibverse{8} Welche aber Dornen und Disteln träget, die ist
untüchtig und dem Fluch nahe, welche man zuletzt verbrennet.
\bibverse{9} Wir versehen uns aber, ihr Liebsten, Besseres zu euch, und
daß die Seligkeit näher sei, ob wir wohl also reden. \bibverse{10} Denn
GOtt ist nicht ungerecht, daß er vergesse eures Werks und Arbeit der
Liebe, die ihr beweiset habt an seinem Namen, da ihr den Heiligen
dientet und noch dienet. \bibverse{11} Wir begehren aber, daß euer
jeglicher denselbigen Fleiß beweise, die Hoffnung festzuhalten bis ans
Ende, \bibverse{12} daß ihr nicht träge werdet, sondern Nachfolger
derer, die durch den Glauben und Geduld ererben die Verheißungen.
\bibverse{13} Denn als GOtt Abraham verhieß, da er bei keinem Größeren
zu schwören hatte, schwur er bei sich selbst \bibverse{14} und sprach:
Wahrlich, ich will dich segnen und vermehren. \bibverse{15} Und also
trug er Geduld und erlangte die Verheißung. \bibverse{16} Die Menschen
schwören wohl bei einem Größeren, denn sie sind; und der Eid macht ein
Ende alles Haders, dabei es fest bleibt unter ihnen. \bibverse{17} Aber
GOtt, da er wollte den Erben der Verheißung überschwenglich beweisen,
daß sein Rat nicht wankete, hat er einen Eid dazugetan, \bibverse{18}
auf daß wir durch zwei Stücke, die nicht wanken (denn es ist unmöglich,
daß GOtt lüge), einen starken Trost haben, die wir Zuflucht haben und
halten an der angebotenen Hoffnung, \bibverse{19} welche wir haben als
einen sicheren und festen Anker unserer Seele, der auch hineingehet in
das Inwendige des Vorhangs, \bibverse{20} dahin der Vorläufer für uns
eingegangen, JEsus, ein Hoherpriester worden in Ewigkeit nach der
Ordnung Melchisedeks.

\hypertarget{section-5}{%
\section{7}\label{section-5}}

\bibverse{1} Dieser Melchisedek aber war ein König zu Salem, ein
Priester GOttes, des Allerhöchsten, der Abraham entgegenging, da er von
der Könige Schlacht wiederkam, und segnete ihn, \bibverse{2} welchem
auch Abraham gab den Zehnten aller Güter. Aufs erste wird er
verdolmetscht ein König der Gerechtigkeit; danach aber ist er auch ein
König Salem, das ist, ein König des Friedens; \bibverse{3} ohne Vater,
ohne Mutter, ohne Geschlecht; und hat weder Anfang der Tage noch Ende
des Lebens. Er ist aber verglichen dem Sohn GOttes und bleibet Priester
in Ewigkeit. \bibverse{4} Schauet aber, wie groß ist der, dem auch
Abraham, der Patriarch, den Zehnten gibt von der eroberten Beute!
\bibverse{5} Zwar die Kinder Levi, da sie das Priestertum empfangen,
haben sie ein Gebot, den Zehnten vom Volk, das ist, von ihren Brüdern,
zu nehmen nach dem Gesetz, wiewohl auch dieselben aus den Lenden
Abrahams kommen sind. \bibverse{6} Aber der, des Geschlecht nicht
genannt wird unter ihnen, der nahm den Zehnten von Abraham und segnete
den, der die Verheißung hatte. \bibverse{7} Nun ist's ohne alles
Widersprechen also, daß das Geringere von dem Besseren gesegnet wird.
\bibverse{8} Und hier nehmen den Zehnten die sterbenden Menschen; aber
dort bezeuget er, daß er lebe. \bibverse{9} Und daß ich also sage, es
ist auch Levi, der den Zehnten nimmt, verzehntet durch Abraham.
\bibverse{10} Denn er war je noch in den Lenden des Vaters, da ihm
Melchisedek entgegenging. \bibverse{11} Ist nun die Vollkommenheit durch
das levitische Priestertum geschehen (denn unter demselbigen hat das
Volk das Gesetz empfangen), was ist denn weiter not zu sagen, daß ein
anderer Priester aufkommen solle nach der Ordnung Melchisedeks und nicht
nach der Ordnung Aarons? \bibverse{12} Denn wo das Priestertum verändert
wird, da muß auch das Gesetz verändert werden. \bibverse{13} Denn von
dem solches gesagt ist, der ist von einem andern Geschlecht, aus welchem
nie keiner des Altars gepfleget hat. \bibverse{14} Denn es ist ja
offenbar, daß von Juda aufgegangen ist unser HErr; zu welchem Geschlecht
Mose nicht geredet hat vom Priestertum. \bibverse{15} Und es ist noch
viel klarer, so nach der Weise Melchisedeks ein anderer Priester
aufkommt, \bibverse{16} welcher nicht nach dem Gesetz des fleischlichen
Gebots gemacht ist, sondern nach der Kraft des unendlichen Lebens.
\bibverse{17} Denn er bezeuget: Du bist ein Priester ewiglich nach der
Ordnung Melchisedeks. \bibverse{18} Denn damit wird das vorige Gesetz
aufgehoben, darum daß es zu schwach und nicht nütze war \bibverse{19}
(denn das Gesetz konnte nichts vollkommen machen), und wird eingeführet
eine bessere Hoffnung, durch welche wir zu GOtt nahen; \bibverse{20} und
dazu, das viel ist, nicht ohne Eid. Denn jene sind ohne Eid Priester
worden; \bibverse{21} dieser aber mit dem Eid durch den, der zu ihm
spricht: Der HErr hat geschworen, und wird ihn nicht gereuen: Du bist
ein Priester in Ewigkeit nach der Ordnung Melchisedeks. \bibverse{22}
Also eines so viel besseren Testaments Ausrichter ist JEsus worden.
\bibverse{23} Und jener sind viel, die Priester wurden, darum daß sie
der Tod nicht bleiben ließ; \bibverse{24} dieser aber darum, daß er
bleibet ewiglich, hat er ein unvergänglich Priestertum; \bibverse{25}
daher er auch selig machen kann immerdar, die durch ihn zu GOtt kommen,
und lebet immerdar und bittet für sie. \bibverse{26} Denn einen solchen
Hohenpriester sollten wir haben, der da wäre heilig, unschuldig,
unbefleckt, von den Sündern abgesondert und höher, denn der Himmel ist,
\bibverse{27} dem nicht täglich not wäre wie jenen Hohenpriestern,
zuerst für eigene Sünden Opfer zu tun, danach für des Volks Sünden; denn
das hat er getan einmal, da er sich selbst opferte. \bibverse{28} Denn
das Gesetz macht Menschen zu Hohenpriestern, die da Schwachheit haben;
dies Wort aber des Eides, das nach dem Gesetz gesagt ist, setzet den
Sohn ewig und vollkommen.

\hypertarget{section-6}{%
\section{8}\label{section-6}}

\bibverse{1} Das ist nun die Summa, davon wir reden: Wir haben einen
solchen Hohenpriester, der da sitzet zu der Rechten auf dem Stuhl der
Majestät im Himmel; \bibverse{2} und ist ein Pfleger der heiligen Güter
und der wahrhaftigen Hütte, welche GOtt aufgerichtet hat und kein
Mensch. \bibverse{3} Denn ein jeglicher Hoherpriester wird eingesetzt,
zu opfern Gaben und Opfer. Darum muß auch dieser etwas haben, das er
opfere. \bibverse{4} Wenn er nun auf Erden wäre, so wäre er nicht
Priester, dieweil da Priester sind, die nach dem Gesetz die Gaben
opfern, \bibverse{5} welche dienen dem Vorbilde und dem Schatten der
himmlischen Güter; wie die göttliche Antwort zu Mose sprach, da er
sollte die Hütte vollenden: Schaue zu, sprach er, daß du machest alles
nach dem Bilde, das dir auf dem Berge gezeiget ist. \bibverse{6} Nun
aber hat er ein besser Amt erlanget, als der eines besseren Testaments
Mittler ist, welches auch auf besseren Verheißungen stehet. \bibverse{7}
Denn so jenes, das erste, untadelig gewesen wäre, würde nicht Raum zu
einem andern gesucht. \bibverse{8} Denn er tadelt sie und sagt: Siehe,
es kommen die Tage, spricht der HErr, daß ich über das Haus Israel und
über das Haus Juda ein neu Testament machen will; \bibverse{9} nicht
nach dem Testament, das ich gemacht habe mit ihren Vätern an dem Tage,
da ich ihre Hand ergriff, sie auszuführen aus Ägyptenland. Denn sie sind
nicht geblieben in meinem Testament; so habe ich ihrer auch nicht wollen
achten, spricht der HErr. \bibverse{10} Denn das ist das Testament, das
ich machen will dem Hause Israel nach diesen Tagen, spricht der HErr:
Ich will geben meine Gesetze in ihren Sinn, und in ihr Herz will ich sie
schreiben, und will ihr GOtt sein, und sie sollen mein Volk sein.
\bibverse{11} Und soll nicht lehren jemand seinen Nächsten noch jemand
seinen Bruder und sagen: Erkenne den HErrn! Denn sie sollen mich alle
kennen, von dem Kleinsten an bis zu dem Größten. \bibverse{12} Denn ich
will gnädig sein ihrer Untugend und ihren Sünden, und ihrer
Ungerechtigkeit will ich nicht mehr gedenken. \bibverse{13} Indem er
sagt: Ein neues, macht er das erste alt. Was aber alt und überjahret
ist, das ist nahe bei seinem Ende.

\hypertarget{section-7}{%
\section{9}\label{section-7}}

\bibverse{1} Es hatte zwar auch das erste seine Rechte des
Gottesdienstes und äußerliche Heiligkeit. \bibverse{2} Denn es war da
aufgerichtet das Vorderteil der Hütte, darinnen war der Leuchter und der
Tisch und die Schaubrote; und diese heißt das Heilige. \bibverse{3}
Hinter dem andern Vorhang aber war die Hütte, die da heißt das
Allerheiligste. \bibverse{4} Die hatte das güldene Rauchfaß und die Lade
des Testaments, allenthalben mit Gold überzogen, in welcher war die
güldene Gelte, die das Himmelbrot hatte, und die Rute Aarons, die
gegrünet hatte, und die Tafeln des Testaments. \bibverse{5} Oben drüber
aber waren die Cherubim der Herrlichkeit, die überschatteten den
Gnadenstuhl; von welchem jetzt nicht zu sagen ist insonderheit.
\bibverse{6} Da nun solches also zugerichtet war gingen die Priester
allezeit in die vorderste Hütte und richteten aus den Gottesdienst.
\bibverse{7} In die andere aber ging nur einmal im Jahr allein der
Hohepriester, nicht ohne Blut, daß er opferte für sein selbst und des
Volks Unwissenheit. \bibverse{8} Damit der Heilige Geist deutete, daß
noch nicht offenbart wäre der Weg zur Heiligkeit, solange die erste
Hütte stünde, \bibverse{9} welche mußte zu derselbigen Zeit ein Vorbild
sein, in welcher Gaben und Opfer geopfert wurden, und konnten nicht
vollkommen machen nach dem Gewissen den, der da Gottesdienst tut
\bibverse{10} allein mit Speise und Trank und mancherlei Taufen und
äußerlicher Heiligkeit, die bis auf die Zeit der Besserung sind
aufgelegt. \bibverse{11} Christus aber ist kommen, daß er sei ein
Hoherpriester der zukünftigen Güter, durch eine größere und
vollkommenere Hütte, die nicht mit der Hand gemacht ist, das ist, die
nicht also gebauet ist; \bibverse{12} Auch nicht durch der Böcke oder
Kälber Blut, sondern er ist durch sein eigen Blut einmal in das Heilige
eingegangen und hat eine ewige Erlösung erfunden. \bibverse{13} Denn so
der Ochsen und der Böcke Blut und die Asche, von der Kuh gesprenget,
heiliget die Unreinen zu der leiblichen Reinigkeit, \bibverse{14}
wieviel mehr wird das Blut Christi, der sich selbst ohne allen Wandel
durch den Heiligen Geist GOtt geopfert hat, unser Gewissen reinigen von
den toten Werken, zu dienen dem lebendigen GOtt! \bibverse{15} Und darum
ist er auch ein Mittler des Neuen Testaments, auf daß durch den Tod, so
geschehen ist zur Erlösung von den Übertretungen, die unter dem ersten
Testament waren, die, so berufen sind, das verheißene ewige Erbe
empfangen. \bibverse{16} Denn wo ein Testament ist, da muß der Tod
geschehen des, der das Testament machte. \bibverse{17} Denn ein
Testament wird fest durch den Tod, anders hat es noch nicht Macht, wenn
der noch lebet, der es gemacht hat. \bibverse{18} Daher auch das erste
nicht ohne Blut gestiftet ward. \bibverse{19} Denn als Mose ausgeredet
hatte von allen Geboten nach dem Gesetz zu allem Volk, nahm er Kälber -
und Bocksblut mit Wasser und Purpurwolle und Ysop und besprengete das
Buch und alles Volk. \bibverse{20} Und sprach: Das ist das Blut des
Testaments, das GOtt euch geboten hat. \bibverse{21} Und die Hütte und
alles Geräte des Gottesdienstes besprengete er desselbigengleichen mit
Blut. \bibverse{22} Und wird fast alles mit Blut gereiniget nach dem
Gesetz. Und ohne Blutvergießen geschieht keine Vergebung. \bibverse{23}
So mußten nun der himmlischen Dinge Vorbilder mit solchem gereiniget
werden; aber sie selbst, die himmlischen, müssen bessere Opfer haben,
denn jene waren. \bibverse{24} Denn Christus ist nicht eingegangen in
das Heilige, so mit Händen gemacht ist (welches ist ein Gegenbild des
rechtschaffenen), sondern in den Himmel selbst, nun zu erscheinen vor
dem Angesichte GOttes für uns. \bibverse{25} Auch nicht, daß er sich
oftmals opfere, gleichwie der Hohepriester gehet alle Jahr in das
Heilige mit fremdem Blut. \bibverse{26} Sonst hätte er oft müssen leiden
von Anfang der Welt her. Nun aber am Ende der Welt ist er einmal
erschienen, durch sein eigen Opfer die Sünde aufzuheben. \bibverse{27}
Und wie den Menschen ist gesetzt, einmal zu sterben, danach aber das
Gericht, \bibverse{28} also ist Christus einmal geopfert, wegzunehmen
vieler Sünden. Zum andernmal aber wird er ohne Sünde erscheinen denen,
die auf ihn warten, zur Seligkeit.

\hypertarget{section-8}{%
\section{10}\label{section-8}}

\bibverse{1} Denn das Gesetz hat den Schatten von den zukünftigen
Gütern, nicht das Wesen der Güter selbst. Alle Jahr muß man opfern immer
einerlei Opfer und kann nicht, die da opfern, vollkommen machen;
\bibverse{2} sonst hätte das Opfern aufgehöret, wo die, so am
Gottesdienst sind, kein Gewissen mehr hätten von den Sünden, wenn sie'
einmal gereiniget wären; \bibverse{3} sondern es geschieht nur durch
dieselbigen ein Gedächtnis der Sünden alle Jahr. \bibverse{4} Denn es
ist unmöglich, durch Ochsen - und Bocksblut Sünden wegzunehmen.
\bibverse{5} Darum, da er in die Welt kommt, spricht er: Opfer und Gaben
hast du nicht gewollt; den Leib aber hast du mir zubereitet.
\bibverse{6} Brandopfer und Sündopfer gefallen dir nicht. \bibverse{7}
Da sprach ich: Siehe, ich komme; im Buch stehet vornehmlich von mir
geschrieben, daß ich tun soll, GOtt, deinen Willen. \bibverse{8} Droben,
als er gesagt hatte: Opfer und Gaben, Brandopfer und Sündopfer hast du
nicht gewollt; sie gefallen dir auch nicht (welche nach dem Gesetz
geopfert werden), \bibverse{9} da sprach er: Siehe, ich komme zu tun,
GOtt, deinen Willen. Da hebt er das erste auf, daß er das andere
einsetze. \bibverse{10} In welchem Willen wir sind geheiliget, einmal
geschehen durch das Opfer des Leibes JEsu Christi. \bibverse{11} Und ein
jeglicher Priester ist eingesetzt, daß er alle Tage Gottesdienst pflege
und oftmals einerlei Opfer tue, welche nimmermehr können die Sünden
abnehmen. \bibverse{12} Dieser aber, da er hat ein Opfer für die Sünden
geopfert, das ewiglich gilt, sitzt er nun zur Rechten GOttes
\bibverse{13} und wartet hinfort, bis daß seine Feinde zum Schemel
seiner Füße gelegt werden. \bibverse{14} Denn mit einem Opfer hat er in
Ewigkeit vollendet, die geheiliget werden. \bibverse{15} Es bezeuget uns
aber das auch der Heilige Geist. Denn nachdem er zuvor gesagt hatte:
\bibverse{16} Das ist das Testament, das ich ihnen machen will nach
diesen Tagen, spricht der HErr: Ich will mein Gesetz in ihr Herz geben,
und in ihre Sinne will ich es schreiben, \bibverse{17} und ihrer Sünden
und ihrer Ungerechtigkeit will ich nicht mehr gedenken. \bibverse{18} Wo
aber derselbigen Vergebung ist, da ist nicht mehr Opfer für die Sünde.
\bibverse{19} So wir denn nun haben, liebe Brüder, die Freudigkeit zum
Eingang in das Heilige durch das Blut JEsu, \bibverse{20} welchen er uns
zubereitet hat zum neuen und lebendigen Wege durch den Vorhang, das ist,
durch sein Fleisch, \bibverse{21} und haben einen Hohenpriester über das
Haus GOttes: \bibverse{22} so lasset uns hinzugehen mit wahrhaftigem
Herzen, in völligem Glauben, besprenget in unsern Herzen und los von dem
bösen Gewissen und gewaschen am Leibe mit reinem Wasser; \bibverse{23}
und lasset uns halten an dem Bekenntnis der Hoffnung und nicht wanken
denn er ist treu, der sie verheißen hat. \bibverse{24} Und lasset uns
untereinander unser selbst wahrnehmen mit Reizen zur Liebe und guten
Werken \bibverse{25} und nicht verlassen unsere Versammlung, wie etliche
pflegen, sondern untereinander ermahnen, und das viel mehr, soviel ihr
sehet, daß sich der Tag nahet. \bibverse{26} Denn so wir mutwillig
sündigen, nachdem wir die Erkenntnis der Wahrheit empfangen haben, haben
wir fürder kein ander Opfer mehr für die Sünden, \bibverse{27} sondern
ein schrecklich Warten des Gerichts und des Feuereifers, der die
Widerwärtigen verzehren wird. \bibverse{28} Wenn jemand das Gesetz
Mose's bricht, der muß sterben ohne Barmherzigkeit durch zween oder drei
Zeugen. \bibverse{29} Wieviel meinet ihr, ärgere Strafe wird der
verdienen, der den Sohn GOttes mit Füßen tritt und das Blut des
Testaments unrein achtet, durch welches er geheiliget ist, und den Geist
der Gnaden schmähet? \bibverse{30} Denn wir wissen den, der da sagte:
Die Rache ist mein; ich will vergelten, spricht der HErr. Und abermal:
Der HErr wird sein Volk richten. \bibverse{31} Schrecklich ist's, in die
Hände des lebendigen GOttes zu fallen. \bibverse{32} Gedenket aber an
die vorigen Tage, in welchen ihr, erleuchtet, erduldet habt einen großen
Kampf des Leidens, \bibverse{33} zum Teil selbst durch Schmach und
Trübsal ein Schauspiel worden, zum Teil Gemeinschaft gehabt mit denen,
denen es also gehet. \bibverse{34} Denn ihr habt mit meinen Banden
Mitleid gehabt und den Raub eurer Güter mit Freuden erduldet, als die
ihr wisset, daß ihr bei euch selbst eine bessere und bleibende Habe im
Himmel habt. \bibverse{35} Werfet euer Vertrauen nicht weg, welches eine
große Belohnung hat. \bibverse{36} Geduld aber ist euch not, auf daß ihr
den Willen GOttes tut und die Verheißung empfanget. \bibverse{37} Denn
noch über eine kleine Weile, so wird kommen, der da kommen soll, und
nicht verziehen. \bibverse{38} Der Gerechte aber wird des Glaubens
leben. Wer aber weichen wird, an dem wird meine Seele kein Gefallen
haben. \bibverse{39} Wir aber sind nicht von denen, die da weichen und
verdammt werden, sondern von denen, die da glauben und die Seele
erretten.

\hypertarget{section-9}{%
\section{11}\label{section-9}}

\bibverse{1} Es ist aber der Glaube eine gewisse Zuversicht des, das man
hoffet, und nicht zweifeln an dem, das man nicht siehet. \bibverse{2}
Durch den haben die Alten Zeugnis überkommen. \bibverse{3} Durch den
Glauben merken wir, daß die Welt durch GOttes Wort fertig ist, daß
alles, was man siehet, aus nichts worden ist. \bibverse{4} Durch den
Glauben hat Abel GOtt ein größer Opfer getan denn Kain, durch welchen er
Zeugnis überkommen hat, daß er gerecht sei, da GOtt zeugete von seiner
Gabe; und durch denselbigen redet er noch, wiewohl er gestorben ist.
\bibverse{5} Durch den Glauben ward Enoch weggenommen, daß er den Tod
nicht sähe, und ward nicht gefunden, darum daß ihn GOtt wegnahm; denn
vor seinem Wegnehmen hat er Zeugnis gehabt, daß er GOtt gefallen habe.
\bibverse{6} Aber ohne Glauben ist's unmöglich, GOtt gefallen; denn wer
zu GOtt kommen will, der muß glauben, daß er sei und denen, die ihn
suchen, ein Vergelter sein werde. \bibverse{7} Durch den Glauben hat
Noah GOtt geehret und die Arche zubereitet zum Heil seines Hauses, da er
einen göttlichen Befehl empfing von dem das man noch nicht sah; durch
welchen er verdammte die Welt und hat ererbet die Gerechtigkeit, die
durch den Glauben kommt. \bibverse{8} Durch den Glauben ward gehorsam
Abraham, da er berufen ward, auszugehen in das Land, das er ererben
sollte; und ging aus und wußte nicht, wo er hinkäme. \bibverse{9} Durch
den Glauben ist er ein Fremdling gewesen in dem verheißenen Lande als in
einem fremden und wohnete in Hütten mit Isaak und Jakob, den Miterben
derselbigen Verheißung. \bibverse{10} Denn er wartete auf eine Stadt,
die einen Grund hat, welcher Baumeister und Schöpfer GOtt ist.
\bibverse{11} Durch den Glauben empfing auch Sara Kraft, daß sie
schwanger ward, und gebar über die Zeit ihres Alters; denn sie achtete
ihn treu, der es verheißen hatte. \bibverse{12} Darum sind auch von
einem, wie wohl erstorbenen Leibes, viele geboren wie die Sterne am
Himmel und wie der Sand am Rande des Meeres, der unzählig ist.
\bibverse{13} Diese alle sind gestorben im Glauben und haben die
Verheißung nicht empfangen, sondern sie von ferne gesehen und sich der
vertröstet und wohl genügen lassen und bekannt, daß sie Gäste und
Fremdlinge auf Erden sind. \bibverse{14} Denn die solches sagen, die
geben zu verstehen, daß sie ein Vaterland suchen. \bibverse{15} Und
zwar, wo sie das gemeinet hätten, von welchem sie waren ausgezogen,
hatten sie ja Zeit, wieder umzukehren. \bibverse{16} Nun aber begehren
sie eines besseren, nämlich eines himmlischen. Darum schämet sich GOtt
ihrer nicht, zu heißen ihr GOtt; denn er hat ihnen eine Stadt
zubereitet. \bibverse{17} Durch den Glauben opferte Abraham den Isaak,
da er versucht ward, und gab dahin den Eingebornen, da er schon die
Verheißung empfangen hatte, \bibverse{18} von welchem gesagt war: In
Isaak wird dir dein Same geheißen werden, \bibverse{19} und dachte: GOtt
kann auch wohl von den Toten erwecken; daher er auch ihn zum Vorbilde
wieder nahm. \bibverse{20} Durch den Glauben segnete Isaak von den
zukünftigen Dingen den Jakob und Esau. \bibverse{21} Durch den Glauben
segnete Jakob, da er starb, beide Söhne Josephs und neigete sich gegen
seines Zepters Spitze. \bibverse{22} Durch den Glauben redete Joseph vom
Auszug der Kinder Israel, da er starb, und tat Befehl von seinen
Gebeinen. \bibverse{23} Durch den Glauben ward Mose, da er geboren war,
drei Monden verborgen von seinen Eltern, darum daß sie sahen, wie er ein
schön Kind war, und fürchteten sich nicht vor des Königs Gebot.
\bibverse{24} Durch den Glauben wollte Mose, da er groß ward, nicht mehr
ein Sohn heißen der Tochter Pharaos \bibverse{25} und erwählete viel
lieber, mit dem Volk GOttes Ungemach zu leiden, denn die zeitliche
Ergötzung der Sünde zu haben, \bibverse{26} und achtete die Schmach
Christi für größeren Reichtum denn die Schätze Ägyptens; denn er sah an
die Belohnung. \bibverse{27} Durch den Glauben verließ er Ägypten und
fürchtete nicht des Königs Grimm; denn er hielt sich an den, den er
nicht sah, als sähe er ihn. \bibverse{28} Durch den Glauben hielt er
Ostern und das Blutvergießen, auf daß, der die Erstgeburten würgete, sie
nicht träfe. \bibverse{29} Durch den Glauben gingen sie durch das Rote
Meer als durch trocken Land; welches die Ägypter auch versuchten und
ersoffen. \bibverse{30} Durch den Glauben fielen die Mauern Jerichos, da
sie sieben Tage umhergegangen waren. \bibverse{31} Durch den Glauben
ward die Hure Rahab nicht verloren mit den Ungläubigen, da sie die
Kundschafter freundlich aufnahm. \bibverse{32} Und was soll ich mehr
sagen? Die Zeit würde mir zu kurz, wenn ich sollte erzählen von Gideon
und Barak und Simson und Jephthah und David und Samuel und den
Propheten, \bibverse{33} welche haben durch den Glauben Königreiche
bezwungen, Gerechtigkeit gewirket, die Verheißung erlanget, der Löwen
Rachen verstopfet, \bibverse{34} des Feuers Kraft ausgelöscht, sind des
Schwerts Schärfe entronnen, sind kräftig worden aus der Schwachheit,
sind stark worden im Streit, haben der Fremden Heer daniedergelegt.
\bibverse{35} Die Weiber haben ihre Toten von der Auferstehung wieder
genommen; die andern aber sind zerschlagen und haben keine Erlösung
angenommen, auf daß sie die Auferstehung, die besser ist, erlangeten.
\bibverse{36} Etliche haben Spott und Geißeln erlitten, dazu Bande und
Gefängnis. \bibverse{37} Sie sind gesteiniget, zerhackt, zerstochen,
durchs Schwert getötet; sie sind umhergegangen in Pelzen und
Ziegenfellen, mit Mangel, mit Trübsal, mit Ungemach \bibverse{38} (deren
die Welt nicht wert war) und sind im Elend gegangen in den Wüsten, auf
den Bergen und in den Klüften und Löchern der Erde. \bibverse{39} Diese
alle haben durch den Glauben Zeugnis überkommen und nicht empfangen die
Verheißung, \bibverse{40} darum daß GOtt etwas Besseres für uns zuvor
versehen hat, daß sie nicht ohne uns vollendet würden.

\hypertarget{section-10}{%
\section{12}\label{section-10}}

\bibverse{1} Darum auch wir, dieweil wir solchen Haufen Zeugen um uns
haben, lasset uns ablegen die Sünde, so uns immer anklebt und träge
macht, und lasset uns laufen durch Geduld in dem Kampf, der uns
verordnet ist, \bibverse{2} und aufsehen auf JEsum, den Anfänger und
Vollender des Glaubens, welcher, da er wohl hätte mögen Freude haben,
erduldete er das Kreuz und achtete der Schande nicht und ist gesessen
zur Rechten auf dem Stuhl GOttes. \bibverse{3} Gedenket an den, der ein
solches Widersprechen von den Sündern wider sich erduldet hat, daß ihr
nicht in eurem Mut matt werdet und ablasset. \bibverse{4} Denn ihr habt
noch nicht bis aufs Blut widerstanden über dem Kämpfen wider die Sünde
\bibverse{5} und habt bereits vergessen des Trostes, der zu euch redet
als zu den Kindern: Mein Sohn, achte nicht gering die Züchtigung des
HErrn und verzage nicht, wenn du von ihm gestraft wirst; \bibverse{6}
denn welchen der HErr liebhat, den züchtiget er; er stäupt aber einen
jeglichen Sohn, den er aufnimmt. \bibverse{7} So ihr die Züchtigung
erduldet, so erbeut sich euch GOtt als Kindern; denn wo ist ein Sohn,
den der Vater nicht züchtiget? \bibverse{8} Seid ihr aber ohne
Züchtigung, welcher sie alle sind teilhaftig worden, so seid ihr
Bastarde und nicht Kinder. \bibverse{9} Auch so wir haben unsere
leiblichen Väter zu Züchtigern gehabt und sie gescheuet, sollten wir
denn nicht viel mehr untertan sein dem geistlichen Vater, daß wir leben?
\bibverse{10} Und jene zwar haben uns gezüchtiget wenige Tage nach ihrem
Dünken, dieser aber zu Nutz, auf daß wir seine Heiligung erlangen.
\bibverse{11} Alle Züchtigung aber, wenn sie da ist, dünkt sie uns nicht
Freude, sondern Traurigkeit sein; aber danach wird sie geben eine
friedsame Frucht der Gerechtigkeit denen, die dadurch geübet sind.
\bibverse{12} Darum richtet wieder auf die lässigen Hände und die müden
Kniee \bibverse{13} und tut gewissen Tritt mit euren Füßen, daß nicht
jemand strauchele wie ein Lahmer, sondern vielmehr gesund werde.
\bibverse{14} Jaget nach dem Frieden gegen jedermann und der Heiligung,
ohne welche wird niemand den HErrn sehen. \bibverse{15} Und sehet
darauf, daß nicht jemand GOttes Gnade versäume, daß nicht etwa eine
bittere Wurzel aufwachse und Unfrieden anrichte, und viele durch
dieselbige verunreiniget werden; \bibverse{16} daß nicht jemand sei ein
Hurer oder ein Gottloser wie Esau, der um einer Speise willen seine
Erstgeburt verkaufte. \bibverse{17} Wisset aber, daß er hernach, da er
den Segen ererben wollte, verworfen ist; denn er fand keinen Raum zur
Buße, wiewohl er sie mit Tränen suchte. \bibverse{18} Denn ihr seid
nicht kommen zu dem Berge, den man anrühren konnte, und mit Feuer
brannte, noch zu dem Dunkel und Finsternis und Ungewitter \bibverse{19}
noch zu dem Hall der Posaune und zur Stimme der Worte, welcher sich
weigerten, die sie höreten, daß ihnen das Wort ja nicht gesagt würde
\bibverse{20} (denn sie mochten's nicht ertragen, was da gesagt ward.
Und wenn ein Tier den Berg anrührete, sollte es gesteiniget oder mit
einem Geschoß erschossen werden. \bibverse{21} Und also erschrecklich
war das Gesicht, daß Mose sprach: Ich bin erschrocken und zittere),
\bibverse{22} sondern ihr seid kommen zu dem Berge Zion und zu der Stadt
des lebendigen GOttes, zu dem himmlischen Jerusalem, und zu der Menge
vieler tausend Engel \bibverse{23} und zu der Gemeinde der Erstgebornen,
die im Himmel angeschrieben sind, und zu GOtt, dem Richter über alle,
und zu den Geistern der vollkommenen Gerechten \bibverse{24} und zu dem
Mittler des Neuen Testaments, JEsus, und zu dem Blut der Besprengung,
das da besser redet denn Abels. \bibverse{25} Sehet zu, daß ihr euch des
nicht weigert, der da redet! Denn jene nicht entflohen sind, die sich
weigerten, da er auf Erden redete, viel weniger wir, so wir uns des
weigern, der vom Himmel redet, \bibverse{26} welches Stimme zu der Zeit
die Erde bewegete. Nun aber verheißet er und spricht: Noch einmal will
ich bewegen nicht allein die Erde, sondern auch den Himmel.
\bibverse{27} Aber solches ``Noch einmal'' zeigt an, daß das Bewegliche
soll verändert werden, als das gemacht ist, auf daß da bleibe das
Unbewegliche. \bibverse{28} Darum, dieweil wir empfangen ein unbeweglich
Reich, haben wir Gnade, durch welche wir sollen GOtt dienen, ihm zu
gefallen, mit Zucht und Furcht. \bibverse{29} Denn unser GOtt ist ein
verzehrend Feuer.

\hypertarget{section-11}{%
\section{13}\label{section-11}}

\bibverse{1} Bleibet fest in der brüderlichen Liebe! \bibverse{2}
Gastfrei zu sein vergesset nicht; denn durch dasselbige haben etliche
ohne ihr Wissen Engel beherberget. \bibverse{3} Gedenket der Gebundenen
als die Mitgebundenen und derer, die Trübsal leiden, als die ihr auch
noch im Leibe lebet. \bibverse{4} Die Ehe soll ehrlich gehalten werden
bei allen und das Ehebett unbefleckt; die Hurer aber und Ehebrecher wird
GOtt richten. \bibverse{5} Der Wandel sei ohne Geiz; und lasset euch
begnügen an dem, was da ist. Denn er hat gesagt: Ich will dich nicht
verlassen noch versäumen, \bibverse{6} also daß wir dürfen sagen: Der
HErr ist mein Helfer, und will mich nicht fürchten; was sollte mir ein
Mensch tun? \bibverse{7} Gedenket an eure Lehrer, die euch das Wort
GOttes gesagt haben, welcher Ende schauet an und folget ihrem Glauben
nach. \bibverse{8} JEsus Christus gestern und heute und derselbe auch in
Ewigkeit. \bibverse{9} Lasset euch nicht mit mancherlei und fremden
Lehren umtreiben; denn es ist ein köstlich Ding, daß das Herz fest
werde, welches geschieht durch Gnade, nicht durch Speisen, davon keinen
Nutzen haben, so damit umgehen. \bibverse{10} Wir haben einen Altar,
davon nicht Macht haben zu essen, die der Hütte pflegen. \bibverse{11}
Denn welcher Tiere Blut getragen wird durch den Hohenpriester in das
Heilige für die Sünde, derselbigen Leichname werden verbrannt außer dem
Lager. \bibverse{12} Darum auch JEsus, auf daß er heiligte das Volk
durch sein eigen Blut, hat er gelitten außen vor dem Tor. \bibverse{13}
So lasset uns nun zu ihm hinausgehen außer dem Lager und seine Schmach
tragen. \bibverse{14} Denn wir haben hier keine bleibende Stadt, sondern
die zukünftige suchen wir. \bibverse{15} So lasset uns nun opfern durch
ihn das Lobopfer GOtt allezeit, das ist, die Frucht der Lippen, die
seinen Namen bekennen. \bibverse{16} Wohlzutun und mitzuteilen vergesset
nicht; denn solche Opfer gefallen GOtt wohl. \bibverse{17} Gehorchet
euren Lehrern und folget ihnen; denn sie wachen über eure Seelen, als
die da Rechenschaft dafür geben sollen, auf daß sie das mit Freuden tun
und nicht mit Seufzen; denn das ist euch nicht gut. \bibverse{18} Betet
für uns! Unser Trost ist der, daß wir ein gut Gewissen haben und
fleißigen uns, guten Wandel zu führen bei allen. \bibverse{19} Ich
ermahne euch aber zum Überfluß, solches zu tun, auf daß ich aufs
schierste wieder zu euch komme. \bibverse{20} GOtt aber des Friedens,
der von den Toten ausgeführet hat den großen Hirten der Schafe durch das
Blut des ewigen Testaments, unsern HErrn JEsum, \bibverse{21} der mache
euch fertig in allem guten Werk, zu tun seinen Willen, und schaffe in
euch, was vor ihm gefällig ist, durch JEsum Christum, welchem sei Ehre
von Ewigkeit zu Ewigkeit! Amen. \bibverse{22} Ich ermahne euch aber,
liebe Brüder, haltet das Wort der Ermahnung zugute; denn ich habe euch
kurz geschrieben. \bibverse{23} Wisset, daß der Bruder Timotheus wieder
ledig ist, mit welchem, so er bald kommt, will ich euch sehen.
\bibverse{24} Grüßet alle eure Lehrer und alle Heiligen. Es grüßen euch
die Brüder aus Italien. \bibverse{25} Die Gnade sei mit euch allen!
Amen.
