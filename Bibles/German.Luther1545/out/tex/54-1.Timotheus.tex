\hypertarget{section}{%
\section{1}\label{section}}

\bibverse{1} Paulus, ein Apostel JEsu Christi, nach dem Befehl GOttes,
unsers Heilandes, und des HErrn JEsu Christi, der unsere Hoffnung ist:
\bibverse{2} Timotheus, meinem rechtschaffenen Sohn im Glauben. Gnade,
Barmherzigkeit, Friede von GOtt, unserm Vater, und unserm HErrn JEsu
Christo. \bibverse{3} Wie ich dich ermahnet habe, daß du zu Ephesus
bliebest, da ich nach Mazedonien zog, und gebötest etlichen, daß sie
nicht anders lehreten, \bibverse{4} auch nicht achthätten auf die Fabeln
und der Geschlechtsregister, die kein Ende haben, und bringen Fragen
auf, mehr denn Besserung zu GOtt im Glauben. \bibverse{5} Denn die
Hauptsumme des Gebots ist Liebe von reinem Herzen und von gutem Gewissen
und von ungefärbtem Glauben, \bibverse{6} welcher haben etliche gefehlet
und sind umgewandt zu unnützem Geschwätz, \bibverse{7} wollen der
Schrift Meister sein und verstehen nicht, was sie sagen, oder was sie
setzen. \bibverse{8} Wir wissen aber, daß das Gesetz gut ist, so sein
jemand recht brauchet, \bibverse{9} und weiß solches, daß dem Gerechten
kein Gesetz gegeben ist, sondern den Ungerechten und Ungehorsamen, den
Gottlosen und Sündern, den Unheiligen und Ungeistlichen, den
Vatermördern und Muttermördern, den Totschlägern, \bibverse{10} den
Hurern, den Knabenschändern, den Menschendieben, den Lügnern, den
Meineidigen, und so etwas mehr der heilsamen Lehre wider ist,
\bibverse{11} nach dem herrlichen Evangelium des seligen GOttes, welches
mir vertrauet ist. \bibverse{12} Und ich danke unserm HErrn Christo
JEsu, der mich stark gemacht und treu geachtet hat und gesetzt in das
Amt, \bibverse{13} der ich zuvor war ein Lästerer und ein Verfolger und
ein Schmäher. Aber mir ist Barmherzigkeit widerfahren; denn ich hab's
unwissend getan, im Unglauben. \bibverse{14} Es ist aber desto reicher
gewesen die Gnade unsers HErrn samt dem Glauben und der Liebe, die in
Christo JEsu ist. \bibverse{15} Denn das ist je gewißlich wahr und ein
teuer wertes Wort, daß Christus JEsus kommen ist in die Welt, die Sünder
selig zu machen, unter welchen ich der vornehmste bin. \bibverse{16}
Aber darum ist mir Barmherzigkeit widerfahren, auf daß an mir
vornehmlich JEsus Christus erzeigete alle Geduld zum Exempel denen, die
an ihn glauben sollten zum ewigen Leben. \bibverse{17} Aber GOtt, dem
ewigen Könige, dem Unvergänglichen und Unsichtbaren und allein Weisen,
sei Ehre und Preis in Ewigkeit! Amen. \bibverse{18} Dies Gebot befehle
ich dir, mein Sohn Timotheus, nach den vorigen Weissagungen über dir,
daß du in denselbigen eine gute Ritterschaft übest \bibverse{19} und
habest den Glauben und gut Gewissen, welches etliche von sich gestoßen
und am Glauben Schiffbruch erlitten haben; \bibverse{20} unter welchen
ist Hymenäus und Alexander, welche ich habe dem Satan übergeben, daß sie
gezüchtiget werden, nicht mehr zu lästern.

\hypertarget{section-1}{%
\section{2}\label{section-1}}

\bibverse{1} So ermahne ich nun, daß man vor allen Dingen zuerst tue
Bitte, Gebet, Fürbitte und Danksagung für alle Menschen, \bibverse{2}
für die Könige und für alle Obrigkeit, auf daß wir ein ruhig und stilles
Leben führen mögen in aller Gottseligkeit und Ehrbarkeit. \bibverse{3}
Denn solches ist gut, dazu auch angenehm vor GOtt, unserm Heilande,
\bibverse{4} welcher will, daß allen Menschen geholfen werde, und zur
Erkenntnis der Wahrheit kommen. \bibverse{5} Denn es ist ein GOtt und
ein Mittler zwischen GOtt und den Menschen, nämlich der Mensch Christus
JEsus, \bibverse{6} der sich selbst gegeben hat für alle zur Erlösung,
daß solches zu seiner Zeit geprediget würde; \bibverse{7} dazu ich
gesetzt bin ein Prediger und Apostel (ich sage die Wahrheit in Christo
und lüge nicht), ein Lehrer der Heiden im Glauben und in der Wahrheit.
\bibverse{8} So will ich nun, daß die Männer beten an allen Orten und
aufheben heilige Hände, ohne Zorn und Zweifel. \bibverse{9}
Desselbigengleichen die Weiber, daß sie in zierlichem Kleide mit Scham
und Zucht sich schmücken, nicht mit Zöpfen oder Gold oder Perlen oder
köstlichem Gewand, \bibverse{10} sondern wie sich's ziemet den Weibern,
die da Gottseligkeit beweisen durch gute Werke. \bibverse{11} Ein Weib
lerne in der Stille mit aller Untertänigkeit. \bibverse{12} Einem Weibe
aber gestatte ich nicht, daß sie lehre, auch nicht, daß sie des Mannes
Herr sei, sondern stille sei. \bibverse{13} Denn Adam ist am ersten
gemacht, danach Eva. \bibverse{14} Und Adam ward nicht verführet; das
Weib aber ward verführet und hat die Übertretung eingeführet.
\bibverse{15} Sie wird aber selig werden durch Kinderzeugen, so sie
bleiben im Glauben und in der Liebe und in der Heiligung samt der Zucht.

\hypertarget{section-2}{%
\section{3}\label{section-2}}

\bibverse{1} Das ist je gewißlich wahr, so jemand ein Bischofsamt
begehret, der begehret ein köstlich Werk. \bibverse{2} Es soll aber ein
Bischof unsträflich sein, eines Weibes Mann, nüchtern, mäßig, sittig,
gastfrei, lehrhaftig, \bibverse{3} nicht ein Weinsäufer, nicht pochen,
nicht unehrliche Hantierung treiben, sondern gelinde, nicht haderhaftig,
nicht geizig, \bibverse{4} der seinem eigenen Hause wohl vorstehe, der
gehorsame Kinder habe mit aller Ehrbarkeit \bibverse{5} (so aber jemand
seinem eigenen Hause nicht weiß vorzustehen, wie wird er die Gemeinde
GOttes versorgen?), \bibverse{6} nicht ein Neuling, auf daß er sich
nicht aufblase und dem Lästerer ins Urteil falle. \bibverse{7} Er muß
aber auch ein gut Zeugnis haben von denen, die draußen sind, auf daß er
nicht falle dem Lästerer in die Schmach und Strick. \bibverse{8}
Desselbigengleichen die Diener sollen ehrbar sein, nicht zweizüngig,
nicht Weinsäufer, nicht unehrliche Hantierung treiben \bibverse{9} die
das Geheimnis des Glaubens in reinem Gewissen haben. \bibverse{10} Und
dieselbigen lasse man zuvor versuchen; danach lasse man sie dienen, wenn
sie unsträflich sind. \bibverse{11} Desselbigengleichen ihre Weiber
sollen ehrbar sein, nicht Lästerinnen, nüchtern, treu in allen Dingen.
\bibverse{12} Die Diener laß einen jeglichen sein eines Weibes Mann, die
ihren Kindern wohl vorstehen und ihren eigenen Häusern. \bibverse{13}
Welche aber wohl dienen, die erwerben sich selbst eine gute Stufe und
eine große Freudigkeit im Glauben in Christo JEsu. \bibverse{14} Solches
schreibe ich dir und hoffe, aufs schierste zu dir zu kommen.
\bibverse{15} So ich aber verzöge, daß du wissest, wie du wandeln sollst
in dem Hause GOttes, welches ist die Gemeinde des lebendigen GOttes, ein
Pfeiler und Grundfeste der Wahrheit. \bibverse{16} Und kündlich groß ist
das gottselige Geheimnis: GOtt ist offenbaret im Fleisch,
gerechtfertiget im Geist, erschienen den Engeln, geprediget den Heiden,
geglaubet von der Welt, aufgenommen in die Herrlichkeit.

\hypertarget{section-3}{%
\section{4}\label{section-3}}

\bibverse{1} Der Geist aber sagt deutlich, daß in den letzten Zeiten
werden etliche von dem Glauben abtreten und anhangen den verführerischen
Geistern und Lehren der Teufel \bibverse{2} durch die, so in Gleisnerei
Lügenredner sind und Brandmal in ihrem Gewissen haben \bibverse{3} und
verbieten, ehelich zu werden und zu meiden die Speisen, die GOtt
geschaffen hat, zu nehmen mit Danksagung, den Gläubigen und denen, die
die Wahrheit erkennen. \bibverse{4} Denn alle Kreatur GOttes ist gut und
nichts verwerflich, was mit Danksagung empfangen wird. \bibverse{5} Denn
es wird geheiliget durch das Wort GOttes und Gebet. \bibverse{6} Wenn du
den Brüdern solches vorhältst, so wirst du ein guter Diener JEsu Christi
sein, auferzogen in den Worten des Glaubens und der guten Lehre, bei
welcher du immerdar gewesen bist. \bibverse{7} Der ungeistlichen aber
und altvettelischen Fabeln entschlage dich. Übe dich selbst aber an der
Gottseligkeit. \bibverse{8} Denn die leibliche Übung ist wenig nütz;
aber die Gottseligkeit ist zu allen Dingen nütz und hat die Verheißung
dieses und des zukünftigen Lebens. \bibverse{9} Das ist je gewißlich
wahr und ein teuer wertes Wort. \bibverse{10} Denn dahin arbeiten wir
auch und werden geschmähet, daß wir auf den lebendigen GOtt gehoffet
haben, welcher ist der Heiland aller Menschen, sonderlich aber der
Gläubigen. \bibverse{11} Solches gebiete und lehre! \bibverse{12}
Niemand verachte deine Jugend, sondern sei ein Vorbild den Gläubigen im
Wort, im Wandel, in der Liebe, im Geist, im Glauben, in der Keuschheit.
\bibverse{13} Halt an mit Lesen, mit Ermahnen, mit Lehren, bis ich
komme! \bibverse{14} Laß nicht aus der Acht die Gabe, die dir gegeben
ist durch die Weissagung mit Handauflegung der Ältesten. \bibverse{15}
Solches warte, damit gehe um auf daß dein Zunehmen in allen Dingen
offenbar sei. \bibverse{16} Hab acht auf dich selbst und auf die Lehre;
beharre in diesen Stücken! Denn wo du solches tust, wirst du dich selbst
selig machen, und die dich hören.

\hypertarget{section-4}{%
\section{5}\label{section-4}}

\bibverse{1} Einen Alten schilt nicht, sondern ermahne ihn als einen
Vater, die Jungen als die Brüder, \bibverse{2} die alten Weiber als die
Mütter, die jungen als die Schwestern mit aller Keuschheit. \bibverse{3}
Ehre die Witwen, welche rechte Witwen sind. \bibverse{4} So aber eine
Witwe Kinder oder Neffen hat, solche laß zuvor lernen ihre eigenen
Häuser göttlich regieren und den Eltern Gleiches vergelten; denn das ist
wohl getan und angenehm vor GOtt. \bibverse{5} Das ist aber eine rechte
Witwe, die einsam ist, die ihre Hoffnung auf GOtt stellet und bleibet am
Gebet und Flehen Tag und Nacht. \bibverse{6} Welche aber in Wollüsten
lebet, die ist lebendig tot. \bibverse{7} Solches gebiete, auf daß sie
untadelig seien. \bibverse{8} So aber jemand die Seinen, sonderlich
seine Hausgenossen, nicht versorget, der hat den Glauben verleugnet und
ist ärger denn ein Heide. \bibverse{9} Laß keine Witwe erwählet werden
unter sechzig Jahren, und die da gewesen sei eines Mannes Weib,
\bibverse{10} und die ein Zeugnis habe guter Werke, so sie Kinder
aufgezogen hat, so sie gastfrei gewesen ist, so sie der Heiligen Füße
gewaschen hat, so sie den Trübseligen Handreichung getan hat, so sie
allem guten Werk nachkommen ist. \bibverse{11} Der jungen Witwen aber
entschlage dich; denn wenn sie geil worden sind wider Christum, so
wollen sie freien \bibverse{12} und haben ihr Urteil, daß sie den ersten
Glauben gebrochen haben. \bibverse{13} Daneben sind sie faul und lernen
umlaufen durch die Häuser; nicht allein aber sind sie faul, sondern auch
schwätzig und vorwitzig und reden, was nicht sein soll. \bibverse{14} So
will ich nun, daß die jungen Witwen freien, Kinder zeugen, haushalten,
dem Widersacher keine Ursache geben zu schelten. \bibverse{15} Denn es
sind schon etliche umgewandt dem Satan nach. \bibverse{16} So aber ein
Gläubiger oder Gläubigin Witwen hat, der versorge dieselbigen und lasse
die Gemeinde nicht beschweret werden, auf daß die, so rechte Witwen
sind, mögen genug haben. \bibverse{17} Die Ältesten, die wohl vorstehen,
die halte man zwiefacher Ehre wert, sonderlich die da arbeiten im Wort
und in der Lehre. \bibverse{18} Denn es spricht die Schrift: Du sollst
nicht dem Ochsen das Maul verbinden, der da drischt, und: Ein Arbeiter
ist seines Lohnes wert. \bibverse{19} Wider einen Ältesten nimm keine
Klage auf außer zweien oder dreien Zeugen. \bibverse{20} Die da
sündigen, die strafe vor allen, auf daß sich auch die andern fürchten.
\bibverse{21} Ich bezeuge vor GOtt und dem HErrn JEsu Christo und den
auserwählten Engeln, daß du solches haltest ohne eigen Gutdünken und
nichts tuest nach Gunst. \bibverse{22} Die Hände lege niemand bald auf;
mache dich auch nicht teilhaftig fremder Sünden. Halte dich selber
keusch! \bibverse{23} Trinke nicht mehr Wasser, sondern brauche ein
wenig Wein um deines Magens willen, und daß du oft krank bist.
\bibverse{24} Etlicher Menschen Sünden sind offenbar, daß man sie vorhin
richten kann; etlicher aber werden hernach offenbar. \bibverse{25}
Desselbigengleichen auch etlicher gute Werkes sind zuvor offenbar; und
die andern bleiben auch nicht verborgen.

\hypertarget{section-5}{%
\section{6}\label{section-5}}

\bibverse{1} Die Knechte, so unter dem Joch sind, sollen ihre Herren
aller Ehren wert halten, auf daß nicht der Name GOttes und die Lehre
verlästert werde. \bibverse{2} Welche aber gläubige Herren haben, sollen
dieselbigen nicht verachten (mit dem Schein), daß sie Brüder sind,
sondern sollen vielmehr dienstbar sein, dieweil sie gläubig und geliebt
und der Wohltat teilhaftig sind. Solches lehre und ermahne! \bibverse{3}
So jemand anders lehret und bleibet nicht bei den heilsamen Worten
unsers HErrn JEsu Christi und bei der Lehre von der Gottseligkeit,
\bibverse{4} der ist verdüstert und weiß nichts, sondern ist seuchtig in
Fragen und Wortkriegen, aus welchen entspringet Neid, Hader, Lästerung,
böser Argwohn, \bibverse{5} Schulgezänke solcher Menschen, die
zerrüttete Sinne haben und der Wahrheit beraubt sind, die da meinen,
Gottseligkeit sei ein Gewerbe. Tue dich von solchen! \bibverse{6} Es ist
aber ein großer Gewinn, wer gottselig ist und lässet sich genügen.
\bibverse{7} Denn wir haben nichts in die Welt gebracht, darum offenbar
ist, wir werden auch nichts hinausbringen. \bibverse{8} Wenn wir aber
Nahrung und Kleider haben, so lasset uns begnügen. \bibverse{9} Denn die
da reich werden wollen, die fallen in Versuchung und Stricke und viel
törichter und schädlicher Lüste, welche versenken die Menschen ins
Verderben und Verdammnis. \bibverse{10} Denn Geiz ist eine Wurzel alles
Übels, welches hat etliche gelüstet, und sind vom Glauben irregegangen
und machen sich selbst viel Schmerzen, \bibverse{11} Aber du,
Gottesmensch, flieh solches! Jage aber nach der Gerechtigkeit, der
Gottseligkeit, dem Glauben, der Liebe, der Geduld, der Sanftmut.
\bibverse{12} Kämpfe den guten Kampf des Glaubens; ergreife das ewige
Leben, dazu du auch berufen bist und bekannt hast ein gut Bekenntnis vor
vielen Zeugen. \bibverse{13} Ich gebiete dir vor GOtt, der alle Dinge
lebendig macht, und vor Christo JEsu, der unter Pontius Pilatus bezeuget
hat ein gut Bekenntnis, \bibverse{14} daß du haltest das Gebot ohne
Flecken, untadelig, bis auf die Erscheinung unsers HErrn JEsu Christi,
\bibverse{15} welche wird zeigen zu seiner Zeit der Selige und allein
Gewaltige, der König aller Könige, und HErr aller Herren, \bibverse{16}
der allein Unsterblichkeit hat; der da wohnet in einem Licht, da niemand
zukommen kann; welchen kein Mensch gesehen hat noch sehen kann: dem sei
Ehre und ewiges Reich! Amen. \bibverse{17} Den Reichen von dieser Welt
gebeut, daß sie nicht stolz seien, auch nicht hoffen auf den ungewissen
Reichtum, sondern auf den lebendigen GOtt, der uns dar gibt reichlich,
allerlei zu genießen, \bibverse{18} daß sie Gutes tun, reich werden an
guten Werken, gerne geben, behilflich seien, \bibverse{19} Schätze
sammeln, sich selbst einen guten Grund aufs Zukünftige, daß sie
ergreifen das ewige Leben. \bibverse{20} O Timotheus, bewahre, was dir
vertrauet ist, und meide die ungeistlichen losen Geschwätze und das
Gezänke der falschberühmten Kunst, \bibverse{21} welche etliche vorgeben
und fehlen des Glaubens. Die Gnade sei mir dir! Amen.
