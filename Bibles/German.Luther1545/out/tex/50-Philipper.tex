\hypertarget{section}{%
\section{1}\label{section}}

\bibverse{1} Paulus und Timotheus, Knechte JEsu Christi: Allen Heiligen
in Christo JEsu zu Philippi samt den Bischöfen und Dienern. \bibverse{2}
Gnade sei mit euch und Friede von GOtt, unserm Vater, und dem HErrn JEsu
Christo! \bibverse{3} Ich danke meinem GOtt, so oft ich euer gedenke
\bibverse{4} (welches ich allezeit tue in allem meinem Gebet für euch
alle, und tue das Gebet mit Freuden), \bibverse{5} über eurer
Gemeinschaft am Evangelium vom ersten Tage an bis her. \bibverse{6} Und
bin desselbigen in guter Zuversicht, daß, der in euch angefangen hat das
gute Werk, der wird's auch vollführen bis an den Tag JEsu Christi.
\bibverse{7} Wie es denn mir billig ist, daß ich dermaßen von euch allen
halte, darum daß ich euch in meinem Herzen habe in diesem meinem
Gefängnis, darin ich das Evangelium verantworte und bekräftige, als die
ihr alle mit mir der Gnade teilhaftig seid. \bibverse{8} Denn GOtt ist
mein Zeuge, wie mich nach euch allen verlanget von Herzensgrund in JEsu
Christo. \bibverse{9} Und darum bete ich, daß eure Liebe je mehr und
mehr reich werde in allerlei Erkenntnis und Erfahrung, \bibverse{10} daß
ihr prüfen möget, was das Beste sei, auf daß ihr seid lauter und
unanstößig bis auf den Tag Christi, \bibverse{11} erfüllet mit Früchten
der Gerechtigkeit, die durch JEsum Christum geschehen (in euch) zu Ehre
und Lobe GOttes.{]} \bibverse{12} Ich lasse euch aber wissen, liebe
Brüder, daß, wie es um mich stehet, das ist nur mehr zur Förderung des
Evangeliums geraten, \bibverse{13} also daß meine Bande offenbar worden
sind in Christo in dem ganzen Richthause und bei den andern allen,
\bibverse{14} und viel Brüder in dem HErrn aus meinen Banden Zuversicht
gewonnen, desto türstiger worden sind, das Wort zu reden ohne Scheu.
\bibverse{15} Etliche zwar predigen Christum auch um Hasses und Haders
willen, etliche aber aus guter Meinung. \bibverse{16} Jene verkündigen
Christum aus Zank und nicht lauter; denn sie meinen, sie wollen eine
Trübsal zuwenden meinen Banden. \bibverse{17} Diese aber aus Liebe; denn
sie wissen, daß ich zur Verantwortung des Evangeliums hier liege.
\bibverse{18} Was ist ihm aber denn? Daß nur Christus verkündiget werde
allerlei Weise, es geschehe Zufalles oder rechter Weise; so freue ich
mich doch darinnen und will mich auch freuen. \bibverse{19} Denn ich
weiß, daß mir dasselbige gelinget zur Seligkeit durch euer Gebet und
durch Handreichung des Geistes JEsu Christi \bibverse{20} wie ich
endlich warte und hoffe, daß ich in keinerlei Stück zuschanden werde,
sondern daß mit aller Freudigkeit, gleichwie sonst allezeit, also auch
jetzt, Christus hoch gepreiset werde an meinem Leibe, es sei durch Leben
oder durch Tod. \bibverse{21} Denn Christus ist mein Leben, und Sterben
ist mein Gewinn. \bibverse{22} Sintemal aber im Fleisch leben dienet,
mehr Frucht zu schaffen, so weiß ich nicht, welches ich erwählen soll.
\bibverse{23} Denn es liegt mir beides hart an: Ich habe Lust
abzuscheiden und bei Christo zu sein, welches auch viel besser wäre:
\bibverse{24} Aber es ist nötiger, im Fleisch bleiben um euretwillen.
\bibverse{25} Und in guter Zuversicht weiß ich, daß ich bleiben und bei
euch allen sein werde euch zur Förderung und zur Freude des Glaubens,
\bibverse{26} auf daß ihr euch sehr rühmen möget in Christo JEsu an mir
durch meine Ankunft wieder zu euch. \bibverse{27} Wandelt nur würdiglich
dem Evangelium Christi, auf daß, ob ich komme und sehe euch oder
abwesend von, euch höre, daß ihr stehet in einem Geist und einer Seele
und samt uns kämpfet für den Glauben des Evangeliums \bibverse{28} und
euch in keinem Wege erschrecken lasset von den Widersachern, welches ist
ein Anzeichen, ihnen der Verdammnis euch aber der Seligkeit, und
dasselbige von GOtt. \bibverse{29} Denn euch ist gegeben, um Christi
willen zu tun, daß ihr nicht allein an ihn glaubet, sondern auch um
seinetwillen leidet, \bibverse{30} und habet denselbigen Kampf, welchen
ihr an mir gesehen habt und nun von mir höret.

\hypertarget{section-1}{%
\section{2}\label{section-1}}

\bibverse{1} Ist nun bei euch Ermahnung in Christo, ist Trost der Liebe,
ist Gemeinschaft des Geistes, ist herzliche Liebe und Barmherzigkeit,
\bibverse{2} so erfüllet meine Freude, daß ihr eines Sinnes seid,
gleiche Liebe habet, einmütig und einhellig seid, \bibverse{3} nichts
tut durch Zank oder eitle Ehre, sondern durch Demut achtet euch
untereinander einer den andern höher denn sich selbst. \bibverse{4} Und
ein jeglicher sehe nicht auf das Seine, sondern auf das, was des andern
ist. \bibverse{5} Ein jeglicher sei gesinnet, wie JEsus Christus auch
war, \bibverse{6} welcher, ob er wohl in göttlicher Gestalt war, hielt
er's nicht für einen Raub, GOtt gleich sein, \bibverse{7} sondern
entäußerte sich selbst und nahm Knechtsgestalt an, ward gleich wie ein
anderer Mensch und an Gebärden als ein Mensch erfunden, \bibverse{8}
erniedrigte sich selbst und ward gehorsam bis zum Tode, ja zum Tode am
Kreuz. \bibverse{9} Darum hat ihn auch GOtt erhöhet und hat ihm einen
Namen gegeben, der über alle Namen ist, \bibverse{10} daß in dem Namen
JEsu sich beugen sollen alle derer Kniee, die im Himmel und auf Erden
und unter der Erde sind, \bibverse{11} und alle Zungen bekennen sollen,
daß JEsus Christus der HErr sei, zur Ehre GOttes des Vaters.{]}
\bibverse{12} Also, meine Liebsten, wie ihr allezeit seid gehorsam
gewesen, nicht allein in meiner Gegenwart, sondern auch nun viel mehr in
meiner Abwesenheit: Schaffet, daß ihr selig werdet, mit Furcht und
Zittern! \bibverse{13} Denn GOtt ist's, der in euch wirket beides, das
Wollen und das Vollbringen, nach seinem Wohlgefallen. \bibverse{14} Tut
alles ohne Murmeln und ohne Zweifel, \bibverse{15} auf daß ihr seid ohne
Tadel und lauter und GOttes Kinder, unsträflich mitten unter dem
unschlachtigen und verkehrten Geschlecht, unter welchem ihr scheinet als
Lichter in der Welt \bibverse{16} damit, daß ihr haltet ob dem Wort des
Lebens, mir zu einem Ruhm an dem Tage Christi, als der ich nicht
vergeblich gelaufen noch vergeblich gearbeitet habe. \bibverse{17} Und
ob ich geopfert werde über dem Opfer und Gottesdienst eures Glaubens, so
freue ich mich und freue mich mit euch allen. \bibverse{18} Desselbigen
sollt ihr euch auch freuen und sollt euch mit mir freuen. \bibverse{19}
Ich hoffe aber in dem HErrn JEsu, daß ich Timotheus bald werde zu euch
senden, daß ich auch erquicket werde, wenn ich erfahre, wie es um euch
stehet. \bibverse{20} Denn ich habe keinen, der so gar meines Sinnes
sei, der so herzlich für euch sorget. \bibverse{21} Denn sie suchen alle
das Ihre, nicht das Christi JEsu ist. \bibverse{22} Ihr aber wisset, daß
er rechtschaffen ist; denn wie ein Kind dem Vater hat er mit mir
gedienet am Evangelium. \bibverse{23} Denselbigen, hoffe ich, werde ich
senden von Stund' an, wenn ich erfahren habe, wie es um mich stehet.
\bibverse{24} Ich vertraue aber in dem HErrn, daß auch ich selbst bald
kommen werde. \bibverse{25} Ich hab's aber für nötig angesehen, den
Bruder Epaphroditus zu euch zu senden, der mein Gehilfe und Mitstreiter
und euer Apostel und meiner Notdurft Diener ist, \bibverse{26} sintemal
er nach euch allen Verlangen hatte und war hoch bekümmert darum, daß ihr
gehöret hattet, daß er krank war gewesen. \bibverse{27} Und er war zwar
todkrank, aber GOtt hat sich über ihn erbarmet, nicht allein aber über
ihn, sondern auch über mich, auf daß ich, nicht eine Traurigkeit über
die andere hätte. \bibverse{28} Ich habe ihn aber desto eilender
gesandt, auf daß ihr ihn sehet und wieder fröhlich werdet, und ich auch
der Traurigkeit weniger habe. \bibverse{29} So nehmet ihn nun auf in dem
HErrn mit allen Freuden und habt solche in Ehren. \bibverse{30} Denn um
des Werks Christi willen ist er dem Tode so nahe kommen, da er sein
Leben gering bedachte, auf daß er mir dienete an eurer Statt.

\hypertarget{section-2}{%
\section{3}\label{section-2}}

\bibverse{1} Weiter, liebe Brüder, freuet euch in dem HErrn! Daß ich
euch immer einerlei schreibe, verdrießt mich nicht und macht euch desto
gewisser. \bibverse{2} Sehet auf die Hunde, sehet auf die bösen
Arbeiter, sehet auf die Zerschneidung! \bibverse{3} Denn wir sind die
Beschneidung; die wir GOtt im Geist dienen und rühmen uns von Christo
JEsu und verlassen uns nicht auf Fleisch. \bibverse{4} Wiewohl ich auch
habe, daß ich mich Fleisches rühmen möchte. So ein anderer sich dünken
lässet, er möge sich Fleisches rühmen, ich viel mehr, \bibverse{5} der
ich am achten Tage beschnitten bin, einer aus dem Volk von Israel, des
Geschlechts Benjamin, ein Ebräer aus den Ebräern und nach dem Gesetz ein
Pharisäer, \bibverse{6} nach dem Eifer ein Verfolger der Gemeinde, nach
der Gerechtigkeit im Gesetz gewesen unsträflich. \bibverse{7} Aber was
mir Gewinn war, das habe ich um Christi willen für Schaden geachtet.
\bibverse{8} Denn ich achte es alles für Schaden gegen die
überschwengliche Erkenntnis Christi JEsu, meines HErrn, um welches
willen ich alles habe für Schaden gerechnet und achte es für Dreck, auf
daß ich Christum gewinne \bibverse{9} und in ihm erfunden werde, daß ich
nicht habe meine Gerechtigkeit, die aus dem Gesetz, sondern die durch
den Glauben an Christum kommt, nämlich die Gerechtigkeit, die von GOtt
dem Glauben zugerechnet wird, \bibverse{10} zu erkennen ihn und die
Kraft seiner Auferstehung und die Gemeinschaft seiner Leiden, daß ich
seinem Tode ähnlich werde, \bibverse{11} damit ich entgegenkomme zur
Auferstehung der Toten. \bibverse{12} Nicht daß ich's schon ergriffen
habe oder schon vollkommen sei; ich jage ihm aber nach, ob ich's auch
ergreifen möchte, nachdem ich von Christo JEsu ergriffen bin.
\bibverse{13} Meine Brüder, ich schätze mich selbst noch nicht, daß
ich's ergriffen habe. Eines aber sage ich: Ich vergesse, was dahinten
ist, und strecke mich zu dem, das da vorne ist, \bibverse{14} und jage
nach dem vorgesteckten Ziel, nach dem Kleinod, welches vorhält die
himmlische Berufung GOttes in Christo JEsu. \bibverse{15} Wieviel nun
unser vollkommen sind, die lasset uns also gesinnet sein. Und sollt ihr
sonst etwas halten, das lasset euch GOtt offenbaren, \bibverse{16} doch
so ferne, daß wir nach einer Regel, darein wir kommen sind, wandeln und
gleichgesinnet seien. \bibverse{17} Folget mir, liebe Brüder, und sehet
auf die, die also wandeln, wie ihr uns habt zum Vorbilde. \bibverse{18}
Denn viele wandeln, von welchen ich euch oft gesagt habe, nun aber sage
ich auch mit Weinen: Die Feinde des Kreuzes Christi; \bibverse{19}
welcher Ende ist die Verdammnis, welchen der Bauch ihr GOtt ist, und
ihre Ehre zuschanden wird, derer, die irdisch gesinnet sind.
\bibverse{20} Unser Wandel aber ist im Himmel von dannen wir auch warten
des Heilandes JEsu Christi, des HErrn, \bibverse{21} welcher unsern
nichtigen Leib verklären wird, daß er ähnlich werde seinem verklärten
Leibe, nach der Wirkung, damit er kann auch alle Dinge ihm untertänig
machen.{]}

\hypertarget{section-3}{%
\section{4}\label{section-3}}

\bibverse{1} Also, meine lieben und gewünschten Brüder, meine Freude und
meine Krone, bestehet also in dem HErrn, ihr Lieben! \bibverse{2} Die
Evodia ermahne ich, und die Syntyche ermahne ich, daß sie eines Sinnes
seien in dem HErrn. \bibverse{3} Ja, ich bitte auch dich, mein treuer
Geselle, stehe ihnen bei, die samt mir über dem Evangelium gekämpft
haben mit Clemens und den andern meinen Gehilfen, welcher Namen sind in
dem Buch des Lebens. \bibverse{4} Freuet euch in dem HErrn allewege; und
abermal sage ich: Freuet euch! \bibverse{5} Eure Lindigkeit lasset kund
sein allen Menschen. Der HErr ist nahe. \bibverse{6} Sorget nichts,
sondern in allen Dingen lasset eure Bitte im Gebet und Flehen mit
Danksagung vor GOtt kund werden. \bibverse{7} Und der Friede GOttes,
welcher höher ist denn alle Vernunft, bewahre eure Herzen und Sinne in
Christo JEsu!{]} \bibverse{8} Weiter, liebe Brüder, was wahrhaftig ist,
was ehrbar, was gerecht, was keusch, was lieblich, was wohl lautet, ist
etwa eine Tugend, ist etwa ein Lob, dem denket nach. \bibverse{9}
Welches ihr auch gelernet und empfangen und gehöret und gesehen habt an
mir, das tut, so wird der HErr des Friedens mit euch sein. \bibverse{10}
Ich bin aber hoch erfreuet in dem HErrn, daß ihr wieder wacker worden
seid, für mich zu sorgen, wiewohl ihr allewege gesorget habt; aber die
Zeit hat's nicht wollen leiden. \bibverse{11} Nicht sage ich das des
Mangels halben; denn ich habe gelernet, bei welchen ich bin, mir genügen
lassen. \bibverse{12} Ich kann niedrig sein und kann hoch sein; ich bin
in allen Dingen und bei allen geschickt, beide, satt sein und hungern,
beide, übrig haben und Mangel leiden. \bibverse{13} Ich vermag alles
durch den, der mich mächtig macht, Christus. \bibverse{14} Doch ihr habt
wohl getan, daß ihr euch meiner Trübsal angenommen habt. \bibverse{15}
Ihr aber von Philippi wisset, daß von Anfang des Evangeliums, da ich
auszog aus Mazedonien, keine Gemeinde mit mir geteilet hat nach der
Rechnung der Ausgabe und Einnahme denn ihr alleine. \bibverse{16} Denn
gen Thessalonich sandtet ihr zu meiner Notdurft einmal und danach aber
einmal. \bibverse{17} Nicht, daß ich das Geschenk suche, sondern ich
suche die Frucht daß sie überflüssig in eurer Rechnung sei.
\bibverse{18} Denn ich habe alles und habe überflüssig. Ich bin
erfüllet, da ich empfing durch Epaphroditus, was von euch kam, ein süßer
Geruch, ein angenehm Opfer, GOtt gefällig. \bibverse{19} Mein GOtt aber
erfülle alle eure Notdurft nach seinem Reichtum in der Herrlichkeit in
Christo JEsu! \bibverse{20} Dem GOtt aber und unserm Vater sei Ehre von
Ewigkeit zu Ewigkeit! Amen. \bibverse{21} Grüßet alle Heiligen in
Christo JEsu. Es grüßen euch die Brüder, die bei mir sind. \bibverse{22}
Es grüßen euch alle Heiligen, sonderlich aber die von des Kaisers Hause.
\bibverse{23} Die Gnade unsers HErrn JEsu Christi sei mit euch allen!
Amen.
