\hypertarget{section}{%
\section{1}\label{section}}

\bibverse{1} Paulus, ein Knecht GOttes, aber ein Apostel JEsu Christi
nach dem Glauben der Auserwählten GOttes und der Erkenntnis der Wahrheit
zur Gottseligkeit, \bibverse{2} auf Hoffnung des ewigen Lebens, welches
verheißen hat, der nicht lüget, GOtt, vor den Zeiten der Welt,
\bibverse{3} hat aber offenbaret zu seiner Zeit sein Wort durch die
Predigt, die mir vertrauet ist nach dem Befehl GOttes, unsers Heilandes:
\bibverse{4} Titus, meinem rechtschaffenen Sohn, nach unser beider
Glauben: Gnade, Barmherzigkeit, Friede von GOtt dem Vater und dem HErrn
JEsu Christo, unserm Heilande. \bibverse{5} Derhalben ließ ich dich in
Kreta, daß du solltest vollends anrichten, da ich's gelassen habe und
besetzen die Städte hin und her mit Ältesten, wie ich dir befohlen habe;
\bibverse{6} wo einer ist untadelig, eines Weibes Mann, der gläubige
Kinder habe, nicht berüchtiget, daß sie Schwelger und ungehorsam sind.
\bibverse{7} Denn ein Bischof soll untadelig sein, als ein Haushalter
GOttes, nicht eigensinnig, nicht zornig, nicht ein Weinsäufer, nicht
pochen, nicht unehrliche Hantierung treiben, \bibverse{8} sondern
gastfrei, gütig, züchtig, gerecht, heilig, keusch \bibverse{9} und halte
ob dem Wort, das gewiß ist und lehren kann, auf daß er mächtig sei, zu
ermahnen durch die heilsame Lehre und zu strafen die Widersprecher.
\bibverse{10} Denn es sind viele freche und unnütze Schwätzer und
Verführer, sonderlich die aus der Beschneidung, \bibverse{11} welchen
man muß das Maul stopfen, die da ganze Häuser verkehren und lehren, was
nicht taugt, um schändliches Gewinns willen. \bibverse{12} Es hat einer
aus ihnen gesagt, ihr eigener Prophet: Die Kreter sind immer Lügner,
böse Tiere und faule Bäuche. \bibverse{13} Dies Zeugnis ist wahr. Um der
Sache willen strafe sie scharf, auf daß sie gesund seien im Glauben
\bibverse{14} und nicht achten auf die jüdischen Fabeln und
Menschengebote, welche sich von der Wahrheit abwenden. \bibverse{15} Den
Reinen ist alles rein; den Unreinen aber und Ungläubigen ist nichts
rein, sondern unrein ist beides, ihr Sinn und Gewissen. \bibverse{16}
Sie sagen, sie erkennen GOtt; aber mit den Werken verleugnen sie es,
sintemal sie sind, an welchen GOtt Greuel hat, und gehorchen nicht und
sind zu allem guten Werk untüchtig.

\hypertarget{section-1}{%
\section{2}\label{section-1}}

\bibverse{1} Du aber rede, wie sich's ziemet nach der heilsamen Lehre:
\bibverse{2} den Alten, daß sie nüchtern seien, ehrbar, züchtig, gesund
im Glauben, in der Liebe, in der Geduld; \bibverse{3} den alten Weibern
desselbigengleichen, daß sie sich stellen, wie den Heiligen ziemet,
nicht Lästerinnen seien, nicht Weinsäuferinnen, gute Lehrerinnen,
\bibverse{4} daß sie die jungen Weiber lehren züchtig sein, ihre Männer
lieben, Kinder lieben, \bibverse{5} sittig sein, keusch, häuslich,
gütig, ihren Männern untertan, auf daß nicht das Wort GOttes verlästert
werde. \bibverse{6} Desselbigengleichen die jungen Männer ermahne, daß
sie züchtig seien. \bibverse{7} Allenthalben aber stelle dich selbst zum
Vorbilde guter Werke mit unverfälschter Lehre, mit Ehrbarkeit,
\bibverse{8} mit heilsamem und untadeligem Wort, auf daß der
Widerwärtige sich schäme und nichts habe, daß er von uns möge Böses
sagen. \bibverse{9} Den Knechten daß sie ihren Herren untertänig seien,
in allen Dingen zu Gefallen tun, nicht widerbellen, \bibverse{10} nicht
veruntreuen, sondern alle gute Treue erzeigen, auf daß sie die Lehre
GOttes, unsers Heilandes, zieren in allen Stücken. \bibverse{11} Denn es
ist erschienen die heilsame Gnade GOttes allen Menschen \bibverse{12}
und züchtiget uns, daß wir sollen verleugnen das ungöttliche Wesen und
die weltlichen Lüste und züchtig, gerecht und gottselig leben in dieser
Welt \bibverse{13} und warten auf die selige Hoffnung und Erscheinung
der Herrlichkeit des großen GOttes und unsers Heilandes JEsu Christi,
\bibverse{14} der sich selbst für uns gegeben bat, auf daß er uns
erlösete von aller Ungerechtigkeit und reinigte sich selbst ein Volk zum
Eigentum, das fleißig wäre zu guten Werken. \bibverse{15} Solches rede
und ermahne und strafe mit ganzem Ernst. Laß dich niemand verachten!

\hypertarget{section-2}{%
\section{3}\label{section-2}}

\bibverse{1} Erinnere sie, daß sie den Fürsten und der Obrigkeit
untertan und gehorsam seien, zu allem guten Werk bereit seien,
\bibverse{2} niemand lästern, nicht hadern, gelinde seien, alle
Sanftmütigkeit beweisen gegen alle Menschen. \bibverse{3} Denn wir waren
auch weiland unweise, ungehorsam, irrig, dienend den Lüsten und
mancherlei Wollüsten und wandelten in Bosheit und Neid und hasseten uns
untereinander. \bibverse{4} Da aber erschien die Freundlichkeit und
Leutseligkeit GOttes, unsers Heilandes, \bibverse{5} nicht um der Werke
willen der Gerechtigkeit, die wir getan hatten, sondern nach seiner
Barmherzigkeit machte er uns selig durch das Bad der Wiedergeburt und
Erneuerung des Heiligen Geistes, \bibverse{6} welchen er,ausgegossen hat
über uns reichlich durch JEsum Christum, unsern Heiland, \bibverse{7}
auf daß wir durch desselbigen Gnade gerecht und Erben seien des ewigen
Lebens nach der Hoffnung. \bibverse{8} Das ist je gewißlich wahr.
Solches will ich, daß du fest lehrest, auf daß die, so an GOtt gläubig
sind worden, in einem Stand guter Werke funden werden. Solches ist gut
und nütze den Menschen. \bibverse{9} Der törichten Fragen aber, der
Geschlechtsregister, des Zankes und Streites über dem Gesetz entschlage
dich; denn sie sind unnütz und eitel. \bibverse{10} Einen ketzerischen
Menschen meide, wenn er einmal und abermal ermahnet ist, \bibverse{11}
und wisse, daß ein solcher verkehrt ist und sündiget, als der sich
selbst verurteilet hat. \bibverse{12} Wenn ich zu dir senden werde
Artemas oder Tychikus, so komm eilend zu mir gen Nikopolis; denn
daselbst habe ich beschlossen, den Winter zu bleiben. \bibverse{13}
Zenäs, den Schriftgelehrten, und Apollos fertige ab mit Fleiß, auf daß
ihnen nichts gebreche. \bibverse{14} Laß aber auch die Unsern lernen,
daß sie im Stand guter Werke sich finden lassen, wo man ihrer bedarf,
auf daß sie nicht unfruchtbar seien: \bibverse{15} Es grüßen dich alle,
die mit mir sind. Grüße alle, die uns lieben im Glauben. Die Gnade sei
mit euch allen! Amen.
