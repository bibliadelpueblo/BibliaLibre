\hypertarget{section}{%
\section{1}\label{section}}

\bibverse{1} Paulus, berufen zum Apostel JEsu Christi durch den Willen
GOttes, und Bruder Sosthenes: \bibverse{2} Der Gemeinde GOttes zu
Korinth, den Geheiligten in Christo JEsu, den berufenen Heiligen samt
allen denen, die anrufen den Namen unsers HErrn JEsu Christi an allen
ihren und unsern Orten. \bibverse{3} Gnade sei mit euch und Friede von
GOtt, unserm Vater, und dem HErrn JEsu Christo! \bibverse{4} Ich danke
meinem GOtt allezeit eurethalben für die Gnade GOttes, die euch gegeben
ist in Christo JEsu, \bibverse{5} daß ihr seid durch ihn an allen
Stücken reich gemacht, an aller Lehre und in aller Erkenntnis
\bibverse{6} wie denn die Predigt von Christo in euch kräftig worden
ist, \bibverse{7} also daß ihr keinen Mangel habt an irgendeiner Gabe
und wartet nur auf die Offenbarung unsers HErrn JEsu Christi.
\bibverse{8} welcher auch wird euch fest behalten bis ans Ende, daß ihr
unsträflich seid auf den Tag unsers HErrn JEsu Christi. \bibverse{9}
Denn GOtt ist treu, durch welchen ihr berufen seid zur Gemeinschaft
seines Sohnes JEsu Christi, unsers HErrn. \bibverse{10} Ich ermahne euch
aber, liebe Brüder, durch den Namen unsers HErrn JEsu Christi; daß ihr
allzumal einerlei Rede führet und lasset nicht Spaltungen unter euch
sein, sondern haltet fest aneinander in einem Sinn und in einerlei
Meinung. \bibverse{11} Denn mir ist vorkommen, liebe Brüder, durch die
aus Chloes Gesinde von euch, daß Zank unter euch sei. \bibverse{12} Ich
sage aber davon, daß unter euch einer spricht: Ich bin paulisch; der
andere: Ich bin apollisch; der dritte: Ich bin kephisch; der vierte: Ich
bin christisch. \bibverse{13} Wie? ist Christus nun zertrennet? Ist denn
Paulus für euch gekreuziget, oder seid ihr auf des Paulus Namen getauft?
\bibverse{14} Ich danke GOtt, daß ich niemand unter euch getauft habe
außer Crispus und Gajus, \bibverse{15} daß nicht jemand sagen möge, ich
hätte auf meinen Namen getauft. \bibverse{16} Ich habe aber auch getauft
des Stephanas Hausgesinde; danach weiß ich nicht, ob ich etliche andere
getauft habe. \bibverse{17} Denn Christus hat mich nicht gesandt zu
taufen, sondern das Evangelium zu predigen, nicht mit klugen Worten, auf
daß nicht das Kreuz Christi zunichte werde. \bibverse{18} Denn das Wort
vom Kreuz ist eine Torheit denen, die verloren werden; uns aber, die wir
selig werden, ist es eine Gotteskraft. \bibverse{19} Denn es stehet
geschrieben: Ich will zunichte machen die Weisheit der Weisen, und den
Verstand der Verständigen will ich verwerfen. \bibverse{20} Wo sind die
Klugen? Wo sind die Schriftgelehrten? Wo sind die Weltweisen? Hat nicht
GOtt die Weisheit dieser Welt zur Torheit gemacht? \bibverse{21} Denn
dieweil die Welt durch ihre Weisheit GOtt in seiner Weisheit nicht
erkannte, gefiel es GOtt wohl, durch törichte Predigt selig zu machen
die, so daran glauben, \bibverse{22} sintemal die Juden Zeichen fordern,
und die Griechen nach Weisheit fragen. \bibverse{23} Wir aber predigen
den gekreuzigten Christum, den Juden ein Ärgernis und den Griechen eine
Torheit. \bibverse{24} Denen aber, die berufen sind, beide, Juden und
Griechen, predigen wir Christum göttliche Kraft und göttliche Weisheit.
\bibverse{25} Denn die göttliche Torheit ist weiser, denn die Menschen
sind, und die göttliche Schwachheit ist stärker, denn die Menschen sind.
\bibverse{26} Sehet an, liebe Brüder, euren Beruf: nicht viel Weise nach
dem Fleisch, nicht viel Gewaltige, nicht viel Edle sind berufen.
\bibverse{27} sondern was töricht ist vor der Welt, das hat GOtt
erwählet, daß er die Weisen zuschanden machte; und was schwach ist vor
der Welt, das hat GOtt erwählet, daß er zuschanden machte, was stark
ist; \bibverse{28} und das Unedle vor der Welt und das Verachtete hat
GOtt erwählet, und das da nichts ist, daß er zunichte machte, was etwas
ist, \bibverse{29} auf daß sich vor ihm kein Fleisch rühme.
\bibverse{30} Von welchem auch ihr herkommt in Christo JEsu, welcher uns
gemacht ist von GOtt zur Weisheit und zur Gerechtigkeit und zur
Heiligung und zur Erlösung, \bibverse{31} auf daß (wie geschrieben
stehet), wer sich rühmet, der rühme sich des HErrn.

\hypertarget{section-1}{%
\section{2}\label{section-1}}

\bibverse{1} Und ich, liebe Brüder, da ich zu euch kam, kam ich nicht
mit hohen Worten oder hoher Weisheit, euch zu verkündigen die göttliche
Predigt. \bibverse{2} Denn ich hielt mich nicht dafür, daß ich etwas
wüßte unter euch ohne allein JEsum Christum, den Gekreuzigten.
\bibverse{3} Und ich war bei euch mit Schwachheit und mit Furcht und mit
großem Zittern. \bibverse{4} Und mein Wort und meine Predigt war nicht
in vernünftigen Reden menschlicher Weisheit, sondern in Beweisung des
Geistes und der Kraft, \bibverse{5} auf daß euer Glaube bestehe nicht
auf Menschenweisheit, sondern auf GOttes Kraft. \bibverse{6} Wovon wir
aber reden, das ist dennoch Weisheit bei den Vollkommenen; nicht eine
Weisheit dieser Welt, auch nicht der Obersten dieser Welt, welche
vergehen; \bibverse{7} sondern wir reden von der heimlichen, verborgenen
Weisheit GOttes, welche GOtt verordnet hat vor der Welt zu unserer
Herrlichkeit, \bibverse{8} welche keiner von den Obersten dieser Welt
erkannt hat; denn wo sie die erkannt hätten, hätten sie den HErrn der
Herrlichkeit nicht gekreuziget; \bibverse{9} sondern wie geschrieben
stehet: Das kein Auge gesehen hat und kein Ohr gehöret hat und in keines
Menschen Herz kommen ist, das GOtt bereitet hat denen, die ihn lieben.
\bibverse{10} Uns aber hat es GOtt offenbaret durch seinen Geist; denn
der Geist erforschet alle Dinge, auch die Tiefen der Gottheit.
\bibverse{11} Denn welcher Mensch weiß, was im Menschen ist, ohne der
Geist des Menschen, der in ihm ist? Also auch weiß niemand, was in GOtt
ist, ohne der Geist GOttes. \bibverse{12} Wir aber haben nicht empfangen
den Geist der Welt, sondern den Geist aus GOtt, daß wir wissen können,
was uns von GOtt gegeben ist. \bibverse{13} Welches wir auch reden,
nicht mit Worten, welche menschliche Weisheit lehren kann, sondern mit
Worten, die der Heilige Geist lehret, und richten geistliche Sachen
geistlich. \bibverse{14} Der natürliche Mensch aber vernimmt nichts vom
Geist GOttes; es ist ihm eine Torheit, und kann es nicht erkennen; denn
es muß geistlich gerichtet sein. \bibverse{15} Der Geistliche aber
richtet alles und wird von niemand gerichtet. \bibverse{16} Denn wer hat
des HErrn Sinn erkannt, oder wer will ihn unterweisen? Wir aber haben
Christi Sinn.

\hypertarget{section-2}{%
\section{3}\label{section-2}}

\bibverse{1} Und ich, liebe Brüder, konnte nicht mit euch reden als mit
Geistlichen, sondern als mit Fleischlichen, wie mit jungen Kindern in
Christo. \bibverse{2} Milch habe ich euch zu trinken gegeben und nicht
Speise; denn ihr konntet noch nicht; auch könnt ihr noch jetzt nicht,
\bibverse{3} dieweil ihr noch fleischlich seid. Denn sintemal Eifer und
Zank und Zwietracht unter euch sind, seid ihr denn nicht fleischlich und
wandelt nach menschlicher Weise? \bibverse{4} Denn so einer sagt: Ich
bin paulisch, der andere aber: Ich bin apollisch, seid ihr denn nicht
fleischlich? \bibverse{5} Wer ist nun Paulus? Wer ist Apollo? Diener
sind sie, durch welche ihr seid gläubig worden, und dasselbige, wie der
HErr einem jeglichen gegeben hat. \bibverse{6} Ich habe gepflanzet,
Apollo hat begossen, aber GOtt hat das Gedeihen gegeben. \bibverse{7} So
ist nun weder der da pflanzet, noch der da begießt, etwas, sondern GOtt,
der das Gedeihen gibt. \bibverse{8} Der aber pflanzet und der da
begießt, ist einer wie der andere. Ein jeglicher aber wird seinen Lohn
empfangen nach seiner Arbeit. \bibverse{9} Denn wir sind GOttes
Mitarbeiter; ihr seid GOttes Ackerwerk und GOttes Gebäu. \bibverse{10}
Ich von GOttes Gnaden, die mir gegeben ist, habe den Grund gelegt als
ein weiser Baumeister; ein anderer bauet darauf. Ein jeglicher aber sehe
zu, wie er darauf baue. \bibverse{11} Einen andern Grund kann zwar
niemand legen außer dem, der gelegt ist, welcher ist JEsus Christus.
\bibverse{12} So aber jemand auf diesen Grund bauet Gold, Silber,
Edelsteine, Holz, Heu, Stoppeln, \bibverse{13} so wird eines jeglichen
Werk offenbar werden; der Tag wird's klar machen. Denn es wird durchs
Feuer offenbar werden, und welcherlei eines jeglichen Werk sei, wird das
Feuer bewähren. \bibverse{14} Wird jemandes Werk bleiben, das er darauf
gebauet hat, so wird er Lohn empfangen. \bibverse{15} Wird aber jemandes
Werk verbrennen, so wird er des Schaden leiden; er selbst aber wird
selig werden, so doch wie durchs Feuer. \bibverse{16} Wisset ihr nicht,
daß ihr GOttes Tempel seid, und der Geist GOttes in euch wohnet?
\bibverse{17} So jemand den Tempel GOttes verderbet, den wird GOtt
verderben; denn der Tempel GOttes ist heilig; der seid ihr.
\bibverse{18} Niemand betrüge sich selbst! Welcher sich unter euch
dünkt, weise zu sein, der werde ein Narr in dieser Welt, daß er möge
weise sein. \bibverse{19} Denn dieser Welt Weisheit ist Torheit bei
GOtt. Denn es stehet geschrieben: Die Weisen erhaschet er in ihrer
Klugheit. \bibverse{20} Und abermal: Der HErr weiß der Weisen Gedanken,
daß sie eitel sind. \bibverse{21} Darum rühme sich niemand eines
Menschen! Es ist alles euer, \bibverse{22} es sei Paulus oder Apollo, es
sei Kephas oder die Welt, es sei das Leben oder der Tod, es sei das
Gegenwärtige oder das Zukünftige: alles ist euer. \bibverse{23} Ihr aber
seid Christi; Christus aber ist GOttes.

\hypertarget{section-3}{%
\section{4}\label{section-3}}

\bibverse{1} Dafür halte uns jedermann, nämlich für Christi Diener und
Haushalter über GOttes Geheimnisse. \bibverse{2} Nun sucht man nicht
mehr an den Haushaltern, denn daß sie treu erfunden werden. \bibverse{3}
Mir aber ist's ein Geringes, daß ich von euch gerichtet werde oder von
einem menschlichen Tage; auch richte ich mich selbst nicht. \bibverse{4}
Ich bin mir wohl nichts bewußt, aber darinnen bin ich nicht
gerechtfertiget; der HErr ist's aber, der mich richtet. \bibverse{5}
Darum richtet nicht vor der Zeit, bis der HErr komme, welcher auch wird
ans Licht bringen, was im Finstern verborgen ist, und den Rat der Herzen
offenbaren; alsdann wird einem jeglichen von GOtt Lob widerfahren.
\bibverse{6} Solches aber, liebe Brüder, habe ich auf mich und Apollo
gedeutet um euretwillen, daß ihr an uns lernet, daß niemand höher von
sich halte, denn jetzt geschrieben ist, auf daß sich nicht einer wider
den andern um jemandes willen aufblase. \bibverse{7} Denn wer hat dich
vorgezogen? Was hast du aber, das du nicht empfangen hast? So du es aber
empfangen hast, was rühmest du dich denn, als der es nicht empfangen
hätte? \bibverse{8} Ihr seid schon satt worden; ihr seid schon reich
worden; ihr herrschet ohne uns. Und wollte GOtt, ihr herrschet, auf daß
auch wir mit euch herrschen möchten. \bibverse{9} Ich halte aber, GOtt
habe uns Apostel für die Allergeringsten dargestellet, als dem Tode
übergeben. Denn wir sind ein Schauspiel worden der Welt und den Engeln
und den Menschen. \bibverse{10} Wir sind Narren um Christi willen, ihr
aber seid klug in Christo; wir schwach, ihr aber stark; ihr herrlich,
wir aber verachtet. \bibverse{11} Bis auf diese Stunde leiden wir Hunger
und Durst und sind nackend und werden geschlagen und haben keine gewisse
Stätte \bibverse{12} und arbeiten und wirken mit unsern eigenen Händen.
Man schilt uns, so segnen wir; man verfolgt uns, so dulden wir's, man
lästert uns, so flehen wir. \bibverse{13} Wir sind stets als ein Fluch
der Welt und ein Fegopfer aller Leute. \bibverse{14} Nicht schreibe ich
solches, daß ich euch beschäme, sondern ich ermahne euch als meine
lieben Kinder. \bibverse{15} Denn ob ihr gleich zehntausend Zuchtmeister
hättet in Christo, so habt ihr doch nicht viele Väter. Denn ich habe
euch gezeuget in Christo JEsu durch das Evangelium. \bibverse{16} Darum
ermahne ich euch: Seid meine Nachfolger! \bibverse{17} Aus derselben
Ursache habe ich Timotheus zu euch gesandt, welcher ist mein lieber und
getreuer Sohn in dem HErrn, daß er euch erinnere meiner Wege, die da in
Christo sind, gleichwie ich an allen Enden in allen Gemeinden lehre.
\bibverse{18} Es blähen sich etliche auf, als würde ich nicht zu euch
kommen. \bibverse{19} Ich will aber gar kürzlich zu euch kommen, so der
HErr will, und erlernen nicht die Worte der Aufgeblasenen, sondern die
Kraft. \bibverse{20} Denn das Reich GOttes stehet nicht in Worten,
sondern in Kraft. \bibverse{21} Was wollet ihr? Soll ich mit der Rute zu
euch kommen oder mit Liebe und sanftmütigem Geist?

\hypertarget{section-4}{%
\section{5}\label{section-4}}

\bibverse{1} Es gehet ein gemein Geschrei, daß Hurerei unter euch ist,
und eine solche Hurerei, da auch die Heiden nicht von zu sagen wissen,
daß einer seines Vaters Weib habe. \bibverse{2} Und ihr seid aufgeblasen
und habt nicht vielmehr Leid getragen, auf daß, der das Werk getan hat,
von euch getan würde. \bibverse{3} Ich zwar, als der ich mit dem Leibe
nicht da bin, doch mit dem Geist gegenwärtig, habe schon als gegenwärtig
beschlossen über den, den solches also getan hat: \bibverse{4} in dem
Namen unsers HErrn JEsu Christi, in eurer Versammlung mit meinem Geist
und mit der Kraft unsers HErrn JEsu Christi. \bibverse{5} ihn zu
übergeben dem Satan zum Verderben des Fleisches, auf daß der Geist selig
werde am Tage des HErrn JEsu. \bibverse{6} Euer Ruhm ist nicht fein.
Wisset ihr nicht, daß ein wenig Sauerteig den ganzen Teig versäuert?
\bibverse{7} Darum feget den alten Sauerteig aus, auf daß ihr ein neuer
Teig seid, gleichwie ihr ungesäuert seid. Denn wir haben auch ein
Osterlamm, das ist Christus, für uns geopfert. \bibverse{8} Darum lasset
uns Ostern halten, nicht im alten Sauerteig, auch nicht im Sauerteig der
Bosheit und Schalkheit, sondern in dem Süßteig der Lauterkeit und der
Wahrheit. \bibverse{9} Ich habe euch geschrieben in dem Briefe, daß ihr
nichts sollet zu schaffen haben mit den Hurern. \bibverse{10} Das meine
ich gar nicht von den Hurern in dieser Welt oder von den Geizigen oder
von den Räubern oder von den Abgöttischen; sonst müßtet ihr die Welt
räumen. \bibverse{11} Nun aber habe ich euch geschrieben, ihr sollet
nichts mit ihnen zu schaffen haben; nämlich, so jemand ist, der sich
lässet einen Bruder nennen, und ist ein Hurer oder ein Geiziger oder ein
Abgöttischer oder ein Lästerer oder ein Trunkenbold oder ein Räuber, mit
demselbigen sollet ihr auch nicht essen. \bibverse{12} Denn was gehen
mich die draußen an, daß ich sie sollte richten? Richtet ihr nicht, die
da drinnen sind? \bibverse{13} GOtt aber wird, die draußen sind,
richten. Tut von euch selbst hinaus, wer da böse ist!

\hypertarget{section-5}{%
\section{6}\label{section-5}}

\bibverse{1} Wie darf jemand unter euch, so er einen Handel hat mit
einem andern, hadern vor den Ungerechten und nicht vor den Heiligen?
\bibverse{2} Wisset ihr nicht, daß die Heiligen die Welt richten werden?
So denn nun die Welt soll von euch gerichtet werden, seid ihr denn nicht
gut genug, geringere Sachen zu richten? \bibverse{3} Wisset ihr nicht,
daß wir über die Engel richten werden? wieviel mehr über die zeitlichen
Güter! \bibverse{4} Ihr aber, wenn ihr über zeitlichen Gütern Sachen
habt, so nehmet ihr die, so bei der Gemeinde verachtet sind, und setzet
sie zu Richtern. \bibverse{5} Euch zur Schande muß ich das sagen. Ist so
gar kein Weiser unter euch oder doch nicht einer, der da könnte richten
zwischen Bruder und Bruder? \bibverse{6} Sondern ein Bruder mit dem
andern hadert, dazu vor den Ungläubigen. \bibverse{7} Es ist schon ein
Fehl unter euch, daß ihr miteinander rechtet. Warum lasset ihr euch
nicht viel lieber unrecht tun? Warum lasset ihr euch nicht viel lieber
übervorteilen? \bibverse{8} Sondern ihr tut unrecht und übervorteilet,
und solches an den Brüdern. \bibverse{9} Wisset ihr nicht, daß die
Ungerechten werden das Reich GOttes nicht ererben? Lasset euch nicht
verführen: weder die Hurer noch die Abgöttischen noch die Ehebrecher
noch die Weichlinge noch die Knabenschänder \bibverse{10} noch die Diebe
noch die Geizigen noch die Trunkenbolde noch die Lästerer noch die
Räuber werden das Reich GOttes ererben. \bibverse{11} Und solche sind
euer etliche gewesen; aber ihr seid abgewaschen, ihr seid geheiliget,
ihr seid gerecht worden durch den Namen des HErrn JEsu und durch den
Geist unsers GOttes. \bibverse{12} Ich hab' es alles Macht; es frommet
aber nicht alles. Ich hab' es alles Macht; es soll mich aber nichts
gefangennehmen. \bibverse{13} Die Speise dem Bauche und der Bauch der
Speise; aber GOtt wird diesen und jene hinrichten. Der Leib aber nicht
der Hurerei, sondern dem HErrn und der HErr dem Leibe. \bibverse{14}
GOtt aber hat den HErrn auferwecket und wird uns auch auferwecken durch
seine Kraft. \bibverse{15} Wisset ihr nicht, daß eure Leiber Christi
Glieder sind? Sollte ich nun die Glieder Christi nehmen und Hurenglieder
daraus machen? Das sei ferne! \bibverse{16} Oder wisset ihr nicht, daß,
wer an der Hure hanget, der ist ein Leib mit ihr? Denn sie werden
(spricht er) zwei in einem Fleische sein. \bibverse{17} Wer aber dem
HErrn anhanget, der ist ein Geist mit ihm. \bibverse{18} Fliehet die
Hurerei! Alle Sünden, die der Mensch tut, sind außer seinem Leibe; wer
aber huret, der sündiget an seinem eigenen Leibe. \bibverse{19} Oder
wisset ihr nicht, daß euer Leib ein Tempel des Heiligen Geistes ist, der
in euch ist, welchen ihr habt von GOtt, und seid nicht euer selbst?
\bibverse{20} Denn ihr seid teuer erkauft. Darum so preiset GOtt an
eurem Leibe und in eurem Geiste, welche sind GOttes.

\hypertarget{section-6}{%
\section{7}\label{section-6}}

\bibverse{1} Von dem ihr aber mir geschrieben habt, antworte ich: Es ist
dem Menschen gut, daß er kein Weib berühre. \bibverse{2} Aber um der
Hurerei willen habe ein jeglicher sein eigen Weib, und eine jegliche
habe ihren eigenen Mann. \bibverse{3} Der Mann leiste dem Weibe die
schuldige Freundschaft, desselbigengleichen das Weib dem Manne.
\bibverse{4} Das Weib ist ihres Leibes nicht mächtig, sondern der Mann.
Desselbigengleichen der Mann ist seines Leibes nicht mächtig, sondern
das Weib. \bibverse{5} Entziehe sich nicht eins dem andern, es sei denn
aus beider Bewilligung eine Zeitlang, daß ihr zum Fasten und Beten Muße
habet; und kommet wiederum zusammen; auf daß euch der Satan nicht
versuche um eurer Unkeuschheit willen. \bibverse{6} Solches sage ich
aber aus Vergunst und nicht aus Gebot. \bibverse{7} Ich wollte aber
lieber, alle Menschen wären, wie ich bin; aber ein jeglicher hat seine
eigene Gabe von GOtt, einer so, der andere so. \bibverse{8} Ich sage
zwar den Ledigen und Witwen: Es ist ihnen gut, wenn sie auch bleiben wie
ich. \bibverse{9} So sie aber sich nicht enthalten, so laß sie freien;
es ist besser freien, denn Brunst leiden. \bibverse{10} Den Ehelichen
aber gebiete nicht ich, sondern der HErr, daß das Weib sich nicht
scheide von dem Manne. \bibverse{11} So sie sich aber scheidet, daß sie
ohne Ehe bleibe oder sich mit dem Manne versöhne, und daß der Mann das
Weib nicht von sich lasse. \bibverse{12} Den andern aber sage ich, nicht
der HErr: So ein Bruder ein ungläubig Weib hat, und dieselbige läßt es
sich gefallen, bei ihm zu wohnen, der scheide sich nicht von ihr.
\bibverse{13} Und so ein Weib einen ungläubigen Mann hat, und er läßt es
sich gefallen, bei ihr zu wohnen, die scheide sich nicht von ihm.
\bibverse{14} Denn der ungläubige Mann ist geheiliget durch das Weib,
und das ungläubige Weib wird geheiliget durch den Mann. Sonst wären eure
Kinder unrein; nun aber sind sie heilig. \bibverse{15} So aber der
Ungläubige sich scheidet, so laß ihn sich scheiden. Es ist der Bruder
oder die Schwester nicht gefangen in solchen Fällen. Im Frieden aber hat
uns GOtt berufen. \bibverse{16} Was weißt du aber, du Weib, ob du den
Mann werdest selig machen? Oder du Mann was weißt du, ob du das Weib
werdest selig machen? \bibverse{17} Doch wie einem jeglichen GOtt hat
ausgeteilet. Ein jeglicher, wie ihn der HErr berufen hat, also wandele
er. Und also schaffe ich's in allen Gemeinden. \bibverse{18} Ist jemand
beschnitten berufen, der zeuge keine Vorhaut. Ist jemand berufen in der
Vorhaut, der lasse sich nicht beschneiden. \bibverse{19} Die
Beschneidung ist nichts, und die Vorhaut ist nichts, sondern GOttes
Gebote halten; \bibverse{20} Ein jeglicher bleibe in dem Beruf, darinnen
er berufen ist. \bibverse{21} Bist du als Knecht berufen, sorge dich
nicht; doch kannst du frei werden, so brauche des viel lieber.
\bibverse{22} Denn wer als Knecht berufen ist in dem HErrn, der ist ein
Gefreiter des HErrn; desselbigengleichen, wer als Freier berufen ist,
der ist ein Knecht Christi. \bibverse{23} Ihr seid teuer erkauft; werdet
nicht der Menschen Knechte! \bibverse{24} Ein jeglicher, liebe Brüder,
worinnen er berufen ist, darinnen bleibe er bei GOtt. \bibverse{25} Von
den Jungfrauen aber habe ich kein Gebot des HErrn; ich sage aber meine
Meinung, als ich Barmherzigkeit erlanget habe von dem HErrn, treu zu
sein. \bibverse{26} So meine ich nun, solches sei gut um der
gegenwärtigen Not willen, daß es dem Menschen gut sei, also zu sein.
\bibverse{27} Bist du an ein Weib gebunden, so suche nicht los zu
werden; bist du aber los vom Weibe, so suche kein Weib. \bibverse{28} So
du aber freiest, sündigest du nicht; und so eine Jungfrau freiet,
sündiget sie nicht; doch werden solche leibliche Trübsal haben. Ich
verschone aber euer gerne. \bibverse{29} Das sage ich aber, liebe
Brüder: Die Zeit ist kurz. Weiter ist das die Meinung: Die da Weiber
haben, daß sie seien, als hätten sie keine, und die da weinen, als
weineten sie nicht, \bibverse{30} und die sich freuen, als freueten sie
sich nicht, und die da kaufen, als besäßen sie es nicht, \bibverse{31}
und die diese Welt gebrauchen, daß sie dieselbige nicht mißbrauchen;
denn das Wesen dieser Welt vergehet. \bibverse{32} Ich wollte aber, daß
ihr ohne Sorge wäret. Wer ledig ist, der sorget, was dem HErrn
angehöret, wie er dem HErrn gefalle. \bibverse{33} Wer aber freiet, der
sorget, was der Welt angehöret, wie er dem Weibe gefalle. Es ist ein
Unterschied zwischen einem Weibe und einer Jungfrau. \bibverse{34}
Welche nicht freiet, die sorget, was dem HErrn angehöret, daß sie heilig
sei, beide, am Leibe und auch am Geist; die aber freiet, die sorget, was
der Welt angehöret, wie sie dem Manne gefalle. \bibverse{35} Solches
aber sage ich zu eurem Nutz; nicht daß ich euch einen Strick an den Hals
werfe, sondern dazu, daß es fein ist, und ihr stets und unverhindert dem
HErrn dienen könnet. \bibverse{36} So aber jemand sich lässet dünken, es
wolle sich nicht schicken mit seiner Jungfrau, weil sie eben wohl
mannbar ist, und es will nicht anders sein, so tue er, was er will; er
sündiget nicht, er lasse sie freien. \bibverse{37} Wenn einer aber sich
fest vornimmt, weil er ungezwungen ist und seinen freien Willen hat, und
beschließt solches in seinem Herzen, seine Jungfrau also bleiben zu
lassen, der tut wohl. \bibverse{38} Endlich, welcher verheiratet, der
tut wohl; welcher aber nicht verheiratet, der tut besser. \bibverse{39}
Ein Weib ist gebunden an das Gesetz, solange ihr Mann lebet; so aber ihr
Mann entschläft, ist sie frei, sich zu verheiraten, welchem sie will;
allein, daß es in dem HErrn geschehe. \bibverse{40} Seliger ist sie
aber, wo sie also bleibet, nach meiner Meinung. Ich halte aber, ich habe
auch den Geist GOttes.

\hypertarget{section-7}{%
\section{8}\label{section-7}}

\bibverse{1} Von dem Götzenopfer aber wissen wir; denn wir haben alle
das Wissen. Das Wissen bläset auf; aber die Liebe bessert. \bibverse{2}
So aber sich jemand dünken lässet, er wisse etwas, der weiß noch nichts,
wie er wissen soll. \bibverse{3} So aber jemand GOtt liebet, derselbige
ist von ihm erkannt. \bibverse{4} So wissen wir nun von der Speise des
Götzenopfers, daß ein Götze nichts in der Welt sei, und daß kein anderer
GOtt sei ohne der einige. \bibverse{5} Und wiewohl es sind, die Götter
genannt werden, es sei, im Himmel oder auf Erden, sintemal es sind viel
Götter und viel Herren: \bibverse{6} so haben wir doch nur einen GOtt,
den Vater, von welchem alle Dinge sind und wir in ihm, und einen HErrn,
JEsum Christum, durch welchen alle Dinge sind und wir durch ihn.
\bibverse{7} Es hat aber nicht jedermann das Wissen. Denn etliche machen
sich noch ein Gewissen über dem Götzen und essen es für Götzenopfer;
damit wird ihr Gewissen, weil es so schwach ist, beflecket. \bibverse{8}
Aber die Speise fördert uns nicht vor GOtt. Essen wir, so werden wir
darum nicht besser sein; essen wir nicht, so werden wir darum nichts
weniger sein. \bibverse{9} Sehet aber zu, daß diese eure Freiheit nicht
gerate zu einem Anstoß der Schwachen. \bibverse{10} Denn so dich, der du
die Erkenntnis hast, jemand sähe zu Tische sitzen im Götzenhause, wird
nicht sein Gewissen dieweil er schwach ist, verursacht, das Götzenopfer
zu essen? \bibverse{11} Und wird also über deiner Erkenntnis der
schwache Bruder umkommen, um welches willen doch Christus gestorben ist.
\bibverse{12} Wenn ihr aber also sündiget an den Brüdern und schlaget
ihr schwaches Gewissen, so sündiget ihr an Christo. \bibverse{13} Darum,
so die Speise meinen Bruder ärgert, wollte ich nimmermehr Fleisch essen,
auf daß ich meinen Bruder nicht ärgerte.

\hypertarget{section-8}{%
\section{9}\label{section-8}}

\bibverse{1} Bin ich nicht ein Apostel? Bin ich nicht frei? Habe ich
nicht unsern HErrn JEsum Christum gesehen? Seid nicht ihr mein Werk in
dem HErrn? \bibverse{2} Bin ich andern nicht ein Apostel, so bin ich
doch euer Apostel; denn das Siegel meines Apostelamts seid ihr in dem
HErrn. \bibverse{3} Wenn man mich fragt, so antworte ich also:
\bibverse{4} Haben wir nicht Macht zu essen und zu trinken? \bibverse{5}
Haben wir nicht auch Macht, eine Schwester zum Weibe mit umherzuführen
wie die andern Apostel und des HErrn Brüder und Kephas? \bibverse{6}
Oder haben allein ich und Barnaba nicht Macht, solches zu tun?
\bibverse{7} Welcher zieht jemals in den Krieg auf seinen eigenen Sold?
Welcher pflanzet einen Weinberg und isset nicht von seiner Frucht, oder
welcher weidet eine Herde und isset nicht von der Milch der Herde?
\bibverse{8} Rede ich aber solches auf Menschenweise? Sagt nicht solches
das Gesetz auch? \bibverse{9} Denn im Gesetz Mose's stehet geschrieben:
Du sollst dem Ochsen nicht das Maul verbinden, der da drischet. Sorget
GOtt für die Ochsen? \bibverse{10} Oder sagt er's nicht allerdinge um
unsertwillen? Denn es ist ja um unsertwillen geschrieben. Denn der da
pflüget, soll auf Hoffnung pflügen, und der da drischt, soll auf
Hoffnung dreschen, daß er seiner Hoffnung teilhaftig werde.
\bibverse{11} So wir euch das Geistliche säen, ist's ein groß Ding, ob
wir euer Leibliches ernten? \bibverse{12} So aber andere dieser Macht an
euch teilhaftig sind, warum nicht viel mehr wir? Aber wir haben solche
Macht nicht gebraucht, sondern wir vertragen allerlei, daß wir nicht dem
Evangelium Christi ein Hindernis machen. \bibverse{13} Wisset ihr nicht,
daß, die da opfern essen vom Opfer, und die des Altars pflegen, genießen
des Altars? \bibverse{14} Also hat auch der HErr befohlen daß, die das
Evangelium verkündigen; sollen sich vom Evangelium nähren. \bibverse{15}
Ich aber habe der keines gebraucht. Ich schreibe auch nicht darum davon,
daß es mit mir also sollte gehalten werden. Es wäre mir lieber, ich
stürbe, denn daß mir jemand meinen Ruhm sollte zunichte machen.
\bibverse{16} Denn daß ich das Evangelium predige, darf ich mich nicht
rühmen; denn ich muß es tun. Und wehe mir, wenn ich das Evangelium nicht
predigte! \bibverse{17} Tue ich's gerne, so wird mir gelohnet; tue ich's
aber ungerne, so ist mir das Amt doch befohlen. \bibverse{18} Was ist
denn nun mein Lohn? Nämlich daß ich predige das Evangelium Christi und
tue dasselbige frei, umsonst, auf daß ich nicht meiner Freiheit
mißbrauche am Evangelium. \bibverse{19} Denn wiewohl ich frei bin von
jedermann, hab' ich mich doch selbst jedermann zum Knechte gemacht, auf
daß ich ihrer viel gewinne. \bibverse{20} Den Juden bin ich worden als
ein Jude, auf daß ich die Juden gewinne. Denen, die unter dem Gesetz
sind, bin ich worden als unter dem Gesetz, auf daß ich, die, so unter
dem Gesetz sind, gewinne. \bibverse{21} Denen, die ohne Gesetz sind, bin
ich als ohne Gesetz worden (so ich doch nicht ohne Gesetz bin vor GOtt,
sondern bin in dem Gesetz Christi), auf daß ich die, so ohne Gesetz
sind, gewinne. \bibverse{22} Den Schwachen bin ich worden als ein
Schwacher, auf daß ich die Schwachen gewinne. Ich bin jedermann allerlei
worden, auf daß ich allenthalben ja etliche selig mache. \bibverse{23}
Solches aber tue ich um des Evangeliums willen, auf daß ich sein
teilhaftig werde. \bibverse{24} Wisset ihr nicht, daß die, so in den
Schranken laufen, die laufen alle, aber einer erlanget das Kleinod?
Laufet nun also, daß ihr es ergreifet! \bibverse{25} Ein jeglicher aber,
der da kämpfet, enthält sich alles Dinges: jene also, daß sie eine
vergängliche Krone empfangen, wir aber eine unvergängliche.
\bibverse{26} Ich laufe aber also, nicht als aufs Ungewisse; ich fechte
also, nicht als, der in die Luft streichet, \bibverse{27} sondern ich
betäube meinen Leib und zähme ihn, daß ich nicht den andern predige und
selbst verwerflich werde.

\hypertarget{section-9}{%
\section{10}\label{section-9}}

\bibverse{1} Ich will euch aber, liebe Brüder, nicht verhalten, daß
unsere Väter sind alle unter der Wolke gewesen und sind alle durchs Meer
gegangen \bibverse{2} und sind alle unter Mose getauft mit der Wolke und
mit dem Meer; \bibverse{3} und haben alle einerlei geistliche Speise
gegessen \bibverse{4} und haben alle einerlei geistlichen Trank
getrunken; sie tranken aber von dem geistlichen Fels, der mitfolgte,
welcher war Christus. \bibverse{5} Aber an ihrer vielen hatte GOtt kein
Wohlgefallen; denn sie sind niedergeschlagen in der Wüste. \bibverse{6}
Das ist aber uns zum Vorbilde geschehen, daß wir uns nicht gelüsten
lassen des Bösen, gleichwie jene gelüstet hat. \bibverse{7} Werdet auch
nicht Abgöttische, gleichwie jener etliche wurden, als geschrieben
stehet: Das Volk setzte sich nieder, zu essen und zu trinken, und stund
auf, zu spielen. \bibverse{8} Auch lasset uns nicht Hurerei treiben, wie
etliche unter jenen Hurerei trieben, und fielen auf einen Tag
dreiundzwanzigtausend. \bibverse{9} Lasset uns aber auch Christum nicht
versuchen, wie etliche von jenen ihn versuchten und wurden von, den
Schlangen umgebracht. \bibverse{10} Murret auch nicht, gleichwie jener
etliche murreten und wurden umgebracht durch den Verderber.
\bibverse{11} Solches alles widerfuhr ihnen zum Vorbilde; es ist aber
geschrieben uns zur Warnung, auf welche das Ende der Welt kommen ist.
\bibverse{12} Darum wer, sich lässet dünken, er stehe, mag wohl zusehen,
daß er nicht falle. \bibverse{13} Es hat euch noch keine denn
menschliche Versuchung betreten; aber GOtt ist getreu, der euch nicht
lässet versuchen über euer Vermögen, sondern machet, daß die Versuchung
so ein Ende gewinne, daß ihr's könnet ertragen. \bibverse{14} Darum,
meine Liebsten; fliehet von dem Götzendienst! \bibverse{15} Als mit den
Klugen rede ich; richtet ihr, was ich sage! \bibverse{16} Der gesegnete
Kelch, welchen wir segnen, ist der nicht die Gemeinschaft des Blutes
Christi? Das Brot, das wir brechen, ist das nicht die Gemeinschaft des
Leibes Christi? \bibverse{17} Denn ein Brot ist's; so sind wir viele ein
Leib, dieweil wir alle eines Brotes teilhaftig sind. \bibverse{18} Sehet
an den Israel nach dem Fleisch. Welche die Opfer essen, sind die nicht
in der Gemeinschaft des Altars? \bibverse{19} Was soll ich denn nun
sagen? Soll ich sagen, daß der Götze etwas sei, oder daß das Götzenopfer
etwas sei? \bibverse{20} Aber ich sage, daß die Heiden, was sie opfern,
das opfern sie den Teufeln und nicht GOtt. Nun will ich nicht, daß ihr
in der Teufel Gemeinschaft sein sollet. \bibverse{21} Ihr könnt nicht
zugleich trinken des HErrn Kelch und der Teufel Kelch; ihr könnt nicht
zugleich teilhaftig sein des Tisches des HErrn und des Tisches der
Teufel. \bibverse{22} Oder wollen wir dem HErrn trotzen? Sind wir
stärker denn er? \bibverse{23} Ich habe es zwar alles Macht; aber es
frommet nicht alles. Ich habe es alles Macht; aber es bessert nicht
alles. \bibverse{24} Niemand suche was sein ist, sondern ein jeglicher,
was des andern ist. \bibverse{25} Alles was feil ist auf dem
Fleischmarkt, das esset und forschet nichts, auf daß ihr des Gewissens
verschonet. \bibverse{26} Denn die Erde ist des HErrn, und was darinnen
ist. \bibverse{27} So aber jemand von den Ungläubigen euch ladet, und
ihr wollt hingehen, so esset alles, was euch vorgetragen wird, und
forschet nichts, auf daß ihr des Gewissens verschonet. \bibverse{28} Wo
aber jemand würde zu euch sagen: Das ist Götzenopfer, so esset nicht, um
deswillen, der es anzeigte, auf daß ihr des Gewissens verschonet. Die
Erde ist des HErrn, und was darinnen ist. \bibverse{29} Ich sage aber
vom Gewissen nicht dein selbst, sondern des andern. Denn warum sollte
ich meine Freiheit lassen urteilen von eines andern Gewissen?
\bibverse{30} Denn so ich's mit Danksagung genieße, was sollte ich denn
verlästert werden über dem, dafür ich danke?, \bibverse{31} Ihr esset
nun oder trinket, oder was ihr tut, so tut es alles zu GOttes Ehre.
\bibverse{32} Seid nicht ärgerlich weder den Juden noch den Griechen
noch der Gemeinde GOttes, \bibverse{33} gleichwie ich auch jedermann in
allerlei mich gefällig mache und suche nicht, was mir, sondern was
vielen frommet, daß sie selig werden.

\hypertarget{section-10}{%
\section{11}\label{section-10}}

\bibverse{1} Seid meine Nachfolger, gleichwie ich Christi! \bibverse{2}
Ich lobe euch, liebe Brüder, daß ihr an mich gedenket in allen Stücken
und haltet die Weise, gleichwie ich euch gegeben habe. \bibverse{3} Ich
lasse euch aber wissen, daß Christus ist eines jeglichen Mannes Haupt,
der Mann aber ist des Weibes Haupt; GOtt aber ist Christi Haupt.
\bibverse{4} Ein jeglicher Mann, der da betet oder weissaget und hat
etwas auf dem Haupt, der schändet sein Haupt. \bibverse{5} Ein Weib
aber, das da betet oder weissaget mit unbedecktem Haupt, die schändet
ihr Haupt; denn es ist ebensoviel, als wäre sie beschoren. \bibverse{6}
Will sie sich nicht bedecken, so schneide man ihr auch das Haar ab. Nun
es aber übel stehet, daß ein Weib verschnitten Haar habe oder beschoren
sei, so lasset sie das Haupt bedecken. \bibverse{7} Der Mann aber soll
das Haupt nicht bedecken, sintemal er ist GOttes Bild und Ehre; das Weib
aber ist des Mannes Ehre. \bibverse{8} Denn der Mann ist nicht vom
Weibe, sondern das Weib ist vom Manne. \bibverse{9} Und der Mann ist
nicht geschaffen um des Weibes willen; sondern das Weib um des Mannes
willen. \bibverse{10} Darum soll das Weib eine Macht auf dem Haupt haben
um der Engel willen. \bibverse{11} Doch ist weder der Mann ohne das
Weib, noch das Weib ohne den Mann in dem HErrn. \bibverse{12} Denn wie
das Weib von dem Manne, also kommt auch der Mann durch das Weib, aber
alles kommt von GOtt. \bibverse{13} Richtet bei euch selbst, ob es wohl
stehet, daß ein Weib unbedeckt vor GOtt bete. \bibverse{14} Oder lehret
euch auch nicht die Natur, daß einem Manne eine Unehre ist, so er lange
Haare zeuget, \bibverse{15} und dem Weibe eine Ehre, so sie lange Haare
zeuget? Das Haar ist ihr zur Decke gegeben. \bibverse{16} Ist aber
jemand unter euch, der Lust zu zanken hat, der wisse, daß wir solche
Weise nicht haben, die Gemeinden GOttes auch nicht. \bibverse{17} Ich
muß aber dies befehlen: Ich kann's nicht loben, daß ihr nicht auf
bessere Weise, sondern auf ärgere Weise zusammenkommet. \bibverse{18}
Zum ersten, wenn ihr zusammen kommt in der Gemeinde, höre ich, es seien
Spaltungen unter euch; und zum Teil glaube ich's. \bibverse{19} Denn es
müssen Rotten unter euch sein, auf daß die, so rechtschaffen sind,
offenbar unter euch werden. \bibverse{20} Wenn ihr nun zusammenkommet,
so hält man da nicht des HErrn Abendmahl. \bibverse{21} Denn so man das
Abendmahl halten soll, nimmt ein jeglicher sein eigenes vorhin, und
einer ist hungrig, der andere ist trunken. \bibverse{22} Habt ihr aber
nicht Häuser, da ihr essen und trinken möget? Oder verachtet ihr die
Gemeinde GOttes und beschämet die, so da nichts haben? Was soll ich euch
sagen? Soll ich euch loben? Hierinnen lobe ich euch nicht. \bibverse{23}
Ich habe von dem HErrn empfangen, das ich euch gegeben habe. Denn der
HErr JEsus in der Nacht, da er verraten ward, nahm er das Brot,
\bibverse{24} dankete und brach's und sprach: Nehmet, esset; das ist
mein Leib der für euch gebrochen wird. Solches tut zu meinem Gedächtnis!
\bibverse{25} Desselbigengleichen auch den Kelch nach dem Abendmahl und
sprach: Dieser Kelch ist das neue Testament in meinem Blut. Solches tut,
so oft ihr's trinket, zu meinem Gedächtnis! \bibverse{26} Denn so oft
ihr von diesem Brot esset und von diesem Kelch trinket, sollt ihr des
HErrn Tod verkündigen, bis daß er kommt. \bibverse{27} Welcher nun
unwürdig von diesem Brot isset oder von dem Kelch des HErrn trinket, der
ist schuldig an dem Leib und Blut des HErrn. \bibverse{28} Der Mensch
prüfe aber sich selbst und also esse er von diesem Brot und trinke von
diesem Kelch. \bibverse{29} Denn welcher unwürdig isset und trinket, der
isset und trinket ihm selber das Gericht damit, daß er nicht
unterscheidet den Leib des HErrn. \bibverse{30} Darum sind auch so viel
Schwache und Kranke unter euch, und ein gut Teil schlafen. \bibverse{31}
Denn so wir uns selber richteten, so würden wir nicht gerichtet.
\bibverse{32} Wenn wir aber gerichtet werden, so werden wir von dem
HErrn gezüchtiget, auf daß wir nicht samt der Welt verdammet werden.
\bibverse{33} Darum, meine lieben Brüder, wenn ihr zusammenkommet, zu
essen, so harre einer des andern. \bibverse{34} Hungert aber jemand, der
esse daheim, auf daß ihr nicht zum Gerichte zusammenkommet. Das andere
will ich ordnen, wenn ich komme.

\hypertarget{section-11}{%
\section{12}\label{section-11}}

\bibverse{1} Von den geistlichen Gaben aber will ich euch, liebe Brüder,
nicht verhalten. \bibverse{2} Ihr wisset, daß ihr Heiden seid gewesen
und hingegangen zu den stummen Götzen, wie ihr geführt wurdet.
\bibverse{3} Darum tue ich euch kund, daß niemand JEsum verfluchet, der
durch den Geist GOttes redet; und niemand kann JEsum einen HErrn heißen
ohne durch den Heiligen Geist. \bibverse{4} Es sind mancherlei Gaben,
aber es ist ein Geist. \bibverse{5} Und es sind mancherlei Ämter, aber
es ist ein HErr. \bibverse{6} Und es sind mancherlei Kräfte, aber es ist
ein GOtt, der da wirket alles in allen. \bibverse{7} In einem jeglichen
erzeigen sich die Gaben des Geistes zum gemeinen Nutzen. \bibverse{8}
Einem wird gegeben durch den Geist, zu reden von der Weisheit; dem
andern wird gegeben, zu reden von der Erkenntnis nach demselbigen Geist;
\bibverse{9} einem andern der Glaube in demselbigen Geist; einem andern
die Gabe, gesund zu machen, in demselbigen Geist; \bibverse{10} einem
andern, Wunder zu tun; einem andern Weissagung; einem andern, Geister zu
unterscheiden; einem andern mancherlei Sprachen; einem andern, die
Sprachen auszulegen. \bibverse{11} Dies aber alles wirket derselbige
einige Geist und teilet einem jeglichen seines zu, nachdem er will.
\bibverse{12} Denn gleichwie ein Leib ist und hat doch viel Glieder,
alle Glieder aber eines Leibes, wiewohl ihrer viel sind, sind sie doch
ein Leib: also auch Christus. \bibverse{13} Denn wir sind durch einen
Geist alle zu einem Leibe getauft, wir seien Juden oder Griechen,
Knechte oder Freie, und sind alle zu einem Geist getränket.
\bibverse{14} Denn auch der Leib ist nicht ein Glied, sondern viele.
\bibverse{15} So aber der Fuß spräche: Ich bin keine Hand, darum bin ich
des Leibes Glied nicht, sollte er um deswillen nicht des Leibes Glied
sein? \bibverse{16} Und so das Ohr spräche: Ich bin kein Auge, darum bin
ich nicht des Leibes Glied, sollte es um deswillen nicht des Leibes
Glied sein? \bibverse{17} Wenn der ganze Leib Auge wäre, wo bliebe das
Gehör? So er ganz Gehör wäre, wo bliebe der Geruch? \bibverse{18} Nun
aber hat GOtt die Glieder gesetzt, ein jegliches sonderlich am Leibe,
wie er gewollt hat. \bibverse{19} So aber alle Glieder ein Glied wären,
wo bliebe der Leib? \bibverse{20} Nun aber sind der Glieder viele, aber
der Leib ist einer. \bibverse{21} Es kann das Auge nicht sagen zu der
Hand: Ich bedarf dein nicht; oder wiederum das Haupt zu den Füßen: Ich
bedarf euer nicht; \bibverse{22} sondern vielmehr, die Glieder des
Leibes, die uns dünken, die schwächsten zu sein, sind die nötigsten,
\bibverse{23} und die uns dünken, die unehrlichsten sein, denselbigen
legen wir am meisten Ehre an, und die uns übel anstehen, die schmücket
man am meisten. \bibverse{24} Denn die uns wohl anstehen, die bedürfen's
nicht. Aber GOtt hat den Leib also vermenget und dem dürftigen Glied am
meisten Ehre gegeben, \bibverse{25} auf daß nicht eine Spaltung im Leibe
sei, sondern die Glieder füreinander gleich sorgen. \bibverse{26} Und so
ein Glied leidet, so leiden alle Glieder mit; und so ein Glied wird
herrlich gehalten, so freuen sich alle Glieder, mit. \bibverse{27} Ihr
seid aber der Leib Christi und Glieder, ein jeglicher nach seinem Teil.
\bibverse{28} Und GOtt hat gesetzt in der Gemeinde aufs erste die
Apostel, aufs andere die Propheten, aufs dritte die Lehrer, danach die
Wundertäter, danach die Gaben, gesund zu machen, Helfer, Regierer,
mancherlei Sprachen. \bibverse{29} Sind sie alle Apostel? Sind sie alle
Propheten? Sind sie alle Lehrer? Sind sie alle Wundertäter?
\bibverse{30} Haben sie alle Gaben gesund zu machen? Reden sie alle mit
mancherlei Sprachen? Können sie alle auslegen? \bibverse{31} Strebet
aber nach den besten Gaben! Und ich will euch noch einen köstlichern Weg
zeigen.

\hypertarget{section-12}{%
\section{13}\label{section-12}}

\bibverse{1} Wenn ich mit Menschen- und mit Engelzungen redete und hätte
der Liebe nicht, so wäre ich ein tönend Erz oder eine klingende Schelle.
\bibverse{2} Und wenn ich weissagen könnte und wüßte alle Geheimnisse
und alle Erkenntnis und hätte allen Glauben, also daß ich Berge
versetzte, und hätte der Liebe nicht, so wäre ich nichts. \bibverse{3}
Und wenn ich alle meine Habe den Armen gäbe und ließe meinen Leib
brennen und hätte der Liebe nicht, so wäre mir's nichts nütze.
\bibverse{4} Die Liebe ist langmütig und freundlich; die Liebe eifert
nicht; die Liebe treibt nicht Mutwillen; sie blähet sich nicht;
\bibverse{5} sie stellet sich nicht ungebärdig; sie suchet nicht das
Ihre; sie lässet sich nicht erbittern; sie trachtet nicht nach Schaden;
\bibverse{6} sie freuet sich nicht der Ungerechtigkeit; sie freuet sich
aber der Wahrheit; \bibverse{7} sie verträget alles, sie glaubet alles,
sie hoffet alles, sie duldet alles. \bibverse{8} Die Liebe höret nimmer
auf, so doch die Weissagungen aufhören werden, und die Sprachen aufhören
werden, und die Erkenntnis aufhören wird. \bibverse{9} Denn unser Wissen
ist Stückwerk, und unser Weissagen ist Stückwerk. \bibverse{10} Wenn
aber kommen wird das Vollkommene, so wird das Stückwerk aufhören.
\bibverse{11} Da ich ein Kind war, da redete ich wie ein Kind und war
klug wie ein Kind und hatte kindische Anschläge; da ich aber ein Mann
ward, tat ich ab, was kindisch war. \bibverse{12} Wir sehen jetzt durch
einen Spiegel in einem dunklen Wort, dann aber von Angesicht zu
Angesichte. Jetzt erkenne ich's stückweise; dann aber werde ich
erkennen, gleichwie ich erkannt bin. \bibverse{13} Nun aber bleibt
Glaube, Hoffnung, Liebe, diese drei; aber die Liebe ist die größte unter
ihnen.

\hypertarget{section-13}{%
\section{14}\label{section-13}}

\bibverse{1} Strebet nach der Liebe! Fleißiget euch der geistlichen
Gaben, am meisten aber, daß ihr weissagen möget. \bibverse{2} Denn der
mit der Zunge redet, der redet nicht den Menschen, sondern GOtt. Denn
ihm höret niemand zu; im Geist aber redet er die Geheimnisse.
\bibverse{3} Wer aber weissaget, der redet den Menschen zur Besserung
und zur Ermahnung und zur Tröstung. \bibverse{4} Wer mit Zungen redet,
der bessert sich selbst; wer aber weissaget, der bessert die Gemeinde.
\bibverse{5} Ich wollte, daß ihr alle mit Zungen reden könntet, aber
viel mehr, daß ihr weissagetet. Denn der da weissaget, ist größer, denn
der mit Zungen redet, es sei denn, daß er es auch auslege, daß die
Gemeinde davon gebessert werde. \bibverse{6} Nun aber, liebe Brüder,
wenn ich zu euch käme und redete mit Zungen, was wäre ich euch nütze, so
ich nicht mit euch redete entweder durch Offenbarung oder durch
Erkenntnis oder durch Weissagung oder durch Lehre? \bibverse{7} Hält
sich's doch auch also in den Dingen, die da lauten und doch nicht leben,
es sei eine Pfeife oder eine Harfe; wenn sie nicht unterschiedliche
Stimmen von sich geben, wie kann man wissen, was gepfiffen oder geharfet
ist? \bibverse{8} Und so die Posaune einen undeutlichen Ton gibt, wer
will sich zum Streit rüsten? \bibverse{9} Also auch ihr, wenn ihr mit
Zungen redet, so ihr nicht eine deutliche Rede gebet, wie kann man
wissen, was geredet ist? Denn ihr werdet in den Wind reden.
\bibverse{10} Zwar es ist mancherlei Art der Stimmen in der Welt, und
derselbigen ist doch keine undeutlich. \bibverse{11} So ich nun nicht
weiß der Stimme Deutung, werde ich undeutsch sein dem, der da redet, und
der da redet, wird mir undeutsch sein. \bibverse{12} Also auch ihr,
sintemal ihr euch fleißiget der geistlichen Gaben, trachtet danach, daß
ihr die Gemeinde bessert, auf daß ihr alles reichlich habet.
\bibverse{13} Darum, welcher mit Zungen redet, der bete also, daß er's
auch auslege. \bibverse{14} So ich aber mit Zungen bete, so betet mein
Geist; aber mein Sinn bringet niemand Frucht. \bibverse{15} Wie soll es
aber denn sein? Nämlich also: Ich will beten mit dem Geist und will
beten auch im Sinn; ich will Psalmen singen im Geist und will auch
Psalmen singen mit dem Sinn. \bibverse{16} Wenn du aber segnest im
Geist, wie soll der, so anstatt des Laien stehet, Amen sagen auf deine
Danksagung, sintemal er nicht weiß, was du sagest? \bibverse{17} Du
danksagest wohl fein; aber der andere wird nicht davon gebessert.
\bibverse{18} Ich danke meinem GOtt, daß ich mehr mit Zungen rede denn
ihr alle. \bibverse{19} Aber ich will in der Gemeinde lieber fünf Worte
reden mit meinem Sinn, auf daß ich auch andere unterweise, denn sonst
zehntausend Worte mit Zungen. \bibverse{20} Liebe Brüder, werdet nicht
Kinder an dem Verständnis, sondern an der Bosheit seid Kinder; an dem
Verständnis aber seid vollkommen. \bibverse{21} Im Gesetz stehet
geschrieben: Ich will mit andern Zungen und mit andern Lippen reden zu
diesem Volk, und sie werden mich auch also nicht hören, spricht der
HErr. \bibverse{22} Darum so sind die Zungen zum Zeichen, nicht den
Gläubigen, sondern den Ungläubigen; die Weissagung aber nicht den
Ungläubigen, sondern den Gläubigen. \bibverse{23} Wenn nun die ganze
Gemeinde zusammenkäme an einem Ort und redeten alle mit Zungen, es kämen
aber hinein Laien oder Ungläubige, würden sie nicht sagen, ihr wäret
unsinnig? \bibverse{24} So sie aber alle weissageten und käme dann ein
Ungläubiger oder Laie hinein, der würde von denselbigen allen gestraft
und von allen gerichtet. \bibverse{25} Und also würde das Verborgene
seines Herzens offenbar, und er würde also fallen auf sein Angesicht,
GOtt anbeten und bekennen, daß GOtt wahrhaftig in euch sei.
\bibverse{26} Wie ist ihm denn nun, liebe Brüder? Wenn ihr
zusammenkommt, so hat ein jeglicher Psalmen, er hat eine Lehre, er hat
Zungen, er hat Offenbarung, er hat Auslegung. Lasset es alles geschehen
zur Besserung! \bibverse{27} So jemand mit der Zunge redet oder zween
oder aufs meiste' drei, eins ums andere; so lege es einer aus.
\bibverse{28} Ist er aber nicht ein Ausleger, so schweige er unter der
Gemeinde, rede aber sich selber, und GOtt, \bibverse{29} Die Weissager
aber lasset reden, zween oder drei, und die andern lasset richten.
\bibverse{30} So aber eine Offenbarung geschieht einem andern, der da
sitzet, so schweige der erste. \bibverse{31} Ihr könnet wohl alle
weissagen, einer nach dem andern, auf daß sie alle lernen und alle
ermahnet werden. \bibverse{32} Und die Geister der Propheten sind den
Propheten untertan. \bibverse{33} Denn GOtt ist, nicht ein GOtt der
Unordnung, sondern des Friedens wie in allen Gemeinden der Heiligen.
\bibverse{34} Eure Weiber lasset schweigen unter der Gemeinde; denn es
soll ihnen nicht zugelassen werden, daß sie reden, sondern untertan
sein, wie auch das Gesetz sagt. \bibverse{35} Wollen sie aber etwas
lernen, so lasset sie daheim ihre Männer fragen. Es stehet den Weibern
übel an, unter der Gemeinde reden. \bibverse{36} Oder ist das Wort
GOttes von euch auskommen, oder ist's allein zu euch kommen?
\bibverse{37} So sich jemand lässet dünken, er sei ein Prophet oder
geistlich, der erkenne, was ich euch schreibe; denn es sind des HErrn
Gebote. \bibverse{38} Ist aber jemand unwissend, der sei, unwissend.
\bibverse{39} Darum, liebe Brüder, fleißiget euch des Weissagens und
wehret nicht, mit Zungen zu reden. \bibverse{40} Lasset alles ehrlich
und ordentlich zugehen!

\hypertarget{section-14}{%
\section{15}\label{section-14}}

\bibverse{1} Ich erinnere euch, aber, liebe Brüder, des Evangeliums, das
ich euch verkündiget habe, welches ihr auch angenommen habt, in welchem
ihr auch stehet, \bibverse{2} durch welches ihr auch selig werdet,
welcher Gestalt ich es euch verkündiget habe, so ihr's behalten habt, es
wäre, denn, daß ihr's umsonst geglaubet hättet. \bibverse{3} Denn ich
habe euch zuvörderst gegeben, welches ich auch empfangen habe, daß
Christus gestorben sei für unsere Sünden nach der Schrift, \bibverse{4}
und daß er begraben sei, und daß er auferstanden sei am dritten Tage
nach der Schrift, \bibverse{5} und daß er gesehen worden ist von Kephas,
danach von den Zwölfen. \bibverse{6} Danach ist er gesehen worden von
mehr denn fünfhundert Brüdern auf einmal, deren noch viel leben, etliche
aber sind entschlafen. \bibverse{7} Danach ist er gesehen worden von
Jakobus, danach von allen Aposteln. \bibverse{8} Am letzten nach allen
ist er auch von mir, als einer unzeitigen Geburt, gesehen worden;
\bibverse{9} denn ich bin der geringste unter den Aposteln, als der ich
nicht wert bin, daß ich ein Apostel heiße, darum daß ich die Gemeinde
GOttes verfolget habe. \bibverse{10} Aber von GOttes Gnaden bin ich, das
ich bin, und seine Gnade an mir ist nicht vergeblich gewesen, sondern
ich habe viel mehr gearbeitet denn sie alle, nicht aber ich, sondern
GOttes Gnade, die mit mir ist. \bibverse{11} Es sei nun ich oder jene,
also predigen wir, und also habt ihr geglaubet. \bibverse{12} So aber
Christus geprediget wird, daß er sei von den Toten auferstanden, wie
sagen denn etliche unter euch, die Auferstehung der Toten sei nichts ?
\bibverse{13} Ist aber die Auferstehung der Toten nichts, so ist auch
Christus nicht auferstanden. \bibverse{14} Ist aber Christus nicht
auferstanden, so ist unsere Predigt vergeblich, so ist auch euer Glaube
vergeblich. \bibverse{15} Wir würden aber auch erfunden falsche Zeugen
GOttes, daß wir wider GOtt gezeuget hätten, er hätte Christum
auferwecket, den er nicht auferwecket hätte, sintemal die Toten nicht
auferstehen. \bibverse{16} Denn so die Toten nicht auferstehen, so ist
Christus auch nicht auferstanden. \bibverse{17} Ist Christus aber nicht
auferstanden, so ist euer Glaube eitel, so seid ihr noch in euren
Sünden, \bibverse{18} so sind auch die, so in Christo entschlafen sind,
verloren. \bibverse{19} Hoffen wir allein in diesem Leben auf Christum,
so sind wir die elendesten unter allen Menschen. \bibverse{20} Nun aber
ist Christus auferstanden von den Toten und der Erstling worden unter
denen, die da schlafen, \bibverse{21} sintemal durch einen Menschen der
Tod und durch einen Menschen die Auferstehung der Toten kommt.
\bibverse{22} Denn gleichwie sie in Adam alle sterben, also werden sie
in Christo alle lebendig gemacht werden. \bibverse{23} Ein jeglicher
aber in seinerOrdnung. Der Erstling Christus, danach die Christo
angehören, wenn er kommen wird. \bibverse{24} Danach das Ende, wenn er
das Reich GOtt und dem Vater überantworten wird, wenn er aufheben wird
alle Herrschaft und alle Obrigkeit und Gewalt. \bibverse{25} Er muß aber
herrschen, bis daß er alle seine Feinde unter seine Füße lege.
\bibverse{26} Der letzte Feind, der aufgehoben wird, ist der Tod.
\bibverse{27} Denn er hat ihm alles unter seine Füße getan. Wenn er aber
sagt; daß es alles untertan sei, ist's offenbar, daß ausgenommen ist,
der ihm alles untertan hat. \bibverse{28} Wenn aber alles ihm untertan
sein wird, alsdann wird auch der Sohn selbst untertan sein dem, der ihm
alles untertan hat, auf daß GOtt sei alles in allen. \bibverse{29} Was
machen sonst, die sich taufen lassen über den Toten, so allerdinge die
Toten nicht auferstehen? Was lassen sie sich taufen über den Toten?
\bibverse{30} Und was stehen wir alle Stunde in der Gefahr?
\bibverse{31} Bei unserm Ruhm den ich habe in Christo JEsu, unserm
HErrn, ich sterbe täglich. \bibverse{32} Hab' ich menschlicher Meinung
zu Ephesus mit den wilden Tieren gefochten, was hilft's mir, so die
Toten nicht auferstehen? Lasset uns essen und trinken; denn morgen sind
wir tot. \bibverse{33} Lasset euch nicht verführen! Böse Geschwätze
verderben gute Sitten. \bibverse{34} Werdet doch einmal recht nüchtern
und sündiget nicht; denn etliche wissen nichts von GOtt, das sage ich
euch zur Schande. \bibverse{35} Möchte aber jemand, sagen: Wie werden
die Toten auferstehen; und mit welcherlei Leibe werden sie kommen?
\bibverse{36} Du Narr, was du säest, wird nicht lebendig, es sterbe
denn. \bibverse{37} Und was du säest, ist ja nicht der Leib, der werden
soll, sondern ein bloßes Korn, nämlich Weizen oder der andern eines.
\bibverse{38} GOtt aber gibt ihm einen Leib, wie er will, und einem
jeglichen von den Samen seinen eigenen Leib. \bibverse{39} Nicht ist
alles Fleisch einerlei Fleisch, sondern ein ander Fleisch ist der
Menschen, ein anderes des Viehes, ein anderes der Fische, ein anderes
der Vögel. \bibverse{40} Und es sind himmlische Körper und irdische
Körper. Aber eine andere Herrlichkeit haben die himmlischen und eine
andere die irdischen. \bibverse{41} Eine andere Klarheit hat die Sonne,
eine andere Klarheit hat der Mond, eine andere Klarheit haben die
Sterne; denn ein Stern übertrifft den andern an Klarheit. \bibverse{42}
Also auch die Auferstehung der Toten. Es wird gesäet verweslich und wird
auferstehen unverweslich. \bibverse{43} Es wird gesäet in Unehre und
wird auferstehen in Herrlichkeit. Es wird gesäet in Schwachheit und wird
auferstehen in Kraft. \bibverse{44} Es wird gesäet ein natürlicher Leib,
und wird auferstehen ein geistlicher Leib. Hat man einen natürlichen
Leib, so hat man auch einen geistlichen Leib, \bibverse{45} wie es
geschrieben stehet: Der erste Mensch, Adam, ist gemacht ins natürliche
Leben und der letzte Adam ins geistliche Leben. \bibverse{46} Aber der
geistliche Leib ist nicht erste, sondern der natürliche, danach der
geistliche. \bibverse{47} Der erste Mensch ist von der Erde und irdisch;
der andere Mensch ist der HErr vom Himmel. \bibverse{48} Welcherlei der
irdische ist, solcherlei sind auch die irdischen; und welcherlei der
himmlische ist, solcherlei sind auch die himmlischen. \bibverse{49} Und
wie wir getragen haben das Bild des irdischen, also werden wir auch
tragen das Bild des himmlischen. \bibverse{50} Davon sage ich aber,
liebe Brüder, daß Fleisch und Blut nicht können das Reich GOttes
ererben; auch wird das Verwesliche nicht erben das Unverwesliche,
\bibverse{51} Siehe, ich sage euch ein Geheimnis: Wir werden nicht alle
entschlafen wir werden aber alle verwandelt werden, \bibverse{52} und
dasselbige plötzlich, in einem Augenblick, zu der Zeit der letzten
Posaune. Denn es wird die Posaune schallen und die Toten werden
auferstehen unverweslich, und wir werden verwandelt werden.
\bibverse{53} Denn dies Verwesliche muß anziehen das Unverwesliche, und
dies Sterbliche muß anziehen die Unsterblichkeit. \bibverse{54} Wenn
aber dies Verwesliche wird anziehen das Unverwesliche, und dies
Sterbliche wird anziehen die Unsterblichkeit, dann wird erfüllet werden
das Wort, das geschrieben stehet: \bibverse{55} Der Tod ist verschlungen
in den Sieg. Tod, wo ist dein Stachel? Hölle, wo ist dein Sieg?
\bibverse{56} Aber der Stachel des Todes ist die Sünde; die Kraft aber
der Sünde ist das Gesetz. \bibverse{57} GOtt aber sei Dank, der uns den
Sieg gegeben hat durch unsern HErrn JEsum Christum! \bibverse{58} Darum,
meine lieben Brüder, seid fest, unbeweglich und nehmet immer zu in dem
Werk des HErrn, sintemal ihr wisset, daß eure Arbeit nicht vergeblich
ist in dem HErrn.

\hypertarget{section-15}{%
\section{16}\label{section-15}}

\bibverse{1} Von der Steuer aber, die den Heiligen geschieht, wie ich
den Gemeinden in Galatien geordnet habe, also tut auch ihr. \bibverse{2}
Auf je der Sabbate einen lege bei sich selbst ein jeglicher unter euch
und sammle, was ihn gut dünkt, auf daß nicht, wenn ich komme, dann
allererst die Steuer zu sammeln sei. \bibverse{3} Wenn ich aber
darkommen bin, welche ihr durch Briefe dafür ansehet, die will ich
senden, daß sie hinbringen eure Wohltat gen Jerusalem. \bibverse{4} So
es aber wert ist, daß ich auch hinreise, sollen sie mit mir reisen.
\bibverse{5} Ich will aber zu euch kommen, wenn ich durch Mazedonien
ziehe; denn durch Mazedonien werde ich ziehen. \bibverse{6} Bei euch
aber werde ich vielleicht bleiben oder auch wintern, auf daß ihr mich
geleitet, wo ich hinziehen werde. \bibverse{7} Ich will euch jetzt nicht
sehen im Vorüberziehen; denn ich hoffe, ich wolle etliche Zeit bei euch
bleiben, so es der HErr zuläßt. \bibverse{8} Ich werde aber zu Ephesus
bleiben bis Pfingsten. \bibverse{9} Denn mir ist eine große Tür
aufgetan, die viele Frucht wirket, und sind viel Widerwärtige da.
\bibverse{10} So Timotheus kommt, so sehet zu, daß er ohne Furcht bei
euch sei; denn er treibet auch das Werk des HErrn wie ich. \bibverse{11}
Daß ihn nun nicht jemand verachte! Geleitet ihn aber im Frieden, daß er
zu mir komme; denn ich warte sein mit den Brüdern. \bibverse{12} Von
Apollos, dem Bruder, aber wisset, daß ich ihn sehr viel ermahnet habe,
daß er zu euch käme mit den Brüdern, und es war allerdinge sein Wille
nicht, daß er jetzt käme; er wird aber kommen, wenn es ihm gelegen sein
wird. \bibverse{13} Wachet, stehet im Glauben, seid männlich und seid
stark! \bibverse{14} Alle eure Dinge lasset in der Liebe geschehen.
\bibverse{15} Ich ermahne euch aber, liebe Brüder, ihr kennet das Haus
Stephanas, daß sie sind die Erstlinge in Achaja und haben sich selbst
verordnet zum Dienst den Heiligen, \bibverse{16} auf daß auch ihr
solchen untertan seiet und allen, die mitwirken und arbeiten.
\bibverse{17} Ich freue mich über die Zukunft Stephanas und Fortunatus
und Achaicus; denn wo ich euer Mangel hatte, das haben sie erstattet.
\bibverse{18} Sie haben erquicket meinen und euren Geist. Erkennet, die
solche sind! \bibverse{19} Es grüßen euch die Gemeinden in Asien. Es
grüßen euch sehr in dem HErrn Aquila und Priscilla samt der Gemeinde in
ihrem Hause. \bibverse{20} Es grüßen euch alle Brüder. Grüßet euch
untereinander mit dem heiligen Kuß. \bibverse{21} Ich, Paulus, grüße
euch mit meiner Hand. \bibverse{22} So jemand den HErrn JEsum Christum
nicht liebhat, der sei Anathema, Maharam Motha. \bibverse{23} Die Gnade
des HErrn JEsu Christi sei mit euch! \bibverse{24} Meine Liebe sei mit
euch allen in Christo JEsu! Amen.
