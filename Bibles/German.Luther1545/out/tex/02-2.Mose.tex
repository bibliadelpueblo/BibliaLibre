\hypertarget{section}{%
\section{1}\label{section}}

\bibverse{1} Dies sind die Namen der Kinder Israels, die mit Jakob nach
Ägypten kamen; ein jeglicher kam mit seinem Hause hinein: \bibverse{2}
Ruben, Simeon, Levi, Juda, \bibverse{3} Isaschar, Sebulon, Benjamin,
\bibverse{4} Dan, Naphthali, Gad, Asser. \bibverse{5} Und aller Seelen,
die aus den Lenden Jakobs kommen waren, der waren siebenzig. Joseph aber
war zuvor in Ägypten. \bibverse{6} Da nun Joseph gestorben war und alle
seine Brüder und alle, die zu der Zeit gelebt hatten, \bibverse{7}
wuchsen die Kinder Israel und zeugeten Kinder und mehreten sich; und
wurden ihrer sehr viel, daß ihrer das Land voll ward. \bibverse{8} Da
kam ein neuer König auf in Ägypten, der wußte nichts von Joseph;
\bibverse{9} und sprach zu seinem Volk: Siehe, des Volks der Kinder
Israel ist viel und mehr denn wir. \bibverse{10} Wohlan, wir wollen sie
mit Listen dämpfen, daß ihrer nicht so viel werden. Denn wo sich ein
Krieg erhübe, möchten sie sich auch zu unsern Feinden schlagen und wider
uns streiten und zum Lande ausziehen. \bibverse{11} Und man setzte
Fronvögte über sie, die sie mit schweren Diensten drücken sollten; denn
man bauete dem Pharao die Städte Pithon und Raemses zu Schatzhäusern.
\bibverse{12} Aber je mehr sie das Volk drückten, je mehr sich es
mehrete und ausbreitete. Und sie hielten die Kinder Israel wie einen
Greuel. \bibverse{13} Und die Ägypter zwangen die Kinder Israel zu
Dienst mit Unbarmherzigkeit \bibverse{14} und machten ihnen ihr Leben
sauer mit schwerer Arbeit im Ton und Ziegeln und mit allerlei Frönen auf
dem Felde und mit allerlei Arbeit, die sie ihnen auflegten mit
Unbarmherzigkeit. \bibverse{15} Und der König in Ägypten sprach zu den
ebräischen Wehmüttern, deren eine hieß Siphra und die andere Pua:
\bibverse{16} Wenn ihr den ebräischen Weibern helfet und auf dem Stuhl
sehet, daß es ein Sohn ist, so tötet ihn; ist's aber eine Tochter, so
lasset sie leben. \bibverse{17} Aber die Wehmütter fürchteten GOtt und
taten nicht, wie der König in Ägypten zu ihnen gesagt hatte, sondern
ließen die Kinder leben. \bibverse{18} Da rief der König in Ägypten den
Wehmüttern und sprach zu ihnen: Warum tut ihr das, daß ihr die Kinder
leben lasset? \bibverse{19} Die Wehmütter antworteten Pharao: Die
ebräischen Weiber sind nicht wie die ägyptischen, denn sie sind harte
Weiber; ehe die Wehmutter zu ihnen kommt, haben sie geboren.
\bibverse{20} Darum tat GOtt den Wehmüttern Gutes. Und das Volk mehrete
sich und ward sehr viel. \bibverse{21} Und weil die Wehmütter GOtt
fürchteten, bauete er ihnen Häuser. \bibverse{22} Da gebot Pharao all
seinem Volk und sprach: Alle Söhne, die geboren werden, werfet ins
Wasser und alle Töchter lasset leben.

\hypertarget{section-1}{%
\section{2}\label{section-1}}

\bibverse{1} Und es ging hin ein Mann vom Hause Levi und nahm eine
Tochter Levis. \bibverse{2} Und das Weib ward schwanger und gebar einen
Sohn. Und da sie sah, daß es ein fein Kind war, verbarg sie ihn drei
Monden. \bibverse{3} Und da sie ihn nicht länger verbergen konnte,
machte sie ein Kästlein von Rohr und verklebte es mit Ton und Pech und
legte das Kind drein und legte ihn in das Schilf am Ufer des Wassers.
\bibverse{4} Aber seine Schwester stund von ferne, daß sie erfahren
wollte, wie es ihm gehen würde. \bibverse{5} Und die Tochter Pharaos
ging hernieder und wollte baden im Wasser; und ihre Jungfrauen gingen an
dem Rande des Wassers. Und da sie das Kästlein im Schilf sah, sandte sie
ihre Magd hin und ließ es holen. \bibverse{6} Und da sie es auftat, sah
sie das Kind; und siehe, das Knäblein weinete. Da jammerte es sie, und
sprach: Es ist der ebräischen Kindlein eins. \bibverse{7} Da sprach
seine Schwester zu der Tochter Pharaos: Soll ich hingehen und der
ebräischen Weiber eine rufen, die da säuget, daß sie dir das Kindlein
säuge? \bibverse{8} Die Tochter Pharaos sprach zu ihr: Gehe hin! Die
Jungfrau ging hin und rief des Kindes Mutter. \bibverse{9} Da sprach
Pharaos Tochter zu ihr: Nimm hin das Kindlein und säuge mir's, ich will
dir lohnen. Das Weib nahm das Kind und säugete es. \bibverse{10} Und da
das Kind groß ward, brachte sie es der Tochter Pharaos, und es ward ihr
Sohn; und hieß ihn Mose, denn sie sprach: Ich habe ihn aus dem Wasser
gezogen. \bibverse{11} Zu den Zeiten, da Mose war groß worden, ging er
aus zu seinen Brüdern und sah ihre Last und ward gewahr, daß ein Ägypter
schlug seiner Brüder, der ebräischen, einen. \bibverse{12} Und er wandte
sich hin und her, und da er sah, daß kein Mensch da war, erschlug er den
Ägypter und verscharrete ihn in den Sand. \bibverse{13} Auf einen andern
Tag ging er auch aus und sah zween ebräische Männer sich miteinander
zanken; und sprach zu dem Ungerechten: Warum schlägest du deinen
Nächsten? \bibverse{14} Er aber sprach: Wer hat dich zum Obersten oder
Richter über uns gesetzt? Willst du mich auch erwürgen, wie du den
Ägypter erwürget hast? Da fürchtete sich Mose und sprach: Wie ist das
laut worden? \bibverse{15} Und es kam vor Pharao, der trachtete nach
Mose, daß er ihn erwürgete. Aber Mose floh vor Pharao und hielt sich im
Lande Midian und wohnete bei einem Brunnen. \bibverse{16} Der Priester
aber in Midian hatte sieben Töchter, die kamen, Wasser zu schöpfen, und
fülleten die Rinnen, daß sie ihres Vaters Schafe tränketen.
\bibverse{17} Da kamen die Hirten und stießen sie davon. Aber Mose
machte sich auf und half ihnen und tränkte ihre Schafe. \bibverse{18}
Und da sie zu ihrem Vater Reguel kamen, sprach er: Wie seid ihr heute so
bald kommen? \bibverse{19} Sie sprachen: Ein ägyptischer Mann errettete
uns von den Hirten und schöpfte uns und tränkte die Schafe.
\bibverse{20} Er sprach zu seinen Töchtern: Wo ist er? Warum habt ihr
den Mann gelassen, daß ihr ihn nicht ludet, mit uns zu essen?
\bibverse{21} Und Mose bewilligte, bei dem Manne zu bleiben. Und er gab
Mose seine Tochter Zipora. \bibverse{22} Die gebar einen Sohn: und er
hieß ihn Gersom; denn er sprach: Ich bin ein Fremdling worden im fremden
Lande. (Und sie gebar noch einen Sohn, den hieß er Elieser, und sprach:
Der GOtt meines Vaters ist mein Helfer und hat mich von der Hand Pharaos
errettet.) \bibverse{23} Lange Zeit aber danach starb der König in
Ägypten. Und die Kinder Israel seufzeten über ihre Arbeit und schrieen;
und ihr Schreien über ihre Arbeit kam vor GOtt. \bibverse{24} Und GOtt
erhörete ihr Wehklagen und gedachte an seinen Bund mit Abraham, Isaak
und Jakob. \bibverse{25} Und er sah drein und nahm sich ihrer an.

\hypertarget{section-2}{%
\section{3}\label{section-2}}

\bibverse{1} Mose aber hütete die Schafe Jethros, seines Schwähers, des
Priesters in Midian, und trieb die Schafe hinter in die Wüste und kam an
den Berg GOttes Horeb. \bibverse{2} Und der Engel des HErrn erschien ihm
in einer feurigen Flamme aus dem Busch. Und er sah, daß der Busch mit
Feuer brannte, und ward doch nicht verzehret. \bibverse{3} Und sprach:
Ich will dahin und besehen dies große Gesicht, warum der Busch nicht
verbrennet. \bibverse{4} Da aber der HErr sah, daß er hinging zu sehen,
rief ihm GOtt aus dem Busch und sprach: Mose, Mose! Er antwortete: Hie
bin ich. \bibverse{5} Er sprach: Tritt nicht herzu! Zeuch deine Schuhe
aus von deinen Füßen; denn der Ort, da du auf stehest, ist ein heilig
Land. \bibverse{6} Und sprach weiter: Ich bin der GOtt deines Vaters,
der GOtt Abrahams, der GOtt Isaaks und der GOtt Jakobs. Und Mose
verhüllete sein Angesicht, denn er fürchtete sich, GOtt anzuschauen.
\bibverse{7} Und der HErr sprach: Ich habe gesehen das Elend meines
Volks in Ägypten und habe ihr Geschrei gehöret über die, so sie treiben;
ich habe ihr Leid erkannt. \bibverse{8} Und bin herniedergefahren, daß
ich sie errette von der Ägypter Hand und sie ausführe aus diesem Lande
in ein gut und weit Land, in ein Land, darinnen Milch und Honig fleußt,
nämlich an den Ort der Kanaaniter, Hethiter, Amoriter, Pheresiter,
Heviter und Jebusiter. \bibverse{9} Weil denn nun das Geschrei der
Kinder Israel vor mich kommen ist und habe auch dazu gesehen ihre Angst,
wie sie die Ägypter ängsten, \bibverse{10} so gehe nun hin, ich will
dich zu Pharao senden, daß du mein Volk, die Kinder Israel, aus Ägypten
führest. \bibverse{11} Mose sprach zu GOtt: Wer bin ich, daß ich zu
Pharao gehe und führe die Kinder Israel aus Ägypten? \bibverse{12} Er
sprach: Ich will mit dir sein. Und das soll dir das Zeichen sein, daß
ich dich gesandt habe: Wenn du mein Volk aus Ägypten geführet hast,
werdet ihr GOtt opfern auf diesem Berge. \bibverse{13} Mose sprach zu
GOtt: Siehe, wenn ich zu den Kindern Israel komme und spreche zu ihnen:
Der GOtt eurer Väter hat mich zu euch gesandt, und sie mir sagen werden:
Wie heißt sein Name? was soll ich ihnen sagen? \bibverse{14} GOtt sprach
zu Mose: Ich werde sein, der ich sein werde. Und sprach: Also sollst du
den Kindern Israel sagen: Ich werd's sein, der hat mich zu euch gesandt.
\bibverse{15} Und GOtt sprach weiter zu Mose: Also sollst du zu den
Kindern Israel sagen: Der HErr, eurer Väter GOtt, der GOtt Abrahams, der
GOtt Isaaks, der GOtt Jakobs, hat mich zu euch gesandt. Das ist mein
Name ewiglich, dabei soll man mein gedenken für und für. \bibverse{16}
Darum so gehe hin und versammle die Ältesten in Israel und sprich zu
ihnen: Der HErr, eurer Väter GOtt, ist mir er schienen, der GOtt
Abrahams, der GOtt Isaaks, der GOtt Jakobs, und hat gesagt: Ich habe
euch heimgesucht und gesehen, was euch in Ägypten widerfahren ist.
\bibverse{17} Und habe gesagt: Ich will euch aus dem Elende Ägyptens
führen in das Land der Kanaaniter, Hethiter, Amoriter, Pheresiter,
Heviter und Jebusiter, in das Land, darinnen Milch und Honig fleußt.
\bibverse{18} Und wenn sie deine Stimme hören, so sollst du und die
Ältesten in Israel hineingehen zum Könige in Ägypten und zu ihm sagen:
Der HErr, der Ebräer GOtt, hat uns gerufen. So laß uns nun gehen drei
Tagesreisen in die Wüste, daß wir opfern dem HErrn, unserm GOtt.
\bibverse{19} Aber ich weiß, daß euch der König in Ägypten nicht wird
ziehen lassen ohne durch eine starke Hand. \bibverse{20} Denn ich werde
meine Hand aus strecken und Ägypten schlagen mit allerlei Wundern, die
ich drinnen tun werde. Danach wird er euch ziehen lassen. \bibverse{21}
Und ich will diesem Volk Gnade geben vor den Ägyptern, daß, wenn ihr
ausziehet, nicht leer ausziehet; \bibverse{22} sondern ein jeglich Weib
soll von ihrer Nachbarin und Hausgenossin fordern silberne und güldene
Gefäße und Kleider; die sollt ihr auf eure Söhne und Töchter legen und
den Ägyptern entwenden.

\hypertarget{section-3}{%
\section{4}\label{section-3}}

\bibverse{1} Mose antwortete und sprach: Siehe, sie werden mir nicht
glauben noch meine Stimme hören, sondern werden sagen: Der HErr ist dir
nicht erschienen. \bibverse{2} Der HErr sprach zu ihm: Was ist, das du
in deiner Hand hast? Er sprach: Ein Stab. \bibverse{3} Er sprach: Wirf
ihn von dir auf die Erde! Und er warf ihn von sich; da ward er zur
Schlange. Und Mose floh vor ihr. \bibverse{4} Aber der HErr sprach zu
ihm: Strecke deine Hand aus und erhasche sie bei dem Schwanz. Da
streckte er seine Hand aus und hielt sie; und sie ward zum Stab in
seiner Hand. \bibverse{5} Darum werden sie glauben, daß dir erschienen
sei der HErr, der GOtt ihrer Väter, der GOtt Abrahams, der GOtt Isaaks,
der GOtt Jakobs. \bibverse{6} Und der HErr sprach weiter zu ihm: Stecke
deine Hand in deinen Busen. Und er steckte sie in seinen Busen und zog
sie heraus; siehe, da war sie aussätzig wie Schnee. \bibverse{7} Und er
sprach: Tu sie wieder in den Busen. Und er tat sie wieder in den Busen
und zog sie heraus; siehe, da ward sie wieder wie sein ander Fleisch.
\bibverse{8} Wenn sie dir nun nicht werden glauben noch deine Stimme
hören bei einem Zeichen, so werden sie doch glauben deiner Stimme bei
dem andern Zeichen. \bibverse{9} Wenn sie aber diesen zweien Zeichen
nicht glauben werden noch deine Stimme hören, so nimm des Wassers aus
dem Strom und geuß es auf das trockne Land, so wird dasselbe Wasser, das
du aus dem Strom genommen hast, Blut werden auf dem trocknen Lande.
\bibverse{10} Mose aber sprach zu dem HErrn: Ach, mein HErr, ich bin je
und je nicht wohl beredt gewesen, seit der Zeit du mit deinem Knecht
geredet hast; denn ich habe eine schwere Sprache und eine schwere Zunge.
\bibverse{11} Der HErr sprach zu ihm: Wer hat dem Menschen den Mund
geschaffen oder wer hat den Stummen oder Tauben oder Sehenden oder
Blinden gemacht? Habe ich's nicht getan, der HErr? \bibverse{12} So gehe
nun hin: Ich will mit deinem Munde sein und dich lehren, was du sagen
sollst. \bibverse{13} Mose sprach aber: Mein HErr, sende, welchen du
senden willst! \bibverse{14} Da ward der HErr sehr zornig über Mose und
sprach: Weiß ich denn nicht, daß dein Bruder Aaron aus dem Stamm Levi
beredt ist? Und siehe, er wird herausgehen dir entgegen, und wenn er
dich siehet, wird er sich von Herzen freuen. \bibverse{15} Du sollst zu
ihm reden und die Worte in seinen Mund legen. Und ich will mit deinem
und seinem Munde sein und euch lehren; was ihr tun sollt. \bibverse{16}
Und er soll für dich zum Volk reden; er soll dein Mund sein, und du
sollst sein Gott sein. \bibverse{17} Und diesen Stab nimm in deine Hand,
damit du Zeichen tun sollst. \bibverse{18} Mose ging hin und kam wieder
zu Jethro, seinem Schwäher, und sprach zu ihm: Lieber, laß mich gehen,
daß ich wieder zu meinen Brüdern komme, die in Ägypten sind, und sehe,
ob sie noch leben. Jethro sprach zu ihm: Gehe hin mit Frieden.
\bibverse{19} Auch sprach der HErr zu ihm in Midian: Gehe hin und zeuch
wieder nach Ägypten; denn die Leute sind tot, die nach deinem Leben
stunden. \bibverse{20} Also nahm Mose sein Weib und seine Söhne und
führete sie auf einem Esel und zog wieder nach Ägyptenland; und nahm den
Stab GOttes in seine Hand! \bibverse{21} Und der HErr sprach zu Mose:
Siehe zu, wenn du wieder nach Ägypten kommst, daß du alle die Wunder
tust vor Pharao, die ich dir in deine Hand gegeben habe; ich aber will
sein Herz verstocken, daß er das Volk nicht lassen wird. \bibverse{22}
Und sollst zu ihm sagen: So saget der HErr: Israel ist mein erstgeborner
Sohn; \bibverse{23} und ich gebiete dir, daß du meinen Sohn ziehen
lassest, daß er mir diene. Wirst du dich des weigern, so will ich deinen
erstgebornen Sohn erwürgen. \bibverse{24} Und als er unterwegen in der
Herberge war, kam ihm der HErr entgegen und wollte ihn töten.
\bibverse{25} Da nahm Zipora einen Stein und beschnitt ihrem Sohn die
Vorhaut; und rührete ihm seine Füße an und sprach: Du bist mir ein
Blutbräutigam. \bibverse{26} Da ließ er von ihm ab. Sie sprach aber
Blutbräutigam um der Beschneidung willen. \bibverse{27} Und der HErr
sprach zu Aaron: Gehe hin Mose entgegen in die Wüste. Und er ging hin
und begegnete ihm am Berge GOttes und küssete ihn. \bibverse{28} Und
Mose sagte Aaron alle Worte des HErrn, der ihn gesandt hatte, und alle
Zeichen, die er ihm befohlen hatte. \bibverse{29} Und sie gingen hin und
versammelten alle Ältesten von den Kindern Israel. \bibverse{30} Und
Aaron redete alle Worte, die der HErr mit Mose geredet hatte, und tat
die Zeichen vor dem Volk. \bibverse{31} Und das Volk glaubete. Und da
sie höreten, daß der HErr die Kinder Israel heimgesucht und ihr Elend
angesehen hätte, neigeten sie sich und beteten an.

\hypertarget{section-4}{%
\section{5}\label{section-4}}

\bibverse{1} Danach gingen Mose und Aaron hinein und sprachen zu Pharao:
So sagt der HErr, der GOtt Israels: Laß mein Volk ziehen, daß mir's ein
Fest halte in der Wüste. \bibverse{2} Pharao antwortete: Wer ist der
HErr, des Stimme ich hören müsse und Israel ziehen lassen? Ich weiß
nicht von dem HErrn, will auch Israel nicht lassen ziehen. \bibverse{3}
Sie sprachen: Der Ebräer GOtt hat uns gerufen; so laß uns nun hinziehen
drei Tagereisen in die Wüste und dem HErrn, unserm GOtt, opfern, daß uns
nicht widerfahre Pestilenz oder Schwert. \bibverse{4} Da sprach der
König in Ägypten zu ihnen: Du, Mose und Aaron, warum wollt ihr das Volk
von seiner Arbeit frei machen? Gehet hin an eure Dienste! \bibverse{5}
Weiter sprach Pharao: Siehe, des Volks ist schon zu viel im Lande, und
ihr wollt sie noch feiern heißen von ihrem Dienst. \bibverse{6} Darum
befahl Pharao desselben Tages den Vögten des Volks und ihren Amtleuten
und sprach: \bibverse{7} Ihr sollt dem Volk nicht mehr Stroh sammeln und
geben, daß sie Ziegel brennen, wie bis anher; lasset sie selbst hingehen
und Stroh zusammenlesen; \bibverse{8} und die Zahl der Ziegel, die sie
bisher gemacht haben, sollt ihr ihnen gleichwohl auflegen und nichts
mindern; denn sie gehen müßig, darum schreien sie und sprechen: Wir
wollen hinziehen und unserm GOtt opfern. \bibverse{9} Man drücke die
Leute mit Arbeit, daß sie zu schaffen haben und sich nicht kehren an
falsche Rede! \bibverse{10} Da gingen die Vögte des Volks und ihre
Amtleute aus und sprachen zum Volk: So spricht Pharao: Man wird euch
kein Stroh geben. \bibverse{11} Gehet ihr selbst hin und sammelt euch
Stroh, wo ihr's findet; aber von eurer Arbeit soll nichts gemindert
werden. \bibverse{12} Da zerstreute sich das Volk ins ganze Land
Ägypten, daß es Stoppeln sammelte, damit sie Stroh hätten. \bibverse{13}
Und die Vögte trieben sie und sprachen: Erfüllet euer Tagwerk, gleich
als da ihr Stroh hattet! \bibverse{14} Und die Amtleute der Kinder
Israel, welche die Vögte Pharaos über sie gesetzet hatten, wurden
geschlagen, und ward zu ihnen gesagt: Warum habt ihr weder heute noch
gestern euer gesetzt Tagwerk getan, wie vorhin? \bibverse{15} Da gingen
hinein die Amtleute der Kinder Israel und schrieen zu Pharao: Warum
willst du mit deinen Knechten also fahren? \bibverse{16} Man gibt deinen
Knechten kein Stroh, und sollen die Ziegel machen, die uns bestimmt
sind; und siehe, deine Knechte werden geschlagen, und dein Volk muß
Sünder sein. \bibverse{17} Pharao sprach: Ihr seid müßig, müßig seid
ihr; darum sprechet ihr: Wir wollen hinziehen und dem HErrn opfern.
\bibverse{18} So gehet nun hin und frönet! Stroh soll man euch nicht
geben, aber die Anzahl der Ziegel sollt ihr reichen. \bibverse{19} Da
sahen die Amtleute der Kinder Israel, daß es ärger ward, weil man sagte:
Ihr sollt nichts mindern von dem Tagwerk an den Ziegeln. \bibverse{20}
Und da sie von Pharao gingen, begegneten sie Mose und Aaron und traten
gegen sie \bibverse{21} und sprachen zu ihnen: Der HErr sehe auf euch
und richte es, daß ihr unsern Geruch habt stinken gemacht vor Pharao und
seinen Knechten und habt ihnen das Schwert in ihre Hände gegeben, uns zu
töten. \bibverse{22} Mose aber kam wieder zu dem HErrn und sprach: HErr,
warum tust du so übel an diesem Volk? Warum hast du mich hergesandt?
\bibverse{23} Denn seit dem, daß ich hinein bin gegangen zu Pharao, mit
ihm zu reden in deinem Namen, hat er das Volk noch härter geplagt; und
du hast dein Volk nicht errettet.

\hypertarget{section-5}{%
\section{6}\label{section-5}}

\bibverse{1} Der HErr sprach zu Mose: Nun sollst du sehen, was ich
Pharao tun werde; denn durch eine starke Hand muß er sie lassen ziehen,
er muß sie noch durch eine starke Hand aus seinem Lande von sich
treiben. \bibverse{2} Und GOtt redete mit Mose und sprach zu ihm: Ich
bin der HErr, \bibverse{3} und bin erschienen Abraham, Isaak und Jakob,
daß ich ihr allmächtiger GOtt sein wollte; aber mein Name, HErr, ist
ihnen nicht offenbaret worden. \bibverse{4} Auch habe ich meinen Bund
mit ihnen aufgerichtet, daß ich ihnen geben will das Land Kanaan, das
Land ihrer Wallfahrt, darinnen sie Fremdlinge gewesen sind. \bibverse{5}
Auch habe ich gehöret die Wehklage der Kinder Israel, welche die Ägypter
mit Frönen beschweren, und habe an meinen Bund gedacht. \bibverse{6}
Darum sage den Kindern Israel: Ich bin der HErr und will euch ausführen
von euren Lasten in Ägypten und will euch erretten von eurem Fronen und
will euch erlösen durch einen ausgereckten Arm und große Gerichte;
\bibverse{7} und will euch annehmen zum Volk und will euer GOtt sein,
daß ihr's erfahren sollt, daß ich der HErr bin euer GOtt, der euch
ausgeführet habe von der Last Ägyptens \bibverse{8} und euch gebracht in
das Land, darüber ich habe meine Hand gehoben, daß ich's gäbe Abraham,
Isaak und Jakob; das will ich euch geben zu eigen, ich, der HErr.
\bibverse{9} Mose sagte solches den Kindern Israel; aber sie höreten ihn
nicht vor Seufzen und Angst und harter Arbeit. \bibverse{10} Da redete
der HErr mit Mose und sprach: \bibverse{11} Gehe hinein und rede mit
Pharao, dem Könige in Ägypten, daß er die Kinder Israel aus seinem Lande
lasse. \bibverse{12} Mose aber redete vor dem HErrn und sprach: Siehe,
die Kinder Israel hören mich nicht, wie sollte mich denn Pharao hören?
Dazu bin ich von unbeschnittenen Lippen. \bibverse{13} Also redete der
HErr mit Mose und Aaron und tat ihnen Befehl an die Kinder Israel und
Pharao, den König in Ägypten, daß sie die Kinder Israel aus Ägypten
führeten. \bibverse{14} Dies sind die Häupter in jeglichem Geschlecht
der Väter. Die Kinder Rubens, des ersten Sohnes Israels, sind diese:
Hanoch, Pallu, Hezron, Charmi. Das sind die Geschlechter von Ruben.
\bibverse{15} Die Kinder Simeons sind diese: Jemuel, Jamin, Ohad,
Jachin, Zohar und Saul, der Sohn des kanaanäischen Weibes, das sind
Simeons Geschlechter. \bibverse{16} Dies sind die Namen der Kinder Levis
in ihren Geschlechtern: Gerson, Kahath, Merari. Aber Levi ward
hundertundsiebenunddreißig Jahre alt. \bibverse{17} Die Kinder Gersons
sind diese: Libni und Simei in ihren Geschlechtern. \bibverse{18} Die
Kinder Kahaths sind diese: Amram, Jezear, Hebron, Usiel. Kahath aber
ward hundertunddreiunddreißig Jahre alt. \bibverse{19} Die Kinder
Meraris sind diese: Maheli und Musi. Das sind die Geschlechter Levis in
ihren Stämmen. \bibverse{20} Und Amram nahm seine Muhme Jochebed zum
Weibe, die gebar ihm Aaron und Mose. Aber Amram ward
hundertundsiebenunddreißig Jahre alt. \bibverse{21} Die Kinder Jezears
sind diese: Korah, Nepheg, Sichri. \bibverse{22} Die Kinder Usiels sind
diese: Misael, Elzaphan, Sithri. \bibverse{23} Aaron nahm zum Weibe
Eliseba, die Tochter Amminadabs, Nahassons Schwester; die gebar ihm
Nadab, Abihu, Eleasar, Ithamar. \bibverse{24} Die Kinder Korahs sind
diese: Assir, Elkana, Abiasaph. Das sind die Geschlechter der Korahiter.
\bibverse{25} Eleasar aber, Aarons Sohn, der nahm von den Töchtern
Putiels ein Weib; die gebar ihm den Pinehas. Das sind die Häupter unter
den Vätern der Levitengeschlechter. \bibverse{26} Das ist der Aaron und
Mose, zu denen der HErr sprach: Führet die Kinder Israel aus Ägyptenland
mit ihrem Heer. \bibverse{27} Sie sind's, die mit Pharao, dem Könige in
Ägypten, redeten, daß sie die Kinder Israel aus Ägypten führeten,
nämlich Mose und Aaron. \bibverse{28} Und des Tages redete der HErr mit
Mose in Ägyptenland \bibverse{29} und sprach zu ihm: Ich bin der HErr;
rede mit Pharao, dem Könige in Ägypten, alles was ich mit dir rede.
\bibverse{30} Und er antwortete vor dem HErrn: Siehe, ich bin von
unbeschnittenen Lippen; wie wird mich denn Pharao hören?

\hypertarget{section-6}{%
\section{7}\label{section-6}}

\bibverse{1} Der HErr sprach zu Mose: Siehe, ich habe dich einen Gott
gesetzt über Pharao; und Aaron, dein Bruder, soll dein Prophet sein.
\bibverse{2} Du sollst reden alles, was ich dir gebieten werde; aber
Aaron, dein Bruder, soll es vor Pharao reden, daß er die Kinder Israel
aus seinem Lande lasse. \bibverse{3} Aber ich will Pharaos Herz
verhärten, daß ich meiner Zeichen und Wunder viel tue in Ägyptenland.
\bibverse{4} Und Pharao wird euch nicht hören, auf daß ich meine Hand in
Ägypten beweise und führe mein Heer, mein Volk, die Kinder Israel, aus
Ägyptenland durch große Gerichte. \bibverse{5} Und die Ägypter sollen's
inne werden, daß ich der HErr bin, wenn ich nun meine Hand ausstrecke
über Ägypten und die Kinder Israel von ihnen wegführen werde.
\bibverse{6} Mose und Aaron taten, wie ihnen der HErr geboten hatte.
\bibverse{7} Und Mose war achtzig Jahre alt und Aaron dreiundachtzig
Jahre alt, da sie mit Pharao redeten. \bibverse{8} Und der HErr sprach
zu Mose und Aaron: \bibverse{9} Wenn Pharao zu euch sagen wird: Beweiset
eure Wunder, so sollst du zu Aaron sagen: Nimm deinen Stab und wirf ihn
vor Pharao, daß er zur Schlange werde. \bibverse{10} Da gingen Mose und
Aaron hinein zu Pharao und taten wie ihnen der HErr geboten hatte. Und
Aaron warf seinen Stab vor Pharao und vor seinen Knechten; und er ward
zur Schlange. \bibverse{11} Da forderte Pharao die Weisen und Zauberer.
Und die ägyptischen Zauberer taten auch also mit ihrem Beschwören.
\bibverse{12} Ein jeglicher warf seinen Stab von sich, da wurden
Schlangen draus; aber Aarons Stab verschlang ihre Stäbe. \bibverse{13}
Also ward das Herz Pharaos verstockt und hörete sie nicht, wie denn der
HErr geredet hatte. \bibverse{14} Und der HErr sprach zu Mose: Das Herz
Pharaos ist hart; er weigert sich, das Volk zu lassen. \bibverse{15}
Gehe hin zu Pharao morgen. Siehe, er wird ans Wasser gehen; so tritt
gegen ihm an das Ufer des Wassers und nimm den Stab in deine Hand, der
zur Schlange ward, \bibverse{16} und sprich zu ihm: Der HErr, der Ebräer
GOtt, hat mich zu dir gesandt und lassen sagen: Laß mein Volk, daß mir's
diene in der Wüste! Aber du hast bisher nicht wollen hören.
\bibverse{17} Darum spricht der HErr also: Daran sollst du erfahren, daß
ich der HErr bin. Siehe, ich will mit dem Stabe, den ich in meiner Hand
habe, das Wasser schlagen, das in dem Strom ist, und es soll in Blut
verwandelt werden, \bibverse{18} daß die Fische im Strom sterben sollen
und der Strom stinken, und den Ägyptern wird ekeln, zu trinken des
Wassers aus dem Strom. \bibverse{19} Und der HErr sprach zu Mose: Sage
Aaron: Nimm deinen Stab und recke deine Hand aus über die Wasser in
Ägypten, über ihre Bäche und Ströme und Seen und über alle Wassersümpfe,
daß sie Blut werden, und sei Blut in ganz Ägyptenland, beide in
hölzernen und steinernen Gefäßen. \bibverse{20} Mose und Aaron taten,
wie ihnen der HErr geboten hatte, und hub den Stab auf und schlug ins
Wasser, das im Strom war, vor Pharao und seinen Knechten. Und alles
Wasser im Strom ward in Blut verwandelt. \bibverse{21} Und die Fische im
Strom starben, und der Strom ward stinkend, daß die Ägypter nicht
trinken konnten des Wassers aus dem Strom; und ward Blut in ganz
Ägyptenland. \bibverse{22} Und die ägyptischen Zauberer taten auch also
mit ihrem Beschwören. Also ward das Herz Pharaos verstockt und hörete
sie nicht, wie denn der HErr geredet hatte. \bibverse{23} Und Pharao
wandte sich und ging heim und nahm's nicht zu Herzen. \bibverse{24} Aber
alle Ägypter gruben nach Wasser um den Strom her, zu trinken; denn des
Wassers aus dem Strom konnten sie nicht trinken. \bibverse{25} Und das
währete sieben Tage lang, daß der HErr den Strom schlug.

\hypertarget{section-7}{%
\section{8}\label{section-7}}

\bibverse{1} Der HErr sprach zu Mose: Gehe hinein zu Pharao und sprich
zu ihm: So sagt der HErr: Laß mein Volk, daß mir's diene! \bibverse{2}
Wo du dich des weigerst, siehe, so will ich alle deine Grenze mit
Fröschen plagen, \bibverse{3} daß der Strom soll von Fröschen wimmeln;
die sollen heraufkriechen und kommen in dein Haus, in deine Kammer, auf
dein Lager, auf dein Bett; auch in die Häuser deiner Knechte, unter dein
Volk, in deine Backöfen und in deine Teige; \bibverse{4} und sollen die
Frösche auf dich und auf dein Volk und auf alle deine Knechte kriechen.
\bibverse{5} Und der HErr sprach zu Mose: Sage Aaron: Recke deine Hand
aus mit deinem Stabe über die Bäche und Ströme und Seen und laß Frösche
über Ägyptenland kommen. \bibverse{6} Und Aaron reckte seine Hand über
die Wasser in Ägypten; und kamen Frösche herauf, daß Ägyptenland bedeckt
ward. \bibverse{7} Da taten die Zauberer auch also mit ihrem Beschwören
und ließen Frösche über Ägyptenland kommen. \bibverse{8} Da forderte
Pharao Mose und Aaron und sprach: Bittet den HErrn für mich, daß er die
Frösche von mir und von meinem Volk nehme, so will ich das Volk lassen,
daß es dem HErrn opfere. \bibverse{9} Mose sprach: Habe du die Ehre vor
mir und stimme mir, wann ich für dich, für deine Knechte und für dein
Volk bitten soll, daß die Frösche von dir und von deinem Hause
vertrieben werden und allein im Strom bleiben. \bibverse{10} Er sprach:
Morgen. Er sprach: Wie du gesagt hast. Auf daß du erfahrest, daß niemand
ist wie der HErr, unser GOtt, \bibverse{11} so sollen die Frösche von
dir, von deinem Hause von deinen Knechten und von deinem Volk genommen
werden und allein im Strom bleiben. \bibverse{12} Also ging Mose und
Aaron von Pharao. Und Mose schrie zu dem HErrn der Frösche halben, wie
er Pharao hatte zugesagt. \bibverse{13} Und der HErr tat, wie Mose
gesagt hatte; und die Frösche starben in den Häusern, in den Höfen und
auf dem Felde. \bibverse{14} Und sie häuften sie zusammen, hie einen
Haufen und da einen Haufen; und das Land stank davon. \bibverse{15} Da
aber Pharao sah, daß er Luft gekriegt hatte, ward sein Herz verhärtet
und hörete sie nicht, wie denn der HErr geredet hatte. \bibverse{16} Und
der HErr sprach zu Mose: Sage Aaron: Recke deinen Stab aus und schlag in
den Staub auf Erden, daß Läuse werden in ganz Ägyptenland. \bibverse{17}
Sie taten also, und Aaron reckte seine Hand aus mit seinem Stabe und
schlug in den Staub auf Erden; und es wurden Läuse an den Menschen und
an dem Vieh; aller Staub des Landes ward Läuse in ganz Ägyptenland.
\bibverse{18} Die Zauberer taten auch also mit ihrem Beschwören, daß sie
Läuse heraus brächten, aber sie konnten nicht. Und die Läuse waren beide
an Menschen und an Vieh. \bibverse{19} Da sprachen die Zauberer zu
Pharao: Das ist GOttes Finger. Aber das Herz Pharaos ward verstockt und
hörete sie nicht, wie denn der HErr gesagt hatte. \bibverse{20} Und der
HErr sprach zu Mose: Mache dich morgen frühe auf und tritt vor Pharao
(siehe, er wird ans Wasser gehen) und sprich zu ihm: So sagt der HErr:
Laß mein Volk, daß es mir diene; \bibverse{21} wo nicht, siehe, so will
ich allerlei Ungeziefer lassen kommen über dich, deine Knechte, dein
Volk und dein Haus, daß aller Ägypter Häuser und das Feld und was drauf
ist, voll Ungeziefer werden sollen. \bibverse{22} Und will des Tages ein
Besonderes tun mit dem Lande Gosen, da sich mein Volk enthält, daß kein
Ungeziefer da sei, auf daß du inne werdest, daß ich der HErr bin auf
Erden allenthalben. \bibverse{23} Und will eine Erlösung setzen zwischen
meinem und deinem Volk: Morgen soll das Zeichen geschehen. \bibverse{24}
Und der HErr tat also, und es kam viel Ungeziefers in Pharaos Haus, in
seiner Knechte Häuser und über ganz Ägyptenland; und das Land ward
verderbet von dem Ungeziefer. \bibverse{25} Da forderte Pharao Mose und
Aaron und sprach: Gehet hin, opfert eurem GOtt hie im Lande.
\bibverse{26} Mose sprach: Das taugt nicht, daß wir also tun; denn wir
würden der Ägypter Greuel opfern unserm GOtt, dem HErrn; siehe, wenn wir
denn der Ägypter Greuel vor ihren Augen opferten, würden sie uns nicht
steinigen? \bibverse{27} Drei Tagereisen wollen wir gehen in die Wüste
und dem HErrn, unserm GOtt, opfern, wie er uns gesagt hat. \bibverse{28}
Pharao sprach: Ich will euch lassen, daß ihr dem HErrn, eurem GOtt
opfert in der Wüste; allein; daß ihr nicht ferner ziehet, und bittet für
mich. \bibverse{29} Mose sprach: Siehe, wenn ich hinaus von dir komme so
will ich den HErrn bitten, daß dies Ungeziefer von Pharao und seinen
Knechten und von seinem Volk genommen werde, morgen des Tages; allein
täusche mich nicht mehr, daß du das Volk nicht lassest, dem HErrn zu
opfern. \bibverse{30} Und Mose ging hinaus von Pharao und bat den HErrn.
\bibverse{31} Und der HErr tat, wie Mose gesagt hatte, und schaffte das
Ungeziefer weg von Pharao, von seinen Knechten und von seinem Volk, daß
nicht eins überblieb. \bibverse{32} Aber Pharao verhärtete sein Herz
auch dasselbe Mal und ließ das Volk nicht.

\hypertarget{section-8}{%
\section{9}\label{section-8}}

\bibverse{1} Der HErr sprach zu Mose: Gehe hinein zu Pharao und sprich
zu ihm: Also sagt der HErr, der GOtt der Ebräer: Laß mein Volk, daß sie
mir dienen! \bibverse{2} Wo du dich des weigerst und sie weiter
aufhältst, \bibverse{3} siehe, so wird Hand des HErrn sein über dein
Vieh auf dem Felde, über Pferde, über Esel, über Kamele, über Ochsen,
über Schafe mit einer fast schweren Pestilenz. \bibverse{4} Und der HErr
wird ein Besonderes tun zwischen dem Vieh der Israeliten und der
Ägypter, daß nichts sterbe aus allem, das die Kinder Israel haben.
\bibverse{5} Und der HErr bestimmte eine Zeit und sprach: Morgen wird
der HErr solches auf Erden tun. \bibverse{6} Und der HErr tat solches
des Morgens; und starb allerlei Vieh der Ägypter; aber des Viehes der
Kinder Israel starb nicht eins. \bibverse{7} Und Pharao sandte danach,
und siehe, es war des Viehes Israel nicht eins gestorben. Aber das Herz
Pharaos ward verstockt und ließ das Volk nicht. \bibverse{8} Da sprach
der HErr zu Mose und Aaron: Nehmet eure Fäuste voll Ruß aus dem Ofen,
und Mose sprenge ihn gen Himmel vor Pharao, \bibverse{9} daß über ganz
Ägyptenland stäube, und böse schwarze Blattern auffahren, beide an
Menschen und an Vieh, in ganz Ägyptenland. \bibverse{10} Und sie nahmen
Ruß aus dem Ofen und traten vor Pharao, und Mose sprengete ihn gen
Himmel. Da fuhren auf böse schwarze Blattern, beide an Menschen und an
Vieh, \bibverse{11} also daß die Zauberer nicht konnten vor Mose stehen
vor den bösen Blattern; denn es waren an den Zauberern ebensowohl böse
Blattern als an allen Ägyptern. \bibverse{12} Aber der HErr verstockte
das Herz Pharaos, daß er sie nicht hörete, wie denn der HErr zu Mose
gesagt hatte. \bibverse{13} Da sprach der HErr zu Mose: Mache dich
morgen frühe auf und tritt vor Pharao und sprich zu ihm: So sagt der
HErr, der Ebräer GOtt: Laß mein Volk, daß mir's diene! \bibverse{14} Ich
will anders diesmal alle meine Plagen über dich selbst senden, über
deine Knechte und über dein Volk, daß du inne werden sollst, daß
meinesgleichen nicht ist in allen Landen. \bibverse{15} Denn ich will
jetzt meine Hand ausrecken und dich und dein Volk mit Pestilenz
schlagen, daß du von der Erde sollst vertilget werden. \bibverse{16} Und
zwar darum habe ich dich erwecket, daß meine Kraft an dir erscheine, und
mein Name verkündiget werde in allen Landen. \bibverse{17} Du trittst
mein Volk noch unter dich und willst es nicht lassen. \bibverse{18}
Siehe, ich will morgen um diese Zeit einen sehr großen Hagel regnen
lassen, desgleichen in Ägypten nicht gewesen ist, seit der Zeit es
gegründet ist, bisher. \bibverse{19} Und nun sende hin und verwahre dein
Vieh und alles, was du auf dem Felde hast. Denn alle Menschen und Vieh,
das auf dem Felde funden wird und nicht in die Häuser versammelt ist, so
der Hagel auf sie fällt, werden sterben. \bibverse{20} Wer nun unter den
Knechten Pharaos des HErrn Wort fürchtete, der ließ seine Knechte und
Vieh in die Häuser fliehen. \bibverse{21} Welcher Herz aber sich nicht
kehrete an des HErrn Wort, ließen ihre Knechte und Vieh auf dem Felde.
\bibverse{22} Da sprach der HErr zu Mose: Recke deine Hand gen Himmel,
daß es hagele über ganz Ägyptenland, über Menschen, über Vieh und über
alles Kraut auf dem Felde in Ägyptenland. \bibverse{23} Also reckte Mose
seinen Stab gen Himmel; und der HErr ließ donnern und hageln, daß das
Feuer auf die Erde schoß. Also ließ der HErr Hagel regnen über
Ägyptenland, \bibverse{24} daß Hagel und Feuer untereinander fuhren, so
grausam, daß desgleichen in ganz Ägyptenland nie gewesen war, seit der
Zeit Leute drinnen gewesen sind. \bibverse{25} Und der Hagel schlug in
ganz Ägyptenland alles, was auf dem Felde war, beide Menschen und Vieh,
und schlug alles Kraut auf dem Felde und zerbrach alle Bäume auf dem
Felde. \bibverse{26} Ohne allein im Lande Gosen, da die Kinder Israel
waren, da hagelte es nicht. \bibverse{27} Da schickte Pharao hin und
ließ Mose und Aaron rufen und sprach zu ihnen: Ich habe dasmal mich
versündiget; der HErr ist gerecht, ich aber und mein Volk sind Gottlose.
\bibverse{28} Bittet aber den HErrn, daß aufhöre solch Donnern und
Hageln GOttes, so will ich euch lassen, daß ihr nicht länger hie
bleibet. \bibverse{29} Mose sprach zu ihm: Wenn ich zur Stadt hinaus
komme, will ich meine Hände ausbreiten gegen den HErrn, so wird der
Donner aufhören, und kein Hagel mehr sein, auf daß du inne werdest, daß
die Erde des HErrn sei. \bibverse{30} Ich weiß aber, daß du und deine
Knechte euch noch nicht fürchtet vor GOtt dem HErrn. \bibverse{31} Also
ward geschlagen der Flachs und die Gerste; denn die Gerste hatte
geschosset und der Flachs Knoten gewonnen. \bibverse{32} Aber der Weizen
und Roggen ward nicht geschlagen, denn es war spät Getreide.
\bibverse{33} So ging nun Mose von Pharao zur Stadt hinaus und breitete
seine Hände gegen den HErrn, und der Donner und Hagel höreten auf, und
der Regen troff nicht mehr auf Erden. \bibverse{34} Da aber Pharao sah,
daß der Regen und Donner und Hagel aufhörete, versündigte er sich weiter
und verhärtete sein Herz, er und seine Knechte. \bibverse{35} Also ward
des Pharao Herz verstockt, daß er die Kinder Israel nicht ließ, wie denn
der HErr geredet hatte durch Mose.

\hypertarget{section-9}{%
\section{10}\label{section-9}}

\bibverse{1} Und der HErr sprach zu Mose: Gehe hinein zu Pharao; denn
ich habe sein und seiner Knechte Herz verhärtet, auf daß ich diese meine
Zeichen unter ihnen tue, \bibverse{2} und daß du verkündigest vor den
Ohren deiner Kinder und deiner Kindeskinder, was ich in Ägypten
ausgerichtet habe und wie ich meine Zeichen unter ihnen beweiset habe,
daß ihr wisset; ich bin der HErr. \bibverse{3} Also gingen Mose und
Aaron hinein zu Pharao und sprachen zu ihm: So spricht der HErr, der
Ebräer GOtt: Wie lange weigerst du, dich vor mir zu demütigen, daß du
mein Volk lassest, mir zu dienen? \bibverse{4} Weigerst du dich, mein
Volk zu lassen, siehe, so will ich morgen Heuschrecken kommen lassen an
allen Orten, \bibverse{5} daß sie das Land bedecken, also daß man das
Land nicht sehen könne; und sollen fressen, was euch übrig und errettet
ist von dem Hagel, und sollen alle eure grünenden Bäume fressen auf dem
Felde; \bibverse{6} und sollen erfüllen dein Haus, aller deiner Knechte
Häuser und aller Ägypter Häuser, desgleichen nicht gesehen haben deine
Väter und deiner Väter Väter, seit der Zeit sie auf Erden gewesen, bis
auf diesen Tag. Und er wandte sich und ging von Pharao hinaus.
\bibverse{7} Da sprachen die Knechte Pharaos zu ihm: Wie lange sollen
wir damit geplagt sein? Laß die Leute ziehen, daß sie dem HErrn, ihrem
GOtt, dienen! Willst du zu vor erfahren, daß Ägypten untergegangen sei?
\bibverse{8} Mose und Aaron wurden wieder zu Pharao gebracht, der sprach
zu ihnen: Gehet hin und dienet dem HErrn, eurem GOtt. Welche sind sie
aber, die hinziehen sollen? \bibverse{9} Mose sprach: Wir wollen ziehen
mit jung und alt, mit Söhnen und Töchtern, mit Schafen und Rindern; denn
wir haben ein Fest des HErrn. \bibverse{10} Er sprach zu ihnen: Awe ja,
der HErr sei mit euch! Sollte ich euch und eure Kinder dazu ziehen
lassen? Sehet da, ob ihr nicht Böses vorhabt! \bibverse{11} Nicht also,
sondern ihr Männer ziehet hin und dienet dem HErrn; denn das habt ihr
auch gesucht. Und man stieß sie heraus von Pharao. \bibverse{12} Da
sprach der HErr zu Mose: Recke deine Hand über Ägyptenland um die
Heuschrecken, daß sie auf Ägyptenland kommen und fressen alles Kraut im
Lande auf samt alle dem, das dem Hagel überblieben ist. \bibverse{13}
Mose reckte seinen Stab über Ägyptenland. Und der HErr trieb einen
Ostwind ins Land den ganzen Tag und die ganze Nacht; und des Morgens
führete der Ostwind die Heuschrecken her. \bibverse{14} Und sie kamen
über ganz Ägyptenland und ließen sich nieder an allen Orten in Ägypten,
so sehr viel, daß zuvor des gleichen nie gewesen ist, noch hinfort sein
wird. \bibverse{15} Denn sie bedeckten das Land und verfinsterten es.
Und sie fraßen alles Kraut im Lande auf und alle Früchte auf den Bäumen,
die dem Hagel waren überblieben, und ließen nichts Grünes übrig an den
Bäumen und am Kraut auf dem Felde in ganz Ägyptenland. \bibverse{16} Da
forderte Pharao eilend Mose und Aaron und sprach: Ich habe mich
versündiget an dem HErrn, eurem GOtt, und an euch. \bibverse{17}
Vergebet mir meine Sünde diesmal auch und bittet den HErrn, euren GOtt,
daß er doch nur diesen Tod von mir wegnehme. \bibverse{18} Und er ging
aus von Pharao und bat den HErrn. \bibverse{19} Da wendete der HErr
einen sehr starken Westwind und hub die Heuschrecken auf und warf sie
ins Schilfmeer, daß nicht eine übrig blieb an allen Orten Ägyptens.
\bibverse{20} Aber der HErr verstockte Pharaos Herz, daß er die Kinder
Israel nicht ließ. \bibverse{21} Der HErr sprach zu Mose: Recke deine
Hand gen Himmel, daß es so finster werde in Ägyptenland, daß man's
greifen mag. \bibverse{22} Und Mose reckte seine Hand gen Himmel; da
ward eine dicke Finsternis in ganz Ägyptenland drei Tage, \bibverse{23}
daß niemand den andern sah noch aufstund von dem Ort, da er war, in
dreien Tagen. Aber bei allen Kindern Israel war es licht in ihren
Wohnungen. \bibverse{24} Da forderte Pharao Mose und sprach: Ziehet hin
und dienet dem HErrn; allein eure Schafe und Rinder lasset hie; lasset
auch eure Kindlein mit euch ziehen. \bibverse{25} Mose sprach: Du mußt
uns auch Opfer und Brandopfer geben, das wir unserm GOtt, dem HErrn, tun
mögen. \bibverse{26} Unser Vieh soll mit uns gehen und nicht eine Klaue
dahinten bleiben; denn von dem Unsern werden wir nehmen zum Dienst
unsers GOttes, des HErrn. Denn wir wissen nicht, womit wir dem HErrn
dienen sollen, bis wir dahin kommen. \bibverse{27} Aber der HErr
verstockte das Herz Pharaos, daß er sie nicht lassen wollte.
\bibverse{28} Und Pharao sprach zu ihm: Gehe von mir und hüte dich, daß
du nicht mehr vor meine Augen kommest; denn welches Tages du vor meine
Augen kommst, sollst du sterben. \bibverse{29} Mose antwortete: Wie du
gesagt hast; ich will nicht mehr vor deine Augen kommen.

\hypertarget{section-10}{%
\section{11}\label{section-10}}

\bibverse{1} Und der HErr sprach zu Mose: Ich will noch eine Plage über
Pharao und Ägypten kommen lassen, danach wird er euch lassen von hinnen,
und wird nicht allein alles lassen, sondern euch auch von hinnen
treiben. \bibverse{2} So sage nun vor dem Volk, daß ein jeglicher von
seinem Nächsten und eine jegliche von ihrer Nächstin silberne und
güldene Gefäße fordere. \bibverse{3} Denn der HErr wird dem Volk Gnade
geben vor den Ägyptern. Und Mose war ein sehr großer Mann in Ägyptenland
vor den Knechten Pharaos und vor dem Volk. \bibverse{4} Und Mose sprach:
So sagt der HErr: Ich will zur Mitternacht ausgehen in Ägyptenland;
\bibverse{5} und alle Erstgeburt in Ägyptenland soll sterben, von dem
ersten Sohn Pharaos an, der auf seinem Stuhl sitzt, bis an den ersten
Sohn der Magd, die hinter der Mühle ist, und alle Erstgeburt unter dem
Vieh. \bibverse{6} Und wird ein groß Geschrei sein in ganz Ägyptenland,
desgleichen nie gewesen ist noch werden wird; \bibverse{7} aber bei
allen Kindern Israel soll nicht ein Hund mucken, beide unter Menschen
und Vieh, auf daß ihr erfahret, wie der HErr Ägypten und Israel scheide.
\bibverse{8} Dann werden zu mir herabkommen alle diese deine Knechte und
mir zu Fuße fallen und sagen: Zeuch aus, du und alles Volk, das unter
dir ist. Danach will ich ausziehen, und er ging von Pharao mit grimmigem
Zorn. \bibverse{9} Der HErr aber sprach zu Mose: Pharao höret euch
nicht, auf daß viele Wunder geschehen in Ägyptenland. \bibverse{10} Und
Mose und Aaron haben diese Wunder alle getan vor Pharao; aber der HErr
verstockte ihm sein Herz, daß er die Kinder Israel nicht lassen wollte
aus seinem Lande.

\hypertarget{section-11}{%
\section{12}\label{section-11}}

\bibverse{1} Der HErr aber sprach zu Mose und Aaron in Ägyptenland:
\bibverse{2} Dieser Mond soll bei euch der erste Mond sein; und von ihm
sollt ihr die Monde des Jahrs anheben. \bibverse{3} Saget der ganzen
Gemeine Israel und sprechet: Am zehnten Tage dieses Monden nehme ein
jeglicher ein Lamm, wo ein Hausvater ist, je ein Lamm zu einem Hause!
\bibverse{4} Wo ihrer aber in einem Hause zum Lamm zu wenig sind, so
nehme er's und sein nächster Nachbar an seinem Hause, bis ihrer so viel
wird, daß sie das Lamm aufessen mögen. \bibverse{5} Ihr sollt aber ein
solch Lamm nehmen, da kein Fehl an ist, ein Männlein und eines Jahrs
alt; von den Lämmern und Ziegen sollt ihr's nehmen. \bibverse{6} Und
sollt es behalten bis auf den vierzehnten Tag des Monden. Und ein
jegliches Häuflein im ganzen Israel soll es schlachten zwischen Abends.
\bibverse{7} Und sollt seines Bluts nehmen und beide Pfosten an der Tür
und die oberste Schwelle damit bestreichen an den Häusern, da sie es
innen essen. \bibverse{8} Und sollt also Fleisch essen in derselben
Nacht, am Feuer gebraten, und ungesäuert Brot, und sollt es mit bittern
Salsen essen. \bibverse{9} Ihr sollt es nicht roh essen, noch mit Wasser
gesotten, sondern am Feuer gebraten, sein Haupt mit seinen Schenkeln und
Eingeweide. \bibverse{10} Und sollt nichts davon überlassen bis morgen;
wo aber etwas überbleibet bis morgen, sollt ihr's mit Feuer verbrennen.
\bibverse{11} Also sollt ihr's aber essen: Um eure Lenden sollt ihr
gegürtet sein und eure Schuhe an euren Füßen haben und Stäbe in euren
Händen, und sollt es essen, als die hinwegeilen; denn es ist des HErrn
Passah. \bibverse{12} Denn ich will in derselbigen Nacht durch
Ägyptenland gehen und alle Erstgeburt schlagen in Ägyptenland, beide
unter Menschen und Vieh. Und will meine Strafe beweisen an allen Göttern
der Ägypter, ich, der HErr. \bibverse{13} Und das Blut soll euer Zeichen
sein an den Häusern, darin ihr seid, daß, wenn ich das Blut sehe, vor
euch übergehe, und euch nicht die Plage widerfahre, die euch verderbe,
wenn ich Ägyptenland schlage. \bibverse{14} Und sollt diesen Tag haben
zum Gedächtnis und sollt ihn feiern dem HErrn zum Fest, ihr und alle
eure Nachkommen, zur ewigen Weise. \bibverse{15} Sieben Tage sollt ihr
ungesäuert Brot essen; nämlich am ersten Tage sollt ihr aufhören mit
gesäuertem Brot in euren Häusern. Wer gesäuert Brot isset vom ersten
Tage an, bis auf den siebenten, des Seele soll ausgerottet werden von
Israel. \bibverse{16} Der erste Tag soll heilig sein, daß ihr
zusammenkommet; und der siebente soll auch heilig sein, daß ihr
zusammenkommet. Keine Arbeit sollt ihr drinnen tun, ohne was zur Speise
gehöret für allerlei Seelen, dasselbe allein möget ihr für euch tun.
\bibverse{17} Und haltet ob dem ungesäuerten Brot, denn eben an
demselben Tage habe ich euer Heer aus Ägyptenland geführet; darum sollt
ihr diesen Tag halten und alle eure Nachkommen zur ewigen Weise.
\bibverse{18} Am vierzehnten Tage des ersten Monden, des Abends, sollt
ihr ungesäuert Brot essen, bis an den einundzwanzigsten Tag des Monden
an dem Abend, \bibverse{19} daß man sieben Tage kein gesäuert Brot finde
in euren Häusern. Denn wer gesäuert Brot isset, des Seele soll
ausgerottet werden von der Gemeinde Israel, es sei ein Fremdling oder
Einheimischer im Lande. \bibverse{20} Darum so esset kein gesäuert Brot,
sondern eitel ungesäuert Brot in allen euren Wohnungen. \bibverse{21}
Und Mose forderte alle Ältesten in Israel und sprach zu ihnen: Leset aus
und nehmet Schafe jedermann für sein Gesinde und schlachtet das Passah.
\bibverse{22} Und nehmet ein Büschel Ysop und tunket in das Blut in dem
Becken und berühret damit die Überschwelle und die zween Pfosten. Und
gehe kein Mensch zu seiner Haustür heraus bis an den Morgen.
\bibverse{23} Denn der, HErr wird umhergehen und die Ägypter plagen. Und
wenn er das Blut sehen wird an der Überschwelle und an den zween
Pfosten, wird er vor der Tür übergehen und den Verderber nicht in eure
Häuser kommen lassen zu plagen. \bibverse{24} Darum so halte diese Weise
für dich und deine Kinder ewiglich. \bibverse{25} Und wenn ihr ins Land
kommet, das euch der HErr geben wird, wie er geredet hat, so haltet
diesen Dienst. \bibverse{26} Und wenn eure Kinder werden zu euch sagen:
Was habt ihr da für einen Dienst? \bibverse{27} sollt ihr sagen: Es ist
das Passahopfer des HErrn, der vor den Kindern Israel überging in
Ägypten, da er die Ägypter plagte und unsere Häuser errettete. Da neigte
sich das Volk und bückte sich. \bibverse{28} Und die Kinder Israel
gingen hin und taten, wie der HErr Mose und Aaron geboten hätte.
\bibverse{29} Und zur Mitternacht schlug der HErr alle Erstgeburt in
Ägyptenland, von dem ersten Sohn Pharaos an, der auf seinem Stuhl saß,
bis auf den ersten Sohn des Gefangenen im Gefängnis, und alle Erstgeburt
des Viehes. \bibverse{30} Da stand Pharao auf und alle seine Knechte in
derselben Nacht und alle Ägypter, und ward ein groß Geschrei in Ägypten;
denn es war kein Haus, da nicht ein Toter innen wäre. \bibverse{31} Und
er forderte Mose und Aaron in der Nacht und sprach: Machet euch auf und
ziehet aus von meinem Volk, ihr und die Kinder Israel; gehet hin und
dienet dem HErrn, wie ihr gesagt habt. \bibverse{32} Nehmet auch mit
euch eure Schafe und Rinder, wie ihr gesagt habt; gehet hin und segnet
mich auch. \bibverse{33} Und die Ägypter drungen das Volk, daß sie es
eilend aus dem Lande trieben; denn sie sprachen: Wir sind alle des
Todes. \bibverse{34} Und das Volk trug den rohen Teig, ehe denn er
versäuert war, zu ihrer Speise, gebunden in ihren Kleidern, auf ihren
Achseln. \bibverse{35} Und die Kinder Israel hatten getan, wie Mose
gesagt hatte, und von den Ägyptern gefordert silberne und güldene Geräte
und Kleider. \bibverse{36} Dazu hatte der HErr dem Volk Gnade gegeben
vor den Ägyptern, daß sie ihnen leiheten; und entwandten es den
Ägyptern. \bibverse{37} Also zogen aus die Kinder Israel von Raemses gen
Suchoth, sechshunderttausend Mann zu Fuß ohne die Kinder. \bibverse{38}
Und zog auch mit ihnen viel Pöbelvolk und Schafe und Rinder und fast
viel Viehes. \bibverse{39} Und sie buken aus dem rohen Teige, den sie
aus Ägypten brachten, ungesäuerte Kuchen; denn es war nicht gesäuert,
weil sie aus Ägypten gestoßen wurden, und konnten nicht verziehen und
hatten ihnen sonst keine Zehrung zubereitet. \bibverse{40} Die Zeit
aber, die die Kinder Israel in Ägypten gewohnet haben, ist
vierhundertunddreißig Jahre. \bibverse{41} Da dieselben um waren, ging
das ganze Heer des HErrn auf einen Tag aus Ägyptenland. \bibverse{42}
Darum wird diese Nacht dem HErrn gehalten, daß er sie aus Ägyptenland
geführet hat; und die Kinder Israel sollen sie dem HErrn halten, sie und
ihre Nachkommen. \bibverse{43} Und der HErr sprach zu Mose und Aaron:
Dies ist die Weise, Passah zu halten: Kein Fremder soll davon essen.
\bibverse{44} Aber wer ein erkaufter Knecht ist, den beschneide man, und
dann esse er davon. \bibverse{45} Ein Hausgenoß und Mietling sollen
nicht davon essen. \bibverse{46} In einem Hause soll man's essen; ihr
sollt nichts von seinem Fleisch hinaus vor das Haus tragen; und sollt
kein Bein an ihm zerbrechen. \bibverse{47} Die ganze Gemeine Israel soll
solches tun. \bibverse{48} So aber ein Fremdling bei dir wohnet und dem
HErrn das Passah halten will, der beschneide alles, was männlich ist;
als dann mache er sich herzu, daß er solches tue, und sei wie ein
Einheimischer des Landes; denn kein Unbeschnittener soll davon essen.
\bibverse{49} Einerlei Gesetz sei dem Einheimischen und dem Fremdling,
der unter euch wohnet. \bibverse{50} Und alle Kinder Israel taten, wie
der HErr Mose und Aaron hatte geboten. \bibverse{51} Also führete der
HErr auf einen Tag die Kinder Israel aus Ägyptenland mit ihrem Heer.

\hypertarget{section-12}{%
\section{13}\label{section-12}}

\bibverse{1} Und der HErr redete mit Mose und sprach: \bibverse{2}
Heilige mir alle Erstgeburt, die allerlei Mutter bricht, bei den Kindern
Israel, beide unter den Menschen und dem Vieh; denn sie sind mein.
\bibverse{3} Da sprach Mose zum Volk: Gedenket an diesen Tag, an dem ihr
aus Ägypten, aus dem Diensthause, gegangen seid, daß der HErr euch mit
mächtiger Hand von hinnen hat ausgeführet; darum sollst du nicht
Sauerteig essen. \bibverse{4} Heute seid ihr ausgegangen, in dem Mond
Abib. \bibverse{5} Wenn dich nun der HErr bringen wird in das Land der
Kanaaniter, Hethiter, Amoriter, Heviter und Jebusiter, das er deinen
Vätern geschworen hat, dir zu geben, ein Land, da Milch und Honig innen
fleußt, so sollst du diesen Dienst halten in diesem Mond. \bibverse{6}
Sieben Tage sollst du ungesäuert Brot essen, und am siebenten Tage ist
des HErrn Fest. \bibverse{7} Darum sollst du sieben Tage ungesäuert Brot
essen, daß bei dir kein Sauerteig noch gesäuert Brot gesehen werde an
allen deinen Orten. \bibverse{8} Und sollt euren Söhnen sagen an dem
selbigen Tage: Solches halten wir um deswillen, das uns der HErr getan
hat, da wir aus Ägypten zogen. \bibverse{9} Darum soll dir's sein ein
Zeichen in deiner Hand und ein Denkmal vor deinen Augen, auf daß des
HErrn Gesetz sei in deinem Munde, daß der HErr dich mit mächtiger Hand
aus Ägypten geführet hat. \bibverse{10} Darum halte diese Weise zu
seiner Zeit jährlich. \bibverse{11} Wenn dich nun der HErr ins Land der
Kanaaniter gebracht hat, wie er dir und deinen Vätern geschworen hat,
und dir's gegeben, \bibverse{12} so sollst du aussondern dem HErrn
alles, was die Mutter bricht, und die Erstgeburt unter dem Vieh, das ein
Männlein ist. \bibverse{13} Die Erstgeburt vom Esel sollst du lösen mit
einem Schaf; wo du es aber nicht lösest, so brich ihm das Genick. Aber
alle erste Menschengeburt unter deinen Kindern sollst du lösen.
\bibverse{14} Und wenn dich heute oder morgen dein Kind wird fragen: Was
ist das? sollst du ihm sagen: Der HErr hat uns mit mächtiger Hand aus
Ägypten, von dem Diensthause, geführet. \bibverse{15} Denn da Pharao
hart war, uns loszulassen, erschlug der HErr alle Erstgeburt in
Ägyptenland, von der Menschen Erstgeburt an bis an die Erstgeburt des
Viehes. Darum opfere ich dem HErrn alles, was die Mutter bricht, das ein
Männlein ist, und die Erstgeburt meiner Kinder löse ich. \bibverse{16}
Und das soll dir ein Zeichen in deiner Hand sein und ein Denkmal vor
deinen Augen, daß uns der HErr hat mit mächtiger Hand aus Ägypten
geführet. \bibverse{17} Da nun Pharao das Volk gelassen hatte, führete
sie GOtt nicht auf die Straße durch der Philister Land, die am nächsten
war; denn GOtt gedachte, es möchte das Volk gereuen, wenn sie den Streit
sahen, und wieder nach Ägypten umkehren. \bibverse{18} Darum führete er
das Volk um auf die Straße durch die Wüste am Schilfmeer. Und die Kinder
Israel zogen gerüstet aus Ägyptenland. \bibverse{19} Und Mose nahm mit
sich die Gebeine Josephs. Denn er hatte einen Eid von den Kindern Israel
genommen und gesprochen: GOtt wird euch heimsuchen; so führet meine
Gebeine mit euch von hinnen. \bibverse{20} Also zogen sie aus von
Suchoth und lagerten sich in Etham, vorn an der Wüste. \bibverse{21} Und
der HErr zog vor ihnen her, des Tages in einer Wolkensäule, daß er sie
den rechten Weg führete, und des Nachts in einer Feuersäule, daß er
ihnen leuchtete, zu reisen Tag und Nacht. \bibverse{22} Die Wolkensäule
wich nimmer von dem Volk des Tages, noch die Feuersäule des Nachts.

\hypertarget{section-13}{%
\section{14}\label{section-13}}

\bibverse{1} Und der HErr redete mit Mose und sprach: \bibverse{2} Rede
mit den Kindern Israel und sprich, daß sie sich herumlenken und sich
lagern gegen dem Tal Hiroth, zwischen Migdol und dem Meer, gegen
Baal-Zephon, und daselbst gegenüber sich lagern ans Meer. \bibverse{3}
Denn Pharao wird sagen von den Kindern Israel: Sie sind verirret im
Lande, die Wüste hat sie beschlossen. \bibverse{4} Und ich will sein
Herz verstocken, daß er ihnen nachjage, und will an Pharao und an aller
seiner Macht Ehre einlegen, und die Ägypter sollen inne werden, daß ich
der HErr bin. Und sie taten also. \bibverse{5} Und da es dem Könige in
Ägypten ward angesagt, daß das Volk war geflohen, ward sein Herz
verwandelt und seiner Knechte gegen das Volk, und sprachen: Warum haben
wir das getan, daß wir Israel haben gelassen, daß sie uns nicht
dieneten? \bibverse{6} Und er spannte seinen Wagen an und nahm sein Volk
mit ihm \bibverse{7} und nahm sechshundert auserlesene Wagen, und was
sonst von Wagen in Ägypten war, und die Hauptleute über all sein Heer.
\bibverse{8} Denn der HErr verstockte das Herz Pharaos, des Königs in
Ägypten, daß er den Kindern Israel nachjagete. Aber die Kinder Israel
waren durch eine hohe Hand ausgegangen. \bibverse{9} Und die Ägypter
jagten ihnen nach und ereileten sie (da sie sich gelagert hatten am
Meer) mit Rossen und Wagen und Reitern und allem Heer des Pharao im Tal
Hiroth, gegen Baal-Zephon. \bibverse{10} Und da Pharao nahe zu ihnen
kam, huben die Kinder Israel ihre Augen auf, und siehe, die Ägypter
zogen hinter ihnen her; und sie fürchteten sich sehr und schrieen zu dem
HErrn. \bibverse{11} Und sprachen zu Mose: Waren nicht Gräber in
Ägypten, daß du uns mußtest wegführen, daß wir in der Wüste sterben?
Warum hast du das getan, daß du uns aus Ägypten geführet hast?
\bibverse{12} Ist's nicht das, das wir dir sagten in Ägypten: Höre auf
und laß uns den Ägyptern dienen? Denn es wäre uns ja besser, den
Ägyptern zu dienen, denn in der Wüste sterben. \bibverse{13} Mose sprach
zum Volk: Fürchtet euch nicht, stehet fest und sehet zu was für ein Heil
der HErr heute an euch tun wird. Denn diese Ägypter, die ihr heute
sehet, werdet ihr nimmermehr sehen ewiglich. \bibverse{14} Der HErr wird
für euch streiten, und ihr werdet stille sein. \bibverse{15} Der HErr
sprach zu Mose: Was schreiest du zu mir? Sage den Kindern Israel, daß
sie ziehen! \bibverse{16} Du aber heb deinen Stab auf und recke deine
Hand über das Meer und teile es voneinander, daß die Kinder Israel
hineingehen, mitten hindurch auf dem Trockenen. \bibverse{17} Siehe, ich
will das Herz der Ägypter verstocken, daß sie euch nachfolgen. So will
ich Ehre einlegen an dem Pharao und an aller seiner Macht, an seinen
Wagen und Reitern. \bibverse{18} Und die Ägypter sollen's inne werden,
daß ich der HErr bin, wenn ich Ehre eingelegt habe an Pharao und an
seinen Wagen und Reitern. \bibverse{19} Da erhub sich der Engel GOttes,
der vor dem Heer Israels herzog, und machte sich hinter sie; und die
Wolkensäule machte sich auch von ihrem Angesicht und trat hinter sie
\bibverse{20} und kam zwischen das Heer der Ägypter und das Heer
Israels. Es war aber eine finstere Wolke und erleuchtete die Nacht, daß
sie die ganze Nacht, diese und jene, nicht zusammenkommen konnten.
\bibverse{21} Da nun Mose seine Hand reckte über das Meer, ließ es der
HErr hinwegfahren durch einen starken Ostwind die ganze Nacht und machte
das Meer trocken; und die Wasser teilten sich voneinander. \bibverse{22}
Und die Kinder Israel gingen hinein, mitten ins Meer auf dem Trockenen;
und das Wasser war ihnen für Mauern zur Rechten und zur Linken.
\bibverse{23} Und die Ägypter folgten und gingen hinein ihnen nach, alle
Rosse Pharaos und Wagen und Reiter, mitten ins Meer. \bibverse{24} Als
nun die Morgenwache kam, schauete der HErr auf der Ägypter Heer aus der
Feuersäule und Wolke und machte ein Schrecken in ihrem Heer;
\bibverse{25} und stieß die Räder von ihren Wagen, stürzte sie mit
Ungestüm. Da sprachen die Ägypter: Lasset uns fliehen von Israel! Der
HErr streitet für sie wider die Ägypter. \bibverse{26} Aber der HErr
sprach zu Mose: Recke deine Hand aus über das Meer, daß das Wasser
wieder herfalle über die Ägypter, über ihre Wagen und Reiter.
\bibverse{27} Da reckte Mose seine Hand aus über das Meer; und das Meer
kam wieder vor Morgens in seinen Strom, und die Ägypter flohen ihm
entgegen. Also stürzte sie der HErr mitten ins Meer, \bibverse{28} daß
das Wasser wiederkam und bedeckte Wagen und Reiter und alle Macht des
Pharao, die ihnen nachgefolget waren ins Meer, daß nicht einer aus ihnen
überblieb. \bibverse{29} Aber die Kinder Israel gingen trocken mitten
durchs Meer; und das Wasser war ihnen für Mauern zur Rechten und zur
Linken. \bibverse{30} Also half der HErr Israel an dem Tage von der
Ägypter Hand. Und sie sahen die Ägypter tot am Ufer des Meers,
\bibverse{31} und die große Hand, die der HErr an den Ägyptern erzeigt
hatte. Und das Volk fürchtete den HErrn, und glaubten ihm und seinem
Knechte Mose.

\hypertarget{section-14}{%
\section{15}\label{section-14}}

\bibverse{1} Da sang Mose und die Kinder Israel dies Lied dem HErrn und
sprachen: Ich will dem HErrn singen; denn er hat eine herrliche Tat
getan: Roß und Wagen hat er ins Meer gestürzt. \bibverse{2} Der HErr ist
meine Stärke und Lobgesang und ist mein Heil. Das ist mein GOtt, ich
will ihn preisen; er ist meines Vaters GOtt, ich will ihn erheben.
\bibverse{3} Der HErr ist der rechte Kriegsmann. HErr ist sein Name.
\bibverse{4} Die Wagen Pharaos und seine Macht warf er ins Meer, seine
auserwählten Hauptleute versanken im Schilfmeer. \bibverse{5} Die Tiefe
hat sie bedeckt, sie fielen zu Grund wie die Steine. \bibverse{6} HErr,
deine rechte Hand tut große Wunder; HErr, deine rechte Hand hat die
Feinde zerschlagen. \bibverse{7} Und mit deiner großen Herrlichkeit hast
du deine Widerwärtigen gestürzt; denn da du deinen Grimm ausließest,
verzehrte er sie wie Stoppeln. \bibverse{8} Durch dein Blasen taten sich
die Wasser auf, und die Fluten stunden auf Haufen; die Tiefe wallete
voneinander mitten im Meer. \bibverse{9} Der Feind gedachte: Ich will
ihnen nachjagen und erhaschen und den Raub austeilen und meinen Mut an
ihnen kühlen; ich will mein Schwert ausziehen, und meine Hand soll sie
verderben. \bibverse{10} Da ließest du deinen Wind blasen, und das Meer
bedeckte sie, und sanken unter wie Blei im mächtigen Wasser.
\bibverse{11} HErr, wer ist dir gleich unter den Göttern? Wer ist dir
gleich, der so mächtig, heilig, schrecklich, löblich und wundertätig
sei? \bibverse{12} Da du deine rechte Hand ausrecktest, verschlang sie
die Erde. \bibverse{13} Du hast geleitet durch deine Barmherzigkeit dein
Volk, das du erlöset hast, und hast sie geführet durch deine Stärke zu
deiner heiligen Wohnung. \bibverse{14} Da das die Völker höreten,
erbebeten sie; Angst kam die Philister an; \bibverse{15} da erschraken
die Fürsten Edoms; Zittern kam die Gewaltigen Moabs an; alle Einwohner
Kanaans wurden feig. \bibverse{16} Laß über sie fallen Erschrecken und
Furcht durch deinen großen Arm, daß sie erstarren wie die Steine, bis
dein Volk, HErr, hindurchkomme, bis das Volk hindurchkomme, das du
erworben hast. \bibverse{17} Bringe sie hinein und pflanze sie auf dem
Berge deines Erbteils, den du, HErr, dir zur Wohnung gemacht hast, zu
deinem Heiligtum, HErr, das deine Hand bereitet hat. \bibverse{18} Der
HErr wird König sein immer und ewig. \bibverse{19} Denn Pharao zog
hinein ins Meer mit Rossen und Wagen und Reitern, und der HErr ließ das
Meer wieder über sie fallen. Aber die Kinder Israel gingen trocken
mitten durchs Meer. \bibverse{20} Und Mirjam, die Prophetin, Aarons
Schwester, nahm eine Pauke in ihre Hand; und alle Weiber folgten ihr
nach hinaus mit Pauken am Reigen. \bibverse{21} Und Mirjam sang ihnen
vor: Lasset uns dem HErrn singen; denn er hat eine herrliche Tat getan,
Mann und Roß hat er ins Meer gestürzt. \bibverse{22} Mose ließ die
Kinder Israel ziehen vom Schilfmeer hinaus zu der Wüste Sur. Und sie
wanderten drei Tage in der Wüste, daß sie kein Wasser fanden.
\bibverse{23} Da kamen sie gen Mara; aber sie konnten des Wassers zu
Mara nicht trinken, denn es war fast bitter. Daher hieß man den Ort
Mara. \bibverse{24} Da murrete das Volk wider Mose und sprach: Was
sollen wir trinken? \bibverse{25} Er schrie zu dem HErrn; und der HErr
weisete ihm einen Baum, den tat er ins Wasser; da ward es süß. Daselbst
stellete er ihnen ein Gesetz und ein Recht und versuchte sie
\bibverse{26} und sprach: Wirst du der Stimme des HErrn, deines GOttes,
gehorchen und tun, was recht ist vor ihm, und zu Ohren fassen seine
Gebote und halten alle seine Gesetze, so will ich der Krankheit keine
auf dich legen, die ich auf Ägypten gelegt habe; denn ich bin der HErr,
dein Arzt. \bibverse{27} Und sie kamen nach Elim, da waren zwölf
Wasserbrunnen und siebenzig Palmbäume; und lagerten sich daselbst ans
Wasser.

\hypertarget{section-15}{%
\section{16}\label{section-15}}

\bibverse{1} Von Elim zogen sie, und kam die ganze Gemeine der Kinder
Israel in die Wüste Sin, die da liegt zwischen Elim und Sinai, am
fünfzehnten Tage des andern Monden, nachdem sie aus Ägypten gezogen
waren. \bibverse{2} Und es murrete die ganze Gemeine der Kinder Israel
wider Mose und Aaron in der Wüste \bibverse{3} und sprachen: Wollte
GOtt, wir wären in Ägypten gestorben durch des HErrn Hand, da wir bei
den Fleischtöpfen saßen und hatten die Fülle Brot zu essen! Denn ihr
habt uns darum ausgeführet in diese Wüste, daß ihr diese ganze Gemeine
Hungers sterben lasset. \bibverse{4} Da sprach der HErr zu Mose: Siehe,
ich will euch Brot vom Himmel regnen lassen, und das Volk soll
hinausgehen und sammeln täglich, was es des Tages bedarf, daß ich's
versuche, ob es in meinem Gesetz wandele oder nicht. \bibverse{5} Des
sechsten Tages aber sollen sie sich schicken, daß sie zwiefältig
eintragen, weder sie sonst täglich sammeln. \bibverse{6} Mose und Aaron
sprachen zu allen Kindern Israel: Am Abend sollt ihr inne werden, daß
euch der HErr aus Ägyptenland geführet bat, \bibverse{7} und des Morgens
werdet ihr des HErrn Herrlichkeit sehen; denn er hat euer Murren wider
den HErrn gehöret. Was sind wir, daß ihr wider uns murret? \bibverse{8}
Weiter sprach Mose: Der HErr wird euch am Abend Fleisch zu essen geben
und am Morgen Brots die Fülle, darum daß der HErr euer Murren gehöret
hat, das ihr wider ihn gemurret habt. Denn was sind wir? Euer Murren ist
nicht wider uns, sondern wider den HErrn. \bibverse{9} Und Mose sprach
zu Aaron: Sage der ganzen Gemeine der Kinder Israel: Kommt herbei vor
den HErrn; denn er hat euer Murren gehöret. \bibverse{10} Und da Aaron
also redete zu der ganzen Gemeine der Kinder Israel, wandten sie sich
gegen die Wüste; und siehe, die Herrlichkeit des HErrn erschien in einer
Wolke. \bibverse{11} Und der HErr sprach zu Mose: \bibverse{12} Ich habe
der Kinder Israel Murren gehöret. Sage ihnen: Zwischen Abend sollt ihr
Fleisch zu essen haben und am Morgen Brots satt werden und inne werden,
daß ich der HErr, euer GOtt bin. \bibverse{13} Und am Abend kamen
Wachteln herauf und bedeckten das Heer. Und am Morgen lag der Tau um das
Heer her. \bibverse{14} Und als der Tau weg war, siehe, da lag es in der
Wüste rund und klein, wie der Reif auf dem Lande. \bibverse{15} Und da
es die Kinder Israel sahen, sprachen sie untereinander: Das ist Man;
denn sie wußten nicht, was es war. Mose aber sprach zu ihnen: Es ist das
Brot, das euch der HErr zu essen gegeben hat. \bibverse{16} Das ist's
aber, das der HErr geboten hat: Ein jeglicher sammle des, soviel er für
sich essen mag, und nehme ein Gomor auf ein jeglich Haupt, nach der Zahl
der Seelen in seiner Hütte. \bibverse{17} Und die Kinder Israel taten
also und sammelten, einer viel, der andere wenig. \bibverse{18} Aber da
man's mit dem Gomor maß, fand der nicht drüber, der viel gesammelt
hatte, und der nicht drunter, der wenig gesammelt hatte, sondern ein
jeglicher hatte gesammelt, soviel er für sich essen mochte.
\bibverse{19} Und Mose sprach zu ihnen: Niemand lasse etwas davon über
bis morgen. \bibverse{20} Aber sie gehorchten Mose nicht. Und etliche
ließen davon über bis morgen; da wuchsen Würmer drinnen, und ward
stinkend. Und Mose ward zornig auf sie. \bibverse{21} Sie sammelten aber
desselben alle Morgen, soviel ein jeglicher für sich essen mochte. Wenn
aber die Sonne heiß schien, zerschmolz es. \bibverse{22} Und des
sechsten Tages sammelten sie des Brots zwiefältig, je zwei Gomor für
einen. Und alle Obersten der Gemeine kamen hinein und verkündigten es
Mose. \bibverse{23} Und er sprach zu ihnen: Das ist's, das der HErr
gesagt hat: Morgen ist der Sabbat der heiligen Ruhe des HErrn; was ihr
backen wollt, das backet, und was ihr kochen wollt, das kochet; was aber
übrig ist, das lasset bleiben, daß es behalten werde bis morgen.
\bibverse{24} Und sie ließen's bleiben bis morgen, wie Mose geboten
hatte; da ward es nicht stinkend, und war auch kein Wurm drinnen.
\bibverse{25} Da sprach Mose: Esset das heute, denn es ist heute der
Sabbat des HErrn; ihr werdet es heute nicht finden auf dem Felde.
\bibverse{26} Sechs Tage sollt ihr sammeln, aber der siebente Tag ist
der Sabbat, darinnen wird's nicht sein. \bibverse{27} Aber am siebenten
Tage gingen etliche vom Volk hinaus zu sammeln, und fanden nichts.
\bibverse{28} Da sprach der HErr zu Mose: Wie lange weigert ihr euch, zu
halten meine Gebote und Gesetze? \bibverse{29} Sehet, der HErr hat euch
den Sabbat gegeben; darum gibt er euch am sechsten Tage zweier Tage
Brot. So bleibe nun ein jeglicher in dem Seinen, und niemand gehe heraus
von seinem Ort des siebenten Tages! \bibverse{30} Also feierte das Volk
des siebenten Tages. \bibverse{31} Und das Haus Israel hieß es Man. Und
es war wie Koriandersamen und weiß, und hatte einen Schmack wie Semmel
mit Honig. \bibverse{32} Und Mose sprach: Das ist's, das der HErr
geboten hat: Fülle ein Gomor davon, zu behalten auf eure Nachkommen, auf
daß man sehe das Brot, damit ich euch gespeiset habe in der Wüste, da
ich euch aus Ägyptenland führte. \bibverse{33} Und Mose sprach zu Aaron:
Nimm ein Krüglein und tu ein Gomor voll Man drein; und laß es vor dem
HErrn, zu behalten auf eure Nachkommen. \bibverse{34} Wie der HErr Mose
geboten hat, also ließ es Aaron daselbst vor dem Zeugnis, zu behalten.
\bibverse{35} Und die Kinder Israel aßen Man vierzig Jahre, bis daß sie
zu dem Lande kamen, da sie wohnen sollten; bis an die Grenze des Landes
Kanaan aßen sie Man. \bibverse{36} Ein Gomor aber ist das zehnte Teil
eines Epha.

\hypertarget{section-16}{%
\section{17}\label{section-16}}

\bibverse{1} Und die ganze Gemeine der Kinder Israel zog aus der Wüste
Sin, ihre Tagereisen, wie ihnen der HErr befahl, und lagerten sich in
Raphidim. Da hatte das Volk kein Wasser zu trinken. \bibverse{2} Und sie
zankten mit Mose und sprachen: Gebet uns Wasser, daß wir trinken! Mose
sprach zu ihnen: Was zanket ihr mit mir? Warum versuchet ihr den HErrn?
\bibverse{3} Da aber das Volk daselbst dürstete nach Wasser, murreten
sie wider Mose und sprachen: Warum hast du uns lassen aus Ägypten
ziehen, daß du uns, unsere Kinder und Vieh Durst sterben ließest?
\bibverse{4} Mose schrie zum HErrn und sprach: Wie soll ich mit dem Volk
tun? Es fehlet nicht weit, sie werden mich noch steinigen. \bibverse{5}
Der HErr sprach zu ihm: Gehe vorhin vor dem Volk und nimm etliche
Älteste von Israel mit dir; und nimm deinen Stab in deine Hand, damit du
das Wasser schlugest, und gehe hin. \bibverse{6} Siehe, ich will
daselbst stehen vor dir auf einem Fels in Horeb; da sollst du den Felsen
schlagen, so wird Wasser heraus laufen, daß das Volk trinke. Mose tat
also vor den Ältesten von Israel. \bibverse{7} Da hieß man den Ort Massa
und Meriba um des Zanks willen der Kinder Israel, und daß sie den HErrn
versucht und gesagt hatten: Ist der HErr unter uns oder nicht?
\bibverse{8} Da kam Amalek und stritt wider Israel in Raphidim.
\bibverse{9} Und Mose sprach zu Josua: Erwähle uns Männer, zeuch aus und
streite wider Amalek; morgen will ich auf des Hügels Spitze stehen und
den Stab GOttes, in meiner Hand haben. \bibverse{10} Und Josua tat, wie
Mose ihm sagte, daß er wider Amalek stritte. Mose aber und Aaron und Hur
gingen auf die Spitze des Hügels. \bibverse{11} Und dieweil Mose seine
Hände emporhielt, siegte Israel; wenn er aber seine Hände niederließ,
siegte Amalek. \bibverse{12} Aber die Hände Moses waren schwer; darum
nahmen sie einen Stein und legten ihn unter ihn, daß er sich darauf
setzte. Aaron aber und Hur unter hielten ihm seine Hände, auf jeglicher
Seite einer. Also blieben seine Hände steif, bis die Sonne unterging.
\bibverse{13} Und Josua dämpfte den Amalek und sein Volk durch des
Schwerts Schärfe. \bibverse{14} Und der HErr sprach zu Mose: Schreibe
das zum Gedächtnis in ein Buch und befiehl es in die Ohren Josuas; denn
ich will den Amalek unter dem Himmel austilgen, daß man sein nicht mehr
gedenke. \bibverse{15} Und Mose bauete einen Altar und hieß ihn: Der
HErr Nissi. \bibverse{16} Denn er sprach: Es ist ein Malzeichen bei dem
Stuhl des HErrn, daß der HErr streiten wird wider Amalek von Kind zu
Kindeskind.

\hypertarget{section-17}{%
\section{18}\label{section-17}}

\bibverse{1} Und da Jethro, der Priester in Midian, Moses
Schwiegervater, hörete alles, was GOtt getan hatte mit Mose und seinem
Volk Israel, daß der HErr Israel hätte aus Ägypten geführet,
\bibverse{2} nahm er Zipora, Moses Weib, die er hatte zurückgesandt,
\bibverse{3} samt ihren zween Söhnen, der eine hieß Gersom; denn er
sprach: Ich bin ein Gast worden in fremdem Lande; \bibverse{4} und der
andere Elieser; denn er sprach: Der GOtt meines Vaters ist meine Hilfe
gewesen und hat mich errettet von dem Schwert Pharaos. \bibverse{5} Da
nun Jethro, Moses Schwäher, und seine Söhne und sein Weib zu ihm kamen
in die Wüste, an den Berg GOttes, da er sich gelagert hatte,
\bibverse{6} ließ er Mose sagen: Ich, Jethro, dein Schwäher, bin zu dir
kommen, und dein Weib und ihre beiden Söhne mit ihr. \bibverse{7} Da
ging ihm Mose entgegen hinaus und neigte sich vor ihm und küssete ihn.
Und da sie sich untereinander gegrüßet hatten, gingen sie in die Hütte.
\bibverse{8} Da erzählete Mose seinem Schwäher alles, was der HErr
Pharao und den Ägyptern getan hatte Israels halben, und alle die Mühe,
die ihnen auf dem Wege begegnet war, und daß sie der HErr errettet
hätte. \bibverse{9} Jethro aber freuete sich all des Guten, das der HErr
Israel getan hatte, daß er sie errettet hatte von der Ägypter Hand.
\bibverse{10} Und Jethro sprach: Gelobet sei der HErr, der euch errettet
hat von der Ägypter und Pharaos Hand, der weiß sein Volk von der Ägypter
Hand zu erretten. \bibverse{11} Nun weiß ich, daß der HErr größer ist
denn alle Götter, darum daß sie Hochmut an ihnen geübet haben.
\bibverse{12} Und Jethro, Moses Schwäher, nahm Brandopfer und opferte
GOtt. Da kam Aaron und alle Ältesten in Israel, mit Moses Schwäher das
Brot zu essen vor GOtt. \bibverse{13} Des andern Morgens setzte sich
Mose, das Volk zu richten; und das Volk stund um Mose her von Morgen an
bis zu Abend. \bibverse{14} Da aber sein Schwäher sah alles, was er mit
dem Volk tat, sprach er: Was ist, das du tust mit dem Volk? Warum
sitzest du allein, und alles Volk stehet um dich her von Morgen an bis
zu Abend? \bibverse{15} Mose antwortete ihm: Das Volk kommt zu mir und
fragen GOtt um Rat. \bibverse{16} Denn wo sie was zu schaffen haben,
kommen sie zu mir, daß ich richte zwischen einem jeglichen und seinem
Nächsten und zeige ihnen GOttes Rechte und seine Gesetze. \bibverse{17}
Sein Schwäher sprach zu ihm: Es ist nicht gut, das du tust.
\bibverse{18} Du machest dich zu müde, dazu das Volk auch, das mit dir
ist. Das Geschäft ist dir zu schwer, du kannst es allein nicht
ausrichten. \bibverse{19} Aber gehorche meiner Stimme; ich will dir
raten, und GOtt wird mit dir sein. Pflege du des Volks vor GOtt und
bringe die Geschäfte vor GOtt; \bibverse{20} und stelle ihnen Rechte und
Gesetze, daß du sie lehrest den Weg, darin sie wandeln, und die Werke,
die sie tun sollen. \bibverse{21} Sieh dich aber um unter allem Volk
nach redlichen Leuten, die GOtt fürchten, wahrhaftig und dem Geiz feind
sind; die setze über sie, etliche über tausend, über hundert, über
fünfzig und über zehn, \bibverse{22} daß sie das Volk allezeit richten;
wo aber eine große Sache ist, daß sie dieselbe an dich bringen, und sie
alle geringen Sachen richten. So wird dir's leichter werden, und sie mit
dir tragen. \bibverse{23} Wirst du das tun, so kannst du ausrichten, was
dir GOtt gebeut, und all dies Volk kann mit Frieden an seinen Ort
kommen. \bibverse{24} Mose gehorchte seines Schwähers Wort und tat
alles, was er sagte, \bibverse{25} und redliche Leute aus dem ganzen
Israel und machte sie zu Häuptern über das Volk, etliche über tausend,
über hundert, über fünfzig und über zehn \bibverse{26} daß sie das Volk
allezeit richteten, was aber schwere Sachen wären, zu Mose brächten, und
die kleinen Sachen sie richteten. \bibverse{27} Also ließ Mose seinen
Schwäher in sein Land ziehen.

\hypertarget{section-18}{%
\section{19}\label{section-18}}

\bibverse{1} Im dritten Mond nach dem Ausgang der Kinder Israel aus
Ägyptenland kamen sie dieses Tages in die Wüste Sinai. \bibverse{2} Denn
sie waren ausgezogen von Raphidim und wollten in die Wüste Sinai; und
lagerten sich in der Wüste daselbst gegen dem Berg. \bibverse{3} Und
Mose stieg hinauf zu GOtt. Und der HErr rief ihm vom Berge und sprach:
So sollst du sagen zu dem Hause Jakob und verkündigen den Kindern
Israel: \bibverse{4} Ihr habt gesehen, was ich den Ägyptern getan habe,
und wie ich euch getragen habe auf Adlersflügeln und habe euch zu mir
gebracht. \bibverse{5} Werdet ihr nun meiner Stimme gehorchen und meinen
Bund halten, so sollt ihr mein Eigentum sein vor allen Völkern; denn die
ganze Erde ist mein. \bibverse{6} Und ihr sollt mir ein priesterlich
Königreich und ein heiliges Volk sein. Das sind die Worte, die du den
Kindern Israel sagen sollst. \bibverse{7} Mose kam und forderte die
Ältesten im Volk und legte ihnen alle diese Worte vor, die der HErr
geboten hatte. \bibverse{8} Und alles Volk antwortete zugleich und
sprachen: Alles, was der HErr geredet hat, wollen wir tun. Und Mose
sagte die Rede des Volks dem HErrn wieder. \bibverse{9} Und der HErr
sprach zu Mose: Siehe, ich will zu dir kommen in einer dicken Wolke, auf
daß dies Volk meine Worte höre, die ich mit dir rede, und glaube dir
ewiglich. Und Mose verkündigte dem HErrn die Rede des Volks.
\bibverse{10} Der HErr sprach zu Mose: Gehe hin zum Volk und heilige sie
heute und morgen, daß sie ihre Kleider waschen. \bibverse{11} und bereit
seien auf den dritten Tag. Denn am dritten Tage wird der HErr vor allem
Volk herabfahren auf den Berg Sinai. \bibverse{12} Und mache dem Volk
ein Gehege umher und sprich zu ihnen: Hütet euch, daß ihr nicht auf den
Berg steiget, noch sein Ende anrühret; denn wer den Berg anrühret, soll
des Todes sterben. \bibverse{13} Keine Hand soll ihn anrühren, sondern
er soll gesteinigt oder mit Geschoß erschossen werden; es sei ein Tier
oder Mensch, so soll er nicht leben. Wenn es aber lange tönen wird, dann
sollen sie an den Berg gehen. \bibverse{14} Mose stieg vom Berge zum
Volk und heiligte sie, und sie wuschen ihre Kleider. \bibverse{15} Und
er sprach zu ihnen: Seid bereit auf den dritten Tag, und keiner nahe
sich zum Weibe. \bibverse{16} Als nun der dritte Tag kam und Morgen war,
da hub sich ein Donnern und Blitzen und eine dicke Wolke auf dem Berge
und ein Ton einer sehr starken Posaune. Das ganze Volk aber, das im
Lager war, erschrak. \bibverse{17} Und Mose führete das Volk aus dem
Lager GOtt entgegen; und sie traten unten an den Berg. \bibverse{18} Der
ganze Berg aber Sinai rauchte, darum daß der HErr herab auf den Berg
fuhr mit Feuer; und sein Rauch ging auf wie ein Rauch vom Ofen, daß der
ganze Berg sehr bebete. \bibverse{19} Und der Posaunen Ton ward immer
stärker. Mose redete, und GOtt antwortete ihm laut. \bibverse{20} Als
nun der HErr herniederkommen war auf den Berg Sinai, oben auf seine
Spitze, forderte er Mose oben auf die Spitze des Berges, und Mose stieg
hinauf. \bibverse{21} Da sprach der HErr zu ihm: Steig hinab und zeuge
dem Volk, daß sie nicht herzubrechen zum HErrn, daß sie sehen, und viele
aus ihnen fallen. \bibverse{22} Dazu die Priester, die zum HErrn nahen,
sollen sich heiligen, daß sie der HErr nicht zerschmettere.
\bibverse{23} Mose aber sprach zum HErrn: Das Volk kann nicht auf den
Berg Sinai steigen; denn du hast uns bezeuget und gesagt: Mache ein
Gehege um den Berg und heilige ihn. \bibverse{24} Und der HErr sprach zu
ihm: Gehe hin, steige hinab; du und Aaron mit dir soll heraufsteigen;
aber die Priester und das Volk sollen nicht herzubrechen, daß sie
hinaufsteigen zu dem HErrn, daß er sie nicht zerschmettere.
\bibverse{25} Und Mose stieg herunter zum Volk und sagte es ihnen.

\hypertarget{section-19}{%
\section{20}\label{section-19}}

\bibverse{1} Und GOtt redete alle diese Worte: \bibverse{2} Ich bin der
HErr, dein GOtt, der ich dich aus Ägyptenland, aus dem Diensthause,
geführet habe. \bibverse{3} Du sollst keine andern Götter neben mir
haben. \bibverse{4} Du sollst dir kein Bildnis noch irgend ein Gleichnis
machen weder des, das oben im Himmel, noch des, das unten auf Erden,
oder des, das im Wasser unter der Erde ist. \bibverse{5} Bete sie nicht
an und diene ihnen nicht. Denn ich, der HErr, dein GOtt, bin ein
eifriger GOtt, der da heimsuchet der Väter Missetat an den Kindern bis
in das dritte und vierte Glied, die mich hassen, \bibverse{6} und tue
Barmherzigkeit an vielen Tausenden, die mich liebhaben und meine Gebote
halten. \bibverse{7} Du sollst den Namen des HErrn, deines GOttes, nicht
mißbrauchen; denn der HErr wird den nicht ungestraft lassen, der seinen
Namen mißbraucht. \bibverse{8} Gedenke des Sabbattages, daß du ihn
heiligest. \bibverse{9} Sechs Tage sollst du arbeiten und alle deine
Dinge beschicken; \bibverse{10} aber am siebenten Tage ist der Sabbat
des HErrn, deines GOttes. Da sollst du kein Werk tun, noch dein Sohn,
noch deine Tochter, noch dein Knecht, noch deine Magd, noch dein Vieh,
noch dein Fremdling, der in deinen Toren ist. \bibverse{11} Denn in
sechs Tagen hat der HErr Himmel und Erde gemacht und das Meer und alles,
was drinnen ist, und ruhete am siebenten Tage. Darum segnete der HErr
den Sabbattag und heiligte ihn. \bibverse{12} Du sollst deinen Vater und
deine Mutter ehren, auf daß du lange lebest im Lande, das dir der HErr,
dein GOtt gibt. \bibverse{13} Du sollst nicht töten. \bibverse{14} Du
sollst nicht ehebrechen. \bibverse{15} Du sollst nicht stehlen.
\bibverse{16} Du sollst kein falsch Zeugnis reden wider deinen Nächsten.
\bibverse{17} Laß dich nicht gelüsten deines Nächsten Hauses. Laß dich
nicht gelüsten deines Nächsten Weibes, noch seines Knechts, noch seiner
Magd, noch seines Ochsen, noch seines Esels, noch alles, das dein
Nächster hat. \bibverse{18} Und alles Volk sah den Donner und Blitz und
den Ton der Posaune und den Berg rauchen. Da sie aber solches sahen,
flohen sie und traten von ferne; \bibverse{19} und sprachen zu Mose:
Rede du mit uns, wir wollen gehorchen, und laß GOtt nicht mit uns reden,
wir möchten sonst sterben. \bibverse{20} Mose aber sprach zum Volk:
Fürchtet euch nicht; denn GOtt ist kommen, daß er euch versuchte, und
daß seine Furcht euch vor Augen wäre, daß ihr nicht sündiget.
\bibverse{21} Also trat das Volk von ferne; aber Mose machte sich hinzu
ins Dunkel, da GOtt innen war. \bibverse{22} Und der HErr sprach zu ihm:
Also sollst du den Kindern Israel sagen: Ihr habt gesehen, daß ich mit
euch vom Himmel geredet habe. \bibverse{23} Darum sollt ihr nichts neben
mir machen, silberne und güldene Götter sollt ihr nicht machen.
\bibverse{24} Einen Altar von Erde mache mir, darauf du dein Brandopfer
und Dankopfer, deine Schafe und Rinder opferst. Denn an welchem Ort ich
meines Namens Gedächtnis stiften werde, da will ich zu dir kommen und
dich segnen. \bibverse{25} Und so du mir einen steinernen Altar willst
machen, sollst du ihn nicht von gehauenen Steinen bauen; denn wo du mit
deinem Messer darüber fährest, so wirst du ihn entweihen. \bibverse{26}
Du sollst auch nicht auf Stufen zu meinem Altar steigen, daß nicht deine
Scham aufgedeckt werde vor ihm.

\hypertarget{section-20}{%
\section{21}\label{section-20}}

\bibverse{1} Dies sind die Rechte, die du ihnen sollst vorlegen:
\bibverse{2} So du einen ebräischen Knecht kaufest, der soll dir sechs
Jahre dienen; im siebenten Jahr soll er frei ledig ausgehen.
\bibverse{3} Ist er ohne Weib kommen, so, soll er auch ohne Weib
ausgehen. Ist er aber mit Weib kommen, so soll sein Weib mit ihm
ausgehen. \bibverse{4} Hat ihm aber sein Herr ein Weib gegeben und hat
Söhne oder Töchter gezeuget, so soll das Weib und die Kinder seines
Herrn sein; er aber soll ohne Weib ausgehen. \bibverse{5} Spricht aber
der Knecht: Ich habe meinen Herrn lieb und mein Weib und Kind, ich will
nicht frei werden, \bibverse{6} So bringe ihn sein Herr vor die Götter
und halte ihn an die Tür oder Pfosten und bohre ihm mit einem Pfriemen
durch sein Ohr; und er sei sein Knecht ewig. \bibverse{7} Verkauft
jemand seine Tochter zur Magd, so soll sie nicht ausgehen wie die
Knechte. \bibverse{8} Gefällt sie aber ihrem Herrn nicht und will ihr
nicht zur Ehe helfen, so soll er sie zu lösen geben. Aber unter ein
fremd Volk sie zu verkaufen, hat er nicht Macht, weil er sie verschmähet
hat. \bibverse{9} Vertrauet er sie aber seinem Sohn, so soll er
Tochterrecht an ihr tun. \bibverse{10} Gibt er ihm aber eine andere, so
soll er ihr an ihrem Futter, Decke und Eheschuld nicht abbrechen.
\bibverse{11} Tut er diese drei nicht, so soll sie frei ausgehen ohne
Lösegeld. \bibverse{12} Wer einen Menschen schlägt, daß er stirbt, der
soll des Todes sterben. \bibverse{13} Hat er ihm aber nicht
nachgestellet sondern GOtt hat ihn lassen ohngefähr in seine Hände
fallen, so will ich dir einen Ort bestimmen, dahin er fliehen soll.
\bibverse{14} Wo aber jemand an seinem Nächsten frevelt und ihn mit List
erwürget, so sollst du denselben von meinem Altar nehmen, daß man ihn
töte. \bibverse{15} Wer seinen Vater oder Mutter schlägt, der soll des
Todes sterben. \bibverse{16} Wer einen Menschen stiehlt und verkaufet,
daß man ihn bei ihm findet, der soll des Todes sterben. \bibverse{17}
Wer Vater oder Mutter flucht, der soll des Todes sterben. \bibverse{18}
Wenn sich Männer miteinander hadern, und einer schlägt den andern mit
einem Stein oder mit einer Faust, daß er nicht stirbt, sondern zu Bette
liegt: \bibverse{19} kommt er auf, daß er ausgehet an seinem Stabe so
soll, der ihn schlug, unschuldig sein, ohne daß er ihm bezahle, was er
versäumet hat, und das Arztgeld gebe. \bibverse{20} Wer seinen Knecht
oder Magd schlägt mit einem Stabe, daß er stirbt unter seinen Händen,
der, soll darum gestraft werden. \bibverse{21} Bleibt er aber einen oder
zween Tage, so soll er nicht darum gestraft werden; denn es ist sein
Geld. \bibverse{22} Wenn sich Männer hadern und verletzen ein schwanger
Weib, daß ihr die Frucht abgehet, und ihr kein Schade widerfährt, so
soll man ihn um Geld strafen, wieviel des Weibes Mann ihm auflegt, und
soll's geben nach der Teidingsleute Erkennen. \bibverse{23} Kommt ihr
aber ein Schade daraus, so soll er lassen Seele um Seele, \bibverse{24}
Auge um Auge, Zahn um Zahn, Hand um Hand, Fuß um Fuß, \bibverse{25}
Brand um Brand, Wunde um Wunde, Beule um Beule. \bibverse{26} Wenn
jemand seinen Knecht oder seine Magd in ein Auge schlägt und verderbet
es, der soll sie frei loslassen um das Auge. \bibverse{27}
Desselbigengleichen, wenn er seinem Knecht oder Magd einen Zahn
ausschlägt, soll er sie frei loslassen um den Zahn. \bibverse{28} Wenn
ein Ochse einen Mann oder Weib stößet, daß er stirbt, so soll man den
Ochsen steinigen und sein Fleisch nicht essen; so ist der Herr des
Ochsen unschuldig. \bibverse{29} Ist aber der Ochse vorhin stößig
gewesen, und seinem Herrn ist's angesagt, und er ihn nicht verwahret
hat, und tötet darüber einen Mann oder Weib, soll man den Ochsen
steinigen, und sein Herr soll sterben. \bibverse{30} Wird man aber ein
Geld auf ihn legen, so soll er geben, sein Leben zu lösen, was man ihm
auflegt. \bibverse{31} Desselbigengleichen soll man mit ihm handeln,
wenn er Sohn oder Tochter stößet. \bibverse{32} Stößet er aber einen
Knecht oder Magd, so soll er ihrem Herrn dreißig silberne Sekel geben,
und den Ochsen soll man steinigen. \bibverse{33} so jemand eine Grube
auftut, oder gräbt eine Grube und decket sie nicht zu, und fällt darüber
ein Ochse oder Esel hinein, \bibverse{34} so soll's der Herr der Grube
mit Geld dem andern wieder bezahlen; das Aas aber soll sein sein.
\bibverse{35} Wenn jemandes Ochse eines andern Ochsen stößet, daß er
stirbt, so sollen sie den lebendigen Ochsen verkaufen und das Geld
teilen und das Aas auch teilen. \bibverse{36} Ist's aber kund gewesen,
daß der Ochse stößig vorhin gewesen ist, und sein Herr hat ihn nicht
verwahret, so soll er einen Ochsen um den andern vergelten und das Aas
haben.

\hypertarget{section-21}{%
\section{22}\label{section-21}}

\bibverse{1} Wenn jemand einen Ochsen oder Schaf stiehlt und schlachtet
es oder verkauft es, der soll fünf Ochsen für einen Ochsen wiedergeben
und vier Schafe für ein Schaf. \bibverse{2} Wenn ein Dieb ergriffen
wird, daß er einbricht, und wird drob geschlagen, daß er stirbt, so soll
man kein Blutgericht über jenen lassen gehen. \bibverse{3} Ist aber die
Sonne über ihm aufgegangen, so soll man das Blutgericht gehen lassen. Es
soll aber ein Dieb wiedererstatten. Hat er nichts, so verkaufe man ihn
um seinen Diebstahl. \bibverse{4} Findet man aber bei ihm den Diebstahl
lebendig, es sei Ochse, Esel oder Schaf, so soll er's zwiefältig
wiedergeben. \bibverse{5} Wenn jemand einen Acker oder Weinberg
beschädiget, daß er sein Vieh lässet Schaden tun in eines andern Acker,
der soll von dem Besten auf seinem Acker und Weinberge wiedererstatten.
\bibverse{6} Wenn ein Feuer auskommt und ergreift die Dornen und
verbrennet die Garben oder Getreide, das noch stehet, oder den Acker, so
soll der wiedererstatten, der das Feuer angezündet hat. \bibverse{7}
Wenn jemand seinem Nächsten Geld oder Geräte zu behalten tut, und wird
demselbigen aus seinem Hause gestohlen: findet man den Dieb, so soll
er's zwiefältig wiedergeben. \bibverse{8} Findet man aber den Dieb
nicht, so soll man den Hauswirt vor die Götter bringen, ob er nicht
seine Hand habe an seines Nächsten Habe gelegt. \bibverse{9} Wo einer
den andern schuldiget um einigerlei Unrecht, es sei um Ochsen oder Esel
oder Schaf oder Kleider oder allerlei, das verloren ist, so sollen
beider Sachen vor die Götter kommen. Welchen die Götter verdammen, der
soll's zwiefältig seinem Nächsten wiedergeben. \bibverse{10} Wenn jemand
seinem Nächsten einen Esel oder Ochsen oder Schaf oder irgend ein Vieh
zu behalten tut, und stirbt ihm, oder wird beschädiget, oder wird ihm
weggetrieben, daß es niemand siehet, \bibverse{11} so soll man's unter
ihnen auf einen Eid bei dem HErrn kommen lassen, ob er nicht habe seine
Hand an seines Nächsten Habe gelegt; und des Guts Herr soll's annehmen,
daß jener nicht bezahlen müsse. \bibverse{12} Stiehlt es ihm aber ein
Dieb, so soll er's seinem Herrn bezahlen. \bibverse{13} Wird es aber
zerrissen, so soll er Zeugnis davon bringen und nicht bezahlen.
\bibverse{14} Wenn es jemand von seinem Nächsten entlehnet, und wird
beschädiget oder stirbt, daß sein Herr nicht dabei ist, so soll er's
bezahlen. \bibverse{15} Ist aber sein Herr dabei, so soll er's nicht
bezahlen, so er's um sein Geld gedinget hat. \bibverse{16} Wenn jemand
eine Jungfrau beredet, die noch nicht vertrauet ist, und beschläft sie,
der soll ihr geben ihre Morgengabe und sie zum Weibe haben.
\bibverse{17} Weigert sich aber ihr Vater, sie ihm zu geben, so soll er
Geld darwägen, wieviel einer Jungfrau zur Morgengabe gebührt.
\bibverse{18} Die Zauberinnen sollst du nicht leben lassen.
\bibverse{19} Wer ein Vieh beschläft, der soll des Todes sterben.
\bibverse{20} Wer den Göttern opfert, ohne dem HErrn allein, der sei
verbannet, \bibverse{21} Die Fremdlinge sollst du nicht schinden noch
unterdrücken; denn ihr seid auch Fremdlinge in Ägyptenland gewesen.
\bibverse{22} Ihr sollt keine Witwen und Waisen beleidigen.
\bibverse{23} Wirst du sie beleidigen, so werden sie zu mir schreien,
und ich werde ihr Schreien erhören; \bibverse{24} so wird mein Zorn
ergrimmen, daß ich euch mit dem Schwert töte und eure Weiber Witwen und
eure Kinder Waisen werden. \bibverse{25} Wenn du Geld leihest meinem
Volk, das arm ist bei dir, sollst du ihn nicht zu Schaden dringen und
keinen Wucher auf ihn treiben. \bibverse{26} Wenn du von deinem Nächsten
ein Kleid zum Pfande nimmst, sollst du es ihm wiedergeben, ehe die Sonne
untergehet. \bibverse{27} Denn sein Kleid ist seine einige Decke seiner
Haut, darin er schläft. Wird er aber zu mir schreien, so werde ich, ihn
erhören; denn ich bin gnädig. \bibverse{28} Den Göttern sollst du nicht
fluchen und den Obersten in deinem Volk sollst du nicht lästern.
\bibverse{29} Deine Fülle und Tränen sollst du nicht verziehen. Deinen
ersten Sohn sollst du mir geben. \bibverse{30} So sollst du auch tun mit
deinem Ochsen und Schaf. Sieben Tage laß es bei seiner Mutter sein, am
achten Tage sollst du mir's geben. \bibverse{31} Ihr sollt heilige Leute
vor mir sein: darum sollt ihr kein Fleisch essen, das auf dem Felde von
Tieren zerrissen ist, sondern vor die Hunde werfen.

\hypertarget{section-22}{%
\section{23}\label{section-22}}

\bibverse{1} Du sollst falscher Anklage nicht glauben, daß du einem
Gottlosen Beistand tust und ein falscher Zeuge seiest. \bibverse{2} Du
sollst nicht folgen der Menge zum Bösen und nicht antworten vor Gericht,
daß du der Menge nach vom Rechten weichest. \bibverse{3} Du sollst den
Geringen nicht schmücken in seiner Sache. \bibverse{4} Wenn du deines
Feindes Ochsen oder Esel begegnest, daß er irret, so sollst du ihm
denselben wieder zuführen. \bibverse{5} Wenn du des, der dich hasset,
Esel siehest unter seiner Last liegen, hüte dich und laß ihn nicht,
sondern versäume gerne das Deine, um seinetwillen. \bibverse{6} Du
sollst das Recht deines Armen nicht beugen in seiner Sache. \bibverse{7}
Sei ferne von falschen Sachen. Den Unschuldigen und Gerechten sollst du
nicht erwürgen; denn ich lasse den Gottlosen nicht recht haben.
\bibverse{8} Du sollst nicht Geschenke nehmen; denn Geschenke machen die
Sehenden blind und verkehren die Sachen der Gerechten. \bibverse{9} Die
Fremdlinge sollt ihr nicht unter, drücken; denn ihr wisset um der
Fremdlinge Herz, dieweil ihr auch seid Fremdlinge in Ägyptenland
gewesen. \bibverse{10} Sechs Jahre sollst du dein Land besäen und seine
Früchte einsammeln. \bibverse{11} Im siebenten Jahr sollst du es ruhen
und liegen lassen, daß die Armen unter deinem Volk davon essen; und was
über bleibet, laß das Wild auf dem Felde essen. Also sollst du auch tun
mit deinem Weinberge und Ölberge. \bibverse{12} Sechs Tage sollst du
deine Arbeit tun, aber des siebenten Tages sollst du feiern, auf daß
dein Ochse und Esel ruhen und deiner Magd Sohn und Fremdling sich
erquicken. \bibverse{13} Alles, was ich euch gesagt habe, das haltet.
Und anderer Götter Namen sollt ihr nicht gedenken, und aus eurem Munde
sollen sie nicht gehöret werden. \bibverse{14} Dreimal sollt ihr mir
Fest halten im Jahr. \bibverse{15} Nämlich das Fest der ungesäuerten
Brote sollst du halten, daß du sieben Tage ungesäuert Brot essest (wie
ich dir geboten habe) um die Zeit des Monden Abib; denn in demselbigen
bist du aus Ägypten gezogen. Erscheinet aber nicht leer vor mir!
\bibverse{16} Und das Fest der ersten Ernte der Früchte, die du auf dem
Felde gesäet hast. Und das Fest der Einsammlung im Ausgang des Jahrs,
wenn du deine Arbeit eingesammelt hast vom Felde. \bibverse{17} Dreimal
im Jahr sollen erscheinen vor dem HErrn, dem Herrscher, alle deine
Mannsbilde. \bibverse{18} Du sollst das Blut meines Opfers nicht neben
dem Sauerteig opfern, und das Fette von meinem Fest soll nicht bleiben
bis auf morgen. \bibverse{19} Das Erstling von der ersten Frucht auf
deinem Felde sollst du bringen in das Haus des HErrn, deines GOttes. Und
sollst das Böcklein nicht kochen, dieweil es an seiner Mutter Milch ist.
\bibverse{20} Siehe, ich sende einen Engel vor dir her, der dich behüte
auf dem Wege und bringe dich an den Ort, den ich bereitet habe.
\bibverse{21} Darum hüte dich vor seinem Angesicht und gehorche seiner
Stimme und erbittere ihn nicht; denn er wird euer Übertreten nicht
vergeben, und mein Name ist in ihm. \bibverse{22} Wirst du aber seine
Stimme hören und tun alles, was ich dir sagen werde, so will ich deiner
Feinde Feind und deiner Widerwärtigen Widerwärtiger sein. \bibverse{23}
Wenn nun mein Engel vor dir hergehet und dich bringet an die Amoriter,
Hethiter, Pheresiter, Kanaaniter, Heviter und Jebusiter, und ich sie
vertilge, \bibverse{24} so sollst du ihre Götter nicht anbeten noch
ihnen dienen und nicht tun, wie sie tun, sondern du sollst ihre Götzen
umreißen und zerbrechen. \bibverse{25} Aber dem HErrn, eurem GOtt, sollt
ihr dienen, so wird er dein Brot und dein Wasser segnen, und ich will
alle Krankheit von dir wenden. \bibverse{26} Und soll nichts
Unträchtiges noch Unfruchtbares sein in deinem Lande, und will dich
lassen alt werden. \bibverse{27} Ich will mein Schrecken vor dir
hersenden und alles Volk verzagt machen, dahin du kommst; und will dir
geben alle deine Feinde in die Flucht. \bibverse{28} Ich will Hornissen
vor dir hersenden, die vor dir her ausjagen die Heviter, Kanaaniter und
Hethiter. \bibverse{29} Ich will sie nicht auf ein Jahr ausstoßen vor
dir, auf daß nicht das Land wüste werde, und sich wilde Tiere wider dich
mehren. \bibverse{30} Einzeln nacheinander will ich sie vor dir her
ausstoßen, bis daß du wachsest und das Land besitzest. \bibverse{31} Und
will deine Grenze setzen das Schilfmeer und das Philistermeer und die
Wüste bis an das Wasser. Denn ich will dir in deine Hand geben die
Einwohner des Landes, daß du sie sollst ausstoßen vor dir her.
\bibverse{32} Du sollst mit ihnen oder mit ihren Göttern keinen Bund
machen, \bibverse{33} sondern laß sie nicht wohnen in deinem Lande, daß
sie dich nicht verführen wider mich. Denn wo du ihren Göttern dienest,
wird dir's zum Ärgernis geraten.

\hypertarget{section-23}{%
\section{24}\label{section-23}}

\bibverse{1} Und zu Mose sprach er: Steig herauf zum HErrn, du und
Aaron, Nadab und Abihu, und die siebenzig Ältesten Israels, und betet an
von ferne. \bibverse{2} Aber Mose alleine nahe sich zum HErrn, und laß
jene sich nicht herzunahen; und das Volk komme auch nicht mit ihm
herauf. \bibverse{3} Mose kam und erzählete dem Volk alle Worte des
HErrn und alle Rechte. Da antwortete alles Volk mit einer Stimme und
sprachen: Alle Worte, die der HErr gesagt hat, wollen wir tun.
\bibverse{4} Da schrieb Mose alle Worte des HErrn und machte sich des
Morgens frühe auf und bauete einen Altar unten am Berge mit zwölf Säulen
nach den zwölf Stämmen Israels. \bibverse{5} Und sandte hin Jünglinge
aus den Kindern Israel, daß sie Brandopfer darauf opferten und Dankopfer
dem HErrn von Farren. \bibverse{6} Und Mose nahm die Hälfte des Bluts
und tat's in ein Becken; die andere Hälfte sprengete er auf den Altar.
\bibverse{7} Und nahm das Buch des Bundes und las es vor den Ohren des
Volks. Und da sie sprachen: Alles, was der HErr gesagt hat, wollen wir
tun und gehorchen, \bibverse{8} da nahm Mose das Blut und sprengete das
Volk damit und sprach: Sehet, das ist Blut des Bundes, den der HErr mit
euch machte über allen diesen Worten. \bibverse{9} Da stiegen Mose und
Aaron, Nadab und Abihu und die siebenzig Ältesten Israels hinauf
\bibverse{10} und sahen den GOtt Israels. Unter seinen Füßen war es wie
ein schöner Saphir und wie die Gestalt des Himmels, wenn es klar ist.
\bibverse{11} Und er ließ seine Hand nicht über dieselben Obersten in
Israel. Und da sie GOtt geschaute hatten, aßen und tranken sie.
\bibverse{12} Und der HErr sprach zu Mose: Komm herauf zu mir auf den
Berg und bleibe daselbst, daß ich dir gebe steinerne Tafeln und Gesetze
und Gebote, die ich geschrieben habe, die du sie lehren sollst.
\bibverse{13} Da machte sich Mose auf und sein Diener Josua und stieg
auf den Berg GOttes. \bibverse{14} Und sprach zu den Ältesten: Bleibet
hie, bis wir wieder zu euch kommen. Siehe, Aaron und Hur sind bei euch;
hat jemand eine Sache, der komme vor dieselben. \bibverse{15} Da nun
Mose auf den Berg kam, bedeckte eine Wolke den Berg. \bibverse{16} Und
die Herrlichkeit des HErrn wohnete auf dem Berge Sinai und deckte ihn
mit der Wolke sechs Tage; und rief Mose am siebenten Tage aus der Wolke.
\bibverse{17} Und das Ansehen der Herrlichkeit des HErrn war wie ein
verzehrend Feuer auf der Spitze des Berges vor den Kindern Israel.
\bibverse{18} Und Mose ging mitten in die Wolke und stieg auf den Berg;
und blieb auf dem Berge vierzig Tage und vierzig Nächte.

\hypertarget{section-24}{%
\section{25}\label{section-24}}

\bibverse{1} Und der HErr redete mit Mose und sprach: \bibverse{2} Sage
den Kindern Israel, daß sie mir ein Hebopfer geben; und nehmet dasselbe
von jedermann, der es williglich gibt. \bibverse{3} Das ist aber das
Hebopfer, das ihr von ihnen nehmen sollt: Gold, Silber, Erz,
\bibverse{4} gelbe Seide, Scharlaken, Rosinrot, weiße Seide, Ziegenhaar,
\bibverse{5} rötliche Widderfelle, Dachsfelle, Föhrenholz, \bibverse{6}
Öl zur Lampe, Spezerei zur Salbe und gutem Räuchwerk, \bibverse{7}
Onyxsteine und eingefaßte Steine zum Leibrock und zum Schildlein.
\bibverse{8} Und sie sollen mir ein Heiligtum machen, daß ich unter
ihnen wohne. \bibverse{9} Wie ich dir ein Vorbild der Wohnung und alles
seines Geräts zeigen werde, so sollt ihr's machen. \bibverse{10} Machet
eine Lade von Föhrenholz. Dritthalb Ellen soll die Länge sein,
anderthalb Ellen die Breite und anderthalb Ellen die Höhe. \bibverse{11}
Und sollst sie mit feinem Golde überziehen, inwendig und auswendig; und
mache einen güldenen Kranz oben umher. \bibverse{12} Und geuß vier
güldene Rinken und mache sie an ihre vier Ecken, also daß zween Rinken
seien auf einer Seite und zween auf der andern Seite. \bibverse{13} Und
mache Stangen von Föhrenholz und überzeuch sie mit Golde. \bibverse{14}
Und stecke sie in die Rinken an der Lade Seiten, daß man sie dabei
trage; \bibverse{15} und sollen in den Rinken bleiben und nicht
herausgetan werden. \bibverse{16} Und sollt in die Lade das Zeugnis
legen, das ich dir geben werde. \bibverse{17} Du sollst auch einen
Gnadenstuhl machen von feinem Golde; dritthalb Ellen soll seine Länge
sein und anderthalb Ellen seine Breite. \bibverse{18} Und sollst zween
Cherubim machen von dichtem Golde, zu beiden Enden des Gnadenstuhls,
\bibverse{19} daß ein Cherub sei an diesem Ende, der andere an dem
andern Ende, und also zween Cherubim seien an des Gnadenstuhls Enden.
\bibverse{20} Und die Cherubim sollen Flügel ausbreiten, oben überher,
daß sie mit ihren Flügeln den Gnadenstuhl bedecken, und eines jeglichen
Antlitz gegen dem andern stehe; und ihre Antlitze sollen auf den
Gnadenstuhl sehen. \bibverse{21} Und sollt den Gnadenstuhl oben auf die
Lade tun und in die Lade das Zeugnis legen, das ich dir geben werde.
\bibverse{22} Von dem Ort will ich dir zeugen und mit dir reden, nämlich
von dem Gnadenstuhl zwischen den zween Cherubim, der auf der Lade des
Zeugnisses ist, alles, was ich dir gebieten will an die Kinder Israel.
\bibverse{23} Du sollst auch einen Tisch machen von Föhrenholz; zwo
Ellen soll seine Länge sein und eine Elle seine Breite und anderthalb
Ellen seine Höhe. \bibverse{24} Und sollst ihn überziehen mit feinem
Golde und einen güldenen Kranz umher machen \bibverse{25} und eine
Leiste umher, einer Hand breit hoch, und einen güldenen Kranz um die
Leiste her. \bibverse{26} Und sollst vier güldene Ringe dran machen an
die vier Orte an seinen vier Füßen. \bibverse{27} Hart unter der Leiste
sollen die Ringe sein, daß man Stangen drein tue und den Tisch trage;
\bibverse{28} Und sollst die Stangen von Föhrenholz machen und sie mit
Golde überziehen, daß der Tisch damit getragen werde. \bibverse{29} Du
sollst auch seine Schüsseln, Becher, Kannen, Schalen aus feinem Golde
machen, damit man aus- und einschenke. \bibverse{30} Und sollst auf den
Tisch allezeit Schaubrote legen vor mir. \bibverse{31} Du sollst auch
einen Leuchter von feinem dichten Golde machen; daran soll der Schaft
mit Röhren, Schalen, Knäufen und Blumen sein. \bibverse{32} Sechs Röhren
sollen aus dem Leuchter zu den Seiten ausgehen, aus jeglicher Seite drei
Röhren. \bibverse{33} Eine jegliche Röhre soll drei offene Schalen,
Knäufe und Blumen haben; das sollen sein die sechs Röhren aus dem
Leuchter. \bibverse{34} Aber der Schaft am Leuchter soll vier offene
Schalen mit Knäufen und Blumen haben \bibverse{35} und je einen Knauf
unter zwo Röhren, welcher sechs aus dem Leuchter gehen. \bibverse{36}
Denn beide ihre Knäufe und Röhren sollen aus ihm gehen, alles ein dicht
lauter Gold. \bibverse{37} Und sollst sieben Lampen machen obenauf, daß
sie gegeneinander leuchten, \bibverse{38} und Lichtschneuzen und
Löschnäpfe von feinem Golde. \bibverse{39} Aus einem Zentner feinen
Goldes sollst du das machen mit allem diesem Geräte. \bibverse{40} Und
siehe zu, daß du es machest nach ihrem Bilde, das du auf dem Berge
gesehen hast.

\hypertarget{section-25}{%
\section{26}\label{section-25}}

\bibverse{1} Die Wohnung sollst du machen von zehn Teppichen, von weißer
gezwirnter Seide, von gelber Seide, von Scharlaken und Rosinrot.
Cherubim sollst du dran machen künstlich. \bibverse{2} Die Länge eines
Teppichs soll achtundzwanzig Ellen sein, die Breite vier Ellen; und
sollen alle zehn gleich sein. \bibverse{3} Und sollen je fünf
zusammengefüget sein; einer an den andern. \bibverse{4} Und sollst
Schläuflein machen von gelber Seide an jeglichen Teppichs Orten, da sie
sollen zusammengefüget sein, daß je zween und zween an ihren Orten
zusammengeheftet werden, \bibverse{5} fünfzig Schläuflein an jeglichem
Teppich, daß einer den andern zusammenfasse. \bibverse{6} Und sollst
fünfzig güldene Hefte machen, damit man die Teppiche zusammenhefte,
einen an den andern, auf daß es eine Wohnung werde. \bibverse{7} Du
sollst auch eine Decke aus Ziegenhaar machen zur Hütte über die Wohnung
von elf Teppichen. \bibverse{8} Die Länge eines Teppichs soll dreißig
Ellen sein, die Breite aber vier Ellen; und sollen alle elf gleich groß
sein. \bibverse{9} Fünf sollst du aneinanderfügen und sechs auch
aneinander, daß du den sechsten Teppich zwiefältig machest vorne an der
Hütte. \bibverse{10} Und sollst an einem jeglichen Teppich fünfzig
Schläuflein machen, an ihren Orten, daß sie aneinander bei den Enden
gefüget werden. \bibverse{11} Und sollst fünfzig eherne Hefte machen und
die Hefte in die Schläuflein tun, daß die Hütte zusammengefüget und eine
Hütte werde. \bibverse{12} Aber das Überlänge an den Teppichen der Hütte
sollst du die Hälfte lassen überhangen an der Hütte, \bibverse{13} auf
beiden Seiten eine Elle lang, daß das übrige sei an der Hütte Seiten und
auf beiden Seiten sie bedecke. \bibverse{14} Über diese Decke sollst du
eine Decke machen von rötlichen Widderfellen, dazu über sie eine Decke
von Dachsfellen. \bibverse{15} Du sollst auch Bretter machen zu der
Wohnung von Föhrenholz, die stehen sollen. \bibverse{16} Zehn Ellen lang
soll ein Brett sein und anderthalb Ellen breit. \bibverse{17} Zween
Zapfen soll ein Brett haben, daß eins an das andere möge gesetzt werden.
Also sollst du alle Bretter der Wohnung machen. \bibverse{18} Zwanzig
sollen ihrer stehen gegen dem Mittag. \bibverse{19} Die sollen vierzig
silberne Füße unten haben, je zween Füße unter jeglichem Brett an seinen
zween Zapfen. \bibverse{20} Also auf der andern Seite, gegen
Mitternacht, sollen auch zwanzig Bretter stehen \bibverse{21} und
vierzig silberne Füße, je zween Füße unter jeglichem Brett.
\bibverse{22} Aber hinten an der Wohnung, gegen dem Abend, sollst du
sechs Bretter machen. \bibverse{23} Dazu zwei Bretter hinten an die zwo
Ecken der Wohnung, \bibverse{24} daß ein jegliches der beiden sich mit
seinem Ortbrett von unten auf geselle und oben am Haupt gleich
zusammenkomme mit einer Klammer, \bibverse{25} daß acht Bretter seien
mit ihren silbernen Füßen; deren sollen sechzehn sein, je zween unter
einem Brett. \bibverse{26} Und, sollst Riegel machen von Föhrenholz,
fünf zu den Brettern auf einer Seite der Wohnung \bibverse{27} und fünf
zu den Brettern auf der andern Seite der Wohnung und fünf zu den
Brettern hinten an der Wohnung gegen dem Abend. \bibverse{28} Und sollst
die Riegel mitten an den Brettern durchhinstoßen und alles
zusammenfassen von einem Ort zu dem andern. \bibverse{29} Und sollst die
Bretter mit Golde überziehen und ihre Rinken von Golde machen, daß man
die Riegel drein tue. \bibverse{30} Und die Riegel sollst du mit Gold
überziehen. Und also sollst du denn die Wohnung aufrichten nach der
Weise, wie du gesehen hast auf dem Berge. \bibverse{31} Und sollst einen
Vorhang machen von gelber Seide, Scharlaken und Rosinrot und gezwirnter
weißer Seide; und sollst Cherubim dran machen künstlich. \bibverse{32}
Und sollst ihn hängen an vier Säulen von Föhrenholz, die mit Gold
überzogen sind und güldene Knäufe und vier silberne Füße haben.
\bibverse{33} Und sollst den Vorhang mit Heften anheften und die Lade
des Zeugnisses inwendig des Vorhangs setzen, daß er euch ein Unterschied
sei zwischen dem Heiligen und dem Allerheiligsten. \bibverse{34} Und
sollst den Gnadenstuhl tun auf die Lade des Zeugnisses in dem
Allerheiligsten. \bibverse{35} Den Tisch aber setze außer dem Vorhange
und den Leuchter gegen dem Tisch über, zu mittagwärts der Wohnung, daß
der Tisch stehe gegen Mitternacht. \bibverse{36} Und sollst ein Tuch
machen in die Tür der Hütte, gewirkt von gelber Seide, Rosinrot,
Scharlaken und gezwirnter weißer Seide. \bibverse{37} Und sollst
demselben Tuch fünf Säulen machen von Föhrenholz, mit Gold überzogen,
mit güldenen Knäufen, und sollst ihnen fünf eherne Füße gießen.

\hypertarget{section-26}{%
\section{27}\label{section-26}}

\bibverse{1} Und sollst einen Altar machen von Föhrenholz, fünf Ellen
lang und breit, daß er gleich viereckig sei, und drei Ellen hoch.
\bibverse{2} Hörner sollst du auf seine vier Ecken machen, und sollst
ihn mit Erz überziehen. \bibverse{3} Mache auch Aschentöpfe, Schaufeln,
Becken, Kreuel, Kohlpfannen; alle seine Geräte sollst du von Erz machen.
\bibverse{4} Du sollst auch ein ehern Gitter machen wie ein Netz und
vier eherne Ringe an seine vier Orte. \bibverse{5} Du sollst es aber von
unten auf um den Altar machen, daß das Gitter reiche bis mitten an den
Altar. \bibverse{6} Und sollst auch Stangen machen zu dem Altar von
Föhrenholz, mit Erz überzogen. \bibverse{7} Und sollst die Stangen in
die Ringe tun, daß die Stangen seien an beiden Seiten des Altars, damit
man ihn tragen möge. \bibverse{8} Und sollst ihn also von Brettern
machen, daß er inwendig hohl sei, wie dir auf dem Berge gezeiget ist.
\bibverse{9} Du sollst auch der Wohnung einen Hof machen, einen Umhang
von gezwirnter weißer Seide, auf einer Seite hundert Ellen lang, gegen
dem Mittag, \bibverse{10} Und zwanzig Säulen auf zwanzig ehernen Füßen,
und ihre Knäufe mit ihren Reifen von Silber. \bibverse{11} Also auch
gegen Mitternacht soll sein ein Umhang, hundert Ellen lang; zwanzig
Säulen auf zwanzig ehernen Füßen, und ihre Knäufe mit ihren Reifen von
Silber. \bibverse{12} Aber gegen dem Abend soll die Breite des Hofes
haben einen Umhang fünfzig Ellen lang, zehn Säulen auf zehn Füßen.
\bibverse{13} Gegen dem Morgen aber soll die Breite des Hofes haben
fünfzig Ellen, \bibverse{14} also daß der Umhang habe auf einer Seite
fünfzehn Ellen, dazu drei Säulen auf dreien Füßen, \bibverse{15} und
aber fünfzehn Ellen auf der andern Seite, dazu drei Säulen auf dreien
Füßen. \bibverse{16} Aber in dem Tor des Hofes soll ein Tuch sein,
zwanzig Ellen breit, gewirket von gelber Seide, Scharlaken, Rosinrot und
gezwirnter weißer Seide, dazu vier Säulen auf ihren vier Füßen.
\bibverse{17} Alle Säulen um den Hof her sollen silberne Reife und
silberne Knäufe und eherne Füße haben. \bibverse{18} Und die Länge des
Hofes soll hundert Ellen sein, die Breite fünfzig Ellen, die Höhe fünf
Ellen, von gezwirnter weißer Seide; und seine Füße sollen ehern sein.
\bibverse{19} Auch alle Geräte der Wohnung zu allerlei Amt und alle
seine Nägel und alle Nägel des Hofes sollen ehern sein. \bibverse{20}
Gebeut den Kindern Israel, daß sie zu dir bringen das allerreinste
lautere Öl von Ölbäumen gestoßen, zur Leuchte, das man allezeit oben in
die Lampen tue \bibverse{21} in der Hütte des Stifts außer dem Vorhang,
der vor dem Zeugnis hanget. Und Aaron und seine Söhne sollen sie
zurichten, beide des Morgens und des Abends, vor dem HErrn. Das soll
euch eine ewige Weise sein auf eure Nachkommen unter den Kindern Israel.

\hypertarget{section-27}{%
\section{28}\label{section-27}}

\bibverse{1} Und sollst Aaron, deinen Bruder, und seine Söhne zu dir
nehmen aus den Kindern Israel, daß er mein Priester sei, nämlich Aaron
und seine Söhne, Nadab, Abihu, Eleasar und Ithamar. \bibverse{2} Und
sollst Aaron, deinem Bruder heilige Kleider machen, die herrlich und
schön seien. \bibverse{3} Und sollst reden mit allen, die eines weisen
Herzens sind, die ich mit dem Geist der Weisheit erfüllet habe, Aaron
Kleider machen zu seiner Weihe, daß er mein Priester sei. \bibverse{4}
Das sind aber die Kleider, die sie machen sollen: das Schildlein,
Leibrock, Seidenrock, engen Rock, Hut und Gürtel. Also sollen sie
heilige Kleider machen deinem Bruder Aaron und seinen Söhnen, daß er
mein Priester sei. \bibverse{5} Dazu sollen sie nehmen Gold, gelbe
Seide, Scharlaken, Rosinrot und weiße Seide. \bibverse{6} Den Leibrock
sollen sie machen von Gold, gelber Seide, Scharlaken, Rosinrot und
gezwirnter weißer Seide, künstlich, \bibverse{7} daß er auf beiden
Achseln zusammengefüget und an beiden Seiten zusammengebunden werde.
\bibverse{8} Und sein Gurt drauf soll derselben Kunst und Werks sein,
von Gold, gelber Seide, Scharlaken, Rosinrot und gezwirnter weißer
Seide. \bibverse{9} Und sollst zween Onyxsteine nehmen und drauf graben
die Namen der Kinder Israel, \bibverse{10} auf jeglichen sechs Namen,
nach der Ordnung ihres Alters. \bibverse{11} Das sollst du tun durch die
Steinschneider, die da Siegel graben, also daß sie mit Gold umher
gefasset werden. \bibverse{12} Und sollst sie auf die Schultern des
Leibrocks heften, daß es Steine seien zum Gedächtnis für die Kinder
Israel, daß Aaron ihre Namen auf seinen beiden Schultern trage vor dem
HErrn zum Gedächtnis. \bibverse{13} Und sollst güldene Spangen machen
\bibverse{14} und zwo Ketten von feinem Golde mit zwei Enden, aber die
Glieder ineinander hangend, und sollst sie an die Spangen tun.
\bibverse{15} Das Amtsschildlein sollst du machen nach der Kunst wie den
Leibrock, von Gold, gelber Seide, Scharlaken, Rosinrot und gezwirnter
weißer Seide. \bibverse{16} Viereckig soll es sein und zwiefach; eine
Hand breit soll seine Länge sein und eine Hand breit seine Breite.
\bibverse{17} Und sollst es füllen mit vier Riegen voll Steine. Die
erste Riege sei ein Sarder, Topaser; Smaragd; \bibverse{18} die andere
ein Rubin, Saphir, Demant; \bibverse{19} die dritte ein Lynkurer, Achat,
Amethyst; \bibverse{20} die vierte ein Türkis, Onyx, Jaspis. In Gold
sollen sie gefasset sein in allen Riegen. \bibverse{21} Und sollen nach
den zwölf Namen der Kinder Israel stehen, gegraben vom Steinschneider,
ein jeglicher seines Namens, nach den zwölf Stämmen. \bibverse{22} Und
sollst Ketten zu dem Schildlein machen mit zwei Enden, aber die Glieder
ineinander hangend, von feinem Golde, \bibverse{23} und zween güldene
Ringe an das Schildlein, also daß du dieselben zween Ringe heftest an
zwo Ecken des Schildleins \bibverse{24} und die zwo güldenen Ketten in
die selben zween Ringe an den beiden Ecken des Schildleins tust.
\bibverse{25} Aber die zwei Enden der zwo Ketten sollst du in zwo
Spangen tun und sie heften auf die Schultern am Leibrock, gegeneinander
über. \bibverse{26} Und sollst zween andere güldene Ringe machen und an
die zwo andern Ecken des Schildleins heften an seinem Ort, inwendig
gegen dem Leibrock. \bibverse{27} Und sollst aber zween güldene Ringe
machen und an die zwo Ecken unten am Leibrock gegeneinander heften, da
der Leibrock zusammengehet, oben an dem Leibrock, künstlich.
\bibverse{28} Und man soll das Schildlein mit feinen Ringen mit einer
gelben Schnur an die Ringe des Leibrocks knüpfen, daß es auf dem
künstlich gemachten Leibrock hart anliege und das Schildlein sich nicht
von dem Leibrock losmache. \bibverse{29} Also soll Aaron die Namen der
Kinder Israel tragen in dem Amtsschildlein auf seinem Herzen, wenn er in
das Heilige gehet, zum Gedächtnis vor dem HErrn allezeit. \bibverse{30}
Und sollst in das Amtsschildlein tun Licht und Recht, daß sie auf dem
Herzen Aarons seien, wenn er eingehet vor den HErrn, und trage das Amt
der Kinder Israel auf seinem Herzen vor dem HErrn allewege.
\bibverse{31} Du sollst auch den Seidenrock unter den Leibrock machen,
ganz von gelber Seide. \bibverse{32} Und oben mitten inne soll ein Loch
sein und eine Borte um das Loch her zusammengefaltet, daß es nicht
zerreiße. \bibverse{33} Und unten an seinem Saum sollst du Granatäpfel
machen von gelber Seide, Scharlaken, Rosinrot um und um, und zwischen
dieselben güldene Schellen, auch um und um, \bibverse{34} daß eine
güldene Schelle sei, danach ein Granatapfel und aber eine güldene
Schelle und wieder ein Granatapfel um und um an dem Saum desselben
Seidenrocks. \bibverse{35} Und Aaron soll ihn anhaben wenn er dienet,
daß man seinen Klang höre, wenn er aus und ein gehet in das Heilige vor
dem HErrn, auf daß er nicht sterbe. \bibverse{36} Du sollst auch ein
Stirnblatt machen von feinem Golde und ausgraben, wie man die Siegel
ausgräbt: Die Heiligkeit des HErrn. \bibverse{37} Und sollst es heften
an eine gelbe Schnur vorne an den Hut, \bibverse{38} auf der Stirn
Aarons, daß also Aaron trage die Missetat des Heiligen, das die Kinder
Israel heiligen in allen Gaben ihrer Heiligung; und es soll allewege an
seiner Stirn sein, daß er sie versöhne vor dem HErrn. \bibverse{39} Du
sollst auch den engen Rock machen von weißer Seide und einen Hut von
weißer Seide machen und einen gestickten Gürtel. \bibverse{40} Und den
Söhnen Aarons sollst du Röcke, Gürtel und Hauben machen, die herrlich
und schön seien. \bibverse{41} Und sollst sie deinem Bruder Aaron samt
seinen Söhnen anziehen und sollst sie salben und ihre Hände füllen und
sie weihen, daß sie meine Priester seien. \bibverse{42} Und sollst ihnen
leinene Niederkleider machen, zu bedecken das Fleisch der Scham, von den
Lenden bis an die Hüften. \bibverse{43} Und Aaron und seine Söhne sollen
sie anhaben, wenn sie in die Hütte des Stifts gehen oder hinzutreten zum
Altar, daß sie dienen in dem Heiligtum, daß sie nicht ihre Missetat
tragen und sterben müssen. Das soll ihm und seinem Samen nach ihm eine
ewige Weise sein.

\hypertarget{section-28}{%
\section{29}\label{section-28}}

\bibverse{1} Das ist's auch, das du ihnen tun sollst, daß sie mir zu
Priestern geweihet werden: Nimm einen jungen Farren und zween Widder
ohne Wandel, \bibverse{2} ungesäuert Brot und ungesäuerte Kuchen, mit Öl
gemenget, und ungesäuerte Fladen, mit Öl gesalbet. Von Weizenmehl sollst
du solches alles machen; \bibverse{3} und sollst es in einen Korb legen
und in dem Korbe herzubringen samt dem Farren und den zween Widdern.
\bibverse{4} Und sollst Aaron und seine Söhne vor die Tür der Hütte des
Stifts führen und mit Wasser waschen; \bibverse{5} und die Kleider
nehmen und Aaron anziehen den engen Rock und den Seidenrock und den
Leibrock und das Schildlein zu dem Leibrock; und sollst ihn gürten außen
auf den Leibrock \bibverse{6} und den Hut auf sein Haupt setzen und die
heilige Krone an den Hut. \bibverse{7} Und sollst nehmen das Salböl und
auf sein Haupt schütten und ihn salben. \bibverse{8} Und seine Söhne
sollst du auch herzu führen und den engen Rock ihnen anziehen;
\bibverse{9} und beide Aaron und auch sie mit Gürteln gürten und ihnen
die Hauben aufbinden, daß sie das Priestertum haben zu ewiger Weise. Und
sollst Aaron und seinen Söhnen die Hände füllen \bibverse{10} und den
Farren herzuführen vor die Hütte des Stifts; und Aaron samt seinen
Söhnen sollen ihre Hände auf des Farren Haupt legen. \bibverse{11} Und
sollst den Farren schlachten vor dem HErrn, vor der Tür der Hütte des
Stifts. \bibverse{12} Und sollst seines Bluts nehmen und auf des Altars
Hörner tun mit deinem Finger und alles andere Blut an des Altars Boden
schütten. \bibverse{13} Und sollst alles Fett nehmen am Eingeweide und
das Netz über der Leber und die zwo Nieren mit dem Fett, das drüber
liegt, und sollst es auf dem Altar anzünden. \bibverse{14} Aber des
Farren Fleisch, Fell und Mist sollst du außen vor dem Lager mit Feuer
verbrennen, denn es ist ein Sündopfer. \bibverse{15} Aber den einen
Widder sollst du nehmen, und Aaron samt seinen Söhnen sollen ihre Hände
auf sein Haupt legen. \bibverse{16} Dann sollst du ihn schlachten und
seines Bluts nehmen und auf den Altar sprengen ringsherum. \bibverse{17}
Aber den Widder sollst du zerlegen in Stücke und sein Eingeweide waschen
und Schenkel; und sollst es auf seine Stücke und Haupt legen
\bibverse{18} und den ganzen Widder anzünden auf dem Altar; denn es ist
dem HErrn ein Brandopfer, ein süßer Geruch, ein Feuer des HErrn.
\bibverse{19} Den andern Widder aber sollst du nehmen und Aaron samt
seinen Söhnen sollen ihre Hände auf sein Haupt legen. \bibverse{20} Und
sollst ihn schlachten und seines Bluts nehmen und Aaron und seinen
Söhnen auf den rechten Ohrknorpel tun und auf den Daumen ihrer rechten
Hand und auf den großen Zehen ihres rechten Fußes; und sollst das Blut
auf den Altar sprengen ringsherum. \bibverse{21} Und sollst das Blut auf
dem Altar nehmen und Salböl und Aaron und seine Kleider, seine Söhne und
ihre Kleider besprengen, so wird er und seine Kleider, seine Söhne und
ihre Kleider geweihet. \bibverse{22} Danach sollst du nehmen das Fett
von dem Widder, den Schwanz und das Fett am Eingeweide, das Netz über
der Leber und die zwo Nieren mit dem Fett drüber und die rechte Schulter
(denn es ist ein Widder der Fülle) \bibverse{23} und ein Brot und einen
Ölkuchen und einen Fladen aus dem Korbe des ungesäuerten Brots, der vor
dem HErrn stehet. \bibverse{24} Und lege es alles auf die Hände Aarons
und seiner Söhne und webe es dem HErrn. \bibverse{25} Danach nimm's von
ihren Händen und zünde es an auf dem Altar zum Brandopfer, zum süßen
Geruch vor dem HErrn; denn das ist ein Feuer des HErrn. \bibverse{26}
Und sollst die Brust nehmen vom Widder der Fülle Aarons und sollst es
vor dem HErrn weben. Das soll dein Teil sein. \bibverse{27} Und sollst
also heiligen die Webebrust und die Hebeschulter, die gewebet und
gehebet sind von dem Widder der Fülle Aarons und seiner Söhne.
\bibverse{28} Und soll Aarons und seiner Söhne sein ewiger Weise von den
Kindern Israel; denn es ist ein Hebopfer. Und die Hebopfer sollen des
HErrn sein von den Kindern Israel an ihren Dankopfern und Hebopfern.
\bibverse{29} Aber die heiligen Kleider Aarons sollen seine Söhne haben
nach ihm, daß sie darinnen gesalbet und ihre Hände gefüllet werden.
\bibverse{30} Welcher unter seinen Söhnen an seiner Statt Priester wird,
der soll sie sieben Tage anziehen, daß er gehe in die Hütte des Stifts,
zu dienen im Heiligen. \bibverse{31} Du sollst aber nehmen den Widder
der Füllung und sein Fleisch an einem heiligen Ort kochen. \bibverse{32}
Und Aaron mit seinen Söhnen soll desselben Widders Fleisch essen samt
dem Brot im Korbe vor der Tür der Hütte des Stifts. \bibverse{33} Denn
es ist Versöhnung damit geschehen, zu füllen ihre Hände, daß sie
geweihet werden. Kein anderer soll es essen, denn es ist heilig.
\bibverse{34} Wo aber etwas überbleibet von dem Fleisch der Füllung und
von dem Brot bis an den Morgen, das sollst du mit Feuer verbrennen und
nicht essen lassen; denn es ist heilig. \bibverse{35} Und sollst also
mit Aaron und seinen Söhnen tun alles, was ich dir geboten habe. Sieben
Tage sollst du ihre Hände füllen \bibverse{36} und täglich einen Farren
zum Sündopfer schlachten zur Versöhnung. Und sollst den Altar
entsündigen, wenn du ihn versöhnest, und sollst ihn salben, daß er
geweihet werde. \bibverse{37} Sieben Tage sollst du den Altar versöhnen
und ihn weihen, daß er sei ein Altar, das Allerheiligste. Wer den Altar
anrühren will, der soll geweihet sein. \bibverse{38} Und das sollst du
mit dem Altar tun. Zwei jährige Lämmer sollst du allewege des Tages
drauf opfern, \bibverse{39} ein Lamm des Morgens, das andere zwischen
Abends. \bibverse{40} Und zu einem Lamm ein Zehnten Semmelmehls,
gemenget mit einem Vierteil von einem Hin gestoßenen Öls und einem
Vierteil vom Hin Weins zum Trankopfer. \bibverse{41} Mit dem andern Lamm
zwischen Abends sollst du tun wie mit dem Speisopfer und Trankopfer des
Morgens, zu süßem Geruch, ein Feuer dem HErrn. \bibverse{42} Das ist das
tägliche Brandopfer bei euren Nachkommen vor der Tür der Hütte des
Stifts, vor dem HErrn, da ich euch zeugen und mit dir reden will.
\bibverse{43} Daselbst will ich den Kindern Israel erkannt und
geheiliget werden in meiner Herrlichkeit. \bibverse{44} Und will die
Hütte des Stifts mit dem Altar heiligen und Aaron und seine Söhne mir zu
Priestern weihen. \bibverse{45} Und will unter den Kindern Israel wohnen
und ihr GOtt sein, \bibverse{46} daß sie wissen sollen, ich sei der
HErr, ihr GOtt, der sie aus Ägyptenland führete, daß ich unter ihnen
wohne, ich, der HErr, ihr GOtt.

\hypertarget{section-29}{%
\section{30}\label{section-29}}

\bibverse{1} Du sollst auch einen Räuchaltar machen zu räuchern, von
Föhrenholz, \bibverse{2} einer Eile lang und breit, gleich viereckig,
und zwo Ellen hoch mit seinen Hörnern. \bibverse{3} Und sollst ihn mit
feinem Golde überziehen, sein Dach und seine Wände rings umher und seine
Hörner. Und sollst einen Kranz von Gold umher machen \bibverse{4} und
zween güldene Ringe unter dem Kranz zu beiden Seiten, daß man Stangen
drein tue und ihn damit trage. \bibverse{5} Die Stangen sollst du auch
von Föhrenholz machen und mit Gold überziehen. \bibverse{6} Und sollst
ihn setzen vor den Vorhang, der vor der Lade des Zeugnisses hanget, und
vor dem Gnadenstuhl, der auf dem Zeugnis ist, von dannen ich dir werde
zeugen. \bibverse{7} Und Aaron soll drauf räuchern gut Räuchwerk alle
Morgen, wenn er die Lampen zurichtet. \bibverse{8} Desselbigengleichen,
wenn er die Lampen anzündet zwischen Abends, soll er solch Geräuch auch
räuchern. Das soll das tägliche Geräuch sein vor dem HErrn bei euren
Nachkommen. \bibverse{9} Ihr sollt kein fremd Geräuch drauf tun, auch
kein Brandopfer noch Speisopfer und kein Trankopfer drauf opfern.
\bibverse{10} Und Aaron soll auf seinen Hörnern versöhnen einmal im Jahr
mit dem Blut des Sündopfers zur Versöhnung. Solche Versöhnung soll
jährlich einmal geschehen bei euren Nachkommen; denn das ist dem HErrn
das Allerheiligste. \bibverse{11} Und der HErr redete mit Mose und
sprach: \bibverse{12} Wenn du die Häupter der Kinder Israel zählest, so
soll ein jeglicher dem HErrn geben die Versöhnung seiner Seele, auf daß
ihnen nicht eine Plage widerfahre, wenn sie gezählet werden.
\bibverse{13} Es soll aber ein, jeglicher, der mit in der Zahl ist,
einen halben Sekel geben, nach dem Sekel des Heiligtums (ein Sekel gilt
zwanzig Gera). Solcher halber Sekel soll das Hebopfer des HErrn sein.
\bibverse{14} Wer in der Zahl ist von zwanzig Jahren und drüber, der
soll solch Hebopfer dem HErrn geben. \bibverse{15} Der Reiche soll nicht
mehr geben und der Arme nicht weniger als den halben Sekel, den man dem
HErrn zur Hebe gibt für die Versöhnung ihrer Seelen. \bibverse{16} Und
du sollst solch Geld der Versöhnung nehmen von den Kindern Israel und an
den Gottesdienst der Hütte des Stifts legen, daß es sei den Kindern
Israel ein Gedächtnis vor dem HErrn, daß er sich über ihre Seelen
versöhnen lasse. \bibverse{17} Und der HErr redete mit Mose und sprach:
\bibverse{18} Du sollst auch ein ehern Handfaß machen mit einem ehernen
Fuß, zu waschen; und sollst es setzen zwischen der Hütte des Stifts und
dem Altar und Wasser drein tun, \bibverse{19} daß Aaron und seine Söhne
ihre Hände und Füße draus waschen, \bibverse{20} wenn sie in die Hütte
des Stifts gehen oder zum Altar, daß sie dienen mit Räuchern, einem
Feuer des HErrn, \bibverse{21} auf daß sie nicht sterben. Das soll eine
ewige Weise sein ihm und seinem Samen bei ihren Nachkommen.
\bibverse{22} Und der HErr redete mit Mose und sprach:. \bibverse{23}
Nimm zu dir die besten Spezereien, die edelsten Myrrhen, fünfhundert
(Sekel), und Zinnamet, die Hälfte so viel, zweihundertundfünfzig, und
Kalmus, auch zweihundertundfünfzig, \bibverse{24} und Kassia,
fünfhundert, nach dem Sekel des Heiligtums, und Öl vom Ölbaum ein Hin,
\bibverse{25} und mache ein heiliges Salböl nach der Apotheker Kunst.
\bibverse{26} Und sollst damit salben die Hütte des Stifts und die Lade
des Zeugnisses, \bibverse{27} den Tisch mit all seinem Geräte, den
Leuchter mit seinem Geräte, den Räuchaltar, \bibverse{28} den
Brandopferaltar mit all seinem Geräte und das Handfaß mit seinem Fuß.
\bibverse{29} Und sollst sie also weihen, daß sie das Allerheiligste
seien; denn wer sie anrühren will, der soll geweihet sein. \bibverse{30}
Aaron und seine Söhne sollst du auch salben und sie mir zu Priestern
weihen. \bibverse{31} Und sollst mit den Kindern Israel reden und
sprechen: Dies Öl soll mir eine heilige Salbe sein bei euren Nachkommen.
\bibverse{32} Auf Menschen Leib soll's nicht gegossen werden, sollst
auch seinesgleichen nicht machen; denn es ist heilig, darum soll's euch
heilig sein. \bibverse{33} Wer ein solches macht oder einem andern davon
gibt, der soll von seinem Volk ausgerottet werden. \bibverse{34} Und der
HErr sprach zu Mose: Nimm zu dir Spezerei, Balsam, Stakte, Galban und
reinen Weihrauch, eines so viel als des andern, \bibverse{35} und mache
Räuchwerk draus, nach Apothekerkunst gemenget, daß es rein und heilig
sei. \bibverse{36} Und sollst es zu Pulver stoßen und sollst desselben
tun vor das Zeugnis in der Hütte des Stifts, von dannen ich dir zeugen
werde. Das soll euch das Allerheiligste sein. \bibverse{37} Und
desgleichen Räucherwerk sollt ihr euch nicht machen, sondern es soll dir
heilig sein dem HErrn. \bibverse{38} Wer ein solches machen wird daß er
damit räuchere, der wird ausgerottet deswillen von seinem Volk.

\hypertarget{section-30}{%
\section{31}\label{section-30}}

\bibverse{1} Und der HErr redete mit Mose und sprach: \bibverse{2}
Siehe, ich habe mit Namen berufen Bezaleel, den Sohn Uris, des Sohns
Hurs, vom Stamm Juda, \bibverse{3} und habe ihn erfüllet mit dem Geist
GOttes, mit Weisheit und Verstand und Erkenntnis und mit allerlei Werk,
\bibverse{4} künstlich zu arbeiten am Gold, Silber, Erz, \bibverse{5}
künstlich Stein zu schneiden und einzusetzen und künstlich zu zimmern am
Holz, zu machen allerlei Werk. \bibverse{6} Und siehe, ich habe ihm
zugegeben Ahaliab, den Sohn Ahisamachs, vom Stamm Dan, und habe allerlei
Weisen die Weisheit ins Herz gegeben, daß sie machen sollen alles, was
ich dir geboten habe: \bibverse{7} die Hütte des Stifts, die Lade des
Zeugnisses, den Gnadenstuhl drauf und alle Geräte der Hütte,
\bibverse{8} den Tisch und sein Gerät, den feinen Leuchter und all sein
Gerät, den Räuchaltar, \bibverse{9} den Brandopferaltar mit all seinem
Geräte, das Handfaß mit seinem Fuße, \bibverse{10} die Amtskleider und
die heiligen Kleider des Priesters Aaron und die Kleider seiner Söhne,
priesterlich zu dienen, \bibverse{11} das Salböl und das Räuchwerk von
Spezerei zum Heiligtum. Alles, was ich dir geboten habe, werden sie
machen; \bibverse{12} Und der HErr redete mit Mose und sprach:
\bibverse{13} Sage den Kindern Israel und sprich: Haltet meinen Sabbat;
denn derselbe ist ein Zeichen zwischen mir und euch auf eure Nachkommen,
daß ihr wisset, daß ich der HErr bin, der euch heiliget. \bibverse{14}
Darum so haltet meinen Sabbat; denn er soll euch heilig sein. Wer ihn
entheiliget, der soll des Todes sterben, Denn wer eine Arbeit darinnen
tut, des Seele soll ausgerottet werden von seinem Volk. \bibverse{15}
Sechs Tage soll man arbeiten; aber am siebenten Tage ist Sabbat, die
heilige Ruhe des HErrn. Wer eine Arbeit tut am Sabbattage, soll des
Todes sterben. \bibverse{16} Darum sollen die Kinder Israel den Sabbat
halten, daß sie ihn auch bei ihren Nachkommen halten zum ewigen Bunde.
\bibverse{17} Er ist ein ewig Zeichen zwischen mir und den Kindern
Israel. Denn in sechs Tagen machte der HErr Himmel und Erde; aber am
siebenten Tage ruhete er und erquickte sich. \bibverse{18} Und da der
HErr ausgeredet hatte mit Mose auf dem Berge Sinai, gab er ihm zwo
Tafeln des Zeugnisses; die waren steinern und geschrieben mit dem Finger
GOttes.

\hypertarget{section-31}{%
\section{32}\label{section-31}}

\bibverse{1} Da aber das Volk sah, daß Mose verzog, von dem Berge zu
kommen, sammelte sich's wider Aaron und sprach zu ihm: Auf, und mach uns
Götter, die vor uns hergehen! Denn wir wissen nicht, was diesem Mann
Mose widerfahren ist, der uns aus Ägyptenland geführet hat. \bibverse{2}
Aaron sprach zu ihnen: Reißet ab die güldenen Ohrenringe an den Ohren
eurer Weiber, eurer Söhne und eurer Töchter und bringet sie zu mir.
\bibverse{3} Da riß alles Volk seine güldenen Ohrenringe von ihren Ohren
und brachten sie zu Aaron. \bibverse{4} Und er nahm sie von ihren Händen
und entwarf es mit einem Griffel und machte ein gegossen Kalb. Und sie
sprachen: Das sind deine Götter, Israel, die dich aus Ägyptenland
geführet haben! \bibverse{5} Da das Aaron sah, bauete er einen Altar vor
ihm und ließ ausrufen und sprach: Morgen ist des HErrn Fest!
\bibverse{6} Und stunden des Morgens frühe auf und opferten Brandopfer
und brachten dazu Dankopfer. Danach setzte sich das Volk zu essen und zu
trinken, und stunden auf zu spielen. \bibverse{7} Der HErr aber sprach
zu Mose: Gehe, steig hinab; denn dein Volk, das du aus Ägyptenland
geführet hast, hat's verderbet. \bibverse{8} Sie sind schnell von dem
Wege getreten, den ich ihnen geboten habe. Sie haben ihnen ein gegossen
Kalb gemacht und haben's angebetet und ihm geopfert und gesagt: Das sind
deine Götter, Israel, die dich aus Ägyptenland geführet haben.
\bibverse{9} Und den HErr sprach zu Mose: Ich sehe, daß es ein
halsstarrig Volk ist. \bibverse{10} Und nun laß mich, daß mein Zorn über
sie ergrimme und sie auffresse, so will ich dich zum großen Volk machen.
\bibverse{11} Mose aber flehete von dem HErrn, seinem GOtt, und sprach:
Ach, HErr, war um will dein Zorn ergrimmen über dein Volk, das du mit
großer Kraft und starker Hand hast aus Ägyptenland geführet?
\bibverse{12} Warum sollen die Ägypter sagen und sprechen: Er hat sie zu
ihrem Unglück ausgeführt, daß er sie erwürgete im Gebirge und vertilgete
sie von dem Erdboden? Kehre dich von dem Grimm deines Zorns und sei
gnädig über die Bosheit deines Volks! \bibverse{13} Gedenk an deine
Diener, Abraham, Isaak und Israel, denen du bei dir selbst geschworen
und ihnen verheißen hast: Ich will euren Samen mehren wie die Sterne am
Himmel, und alles Land, das ich verheißen habe, will ich eurem Samen
geben, und sollen es besitzen ewiglich. \bibverse{14} Also gereuete den
HErrn das Übel, das er dräuete seinem Volk zu tun. \bibverse{15} Mose
wandte sich und stieg vom Berge und hatte zwo Tafeln des Zeugnisses in
seiner Hand, die waren geschrieben auf beiden Seiten. \bibverse{16} Und
GOtt hatte sie selbst gemacht und selbst die Schrift drein gegraben.
\bibverse{17} Da nun Josua hörete des Volks Geschrei, daß sie
jauchzeten, sprach er zu Mose: Es ist ein Geschrei im Lager wie im
Streit. \bibverse{18} Er antwortete: Es ist nicht ein Geschrei
gegeneinander, deren, die obliegen und unterliegen, sondern ich höre ein
Geschrei eines Singetanzes. \bibverse{19} Als er aber nahe zum Lager kam
und das Kalb und den Reigen sah, ergrimmete er mit Zorn und warf die
Tafeln aus seiner Hand und zerbrach sie unten am Berge. \bibverse{20}
Und nahm das Kalb, das sie gemacht hatten, und verbrannte es mit Feuer
und zermalmete es zu Pulver und stäubte es aufs Wasser und gab's den
Kindern Israel zu trinken. \bibverse{21} Und sprach zu Aaron: Was hat
dir das Volk getan, daß du eine so große Sünde üben sie gebracht hast?
\bibverse{22} Aaron sprach: Mein Herr lasse seinen Zorn nicht ergrimmen.
Du weißt, daß dies Volk böse ist. \bibverse{23} Sie sprachen zu mir:
Mache uns Götter, die vor uns hergehen; denn wir wissen nicht, wie es
diesem Mann Mose gehet, der uns aus Ägyptenland geführet hat.
\bibverse{24} Ich sprach zu ihnen: Wer hat Gold, der reiße es ab und
gebe es mir. Und ich warf es ins Feuer; daraus ist das Kalb geworden.
\bibverse{25} Da nun Mose sah, daß das Volk los worden war (denn Aaron
hatte sie losgemacht durch ein Geschwätz, damit er sie fein wollte
anrichten), \bibverse{26} trat er in das Tor des Lagers und sprach: Her
zu mir, wer dem HErrn angehöret! Da sammelten sich zu ihm alle Kinder
Levi. \bibverse{27} Und er sprach zu ihnen: So spricht der HErr, der
GOtt Israels: Gürte ein jeglicher sein Schwert auf seine Lenden und
durchgehet hin und wieder von einem Tor zum andern im Lager und erwürge
ein jeglicher seinen Bruder, Freund und Nächsten. \bibverse{28} Die
Kinder Levi taten, wie ihnen Mose gesagt hatte, und fiel des Tages vom
Volk dreitausend Mann. \bibverse{29} Da sprach Mose: Füllet heute eure
Hände dem HErrn, ein jeglicher an seinem Sohn und Bruder, daß heute über
euch der Segen gegeben werde. \bibverse{30} Des Morgens sprach Mose zum
Volk: Ihr habt eine große Sünde getan; nun will ich hinaufsteigen zu dem
HErrn, ob ich vielleicht eure Sünde versöhnen möge. \bibverse{31} Als
nun Mose wieder zum HErrn kam, sprach er: Ach, das Volk hat eine große
Sünde getan und haben ihnen güldene Götter gemacht. \bibverse{32} Nun
vergib ihnen ihre Sünde! Wo nicht, so tilge mich auch aus deinem Buch,
das du geschrieben hast. \bibverse{33} Der HErr sprach zu Mose: Was? Ich
will den aus meinem Buch tilgen, der an mir sündiget. \bibverse{34} So
gehe nun hin und führe das Volk, dahin ich dir gesagt habe. Siehe, mein
Engel soll vor dir hergehen. Ich werde ihre Sünde wohl heimsuchen, wenn
meine Zeit kommt heimzusuchen. \bibverse{35} Also strafte der HErr das
Volk, daß sie das Kalb hatten gemacht, welches Aaron gemacht hatte.

\hypertarget{section-32}{%
\section{33}\label{section-32}}

\bibverse{1} Der HErr sprach zu Mose: Gehe, zeuch von dannen, du und das
Volk, das du aus Ägyptenland geführet hast, ins Land, das ich Abraham,
Isaak und Jakob geschworen habe und gesagt: Deinem Samen will ich's
geben. \bibverse{2} Ich will vor dir hersenden einen Engel und ausstoßen
die Kanaaniter, Amoriter, Hethiter, Pheresiter, Heviter und Jebusiter;
\bibverse{3} ins Land, da Milch und Honig innen fleußt. Ich will nicht
mit dir hinaufziehen; denn du bist ein halsstarrig Volk. Ich möchte dich
unterwegen auffressen. \bibverse{4} Da das Volk diese böse Rede hörete,
trugen sie Leide, und niemand trug seinen Schmuck an ihm. \bibverse{5}
Und der HErr sprach zu Mose: Sage zu den Kindern Israel: Ihr seid ein
halsstarrig Volk. Ich werde einmal plötzlich über dich kommen und dich
vertilgen. Und nun lege deinen Schmuck von dir, daß ich wisse, was ich
dir tun soll. \bibverse{6} Also taten die Kinder Israel ihren Schmuck
von sich vor dem Berge Horeb. \bibverse{7} Mose aber nahm die Hütte und
schlug sie auf, außen ferne vor dem Lager, und hieß sie eine Hütte des
Stifts. Und wer den HErrn fragen wollte, mußte herausgehen zur Hütte des
Stifts vor das Lager. \bibverse{8} Und wenn Mose ausging zur Hütte, so
stund alles Volk auf, und trat ein jeglicher in seiner Hütte Tür und
sahen ihm nach, bis er in die Hütte kam. \bibverse{9} Und wenn Mose in
die Hütte kam, so kam die Wolkensäule hernieder und stund in der Hütte
Tür und redete mit Mose. \bibverse{10} Und alles Volk sah die
Wolkensäule in der Hütte Tür stehen, und stunden auf und neigten sich,
ein jeglicher in seiner Hütte Tür. \bibverse{11} Der HErr aber redete
mit Mose von Angesicht zu Angesicht, wie ein Mann mit seinem Freunde
redet. Und wenn er wiederkehrete zum Lager, so wich sein Diener Josua,
der Sohn Nuns, der Jüngling, nicht aus der Hütte. \bibverse{12} Und Mose
sprach zu dem HErrn: Siehe, du sprichst zu mir: Führe das Volk hinauf,
und lässest mich nicht wissen, wen du mit mir senden willst, so du doch
gesagt hast: Ich kenne dich mit Namen, und hast Gnade vor meinen Augen
funden. \bibverse{13} Habe ich denn Gnade vor deinen Augen funden, so
laß mich deinen Weg wissen, damit ich dich kenne und Gnade vor deinen
Augen finde. Und siehe doch, daß dies Volk dein Volk ist. \bibverse{14}
Er sprach: Mein Angesicht soll gehen, damit will ich dich leiten.
\bibverse{15} Er aber sprach zu ihm: Wo nicht dein Angesicht gehet, so
führe uns nicht von dannen hinauf. \bibverse{16} Denn wobei soll doch
erkannt werden, daß ich und dein Volk vor deinen Augen Gnade funden
haben, ohne wenn du mit uns gehest? auf daß ich und dein Volk gerühmet
werden vor allem Volk, das auf dem Erdboden ist. \bibverse{17} Der HErr
sprach zu Mose: Was du jetzt geredet hast, will ich auch tun: denn du
hast Gnade vor meinen Augen funden, und ich kenne dich mit Namen.
\bibverse{18} Er aber sprach: So laß mich deine Herrlichkeit sehen.
\bibverse{19} Und er sprach: Ich will vor deinem Angesicht her alle
meine Güte gehen lassen und will lassen predigen des HErrn Namen vor
dir. Wem ich aber gnädig bin, dem bin ich gnädig; und wes ich mich
erbarme, des erbarme ich mich. \bibverse{20} Und sprach weiter: Mein
Angesicht kannst du nicht sehen; denn kein Mensch wird leben, der mich
siehet. \bibverse{21} Und der HErr sprach weiter: Siehe, es ist ein Raum
bei mir, da sollst du auf dem Fels stehen. \bibverse{22} Wenn denn nun
meine Herrlichkeit vorübergehet, will ich dich in der Felskluft lassen
stehen, und meine Hand soll ob dir halten, bis ich vorübergehe.
\bibverse{23} Und wenn ich meine Hand von dir tue, wirst du mir hinten
nachsehen; aber mein Angesicht kann man nicht sehen.

\hypertarget{section-33}{%
\section{34}\label{section-33}}

\bibverse{1} Und der HErr sprach zu Mose: Haue dir zwo steinerne Tafeln,
wie die ersten waren, daß ich die Worte darauf schreibe, die in den
ersten Tafeln waren, welche du zerbrochen hast. \bibverse{2} Und sei
morgen bereit, daß du frühe auf den Berg Sinai steigest und daselbst zu
mir tretest auf des Berges Spitze. \bibverse{3} Und laß niemand mit dir
hinaufsteigen, daß niemand gesehen werde um den ganzen Berg her; auch
kein Schaf noch Rind laß weiden gegen diesem Berge. \bibverse{4} Und
Mose hieb zwo steinerne Tafeln, wie die ersten waren, und stund des
Morgens frühe auf und stieg auf den Berg Sinai wie ihm der HErr geboten
hatte, und nahm die zwo steinernen Tafeln in seine Hand. \bibverse{5} Da
kam der HErr hernieder in einer Wolke und trat daselbst bei ihn und
predigte von des HErrn Namen. \bibverse{6} Und da der HErr vor seinem
Angesicht überging, rief er: HErr, HErr GOtt, barmherzig und gnädig und
geduldig und von großer Gnade und Treue; \bibverse{7} der du beweisest
Gnade in tausend Glied und vergibst Missetat, Übertretung und Sünde, und
vor welchem niemand unschuldig ist; der du die Missetat der Väter
heimsuchest auf Kinder und Kindeskinder bis ins dritte und vierte Glied.
\bibverse{8} Und Mose neigete sich eilend zu der Erde und betete ihn an,
\bibverse{9} und sprach: Habe ich, HErr, Gnade vor deinen Augen funden,
so gehe der HErr mit uns; denn es ist ein halsstarrig Volk, daß du
unserer Missetat und Sünde gnädig seiest und lassest uns dein Erbe sein.
\bibverse{10} Und er sprach: Siehe, ich will einen Bund machen vor all
deinem Volk und will Wunder tun, dergleichen nicht geschaffen sind in
allen Landen und unter allen Völkern; und alles Volk, darunter du bist,
soll sehen des HErrn Werk; denn wunderbarlich soll es sein, das ich bei
dir tun werde. \bibverse{11} Halte, was ich dir heute gebiete. Siehe,
ich will vor dir her ausstoßen die Amoriter, Kanaaniter, Hethiter,
Pheresiter, Heviter und Jebusiter. \bibverse{12} Hüte dich, daß du nicht
einen Bund machest mit den Einwohnern des Landes, da du einkommst, daß
sie dir nicht ein Ärgernis unter dir werden; \bibverse{13} sondern ihre
Altäre sollst du umstürzen und ihre Götzen zerbrechen und ihre Haine
ausrotten. \bibverse{14} Denn du sollst keinen andern Gott anbeten. Denn
der HErr heißet ein Eiferer, darum daß er ein eifriger GOtt ist.
\bibverse{15} Auf daß, wo du einen Bund mit des Landes Einwohnern
machest, und wenn sie huren ihren Göttern nach und opfern ihren Göttern,
daß sie dich nicht laden, und du von ihrem Opfer essest; \bibverse{16}
und nehmest deinen Söhnen ihre Töchter zu Weibern, und dieselben dann
huren ihren Göttern nach und machen deine Söhne auch ihren Göttern
nachhuren. \bibverse{17} Du sollst dir keine gegossenen Götter machen.
\bibverse{18} Das Fest der ungesäuerten Brote sollst du halten. Sieben
Tage sollst du ungesäuert Brot essen, wie ich dir geboten habe, um die
Zeit des Mondes Abib; denn in dem Mond Abib bist du aus Ägypten gezogen.
\bibverse{19} Alles, was seine Mutter am ersten bricht, ist mein; was
männlich sein wird in deinem Vieh, das seine Mutter bricht, es sei Ochse
oder Schaf. \bibverse{20} Aber den Erstling des Esels sollst du mit
einem Schaf lösen. Wo du es aber nicht lösest, so brich ihm das Genick.
Alle Erstgeburt deiner Söhne sollst du lösen. Und daß niemand vor mir
leer erscheine! \bibverse{21} Sechs Tage sollst du arbeiten; am
siebenten Tage sollst du feiern, beide mit Pflügen und mit Ernten:
\bibverse{22} Das Fest der Wochen sollst du halten mit den Erstlingen
der Weizenernte und das Fest der Einsammlung, wenn das Jahr um ist.
\bibverse{23} Dreimal im Jahr sollen alle Mannsnamen erscheinen vor dem
Herrscher, dem HErrn und GOtt Israels. \bibverse{24} Wenn ich die Heiden
vor dir ausstoßen und deine Grenze weitern werde, soll niemand deines
Landes begehren, dieweil du hinaufgehest dreimal im Jahr, zu erscheinen
vor dem HErrn, deinem GOtt. \bibverse{25} Du sollst das Blut meines
Opfers nicht opfern auf dem gesäuerten Brot; und das Opfer des
Osterfestes bleiben bis an den Morgen. \bibverse{26} Das Erstling von
den ersten Früchten deines Ackers sollst du in das Haus des HErrn deines
GOttes, bringen. Du sollst das Böcklein nicht kochen, wenn es noch an
seiner Mutter Milch ist. \bibverse{27} Und der HErr sprach zu Mose:
Schreibe diese Worte; denn nach diesen Worten habe ich mit dir und mit
Israel einen Bund gemacht. \bibverse{28} Und er war allda bei dem HErrn
vierzig Tage und vierzig Nächte und aß kein Brot und trank kein Wasser.
Und er schrieb auf die Tafeln solchen Bund, die zehn Worte.
\bibverse{29} Da nun Mose vom Berge Sinai ging, hatte er die zwo Tafeln
des Zeugnisses in seiner Hand; und wußte nicht, daß die Haut seines
Angesichts glänzete davon, daß er mit ihm geredet hatte. \bibverse{30}
Und da Aaron und alle Kinder Israel sahen, daß die Haut seines
Angesichts glänzete, fürchteten sie sich, zu ihm zu nahen. \bibverse{31}
Da rief ihnen Mose; und sie wandten sich zu ihm, beide Aaron und alle
Obersten der Gemeine; und er redete mit ihnen. \bibverse{32} Danach
naheten alle Kinder Israel zu ihm. Und er gebot ihnen alles, was der
HErr mit ihm geredet hatte auf dem Berge Sinai. \bibverse{33} Und wenn
er solches alles mit ihnen redete, legte er eine Decke auf sein
Angesicht. \bibverse{34} Und wenn er hineinging vor den HErrn, mit ihm
zu reden, tat er die Decke ab, bis er wieder herausging. Und wenn er
herauskam und redete mit den Kindern Israel, was ihm geboten war,
\bibverse{35} so sahen dann die Kinder Israel sein Angesicht an, wie daß
die Haut seines Angesichts glänzete; so tat er die Decke wieder auf sein
Angesicht, bis er wieder hineinging, mit ihm zu reden.

\hypertarget{section-34}{%
\section{35}\label{section-34}}

\bibverse{1} Und Mose versammelte die ganze Gemeine der Kinder Israel
und sprach zu ihnen: Das ist's, das der HErr geboten hat, das ihr tun
sollt: \bibverse{2} Sechs Tage sollt ihr arbeiten; den siebenten Tag
aber sollt ihr heilig halten, einen Sabbat der Ruhe des HErrn. Wer
darinnen arbeitet, soll sterben. \bibverse{3} Ihr sollt kein Feuer
anzünden am Sabbattage in allen euren Wohnungen. \bibverse{4} Und Mose
sprach zu der ganzen Gemeine der Kinder Israel: Das ist's, das der HErr
geboten hat: \bibverse{5} Gebt unter euch Hebopfer dem HErrn, also daß
das Hebopfer des HErrn ein jeglicher williglich bringe, Gold, Silber,
Erz, \bibverse{6} gelbe Seide, Scharlaken, Rosinrot, weiße Seide und
Ziegenhaar, \bibverse{7} rötlich Widderfell, Dachsfell und Föhrenholz,
\bibverse{8} Öl zur Lampe und Spezerei zur Salbe und zu gutem Räuchwerk,
\bibverse{9} Onyx und eingefaßte Steine zum Leibrock und zum Schildlein.
\bibverse{10} Und wer unter euch verständig, ist, der komme und mache,
was der HErr geboten hat: \bibverse{11} nämlich die Wohnung mit ihrer
Hütte und Decke, Rinken, Brettern, Riegeln, Säulen und Füßen;
\bibverse{12} die Lade mit ihren Stangen, den Gnadenstuhl und Vorhang;
\bibverse{13} den Tisch mit seinen Stangen und alle seinem Geräte und
die Schaubrote; \bibverse{14} den Leuchter, zu leuchten, und sein Gerät
und seine Lampen und das Öl zum Licht; \bibverse{15} den Räuchaltar mit
seinen Stangen, die Salbe und Spezerei zum Räuchwerk; das Tuch vor der
Wohnung Tür; \bibverse{16} den Brandopferaltar mit seinem ehernen
Gitter, Stangen und alle seinem Gerät; das Handfaß mit seinem Fuße;
\bibverse{17} den Umhang des Vorhofs mit seinen Säulen und Füßen und das
Tuch des Tors am Vorhof; \bibverse{18} die Nägel der Wohnung und des
Vorhofs mit ihren Säulen \bibverse{19} die Kleider des Amts zum Dienst
im Heiligen, die heiligen Kleider Aarons, des Priesters, mit den
Kleidern seiner Söhne zum Priestertum. \bibverse{20} Da ging die ganze
Gemeine der Kinder Israel aus von Mose. \bibverse{21} Und alle, die es
gerne und williglich gaben, kamen und brachten das Hebopfer dem HErrn
zum Werk der Hütte des Stifts und zu alle seinem Dienst und zu den
heiligen Kleidern. \bibverse{22} Es brachten aber beide, Mann und Weib,
wer es williglich tat, Hefte, Ohrenrinken, Ringe und Spangen und
allerlei gülden Gerät. Dazu brachte jedermann Gold zur Webe dem HErrn.
\bibverse{23} Und wer bei ihm fand gelbe Seide, Scharlaken, Rosinrot,
weiße Seide, Ziegenhaar, rötlich Widderfell und Dachsfell, der brachte
es. \bibverse{24} Und wer Silber und Erz hub, der brachte es zur Hebe
dem HErrn. Und wer Föhrenholz bei ihm fand, der brachte es zu allerlei
Werk des Gottesdienstes. \bibverse{25} Und welche verständige Weiber
waren, die wirkten mit ihren Händen und brachten ihr Werk von gelber
Seide, Scharlaken, Rosinrot und weißer Seide. \bibverse{26} Und welche
Weiber solche Arbeit konnten und willig dazu waren, die wirkten
Ziegenhaar. \bibverse{27} Die Fürsten aber brachten Onyx und eingefaßte
Steine zum Leibrock und zum Schildlein \bibverse{28} und Spezerei und Öl
zu den Lichtern und zur Salbe und zu gutem Räuchwerk. \bibverse{29} Also
brachten die Kinder Israel williglich, beide Mann und Weib, zu allerlei
Werk, das der HErr geboten hatte durch Mose, daß man's machen sollte.
\bibverse{30} Und Mose sprach zu den Kindern Israel: Sehet, der HErr hat
mit Namen berufen den Bezaleel, den Sohn Uris, des Sohns Hurs, vom Stamm
Juda, \bibverse{31} und hat ihn erfüllet mit dem Geist GOttes, daß er
weise, verständig, geschickt sei zu allerlei Werk, \bibverse{32}
künstlich zu arbeiten am Gold, Silber und Erz, \bibverse{33} Edelstein
schneiden und einsetzen, Holz zimmern, zu machen allerlei künstliche
Arbeit. \bibverse{34} Und hat ihm sein Herz unterweiset samt Ahaliab,
dem Sohne Ahisamachs, vom Stamm Dan. \bibverse{35} Er hat ihr Herz mit
Weisheit erfüllet, zu machen allerlei Werk, zu schneiden, wirken und zu
sticken mit gelber Seide, Scharlaken, Rosinrot und weißer Seide und mit
Weben, daß sie machen allerlei Werk und künstliche Arbeit erfinden.

\hypertarget{section-35}{%
\section{36}\label{section-35}}

\bibverse{1} Da arbeiteten Bezaleel und Ahaliab und alle weisen Männer,
denen der HErr Weisheit und Verstand gegeben hatte, zu wissen, wie sie
allerlei Werk machen sollten, zum Dienst des Heiligtums nach allem, das
der HErr geboten hatte. \bibverse{2} Und Mose rief dem Bezaleel und
Ahaliab und allen weisen Männern, denen der HErr Weisheit gegeben hatte
in ihr Herz, nämlich allen, die sich willig darerboten und hinzutraten,
zu arbeiten an dem Werke. \bibverse{3} Und sie nahmen zu sich von Mose
alle Hebe, die die Kinder Israel brachten zu dem Werk des Dienstes des
Heiligtums, daß es gemacht würde. Denn sie brachten alle Morgen ihre
willige Gabe zu ihm. \bibverse{4} Da kamen alle Weisen, die am Werk des
Heiligtums arbeiteten, ein jeglicher seines Werks, das sie machten,
\bibverse{5} und sprachen zu Mose: Das Volk bringet zu viel, mehr denn
zum Werk dieses Dienstes not ist, das der HErr zu machen geboten hat.
\bibverse{6} Da gebot Mose, daß man rufen ließ durchs Lager: Niemand tue
mehr zur Hebe des Heiligtums! Da hörete das Volk auf zu bringen.
\bibverse{7} Denn des Dinges war genug zu allerlei Werk, das zu machen
war, und noch übrig. \bibverse{8} Also machten alle weisen Männer unter
den Arbeitern am Werk die Wohnung, zehn Teppiche von gezwirnter weißer
Seide, gelber Seide, Scharlaken, Rosinrot, Cherubim, künstlich.
\bibverse{9} Die Länge eines Teppichs war achtundzwanzig Ellen und die
Breite vier Ellen, und waren alle in einem Maß. \bibverse{10} Und er
heftete je fünf Teppiche zusammen, einen an den andern. \bibverse{11}
Und machte gelbe Schläuflein an eines jeglichen Teppichs Ort, da sie
zusammengefüget werden. \bibverse{12} je fünfzig Schläuflein an einen
Teppich, damit einer den andern faßte. \bibverse{13} Und machte fünfzig
güldene Häklein; und fügte die Teppiche mit den Häklein einen an den
andern zusammen, daß es eine Wohnung würde. \bibverse{14} Und er machte
elf Teppiche von Ziegenhaaren zur Hütte üben die Wohnung, \bibverse{15}
dreißig Ellen lang und vier Ellen breit, alle in einem Maß.
\bibverse{16} Und fügte ihrer fünf zusammen auf ein Teil und sechs
zusammen aufs andere Teil. \bibverse{17} Und machte je fünfzig
Schläuflein an jeglichen Teppich am Ort, damit sie zusammengeheftet
würden. \bibverse{18} Und machte je fünfzig eherne Häklein, damit die
Hütte zusammen in eins gefüget würde. \bibverse{19} Und machte eine
Decke über die Hütte von rötlichen Widderfellen und über die noch eine
Decke von Dachsfellen. \bibverse{20} Und machte Bretter zur Wohnung von
Föhrenholz, die stehen sollten, \bibverse{21} ein jegliches zehn Ellen
lang und anderthalb Ellen breit, \bibverse{22} und an jeglichem zween
Zapfen, da mit eins an das andere gesetzt würde. Also machte er alle
Bretter zur Wohnung, \bibverse{23} daß derselben Bretter zwanzig gegen
Mittag stunden; \bibverse{24} und machte vierzig silberne Füße drunter,
unter jeglich Brett zween Füße an seinen zween Zapfen. \bibverse{25}
Also zur andern Seite der Wohnung, gegen Mitternacht, machte er auch
zwanzig Bretter \bibverse{26} mit vierzig silbernen Füßen, unter jeglich
Brett zween Füße. \bibverse{27} Aber hinten an der Wohnung gegen den
Abend machte er sechs Bretter, \bibverse{28} und zwei andere hinten an
den zwo Ecken der Wohnung, \bibverse{29} daß ein jegliches der beiden
sich mit seinem Ortbrett von unten auf gesellete und oben am Haupt
zusammenkäme mit einer Klammer, \bibverse{30} daß der Bretter acht
würden und sechzehn silberne Füße, unter jeglichem zween Füße.
\bibverse{31} Und er machte Riegel von Föhrenholz, fünf zu den Brettern
auf der einen Seite der Wohnung \bibverse{32} und fünf auf der andern
Seite und fünf hinten an, gegen den Abend. \bibverse{33} Und machte die
Riegel, daß sie mitten an den Brettern durchhingestoßen würden, von
einem Ende zum andern. \bibverse{34} Und überzog die Bretter mit Golde;
aber ihre Rinken machte er von Gold zu den Riegeln und überzog die
Riegel mit Golde. \bibverse{35} Und machte den Vorhang mit den Cherubim
dran künstlich mit gelber Seide, Scharlaken, Rosinrot und gezwirnter
weißer Seide. \bibverse{36} Und machte zu demselben vier Säulen von
Föhrenholz und überzog sie mit Gold und ihre Köpfe von Golde; und goß
dazu vier silberne Füße. \bibverse{37} Und machte ein Tuch in der Tür
der Hütte; von gelber Seide, Scharlaken, Rosenrot und gezwirnter weißer
Seide gestickt, \bibverse{38} und fünf Säulen dazu mit ihren Köpfen und
überzog ihre Köpfe und Reife mit Golde; und fünf eherne Füße dran.

\hypertarget{section-36}{%
\section{37}\label{section-36}}

\bibverse{1} Und Bezaleel machte die Lade von Föhrenholz, dritthalb
Ellen lang, anderthalb Ellen breit und hoch, \bibverse{2} und überzog
sie mit feinem Golde, inwendig und auswendig; und machte ihr einen
güldenen Kranz umher. \bibverse{3} Und goß vier güldene Rinken an ihre
vier Ecken, auf jeglicher Seite zween. \bibverse{4} Und machte Stangen
von Föhrenholz und überzog sie mit Golde \bibverse{5} und tat sie in die
Rinken an der Lade Seiten, daß man sie tragen konnte. \bibverse{6} Und
machte den Gnadenstuhl von feinem Golde, dritthalb Ellen lang und
anderthalb Ellen breit. \bibverse{7} Und machte zween Cherubim von
dichtem Golde an die zwei Enden des Gnadenstuhls, \bibverse{8} einen
Cherub an diesem Ende, den andern an jenem Ende. \bibverse{9} Und die
Cherubim breiteten ihre Flügel aus von oben her und deckten damit den
Gnadenstuhl; und ihre Antlitze stunden gegeneinander und sahen auf den
Gnadenstuhl. \bibverse{10} Und er machte den Tisch von Föhrenholz, zwo
Ellen lang, eine Elle breit und anderthalb Ellen hoch. \bibverse{11} Und
überzog ihn mit feinem Golde und machte ihm einen güldenen Kranz umher.
\bibverse{12} Und machte ihm eine Leiste umher, einer Hand breit hoch;
und machte einen güldenen Kranz um die Leiste her. \bibverse{13} Und goß
dazu vier güldene Rinken und tat sie an die vier Orte an seinen vier
Füßen \bibverse{14} hart an der Leiste, daß die Stangen drinnen wären,
damit man den Tisch trüge. \bibverse{15} Und machte die Stangen von
Föhrenholz und überzog sie mit Gold, daß man den Tisch damit trüge.
\bibverse{16} Und machte auch von feinem Golde das Geräte auf den Tisch:
Schüsseln, Becher, Kannen und Schalen, damit man aus- und einschenkte.
\bibverse{17} Und machte den Leuchter von feinem, dichtem Golde. Daran
waren der Schaft mit Röhren, Schalen, Knäufen und Blumen. \bibverse{18}
Sechs Röhren gingen zu seinen Seiten aus, zu jeglicher Seite drei
Röhren. \bibverse{19} Drei Schalen waren an jeglichem Rohr mit Knäufen
und Blumen. \bibverse{20} An dem Leuchter aber waren vier Schalen mit
Knäufen und Blumen, \bibverse{21} je unter zwo Röhren ein Knauf, daß
also sechs Röhren aus ihm gingen, \bibverse{22} und ihre Knäufe und
Röhren daran, und war alles aus dichtem, feinem Golde. \bibverse{23} Und
machte die sieben Lampen mit ihren Lichtschneuzen und Löschnäpfen von
feinem Golde. \bibverse{24} Aus einem Zentner feines Goldes machte er
ihn und alle seine Geräte. \bibverse{25} Er machte auch den Räuchaltar
von Föhrenholz, eine Elle lang und breit, gleich viereckig und zwo Ellen
hoch, mit seinen Hörnern. \bibverse{26} Und überzog ihn mit feinem
Golde, sein Dach und seine Wände rings umher und seine Hörner. Und
machte ihm einen Kranz umher von Golde \bibverse{27} und zween güldene
Rinken unter dem Kranz zu beiden Seiten, daß man Stangen drein täte und
ihn damit trüge. \bibverse{28} Aber die Stangen machte er von Föhrenholz
und überzog sie mit Golde. \bibverse{29} Und machte die heilige Salbe
und Räuchwerk von reiner Spezerei nach Apothekerkunst.

\hypertarget{section-37}{%
\section{38}\label{section-37}}

\bibverse{1} Und machte den Brandopferaltar von Föhrenholz, fünf Ellen
lang und, breit, gleich viereckig und drei Ellen hoch. \bibverse{2} Und
machte vier Hörner, die aus ihm gingen, auf seinen vier Ecken; und über
zog ihn mit Erz. \bibverse{3} Und machte allerlei Geräte zu dem Altar
Aschentöpfe, Schaufeln, Becken, Kreuel, Kohlpfannen: alles von Erz.
\bibverse{4} Und machte am Altar ein Gitter, wie ein Netz, von Erz
umher, von unten auf bis an die Hälfte des Altars. \bibverse{5} Und goß
vier Rinken an die vier Orte des ehernen Gitters zu Stangen.
\bibverse{6} Dieselben machte er von Föhrenholz und überzog sie mit Erz.
\bibverse{7} Und tat sie in die Rinken an den Seiten des Altars, daß man
ihn damit trüge; und machte ihn inwendig hohl. \bibverse{8} Und machte
das Handfaß von Erz und seinen Fuß auch von Erz, gegen den Weibern, die
vor der Tür der Hütte des Stifts dieneten. \bibverse{9} Und er machte
einen Vorhof gegen Mittag mit einem Umhang hundert Ellen lang von
gezwirnter weißer Seide, \bibverse{10} mit ihren zwanzig Säulen und
zwanzig Füßen von Erz, aber ihre Knäufe und Reife von Silber;
\bibverse{11} desselbengleichen gegen Mitternacht hundert Ellen mit
zwanzig Säulen und zwanzig Füßen von Erz, aber ihre Knäufe und Reife von
Silber; \bibverse{12} gegen den Abend aber fünfzig Ellen mit zehn Säulen
und zehn Füßen, aber ihre Knäufe und Reife von Silber;: \bibverse{13}
gegen den Morgen aber fünfzig Ellen, \bibverse{14} fünfzehn Ellen auf
jeglicher Seite des Tors am Vorhof, je mit drei Säulen und drei Füßen,
\bibverse{15} und auf der andern Seite fünfzehn Ellen, daß ihrer so viel
war an der einen Seite des Tors am Vorhofe als auf der andern, mit drei
Säulen und drei Füßen, \bibverse{16} daß alle Umhänge des Vorhofs waren
von gezwirnter weißer Seide \bibverse{17} und die Füße der Säulen von
Erz und ihre Knäufe und Reife von Silber, also daß ihre Köpfe überzogen
waren mit Silber; aber ihre Reife waren silbern an allen Säulen des
Vorhofs. \bibverse{18} Und das Tuch in dem Tor des Vorhofs machte er
gestickt, von gelber Seide, Scharlaken, Rosinrot und gezwirnter weißer
Seide, zwanzig Ellen lang und fünf Ellen hoch, nach dem Maß der Umhänge
des Vorhofs; \bibverse{19} dazu vier Säulen und vier Füße von Erz und
ihre Knäufe von Silber und ihre Köpfe überzogen und ihre Reife silbern.
\bibverse{20} Und alle Nägel der Wohnung und des Vorhofs ringsherum
waren von Erz: \bibverse{21} Das ist nun die Summa zu der Wohnung des
Zeugnisses, die erzählet ist, wie Mose gesagt hat, zum Gottesdienst der
Leviten unter der Hand Ithamars, Aarons, des Priesters, Sohnes,
\bibverse{22} die Bezaleel, der Sohn Uris, des Sohns Hurs, vom Stamm
Juda machte, alles, wie der HErr Mose geboten hatte; \bibverse{23} und
mit ihm Ahaliab, der Sohn Ahisamachs, vom Stamm Dan, ein Meister zu
schneiden, zu wirken und zu sticken mit gelber Seide, Scharlaken,
Rosinrot und weißer Seide. \bibverse{24} Alles Gold, das verarbeitet ist
in diesem ganzen Werk des Heiligtums, das zur Webe gegeben ward, ist
neunundzwanzig Zentner, siebenhundertunddreißig Sekel nach dem Sekel des
Heiligtums. \bibverse{25} Des Silbers aber, das von der Gemeine kam, war
hundert Zentner, tausendsiebenhundertfünfundsiebenzig Sekel, nach dem
Sekel des Heiligtums. \bibverse{26} So manch Haupt, so mancher halber
Sekel, nach dem Sekel des Heiligtums, von allen, die gezählet wurden,
von zwanzig Jahren an und drüber, sechshundertmal tausend
dreitausendfünfhundertundfünfzig. \bibverse{27} Aus den hundert Zentnern
Silbers goß man die Füße des Heiligtums und die Füße des Vorhangs,
hundert Füße aus hundert Zentnern, je einen Zentner zum Fuß.
\bibverse{28} Aber aus den tausend siebenhundertundfünfundsiebenzig
Sekeln wurden gemacht der Säulen Knäufe, und ihre Köpfe überzogen und
ihre Reife. \bibverse{29} Die Webe aber des Erzes war siebenzig Zentner,
zweitausendundvierhundert Sekel. \bibverse{30} Daraus wurden gemacht die
Füße in der Tür der Hütte des Stifts und der eherne Altar und das eherne
Gitter dran und alles Geräte des Altars, \bibverse{31} dazu die Füße des
Vorhofs ringsherum und die Füße des Tors am Vorhof, alle Nägel der
Wohnung und alle Nägel des Vorhofs ringsherum.

\hypertarget{section-38}{%
\section{39}\label{section-38}}

\bibverse{1} Aber von der gelben Seide, Scharlaken und Rosinrot machten
sie Aaron Kleider, zu dienen im Heiligtum, wie der HErr Mose geboten
hatte. \bibverse{2} Und er machte den Leibrock mit Golde, gelber Seide,
Scharlaken, Rosinrot und gezwirnter weißer Seide. \bibverse{3} Und
schlug das Gold und schnitt's zu Faden, daß man's künstlich wirken
konnte unter die gelbe Seide, Scharlaken, Rosinrot und weiße Seide,
\bibverse{4} daß man's auf beiden Achseln zusammenfügte und an beiden
Seiten zusammenbände. \bibverse{5} Und sein Gurt war nach derselben
Kunst und Werk von Gold, gelber Seide, Scharlaken, Rosinrot und
gezwirnter weißer Seide, wie der HErr Mose geboten hatte. \bibverse{6}
Und sie machten zween Onyxsteine, umher gefasset mit Gold, gegraben
durch die Steinschneider, mit den Namen der Kinder Israel, \bibverse{7}
und heftete sie auf die Schultern des Leibrocks, daß es Steine seien zum
Gedächtnis der Kinder Israel, wie der HErr Mose geboten hatte.
\bibverse{8} Und sie machten das Schildlein nach der Kunst und Werk des
Leibrocks von Gold, gelber Seide, Scharlaken, Rosinrot und gezwirnter
weißer Seide, \bibverse{9} daß es viereckig und zwiefach war, einer Hand
lang und breit. \bibverse{10} Und fülleten es mit vier Riegen Steinen.
Die erste Riege war ein Sarder, Topaser und Smaragd; \bibverse{11} die
andere ein Rubin, Saphir und Demant; \bibverse{12} die dritte ein
Lynkurer, Achat und Amethyst; \bibverse{13} die vierte ein Türkis, Onyx
und Jaspis, umher gefasset mit Gold in allen Riegen. \bibverse{14} Und
die Steine stunden nach den zwölf Namen der Kinder Israel, gegraben
durch die Steinschneider, ein jeglicher seines Namens, nach den zwölf
Stämmen. \bibverse{15} Und sie machten am Schildlein Ketten mit zwei
Enden von feinem Gold \bibverse{16} und zwo güldene Spangen und zween
güldene Ringe: und hefteten die zween Ringe auf die zwo Ecken des
Schildleins. \bibverse{17} Und die zwo güldenen Ketten taten sie in die
zween Ringe auf den Ecken des Schildleins. \bibverse{18} Aber die zwei
Enden der Ketten taten sie an die zwo Spangen und hefteten sie auf die
Ecken des Leibrocks gegeneinander über. \bibverse{19} Und machten zween
andere güldene Ringe und hefteten sie an die zwo andern Ecken des
Schildleins an seinen Ort, daß es fein anläge auf dem Leibrock.
\bibverse{20} Und machten zween andere güldene Ringe, die taten sie an
die zwo Ecken unten am Leibrock gegeneinander über, da der Leibrock
unten zusammengehet, \bibverse{21} daß das Schildlein mit seinen Ringen
an die Ringe des Leibrocks geknüpft würde mit einer gelben Schnur, daß
es auf dem Leibrock hart anläge und nicht von dem Leibrock los würde,
wie der HErr Mose geboten hatte. \bibverse{22} Und er machte den
Seidenrock zum Leibrock, gewirkt ganz von gelber Seide, \bibverse{23}
und sein Loch oben mitten inne und eine Borte ums Loch her gefaltet, daß
er nicht zerrisse. \bibverse{24} Und sie machten an seinem Saum
Granatäpfel von gelber Seide, Scharlaken, Rosinrot und gezwirnter weißer
Seide. \bibverse{25} Und machten Schellen von feinem Golde; die taten
sie zwischen die Granatäpfel ringsumher am Saum des Seidenrocks,
\bibverse{26} je ein Granatapfel und eine Schelle um und um am Saum,
darin zu dienen, wie der HErr Mose geboten hatte. \bibverse{27} Und
machten auch die engen Röcke, von weißer Seide gewirkt, Aaron und seinen
Söhnen, \bibverse{28} und den Hut von weißer Seide und die schönen
Hauben von weißer Seide und Niederkleider von gezwirnter weißer Leinwand
\bibverse{29} und den gestickten Gürtel von gezwirnter weißer Seide,
gelber Seide, Scharlaken, Rosinrot, wie der HErr Mose geboten hatte.
\bibverse{30} Sie machten auch das Stirnblatt, nämlich die heilige
Krone, von feinem Golde und gruben Schrift drein: Die Heiligkeit des
HErrn. \bibverse{31} Und banden eine gelbe Schnur dran daß sie an den
Hut von oben her geheftet würde, wie der HErr Mose geboten hatte.
\bibverse{32} Also ward vollendet das ganze Werk der Wohnung der Hütte
des Stifts. Und die Kinder Israel taten alles, was der HErr Mose geboten
hatte, \bibverse{33} und brachten die Wohnung zu Mose: die Hütte und
alle ihre Geräte, Häklein, Bretter, Riegel, Säulen, Füße; \bibverse{34}
die Decke von rötlichen Widderfellen, die Decke von Dachsfellen und den
Vorhang; \bibverse{35} die Lade des Zeugnisses mit ihren Stangen; den
Gnadenstuhl; \bibverse{36} den Tisch und alle seine Geräte und die
Schaubrote; \bibverse{37} den schönen Leuchter, mit den Lampen
zubereitet, und alle seinem Geräte, und Öl zu Lichtern; \bibverse{38}
den güldenen Altar und die Salbe und gut Räuchwerk; das Tuch in der
Hütte Tür; \bibverse{39} den ehernen Altar und sein ehern Gitter mit
seinen Stangen und alle seinem Gerät; das Handfaß mit seinem Fuß;
\bibverse{40} die Umhänge des Vorhofs mit: seinen Säulen und Füßen; das
Tuch im Tor des Vorhofs mit seinen Seilen und Nägeln und allem Geräte
zum Dienst der Wohnung der Hütte des Stifts; \bibverse{41} die
Amtskleider des Priesters Aaron, zu dienen im Heiligtum, und die Kleider
seiner Söhne, daß sie Priesteramt täten. \bibverse{42} Alles, wie der
HErr Mose geboten hatte, taten die Kinder Israel an alle diesem Dienst.
\bibverse{43} Und Mose sah an alle dies Werk; und siehe, sie hatten es
gemacht, wie der HErr geboten hatte. Und er segnete sie.

\hypertarget{section-39}{%
\section{40}\label{section-39}}

\bibverse{1} Und der HErr redete mit Mose und sprach: \bibverse{2} Du
sollst die Wohnung der Hütte des Stifts aufrichten am ersten Tage des
ersten Monden. \bibverse{3} Und sollst darein setzen die Lade des
Zeugnisses und vor die Lade den Vorhang hängen. \bibverse{4} Und sollst
den Tisch darbringen und ihn zubereiten und den Leuchter darstellen und
die Lampen drauf setzen. \bibverse{5} Und sollst den güldenen Räuchaltar
setzen vor die Lade des Zeugnisses und das Tuch in der Tür der Wohnung
aufhängen. \bibverse{6} Den Brandopferaltar aber sollst du setzen heraus
vor die Tür der Wohnung der Hütte des Stifts \bibverse{7} und das
Handfaß zwischen der Hütte des Stifts und dem Altar, und Wasser drein
tun; \bibverse{8} und den Vorhof stellen umher und das Tuch in der Tür
des Vorhofs aufhängen. \bibverse{9} Und sollst die Salbe nehmen und die
Wohnung und alles, was drinnen ist, salben; und sollst sie weihen mit
alle ihrem Geräte, daß sie heilig sei. \bibverse{10} Und sollst den
Brandopferaltar salben mit alle seinem Geräte und weihen, daß er
allerheiligst sei. \bibverse{11} Sollst auch das Handfaß und seinen Fuß
salben und weihen. \bibverse{12} Und sollst Aaron und seine Söhne vor
die Tür der Hütte des Stifts führen und mit Wasser waschen;
\bibverse{13} und Aaron die heiligen Kleider anziehen und salben und
weihen, daß er mein Priester sei; \bibverse{14} und seine Söhne auch
herzuführen und ihnen die engen Röcke anziehen; \bibverse{15} und sie
salben, wie du ihren Vater gesalbet hast, daß sie meine Priester seien.
Und die Salbung sollen sie haben zum ewigen Priestertum bei ihren
Nachkommen. \bibverse{16} Und Mose tat alles, wie ihm der HErr geboten
hatte. \bibverse{17} Also ward die Wohnung aufgerichtet im andern Jahr,
am ersten Tage des ersten Monds. \bibverse{18} Und da Mose sie
aufrichtete, setzte er die Füße und die Bretter und Riegel und richtete
die Säulen auf. \bibverse{19} Und breitete die Hütte aus zur Wohnung und
legte die Decke der Hütte oben drauf, wie der HErr ihm geboten hatte.
\bibverse{20} Und nahm das Zeugnis und legte es in die Lade; und tat die
Stangen an die Lade und tat den Gnadenstuhl oben auf die Lade.
\bibverse{21} Und brachte die Lade in die Wohnung und hing den Vorhang
vor die Lade des Zeugnisses, wie ihm der HErr geboten hatte.
\bibverse{22} Und setzte den Tisch in die Hütte des Stifts, in den
Winkel der Wohnung gegen Mitternacht, außen vor dem Vorhang.
\bibverse{23} Und bereitete Brot darauf vor dem HErrn, wie ihm der HErr
geboten hatte. \bibverse{24} Und setzte den Leuchter auch hinein gegen
dem Tisch über, in den Winkel der Wohnung gegen Mittag. \bibverse{25}
Und tat Lampen drauf vor dem HErrn, wie ihm der HErr geboten hatte.
\bibverse{26} Und setzte den güldenen Altar hinein, vor den Vorhang.
\bibverse{27} Und räucherte drauf mit gutem Räuchwerk, wie ihm der HErr
geboten hatte. \bibverse{28} Und hing das Tuch in die Tür der Wohnung.
\bibverse{29} Aber den Brandopferaltar setzte er vor die Tür der Wohnung
der Hütte des Stifts; und opferte drauf Brandopfer und Speisopfer, wie
ihm der HErr geboten hatte. \bibverse{30} Und das Handfaß setzte er
zwischen die Hütte des Stifts und den Altar; und tat Wasser drein zu
waschen. \bibverse{31} Und Mose, Aaron und seine Söhne wuschen ihre
Hände und Füße draus. \bibverse{32} Denn sie müssen sich waschen, wenn
sie in die Hütte des Stifts gehen oder hinzutreten zum Altar, wie ihm
der HErr geboten hatte. \bibverse{33} Und er richtete den Vorhof auf, um
die Wohnung und um den Altar her, und hing den Vorhang in das Tor des
Vorhofs. Also vollendete Mose das ganze Werk. \bibverse{34} Da bedeckte
eine Wolke die Hütte des Stifts, und die Herrlichkeit des HErrn füllete
die Wohnung. \bibverse{35} Und Mose konnte nicht in die Hütte des Stifts
gehen, weil die Wolke drauf blieb, und die Herrlichkeit des HErrn die
Wohnung füllete. \bibverse{36} Und wenn die Wolke sich aufhub von der
Wohnung, so zogen die Kinder Israel, so oft sie reiseten. \bibverse{37}
Wenn sich aber die Wolke nicht aufhub, so zogen sie nicht, bis an den
Tag, da sie sich aufhub. \bibverse{38} Denn die Wolke des HErrn war des
Tages auf der Wohnung, und des Nachts war sie feurig, vor den Augen des
ganzen Hauses Israel, solange sie reiseten.
