\hypertarget{section}{%
\section{1}\label{section}}

\bibverse{1} Paulus, berufen zum Apostel JEsu Christi durch den Willen
GOttes, undBruder Sosthenes: \bibverse{2} Der Gemeinde GOttes zu
Korinth, den Geheiligten in Christo JEsu, denberufenen Heiligen samt
allen denen, die anrufen den Namen unsers HErrn JEsu Christian allen
ihren und unsern Orten. \bibverse{3} Gnade sei mit euch und Friede von
GOtt, unserm Vater, und dem HErrnJEsu Christo! \bibverse{4} Ich danke
meinem GOtt allezeit eurethalben für die Gnade GOttes, dieeuch gegeben
ist in Christo JEsu, \bibverse{5} daß ihr seid durch ihn an allen
Stücken reich gemacht, an aller Lehreund in aller Erkenntnis
\bibverse{6} wie denn die Predigt von Christo in euch kräftig worden
ist, \bibverse{7} also daß ihr keinen Mangel habt an irgendeiner Gabe
und wartet nur aufdie Offenbarung unsers HErrn JEsu Christi.
\bibverse{8} welcher auch wird euch fest behalten bis ans Ende, daß ihr
unsträflichseid auf den Tag unsers HErrn JEsu Christi. \bibverse{9} Denn
GOtt ist treu, durch welchen ihr berufen seid zur Gemeinschaftseines
Sohnes JEsu Christi, unsers HErrn. \bibverse{10} Ich ermahne euch aber,
liebe Brüder, durch den Namen unsers HErrnJEsu Christi; daß ihr allzumal
einerlei Rede führet und lasset nicht Spaltungen unter euchsein, sondern
haltet fest aneinander in einem Sinn und in einerlei Meinung.
\bibverse{11} Denn mir ist vorkommen, liebe Brüder, durch die aus Chloes
Gesindevon euch, daß Zank unter euch sei. \bibverse{12} Ich sage aber
davon, daß unter euch einer spricht: Ich bin paulisch;der andere: Ich
bin apollisch; der dritte: Ich bin kephisch; der vierte: Ich bin
christisch. \bibverse{13} Wie? ist Christus nun zertrennet? Ist denn
Paulus für euchgekreuziget, oder seid ihr auf des Paulus Namen getauft?
\bibverse{14} Ich danke GOtt, daß ich niemand unter euch getauft habe
außerCrispus und Gajus, \bibverse{15} daß nicht jemand sagen möge, ich
hätte auf meinen Namen getauft. \bibverse{16} Ich habe aber auch getauft
des Stephanas Hausgesinde; danach weißich nicht, ob ich etliche andere
getauft habe. \bibverse{17} Denn Christus hat mich nicht gesandt zu
taufen, sondern dasEvangelium zu predigen, nicht mit klugen Worten, auf
daß nicht das Kreuz Christizunichte werde. \bibverse{18} Denn das Wort
vom Kreuz ist eine Torheit denen, die verloren werden;uns aber, die wir
selig werden, ist es eine Gotteskraft. \bibverse{19} Denn es stehet
geschrieben: Ich will zunichte machen die Weisheit derWeisen, und den
Verstand der Verständigen will ich verwerfen. \bibverse{20} Wo sind die
Klugen? Wo sind die Schriftgelehrten? Wo sind dieWeltweisen? Hat nicht
GOtt die Weisheit dieser Welt zur Torheit gemacht? \bibverse{21} Denn
dieweil die Welt durch ihre Weisheit GOtt in seiner Weisheit
nichterkannte, gefiel es GOtt wohl, durch törichte Predigt selig zu
machen die, so daranglauben, \bibverse{22} sintemal die Juden Zeichen
fordern, und die Griechen nach Weisheitfragen. \bibverse{23} Wir aber
predigen den gekreuzigten Christum, den Juden ein Ärgernisund den
Griechen eine Torheit. \bibverse{24} Denen aber, die berufen sind,
beide, Juden und Griechen, predigen wirChristum göttliche Kraft und
göttliche Weisheit. \bibverse{25} Denn die göttliche Torheit ist weiser,
denn die Menschen sind, und diegöttliche Schwachheit ist stärker, denn
die Menschen sind. \bibverse{26} Sehet an, liebe Brüder, euren Beruf:
nicht viel Weise nach demFleisch, nicht viel Gewaltige, nicht viel Edle
sind berufen. \bibverse{27} sondern was töricht ist vor der Welt, das
hat GOtt erwählet, daß er dieWeisen zuschanden machte; und was schwach
ist vor der Welt, das hat GOtt erwählet,daß er zuschanden machte, was
stark ist; \bibverse{28} und das Unedle vor der Welt und das Verachtete
hat GOtt erwählet,und das da nichts ist, daß er zunichte machte, was
etwas ist, \bibverse{29} auf daß sich vor ihm kein Fleisch rühme.
\bibverse{30} Von welchem auch ihr herkommt in Christo JEsu, welcher uns
gemachtist von GOtt zur Weisheit und zur Gerechtigkeit und zur Heiligung
und zur Erlösung, \bibverse{31} auf daß (wie geschrieben stehet), wer
sich rühmet, der rühme sich desHErrn.

\hypertarget{section-1}{%
\section{2}\label{section-1}}

\bibverse{1} Und ich, liebe Brüder, da ich zu euch kam, kam ich nicht
mit hohenWorten oder hoher Weisheit, euch zu verkündigen die göttliche
Predigt. \bibverse{2} Denn ich hielt mich nicht dafür, daß ich etwas
wüßte unter euch ohneallein JEsum Christum, den Gekreuzigten.
\bibverse{3} Und ich war bei euch mit Schwachheit und mit Furcht und mit
großemZittern. \bibverse{4} Und mein Wort und meine Predigt war nicht in
vernünftigen Redenmenschlicher Weisheit, sondern in Beweisung des
Geistes und der Kraft, \bibverse{5} auf daß euer Glaube bestehe nicht
auf Menschenweisheit, sondern aufGOttes Kraft. \bibverse{6} Wovon wir
aber reden, das ist dennoch Weisheit bei denVollkommenen; nicht eine
Weisheit dieser Welt, auch nicht der Obersten dieser Welt,welche
vergehen; \bibverse{7} sondern wir reden von der heimlichen, verborgenen
Weisheit GOttes,welche GOtt verordnet hat vor der Welt zu unserer
Herrlichkeit, \bibverse{8} welche keiner von den Obersten dieser Welt
erkannt hat; denn wo siedie erkannt hätten, hätten sie den HErrn der
Herrlichkeit nicht gekreuziget; \bibverse{9} sondern wie geschrieben
stehet: Das kein Auge gesehen hat und keinOhr gehöret hat und in keines
Menschen Herz kommen ist, das GOtt bereitet hat denen,die ihn lieben.
\bibverse{10} Uns aber hat es GOtt offenbaret durch seinen Geist; denn
der Geisterforschet alle Dinge, auch die Tiefen der Gottheit.
\bibverse{11} Denn welcher Mensch weiß, was im Menschen ist, ohne der
Geist desMenschen, der in ihm ist? Also auch weiß niemand, was in GOtt
ist, ohne der GeistGOttes. \bibverse{12} Wir aber haben nicht empfangen
den Geist der Welt, sondern denGeist aus GOtt, daß wir wissen können,
was uns von GOtt gegeben ist. \bibverse{13} Welches wir auch reden,
nicht mit Worten, welche menschlicheWeisheit lehren kann, sondern mit
Worten, die der Heilige Geist lehret, und richtengeistliche Sachen
geistlich. \bibverse{14} Der natürliche Mensch aber vernimmt nichts vom
Geist GOttes; es istihm eine Torheit, und kann es nicht erkennen; denn
es muß geistlich gerichtet sein. \bibverse{15} Der Geistliche aber
richtet alles und wird von niemand gerichtet. \bibverse{16} Denn wer hat
des HErrn Sinn erkannt, oder wer will ihn unterweisen?Wir aber haben
Christi Sinn.

\hypertarget{section-2}{%
\section{3}\label{section-2}}

\bibverse{1} Und ich, liebe Brüder, konnte nicht mit euch reden als mit
Geistlichen,sondern als mit Fleischlichen, wie mit jungen Kindern in
Christo. \bibverse{2} Milch habe ich euch zu trinken gegeben und nicht
Speise; denn ihrkonntet noch nicht; auch könnt ihr noch jetzt nicht,
\bibverse{3} dieweil ihr noch fleischlich seid. Denn sintemal Eifer und
Zank undZwietracht unter euch sind, seid ihr denn nicht fleischlich und
wandelt nach menschlicherWeise? \bibverse{4} Denn so einer sagt: Ich bin
paulisch, der andere aber: Ich bin apollisch,seid ihr denn nicht
fleischlich? \bibverse{5} Wer ist nun Paulus? Wer ist Apollo? Diener
sind sie, durch welche ihrseid gläubig worden, und dasselbige, wie der
HErr einem jeglichen gegeben hat. \bibverse{6} Ich habe gepflanzet,
Apollo hat begossen, aber GOtt hat das Gedeihengegeben. \bibverse{7} So
ist nun weder der da pflanzet, noch der da begießt, etwas, sondernGOtt,
der das Gedeihen gibt. \bibverse{8} Der aber pflanzet und der da
begießt, ist einer wie der andere. Einjeglicher aber wird seinen Lohn
empfangen nach seiner Arbeit. \bibverse{9} Denn wir sind GOttes
Mitarbeiter; ihr seid GOttes Ackerwerk und GOttesGebäu. \bibverse{10}
Ich von GOttes Gnaden, die mir gegeben ist, habe den Grund gelegtals ein
weiser Baumeister; ein anderer bauet darauf. Ein jeglicher aber sehe zu,
wie erdarauf baue. \bibverse{11} Einen andern Grund kann zwar niemand
legen außer dem, der gelegtist, welcher ist JEsus Christus.
\bibverse{12} So aber jemand auf diesen Grund bauet Gold, Silber,
Edelsteine, Holz,Heu, Stoppeln, \bibverse{13} so wird eines jeglichen
Werk offenbar werden; der Tag wird's klarmachen. Denn es wird durchs
Feuer offenbar werden, und welcherlei eines jeglichenWerk sei, wird das
Feuer bewähren. \bibverse{14} Wird jemandes Werk bleiben, das er darauf
gebauet hat, so wird erLohn empfangen. \bibverse{15} Wird aber jemandes
Werk verbrennen, so wird er des Schaden leiden;er selbst aber wird selig
werden, so doch wie durchs Feuer. \bibverse{16} Wisset ihr nicht, daß
ihr GOttes Tempel seid, und der Geist GOttes ineuch wohnet?
\bibverse{17} So jemand den Tempel GOttes verderbet, den wird GOtt
verderben;denn der Tempel GOttes ist heilig; der seid ihr. \bibverse{18}
Niemand betrüge sich selbst! Welcher sich unter euch dünkt, weise
zusein, der werde ein Narr in dieser Welt, daß er möge weise sein.
\bibverse{19} Denn dieser Welt Weisheit ist Torheit bei GOtt. Denn es
stehetgeschrieben: Die Weisen erhaschet er in ihrer Klugheit.
\bibverse{20} Und abermal: Der HErr weiß der Weisen Gedanken, daß sie
eitel sind. \bibverse{21} Darum rühme sich niemand eines Menschen! Es
ist alles euer, \bibverse{22} es sei Paulus oder Apollo, es sei Kephas
oder die Welt, es sei dasLeben oder der Tod, es sei das Gegenwärtige
oder das Zukünftige: alles ist euer. \bibverse{23} Ihr aber seid
Christi; Christus aber ist GOttes.

\hypertarget{section-3}{%
\section{4}\label{section-3}}

\bibverse{1} Dafür halte uns jedermann, nämlich für Christi Diener und
Haushalterüber GOttes Geheimnisse. \bibverse{2} Nun sucht man nicht mehr
an den Haushaltern, denn daß sie treuerfunden werden. \bibverse{3} Mir
aber ist's ein Geringes, daß ich von euch gerichtet werde oder voneinem
menschlichen Tage; auch richte ich mich selbst nicht. \bibverse{4} Ich
bin mir wohl nichts bewußt, aber darinnen bin ich nichtgerechtfertiget;
der HErr ist's aber, der mich richtet. \bibverse{5} Darum richtet nicht
vor der Zeit, bis der HErr komme, welcher auchwird ans Licht bringen,
was im Finstern verborgen ist, und den Rat der Herzenoffenbaren; alsdann
wird einem jeglichen von GOtt Lob widerfahren. \bibverse{6} Solches
aber, liebe Brüder, habe ich auf mich und Apollo gedeutet umeuretwillen,
daß ihr an uns lernet, daß niemand höher von sich halte, denn
jetztgeschrieben ist, auf daß sich nicht einer wider den andern um
jemandes willen aufblase. \bibverse{7} Denn wer hat dich vorgezogen? Was
hast du aber, das du nichtempfangen hast? So du es aber empfangen hast,
was rühmest du dich denn, als der esnicht empfangen hätte? \bibverse{8}
Ihr seid schon satt worden; ihr seid schon reich worden; ihr
herrschetohne uns. Und wollte GOtt, ihr herrschet, auf daß auch wir mit
euch herrschen möchten. \bibverse{9} Ich halte aber, GOtt habe uns
Apostel für die Allergeringstendargestellet, als dem Tode übergeben.
Denn wir sind ein Schauspiel worden der Weltund den Engeln und den
Menschen. \bibverse{10} Wir sind Narren um Christi willen, ihr aber seid
klug in Christo; wirschwach, ihr aber stark; ihr herrlich, wir aber
verachtet. \bibverse{11} Bis auf diese Stunde leiden wir Hunger und
Durst und sind nackendund werden geschlagen und haben keine gewisse
Stätte \bibverse{12} und arbeiten und wirken mit unsern eigenen Händen.
Man schilt uns,so segnen wir; man verfolgt uns, so dulden wir's, man
lästert uns, so flehen wir. \bibverse{13} Wir sind stets als ein Fluch
der Welt und ein Fegopfer aller Leute. \bibverse{14} Nicht schreibe ich
solches, daß ich euch beschäme, sondern ichermahne euch als meine lieben
Kinder. \bibverse{15} Denn ob ihr gleich zehntausend Zuchtmeister hättet
in Christo, so habtihr doch nicht viele Väter. Denn ich habe euch
gezeuget in Christo JEsu durch dasEvangelium. \bibverse{16} Darum
ermahne ich euch: Seid meine Nachfolger! \bibverse{17} Aus derselben
Ursache habe ich Timotheus zu euch gesandt, welcherist mein lieber und
getreuer Sohn in dem HErrn, daß er euch erinnere meiner Wege, dieda in
Christo sind, gleichwie ich an allen Enden in allen Gemeinden lehre.
\bibverse{18} Es blähen sich etliche auf, als würde ich nicht zu euch
kommen. \bibverse{19} Ich will aber gar kürzlich zu euch kommen, so der
HErr will, underlernen nicht die Worte der Aufgeblasenen, sondern die
Kraft. \bibverse{20} Denn das Reich GOttes stehet nicht in Worten,
sondern in Kraft. \bibverse{21} Was wollet ihr? Soll ich mit der Rute zu
euch kommen oder mit Liebeund sanftmütigem Geist?

\hypertarget{section-4}{%
\section{5}\label{section-4}}

\bibverse{1} Es gehet ein gemein Geschrei, daß Hurerei unter euch ist,
und einesolche Hurerei, da auch die Heiden nicht von zu sagen wissen,
daß einer seines VatersWeib habe. \bibverse{2} Und ihr seid aufgeblasen
und habt nicht vielmehr Leid getragen, aufdaß, der das Werk getan hat,
von euch getan würde. \bibverse{3} Ich zwar, als der ich mit dem Leibe
nicht da bin, doch mit dem Geistgegenwärtig, habe schon als gegenwärtig
beschlossen über den, den solches also getanhat: \bibverse{4} in dem
Namen unsers HErrn JEsu Christi, in eurer Versammlung mitmeinem Geist
und mit der Kraft unsers HErrn JEsu Christi. \bibverse{5} ihn zu
übergeben dem Satan zum Verderben des Fleisches, auf daß derGeist selig
werde am Tage des HErrn JEsu. \bibverse{6} Euer Ruhm ist nicht fein.
Wisset ihr nicht, daß ein wenig Sauerteig denganzen Teig versäuert?
\bibverse{7} Darum feget den alten Sauerteig aus, auf daß ihr ein neuer
Teig seid,gleichwie ihr ungesäuert seid. Denn wir haben auch ein
Osterlamm, das ist Christus, füruns geopfert. \bibverse{8} Darum lasset
uns Ostern halten, nicht im alten Sauerteig, auch nicht imSauerteig der
Bosheit und Schalkheit, sondern in dem Süßteig der Lauterkeit und
derWahrheit. \bibverse{9} Ich habe euch geschrieben in dem Briefe, daß
ihr nichts sollet zuschaffen haben mit den Hurern. \bibverse{10} Das
meine ich gar nicht von den Hurern in dieser Welt oder von denGeizigen
oder von den Räubern oder von den Abgöttischen; sonst müßtet ihr die
Welträumen. \bibverse{11} Nun aber habe ich euch geschrieben, ihr sollet
nichts mit ihnen zuschaffen haben; nämlich, so jemand ist, der sich
lässet einen Bruder nennen, und ist einHurer oder ein Geiziger oder ein
Abgöttischer oder ein Lästerer oder ein Trunkenboldoder ein Räuber, mit
demselbigen sollet ihr auch nicht essen. \bibverse{12} Denn was gehen
mich die draußen an, daß ich sie sollte richten?Richtet ihr nicht, die
da drinnen sind? \bibverse{13} GOtt aber wird, die draußen sind,
richten. Tut von euch selbst hinaus,wer da böse ist!

\hypertarget{section-5}{%
\section{6}\label{section-5}}

\bibverse{1} Wie darf jemand unter euch, so er einen Handel hat mit
einem andern,hadern vor den Ungerechten und nicht vor den Heiligen?
\bibverse{2} Wisset ihr nicht, daß die Heiligen die Welt richten werden?
So denn nundie Welt soll von euch gerichtet werden, seid ihr denn nicht
gut genug, geringere Sachenzu richten? \bibverse{3} Wisset ihr nicht,
daß wir über die Engel richten werden? wieviel mehrüber die zeitlichen
Güter! \bibverse{4} Ihr aber, wenn ihr über zeitlichen Gütern Sachen
habt, so nehmet ihrdie, so bei der Gemeinde verachtet sind, und setzet
sie zu Richtern. \bibverse{5} Euch zur Schande muß ich das sagen. Ist so
gar kein Weiser unter euchoder doch nicht einer, der da könnte richten
zwischen Bruder und Bruder? \bibverse{6} Sondern ein Bruder mit dem
andern hadert, dazu vor den Ungläubigen. \bibverse{7} Es ist schon ein
Fehl unter euch, daß ihr miteinander rechtet. Warumlasset ihr euch nicht
viel lieber unrecht tun? Warum lasset ihr euch nicht viel
lieberübervorteilen? \bibverse{8} Sondern ihr tut unrecht und
übervorteilet, und solches an den Brüdern. \bibverse{9} Wisset ihr
nicht, daß die Ungerechten werden das Reich GOttes nichtererben? Lasset
euch nicht verführen: weder die Hurer noch die Abgöttischen noch
dieEhebrecher noch die Weichlinge noch die Knabenschänder \bibverse{10}
noch die Diebe noch die Geizigen noch die Trunkenbolde noch dieLästerer
noch die Räuber werden das Reich GOttes ererben. \bibverse{11} Und
solche sind euer etliche gewesen; aber ihr seid abgewaschen, ihrseid
geheiliget, ihr seid gerecht worden durch den Namen des HErrn JEsu und
durch denGeist unsers GOttes. \bibverse{12} Ich hab' es alles Macht; es
frommet aber nicht alles. Ich hab' es allesMacht; es soll mich aber
nichts gefangennehmen. \bibverse{13} Die Speise dem Bauche und der Bauch
der Speise; aber GOtt wirddiesen und jene hinrichten. Der Leib aber
nicht der Hurerei, sondern dem HErrn und derHErr dem Leibe.
\bibverse{14} GOtt aber hat den HErrn auferwecket und wird uns auch
auferweckendurch seine Kraft. \bibverse{15} Wisset ihr nicht, daß eure
Leiber Christi Glieder sind? Sollte ich nun dieGlieder Christi nehmen
und Hurenglieder daraus machen? Das sei ferne! \bibverse{16} Oder wisset
ihr nicht, daß, wer an der Hure hanget, der ist ein Leib mitihr? Denn
sie werden (spricht er) zwei in einem Fleische sein. \bibverse{17} Wer
aber dem HErrn anhanget, der ist ein Geist mit ihm. \bibverse{18}
Fliehet die Hurerei! Alle Sünden, die der Mensch tut, sind außerseinem
Leibe; wer aber huret, der sündiget an seinem eigenen Leibe.
\bibverse{19} Oder wisset ihr nicht, daß euer Leib ein Tempel des
Heiligen Geistesist, der in euch ist, welchen ihr habt von GOtt, und
seid nicht euer selbst? \bibverse{20} Denn ihr seid teuer erkauft. Darum
so preiset GOtt an eurem Leibeund in eurem Geiste, welche sind GOttes.

\hypertarget{section-6}{%
\section{7}\label{section-6}}

\bibverse{1} Von dem ihr aber mir geschrieben habt, antworte ich: Es ist
demMenschen gut, daß er kein Weib berühre. \bibverse{2} Aber um der
Hurerei willen habe ein jeglicher sein eigen Weib, und einejegliche habe
ihren eigenen Mann. \bibverse{3} Der Mann leiste dem Weibe die schuldige
Freundschaft,desselbigengleichen das Weib dem Manne. \bibverse{4} Das
Weib ist ihres Leibes nicht mächtig, sondern der
Mann.Desselbigengleichen der Mann ist seines Leibes nicht mächtig,
sondern das Weib. \bibverse{5} Entziehe sich nicht eins dem andern, es
sei denn aus beider Bewilligungeine Zeitlang, daß ihr zum Fasten und
Beten Muße habet; und kommet wiederumzusammen; auf daß euch der Satan
nicht versuche um eurer Unkeuschheit willen. \bibverse{6} Solches sage
ich aber aus Vergunst und nicht aus Gebot. \bibverse{7} Ich wollte aber
lieber, alle Menschen wären, wie ich bin; aber einjeglicher hat seine
eigene Gabe von GOtt, einer so, der andere so. \bibverse{8} Ich sage
zwar den Ledigen und Witwen: Es ist ihnen gut, wenn sie auchbleiben wie
ich. \bibverse{9} So sie aber sich nicht enthalten, so laß sie freien;
es ist besser freien,denn Brunst leiden. \bibverse{10} Den Ehelichen
aber gebiete nicht ich, sondern der HErr, daß das Weibsich nicht scheide
von dem Manne. \bibverse{11} So sie sich aber scheidet, daß sie ohne Ehe
bleibe oder sich mit demManne versöhne, und daß der Mann das Weib nicht
von sich lasse. \bibverse{12} Den andern aber sage ich, nicht der HErr:
So ein Bruder ein ungläubigWeib hat, und dieselbige läßt es sich
gefallen, bei ihm zu wohnen, der scheide sich nichtvon ihr.
\bibverse{13} Und so ein Weib einen ungläubigen Mann hat, und er läßt es
sichgefallen, bei ihr zu wohnen, die scheide sich nicht von ihm.
\bibverse{14} Denn der ungläubige Mann ist geheiliget durch das Weib,
und dasungläubige Weib wird geheiliget durch den Mann. Sonst wären eure
Kinder unrein; nunaber sind sie heilig. \bibverse{15} So aber der
Ungläubige sich scheidet, so laß ihn sich scheiden. Es istder Bruder
oder die Schwester nicht gefangen in solchen Fällen. Im Frieden aber hat
unsGOtt berufen. \bibverse{16} Was weißt du aber, du Weib, ob du den
Mann werdest selig machen?Oder du Mann was weißt du, ob du das Weib
werdest selig machen? \bibverse{17} Doch wie einem jeglichen GOtt hat
ausgeteilet. Ein jeglicher, wie ihnder HErr berufen hat, also wandele
er. Und also schaffe ich's in allen Gemeinden. \bibverse{18} Ist jemand
beschnitten berufen, der zeuge keine Vorhaut. Ist jemandberufen in der
Vorhaut, der lasse sich nicht beschneiden. \bibverse{19} Die
Beschneidung ist nichts, und die Vorhaut ist nichts, sondernGOttes
Gebote halten; \bibverse{20} Ein jeglicher bleibe in dem Beruf, darinnen
er berufen ist. \bibverse{21} Bist du als Knecht berufen, sorge dich
nicht; doch kannst du freiwerden, so brauche des viel lieber.
\bibverse{22} Denn wer als Knecht berufen ist in dem HErrn, der ist ein
Gefreiter desHErrn; desselbigengleichen, wer als Freier berufen ist, der
ist ein Knecht Christi. \bibverse{23} Ihr seid teuer erkauft; werdet
nicht der Menschen Knechte! \bibverse{24} Ein jeglicher, liebe Brüder,
worinnen er berufen ist, darinnen bleibe erbei GOtt. \bibverse{25} Von
den Jungfrauen aber habe ich kein Gebot des HErrn; ich sage abermeine
Meinung, als ich Barmherzigkeit erlanget habe von dem HErrn, treu zu
sein. \bibverse{26} So meine ich nun, solches sei gut um der
gegenwärtigen Not willen,daß es dem Menschen gut sei, also zu sein.
\bibverse{27} Bist du an ein Weib gebunden, so suche nicht los zu
werden; bist duaber los vom Weibe, so suche kein Weib. \bibverse{28} So
du aber freiest, sündigest du nicht; und so eine Jungfrau
freiet,sündiget sie nicht; doch werden solche leibliche Trübsal haben.
Ich verschone aber euergerne. \bibverse{29} Das sage ich aber, liebe
Brüder: Die Zeit ist kurz. Weiter ist das dieMeinung: Die da Weiber
haben, daß sie seien, als hätten sie keine, und die da weinen,als
weineten sie nicht, \bibverse{30} und die sich freuen, als freueten sie
sich nicht, und die da kaufen, alsbesäßen sie es nicht, \bibverse{31}
und die diese Welt gebrauchen, daß sie dieselbige nicht mißbrauchen;denn
das Wesen dieser Welt vergehet. \bibverse{32} Ich wollte aber, daß ihr
ohne Sorge wäret. Wer ledig ist, der sorget,was dem HErrn angehöret, wie
er dem HErrn gefalle. \bibverse{33} Wer aber freiet, der sorget, was der
Welt angehöret, wie er dem Weibegefalle. Es ist ein Unterschied zwischen
einem Weibe und einer Jungfrau. \bibverse{34} Welche nicht freiet, die
sorget, was dem HErrn angehöret, daß sieheilig sei, beide, am Leibe und
auch am Geist; die aber freiet, die sorget, was der Weltangehöret, wie
sie dem Manne gefalle. \bibverse{35} Solches aber sage ich zu eurem
Nutz; nicht daß ich euch einen Strickan den Hals werfe, sondern dazu,
daß es fein ist, und ihr stets und unverhindert demHErrn dienen könnet.
\bibverse{36} So aber jemand sich lässet dünken, es wolle sich nicht
schicken mitseiner Jungfrau, weil sie eben wohl mannbar ist, und es will
nicht anders sein, so tue er,was er will; er sündiget nicht, er lasse
sie freien. \bibverse{37} Wenn einer aber sich fest vornimmt, weil er
ungezwungen ist undseinen freien Willen hat, und beschließt solches in
seinem Herzen, seine Jungfrau alsobleiben zu lassen, der tut wohl.
\bibverse{38} Endlich, welcher verheiratet, der tut wohl; welcher aber
nichtverheiratet, der tut besser. \bibverse{39} Ein Weib ist gebunden an
das Gesetz, solange ihr Mann lebet; so aberihr Mann entschläft, ist sie
frei, sich zu verheiraten, welchem sie will; allein, daß es indem HErrn
geschehe. \bibverse{40} Seliger ist sie aber, wo sie also bleibet, nach
meiner Meinung. Ichhalte aber, ich habe auch den Geist GOttes.

\hypertarget{section-7}{%
\section{8}\label{section-7}}

\bibverse{1} Von dem Götzenopfer aber wissen wir; denn wir haben alle
das Wissen.Das Wissen bläset auf; aber die Liebe bessert. \bibverse{2}
So aber sich jemand dünken lässet, er wisse etwas, der weiß nochnichts,
wie er wissen soll. \bibverse{3} So aber jemand GOtt liebet, derselbige
ist von ihm erkannt. \bibverse{4} So wissen wir nun von der Speise des
Götzenopfers, daß ein Götzenichts in der Welt sei, und daß kein anderer
GOtt sei ohne der einige. \bibverse{5} Und wiewohl es sind, die Götter
genannt werden, es sei, im Himmeloder auf Erden, sintemal es sind viel
Götter und viel Herren: \bibverse{6} so haben wir doch nur einen GOtt,
den Vater, von welchem alle Dingesind und wir in ihm, und einen HErrn,
JEsum Christum, durch welchen alle Dinge sindund wir durch ihn.
\bibverse{7} Es hat aber nicht jedermann das Wissen. Denn etliche machen
sichnoch ein Gewissen über dem Götzen und essen es für Götzenopfer;
damit wird ihrGewissen, weil es so schwach ist, beflecket. \bibverse{8}
Aber die Speise fördert uns nicht vor GOtt. Essen wir, so werden
wirdarum nicht besser sein; essen wir nicht, so werden wir darum nichts
weniger sein. \bibverse{9} Sehet aber zu, daß diese eure Freiheit nicht
gerate zu einem Anstoß derSchwachen. \bibverse{10} Denn so dich, der du
die Erkenntnis hast, jemand sähe zu Tischesitzen im Götzenhause, wird
nicht sein Gewissen dieweil er schwach ist, verursacht, dasGötzenopfer
zu essen? \bibverse{11} Und wird also über deiner Erkenntnis der
schwache Bruderumkommen, um welches willen doch Christus gestorben ist.
\bibverse{12} Wenn ihr aber also sündiget an den Brüdern und schlaget
ihrschwaches Gewissen, so sündiget ihr an Christo. \bibverse{13} Darum,
so die Speise meinen Bruder ärgert, wollte ich nimmermehrFleisch essen,
auf daß ich meinen Bruder nicht ärgerte.

\hypertarget{section-8}{%
\section{9}\label{section-8}}

\bibverse{1} Bin ich nicht ein Apostel? Bin ich nicht frei? Habe ich
nicht unsern HErrnJEsum Christum gesehen? Seid nicht ihr mein Werk in
dem HErrn? \bibverse{2} Bin ich andern nicht ein Apostel, so bin ich
doch euer Apostel; denn dasSiegel meines Apostelamts seid ihr in dem
HErrn. \bibverse{3} Wenn man mich fragt, so antworte ich also:
\bibverse{4} Haben wir nicht Macht zu essen und zu trinken? \bibverse{5}
Haben wir nicht auch Macht, eine Schwester zum Weibe mitumherzuführen
wie die andern Apostel und des HErrn Brüder und Kephas? \bibverse{6}
Oder haben allein ich und Barnaba nicht Macht, solches zu tun?
\bibverse{7} Welcher zieht jemals in den Krieg auf seinen eigenen Sold?
Welcherpflanzet einen Weinberg und isset nicht von seiner Frucht, oder
welcher weidet eineHerde und isset nicht von der Milch der Herde?
\bibverse{8} Rede ich aber solches auf Menschenweise? Sagt nicht solches
dasGesetz auch? \bibverse{9} Denn im Gesetz Mose's stehet geschrieben:
Du sollst dem Ochsen nichtdas Maul verbinden, der da drischet. Sorget
GOtt für die Ochsen? \bibverse{10} Oder sagt er's nicht allerdinge um
unsertwillen? Denn es ist ja umunsertwillen geschrieben. Denn der da
pflüget, soll auf Hoffnung pflügen, und der dadrischt, soll auf Hoffnung
dreschen, daß er seiner Hoffnung teilhaftig werde. \bibverse{11} So wir
euch das Geistliche säen, ist's ein groß Ding, ob wir euerLeibliches
ernten? \bibverse{12} So aber andere dieser Macht an euch teilhaftig
sind, warum nicht vielmehr wir? Aber wir haben solche Macht nicht
gebraucht, sondern wir vertragen allerlei,daß wir nicht dem Evangelium
Christi ein Hindernis machen. \bibverse{13} Wisset ihr nicht, daß, die
da opfern essen vom Opfer, und die desAltars pflegen, genießen des
Altars? \bibverse{14} Also hat auch der HErr befohlen daß, die das
Evangelium verkündigen;sollen sich vom Evangelium nähren. \bibverse{15}
Ich aber habe der keines gebraucht. Ich schreibe auch nicht darumdavon,
daß es mit mir also sollte gehalten werden. Es wäre mir lieber, ich
stürbe, denndaß mir jemand meinen Ruhm sollte zunichte machen.
\bibverse{16} Denn daß ich das Evangelium predige, darf ich mich nicht
rühmen;denn ich muß es tun. Und wehe mir, wenn ich das Evangelium nicht
predigte! \bibverse{17} Tue ich's gerne, so wird mir gelohnet; tue ich's
aber ungerne, so istmir das Amt doch befohlen. \bibverse{18} Was ist
denn nun mein Lohn? Nämlich daß ich predige das EvangeliumChristi und
tue dasselbige frei, umsonst, auf daß ich nicht meiner Freiheit
mißbraucheam Evangelium. \bibverse{19} Denn wiewohl ich frei bin von
jedermann, hab' ich mich doch selbstjedermann zum Knechte gemacht, auf
daß ich ihrer viel gewinne. \bibverse{20} Den Juden bin ich worden als
ein Jude, auf daß ich die Juden gewinne.Denen, die unter dem Gesetz
sind, bin ich worden als unter dem Gesetz, auf daß ich,die, so unter dem
Gesetz sind, gewinne. \bibverse{21} Denen, die ohne Gesetz sind, bin ich
als ohne Gesetz worden (so ichdoch nicht ohne Gesetz bin vor GOtt,
sondern bin in dem Gesetz Christi), auf daß ich die,so ohne Gesetz sind,
gewinne. \bibverse{22} Den Schwachen bin ich worden als ein Schwacher,
auf daß ich dieSchwachen gewinne. Ich bin jedermann allerlei worden, auf
daß ich allenthalben jaetliche selig mache. \bibverse{23} Solches aber
tue ich um des Evangeliums willen, auf daß ich seinteilhaftig werde.
\bibverse{24} Wisset ihr nicht, daß die, so in den Schranken laufen, die
laufen alle,aber einer erlanget das Kleinod? Laufet nun also, daß ihr es
ergreifet! \bibverse{25} Ein jeglicher aber, der da kämpfet, enthält
sich alles Dinges: jene also,daß sie eine vergängliche Krone empfangen,
wir aber eine unvergängliche. \bibverse{26} Ich laufe aber also, nicht
als aufs Ungewisse; ich fechte also, nicht als,der in die Luft
streichet, \bibverse{27} sondern ich betäube meinen Leib und zähme ihn,
daß ich nicht denandern predige und selbst verwerflich werde.

\hypertarget{section-9}{%
\section{10}\label{section-9}}

\bibverse{1} Ich will euch aber, liebe Brüder, nicht verhalten, daß
unsere Väter sindalle unter der Wolke gewesen und sind alle durchs Meer
gegangen \bibverse{2} und sind alle unter Mose getauft mit der Wolke und
mit dem Meer; \bibverse{3} und haben alle einerlei geistliche Speise
gegessen \bibverse{4} und haben alle einerlei geistlichen Trank
getrunken; sie tranken abervon dem geistlichen Fels, der mitfolgte,
welcher war Christus. \bibverse{5} Aber an ihrer vielen hatte GOtt kein
Wohlgefallen; denn sie sindniedergeschlagen in der Wüste. \bibverse{6}
Das ist aber uns zum Vorbilde geschehen, daß wir uns nicht
gelüstenlassen des Bösen, gleichwie jene gelüstet hat. \bibverse{7}
Werdet auch nicht Abgöttische, gleichwie jener etliche wurden,
alsgeschrieben stehet: Das Volk setzte sich nieder, zu essen und zu
trinken, und stund auf,zu spielen. \bibverse{8} Auch lasset uns nicht
Hurerei treiben, wie etliche unter jenen Hurereitrieben, und fielen auf
einen Tag dreiundzwanzigtausend. \bibverse{9} Lasset uns aber auch
Christum nicht versuchen, wie etliche von jenenihn versuchten und wurden
von, den Schlangen umgebracht. \bibverse{10} Murret auch nicht,
gleichwie jener etliche murreten und wurdenumgebracht durch den
Verderber. \bibverse{11} Solches alles widerfuhr ihnen zum Vorbilde; es
ist aber geschriebenuns zur Warnung, auf welche das Ende der Welt kommen
ist. \bibverse{12} Darum wer, sich lässet dünken, er stehe, mag wohl
zusehen, daß ernicht falle. \bibverse{13} Es hat euch noch keine denn
menschliche Versuchung betreten; aberGOtt ist getreu, der euch nicht
lässet versuchen über euer Vermögen, sondern machet,daß die Versuchung
so ein Ende gewinne, daß ihr's könnet ertragen. \bibverse{14} Darum,
meine Liebsten; fliehet von dem Götzendienst! \bibverse{15} Als mit den
Klugen rede ich; richtet ihr, was ich sage! \bibverse{16} Der gesegnete
Kelch, welchen wir segnen, ist der nicht dieGemeinschaft des Blutes
Christi? Das Brot, das wir brechen, ist das nicht dieGemeinschaft des
Leibes Christi? \bibverse{17} Denn ein Brot ist's; so sind wir viele ein
Leib, dieweil wir alle einesBrotes teilhaftig sind. \bibverse{18} Sehet
an den Israel nach dem Fleisch. Welche die Opfer essen, sinddie nicht in
der Gemeinschaft des Altars? \bibverse{19} Was soll ich denn nun sagen?
Soll ich sagen, daß der Götze etwassei, oder daß das Götzenopfer etwas
sei? \bibverse{20} Aber ich sage, daß die Heiden, was sie opfern, das
opfern sie denTeufeln und nicht GOtt. Nun will ich nicht, daß ihr in der
Teufel Gemeinschaft sein sollet. \bibverse{21} Ihr könnt nicht zugleich
trinken des HErrn Kelch und der TeufelKelch; ihr könnt nicht zugleich
teilhaftig sein des Tisches des HErrn und des Tisches derTeufel.
\bibverse{22} Oder wollen wir dem HErrn trotzen? Sind wir stärker denn
er? \bibverse{23} Ich habe es zwar alles Macht; aber es frommet nicht
alles. Ich habees alles Macht; aber es bessert nicht alles.
\bibverse{24} Niemand suche was sein ist, sondern ein jeglicher, was des
andernist. \bibverse{25} Alles was feil ist auf dem Fleischmarkt, das
esset und forschet nichts,auf daß ihr des Gewissens verschonet.
\bibverse{26} Denn die Erde ist des HErrn, und was darinnen ist.
\bibverse{27} So aber jemand von den Ungläubigen euch ladet, und ihr
wollthingehen, so esset alles, was euch vorgetragen wird, und forschet
nichts, auf daß ihr desGewissens verschonet. \bibverse{28} Wo aber
jemand würde zu euch sagen: Das ist Götzenopfer, so essetnicht, um
deswillen, der es anzeigte, auf daß ihr des Gewissens verschonet. Die
Erde istdes HErrn, und was darinnen ist. \bibverse{29} Ich sage aber vom
Gewissen nicht dein selbst, sondern des andern.Denn warum sollte ich
meine Freiheit lassen urteilen von eines andern Gewissen? \bibverse{30}
Denn so ich's mit Danksagung genieße, was sollte ich denn
verlästertwerden über dem, dafür ich danke?, \bibverse{31} Ihr esset nun
oder trinket, oder was ihr tut, so tut es alles zu GOttesEhre.
\bibverse{32} Seid nicht ärgerlich weder den Juden noch den Griechen
noch derGemeinde GOttes, \bibverse{33} gleichwie ich auch jedermann in
allerlei mich gefällig mache undsuche nicht, was mir, sondern was vielen
frommet, daß sie selig werden.

\hypertarget{section-10}{%
\section{11}\label{section-10}}

\bibverse{1} Seid meine Nachfolger, gleichwie ich Christi! \bibverse{2}
Ich lobe euch, liebe Brüder, daß ihr an mich gedenket in allen
Stückenund haltet die Weise, gleichwie ich euch gegeben habe.
\bibverse{3} Ich lasse euch aber wissen, daß Christus ist eines
jeglichen MannesHaupt, der Mann aber ist des Weibes Haupt; GOtt aber ist
Christi Haupt. \bibverse{4} Ein jeglicher Mann, der da betet oder
weissaget und hat etwas aufdem Haupt, der schändet sein Haupt.
\bibverse{5} Ein Weib aber, das da betet oder weissaget mit unbedecktem
Haupt,die schändet ihr Haupt; denn es ist ebensoviel, als wäre sie
beschoren. \bibverse{6} Will sie sich nicht bedecken, so schneide man
ihr auch das Haar ab.Nun es aber übel stehet, daß ein Weib verschnitten
Haar habe oder beschoren sei, solasset sie das Haupt bedecken.
\bibverse{7} Der Mann aber soll das Haupt nicht bedecken, sintemal er
ist GOttesBild und Ehre; das Weib aber ist des Mannes Ehre. \bibverse{8}
Denn der Mann ist nicht vom Weibe, sondern das Weib ist vom Manne.
\bibverse{9} Und der Mann ist nicht geschaffen um des Weibes willen;
sondern dasWeib um des Mannes willen. \bibverse{10} Darum soll das Weib
eine Macht auf dem Haupt haben um der Engelwillen. \bibverse{11} Doch
ist weder der Mann ohne das Weib, noch das Weib ohne denMann in dem
HErrn. \bibverse{12} Denn wie das Weib von dem Manne, also kommt auch
der Manndurch das Weib, aber alles kommt von GOtt. \bibverse{13} Richtet
bei euch selbst, ob es wohl stehet, daß ein Weib unbedecktvor GOtt bete.
\bibverse{14} Oder lehret euch auch nicht die Natur, daß einem Manne
eine Unehreist, so er lange Haare zeuget, \bibverse{15} und dem Weibe
eine Ehre, so sie lange Haare zeuget? Das Haar istihr zur Decke gegeben.
\bibverse{16} Ist aber jemand unter euch, der Lust zu zanken hat, der
wisse, daßwir solche Weise nicht haben, die Gemeinden GOttes auch nicht.
\bibverse{17} Ich muß aber dies befehlen: Ich kann's nicht loben, daß
ihr nicht aufbessere Weise, sondern auf ärgere Weise zusammenkommet.
\bibverse{18} Zum ersten, wenn ihr zusammen kommt in der Gemeinde, höre
ich,es seien Spaltungen unter euch; und zum Teil glaube ich's.
\bibverse{19} Denn es müssen Rotten unter euch sein, auf daß die,
sorechtschaffen sind, offenbar unter euch werden. \bibverse{20} Wenn ihr
nun zusammenkommet, so hält man da nicht des HErrnAbendmahl.
\bibverse{21} Denn so man das Abendmahl halten soll, nimmt ein jeglicher
seineigenes vorhin, und einer ist hungrig, der andere ist trunken.
\bibverse{22} Habt ihr aber nicht Häuser, da ihr essen und trinken
möget? Oderverachtet ihr die Gemeinde GOttes und beschämet die, so da
nichts haben? Was soll icheuch sagen? Soll ich euch loben? Hierinnen
lobe ich euch nicht. \bibverse{23} Ich habe von dem HErrn empfangen, das
ich euch gegeben habe.Denn der HErr JEsus in der Nacht, da er verraten
ward, nahm er das Brot, \bibverse{24} dankete und brach's und sprach:
Nehmet, esset; das ist mein Leibder für euch gebrochen wird. Solches tut
zu meinem Gedächtnis! \bibverse{25} Desselbigengleichen auch den Kelch
nach dem Abendmahl undsprach: Dieser Kelch ist das neue Testament in
meinem Blut. Solches tut, so oft ihr'strinket, zu meinem Gedächtnis!
\bibverse{26} Denn so oft ihr von diesem Brot esset und von diesem Kelch
trinket,sollt ihr des HErrn Tod verkündigen, bis daß er kommt.
\bibverse{27} Welcher nun unwürdig von diesem Brot isset oder von dem
Kelch desHErrn trinket, der ist schuldig an dem Leib und Blut des HErrn.
\bibverse{28} Der Mensch prüfe aber sich selbst und also esse er von
diesem Brotund trinke von diesem Kelch. \bibverse{29} Denn welcher
unwürdig isset und trinket, der isset und trinket ihmselber das Gericht
damit, daß er nicht unterscheidet den Leib des HErrn. \bibverse{30}
Darum sind auch so viel Schwache und Kranke unter euch, und eingut Teil
schlafen. \bibverse{31} Denn so wir uns selber richteten, so würden wir
nicht gerichtet. \bibverse{32} Wenn wir aber gerichtet werden, so werden
wir von dem HErrngezüchtiget, auf daß wir nicht samt der Welt verdammet
werden. \bibverse{33} Darum, meine lieben Brüder, wenn ihr
zusammenkommet, zu essen,so harre einer des andern. \bibverse{34}
Hungert aber jemand, der esse daheim, auf daß ihr nicht zumGerichte
zusammenkommet. Das andere will ich ordnen, wenn ich komme.

\hypertarget{section-11}{%
\section{12}\label{section-11}}

\bibverse{1} Von den geistlichen Gaben aber will ich euch, liebe Brüder,
nichtverhalten. \bibverse{2} Ihr wisset, daß ihr Heiden seid gewesen und
hingegangen zu denstummen Götzen, wie ihr geführt wurdet. \bibverse{3}
Darum tue ich euch kund, daß niemand JEsum verfluchet, der durchden
Geist GOttes redet; und niemand kann JEsum einen HErrn heißen ohne durch
denHeiligen Geist. \bibverse{4} Es sind mancherlei Gaben, aber es ist
ein Geist. \bibverse{5} Und es sind mancherlei Ämter, aber es ist ein
HErr. \bibverse{6} Und es sind mancherlei Kräfte, aber es ist ein GOtt,
der da wirket allesin allen. \bibverse{7} In einem jeglichen erzeigen
sich die Gaben des Geistes zum gemeinenNutzen. \bibverse{8} Einem wird
gegeben durch den Geist, zu reden von der Weisheit; demandern wird
gegeben, zu reden von der Erkenntnis nach demselbigen Geist;
\bibverse{9} einem andern der Glaube in demselbigen Geist; einem andern
dieGabe, gesund zu machen, in demselbigen Geist; \bibverse{10} einem
andern, Wunder zu tun; einem andern Weissagung; einemandern, Geister zu
unterscheiden; einem andern mancherlei Sprachen; einem andern,die
Sprachen auszulegen. \bibverse{11} Dies aber alles wirket derselbige
einige Geist und teilet einemjeglichen seines zu, nachdem er will.
\bibverse{12} Denn gleichwie ein Leib ist und hat doch viel Glieder,
alle Gliederaber eines Leibes, wiewohl ihrer viel sind, sind sie doch
ein Leib: also auch Christus. \bibverse{13} Denn wir sind durch einen
Geist alle zu einem Leibe getauft, wir seienJuden oder Griechen, Knechte
oder Freie, und sind alle zu einem Geist getränket. \bibverse{14} Denn
auch der Leib ist nicht ein Glied, sondern viele. \bibverse{15} So aber
der Fuß spräche: Ich bin keine Hand, darum bin ich desLeibes Glied
nicht, sollte er um deswillen nicht des Leibes Glied sein? \bibverse{16}
Und so das Ohr spräche: Ich bin kein Auge, darum bin ich nicht desLeibes
Glied, sollte es um deswillen nicht des Leibes Glied sein? \bibverse{17}
Wenn der ganze Leib Auge wäre, wo bliebe das Gehör? So er ganzGehör
wäre, wo bliebe der Geruch? \bibverse{18} Nun aber hat GOtt die Glieder
gesetzt, ein jegliches sonderlich amLeibe, wie er gewollt hat.
\bibverse{19} So aber alle Glieder ein Glied wären, wo bliebe der Leib?
\bibverse{20} Nun aber sind der Glieder viele, aber der Leib ist einer.
\bibverse{21} Es kann das Auge nicht sagen zu der Hand: Ich bedarf dein
nicht;oder wiederum das Haupt zu den Füßen: Ich bedarf euer nicht;
\bibverse{22} sondern vielmehr, die Glieder des Leibes, die uns dünken,
dieschwächsten zu sein, sind die nötigsten, \bibverse{23} und die uns
dünken, die unehrlichsten sein, denselbigen legen wir ammeisten Ehre an,
und die uns übel anstehen, die schmücket man am meisten. \bibverse{24}
Denn die uns wohl anstehen, die bedürfen's nicht. Aber GOtt hat denLeib
also vermenget und dem dürftigen Glied am meisten Ehre gegeben,
\bibverse{25} auf daß nicht eine Spaltung im Leibe sei, sondern die
Gliederfüreinander gleich sorgen. \bibverse{26} Und so ein Glied leidet,
so leiden alle Glieder mit; und so ein Gliedwird herrlich gehalten, so
freuen sich alle Glieder, mit. \bibverse{27} Ihr seid aber der Leib
Christi und Glieder, ein jeglicher nach seinemTeil. \bibverse{28} Und
GOtt hat gesetzt in der Gemeinde aufs erste die Apostel, aufsandere die
Propheten, aufs dritte die Lehrer, danach die Wundertäter, danach die
Gaben,gesund zu machen, Helfer, Regierer, mancherlei Sprachen.
\bibverse{29} Sind sie alle Apostel? Sind sie alle Propheten? Sind sie
alle Lehrer?Sind sie alle Wundertäter? \bibverse{30} Haben sie alle
Gaben gesund zu machen? Reden sie alle mitmancherlei Sprachen? Können
sie alle auslegen? \bibverse{31} Strebet aber nach den besten Gaben! Und
ich will euch noch einenköstlichern Weg zeigen.

\hypertarget{section-12}{%
\section{13}\label{section-12}}

\bibverse{1} Wenn ich mit Menschen- und mit Engelzungen redete und hätte
derLiebe nicht, so wäre ich ein tönend Erz oder eine klingende Schelle.
\bibverse{2} Und wenn ich weissagen könnte und wüßte alle Geheimnisse
und alleErkenntnis und hätte allen Glauben, also daß ich Berge
versetzte, und hätte der Liebenicht, so wäre ich nichts. \bibverse{3}
Und wenn ich alle meine Habe den Armen gäbe und ließe meinen Leibbrennen
und hätte der Liebe nicht, so wäre mir's nichts nütze. \bibverse{4} Die
Liebe ist langmütig und freundlich; die Liebe eifert nicht; die
Liebetreibt nicht Mutwillen; sie blähet sich nicht; \bibverse{5} sie
stellet sich nicht ungebärdig; sie suchet nicht das Ihre; sie lässetsich
nicht erbittern; sie trachtet nicht nach Schaden; \bibverse{6} sie
freuet sich nicht der Ungerechtigkeit; sie freuet sich aber derWahrheit;
\bibverse{7} sie verträget alles, sie glaubet alles, sie hoffet alles,
sie duldet alles. \bibverse{8} Die Liebe höret nimmer auf, so doch die
Weissagungen aufhörenwerden, und die Sprachen aufhören werden, und die
Erkenntnis aufhören wird. \bibverse{9} Denn unser Wissen ist Stückwerk,
und unser Weissagen ist Stückwerk. \bibverse{10} Wenn aber kommen wird
das Vollkommene, so wird das Stückwerkaufhören. \bibverse{11} Da ich ein
Kind war, da redete ich wie ein Kind und war klug wie einKind und hatte
kindische Anschläge; da ich aber ein Mann ward, tat ich ab, was
kindischwar. \bibverse{12} Wir sehen jetzt durch einen Spiegel in einem
dunklen Wort, dannaber von Angesicht zu Angesichte. Jetzt erkenne ich's
stückweise; dann aber werde icherkennen, gleichwie ich erkannt bin.
\bibverse{13} Nun aber bleibt Glaube, Hoffnung, Liebe, diese drei; aber
die Liebe istdie größte unter ihnen.

\hypertarget{section-13}{%
\section{14}\label{section-13}}

\bibverse{1} Strebet nach der Liebe! Fleißiget euch der geistlichen
Gaben, ammeisten aber, daß ihr weissagen möget. \bibverse{2} Denn der
mit der Zunge redet, der redet nicht den Menschen, sondernGOtt. Denn ihm
höret niemand zu; im Geist aber redet er die Geheimnisse. \bibverse{3}
Wer aber weissaget, der redet den Menschen zur Besserung und
zurErmahnung und zur Tröstung. \bibverse{4} Wer mit Zungen redet, der
bessert sich selbst; wer aber weissaget,der bessert die Gemeinde.
\bibverse{5} Ich wollte, daß ihr alle mit Zungen reden könntet, aber
viel mehr, daßihr weissagetet. Denn der da weissaget, ist größer, denn
der mit Zungen redet, es seidenn, daß er es auch auslege, daß die
Gemeinde davon gebessert werde. \bibverse{6} Nun aber, liebe Brüder,
wenn ich zu euch käme und redete mitZungen, was wäre ich euch nütze, so
ich nicht mit euch redete entweder durchOffenbarung oder durch
Erkenntnis oder durch Weissagung oder durch Lehre? \bibverse{7} Hält
sich's doch auch also in den Dingen, die da lauten und doch nichtleben,
es sei eine Pfeife oder eine Harfe; wenn sie nicht unterschiedliche
Stimmen vonsich geben, wie kann man wissen, was gepfiffen oder geharfet
ist? \bibverse{8} Und so die Posaune einen undeutlichen Ton gibt, wer
will sich zumStreit rüsten? \bibverse{9} Also auch ihr, wenn ihr mit
Zungen redet, so ihr nicht eine deutlicheRede gebet, wie kann man
wissen, was geredet ist? Denn ihr werdet in den Wind reden.
\bibverse{10} Zwar es ist mancherlei Art der Stimmen in der Welt, und
derselbigenist doch keine undeutlich. \bibverse{11} So ich nun nicht
weiß der Stimme Deutung, werde ich undeutsch seindem, der da redet, und
der da redet, wird mir undeutsch sein. \bibverse{12} Also auch ihr,
sintemal ihr euch fleißiget der geistlichen Gaben,trachtet danach, daß
ihr die Gemeinde bessert, auf daß ihr alles reichlich habet.
\bibverse{13} Darum, welcher mit Zungen redet, der bete also, daß er's
auchauslege. \bibverse{14} So ich aber mit Zungen bete, so betet mein
Geist; aber mein Sinnbringet niemand Frucht. \bibverse{15} Wie soll es
aber denn sein? Nämlich also: Ich will beten mit demGeist und will beten
auch im Sinn; ich will Psalmen singen im Geist und will auchPsalmen
singen mit dem Sinn. \bibverse{16} Wenn du aber segnest im Geist, wie
soll der, so anstatt des Laienstehet, Amen sagen auf deine Danksagung,
sintemal er nicht weiß, was du sagest? \bibverse{17} Du danksagest wohl
fein; aber der andere wird nicht davongebessert. \bibverse{18} Ich danke
meinem GOtt, daß ich mehr mit Zungen rede denn ihr alle. \bibverse{19}
Aber ich will in der Gemeinde lieber fünf Worte reden mit meinemSinn,
auf daß ich auch andere unterweise, denn sonst zehntausend Worte mit
Zungen. \bibverse{20} Liebe Brüder, werdet nicht Kinder an dem
Verständnis, sondern ander Bosheit seid Kinder; an dem Verständnis aber
seid vollkommen. \bibverse{21} Im Gesetz stehet geschrieben: Ich will
mit andern Zungen und mitandern Lippen reden zu diesem Volk, und sie
werden mich auch also nicht hören, sprichtder HErr. \bibverse{22} Darum
so sind die Zungen zum Zeichen, nicht den Gläubigen,sondern den
Ungläubigen; die Weissagung aber nicht den Ungläubigen, sondern
denGläubigen. \bibverse{23} Wenn nun die ganze Gemeinde zusammenkäme an
einem Ort undredeten alle mit Zungen, es kämen aber hinein Laien oder
Ungläubige, würden sie nichtsagen, ihr wäret unsinnig? \bibverse{24} So
sie aber alle weissageten und käme dann ein Ungläubiger oderLaie hinein,
der würde von denselbigen allen gestraft und von allen gerichtet.
\bibverse{25} Und also würde das Verborgene seines Herzens offenbar, und
erwürde also fallen auf sein Angesicht, GOtt anbeten und bekennen, daß
GOtt wahrhaftigin euch sei. \bibverse{26} Wie ist ihm denn nun, liebe
Brüder? Wenn ihr zusammenkommt, sohat ein jeglicher Psalmen, er hat eine
Lehre, er hat Zungen, er hat Offenbarung, er hatAuslegung. Lasset es
alles geschehen zur Besserung! \bibverse{27} So jemand mit der Zunge
redet oder zween oder aufs meiste' drei,eins ums andere; so lege es
einer aus. \bibverse{28} Ist er aber nicht ein Ausleger, so schweige er
unter der Gemeinde,rede aber sich selber, und GOtt, \bibverse{29} Die
Weissager aber lasset reden, zween oder drei, und die andernlasset
richten. \bibverse{30} So aber eine Offenbarung geschieht einem andern,
der da sitzet, soschweige der erste. \bibverse{31} Ihr könnet wohl alle
weissagen, einer nach dem andern, auf daß siealle lernen und alle
ermahnet werden. \bibverse{32} Und die Geister der Propheten sind den
Propheten untertan. \bibverse{33} Denn GOtt ist, nicht ein GOtt der
Unordnung, sondern des Friedenswie in allen Gemeinden der Heiligen.
\bibverse{34} Eure Weiber lasset schweigen unter der Gemeinde; denn es
sollihnen nicht zugelassen werden, daß sie reden, sondern untertan sein,
wie auch dasGesetz sagt. \bibverse{35} Wollen sie aber etwas lernen, so
lasset sie daheim ihre Männerfragen. Es stehet den Weibern übel an,
unter der Gemeinde reden. \bibverse{36} Oder ist das Wort GOttes von
euch auskommen, oder ist's allein zueuch kommen? \bibverse{37} So sich
jemand lässet dünken, er sei ein Prophet oder geistlich, dererkenne, was
ich euch schreibe; denn es sind des HErrn Gebote. \bibverse{38} Ist aber
jemand unwissend, der sei, unwissend. \bibverse{39} Darum, liebe Brüder,
fleißiget euch des Weissagens und wehret nicht,mit Zungen zu reden.
\bibverse{40} Lasset alles ehrlich und ordentlich zugehen!

\hypertarget{section-14}{%
\section{15}\label{section-14}}

\bibverse{1} Ich erinnere euch, aber, liebe Brüder, des Evangeliums, das
ich euchverkündiget habe, welches ihr auch angenommen habt, in welchem
ihr auch stehet, \bibverse{2} durch welches ihr auch selig werdet,
welcher Gestalt ich es euchverkündiget habe, so ihr's behalten habt, es
wäre, denn, daß ihr's umsonst geglaubethättet. \bibverse{3} Denn ich
habe euch zuvörderst gegeben, welches ich auch empfangenhabe, daß
Christus gestorben sei für unsere Sünden nach der Schrift, \bibverse{4}
und daß er begraben sei, und daß er auferstanden sei am dritten Tagenach
der Schrift, \bibverse{5} und daß er gesehen worden ist von Kephas,
danach von den Zwölfen. \bibverse{6} Danach ist er gesehen worden von
mehr denn fünfhundert Brüdern aufeinmal, deren noch viel leben, etliche
aber sind entschlafen. \bibverse{7} Danach ist er gesehen worden von
Jakobus, danach von allenAposteln. \bibverse{8} Am letzten nach allen
ist er auch von mir, als einer unzeitigen Geburt,gesehen worden;
\bibverse{9} denn ich bin der geringste unter den Aposteln, als der ich
nicht wertbin, daß ich ein Apostel heiße, darum daß ich die Gemeinde
GOttes verfolget habe. \bibverse{10} Aber von GOttes Gnaden bin ich, das
ich bin, und seine Gnade an mirist nicht vergeblich gewesen, sondern ich
habe viel mehr gearbeitet denn sie alle, nichtaber ich, sondern GOttes
Gnade, die mit mir ist. \bibverse{11} Es sei nun ich oder jene, also
predigen wir, und also habt ihrgeglaubet. \bibverse{12} So aber Christus
geprediget wird, daß er sei von den Totenauferstanden, wie sagen denn
etliche unter euch, die Auferstehung der Toten sei nichts ?
\bibverse{13} Ist aber die Auferstehung der Toten nichts, so ist auch
Christus nichtauferstanden. \bibverse{14} Ist aber Christus nicht
auferstanden, so ist unsere Predigt vergeblich,so ist auch euer Glaube
vergeblich. \bibverse{15} Wir würden aber auch erfunden falsche Zeugen
GOttes, daß wir widerGOtt gezeuget hätten, er hätte Christum
auferwecket, den er nicht auferwecket hätte,sintemal die Toten nicht
auferstehen. \bibverse{16} Denn so die Toten nicht auferstehen, so ist
Christus auch nichtauferstanden. \bibverse{17} Ist Christus aber nicht
auferstanden, so ist euer Glaube eitel, so seidihr noch in euren Sünden,
\bibverse{18} so sind auch die, so in Christo entschlafen sind,
verloren. \bibverse{19} Hoffen wir allein in diesem Leben auf Christum,
so sind wir dieelendesten unter allen Menschen. \bibverse{20} Nun aber
ist Christus auferstanden von den Toten und der Erstlingworden unter
denen, die da schlafen, \bibverse{21} sintemal durch einen Menschen der
Tod und durch einen Menschendie Auferstehung der Toten kommt.
\bibverse{22} Denn gleichwie sie in Adam alle sterben, also werden sie
in Christoalle lebendig gemacht werden. \bibverse{23} Ein jeglicher aber
in seinerOrdnung. Der Erstling Christus, danach dieChristo angehören,
wenn er kommen wird. \bibverse{24} Danach das Ende, wenn er das Reich
GOtt und dem Vaterüberantworten wird, wenn er aufheben wird alle
Herrschaft und alle Obrigkeit undGewalt. \bibverse{25} Er muß aber
herrschen, bis daß er alle seine Feinde unter seine Füßelege.
\bibverse{26} Der letzte Feind, der aufgehoben wird, ist der Tod.
\bibverse{27} Denn er hat ihm alles unter seine Füße getan. Wenn er aber
sagt;daß es alles untertan sei, ist's offenbar, daß ausgenommen ist, der
ihm alles untertanhat. \bibverse{28} Wenn aber alles ihm untertan sein
wird, alsdann wird auch der Sohnselbst untertan sein dem, der ihm alles
untertan hat, auf daß GOtt sei alles in allen. \bibverse{29} Was machen
sonst, die sich taufen lassen über den Toten, soallerdinge die Toten
nicht auferstehen? Was lassen sie sich taufen über den Toten?
\bibverse{30} Und was stehen wir alle Stunde in der Gefahr?
\bibverse{31} Bei unserm Ruhm den ich habe in Christo JEsu, unserm
HErrn, ichsterbe täglich. \bibverse{32} Hab' ich menschlicher Meinung zu
Ephesus mit den wilden Tierengefochten, was hilft's mir, so die Toten
nicht auferstehen? Lasset uns essen und trinken;denn morgen sind wir
tot. \bibverse{33} Lasset euch nicht verführen! Böse Geschwätze
verderben gute Sitten. \bibverse{34} Werdet doch einmal recht nüchtern
und sündiget nicht; denn etlichewissen nichts von GOtt, das sage ich
euch zur Schande. \bibverse{35} Möchte aber jemand, sagen: Wie werden
die Toten auferstehen; undmit welcherlei Leibe werden sie kommen?
\bibverse{36} Du Narr, was du säest, wird nicht lebendig, es sterbe
denn. \bibverse{37} Und was du säest, ist ja nicht der Leib, der werden
soll, sondern einbloßes Korn, nämlich Weizen oder der andern eines.
\bibverse{38} GOtt aber gibt ihm einen Leib, wie er will, und einem
jeglichen vonden Samen seinen eigenen Leib. \bibverse{39} Nicht ist
alles Fleisch einerlei Fleisch, sondern ein ander Fleisch istder
Menschen, ein anderes des Viehes, ein anderes der Fische, ein anderes
der Vögel. \bibverse{40} Und es sind himmlische Körper und irdische
Körper. Aber eine andereHerrlichkeit haben die himmlischen und eine
andere die irdischen. \bibverse{41} Eine andere Klarheit hat die Sonne,
eine andere Klarheit hat derMond, eine andere Klarheit haben die Sterne;
denn ein Stern übertrifft den andern anKlarheit. \bibverse{42} Also auch
die Auferstehung der Toten. Es wird gesäet verweslich undwird
auferstehen unverweslich. \bibverse{43} Es wird gesäet in Unehre und
wird auferstehen in Herrlichkeit. Eswird gesäet in Schwachheit und wird
auferstehen in Kraft. \bibverse{44} Es wird gesäet ein natürlicher Leib,
und wird auferstehen eingeistlicher Leib. Hat man einen natürlichen
Leib, so hat man auch einen geistlichen Leib, \bibverse{45} wie es
geschrieben stehet: Der erste Mensch, Adam, ist gemacht insnatürliche
Leben und der letzte Adam ins geistliche Leben. \bibverse{46} Aber der
geistliche Leib ist nicht erste, sondern der natürliche,danach der
geistliche. \bibverse{47} Der erste Mensch ist von der Erde und irdisch;
der andere Mensch istder HErr vom Himmel. \bibverse{48} Welcherlei der
irdische ist, solcherlei sind auch die irdischen; undwelcherlei der
himmlische ist, solcherlei sind auch die himmlischen. \bibverse{49} Und
wie wir getragen haben das Bild des irdischen, also werden wirauch
tragen das Bild des himmlischen. \bibverse{50} Davon sage ich aber,
liebe Brüder, daß Fleisch und Blut nicht könnendas Reich GOttes ererben;
auch wird das Verwesliche nicht erben das Unverwesliche, \bibverse{51}
Siehe, ich sage euch ein Geheimnis: Wir werden nicht alleentschlafen wir
werden aber alle verwandelt werden, \bibverse{52} und dasselbige
plötzlich, in einem Augenblick, zu der Zeit der letztenPosaune. Denn es
wird die Posaune schallen und die Toten werden auferstehenunverweslich,
und wir werden verwandelt werden. \bibverse{53} Denn dies Verwesliche
muß anziehen das Unverwesliche, und diesSterbliche muß anziehen die
Unsterblichkeit. \bibverse{54} Wenn aber dies Verwesliche wird anziehen
das Unverwesliche, unddies Sterbliche wird anziehen die Unsterblichkeit,
dann wird erfüllet werden das Wort,das geschrieben stehet: \bibverse{55}
Der Tod ist verschlungen in den Sieg. Tod, wo ist dein Stachel? Hölle,wo
ist dein Sieg? \bibverse{56} Aber der Stachel des Todes ist die Sünde;
die Kraft aber der Sündeist das Gesetz. \bibverse{57} GOtt aber sei
Dank, der uns den Sieg gegeben hat durch unsernHErrn JEsum Christum!
\bibverse{58} Darum, meine lieben Brüder, seid fest, unbeweglich und
nehmetimmer zu in dem Werk des HErrn, sintemal ihr wisset, daß eure
Arbeit nicht vergeblichist in dem HErrn.

\hypertarget{section-15}{%
\section{16}\label{section-15}}

\bibverse{1} Von der Steuer aber, die den Heiligen geschieht, wie ich
denGemeinden in Galatien geordnet habe, also tut auch ihr. \bibverse{2}
Auf je der Sabbate einen lege bei sich selbst ein jeglicher unter
euchund sammle, was ihn gut dünkt, auf daß nicht, wenn ich komme, dann
allererst dieSteuer zu sammeln sei. \bibverse{3} Wenn ich aber darkommen
bin, welche ihr durch Briefe dafür ansehet,die will ich senden, daß sie
hinbringen eure Wohltat gen Jerusalem. \bibverse{4} So es aber wert ist,
daß ich auch hinreise, sollen sie mit mir reisen. \bibverse{5} Ich will
aber zu euch kommen, wenn ich durch Mazedonien ziehe;denn durch
Mazedonien werde ich ziehen. \bibverse{6} Bei euch aber werde ich
vielleicht bleiben oder auch wintern, auf daßihr mich geleitet, wo ich
hinziehen werde. \bibverse{7} Ich will euch jetzt nicht sehen im
Vorüberziehen; denn ich hoffe, ichwolle etliche Zeit bei euch bleiben,
so es der HErr zuläßt. \bibverse{8} Ich werde aber zu Ephesus bleiben
bis Pfingsten. \bibverse{9} Denn mir ist eine große Tür aufgetan, die
viele Frucht wirket, und sindviel Widerwärtige da. \bibverse{10} So
Timotheus kommt, so sehet zu, daß er ohne Furcht bei euch sei;denn er
treibet auch das Werk des HErrn wie ich. \bibverse{11} Daß ihn nun nicht
jemand verachte! Geleitet ihn aber im Frieden, daßer zu mir komme; denn
ich warte sein mit den Brüdern. \bibverse{12} Von Apollos, dem Bruder,
aber wisset, daß ich ihn sehr viel ermahnethabe, daß er zu euch käme mit
den Brüdern, und es war allerdinge sein Wille nicht, daßer jetzt käme;
er wird aber kommen, wenn es ihm gelegen sein wird. \bibverse{13}
Wachet, stehet im Glauben, seid männlich und seid stark! \bibverse{14}
Alle eure Dinge lasset in der Liebe geschehen. \bibverse{15} Ich ermahne
euch aber, liebe Brüder, ihr kennet das HausStephanas, daß sie sind die
Erstlinge in Achaja und haben sich selbst verordnet zumDienst den
Heiligen, \bibverse{16} auf daß auch ihr solchen untertan seiet und
allen, die mitwirken undarbeiten. \bibverse{17} Ich freue mich über die
Zukunft Stephanas und Fortunatus undAchaicus; denn wo ich euer Mangel
hatte, das haben sie erstattet. \bibverse{18} Sie haben erquicket meinen
und euren Geist. Erkennet, die solchesind! \bibverse{19} Es grüßen euch
die Gemeinden in Asien. Es grüßen euch sehr in demHErrn Aquila und
Priscilla samt der Gemeinde in ihrem Hause. \bibverse{20} Es grüßen euch
alle Brüder. Grüßet euch untereinander mit demheiligen Kuß.
\bibverse{21} Ich, Paulus, grüße euch mit meiner Hand. \bibverse{22} So
jemand den HErrn JEsum Christum nicht liebhat, der seiAnathema, Maharam
Motha. \bibverse{23} Die Gnade des HErrn JEsu Christi sei mit euch!
\bibverse{24} Meine Liebe sei mit euch allen in Christo JEsu! Amen.
