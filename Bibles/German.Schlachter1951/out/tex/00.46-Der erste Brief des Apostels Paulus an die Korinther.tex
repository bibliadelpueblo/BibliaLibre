\hypertarget{section}{%
\section{1}\label{section}}

\bibleverse{1} Paulus, berufener Apostel Jesu Christi durch Gottes
Willen, und Sosthenes, der Bruder, \bibleverse{2} an die Gemeinde
Gottes, die in Korinth ist, an die Geheiligten in Christus Jesus, an die
berufenen Heiligen, samt allen, die den Namen unsres Herrn Jesus
Christus anrufen an jedem Ort, bei ihnen und bei uns. \bibleverse{3}
Gnade sei mit euch und Friede von Gott, unsrem Vater und dem Herrn Jesus
Christus! \bibleverse{4} Ich danke meinem Gott allezeit eurethalben für
die Gnade Gottes, die euch in Christus Jesus gegeben ist, \bibleverse{5}
daß ihr an allem reich gemacht worden seid in ihm, an aller Lehre und an
aller Erkenntnis, \bibleverse{6} wie denn das Zeugnis von Christus unter
euch befestigt worden ist, \bibleverse{7} so daß ihr keinen Mangel habt
an irgend einer Gnadengabe, während ihr die Offenbarung unsres Herrn
Jesus Christus erwartet, \bibleverse{8} welcher euch auch bis ans Ende
befestigen wird, so daß ihr unverklagbar seid am Tage unsres Herrn Jesus
Christus. \bibleverse{9} Treu ist Gott, durch welchen ihr berufen seid
zur Gemeinschaft seines Sohnes Jesus Christus unsres Herrn.
\bibleverse{10} Ich ermahne euch aber, ihr Brüder, kraft des Namens
unsres Herrn Jesus Christus, daß ihr alle einerlei Rede führet und nicht
Spaltungen unter euch sein lasset, sondern zusammenhaltet in derselben
Gesinnung und in derselben Meinung. \bibleverse{11} Mir ist nämlich,
meine Brüder, durch die Leute der Chloe bekanntgeworden, daß
Zwistigkeiten unter euch sind. \bibleverse{12} Ich rede aber davon, daß
unter euch der eine spricht: Ich halte zu Paulus; der andere: Ich zu
Apollos; der dritte: Ich zu Kephas; der vierte: Ich zu Christus!
\bibleverse{13} Ist Christus zerteilt? Ist etwa Paulus für euch
gekreuzigt worden, oder seid ihr auf des Paulus Namen getauft?
\bibleverse{14} Ich danke Gott, daß ich niemand von euch getauft habe,
außer Krispus und Gajus; \bibleverse{15} so kann doch niemand sagen, ihr
seiet auf meinen Namen getauft! \bibleverse{16} Ich habe aber auch das
Haus des Stephanas getauft. Sonst weiß ich nicht, ob ich noch jemand
getauft habe; \bibleverse{17} denn Christus hat mich nicht gesandt zu
taufen, sondern das Evangelium zu verkündigen, nicht in Redeweisheit,
damit nicht das Kreuz Christi entkräftet werde. \bibleverse{18} Denn das
Wort vom Kreuz ist eine Torheit denen, die verloren gehen; uns aber, die
wir gerettet werden, ist es eine Gotteskraft, \bibleverse{19} denn es
steht geschrieben: ``Ich will zunichte machen die Weisheit der Weisen,
und den Verstand der Verständigen will ich verwerfen.'' \bibleverse{20}
Wo ist der Weise, wo der Schriftgelehrte, wo der Disputiergeist dieser
Welt? Hat nicht Gott die Weisheit dieser Welt zur Torheit gemacht?
\bibleverse{21} Denn weil die Welt durch ihre Weisheit Gott in seiner
Weisheit nicht erkannte, gefiel es Gott, durch die Torheit der Predigt
diejenigen zu retten, welche glauben. \bibleverse{22} Während nämlich
die Juden Zeichen fordern und die Griechen Weisheit verlangen,
\bibleverse{23} predigen wir Christus den Gekreuzigten, den Juden ein
Ärgernis, den Griechen eine Torheit; \bibleverse{24} jenen, den
Berufenen aber, sowohl Juden als Griechen, predigen wir Christus, Gottes
Kraft und Gottes Weisheit. \bibleverse{25} Denn Gottes ``Torheit'' ist
weiser als die Menschen sind, und Gottes ``Schwachheit'' ist stärker als
die Menschen sind. \bibleverse{26} Sehet doch eure Berufung an, ihr
Brüder! Da sind nicht viele Weise nach dem Fleisch, nicht viele
Mächtige, nicht viel Adelige; \bibleverse{27} sondern das Törichte der
Welt hat Gott auserwählt, um die Weisen zuschanden zu machen, und das
Schwache der Welt hat Gott erwählt, um das Starke zuschanden zu machen,
\bibleverse{28} und das Unedle der Welt und das Verachtete hat Gott
erwählt und das, was nichts ist, damit er zunichte mache, was etwas ist;
\bibleverse{29} auf daß sich vor Gott kein Fleisch rühme.
\bibleverse{30} Durch ihn aber seid ihr in Christus Jesus, welcher uns
von Gott gemacht worden ist zur Weisheit, zur Gerechtigkeit, zur
Heiligung und zur Erlösung, \bibleverse{31} auf daß, wie geschrieben
steht: ``Wer sich rühmt, der rühme sich im Herrn!''

\hypertarget{section-1}{%
\section{2}\label{section-1}}

\bibleverse{1} So bin auch ich, meine Brüder, als ich zu euch kam, nicht
gekommen, um euch in hervorragender Rede oder Weisheit das Zeugnis
Gottes zu verkündigen. \bibleverse{2} Denn ich hatte mir vorgenommen,
unter euch nichts anderes zu wissen, als nur Jesus Christus, und zwar
als Gekreuzigten. \bibleverse{3} Und ich war in Schwachheit und mit viel
Furcht und Zittern unter euch. \bibleverse{4} Und meine Rede und meine
Predigt bestand nicht in überredenden Worten menschlicher Weisheit,
sondern in Beweisung des Geistes und der Kraft, \bibleverse{5} auf daß
euer Glaube nicht auf Menschenweisheit beruhe, sondern auf Gotteskraft.
\bibleverse{6} Wir reden allerdings Weisheit, unter den Gereiften; aber
keine Weisheit dieser Welt, auch nicht der Obersten dieser Welt, welche
vergehen. \bibleverse{7} Sondern wir reden Gottes Weisheit im Geheimnis,
die verborgene, welche Gott vor den Weltzeiten zu unserer Herrlichkeit
vorherbestimmt hat, \bibleverse{8} welche keiner der Obersten dieser
Welt erkannt hat; denn hätten sie sie erkannt, so würden sie den Herrn
der Herrlichkeit nicht gekreuzigt haben. \bibleverse{9} Sondern, wie
geschrieben steht: ``Was kein Auge gesehen und kein Ohr gehört und
keinem Menschen in den Sinn gekommen ist, was Gott denen bereitet hat,
die ihn lieben'', \bibleverse{10} hat Gott uns aber geoffenbart durch
seinen Geist; denn der Geist erforscht alles, auch die Tiefen der
Gottheit. \bibleverse{11} Denn welcher Mensch weiß, was im Menschen ist,
als nur der Geist des Menschen, der in ihm ist? So weiß auch niemand,
was in Gott ist, als nur der Geist Gottes. \bibleverse{12} Wir aber
haben nicht den Geist der Welt empfangen, sondern den Geist aus Gott, so
daß wir wissen können, was uns von Gott gegeben ist; \bibleverse{13} und
davon reden wir auch, nicht in Worten, die von menschlicher Weisheit
gelehrt sind, sondern in solchen, die vom Geist gelehrt sind, indem wir
Geistliches geistlich beurteilen. \bibleverse{14} Der seelische Mensch
aber nimmt nicht an, was vom Geiste Gottes ist; denn es ist ihm eine
Torheit, und er kann es nicht verstehen, weil es geistlich beurteilt
werden muß. \bibleverse{15} Der geistliche Mensch aber erforscht alles,
er selbst jedoch wird von niemand erforscht; \bibleverse{16} denn wer
hat des Herrn Sinn erkannt, daß er ihn belehre? Wir aber haben Christi
Sinn.

\hypertarget{section-2}{%
\section{3}\label{section-2}}

\bibleverse{1} Und ich, meine Brüder, konnte nicht mit euch reden als
mit geistlichen, sondern als mit fleischlichen Menschen, als mit
Unmündigen in Christus. \bibleverse{2} Milch habe ich euch zu trinken
gegeben, und nicht feste Speise; denn ihr vertruget sie nicht, ja ihr
vertraget sie jetzt noch nicht; \bibleverse{3} denn ihr seid noch
fleischlich. Solange nämlich Eifersucht und Zank und Zwietracht unter
euch sind, seid ihr da nicht fleischlich und wandelt nach Menschenweise?
\bibleverse{4} Denn wenn einer sagt: Ich halte zu Paulus, der andere
aber: Ich zu Apollos! seid ihr da nicht fleischlich? \bibleverse{5} Was
ist nun Apollos, was ist Paulus? Diener sind sie, durch welche ihr
gläubig geworden seid, und zwar, wie es der Herr einem jeglichen gegeben
hat. \bibleverse{6} Ich habe gepflanzt, Apollos hat begossen, Gott aber
hat das Gedeihen gegeben. \bibleverse{7} So ist also weder der etwas,
welcher pflanzt, noch der, welcher begießt, sondern Gott, der das
Gedeihen gibt. \bibleverse{8} Der aber, welcher pflanzt und der, welcher
begießt, sind einer wie der andere; jeder aber wird seinen eigenen Lohn
empfangen nach seiner eigenen Arbeit. \bibleverse{9} Denn wir sind
Gottes Mitarbeiter; ihr aber seid Gottes Ackerfeld und Gottes Bau.
\bibleverse{10} Nach der Gnade Gottes, die mir gegeben ist, habe ich als
ein weiser Baumeister den Grund gelegt; ein anderer aber baut darauf.
Ein jeglicher sehe zu, wie er darauf baue. \bibleverse{11} Denn einen
andern Grund kann niemand legen, außer dem, der gelegt ist, welcher ist
Jesus Christus. \bibleverse{12} Wenn aber jemand auf diesen Grund Gold,
Silber, kostbare Steine, Holz, Heu, Stroh baut, \bibleverse{13} so wird
eines jeden Werk offenbar werden; der Tag wird es klar machen, weil es
durchs Feuer offenbar wird. Und welcher Art eines jeden Werk ist, wird
das Feuer erproben. \bibleverse{14} Wird jemandes Werk, das er darauf
gebaut hat, bleiben, so wird er Lohn empfangen; \bibleverse{15} wird
aber jemandes Werk verbrennen, so wird er Schaden leiden, er selbst aber
wird gerettet werden, doch so, wie durchs Feuer hindurch.
\bibleverse{16} Wisset ihr nicht, daß ihr Gottes Tempel seid und der
Geist Gottes in euch wohnt? \bibleverse{17} Wenn jemand den Tempel
Gottes verderbt, den wird Gott verderben; denn der Tempel Gottes ist
heilig, und der seid ihr. \bibleverse{18} Niemand betrüge sich selbst!
Dünkt sich jemand unter euch weise zu sein in dieser Weltzeit, so werde
er ein Tor, damit er weise werde! \bibleverse{19} Denn die Weisheit
dieser Welt ist Torheit vor Gott; denn es steht geschrieben: ``Er fängt
die Weisen in ihrer List.'' \bibleverse{20} Und wiederum: ``Der Herr
kennt die Gedanken der Weisen, daß sie eitel sind.'' \bibleverse{21} So
brüste sich nun niemand mit Menschen; denn alles ist euer:
\bibleverse{22} es sei Paulus oder Apollos, Kephas oder die Welt, das
Leben oder der Tod, das Gegenwärtige oder das Zukünftige; alles ist
euer; \bibleverse{23} ihr aber seid Christi, Christus aber ist Gottes.

\hypertarget{section-3}{%
\section{4}\label{section-3}}

\bibleverse{1} So soll man uns betrachten: als Christi Diener und
Verwalter göttlicher Geheimnisse. \bibleverse{2} Im übrigen wird von
Verwaltern nur verlangt, daß einer treu erfunden werde. \bibleverse{3}
Mir aber ist es das Geringste, daß ich von euch oder von einem
menschlichen Gerichtstage beurteilt werde; auch beurteile ich mich nicht
selbst. \bibleverse{4} Denn ich bin mir nichts bewußt; aber damit bin
ich nicht gerechtfertigt, sondern der Herr ist es, der mich beurteilt.
\bibleverse{5} Darum richtet nichts vor der Zeit, bis der Herr kommt,
welcher auch das im Finstern Verborgene ans Licht bringen und den Rat
der Herzen offenbaren wird; und dann wird einem jeden das Lob von Gott
zuteil werden. \bibleverse{6} Das aber, meine Brüder, habe ich auf mich
und Apollos bezogen, damit ihr an uns lernet, nicht über das
hinauszugehen, was geschrieben steht, damit ihr euch nicht für den einen
auf Kosten des andern aufblähet. \bibleverse{7} Denn wer gibt dir den
Vorzug? Was besitzest du aber, das du nicht empfangen hast? Wenn du es
aber empfangen hast, was rühmst du dich, wie wenn du es nicht empfangen
hättest? \bibleverse{8} Ihr seid schon satt geworden, ihr seid schon
reich geworden, ihr herrschet ohne uns! Möchtet ihr wenigstens so
herrschen, daß auch wir mit euch herrschen könnten! \bibleverse{9} Es
dünkt mich nämlich, Gott habe uns Apostel als die Letzten hingestellt,
gleichsam zum Tode bestimmt; denn wir sind ein Schauspiel geworden der
Welt, sowohl Engeln als Menschen. \bibleverse{10} Wir sind Narren um
Christi willen, ihr aber seid klug in Christus; wir schwach, ihr aber
stark; ihr in Ehren, wir aber verachtet. \bibleverse{11} Bis auf diese
Stunde leiden wir Hunger, Durst und Blöße, werden geschlagen und haben
keine Bleibe und arbeiten mühsam mit unsern eigenen Händen.
\bibleverse{12} Wir werden geschmäht und segnen, wir leiden Verfolgung
und halten stand; wir werden gelästert und spenden Trost;
\bibleverse{13} zum Auswurf der Welt sind wir geworden, zum Abschaum
aller bis jetzt. \bibleverse{14} Nicht zu eurer Beschämung schreibe ich
das, sondern ich ermahne euch als meine geliebten Kinder.
\bibleverse{15} Denn wenn ihr auch zehntausend Erzieher hättet in
Christus, so habt ihr doch nicht viele Väter; denn ich habe euch in
Christus Jesus durch das Evangelium gezeugt. \bibleverse{16} So ermahne
ich euch nun: Werdet meine Nachahmer! \bibleverse{17} Deshalb habe ich
Timotheus zu euch gesandt, der mein geliebter und treuer Sohn im Herrn
ist; der wird euch an meine Wege in Christus erinnern, wie ich
allenthalben in jeder Gemeinde lehre. \bibleverse{18} Weil ich aber
nicht selbst zu euch komme, haben sich etliche aufgebläht;
\bibleverse{19} ich werde aber bald zu euch kommen, so der Herr will,
und Kenntnis nehmen, nicht von den Worten der Aufgeblähten, sondern von
der Kraft. \bibleverse{20} Denn das Reich Gottes besteht nicht in
Worten, sondern in Kraft! \bibleverse{21} Was wollt ihr? Soll ich mit
der Rute zu euch kommen, oder mit Liebe und dem Geiste der Sanftmut?

\hypertarget{section-4}{%
\section{5}\label{section-4}}

\bibleverse{1} Überhaupt hört man von Unzucht unter euch, und zwar von
einer solchen Unzucht, die nicht einmal unter den Heiden vorkommt, daß
nämlich einer seines Vaters Frau habe! \bibleverse{2} Und ihr seid
aufgebläht und hättet doch eher Leid tragen sollen, damit der, welcher
diese Tat begangen hat, aus eurer Mitte getan würde! \bibleverse{3} Denn
ich, der ich zwar dem Leibe nach abwesend, dem Geiste nach aber anwesend
bin, habe schon, als wäre ich anwesend, über den, welcher solches
begangen hat, beschlossen: \bibleverse{4} im Namen unsres Herrn Jesus
Christus und nachdem euer und mein Geist sich mit der Kraft unsres Herrn
Jesus Christus vereinigt hat, \bibleverse{5} den Betreffenden dem Satan
zu übergeben zum Verderben des Fleisches, damit der Geist gerettet werde
am Tage des Herrn Jesus. \bibleverse{6} Euer Rühmen ist nicht fein!
Wisset ihr nicht, daß ein wenig Sauerteig den ganzen Teig durchsäuert?
\bibleverse{7} Feget den alten Sauerteig aus, damit ihr ein neuer Teig
seid, gleichwie ihr ja ungesäuert seid! Denn auch für uns ist ein
Passahlamm geschlachtet worden: Christus. \bibleverse{8} So wollen wir
denn nicht mit altem Sauerteig Fest feiern, auch nicht mit Sauerteig der
Bosheit und Schlechtigkeit, sondern mit ungesäuerten Broten der
Lauterkeit und Wahrheit. \bibleverse{9} Ich habe euch in dem Brief
geschrieben, daß ihr keinen Umgang mit Unzüchtigen haben sollt;
\bibleverse{10} nicht überhaupt mit den Unzüchtigen dieser Welt, oder
den Habsüchtigen und Räubern oder Götzendienern; sonst müßtet ihr ja die
Welt räumen. \bibleverse{11} Nun aber habe ich euch geschrieben, daß ihr
keinen Umgang haben sollt mit jemandem, der sich Bruder nennen läßt und
dabei ein Unzüchtiger oder Habsüchtiger oder Götzendiener oder Lästerer
oder Trunkenbold oder Räuber ist; mit einem solchen sollt ihr nicht
einmal essen. \bibleverse{12} Denn was soll ich die richten, die
außerhalb der Gemeinde sind? Ihr richtet nicht einmal die, welche
drinnen sind? \bibleverse{13} Die aber draußen sind, wird Gott richten.
Tut den Bösen aus eurer Mitte hinweg!

\hypertarget{section-5}{%
\section{6}\label{section-5}}

\bibleverse{1} Wie darf jemand von euch, der eine Beschwerde gegen einen
andern hat, sich bei den Ungerechten richten lassen, anstatt bei den
Heiligen? \bibleverse{2} Wisset ihr nicht, daß die Heiligen die Welt
richten werden? Wenn nun durch euch die Welt gerichtet werden soll, seid
ihr dann unwürdig, über die allergeringsten Dinge zu entscheiden?
\bibleverse{3} Wisset ihr nicht, daß wir Engel richten werden? Warum
denn nicht auch Dinge dieses Lebens? \bibleverse{4} Wenn ihr nun über
Dinge dieses Lebens Entscheidungen zu treffen habt, so setzet ihr solche
zu Richtern, die bei der Gemeinde nichts gelten! \bibleverse{5} Zur
Beschämung sage ich's euch: demnach ist also nicht ein einziger
Sachverständiger unter euch, der ein unparteiisches Urteil fällen könnte
für seinen Bruder; \bibleverse{6} sondern ein Bruder rechtet mit dem
andern, und das vor Ungläubigen! \bibleverse{7} Es ist überhaupt schon
schlimm genug für euch, daß ihr Prozesse miteinander führet. Warum
lasset ihr euch nicht lieber Unrecht tun? Warum lasset ihr euch nicht
lieber übervorteilen? \bibleverse{8} Sondern ihr übet Unrecht und
Übervorteilung, und zwar an Brüdern! \bibleverse{9} Wisset ihr denn
nicht, daß Ungerechte das Reich Gottes nicht ererben werden? Irret euch
nicht: Weder Unzüchtige noch Götzendiener, weder Ehebrecher noch
Weichlinge, noch Knabenschänder, \bibleverse{10} weder Diebe noch
Habsüchtige, noch Trunkenbolde, noch Lästerer, noch Räuber werden das
Reich Gottes ererben. \bibleverse{11} Und solche sind etliche von euch
gewesen; aber ihr seid abgewaschen, ihr seid geheiligt, ihr seid
gerechtfertigt worden in dem Namen unsres Herrn Jesus Christus und in
dem Geist unsres Gottes! \bibleverse{12} Alles ist mir erlaubt; aber
nicht alles frommt! Alles ist mir erlaubt; aber ich will mich von nichts
beherrschen lassen. \bibleverse{13} Die Speisen sind für den Bauch und
der Bauch für die Speisen; Gott aber wird diesen und jene abtun. Der
Leib aber ist nicht für die Unzucht, sondern für den Herrn, und der Herr
für den Leib. \bibleverse{14} Gott aber hat den Herrn auferweckt und
wird auch uns auferwecken durch seine Kraft. \bibleverse{15} Wisset ihr
nicht, daß eure Leiber Christi Glieder sind? Soll ich nun die Glieder
Christi nehmen und Hurenglieder daraus machen? Das sei ferne!
\bibleverse{16} Wisset ihr aber nicht, daß, wer einer Hure anhängt, ein
Leib mit ihr ist? ``Denn es werden'', spricht er, ``die zwei ein Fleisch
sein.'' \bibleverse{17} Wer aber dem Herrn anhängt, ist ein Geist mit
ihm. \bibleverse{18} Fliehet die Unzucht! Jede Sünde, die ein Mensch
sonst begeht, ist außerhalb des Leibes; der Unzüchtige aber sündigt an
seinem eigenen Leib. \bibleverse{19} Oder wisset ihr nicht, daß euer
Leib ein Tempel des in euch wohnenden heiligen Geistes ist, welchen ihr
von Gott empfangen habt, und daß ihr nicht euch selbst angehöret?
\bibleverse{20} Denn ihr seid teuer erkauft; darum verherrlichet Gott
mit eurem Leibe!

\hypertarget{section-6}{%
\section{7}\label{section-6}}

\bibleverse{1} Was aber das betrifft, wovon ihr mir geschrieben habt, so
ist es ja gut für den Menschen, kein Weib zu berühren; \bibleverse{2} um
aber Unzucht zu vermeiden, habe ein jeglicher seine eigene Frau und eine
jegliche ihren eigenen Mann. \bibleverse{3} Der Mann leiste der Frau die
schuldige Pflicht, ebenso aber auch die Frau dem Manne. \bibleverse{4}
Die Frau verfügt nicht selbst über ihren Leib, sondern der Mann;
gleicherweise verfügt aber auch der Mann nicht selbst über seinen Leib,
sondern die Frau. \bibleverse{5} Entziehet euch einander nicht, außer
nach Übereinkunft auf einige Zeit, damit ihr zum Gebet Muße habt, und
kommet wieder zusammen, damit euch der Satan nicht versuche um eurer
Unenthaltsamkeit willen. \bibleverse{6} Das sage ich aber aus Nachsicht
und nicht als Befehl. \bibleverse{7} Denn ich wollte, alle Menschen
wären wie ich; aber jeder hat seine eigene Gnadengabe von Gott, der eine
so, der andere so. \bibleverse{8} Ich sage aber den Ledigen und den
Witwen: Es ist gut für sie, wenn sie bleiben wie ich. \bibleverse{9}
Können sie sich aber nicht enthalten, so sollen sie heiraten; denn
heiraten ist besser als in Glut geraten. \bibleverse{10} Den
Verheirateten aber gebiete nicht ich, sondern der Herr, daß eine Frau
sich nicht scheide von dem Manne; \bibleverse{11} wäre sie aber schon
geschieden, so bleibe sie unverheiratet oder versöhne sich mit dem
Manne. Der Mann aber soll die Frau nicht verstoßen. \bibleverse{12} Den
übrigen aber sage ich, nicht der Herr: Wenn ein Bruder eine ungläubige
Frau hat, und diese ist einverstanden, bei ihm zu wohnen, so soll er sie
nicht verstoßen; \bibleverse{13} und wenn eine Frau einen ungläubigen
Mann hat, und dieser ist einverstanden, bei ihr zu wohnen, so soll sie
den Mann nicht verlassen. \bibleverse{14} Denn der ungläubige Mann ist
geheiligt durch die Frau, und die ungläubige Frau ist geheiligt durch
den Bruder; sonst wären eure Kinder unrein, nun aber sind sie heilig.
\bibleverse{15} Will sich aber der ungläubige Teil scheiden, so scheide
er! Der Bruder oder die Schwester ist in solchen Fällen nicht gebunden.
In Frieden aber hat uns Gott berufen. \bibleverse{16} Denn was weißt du,
Frau, ob du den Mann retten kannst? Oder was weißt du, Mann, ob du die
Frau retten kannst? \bibleverse{17} Doch wie der Herr einem jeden
zugeteilt hat, wie der Herr einen jeden berufen hat, so wandle er! Und
so verordne ich es in allen Gemeinden. \bibleverse{18} Ist jemand nach
erfolgter Beschneidung berufen worden, so lasse er sich von ihr nicht
wieder gewinnen; ist jemand in unbeschnittenem Zustand berufen worden,
so lasse er sich nicht beschneiden. \bibleverse{19} Beschnitten sein ist
nichts und unbeschnitten sein ist auch nichts, wohl aber Gottes Gebote
halten. \bibleverse{20} Jeder bleibe in dem Stand, darin er berufen
worden ist. \bibleverse{21} Bist du als Sklave berufen worden, so sei
deshalb ohne Sorge! Kannst du aber frei werden, so benütze es lieber.
\bibleverse{22} Denn der im Herrn berufene Sklave ist ein Freigelassener
des Herrn; desgleichen ist der berufene Freie ein Knecht Christi.
\bibleverse{23} Ihr seid teuer erkauft; werdet nicht der Menschen
Knechte! \bibleverse{24} Brüder, es bleibe ein jeglicher vor Gott in dem
Stand, worin er berufen worden ist. \bibleverse{25} Betreffs der
Jungfrauen aber habe ich keinen Auftrag vom Herrn; ich gebe aber ein
Gutachten ab als einer, der vom Herrn begnadigt worden ist, treu zu
sein. \bibleverse{26} So halte ich nun, um der bevorstehenden Not
willen, für richtig, daß es nämlich für einen Menschen gut sei, so zu
sein. \bibleverse{27} Bist du an eine Frau gebunden, so suche keine
Lösung; bist du los von der Frau, so suche keine Frau. \bibleverse{28}
Wenn du aber auch heiratest, so sündigest du nicht; und wenn die
Jungfrau heiratet, so sündigt sie nicht; doch werden solche leibliche
Trübsal haben, die ich euch gerne ersparen möchte. \bibleverse{29} Das
aber sage ich, ihr Brüder: Die Zeit ist beschränkt! So mögen nun in der
noch verbleibenden Frist die, welche Frauen haben, sein, als hätten sie
keine, \bibleverse{30} und die da weinen, als weinten sie nicht, und die
sich freuen, als freuten sie sich nicht, und die da kaufen, als besäßen
sie es nicht, \bibleverse{31} und die diese Welt gebrauchen, als
brauchten sie sie gar nicht; denn die Gestalt dieser Welt vergeht.
\bibleverse{32} Ich will aber, daß ihr ohne Sorgen seid! Der
Unverheiratete ist für die Sache des Herrn besorgt, wie er dem Herrn
gefalle; \bibleverse{33} der Verheiratete aber sorgt für die Dinge der
Welt, wie er der Frau gefalle, und er ist geteilt. \bibleverse{34} So
ist auch die Frau, die keinen Mann hat, und die Jungfrau besorgt um die
Sache des Herrn, daß sie heilig sei am Leibe und am Geist; die
Verheiratete aber sorgt für die Dinge der Welt, wie sie dem Manne
gefalle. \bibleverse{35} Das sage ich aber zu eurem eigenen Nutzen,
nicht um euch eine Schlinge um den Hals zu werfen, sondern damit ihr in
allem Anstand und ungeteilt bei dem Herrn verharren könnet.
\bibleverse{36} Wenn aber jemand meint, daß es für seine Jungfrau
unschicklich sei, über die Jahre der Reife hinauszukommen, und wenn es
dann so sein muß, der tue, was er will; er sündigt nicht, sie mögen
heiraten! \bibleverse{37} Wenn aber einer in seinem Herzen fest geworden
ist und keine Verpflichtung hat, sondern Macht, nach seinem eigenen
Willen zu handeln, und in seinem eigenen Herzen beschlossen hat, seine
Jungfrau zu behalten, der tut wohl. \bibleverse{38} Doch tut auch der
wohl, welcher sie zur Ehe gibt; wer sie aber nicht gibt, tut besser.
\bibleverse{39} Eine Frau ist gebunden, solange ihr Mann lebt; wenn aber
ihr Mann entschlafen ist, so ist sie frei, sich zu verheiraten, mit wem
sie will; nur daß es im Herrn geschehe. \bibleverse{40} Seliger aber ist
sie, wenn sie so bleibt, nach meiner Meinung; ich glaube aber auch den
heiligen Geist zu haben.

\hypertarget{section-7}{%
\section{8}\label{section-7}}

\bibleverse{1} Betreffs der Götzenopfer aber wissen wir, da wir alle
Erkenntnis haben; die Erkenntnis bläht auf, aber die Liebe erbaut.
\bibleverse{2} Wenn aber jemand meint, etwas erkannt zu haben, der hat
noch nicht erkannt, wie man erkennen soll; \bibleverse{3} wenn aber
jemand Gott liebt, der ist von ihm erkannt, \bibleverse{4} was also das
Essen der Götzenopfer betrifft, so wissen wir, daß kein Götze in der
Welt ist und daß es keinen Gott gibt außer dem Einen. \bibleverse{5}
Denn wenn es auch sogenannte Götter gibt, sei es im Himmel oder auf
Erden (wie es ja wirklich viele Götter und viele Herren gibt),
\bibleverse{6} so haben wir doch nur einen Gott, den Vater, von welchem
alle Dinge sind und wir für ihn; und einen Herrn, Jesus Christus, durch
welchen alle Dinge sind, und wir durch ihn. \bibleverse{7} Aber nicht
alle haben die Erkenntnis, sondern etliche essen infolge ihrer Gewöhnung
an den Götzen das Fleisch noch immer als Götzenopferfleisch, und so wird
ihr Gewissen, weil es schwach ist, befleckt. \bibleverse{8} Nun
verschafft uns aber das Essen keine Bedeutung bei Gott; wir sind nicht
mehr, wenn wir essen, und sind nicht weniger, wenn wir nicht essen.
\bibleverse{9} Sehet aber zu, daß diese eure Freiheit den Schwachen
nicht zum Anstoß werde! \bibleverse{10} Denn wenn jemand dich, der du
die Erkenntnis hast, im Götzenhause zu Tische sitzen sieht, wird nicht
sein Gewissen, weil es schwach ist, ermutigt werden, Götzenopferfleisch
zu essen? \bibleverse{11} Und so wird durch deine Erkenntnis der
schwache Bruder verdorben, um dessen willen Christus gestorben ist.
\bibleverse{12} Wenn ihr aber auf solche Weise an den Brüdern sündiget
und ihr schwaches Gewissen verletzet, so sündiget ihr gegen Christus.
\bibleverse{13} Darum wenn eine Speise meinem Bruder zum Anstoß wird, so
will ich lieber in Ewigkeit kein Fleisch essen, damit ich meinem Bruder
keinen Anstoß gebe.

\hypertarget{section-8}{%
\section{9}\label{section-8}}

\bibleverse{1} Bin ich nicht frei? Bin ich nicht ein Apostel? Habe ich
nicht unsern Herrn Jesus Christus gesehen? Seid nicht ihr mein Werk im
Herrn? \bibleverse{2} Bin ich für andere kein Apostel, so bin ich es
doch für euch; denn das Siegel meines Apostelamts seid ihr in dem Herrn.
\bibleverse{3} Dies ist meine Verteidigung denen gegenüber, die mich zur
Rede stellen: \bibleverse{4} Haben wir nicht Vollmacht, zu essen und zu
trinken? \bibleverse{5} Haben wir nicht Vollmacht, eine Schwester als
Gattin mit uns zu führen, wie auch die andern Apostel und die Brüder des
Herrn und Kephas? \bibleverse{6} Oder haben nur ich und Barnabas keine
Vollmacht, die Arbeit zu unterlassen? \bibleverse{7} Wer zieht je auf
eigene Kosten ins Feld? Wer pflanzt einen Weinberg und ißt nicht von
dessen Frucht? Oder wer weidet eine Herde und nährt sich nicht von der
Milch der Herde? \bibleverse{8} Sage ich das nur nach menschlicher
Weise? Sagt es nicht auch das Gesetz? \bibleverse{9} Ja, im Gesetz Moses
steht geschrieben: ``Du sollst dem Ochsen das Maul nicht verbinden, wenn
er drischt.'' \bibleverse{10} Kümmert sich Gott nur um die Ochsen? Sagt
er das nicht vielmehr wegen uns? Denn unsertwegen steht ja geschrieben,
daß, wer pflügt, auf Hoffnung hin pflügen, und wer drischt, auf Hoffnung
hin dreschen soll, daß er des Gehofften teilhaftig werde.
\bibleverse{11} Wenn wir euch die geistlichen Güter gesät haben, ist es
etwas Großes, wenn wir von euch diejenigen für den Leib ernten?
\bibleverse{12} Wenn andere an diesem Recht über euch Anteil haben,
sollten wir es nicht viel eher? Aber wir haben uns dieses Rechtes nicht
bedient, sondern wir ertragen alles, damit wir dem Evangelium Christi
kein Hindernis bereiten. \bibleverse{13} Wisset ihr nicht, daß die,
welche die heiligen Dienstverrichtungen besorgen, auch vom Heiligtum
essen, und daß die, welche des Altars warten, vom Altar ihren Anteil
erhalten? \bibleverse{14} So hat auch der Herr verordnet, daß die,
welche das Evangelium verkündigen, vom Evangelium leben sollen.
\bibleverse{15} Ich aber habe davon keinerlei Gebrauch gemacht; ich habe
auch solches nicht darum geschrieben, damit es mit mir so gehalten
werde. Viel lieber wollte ich sterben, als daß mir jemand meinen Ruhm
zunichte machte! \bibleverse{16} Denn wenn ich das Evangelium predige,
so ist das kein Ruhm für mich; denn ich bin dazu verpflichtet, und wehe
mir, wenn ich das Evangelium nicht predigte! \bibleverse{17} Tue ich es
freiwillig, so habe ich Lohn; wenn aber unfreiwillig, bin ich gleichwohl
mit dem Verwalteramt betraut. \bibleverse{18} Was ist denn nun mein
Lohn? Daß ich bei meiner Verkündigung des Evangeliums dieses kostenfrei
darbiete, so daß ich von meinem Anspruch ans Evangelium keinen Gebrauch
mache. \bibleverse{19} Denn wiewohl ich frei bin von allen, habe ich
mich doch allen zum Knecht gemacht, um ihrer desto mehr zu gewinnen.
\bibleverse{20} Den Juden bin ich wie ein Jude geworden, auf daß ich die
Juden gewinne; denen, die unter dem Gesetz sind, bin ich geworden, als
wäre ich unter dem Gesetz (obschon ich nicht unter dem Gesetz bin),
damit ich die unter dem Gesetz gewinne; \bibleverse{21} denen, die ohne
Gesetz sind, bin ich geworden, als wäre ich ohne Gesetz (wiewohl ich
nicht ohne göttliches Gesetz lebe, sondern in dem Gesetz Christi), damit
ich die gewinne, welche ohne Gesetz sind. \bibleverse{22} Den Schwachen
bin ich wie ein Schwacher geworden, damit ich die Schwachen gewinne; ich
bin allen alles geworden, damit ich allenthalben etliche rette.
\bibleverse{23} Alles aber tue ich um des Evangeliums willen, um an ihm
teilzuhaben. \bibleverse{24} Wisset ihr nicht, daß die, welche in der
Rennbahn laufen, zwar alle laufen, aber nur einer den Preis erlangt?
Laufet so, daß ihr ihn erlanget! \bibleverse{25} Jeder aber, der sich am
Wettlauf beteiligt, ist enthaltsam in allem; jene, um einen
vergänglichen Kranz zu empfangen, wir aber einen unvergänglichen.
\bibleverse{26} So laufe ich nun nicht wie aufs Ungewisse; ich führe
meinen Faustkampf nicht mit bloßen Luftstreichen, \bibleverse{27}
sondern ich zerschlage meinen Leib und behandle ihn als Sklaven, damit
ich nicht andern predige und selbst verwerflich werde.

\hypertarget{section-9}{%
\section{10}\label{section-9}}

\bibleverse{1} Ich will aber nicht, meine Brüder, daß ihr außer acht
lasset, daß unsre Väter alle unter der Wolke gewesen und alle durchs
Meer hindurch gegangen sind. \bibleverse{2} Sie wurden auch alle auf
Mose getauft in der Wolke und im Meer, \bibleverse{3} und sie haben alle
dieselbe geistliche Speise gegessen und alle denselben geistlichen Trank
getrunken; \bibleverse{4} denn sie tranken aus einem geistlichen Felsen,
der ihnen folgte. Der Fels aber war Christus. \bibleverse{5} Aber an der
Mehrzahl von ihnen hatte Gott kein Wohlgefallen; denn sie wurden in der
Wüste niedergestreckt. \bibleverse{6} Diese Dinge aber sind zum Vorbild
für uns geschehen, damit wir uns nicht des Bösen gelüsten lassen,
gleichwie jene gelüstet hat. \bibleverse{7} Werdet auch nicht
Götzendiener, gleichwie etliche von ihnen, wie geschrieben steht: ``Das
Volk setzte sich nieder, um zu essen und zu trinken, und stand auf, um
zu spielen.'' \bibleverse{8} Lasset uns auch nicht Unzucht treiben,
gleichwie etliche von ihnen Unzucht trieben, und es fielen an einem Tage
ihrer dreiundzwanzigtausend. \bibleverse{9} Lasset uns auch nicht
Christus versuchen, gleichwie etliche von ihnen ihn versuchten und von
den Schlangen umgebracht wurden. \bibleverse{10} Murret auch nicht,
gleichwie etliche von ihnen murrten und durch den Verderber umgebracht
wurden. \bibleverse{11} Das alles, was jenen widerfuhr, ist ein Vorbild
und wurde zur Warnung geschrieben für uns, auf welche das Ende der
Zeitalter gekommen ist. \bibleverse{12} Darum, wer sich dünkt, er stehe,
der sehe wohl zu, daß er nicht falle! \bibleverse{13} Es hat euch bisher
nur menschliche Versuchung betroffen. Gott aber ist treu; der wird euch
nicht über euer Vermögen versucht werden lassen, sondern wird zugleich
mit der Versuchung auch den Ausgang schaffen, daß ihr sie ertragen
könnt. \bibleverse{14} Darum, meine Geliebten, fliehet vor dem
Götzendienst! \bibleverse{15} Ich rede mit Verständigen; beurteilet ihr,
was ich sage: \bibleverse{16} Der Kelch des Segens, den wir segnen, ist
er nicht Gemeinschaft mit dem Blute Christi? Das Brot, das wir brechen,
ist es nicht Gemeinschaft mit dem Leibe Christi? \bibleverse{17} Denn
ein Brot ist es, so sind wir, die vielen, ein Leib; denn wir sind alle
des einen Brotes teilhaftig. \bibleverse{18} Sehet an das Israel nach
dem Fleisch! Stehen nicht die, welche die Opfer essen, in Gemeinschaft
mit dem Opferaltar? \bibleverse{19} Was sage ich nun? Daß das
Götzenopfer etwas sei, oder daß ein Götze etwas sei? \bibleverse{20}
Nein, aber daß sie das, was sie opfern, den Dämonen opfern und nicht
Gott! Ich will aber nicht, daß ihr in Gemeinschaft der Dämonen geratet.
\bibleverse{21} Ihr könnet nicht des Herrn Kelch trinken und der Dämonen
Kelch; ihr könnet nicht am Tische des Herrn teilhaben und am Tische der
Dämonen! \bibleverse{22} Oder wollen wir den Herrn zur Eifersucht
reizen? Sind wir stärker als er? \bibleverse{23} Es ist alles erlaubt;
aber es frommt nicht alles! Es ist alles erlaubt; aber es erbaut nicht
alles! \bibleverse{24} Niemand suche das Seine, sondern ein jeder das
des andern. \bibleverse{25} Alles, was auf dem Fleischmarkt feil ist,
das esset, ohne um des Gewissens willen nachzuforschen; \bibleverse{26}
denn ``die Erde ist des Herrn und was sie erfüllt''. \bibleverse{27}
Wenn aber jemand von den Ungläubigen euch einladet und ihr hingehen
wollt, so esset alles, was euch vorgesetzt wird, und forschet nicht nach
um des Gewissens willen. \bibleverse{28} Wenn aber jemand zu euch sagen
würde: Das ist Götzenopferfleisch! so esset es nicht, um deswillen, der
es anzeigt, und um des Gewissens willen. \bibleverse{29} Ich rede aber
nicht von deinem eigenen Gewissen, sondern von dem des andern; denn
warum sollte meine Freiheit von dem Gewissen eines andern gerichtet
werden? \bibleverse{30} Wenn ich es dankbar genieße, warum sollte ich
gelästert werden über dem, wofür ich danke? \bibleverse{31} Ihr esset
nun oder trinket oder was ihr tut, so tut es alles zu Gottes Ehre!
\bibleverse{32} Seid unanstößig den Juden und Griechen und der Gemeinde
Gottes, \bibleverse{33} gleichwie auch ich in allen Stücken allen zu
Gefallen lebe und nicht suche, was mir, sondern was vielen frommt, damit
sie gerettet werden.

\hypertarget{section-10}{%
\section{11}\label{section-10}}

\bibleverse{1} Werdet meine Nachahmer, gleichwie ich Christi!
\bibleverse{2} Ich lobe euch, Brüder, daß ihr in allen Dingen meiner
eingedenk seid und an den Überlieferungen festhaltet, so wie ich sie
euch übergeben habe. \bibleverse{3} Ich will aber, daß ihr wisset, daß
Christus eines jeglichen Mannes Haupt ist, der Mann aber des Weibes
Haupt, Gott aber Christi Haupt. \bibleverse{4} Ein jeglicher Mann,
welcher betet oder weissagt und etwas auf dem Haupte hat, schändet sein
Haupt. \bibleverse{5} Jedes Weib aber, welches betet und weissagt mit
unverhülltem Haupt, schändet ihr Haupt; es ist ein und dasselbe, wie
wenn sie geschoren wäre! \bibleverse{6} Denn wenn sich ein Weib nicht
verhüllen will, so lasse sie sich das Haar abschneiden! Nun es aber
einem Weibe übel ansteht, sich das Haar abschneiden oder abscheren zu
lassen, so soll sie sich verhüllen. \bibleverse{7} Der Mann hat nämlich
darum nicht nötig, das Haupt zu verhüllen, weil er Gottes Bild und Ehre
ist; das Weib aber ist des Mannes Ehre. \bibleverse{8} Denn der Mann
kommt nicht vom Weibe, sondern das Weib vom Mann; \bibleverse{9} auch
wurde der Mann nicht um des Weibes willen erschaffen, sondern das Weib
um des Mannes willen. \bibleverse{10} Darum muß das Weib ein Zeichen der
Gewalt auf dem Haupte haben, um der Engel willen. \bibleverse{11} Doch
ist im Herrn weder das Weib ohne den Mann, noch der Mann ohne das Weib.
\bibleverse{12} Denn gleichwie das Weib vom Manne kommt, so auch der
Mann durch das Weib; aber das alles von Gott. \bibleverse{13} Urteilet
bei euch selbst, ob es schicklich sei, daß ein Weib unverhüllt Gott
anbete! \bibleverse{14} Oder lehrt euch nicht schon die Natur, daß es
für einen Mann eine Unehre ist, langes Haar zu tragen? \bibleverse{15}
Dagegen gereicht es einem Weibe zur Ehre, wenn sie langes Haar trägt;
denn das Haar ist ihr statt eines Schleiers gegeben. \bibleverse{16}
Will aber jemand rechthaberisch sein, so haben wir solche Gewohnheit
nicht, die Gemeinden Gottes auch nicht. \bibleverse{17} Das aber kann
ich, da ich am Verordnen bin, nicht loben, daß eure Zusammenkünfte nicht
besser, sondern eher schlechter werden. \bibleverse{18} Denn erstens
höre ich, daß, wenn ihr in der Gemeinde zusammenkommt, Spaltungen unter
euch sind, und zum Teil glaube ich es; \bibleverse{19} denn es müssen ja
auch Parteiungen unter euch sein, damit die Bewährten offenbar werden
unter euch! \bibleverse{20} Wenn ihr nun auch am selben Orte
zusammenkommt, so ist das doch nicht, um des Herrn Mahl zu essen;
\bibleverse{21} denn ein jeder nimmt beim Essen sein eigenes Mahl
vorweg, so daß der eine hungrig, der andere trunken ist. \bibleverse{22}
Habt ihr denn keine Häuser, wo ihr essen und trinken könnt? Oder
verachtet ihr die Gemeinde Gottes und beschämet die, welche nichts
haben? Was soll ich euch sagen? Soll ich euch loben? Dafür lobe ich
nicht! \bibleverse{23} Denn ich habe vom Herrn empfangen, was ich auch
euch überliefert habe, nämlich daß der Herr Jesus in der Nacht, da er
verraten wurde, Brot nahm, es mit Danksagung brach und sprach:
\bibleverse{24} Nehmet, esset, das ist mein Leib, der für euch gebrochen
wird, solches tut zu meinem Gedächtnis! \bibleverse{25} Desgleichen auch
den Kelch, nach dem Mahl, indem er sprach: Dieser Kelch ist der neue
Bund in meinem Blut; solches tut, so oft ihr ihn trinket, zu meinem
Gedächtnis! \bibleverse{26} Denn so oft ihr dieses Brot esset und den
Kelch trinket, verkündiget ihr den Tod des Herrn, bis daß er kommt.
\bibleverse{27} Wer also unwürdig das Brot ißt oder den Kelch des Herrn
trinkt, der ist schuldig am Leib und am Blut des Herrn. \bibleverse{28}
Es prüfe aber ein Mensch sich selbst, und also esse er von dem Brot und
trinke aus dem Kelch; \bibleverse{29} denn wer unwürdig ißt und trinkt,
der ißt und trinkt sich selbst ein Gericht, weil er den Leib des Herrn
nicht unterscheidet. \bibleverse{30} Deshalb sind unter euch viele
Schwache und Kranke, und eine beträchtliche Zahl sind entschlafen;
\bibleverse{31} denn wenn wir uns selbst richteten, würden wir nicht
gerichtet werden; \bibleverse{32} werden wir aber vom Herrn gerichtet,
so geschieht es zu unserer Züchtigung, damit wir nicht samt der Welt
verdammt werden. \bibleverse{33} Darum, meine Brüder, wenn ihr zum Essen
zusammenkommt, so wartet aufeinander! \bibleverse{34} Hungert aber
jemand, so esse er daheim, damit ihr nicht zum Gericht zusammenkommt.
Das übrige will ich anordnen, sobald ich komme.

\hypertarget{section-11}{%
\section{12}\label{section-11}}

\bibleverse{1} Über die Geistesgaben aber, meine Brüder, will ich euch
nicht in Unwissenheit lassen. \bibleverse{2} Ihr wisset, daß ihr, als
ihr Heiden waret, euch zu den stummen Götzen hinziehen ließet, wie ihr
geleitet wurdet. \bibleverse{3} Darum tue ich euch kund, daß niemand,
der im Geiste Gottes redet, sagt: ``Verflucht sei Jesus!'' es kann aber
auch niemand sagen: ``Herr Jesus!'' als nur im heiligen Geist.
\bibleverse{4} Es bestehen aber Unterschiede in den Gnadengaben, doch
ist es derselbe Geist; \bibleverse{5} auch gibt es verschiedene
Dienstleistungen, doch ist es derselbe Herr; \bibleverse{6} und auch die
Kraftwirkungen sind verschieden, doch ist es derselbe Gott, der alles in
allen wirkt. \bibleverse{7} Einem jeglichen aber wird die Offenbarung
des Geistes zum allgemeinen Nutzen verliehen. \bibleverse{8} Dem einen
nämlich wird durch den Geist die Rede der Weisheit gegeben, einem andern
aber die Rede der Erkenntnis nach demselben Geist; \bibleverse{9} einem
andern Glauben in demselben Geist; einem andern die Gabe gesund zu
machen in dem gleichen Geist; \bibleverse{10} einem andern Wunder zu
wirken, einem andern Weissagung, einem andern Geister zu unterscheiden,
einem andern verschiedene Arten von Sprachen, einem andern die Auslegung
der Sprachen. \bibleverse{11} Dieses alles aber wirkt ein und derselbe
Geist, der einem jeden persönlich zuteilt, wie er will. \bibleverse{12}
Denn gleichwie der Leib einer ist und doch viele Glieder hat, alle
Glieder des Leibes aber, wiewohl ihrer viele sind, doch nur einen Leib
bilden, also auch Christus. \bibleverse{13} Denn wir wurden alle in
einem Geist zu einem Leibe getauft, seien wir Juden oder Griechen,
Knechte oder Freie, und wurden alle mit einem Geist getränkt.
\bibleverse{14} Denn auch der Leib ist nicht ein Glied, sondern viele.
\bibleverse{15} Wenn der Fuß spräche: Ich bin keine Hand, darum gehöre
ich nicht zum Leib, so gehört er deswegen nicht weniger dazu!
\bibleverse{16} Und wenn das Ohr spräche: Ich bin kein Auge, darum
gehöre ich nicht zum Leib; so gehört es deswegen nicht weniger dazu!
\bibleverse{17} Wäre der ganze Leib Auge, wo bliebe das Gehör? Wäre er
ganz Ohr, wo bliebe der Geruch? \bibleverse{18} Nun aber hat Gott die
Glieder, jedes einzelne von ihnen, so am Leibe gesetzt, wie er gewollt
hat. \bibleverse{19} Wenn aber alles ein Glied wäre, wo bliebe der Leib?
\bibleverse{20} Nun aber gibt es viele Glieder, doch nur einen Leib.
\bibleverse{21} Das Auge kann nicht zur Hand sagen: Ich bedarf deiner
nicht, oder das Haupt zu den Füßen: Ich bedarf euer nicht!
\bibleverse{22} Vielmehr sind gerade die scheinbar schwächern Glieder
des Leibes notwendig, \bibleverse{23} und die wir für weniger ehrbar am
Leibe halten, die umgeben wir mit desto größerer Ehre, und die uns übel
anstehen, die schmückt man am meisten; \bibleverse{24} denn die uns wohl
anstehen, bedürfen es nicht. Gott aber hat den Leib so zusammengefügt,
daß er dem dürftigeren Glied um so größere Ehre gab, \bibleverse{25}
damit es keinen Zwiespalt im Leibe gebe, sondern die Glieder gleichmäßig
füreinander sorgen. \bibleverse{26} Und wenn ein Glied leidet, so leiden
alle Glieder mit; und wenn ein Glied geehrt wird, so freuen sich alle
Glieder mit. \bibleverse{27} Ihr aber seid Christi Leib, und jedes in
seinem Teil Glieder. \bibleverse{28} Und so hat Gott in der Gemeinde
gesetzt erstens Apostel, zweitens Propheten, drittens Lehrer, darnach
Wundertäter, sodann die Gaben der Heilung, der Hilfeleistung, der
Verwaltung, verschiedene Sprachen. \bibleverse{29} Es sind doch nicht
alle Apostel, nicht alle Propheten, nicht alle Lehrer, nicht alle
Wundertäter? \bibleverse{30} Haben alle die Gaben der Heilung? Reden
alle mit Zungen? Können alle auslegen? \bibleverse{31} Strebet aber nach
den besten Gaben; doch zeige ich euch jetzt einen noch weit
vortrefflicheren Weg:

\hypertarget{section-12}{%
\section{13}\label{section-12}}

\bibleverse{1} Wenn ich mit Menschen und Engelzungen rede, aber keine
Liebe habe, so bin ich ein tönendes Erz oder eine klingende Schelle.
\bibleverse{2} Und wenn ich weissagen kann und alle Geheimnisse weiß und
alle Erkenntnis habe, und wenn ich allen Glauben besitze, so daß ich
Berge versetze, habe aber keine Liebe, so bin ich nichts. \bibleverse{3}
Und wenn ich alle meine Habe austeile und meinen Leib hergebe, damit ich
verbrannt werde, habe aber keine Liebe, so nützt es mir nichts!
\bibleverse{4} Die Liebe ist langmütig und gütig, die Liebe beneidet
nicht, sie prahlt nicht, sie bläht sich nicht auf; \bibleverse{5} sie
ist nicht unanständig, sie sucht nicht das Ihre, sie läßt sich nicht
erbittern, sie rechnet das Böse nicht zu; \bibleverse{6} sie freut sich
nicht über die Ungerechtigkeit, sie freut sich aber der Wahrheit;
\bibleverse{7} sie erträgt alles, sie glaubt alles, sie hofft alles, sie
duldet alles. \bibleverse{8} Die Liebe hört nimmer auf, wo doch die
Prophezeiungen ein Ende haben werden, das Zungenreden aufhören wird und
die Erkenntnis aufgehoben werden soll. \bibleverse{9} Denn wir erkennen
stückweise und wir weissagen stückweise; \bibleverse{10} wenn aber
einmal das Vollkommene da ist, dann wird das Stückwerk abgetan.
\bibleverse{11} Als ich ein Kind war, redete ich wie ein Kind, dachte
wie ein Kind und urteilte wie ein Kind; als ich aber ein Mann wurde, tat
ich ab, was kindisch war. \bibleverse{12} Wir sehen jetzt durch einen
Spiegel wie im Rätsel, dann aber von Angesicht zu Angesicht; jetzt
erkenne ich stückweise, dann aber werde ich erkennen, gleichwie ich
erkannt bin. \bibleverse{13} Nun aber bleibt Glaube, Hoffnung, Liebe,
diese drei; die größte aber von diesen ist die Liebe.

\hypertarget{section-13}{%
\section{14}\label{section-13}}

\bibleverse{1} Strebet nach der Liebe; doch eifert auch nach den
Geistesgaben, am meisten aber, daß ihr weissagen könnet! \bibleverse{2}
Denn wer in Zungen redet, der redet nicht für Menschen, sondern für
Gott; denn niemand vernimmt es, im Geiste aber redet er Geheimnisse.
\bibleverse{3} Wer aber weissagt, der redet für Menschen zur Erbauung,
zur Ermahnung und zum Trost. \bibleverse{4} Wer in Zungen redet, erbaut
sich selbst; wer aber weissagt, erbaut die Gemeinde. \bibleverse{5} Ich
wünschte, daß ihr alle in Zungen redetet, noch viel mehr aber, daß ihr
weissagen könntet. Denn wer weissagt, ist größer, als wer in Zungen
redet; es sei denn, daß er es auslege, damit die Gemeinde Erbauung
empfange. \bibleverse{6} Nun aber, ihr Brüder, wenn ich zu euch käme und
in Zungen redete, was würde ich euch nützen, wenn ich nicht zu euch
redete, sei es durch Offenbarung oder durch Erkenntnis oder durch
Weissagung oder durch Lehre? \bibleverse{7} Ist es doch ebenso mit den
leblosen Instrumenten, die einen Laut von sich geben, sei es eine Flöte
oder eine Harfe; wenn sie nicht bestimmte Töne geben, wie kann man
erkennen, was auf der Flöte oder auf der Harfe gespielt wird?
\bibleverse{8} Ebenso auch, wenn die Posaune einen undeutlichen Ton
gibt, wer wird sich zum Kampfe rüsten? \bibleverse{9} Also auch ihr,
wenn ihr durch die Zunge nicht eine verständliche Rede gebet, wie kann
man verstehen, was geredet wird? Denn ihr werdet in den Wind reden.
\bibleverse{10} So viele Arten von Sprachen mögen wohl in der Welt sein,
und keine ist ohne Laut. \bibleverse{11} Wenn ich nun den Sinn des
Lautes nicht kenne, so werde ich dem Redenden ein Fremder sein und der
Redende für mich ein Fremder. \bibleverse{12} Also auch ihr, da ihr
eifrig nach Geistesgaben trachtet, suchet, zur Erbauung der Gemeinde
daran Überfluß zu haben! \bibleverse{13} Darum: wer in Zungen redet, der
bete, daß er es auch auslegen kann. \bibleverse{14} Denn wenn ich in
Zungen bete, so betet zwar mein Geist, aber mein Verstand ist ohne
Frucht. \bibleverse{15} Wie soll es nun sein? Ich will im Geiste beten,
ich will aber auch mit dem Verstande beten; ich will im Geiste
lobsingen, ich will aber auch mit dem Verstande lobsingen.
\bibleverse{16} Sonst, wenn du im Geiste lobpreisest, wie soll der,
welcher die Stelle des Unkundigen einnimmt, das Amen sprechen zu deiner
Danksagung, da er nicht weiß, was du sagst? \bibleverse{17} Du magst
wohl schön danksagen, aber der andere wird nicht erbaut. \bibleverse{18}
Ich danke Gott, daß ich mehr als ihr alle in Zungen rede.
\bibleverse{19} Aber in der Gemeinde will ich lieber fünf Worte mit
meinem Verstande reden, damit ich auch andere unterrichte, als
zehntausend Worte in Zungen. \bibleverse{20} Ihr Brüder, werdet nicht
Kinder im Verständnis, sondern an Bosheit seid Kinder, am Verständnis
aber werdet vollkommen. \bibleverse{21} Im Gesetz steht geschrieben:
``Ich will mit fremden Zungen und mit fremden Lippen zu diesem Volke
reden, aber auch so werden sie mich nicht hören, spricht der Herr.''
\bibleverse{22} Darum sind die Zungen zum Zeichen nicht für die
Gläubigen, sondern für die Ungläubigen; die Weissagung aber ist nicht
für die Ungläubigen, sondern für die Gläubigen. \bibleverse{23} Wenn nun
die ganze Gemeinde am selben Ort zusammenkäme, und alle würden in Zungen
reden, und es kämen Unkundige oder Ungläubige herein, würden sie nicht
sagen, ihr wäret von Sinnen? \bibleverse{24} Wenn aber alle weissagten,
und es käme ein Ungläubiger oder Unkundiger herein, so würde er von
allen überführt, von allen erforscht; \bibleverse{25} das Verborgene
seines Herzens würde offenbar, und so würde er auf sein Angesicht fallen
und Gott anbeten und bekennen, daß Gott wahrhaftig in euch sei.
\bibleverse{26} Wie ist es nun, ihr Brüder? Wenn ihr zusammenkommt, so
hat jeder von euch etwas: einen Psalm, eine Lehre, eine Offenbarung,
eine Zungenrede, eine Auslegung; alles geschehe zur Erbauung!
\bibleverse{27} Will jemand in Zungen reden, so seien es je zwei,
höchstens drei, und der Reihe nach, und einer lege es aus.
\bibleverse{28} Ist aber kein Ausleger da, so schweige er in der
Gemeinde; er rede aber für sich selbst und zu Gott. \bibleverse{29}
Propheten aber sollen zwei oder drei reden, und die andern sollen es
beurteilen. \bibleverse{30} Wenn aber einem andern, der dasitzt, eine
Offenbarung zuteil wird, so soll der erste schweigen. \bibleverse{31}
Denn ihr könnet einer nach dem andern alle weissagen, damit alle lernen
und alle getröstet werden. \bibleverse{32} Und die Geister der Propheten
sind den Propheten untertan. \bibleverse{33} Denn Gott ist nicht ein
Gott der Unordnung, sondern des Friedens. \bibleverse{34} Wie in allen
Gemeinden der Heiligen, so sollen die Frauen in den Gemeinden schweigen;
denn es ist ihnen nicht gestattet zu reden, sondern sie sollen untertan
sein, wie auch das Gesetz sagt. \bibleverse{35} Wollen sie aber etwas
lernen, so mögen sie daheim ihre Männer fragen; denn es steht einem
Weibe übel an, in der Gemeinde zu reden. \bibleverse{36} Oder ist von
euch das Wort Gottes ausgegangen? Oder ist es zu euch allein gekommen?
\bibleverse{37} Glaubt jemand ein Prophet oder ein Geistbegabter zu
sein, der erkenne, daß das, was ich euch schreibe, des Herrn Gebot ist.
\bibleverse{38} Will es aber jemand mißachten, der mißachte es!
\bibleverse{39} Also, meine Brüder, strebet nach der Weissagung, und das
Reden in Zungen wehret nicht; \bibleverse{40} alles aber geschehe mit
Anstand und in Ordnung!

\hypertarget{section-14}{%
\section{15}\label{section-14}}

\bibleverse{1} Ich mache euch aber, ihr Brüder, auf das Evangelium
aufmerksam, das ich euch gepredigt habe, welches ihr auch angenommen
habt, in welchem ihr auch stehet; \bibleverse{2} durch welches ihr auch
gerettet werdet, wenn ihr an dem Worte festhaltet, das ich euch
verkündigt habe, es wäre denn, daß ihr vergeblich geglaubt hättet.
\bibleverse{3} Denn ich habe euch in erster Linie das überliefert, was
ich auch empfangen habe, nämlich daß Christus für unsre Sünden gestorben
ist, nach der Schrift, \bibleverse{4} und daß er begraben worden und daß
er auferstanden ist am dritten Tage, nach der Schrift, \bibleverse{5}
und daß er dem Kephas erschienen ist, hernach den Zwölfen.
\bibleverse{6} Darnach ist er mehr als fünfhundert Brüdern auf einmal
erschienen, von welchen die meisten noch leben, etliche aber auch
entschlafen sind. \bibleverse{7} Darnach erschien er dem Jakobus,
hierauf sämtlichen Aposteln. \bibleverse{8} Zuletzt aber von allen
erschien er auch mir, der ich gleichsam eine unzeitige Geburt bin.
\bibleverse{9} Denn ich bin der geringste von den Aposteln, nicht wert
ein Apostel zu heißen, weil ich die Gemeinde Gottes verfolgt habe.
\bibleverse{10} Aber durch Gottes Gnade bin ich, was ich bin, und seine
Gnade gegen mich ist nicht vergeblich gewesen, sondern ich habe mehr
gearbeitet als sie alle; nicht aber ich, sondern die Gnade Gottes, die
mit mir ist. \bibleverse{11} Ob es nun aber ich sei oder jene, so
predigen wir, und so habt ihr geglaubt. \bibleverse{12} Wenn aber
Christus gepredigt wird, daß er von den Toten auferstanden sei, wie
sagen denn etliche unter euch, es gebe keine Auferstehung der Toten?
\bibleverse{13} Gibt es wirklich keine Auferstehung der Toten, so ist
auch Christus nicht auferstanden! \bibleverse{14} Ist aber Christus
nicht auferstanden, so ist also unsre Predigt vergeblich, vergeblich
auch euer Glaube! \bibleverse{15} Wir werden aber auch als falsche
Zeugen Gottes erfunden, weil wir wider Gott gezeugt haben, er habe
Christus auferweckt, während er ihn doch nicht auferweckt hat, wenn also
Tote nicht auferstehen! \bibleverse{16} Denn wenn Tote nicht
auferstehen, so ist auch Christus nicht auferstanden. \bibleverse{17}
Ist aber Christus nicht auferstanden, so ist euer Glaube nichtig, so
seid ihr noch in euren Sünden; \bibleverse{18} dann sind auch die in
Christus Entschlafenen verloren. \bibleverse{19} Hoffen wir allein in
diesem Leben auf Christus, so sind wir die elendesten unter allen
Menschen! \bibleverse{20} Nun aber ist Christus von den Toten
auferstanden, als Erstling der Entschlafenen. \bibleverse{21} Denn weil
der Tod kam durch einen Menschen, so kommt auch die Auferstehung der
Toten durch einen Menschen; \bibleverse{22} denn gleichwie in Adam alle
sterben, so werden auch in Christus alle lebendig gemacht werden.
\bibleverse{23} Ein jeglicher aber in seiner Ordnung: Als Erstling
Christus, darnach die, welche Christus angehören, bei seiner
Wiederkunft; \bibleverse{24} hernach das Ende, wenn er das Reich Gott
und dem Vater übergibt, wenn er abgetan hat jede Herrschaft, Gewalt und
Macht. \bibleverse{25} Denn er muß herrschen, ``bis er alle Feinde unter
seine Füße gelegt hat''. \bibleverse{26} Als letzter Feind wird der Tod
abgetan. \bibleverse{27} Denn ``alles hat er unter seine Füße getan''.
Wenn er aber sagt, daß ihm alles unterworfen sei, so ist offenbar, daß
der ausgenommen ist, welcher ihm alles unterworfen hat. \bibleverse{28}
Wenn ihm aber alles unterworfen sein wird, dann wird auch der Sohn
selbst sich dem unterwerfen, der ihm alles unterworfen hat, auf daß Gott
sei alles in allen. \bibleverse{29} Was würden sonst die tun, welche
sich für die Toten taufen lassen? Wenn die Toten gar nicht auferstehen,
was lassen sie sich für die Toten taufen? \bibleverse{30} Und warum
stehen auch wir stündlich in Gefahr? \bibleverse{31} Täglich sterbe ich,
ja, sowahr ihr, Brüder, mein Ruhm seid, den ich in Christus Jesus habe,
unserm Herrn! \bibleverse{32} Habe ich als Mensch zu Ephesus mit wilden
Tieren gekämpft, was nützt es mir? Wenn die Toten nicht auferstehen, so
``lasset uns essen und trinken, denn morgen sind wir tot!''
\bibleverse{33} Lasset euch nicht irreführen: Schlechte Gesellschaften
verderben gute Sitten. \bibleverse{34} Werdet ganz nüchtern und sündiget
nicht! Denn etliche haben keine Erkenntnis Gottes; das sage ich euch zur
Beschämung. \bibleverse{35} Aber, wird jemand sagen, wie sollen die
Toten auferstehen? Mit was für einem Leibe sollen sie kommen?
\bibleverse{36} Du Gedankenloser, was du säst, wird nicht lebendig, es
sterbe denn! \bibleverse{37} Und was du säst, das ist ja nicht der Leib,
der werden soll, sondern ein bloßes Korn, etwa von Weizen, oder von
einer andern Frucht. \bibleverse{38} Gott aber gibt ihm einen Leib, wie
er es gewollt hat, und zwar einem jeglichen Samen seinen besonderen
Leib. \bibleverse{39} Nicht alles Fleisch ist von gleicher Art; sondern
anders ist das der Menschen, anders das Fleisch vom Vieh, anders das
Fleisch der Vögel, anders das der Fische. \bibleverse{40} Und es gibt
himmlische Körper und irdische Körper; aber anders ist der Glanz der
Himmelskörper, anders der der irdischen; \bibleverse{41} einen andern
Glanz hat die Sonne und einen andern Glanz der Mond, und einen andern
Glanz haben die Sterne; denn ein Stern unterscheidet sich vom andern
durch den Glanz. \bibleverse{42} So ist es auch mit der Auferstehung der
Toten: Es wird gesät verweslich und wird auferstehen unverweslich;
\bibleverse{43} es wird gesät in Unehre und wird auferstehen in
Herrlichkeit; es wird gesät in Schwachheit und wird auferstehen in
Kraft; \bibleverse{44} es wird gesät ein natürlicher Leib und wird
auferstehen ein geistiger Leib. Gibt es einen natürlichen Leib, so gibt
es auch einen geistigen Leib. \bibleverse{45} So steht auch geschrieben:
Der erste Mensch, Adam, wurde zu einer lebendigen Seele; der letzte Adam
zu einem lebendigmachenden Geiste. \bibleverse{46} Aber nicht das
Geistige ist das erste, sondern das Seelische, darnach kommt das
Geistige. \bibleverse{47} Der erste Mensch ist von Erde, irdisch; der
zweite Mensch ist der Herr vom Himmel. \bibleverse{48} Wie der Irdische
beschaffen ist, so sind auch die Irdischen; und wie der Himmlische
beschaffen ist, so sind auch die Himmlischen. \bibleverse{49} Und wie
wir das Bild des Irdischen getragen haben, so werden wir auch das Bild
des Himmlischen tragen. \bibleverse{50} Das aber sage ich, Brüder, daß
Fleisch und Blut das Reich Gottes nicht ererben können; auch wird das
Verwesliche nicht ererben die Unverweslichkeit. \bibleverse{51} Siehe,
ich sage euch ein Geheimnis: Wir werden nicht alle entschlafen, wir
werden aber alle verwandelt werden, \bibleverse{52} plötzlich, in einem
Augenblick, zur Zeit der letzten Posaune; denn die Posaune wird
erschallen, und die Toten werden auferstehen unverweslich, und wir
werden verwandelt werden. \bibleverse{53} Denn dieses Verwesliche muß
anziehen Unverweslichkeit, und dieses Sterbliche muß anziehen
Unsterblichkeit. \bibleverse{54} Wenn aber dieses Verwesliche
Unverweslichkeit anziehen und dieses Sterbliche Unsterblichkeit anziehen
wird, dann wird das Wort erfüllt werden, das geschrieben steht:
\bibleverse{55} ``Der Tod ist verschlungen in Sieg! Tod, wo ist dein
Stachel? Totenreich, wo ist dein Sieg?'' \bibleverse{56} Aber der
Stachel des Todes ist die Sünde, die Kraft der Sünde aber ist das
Gesetz. \bibleverse{57} Gott aber sei Dank, der uns den Sieg gibt durch
unsern Herrn Jesus Christus! \bibleverse{58} Darum, meine geliebten
Brüder, seid fest, unbeweglich, nehmet immer zu in dem Werke des Herrn,
weil ihr wisset, daß eure Arbeit nicht vergeblich ist in dem Herrn!

\hypertarget{section-15}{%
\section{16}\label{section-15}}

\bibleverse{1} Was aber die Sammlung für die Heiligen anbelangt, so
handelt auch ihr so, wie ich es für die Gemeinden in Galatien angeordnet
habe. \bibleverse{2} An jedem ersten Wochentag lege ein jeder unter euch
etwas beiseite und sammle, je nachdem es ihm wohl geht; damit nicht erst
dann, wenn ich komme, die Sammlungen gemacht werden müssen.
\bibleverse{3} Wenn ich aber angekommen bin, will ich die, welche ihr
als geeignet erachtet, mit Briefen absenden, damit sie eure Liebesgabe
nach Jerusalem überbringen. \bibleverse{4} Wenn es aber der Mühe wert
ist, daß auch ich hinreise, sollen sie mit mir reisen. \bibleverse{5}
Ich werde aber zu euch kommen, wenn ich Mazedonien durchzogen habe, denn
durch Mazedonien werde ich ziehen. \bibleverse{6} Bei euch aber werde
ich vielleicht verweilen oder auch überwintern, damit ihr mich geleitet,
wohin ich reise. \bibleverse{7} Denn ich will euch jetzt nicht nur im
Vorbeigehen sehen, sondern ich hoffe, einige Zeit bei euch zu bleiben,
wenn der Herr es zuläßt. \bibleverse{8} Ich werde aber zu Ephesus
bleiben bis Pfingsten. \bibleverse{9} Denn eine Tür hat sich mir
aufgetan, weit und vielversprechend; auch der Widersacher sind viele.
\bibleverse{10} Wenn aber Timotheus kommt, so sehet zu, daß er ohne
Furcht bei euch sei, denn er treibt des Herrn Werk, wie ich auch.
\bibleverse{11} Darum soll ihn niemand geringschätzen! Geleitet ihn
vielmehr in Frieden, damit er zu mir komme; denn ich erwarte ihn mit den
Brüdern. \bibleverse{12} Was aber den Bruder Apollos betrifft, so habe
ich ihn vielfach ermahnt, mit den Brüdern zu euch zu kommen; doch er war
durchaus nicht willens, jetzt zu kommen. Er wird aber kommen, wenn die
Zeit es ihm erlaubt. \bibleverse{13} Wachet, stehet fest im Glauben,
seid männlich, seid stark! \bibleverse{14} Möge alles bei euch in Liebe
geschehen! \bibleverse{15} Ich ermahne euch aber, ihr Brüder: Ihr kennet
das Haus des Stephanas, daß es die Erstlingsfrucht von Achaja ist, und
daß sie sich dem Dienste der Heiligen gewidmet haben; \bibleverse{16}
seid auch ihr solchen untertan und einem jeden, der mitwirkt und
arbeitet. \bibleverse{17} Ich freue mich aber über die Ankunft des
Stephanas und Fortunatus und Achaikus; denn diese haben mir ersetzt, daß
ich euer ermangeln muß; \bibleverse{18} denn sie haben meinen Geist und
den eurigen erquickt. Darum erkennet solche Männer an! \bibleverse{19}
Es grüßen euch die Gemeinden in Asien. Es grüßen euch vielmals im Herrn
Aquila und Priscilla samt der Gemeinde in ihrem Hause. \bibleverse{20}
Es grüßen euch die Brüder alle. Grüßet euch untereinander mit dem
heiligen Kuß! \bibleverse{21} Das ist mein, des Paulus,
handschriftlicher Gruß. \bibleverse{22} So jemand den Herrn Jesus
Christus nicht liebt, der sei verflucht! Maranatha! \bibleverse{23} Die
Gnade des Herrn Jesus Christus sei mit euch! \bibleverse{24} Meine Liebe
sei mit euch allen in Christus Jesus! Amen.
