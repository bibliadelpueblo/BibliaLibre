\hypertarget{section}{%
\section{1}\label{section}}

\bibleverse{1} Als aber der König David alt und hochbetagt war, konnte
er nicht warm werden, obgleich man ihn mit Kleidern bedeckte.
\bibleverse{2} Da sprachen seine Knechte zu ihm: Man sollte unserm
Herrn, dem König, ein Mädchen, eine Jungfrau suchen, daß sie vor dem
König stehe und seiner pflege und an seinem Busen schlafe und unsern
Herrn, den König, wärme. \bibleverse{3} Und sie suchten ein schönes
Mädchen in allen Landmarken Israels und fanden Abisag von Sunem, die
brachten sie dem König. \bibleverse{4} Sie war aber ein sehr schönes
Mädchen und pflegte den König und diente ihm. Aber der König erkannte
sie nicht. \bibleverse{5} Adonia aber, der Sohn der Haggit, erhob sich
und sprach: Ich will König werden! Und er verschaffte sich Wagen und
Reiter und fünfzig Mann, die vor ihm herliefen. \bibleverse{6} Aber sein
Vater hatte ihn nie betrübt Zeit seines Lebens, so daß er gesagt hätte:
Warum tust du also? Auch war er sehr schön von Gestalt; und seine Mutter
hatte ihn nach Absalom geboren. \bibleverse{7} Und er hatte eine
Unterredung mit Joab, dem Sohne der Zeruja, und mit Abjatar, dem
Priester; die halfen dem Adonia. \bibleverse{8} Aber der Priester Zadok
und Benaja, Jojadas Sohn, und der Prophet Natan und Simei und Rei und
die Helden Davids hielten es nicht mit Adonia. \bibleverse{9} Und als
Adonia Schafe und Rinder und Mastvieh opferte bei dem Stein Sochelet,
der neben dem Brunnen Rogel liegt, lud er alle seine Brüder, des Königs
Söhne, und alle Männer Judas, des Königs Knechte, ein. \bibleverse{10}
Aber den Propheten Natan und Benaja und die Helden und seinen Bruder
Salomo lud er nicht ein. \bibleverse{11} Da sprach Natan zu Batseba, der
Mutter Salomos: Hast du nicht gehört, daß Adonia, der Sohn der Haggit,
König geworden ist ohne Wissen Davids, unseres Herrn? \bibleverse{12}
Komm nun, ich will dir doch einen Rat geben, daß du dein Leben und das
Leben deines Sohnes Salomo errettest. \bibleverse{13} Komm und gehe
hinein zum König David und sprich zu ihm: ``Hast du nicht, mein Herr und
König, deiner Magd geschworen und gesagt: Dein Sohn Salomo soll König
sein nach mir, und er soll auf meinem Throne sitzen? Warum ist denn
Adonia König geworden?'' \bibleverse{14} Siehe, während du noch dort
bist und mit dem König redest, will ich nach dir hineinkommen und deine
Worte bestätigen. \bibleverse{15} Da ging Batseba zum König in die
Kammer hinein. Der König aber war sehr alt, und Abisag von Sunem diente
dem König. \bibleverse{16} Und Batseba neigte und verbeugte sich vor dem
König. Der König aber sprach: Was willst du? \bibleverse{17} Sie sprach
zu ihm: Mein Herr, du hast deiner Magd bei dem \textsc{Herrn}, deinem
Gott, geschworen: Dein Sohn Salomo soll König sein nach mir, und er soll
auf meinem Throne sitzen! \bibleverse{18} Nun aber, siehe, ist Adonia
König geworden; und mein Herr und König weiß nichts darum.
\bibleverse{19} Er hat Ochsen und Mastvieh und viele Schafe geopfert und
hat alle Söhne des Königs eingeladen, dazu Abjatar, den Priester, und
Joab, den Feldhauptmann. Aber deinen Knecht Salomo hat er nicht
eingeladen. \bibleverse{20} Du bist es aber, mein Herr und König, auf
den die Augen von ganz Israel sehen, daß du anzeigest, wer nach meinem
Herrn und König auf seinem Throne sitzen soll. \bibleverse{21} Wenn aber
mein Herr und König bei seinen Vätern liegt, so werden ich und mein Sohn
Salomo es büßen müssen! \bibleverse{22} Während sie noch also mit dem
König redete, siehe, da kam der Prophet Natan. \bibleverse{23} Da
meldete man dem König und sprach: Siehe, der Prophet Natan ist da! Und
als er vor den König hineinkam, bückte er sich vor dem König mit dem
Angesicht zur Erde. \bibleverse{24} Und Natan sprach: Mein Herr und
König, hast du gesagt: ``Adonia soll nach mir König sein und soll auf
meinem Throne sitzen?'' \bibleverse{25} Denn er ist heute hinabgegangen
und hat Ochsen und Mastvieh und viele Schafe geopfert und hat alle Söhne
des Königs eingeladen und die Hauptleute, dazu den Priester Abjatar. Und
siehe, sie essen und trinken vor ihm und sagen: Es lebe der König
Adonia! \bibleverse{26} Aber mich, deinen Knecht, und Zadok, den
Priester, und Benaja, den Sohn Jojadas, und deinen Knecht Salomo hat er
nicht eingeladen. \bibleverse{27} Ist das alles von meinem Herrn, dem
König, befohlen worden, und hast du deinen Knecht nicht wissen lassen,
wer auf dem Throne meines Herrn, des Königs, nach ihm sitzen soll?
\bibleverse{28} Der König David antwortete und sprach: Rufe mir Batseba!
Und sie kam hinein vor den König. \bibleverse{29} Und als sie vor dem
König stand, schwur der König und sprach: So wahr der \textsc{Herr}
lebt, der meine Seele aus aller Not erlöst hat, \bibleverse{30} ich will
heute also tun, wie ich dir bei dem \textsc{Herrn}, dem Gott Israels,
geschworen und gesagt habe: Salomo, dein Sohn, soll König nach mir sein,
und er soll für mich auf meinem Throne sitzen! \bibleverse{31} Da
verneigte sich Batseba mit ihrem Angesicht zur Erde und dankte dem König
und sprach: Mein Herr, der König David, lebe ewiglich! \bibleverse{32}
Und der König David sprach: Ruft mir den Priester Zadok und den
Propheten Natan und Benaja, den Sohn Jojadas! Und als sie vor den König
hineinkamen, \bibleverse{33} sprach der König zu ihnen: Nehmt die
Knechte eures Herrn mit euch und setzet meinen Sohn Salomo auf mein
Maultier und führet ihn hinab gen Gihon. \bibleverse{34} Und der
Priester Zadok und der Prophet Natan sollen ihn daselbst salben zum
König über Israel; und stoßet in die Posaune und sprechet: Es lebe der
König Salomo! \bibleverse{35} Und ziehet hinter ihm herauf, und er soll
kommen und auf meinem Throne sitzen und für mich König sein; denn ich
habe verordnet, daß er Fürst über Israel und Juda sei. \bibleverse{36}
Da antwortete Benaja, der Sohn Jojadas, dem König und sprach: Amen! Der
\textsc{Herr}, der Gott meines Herrn, des Königs, sage auch also!
\bibleverse{37} Wie der \textsc{Herr} mit meinem Herrn, dem König,
gewesen ist, so sei er auch mit Salomo, und er mache seinen Thron noch
größer als den Thron meines Herrn, des Königs David! \bibleverse{38} Da
gingen der Priester Zadok und der Prophet Natan und Benaja, der Sohn
Jojadas, und die Kreter und Pleter hinab und setzten Salomo auf das
Maultier des Königs David und führten ihn gen Gihon. \bibleverse{39} Und
der Priester Zadok nahm das Ölhorn aus dem Zelte und salbte Salomo, und
sie stießen in die Posaune, und alles Volk sprach: Es lebe der König
Salomo! \bibleverse{40} Und alles Volk zog hinter ihm herauf, und das
Volk blies auf Flöten und war sehr fröhlich, so daß die Erde von ihrem
Geschrei erzitterte. \bibleverse{41} Adonia aber hörte es samt allen
Gästen, die bei ihm waren, da sie eben das Mahl beendigt hatten. Als
aber Joab den Schall der Posaune hörte, sprach er: Was soll das Geschrei
in der Stadt? \bibleverse{42} Als er aber noch redete, siehe, da kam
Jonatan, der Sohn des Priesters Abjatar. Und Adonia sprach: Komm herein;
denn du bist ein wackerer Mann und bringst eine gute Botschaft!
\bibleverse{43} Jonatan aber antwortete und sprach zu Adonia: Fürwahr,
unser Herr, der König David, hat Salomo zum König gemacht!
\bibleverse{44} Und der König hat mit ihm gesandt den Priester Zadok und
den Propheten Natan und Benaja, den Sohn Jojadas, und die Kreter und
Pleter, und sie haben ihn auf des Königs Maultier gesetzt.
\bibleverse{45} Und der Priester Zadok und der Prophet Natan haben ihn
zum König gesalbt zu Gihon, und sie sind mit Freuden von dannen
heraufgezogen, so daß die ganze Stadt in Bewegung ist. Das ist das
Geschrei, das ihr gehört habt. \bibleverse{46} Dazu sitzt Salomo auf dem
königlichen Throne. \bibleverse{47} Und auch die Knechte des Königs sind
hineingegangen, unserm Herrn, dem König David, Glück zu wünschen, und
sie haben gesagt: ``Dein Gott mache den Namen Salomos noch herrlicher
als deinen Namen und mache seinen Thron noch größer als deinen Thron!''.
Und der König hat sich auf seinem Lager verneigt! \bibleverse{48} Zudem
hat der König also gesagt: Gelobet sei der \textsc{Herr}, der Gott
Israels, der mir heute einen Thronerben bestellt hat vor meinen Augen!
\bibleverse{49} Da erschraken die Gäste, die bei Adonia waren, und
machten sich auf und gingen ein jeder seines Weges. \bibleverse{50}
Adonia aber fürchtete sich vor Salomo und machte sich auf, ging hin und
ergriff die Hörner des Altars. \bibleverse{51} Das meldete man Salomo
und sprach: Siehe, Adonia fürchtet den König Salomo; und siehe, er hält
sich an den Hörnern des Altars und spricht: Der König Salomo schwöre mir
heute, daß er seinen Knecht nicht mit dem Schwerte töten wolle!
\bibleverse{52} Salomo sprach: Wird er sich wacker halten, so soll kein
Haar von ihm auf die Erde fallen; wird aber Böses an ihm gefunden, so
muß er sterben! \bibleverse{53} Und der König Salomo sandte hin und ließ
ihn vom Altar herabholen. Und als er kam, fiel er vor dem König Salomo
nieder. Salomo aber sprach zu ihm: Gehe hin in dein Haus!

\hypertarget{section-1}{%
\section{2}\label{section-1}}

\bibleverse{1} Als nun die Zeit kam, daß David sterben sollte, gebot er
seinem Sohne Salomo und sprach: \bibleverse{2} Ich gehe hin den Weg
aller Welt. So sei nun stark und sei ein Mann \bibleverse{3} und
beobachte die Verordnungen des \textsc{Herrn}, deines Gottes, daß du in
seinen Wegen wandelst, seine Satzungen, seine Gebote, seine Rechte und
seine Zeugnisse haltest, wie im Gesetze Moses geschrieben steht, auf daß
du weislich vollbringest alles, was du tust und wohin du dich wendest;
\bibleverse{4} daß der \textsc{Herr} sein Wort bestätige, das er über
mich geredet hat, indem er sagte: Werden deine Kinder auf ihre Wege
achten, daß sie in Wahrheit vor mir wandeln, mit ihrem ganzen Herzen und
mit ihrer ganzen Seele, so soll es dir nimmer (sprach er) an einem Manne
fehlen auf dem Throne Israels! \bibleverse{5} Du weißt aber auch, was
mir Joab, der Sohn der Zeruja, getan hat, wie er an den beiden
Heerführern Israels, an Abner, dem Sohne Ners, und an Amasa, dem Sohne
Jeters, gehandelt hat, wie er sie umgebracht und also Kriegsblut mitten
im Frieden vergossen und Kriegsblut an seinen Gürtel getan hat, der um
seine Lenden war, und an die Schuhe, die an seinen Füßen waren.
\bibleverse{6} So handle nun nach deiner Weisheit, daß du seine grauen
Haare nicht in Frieden ins Totenreich fahren lässest! \bibleverse{7}
Aber den Kindern Barsillais, des Gileaditers, sollst du Barmherzigkeit
erweisen, daß sie unter denen seien, die an deinem Tische essen, denn
ebenso hielten sie sich zu mir, als ich vor deinem Bruder Absalom floh.
\bibleverse{8} Und siehe, du hast bei dir Simei, den Sohn Geras, den
Benjaminiter, von Bachurim, der mir bitter und schändlich fluchte zu der
Zeit, als ich nach Mahanaim ging. Als er aber dann an den Jordan herab
mir entgegenkam, da schwur ich ihm bei dem \textsc{Herrn} und sprach:
Ich will dich nicht mit dem Schwerte töten! \bibleverse{9} Nun aber laß
du ihn nicht ungestraft; denn du bist ein weiser Mann und wirst wohl
wissen, was du ihm tun sollst, daß du seine grauen Haare mit Blut ins
Totenreich hinunter bringest. \bibleverse{10} Und David entschlief mit
seinen Vätern und ward begraben in der Stadt Davids. \bibleverse{11} Die
Zeit aber, die David über Israel regierte, betrug vierzig Jahre. Sieben
Jahre lang war er König zu Hebron und dreiunddreißig Jahre lang zu
Jerusalem. \bibleverse{12} Und Salomo saß auf dem Throne seines Vaters
David, und sein Königtum ward fest gegründet. \bibleverse{13} Adonia
aber, der Sohn der Haggit, kam hinein zu Batseba, der Mutter Salomos.
Und sie sprach: Kommst du auch in Frieden? Er sprach: Ja, in Frieden!
\bibleverse{14} Und er sprach: Ich habe mit dir zu reden.
\bibleverse{15} Sie sprach: Sage her! Er sprach: Du weißt, daß das
Königtum mein war und daß ganz Israel sein Angesicht auf mich gerichtet
hatte, daß ich König sein sollte; nun aber ist mir das Königtum
entgangen und meinem Bruder zugefallen; denn es war ihm vom
\textsc{Herrn} bestimmt. \bibleverse{16} Nun habe ich eine Bitte an
dich; die wollest du mir nicht abschlagen. Sie sprach zu ihm:
\bibleverse{17} Sage her! Er sprach: Rede doch mit dem König Salomo
(denn dich wird er nicht abweisen), daß er mir Abisag von Sunem zum
Weibe gebe. \bibleverse{18} Batseba sprach: Gut, ich will deinetwegen
mit dem König reden! \bibleverse{19} Also kam Batseba hinein zum König
Salomo, mit ihm zu reden wegen Adonia. Und der König stand auf und ging
ihr entgegen und verneigte sich vor ihr und setzte sich auf seinen
Thron. Und auch der Mutter des Königs ward ein Thron hingestellt, daß
sie sich zu seiner Rechten setzte. \bibleverse{20} Und sie sprach: Ich
habe eine kleine Bitte an dich, die wollest du mir nicht abschlagen! Der
König sprach zu ihr: Bitte, meine Mutter; denn dich werde ich nicht
abweisen! \bibleverse{21} Sie sprach: Man gebe Abisag von Sunem deinem
Bruder Adonia zum Weibe! \bibleverse{22} Da antwortete der König Salomo
und sprach zu seiner Mutter: Und warum bittest du für Adonia um Abisag
von Sunem? Verlange für ihn auch das Königreich; denn er ist mein
älterer Bruder und hat Abjatar, den Priester, und Joab, den Sohn der
Zeruja, auf seiner Seite! \bibleverse{23} Und der König Salomo schwur
bei dem \textsc{Herrn} und sprach: Gott tue mir dies und das! Adonia
soll das wider sein Leben geredet haben! \bibleverse{24} Und nun, so
wahr der \textsc{Herr} lebt, der mich bestätigt und mich auf den Thron
meines Vaters David gesetzt und mir ein Haus gemacht, wie er gesagt hat:
Heute soll Adonia sterben! \bibleverse{25} Und der König Salomo sandte
Benaja, den Sohn Jojadas; der schlug ihn, daß er starb. \bibleverse{26}
Und zu dem Priester Abjatar sprach der König: Gehe hin nach Anatot, auf
deinen Acker; denn du bist ein Mann des Todes; aber ich will dich heute
nicht töten, denn du hast die Lade Gottes, des \textsc{Herrn}, getragen
vor meinem Vater David und hast mitgelitten alles, was mein Vater
gelitten hat. \bibleverse{27} Also verstieß Salomo den Abjatar, daß er
nicht mehr Priester des \textsc{Herrn} sein durfte, wodurch das Wort des
\textsc{Herrn} erfüllt wurde, das er zu Silo über das Haus Elis geredet
hatte. \bibleverse{28} Und das Gerücht davon kam vor Joab; denn Joab
hing an Adonia, während er sich nicht zu Absalom gehalten hatte. Da floh
Joab in das Zelt des \textsc{Herrn} und faßte die Hörner des Altars.
\bibleverse{29} Und es ward dem König Salomo gesagt: Joab ist zum Zelte
des \textsc{Herrn} geflohen und siehe, er steht am Altar! Da sandte
Salomo Benaja, den Sohn Jojadas, und sprach: Geh, erschlage ihn!
\bibleverse{30} Als nun Benaja zum Zelte des \textsc{Herrn} kam, sprach
er zu ihm: So spricht der König: Gehe heraus! Er sprach: Nein, sondern
hier will ich sterben! Und Benaja sagte solches dem König wieder und
sprach: Also hat Joab gesprochen, und also hat er mir geantwortet!
\bibleverse{31} Der König sprach zu ihm: Tue, wie er gesagt hat;
erschlage ihn und begrabe ihn, daß du das Blut, das Joab ohne Grund
vergossen hat, von mir und meines Vaters Hause wendest \bibleverse{32}
und daß der \textsc{Herr} sein Blut auf seinen eigenen Kopf kommen
lasse, weil er zwei Männer erschlagen hat, die gerechter und besser
waren als er, und sie mit dem Schwert umgebracht hat, da mein Vater
David nichts darum wußte: nämlich Abner, den Sohn Ners, den
Feldhauptmann Israels, und Amasa, den Sohn Jeters, den Feldhauptmann
Judas. \bibleverse{33} Ihr Blut komme auf Joabs Kopf und auf den Kopf
seines Samens ewiglich; David aber und sein Same, sein Haus und sein
Thron, habe ewiglich Frieden von dem \textsc{Herrn}! \bibleverse{34} Da
ging Benaja, der Sohn Jojadas, hinauf und schlug ihn und tötete ihn; und
er ward in seinem Hause begraben in der Wüste. \bibleverse{35} Da setzte
der König Benaja, den Sohn Jojadas, an seine Statt über das Heer; den
Priester Zadok aber setzte der König an Abjatars Statt. \bibleverse{36}
Und der König sandte hin und ließ Simei rufen und sprach zu ihm: Baue
dir ein Haus zu Jerusalem und wohne daselbst und gehe nicht von dannen
heraus, weder hierhin noch dorthin! \bibleverse{37} An welchem Tage du
hinausgehen und den Bach Kidron überschreiten wirst, sollst du wissen,
daß du gewiß sterben mußt; dein Blut sei auf deinem Kopf!
\bibleverse{38} Simei sprach zum König: Das Wort ist gut; wie mein Herr,
der König, gesagt hat, so wird dein Knecht tun! Also wohnte Simei zu
Jerusalem lange Zeit. \bibleverse{39} Es begab sich aber nach drei
Jahren, daß dem Simei zwei Knechte entliefen zu Achis, dem Sohne
Maachas, dem König zu Gat. Und es ward dem Simei angezeigt: Siehe, deine
Knechte sind zu Gat! \bibleverse{40} Da machte sich Simei auf und
sattelte seinen Esel und ritt nach Gat zu Achis, um seine Knechte zu
suchen. Und Simei kam wieder und brachte seine Knechte von Gat zurück.
\bibleverse{41} Da ward dem Salomo angezeigt, daß Simei von Jerusalem
nach Gat gegangen und wiedergekommen sei. \bibleverse{42} Da sandte der
König hin und ließ Simei rufen und sprach zu ihm: Habe ich von dir nicht
einen Eid genommen bei dem \textsc{Herrn} und dir bezeugt und gesagt: An
welchem Tage du ausziehen und hierhin oder dorthin gehen wirst, mußt du
gewiß sterben? Und du sprachst zu mir: Das Wort ist gut; ich habe es
gehört! \bibleverse{43} Warum hast du dich denn nicht gehalten an den
Eid bei dem \textsc{Herrn} und an das Gebot, das ich dir gegeben habe?
\bibleverse{44} Und der König sprach zu Simei: Du weißt alle Bosheit,
deren dein Herz bewußt ist, die du meinem Vater David zugefügt hast. So
möge nun der \textsc{Herr} deine Bosheit auf deinen eigenen Kopf kommen
lassen! \bibleverse{45} Aber der König Salomo sei gesegnet, und der
Thron Davids stehe fest vor dem \textsc{Herrn} ewiglich! \bibleverse{46}
Und der König gebot Benaja, dem Sohne Jojadas; der ging hinaus und
schlug ihn, daß er starb.

\hypertarget{section-2}{%
\section{3}\label{section-2}}

\bibleverse{1} Als nun die Königsherrschaft in Salomos Hand befestigt
war, verschwägerte sich Salomo mit dem Pharao, dem König von Ägypten,
und nahm die Tochter des Pharao und brachte sie in die Stadt Davids, bis
er sein Haus und das Haus des \textsc{Herrn} und die Mauern um Jerusalem
her fertiggebaut hatte. \bibleverse{2} Nur opferte das Volk noch auf den
Höhen; denn dem Namen des \textsc{Herrn} war noch kein Haus gebaut bis
auf jene Zeit. \bibleverse{3} Salomo aber liebte den \textsc{Herrn}, so
daß er in den Satzungen seines Vaters David wandelte; nur opferte und
räucherte er auf den Höhen. \bibleverse{4} Und der König ging nach
Gibeon, um daselbst zu opfern; denn das war die bedeutendste Höhe. Und
Salomo opferte tausend Brandopfer auf jenem Altar. \bibleverse{5} Zu
Gibeon erschien der \textsc{Herr} dem Salomo des Nachts im Traume. Und
Gott sprach: Bitte, was ich dir geben soll! \bibleverse{6} Salomo
sprach: Du hast deinem Knechte, meinem Vater David, große Gnade
erwiesen, wie er denn vor dir gewandelt ist in Wahrheit und
Gerechtigkeit und mit aufrichtigem Herzen gegen dich, und du hast ihm
diese große Gnade bewahrt und ihm einen Sohn gegeben, der heute auf
seinem Throne sitzt. \bibleverse{7} Weil nun du, o \textsc{Herr}, mein
Gott, deinen Knecht an meines Vaters David Statt zum König gemacht hast,
ich aber ein junger Knabe bin, der weder ein noch auszugehen weiß;
\bibleverse{8} und weil dein Knecht mitten unter deinem Volke ist, das
du erwählt hast, welches so groß ist, daß es niemand zählen noch vor
Menge berechnen kann; \bibleverse{9} so wollest du deinem Knecht ein
verständiges Herz geben, daß er dein Volk zu richten wisse und
unterscheiden könne, was gut und böse ist. Denn wer vermag dieses dein
ansehnliches Volk zu richten? \bibleverse{10} Das gefiel dem
\textsc{Herrn} wohl, daß Salomo um solches bat. \bibleverse{11} Und Gott
sprach zu ihm: Weil du um solches bittest, und bittest nicht um Reichtum
und bittest nicht um das Leben deiner Feinde, sondern bittest um
Verstand zur Rechtspflege, \bibleverse{12} siehe, so habe ich nach
deinen Worten getan. Siehe, ich habe dir ein weises und verständiges
Herz gegeben, daß deinesgleichen vor dir nicht gewesen ist und auch nach
dir nicht aufkommen wird; \bibleverse{13} dazu habe ich dir auch
gegeben, was du nicht gebeten hast, Reichtum und Ehre, daß
deinesgleichen nicht sein soll unter den Königen dein ganzes Leben lang.
\bibleverse{14} Und so du in meinen Wegen wandeln wirst, daß du meine
Satzungen und Gebote beobachtest, wie dein Vater David gewandelt ist, so
will ich dir ein langes Leben geben! \bibleverse{15} Und als Salomo
erwachte, siehe, da war es ein Traum. Als er nun nach Jerusalem kam,
trat er vor die Bundeslade des \textsc{Herrn} und opferte Brandopfer und
Dankopfer und machte allen seinen Knechten ein Mahl. \bibleverse{16} Zu
der Zeit kamen zwei Dirnen zum König und traten vor ihn. \bibleverse{17}
Und das eine Weib sprach: Ach, mein Herr, ich und dieses Weib wohnten in
einem Hause, und ich gebar bei ihr im Hause; \bibleverse{18} und drei
Tage, nachdem ich geboren hatte, gebar sie auch. Und wir waren
beieinander, und kein Fremder war mit uns im Hause, nur wir beide waren
im Hause. \bibleverse{19} Und der Sohn dieses Weibes starb in der Nacht;
denn sie hatte ihn im Schlafe erdrückt. \bibleverse{20} Und sie stand
mitten in der Nacht auf und nahm meinen Sohn von meiner Seite, als deine
Magd schlief, und legte ihn an ihren Busen, und ihren toten Sohn legte
sie an meinen Busen. \bibleverse{21} Und als ich am Morgen aufstand,
meinen Sohn zu säugen, siehe, da war er tot! Aber am Morgen sah ich ihn
genau an und siehe, es war nicht mein Sohn, den ich geboren hatte.
\bibleverse{22} Das andere Weib sprach: Nicht also, sondern mein Sohn
lebt, und dein Sohn ist tot! Jene aber sprach: Nicht also, dein Sohn ist
tot und mein Sohn lebt! Also redeten sie vor dem König. \bibleverse{23}
Und der König sprach: Diese spricht: Der Sohn, der lebt, ist mein Sohn,
und dein Sohn ist tot! Jene spricht: Nicht also, dein Sohn ist tot, und
mein Sohn lebt! \bibleverse{24} Da sprach der König: Bringet mir ein
Schwert! Und als das Schwert vor den König gebracht ward,
\bibleverse{25} sprach der König: Zerschneide das lebendige Kind in zwei
Teile und gebt dieser die eine Hälfte und jener die andere Hälfte!
\bibleverse{26} Da sprach die Frau, welchem der lebendige Sohn gehörte,
zum König (denn ihr Erbarmen über ihren Sohn regte sich in ihr) und
sagte: Bitte, mein Herr, gebt ihr das lebendige Kind und tötet es nicht!
Jene aber sprach: Es sei weder mein noch dein; teilet es!
\bibleverse{27} Da antwortete der König und sprach: Gebt dieser das
lebendige Kind und tötet es nicht! Sie ist seine Mutter! \bibleverse{28}
Als nun ganz Israel vernahm, was für ein Urteil der König gefällt hatte,
fürchteten sie sich vor dem König; denn sie sahen, daß die Weisheit
Gottes in seinem Herzen war, um Recht zu schaffen.

\hypertarget{section-3}{%
\section{4}\label{section-3}}

\bibleverse{1} Und der König Salomo regierte über ganz Israel. Und dies
waren seine obersten Beamten: \bibleverse{2} Asarja, der Sohn Zadoks,
war Priester. \bibleverse{3} Elihoreph und Achija, die beiden Söhne
Sisas, waren Schreiber; Josaphat, der Sohn Achiluds, war Kanzler,
\bibleverse{4} und Benaja, der Sohn Jojadas, war Feldhauptmann; Zadok
aber und Abjatar waren Priester; \bibleverse{5} Asarja, der Sohn Natans,
war über die Beamten; Sabud, der Sohn Natans, war Priester, des Königs
Freund. \bibleverse{6} Achisar war über das Haus gesetzt, und Adoniram,
der Sohn Abdas, war Fronmeister. \bibleverse{7} Und Salomo hatte zwölf
Vögte über ganz Israel, die den König und sein Haus mit Speise
versorgten; je einen Monat im Jahre lag jedem die Versorgung ob.
\bibleverse{8} Und sie hießen also: Der Sohn Churs auf dem Gebirge
Ephraim; \bibleverse{9} der Sohn Dekers zu Makaz und zu Saalbim und zu
Beth-Semes und zu Elon bis Beth-Chanan; \bibleverse{10} der Sohn Heseds
zu Arubbot, über Socho und das ganze Land Hepher; \bibleverse{11} der
Sohn Abinadabs über ganz Naphet-Dor. Dieser hatte Taphat, Salomos
Tochter, zum Weibe. \bibleverse{12} Baana, der Sohn Achiluds, zu Taanach
und Megiddo und über das ganze Beth-Sean, welches neben Zartan unterhalb
Jesreel liegt, von Beth-Sean bis nach Abel-Mechola, bis jenseits
Jokmeam. \bibleverse{13} Der Sohn Gebers zu Ramot in Gilead, der hatte
die Dörfer Jairs, des Sohnes Manasses, in Gilead und die Gegend Argob,
die in Basan liegt; sechzig große Städte, mit Mauern und ehernen Riegeln
verwahrt. \bibleverse{14} Achinadab, der Sohn Iddos, zu Mahanaim;
Achimaaz in Naphtali; \bibleverse{15} auch er hatte eine Tochter
Salomos, Basmat, zum Weibe. \bibleverse{16} Baana, der Sohn Husais, in
Asser und Bealot. \bibleverse{17} Josaphat, der Sohn Paruahas, in
Issaschar. \bibleverse{18} Simei, der Sohn Elas, in Benjamin.
\bibleverse{19} Geber, der Sohn Uris, im Lande Gilead, im Lande Sihons,
des Königs der Amoriter, und Ogs, des Königs von Basan. Nur ein Vogt war
in diesem Lande. \bibleverse{20} Aber Juda und Israel waren zahlreich
wie der Sand am Meer. Sie aßen und tranken und waren fröhlich.
\bibleverse{21} Also war Salomo Herrscher über alle Königreiche, vom
Euphrat-Strome bis zum Philisterlande und bis an die Grenze Ägyptens;
sie brachten ihm Gaben und dienten ihm sein Leben lang. \bibleverse{22}
Salomo aber bedurfte zum Unterhalt täglich dreißig Kor Semmelmehl und
dreißig Kor anderes Mehl; \bibleverse{23} zehn gemästete Rinder und
zwanzig Weiderinder und hundert Schafe, ausgenommen die Hirsche und
Gazellen und Damhirsche und das gemästete Geflügel. \bibleverse{24} Denn
er herrschte im ganzen Lande diesseits des Euphrat-Stromes von Tiphsach
bis nach Gaza, über alle Könige diesseits des Stromes und hatte Frieden
auf allen Seiten ringsum; \bibleverse{25} so daß Juda und Israel sicher
wohnten, ein jeder unter seinem Weinstock und unter seinem Feigenbaum,
von Dan bis Beerseba, solange Salomo lebte. \bibleverse{26} Und Salomo
hatte vierzigtausend Gespann Rosse für seine Wagen und zwölftausend
Reitpferde. \bibleverse{27} Und jene Vögte versorgten den König Salomo
mit Speise; und alles, was zum Tische des Königs Salomo gehörte, brachte
ein jeder in seinem Monat; sie ließen es an nichts mangeln.
\bibleverse{28} Auch die Gerste und das Stroh für die Rosse und
Wagenpferde brachten sie an den Ort, da er war, ein jeder nach seiner
Ordnung. \bibleverse{29} Und Gott gab Salomo Weisheit und sehr viel
Verstand und Weite des Herzens, wie der Sand, der am Meeresufer liegt.
\bibleverse{30} Und die Weisheit Salomos war größer als die Weisheit
aller Söhne des Morgenlandes und als alle Weisheit der Ägypter.
\bibleverse{31} Ja, er war weiser als alle Menschen, auch weiser als
Etan, der Esrachiter, und Heman und Kalkol und Dardan, die Söhne
Machols; und er ward berühmt unter allen Nationen ringsum.
\bibleverse{32} Und er redete dreitausend Sprüche; und seiner Lieder
waren tausendundfünf. \bibleverse{33} Er redete auch von den Bäumen, von
der Zeder auf dem Libanon bis zum Ysop, der aus der Mauer wächst. Auch
redete er vom Vieh, von den Vögeln, von den Reptilien und von den
Fischen. \bibleverse{34} Und es kamen aus allen Völkern, Salomos
Weisheit zu hören, von allen Königen auf Erden, die von seiner Weisheit
gehört hatten.

\hypertarget{section-4}{%
\section{5}\label{section-4}}

\bibleverse{1} Und Hiram, der König zu Tyrus, sandte seine Knechte zu
Salomo; denn er hatte gehört, daß man ihn an seines Vaters Statt zum
König gesalbt hatte; denn Hiram liebte David sein Leben lang.
\bibleverse{2} Und Salomo sandte zu Hiram und ließ ihm sagen:
\bibleverse{3} Du weißt, daß mein Vater David dem Namen des
\textsc{Herrn}, seines Gottes, kein Haus bauen konnte wegen der Kriege,
in die seine Nachbarn ihn verwickelten, bis der \textsc{Herr} sie unter
seine Fußsohlen legte. \bibleverse{4} Nun aber hat mir der
\textsc{Herr}, mein Gott, ringsum Ruhe gegeben, so daß weder ein
Widersacher noch ein boshafter Angriff zu erwarten ist. \bibleverse{5}
Siehe, nun habe ich gedacht, dem Namen des \textsc{Herrn}, meines
Gottes, ein Haus zu bauen, wie der \textsc{Herr} zu meinem Vater David
gesagt hat, indem er sprach: Dein Sohn, den ich an deiner Statt auf den
Thron setzen werde, der soll meinem Namen ein Haus bauen! \bibleverse{6}
So gebiete nun, daß man mir Zedern vom Libanon haue; und meine Knechte
sollen mit deinen Knechten sein, und den Lohn deiner Knechte will ich
dir geben, soviel du verlangst; denn dir ist bekannt, daß niemand unter
uns ist, der Holz zu hauen weiß wie die Zidonier. \bibleverse{7} Als nun
Hiram die Worte Salomos hörte, freute er sich hoch und sprach: Der
\textsc{Herr} sei heute gelobt, der David einen weisen Sohn gegeben hat
über dieses große Volk! \bibleverse{8} Und Hiram sandte zu Salomo und
ließ ihm sagen: Ich habe die Botschaft gehört, die du mir gesandt hast;
ich will tun nach all deinem Begehren betreffs des Zedern und
Zypressenholzes. \bibleverse{9} Meine Knechte sollen die Stämme vom
Libanon an das Meer hinabbringen; darauf will ich sie zu Flößen machen
auf dem Meere bis an den Ort, den du mir wirst sagen lassen, und ich
will sie selbst wieder zerlegen, und du sollst sie holen lassen. Aber du
sollst auch mein Begehren erfüllen und meinem Gesinde Speise geben!
\bibleverse{10} Also gab Hiram dem Salomo Zedern und Zypressenholz nach
all seinem Begehren. \bibleverse{11} Salomo aber gab dem Hiram
zwanzigtausend Kor Weizen zur Speise für sein Gesinde und zwanzig Kor
gestoßenes Öl. Solches gab Salomo dem Hiram alljährlich. \bibleverse{12}
Und der \textsc{Herr} gab Salomo Weisheit, wie er ihm verheißen hatte;
und es war Friede zwischen Hiram und Salomo; und die beiden machten
einen Bund miteinander. \bibleverse{13} Der König Salomo hob auch aus
ganz Israel Fronarbeiter aus, und es waren ihrer dreißigtausend Mann.
\bibleverse{14} Und er sandte sie abwechselnd auf den Libanon, jeden
Monat zehntausend Mann, so daß sie einen Monat auf dem Libanon waren und
zwei Monate daheim. Und Adoniram war über die Fronarbeiter gesetzt.
\bibleverse{15} Und Salomo hatte siebzigtausend Lastträger und
achtzigtausend Steinmetzen im Gebirge, \bibleverse{16} ohne die
Oberaufseher Salomos, die über das Werk gesetzt waren, nämlich
dreitausenddreihundert, welche über das Volk, das am Werk arbeitete, zu
gebieten hatten. \bibleverse{17} Und der König gebot, und sie brachen
große Steine aus, kostbare Steine, nämlich Quadersteine zum Grunde des
Hauses. \bibleverse{18} Und die Bauleute Salomos und die Bauleute Hirams
und die Gibliter behieben sie und richteten das Holz und die Steine zum
Bau des Hauses.

\hypertarget{section-5}{%
\section{6}\label{section-5}}

\bibleverse{1} Im 480. Jahr nach dem Auszug der Kinder Israel aus
Ägypten, im vierten Jahre der Regierung Salomos über Israel, im Monat
Siv, das ist der zweite Monat, baute er dem \textsc{Herrn} das Haus.
\bibleverse{2} Das Haus aber, das der König Salomo dem \textsc{Herrn}
baute, war sechzig Ellen lang, zwanzig Ellen breit und dreißig Ellen
hoch. \bibleverse{3} Und er baute eine Halle vor dem Tempelhaus, zwanzig
Ellen lang, gemäß der Breite des Hauses, und zehn Ellen breit, vor dem
Hause her. \bibleverse{4} Und er machte an das Haus Fenster mit
festeingefügtem Gitterwerk. \bibleverse{5} Und er baute an die Wand des
Hauses einen Anbau ringsum gegen die Wand des Hauses, sowohl des
Tempels, als auch des Chors, und machte Seitengemächer ringsum.
\bibleverse{6} Das unterste Stockwerk war fünf Ellen breit, das mittlere
sechs Ellen und das dritte sieben Ellen breit; denn er machte Absätze an
der Außenseite des Hauses ringsum, so daß sie nicht in die Wände des
Hauses eingriffen. \bibleverse{7} Und als das Haus gebaut ward, wurde es
aus Steinen, die fertig behauen aus dem Bruch kamen, gebaut, so daß man
weder Hammer noch Meißel noch sonst ein eisernes Werkzeug im Hause
hörte, während es gebaut wurde. \bibleverse{8} Der Eingang zum untersten
Stockwerk befand sich an der rechten Seite des Hauses, und man stieg auf
Wendeltreppen hinauf zum mittleren und vom mittleren zum dritten
Stockwerk. \bibleverse{9} Also baute er das Haus und vollendete es und
deckte das Haus mit Brettern und Balkenreihen von Zedernholz.
\bibleverse{10} Er baute auch die Stockwerke am ganzen Haus, jedes fünf
Ellen hoch, und verband sie mit dem Haus durch Zedernbalken.
\bibleverse{11} Und es erging ein Wort des \textsc{Herrn} an Salomo, das
lautete also: \bibleverse{12} Dieses Haus betreffend, das du gebaut
hast: Wirst du in meinen Satzungen wandeln und alle meine Gebote halten
und beobachten, so daß du darin wandelst, so will ich mein Wort an dir
erfüllen, das ich deinem Vater David verheißen habe; \bibleverse{13} und
ich will mitten unter den Kindern Israel wohnen und will mein Volk
Israel nicht verlassen! \bibleverse{14} Also baute Salomo das Haus und
vollendete es. \bibleverse{15} Und er bekleidete die Wände des Hauses
inwendig mit Brettern von Zedern, von des Hauses Boden an bis an die
Balken der Decke, und täfelte es mit Holz inwendig und belegte den Boden
des Hauses mit Brettern von Zypressenholz. \bibleverse{16} Und er baute
in einem Abstand von zwanzig Ellen von der hintern Seite des Hauses eine
zederne Wand, vom Boden bis zum Gebälk, und baute es inwendig aus zum
Chor, zum Allerheiligsten. \bibleverse{17} Aber das Tempelhaus vor dem
Chor war vierzig Ellen lang. \bibleverse{18} Und das Zedernholz inwendig
am Hause war Schnitzwerk von Koloquinten und aufgebrochenen Blumen.
Alles war von Zedernholz, so daß man keinen Stein sah. \bibleverse{19}
Aber den Chor richtete er im Innern des Hauses her, um die Bundeslade
des \textsc{Herrn} dorthin zu stellen. \bibleverse{20} Und vor dem Chor;
welcher zwanzig Ellen lang und zwanzig Ellen breit und zwanzig Ellen
hoch und mit feinem Gold überzogen war; täfelte er den Altar mit
Zedernholz. \bibleverse{21} Und Salomo überzog das Haus inwendig mit
feinem Gold, und er zog goldene Ketten vor dem Chore her, den er mit
Gold überzogen hatte, \bibleverse{22} also daß das ganze Haus völlig mit
Gold überzogen war. Auch den Altar, der vor dem Chor stand, überzog er
mit Gold. \bibleverse{23} Er machte im Chor auch zwei Cherubim von
Ölbaumholz, zehn Ellen hoch. \bibleverse{24} Fünf Ellen maß der eine
Flügel des Cherubs und fünf Ellen der andere Flügel des Cherubs; zehn
Ellen waren vom Ende des einen Flügels bis zum Ende des andern Flügels.
\bibleverse{25} Auch der andere Cherub hatte zehn Ellen Flügelweite.
Beide Cherubim hatten gleiches Maß und gleiche Form. \bibleverse{26} Die
Höhe des einen Cherubs betrug zehn Ellen, ebenso die Höhe des andern
Cherubs. \bibleverse{27} Und er tat die Cherubim mitten ins innerste
Haus. Und die Cherubim breiteten ihre Flügel aus, so daß der Flügel des
einen Cherubs die eine Wand und der Flügel des andern Cherubs die andere
Wand berührte. Aber in der Mitte des Hauses berührte ein Flügel den
andern. \bibleverse{28} Und er überzog die Cherubim mit Gold.
\bibleverse{29} Und an allen Wänden des Hauses ließ er Schnitzwerk
anbringen von Cherubim, Palmen und aufgebrochenen Blumen, innerhalb und
außerhalb. \bibleverse{30} Auch den Boden des Hauses überzog er mit
Gold, innerhalb und außerhalb. \bibleverse{31} Den Eingang zum Chor
machte er mit Türen von Ölbaumholz, Gesimse und Pfosten im Fünfeck.
\bibleverse{32} Und er machte zwei Türflügel von Ölbaumholz und ließ
darauf Schnitzwerk von Cherubim, Palmen und aufgebrochenen Blumen
anbringen und überzog sie mit Gold; auch die Cherubim und die Palmen
überzog er mit Gold. \bibleverse{33} Ebenso machte er auch am Eingang
des Tempels viereckige Pfosten von Ölbaumholz \bibleverse{34} und zwei
Türflügel aus Zypressenholz; aus zwei drehbaren Blättern bestand der
eine Flügel, und aus zwei drehbaren Blättern der andere Flügel.
\bibleverse{35} Und er machte darauf Schnitzwerk von Cherubim, Palmen
und aufgebrochenen Blumen und überzog sie mit Gold, das dem Schnitzwerk
angepaßt war. \bibleverse{36} Auch baute er den innern Vorhof mit drei
Lagen Quadersteinen und einer Lage Zedernbalken. \bibleverse{37} Im
vierten Jahre, im Monat Siv, war der Grund zum Hause des \textsc{Herrn}
gelegt worden. \bibleverse{38} Und im elften Jahre, im Monat Bul, das
ist im achten Monat, ward das Haus vollendet nach allen seinen Plänen
und Vorschriften, so daß er sieben Jahre lang daran gebaut hatte.

\hypertarget{section-6}{%
\section{7}\label{section-6}}

\bibleverse{1} Aber an seinem Hause baute Salomo dreizehn Jahre lang,
bis er es vollendet hatte. \bibleverse{2} Er baute nämlich das Haus des
Libanon-Waldes; hundert Ellen lang, fünfzig Ellen breit und dreißig
Ellen hoch; auf vier Reihen von zedernen Säulen, auf denen zederne
Balken lagen; \bibleverse{3} und ein Dach von Zedernholz oben über den
Gemächern, die über den Säulen lagen, deren fünfundvierzig waren, je
fünfzehn auf einer Reihe. \bibleverse{4} Und es waren drei Reihen
Balken, und die Fenster lagen einander gegenüber, dreimal.
\bibleverse{5} Und alle Türen und Pfosten waren viereckig, aus Gebälk,
und ein Fenster dem andern gegenüber, dreimal. \bibleverse{6} Und er
machte eine Säulenhalle, fünfzig Ellen lang und dreißig Ellen breit, und
noch eine Vorhalle mit Säulen und einer Schwelle davor. \bibleverse{7}
Dazu machte er eine Thronhalle, um dort zu richten, nämlich die
Gerichtshalle, und er täfelte sie mit Zedernholz vom Fußboden bis zu den
Balken der Decke. \bibleverse{8} Und sein Haus, da er wohnte, im andern
Hof, einwärts von der Halle, war von der gleichen Bauart. Salomo baute
auch für die Tochter des Pharao, die er zur Gemahlin genommen hatte, ein
Haus gleich dieser Halle. \bibleverse{9} Solches alles ward gemacht aus
kostbaren Steinen, nach der Schnur behauen, mit der Säge geschnitten auf
der Innen und Außenseite, vom Grunde an bis zum Dach, und draußen bis
zum großen Hof. \bibleverse{10} Die Grundfesten aber bestanden aus
kostbaren, großen Steinen, aus Steinen von zehn Ellen und Steinen von
acht Ellen Länge, \bibleverse{11} und darüber lagen kostbare Steine,
nach dem Maß behauen, und Zedernbalken. \bibleverse{12} Aber der große
Hof, ringsumher, hatte eine Mauer von drei Lagen behauener Steine und
einer Lage Zedernbalken; ebenso der innere Hof des Hauses des
\textsc{Herrn} und die Halle des Hauses. \bibleverse{13} Und der König
Salomo sandte hin und ließ Hiram von Tyrus holen; \bibleverse{14} der
war Sohn einer Witwe aus dem Stamme Naphtali, sein Vater war ein Mann
von Tyrus, ein Erzschmied. Der war voll Weisheit, Verstand und
Kunstsinn, um allerlei Arbeiten in Erz auszuführen. Er kam zum König
Salomo und führte alle Arbeiten für ihn aus. \bibleverse{15} Er goß die
beiden ehernen Säulen; achtzehn Ellen hoch war jede Säule, ein Faden von
zwölf Ellen vermochte sie zu umspannen. \bibleverse{16} Und er machte
zwei Knäufe, aus Erz gegossen, um sie oben auf die Säulen zu setzen, und
jeder Knauf war fünf Ellen hoch. \bibleverse{17} Kränze, als wären sie
geflochten, und Schnüre wie Ketten, waren an den Knäufen oben auf den
Säulen, sieben an dem einen Knauf und sieben an dem andern Knauf.
\bibleverse{18} Und so machte er die Säulen; und zwei Reihen von
Granatäpfeln gingen rings um das eine Flechtwerk, um die Knäufe zu
bedecken, die oben auf den Säulen waren, und ebenso machte er es an dem
andern Knauf. \bibleverse{19} Und die Knäufe oben auf den Säulen waren
gemacht wie Lilien, vier Ellen hoch. \bibleverse{20} Und die Knäufe auf
den beiden Säulen hatten auch oberhalb, nahe bei der Ausbauchung, welche
über dem Flechtwerk war, zweihundert Granatäpfel, ringsum in Reihen
geordnet. \bibleverse{21} Und er richtete die Säulen auf bei der Halle
des Tempels und nannte die, welche er zur Rechten setzte, Jachin, und
die zur Linken hieß er Boas. \bibleverse{22} Und oben auf die Säulen kam
das Lilienwerk. Damit war die Arbeit an den Säulen vollendet.
\bibleverse{23} Er machte auch das gegossene Meer, zehn Ellen weit von
einem Rande bis zum andern, es war ringsherum rund und fünf Ellen hoch.
Und eine dreißig Ellen lange Schnur vermochte es zu umspannen.
\bibleverse{24} Unterhalb seines Randes umgaben es Koloquinten, je zehn
auf die Elle. Der Koloquinten aber waren zwei Reihen, gegossen aus einem
Guß mit dem Meer. \bibleverse{25} Es stand auf zwölf Rindern, deren drei
gegen Mitternacht, drei gegen Abend, drei gegen Mittag und drei gegen
Morgen sahen; und das Meer ruhte oben auf ihnen, und das Hinterteil von
allen war einwärts gekehrt. \bibleverse{26} Seine Dicke aber betrug eine
Handbreite, und sein Rand war wie der Rand eines Bechers, wie die Blüte
einer Lilie, und es faßte zweitausend Bat. \bibleverse{27} Er machte
auch zehn eherne Ständer. Jeder Ständer war vier Ellen lang und vier
Ellen breit und drei Ellen hoch. \bibleverse{28} Diese Ständer aber
waren so eingerichtet, daß sie Felder zwischen den Eckleisten hatten.
\bibleverse{29} Und auf den Feldern zwischen den Eckleisten waren Löwen,
Rinder und Cherubim; und auf den Eckleisten war es oben ebenso, und
unterhalb der Löwen und Rinder waren herabhängende Kränze.
\bibleverse{30} Und jeder Ständer hatte vier eherne Räder mit ehernen
Achsen; an seinen vier Ecken waren Schulterstücke; unter dem Becken
waren die Schulterstücke angegossen, gegenüber den Kränzen.
\bibleverse{31} Und seine Öffnung, innerhalb des Kopfstückes und
darüber, maß eine Elle, und seine Öffnung war rund, nach Art eines
Säulenfußes, anderthalb Ellen; auch an seiner Öffnung war Bildwerk; ihre
Felder waren viereckig, nicht rund. \bibleverse{32} Die vier Räder aber
standen unterhalb der Leisten, und die Achsen der Räder waren an dem
Ständer. Jedes Rad war anderthalb Ellen hoch. \bibleverse{33} Und es
waren Räder wie Wagenräder. Und ihre Achsen, Naben, Speichen und Felgen
waren alle gegossen. \bibleverse{34} Es waren auch vier Schulterstücke
an den vier Ecken eines jeden Ständers, die waren aus einem Guß mit dem
Ständer. \bibleverse{35} Oben an dem Ständer lief eine Art von Gestell
von der Höhe einer halben Elle ringsherum, und oben am Ständer waren
seine Halter; diese und die Felder aus einem Guß mit ihm.
\bibleverse{36} Und er grub auf die Tafeln seiner Seiten und auf seine
Leisten Cherubim, Löwen und Palmbäume ein, je nachdem Raum vorhanden
war, und Kränze ringsum. \bibleverse{37} So machte er die zehn Ständer
alle aus einem Guß, nach einerlei Maß und Form. \bibleverse{38} Und er
machte zehn eherne Kessel, vierzig Bat gingen in einen Kessel; ein jeder
war vier Ellen weit, und auf jedem der zehn Ständer war ein Kessel.
\bibleverse{39} Er setzte aber fünf Ständer an die rechte Seite und die
andern fünf an die linke Seite des Hauses. Aber das Meer stellte er auf
die rechte Seite des Hauses, nach Südosten hin. \bibleverse{40} Und
Hiram machte die Töpfe, Schaufeln und Becken; so vollendete er das ganze
Werk, welches er dem König Salomo für das Haus des \textsc{Herrn} zu
machen hatte: \bibleverse{41} die beiden Säulen und die Kugeln der
Knäufe oben auf den beiden Säulen, und die beiden Kränze, um die Kugeln
der Knäufe auf den Säulen zu decken. \bibleverse{42} Auch die
vierhundert Granatäpfel an den beiden Kränzen, je zwei Reihen
Granatäpfel an einem Kranz, um die zwei Kugeln der Knäufe auf den Säulen
zu bedecken. \bibleverse{43} Dazu die zehn Ständer und die zehn Kessel
oben auf den Ständern. \bibleverse{44} Und das eine Meer und die zwölf
Rinder unter dem Meer. \bibleverse{45} Und die Töpfe, Schaufeln und
Becken. Und alle diese Geräte, die Hiram dem König Salomo machte für das
Haus des \textsc{Herrn}, waren von glänzendem Erz. \bibleverse{46} In
der Gegend am Jordan ließ sie der König gießen in lehmiger Erde,
zwischen Sukkot und Zartan. \bibleverse{47} Und Salomo ließ alle diese
Geräte ungewogen wegen der sehr großen Menge des Erzes; denn das Gewicht
des Erzes konnte man nicht ermitteln. \bibleverse{48} Salomo machte auch
alle Geräte, die zum Hause des \textsc{Herrn} gehörten: den goldenen
Altar, den goldenen Tisch, worauf die Schaubrote lagen; \bibleverse{49}
fünf Leuchter zur Rechten und fünf Leuchter zur Linken, vor dem Chor,
von feinem Gold, mit goldenen Blumen, Lampen und Lichtscheren.
\bibleverse{50} Dazu Schalen, Messer, Becken, Pfannen und Rauchnäpfe von
feinem Gold. Auch die Angeln an den Türen des innern Hauses, des
Allerheiligsten, und an den Türen des Tempelhauses waren von Gold.
\bibleverse{51} Als nun das ganze Werk vollendet war, welches der König
Salomo am Hause des \textsc{Herrn} machte, brachte Salomo hinein, was
sein Vater David geheiligt hatte: das Silber und das Gold und die Geräte
legte er in den Schatz des Hauses des \textsc{Herrn}.

\hypertarget{section-7}{%
\section{8}\label{section-7}}

\bibleverse{1} Darnach versammelte Salomo die Ältesten Israels und alle
Häupter der Stämme, die Fürsten der israelitischen Geschlechter bei sich
in Jerusalem, um die Bundeslade des \textsc{Herrn} hinaufzubringen aus
der Stadt Davids, das ist Zion. \bibleverse{2} Und es versammelten sich
alle Männer Israels beim König Salomo zum Fest im Monat Etanim, das ist
der siebente Monat. \bibleverse{3} Als nun alle Ältesten Israels kamen,
trugen die Priester die Lade des \textsc{Herrn} \bibleverse{4} und
brachten die Lade des \textsc{Herrn} hinauf, dazu die Stiftshütte und
alle Geräte des Heiligtums, die in der Stiftshütte waren. Das trugen die
Priester und Leviten hinauf. \bibleverse{5} Und der König Salomo und die
ganze Gemeinde Israel, die sich zu ihm versammelt hatte, standen vor der
Lade und opferten Schafe und Rinder, so viele, daß man sie vor Menge
nicht zählen noch berechnen konnte. \bibleverse{6} Also brachten die
Priester die Bundeslade an ihren Ort, in den Chor des Hauses, in das
Allerheiligste, unter die Flügel der Cherubim. \bibleverse{7} Denn die
Cherubim breiteten die Flügel aus über den Ort, wo die Lade stand, und
bedeckten die Lade und ihre Stangen von oben her. \bibleverse{8} Die
Stangen aber waren so lang, daß ihre Spitzen im Heiligtum vor dem Chor
gesehen wurden; aber draußen wurden sie nicht gesehen, und sie blieben
daselbst bis auf diesen Tag. \bibleverse{9} Es war nichts in der Lade
als nur die zwei steinernen Tafeln, welche Mose am Horeb hineingelegt
hatte, als der \textsc{Herr} mit den Kindern Israel einen Bund machte,
als sie aus dem Lande Ägypten gezogen waren. \bibleverse{10} Als aber
die Priester aus dem Heiligtum traten, erfüllte die Wolke das Haus des
\textsc{Herrn}, \bibleverse{11} also daß die Priester wegen der Wolke
nicht hintreten konnten, um ihren Dienst zu verrichten; denn die
Herrlichkeit des \textsc{Herrn} erfüllte das Haus des \textsc{Herrn}.
\bibleverse{12} Damals sprach Salomo: Der \textsc{Herr} hat gesagt, er
wolle im Dunkeln wohnen. \bibleverse{13} So habe ich nun ein Haus
gebaut, dir zur Wohnung; einen Sitz, daß du da ewiglich bleiben mögest!
\bibleverse{14} Und der König wandte sein Angesicht und segnete die
ganze Gemeinde Israel; und die ganze Gemeinde Israel stand.
\bibleverse{15} Und er sprach: Gepriesen sei der \textsc{Herr}, der Gott
Israels, der meinem Vater David durch seinen Mund verheißen und es auch
durch seine Hand erfüllt hat, da er sagte: \bibleverse{16} Seit dem
Tage, da ich mein Volk Israel aus Ägypten führte, habe ich unter allen
Stämmen Israels niemals eine Stadt erwählt, daß mir dort ein Haus gebaut
würde, damit mein Name daselbst wäre; aber ich habe David erwählt, daß
er über mein Volk Israel herrsche. \bibleverse{17} Nun hatte zwar mein
Vater David im Sinn, dem Namen des \textsc{Herrn}, des Gottes Israels,
ein Haus zu bauen. \bibleverse{18} Aber der \textsc{Herr} sprach zu
meinem Vater David: Daß du dir vornahmst, meinem Namen ein Haus zu
bauen, da hast du wohlgetan, dir solches vorzunehmen; \bibleverse{19}
doch sollst nicht du das Haus bauen, sondern dein Sohn, der aus deinen
Lenden hervorgehen wird, der soll meinem Namen ein Haus bauen.
\bibleverse{20} Und der \textsc{Herr} hat sein Wort erfüllt, das er
geredet hat; denn ich bin an meines Vaters David Statt getreten und
sitze auf dem Thron Israels, wie der \textsc{Herr} geredet hat, und ich
habe dem Namen des \textsc{Herrn}, des Gottes Israels, \bibleverse{21}
ein Haus gebaut und daselbst einen Ort zugerichtet für die Lade, darin
das Gesetz des Bundes des \textsc{Herrn} ist, den er mit unsern Vätern
gemacht hat, als er sie aus dem Lande Ägypten führte. \bibleverse{22}
Darnach trat Salomo vor den Altar des \textsc{Herrn}, angesichts der
ganzen Gemeinde Israel und breitete seine Hände aus gen Himmel und
sprach: \bibleverse{23} O \textsc{Herr}, Gott Israels! Dir, o Gott, ist
niemand gleich, weder oben im Himmel noch unten auf Erden, der du den
Bund und die Gnade bewahrst deinen Knechten, die vor dir wandeln;
\bibleverse{24} der du deinem Knechte, meinem Vater David, gehalten, was
du ihm versprochen hast; ja, was du mit deinem Munde geredet hattest,
das hast du mit deiner Hand erfüllt, wie es heute der Fall ist.
\bibleverse{25} Und nun, \textsc{Herr}, Gott Israels, halte deinem
Knecht, meinem Vater David, was du ihm versprochen hast, als du sagtest:
Es soll dir nicht mangeln an einem Mann vor mir, welcher auf dem Thron
Israels sitze, insofern deine Kinder ihren Weg bewahren, daß sie vor mir
wandeln, wie du vor mir gewandelt bist! \bibleverse{26} Und nun, o Gott
Israels, laß doch dein Wort wahr werden, welches du zu deinem Knecht,
meinem Vater David, geredet hast! \bibleverse{27} Aber wohnt Gott
wirklich auf Erden? Siehe, die Himmel und aller Himmel Himmel mögen dich
nicht fassen; wie sollte es denn dieses Haus tun, das ich gebaut habe?
\bibleverse{28} Wende dich aber zum Gebet deines Knechtes und zu seinem
Flehen, o \textsc{Herr}, mein Gott, daß du hörest das Flehen und das
Gebet, welches dein Knecht heute vor dir tut, \bibleverse{29} daß deine
Augen Tag und Nacht offen stehen über diesem Haus, über dem Ort, davon
du gesagt hast: Mein Name soll daselbst sein. So wollest du denn hören
das Gebet, welches dein Knecht an dieser Stätte tut, \bibleverse{30} und
wollest erhören das Flehen deines Knechtes und deines Volkes Israel, das
sie an diesem Ort tun; ja, du wollest es hören am Ort deiner Wohnung im
Himmel, und wenn du es hörst, so vergib! \bibleverse{31} Wenn jemand
wider seinen Nächsten sündigt, und man ihm einen Eid auferlegt, den er
schwören soll, und er kommt und schwört vor deinem Altar in diesem
Hause, \bibleverse{32} so wollest du hören im Himmel und verschaffen,
daß deinen Knechten Recht gesprochen wird, indem du den Schuldigen
verurteilst, sein Tun auf sein Haupt zurückfallen lässest, den Gerechten
aber rechtfertigst, ihm nach seiner Gerechtigkeit vergiltst.
\bibleverse{33} Wenn dein Volk Israel vor dem Feind geschlagen wird,
weil sie an dir gesündigt haben, und sie kehren wieder zu dir zurück und
bekennen deinen Namen, beten und flehen zu dir in diesem Hause,
\bibleverse{34} so wollest du hören im Himmel und die Sünde deines
Volkes vergeben und sie wieder in das Land bringen, das du ihren Vätern
gegeben hast. \bibleverse{35} Wenn der Himmel verschlossen ist und es
nicht regnet, weil sie an dir gesündigt haben, und sie dann an diesem
Orte beten und deinen Namen bekennen und sich von ihren Sünden abwenden,
weil du sie demütigst, \bibleverse{36} so wollest du es hören im Himmel
und die Sünde deiner Knechte und deines Volkes Israel vergeben, indem du
sie den guten Weg lehrest, auf dem sie wandeln sollen, und wollest
regnen lassen auf dein Land, welches du deinem Volk zum Erbe gegeben
hast. \bibleverse{37} Wenn Hungersnot im Lande sein wird, wenn eine
Pestilenz ausbricht, wenn Kornbrand, Vergilben des Getreides,
Heuschrecken und Fresser sein werden, wenn sein Feind es belagert im
Lande seiner Tore, wenn irgend eine Plage, irgend eine Krankheit
auftritt, \bibleverse{38} was immer alsdann irgend ein Mensch von deinem
ganzen Volke Israel bittet und fleht, wenn sie spüren, wie ihnen das
Gewissen schlägt, und sie ihre Hände ausbreiten nach diesem Hause,
\bibleverse{39} so mögest du es hören in deiner Wohnung im Himmel und
vergeben und eingreifen und einem jeden geben, wie er gewandelt hat, wie
du sein Herz kennst (denn du allein kennst das Herz aller
Menschenkinder) \bibleverse{40} auf daß sie dich fürchten allezeit,
solange sie leben im Lande, das du ihren Vätern gegeben hast!
\bibleverse{41} Aber auch wenn ein Fremdling, der nicht zu deinem Volke
Israel gehört, aus fernem Lande kommt um deines Namens willen
\bibleverse{42} denn sie werden hören von deinem großen Namen und von
deiner mächtigen Hand und von deinem ausgestreckten Arm, wenn er kommt,
um in diesem Hause zu beten, \bibleverse{43} so wollest du es hören in
deiner Wohnung im Himmel und alles tun, um was der Fremdling dich
anrufen wird, auf daß alle Völker auf Erden deinen Namen erkennen und
dich fürchten wie dein Volk Israel und erfahren, daß dieses Haus,
welches ich gebaut habe, nach deinem Namen genannt ist. \bibleverse{44}
Wenn dein Volk in den Krieg zieht wider seine Feinde, auf dem Weg, den
du sie senden wirst, und sie zum \textsc{Herrn} beten nach der Stadt
gewandt, die du erwählt hast, und nach dem Hause, das ich deinem Namen
erbaut habe, \bibleverse{45} so wollest du im Himmel ihr Gebet und ihr
Flehen hören und ihnen Recht verschaffen! \bibleverse{46} Wenn sie wider
dich sündigen (denn es ist kein Mensch, der nicht sündigt) und du wider
sie zürnst und sie ihren Feinden übergibst, so daß diese sie gefangen
abführen in das Land ihrer Feinde, es sei fern oder nah, \bibleverse{47}
und sie in dem Lande, wo sie gefangen sind, in sich gehen und umkehren
und im Lande ihrer Gefangenschaft zu dir flehen und sprechen: Wir haben
gesündigt und Unrecht getan und sind gottlos gewesen! \bibleverse{48}
Wenn sie sich also zu dir kehren mit ihrem ganzen Herzen und mit ihrer
ganzen Seele im Lande ihrer Feinde, die sie weggeführt haben, und zu dir
beten, nach ihrem Lande gewandt, das du ihren Vätern gegeben hast, und
nach der Stadt, welche du erwählt hast, und nach dem Hause, das ich
deinem Namen gebaut habe, \bibleverse{49} so wollest du in deiner
Wohnung im Himmel ihr Gebet und ihr Flehen hören und ihnen Recht
schaffen, und wollest deinem Volke vergeben, was es an dir gesündigt
hat, \bibleverse{50} und alle Übertretungen, die es wider dich begangen
hat, und du wollest sie Barmherzigkeit finden lassen bei denen, die sie
gefangen halten, so daß sie sich ihrer erbarmen; \bibleverse{51} denn
sie sind dein Volk und dein Erbe, das du aus Ägypten, mitten aus dem
eisernen Ofen, geführt hast! \bibleverse{52} So wollest du denn deine
Augen offen halten für das Flehen deines Knechtes und für das Flehen
deines Volkes Israel, daß du sie erhörest in allem, um was sie dich
anrufen! \bibleverse{53} Denn du hast sie ausgesondert aus allen Völkern
auf Erden dir zum Erbe, wie du durch deinen Knecht Mose geredet hast,
als du unsre Väter aus Ägypten führtest, o Herr, \textsc{Herr}!
\bibleverse{54} Als nun Salomo dieses ganze Gebet und Flehen vor dem
\textsc{Herrn} vollendet hatte, stand er auf von seinem Platz vor dem
Altar des \textsc{Herrn}, wo er gekniet hatte, seine Hände gen Himmel
gebreitet, \bibleverse{55} und er trat hin und segnete die ganze
Gemeinde Israel mit lauter Stimme und sprach: \bibleverse{56} Gelobet
sei der \textsc{Herr}, der seinem Volk Israel Ruhe gegeben hat, ganz wie
er versprochen hat! Von allen seinen guten Worten, welche er durch
seinen Knecht Mose geredet hat, ist nicht eines dahingefallen.
\bibleverse{57} Der \textsc{Herr}, unser Gott, sei mit uns, wie er mit
unsern Vätern gewesen ist! Er verlasse uns nicht und ziehe die Hand
nicht von uns ab, \bibleverse{58} unser Herz zu ihm zu neigen, daß wir
in allen seinen Wegen wandeln und seine Gebote, seine Satzungen und
seine Rechte halten, welche er unsern Vätern geboten hat!
\bibleverse{59} Und mögen diese meine Worte, die ich vor dem
\textsc{Herrn} gefleht habe, gegenwärtig sein vor dem \textsc{Herrn},
unserm Gott, bei Tag und bei Nacht, daß er Recht schaffe seinem Knecht
und Recht seinem Volke Israel, Tag für Tag, \bibleverse{60} auf daß alle
Völker auf Erden erkennen, daß er, der \textsc{Herr}, Gott ist, und
keiner sonst! \bibleverse{61} Euer Herz aber sei ungeteilt mit dem
\textsc{Herrn}, unserm Gott, daß ihr in seinen Satzungen wandelt und
seine Gebote bewahrt, wie an diesem Tage! \bibleverse{62} Und der König
und ganz Israel mit ihm brachten Opfer dar vor dem \textsc{Herrn}.
\bibleverse{63} Und zwar brachte Salomo zum Dankopfer, das er dem
\textsc{Herrn} opferte, 22000 Ochsen und 120000 Schafe. Also weihten der
König und alle Kinder Israel das Haus des \textsc{Herrn} ein.
\bibleverse{64} An jenem Tage weihte der König den innern Vorhof, der
vor dem Hause des \textsc{Herrn} war, damit, daß er Brandopfer,
Speisopfer und das Fett der Dankopfer daselbst zurichtete; denn der
eherne Altar, der vor dem \textsc{Herrn} stand, war zu klein für die
Brandopfer, Speisopfer und für das Fett der Dankopfer. \bibleverse{65}
So feierte Salomo zu jener Zeit ein Fest (und ganz Israel mit ihm, eine
große Versammlung des Volkes von den Grenzen Chamats bis an den Bach
Ägyptens) vor dem \textsc{Herrn}, unserm Gott, sieben Tage und nochmals
sieben Tage lang; das waren vierzehn Tage. \bibleverse{66} Am achten
Tage entließ er das Volk. Und sie segneten den König und gingen hin zu
ihren Hütten, fröhlich und guten Mutes, wegen all des Guten, das der
\textsc{Herr} an seinem Knechte David und an seinem Volke Israel getan
hatte.

\hypertarget{section-8}{%
\section{9}\label{section-8}}

\bibleverse{1} Und als Salomo das Haus des \textsc{Herrn} und das Haus
des Königs vollendet hatte und alles, was er zu machen begehrte und wozu
er Lust hatte, \bibleverse{2} da erschien ihm der \textsc{Herr} zum
zweiten Mal, wie er ihm zu Gibeon erschienen war. \bibleverse{3} Und der
\textsc{Herr} sprach zu ihm: Ich habe dein Gebet und dein Flehen erhört,
das du vor mir gebetet hast. Ich habe dieses Haus geheiligt, welches du
gebaut hast, daß ich meinen Namen daselbst wohnen lasse ewiglich, und
meine Augen und mein Herz sollen daselbst sein allezeit. \bibleverse{4}
Und was dich betrifft, wenn du vor mir wandelst, wie dein Vater David
gewandelt ist, mit rechtschaffenem Herzen und aufrichtig, so daß du
alles tust, was ich dir geboten habe, und meine Satzungen und meine
Rechte beobachtest, \bibleverse{5} so will ich den Thron deines
Königtums über Israel auf ewig befestigen, wie ich deinem Vater David
versprochen habe, indem ich sagte: Es soll dir nicht mangeln an einem
Mann auf dem Throne Israels! \bibleverse{6} Werdet ihr euch aber von mir
abwenden, ihr und eure Kinder, und meine Gebote und meine Satzungen, die
ich euch vorgelegt habe, nicht beobachten, sondern hingehen und anderen
Göttern dienen und sie anbeten, \bibleverse{7} so werde ich Israel
ausrotten aus dem Lande, das ich ihnen gegeben habe, und das Haus, das
ich meinem Namen geheiligt habe, von meinem Angesicht verwerfen, und
Israel soll zum Sprichwort und zum Spott werden unter allen Völkern.
\bibleverse{8} Und über dieses Haus, so hoch es gewesen ist, wird jeder,
der da vorübergeht, sich entsetzen und spotten und sagen: Warum hat der
\textsc{Herr} diesem Lande und diesem Hause also getan? \bibleverse{9}
Alsdann wird man antworten: Weil sie den \textsc{Herrn}, ihren Gott, der
ihre Väter aus Ägyptenland geführt hat, verlassen und sich an andere
Götter gehalten und sie angebetet und ihnen gedient haben. Darum hat der
\textsc{Herr} all dies Unglück über sie gebracht! \bibleverse{10} Als
nun die zwanzig Jahre verflossen waren, in welchen Salomo die beiden
Häuser, das Haus des \textsc{Herrn} und das Haus des Königs gebaut
hatte, \bibleverse{11} wozu Hiram, der König von Tyrus, Salomo mit
Zedern und Zypressenholz und Gold nach all seinem Begehr versorgt hatte,
da gab der König Salomo dem Hiram zwanzig Städte im Lande Galiläa.
\bibleverse{12} Und Hiram zog aus von Tyrus, die Städte zu besehen, die
ihm Salomo gegeben hatte; aber sie gefielen ihm nicht. \bibleverse{13}
Und er sprach: Was sind das für Städte, mein Bruder, die du mir gegeben
hast? Und er hieß sie ``Land Kabul'' bis auf diesen Tag. \bibleverse{14}
Denn Hiram hatte dem König hundertundzwanzig Talente Gold gesandt.
\bibleverse{15} Und so verhielt es sich mit den Fronarbeitern, welche
der König Salomo aushob, um das Haus des \textsc{Herrn} und sein Haus
und den Millo und die Mauern Jerusalems und Hazor und Megiddo und Geser
zu bauen. \bibleverse{16} Denn Pharao, der König von Ägypten, war
heraufgekommen und hatte Geser eingenommen und mit Feuer verbrannt und
die Kanaaniter, die in der Stadt wohnten, getötet und hatte sie seiner
Tochter, der Gemahlin Salomos, zur Mitgift gegeben. \bibleverse{17} Also
baute Salomo Geser und das untere Beth-Horon; \bibleverse{18} auch
Bahalat und Tadmor in der Wüste, im Lande, und alle Vorratsstädte,
\bibleverse{19} die Salomo hatte, und die Wagenstädte und die
Reiterstädte und wozu Salomo Lust hatte zu bauen zu Jerusalem und im
Libanon und im ganzen Land seiner Herrschaft. \bibleverse{20} Alles
Volk, das von den Amoritern, Hetitern, Pheresitern, Hevitern und
Jebusitern übriggeblieben war und nicht zu den Kindern Israel gehörte,
\bibleverse{21} ihre Söhne, die im Lande nach ihnen übriggeblieben
waren, welche die Kinder Israel nicht ausrotten konnten, die hob Salomo
zum Frondienst aus bis auf diesen Tag. \bibleverse{22} Aber von den
Kindern Israel machte er keine zu Leibeigenen, sondern sie waren
Kriegsleute und seine Diener und seine Fürsten und seine Wagenkämpfer
und Oberste über seine Wagen und über seine Reiter. \bibleverse{23} Die
Zahl der Oberaufseher, die Salomo über das Werk gesetzt hatte, war 550;
sie herrschten über das Volk, welches an dem Werk arbeitete.
\bibleverse{24} Sobald die Tochter des Pharao heraufgezogen war von der
Stadt Davids in ihr Haus, das er für sie gebaut hatte, da baute er auch
den Millo. \bibleverse{25} Und Salomo opferte dreimal im Jahre
Brandopfer und Dankopfer auf dem Altar, den er dem \textsc{Herrn} gebaut
hatte, und ließ zugleich räuchern auf demjenigen, welcher vor dem
\textsc{Herrn} stand. \bibleverse{26} Und als er das Haus vollendet
hatte, baute der König Salomo Schiffe zu Ezjon-Geber, welches bei Elot
liegt, am Ufer des Schilfmeers im Lande der Edomiter. \bibleverse{27}
Und Hiram sandte seine Knechte, die sich auf die Schiffe verstanden und
auf dem Meere erfahren waren, mit den Knechten Salomos auf die Fahrt;
\bibleverse{28} und sie gelangten bis nach Ophir und holten dort
vierhundertundzwanzig Talente Gold und brachten es dem König Salomo.

\hypertarget{section-9}{%
\section{10}\label{section-9}}

\bibleverse{1} Als aber die Königin von Saba den Ruhm Salomos wegen des
Namens des \textsc{Herrn} vernahm, kam sie, ihn mit Rätseln zu erproben.
\bibleverse{2} Sie kam aber nach Jerusalem mit sehr großem Reichtum, mit
Kamelen, die Spezereien und sehr viel Gold und Edelsteine trugen. Und
als sie zu Salomo kam, sagte sie ihm alles, was sie auf dem Herzen
hatte. \bibleverse{3} Und Salomo erklärte ihr alles; es war dem König
nichts verborgen, daß er es ihr nicht erklärt hätte. \bibleverse{4} Als
aber die Königin von Saba alle Weisheit Salomos sah und das Haus, das er
gebaut hatte, \bibleverse{5} und die Speise für seinen Tisch und die
Wohnung seiner Knechte und das Auftreten seiner Dienerschaft und ihre
Kleidung, auch sein Geschirr und die Brandopfer, die er im Hause des
\textsc{Herrn} darbrachte, da geriet sie außer sich vor Erstaunen und
sprach zum König: \bibleverse{6} Das Wort ist wahr, welches ich in
meinem Lande über deine Sachen und über deine Weisheit gehört habe!
\bibleverse{7} Und ich habe den Worten nicht geglaubt, bis ich gekommen
bin und es mit eigenen Augen gesehen habe. Und siehe, es ist mir nicht
die Hälfte gesagt worden; du hast mehr Weisheit und Gut, als das Gerücht
sagt, das ich vernommen habe. \bibleverse{8} Selig sind deine Leute,
selig diese deine Knechte, die allezeit vor dir stehen und deine
Weisheit hören! \bibleverse{9} Gepriesen sei der \textsc{Herr}, dein
Gott, der Gefallen an dir gehabt hat, so daß er dich auf den Thron
Israels setzte! Weil der \textsc{Herr} Israel lieb hat ewiglich, darum
hat er dich zum König eingesetzt, daß du Recht und Gerechtigkeit übest!
\bibleverse{10} Und sie gab dem König hundertundzwanzig Talente Gold und
sehr viel Gewürz und Edelsteine; nie wieder kam so viel Gewürz, wie die
Königin von Saba dem König Salomo gab. \bibleverse{11} Dazu brachten die
Schiffe Hirams, welche Gold aus Ophir führten, sehr viel Sandelholz und
Edelsteine von Ophir. \bibleverse{12} Und der König ließ aus Sandelholz
Treppen machen für das Haus des \textsc{Herrn} und für das Haus des
Königs und Harfen und Psalter für die Sänger; soviel Sandelholz ist nie
mehr ins Land gekommen noch gesehen worden bis auf diesen Tag.
\bibleverse{13} Und der König Salomo gab der Königin von Saba alles, was
sie wünschte und erbat, außer dem, womit Salomo sie königlich
beschenkte. Da kehrte sie um und zog in ihr Land samt ihren Knechten.
\bibleverse{14} Das Gewicht des Goldes aber, das bei Salomo in einem
Jahr einging, betrug 666 Talente, \bibleverse{15} außer dem, was die
Karawanen und der Handel der Kaufleute und alle Könige Arabiens und die
Statthalter des Landes brachten. \bibleverse{16} Und der König Salomo
ließ zweihundert Schilde von gehämmertem Golde machen; sechshundert
Schekel Gold verwendete er für jeden Schild; \bibleverse{17} und
dreihundert Tartschen von gehämmertem Gold; je drei Minen Gold
verwendete er für eine Tartsche. Und der König tat sie in das Haus des
Libanonwaldes. \bibleverse{18} Ferner machte der König einen großen
Thron von Elfenbein und überzog ihn mit dem edelsten Golde.
\bibleverse{19} Dieser Thron hatte sechs Stufen, und das Kopfstück des
Thrones war hinten rund, und auf beiden Seiten um den Sitz waren Lehnen,
und zwei Löwen standen an den Lehnen. \bibleverse{20} Und zwölf Löwen
standen auf den sechs Stufen zu beiden Seiten. Dergleichen ist niemals
in irgend einem Königreiche gemacht worden. \bibleverse{21} Auch alle
Trinkgeschirre des Königs Salomo waren golden, und alles Geschirr im
Hause des Libanonwaldes war von feinem Gold; denn zu den Zeiten Salomos
achtete man das Silber gar nicht. \bibleverse{22} Denn des Königs
Tarsisschiffe fuhren auf dem Meer mit den Schiffen Hirams. Diese
Tarsisschiffe kamen in drei Jahren einmal und brachten Gold, Silber,
Elfenbein, Affen und Pfauen. \bibleverse{23} Also ward der König Salomo
größer an Reichtum und Weisheit als alle Könige auf Erden.
\bibleverse{24} Und alle Welt begehrte Salomo zu sehen, um seine
Weisheit zu hören, die ihm Gott ins Herz gegeben hatte. \bibleverse{25}
Und sie brachten ihm ein jeder sein Geschenk: silberne und goldene
Geschirre, Kleider, Rüstungen, Gewürz, Pferde und Maultiere, Jahr für
Jahr. \bibleverse{26} Salomo brachte auch Wagen und Reiter zusammen, so
daß er tausendvierhundert Wagen und zwölftausend Reiter hatte, die er in
den Wagenstädten und bei dem König zu Jerusalem unterbrachte.
\bibleverse{27} Und der König machte, daß zu Jerusalem so viel Silber
war wie Steine und so viel Zedernholz wie wilde Feigenbäume in den
Gründen. \bibleverse{28} Und man brachte dem Salomo Pferde aus Ägypten
und ein Zug von Kaufleuten des Königs holte sie scharenweise um den
Kaufpreis. \bibleverse{29} Und ein Wagen aus Ägypten kam auf
sechshundert Silberlinge zu stehen, und ein Pferd auf hundertundfünfzig;
ebenso brachte man sie durch ihre Vermittlung auch allen Königen der
Hetiter und den Königen von Syrien.

\hypertarget{section-10}{%
\section{11}\label{section-10}}

\bibleverse{1} Aber der König Salomo liebte viele ausländische Frauen
neben der Tochter des Pharao: moabitische, ammonitische, edomitische,
zidonische und hetitische, \bibleverse{2} aus den Völkern, von denen der
\textsc{Herr} den Kindern Israel gesagt hatte: Geht nicht zu ihnen und
lasset sie nicht zu euch kommen, denn sie werden gewiß eure Herzen zu
ihren Göttern neigen! An diesen hing Salomo mit Liebe. \bibleverse{3}
Und er hatte siebenhundert fürstliche Frauen und dreihundert
Nebenfrauen; und seine Frauen neigten sein Herz. \bibleverse{4} Denn als
Salomo alt war, neigten seine Frauen sein Herz fremden Göttern zu, daß
sein Herz nicht mehr so vollkommen mit dem \textsc{Herrn}, seinem Gott,
war wie das Herz seines Vaters David. \bibleverse{5} Also lief Salomo
der Astarte nach, der Gottheit der Zidonier, und Milkom, dem Greuel der
Ammoniter. \bibleverse{6} Und Salomo tat, was böse war in den Augen des
\textsc{Herrn}, und folgte dem Herrn nicht gänzlich nach, wie sein Vater
David. \bibleverse{7} Auch baute Salomo eine Höhe dem Kamos, dem Greuel
der Moabiter, auf dem Berge, der vor Jerusalem liegt, und dem Moloch,
dem Greuel der Kinder Ammon. \bibleverse{8} Und ebenso tat er für alle
seine ausländischen Frauen, die ihren Göttern räucherten und opferten.
\bibleverse{9} Aber der \textsc{Herr} ward zornig über Salomo, weil sein
Herz sich abgewandt hatte von dem \textsc{Herrn}, dem Gott Israels, der
ihm zweimal erschienen war, \bibleverse{10} ja, der ihm gerade darüber
Befehl gegeben hatte, daß er nicht andern Göttern nachwandeln solle;
aber er beachtete nicht, was ihm der \textsc{Herr} geboten hatte.
\bibleverse{11} Darum sprach der \textsc{Herr} zu Salomo: Weil solches
von dir geschehen ist und du meinen Bund nicht bewahrt hast, noch meine
Satzungen, die ich dir geboten habe, so will ich dir gewiß das
Königreich entreißen und es deinem Knechte geben. \bibleverse{12} Doch
zu deiner Zeit will ich es nicht tun, um deines Vaters David willen; der
Hand deines Sohnes will ich es entreißen. \bibleverse{13} Nur will ich
ihm nicht das ganze Reich entreißen; einen Stamm will ich deinem Sohne
geben, um meines Knechtes David und um Jerusalems willen, die ich
erwählt habe. \bibleverse{14} Und der \textsc{Herr} erweckte dem Salomo
einen Widersacher, Hadad, den Edomiter, er war vom königlichen Samen in
Edom. \bibleverse{15} Denn als David in Edom war, und als Joab, der
Feldhauptmann, hinaufzog, um die Erschlagenen zu begraben, erschlug er,
was männlich war in Edom. \bibleverse{16} Denn Joab blieb sechs Monate
lang daselbst mit ganz Israel, bis er alles ausgerottet hatte, was in
Edom männlich war. \bibleverse{17} Da floh Hadad und mit ihm etliche
Edomiter von den Knechten seines Vaters, um nach Ägypten zu gehen; Hadad
aber war noch ein kleiner Knabe. \bibleverse{18} Und sie machten sich
auf von Midian und kamen gen Paran, nahmen Leute mit sich aus Paran und
kamen nach Ägypten zum Pharao, dem König von Ägypten; der gab ihm ein
Haus, versprach ihm Nahrung und gab ihm Land. \bibleverse{19} Und Hadad
fand große Gnade beim Pharao, so daß er ihm auch seiner Gemahlin
Schwester, die Schwester der Gebieterin Tachpenes, zum Weibe gab.
\bibleverse{20} Und die Schwester der Tachpenes gebar ihm Genubat,
seinen Sohn, und Tachpenes zog ihn auf im Hause des Pharao, so daß
Genubat im Hause des Pharao unter den Kindern des Pharao war.
\bibleverse{21} Als nun Hadad in Ägypten hörte, daß David sich zu seinen
Vätern gelegt habe, und daß Joab, der Feldhauptmann, tot sei, sprach
Hadad zum Pharao: Laß mich in mein Land ziehen! \bibleverse{22} Der
Pharao sprach zu ihm: Was mangelt dir bei mir, daß du in dein Land
ziehen willst? Er sprach: Nichts; aber laß mich gehen! \bibleverse{23}
Und Gott erweckte ihm noch einen Widersacher, Reson, den Sohn Eljadas,
der von seinem Herrn Hadad-Eser, dem König zu Zoba, geflohen war.
\bibleverse{24} Der sammelte Männer um sich und war Rottenführer, als
David sie schlug; und sie zogen nach Damaskus und wohnten daselbst und
regierten zu Damaskus. \bibleverse{25} Und er war Israels Widersacher,
solange Salomo lebte, außer dem Übel, das Hadad tat; und er hatte einen
Widerwillen gegen Israel und wurde König über Syrien. \bibleverse{26}
Auch Jerobeam, der Sohn Nebats, ein Ephraimiter von Zareda, Salomos
Knecht, dessen Mutter, eine Witwe, Zeruha hieß, erhob die Hand wider den
König. \bibleverse{27} Und dies ist der Anlaß, bei welchem er die Hand
wider den König erhob: Salomo baute Millo und schloß damit eine Lücke an
der Stadt Davids, seines Vaters. \bibleverse{28} Nun war Jerobeam ein
tüchtiger Mann; als aber Salomo sah, daß der Jüngling arbeitsam war,
setzte er ihn über alle Lastträger des Hauses Josephs. \bibleverse{29}
Es begab sich aber zu jener Zeit, als Jerobeam aus Jerusalem wegging,
fand ihn der Prophet Achija von Silo auf dem Wege; der hatte einen neuen
Mantel an, und sie waren beide allein auf dem Felde. \bibleverse{30} Und
Achija faßte den neuen Mantel, den er anhatte, und zerriß ihn in zwölf
Stücke \bibleverse{31} und sprach zu Jerobeam: Nimm die zehn Stücke!
Denn also spricht der \textsc{Herr}, der Gott Israels: Siehe, ich will
das Königreich der Hand Salomos entreißen und dir zehn Stämme geben.
\bibleverse{32} Einen Stamm soll er haben, um meines Knechtes David und
um der Stadt Jerusalem willen, die ich aus allen Stämmen Israels erwählt
habe; \bibleverse{33} darum, weil sie mich verlassen und Astarte, die
Gottheit der Zidonier, Kamos, den Gott der Moabiter, und Milkom, den
Gott der Ammoniter, angebetet haben und nicht in meinen Wegen gewandelt
sind, um zu tun, was in meinen Augen recht ist, nach meinen Satzungen
und Rechten, wie sein Vater David getan hat. \bibleverse{34} Doch will
ich nicht das ganze Reich aus seiner Hand nehmen, sondern ich will ihn
als Fürst belassen sein Leben lang, um meines Knechtes David willen, den
ich erwählt habe, der meine Gebote und Satzungen beobachtet hat.
\bibleverse{35} Aber ich will das Königreich aus der Hand seines Sohnes
nehmen und dir zehn Stämme geben; \bibleverse{36} und will seinem Sohn
einen Stamm geben, daß mein Knecht David immerdar vor mir eine Leuchte
habe in der Stadt Jerusalem, die ich mir erwählt habe, um meinen Namen
dahin zu setzen. \bibleverse{37} So will ich nun dich nehmen, und du
sollst regieren über alles, was dein Herz begehrt und König sein über
Israel. \bibleverse{38} Wirst du nun allem gehorchen, was ich dir
gebieten werde, und in meinen Wegen wandeln und tun, was in meinen Augen
recht ist, daß du meine Satzungen und meine Gebote beobachtest, wie mein
Knecht David getan hat, so will ich mit dir sein und dir ein beständiges
Haus bauen, wie ich es David gebaut habe, und will dir Israel geben.
\bibleverse{39} Und ich will den Samen Davids um deswillen demütigen,
doch nicht immerdar! \bibleverse{40} Salomo aber trachtete, Jerobeam zu
töten; da machte sich Jerobeam auf und floh nach Ägypten zu Sisak, dem
König von Ägypten, und verblieb in Ägypten, bis Salomo starb.
\bibleverse{41} Was aber mehr von Salomo zu sagen ist, und alles, was er
getan hat, und seine Weisheit, ist das nicht geschrieben im Buche der
Geschichte Salomos? \bibleverse{42} Die Zeit aber, während welcher
Salomo über ganz Israel zu Jerusalem regierte, beträgt vierzig Jahre.
\bibleverse{43} Und Salomo entschlief mit seinen Vätern und ward
begraben in der Stadt Davids, seines Vaters; und Rehabeam, sein Sohn,
ward König an seiner Statt.

\hypertarget{section-11}{%
\section{12}\label{section-11}}

\bibleverse{1} Und Rehabeam zog gen Sichem; denn ganz Israel war gen
Sichem gekommen, um ihn zum König zu machen. \bibleverse{2} Als aber
Jerobeam, der Sohn Nebats, solches hörte, da er noch in Ägypten war
(denn dahin war er vor dem König Salomo geflohen), kehrte er aus Ägypten
zurück. \bibleverse{3} Und man sandte hin und ließ ihn rufen. Da kamen
Jerobeam und die ganze Gemeinde Israel und redeten mit Rehabeam und
sprachen: \bibleverse{4} Dein Vater hat unser Joch zu hart gemacht; so
mache du nun den harten Dienst deines Vaters und das schwere Joch,
welches er uns aufgelegt hat, leichter, so wollen wir dir untertänig
sein! \bibleverse{5} Er aber sprach zu ihnen: Geht hin für drei Tage,
alsdann kommt wieder zu mir! Und das Volk ging hin. \bibleverse{6} Da
hielt der König Rehabeam einen Rat mit den Ältesten, die vor seinem
Vater Salomo gestanden, als er noch lebte, und sprach: Wie ratet ihr,
daß wir diesem Volk antworten sollen? \bibleverse{7} Sie sprachen zu
ihm: Wirst du heute diesem Volk einen Gefallen tun und ihm zu Willen
sein und sie erhören und ihm gute Worte geben, so werden sie dir dienen
dein Leben lang! \bibleverse{8} Aber er verließ den Rat der Ältesten,
den sie ihm gegeben hatten, und hielt Rat mit den Jungen, die mit ihm
aufgewachsen waren, die vor ihm standen. \bibleverse{9} Und er sprach zu
ihnen: Was ratet ihr, daß wir diesem Volk antworten, welches zu mir
gesagt und gesprochen hat: Mache das Joch leichter, welches dein Vater
auf uns gelegt hat? \bibleverse{10} Da sprachen zu ihm die Jungen, die
mit ihm aufgewachsen waren: Du sollst dem Volk, das zu dir gesagt hat:
Dein Vater hat unser Joch zu schwer gemacht, du aber mache es uns
leichter, dem sollst du so antworten: ``Mein kleiner Finger ist dicker
als meines Vaters Lenden! \bibleverse{11} Und nun, hat mein Vater ein
schweres Joch auf euch geladen, so will ich euch noch mehr aufladen! Hat
mein Vater euch mit Geißeln gezüchtigt, so will ich euch mit Skorpionen
züchtigen!'' \bibleverse{12} Als nun Jerobeam samt dem ganzen Volk am
dritten Tage zu Rehabeam kam, wie der König gesagt hatte: ``Kommt am
dritten Tag zu mir!'' \bibleverse{13} da gab der König dem Volk eine
harte Antwort und verließ den Rat, welchen ihm die Ältesten gegeben
hatten, \bibleverse{14} und redete mit ihnen nach dem Rat der Jungen und
sprach: Mein Vater hat euch mit Geißeln gezüchtigt, ich aber will euch
mit Skorpionen züchtigen! \bibleverse{15} Also willfahrte der König dem
Volke nicht; denn es ward so vom \textsc{Herrn} gefügt, auf daß er sein
Wort erfülle, welches der \textsc{Herr} durch Achija von Silo zu
Jerobeam, dem Sohne Nebats, geredet hatte. \bibleverse{16} Als nun ganz
Israel sah, daß der König ihnen kein Gehör schenkte, antwortete das Volk
dem König und sprach: Was haben wir für Anteil an David? Wir haben
nichts zu erben von dem Sohne Isais! Israel, auf zu deinen Hütten! Und
du, David, sieh zu deinem Haus! \bibleverse{17} Also ging Israel in
seine Hütten, und Rehabeam regierte nur über die Kinder Israel, die in
den Städten Judas wohnten. \bibleverse{18} Als aber der König Rehabeam
den Fronmeister Adoram hinsandte, bewarf ihn ganz Israel mit Steinen;
der König Rehabeam aber sputete sich und stieg auf seinen Wagen, um nach
Jerusalem zu fliehen. \bibleverse{19} Also fiel Israel ab vom Hause
Davids bis auf diesen Tag. \bibleverse{20} Als nun ganz Israel hörte,
daß Jerobeam wiedergekommen sei, sandten sie hin und beriefen ihn in die
Volksversammlung und machten ihn zum König über ganz Israel, und niemand
folgte dem Hause Davids als allein der Stamm Juda. \bibleverse{21} Als
aber Rehabeam nach Jerusalem kam, versammelte er das ganze Haus Juda und
den Stamm Benjamin, etwa 180000 junge Männer, um wider das Haus Israel
zu streiten und das Königtum wieder an Rehabeam, den Sohn Salomos, zu
bringen. \bibleverse{22} Aber das Wort Gottes erging an Semaja, den Mann
Gottes, also: \bibleverse{23} Sage zu Rehabeam, dem Sohne Salomos, dem
König Judas, und zum Hause Juda und zu Benjamin und dem übrigen Volk und
sprich: \bibleverse{24} So spricht der \textsc{Herr}: Ihr sollt nicht
hinaufziehen, um wider eure Brüder, die Kinder Israel, zu streiten!
Jedermann gehe wieder heim; denn solches ist von mir geschehen! Und sie
folgten dem Worte des \textsc{Herrn} und kehrten um, wie der
\textsc{Herr} gesagt hatte. \bibleverse{25} Jerobeam aber baute Sichem
auf dem Gebirge Ephraim und wohnte darin und zog aus von dort und baute
Pnuel. \bibleverse{26} Jerobeam aber gedachte in seinem Herzen: Das
Königreich wird nun wieder dem Hause Davids zufallen! \bibleverse{27}
Wenn dieses Volk hinaufgehen soll, um im Hause des \textsc{Herrn} zu
Jerusalem zu opfern, so wird sich das Herz dieses Volkes zu ihrem Herrn,
zu Rehabeam, dem König von Juda, wenden, und sie werden mich töten und
sich wieder Rehabeam, dem König von Juda, zuwenden! \bibleverse{28}
Darum hielt der König einen Rat und machte zwei goldene Kälber und
sprach zum Volk: Es ist zu viel für euch, nach Jerusalem hinaufzugehen!
Siehe, das sind deine Götter, Israel, die dich aus Ägyptenland geführt
haben! \bibleverse{29} Und er stellte das eine zu Bethel auf und tat das
andere nach Dan. \bibleverse{30} Aber diese Tat ward Israel zur Sünde;
und das Volk lief zu dem einen Kalbe bis gen Dan. \bibleverse{31} Er
machte auch ein Höhenheiligtum und bestellte aus dem ganzen Volk Leute
zu Priestern, die nicht von den Kindern Levi waren. \bibleverse{32}
Ferner setzte Jerobeam ein Fest an, am fünfzehnten Tag des achten
Monats, wie das Fest in Juda, und opferte auf dem Altar. Also tat er in
Bethel, indem er den Kälbern opferte, die er gemacht hatte, und er
bestellte zu Bethel die Priester der Höhen, die er gemacht hatte.
\bibleverse{33} Und er opferte auf dem Altar, den er in Bethel gemacht
hatte, am fünfzehnten Tage des achten Monats, des Monats, welchen er aus
eigenem Herzen erdacht hatte; und er veranstaltete den Kindern Israel
ein Fest und opferte auf dem Altar und räucherte.

\hypertarget{section-12}{%
\section{13}\label{section-12}}

\bibleverse{1} Aber siehe, ein Mann Gottes kam von Juda durch das Wort
des \textsc{Herrn} gen Bethel, als Jerobeam eben bei dem Altar stand, um
zu räuchern. \bibleverse{2} Und er rief wider den Altar durch das Wort
des \textsc{Herrn} und sprach: Altar! Altar! So spricht der
\textsc{Herr}: Siehe, es wird dem Hause Davids ein Sohn geboren werden
namens Josia, der wird auf dir die Priester der Höhen opfern, die auf
dir räuchern, und man wird Menschengebeine auf dir verbrennen!
\bibleverse{3} Und am gleichen Tage gab er ein Zeichen, und sprach: Das
ist das Zeichen, daß der \textsc{Herr} solches geredet hat: Siehe, der
Altar wird bersten und die Asche, die darauf ist, verschüttet werden!
\bibleverse{4} Als aber der König das Wort des Mannes Gottes hörte, der
wider den Altar zu Bethel rief, streckte Jerobeam seine Hand aus bei dem
Altar und sprach: Greifet ihn! Aber seine Hand, die er wider ihn
ausgestreckt hatte, ward steif, so daß er sie nicht wieder zu sich
ziehen konnte. \bibleverse{5} Und der Altar barst, und die Asche ward
vom Altar verschüttet, gemäß dem Zeichen, das der Mann Gottes durch das
Wort des \textsc{Herrn} angekündigt hatte. \bibleverse{6} Da hob der
König an und sprach zu dem Manne Gottes: Besänftige doch das Angesicht
des \textsc{Herrn}, deines Gottes, und bitte für mich, daß meine Hand
mir wieder gegeben werde! Da besänftigte der Mann Gottes das Angesicht
des \textsc{Herrn}. Und der König konnte die Hand wieder an sich ziehen,
und sie ward wieder wie zuvor. \bibleverse{7} Da sprach der König zu dem
Manne Gottes: Komm mit mir heim und erlabe dich! Ich will dir auch ein
Geschenk geben. \bibleverse{8} Aber der Mann Gottes sprach zum König:
Wenn du mir auch dein halbes Haus gäbest, so käme ich nicht mit dir;
denn ich würde an diesem Ort kein Brot essen und kein Wasser trinken.
\bibleverse{9} Denn also wurde mir durch das Wort des \textsc{Herrn}
geboten und gesagt: Du sollst daselbst kein Brot essen und kein Wasser
trinken und nicht wieder auf dem Wege zurückkehren, den du gegangen
bist! \bibleverse{10} Und er ging einen andern Weg und kehrte nicht
wieder auf dem gleichen Wege zurück, auf welchem er nach Bethel gekommen
war. \bibleverse{11} Aber in Bethel wohnte ein alter Prophet. Zu dem
kamen seine Söhne und erzählten ihm alles, was der Mann Gottes an jenem
Tage in Bethel getan hatte; auch die Worte, die er zum Könige geredet
hatte, erzählten sie ihrem Vater. \bibleverse{12} Da sprach ihr Vater zu
ihnen: Welchen Weg ist er gegangen? Da zeigten ihm seine Söhne den Weg,
den der Mann Gottes, der von Juda gekommen, eingeschlagen hatte.
\bibleverse{13} Er aber sprach zu seinen Söhnen: Sattelt mir den Esel!
Und als sie ihm den Esel gesattelt hatten, \bibleverse{14} setze er sich
darauf und ritt dem Manne Gottes nach und fand ihn unter einer Eiche
sitzen und sprach zu ihm: Bist du der Mann Gottes, der von Juda gekommen
ist? Er sprach: Ja! \bibleverse{15} Da sprach er zu ihm: Komm mit mir
heim und iß etwas! \bibleverse{16} Er aber sprach: Ich kann nicht
umkehren und mit dir kommen; ich will auch mit dir weder Brot essen noch
Wasser trinken an diesem Ort; \bibleverse{17} denn durch das Wort des
\textsc{Herrn} ist zu mir gesagt worden: Du sollst daselbst weder Brot
essen noch Wasser trinken; du sollst nicht auf dem gleichen Wege
zurückkehren, auf dem du hingegangen bist! \bibleverse{18} Aber jener
sprach zu ihm: Ich bin auch ein Prophet wie du, und ein Engel hat durch
das Wort des \textsc{Herrn} folgendermaßen mit mir geredet: Führe ihn
zurück in dein Haus, damit er Brot esse und Wasser trinke!
\bibleverse{19} Er log es ihm aber vor. Da kehrte er mit ihm um, daß er
in seinem Hause Brot esse und Wasser trinke. \bibleverse{20} Als sie
aber zu Tische saßen, kam das Wort des \textsc{Herrn} zum Propheten, der
ihn zurückgeführt hatte, \bibleverse{21} und er rief dem Manne Gottes
zu, der von Juda gekommen war, und sprach: So spricht der \textsc{Herr}:
Weil du dem Munde des \textsc{Herrn} ungehorsam gewesen bist und das
Gebot nicht gehalten hast, das dir der \textsc{Herr}, dein Gott, geboten
hat, \bibleverse{22} sondern umgekehrt bist und Brot gegessen und Wasser
getrunken hast an diesem Ort, davon er dir sagte, du solltest weder Brot
essen noch Wasser trinken, so soll dein Leichnam nicht in deiner Väter
Grab kommen! \bibleverse{23} Und nachdem er Brot gegessen und getrunken
hatte, sattelte man dem Propheten, den jener zurückgeführt hatte, den
Esel. \bibleverse{24} Als er nun ging, traf ihn auf dem Wege ein Löwe;
der tötete ihn, und sein Leichnam lag hingestreckt auf dem Wege. Und der
Esel stand neben ihm, und der Löwe stand neben dem Leichnam.
\bibleverse{25} Und siehe, als Leute vorbeigingen, sahen sie den
Leichnam auf dem Wege liegen und den Löwen bei dem Leichnam stehen, und
sie kamen und sagten es in der Stadt, wo der alte Prophet wohnte.
\bibleverse{26} Als nun der Prophet, der ihn vom Wege zurückgeholt
hatte, das hörte, sprach er: Es ist der Mann Gottes, der dem Munde des
\textsc{Herrn} ungehorsam gewesen ist; darum hat ihn der \textsc{Herr}
dem Löwen übergeben, der hat ihn zerrissen und getötet nach dem Wort,
das ihm der \textsc{Herr} gesagt hat! \bibleverse{27} Und er redete mit
seinen Söhnen und sprach: Sattelt mir den Esel! Und als sie ihn
gesattelt hatten, \bibleverse{28} ging er hin und fand seinen Leichnam
auf dem Wege liegen und den Esel und den Löwen neben dem Leichnam
stehen. Der Löwe hatte den Leichnam nicht gefressen und den Esel nicht
zerrissen. \bibleverse{29} Da hob der Prophet den Leichnam des Mannes
Gottes auf und legte ihn auf den Esel und führte ihn zurück, und sie
kamen in die Stadt des alten Propheten, um ihn zu beklagen und zu
begraben. \bibleverse{30} Und er legte dessen Leichnam in sein eigenes
Grab, und sie klagten um ihn: Ach, mein Bruder! \bibleverse{31} Und als
er ihn begraben hatte, sprach er zu seinen Söhnen: Wenn ich sterbe, so
begrabt mich in dem Grabe, darin der Mann Gottes begraben worden ist,
und legt meine Gebeine neben seine Gebeine; \bibleverse{32} denn es wird
gewiß geschehen, was er durch das Wort des \textsc{Herrn} ausgerufen hat
wider den Altar zu Bethel und wider alle Höhenheiligtümer, die in den
Städten Samarias sind. \bibleverse{33} Aber nach dieser Geschichte
kehrte sich Jerobeam nicht von seinem bösen Wege, sondern bestellte
wieder Höhenpriester aus dem gesamten Volk; zu wem er Lust hatte, dessen
Hand füllte er, und der ward Höhenpriester. \bibleverse{34} Und dies
wurde dem Hause Jerobeams zur Sünde, so daß er vernichtet und aus dem
Lande vertilgt werden mußte.

\hypertarget{section-13}{%
\section{14}\label{section-13}}

\bibleverse{1} Zu jener Zeit ward Abija, der Sohn Jerobeams, krank.
\bibleverse{2} Und Jerobeam sprach zu seinem Weibe: Mache dich auf und
verstelle dich, daß niemand merke, daß du Jerobeams Weib bist, und gehe
nach Silo; siehe, daselbst ist der Prophet Achija, der von mir geredet
hat, daß ich König über dieses Volk sein sollte; \bibleverse{3} und nimm
mit dir zehn Brote und Kuchen und einen Krug Honig und gehe zu ihm, daß
er dir kundtue, wie es dem Knaben gehen wird! \bibleverse{4} Und das
Weib Jerobeams tat so und machte sich auf und ging hin nach Silo und kam
in das Haus Achijas. Achija aber konnte nicht sehen, denn seine Augen
waren starr geworden vor Alter. \bibleverse{5} Aber der \textsc{Herr}
hatte zu Achija gesprochen: Siehe, das Weib Jerobeams kommt, um von dir
ein Wort zu erlangen betreffs ihres Sohnes; denn er ist krank. So rede
nun mit ihr so und so! Als sie nun hineinkam, stellte sie sich fremd.
\bibleverse{6} Als aber Achija das Geräusch ihrer Füße hörte, wie sie
zur Tür hereinkam, sprach er: Komm herein, du Weib Jerobeams! Warum
stellst du dich so fremd? Ich bin mit einer harten Botschaft an dich
beauftragt! \bibleverse{7} Gehe hin, sage Jerobeam: So spricht der
\textsc{Herr}, der Gott Israels: Weil ich dich aus dem Volk erhöht und
zum Fürsten über mein Volk Israel gesetzt habe, \bibleverse{8} also daß
ich das Königreich dem Hause Davids entrissen und es dir gegeben habe,
du aber nicht gewesen bist wie mein Knecht David, der meine Gebote
beobachtete und mir nachwandelte von ganzem Herzen, so daß er nur tat,
was in meinen Augen recht ist; \bibleverse{9} weil du aber mehr Böses
getan hast als alle, die vor dir gewesen sind; weil du hingegangen bist
und dir andere Götter und gegossene Bilder gemacht hast, so daß du mich
zum Zorne reiztest, und mich hinter deinen Rücken geworfen hast;
\bibleverse{10} darum siehe, bringe ich Unglück über das Haus Jerobeams,
und ich will ausrotten von Jerobeam, was männlich ist, Mündige und
Unmündige in Israel, und ich will die Nachkommen des Hauses Jerobeams
ausfegen, wie man Kot ausfegt, bis es ganz aus sei mit ihm.
\bibleverse{11} Wer von Jerobeam in der Stadt stirbt, den sollen die
Hunde fressen; wer aber auf dem Felde stirbt, den sollen die Vögel des
Himmels fressen; denn der \textsc{Herr} hat es gesagt! \bibleverse{12}
So mache dich nun auf und gehe heim, und wenn dein Fuß die Stadt
betritt, wird der Knabe sterben. \bibleverse{13} Und ganz Israel wird
ihn beklagen, und sie werden ihn begraben; denn von Jerobeam wird dieser
allein in ein Grab kommen, weil an ihm vor dem \textsc{Herrn}, dem Gott
Israels, etwas Gutes gefunden worden ist im Hause Jerobeams.
\bibleverse{14} Der \textsc{Herr} aber wird einen König über Israel
erwecken, der das Haus Jerobeams ausrotten soll an jenem Tag. Und wie
steht es schon jetzt! \bibleverse{15} Und der \textsc{Herr} wird Israel
schlagen, daß es schwankt wie ein Rohr im Wasser; und er wird Israel
ausrotten aus diesem guten Land, welches er ihren Vätern gegeben hat,
und wird sie zerstreuen jenseits des Stromes Euphrat, weil sie ihre
Ascheren gemacht haben, den \textsc{Herrn} zu erzürnen. \bibleverse{16}
Und er wird Israel dahingeben um der Sünde Jerobeams willen, die er
begangen und zu welcher er Israel verführt hat. \bibleverse{17} Da
machte sich das Weib Jerobeams auf, ging hin und kam gen Tirza. Und als
sie die Schwelle des Hauses betrat, starb der Knabe. \bibleverse{18} Und
sie begruben ihn, und ganz Israel beklagte ihn nach dem Wort des
\textsc{Herrn}, das er durch seinen Knecht, den Propheten Achija,
geredet hatte. \bibleverse{19} Was aber mehr von Jerobeam zu sagen ist,
wie er gestritten und wie er regiert hat, siehe, das ist geschrieben im
Buch der Chronik der Könige von Israel. \bibleverse{20} Die Zeit aber,
während der Jerobeam regiert hat, betrug zweiundzwanzig Jahre. Und er
legte sich zu seinen Vätern. Und sein Sohn Nadab ward König an seiner
Statt. \bibleverse{21} Rehabeam aber, der Sohn Salomos, regierte in
Juda. Einundvierzig Jahre alt war Rehabeam, als er König ward, und
regierte siebzehn Jahre lang zu Jerusalem, in der Stadt, die der
\textsc{Herr} aus allen Stämmen Israels erwählt hatte, um seinen Namen
daselbst wohnen zu lassen. Und seine Mutter hieß Naama, eine
Ammoniterin. \bibleverse{22} Und Juda tat, was dem \textsc{Herrn} übel
gefiel, und sie reizten ihn zum Eifer durch ihre Sünden, welche sie
taten, mehr als alles, was ihre Väter getan hatten. \bibleverse{23} Denn
sie bauten auch Höhen und Säulen und Ascheren auf allen hohen Hügeln und
unter allen grünen Bäumen. \bibleverse{24} Und es waren auch Schandbuben
im Lande; die taten nach allen Greueln der Heiden, die der \textsc{Herr}
vor den Kindern Israel vertrieben hatte. \bibleverse{25} Es begab sich
aber im fünften Jahre des Königs Rehabeam, daß Sisak, der König von
Ägypten, wider Jerusalem heraufzog; \bibleverse{26} der nahm die Schätze
des Hauses des \textsc{Herrn} und die Schätze des Palastes des Königs,
alles nahm er, auch alle goldenen Schilde, die Salomo hatte machen
lassen. \bibleverse{27} An deren Statt ließ der König Rehabeam eherne
Schilde machen und versah damit die obersten Trabanten, welche die Tür
am Hause des Königs hüteten. \bibleverse{28} Und sooft der König in das
Haus des \textsc{Herrn} ging, trugen sie die Trabanten und brachten sie
darnach wieder in die Kammern der Trabanten. \bibleverse{29} Was aber
mehr von Rehabeam zu sagen ist, und alles, was er getan hat, ist das
nicht geschrieben im Buch der Chronik der Könige von Juda?
\bibleverse{30} Es war aber Krieg zwischen Rehabeam und Jerobeam, ihr
Leben lang. \bibleverse{31} Und Rehabeam legte sich zu seinen Vätern und
ward begraben mit seinen Vätern in der Stadt Davids; seine Mutter aber
hieß Naama, eine Ammoniterin. Und sein Sohn Abijam ward König an seiner
Statt.

\hypertarget{section-14}{%
\section{15}\label{section-14}}

\bibleverse{1} Im achtzehnten Regierungs-Jahre des Königs Jerobeam, des
Sohnes Nebats, ward Abija König über Juda. \bibleverse{2} Er regierte
drei Jahre lang zu Jerusalem. Seine Mutter hieß Maacha, eine Tochter
Abisaloms. \bibleverse{3} Und er wandelte in allen Sünden seines Vaters,
die er vor ihm getan hatte, und sein Herz war nicht völlig mit dem
\textsc{Herrn}, seinem Gott, wie das Herz seines Vaters David.
\bibleverse{4} Doch um Davids willen gab der \textsc{Herr}, sein Gott,
ihm eine Leuchte zu Jerusalem, indem er seinen Sohn ihm nachfolgen und
Jerusalem bestehen ließ, \bibleverse{5} weil David getan hatte, was in
den Augen des \textsc{Herrn} recht war, und nicht gewichen war von
allem, was er ihm gebot, sein Leben lang, außer in der Sache Urijas, des
Hetiters. \bibleverse{6} Es war aber Krieg zwischen Rehabeam und
Jerobeam, ihr Leben lang. \bibleverse{7} Was aber mehr von Abija zu
sagen ist, und was er getan hat, ist das nicht geschrieben in der
Chronik der Könige von Juda? Es war aber Krieg zwischen Abija und
Jerobeam. \bibleverse{8} Und Abija legte sich zu seinen Vätern, und sie
begruben ihn in der Stadt Davids. Und Asa, sein Sohn, ward König an
seiner Statt. \bibleverse{9} Im zwanzigsten Jahre Jerobeams, des Königs
von Israel, ward Asa König über Juda; \bibleverse{10} er regierte
einundvierzig Jahre lang zu Jerusalem. Seine Mutter hieß Maacha, eine
Tochter Abisaloms. \bibleverse{11} Und Asa tat, was dem \textsc{Herrn}
wohlgefiel, wie sein Vater David. \bibleverse{12} Denn er schaffte die
Schandbuben aus dem Lande und entfernte alle schändlichen Götzen, die
seine Väter gemacht hatten. \bibleverse{13} Dazu setzte er auch seine
Mutter Maacha ab, daß sie nicht mehr Gebieterin war, weil sie ein
Götzenbild der Aschera gemacht hatte. Und Asa rottete ihr Götzenbild aus
und verbrannte es am Bach Kidron. Aber die Höhen tat er nicht ab.
\bibleverse{14} Doch war das Herz Asas völlig mit dem \textsc{Herrn}
sein Leben lang. \bibleverse{15} Und das Silber und Gold und die Geräte,
was sein Vater geweiht hatte und was er selbst weihte, das brachte er in
das Haus des \textsc{Herrn}. \bibleverse{16} Es war aber Krieg zwischen
Asa und Baesa, dem Könige von Israel, ihr Leben lang. \bibleverse{17}
Denn Baesa, der König von Israel, zog herauf wider Juda und baute Rama,
um dem König von Juda keinen Ausgang und Eingang mehr zu lassen.
\bibleverse{18} Da nahm Asa alles Silber und Gold, das im Schatze des
Hauses des \textsc{Herrn} und im Schatze des Königshauses übrig war, und
gab es in die Hand seiner Knechte; und der König Asa sandte sie zu
Benhadad, dem Sohne Tabrimmons, des Sohnes Hesions, dem König von
Syrien, der zu Damaskus wohnte, und ließ ihm sagen: \bibleverse{19} Es
besteht ein Bund zwischen mir und dir, zwischen meinem Vater und deinem
Vater, siehe, ich sende dir ein Geschenk von Silber und Gold; gehe hin,
löse das Bündnis auf, das du mit Baesa, dem König von Israel, hast, daß
er von mir abziehe! \bibleverse{20} Und Benhadad willfahrte dem König
Asa und sandte seine Hauptleute wider die Städte Israels und schlug Jjon
und Dan und Abel-Bet-Maacha und ganz Genezareth, samt dem ganzen Lande
Naphtali. \bibleverse{21} Als Baesa solches hörte, ließ er ab, Rama zu
bauen, und blieb zu Tirza. \bibleverse{22} Der König Asa aber erließ ein
Aufgebot im ganzen Juda, so daß keiner frei blieb; Und sie nahmen von
Rama die Steine und das Holz weg, womit Baesa gebaut hatte. Und der
König Asa baute damit Geba in Benjamin, und Mizpa. \bibleverse{23} Was
aber mehr von Asa zu sagen ist, und alle seine Macht und alles, was er
getan hat, ist das nicht geschrieben in der Chronik der Könige von Juda?
Doch ward er in seinem Alter krank an den Füßen. \bibleverse{24} Und Asa
legte sich zu seinen Vätern und ward begraben mit seinen Vätern in der
Stadt Davids, seines Vaters. Und Josaphat, sein Sohn, ward König an
seiner Statt. \bibleverse{25} Nadab aber, der Sohn Jerobeams, ward König
über Israel im zweiten Jahre der Regierung Asas, des Königs von Juda,
und regierte zwei Jahre lang über Israel \bibleverse{26} und tat, was
dem \textsc{Herrn} übel gefiel, und wandelte in dem Wege seines Vaters
und in seiner Sünde, wodurch er Israel zur Sünde verführt hatte.
\bibleverse{27} Aber Baesa, der Sohn Achijas, aus dem Hause Issaschar,
machte eine Verschwörung wider ihn, und Baesa erschlug ihn zu Gibbeton,
das den Philistern gehörte; denn Nadab und ganz Israel belagerten
Gibbeton. \bibleverse{28} Also tötete ihn Baesa im dritten Jahre Asas,
des Königs von Juda, und ward König an seiner Statt. \bibleverse{29} Als
er nun König geworden war, erschlug er das ganze Haus Jerobeams und ließ
Jerobeam keine Seele übrig, bis er ihn vertilgt hatte nach dem Worte des
\textsc{Herrn}, das er durch seinen Knecht Achija von Silo geredet
hatte, um der Sünden Jerobeams willen, \bibleverse{30} die er tat, und
zu denen er Israel verführt hatte, wegen seines Reizens, womit er den
\textsc{Herrn}, den Gott Israels, zum Zorne reizte. \bibleverse{31} Was
aber mehr von Nadab zu sagen ist, und alles, was er getan hat, ist das
nicht geschrieben in der Chronik der Könige von Israel? \bibleverse{32}
Und es war Krieg zwischen Asa und Baesa, dem König von Israel, ihr Leben
lang. \bibleverse{33} Im dritten Jahr Asas, des Königs von Juda, ward
Baesa, der Sohn Achijas, König über ganz Israel zu Tirza, und er
regierte vierundzwanzig Jahre lang. \bibleverse{34} Und er tat, was dem
\textsc{Herrn} übel gefiel, und wandelte in dem Wege Jerobeams und in
seiner Sünde, wodurch er Israel zur Sünde verführt hatte.

\hypertarget{section-15}{%
\section{16}\label{section-15}}

\bibleverse{1} Aber das Wort des \textsc{Herrn} erging an Jehu, den Sohn
Hananis, wider Baesa, also: \bibleverse{2} Weil ich dich aus dem Staube
erhoben und dich zum Fürsten über mein Volk Israel gemacht habe und du
in dem Wege Jerobeams wandelst und mein Volk Israel zur Sünde verführst,
so daß du mich durch ihre Sünden erzürnst, \bibleverse{3} siehe, so will
ich die Nachkommen Baesas und die Nachkommen seines Hauses ausrotten und
mit deinem Haus verfahren wie mit dem Hause Jerobeams, des Sohnes
Nebats. \bibleverse{4} Wer von Baesa in der Stadt stirbt, den sollen die
Hunde fressen, und wer von ihm auf dem Felde stirbt, den sollen die
Vögel des Himmels fressen! \bibleverse{5} Was aber mehr von Baesa zu
sagen ist, und was er getan hat und seine Macht, ist das nicht
geschrieben in der Chronik der Könige von Israel? \bibleverse{6} Und
Baesa legte sich zu seinen Vätern und ward begraben zu Tirza, und sein
Sohn Ela ward König an seiner Statt. \bibleverse{7} Auch erging das Wort
des \textsc{Herrn} durch den Propheten Jehu, den Sohn Hananis, wider
Baesa und wider sein Haus, um all des Bösen willen, das er vor dem
\textsc{Herrn} tat, indem er ihn durch die Werke seiner Hände erzürnte,
so daß es wurde wie das Haus Jerobeams, und weil er dasselbe erschlagen
hatte. \bibleverse{8} Im sechsundzwanzigsten Jahre Asas, des Königs von
Juda, ward Ela, der Sohn Baesas, König über Israel zu Tirza und regierte
zwei Jahre lang. \bibleverse{9} Und sein Knecht Simri, der Oberste über
die Hälfte der Streitwagen, machte eine Verschwörung wider ihn. Er aber
war zu Tirza, trank und ward trunken im Hause Arzas, welcher über das
Haus gesetzt war zu Tirza. \bibleverse{10} Und Simri kam hinein und
schlug ihn tot im siebenundzwanzigsten Jahr Asas, des Königs von Juda,
und er ward König an seiner Statt. \bibleverse{11} Als er nun König war
und auf seinem Throne saß, erschlug er das ganze Haus Baesas und ließ
nichts von ihm übrig, was männlich war, auch dessen Bluträcher und
Freunde nicht. \bibleverse{12} Also vertilgte Simri das ganze Haus
Baesas nach dem Worte des \textsc{Herrn}, das er durch den Propheten
Jehu über Baesa geredet hatte: \bibleverse{13} um aller Sünden Baesas
und um der Sünden seines Sohnes Ela willen, die sie taten und wodurch
sie Israel zur Sünde verführten und den \textsc{Herrn}, den Gott
Israels, durch ihre Götzen erzürnten. \bibleverse{14} Was aber mehr von
Ela zu sagen ist, und alles, was er getan hat, ist das nicht geschrieben
in der Chronik der Könige von Israel? \bibleverse{15} Im
siebenundzwanzigsten Jahre Asas, des Königs von Juda, ward Simri König
zu Tirza sieben Tage lang, und das Volk lag vor Gibbeton der Philister.
\bibleverse{16} Als aber das Volk im Lager sagen hörte: Simri hat eine
Verschwörung gemacht und hat auch den König erschlagen, da machte am
selben Tage das ganze Israel im Lager Omri, den Feldhauptmann, zum König
über Israel. \bibleverse{17} Und Omri und ganz Israel mit ihm zog von
Gibbeton hinauf, und sie belagerten Tirza. \bibleverse{18} Als aber
Simri sah, daß die Stadt eingenommen war, ging er in die Burg des
Königshauses und verbrannte sich samt dem Hause des Königs
\bibleverse{19} und starb um seiner Sünden willen, die er getan hatte,
indem er tat, was dem \textsc{Herrn} übel gefiel, und indem er wandelte
in dem Wege Jerobeams und in seiner Sünde, die er tat, wodurch er Israel
zur Sünde verführt hatte. \bibleverse{20} Was aber mehr von Simri zu
sagen ist, und seine Verschwörung, die er gemacht hat, ist das nicht
geschrieben in der Chronik der Könige von Israel? \bibleverse{21} Damals
teilte sich das Volk Israel in zwei Parteien: die eine Hälfte des Volkes
hing an Tibni, dem Sohne Ginats, um ihn zum König zu machen, die andere
Hälfte aber an Omri. \bibleverse{22} Aber das Volk, das an Omri hing,
war stärker als das Volk, das an Tibni, dem Sohne Ginats, hing. Und
Tibni starb und Omri ward König. \bibleverse{23} Im einunddreißigsten
Jahre Asas, des Königs von Juda, ward Omri König über Israel und
regierte zwölf Jahre lang. \bibleverse{24} Er kaufte aber den Berg
Samaria von Semer um zwei Talente Silber und baute auf dem Berge und
hieß die Stadt, die er baute, Samaria nach dem Namen Semers, des Herrn
des Berges. \bibleverse{25} Und Omri tat, was dem \textsc{Herrn} übel
gefiel, und war ärger als alle, die vor ihm gewesen. \bibleverse{26} Und
er wandelte in allen Wegen Jerobeams, des Sohnes Nebats, und in seinen
Sünden, wodurch er Israel zur Sünde verführte, so daß sie den
\textsc{Herrn}, den Gott Israels, durch ihre Götzen erzürnten.
\bibleverse{27} Was aber mehr von Omri zu sagen ist, was er getan, und
seine Tapferkeit, die er bewiesen hat, ist das nicht geschrieben in der
Chronik der Könige von Israel? \bibleverse{28} Und Omri legte sich zu
seinen Vätern und ward begraben zu Samaria, und Ahab, sein Sohn, ward
König an seiner Statt. \bibleverse{29} Im achtunddreißigsten Jahre Asas,
des Königs von Juda, ward Ahab, der Sohn Omris, König über Israel und
regierte zu Samaria zweiundzwanzig Jahre lang über Israel.
\bibleverse{30} Und Ahab, der Sohn Omris, tat, was dem \textsc{Herrn}
übel gefiel, mehr als alle, die vor ihm gewesen waren. \bibleverse{31}
Denn das war noch das Geringste, daß er in den Sünden Jerobeams, des
Sohnes Nebats, wandelte; er nahm sogar Isebel, die Tochter Et-Baals, des
Königs der Zidonier, zum Weibe und ging hin und diente dem Baal und
betete ihn an. \bibleverse{32} Und er richtete dem Baal einen Altar auf
im Hause Baals, welches er zu Samaria baute. \bibleverse{33} Ahab machte
auch eine Aschera, also daß Ahab mehr tat, was den \textsc{Herrn}, den
Gott Israels, erzürnte, als alle Könige von Israel, die vor ihm gewesen
waren. \bibleverse{34} Zu derselben Zeit baute Hiel von Bethel Jericho
wieder auf. Es kostete ihn seinen erstgeborenen Sohn Abiram, als er
ihren Grund legte, und seinen jüngsten Sohn Segub, als er ihre Tore
setzte, nach dem Worte des \textsc{Herrn}, welches er durch Josua, den
Sohn Nuns, geredet hatte.

\hypertarget{section-16}{%
\section{17}\label{section-16}}

\bibleverse{1} Und Elia, der Tisbiter, aus Tisbe-Gilead, sprach zu Ahab:
So wahr der \textsc{Herr}, der Gott Israels, lebt, vor dessen Angesicht
ich stehe, es soll diese Jahre weder Tau noch Regen fallen, es sei denn,
daß ich es sage! \bibleverse{2} Und das Wort des \textsc{Herrn} erging
an ihn also: \bibleverse{3} Gehe fort von hier und wende dich gegen
Morgen und verbirg dich am Bache Krit, der gegen den Jordan fließt!
\bibleverse{4} Und du sollst aus dem Bache trinken, und ich habe den
Raben geboten, daß sie dich daselbst versorgen. \bibleverse{5} Da ging
er hin und tat nach dem Worte des \textsc{Herrn}; er ging und setzte
sich an den Bach Krit, der gegen den Jordan fließt. \bibleverse{6} Und
die Raben brachten ihm Brot und Fleisch am Morgen und am Abend, und er
trank aus dem Bache. \bibleverse{7} Es begab sich aber nach einiger
Zeit, daß der Bach vertrocknete; denn es war kein Regen im Lande.
\bibleverse{8} Da erging das Wort des \textsc{Herrn} an ihn also:
\bibleverse{9} Mache dich auf und gehe nach Zarpat, das bei Zidon liegt,
und bleibe daselbst; siehe, ich habe daselbst einer Witwe geboten, daß
sie dich mit Nahrung versorge! \bibleverse{10} Und er machte sich auf
und ging nach Zarpat. Und als er an das Stadttor kam, siehe, da war eine
Witwe, die Holz auflas. Und er rief sie an und sprach: Hole mir doch ein
wenig Wasser im Geschirr, daß ich trinke! \bibleverse{11} Als sie nun
hinging zu holen, rief er ihr nach und sprach: Ich bitte dich, bringe
mir auch einen Bissen Brot mit! \bibleverse{12} Sie sprach: So wahr der
\textsc{Herr}, dein Gott, lebt, ich habe nichts Gebackenes, sondern nur
eine Handvoll Mehl im Faß und ein wenig Öl im Krug! Und siehe, ich habe
ein paar Hölzer aufgelesen und gehe hin und will mir und meinem Sohn
etwas zurichten, daß wir es essen und darnach sterben. \bibleverse{13}
Elia sprach zu ihr: Fürchte dich nicht! Gehe hin und mache es, wie du
gesagt hast; doch mache mir davon zuerst ein kleines Gebackenes und
bringe es mir heraus; dir aber und deinem Sohne sollst du hernach etwas
machen. \bibleverse{14} Denn also spricht der \textsc{Herr}, der Gott
Israels: Das Mehlfaß soll nicht leer werden und das Öl im Kruge nicht
mangeln bis auf den Tag, da der \textsc{Herr} auf Erden regnen lassen
wird! \bibleverse{15} Sie ging hin und tat, wie Elia gesagt hatte. Und
er aß und sie auch und ihr Haus eine Zeitlang. \bibleverse{16} Das
Mehlfaß ward nicht leer, und das Öl im Kruge mangelte nicht, nach dem
Worte des \textsc{Herrn}, das er durch Elia geredet hatte.
\bibleverse{17} Aber nach diesen Geschichten ward der Sohn des Weibes,
der Hauswirtin, krank, und seine Krankheit ward so schwer, daß kein Atem
mehr in ihm blieb. \bibleverse{18} Und sie sprach zu Elia: Du Mann
Gottes, was habe ich mit dir zu schaffen? Du bist zu mir hergekommen,
daß meiner Missetat gedacht werde und mein Sohn sterbe! \bibleverse{19}
Er sprach zu ihr: Gib mir deinen Sohn her! Und er nahm ihn von ihrem
Schoß und trug ihn hinauf in das Obergemach, wo er wohnte, und legte ihn
auf sein Bett; \bibleverse{20} und er rief den \textsc{Herrn} an und
sprach: \textsc{Herr}, mein Gott, hast du auch der Witwe, bei der ich zu
Gaste bin, so übel getan, daß du ihren Sohn sterben lässest?
\bibleverse{21} Und er streckte sich dreimal über das Kind aus und rief
den \textsc{Herrn} an und sprach: \textsc{Herr}, mein Gott, laß doch die
Seele dieses Kindes wieder in dasselbe zurückkehren! \bibleverse{22} Und
der \textsc{Herr} erhörte die Stimme des Elia. Und die Seele des Kindes
kam wieder in dasselbe, und es ward lebendig. \bibleverse{23} Und Elia
nahm das Kind und brachte es von dem Obergemach ins Haus hinab und gab
es seiner Mutter und sprach: Siehe da, dein Sohn lebt! \bibleverse{24}
Da sprach das Weib zu Elia: Nun erkenne ich, daß du ein Mann Gottes bist
und daß das Wort des \textsc{Herrn} in deinem Munde Wahrheit ist!

\hypertarget{section-17}{%
\section{18}\label{section-17}}

\bibleverse{1} Und nach langer Zeit, im dritten Jahre, erging das Wort
des \textsc{Herrn} an Elia also: Gehe hin, zeige dich Ahab, damit ich
regen lasse auf den Erdboden. \bibleverse{2} Und Elia ging hin, um sich
Ahab zu zeigen. Es war aber eine große Hungersnot zu Samaria.
\bibleverse{3} Und Ahab rief Obadja, seinen Hofmeister. Obadja aber
fürchtete den \textsc{Herrn} sehr. \bibleverse{4} Denn als Isebel die
Propheten des \textsc{Herrn} ausrottete, nahm Obadja hundert Propheten
und verbarg sie in den Höhlen, hier fünfzig und dort fünfzig, und
versorgte sie mit Brot und Wasser. \bibleverse{5} So sprach nun Ahab zu
Obadja: Ziehe durch das Land, zu allen Wasserbrunnen und zu allen
Bächen; vielleicht finden wir Gras, um die Pferde und Maultiere am Leben
zu erhalten, daß nicht alles Vieh umkomme! \bibleverse{6} Und sie
teilten das Land unter sich, um es zu durchziehen. Ahab zog allein auf
einem Wege und Obadja auch allein auf einem andern Weg. \bibleverse{7}
Als nun Obadja auf dem Wege war, siehe, da begegnete ihm Elia. Und als
er ihn erkannte, fiel er auf sein Angesicht und sprach: Bist du nicht
mein Herr Elia? \bibleverse{8} Er sprach zu ihm: Doch! Gehe hin und sage
deinem Herrn: Siehe, Elia ist hier! \bibleverse{9} Er aber sprach: Was
habe ich gesündigt, daß du deinen Knecht in die Hand Ahabs geben willst,
daß er mich töte? \bibleverse{10} So wahr der \textsc{Herr}, dein Gott,
lebt, es gibt kein Volk noch Königreich, dahin mein Herr nicht gesandt
hätte, dich zu suchen. Und wenn sie sprachen: ``Er ist nicht hier'',
nahm er einen Eid von jenem Königreich und von jenem Volk, daß man dich
nicht gefunden habe. \bibleverse{11} Und du sprichst nun: Gehe hin, sage
deinem Herrn: Siehe, Elia ist hier! \bibleverse{12} Wenn ich nun von dir
ginge, so würde dich der Geist des \textsc{Herrn} hinwegnehmen, ich weiß
nicht wohin; und wenn ich dann käme und es Ahab sagte, und er fände dich
nicht, so würde er mich töten; und doch fürchtet dein Knecht den
\textsc{Herrn} von Jugend auf! \bibleverse{13} Ist meinem \textsc{Herrn}
nicht gesagt worden, was ich getan habe, als Isebel die Propheten des
\textsc{Herrn} tötete, daß ich von den Propheten des \textsc{Herrn}
hundert verbarg, hier fünfzig und dort fünfzig, in Höhlen, und sie mit
Brot und Wasser versorgte? \bibleverse{14} Und du sprichst nun: Gehe
hin, sage deinem Herrn: Siehe, Elia ist hier! Er würde mich ja töten!
\bibleverse{15} Elia sprach: So wahr der \textsc{Herr} der Heerscharen
lebt, vor dem ich stehe, ich will mich ihm heute zeigen! \bibleverse{16}
Da ging Obadja hin, Ahab entgegen, und sagte es ihm; Ahab aber kam Elia
entgegen. \bibleverse{17} Und als Ahab den Elia sah, sprach Ahab zu ihm:
Bist du da, der Israel ins Unglück bringt? \bibleverse{18} Er aber
sprach: Nicht ich bringe Israel ins Unglück, sondern du und deines
Vaters Haus, weil ihr die Gebote des \textsc{Herrn} verlassen habt und
den Baalen nachwandelt! \bibleverse{19} Wohlan, so sende nun hin und
versammle zu mir ganz Israel auf den Berg Karmel, dazu die 450 Propheten
des Baal und die 400 Propheten der Aschera, die am Tische der Isebel
essen! \bibleverse{20} Also sandte Ahab hin unter alle Kinder Israel und
versammelte die Propheten auf dem Karmel. \bibleverse{21} Da trat Elia
zu allem Volk und sprach: Wie lange hinket ihr nach beiden Seiten? Ist
der \textsc{Herr} Gott, so folget ihm nach, ist es aber Baal, so folget
ihm! Und das Volk antwortete ihm nichts. \bibleverse{22} Da sprach Elia
zum Volk: Ich bin allein übriggeblieben als Prophet des \textsc{Herrn},
der Propheten Baals aber sind 450 Mann. \bibleverse{23} So gebt uns nun
zwei Farren und lasset sie den einen Farren erwählen und ihn zerstücken
und auf das Holz legen und kein Feuer daran legen; so will ich den
andern Farren zurichten und auf das Holz legen und auch kein Feuer daran
legen. \bibleverse{24} Dann rufet ihr den Namen eures Gottes an, und ich
will den Namen des \textsc{Herrn} anrufen. Welcher Gott mit Feuer
antworten wird, der sei Gott! Da antwortete das ganze Volk und sprach:
Das Wort ist gut! \bibleverse{25} Und Elia sprach zu den Propheten
Baals: Erwählet euch den einen Farren und bereitet ihn zuerst zu, denn
euer sind viele, und rufet den Namen eures Gottes an und leget kein
Feuer daran! \bibleverse{26} Und sie nahmen den Farren, den er ihnen
gab, und richteten ihn zu und riefen den Namen Baals an vom Morgen bis
zum Mittag und sprachen: O Baal, erhöre uns! Aber da war keine Stimme
noch Antwort. Und sie hüpften um den Altar, den man gemacht hatte.
\bibleverse{27} Als es nun Mittag war, spottete Elia ihrer und sprach:
Rufet laut! denn er ist ja ein Gott; vielleicht denkt er nach oder hat
zu schaffen oder ist auf Reisen oder schläft vielleicht und wird
aufwachen! \bibleverse{28} Und sie riefen laut und machten Einschnitte
nach ihrer Weise mit Schwertern und Spießen, bis das Blut über sie floß.
\bibleverse{29} Als aber der Mittag vergangen war, weissagten sie, bis
es Zeit war, das Speisopfer darzubringen; aber da war keine Stimme noch
Antwort noch Aufmerken. \bibleverse{30} Da sprach Elia zu allem Volk:
Tretet heran zu mir! Als nun alles Volk zu ihm trat, stellte er den
Altar des \textsc{Herrn}, der zerbrochen war, wieder her.
\bibleverse{31} Und Elia nahm zwölf Steine, nach der Zahl der Stämme der
Kinder Jakobs, an welchen das Wort des \textsc{Herrn} also ergangen war:
Du sollst Israel heißen! \bibleverse{32} Und er baute von den Steinen
einen Altar im Namen des \textsc{Herrn} und machte um den Altar her
einen Graben von der Tiefe eines Getreidedoppelmaßes; \bibleverse{33}
und er richtete das Holz zu und zerstückte den Farren und legte ihn auf
das Holz und sprach: \bibleverse{34} Füllet vier Krüge mit Wasser und
gießet es auf das Brandopfer und auf das Holz! Und er sprach: Tut es
noch einmal! Und sie taten es noch einmal. Und er sprach: Tut es zum
drittenmal! Und sie taten es zum drittenmal. \bibleverse{35} Und das
Wasser lief um den Altar her, und der Graben ward auch voll Wasser.
\bibleverse{36} Und um die Zeit, da man das Speisopfer darbringt, trat
der Prophet Elia herzu und sprach: O \textsc{Herr}, Gott Abrahams,
Isaaks und Israels, laß heute kund werden, daß du Gott in Israel bist
und ich dein Knecht und daß ich solches alles nach deinem Wort getan
habe! \bibleverse{37} Erhöre mich, o \textsc{Herr}, erhöre mich, daß
dieses Volk erkenne, daß du, \textsc{Herr}, Gott bist, und daß du ihr
Herz herumgewendet hast! \bibleverse{38} Da fiel das Feuer des
\textsc{Herrn} herab und fraß das Brandopfer und das Holz und die Steine
und die Erde; und es leckte das Wasser auf in dem Graben.
\bibleverse{39} Als alles Volk solches sah, fielen sie auf ihr Angesicht
und sprachen: Der \textsc{Herr} ist Gott! der \textsc{Herr} ist Gott!
\bibleverse{40} Elia aber sprach zu ihnen: Fanget die Propheten Baals,
daß ihrer keiner entrinne! Und sie fingen sie. Und Elia führte sie hinab
an den Bach Kison und schlachtete sie daselbst. \bibleverse{41} Und Elia
sprach zu Ahab: Ziehe hinauf, iß und trink, denn es rauscht, als wolle
es reichlich regnen! \bibleverse{42} Und als Ahab hinaufzog, um zu essen
und zu trinken, ging Elia auf die Spitze des Karmel und beugte sich zur
Erde und tat sein Angesicht zwischen seine Knie \bibleverse{43} und
sprach zu seinem Knaben: Gehe doch hinauf und siehe nach dem Meere hin!
Da ging er hinauf und schaute hin und sprach: Es ist nichts da! Er
sprach: Gehe wieder hin, siebenmal! \bibleverse{44} Und beim siebenten
Mal sprach er: Siehe, es steigt eine kleine Wolke aus dem Meere auf, wie
eines Mannes Hand. Er sprach: Gehe hin und sage zu Ahab: Spanne an und
fahre hinab, daß dich der Regen nicht zurückhalte! \bibleverse{45} Und
ehe man zusah, ward der Himmel schwarz von Wolken und Wind, und es kam
ein gewaltiger Regen. Ahab aber stieg auf und fuhr nach Jesreel.
\bibleverse{46} Und die Hand des \textsc{Herrn} kam über Elia; und er
gürtete seine Lenden und lief vor Ahab her bis gen Jesreel.

\hypertarget{section-18}{%
\section{19}\label{section-18}}

\bibleverse{1} Und Ahab sagte der Isebel alles, was Elia getan und wie
er alle Propheten Baals mit dem Schwerte umgebracht hatte.
\bibleverse{2} Da sandte Isebel einen Boten zu Elia und ließ ihm sagen:
Die Götter sollen mir dies und das tun, wenn ich morgen um diese Zeit
mit deinem Leben nicht also verfahre wie du mit jener Leben!
\bibleverse{3} Als er solches vernahm, machte er sich auf und ging fort
um seines Lebens willen und kam nach Beerseba in Juda und ließ seinen
Knaben daselbst. \bibleverse{4} Er aber ging hin in die Wüste, eine
Tagereise weit, kam und setzte sich unter einen Ginsterstrauch und erbat
sich den Tod und sprach: Es ist genug! So nimm nun, \textsc{Herr}, meine
Seele; denn ich bin nicht besser als meine Väter! \bibleverse{5} Und er
legte sich und schlief ein unter dem Ginsterstrauch. Und siehe, ein
Engel rührte ihn an und sprach zu ihm: Stehe auf und iß! \bibleverse{6}
Und als er sich umsah, siehe, da war zu seinen Häupten ein auf heißen
Steinen gebackener Brotkuchen und ein Krug Wasser. Und als er gegessen
und getrunken hatte, legte er sich wieder schlafen. \bibleverse{7} Und
der Engel des \textsc{Herrn} kam zum zweitenmal und rührte ihn an und
sprach: Stehe auf und iß; denn du hast einen weiten Weg vor dir!
\bibleverse{8} Und er stand auf, aß und trank und ging kraft dieser
Speise vierzig Tage und vierzig Nächte lang, bis an den Berg Gottes
Horeb. \bibleverse{9} Und er ging daselbst in eine Höhle hinein und
blieb dort über Nacht. Und siehe, das Wort des \textsc{Herrn} kam zu ihm
und sprach: Was willst du hier, Elia? \bibleverse{10} Er sprach: Ich
habe heftig für den \textsc{Herrn}, den Gott der Heerscharen, geeifert;
denn die Kinder Israel haben deinen Bund verlassen und deine Altäre
zerbrochen und deine Propheten mit dem Schwert umgebracht, und ich bin
allein übriggeblieben, und sie trachten darnach, mir das Leben zu
nehmen! \bibleverse{11} Er aber sprach: Komm heraus und tritt auf den
Berg vor den \textsc{Herrn}! Und siehe, der \textsc{Herr} ging vorüber;
und ein großer, starker Wind, der die Berge zerriß und die Felsen
zerbrach, ging vor dem \textsc{Herrn} her; der \textsc{Herr} aber war
nicht im Winde. Nach dem Winde aber kam ein Erdbeben; aber der Herr war
nicht im Erdbeben. \bibleverse{12} Und nach dem Erdbeben kam ein Feuer;
aber der \textsc{Herr} war nicht im Feuer. Und nach dem Feuer kam die
Stimme eines sanften Säuselns. \bibleverse{13} Als Elia dieses hörte,
verhüllte er sein Angesicht mit seinem Mantel und ging hinaus und trat
an den Eingang der Höhle. Und siehe, da kam eine Stimme zu ihm, die
sprach: Was willst du hier, Elia? \bibleverse{14} Er sprach: Ich habe
heftig für den \textsc{Herrn}, den Gott der Heerscharen, geeifert; denn
die Kinder Israel haben deinen Bund verlassen, deine Altäre zerbrochen
und deine Propheten mit dem Schwerte umgebracht, und ich bin allein
übriggeblieben, und sie trachten darnach, mir das Leben zu nehmen!
\bibleverse{15} Aber der \textsc{Herr} sprach zu ihm: Kehre wieder auf
deinen Weg zurück nach der Wüste und wandere gen Damaskus und gehe
hinein und salbe Hasael zum König über Syrien. \bibleverse{16} Auch
sollst du Jehu, den Sohn Nimsis, zum König über Israel salben und sollst
Elisa, den Sohn Saphats, von Abel-Mechola, zum Propheten salben an
deiner Statt. \bibleverse{17} Und es soll geschehen, wer dem Schwerte
Hasaels entrinnt, den soll Jehu töten; und wer dem Schwerte Jehus
entrinnt, den soll Elisa töten. \bibleverse{18} Ich aber will in Israel
siebentausend übriglassen, nämlich alle, die ihre Knie nicht gebeugt
haben vor Baal und deren Mund ihn nicht geküßt hat. \bibleverse{19} Und
er ging von dannen und fand Elisa, den Sohn Saphats; der pflügte mit
zwölf Joch Rindern vor sich her, und er selbst war beim zwölften. Und
Elia ging zu ihm und warf seinen Mantel über ihn. \bibleverse{20} Er
aber verließ die Rinder und lief Elia nach und sprach: Laß mich noch
meinen Vater und meine Mutter küssen, dann will ich dir nachfolgen! Er
aber sprach zu ihm: Gehe hin und komm wieder! \bibleverse{21} Denn was
habe ich dir getan? Da wandte er sich von ihm und nahm ein Joch Rinder
und opferte sie und kochte das Fleisch mit dem Geschirr der Rinder und
gab es dem Volk, daß sie aßen; dann machte er sich auf und folgte Elia
nach und diente ihm.

\hypertarget{section-19}{%
\section{20}\label{section-19}}

\bibleverse{1} Benhadad aber, der König von Syrien, versammelte seine
ganze Macht, und zweiunddreißig Könige waren mit ihm und Pferde und
Wagen; und er zog herauf und belagerte Samaria und bestürmte es.
\bibleverse{2} Und er sandte Boten in die Stadt zu Ahab, dem König von
Israel, und ließ ihm sagen: \bibleverse{3} So spricht Benhadad: Dein
Silber und dein Gold ist mein, und deine schönen Frauen und Kinder sind
auch mein! \bibleverse{4} Der König von Israel antwortete und sprach:
Mein Herr und König, wie du gesagt hast: ich bin dein und alles, was ich
habe! \bibleverse{5} Und die Boten kamen wieder und sprachen: So spricht
Benhadad: Weil ich zu dir gesandt und dir habe sagen lassen: Du sollst
mir dein Silber und dein Gold und deine Weiber und deine Söhne geben,
\bibleverse{6} so will ich morgen um diese Zeit meine Knechte zu dir
senden, daß sie dein Haus und deiner Knechte Häuser durchsuchen; und was
in deinen Augen lieblich ist, sollen sie zuhanden nehmen und forttragen.
\bibleverse{7} Da berief der König von Israel alle Ältesten des Landes
und sprach: Merket doch und sehet, daß dieser Böses vorhat! Denn er hat
zu mir gesandt, um meine Weiber und meine Söhne, mein Silber und mein
Gold zu fordern, und ich habe ihm dieses nicht verweigert.
\bibleverse{8} Da sprachen alle Ältesten und alles Volk zu ihm: Du
sollst nicht darauf hören und nicht einwilligen! \bibleverse{9} Und er
sprach zu den Boten Benhadads: Saget meinem Herrn, dem König: Alles, was
du deinem Knecht zuerst entboten hast, habe ich tun wollen, aber dieses
kann ich nicht tun! Und die Boten gingen hin und meldeten solches.
\bibleverse{10} Da sandte Benhadad zu ihm und ließ ihm sagen: Die Götter
sollen mir dies und das tun, wenn der Staub Samarias hinreicht, daß
jeder vom Volk, das ich anführe, nur eine Handvoll davon nehme!
\bibleverse{11} Aber der König von Israel antwortete und sprach: Saget:
Wer das Schwert umgürtet, soll sich nicht rühmen wie der, der es
abgürtet! \bibleverse{12} Als Benhadad solches hörte und er gerade mit
den Königen in den Zelten trank, sprach er zu seinen Knechten: Stellet
euch zum Angriff! Da stellen sie sich zum Angriff gegen die Stadt.
\bibleverse{13} Aber siehe, ein Prophet trat zu Ahab, dem König von
Israel, und sprach: So spricht der \textsc{Herr}: Hast du diesen ganzen
großen Haufen gesehen? Siehe, ich will ihn heute in deine Hand geben,
daß du erfahren sollst, daß ich der \textsc{Herr} bin! \bibleverse{14}
Ahab sprach: Durch wen? Er sprach: So spricht der \textsc{Herr}: Durch
die Knappen der Bezirkshauptleute! Er sprach: Wer soll den Kampf
beginnen? Er sprach: Du! \bibleverse{15} Da musterte er die Knappen der
Bezirkshauptleute, und es waren ihrer 232; und nach ihnen musterte er
das ganze Volk, alle Kinder Israel, 7000 Mann. \bibleverse{16} Und sie
zogen aus am Mittag. Benhadad aber zechte und betrank sich in den Zelten
samt den zweiunddreißig Königen, die ihm zu Hilfe gekommen waren.
\bibleverse{17} Aber die Knappen der Bezirkshauptleute zogen zuerst aus.
Und Benhadads Kundschafter meldeten ihm: Es kommen Männer aus Samaria!
\bibleverse{18} Er sprach: Fanget sie lebendig, sie seien zum Frieden
oder zum Streit ausgezogen! \bibleverse{19} Jene aber zogen zur Stadt
hinaus, nämlich die Knappen der Bezirkshauptleute und das Heer hinter
ihnen her, \bibleverse{20} und ein jeder schlug seinen Mann, so daß die
Syrer flohen und Israel ihnen nachjagte. Benhadad aber, der König von
Syrien, entrann auf einem Pferd mit den Reitern. \bibleverse{21} Und der
König von Israel zog aus und schlug Pferde und Wagen und brachte den
Syrern eine große Niederlage bei. \bibleverse{22} Da trat der Prophet
zum König von Israel und sprach zu ihm: Gehe hin, stärke dich und merke
und siehe zu, was du zu tun hast; denn der König von Syrien wird wider
dich heraufziehen, wenn das Jahr vorbei ist! \bibleverse{23} Aber die
Knechte des Königs von Syrien sprachen zu ihm: Ihre Götter sind
Berggötter, darum haben sie uns überwunden. O daß wir mit ihnen auf der
Ebene streiten könnten; gewiß würden wir sie überwinden! \bibleverse{24}
Darum tue also: Setze die Könige ab von ihren Posten, und ernenne
Statthalter an ihrer Stelle! \bibleverse{25} Du aber verschaffe dir ein
Heer wie das Heer, das du verloren hast, und Pferde und Wagen, wie jene
waren, und laß uns in der Ebene wider sie streiten, so werden wir sie
gewiß überwinden! Und er gehorchte ihrer Stimme und tat also.
\bibleverse{26} Als nun das Jahr vorbei war, musterte Benhadad die Syrer
und zog herauf gen Aphek, um wider Israel zu streiten. \bibleverse{27}
Und die Kinder Israel wurden auch gemustert und mit Lebensmitteln
versehen und zogen ihnen entgegen; und die Kinder Israel lagerten sich
ihnen gegenüber wie zwei kleine Herden Ziegen; von den Syrern aber war
das Land voll. \bibleverse{28} Und der Mann Gottes trat herzu und sprach
zum König von Israel: So spricht der \textsc{Herr}: Weil die Syrer
gesagt haben, der \textsc{Herr} sei ein Gott der Berge und nicht ein
Gott der Talgründe, so habe ich diese ganze große Menge in deine Hand
gegeben, damit ihr erfahret, daß ich der \textsc{Herr} bin!
\bibleverse{29} Und sie lagerten sieben Tage lang einander gegenüber.
Aber am siebenten Tag kam es zur Schlacht, und die Kinder Israel
erschlugen von den Syrern an einem Tage 100000 Mann Fußvolk.
\bibleverse{30} Und die Übriggebliebenen flohen gen Aphek in die Stadt,
und die Mauer fiel auf die übrigen 27000 Mann. Und Benhadad floh auch
und kam in die Stadt, von einem Gemach in das andere. \bibleverse{31} Da
sprachen seine Knechte zu ihm: Siehe doch, wir haben gehört, daß die
Könige des Hauses Israel barmherzige Könige seien; so laßt uns nun Säcke
um unsre Lenden tun und Stricke um unser Haupt und zum König von Israel
hinausgehen; vielleicht läßt er deine Seele leben! \bibleverse{32} Und
sie gürteten Säcke um ihre Lenden und legten Stricke um ihre Häupter und
kamen zum König von Israel und sprachen: Benhadad, dein Knecht, läßt dir
sagen: Laß doch meine Seele leben! Er aber sprach: Lebt er noch? Er ist
mein Bruder! \bibleverse{33} Und die Männer hielten das für günstig und
griffen es eilends von ihm auf und sprachen: Benhadad ist dein Bruder!
Er sprach: kommt und bringt ihn! Da ging Benhadad zu ihm hinaus, und er
ließ ihn auf den Wagen steigen. \bibleverse{34} Und Benhadad sprach: Die
Städte, die mein Vater deinem Vater genommen hat, will ich dir
wiedergeben; und mache dir Freiplätze zu Damaskus, wie mein Vater zu
Samaria getan hat! Ich aber , antwortete Ahab, lasse dich unter diesen
Bedingungen frei! Und er machte einen Bund mit ihm und ließ ihn frei.
\bibleverse{35} Da sprach ein Mann unter den Prophetensöhnen zu seinem
Nächsten durch das Wort des \textsc{Herrn}: Schlage mich doch! Der Mann
aber weigerte sich, ihn zu schlagen. \bibleverse{36} Da sprach er zu
ihm: Weil du der Stimme des \textsc{Herrn} nicht gehorcht hast, siehe,
so wird dich ein Löwe töten, wenn du von mir gehst! Und als er von ihm
ging, fand ihn ein Löwe und tötete ihn. \bibleverse{37} Und er fand
einen andern Mann und sprach: Schlage mich doch! Und der Mann schlug ihn
wund. \bibleverse{38} Da ging der Prophet hin und trat zum König an den
Weg und machte sich unkenntlich durch eine Binde über seinen Augen.
\bibleverse{39} Und als der König vorbeiging, rief er den König an und
sprach: Dein Knecht war in den Streit gezogen und siehe, ein fremder
Mann trat herzu und brachte einen Mann zu mir und sprach: Bewache diesen
Mann! Wird man ihn vermissen, so soll dein Leben für sein Leben haften,
oder du sollst ein Talent Silber darwägen! \bibleverse{40} Und während
dein Knecht hier und dort zu tun hatte, da war er verschwunden! Der
König von Israel sprach zu ihm: Also ist dein Urteil, du hast es selbst
gefällt. \bibleverse{41} Da tat er eilends die Binde weg von seinen
Augen. Und der König von Israel erkannte, daß er einer von den Propheten
war. \bibleverse{42} Er aber sprach zu ihm: So spricht der
\textsc{Herr}: Weil du den von mir mit dem Bann belegten Mann deiner
Hand entrinnen ließest, soll dein Leben für sein Leben und dein Volk für
sein Volk haften! \bibleverse{43} Also ging der König von Israel
mißmutig und zornig nach Hause und kam nach Samaria.

\hypertarget{section-20}{%
\section{21}\label{section-20}}

\bibleverse{1} Nach diesen Geschichten begab sich folgendes: Nabot, der
Jesreelit, hatte einen Weinberg zu Jesreel beim Palast Ahabs, des Königs
von Samaria. \bibleverse{2} Und Ahab redete mit Nabot und sprach: Gib
mir deinen Weinberg, ich will einen Gemüsegarten daraus machen, weil er
so nahe an meinem Hause liegt, und ich will dir einen bessern Weinberg
dafür geben; oder, wenn es dir gefällt, will ich dir Geld dafür geben,
so viel er gilt. \bibleverse{3} Aber Nabot sprach zu Ahab: Das lasse der
\textsc{Herr} ferne von mir sein, daß ich dir das Erbe meiner Väter
geben sollte! \bibleverse{4} Da kam Ahab heim, mißmutig und zornig, um
des Wortes willen, das Nabot, der Jesreelit, zu ihm gesprochen hatte:
Ich will dir das Erbe meiner Väter nicht geben! Und er legte sich auf
sein Bett, wandte sein Angesicht ab und aß nichts. \bibleverse{5} Da kam
sein Weib Isebel zu ihm hinein und redete mit ihm: Warum bist du so
mißmutig und issest nichts? \bibleverse{6} Er sprach zu ihr: Ich habe
mit Nabot, dem Jesreeliten, geredet und zu ihm gesagt: Gib mir deinen
Weinberg um Geld, oder, wenn es dir lieber ist, will ich dir einen
andern dafür geben. Er aber sprach: Ich will dir meinen Weinberg nicht
geben! \bibleverse{7} Da sprach sein Weib Isebel zu ihm: Erzeige dich
jetzt als König über Israel! Stehe auf und iß etwas und sei guten Muts!
Ich will dir den Weinberg Nabots, des Jesreeliten, verschaffen!
\bibleverse{8} Und sie schrieb Briefe in Ahabs Namen und versiegelte sie
mit seinem Siegel und sandte sie an die Ältesten und Obersten, die mit
Nabot zusammen in der Stadt wohnten; \bibleverse{9} und sie schrieb in
den Briefen also: Ruft ein Fasten aus und setzet Nabot oben an unter dem
Volk; \bibleverse{10} und stellt ihm gegenüber zwei Männer auf,
nichtswürdige Leute, welche wider ihn zeugen und sagen sollen: ``Du hast
Gott und dem König geflucht!'' Und führt ihm hinaus und steinigt ihn,
daß er sterbe! \bibleverse{11} Und die Männer seiner Stadt, die Ältesten
und Vornehmsten, die in seiner Stadt wohnten, taten, wie Isebel ihnen
aufgetragen hatte, wie in den Briefen geschrieben stand, die sie ihnen
zugesandt. \bibleverse{12} Sie ließen ein Fasten ausrufen und setzten
Nabot obenan unter dem Volk. \bibleverse{13} Da kamen die beiden Männer,
die nichtswürdigen Leute, und traten gegen ihn auf und zeugten wider
Nabot vor dem Volk und sprachen: Nabot hat Gott und dem König geflucht!
Da führten sie ihn vor die Stadt hinaus und steinigten ihn, daß er
starb. \bibleverse{14} Und sie sandten zu Isebel und ließen ihr sagen:
Nabot ist gesteinigt worden und ist tot! \bibleverse{15} Als aber Isebel
hörte, daß Nabot gesteinigt worden und tot sei, sprach Isebel zu Ahab:
Stehe auf und nimm den Weinberg Nabots, des Jesreeliten, in Besitz,
welchen er dir nicht um Geld geben wollte; denn Nabot lebt nicht mehr,
er ist tot! \bibleverse{16} Als nun Ahab hörte, daß Nabot tot sei, stand
er auf, um zum Weinberg Nabots, des Jesreeliten, hinabzugehen und ihn in
Besitz zu nehmen. \bibleverse{17} Aber das Wort des \textsc{Herrn}
erging an Elia, den Tisbiter, also: \bibleverse{18} Mache dich auf und
gehe hinab, Ahab, dem König von Israel, der zu Samaria ist, entgegen!
Siehe, er ist im Weinberg Nabots, dahin er gegangen, um ihn in Besitz zu
nehmen. \bibleverse{19} Du sollst aber zu ihm sagen: So spricht der
\textsc{Herr}: ``Hast du gemordet und geraubt?'' Und du sollst ferner
mit ihm reden und sagen: So spricht der \textsc{Herr}: An der Stelle, wo
die Hunde das Blut Nabots geleckt haben, sollen die Hunde auch dein Blut
lecken, ja, das deinige! \bibleverse{20} Und Ahab sprach zu Elia: Hast
du mich gefunden, du mein Feind? Er aber sprach: Ja, ich habe dich
gefunden, weil du dich verkauft hast, das zu tun, was böse ist vor dem
\textsc{Herrn}! \bibleverse{21} Siehe, ich will Unglück über dich
bringen und deine Nachkommen wegfegen und von Ahab ausrotten, was
männlich ist, Mündige und Unmündige in Israel; \bibleverse{22} und will
dein Haus machen wie das Haus Jerobeams, des Sohnes Nebats, und wie das
Haus Baesas, des Sohnes Achijas, um der Herausforderung willen, womit du
mich zum Zorn gereizt und Israel zur Sünde verführt hast!
\bibleverse{23} Und auch über Isebel redete der \textsc{Herr} und
sprach: Die Hunde sollen Isebel fressen auf dem Acker von Jesreel!
\bibleverse{24} Wer von Ahab in der Stadt stirbt, den sollen die Hunde
fressen, und wer auf dem Felde stirbt, den sollen die Vögel des Himmels
fressen! \bibleverse{25} Gar niemand war wie Ahab, der sich verkauft
hatte, Übles zu tun vor dem \textsc{Herrn}, wozu sein Weib Isebel ihn
überredete. \bibleverse{26} Und er verübte sehr viele Greuel, indem er
den Götzen nachwandelte, ganz wie die Amoriter getan, die der
\textsc{Herr} vor den Kindern Israel vertrieben hatte. \bibleverse{27}
Als aber Ahab diese Worte hörte, zerriß er seine Kleider und legte einen
Sack um seinen Leib und fastete und schlief im Sack und ging langsam
einher. \bibleverse{28} Da erging das Wort des \textsc{Herrn} an Elia,
den Tisbiter, und sprach: \bibleverse{29} Hast du nicht gesehen, wie
sich Ahab vor mir demütigt? Weil er sich nun vor mir demütigt, will ich
das Unglück nicht zu seinen Lebzeiten hereinbrechen lassen; erst bei
seines Sohnes Lebzeiten will ich das Unglück über sein Haus bringen.

\hypertarget{section-21}{%
\section{22}\label{section-21}}

\bibleverse{1} Und sie blieben drei Jahre lang ruhig, und es war kein
Krieg zwischen den Syrern und Israel. \bibleverse{2} Im dritten Jahre
aber zog Josaphat, der König von Juda, zum König von Israel hinab.
\bibleverse{3} Und der König von Israel sprach zu seinen Knechten:
Wisset ihr nicht, daß Ramot in Gilead uns gehört? Und wir sitzen still
und entreißen es nicht der Hand des Königs von Syrien? \bibleverse{4}
Und er sprach zu Josaphat: Willst du mit mir gen Ramot in Gilead in den
Krieg ziehen? Josaphat sprach zum König von Israel: Ich will sein wie
du, mein Volk wie dein Volk, meine Pferde wie deine Pferde!
\bibleverse{5} Josaphat sprach weiter zum König von Israel: Befrage doch
heute das Wort des \textsc{Herrn}! \bibleverse{6} Da versammelte der
König von Israel die Propheten, etwa vierhundert Mann, und sprach zu
ihnen: Soll ich gen Ramot in Gilead in den Streit ziehen, oder soll ich
es lassen? Sie sprachen: Ziehe hinauf, und der \textsc{Herr} wird sie in
des Königs Hand geben! \bibleverse{7} Josaphat aber sprach: Ist sonst
kein Prophet des \textsc{Herrn} hier, den wir fragen könnten?
\bibleverse{8} Der König von Israel sprach zu Josaphat: Es ist noch ein
Mann, durch den man den \textsc{Herrn} befragen kann; aber ich bin ihm
gram; denn er weissagt mir nichts Gutes, sondern eitel Böses: Michajah,
der Sohn Jimlas! Josaphat sprach: Der König rede nicht also!
\bibleverse{9} Da rief der König von Israel einen Kämmerer und sprach:
Bringe Michajah, den Sohn Jimlas, eilends her! \bibleverse{10} Der König
von Israel aber und Josaphat, der König von Juda, saßen ein jeder auf
seinem Thron, angetan mit königlichen Kleidern, auf dem Platze vor der
Tür, am Tor Samarias, und alle Propheten weissagten vor ihnen.
\bibleverse{11} Und Zedekia, der Sohn Kenaanas, hatte sich eiserne
Hörner gemacht und sprach: So spricht der \textsc{Herr}: Hiermit wirst
du die Syrer stoßen, bis du sie aufgerieben hast! \bibleverse{12} Und
alle Propheten weissagten ebenso und sprachen: Ziehe hinauf gen Ramot in
Gilead, und du wirst Gelingen haben, und der \textsc{Herr} wird es in
des Königs Hand geben! \bibleverse{13} Und der Bote, der hingegangen
war, Michajah zu rufen, sprach zu ihm also: Siehe doch, die Reden der
Propheten sind einstimmig gut für den König; so laß nun dein Wort auch
sein wie das Wort eines jeden von ihnen und rede Gutes! \bibleverse{14}
Michajah sprach: So wahr der \textsc{Herr} lebt, ich will reden, was mir
der \textsc{Herr} sagen wird! \bibleverse{15} Und als er zum König kam,
sprach der König zu ihm: Michajah, sollen wir gen Ramot in Gilead in den
Krieg ziehen, oder sollen wir es lassen? Er sprach: Ziehe hinauf! Es
soll dir gelingen, und der \textsc{Herr} wird es in des Königs Hand
geben! \bibleverse{16} Der König sprach abermal zu ihm: Wie oft muß ich
dich beschwören, daß du mir nichts anderes als die Wahrheit sagest im
Namen des \textsc{Herrn}? \bibleverse{17} Er sprach: ``Ich sah ganz
Israel auf den Bergen zerstreut wie Schafe, die keinen Hirten haben; und
der \textsc{Herr} sprach: Diese haben keinen Herrn! Ein jeder kehre
wieder heim in Frieden!'' \bibleverse{18} Da sprach der König von Israel
zu Josaphat: Habe ich dir nicht gesagt, daß er mir nichts Gutes
weissagt, sondern eitel Böses? \bibleverse{19} Er sprach: Darum höre das
Wort des \textsc{Herrn}! Ich sah den \textsc{Herrn} auf seinem Throne
sitzen und das ganze himmlische Heer neben ihm zu seiner Rechten und zu
seiner Linken stehen. \bibleverse{20} Und der \textsc{Herr} sprach: Wer
will Ahab überreden, daß er hinaufziehe und zu Ramot in Gilead falle?
Und einer sagte dies, der andere das. \bibleverse{21} Da ging ein Geist
aus und trat vor den \textsc{Herrn} und sprach: Ich will ihn überreden!
Der \textsc{Herr} sprach zu ihm: Womit? \bibleverse{22} Er sprach: Ich
will ausgehen und ein Lügengeist sein im Munde aller seiner Propheten!
Er sprach: Du sollst ihn überreden, und du wirst es auch vermögen! Gehe
aus und tue also! \bibleverse{23} Und nun siehe, der \textsc{Herr} hat
in den Mund aller dieser deiner Propheten einen Lügengeist gelegt, und
der \textsc{Herr} hat Unglück über dich beschlossen! \bibleverse{24} Da
trat Zedekia, der Sohn Kenaanas, herzu und schlug Michajah auf den
Backen und sprach: Ist etwa der Geist des \textsc{Herrn} von mir
gewichen, um mit dir zu reden? \bibleverse{25} Michajah sprach: Siehe,
du wirst es sehen an dem Tage, da du von einer Kammer in die andere
gehen wirst, um dich zu verbergen! \bibleverse{26} Der König sprach:
Nimm Michajah und führe ihn wieder zu Amon, dem Obersten der Stadt, und
zu Joas, dem Sohne des Königs, und sprich: \bibleverse{27} So spricht
der König: Leget diesen in den Kerker und speiset ihn mit Brot der
Trübsal, bis ich in Frieden wiederkomme! \bibleverse{28} Michajah
sprach: Kommst du in Frieden wieder, so hat der \textsc{Herr} nicht
durch mich geredet! Und er sprach: Hört es, ihr Völker alle!
\bibleverse{29} Da zogen der König von Israel und Josaphat, der König
von Juda, hinauf gen Ramot in Gilead. \bibleverse{30} Und der König von
Israel sprach zu Josaphat: Ich will verkleidet in den Streit ziehen; du
aber ziehe deine Kleider an! Also verkleidete sich der König von Israel
und zog in den Streit. \bibleverse{31} Aber der König von Syrien hatte
den Obersten über seine Wagen, deren zweiunddreißig waren, geboten und
gesagt: Ihr sollt nicht streiten wider Kleine noch Große, sondern nur
wider den König von Israel! \bibleverse{32} Als nun die Obersten der
Wagen Josaphat sahen, sprachen sie: Gewiß ist dieser der König von
Israel! Und sie wandten sich zum Kampf gegen ihn; und Josaphat schrie.
\bibleverse{33} Als aber die Obersten der Wagen sahen, daß er nicht der
König von Israel sei, ließen sie von ihm ab. \bibleverse{34} Ein Mann
aber spannte den Bogen von ungefähr und traf den König von Israel
zwischen den Fugen des Panzers. Da sprach er zu seinem Wagenlenker:
Wende um und führe mich aus dem Heer; denn ich bin verwundet!
\bibleverse{35} Da aber gerade um diese Zeit der Streit zunahm, mußte
der König auf dem Wagen stehen bleiben, den Syrern gegenüber, und er
starb am Abend, und das Blut floß von der Wunde mitten in den Wagen.
\bibleverse{36} Und als die Sonne unterging, erscholl die Klage durch
das Lager: Jedermann gehe in seine Stadt und in sein Land; denn der
König ist tot! \bibleverse{37} Als sie nun nach Samaria kamen, begruben
sie den König zu Samaria. \bibleverse{38} Und als man den Wagen beim
Teiche Samarias wusch, leckten die Hunde sein Blut, und die Dirnen
wuschen sich damit, nach dem Worte des \textsc{Herrn}, das er geredet
hatte. \bibleverse{39} Was aber mehr von Ahab zu sagen ist, und alles,
was er getan hat, und das elfenbeinerne Haus, das er gebaut, und alle
Städte, die er gebaut hat, steht das nicht geschrieben in der Chronik
der Könige von Israel? \bibleverse{40} Also legte sich Ahab zu seinen
Vätern; und Ahasia, sein Sohn, ward König an seiner Statt.
\bibleverse{41} Josaphat aber, der Sohn Asas, war König über Juda
geworden im vierten Jahre Ahabs, des Königs von Israel. \bibleverse{42}
Und Josaphat war fünfunddreißig Jahre alt, als er König ward, und
regierte fünfundzwanzig Jahre lang zu Jerusalem. Und seine Mutter hieß
Asuba, eine Tochter Silhis. \bibleverse{43} Und er wandelte durchaus in
den Wegen seines Vaters Asa und wich nicht davon, indem er tat, was dem
\textsc{Herrn} wohlgefiel. Doch kamen die Höhen nicht weg; denn das Volk
opferte und räucherte noch auf den Höhen. \bibleverse{44} Und Josaphat
hatte Frieden mit dem König von Israel. \bibleverse{45} Was aber mehr
von Josaphat zu sagen ist, und seine Tapferkeit, die er bewiesen, und
wie er gestritten hat, ist das nicht aufgezeichnet in der Chronik der
Könige von Juda? \bibleverse{46} Er rottete auch aus dem Lande die noch
übrigen Schandbuben aus, die zur Zeit seines Vaters Asa übriggeblieben
waren. \bibleverse{47} Und es war damals kein König in Edom; ein
Statthalter regierte. \bibleverse{48} Und Josaphat hatte Tarsis-Schiffe
machen lassen, die nach Ophir fahren sollten, um Gold zu holen; aber sie
fuhren nicht, denn sie scheiterten zu Ezjon-Geber. \bibleverse{49}
Damals sprach Ahasia, der Sohn Ahabs, zu Josaphat: Laß meine Knechte mit
deinen Knechten auf den Schiffen fahren! Josaphat aber wollte nicht.
\bibleverse{50} Und Josaphat legte sich zu seinen Vätern und ward
begraben bei seinen Vätern in der Stadt Davids, seines Vaters, und
Joram, sein Sohn, ward König an seiner Statt. \bibleverse{51} Ahasia,
der Sohn Ahabs, ward König über Israel zu Samaria, im siebzehnten Jahre
Josaphats, des Königs von Juda, und regierte zwei Jahre lang über
Israel. \bibleverse{52} Er tat, was dem \textsc{Herrn} übel gefiel, und
wandelte auf dem Wege seines Vaters und seiner Mutter und auf dem Wege
Jerobeams, des Sohnes Nebats, der Israel zur Sünde verführt hatte.
\bibleverse{53} Und er diente dem Baal und betete ihn an und erzürnte
den \textsc{Herrn}, den Gott Israels, ganz wie sein Vater getan hatte.
