\hypertarget{section}{%
\section{1}\label{section}}

\bibleverse{1} Und er rief Mose, und der \textsc{Herr} redete zu ihm von
der Stiftshütte aus und sprach: \bibleverse{2} Rede zu den Kindern
Israel und sprich zu ihnen: Will jemand von euch dem \textsc{Herrn} ein
Opfer bringen, so soll euer Opfer, das ihr darbringet, vom Vieh, von
Rindern oder Schafen genommen sein. \bibleverse{3} Ist seine Gabe ein
Brandopfer von Rindern, so soll er ein tadelloses männliches Tier
darbringen; zur Tür der Stiftshütte soll er es bringen, daß es ihn
angenehm mache vor dem \textsc{Herrn}; \bibleverse{4} und er soll seine
Hand auf den Kopf des Brandopfers stützen, so wird es ihm wohlgefällig
aufgenommen und für ihn Sühne erwirken. \bibleverse{5} Dann soll er den
jungen Ochsen schächten vor dem \textsc{Herrn}; die Söhne Aarons aber,
die Priester, sollen das Blut darbringen und es ringsum an den Altar
sprengen, der vor der Tür der Stiftshütte steht. \bibleverse{6} Er aber
soll dem Brandopfer die Haut abziehen und es in Stücke zerlegen;
\bibleverse{7} und die Söhne Aarons, des Priesters, sollen Feuer auf den
Altar tun und Holz aufschichten über dem Feuer; \bibleverse{8} auf das
Holz aber über dem Feuer sollen die Priester, die Söhne Aarons, die
Fleischstücke legen, dazu den Kopf und das Fett; \bibleverse{9} sein
Eingeweide aber und seine Schenkel soll man mit Wasser waschen; und der
Priester soll das Ganze auf dem Altar verbrennen als ein Brandopfer, ein
wohlriechendes Feuer für den \textsc{Herrn}. \bibleverse{10} Entnimmt er
aber seine Opfergabe dem Kleinvieh, so bringe er zum Brandopfer ein
tadelloses männliches Tier von den Lämmern oder Ziegen dar
\bibleverse{11} und schächte es an der nördlichen Seite des Altars vor
dem \textsc{Herrn}, und die Söhne Aarons, die Priester, sollen das Blut
ringsum an den Altar sprengen; \bibleverse{12} er aber zerlege es in
Stücke, und der Priester lege sie samt dem Kopf und dem Fett ordentlich
auf das Holz über dem Feuer auf dem Altar. \bibleverse{13} Aber das
Eingeweide und die Schenkel soll man mit Wasser waschen; und der
Priester soll das Ganze darbringen und verbrennen auf dem Altar; es ist
ein Brandopfer, ein wohlriechendes Feuer für den \textsc{Herrn}.
\bibleverse{14} Soll aber seine Brandopfergabe für den \textsc{Herrn}
aus Geflügel bestehen, so bringe er von Turteltauben oder von jungen
Tauben sein Opfer dar. \bibleverse{15} Dieses soll der Priester zum
Altar bringen und ihm den Kopf abkneifen und ihn auf dem Altar
verbrennen; sein Blut aber soll an der Wand des Altars ausgepreßt
werden. \bibleverse{16} Den Kropf aber samt dem Unrat soll er entfernen
und ihn auf den Aschenhaufen werfen, östlich vom Altar. \bibleverse{17}
Sodann soll er den Vogel an den Flügeln einreißen, sie aber nicht
abtrennen, und der Priester soll ihn auf dem Altar verbrennen, auf dem
Holz über dem Feuer; es ist ein Brandopfer, ein wohlriechendes Feuer für
den \textsc{Herrn}.

\hypertarget{section-1}{%
\section{2}\label{section-1}}

\bibleverse{1} Will aber eine Seele dem \textsc{Herrn} ein Speisopfer
bringen, so soll ihre Opfergabe aus Semmelmehl sein, und man soll Öl
darüber gießen und Weihrauch darauf tun. \bibleverse{2} Also soll man
sie zu den Söhnen Aarons, zu den Priestern bringen, und er soll davon
eine Handvoll nehmen, von dem Semmelmehl und dem Öl, samt allem
Weihrauch, und der Priester soll dieses Gedächtnisopfer auf dem Altar
verbrennen als ein wohlriechendes Feuer für den \textsc{Herrn}.
\bibleverse{3} Das Übrige aber vom Speisopfer gehört Aaron und seinen
Söhnen, als ein hochheiliger Anteil an den Feueropfern des
\textsc{Herrn}. \bibleverse{4} Willst du aber ein Speisopfer darbringen
von dem, was im Ofen gebacken wird, so nimm ungesäuerte Semmelkuchen,
mit Öl gemengt, und ungesäuerte Fladen, mit Öl gesalbt. \bibleverse{5}
Ist aber dein Speisopfer in der Pfanne bereitet, so soll es von
ungesäuertem Semmelmehl sein, mit Öl gemengt; \bibleverse{6} du sollst
es in Brocken zerbrechen und Öl darauf gießen, so ist es ein Speisopfer.
\bibleverse{7} Willst du aber ein gekochtes Speisopfer darbringen, so
bereite man es von Semmelmehl mit Öl; \bibleverse{8} und du sollst das
Speisopfer, das von solchem bereitet ist, zum \textsc{Herrn} bringen und
sollst es dem Priester übergeben, der trage es zum Altar; \bibleverse{9}
und der Priester soll von dem Speisopfer abheben, was davon zum
Gedächtnis bestimmt ist, und soll es auf dem Altar verbrennen zu einem
wohlriechenden Feuer vor dem \textsc{Herrn}. \bibleverse{10} Das Übrige
aber vom Speisopfer gehört Aaron und seinen Söhnen, als hochheiliger
Anteil an den Feueropfern des \textsc{Herrn}. \bibleverse{11} Kein
Speisopfer, das ihr dem \textsc{Herrn} darbringet, soll gesäuert werden;
denn ihr sollt dem \textsc{Herrn} weder Sauerteig noch Honig verbrennen.
\bibleverse{12} Als eine Erstlingsgabe mögt ihr solches dem
\textsc{Herrn} darbringen; aber auf den Altar soll es nicht kommen zum
lieblichen Geruch. \bibleverse{13} Dagegen sollst du alle deine
Speisopfergaben mit Salz würzen und sollst das Bundessalz deines Gottes
nicht fehlen lassen in deinem Speisopfer; sondern zu allen deinen
Opfergaben sollst du Salz darbringen. \bibleverse{14} Willst du aber dem
\textsc{Herrn}, deinem Gott, ein Erstlingsopfer darbringen, so sollst du
am Feuer geröstete Ähren, geschrotete Körner als Erstlingsspeisopfer
bringen; \bibleverse{15} und sollst Öl darauf tun und Weihrauch darauf
legen, so ist es ein Speisopfer. \bibleverse{16} Und der Priester soll,
was davon zum Gedächtnis bestimmt ist, verbrennen, von der Grütze und
vom Öl, dazu allen Weihrauch, daß es ein Feuer sei für den
\textsc{Herrn}.

\hypertarget{section-2}{%
\section{3}\label{section-2}}

\bibleverse{1} Ist aber seine Gabe ein Dankopfer, und bringt er es von
den Rindern dar, es sei ein Ochs oder eine Kuh, so soll er ein
tadelloses Tier herbringen vor den \textsc{Herrn}. \bibleverse{2} Und er
soll seine Hand stützen auf seines Opfers Haupt und es schächten vor der
Tür der Stiftshütte, und Aarons Söhne, die Priester, sollen sein Blut
ringsum an den Altar sprengen. \bibleverse{3} Dann soll er von dem
Dankopfer zur Verbrennung für den \textsc{Herrn} das Fett herzubringen,
welches das Eingeweide bedeckt, auch alles Fett, das am Eingeweide
hängt; \bibleverse{4} dazu die beiden Nieren samt dem Fett daran, das an
den Lenden ist, und was über die Leber hervorragt; oberhalb der Nieren
soll er es wegnehmen. \bibleverse{5} Und Aarons Söhne sollen es
verbrennen auf dem Altar, über dem Brandopfer, auf dem Holz, das über
dem Feuer liegt, als ein wohlriechendes Feuer für den \textsc{Herrn}.
\bibleverse{6} Besteht aber seine Gabe, die er dem \textsc{Herrn} zum
Dankopfer darbringt, in Kleinvieh, es sei ein Männchen oder Weibchen, so
soll es tadellos sein. \bibleverse{7} Bringt er ein Lamm zum Opfer dar,
so bringe er es vor den \textsc{Herrn} \bibleverse{8} und stütze seine
Hand auf des Opfers Haupt und schächte es vor der Stiftshütte; die Söhne
Aarons aber sollen das Blut ringsum an den Altar sprengen.
\bibleverse{9} Darnach bringe er von dem Dankopfer das Fett dem
\textsc{Herrn} zur Verbrennung dar, dazu das Fett, welches das
Eingeweide bedeckt, samt allem Fett an den Eingeweiden; \bibleverse{10}
auch die beiden Nieren mit dem Fett daran, das an den Lenden ist, samt
dem, was über die Leber hervorragt; oberhalb der Nieren soll er es
wegnehmen; \bibleverse{11} und der Priester soll es auf dem Altar
verbrennen als Nahrung für das Feuer des \textsc{Herrn}. \bibleverse{12}
Besteht aber sein Opfer in einer Ziege, so bringe er sie vor den
\textsc{Herrn} \bibleverse{13} und stütze seine Hand auf ihr Haupt und
schächte sie vor der Stiftshütte; die Söhne Aarons aber sollen das Blut
ringsum an den Altar sprengen. \bibleverse{14} Darnach bringe er sein
Opfer dar zur Verbrennung für den \textsc{Herrn}, nämlich das Fett,
welches das Eingeweide bedeckt, samt allem Fett, das am Eingeweide
hängt; \bibleverse{15} dazu die beiden Nieren mit dem Fett daran, das an
den Lenden ist, samt dem, was über die Leber hervorragt; oberhalb der
Nieren soll er es wegnehmen. \bibleverse{16} Das soll der Priester auf
dem Altar verbrennen als Nahrung für das Feuer, zum lieblichen Geruch.
Alles Fett gehört dem \textsc{Herrn}. \bibleverse{17} Das ist eine ewige
Satzung für eure Geschlechter an allen euren Wohnorten, daß ihr weder
Fett noch Blut essen sollt.

\hypertarget{section-3}{%
\section{4}\label{section-3}}

\bibleverse{1} Und der \textsc{Herr} redete mit Mose und sprach:
\bibleverse{2} Sage zu den Kindern Israel und sprich: Wenn sich eine
Seele aus Versehen versündigt gegen irgend eines der Gebote des
\textsc{Herrn}, also daß sie etwas von dem tut, was man nicht tun soll,
so gelte folgende Vorschrift: \bibleverse{3} Wenn der gesalbte Priester
sündigt, sodaß er sich am Volk verschuldet, so soll er für seine Sünde,
die er begangen hat, einen jungen, tadellosen Farren dem \textsc{Herrn}
zum Sündopfer darbringen; \bibleverse{4} und zwar soll er den Farren zur
Tür der Stiftshütte bringen, vor den \textsc{Herrn}, und seine Hand
stützen auf des Farren Haupt und ihn schächten vor dem \textsc{Herrn}.
\bibleverse{5} Und der gesalbte Priester soll von dem Blut des Farren
nehmen und es in die Stiftshütte bringen; \bibleverse{6} und der
Priester soll seinen Finger in das Blut tauchen und von dem Blut
siebenmal an die Vorderseite des Vorhangs im Heiligtum sprengen, vor dem
Angesicht des \textsc{Herrn}. \bibleverse{7} Auch soll der Priester von
dem Blut auf die Hörner des wohlriechenden Räucheraltars tun, der in der
Stitshütte ist vor dem \textsc{Herrn}; alles übrige Blut des Farren aber
soll er an den Fuß des Brandopferaltars gießen, der vor der Tür der
Stiftshütte ist. \bibleverse{8} Und alles Fett des Sündopferfarren soll
er von ihm ablösen, das Fett, welches das Eingeweide bedeckt, und alles
Fett, das am Eingeweide hängt; \bibleverse{9} dazu die beiden Nieren,
samt dem Fett daran, das an den Lenden ist, auch das, was über die Leber
hervorragt; oberhalb der Nieren soll er es wegnehmen, gleich wie man es
von dem Stier des Dankopfers abhebt; \bibleverse{10} und der Priester
soll es auf dem Brandopferaltar verbrennen. \bibleverse{11} Aber das
Fell des Farren und all sein Fleisch samt seinem Kopf, seinen Schenkeln,
seinen Eingeweiden und seinem Mist, \bibleverse{12} kurz den ganzen
Farren soll man hinaus vor das Lager führen, an einen reinen Ort, wohin
man die Asche zu schütten pflegt, und ihn bei einem Holzfeuer
verbrennen; am Aschenplatz soll er verbrannt werden. \bibleverse{13}
Wenn sich aber die ganze Gemeinde Israel vergeht, und es ist vor den
Augen der Versammlung verborgen, daß sie etwas getan, davon der
\textsc{Herr} geboten hat, daß man es nicht tun soll, sodaß sie sich
verschuldet hat; \bibleverse{14} sie kommt aber zur Erkenntnis der
Sünde, die sie begangen hat wider dasselbe Gebot, so soll die
Versammlung einen jungen Farren darbringen und ihn vor die Stiftshütte
führen. \bibleverse{15} Dann sollen die Ältesten der Gemeinde ihre Hände
auf des Farren Haupt stützen vor dem \textsc{Herrn}, und man soll den
Farren schächten vor dem \textsc{Herrn}. \bibleverse{16} Der gesalbte
Priester aber soll von dem Blut des Farren in die Stiftshütte bringen,
\bibleverse{17} und der Priester soll seinen Finger in das Blut tauchen
und davon siebenmal an die Vorderseite des Vorhangs sprengen vor dem
\textsc{Herrn}; \bibleverse{18} und er soll von dem Blut auf die Hörner
des Altars tun, der vor dem \textsc{Herrn} in der Stiftshütte steht;
alles übrige Blut aber soll er an den Fuß des Brandopferaltars gießen,
der vor der Tür der Stiftshütte steht. \bibleverse{19} Aber all sein
Fett soll er von ihm ablösen und es auf dem Altar verbrennen.
\bibleverse{20} Er soll diesen Farren behandeln, wie er den Farren des
Sündopfers behandelt hat; ganz gleich soll auch dieser behandelt werden,
und der Priester soll sie versühnen, und es soll ihr vergeben werden.
\bibleverse{21} Und man soll den Farren hinaus vor das Lager schaffen
und ihn verbrennen, wie man den ersten Farren verbrannt hat. Er ist ein
Sündopfer der Gemeinde. \bibleverse{22} Kommt es aber vor, daß ein Fürst
sündigt und aus Versehen irgend etwas tut, wovon der \textsc{Herr}, sein
Gott, geboten, daß man es nicht tun soll, und sich also verschuldet,
\bibleverse{23} man hält ihm aber seine Sünde vor, die er daran begangen
hat, so soll er einen tadellosen Ziegenbock zum Opfer bringen
\bibleverse{24} und soll seine Hand auf des Bockes Haupt stützen und ihn
schächten an dem Ort, da man das Brandopfer zu schächten pflegt vor dem
\textsc{Herrn}; es ist ein Sündopfer. \bibleverse{25} Und der Priester
soll mit seinem Finger von dem Blut des Sündopfers nehmen und es auf die
Hörner des Brandopferaltars tun; das übrige Blut aber soll er an den Fuß
des Brandopferaltars gießen; \bibleverse{26} und all sein Fett soll er
auf dem Altar verbrennen, gleich dem Fett des Dankopfers. Also soll der
Priester ihm Sühne erwirken für seine Sünde, und es wird ihm vergeben
werden. \bibleverse{27} Wenn aber jemand vom Landvolk aus Versehen
sündigt, indem er etwas tut, davon der \textsc{Herr} geboten hat, daß
man es nicht tun soll, und sich verschuldet, \bibleverse{28} man hält
ihm aber seine Sünde vor, die er begangen hat, so soll er eine tadellose
Ziege zum Opfer bringen für seine Sünde, die er begangen hat,
\bibleverse{29} und er soll seine Hand auf des Sündopfers Haupt stützen
und das Sündopfer schächten an der Stätte des Brandopfers.
\bibleverse{30} Der Priester aber soll mit seinem Finger von dem Blut
der Ziege nehmen und es auf die Hörner des Brandopferaltars tun und
alles übrige Blut an den Fuß des Brandopferaltars gießen.
\bibleverse{31} Alles Fett aber soll er von ihr nehmen, wie das Fett von
dem Dankopfer genommen wird, und der Priester soll es auf dem Altar
verbrennen zum lieblichen Geruch dem \textsc{Herrn}. Also soll der
Priester für ihn Sühne erwirken, und es wird ihm vergeben werden.
\bibleverse{32} Will er aber ein Lamm darbringen zum Sündopfer, so soll
es ein tadelloses Weibchen sein; \bibleverse{33} und er soll seine Hand
stützen auf des Sündopfers Haupt und es schächten als Sündopfer an dem
Ort, wo man das Brandopfer zu schächten pflegt. \bibleverse{34} Und der
Priester soll mit seinem Finger von dem Blut des Sündopfers nehmen und
es auf die Hörner des Brandopferaltars tun, alles übrige Blut aber an
den Fuß des Altars gießen. \bibleverse{35} Und er soll alles Fett davon
nehmen, wie das Fett von dem Lamm des Dankopfers genommen wird, und der
Priester soll es auf dem Altar verbrennen, über den Feuerflammen des
\textsc{Herrn}, und also soll er ihm Sühne erwirken wegen seiner Sünde,
die er begangen hat, so wird ihm vergeben werden.

\hypertarget{section-4}{%
\section{5}\label{section-4}}

\bibleverse{1} Wenn eine Seele dadurch sündigt, daß sie etwas nicht
anzeigt, wiewohl sie die Beschwörung vernommen hat und Zeuge ist, daß
sie es entweder gesehen oder erfahren hat, so daß sie nun ihre Missetat
trägt; \bibleverse{2} oder wenn eine Seele irgend etwas Unreines
anrührt, sei es das Aas eines unreinen Wildes oder das Aas eines
unreinen Viehs oder das Aas eines unreinen Reptils, und es ist ihr
verborgen gewesen, sie fühlt sich nun aber unrein und schuldig;
\bibleverse{3} oder wenn jemand menschliche Unreinheit anrührt, irgend
etwas von alledem, womit man sich verunreinigen kann, und es ist ihm
verborgen gewesen, er hat es aber nun erkannt und fühlt sich schuldig;
\bibleverse{4} oder wenn eine Seele leichtfertig mit ihren Lippen
schwört, Gutes oder Böses tun zu wollen, irgend etwas von dem, was so
ein Mensch leichtfertig schwören mag, und es war ihm verborgen, er
erkennt es aber nun und fühlt sich einer dieser Sachen schuldig
\bibleverse{5} ist er nun wirklich in einem dieser Punkte schuldig, so
bekenne er, woran er sich versündigt hat, \bibleverse{6} und bringe dem
\textsc{Herrn} als Schuldopfer für seine Sünde, die er begangen hat, ein
Weibchen vom Kleinvieh, ein Lamm oder eine Ziege zum Sündopfer, und der
Priester soll ihm damit Sühne erwirken wegen seiner Sünde.
\bibleverse{7} Kann er aber nicht soviel zusammenbringen, daß es zu
einem Schäflein langt, so bringe er dem \textsc{Herrn} zu seinem
Schuldopfer, das er schuldig ist, zwei Turteltauben oder zwei junge
Tauben, eine zum Schuldopfer, die andere zum Brandopfer. \bibleverse{8}
Er soll sie zum Priester bringen; dieser aber soll zuerst die zum
Sündopfer bestimmte darbringen und ihr unterhalb des Genicks den Kopf
abkneifen, ihn aber nicht abtrennen. \bibleverse{9} Und vom Blut des
Sündopfers sprenge er an die Wand des Altars, das übrige Blut aber soll
an den Fuß des Altars ausgepreßt werden, weil es ein Sündopfer ist.
\bibleverse{10} Aus der andern aber soll er ein Brandopfer machen, wie
es verordnet ist. Also soll der Priester für ihn Sühne erwirken wegen
seiner Sünde, die er begangen hat, so wird ihm vergeben werden.
\bibleverse{11} Vermag er aber auch die zwei Turteltauben oder die zwei
jungen Tauben nicht, so bringe er zu seinem Opfer, das er schuldig ist,
ein Zehntel Epha Semmelmehl als Sündopfer. Er soll aber weder Öl daran
tun, noch Weihrauch darauf legen, weil es ein Sündopfer ist.
\bibleverse{12} Er soll es zum Priester bringen, und der Priester nehme
eine Handvoll davon, soviel als zum Gedächtnis bestimmt ist, und
verbrenne es auf dem Altar über dem Feuer des \textsc{Herrn}. Es ist ein
Sündopfer. \bibleverse{13} Also soll ihm der Priester Sühne erwirken
wegen seiner Sünde, die er begangen hat in einem jener Fälle, so wird
ihm vergeben werden. Das Opfer aber soll dem Priester gehören wie das
Speisopfer. \bibleverse{14} Und der \textsc{Herr} redete zu Mose und
sprach: \bibleverse{15} Wenn sich eine Seele aus Versehen vergreift und
versündigt an heiligen Dingen des \textsc{Herrn}, so soll sie dem
\textsc{Herrn} ihr Schuldopfer bringen, nämlich einen tadellosen Widder
von der Herde, nach deiner Schätzung im Wert von zwei Schekeln, nach dem
Schekel des Heiligtums, zum Schuldopfer. \bibleverse{16} Den Schaden
aber, den er dem Heiligtum zugefügt hat, soll er vergüten und einen
Fünftel dazufügen und es dem Priester geben; der soll für ihn Sühne
erwirken mit dem Widder des Schuldopfers, so wird ihm vergeben werden.
\bibleverse{17} Und wenn eine Seele sündigt und irgend etwas von alledem
tut, was der \textsc{Herr} verboten hat und man nicht tun soll, hat es
aber nicht gewußt und fühlt sich nun schuldig und trägt ihre Missetat;
\bibleverse{18} so soll der Betreffende dem Priester einen tadellosen
Widder von seiner Herde nach deiner eigenen Schätzung zum Schuldopfer
bringen, und der Priester soll ihm Sühne erwirken wegen seines
Versehens, das er unwissentlich begangen hat; so wird ihm vergeben
werden. \bibleverse{19} Es ist ein Schuldopfer, das er dem
\textsc{Herrn} schuldig ist.

\hypertarget{section-5}{%
\section{6}\label{section-5}}

\bibleverse{1} Und der \textsc{Herr} redete zu Mose und sprach:
\bibleverse{2} Wenn sich jemand dadurch versündigt und vergreift am
\textsc{Herrn}, daß er seinem Volksgenossen etwas Anvertrautes oder
Hinterlegtes ableugnet oder gewalttätigerweise raubt; \bibleverse{3}
oder wenn er etwas Verlorenes gefunden hat und es ableugnet und schwört
einen falschen Eid wegen irgend etwas von alledem, womit sich ein Mensch
versündigen mag; \bibleverse{4} wenn er nun, nachdem er also gesündigt
hat, sich schuldig fühlt, so soll er den Raub, den er genommen hat, oder
das erpreßte Gut, das er sich gewalttätigerweise angeeignet hat, oder
das anvertraute Gut, das ihm anvertraut worden, oder das Verlorene, das
er gefunden hat, wiedergeben; \bibleverse{5} auch alles, worüber er
einen falschen Eid geschworen hat, soll er nach seinem vollen Wert
zurückerstatten und noch einen Fünftel dazulegen; und zwar soll er es
dem geben, dem es gehört, an dem Tage, da er sein Schuldopfer
entrichtet. \bibleverse{6} Sein Schuldopfer aber soll er dem
\textsc{Herrn} bringen, einen tadellosen Widder von der Herde nach
deiner Schätzung als Schuldopfer, zum Priester. \bibleverse{7} Und der
Priester soll ihm Sühne erwirken vor dem \textsc{Herrn}, so wird ihm
vergeben werden, was irgend er getan hat von alledem, womit man sich
verschulden kann. \bibleverse{8} Und der \textsc{Herr} redete zu Mose
und sprach: Gebiete Aaron und seinen Söhnen und sprich: \bibleverse{9}
Dies ist das Gesetz vom Brandopfer: Das Brandopfer soll auf seiner Glut
auf dem Altar die ganze Nacht bis zum Morgen verbleiben, daß das Feuer
des Altars dadurch genährt werde. \bibleverse{10} Und der Priester soll
ein leinenes Kleid anziehen und sein Fleisch in die leinenen Beinkleider
hüllen und soll die Asche abheben, nachdem das Feuer auf dem Altar das
Brandopfer verzehrt hat, und sie neben den Altar tun. \bibleverse{11}
Dann lege er seine Kleider ab und ziehe andere Kleider an und schaffe
die Asche hinaus vor das Lager an einen reinen Ort. \bibleverse{12} Aber
das Feuer auf dem Altar soll auf demselben brennend erhalten werden; es
soll nicht erlöschen; darum soll der Priester alle Morgen Holz darauf
anzünden und das Brandopfer darauf zurichten und das Fett der Dankopfer
darauf verbrennen. \bibleverse{13} Ein beständiges Feuer soll auf dem
Altar brennen; es soll nie erlöschen! \bibleverse{14} Und dies ist das
Gesetz vom Speisopfer: Die Söhne Aarons sollen es vor dem \textsc{Herrn}
darbringen, vor dem Altar. \bibleverse{15} Und dann hebe einer davon
eine Handvoll ab, von dem Semmelmehl des Speisopfers und von seinem Öl,
auch allen Weihrauch, der auf dem Speisopfer ist, und verbrenne also,
was davon zum Gedächtnis bestimmt ist, auf dem Altar zum lieblichen
Geruch dem \textsc{Herrn}. \bibleverse{16} Das Übrige aber sollen Aaron
und seine Söhne essen; ungesäuert soll es gegessen werden an einem
heiligen Ort; im Vorhof der Stiftshütte sollen sie es essen.
\bibleverse{17} Es soll ungesäuert gebacken werden. Ich habe es ihnen
gegeben als ihren Anteil an meinen Feueropfern; es ist hochheilig wie
das Sündopfer und wie das Schuldopfer. \bibleverse{18} Alles, was
männlich ist unter Aarons Nachkommen, darf davon essen; es ist ein auf
ewig festgesetzter Anteil an den Feueropfern des \textsc{Herrn} für alle
eure Geschlechter. Jeder, der es anrührt, soll heilig sein!
\bibleverse{19} Und der \textsc{Herr} redete zu Mose und sprach:
\bibleverse{20} Dies ist die Opfergabe Aarons und seiner Söhne, welche
sie dem \textsc{Herrn} darbringen sollen am Tage seiner Salbung. Ein
Zehntel Epha Semmelmehl als beständiges Speisopfer, die eine Hälfte am
Morgen, die andere am Abend. \bibleverse{21} Es soll in der Pfanne mit
Öl angemacht werden, durcheinandergerührt soll man es darbringen, in
Kuchenform, in Bissen zerlegt soll man das Speisopfer darbringen zum
lieblichen Geruch dem \textsc{Herrn}. \bibleverse{22} Und zwar soll es
der Priester, der an Aarons Statt aus seinen Söhnen gesalbt wird,
bereiten, dem \textsc{Herrn} zum ewigen Recht; es soll gänzlich
verbrannt werden. \bibleverse{23} Jedes Speisopfer eines Priesters soll
ganz verbrannt werden; es darf nicht gegessen werden. \bibleverse{24}
Und der \textsc{Herr} redete zu Mose und sprach: \bibleverse{25} Sage zu
Aaron und zu seinen Söhnen und sprich: Dies ist das Gesetz vom
Sündopfer: Am gleichen Ort, da man das Brandopfer schächtet, soll auch
das Sündopfer geschächtet werden vor dem \textsc{Herrn}, weil es
hochheilig ist. \bibleverse{26} Der Priester, der das Sündopfer
darbringt, darf es essen; es soll aber an heiliger Stätte gegessen
werden, im Vorhof der Stiftshütte. \bibleverse{27} Jeder, der sein
Fleisch anrührt, soll heilig sein! Wenn aber etwas von seinem Blut auf
ein Kleid spritzt, so sollst du das, was bespritzt worden ist, an
heiliger Stätte waschen. \bibleverse{28} Ist es in einem irdenen
Geschirr gekocht worden, so soll man dasselbe zerbrechen, wenn aber in
einem ehernen, so muß es gescheuert und mit Wasser gespült werden.
\bibleverse{29} Alles, was männlich ist unter den Priestern, darf davon
essen; es ist hochheilig. \bibleverse{30} Dagegen soll man kein
Sündopfer essen, von dessen Blut in die Stiftshütte hineingebracht wird,
um Sühne zu erwirken im Heiligtum; es soll mit Feuer verbrannt werden.

\hypertarget{section-6}{%
\section{7}\label{section-6}}

\bibleverse{1} Und dies ist das Gesetz vom Schuldopfer, welches
hochheilig ist: \bibleverse{2} Am gleichen Ort, wo man das Brandopfer
schächtet, soll man auch das Schuldopfer schächten und sein Blut ringsum
an den Altar sprengen. \bibleverse{3} Auch soll man von ihm all sein
Fett darbringen, den Fettschwanz samt dem Fett, welches die Eingeweide
bedeckt; \bibleverse{4} dazu die beiden Nieren mit dem Fett daran, das
an den Lenden ist, samt dem, was über die Leber hervorragt; über den
Nieren soll man es wegnehmen. \bibleverse{5} Und der Priester soll es
auf dem Altar verbrennen, daß solches Schuldopfer zu einem Feuer werde
für den \textsc{Herrn}. \bibleverse{6} Alles, was männlich ist unter den
Priestern, darf es essen; es soll aber an heiliger Stätte gegessen
werden, weil es hochheilig ist. \bibleverse{7} Wie das Sündopfer, so das
Schuldopfer; für beide gilt ein und dasselbe Gesetz: Es gehört dem
Priester, der die Sühne damit vollzieht. \bibleverse{8} Dem Priester,
der jemandes Brandopfer darbringt, gehört auch das Fell des Brandopfers,
welches er dargebracht hat. \bibleverse{9} Desgleichen alle Speisopfer,
die im Ofen gebacken, im Topf gekocht oder auf der Pfanne bereitet
werden, fallen dem Priester zu, der sie darbringt. \bibleverse{10} Alle
Speisopfer, seien sie nun mit Öl vermengt oder trocken, gehören allen
Söhnen Aarons, einem wie dem andern. \bibleverse{11} Und dies ist das
Gesetz des Dankopfers, das man dem \textsc{Herrn} darbringen soll:
\bibleverse{12} Will er es zum Lobe opfern, so bringe er zu seinem
Lob-Schlachtopfer hinzu ungesäuerte Kuchen dar, mit Öl gemengt, und
ungesäuerte Fladen, mit Öl bestrichen, und eingerührtes Semmelmehl, mit
Öl gemengte Kuchen. \bibleverse{13} Auf einem gesäuerten Brotkuchen soll
er seine Opfergabe darbringen, zum Schlachtopfer seines Lob und
Dankopfers hinzu. \bibleverse{14} Von allen Opfergaben aber soll er dem
\textsc{Herrn} je ein Stück als Hebe darbringen; das soll dem Priester
gehören, der das Blut der Dankopfer sprengt. \bibleverse{15} Es soll
aber das Fleisch des Lob und Dankopfers am Tage seiner Darbringung
gegessen werden; man darf nichts davon übriglassen bis zum Morgen.
\bibleverse{16} Beruht aber das Opfer, das er darbringt, auf einem
Gelübde, oder ist es freiwillig, so soll es am Tage seiner Darbringung
gegessen werden und am folgenden Tag, so daß, was davon übrigbleibt,
gegessen werden darf. \bibleverse{17} Was aber vom Opferfleisch bis zum
dritten Tag übrigbleibt, das soll man mit Feuer verbrennen.
\bibleverse{18} Sollte aber trotzdem am dritten Tage von dem Fleisch
seines Dankopfers gegessen werden, so würde der, welcher es dargebracht
hat, nicht angenehm sein; es würde ihm nicht zugerechnet, sondern für
verdorben gelten, und die Seele, die davon äße, müßte ihre Schuld
tragen. \bibleverse{19} Auch wenn das Fleisch mit irgend etwas Unreinem
in Berührung kommt, so darf man es nicht essen, sondern muß es mit Feuer
verbrennen; sonst aber darf jedermann von diesem Fleisch essen, wenn er
rein ist. \bibleverse{20} Eine Seele aber, die ihre Unreinigkeit an sich
hat und doch von dem Fleisch des Dankopfers ißt, das dem \textsc{Herrn}
gehört, die soll ausgerottet werden aus ihrem Volk. \bibleverse{21} Auch
wenn eine Seele irgend etwas Unreines anrührt, es sei die Unreinigkeit
eines Menschen oder ein unreines Vieh oder irgend ein unreines Reptil,
und ißt doch von dem Fleisch des Dankopfers, das dem \textsc{Herrn}
gehört, so soll eine solche Seele ausgerottet werden von ihrem Volk.
\bibleverse{22} Und der \textsc{Herr} redete zu Mose und sprach:
\bibleverse{23} Sage den Kindern Israel und sprich: Ihr sollt kein Fett
essen von Ochsen, Lämmern und Ziegen! \bibleverse{24} Das Fett von Aas
oder Zerrissenem darf zu allerlei Zwecken verwendet werden, aber essen
sollt ihr es nicht. \bibleverse{25} Denn wer Fett ißt von dem Vieh, von
welchem man dem \textsc{Herrn} Feueropfer darzubringen pflegt, der soll
ausgerottet werden aus seinem Volk! \bibleverse{26} Ihr sollt auch kein
Blut essen in allen euren Wohnungen, weder von Geflügel noch vom Vieh;
\bibleverse{27} jede Seele, die irgendwelches Blut ißt, soll ausgerottet
werden aus ihrem Volk! \bibleverse{28} Und der \textsc{Herr} redete zu
Mose und sprach: \bibleverse{29} Sage zu den Kindern Israel und sprich:
Wer dem \textsc{Herrn} ein Dankopfer darbringen will, der lasse dem
\textsc{Herrn} seine Gabe zukommen von seinem Dankopfer. \bibleverse{30}
Eigenhändig soll er herzubringen, was dem \textsc{Herrn} verbrannt
werden soll: Das Fett samt dem Kern stück der Brust soll er bringen, den
Brustkern, um ihn als Webopfer vor dem \textsc{Herrn} zu weben.
\bibleverse{31} Der Priester aber soll das Fett auf dem Altar
verbrennen; und der Brustkern fällt Aaron und seinen Söhnen zu.
\bibleverse{32} Dazu sollt ihr die rechte Keule von euren Dankopfern dem
Priester als Hebe geben; \bibleverse{33} und zwar soll derjenige von
Aarons Söhnen, der das Blut der Dankopfer und das Fett darbringt, die
rechte Keule zum Anteil erhalten. \bibleverse{34} Denn ich habe die
Webebrust und die Hebekeule von den Kindern Israel, von ihren Dankopfern
genommen und habe sie dem Priester Aaron und seinen Söhnen gegeben zum
ewigen Anrecht, das sie zu beanspruchen haben von den Kindern Israel.
\bibleverse{35} Das ist das Salbungsgeschenk, welches Aaron und seinen
Söhnen gemacht wurde von den Feueropfern des \textsc{Herrn} an dem Tage,
da er sie herzunahen ließ, dem \textsc{Herrn} Priesterdienst zu tun,
\bibleverse{36} davon der \textsc{Herr} am Tage ihrer Salbung befahl,
daß es ihnen gegeben werde von den Kindern Israel als ewiges Recht in
ihren Geschlechtern. \bibleverse{37} Dies ist das Gesetz vom Brandopfer,
vom Speisopfer, vom Sündopfer, vom Schuldopfer, vom Einweihungsopfer und
vom Dankopfer, \bibleverse{38} welches der \textsc{Herr} Mose auf dem
Berge Sinai gegeben hat, des Tages, da er den Kindern Israel befahl, dem
\textsc{Herrn} ihre Opfer darzubringen, in der Wüste Sinai.

\hypertarget{section-7}{%
\section{8}\label{section-7}}

\bibleverse{1} Und der \textsc{Herr} redete zu Mose und sprach:
\bibleverse{2} Nimm Aaron und seine Söhne mit ihm, dazu die Kleider und
das Salböl und einen Farren zum Sündopfer, zwei Widder und einen Korb
mit ungesäuertem Brot, \bibleverse{3} und versammle die ganze Gemeinde
vor der Tür der Stiftshütte. \bibleverse{4} Mose tat, wie ihm der
\textsc{Herr} befahl, und versammelte die Gemeinde vor der Tür der
Stiftshütte. \bibleverse{5} Und Mose sprach zu der Gemeinde: Das ist's,
was der \textsc{Herr} zu tun geboten hat. \bibleverse{6} Und Mose
brachte Aaron und seine Söhne herzu und wusch sie mit Wasser.
\bibleverse{7} Und er legte ihm den Leibrock an und gürtete ihn mit dem
Gürtel des Ephod und befestigte es damit. \bibleverse{8} Darnach legte
er ihm das Brustschildlein an und tat in das Brustschildlein das Licht
und das Recht; \bibleverse{9} und er setzte ihm den Hut auf das Haupt
und heftete an den Hut, vorn an seine Stirne, das goldene Stirnblatt,
das heilige Diadem, wie der \textsc{Herr} Mose geboten hatte.
\bibleverse{10} Und Mose nahm das Salböl und salbte die Wohnung und
alles, was darin war, und weihte es. \bibleverse{11} Auch sprengte er
davon siebenmal auf den Altar und salbte den Altar samt allen seinen
Geräten, auch das Becken samt seinem Fuß, um es zu weihen.
\bibleverse{12} Und er goß von dem Salböl auf das Haupt Aarons und
salbte ihn, um ihn zu weihen. \bibleverse{13} Er brachte auch die Söhne
Aarons herzu und zog ihnen Leibröcke an und gürtete sie mit dem Gürtel
und band ihnen hohe Mützen auf, wie der \textsc{Herr} Mose geboten
hatte. \bibleverse{14} Und ließ einen Farren herzuführen zum Sündopfer;
und Aaron und seine Söhne stützten ihre Hände auf den Kopf des Farren,
des Sündopfers. \bibleverse{15} Und Mose schächtete ihn und nahm das
Blut und tat es mit seinem Finger auf die Hörner des Altars ringsum;
also entsündigte er den Altar und goß das übrige Blut an den Grund des
Altars und weihte ihn, indem er ihm Sühne erwirkte. \bibleverse{16}
Sodann nahm er alles Fett am Eingeweide und was über die Leber
hervorragt und die beiden Nieren mit dem Fett daran, und Mose verbrannte
es auf dem Altar. \bibleverse{17} Aber den Farren samt seinem Fell und
seinem Fleisch und Mist verbrannte er mit Feuer außerhalb des Lagers,
wie der \textsc{Herr} Mose geboten hatte. \bibleverse{18} Er brachte
auch den Widder herzu zum Brandopfer. Und Aaron und seine Söhne stützten
ihre Hände auf des Widders Kopf. \bibleverse{19} Und Mose schächtete ihn
und sprengte das Blut ringsrum an den Altar \bibleverse{20} und zerlegte
den Widder in seine Stücke, und Mose verbrannte den Kopf, die Stücke und
das Fett, \bibleverse{21} und wusch die Eingeweide und die Schenkel mit
Wasser; also verbrannte Mose den ganzen Widder auf dem Altar. Das war
ein Brandopfer zum lieblichen Geruch, ein Feueropfer dem \textsc{Herrn},
wie der \textsc{Herr} Mose geboten hatte. \bibleverse{22} Er brachte
auch den andern Widder herzu, den Widder des Einweihungsopfers. Und
Aaron und seine Söhne stützten ihre Hände auf des Widders Kopf.
\bibleverse{23} Mose aber schächtete ihn und nahm von seinem Blut, und
tat es Aaron auf sein rechtes Ohrläpplein und auf den Daumen seiner
rechten Hand und auf die große Zehe seines rechten Fußes,
\bibleverse{24} und er brachte auch Aarons Söhne herzu und tat von dem
Blut auf ihr rechtes Ohrläpplein und auf den Daumen ihrer rechten Hand
und auf die große Zehe ihres rechten Fußes und sprengte das Blut ringsum
an den Altar. \bibleverse{25} Und er nahm das Fett und den Fettschwanz
und alles Fett am Eingeweide und was über die Leber hervorragt und die
beiden Nieren mit dem Fett daran und die rechte Keule; \bibleverse{26}
dazu nahm er aus dem Korbe mit dem ungesäuerten Brot vor dem
\textsc{Herrn} einen ungesäuerten Kuchen und einen Brotkuchen mit Öl und
einen Fladen und legte es auf die Fettstücke und auf die rechte Keule,
\bibleverse{27} und legte das alles auf die Hände Aarons und auf die
Hände seiner Söhne und webte es zum Webopfer vor dem \textsc{Herrn}.
\bibleverse{28} Darnach nahm Mose das alles wieder von ihren Händen und
verbrannte es auf dem Altar über dem Brandopfer. Das war das
Einweihungsopfer zum lieblichen Geruch, ein Feueropfer dem
\textsc{Herrn}. \bibleverse{29} Und Mose nahm die Brust und webte sie
zum Webopfer vor dem \textsc{Herrn}; das war Moses Anteil von dem Widder
des Einweihungsopfers, wie der \textsc{Herr} Mose geboten hatte.
\bibleverse{30} Und Mose nahm von dem Salböl und von dem Blut auf dem
Altar und sprengte es auf Aaron und seine Kleider, auf seine Söhne und
ihre Kleider und weihte also Aaron und seine Kleider und mit ihm seine
Söhne und seiner Söhne Kleider. \bibleverse{31} Und Mose sprach zu Aaron
und zu seinen Söhnen: Kocht das Fleisch vor der Tür der Stiftshütte und
esset es daselbst, wie ich geboten und gesagt habe: Aaron und seine
Söhne sollen es essen. \bibleverse{32} Was aber vom Fleisch und Brot
übrigbleibt, das sollt ihr mit Feuer verbrennen. \bibleverse{33} Und ihr
sollt sieben Tage lang nicht hinausgehen vor die Tür der Stiftshütte,
bis an den Tag, an welchem die Tage eures Weihopfers erfüllt sind; denn
sieben Tage lang soll man euch die Hände füllen. \bibleverse{34} Was man
heute getan hat, das hat der \textsc{Herr} zu tun befohlen, um für euch
Sühne zu erwirken. \bibleverse{35} Sieben Tage lang sollt ihr Tag und
Nacht an der Tür der Stiftshütte bleiben und die Anordnungen des
\textsc{Herrn} befolgen, daß ihr nicht sterbet; denn also ist es mir
geboten worden. \bibleverse{36} Und Aaron und seine Söhne taten alles,
was der \textsc{Herr} durch Mose geboten hatte.

\hypertarget{section-8}{%
\section{9}\label{section-8}}

\bibleverse{1} Und am achten Tage rief Mose den Aaron und seine Söhne
und die Ältesten von Israel und sprach zu Aaron: \bibleverse{2} Nimm dir
ein Stierkalb zum Sündopfer und einen Widder zum Brandopfer, beide
tadellos, und bringe sie vor den \textsc{Herrn} \bibleverse{3} und sage
zu den Kindern Israel und sprich: Nehmt einen Ziegenbock zum Sündopfer
und ein Kalb und ein Schaf, beide ein Jahr alt und tadellos, zum
Brandopfer, \bibleverse{4} ferner einen Ochsen und einen Widder zum
Dankopfer, vor dem \textsc{Herrn} zu opfern, und ein mit Öl gemengtes
Speisopfer; denn heute wird euch der \textsc{Herr} erscheinen.
\bibleverse{5} Und sie brachten, was Mose geboten hatte, vor die Tür der
Stiftshütte, und die ganze Gemeinde trat herzu und stand vor dem
\textsc{Herrn}. \bibleverse{6} Da sprach Mose: Was der \textsc{Herr}
geboten hat, das sollt ihr tun, so wird euch die Herrlichkeit des
\textsc{Herrn} erscheinen! \bibleverse{7} Und Mose sprach zu Aaron:
Tritt zum Altar und verrichte dein Sündopfer und dein Brandopfer und
erwirke Sühne für dich und das Volk. Darnach bringe das Opfer des Volkes
dar und erwirke Sühnung für sie, wie der \textsc{Herr} geboten hat!
\bibleverse{8} Da trat Aaron zum Altar und schächtete das Kalb zum
Sündopfer. \bibleverse{9} Und die Söhne Aarons brachten das Blut zu ihm,
und er tauchte seinen Finger in das Blut und tat es auf die Hörner des
Altars und goß das übrige Blut an den Grund des Altars. \bibleverse{10}
Aber das Fett und die Nieren und was von der Leber des Sündopfers
hervorragt, verbrannte er auf dem Altar, wie der \textsc{Herr} Mose
geboten hatte. \bibleverse{11} Und das Fleisch und das Fell verbrannte
er mit Feuer außerhalb des Lagers. \bibleverse{12} Darnach schächtete er
das Brandopfer, und die Söhne Aarons brachten das Blut zu ihm, und er
sprengte es ringsum an den Altar. \bibleverse{13} Und sie brachten das
Brandopfer, in seine Stücke zerlegt, samt dem Kopf, zu ihm, und er
verbrannte es auf dem Altar. \bibleverse{14} Und er wusch die Eingeweide
und die Schenkel und verbrannte es über dem Brandopfer auf dem Altar.
\bibleverse{15} Darnach brachte er das Opfer des Volkes herzu und nahm
den Bock, das Sündopfer des Volkes, und schächtete ihn und machte ein
Sündopfer daraus, wie das vorige. \bibleverse{16} Darnach brachte er das
Brandopfer herzu und verrichtete es nach Vorschrift. \bibleverse{17} Er
brachte auch das Speisopfer herzu und nahm eine Handvoll davon und
verbrannte es auf dem Altar, außer dem Brandopfer des Morgens.
\bibleverse{18} Darnach schächtete er den Ochsen und den Widder zum
Dankopfer des Volks. Und die Söhne Aarons brachten ihm das Blut; das
sprengte er ringsum an den Altar. \bibleverse{19} Aber die Fettstücke
vom Ochsen und vom Widder, den Fettschwanz und das Fett, welches die
Eingeweide bedeckt, und die Nieren und was über die Leber hervorragt,
\bibleverse{20} alle diese Fettstücke legten sie auf die Brust; und er
verbrannte die Fettstücke auf dem Altar. \bibleverse{21} Aber die Brust
und die rechte Schulter webte Aaron zum Webopfer vor dem \textsc{Herrn},
wie Mose geboten hatte. \bibleverse{22} Darnach streckte Aaron seine
Hand aus gegen das Volk und segnete es und stieg herab, nachdem er das
Sündopfer, das Brandopfer und das Dankopfer dargebracht hatte.
\bibleverse{23} Und Mose und Aaron gingen in die Stiftshütte hinein. Und
als sie wieder herauskamen, segneten sie das Volk. Da erschien die
Herrlichkeit des \textsc{Herrn} allem Volk; \bibleverse{24} und es ging
Feuer aus von dem \textsc{Herrn} und verzehrte das Brandopfer und die
Fettstücke auf dem Altar. Als alles Volk solches sah, jubelten sie und
fielen auf ihre Angesichter.

\hypertarget{section-9}{%
\section{10}\label{section-9}}

\bibleverse{1} Aber die Söhne Aarons, Nadab und Abihu, nahmen ein jeder
seine Räucherpfanne und taten Feuer hinein und legten Räucherwerk darauf
und brachten fremdes Feuer vor den \textsc{Herrn}, das er ihnen nicht
geboten hatte. \bibleverse{2} Da ging Feuer aus von dem \textsc{Herrn}
und verzehrte sie, daß sie starben vor dem \textsc{Herrn}.
\bibleverse{3} Da sprach Mose zu Aaron: Das hat der \textsc{Herr}
gemeint, als er sprach: Ich will geheiligt werden durch die, welche zu
mir nahen, und geehrt werden vor allem Volk! Und Aaron schwieg still.
\bibleverse{4} Mose aber rief Misael und Elzaphan, die Söhne Ussiels,
des Oheims Aarons, und sprach zu ihnen: Tretet herzu und traget eure
Brüder vom Heiligtum hinweg, vor das Lager hinaus! \bibleverse{5} Und
sie traten herzu und trugen sie in ihren Leibröcken vor das Lager
hinaus, wie Mose befohlen hatte. \bibleverse{6} Da sprach Mose zu Aaron
und seinen Söhnen Eleasar und Itamar: Ihr sollt eure Häupter nicht
entblößen, noch eure Kleider zerreißen, damit ihr nicht sterbet und der
Zorn über die ganze Gemeinde komme. Lasset eure Brüder, das ganze Haus
Israel, weinen über diesen Brand, den der \textsc{Herr} angezündet hat.
\bibleverse{7} Ihr aber sollt nicht vor die Tür der Stiftshütte
hinausgehen, auf daß ihr nicht sterbet; denn das Salböl des
\textsc{Herrn} ist auf euch! Und sie taten, wie Mose sagte.
\bibleverse{8} \textsc{Der herr} aber redete mit Aaron und sprach:
\bibleverse{9} Du und deine Söhne mit dir sollen keinen Wein noch
starkes Getränk trinken, wenn ihr in die Stiftshütte geht, damit ihr
nicht sterbet. Das sei eine ewige Ordnung für eure Geschlechter,
\bibleverse{10} damit ihr unterscheiden könnet zwischen heilig und
gemein, zwischen unrein und rein, \bibleverse{11} und damit ihr die
Kinder Israel alle Rechte lehret, die der \textsc{Herr} zu ihnen durch
Mose geredet hat. \bibleverse{12} Und Mose redete mit Aaron und mit
seinen übrigen Söhnen, Eleasar und Itamar: Nehmt das Speisopfer, das von
den Feueropfern des \textsc{Herrn} übrigbleibt, und esset es ungesäuert
beim Altar, denn es ist hochheilig. \bibleverse{13} Ihr sollt es essen
an heiliger Stätte; denn es ist das, was dir und deinen Söhnen bestimmt
ist von den Feueropfern des \textsc{Herrn}; denn also ist es mir geboten
worden. \bibleverse{14} Desgleichen die Webebrust und die Hebeschulter
sollst du und deine Söhne und deine Töchter mit dir an reiner Stätte
essen. Denn solches ist dir und deinen Kindern bestimmt von den
Dankopfern der Kinder Israel. \bibleverse{15} Die Hebekeule und die
Webebrust soll man mit den Feueropfern der Fettstücke herzubringen, daß
man sie webe zum Webopfer vor dem \textsc{Herrn}. Solches soll dir und
deinen Söhnen mit dir als ein ewiges Recht zufallen, wie der
\textsc{Herr} geboten hat. \bibleverse{16} Und Mose suchte den Bock des
Sündopfers; und siehe, er war verbrannt. Da ward er zornig über Eleasar
und Itamar, die Söhne Aarons, die noch übrig waren, und sprach:
\bibleverse{17} Warum habt ihr das Sündopfer nicht gegessen an heiliger
Stätte? Denn es ist hochheilig, und er hat es euch gegeben, daß ihr die
Missetat der Gemeinde traget, um für sie Sühne zu erwirken vor dem
\textsc{Herrn}! \bibleverse{18} Siehe, sein Blut ist nicht in das Innere
des Heiligtums hineingekommen; ihr hättet ihn im Heiligtum essen sollen,
wie ich geboten habe. \bibleverse{19} Aaron aber sprach zu Mose: Siehe,
heute haben sie ihr Sündopfer vor dem \textsc{Herrn} geopfert, und es
ist mir solches widerfahren; sollte ich heute vom Sündopfer essen? Wäre
es auch wohl getan in den Augen des \textsc{Herrn}? \bibleverse{20} Als
Mose solches hörte, war es wohlgefällig in seinen Augen.

\hypertarget{section-10}{%
\section{11}\label{section-10}}

\bibleverse{1} Und der \textsc{Herr} redete zu Mose und Aaron und sprach
zu ihnen: \bibleverse{2} Redet mit den Kindern Israel und sprechet: Das
sind die Tiere, die ihr von allem Vieh auf Erden essen dürft:
\bibleverse{3} Alle Vielhufer, die ganz gespaltene Klauen haben und
wiederkäuen, dürft ihr essen. \bibleverse{4} Aber von den Wiederkäuern
und Vielhufern sollt ihr die folgenden nicht essen: das Kamel; denn
obschon es wiederkäut, hat es doch keine gespaltenen Klauen; darum soll
es euch unrein sein. \bibleverse{5} Desgleichen der Klippdachs; denn
obschon er wiederkäut, hat er doch keine gespaltenen Klauen; darum ist
er euch unrein. \bibleverse{6} Auch der Hase, der zwar wiederkäut, aber
er hat keine gespaltenen Klauen; darum ist er euch unrein.
\bibleverse{7} Ferner das Schwein; es ist zwar ein Vielhufer mit
durchgespaltenen Klauen, aber kein Wiederkäuer; darum ist es euch
unrein. \bibleverse{8} Von ihrem Fleisch sollt ihr nicht essen, auch ihr
Aas nicht anrühren, denn sie sind euch unrein. \bibleverse{9} Folgende
Tiere dürft ihr essen von allem, was in den Wassern ist: Alles, was
Flossen und Schuppen hat im Wasser, im Meer und in Bächen, dürft ihr
essen. \bibleverse{10} Aber alles, was keine Flossen und Schuppen hat,
im Meer und in Bächen, unter allem Getier, das sich in den Wassern regt,
und von allem, was im Wasser lebt, das soll euch ein Greuel sein.
\bibleverse{11} Ein Greuel sollen sie euch sein; von ihrem Fleisch sollt
ihr nicht essen und vor ihrem Aas euch scheuen. \bibleverse{12} Alle
Wassertiere, die keine Flossen und Schuppen haben, sollen euch ein
Greuel sein. \bibleverse{13} Von den Vögeln aber sollt ihr folgende
verabscheuen; man soll sie nicht essen, weil sie ein Greuel sind: Den
Adler, den Lämmergeier und den Seeadler, \bibleverse{14} die Weihe und
das Falkengeschlecht, \bibleverse{15} die ganze Rabenfamilie,
\bibleverse{16} den Strauß, die Eule, die Möwe und die Habichtarten;
\bibleverse{17} das Käuzchen, den Reiher, den Ibis, \bibleverse{18} das
Purpurhuhn, den Pelikan, den Schwan, \bibleverse{19} den Storch, die
verschiedenen Strandläufer, den Wiedehopf und die Fledermaus.
\bibleverse{20} Jedes geflügelte Insekt, das auf vier Füßen geht, soll
euch ein Greuel sein. \bibleverse{21} Doch dürft ihr von den geflügelten
Insekten, welche auf vier Füßen gehen, diejenigen essen, welche oberhalb
ihrer Füße zwei Schenkel haben, vermittelst deren sie über den Erdboden
hüpfen können. \bibleverse{22} Von diesen dürft ihr essen die
verschiedenen Arten der Wanderheuschrecke, der Feldheuschrecke, der
Laubheuschrecke und der Fangheuschrecke. \bibleverse{23} Aber alle
übrigen geflügelten Insekten mit vier Füßen sollen euch ein Greuel sein,
\bibleverse{24} und ihr würdet euch an ihnen verunreinigen; wer ihr Aas
anrührt, der soll unrein sein bis zum Abend; \bibleverse{25} wer aber
eines ihrer Aase aufhebt, der soll seine Kleider waschen und bleibt
unrein bis zum Abend. \bibleverse{26} Jeder Vielhufer, der nicht
zugleich durchgespaltene Klauen hat und wiederkäut, soll euch unrein
sein; wer ihn anrührt, wird unrein. \bibleverse{27} Auch alles, was auf
seinen Tatzen geht unter den Vierfüßlern, soll euch unrein sein; wer ihr
Aas anrührt, wird unrein sein bis zum Abend; \bibleverse{28} und wer ihr
Aas aufhebt, der soll seine Keider waschen und bleibt unrein bis zum
Abend; unrein sollen sie euch sein. \bibleverse{29} Auch diese sollen
euch unrein sein von den Tieren, die auf der Erde kriechen: Das Wiesel,
die Maus, die verschiedenen Eidechsenarten; \bibleverse{30} der
Mauergeko, der Dornschwanz, der Schleuderschwanz, der Salamander und das
Chamäleon. \bibleverse{31} Diese sollen euch unrein sein unter allem,
was da kriecht; wer sie anrührt, wenn sie tot sind, bleibt unrein bis
zum Abend. \bibleverse{32} Auch wird alles unrein, worauf eins von
diesen Tieren fällt, wenn es tot ist, sei es ein hölzernes Gefäß oder
ein Kleid, ein Fell oder ein Sack; ein Gerät aber, damit man Arbeit
verrichtet, soll man ins Wasser legen, und es soll unrein bleiben bis
zum Abend; dann wird es rein. \bibleverse{33} Fällt aber eines jener
Tiere in ein irdenes Geschirr, so wird sein ganzer Inhalt unrein, und
ihr müßt es zerbrechen. \bibleverse{34} Kommt von dem Wasser an
irgendeine Speise, die man essen will, so wird sie unrein, desgleichen
jedes Getränk, das man aus einem solchen Gefäß trinken würde.
\bibleverse{35} Alles wird unrein, worauf ein solches Aas fällt; wäre es
ein Backofen oder Kochherd, so müßte er eingerissen werden; denn er wäre
unrein und müßte euch für unrein gelten. \bibleverse{36} Nur ein
Wassersammler, der von einer Quelle oder von einem Brunnen gespeist
wird, bleibt rein; wer aber ein Aas anrührt, das hineinfällt, wird
gleichwohl unrein. \bibleverse{37} Auch wenn von solchem Aas auf
irgendwelche Sämereien fällt, die man aussäen will, so bleiben sie rein;
\bibleverse{38} wäre aber Wasser auf den Samen gegossen worden, und es
fiele von solchem Aas darauf, so müßte er euch für unrein gelten.
\bibleverse{39} Stirbt ein Stück Vieh, das man sonst zu essen pflegt, so
wird, wer sein Aas anrührt, unrein sein bis zum Abend; \bibleverse{40}
wer aber von seinem Aase ißt, der soll seine Kleider waschen und bleibt
unrein bis zum Abend; auch wer sein Aas aufhebt, muß seine Kleider
waschen und bleibt unrein bis zum Abend. \bibleverse{41} Alles, was auf
der Erde kriecht, soll euch ein Greuel sein und darf nicht gegessen
werden. \bibleverse{42} Alles, was auf dem Bauche kriecht, samt allem,
was auf vier und mehr Füßen geht von dem, was auf der Erde kriecht, das
sollt ihr nicht essen, sondern es soll euch ein Greuel sein.
\bibleverse{43} Macht eure Seelen nicht verabscheuungswürdig durch
irgendein kriechendes Tier und verunreinigt euch nicht an ihnen, daß ihr
durch sie verunreinigt werdet! \bibleverse{44} Denn ich, der
\textsc{Herr}, bin euer Gott; darum sollt ihr euch heiligen und sollt
heilig sein; denn ich bin heilig; und ihr sollt eure Seelen nicht
verunreinigen mit allerlei Gewürm, das auf der Erde kriecht!
\bibleverse{45} Denn ich, der \textsc{Herr}, bin es, der euch aus
Ägyptenland heraufgeführt hat, um euer Gott zu sein; darum sollt ihr
heilig sein; denn ich bin heilig! \bibleverse{46} Dies ist das Gesetz
von Vieh und Vögeln und allen lebendigen Wesen, die sich im Wasser regen
und von allem Lebendigen, was auf Erden kriecht, \bibleverse{47} damit
man unterscheide zwischen unrein und rein, zwischen dem, was man essen,
und dem, was man nicht essen soll.

\hypertarget{section-11}{%
\section{12}\label{section-11}}

\bibleverse{1} Und der \textsc{Herr} redete zu Mose und sprach: Sage zu
den Kindern Israel und sprich: \bibleverse{2} Wenn ein Weib fruchtbar
wird und ein Knäblein gebiert, so soll sie sieben Tage lang unrein sein,
ebenso lange wie sie unrein ist, wenn sie unwohl wird, soll sie unrein
sein. \bibleverse{3} Und am achten Tage soll man das Fleisch seiner
Vorhaut beschneiden. \bibleverse{4} Und sie soll daheim bleiben
dreiunddreißig Tage lang im Blut ihrer Reinigung; sie soll nichts
Heiliges anrühren und nicht kommen zum Heiligtum, bis die Tage ihrer
Reinigung erfüllt sind. \bibleverse{5} Gebiert sie aber ein Mägdlein, so
soll sie zwei Wochen lang unrein sein, wie bei ihrem Unwohlsein, und
soll sechsundsechzig Tage lang daheim bleiben in dem Blut ihrer
Reinigung. \bibleverse{6} Und wenn die Tage ihrer Reinigung erfüllt sind
für den Sohn oder für die Tochter, so soll sie dem Priester vor die Tür
der Stiftshütte ein einjähriges Lamm zum Brandopfer und eine junge Taube
oder eine Turteltaube zum Sündopfer bringen; \bibleverse{7} der soll es
vor dem \textsc{Herrn} opfern und für sie Sühne erwirken, so wird sie
rein von ihrem Blutfluß. Das ist das Gesetz für die, welche ein Knäblein
oder Mägdlein gebiert. \bibleverse{8} Vermag aber ihre Hand den Preis
eines Schafes nicht, so nehme sie zwei Turteltauben oder zwei junge
Tauben, eine zum Brandopfer und die andere zum Sündopfer, so soll der
Priester für sie Sühne erwirken, daß sie rein werde.

\hypertarget{section-12}{%
\section{13}\label{section-12}}

\bibleverse{1} Und der \textsc{Herr} redete zu Mose und Aaron und
sprach: \bibleverse{2} Wenn sich bei einem Menschen an der Haut seines
Fleisches eine Geschwulst oder ein Schorf oder ein weißer Fleck zeigt,
als wollte sich ein Aussatz bilden an der Haut des Fleisches, so soll
man ihn zum Priester Aaron oder zu einem seiner Söhne unter den
Priestern führen. \bibleverse{3} Und wenn der Priester das Mal an der
Haut seines Fleisches besieht und findet, daß die Haare im Mal weiß
geworden sind, und daß das Mal tieferliegend erscheint als die Haut
seines Fleisches, so ist es der Aussatz; sobald der Priester das sieht,
soll er ihn für unrein erklären! \bibleverse{4} Wenn aber der Fleck auf
der Haut seines Fleisches weiß ist und nicht tieferliegend erscheint als
die übrige Haut des Fleisches und seine Haare nicht weiß geworden sind,
so soll der Priester das Mal sieben Tage lang einschließen,
\bibleverse{5} und am siebenten Tag soll er es besichtigen. Ist das Mal
gleich geblieben wie zuvor und hat nicht weitergefressen an der Haut, so
soll es der Priester abermal sieben Tage lang einschließen.
\bibleverse{6} Und wenn ihn der Priester am siebenten Tage nochmals
besieht und findet, daß das Mal blässer ist und nicht weitergefressen
hat an der Haut, so soll ihn der Priester für rein erklären, denn es ist
ein Ausschlag; und er soll seine Kleider waschen, so ist er rein.
\bibleverse{7} Wenn aber der Ausschlag weiter um sich greift an der
Haut, nachdem er vom Priester besehen worden ist zu seiner Reinigung, so
soll er sich dem Priester nochmals zeigen. \bibleverse{8} Wenn dann der
Priester sieht, daß der Ausschlag an der Haut weiter um sich gegriffen
hat, so soll ihn der Priester für unrein erklären; denn es ist ein
Aussatz. \bibleverse{9} Zeigt sich ein Aussatzmal an einem Menschen, so
soll man ihn zum Priester bringen; \bibleverse{10} sieht dieser an der
Haut eine weiße Geschwulst und daß die Haare weiß geworden sind und daß
rohes Fleisch in der Geschwulst ist, \bibleverse{11} so ist es ein alter
Aussatz in der Haut seines Fleisches; darum soll ihn der Priester für
unrein erklären und nicht einschließen; denn er ist schon unrein.
\bibleverse{12} Wenn aber der Aussatz an der Haut ausbricht und die
ganze Haut des Betroffenen vom Kopf bis zu den Füßen bedeckt, soweit der
Priester sehen kann, \bibleverse{13} und der Priester sieht, daß der
Aussatz das ganze Fleisch bedeckt, so soll er den Betroffenen für rein
erklären, weil er ganz weiß geworden ist; dann ist er rein.
\bibleverse{14} Sobald sich aber rohes Fleisch an ihm zeigt, so ist er
unrein. \bibleverse{15} Und wenn der Priester das rohe Fleisch sieht,
soll er ihn für unrein erklären; denn das rohe Fleisch ist unrein; es
ist der Aussatz. \bibleverse{16} Verwandelt sich aber das rohe Fleisch
wieder und wird weiß, so soll er zum Priester kommen. \bibleverse{17}
Und wenn der Priester bei der Besichtigung findet, daß das Mal weiß
geworden ist, so soll er den Betroffenen für rein erklären, denn er ist
rein. \bibleverse{18} Wenn in jemandes Fleisch an der Haut ein Geschwür
entsteht und wieder heilt, \bibleverse{19} es bildet sich aber an der
Stelle des Geschwürs eine weiße Geschwulst oder ein weiß-rötlicher
Fleck, so soll er sich dem Priester zeigen. \bibleverse{20} Sieht aber
der Priester, daß es tieferliegend erscheint als die übrige Haut und daß
das Haar weiß geworden ist, so soll er ihn für unrein erklären; denn es
ist ein Aussatzmal in dem Geschwür ausgebrochen. \bibleverse{21} Sieht
aber der Priester, daß die Haare nicht weiß sind, und daß es nicht
tieferliegend ist als die übrige Haut, sondern blässer, so soll er ihn
sieben Tage lang einschließen. \bibleverse{22} Greift es weiter um sich
an der Haut, so soll er ihn für unrein erklären; denn es ist ein
Aussatzmal. \bibleverse{23} Bleibt aber der weiße Fleck stehen und frißt
nicht weiter, so ist es die Narbe des Geschwürs, und der Priester soll
ihn für rein erklären. \bibleverse{24} Wenn jemandes Fleisch an der Haut
eine Brandwunde erhält, und es bildet sich in der Brandwunde ein
weißrötlicher oder weißer Fleck; \bibleverse{25} und wenn der Priester
es besieht und findet, daß das Haar weiß geworden ist an dem Fleck und
daß er tieferliegend erscheint als die übrige Haut, so ist ein Aussatz
in der Brandwunde entstanden; darum soll ihn der Priester für unrein
erklären; denn es ist ein Aussatzmal. \bibleverse{26} Sieht aber der
Priester, daß die Haare an dem Fleck nicht weiß geworden sind und daß er
nicht tieferliegend ist als die übrige Haut, so soll er ihn sieben Tage
lang einschließen. \bibleverse{27} Und am siebenten Tage soll er ihn
besichten; hat es weitergefressen an der Haut, so soll er ihn für unrein
erklären; denn es ist ein Aussatzmal. \bibleverse{28} Ist aber der Fleck
stehengeblieben und hat nicht weitergefressen an der Haut, so ist es
eine Geschwulst des Brandmals, und der Priester soll ihn für rein
erklären; denn es ist die Narbe des Brandmals. \bibleverse{29} Wenn ein
Mann oder ein Weib auf dem Haupt oder am Bart ein Mal hat,
\bibleverse{30} und der Priester das Mal besieht und findet, daß es
tieferliegend erscheint als die übrige Haut, und das Haar daselbst
goldgelb und dünn ist, so soll er ihn für unrein erklären; denn es ist
der Grind, ein Aussatz am Haupt oder am Bart. \bibleverse{31} Sieht aber
der Priester, daß der Grind nicht tieferliegend erscheint als die Haut,
und daß das Haar nicht goldgelb ist, so soll er den, der das Mal hat,
sieben Tage lang einschließen. \bibleverse{32} Und wenn er das Mal am
siebenten Tage besieht und findet, daß der Grind nicht weitergefressen
hat, und kein goldgelbes Haar da ist, und der Grind nicht tieferliegend
erscheint als die übrige Haut, \bibleverse{33} so soll er sich scheren
lassen, und der Priester soll den Grindigen abermal sieben Tage lang
einschließen. \bibleverse{34} Und wenn er ihn am siebenten Tage besieht
und findet, daß der Grind in der Haut nicht weitergefressen hat und
nicht tieferliegend erscheint als die übrige Haut, so soll ihn der
Priester für rein erklären, und er soll seine Kleider waschen; denn er
ist rein. \bibleverse{35} Frißt aber der Grind weiter an der Haut, nach
seiner Reinigung, \bibleverse{36} und der Priester besieht ihn und
findet, daß der Grind an der Haut weitergefressen hat, so soll er nicht
mehr darnach fragen, ob die Haare goldgelb seien; denn er ist unrein.
\bibleverse{37} Ist aber das Aussehen des Grindes gleich geblieben und
schwarzes Haar darin gewachsen, so ist der Grind geheilt, und er ist
rein; darum soll ihn der Priester für rein erklären. \bibleverse{38}
Wenn sich bei einem Manne oder einem Weibe an der Haut ihres Fleisches
weiße Flecken zeigen, \bibleverse{39} und der Priester sieht nach und
findet in der Haut ihres Fleisches blasse weiße Flecken, so ist es ein
Ausschlag, der an der Haut ausgebrochen ist, und der Betreffende ist
rein. \bibleverse{40} Wenn einem Mann die Haupthaare ausfallen, daß er
hinten kahl wird, der ist rein. \bibleverse{41} Fallen sie ihm vorn am
Haupt aus, daß er vorn eine Glatze hat, so ist er rein. \bibleverse{42}
Entsteht aber an der hintern oder vordern Glatze ein weißrötliches Mal,
so ist ihm ein Aussatz ausgebrochen an seiner hintern oder vordern
Glatze. \bibleverse{43} Darum soll ihn der Priester besehen, und wenn er
findet, daß die Geschwulst des Males an seiner Hinter oder Vorderglatze
weißrötlich ist, und wie ein Aussatz an der Haut des Fleisches anzusehen
ist, \bibleverse{44} so ist er ein aussätziger Mann und unrein, und der
Priester soll ihn für unrein erklären wegen des Mals auf seinem Kopf.
\bibleverse{45} Es soll aber der Aussätzige, der ein Mal an sich hat, in
zerrissenen Kleidern einhergehen, mit entblößtem Haupt und verhüllten
Lippen, und er soll ausrufen: Unrein, unrein! \bibleverse{46} Solange
das Mal an ihm ist, soll er unrein bleiben, denn er ist unrein; er soll
abgesondert wohnen und außerhalb des Lagers seine Wohnung haben.
\bibleverse{47} Wenn an einem Kleide ein Aussatzmal ist, es sei wollen
oder leinen; \bibleverse{48} am Zettel oder am Eintrag, es sei leinen
oder wollen, oder an einem Fell, oder an irgend etwas, das aus Fellen
gemacht wird; \bibleverse{49} und wenn das Mal grünlich oder rötlich ist
am Kleid oder am Fell, oder am Zettel oder am Eintrag, oder an irgend
etwas, das von Fellen gemacht wird, so ist es gewiß ein Aussatzmal.
Darum soll es der Priester besehen. \bibleverse{50} Und wenn er das Mal
besehen hat, soll er es sieben Tage lang einschließen. \bibleverse{51}
Und wenn er am siebenten Tage sieht, daß das Mal weitergefressen hat am
Kleid, am Zettel oder am Eintrag, am Fell oder an irgend etwas, das man
aus Fellen macht, so ist es ein fressendes Aussatzmal und der Gegenstand
ist unrein; \bibleverse{52} und er soll das Kleid verbrennen, oder den
Zettel oder Eintrag, es sei wollen oder leinen, oder allerlei Fellwerk,
darin ein solches Mal ist; denn es ist ein fressender Aussatz, und man
soll es mit Feuer verbrennen. \bibleverse{53} Sieht aber der Priester,
daß das Mal nicht weitergefressen hat am Kleid oder am Zettel oder am
Eintrag, oder an allerlei Fellwerk, \bibleverse{54} so soll er gebieten,
daß man den Gegenstand, an welchem das Mal ist, wasche, und er soll es
weitere sieben Tage lang einschließen. \bibleverse{55} Und wenn der
Priester sieht, nachdem das Mal gewaschen ist, daß das Mal seine Farbe
nicht verändert und auch nicht weitergefressen hat, so ist es unrein; du
sollst es mit Feuer verbrennen; es ist eine eingefressene Vertiefung an
der hintern und vordern Seite. \bibleverse{56} Wenn aber der Priester
sieht, daß das Mal, nachdem es gewaschen worden, verblaßt ist, so soll
er es abreißen vom Kleid, vom Fell, vom Zettel oder vom Eintrag.
\bibleverse{57} Wird es aber noch gesehen am Kleid, am Zettel, am
Eintrag oder an allerlei Fellwerk, so ist es ein ausbrechender Aussatz;
und du sollst den Gegenstand, an welchem ein solches Mal ist, mit Feuer
verbrennen. \bibleverse{58} Das Kleid aber oder den Zettel oder den
Eintrag oder allerlei Fellwerk, das gewaschen und wovon das Mal entfernt
ist, soll man nochmals waschen, so ist es rein. \bibleverse{59} Das ist
das Gesetz über das Aussatzmal an Kleidern, sie seien wollen oder
leinen, am Zettel und am Eintrag und an allerlei Fellwerk, wonach sie
für rein oder unrein zu erklären sind.

\hypertarget{section-13}{%
\section{14}\label{section-13}}

\bibleverse{1} Und der \textsc{Herr} redete zu Mose und sprach: Dieses
Gesetz gilt für den Aussätzigen am Tage seiner Reinigung: \bibleverse{2}
Er soll zum Priester kommen. \bibleverse{3} Und der Priester soll hinaus
vor das Lager gehen, und wenn er nachsieht und findet, daß das Mal des
Aussätzigen heil geworden ist, \bibleverse{4} so soll er gebieten, daß
man für den, der sich reinigen läßt, zwei lebendige Vögel bringe, welche
rein sind, und Zedernholz, Karmesin und Ysop; \bibleverse{5} und der
Priester soll gebieten, daß man den einen Vogel schächte über einem
irdenen Geschirr, darin lebendiges Wasser ist. \bibleverse{6} Den
lebendigen Vogel aber soll man nehmen samt dem Zedernholz, dem Karmesin
und Ysop, und es samt dem lebendigen Vogel in des geschächteten Vogels
Blut tauchen, das mit dem lebendigen Wasser vermischt worden ist;
\bibleverse{7} und soll denjenigen siebenmal besprengen, der sich vom
Aussatz reinigen läßt; also reinige er ihn und lasse den lebendigen
Vogel in das freie Feld fliegen. \bibleverse{8} Der zu Reinigende aber
soll seine Kleider waschen und alle seine Haare abschneiden und sich mit
Wasser baden; so ist er rein. Darnach gehe er in das Lager; doch soll er
sieben Tage lang außerhalb seiner Hütte bleiben. \bibleverse{9} Und am
siebenten Tage soll er alle seine Haare abschneiden auf dem Haupte, am
Bart und an den Augenbrauen, daß alle Haare abgeschoren seien, und soll
seine Kleider waschen und sein Fleisch im Wasser baden, so ist er rein.
\bibleverse{10} Und am achten Tag soll er zwei tadellose Lämmer nehmen,
und ein tadelloses jähriges Schaf, und drei Zehntel Semmelmehl zum
Speisopfer, mit Öl gemengt, und ein Log Öl. \bibleverse{11} Da soll dann
der Priester, der die Reinigung vollzogen hat, den zu Reinigenden und
diese Dinge vor den \textsc{Herrn} stellen, vor die Tür der Stiftshütte;
\bibleverse{12} und er soll das eine Lamm nehmen und es zum Schuldopfer
darbringen samt dem Log Öl, und soll solches vor dem \textsc{Herrn} hin
und her weben. \bibleverse{13} Darnach soll er das Lamm schächten an dem
Ort, da man das Sündopfer und das Brandopfer schächtet, an heiliger
Stätte. Denn wie das Sündopfer, also gehört auch das Schuldopfer dem
Priester: es ist hochheilig. \bibleverse{14} Und der Priester soll von
dem Blut des Schuldopfers nehmen und dem, der gereinigt werden soll, auf
das rechte Ohrläpplein tun und auf den Daumen seiner rechten Hand und
auf die große Zehe seines rechten Fußes. \bibleverse{15} Darnach soll er
von dem Log Öl nehmen und auf des Priesters linke Hand gießen,
\bibleverse{16} und der Priester soll mit seinem rechten Finger in das
Öl tunken, das in seiner linken Hand ist, und mit seinem Finger von dem
Öl siebenmal vor dem \textsc{Herrn} sprengen. \bibleverse{17} Das übrige
Öl aber in seiner Hand soll er dem, der gereinigt werden soll, auf das
rechte Ohrläpplein tun und auf den Daumen seiner rechten Hand und auf
die große Zehe seines rechten Fußes, oben auf das Blut des Schuldopfers.
\bibleverse{18} Das übrige Öl aber in seiner Hand soll er auf des zu
Reinigenden Haupt tun und ihn vor dem \textsc{Herrn} versühnen.
\bibleverse{19} Und der Priester soll das Sündopfer zurichten und für
den zu Reinigenden Sühne erwirken wegen seiner Unreinigkeit, und soll
darnach das Brandopfer schächten. \bibleverse{20} Und er soll es auf dem
Altar opfern samt dem Speisopfer und ihm Sühne erwirken; so ist er rein.
\bibleverse{21} Ist er aber arm und vermag nicht so viel, so nehme er
ein Lamm zum Schuldopfer, zum Webopfer, um für ihn Sühne zu erwirken,
und einen Zehntel Semmelmehl, mit Öl gemengt, zum Speisopfer, und ein
Log Öl, \bibleverse{22} und zwei Turteltauben oder zwei junge Tauben, je
nach seinem Vermögen, die eine zum Sündopfer, die andere zum Brandopfer,
\bibleverse{23} und bringe sie am achten Tage seiner Reinigung zum
Priester, vor die Tür der Stiftshütte, vor den \textsc{Herrn}.
\bibleverse{24} Da soll der Priester das Lamm zum Schuldopfer nehmen,
und das Öl, und soll beides vor dem \textsc{Herrn} zu einem Webopfer
weben. \bibleverse{25} Und er soll das Lamm des Schuldopfers schächten
und das Blut von demselben nehmen und dem, der gereinigt werden soll,
auf sein rechtes Ohrläpplein tun und auf den Daumen seiner rechten Hand
und auf die große Zehe seines rechten Fußes; \bibleverse{26} und von dem
Öl soll der Priester in seine linke Hand gießen, \bibleverse{27} und mit
seinem rechten Finger vom Öl, das in seiner linken Hand ist, siebenmal
vor dem \textsc{Herrn} sprengen. \bibleverse{28} Darnach soll der
Priester vom Öl in seiner Hand dem, der gereinigt werden soll, auf sein
rechtes Ohrläpplein und auf den Daumen seiner rechten Hand und auf die
große Zehe seines rechten Fußes tun, oben auf das Blut des Schuldopfers.
\bibleverse{29} Das übrige Öl in seiner Hand aber soll er dem zu
Reinigenden auf das Haupt tun, um für ihn Sühne zu erwirken vor dem
\textsc{Herrn}. \bibleverse{30} Darnach soll er eine der Turteltauben
oder der jungen Tauben (was seine Hand aufzubringen vermochte)
\bibleverse{31} zum Sündopfer zubereiten und die andere zum Brandopfer,
samt dem Speisopfer; so soll der Priester für den, der gereinigt werden
soll, Sühne erwirken vor dem \textsc{Herrn}. \bibleverse{32} Das ist das
Gesetz für den Aussätzigen, der mit seiner Hand nicht aufbringen kann,
was zu seiner Reinigung gehört. \bibleverse{33} Und der \textsc{Herr}
redete zu Mose und Aaron: \bibleverse{34} Wenn ihr in das Land Kanaan
kommt, das ich euch zur Besitzung gebe, und ich irgendein Haus des
Landes eurer Besitzung mit einem Aussatz belege, \bibleverse{35} so soll
der, dem das Haus gehört, kommen und es dem Priester anzeigen und
sprechen: Es dünkt mich, als sei ein Aussatzmal an meinem Hause.
\bibleverse{36} Dann soll der Priester gebieten, daß man das Haus
ausräume, ehe der Priester hineingeht, das Mal zu besehen, damit nicht
alles unrein werde, was im Hause ist; darnach soll der Priester
hineingehen, das Haus zu besehen. \bibleverse{37} Wenn er nun das Mal
besieht und findet, daß an der Wand des Hauses grüne oder rötliche
Grüblein sind, die tieferliegend erscheinen als die übrige Wand,
\bibleverse{38} so soll er zur Tür des Hauses hinausgehen und das Haus
sieben Tage lang verschließen. \bibleverse{39} Und wenn er am siebenten
Tage wiederkommt und nachsieht und findet, daß das Mal an der Wand des
Hauses weitergefressen hat, \bibleverse{40} so soll der Priester
befehlen, daß man die Steine ausbreche, wo das Mal ist, und daß man sie
vor die Stadt hinaus an einen unreinen Ort werfe; \bibleverse{41} und er
soll befehlen, das Haus inwendig ringsum abzuschaben, und man soll den
Schutt, den man abgeschabt hat, vor die Stadt hinaus an einen unreinen
Ort schütten \bibleverse{42} und andere Steine nehmen und an jene Stelle
tun und andern Mörtel nehmen und das Haus bewerfen. \bibleverse{43} Wenn
dann das Mal wieder kommt und am Hause ausbricht, nachdem man die Steine
ausgebrochen und das Haus abgekratzt und neu beworfen hat,
\bibleverse{44} so soll der Priester hineingehen; und wenn er sieht, daß
das Mal am Hause weitergefressen hat, so ist es ein fressender Aussatz
am Hause, und es ist unrein. \bibleverse{45} Darum soll man das Haus
abbrechen, seine Steine und sein Holz und allen Mörtel am Hause, und man
soll es vor die Stadt hinaus an einen unreinen Ort führen.
\bibleverse{46} Und wer in das Haus geht, solang es verschlossen ist,
der ist unrein bis zum Abend. \bibleverse{47} Und wer darin liegt, der
soll seine Kleider waschen; auch wer darin ißt, der soll seine Kleider
waschen. \bibleverse{48} Wenn aber der Priester beim Betreten des Hauses
sieht, daß das Mal nicht weitergefressen hat am Hause, nachdem das Haus
neu beworfen ist, so soll er es für rein erklären; denn das Mal ist heil
geworden. \bibleverse{49} Und er soll für das Haus zum Sündopfer zwei
Vögel nehmen, Zedernholz, Karmesin und Ysop, \bibleverse{50} und soll
den einen Vogel schächten über einem irdenen Geschirr, darin lebendiges
Wasser ist, \bibleverse{51} und soll das Zedernholz nehmen, den Ysop und
das Karmesin und den lebendigen Vogel, und es in des geschächteten
Vogels Blut tauchen und in das lebendige Wasser, und soll das Haus
siebenmal besprengen. \bibleverse{52} Und soll also das Haus entsündigen
mit dem Blut des Vogels, mit dem lebendigen Wasser, mit dem lebendigen
Vogel, mit dem Zedernholz, dem Ysop und Karmesin, \bibleverse{53} und
soll den lebendigen Vogel vor die Stadt hinaus in das freie Feld fliegen
lassen und für das Haus Sühne erwirken; so ist es rein. \bibleverse{54}
Dies ist das Gesetz über allerlei Aussatzmale und über den Grind,
\bibleverse{55} auch über den Aussatz der Kleider und der Häuser
\bibleverse{56} und über die Geschwulst, den Ausschlag und die weißen
Flecken, \bibleverse{57} um Belehrung zu geben für den Tag der
Verunreinigung und der Reinigung. Es ist das Gesetz vom Aussatz.

\hypertarget{section-14}{%
\section{15}\label{section-14}}

\bibleverse{1} Und der \textsc{Herr} redete zu Mose und Aaron und
sprach: \bibleverse{2} Redet mit den Kindern Israel und sprecht zu
ihnen: Wenn ein Mann einen Ausfluß hat, der von seinem Fleische fließt,
so ist er unrein. \bibleverse{3} Und zwar ist er unrein an diesem
Flusse, wenn sein Fleisch den Ausfluß frei fließen läßt; auch wenn sein
Fleisch verstopft wird von dem Ausflusse, so ist er unrein.
\bibleverse{4} Jedes Lager, worauf der mit einem Ausfluß Behaftete
liegt, und alles, worauf er sitzt, wird unrein; \bibleverse{5} und wer
sein Lager anrührt, soll seine Kleider waschen und sich mit Wasser baden
und unrein sein bis zum Abend; \bibleverse{6} und wer sich auf etwas
setzt, worauf der mit einem Ausfluß Behaftete gesessen hat, der soll
seine Kleider waschen und sich mit Wasser baden und unrein sein bis zum
Abend. \bibleverse{7} Wer sein Fleisch anrührt, der soll seine Kleider
waschen und sich mit Wasser baden und unrein sein bis zum Abend.
\bibleverse{8} Wenn aber der mit einem Ausfluß Behaftete seinen Speichel
auswirft auf einen, der rein ist, so soll dieser seine Kleider waschen
und sich mit Wasser baden und unrein sein bis zum Abend. \bibleverse{9}
Auch der Sattel und alles, worauf der mit einem Ausfluß Behaftete
reitet, wird unrein; \bibleverse{10} und wer immer etwas anrührt, das
unter ihm gewesen ist, der wird unrein sein bis zum Abend. Und wer etwas
solches trägt, der soll seine Kleider waschen und sich mit Wasser baden
und unrein sein bis zum Abend. \bibleverse{11} Und wen der mit einem
Ausfluß Behaftete anrührt, ohne daß er zuvor die Hände mit Wasser
gewaschen hat, der soll seine Kleider waschen und sich mit Wasser baden
und unrein sein bis zum Abend. \bibleverse{12} Wenn er ein irdenes
Geschirr anrührt, so soll man es zerbrechen; aber jedes hölzerne
Geschirr soll man mit Wasser waschen. \bibleverse{13} Und wenn er von
seinem Ausfluß rein wird, so soll er sieben Tage zählen zu seiner
Reinigung, und seine Kleider waschen und sein Fleisch mit lebendigem
Wasser baden; so ist er rein. \bibleverse{14} Und am achten Tage soll er
zwei Turteltauben oder zwei junge Tauben nehmen und vor den
\textsc{Herrn} kommen, vor die Tür der Stiftshütte, und soll sie dem
Priester geben. \bibleverse{15} Und der Priester soll die eine zum
Sündopfer, die andere zum Brandopfer machen und für ihn Sühne erwirken
vor dem \textsc{Herrn}, wegen seines Ausflusses. \bibleverse{16} Wenn
einem Mann der Same entgeht, so soll er sein ganzes Fleisch mit Wasser
baden und unrein sein bis zum Abend. \bibleverse{17} Und jedes Kleid und
jedes Fell, das mit solchem Samen befleckt ist, soll man mit Wasser
waschen, und es bleibt unrein bis zum Abend. \bibleverse{18} Und wenn
ein Mann bei einem Weibe liegt, daß ihm der Same entgeht, so sollen sie
sich mit Wasser baden und unrein sein bis zum Abend. \bibleverse{19}
Wenn ein Weib ihres Fleisches Blutfluß hat, so soll sie sieben Tage lang
in ihrer Unreinigkeit verbleiben. Wer sie anrührt, der bleibt unrein bis
zum Abend. \bibleverse{20} Und alles, worauf sie in ihrer Unreinigkeit
liegt, wird unrein; auch alles, worauf sie sitzt. \bibleverse{21} Und
wer ihr Lager anrührt, der soll seine Kleider waschen und sich mit
Wasser baden und bleibt unrein bis zum Abend. \bibleverse{22} Und wer
immer etwas anrührt, worauf sie gesessen hat, der soll seine Kleider
waschen und sich mit Wasser baden und unrein sein bis zum Abend.
\bibleverse{23} Auch wer etwas anrührt, das auf ihrem Bette war oder
worauf sie gesessen hat, soll unrein sein bis zum Abend. \bibleverse{24}
Und wenn ein Mann bei ihr liegt, und es kommt ihre Unreinigkeit an ihn,
der wird sieben Tage lang unrein sein, und das Lager, worauf er gelegen
hat, wird unrein sein. \bibleverse{25} Wenn aber ein Weib ihren Blutfluß
eine lange Zeit hat, nicht nur zur gewöhnlichen Zeit, sondern auch über
die gewöhnliche Zeit hinaus, so wird sie unrein sein während der ganzen
Dauer ihres Flusses; wie in den Tagen ihrer Unreinigkeit soll sie auch
dann unrein sein. \bibleverse{26} Alles, worauf sie liegt während der
ganzen Zeit ihres Flusses, soll sein wie das Lager ihrer monatlichen
Unreinigkeit; auch alles, worauf sie sitzt, wird unrein sein, gleich wie
zur Zeit ihrer monatlichen Unreinigkeit. \bibleverse{27} Wer etwas davon
anrührt, der wird unrein und soll seine Kleider waschen und sich mit
Wasser baden und unrein sein bis zum Abend. \bibleverse{28} Wird sie
aber rein von ihrem Flusse, so soll sie sieben Tage zählen, darnach soll
sie rein sein. \bibleverse{29} Und am achten Tage soll sie zwei
Turteltauben oder zwei junge Tauben nehmen und sie zum Priester bringen
vor die Tür der Stiftshütte. \bibleverse{30} Und der Priester soll die
eine zum Sündopfer, die andere zum Brandopfer machen und ihr wegen des
Flusses ihrer Unreinigkeit Sühne erwirken vor dem \textsc{Herrn}.
\bibleverse{31} Also sollt ihr die Kinder Israel absondern um ihrer
Unreinigkeit willen, damit sie in ihrer Unreinigkeit nicht sterben, wenn
sie meine Wohnung verunreinigen, die unter ihnen ist. \bibleverse{32}
Dies ist das Gesetz über den, der einen Ausfluß hat, und über den, dem
der Same entgeht, daß er unrein wird, \bibleverse{33} und über die,
welche an ihrer Unreinigkeit leidet, und über solche, die einen Fluß
haben, es sei ein Mann oder ein Weib, und über einen Mann, der bei einer
Unreinen liegt.

\hypertarget{section-15}{%
\section{16}\label{section-15}}

\bibleverse{1} Und der Herr redete mit Mose nach dem Tod der beiden
Söhne Aarons, als sie vor den \textsc{Herrn} traten. \bibleverse{2} Und
der \textsc{Herr} sprach zu Mose: Sage deinem Bruder Aaron, daß er nicht
zu allen Zeiten in das Heiligtum hineingehe hinter den Vorhang vor den
Sühndeckel, der auf der Lade ist, damit er nicht sterbe; denn ich will
auf dem Sühndeckel in einer Wolke erscheinen. \bibleverse{3} Damit soll
Aaron hineingehen in das Heiligtum: mit einem jungen Farren zum
Sündopfer und mit einem Widder zum Brandopfer; \bibleverse{4} und er
soll den heiligen leinenen Leibrock anziehen und soll ein leinenes
Unterkleid an seinem Leibe haben und sich mit einem leinenen Gürtel
gürten und einen leinenen Kopfbund umbinden (denn das sind die heiligen
Kleider) und soll seinen Leib mit Wasser baden und sie anziehen.
\bibleverse{5} Dann soll er von der Gemeinde der Kinder Israel zwei
Ziegenböcke nehmen zum Sündopfer und einen Widder zum Brandopfer.
\bibleverse{6} Und Aaron soll den Farren zum Sündopfer für sich selbst
herzubringen und sich und seinem Haus Sühne erwirken. \bibleverse{7}
Darnach soll er die beiden Böcke nehmen und sie vor den \textsc{Herrn}
stellen, vor die Tür der Stiftshütte, \bibleverse{8} und soll das Los
werfen über die beiden Böcke, ein Los für den \textsc{Herrn} und ein Los
für den Asasel. \bibleverse{9} Und Aaron soll den Bock, auf welchen des
\textsc{Herrn} Los fällt, zum Sündopfer machen. \bibleverse{10} Aber den
Bock, auf welchen das Los Asasels fällt, soll er lebendig vor den
\textsc{Herrn} stellen, daß er über ihm die Sühne vollziehe und ihn zum
Asasel in die Wüste jage. \bibleverse{11} Und Aaron soll den Farren des
Sündopfers, das für ihn selbst bestimmt ist, herzubringen und sich und
seinem Haus Sühne erwirken und soll den Farren schächten zum Sündopfer
für sich selbst. \bibleverse{12} Darnach nehme er die Pfanne voll Glut
vom Altar, der vor dem \textsc{Herrn} steht, und eine Handvoll
wohlriechenden zerstoßenen Räucherwerks und bringe es hinein hinter den
Vorhang; \bibleverse{13} und er tue das Räucherwerk auf das Feuer vor
dem \textsc{Herrn}, damit die Wolke vom Räucherwerk den Sühndeckel, der
auf dem Zeugnis ist, verhülle, damit er nicht sterbe. \bibleverse{14} Er
soll auch von dem Blut des Farren nehmen und mit seinem Finger gegen den
Sühndeckel sprengen, gegen Aufgang. Siebenmal soll er also vor dem
Sühndeckel mit seinem Finger vom Blute sprengen. \bibleverse{15} Darnach
soll er den Bock, das Sündopfer des Volkes schächten und von dessen Blut
hinein hinter den Vorhang bringen, und soll mit dessen Blute tun, wie er
mit des Farren Blut getan hat, und auch damit sprengen auf den
Sühndeckel und vor denselben. \bibleverse{16} Also soll er Sühne
erwirken für das Heiligtum wegen der Unreinigkeiten der Kinder Israel
und wegen ihrer Übertretungen und aller ihrer Sünden, und soll also tun
mit der Stiftshütte, welche sich mitten unter ihren Unreinigkeiten
befindet. \bibleverse{17} Kein Mensch soll in der Stiftshütte sein, wenn
er hineingeht, um im Heiligtum die Sühne zu vollziehen, bis er wieder
hinausgeht und die Sühne erwirkt hat für sich und sein Haus und die
ganze Gemeinde Israel. \bibleverse{18} Und wenn er zum Altar
herauskommt, der vor dem \textsc{Herrn} steht, so soll er von dem Blut
des Farren und von dem Blut des Bocks nehmen und auf die Hörner des
Altars tun, ringsum, \bibleverse{19} und soll mit seinem Finger vom Blut
siebenmal darauf sprengen und ihn reinigen und von der Unreinigkeit der
Kinder Israel heiligen. \bibleverse{20} Und wenn er die Sühne für das
Heiligtum und die Stiftshütte und den Altar erwirkt hat, so soll er den
lebendigen Bock herzu bringen, \bibleverse{21} und Aaron soll seine
beiden Hände auf dieses lebendigen Bockes Kopf stützen und auf ihn alle
Missetaten der Kinder Israel und alle ihre Übertretungen samt ihren
Sünden bekennen, und soll sie dem Bock auf den Kopf legen und ihn durch
einen Mann, der bereitsteht, in die Wüste jagen lassen; \bibleverse{22}
daß also der Bock alle ihre Missetaten auf sich in eine Wildnis trage;
und er soll ihn in der Wüste loslassen. \bibleverse{23} Und Aaron soll
in die Stiftshütte gehen und die leinenen Kleider ausziehen, die er
anzog, als er in das Heiligtum ging, und soll sie daselbst lassen,
\bibleverse{24} und soll seinen Leib mit Wasser baden an heiliger Stätte
und seine eigenen Kleider anziehen und hinausgehen und sein und des
Volkes Brandopfer verrichten, und Sühnung tun für sich und das Volk.
\bibleverse{25} Und das Fett des Sündopfers soll er auf dem Altar
verbrennen. \bibleverse{26} Der aber, welcher den Bock zum Asasel gejagt
hat, soll seine Kleider waschen und seinen Leib mit Wasser baden und
darnach in das Lager kommen. \bibleverse{27} Den Farren des Sündopfers
aber und den Bock des Sündopfers, deren Blut zur Sühnung in das
Heiligtum gebracht worden ist, soll man hinaus vor das Lager führen und
mit Feuer verbrennen, ihre Haut und ihr Fleisch und ihren Mist.
\bibleverse{28} Und der sie verbrannt hat, soll seine Kleider waschen
und seinen Leib mit Wasser baden und darnach in das Lager kommen.
\bibleverse{29} Und das soll euch eine ewig gültige Ordnung sein: Am
zehnten Tage des siebenten Monats sollt ihr eure Seelen demütigen und
kein Werk tun, weder der Einheimische noch der Fremdling, der unter euch
weilt. \bibleverse{30} Denn an diesem Tage wird für euch Sühne erwirkt,
euch zu reinigen; von allen euren Sünden sollt ihr vor dem
\textsc{Herrn} gereinigt werden. \bibleverse{31} Darum soll es euch ein
Ruhe-Sabbat sein, und ihr sollt eure Seelen demütigen. Das sei eine
ewige Ordnung. \bibleverse{32} Diese Sühne soll ein Priester vollziehen,
den man gesalbt und dessen Hand man gefüllt hat, daß er an seines Vaters
Statt Priester sei; und er soll die leinenen Kleider anziehen, die
heiligen Kleider, \bibleverse{33} und soll für das Heiligtum und die
Stiftshütte und den Altar Sühne erwirken; auch den Priestern und der
ganzen Volksgemeinde soll er Sühne schaffen. \bibleverse{34} Das soll
euch zur ewigen Gewohnheit weden, daß ihr für die Kinder Israel Sühne
erwirkt wegen allen ihren Sünden, einmal im Jahr. Und man tat, wie der
\textsc{Herr} Mose geboten hatte.

\hypertarget{section-16}{%
\section{17}\label{section-16}}

\bibleverse{1} Und der \textsc{Herr} redete zu Mose und sprach:
\bibleverse{2} Sage Aaron und seinen Söhnen und allen Kindern Israel und
sprich zu ihnen: Das ist's, was der \textsc{Herr} geboten hat, indem er
sprach: \bibleverse{3} Jedermann aus dem Hause Israel, der einen Ochsen,
oder ein Lamm, oder eine Ziege im Lager oder außerhalb des Lagers
schächtet, \bibleverse{4} und es nicht vor die Tür der Stiftshütte
bringt, daß es dem \textsc{Herrn} zum Opfer gebracht werde vor der
Wohnung des \textsc{Herrn}, dem soll es für eine Blutschuld gerechnet
werden; er hat Blut vergossen, und es soll derselbe Mensch aus seinem
Volk ausgerottet werden. \bibleverse{5} Darum sollen die Kinder Israel
fortan ihre Opfer, die sie jetzt noch auf freiem Felde opfern, vor den
\textsc{Herrn} bringen, vor die Tür der Stiftshütte, zum Priester, um
sie daselbst dem \textsc{Herrn} als Dankopfer darzubringen.
\bibleverse{6} Und der Priester soll das Blut auf den Altar des
\textsc{Herrn} sprengen vor der Tür der Stiftshütte und das Fett
verbrennen zum lieblichen Geruch dem \textsc{Herrn}. \bibleverse{7} Und
sie sollen forthin ihre Opfer nicht mehr den Dämonen opfern, denen sie
nachbuhlen. Das soll ihnen eine ewig gültige Ordnung sein auf alle ihre
Geschlechter. \bibleverse{8} Und du sollst zu ihnen sagen: Welcher
Mensch aus dem Hause Israel oder welcher Fremdling, der unter ihnen
wohnt, ein Brandopfer oder sonst ein Schlachtopfer verrichten will
\bibleverse{9} und es nicht vor die Türe der Stiftshütte bringt, daß er
es dem \textsc{Herrn} zurichte, der soll ausgerottet werden aus seinem
Volk. \bibleverse{10} Und wenn ein Mensch vom Hause Israel oder ein
Fremdling, der unter ihnen wohnt, irgend Blut ißt, wider einen solchen,
der Blut ißt, will ich mein Angesicht richten und ihn ausrotten aus
seinem Volk; \bibleverse{11} denn die Seele des Fleisches ist im Blut,
und ich habe es euch auf den Altar gegeben, um Sühne zu erwirken für
eure Seelen. Denn das Blut ist es, das Sühne erwirkt durch die in ihm
wohnende Seele. \bibleverse{12} Darum habe ich den Kindern Israel
gesagt: Keine Seele unter euch soll Blut essen; auch kein Fremdling
unter euch soll Blut essen. \bibleverse{13} Und wenn ein Mensch von den
Kindern Israel oder ein Fremdling, der unter ihnen wohnt, auf der Jagd
ein Wildpret oder Geflügel erwischt, das man essen darf, der soll
desselben Blut ausgießen und mit Erde bedecken; \bibleverse{14} denn
alles Fleisches Seele ist sein Blut; es ist mit seiner Seele verbunden.
Darum habe ich den Kindern Israel gesagt: Ihr sollt keines Fleisches
Blut essen; denn alles Fleisches Seele ist sein Blut. Wer es aber ißt,
der soll ausgerottet werden. \bibleverse{15} Jeder aber, der ein Aas
oder Zerrissenes genießt, er sei ein Einheimischer oder ein Fremdling,
der soll seine Kleider waschen und sich mit Wasser baden und unrein
bleiben bis zum Abend, dann wird er rein. \bibleverse{16} Wenn er aber
sein Kleid nicht waschen und sein Fleisch nicht baden wird, so soll er
seine Schuld tragen.

\hypertarget{section-17}{%
\section{18}\label{section-17}}

\bibleverse{1} Und der \textsc{Herr} redete zu Mose und sprach: Rede mit
den Kindern Israel und sprich zu ihnen: \bibleverse{2} Ich, der
\textsc{Herr}, bin euer Gott! \bibleverse{3} Ihr sollt nicht tun, wie
man im Lande Ägypten tut, wo ihr gewohnt habt, und sollt auch nicht tun,
wie man in Kanaan tut, dahin ich euch führen will, und ihr sollt nicht
nach ihren Satzungen wandeln; \bibleverse{4} sondern meine Rechte sollt
ihr halten und meine Satzungen beobachten, daß ihr darin wandelt; denn
ich bin der \textsc{Herr}, euer Gott. \bibleverse{5} Und zwar sollt ihr
meine Satzungen und meine Rechte beobachten, weil der Mensch, der sie
tut, dadurch leben wird. Ich bin der \textsc{Herr}! \bibleverse{6}
Niemand soll sich seiner Blutsverwandten nahen, ihre Scham zu entblößen;
ich bin der \textsc{Herr}! \bibleverse{7} Du sollst die Scham deines
Vaters und deiner Mutter nicht entblößen. Es ist deine Mutter, darum
sollst du ihre Scham nicht entblößen. \bibleverse{8} Du sollst die Scham
des Weibes deines Vaters nicht entblößen; denn es ist die Scham deines
Vaters. \bibleverse{9} Du sollst die Scham deiner Schwester, die deines
Vaters oder deiner Mutter Tochter ist, daheim oder draußen geboren,
nicht entblößen. \bibleverse{10} Die Scham der Tochter deines Sohns oder
deiner Tochter Tochter, ihre Scham sollst du nicht entblößen, denn es
ist deine Scham. \bibleverse{11} Du sollst die Scham der Tochter deines
Vaters Weibes, die deinem Vater geboren und deine Schwester ist, nicht
entblößen. \bibleverse{12} Du sollst die Scham der Schwester deines
Vaters nicht entblößen, denn sie ist deines Vaters nächste
Blutsverwandte. \bibleverse{13} Du sollst die Scham der Schwester deiner
Mutter nicht entblößen; denn sie ist deiner Mutter nächste
Blutsverwandte. \bibleverse{14} Du sollst die Scham des Bruders deines
Vaters nicht entblößen, du sollst nicht zu seinem Weibe gehen; denn sie
ist deine Base. \bibleverse{15} Du sollst die Scham deiner Sohnsfrau
nicht entblößen; denn sie ist deines Sohnes Weib, darum sollst du ihre
Scham nicht entblößen. \bibleverse{16} Du sollst die Scham des Weibes
deines Bruders nicht entblößen, denn es ist deines Bruders Scham.
\bibleverse{17} Du sollst nicht zugleich die Scham eines Weibes und
ihrer Tochter entblößen, noch ihres Sohnes Tochter oder ihrer Tochter
Tochter nehmen, ihre Scham zu entblößen; denn sie ist ihre nächste
Blutsverwandte; es wäre eine Schandtat. \bibleverse{18} Du sollst auch
nicht ein Weib zu ihrer Schwester hinzunehmen, wodurch Eifersucht erregt
würde, wenn du ihre Scham entblößtest, während jene noch lebt.
\bibleverse{19} Du sollst nicht zum Weibe gehen während ihrer
monatlichen Unreinigkeit, ihre Scham zu entblößen. \bibleverse{20} Auch
sollst du deines Nächsten Weib keinen Beischlaf gewähren, sie zu
besamen, daß du dich mit ihr verunreinigest. \bibleverse{21} Du sollst
auch von deinen Kindern keines hergeben, daß es dem Moloch geopfert
werde, damit du den Namen deines Gottes nicht entweihest; ich bin der
\textsc{Herr}! \bibleverse{22} Du sollst bei keiner Mannsperson liegen
wie beim Weib; denn das ist ein Greuel. \bibleverse{23} Auch sollst du
den Beischlaf mit keinem Vieh vollziehen, daß du dich mit ihm
verunreinigest. Und kein Weib soll sich zur Begattung vor ein Vieh
stellen; das wäre abscheulich! \bibleverse{24} Ihr sollt euch durch
nichts derartiges verunreinigen. Denn durch das alles haben sich die
Heiden verunreinigt, die ich vor euch her ausstoßen will.
\bibleverse{25} Und dadurch ist das Land verunreinigt worden. Darum will
ich ihre Missetat an ihm heimsuchen, daß das Land seine Einwohner
ausspeie. \bibleverse{26} Ihr aber sollt meine Satzungen und Rechte
beobachten und keinen dieser Greuel verüben, weder der Einheimische noch
der Fremdling, der unter euch wohnt; \bibleverse{27} denn alle diese
Greuel haben die Leute dieses Landes getan, die vor euch waren, wodurch
das Land verunreinigt worden ist. \bibleverse{28} Damit euch nun das
Land nicht ausspeie, wenn ihr es verunreiniget, wie es die Heiden
ausgespieen hat, die vor euch gewesen sind, \bibleverse{29} so soll
jeder, der einen dieser Greuel tut, jede Seele, die dergleichen verübt,
mitten aus ihrem Volk ausgerottet werden. \bibleverse{30} So beobachtet
denn meine Verordnungen, daß ihr keinen von den greulichen Gebräuchen
übet, die man vor euch geübt hat, und euch dadurch nicht verunreiniget.
Ich, der \textsc{Herr}, bin euer Gott!

\hypertarget{section-18}{%
\section{19}\label{section-18}}

\bibleverse{1} Und der \textsc{Herr} redete zu Mose und sprach:
\bibleverse{2} Rede mit der ganzen Gemeinde der Kinder Israel und sprich
zu ihnen: Ihr sollt heilig sein, denn Ich bin heilig, der \textsc{Herr},
euer Gott! \bibleverse{3} Jedermann fürchte seine Mutter und seinen
Vater und beobachte meine Sabbate; denn Ich, der \textsc{Herr}, bin euer
Gott. \bibleverse{4} Ihr sollt euch nicht an die Götzen wenden und sollt
euch keine gegossenen Götter machen, denn ich, der \textsc{Herr}, bin
euer Gott. \bibleverse{5} Und wenn ihr dem \textsc{Herrn} ein Dankopfer
schlachten wollt, sollt ihr's so opfern, daß es euch angenehm macht.
\bibleverse{6} Es soll aber gegessen werden an dem Tage, da ihr es
opfert, und am folgenden Tag; was aber bis zum dritten Tag übrigbleibt,
das soll man mit Feuer verbrennen. \bibleverse{7} Wird aber am dritten
Tage davon gegessen, so ist es ein Greuel und wird nicht angenehm sein;
\bibleverse{8} und wer davon ißt, wird seine Missetat tragen, weil er
das Heiligtum des \textsc{Herrn} entheiligt hat, und eine solche Seele
soll ausgerottet werden aus ihrem Volk. \bibleverse{9} Wenn ihr die
Ernte eures Landes einbringt, sollst du den Rand deines Ackers nicht
vollständig abernten und keine Nachlese nach deiner Ernte halten.
\bibleverse{10} Auch sollst du nicht Nachlese halten in deinem Weinberg,
noch die abgefallenen Beeren deines Weinberges auflesen, sondern du
sollst es den Armen und Fremdlingen lassen; denn ich, der \textsc{Herr},
bin euer Gott. \bibleverse{11} Ihr sollt einander nicht bestehlen, nicht
belügen noch betrügen! \bibleverse{12} Ihr sollt nicht falsch schwören
bei meinem Namen und nicht entheiligen den Namen deines Gottes! Denn Ich
bin der \textsc{Herr}. \bibleverse{13} Du sollst deinen Nächsten weder
bedrücken noch berauben. Des Taglöhners Lohn soll nicht über Nacht bei
dir bleiben bis zum Morgen. \bibleverse{14} Du sollst dem Tauben nicht
fluchen. Du sollst dem Blinden nichts in den Weg legen, sondern sollst
dich fürchten vor deinem Gott; denn Ich bin der \textsc{Herr}!
\bibleverse{15} Ihr sollt keine Ungerechtigkeit begehen im Gericht; du
sollst weder die Person des Geringen ansehen, noch die Person des Großen
ehren; sondern du sollst deinen Nächsten recht richten. \bibleverse{16}
Du sollst nicht als Verleumder umhergehen unter deinem Volk! Du sollst
auch nicht auftreten wider deines Nächsten Blut! \bibleverse{17} Ich bin
der \textsc{Herr}. Du sollst deinen Bruder nicht hassen in deinem
Herzen; strafen sollst du deinen Nächsten, daß du nicht seinethalben
Schuld tragen müssest! \bibleverse{18} Du sollst nicht Rache üben, noch
Groll behalten gegen die Kinder deines Volkes, sondern du sollst deinen
Nächsten lieben wie dich selbst! Denn ich bin der \textsc{Herr}.
\bibleverse{19} Meine Satzungen sollt ihr beobachten. Du sollst bei
deinem Vieh nicht zweierlei Arten sich begatten lassen und dein Feld
nicht besäen mit vermischtem Samen, und es soll kein Kleid auf deinen
Leib kommen, das von zweierlei Garn gewoben ist. \bibleverse{20} Wenn
ein Mann bei einem Weibe liegt und sie beschläft, die eine Dienstmagd
und einem andern versprochen, doch nicht losgekauft ist und die Freiheit
nicht erlangt hat, die sollen gestraft werden, aber sie sollen nicht
sterben; denn sie ist nicht frei gewesen. \bibleverse{21} Er soll aber
für seine Schuld dem \textsc{Herrn} vor die Tür der Stiftshütte einen
Widder zum Schuldopfer bringen. \bibleverse{22} Und der Priester soll
ihm Sühne erwirken mit dem Schuldopferwidder vor dem \textsc{Herrn}
wegen der Sünde, die er begangen hat; so wird ihm seine Sünde, die er
getan hat, vergeben werden. \bibleverse{23} Wenn ihr in das Land kommt
und allerlei Bäume pflanzet, wovon man ißt, sollt ihr die ersten Früchte
derselben als Vorhaut betrachten. Drei Jahre lang sollt ihr sie für
unbeschnitten achten und nicht davon essen. \bibleverse{24} Im vierten
Jahr aber sollen alle ihre Früchte heilig sein zu einer Jubelfeier für
den \textsc{Herrn}. \bibleverse{25} Und im fünften Jahre sollt ihr die
Früchte essen, daß der Ertrag umso größer werde; ich, der \textsc{Herr},
bin euer Gott. \bibleverse{26} Ihr sollt nichts mit Blut essen, ihr
sollt keine Wahrsagerei, keine Zeichendeuterei treiben. \bibleverse{27}
Ihr sollt den Rand eures Haupthaares nicht rundum stutzen, auch sollst
du den Rand deines Bartes nicht beschädigen. \bibleverse{28} Ihr sollt
keine Einschnitte an eurem Leibe machen für eine abgeschiedene Seele und
sollt euch nicht tätowieren! Ich bin der \textsc{Herr}. \bibleverse{29}
Du sollst deine Tochter nicht preisgeben, sie zur Unzucht anzuhalten,
damit das Land nicht Unzucht treibe und voller Laster werde!
\bibleverse{30} Beobachtet meine Sabbattage und verehret mein Heiligtum!
Ich bin der \textsc{Herr}. \bibleverse{31} Ihr sollt euch nicht an die
Totenbeschwörer wenden, noch an die Zeichendeuter; ihr sollt sie nicht
fragen, auf daß ihr durch sie nicht verunreinigt werdet; denn ich, der
\textsc{Herr}, bin euer Gott. \bibleverse{32} Vor einem grauen Haupte
sollst du aufstehen und alte Leute ehren und sollst dich fürchten vor
deinem Gott; ich bin der \textsc{Herr}. \bibleverse{33} Wenn ein
Fremdling bei dir in eurem Lande wohnen wird, so sollt ihr ihn nicht
beleidigen. \bibleverse{34} Ihr sollt euch gegen den Fremdling, der sich
bei euch aufhält, benehmen, als wäre er bei euch geboren, und du sollst
ihn lieben wie dich selbst; denn ihr seid auch Fremdlinge in Ägypten
gewesen. Ich, der \textsc{Herr}, bin euer Gott. \bibleverse{35} Ihr
sollt euch nicht vergreifen weder am Recht noch an der Elle, noch am
Gewicht, noch am Maß. \bibleverse{36} Rechte Waage, gutes Gewicht,
richtige Scheffel und rechte Eimer sollt ihr haben! Ich, der
\textsc{Herr}, bin euer Gott, der ich euch aus Ägypten geführt habe;
\bibleverse{37} darum sollt ihr alle meine Satzungen und alle meine
Rechte beobachten und tun; ich bin der \textsc{Herr}.

\hypertarget{section-19}{%
\section{20}\label{section-19}}

\bibleverse{1} Und der \textsc{Herr} redete zu Mose und sprach: Sage den
Kindern Israel: \bibleverse{2} Wer von den Kindern Israel oder den
Fremdlingen, die in Israel wohnen, von seinem Samen dem Moloch gibt, der
soll des Todes sterben; das Volk des Landes soll ihn steinigen!
\bibleverse{3} Und ich will mein Angesicht wider einen solchen Menschen
setzen und ihn mitten aus seinem Volk ausrotten, weil er dem Moloch von
seinem Samen gegeben und mein Heiligtum verunreinigt und meinen heiligen
Namen entheiligt hat. \bibleverse{4} Und wenn das Volk des Landes
nachsichtig wäre gegen einen solchen Menschen, der von seinem Samen dem
Moloch gegeben hat, daß es ihn nicht tötete, \bibleverse{5} so würde ich
mein Angesicht wider jenen Menschen und wider sein Geschlecht richten
und ihn und alle, die mit ihm dem Moloch nachgebuhlt haben, aus der
Mitte ihres Volkes ausrotten. \bibleverse{6} Auch wenn sich eine Seele
zu den Totenbeschwörern und Zeichendeutern wendet und ihnen nachbuhlt,
so will ich mein Angesicht wider diese Seele richten und sie aus der
Mitte ihres Volkes ausrotten. \bibleverse{7} Darum heiligt euch und seid
heilig; denn ich, der \textsc{Herr}, bin euer Gott! \bibleverse{8} Darum
beobachtet meine Satzungen und tut sie; denn ich, der \textsc{Herr}, bin
es, der euch heiligt. \bibleverse{9} Wer seinem Vater oder seiner Mutter
flucht, der soll unbedingt sterben! Sein Blut sei auf ihm; er hat seinem
Vater oder seiner Mutter geflucht. \bibleverse{10} Wenn einer die Ehe
bricht mit einem Eheweib, so sollen beide unbedingt sterben, der
Ehebrecher und die Ehebrecherin, weil er mit seines Nächsten Weib die
Ehe gebrochen hat. \bibleverse{11} Wer bei seines Vaters Weibe schläft,
der hat die Scham seines Vaters entblößt; sie sollen beide unbedingt
sterben; ihr Blut sei auf ihnen. \bibleverse{12} Wenn jemand bei seiner
Sohnsfrau schläft, so sollen sie beide unbedingt sterben; sie haben
einen Greuel begangen; ihr Blut sei auf ihnen! \bibleverse{13} Wenn ein
Mann bei einer männlichen Person schläft, als wäre es ein Weib, die
haben beide einen Greuel getan, und sie sollen unbedingt sterben; ihr
Blut sei auf ihnen! \bibleverse{14} Wenn jemand ein Weib nimmt und ihre
Mutter dazu, so ist das eine Schandtat; man soll ihn samt den beiden
Weibern mit Feuer verbrennen, damit keine solche Schandtat unter euch
sei. \bibleverse{15} Wenn ein Mann seinen Samen an ein Vieh abgibt, so
soll er unbedingt sterben, und das Vieh soll man erwürgen.
\bibleverse{16} Kommt ein Weib einem Vieh zu nahe, um sich mit ihm zu
vermischen, so sollst du sie töten und das Vieh auch; sie sollen
unbedingt sterben; ihr Blut sei auf ihnen! \bibleverse{17} Wenn jemand
seine Schwester nimmt, seines Vaters Tochter oder seiner Mutter Tochter,
und ihre Scham beschaut und sie wieder seine Scham, so ist das eine
Blutschande. Sie sollen ausgerottet werden vor den Augen ihres Volkes.
Er hat seiner Schwester Scham entblößt, so trage er seine Schuld.
\bibleverse{18} Wenn ein Mann bei einer Frau liegt zur Zeit ihrer
Krankheit und ihre Scham entblößt und ihren Brunnen aufdeckt, und sie
den Brunnen ihres Blutes entblößt, so sollen beide aus ihrem Volk
ausgerottet werden! \bibleverse{19} Die Scham deiner Mutter Schwester
und die Scham deines Vaters Schwester sollst du nicht entblößen; denn
wer dies tut, hat seine Blutsverwandten entblößt; sie sollen ihre Schuld
tragen! \bibleverse{20} Wenn jemand bei der Frau des Bruders seines
Vaters schläft, der hat die Scham seines Oheims entblößt; sie sollen
ihre Sünde tragen, sie sollen kinderlos sterben! \bibleverse{21} Wenn
jemand seines Bruders Weib nimmt, so ist das eine schändliche Tat; sie
sollen kinderlos bleiben, weil er die Scham seines Bruders entblößt hat.
\bibleverse{22} So beobachtet nun alle meine Satzungen und meine Rechte
und tut sie, damit euch nicht das Land ausspeie, wohin ich euch führe,
daß ihr darinnen wohnet! \bibleverse{23} Und wandelt nicht nach den
Satzungen der Heiden, die ich vor euch her ausstoßen werde. Denn solches
alles haben sie getan, daß mir vor ihnen ekelte. \bibleverse{24} Euch
aber habe ich gesagt: Ihr sollt ihr Land erblich besitzen; denn ich will
euch dasselbe zum Erbe geben, ein Land, das von Milch und Honig fließt.
Ich, der \textsc{Herr}, bin euer Gott, der ich euch von den Völkern
abgesondert habe. \bibleverse{25} So sollt nun auch ihr das reine Vieh
vom unreinen absondern und die unreinen Vögel von den reinen, und sollt
eure Seelen nicht verabscheuungswürdig machen durch Vieh, Vögel und
alles, was auf Erden kriecht, was ich euch als unrein abgesondert habe;
\bibleverse{26} sondern ihr sollt mir heilig sein, denn ich, der
\textsc{Herr}, bin heilig, der ich euch von den Völkern abgesondert
habe, daß ihr mir angehöret! \bibleverse{27} Wenn in einem Mann oder
einem Weib ein Totenbeschwörer oder Wahrsagergeist steckt, so sollen sie
unbedingt sterben. Man soll sie steinigen, ihr Blut sei auf ihnen!

\hypertarget{section-20}{%
\section{21}\label{section-20}}

\bibleverse{1} Und der \textsc{Herr} sprach zu Mose: Sage den Priestern,
Aarons Söhnen, und sprich zu ihnen: Ein Priester soll sich an keinem
Toten seines Volkes verunreinigen, \bibleverse{2} außer an seinem
nächsten Blutsverwandten, der ihm zugehört; an seiner Mutter, an seinem
Vater, an seinem Sohn, an seiner Tochter, an seinem Bruder,
\bibleverse{3} und an seiner Schwester, die noch eine Jungfrau ist, die
ihm nahesteht, weil sie noch keines Mannes Weib gewesen ist, an dieser
mag er sich verunreinigen. \bibleverse{4} Es soll sich der Vorgesetzte
an seinem Volk nicht verunreinigen, damit er sich nicht entweihe.
\bibleverse{5} Sie sollen sich keine Glatze scheren auf ihrem Haupt,
noch die Enden ihres Bartes stutzen, noch an ihrem Leibe Einschnitte
machen. \bibleverse{6} Sie sollen ihrem Gott heilig sein und den Namen
ihres Gottes nicht entheiligen; denn sie opfern des \textsc{Herrn}
Feueropfer, das Brot ihres Gottes, darum sollen sie heilig sein.
\bibleverse{7} Sie sollen keine Hure zum Weibe nehmen, auch keine
Entehrte, noch eine, die von ihrem Mann verstoßen ist; denn der Priester
ist heilig seinem Gott. \bibleverse{8} Darum sollst du ihn für heilig
halten; denn er opfert das Brot deines Gottes. Er soll dir heilig sein;
denn heilig bin ich, der \textsc{Herr}, der euch heiligt. \bibleverse{9}
Wenn eines Priesters Tochter sich durch Unzucht entweiht, so hat sie
ihren Vater entweiht; man soll sie mit Feuer verbrennen! \bibleverse{10}
Wer aber Hoherpriester ist unter seinen Brüdern, auf dessen Haupt das
Salböl gegossen worden, und dem man die Hand gefüllt hat bei der
Einkleidung, der soll sein Haupt nicht entblößen und seine Kleider nicht
zerreißen. \bibleverse{11} Er soll auch zu keinem Toten kommen und soll
sich weder an seinem Vater noch an seiner Mutter verunreinigen.
\bibleverse{12} Er soll das Heiligtum nicht verlassen, noch das
Heiligtum seines Gottes entheiligen; denn die Weihe des Salböls seines
Gottes ist auf ihm; ich bin der \textsc{Herr}. \bibleverse{13} Er soll
eine Jungfrau zum Weibe nehmen. \bibleverse{14} Eine Witwe, oder eine
Verstoßene, oder eine Entehrte, oder eine Hure soll er nicht nehmen;
sondern eine Jungfrau aus seinem Volk soll er zum Weibe nehmen,
\bibleverse{15} daß er seinen Samen nicht entweihe unter seinem Volk.
Denn ich, der \textsc{Herr}, heilige ihn. \bibleverse{16} Und der
\textsc{Herr} redete zu Mose und sprach: \bibleverse{17} Rede mit Aaron
und sprich: Sollte jemand von deinen Nachkommen in ihren künftigen
Geschlechtern mit irgend einem Gebrechen behaftet sein, so darf er sich
nicht herzunahen, das Brot seines Gottes darzubringen. \bibleverse{18}
Nein, keiner, an dem ein Gebrechen ist, soll sich herzunahen, er sei
blind oder lahm oder verstümmelt, oder habe ein zu langes Glied;
\bibleverse{19} auch keiner, der einen gebrochenen Fuß oder eine
gebrochene Hand hat, \bibleverse{20} oder der bucklig oder
schwindsüchtig ist, oder der einen Fleck auf seinem Auge hat, oder die
Krätze oder Flechten oder einen Hodenbruch. \bibleverse{21} Wer nun von
dem Samen Aarons, des Priesters, ein solches Gebrechen an sich hat, der
soll sich nicht herzunahen, die Feueropfer des \textsc{Herrn}
darzubringen; er hat ein Gebrechen; darum soll er das Brot seines Gottes
nicht herzubringen, daß er es opfere. \bibleverse{22} Doch darf er das
Brot seines Gottes essen, vom Heiligen und vom Allerheiligsten.
\bibleverse{23} Aber zum Vorhang soll er nicht kommen, noch sich dem
Altar nahen, weil er ein Gebrechen hat, daß er mein Heiligtum nicht
entweihe; denn ich, der \textsc{Herr}, heilige sie. \bibleverse{24} Und
Mose sagte es Aaron und seinen Söhnen und allen Kindern Israel.

\hypertarget{section-21}{%
\section{22}\label{section-21}}

\bibleverse{1} Und der \textsc{Herr} redete zu Mose und sprach: Sage
Aaron und seinen Söhnen, \bibleverse{2} daß sie sich enthalten sollen
der heiligen Gaben der Kinder Israel, und meinen heiligen Namen nicht
entweihen, in den Dingen, die sie mir geheiligt haben, mir, dem
\textsc{Herrn}. \bibleverse{3} So sage ihnen nun: Wer von euren
Nachkommen, der von eurem Samen ist, sich dem Heiligen naht, das die
Kinder Israel dem \textsc{Herrn} geheiligt haben, während er eine
Unreinigkeit an sich hat; eine solche Seele soll von meinem Angesicht
ausgerottet werden; ich bin der \textsc{Herr}! \bibleverse{4} Ist irgend
jemand vom Samen Aarons aussätzig oder mit einem Ausfluß behaftet, so
soll er von dem Heiligen nicht essen, bis er rein wird. Und wer etwas
durch einen Entseelten Verunreinigtes anrührt, oder wem der Same
entgeht, \bibleverse{5} oder wer irgend ein Gewürm anrührt, durch das
man unrein wird, oder einen Menschen, an dem man sich verunreinigen kann
wegen irgend etwas, was ihn unrein macht; \bibleverse{6} welche Seele
solches anrührt, die ist unrein bis zum Abend und soll nicht von dem
Heiligen essen, sondern soll zuvor ihren Leib mit Wasser baden.
\bibleverse{7} Und wenn die Sonne untergegangen und sie rein geworden
ist, dann mag sie von dem Heiligen essen; denn es ist ihre Speise.
\bibleverse{8} Kein Aas noch Zerrissenes soll er essen, daß er nicht
unrein davon werde; ich bin der \textsc{Herr}! \bibleverse{9} Sie sollen
meine Anordnungen beobachten, damit sie nicht Sünde auf sich laden und
daran sterben, wenn sie dieselben entheiligen; denn ich, der
\textsc{Herr}, heilige sie. \bibleverse{10} Kein Fremdling soll von dem
Heiligen essen. \bibleverse{11} Wenn aber der Priester eine Seele um
Geld erkauft, so mag dieselbe davon essen. Und wer ihm in seinem Hause
geboren wird, der mag auch von seinem Brot essen. \bibleverse{12} Wenn
aber des Priesters Tochter eines Fremdlings Weib wird, soll sie nicht
von dem Hebopfer des Heiligen essen. \bibleverse{13} Wird aber des
Priesters Tochter eine Witwe oder eine Verstoßene und hat keine Kinder
und kommt wieder in ihres Vaters Haus, wie in ihrer Jugend, so soll sie
von ihres Vaters Brot essen. Aber kein Fremdling soll davon essen.
\bibleverse{14} Wer sonst aber aus Versehen von dem Geheiligten ißt, der
soll den fünften Teil dazutun und es dem Priester mit dem Geheiligten
geben, \bibleverse{15} damit sie nicht die heiligen Gaben der Kinder
Israel entheiligen, welche diese dem \textsc{Herrn} heben,
\bibleverse{16} daß sie sich nicht mit Missetat und Schuld beladen, wenn
sie ihr Geheiligtes essen; denn ich, der \textsc{Herr}, heilige sie.
\bibleverse{17} Weiter redete der \textsc{Herr} zu Mose und sprach:
\bibleverse{18} Sage Aaron und seinen Söhnen und allen Kindern Israel
und sprich zu ihnen: Wer vom Hause Israel oder von den Fremdlingen in
Israel sein Opfer bringen will (sei es, daß sie es nach ihren Gelübden
oder ganz freiwillig dem \textsc{Herrn} zum Brandopfer darbringen
wollen), \bibleverse{19} der opfere, damit es euch angenehm mache, ein
tadelloses Männlein, von den Rindern, Lämmern oder Ziegen.
\bibleverse{20} Nichts Gebrechliches sollt ihr opfern; denn es würde
euch nicht angenehm machen. \bibleverse{21} Und wenn jemand dem
\textsc{Herrn} ein Dankopfer bringen will, sei es zur Erfüllung eines
Gelübdes oder als freiwillige Gabe, von Rindern oder Schafen, so soll es
tadellos sein, zum Wohlgefallen. Es soll keinen Mangel haben.
\bibleverse{22} Ein Blindes oder Gebrochenes oder Verwundetes oder
eines, das Geschwüre oder die Krätze oder Flechten hat, sollt ihr dem
\textsc{Herrn} nicht opfern und davon kein Feueropfer bringen auf den
Altar des \textsc{Herrn}. \bibleverse{23} Einen Ochsen, oder ein Schaf,
das zu lange oder zu kurze Glieder hat, magst du als freiwillige Gabe
opfern, aber zur Erfüllung eines Gelübdes wäre es nicht angenehm.
\bibleverse{24} Ihr sollt auch dem \textsc{Herrn} kein Tier darbringen,
welches verschnittene oder zerdrückte oder abgerissene oder
abgeschnittene Hoden hat, und sollt in eurem Lande solches gar nicht
tun. \bibleverse{25} Auch von der Hand eines Fremdlings sollt ihr deren
keines eurem Gott zur Speise darbringen; denn sie haben eine
Verstümmelung, einen Makel an sich; sie werden euch nicht gnädig
aufgenommen. \bibleverse{26} Und der \textsc{Herr} redete zu Mose und
sprach: \bibleverse{27} Wenn ein Rind oder ein Lamm, oder eine Ziege
geboren wird, so soll es sieben Tage lang bei seiner Mutter bleiben;
erst vom achten Tag an ist es angenehm zum Feueropfer für den
\textsc{Herrn}. \bibleverse{28} Ihr sollt aber kein Rind noch Schaf
zugleich mit seinem Jungen am gleichen Tag schächten. \bibleverse{29}
Wenn ihr aber dem \textsc{Herrn} ein Lobopfer darbringen wollt, so
opfert es zu eurer Begnadigung. \bibleverse{30} Ihr sollt es am gleichen
Tag essen und nichts übriglassen bis zum Morgen; ich bin der
\textsc{Herr}. \bibleverse{31} Ihr aber sollt meine Gebote beobachten
und sie tun; ich bin der \textsc{Herr}! \bibleverse{32} Und ihr sollt
meinen heiligen Namen nicht entheiligen; sondern ich will geheiligt
werden unter den Kindern Israel, ich, der \textsc{Herr}, der euch
heiligt; \bibleverse{33} der ich euch aus dem Lande Ägypten geführt
habe, um euer Gott zu sein, ich, der \textsc{Herr}.

\hypertarget{section-22}{%
\section{23}\label{section-22}}

\bibleverse{1} Und der \textsc{Herr} redete zu Mose und sprach: Sage den
Kindern Israel und sprich zu ihnen: \bibleverse{2} Das sind die Feste
des \textsc{Herrn}, da ihr heilige Festversammlungen einberufen sollt;
das sind meine Feste: \bibleverse{3} Sechs Tage lang soll man arbeiten,
aber am siebenten Tag ist die Sabbatfeier, eine heilige Versammlung; da
sollt ihr kein Werk tun; denn es ist der Sabbat des \textsc{Herrn}, in
allen euren Wohnorten. \bibleverse{4} Das sind aber die Feste des
\textsc{Herrn}, die heiligen Versammlungen, die ihr zu festgesetzten
Zeiten einberufen sollt: \bibleverse{5} Am vierzehnten Tag des ersten
Monats, gegen Abend, ist das Passah des \textsc{Herrn}. \bibleverse{6}
Und am fünfzehnten Tage desselben Monats ist das Fest der ungesäuerten
Brote des \textsc{Herrn}. Da sollt ihr sieben Tage lang ungesäuertes
Brot essen. \bibleverse{7} Am ersten Tag sollt ihr eine heilige
Versammlung halten; \bibleverse{8} da sollt ihr keine Werktagsarbeit
verrichten und ihr sollt dem \textsc{Herrn} sieben Tage lang Feueropfer
darbringen. Am siebenten Tag ist heilige Versammlung, da sollt ihr keine
Werktagsarbeit verrichten. \bibleverse{9} Und der \textsc{Herr} redete
zu Mose und sprach: \bibleverse{10} Sage den Kindern Israel und sprich
zu ihnen: Wenn ihr in das Land kommt, das ich euch geben werde, und
seine Ernte einheimset, so sollt ihr die Erstlingsgarbe von eurer Ernte
zum Priester bringen. \bibleverse{11} Der soll die Garbe weben vor dem
\textsc{Herrn}, zu eurer Begnadigung; am Tage nach dem Sabbat soll sie
der Priester weben. \bibleverse{12} Ihr sollt aber an dem Tage, wenn
eure Garbe gewebt wird, dem \textsc{Herrn} ein Brandopfer zurichten von
einem tadellosen einjährigen Lamm; \bibleverse{13} dazu sein Speisopfer,
zwei Zehntel Semmelmehl, mit Öl gemengt, ein Feueropfer dem
\textsc{Herrn} zum lieblichen Geruch; samt seinem Trankopfer, einem
Viertel Hin Wein. \bibleverse{14} Ihr sollt aber weder Brot noch
geröstetes Korn noch zerriebene Körner essen bis zu dem Tag, da ihr
eurem Gott diese Gabe darbringt. Das ist eine ewig gültige Ordnung für
alle eure Geschlechter. \bibleverse{15} Darnach sollt ihr vom Tage nach
dem Sabbat, von dem Tage, da ihr die Webegarbe darbringt, sieben volle
Wochen abzählen bis zum Tag, \bibleverse{16} der auf den siebenten
Sabbat folgt, nämlich fünfzig Tage sollt ihr zählen, und alsdann dem
\textsc{Herrn} ein neues Speisopfer darbringen. \bibleverse{17} Ihr
sollt nämlich aus euren Wohnsitzen zwei Webebrote bringen, von zwei
Zehntel Semmelmehl zubereitet; die sollen gesäuert und dem
\textsc{Herrn} zu Erstlingen gebacken werden. \bibleverse{18} Zu dem
Brot aber sollt ihr sieben einjährige, tadellose Lämmer darbringen und
einen jungen Farren und zwei Widder; das soll des \textsc{Herrn}
Brandopfer sein; dazu ihr Speisopfer und ihr Trankopfer; ein Feueropfer,
dem \textsc{Herrn} zum lieblichen Geruch. \bibleverse{19} Ihr sollt auch
einen Ziegenbock zum Sündopfer und zwei einjährige Lämmer zum Dankopfer
zurichten; \bibleverse{20} und der Priester soll sie samt den
Erstlingsbroten weben, nebst den beiden Lämmern, als Webopfer vor dem
\textsc{Herrn}. Die sollen dem \textsc{Herrn} heilig sein und dem
Priester gehören. \bibleverse{21} Und ihr sollt an demselben Tag
ausrufen lassen: ``Man soll eine heilige Versammlung abhalten und keine
Werktagsarbeit verrichten!'' Das ist eine ewig gültige Satzung für alle
eure Wohnorte und Geschlechter. \bibleverse{22} Wenn ihr aber die Ernte
eures Landes einbringt, so sollst du dein Feld nicht bis an den Rand
abernten und nicht selbst Nachlese halten, sondern es dem Armen und
Fremdling überlassen. Ich, der \textsc{Herr}, bin euer Gott.
\bibleverse{23} Und der \textsc{Herr} redete zu Mose und sprach:
\bibleverse{24} Rede mit den Kindern Israel und sprich: Am ersten Tag
des siebenten Monats sollt ihr einen Feiertag halten, einen
Gedächtnistag unter Posaunenklang, eine heilige Versammlung.
\bibleverse{25} Ihr sollt keine Werktagsarbeit verrichten, sondern dem
\textsc{Herrn} Feueropfer darbringen. \bibleverse{26} Und der
\textsc{Herr} redete zu Mose und sprach: \bibleverse{27} Am zehnten Tag
in diesem siebenten Monat ist der Versöhnungstag, da sollt ihr eine
heilige Versammlung halten und eure Seelen demütigen und dem
\textsc{Herrn} Feueropfer darbringen; \bibleverse{28} und ihr sollt an
diesem Tage keine Arbeit verrichten; denn es ist der Versöhnungstag, zu
eurer Versöhnung vor dem \textsc{Herrn}, eurem Gott. \bibleverse{29}
Welche Seele sich aber an diesem Tage nicht demütigt, die soll
ausgerottet werden aus ihrem Volk; \bibleverse{30} und welche Seele an
diesem Tag irgend eine Arbeit verrichtet, die will ich vertilgen mitten
aus ihrem Volk. \bibleverse{31} Ihr sollt keine Arbeit verrichten. Das
ist eine ewig gültige Ordnung für eure Geschlechter an allen euren
Wohnorten. \bibleverse{32} Ihr sollt Sabbatruhe halten und eure Seelen
demütigen. Am neunten Tage des Monats, am Abend, sollt ihr die Feier
beginnen, und sie soll währen von einem Abend bis zum andern.
\bibleverse{33} Und der \textsc{Herr} redete zu Mose und sprach:
\bibleverse{34} Rede mit den Kindern Israel und sprich: Am fünfzehnten
Tage des siebenten Monats soll dem \textsc{Herrn} das Laubhüttenfest
gefeiert werden, sieben Tage lang. \bibleverse{35} Am ersten Tage ist
heilige Versammlung; da sollt ihr keine Arbeit verrichten.
\bibleverse{36} Sieben Tage lang sollt ihr dem \textsc{Herrn} Feueropfer
darbringen und am achten Tag eine heilige Versammlung halten und dem
\textsc{Herrn} Feueropfer darbringen; es ist Festversammlung; da sollt
ihr keine Arbeit verrichten. \bibleverse{37} Das sind die Feste des
\textsc{Herrn}, da ihr heilige Versammlungen einberufen sollt, um dem
\textsc{Herrn} Feueropfer, Brandopfer, Speisopfer, Schlachtopfer und
Trankopfer darzubringen, ein jedes an seinem Tag \bibleverse{38} außer
den Sabbaten des \textsc{Herrn} und außer euren Gaben, den gelobten und
freiwilligen Gaben, die ihr dem \textsc{Herrn} gebet. \bibleverse{39} So
sollt ihr nun am fünfzehnten Tage des siebenten Monats, wenn ihr den
Ertrag des Landes eingebracht habt, das Fest des \textsc{Herrn} halten,
sieben Tage lang; am ersten Tag ist Feiertag und am achten Tag ist auch
Feiertag. \bibleverse{40} Ihr sollt aber am ersten Tag Früchte nehmen
von schönen Bäumen, Palmenzweige und Zweige von dichtbelaubten Bäumen
und Bachweiden, und sieben Tage lang fröhlich sein vor dem
\textsc{Herrn}, eurem Gott. \bibleverse{41} Und sollt also dem
\textsc{Herrn} das Fest halten, sieben Tage lang im Jahr. Das soll eine
ewige Ordnung sein für eure Geschlechter, daß ihr im siebenten Monat
also feiert. \bibleverse{42} Sieben Tage lang sollt ihr in Laubhütten
wohnen; alle Landeskinder in Israel sollen in Laubhütten wohnen,
\bibleverse{43} damit eure Nachkommen wissen, wie ich die Kinder Israel
in Hütten wohnen ließ, als ich sie aus Ägypten führte; ich, der Herr,
euer Gott. \bibleverse{44} Und Mose erklärte den Kindern Israel die
Feiertage des \textsc{Herrn}.

\hypertarget{section-23}{%
\section{24}\label{section-23}}

\bibleverse{1} Und der \textsc{Herr} redete zu Mose und sprach: Gebiete
den Kindern Israel, \bibleverse{2} daß sie zu dir bringen lauteres Öl
aus zerstossenen Oliven für den Leuchter, um beständig Licht zu
unterhalten! \bibleverse{3} Draußen vor dem Vorhang des Zeugnisses, in
der Stiftshütte, soll es Aaron zurichten, daß es stets brenne vor dem
\textsc{Herrn}, vom Abend bis zum Morgen; eine ewige Ordnung für eure
Geschlechter. \bibleverse{4} Auf dem reinen Leuchter soll er die Lampen
zurichten, vor dem \textsc{Herrn}, beständig. \bibleverse{5} Und du
sollst Semmelmehl nehmen und davon zwölf Kuchen backen; ein Kuchen soll
aus zwei Zehnteln bestehen. \bibleverse{6} Du sollst sie in zwei
Schichten von je sechs Stück auf den reinen Tisch legen vor den
\textsc{Herrn}. \bibleverse{7} Du sollst auf jede Schicht reinen
Weihrauch legen, damit dieser die Brote in Erinnerung bringe, wenn er
verbrannt wird vor dem \textsc{Herrn}. \bibleverse{8} Jeden Sabbat soll
er sie stets vor dem \textsc{Herrn} aufschichten als Gabe von den
Kindern Israel, laut ewigem Bund. \bibleverse{9} Und sie sollen Aaron
und seinen Söhnen gehören; die sollen sie essen an heiliger Stätte; denn
das ist ein hochheiliger, ewig festgesetzter Anteil für ihn von den
Feueropfern des \textsc{Herrn}. \bibleverse{10} Es ging aber der Sohn
eines israelitischen Weibes, der einen ägyptischen Vater hatte, unter
den Kindern Israel aus und ein. Dieser Sohn des israelitischen Weibes
und ein Israelite zankten im Lager miteinander. \bibleverse{11} Da
lästerte der Sohn des israelitischen Weibes den Namen Gottes und
fluchte. Darum brachte man ihn zu Mose. Seine Mutter aber hieß Selomit
und war die Tochter Dibris, vom Stamme Dan. \bibleverse{12} Und sie
behielten ihn in Haft, bis ihnen Bescheid würde durch den Mund des
\textsc{Herrn}. \bibleverse{13} Und der \textsc{Herr} redete zu Mose und
sprach: \bibleverse{14} Führe den Flucher vor das Lager hinaus und laß
alle, die es gehört haben, ihre Hand auf sein Haupt stützen, und die
ganze Gemeinde soll ihn steinigen. \bibleverse{15} Und sage den Kindern
Israel und sprich: Wer seinem Gott flucht, der soll seine Sünde tragen;
\bibleverse{16} und wer den Namen des \textsc{Herrn} lästert, der soll
unbedingt sterben! Die ganze Gemeinde soll ihn steinigen, er sei ein
Fremdling oder ein Einheimischer; wenn er den Namen lästert, so soll er
sterben! \bibleverse{17} Auch wenn jemand einen Menschen erschlägt, so
soll er unbedingt sterben. \bibleverse{18} Wer aber ein Vieh erschlägt,
der soll es bezahlen; Seele um Seele! \bibleverse{19} Bringt aber einer
seinem Nächsten eine Verletzung bei, so soll man ihm tun, wie er getan
hat: \bibleverse{20} Bruch um Bruch, Auge um Auge, Zahn um Zahn; die
Verletzung, die er dem andern zugefügt hat, soll man ihm auch zufügen;
\bibleverse{21} also daß, wer ein Vieh erschlägt, der soll es bezahlen;
wer aber einen Menschen erschlägt, der soll sterben. \bibleverse{22} Ihr
sollt ein einheitliches Recht haben für Fremdlinge und Einheimische;
denn ich, der \textsc{Herr}, bin euer Gott. \bibleverse{23} Mose aber
sagte solches den Kindern Israel; die führten den Flucher vor das Lager
hinaus und steinigten ihn. Also taten die Kinder Israel, wie der
\textsc{Herr} Mose geboten hatte.

\hypertarget{section-24}{%
\section{25}\label{section-24}}

\bibleverse{1} Und der \textsc{Herr} redete zu Mose auf dem Berge Sinai
und sprach: \bibleverse{2} Rede mit den Kindern Israel und sprich zu
ihnen: Wenn ihr in das Land kommt, das ich euch geben werde, so soll das
Land dem \textsc{Herrn} einen Sabbat feiern. \bibleverse{3} Sechs Jahre
lang sollst du dein Feld besäen und sechs Jahre lang deine Reben
beschneiden und ihre Früchte einsammeln. \bibleverse{4} Aber im
siebenten Jahr soll das Land seinen Ruhesabbat haben, den Sabbat des
\textsc{Herrn}, da du dein Feld nicht besäen, noch deine Reben
beschneiden sollst. \bibleverse{5} Auch was nach deiner Ernte von sich
selber wächst, sollst du nicht ernten; und die Trauben deines
unbeschnittenen Weinstocks sollst du nicht ablesen, weil es ein
Sabbatjahr des Landes ist. \bibleverse{6} Und dieser Landessabbat soll
euch Nahrung bringen, dir und deinen Knechten und deiner Magd, deinem
Taglöhner, deinen Beisaßen und deinem Fremdling bei dir; \bibleverse{7}
deinem Vieh und den Tieren in deinem Lande soll sein ganzer Ertrag zur
Speise dienen. \bibleverse{8} Und du sollst dir sieben solche
Sabbatjahre abzählen, daß siebenmal sieben Jahre gezählt werden, und die
Zeit der sieben Sabbatjahre beträgt neunundvierzig Jahre. \bibleverse{9}
Da sollst du den Schall der Posaune ertönen lassen am zehnten Tage des
siebenten Monats; am Tage der Versöhnung sollt ihr den Schall durch euer
ganzes Land ergehen lassen. \bibleverse{10} Und ihr sollt das fünfzigste
Jahr heiligen und sollt ein Freijahr ausrufen im Lande allen, die darin
wohnen, denn es ist das Jubeljahr. Da soll ein jeder bei euch wieder zu
seiner Habe und zu seinem Geschlecht kommen. \bibleverse{11} Denn das
fünfzigste ist das Jubeljahr. Ihr sollt nicht säen, auch nicht ernten,
was von sich selber wächst, auch den unbeschnittenen Weinstock nicht
ablesen. \bibleverse{12} Denn das Jubeljahr soll unter euch heilig sein;
vom Feld weg dürft ihr essen, was es trägt. \bibleverse{13} In diesem
Jubeljahr soll jedermann wieder zu seinem Besitztum kommen.
\bibleverse{14} Wenn du nun deinem Nächsten etwas verkaufst oder
demselben etwas abkaufst, so soll keiner seinen Bruder übervorteilen;
\bibleverse{15} sondern nach der Zahl der Jahre, nach dem Jubeljahr
sollst du es von ihm kaufen; und nach der Zahl des jährlichen Ertrages
soll er es dir verkaufen. \bibleverse{16} Nach der Menge der Jahre
sollst du den Kaufpreis steigern, und nach der geringen Anzahl der Jahre
sollst du den Kaufpreis verringern; denn eine bestimmte Anzahl von
Ernten verkauft er dir. \bibleverse{17} So übervorteile nun keiner
seinen Nächsten; sondern fürchte dich vor deinem Gott; denn ich, der
\textsc{Herr}, bin euer Gott! \bibleverse{18} Darum haltet meine
Satzungen und beobachtet meine Rechte, daß ihr sie tut; so sollst du
sicher wohnen im Lande, \bibleverse{19} und das Land soll euch seine
Früchte geben, daß ihr genug zu essen habt und sicher darin wohnt.
\bibleverse{20} Und wenn ihr sagen würdet: Was sollen wir im siebenten
Jahre essen? Denn wir säen nicht und sammeln auch keine Früchte ein!
\bibleverse{21} so will ich im sechsten Jahr meinem Segen gebieten, daß
es euch Früchte für drei Jahre liefern soll; \bibleverse{22} daß, wenn
ihr im achten Jahre säet, ihr noch von den alten Früchten esset bis in
das neunte Jahr; daß ihr von dem Alten esset bis wieder neue Früchte
kommen. \bibleverse{23} Ihr sollt das Land nicht als unablöslich
verkaufen; denn das Land ist mein, und ihr seid Fremdlinge und Beisaßen.
\bibleverse{24} Und ihr sollt im ganzen Lande eurer Besitzung die
Wiedereinlösung des Landes zulassen. \bibleverse{25} Wenn dein Bruder
verarmt und dir etwas von seiner Habe verkauft, so soll derjenige als
Löser für ihn eintreten, der sein nächster Verwandter ist; derselbe soll
lösen, was sein Bruder verkauft hat. \bibleverse{26} Wenn aber jemand
keinen Löser hat, kann aber mit seiner Hand so viel zuwege bringen, als
zur Wiedereinlösung nötig ist, \bibleverse{27} so soll er die Jahre, die
seit dem Verkauf verflossen sind, abrechnen und für den Rest den Käufer
entschädigen, damit er selbst wieder zu seiner Habe komme.
\bibleverse{28} Vermag er ihn aber nicht zu entschädigen, so soll das,
was er verkauft hat, in der Hand des Käufers bleiben bis zum Jubeljahr;
alsdann soll es frei ausgehen, und er soll wieder zu seiner Habe kommen.
\bibleverse{29} Wer ein Wohnhaus verkauft innerhalb der Stadtmauern, der
hat zur Wiedereinlösung Frist bis zur Vollendung des Verkaufsjahres. Ein
Jahr lang besteht für ihn das Rückkaufsrecht. \bibleverse{30} Wenn es
aber nicht gelöst wird bis zum Ablauf eines vollen Jahres, so sollen der
Käufer und seine Nachkommen dasselbe Haus innerhalb der Stadtmauern als
unablöslich behalten; es soll im Jubeljahr nicht frei ausgehen.
\bibleverse{31} Dagegen sind die Häuser in den Dörfern ohne Ringmauern
dem Ackerland gleich zu rechnen; sie sind ablösbar und sollen im
Jubeljahr frei ausgehen. \bibleverse{32} Was aber die Levitenstädte
anbetrifft, die Häuser in den Städten ihres Besitztums, so haben die
Leviten das ewige Einlösungsrecht. \bibleverse{33} Und wenn jemand etwas
von den Leviten erwirbt, so geht das verkaufte Haus in der Stadt seines
Besitztums im Jubeljahr frei aus; denn die Häuser in den Städten der
Leviten sind ihr Besitztum unter den Kindern Israel; \bibleverse{34} und
die Weideplätze bei ihren Städten dürfen nicht verkauft werden, denn sie
sind ihr ewiges Eigentum. \bibleverse{35} Wenn dein Bruder verarmt neben
dir und sich nicht mehr zu halten vermag, so sollst du ihm Hilfe
leisten, er sei ein Fremdling oder Beisaße, daß er bei dir leben kann.
\bibleverse{36} Du sollst keinen Zins noch Wucher von ihm nehmen,
sondern sollst dich fürchten vor deinem Gott, daß dein Bruder neben dir
leben könne. \bibleverse{37} Du sollst ihm dein Geld nicht auf Zins,
noch deine Speise um Wucherpreise geben. \bibleverse{38} Ich, der
\textsc{Herr}, bin euer Gott, der ich euch aus Ägyptenland geführt habe,
daß ich euch das Land Kanaan gebe und euer Gott sei. \bibleverse{39}
Wenn dein Bruder neben dir verarmt und dir sich selbst verkauft, sollst
du ihn im Dienst nicht als einen leibeigenen Knecht halten;
\bibleverse{40} als Taglöhner und Beisaße soll er bei dir gelten und dir
bis zum Jubeljahr dienen. \bibleverse{41} Alsdann soll er frei von dir
ausgehen, und seine Kinder mit ihm, und soll wieder zu seinem Geschlecht
und zu seiner Väter Habe kommen. \bibleverse{42} Denn auch sie sind
meine Knechte, die ich aus Ägyptenland geführt habe. Darum soll man sie
nicht wie Sklaven verkaufen! \bibleverse{43} Du sollst nicht mit Strenge
über ihn herrschen, sondern sollst dich fürchten vor deinem Gott.
\bibleverse{44} Willst du aber leibeigene Knechte und Mägde haben, so
sollst du sie kaufen von den Heiden, die um euch her sind.
\bibleverse{45} Ihr könnt sie auch kaufen von den Kindern der Beisaßen,
die sich bei euch aufhalten, und von ihren Geschlechtern bei euch, die
in eurem Lande geboren sind; dieselben sollt ihr zu eigen haben,
\bibleverse{46} und sollst sie vererben auf eure Kinder nach euch zum
leibeigenen Besitz, daß sie euch ewiglich dienen. Aber über eure Brüder,
die Kinder Israel, sollt ihr nicht, einer über den andern, mit Strenge
herrschen! \bibleverse{47} Wenn die Hand eines Fremdlings oder Beisaßen
bei dir etwas erwirbt, und dein Bruder neben ihm verarmt und sich dem
Fremdling, welcher ein Beisaße bei dir ist, oder einem Abkömmling von
seinem Stamm verkauft, \bibleverse{48} so soll er, nachdem er sich
verkauft hat, das Loskaufsrecht behalten; einer von seinen Brüdern soll
ihn lösen; \bibleverse{49} oder sein Vetter oder seines Vetters Sohn mag
ihn lösen, oder sonst sein nächster Blutsverwandter aus seinem
Geschlecht kann ihn lösen; oder wenn seine Hand so viel erwirbt, so soll
er sich selbst lösen. \bibleverse{50} Er soll aber mit seinem Käufer
rechnen von dem Jahr an, da er sich ihm verkauft hat, bis zum Jubeljahr.
Und der Preis seines Verkaufs soll nach der Zahl der Jahre berechnet
werden, und er soll diese Zeit wie ein Taglöhner bei ihm sein.
\bibleverse{51} Sind noch viele Jahre übrig, so soll er dementsprechend
von dem Kaufpreis als Lösegeld zurückerstatten; \bibleverse{52} sind
aber wenig Jahre übrig bis zum Jubeljahr, so soll er darauf Rücksicht
nehmen; nach der Zahl der Jahre soll er sein Lösegeld bezahlen.
\bibleverse{53} Wie ein Taglöhner soll er Jahr für Jahr bei ihm sein; er
aber soll nicht mit Strenge über ihn herrschen vor deinen Augen.
\bibleverse{54} Löst er sich aber nicht auf einem dieser Wege, so soll
er im Jubeljahr frei ausgehen und seine Kinder mit ihm; \bibleverse{55}
denn die Kinder Israel sind mir dienstbar; sie sind meine Knechte, die
ich aus Ägypten geführt habe, ich, der \textsc{Herr}, euer Gott.

\hypertarget{section-25}{%
\section{26}\label{section-25}}

\bibleverse{1} Ihr sollt keine Götzen machen, keine gemeißelten Bilder,
und sollt euch keine Säulen aufrichten, auch keine Steinbilder setzen in
eurem Lande, daß ihr euch davor bücket; denn ich, der \textsc{Herr}, bin
euer Gott. \bibleverse{2} Beobachtet meine Sabbate und verehret mein
Heiligtum; ich bin der \textsc{Herr}! \bibleverse{3} Werdet ihr nun in
meinen Satzungen wandeln und meine Gebote befolgen und sie tun,
\bibleverse{4} so will ich euch Regen geben zu seiner Zeit, und das Land
soll sein Gewächs geben und die Bäume auf dem Felde ihre Früchte
bringen. \bibleverse{5} Und die Dreschzeit wird reichen bis zur
Weinlese, und die Weinlese bis zur Saatzeit, und ihr werdet euch von
eurem Brot satt essen und sollt sicher wohnen in eurem Lande.
\bibleverse{6} Denn ich will Frieden geben im Lande, daß ihr schlafet
und euch niemand erschrecke. Ich will die bösen Tiere aus eurem Lande
vertreiben, und es soll kein Schwert über euer Land kommen.
\bibleverse{7} Ihr werdet eure Feinde jagen, daß sie vor euch her durchs
Schwert fallen. \bibleverse{8} Euer fünf werden hundert jagen, und euer
hundert werden zehntausend jagen, und eure Feinde werden vor euch her
durchs Schwert fallen. \bibleverse{9} Und ich will mich zu euch wenden
und euch wachsen und zunehmen lassen und meinen Bund mit euch
aufrechthalten. \bibleverse{10} Und ihr werdet von dem Vorjährigen essen
und das Vorjährige wegen der Menge des Neuen hinwegtun. \bibleverse{11}
Ich will meine Wohnung unter euch haben, und meine Seele soll euch nicht
verwerfen: \bibleverse{12} und ich will unter euch wandeln und euer Gott
sein, und ihr sollt mein Volk sein. \bibleverse{13} Ich, der
\textsc{Herr}, bin euer Gott, der ich euch aus Ägypten geführt habe, daß
ihr nicht ihre Knechte sein solltet; und ich zerbrach die Stäbe eures
Joches und ließ euch aufrecht gehen. \bibleverse{14} Werdet ihr mir aber
nicht folgen und nicht alle diese Gebote erfüllen, \bibleverse{15} und
werdet ihr meine Satzungen verachten, und wird eure Seele gegen meine
Rechte einen Widerwillen haben, daß ihr nicht alle meine Gebote tut,
sondern meinen Bund brechet, \bibleverse{16} so will auch ich euch
solches tun: Ich will euch heimsuchen mit Schrecken, Schwindsucht und
Fieberhitze, davon die Augen matt werden und die Seele verschmachtet.
Ihr werdet eure Saat vergeblich bestellen; denn eure Feinde sollen sie
essen. \bibleverse{17} Und ich will mein Angesicht gegen euch richten,
daß ihr vor euren Feinden geschlagen werdet; und die euch hassen, sollen
über euch herrschen, und ihr werdet fliehen, wenn euch niemand jagt.
\bibleverse{18} Werdet ihr mir aber daraufhin noch nicht gehorchen, so
will ich euch noch siebenmal ärger strafen um eurer Sünden willen,
\bibleverse{19} daß ich euren harten Stolz breche. Ich will euren Himmel
machen wie Eisen und eure Erde wie Erz, \bibleverse{20} daß eure Mühe
und Arbeit vergeblich aufgewendet sei, und euer Land sein Gewächs nicht
gebe und die Bäume des Landes ihre Früchte nicht bringen.
\bibleverse{21} Setzet ihr mir aber noch weitern Widerstand entgegen und
wollt mir nicht gehorchen, so will ich euch noch siebenmal mehr
schlagen, entsprechend euren Sünden. \bibleverse{22} Und ich will wilde
Tiere unter euch senden, die sollen euch euer Kinder berauben und euer
Vieh verderben und euer weniger machen, und eure Straßen sollen wüste
werden. \bibleverse{23} Werdet ihr euch aber dadurch noch nicht
züchtigen lassen, sondern mir trotzig begegnen, \bibleverse{24} so will
auch ich euch trotzig begegnen und euch siebenfältig schlagen um eurer
Sünden willen. \bibleverse{25} Und ich will ein Racheschwert über euch
kommen lassen, eine Bundesrache! Da werdet ihr euch in euren Städten
sammeln; ich aber will Pestilenz unter euch senden und euch in eurer
Feinde Hand geben. \bibleverse{26} Und ich werde euch den Stab des
Brotes zerbrechen, daß zehn Weiber euer Brot in einem Ofen backen mögen,
und man wird euch das Brot nach dem Gewicht zuteilen; und ihr werdet es
essen, aber nicht satt werden. \bibleverse{27} Werdet ihr aber auch
dadurch noch nicht zum Gehorsam gegen mich gebracht, sondern mir trotzig
begegnen, \bibleverse{28} so will ich auch euch mit grimmigem Trotz
begegnen und euch siebenfältig strafen um eurer Sünden willen,
\bibleverse{29} daß ihr eurer Söhne und Töchter Fleisch fressen müßt!
\bibleverse{30} Und ich will eure Höhen vertilgen und eure Sonnensäulen
abhauen und eure Leichname auf die Leichname eurer Götzen werfen, und
meine Seele wird euch verabscheuen. \bibleverse{31} Und ich will eure
Städte öde machen und eure heiligen Städte verwüsten und euren
lieblichen Geruch verabscheuen. \bibleverse{32} Also will ich das Land
wüste machen, daß eure Feinde, die darinnen wohnen werden, sich davor
entsetzen sollen. \bibleverse{33} Euch aber will ich unter die Heiden
zerstreuen und das Schwert hinter euch her ausziehen, daß euer Land zur
Wüste und eure Städte zu Ruinen werden. \bibleverse{34} Alsdann wird das
Land seine Sabbate genießen, solange es wüste liegt, und ihr in eurer
Feinde Land seid. Ja alsdann wird das Land feiern und seine Sabbate
genießen dürfen. \bibleverse{35} Solange es wüste liegt, wird es feiern,
weil es nicht feiern konnte an euren Sabbaten, als ihr darin wohntet.
\bibleverse{36} Denen aber, die von euch übrigbleiben, will ich das Herz
verzagt machen in ihrer Feinde Land, daß ein rauschendes Blatt sie jagen
wird; und sie werden davonfliehen, als jage sie ein Schwert, und fallen,
ohne daß sie jemand verfolgt. \bibleverse{37} Und sie sollen
übereinander fallen; wie vor dem Schwert, obschon sie niemand jagt; und
ihr werdet euren Feinden nicht widerstehen können, \bibleverse{38}
sondern werdet unter den Heiden umkommen, und eurer Feinde Land wird
euch fressen. \bibleverse{39} Welche von euch aber übrigbleiben, die
sollen ob ihrer Missetat verschmachten, in eurer Feinde Land; und ob der
Missetat ihrer Väter sollen sie verschmachten wie sie. \bibleverse{40}
Werden sie aber ihre und ihrer Väter Missetat bekennen samt ihrer
Übertretung, womit sie sich an mir vergriffen haben und mir trotzig
begegnet sind, \bibleverse{41} weswegen auch ich ihnen widerstand und
sie in ihrer Feinde Land brachte; und wird sich alsdann ihr
unbeschnittenes Herz demütigen, so daß sie dann ihre Schuld büßen,
\bibleverse{42} so will ich gedenken an meinen Bund mit Jakob und an
meinen Bund mit Isaak und an meinen Bund mit Abraham, und will des
Landes gedenken. \bibleverse{43} Aber das Land wird von ihnen verlassen
sein und seine Sabbate genießen, indem es um ihretwillen wüste liegt,
und sie werden ihre Schuld büßen, darum und deswegen, weil sie meine
Rechte verachtet und ihre Seele meine Satzungen verabscheut hat.
\bibleverse{44} Jedoch, wenn sie gleich in der Feinde Land sein werden,
so will ich sie nicht gar verwerfen und sie nicht also verabscheuen, daß
ich sie gar aufreibe oder meinen Bund mit ihnen breche; denn ich, der
\textsc{Herr}, bin ihr Gott. \bibleverse{45} Und ich will für sie an
meinen ersten Bund gedenken, als ich sie aus Ägypten führte vor den
Augen der Heiden, daß ich ihr Gott wäre, ich, der \textsc{Herr}.
\bibleverse{46} Das sind die Satzungen, die Rechte und Gesetze, die der
\textsc{Herr} auf dem Berge Sinai durch die Hand Moses gegeben hat, daß
sie zwischen ihm und den Kindern Israel bestehen sollten.

\hypertarget{section-26}{%
\section{27}\label{section-26}}

\bibleverse{1} Und der \textsc{Herr} redete zu Mose und sprach: Rede mit
den Kindern Israel und sprich zu ihnen: \bibleverse{2} Wenn jemand dem
\textsc{Herrn} ein besonderes Gelübde tut, wenn er nach deiner Schätzung
Seelen gelobt, \bibleverse{3} so sollst du sie also schätzen: Einen
Mann, vom zwanzigsten bis zum sechzigsten Jahr sollst du schätzen auf
fünfzig Schekel Silber, nach dem Schekel des Heiligtums. \bibleverse{4}
Ist es aber ein Weib, so sollst du sie auf dreißig Schekel schätzen.
\bibleverse{5} Im Alter von fünf bis zwanzig Jahren sollst du ihn
schätzen auf zwanzig Schekel, wenn es ein Knabe ist, aber auf zehn
Schekel, wenn es ein Mädchen ist. \bibleverse{6} Im Alter von einem
Monat bis zu fünf Jahren sollst du ihn schätzen auf fünf Schekel Silber,
wenn es ein Knabe ist, aber auf drei Schekel Silber, wenn es ein Mädchen
ist. \bibleverse{7} Im Alter von sechzig aber und darüber sollst du ihn
auf fünfzehn Schekel schätzen, wenn es ein Mann ist, auf zehn Schekel,
wenn es ein Weib ist. \bibleverse{8} Vermag er aber nicht soviel zu
bezahlen, wie du ihn schätzest, so soll er sich vor den Priester
stellen, und der Priester soll ihn schätzen nach dem Vermögen dessen,
der das Gelübde getan hat. \bibleverse{9} Ist es aber ein Vieh, von dem,
was man dem \textsc{Herrn} opfern kann, so soll jedes Stück, das man von
solchem Vieh dem \textsc{Herrn} gibt, heilig sein. \bibleverse{10} Man
soll es nicht auswechseln noch vertauschen, ein gutes für ein schlechtes
oder ein schlechtes für ein gutes; sollte es aber jemand auswechseln,
ein Vieh für das andere, so würde es samt dem zur Auswechslung
bestimmten Stück dem \textsc{Herrn} heilig sein. \bibleverse{11} Ist
aber das Tier unrein, daß man es dem \textsc{Herrn} nicht opfern darf,
so soll man es vor den Priester stellen; \bibleverse{12} und der
Priester soll es schätzen, je nachdem es gut oder schlecht ist; und bei
der Schätzung des Priesters soll es bleiben. \bibleverse{13} Will es
aber jemand lösen, so soll er den fünften Teil deiner Schätzung
dazugeben. \bibleverse{14} Wenn jemand sein Haus dem \textsc{Herrn} zum
Heiligtum weiht, so soll es der Priester schätzen, je nachdem es gut
oder schlecht ist; und wie es der Priester schätzt, so soll es gelten.
\bibleverse{15} Will es aber derjenige lösen, der es geheiligt hat, so
soll er den fünften Teil dazulegen; dann gehört es ihm. \bibleverse{16}
Wenn jemand dem \textsc{Herrn} ein Stück Feld von seinem Erbgut weiht,
so soll es von dir geschätzt werden nach dem Maß der Aussaat; der Raum
für die Aussaat von einem Homer Gerste soll fünfzig Schekel Silber
gelten. \bibleverse{17} Weiht er sein Feld vor dem Jubeljahr, so soll es
nach deiner Schatzung gelten. \bibleverse{18} Weiht er aber das Feld
nach dem Jubeljahr, so soll der Priester den Betrag berechnen nach den
übrigen Jahren bis zum nächsten Jubeljahr und es je nachdem geringer
schätzen. \bibleverse{19} Wenn aber der, welcher das Feld geweiht hat,
es lösen will, so soll er den fünften Teil über die Schatzungssumme
dazulegen, dann bleibt es sein. \bibleverse{20} Will er es aber nicht
lösen, sondern verkauft es einem andern, so kann es nicht mehr gelöst
werden; \bibleverse{21} sondern es soll dasselbige Feld, wenn es im
Jubeljahr frei ausgeht, dem \textsc{Herrn} heilig sein, wie ein mit dem
Bann belegtes Feld; es fällt dem Priester als Erbgut zu. \bibleverse{22}
Wenn aber jemand dem \textsc{Herrn} ein Stück Feld weiht, das er gekauft
hat und das nicht sein Erbgut ist, \bibleverse{23} so soll ihm der
Priester den Betrag nach deiner Schatzung berechnen bis zum Jubeljahr,
und er soll an demselben Tage den Schatzungswert geben, daß es dem
\textsc{Herrn} geweiht sei. \bibleverse{24} Aber im Jubeljahr soll das
Feld wieder an den Verkäufer zurückfallen, nämlich an den, welchem das
Land als Erbteil gehört. \bibleverse{25} Alle deine Schätzung aber soll
nach dem Schekel des Heiligtums geschehen. Ein Schekel macht zwanzig
Gera. \bibleverse{26} Doch soll niemand die Erstgeburt unter dem Vieh
weihen, die dem \textsc{Herrn} schon als Erstgeburt gehört, es sei ein
Ochs oder Schaf; es ist des \textsc{Herrn}. \bibleverse{27} Ist es aber
ein unreines Vieh, so soll man es lösen nach deiner Schätzung und den
fünften Teil darüber geben. Will man es nicht lösen, so soll es nach
deiner Schätzung verkauft werden. \bibleverse{28} Nur soll man kein mit
dem Bann Belegtes verkaufen oder lösen, nichts, das jemand dem
\textsc{Herrn} gebannt, von allem, was sein ist, es seien Menschen, Vieh
oder Äcker seines Besitztums; denn alles Gebannte ist dem \textsc{Herrn}
hochheilig! \bibleverse{29} Man soll auch keinen mit dem Bann belegten
Menschen lösen, sondern er soll unbedingt sterben! \bibleverse{30} Alle
Zehnten des Landes, sowohl von der Saat des Landes als auch von den
Früchten der Bäume, gehören dem \textsc{Herrn} und sollen dem
\textsc{Herrn} heilig sein. \bibleverse{31} Will aber jemand etwas von
seinem Zehnten lösen, der soll den fünften Teil darübergeben.
\bibleverse{32} Und alle Zehnten von Rindern und Schafen, von allem, was
unter dem Hirtenstab hindurchgeht, soll jedes zehnte Stück dem
\textsc{Herrn} heilig sein. \bibleverse{33} Man soll nicht untersuchen,
ob es gut oder schlecht sei, man soll es auch nicht auswechseln; sollte
es aber jemand auswechseln, so würde es samt dem zur Auswechslung
bestimmten Stück heilig sein und könnte nicht gelöst werden.
\bibleverse{34} Das sind die Gebote, die der \textsc{Herr} Mose befohlen
hat an die Kinder Israel, auf dem Berge Sinai.
