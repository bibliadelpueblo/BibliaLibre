\hypertarget{section}{%
\section{1}\label{section}}

\bibleverse{1} Als aber Ahab tot war, wurden die Moabiter von Israel
abtrünnig. Und Ahasia fiel in seinem Obergemach zu Samaria durch das
Gitter und ward krank; \bibleverse{2} und er sandte Boten und sprach zu
ihnen: Geht hin und befraget Baal-Sebub, den Gott zu Ekron, ob ich von
dieser Krankheit genesen werde! \bibleverse{3} Aber der Engel des
\textsc{Herrn} sprach zu Elia, dem Tisbiter: Mache dich auf und geh den
Boten des Königs von Samaria entgegen und sprich zu ihnen: Ist denn kein
Gott in Israel, daß ihr hingehet, Baal-Sebub, den Gott zu Ekron, zu
befragen? \bibleverse{4} Und darum spricht der \textsc{Herr} also: Du
sollst von dem Bette, darauf du dich gelegt hast, nicht herunterkommen,
sondern gewiß sterben! \bibleverse{5} Und Elia ging. Die Boten aber
kehrten wieder zum König zurück. Da fragte er sie: \bibleverse{6} Warum
kommt ihr wieder? Sie sprachen zu ihm: Ein Mann kam herauf, uns
entgegen, der sprach zu uns: Kehret wieder zurück zu dem König, der euch
gesandt hat, und saget zu ihm: So spricht der \textsc{Herr}: Ist denn
kein Gott in Israel, daß du hinsendest, Baal-Sebub, den Gott zu Ekron,
zu befragen? Darum sollst du von dem Bette, darauf du dich gelegt hast,
nicht herunterkommen, sondern gewiß sterben! \bibleverse{7} Er sprach zu
ihnen: Wie sah der Mann aus, der euch begegnete und solches zu euch
sagte? \bibleverse{8} Sie sprachen zu ihm: Der Mann trug einen härenen
Mantel und einen ledernen Gürtel um seine Lenden. Er aber sprach: Es ist
Elia, der Tisbiter! \bibleverse{9} Und er sandte einen Hauptmann über
fünfzig zu ihm, mit seinen fünfzig Leuten. Als der zu ihm hinaufkam,
siehe, da saß er oben auf dem Berge. Er aber sprach zu ihm: Du Mann
Gottes, \bibleverse{10} der König sagt, du sollst herabkommen! Aber Elia
antwortete dem Hauptmann über fünfzig und sprach zu ihm: Bin ich ein
Mann Gottes, so falle Feuer vom Himmel und verzehre dich und deine
Fünfzig! Da fiel Feuer vom Himmel und verzehrte ihn und seine Fünfzig.
\bibleverse{11} Und er sandte wieder einen andern Hauptmann über fünfzig
zu ihm mit seinen Fünfzigen, der antwortete und sprach zu ihm: Du Mann
Gottes, so spricht der König: Komm eilends herab! \bibleverse{12} Elia
antwortete und sprach zu ihm: Bin ich ein Mann Gottes, so falle Feuer
vom Himmel und verzehre dich und deine Fünfzig! Da fiel das Feuer Gottes
vom Himmel und verzehrte ihn und seine Fünfzig. \bibleverse{13} Da
sandte er noch einen dritten Hauptmann über fünfzig mit seinen
Fünfzigen. Als nun dieser dritte Hauptmann über fünfzig zu ihm
hinaufkam, beugte er seine Knie gegen Elia und bat ihn und sprach zu
ihm: Du Mann Gottes, laß doch mein Leben und das Leben deiner Knechte,
dieser Fünfzig, etwas vor dir gelten! \bibleverse{14} Siehe, das Feuer
ist vom Himmel gefallen und hat die ersten zwei Hauptleute über fünfzig
samt ihren Fünfzigen vertilgt. Nun aber laß mein Leben etwas vor dir
gelten! \bibleverse{15} Da sprach der Engel des \textsc{Herrn} zu Elia:
Gehe mit ihm hinab und fürchte dich nicht vor ihm! Und er machte sich
auf und ging mit ihm hinab zum König. \bibleverse{16} Und er sprach zu
ihm: So spricht der \textsc{Herr}: Weil du Boten hingesandt hast,
Baal-Sebub, den Gott zu Ekron, befragen zu lassen, als wäre kein Gott in
Israel, dessen Wort man befragen könnte, sollst du von dem Bette, darauf
du dich gelegt hast, nicht herunterkommen, sondern gewiß sterben!
\bibleverse{17} Also starb er, nach dem Worte des \textsc{Herrn}, das
Elia geredet hatte. Und Joram ward König an seiner Statt im zweiten Jahr
Jorams, des Sohnes Josaphats, des Königs von Juda; denn er hatte keinen
Sohn. \bibleverse{18} Was aber mehr von Ahasia zu sagen ist, das er
getan hat, ist das nicht beschrieben in der Chronik der Könige von
Israel?

\hypertarget{section-1}{%
\section{2}\label{section-1}}

\bibleverse{1} Als aber der \textsc{Herr} den Elia im Wetter gen Himmel
holen wollte, ging Elia mit Elisa von Gilgal hinweg. \bibleverse{2} Und
Elia sprach zu Elisa: Bleibe doch hier; der \textsc{Herr} hat mich gen
Bethel gesandt! Elisa aber sprach: So wahr der \textsc{Herr} lebt, und
so wahr deine Seele lebt, ich verlasse dich nicht! Also kamen sie hinab
gen Bethel. \bibleverse{3} Da gingen die Prophetensöhne, die zu Bethel
waren, heraus zu Elisa und sprachen zu ihm: Weißt du auch, daß der
\textsc{Herr} deinen Herrn heute über deinem Haupte hinwegnehmen wird?
Er aber sprach: Ich weiß es auch; schweigt nur still! \bibleverse{4} Und
Elia sprach zu ihm: Elisa, bleibe doch hier; denn der \textsc{Herr} hat
mich gen Jericho gesandt! Er aber sprach: So wahr der \textsc{Herr}
lebt, und so wahr deine Seele lebt, ich verlasse dich nicht! Also kamen
sie gen Jericho. \bibleverse{5} Da traten die Prophetensöhne, die zu
Jericho waren, zu Elisa und sprachen: Weißt du auch, daß der
\textsc{Herr} deinen Herrn heute über deinem Haupte hinwegnehmen wird?
Er aber sprach: Ich weiß es auch; schweigt nur still! \bibleverse{6} Und
Elia sprach zu ihm: Bleibe doch hier, denn der \textsc{Herr} hat mich an
den Jordan gesandt! Er aber sprach: So wahr der \textsc{Herr} lebt, und
so wahr deine Seele lebt, ich verlasse dich nicht! Und sie gingen beide
miteinander. \bibleverse{7} Und fünfzig Mann von den Prophetensöhnen
gingen hin und stellten sich abseits in einiger Entfernung auf, während
diese beiden am Jordan standen. \bibleverse{8} Da nahm Elia seinen
Mantel und wickelte ihn zusammen und schlug damit das Wasser; das teilte
sich nach beiden Seiten, so daß sie beide trocken hindurchgingen.
\bibleverse{9} Während sie aber hinübergingen, sprach Elia zu Elisa:
Erbitte, was ich dir tun soll, ehe ich von dir genommen werde! Elisa
sprach: Möchte mir doch ein zweifacher Anteil an deinem Geiste beschert
werden! \bibleverse{10} Er sprach: Du hast eine schwer zu erfüllende
Bitte getan: wirst du mich sehen, wenn ich von dir genommen werde, so
wird es geschehen, wo aber nicht, so wird es nicht sein! \bibleverse{11}
Und während sie noch miteinander gingen und redeten, siehe, da kam ein
feuriger Wagen mit feurigen Pferden und trennte beide voneinander. Und
Elia fuhr also im Wetter gen Himmel. \bibleverse{12} Elisa aber sah ihn
und schrie: Mein Vater! mein Vater! Wagen Israels und seine Reiter! Und
da er ihn nicht mehr sah, \bibleverse{13} faßte er seine Kleider und
zerriß sie in zwei Stücke und hob den Mantel auf, der Elia entfallen
war, und kehrte um und trat an das Gestade des Jordan. \bibleverse{14}
Darnach nahm er den Mantel, der Elia entfallen war, und schlug damit das
Wasser und sprach: Wo ist der \textsc{Herr}, der Gott des Elia? Und als
er so das Wasser schlug, teilte es sich nach beiden Seiten, und Elisa
ging hindurch. \bibleverse{15} Als aber die Prophetensöhne, die bei
Jericho ihm gegenüber standen, das sahen, sprachen sie: Der Geist des
Elia ruht auf Elisa! Und sie gingen ihm entgegen, \bibleverse{16}
bückten sich vor ihm zur Erde und sprachen zu ihm: Siehe doch, es sind
unter deinen Knechten fünfzig Männer, wackere Leute, laß dieselben gehen
und deinen Herrn suchen! Vielleicht hat ihn der Geist des \textsc{Herrn}
genommen und auf irgend einen Berg oder in irgend ein Tal geworfen? Er
aber sprach: Schicket sie nicht! \bibleverse{17} Aber sie drangen in
ihn, bis er ganz verlegen ward und sprach: So laßt sie gehen! Da sandten
sie fünfzig Männer, die suchten ihn drei Tage lang, aber sie fanden ihn
nicht. \bibleverse{18} Und als sie wieder zu ihm zurückkehrten, da er
noch zu Jericho war, sprach er zu ihnen: Habe ich euch nicht gesagt, ihr
solltet nicht hingehen? \bibleverse{19} Und die Männer der Stadt
sprachen zu Elisa: Siehe doch, in dieser Stadt ist gut wohnen, wie mein
Herr sieht; aber das Wasser ist schlecht, und das Land macht kinderlos!
\bibleverse{20} Er sprach: Bringt mir eine neue Schale und tut Salz
darein! Und sie brachten es ihm. \bibleverse{21} Da ging er hinaus zu
der Wasserquelle und warf das Salz hinein und sprach: So spricht der
\textsc{Herr}: Ich habe dieses Wasser gesund gemacht, es soll fortan
weder Tod noch Kinderlosigkeit daher kommen! \bibleverse{22} Also ward
das Wasser gesund bis auf diesen Tag nach dem Worte Elisas, das er
geredet hatte. \bibleverse{23} Und er ging von dannen hinauf nach
Bethel. Als er nun den Weg hinaufging, kamen kleine Knaben zur Stadt
hinaus; die verspotteten ihn und sprachen zu ihm: Kahlkopf, komm herauf!
Kahlkopf, komm herauf! \bibleverse{24} Da wandte er sich um, und da er
sie sah, fluchte er ihnen im Namen des \textsc{Herrn}. Da kamen zwei
Bären aus dem Walde und zerrissen zweiundvierzig Kinder. \bibleverse{25}
Von dort ging er auf den Berg Karmel und kehrte von da wieder nach
Samaria zurück.

\hypertarget{section-2}{%
\section{3}\label{section-2}}

\bibleverse{1} Und Joram, der Sohn Ahabs, ward König über Israel zu
Samaria, im achtzehnten Jahre Josaphats, des Königs von Juda, und
regierte zwölf Jahre lang. \bibleverse{2} Und er tat, was dem
\textsc{Herrn} übel gefiel, doch nicht wie sein Vater und seine Mutter;
denn er beseitigte die Säule Baals, welche sein Vater gemacht hatte.
\bibleverse{3} Aber er hielt fest an den Sünden, zu denen Jerobeam, der
Sohn Nebats, Israel verführt hatte, und ließ nicht davon. \bibleverse{4}
Mesa aber, der König der Moabiter, hatte viel Vieh und zinsete dem König
von Israel hunderttausend Lämmer und hunderttausend Widder samt der
Wolle. \bibleverse{5} Als aber Ahab tot war, fiel der König der Moabiter
von dem König von Israel ab. \bibleverse{6} Zu jener Zeit zog der König
Joram von Samaria aus und musterte ganz Israel; \bibleverse{7} und er
sandte zu Josaphat, dem König von Juda, und ließ ihm sagen: Der König
der Moabiter ist von mir abgefallen! Willst du mit mir kommen, wider die
Moabiter zu streiten? Er sprach: Ich will hinaufkommen! Ich bin wie du,
mein Volk ist wie dein Volk, und meine Pferde wie deine Pferde!
\bibleverse{8} Und er sprach: Auf welchem Wege wollen wir hinaufziehen?
Er sprach: Auf dem Wege durch die Wüste Edom! \bibleverse{9} Da zogen
aus der König von Israel, der König von Juda und der König von Edom. Als
sie aber einen Weg von sieben Tagen zurückgelegt hatten, hatte das Heer
und das Vieh, das ihnen folgte, kein Wasser mehr. \bibleverse{10} Da
sprach der König von Israel: Wehe! der \textsc{Herr} hat diese drei
Könige gerufen, um sie in die Hand der Moabiter zu geben!
\bibleverse{11} Josaphat aber sprach: Ist kein Prophet des
\textsc{Herrn} hier, daß wir durch ihn den \textsc{Herrn} um Rat fragen
könnten? Da antwortete einer von den Knechten des Königs von Israel und
sprach: Hier ist Elisa, der Sohn Saphats, der dem Elia Wasser auf die
Hände goß! \bibleverse{12} Josaphat sprach: Das Wort des \textsc{Herrn}
ist bei ihm! Also zogen der König von Israel und Josaphat und der König
von Edom zu ihm hinab. \bibleverse{13} Elisa aber sprach zum König von
Israel: Was habe ich mit dir zu schaffen? Gehe hin zu den Propheten
deines Vaters und zu den Propheten deiner Mutter! Der König von Israel
sprach zu ihm: Nein! Hat der \textsc{Herr} nicht diese drei Könige
gerufen, um sie in die Hand der Moabiter zu geben? \bibleverse{14} Elisa
sprach: So wahr der \textsc{Herr} der Heerscharen lebt, vor dessen
Angesicht ich stehe, wenn ich nicht auf Josaphat, den König von Juda,
Rücksicht nähme, ich wollte dich nicht ansehen noch beachten!
\bibleverse{15} So bringt mir nun einen Saitenspieler! Und als der
Saitenspieler die Saiten schlug, kam die Hand des \textsc{Herrn} über
ihn. \bibleverse{16} Und er sprach: So spricht der \textsc{Herr}: Machet
im Tale Grube an Grube! \bibleverse{17} Denn also spricht der
\textsc{Herr}: Ihr werdet keinen Wind noch Regen sehen; dennoch soll
dieses Tal voll Wasser werden, so daß ihr zu trinken habt,
\bibleverse{18} auch euer kleines und großes Vieh. Und zwar ist das ein
Geringes vor dem \textsc{Herrn}; er wird auch die Moabiter in eure Hand
geben, \bibleverse{19} so daß ihr alle festen Städte und alle
auserlesenen Städte schlagen werdet; und ihr werdet alle guten Bäume
fällen und alle Wasserbrunnen verstopfen und alle guten Äcker mit
Steinen verderben. \bibleverse{20} Am Morgen nun, zur Zeit des
Speisopfers, siehe, da kam ein Gewässer den Weg von Edom her, und das
Land wurde voll Wasser. \bibleverse{21} Als aber ganz Moab hörte, daß
die Könige heraufzogen, wider sie zu streiten, wurden alle, die das
Schwert umgürten konnten, aufgeboten; und sie besetzten die Grenze.
\bibleverse{22} Und als sie sich am Morgen früh aufmachten und die Sonne
über dem Wasser aufging, erschien den Moabitern das Wasser drüben rot
wie Blut. \bibleverse{23} Und sie sprachen: Es ist Blut! Die Könige
haben sich gewiß mit dem Schwerte bekämpft, und einer wird den andern
erschlagen haben! Und nun, Moab, mache dich auf zur Plünderung!
\bibleverse{24} Als sie aber zum Lager Israels kamen, machten sich die
Israeliten auf und schlugen die Moabiter, daß sie vor ihnen flohen.
\bibleverse{25} Jene aber drangen ins Land ein und schlugen Moab und
rissen die Städte nieder und warfen ein jeder seinen Stein auf alle
guten Äcker, bis sie voll waren, und verstopften alle Wasserbrunnen und
fällten alle guten Bäume, bis in Kir-Hareset nur noch dessen Steinmauern
übrigblieben. Und die Schleuderer umzingelten und beschossen es.
\bibleverse{26} Als aber der König der Moabiter sah, daß ihm der Streit
zu stark ward, nahm er siebenhundert Mann mit sich, die das Schwert
zogen, um gegen den König von Edom durchzubrechen; aber sie konnten
nicht. \bibleverse{27} Da nahm er seinen erstgeborenen Sohn, der an
seiner Statt König werden sollte, und opferte ihn zum Brandopfer auf der
Mauer. Und es entstand großer Unwille wider Israel, so daß sie von ihm
abzogen und wieder in ihr Land zurückkehrten.

\hypertarget{section-3}{%
\section{4}\label{section-3}}

\bibleverse{1} Und eine Frau unter den Frauen der Prophetensöhne schrie
zu Elisa und sprach: Dein Knecht, mein Mann, ist gestorben; aber du
weißt, daß er, dein Knecht, den \textsc{Herrn} fürchtete. Nun kommt der
Gläubiger und will sich meine beiden Söhne zu Knechten nehmen!
\bibleverse{2} Elisa sprach zu ihr: Was soll ich für dich tun? Sage mir,
was hast du im Hause? Sie sprach: Deine Magd hat nichts im Hause als
einen Krug mit Öl! \bibleverse{3} Er sprach: Gehe hin und erbitte dir
draußen Gefäße von allen deinen Nachbarinnen, leere Gefäße, und
derselben nicht wenige; \bibleverse{4} und gehe hinein und schließe die
Tür hinter dir und deinen Söhnen zu und gieße in alle diese Gefäße; und
was voll ist, trage weg! \bibleverse{5} Sie ging von ihm und schloß die
Tür hinter sich und ihren Söhnen zu; die brachten ihr die Gefäße, und
sie goß ein. \bibleverse{6} Und als die Gefäße voll waren, sprach sie zu
ihrem Sohn: Reiche mir noch ein Gefäß her! Er sprach zu ihr: Es ist kein
Gefäß mehr hier! Da stockte das Öl. \bibleverse{7} Und sie ging hin und
sagte es dem Manne Gottes. Er sprach: Gehe hin, verkaufe das Öl und
bezahle deine Schuld; du aber und deine Söhne möget von dem Übrigen
leben! \bibleverse{8} Und es begab sich eines Tages, daß Elisa nach
Sunem ging. Dort wohnte eine vornehme Frau, die nötigte ihn, bei ihr zu
essen. Sooft er nun daselbst durchzog, kehrte er dort ein, um zu essen.
\bibleverse{9} Und sie sprach zu ihrem Mann: Siehe doch, ich merke, daß
dies ein heiliger Mann Gottes ist, der stets bei uns vorbeikommt.
\bibleverse{10} Laß uns doch eine kleine Dachstube herrichten und Bett,
Tisch, Stuhl und Leuchter hineinstellen, damit, wenn er zu uns kommt, er
sich dahin verfüge! \bibleverse{11} Es begab sich nun eines Tages, daß
er hineinkam und sich in die Dachstube verfügte und darin schlief.
\bibleverse{12} Dann sprach er zu seinem Burschen Gehasi: Rufe diese
Sunamitin! Da rief er sie, und sie trat vor ihn hin. \bibleverse{13} Und
er sprach zu ihm: Sage ihr doch: Siehe, du hast unsertwegen so viel
Sorge gehabt; was kann ich für dich tun? Hast du etwas, weswegen ich mit
dem König oder mit dem Feldhauptmann für dich reden sollte? Sie sprach:
Ich wohne ja mitten unter meinem Volk! \bibleverse{14} Er sprach: Was
könnte man für sie tun? Gehasi sprach: Ach, sie hat keinen Sohn, und ihr
Mann ist alt! \bibleverse{15} Da sagte er: Rufe sie! Und als er sie
rief, trat sie unter die Tür. \bibleverse{16} Und er sprach: Um diese
bestimmte Zeit übers Jahr wirst du einen Sohn herzen! Sie sprach: Ach
nein, mein Herr, du Mann Gottes, spotte deiner Magd nicht!
\bibleverse{17} Aber das Weib empfing und gebar einen Sohn um dieselbe
Zeit, im nächsten Jahre, wie Elisa ihr verheißen hatte. \bibleverse{18}
Als aber der Knabe heranwuchs, begab es sich eines Tages, daß er zu
seinem Vater, zu den Schnittern hinausging. \bibleverse{19} Da sprach er
zu seinem Vater: O mein Kopf, mein Kopf! Jener befahl einem Knecht:
Führe ihn zu seiner Mutter! \bibleverse{20} Der nahm ihn und brachte ihn
zu seiner Mutter. Und er saß auf ihrem Schoße bis zum Mittag, dann starb
er. \bibleverse{21} Da ging sie hinauf und legte ihn auf das Bett des
Mannes Gottes, schloß hinter ihm zu und ging hinaus, \bibleverse{22}
rief ihren Mann und sprach: Sende mir doch einen von den Knechten und
eine Eselin, ich will eilends zu dem Manne Gottes gehen, aber bald
wiederkommen! \bibleverse{23} Er sprach: Warum gehst du heute zu ihm? Es
ist doch weder Neumond noch Sabbat! Sie sprach: Lebe wohl!
\bibleverse{24} Und sie sattelte die Eselin und sprach zu ihrem Knechte:
Treibe das Tier beständig an und mache keinen Aufenthalt, es sei denn,
daß ich es sage! \bibleverse{25} So ging sie denn und kam zu dem Manne
Gottes auf den Berg Karmel. Als aber der Mann Gottes sie aus einiger
Entfernung sah, sprach er zu seinem Diener Gehasi: Sieh dort die
Sunamitin! \bibleverse{26} Nun laufe ihr doch entgegen und sprich zu
ihr: Geht es dir wohl? Geht es deinem Manne wohl? Sie sprach: Jawohl!
\bibleverse{27} Als sie aber zu dem Manne Gottes kam, umfaßte sie seine
Füße; da machte sich Gehasi herzu, um sie wegzustoßen. Aber der Mann
Gottes sprach: Laß sie, denn ihre Seele ist betrübt, und der
\textsc{Herr} hat es mir verborgen und nicht kundgetan! \bibleverse{28}
Sie sprach: Habe ich denn von meinem Herrn einen Sohn erbeten? Sagte ich
nicht, du solltest meiner nicht spotten? \bibleverse{29} Er sprach zu
Gehasi: Gürte deine Lenden und nimm einen Stab in deine Hand und gehe
hin! Wenn dir jemand begegnet, so grüße ihn nicht, und grüßt dich
jemand, so antworte ihm nicht, und lege meinen Stab auf des Knaben
Angesicht! \bibleverse{30} Aber die Mutter des Knaben sprach: So wahr
der \textsc{Herr} lebt, und so wahr deine Seele lebt, ich lasse nicht
von dir! Da machte er sich auf und folgte ihr. \bibleverse{31} Gehasi
aber ging vor ihnen hin und legte dem Knaben den Stab auf das Angesicht;
aber da war keine Stimme noch Aufmerken. Und er kehrte um, ihm entgegen,
und zeigte es ihm an und sprach: Der Knabe ist nicht aufgewacht!
\bibleverse{32} Als nun Elisa in das Haus kam, siehe, da lag der Knabe
tot auf seinem Bett. \bibleverse{33} Und er ging hinein und schloß die
Tür hinter ihnen beiden zu und betete zu dem \textsc{Herrn}.
\bibleverse{34} Dann stieg er hinauf und legte sich auf das Kind und
legte seinen Mund auf des Kindes Mund und seine Augen auf desselben
Augen und seine Hände auf desselben Hände und breitete sich also über
dasselbe, daß des Kindes Leib warm wurde. \bibleverse{35} Darnach stand
er auf und ging im Hause einmal hierhin, einmal dorthin, stieg dann
wieder hinauf und breitete sich über ihn. Da nieste der Knabe siebenmal;
darnach tat der Knabe die Augen auf. \bibleverse{36} Und er rief Gehasi
und sprach: Rufe die Sunamitin! Da rief er sie, und als sie zu ihm
hereinkam, sprach er: Da nimm deinen Sohn! \bibleverse{37} Da kam sie
und fiel zu seinen Füßen und bückte sich zur Erde und nahm ihren Sohn
und ging hinaus. \bibleverse{38} Elisa aber kam wieder nach Gilgal. Und
es war eine Hungersnot im Lande. Und die Prophetensöhne saßen vor ihm,
und er sprach zu seinem Burschen: Setze den großen Topf auf und koche
ein Gemüse für die Prophetensöhne! \bibleverse{39} Da ging einer aufs
Feld hinaus, um Kräuter zu suchen, und er fand wilde Gurken und las
davon sein Kleid voll wilde Gurken; und als er heimkam, zerschnitt er
sie in den Gemüsetopf; denn sie kannten sie nicht. \bibleverse{40} Als
man es aber zum Essen vor die Männer ausschüttete und sie von dem Gemüse
aßen, schrieen sie und sprachen: Der Tod ist im Topf, Mann Gottes! Und
sie konnten es nicht essen. \bibleverse{41} Er aber sprach: Gebt Mehl!
Und er warf es in den Topf und sprach: Schütte es aus für die Leute, daß
sie essen! Da war nichts Böses mehr im Topf. \bibleverse{42} Aber ein
Mann von Baal-Schalischa kam und brachte dem Manne Gottes
Erstlingsbrote, zwanzig Gerstenbrote und zerriebene Körner in seinem
Sack. Er aber sprach: Gib es dem Volk, daß sie essen! \bibleverse{43}
Sein Diener sprach: Wie kann ich das hundert Männern vorsetzen? Er aber
sprach: Gib es dem Volk, daß sie essen! Denn also spricht der
\textsc{Herr}: Man wird essen, und es wird übrigbleiben! \bibleverse{44}
Und er legte es ihnen vor, und sie aßen; und es blieb noch übrig, nach
dem Worte des \textsc{Herrn}.

\hypertarget{section-4}{%
\section{5}\label{section-4}}

\bibleverse{1} Naeman, der Feldhauptmann des Königs von Syrien, war ein
geschätzter Mann vor seinem Herrn und hochangesehen; denn durch ihn gab
der \textsc{Herr} den Syrern Heil. Aber dieser gewaltige, tapfere Mann
war aussätzig. \bibleverse{2} Und die Syrer waren in Streifscharen
ausgezogen und hatten ein kleines Mägdlein aus dem Lande Israel
entführt, das nun im Dienste von Naemans Frau war. \bibleverse{3} Und es
sprach zu seiner Herrin: Ach, daß mein Herr bei dem Propheten zu Samaria
wäre; der würde ihn von seinem Aussatz befreien! \bibleverse{4} Da ging
Naeman hinein zu seinem Herrn und sagte es ihm und sprach: So und so hat
das Mägdlein aus dem Lande Israel geredet! \bibleverse{5} Da sprach der
König von Syrien: Gehe hin, ich will dem König von Israel einen Brief
schicken! Da ging er und nahm zehn Talente Silber und sechstausend
Goldstücke und zehn Feierkleider mit sich. \bibleverse{6} Und er brachte
dem König von Israel den Brief; darin stand: «Und nun, wenn dieser Brief
zu dir kommt, so siehe: ich habe meinen Knecht Naeman zu dir gesandt,
damit du ihn von seinem Aussatz befreiest!'' \bibleverse{7} Als nun der
König von Israel den Brief gelesen hatte, zerriß er seine Kleider und
sprach: Bin ich denn Gott, daß ich töten und lebendig machen kann, daß
man von mir verlangt, ich solle einen Mann von seinem Aussatz befreien?
Da seht doch, daß der einen Anlaß zum Streit mit mir sucht!
\bibleverse{8} Als aber Elisa, der Mann Gottes, hörte, daß der König
seine Kleider zerrissen habe, sandte er zum König und ließ ihm sagen:
Warum hast du deine Kleider zerrissen? Er soll zu mir kommen, so wird er
innewerden, daß ein Prophet in Israel ist! \bibleverse{9} Also kam
Naeman mit seinen Pferden und mit seinen Wagen und hielt vor der Tür des
Hauses Elisas. \bibleverse{10} Da sandte Elisa einen Boten zu ihm und
ließ ihm sagen: Gehe hin und wasche dich siebenmal im Jordan, so wird
dir dein Fleisch wieder erstattet, und du wirst rein werden!
\bibleverse{11} Da ward Naeman zornig, ging weg und sprach: Siehe, ich
dachte, er werde zu mir herauskommen und herzutreten und den Namen des
\textsc{Herrn}, seines Gottes, anrufen und mit seiner Hand über die
Stelle fahren und den Aussatz wegnehmen! \bibleverse{12} Sind nicht die
Flüsse Abama und Pharphar zu Damaskus besser als alle Wasser in Israel?
Kann ich mich nicht darin waschen und rein werden? Und er wandte sich
und ging zornig davon. \bibleverse{13} Da traten seine Knechte zu ihm,
redeten mit ihm und sprachen: Mein Vater, wenn dir der Prophet etwas
Großes befohlen hätte, würdest du es nicht tun? Wieviel mehr denn, da er
zu dir sagt: Wasche dich, so wirst du rein! \bibleverse{14} Da stieg er
hinab und tauchte sich im Jordan siebenmal unter, wie der Mann Gottes
gesagt hatte; und sein Fleisch ward wieder wie das Fleisch eines jungen
Knaben, und er ward rein. \bibleverse{15} Und er kehrte wieder zu dem
Manne Gottes zurück, er und sein ganzes Gefolge. Und er ging hinein,
trat vor ihn und sprach: Siehe, nun weiß ich, daß kein Gott auf der
ganzen Erde ist, außer in Israel! Und nun nimm doch ein Geschenk an von
deinem Knechte! \bibleverse{16} Er aber sprach: So wahr der
\textsc{Herr} lebt, vor dessen Angesicht ich stehe, ich nehme nichts! Da
nötigte er ihn, es zu nehmen, aber er wollte nicht. \bibleverse{17} Da
sprach Naeman: Könnte deinem Knechte nicht eine doppelte Maultierlast
Erde gegeben werden? Denn dein Knecht will nicht mehr andern Göttern
Brandopfer und Schlachtopfer bringen, sondern nur dem \textsc{Herrn}.
\bibleverse{18} Nur darin wolle der \textsc{Herr} deinem Knechte gnädig
sein: Wenn mein Herr in das Haus Rimmons geht, daselbst anzubeten, und
er sich auf meinen Arm stützt und ich in dem Hause Rimmons niederfalle,
wenn er dort niederfällt, so wolle der \textsc{Herr} deinem Knecht aus
diesem Grunde vergeben! \bibleverse{19} Er sprach zu ihm: Gehe hin in
Frieden! \bibleverse{20} Als er nun eine Strecke Weges von ihm entfernt
war, dachte Gehasi, der Diener Elisas, des Mannes Gottes: Siehe, mein
Herr hat Naeman, diesen Syrer, geschont, indem er nichts von ihm
genommen, was er gebracht hat; so wahr der \textsc{Herr} lebt, ich will
ihm nachlaufen und etwas von ihm annehmen! \bibleverse{21} Also jagte
Gehasi dem Naeman nach. Und als Naeman sah, daß er ihm nachlief, sprang
er vom Wagen, ihm entgegen, und sprach: \bibleverse{22} Bringst du gute
Botschaft? Er sprach: Ja! Mein Herr hat mich gesandt, dir zu sagen:
Siehe, eben jetzt sind zwei Jünglinge von den Prophetensöhnen vom
Gebirge Ephraim zu mir gekommen. Gib ihnen doch ein Talent Silber und
zwei Feierkleider! \bibleverse{23} Naeman sprach: Tu mir den Gefallen
und nimm zwei Talente! Und er nötigte ihn und band zwei Talente Silber
in zwei Beutel und zwei Feierkleider und gab es seinen beiden Knappen,
die trugen es vor ihm her. \bibleverse{24} Und als er auf den Hügel kam,
nahm er es von ihrer Hand und legte es in das Haus und ließ die Männer
gehen. \bibleverse{25} Und sie gingen. Er aber kam und trat vor seinen
Herrn. Da sprach Elisa zu ihm: Woher, Gehasi? Er sprach: Dein Knecht ist
weder hierhin noch dorthin gegangen! \bibleverse{26} Er aber sprach zu
ihm: Wandelte nicht mein Herz mit dir, als der Mann von seinem Wagen
umkehrte, dir entgegen? War es auch an der Zeit, Silber zu nehmen und
Kleider, oder Ölbäume, Weinberge, Schafe, Rinder, Knechte und Mägde?
\bibleverse{27} So soll nun der Aussatz Naemans dir und deinem Samen
ewiglich anhangen! Da ging er von ihm hinaus, aussätzig wie Schnee.

\hypertarget{section-5}{%
\section{6}\label{section-5}}

\bibleverse{1} Und die Söhne der Propheten sprachen zu Elisa: Siehe
doch, der Ort, wo wir vor dir wohnen, ist uns zu eng! \bibleverse{2} Wir
wollen doch an den Jordan gehen und daselbst ein jeder einen Balken
holen, damit wir uns dort eine Niederlassung bauen. \bibleverse{3} Er
sprach: Geht hin! Es sprach aber einer: Tu uns doch den Gefallen und
komm mit deinen Knechten! \bibleverse{4} Er sprach: Ich will mitkommen!
Und er ging mit ihnen. Als sie nun an den Jordan kamen, schnitten sie
Holz. \bibleverse{5} Und als einer einen Stamm fällte, fiel das Eisen
ins Wasser. Da schrie er und sprach: O weh, mein Herr! Dazu ist es
entlehnt! \bibleverse{6} Aber der Mann Gottes sprach: Wohin ist es
gefallen? Und als er ihm den Ort zeigte, schnitt er ein Holz ab und warf
es dort hinein. Da schwamm das Eisen empor. \bibleverse{7} Und er
sprach: Hebe es auf! Da streckte er seine Hand aus und nahm es.
\bibleverse{8} Und der König von Syrien führte Krieg wider Israel und
beratschlagte sich mit seinen Knechten und sprach: Da und da soll mein
Lager sein! \bibleverse{9} Aber der Mann Gottes sandte zum König von
Israel und ließ ihm sagen: Hüte dich, an jenem Orte vorbeizugehen; denn
die Syrer begeben sich dorthin! \bibleverse{10} Und der König von Israel
sandte hin an den Ort, den ihm der Mann Gottes genannt und vor welchem
er ihn gewarnt hatte, und er nahm sich daselbst in acht, nicht bloß
einmal oder zweimal. \bibleverse{11} Da ward das Herz des Königs von
Syrien unruhig darüber, und er berief seine Knechte und sprach zu ihnen:
Wollt ihr mir denn nicht sagen, wer von uns es mit dem König von Israel
hält? \bibleverse{12} Da sprach einer seiner Knechte: Nicht also, mein
Herr und König; sondern Elisa, der Prophet in Israel, verrät dem König
von Israel alles, was du in deiner Schlafkammer redest! \bibleverse{13}
Er sprach: So geht hin und seht, wo er ist, daß ich ihn greifen lasse.
Und sie zeigten es ihm an und sprachen: Siehe, er ist in Dotan!
\bibleverse{14} Da sandte er Pferde und Wagen und eine große Macht
dorthin. Und sie kamen bei Nacht und umzingelten die Stadt.
\bibleverse{15} Als nun der Diener des Mannes Gottes am Morgen früh
aufstand und hinausging, siehe, da lag um die Stadt ein Heer mit Pferden
und Wagen. Da sprach sein Knecht zu ihm: O weh, mein Herr! was wollen
wir nun tun? \bibleverse{16} Er sprach: Fürchte dich nicht! Denn derer,
die bei uns sind, sind mehr, als derer, die bei ihnen sind!
\bibleverse{17} Und Elisa betete und sprach: \textsc{Herr}, öffne ihm
doch die Augen, daß er sehe! Da öffnete der \textsc{Herr} dem Knecht die
Augen, daß er sah. Und siehe, da war der Berg voll feuriger Rosse und
Wagen rings um Elisa her. \bibleverse{18} Und als sie zu ihm hinkamen,
bat Elisa den \textsc{Herrn} und sprach: Schlage doch diese Heiden mit
Blindheit! Da schlug er sie mit Blindheit nach dem Worte Elisas.
\bibleverse{19} Und Elisa sprach zu ihnen: Das ist nicht der Weg noch
die Stadt; folget mir nach, so will ich euch zu dem Manne führen, den
ihr suchet! Und er führte sie gen Samaria. \bibleverse{20} Und als sie
nach Samaria kamen, sprach Elisa: \textsc{Herr}, öffne diesen die Augen,
daß sie sehen! Und der \textsc{Herr} öffnete ihnen die Augen, daß sie
sahen. Und siehe, da waren sie mitten in Samaria. \bibleverse{21} Und
als der König von Israel sie sah, sprach er zu Elisa: Mein Vater, soll
ich sie schlagen? Soll ich sie schlagen? \bibleverse{22} Er sprach: Du
sollst sie nicht schlagen! Würdest du die, welche du mit deinem Schwert
und mit deinem Bogen gefangen nimmst, schlagen? Setze ihnen Brot und
Wasser vor, daß sie essen und trinken und zu ihrem Herrn ziehen!
\bibleverse{23} Da ward ein großes Mal zugerichtet. Und als sie gegessen
und getrunken hatten, ließ er sie gehen, und sie zogen zu ihrem Herrn.
Von da an kamen die Streifscharen der Syrer nicht mehr in das Land
Israel. \bibleverse{24} Darnach begab es sich, daß Benhadad, der König
von Syrien, sein ganzes Heer versammelte und heraufzog und Samaria
belagerte. \bibleverse{25} Da entstand in Samaria eine große Hungersnot;
und siehe, sie belagerten die Stadt so lange, bis ein Eselskopf achtzig
Silberlinge und ein Viertel Kab Taubenmist fünf Silberlinge galt.
\bibleverse{26} Und als der König von Israel auf der Mauer einherging,
flehte ihn ein Weib an und sprach: Hilf mir, mein Herr und König!
\bibleverse{27} Er aber sprach: Hilft dir der \textsc{Herr} nicht, von
woher soll ich dir Hilfe bringen? \bibleverse{28} Von der Tenne oder von
der Kelter? Und der König fragte sie: Was willst du? Sie sprach: Dieses
Weib sprach zu mir: Gib deinen Sohn her, daß wir ihn heute essen; morgen
wollen wir dann meinen Sohn essen! \bibleverse{29} So haben wir meinen
Sohn gekocht und ihn gegessen; und am andern Tage sprach ich zu ihr: Gib
deinen Sohn her, daß wir ihn essen! Aber sie hat ihren Sohn verborgen.
\bibleverse{30} Als der König die Worte des Weibes hörte, zerriß er
seine Kleider, während er auf der Mauer einherging. Da sah das Volk, daß
er darunter auf seinem Leibe einen Sack trug. \bibleverse{31} Und er
sprach: Gott tue mir dies und das, wenn das Haupt Elisas, des Sohnes
Saphats, heute auf ihm bleibt! \bibleverse{32} Elisa aber saß in seinem
Hause, und die Ältesten saßen bei ihm. Und der König sandte einen Mann
vor sich her; aber ehe der Bote zu ihm kam, sprach er zu den Ältesten:
Sehet ihr nicht, wie dieser Mördersohn hersendet, um mir den Kopf
abzuhauen? Sehet zu, wenn der Bote kommt, verschließet die Tür und
drängt ihn mit der Tür hinweg! Höre ich nicht die Fußtritte seines Herrn
hinter ihm her? \bibleverse{33} Während er noch mit ihnen redete, siehe,
da kam der Bote zu ihm hinab, und er sprach: Siehe, solches Übel kommt
vom \textsc{Herrn}, was soll ich noch auf den \textsc{Herrn} warten?

\hypertarget{section-6}{%
\section{7}\label{section-6}}

\bibleverse{1} Da sprach Elisa: Höret das Wort des \textsc{Herrn}! So
spricht der \textsc{Herr}: Morgen um diese Zeit wird im Tore zu Samaria
ein Maß Semmelmehl einen Silberling gelten und zwei Maß Gerste einen
Silberling! \bibleverse{2} Da antwortete der Ritter, auf dessen Arm sich
der König stützte, dem Manne Gottes und sprach: Siehe, und wenn der
\textsc{Herr} Fenster am Himmel machte, wie könnte solches geschehen? Er
aber sprach: Siehe, du wirst es mit eigenen Augen sehen, aber nicht
davon essen! \bibleverse{3} Es waren aber vier aussätzige Männer am
Eingang des Tores, und einer sprach zum anderen: Was wollen wir hier
bleiben, bis wir sterben? Wenn wir sprächen: Wir wollen in die Stadt
gehen, wo doch Hungersnot in der Stadt herrscht, so müßten wir dort
sterben; bleiben wir aber hier, so müssen wir auch sterben!
\bibleverse{4} So kommt nun, wir wollen zum Heere der Syrer übergehen!
Lassen sie uns leben, so leben wir, töten sie uns, so sind wir tot!
\bibleverse{5} Und sie machten sich in der Dämmerung auf, um in das
Lager der Syrer zu gehen. Als sie nun an den Rand des Lagers der Syrer
kamen, siehe, da war kein Mensch zugegen! \bibleverse{6} Denn der
\textsc{Herr} hatte das Heer der Syrer ein Getöse von Wagen, auch ein
Getümmel von Pferden und ein Geschrei einer großen Heeresmacht hören
lassen, so daß sie untereinander sprachen: Siehe, der König von Israel
hat die Könige der Hetiter und die Könige der Ägypter wider uns
gedungen, daß sie uns überfallen sollen! \bibleverse{7} Und sie machten
sich auf und flohen in der Dämmerung und ließen ihre Zelte und ihre
Pferde und ihre Esel, das Lager, wie es stand, und flohen, um ihr Leben
zu retten. \bibleverse{8} Als nun jene Aussätzigen an den Rand des
Lagers kamen, gingen sie in ein Zelt, aßen und tranken und nahmen
Silber, Gold und Kleider daraus mit und gingen hin und verbargen es, und
gingen in ein anderes Zelt und nahmen daraus, gingen fort und verbargen
es. \bibleverse{9} Aber einer sprach zum andern: Wir handeln nicht
recht. Dieser Tag ist ein Tag guter Botschaft; wenn wir schweigen und
warten, bis es heller Morgen wird, so wird uns Strafe treffen. So kommt
nun, wir wollen gehen und es dem Hause des Königs melden.
\bibleverse{10} Und sie kamen und riefen dem Torhüter der Stadt und
verkündigten ihnen und sprachen: Wir sind zum Lager der Syrer gekommen,
und siehe, es ist niemand da, und man hört auch keinen Menschen, sondern
nur Pferde und Esel; die sind angebunden, und die Zelte, wie sie waren.
\bibleverse{11} Da riefen es die Torhüter, und man berichtete es drinnen
im Hause des Königs. \bibleverse{12} Und der König stand in der Nacht
auf und sprach zu seinen Knechten: Ich will euch doch sagen, was die
Syrer mit uns vorhaben: Sie wissen, daß wir Hunger leiden, und sind aus
dem Lager gegangen, um sich im Felde zu verbergen, und denken: Wenn die
aus der Stadt gehen, wollen wir sie lebendig fangen und in die Stadt
eindringen. \bibleverse{13} Da antwortete einer seiner Knechte und
sprach: Man nehme doch fünf von den übriggebliebenen Pferden, die noch
da sind (siehe, es geht ihnen doch wie der ganzen Menge Israels, die
darin übriggeblieben ist; siehe, es geht ihnen wie der ganzen Menge
Israels, welche aufgerieben ist), die lasset uns senden und dann
schauen! \bibleverse{14} Da nahmen sie zwei Gespanne Pferde, und der
König sandte sie dem Heere der Syrer nach und sprach: Gehet hin und
sehet nach! \bibleverse{15} Als sie ihnen nun bis an den Jordan
nachzogen, siehe, da lagen alle Wege voll Kleider und Waffen, welche die
Syrer auf ihrer eiligen Flucht von sich geworfen hatten. Und die Boten
kamen wieder und sagten es dem König. \bibleverse{16} Da ging das Volk
hinaus und plünderte das Lager der Syrer, so daß ein Maß Semmelmehl
einen Silberling galt und zwei Maß Gerste auch einen Silberling, nach
dem Worte des \textsc{Herrn}. \bibleverse{17} Und der König bestellte
den Ritter, auf dessen Arm er sich stützte, zur Aufsicht über das Tor;
und das Volk zertrat ihn im Tor, so daß er starb, wie der Mann Gottes
gesagt hatte, als der König zu ihm hinabkam. \bibleverse{18} Denn es
geschah, wie der Mann Gottes dem König gesagt hatte, als er sprach:
Morgen um diese Zeit werden im Tore zu Samaria zwei Maß Gerste einen
Silberling gelten und ein Maß Semmelmehl einen Silberling;
\bibleverse{19} worauf der Ritter dem Manne Gottes geantwortet hatte:
Ja, siehe, und wenn der \textsc{Herr} Fenster am Himmel machte, wie
könnte solches geschehen? Er aber hatte gesagt: Siehe, du wirst es mit
deinen Augen sehen, aber nicht davon essen! \bibleverse{20} Also erging
es ihm jetzt; denn das Volk zertrat ihn im Tore, so daß er starb.

\hypertarget{section-7}{%
\section{8}\label{section-7}}

\bibleverse{1} Und Elisa redete mit dem Weibe, deren Sohn er lebendig
gemacht hatte, und sprach: Mache dich auf und gehe hin mit deinem
Haushalt und halte dich in der Fremde auf, wo du kannst! Denn der
\textsc{Herr} hat eine Hungersnot herbeigerufen. Und sie kommt auch in
das Land, sieben Jahre lang. \bibleverse{2} Das Weib machte sich auf und
tat, wie der Mann Gottes sagte, und zog hin mit ihren Hausgenossen und
hielt sich im Lande der Philister auf, sieben Jahre lang. \bibleverse{3}
Als aber die sieben Jahre vorbei waren, kam das Weib wieder aus dem
Lande der Philister, und sie ging hin, um den König wegen ihres Hauses
und wegen ihres Ackers anzurufen. \bibleverse{4} Der König aber redete
eben mit Gehasi, dem Knechte des Mannes Gottes, und sprach: Erzähle mir
doch alle die großen Taten, welche Elisa getan hat! \bibleverse{5}
Während er aber dem Könige erzählte, wie jener einen Toten lebendig
gemacht hatte, siehe, da kam eben das Weib, deren Sohn er lebendig
gemacht hatte, dazu und rief den König an wegen ihres Hauses und wegen
ihres Ackers. Da sprach Gehasi: Mein Herr und König, hier ist das Weib,
und dies ist ihr Sohn, den Elisa lebendig gemacht hat! \bibleverse{6} Da
fragte der König das Weib, und sie erzählte es ihm. Da gab ihr der König
einen Kämmerer und sprach: Verschaffe ihr alles wieder, was ihr gehört;
dazu allen Ertrag des Ackers seit der Zeit, da sie das Land verlassen
hat, bis jetzt. \bibleverse{7} Und Elisa kam nach Damaskus. Da lag
Benhadad, der König von Syrien, krank. Und man sagte es ihm und sprach:
Der Mann Gottes ist hierher gekommen! \bibleverse{8} Da sprach der König
zu Hasael: Nimm Geschenke mit dir und gehe dem Manne Gottes entgegen und
befrage den \textsc{Herrn} durch ihn und sprich: Werde ich von dieser
Krankheit genesen können? \bibleverse{9} Hasael ging ihm entgegen und
nahm Geschenke mit sich und allerlei Güter von Damaskus, eine Last für
vierzig Kamele. Und als er kam, trat er vor ihn hin und sprach: Dein
Sohn Benhadad, der König von Syrien, hat mich zu dir gesandt und läßt
dir sagen: Kann ich auch von dieser Krankheit genesen? \bibleverse{10}
Elisa sprach zu ihm: Gehe hin und sage ihm: Du wirst genesen! Aber der
\textsc{Herr} hat mir gezeigt, daß er gewiß sterben wird.
\bibleverse{11} Und der Mann Gottes richtete sein Angesicht auf ihn und
starrte ihn unverwandt an, bis er sich schämte; dann weinte er.
\bibleverse{12} Da sprach Hasael: Warum weint mein Herr? Er sprach: Weil
ich weiß, was für Unheil du den Kindern Israel antun wirst! Du wirst
ihre festen Städte mit Feuer verbrennen und ihre junge Mannschaft mit
dem Schwert töten und ihre Kindlein zerschmettern und die Frauen
aufschlitzen, die guter Hoffnung sind. \bibleverse{13} Hasael sprach:
Was ist dein Knecht, der Hund, daß er solch große Dinge tun sollte?
Elisa sprach: Der \textsc{Herr} hat mir gezeigt, daß du König über
Syrien wirst! \bibleverse{14} Und er ging von Elisa weg und kam zu
seinem Herrn; der sprach zu ihm: Was sagte dir Elisa? Er sprach: Er
sagte mir, du werdest gewiß genesen! \bibleverse{15} Am folgenden Tage
aber nahm er die Decke und tauchte sie ins Wasser und breitete sie über
sein Angesicht, so daß er starb. Und Hasael ward König an seiner Statt.
\bibleverse{16} Im fünften Jahre Jorams, des Sohnes Ahabs, des Königs
von Israel, ward Jehoram, der Sohn Josaphats, König in Juda.
\bibleverse{17} Zweiunddreißig Jahre alt war er, als er König ward, und
regierte acht Jahre lang zu Jerusalem. \bibleverse{18} Und er wandelte
auf dem Wege der Könige von Israel, wie das Haus Ahabs tat; denn die
Tochter Ahabs war sein Weib, und er tat, was dem \textsc{Herrn} übel
gefiel. \bibleverse{19} Aber der \textsc{Herr} wollte Juda nicht
verderben, um seines Knechtes David willen, wie er ihm verheißen hatte,
ihm unter seinen Söhnen eine Leuchte zu geben immerdar. \bibleverse{20}
Zu seiner Zeit fielen die Edomiter von Juda ab und machten einen König
über sich. \bibleverse{21} Da zog Jehoram gen Zair und alle Wagen mit
ihm; und er machte sich auf bei Nacht und schlug die Edomiter, die ihn
und die Obersten über die Wagen umzingelt hatten, so daß das Volk in
seine Hütten floh. \bibleverse{22} Dennoch fielen die Edomiter von Juda
ab bis auf diesen Tag. Auch Libna fiel zu jener Zeit ab. \bibleverse{23}
Was aber mehr von Jehoram zu sagen ist, und alles, was er getan hat, ist
das nicht geschrieben in der Chronik der Könige von Juda?
\bibleverse{24} Und Jehoram legte sich zu seinen Vätern in der Stadt
Davids; und Ahasia, sein Sohn, ward König an seiner Statt.
\bibleverse{25} Im elften Jahre Jorams, des Sohnes Ahabs, des Königs von
Israel, wurde Ahasia, der Sohn Jehorams, König in Juda. \bibleverse{26}
Zweiundzwanzig Jahre alt war Ahasia, als er König ward, und regierte ein
Jahr lang zu Jerusalem. Und seine Mutter hieß Atalia, eine Tochter
Omris, des Königs von Israel. \bibleverse{27} Und er wandelte auf dem
Wege des Hauses Ahabs und tat, was böse war in den Augen des
\textsc{Herrn}, wie das Haus Ahabs; denn er war Tochtermann im Hause
Ahabs. \bibleverse{28} Und er zog mit Joram, dem Sohne Ahabs, in den
Krieg wider Hasael, den König von Syrien, nach Ramot in Gilead; aber die
Syrer verwundeten Joram. \bibleverse{29} Da kehrte der König Joram
zurück, um sich zu Jesreel heilen zu lassen von den Wunden, welche ihm
die Syrer in Ramot geschlagen hatten, als er mit Hasael, dem König von
Syrien, stritt. Und Ahasia, der Sohn Jehorams, der König in Juda, kam
hinab, um Joram, den Sohn Ahabs, in Jesreel zu besuchen; denn er lag
krank.

\hypertarget{section-8}{%
\section{9}\label{section-8}}

\bibleverse{1} Elisa aber, der Prophet, rief einen der Prophetensöhne
und sprach zu ihm: Gürte deine Lenden und nimm diese Ölflasche mit dir
und gehe hin nach Ramot in Gilead. \bibleverse{2} Und wenn du dahin
kommst, so schau, wo Jehu, der Sohn Josaphats, des Sohnes Nimsis, ist,
und gehe hinein und heiße ihn aufstehen aus der Mitte seiner Brüder und
führe ihn in die innerste Kammer; \bibleverse{3} und nimm die Ölflasche
und gieße sie auf sein Haupt aus und sprich: So spricht der
\textsc{Herr}: Ich habe dich zum König über Israel gesalbt! Und du
sollst die Tür öffnen und fliehen und nicht verweilen! \bibleverse{4}
Also ging der Jüngling, der Diener des Propheten, hin gen Ramot in
Gilead. \bibleverse{5} Und als er hineinkam, siehe, da saßen die
Hauptleute des Heeres beisammen, und er sprach: Ein Wort habe ich an
dich, o Hauptmann! Jehu sprach: An welchen von uns allen? Er sprach: An
dich, o Hauptmann! \bibleverse{6} Da stand er auf und ging in das Haus
hinein. Er aber goß das Öl auf sein Haupt und sprach zu ihm: So spricht
der \textsc{Herr}, der Gott Israels: Ich habe dich zum König gesalbt
über das Volk des \textsc{Herrn}, über Israel! \bibleverse{7} Und du
sollst das Haus Ahabs, deines Herrn, erschlagen; so will ich das Blut
der Propheten, meiner Knechte, und das Blut aller Knechte des
\textsc{Herrn} an Isebel rächen. \bibleverse{8} Ja, das ganze Haus Ahabs
soll umkommen; und ich will von Ahab alles ausrotten, was männlich ist,
Mündige und Unmündige in Israel. \bibleverse{9} Und ich will das Haus
Ahabs machen wie das Haus Jerobeams, des Sohnes Nebats, und wie das Haus
Baesas, des Sohnes Achijas. \bibleverse{10} Und die Hunde sollen Isebel
fressen auf dem Acker zu Jesreel, und niemand soll sie begraben! Und er
öffnete die Tür und floh. \bibleverse{11} Als nun Jehu zu den Knechten
seines Herrn herausging, sprach man zu ihm: Bedeutet es Friede? Warum
ist dieser Unsinnige zu dir gekommen? Er sprach zu ihnen: Ihr kennt doch
den Mann und seine Rede? \bibleverse{12} Sie sprachen: Das ist nicht
wahr; sage es uns doch! Er sprach: So und so hat er mit mir geredet und
gesagt: So spricht der \textsc{Herr}: Ich habe dich zum König über
Israel gesalbt! \bibleverse{13} Da eilten sie und nahmen ein jeder sein
Kleid und legten sie unter ihn auf die bloßen Stufen; und sie stießen in
die Posaune und riefen: Jehu ist König geworden! \bibleverse{14} Also
machte Jehu, der Sohn Josaphats, des Sohnes Nimsis, eine Verschwörung
wider Joram. Joram aber hatte mit ganz Israel zu Ramot in Gilead wider
Hasael, den König von Syrien, Wache gehalten. \bibleverse{15} Aber der
König Joram war wieder umgekehrt, um sich zu Jesreel heilen zu lassen
von den Wunden, welche ihm die Syrer geschlagen hatten, als er mit
Hasael, dem König von Syrien, stritt. Und Jehu sprach: Wenn es euch
recht ist, so soll niemand aus der Stadt entfliehen, um hinzugehen und
es in Jesreel zu berichten! \bibleverse{16} Und Jehu ritt nach Jesreel;
denn Joram lag daselbst; auch war Ahasia, der König von Juda,
herabgekommen, Joram zu besuchen. \bibleverse{17} Der Wächter aber, der
auf dem Turm zu Jesreel stand, sah Jehus Schar kommen und sprach: Ich
sehe eine Schar! Da sprach Joram: Nimm einen Reiter und sende ihnen den
entgegen und frage: Bedeutet es Friede? \bibleverse{18} Und der Reiter
ritt ihm entgegen und sprach: So spricht der König: Bedeutet es Friede?
Jehu sprach: Was geht dich der Friede an? Wende dich, folge mir! Der
Wächter verkündigte es und sprach: Der Bote ist zu ihnen gekommen und
kehrt nicht zurück! \bibleverse{19} Da sandte er einen andern Reiter.
Als der zu ihm kam, sprach er: So spricht der König: Bedeutet es Friede?
Jehu sprach: Was geht dich der Friede an? Wende dich, folge mir!
\bibleverse{20} Das verkündigte der Wächter und sprach: Der ist auch zu
ihnen gekommen und kehrt nicht zurück; und es ist ein Jagen wie das
Jagen Jehus, des Sohnes Nimsis, denn er jagt, als wäre er rasend!
\bibleverse{21} Da sprach Joram: Spanne an! Und man spannte seinen Wagen
an, und sie zogen aus, Joram, der König von Israel, und Ahasia, der
König von Juda, jeder auf seinem Wagen; sie fuhren Jehu entgegen, und
sie trafen ihn auf dem Acker Nabots, des Jesreeliten. \bibleverse{22}
Als nun Joram den Jehu sah, sprach er: Jehu, bedeutet das Friede? Er
aber sprach: Was Friede, bei all der Buhlerei und Zauberei deiner Mutter
Isebel? \bibleverse{23} Da wandte sich Joram zur Flucht und sprach zu
Ahasia: Verrat, Ahasia! \bibleverse{24} Aber Jehu nahm den Bogen zur
Hand und schoß Joram zwischen die Schultern, so daß der Pfeil durch sein
Herz fuhr und er in seinen Wagen sank. \bibleverse{25} Und Jehu sprach
zu Bidekar, seinem Wagenkämpfer: Nimm ihn und wirf ihn auf das Ackerfeld
Nabots, des Jesreeliten; denn gedenke, wie wir, ich und du,
nebeneinander hinter seinem Vater Ahab herritten, als der \textsc{Herr}
diesen Ausspruch über ihn tat: \bibleverse{26} «Fürwahr, das Blut Nabots
und das Blut seiner Söhne habe ich gestern gesehen, spricht der
\textsc{Herr}; und ich werde es dir auf diesem Acker vergelten, spricht
der \textsc{Herr}!'' So nimm ihn und wirf ihn auf den Acker, nach dem
Worte des \textsc{Herrn}! \bibleverse{27} Als aber Ahasia, der König von
Juda, solches sah, floh er dem Gartenhause zu. Jehu aber jagte ihm nach
und sprach: Erschießt ihn auch! Da schossen sie ihn nieder auf seinem
Wagen, beim Aufstieg nach Gur, das bei Jibleam liegt; und er floh gen
Megiddo und starb daselbst. \bibleverse{28} Und seine Knechte ließen ihn
nach Jerusalem führen und begruben ihn in seinem Grabe mit seinen Vätern
in der Stadt Davids. \bibleverse{29} Ahasia aber war König geworden über
Juda im elften Jahre Jorams, des Sohnes Ahabs. \bibleverse{30} Als nun
Jehu nach Jesreel kam und Isebel solches hörte, schminkte sie ihr
Angesicht und schmückte ihr Haupt und schaute zum Fenster hinaus.
\bibleverse{31} Und als Jehu in das Tor kam, sprach sie: Ist es Simri
wohl ergangen, der seinen Herrn ermordete? \bibleverse{32} Da schaute er
zum Fenster empor und sprach: Wer hält es mit mir? Wer? Da sahen zwei
oder drei Kämmerer zu ihm hinab. \bibleverse{33} Er sprach: Stürzet sie
herab! Und sie stürzten sie hinunter, daß die Wände und die Pferde mit
ihrem Blut bespritzt wurden; und sie zertraten sie. \bibleverse{34} Und
als er hineinkam und gegessen und getrunken hatte, sprach er: Sehet doch
nach dieser Verfluchten und begrabet sie, denn sie ist eines Königs
Tochter! \bibleverse{35} Als sie aber hingingen, sie zu begraben, fanden
sie nichts mehr von ihr als den Schädel, die Füße und die Handflächen;
\bibleverse{36} und sie kamen wieder und sagten es ihm. Er aber sprach:
Es erfüllt sich, was der \textsc{Herr} durch seinen Knecht Elia, den
Tisbiter, gesagt hat, als er sprach: «Auf dem Acker Jesreels sollen die
Hunde das Fleisch der Isebel fressen! \bibleverse{37} So wird der
Leichnam Isebels sein wie Dünger auf dem Felde im Acker Jesreels», daß
man nicht sagen kann: Dies ist Isebel!

\hypertarget{section-9}{%
\section{10}\label{section-9}}

\bibleverse{1} Ahab aber hatte siebzig Söhne zu Samaria. Und Jehu
schrieb Briefe und sandte sie nach Samaria an die Obersten von Israel,
an die Ältesten und Hofmeister Ahabs, die lauteten also: \bibleverse{2}
Sobald dieser Brief zu euch kommt, die ihr über eures Herrn Söhne,
Wagen, Pferde und über feste Städte und Rüstungen verfügt, so sehet,
\bibleverse{3} welcher der beste und rechtschaffenste unter den Söhnen
eures Herrn sei, und setzet ihn auf seines Vaters Thron und kämpfet für
das Haus eures Herrn! \bibleverse{4} Sie aber fürchteten sich sehr und
sprachen: Siehe, die zwei Könige konnten nicht vor ihm bestehen, wie
wollen denn wir bestehen? \bibleverse{5} Und sie, die Vorgesetzten des
Hauses und der Stadt, und die Ältesten und Hofmeister sandten hin zu
Jehu und ließen ihm sagen: Wir sind deine Knechte und wollen alles tun,
was du uns sagst! Wir wollen niemand zum König machen; tue was dir
gefällt! \bibleverse{6} Da schrieb er einen andern Brief an sie, der
lautete also: Wollt ihr es mit mir halten und meiner Stimme gehorchen,
so nehmet die Köpfe der Männer, der Söhne eures Herrn, und bringet sie
morgen um diese Zeit zu mir gen Jesreel! Aber die Königssöhne, siebzig
Mann, waren bei den Vornehmsten der Stadt, die sie auferzogen.
\bibleverse{7} Als nun der Brief zu ihnen kam, nahmen sie die
Königssöhne und töteten sie, siebzig Mann, und legten ihre Köpfe in
Körbe und sandten sie zu ihm nach Jesreel. \bibleverse{8} Und als der
Bote kam und es ihm berichtete und sprach: Sie haben die Köpfe der
Königssöhne gebracht! da sprach er: Leget sie in zwei Haufen draußen vor
das Tor bis zum Morgen! \bibleverse{9} Und am Morgen, als er hinausging,
trat er hin und sprach zu allem Volk: Ihr seid gerecht! Siehe, ich habe
wider meinen Herrn eine Verschwörung gemacht und ihn umgebracht. Wer
aber hat diese alle erschlagen? \bibleverse{10} So erkennet nun, daß
kein Wort des \textsc{Herrn} auf die Erde gefallen ist, das der
\textsc{Herr} wider das Haus Ahabs geredet hat, sondern der
\textsc{Herr} hat getan, wie er durch seinen Knecht Elia geredet hat.
\bibleverse{11} Also erschlug Jehu zu Jesreel alle Übrigen vom Hause
Ahabs und seine Gewaltigen, seine Bekannten und seine vertrauten Räte,
daß ihm nicht einer übrigblieb. \bibleverse{12} Darnach machte er sich
auf und zog nach Samaria. Unterwegs aber war ein Hirtenhaus.
\bibleverse{13} Da traf Jehu die Brüder Ahasias, des Königs von Juda, an
und sprach: Wer seid ihr? Sie sprachen: Wir sind die Brüder Ahasias und
ziehen hinab, um die Söhne des Königs und die Söhne der Gebieterin zu
begrüßen. \bibleverse{14} Er aber sprach: Greift sie lebendig! Und sie
griffen sie lebendig und erstachen sie bei dem Brunnen am Hirtenhause,
zweiundvierzig Mann; und er ließ nicht einen von ihnen übrig.
\bibleverse{15} Und als er von dannen zog, fand er Jonadab, den Sohn
Rechabs, der ihm entgegenkam; und er grüßte ihn und sprach zu ihm: Ist
dein Herz aufrichtig, wie mein Herz mit deinem Herzen? Jonadab sprach:
Ja! Wenn es so ist, so gib mir deine Hand! Und er gab ihm seine Hand.
\bibleverse{16} Da ließ er ihn zu sich auf den Wagen sitzen und sprach:
Komm mit mir und siehe meinen Eifer für den \textsc{Herrn}! Und er
führte ihn auf seinem Wagen. \bibleverse{17} Und als er nach Samaria
kam, erschlug er alles, was von Ahab in Samaria noch übrig war, bis er
ihn vertilgt hatte, gemäß dem Worte des \textsc{Herrn}, das er zu Elia
geredet hatte. \bibleverse{18} Und Jehu versammelte alles Volk und
sprach zu ihnen: Ahab hat dem Baal wenig gedient, Jehu will ihm besser
dienen! \bibleverse{19} So beruft nun alle Propheten Baals, alle seine
Knechte und alle seine Priester zu mir, daß niemand fehle; denn ich habe
dem Baal ein großes Opfer zu bringen. Wen man vermissen wird, der soll
nicht leben! Aber Jehu tat es aus List, um die Diener Baals auszurotten.
\bibleverse{20} Und Jehu sprach: Heiligt dem Baal ein Fest! Und sie
ließen ein solches ausrufen. \bibleverse{21} Jehu sandte auch Boten in
ganz Israel und ließ alle Diener Baals kommen, also daß niemand
übrigblieb, der nicht gekommen wäre. Und sie kamen in das Haus Baals, so
daß das Haus Baals voll ward, von einem Ende bis zum andern.
\bibleverse{22} Da sprach er zu dem, der über das Kleiderhaus verordnet
war: Bringe die Kleider aller Diener Baals heraus! Und er brachte ihre
Kleider heraus. \bibleverse{23} Und Jehu ging mit Jonadab, dem Sohne
Rechabs, in das Haus Baals und sprach zu den Dienern Baals: Forschet
nach und sehet zu, daß hier unter euch nicht jemand von den Dienern des
\textsc{Herrn} sei, sondern ausschließlich Diener des Baal!
\bibleverse{24} Und als sie hineinkamen, Opfer und Brandopfer
darzubringen, bestellte Jehu draußen achtzig Mann und sprach: Wenn
jemand einen von den Männern entrinnen läßt, die ich in eure Hand gebe,
so soll sein Leben für dessen Leben haften! \bibleverse{25} Als man nun
die Brandopfer vollendet hatte, sprach Jehu zu den Trabanten und
Rittern: Geht hinein und erschlaget sie, daß niemand davonkomme! Und sie
erschlugen sie mit der Schärfe des Schwertes. Und die Trabanten und
Ritter warfen sie weg und gingen in den Hinterraum des Baalstempels
\bibleverse{26} und brachten die Bildsäulen des Baalstempels heraus und
verbrannten sie und rissen die Bildsäule des Baal nieder.
\bibleverse{27} Sie zerstörten auch den Baalstempel und machten Kloaken
daraus, die sind dort bis auf diesen Tag. \bibleverse{28} Also vertilgte
Jehu den Baal aus Israel. \bibleverse{29} Aber von den Sünden Jerobeams,
des Sohnes Nebats, wodurch er Israel zur Sünde verführt hatte, ließ Jehu
nicht, nämlich von den goldenen Kälbern zu Bethel und zu Dan.
\bibleverse{30} Doch sprach der \textsc{Herr} zu Jehu: Weil du dich wohl
gehalten und getan hast, was recht ist in meinen Augen, weil du am Hause
Ahabs getan hast nach allem, was in meinem Herzen war, so sollen
Nachkommen von dir bis in das vierte Glied auf dem Throne Israels
sitzen! \bibleverse{31} Aber Jehu achtete nicht darauf, von ganzem
Herzen nach dem Gesetze des \textsc{Herrn}, des Gottes Israels, zu
wandeln; denn er wich nicht von den Sünden Jerobeams, wodurch er Israel
zur Sünde verführt hatte. \bibleverse{32} Zu jener Zeit fing der
\textsc{Herr} an, Israel zu schmälern; denn Hasael schlug sie an allen
Grenzen Israels: östlich vom Jordan, \bibleverse{33} das ganze Land
Gilead, die Gaditer, Rubeniter und Manassiter, von Aroer an, das am
Bache Arnon liegt, Gilead und Basan. \bibleverse{34} Was aber mehr von
Jehu zu sagen ist, und alles, was er getan hat, und alle seine Macht,
ist das nicht geschrieben in der Chronik der Könige von Israel?
\bibleverse{35} Und Jehu legte sich zu seinen Vätern; und sie begruben
ihn zu Samaria. Und Joahas, sein Sohn, ward König an seiner Statt.
\bibleverse{36} Die Zeit aber, die Jehu über Israel zu Samaria regiert
hat, beträgt achtundzwanzig Jahre.

\hypertarget{section-10}{%
\section{11}\label{section-10}}

\bibleverse{1} Als aber Atalia, die Mutter Ahasias, sah, daß ihr Sohn
tot war, machte sie sich auf und brachte allen königlichen Samen um.
\bibleverse{2} Joscheba aber, die Tochter des Königs Joram, Ahasias
Schwester, nahm Joas, den Sohn Ahasias, und stahl in weg aus der Mitte
der Königssöhne, die getötet wurden, und tat ihn samt seiner Amme in
eine Schlafkammer; und sie verbargen ihn vor Atalia, daß er nicht
getötet wurde. \bibleverse{3} Und er war mit ihr sechs Jahre lang
verborgen im Hause des \textsc{Herrn}. Atalia aber war Königin im Lande.
\bibleverse{4} Aber im siebenten Jahre ließ Jojada die Obersten über die
Hundertschaften der Karier und der Trabanten holen und zu sich in das
Haus des \textsc{Herrn} kommen; und er machte mit ihnen einen Bund und
nahm einen Eid von ihnen im Hause des \textsc{Herrn} \bibleverse{5} und
zeigte ihnen den Sohn des Königs und gebot ihnen und sprach: Das ist es,
was ihr tun sollt: Der dritte Teil von euch, die ihr am Sabbat antretet,
soll Wache halten im Hause des Königs; \bibleverse{6} und ein Drittel am
Tore Sur und ein Drittel am Tore hinter den Trabanten; und ihr sollt
Wache halten beim Hause. \bibleverse{7} Und die zwei andern Teile von
euch, alle, die am Sabbat abtreten, sollen im Hause des \textsc{Herrn}
um den König Wache halten. \bibleverse{8} Und ihr sollt euch rings um
den König scharen, ein jeglicher mit seinen Waffen in der Hand; wer aber
in die Reihen eindringt, der soll getötet werden; und ihr sollt bei dem
König sein, wenn er aus und eingeht. \bibleverse{9} Und die Obersten
über Hundert taten alles, wie ihnen der Priester Jojada geboten hatte;
und sie nahmen zu sich ihre Männer, die am Sabbat antraten, samt denen,
die am Sabbat abtraten, und kamen zu Jojada, dem Priester.
\bibleverse{10} Und der Priester gab den Obersten über Hundert Speere
und Schilde, welche dem König David gehört hatten, und die im Hause des
\textsc{Herrn} waren. \bibleverse{11} Und die Trabanten standen um den
König her, ein jeglicher mit seinen Waffen in der Hand, von der rechten
Seite des Hauses bis zur linken Seite des Hauses, bei dem Altar und bei
dem Hause. \bibleverse{12} Und er führte den Sohn des Königs heraus und
setzte ihm die Krone auf und gab ihm das Zeugnis; und sie machten ihn
zum König und salbten ihn und klatschten in die Hände und sprachen: Es
lebe der König! \bibleverse{13} Als aber Atalia das Geschrei der
Trabanten und des Volkes hörte, kam sie zum Volk in das Haus des
\textsc{Herrn}. \bibleverse{14} Und sie sah zu, und siehe, da stand der
König an der Säule, wie es Sitte war, und die Obersten und Trabanten bei
dem König; und alles Volk des Landes war fröhlich und stieß in die
Trompeten. Atalia aber zerriß ihre Kleider und schrie: Verschwörung!
Verschwörung! \bibleverse{15} Aber Jojada, der Priester, gebot den
Obersten über Hundert, die über das Heer gesetzt waren, und sprach:
Führt sie hinaus, zwischen den Reihen hindurch, und wer ihr folgt, der
soll durch das Schwert sterben! Denn der Priester sprach: Sie soll nicht
im Hause des \textsc{Herrn} getötet werden! \bibleverse{16} Da legte man
Hand an sie. Und sie ging durch den Eingang für die Pferde zum Hause des
Königs und ward daselbst getötet. \bibleverse{17} Da machte Jojada einen
Bund zwischen dem \textsc{Herrn} und dem König und dem Volk, daß sie das
Volk des \textsc{Herrn} sein sollten; ebenso zwischen dem König und dem
Volk. \bibleverse{18} Da ging alles Volk des Landes in den Baalstempel
und zerstörte ihn, seine Altäre und Bilder zerbrachen sie in Stücke; und
Mattan, den Baalspriester, töteten sie vor den Altären. \bibleverse{19}
Der Priester aber bestellte Wachen über das Haus des \textsc{Herrn}. Und
er nahm die Obersten über Hundert, und die Leibwache und die Trabanten
und alles Volk des Landes, und sie führten den König hinab vom Hause des
\textsc{Herrn} und kamen durch das Tor der Trabanten zum Hause des
Königs; und er setzte sich auf den Thron der Könige. \bibleverse{20} Und
alles Volk im Lande ward fröhlich, und die Stadt blieb stille. Atalia
aber hatten sie mit dem Schwerte getötet beim Hause des Königs.
\bibleverse{21} Joas war sieben Jahre alt, als er König ward.

\hypertarget{section-11}{%
\section{12}\label{section-11}}

\bibleverse{1} Im siebenten Jahre Jehus ward Joas König und regierte
vierzig Jahre lang zu Jerusalem. Seine Mutter hieß Zibia, von Beerseba.
\bibleverse{2} Und Joas tat, was recht war in den Augen des
\textsc{Herrn}, solange ihn der Priester Jojada unterwies.
\bibleverse{3} Doch kamen die Höhen nicht weg; denn das Volk opferte und
räucherte noch auf den Höhen. \bibleverse{4} Und Joas sprach zu den
Priestern: Alles geheiligte Geld, welches in das Haus des \textsc{Herrn}
gebracht wird, das Geld, welches gelegentlich eingeht, wie das, was
jeder nach seiner Schatzung gibt, auch alles Geld, das jemand freiwillig
ins Haus des \textsc{Herrn} bringt, \bibleverse{5} das sollen die
Priester zu sich nehmen, ein jeder von seinem Bekannten; davon sollen
sie ausbessern, was am Hause baufällig ist; alles, was baufällig
erfunden wird, sollen sie ausbessern. \bibleverse{6} Als aber die
Priester im dreiundzwanzigsten Jahre des Königs Joas nicht ausgebessert
hatten, was am Hause baufällig war, \bibleverse{7} berief der König den
Priester Jojada und die übrigen Priester und sprach zu ihnen: Warum
bessert ihr nicht aus, was am Hause baufällig ist? So sollt ihr nun das
Geld nicht mehr nehmen von euren Bekannten, sondern sollt es für die
Ausbesserung des Hauses geben! \bibleverse{8} Und die Priester waren
damit einverstanden, von dem Volk kein Geld mehr zu nehmen und auch für
die Ausbesserung des Hauses nicht mehr zu sorgen. \bibleverse{9} Da nahm
Jojada, der Priester, eine Lade und bohrte ein Loch in deren Deckel und
stellte sie zur rechten Hand neben den Altar, da man in das Haus des
\textsc{Herrn} geht. Und die Priester, welche die Schwelle hüteten,
taten alles Geld, das zum Hause des \textsc{Herrn} gebracht ward,
dahinein. \bibleverse{10} Wenn sie dann sahen, daß viel Geld in der Lade
war, kamen des Königs Schreiber und der Hohepriester herauf und banden
das Geld zusammen und zählten, was im Hause des \textsc{Herrn} gefunden
ward. \bibleverse{11} Und man gab das abgewogene Geld denen, die die
Arbeit verrichteten, die über das Haus des \textsc{Herrn} bestellt
waren; die zahlten es an die Zimmerleute und Bauleute, welche am Hause
des \textsc{Herrn} arbeiteten, \bibleverse{12} an die Maurer und
Steinmetzen, und um Holz und behauene Steine zu kaufen, damit
auszubessern, was am Hause des \textsc{Herrn} baufällig war, und für
alle übrigen Ausgaben zur Ausbesserung des Hauses. \bibleverse{13} Doch
ließ man für das Haus des \textsc{Herrn} keine silbernen Schalen,
Messer, Becken, Trompeten, noch irgend ein goldenes oder silbernes Gerät
von dem Gelde machen, welches in das Haus des \textsc{Herrn} gebracht
worden war, \bibleverse{14} sondern man gab es den Arbeitern, daß sie
damit die Schäden am Hause des \textsc{Herrn} ausbesserten.
\bibleverse{15} Sie rechneten auch nicht ab mit den Männern, denen man
das Geld einhändigte, um es den Arbeitern zu geben, sondern sie
handelten mit Redlichkeit. \bibleverse{16} Das Geld von Schuldopfern
aber und das Geld von Sündopfern wurde nicht in das Haus des
\textsc{Herrn} gebracht, denn es gehörte den Priestern. \bibleverse{17}
Zu der Zeit zog Hasael, der König von Syrien, hinauf und stritt wider
Gat und eroberte es. Und als Hasael Miene machte, wider Jerusalem
hinaufzuziehen, \bibleverse{18} nahm Joas, der König von Juda, alles,
was geheiligt war, was seine Väter Josaphat, Joram und Ahasia, die
Könige von Juda, geheiligt hatten, und was er selbst geheiligt hatte,
dazu alles Gold, das man in den Schätzen im Hause des \textsc{Herrn}
vorfand, und sandte es Hasael, dem König von Syrien. Da zog er ab von
Jerusalem. \bibleverse{19} Was aber mehr von Joas zu sagen ist, und
alles, was er getan hat, ist das nicht geschrieben in der Chronik der
Könige von Juda? \bibleverse{20} Und seine Knechte erhoben sich und
machten eine Verschwörung und erschlugen Joas im Hause Millo, da man
gegen Silla hinabgeht. \bibleverse{21} Denn Josachar, der Sohn Simeas,
und Josabad, der Sohn Somers, seine Knechte, erschlugen ihn, und er
starb; und man begrub ihn mit seinen Vätern in der Stadt Davids; und
Amazia, sein Sohn, ward König an seiner Statt.

\hypertarget{section-12}{%
\section{13}\label{section-12}}

\bibleverse{1} Im dreiundzwanzigsten Jahre des Joas, des Sohnes Ahasias,
des Königs von Juda, ward Joahas, der Sohn Jehus, König über Israel zu
Samaria, und regierte siebzehn Jahre lang. \bibleverse{2} Er tat, was
böse war in den Augen des \textsc{Herrn}, und wandelte in den Sünden
Jerobeams, des Sohnes Nebats, der Israel zur Sünde verführt hatte, und
ließ nicht davon. \bibleverse{3} Deswegen ergrimmte der Zorn des
\textsc{Herrn} über Israel, und er gab sie in die Hand Hasaels, des
Königs von Syrien, und in die Hand Benhadads, des Sohnes Hasaels, ihr
Leben lang. \bibleverse{4} Aber Joahas besänftigte das Angesicht des
\textsc{Herrn}, und der \textsc{Herr} erhörte ihn; denn er sah die
Bedrängnis Israels, wie der König von Syrien sie bedrängte.
\bibleverse{5} Und der \textsc{Herr} gab Israel einen Helfer, und sie
kamen aus der Gewalt der Syrer, und die Kinder Israel wohnten in ihren
Hütten wie zuvor. \bibleverse{6} Doch ließen sie nicht von den Sünden,
zu denen das Haus Jerobeams Israel verführt hatte, sondern wandelten
darin. Auch blieb die Aschera in Samaria stehen. \bibleverse{7} Von dem
Kriegsvolke ließ der \textsc{Herr} dem Joahas nicht mehr übrig als
fünfzig Reiter, zehn Wagen und zehntausend Mann Fußvolk; denn der König
von Syrien hatte sie vertilgt und sie gemacht wie Staub beim Dreschen.
\bibleverse{8} Was aber mehr von Joahas zu sagen ist, und alles, was er
getan hat, und seine Macht, ist das nicht geschrieben in der Chronik der
Könige von Israel? \bibleverse{9} Und Joahas entschlief mit seinen
Vätern, und man begrub ihn zu Samaria, und Joas, sein Sohn, ward König
an seiner Statt. \bibleverse{10} Im siebenunddreißigsten Jahre des
Königs Joas von Juda ward Joas, der Sohn des Joahas, König über Israel
zu Samaria, und regierte sechzehn Jahre lang. \bibleverse{11} Und er
tat, was böse war in den Augen des \textsc{Herrn}, und ließ nicht ab von
allen Sünden, zu denen Jerobeam, der Sohn Nebats, Israel verführt hatte,
sondern wandelte darin. \bibleverse{12} Was aber mehr von Joas zu sagen
ist und was er getan hat, und seine Macht, wie er mit Amazia, dem König
von Juda, gestritten, ist das nicht geschrieben in der Chronik der
Könige von Israel? \bibleverse{13} Und Joas legte sich zu seinen Vätern,
und Jerobeam setzte sich auf seinen Thron. Und Joas ward zu Samaria bei
den Königen von Israel begraben. \bibleverse{14} Elisa aber ward von der
Krankheit befallen, an der er sterben sollte. Und Joas, der König von
Israel, kam zu ihm hinab, weinte vor ihm und sprach: O mein Vater, mein
Vater! Wagen Israels und seine Reiter! \bibleverse{15} Elisa aber sprach
zu ihm: Nimm den Bogen und die Pfeile! Und als er den Bogen und die
Pfeile nahm, \bibleverse{16} sprach Elisa zum König von Israel: Spanne
mit deiner Hand den Bogen! Und er spannte ihn mit seiner Hand. Und Elisa
legte seine Hände auf die Hände des Königs und sprach: \bibleverse{17}
Tue das Fenster auf gegen Morgen! Und er tat es auf. Und Elisa sprach:
Schieße! Und er schoß. Er aber sprach: Ein Pfeil des Heils vom
\textsc{Herrn}, ein Pfeil des Heils wider die Syrer! Du wirst die Syrer
schlagen zu Aphek, bis sie aufgerieben sind! \bibleverse{18} Und er
sprach: Nimm die Pfeile! Und als er sie nahm, sprach er zum König von
Israel: Schlage auf die Erde! Da schlug er dreimal und hielt inne.
\bibleverse{19} Da ward der Mann Gottes zornig über ihn und sprach:
Hättest du fünf oder sechsmal geschlagen, so würdest du die Syrer bis
zur Vernichtung geschlagen haben; nun aber wirst du die Syrer nur
dreimal schlagen! \bibleverse{20} Und Elisa starb und ward begraben. Im
folgenden Jahre fielen die Streifscharen der Moabiter ins Land.
\bibleverse{21} Und es begab sich, als man einen Mann begrub, sahen sie
plötzlich die Streifschar kommen; da warfen sie den Mann in Elisas Grab.
Und sobald der Mann die Gebeine Elisas berührte, ward er lebendig und
stand auf seine Füße. \bibleverse{22} Hasael aber, der König von Syrien,
bedrängte Israel, solange Joahas lebte. \bibleverse{23} Aber der
\textsc{Herr} war ihnen gnädig und erbarmte sich ihrer und wandte sich
zu ihnen um seines Bundes willen mit Abraham, Isaak und Jakob; er wollte
sie nicht verderben und hatte sie bis dahin noch nicht von seinem
Angesichte verworfen. \bibleverse{24} Und Hasael, der König von Syrien,
starb, und sein Sohn Benhadad ward König an seiner Statt.
\bibleverse{25} Joas aber, der Sohn des Joahas, entriß der Hand
Benhadads, des Sohnes Hasaels, die Städte wieder, die dieser im Krieg
aus der Hand seines Vaters Joahas genommen hatte; dreimal schlug ihn
Joas und eroberte die Städte Israels zurück.

\hypertarget{section-13}{%
\section{14}\label{section-13}}

\bibleverse{1} Im zweiten Jahre des Joas, des Sohnes Joahas, des Königs
von Israel, ward Amazia König, der Sohn des Königs Joas von Juda.
\bibleverse{2} Mit fünfundzwanzig Jahren ward er König und regierte
neunundzwanzig Jahre lang zu Jerusalem. \bibleverse{3} Und er tat, was
dem \textsc{Herrn} wohlgefiel, doch nicht wie sein Vater David, sondern
ganz so, wie sein Vater Joas getan hatte. \bibleverse{4} Nur die Höhen
kamen nicht weg, sondern das Volk opferte und räucherte noch auf den
Höhen. \bibleverse{5} Sobald er nun die Herrschaft fest in Händen hatte,
erschlug er seine Knechte, die seinen königlichen Vater erschlagen
hatten. \bibleverse{6} Aber die Söhne der Mörder tötete er nicht, wie
denn im Gesetzbuch Moses geschrieben steht, wo der \textsc{Herr} geboten
und gesagt hat: Die Väter sollen nicht um der Söhne willen sterben, und
die Söhne sollen nicht um der Väter willen getötet werden; sondern ein
jeder soll um seiner Sünde willen sterben. \bibleverse{7} Er schlug auch
die Edomiter im Salztal, zehntausend Mann und gewann Sela im Kampfe und
hieß die Stadt Jokteel, wie sie heute noch heißt. \bibleverse{8} Darnach
sandte Amazia Boten zu dem König Joas von Israel, dem Sohne des Joahas,
des Sohnes Jehus, und ließ ihm sagen: Komm her, wir wollen uns ins
Angesicht sehen! \bibleverse{9} Aber Joas, der König von Israel, sandte
zu Amazia, dem König von Juda, und ließ ihm sagen: Der Dornstrauch am
Libanon sandte zur Zeder am Libanon und ließ ihr sagen: Gib deine
Tochter meinem Sohn zum Weibe! Aber das Wild auf dem Libanon lief über
den Dornstrauch und zertrat ihn. \bibleverse{10} Du hast die Edomiter
gänzlich geschlagen; dessen erhebt sich dein Herz. Trage Sorge zu deinem
Ruhm und bleibe daheim! Warum willst du dich ins Unglück stürzen, daß du
fallest und Juda mit dir? \bibleverse{11} Aber Amazia wollte nicht
hören. Da zog Joas, der König von Israel, herauf, und sie schauten sich
ins Angesicht, er und Amazia, der König von Juda, zu Beth-Semes, das in
Juda liegt. \bibleverse{12} Aber Juda ward vor Israel geschlagen, so daß
ein jeder in seine Hütte floh. \bibleverse{13} Und Joas, der König von
Israel, nahm Amazia, den König von Juda, den Sohn des Joas, des Sohnes
Ahasias, zu Beth-Semes gefangen und kam gen Jerusalem und riß die
Stadtmauern ein, vom Tor Ephraim an bis an das Ecktor, vierhundert Ellen
Länge. \bibleverse{14} Und er nahm alles Gold und Silber und alle
Geräte, welche im Hause des \textsc{Herrn} und in den Schätzen des
königlichen Hauses gefunden wurden, dazu Geiseln und kehrte wieder nach
Samaria zurück. \bibleverse{15} Was aber mehr von Joas zu sagen ist, was
er getan, und seine Macht, und wie er mit Amazia, dem König von Juda,
gestritten hat, ist das nicht geschrieben in der Chronik der Könige von
Israel? \bibleverse{16} Und Joas legte sich zu seinen Vätern und ward zu
Samaria bei den Königen von Israel begraben. Und Jerobeam, sein Sohn,
ward König an seiner Statt. \bibleverse{17} Amazia aber, der Sohn des
Joas, der König von Juda, lebte nach dem Tode des Königs Joas von
Israel, des Sohnes des Joahas, noch fünfzehn Jahre lang. \bibleverse{18}
Was aber Amazias weitere Geschichte betrifft, ist die nicht geschrieben
in der Chronik der Könige von Juda? \bibleverse{19} Und sie machten eine
Verschwörung wider ihn zu Jerusalem. Er aber floh gen Lachis. Da sandten
sie ihm nach gen Lachis und töteten ihn daselbst \bibleverse{20} und
brachten ihn auf Pferden, und er ward begraben in Jerusalem bei seinen
Vätern in der Stadt Davids. \bibleverse{21} Und das ganze Volk Juda nahm
Asaria in seinem sechzehnten Lebensjahre und machten ihn zum König an
Stelle seines Vaters Amazia. \bibleverse{22} Er baute Elat und brachte
es wieder an Juda, nachdem der König sich zu seinen Vätern gelegt hatte.
\bibleverse{23} Im fünfzehnten Jahre Amazias, des Sohnes des Joas, des
Königs von Juda, ward Jerobeam, der Sohn des Joas, König über Israel zu
Samaria, und regierte einundvierzig Jahre lang. \bibleverse{24} Er tat
aber, was dem \textsc{Herrn} übel gefiel, und ließ nicht ab von allen
Sünden Jerobeams, des Sohnes Nebats, der Israel zur Sünde verführt
hatte. \bibleverse{25} Dieser eroberte das Gebiet Israels zurück, von
Chamat an bis an das Meer der Ebene, nach dem Worte des \textsc{Herrn},
des Gottes Israels, das er geredet hatte durch seinen Knecht Jona, den
Sohn Amitais, den Propheten von Gat-Hepher. \bibleverse{26} Denn der
\textsc{Herr} sah das so bittere Elend Israels, daß Mündige und
Unmündige dahin waren und es keinen Helfer für Israel gab.
\bibleverse{27} Und der \textsc{Herr} hatte nicht gesagt, daß er den
Namen Israels unter dem Himmel austilgen wolle; deswegen half er ihnen
durch Jerobeam, den Sohn des Joas. \bibleverse{28} Was aber mehr von
Jerobeam zu sagen ist, und alles, was er getan, und seine Macht, wie er
gestritten und wie er Damaskus und Chamat, die zu Juda gehört hatten, an
Israel zurückgebracht hat, ist das nicht geschrieben in der Chronik der
Könige von Israel? \bibleverse{29} Und Jerobeam legte sich zu seinen
Vätern, den Königen von Israel. Und Sacharia, sein Sohn, ward König an
seiner Statt.

\hypertarget{section-14}{%
\section{15}\label{section-14}}

\bibleverse{1} Im siebenundzwanzigsten Jahre Jerobeams, des Königs von
Israel, ward Asaria König, der Sohn Amazias, des Königs in Juda.
\bibleverse{2} Mit sechzehn Jahren ward er König und regierte
zweiundfünfzig Jahre lang zu Jerusalem. \bibleverse{3} Und er tat, was
dem \textsc{Herrn} wohlgefiel, ganz wie sein Vater Amazia getan hatte;
\bibleverse{4} nur daß die Höhen nicht wegkamen; denn das Volk opferte
und räucherte noch auf den Höhen. \bibleverse{5} Der \textsc{Herr} aber
plagte den König, so daß er aussätzig ward bis an den Tag seines Todes,
und er wohnte in einem abgesonderten Hause. Jotam aber, der Sohn des
Königs, regierte den Palast und richtete das Volk des Landes.
\bibleverse{6} Was aber mehr von Asaria zu sagen ist, und alles, was er
getan hat, ist das nicht geschrieben in der Chronik der Könige von Juda?
\bibleverse{7} Und Asaria legte sich zu seinen Vätern, und man begrub
ihn bei seinen Vätern in der Stadt Davids, und Jotam, sein Sohn, ward
König an seiner Statt. \bibleverse{8} Im achtunddreißigsten Jahre
Asarias, des Königs von Juda, ward Sacharia, der Sohn Jerobeams, König
über Israel zu Samaria, und regierte sechs Monate lang. \bibleverse{9}
Der tat, was dem \textsc{Herrn} übel gefiel, wie seine Väter getan
hatten; er ließ nicht ab von den Sünden, zu denen Jerobeam, der Sohn
Nebats, Israel verführt hatte. \bibleverse{10} Und Sallum, der Sohn des
Jabes, machte eine Verschwörung wider ihn und schlug ihn vor dem Volk
und tötete ihn und ward König an seiner Statt. \bibleverse{11} Was aber
mehr von Sacharia zu sagen ist, siehe, das ist geschrieben in der
Chronik der Könige von Israel. \bibleverse{12} So erfüllte sich das
Wort, das der \textsc{Herr} zu Jehu gesagt hatte, als er sprach: Es
sollen Nachkommen von dir bis in das vierte Glied auf dem Throne Israels
sitzen! Es geschah also. \bibleverse{13} Sallum aber, der Sohn des
Jabes, ward König im neununddreißigsten Jahre Ussijas, des Königs von
Juda, und regierte einen Monat lang zu Samaria. \bibleverse{14} Da zog
Menachem, der Sohn Gadis, von Tirza herauf und kam nach Samaria und
schlug Sallum, den Sohn des Jabes, zu Samaria und tötete ihn; und er
ward König an seiner Statt. \bibleverse{15} Was aber mehr von Sallum zu
sagen ist und von seiner Verschwörung, die er gemacht hat, siehe, das
ist geschrieben in der Chronik der Könige von Israel. \bibleverse{16}
Dazumal schlug Menachem die Stadt Tiphsach und alle, die darin waren,
und ihr Gebiet von Tirza an; weil sie ihn nicht einlassen wollten, darum
schlug er sie und ließ alle aufschneiden, die guter Hoffnung waren.
\bibleverse{17} Im neununddreißigsten Jahre Asarias, des Königs von
Juda, ward Menachem, der Sohn Gadis, König über Israel zu Samaria, und
regierte zehn Jahre lang. \bibleverse{18} Und er tat, was böse war in
den Augen des \textsc{Herrn}; er ließ sein Leben lang nicht von den
Sünden, zu denen Jerobeam, der Sohn Nebats, Israel verführt hatte.
\bibleverse{19} Und Phul, der König von Assyrien, kam in das Land. Und
Menachem gab Phul tausend Talente Silber, damit er zu ihm halte und ihm
das Königreich bestätige. \bibleverse{20} Und Menachem erhob das Geld
von Israel, von allen begüterten Leuten, fünfzig Schekel Silber von
jedem Mann, um es dem König von Assyrien zu geben. Also zog der König
von Assyrien wieder heim und blieb nicht dort im Lande. \bibleverse{21}
Was aber mehr von Menachem zu sagen ist, und alles, was er getan hat,
ist das nicht geschrieben in der Chronik der Könige von Israel?
\bibleverse{22} Und Menachem legte sich zu seinen Vätern. Und Pekachja,
sein Sohn, ward König an seiner Statt. \bibleverse{23} Im fünfzigsten
Jahre Asarias, des Königs von Juda, ward Pekachja, der Sohn Menachems,
König über Israel zu Samaria, und regierte zwei Jahre lang.
\bibleverse{24} Und er tat, was dem \textsc{Herrn} übel gefiel; er ließ
nicht ab von den Sünden, zu denen Jerobeam, der Sohn Nebats, Israel
verführt hatte. \bibleverse{25} Pekach aber, der Sohn Remaljas, sein
Hauptmann, machte eine Verschwörung wider ihn und erschlug ihn zu
Samaria, in der Burg des königlichen Hauses, ebenso Argob und Arje. Mit
ihm aber waren fünfzig Mann von den Söhnen der Gileaditer. Und er tötete
ihn und ward König an seiner Statt. \bibleverse{26} Was aber mehr von
Pekachja zu sagen ist, und alles, was er getan hat, siehe, das ist
aufgezeichnet in der Chronik der Könige von Israel. \bibleverse{27} Im
zweiundfünfzigsten Jahre Asarias, des Königs von Juda, ward Pekach, der
Sohn Remaljas, König über Israel zu Samaria, und regierte zwanzig Jahre
lang. \bibleverse{28} Und er tat, was dem \textsc{Herrn} übel gefiel. Er
ließ nicht ab von den Sünden, zu denen Jerobeam, der Sohn Nebats, Israel
verführt hatte. \bibleverse{29} Zu den Zeiten Pekachs, des Königs von
Israel, kam Tiglat-Pileser, der König von Assyrien, und nahm Ijon,
Abel-Beth-Maacha, Janoach, Kedesch, Hazor, Gilead, Galiläa und das ganze
Land Naphtali ein und führte die Bewohner gefangen nach Assyrien.
\bibleverse{30} Und Hosea, der Sohn Elas, machte eine Verschwörung wider
Pekach, den Sohn Remaljas, schlug ihn tot und ward König an seiner Statt
im zwanzigsten Jahre Jotams, des Sohnes Ussijas. \bibleverse{31} Was
aber mehr von Pekach zu sagen ist, und alles, was er getan hat, siehe,
das ist geschrieben in der Chronik der Könige von Israel.
\bibleverse{32} Im zweiten Jahre Pekachs, des Sohnes Remaljas, des
Königs von Israel, ward Jotam König, der Sohn Ussijas, des Königs von
Juda. \bibleverse{33} Fünfundzwanzig Jahre alt war er, als er König
ward, und regierte sechzehn Jahre lang zu Jerusalem. Seine Mutter hieß
Jerusa, eine Tochter Zadoks. \bibleverse{34} Und er tat, was recht war
in den Augen des \textsc{Herrn}; ganz wie sein Vater Ussija getan hatte,
so tat auch er. \bibleverse{35} Nur daß die Höhen nicht wegkamen; denn
das Volk opferte und räucherte noch auf den Höhen. Er baute das obere
Tor am Hause des \textsc{Herrn}. \bibleverse{36} Was aber mehr von Jotam
zu sagen ist, und alles, was er getan hat, ist das nicht geschrieben in
der Chronik der Könige von Juda? \bibleverse{37} Zu derselben Zeit fing
der \textsc{Herr} an, Rezin, den König von Syrien, und Pekach, den Sohn
Remaljas, wider Juda zu senden. \bibleverse{38} Und Jotam legte sich zu
seinen Vätern und ward begraben bei seinen Vätern in der Stadt seines
Vaters David. Und Ahas, sein Sohn, ward König an seiner Statt.

\hypertarget{section-15}{%
\section{16}\label{section-15}}

\bibleverse{1} Im siebzehnten Jahre Pekachs, des Sohnes Remaljas, ward
Ahas König, der Sohn Jotams, des Königs in Juda. \bibleverse{2} Zwanzig
Jahre alt war Ahas, als er König ward, und regierte sechzehn Jahre lang
zu Jerusalem und tat nicht, was dem \textsc{Herrn}, seinem Gott,
wohlgefiel, wie sein Vater David. \bibleverse{3} Denn er wandelte auf
dem Wege der Könige von Israel; dazu ließ er seinen Sohn durchs Feuer
gehen nach den Greueln der Heiden, die der \textsc{Herr} vor den Kindern
Israel vertrieben hatte. \bibleverse{4} Und er opferte und räucherte auf
den Höhen und auf den Hügeln und unter allen grünen Bäumen.
\bibleverse{5} Da zogen Rezin, der König von Syrien, und Pekach, der
Sohn Remaljas, der König von Israel, zum Kampfe herauf wider Jerusalem
und belagerten Ahas, konnten die Stadt aber nicht erstürmen.
\bibleverse{6} Zu jener Zeit brachte Rezin, der König von Syrien, Elat
wieder an Edom; denn er vertrieb die Juden aus Elat; und Syrer kamen gen
Elat und wohnten darin bis auf diesen Tag. \bibleverse{7} Ahas aber
sandte Boten zu Tiglat-Pileser, dem König von Assyrien, und ließ ihm
sagen: Ich bin dein Knecht und dein Sohn; komm herauf und errette mich
aus der Hand des Königs von Syrien und aus der Hand des Königs von
Israel, die sich wider mich aufgemacht haben! \bibleverse{8} Und Ahas
nahm das Silber und das Gold, das sich im Hause des \textsc{Herrn} und
in den Schätzen des königlichen Hauses vorfand, und sandte es dem König
von Assyrien zum Geschenk. \bibleverse{9} Und der König von Assyrien
willfahrte ihm. Und der König von Assyrien zog herauf gen Damaskus und
nahm es ein und führte die Leute gefangen nach Kir und tötete Rezin.
\bibleverse{10} Da zog der König Ahas Tiglat-Pileser, dem König von
Assyrien, entgegen nach Damaskus. Und als er den Altar sah, der zu
Damaskus war, sandte der König Ahas das Modell des Altars und eine
genaue Abbildung, wie er gemacht war, dem Priester Urija.
\bibleverse{11} Und der Priester Urija baute den Altar genau so, wie der
König Ahas von Damaskus aus befohlen hatte; so verfertigte ihn der
Priester Urija, ehe der König Ahas von Damaskus kam. \bibleverse{12} Und
als der König von Damaskus kam und den Altar sah, trat er zum Altar und
opferte darauf \bibleverse{13} und zündete darauf sein Brandopfer und
sein Speisopfer an und goß sein Trankopfer darauf und ließ das Blut der
Dankopfer, die er darbrachte, auf den Altar sprengen. \bibleverse{14}
Aber den ehernen Altar, der vor dem \textsc{Herrn} stand, rückte er von
der Vorderseite des Hauses weg aus dem Zwischenraum zwischen dem neuen
Altar und dem Hause des \textsc{Herrn} und stellte ihn nördlich vom
Altar auf. \bibleverse{15} Und der König Ahas gebot dem Priester Urija
und sprach: Auf dem großen Altar sollst du das Brandopfer anzünden am
Morgen und das Speisopfer am Abend und das Brandopfer des Königs und
sein Speisopfer, auch das Brandopfer aller Leute im Lande samt ihrem
Speisopfer und ihren Trankopfern; und alles Blut des Brandopfers und
alles andere Opferblut sollst du daraufsprengen; wegen des ehernen
Altars aber will ich mich noch bedenken. \bibleverse{16} Und der
Priester Urija machte alles genau, wie ihm der König Ahas befahl.
\bibleverse{17} Der König Ahas ließ auch die Seitenfelder an den
Ständern herausbrechen und den Kessel oben hinwegtun; und das Meer nahm
er von den ehernen Rindern, die darunter waren, herab und setzte es auf
ein steinernes Pflaster. \bibleverse{18} Auch die Sabbathalle, die man
am Hause gebaut hatte, und den äußern Eingang des Königs verlegte er am
Hause des \textsc{Herrn} wegen des Königs von Assyrien. \bibleverse{19}
Was aber mehr von Ahas zu sagen ist, was er getan hat, ist das nicht
geschrieben in der Chronik der Könige von Juda? \bibleverse{20} Und Ahas
legte sich zu seinen Vätern und ward begraben bei seinen Vätern in der
Stadt Davids. Und Hiskia, sein Sohn, ward König an seiner Statt.

\hypertarget{section-16}{%
\section{17}\label{section-16}}

\bibleverse{1} Im zwölften Jahre Ahas, des Königs von Juda, ward Hosea,
der Sohn Elas, König über Israel zu Samaria, und regierte neun Jahre
lang; \bibleverse{2} und er tat, was dem \textsc{Herrn} übel gefiel,
doch nicht wie die Könige von Israel, die vor ihm gewesen.
\bibleverse{3} Wider denselben zog Salmanassar, der König von Assyrien,
herauf; und Hosea ward ihm untertan und zahlte ihm Tribut.
\bibleverse{4} Als aber der König von Assyrien erfuhr, daß Hosea eine
Verschwörung gemacht und Boten zu So, dem König von Ägypten, gesandt und
dem König von Assyrien nicht wie alle Jahre Tribut gezahlt hatte,
verhaftete er ihn und legte ihn gebunden ins Gefängnis. \bibleverse{5}
Und der König von Assyrien durchzog das ganze Land und kam vor Samaria
und belagerte es drei Jahre lang. \bibleverse{6} Im neunten Jahre Hoseas
eroberte der König von Assyrien Samaria und führte Israel gefangen nach
Assyrien und siedelte sie in Chalach und Chabor, am Flusse Gosan, und in
den Städten der Meder an. \bibleverse{7} Solches geschah darum, weil die
Kinder Israel gesündigt hatten wider den \textsc{Herrn}, ihren Gott, der
sie aus Ägyptenland, aus der Hand des Pharao, des Königs von Ägypten,
geführt hatte, und weil sie andere Götter fürchteten \bibleverse{8} und
nach den Satzungen der Heiden wandelten, welche der \textsc{Herr} vor
den Kindern Israel vertrieben hatte, und nach den Satzungen der Könige
von Israel, welche diese gemacht hatten. \bibleverse{9} So hatten die
Kinder Israel wider den \textsc{Herrn}, ihren Gott, heimlich Dinge
getrieben, die nicht recht waren: sie bauten sich Höhen an allen ihren
Wohnorten, von den Wachttürmen bis zu den festen Städten,
\bibleverse{10} errichteten sich Säulen und Ascheren auf allen hohen
Hügeln und unter allen grünen Bäumen, \bibleverse{11} räucherten auf
allen Höhen wie die Heiden, welche der \textsc{Herr} vor ihnen
vertrieben hatte, und trieben böse Dinge und erzürnten damit den
\textsc{Herrn}, \bibleverse{12} und sie dienten den Götzen, wovon der
\textsc{Herr} ihnen gesagt hatte: Ihr sollt solches nicht tun!
\bibleverse{13} Ja, wenn der \textsc{Herr} gegen Israel und Juda durch
alle Propheten und alle Seher zeugte, indem er ihnen sagen ließ: Wendet
euch ab von euren bösen Wegen und haltet meine Gebote und meine
Satzungen nach all dem Gesetz, das ich euren Vätern geboten habe und das
ich durch meine Knechte, die Propheten, zu euch gesandt habe,
\bibleverse{14} so gehorchten sie nicht, sondern machten ihren Nacken
hart, gleich dem Nacken ihrer Väter, die nicht an den \textsc{Herrn},
ihren Gott, geglaubt hatten. \bibleverse{15} Dazu verachteten sie seine
Satzungen und seinen Bund, den er mit ihren Vätern geschlossen, und
seine Zeugnisse, die er wider sie abgelegt hatte; und wandelten der
Eitelkeit nach und wurden eitel; und folgten den Heiden nach, die um sie
her wohnten, betreffs derer der \textsc{Herr} ihnen geboten hatte, sie
sollten nicht tun wie diese. \bibleverse{16} Aber sie verließen alle
Gebote des \textsc{Herrn}, ihres Gottes, und machten sich zwei gegossene
Kälber und machten Ascheren und beteten das ganze Heer des Himmels an
und dienten dem Baal; \bibleverse{17} und ließen ihre Söhne und ihre
Töchter durchs Feuer gehen und gaben sich ab mit Wahrsagen und
Zeichendeuterei und verkauften sich, zu tun, was böse war in den Augen
des \textsc{Herrn}, um ihn zu erzürnen. \bibleverse{18} Da ward der
\textsc{Herr} zornig über Israel und tat sie von seinem Angesicht weg,
so daß nur der Stamm Juda übrigblieb. \bibleverse{19} Aber auch Juda
beobachtete die Gebote des \textsc{Herrn}, seines Gottes, nicht, sondern
wandelte nach den Gebräuchen Israels, die sie getan hatten.
\bibleverse{20} Darum verwarf der \textsc{Herr} den ganzen Samen Israels
und demütigte sie und gab sie in die Hände der Räuber, bis er sie von
seinem Angesicht verstieß. \bibleverse{21} Denn Israel hatte sich vom
Hause Davids losgerissen und hatte Jerobeam, den Sohn Nebats, zum König
gemacht; und Jerobeam wandte Israel ab von der Nachfolge des
\textsc{Herrn} und verführte es zu schwerer Sünde; \bibleverse{22} und
die Kinder Israel wandelten in allen Sünden Jerobeams, die er getan
hatte, und ließen nicht davon, \bibleverse{23} bis der \textsc{Herr}
Israel von seinem Angesicht verwarf, wie er durch alle seine Knechte,
die Propheten, gesagt hatte. Also ward Israel aus seinem Lande nach
Assyrien weggeführt, bis auf diesen Tag. \bibleverse{24} Aber der König
von Assyrien ließ Leute von Babel, Kuta, Awa, Chamat und Sepharwaim
kommen und siedelte sie an Stelle der Kinder Israel in den Städten
Samarias an. Und sie nahmen Samaria ein und wohnten in dessen Städten.
\bibleverse{25} Als sie aber anfingen daselbst zu wohnen und den
\textsc{Herrn} nicht fürchteten, sandte der \textsc{Herr} Löwen unter
sie; die richteten Verheerung unter ihnen an. \bibleverse{26} Darum
ließen sie dem König von Assyrien sagen: Die Völker, welche du
hergebracht und in den Städten Samarias angesiedelt hast, kennen das
Recht des Landesgottes nicht, darum hat er Löwen unter sie gesandt; und
siehe, diese töten sie, weil sie das Recht des Landesgottes nicht
kennen! \bibleverse{27} Da befahl der König von Assyrien und sprach:
Bringet einen der Priester dahin, die ihr von dort weggeführt habt; der
soll hinziehen und daselbst wohnen; und er soll sie das Recht des
Landesgottes lehren! \bibleverse{28} Da kam einer von den Priestern, die
sie von Samaria weggeführt hatten, und ließ sich zu Bethel nieder und
lehrte sie, wie sie den \textsc{Herrn} fürchten sollten. \bibleverse{29}
Aber ein jedes Volk machte seine eigenen Götter und tat sie in die
Höhenhäuser, welche die Samariter gemacht hatten. \bibleverse{30} Die
Leute von Babel machten Sukkot-Benot, und die Leute von Kut machten
Nergal, und die Leute von Charmat machten Aschima; \bibleverse{31} und
die von Awa machten Nibchas und Tartak; aber die von Sepharwaim
verbrannten ihre Söhne mit Feuer dem Adrammelech und dem Anammelech, den
Göttern von Sepharwaim. \bibleverse{32} Doch verehrten sie auch den
\textsc{Herrn} und bestellten sich Höhenpriester aus dem gesamten Volk,
die für sie in den Höhenhäusern opferten. \bibleverse{33} Also verehrten
sie den \textsc{Herrn} und dienten auch ihren Göttern nach der
Gewohnheit eines jeden Volkes, von welchem sie hergebracht waren.
\bibleverse{34} Und bis auf diesen Tag tun sie nach der früheren Weise;
sie fürchten den \textsc{Herrn} nicht; sie tun auch nicht nach ihren
Satzungen und Ordnungen, noch nach dem Gesetz und Gebot, welches der
\textsc{Herr} den Kindern Jakobs geboten hat, dem er den Namen Israel
gab. \bibleverse{35} Und doch hat der \textsc{Herr} mit ihnen einen Bund
gemacht und ihnen geboten und gesagt: Fürchtet keine anderen Götter,
betet sie nicht an, dienet ihnen nicht und opfert ihnen nicht,
\bibleverse{36} sondern den \textsc{Herrn}, der euch mit großer Kraft
und ausgestrecktem Arm aus Ägyptenland geführt hat, den sollt ihr
fürchten, ihn betet an, ihm sollt ihr opfern! \bibleverse{37} Die
Satzungen, Rechte, Gesetze und Gebote, die er euch vorgeschrieben hat,
sollt ihr beobachten, daß ihr darnach tuet immerdar; und fürchtet nicht
andere Götter! \bibleverse{38} Und den Bund, den ich mit euch
geschlossen habe, vergesset nicht und fürchtet nicht andere Götter,
\bibleverse{39} sondern fürchtet den \textsc{Herrn}, euren Gott, der
wird euch von der Hand aller eurer Feinde erretten! \bibleverse{40} Aber
sie gehorchten nicht, sondern tun nach ihrer früheren Weise.
\bibleverse{41} So kam es, daß diese Völker den \textsc{Herrn} verehrten
und zugleich ihren Götzen dienten; auch ihre Kinder und ihre
Kindeskinder tun so, wie ihre Väter getan haben, bis auf diesen Tag.

\hypertarget{section-17}{%
\section{18}\label{section-17}}

\bibleverse{1} Im dritten Jahre Hoseas, des Sohnes Elas, des Königs von
Israel, ward Hiskia König, der Sohn des Ahas, des Königs von Juda.
\bibleverse{2} Mit fünfundzwanzig Jahren ward er König und regierte
neunundzwanzig Jahre lang zu Jerusalem. Seine Mutter hieß Abia, eine
Tochter Sacharias. \bibleverse{3} Und er tat, was dem \textsc{Herrn}
wohlgefiel, ganz wie sein Vater David getan hatte. \bibleverse{4} Er tat
die Höhen ab und zerbrach die Säulen und hieb die Ascheren um und
zerstieß die eherne Schlange, welche Mose gemacht hatte; denn bis zu
dieser Zeit hatten die Kinder Israel ihr geräuchert, und man hieß sie
Nechuschtan. \bibleverse{5} Er vertraute dem \textsc{Herrn}, dem Gott
Israels, so daß unter allen Königen von Juda keiner seinesgleichen war,
weder nach ihm noch vor ihm. \bibleverse{6} Er hing dem \textsc{Herrn}
an, wich nicht von ihm ab und beobachtete die Gebote, welche der
\textsc{Herr} dem Mose geboten hatte. \bibleverse{7} Und der
\textsc{Herr} war mit ihm; und wo er hinzog, handelte er weislich. Er
fiel auch ab von dem assyrischen König und diente ihm nicht.
\bibleverse{8} Und er schlug die Philister bis hin nach Gaza und dessen
Gebiet, vom Wachtturm bis an die festen Städte. \bibleverse{9} Es
geschah aber im vierten Jahr des Königs Hiskia (das war das siebente
Jahr Hoseas, des Sohnes Elas, des Königs von Israel), da zog
Salmanassar, der König von Assyrien, wider Samaria herauf und belagerte
es. \bibleverse{10} Und sie eroberten es nach drei Jahren; im sechsten
Jahre Hiskias (das ist das neunte Jahr Hoseas, des Königs von Israel)
ward Samaria genommen. \bibleverse{11} Und der König von Assyrien führte
Israel nach Assyrien hinweg und siedelte sie in Chalach und Chabor, am
Flusse Gosan, und in den Städten der Meder an, \bibleverse{12} weil sie
der Stimme des \textsc{Herrn}, ihres Gottes, nicht gehorcht und seinen
Bund gebrochen hatten, alles, was Mose, der Knecht des \textsc{Herrn},
geboten; sie hatten nicht darauf gehört und es nicht getan.
\bibleverse{13} Aber im vierzehnten Jahre des Königs Hiskia zog
Sanherib, der König von Assyrien, herauf wider alle festen Städte Judas
und nahm sie ein. \bibleverse{14} Da sandte Hiskia, der König von Juda,
Boten zum König von Assyrien nach Lachis und ließ ihm sagen: Ich habe
mich versündigt! Ziehe ab von mir; was du mir auferlegst, will ich
tragen! Da legte der König von Assyrien Hiskia, dem König von Juda,
dreihundert Talente Silber und dreißig Talente Gold auf. \bibleverse{15}
Und Hiskia gab ihm alles Silber, das sich im Hause des \textsc{Herrn}
und in den Schätzen des königlichen Hauses vorfand. \bibleverse{16} Zu
jener Zeit ließ Hiskia, der König von Juda, das Gold abschneiden von den
Türen am Tempel des \textsc{Herrn} und von den Pfosten, die er selbst
hatte überziehen lassen, und gab es dem König von Assyrien.
\bibleverse{17} Und der König von Assyrien sandte den Tartan und den
Rabsaris und den Rabschake mit großer Macht von Lachis aus zum König
Hiskia gen Jerusalem. Und sie zogen herauf, und als sie vor Jerusalem
anlangten, hielten sie an der Wasserleitung des obern Teiches, die an
der Straße des Walkerfeldes liegt; und sie riefen den König.
\bibleverse{18} Da gingen zu ihnen hinaus Eljakim, der Sohn Hilkias, der
über das Haus gesetzt war, und Sebna, der Schreiber, und Joah, der Sohn
Asaphs, der Kanzler. \bibleverse{19} Und Rabschake sprach zu ihnen:
Saget doch dem Hiskia: So spricht der große König, der König von
Assyrien: Was ist das für ein Trost, darauf du dich vertröstest?
\bibleverse{20} Wenn du sagst: «Es ist Rat und Macht zum Krieg
vorhanden», so ist das leeres Geschwätz! Auf wen vertraust du denn, daß
du von mir abtrünnig geworden bist? \bibleverse{21} Siehe, du vertraust
jetzt auf jenen zerbrochenen Rohrstab, auf Ägypten, welcher jedem, der
sich darauf lehnt, in die Hand fährt und sie durchbohrt! Also ist der
Pharao, der König von Ägypten, allen denen, die auf ihn vertrauen!
\bibleverse{22} Wenn ihr mir aber sagen wolltet: Wir vertrauen auf den
\textsc{Herrn}, unsern Gott! ist das nicht der, dessen Höhen und Altäre
Hiskia abgetan hat, während er zu Juda und zu Jerusalem sprach: Vor
diesem Altar sollt ihr anbeten zu Jerusalem? \bibleverse{23} Wette doch
jetzt einmal mit meinem Herrn, dem König von Assyrien: ich will dir
zweitausend Pferde geben, wenn du die Reiter dazu stellen kannst!
\bibleverse{24} Wie wolltest du auch nur einem der geringsten Fürsten
von meines Herrn Knechten begegnen? Doch du vertraust ja auf Ägypten,
wegen der Wagen und Reiter! \bibleverse{25} Bin ich nun aber etwa ohne
den \textsc{Herrn} gegen diesen Ort heraufgezogen, ihn zu verderben? Der
\textsc{Herr} hat zu mir gesagt: Ziehe wider dieses Land hinauf und
verderbe es! \bibleverse{26} Da sprachen Eljakim, der Sohn Hilkias, und
Sebna und Joah zu Rabschake: Rede doch mit deinen Knechten aramäisch;
denn wir verstehen es, und rede nicht judäisch mit uns vor den Ohren des
Volkes, das auf der Mauer ist! \bibleverse{27} Rabschake aber sprach zu
ihnen: Hat mich denn mein Herr zu deinem Herrn oder zu dir gesandt,
solche Worte zu reden, und nicht vielmehr zu den Männern, die auf der
Mauer sitzen, um mit euch ihren Kot zu essen und ihren Harn zu trinken?
\bibleverse{28} Und Rabschake trat vor und rief mit lauter Stimme auf
judäisch, redete und sprach: Hört das Wort des großen Königs, des Königs
von Assyrien! \bibleverse{29} So spricht der König: Laßt euch von Hiskia
nicht verführen; denn er kann euch nicht aus meiner Hand erretten!
\bibleverse{30} Laßt euch auch von Hiskia nicht auf den \textsc{Herrn}
vertrösten, wenn er sagt: Der \textsc{Herr} wird uns gewiß erretten, und
diese Stadt wird nicht in die Hand des Königs von Assyrien gegeben
werden! \bibleverse{31} Höret nicht auf Hiskia; denn also spricht der
König von Assyrien: Macht Frieden mit mir und kommt zu mir heraus; so
soll ein jeder von seinem Weinstock und von seinem Feigenbaum essen und
das Wasser seines Brunnens trinken, \bibleverse{32} bis daß ich komme
und euch in das Land hole, das eurem Lande gleich ist; ein Land von Korn
und Most, ein Land von Brot und Weinbergen, ein Land von Ölbäumen und
Honig; so werdet ihr am Leben bleiben und nicht sterben. Hört nicht auf
Hiskia; denn er verführt euch, wenn er sagt: Der \textsc{Herr} wird uns
erretten! \bibleverse{33} Haben auch die Götter der Völker ein jeder
sein Land aus der Hand des Königs von Assyrien errettet? \bibleverse{34}
Wo sind die Götter zu Chamat und Arpad? Wo sind die Götter zu
Sepharwaim, Hena und Iwa? Haben sie etwa Samaria von meiner Hand
errettet? \bibleverse{35} Wo ist einer unter allen Göttern der Länder,
der sein Land aus meiner Hand errettet hätte, daß der \textsc{Herr}
Jerusalem aus meiner Hand erretten sollte! \bibleverse{36} Das Volk aber
schwieg still und antwortete ihm nichts; denn der König hatte geboten
und gesagt: Antwortet ihm nichts! \bibleverse{37} Da kamen Eljakim, der
Sohn Hilkias, der über das Haus gesetzt war, und Sebna, der Schreiber,
und Joah, der Sohn Asaphs, der Kanzler, zu Hiskia, mit zerrissenen
Kleidern und meldeten ihm die Worte Rabschakes.

\hypertarget{section-18}{%
\section{19}\label{section-18}}

\bibleverse{1} Als aber der König Hiskia solches hörte, zerriß er seine
Kleider, legte einen Sack an und ging in das Haus des \textsc{Herrn}.
\bibleverse{2} Und er sandte Eljakim, der über das Haus gesetzt war, und
Sebna, den Schreiber, und die ältesten Priester, mit Säcken angetan, zum
Propheten Jesaja, dem Sohne des Amoz. \bibleverse{3} Und sie sprachen zu
ihm: So spricht Hiskia: Das ist ein Tag der Not und der Vorwürfe und ein
Tag der Schmach; denn die Kinder sind bis zum Durchbruch gekommen, aber
da ist keine Kraft zur Geburt! \bibleverse{4} Vielleicht wird der
\textsc{Herr}, dein Gott, hören alle Worte Rabschakes, den sein Herr,
der König von Assyrien, gesandt hat, den lebendigen Gott zu höhnen, und
wird die Reden strafen, welche der \textsc{Herr}, dein Gott, gehört hat.
So wollest du denn Fürbitte einlegen für den Rest, der noch vorhanden
ist! \bibleverse{5} Und als die Knechte des Königs Hiskia zu Jesaja
kamen, \bibleverse{6} sprach Jesaja zu ihnen: Also sollt ihr eurem Herrn
sagen: So spricht der \textsc{Herr}: Fürchte dich nicht vor den Worten,
die du gehört hast, womit die Knaben des Königs von Assyrien mich
gelästert haben! \bibleverse{7} Siehe, ich will ihm einen Geist
eingeben, daß er ein Gerücht hören und wieder in sein Land ziehen wird,
und ich will ihn in seinem Lande durch das Schwert fällen!
\bibleverse{8} Und als Rabschake zurückkehrte, fand er den König von
Assyrien im Kampfe wider Libna; denn er hatte gehört, daß er von Lachis
abgezogen war. \bibleverse{9} Da hörte Sanherib in betreff Tirhakas, des
Königs von Äthiopien, sagen: Siehe, er ist ausgezogen, mit dir zu
streiten! Da sandte er nochmals Boten zu Hiskia und sprach:
\bibleverse{10} Redet mit Hiskia, dem König von Juda, und saget ihm: Laß
dich von deinem Gott, auf den du dich verlässest, nicht verführen, indem
du sprichst: Jerusalem wird nicht in die Hand des Königs von Assyrien
gegeben werden! \bibleverse{11} Siehe, du hast gehört, was die Könige
von Assyrien allen Ländern getan, wie sie den Bann an ihnen vollstreckt
haben; und du solltest errettet werden? \bibleverse{12} Haben die Götter
der Heiden auch die errettet, welche meine Väter vernichtet haben,
nämlich Gosan, Haran, Rezeph und die Kinder von Eden zu Telassar?
\bibleverse{13} Wo ist der König zu Chamat und der König zu Arpad und
der König der Stadt Sepharwaim, Hena und Iwa? \bibleverse{14} Als nun
Hiskia den Brief aus der Hand der Boten empfangen und gelesen hatte,
ging er zum Hause des \textsc{Herrn} hinauf, und Hiskia breitete ihn aus
vor dem \textsc{Herrn}. \bibleverse{15} Darnach betete Hiskia vor dem
\textsc{Herrn} und sprach: O \textsc{Herr}, Gott Israels, der du über
den Cherubim thronst, du bist allein der Gott über alle Königreiche auf
Erden! Du hast den Himmel und die Erde gemacht. \bibleverse{16}
\textsc{Herr}, neige dein Ohr und höre! Tue deine Augen auf, o
\textsc{Herr}, und siehe! Vernimm die Worte Sanheribs, der hierher
gesandt hat, den lebendigen Gott zu schmähen! \bibleverse{17} Es ist
wahr, \textsc{Herr}, die Könige von Assyrien haben die Völker und ihre
Länder verwüstet \bibleverse{18} und haben ihre Götter ins Feuer
geworfen; denn sie waren nicht Götter, sondern Werke von Menschenhand,
Holz und Stein; darum haben sie sie verderbt. \bibleverse{19} Nun aber,
\textsc{Herr}, unser Gott, hilf uns doch aus seiner Hand, damit alle
Königreiche auf Erden erkennen, daß du, \textsc{Herr}, allein Gott bist!
\bibleverse{20} Da sandte Jesaja, der Sohn des Amoz, zu Hiskia und ließ
ihm sagen: So spricht der \textsc{Herr}, der Gott Israels: Was du wegen
Sanheribs, des Königs von Assyrien, zu mir gebetet hast, das habe ich
gehört. \bibleverse{21} Dies ist das Wort, welches der \textsc{Herr}
wider ihn geredet hat: Die Jungfrau, die Tochter Zion, verachtet dich
und spottet dein! Die Tochter Jerusalem schüttelt das Haupt über dich!
\bibleverse{22} Wen hast du geschmäht und gelästert? Und gegen wen hast
du deine Stimme erhoben und deine Augen stolz emporgeschlagen? Gegen den
Heiligen Israels! \bibleverse{23} Du hast durch deine Boten den
\textsc{Herrn} geschmäht und gesagt: Ich bin mit der Menge meiner Wagen
auf die Höhen der Berge gestiegen, an die Seiten des Libanon. Und ich
will seine hohen Zedern und seine auserlesenen Zypressen abhauen und in
seine äußerste Herberge, zum Walde seines Lustgartens kommen.
\bibleverse{24} Ich habe fremde Wasser gegraben und ausgetrunken und
trockne mit meinen Fußsohlen alle Ströme Ägyptens! \bibleverse{25} Hast
du es aber nicht gehört, daß ich solches längst vorbereitet und von
Anfang bestimmt habe? Nun aber habe ich es kommen lassen, daß du feste
Städte zerstörtest zu wüsten Steinhaufen. \bibleverse{26} Und die darin
wohnten, deren Hand zu schwach war, erschraken und wurden zuschanden;
sie wurden wie das Gras auf dem Felde und wie grünes Kraut und Gras auf
den Dächern und verwelktes Getreide, ehe es Halme gewinnt.
\bibleverse{27} Ich weiß dein Wohnen und dein Aus und Einziehen und daß
du wider mich tobst. \bibleverse{28} Weil du denn wider mich tobst und
dein Übermut vor meine Ohren heraufgekommen ist, so will ich dir meinen
Ring in die Nase legen und mein Gebiß ins Maul und will dich den Weg
zurückführen, den du gekommen bist! \bibleverse{29} Und das soll dir,
Hiskia, zum Zeichen sein: In diesem Jahre werdet ihr Brachwuchs essen,
und im zweiten Jahre, was von selbst wachsen wird; im dritten Jahre aber
sollt ihr säen und ernten und Weinberge pflanzen und deren Früchte
essen! \bibleverse{30} Und was vom Hause Juda entronnen und
übriggeblieben ist, wird forthin unter sich Wurzel schlagen und über
sich Früchte tragen; \bibleverse{31} denn von Jerusalem wird ein
Überrest ausgehen und Entronnene vom Berge Zion. Der Eifer des
\textsc{Herrn} der Heerscharen wird solches tun! \bibleverse{32} Darum
spricht der \textsc{Herr} von dem assyrischen König also: Er soll nicht
in diese Stadt hineinkommen und keinen Pfeil darein schießen und mit
keinem Schilde gegen sie anrücken und keinen Wall gegen sie aufwerfen.
\bibleverse{33} Auf dem Wege, den er gekommen ist, soll er wieder
zurückkehren und in diese Stadt nicht eindringen; der \textsc{Herr} sagt
es! \bibleverse{34} Und ich will diese Stadt beschirmen, daß ich ihr
helfe um meinetwillen und um meines Knechtes David willen.
\bibleverse{35} Und es begab sich in derselben Nacht, da ging der Engel
des \textsc{Herrn} aus und erschlug im Lager der Assyrer 185000 Mann.
Und als man sich am Morgen früh aufmachte, siehe, da waren sie alle tot,
lauter Leichen. \bibleverse{36} Da brach Sanherib, der König von
Assyrien, auf und ging weg und kehrte zurück und blieb zu Ninive.
\bibleverse{37} Und als er im Hause seines Gottes Nisroch anbetete,
erschlugen ihn seine Söhne Adrammalech und Sarezer mit dem Schwerte, und
sie entrannen ins Land Ararat. Und sein Sohn Esarhaddon ward König an
seiner Statt.

\hypertarget{section-19}{%
\section{20}\label{section-19}}

\bibleverse{1} Zu der Zeit ward Hiskia todkrank. Und der Prophet Jesaja,
der Sohn des Amoz, kam und sprach zu ihm: So spricht der \textsc{Herr}:
Bestelle dein Haus; denn du wirst sterben und nicht lebendig bleiben!
\bibleverse{2} Er aber wandte sein Angesicht gegen die Wand, betete zum
\textsc{Herrn} und sprach: \bibleverse{3} Ach, \textsc{Herr}, gedenke
doch, daß ich in Wahrheit und von ganzem Herzen vor dir gewandelt und
getan habe, was gut ist in deinen Augen! \bibleverse{4} Und Hiskia
weinte sehr. Als aber Jesaja noch nicht zur mittleren Stadt
hinausgegangen war, erging das Wort des \textsc{Herrn} an ihn und
sprach: \bibleverse{5} Kehre um und sage zu Hiskia, dem Fürsten meines
Volks: So spricht der \textsc{Herr}, der Gott deines Vaters David: Ich
habe dein Gebet erhört und deine Tränen gesehen. Siehe, ich will dich
gesund machen; am dritten Tage wirst du in das Haus des \textsc{Herrn}
hinaufgehen; \bibleverse{6} und ich will fünfzehn Jahre zu deinem Leben
hinzutun und dich und diese Stadt von der Hand des Königs von Assyrien
erretten und diese Stadt beschirmen, um meinetwillen und um meines
Knechtes David willen. \bibleverse{7} Und Jesaja sprach: Bringt eine
getrocknete Feigenmasse her! Und als sie eine solche brachten, legten
sie dieselbe auf das Geschwür; und er ward gesund. \bibleverse{8} Hiskia
aber sprach zu Jesaja: Welches ist das Zeichen, daß mich der
\textsc{Herr} gesund machen wird und daß ich am dritten Tage in das Haus
des \textsc{Herrn} hinaufgehen werde? \bibleverse{9} Jesaja sprach: Dies
sei dir das Zeichen vom \textsc{Herrn}, daß der \textsc{Herr} tun wird,
was er gesagt hat: Soll der Schatten zehn Stufen vorwärtsgehen, oder
zehn Stufen zurückkehren? \bibleverse{10} Hiskia sprach: Es ist ein
Leichtes, daß der Schatten zehn Stufen abwärtsgehe; nicht also, sondern
der Schatten soll zehn Stufen zurückgehen! \bibleverse{11} Da rief der
Prophet Jesaja den \textsc{Herrn} an; und er ließ an dem Sonnenzeiger
des Ahas den Schatten, welcher abwärts gegangen war, um zehn Stufen
zurückgehen. \bibleverse{12} Zu der Zeit sandte Berodach-Baladan, der
Sohn Baladans, König zu Babel, Briefe und Geschenke zu Hiskia; denn er
hatte gehört, daß Hiskia krank gewesen. \bibleverse{13} Hiskia aber
schenkte ihnen Gehör und zeigte ihnen sein ganzes Schatzhaus, das Silber
und das Gold und die Spezereien und das beste Öl und das Zeughaus und
alles, was in seinen Schatzhäusern vorhanden war. Es war nichts in
seinem Hause und in seiner ganzen Herrschaft, das Hiskia ihnen nicht
zeigte. \bibleverse{14} Da kam der Prophet Jesaja zum König Hiskia und
sprach zu ihm: Was haben diese Leute gesagt? Und woher sind sie zu dir
gekommen? Hiskia sprach: Sie sind aus fernem Lande zu mir gekommen, von
Babel. \bibleverse{15} Er sprach: Was haben sie in deinem Hause gesehen?
Hiskia sprach: Sie haben alles gesehen, was in meinem Hause ist, und es
ist nichts in meinen Schatzhäusern, was ich ihnen nicht gezeigt habe.
\bibleverse{16} Da sprach Jesaja zu Hiskia: \bibleverse{17} Höre das
Wort des \textsc{Herrn}! Siehe, es kommt die Zeit, daß alles, was in
deinem Hause ist und was deine Väter bis auf diesen Tag gesammelt haben,
gen Babel hinweggetragen werden wird; es wird nichts übriggelassen
werden, spricht der \textsc{Herr}! \bibleverse{18} Auch von deinen
Söhnen, die von dir abstammen werden, die du zeugen wirst, wird man
nehmen, daß sie Kämmerer seien im Palast des Königs zu Babel!
\bibleverse{19} Hiskia aber sprach zu Jesaja: Das Wort des
\textsc{Herrn}, welches du geredet hast, ist gut. Und er sprach: Es wird
ja doch Friede und Sicherheit sein zu meinen Lebzeiten! \bibleverse{20}
Was aber mehr von Hiskia zu sagen ist, und alle seine Macht, und wie er
den Teich und die Wasserleitung gemacht, womit er Wasser in die Stadt
geleitet hat, ist das nicht beschrieben in der Chronik der Könige von
Juda? \bibleverse{21} Und Hiskia legte sich zu seinen Vätern; und sein
Sohn Manasse ward König an seiner Statt.

\hypertarget{section-20}{%
\section{21}\label{section-20}}

\bibleverse{1} Manasse war zwölf Jahre alt, als er König ward, und
regierte fünfundfünfzig Jahre lang zu Jerusalem. Seine Mutter hieß
Chephziba. \bibleverse{2} Und er tat, was böse war in den Augen des
\textsc{Herrn}, nach den Greueln der Heiden, die der \textsc{Herr} vor
den Kindern Israel vertrieben hatte. \bibleverse{3} Er baute die Höhen
wieder auf, die sein Vater Hiskia abgetan hatte, und richtete dem Baal
Altäre auf und machte eine Aschera, wie Ahab, der König von Israel,
getan hatte, und betete das ganze Heer des Himmels an und diente ihnen.
\bibleverse{4} Und er baute Altäre im Hause des \textsc{Herrn}, von
welchem der \textsc{Herr} gesagt hatte: Mein Name soll in Jerusalem
wohnen! \bibleverse{5} Er baute auch dem ganzen Heer des Himmels Altäre
in beiden Vorhöfen am Hause des \textsc{Herrn} \bibleverse{6} und ließ
seinen Sohn durchs Feuer gehen und trieb Wolkendeuterei und
Schlangenbeschwörung und hielt Totenbeschwörer und Wahrsager; er tat
viel von dem, was böse ist in den Augen des \textsc{Herrn}, wodurch er
ihn erzürnte. \bibleverse{7} Er setzte auch das Bild der Aschera, das er
gemacht hatte, in das Haus, von welchem der \textsc{Herr} zu David und
zu seinem Sohne Salomo gesagt hatte: In diesem Hause und in Jerusalem,
das ich aus allen Stämmen Israels erwählt habe, will ich meinen Namen
wohnen lassen ewiglich, \bibleverse{8} und ich will den Fuß Israels
nicht mehr aus dem Lande wandern lassen, daß ich ihren Vätern gegeben
habe; wenn sie nur darauf achten, zu tun nach allem, was ich ihnen
geboten habe, und nach dem ganzen Gesetz, das mein Knecht Mose ihnen
befohlen hat. \bibleverse{9} Aber sie gehorchten nicht, sondern Manasse
verführte sie, daß sie Schlimmeres taten als die Heiden, die der
\textsc{Herr} vor den Kindern Israel vertilgt hatte. \bibleverse{10} Da
redete der \textsc{Herr} durch seine Knechte, die Propheten, und sprach:
\bibleverse{11} Weil Manasse, der König von Juda, diese Greuel verübt
hat, die ärger sind, als alle Greuel, welche die Amoriter getan haben,
die vor ihm gewesen sind, und weil er auch Juda mit seinen Götzen zur
Sünde verführt hat, \bibleverse{12} darum spricht der \textsc{Herr}, der
Gott Israels, also: Siehe, ich will Unglück über Jerusalem und über Juda
bringen, daß allen, die es hören werden, beide Ohren gellen sollen;
\bibleverse{13} und ich will über Jerusalem die Meßschnur Samarias
ausspannen und das Senkblei des Hauses Ahabs, und ich will Jerusalem
auswischen, wie man eine Schüssel auswischt und sie umkehrt.
\bibleverse{14} Und das Übriggebliebene meines Erbteils will ich
verwerfen und sie in die Hand ihrer Feinde geben, und sie sollen allen
ihren Feinden zum Raub und zur Beute werden; \bibleverse{15} weil sie
getan, was böse ist in meinen Augen, und mich erzürnt haben, von dem
Tage an, da ihre Väter aus Ägypten gezogen sind, bis auf diesen Tag!
\bibleverse{16} Auch vergoß Manasse sehr viel unschuldiges Blut, so daß
er Jerusalem damit erfüllte, von einem Ende bis zum andern, abgesehen
von seiner Sünde, zu der er Juda verführt hatte, so daß sie taten, was
böse war in den Augen des \textsc{Herrn}. \bibleverse{17} Was aber mehr
von Manasse zu sagen ist, und alles, was er getan hat, und seine Sünde,
die er tat, ist das nicht beschrieben in der Chronik der Könige von
Juda? \bibleverse{18} Und Manasse legte sich zu seinen Vätern und ward
begraben im Garten seines Hauses, im Garten Ussas. Und sein Sohn Amon
ward König an seiner Statt. \bibleverse{19} Zweiundzwanzig Jahre alt war
Amon, als er König ward, und regierte zwei Jahre zu Jerusalem. Seine
Mutter hieß Messulemet, eine Tochter des Charuz von Jothba.
\bibleverse{20} Und er tat, was böse war in den Augen des
\textsc{Herrn}, wie sein Vater Manasse getan hatte. \bibleverse{21} Und
er wandelte ganz auf dem Wege, den sein Vater gewandelt war, und diente
den Götzen, welchen sein Vater gedient hatte, und betete sie an;
\bibleverse{22} und verließ den \textsc{Herrn}, den Gott seiner Väter,
und wandelte nicht im Wege des \textsc{Herrn}. \bibleverse{23} Und die
Knechte Amons machten eine Verschwörung wider ihn und töteten ihn in
seinem Hause. \bibleverse{24} Aber das Landvolk erschlug alle, welche
die Verschwörung wider den König Amon gemacht hatten. Und das Landvolk
machte Josia, seinen Sohn, zum König an seiner Statt. \bibleverse{25}
Was aber Amon mehr getan hat, ist das nicht beschrieben in der Chronik
der Könige von Juda? \bibleverse{26} Und er wurde begraben in seiner
Grabstätte im Garten Ussas, und sein Sohn Josia ward König an seiner
Statt.

\hypertarget{section-21}{%
\section{22}\label{section-21}}

\bibleverse{1} Josia war acht Jahre alt, als er König ward, und regierte
einunddreißig Jahre lang zu Jerusalem. Seine Mutter hieß Jedida, eine
Tochter Adajas von Bozkat. \bibleverse{2} Und er tat, was recht war in
den Augen des \textsc{Herrn}, und wandelte in allen Wegen Davids, seines
Vaters, und wich nicht davon, weder zur Rechten noch zur Linken.
\bibleverse{3} Und im achtzehnten Jahre des Königs Josia sandte der
König Saphan, den Sohn Azaljas, des Sohnes Mesullams, den Schreiber, in
das Haus des \textsc{Herrn} und sprach: \bibleverse{4} Gehe hinauf zu
Hilkia, dem Hohenpriester, er soll das Geld auszahlen, das zum Hause des
\textsc{Herrn} gebracht worden ist, welches die Schwellenhüter vom Volk
gesammelt haben, \bibleverse{5} damit man es den Aufsehern über die
Arbeiter im Hause des \textsc{Herrn} gebe; diese sollen es den Arbeitern
am Hause des \textsc{Herrn} geben, daß sie ausbessern, was am Hause
baufällig ist; \bibleverse{6} nämlich den Zimmerleuten und Bauleuten und
den Maurern, um Holz und behauene Steine für die Ausbesserung des Hauses
zu kaufen; \bibleverse{7} doch soll man nicht abrechnen mit ihnen
betreffs des Geldes, das ihnen eingehändigt wird, denn sie handeln auf
Treu und Glauben! \bibleverse{8} Da sprach Hilkia, der Hohepriester, zu
Saphan, dem Schreiber: Ich habe das Gesetzbuch im Hause des
\textsc{Herrn} gefunden! Und Hilkia übergab Saphan das Buch, und er las
es. \bibleverse{9} Und Saphan, der Schreiber, kam zum König und brachte
dem König Bericht und sprach: Deine Knechte haben das Geld
ausgeschüttet, das im Hause vorhanden war, und haben es den Aufsehern
über die Arbeiter im Hause des \textsc{Herrn} gegeben. \bibleverse{10}
Auch sagte Saphan, der Schreiber, dem König und sprach: Hilkia, der
Priester, gab mir ein Buch. Und Saphan las es vor dem König.
\bibleverse{11} Als aber der König die Worte des Gesetzbuches hörte,
zerriß er seine Kleider. \bibleverse{12} Und der König gebot dem
Priester Hilkia und Achikam, dem Sohne Saphans, und Achbor, dem Sohne
Michajas, und dem Schreiber Saphan und Asaja, dem Knechte des Königs,
und sprach: \bibleverse{13} Gehet hin und befraget den \textsc{Herrn}
für mich und das Volk und für ganz Juda wegen der Worte dieses Buches,
das gefunden worden ist; denn groß ist der Zorn des \textsc{Herrn}, der
wider uns entbrannt ist, weil unsre Väter nicht auf die Worte dieses
Buches gehört haben, daß sie getan hätten alles, was uns darin
vorgeschrieben ist! \bibleverse{14} Da gingen der Priester Hilkia,
Achikam, Achbor, Saphan und Asaja zu der Prophetin Hulda, dem Weibe
Sallums, des Sohnes, Tikwas, des Sohnes Harhas, des Hüters der Kleider.
Sie wohnte aber zu Jerusalem, im andern Stadtteil. Und sie redeten mit
ihr. \bibleverse{15} Sie aber sprach zu ihnen: So spricht der
\textsc{Herr}, der Gott Israels: Sagt dem Manne, der euch zu mir gesandt
hat: \bibleverse{16} So spricht der \textsc{Herr}: Siehe, ich will
Unglück bringen über diesen Ort und über seine Bewohner, nämlich alle
Worte des Buches, welches der König von Juda gelesen hat,
\bibleverse{17} weil sie mich verlassen und andern Göttern geräuchert
haben, so daß sie mich erzürnten mit allen Werken ihrer Hände; darum
wird mein Grimm wider diesen Ort entbrennen und nicht ausgelöscht
werden. \bibleverse{18} Aber dem König von Juda, der euch gesandt hat,
den \textsc{Herrn} zu befragen, sollt ihr also sagen: So spricht der
\textsc{Herr}, der Gott Israels, betreffs der Worte, welche du gehört
hast: \bibleverse{19} Weil dein Herz weich geworden ist und du dich vor
dem \textsc{Herrn} gedemütigt hast, als du hörtest, was ich wider diesen
Ort und seine Bewohner geredet habe, daß sie zum Entsetzen und zum Fluch
werden sollen; und weil du deine Kleider zerrissen und vor mir geweint
hast, so habe auch ich darauf gehört, spricht der \textsc{Herr};
\bibleverse{20} und darum, siehe, will ich dich zu deinen Vätern
versammeln, daß du in Frieden in dein Grab gebracht werdest, und deine
Augen sollen alles Unglück, das ich über diesen Ort bringen will, nicht
sehen. Und sie brachten dem König diese Antwort.

\hypertarget{section-22}{%
\section{23}\label{section-22}}

\bibleverse{1} Da sandte der König hin und ließ alle Ältesten von Juda
und Jerusalem zu sich versammeln. \bibleverse{2} Und der König ging
hinauf in das Haus des \textsc{Herrn}, und alle Männer von Juda und alle
Einwohner von Jerusalem mit ihm, auch die Priester und Propheten und
alles Volk, klein und groß, und man las vor ihren Ohren alle Worte des
Bundesbuches, das man im Hause des \textsc{Herrn} gefunden hatte.
\bibleverse{3} Der König aber trat an die Säule und machte einen Bund
vor dem \textsc{Herrn}, dem \textsc{Herrn} nachzuwandeln und seine
Gebote und Zeugnisse und Satzungen von ganzem Herzen und von ganzer
Seele zu beobachten, die Worte dieses Bundes, welche in diesem Buche
geschrieben standen, auszuführen. Und das ganze Volk trat in den Bund.
\bibleverse{4} Und der König gebot dem Hohenpriester Hilkia und den
Priestern der zweiten Ordnung und den Hütern der Schwelle, daß sie aus
dem Tempel des \textsc{Herrn} alle Geräte entfernen sollten, die man dem
Baal und der Aschera und dem ganzen Heer des Himmels gemacht hatte; und
er verbrannte sie draußen vor Jerusalem, auf den Feldern des Kidron, und
trug ihren Staub nach Bethel. \bibleverse{5} Und er setzte die
Götzenpriester ab, welche die Könige von Juda eingesetzt hatten und die
auf den Höhen in den Städten Judas und um Jerusalem her räucherten; auch
die, welche dem Baal, der Sonne und dem Mond und den Gestirnen und dem
ganzen Heer des Himmels räucherten. \bibleverse{6} Er ließ auch die
Aschera aus dem Hause des \textsc{Herrn} hinausführen vor Jerusalem, an
den Bach Kidron, und verbrannte sie beim Bach Kidron und machte sie zu
Staub und warf ihren Staub auf die Gräber der gemeinen Leute.
\bibleverse{7} Und er brach die Häuser der Buhler ab, die am Hause des
\textsc{Herrn} waren, darin die Weiber für die Aschera Zelte wirkten.
\bibleverse{8} Auch ließ er alle Priester aus den Städten kommen und
verunreinigte die Höhen, wo die Priester geräuchert hatten, von Geba an
bis gen Beerseba; und er brach die Höhen der Tore ab, die am Eingang des
Tores Josuas, des Stadtobersten, waren, zur Linken, wenn man zum
Stadttor kommt. \bibleverse{9} Doch durften die Höhenpriester nicht auf
dem Altar des \textsc{Herrn} zu Jerusalem opfern, dagegen aßen sie von
dem ungesäuerten Brot unter ihren Brüdern. \bibleverse{10} Er
verunreinigte auch das Tophet im Tale der Söhne Hinnom, damit niemand
mehr seinen Sohn oder seine Tochter dem Moloch durchs Feuer gehen lasse.
\bibleverse{11} Und er schaffte die Rosse ab, welche die Könige von Juda
der Sonne geweiht hatten, beim Eingang des Hauses des \textsc{Herrn},
gegen die Kammer Netan-Melechs, des Kämmerers, die im Anbau war; und die
Wagen der Sonne verbrannte er mit Feuer. \bibleverse{12} Der König brach
auch die Altäre ab auf dem Dache, dem Söller des Ahas, welche die Könige
von Juda gemacht hatten; desgleichen die Altäre, welche Manasse in den
beiden Vorhöfen des Hauses des \textsc{Herrn} gemacht hatte, er
zerstörte sie und schaffte sie fort und warf ihren Staub in den Bach
Kidron. \bibleverse{13} Auch die Höhen, die gegenüber von Jerusalem, zur
Rechten am Berge des Verderbens waren, welche Salomo, der König von
Israel, der Astarte, dem Greuel der Zidonier, und Kamos, dem Greuel der
Moabiter, und Milkom, dem Greuel der Kinder Ammon, gebaut hatte,
verunreinigte der König. \bibleverse{14} Er zerbrach die Säulen und hieb
die Ascheren um und füllte ihren Platz mit Menschengebeinen.
\bibleverse{15} Desgleichen auch den Altar zu Bethel und die Höhe, die
Jerobeam, der Sohn Nebats, der Israel zur Sünde verführte, erbaut hatte:
auch diesen Altar und die Höhe brach er ab und verbrannte die Höhe und
machte sie zu Staub und verbrannte die Aschera. \bibleverse{16} Und
Josia sah sich um und erblickte die Gräber, welche dort auf dem Berge
waren, und sandte hin und ließ die Gebeine aus den Gräbern nehmen und
verbrannte sie auf dem Altar und verunreinigte ihn, nach dem Worte des
\textsc{Herrn}, welches der Mann Gottes verkündigt hatte, als er solches
ausrief. \bibleverse{17} Und er sprach: Was ist das für ein Grabmal, das
ich hier sehe? Da sprachen die Leute der Stadt zu ihm: Es ist das Grab
des Mannes Gottes, der von Juda kam, und diese Dinge, die du getan hast,
wider den Altar zu Bethel ankündigte. \bibleverse{18} Da sprach er: So
laßt ihn liegen; niemand rühre seine Gebeine an! Also blieben seine
Gebeine erhalten, samt den Gebeinen des Propheten, der von Samaria
gekommen war. \bibleverse{19} Josia beseitigte auch alle Höhenhäuser in
den Städten Samarias, welche die Könige von Israel gemacht hatten, den
\textsc{Herrn} zu erzürnen, und verfuhr mit ihnen ganz so, wie er zu
Bethel getan hatte. \bibleverse{20} Und er opferte alle Höhenpriester,
die daselbst waren, auf den Altären; und verbrannte also Menschengebeine
darauf und kehrte dann nach Jerusalem zurück. \bibleverse{21} Dann gebot
der König allem Volk und sprach: Feiert dem \textsc{Herrn}, eurem Gott,
das Passah, wie es in diesem Bundesbuch geschrieben steht!
\bibleverse{22} Denn es war kein solches Passah gehalten worden, seit
der Zeit der Richter, die Israel gerichtet hatten, und während der
ganzen Zeit der Könige von Israel und der Könige von Juda;
\bibleverse{23} erst im achtzehnten Jahre des Königs Josia ist dieses
Passah dem \textsc{Herrn} zu Jerusalem gefeiert worden. \bibleverse{24}
Auch die Totenbeschwörer und Zeichendeuter, die Teraphim und Götzen und
alle Greuel, die im Lande Juda und zu Jerusalem gesehen wurden, rottete
Josia aus, um die Worte des Gesetzes zu vollstrecken, die geschrieben
standen in dem Buche, welches der Priester Hilkia im Hause des
\textsc{Herrn} gefunden hatte. \bibleverse{25} Und seinesgleichen ist
vor ihm kein König gewesen, der sich also von ganzem Herzen und von
ganzer Seele und mit allen seinen Kräften dem \textsc{Herrn} zuwandte,
ganz nach dem Gesetze Moses; auch nach ihm ist keiner seinesgleichen
aufgestanden. \bibleverse{26} Doch kehrte sich der \textsc{Herr} nicht
von dem Grimm seines großen Zornes, womit er über Juda erzürnt war, um
aller Ärgernisse willen, womit Manasse ihn gereizt hatte.
\bibleverse{27} Denn der \textsc{Herr} sprach: Ich will auch Juda von
meinem Angesicht hinwegtun, wie ich Israel hinweggetan habe, und ich
will diese Stadt Jerusalem, die ich erwählt hatte, verwerfen, und das
Haus, von dem ich gesagt habe: Mein Name soll daselbst sein!
\bibleverse{28} Was aber mehr von Josia zu sagen ist, und alles, was er
getan hat, ist das nicht geschrieben in der Chronik der Könige von Juda?
\bibleverse{29} Zu seiner Zeit zog der Pharao Necho, der König von
Ägypten, herauf wider den König von Assyrien an den Euphratstrom; dem
zog der König Josia entgegen; aber der Pharao tötete ihn zu Megiddo,
sowie er ihn gesehen hatte. \bibleverse{30} Und seine Knechte führten
ihn tot von Megiddo weg und brachten ihn nach Jerusalem und begruben ihn
in seinem Grabe. Da nahm das Volk des Landes Joahas, den Sohn Josias,
und sie salbten ihn und machten ihn zum König an seines Vaters Statt.
\bibleverse{31} Dreiundzwanzig Jahre alt war Joahas, als er König ward,
und regierte drei Monate lang zu Jerusalem. Seine Mutter hieß Hamutal,
die Tochter Jeremias von Libna. \bibleverse{32} Er tat, was dem
\textsc{Herrn} übel gefiel, ganz wie seine Väter getan hatten.
\bibleverse{33} Aber der Pharao Necho nahm ihn gefangen zu Ribla, im
Lande Chamat, so daß er nicht mehr König war zu Jerusalem; und legte
eine Geldbuße auf das Land, hundert Talente Silber und ein Talent Gold.
\bibleverse{34} Und der Pharao Necho machte Eljakim, den Sohn Josias,
zum König an Stelle seines Vaters Josia und veränderte seinen Namen in
Jehojakim. Aber den Joahas nahm er und brachte ihn nach Ägypten, wo er
starb. \bibleverse{35} Und Jehojakim gab das Silber und das Gold dem
Pharao; doch schätzte er das Land ein, um das Silber nach dem Befehl des
Pharao geben zu können; er zwang das Volk des Landes, daß ein jeder nach
seiner Schätzung Silber und Gold dem Pharao Necho gäbe. \bibleverse{36}
Fünfundzwanzig Jahre alt war Jehojakim, als er König ward, und regierte
elf Jahre lang zu Jerusalem. Seine Mutter hieß Sebudda, die Tochter
Pedajas von Ruma. \bibleverse{37} Und er tat, was dem \textsc{Herrn}
übel gefiel, ganz wie seine Väter getan hatten.

\hypertarget{section-23}{%
\section{24}\label{section-23}}

\bibleverse{1} Zu seiner Zeit zog Nebukadnezar, der König von Babel,
herauf, und Jehojakim ward ihm untertan drei Jahre lang. Darnach fiel er
wieder von ihm ab. \bibleverse{2} Da sandte der \textsc{Herr} Truppen
wider ihn aus Chaldäa, aus Syrien, aus Moab und von den Kindern Ammon;
die sandte er gegen Juda, um es zugrunde zu richten, nach dem Worte des
\textsc{Herrn}, das er durch seine Knechte, die Propheten, geredet
hatte. \bibleverse{3} Fürwahr, nach dem Worte des \textsc{Herrn} kam das
über Juda, daß er sie von seinem Angesicht täte, um der Sünden Manasses
willen, für all das, was er getan hatte; \bibleverse{4} und auch um des
unschuldigen Blutes willen, das er vergossen, da er Jerusalem mit
unschuldigem Blute erfüllt hatte; darum wollte der \textsc{Herr} nicht
vergeben. \bibleverse{5} Was aber mehr von Jehojakim zu sagen ist, und
alles, was er getan hat, ist das nicht geschrieben in der Chronik der
Könige von Juda? \bibleverse{6} Und Jehojakim legte sich zu seinen
Vätern. Und Jehojachin, sein Sohn, ward König an seiner Statt.
\bibleverse{7} Aber der König von Ägypten zog nicht mehr aus seinem
Lande; denn der König von Babel hatte alles eingenommen, was dem König
von Ägypten gehörte, vom Bache Ägyptens bis an den Euphratstrom.
\bibleverse{8} Achtzehn Jahre alt war Jehojachin, als er König ward, und
regierte drei Monate lang zu Jerusalem. Seine Mutter hieß Nehusta, die
Tochter Elnatans von Jerusalem. \bibleverse{9} Er tat aber, was dem
\textsc{Herrn} mißfiel, ganz wie sein Vater getan hatte. \bibleverse{10}
Zu jener Zeit zogen die Knechte Nebukadnezars, des Königs von Babel, gen
Jerusalem herauf, und die Stadt ward belagert. \bibleverse{11} Und
Nebukadnezar, der König von Babel, kam zur Stadt, und seine Knechte
belagerten sie. \bibleverse{12} Aber Jehojachin, der König von Juda,
ging zum König von Babel hinaus, er samt seiner Mutter, seinen Knechten,
seinen Obersten und seinen Kämmerern; und der König von Babel nahm ihn
gefangen im achten Jahre seiner Regierung. \bibleverse{13} Und er
brachte von dannen heraus alle Schätze im Hause des \textsc{Herrn} und
die Schätze im Hause des Königs und zerschlug alle goldenen Geräte,
welche Salomo, der König von Israel, im Tempel des \textsc{Herrn}
gemacht; wie der \textsc{Herr} gesagt hatte. \bibleverse{14} Und er
führte ganz Jerusalem gefangen hinweg, nämlich alle Obersten und alle
kriegstüchtigen Männer, zehntausend Gefangene, auch alle Schlosser und
alle Schmiede, und ließ nichts übrig als geringes Landvolk.
\bibleverse{15} Also führte er Jehojachin nach Babel hinweg, auch die
Mutter des Königs und die Frauen des Königs und seine Kämmerer. Dazu
führte er die Mächtigen des Landes von Jerusalem gefangen nach Babel,
\bibleverse{16} auch alle Kriegsleute, siebentausend, dazu die Schlosser
und die Schmiede, im ganzen tausend, alles kriegstüchtige Männer; und
der König von Babel brachte sie gefangen nach Babel. \bibleverse{17} Und
der König von Babel machte Matanja, Jehojachins Oheim, zum König an
seiner Statt, und änderte seinen Namen in Zedekia. \bibleverse{18}
Einundzwanzig Jahre alt war Zedekia, als er König ward, und regierte elf
Jahre zu Jerusalem. Seine Mutter hieß Hamutal, die Tochter Jeremias von
Libna. \bibleverse{19} Und er tat, was dem \textsc{Herrn} mißfiel, ganz
wie Jehojachin getan hatte. \bibleverse{20} Denn wegen des Zornes des
\textsc{Herrn} kam es so weit mit Jerusalem und Juda, daß er sie von
seinem Angesicht verwarf. Und Zedekia fiel ab von dem König zu Babel.

\hypertarget{section-24}{%
\section{25}\label{section-24}}

\bibleverse{1} Und es begab sich im neunten Jahre seines Königreichs, am
zehnten Tage des zehnten Monats, da kam Nebukadnezar, der König von
Babel, mit aller seiner Macht wider Jerusalem und belagerte die Stadt;
und sie bauten Belagerungstürme um sie her. \bibleverse{2} Und die Stadt
wurde belagert bis ins elfte Jahr des Königs Zedekia. \bibleverse{3} Am
neunten Tage des vierten Monats aber ward die Hungersnot in der Stadt so
stark, daß das Landvolk nichts zu essen hatte. \bibleverse{4} Da brach
der Feind in die Stadt ein, und alle Kriegsleute flohen bei Nacht durch
das Tor zwischen den beiden Mauern, beim Garten des Königs; und da die
Chaldäer rings um die Stadt her lagen, zog man den Weg nach der Ebene.
\bibleverse{5} Aber das Heer der Chaldäer jagte dem König nach und holte
ihn ein auf den Ebenen von Jericho, nachdem sein ganzes Heer sich von
ihm zerstreut hatte. \bibleverse{6} Sie aber fingen den König und
führten ihn hinauf zum König von Babel nach Ribla und sprachen das
Urteil über ihn. \bibleverse{7} Und sie metzelten Zedekias Söhne vor
dessen Augen nieder; darnach blendeten sie Zedekia und banden ihn mit
zwei ehernen Ketten und führten ihn nach Babel. \bibleverse{8} Am
siebenten Tage des fünften Monats (das ist das neunzehnte Jahr
Nebukadnezars, des Königs von Babel), kam Nebusaradan, der Oberste der
Leibwache, der Diener des Königs von Babel, \bibleverse{9} nach
Jerusalem und verbrannte das Haus des \textsc{Herrn} und das Haus des
Königs und alle Häuser zu Jerusalem, und alle großen Häuser verbrannte
er mit Feuer. \bibleverse{10} Und das ganze Heer der Chaldäer, das bei
dem Obersten der Leibwache war, riß die Mauern der Stadt Jerusalem
ringsum nieder. \bibleverse{11} Den Rest des Volkes aber, der in der
Stadt noch übriggeblieben war, und die Überläufer, welche zum König von
Babel übergegangen waren, und den Rest der Menge führte Nebusaradan, der
Oberste der Leibwache, hinweg. \bibleverse{12} Doch von den Geringsten
im Lande ließ der Oberste der Leibwache Weingärtner und Ackerleute
zurück. \bibleverse{13} Aber die ehernen Säulen am Hause des
\textsc{Herrn} und die Ständer und das eherne Meer, das im Hause des
\textsc{Herrn} war, zerbrachen die Chaldäer und führten das Erz nach
Babel. \bibleverse{14} Auch die Töpfe, Schaufeln, Messer, Schalen und
alle ehernen Geräte, womit man diente, nahmen sie weg. \bibleverse{15}
Dazu nahm der Oberste der Leibwache die Räucherpfannen und
Sprengschalen, alles, was von Gold, und alles, was von Silber war.
\bibleverse{16} Die beiden Säulen, das eine Meer und die Ständer, welche
Salomo zum Hause des \textsc{Herrn} gemacht hatte, das Erz aller dieser
Geräte war nicht zu wägen. \bibleverse{17} Achtzehn Ellen hoch war eine
Säule, und es war auf ihr ein Knauf von Erz, drei Ellen hoch, und um den
Knauf ein Netzwerk und Granatäpfel, ganz von Erz. Ebensolche Granatäpfel
hatte auch die andere Säule um das Netzwerk. \bibleverse{18} Und der
Oberste der Leibwache nahm Seraja, den Hauptpriester, und Zephanja, den
zweiten Priester, und die drei Schwellenhüter; \bibleverse{19} er nahm
auch einen Kämmerer aus der Stadt, der über die Kriegsleute gesetzt war,
und fünf Männer, die stets vor dem König waren, die in der Stadt
gefunden wurden, und den Schreiber, den Feldhauptmann, der das Volk des
Landes zum Heere aushob, und sechzig Männer von dem Landvolk, die in der
Stadt gefunden wurden; \bibleverse{20} diese nahm Nebusaradan, der
Oberste der Leibwache, und brachte sie zum König von Babel, nach Ribla.
\bibleverse{21} Und der König von Babel schlug sie tot zu Ribla im Lande
Chamat. Also ward Juda aus seinem Lande gefangen hinweggeführt.
\bibleverse{22} Über das Volk aber, das im Lande Juda blieb, das
Nebukadnezar, der König von Babel, übriggelassen hatte, setzte er
Gedalja, den Sohn Ahikams, des Sohnes Saphans. \bibleverse{23} Als nun
alle Obersten des Heeres und ihre Leute hörten, daß der König von Babel
den Gedalja eingesetzt hatte, kamen sie zu Gedalja gen Mizpa; nämlich
Ismael, der Sohn Netanjas, und Johanan, der Sohn Kareachs, und Seraja,
der Sohn Tanchumets, des Netophatiters, und Jaasanja, der Sohn des
Maachatiters, samt ihren Männern. \bibleverse{24} Und Gedalja schwur
ihnen und ihren Männern und sprach zu ihnen: Fürchtet euch nicht vor den
Knechten der Chaldäer; bleibet im Lande und seid dem König von Babel
untertan, so wird es euch wohlgehen! \bibleverse{25} Aber im siebenten
Monat kam Ismael, der Sohn Netanjas, des Sohnes Elisamas, von
königlichem Geschlecht, und zehn Männer mit ihm und schlugen Gedalja
tot; dazu die Juden und die Chaldäer, die zu Mizpa bei ihm waren.
\bibleverse{26} Da machte sich alles Volk, klein und groß, und die
Obersten des Heeres auf und zogen nach Ägypten; denn sie fürchteten sich
vor den Chaldäern. \bibleverse{27} Aber im siebenunddreißigsten Jahre,
nachdem Jehojachin, der König von Juda, gefangen hinweggeführt worden,
am siebenundzwanzigsten Tage des zwölften Monats, erhob Evil-Merodach,
der König von Babel, im ersten Jahre seiner Regierung das Haupt
Jehojachins, des Königs von Juda, und entließ ihn aus dem Kerker;
\bibleverse{28} und redete freundlich mit ihm und setzte seinen Thron
über die Throne der Könige, die bei ihm zu Babel waren; \bibleverse{29}
und er ließ ihn seine Gefängniskleider ablegen; und er durfte stets vor
ihm essen, sein ganzes Leben lang. \bibleverse{30} Und sein Unterhalt,
der beständige Unterhalt, ward ihm vom König gegeben, für jeden Tag sein
bestimmtes Teil, sein ganzes Leben lang.
