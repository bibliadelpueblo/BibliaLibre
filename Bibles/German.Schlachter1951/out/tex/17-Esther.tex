\hypertarget{section}{%
\section{1}\label{section}}

\bibleverse{1} Und es begab sich in den Tagen des Ahasveros (des
Ahasveros, welcher von Indien bis Äthiopien über hundertsiebenundzwanzig
Provinzen regierte), \bibleverse{2} in jenen Tagen, als der König
Ahasveros im Schlosse Susan auf seinem königlichen Throne saß,
\bibleverse{3} im dritten Jahre seiner Regierung, daß er allen seinen
Fürsten und Knechten ein Mahl machte. Die Gewaltigen von Persien und
Medien, die Edelsten und Obersten seiner Provinzen waren vor ihm,
\bibleverse{4} als er den Reichtum der Herrlichkeit seines Königreichs
und die kostbare Pracht seiner Majestät sehen ließ viele Tage, nämlich
hundertachtzig Tage lang. \bibleverse{5} Und als diese Tage vollendet
waren, machte der König ein Mahl allem Volk, das im Schlosse Susan
zugegen war, den Großen und den Kleinen, sieben Tage lang, im Hofe des
Gartens beim königlichen Palaste. \bibleverse{6} Da waren feine Tücher
von weißer Baumwolle und blauem Purpur mit Schnüren von feiner weißer
Baumwolle und rotem Purpur an silbernen Ringen und Säulen von weißem
Marmor aufgehängt. Goldene und silberne Ruhebetten standen auf einem
Steinpflaster von Alabaster, weißem Marmor, Perlmutter und dunklem
Marmor. \bibleverse{7} Und man gab zu trinken aus goldenen Gefäßen, und
die Gefäße waren voneinander verschieden; königlicher Wein war in Menge
vorhanden, nach königlicher Freigebigkeit. \bibleverse{8} Und das
Trinken war der Verordnung gemäß ohne Zwang; denn also hatte der König
allen seinen Hofmeistern befohlen, daß man jedermann machen ließe, wie
es ihm gefiele. \bibleverse{9} Auch die Königin Vasti veranstaltete ein
Mahl für die Frauen des königlichen Palastes, welcher dem König
Ahasveros gehörte. \bibleverse{10} Und am siebenten Tage, als des Königs
Herz vom Wein fröhlich war, befahl er Mehuman, Bista, Charbona, Bigta,
Abagta, Setar und Karkas, den sieben Kämmerern, die vor dem König
Ahasveros dienten, \bibleverse{11} die Königin Vasti mit der königlichen
Krone vor den König zu bringen, damit er den Völkern und Fürsten ihre
Schönheit zeigte, denn sie war von schöner Gestalt. \bibleverse{12} Aber
die Königin Vasti wollte nicht kommen auf den Befehl des Königs, den er
durch seine Kämmerer gegeben hatte. Da ward der König sehr zornig, und
sein Grimm entbrannte in ihm. \bibleverse{13} Und der König sprach zu
den Weisen, die sich auf die Zeiten verstanden (denn des Königs Wort
erging in Gegenwart aller Gesetzes und Rechtskundigen; \bibleverse{14}
und ihm zunächst saßen Karschena, Setar, Admata, Tarsis, Meres, Marsena
und Memuchan, die sieben Fürsten der Perser und Meder, die das Angesicht
des Königs sahen und die ersten Stellen im Königreich innehatten):
\bibleverse{15} Nach welchem Gesetz soll man die Königin Vasti
behandeln, dafür daß sie nicht gehandelt hat nach dem Befehl des Königs
Ahasveros, der ihr durch die Kämmerer übermittelt wurde? \bibleverse{16}
Da sprach Memuchan vor dem König und den Fürsten: Die Königin Vasti hat
sich nicht nur an dem König vergangen, sondern auch an allen Fürsten und
an allen Völkern, die in allen Provinzen des Königs Ahasveros sind.
\bibleverse{17} Denn das Verhalten der Königin wird allen Frauen bekannt
werden, so daß ihre Männer in ihren Augen verächtlich werden, da es
heißen wird: Der König Ahasveros hieß die Königin Vasti vor sich kommen,
aber sie kam nicht! \bibleverse{18} Das werden die Fürstinnen der Perser
und Meder heute schon allen Fürsten des Königs erzählen, wenn sie das
Verhalten der Königin hören, woraus Verachtung und Verdruß genug
entstehen wird. \bibleverse{19} Gefällt es dem König, so gehe ein
königlicher Befehl von ihm aus und werde aufgezeichnet unter die Gesetze
der Perser und Meder, damit er nicht bloß vorübergehend gelte: daß Vasti
nicht mehr vor dem König Ahasveros erscheinen dürfe, und daß der König
ihre königliche Würde einer andern gebe, die besser ist als sie.
\bibleverse{20} Wird dann dieser Befehl des Königs, den er geben wird,
in seinem ganzen Königreich, welches groß ist, bekannt gemacht, so
werden alle Frauen ihre Männer in Ehren halten, vornehme und geringe.
\bibleverse{21} Diese Rede gefiel dem König und den Fürsten; und der
König tat nach den Worten Memuchans \bibleverse{22} und sandte Briefe in
alle Provinzen des Königs, in jede Provinz nach ihrer Schrift und zu
jedem Volk nach seiner Sprache; daß jeder Mann Herr sein solle in seinem
Hause. Das ließ er bekanntmachen in der Sprache seines Volkes.

\hypertarget{section-1}{%
\section{2}\label{section-1}}

\bibleverse{1} Nach diesen Begebenheiten, als sich der Grimm des Königs
Ahasveros gelegt hatte, dachte er an Vasti und daran, was sie getan
hatte und was über sie beschlossen worden war. \bibleverse{2} Da
sprachen die Knappen des Königs, die ihm dienten: Man suche für den
König Mädchen, Jungfrauen von schöner Gestalt; \bibleverse{3} und der
König bestellte Beamte in allen Provinzen seines Königreichs, damit sie
alle Mädchen, Jungfrauen von schöner Gestalt, in das Schloß Susan
zusammenbringen, in das Frauenhaus, unter die Obhut Hegais, des
königlichen Kämmerers, des Hüters der Frauen; und man lasse ihnen ihre
Reinigungssalben geben; \bibleverse{4} und welche Jungfrau dem König
gefällt, die werde Königin an Vastis Statt! Dieser Vorschlag gefiel dem
König, und er tat also. \bibleverse{5} Es war aber ein jüdischer Mann im
Schloß Susan, der hieß Mardochai, ein Sohn Jairs, des Sohnes Simeis, des
Sohnes des Kis, \bibleverse{6} ein Benjaminiter, der von Jerusalem
weggeführt worden war mit den Gefangenen, die mit Jechonja, dem König
von Juda, hinweggeführt worden waren, welche Nebukadnezar, der König von
Babel, gefangen weggeführt hatte. \bibleverse{7} Und dieser war
Pflegevater der Hadassa (das ist Esther), der Tochter seines Oheims;
denn sie hatte weder Vater noch Mutter. Diese Jungfrau aber war von
schöner Gestalt und lieblichem Aussehen. Und als ihr Vater und ihre
Mutter gestorben waren, hatte Mardochai sie als seine Tochter
angenommen. \bibleverse{8} Als nun das Gebot des Königs und das Gesetz
bekanntgemacht war und viele Jungfrauen in das Schloß Susan unter die
Obhut Hegais zusammengebracht wurden, da ward auch Esther in des Königs
Haus geholt, unter die Obhut Hegais, des Hüters der Frauen.
\bibleverse{9} Und die Jungfrau gefiel ihm, und sie fand Gunst bei ihm.
Und er sorgte dafür, daß sie ihre Reinigungssalben und ihre Gerichte
bald erhielt; auch gab er ihr sieben auserlesene Mägde aus des Königs
Hause. Und er wies ihr samt ihren Mägden den besten Ort im Frauenhause
an. \bibleverse{10} Esther aber zeigte ihr Volk und ihre Herkunft nicht
an; denn Mardochai hatte ihr geboten, es nicht zu sagen. \bibleverse{11}
Und Mardochai ging alle Tage vor dem Hof am Frauenhause auf und ab, um
zu erfahren, ob es Esther wohlgehe und was mit ihr geschehe.
\bibleverse{12} Wenn die Reihe an eine jede Jungfrau kam, zum König
Ahasveros zu kommen, nachdem sie zwölf Monate lang gemäß der Verordnung
für die Frauen, behandelt worden war (denn damit wurden die Tage ihrer
Reinigung ausgefüllt: sechs Monate wurden sie mit Myrrhenöl und sechs
Monate mit Balsam und mit den Reinigungssalben der Frauen behandelt);
\bibleverse{13} alsdann kam die Jungfrau zum König; dann gab man ihr
alles, was sie begehrte, um damit von dem Frauenhause zu des Königs
Hause zu gehen. \bibleverse{14} Am Abend ging sie hinein, und am Morgen
kam sie zurück, in das andere Frauenhaus, unter die Obhut Schaaschgas,
des Kämmerers des Königs, des Hüters der Nebenfrauen; sie kam nicht
wieder zum König, außer wenn der König nach ihr verlangte; alsdann wurde
sie mit Namen gerufen. \bibleverse{15} Als nun Esther, die Tochter
Abichails, des Oheims Mardochais, die er als Tochter angenommen hatte,
an die Reihe kam, zum König zu kommen, begehrte sie nichts, als was
Hegai, der Kämmerer des Königs, der Hüter der Frauen, ihr riet. Und
Esther fand Gnade vor allen, die sie sahen. \bibleverse{16} Und Esther
ward zum König Ahasveros, in sein königliches Haus genommen, im zehnten
Monat, das ist der Monat Thebet, im siebenten Jahre seiner Regierung.
\bibleverse{17} Und der König gewann Esther lieber als alle andern
Frauen. Sie fand Gnade und Gunst vor ihm, mehr als alle Jungfrauen; und
er setzte die königliche Krone auf ihr Haupt und machte sie zur Königin
an Vastis Statt. \bibleverse{18} Und der König machte allen seinen
Fürsten ein großes Mahl, das Mahl der Esther. Und er veranstaltete eine
Feier in den Provinzen und teilte Gaben aus mit königlicher Hand.
\bibleverse{19} Und als man zum zweitenmal Jungfrauen zusammenbrachte,
saß Mardochai im Tore des Königs. \bibleverse{20} Esther aber hatte
weder ihre Herkunft noch ihr Volk angezeigt, wie ihr Mardochai geboten
hatte. Denn Esther tat nach der Weisung Mardochais, wie zu der Zeit, als
sie von ihm erzogen wurde. \bibleverse{21} In jenen Tagen, als Mardochai
im Tore des Königs saß, waren zwei Kämmerer des Königs, Bigtan und
Teres, welche die Schwelle hüteten, unzufrieden und trachteten, Hand an
den König Ahasveros zu legen. \bibleverse{22} Das ward dem Mardochai
bekannt, und er sagte es der Königin Esther; Esther aber sagte es dem
König in Mardochais Namen. \bibleverse{23} Da wurde die Sache untersucht
und richtig befunden, und die beiden wurden an das Holz gehängt; und
solches ward vor dem König in das Buch der Chronik geschrieben.

\hypertarget{section-2}{%
\section{3}\label{section-2}}

\bibleverse{1} Nach diesen Begebenheiten erhob der König Ahasveros
Haman, den Sohn Hamedatas, den Agagiter, zu höherer Macht und Würde und
setzte seinen Stuhl über alle Fürsten, die bei ihm waren. \bibleverse{2}
Und alle Knechte des Königs, die im Königstore waren, beugten die Knie
und fielen vor Haman nieder; denn der König hatte es also geboten. Aber
Mardochai beugte die Knie nicht und fiel nicht nieder. \bibleverse{3} Da
sprachen die Knechte des Königs, die im Königstore waren, zu Mardochai:
Warum übertrittst du des Königs Gebot? \bibleverse{4} Und als sie
solches täglich zu ihm sagten und er ihnen nicht gehorchte, sagten sie
es Haman, um zu sehen, ob Mardochai auf seiner Weigerung bestehen würde;
denn er hatte ihnen gesagt, daß er ein Jude sei. \bibleverse{5} Als nun
Haman sah, daß Mardochai die Knie nicht vor ihm beugte und ihm nicht
huldigte, ward er voll Grimm. \bibleverse{6} Doch war ihm das zu gering,
an Mardochai allein Hand zu legen; sondern weil man ihm das Volk
Mardochais genannt hatte, trachtete Haman darnach, alle Juden im ganzen
Königreich des Ahasveros, die Volksgenossen Mardochais, zu vertilgen.
\bibleverse{7} Im ersten Monat, das ist der Monat Nisan, im zwölften
Jahre des Königs Ahasveros, ward das Pur, das ist das Los, vor Haman
geworfen über die Tage und Monate, und es fiel auf den dreizehnten Tag
im zwölften Monat, das ist der Monat Adar. \bibleverse{8} Und Haman
sprach zum König Ahasveros: Es gibt ein Volk, das lebt zerstreut und
abgesondert unter allen Völkern in allen Provinzen deines Königreichs,
und ihr Gesetz ist anders als dasjenige aller Völker, und sie tun nicht
nach des Königs Gesetzen; also daß es dem König nicht geziemt, sie in
Ruhe zu lassen! \bibleverse{9} Gefällt es dem König, so schreibe er, daß
man sie umbringe; dann will ich zehntausend Talente Silber darwägen in
die Hände der Schaffner, damit man es in des Königs Schatzkammern
bringe! \bibleverse{10} Da zog der König seinen Siegelring von der Hand
und gab ihn Haman, dem Sohne Hamedatas, dem Agagiter, dem Feinde der
Juden. \bibleverse{11} Und der König sprach zu Haman: Das Silber sei dir
geschenkt, dazu das Volk, damit du mit ihm tuest, was dir gefällt!
\bibleverse{12} Da berief man die Schreiber des Königs am dreizehnten
Tage des ersten Monats, und es ward geschrieben, ganz wie Haman befahl,
an die Fürsten des Königs und an die Landpfleger in allen Provinzen und
an die Hauptleute eines jeden Volkes, in der Schrift einer jeden Provinz
und in der Sprache eines jeden Volkes; im Namen des Königs Ahasveros
ward es geschrieben und mit des Königs Ring versiegelt. \bibleverse{13}
Und die Briefe wurden durch die Läufer in alle Provinzen des Königs
gesandt, daß man alle Juden vertilgen, erwürgen und umbringen solle,
Junge und Alte, Kinder und Frauen, an einem Tage, nämlich am dreizehnten
des zwölften Monats, das ist der Monat Adar, und daß man zugleich ihr
Gut rauben dürfe. \bibleverse{14} Die Schrift aber lautete also, es sei
ein Befehl zu erlassen und in allen Provinzen zu eröffnen, daß sie sich
auf diesen Tag rüsten sollten. \bibleverse{15} Und die Läufer gingen
eilends mit des Königs Gebot, sobald es im Schlosse Susan erlassen war.
Der König aber und Haman setzten sich, um zu trinken, während die Stadt
Susan in Bestürzung geriet.

\hypertarget{section-3}{%
\section{4}\label{section-3}}

\bibleverse{1} Als nun Mardochai alles erfuhr, was geschehen war, zerriß
Mardochai seine Kleider und kleidete sich in Sack und Asche und ging in
die Stadt hinein und klagte laut und bitterlich. \bibleverse{2} Und er
kam bis vor das Königstor; denn es durfte niemand zum Königstor
eingehen, der einen Sack anhatte. \bibleverse{3} Da war auch in allen
Provinzen, wo immer des Königs Wort und Gebot hinkam, unter den Juden
große Klage und Fasten und Weinen und Leidtragen, und viele lagen in
Säcken und in der Asche. \bibleverse{4} Da kamen die Mägde der Esther
und ihre Kämmerer und sagten es ihr; und dieses bekümmerte die Königin
sehr. Und sie sandte Kleider, damit Mardochai sie anzöge und den Sack
von sich legte. Aber er nahm sie nicht an. \bibleverse{5} Da rief Esther
den Hatach, einen Kämmerer des Königs, den er für sie bestellt hatte,
und gab ihm Befehl, bei Mardochai in Erfahrung zu bringen, was das
bedeute und warum es geschehe. \bibleverse{6} Da ging Hatach zu
Mardochai hinaus auf den Platz der Stadt, vor das Königstor.
\bibleverse{7} Und Mardochai tat ihm alles kund, was ihm begegnet war,
auch die genaue Summe Silbers, die Haman versprochen hatte, in der
Schatzkammer des Königs darzuwägen als Entgelt für die Vertilgung der
Juden. \bibleverse{8} Und er gab ihm die Abschrift des Gebots, das
betreffs der Vertilgung zu Susan erlassen worden war, damit er es Esther
zeige und ihr kundtue und ihr gebiete, zum König hineinzugehen, um seine
Gnade zu erflehen und vor seinem Angesicht für ihr Volk zu bitten.
\bibleverse{9} Da ging Hatach hinein und tat Esther die Worte Mardochais
kund. \bibleverse{10} Da sprach Esther zu Hatach und befahl ihm,
Mardochai zu sagen: \bibleverse{11} Alle Knechte des Königs und die
Leute in den königlichen Provinzen wissen, daß, wer irgend in den innern
Hof zum König hineingeht, es sei Mann oder Weib, ohne gerufen zu sein,
nach dem gleichen Gesetz sterben muß, es sei denn, daß der König das
goldene Zepter gegen ihn ausstreckt, damit er am Leben bleibe. Ich aber
bin nun seit dreißig Tagen nicht gerufen worden, zum König
hineinzugehen. \bibleverse{12} Als nun Esthers Worte dem Mardochai
mitgeteilt wurden, \bibleverse{13} ließ Mardochai der Esther antworten:
Bilde dir ja nicht ein, daß du vor allen Juden entrinnen werdest, weil
du in des Königs Hause bist! \bibleverse{14} Denn wenn du unter diesen
Umständen schweigst, so wird den Juden von einer andern Seite her Trost
und Rettung erstehen, du aber und deines Vaters Haus werden umkommen.
Und wer weiß, ob du nicht um dieser Umstände willen zum Königtum
gekommen bist? \bibleverse{15} Da ließ Esther dem Mardochai antworten:
\bibleverse{16} So gehe hin, versammle alle Juden, die zu Susan anwesend
sind, und fastet für mich, drei Tage lang bei Tag und Nacht, esset und
trinket nicht. Auch ich will mit meinen Mägden also fasten, und alsdann
will ich zum König hineingehen, wiewohl es nicht nach dem Gesetze ist.
Komme ich um, so komme ich um! \bibleverse{17} Mardochai ging hin und
tat alles ganz so, wie Esther ihm befohlen hatte.

\hypertarget{section-4}{%
\section{5}\label{section-4}}

\bibleverse{1} Und am dritten Tage legte Esther ihre königliche Kleidung
an und stellte sich in den innern Hof am Hause des Königs, dem Hause des
Königs gegenüber, während der König auf seinem königlichen Throne im
königlichen Hause saß, gegenüber dem Eingang zum Hause. \bibleverse{2}
Als nun der König die Königin Esther im Hofe stehen sah, fand sie Gnade
vor seinen Augen; denn der König streckte das goldene Zepter in seiner
Hand Esther entgegen. Da trat Esther herzu und rührte die Spitze des
Zepters an. \bibleverse{3} Da sprach der König zu ihr: Was hast du,
Königin Esther, und was forderst du? Es soll dir gewährt werden, und
wäre es auch die Hälfte des Königreichs! \bibleverse{4} Esther sprach:
Gefällt es dem König, so komme der König heute mit Haman zu dem Mahl,
das ich ihm zubereitet habe! \bibleverse{5} Der König sprach: Sorget
dafür, daß Haman eilends tue, was Esther gesagt hat! Als nun der König
und Haman zu dem Mahl kamen, welches Esther zugerichtet hatte,
\bibleverse{6} sprach der König zu Esther beim Weingelage: Was bittest
du, Esther? Es soll dir gegeben werden! Und was forderst du? Wäre es
auch die Hälfte des Königreichs, es soll geschehen! \bibleverse{7} Da
antwortete Esther und sprach: Meine Bitte und mein Begehren ist:
\bibleverse{8} Habe ich Gnade gefunden vor dem König, und gefällt es dem
König, mir meine Bitte zu gewähren und meinen Wunsch zu erfüllen, so
komme der König mit Haman zu dem Mahl, das ich für sie zurichten will;
dann will ich morgen tun, was der König gesagt hat! \bibleverse{9} Da
ging Haman an jenem Tage fröhlich und guten Mutes hinaus. Aber als Haman
den Mardochai im Königstore sah, wie er nicht aufstand, noch sich vor
ihm verbeugte, ward er voll Zorn über Mardochai. \bibleverse{10} Doch
Haman überwand sich; als er aber heimkam, sandte er hin und ließ seine
Freunde und sein Weib Seres holen. \bibleverse{11} Und Haman zählte
ihnen die Herrlichkeit seines Reichtums auf und die Menge seiner Söhne
und wie ihn der König so gar groß gemacht und ihn über die Fürsten und
Knechte erhoben habe. \bibleverse{12} Auch sprach Haman: Und die Königin
Esther hat niemand mit dem König zum Mahle kommen lassen, das sie
zugerichtet hat, als mich, und ich bin auch morgen mit dem König zu ihr
geladen! \bibleverse{13} Aber das alles befriedigt mich nicht, solange
ich Mardochai, den Juden, im Königstor sitzen sehe. \bibleverse{14} Da
sprachen sein Weib Seres und alle seine Freunde zu ihm: Man soll einen
Galgen machen, fünfzig Ellen hoch; dann sage du morgen dem König, daß
man Mardochai daran hängen soll, so kannst du fröhlich mit dem König zum
Mahl gehen. Das gefiel Haman wohl, und er ließ den Galgen zurichten.

\hypertarget{section-5}{%
\section{6}\label{section-5}}

\bibleverse{1} In derselben Nacht konnte der König nicht schlafen, und
er ließ das Buch der Denkwürdigkeiten, die Chronik, herbringen; daraus
wurde dem Könige vorgelesen. \bibleverse{2} Da fand sich, daß darin
geschrieben war, wie Mardochai angezeigt habe, daß Bigtana und Teres,
die beiden Kämmerer des Königs, die an der Schwelle hüteten, darnach
getrachtet hatten, Hand an den König Ahasveros zu legen. \bibleverse{3}
Und der König sprach: Was für Ehre und Würde haben wir dafür Mardochai
zuteilwerden lassen? Da sprachen die Knappen des Königs, die ihm
dienten: Man hat ihm gar nichts gegeben! \bibleverse{4} Und der König
fragte: Wer ist im Hofe? Nun war Haman gerade in den äußern Hof des
königlichen Hauses gekommen, um dem König zu sagen, er solle Mardochai
an den Galgen hängen lassen, den er für ihn bereitet hatte.
\bibleverse{5} Da sprachen des Königs Knappen zu ihm: Siehe, Haman steht
im Hof! Der König sprach: Er soll hereinkommen! \bibleverse{6} Als nun
Haman hereinkam, sprach der König zu ihm: Was soll man dem Manne tun,
den der König gern ehren wollte? Haman aber dachte in seinem Herzen: Wem
anders sollte der König Ehre erweisen wollen als mir? \bibleverse{7} Und
Haman sprach zum König: Für den Mann, welchen der König gern ehren
wollte, \bibleverse{8} soll man ein königliches Kleid, welches der König
selbst getragen hat, herbringen und ein Pferd, darauf der König reitet
und auf dessen Kopf eine königliche Krone gesetzt worden ist.
\bibleverse{9} Und man soll das Kleid und das Pferd den Händen eines der
vornehmsten Fürsten des Königs übergeben, damit man den Mann bekleide,
welchen der König gern ehren wollte, und ihn auf dem Pferde in den
Gassen der Stadt umherführen und vor ihm her ausrufen lassen: So tut man
dem Manne, den der König gern ehren will! \bibleverse{10} Da sprach der
König zu Haman: Eile, nimm das Kleid und das Pferd, wie du gesagt hast,
und tue also mit Mardochai, dem Juden, der vor dem Königstor sitzt; es
soll nichts fehlen von allem, was du gesagt hast! \bibleverse{11} Da
nahm Haman das Kleid und das Pferd und bekleidete Mardochai und führte
ihn auf die Gassen der Stadt und rief vor ihm her: So tut man dem Manne,
den der König gern ehren will! \bibleverse{12} Darauf kehrte Mardochai
zum Königstor zurück; Haman aber eilte traurig und mit verhülltem Haupte
nach Hause. \bibleverse{13} Und Haman erzählte seinem Weibe Seres und
allen seinen Freunden alles, was ihm begegnet war. Da sprachen seine
Weisen und sein Weib Seres zu ihm: Wenn Mardochai, vor dem du zu fallen
angefangen hast, von dem Samen der Juden ist, so vermagst du nichts
wider ihn, sondern du wirst gänzlich vor ihm fallen! \bibleverse{14}
Während sie aber noch mit ihm redeten, kamen die Kämmerer des Königs und
beeilten sich, Haman zu dem Mahle zu bringen, welches Esther zugerichtet
hatte.

\hypertarget{section-6}{%
\section{7}\label{section-6}}

\bibleverse{1} So kam nun der König mit Haman zum Trinkgelage bei der
Königin Esther. \bibleverse{2} Da sprach der König zu Esther auch am
zweiten Tage beim Weintrinken: Was bittest du, Königin Esther? Es soll
dir gegeben werden! Und was forderst du? Wäre es auch die Hälfte des
Königreichs, es soll geschehen! \bibleverse{3} Da antwortete die Königin
Esther und sprach: Habe ich Gnade vor dir gefunden, o König, und gefällt
es dem König, so schenke mir das Leben um meiner Bitte willen, und mein
Volk um meines Begehrens willen! \bibleverse{4} Denn wir sind verkauft,
ich und mein Volk, um vertilgt, erwürgt und umgebracht zu werden. Wenn
wir nur zu Knechten und Mägden verkauft würden, so wollte ich schweigen;
aber der Feind wäre nicht imstande, den Schaden des Königs zu ersetzen!
\bibleverse{5} Da sprach der König Ahasveros zu der Königin Esther: Wer
ist der, und wo ist der, welcher sich vorgenommen hat, solches zu tun?
\bibleverse{6} Esther sprach: Der Widersacher und Feind ist dieser böse
Haman! Da erschrak Haman vor dem König und der Königin. \bibleverse{7}
Der König aber stand in seinem Grimm auf vom Weintrinken und ging in den
Garten am Hause. Haman aber stand auf und bat die Königin Esther um sein
Leben; denn er sah, daß sein Verderben beim König beschlossen war.
\bibleverse{8} Und als der König aus dem Garten am Hause wieder in das
Haus kam, wo man den Wein getrunken hatte, lag Haman an dem Polster, auf
welchem Esther saß. Da sprach der König: Will er sogar der Königin
Gewalt antun in meinem Hause? Das Wort war kaum aus des Königs Munde
gegangen, so verhüllten sie dem Haman das Angesicht. \bibleverse{9} Und
Harbona, einer der Kämmerer vor dem König, sprach: Siehe, der Galgen,
welchen Haman für Mardochai gemacht hat, der Gutes für den König geredet
hat, steht schon bei Hamans Hause, fünfzig Ellen hoch! Der König sprach:
Hängt ihn daran! \bibleverse{10} Also hängte man Haman an den Galgen,
welchen er für Mardochai gemacht hatte. Da legte sich der Zorn des
Königs.

\hypertarget{section-7}{%
\section{8}\label{section-7}}

\bibleverse{1} An demselben Tage gab der König Ahasveros der Königin
Esther das Haus Hamans, des Feindes der Juden. Mardochai aber bekam
Zutritt beim König; denn Esther hatte gesagt, wie er ihr zugehörte.
\bibleverse{2} Und der König tat seinen Siegelring ab, den er Haman
abgenommen hatte, und gab ihn Mardochai. Und Esther setzte Mardochai
über das Haus Hamans. \bibleverse{3} Und Esther redete weiter vor dem
König und fiel ihm zu Füßen, weinte und flehte ihn an, daß er die
Bosheit Hamans, des Agagiters, nämlich seinen Anschlag, den er wider die
Juden erdacht hatte, abwenden möchte. \bibleverse{4} Und der König
streckte Esther das goldene Zepter entgegen. Da stand Esther auf und
trat vor den König und sprach: \bibleverse{5} Gefällt es dem König, und
habe ich Gnade vor ihm gefunden, und dünkt es den König gut, und gefalle
ich ihm, so schreibe man, daß die Briefe mit dem Anschlag Hamans, des
Sohnes Hammedatas, des Agagiters, widerrufen werden, welche er
geschrieben hat, um die Juden in allen Provinzen des Königs umzubringen.
\bibleverse{6} Denn wie könnte ich dem Unglück zusehen, das mein Volk
treffen würde? Und wie könnte ich zusehen, wie mein Geschlecht umkommt?
\bibleverse{7} Da sprach der König Ahasveros zur Königin Esther und zu
Mardochai, dem Juden: Seht, ich habe Esther das Haus Hamans gegeben, und
man hat ihn an den Galgen gehängt, weil er seine Hand gegen die Juden
ausgestreckt hat. \bibleverse{8} So schreibt nun betreffs der Juden, wie
es euch gut dünkt, in des Königs Namen, und versiegelt es mit des Königs
Ring; denn die Schrift, die in des Königs Namen geschrieben und mit des
Königs Ring versiegelt worden ist, kann nicht widerrufen werden.
\bibleverse{9} Da wurden des Königs Schreiber zu jener Zeit berufen, im
dritten Monat, das ist der Monat Sivan, am dreiundzwanzigsten Tage
desselben. Und es ward geschrieben, ganz wie Mardochai gebot, an die
Juden und an die Fürsten und Landpfleger und Hauptleute der Provinzen
von Indien bis Äthiopien, nämlich 127 Provinzen, einer jeden Provinz in
ihrer Schrift, und einem jeden Volk in seiner Sprache, auch an die Juden
in ihrer Schrift und in ihrer Sprache. \bibleverse{10} Und es ward
geschrieben im Namen des Königs Ahasveros und versiegelt mit dem Ring
des Königs. Und er sandte Briefe durch reitende Eilboten, die auf
schnellen Rossen aus den königlichen Gestüten ritten; \bibleverse{11}
darin gestattete der König den Juden, sich in allen Städten zu
versammeln und für ihr Leben einzustehen und zu vertilgen, zu erwürgen
und umzubringen alle Volks und Bezirkstruppen, die sie befehden sollten,
mitsamt den Kindern und Frauen, und die ihr Gut rauben wollten;
\bibleverse{12} und zwar an einem Tage in allen Provinzen des Königs
Ahasveros, nämlich am dreizehnten Tage des zwölften Monats, das ist der
Monat Adar. \bibleverse{13} Der Inhalt der Schrift aber war, es sei in
allen Provinzen ein Gebot zu erlassen und allen Völkern zu eröffnen, daß
die Juden auf jenen Tag bereit sein sollten, sich an ihren Feinden zu
rächen. \bibleverse{14} Und Eilboten, die auf schnellen Rossen aus den
königlichen Gestüten ritten, gingen auf Befehl des Königs schleunigst
und eilend aus, sobald das Gesetz im Schloß Susan erlassen war.
\bibleverse{15} Mardochai aber verließ den König in königlichen
Kleidern, in blauem Purpur und feiner weißer Baumwolle und mit einer
großen goldenen Krone und einem Mantel von weißer Baumwolle und rotem
Purpur; und die Stadt Susan jauchzte und war fröhlich. \bibleverse{16}
Für die Juden aber war Licht und Freude, Wonne und Ehre entstanden.
\bibleverse{17} Und in allen Provinzen und in allen Städten, wohin des
Königs Wort und Gebot gelangte, da war Freude und Wonne unter den Juden,
Gastmahl und Festtag, so daß viele von der Bevölkerung des Landes Juden
wurden; denn die Furcht vor den Juden war auf sie gefallen.

\hypertarget{section-8}{%
\section{9}\label{section-8}}

\bibleverse{1} Im zwölften Monat nun, das ist der Monat Adar, am
dreizehnten Tage, an welchem des Königs Wort und Gebot in Erfüllung
gehen sollte, an eben dem Tage, an welchem die Feinde der Juden hofften,
sie zu überwältigen, wandte es sich so, daß die Juden ihre Hasser
überwältigen durften. \bibleverse{2} Da versammelten sich die Juden in
ihren Städten, in sämtlichen Provinzen des Königs Ahasveros, um Hand an
die zu legen, die ihnen übel wollten, und niemand konnte ihnen
widerstehen; denn die Furcht vor ihnen war auf alle Völker gefallen.
\bibleverse{3} Auch alle Vorsteher der Provinzen und die Satrapen und
Landpfleger und die Amtleute des Königs unterstützten die Juden; denn
die Furcht vor Mardochai war auf sie gefallen. \bibleverse{4} Denn
Mardochai galt viel am Hofe des Königs, und sein Ruf erscholl in alle
Provinzen; denn der Mann Mardochai ward immer größer. \bibleverse{5}
Also schlugen die Juden alle ihre Feinde mit dem Schwerte und erwürgten
und brachten sie um und verfuhren mit ihren Hassern nach ihrem Belieben.
\bibleverse{6} Auch in der Burg Susan erwürgten die Juden und brachten
500 Mann um. \bibleverse{7} Dazu erschlugen sie Parsandata,
\bibleverse{8} Dalphon, Aspata, Porata, \bibleverse{9} Adalja, Aridata,
Parmasta, Arisai, Aridai und Vajesata, \bibleverse{10} die zehn Söhne
Hamans, des Sohnes Hammedatas, des Feindes der Juden; aber an ihre Güter
legten sie die Hand nicht. \bibleverse{11} An jenem Tage erfuhr der
König die Zahl der in der Burg Susan Getöteten. \bibleverse{12} Und der
König sprach zu der Königin Esther: Die Juden haben in der Burg Susan
500 Mann erschlagen und umgebracht, dazu die zehn Söhne Hamans. Was
haben sie getan in den andern Provinzen des Königs? Was bittest du nun?
Es soll dir gegeben werden. Und was forderst du mehr? Es soll geschehen!
\bibleverse{13} Esther sprach: Gefällt es dem König, so lasse er auch
morgen die Juden zu Susan handeln nach der heutigen Verordnung; die zehn
Söhne Hamans aber soll man an den Galgen hängen! \bibleverse{14} Da
befahl der König, solches zu tun, und das Gebot ward zu Susan erlassen,
und die zehn Söhne Hamans wurden gehängt. \bibleverse{15} Und die Juden
versammelten sich zu Susan am vierzehnten Tage des Monats Adar und
erwürgten zu Susan 300 Mann; aber an ihre Güter legten sie die Hand
nicht. \bibleverse{16} Auch die übrigen Juden, die in den Ländern des
Königs waren, kamen zusammen und standen für ihr Leben ein und
verschafften sich Ruhe vor ihren Feinden, und sie töteten von ihren
Feinden 75000; aber an ihre Güter legten sie die Hand nicht.
\bibleverse{17} Das geschah am dreizehnten Tage des Monats Adar, und sie
ruhten am vierzehnten Tage desselben Monats und machten ihn zu einem
Tage des Gastmahls und der Freuden. \bibleverse{18} Aber die Juden zu
Susan waren am dreizehnten und vierzehnten Tage dieses Monats
zusammengekommen und ruhten am fünfzehnten Tage; und sie machten diesen
Tag zu einem Tage des Gastmahls und der Freuden. \bibleverse{19} Darum
machen die Juden auf dem Lande, welche in den offenen Städten wohnen,
den vierzehnten Tag des Monats Adar zu einem Tage der Freude, des
Gastmahls und zum Festtag und senden einander Geschenke. \bibleverse{20}
Und Mardochai schrieb diese Begebenheiten auf und sandte Briefe an alle
Juden, die in allen Provinzen des Königs Ahasveros wohnten, in der Nähe
und in der Ferne, \bibleverse{21} indem er ihnen verordnete, daß sie den
vierzehnten und fünfzehnten Tag des Monats Adar alle Jahre feiern
sollten als Tage, \bibleverse{22} an denen die Juden vor ihren Feinden
zur Ruhe gekommen waren, und als Monat, in welchem ihr Kummer in Freude
und ihr Leid in gute Tage verwandelt worden war; daß sie die feiern
sollten als Tage des Gastmahls und der Freuden, an denen sie einander
Geschenke machen und die Armen beschenken sollten. \bibleverse{23} Und
die Juden machten sich das, was sie zu tun angefangen hatten und was
ihnen Mardochai vorgeschrieben hatte, zur Gewohnheit. \bibleverse{24}
Weil Haman, der Sohn Hammedatas, der Agagiter, aller Juden Feind, den
Plan gefaßt hatte, alle Juden umzubringen, und weil er das Pur, das ist
das Los, hatte werfen lassen, um sie aufzureiben und umzubringen;
\bibleverse{25} während Esther dadurch, daß sie vor den König kam,
bewirkte, daß er durch Briefe befahl, Hamans bösen Anschlag, den er
wider die Juden erdacht hatte, auf seinen eigenen Kopf zu lenken, so daß
man ihn und seine Söhne an den Galgen hängte. \bibleverse{26} Darum
werden diese Tage Purim genannt, nach dem Worte Pur. Um deswillen und
wegen alles dessen, was in dem Schriftstücke stand, was sie selbst
gesehen und erfahren hatten, \bibleverse{27} setzten die Juden solches
fest und nahmen es an für sich und ihre Nachkommen und alle, die sich
ihnen anschließen würden, daß sie nicht davon abgehen wollten, jährlich
diese zwei Tage zu halten, wie sie vorgeschrieben und bestimmt worden
waren. \bibleverse{28} Und so sollen diese Tage im Gedächtnis bleiben
und gefeiert werden von Geschlecht zu Geschlecht, in allen Provinzen und
Städten; so daß diese Purimtage nie verschwinden sollen aus der Mitte
der Juden und ihr Gedächtnis bei ihren Nachkommen nicht aufhören soll.
\bibleverse{29} Und die Königin Esther, die Tochter Abichails, und
Mardochai, der Jude, schrieben mit allem Nachdruck, um diesen zweiten
Brief betreffend die Purim zu bestätigen. \bibleverse{30} Und er sandte
Briefe an alle Juden in den 127 Provinzen des Königreichs Ahasveros`,
Worte des Friedens und der Wahrheit, \bibleverse{31} um diese Purimtage
zu ihren bestimmten Zeiten festzusetzen, wie Mardochai, der Jude, und
die Königin Ester ihnen verordnet und wie sie sie auch für sich selbst
und für ihre Nachkommen festgesetzt hatten, nämlich die Geschichten der
Fasten und ihrer Wehklage. \bibleverse{32} Und der Befehl Esthers
bestätigte diese Purimgeschichte, und er wurde in einem Buche
aufgezeichnet.

\hypertarget{section-9}{%
\section{10}\label{section-9}}

\bibleverse{1} Und der König Ahasveros legte dem Festland und den Inseln
des Meeres einen Tribut auf. \bibleverse{2} Aber alle Werke seiner
Gewalt und seiner Macht und die Beschreibung der Größe Mardochais, zu
welcher ihn der König erhob, ist das nicht aufgezeichnet in der Chronik
der Könige von Medien und Persien? \bibleverse{3} Denn der Jude
Mardochai war der Nächste nach dem König Ahasveros und groß unter den
Juden und beliebt bei der Menge seiner Brüder, weil er das Beste seines
Volkes suchte und mit all seinem Geschlecht freundlich redete!
