\hypertarget{section}{%
\section{1}\label{section}}

\bibleverse{1} Paulus, Knecht Jesu Christi, berufener Apostel,
ausgesondert zum Evangelium Gottes, \bibleverse{2} welches vorher
verheißen wurde durch seine Propheten in heiligen Schriften,
\bibleverse{3} betreffs seines Sohnes, der hervorgegangen ist aus dem
Samen Davids nach dem Fleisch \bibleverse{4} und erwiesen als Sohn
Gottes in Kraft nach dem Geiste der Heiligkeit durch die Auferstehung
von den Toten, Jesus Christus, unser Herr; \bibleverse{5} durch welchen
wir Gnade und Apostelamt empfangen haben, um für seinen Namen
Glaubensgehorsam zu verlangen unter allen Völkern, \bibleverse{6} unter
welchen auch ihr seid, Berufene Jesu Christi; \bibleverse{7} allen zu
Rom anwesenden Geliebten Gottes, den berufenen Heiligen: Gnade sei mit
euch und Friede von Gott, unsrem Vater, und dem Herrn Jesus Christus!
\bibleverse{8} Zuerst danke ich meinem Gott durch Jesus Christus für
euch alle, daß euer Glaube in der ganzen Welt verkündigt wird.
\bibleverse{9} Denn Gott, welchem ich in meinem Geist diene am
Evangelium seines Sohnes, ist mein Zeuge, wie unablässig ich euer
gedenke, \bibleverse{10} indem ich allezeit in meinen Gebeten flehe, ob
mir nicht endlich einmal durch den Willen Gottes das Glück zuteil werden
möchte, zu euch zu kommen. \bibleverse{11} Denn mich verlangt darnach,
euch zu sehen, um euch etwas geistliche Gabe mitzuteilen, damit ihr
gestärkt werdet, \bibleverse{12} das heißt aber, daß ich mitgetröstet
werde unter euch durch den gemeinschaftlichen Glauben, den euren und den
meinen. \bibleverse{13} Ich will euch aber nicht verschweigen, meine
Brüder, daß ich mir schon oftmals vorgenommen habe, zu euch zu kommen
(ich wurde aber verhindert bis jetzt), um auch unter euch etwas Frucht
zu schaffen, gleichwie unter den übrigen Nationen; \bibleverse{14} denn
ich bin ein Schuldner sowohl den Griechen als den Barbaren, sowohl den
Weisen als den Unverständigen; \bibleverse{15} darum bin ich, soviel an
mir liegt, bereit, auch euch in Rom das Evangelium zu verkündigen.
\bibleverse{16} Denn ich schäme mich des Evangeliums nicht; denn es ist
Gottes Kraft zur Rettung für jeden, der glaubt, zuerst für den Juden,
dann auch für den Griechen; \bibleverse{17} denn es wird darin
geoffenbart die Gerechtigkeit Gottes aus Glauben zum Glauben, wie
geschrieben steht: ``Der Gerechte wird infolge von Glauben leben''.
\bibleverse{18} Es offenbart sich nämlich Gottes Zorn vom Himmel her
über alle Gottlosigkeit und Ungerechtigkeit der Menschen, welche die
Wahrheit durch Ungerechtigkeit aufhalten, \bibleverse{19} weil das von
Gott Erkennbare unter ihnen offenbar ist, da Gott es ihnen geoffenbart
hat; \bibleverse{20} denn sein unsichtbares Wesen, das ist seine ewige
Kraft und Gottheit, wird seit Erschaffung der Welt an den Werken durch
Nachdenken wahrgenommen, so daß sie keine Entschuldigung haben.
\bibleverse{21} Denn obschon sie Gott erkannten, haben sie ihn doch
nicht als Gott gepriesen und ihm nicht gedankt, sondern sind in ihren
Gedanken in eitlen Wahn verfallen, und ihr unverständiges Herz wurde
verfinstert. \bibleverse{22} Da sie sich für weise hielten, sind sie zu
Narren geworden \bibleverse{23} und haben die Herrlichkeit des
unvergänglichen Gottes vertauscht mit dem Bild vom vergänglichen
Menschen, von Vögeln und vierfüßigen und kriechenden Tieren.
\bibleverse{24} Darum hat sie auch Gott dahingegeben in die Gelüste
ihrer Herzen, zur Unreinigkeit, daß sie ihre eigenen Leiber
untereinander entehren, \bibleverse{25} sie, welche die Wahrheit Gottes
mit der Lüge vertauschten und dem Geschöpf mehr Ehre und Dienst erwiesen
als dem Schöpfer, der da gelobt ist in Ewigkeit. Amen! \bibleverse{26}
Darum hat sie Gott auch dahingegeben in entehrende Leidenschaften. Denn
ihre Frauen haben den natürlichen Gebrauch vertauscht mit dem
widernatürlichen; \bibleverse{27} gleicherweise haben auch die Männer
den natürlichen Verkehr mit der Frau verlassen und sind gegeneinander
entbrannt in ihrer Begierde und haben Mann mit Mann Schande getrieben
und den verdienten Lohn ihrer Verirrung an sich selbst empfangen.
\bibleverse{28} Und gleichwie sie Gott nicht der Anerkennung würdigten,
hat Gott auch sie dahingegeben in unwürdigen Sinn, zu verüben, was sich
nicht geziemt, \bibleverse{29} als solche, die voll sind von aller
Ungerechtigkeit, Schlechtigkeit, Habsucht, Bosheit; voll Neid, Mordlust,
Zank, Trug und Tücke, \bibleverse{30} Ohrenbläser, Verleumder,
Gottesverächter, Freche, Übermütige, Prahler, erfinderisch im Bösen, den
Eltern ungehorsam; \bibleverse{31} unverständig, unbeständig, lieblos,
unversöhnlich, unbarmherzig; \bibleverse{32} welche, wiewohl sie das
Urteil Gottes kennen, daß die, welche solches verüben, des Todes würdig
sind, es nicht nur selbst tun, sondern auch Gefallen haben an denen, die
es verüben.

\hypertarget{section-1}{%
\section{2}\label{section-1}}

\bibleverse{1} Darum bist du nicht zu entschuldigen, o Mensch, wer du
seist, der du richtest! Denn indem du den andern richtest, verdammst du
dich selbst; denn du verübst ja dasselbe, was du richtest!
\bibleverse{2} Wir wissen aber, daß das Gericht Gottes dem wahren
Sachverhalt entsprechend über die ergeht, welche solches verüben.
\bibleverse{3} Oder denkst du, o Mensch, der du die richtest, welche
solches verüben, und doch das Gleiche tust, daß du dem Gerichte Gottes
entrinnen werdest? \bibleverse{4} Oder verachtest du den Reichtum seiner
Güte, Geduld und Langmut, ohne zu erkennen, daß dich Gottes Güte zur
Buße leitet? \bibleverse{5} Aber nach deinem verstockten und
unbußfertigen Herzen häufst du dir selbst Zorn auf für den Tag des Zorns
und der Offenbarung des gerechten Gerichtes Gottes, \bibleverse{6}
welcher einem jeglichen vergelten wird nach seinen Werken;
\bibleverse{7} denen nämlich, die mit Ausdauer im Wirken des Guten
Herrlichkeit, Ehre und Unsterblichkeit erstreben, ewiges Leben;
\bibleverse{8} den Streitsüchtigen aber, welche der Wahrheit ungehorsam
sind, dagegen der Ungerechtigkeit gehorchen, Zorn und Grimm!
\bibleverse{9} Trübsal und Angst über jede Menschenseele, die das Böse
vollbringt, zuerst über den Juden, dann auch über den Griechen;
\bibleverse{10} Herrlichkeit aber und Ehre und Friede jedem, der das
Gute wirkt, zuerst dem Juden, dann auch dem Griechen; \bibleverse{11}
denn es gibt kein Ansehen der Person bei Gott: \bibleverse{12} Welche
ohne Gesetz gesündigt haben, die werden auch ohne Gesetz verloren gehen;
und welche unter dem Gesetz gesündigt haben, die werden durch das Gesetz
verurteilt werden. \bibleverse{13} Denn vor Gott sind nicht die gerecht,
welche das Gesetz hören; sondern die, welche das Gesetz befolgen, sollen
gerechtfertigt werden. \bibleverse{14} Denn wenn die Heiden, die das
Gesetz nicht haben, doch von Natur tun, was das Gesetz verlangt, so sind
sie, die das Gesetz nicht haben, sich selbst ein Gesetz; \bibleverse{15}
da sie ja beweisen, daß des Gesetzes Werk in ihre Herzen geschrieben
ist, was auch ihr Gewissen bezeugt, dazu ihre Überlegungen, welche sich
untereinander verklagen oder entschuldigen. \bibleverse{16} Das wird an
dem Tage offenbar werden, da Gott das Verborgene der Menschen richten
wird, laut meinem Evangelium, durch Jesus Christus. \bibleverse{17} Wenn
du dich aber einen Juden nennst und dich auf das Gesetz verlässest und
dich Gottes rühmst, \bibleverse{18} wenn du seinen Willen weißt und
verschiedenartige Dinge zu unterscheiden verstehst, weil du aus dem
Gesetz unterrichtet bist; \bibleverse{19} wenn du dir zutraust, ein
Leiter der Blinden, ein Licht derer zu sein, die in der Finsternis sind,
\bibleverse{20} ein Erzieher der Unverständigen, ein Lehrer der
Unmündigen, der den Inbegriff der Erkenntnis und der Wahrheit im Gesetze
hat: \bibleverse{21} nun also, du lehrst andere, dich selbst aber lehrst
du nicht? Du predigst, man solle nicht stehlen, und stiehlst selber?
\bibleverse{22} Du sagst, man solle nicht ehebrechen, und brichst selbst
die Ehe? Du verabscheust die Götzen und begehst dabei Tempelraub?
\bibleverse{23} Du rühmst dich des Gesetzes und verunehrst doch Gott
durch Übertretung des Gesetzes? \bibleverse{24} wie geschrieben steht:
``Der Name Gottes wird um euretwillen unter den Heiden gelästert.''
\bibleverse{25} Denn die Beschneidung hat nur Wert, wenn du das Gesetz
hältst; bist du aber ein Übertreter des Gesetzes, so ist deine
Beschneidung schon zur Unbeschnittenheit geworden. \bibleverse{26} Wenn
nun der Unbeschnittene die Forderungen des Gesetzes beobachtet, wird ihm
nicht seine Unbeschnittenheit als Beschneidung angerechnet werden?
\bibleverse{27} Und wird nicht der von Natur Unbeschnittene, der das
Gesetz erfüllt, dich richten, der du trotz Buchstabe und Beschneidung
ein Übertreter des Gesetzes bist? \bibleverse{28} Denn nicht der ist ein
Jude, der es äußerlich ist; auch ist nicht das die Beschneidung, die
äußerlich am Fleisch geschieht; \bibleverse{29} sondern der ist ein
Jude, der es innerlich ist, und das ist eine Beschneidung, die am
Herzen, im Geiste, nicht dem Buchstaben nach vollzogen wird. Eines
solchen Lob kommt nicht von Menschen, sondern von Gott.

\hypertarget{section-2}{%
\section{3}\label{section-2}}

\bibleverse{1} Was hat nun der Jude für einen Vorzug, oder was nützt die
Beschneidung? \bibleverse{2} Viel, in jeder Hinsicht! Erstens sind ihnen
die Aussprüche Gottes anvertraut worden! \bibleverse{3} Wie denn? Wenn
auch etliche ungläubig waren, hebt etwa ihr Unglaube die Treue Gottes
auf? \bibleverse{4} Das sei ferne! Vielmehr erweist sich Gott als
wahrhaftig, jeder Mensch aber als Lügner, wie geschrieben steht: ``Auf
daß du gerecht befunden werdest in deinen Worten und siegreich, wenn du
gerichtet wirst.'' \bibleverse{5} Wenn aber unsere Ungerechtigkeit
Gottes Gerechtigkeit beweist, was sollen wir sagen? Ist dann Gott nicht
ungerecht, wenn er darüber zürnt? (Ich rede nach Menschenweise.)
\bibleverse{6} Das sei ferne! Wie könnte Gott sonst die Welt richten?
\bibleverse{7} Wenn aber die Wahrhaftigkeit Gottes durch meine Lüge
überfließender wird zu seinem Ruhm, was werde ich dann noch als Sünder
gerichtet? \bibleverse{8} Müßte man dann nicht so reden, wie wir
verleumdet werden und wie etliche behaupten, daß wir sagen: ``Lasset uns
Böses tun, damit Gutes daraus komme''? Ihre Verurteilung ist gerecht!
\bibleverse{9} Wie nun? Haben wir etwas voraus? Ganz und gar nichts!
Denn wir haben ja vorhin sowohl Juden als Griechen beschuldigt, daß sie
alle unter der Sünde sind, \bibleverse{10} wie geschrieben steht: ``Es
ist keiner gerecht, auch nicht einer; \bibleverse{11} es ist keiner
verständig, keiner fragt nach Gott; \bibleverse{12} alle sind
abgewichen, sie taugen alle zusammen nichts; es ist keiner, der Gutes
tut, auch nicht einer! \bibleverse{13} Ihre Kehle ist ein offenes Grab,
mit ihren Zungen trügen sie; Otterngift ist unter ihren Lippen;
\bibleverse{14} ihr Mund ist voll Fluchens und Bitterkeit,
\bibleverse{15} ihre Füße sind eilig, um Blut zu vergießen;
\bibleverse{16} Verwüstung und Jammer bezeichnen ihre Bahn,
\bibleverse{17} und den Weg des Friedens kennen sie nicht.
\bibleverse{18} Es ist keine Gottesfurcht vor ihren Augen.''
\bibleverse{19} Wir wissen aber, daß das Gesetz alles, was es spricht,
denen sagt, die unter dem Gesetze sind, auf daß jeder Mund verstopft
werde und alle Welt vor Gott schuldig sei, \bibleverse{20} weil aus
Gesetzeswerken kein Fleisch vor ihm gerechtfertigt werden kann; denn
durch das Gesetz kommt Erkenntnis der Sünde. \bibleverse{21} Nun aber
ist außerhalb vom Gesetz die Gerechtigkeit Gottes geoffenbart worden,
die von dem Gesetz und den Propheten bezeugt wird, \bibleverse{22}
nämlich die Gerechtigkeit Gottes, veranlaßt durch den Glauben an Jesus
Christus, für alle, die da glauben. \bibleverse{23} Denn es ist kein
Unterschied: Alle haben gesündigt und ermangeln der Herrlichkeit Gottes,
\bibleverse{24} so daß sie gerechtfertigt werden ohne Verdienst, durch
seine Gnade, mittels der Erlösung, die in Christus Jesus ist.
\bibleverse{25} Ihn hat Gott zum Sühnopfer verordnet, durch sein Blut,
für alle, die glauben, zum Erweis seiner Gerechtigkeit, wegen der
Nachsicht mit den Sünden, die zuvor geschehen waren unter göttlicher
Geduld, \bibleverse{26} zur Erweisung seiner Gerechtigkeit in der
jetzigen Zeit, damit er selbst gerecht sei und zugleich den
rechtfertige, der aus dem Glauben an Jesus ist. \bibleverse{27} Wo
bleibt nun das Rühmen? Es ist ausgeschlossen? Durch welches Gesetz? Das
der Werke? Nein, sondern durch das Gesetz des Glaubens! \bibleverse{28}
So kommen wir zu dem Schluß, daß der Mensch durch den Glauben
gerechtfertigt werde, ohne Gesetzeswerke. \bibleverse{29} Oder ist Gott
nur der Juden Gott, nicht auch der Heiden? Ja freilich, auch der Heiden!
\bibleverse{30} Denn es ist ja ein und derselbe Gott, welcher die
Beschnittenen aus Glauben und die Unbeschnittenen durch den Glauben
rechtfertigt. \bibleverse{31} Heben wir nun das Gesetz auf durch den
Glauben? Das sei ferne! Vielmehr richten wir das Gesetz auf.

\hypertarget{section-3}{%
\section{4}\label{section-3}}

\bibleverse{1} Was wollen wir nun von dem sagen, was unser Vater Abraham
erlangt hat nach dem Fleisch? \bibleverse{2} Wenn Abraham aus Werken
gerechtfertigt worden ist, hat er zwar Ruhm, aber nicht vor Gott.
\bibleverse{3} Denn was sagt die Schrift? ``Abraham aber glaubte Gott,
und das wurde ihm zur Gerechtigkeit angerechnet.'' \bibleverse{4} Wer
aber Werke verrichtet, dem wird der Lohn nicht als Gnade angerechnet,
sondern nach Schuldigkeit; \bibleverse{5} wer dagegen keine Werke
verrichtet, sondern an den glaubt, der den Gottlosen rechtfertigt, dem
wird sein Glaube als Gerechtigkeit angerechnet. \bibleverse{6} Ebenso
spricht auch David die Seligpreisung des Menschen aus, welchem Gott
Gerechtigkeit anrechnet ohne Werke: \bibleverse{7} ``Selig sind die,
welchen die Übertretungen vergeben und deren Sünden zugedeckt sind;
\bibleverse{8} selig ist der Mann, welchem der Herr die Sünde nicht
zurechnet!'' \bibleverse{9} Gilt nun diese Seligpreisung den
Beschnittenen oder auch den Unbeschnittenen? Wir sagen ja, daß dem
Abraham der Glaube als Gerechtigkeit angerechnet worden sei.
\bibleverse{10} Wie wurde er ihm nun angerechnet? Als er beschnitten
oder als er noch unbeschnitten war? Nicht als er beschnitten, sondern
als er noch unbeschnitten war! \bibleverse{11} Und er empfing das
Zeichen der Beschneidung als Siegel der Gerechtigkeit des Glaubens,
welchen er schon vor der Beschneidung hatte; auf daß er ein Vater aller
unbeschnittenen Gläubigen sei, damit auch ihnen die Gerechtigkeit
zugerechnet werde; \bibleverse{12} und auch ein Vater der Beschnittenen,
die nicht nur aus der Beschneidung sind, sondern auch wandeln in den
Fußstapfen des Glaubens, den unser Vater Abraham hatte, als er noch
unbeschnitten war. \bibleverse{13} Denn nicht durch das Gesetz erhielt
Abraham und sein Same die Verheißung, daß er der Welt Erbe sein solle,
sondern durch die Gerechtigkeit des Glaubens. \bibleverse{14} Denn wenn
die vom Gesetz Erben sind, so ist der Glaube wertlos geworden und die
Verheißung entkräftet. \bibleverse{15} Denn das Gesetz bewirkt Zorn; wo
aber kein Gesetz ist, da ist auch keine Übertretung. \bibleverse{16}
Darum geschah es durch den Glauben, damit es aus Gnaden sei, auf daß die
Verheißung dem ganzen Samen gesichert sei, nicht nur demjenigen aus dem
Gesetz, sondern auch dem vom Glauben Abrahams, welcher unser aller Vater
ist; \bibleverse{17} wie geschrieben steht: ``Ich habe dich zum Vater
vieler Völker gesetzt'' vor dem Gott, dem er glaubte, welcher die Toten
lebendig macht und dem ruft, was nicht ist, als wäre es da.
\bibleverse{18} Er hat gegen alle Hoffnung auf Hoffnung hin geglaubt,
daß er ein Vater vieler Völker werde, wie zu ihm gesagt worden war:
``Also soll dein Same sein!'' \bibleverse{19} Und er wurde nicht schwach
im Glauben, so daß er seinen schon erstorbenen Leib in Betracht gezogen
hätte, weil er schon hundertjährig war; auch nicht den erstorbenen
Mutterleib der Sara. \bibleverse{20} Er zweifelte nicht an der
Verheißung Gottes durch Unglauben, sondern wurde stark durch den
Glauben, indem er Gott die Ehre gab \bibleverse{21} und völlig überzeugt
war, daß Gott das, was er verheißen habe, auch zu tun vermöge.
\bibleverse{22} Darum wurde es ihm auch als Gerechtigkeit angerechnet.
\bibleverse{23} Es ist aber nicht allein um seinetwillen geschrieben,
daß es ihm zugerechnet worden ist, \bibleverse{24} sondern auch um
unsertwillen, denen es zugerechnet werden soll, wenn wir an den glauben,
der unsren Herrn Jesus Christus von den Toten auferweckt hat,
\bibleverse{25} welcher um unserer Übertretungen willen dahingegeben und
zu unserer Rechtfertigung auferweckt worden ist.

\hypertarget{section-4}{%
\section{5}\label{section-4}}

\bibleverse{1} Da wir nun durch den Glauben gerechtfertigt sind, so
haben wir Frieden mit Gott durch unsren Herrn Jesus Christus,
\bibleverse{2} durch welchen wir auch im Glauben Zutritt erlangt haben
zu der Gnade, in der wir stehen, und rühmen uns der Hoffnung auf die
Herrlichkeit Gottes. \bibleverse{3} Aber nicht nur das, sondern wir
rühmen uns auch in den Trübsalen, weil wir wissen, daß die Trübsal
Standhaftigkeit wirkt; \bibleverse{4} die Standhaftigkeit aber
Bewährung, die Bewährung aber Hoffnung; \bibleverse{5} die Hoffnung aber
läßt nicht zuschanden werden; denn die Liebe Gottes ist ausgegossen in
unsre Herzen durch den heiligen Geist, welcher uns gegeben worden ist.
\bibleverse{6} Denn Christus ist, als wir noch schwach waren, zur
rechten Zeit für Gottlose gestorben. \bibleverse{7} Nun stirbt kaum
jemand für einen Gerechten; für einen Wohltäter entschließt sich
vielleicht jemand zu sterben. \bibleverse{8} Gott aber beweist seine
Liebe gegen uns damit, daß Christus für uns gestorben ist, als wir noch
Sünder waren. \bibleverse{9} Wieviel mehr werden wir nun, nachdem wir
durch sein Blut gerechtfertigt worden sind, durch ihn vor dem
Zorngericht errettet werden! \bibleverse{10} Denn, wenn wir, als wir
noch Feinde waren, mit Gott versöhnt worden sind durch den Tod seines
Sohnes, wieviel mehr werden wir als Versöhnte gerettet werden durch sein
Leben! \bibleverse{11} Aber nicht nur das, sondern wir rühmen uns auch
Gottes durch unsren Herrn Jesus Christus, durch welchen wir nun die
Versöhnung empfangen haben. \bibleverse{12} Darum, gleichwie durch einen
Menschen die Sünde in die Welt gekommen ist und durch die Sünde der Tod,
und so der Tod zu allen Menschen hindurchgedrungen ist, weil sie alle
gesündigt haben \bibleverse{13} denn schon vor dem Gesetz war die Sünde
in der Welt; wo aber kein Gesetz ist, da wird die Sünde nicht
angerechnet. \bibleverse{14} Dennoch herrschte der Tod von Adam bis Mose
auch über die, welche nicht mit gleicher Übertretung gesündigt hatten
wie Adam, der ein Vorbild des Zukünftigen ist. \bibleverse{15} Aber es
verhält sich mit dem Sündenfall nicht wie mit der Gnadengabe. Denn wenn
durch des einen Sündenfall die vielen gestorben sind, wieviel mehr ist
die Gnade Gottes und das Gnadengeschenk durch den einen Menschen Jesus
Christus den vielen reichlich zuteil geworden. \bibleverse{16} Und es
verhält sich mit der Sünde durch den einen nicht wie mit dem Geschenk.
Denn das Urteil wurde wegen des einen zur Verurteilung; die Gnadengabe
aber wird trotz vieler Sündenfälle zur Rechtfertigung. \bibleverse{17}
Denn wenn infolge des Sündenfalles des einen der Tod zur Herrschaft kam
durch den einen, wieviel mehr werden die, welche den Überfluß der Gnade
und der Gabe der Gerechtigkeit empfangen, im Leben herrschen durch den
Einen, Jesus Christus! \bibleverse{18} Also: wie der Sündenfall des
einen zur Verurteilung aller Menschen führte, so führt auch das gerechte
Tun des Einen alle Menschen zur lebenbringenden Rechtfertigung.
\bibleverse{19} Denn gleichwie durch den Ungehorsam des einen Menschen
die vielen zu Sündern gemacht worden sind, so werden auch durch den
Gehorsam des Einen die vielen zu Gerechten gemacht. \bibleverse{20} Das
Gesetz aber ist daneben hereingekommen, damit das Maß der Sünden voll
würde. Wo aber das Maß der Sünde voll geworden ist, da ist die Gnade
überfließend geworden, \bibleverse{21} auf daß, gleichwie die Sünde
geherrscht hat im Tode, also auch die Gnade herrsche durch Gerechtigkeit
zu ewigem Leben, durch Jesus Christus, unsren Herrn.

\hypertarget{section-5}{%
\section{6}\label{section-5}}

\bibleverse{1} Was wollen wir nun sagen? Sollen wir in der Sünde
verharren, damit das Maß der Gnade voll werde? \bibleverse{2} Das sei
ferne! Wie sollten wir, die wir der Sünde gestorben sind, noch in ihr
leben? \bibleverse{3} Oder wisset ihr nicht, daß wir alle, die wir auf
Jesus Christus getauft sind, auf seinen Tod getauft sind? \bibleverse{4}
Wir sind also mit ihm begraben worden durch die Taufe auf den Tod, auf
daß, gleichwie Christus durch die Herrlichkeit des Vaters von den Toten
auferweckt worden ist, so auch wir in einem neuen Leben wandeln.
\bibleverse{5} Denn wenn wir mit ihm verwachsen sind zur Ähnlichkeit
seines Todes, so werden wir es auch zu der seiner Auferstehung sein,
\bibleverse{6} wissen wir doch, daß unser alter Mensch mitgekreuzigt
worden ist, damit der Leib der Sünde außer Wirksamkeit gesetzt sei, so
daß wir der Sünde nicht mehr dienen; \bibleverse{7} denn wer gestorben
ist, der ist von der Sünde losgesprochen. \bibleverse{8} Sind wir aber
mit Christus gestorben, so glauben wir, daß wir auch mit ihm leben
werden, \bibleverse{9} da wir wissen, daß Christus, von den Toten
erweckt, nicht mehr stirbt; der Tod herrscht nicht mehr über ihn;
\bibleverse{10} denn was er gestorben ist, das ist er der Sünde
gestorben, ein für allemal; was er aber lebt, das lebt er für Gott.
\bibleverse{11} Also auch ihr: Haltet euch selbst dafür, daß ihr für die
Sünde tot seid, aber für Gott lebet in Christus Jesus, unsrem Herrn!
\bibleverse{12} So soll nun die Sünde nicht herrschen in eurem
sterblichen Leibe, so daß ihr seinen Lüsten gehorchet; \bibleverse{13}
gebet auch nicht eure Glieder der Sünde hin, als Waffen der
Ungerechtigkeit, sondern gebet euch selbst Gott hin, als solche, die aus
Toten lebendig geworden sind, und eure Glieder Gott, als Waffen der
Gerechtigkeit. \bibleverse{14} Denn die Sünde wird nicht herrschen über
euch, weil ihr nicht unter dem Gesetz, sondern unter der Gnade seid.
\bibleverse{15} Wie nun, sollen wir sündigen, weil wir nicht unter dem
Gesetz, sondern unter der Gnade sind? Das sei ferne! \bibleverse{16}
Wisset ihr nicht: wem ihr euch als Knechte hingebet, ihm zu gehorchen,
dessen Knechte seid ihr und müßt ihm gehorchen, es sei der Sünde zum
Tode, oder dem Gehorsam zur Gerechtigkeit? \bibleverse{17} Gott aber sei
Dank, daß ihr Knechte der Sünde gewesen, nun aber von Herzen gehorsam
geworden seid dem Vorbild der Lehre, dem ihr euch übergeben habt.
\bibleverse{18} Nachdem ihr aber von der Sünde befreit wurdet, seid ihr
der Gerechtigkeit dienstbar geworden. \bibleverse{19} Ich muß menschlich
davon reden wegen der Schwachheit eures Fleisches. Gleichwie ihr eure
Glieder in den Dienst der Unreinigkeit und der Gesetzwidrigkeit gestellt
habt, um gesetzwidrig zu handeln, so stellet nun eure Glieder in den
Dienst der Gerechtigkeit zur Heiligung. \bibleverse{20} Denn als ihr
Knechte der Sünde waret, da waret ihr frei gegenüber der Gerechtigkeit.
\bibleverse{21} Was hattet ihr nun damals für Frucht? Solche, deren ihr
euch jetzt schämet; denn das Ende derselben ist der Tod. \bibleverse{22}
Nun aber, da ihr von der Sünde frei und Gott dienstbar geworden seid,
habt ihr als eure Frucht die Heiligung, als Ende aber das ewige Leben.
\bibleverse{23} Denn der Tod ist der Sünde Sold; aber die Gnadengabe
Gottes ist das ewige Leben in Christus Jesus, unsrem Herrn.

\hypertarget{section-6}{%
\section{7}\label{section-6}}

\bibleverse{1} Oder wisset ihr nicht, Brüder (denn ich rede ja mit
Gesetzeskundigen), daß das Gesetz nur so lange über den Menschen
herrscht, als er lebt? \bibleverse{2} Denn die verheiratete Frau ist
durchs Gesetz an ihren Mann gebunden, solange er lebt; wenn aber der
Mann stirbt, so ist sie von dem Gesetz des Mannes befreit.
\bibleverse{3} So wird sie nun bei Lebzeiten des Mannes eine
Ehebrecherin genannt, wenn sie einem andern Manne zu eigen wird; stirbt
aber der Mann, so ist sie vom Gesetze frei, so daß sie keine
Ehebrecherin ist, wenn sie einem andern Manne zu eigen wird.
\bibleverse{4} Also seid auch ihr, meine Brüder, dem Gesetze getötet
worden durch den Leib Christi, auf daß ihr einem andern angehöret,
nämlich dem, der von den Toten auferstanden ist, damit wir Gott Frucht
bringen. \bibleverse{5} Denn als wir im Fleische waren, da wirkten die
sündlichen Leidenschaften, durch das Gesetz erregt, in unsren Gliedern,
um dem Tode Frucht zu bringen. \bibleverse{6} Nun aber sind wir vom
Gesetz frei geworden, da wir dem gestorben sind, worin wir festgehalten
wurden, so daß wir dienen im neuen Wesen des Geistes und nicht im alten
Wesen des Buchstabens. \bibleverse{7} Was wollen wir nun sagen? Ist das
Gesetz Sünde? Das sei ferne! Aber die Sünde hätte ich nicht erkannt,
außer durch das Gesetz; denn von der Lust hätte ich nichts gewußt, wenn
das Gesetz nicht gesagt hätte: Laß dich nicht gelüsten! \bibleverse{8}
Da nahm aber die Sünde einen Anlaß und bewirkte durch das Verbot in mir
allerlei Gelüste; denn ohne das Gesetz ist die Sünde tot. \bibleverse{9}
Ich aber lebte, als ich noch ohne Gesetz war; als aber das Gesetz kam,
lebte die Sünde auf; \bibleverse{10} ich aber starb, und das zum Leben
gegebene Gesetz erwies sich mir todbringend. \bibleverse{11} Denn die
Sünde nahm einen Anlaß und verführte mich durch das Gebot und tötete
mich durch dasselbe. \bibleverse{12} So ist nun das Gesetz heilig, und
das Gebot ist heilig, gerecht und gut! \bibleverse{13} Gereichte nun das
Gute mir zum Tode? Das sei ferne! Sondern die Sünde, damit sie als Sünde
erscheine, hat mir durch das Gute den Tod bewirkt, auf daß die Sünde
überaus sündig würde durch das Gebot. \bibleverse{14} Denn wir wissen,
daß das Gesetz geistlich ist; ich aber bin fleischlich, unter die Sünde
verkauft. \bibleverse{15} Denn was ich vollbringe, billige ich nicht;
denn ich tue nicht, was ich will, sondern was ich hasse, das übe ich
aus. \bibleverse{16} Wenn ich aber das tue, was ich nicht will, so
stimme ich dem Gesetz bei, daß es trefflich ist. \bibleverse{17} Nun
aber vollbringe nicht mehr ich dasselbe, sondern die Sünde, die in mir
wohnt. \bibleverse{18} Denn ich weiß, daß in mir, das ist in meinem
Fleische, nichts Gutes wohnt; das Wollen ist zwar bei mir vorhanden,
aber das Vollbringen des Guten gelingt mir nicht! \bibleverse{19} Denn
nicht das Gute, das ich will, tue ich, sondern das Böse, das ich nicht
will, übe ich aus. \bibleverse{20} Wenn ich aber das tue, was ich nicht
will, so vollbringe nicht mehr ich dasselbe, sondern die Sünde, die in
mir wohnt. \bibleverse{21} Ich finde also das Gesetz vor, wonach mir,
der ich das Gute tun will, das Böse anhängt. \bibleverse{22} Denn ich
habe Lust an dem Gesetz Gottes nach dem inwendigen Menschen;
\bibleverse{23} ich sehe aber ein anderes Gesetz in meinen Gliedern, das
dem Gesetz meiner Vernunft widerstreitet und mich gefangen nimmt in dem
Gesetz der Sünde, das in meinen Gliedern ist. \bibleverse{24} Ich
elender Mensch! Wer wird mich erlösen von diesem Todesleib?
\bibleverse{25} Ich danke Gott durch Jesus Christus, unsren Herrn! So
diene nun ich selbst mit der Vernunft dem Gesetz Gottes, mit dem
Fleische aber dem Gesetz der Sünde.

\hypertarget{section-7}{%
\section{8}\label{section-7}}

\bibleverse{1} So gibt es nun keine Verdammnis mehr für die, welche in
Christus Jesus sind. \bibleverse{2} Denn das Gesetz des Geistes des
Lebens in Christus Jesus hat mich frei gemacht von dem Gesetz der Sünde
und des Todes. \bibleverse{3} Denn was dem Gesetz unmöglich war (weil es
durch das Fleisch geschwächt wurde), das hat Gott getan, nämlich die
Sünde im Fleische verdammt, indem er seinen Sohn sandte in der
Ähnlichkeit des sündlichen Fleisches und um der Sünde willen,
\bibleverse{4} damit die vom Gesetz geforderte Gerechtigkeit in uns
erfüllt würde, die wir nicht nach dem Fleische wandeln, sondern nach dem
Geist. \bibleverse{5} Denn die nach dem Fleische leben, sinnen auf das,
was des Fleisches ist, die aber nach dem Geiste leben, auf das, was des
Geistes ist. \bibleverse{6} Denn die Gesinnung des Fleisches ist Tod,
die Gesinnung des Geistes aber Leben und Friede, \bibleverse{7} darum,
weil die Gesinnung des Fleisches Feindschaft wider Gott ist; denn sie
ist dem Gesetz Gottes nicht untertan, sie kann es auch nicht.
\bibleverse{8} Die aber im Fleische sind, vermögen Gott nicht zu
gefallen. \bibleverse{9} Ihr aber seid nicht im Fleische, sondern im
Geiste, wenn anders Gottes Geist in euch wohnt; wer aber Christi Geist
nicht hat, der ist nicht sein. \bibleverse{10} Wenn aber Christus in
euch ist, so ist der Leib zwar tot um der Sünde willen, der Geist aber
ist Leben um der Gerechtigkeit willen. \bibleverse{11} Wenn aber der
Geist dessen, der Jesus von den Toten auferweckt hat, in euch wohnt, so
wird derselbe, der Christus von den Toten auferweckt hat, auch eure
sterblichen Leiber lebendig machen durch seinen Geist, der in euch
wohnt. \bibleverse{12} So sind wir also, ihr Brüder, dem Fleische nicht
schuldig, nach dem Fleische zu leben! \bibleverse{13} Denn wenn ihr nach
dem Fleische lebet, so müßt ihr sterben; wenn ihr aber durch den Geist
die Geschäfte des Leibes tötet, so werdet ihr leben. \bibleverse{14}
Denn alle, die sich vom Geiste Gottes leiten lassen, sind Gottes Kinder.
\bibleverse{15} Denn ihr habt nicht einen Geist der Knechtschaft
empfangen, daß ihr euch abermal fürchten müßtet, sondern ihr habt einen
Geist der Kindschaft empfangen, in welchem wir rufen: Abba, Vater!
\bibleverse{16} Dieser Geist gibt Zeugnis unsrem Geist, daß wir Gottes
Kinder sind. \bibleverse{17} Sind wir aber Kinder, so sind wir auch
Erben, nämlich Gottes Erben und Miterben Christi; wenn anders wir mit
ihm leiden, auf daß wir auch mit ihm verherrlicht werden.
\bibleverse{18} Denn ich halte dafür, daß die Leiden der jetzigen Zeit
nicht in Betracht kommen gegenüber der Herrlichkeit, die an uns
geoffenbart werden soll. \bibleverse{19} Denn die gespannte Erwartung
der Kreatur sehnt die Offenbarung der Kinder Gottes herbei.
\bibleverse{20} Die Kreatur ist nämlich der Vergänglichkeit unterworfen,
nicht freiwillig, sondern durch den, der sie unterworfen hat, auf
Hoffnung hin, \bibleverse{21} daß auch sie selbst, die Kreatur, befreit
werden soll von der Knechtschaft der Sterblichkeit zur Freiheit der
Herrlichkeit der Kinder Gottes. \bibleverse{22} Denn wir wissen, daß die
ganze Schöpfung mitseufzt und mit in Wehen liegt bis jetzt;
\bibleverse{23} und nicht nur sie, sondern auch wir selbst, die wir die
Erstlingsgabe des Geistes haben, auch wir erwarten seufzend die
Sohnesstellung, die Erlösung unsres Leibes. \bibleverse{24} Denn auf
Hoffnung hin sind wir errettet worden. Eine Hoffnung aber, die man
sieht, ist keine Hoffnung; denn was einer sieht, das hofft er doch nicht
mehr! \bibleverse{25} Wenn wir aber auf das hoffen, was wir nicht sehen,
so warten wir es ab in Geduld. \bibleverse{26} Ebenso kommt aber auch
der Geist unserer Schwachheit zu Hilfe. Denn wir wissen nicht, was wir
beten sollen, wie sich\textquotesingle s gebührt; aber der Geist selbst
tritt für uns ein mit unausgesprochenen Seufzern. \bibleverse{27} Der
aber die Herzen erforscht, weiß, was des Geistes Sinn ist; denn er
vertritt die Heiligen so, wie es Gott angemessen ist. \bibleverse{28}
Wir wissen aber, daß denen, die Gott lieben, alles zum Besten mitwirkt,
denen, die nach dem Vorsatz berufen sind. \bibleverse{29} Denn welche er
zuvor ersehen hat, die hat er auch vorherbestimmt, dem Ebenbilde seines
Sohnes gleichgestaltet zu werden, damit er der Erstgeborene sei unter
vielen Brüdern. \bibleverse{30} Welche er aber vorherbestimmt hat, die
hat er auch berufen, welche er aber berufen hat, die hat er auch
gerechtfertigt, welche er aber gerechtfertigt hat, die hat er auch
verherrlicht. \bibleverse{31} Was wollen wir nun hierzu sagen? Ist Gott
für uns, wer mag wider uns sein? \bibleverse{32} Welcher sogar seines
eigenen Sohnes nicht verschont, sondern ihn für uns alle dahingegeben
hat, wie sollte er uns mit ihm nicht auch alles schenken?
\bibleverse{33} Wer will gegen die Auserwählten Gottes Anklage erheben?
Gott, der sie rechtfertigt? \bibleverse{34} Wer will verdammen?
Christus, der gestorben ist, ja vielmehr, der auch auferweckt ist, der
auch zur Rechten Gottes ist, der uns auch vertritt? \bibleverse{35} Wer
will uns scheiden von der Liebe Christi? Trübsal oder Angst oder
Verfolgung oder Hunger oder Blöße oder Gefahr oder Schwert?
\bibleverse{36} Wie geschrieben steht: ``Um deinetwillen werden wir
getötet den ganzen Tag, wir sind geachtet wie Schlachtschafe!''
\bibleverse{37} Aber in dem allen überwinden wir weit durch den, der uns
geliebt hat! \bibleverse{38} Denn ich bin überzeugt, daß weder Tod noch
Leben, weder Engel noch Fürstentümer noch Gewalten, weder Gegenwärtiges
noch Zukünftiges, \bibleverse{39} weder Hohes noch Tiefes, noch irgend
ein anderes Geschöpf uns zu scheiden vermag von der Liebe Gottes, die in
Christus Jesus ist, unsrem Herrn!

\hypertarget{section-8}{%
\section{9}\label{section-8}}

\bibleverse{1} Ich sage die Wahrheit in Christus, ich lüge nicht, wie
mir mein Gewissen bezeugt im heiligen Geist, \bibleverse{2} daß ich
große Traurigkeit und unablässigen Schmerz in meinem Herzen habe.
\bibleverse{3} Ich wünschte nämlich, selber von Christus verbannt zu
sein für meine Brüder, meine Verwandten nach dem Fleisch, \bibleverse{4}
welche Israeliten sind, denen die Kindschaft und die Herrlichkeit und
die Bündnisse und die Gesetzgebung und der Gottesdienst und die
Verheißungen gehören; \bibleverse{5} ihnen gehören auch die Väter an,
und von ihnen stammt dem Fleische nach Christus, der da ist über alle,
hochgelobter Gott, in Ewigkeit. Amen! \bibleverse{6} Nicht aber, als ob
das Wort Gottes nun hinfällig wäre! Denn nicht alle, die von Israel
abstammen, sind Israel; \bibleverse{7} auch sind nicht alle, weil sie
Abrahams Same sind, seine Kinder, sondern ``in Isaak soll dir ein Same
berufen werden''; \bibleverse{8} das heißt: Nicht die Kinder des
Fleisches sind Kinder Gottes, sondern die Kinder der Verheißung werden
als Same gerechnet. \bibleverse{9} Denn das ist ein Wort der Verheißung:
``Um diese Zeit will ich kommen, und Sara soll einen Sohn haben.''
\bibleverse{10} Und nicht dieses allein, sondern auch, als Rebekka von
ein und demselben, von unserm Vater Isaak schwanger war, \bibleverse{11}
ehe die Kinder geboren waren und weder Gutes noch Böses getan hatten
(auf daß der nach der Erwählung gefaßte Vorsatz Gottes bestehe, nicht um
der Werke, sondern um des Berufers willen), \bibleverse{12} wurde zu ihr
gesagt: ``Der Größere wird dem Kleineren dienen''; \bibleverse{13} wie
auch geschrieben steht: ``Jakob habe ich geliebt, aber Esau habe ich
gehaßt.'' \bibleverse{14} Was wollen wir nun sagen! Ist etwa bei Gott
Ungerechtigkeit? Das sei ferne! \bibleverse{15} Denn zu Mose spricht er:
``Welchem ich gnädig bin, dem bin ich gnädig, und wessen ich mich
erbarme, dessen erbarme ich mich.'' \bibleverse{16} So liegt es nun
nicht an jemandes Wollen oder Laufen, sondern an Gottes Erbarmen.
\bibleverse{17} Denn die Schrift sagt zum Pharao: ``Eben dazu habe ich
dich erweckt, daß ich an dir meine Macht erweise und daß mein Name
verkündigt werde auf der ganzen Erde.'' \bibleverse{18} So erbarmt er
sich nun, wessen er will, und verstockt, wen er will. \bibleverse{19}
Nun wirst du mich fragen: Warum tadelt er dann noch? Wer kann seinem
Willen widerstehen? \bibleverse{20} Nun ja, lieber Mensch, wer bist denn
du, daß du mit Gott rechten willst? Spricht auch das Gebilde zu seinem
Bildner: Warum hast du mich so gemacht? \bibleverse{21} Hat nicht der
Töpfer Macht über den Ton, aus derselben Masse das eine Gefäß zur Ehre,
das andere zur Unehre zu machen? \bibleverse{22} Wenn aber Gott, da er
seinen Zorn erzeigen und seine Macht kundtun wollte, mit großer Geduld
die Gefäße des Zorns getragen hat, die zum Verderben zugerichtet sind,
\bibleverse{23} damit er auch den Reichtum seiner Herrlichkeit an den
Gefäßen der Barmherzigkeit kundtäte, die er zuvor zur Herrlichkeit
bereitet hat, \bibleverse{24} wie er denn als solche auch uns berufen
hat, nicht allein aus den Juden, sondern auch aus den Heiden, was dann?
\bibleverse{25} Wie er auch durch Hosea spricht: ``Ich will das mein
Volk nennen, was nicht mein Volk war, und Geliebte, die nicht die
Geliebte war, \bibleverse{26} und es soll geschehen an dem Ort, wo zu
ihnen gesagt wurde: Ihr seid nicht mein Volk, da sollen sie Kinder des
lebendigen Gottes genannt werden.'' \bibleverse{27} Jesaja aber ruft
über Israel aus: ``Wenn die Zahl der Kinder Israel wäre wie der Sand am
Meer, so wird doch nur der Überrest gerettet werden; \bibleverse{28}
denn eine abschließende und beschleunigte Abrechnung in Gerechtigkeit
wird der Herr auf Erden veranstalten, ja eine summarische Abrechnung!''
\bibleverse{29} Und, wie Jesaja vorhergesagt hat: ``Hätte der Herr der
Heerscharen uns nicht eine Nachkommenschaft übrigbleiben lassen, so
wären wir wie Sodom geworden und gleich wie Gomorra!'' \bibleverse{30}
Was wollen wir nun sagen? Daß Heiden, welche nicht nach Gerechtigkeit
strebten, Gerechtigkeit erlangt haben, nämlich Gerechtigkeit, die aus
dem Glauben kommt, \bibleverse{31} daß aber Israel, welches dem Gesetz
der Gerechtigkeit nachjagte, dem Gesetz nicht nachgekommen ist.
\bibleverse{32} Warum? Weil es nicht aus Glauben geschah, sondern aus
Werken. Sie haben sich gestoßen an dem Stein des Anstoßes,
\bibleverse{33} wie geschrieben steht: ``Siehe, ich lege in Zion einen
Stein des Anstoßes und einen Fels des Ärgernisses; und wer an ihn
glaubt, wird nicht zuschanden werden!''

\hypertarget{section-9}{%
\section{10}\label{section-9}}

\bibleverse{1} Brüder, meines Herzens Wunsch und mein Flehen zu Gott für
Israel ist auf ihr Heil gerichtet. \bibleverse{2} Denn ich gebe ihnen
das Zeugnis, daß sie eifern um Gott, aber mit Unverstand. \bibleverse{3}
Denn weil sie die Gerechtigkeit Gottes nicht erkennen und ihre eigene
Gerechtigkeit aufzurichten trachten, sind sie der Gerechtigkeit Gottes
nicht untertan. \bibleverse{4} Denn Christus ist des Gesetzes Ende zur
Gerechtigkeit für einen jeden, der da glaubt. \bibleverse{5} Mose
beschreibt nämlich die Gerechtigkeit, die durch das Gesetz kommt, also:
``Der Mensch, welcher sie tut, wird dadurch leben.'' \bibleverse{6} Aber
die Gerechtigkeit durch den Glauben redet so: ``Sprich nicht in deinem
Herzen: Wer will in den Himmel hinaufsteigen?'' (nämlich um Christus
herabzuholen) \bibleverse{7} oder: ``wer will in den Abgrund
hinuntersteigen?'' nämlich um Christus von den Toten zu holen!
\bibleverse{8} Sondern was sagt sie? ``Das Wort ist dir nahe, in deinem
Munde und in deinem Herzen!'' nämlich das Wort des Glaubens, das wir
predigen. \bibleverse{9} Denn wenn du mit deinem Munde Jesus als den
Herrn bekennst und in deinem Herzen glaubst, daß Gott ihn von den Toten
auferweckt hat, so wirst du gerettet; \bibleverse{10} denn mit dem
Herzen glaubt man, um gerecht, und mit dem Munde bekennt man, um
gerettet zu werden; \bibleverse{11} denn die Schrift spricht: ``Wer an
ihn glaubt, wird nicht zuschanden werden!'' \bibleverse{12} Denn es ist
kein Unterschied zwischen Juden und Griechen: alle haben denselben
Herrn, der reich ist für alle, die ihn anrufen; \bibleverse{13} denn
``wer den Namen des Herrn anrufen wird, der soll gerettet werden''.
\bibleverse{14} Wie sollen sie ihn aber anrufen, wenn sie nicht an ihn
glauben? Wie sollen sie aber glauben, wenn sie nichts von ihm gehört
haben? Wie sollen sie aber hören ohne Prediger? \bibleverse{15} Wie
sollen sie aber predigen, wenn sie nicht ausgesandt werden? Wie
geschrieben steht: ``Wie lieblich sind die Füße derer, die das
Evangelium des Friedens, die das Evangelium des Guten verkündigen!''
\bibleverse{16} Aber nicht alle haben dem Evangelium gehorcht; denn
Jesaja spricht: ``Herr, wer hat unsrer Predigt geglaubt?''
\bibleverse{17} Demnach kommt der Glaube aus der Predigt, die Predigt
aber durch Gottes Wort. \bibleverse{18} Aber ich frage: Haben sie etwa
nicht gehört? Doch ja, ``es ist in alle Lande ausgegangen ihr Schall und
bis an die Enden der Erde ihre Worte''. \bibleverse{19} Aber ich frage:
Hat es Israel nicht gewußt? Schon Mose sagt: ``Ich will euch zur
Eifersucht reizen durch das, was kein Volk ist, durch ein unverständiges
Volk will ich euch erzürnen.'' \bibleverse{20} Jesaja aber wagt sogar zu
sagen: ``Ich bin von denen gefunden worden, welche mich nicht suchten,
bin denen offenbar geworden, die nicht nach mir fragten.''
\bibleverse{21} In bezug auf Israel aber spricht er: ``Den ganzen Tag
habe ich meine Hände ausgestreckt nach einem ungehorsamen und
widerspenstigen Volk!''

\hypertarget{section-10}{%
\section{11}\label{section-10}}

\bibleverse{1} Ich frage nun: Hat etwa Gott sein Volk verstoßen? Das sei
ferne! Denn auch ich bin ein Israelit, aus dem Samen Abrahams, aus dem
Stamme Benjamin. \bibleverse{2} Gott hat sein Volk nicht verstoßen,
welches er zuvor ersehen hat! Oder wisset ihr nicht, was die Schrift bei
der Geschichte von Elia spricht, wie er sich an Gott gegen Israel
wendet: \bibleverse{3} ``Herr, sie haben deine Propheten getötet und
deine Altäre zerstört, und ich bin allein übriggeblieben, und sie
trachten mir nach dem Leben!'' \bibleverse{4} Aber was sagt ihm die
göttliche Antwort? ``Ich habe mir siebentausend Mann übrigbleiben
lassen, die kein Knie gebeugt haben vor Baal.'' \bibleverse{5} So ist
auch in der jetzigen Zeit ein Rest vorhanden, dank der Gnadenwahl.
\bibleverse{6} Wenn aber aus Gnade, so ist es nicht mehr um der Werke
willen, sonst würde die Gnade nicht mehr Gnade sein; wenn aber um der
Werke willen, so ist es nicht mehr aus Gnade, sonst wäre das Werk nicht
mehr Werk. \bibleverse{7} Wie nun? Was Israel sucht, das hat es nicht
erlangt; die Auswahl aber hat es erlangt, die übrigen aber wurden
verstockt, \bibleverse{8} wie geschrieben steht: ``Gott hat ihnen einen
Geist der Schlafsucht gegeben, Augen, um nicht zu sehen, und Ohren, um
nicht zu hören, bis zum heutigen Tag.'' \bibleverse{9} Und David
spricht: ``Ihr Tisch werde ihnen zur Schlinge und zum Fallstrick und zum
Anstoß und zur Vergeltung; \bibleverse{10} ihre Augen sollen verfinstert
werden, daß sie nicht sehen, und ihren Rücken beuge allezeit!''
\bibleverse{11} Ich frage nun: Sind sie denn darum gestrauchelt, damit
sie fallen sollten? Das sei ferne! Sondern durch ihren Fall wurde das
Heil den Heiden zuteil, damit sie diesen nacheifern möchten.
\bibleverse{12} Wenn aber ihr Fall der Reichtum der Welt und ihr Verlust
der Reichtum der Heiden geworden ist, wieviel mehr ihre volle Zahl!
\bibleverse{13} Zu euch, den Heiden, rede ich (da ich nun eben
Heidenapostel bin, rühme ich mein Amt, \bibleverse{14} ob ich nicht etwa
meine Volksgenossen zum Nacheifern reizen und etliche von ihnen erretten
könnte); \bibleverse{15} darum sage ich: Wenn ihre Verwerfung die
Versöhnung der Welt geworden ist, was würde ihre Annahme anderes sein,
als Leben aus den Toten? \bibleverse{16} Ist aber der Anbruch heilig, so
ist es auch der Teig, und ist die Wurzel heilig, so sind es auch die
Zweige. \bibleverse{17} Wenn aber etliche der Zweige ausgebrochen wurden
und du als ein wilder Ölzweig unter sie eingepfropft und der Wurzel und
der Fettigkeit des Ölbaums teilhaftig geworden bist, \bibleverse{18} so
rühme dich nicht wider die Zweige! Rühmst du dich aber, so wisse, daß
nicht du die Wurzel trägst, sondern die Wurzel trägt dich!
\bibleverse{19} Nun sagst du aber: Die Zweige sind ausgebrochen worden,
damit ich eingepfropft würde! \bibleverse{20} Gut! Um ihres Unglaubens
willen sind sie ausgebrochen worden; du aber stehst durch den Glauben.
Sei nicht stolz, sondern fürchte dich! \bibleverse{21} Denn wenn Gott
die natürlichen Zweige nicht verschont hat, so wird er wohl auch dich
nicht verschonen. \bibleverse{22} So schaue nun die Güte und die Strenge
Gottes; die Strenge an denen, die gefallen sind; die Güte aber an dir,
sofern du in der Güte bleibst, sonst wirst auch du abgehauen werden!
\bibleverse{23} Jene dagegen, wenn sie nicht im Unglauben verharren,
sollen wieder eingepfropft werden; denn Gott vermag sie wohl wieder
einzupfropfen. \bibleverse{24} Denn wenn du aus dem von Natur wilden
Ölbaum herausgeschnitten und wider die Natur in den edlen Ölbaum
eingepfropft worden bist, wieviel eher können diese, die natürlichen
Zweige, wieder in ihren eigenen Ölbaum eingepfropft werden!
\bibleverse{25} Denn ich will nicht, meine Brüder, daß euch dieses
Geheimnis unbekannt bleibe, damit ihr euch nicht selbst klug dünket, daß
Israel zum Teil Verstockung widerfahren ist, bis daß die Vollzahl der
Heiden eingegangen sein wird \bibleverse{26} und also ganz Israel
gerettet werde, wie geschrieben steht: ``Aus Zion wird der Erlöser
kommen und die Gottlosigkeiten von Jakob abwenden'', \bibleverse{27}
und: ``das ist mein Bund mit ihnen, wenn ich ihre Sünden wegnehmen
werde''. \bibleverse{28} Nach dem Evangelium zwar sind sie Feinde um
euretwillen, nach der Erwählung aber Geliebte um der Väter willen.
\bibleverse{29} Denn Gottes Gnadengaben und Berufung sind
unwiderruflich. \bibleverse{30} Denn gleichwie auch ihr einst Gott nicht
gehorcht habt, nun aber begnadigt worden seid infolge ihres Ungehorsams,
\bibleverse{31} so haben auch sie jetzt nicht gehorcht infolge eurer
Begnadigung, damit auch sie begnadigt würden. \bibleverse{32} Denn Gott
hat alle miteinander in den Unglauben verschlossen, damit er sich aller
erbarme. \bibleverse{33} O welch eine Tiefe des Reichtums, der Weisheit
und der Erkenntnis Gottes! Wie unergründlich sind seine Gerichte und
unausforschlich seine Wege! \bibleverse{34} Denn ``wer hat des Herrn
Sinn erkannt, oder wer ist sein Ratgeber gewesen? \bibleverse{35} Oder
wer hat ihm etwas zuvor gegeben, daß es ihm wiedervergolten werde?''
\bibleverse{36} Denn von ihm und durch ihn und zu ihm sind alle Dinge;
ihm sei Ehre von Ewigkeit zu Ewigkeit! Amen.

\hypertarget{section-11}{%
\section{12}\label{section-11}}

\bibleverse{1} Ich ermahne euch nun, ihr Brüder, kraft der
Barmherzigkeit Gottes, daß ihr eure Leiber darbringet als ein
lebendiges, heiliges, Gott wohlgefälliges Opfer: das sei euer
vernünftiger Gottesdienst! \bibleverse{2} Und passet euch nicht diesem
Weltlauf an, sondern verändert euer Wesen durch die Erneuerung eures
Sinnes, um prüfen zu können, was der Wille Gottes sei, der gute und
wohlgefällige und vollkommene. \bibleverse{3} Denn ich sage kraft der
Gnade, die mir gegeben ist, einem jeden unter euch, daß er nicht höher
von sich denke, als sich zu denken gebührt, sondern daß er auf
Bescheidenheit bedacht sei, wie Gott einem jeden das Maß des Glaubens
zugeteilt hat. \bibleverse{4} Denn gleichwie wir an einem Leibe viele
Glieder besitzen, nicht alle Glieder aber dieselbe Verrichtung haben,
\bibleverse{5} so sind auch wir, die vielen, ein Leib in Christus, als
einzelne aber untereinander Glieder. \bibleverse{6} Wenn wir aber auch
verschiedene Gaben haben nach der uns verliehenen Gnade, zum Beispiel
Weissagung, so stimmen sie doch mit dem Glauben überein! \bibleverse{7}
Wenn einer dient, sei es so in dem Dienst; wenn einer lehrt, in der
Lehre; \bibleverse{8} wenn einer ermahnt, in der Ermahnung. Wer gibt,
gebe in Einfalt; wer vorsteht, tue es mit Fleiß; wer Barmherzigkeit übt,
mit Freudigkeit! \bibleverse{9} Die Liebe sei ungeheuchelt! Hasset das
Böse, hanget dem Guten an! \bibleverse{10} In der Bruderliebe seid
gegeneinander herzlich, in der Ehrerbietung komme einer dem andern
zuvor! \bibleverse{11} Im Fleiß lasset nicht nach, seid brennend im
Geist, dienet dem Herrn! \bibleverse{12} Seid fröhlich in Hoffnung, in
Trübsal haltet stand, seid beharrlich im Gebet! \bibleverse{13} Nehmet
Anteil an den Nöten der Heiligen, befleißiget euch der Gastfreundschaft!
\bibleverse{14} Segnet die euch verfolgen, segnet und fluchet nicht!
\bibleverse{15} Freuet euch mit den Fröhlichen und weinet mit den
Weinenden! \bibleverse{16} Seid gleichgesinnt gegeneinander; trachtet
nicht nach hohen Dingen, sondern haltet euch herunter zu den Niedrigen;
haltet euch nicht selbst für klug! \bibleverse{17} Vergeltet niemandem
Böses mit Bösem! Befleißiget euch dessen, was in aller Menschen Augen
edel ist! \bibleverse{18} Ist es möglich, soviel an euch liegt, so habt
mit allen Menschen Frieden. \bibleverse{19} Rächet euch nicht selbst,
ihr Lieben, sondern gebet Raum dem Zorne Gottes; denn es steht
geschrieben: ``Die Rache ist mein, ich will vergelten, spricht der
Herr.'' \bibleverse{20} Wenn nun deinen Feind hungert, so speise ihn;
dürstet ihn, so tränke ihn! Wenn du das tust, wirst du feurige Kohlen
auf sein Haupt sammeln. \bibleverse{21} Laß dich nicht vom Bösen
überwinden, sondern überwinde das Böse mit Gutem!

\hypertarget{section-12}{%
\section{13}\label{section-12}}

\bibleverse{1} Jedermann sei den obrigkeitlichen Gewalten untertan; denn
es gibt keine Obrigkeit, die nicht von Gott wäre; die vorhandenen aber
sind von Gott verordnet. \bibleverse{2} Wer sich also der Obrigkeit
widersetzt, der widerstrebt der Ordnung Gottes; die aber widerstreben,
ziehen sich selbst die Verurteilung zu. \bibleverse{3} Denn die
Herrscher sind nicht wegen guten Werken zu fürchten, sondern wegen
bösen! Willst du also die Obrigkeit nicht fürchten, so tue das Gute,
dann wirst du Lob von ihr empfangen! \bibleverse{4} Denn sie ist Gottes
Dienerin, zu deinem Besten. Tust du aber Böses, so fürchte dich! Denn
sie trägt das Schwert nicht umsonst; Gottes Dienerin ist sie, eine
Rächerin zur Strafe an dem, der das Böse tut. \bibleverse{5} Darum ist
es notwendig, untertan zu sein, nicht allein um der Strafe, sondern auch
um des Gewissens willen. \bibleverse{6} Deshalb zahlet ihr ja auch
Steuern; denn sie sind Gottes Diener, die eben dazu bestellt sind.
\bibleverse{7} So gebet nun jedermann, was ihr schuldig seid: Steuer,
dem die Steuer, Zoll, dem der Zoll, Furcht, dem die Furcht, Ehre, dem
die Ehre gebührt. \bibleverse{8} Seid niemand etwas schuldig, als daß
ihr einander liebet; denn wer den andern liebt, hat das Gesetz erfüllt.
\bibleverse{9} Denn die Forderung: ``Du sollst nicht ehebrechen, du
sollst nicht töten, du sollst nicht stehlen, laß dich nicht gelüsten''
(und welches andere Gebot noch sei), wird zusammengefaßt in diesem Wort:
``Du sollst deinen Nächsten lieben wie dich selbst!'' \bibleverse{10}
Die Liebe tut dem Nächsten nichts Böses; so ist nun die Liebe des
Gesetzes Erfüllung. \bibleverse{11} Und dieses sollen wir tun als
solche, die die Zeit verstehen, daß nämlich die Stunde schon da ist, wo
wir vom Schlafe aufwachen sollten; denn jetzt ist unser Heil näher, als
da wir gläubig wurden; \bibleverse{12} die Nacht ist vorgerückt, der Tag
aber nahe. So lasset uns nun ablegen die Werke der Finsternis und
anziehen die Waffen des Lichts; \bibleverse{13} laßt uns anständig
wandeln als am Tage, nicht in Schmausereien und Schlemmereien, nicht in
Unzucht und Ausschweifungen, nicht in Hader und Neid; \bibleverse{14}
sondern ziehet den Herrn Jesus Christus an und pfleget das Fleisch nicht
bis zur Erregung von Begierden!

\hypertarget{section-13}{%
\section{14}\label{section-13}}

\bibleverse{1} Des Schwachen im Glauben nehmet euch an, doch nicht um
über Meinungen zu streiten. \bibleverse{2} Einer glaubt, alles essen zu
dürfen; wer aber schwach ist, ißt Gemüse. \bibleverse{3} Wer ißt,
verachte den nicht, der nicht ißt; und wer nicht ißt, richte den nicht,
der ißt; denn Gott hat ihn angenommen. \bibleverse{4} Wer bist du, daß
du einen fremden Knecht richtest? Er steht oder fällt seinem Herrn. Er
wird aber aufgerichtet werden; denn der Herr vermag ihn aufzurichten.
\bibleverse{5} Dieser achtet einen Tag höher als den andern, jener hält
alle Tage gleich; ein jeglicher sei seiner Meinung gewiß! \bibleverse{6}
Wer auf den Tag schaut, schaut darauf für den Herrn, und wer nicht auf
den Tag schaut, schaut nicht darauf für den Herrn. Wer ißt, der ißt für
den Herrn; denn er dankt Gott, und wer nicht ißt, der ißt nicht für den
Herrn und dankt Gott. \bibleverse{7} Denn keiner von uns lebt sich
selbst und keiner stirbt sich selbst. \bibleverse{8} Leben wir, so leben
wir dem Herrn, sterben wir, so sterben wir dem Herrn; ob wir nun leben
oder sterben, so sind wir des Herrn. \bibleverse{9} Denn dazu ist
Christus gestorben und wieder lebendig geworden, daß er sowohl über Tote
als auch über Lebende Herr sei. \bibleverse{10} Du aber, was richtest du
deinen Bruder? Oder du, was verachtest du deinen Bruder? Wir werden alle
vor dem Richterstuhl Christi erscheinen; \bibleverse{11} denn es steht
geschrieben: ``So wahr ich lebe, spricht der Herr, mir soll sich beugen
jedes Knie, und jede Zunge wird Gott bekennen.'' \bibleverse{12} So wird
also ein jeglicher für sich selbst Gott Rechenschaft geben.
\bibleverse{13} Darum laßt uns nicht mehr einander richten, sondern das
richtet vielmehr, daß dem Bruder weder Anstoß noch Ärgernis gegeben
werde! \bibleverse{14} Ich weiß und bin in dem Herrn Jesus davon
überzeugt, daß nichts an sich selbst unrein ist; sondern nur für den,
der etwas für unrein hält, ist es unrein. \bibleverse{15} Wenn aber dein
Bruder um einer Speise willen betrübt wird, so wandelst du schon nicht
nach der Liebe. Verdirb mit deiner Speise nicht den, für welchen
Christus gestorben ist! \bibleverse{16} So soll nun euer Bestes nicht
verlästert werden! \bibleverse{17} Denn das Reich Gottes ist nicht Essen
und Trinken, sondern Gerechtigkeit, Friede und Freude im heiligen Geist;
\bibleverse{18} wer darin Christus dient, der ist Gott wohlgefällig und
auch von den Menschen gebilligt. \bibleverse{19} So laßt uns nun dem
nachjagen, was zum Frieden und zur Erbauung untereinander dient.
\bibleverse{20} Zerstöre nicht wegen einer Speise Gottes Werk! Es ist
zwar alles rein, aber es ist demjenigen schädlich, welcher es mit Anstoß
ißt. \bibleverse{21} Es ist gut, wenn du kein Fleisch issest und keinen
Wein trinkst, noch sonst etwas tust, woran dein Bruder Anstoß oder
Ärgernis nehmen oder schwach werden könnte. \bibleverse{22} Du hast
Glauben? Habe ihn für dich selbst vor Gott! Selig, wer sich selbst nicht
beschuldigt in dem, was er billigt; \bibleverse{23} wer aber zweifelt
und doch ißt, der ist verurteilt, weil es nicht aus Glauben geschieht.
Alles aber, was nicht aus Glauben geschieht, ist Sünde.

\hypertarget{section-14}{%
\section{15}\label{section-14}}

\bibleverse{1} Es ist aber unsere, der Starken Pflicht, die
Schwachheiten der Kraftlosen zu tragen und nicht Gefallen an uns selber
zu haben. \bibleverse{2} Es soll aber ein jeder von uns seinem Nächsten
gefallen zum Guten, zur Erbauung. \bibleverse{3} Denn auch Christus
hatte nicht an sich selbst Gefallen, sondern wie geschrieben steht:
``Die Schmähungen derer, die dich geschmäht haben, sind auf mich
gefallen.'' \bibleverse{4} Was aber zuvor geschrieben worden ist, das
wurde zu unserer Belehrung geschrieben, damit wir durch die Geduld und
durch den Trost der Schrift Hoffnung fassen. \bibleverse{5} Der Gott der
Geduld und des Trostes aber gebe euch, untereinander eines Sinnes zu
sein, Christus Jesus gemäß, \bibleverse{6} damit ihr einmütig, mit einem
Munde Gott und den Vater unsres Herrn Jesus Christus lobet.
\bibleverse{7} Darum nehmet euch einer des andern an, gleichwie auch
Christus sich euer angenommen hat, zu Gottes Ehre! \bibleverse{8} Ich
sage aber, daß Jesus Christus ein Diener der Beschneidung geworden ist
um der Wahrhaftigkeit Gottes willen, um die Verheißungen an die Väter zu
bestätigen, \bibleverse{9} daß aber die Heiden Gott loben um der
Barmherzigkeit willen, wie geschrieben steht: ``Darum will ich dich
preisen unter den Heiden und deinem Namen lobsingen!'' \bibleverse{10}
Und wiederum spricht er: ``Freuet euch, ihr Heiden, mit seinem Volk!''
\bibleverse{11} Und wiederum: ``Lobet den Herrn, alle Heiden, preiset
ihn, alle Völker!'' \bibleverse{12} Und wiederum spricht Jesaja: ``Es
wird aus der Wurzel Jesses sprossen der, welcher aufsteht, um über die
Heiden zu herrschen; auf ihn werden die Heiden hoffen.'' \bibleverse{13}
Der Gott der Hoffnung aber erfülle euch mit aller Freude und mit Frieden
im Glauben, daß ihr überströmet an Hoffnung, in der Kraft des heiligen
Geistes! \bibleverse{14} Ich habe aber, meine Brüder, die feste
Überzeugung von euch, daß auch ihr selbst voll Gütigkeit seid, erfüllt
mit aller Erkenntnis und fähig, einander zu ermahnen. \bibleverse{15}
Das machte mir aber zum Teil um so mehr Mut, euch zu schreiben, um euer
Gedächtnis wieder aufzufrischen, wegen der Gnade, die mir von Gott
gegeben ist, \bibleverse{16} daß ich ein Diener Jesu Christi für die
Heiden sein soll, der das Evangelium Gottes priesterlich verwaltet, auf
daß das Opfer der Heiden angenehm werde, geheiligt im heiligen Geist.
\bibleverse{17} Ich habe also Grund zum Rühmen in Christus Jesus, vor
Gott. \bibleverse{18} Denn ich würde nicht wagen, etwas davon zu sagen,
wenn nicht Christus es durch mich gewirkt hätte, um die Heiden zum
Gehorsam zu bringen durch Wort und Werk, \bibleverse{19} in Kraft von
Zeichen und Wundern, in Kraft des heiligen Geistes, also daß ich von
Jerusalem an und ringsumher bis nach Illyrien das Evangelium von
Christus völlig ausgerichtet habe, \bibleverse{20} wobei ich es mir zur
Ehre mache, das Evangelium nicht dort zu verkündigen, wo Christi Name
schon bekannt ist, damit ich nicht auf einen fremden Grund baue,
\bibleverse{21} sondern, wie geschrieben steht: ``Welchen nicht von ihm
verkündigt worden ist, die sollen es sehen, und welche es nicht gehört
haben, die sollen es vernehmen.'' \bibleverse{22} Darum bin ich auch
oftmals verhindert worden, zu euch zu kommen. \bibleverse{23} Da ich
jetzt aber in diesen Gegenden keinen Raum mehr habe, wohl aber seit
vielen Jahren ein Verlangen hege, zu euch zu kommen, \bibleverse{24} so
werde ich auf der Reise nach Spanien zu euch kommen; denn ich hoffe,
euch auf der Durchreise zu sehen und von euch dorthin geleitet zu
werden, wenn ich mich zuvor ein wenig an euch erquickt habe.
\bibleverse{25} Nun aber reise ich nach Jerusalem, im Dienste der
Heiligen. \bibleverse{26} Es hat nämlich Mazedonien und Achaja gefallen,
eine Sammlung für die Armen unter den Heiligen in Jerusalem zu
veranstalten; \bibleverse{27} es hat ihnen gefallen, und sie sind es
ihnen auch schuldig; denn wenn die Heiden an ihren geistlichen Gütern
Anteil erhalten haben, so sind sie auch verpflichtet, jenen in den
leiblichen zu dienen. \bibleverse{28} Wenn ich nun das ausgerichtet und
ihnen diese Frucht gesichert habe, will ich bei euch durchreisen nach
Spanien. \bibleverse{29} Ich weiß aber, daß, wenn ich zu euch komme, es
in der Fülle des Segens Christi geschehen wird. \bibleverse{30} Ich
ermahne euch aber, ihr Brüder, durch unsern Herrn Jesus Christus und
durch die Liebe des Geistes, daß ihr mit mir kämpfet in den Gebeten für
mich zu Gott, \bibleverse{31} daß ich errettet werde von den Ungläubigen
in Judäa und daß meine Dienstleistung für Jerusalem den Heiligen
angenehm sei, \bibleverse{32} auf daß ich durch Gottes Willen mit
Freuden zu euch komme und mich mit euch erquicke. \bibleverse{33} Der
Gott aber des Friedens sei mit euch allen! Amen.

\hypertarget{section-15}{%
\section{16}\label{section-15}}

\bibleverse{1} Ich empfehle euch aber unsere Schwester Phöbe, welche
Dienerin der Gemeinde zu Kenchreä ist, \bibleverse{2} damit ihr sie
aufnehmet im Herrn, wie es Heiligen geziemt, und ihr beistehet, in
welcher Sache sie euer bedarf; denn auch sie ist vielen eine
Beschützerin gewesen, auch mir selbst. \bibleverse{3} Grüßet Prisca und
Aquila, meine Mitarbeiter in Christus Jesus, \bibleverse{4} welche für
mein Leben ihren Nacken dargeboten haben, denen nicht allein ich danke,
sondern auch alle Gemeinden der Heiden; grüßet auch die Gemeinde in
ihrem Hause. \bibleverse{5} Grüßet den Epänetus, meinen Geliebten,
welcher ein Erstling von Asien ist für Christus. \bibleverse{6} Grüßet
Maria, welche viel für uns gearbeitet hat. \bibleverse{7} Grüßet
Andronicus und Junias, meine Verwandten und Mitgefangenen, welche unter
den Aposteln angesehen und vor mir in Christus gewesen sind.
\bibleverse{8} Grüßet Amplias, meinen Geliebten im Herrn. \bibleverse{9}
Grüßet Urbanus, unsern Mitarbeiter in Christus, und Stachys, meinen
Geliebten. \bibleverse{10} Grüßet Apelles, den in Christus Bewährten,
grüßet die vom Hause des Aristobulus. \bibleverse{11} Grüßet Herodion,
meinen Verwandten; grüßet die vom Hause des Narcissus, die im Herrn
sind. \bibleverse{12} Grüßet die Tryphena und die Tryphosa, die im Herrn
arbeiten; grüßet Persis, die Geliebte, die viel gearbeitet hat im Herrn.
\bibleverse{13} Grüßet Rufus, den Auserwählten im Herrn, und seine und
meine Mutter. \bibleverse{14} Grüßet Asynkritus, Phlegon, Hermes,
Patrobas, Hermas und die Brüder bei ihnen. \bibleverse{15} Grüßet
Philologus und Julia, Nereus und seine Schwester, auch Olympas und alle
Heiligen bei ihnen. \bibleverse{16} Grüßet einander mit dem heiligen
Kuß! Es grüßen euch alle Gemeinden Christi. \bibleverse{17} Ich ermahne
euch aber, ihr Brüder, gebet acht auf die, welche Trennungen und
Ärgernisse anrichten abseits von der Lehre, die ihr gelernt habt, und
meidet sie. \bibleverse{18} Denn solche dienen nicht dem Herrn Jesus
Christus, sondern ihrem eigenen Bauch, und durch gleisnerische Reden und
schöne Worte verführen sie die Herzen der Arglosen. \bibleverse{19} Denn
euer Gehorsam ist überall bekanntgeworden. Darum freue ich mich über
euch, möchte aber, daß ihr weise wäret zum Guten und unvermischt bliebet
mit dem Bösen. \bibleverse{20} Der Gott des Friedens aber wird den Satan
unter euren Füßen zermalmen in kurzem! Die Gnade unsres Herrn Jesus
Christus sei mit euch! \bibleverse{21} Es grüßen euch Timotheus, mein
Mitarbeiter, und Lucius und Jason und Sosipater, meine Verwandten.
\bibleverse{22} Ich, Tertius, der ich den Brief geschrieben habe, grüße
euch im Herrn. \bibleverse{23} Es grüßt euch Gajus, der mich und die
ganze Gemeinde beherbergt. Es grüßt euch Erastus, der Stadtverwalter,
und Quartus, der Bruder. \bibleverse{24} Die Gnade unsres Herrn Jesus
Christus sei mit euch allen! Amen. \bibleverse{25} Dem aber, der euch
stärken kann laut meines Evangeliums und der Predigt von Jesus Christus,
gemäß der Offenbarung des Geheimnisses, das von ewigen Zeiten her
verschwiegen gewesen, \bibleverse{26} jetzt aber geoffenbart und durch
prophetische Schriften auf Befehl des ewigen Gottes kundgetan worden
ist, zum Gehorsam des Glaubens, für alle Völker, \bibleverse{27} ihm,
dem allein weisen Gott, durch Jesus Christus, sei die Ehre von Ewigkeit
zu Ewigkeit! Amen. (An die Römer gesandt von Korinth durch Phöbe, die
Dienerin der Gemeinde zu Kenchreä.)
