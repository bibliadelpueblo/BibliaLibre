\hypertarget{section}{%
\section{1}\label{section}}

\bibleverse{1} Und Salomo, der Sohn Davids, erlangte Macht über sein
Reich; und der \textsc{Herr}, sein Gott, war mit ihm und machte ihn sehr
groß. \bibleverse{2} Und Salomo redete mit ganz Israel, mit den Obersten
der Tausendschaften und der Hundertschaften, mit den Richtern und mit
allen Fürsten in ganz Israel, mit den Familienhäuptern, \bibleverse{3}
und sie gingen, Salomo und die ganze Gemeinde mit ihm, hin zu der Höhe,
die zu Gibeon war; denn daselbst war die Stiftshütte Gottes, welche
Mose, der Knecht des \textsc{Herrn}, in der Wüste gemacht hatte.
\bibleverse{4} Die Lade Gottes aber hatte David von Kirjat-Jearim
heraufgebracht an den Ort, welchen David ihr bereitet hatte; denn er
hatte für sie in Jerusalem ein Zelt aufgeschlagen. \bibleverse{5} Aber
der eherne Altar, welchen Bezaleel, der Sohn Uris, des Sohnes Churs,
gemacht hatte, war daselbst vor der Hütte des \textsc{Herrn}, und Salomo
und die Gemeinde pflegten ihn zu benutzen. \bibleverse{6} Und Salomo
opferte daselbst vor dem \textsc{Herrn} auf dem ehernen Altar, der vor
der Stiftshütte stand, tausend Brandopfer. \bibleverse{7} In derselben
Nacht erschien Gott dem Salomo und sprach zu ihm: Bitte, was ich dir
geben soll! \bibleverse{8} Und Salomo sprach zu Gott: Du hast an meinem
Vater David große Barmherzigkeit erzeigt, und du hast mich an seiner
Statt zum Könige gemacht. \bibleverse{9} So laß nun, o Gott,
\textsc{Herr}, deine Zusage an meinen Vater David wahr werden! Denn du
hast mich zum Könige gemacht über ein Volk, das so zahlreich ist wie der
Staub auf Erden. \bibleverse{10} So gib mir nun Weisheit und Erkenntnis,
daß ich vor diesem Volk aus und einzugehen weiß; denn wer kann dieses
dein großes Volk richten? \bibleverse{11} Da sprach Gott zu Salomo: Weil
du das im Sinne hast und nicht um Schätze, Reichtum, Ehre, noch um den
Tod deiner Feinde, noch um langes Leben gebeten hast, sondern um
Weisheit und Erkenntnis, mein Volk zu richten, über das ich dich zum
König gemacht habe, \bibleverse{12} so sei dir nun Weisheit und
Erkenntnis gegeben! Dazu will ich dir Reichtum, Schätze und Ehre geben,
dergleichen kein König vor dir gehabt hat, noch nach dir haben soll!
\bibleverse{13} Also kam Salomo von der Höhe, die zu Gibeon war, von der
Stiftshütte her, nach Jerusalem und regierte über Israel.
\bibleverse{14} Und Salomo sammelte Wagen und Reiter, also daß er
tausendvierhundert Wagen und zwölftausend Reiter hatte; die tat er in
die Wagenstädte und zu dem König nach Jerusalem. \bibleverse{15} Und der
König machte, daß in Jerusalem Silber und Gold war so viel wie Steine,
und Zedernholz so viel wie wilde Feigenbäume in den Gründen.
\bibleverse{16} Und man brachte dem Salomo Pferde aus Ägypten. Und je
ein Zug von Kaufleuten des Königs holte sie scharenweise um den
Kaufpreis. \bibleverse{17} Und sie brachten Wagen aus Ägypten herauf,
die kamen auf je sechshundert Schekel Silber zu stehen, und ein Pferd
auf hundertfünfzig. Ebenso brachten sie durch ihre Vermittlung allen
Königen der Hetiter und den Königen in Syrien.

\hypertarget{section-1}{%
\section{2}\label{section-1}}

\bibleverse{1} Und Salomo gedachte, dem Namen des \textsc{Herrn} ein
Haus zu bauen und ein Haus zu seiner Residenz. \bibleverse{2} Und Salomo
zählte 70000 Lastträger ab und 80000 Holzhauer im Gebirge und 3600
Aufseher über sie. \bibleverse{3} Und Salomo sandte zu Huram, dem König
zu Tyrus, und ließ ihm sagen: Wie damals, als du meinem Vater David
Zedern sandtest, daß er sich ein Haus baute, um darin zu wohnen , so tue
auch an mir. \bibleverse{4} Siehe, ich will dem Namen des
\textsc{Herrn}, meines Gottes, ein Haus bauen, um es ihm zu weihen, um
wohlriechendes Räucherwerk vor ihm zu räuchern und allezeit Schaubrote
zuzurüsten und Brandopfer zu opfern, am Morgen und am Abend, an den
Sabbaten und Neumonden und an den Festen des \textsc{Herrn}, unseres
Gottes, was um Israels willen stets geschehen soll. \bibleverse{5} Das
Haus aber, das ich bauen will, soll groß sein; denn unser Gott ist
größer als alle Götter. \bibleverse{6} Aber wer vermag es, ihm ein Haus
zu bauen? Denn der Himmel und aller Himmel Himmel mögen ihn nicht
fassen; und wer bin ich, daß ich ihm ein Haus baue, es sei denn, um vor
ihm zu räuchern? \bibleverse{7} So sende mir nun einen weisen Mann, der
zu arbeiten versteht in Gold, Silber, Erz, Eisen, Purpur, in Stoffen von
Karmesinfarbe und von blauem Purpur, und der sich auf die Bildhauerei
versteht, Damit er arbeite mit den Weisen, die bei mir sind, in Juda und
Jerusalem, für welche mein Vater David gesorgt hat. \bibleverse{8} Und
sende mir Zedern, Zypressen und Sandelholz vom Libanon; denn ich weiß,
daß deine Knechte es verstehen, die Bäume auf dem Libanon zu fällen. Und
siehe, meine Knechte sollen mit deinen Knechten sein, \bibleverse{9}
damit man mir viel Holz zurichte; denn das Haus, das ich bauen will,
soll groß und wunderbar sein. \bibleverse{10} Und siehe, ich will den
Zimmerleuten, deinen Knechten, die das Holz hauen, zwanzigtausend Kor
gestoßenen Weizen, zwanzigtausend Kor Gerste, zwanzigtausend Bat Wein
und zwanzigtausend Bat Öl geben. \bibleverse{11} Da antwortete Huram,
der König von Tyrus, schriftlich und ließ Salomo sagen: Weil der
\textsc{Herr} sein Volk liebt, hat er dich zum König über sie gemacht.
\bibleverse{12} Und Huram sprach weiter: Gelobt sei der \textsc{Herr},
der Gott Israels, der Himmel und Erde gemacht hat, welcher dem König
David einen weisen Sohn gegeben hat, der so klug und verständig ist, daß
er dem \textsc{Herrn} ein Haus bauen kann und für sich selbst ein Haus
zur Residenz! \bibleverse{13} So sende ich nun einen weisen Mann, den
Künstler Huram-Abi. \bibleverse{14} Derselbe ist der Sohn eines Weibes
aus den Töchtern Dans, sein Vater ist ein Tyrer gewesen. Der weiß in
Gold, Silber, Erz, Eisen, Stein und Holz, in rotem und blauem Purpur, in
feiner Baumwolle und in Stoffen von Karmesinfarbe zu arbeiten und
versteht alle Arten von Bildhauerei und weiß jedes Kunstwerk, das ihm
aufgegeben wird, auszuführen mit Hilfe deiner Künstler und der Künstler
meines Herrn David, deines Vaters. \bibleverse{15} So wolle nun mein
\textsc{Herr} seinen Knechten den Weizen, die Gerste, das Öl und den
Wein senden, wie er versprochen hat; \bibleverse{16} und wir werden das
Holz auf dem Libanon hauen, soviel du bedarfst, und es als Flöße auf dem
Meer nach Japho bringen, von wo du es nach Jerusalem hinaufholen kannst.
\bibleverse{17} Und Salomo zählte alle Fremdlinge im Lande Israel, nach
der früheren Zählung, die sein Vater David angeordnet hatte, und es
wurden 153600 gefunden. \bibleverse{18} Von diesen machte er 70000 zu
Lastträgern und 80000 zu Steinhauern im Gebirge und 3600 zu Aufsehern,
die das Volk zur Arbeit anzuhalten hatten.

\hypertarget{section-2}{%
\section{3}\label{section-2}}

\bibleverse{1} Und Salomo fing an, das Haus des \textsc{Herrn} zu bauen
zu Jerusalem, auf dem Berge Morija, wo er seinem Vater David erschienen
war, an dem Orte, welchen David bestimmt hatte, auf der Tenne Ornans,
des Jebusiters. \bibleverse{2} Er fing aber an zu bauen im zweiten
Monat, am zweiten Tage, im vierten Jahre seiner Regierung.
\bibleverse{3} Und also legte Salomo den Grund, das Haus Gottes zu
bauen: die Länge betrug nach altem Maß sechzig Ellen und die Breite
zwanzig Ellen. \bibleverse{4} Die Halle aber, welche der ganzen Breite
des Hauses entlang ging, war zwanzig Ellen lang und hundertzwanzig hoch.
Er überzog sie inwendig mit lauterm Gold. \bibleverse{5} Das große Haus
aber täfelte er mit Zypressenholz und überzog es mit gutem Gold und
machte darauf Palmen und Kettenwerk \bibleverse{6} und überzog das Haus
mit kostbaren Steinen zur Zierde; das Gold aber war Parvaimgold.
\bibleverse{7} Und er überzog das Haus, die Balken, die Schwellen, seine
Wände und seine Türen mit Gold und ließ Cherubim an den Wänden
einschnitzen. \bibleverse{8} Er machte auch den Raum des
Allerheiligsten: seine Länge war zwanzig Ellen, nach der Breite des
Hauses; und seine Breite war auch zwanzig Ellen. Und er überzog es mit
gutem Gold, im Betrage von sechshundert Talenten. \bibleverse{9} Und das
Gewicht der Nägel betrug fünfzig Schekel Gold; er überzog auch die
Söller mit Gold. \bibleverse{10} Er machte im Raume des Allerheiligsten
auch zwei Cherubim von Bildhauerarbeit und überzog sie mit Gold.
\bibleverse{11} Und die Länge der Flügel der Cherubim betrug insgesamt
zwanzig Ellen; ein Flügel des einen Cherubs, fünf Ellen lang, berührte
die Wand des Raumes, und der andere Flügel, auch fünf Ellen lang,
berührte den Flügel des anderen Cherubs. \bibleverse{12} Ebenso maß ein
Flügel des zweiten Cherubs fünf Ellen und berührte die Wand des Raumes,
und der andere Flügel, fünf Ellen lang, berührte den Flügel des anderen
Cherubs, \bibleverse{13} so daß sich die Flügel dieser Cherubim zwanzig
Ellen weit ausbreiteten. Und sie standen auf ihren Füßen, und ihre
Angesichter waren einwärts gewandt. \bibleverse{14} Er machte auch einen
Vorhang von blauem und rotem Purpur und Stoffen von Karmesinfarbe und
feiner Baumwolle und brachte Cherubim darauf an. \bibleverse{15} Und er
machte vor dem Hause zwei Säulen, fünfunddreißig Ellen hoch, und oben
darauf einen Knauf, fünf Ellen hoch. \bibleverse{16} Und er machte
Kettenwerk und tat solches oben auf die Säulen und machte hundert
Granatäpfel und tat sie an das Kettenwerk. \bibleverse{17} Und er
richtete die Säulen vor dem Tempel auf, eine zur Rechten, die andere zur
Linken; und er hieß die zur Rechten Jachin und die zur Linken Boas.

\hypertarget{section-3}{%
\section{4}\label{section-3}}

\bibleverse{1} Er machte auch einen ehernen Altar, zwanzig Ellen lang
und zwanzig Ellen breit und zehn Ellen hoch. \bibleverse{2} Und er
machte das gegossene Meer, zehn Ellen weit von einem Rand bis zum
anderen, gerundet ringsum, und fünf Ellen hoch; und eine Meßschnur von
dreißig Ellen konnte es umfassen. \bibleverse{3} Und es waren Gebilde
von Rindern unter ihm ringsum, die es umgaben, zehn auf die Elle, rings
um das Meer herum; zwei Reihen Rinder waren es, gegossen aus einem Guß
mit dem Meer. \bibleverse{4} Es stand auf zwölf Rindern, deren drei
gegen Mitternacht, drei gegen Abend, drei gegen Mittag und drei gegen
Morgen gewendet waren; und das Meer ruhte oben auf ihnen, und das
Hinterteil von ihnen allen war einwärts gekehrt. \bibleverse{5} Seine
Dicke betrug eine Handbreite, und sein Rand war wie der eines Bechers,
wie die Blüte einer Lilie; und es faßte dreitausend Bat. \bibleverse{6}
Und er machte zehn Kessel und setzte fünf zur Rechten und fünf zur
Linken, um darin zu waschen; was zum Brandopfer gehörte, spülte man
darin ab; das Meer aber war für die Waschungen der Priester bestimmt.
\bibleverse{7} Er machte auch zehn goldene Leuchter, wie sie sein
sollten, und setzte sie in den Tempel; fünf zur Rechten und fünf zur
Linken. \bibleverse{8} Und er machte zehn Tische und tat sie in den
Tempel; fünf zur Rechten und fünf zur Linken. Auch machte er hundert
goldene Becken. \bibleverse{9} Er machte auch einen Vorhof für die
Priester und den großen Vorhof; und Türen für den Vorhof, und er überzog
die Türen mit Erz. \bibleverse{10} Und er setzte das Meer auf die rechte
Seite gegen Morgen, südwärts. \bibleverse{11} Und Huram machte Töpfe,
Schaufeln und Becken. Also vollendete Huram die Arbeit, die er für den
König Salomo am Hause Gottes machte, \bibleverse{12} nämlich die zwei
Säulen und die Kugeln der Knäufe oben an den Säulen und die beiden
Geflechte, um die zwei Kugeln der Knäufe zu bedecken, die oben auf
beiden Säulen waren. \bibleverse{13} Und die vierhundert Granatäpfel an
beiden Geflechten, zwei Reihen Granatäpfel an einem jeden Geflecht, um
die zwei Kugeln der Knäufe oben auf den Säulen zu bedecken.
\bibleverse{14} Auch machte er die Ständer und die Kessel auf den
Ständern; \bibleverse{15} und das eine Meer und die zwölf Rinder
darunter. \bibleverse{16} Und die Töpfe, Schaufeln, Gabeln und alle ihre
Geräte machte Vater Huram dem König Salomo für das Haus des
\textsc{Herrn} aus glänzendem Erz. \bibleverse{17} In der Gegend des
Jordan ließ sie der König gießen in dicker Erde, zwischen Sukkot und
Zareda. \bibleverse{18} Und Salomo machte alle diese Geräte in sehr
großer Menge, also daß das Gewicht des Erzes nicht zu ermitteln war.
\bibleverse{19} Und Salomo machte alle Geräte zum Hause Gottes; nämlich
den goldenen Altar, die Tische, auf denen die Schaubrote liegen;
\bibleverse{20} und die Leuchter mit ihren Lampen von lauterm Gold, um
sie vor dem Allerheiligsten anzuzünden, wie es sich gebührt;
\bibleverse{21} und das Blumenwerk und die Lampen und die Lichtscheren
von Gold. Das alles war von feinstem Gold; \bibleverse{22} dazu die
Messer, Becken, Pfannen und Räuchernäpfe von feinem Gold; auch der
Eingang des Hauses, seine innern Türen zum Allerheiligsten und die Türen
am Hause des Tempels waren vergoldet.

\hypertarget{section-4}{%
\section{5}\label{section-4}}

\bibleverse{1} Also ward alle Arbeit vollendet, die Salomo am Hause des
\textsc{Herrn} machte. Und Salomo brachte hinein, was sein Vater David
geheiligt hatte, dazu das Silber und das Gold und alle Geräte und legte
es in die Schatzkammern des Hauses Gottes. \bibleverse{2} Da versammelte
Salomo die Ältesten in Israel und alle Häupter der Stämme, die Fürsten
der Vaterhäuser der Kinder Israel, in Jerusalem, um die Lade des Bundes
des \textsc{Herrn} heraufzubringen aus der Stadt Davids; das ist Zion.
\bibleverse{3} Und alle Männer Israels versammelten sich beim König zum
Fest, das heißt im siebenten Monat. \bibleverse{4} Und alle Ältesten
Israels kamen; und die Leviten nahmen die Lade \bibleverse{5} und
brachten die Lade hinauf, samt der Stiftshütte und allen heiligen
Geräten, die in dem Zelte waren. Die Priester und die Leviten brachten
sie hinauf. \bibleverse{6} Der König Salomo aber und die ganze Gemeinde
Israel, die vor der Lade bei ihm versammelt war, opferten Schafe und
Rinder, so viel, daß ihre Menge weder zu zählen noch zu berechnen war.
\bibleverse{7} Also brachten die Priester die Bundeslade des
\textsc{Herrn} an ihren Ort in den Chor des Hauses, in das
Allerheiligste, unter die Flügel der Cherubim. \bibleverse{8} Denn die
Cherubim breiteten beide Flügel aus über den Ort der Lade; und die
Cherubim bedeckten die Lade und deren Stangen von oben her.
\bibleverse{9} Die Stangen aber waren so lang, daß man die Enden der
Stangen von der Lade aus, vor dem Chor sehen konnte, aber von außen sah
man sie nicht. Und sie blieb daselbst bis auf diesen Tag.
\bibleverse{10} Es war nichts in der Lade, als die beiden Tafeln, die
Mose am Horeb darein getan hatte, als der \textsc{Herr} mit den Kindern
Israel einen Bund machte, da sie aus Ägypten zogen. \bibleverse{11} Und
als die Priester aus dem Heiligtum herausgingen (denn alle Priester, die
vorhanden waren, hatten sich geheiligt, ohne Rücksicht auf die
Abteilungen), \bibleverse{12} und als auch die Leviten, alle Sänger,
Asaph, Heman, Jedutun und ihre Söhne und ihre Brüder, in weiße Baumwolle
gekleidet, dastanden mit Zimbeln, Psaltern und Harfen östlich vom Altar,
und bei ihnen hundertundzwanzig Priester, die auf Trompeten bliesen,
\bibleverse{13} da war es, wie wenn die, welche die Trompeten bliesen
und sangen, nur eine Stimme hören ließen, zu loben und zu danken dem
\textsc{Herrn}. Und als sie die Stimme erhoben mit Trompeten, Zimbeln
und Saitenspiel und mit dem Lobe des \textsc{Herrn}, daß er freundlich
ist und seine Güte ewig währt, da ward das Haus des \textsc{Herrn} mit
einer Wolke erfüllt, \bibleverse{14} so daß die Priester wegen der Wolke
nicht zum Dienste antreten konnten, denn die Herrlichkeit des
\textsc{Herrn} erfüllte das Haus Gottes.

\hypertarget{section-5}{%
\section{6}\label{section-5}}

\bibleverse{1} Damals sprach Salomo: Der \textsc{Herr} hat gesagt, er
wolle im Dunkeln wohnen. \bibleverse{2} Ich aber habe ein Haus gebaut,
dir zur Wohnung, und einen Sitz, da du ewiglich wohnen mögest.
\bibleverse{3} Und der König wandte sein Angesicht und segnete die ganze
Gemeinde Israel; denn die ganze Gemeinde Israel stand da. \bibleverse{4}
Und er sprach: Gelobt sei der \textsc{Herr}, der Gott Israels, der durch
seinen Mund meinem Vater David verheißen und es auch mit seiner Hand
erfüllt hat, da er sagte: \bibleverse{5} Seit der Zeit, als ich mein
Volk aus Ägyptenland führte, habe ich in allen Stämmen Israels keine
Stadt erwählt, um ein Haus zu bauen, daß mein Name daselbst sei, und
habe auch keinen Mann erwählt, daß er über mein Volk Israel Fürst sei.
\bibleverse{6} Aber Jerusalem habe ich erwählt, daß mein Name daselbst
sei; und David habe ich erwählt, daß er über mein Volk Israel König sei.
\bibleverse{7} Und mein Vater David hatte im Sinne, dem Namen des
\textsc{Herrn}, des Gottes Israels, ein Haus zu bauen. \bibleverse{8}
Aber der \textsc{Herr} sprach zu meinem Vater David: Daß du im Sinne
gehabt hast, meinem Namen ein Haus zu bauen, daran hast du wohlgetan,
daß du das im Sinne gehabt hast. \bibleverse{9} Doch sollst nicht du das
Haus bauen, sondern dein Sohn, der aus deinen Lenden hervorgehen wird,
der soll meinem Namen das Haus bauen. \bibleverse{10} Und nun hat der
\textsc{Herr} sein Wort erfüllt, das er geredet hat; denn ich bin an
meines Vaters Statt getreten und sitze auf dem Throne Israels, wie der
\textsc{Herr} versprochen hat, und ich habe dem Namen des
\textsc{Herrn}, des Gottes Israels, ein Haus gebaut \bibleverse{11} und
dorthinein die Lade gestellt, worin der Bund des \textsc{Herrn} ist, den
er mit den Kindern Israel gemacht hat. \bibleverse{12} Und er trat vor
den Altar des \textsc{Herrn}, angesichts der ganzen Gemeinde Israel, und
breitete seine Hände aus. \bibleverse{13} Denn Salomo hatte eine eherne
Kanzel gemacht und mitten in den Hof gestellt, fünf Ellen lang und fünf
Ellen breit und drei Ellen hoch; darauf trat er und fiel auf seine Knie
nieder, angesichts der ganzen Gemeinde Israel, und breitete seine Hände
aus gen Himmel \bibleverse{14} und sprach: O \textsc{Herr}, Gott
Israels! Kein Gott ist dir gleich, weder im Himmel noch auf Erden, der
du den Bund und die Barmherzigkeit beobachtest deinen Knechten, die von
ganzem Herzen vor dir wandeln! \bibleverse{15} Der du deinem Knecht
David, meinem Vater, gehalten, was du ihm verheißen hast. Mit deinem
Munde hattest du es geredet, und mit deiner Hand hast du es erfüllet,
wie es heute der Fall ist. \bibleverse{16} So halte nun, o
\textsc{Herr}, Gott Israels, deinem Knechte David, meinem Vater, was du
zu ihm gesagt hast, als du sprachest: Es soll dir nicht mangeln an einem
Mann vor mir, der auf dem Throne Israels sitze, wenn nur deine Kinder
auf ihren Weg achthaben, daß sie in meinem Gesetze wandeln, wie du vor
mir gewandelt bist! \bibleverse{17} Und nun, \textsc{Herr}, Gott
Israels, laß dein Wort wahr werden, welches du zu deinem Knecht David
gesprochen hast! \bibleverse{18} Sollte aber Gott wahrhaftig bei den
Menschen auf Erden wohnen? Siehe, der Himmel und aller Himmel Himmel
können dich nicht fassen; wie sollte es denn dieses Haus tun, das ich
gebaut habe? \bibleverse{19} Wende dich aber, o \textsc{Herr}, mein
Gott, zum Gebet deines Knechtes und zu seinem Flehen, daß du erhörest
das Lob und die Bitte, die dein Knecht vor dir tut, \bibleverse{20} also
daß deine Augen Tag und Nacht offenstehen über dem Ort, davon du gesagt
hast, daß du deinen Namen dahin setzen wollest; daß du erhörest das
Gebet, das dein Knecht für diesen Ort tun will. \bibleverse{21} So höre
nun das Flehen deines Knechtes und deines Volkes Israel, was sie an
dieser Stätte bitten werden! Höre du es an dem Ort deiner Wohnung, im
Himmel, und wenn du es hörst, so vergib! \bibleverse{22} Wenn jemand
wider seinen Nächsten sündigt, und man legt ihm einen Eid auf, den er
schwören soll, und der Eid kommt in diesem Hause vor deinen Altar,
\bibleverse{23} so wollest du hören im Himmel und verschaffen, daß
deinen Knechten Recht gesprochen wird, indem du dem Gottlosen vergiltst
und seinen Weg auf seinen Kopf fallen lässest und den Gerechten
rechtfertigst und ihm gibst nach seiner Gerechtigkeit. \bibleverse{24}
Und wenn dein Volk Israel vor seinen Feinden geschlagen wird, weil sie
an dir gesündigt haben, und sie kehren um und bekennen deinen Namen und
beten und flehen in diesem Hause vor dir, \bibleverse{25} so wollest du
hören im Himmel und die Sünden deines Volkes Israel vergeben und sie
wieder in das Land bringen, das du ihnen und ihren Vätern gegeben hast!
\bibleverse{26} Wenn sich der Himmel verschließt, daß es nicht regnet,
weil sie an dir gesündigt haben, und sie an diesem Ort beten und deinen
Namen bekennen, weil du sie gedemütigt hast, \bibleverse{27} so wollest
du hören im Himmel und die Sünden deiner Knechte und deines Volkes
Israel vergeben, daß du sie den rechten Weg lehrest, darin sie wandeln
sollen, und regnen lässest auf dein Land, das du deinem Volk zu besitzen
gegeben hast. \bibleverse{28} Wenn Hungersnot im Lande herrscht; wenn
Pestilenz, Getreidebrand, Vergilben, wenn Heuschrecken und Fresser sein
werden; wenn sein Feind im Lande seine Tore belagert, oder sonst eine
Plage und Krankheit herrschen wird; \bibleverse{29} was man alsdann
bittet und fleht, es geschehe von Menschen, wer sie seien, oder von
deinem ganzen Volk Israel, wenn sie innewerden, ein jeder seine Plage
und seinen Schmerz, und sie ihre Hände zu diesem Hause ausbreiten,
\bibleverse{30} so wollest du hören im Himmel, am Sitze deiner Wohnung,
und vergeben und jedermann geben nach all seinem Weg, wie du sein Herz
erkennst; denn du allein erkennst das Herz der Menschenkinder,
\bibleverse{31} auf daß sie dich fürchten, um in deinen Wegen zu wandeln
alle Tage, solange sie in dem Lande leben, das du unsern Vätern gegeben
hast. \bibleverse{32} Und wenn auch ein Fremdling, der nicht von deinem
Volk Israel ist, aus fernen Landen kommt, um deines großen Namens und
deiner mächtigen Hand und deines ausgestreckten Arms willen, und kommt
und in diesem Hause betet, \bibleverse{33} so wollest du im Himmel, am
Sitze deiner Wohnung, hören und alles tun, um was dieser Fremdling dich
anruft, auf daß alle Völker auf Erden deinen Namen erkennen und dich
fürchten, wie dein Volk Israel, und erfahren, daß dieses Haus, das ich
gebaut habe, nach deinem Namen genannt sei. \bibleverse{34} Wenn dein
Volk wider seine Feinde in den Krieg zieht, auf dem Wege, den du sie
senden wirst, und sie zu dir beten, nach dieser Stadt gewandt, die du
erwählt hast, und dem Hause, das ich deinem Namen gebaut habe,
\bibleverse{35} so wollest du ihr Gebet und ihr Flehen im Himmel hören
und ihnen zu ihrem Recht verhelfen! \bibleverse{36} Wenn sie an dir
sündigen werden (da kein Mensch ist, der nicht sündigt), und du über sie
zürnst und sie vor ihren Feinden dahingibst, so daß dieselben sie in ein
fernes oder nahes Land gefangen hinwegführen, \bibleverse{37} und sie in
ihrem Herzen umkehren, im Lande ihrer Gefangenschaft, und sprechen:
``Wir haben gesündigt, böse gehandelt und sind gottlos gewesen'',
\bibleverse{38} und sie also von ganzem Herzen und von ganzer Seele zu
dir zurückkehren im Lande ihrer Gefangenschaft, da man sie gefangen
hält, und sie beten, gegen ihr Land gewandt, das du ihren Vätern gegeben
hast, und nach der Stadt hin, die du erwählt hast, und nach dem Hause
hin, das ich deinem Namen gebaut habe, \bibleverse{39} so wollest du ihr
Gebet und ihr Flehen hören im Himmel, am Sitze deiner Wohnung, und ihnen
zu ihrem Recht verhelfen und deinem Volk vergeben, das an dir gesündigt
hat! \bibleverse{40} So laß nun doch, mein Gott, deine Augen offen sein
und deine Ohren aufmerken auf das Gebet an diesem Ort! \bibleverse{41}
Und mache dich nun auf, o Gott, \textsc{Herr}, zu deiner Ruhe, du und
die Lade deiner Macht! Laß deine Priester, o Gott, \textsc{Herr}, mit
Heil angetan werden und deine Frommen sich freuen über das Gute!
\bibleverse{42} O Gott, \textsc{Herr}, weise nicht ab das Angesicht
deines Gesalbten! Gedenke der Gnaden, die du deinem Knechte David
verheißen hast!

\hypertarget{section-6}{%
\section{7}\label{section-6}}

\bibleverse{1} Als nun Salomo sein Gebet vollendet hatte, fiel das Feuer
vom Himmel und verzehrte das Brandopfer und die Schlachtopfer. Und die
Herrlichkeit des \textsc{Herrn} erfüllte das Haus; \bibleverse{2} also
daß die Priester nicht in das Haus des \textsc{Herrn} hineingehen
konnten, weil die Herrlichkeit des \textsc{Herrn} das Haus des
\textsc{Herrn} erfüllte. \bibleverse{3} Als aber alle Kinder Israel das
Feuer herabfallen sahen und die Herrlichkeit des \textsc{Herrn} über dem
Hause, fielen sie auf ihre Knie, mit dem Angesicht zur Erde, auf das
Pflaster, und beteten an und dankten dem \textsc{Herrn}, daß er
freundlich ist und seine Güte ewiglich währt. \bibleverse{4} Und der
König und alles Volk opferten Schlachtopfer vor dem \textsc{Herrn}.
\bibleverse{5} Der König Salomo opferte als Schlachtopfer 22000 Rinder
und 120000 Schafe. Also weihten der König und das ganze Volk das Haus
Gottes ein. \bibleverse{6} Die Priester aber standen auf ihren Posten
und die Leviten mit den Musikinstrumenten des \textsc{Herrn}, welche der
König David hatte machen lassen, um dem \textsc{Herrn} zu danken, daß
seine Güte ewig währt, wenn David durch sie den Lobpreis darbrachte. Und
die Priester bliesen die Trompeten, ihnen gegenüber, und ganz Israel
stand dabei. \bibleverse{7} Und Salomo heiligte den innern Vorhof, der
vor dem Hause des \textsc{Herrn} war; denn er brachte daselbst
Brandopfer dar und die Fettstücke der Dankopfer; denn der eherne Altar,
welchen Salomo hatte machen lassen, konnte die Brandopfer und Speisopfer
und die Fettstücke nicht fassen. \bibleverse{8} Und zu jener Zeit hielt
Salomo ein Fest, sieben Tage lang, und ganz Israel mit ihm, eine sehr
große Gemeinde, von Chamat an bis an den Bach Ägyptens; \bibleverse{9}
und sie hielten am achten Tage eine Festversammlung. Denn die Einweihung
des Altars hatten sie sieben Tage lang gefeiert und das Fest auch sieben
Tage lang. \bibleverse{10} Aber am dreiundzwanzigsten Tage des siebenten
Monats ließ er das Volk in ihre Hütten ziehen, fröhlich und guten Muts,
wegen all des Guten, das der \textsc{Herr} an David, Salomo und seinem
Volk Israel getan. \bibleverse{11} Als nun Salomo das Haus des
\textsc{Herrn} und das Haus des Königs vollendet hatte und alles
glücklich ausgeführt war, was in Salomos Herz gekommen war, es im Hause
des \textsc{Herrn} und in seinem Hause zu machen, \bibleverse{12} da
erschien der \textsc{Herr} dem Salomo in der Nacht und sprach zu ihm:
Ich habe dein Gebet erhört und mir diesen Ort zur Opferstätte erwählt.
\bibleverse{13} Siehe, wenn ich den Himmel zuschließe, daß es nicht
regnet, oder den Heuschrecken gebiete, das Land abzufressen, oder wenn
ich eine Pestilenz unter mein Volk sende, \bibleverse{14} und sich mein
Volk, das nach meinem Namen genannt ist, demütigt, und sie beten und
suchen mein Angesicht und wenden sich ab von ihren bösen Wegen, so will
ich im Himmel hören und ihre Sünden vergeben und ihr Land heilen.
\bibleverse{15} So sollen nun meine Augen offen stehen und meine Ohren
aufmerken auf das Gebet an diesem Ort. \bibleverse{16} Ich habe nun
dieses Haus erwählt und geheiligt, daß mein Name daselbst sein soll
ewiglich; und meine Augen und mein Herz sollen da sein täglich.
\bibleverse{17} Du aber, wenn du vor mir wandeln wirst, wie dein Vater
David gewandelt ist, und du alles tust, was ich dich heiße, und meine
Satzungen und Rechte bewahrst, \bibleverse{18} so will ich den Thron
deines Königreichs befestigen, gemäß dem Bund, den ich mit deinem Vater
David gemacht habe, indem ich sagte: Es soll dir nicht mangeln an einem
Mann, der über Israel herrsche. \bibleverse{19} Werdet ihr euch aber
abwenden und meine Satzungen und Gebote, die ich euch vorgelegt habe,
verlassen und hingehen und anderen Göttern dienen und sie anbeten,
\bibleverse{20} so werde ich sie aus meinem Lande ausrotten, das ich
ihnen gegeben habe; und dieses Haus, welches ich meinem Namen geheiligt
habe, werde ich von meinem Angesicht verwerfen und es zum Sprichwort
setzen und zum Spott unter allen Völkern, \bibleverse{21} daß jedermann,
der an diesem Hause, welches das höchste gewesen ist, vorübergeht, sich
entsetzen und sagen wird: Warum ist der \textsc{Herr} mit diesem Lande
und diesem Hause also verfahren? \bibleverse{22} So wird man sagen: Weil
sie den \textsc{Herrn}, den Gott ihrer Väter, der sie aus Ägyptenland
geführt hat, verlassen und sich an andere Götter gehängt und sie
angebetet und ihnen gedient haben, darum hat er all dieses Unglück über
sie gebracht!

\hypertarget{section-7}{%
\section{8}\label{section-7}}

\bibleverse{1} Und nach zwanzig Jahren, in welchen Salomo das Haus des
\textsc{Herrn} und sein eigenes Haus gebaut hatte, \bibleverse{2} baute
Salomo auch die Städte aus, die Huram dem Salomo gegeben, und ließ die
Kinder Israel darin wohnen. \bibleverse{3} Und Salomo zog gen
Chamat-Zoba und überwältigte es, \bibleverse{4} und baute Tadmor in der
Wüste und alle Vorratsstädte, die er baute in Chamat. \bibleverse{5} Er
baute auch das obere Beth-Horon und das untere Beth-Horon, feste Städte
mit Mauern, Toren und Riegeln. \bibleverse{6} Auch Baalat und alle
Vorratsstädte, die Salomo gehörten, und alle Wagenstädte und
Reiterstädte und alles, wozu Salomo zu bauen Lust hatte in Jerusalem und
auf dem Libanon und im ganzen Lande seiner Herrschaft. \bibleverse{7}
Und alles Volk, das von den Hetitern, Amoritern, Pheresitern, Hevitern
und Jebusitern noch übrig war, die nicht zu den Kindern Israel gehörten,
\bibleverse{8} ihre Nachkommen, die sie im Lande gelassen, welche die
Kinder Israel nicht vertilgt hatten, machte Salomo fronpflichtig, bis
auf diesen Tag. \bibleverse{9} Aber von den Kindern Israel machte er
keine zu Leibeigenen für seine Arbeit, sondern sie waren seine
Kriegsleute und Oberste seiner Wagenkämpfer und seine Reiter.
\bibleverse{10} Und die Zahl dieser obersten Amtleute des Königs Salomo
betrug zweihundertfünfzig, die hatten die Aufsicht über das Volk.
\bibleverse{11} Und Salomo brachte die Tochter des Pharao aus der Stadt
Davids herauf in das Haus, das er für sie gebaut hatte. Denn er sprach:
Mein Weib soll nicht im Hause Davids, des Königs Israels, wohnen; denn
heilig ist die Stätte, weil die Lade des \textsc{Herrn} hineingekommen
ist! \bibleverse{12} Von da an opferte Salomo dem \textsc{Herrn}
Brandopfer auf dem Altar des \textsc{Herrn}, den er vor der Halle gebaut
hatte, \bibleverse{13} was an jedem Tag zu opfern war nach dem Gesetze
Moses, an den Sabbaten und Neumonden und an den Festzeiten, dreimal im
Jahre, nämlich am Fest der ungesäuerten Brote, am Fest der Wochen und am
Laubhüttenfest. \bibleverse{14} Und er bestellte die Abteilungen der
Priester, wie sein Vater David sie geordnet hatte, zu ihrem Amt, und die
Leviten zu ihren Posten, um zu loben und zu dienen vor den Priestern,
wie es ein jeder Tag erforderte; und die Torhüter nach ihren Abteilungen
zu einem jeden Tor; denn also hatte es David, der Mann Gottes, befohlen.
\bibleverse{15} Und sie wichen nicht vom Gebot des Königs betreffs der
Priester und Leviten, in keinem Wort, auch hinsichtlich der Schätze
nicht. \bibleverse{16} Also kam das ganze Werk Salomos zustande, von dem
Tage an, als das Haus des \textsc{Herrn} gegründet ward, bis zu seiner
Vollendung; er machte das Haus des \textsc{Herrn} fertig.
\bibleverse{17} Damals ging Salomo nach Ezjon-Geber und Elot, das am
Gestade des Meeres liegt, im Lande Edom. \bibleverse{18} Und Huram
sandte ihm Schiffe durch seine Knechte, die des Meeres kundig waren; die
fuhren mit den Knechten Salomos nach Ophir und holten von dort 450
Talente Gold und brachten es dem König Salomo.

\hypertarget{section-8}{%
\section{9}\label{section-8}}

\bibleverse{1} Und als die Königin von Saba das Gerücht von Salomo
hörte, kam sie, um Salomo mit Rätseln zu erproben, nach Jerusalem mit
einem sehr großen Gefolge und mit Kamelen, die Gewürz und Gold in Menge
und Edelsteine trugen. Und als sie zu Salomo kam, redete sie mit ihm
alles, was sie in ihrem Herzen hatte. \bibleverse{2} Und Salomo gab ihr
über alles Aufschluß; es war dem Salomo nichts verborgen, daß er ihr
darüber nicht Aufschluß gegeben hätte. \bibleverse{3} Als nun die
Königin von Saba die Weisheit Salomos sah und das Haus, das er gebaut
hatte, \bibleverse{4} und die Speise auf seinem Tisch, die Wohnung
seiner Knechte und das Auftreten seiner Diener und ihre Kleider, seine
Mundschenken und ihre Kleider und seine Brandopfer, die er im Haus des
\textsc{Herrn} darbrachte, kam sie außer sich vor Erstaunen
\bibleverse{5} und sprach zum König: Es ist wahr, was ich in meinem
Lande von deinen Sachen und von deiner Weisheit gehört habe!
\bibleverse{6} Ich aber wollte ihren Worten nicht glauben, bis ich
gekommen bin und es mit eigenen Augen gesehen habe. Und siehe, es ist
mir nicht die Hälfte von deiner Weisheit gesagt worden; du hast das
Gerücht übertroffen, das ich vernommen habe. \bibleverse{7} Selig sind
deine Leute, ja, selig diese deine Knechte, die allezeit vor dir stehen
und deine Weisheit hören! \bibleverse{8} Der \textsc{Herr}, dein Gott,
sei gelobt, der Lust zu dir hatte, daß er dich auf seinen Thron setzte
als König vor dem \textsc{Herrn}, deinem Gott! Darum, weil dein Gott
Israel liebt und es ewiglich erhalten will, hat er dich zum König
gesetzt, daß du Recht und Gerechtigkeit übest! \bibleverse{9} Und sie
gab dem König hundertzwanzig Talente Gold und sehr viel Gewürz und
Edelsteine; es gab sonst kein solches Gewürz wie das, welches die
Königin von Saba dem König Salomo schenkte. \bibleverse{10} (Dazu
brachten die Knechte Hurams und die Knechte Salomos, welche Gold aus
Ophir holten, auch Sandelholz und Edelsteine. \bibleverse{11} Und der
König ließ aus dem Sandelholz Geländer machen im Hause des
\textsc{Herrn} und im Hause des Königs, und Harfen und Psalter für die
Sänger: dergleichen war zuvor im Lande Juda niemals gesehen worden.)
\bibleverse{12} Und der König Salomo gab der Königin von Saba alles, was
sie begehrte und bat, viel mehr als das, was sie selbst dem König
gebracht hatte. Dann kehrte sie in ihr Land zurück samt ihren Knechten.
\bibleverse{13} Das Gewicht des Goldes aber, das dem Salomo in einem
Jahre einging, betrug 666 Talente Gold, \bibleverse{14} außer dem, was
die Handelsleute und die Kaufleute brachten. Es brachten auch alle
Könige der Araber und die Gewaltigen des Landes Gold und Silber zu
Salomo. \bibleverse{15} Und der König Salomo machte zweihundert Schilde
von geschlagenem Gold, wobei sechshundert Schekel geschlagenen Goldes
auf einen Schild kamen. \bibleverse{16} Und dreihundert Tartschen von
geschlagenem Gold, wobei dreihundert Schekel geschlagenen Goldes auf
eine Tartsche kamen; und der König tat sie ins Haus vom Libanonwalde.
\bibleverse{17} Und der König machte einen großen Thron von Elfenbein
und überzog ihn mit reinem Gold. \bibleverse{18} Und der Thron hatte
sechs Stufen und einen goldenen Fußschemel, an dem Thron befestigt, und
es befanden sich Lehnen zu beiden Seiten des Sitzes, und zwei Löwen
standen an den Lehnen. \bibleverse{19} Ferner standen zwölf Löwen
daselbst auf den sechs Stufen zu beiden Seiten. Solches ist in keinem
Königreich jemals gemacht worden. \bibleverse{20} Und alle
Trinkgeschirre des Königs Salomo waren golden, und alle Geschirre im
Hause vom Libanonwald feines Gold; denn zu Salomos Zeit wurde das Silber
für nichts gerechnet. \bibleverse{21} Denn die Schiffe des Königs fuhren
gen Tarsis mit den Knechten Hurams; einmal in drei Jahren kamen die
Tarsis-Schiffe und brachten Gold, Silber, Elfenbein, Affen und Pfauen.
\bibleverse{22} Also war der König Salomo an Reichtum und Weisheit
größer als alle Könige auf Erden. \bibleverse{23} Und alle Könige auf
Erden begehrten das Angesicht Salomos zu sehen und seine Weisheit zu
hören, die ihm Gott in sein Herz gegeben hatte. \bibleverse{24} Und sie
brachten ihm jährlich ein jeder sein Geschenk, silberne und goldene
Geräte, Kleider, Waffen und Gewürze, Pferde und Maultiere.
\bibleverse{25} Und Salomo hatte viertausend Pferdestände und Wagen und
zwölftausend Reiter; die ließ man in den Wagenstädten und bei dem König
zu Jerusalem. \bibleverse{26} Und er war Herrscher über alle Könige, vom
Euphratstrom an bis an das Land der Philister und bis an die Grenzen
Ägyptens. \bibleverse{27} Und der König machte, daß es zu Jerusalem so
viel Silber gab wie Steine und so viel Zedern wie wilde Feigenbäume in
den Tälern. \bibleverse{28} Und man brachte Salomo Pferde aus Ägypten
und aus allen Ländern. \bibleverse{29} Die weitern Geschichten Salomos
aber, die ersten und die letzten, sind die nicht aufgezeichnet in den
Schriften des Propheten Natan und in der Weissagung Achijas von Silo und
in den Gesichten Iddos, des Sehers, wider Jerobeam, den Sohn Nebats?
\bibleverse{30} Und Salomo regierte zu Jerusalem über ganz Israel
vierzig Jahre lang. \bibleverse{31} Und Salomo legte sich zu seinen
Vätern, und man begrub ihn in der Stadt seines Vaters David; und
Rehabeam, sein Sohn, ward König an seiner Statt.

\hypertarget{section-9}{%
\section{10}\label{section-9}}

\bibleverse{1} Und Rehabeam zog nach Sichem; denn ganz Israel war nach
Sichem gekommen, ihn zum König zu machen. \bibleverse{2} Als aber
Jerobeam, der Sohn Nebats, der in Ägypten war, wohin er sich vor dem
König Salomo geflüchtet hatte, solches vernahm, kehrte er aus Ägypten
zurück. \bibleverse{3} Da sandten sie hin und ließen ihn rufen. Und
Jerobeam kam mit ganz Israel, und sie redeten mit Rehabeam und sprachen:
\bibleverse{4} Dein Vater hat unser Joch zu hart gemacht; erleichtere
nun du den harten Dienst deines Vaters und das schwere Joch, das er uns
auferlegt hat, so wollen wir dir untertan sein! \bibleverse{5} Er sprach
zu ihnen: Kommt in drei Tagen wieder zu mir! Und das Volk ging hin.
\bibleverse{6} Da beriet sich der König Rehabeam mit den Ältesten, die
vor seinem Vater Salomo zu dessen Lebzeiten gestanden hatten, und
sprach: Wie ratet ihr, daß man diesem Volk antworten soll?
\bibleverse{7} Sie sagten zu ihm und sprachen: Wirst du gegen dieses
Volk freundlich und ihm gefällig sein und ihnen gute Worte geben, so
werden sie dir allezeit dienen! \bibleverse{8} Aber er verließ den Rat
der Ältesten, den sie ihm gegeben hatten, und beriet sich mit den
Jungen, die mit ihm aufgewachsen waren und vor ihm standen.
\bibleverse{9} Und er sprach zu ihnen: Was ratet ihr, daß wir diesem
Volk antworten, das zu mir gesagt hat: Erleichtere das Joch, das dein
Vater uns auferlegt hat? \bibleverse{10} Da sprachen die Jungen, die mit
ihm aufgewachsen waren: Dem Volk, das zu dir gesagt hat: ``Dein Vater
hat unser Joch zu schwer gemacht; mache du unser Joch leichter'', dem
antworte du: Mein kleiner Finger soll dicker sein, als meines Vaters
Lenden! \bibleverse{11} Hat euch mein Vater ein schweres Joch
aufgeladen, so will ich euch noch mehr aufladen! Hat mein Vater euch mit
Geißeln gezüchtigt, so will ich euch mit Skorpionen züchtigen!
\bibleverse{12} Als nun am dritten Tage Jerobeam und alles Volk zu
Rehabeam kamen, wie der König gesagt hatte: ``Kommt wieder zu mir am
dritten Tag'', \bibleverse{13} da antwortete ihnen der König hart. Denn
der König Rehabeam verließ den Rat der Ältesten \bibleverse{14} und
redete mit ihnen nach dem Rat der Jungen und sprach: Hat mein Vater euer
Joch schwer gemacht, so will ich es noch schwerer machen! Hat mein Vater
euch mit Geißeln gezüchtigt, so will ich euch mit Skorpionen züchtigen!
\bibleverse{15} Also willfahrte der König dem Volke nicht; denn es ward
also von Gott gefügt, damit der \textsc{Herr} sein Wort bekräftige, das
er durch Achija von Silo zu Jerobeam, dem Sohn Nebats, geredet hatte.
\bibleverse{16} Als aber ganz Israel sah, daß der König ihnen nicht
willfahrte, antwortete das Volk dem König und sprach: Was haben wir für
Anteil an David? Wir haben nichts zu erben vom Sohne Isais! In seine
eigene Hütte gehe, wer zu Israel gehört! Du, David, magst selbst zu
deinem Hause sehen! Und ganz Israel ging in seine Hütten;
\bibleverse{17} so daß Rehabeam nur über die Kinder Israel regierte, die
in den Städten Judas wohnten. \bibleverse{18} Und als der König Rehabeam
den Fronmeister Hadoram hinsandte, warfen ihn die Kinder Israel mit
Steinen zu Tode. Der König Rehabeam aber sprang eilends auf seinen
Wagen, um nach Jerusalem zu entfliehen. \bibleverse{19} Also fiel Israel
ab vom Hause Davids bis auf diesen Tag.

\hypertarget{section-10}{%
\section{11}\label{section-10}}

\bibleverse{1} Als aber Rehabeam nach Jerusalem kam, versammelte er das
Haus Juda und Benjamin, 180000 streitbare junge Männer, um wider Israel
zu streiten und das Königreich wieder an Rehabeam zu bringen.
\bibleverse{2} Aber das Wort des \textsc{Herrn} kam zu Semaja, dem Manne
Gottes, also: \bibleverse{3} Sage zu Rehabeam, dem Sohne Salomos, dem
König von Juda, und zu ganz Israel, das unter Juda und Benjamin ist, und
sprich: \bibleverse{4} So spricht der \textsc{Herr}: Ihr sollt nicht
hinaufziehen, noch wider eure Brüder streiten! Jedermann kehre wieder
heim! Denn solches ist von mir so gefügt worden. Sie folgten den Worten
des \textsc{Herrn} und kehrten um und zogen nicht wider Jerobeam.
\bibleverse{5} Und Rehabeam blieb zu Jerusalem und baute Städte in Juda
zu Festungen um, \bibleverse{6} und zwar baute er Bethlehem, Etam,
Tekoa, \bibleverse{7} Beth-Zur, Socho, Adullam, \bibleverse{8} Gat,
Marescha, Siph, \bibleverse{9} Adoraim, Lachis, Aseka, \bibleverse{10}
Zorea, Ajalon und Hebron, welche in Juda und Benjamin waren, zu
Festungen um. \bibleverse{11} Und er verstärkte die Festungen und tat
Befehlshaber hinein und Vorräte an Nahrung, Öl und Wein, \bibleverse{12}
und tat in alle Städte Schilde und Speere und machte sie sehr fest. So
gehörten Juda und Benjamin ihm. \bibleverse{13} Auch die Priester und
Leviten aus ganz Israel und aus allen ihren Gebieten stellten sich zu
ihm. \bibleverse{14} Denn die Leviten verließen ihre Bezirke und ihr
Besitztum und kamen nach Juda und Jerusalem. Denn Jerobeam und seine
Söhne hatten sie verstoßen, also daß sie des Priesteramts nicht pflegen
konnten vor dem \textsc{Herrn}; \bibleverse{15} er bestellte aber für
sich selbst Priester, für die Höhen und für die Böcke und Kälber, welche
er machen ließ. \bibleverse{16} Jenen Leviten aber folgten aus allen
Stämmen Israels die, welche ihr Herz darauf richteten, den
\textsc{Herrn}, den Gott Israels, zu suchen; diese kamen nach Jerusalem,
dem \textsc{Herrn}, dem Gott ihrer Väter, zu opfern. \bibleverse{17}
Diese stärkten das Königreich Juda und ermutigten Rehabeam, den Sohn
Salomos, drei Jahre lang; denn sie wandelten auf dem Wege Davids und
Salomos drei Jahre lang. \bibleverse{18} Und Rehabeam nahm Machalat, die
Tochter Jerimots, des Sohnes Davids, zur Frau, und Abichail, die Tochter
Eliabs, des Sohnes Isais. \bibleverse{19} Die gebar ihm Söhne: Jeusch,
Semarja und Sacham. \bibleverse{20} Nach dieser nahm er Maacha, die
Tochter Absaloms, die gebar ihm Abija, Attai, Sisa und Selomit.
\bibleverse{21} Aber Rehabeam hatte Maacha, die Tochter Absaloms, lieber
als alle seine andern Frauen und Nebenfrauen und zeugte achtundzwanzig
Söhne und sechzig Töchter. \bibleverse{22} Und Rehabeam setze Abija, den
Sohn der Maacha, zum Haupt und zum Fürsten unter seinen Brüdern; denn er
gedachte ihn zum König zu machen. \bibleverse{23} Und er war verständig
und verteilte alle seine Söhne in alle Landschaften von Juda und
Benjamin, in alle festen Städte. Und er gab ihnen reichlichen Unterhalt
und verlangte viele Frauen für sie.

\hypertarget{section-11}{%
\section{12}\label{section-11}}

\bibleverse{1} Als aber Rehabeams Herrschaft befestigt und er stark
geworden war, verließ er das Gesetz des \textsc{Herrn}, und ganz Israel
mit ihm. \bibleverse{2} Da geschah es, daß im fünften Jahre des Königs
Rehabeam Sisak, der König von Ägypten, wider Jerusalem heraufzog (denn
sie hatten sich am \textsc{Herrn} versündigt) \bibleverse{3} mit 1200
Wagen und 60000 Reitern; und das Volk war nicht zu zählen, das mit ihm
aus Ägypten kam: Lybier, Suchiter und Mohren. \bibleverse{4} Und er
eroberte die festen Städte, die in Juda waren, und gelangte bis nach
Jerusalem. \bibleverse{5} Da kam Semaja, der Prophet, zu Rehabeam und zu
den Obersten Judas, die sich um Sisaks willen zu Jerusalem versammelt
hatten, und sprach zu ihnen: So spricht der \textsc{Herr}: Ihr habt mich
verlassen; darum habe auch Ich euch verlassen und in Sisaks Hand
gegeben! \bibleverse{6} Da demütigten sich die Obersten Israels mit dem
König und sprachen: Der \textsc{Herr} ist gerecht! \bibleverse{7} Als
aber der \textsc{Herr} sah, daß sie sich demütigten, erging das Wort des
\textsc{Herrn} an Semaja also: Sie haben sich gedemütigt, darum will ich
sie nicht verderben, sondern ich will ihnen ein wenig Rettung
verschaffen, daß mein Grimm durch die Hand Sisaks nicht auf Jerusalem
ausgegossen werde. \bibleverse{8} Doch sollen sie ihm untertan sein,
damit sie erfahren, was es sei, mir zu dienen, oder den Königreichen der
Länder zu dienen. \bibleverse{9} Also zog Sisak, der König von Ägypten,
nach Jerusalem hinauf und nahm die Schätze im Hause des \textsc{Herrn}
und die Schätze im Hause des Königs und nahm alles hinweg, auch die
goldenen Schilde, welche Salomo hatte machen lassen. \bibleverse{10} An
deren Statt ließ der König Rehabeam eherne Schilde machen und übergab
sie dem Obersten der Trabanten, die an der Tür des Hauses des Königs
hüteten. \bibleverse{11} Und sooft der König in das Haus des
\textsc{Herrn} ging, kamen die Trabanten und trugen sie und brachten sie
wieder in der Trabanten Kammer. \bibleverse{12} Weil er sich nun
demütigte, wandte sich der Zorn des \textsc{Herrn} von ihm, so daß nicht
alles verderbt wurde; denn es war in Juda noch etwas Gutes.
\bibleverse{13} Also erholte sich der König Rehabeam in Jerusalem und
regierte. Denn einundvierzig Jahre alt war Rehabeam, als er König ward,
und regierte siebzehn Jahre lang zu Jerusalem, in der Stadt, die der
\textsc{Herr} aus allen Stämmen Israels erwählt hatte, daß er seinen
Namen daselbst wohnen lasse. Seine Mutter aber hieß Naama, eine
Ammoniterin. \bibleverse{14} Und er tat das Böse; denn er hatte sein
Herz nicht darauf gerichtet, den \textsc{Herrn} zu suchen.
\bibleverse{15} Die Geschichten aber Rehabeams, die früheren und die
späteren, sind sie nicht geschrieben in den Geschichten Semajas, des
Propheten, und Iddos, des Sehers, da die Geschlechter aufgezeichnet
sind; dazu die Kriege Rehabeams und Jerobeams, ihr Leben lang?
\bibleverse{16} Und Rehabeam legte sich zu seinen Vätern und ward
begraben in der Stadt Davids; und Abija, sein Sohn, ward König an seiner
Statt.

\hypertarget{section-12}{%
\section{13}\label{section-12}}

\bibleverse{1} Im achtzehnten Jahre des Königs Jerobeam ward Abija König
in Juda \bibleverse{2} und regierte drei Jahre lang zu Jerusalem. Seine
Mutter hieß Maacha, eine Tochter Uriels von Gibea. Und es war Krieg
zwischen Abija und Jerobeam. \bibleverse{3} Und Abija rüstete sich zum
Krieg mit einem Heere von 400000 Kriegsleuten, auserlesener Mannschaft.
Jerobeam aber rüstete sich, mit ihm zu streiten, mit 800000 Mann,
auserlesenen, tapferen Leuten. \bibleverse{4} Und Abija stellte sich
oben auf den Berg Zemaraim, welcher zu dem Gebirge Ephraim gehört, und
rief: Höret mir zu, Jerobeam und du ganz Israel! \bibleverse{5} Wisset
ihr nicht, daß der \textsc{Herr}, der Gott Israels, das Königtum über
Israel David gegeben hat auf ewige Zeiten, ihm und seinen Söhnen, durch
einen Salzbund? \bibleverse{6} Aber Jerobeam, der Sohn Nebats, der
Knecht Salomos, des Sohnes Davids, erhob sich und ward von seinem Herrn
abtrünnig. \bibleverse{7} Und es haben sich lose Leute, Kinder Belials,
zu ihm geschlagen, die widersetzten sich Rehabeam, dem Sohne Salomos;
denn Rehabeam war noch jung und zu furchtsam, um ihnen zu widerstehen.
\bibleverse{8} Und nun, glaubt ihr, dem Reiche des \textsc{Herrn}
widerstehen zu können, welches in der Hand der Söhne Davids ist, weil
ihr ein großer Haufe seid und ihr bei euch die goldenen Kälber habt,
welche euch Jerobeam zu Göttern gemacht hat? \bibleverse{9} Habt ihr
nicht die Priester des \textsc{Herrn}, die Kinder Aarons, und die
Leviten ausgestoßen und habt euch eigene Priester gemacht, wie die
Völker der Länder? Wer irgend kommt, seine Hand zu füllen mit einem
jungen Farren und sieben Widdern, der wird Priester derer, die doch
nicht Götter sind! \bibleverse{10} Unser Gott aber ist der
\textsc{Herr}, und wir haben ihn nicht verlassen; und die Priester, die
dem \textsc{Herrn} dienen, sind Söhne Aarons, und die Leviten stehen in
ihrem Amt \bibleverse{11} und zünden dem \textsc{Herrn} alle Morgen und
alle Abend Brandopfer an, dazu das gute Räucherwerk, und besorgen die
Zubereitung des Brotes auf dem reinen Tisch und den goldenen Leuchter
mit seinen Lampen, daß sie alle Abend angezündet werden. Denn wir
beobachten die Vorschriften des \textsc{Herrn}, unsres Gottes; ihr aber
habt ihn verlassen! \bibleverse{12} Und siehe, mit uns an unserer Spitze
ist Gott und seine Priester und die Lärmtrompeten, daß man wider euch
Lärm blase. Ihr Kinder Israel, streitet nicht wider den \textsc{Herrn},
den Gott eurer Väter, denn es wird euch nicht gelingen! \bibleverse{13}
Aber Jerobeam hatte den Hinterhalt ausgesandt, sie zu umgehen, so daß er
vor Juda stand, der Hinterhalt aber in ihrem Rücken. \bibleverse{14} Als
sich nun Juda umwandte, siehe, da war Kampf vorne und hinten! Da
schrieen sie zum \textsc{Herrn}, und die Priester bliesen in die
Trompeten. \bibleverse{15} Und die Männer Judas erhoben ein
Feldgeschrei. Und als die Männer Judas ein Feldgeschrei erhoben, schlug
Gott den Jerobeam und ganz Israel vor Abija und Juda. \bibleverse{16}
Und die Kinder Israel flohen vor Juda; denn Gott gab sie in ihre Hand,
\bibleverse{17} also daß Abija mit seinem Volk ihnen eine große
Niederlage zufügte, und aus Israel fielen der Erschlagenen 500000
auserlesene Mannschaft. \bibleverse{18} Also wurden die Kinder Israel zu
jener Zeit gedemütigt, aber die Kinder Juda wurden gestärkt; denn sie
verließen sich auf den \textsc{Herrn}, den Gott ihrer Väter.
\bibleverse{19} Und Abija jagte Jerobeam nach und gewann ihm Städte ab,
nämlich Bethel mit seinen Dörfern und Jeschana mit seinen Dörfern und
Ephron mit seinen Dörfern; \bibleverse{20} so daß Jerobeam forthin nicht
mehr zu Kräften kam, solange Abija lebte. Und der \textsc{Herr} schlug
ihn, daß er starb. \bibleverse{21} Als nun Abija erstarkte, nahm er
vierzehn Frauen und zeugte zweiundzwanzig Söhne und sechzehn Töchter.
\bibleverse{22} Was aber mehr von Abija zu sagen ist, und seine Wege und
seine Reden, das ist geschrieben in der Schrift des Propheten Iddo.

\hypertarget{section-13}{%
\section{14}\label{section-13}}

\bibleverse{1} Und Abija legte sich zu seinen Vätern, und sie begruben
ihn in der Stadt Davids; und Asa, sein Sohn, ward König an seiner Statt.
Zu dessen Zeiten war das Land stille, zehn Jahre lang. \bibleverse{2}
Und Asa tat, was gut und recht war vor dem \textsc{Herrn}, seinem Gott.
\bibleverse{3} Denn er entfernte die fremden Altäre und die Höhen und
zerbrach die Säulen und hieb die Ascheren um; \bibleverse{4} und gebot
Juda, den \textsc{Herrn}, den Gott ihrer Väter, zu suchen und zu tun
nach dem Gesetz und Gebot. \bibleverse{5} Er entfernte auch aus allen
Städten Judas die Höhen und die Sonnensäulen. Und das Königreich hatte
Ruhe unter ihm. \bibleverse{6} Und er baute feste Städte in Juda, weil
in jenen Jahren das Land Ruhe hatte und kein Krieg wider ihn geführt
wurde; denn der \textsc{Herr} gab ihm Ruhe. \bibleverse{7} Und er sprach
zu Juda: Lasset uns diese Städte bauen und sie mit Mauern umgeben und
mit Türmen, Toren und Riegeln, weil das Land noch vor uns liegt! Denn
wir haben den \textsc{Herrn}, unsern Gott, gesucht; wir haben ihn
gesucht, und er hat uns Ruhe gegeben ringsumher. Also bauten sie, und es
gelang ihnen. \bibleverse{8} Und Asa hatte ein Heer, das Schild und
Speer trug, aus Juda 300000 und aus Benjamin 280000, welche die Tartsche
trugen und mit Bogen schossen. Diese waren alle starke Helden.
\bibleverse{9} Aber Serach, der Mohr, zog aus wider sie mit einem Heer
von tausendmal tausend, dazu dreihundert Wagen, und er kam bis Marescha.
\bibleverse{10} Und Asa zog aus, ihm entgegen. Und sie rüsteten sich zum
Kampf im Tal Zephata bei Marescha. \bibleverse{11} Und Asa rief den
\textsc{Herrn}, seinen Gott, an und sprach: \textsc{Herr}, bei dir ist
kein Unterschied, zu helfen, wo viel oder wo keine Kraft ist. Hilf uns,
\textsc{Herr}, unser Gott, denn wir verlassen uns auf dich; und in
deinem Namen sind wir gekommen wider diesen Haufen! Du, \textsc{Herr},
bist unser Gott! Vor dir behält der Sterbliche keine Kraft!
\bibleverse{12} Da schlug der \textsc{Herr} die Mohren vor Asa und vor
Juda, daß die Mohren flohen. \bibleverse{13} Und Asa samt dem Volk, das
bei ihm war, jagte ihnen nach bis gen Gerar. Und von den Mohren fielen
so viele, daß sie sich nicht erholen konnten, sondern sie wurden
geschlagen vor dem \textsc{Herrn} und vor seiner Heerschar; und sie
trugen sehr viel Raub davon. \bibleverse{14} Und sie schlugen alle
Städte um Gerar her; denn die Furcht des \textsc{Herrn} kam über sie.
Und sie plünderten alle Städte; denn es war viel Beute darin.
\bibleverse{15} Auch die Hirtenzelte schlugen sie und führten viele
Schafe und Kamele hinweg und kehrten wieder nach Jerusalem zurück.

\hypertarget{section-14}{%
\section{15}\label{section-14}}

\bibleverse{1} Und der Geist Gottes kam auf Asaria, den Sohn Odeds;
\bibleverse{2} der ging hinaus, Asa entgegen, und sprach zu ihm: Höret
mir zu, Asa, und du, ganz Juda und Benjamin! Der \textsc{Herr} ist mit
euch, wenn ihr mit ihm seid; und wenn ihr ihn suchet, so wird er sich
von euch finden lassen; werdet ihr aber ihn verlassen, so wird er euch
auch verlassen! \bibleverse{3} Israel war lange Zeit ohne den wahren
Gott und ohne einen Priester, welcher lehrt, und ohne Gesetz.
\bibleverse{4} Als es sich aber in seiner Not zu dem \textsc{Herrn}, dem
Gott Israels, kehrte und ihn suchte, da ließ er sich von ihnen finden.
\bibleverse{5} Aber zu jenen Zeiten hatten die, welche aus und
eingingen, keinen Frieden, sondern es kamen große Schrecken über alle
Landesbewohner. \bibleverse{6} Und es schlug sich ein Volk mit dem
andern und eine Stadt mit der andern; denn Gott erschreckte sie durch
allerlei Not. \bibleverse{7} Ihr aber, ermannet euch und laßt eure Hände
nicht sinken; denn euer Werk hat seinen Lohn! \bibleverse{8} Als nun Asa
diese Worte und die Weissagung des Propheten Oded hörte, ermannte er
sich und schaffte die Greuel hinweg aus dem ganzen Lande Juda und
Benjamin und aus den Städten, die er auf dem Gebirge Ephraim erobert
hatte, und erneuerte den Altar des \textsc{Herrn}, der vor der Halle des
\textsc{Herrn} stand. \bibleverse{9} Und er versammelte ganz Juda und
Benjamin und die Fremdlinge bei ihnen aus Ephraim, Manasse und Simeon;
denn es fielen ihm sehr viele Leute aus Israel zu, als sie sahen, daß
der \textsc{Herr}, sein Gott, mit ihm war. \bibleverse{10} Und sie
versammelten sich zu Jerusalem im dritten Monat, im fünfzehnten Jahre
der Regierung Asas. \bibleverse{11} Und sie opferten dem \textsc{Herrn}
an jenem Tage von der Beute, die sie gebracht hatten, siebenhundert
Rinder und siebentausend Schafe. \bibleverse{12} Und sie gingen den Bund
ein, daß sie den \textsc{Herrn}, den Gott ihrer Väter, suchen wollten
mit ihrem ganzen Herzen und ihrer ganzen Seele; \bibleverse{13} wer aber
den \textsc{Herrn}, den Gott Israels, nicht suchen würde, der sollte
sterben, ob klein oder groß, ob Mann oder Weib. \bibleverse{14} Und sie
schwuren dem \textsc{Herrn} mit lauter Stimme, mit Jauchzen, Trompeten
und Posaunen. \bibleverse{15} Und ganz Juda war fröhlich über den Eid;
denn sie hatten mit ihrem ganzen Herzen geschworen; und sie suchten ihn
mit ihrem ganzen Willen; und er ließ sich von ihnen finden. Und der
\textsc{Herr} gab ihnen Ruhe ringsumher. \bibleverse{16} Auch setzte der
König Asa seine Mutter Maacha ab, daß sie nicht mehr Gebieterin wäre,
weil sie der Aschera ein Götzenbild gemacht hatte. Und Asa hieb das
Götzenbild um und zermalmte es und verbrannte es am Bach Kidron.
\bibleverse{17} Aber die Höhen kamen nicht weg aus Israel; doch war das
Herz Asas ungeteilt sein Leben lang. \bibleverse{18} Und er brachte das,
was sein Vater geheiligt und was er selbst geheiligt hatte, in das Haus
Gottes, nämlich Silber, Gold und Geräte. \bibleverse{19} Und es war kein
Krieg bis zum fünfunddreißigsten Jahre der Regierung Asas.

\hypertarget{section-15}{%
\section{16}\label{section-15}}

\bibleverse{1} Im sechsunddreißigsten Jahre der Regierung Asas zog
Baesa, der König Israels, herauf wider Juda und baute Rama, um Asa, dem
König von Juda, weder Ausgang noch Eingang zu lassen. \bibleverse{2}
Aber Asa nahm aus dem Schatz im Hause des \textsc{Herrn} und im Hause
des Königs Silber und Gold und sandte zu Benhadad, dem König von Syrien,
der zu Damaskus wohnte, und ließ ihm sagen: \bibleverse{3} Es ist ein
Bund zwischen mir und dir und zwischen meinem und deinem Vater; siehe,
darum habe ich dir Silber und Gold gesandt. Gehe hin, löse das Bündnis
mit Baesa, dem König von Israel, daß er von mir abziehe! \bibleverse{4}
Und Benhadad gehorchte dem König Asa und sandte seine Heerführer wider
die Städte Israels; die schlugen Ijon, Dan, Abel-Maim und alle
Vorratsstädte in Naphtali. \bibleverse{5} Als Baesa solches hörte, ließ
er ab, Rama zu bauen, und stellte seine Arbeit ein. \bibleverse{6} Da
nahm der König Asa ganz Juda und ließ sie die Steine und das Holz, womit
Baesa baute, von Rama wegtragen, und er baute damit Geba und Mizpa.
\bibleverse{7} Und zu jener Zeit kam Hanani, der Seher, zu Asa, dem
König von Juda, und sprach zu ihm: Weil du dich auf den König von Syrien
verlassen und dich nicht auf den \textsc{Herrn}, deinen Gott, verlassen
hast, darum ist das Heer des Königs von Syrien deiner Hand entronnen!
\bibleverse{8} Waren nicht die Mohren und Lybier ein gewaltiges Heer mit
sehr vielen Wagen und Reitern? Dennoch gab sie der \textsc{Herr} in
deine Hand, als du dich auf ihn verließest. \bibleverse{9} Denn die
Augen des \textsc{Herrn} durchstreifen die ganze Erde, um sich mächtig
zu erzeigen an denen, die von ganzem Herzen ihm ergeben sind. Du hast
hierin töricht gehandelt; darum wirst du von nun an Krieg haben!
\bibleverse{10} Aber Asa ward zornig über den Seher und legte ihn ins
Gefängnis; denn er zürnte ihm deswegen. Asa unterdrückte auch etliche
von dem Volk zu jener Zeit. \bibleverse{11} Und siehe, die Geschichten
Asas, die ersten und die letzten, sind geschrieben im Buch der Könige
von Juda und Israel. \bibleverse{12} Und Asa ward krank an seinen Füßen
im neununddreißigsten Jahr seines Königreichs, und seine Krankheit nahm
sehr zu; doch suchte er auch in seiner Krankheit nicht den
\textsc{Herrn}, sondern die Ärzte. \bibleverse{13} Also legte sich Asa
zu seinen Vätern und starb im einundvierzigsten Jahre seines
Königreichs. \bibleverse{14} Und man begrub ihn in seinem Grabe, das er
sich in der Stadt Davids hatte aushauen lassen. Und sie legten ihn auf
ein Lager, welches man angefüllt hatte mit gutem Räucherwerk und
allerlei Spezereien, nach der Kunst des Salbenbereiters gemacht, und sie
zündeten ihm ein sehr großes Feuer an.

\hypertarget{section-16}{%
\section{17}\label{section-16}}

\bibleverse{1} Und sein Sohn Josaphat ward König an seiner Statt und
ward mächtig wider Israel. \bibleverse{2} Denn er legte Kriegsvolk in
alle festen Städte Judas und legte Besatzungen in das Land Juda und in
die Städte Ephraims, die sein Vater Asa erobert hatte. \bibleverse{3}
Und der \textsc{Herr} war mit Josaphat; denn er wandelte in den früheren
Wegen seines Vaters David und suchte nicht die Baale auf, \bibleverse{4}
sondern den Gott seines Vaters suchte er und wandelte in seinen Geboten
und tat nicht wie Israel. \bibleverse{5} Darum bestätigte ihm der
\textsc{Herr} das Königtum. Und ganz Juda gab Josaphat Geschenke, so daß
er viel Reichtum und Ehre hatte. \bibleverse{6} Und da sein Herz in den
Wegen des \textsc{Herrn} mutig ward, tat er die übrigen Höhen und die
Ascheren aus Juda hinweg. \bibleverse{7} Und im dritten Jahre seiner
Regierung sandte er seine Fürsten Benchail, Obadja, Sacharja, Netaneel
und Michaja, daß sie in den Städten Judas lehren sollten; \bibleverse{8}
und mit ihnen die Leviten Semaja, Netanja, Sebadja, Asahel, Semiramot,
Jonatan, Adonia, Tobia und Tob-Adonia, die Leviten; und mit ihnen
Elisama und Joram, die Priester. \bibleverse{9} Und sie lehrten in Juda
und hatten das Gesetzbuch des \textsc{Herrn} bei sich; sie zogen in
allen Städten Judas umher und lehrten das Volk. \bibleverse{10} Und die
Furcht des \textsc{Herrn} kam über alle Königreiche der Länder, die
rings um Juda lagen, so daß sie nicht wider Josaphat stritten.
\bibleverse{11} Und man brachte Josaphat Geschenke von den Philistern
und Silber als Tribut. Und die Araber brachten ihm Kleinvieh, 7700
Widder und 7700 Böcke. \bibleverse{12} Also nahm Josaphat zu und ward
immer größer. Und er baute Burgen und Vorratsstädte in Juda.
\bibleverse{13} Und er hatte viel Vorrat in den Städten Judas und zu
Jerusalem streitbare Männer, tapfere Helden. \bibleverse{14} Und dies
ist das Ergebnis ihrer Musterung nach ihren Vaterhäusern: In Juda waren
Befehlshaber über Tausende: Odna, ein Oberster, und mit ihm 300000
tapfere Helden. \bibleverse{15} Und neben ihm war Johanan, der Oberste;
und mit ihm 280000. \bibleverse{16} Und neben ihm Amasja, der Sohn
Sichris, der Freiwillige des \textsc{Herrn}; und mit ihm 200000 tapfere
Helden. \bibleverse{17} Von Benjamin war Eljada, ein tapferer Mann; und
mit ihm 200000 Mann, die mit Bogen und Schild bewaffnet waren.
\bibleverse{18} Und neben ihm Josabad; und mit ihm 180000 zum Heer
Gerüstete. \bibleverse{19} Diese standen alle im Dienste des Königs,
außer denen, welche der König in die festen Städte in ganz Juda gelegt
hatte.

\hypertarget{section-17}{%
\section{18}\label{section-17}}

\bibleverse{1} Als nun Josaphat großen Reichtum und Ehre erlangt hatte,
verschwägerte er sich mit Ahab. \bibleverse{2} Und nach etlichen Jahren
zog er zu Ahab hinab, nach Samaria. Und Ahab ließ für ihn und das Volk,
das bei ihm war, viele Schafe und Rinder schlachten und beredete ihn,
gen Ramot in Gilead hinaufzuziehen. \bibleverse{3} Denn Ahab, der König
von Israel, sprach zu Josaphat, dem König von Juda: Willst du mit mir
nach Ramot in Gilead hinaufziehen? Er sprach zu ihm: Ich will sein wie
du, und mein Volk wie dein Volk, und ich will mit dir in den Krieg.
\bibleverse{4} Aber Josaphat sprach zum König von Israel: Befrage doch
heute das Wort des \textsc{Herrn}! \bibleverse{5} Da versammelte der
König von Israel die Propheten, vierhundert Mann, und sprach zu ihnen:
Sollen wir gen Ramot in Gilead in den Krieg ziehen, oder soll ich es
unterlassen? Sie sprachen: Ziehe hinauf, denn Gott wird sie in die Hand
des Königs geben! \bibleverse{6} Josaphat aber sprach: Ist hier kein
Prophet des \textsc{Herrn} mehr, daß wir durch ihn fragen könnten?
\bibleverse{7} Der König von Israel sprach zu Josaphat: Es ist noch ein
Mann, durch den man den \textsc{Herrn} fragen kann; aber ich bin ihm
gram, denn er weissagt mir nichts Gutes, sondern immer nur Böses; das
ist Michaja, der Sohn Jimlas. Josaphat sprach: Der König rede nicht
also! \bibleverse{8} Da rief der König von Israel einen seiner Kämmerer
und sprach: Bringe eilends her Michaja, den Sohn Jimlas! \bibleverse{9}
Und der König von Israel und Josaphat, der König von Juda, saßen ein
jeder auf seinem Throne, mit königlichen Kleidern angetan. Sie saßen
aber auf dem Platze vor dem Tor zu Samaria, und alle Propheten
weissagten vor ihnen. \bibleverse{10} Und Zedekia, der Sohn Kenaanas,
machte sich eiserne Hörner und sprach: So spricht der \textsc{Herr}:
Hiermit wirst du die Syrer stoßen, bis sie aufgerieben sind!
\bibleverse{11} Und alle Propheten weissagten auch also und sprachen:
Ziehe hinauf gen Ramot in Gilead, und es wird dir wohlgehen! Der
\textsc{Herr} wird es in die Hand des Königs geben! \bibleverse{12} Und
der Bote, der hingegangen war, Michaja zu rufen, redete mit ihm und
sprach: Siehe, die Reden der Propheten sind einstimmig gut für den
König. So laß nun dein Wort auch sein wie das ihre und rede Gutes!
\bibleverse{13} Michaja aber sprach: So wahr der \textsc{Herr} lebt: was
mein Gott sagen wird, das will ich reden! \bibleverse{14} Und als er zum
König kam, sprach der König zu ihm: Micha, sollen wir gen Ramot in
Gilead in den Krieg ziehen, oder soll ich es unterlassen? Er sprach:
Ziehet hinauf und fahret wohl; sie sollen in eure Hände gegeben werden!
\bibleverse{15} Aber der König sprach zu ihm: Wie oft muß ich dich
beschwören, daß du mir nichts als die Wahrheit sagest im Namen des
\textsc{Herrn}? \bibleverse{16} Da sprach er: Ich sah ganz Israel auf
den Bergen zerstreut, wie Schafe, die keinen Hirten haben. Und der
\textsc{Herr} sprach: Diese haben keinen Herrn; ein jeder kehre wieder
heim in Frieden! \bibleverse{17} Da sprach der König von Israel zu
Josaphat: Sagte ich dir nicht, er weissage mir nichts Gutes, sondern nur
Böses? \bibleverse{18} Er aber sprach: Darum höret das Wort des
\textsc{Herrn}: Ich sah den \textsc{Herrn} auf seinem Throne sitzen, und
das ganze Heer des Himmels stand zu seiner Rechten und zu seiner Linken.
\bibleverse{19} Und der \textsc{Herr} sprach: Wer will Ahab, den König
von Israel, betören, daß er hinaufziehe und falle zu Ramot in Gilead?
\bibleverse{20} Und nachdem der eine dies, der andere das gesagt hatte,
kam ein Geist hervor und trat vor den \textsc{Herrn} und sprach: Ich
will ihn betören! Der \textsc{Herr} aber sprach zu ihm: Womit?
\bibleverse{21} Er sprach: Ich will ausgehen und ein Geist der Lüge sein
im Munde aller seiner Propheten! Da sprach er: Du sollst ihn betören,
und du wirst es auch vermögen! Gehe aus und tue also! \bibleverse{22}
Und nun siehe, der \textsc{Herr} hat einen Geist der Lüge in den Mund
dieser deiner Propheten gelegt; und der \textsc{Herr} hat Unglück über
dich beschlossen. \bibleverse{23} Da trat Zedekia, der Sohn Kenaanas,
herzu und schlug Michaja auf den Backen und sprach: Auf welchem Weg ist
der Geist des \textsc{Herrn} von mir gewichen, um mit dir zu reden?
\bibleverse{24} Michaja sprach: Siehe, du wirst es sehen an dem Tage,
wenn du von einem Gemach in das andere laufen wirst, um dich zu
verbergen! \bibleverse{25} Da sprach der König von Israel: Nehmt Michaja
und bringet ihn wiederum zu Amon, dem Obersten der Stadt, und zu Joas,
dem Sohn des Königs, \bibleverse{26} und saget: So spricht der König:
Legt diesen ins Gefängnis und speiset ihn mit Brot der Trübsal und mit
Wasser der Trübsal, bis ich in Frieden wiederkomme! \bibleverse{27}
Michaja sprach: Kommst du in Frieden wieder, so hat der \textsc{Herr}
nicht durch mich geredet! Und er sprach noch: Höret zu, ihr Völker alle!
\bibleverse{28} Also zogen der König von Israel und Josaphat, der König
von Juda, hinauf gen Ramot in Gilead. \bibleverse{29} Und der König von
Israel sprach zu Josaphat: Ich will verkleidet in den Kampf ziehen; du
aber bekleide dich mit deinen Kleidern! Und der König von Israel
verkleidete sich, und sie zogen in den Kampf. \bibleverse{30} Aber der
König von Syrien hatte den Obersten über seine Wagen ausdrücklich
geboten: Ihr sollt weder gegen Kleine noch Große streiten, sondern
allein gegen den König von Israel! \bibleverse{31} Als nun die Obersten
der Wagen Josaphat sahen, dachten sie: ``Das ist der König von Israel!''
und gingen auf ihn los zum Kampf. Aber Josaphat schrie, und der
\textsc{Herr} half ihm; und Gott lockte sie von ihm weg. \bibleverse{32}
Als nun die Obersten der Wagen sahen, daß er nicht der König von Israel
sei, wandten sie sich von ihm ab. \bibleverse{33} Aber ein Mann spannte
seinen Bogen von ungefähr und traf den König von Israel zwischen den
Fugen des Panzers. Da sprach er zu seinem Wagenlenker: Wende um und
führe mich aus dem Heer; denn ich bin verwundet! \bibleverse{34} Aber
der Kampf ward heftiger an jenem Tag. Und der König von Israel stand auf
seinem Wagen den Syrern gegenüber, bis zum Abend, und er starb zur Zeit
des Sonnenuntergangs.

\hypertarget{section-18}{%
\section{19}\label{section-18}}

\bibleverse{1} Aber Josaphat, der König von Juda, kehrte in Frieden heim
nach Jerusalem. \bibleverse{2} Und Jehu, der Sohn Hananis, der Seher,
ging hinaus, ihm entgegen, und sprach zum König Josaphat: Solltest du
also dem Gottlosen helfen und die lieben, welche den \textsc{Herrn}
hassen? Deswegen ist der Zorn des \textsc{Herrn} wider dich entbrannt!
\bibleverse{3} Aber doch ist etwas Gutes an dir gefunden worden, weil du
die Ascheren aus dem Lande ausgerottet und dein Herz darauf gerichtet
hast, Gott zu suchen. \bibleverse{4} Darnach verblieb Josaphat zu
Jerusalem; dann ging er wieder aus unter das Volk, von Beerseba bis zum
Gebirge Ephraim, und führte sie zu dem \textsc{Herrn}, dem Gott ihrer
Väter, zurück. \bibleverse{5} Und er bestellte Richter im Lande, in
allen festen Städten Judas, in einer jeden Stadt besonders.
\bibleverse{6} Und er sprach zu den Richtern: Sehet zu, was ihr tut!
Denn ihr haltet das Gericht nicht für Menschen, sondern für den
\textsc{Herrn}, und er ist mit euch beim Urteilsspruch. \bibleverse{7}
Darum sei die Furcht des \textsc{Herrn} über euch; nehmt euch wohl in
acht, was ihr tut! Denn bei dem \textsc{Herrn}, unserm Gott, gibt es
weder Unrecht noch Ansehen der Person noch Bestechlichkeit!
\bibleverse{8} Auch in Jerusalem bestellte Josaphat etliche von den
Leviten und Priestern und Familienhäuptern Israels für das Gericht des
\textsc{Herrn} und für die Rechtshändel derer, die wieder nach Jerusalem
gekommen waren. \bibleverse{9} Und er gebot ihnen und sprach: Also sollt
ihr handeln in der Furcht des \textsc{Herrn}, in Wahrheit und mit
unverletztem Gewissen. \bibleverse{10} In jedem Rechtsstreit, der vor
euch gebracht wird von seiten eurer Brüder, die in ihren Städten wohnen,
betreffe es Blutrache oder Gesetz und Gebot, Satzungen und Rechte, sollt
ihr sie unterrichten, damit sie sich nicht an dem \textsc{Herrn}
versündigen und sein Zorn nicht über euch und eure Brüder komme. Tut
also und versündigt euch nicht! \bibleverse{11} Und siehe, Amarja, der
oberste Priester, ist über euch gesetzt für alle göttlichen
Angelegenheiten; Sebadja aber, der Sohn Ismaels, ist Fürst im Hause Juda
für alle königlichen Geschäfte, und als Amtleute stehen euch die Leviten
vor. Gehet mutig ans Werk! Der \textsc{Herr} aber sei mit dem Guten!

\hypertarget{section-19}{%
\section{20}\label{section-19}}

\bibleverse{1} Darnach kamen die Moabiter und die Ammoniter und mit
ihnen etliche von den Meunitern, um Josaphat zu bekriegen.
\bibleverse{2} Und man kam und verkündigte Josaphat und sprach: Es kommt
eine große Menge wider dich von jenseits des Toten Meeres, aus Syrien;
und siehe, sie sind zu Hazezon-Tamar, das ist Engedi! \bibleverse{3} Da
fürchtete sich Josaphat und befleißigte sich, den \textsc{Herrn} zu
suchen, und ließ in ganz Juda ein Fasten ausrufen. \bibleverse{4} Und
Juda kam zusammen, den \textsc{Herrn} zu suchen; auch aus allen Städten
Judas kamen sie, den \textsc{Herrn} zu suchen. \bibleverse{5} Josaphat
trat unter die Gemeinde von Juda und Jerusalem im Hause des
\textsc{Herrn}, vor dem neuen Vorhofe, \bibleverse{6} und sprach: O
\textsc{Herr}, Gott unsrer Väter, bist du nicht Gott im Himmel und
Herrscher über alle Königreiche der Heiden? In deiner Hand ist Kraft und
Macht, und niemand vermag vor dir zu bestehen! \bibleverse{7} Hast nicht
du, unser Gott, die Einwohner dieses Landes vor deinem Volk Israel
vertrieben und hast es dem Samen Abrahams, deines Freundes, gegeben, auf
ewige Zeiten? \bibleverse{8} Sie haben sich darin niedergelassen und dir
darin ein Heiligtum für deinen Namen gebaut und gesagt: \bibleverse{9}
Wenn Unglück, Schwert des Gerichts, Pestilenz oder Hungersnot über uns
kommt und wir vor diesem Hause und vor dir stehen (da dein Name in
diesem Hause wohnt), und wir in unsrer Not zu dir schreien, so wollest
du hören und helfen! \bibleverse{10} Und nun siehe, die Ammoniter und
Moabiter und die vom Gebirge Seir, durch deren Land zu ziehen du den
Kindern Israel nicht erlaubtest, als sie aus Ägyptenland zogen, sondern
von denen sie sich ferne hielten und die sie nicht vertilgen durften,
\bibleverse{11} siehe, diese lassen uns das entgelten und kommen, um uns
aus deinem Erbe, welches du uns verliehen hast, zu vertreiben.
\bibleverse{12} Unser Gott, willst du sie nicht richten? Denn in uns ist
keine Kraft gegen diesen großen Haufen, der wider uns kommt; und wir
wissen nicht, was wir tun sollen, sondern unsre Augen sehen auf dich!
\bibleverse{13} Und ganz Juda stand vor dem \textsc{Herrn}, samt ihren
Kindern, Frauen und Söhnen. \bibleverse{14} Da kam auf Jehasiel, den
Sohn Sacharias, des Sohnes Benajas, des Sohnes Jehiels, des Sohnes
Mattanjas, den Leviten aus den Kindern Asaphs, der Geist des
\textsc{Herrn} mitten in der Gemeinde, und er sprach: \bibleverse{15}
Merket auf, ganz Juda und ihr Einwohner von Jerusalem und du, König
Josaphat: So spricht der \textsc{Herr} zu euch: Ihr sollt euch nicht
fürchten, noch vor diesem großen Haufen verzagen; denn der Kampf ist
nicht eure Sache, sondern Gottes! \bibleverse{16} Morgen sollt ihr gegen
sie hinabziehen. Siehe, sie kommen auf der Steige Ziz herauf, und ihr
werdet sie antreffen am Ende des Tales, vor der Wüste Jeruel.
\bibleverse{17} Aber es ist nicht an euch, daselbst zu streiten. Tretet
nur hin und bleibet stehen und sehet das Heil des \textsc{Herrn}, mit
welchem er euch hilft! O Juda und Jerusalem, fürchtet euch nicht und
verzaget nicht! Morgen ziehet aus wider sie, der \textsc{Herr} ist mit
euch! \bibleverse{18} Da beugte sich Josaphat mit seinem Angesicht zur
Erde, und ganz Juda und die Einwohner von Jerusalem fielen vor dem
\textsc{Herrn} nieder und beteten den \textsc{Herrn} an. \bibleverse{19}
Und die Leviten von den Söhnen der Kahatiter und von den Söhnen der
Korahiter machten sich auf, den \textsc{Herrn}, den Gott Israels, hoch
zu loben mit lauter Stimme. \bibleverse{20} Und sie machten sich am
Morgen früh auf und zogen nach der Wüste Tekoa. Und als sie auszogen,
trat Josaphat hin und sprach: Höret mir zu, Juda und ihr Einwohner von
Jerusalem: Vertrauet auf den \textsc{Herrn}, euren Gott, so könnt ihr
getrost sein, und glaubet seinen Propheten, so werdet ihr Glück haben!
\bibleverse{21} Und er beriet sich mit dem Volk und stellte die, welche
in heiligem Schmuck dem \textsc{Herrn} singen und ihn preisen sollten,
im Zug vor die Gerüsteten hin, um zu singen: Danket dem \textsc{Herrn},
denn seine Güte währet ewiglich! \bibleverse{22} Und als sie anfingen
mit Jauchzen und Loben, ließ der \textsc{Herr} einen Hinterhalt kommen
über die Ammoniter, Moabiter und die vom Gebirge Seir, die wider Juda
gekommen waren, und sie wurden geschlagen. \bibleverse{23} Und die
Ammoniter und Moabiter stellten sich denen vom Gebirge Seir entgegen,
sie zu vernichten und zu vertilgen. Und als sie die vom Gebirge Seir
aufgerieben hatten, halfen sie selbst einander zur Vertilgung.
\bibleverse{24} Als aber Juda zu der Warte gegen die Wüste kam und sich
gegen den Haufen wenden wollte, siehe, da lagen die Leichen auf dem
Boden; es war niemand entronnen. \bibleverse{25} Und Josaphat kam mit
seinem Volk, um unter ihnen Beute zu machen, und sie fanden dort eine
Menge Fahrhabe und Kleider und kostbare Geräte, und sie raubten so viel,
daß sie es nicht tragen konnten. Und sie plünderten drei Tage lang, weil
so viel vorhanden war. \bibleverse{26} Aber am vierten Tage kamen sie
zusammen im ``Lobetal''; denn daselbst lobten sie den \textsc{Herrn}.
Daher heißt jener Ort Lobetal bis auf diesen Tag. \bibleverse{27}
Darnach kehrte die ganze Mannschaft von Juda und Jerusalem wieder um,
und Josaphat an ihrer Spitze, um mit Freuden gen Jerusalem zu ziehen;
denn der \textsc{Herr} hatte sie durch ihre Feinde erfreut.
\bibleverse{28} Und sie zogen zu Jerusalem ein unter Psalter und Harfen
und mit Trompetenklang, zum Haus des \textsc{Herrn}. \bibleverse{29} Und
der Schrecken Gottes kam über alle Königreiche der Länder, als sie
hörten, daß der \textsc{Herr} wider die Feinde Israels gestritten hatte.
\bibleverse{30} So blieb denn Josaphats Regierung ungestört, und sein
Gott gab ihm Ruhe ringsum. \bibleverse{31} Und Josaphat regierte über
Juda. Mit fünfunddreißig Jahren war er König geworden, und er regierte
fünfundzwanzig Jahre zu Jerusalem. Seine Mutter hieß Asuba, eine Tochter
Silhis. \bibleverse{32} Und er wandelte in dem Wege seines Vaters Asa
und wich nicht davon, sondern tat, was dem \textsc{Herrn} wohlgefiel.
\bibleverse{33} Nur die Höhen wurden nicht abgetan, denn das Volk hatte
sein Herz noch nicht dem Gott ihrer Väter zugewandt. \bibleverse{34} Die
übrigen Geschichten Josaphats aber, die früheren und die späteren,
siehe, die sind aufgezeichnet in den Geschichten Jehus, des Sohnes
Hananis, die er in das Buch der Könige von Israel geschrieben hat.
\bibleverse{35} Darnach verbündete sich Josaphat, der König von Juda,
mit Ahasia, dem König von Israel, welcher in seinem Tun gottlos war.
\bibleverse{36} Und zwar verband er sich mit ihm, um Schiffe zu bauen,
die nach Tarsis fahren sollten; und sie machten die Schiffe zu
Ezjon-Geber. \bibleverse{37} Aber Elieser, der Sohn Dodavahus von
Marescha, weissagte wider Josaphat und sprach: Weil du dich mit Ahasia
verbunden hast, so hat der \textsc{Herr} deine Werke zerrissen! Und die
Schiffe scheiterten wirklich und konnten nicht nach Tarsis fahren.

\hypertarget{section-20}{%
\section{21}\label{section-20}}

\bibleverse{1} Und Josaphat legte sich zu seinen Vätern und ward
begraben bei seinen Vätern in der Stadt Davids, und Jehoram, sein Sohn,
ward König an seiner Statt. \bibleverse{2} Und er hatte Brüder, Söhne
Josaphats, nämlich Asarja, Jechiel und Sacharjahu, Asarjahu, Michael und
Sephatjahu. Diese alle waren Söhne Josaphats, des Königs von Juda.
\bibleverse{3} Und ihr Vater machte ihnen reiche Geschenke von Silber,
Gold und Kleinodien und gab ihnen feste Städte in Juda. Aber das
Königreich gab er Jehoram, denn er war der Erstgeborene. \bibleverse{4}
Als aber Jehoram das Königreich seines Vaters übernommen hatte und
mächtig geworden war, tötete er alle seine Brüder mit dem Schwert; dazu
auch etliche von den Fürsten Israels. \bibleverse{5} Zweiunddreißig
Jahre alt war Jehoram, als er König ward, und regierte acht Jahre lang
zu Jerusalem; \bibleverse{6} und er wandelte in dem Wege der Könige von
Israel, wie das Haus Ahabs getan hatte; denn er hatte die Tochter Ahabs
zur Frau. Und er tat, was böse war in den Augen des \textsc{Herrn}.
\bibleverse{7} Aber der \textsc{Herr} wollte das Haus Davids nicht
verderben, um des Bundes willen, welchen er mit David gemacht, und weil
er ihm verheißen hatte, daß er ihm und seinen Kindern eine Leuchte geben
werde immerdar. \bibleverse{8} Zu seiner Zeit fielen die Edomiter von
Juda ab und setzten einen König über sich. \bibleverse{9} Da zog Jehoram
hinüber mit seinen Obersten und allen Wagen; und er machte sich auf bei
Nacht und schlug die Edomiter, die ihn und die Obersten der Wagen
umzingelten. \bibleverse{10} Aber die Edomiter fielen von Juda ab bis
auf diesen Tag. Zu jener Zeit fiel auch Libna von ihm ab; denn er
verließ den \textsc{Herrn}, den Gott seiner Väter. \bibleverse{11} Auch
machte er Höhen auf den Bergen Judas und verführte die Bewohner
Jerusalems zur Abgötterei und brachte Juda auf Abwege. \bibleverse{12}
Es kam aber ein Schreiben zu ihm von dem Propheten Elia; das lautete
also: So spricht der \textsc{Herr}, der Gott deines Vaters David: Weil
du nicht gewandelt bist in den Wegen deines Vaters Josaphat, noch in den
Wegen Asas, des Königs von Juda, \bibleverse{13} sondern in dem Wege der
Könige von Israel, und verführst Juda und die Bewohner Jerusalems zu
Abgötterei, gleichwie das Haus Ahabs Abgötterei einführte, und hast dazu
deine Brüder aus deines Vaters Haus erwürgt, die besser waren als du;
\bibleverse{14} siehe, darum wird der \textsc{Herr} eine schwere Plage
über dein Volk verhängen, auch über deine Kinder, deine Frauen und alle
deine Habe. \bibleverse{15} Du aber wirst viel zu leiden haben an einer
Krankheit in deinen Eingeweiden, bis deine Eingeweide infolge dieser
Krankheit nach und nach heraustreten werden. \bibleverse{16} Also
erweckte der \textsc{Herr} wider Jehoram den Geist der Philister und
Araber, welche zur Seite der Mohren wohnen; \bibleverse{17} die zogen
herauf gegen Juda und brachen ein und führten alle Habe hinweg, die im
Hause des Königs vorhanden war; dazu seine Söhne und seine Frauen, so
daß ihm kein Sohn übrigblieb, außer Joahas, sein jüngster Sohn.
\bibleverse{18} Und nach alledem schlug ihn der \textsc{Herr} in seinen
Eingeweiden mit einer unheilbaren Krankheit. \bibleverse{19} Und solches
währte zwei Jahre. Als aber nach zwei Jahren seine Eingeweide austraten
infolge seiner Krankheit, starb er unter argen Schmerzen. Und sein Volk
machte ihm zu Ehren kein Feuer, wie man seinen Vätern getan hatte.
\bibleverse{20} Mit zweiunddreißig Jahren war er König geworden, und er
regierte acht Jahre lang zu Jerusalem und ging unbeliebt dahin, und man
begrub ihn in der Stadt Davids, aber nicht in den Gräbern der Könige.

\hypertarget{section-21}{%
\section{22}\label{section-21}}

\bibleverse{1} Und die Einwohner von Jerusalem machten Ahasia, seinen
jüngsten Sohn, zum König an seiner Statt; denn die Truppe, welche mit
den Arabern in das Lager gekommen war, hatte alle älteren getötet. Also
ward Ahasia König, der Sohn Jehorams, des Königs von Juda.
\bibleverse{2} Zweiundzwanzig Jahre alt war Ahasia, als er König ward,
und regierte ein Jahr lang zu Jerusalem. Seine Mutter hieß Atalia, eine
Tochter Omris. \bibleverse{3} Und er wandelte auch in den Wegen des
Hauses Ahabs; denn seine Mutter beriet ihn so, daß er Böses tat.
\bibleverse{4} Und so tat er, was böse war in den Augen des
\textsc{Herrn}, wie das Haus Ahabs; denn nach seines Vaters Tod waren
sie seine Ratgeber, zu seinem Verderben. \bibleverse{5} Und er wandelte
nach ihrem Rat und zog mit Joram, dem Sohn Ahabs, dem König von Israel,
in den Krieg wider Hasael, den König von Syrien, gen Ramot in Gilead.
\bibleverse{6} Aber die Syrer trafen den Joram, so daß er umkehrte, um
sich zu Jesreel heilen zu lassen; denn er hatte Wunden, die ihm zu Rama
geschlagen worden, als er mit Hasael, dem König von Syrien stritt. Und
Asaria, der Sohn Jehorams, der König von Juda, zog hinab, um Joram, den
Sohn Ahabs, in Jesreel zu besuchen, weil er krank lag. \bibleverse{7}
Und das war von Gott, zu Ahasias Untergang, daß er zu Joram ging; denn
als er kam, zog er mit Joram aus wider Jehu, den Sohn Nimsis, welchen
der \textsc{Herr} gesalbt hatte, das Haus Ahabs auszurotten.
\bibleverse{8} Da nun Jehu am Hause Ahabs Strafe übte, traf er die
Fürsten Judas und die Neffen Ahasias, welche Ahasia dienten, und brachte
sie um. \bibleverse{9} Er suchte auch Ahasia; und man fing ihn zu
Samaria, wo er sich verborgen hatte, und brachte ihn zu Jehu; der tötete
ihn. Und man begrub ihn, denn sie sprachen: Er ist Josaphats Sohn, der
von ganzem Herzen den \textsc{Herrn} gesucht hat! Und es war niemand
mehr aus dem Hause Ahasias, der stark genug gewesen wäre zum Regieren.
\bibleverse{10} Als aber Atalia, die Mutter Ahasias, sah, daß ihr Sohn
tot war, machte sie sich auf und brachte allen königlichen Samen im
Hause Judas um. \bibleverse{11} Aber Joschabat, die Tochter des Königs,
nahm Joas, den Sohn Ahasias, und stahl ihn weg mitten aus den Söhnen des
Königs, die getötet wurden, und tat ihn samt seiner Amme in eine
Schlafkammer. Also verbarg ihn Joschabat, die Tochter des Königs
Jehoram, das Weib des Priesters Jojada (denn sie war Ahasias Schwester,
vor Atalia, so daß er nicht getötet wurde. \bibleverse{12} Und er war
mit ihnen im Hause Gottes sechs Jahre lang verborgen, solange Atalia
über das Land regierte.

\hypertarget{section-22}{%
\section{23}\label{section-22}}

\bibleverse{1} Aber im siebenten Jahre ermannte sich Jojada und nahm die
Obersten der Hundertschaften, nämlich Asarja, den Sohn Jerohams, Ismael,
den Sohn Johanans, Asarja, den Sohn Obeds, Maaseja, den Sohn Adajas, und
Elischaphat, den Sohn Sichris, zu Verbündeten. \bibleverse{2} Die zogen
umher und brachten die Leviten zusammen aus allen Städten Judas und die
Familienhäupter von Israel, und sie kamen nach Jerusalem. \bibleverse{3}
Und diese ganze Gemeinde machte im Hause Gottes einen Bund mit dem
König; und er sprach zu ihnen: Siehe, des Königs Sohn soll König sein,
wie der \textsc{Herr} betreffs der Söhne Davids gesagt hat!
\bibleverse{4} So sollt ihr nun also tun: Ein Drittel von euch Priestern
und Leviten, die ihr am Sabbat antretet, sollt als Türhüter an der
Schwelle dienen; \bibleverse{5} und ein Drittel im Hause des Königs und
ein Drittel am Grundtor, während alles Volk in den Vorhöfen am Hause des
\textsc{Herrn} ist \bibleverse{6} es soll aber niemand in das Haus des
\textsc{Herrn} gehen, nur die Priester und die Leviten dürfen
hineingehen, denn sie sind heilig; aber alles Volk soll die Vorschriften
des \textsc{Herrn} befolgen; \bibleverse{7} und die Leviten sollen den
König umringen, ein jeder mit den Waffen in der Hand; und wer ins Haus
eindringt, soll getötet werden; sie aber sollen den König umgeben, wenn
er aus und eingeht. \bibleverse{8} Und die Leviten und ganz Juda
handelten genau nach dem Befehl des Priesters Jojada; und ein jeder nahm
seine Leute, die am Sabbat antraten, samt denen, die am Sabbat abtraten.
Denn der Priester Jojada entließ die Abteilungen nicht. \bibleverse{9}
Und der Priester Jojada gab den Obersten der Hundertschaften Speere und
Schilde und die Tartschen des Königs David, die im Hause Gottes waren,
\bibleverse{10} und stellte alles Volk, einen jeden mit der Waffe in der
Hand, von der rechten Seite des Hauses bis zur linken Seite beim Altar
und beim Hause um den König her auf. \bibleverse{11} Da brachten sie den
Sohn des Königs hervor und setzten ihm die Krone auf und gaben ihm das
Zeugnis und machten ihn zum König. Und Jojada samt seinen Söhnen salbten
ihn und sprachen: Es lebe der König! \bibleverse{12} Als aber Atalia das
Geschrei des Volkes hörte, das zulief und den König lobte, kam sie zu
dem Volk im Hause des \textsc{Herrn}. \bibleverse{13} Und sie schaute,
und siehe, der König stand an seiner Säule im Eingang, und die Obersten
und Trompeter bei dem König, und alles Volk des Landes war fröhlich und
stieß in die Trompeten, und die Sänger sangen zu den Saiteninstrumenten
und verkündigten sein Lob. Da zerriß Atalia ihre Kleider und rief:
Aufruhr, Aufruhr! \bibleverse{14} Aber Jojada, der Priester, ließ die
Obersten über die Hundertschaften, welche über das Heer gesetzt waren,
hinausgehen und sprach zu ihnen: Führet sie hinaus, außerhalb der
Reihen, und wer ihr nachfolgt, den soll man mit dem Schwerte töten!
(denn der Priester sprach: Ihr sollt sie nicht töten im Hause des
\textsc{Herrn}!). \bibleverse{15} Und sie legten Hand an sie. Und als
sie zum Eingang des Roßtors am Hause des Königs kam, tötete man sie
daselbst. \bibleverse{16} Und Jojada machte einen Bund mit dem ganzen
Volk und mit dem König, daß sie des \textsc{Herrn} Volk sein wollten.
\bibleverse{17} Da ging alles Volk zum Hause Baals und zerstörte es, und
sie zerbrachen seine Altäre, seine Bilder, und erwürgten Mattan, den
Priester Baals, vor den Altären. \bibleverse{18} Und Jojada legte die
Ämter im Hause des \textsc{Herrn} in die Hand der Priester und Leviten,
die David über das Haus des \textsc{Herrn} verteilt hatte, um dem
\textsc{Herrn} Brandopfer darzubringen, wie im Gesetze Moses geschrieben
steht, mit Freuden und Gesang, nach der Verordnung Davids.
\bibleverse{19} Und er stellte Torhüter an die Tore des Hauses des
\textsc{Herrn}, damit niemand hineinkäme, der irgendwie unrein wäre.
\bibleverse{20} Und er nahm die Obersten über die Hundertschaften und
die Vornehmen und Herrscher über das Volk, auch alles Volk des Landes,
und führte den König vom Hause des \textsc{Herrn} hinab, und sie kamen
durch das obere Tor zum Hause des Königs und setzten den König auf den
königlichen Thron. \bibleverse{21} Und alles Volk des Landes war
fröhlich, und die Stadt ward stille. Atalia aber hatten sie mit dem
Schwerte getötet.

\hypertarget{section-23}{%
\section{24}\label{section-23}}

\bibleverse{1} Joas war sieben Jahre alt, als er König ward, und
regierte vierzig Jahre lang zu Jerusalem. Seine Mutter hieß Zibja, von
Beerseba. \bibleverse{2} Und Joas tat, was recht war in den Augen des
\textsc{Herrn}, solange der Priester Jojada lebte. \bibleverse{3} Und
Jojada gab ihm zwei Frauen, und er zeugte Söhne und Töchter.
\bibleverse{4} Darnach nahm sich Joas vor, das Haus des \textsc{Herrn}
zu erneuern. \bibleverse{5} Und er versammelte die Priester und Leviten
und sprach zu ihnen: Ziehet aus zu den Städten Judas und sammelt Geld
aus ganz Israel, um das Haus eures Gottes jährlich auszubessern, und
beeilet euch damit! Aber die Leviten beeilten sich nicht. \bibleverse{6}
Da rief der König den Jojada, den Oberpriester, und sprach zu ihm: Warum
verlangst du nicht von den Leviten, daß sie von Juda und Jerusalem die
Steuer einbringen, welche Mose, der Knecht des \textsc{Herrn},
auferlegte und welche die Gemeinde Israel zur Hütte des Zeugnisses
brachte? \bibleverse{7} Denn die gottlose Atalia und ihre Söhne haben
das Haus des \textsc{Herrn} aufgebrochen und alle Heiligtümer, welche
zum Hause des \textsc{Herrn} gehören, den Baalen zugeeignet.
\bibleverse{8} Da befahl der König, daß man eine Lade mache und sie
außerhalb des Tores am Hause des \textsc{Herrn} aufstelle.
\bibleverse{9} Und man ließ in Juda und Jerusalem ausrufen, daß man dem
\textsc{Herrn} die Steuer bringe, welche Mose, der Knecht Gottes, in der
Wüste Israel auferlegt hatte. \bibleverse{10} Da freuten sich alle
Obersten und das ganze Volk und brachten sie und warfen sie in die Lade,
bis sie es alle getan hatten. \bibleverse{11} Und wenn es Zeit war, die
Lade durch die Leviten zu der königlichen Behörde zu bringen, und wenn
man sah, daß viel Geld darin war, so kamen der Schreiber des Königs und
der Verordnete des Oberpriesters und leerten die Lade und trugen sie
wieder an ihren Ort. Also taten sie von Zeit zu Zeit, so daß sie viel
Geld zusammenbrachten. \bibleverse{12} Und der König und Jojada gaben es
denen, welche das Werk des Dienstes am Hause des \textsc{Herrn}
betrieben; die dingten Steinmetzen und Zimmerleute, das Haus des
\textsc{Herrn} zu erneuern, auch Meister in Eisen und Erz, das Haus des
\textsc{Herrn} auszubessern. \bibleverse{13} Und die Handwerker
arbeiteten, so daß die Verbesserung des Werkes unter ihrer Hand zunahm,
und sie stellten das Haus Gottes wieder in seinen rechten Stand und
machten es fest. \bibleverse{14} Und als sie es vollendet hatten,
brachten sie das übrige Geld vor den König und vor Jojada; davon machte
man Geräte für das Haus des \textsc{Herrn}, Geräte für den Dienst und
für die Opfer, Schalen und goldene und silberne Geräte. Und sie opferten
Brandopfer im Hause des \textsc{Herrn} immerdar, solange Jojada lebte.
\bibleverse{15} Jojada aber ward alt und lebenssatt und starb; er war
bei seinem Tod hundertdreißig Jahre alt. \bibleverse{16} Und sie
begruben ihn in der Stadt Davids, bei den Königen, weil er an Israel
wohlgetan hatte, auch an Gott und an seinem Hause. \bibleverse{17} Aber
nach Jojadas Tod kamen die Obersten in Juda und huldigten dem König; da
hörte der König auf sie. \bibleverse{18} Und sie verließen das Haus des
\textsc{Herrn}, des Gottes ihrer Väter, und dienten den Ascheren und
Götzenbildern. Da kam der Zorn Gottes über Juda und Jerusalem um dieser
ihrer Schuld willen. \bibleverse{19} Er sandte aber Propheten zu ihnen,
um sie zum \textsc{Herrn} zurückzubringen; und diese vermahnten sie
ernstlich, aber sie hörten nicht darauf. \bibleverse{20} Da kam der
Geist Gottes über Sacharja, den Sohn Jojadas, des Priesters, so daß er
wider das Volk auftrat und zu ihnen sprach: So spricht Gott: Warum
übertretet ihr die Gebote des \textsc{Herrn}? Das bringt euch kein
Glück, denn weil ihr den \textsc{Herrn} verlassen habt, wird er euch
auch verlassen! \bibleverse{21} Aber sie machten eine Verschwörung wider
ihn und steinigten ihn auf Befehl des Königs im Vorhofe am Hause des
\textsc{Herrn}. \bibleverse{22} Und der König Joas gedachte nicht an die
Liebe, die sein Vater Jojada ihm erwiesen, sondern brachte dessen Sohn
um. Als der aber starb, sprach er: Der \textsc{Herr} wird es sehen und
richten! \bibleverse{23} Und um die Jahreswende zog das Heer der Syrer
wider ihn herauf, und sie kamen nach Juda und Jerusalem und vertilgten
alle Obersten des Volkes aus dem Volk und sandten alle ihre Habe zu dem
König von Damaskus. \bibleverse{24} Denn obwohl das Heer der Syrer nur
aus wenig Leuten bestand, gab doch der \textsc{Herr} ein sehr großes
Heer in ihre Hand, weil jene den \textsc{Herrn}, den Gott ihrer Väter,
verlassen hatten. Also vollzogen sie das Strafgericht an Joas.
\bibleverse{25} Und als sie von ihm abzogen, wobei sie ihn schwer
verwundet zurückließen, machten seine Knechte eine Verschwörung wider
ihn, wegen der Blutschuld an den Söhnen des Priesters Jojada, und
töteten ihn auf seinem Bette; und er starb, und man begrub ihn in der
Stadt Davids, aber nicht in den Gräbern der Könige. \bibleverse{26} Die
sich aber gegen ihn verschworen hatten, waren diese: Sabad, der Sohn der
Ammoniterin Simeat, und Josabad, der Sohn der Moabiterin Simrit.
\bibleverse{27} Aber seine Söhne und die Summe, die ihm auferlegt wurde,
und die Wiederherstellung des Hauses Gottes, siehe, das ist beschrieben
in der Erklärung des Buches der Könige. Und Amazia, sein Sohn, ward
König an seiner Statt.

\hypertarget{section-24}{%
\section{25}\label{section-24}}

\bibleverse{1} Fünfundzwanzig Jahre alt war Amazia, als er König ward,
und regierte neunundzwanzig Jahre lang zu Jerusalem. Seine Mutter hieß
Joaddan, von Jerusalem. \bibleverse{2} Und er tat, was recht war in den
Augen des \textsc{Herrn}, doch nicht von ganzem Herzen. \bibleverse{3}
Als ihm nun das Königreich gesichert war, tötete er seine Knechte,
welche seinen königlichen Vater erschlagen hatten. \bibleverse{4} Aber
ihre Söhne tötete er nicht, sondern tat, wie geschrieben steht im
Gesetzbuche Moses, da der \textsc{Herr} gebietet und spricht: Die Väter
sollen nicht für die Söhne und die Söhne nicht für die Väter sterben,
sondern ein jeder soll um seiner eigenen Sünde willen sterben!
\bibleverse{5} Und Amazia brachte Juda zusammen und stellte sie auf nach
den Vaterhäusern, nach den Obersten über die Tausendschaften und über
die Hundertschaften, von ganz Juda und Benjamin, und musterte sie, von
zwanzig Jahren an und darüber, und fand ihrer 300000 Auserlesene, die in
den Krieg ziehen und Speer und Schild handhaben konnten. \bibleverse{6}
Dazu dingte er aus Israel 100000 starke Kriegsleute um hundert Talente
Silber. \bibleverse{7} Aber ein Mann Gottes kam zu ihm und sprach: O
König, laß das Heer Israels nicht mit dir kommen; denn der \textsc{Herr}
ist nicht mit Israel, er ist nicht mit den Kindern Ephraim;
\bibleverse{8} sondern gehe du hin und mache, daß du selbst stark genug
bist zum Kampf! Gott möchte dich sonst zu Fall bringen vor dem Feind;
denn bei Gott steht die Kraft, zu helfen und zu stürzen. \bibleverse{9}
Amazia sprach zu dem Manne Gottes: Was wird dann aber aus den hundert
Talenten, die ich den israelitischen Truppen gegeben habe? Der Mann
Gottes sprach: Der \textsc{Herr} hat dir noch mehr zu geben als nur das!
\bibleverse{10} Da sonderte Amazia seine Leute ab von den Truppen, die
aus Ephraim zu ihm gekommen waren, und ließ sie an ihren Ort hingehen.
Da ergrimmte ihr Zorn sehr wider Juda, und sie kehrten in glühendem Zorn
wieder heim. \bibleverse{11} Amazia aber ermannte sich und führte sein
Volk aus und zog in das Salztal und schlug von den Kindern Seir
zehntausend Mann. \bibleverse{12} Und die Kinder Juda fingen ihrer
zehntausend lebendig, führten sie auf eine Felsenspitze und stürzten sie
von der Felsenspitze hinunter, daß sie alle zerschmettert wurden.
\bibleverse{13} Aber die Kriegsleute, welche Amazia zurückgeschickt
hatte, daß sie nicht mit ihm in den Krieg zögen, fielen in die Städte
Judas ein, von Samaria bis gen Beth-Horon, erschlugen daselbst
dreitausend Mann und machten große Beute. \bibleverse{14} Als aber
Amazia von der Schlacht der Edomiter heimkehrte, brachte er die Götter
der Kinder von Seir mit und stellte sie für sich als Götter auf und
betete vor ihnen an und räucherte ihnen. \bibleverse{15} Da entbrannte
der Zorn des \textsc{Herrn} über Amazia; und er sandte einen Propheten
zu ihm, der sprach zu ihm: Warum suchst du die Götter des Volkes, die
ihr Volk nicht von deiner Hand errettet haben? \bibleverse{16} Als
dieser aber so zu ihm redete, sprach Amazia zu ihm: Hat man dich zum
Ratgeber des Königs gemacht? Halt inne; warum willst du geschlagen sein?
Da hielt der Prophet inne und sprach: Ich merke wohl, daß Gott
beschlossen hat, dich zu verderben, weil du solches getan und meinem Rat
nicht gehorcht hast! \bibleverse{17} Und Amazia, der König von Juda,
beriet sich und sandte hin zu Joas, dem Sohn des Joahas, des Sohnes
Jehus, dem König von Israel, und ließ ihm sagen: Wir wollen einander ins
Angesicht sehen! \bibleverse{18} Aber Joas, der König von Israel, sandte
zu Amazia, dem König von Juda, und ließ ihm sagen: Der Dornstrauch am
Libanon sandte zur Zeder am Libanon und ließ ihr sagen: Gib deine
Tochter meinem Sohn zum Weibe! Aber das Wild am Libanon lief über den
Dornstrauch und zertrat ihn. \bibleverse{19} Du aber denkst, du habest
die Edomiter geschlagen, und dein Herz verführt dich zum Stolz. Bleibe
du jetzt daheim! Warum willst du das Schicksal herausfordern, daß du zu
Fall kommst und Juda mit dir? \bibleverse{20} Aber Amazia gehorchte
nicht; denn es war von Gott so gefügt, um sie in die Hand der Feinde zu
geben, weil sie die Götter der Edomiter gesucht hatten. \bibleverse{21}
Da zog Joas, der König von Israel, herauf, und sie sahen sich ins
Angesicht, er und Amazia, der König von Juda, zu Beth-Semes, das zu Juda
gehört. \bibleverse{22} Aber Juda ward vor Israel geschlagen, und sie
flohen, ein jeder in seine Hütte. \bibleverse{23} Amazia aber, den König
von Juda, den Sohn des Joas, des Sohnes des Joahas, fing Joas, der König
von Israel, zu Beth-Semes und brachte ihn gen Jerusalem; und er riß die
Mauer von Jerusalem ein, vom Tor Ephraim bis zum Ecktor, auf vierhundert
Ellen Länge. \bibleverse{24} Und er nahm alles Gold und Silber und alle
Geräte, die im Hause Gottes bei Obed-Edom vorhanden waren, auch die
Schätze im Hause des Königs, dazu Geiseln; dann kehrte er nach Samaria
zurück. \bibleverse{25} Aber Amazia, der Sohn des Joas, der König von
Juda, lebte nach dem Tode Joas', des Sohnes Joahas', des Königs von
Israel, noch fünfzehn Jahre lang. \bibleverse{26} Die übrigen
Geschichten Amazias aber, die früheren und die späteren, siehe, sind die
nicht aufgezeichnet im Buch der Könige von Juda und Israel?
\bibleverse{27} Und seit der Zeit, da Amazia von dem \textsc{Herrn}
abwich, bestand zu Jerusalem eine Verschwörung gegen ihn. Er aber floh
nach Lachis; da sandten sie ihm nach gen Lachis und töteten ihn
daselbst. \bibleverse{28} Man brachte ihn aber auf Pferden und begrub
ihn bei seinen Vätern in der Hauptstadt Judas.

\hypertarget{section-25}{%
\section{26}\label{section-25}}

\bibleverse{1} Da nahm das ganze Volk Juda den Ussia, der sechzehn Jahre
alt war, und machte ihn zum König an Stelle seines Vaters Amazia.
\bibleverse{2} Derselbe baute Elot und brachte es wieder an Juda,
nachdem der König sich zu seinen Vätern gelegt hatte. \bibleverse{3}
Sechzehn Jahre alt war Ussia, als er König ward, und regierte
zweiundfünfzig Jahre lang zu Jerusalem. Und seine Mutter hieß Jechalia,
von Jerusalem. \bibleverse{4} Und er tat, was recht war in den Augen des
\textsc{Herrn}, ganz wie sein Vater Amazia getan hatte. \bibleverse{5}
Und er suchte Gott, solange Sacharja lebte, der ihn in der Furcht Gottes
unterwies. Und solange er den \textsc{Herrn} suchte, ließ Gott es ihm
gelingen. \bibleverse{6} Denn er zog aus und stritt wider die Philister
und riß die Mauern von Gat und die Mauern von Jabne und die Mauern von
Asdod nieder und baute Städte bei Asdod und unter den Philistern.
\bibleverse{7} Denn Gott half ihm wider die Philister, wider die Araber,
die zu Gur-Baal wohnten, und wider die Meuniter. \bibleverse{8} Und die
Ammoniter zahlten dem Ussia Tribut; und sein Ruhm verbreitete sich bis
nach Ägypten hin; denn er ward sehr stark. \bibleverse{9} Und Ussia
baute Türme zu Jerusalem, über das Ecktor und über das Taltor und über
den Winkel und befestigte sie. \bibleverse{10} Er baute auch Türme in
der Wüste und grub viele Brunnen; denn er hatte viel Vieh in der
Niederung und in der Ebene, auch Ackerleute und Weingärtner auf den
Bergen und in Karmel; denn er liebte den Ackerbau. \bibleverse{11} Ussia
hatte auch ein kriegstüchtiges Heer, welches truppenweise zu Felde zog,
in der Anzahl, wie sie gemustert worden durch Jehiel, den Schreiber, und
Maaseja, den Amtmann, unter der Leitung Hananjas, eines königlichen
Obersten. \bibleverse{12} Die Gesamtzahl der Familienhäupter der
kriegstüchtigen Mannschaft betrug 2600. \bibleverse{13} Und unter ihrer
Hand war das Kriegsheer, 307500 kriegstüchtige Leute, stark genug, um
dem König wider die Feinde zu helfen. \bibleverse{14} Und Ussia versah
das ganze Heer mit Schilden, Speeren, Helmen, Panzern, Bogen und
Schleudersteinen. \bibleverse{15} Er machte in Jerusalem auch Maschinen,
von Künstlern erfunden, die auf Türmen und Zinnen aufgestellt wurden, um
mit Pfeilen und großen Steinen zu schießen. Also verbreitete sich sein
Ruhm weithin, weil er sich so vorzüglich zu helfen wußte, bis daß er
mächtig ward. \bibleverse{16} Als er sich aber mächtig fühlte, überhob
sich sein Herz zu seinem Verderben, und er vergriff sich an dem
\textsc{Herrn}, seinem Gott, indem er in den Tempel des \textsc{Herrn}
ging, um auf dem Räucheraltar zu räuchern. \bibleverse{17} Aber der
Priester Asaria ging ihm nach, und achtzig Priester des \textsc{Herrn}
mit ihm, wackere Leute; \bibleverse{18} die traten dem König Ussia
entgegen und sprachen zu ihm: Ussia, es steht nicht dir zu, dem
\textsc{Herrn} zu räuchern, sondern den Priestern, den Söhnen Aarons,
die zum Räuchern geheiligt sind! Verlaß das Heiligtum, denn du hast dich
vergangen, und das bringt dir vor Gott, dem \textsc{Herrn}, keine Ehre!
\bibleverse{19} Da ward Ussia zornig, während er die Räucherpfanne in
seiner Hand hielt, um zu räuchern. Als er aber seinen Zorn wider die
Priester ausließ, brach der Aussatz an seiner Stirn aus, vor den
Priestern im Hause des \textsc{Herrn} beim Räucheraltar. \bibleverse{20}
Denn als sich der Oberpriester Asarja und alle Priester zu ihm
hinwandten, siehe, da war er aussätzig an seiner Stirn! Da jagten sie
ihn eilends hinaus; und auch er selbst machte sich schnell davon, weil
der \textsc{Herr} ihn geschlagen hatte. \bibleverse{21} Also war der
König Ussia aussätzig bis an den Tag seines Todes und wohnte als
Aussätziger in einem abgesonderten Hause; denn er war vom Hause des
\textsc{Herrn} ausgeschlossen, und sein Sohn Jotam stand dem Hause des
Königs vor und richtete das Volk des Landes. \bibleverse{22} Aber die
übrigen Geschichten Ussias, die früheren und die späteren, hat der
Prophet Jesaja, der Sohn des Amoz, aufgezeichnet. \bibleverse{23} Und
Ussia entschlief mit seinen Vätern, und sie begruben ihn bei seinen
Vätern auf dem Begräbnisacker der Könige; denn sie sprachen: Er ist
aussätzig! Und sein Sohn Jotam ward König an seiner Statt.

\hypertarget{section-26}{%
\section{27}\label{section-26}}

\bibleverse{1} Jotam war fünfundzwanzig Jahre alt, als er König ward,
und regierte sechzehn Jahre lang zu Jerusalem. Und seine Mutter hieß
Jerusa, eine Tochter Zadoks. \bibleverse{2} Und er tat, was recht war in
den Augen des \textsc{Herrn}, ganz wie sein Vater Ussia getan hatte, nur
daß er nicht in den Tempel des \textsc{Herrn} ging. Aber das Volk
handelte noch verderblich. \bibleverse{3} Er baute das obere Tor am
Hause des \textsc{Herrn}; auch an der Mauer des Ophel baute er viel.
\bibleverse{4} Er baute auch Städte auf dem Gebirge Juda; und in den
Wäldern baute er Burgen und Türme. \bibleverse{5} Und er stritt mit dem
König der Kinder Ammon und überwältigte sie, also daß ihm die Kinder
Ammon in jenem Jahr hundert Talente Silber und zehntausend Kor Weizen
und zehntausend Kor Gerste gaben. Solches entrichteten ihm die Kinder
Ammon auch im zweiten und dritten Jahr. \bibleverse{6} Also erstarkte
Jotam; denn er wandelte richtig vor dem \textsc{Herrn}, seinem Gott.
\bibleverse{7} Was aber mehr von Jotam zu sagen ist, und alle seine Wege
und Kriege, siehe, das ist aufgezeichnet im Buch der Könige von Israel
und Juda. \bibleverse{8} Mit fünfundzwanzig Jahren war er König geworden
und regierte sechzehn Jahre lang zu Jerusalem. \bibleverse{9} Und Jotam
entschlief mit seinen Vätern; und sie begruben ihn in der Stadt Davids;
und sein Sohn Ahas ward König an seiner Statt.

\hypertarget{section-27}{%
\section{28}\label{section-27}}

\bibleverse{1} Ahas war zwanzig Jahre alt, als er König ward, und
regierte sechzehn Jahre lang zu Jerusalem; aber er tat nicht, was recht
war in den Augen des \textsc{Herrn}, wie sein Vater David,
\bibleverse{2} sondern wandelte in den Wegen der Könige von Israel, und
machte sogar gegossene Bilder für die Baale. \bibleverse{3} Und er
räucherte im Tale des Sohnes Hinnoms und verbrannte seine Söhne mit
Feuer, nach den Greueln der Heiden, welche der \textsc{Herr} vor den
Kindern Israel vertrieben hatte. \bibleverse{4} Und er opferte und
räucherte auf den Höhen und auf den Hügeln und unter allen grünen
Bäumen. \bibleverse{5} Darum gab ihn der \textsc{Herr}, sein Gott, in
die Hand des Königs der Syrer, die ihn schlugen und von den Seinigen
eine große Menge hinwegführten und gen Damaskus brachten. Auch ward er
in die Hand des Königs von Israel gegeben, der brachte ihm eine große
Niederlage bei. \bibleverse{6} Denn Pekach, der Sohn Remaljas, machte in
Juda an einem Tage 120000 Mann nieder, lauter tapfere Leute, weil sie
den \textsc{Herrn}, den Gott ihrer Väter, verlassen hatten.
\bibleverse{7} Zudem erschlug Sichri, ein ephraimitischer Held, Maaseja,
den Sohn des Königs, und Asrikan, den Haushofmeister, und Elkana, den
Nächsten nach dem König. \bibleverse{8} Und die Kinder Israel führten
von ihren Brüdern 200000 Frauen, Söhne und Töchter gefangen hinweg und
machten dazu große Beute und brachten die Beute nach Samaria.
\bibleverse{9} Es war aber daselbst ein Prophet des \textsc{Herrn}, der
hieß Oded; der ging hinaus, dem Heer entgegen, das gen Samaria kam, und
sprach zu ihm: Siehe, weil der \textsc{Herr}, der Gott eurer Väter, über
Juda zornig ist, hat er sie in eure Hand gegeben; und ihr habt sie mit
einer Wut, die zum Himmel schreit, niedergemetzelt. \bibleverse{10} Und
nun gedenket ihr die Kinder Judas und Jerusalems so niederzutreten, daß
sie eure Knechte und Mägde werden sollen? Was habt ihr denn anders als
Schulden bei dem \textsc{Herrn}, eurem Gott? \bibleverse{11} So
gehorchet mir nun und schicket die Gefangenen wieder zurück, die ihr von
euren Brüdern weggeführt habt; denn der grimmige Zorn des \textsc{Herrn}
lastet auf euch! \bibleverse{12} Da standen einige Männer von den
Häuptern der Kinder Ephraim auf, nämlich Asaja, der Sohn Johanans,
Berechja, der Sohn Messilemots, Jehiskia, der Sohn Sallums, und Amasa,
der Sohn Hadlais, denen entgegen, welche vom Feldzug zurückkehrten,
\bibleverse{13} und sprachen zu ihnen: Ihr sollt die Gefangenen nicht
hierher bringen, denn das würde uns zur Schuld vor dem \textsc{Herrn}
gereichen! Ihr gedenket unsre Sünde und Schuld zu vermehren; und doch
ist unsre Schuld schon groß genug und der grimmige Zorn über Israel!
\bibleverse{14} Da ließen die Krieger die Gefangenen und die Beute vor
den Obersten und der ganzen Gemeinde frei. \bibleverse{15} Die Männer
aber, die mit Namen genannt sind, machten sich auf und nahmen sich der
Gefangenen an und bekleideten alle, die unter ihnen nackt waren, mit
Kleidern von der Beute, zogen ihnen Schuhe an und gaben ihnen zu essen
und zu trinken und salbten sie und führten alle, die zu schwach waren,
auf Eseln und brachten sie gen Jericho, zur Palmenstadt, in die Nähe
ihrer Brüder, und kehrten dann wieder nach Samaria zurück.
\bibleverse{16} Zu jener Zeit sandte der König Ahas Botschaft zu den
Königen von Assyrien, daß sie ihm helfen sollten. \bibleverse{17} Auch
die Edomiter waren wieder gekommen und hatten Juda geschlagen und
Gefangene gemacht. \bibleverse{18} Dazu fielen die Philister in die
Städte der Ebene und in den Süden von Juda ein und eroberten Beth-Semes,
Ajalon, Gederot und Socho mit seinen Dörfern, Timna mit seinen Dörfern
und Gimso mit seinen Dörfern und wohnten darin. \bibleverse{19} Denn der
\textsc{Herr} demütigte Juda, um Ahas willen, des Königs von Israel,
weil er keine Zucht übte in Juda und sich an dem \textsc{Herrn} schwer
verging. \bibleverse{20} Es kam nun zwar Tiglat-Pilneser, der König von
Assyrien, zu ihm; aber er bedrängte ihn und stärkte ihn nicht.
\bibleverse{21} Denn Ahas beraubte das Haus des \textsc{Herrn} und das
Haus des Königs und die Fürsten und gab alles dem König von Assyrien;
aber es half ihm nichts. \bibleverse{22} Ja, zu der Zeit, als er
bedrängt ward, versündigte er sich an dem \textsc{Herrn}, er, der König
Ahas! \bibleverse{23} Er opferte nämlich den Göttern von Damaskus, die
ihn geschlagen hatten, indem er sprach: Weil die Götter der Könige von
Syrien ihnen helfen, so will ich ihnen opfern, damit sie mir auch
helfen! Aber sie gereichten ihm und ganz Israel zum Fall.
\bibleverse{24} Und Ahas nahm die Geräte des Hauses Gottes weg und
zerbrach die Geräte des Hauses Gottes und verschloß die Türen am Hause
des \textsc{Herrn} und machte sich Altäre an allen Ecken zu Jerusalem.
\bibleverse{25} Und in jeder einzelnen Stadt Judas machte er Höhen, um
andern Göttern zu räuchern, und reizte den \textsc{Herrn}, den Gott
seiner Väter, zum Zorn. \bibleverse{26} Seine übrigen Geschichten aber
und alle seine Wege, die früheren und die späteren, die sind
aufgezeichnet im Buch der Könige von Juda und Israel. \bibleverse{27}
Und Ahas legte sich zu seinen Vätern, und man begrub ihn in der Stadt,
zu Jerusalem; denn man bestattete ihn nicht in den Gräbern der Könige
Israels. Und sein Sohn Hiskia ward König an seiner Statt.

\hypertarget{section-28}{%
\section{29}\label{section-28}}

\bibleverse{1} Hiskia war fünfundzwanzig Jahre alt, als er König ward,
und regierte neunundzwanzig Jahre lang zu Jerusalem. Und seine Mutter
hieß Abija, eine Tochter Sacharjas. \bibleverse{2} Und er tat, was recht
war in den Augen des \textsc{Herrn}, ganz wie sein Vater David getan
hatte. \bibleverse{3} Im ersten Monat des ersten Jahres seiner Regierung
öffnete er die Türen am Hause des \textsc{Herrn} und besserte sie aus.
\bibleverse{4} Und er ließ die Priester und Leviten kommen und
versammelte sie auf dem Platz gegen Aufgang und sprach zu ihnen:
\bibleverse{5} Höret mir zu, ihr Leviten! Nunmehr heiliget euch und
heiliget das Haus des \textsc{Herrn}, des Gottes eurer Väter, und
schaffet den Unflat aus dem Heiligtum heraus! \bibleverse{6} Denn unsre
Väter haben sich versündigt und getan, was in den Augen des
\textsc{Herrn}, unsres Gottes, böse ist, und haben ihn verlassen; denn
sie haben ihr Angesicht von der Wohnung des \textsc{Herrn} abgewandt und
ihr den Rücken gekehrt. \bibleverse{7} Auch haben sie die Türen der
Halle zugeschlossen und die Lampen ausgelöscht und kein Räucherwerk
angezündet und dem Gott Israels im Heiligtum kein Brandopfer
dargebracht. \bibleverse{8} Daher ist der Zorn des \textsc{Herrn} über
Juda und Jerusalem gekommen, und er hat sie der Mißhandlung und
Verwüstung preisgegeben, daß man sie auszischt, wie ihr mit euren Augen
sehet. \bibleverse{9} Denn siehe, um deswillen sind unsre Väter durch
das Schwert gefallen und unsre Söhne, unsre Töchter und unsre Weiber
gefangen weggeführt worden. \bibleverse{10} Nun habe ich im Sinn, einen
Bund zu machen mit dem \textsc{Herrn}, dem Gott Israels, damit sein
grimmiger Zorn sich von uns wende. \bibleverse{11} Nun, meine Söhne,
seid nicht nachlässig; denn euch hat der \textsc{Herr} erwählt, damit
ihr vor ihm stehet und ihm dienet und damit ihr seine Diener und
Räucherer seid! \bibleverse{12} Da machten sich die Leviten auf: Machat,
der Sohn Amasais, und Joel, der Sohn Asarjas, von den Söhnen der
Kahatiter; und von den Söhnen Meraris: Kis, der Sohn Abdis, und Asarja,
der Sohn Jehallelels; und von den Söhnen der Gersoniter: Joach, der Sohn
Simmas, und Eden, der Sohn Joachs; \bibleverse{13} und von den Söhnen
Elizaphans: Simri und Jehiel, und von den Söhnen Asaphs: Sacharja und
Mattanja; \bibleverse{14} und von den Söhnen Hemans; Jechiel und Simei;
und von den Söhnen Jedutuns: Semaja und Ussiel. \bibleverse{15} Und sie
versammelten ihre Brüder und heiligten sich und gingen hinein nach dem
Gebot des Königs und nach den Worten des \textsc{Herrn}, um das Haus des
\textsc{Herrn} zu reinigen. \bibleverse{16} Also gingen die Priester
hinein in das Innere des Hauses des \textsc{Herrn}, um es zu reinigen,
und schafften alles Unreine, das im Tempel des \textsc{Herrn} gefunden
ward, hinaus in den Vorhof am Hause des \textsc{Herrn}; und die Leviten
nahmen es und trugen es hinaus in den Bach Kidron. \bibleverse{17} Und
zwar begannen sie mit der Heiligung am ersten Tage des ersten Monats;
und am achten Tage desselben Monats kamen sie in die Halle des
\textsc{Herrn}, und sie heiligten das Haus des \textsc{Herrn} acht Tage
lang; und am sechzehnten Tag des ersten Monats wurden sie fertig.
\bibleverse{18} Da gingen sie hinein zum König Hiskia und sprachen: Wir
haben das ganze Haus des \textsc{Herrn} gereinigt, den Brandopferaltar
und alle seine Geräte; auch den Schaubrottisch und alle seine Geräte;
\bibleverse{19} auch alle Geräte, welche der König Ahas während seiner
Regierung entweiht hat, als er sich versündigte, haben wir
zurechtgemacht und geheiligt; und siehe, sie sind vor dem Altar des
\textsc{Herrn}! \bibleverse{20} Da machte sich der König Hiskia früh auf
und versammelte die Obersten der Stadt und ging hinauf zum Hause des
\textsc{Herrn}. \bibleverse{21} Und sie brachten sieben Farren, sieben
Widder, sieben Lämmer und sieben Ziegenböcke herbei zum Sündopfer für
das Königreich, für das Heiligtum und für Juda. Und er befahl den Söhnen
Aarons, den Priestern, sie auf dem Altar des \textsc{Herrn} zu opfern.
\bibleverse{22} Da schächteten sie die Rinder, und die Priester nahmen
das Blut und sprengten es an den Altar; und sie schächteten die Widder
und sprengten das Blut an den Altar; und sie schächteten die Lämmer und
sprengten das Blut an den Altar. \bibleverse{23} Und sie brachten die
Böcke zum Sündopfer für den König und die Gemeinde und stützten ihre
Hände auf sie. \bibleverse{24} Und die Priester schächteten sie und
brachten ihr Blut zur Entsündigung auf den Altar, um für ganz Israel
Sühne zu erwirken; denn für ganz Israel hatte der König Brandopfer und
Sündopfer befohlen. \bibleverse{25} Er ließ auch die Leviten sich im
Hause des \textsc{Herrn} aufstellen mit Zimbeln, Psaltern und Harfen,
wie David und Gad, der Seher des Königs, und der Prophet Natan befohlen
hatten; denn es war des \textsc{Herrn} Gebot durch seine Propheten.
\bibleverse{26} Und die Leviten stellten sich auf mit den
Musikinstrumenten Davids und die Priester mit den Trompeten.
\bibleverse{27} Und Hiskia befahl, das Brandopfer auf dem Altar zu
opfern. Und als das Brandopfer begann, fing auch der Gesang zu Ehren des
\textsc{Herrn} an und das Spiel der Trompeten, unter der Führung der
Musikinstrumente Davids, des Königs von Israel. \bibleverse{28} Und die
ganze Gemeinde betete an; und die Sänger sangen, und die Trompeter
schmetterten so lange, bis das Brandopfer vollendet war. \bibleverse{29}
Als nun das Brandopfer vollendet war, kniete der König nieder samt
allen, die sich bei ihm befanden, und sie beteten an. \bibleverse{30}
Und der König Hiskia und die Obersten geboten den Leviten, den
\textsc{Herrn} zu loben mit den Worten Davids und Asaphs, des Sehers.
Und sie lobten mit Freuden und verneigten sich und beteten an.
\bibleverse{31} Und Hiskia hob an und sprach: Nun habt ihr eure Hände
dem \textsc{Herrn} gefüllt. Tretet herzu und bringet die Schlachtopfer
und Lobopfer zum Hause des \textsc{Herrn}! Da brachte die Gemeinde
Schlachtopfer und Lobopfer, und alle, die willigen Herzens waren,
brachten Brandopfer. \bibleverse{32} Und die Zahl der Brandopfer, welche
die Gemeinde herzubrachte, betrug siebzig Rinder, hundert Widder und
zweihundert Lämmer; solches alles dem \textsc{Herrn} zum Brandopfer.
\bibleverse{33} Zudem heiligten sie sechshundert Rinder und dreitausend
Schafe. \bibleverse{34} Nur waren der Priester zu wenige, so daß sie
nicht allen Brandopfern die Haut abziehen konnten; darum halfen ihnen
ihre Brüder, die Leviten, bis das Werk vollendet war, und bis sich die
Priester geheiligt hatten; denn die Leviten waren ernstlicher darauf
bedacht, sich zu heiligen, als die Priester. \bibleverse{35} Es waren
aber auch Brandopfer in Menge darzubringen, samt dem Fett der Dankopfer
und den Trankopfern zu den Brandopfern. So ward der Dienst im Hause des
\textsc{Herrn} wiederhergestellt. \bibleverse{36} Und Hiskia freute sich
samt dem ganzen Volke über das, was Gott dem Volk zubereitet hatte; denn
die Sache war sehr rasch vor sich gegangen.

\hypertarget{section-29}{%
\section{30}\label{section-29}}

\bibleverse{1} Und Hiskia sandte Boten an ganz Israel und Juda und
schrieb auch Briefe an Ephraim und Manasse, daß sie zum Hause des
\textsc{Herrn} nach Jerusalem kommen sollten, um dem \textsc{Herrn}, dem
Gott Israels, Passah zu feiern. \bibleverse{2} Denn der König beschloß
mit seinen Obersten und der ganzen Gemeinde zu Jerusalem, das Passah im
zweiten Monat zu feiern; \bibleverse{3} denn sie konnten es nicht zur
bestimmten Zeit feiern, weil sich die Priester nicht in genügender Zahl
geheiligt hatten und das Volk noch nicht in Jerusalem versammelt war.
\bibleverse{4} Und der Beschluß gefiel dem König und der ganzen Gemeinde
wohl. \bibleverse{5} Und sie verfaßten einen Aufruf, der in ganz Israel,
von Beerseba bis Dan, verkündigt werden sollte, daß sie kämen, um dem
\textsc{Herrn}, dem Gott Israels, zu Jerusalem Passah zu halten; denn
sie hatten es nicht in Menge gefeiert, wie es vorgeschrieben ist.
\bibleverse{6} Und die Läufer gingen mit den Briefen von der Hand des
Königs und seiner Obersten durch ganz Israel und Juda und sprachen nach
dem Befehl des Königs: Ihr Kinder Israel, kehret zurück zum
\textsc{Herrn}, dem Gott Abrahams, Isaaks und Israels, so wird er sich
zu den Entronnenen kehren, die euch aus der Hand der assyrischen Könige
noch übriggeblieben sind, \bibleverse{7} und seid nicht wie eure Väter
und eure Brüder, die sich an dem \textsc{Herrn}, dem Gott ihrer Väter,
versündigt haben, daß er sie der Verwüstung preisgab, wie ihr sehet!
\bibleverse{8} So seid nun nicht halsstarrig wie eure Väter, sondern
reichet dem \textsc{Herrn} die Hand und kommt zu seinem Heiligtum,
welches er auf ewig geheiligt hat, und dienet dem \textsc{Herrn}, eurem
Gott, so wird sich die Glut seines Zorns von euch wenden. \bibleverse{9}
Denn wenn ihr zum \textsc{Herrn} zurückkehret, so werden eure Brüder und
eure Söhne Barmherzigkeit finden vor denen, die sie gefangen halten, daß
sie wieder in dieses Land zurückkehren. Denn der \textsc{Herr}, euer
Gott, ist gnädig und barmherzig, und er wird das Angesicht nicht von
euch wenden, wenn ihr euch zu ihm kehret! \bibleverse{10} Und die Läufer
gingen von einer Stadt zur andern im Lande Ephraim und Manasse und bis
nach Sebulon; aber jene verlachten sie und spotteten ihrer.
\bibleverse{11} Doch etliche von Asser und Manasse und Sebulon
demütigten sich und kamen nach Jerusalem. \bibleverse{12} Auch in Juda
wirkte die Hand Gottes, daß er ihnen ein einmütiges Herz gab, des Königs
und der Obersten Gebot zu erfüllen nach dem Wort des \textsc{Herrn}.
\bibleverse{13} So versammelte sich denn zu Jerusalem eine große
Volksmenge, um das Fest der ungesäuerten Brote zu feiern im zweiten
Monat, eine sehr große Gemeinde. \bibleverse{14} Und sie machten sich
auf und schafften die Altäre weg, die zu Jerusalem waren, auch alle
Räucheraltäre beseitigten sie und warfen sie in den Bach Kidron.
\bibleverse{15} Dann schächteten sie das Passah am vierzehnten Tag des
zweiten Monats. Und die Priester und Leviten schämten sich und heiligten
sich und brachten Brandopfer zum Hause des \textsc{Herrn};
\bibleverse{16} und sie standen auf ihren Posten, wie es sich gebührt,
nach dem Gesetze Moses, des Mannes Gottes. Und die Priester sprengten
das Blut, das sie empfingen, aus der Hand der Leviten. \bibleverse{17}
Denn es waren viele in der Gemeinde, die sich nicht geheiligt hatten;
darum schächteten die Leviten die Passahlämmer für alle, die nicht rein
waren, um sie dem \textsc{Herrn} zu heiligen. \bibleverse{18} Auch waren
viele vom Volk, von Ephraim, Manasse, Issaschar und Sebulon, die sich
nicht gereinigt hatten, so daß sie das Passah nicht aßen, wie es
vorgeschrieben ist; \bibleverse{19} aber Hiskia betete für sie und
sprach: Der \textsc{Herr}, der gütig ist, wolle allen denen vergeben,
die ihr Herz darauf gerichtet haben, Gott zu suchen, den \textsc{Herrn},
den Gott ihrer Väter, auch wenn sie nicht die für das Heiligtum
erforderliche Reinheit besitzen! \bibleverse{20} Und der \textsc{Herr}
erhörte Hiskia und heilte das Volk. \bibleverse{21} Also feierten die
Kinder Israel, die sich zu Jerusalem befanden, das Fest der ungesäuerten
Brote sieben Tage lang mit großer Freude. Und die Leviten und Priester
lobten den \textsc{Herrn} alle Tage mit Instrumenten zum Preise des
\textsc{Herrn}. \bibleverse{22} Und Hiskia sprach allen Mut zu, welche
sich verständig erwiesen in der Erkenntnis des \textsc{Herrn}; und sie
aßen das für das Fest Bestimmte, sieben Tage lang, und opferten
Dankopfer und bekannten sich zum \textsc{Herrn}, dem Gott ihrer Väter.
\bibleverse{23} Und die ganze Gemeinde beschloß, noch weitere sieben
Tage das Fest zu feiern, und so feierten sie noch sieben Tage lang ein
Freudenfest; \bibleverse{24} denn Hiskia, der König von Juda, spendete
für die Gemeinde tausend Farren und zehntausend Schafe. Und es heiligten
sich viele Priester. \bibleverse{25} Und die ganze Gemeinde von Juda
freute sich und die Priester und Leviten und die ganze Gemeinde, die aus
Israel gekommen war, auch die Fremdlinge, die aus dem Lande Israel
gekommen waren, und die in Juda wohnten. \bibleverse{26} Und es war
große Freude zu Jerusalem; denn seit der Zeit Salomos, des Sohnes
Davids, des Königs von Israel, war dergleichen nicht gewesen.
\bibleverse{27} Und die Priester, die Leviten, standen auf und segneten
das Volk, und ihr Rufen ward erhört, und ihr Gebet kam zu Seiner
heiligen Wohnung, in den Himmel.

\hypertarget{section-30}{%
\section{31}\label{section-30}}

\bibleverse{1} Und nach Beendigung aller dieser Festlichkeiten zogen
alle Israeliten, die sich eingefunden hatten, hinaus zu den Städten
Judas, zerbrachen die Säulen und hieben die Ascheren um und zerstörten
die Höhen und die Altäre in ganz Juda und Benjamin, Ephraim und Manasse,
bis sie dieselben gänzlich ausgetilgt hatten. Darnach kehrten alle
Kinder Israel wieder zu ihrer Besitzung, in ihre Städte zurück.
\bibleverse{2} Hiskia aber stellte die Abteilungen der Priester und der
Leviten wieder her, daß jeder seinen Dienst hatte, sowohl die Priester
als auch die Leviten, Brandopfer und Dankopfer darzubringen, zu dienen,
zu danken und zu loben in den Toren des Lagers des \textsc{Herrn}.
\bibleverse{3} Auch gab der König einen Teil seiner Habe für die
Brandopfer, für die Brandopfer am Morgen und am Abend, und für die
Brandopfer an den Sabbaten und Neumonden und Festen, wie im Gesetze des
\textsc{Herrn} geschrieben steht. \bibleverse{4} Und er gebot dem Volk,
das zu Jerusalem wohnte, den Priestern und Leviten ihre Gebühr zu geben,
damit sie am Gesetz des \textsc{Herrn} festhalten könnten.
\bibleverse{5} Als nun dieser Befehl bekannt ward, gaben die Kinder
Israel viele Erstlingsgaben von Korn, Most, Öl, Honig und allem Ertrag
des Feldes und brachten die Zehnten von allem in Menge herbei.
\bibleverse{6} Und auch die Kinder Israel und Juda, die in den Städten
Judas wohnten, brachten den Zehnten von Rindern und Schafen und den
Zehnten von den geheiligten Dingen, die dem \textsc{Herrn}, ihrem Gott,
geheiligt worden waren, und legten es haufenweise hin. \bibleverse{7} Im
dritten Monat fingen sie an, die Haufen aufzuschütten, und im siebenten
Monat waren sie damit fertig. \bibleverse{8} Als nun Hiskia und die
Obersten hineingingen und die Haufen sahen, lobten sie den
\textsc{Herrn} und sein Volk Israel. \bibleverse{9} Und Hiskia befragte
die Priester und Leviten wegen dieser Haufen. \bibleverse{10} Da
antwortete ihm Asarja, der Oberpriester vom Hause Zadok, und sprach:
Seitdem man angefangen hat, das Hebopfer in das Haus des \textsc{Herrn}
zu bringen, hat man gegessen und ist satt geworden und hat noch viel
übriggelassen; denn der \textsc{Herr} hat sein Volk gesegnet; daher ist
eine so große Menge übriggeblieben. \bibleverse{11} Da befahl Hiskia,
daß man Vorratskammern herrichte im Hause des \textsc{Herrn}; und sie
richteten dieselben her \bibleverse{12} und taten treulich hinein das
Hebopfer, die Zehnten und das Geheiligte. Und als Oberaufseher über
dieselben wurden bestellt: Kananja, der Levit, und Simei, sein Bruder,
als zweiter; \bibleverse{13} dazu Jechiel, Asasja, Nahat, Asahel,
Jerimot, Josabad, Eliel, Jismachja, Mahat und Benaja, als Aufseher unter
der Leitung Kananjas und Simeis, seines Bruders, nach dem Befehl des
Königs Hiskia und Asarjas, des Obersten im Hause Gottes. \bibleverse{14}
Und Kore, der Sohn Jimmas, der Levit, der Torhüter gegen Aufgang, war
über die freiwilligen Gaben Gottes gesetzt, um das Hebopfer des
\textsc{Herrn} und die hochheiligen Dinge herauszugeben. \bibleverse{15}
Und unter seiner Leitung waren Eden, Minjamin, Jesua, Semaja, Amarja und
Sechanja, um in den Städten der Priester ihren Brüdern abteilungsweise
getreulich ihren Anteil zu geben, den Kleinen wie den Großen.
\bibleverse{16} Überdies wurden sie in Geschlechtsregister eingetragen,
alles, was männlich war, von drei Jahren an und darüber, alle, die in
das Haus des \textsc{Herrn} gehen sollten nach der täglichen Ordnung an
ihren Dienst auf ihren Posten, nach ihren Abteilungen. \bibleverse{17}
Und zwar erfolgte die Eintragung der Priester nach ihren Vaterhäusern,
und die der Leviten von zwanzig Jahren an und darüber mit Rücksicht auf
ihre Ämter, die sie abteilungsweise zu versehen hatten. \bibleverse{18}
Und sie hatten sich einzutragen samt ihren Kindern, ihren Weibern, ihren
Söhnen und Töchtern, als ganze Gemeinde; denn gewissenhaft heiligten sie
sich für das Heiligtum. \bibleverse{19} Und für die Söhne Aarons, die
Priester, die in den zu ihren Städten gehörenden Ländereien wohnten,
hatte es in jeder Stadt mit Namen bezeichnete Leute, welche die
Austeilung an die männlichen Glieder der Priesterfamilien und an alle,
die in die levitischen Geschlechtsregister eingetragen waren, zu
besorgen hatten. \bibleverse{20} Also handelte Hiskia in ganz Juda und
tat, was gut, recht und getreu war vor dem \textsc{Herrn}, seinem Gott.
\bibleverse{21} Und in all seinem Werk, das er im Dienste des Hauses
Gottes und nach dem Gesetze und Gebot unternahm, um seinen Gott zu
suchen, handelte er von ganzem Herzen, und so gelang es ihm auch.

\hypertarget{section-31}{%
\section{32}\label{section-31}}

\bibleverse{1} Nach diesen Geschichten und dieser bewiesenen Treue kam
Sanherib, der König von Assyrien, und rückte in Juda ein und belagerte
die festen Städte und gedachte sie zu erobern. \bibleverse{2} Als aber
Hiskia sah, daß Sanherib gekommen war und die Absicht hatte, wider
Jerusalem zu streiten, \bibleverse{3} beschloß er mit seinen Obersten
und seinen Gewaltigen, die Wasserquellen draußen vor der Stadt zu
verstopfen; und sie halfen ihm. \bibleverse{4} Und die Leute
versammelten sich in großer Zahl und verstopften alle Brunnen und den
Bach, der mitten durch das Land läuft, und sprachen: Warum sollten die
Könige von Assyrien viel Wasser finden, wenn sie kommen? \bibleverse{5}
Und er faßte Mut und baute die Mauer allenthalben, wo sie zerrissen war,
und erhöhte die Türme und baute draußen noch eine andere Mauer und
befestigte das Millo an der Stadt Davids. Auch machte er viele
Wurfgeschosse und Schilde \bibleverse{6} und setzte kriegstüchtige
Hauptleute über das Volk und versammelte sie zu sich auf den Platz am
Tore der Stadt, sprach ihnen Mut zu und sagte: \bibleverse{7} Seid stark
und fest! Fürchtet euch nicht und erschrecket nicht vor dem König von
Assyrien noch vor dem ganzen Haufen, der bei ihm ist; denn mit uns ist
ein Größerer als mit ihm; \bibleverse{8} mit ihm ist ein fleischlicher
Arm, mit uns aber ist der \textsc{Herr}, unser Gott, um uns zu helfen
und für uns Krieg zu führen! Und das Volk verließ sich auf die Worte
Hiskias, des Königs von Juda. \bibleverse{9} Darnach sandte Sanherib,
der König von Assyrien, seine Knechte gen Jerusalem (denn er lag vor
Lachis mit seinem ganzen Heer) zu Hiskia, dem König von Juda, und zu
ganz Juda, das zu Jerusalem war, und ließ ihm sagen: \bibleverse{10} So
spricht Sanherib, der König von Assyrien: Worauf verlasset ihr euch, die
ihr in dem belagerten Jerusalem sitzet? \bibleverse{11} Verführt euch
nicht Hiskia, vor Hunger und Durst zu sterben, indem er sagt: Der
\textsc{Herr}, unser Gott, wird uns aus der Hand des Königs von Assyrien
erretten? \bibleverse{12} Hat aber nicht derselbe Hiskia seine Höhen und
Altäre weggeschafft und Juda und Jerusalem befohlen: Vor einem einzigen
Altar sollt ihr anbeten und räuchern? \bibleverse{13} Wisset ihr nicht,
was ich und meine Väter allen Völkern der Länder getan haben? Haben auch
die Götter der Nationen in den Ländern jemals ihre Länder aus meiner
Hand zu erretten vermocht? \bibleverse{14} Wer ist unter allen Göttern
dieser Nationen, die meine Väter ganz und gar vernichtet haben, der sein
Volk aus meiner Hand zu erretten vermochte, daß euer Gott euch aus
meiner Hand erretten könnte? \bibleverse{15} So lasset euch nun durch
Hiskia nicht verführen und lasset euch nicht also von ihm bereden und
glaubet ihm nicht! Denn da kein Gott irgend einer Nation oder eines
Königreiches sein Volk aus meiner Hand und aus der Hand meiner Väter zu
erretten vermochte, so wird auch euer Gott euch nicht aus meiner Hand zu
erretten vermögen! \bibleverse{16} Und noch mehr redeten seine Knechte
wider Gott, den \textsc{Herrn}, und wider seinen Knecht Hiskia.
\bibleverse{17} Er schrieb auch Briefe, um den \textsc{Herrn}, den Gott
Israels, zu schmähen, und redete wider ihn und sprach: Wie die Götter
der Nationen in den Ländern ihr Volk nicht aus meiner Hand errettet
haben, so wird auch der Gott Hiskias sein Volk nicht aus meiner Hand
erretten! \bibleverse{18} Und sie riefen mit lauter Stimme, auf jüdisch,
zum Volk von Jerusalem, das auf den Mauern war, um es furchtsam zu
machen und zu erschrecken und so die Stadt gewinnen zu können;
\bibleverse{19} und sie redeten vom Gott Jerusalems wie von den Göttern
der Heidenvölker, die ein Werk von Menschenhänden sind. \bibleverse{20}
Aber der König Hiskia und der Prophet Jesaja, der Sohn des Amoz, beteten
deshalb und schrieen zum Himmel. \bibleverse{21} Und der \textsc{Herr}
sandte einen Engel, der vertilgte alle Gewaltigen des Heeres und die
Fürsten und die Obersten im Lager des Königs von Assyrien, daß er mit
Schanden in sein Land zurückkehrte. Und als er in das Haus seines Gottes
ging, fällten ihn daselbst durchs Schwert einige seiner leiblichen
Söhne. \bibleverse{22} Also rettete der \textsc{Herr} den Hiskia und die
Einwohner von Jerusalem aus der Hand Sanheribs, des Königs von Assyrien,
und aus der Hand aller andern und verschaffte ihnen Ruhe auf allen
Seiten; \bibleverse{23} so daß viele dem \textsc{Herrn} Geschenke
brachten nach Jerusalem und Hiskia, dem König von Juda, Kostbarkeiten;
und er stieg darnach in der Achtung aller Nationen. \bibleverse{24} Zu
jener Zeit ward Hiskia todkrank. Da betete er zum \textsc{Herrn}; der
redete mit ihm und gab ihm ein Wunderzeichen. \bibleverse{25} Aber
Hiskia vergalt die Wohltat nicht, die ihm widerfahren war, sondern sein
Herz erhob sich. Da kam der Zorn über ihn und über Juda und Jerusalem.
\bibleverse{26} Als aber Hiskia sich darüber demütigte, daß sein Herz
sich erhoben hatte, er und die Einwohner von Jerusalem, kam der Zorn des
\textsc{Herrn} nicht über sie, solange Hiskia lebte. \bibleverse{27} Und
Hiskia hatte sehr großen Reichtum und Ehre und sammelte sich Schätze von
Silber, Gold, Edelsteinen, Gewürz, Schilden und allerlei kostbaren
Geräten. \bibleverse{28} Er hatte auch Vorratshäuser für den Ertrag des
Korns, Mosts und Öls; und Ställe für allerlei Vieh und Hürden für die
Schafe. \bibleverse{29} Und er baute Städte und hatte sehr viel Vieh,
Schafe und Rinder; denn Gott gab ihm sehr viele Güter. \bibleverse{30}
Er, Hiskia, war es auch, der den obern Ausfluß des Wassers Gihon
verstopfte und es westlich abwärts, zur Stadt Davids leitete; und Hiskia
hatte Glück in allem, was er unternahm. \bibleverse{31} Als aber die
Gesandten der Fürsten von Babel zu ihm geschickt wurden, sich nach dem
Wunder zu erkundigen, das im Lande geschehen war, verließ ihn Gott, um
ihn auf die Probe zu stellen, damit kund würde alles, was in seinem
Herzen sei. \bibleverse{32} Das Übrige aber von Hiskias Geschichte und
von seiner Frömmigkeit, siehe, das ist aufgezeichnet in der Offenbarung
des Propheten Jesaja, des Sohnes des Amoz, und im Buche der Könige von
Juda und Israel. \bibleverse{33} Und Hiskia entschlief mit seinen
Vätern, und man begrub ihn bei der Treppe, die zu den Gräbern der Söhne
Davids führt. Und ganz Juda und die Einwohner von Jerusalem erwiesen ihm
Ehre bei seinem Tode; und sein Sohn Manasse ward König an seiner Statt.

\hypertarget{section-32}{%
\section{33}\label{section-32}}

\bibleverse{1} Manasse war zwölf Jahre alt, als er König ward, und
regierte fünfundfünfzig Jahre lang zu Jerusalem. \bibleverse{2} Und er
tat, was böse war in den Augen des \textsc{Herrn}, nach den Greueln der
Heiden, welche der \textsc{Herr} vor den Kindern Israel vertrieben
hatte. \bibleverse{3} Er baute die Höhen wieder auf, die sein Vater
Hiskia abgebrochen hatte, und errichtete den Baalen Altäre und machte
Ascheren und betete das ganze Heer des Himmels an und diente ihnen.
\bibleverse{4} Er baute auch Altäre im Hause des \textsc{Herrn}, davon
der \textsc{Herr} gesagt hatte: Zu Jerusalem soll mein Name sein
ewiglich! \bibleverse{5} Und er baute dem ganzen Heere des Himmels
Altäre, in beiden Vorhöfen am Hause des \textsc{Herrn}. \bibleverse{6}
Er führte auch seine Söhne durchs Feuer im Tal des Sohnes Hinnoms und
trieb Wolkendeuterei, Schlangenbeschwörung und Zauberei und hielt
Geisterbanner und Wahrsager und tat vielerlei Böses vor dem
\textsc{Herrn}, um ihn zu kränken. \bibleverse{7} Er setzte auch das
Götzenbild, das er machen ließ, in das Haus Gottes, davon Gott zu David
und seinem Sohn Salomo gesagt hatte: In dieses Haus und nach Jerusalem,
das ich aus allen Stämmen Israels erwählt habe, will ich meinen Namen
setzen ewiglich; \bibleverse{8} und ich will Israels Fuß nicht mehr aus
dem Lande vertreiben, das ich ihren Vätern bestimmt habe, sofern sie
darauf achten, alles zu tun, was ich ihnen geboten habe im ganzen
Gesetz, in den Satzungen und Rechten, durch Mose! \bibleverse{9} Aber
Manasse verführte das Volk Juda und die Einwohner von Jerusalem, so daß
sie Ärgeres taten als die Heiden, die der \textsc{Herr} vor den Kindern
Israel vertilgt hatte. \bibleverse{10} Und der \textsc{Herr} redete zu
Manasse und zu seinem Volk, aber sie merkten nicht darauf.
\bibleverse{11} Da ließ der \textsc{Herr} die Heerführer des Königs von
Assyrien über sie kommen, die fingen Manasse mit Haken, banden ihn mit
zwei ehernen Ketten und verbrachten ihn nach Babel. \bibleverse{12} Als
er nun in der Not war, flehte er den \textsc{Herrn}, seinen Gott, an und
demütigte sich sehr vor dem Gott seiner Väter. \bibleverse{13} Und da er
zu ihm betete, ließ sich Gott von ihm erbitten, also daß er sein Flehen
erhörte und ihn wieder gen Jerusalem zu seinem Königreich brachte. Da
erkannte Manasse, daß der \textsc{Herr} Gott ist. \bibleverse{14}
Darnach baute er eine äußere Mauer an der Stadt Davids, westlich vom
Bach Gihon und bis zum Eingang durch das Fischtor und rings um den
Ophel, machte sie sehr hoch und legte Hauptleute in alle festen Städte
Judas. \bibleverse{15} Er tat auch die fremden Götter weg und entfernte
das Götzenbild aus dem Hause des \textsc{Herrn} und alle Altäre, die er
auf dem Berge des Hauses des \textsc{Herrn} und zu Jerusalem gebaut
hatte, und warf sie vor die Stadt hinaus. \bibleverse{16} Und er baute
den Altar des \textsc{Herrn} und opferte darauf Dankopfer und Lobopfer
und befahl Juda, daß sie dem \textsc{Herrn}, dem Gott Israels, dienen
sollten. \bibleverse{17} Zwar opferte das Volk noch auf den Höhen, aber
nur dem \textsc{Herrn}, seinem Gott. \bibleverse{18} Die weitere
Geschichte Manasses und sein Gebet zu seinem Gott und die Reden der
Seher, die im Namen des \textsc{Herrn}, des Gottes Israels, zu ihm
redeten, siehe, das steht in den Geschichten der Könige von Israel.
\bibleverse{19} Sein Gebet, und wie sich Gott von ihm hat erbitten
lassen, und alle seine Sünde und seine Missetat und die Orte, darauf er
die Höhen baute und Ascheren und Götzenbilder aufstellte, ehe er
gedemütigt ward, siehe, das ist beschrieben in den Geschichten der
Seher. \bibleverse{20} Und Manasse legte sich zu seinen Vätern, und man
begrub ihn in seinem Hause; und sein Sohn Amon ward König an seiner
Statt. \bibleverse{21} Zweiundzwanzig Jahre alt war Amon, als er König
ward, und regierte zwei Jahre lang zu Jerusalem; \bibleverse{22} und er
tat, was böse war in den Augen des \textsc{Herrn}, wie sein Vater
Manasse getan hatte. Und Amon opferte allen Götzen, die sein Vater
Manasse gemacht hatte, und diente ihnen. \bibleverse{23} Aber er
demütigte sich nicht vor dem \textsc{Herrn}, wie sich sein Vater Manasse
gedemütigt hatte, sondern er, Amon, lud große Schuld auf sich.
\bibleverse{24} Und seine Knechte machten eine Verschwörung gegen ihn
und töteten ihn in seinem Hause. \bibleverse{25} Da erschlug die
Landbevölkerung alle, welche die Verschwörung wider den König Amon
gemacht hatten; und die Landbevölkerung machte seinen Sohn Josia zum
König an seiner Statt.

\hypertarget{section-33}{%
\section{34}\label{section-33}}

\bibleverse{1} Acht Jahre alt war Josia, als er König ward, und regierte
einunddreißig Jahre lang zu Jerusalem. \bibleverse{2} Und er tat, was
recht war in den Augen des \textsc{Herrn}, und wandelte in den Wegen
seines Vaters David und wich weder zur Rechten noch zur Linken.
\bibleverse{3} Denn im achten Jahr seines Königreichs, als er noch ein
Knabe war, fing er an, den Gott seines Vaters David zu suchen; und im
zwölften Jahr fing er an, Juda und Jerusalem von den Höhen und den
Ascheren und den geschnitzten und gegossenen Bildern zu reinigen.
\bibleverse{4} Und man brach in seiner Gegenwart die Altäre der Baale
ab; und er hieb die Sonnensäulen, die auf denselben standen, um; und die
Ascheren und die geschnitzten und gegossenen Bilder zerbrach er und
machte sie zu Staub und streute sie auf die Gräber derer, die ihnen
geopfert hatten; \bibleverse{5} er verbrannte auch die Gebeine der
Priester auf ihren Altären, und so reinigte er Juda und Jerusalem.
\bibleverse{6} Ebenso tat er in den Städten von Manasse, Ephraim und
Simeon und bis gen Naphtali, in ihren Ruinen ringsum. \bibleverse{7} Und
als er die Altäre und die Ascheren abgebrochen und die geschnitzten
Bilder zu Staub zermalmt und alle Sonnensäulen im ganzen Lande Israel
abgehauen hatte, kehrte er wieder nach Jerusalem zurück. \bibleverse{8}
Im achtzehnten Jahr seines Königreichs, als er das Land und das Haus
Gottes gereinigt hatte, sandte er Saphan, den Sohn Azaljas, und Maaseja,
den Obersten der Stadt, und Joach, den Sohn des Joahas, den Kanzler, um
das Haus des \textsc{Herrn}, seines Gottes, auszubessern. \bibleverse{9}
Und sie kamen zu dem Hohenpriester Hilkia und übergaben das Geld, das
zum Hause des \textsc{Herrn} gebracht ward, welches die Leviten, die an
der Schwelle hüteten, von Manasse, Ephraim und von allen
Übriggebliebenen in Israel und von ganz Juda und Benjamin und von den
Einwohnern Jerusalems gesammelt hatten. \bibleverse{10} Sie gaben es
aber in die Hand derer, die dazu bestellt waren, das Werk am Hause des
\textsc{Herrn} ausführen zu lassen, und diese gaben es den Werkleuten,
welche am Hause des \textsc{Herrn} arbeiteten, um das Haus wieder
herzustellen und auszubessern; \bibleverse{11} und zwar gaben sie es den
Zimmerleuten und Bauleuten, um gehauene Steine zu kaufen und Holz für
die Bindebalken und für die Balken der Häuser, welche die Könige Judas
verderbt hatten. \bibleverse{12} Und die Leute arbeiteten auf Treu und
Glauben an dem Werk. Und es waren über sie verordnet Jahat und Obadja,
die Leviten von den Kindern Merari, Sacharja und Mesullam von den
Kindern der Kahatiter, und alle diese Leviten verstanden sich auf
Musikinstrumente. \bibleverse{13} Auch über die Lastträger und alle
Arbeitsleute der verschiedenen Gewerbe waren Aufseher, und einige von
den Leviten waren Schreiber, Amtleute und Torhüter. \bibleverse{14} Als
sie aber das Geld herausnahmen, das zum Hause des \textsc{Herrn}
gebracht worden war, fand der Priester Hilkia das Gesetzbuch des
\textsc{Herrn}, durch Mose gegeben. \bibleverse{15} Da hob Hilkia an und
sprach zu Saphan, dem Schreiber: Ich habe das Gesetzbuch im Hause des
\textsc{Herrn} gefunden! Und Hilkia gab das Buch dem Saphan.
\bibleverse{16} Saphan aber brachte das Buch zum König und meldete dem
König und sprach: Deine Knechte besorgen alles, was ihnen in die Hände
gelegt worden ist. \bibleverse{17} Sie haben das Geld ausgeschüttet, das
im Hause des \textsc{Herrn} vorgefunden worden ist, und haben es den
Aufsehern und den Arbeitern gegeben. \bibleverse{18} Dann berichtete der
Schreiber Saphan dem König und sprach: Der Priester Hilkia hat mir ein
Buch gegeben! Und Saphan las daraus dem König vor. \bibleverse{19} Als
nun der König die Worte des Gesetzes hörte, zerriß er seine Kleider.
\bibleverse{20} Und der König gebot Hilkia und Achikam, dem Sohne
Saphans, und Abdon, dem Sohn Michas, und Saphan, dem Schreiber, und
Asaja, dem Knecht des Königs, und sprach: \bibleverse{21} Gehet hin,
fraget den \textsc{Herrn} für mich und für die Übriggebliebenen in
Israel und Juda wegen der Worte des Buches, das gefunden worden ist;
denn groß ist der Grimm des \textsc{Herrn}, der über uns ausgegossen
ist, weil unsre Väter das Wort des \textsc{Herrn} nicht beobachtet
haben, zu tun nach allem, was in diesem Buche geschrieben steht!
\bibleverse{22} Da ging Hilkia mit den andern, die vom König gesandt
waren, zu der Prophetin Hulda, dem Weibe Sallums, des Sohnes Tokhats,
des Sohnes Hasras, des Kleiderhüters, die zu Jerusalem wohnte im andern
Stadtteil, und sie redeten demgemäß mit ihr. \bibleverse{23} Und sie
sprach zu ihnen: So spricht der \textsc{Herr}, der Gott Israels: Saget
dem Mann, der euch zu mir gesandt hat: \bibleverse{24} So spricht der
\textsc{Herr}: Siehe, ich will Unglück bringen über diesen Ort und über
seine Einwohner, nämlich alle die Flüche, welche geschrieben stehen in
dem Buche, das man vor dem König von Juda gelesen hat, \bibleverse{25}
weil sie mich verlassen und andern Göttern geräuchert haben, mich zu
reizen mit allen Werken ihrer Hände; darum soll mein Grimm sich über
diesen Ort ergießen und nicht ausgelöscht werden! \bibleverse{26} Zum
König von Juda aber, der euch gesandt hat, den \textsc{Herrn} zu
befragen, sollt ihr also sagen: So spricht der \textsc{Herr}, der Gott
Israels, betreffs der Worte, die du gehört hast: \bibleverse{27} Weil
dein Herz weich geworden ist und du dich vor Gott gedemütigt hast, als
du seine Worte wider diesen Ort und wider seine Einwohner hörtest, ja,
weil du dich vor mir gedemütigt und deine Kleider zerrissen und vor mir
geweint hast, so habe auch ich dich erhört, spricht der \textsc{Herr}.
\bibleverse{28} Siehe, ich will dich zu deinen Vätern versammeln, daß du
in Frieden in dein Grab gebracht wirst und deine Augen all das Unglück
nicht sehen müssen, das ich über diesen Ort und seine Einwohner bringen
will. \bibleverse{29} Als sie nun dem König diese Antwort brachten,
sandte der König hin und ließ alle Ältesten in Juda und Jerusalem
zusammenkommen. \bibleverse{30} Und der König ging hinauf in das Haus
des \textsc{Herrn} und mit ihm alle Männer von Juda und die Einwohner
von Jerusalem, die Priester, die Leviten und alles Volk, groß und klein,
und man las vor ihren Ohren alle Worte des Bundesbuches, das im Hause
des \textsc{Herrn} gefunden worden war. \bibleverse{31} Und der König
trat an seinen Standort und machte einen Bund vor dem \textsc{Herrn},
daß er dem \textsc{Herrn} nachwandeln wolle, seine Gebote, seine
Zeugnisse und seine Satzungen zu halten von ganzem Herzen und von ganzer
Seele, zu tun nach den Worten des Bundes, die in diesem Buch geschrieben
sind. \bibleverse{32} Und er ließ alle dazu Stellung nehmen, die zu
Jerusalem und in Benjamin anwesend waren. Und die Einwohner von
Jerusalem taten nach dem Bunde Gottes, des Gottes ihrer Väter.
\bibleverse{33} Und Josia schaffte alle Greuel weg aus allen Ländern der
Kinder Israel und verpflichtete alle, die sich in Israel befanden, zum
Dienste des \textsc{Herrn}, ihres Gottes. Solange Josia lebte, wichen
sie nicht von dem \textsc{Herrn}, dem Gott ihrer Väter.

\hypertarget{section-34}{%
\section{35}\label{section-34}}

\bibleverse{1} Und Josia hielt dem \textsc{Herrn} ein Passah zu
Jerusalem, und sie schlachteten das Passah am vierzehnten Tage des
ersten Monats. \bibleverse{2} Und er stellte die Priester auf ihre
Posten und stärkte sie zu ihrem Dienst im Hause des \textsc{Herrn}.
\bibleverse{3} Er sprach auch zu den Leviten, welche ganz Israel lehrten
und die dem \textsc{Herrn} geheiligt waren: Tut die heilige Lade in das
Haus, das Salomo, der Sohn Davids, der König Israels, gebaut hat! Ihr
habt sie nicht mehr auf den Schultern zu tragen; so dienet nun dem
\textsc{Herrn}, eurem Gott, und seinem Volk Israel! \bibleverse{4} Und
seid bereit nach euren Vaterhäusern, in euren Abteilungen, wie sie
David, der König von Israel, und sein Sohn Salomo vorgeschrieben haben,
\bibleverse{5} und stellet euch im Heiligtum auf, entsprechend den
Abteilungen der Stammhäuser eurer Brüder, der Volksgenossen, auch nach
der Einteilung der Stammhäuser der Leviten, \bibleverse{6} und
schlachtet das Passah! Heiliget euch und bereitet zu für eure Brüder,
daß sie tun nach dem Wort des \textsc{Herrn} durch Mose! \bibleverse{7}
Und Josia stiftete für die Volksgenossen Schafe, Lämmer und Ziegen,
alles zu Passahopfern, für alle, die anwesend waren, 30000 an der Zahl;
dazu 3000 Rinder, und solches von der Habe des Königs. \bibleverse{8}
Auch seine Fürsten stifteten freiwillige Gaben für das Volk, für die
Priester und für die Leviten; Hilkia, Sacharja und Jechiel, die
Vorsteher des Hauses Gottes, gaben den Priestern 2600 Passahlämmer, dazu
300 Rinder. \bibleverse{9} Und Kananja, Semaja und Nataneel, seine
Brüder, und Chaschabja, Jechiel und Josabad, die Obersten der Leviten,
stifteten für die Leviten 5000 Lämmer und 500 Rinder. \bibleverse{10}
Nach diesen Vorbereitungen zum Gottesdienst traten die Priester an ihren
Platz und die Leviten in ihre Abteilungen nach dem Gebot des Königs.
\bibleverse{11} Und sie schächteten das Passah; die Priester nahmen das
Blut von ihren Händen und sprengten es, und die Leviten zogen den
Lämmern die Haut ab. \bibleverse{12} Und sie taten das Brandopfer
beiseite, um es den Abteilungen der Stammhäuser der Volksgenossen zu
geben, damit sie es dem \textsc{Herrn} darbrächten, wie im Buche Moses
geschrieben steht. Ebenso machten sie es mit den Rindern.
\bibleverse{13} Und sie brieten das Passah am Feuer, wie es sich
gebührt. Was aber geheiligt war, kochten sie in Töpfen, Kesseln und
Schalen; und sie teilten es eilends unter alles Volk. \bibleverse{14}
Darnach aber bereiteten sie auch für sich und für die Priester zu. Denn
die Priester, die Söhne Aarons, waren mit der Darbringung des
Brandopfers und der Fettstücke bis in die Nacht beschäftigt. Darum
mußten die Leviten für sich und für die Priester, die Söhne Aarons,
zubereiten. \bibleverse{15} Und die Sänger, die Söhne Asaphs, standen an
ihrem Platz nach dem Gebot Davids und Asaphs und Hemans und Jedutuns,
des Sehers des Königs; und die Torhüter waren an allen Toren. Sie
brauchten ihren Dienst nicht zu verlassen; denn ihre Brüder, die
Leviten, bereiteten für sie zu. \bibleverse{16} Also vollzog sich an
jenem Tag der ganze Dienst des \textsc{Herrn} in Ordnung, die
Passahfeier und der Brandopferdienst auf dem Altar des \textsc{Herrn},
nach dem Gebot des Königs Josia. \bibleverse{17} Also feierten die
Kinder Israel, die anwesend waren, zu jener Zeit das Passah und das Fest
der ungesäuerten Brote, sieben Tage lang. \bibleverse{18} Es war aber
kein derartiges Passah in Israel gefeiert worden, seit der Zeit des
Propheten Samuel; und keiner der Könige von Israel hatte ein solches
Passah veranstaltet, wie Josia es hielt mit den Priestern und Leviten
und mit ganz Juda und mit allen, die von Israel anwesend waren, auch mit
den Einwohnern von Jerusalem. \bibleverse{19} Im achtzehnten Jahre der
Regierung Josias wurde dieses Passah gefeiert. \bibleverse{20} Nach
alledem, als Josia das Haus des \textsc{Herrn} wieder hergestellt hatte,
zog Necho, der König von Ägypten, herauf, um bei Karkemisch am Euphrat
eine Schlacht zu liefern. Und Josia zog aus, ihm entgegen.
\bibleverse{21} Jener aber sandte Boten zu ihm und ließ ihm sagen: Was
habe ich mit dir zu schaffen, König von Juda? Nicht wider dich komme ich
heute, sondern wider ein Haus, das mit mir im Streite liegt, und Gott
hat gesagt, ich solle eilen. Laß ab von deinem Widerstand gegen Gott,
der mit mir ist, damit er dich nicht verderbe. \bibleverse{22} Aber
Josia wandte sein Angesicht nicht von ihm ab, sondern verkleidete sich,
um mit ihm zu kämpfen, und gehorchte nicht den Worten Nechos, die aus
dem Munde Gottes kamen, sondern kam, mit ihm zu streiten auf der Ebene
bei Megiddo. \bibleverse{23} Aber die Schützen trafen den König Josia.
Und der König sprach zu seinen Knechten: Traget mich hinüber, denn ich
bin schwer verwundet! \bibleverse{24} Da hoben ihn seine Knechte von dem
Kriegswagen auf seinen andern Wagen hinüber, den er bei sich hatte, und
brachten ihn nach Jerusalem. Und er starb und ward begraben in den
Gräbern seiner Väter. Und ganz Juda und Jerusalem trug Leid um Josia.
\bibleverse{25} Und Jeremia dichtete ein Klagelied auf Josia, und alle
Sänger und Sängerinnen haben seitdem in ihren Klageliedern von Josia
geredet, bis auf diesen Tag; und man machte sie zum Brauch in Israel.
Und siehe, sie sind aufgezeichnet in den Klageliedern. \bibleverse{26}
Was aber mehr von Josia zu sagen ist und seine Frömmigkeit nach der
Vorschrift des Gesetzes des \textsc{Herrn} \bibleverse{27} und seine
Geschichten, die früheren und die späteren, siehe, die sind
aufgezeichnet im Buche der Könige von Israel und Juda.

\hypertarget{section-35}{%
\section{36}\label{section-35}}

\bibleverse{1} Und die Landbevölkerung nahm Joahas, den Sohn Josias, und
machte ihn in Jerusalem zum König an seines Vaters Statt. \bibleverse{2}
Dreiundzwanzig Jahre alt war Joahas, als er König ward, und regierte
drei Monate lang zu Jerusalem. \bibleverse{3} Und der König von Ägypten
setzte ihn ab zu Jerusalem und legte dem Land eine Buße auf von hundert
Talenten Silber und einem Talent Gold. \bibleverse{4} Und der König von
Ägypten machte Eljakim, seinen Bruder, zum König über Juda und
Jerusalem, und änderte seinen Namen in Jehojakim. Necho aber nahm seinen
Bruder Joahas und brachte ihn nach Ägypten. \bibleverse{5}
Fünfundzwanzig Jahre alt war Jehojakim, als er König ward, und regierte
elf Jahre lang zu Jerusalem und tat, was böse war in den Augen des
\textsc{Herrn}, seines Gottes. \bibleverse{6} Da zog Nebukadnezar, der
König von Babel, wider ihn herauf und band ihn mit zwei ehernen Ketten,
um ihn nach Babel zu bringen. \bibleverse{7} Auch schleppte Nebukadnezar
etliche Geräte des Hauses des \textsc{Herrn} nach Babel und tat sie in
seinen Tempel zu Babel. \bibleverse{8} Was aber mehr von Jehojakim zu
sagen ist und seine Greuel, die er tat, und was an ihm erfunden worden,
das ist aufgezeichnet im Buch der Könige von Israel und Juda. Und
Jehojachin, sein Sohn, ward König an seiner Statt. \bibleverse{9}
Achtzehn Jahre alt war Jehojachin, als er König ward, und regierte drei
Monate und zehn Tage lang zu Jerusalem und tat, was böse war in den
Augen des \textsc{Herrn}. \bibleverse{10} Aber um die Jahreswende sandte
Nebukadnezar hin und ließ ihn nach Babel holen samt den kostbaren
Geräten des Hauses des \textsc{Herrn}, und machte Zedekia, seinen
Bruder, zum König über Juda und Jerusalem. \bibleverse{11} Einundzwanzig
Jahre alt war Zedekia, als er König ward, und regierte elf Jahre lang zu
Jerusalem. \bibleverse{12} Und er tat, was in den Augen des
\textsc{Herrn}, seines Gottes, böse war, und demütigte sich nicht vor
dem Propheten Jeremia, der aus dem Munde des \textsc{Herrn} zu ihm
redete. \bibleverse{13} Dazu ward er abtrünnig von dem König
Nebukadnezar, der einen Eid bei Gott von ihm genommen hatte, und ward
halsstarrig und verstockte sein Herz, so daß er nicht zu dem
\textsc{Herrn}, dem Gott Israels, umkehren wollte. \bibleverse{14} Auch
alle Obersten der Priester samt dem Volk vergingen sich schwer nach
allen Greueln der Heiden und verunreinigten das Haus des \textsc{Herrn},
das er geheiligt hatte zu Jerusalem. \bibleverse{15} Und gleichwohl
mahnte sie der \textsc{Herr}, der Gott ihrer Väter, unermüdlich durch
seine Boten; denn er hatte Mitleid mit seinem Volk und seiner Wohnung.
\bibleverse{16} Aber sie verspotteten die Boten Gottes und verachteten
seine Worte und verlachten seine Propheten, bis der Zorn des
\textsc{Herrn} über sein Volk so hoch stieg, daß keine Heilung mehr
möglich war. \bibleverse{17} Da ließ er den König der Chaldäer wider sie
heraufkommen, der tötete ihre Jungmannschaft mit dem Schwert im Hause
ihres Heiligtums und verschonte weder Jünglinge noch Jungfrauen, weder
Alte noch Hochbetagte, alle gab er in seine Hand. \bibleverse{18} Und
alle Geräte des Hauses Gottes, die großen und die kleinen, und die
Schätze des Hauses des \textsc{Herrn} und die Schätze des Königs und
seiner Fürsten, alles ließ er nach Babel führen. \bibleverse{19} Und sie
verbrannten das Haus Gottes und rissen die Mauer von Jerusalem nieder
und verbrannten alle ihre Paläste mit Feuer, so daß alle ihre kostbaren
Geräte zugrunde gingen. \bibleverse{20} Was aber vom Schwert
übriggeblieben war, führte er nach Babel hinweg, und sie wurden seine
und seiner Söhne Knechte, bis das Königreich der Perser zur Herrschaft
kam. \bibleverse{21} Also wurde das Wort des \textsc{Herrn} durch den
Mund Jeremias erfüllt: Bis das Land seine Sabbate gefeiert hat, soll es
ruhen, solange die Verwüstung währt, bis siebzig Jahre vollendet sind!
\bibleverse{22} Aber im ersten Jahr Kores, des Königs von Persien,
(damit das durch den Mund Jeremias geredete Wort des \textsc{Herrn}
erfüllt würde), erweckte der \textsc{Herr} den Geist des Kores, des
Königs von Persien, so daß er durch sein ganzes Königreich, auch
schriftlich, kundmachen und sagen ließ: \bibleverse{23} So spricht
Kores, der König von Persien: Der \textsc{Herr}, der Gott des Himmels,
hat mir alle Königreiche der Erde gegeben, und er selbst hat mir
befohlen, ihm ein Haus zu bauen zu Jerusalem, das in Juda ist. Wer
irgend unter euch zu seinem Volk gehört, mit dem sei der \textsc{Herr},
sein Gott, und er ziehe hinauf!
