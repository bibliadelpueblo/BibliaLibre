\hypertarget{section}{%
\section{1}\label{section}}

\bibleverse{1} Nachdem Gott vor Zeiten manchmal und auf mancherlei Weise
zu den Vätern geredet hat durch die Propheten, hat er zuletzt in diesen
Tagen zu uns geredet durch den Sohn, \bibleverse{2} welchen er zum Erben
von allem eingesetzt, durch welchen er auch die Weltzeiten gemacht hat;
\bibleverse{3} welcher, da er die Ausstrahlung seiner Herrlichkeit und
der Ausdruck seines Wesens ist und alle Dinge trägt mit dem Wort seiner
Kraft, und nachdem er die Reinigung unserer Sünden durch sich selbst
vollbracht, sich zur Rechten der Majestät in der Höhe gesetzt hat
\bibleverse{4} und um so viel mächtiger geworden ist als die Engel, als
der Name, den er ererbt hat, ihn vor ihnen auszeichnet. \bibleverse{5}
Denn zu welchem von den Engeln hat er jemals gesagt: ``Du bist mein
Sohn; heute habe ich dich gezeugt''? Und wiederum: ``Ich werde sein
Vater sein, und er wird mein Sohn sein''? \bibleverse{6} Und wie er den
Erstgeborenen wiederum in die Welt einführt, spricht er: ``Und es sollen
ihn alle Engel Gottes anbeten!'' \bibleverse{7} Von den Engeln zwar
heißt es: ``Er macht seine Engel zu Winden und seine Diener zu
Feuerflammen''; \bibleverse{8} aber von dem Sohn: ``Dein Thron, o Gott,
währt von Ewigkeit zu Ewigkeit. Das Zepter deines Reiches ist ein
gerades Zepter; \bibleverse{9} du hast Gerechtigkeit geliebt und
Ungerechtigkeit gehaßt, darum hat dich, Gott, dein Gott mit Freudenöl
gesalbt, mehr als deine Genossen!'' \bibleverse{10} Und: ``Du, Herr,
hast im Anfang die Erde gegründet, und die Himmel sind deiner Hände
Werk. \bibleverse{11} Sie werden vergehen, du aber bleibst; sie werden
alle veralten wie ein Kleid, \bibleverse{12} und wie einen Mantel wirst
du sie zusammenrollen, und sie sollen verwandelt werden. Du aber
bleibst, der du bist, und deine Jahre nehmen kein Ende.''
\bibleverse{13} Zu welchem von den Engeln aber hat er jemals gesagt:
``Setze dich zu meiner Rechten, bis ich deine Feinde hinlege als Schemel
deiner Füße''? \bibleverse{14} Sind sie nicht allzumal dienstbare
Geister, ausgesandt zum Dienste um derer willen, welche das Heil ererben
sollen?

\hypertarget{section-1}{%
\section{2}\label{section-1}}

\bibleverse{1} Darum sollen wir desto mehr auf das achten, was wir
gehört haben, damit wir nicht etwa daran vorbeigleiten. \bibleverse{2}
Denn wenn das durch Engel gesprochene Wort zuverlässig war und jede
Übertretung und jeder Ungehorsam den gerechten Lohn empfing,
\bibleverse{3} wie wollen wir entfliehen, wenn wir ein so großes Heil
versäumen, welches zuerst durch den Herrn gepredigt wurde und dann von
denen, die ihn gehört hatten, uns bestätigt worden ist? \bibleverse{4}
Und Gott gab sein Zeugnis dazu mit Zeichen und Wundern und mancherlei
Kraftwirkungen und Austeilungen des heiligen Geistes nach seinem Willen.
\bibleverse{5} Denn nicht Engeln hat er die zukünftige Welt, von der wir
reden, unterstellt. \bibleverse{6} Es bezeugt aber einer irgendwo und
spricht: ``Was ist der Mensch, daß du seiner gedenkst, oder des Menschen
Sohn, daß du zu ihm siehst? \bibleverse{7} Du hast ihn ein wenig
niedriger gemacht als die Engel, mit Herrlichkeit und Ehre hast du ihn
gekrönt; alles hast du unter seine Füße getan.'' \bibleverse{8} Indem er
ihm aber alles unterwarf, ließ er ihm nichts ununterworfen; jetzt aber
sehen wir, daß ihm noch nicht alles unterworfen ist; \bibleverse{9} den
aber, der ein wenig unter die Engel erniedrigt worden ist, Jesus, sehen
wir wegen des Todesleidens mit Herrlichkeit und Ehre gekrönt, damit er
durch Gottes Gnade für jedermann den Tod schmeckte. \bibleverse{10} Denn
es ziemte dem, um dessentwillen alles und durch den alles ist, als er
viele Kinder zur Herrlichkeit führte, den Anführer ihres Heils durch
Leiden zu vollenden. \bibleverse{11} Denn sowohl der, welcher heiligt,
als auch die, welche geheiligt werden, stammen alle von einem ab.
\bibleverse{12} Aus diesem Grunde schämt er sich auch nicht, sie Brüder
zu nennen, sondern spricht: ``Ich will deinen Namen meinen Brüdern
verkündigen; inmitten der Gemeinde will ich dir lobsingen!''
\bibleverse{13} Und wiederum: ``Ich will mein Vertrauen auf ihn
setzen''; und wiederum: ``Siehe, ich und die Kinder, die mir Gott
gegeben hat.'' \bibleverse{14} Da nun die Kinder Fleisch und Blut
gemeinsam haben, ist er in ähnlicher Weise dessen teilhaftig geworden,
damit er durch den Tod den außer Wirksamkeit setzte, der des Todes
Gewalt hat, nämlich den Teufel, \bibleverse{15} und alle diejenigen
befreite, welche durch Todesfurcht ihr ganzes Leben hindurch in
Knechtschaft gehalten wurden. \bibleverse{16} Denn er nimmt sich ja
nicht der Engel an, sondern des Samens Abrahams nimmt er sich an.
\bibleverse{17} Daher mußte er in allem den Brüdern ähnlich werden,
damit er barmherzig würde und ein treuer Hoherpriester vor Gott, um die
Sünden des Volkes zu sühnen; \bibleverse{18} denn worin er selbst
gelitten hat, als er versucht wurde, kann er denen helfen, die versucht
werden.

\hypertarget{section-2}{%
\section{3}\label{section-2}}

\bibleverse{1} Daher, ihr heiligen Brüder, Genossen einer himmlischen
Berufung, betrachtet den Apostel und Hohenpriester unsres Bekenntnisses,
Jesus, \bibleverse{2} welcher treu ist dem, der ihn gemacht hat, wie
auch Mose, in seinem ganzen Hause. \bibleverse{3} Denn dieser ist
größerer Ehre wertgeachtet worden als Mose, wie ja doch der, welcher ein
Haus bereitet hat, mehr Ehre verdient als das Haus selbst.
\bibleverse{4} Denn jedes Haus wird von jemand bereitet; der aber alles
bereitet hat, ist Gott. \bibleverse{5} Auch Mose zwar ist treu gewesen
in seinem ganzen Hause als Diener, zum Zeugnis dessen, was gesagt werden
sollte, \bibleverse{6} Christus aber als Sohn über sein eigenes Haus;
sein Haus sind wir, wenn wir die Freimütigkeit und den Ruhm der Hoffnung
bis zum Ende fest behalten. \bibleverse{7} Darum, wie der heilige Geist
spricht: ``Heute, wenn ihr seine Stimme hören werdet, so verstocket eure
Herzen nicht, \bibleverse{8} wie in der Verbitterung am Tage der
Versuchung in der Wüste, da mich eure Väter versuchten; \bibleverse{9}
sie prüften mich und sahen meine Werke vierzig Jahre lang.
\bibleverse{10} Darum ward ich entrüstet über dieses Geschlecht und
sprach: Immerdar irren sie mit ihrem Herzen! \bibleverse{11} Sie aber
erkannten meine Wege nicht, so daß ich schwur in meinem Zorn: Sie sollen
nicht eingehen in meine Ruhe!'' \bibleverse{12} Sehet zu, ihr Brüder,
daß nicht jemand von euch ein böses, ungläubiges Herz habe, im Abfall
begriffen von dem lebendigen Gott; \bibleverse{13} sondern ermahnet
einander jeden Tag, solange es ``heute'' heißt, damit nicht jemand unter
euch verstockt werde durch Betrug der Sünde! \bibleverse{14} Denn wir
sind Christi Genossen geworden, wenn wir die anfängliche Zuversicht bis
ans Ende festbehalten, \bibleverse{15} solange gesagt wird: ``Heute,
wenn ihr seine Stimme hören werdet, so verstocket eure Herzen nicht, wie
in der Verbitterung.'' \bibleverse{16} Welche wurden denn verbittert,
als sie es hörten? Waren es denn nicht alle, die unter Mose aus Ägypten
ausgezogen waren? \bibleverse{17} Welchen zürnte er aber vierzig Jahre
lang? Waren es nicht die, welche gesündigt hatten, deren Leiber in der
Wüste fielen? \bibleverse{18} Welchen schwur er aber, daß sie nicht in
seine Ruhe eingehen sollten, als nur denen, die ungehorsam gewesen
waren? \bibleverse{19} Und wir sehen, daß sie nicht eingehen konnten
wegen des Unglaubens.

\hypertarget{section-3}{%
\section{4}\label{section-3}}

\bibleverse{1} So laßt uns nun fürchten, daß nicht etwa, während doch
eine Verheißung zum Eingang in seine Ruhe hinterlassen ist, jemand von
euch als zu spät gekommen erscheine! \bibleverse{2} Denn auch uns ist
die gute Botschaft verkündigt worden, gleichwie jenen; aber das Wort der
Predigt half jenen nicht, weil es durch die Hörer nicht mit dem Glauben
verbunden wurde. \bibleverse{3} Denn wir, die wir gläubig geworden sind,
gehen in die Ruhe ein, wie er gesagt hat: ``Daß ich schwur in meinem
Zorn, sie sollen nicht in meine Ruhe eingehen''. \bibleverse{4} Und doch
waren die Werke seit Grundlegung der Welt beendigt; denn er hat irgendwo
von dem siebenten Tag also gesprochen: ``Und Gott ruhte am siebenten Tag
von allen seinen Werken'', \bibleverse{5} und in dieser Stelle wiederum:
``Sie sollen nicht in meine Ruhe eingehen!'' \bibleverse{6} Da nun noch
vorbehalten bleibt, daß etliche in sie eingehen sollen, und die, welchen
zuerst die gute Botschaft verkündigt worden ist, wegen ihres Ungehorsams
nicht eingegangen sind, \bibleverse{7} so bestimmt er wiederum einen
Tag, ein ``Heute'', indem er nach so langer Zeit durch David sagt, wie
schon angeführt: ``Heute, wenn ihr seine Stimme hören werdet, so
verstocket eure Herzen nicht!'' \bibleverse{8} Denn hätte Josua sie zur
Ruhe gebracht, so würde nicht hernach von einem anderen Tage gesprochen.
\bibleverse{9} Also bleibt dem Volke Gottes noch eine Sabbatruhe
vorbehalten; \bibleverse{10} denn wer in seine Ruhe eingegangen ist, der
ruht auch selbst von seinen Werken, gleichwie Gott von den seinigen.
\bibleverse{11} So wollen wir uns denn befleißigen, in jene Ruhe
einzugehen, damit nicht jemand als gleiches Beispiel des Unglaubens zu
Fall komme. \bibleverse{12} Denn das Wort Gottes ist lebendig und
wirksam und schärfer als jedes zweischneidige Schwert, und es dringt
durch, bis es scheidet Seele und Geist, auch Mark und Bein, und ist ein
Richter der Gedanken und Gesinnungen des Herzens; \bibleverse{13} und
keine Kreatur ist vor ihm unsichtbar, es ist aber alles bloß und
aufgedeckt vor den Augen dessen, welchem wir Rechenschaft zu geben
haben. \bibleverse{14} Da wir nun einen großen Hohenpriester haben, der
die Himmel durchschritten hat, Jesus, den Sohn Gottes, so lasset uns
festhalten an dem Bekenntnis! \bibleverse{15} Denn wir haben nicht einen
Hohenpriester, der kein Mitleid haben könnte mit unsren Schwachheiten,
sondern der in allem gleich wie wir versucht worden ist, doch ohne
Sünde. \bibleverse{16} So lasset uns nun mit Freimütigkeit hinzutreten
zum Thron der Gnade, damit wir Barmherzigkeit erlangen und Gnade finden
zu rechtzeitiger Hilfe!

\hypertarget{section-4}{%
\section{5}\label{section-4}}

\bibleverse{1} Denn jeder aus Menschen genommene Hohepriester wird für
Menschen eingesetzt, zum Dienst vor Gott, um sowohl Gaben darzubringen,
als auch Opfer für Sünden. \bibleverse{2} Ein solcher kann Nachsicht
üben mit den Unwissenden und Irrenden, da er auch selbst mit Schwachheit
behaftet ist; \bibleverse{3} und ihretwegen muß er, wie für das Volk, so
auch für sich selbst, opfern für die Sünden. \bibleverse{4} Und keiner
nimmt sich selbst die Würde, sondern er wird von Gott berufen, gleichwie
Aaron. \bibleverse{5} So hat auch Christus sich nicht selbst die
hohepriesterliche Würde beigelegt, sondern der, welcher zu ihm sprach:
``Du bist mein Sohn; heute habe ich dich gezeugt.'' \bibleverse{6} Wie
er auch an anderer Stelle spricht: ``Du bist ein Priester in Ewigkeit
nach der Ordnung Melchisedeks.'' \bibleverse{7} Und er hat in den Tagen
seines Fleisches Bitten und Flehen mit starkem Geschrei und Tränen dem
dargebracht, der ihn vom Tode retten konnte, und ist auch erhört und
befreit worden von dem Zagen. \bibleverse{8} Und wiewohl er Sohn war,
hat er doch an dem, was er litt, den Gehorsam gelernt; \bibleverse{9}
und so zur Vollendung gelangt, ist er allen, die ihm gehorchen, der
Urheber ewigen Heils geworden, \bibleverse{10} von Gott zubenannt:
Hoherpriester ``nach der Ordnung Melchisedeks''. \bibleverse{11} Davon
haben wir nun viel zu sagen, und solches, was schwer zu erklären ist,
weil ihr träge geworden seid zum Hören; \bibleverse{12} und obschon ihr
der Zeit nach Lehrer sein solltet, habt ihr wieder nötig, daß man euch
gewisse Anfangsgründe der Aussprüche Gottes lehre, und seid der Milch
bedürftig geworden und nicht fester Speise. \bibleverse{13} Denn wer
noch Milch genießt, der ist unerfahren im Worte der Gerechtigkeit; denn
er ist unmündig. \bibleverse{14} Die feste Speise aber ist für die
Gereiften, deren Sinne durch Übung geschult sind zur Unterscheidung des
Guten und des Bösen.

\hypertarget{section-5}{%
\section{6}\label{section-5}}

\bibleverse{1} Darum wollen wir jetzt die Anfangslehre von Christus
verlassen und zur Vollkommenheit übergehen, nicht abermals den Grund
legen mit der Buße von toten Werken und dem Glauben an Gott,
\bibleverse{2} mit der Lehre von Taufen, von der Handauflegung, der
Totenauferstehung und dem ewigen Gericht. \bibleverse{3} Und das wollen
wir tun, wenn Gott es zuläßt. \bibleverse{4} Denn es ist unmöglich, die,
welche einmal erleuchtet worden sind und die himmlische Gabe geschmeckt
haben und des heiligen Geistes teilhaftig geworden sind \bibleverse{5}
und das gute Wort Gottes, dazu Kräfte der zukünftigen Welt geschmeckt
haben, \bibleverse{6} wenn sie dann abgefallen sind, wieder zu erneuern
zur Buße, während sie sich selbst den Sohn Gottes wiederum kreuzigen und
zum Gespött machen! \bibleverse{7} Denn ein Erdreich, welches den Regen
trinkt, der sich öfters darüber ergießt und nützliches Gewächs
hervorbringt denen, für die es bebaut wird, empfängt Segen von Gott;
\bibleverse{8} welches aber Dornen und Disteln trägt, ist untauglich und
dem Fluche nahe, es wird zuletzt verbrannt. \bibleverse{9} Wir sind aber
überzeugt, Brüder, daß euer Zustand besser ist und dem Heile näher
kommt, obgleich wir so reden. \bibleverse{10} Denn Gott ist nicht
ungerecht, daß er eurer Arbeit und der Liebe vergäße, die ihr gegen
seinen Namen bewiesen habt, indem ihr den Heiligen dientet und noch
dienet. \bibleverse{11} Wir wünschen aber, daß jeder von euch denselben
Fleiß bis ans Ende beweise, entsprechend der vollen Gewißheit der
Hoffnung, \bibleverse{12} daß ihr ja nicht träge werdet, sondern
Nachfolger derer, welche durch Glauben und Geduld die Verheißungen
ererben. \bibleverse{13} Denn als Gott dem Abraham die Verheißung gab,
schwur er, da er bei keinem Größeren schwören konnte, bei sich selbst
\bibleverse{14} und sprach: ``Wahrlich, ich will dich reichlich segnen
und mächtig vermehren!'' \bibleverse{15} Und da er sich so geduldete,
erlangte er die Verheißung. \bibleverse{16} Menschen schwören ja bei dem
Größeren, und für sie ist der Eid das Ende alles Widerspruchs und dient
als Bürgschaft. \bibleverse{17} Darum ist Gott, als er den Erben der
Verheißung in noch stärkerem Maße beweisen wollte, wie unwandelbar sein
Ratschluß sei, mit einem Eid ins Mittel getreten, \bibleverse{18} damit
wir durch zwei unwandelbare Tatsachen, bei welchen Gott unmöglich lügen
konnte, einen starken Trost haben, wir, die wir unsere Zuflucht dazu
nehmen, die dargebotene Hoffnung zu ergreifen, \bibleverse{19} und
welche wir festhalten als einen sicheren und festen Anker der Seele, der
auch hineinreicht ins Innere, hinter den Vorhang, \bibleverse{20} wohin
als Vorläufer Jesus für uns eingegangen ist, nach der Ordnung
Melchisedeks Hoherpriester geworden in Ewigkeit.

\hypertarget{section-6}{%
\section{7}\label{section-6}}

\bibleverse{1} Denn dieser Melchisedek (König zu Salem, Priester Gottes,
des Allerhöchsten, der Abraham entgegenkam, als er von der Niederwerfung
der Könige zurückkehrte, und ihn segnete, \bibleverse{2} dem auch
Abraham den Zehnten von allem gab, der zunächst, wenn man seinen Namen
übersetzt, ``König der Gerechtigkeit'' heißt, dann aber auch ``König von
Salem'', das heißt König des Friedens, \bibleverse{3} ohne Vater, ohne
Mutter, ohne Geschlechtsregister, der weder Anfang der Tage noch Ende
des Lebens hat), der ist mit dem Sohne Gottes verglichen und bleibt
Priester für immerdar. \bibleverse{4} Sehet aber, wie groß der ist, dem
auch Abraham, der Patriarch, den Zehnten von der Beute gab!
\bibleverse{5} Zwar haben auch diejenigen von den Söhnen Levis, welche
das Priesteramt empfangen, den Auftrag, vom Volke den Zehnten zu nehmen
nach dem Gesetz, also von ihren Brüdern, obschon diese aus Abrahams
Lenden hervorgegangen sind; \bibleverse{6} der aber, der sein Geschlecht
nicht von ihnen herleitet, hat von Abraham den Zehnten genommen und den
gesegnet, der die Verheißungen hatte! \bibleverse{7} Nun ist es aber
unwidersprechlich so, daß das Geringere von dem Höheren gesegnet wird;
\bibleverse{8} und hier zwar nehmen sterbliche Menschen den Zehnten,
dort aber einer, von welchem bezeugt wird, daß er lebt. \bibleverse{9}
Und sozusagen ist durch Abraham auch für Levi, den Zehntenempfänger, der
Zehnte entrichtet worden; \bibleverse{10} denn er war noch in der Lende
des Vaters, als dieser mit Melchisedek zusammentraf! \bibleverse{11}
Wenn nun das Vollkommenheit wäre, was durch das levitische Priestertum
kam (denn unter diesem hat das Volk das Gesetz empfangen), wozu wäre es
noch nötig, daß ein anderer Priester ``nach der Ordnung Melchisedeks''
auftrete und nicht einer ``nach der Ordnung Aarons'' bezeichnet werde?
\bibleverse{12} Denn wenn das Priestertum verändert wird, so muß
notwendigerweise auch eine Änderung des Gesetzes erfolgen.
\bibleverse{13} Denn der, auf welchen sich jener Ausspruch bezieht,
gehört einem andern Stamme an, von welchem keiner des Altars gepflegt
hat; \bibleverse{14} denn es ist ja bekannt, daß unser Herr aus Juda
entsprossen ist, zu welchem Stamm Mose nichts auf Priester bezügliches
geredet hat. \bibleverse{15} Und noch viel klarer liegt die Sache, wenn
nach der Ähnlichkeit mit Melchisedek ein anderer Priester aufsteht,
\bibleverse{16} welcher es nicht nach dem Gesetz eines fleischlichen
Gebotes geworden ist, sondern nach der Kraft unauflöslichen Lebens;
\bibleverse{17} denn es wird bezeugt: ``Du bist Priester in Ewigkeit
nach der Ordnung Melchisedeks.'' \bibleverse{18} Da erfolgt ja sogar
eine Aufhebung des vorher gültigen Gebotes, seiner Schwachheit und
Nutzlosigkeit wegen \bibleverse{19} (denn das Gesetz hat nichts zur
Vollkommenheit gebracht), zugleich aber die Einführung einer besseren
Hoffnung, durch welche wir Gott nahen können. \bibleverse{20} Und um so
mehr, als dies nicht ohne Eidschwur geschah; denn jene sind ohne
Eidschwur Priester geworden, \bibleverse{21} dieser aber mit einem Eid
durch den, der zu ihm sprach: ``Der Herr hat geschworen und es wird ihn
nicht gereuen: Du bist Priester in Ewigkeit''; \bibleverse{22} um so
viel mehr ist Jesus auch eines bessern Bundes Bürge geworden.
\bibleverse{23} Und jene sind in großer Anzahl Priester geworden, weil
der Tod sie am Bleiben verhinderte; \bibleverse{24} er aber hat, weil er
in Ewigkeit bleibt, ein unübertragbares Priestertum. \bibleverse{25}
Daher kann er auch bis aufs äußerste die retten, welche durch ihn zu
Gott kommen, da er immerdar lebt, um für sie einzutreten!
\bibleverse{26} Denn ein solcher Hoherpriester geziemte uns, der heilig,
unschuldig, unbefleckt, von den Sündern abgesondert und höher als der
Himmel ist, \bibleverse{27} der nicht wie die Hohenpriester täglich
nötig hat, zuerst für die eigenen Sünden Opfer darzubringen, darnach für
die des Volkes; denn das hat er ein für allemal getan, indem er sich
selbst zum Opfer brachte. \bibleverse{28} Denn das Gesetz macht Menschen
zu Hohenpriestern, die mit Schwachheit behaftet sind, das Wort des
Eidschwurs aber, der nach der Zeit des Gesetzes erfolgte, den Sohn,
welcher für alle Ewigkeit vollendet ist.

\hypertarget{section-7}{%
\section{8}\label{section-7}}

\bibleverse{1} Die Hauptsache aber bei dem, was wir sagten, ist: Wir
haben einen solchen Hohenpriester, der zur Rechten des Thrones der
Majestät im Himmel sitzt, \bibleverse{2} einen Diener des Heiligtums und
der wahrhaftigen Stiftshütte, welche der Herr errichtet hat, und nicht
ein Mensch. \bibleverse{3} Denn jeder Hoherpriester wird eingesetzt, um
Gaben und Opfer darzubringen; daher muß auch dieser etwas haben, was er
darbringen kann. \bibleverse{4} Wenn er sich nun auf Erden befände, so
wäre er nicht einmal Priester, weil hier solche sind, die nach dem
Gesetz die Gaben opfern. \bibleverse{5} Diese dienen einem Abbild und
Schatten des Himmlischen, gemäß der Weisung, die Mose erhielt, als er
die Stiftshütte anfertigen wollte: ``Siehe zu'', hieß es, ``daß du alles
nach dem Vorbild machst, das dir auf dem Berge gezeigt worden ist!''
\bibleverse{6} Nun aber hat er einen um so bedeutenderen Dienst erlangt,
als er auch eines besseren Bundes Mittler ist, der auf besseren
Verheißungen ruht. \bibleverse{7} Denn wenn jener erste Bund tadellos
gewesen wäre, so würde nicht Raum für einen zweiten gesucht.
\bibleverse{8} Denn er tadelt sie doch, indem er spricht: ``Siehe, es
kommen Tage, spricht der Herr, da ich mit dem Hause Israel und mit dem
Hause Juda einen neuen Bund schließen werde; \bibleverse{9} nicht wie
der Bund, den ich mit ihren Vätern gemacht habe an dem Tage, als ich sie
bei der Hand nahm, um sie aus Ägyptenland zu führen (denn sie sind nicht
in meinem Bund geblieben, und ich ließ sie gehen, spricht der Herr),
\bibleverse{10} sondern das ist der Bund, den ich mit dem Hause Israel
machen will nach jenen Tagen, spricht der Herr: Ich will ihnen meine
Gesetze in den Sinn geben und sie in ihre Herzen schreiben, und ich will
ihr Gott sein, und sie sollen mein Volk sein. \bibleverse{11} Und es
wird keiner mehr seinen Mitbürger und keiner mehr seinen Bruder lehren
und sagen: Erkenne den Herrn! denn es werden mich alle kennen, vom
Kleinsten bis zum Größten unter ihnen; \bibleverse{12} denn ich werde
gnädig sein gegen ihre Ungerechtigkeiten und ihrer Sünden nicht mehr
gedenken.'' \bibleverse{13} Indem er sagt: ``Einen neuen'', hat er den
ersten für veraltet erklärt; was aber veraltet ist und sich überlebt
hat, das wird bald verschwinden.

\hypertarget{section-8}{%
\section{9}\label{section-8}}

\bibleverse{1} Es hatte nun zwar auch der erste Bund gottesdienstliche
Ordnungen und das irdische Heiligtum. \bibleverse{2} Denn es war ein
Zelt aufgerichtet, das vordere, in welchem sich der Leuchter und der
Tisch und die Schaubrote befanden; dieses wird das Heilige genannt.
\bibleverse{3} Hinter dem zweiten Vorhang aber befand sich das Zelt,
welches das Allerheiligste heißt; \bibleverse{4} zu diesem gehört der
goldene Räucheraltar und die Bundeslade, allenthalben mit Gold
überzogen, und in dieser war der goldene Krug mit dem Manna und die Rute
Aarons, die geblüht hatte, und die Tafeln des Bundes; \bibleverse{5}
oben über ihr aber die Cherubim der Herrlichkeit, die den Sühndeckel
überschatteten, worüber jetzt nicht im einzelnen zu reden ist.
\bibleverse{6} Da nun dieses so eingerichtet ist, betreten zwar die
Priester allezeit das vordere Zelt zur Verrichtung des Gottesdienstes;
\bibleverse{7} in das zweite Zelt aber geht einmal im Jahr nur der
Hohepriester, nicht ohne Blut, das er für sich selbst und für die
Versehen des Volkes darbringt. \bibleverse{8} Damit zeigt der heilige
Geist deutlich, daß der Weg zum Heiligtum noch nicht geoffenbart sei,
solange das vordere Zelt Bestand habe. \bibleverse{9} Dieses ist ein
Gleichnis für die gegenwärtige Zeit, da noch Gaben und Opfer dargebracht
werden, welche, was das Gewissen anbelangt, den nicht vollkommen machen
können, der den Gottesdienst verrichtet, \bibleverse{10} da er sich nur
auf Speisen und Getränke und verschiedene Waschungen bezieht, auf
fleischliche Verordnungen, welche bis zur Zeit der Zurechtbringung
auferlegt sind. \bibleverse{11} Als aber Christus kam als ein
Hoherpriester der zukünftigen Güter, ist er durch das größere und
vollkommenere Zelt, das nicht mit Händen gemacht, das heißt nicht von
dieser Schöpfung ist, \bibleverse{12} auch nicht durch das Blut von
Böcken und Kälbern, sondern durch sein eigenes Blut ein für allemal in
das Heiligtum eingegangen und hat eine ewige Erlösung erfunden.
\bibleverse{13} Denn wenn das Blut von Böcken und Stieren und die
Besprengung mit der Asche der jungen Kuh die Verunreinigten heiligt zu
leiblicher Reinigkeit, \bibleverse{14} wieviel mehr wird das Blut
Christi, der durch ewigen Geist sich selbst als ein tadelloses Opfer
Gott dargebracht hat, unser Gewissen reinigen von toten Werken, zu
dienen dem lebendigen Gott! \bibleverse{15} Darum ist er auch Mittler
eines neuen Bundes, damit (nach Verbüßung des Todes zur Erlösung von den
unter dem ersten Bunde begangenen Übertretungen) die Berufenen das
verheißene ewige Erbe empfingen. \bibleverse{16} Denn wo ein Testament
ist, da muß notwendig der Tod des Testators erwiesen werden;
\bibleverse{17} denn ein Testament tritt auf Todesfall hin in kraft, da
es keine Gültigkeit hat, solange der Testator lebt. \bibleverse{18}
Daher wurde auch der erste Bund nicht ohne Blut eingeweiht.
\bibleverse{19} Denn nachdem jedes einzelne Gebot nach dem Gesetz von
Mose dem ganzen Volke vorgelegt worden war, nahm er das Blut der Kälber
und Böcke mit Wasser und Purpurwolle und Ysop und besprengte sowohl das
Buch selbst als auch das ganze Volk, \bibleverse{20} wobei er sprach:
``Dies ist das Blut des Bundes, welchen Gott euch verordnet hat!''
\bibleverse{21} Auch das Zelt und alle Geräte des Gottesdienstes
besprengte er in gleicher Weise mit Blut; \bibleverse{22} und fast alles
wird nach dem Gesetz mit Blut gereinigt, und ohne Blutvergießen
geschieht keine Vergebung. \bibleverse{23} So ist es also notwendig, daß
die Abbilder der im Himmel befindlichen Dinge durch solches gereinigt
werden, die himmlischen Dinge selbst aber durch bessere Opfer als diese.
\bibleverse{24} Denn nicht in ein mit Händen gemachtes Heiligtum, in ein
Nachbild des wahrhaften, ist Christus eingegangen, sondern in den Himmel
selbst, um jetzt zu erscheinen vor dem Angesichte Gottes für uns;
\bibleverse{25} auch nicht, um sich selbst öfters zu opfern, gleichwie
der Hohepriester jedes Jahr mit fremdem Blut ins Heiligtum hineingeht;
denn sonst hätte er ja öfters leiden müssen von Grundlegung der Welt an!
\bibleverse{26} Nun aber ist er einmal gegen das Ende der Weltzeiten hin
erschienen zur Aufhebung der Sünde durch das Opfer seiner selbst;
\bibleverse{27} und so gewiß den Menschen bestimmt ist, einmal zu
sterben, darnach aber das Gericht, \bibleverse{28} so wird auch
Christus, nachdem er sich einmal zum Opfer dargebracht hat, um die
Sünden vieler auf sich zu nehmen, zum zweitenmal ohne Sünde denen
erscheinen, die auf ihn warten, zum Heil.

\hypertarget{section-9}{%
\section{10}\label{section-9}}

\bibleverse{1} Denn weil das Gesetz nur einen Schatten der zukünftigen
Güter hat, nicht das Ebenbild der Dinge selbst, so kann es auch mit den
gleichen alljährlichen Opfern, welche man immer wieder darbringt, die
Hinzutretenden niemals vollkommen machen! \bibleverse{2} Hätte man sonst
nicht aufgehört, Opfer darzubringen, wenn die, welche den Gottesdienst
verrichten, einmal gereinigt, kein Bewußtsein von Sünden mehr gehabt
hätten? \bibleverse{3} Statt dessen erfolgt durch dieselben nur alle
Jahre eine Erinnerung an die Sünden. \bibleverse{4} Denn unmöglich kann
Blut von Ochsen und Böcken Sünden wegnehmen! \bibleverse{5} Darum
spricht er bei seinem Eintritt in die Welt: ``Opfer und Gaben hast du
nicht gewollt; einen Leib aber hast du mir zubereitet. \bibleverse{6}
Brandopfer und Sündopfer gefallen dir nicht. \bibleverse{7} Da sprach
ich: Siehe, ich komme (in der Buchrolle steht von mir geschrieben), daß
ich tue, o Gott, deinen Willen.'' \bibleverse{8} Indem er oben sagt:
``Opfer und Gaben, Brandopfer und Sündopfer hast du nicht gewollt, sie
gefallen dir auch nicht'' (die nach dem Gesetz dargebracht werden),
\bibleverse{9} und dann fortfährt: ``Siehe, ich komme, zu tun deinen
Willen'', hebt er das erstere auf, um das andere einzusetzen.
\bibleverse{10} In diesem Willen sind wir geheiligt durch die
Aufopferung des Leibes Jesu Christi ein für allemal. \bibleverse{11} Und
jeder Priester steht da und verrichtet täglich den Gottesdienst und
bringt öfters dieselben Opfer dar, welche doch niemals Sünden wegnehmen
können; \bibleverse{12} dieser aber hat sich, nachdem er ein einziges
Opfer für die Sünden dargebracht hat, für immer zur Rechten Gottes
gesetzt \bibleverse{13} und wartet hinfort, bis alle seine Feinde als
Schemel seiner Füße hingelegt werden; \bibleverse{14} denn mit einem
einzigen Opfer hat er die, welche geheiligt werden, für immer vollendet.
\bibleverse{15} Das bezeugt uns aber auch der heilige Geist;
\bibleverse{16} denn, nachdem gesagt worden ist: ``Das ist der Bund, den
ich mit ihnen schließen will nach diesen Tagen'', spricht der Herr:
``Ich will meine Gesetze in ihre Herzen geben und sie in ihre Sinne
schreiben, \bibleverse{17} und ihrer Sünden und ihrer Ungerechtigkeiten
will ich nicht mehr gedenken.'' \bibleverse{18} Wo aber Vergebung für
diese ist, da ist kein Opfer mehr für Sünde. \bibleverse{19} Da wir nun,
ihr Brüder, kraft des Blutes Jesu Freimütigkeit haben zum Eingang in das
Heiligtum, \bibleverse{20} welchen er uns eingeweiht hat als neuen und
lebendigen Weg durch den Vorhang hindurch, das heißt, durch sein
Fleisch, \bibleverse{21} und einen so großen Priester über das Haus
Gottes haben, \bibleverse{22} so lasset uns hinzutreten mit wahrhaftigem
Herzen, in voller Glaubenszuversicht, durch Besprengung der Herzen los
vom bösen Gewissen und gewaschen am Leibe mit reinem Wasser.
\bibleverse{23} Lasset uns festhalten am Bekenntnis der Hoffnung, ohne
zu wanken (denn er ist treu, der die Verheißung gegeben hat);
\bibleverse{24} und lasset uns aufeinander achten, uns gegenseitig
anzuspornen zur Liebe und zu guten Werken, \bibleverse{25} indem wir
unsere eigene Versammlung nicht verlassen, wie etliche zu tun pflegen,
sondern einander ermahnen, und das um so viel mehr, als ihr den Tag
herannahen sehet! \bibleverse{26} Denn wenn wir freiwillig sündigen,
nachdem wir die Erkenntnis der Wahrheit empfangen haben, so bleibt für
Sünden kein Opfer mehr übrig, \bibleverse{27} sondern ein schreckliches
Erwarten des Gerichts und Feuereifers, der die Widerspenstigen verzehren
wird. \bibleverse{28} Wenn jemand das Gesetz Moses mißachtet, muß er
ohne Barmherzigkeit auf die Aussage von zwei oder drei Zeugen hin
sterben, \bibleverse{29} wieviel ärgerer Strafe, meinet ihr, wird
derjenige schuldig erachtet werden, der den Sohn Gottes mit Füßen
getreten und das Blut des Bundes, durch welches er geheiligt wurde, für
gemein geachtet und den Geist der Gnade geschmäht hat? \bibleverse{30}
Denn wir kennen den, der da sagt: ``Die Rache ist mein; ich will
vergelten!'' und wiederum: ``Der Herr wird sein Volk richten''.
\bibleverse{31} Schrecklich ist es, in die Hände des lebendigen Gottes
zu fallen! \bibleverse{32} Gedenket aber der früheren Tage, in welchen
ihr nach eurer Erleuchtung unter Leiden viel Kampf erduldet habt,
\bibleverse{33} da ihr teils selbst Schmähungen und Drangsalen
öffentlich preisgegeben waret, teils mit denen Gemeinschaft hattet,
welche so behandelt wurden; \bibleverse{34} denn ihr habt den Gefangenen
Teilnahme bewiesen und den Raub eurer Güter mit Freuden hingenommen, in
der Erkenntnis, daß ihr selbst ein besseres und bleibendes Gut besitzet.
\bibleverse{35} So werfet nun eure Freimütigkeit nicht weg, welche eine
große Belohnung hat! \bibleverse{36} Denn Ausdauer tut euch not, damit
ihr nach Erfüllung des göttlichen Willens die Verheißung erlanget.
\bibleverse{37} Denn noch eine kleine, ganz kleine Weile, so wird
kommen, der da kommen soll und nicht verziehen. \bibleverse{38} ``Mein
Gerechter aber wird aus Glauben leben; zieht er sich aber aus Feigheit
zurück, so wird meine Seele kein Wohlgefallen an ihm haben.''
\bibleverse{39} Wir aber sind nicht von denen, die feige zurückweichen
zum Verderben, sondern die da glauben zur Rettung der Seele.

\hypertarget{section-10}{%
\section{11}\label{section-10}}

\bibleverse{1} Es ist aber der Glaube ein Beharren auf dem, was man
hofft, eine Überzeugung von Tatsachen, die man nicht sieht.
\bibleverse{2} Durch solchen haben die Alten ein gutes Zeugnis erhalten.
\bibleverse{3} Durch Glauben erkennen wir, daß die Weltzeiten durch
Gottes Wort bereitet worden sind, also das, was man sieht, aus
Unsichtbarem entstanden ist. \bibleverse{4} Durch Glauben brachte Abel
Gott ein größeres Opfer dar als Kain; durch ihn erhielt er das Zeugnis,
daß er gerecht sei, indem Gott über seine Gaben Zeugnis ablegte, und
durch ihn redet er noch, wiewohl er gestorben ist. \bibleverse{5} Durch
Glauben wurde Enoch entrückt, so daß er den Tod nicht sah, und er wurde
nicht mehr gefunden, weil Gott ihn entrückt hatte; denn vor seiner
Entrückung wurde ihm das Zeugnis gegeben, daß er Gott wohlgefallen habe.
\bibleverse{6} Ohne Glauben aber ist es unmöglich, ihm wohlzugefallen;
denn wer zu Gott kommen soll, muß glauben, daß er ist und die, welche
ihn suchen, belohnen wird. \bibleverse{7} Durch Glauben baute Noah, als
er betreffs dessen, was man noch nicht sah, eine Weissagung empfangen
hatte, in ehrerbietiger Scheu eine Arche zur Rettung seines Hauses;
durch ihn verurteilte er die Welt und wurde ein Erbe der
Glaubensgerechtigkeit. \bibleverse{8} Durch Glauben gehorchte Abraham,
als er berufen wurde, nach einem Ort auszuziehen, den er zum Erbteil
empfangen sollte; und er zog aus, ohne zu wissen, wohin er komme.
\bibleverse{9} Durch Glauben siedelte er sich im Lande der Verheißung
an, als in einem fremden, und wohnte in Zelten mit Isaak und Jakob, den
Miterben derselben Verheißung; \bibleverse{10} denn er wartete auf die
Stadt, welche die Grundfesten hat, deren Baumeister und Schöpfer Gott
ist. \bibleverse{11} Durch Glauben erhielt auch Sara Kraft zur Gründung
einer Nachkommenschaft trotz ihres Alters, weil sie den für treu
achtete, der es verheißen hatte. \bibleverse{12} Darum sind auch von
einem einzigen, und zwar erstorbenen Leibe Kinder entsprossen wie die
Sterne des Himmels an Menge und wie der Sand am Gestade des Meeres, der
nicht zu zählen ist. \bibleverse{13} Diese alle sind im Glauben
gestorben, ohne das Verheißene empfangen zu haben, sondern sie haben es
nur von ferne gesehen und begrüßt und bekannt, daß sie Fremdlinge und
Pilgrime seien auf Erden; \bibleverse{14} denn die solches sagen, zeigen
damit an, daß sie ein Vaterland suchen. \bibleverse{15} Und hätten sie
dabei an jenes gedacht, von welchem sie ausgezogen waren, so hätten sie
ja Zeit gehabt zurückzukehren; \bibleverse{16} nun aber trachten sie
nach einem besseren, nämlich einem himmlischen. Darum schämt sich Gott
nicht, ihr Gott zu heißen; denn er hat ihnen eine Stadt zubereitet.
\bibleverse{17} Durch Glauben brachte Abraham den Isaak dar, als er
versucht wurde, und opferte den Eingeborenen, er, der die Verheißungen
empfangen hatte, \bibleverse{18} zu welchem gesagt worden war: ``In
Isaak soll dir ein Same berufen werden.'' \bibleverse{19} Er zählte eben
darauf, daß Gott imstande sei, auch von den Toten zu erwecken, weshalb
er ihn auch, wie durch ein Gleichnis, wieder erhielt. \bibleverse{20}
Durch Glauben segnete auch Isaak den Jakob und Esau betreffs der
zukünftigen Dinge. \bibleverse{21} Durch Glauben segnete Jakob bei
seinem Sterben einen jeden der Söhne Josephs und betete an, auf seinen
Stab gestützt. \bibleverse{22} Durch Glauben gedachte Joseph bei seinem
Ende des Auszuges der Kinder Israel und gab Befehl wegen seiner Gebeine.
\bibleverse{23} Durch Glauben wurde Mose nach seiner Geburt von seinen
Eltern drei Monate lang verborgen gehalten, weil sie sahen, daß er ein
schönes Kind war, und sie des Königs Gebot nicht fürchteten.
\bibleverse{24} Durch Glauben weigerte sich Mose, als er groß geworden
war, ein Sohn der Tochter des Pharao zu heißen. \bibleverse{25} Er
wollte lieber mit dem Volke Gottes Ungemach leiden, als zeitliche
Ergötzung der Sünde haben, \bibleverse{26} da er die Schmach Christi für
größeren Reichtum hielt als die Schätze Ägyptens; denn er sah die
Belohnung an. \bibleverse{27} Durch Glauben verließ er Ägypten, ohne den
Grimm des Königs zu fürchten; denn er hielt sich an den Unsichtbaren,
als sähe er ihn. \bibleverse{28} Durch Glauben hat er das Passah
veranstaltet und das Besprengen mit Blut, damit der Würgengel ihre
Erstgeborenen nicht anrühre. \bibleverse{29} Durch Glauben gingen sie
durch das Rote Meer wie durch trockenes Land; während die Ägypter, als
sie das auch versuchten, ertranken. \bibleverse{30} Durch Glauben fielen
die Mauern von Jericho, nachdem sie sieben Tage umzogen worden waren.
\bibleverse{31} Durch Glauben kam Rahab, die Dirne, nicht mit den
Ungehorsamen um, weil sie die Kundschafter mit Frieden aufgenommen
hatte. \bibleverse{32} Und was soll ich noch sagen? Die Zeit würde mir
fehlen, wenn ich erzählen wollte von Gideon, Barak, Simson, Jephta,
David und Samuel und den Propheten, \bibleverse{33} welche durch Glauben
Königreiche bezwangen, Gerechtigkeit wirkten, Verheißungen erlangten,
der Löwen Rachen verstopften. \bibleverse{34} Sie haben die Gewalt des
Feuers ausgelöscht, sind des Schwertes Schärfe entronnen, von
Schwachheit zu Kraft gekommen, stark geworden im Streit, haben der
Fremden Heere in die Flucht gejagt. \bibleverse{35} Frauen erhielten
ihre Toten durch Auferstehung wieder; andere aber ließen sich martern
und nahmen die Befreiung nicht an, um eine bessere Auferstehung zu
erlangen. \bibleverse{36} Andere erfuhren Spott und Geißelung, dazu
Ketten und Gefängnis; \bibleverse{37} sie wurden gesteinigt, verbrannt,
zersägt, erlitten den Tod durchs Schwert, zogen umher in Schafspelzen
und Ziegenfellen, erlitten Mangel, Bedrückung, Mißhandlung;
\bibleverse{38} sie, derer die Welt nicht wert war, irrten umher in
Wüsten und Gebirgen, in Höhlen und Löchern der Erde. \bibleverse{39} Und
diese alle, obschon sie hinsichtlich des Glaubens ein gutes Zeugnis
erhielten, haben das Verheißene nicht erlangt, \bibleverse{40} weil Gott
für uns etwas Besseres vorgesehen hat, damit sie nicht ohne uns
vollendet würden.

\hypertarget{section-11}{%
\section{12}\label{section-11}}

\bibleverse{1} Darum auch wir, weil wir eine solche Wolke von Zeugen um
uns haben, lasset uns jede Last und die uns so leicht umstrickende Sünde
ablegen und mit Ausdauer die Rennbahn durchlaufen, welche vor uns liegt,
\bibleverse{2} im Aufblick auf Jesus, den Anfänger und Vollender des
Glaubens, welcher für die vor ihm liegende Freude das Kreuz erduldete,
die Schande nicht achtete und sich zur Rechten des Thrones Gottes
gesetzt hat. \bibleverse{3} Achtet auf ihn, der solchen Widerspruch von
den Sündern gegen sich erduldet hat, damit ihr nicht müde werdet und den
Mut verliert! \bibleverse{4} Ihr habt noch nicht bis aufs Blut
widerstanden im Kampf wider die Sünde \bibleverse{5} und habt das
Trostwort vergessen, womit ihr als Söhne angeredet werdet: ``Mein Sohn,
achte nicht gering die Züchtigung des Herrn und verzage nicht, wenn du
von ihm gestraft wirst! \bibleverse{6} Denn welchen der Herr lieb hat,
den züchtigt er, und er geißelt einen jeglichen Sohn, den er aufnimmt.''
\bibleverse{7} Wenn ihr Züchtigung erduldet, so behandelt euch Gott ja
als Söhne; denn wo ist ein Sohn, den der Vater nicht züchtigt?
\bibleverse{8} Seid ihr aber ohne Züchtigung, derer sie alle teilhaftig
geworden sind, so seid ihr ja unecht und keine Söhne! \bibleverse{9}
Sodann hatten wir auch unsere leiblichen Väter zu Zuchtmeistern und
scheuten sie; sollten wir jetzt nicht vielmehr dem Vater der Geister
untertan sein und leben? \bibleverse{10} Denn jene haben uns für wenige
Tage gezüchtigt, nach ihrem Gutdünken; er aber zu unsrem Besten, damit
wir seiner Heiligkeit teilhaftig werden. \bibleverse{11} Alle Züchtigung
aber, wenn sie da ist, dünkt uns nicht zur Freude, sondern zur
Traurigkeit zu dienen; hernach aber gibt sie eine friedsame Frucht der
Gerechtigkeit denen, die dadurch geübt sind. \bibleverse{12} Darum
``recket wieder aus die schlaff gewordenen Hände und die erlahmten
Knie'' \bibleverse{13} und ``tut gerade Tritte mit euren Füßen'', damit
das Lahme nicht abweiche, sondern vielmehr geheilt werde!
\bibleverse{14} Jaget nach dem Frieden mit jedermann und der Heiligung,
ohne welche niemand den Herrn sehen wird! \bibleverse{15} Und sehet
darauf, daß nicht jemand die Gnade Gottes versäume, daß nicht etwa eine
bittere Wurzel aufwachse und Störungen verursache und viele dadurch
befleckt werden, \bibleverse{16} daß nicht jemand ein Unzüchtiger oder
ein gemeiner Mensch sei wie Esau, der um einer Speise willen sein
Erstgeburtsrecht verkaufte. \bibleverse{17} Denn ihr wisset, daß er
nachher, als er den Segen ererben wollte, verworfen wurde, denn er fand
keinen Raum zur Buße, obschon er den Segen mit Tränen suchte.
\bibleverse{18} Denn ihr seid nicht zu dem Berg gekommen, den man
anrühren konnte, und zu dem glühenden Feuer, noch zu dem Dunkel, der
Finsternis und dem Ungewitter, \bibleverse{19} noch zu dem Schall der
Posaune und der Stimme der Worte, bei der die Zuhörer sich erbaten, daß
nicht weiter zu ihnen geredet werde; denn sie ertrugen nicht, was
befohlen war: \bibleverse{20} ``Und wenn ein Tier den Berg berührt, soll
es gesteinigt werden!'' \bibleverse{21} und so schrecklich war die
Erscheinung, daß Mose sprach: ``Ich bin erschrocken und zittere!''
\bibleverse{22} sondern ihr seid gekommen zu dem Berge Zion und zu der
Stadt des lebendigen Gottes, dem himmlischen Jerusalem, und zu
Zehntausenden von Engeln, \bibleverse{23} zur Festversammlung und
Gemeinde der Erstgeborenen, die im Himmel angeschrieben sind, und zu
Gott, dem Richter über alle, und zu den Geistern der vollendeten
Gerechten \bibleverse{24} und zu Jesus, dem Mittler des neuen Bundes,
und zu dem Blut der Besprengung, das Besseres redet als Abels Blut.
\bibleverse{25} Sehet zu, daß ihr den nicht abweiset, der da redet! Denn
wenn jene nicht entflohen sind, die es sich verbaten, als er auf Erden
redete, wieviel weniger wir, wenn wir uns von dem abwenden, der es vom
Himmel herab tut, \bibleverse{26} dessen Stimme damals die Erde bewegte;
nun aber hat er verheißen: ``Noch einmal will ich bewegen, nicht allein
die Erde, sondern auch den Himmel!'' \bibleverse{27} Dieses ``noch
einmal'' deutet hin auf die Veränderung des Beweglichen, weil
Erschaffenen, damit das Unbewegliche bleibe. \bibleverse{28} Darum, weil
wir ein unbewegliches Reich empfangen, lasset uns Dank beweisen, durch
welchen wir Gott wohlgefällig dienen wollen mit Scheu und Furcht!
\bibleverse{29} Denn auch unser Gott ist ein verzehrendes Feuer.

\hypertarget{section-12}{%
\section{13}\label{section-12}}

\bibleverse{1} Die Bruderliebe soll bleiben! \bibleverse{2} Gastfrei zu
sein vergesset nicht; denn dadurch haben etliche ohne ihr Wissen Engel
beherbergt. \bibleverse{3} Gedenket der Gefangenen als Mitgefangene und
derer, die Ungemach leiden, als solche, die selbst auch noch im Leibe
leben. \bibleverse{4} Die Ehe ist von allen in Ehren zu halten und das
Ehebett unbefleckt; denn Hurer und Ehebrecher wird Gott richten!
\bibleverse{5} Der Wandel sei ohne Geiz! Begnüget euch mit dem
Vorhandenen! Denn er selbst hat gesagt: ``Ich will dich nicht verlassen
noch versäumen!'' \bibleverse{6} Also daß wir getrost sagen mögen: ``Der
Herr ist mein Helfer; ich fürchte mich nicht! Was können Menschen mir
tun?'' \bibleverse{7} Gedenket eurer Führer, die euch das Wort Gottes
gesagt haben; schauet das Ende ihres Wandels an und ahmet ihren Glauben
nach! \bibleverse{8} Jesus Christus ist gestern und heute und derselbe
auch in Ewigkeit! \bibleverse{9} Lasset euch nicht von mancherlei und
fremden Lehren umhertreiben; denn es ist gut, daß das Herz durch Gnade
befestigt werde, nicht durch Speisen, mit welchen sich abzugeben noch
niemand Nutzen gebracht hat. \bibleverse{10} Es gibt einen Altar, von
welchem die Diener der Stiftshütte nicht essen dürfen. \bibleverse{11}
Denn von den Tieren, deren Blut für die Sünde durch den Hohenpriester
ins Allerheiligste getragen wird, werden die Leiber außerhalb des Lagers
verbrannt. \bibleverse{12} Darum hat auch Jesus, um das Volk durch sein
eigenes Blut zu heiligen, außerhalb des Tores gelitten. \bibleverse{13}
So lasset uns nun zu ihm hinausgehen, außerhalb des Lagers, und seine
Schmach tragen! \bibleverse{14} Denn wir haben hier keine bleibende
Stadt, sondern suchen die zukünftige. \bibleverse{15} Durch ihn lasset
uns nun Gott allezeit ein Opfer des Lobes darbringen, das ist die
``Frucht der Lippen'', die seinen Namen bekennen! \bibleverse{16}
Wohlzutun und mitzuteilen vergesset nicht; denn solche Opfer gefallen
Gott wohl! \bibleverse{17} Gehorchet euren Führern und folget ihnen;
denn sie wachen über eure Seelen als solche, die Rechenschaft ablegen
sollen, damit sie das mit Freuden tun mögen und nicht mit Seufzen; denn
das wäre euch zum Schaden! \bibleverse{18} Betet für uns! Denn wir sind
überzeugt, ein gutes Gewissen zu haben, da wir uns allenthalben eines
anständigen Lebenswandels befleißigen. \bibleverse{19} Um so mehr aber
ermahne ich euch, solches zu tun, damit ich euch desto bälder
wiedergeschenkt werde. \bibleverse{20} Der Gott des Friedens aber, der
den großen Hirten der Schafe von den Toten ausgeführt hat, mit dem Blut
eines ewigen Bundes, unsren Herrn Jesus, \bibleverse{21} der rüste euch
mit allem Guten aus, seinen Willen zu tun, indem er selbst in euch
schafft, was vor ihm wohlgefällig ist, durch Jesus Christus. Ihm sei die
Ehre von Ewigkeit zu Ewigkeit! Amen. \bibleverse{22} Ich ermahne euch
aber, ihr Brüder, nehmet das Wort der Ermahnung an! Denn ich habe euch
mit kurzen Worten geschrieben. \bibleverse{23} Wisset, daß unser Bruder
Timotheus freigelassen worden ist; wenn er bald kommt, will ich euch mit
ihm besuchen. \bibleverse{24} Grüßet alle eure Führer und alle Heiligen!
Es grüßen euch die von Italien! \bibleverse{25} Die Gnade sei mit euch
allen!
