\hypertarget{section}{%
\section{1}\label{section}}

\bibleverse{1} Die Reden des Predigers, des Sohnes Davids, des Königs zu
Jerusalem: \bibleverse{2} O Eitelkeit der Eitelkeiten! spricht der
Prediger; o Eitelkeit der Eitelkeiten! Alles ist eitel! \bibleverse{3}
Was bleibt dem Menschen von all seiner Mühe, womit er sich abmüht unter
der Sonne? \bibleverse{4} Ein Geschlecht geht, das andere kommt; die
Erde aber bleibt ewiglich! \bibleverse{5} Die Sonne geht auf, und die
Sonne geht unter und eilt an ihren Ort, wo sie wieder aufgehen soll.
\bibleverse{6} Der Wind weht gegen Süden und wendet sich nach Norden; es
weht und wendet sich der Wind, und weil er sich wendet, so kehrt der
Wind wieder zurück. \bibleverse{7} Alle Flüsse laufen ins Meer, und das
Meer wird doch nicht voll; an den Ort, wohin die Flüsse einmal laufen,
laufen sie immer wieder. \bibleverse{8} Alle Worte sind unzulänglich;
der Mensch kann nicht genug reden, das Auge sieht sich nicht satt, und
das Ohr hört nie genug. \bibleverse{9} Was ist gewesen? Das, was sein
wird! Und was hat man gemacht? Das, was man machen wird! Und es gibt
nichts Neues unter der Sonne. \bibleverse{10} Kann man von irgend etwas
sagen: ``Siehe, das ist neu''? Längst schon war es in unbekannten
Zeiten, die vor uns gewesen sind! \bibleverse{11} Man gedenkt eben des
Frühern nicht mehr, und auch des Spätern, das noch kommen soll, wird man
nicht mehr gedenken bei denen, die noch später sein werden!
\bibleverse{12} Ich, der Prediger, war König über Israel zu Jerusalem.
\bibleverse{13} Ich ergab mein Herz, die Weisheit zu befragen und mich
bei ihr zu erkundigen über alles, was unter dem Himmel getan wird. Das
ist eine leidige Mühe, die Gott den Menschenkindern gegeben hat, daß sie
sich damit plagen sollen. \bibleverse{14} Ich betrachtete alle Werke,
die unter der Sonne gemacht werden, und siehe, es war alles eitel und
ein Haschen nach Wind! \bibleverse{15} Krumme Sachen kann man nicht
gerade machen, und die, welche mangeln, kann man nicht zählen.
\bibleverse{16} Da redete ich mit meinem Herzen und sprach: Siehe, nun
habe ich mehr und größere Weisheit als alle, die vor mir über Jerusalem
waren, und mein Herz hat viel Weisheit und Wissenschaft gesehen;
\bibleverse{17} und ich habe mein Herz ergeben, die Weisheit kennen zu
lernen, desgleichen Übermut und Unverstand; aber ich habe auch das als
ein Haschen nach Wind erkannt; \bibleverse{18} denn wo viel Weisheit
ist, da ist auch viel Ärger, und wer sein Wissen mehrt, der mehrt seinen
Schmerz.

\hypertarget{section-1}{%
\section{2}\label{section-1}}

\bibleverse{1} Ich sprach zu meinem Herzen: Komm, wir wollen es mit der
Freude versuchen, und du sollst es gut haben! Aber siehe, auch das war
vergeblich! \bibleverse{2} Zum Lachen sprach ich: Du bist toll! Und zur
Freude: Was tut diese da? \bibleverse{3} Ich gedachte in meinem Herzen,
mein Fleisch an den Wein zu gewöhnen, doch so, daß mein Herz in Weisheit
die Leitung behielte, und so die Torheit zu ergreifen, bis daß ich sähe,
ob das gut sei, was die Menschenkinder ihr ganzes Leben lang unter dem
Himmel tun. \bibleverse{4} Ich unternahm große Werke, baute mir Häuser,
pflanzte mir Weinberge. \bibleverse{5} Ich legte mir Gärten und Pärke an
und pflanzte darin allerlei Fruchtbäume. \bibleverse{6} Ich machte mir
Wasserteiche, um daraus den sprossenden Baumwald zu tränken.
\bibleverse{7} Ich kaufte Knechte und Mägde und hatte auch solche, die
in meinem eigenen Hause geboren waren; so hatte ich auch größere Rinder
und Schafherden als alle, die vor mir zu Jerusalem gewesen waren.
\bibleverse{8} Ich sammelte mir Silber und Gold, Schätze der Könige und
Länder; ich verschaffte mir Sänger und Sängerinnen und, was die
Menschenkinder ergötzt, eine Gattin und Gattinnen. \bibleverse{9} Und
ich ward größer und reicher als alle, die vor mir zu Jerusalem gewesen
waren; auch blieb meine Weisheit bei mir. \bibleverse{10} Und ich
versagte meinen Augen nichts von allem, was sie wünschten; ich hielt
mein Herz von keiner Freude zurück; denn mein Herz hatte Freude von all
meiner Mühe, und das war mein Teil von aller meiner Mühe.
\bibleverse{11} Als ich mich aber umsah nach all meinen Werken, die
meine Hände gemacht hatten, und nach der Mühe, die ich mir gegeben
hatte, um sie zu vollbringen, siehe, da war alles eitel und ein Haschen
nach Wind und nichts Bleibendes unter der Sonne! \bibleverse{12} Und ich
wandte mich zur Betrachtung der Weisheit, des Übermuts und der Torheit;
denn was wird der Mensch tun, der nach dem König kommt? Das, was man
längst getan hat! \bibleverse{13} Und ich habe eingesehen, daß die
Weisheit einen so großen Vorzug hat vor der Torheit wie das Licht vor
der Finsternis. \bibleverse{14} Der Weise hat seine Augen im Kopf; der
Tor aber wandelt in der Finsternis. Zugleich erkannte ich jedoch, daß
ihnen allen das gleiche Schicksal begegnet. \bibleverse{15} Da sprach
ich in meinem Herzen: Wenn mir doch das gleiche Schicksal begegnet wie
dem Toren, warum bin ich denn so überaus weise geworden? Und ich sprach
in meinem Herzen: Auch das ist eitel! \bibleverse{16} Denn des Weisen
wird ebenso wenig ewiglich gedacht wie des Toren, weil in den künftigen
Tagen längst alles vergessen sein wird; und wie stirbt doch der Weise
samt dem Toren dahin! \bibleverse{17} Da haßte ich das Leben; denn mir
mißfiel das Tun, das unter der Sonne geschieht; denn es ist alles eitel
und ein Haschen nach Wind. \bibleverse{18} Ich haßte auch alle meine
Arbeit, womit ich mich abgemüht hatte unter der Sonne, weil ich sie dem
Menschen überlassen soll, der nach mir kommt. \bibleverse{19} Und wer
weiß, ob derselbe weise sein wird oder ein Tor? Und doch wird er über
all das Macht bekommen, was ich mit Mühe und Weisheit erarbeitet habe
unter der Sonne. Auch das ist eitel! \bibleverse{20} Da wandte ich mich,
mein Herz verzweifeln zu lassen an all der Mühe, womit ich mich abgemüht
hatte unter der Sonne. \bibleverse{21} Denn das Vermögen, das einer sich
erworben hat mit Weisheit, Verstand und Geschick, das muß er einem
andern zum Erbteil geben, der sich nicht darum bemüht hat; das ist auch
eitel und ein großes Unglück! \bibleverse{22} Denn was hat der Mensch
von all seiner Mühe und dem Dichten seines Herzens, womit er sich abmüht
unter der Sonne? \bibleverse{23} Denn er plagt sich täglich mit Kummer
und Verdruß, sogar in der Nacht hat sein Herz keine Ruhe; auch das ist
eitel! \bibleverse{24} Es gibt nichts Besseres für den Menschen, als daß
er esse und trinke und seine Seele Gutes genießen lasse in seiner
Mühsal! Doch habe ich gesehen, daß auch das von der Hand Gottes kommt.
\bibleverse{25} Denn wer kann essen und wer kann genießen ohne Ihn?
\bibleverse{26} Denn dem Menschen, der Ihm wohlgefällt, gibt Er Weisheit
und Erkenntnis und Freude; aber dem Sünder gibt er Plage, daß er sammle
und zusammenscharre, um es dem zu geben, welcher Gott gefällt. Auch das
ist eitel und ein Haschen nach Wind.

\hypertarget{section-2}{%
\section{3}\label{section-2}}

\bibleverse{1} Alles hat seine Zeit und jegliches Vornehmen unter dem
Himmel seine Stunde. \bibleverse{2} Geborenwerden hat seine Zeit, und
Sterben hat seine Zeit; Pflanzen hat seine Zeit, und Gepflanztes
ausreuten hat seine Zeit; \bibleverse{3} Töten hat seine Zeit, und
Heilen hat seine Zeit; Zerstören hat seine Zeit, und Bauen hat seine
Zeit; \bibleverse{4} Weinen hat seine Zeit, und Lachen hat seine Zeit;
Klagen hat seine Zeit, und Tanzen hat seine Zeit; \bibleverse{5} Steine
schleudern hat seine Zeit, und Steine sammeln hat seine Zeit; Umarmen
hat seine Zeit, und sich der Umarmung enthalten hat auch seine Zeit;
\bibleverse{6} Suchen hat seine Zeit, und Verlieren hat seine Zeit;
Aufbewahren hat seine Zeit, und Wegwerfen hat seine Zeit; \bibleverse{7}
Zerreißen hat seine Zeit, und Flicken hat seine Zeit; Schweigen hat
seine Zeit, und Reden hat seine Zeit; \bibleverse{8} Lieben hat seine
Zeit, und Hassen hat seine Zeit; Krieg hat seine Zeit, und Friede hat
seine Zeit. \bibleverse{9} Was hat nun der, welcher solches tut, für
einen Gewinn bei dem, womit er sich abmüht? \bibleverse{10} Ich habe die
Plage gesehen, welche Gott den Menschenkindern gegeben hat, sich damit
abzuplagen. \bibleverse{11} Er hat alles schön gemacht zu seiner Zeit,
auch die Ewigkeit hat er in ihr Herz gelegt, da sonst der Mensch das
Werk, welches Gott getan hat, nicht von Anfang bis zu Ende herausfinden
könnte. \bibleverse{12} Ich habe erkannt, daß es nichts Besseres gibt
unter ihnen, als sich zu freuen und Gutes zu tun in seinem Leben;
\bibleverse{13} und wenn irgend ein Mensch ißt und trinkt und Gutes
genießt bei all seiner Mühe, so ist das auch eine Gabe Gottes.
\bibleverse{14} Ich habe erkannt, daß alles, was Gott tut, für ewig ist;
es ist nichts hinzuzufügen und nichts davon wegzunehmen; und Gott hat es
so gemacht, daß man sich vor ihm fürchte. \bibleverse{15} Was ist
geschehen? Was längst schon war! Und was geschehen soll, das ist längst
gewesen; und Gott sucht das Vergangene wieder hervor. \bibleverse{16}
Und weiter sah ich unter der Sonne eine Stätte des Gerichts, da
herrschte Ungerechtigkeit, und eine Stätte des Rechts, da herrschte
Bosheit. \bibleverse{17} Da sprach ich in meinem Herzen: Gott wird den
Gerechten wie den Gottlosen richten; denn er hat für jegliches Vorhaben
und für jegliches Werk eine Zeit festgesetzt! \bibleverse{18} Ich sprach
in meinem Herzen: Es ist wegen der Menschenkinder, daß Gott sie prüft
und damit sie einsehen, daß sie in sich selbst dem Vieh gleichen.
\bibleverse{19} Denn das Schicksal der Menschenkinder und das Schicksal
des Viehs ist ein und dasselbe: die einen sterben so gut wie die andern,
und sie haben alle einerlei Odem, und der Mensch hat nichts vor dem Vieh
voraus; denn es ist alles eitel. \bibleverse{20} Alle gehen an einen
Ort: alles ist aus dem Staube geworden, und alles kehrt auch wieder zum
Staub zurück. \bibleverse{21} Wer weiß, ob der Geist des Menschen
aufwärts steigt, der Geist des Tieres aber abwärts zur Erde fährt?
\bibleverse{22} So sah ich denn, daß es nichts Besseres gibt, als daß
der Mensch sich freue an seinen Werken; denn das ist sein Teil! Denn wer
will ihn dahin bringen, daß er auf das sehe, was nach ihm sein wird?

\hypertarget{section-3}{%
\section{4}\label{section-3}}

\bibleverse{1} Und wiederum sah ich alle Bedrückungen, die verübt werden
unter der Sonne; und siehe, da flossen Tränen von Unterdrückten, die
keinen Tröster hatten; und weil die Hand ihrer Unterdrücker so stark
war, konnte sie niemand trösten. \bibleverse{2} Da pries ich die Toten,
die längst gestorben sind, glücklicher als die Lebenden, die jetzt noch
am Leben sind. \bibleverse{3} Aber besser als beide ist der daran,
welcher noch gar nicht geboren ist, weil er das leidige Tun, das unter
der Sonne geschieht, gar nicht gesehen hat. \bibleverse{4} Ich sah auch,
daß alle Mühe und alles Gelingen im Geschäft nur den Neid des einen
gegen den andern weckt; und auch das ist eitel und ein Haschen nach
Wind! \bibleverse{5} Der Tor faltet seine Hände und verzehrt sein
eigenes Fleisch. \bibleverse{6} Besser eine Handvoll Ruhe, als beide
Fäuste voll Mühsal und Haschen nach Wind! \bibleverse{7} Und wiederum
sah ich Eitelkeit unter der Sonne: \bibleverse{8} Da steht einer ganz
allein, hat weder Sohn noch Bruder, und doch ist seines Arbeitens kein
Ende, und er sieht nie Reichtum genug und denkt nicht: Für wen mühe ich
mich doch ab und enthalte meiner Seele das Beste vor? Auch das ist
nichtig und eine leidige Plage. \bibleverse{9} Es ist besser, man sei zu
zweien, als allein; denn der Arbeitslohn fällt um so besser aus.
\bibleverse{10} Denn wenn sie fallen, so hilft der eine dem andern auf;
wehe aber dem, der allein ist, wenn er fällt und kein zweiter da ist, um
ihn aufzurichten! \bibleverse{11} Auch wenn zwei beieinander liegen, so
wärmen sie sich gegenseitig; aber wie soll einer warm werden, wenn er
allein ist? \bibleverse{12} Und wenn man den einen angreift, so können
die beiden Widerstand leisten; und eine dreifache Schnur wird nicht so
bald zerrissen. \bibleverse{13} Ein armer, aber gescheiter Jüngling ist
besser, als ein alter, närrischer König, der sich nicht mehr beraten
läßt. \bibleverse{14} Denn aus dem Gefängnis ist er hervorgegangen, um
zu herrschen, obschon er im Königreiche jenes arm geboren ward.
\bibleverse{15} Ich sah, daß alle Lebendigen, die unter der Sonne
wandeln, mit dem Jüngling gehen, dem zweiten, der an jenes Stelle treten
sollte. \bibleverse{16} Des Volkes war kein Ende, vor welchem er
herging; werden nicht auch die Nachkommen sich seiner freuen? Oder ist
auch das eitel und ein Haschen nach Wind?

\hypertarget{section-4}{%
\section{5}\label{section-4}}

\bibleverse{1} Bewahre deinen Fuß, wenn du zum Hause Gottes gehst! Sich
herzunahen, um zu hören, ist besser, als wenn die Toren Opfer bringen;
denn sie haben keine Erkenntnis des Bösen, das sie tun. \bibleverse{2}
Übereile dich nicht mit deinem Mund, und laß dein Herz keine
unbesonnenen Worte vor Gott aussprechen; denn Gott ist im Himmel, und du
bist auf der Erde; darum sollst du nicht viele Worte machen!
\bibleverse{3} Denn der Traum kommt von Vielgeschäftigkeit her und
dummes Geschwätz vom vielen Reden. \bibleverse{4} Wenn du Gott ein
Gelübde tust, so versäume nicht, es zu bezahlen; denn er hat kein
Wohlgefallen an den Toren; darum halte deine Gelübde! \bibleverse{5} Es
ist besser, du gelobest nichts, als daß du gelobest und es nicht
haltest. \bibleverse{6} Laß dich durch deinen Mund nicht in Schuld
stürzen und sage nicht vor dem Boten: ``Es war ein Versehen!'' Warum
soll Gott zürnen ob deiner Worte und das Werk deiner Hände bannen?
\bibleverse{7} Denn wo man viel träumt, da werden auch viel unnütze
Worte gemacht. Du aber fürchte Gott! \bibleverse{8} Wenn du
Unterdrückung des Armen und Beraubung im Namen von Recht und
Gerechtigkeit in deinem Bezirke siehst, so werde darob nicht irre! Denn
es wacht noch ein Höherer über dem Hohen und über ihnen allen der
Höchste; \bibleverse{9} und ein Vorteil für ein Land ist bei alledem ein
König, der dem Ackerbau ergeben ist. \bibleverse{10} Wer Geld liebt,
wird des Geldes nimmer satt, und wer Reichtum liebt, bekommt nie genug.
Auch das ist eitel! \bibleverse{11} Wo viele Güter sind, da sind auch
viele, die davon zehren, und was hat ihr Besitzer mehr davon als eine
Augenweide? \bibleverse{12} Süß ist der Schlaf des Arbeiters, er esse
wenig oder viel; aber den Reichen läßt seine Übersättigung nicht
schlafen. \bibleverse{13} Es gibt ein böses Übel, das ich gesehen habe
unter der Sonne: Reichtum, der von seinem Besitzer zu seinem Schaden
verwahrt wird. \bibleverse{14} Geht solcher Reichtum durch einen
Unglücksfall verloren und hat der Betreffende einen Sohn, so bleibt
diesem gar nichts in der Hand. \bibleverse{15} So nackt, wie er von
seiner Mutter Leibe gekommen ist, geht er wieder dahin und kann gar
nichts für seine Mühe mitnehmen, das er in seiner Hand davontragen
könnte. \bibleverse{16} Das ist auch ein böses Übel, daß er gerade so,
wie er gekommen ist, wieder gehen muß; und was nützt es ihm, daß er sich
um Wind abgemüht hat? \bibleverse{17} Dazu muß er sein Leben lang mit
Kummer essen und hat viel Ärger, Verdruß und Zorn. \bibleverse{18}
Siehe, was ich für gut und für schön ansehe, ist das, daß einer esse und
trinke und Gutes genieße bei all seiner Arbeit, womit er sich abmüht
unter der Sonne alle Tage seines Lebens, welche Gott ihm gibt; denn das
ist sein Teil. \bibleverse{19} Auch wenn Gott irgend einem Menschen
Reichtum und Schätze gibt und ihm gestattet, davon zu genießen und sein
Teil zu nehmen, daß er sich freue in seiner Mühe, so ist das eine Gabe
Gottes. \bibleverse{20} Denn er soll nicht viel denken an seine
Lebenstage; denn Gott stimmt der Freude seines Herzens zu.

\hypertarget{section-5}{%
\section{6}\label{section-5}}

\bibleverse{1} Es gibt ein Übel. das ich gesehen habe unter der Sonne,
und das häufig vorkommt bei den Menschen: \bibleverse{2} Wenn Gott einem
Menschen Reichtum, Schätze und Ehre gibt, also daß ihm gar nichts fehlt,
wonach seine Seele gelüstet; wenn ihm Gott aber nicht gestattet, davon
zu genießen, sondern ein Fremder bekommt es zu genießen, so ist das
eitel und ein schweres Leid! \bibleverse{3} Wenn ein Mann hundert Kinder
zeugte und viele Jahre lebte; so groß auch die Zahl seiner Lebenstage
würde, seine Seele würde aber nicht befriedigt von dem Guten, und es
würde ihm kein Begräbnis zuteil, so sage ich: Eine Fehlgeburt ist
glücklicher als er! \bibleverse{4} Denn sie kam in Nichtigkeit und ging
im Dunklen dahin, und ihr Name ist im Dunklen geblieben; \bibleverse{5}
auch hat sie die Sonne nie gesehen noch gemerkt; ihr ist wohler als
jenem! \bibleverse{6} Und wenn er auch zweitausend Jahre lebte und kein
Gutes sähe, geht denn nicht alles an einen Ort? \bibleverse{7} Alle
Arbeit des Menschen ist für seinen Mund; und die Seele wird nicht
gesättigt! \bibleverse{8} Denn was hat der Weise vor dem Toren voraus,
was der Kranke, der weiß, wie man wandeln soll, vor den Gesunden?
\bibleverse{9} Besser mit den Augen anschauen, als mit der Begierde
herumschweifen! Auch das ist eitel und Haschen nach Wind.
\bibleverse{10} Was immer entstanden ist, längst ward es mit Namen
genannt! Und es ist bekannt, was ein Mensch ist: er kann nicht rechten
mit dem, der mächtiger ist als er; \bibleverse{11} denn wenn er auch
viele Worte macht, so sind sie doch ganz vergeblich; was hat der Mensch
davon? \bibleverse{12} Denn wer weiß, was dem Menschen gut ist im Leben,
die Zahl der Tage seines eitlen Lebens, welche er wie ein Schatten
verbringt? Wer will dem Menschen kundtun, was nach ihm sein wird unter
der Sonne?

\hypertarget{section-6}{%
\section{7}\label{section-6}}

\bibleverse{1} Ein guter Name ist besser als Wohlgeruch, und der Tag des
Todes ist besser als der Tag der Geburt. \bibleverse{2} Besser, man gehe
ins Trauerhaus als ins Trinkhaus; denn dort ist das Ende aller Menschen,
und der Lebendige nimmt es zu Herzen. \bibleverse{3} Verdruß ist besser
als Lachen; denn wenn das Angesicht traurig ist, so wird das Herz
gebessert. \bibleverse{4} Das Herz der Weisen ist im Trauerhaus; aber
das Herz der Narren im Haus der Freude. \bibleverse{5} Es ist besser,
man höre auf das Schelten des Weisen, als daß man lausche dem Gesang der
Narren! \bibleverse{6} Denn das Lachen des Narren ist wie das Knistern
der Dornen unter dem Topf; es ist ebenso eitel! \bibleverse{7} Denn
Gewalttätigkeit betört den Weisen, und Bestechung verderbt das Herz.
\bibleverse{8} Besser ist der Ausgang einer Sache als ihr Anfang, besser
ein Langmütiger als ein Hochmütiger. \bibleverse{9} Laß dich nicht
schnell zum Zorn und Ärger reizen; denn der Ärger wohnt im Busen der
Toren. \bibleverse{10} Sprich nicht: Wie kommt es, daß die frühern Tage
besser waren als diese? Denn nicht aus Weisheit fragst du so!
\bibleverse{11} Weisheit ist so gut wie ein Erbe und ein Vorteil für
die, welche die Sonne sehen. \bibleverse{12} Denn die Weisheit gewährt
Schutz, und auch das Geld gewährt Schutz; aber der Vorzug der Erkenntnis
ist der, daß die Weisheit ihrem Besitzer das Leben erhält.
\bibleverse{13} Betrachte das Werk Gottes! Wer kann gerade machen, was
er krümmt? \bibleverse{14} Am guten Tage sei guter Dinge, und am bösen
Tage bedenke: auch diesen hat Gott gemacht gleich wie jenen, wie ja der
Mensch auch gar nicht erraten kann, was nach demselben kommt.
\bibleverse{15} Allerlei habe ich gesehen in den Tagen meiner Eitelkeit:
Da ist ein Gerechter, welcher umkommt in seiner Gerechtigkeit, und dort
ist ein Gottloser, welcher lange lebt in seiner Bosheit. \bibleverse{16}
Sei nicht allzu gerecht und erzeige dich nicht übermäßig weise! Warum
willst du dich selbst verderben? \bibleverse{17} Werde aber auch nicht
allzu verwegen und sei kein Narr! Warum willst du vor der Zeit sterben?
\bibleverse{18} Es ist am besten, du hältst das eine fest und lässest
auch das andere nicht aus der Hand; denn wer Gott fürchtet, der entgeht
dem allem. \bibleverse{19} Die Weisheit macht den Weisen stärker als
zehn Gewaltige, die in der Stadt sind. \bibleverse{20} Weil kein Mensch
auf Erden so gerecht ist, daß er Gutes tut, ohne zu sündigen,
\bibleverse{21} so höre auch nicht auf alle Worte, die man dir
hinterbringt, und nimm sie nicht zu Herzen, damit du nicht deinen
eigenen Knecht dir fluchen hörest! \bibleverse{22} Denn wie oftmals (das
weiß dein Herz) hast auch du andern geflucht! \bibleverse{23} Dies alles
habe ich der Weisheit zur Prüfung vorgelegt. Ich sprach: Ich will weise
werden! Aber sie blieb fern von mir. \bibleverse{24} Wie weit entfernt
ist das, was geschehen ist, und tief, ja, tief verborgen! Wer will es
ausfindig machen? \bibleverse{25} Ich ging herum, und mein Herz war
dabei, zu erkennen und zu erforschen und zu fragen nach Weisheit und dem
Endergebnis, aber auch kennen zu lernen, wie dumm die Gottlosigkeit und
wie toll die Narrheit ist; \bibleverse{26} und nun finde ich, bitterer
als der Tod ist das Weib, deren Herz ein Fangnetz ist und deren Hände
Fesseln sind; wer Gott gefällt, wird ihr entrinnen, wer aber sündigt,
wird von ihr gefangen. \bibleverse{27} Siehe, das habe ich gefunden,
sprach der Prediger, indem ich eins ums andere prüfte, um zum
Endergebnis zu kommen. \bibleverse{28} Was aber meine Seele noch immer
sucht, habe ich nicht gefunden; einen Mann habe ich unter Tausenden
gefunden; aber ein Weib habe ich unter diesen allen nicht gefunden!
\bibleverse{29} Nur allein, siehe, das habe ich gefunden, daß Gott den
Menschen aufrichtig gemacht hat; sie aber suchen viele Künste.

\hypertarget{section-7}{%
\section{8}\label{section-7}}

\bibleverse{1} Wer ist wie der Weise, und wer versteht die Deutung der
Worte? Die Weisheit eines Menschen erleuchtet sein Angesicht, und die
Kraft seiner Augen wird verdoppelt. \bibleverse{2} Bewahre mein
königliches Wort wie einen göttlichen Schwur! \bibleverse{3} Laß dich
nicht von des Königs Angesicht verscheuchen und vertritt keine schlechte
Sache; denn er tut alles, was er will. \bibleverse{4} Denn des Königs
Wort ist mächtig, und wer darf zu ihm sagen: Was machst du?
\bibleverse{5} Wer das Gebot bewahrt, berücksichtigt keine böse Sache;
aber das Herz des Weisen nimmt Rücksicht auf Zeit und Gericht.
\bibleverse{6} Denn für jegliches Vornehmen gibt es eine Zeit und ein
Gericht; denn das Böse des Menschen lastet schwer auf ihm.
\bibleverse{7} Denn er weiß nicht, was geschehen wird; und wer zeigt ihm
an, wie es geschehen wird? \bibleverse{8} Kein Mensch hat Macht über den
Wind, daß er den Wind zurückhalten könnte; so gebietet auch keiner über
den Tag des Todes, und im Kriege gibt es keine Entlassung, und der
Frevel rettet den nicht, welcher ihn verübt. \bibleverse{9} Dies alles
habe ich gesehen und mein Herz all dem Treiben gewidmet, das unter der
Sonne geschieht, in einer Zeit, da ein Mensch über den andern herrscht
zu seinem Schaden. \bibleverse{10} Ich sah auch, wie Gottlose begraben
wurden und zur Ruhe eingingen, während solche, die ordentlich gelebt
hatten, den heiligen Ort verlassen mußten und vergessen wurden in der
Stadt; auch das ist eitel! \bibleverse{11} Weil der Richterspruch nicht
eilends vollzogen wird, darum ist das Herz der Menschenkinder voll,
Böses zu tun. \bibleverse{12} Wenn auch ein Sünder hundertmal Böses tut
und lange lebt, so weiß ich doch, daß es denen gut gehen wird, die Gott
fürchten, die sich scheuen vor ihm. \bibleverse{13} Aber dem Gottlosen
wird es nicht wohl ergehen, und er wird seine Tage nicht wie ein
Schatten verlängern, da er sich vor Gott nicht fürchtet! \bibleverse{14}
Es ist eine Eitelkeit, die auf Erden geschieht, daß es Gerechte gibt,
welche gleichsam das Schicksal der Gottlosen trifft, und Gottlose, denen
es ergeht nach dem Werk der Gerechten. Ich habe gesagt, daß auch das
eitel sei. \bibleverse{15} Darum habe ich die Freude gepriesen, da es
nichts Besseres gibt für den Menschen unter der Sonne, als zu essen und
zu trinken und fröhlich zu sein, daß ihn das begleiten soll bei seiner
Mühe alle Tage seines Lebens, welche Gott ihm gibt unter der Sonne.
\bibleverse{16} Als ich mein Herz darauf richtete, die Weisheit zu
erkennen und die Mühe zu betrachten, die man sich auf Erden gibt, daß
man auch Tag und Nacht keinen Schlaf in ihren Augen sieht,
\bibleverse{17} da sah ich bezüglich des ganzen Werkes Gottes, daß der
Mensch das Werk nicht ergründen kann, welches unter der Sonne getan
wird. Wiewohl der Mensch sich Mühe gibt, es zu erforschen, so kann er es
nicht ergründen; und wenn auch der Weise behauptet, er verstehe es, so
kann er es nicht finden.

\hypertarget{section-8}{%
\section{9}\label{section-8}}

\bibleverse{1} Dies alles habe ich mir zu Herzen genommen, und dies habe
ich zu erkennen gesucht, daß die Gerechten und die Weisen und ihre Werke
in der Hand Gottes sind. Der Mensch merkt weder Liebe noch Haß; es steht
ihnen alles bevor, den einen wie den andern. \bibleverse{2} Es kann dem
Gerechten dasselbe begegnen wie dem Gottlosen, dem Guten und Reinen wie
dem Unreinen, dem, der opfert, wie dem, der nicht opfert; dem Guten wie
dem Sünder, dem, welcher schwört, wie dem, welcher sich vor dem Eide
fürchtet. \bibleverse{3} Das ist das Schlimme bei allem, was unter der
Sonne geschieht, daß allen dasselbe begegnet; daher wird auch das Herz
der Menschen voll Bosheit, und Übermut ist in ihren Herzen ihr Leben
lang, und darnach müssen sie sterben! \bibleverse{4} Denn für jeden
Lebendigen, wer er auch sei, ist noch Hoffnung (denn ein lebendiger Hund
ist besser als ein toter Löwe); \bibleverse{5} denn die Lebendigen
wissen, daß sie sterben müssen; aber die Toten wissen gar nichts, und es
wird ihnen auch keine Belohnung mehr zuteil; denn man denkt nicht mehr
an sie. \bibleverse{6} Ihre Liebe und ihr Haß wie auch ihr Eifer sind
längst vergangen, und sie haben auf ewig keinen Anteil mehr an allem,
was unter der Sonne geschieht. \bibleverse{7} So gehe nun hin, iß mit
Freuden dein Brot und trinke deinen Wein mit gutem Gewissen; denn Gott
hat dein Tun längst gebilligt! \bibleverse{8} Deine Kleider seien
jederzeit weiß, und laß auf deinem Haupte das Öl nie fehlen.
\bibleverse{9} Genieße das Leben mit dem Weibe, das du liebst, alle Tage
des eitlen Lebens, welches er dir unter der Sonne gibt in dieser
vergänglichen Zeit; denn das ist dein Teil am Leben und an der Mühe,
womit du dich abmühst unter der Sonne. \bibleverse{10} Alles, was deine
Hand zu tun vorfindet, das tue mit deiner ganzen Kraft; denn im
Totenreich, dahin du gehst, ist kein Wirken mehr und kein Planen, keine
Wissenschaft und keine Weisheit! \bibleverse{11} Und wiederum sah ich
unter der Sonne, daß nicht die Schnellen den Wettlauf gewinnen, noch die
Starken die Schlacht, daß nicht die Weisen das Brot, auch nicht die
Verständigen den Reichtum, noch die Erfahrenen Gunst erlangen, sondern
daß alles auf Zeit und Umstände ankommt. \bibleverse{12} Denn auch seine
Zeit kennt der Mensch nicht, so wenig wie die Fische, welche mit dem
bösen Netze gefangen werden, und wie die Vögel, welche man mit der
Schlinge fängt; gleich diesen werden auch die Menschenkinder gefangen
zur Zeit des Unglücks, wenn es plötzlich über sie kommt. \bibleverse{13}
Auch das habe ich als Weisheit angesehen unter der Sonne, und sie schien
mir groß: \bibleverse{14} Gegen eine kleine Stadt, in welcher wenig
Männer waren, kam ein großer König und belagerte sie und baute große
Belagerungstürme wider sie. \bibleverse{15} Da fand sich in derselben
Stadt ein armer, aber weiser Mann, der rettete die Stadt durch seine
Weisheit, und kein Mensch hatte an diesen armen Mann gedacht.
\bibleverse{16} Da sprach ich: Weisheit ist besser als Stärke! Aber die
Weisheit des Armen ist verachtet, und man hört nicht auf ihn.
\bibleverse{17} Die Worte der Weisen, die man in der Stille vernimmt,
sind besser als das Schreien eines Herrschers unter den Narren.
\bibleverse{18} Weisheit ist besser als Kriegsgerät; aber ein einziger
Sünder verdirbt viel Gutes.

\hypertarget{section-9}{%
\section{10}\label{section-9}}

\bibleverse{1} Giftige Fliegen machen das Öl des Salbenbereiters
stinkend und faulend; ein wenig Torheit kommt teurer zu stehen als
Weisheit und Ehre! \bibleverse{2} Der Weise trägt sein Herz auf dem
rechten Fleck, der Narr hat es am unrechten Ort; \bibleverse{3} auf
welchem Wege der Narr auch gehen mag, es fehlt ihm überall an Verstand,
und er sagt jedermann, daß er ein Tor sei. \bibleverse{4} Wenn der Zorn
des Herrschers gegen dich entbrennt, so verlaße deinen Posten nicht;
denn Gelassenheit verhütet große Sünden. \bibleverse{5} Es gibt ein
Übel, das ich unter der Sonne sah, wie ein Mißgriff, von einem
Machthaber getan: \bibleverse{6} Die Torheit ward auf große Höhen
gestellt, und Reiche mußten unten bleiben; \bibleverse{7} ich sah
Knechte auf Pferden, und Fürsten gingen wie Knechte zu Fuß.
\bibleverse{8} Wer eine Grube gräbt, fällt hinein; und wer eine Mauer
einreißt, den wird eine Schlange beißen. \bibleverse{9} Wer Steine
bricht, verwundet sich daran, und wer Holz spaltet, bringt sich in
Gefahr. \bibleverse{10} Wenn ein Eisen stumpf ist und ungeschliffen
bleibt, so muß man um so mehr Kraft anwenden; aber durch Weisheit kommt
man zum Gelingen. \bibleverse{11} Wenn die Schlange beißt, weil man sie
nicht beschworen hat, so hat der Beschwörer keinen Nutzen von seiner
Kunst. \bibleverse{12} Die Reden eines Weisen sind anmutig; aber die
Lippen des Toren verschlingen ihn selbst. \bibleverse{13} Der Anfang
seiner Worte ist Dummheit und das Ende seiner Rede die schlimmste
Tollheit. \bibleverse{14} Auch macht der Tor viele Worte, obgleich kein
Mensch weiß, was geschehen ist; und was nach ihm sein wird, wer kann es
ihm sagen? \bibleverse{15} Die Mühe, die der Tor sich gibt, der den Weg
zur Stadt nicht kennt, ermüdet ihn. \bibleverse{16} Wehe dir, Land,
dessen König ein Knabe ist und dessen Fürsten schon am Morgen schmausen!
\bibleverse{17} Heil dir, du Land, dessen König ein Sohn der Edlen ist
und dessen Fürsten zu rechter Zeit essen, zur Stärkung, und nicht aus
Genußsucht. \bibleverse{18} Durch Faulheit verfault das Gebälk, und
wegen Nachlässigkeit der Hände rinnt das Dach. \bibleverse{19} Zum
Vergnügen backt man Brot, und der Wein erfreut die Lebendigen, und das
Geld gewährt alles. \bibleverse{20} Fluche dem König nicht einmal in
deinen Gedanken, und verwünsche den Reichen auch in deiner Schlafkammer
nicht; denn die Vögel des Himmels tragen den Laut davon, und ein
geflügelter Bote verkündigt das Wort.

\hypertarget{section-10}{%
\section{11}\label{section-10}}

\bibleverse{1} Sende dein Brot übers Wasser, so wirst du es nach langer
Zeit wieder finden! \bibleverse{2} Verteile an sieben und an acht; denn
du weißt nicht, was Schlimmes auf Erden geschehen mag! \bibleverse{3}
Wenn die Wolken voll sind, so gießen sie Regen auf die Erde. Ob der Baum
nach Süden fällt oder nach Norden, nach welchem Ort der Baum fällt, da
bleibt er liegen. \bibleverse{4} Wer auf den Wind achtet, sät nicht, und
wer auf die Wolken sieht, erntet nicht. \bibleverse{5} Gleichwie du
nicht weißt, welches der Weg des Windes ist, noch wie die Gebeine im
Mutterleib bereitet werden, also kennst du auch das Werk Gottes nicht,
der alles wirkt. \bibleverse{6} Frühe säe deinen Samen, und des Abends
laß deine Hand nicht ruhen; denn du weißt nicht, ob dieses oder jenes
geraten, oder ob beides zugleich gut wird. \bibleverse{7} Süß ist das
Licht, und gut ist\textquotesingle s für die Augen, die Sonne zu sehen!
\bibleverse{8} Denn wenn der Mensch auch viele Jahre lebt, so soll er
sich in ihnen allen freuen und soll bedenken, daß der Tage der
Finsternis viele sein werden. Alles, was kommt, ist eitel!
\bibleverse{9} Freue dich, Jüngling, in deiner Jugend, und dein Herz sei
guter Dinge in den Tagen deines Jünglingsalters; wandle die Wege, die
dein Herz erwählt und die deinen Augen gefallen; aber wisse, daß dich
Gott für dies alles vor Gericht ziehen wird! \bibleverse{10} Entferne
alle Verdrießlichkeit von deinem Herzen und halte dir das Übel vom Leibe
fern! Denn Jugend und Morgenrot sind vergänglich!

\hypertarget{section-11}{%
\section{12}\label{section-11}}

\bibleverse{1} Und gedenke an deinen Schöpfer in den Tagen deiner
Jugend, ehe die bösen Tage kommen und die Jahre herzutreten, da du wirst
sagen: ``Sie gefallen mir nicht''; \bibleverse{2} ehe die Sonne und das
Licht, der Mond und die Sterne sich verfinstern und die Wolken
wiederkehren nach dem Regen; \bibleverse{3} zur Zeit, wo die Hüter des
Hauses zittern und die Starken sich krümmen und die Müllerinnen feiern,
weil ihrer zu wenige geworden sind, und finster dreinsehen, die durch
die Fenster schauen; \bibleverse{4} wenn die Türen nach der Straße
geschlossen werden und das Klappern der Mühle leiser wird, wenn man
erwacht vom Vogelsang und gedämpft werden die Töchter des Gesangs;
\bibleverse{5} wenn man sich auch vor jeder Anhöhe fürchtet und
Schrecknisse auf dem Wege sieht; wenn der Mandelbaum blüht und die
Heuschrecke sich mühsam fortschleppt und die Kaper versagt (denn der
Mensch geht in sein ewiges Haus, und die Trauernden gehen auf der Gasse
umher); \bibleverse{6} ehe denn der silberne Strick zerreißt und die
goldene Schale zerspringt und der Krug am Born zerbricht und das Rad
zerbrochen in den Brunnen stürzt \bibleverse{7} und der Staub wieder zur
Erde wird, wie er gewesen ist, und der Geist zu Gott zurückkehrt, der
ihn gegeben hat. \bibleverse{8} O Eitelkeit der Eitelkeiten! spricht der
Prediger; alles ist eitel! \bibleverse{9} Und außerdem, daß der Prediger
weise war, lehrte er das Volk Erkenntnis und erwog und erforschte und
stellte viele Sprichwörter auf. \bibleverse{10} Der Prediger suchte
gefällige Worte zu finden und die Worte der Wahrheit richtig
aufzuzeichnen. \bibleverse{11} Die Worte der Weisen sind wie
Treibstacheln und wie eingeschlagene Nägel die gesammelten Sprüche, von
einem einzigen Hirten gegeben. \bibleverse{12} Und außerdem laß dich
warnen, mein Sohn! Des vielen Büchermachens ist kein Ende, und viel
Studieren ermüdet den Leib. \bibleverse{13} Laßt uns die Summe aller
Lehre hören: Fürchte Gott und halte seine Gebote; denn das soll jeder
Mensch! \bibleverse{14} Denn Gott wird jedes Werk ins Gericht bringen,
samt allem Verborgenen, es sei gut oder böse.
