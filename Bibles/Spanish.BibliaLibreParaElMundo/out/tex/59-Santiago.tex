\hypertarget{salutaciones-y-bendiciones}{%
\subsection{Salutaciones y
Bendiciones}\label{salutaciones-y-bendiciones}}

\hypertarget{section}{%
\section{1}\label{section}}

\bibleverse{1} Santiago, siervo de Dios y del Señor Jesucristo, a las
doce tribus que están en la Dispersión: Saludos.

\hypertarget{comportamiento-correcto-en-las-tentaciones-recordatorio-de-la-disposiciuxf3n-correcta}{%
\subsection{Comportamiento correcto en las tentaciones; Recordatorio de
la disposición
correcta}\label{comportamiento-correcto-en-las-tentaciones-recordatorio-de-la-disposiciuxf3n-correcta}}

\bibleverse{2} Hermanos míos, estad contentos cuando caigáis en diversas
tentaciones, \footnote{\textbf{1:2} Rom 5,3-5; 1Pe 4,13} \bibleverse{3}
sabiendo que la prueba de vuestra fe produce resistencia. \bibleverse{4}
Dejad que la resistencia tenga su obra perfecta, para que seáis
perfectos y completos, sin que os falte nada.

\hypertarget{recordatorio-para-perseverar-pidiendo-sabiduruxeda}{%
\subsection{Recordatorio para perseverar pidiendo
sabiduría}\label{recordatorio-para-perseverar-pidiendo-sabiduruxeda}}

\bibleverse{5} Pero si a alguno de vosotros le falta sabiduría, pídala a
Dios, que da a todos con liberalidad y sin reproche, y le será
concedida. \footnote{\textbf{1:5} Prov 2,3-6; Sant 3,15} \bibleverse{6}
Pero que pida con fe, sin dudar, porque el que duda es como la ola del
mar, impulsada por el viento y zarandeada. \footnote{\textbf{1:6} Mar
  11,24; 1Tim 2,8} \bibleverse{7} Porque ese hombre no debe pensar que
recibirá algo del Señor. \bibleverse{8} Es un hombre de doble ánimo,
inestable en todos sus caminos.

\hypertarget{la-actitud-correcta-hacia-la-pobreza-y-la-riqueza-bendiciuxf3n-de-la-libertad-condicional}{%
\subsection{La actitud correcta hacia la pobreza y la riqueza; Bendición
de la libertad
condicional}\label{la-actitud-correcta-hacia-la-pobreza-y-la-riqueza-bendiciuxf3n-de-la-libertad-condicional}}

\bibleverse{9} Que el hermano de condición humilde se gloríe en su alta
posición; \footnote{\textbf{1:9} Sant 2,5} \bibleverse{10} y el rico, en
que se haga humilde, porque como la flor de la hierba, pasará.
\footnote{\textbf{1:10} 1Pe 1,24; 1Tim 6,17} \bibleverse{11} Porque el
sol se levanta con el viento abrasador y marchita la hierba; y la flor
en ella cae, y la belleza de su aspecto perece. Así también el rico se
desvanecerá en sus afanes. \footnote{\textbf{1:11} Is 40,6-7}

\bibleverse{12} Bienaventurado el que soporta la tentación, porque
cuando haya sido aprobado, recibirá la corona de la vida que el Señor
prometió a los que le aman. \footnote{\textbf{1:12} 2Tim 4,8}

\hypertarget{las-tentaciones-al-mal-provienen-de-la-propia-lujuria-no-de-dios-la-fuente-de-todo-bien}{%
\subsection{Las tentaciones al mal provienen de la propia lujuria, no de
Dios, la fuente de todo
bien}\label{las-tentaciones-al-mal-provienen-de-la-propia-lujuria-no-de-dios-la-fuente-de-todo-bien}}

\bibleverse{13} Que nadie diga cuando es tentado: ``Soy tentado por
Dios'', porque Dios no puede ser tentado por el mal, y él mismo no
tienta a nadie. \bibleverse{14} Pero cada uno es tentado cuando es
atraído por su propia concupiscencia y seducido. \footnote{\textbf{1:14}
  Rom 7,7-8} \bibleverse{15} Entonces la concupiscencia, cuando ha
concebido, engendra el pecado. El pecado, cuando ha crecido, produce la
muerte. \footnote{\textbf{1:15} Rom 7,10} \bibleverse{16} No se dejen
engañar, mis amados hermanos. \bibleverse{17} Toda buena dádiva y todo
don perfecto viene de lo alto, del Padre de las luces, con quien no
puede haber variación ni sombra que se convierta. \footnote{\textbf{1:17}
  Mat 7,11; 1Jn 1,5} \bibleverse{18} De su propia voluntad nos hizo
nacer por la palabra de la verdad, para que seamos una especie de
primicias de sus criaturas. \footnote{\textbf{1:18} Juan 1,13; 1Pe 1,23}

\hypertarget{sea-no-solo-un-oyente-sino-tambiuxe9n-un-hacedor-de-la-palabra}{%
\subsection{Sea no solo un oyente, sino también un hacedor de la
palabra}\label{sea-no-solo-un-oyente-sino-tambiuxe9n-un-hacedor-de-la-palabra}}

\bibleverse{19} Así que, mis amados hermanos, todo hombre sea pronto
para oír, lento para hablar y lento para la ira; \footnote{\textbf{1:19}
  Prov 29,20; Ecl 5,2-3; Ecl 7,9} \bibleverse{20} porque la ira del
hombre no produce la justicia de Dios. \footnote{\textbf{1:20} Prov
  29,22; Efes 4,26} \bibleverse{21} Por tanto, desechando toda
inmundicia y desbordamiento de maldad, recibid con humildad la palabra
implantada, que puede salvar vuestras almas. \footnote{\textbf{1:21} 1Pe
  2,1}

\bibleverse{22} Pero sed hacedores de la palabra, y no sólo oidores,
engañándoos a vosotros mismos. \footnote{\textbf{1:22} Mat 7,26; Rom
  2,13} \bibleverse{23} Porque si alguno es oidor de la palabra y no
hacedor, es como un hombre que mira su rostro natural en un espejo;
\bibleverse{24} porque se ve a sí mismo, y se va, y enseguida se olvida
de la clase de hombre que era. \bibleverse{25} Pero el que mira la ley
perfecta de la libertad y continúa, no siendo un oidor que olvida, sino
un hacedor de la obra, éste será bendecido en lo que hace. \footnote{\textbf{1:25}
  Sant 2,12; Rom 8,2; Juan 13,17}

\hypertarget{algunos-ejemplos-de-cuxf3mo-hacer-los-trabajos-correctos}{%
\subsection{Algunos ejemplos de cómo hacer los trabajos
correctos}\label{algunos-ejemplos-de-cuxf3mo-hacer-los-trabajos-correctos}}

\bibleverse{26} Si alguno de vosotros se cree religioso mientras no
refrena su lengua, sino que engaña a su corazón, la religión de ese
hombre no vale nada. \footnote{\textbf{1:26} 1Pe 3,10} \bibleverse{27}
La religión pura y sin mácula ante nuestro Dios y Padre es ésta: visitar
a los huérfanos y a las viudas en su aflicción, y mantenerse sin mancha
del mundo.

\hypertarget{cuidado-con-la-reputaciuxf3n-de-la-persona-especialmente-de-los-pobres}{%
\subsection{Cuidado con la reputación de la persona, especialmente de
los
pobres}\label{cuidado-con-la-reputaciuxf3n-de-la-persona-especialmente-de-los-pobres}}

\hypertarget{section-1}{%
\section{2}\label{section-1}}

\bibleverse{1} Hermanos míos, no tengáis la fe de nuestro glorioso Señor
Jesucristo con parcialidad. \bibleverse{2} Porque si entra en vuestra
sinagoga un hombre con un anillo de oro, vestido con ropas finas, y
entra también un pobre vestido con ropas sucias, \bibleverse{3} y os
fijáis especialmente en el que lleva las ropas finas y le decís:
``Siéntate aquí en un buen lugar'', y al pobre le decís: ``Ponte ahí'',
o ``Siéntate junto al escabel de mis pies'' \bibleverse{4} ¿no habéis
mostrado parcialidad entre vosotros, y os habéis convertido en jueces
con malos pensamientos? \bibleverse{5} Escuchad, mis queridos hermanos.
¿No ha elegido Dios a los pobres de este mundo para que sean ricos en la
fe y herederos del Reino que prometió a los que le aman? \footnote{\textbf{2:5}
  1Cor 1,26; 1Cor 11,22; Luc 12,21} \bibleverse{6} Pero ustedes han
deshonrado al pobre. ¿No le oprimen los ricos y le arrastran
personalmente ante los tribunales? \bibleverse{7} ¿No blasfeman del
honorable nombre con el que te llaman? \footnote{\textbf{2:7} 1Pe 4,14}

\hypertarget{el-cumplimiento-de-la-ley-mosaica-debe-ser-uniforme-es-decir-sin-excepciuxf3n}{%
\subsection{El cumplimiento de la ley mosaica debe ser uniforme, es
decir, sin
excepción}\label{el-cumplimiento-de-la-ley-mosaica-debe-ser-uniforme-es-decir-sin-excepciuxf3n}}

\bibleverse{8} Sin embargo, si cumplís la ley real según la Escritura:
``Amarás a tu prójimo como a ti mismo'', hacéis bien. \bibleverse{9}
Pero si mostráis parcialidad, cometéis pecado, siendo condenados por la
ley como transgresores. \footnote{\textbf{2:9} Deut 1,17}
\bibleverse{10} Porque el que guarda toda la ley y tropieza en un punto,
se hace culpable de todo. \footnote{\textbf{2:10} Mat 5,19}
\bibleverse{11} Porque el que dijo: ``No cometas adulterio'', también
dijo: ``No cometas homicidio''. Ahora bien, si no cometes adulterio pero
cometes homicidio, te has convertido en transgresor de la ley.
\bibleverse{12} Así pues, hablad y haced como hombres que han de ser
juzgados por la ley de la libertad. \bibleverse{13} Porque el juicio es
sin misericordia para el que no ha mostrado misericordia. La
misericordia triunfa sobre el juicio. \footnote{\textbf{2:13} Mat 5,7;
  Mat 18,30; Mat 18,34; Mat 25,45-46}

\hypertarget{la-fe-sin-obras-estuxe1-muerta-e-inuxfatil-la-verdadera-fe-se-muestra-en-la-abnegaciuxf3n-y-en-las-buenas-obras}{%
\subsection{La fe sin obras está muerta e inútil; la verdadera fe se
muestra en la abnegación y en las buenas
obras}\label{la-fe-sin-obras-estuxe1-muerta-e-inuxfatil-la-verdadera-fe-se-muestra-en-la-abnegaciuxf3n-y-en-las-buenas-obras}}

\bibleverse{14} ¿De qué sirve, hermanos míos, que un hombre diga que
tiene fe, pero no tenga obras? ¿Acaso la fe puede salvarle? \footnote{\textbf{2:14}
  Mat 7,21} \bibleverse{15} Y si un hermano o una hermana están desnudos
y les falta el alimento de cada día, \bibleverse{16} y uno de vosotros
les dice: ``Id en paz. Caliéntate y sáciate''; pero no les has dado lo
que necesita el cuerpo, ¿de qué sirve? \footnote{\textbf{2:16} 1Jn 3,18}
\bibleverse{17} Así también la fe, si no tiene obras, está muerta en sí
misma. \bibleverse{18} Sí, un hombre dirá: ``Tú tienes fe, y yo tengo
obras''. Muéstrame tu fe sin obras, y yo te mostraré mi fe por mis
obras. \footnote{\textbf{2:18} Gal 5,6}

\bibleverse{19} Tú crees que Dios es uno. Haces bien. Los demonios
también creen, y tiemblan. \bibleverse{20} ¿Pero quieres saber, hombre
vano, que la fe sin obras está muerta?

\hypertarget{dos-ejemplos-del-antiguo-testamento-como-evidencia-buxedblica-de-las-obras-que-conducen-a-la-consumaciuxf3n-de-la-fe}{%
\subsection{Dos ejemplos del Antiguo Testamento como evidencia bíblica
de las obras que conducen a la consumación de la
fe}\label{dos-ejemplos-del-antiguo-testamento-como-evidencia-buxedblica-de-las-obras-que-conducen-a-la-consumaciuxf3n-de-la-fe}}

\bibleverse{21} ¿No fue Abraham, nuestro padre, justificado por las
obras, al ofrecer a su hijo Isaac sobre el altar? \footnote{\textbf{2:21}
  Gén 22,1; Heb 11,17} \bibleverse{22} Ya ves que la fe obró con sus
obras, y por las obras se perfeccionó la fe. \bibleverse{23} Así se
cumplió la Escritura que dice: ``Abraham creyó a Dios, y le fue contado
como justicia'', y fue llamado amigo de Dios. \bibleverse{24} Veis,
pues, que por las obras el hombre es justificado, y no sólo por la fe.
\bibleverse{25} Del mismo modo, ¿no fue también justificada por las
obras Rahab, la prostituta, cuando recibió a los mensajeros y los envió
por otro camino? \footnote{\textbf{2:25} Jos 2,1; Heb 11,31}
\bibleverse{26} Porque así como el cuerpo sin espíritu está muerto, así
también la fe sin obras está muerta.

\hypertarget{advertencia-contra-las-prisas-no-solicitadas-para-enseuxf1ar-y-contra-los-pecados-de-la-lengua}{%
\subsection{Advertencia contra las prisas no solicitadas para enseñar y
contra los pecados de la
lengua}\label{advertencia-contra-las-prisas-no-solicitadas-para-enseuxf1ar-y-contra-los-pecados-de-la-lengua}}

\hypertarget{section-2}{%
\section{3}\label{section-2}}

\bibleverse{1} Hermanos míos, no seáis muchos los maestros, sabiendo que
recibiremos un juicio más severo. \bibleverse{2} Porque todos tropezamos
en muchas cosas. El que no tropieza en la palabra es una persona
perfecta, capaz de refrenar también a todo el cuerpo. \bibleverse{3} En
efecto, ponemos bocados en la boca de los caballos para que nos
obedezcan, y guiamos todo su cuerpo. \bibleverse{4} He aquí que también
las naves, aunque son tan grandes y son impulsadas por vientos feroces,
son guiadas por un timón muy pequeño, hacia donde el piloto quiere.
\bibleverse{5} Así también la lengua es un miembro pequeño, y se jacta
de grandes cosas. Mira cómo un pequeño fuego puede extenderse hasta un
gran bosque. \bibleverse{6} Y la lengua es un fuego. El mundo de la
iniquidad entre nuestros miembros es la lengua, que contamina todo el
cuerpo, e incendia el curso de la naturaleza, y es incendiada por la
Gehenna. \footnote{\textbf{3:6} Mat 15,11; Mat 15,18; Mat 12,36-37; Prov
  16,27} \bibleverse{7} Porque toda clase de animal, de ave, de reptil y
de criatura marina está domesticada, y ha sido domesticada por la
humanidad; \bibleverse{8} pero nadie puede domesticar la lengua. Es un
mal inquieto, lleno de veneno mortal. \bibleverse{9} Con ella bendecimos
a nuestro Dios y Padre, y con ella maldecimos a los hombres que están
hechos a imagen de Dios. \footnote{\textbf{3:9} Gén 1,27}
\bibleverse{10} De la misma boca salen bendiciones y maldiciones.
Hermanos míos, estas cosas no deben ser así. \footnote{\textbf{3:10}
  Efes 4,29} \bibleverse{11} ¿Acaso un manantial envía de la misma
abertura agua dulce y amarga? \bibleverse{12} ¿Acaso una higuera,
hermanos míos, puede dar aceitunas, o una vid higos? Así pues, ningún
manantial da a la vez agua salada y agua dulce.

\hypertarget{de-la-sabiduruxeda-falsa-espiritual-terrenal-y-verdadera-espiritual-celestial}{%
\subsection{De la sabiduría falsa, espiritual-terrenal y verdadera,
espiritual-celestial}\label{de-la-sabiduruxeda-falsa-espiritual-terrenal-y-verdadera-espiritual-celestial}}

\bibleverse{13} ¿Quién es sabio y entendido entre vosotros? Que
demuestre con su buena conducta que sus obras son hechas con mansedumbre
y sabiduría. \bibleverse{14} Pero si tienes celos amargos y ambición
egoísta en tu corazón, no te jactes ni mientas contra la verdad.
\bibleverse{15} Esta sabiduría no es la que desciende de lo alto, sino
que es terrenal, sensual y demoníaca. \footnote{\textbf{3:15} Sant 1,5}
\bibleverse{16} Porque donde están los celos y la ambición egoísta, allí
está la confusión y toda obra mala. \bibleverse{17} Pero la sabiduría
que viene de arriba es primero pura, luego pacífica, amable, razonable,
llena de misericordia y de buenos frutos, sin parcialidad y sin
hipocresía. \bibleverse{18} Ahora bien, el fruto de la justicia lo
siembran en paz los que hacen la paz. \footnote{\textbf{3:18} Is 32,17;
  Mat 5,9; Fil 1,11}

\hypertarget{advertencias-contra-la-contienda-la-insatisfacciuxf3n-y-el-cosmopolitismo-contra-el-abuso-y-el-juicio-descuidado}{%
\subsection{Advertencias contra la contienda, la insatisfacción y el
cosmopolitismo, contra el abuso y el juicio
descuidado}\label{advertencias-contra-la-contienda-la-insatisfacciuxf3n-y-el-cosmopolitismo-contra-el-abuso-y-el-juicio-descuidado}}

\hypertarget{section-3}{%
\section{4}\label{section-3}}

\bibleverse{1} ¿De dónde vienen las guerras y las peleas entre vosotros?
¿No provienen de vuestros placeres que combaten en vuestros miembros?
\bibleverse{2} Codiciáis, y no tenéis. Asesináis y codiciáis, y no
podéis obtener. Peleáis y hacéis la guerra. No tenéis, porque no pedís.
\footnote{\textbf{4:2} Gal 5,15} \bibleverse{3} Pedís, y no recibís,
porque pedís con malos motivos, para gastarlo en vuestros placeres.
\bibleverse{4} Adúlteros y adúlteras, ¿no sabéis que la amistad con el
mundo es una hostilidad hacia Dios? Por eso, quien quiere ser amigo del
mundo se hace enemigo de Dios. \footnote{\textbf{4:4} Luc 6,26; Rom 8,7;
  1Jn 2,15} \bibleverse{5} ¿O pensáis que la Escritura dice en vano:
``El Espíritu que vive en nosotros anhela celosamente''? \footnote{\textbf{4:5}
  Éxod 20,3; Éxod 20,5} \bibleverse{6} Pero da más gracia. Por eso dice:
``Dios resiste a los soberbios, pero da gracia a los humildes''.
\footnote{\textbf{4:6} Job 22,29; Mat 23,12; 1Pe 5,5} \bibleverse{7}
Someteos, pues, a Dios. Resistid al diablo, y huirá de vosotros.
\footnote{\textbf{4:7} 1Pe 5,8-9} \bibleverse{8} Acercaos a Dios, y él
se acercará a vosotros. Limpiad vuestras manos, pecadores. Purificad
vuestros corazones, vosotros los de doble ánimo. \footnote{\textbf{4:8}
  Zac 1,3; Is 1,16} \bibleverse{9} Lamentad, lamentad y llorad. Que
vuestra risa se convierta en llanto y vuestra alegría en tristeza.
\bibleverse{10} Humillaos ante el Señor, y él os exaltará. \footnote{\textbf{4:10}
  1Pe 5,6}

\bibleverse{11} No habléis unos contra otros, hermanos. El que habla
contra un hermano y juzga a su hermano, habla contra la ley y juzga a la
ley. Pero si juzgas la ley, no eres hacedor de la ley, sino juez.
\bibleverse{12} Uno solo es el legislador, que puede salvar y destruir.
Pero ¿quién eres tú para juzgar a otro? \footnote{\textbf{4:12} Mat 7,1;
  Rom 14,4}

\hypertarget{contra-la-autoconfianza-medida-en-las-empresas}{%
\subsection{Contra la autoconfianza medida en las
empresas}\label{contra-la-autoconfianza-medida-en-las-empresas}}

\bibleverse{13} Venid ahora, vosotros que decís: ``Hoy o mañana vayamos
a esta ciudad y pasemos un año allí, comerciemos y hagamos ganancias''.
\footnote{\textbf{4:13} Prov 27,1} \bibleverse{14} Pero no sabéis cómo
será vuestra vida mañana. Porque, ¿qué es tu vida? Porque sois un vapor
que aparece por un poco de tiempo y luego se desvanece. \footnote{\textbf{4:14}
  Luc 12,20} \bibleverse{15} Pues deberíais decir: ``Si el Señor quiere,
viviremos y haremos esto o aquello''. \footnote{\textbf{4:15} Hech
  18,21; 1Cor 4,19} \bibleverse{16} Pero ahora os gloriáis en vuestra
jactancia. Toda esa jactancia es mala. \bibleverse{17} Por tanto, el que
sabe hacer el bien y no lo hace, para él es pecado. \footnote{\textbf{4:17}
  Luc 12,47}

\hypertarget{anuncio-del-juicio-inminente-a-los-exuberantes-ricos-y-los-siervos-de-mammon-que-se-olvidan-de-dios}{%
\subsection{Anuncio del juicio inminente a los exuberantes ricos y los
siervos de Mammon que se olvidan de
Dios}\label{anuncio-del-juicio-inminente-a-los-exuberantes-ricos-y-los-siervos-de-mammon-que-se-olvidan-de-dios}}

\hypertarget{section-4}{%
\section{5}\label{section-4}}

\bibleverse{1} Venid ahora, ricos, llorad y aullad por vuestras miserias
que os sobrevienen. \footnote{\textbf{5:1} Luc 6,24-25} \bibleverse{2}
Vuestras riquezas se han corrompido y vuestros vestidos se han
apolillado. \footnote{\textbf{5:2} Mat 6,19} \bibleverse{3} Vuestro oro
y vuestra plata están corroídos, y su corrosión será para testimonio
contra vosotros y comerá vuestra carne como el fuego. Habéis guardado
vuestro tesoro en los últimos días. \bibleverse{4} He aquí que el
salario de los obreros que segaron tus campos, que tú has retenido con
fraude, clama; y los gritos de los que segaron han entrado en los oídos
del Señor de los Ejércitos. \footnote{\textbf{5:4} Deut 24,14-15}
\bibleverse{5} Habéis vivido con lujo en la tierra, y habéis tomado
vuestro placer. Habéis alimentado vuestros corazones como en un día de
matanza. \footnote{\textbf{5:5} Luc 16,19; Luc 16,25; Jer 12,3; Jer
  25,34} \bibleverse{6} Habéis condenado y habéis asesinado al justo. Él
no se resiste a vosotros. \footnote{\textbf{5:6} Sant 2,6}

\hypertarget{exhortaciuxf3n-a-los-creyentes-a-perseverar-con-paciencia-en-vista-del-inminente-regreso-del-seuxf1or}{%
\subsection{Exhortación a los creyentes a perseverar con paciencia en
vista del inminente regreso del
Señor}\label{exhortaciuxf3n-a-los-creyentes-a-perseverar-con-paciencia-en-vista-del-inminente-regreso-del-seuxf1or}}

\bibleverse{7} Tened, pues, paciencia, hermanos, hasta la venida del
Señor. He aquí que el agricultor espera el precioso fruto de la tierra,
siendo paciente sobre él, hasta que recibe la lluvia temprana y tardía.
\footnote{\textbf{5:7} Luc 21,19; Heb 10,36} \bibleverse{8} Sed también
vosotros pacientes. Afirmad vuestros corazones, porque la venida del
Señor está cerca.

\bibleverse{9} Hermanos, no os quejéis los unos de los otros, para que
no seáis juzgados. Mirad, el juez está a la puerta. \bibleverse{10}
Tomad, hermanos, como ejemplo de sufrimiento y de perseverancia a los
profetas que hablaron en nombre del Señor. \footnote{\textbf{5:10} Mat
  5,12} \bibleverse{11} He aquí que llamamos bienaventurados a los que
soportaron. Habéis oído hablar de la perseverancia de Job y habéis visto
al Señor en el desenlace, y cómo el Señor está lleno de compasión y
misericordia. \footnote{\textbf{5:11} Job 1,21; Job 42,10-16}

\hypertarget{advertencias-finales-sobre-el-juramento-y-la-oraciuxf3n-sobre-el-comportamiento-hacia-la-alegruxeda-y-el-dolor-en-la-enfermedad-y-contra-los-descarriados}{%
\subsection{Advertencias finales sobre el juramento y la oración, sobre
el comportamiento hacia la alegría y el dolor, en la enfermedad y contra
los
descarriados}\label{advertencias-finales-sobre-el-juramento-y-la-oraciuxf3n-sobre-el-comportamiento-hacia-la-alegruxeda-y-el-dolor-en-la-enfermedad-y-contra-los-descarriados}}

\bibleverse{12} Pero sobre todo, hermanos míos, no juréis, ni por el
cielo, ni por la tierra, ni por ningún otro juramento; sino que vuestro
``sí'' sea ``sí'', y vuestro ``no'', ``no'', para no caer en la
hipocresía. \footnote{\textbf{5:12} Mat 5,34-37}

\bibleverse{13} ¿Alguno de vosotros está sufriendo? Que rece. ¿Está
alguno alegre? Que cante alabanzas. \footnote{\textbf{5:13} Sal 50,15;
  Col 3,16} \bibleverse{14} ¿Está alguno de vosotros enfermo? Que llame
a los ancianos de la asamblea, y que oren sobre él, ungiéndolo con
aceite en el nombre del Señor; \footnote{\textbf{5:14} Mar 6,13}
\bibleverse{15} y la oración de fe sanará al enfermo, y el Señor lo
resucitará. Si ha cometido pecados, será perdonado. \footnote{\textbf{5:15}
  Mar 16,18} \bibleverse{16} Confiésense unos a otros sus pecados y oren
unos por otros, para que sean sanados. La oración insistente de una
persona justa es poderosamente eficaz. \footnote{\textbf{5:16} Hech 12,5}
\bibleverse{17} Elías era un hombre con una naturaleza como la nuestra,
y oró con insistencia para que no lloviera, y no llovió sobre la tierra
durante tres años y seis meses. \footnote{\textbf{5:17} 1Re 17,1; Luc
  4,25} \bibleverse{18} Volvió a orar, y el cielo dio lluvia, y la
tierra produjo su fruto. \footnote{\textbf{5:18} 1Re 18,41-45}

\bibleverse{19} Hermanos, si alguno de vosotros se aleja de la verdad y
alguien lo hace volver, \footnote{\textbf{5:19} Gal 6,1} \bibleverse{20}
que sepa que quien hace volver a un pecador del error de su camino
salvará un alma de la muerte y cubrirá una multitud de pecados.
\footnote{\textbf{5:20} Sal 51,13; Prov 10,12; 1Pe 4,8}
