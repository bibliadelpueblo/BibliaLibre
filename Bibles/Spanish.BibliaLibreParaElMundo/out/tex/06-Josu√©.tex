\hypertarget{la-comisiuxf3n-de-dios-de-conquistar-y-animar-a-josuuxe9-preparativos-para-cruzar-el-jorduxe1n}{%
\subsection{La comisión de Dios de conquistar y animar a Josué;
Preparativos para cruzar el
Jordán}\label{la-comisiuxf3n-de-dios-de-conquistar-y-animar-a-josuuxe9-preparativos-para-cruzar-el-jorduxe1n}}

\hypertarget{section}{%
\section{1}\label{section}}

\bibleverse{1} Después de la muerte de Moisés, siervo de
Yahvé,\footnote{\textbf{1:1} ``Yahvé'' es el nombre propio de Dios, a
  veces traducido como ``\textsc{Señor}'' (en mayúsculas) en otras
  traducciones.} Yahvé habló a Josué hijo de Nun, siervo de Moisés,
diciendo: \bibleverse{2} ``Moisés, mi siervo, ha muerto. Ahora, pues,
levántate y cruza este Jordán, tú y todo este pueblo, hacia la tierra
que les voy a dar a los hijos de Israel. \bibleverse{3} Os he dado todo
lugar que pise la planta de vuestro pie, como se lo dije a Moisés.
\footnote{\textbf{1:3} Deut 11,24} \bibleverse{4} Desde el desierto y
este Líbano hasta el gran río, el río Éufrates, toda la tierra de los
hititas, y hasta el gran mar hacia la puesta del sol, será vuestro
límite. \bibleverse{5} Ningún hombre podrá hacer frente a ti todos los
días de tu vida. Como estuve con Moisés, así estaré contigo. No te
fallaré ni te abandonaré. \footnote{\textbf{1:5} Deut 31,8; Heb 13,5}

\bibleverse{6} ``Sé fuerte y valiente, porque harás que este pueblo
herede la tierra que juré darles a sus padres. \footnote{\textbf{1:6}
  Deut 3,28; Deut 31,7; Deut 31,23} \bibleverse{7} Sólo sé fuerte y muy
valiente. Tengan cuidado de cumplir con toda la ley que mi siervo Moisés
les ordenó. No te apartes de ella ni a la derecha ni a la izquierda,
para que tengas buen éxito dondequiera que vayas. \footnote{\textbf{1:7}
  Deut 5,29; 1Re 2,3} \bibleverse{8} Este libro de la ley no se apartará
de tu boca, sino que meditarás en él de día y de noche, para que guardes
y hagas conforme a todo lo que en él está escrito; porque entonces harás
próspero tu camino y tendrás buen éxito. \footnote{\textbf{1:8} Sal
  1,2-3} \bibleverse{9} ¿No te lo he ordenado? Sé fuerte y valiente. No
tengas miedo. No te desanimes, porque Yahvé, tu Dios,\footnote{\textbf{1:9}
  La palabra hebrea traducida como ``Dios'' es ``\hebrew{אֱלֹהִ֑ים}''
  (Elohim).} está contigo dondequiera que vayas''.

\hypertarget{josuuxe9-ordena-a-la-gente-que-estuxe9-lista-para-marchar}{%
\subsection{Josué ordena a la gente que esté lista para
marchar}\label{josuuxe9-ordena-a-la-gente-que-estuxe9-lista-para-marchar}}

\bibleverse{10} Entonces Josué ordenó a los oficiales del pueblo,
diciendo: \bibleverse{11} ``Pasen por el medio del campamento y manden
al pueblo, diciendo: ``Preparen la comida, porque dentro de tres días
van a pasar este Jordán, para entrar a poseer la tierra que Yahvé su
Dios les da para que la posean''\,''.

\hypertarget{comportamiento-obediente-de-las-tribus-de-cisjordania}{%
\subsection{Comportamiento obediente de las tribus de
Cisjordania}\label{comportamiento-obediente-de-las-tribus-de-cisjordania}}

\bibleverse{12} Josué habló a los rubenitas, a los gaditas y a la media
tribu de Manasés, diciendo: \bibleverse{13} ``Acuérdate de la palabra
que Moisés, siervo de Yavé, te ordenó, diciendo: `Yavé, tu Dios, te da
descanso y te dará esta tierra. \footnote{\textbf{1:13} Núm 32,20}
\bibleverse{14} Vuestras mujeres, vuestros niños y vuestros ganados
vivirán en la tierra que Moisés os dio al otro lado del Jordán; pero
vosotros pasaréis delante de vuestros hermanos armados, todos los
hombres valientes, y les ayudaréis \bibleverse{15} hasta que Yahvé haya
dado descanso a vuestros hermanos, como os lo ha dado a vosotros, y
ellos también hayan poseído la tierra que Yahvé vuestro Dios les da.
Entonces volveréis a la tierra de vuestra posesión y la poseeréis, que
Moisés, siervo de Yahvé, os dio al otro lado del Jordán, hacia el
amanecer.'\,''

\bibleverse{16} Ellos respondieron a Josué diciendo: ``Haremos todo lo
que nos has mandado, e iremos a donde nos mandes. \bibleverse{17} Así
como escuchamos a Moisés en todo, así te escucharemos a ti. Sólo que el
Señor, su Dios, esté con ustedes, como estuvo con Moisés.
\bibleverse{18} El que se rebele contra tu mandamiento y no escuche tus
palabras en todo lo que le mandes, él mismo morirá. Sólo sé fuerte y
valiente''. \footnote{\textbf{1:18} Jos 1,6}

\hypertarget{el-escultismo-de-jericuxf3-la-salvaciuxf3n-de-los-dos-espuxedas-por-la-prostituta-rahab}{%
\subsection{El escultismo de Jericó; la salvación de los dos espías por
la prostituta
Rahab}\label{el-escultismo-de-jericuxf3-la-salvaciuxf3n-de-los-dos-espuxedas-por-la-prostituta-rahab}}

\hypertarget{section-1}{%
\section{2}\label{section-1}}

\bibleverse{1} Josué, hijo de Nun, envió en secreto a dos hombres desde
Sitim como espías, diciendo: ``Vayan a ver la tierra, incluida Jericó''.
Fueron y entraron en la casa de una prostituta que se llamaba Rahab, y
durmieron allí. \footnote{\textbf{2:1} Sant 2,25; Heb 11,31}

\bibleverse{2} Se le dijo al rey de Jericó: ``He aquí,\footnote{\textbf{2:2}
  ``He aquí'', de ``\hebrew{הִנֵּה}'', significa mirar, fijarse, observar,
  ver o contemplar. Se utiliza a menudo como interjección.} hombres de
los hijos de Israel han entrado aquí esta noche para espiar la tierra''.

\bibleverse{3} El rey de Jericó envió a decir a Rahab: ``Saca a los
hombres que han venido a ti, que han entrado en tu casa, porque han
venido a espiar toda la tierra.''

\bibleverse{4} La mujer tomó a los dos hombres y los escondió. Luego
dijo: ``Sí, los hombres vinieron a mí, pero no sabía de dónde venían.
\bibleverse{5} A la hora de cerrar la puerta, cuando ya estaba oscuro,
los hombres salieron. No sé adónde fueron los hombres. Perseguidlos
rápidamente. Tal vez los alcances''. \bibleverse{6} Pero ella los había
subido al tejado y los había escondido bajo los tallos de lino que había
puesto en orden en el tejado. \bibleverse{7} Los hombres los
persiguieron por el camino hasta los vados del río Jordán. En cuanto
salieron los que los perseguían, cerraron la puerta.

\hypertarget{las-negociaciones-y-citas-fijas-entre-rahab-y-los-espuxedas}{%
\subsection{Las negociaciones y citas fijas entre Rahab y los
espías}\label{las-negociaciones-y-citas-fijas-entre-rahab-y-los-espuxedas}}

\bibleverse{8} Antes de que se acostaran, ella se acercó a ellos en el
tejado. \bibleverse{9} Ella dijo a los hombres: ``Sé que Yavé les ha
dado la tierra, y que el temor a ustedes ha caído sobre nosotros, y que
todos los habitantes de la tierra se derriten ante ustedes. \footnote{\textbf{2:9}
  Éxod 23,27} \bibleverse{10} Porque hemos oído cómo el Señor secó las
aguas del Mar Rojo delante de ustedes, cuando salieron de Egipto, y lo
que hicieron a los dos reyes de los amorreos que estaban al otro lado
del Jordán, a Sehón y a Og, a quienes destruyeron por completo.
\footnote{\textbf{2:10} Éxod 14,21; Núm 21,24; Núm 21,35}
\bibleverse{11} En cuanto lo oímos, se nos derritió el corazón, y no
hubo más espíritu en ningún hombre, a causa de ti; porque Yahvé, tu
Dios, es Dios en lo alto del cielo y en lo bajo de la tierra.
\footnote{\textbf{2:11} Jos 5,1; Éxod 15,14-15; Deut 4,39}
\bibleverse{12} Ahora, pues, júrame por Yahvé, ya que me he portado bien
contigo, que tú también te portarás bien con la casa de mi padre y me
darás una señal verdadera; \footnote{\textbf{2:12} Jos 6,23; Jos 6,25}
\bibleverse{13} y que salvarás con vida a mi padre, a mi madre, a mis
hermanos y a mis hermanas, y a todo lo que tienen, y que librarás
nuestras vidas de la muerte.''

\bibleverse{14} Los hombres le dijeron: ``Nuestra vida por la tuya, si
no hablas de este asunto nuestro; y será, cuando Yahvé nos dé la tierra,
que trataremos amable y verdaderamente contigo.''

\bibleverse{15} Entonces los hizo descender con una cuerda por la
ventana, pues su casa estaba al lado de la muralla, y ella vivía en la
muralla. \bibleverse{16} Les dijo: ``Id al monte, no sea que os
encuentren los perseguidores. Escóndanse allí tres días, hasta que los
perseguidores hayan regresado. Después, podéis seguir vuestro camino''.

\bibleverse{17} Los hombres le dijeron: ``Quedaremos libres de este
juramento que nos has hecho hacer. \bibleverse{18} Mira, cuando
lleguemos a la tierra, ata este cordón de hilo escarlata en la ventana
que usaste para dejarnos bajar. Reúne en la casa a tu padre, a tu madre,
a tus hermanos y a toda la familia de tu padre. \bibleverse{19} El que
salga de las puertas de tu casa a la calle, su sangre estará sobre su
cabeza, y nosotros seremos inocentes. El que esté contigo en la casa, su
sangre será sobre nuestra cabeza, si alguna mano lo toca.
\bibleverse{20} Pero si hablas de este asunto nuestro, seremos inocentes
de tu juramento que nos has hecho hacer.''

\bibleverse{21} Ella dijo: ``Que sea como has dicho''. Los despidió y se
marcharon. Luego ató el cordón de grana en la ventana.

\hypertarget{feliz-regreso-de-los-espuxedas-a-josuuxe9-con-buenas-noticias}{%
\subsection{Feliz regreso de los espías a Josué con buenas
noticias}\label{feliz-regreso-de-los-espuxedas-a-josuuxe9-con-buenas-noticias}}

\bibleverse{22} Fueron y llegaron a la montaña, y se quedaron allí tres
días, hasta que los perseguidores regresaron. Los perseguidores los
buscaron por todo el camino, pero no los encontraron. \bibleverse{23}
Entonces los dos hombres regresaron, bajaron del monte, cruzaron el río
y se acercaron a Josué, hijo de Nun. Le contaron todo lo que les había
sucedido. \bibleverse{24} Le dijeron a Josué: ``Verdaderamente el Señor
ha entregado toda la tierra en nuestras manos. Además, todos los
habitantes de la tierra se derriten ante nosotros''. \footnote{\textbf{2:24}
  Jos 2,9}

\hypertarget{llegada-al-jorduxe1n-anuncio-de-los-presidentes-y-dos-uxf3rdenes-de-joshua-salida-del-pueblo}{%
\subsection{Llegada al Jordán; Anuncio de los presidentes y dos órdenes
de Joshua; Salida del
pueblo}\label{llegada-al-jorduxe1n-anuncio-de-los-presidentes-y-dos-uxf3rdenes-de-joshua-salida-del-pueblo}}

\hypertarget{section-2}{%
\section{3}\label{section-2}}

\bibleverse{1} Josué se levantó temprano por la mañana; partieron de
Sitim y llegaron al Jordán, él y todos los hijos de Israel. Acamparon
allí antes de cruzar. \footnote{\textbf{3:1} Núm 25,1} \bibleverse{2}
Después de tres días, los oficiales pasaron por el medio del campamento;
\bibleverse{3} y ordenaron al pueblo, diciendo: ``Cuando veáis el arca
de la alianza de Yahvé vuestro Dios, y a los sacerdotes levitas que la
llevan, dejad vuestro lugar y seguidla. \bibleverse{4} Pero habrá un
espacio entre vosotros y ella de unos dos mil codos\footnote{\textbf{3:4}
  Un codo es la longitud desde la punta del dedo corazón hasta el codo
  del brazo de un hombre, es decir, unas 18 pulgadas o 46 centímetros,
  por lo que 2.000 codos son unos 920 metros.} por medida --- no os
acerquéis a ella --- para que sepáis el camino por el que debéis ir,
porque nunca habéis pasado por aquí.''

\bibleverse{5} Josué dijo al pueblo: ``Santificaos, porque mañana Yahvé
hará maravillas entre vosotros''. \footnote{\textbf{3:5} Éxod 19,10}

\bibleverse{6} Josué habló a los sacerdotes, diciendo: ``Tomen el arca
de la alianza y pasen delante del pueblo''. Tomaron el arca de la
alianza y pasaron delante del pueblo. \footnote{\textbf{3:6} Jos 6,6}

\hypertarget{la-promesa-de-dios-de-salvaciuxf3n-a-josuuxe9-anuncio-del-milagro-divino-a-travuxe9s-de-josuuxe9}{%
\subsection{La promesa de Dios de salvación a Josué; Anuncio del milagro
divino a través de
Josué}\label{la-promesa-de-dios-de-salvaciuxf3n-a-josuuxe9-anuncio-del-milagro-divino-a-travuxe9s-de-josuuxe9}}

\bibleverse{7} Yahvé dijo a Josué: ``Hoy comenzaré a engrandecerte a los
ojos de todo Israel, para que sepan que como estuve con Moisés, así
estaré contigo. \footnote{\textbf{3:7} Jos 4,14; Jos 1,5; Jos 1,17}
\bibleverse{8} Ordenarás a los sacerdotes que llevan el arca de la
alianza que digan: ``Cuando lleguéis al borde de las aguas del Jordán,
os detendréis en el Jordán.''

\bibleverse{9} Josué dijo a los hijos de Israel: ``Venid aquí y escuchad
las palabras de Yahvé, vuestro Dios''. \bibleverse{10} Josué dijo: ``En
esto sabréis que el Dios vivo está en medio de vosotros, y que sin falta
expulsará de vuestra presencia al cananeo, al hitita, al heveo, al
ferezeo, al gergeseo, al amorreo y al jebuseo. \bibleverse{11} He aquí
que el arca del pacto del Señor\footnote{\textbf{3:11} La palabra
  traducida ``Señor'' es ``Adonai''.} de toda la tierra pasa delante de
vosotros al Jordán. \bibleverse{12} Toma, pues, doce hombres de las
tribus de Israel, un hombre por cada tribu. \bibleverse{13} Cuando las
plantas de los pies de los sacerdotes que llevan el arca de Yavé, el
Señor de toda la tierra, descansen en las aguas del Jordán, las aguas
del Jordán se cortarán. Las aguas que descienden de lo alto se pararán
en un montón''.

\hypertarget{el-jorduxe1n-se-detiene-y-se-divide}{%
\subsection{El Jordán se detiene y se
divide}\label{el-jorduxe1n-se-detiene-y-se-divide}}

\bibleverse{14} Cuando el pueblo salió de sus tiendas para pasar el
Jordán, los sacerdotes que llevaban el arca de la alianza iban delante
del pueblo, \bibleverse{15} y cuando los que llevaban el arca llegaron
al Jordán, y los pies de los sacerdotes que llevaban el arca se
sumergieron en la orilla del agua (porque el Jordán se desborda por
todas sus orillas todo el tiempo de la cosecha), \bibleverse{16} las
aguas que descendían de arriba se detuvieron, y subieron en un montón a
gran distancia, en Adam, la ciudad que está junto a Zaretán; y las que
descendían hacia el mar del Arabá, el Mar Salado, fueron totalmente
cortadas. Luego el pueblo pasó cerca de Jericó. \footnote{\textbf{3:16}
  Éxod 14,21; Sal 114,3} \bibleverse{17} Los sacerdotes que llevaban el
arca de la alianza de Yavé se mantuvieron firmes en tierra seca en medio
del Jordán, y todo Israel cruzó en tierra seca, hasta que toda la nación
pasó completamente el Jordán.

\hypertarget{erigiendo-un-monumento-de-piedra-en-el-lecho-del-jorduxe1n-y-otro-en-la-otra-orilla-de-gilgal}{%
\subsection{Erigiendo un monumento de piedra en el lecho del Jordán y
otro en la otra orilla de
Gilgal}\label{erigiendo-un-monumento-de-piedra-en-el-lecho-del-jorduxe1n-y-otro-en-la-otra-orilla-de-gilgal}}

\hypertarget{section-3}{%
\section{4}\label{section-3}}

\bibleverse{1} Cuando toda la nación hubo cruzado completamente el
Jordán, Yahvé habló a Josué, diciendo: \bibleverse{2} ``Toma doce
hombres del pueblo, un hombre de cada tribu, \bibleverse{3} y mándales
decir: ``Toma de la mitad del Jordán, del lugar donde los pies de los
sacerdotes estaban firmes, doce piedras, llévalas contigo y ponlas en el
lugar donde acamparás esta noche.''

\bibleverse{4} Entonces Josué llamó a los doce hombres que había
preparado de los hijos de Israel, un hombre de cada tribu.
\bibleverse{5} Josué les dijo: ``Crucen delante del arca de Yavé su Dios
hasta la mitad del Jordán, y cada uno de ustedes tome una piedra y
póngasela al hombro, según el número de las tribus de los hijos de
Israel; \bibleverse{6} para que esto sea una señal entre ustedes, de
modo que cuando sus hijos pregunten en el futuro, diciendo: ``¿Qué
significan estas piedras?'' \footnote{\textbf{4:6} Éxod 12,26}
\bibleverse{7} entonces les dirán: ``Porque las aguas del Jordán fueron
cortadas delante del arca del pacto de Yavé. Cuando cruzó el Jordán, las
aguas del Jordán fueron cortadas. Estas piedras serán para memoria de
los hijos de Israel para siempre'\,''.

\bibleverse{8} Los hijos de Israel hicieron lo que Josué les había
ordenado, y tomaron doce piedras del medio del Jordán, tal como el Señor
le había dicho a Josué, según el número de las tribus de los hijos de
Israel. Las llevaron consigo hasta el lugar donde acamparon, y las
depositaron allí. \bibleverse{9} Josué levantó doce piedras en medio del
Jordán, en el lugar donde estaban los pies de los sacerdotes que
llevaban el arca de la alianza; y allí están hasta el día de hoy.
\bibleverse{10} Porque los sacerdotes que llevaban el arca se pararon en
medio del Jordán hasta que se terminó todo lo que el Señor le había
ordenado a Josué que dijera al pueblo, según todo lo que Moisés le había
ordenado a Josué; y el pueblo se apresuró a pasar. \bibleverse{11}
Cuando todo el pueblo hubo cruzado completamente, el arca de Yavé cruzó
con los sacerdotes en presencia del pueblo.

\bibleverse{12} Los hijos de Rubén, los hijos de Gad y la media tribu de
Manasés pasaron armados delante de los hijos de Israel, tal como Moisés
les había dicho. \footnote{\textbf{4:12} Núm 32,21; Núm 32,29}
\bibleverse{13} Unos cuarenta mil hombres, listos y armados para la
guerra, pasaron delante de Yavé a la batalla, a las llanuras de Jericó.

\hypertarget{efecto-del-maravilloso-evento-sobre-los-israelitas-y-sobre-todos-los-pueblos-informaciuxf3n-final}{%
\subsection{Efecto del maravilloso evento sobre los israelitas y sobre
todos los pueblos; información
final}\label{efecto-del-maravilloso-evento-sobre-los-israelitas-y-sobre-todos-los-pueblos-informaciuxf3n-final}}

\bibleverse{14} Aquel día, el Señor engrandeció a Josué a los ojos de
todo Israel, y le temieron como a Moisés todos los días de su vida.
\footnote{\textbf{4:14} Jos 3,7}

\bibleverse{15} Yahvé habló a Josué, diciendo: \bibleverse{16} ``Ordena
a los sacerdotes que llevan el arca de la alianza que suban del
Jordán''.

\bibleverse{17} Por eso Josué ordenó a los sacerdotes diciendo: ``¡Suban
del Jordán!'' \bibleverse{18} Cuando los sacerdotes que llevaban el arca
de la alianza de Yavé subieron por la mitad del Jordán, y las plantas de
los pies de los sacerdotes se alzaron hasta la tierra seca, las aguas
del Jordán volvieron a su lugar y se desbordaron por todas sus orillas,
como antes. \bibleverse{19} El pueblo subió del Jordán el décimo día del
primer mes y acampó en Gilgal, en el límite oriental de Jericó.
\footnote{\textbf{4:19} Jos 5,9}

\bibleverse{20} Josué colocó en Gilgal las doce piedras que sacaron del
Jordán. \bibleverse{21} Habló a los hijos de Israel, diciendo: ``Cuando
vuestros hijos pregunten a sus padres en el futuro, diciendo: ``¿Qué
significan estas piedras?'' \footnote{\textbf{4:21} Jos 4,6}
\bibleverse{22} Entonces se lo haréis saber a vuestros hijos, diciendo:
``Israel pasó este Jordán en seco. \bibleverse{23} Porque Yahvé tu Dios
secó las aguas del Jordán delante de ti hasta que cruzaste, como Yahvé
tu Dios hizo con el Mar Rojo, que secó delante de nosotros hasta que
cruzamos, \footnote{\textbf{4:23} Éxod 14,21-22} \bibleverse{24} para
que todos los pueblos de la tierra sepan que la mano de Yahvé es
poderosa, y para que temas a Yahvé tu Dios para siempre.'\,''

\hypertarget{realizar-la-circuncisiuxf3n-de-israel}{%
\subsection{Realizar la circuncisión de
Israel}\label{realizar-la-circuncisiuxf3n-de-israel}}

\hypertarget{section-4}{%
\section{5}\label{section-4}}

\bibleverse{1} Cuando todos los reyes de los amorreos, que estaban al
otro lado del Jordán, hacia el oeste, y todos los reyes de los cananeos,
que estaban junto al mar, oyeron cómo Yahvé había secado las aguas del
Jordán desde delante de los hijos de Israel hasta que habíamos cruzado,
se les derritió el corazón, y no hubo más espíritu en ellos, a causa de
los hijos de Israel. \footnote{\textbf{5:1} Jos 3,16; Jos 2,24}
\bibleverse{2} En aquel tiempo, Yahvé dijo a Josué: ``Haz cuchillos de
pedernal y vuelve a circuncidar a los hijos de Israel por segunda vez.''
\footnote{\textbf{5:2} Éxod 4,25} \bibleverse{3} Josué se hizo cuchillos
de pedernal y circuncidó a los hijos de Israel en el monte de los
prepucios. \bibleverse{4} Esta es la razón por la que Josué los
circuncidó: todo el pueblo que salió de Egipto, que era varón, incluso
todos los hombres de guerra, murieron en el desierto a lo largo del
camino, después de que salieron de Egipto. \bibleverse{5} Porque todo el
pueblo que salió fue circuncidado; pero todo el pueblo que nació en el
desierto a lo largo del camino al salir de Egipto no había sido
circuncidado. \bibleverse{6} Porque los hijos de Israel anduvieron
cuarenta años en el desierto hasta que toda la nación, incluso los
hombres de guerra que salieron de Egipto, fueron consumidos, porque no
escucharon la voz de Yavé. Yahvé les juró que no les dejaría ver la
tierra que Yahvé juró a sus padres que nos daría, una tierra que mana
leche y miel. \footnote{\textbf{5:6} Núm 14,22-23} \bibleverse{7} Sus
hijos, a los que levantó en su lugar, fueron circuncidados por Josué, ya
que eran incircuncisos, porque no los habían circuncidado en el camino.
\bibleverse{8} Cuando terminaron de circuncidar a toda la nación, se
quedaron en sus lugares en el campamento hasta que se curaron.

\bibleverse{9} El Señor le dijo a Josué: ``Hoy he quitado de ti el
oprobio de Egipto''. Por eso el nombre de aquel lugar se llamó
Gilgal\footnote{\textbf{5:9} ``Gilgal'' suena como el hebreo para
  ``rollo''.} hasta el día de hoy.

\hypertarget{primera-pascua-en-canauxe1n-cese-del-manuxe1}{%
\subsection{Primera Pascua en Canaán; Cese del
maná}\label{primera-pascua-en-canauxe1n-cese-del-manuxe1}}

\bibleverse{10} Los hijos de Israel acamparon en Gilgal. Celebraron la
Pascua el día catorce del mes, al atardecer, en las llanuras de Jericó.
\footnote{\textbf{5:10} Éxod 12,6; Lev 23,5} \bibleverse{11} Comieron
tortas sin levadura y grano tostado de los productos de la tierra al día
siguiente de la Pascua, en el mismo día. \bibleverse{12} El maná cesó al
día siguiente, después de que comieron de los productos de la tierra.
Los hijos de Israel ya no tuvieron maná, sino que comieron del fruto de
la tierra de Canaán ese año. \footnote{\textbf{5:12} Éxod 16,35}

\hypertarget{josuuxe9-se-siente-animado-por-la-apariciuxf3n-del-divino-general}{%
\subsection{Josué se siente animado por la aparición del divino
general}\label{josuuxe9-se-siente-animado-por-la-apariciuxf3n-del-divino-general}}

\bibleverse{13} Cuando Josué estaba junto a Jericó, alzó los ojos y
miró, y he aquí que un hombre estaba frente a él con la espada
desenvainada en la mano. Josué se acercó a él y le dijo: ``¿Estás a
favor nuestro o de nuestros enemigos?''. \footnote{\textbf{5:13} Núm
  22,23; Núm 22,31}

\bibleverse{14} Él dijo: ``No; pero he venido ahora como comandante del
ejército de Yahvé''. Josué se postró en tierra y adoró, y le preguntó:
``¿Qué dice mi señor a su siervo?''. \footnote{\textbf{5:14} Éxod 14,19}

\bibleverse{15} El príncipe del ejército de Yahvé le dijo a Josué:
``Quítate las sandalias, porque el lugar donde estás parado es
sagrado''. Josué así lo hizo. \footnote{\textbf{5:15} Éxod 3,5}

\hypertarget{dios-le-enseuxf1uxf3-a-josuuxe9-cuxf3mo-conquistar-jericuxf3}{%
\subsection{Dios le enseñó a Josué cómo conquistar
Jericó}\label{dios-le-enseuxf1uxf3-a-josuuxe9-cuxf3mo-conquistar-jericuxf3}}

\hypertarget{section-5}{%
\section{6}\label{section-5}}

\bibleverse{1} Jericó estaba fuertemente cerrada a causa de los hijos de
Israel. Nadie salía ni entraba. \bibleverse{2} El Señor le dijo a Josué:
``He aquí que he entregado Jericó en tus manos, con su rey y los hombres
valientes. \bibleverse{3} Todos tus hombres de guerra marcharán
alrededor de la ciudad, dándole una vuelta. Lo harán durante seis días.
\bibleverse{4} Siete sacerdotes llevarán siete trompetas de cuernos de
carnero ante el arca. Al séptimo día, marcharéis alrededor de la ciudad
siete veces, y los sacerdotes tocarán las trompetas. \footnote{\textbf{6:4}
  Lev 25,9} \bibleverse{5} Sucederá que cuando hagan un toque largo con
el cuerno de carnero, y cuando oigáis el sonido de la trompeta, todo el
pueblo gritará con un gran alarido; entonces el muro de la ciudad caerá
de plano, y el pueblo subirá, cada uno delante de sí.''

\hypertarget{los-desfiles-diarios-uxfanicos-por-la-ciudad-durante-los-primeros-seis-duxedas}{%
\subsection{Los desfiles diarios únicos por la ciudad durante los
primeros seis
días}\label{los-desfiles-diarios-uxfanicos-por-la-ciudad-durante-los-primeros-seis-duxedas}}

\bibleverse{6} Josué hijo de Nun llamó a los sacerdotes y les dijo:
``Suban el arca de la alianza y que siete sacerdotes lleven siete
trompetas de cuernos de carnero ante el arca de Yavé.''

\bibleverse{7} Dijeron al pueblo: ``¡Adelante! Marchen alrededor de la
ciudad, y dejen pasar a los hombres armados ante el arca de Yahvé''.

\bibleverse{8} Cuando Josué habló al pueblo, los siete sacerdotes que
llevaban las siete trompetas de cuernos de carnero delante de Yavé
avanzaron y tocaron las trompetas, y el arca de la alianza de Yavé los
siguió. \bibleverse{9} Los hombres armados iban delante de los
sacerdotes que tocaban las trompetas, y el arca iba detrás de ellos. Las
trompetas sonaban mientras avanzaban.

\bibleverse{10} Josué ordenó al pueblo diciendo: ``No gritaréis ni
dejaréis oír vuestra voz, ni saldrá palabra alguna de vuestra boca hasta
el día en que os diga que gritéis. Entonces gritaréis''. \bibleverse{11}
Entonces hizo que el arca de Yahvé diera una vuelta a la ciudad,
rodeándola una vez. Luego entraron en el campamento y se quedaron en él.
\bibleverse{12} Josué se levantó de madrugada, y los sacerdotes subieron
el arca de Yavé. \bibleverse{13} Los siete sacerdotes que llevaban las
siete trompetas de cuernos de carnero delante del arca de Yavé iban
continuamente tocando las trompetas. Los hombres armados iban delante de
ellos. La retaguardia iba detrás del arca de Yahvé. Las trompetas
sonaban a su paso. \bibleverse{14} El segundo día dieron una vuelta a la
ciudad y volvieron al campamento. Hicieron esto durante seis días.

\hypertarget{los-siete-desfiles-del-suxe9ptimo-duxeda-conquista-y-destrucciuxf3n-de-la-ciudad}{%
\subsection{Los siete desfiles del séptimo día; Conquista y destrucción
de la
ciudad}\label{los-siete-desfiles-del-suxe9ptimo-duxeda-conquista-y-destrucciuxf3n-de-la-ciudad}}

\bibleverse{15} Al séptimo día, se levantaron temprano al amanecer y
marcharon alrededor de la ciudad de la misma manera siete veces. Sólo
este día marcharon alrededor de la ciudad siete veces. \bibleverse{16} A
la séptima vez, cuando los sacerdotes tocaron las trompetas, Josué dijo
al pueblo: ``¡Griten, porque el Señor les ha entregado la ciudad!
\bibleverse{17} La ciudad será consagrada, ella y todo lo que hay en
ella, a Yavé. Sólo Rahab la prostituta vivirá, ella y todos los que
están con ella en la casa, porque escondió a los mensajeros que
enviamos. \footnote{\textbf{6:17} Núm 21,2; Jos 2,12-13; Heb 11,31}
\bibleverse{18} En cuanto a ustedes, sólo guárdense de lo que está
consagrado a la destrucción, no sea que cuando lo hayan consagrado,
tomen de lo consagrado; así harían maldito el campamento de Israel y lo
perturbarían. \footnote{\textbf{6:18} Lev 27,28; Deut 13,17}
\bibleverse{19} Pero toda la plata, el oro y los objetos de bronce y de
hierro son sagrados para Yavé. Entrarán en el tesoro de Yahvé''.

\bibleverse{20} Entonces el pueblo gritó y los sacerdotes tocaron las
trompetas. Al oír el sonido de la trompeta, el pueblo gritó con gran
estruendo, y la muralla se derrumbó, de modo que el pueblo subió a la
ciudad, cada uno por su lado, y tomaron la ciudad. \footnote{\textbf{6:20}
  Heb 11,30} \bibleverse{21} Destruyeron todo lo que había en la ciudad,
hombres y mujeres, jóvenes y ancianos, bueyes, ovejas y asnos, a filo de
espada.

\hypertarget{perdonando-a-rahab-y-sus-parientes-maldice-la-reconstrucciuxf3n-de-la-ciudad}{%
\subsection{Perdonando a Rahab y sus parientes; Maldice la
reconstrucción de la
ciudad}\label{perdonando-a-rahab-y-sus-parientes-maldice-la-reconstrucciuxf3n-de-la-ciudad}}

\bibleverse{22} Josué dijo a los dos hombres que habían espiado la
tierra: ``Vayan a la casa de la prostituta y saquen de allí a la mujer y
todo lo que tiene, como se lo juraron''. \footnote{\textbf{6:22} Jos
  2,14} \bibleverse{23} Los jóvenes espías entraron y sacaron a Rahab
con su padre, su madre, sus hermanos y todo lo que tenía. También
sacaron a todos sus parientes, y los pusieron fuera del campamento de
Israel. \footnote{\textbf{6:23} Núm 31,19} \bibleverse{24} Quemaron la
ciudad con fuego y todo lo que había en ella. Sólo pusieron la plata, el
oro y los utensilios de bronce y de hierro en el tesoro de la casa de
Yahvé. \bibleverse{25} Pero Josué salvó con vida a Rahab, la prostituta,
la casa de su padre y todo lo que tenía. Ella vive hasta hoy en medio de
Israel, porque escondió a los mensajeros que Josué envió a espiar
Jericó. \footnote{\textbf{6:25} Mat 1,5; Jue 1,25}

\bibleverse{26} Josué les ordenó con un juramento en ese momento,
diciendo: ``Maldito sea el hombre ante Yahvé que se levante y construya
esta ciudad Jericó. Con la pérdida de su primogénito pondrá sus
cimientos, y con la pérdida de su hijo menor levantará sus puertas.''
\footnote{\textbf{6:26} 1Re 16,34} \bibleverse{27} El Señor estaba con
Josué, y su fama se extendía por todo el país.

\hypertarget{fracaso-del-movimiento-cuidadosamente-preparado-contra-ai-desuxe1nimo-del-pueblo-oraciuxf3n-suplicante-de-josuuxe9}{%
\subsection{Fracaso del movimiento cuidadosamente preparado contra Ai;
Desánimo del pueblo; Oración suplicante de
Josué}\label{fracaso-del-movimiento-cuidadosamente-preparado-contra-ai-desuxe1nimo-del-pueblo-oraciuxf3n-suplicante-de-josuuxe9}}

\hypertarget{section-6}{%
\section{7}\label{section-6}}

\bibleverse{1} Pero los hijos de Israel cometieron una transgresión en
las cosas consagradas, pues Acán, hijo de Carmi, hijo de Zabdi, hijo de
Zera, de la tribu de Judá, tomó algunas de las cosas consagradas. Por
eso la ira de Yahvé ardió contra los hijos de Israel. \footnote{\textbf{7:1}
  Jos 6,18} \bibleverse{2} Josué envió hombres desde Jericó a Hai, que
está junto a Bet-Aven, al este de Bet-El, y les habló diciendo: ``Suban
a espiar la tierra''. Los hombres subieron y divisaron a Hai.
\bibleverse{3} Volvieron a Josué y le dijeron: ``No dejes subir a todo
el pueblo, sino que suban unos dos o tres mil hombres y ataquen a Hai.
No hagas que todo el pueblo trabaje allí, pues son pocos''.
\bibleverse{4} Subieron, pues, unos tres mil hombres del pueblo y
huyeron ante los hombres de Hai. \bibleverse{5} Los hombres de Hai
hirieron a unos treinta y seis hombres de ellos. Los persiguieron desde
delante de la puerta hasta Sebarim, y los hirieron al bajar. El corazón
del pueblo se derritió y se volvió como agua. \bibleverse{6} Josué se
rasgó las vestiduras y se postró en tierra sobre su rostro ante el arca
de Yavé hasta el atardecer, él y los ancianos de Israel, y se pusieron
polvo en la cabeza. \bibleverse{7} Josué dijo: ``Ay, Señor Yahvé, ¿por
qué has hecho pasar a este pueblo por el Jordán, para entregarnos en
manos de los amorreos y hacernos perecer? ¡Ojalá nos hubiéramos
contentado y hubiéramos vivido más allá del Jordán! \bibleverse{8} Oh,
Señor, ¿qué voy a decir, después de que Israel haya dado la espalda ante
sus enemigos? \bibleverse{9} Porque los cananeos y todos los habitantes
del país se enterarán, nos rodearán y borrarán nuestro nombre de la
tierra. ¿Qué harás por tu gran nombre?'' \footnote{\textbf{7:9} Éxod
  32,12}

\hypertarget{dios-le-dice-a-josuuxe9-la-razuxf3n-de-su-enojo-y-le-da-instrucciones-para-determinar-quiuxe9n-es-culpable}{%
\subsection{Dios le dice a Josué la razón de su enojo y le da
instrucciones para determinar quién es
culpable}\label{dios-le-dice-a-josuuxe9-la-razuxf3n-de-su-enojo-y-le-da-instrucciones-para-determinar-quiuxe9n-es-culpable}}

\bibleverse{10} Yahvé dijo a Josué: ``¡Levántate! ¿Por qué has caído de
bruces así? \bibleverse{11} Israel ha pecado. Sí, incluso han
transgredido mi pacto que les ordené. Sí, incluso han tomado algunas de
las cosas consagradas, y también han robado, y también han engañado.
Incluso han puesto entre sus propias cosas. \bibleverse{12} Por eso los
hijos de Israel no pueden resistir ante sus enemigos. Dan la espalda
ante sus enemigos, porque se han convertido en devotos para la
destrucción. No estaré más con ustedes, a menos que destruyan las cosas
consagradas de entre ustedes. \bibleverse{13} ¡Levántate! Santificad al
pueblo y decid: `Santificaos para mañana, porque Yahvé, el Dios de
Israel, dice: ``Hay una cosa consagrada entre vosotros, Israel. No
podrás resistir ante tus enemigos hasta que quites el objeto consagrado
de en medio de ti''. \footnote{\textbf{7:13} Jos 3,5} \bibleverse{14}
Por lo tanto, por la mañana serás acercado por tus tribus. La tribu que
Yahvé seleccione se acercará por familias. La familia que Yahvé
seleccione se acercará por hogares. El hogar que el Señor seleccione se
acercará por medio de hombres. \bibleverse{15} Será que el que sea
tomado con lo consagrado será quemado con fuego, él y todo lo que tiene,
porque ha transgredido el pacto de Yahvé y porque ha hecho una cosa
vergonzosa en Israel.'\,''

\hypertarget{acuxe1n-es-identificado-como-un-malhechor-por-sorteo-y-es-apedreado-hasta-morir-despuuxe9s-de-admitir-su-culpabilidad}{%
\subsection{Acán es identificado como un malhechor por sorteo y es
apedreado hasta morir después de admitir su
culpabilidad}\label{acuxe1n-es-identificado-como-un-malhechor-por-sorteo-y-es-apedreado-hasta-morir-despuuxe9s-de-admitir-su-culpabilidad}}

\bibleverse{16} Entonces Josué se levantó de madrugada y acercó a Israel
por sus tribus. Seleccionó la tribu de Judá. \footnote{\textbf{7:16}
  1Sam 10,20-21; 1Sam 14,41-42} \bibleverse{17} Acercó a la familia de
Judá y seleccionó a la familia de los zeraítas. Acercó a la familia de
los zeraítas hombre por hombre, y seleccionó a Zabdi. \footnote{\textbf{7:17}
  Núm 26,20} \bibleverse{18} Acercó su familia hombre por hombre, y fue
seleccionado Acán, hijo de Carmi, hijo de Zabdi, hijo de Zera, de la
tribu de Judá. \bibleverse{19} Josué le dijo a Acán: ``Hijo mío, por
favor, da gloria a Yavé, el Dios de Israel, y hazle una confesión. ¡Dime
ahora lo que has hecho! No me lo ocultes''.

\bibleverse{20} Acán respondió a Josué y dijo: ``Verdaderamente he
pecado contra Yavé, el Dios de Israel, y esto es lo que he hecho.
\bibleverse{21} Cuando vi entre el botín un hermoso manto babilónico,
doscientos siclos\footnote{\textbf{7:21} Un siclo equivale a unos 10
  gramos o a unas 0,35 onzas.} de plata, y una cuña de oro que pesaba
cincuenta siclos, entonces los codicié y los tomé. He aquí que están
escondidos en el suelo, en medio de mi tienda, con la plata debajo''.

\bibleverse{22} Entonces Josué envió mensajeros, y éstos corrieron a la
tienda. He aquí que estaba escondida en su tienda, con la plata debajo
de ella. \bibleverse{23} La sacaron de en medio de la tienda y la
llevaron a Josué y a todos los hijos de Israel. Los depositaron ante el
Señor. \bibleverse{24} Josué, y todo Israel con él, tomaron a Acán hijo
de Zera, la plata, el manto, la cuña de oro, sus hijos, sus hijas, su
ganado, sus asnos, sus ovejas, su tienda y todo lo que tenía; y los
llevaron al valle de Acor. \bibleverse{25} Josué les dijo: ``¿Por qué
nos han molestado? El Señor los molestará hoy''. Todo Israel lo apedreó,
y los quemaron con fuego y los apedrearon. \bibleverse{26} Levantaron
sobre él un gran montón de piedras que permanece hasta hoy. El Señor se
apartó del ardor de su ira. Por eso el nombre de aquel lugar se llamó
hasta hoy ``Valle de Acor''. \footnote{\textbf{7:26} Is 65,10; Os 2,15}

\hypertarget{por-instrucciuxf3n-divina-josuuxe9-se-mueve-contra-hai-y-prepara-una-emboscada-en-el-oeste-de-la-ciudad}{%
\subsection{Por instrucción divina, Josué se mueve contra Hai y prepara
una emboscada en el oeste de la
ciudad}\label{por-instrucciuxf3n-divina-josuuxe9-se-mueve-contra-hai-y-prepara-una-emboscada-en-el-oeste-de-la-ciudad}}

\hypertarget{section-7}{%
\section{8}\label{section-7}}

\bibleverse{1} El Señor le dijo a Josué: ``No temas ni te desanimes.
Toma a todos los guerreros contigo, y levántate y sube a Hai. He aquí
que he entregado en tu mano al rey de Hai, con su pueblo, su ciudad y su
tierra. \bibleverse{2} Haréis con Hai y con su rey lo mismo que
hicisteis con Jericó y con su rey, salvo que tomaréis para vosotros sus
bienes y su ganado. Poned una emboscada a la ciudad detrás de ella''.
\footnote{\textbf{8:2} Jos 6,21}

\bibleverse{3} Entonces Josué se levantó, con todos los guerreros, para
subir a Hai. Josué escogió treinta mil hombres, los más valientes, y los
envió de noche. \bibleverse{4} Les ordenó, diciendo: ``Mirad, os
pondréis en emboscada contra la ciudad, detrás de la ciudad. No os
alejéis mucho de la ciudad, pero estad todos preparados. \bibleverse{5}
Yo y todo el pueblo que está conmigo nos acercaremos a la ciudad.
Sucederá que cuando salgan contra nosotros, como al principio, huiremos
ante ellos. \bibleverse{6} Saldrán tras nosotros hasta que los hayamos
alejado de la ciudad; porque dirán: ``Huyen ante nosotros, como la
primera vez''. Así que huiremos delante de ellos, \footnote{\textbf{8:6}
  Jos 7,5} \bibleverse{7} y tú te levantarás de la emboscada y tomarás
posesión de la ciudad, porque el Señor, tu Dios, la entregará en tu
mano. \bibleverse{8} Cuando hayáis tomado la ciudad, le prenderéis
fuego. Harás esto según la palabra de Yahvé. He aquí que yo te lo he
ordenado''.

\bibleverse{9} Josué los envió, y ellos fueron a preparar la emboscada,
y se quedaron entre Betel y Hai, al oeste de Hai; pero Josué se quedó en
medio del pueblo esa noche. \bibleverse{10} Josué se levantó de
madrugada, reunió al pueblo y subió, él y los ancianos de Israel,
delante del pueblo a Hai. \bibleverse{11} Todo el pueblo, incluso los
hombres de guerra que estaban con él, subieron y se acercaron, y
llegaron ante la ciudad y acamparon en el lado norte de Hai. Había un
valle entre él y Hai. \bibleverse{12} Tomó unos cinco mil hombres y los
puso en una emboscada entre Betel y Hai, del lado occidental de la
ciudad. \bibleverse{13} Así que puso a la gente, a todo el ejército que
estaba al norte de la ciudad, y su emboscada al oeste de la ciudad; y
Josué fue aquella noche al centro del valle.

\hypertarget{curso-de-la-lucha-quemando-la-ciudad-sin-vigilancia}{%
\subsection{Curso de la lucha; Quemando la ciudad sin
vigilancia}\label{curso-de-la-lucha-quemando-la-ciudad-sin-vigilancia}}

\bibleverse{14} Cuando el rey de Hai lo vio, se apresuró y se levantó
temprano, y los hombres de la ciudad salieron contra Israel para
combatir, él y todo su pueblo, a la hora señalada, frente al Arabá; pero
él no sabía que había una emboscada contra él detrás de la ciudad.
\bibleverse{15} Josué y todo Israel hicieron como si fueran vencidos
ante ellos, y huyeron por el camino del desierto. \bibleverse{16} Todo
el pueblo que estaba en la ciudad fue convocado para perseguirlos.
Persiguieron a Josué, y fueron alejados de la ciudad. \bibleverse{17} No
quedó un solo hombre en Hai o en Betel que no saliera en pos de Israel.
Dejaron la ciudad abierta y persiguieron a Israel.

\bibleverse{18} Yahvé dijo a Josué: ``Extiende la jabalina que tienes en
la mano hacia Hai, porque la entregaré en tu mano''. Josué extendió la
jabalina que tenía en la mano hacia la ciudad. \bibleverse{19} Los
emboscados se levantaron rápidamente de su lugar, y corrieron tan pronto
como él extendió su mano y entraron en la ciudad y la tomaron. Se
apresuraron y prendieron fuego a la ciudad. \bibleverse{20} Cuando los
hombres de Hai miraron a sus espaldas, vieron que el humo de la ciudad
subía hasta el cielo, y no tuvieron fuerzas para huir por un lado o por
otro. El pueblo que huyó al desierto se volvió contra los perseguidores.
\bibleverse{21} Cuando Josué y todo Israel vieron que la emboscada había
tomado la ciudad y que el humo de la ciudad ascendía, se volvieron y
mataron a los hombres de Hai. \bibleverse{22} Los demás salieron de la
ciudad contra ellos, de modo que se pusieron en medio de Israel, unos de
un lado y otros de otro. Los atacaron, de modo que no dejaron que
ninguno de ellos permaneciera ni escapara. \bibleverse{23} Capturaron
vivo al rey de Hai y lo llevaron a Josué.

\hypertarget{ejecuciuxf3n-de-la-prohibiciuxf3n-de-la-ciudad-el-rey-matuxf3-y-ahorcuxf3-hasta-la-noche}{%
\subsection{Ejecución de la prohibición de la ciudad; el rey mató y
ahorcó hasta la
noche}\label{ejecuciuxf3n-de-la-prohibiciuxf3n-de-la-ciudad-el-rey-matuxf3-y-ahorcuxf3-hasta-la-noche}}

\bibleverse{24} Cuando Israel terminó de matar a todos los habitantes de
Hai en el campo, en el desierto donde los perseguían, y todos cayeron a
filo de espada hasta ser consumidos, todo Israel volvió a Hai y la hirió
a filo de espada. \bibleverse{25} Todos los que cayeron aquel día, tanto
hombres como mujeres, fueron doce mil, todo el pueblo de Hai.
\bibleverse{26} Porque Josué no retiró su mano, con la que extendía la
jabalina, hasta que hubo destruido por completo a todos los habitantes
de Hai. \footnote{\textbf{8:26} Éxod 17,11-13} \bibleverse{27} Israel
sólo tomó para sí el ganado y los bienes de esa ciudad, según la palabra
de Yavé que le había ordenado a Josué. \bibleverse{28} Entonces Josué
quemó a Hai y la convirtió en un montón para siempre, en una desolación,
hasta el día de hoy. \bibleverse{29} Colgó al rey de Hai en un árbol
hasta el atardecer. Al anochecer, Josué lo ordenó, y bajaron su cuerpo
del árbol y lo arrojaron a la entrada de la puerta de la ciudad, y
levantaron sobre él un gran montón de piedras que permanece hasta el día
de hoy. \footnote{\textbf{8:29} Jos 10,27; Deut 21,23}

\hypertarget{construcciuxf3n-de-un-altar-en-el-monte-ebal-y-lectura-de-la-ley-por-josuuxe9-despuuxe9s-de-la-fiesta-del-sacrificio}{%
\subsection{Construcción de un altar en el monte Ebal y lectura de la
ley por Josué después de la fiesta del
sacrificio}\label{construcciuxf3n-de-un-altar-en-el-monte-ebal-y-lectura-de-la-ley-por-josuuxe9-despuuxe9s-de-la-fiesta-del-sacrificio}}

\bibleverse{30} Entonces Josué edificó un altar a Yavé, el Dios de
Israel, en el monte Ebal, \footnote{\textbf{8:30} Deut 27,2-8}
\bibleverse{31} tal como Moisés, siervo de Yavé, lo había ordenado a los
hijos de Israel, como está escrito en el libro de la ley de Moisés: un
altar de piedras sin cortar, en el que nadie había levantado hierro.
Sobre él ofrecían holocaustos a Yahvé y sacrificaban ofrendas de paz.
\bibleverse{32} Allí escribió en las piedras una copia de la ley de
Moisés, que escribió en presencia de los hijos de Israel.
\bibleverse{33} Todo Israel, con sus ancianos, oficiales y jueces, se
puso de pie a ambos lados del arca, delante de los sacerdotes levitas
que llevaban el arca de la alianza de Yavé, tanto los extranjeros como
los nativos; la mitad de ellos frente al monte Gerizim, y la otra mitad
frente al monte Ebal, tal como Moisés, siervo de Yavé, lo había ordenado
al principio, para que bendijesen al pueblo de Israel. \footnote{\textbf{8:33}
  Deut 11,29; Deut 27,12-13} \bibleverse{34} Después leyó todas las
palabras de la ley, la bendición y la maldición, según todo lo que está
escrito en el libro de la ley. \bibleverse{35} No hubo palabra de todo
lo que Moisés mandó que Josué no leyera ante toda la asamblea de Israel,
con las mujeres, los niños y los extranjeros que estaban entre ellos.

\hypertarget{los-reyes-cananeos-hacen-un-pacto-contra-israel}{%
\subsection{Los reyes cananeos hacen un pacto contra
Israel}\label{los-reyes-cananeos-hacen-un-pacto-contra-israel}}

\hypertarget{section-8}{%
\section{9}\label{section-8}}

\bibleverse{1} Cuando todos los reyes que estaban al otro lado del
Jordán, en la región montañosa y en la llanura, y en toda la orilla del
gran mar frente al Líbano, el hitita, el amorreo, el cananeo, el
ferezeo, el heveo y el jebuseo, se enteraron de ello \bibleverse{2} se
reunieron para luchar con Josué y con Israel, de común acuerdo.

\hypertarget{los-gabaonitas-envuxedan-una-delegaciuxf3n-y-mediante-engauxf1os-consiguen-un-acuerdo-pacuxedfico-con-los-israelitas}{%
\subsection{Los gabaonitas envían una delegación y, mediante engaños,
consiguen un acuerdo pacífico con los
israelitas}\label{los-gabaonitas-envuxedan-una-delegaciuxf3n-y-mediante-engauxf1os-consiguen-un-acuerdo-pacuxedfico-con-los-israelitas}}

\bibleverse{3} Pero cuando los habitantes de Gabaón se enteraron de lo
que Josué había hecho a Jericó y a Hai, \footnote{\textbf{9:3} Jos
  6,20-21; Jos 8,26; Jos 8,28} \bibleverse{4} también recurrieron a un
ardid, y fueron y se hicieron pasar por embajadores, y tomaron sacos
viejos en sus asnos, y cueros de vino viejos, rotos y atados,
\bibleverse{5} y sandalias viejas y remendadas en sus pies, y llevaban
vestidos viejos. Todo el pan de sus provisiones estaba seco y mohoso.
\bibleverse{6} Se dirigieron a Josué en el campamento de Gilgal y le
dijeron a él y a los hombres de Israel: ``Hemos venido de un país
lejano. Ahora, pues, haz un pacto con nosotros''.

\bibleverse{7} Los hombres de Israel dijeron a los heveos: ``¿Y si vivís
entre nosotros? ¿Cómo podríamos hacer un pacto con ustedes?''
\footnote{\textbf{9:7} Jos 11,19; Éxod 23,32}

\bibleverse{8} Dijeron a Josué: ``Somos tus siervos''. Josué les dijo:
``¿Quiénes sois? ¿De dónde venís?''

\bibleverse{9} Le dijeron: ``Tus siervos han venido de un país muy
lejano por el nombre de Yavé, tu Dios; porque hemos oído hablar de su
fama, de todo lo que hizo en Egipto, \bibleverse{10} y de todo lo que
hizo a los dos reyes de los amorreos que estaban al otro lado del
Jordán, a Sehón, rey de Hesbón, y a Og, rey de Basán, que estaba en
Astarot. \footnote{\textbf{9:10} Núm 21,21-35} \bibleverse{11} Nuestros
ancianos y todos los habitantes de nuestro país nos hablaron diciendo:
``Tomen en sus manos provisiones para el viaje y vayan a recibirlos.
Díganles: ``Somos sus siervos. Hagan un pacto con nosotros''.
\bibleverse{12} Este pan nuestro lo sacamos caliente para nuestras
provisiones de nuestras casas el día que salimos para ir a ustedes; pero
ahora, he aquí, está seco y se ha enmohecido. \bibleverse{13} Estos
cueros de vino, que llenamos, eran nuevos; y he aquí que están rotos.
Estos nuestros vestidos y nuestras sandalias se han envejecido a causa
del larguísimo viaje''.

\bibleverse{14} Los hombres probaron sus provisiones y no pidieron
consejo a la boca de Yavé. \footnote{\textbf{9:14} Núm 27,21}
\bibleverse{15} Josué hizo las paces con ellos y pactó con ellos que los
dejaría vivir. Los príncipes de la congregación les prestaron juramento.
\footnote{\textbf{9:15} Jos 9,7}

\hypertarget{hizo-a-los-gabaonitas-siervos-de-la-iglesia-y-del-templo-despuuxe9s-de-que-se-descubriuxf3-su-engauxf1o}{%
\subsection{Hizo a los gabaonitas siervos de la iglesia y del templo
después de que se descubrió su
engaño}\label{hizo-a-los-gabaonitas-siervos-de-la-iglesia-y-del-templo-despuuxe9s-de-que-se-descubriuxf3-su-engauxf1o}}

\bibleverse{16} Al cabo de tres días después de haber hecho un pacto con
ellos, se enteraron de que eran sus vecinos y que vivían entre ellos.
\bibleverse{17} Los hijos de Israel viajaron y llegaron a sus ciudades
al tercer día. Sus ciudades eran Gabaón, Quefira, Beerot y Quiriat
Jearim. \bibleverse{18} Los hijos de Israel no los atacaron, porque los
príncipes de la congregación les habían jurado por Yahvé, el Dios de
Israel. Toda la congregación murmuró contra los príncipes.
\bibleverse{19} Pero todos los príncipes dijeron a toda la congregación:
``Les hemos jurado por Yahvé, el Dios de Israel. Ahora, pues, no podemos
tocarlos. \bibleverse{20} Haremos esto con ellos y los dejaremos vivir,
para que no caiga sobre nosotros la ira por el juramento que les
hicimos.'' \footnote{\textbf{9:20} 2Sam 21,1-2} \bibleverse{21} Los
príncipes les dijeron: ``Déjenlos vivir''. Así que se convirtieron en
cortadores de leña y sacadores de agua para toda la congregación, como
los príncipes les habían dicho.

\bibleverse{22} Josué los llamó y les habló diciendo: ``¿Por qué nos
habéis engañado, diciendo: `Estamos muy lejos de vosotros', cuando vivís
entre nosotros? \bibleverse{23} Ahora, pues, estáis malditos, y algunos
de vosotros no dejarán de ser esclavos, ni cortadores de leña ni
sacadores de agua para la casa de mi Dios.''

\bibleverse{24} Ellos respondieron a Josué y dijeron: ``Porque
ciertamente a tus siervos se les contó cómo Yahvé, tu Dios, le ordenó a
su siervo Moisés que te diera toda la tierra, y que destruyera a todos
los habitantes de la tierra de delante de ti. Por eso temimos mucho por
nuestras vidas a causa de ustedes, y hemos hecho esto. \bibleverse{25}
Ahora, he aquí que estamos en tu mano. Haz con nosotros lo que te
parezca bueno y correcto hacer''.

\bibleverse{26} Así lo hizo con ellos, y los libró de la mano de los
hijos de Israel, para que no los mataran. \bibleverse{27} Aquel día
Josué les hizo cortadores de madera y sacadores de agua para la
congregación y para el altar de Yavé hasta el día de hoy, en el lugar
que él eligiera. \footnote{\textbf{9:27} Deut 29,11}

\hypertarget{la-procesiuxf3n-de-los-cinco-reyes-contra-gabauxf3n-victoria-de-josuuxe9-en-gabauxf3n}{%
\subsection{La procesión de los cinco reyes contra Gabaón; Victoria de
Josué en
Gabaón}\label{la-procesiuxf3n-de-los-cinco-reyes-contra-gabauxf3n-victoria-de-josuuxe9-en-gabauxf3n}}

\hypertarget{section-9}{%
\section{10}\label{section-9}}

\bibleverse{1} Cuando Adoni-Zedec, rey de Jerusalén, oyó que Josué había
tomado a Hai y la había destruido por completo, como había hecho con
Jericó y su rey, así había hecho con Hai y su rey, y que los habitantes
de Gabaón habían hecho la paz con Israel y estaban en medio de ellos,
\footnote{\textbf{10:1} Jos 8,1; Jos 9,1-9} \bibleverse{2} tuvieron
mucho miedo, porque Gabaón era una gran ciudad, como una de las ciudades
reales, y porque era más grande que Hai, y todos sus hombres eran
poderosos. \bibleverse{3} Por lo tanto, Adoni-Zedec, rey de Jerusalén,
envió a Hoham, rey de Hebrón, a Piram, rey de Jarmut, a Jafía, rey de
Laquis, y a Debir, rey de Eglón, diciendo: \bibleverse{4} ``Subid a mí y
ayudadme. Ataquemos a Gabaón; porque ellos han hecho la paz con Josué y
con los hijos de Israel''. \bibleverse{5} Entonces los cinco reyes de
los amorreos, el rey de Jerusalén, el rey de Hebrón, el rey de Jarmut,
el rey de Laquis y el rey de Eglón, se reunieron y subieron, ellos y
todos sus ejércitos, y acamparon contra Gabaón y le hicieron la guerra.
\bibleverse{6} Los hombres de Gabaón enviaron a Josué al campamento de
Gilgal, diciendo: ``¡No abandones a tus siervos! ¡Sube a nosotros
rápidamente y sálvanos! Ayúdanos, porque todos los reyes de los amorreos
que habitan en la región montañosa se han reunido contra nosotros''.

\bibleverse{7} Entonces Josué subió de Gilgal, él y todo el ejército que
lo acompañaba, incluidos todos los hombres valientes. \bibleverse{8} El
Señor le dijo a Josué: ``No los temas, porque los he entregado en tus
manos. Ni un solo hombre de ellos se pondrá en pie ante ti''.

\bibleverse{9} Josué, por lo tanto, llegó a ellos repentinamente. Marchó
desde Gilgal toda la noche. \bibleverse{10} El Señor los confundió ante
Israel. Los mató con una gran matanza en Gabaón, y los persiguió por el
camino de la subida de Bet Horón, y los golpeó hasta Azeca y hasta
Maceda.

\hypertarget{los-dos-grandes-milagros-de-dios-granizo-de-piedras-y-parada}{%
\subsection{Los dos grandes milagros de Dios: granizo de piedras y
parada}\label{los-dos-grandes-milagros-de-dios-granizo-de-piedras-y-parada}}

\bibleverse{11} Cuando huían de delante de Israel, mientras estaban en
la bajada de Bet Horón, el Señor arrojó sobre ellos grandes piedras del
cielo hasta Azeca, y murieron. Fueron más los que murieron a causa del
granizo que los que los hijos de Israel mataron a espada. \footnote{\textbf{10:11}
  Éxod 9,22-25}

\bibleverse{12} Entonces Josué habló a Yahvé el día en que Yahvé entregó
a los amorreos ante los hijos de Israel. Dijo a la vista de Israel:
``¡Sol, detente en Gabaón! Tú, luna, detente en el valle de Ajalón''.

\bibleverse{13} El sol se detuvo y la luna permaneció, hasta que la
nación se vengó de sus enemigos. ¿No está esto escrito en el libro de
Jashar? El sol permaneció en medio del cielo, y no se apresuró a bajar
durante todo un día. \footnote{\textbf{10:13} Hab 3,11; 2Sam 1,18}
\bibleverse{14} No hubo un día como ése, ni antes ni después, en que
Yavé escuchara la voz de un hombre; porque Yavé luchó por Israel.
\footnote{\textbf{10:14} Jos 10,42; Éxod 14,25}

\bibleverse{15} Josué regresó, y todo Israel con él, al campamento de
Gilgal.

\hypertarget{persecucion-los-cinco-reyes-amorreos-atrapados-en-una-cueva-son-asesinados-y-ahorcados}{%
\subsection{Persecucion; los cinco reyes amorreos atrapados en una cueva
son asesinados y
ahorcados}\label{persecucion-los-cinco-reyes-amorreos-atrapados-en-una-cueva-son-asesinados-y-ahorcados}}

\bibleverse{16} Estos cinco reyes huyeron y se escondieron en la cueva
de Macedá. \bibleverse{17} Le avisaron a Josué, diciendo: ``Los cinco
reyes han sido encontrados, escondidos en la cueva de Macedá.''

\bibleverse{18} Josué dijo: ``Hagan rodar grandes piedras para cubrir la
entrada de la cueva, y pongan hombres a su lado para vigilarla;
\bibleverse{19} pero no se queden allí. Persigue a tus enemigos y
atácalos por la retaguardia. No les permitas entrar en sus ciudades,
porque el Señor, tu Dios, los ha entregado en tu mano.''

\bibleverse{20} Cuando Josué y los hijos de Israel terminaron de
matarlos con una matanza muy grande hasta consumirlos, y el remanente
que quedó de ellos entró en las ciudades fortificadas, \bibleverse{21}
todo el pueblo regresó al campamento de Josué en Macedá en paz. Ninguno
movió su lengua contra ninguno de los hijos de Israel. \bibleverse{22}
Entonces Josué dijo: ``Abran la entrada de la cueva y tráiganme a esos
cinco reyes de la cueva''.

\bibleverse{23} Así lo hicieron, y sacaron a esos cinco reyes de la
cueva hacia él: el rey de Jerusalén, el rey de Hebrón, el rey de Jarmut,
el rey de Laquis y el rey de Eglón. \bibleverse{24} Cuando sacaron a
esos reyes ante Josué, éste llamó a todos los hombres de Israel y dijo a
los jefes de los hombres de guerra que iban con él: ``Acérquense. Pongan
sus pies sobre los cuellos de estos reyes''. Se acercaron y les pusieron
los pies en el cuello.

\bibleverse{25} Josué les dijo: ``No tengan miedo, ni se acobarden. Sean
fuertes y valientes, porque Yahvé hará esto con todos sus enemigos
contra los que luchen''.

\bibleverse{26} Después Josué los golpeó, los mató y los colgó en cinco
árboles. Estuvieron colgados en los árboles hasta el atardecer.
\bibleverse{27} A la hora de la puesta del sol, Josué ordenó que los
bajaran de los árboles y los arrojaran a la cueva en la que se habían
escondido, y colocaron grandes piedras en la boca de la cueva, las
cuales permanecen hasta el día de hoy. \footnote{\textbf{10:27} Jos
  8,29; Deut 21,23}

\hypertarget{subyugaciuxf3n-de-toda-la-mitad-sur-de-canauxe1n-el-regreso-de-joshua-a-gilgal}{%
\subsection{Subyugación de toda la mitad sur de Canaán; El regreso de
Joshua a
Gilgal}\label{subyugaciuxf3n-de-toda-la-mitad-sur-de-canauxe1n-el-regreso-de-joshua-a-gilgal}}

\bibleverse{28} Ese día Josué tomó a Macedá y la hirió a filo de espada,
junto con su rey. La destruyó por completo y a todas las almas que
estaban en ella. No dejó a nadie en pie. Hizo con el rey de Maceda lo
mismo que había hecho con el rey de Jericó. \footnote{\textbf{10:28} Jos
  6,21}

\bibleverse{29} Josué pasó de Maceda, y todo Israel con él, a Libna, y
combatió contra Libna. \bibleverse{30} El Señor la entregó, junto con su
rey, en manos de Israel. La hirió con el filo de la espada, y a todas
las almas que estaban en ella. No dejó a nadie en ella. Hizo con su rey
lo mismo que había hecho con el rey de Jericó.

\bibleverse{31} Josué pasó de Libna, y todo Israel con él, a Laquis, y
acampó contra ella y la combatió. \bibleverse{32} El Señor entregó
Laquis en manos de Israel. La tomó al segundo día y la hirió a filo de
espada, con todas las almas que había en ella, conforme a todo lo que
había hecho con Libna. \bibleverse{33} Entonces Horam, rey de Gezer,
subió a ayudar a Laquis; y Josué lo hirió a él y a su pueblo, hasta no
dejarle ninguno.

\bibleverse{34} Josué pasó de Laquis, y todo Israel con él, a Eglón;
acamparon contra ella y la combatieron. \bibleverse{35} Aquel día la
tomaron y la hirieron a filo de espada. Destruyó por completo a todos
los que estaban en ella aquel día, conforme a todo lo que había hecho
con Laquis.

\bibleverse{36} Josué subió de Eglón, y todo Israel con él, a Hebrón, y
la combatieron. \bibleverse{37} La tomaron y la hirieron a filo de
espada, con su rey y todas sus ciudades, y todas las personas que
estaban en ella. No dejó a nadie, conforme a todo lo que había hecho a
Eglón, sino que la destruyó por completo, con todas las almas que había
en ella.

\bibleverse{38} Josué volvió, y todo Israel con él, a Debir, y luchó
contra ella. \bibleverse{39} La tomó, con su rey y todas sus ciudades.
Los hirieron a filo de espada, y destruyeron por completo a todos los
que estaban en ella. No dejó a nadie en pie. Como había hecho con
Hebrón, así hizo con Debir y con su rey; como también hizo con Libna y
con su rey. \bibleverse{40} Así, Josué atacó toda la tierra, la región
de las colinas, el sur, las tierras bajas, las laderas y a todos sus
reyes. No dejó a nadie en pie, sino que destruyó por completo todo lo
que respiraba, como lo ordenó el Señor, el Dios de Israel. \footnote{\textbf{10:40}
  Núm 21,2; Deut 20,16-18} \bibleverse{41} Josué los hirió desde Cades
Barnea hasta Gaza, y todo el país de Gosén, hasta Gabaón. \footnote{\textbf{10:41}
  Jos 11,16} \bibleverse{42} Josué tomó a todos estos reyes y su tierra
de una sola vez, porque Yahvé, el Dios de Israel, luchó por Israel.
\footnote{\textbf{10:42} Jos 10,14} \bibleverse{43} Josué regresó, y
todo Israel con él, al campamento de Gilgal. \footnote{\textbf{10:43}
  Jos 10,15}

\hypertarget{los-reyes-unidos-por-jabuxedn-son-destruidos-por-josuuxe9}{%
\subsection{Los reyes unidos por Jabín son destruidos por
Josué}\label{los-reyes-unidos-por-jabuxedn-son-destruidos-por-josuuxe9}}

\hypertarget{section-10}{%
\section{11}\label{section-10}}

\bibleverse{1} Cuando Jabín, rey de Hazor, se enteró de ello, envió a
Jobab, rey de Madón, al rey de Simrón, al rey de Ajsaf, \bibleverse{2} y
a los reyes que estaban al norte, en la región montañosa, en el Arabá al
sur de Cinnerot en la llanura, y en las alturas de Dor al oeste,
\bibleverse{3} al cananeo al este y al oeste, al amorreo, al hitita, al
ferezeo, al jebuseo en la región montañosa, y al heveo bajo Hermón en la
tierra de Mizpa. \bibleverse{4} Salieron, ellos y todos sus ejércitos
con ellos, mucha gente, como la arena que está a la orilla del mar en
multitud, con muchísimos caballos y carros. \bibleverse{5} Todos estos
reyes se reunieron, y vinieron y acamparon juntos junto a las aguas de
Merom, para luchar contra Israel.

\bibleverse{6} El Señor le dijo a Josué: ``No temas por ellos, porque
mañana a esta hora los entregaré todos muertos ante Israel. Atormentarás
sus caballos y quemarás sus carros con fuego''.

\bibleverse{7} Entonces Josué llegó de repente, con todos los guerreros,
contra ellos junto a las aguas de Merom, y los atacó. \bibleverse{8} El
Señor los entregó en manos de Israel, y ellos los hirieron y los
persiguieron hasta la gran Sidón, hasta Misrefot Maim y hasta el valle
de Mizpa, al este. Los hirieron hasta no dejarles ninguno. \footnote{\textbf{11:8}
  Jos 13,6} \bibleverse{9} Josué hizo con ellos lo que Yahvé le dijo.
Les ató los caballos y quemó sus carros con fuego.

\hypertarget{subyugaciuxf3n-de-toda-la-mitad-norte-de-canauxe1n}{%
\subsection{Subyugación de toda la mitad norte de
Canaán}\label{subyugaciuxf3n-de-toda-la-mitad-norte-de-canauxe1n}}

\bibleverse{10} Josué dio la vuelta en ese momento, tomó Hazor e hirió a
su rey con la espada, pues Hazor era la cabeza de todos esos reinos.
\bibleverse{11} Hirieron con el filo de la espada a todos los que
estaban en ella, destruyéndolos por completo. No quedó nadie que
respirara. Quemó Hazor con fuego. \footnote{\textbf{11:11} Núm 21,2}
\bibleverse{12} Josué capturó todas las ciudades de esos reyes, con sus
reyes, y las hirió a filo de espada, destruyéndolas por completo, como
lo había ordenado Moisés, siervo de Yavé. \bibleverse{13} Pero en cuanto
a las ciudades que estaban sobre sus montículos, Israel no quemó ninguna
de ellas, excepto Hazor solamente. Josué la quemó. \bibleverse{14} Los
hijos de Israel tomaron todo el botín de estas ciudades, con el ganado,
como botín para ellos; pero a todo hombre lo golpearon con el filo de la
espada, hasta destruirlo. No dejaron ninguno que respirara.

\bibleverse{15} Como el Señor le ordenó a Moisés, su siervo, así le
ordenó Moisés a Josué. Josué lo hizo. No dejó nada sin hacer de todo lo
que el Señor le ordenó a Moisés.

\hypertarget{revisiuxf3n-cumplimiento-de-la-voluntad-divina-de-destruir-endureciendo-a-los-cananeos}{%
\subsection{Revisión; Cumplimiento de la voluntad divina de destruir
endureciendo a los
cananeos}\label{revisiuxf3n-cumplimiento-de-la-voluntad-divina-de-destruir-endureciendo-a-los-cananeos}}

\bibleverse{16} Josué capturó toda aquella tierra, la región montañosa,
todo el sur, toda la tierra de Gosén, la llanura, el Arabá, la región
montañosa de Israel y la llanura de la misma, \footnote{\textbf{11:16}
  Jos 10,41} \bibleverse{17} desde el monte Halak, que sube hasta Seir,
hasta Baal Gad en el valle del Líbano, bajo el monte Hermón. Tomó a
todos sus reyes, los hirió y los mató. \bibleverse{18} Josué hizo la
guerra durante mucho tiempo a todos esos reyes. \bibleverse{19} No hubo
ninguna ciudad que hiciera la paz con los hijos de Israel, excepto los
heveos, habitantes de Gabaón. A todos los tomaron en batalla.
\footnote{\textbf{11:19} Jos 9,15} \bibleverse{20} Porque fue de Yahvé
endurecer sus corazones, para venir contra Israel en la batalla, a fin
de destruirlos totalmente, para que no tuvieran ningún favor, sino que
los destruyera, como Yahvé lo mandó a Moisés. \footnote{\textbf{11:20}
  Deut 7,2}

\hypertarget{exterminio-de-los-enakitas-es-decir-gigantes-de-la-tierra}{%
\subsection{Exterminio de los Enakitas (es decir, gigantes) de la
tierra}\label{exterminio-de-los-enakitas-es-decir-gigantes-de-la-tierra}}

\bibleverse{21} En aquel tiempo vino Josué y eliminó a los anakim de la
región montañosa, de Hebrón, de Debir, de Anab y de toda la región
montañosa de Judá y de toda la región montañosa de Israel. Josué los
destruyó por completo con sus ciudades. \footnote{\textbf{11:21} Núm
  13,22; Deut 1,28} \bibleverse{22} No quedó ninguno de los anakim en la
tierra de los hijos de Israel. Sólo en Gaza, en Gat y en Asdod quedaron
algunos. \footnote{\textbf{11:22} 1Sam 17,4} \bibleverse{23} Así que
Josué tomó toda la tierra, de acuerdo con todo lo que Yahvé habló a
Moisés; y Josué la dio en herencia a Israel según sus divisiones por sus
tribus. Entonces la tierra descansó de la guerra. \footnote{\textbf{11:23}
  Jos 14,15}

\hypertarget{los-dos-reyes-de-cisjordania-derrotados-por-moisuxe9s}{%
\subsection{Los dos reyes de Cisjordania derrotados por
Moisés}\label{los-dos-reyes-de-cisjordania-derrotados-por-moisuxe9s}}

\hypertarget{section-11}{%
\section{12}\label{section-11}}

\bibleverse{1} Estos son los reyes de la tierra, a quienes los hijos de
Israel hirieron, y poseyeron su tierra al otro lado del Jordán, hacia la
salida del sol, desde el valle de Arnón hasta el monte Hermón, y todo el
Arabá hacia el oriente: \bibleverse{2} Sehón, rey de los amorreos, que
vivía en Hesbón, y gobernaba desde Aroer, que está al borde del valle de
Arnón, y la mitad del valle, y la mitad de Galaad, hasta el río Jaboc,
el límite de los hijos de Amón; \footnote{\textbf{12:2} Núm 21,24}
\bibleverse{3} y el Arabá hasta el mar de Cinerot, al este, y hasta el
mar del Arabá, el Mar Salado, al este, el camino de Bet Jeshimot; y al
sur, bajo las laderas de Pisga \bibleverse{4} y el límite de Og, rey de
Basán, del remanente de los refaítas, que vivía en Astarot y en Edrei,
\footnote{\textbf{12:4} Núm 21,33; Deut 3,11} \bibleverse{5} y gobernaba
en el monte Hermón, y en Salecá, y en todo Basán, hasta el límite de los
guesuritas y de los maacatitas, y la mitad de Galaad, el límite de
Sehón, rey de Hesbón. \bibleverse{6} Moisés, siervo del Señor, y los
hijos de Israel los atacaron. Moisés, siervo de Yavé, la dio en posesión
a los rubenitas, a los gaditas y a la media tribu de Manasés.
\footnote{\textbf{12:6} Núm 32,33}

\hypertarget{los-31-reyes-derrotados-por-josuuxe9-en-cisjordania}{%
\subsection{Los 31 reyes derrotados por Josué en
Cisjordania}\label{los-31-reyes-derrotados-por-josuuxe9-en-cisjordania}}

\bibleverse{7} Estos son los reyes de la tierra que Josué y los hijos de
Israel hirieron al otro lado del Jordán, hacia el oeste, desde Baal Gad,
en el valle del Líbano, hasta el monte Halak, que sube a Seir. Josué la
dio en posesión a las tribus de Israel según sus divisiones;
\bibleverse{8} en la región montañosa, en la llanura, en el Arabá, en
las laderas, en el desierto y en el sur; el hitita, el amorreo, el
cananeo, el ferezeo, el heveo y el jebuseo: \footnote{\textbf{12:8} Jos
  11,3} \bibleverse{9} el rey de Jericó, uno; el rey de Hai, que está
junto a Betel, uno; \footnote{\textbf{12:9} Jos 6,2; Jos 8,29}
\bibleverse{10} el rey de Jerusalén, uno; el rey de Hebrón, uno;
\footnote{\textbf{12:10} Jos 10,1; Jos 10,3} \bibleverse{11} el rey de
Jarmuth, uno; el rey de Laquis, uno; \footnote{\textbf{12:11} Jos 10,3}
\bibleverse{12} el rey de Eglon, uno; el rey de Gezer, uno; \footnote{\textbf{12:12}
  Jos 10,3; Jos 10,26; Jos 10,33} \bibleverse{13} el rey de Debir, uno;
el rey de Geder, uno; \footnote{\textbf{12:13} Jos 10,39; Jue 1,11}
\bibleverse{14} el rey de Hormah, uno; el rey de Arad, uno; \footnote{\textbf{12:14}
  Jue 1,17; Núm 21,1} \bibleverse{15} el rey de Libna, uno; el rey de
Adulam, uno; \footnote{\textbf{12:15} Jos 10,29-30} \bibleverse{16} el
rey de Makkedah, uno; el rey de Betel, uno; \footnote{\textbf{12:16} Jos
  10,28} \bibleverse{17} el rey de Tappuah, uno; el rey de Hepher, uno;
\bibleverse{18} el rey de Afec, uno; el rey de Lassharon, uno;
\footnote{\textbf{12:18} Jos 15,53; 1Sam 4,1} \bibleverse{19} el rey de
Madón, uno; el rey de Hazor, uno; \footnote{\textbf{12:19} Jos 11,1; Jos
  11,10} \bibleverse{20} el rey de Shimron Meron, uno; el rey de
Achshaph, uno; \footnote{\textbf{12:20} Jos 11,1} \bibleverse{21} el rey
de Taanac, uno; el rey de Meguido, uno; \bibleverse{22} el rey de
Kedesh, uno; el rey de Jokneam en el Carmelo, uno; \bibleverse{23} el
rey de Dor en la altura de Dor, uno; el rey de Goiim en Gilgal, uno;
\footnote{\textbf{12:23} Jos 11,2} \bibleverse{24} el rey de Tirsa, uno:
todos los reyes treinta y uno.

\hypertarget{enumeraciuxf3n-de-las-uxe1reas-previamente-no-conquistadas-el-mandato-de-dios-de-distribuir}{%
\subsection{Enumeración de las áreas previamente no conquistadas; El
mandato de Dios de
distribuir}\label{enumeraciuxf3n-de-las-uxe1reas-previamente-no-conquistadas-el-mandato-de-dios-de-distribuir}}

\hypertarget{section-12}{%
\section{13}\label{section-12}}

\bibleverse{1} Josué era ya viejo y de edad avanzada. Yahvé le dijo:
``Eres viejo y avanzado en años, y aún queda mucha tierra por poseer.

\bibleverse{2} ``Esta es la tierra que aún queda todas las regiones de
los filisteos, y todos los guesuritas; \bibleverse{3} desde el Shihor,
que está delante de Egipto, hasta la frontera de Ecrón hacia el norte,
que se cuenta como cananea; los cinco señores de los filisteos: los
gazitas, los asdoditas, los ascalonitas, los gittitas y los ecronitas;
también los avvim, \bibleverse{4} al sur; toda la tierra de los
cananeos, y Meará que pertenece a los sidonios, hasta Afec, hasta la
frontera de los amorreos; \bibleverse{5} y la tierra de los gebalitas, y
todo el Líbano, hacia la salida del sol, desde Baal Gad bajo el monte
Hermón hasta la entrada de Hamat; \bibleverse{6} todos los habitantes de
la región montañosa desde el Líbano hasta Misrefot Maim, todos los
sidonios. Los expulsaré de delante de los hijos de Israel. Sólo asigna a
Israel como herencia, como te he ordenado. \footnote{\textbf{13:6} Jos
  11,8} \bibleverse{7} Ahora, pues, reparte esta tierra en herencia a
las nueve tribus y a la media tribu de Manasés.''

\hypertarget{informaciuxf3n-general-sobre-la-distribuciuxf3n-de-cisjordania-por-moses-adiciones-posteriores}{%
\subsection{Información general sobre la distribución de Cisjordania por
Moses; adiciones
posteriores}\label{informaciuxf3n-general-sobre-la-distribuciuxf3n-de-cisjordania-por-moses-adiciones-posteriores}}

\bibleverse{8} Con él, los rubenitas y los gaditas recibieron la
herencia que les dio Moisés, al otro lado del Jordán, hacia el este, tal
como se la dio Moisés, siervo de Yavé: \footnote{\textbf{13:8} Jos
  13,15-32} \bibleverse{9} desde Aroer, que está al borde del valle de
Arnón, y la ciudad que está en medio del valle, y toda la llanura de
Medeba hasta Dibón; \bibleverse{10} y todas las ciudades de Sehón, rey
de los amorreos, que reinaba en Hesbón, hasta el límite de los hijos de
Amón \bibleverse{11} y Galaad, y el límite de los guesuritas y
maacatitas, y todo el monte Hermón, y todo Basán hasta Salecá;
\bibleverse{12} todo el reino de Og en Basán, que reinaba en Astarot y
en Edrei (que quedó del resto de los refaítas); porque Moisés atacó a
éstos y los expulsó. \bibleverse{13} Sin embargo, los hijos de Israel no
expulsaron a los guesuritas ni a los maacatíes, sino que Guesur y Maacat
viven dentro de Israel hasta el día de hoy. \bibleverse{14} Sólo que no
dio herencia a la tribu de Leví. Las ofrendas a Yavé, el Dios de Israel,
hechas por fuego, son su herencia, como él le dijo. \footnote{\textbf{13:14}
  Jos 13,33}

\hypertarget{informaciuxf3n-muxe1s-detallada-sobre-las-uxe1reas-de-las-tribus-rubuxe9n-gad-y-mitad-manasuxe9s-distribuidas-por-moisuxe9s}{%
\subsection{Información más detallada sobre las áreas de las tribus
Rubén, Gad y mitad Manasés distribuidas por
Moisés}\label{informaciuxf3n-muxe1s-detallada-sobre-las-uxe1reas-de-las-tribus-rubuxe9n-gad-y-mitad-manasuxe9s-distribuidas-por-moisuxe9s}}

\bibleverse{15} Moisés dio a la tribu de los hijos de Rubén según sus
familias. \footnote{\textbf{13:15} Núm 32,1} \bibleverse{16} Su frontera
fue desde Aroer, que está a la orilla del valle de Arnón, y la ciudad
que está en medio del valle, y toda la llanura junto a Medeba;
\bibleverse{17} Hesbón, y todas sus ciudades que están en la llanura;
Dibón, Bamoth Baal, Beth Baal Meón, \bibleverse{18} Jahaz, Kedemoth,
Mephaath, \bibleverse{19} Kiriathaim, Sibmah, Zereth Shahar en el monte
del valle, \bibleverse{20} Beth Peor, las laderas de Pisga, Beth
Jeshimoth, \bibleverse{21} todas las ciudades de la llanura, y todo el
reino de Sehón, rey de los amorreos, que reinaba en Hesbón, a quien
Moisés hirió con los jefes de Madián, Evi, Rekem, Zur, Hur y Reba,
príncipes de Sehón, que vivían en la tierra. \bibleverse{22} Los hijos
de Israel también mataron a espada a Balaam, hijo de Beor, el adivino,
entre el resto de sus muertos. \footnote{\textbf{13:22} Núm 22,5; Núm
  31,8}

\bibleverse{23} El límite de los hijos de Rubén era la ribera del
Jordán. Esta fue la herencia de los hijos de Rubén según sus familias,
las ciudades y sus aldeas.

\bibleverse{24} Moisés dio a la tribu de Gad, a los hijos de Gad, según
sus familias. \bibleverse{25} Su límite fue Jazer, y todas las ciudades
de Galaad, y la mitad de la tierra de los hijos de Amón, hasta Aroer que
está cerca de Rabá; \bibleverse{26} y desde Hesbón hasta Ramat Mizpa, y
Betonim; y desde Mahanaim hasta el límite de Debir \bibleverse{27} y en
el valle, Bet Haram, Bet Nimra, Sucot y Zafón, el resto del reino de
Sehón, rey de Hesbón, la ribera del Jordán, hasta el extremo del mar de
Cineret, al otro lado del Jordán, hacia el este. \bibleverse{28} Esta es
la herencia de los hijos de Gad según sus familias, las ciudades y sus
aldeas.

\bibleverse{29} Moisés dio una herencia a la media tribu de Manasés. Fue
para la media tribu de los hijos de Manasés según sus familias.
\bibleverse{30} Su frontera era desde Mahanaim, todo Basán, todo el
reino de Og, rey de Basán, y todas las aldeas de Jair, que están en
Basán, sesenta ciudades. \footnote{\textbf{13:30} Jue 10,3-4}
\bibleverse{31} La mitad de Galaad, Astarot y Edrei, ciudades del reino
de Og en Basán, eran para los hijos de Maquir hijo de Manasés, para la
mitad de los hijos de Maquir según sus familias.

\bibleverse{32} Estas son las herencias que Moisés repartió en las
llanuras de Moab, al otro lado del Jordán, en Jericó, hacia el este.
\bibleverse{33} Pero Moisés no dio ninguna herencia a la tribu de Leví.
Yahvé, el Dios de Israel, es su herencia, como él les habló. \footnote{\textbf{13:33}
  Núm 18,20-21}

\hypertarget{comentarios-introductorios-la-herencia-de-caleb-en-hebruxf3n}{%
\subsection{Comentarios introductorios; la herencia de Caleb en
Hebrón}\label{comentarios-introductorios-la-herencia-de-caleb-en-hebruxf3n}}

\hypertarget{section-13}{%
\section{14}\label{section-13}}

\bibleverse{1} Estas son las herencias que los hijos de Israel tomaron
en la tierra de Canaán, y que el sacerdote Eleazar, Josué hijo de Nun, y
los jefes de las casas paternas de las tribus de los hijos de Israel,
les repartieron, \footnote{\textbf{14:1} Núm 34,17} \bibleverse{2} por
la suerte de su herencia, como Yahvé lo mandó por medio de Moisés, para
las nueve tribus y para la media tribu. \footnote{\textbf{14:2} Núm
  26,55} \bibleverse{3} Porque Moisés había dado la herencia de las dos
tribus y de la media tribu al otro lado del Jordán; pero a los levitas
no les dio herencia entre ellos. \footnote{\textbf{14:3} Jos 13,14-33}
\bibleverse{4} Porque los hijos de José eran dos tribus, Manasés y
Efraín. A los levitas no les dio ninguna porción en la tierra, excepto
ciudades para habitar, con sus tierras de pastoreo para su ganado y para
sus propiedades. \footnote{\textbf{14:4} Jos 21,1} \bibleverse{5} Los
hijos de Israel hicieron lo que Yahvé ordenó a Moisés, y repartieron la
tierra.

\hypertarget{a-peticiuxf3n-suya-kaleb-recibe-el-distrito-de-hebruxf3n-como-herencia}{%
\subsection{A petición suya, Kaleb recibe el distrito de Hebrón como
herencia}\label{a-peticiuxf3n-suya-kaleb-recibe-el-distrito-de-hebruxf3n-como-herencia}}

\bibleverse{6} Entonces los hijos de Judá se acercaron a Josué en
Gilgal. Caleb, hijo de Jefone cenecista, le dijo: ``Tú sabes lo que
Yahvé habló a Moisés, el hombre de Dios, acerca de mí y de ti en Cades
Barnea. \footnote{\textbf{14:6} Núm 14,24; Deut 1,36} \bibleverse{7} Yo
tenía cuarenta años cuando Moisés, el siervo de Yavé, me envió desde
Cades Barnea a espiar la tierra. Le llevé la noticia tal como estaba en
mi corazón. \footnote{\textbf{14:7} Núm 13,6; Núm 13,30} \bibleverse{8}
Sin embargo, mis hermanos que subieron conmigo hicieron que el corazón
del pueblo se derritiera; pero yo seguí totalmente a Yavé, mi Dios.
\bibleverse{9} Aquel día Moisés juró diciendo: `Ciertamente la tierra
por la que has caminado será una herencia para ti y para tus hijos para
siempre, porque has seguido enteramente a Yahvé mi Dios'.

\bibleverse{10} ``Ahora bien, he aquí que el Señor me ha mantenido con
vida, tal como habló, estos cuarenta y cinco años, desde el momento en
que el Señor habló esta palabra a Moisés, mientras Israel caminaba por
el desierto. Hoy tengo ochenta y cinco años. \bibleverse{11} Pero hoy
soy tan fuerte como el día en que Moisés me envió. Como era mi fuerza
entonces, así es ahora mi fuerza para la guerra, para salir y para
entrar. \footnote{\textbf{14:11} Deut 34,7} \bibleverse{12} Ahora, pues,
dame esta región montañosa, de la cual habló el Señor en aquel día;
porque tú oíste en aquel día cómo estaban allí los anakim, y ciudades
grandes y fortificadas. Puede ser que Yahvé esté conmigo y los expulse,
como dijo Yahvé''. \footnote{\textbf{14:12} Jos 11,21}

\bibleverse{13} Josué lo bendijo, y le dio Hebrón a Caleb, hijo de
Jefone, como herencia. \footnote{\textbf{14:13} Jos 15,13-19; Jos
  21,11-12} \bibleverse{14} Por lo tanto, Hebrón pasó a ser la herencia
de Caleb, hijo de Jefone, el cenecista, hasta el día de hoy, porque
siguió a Yavé, el Dios de Israel, de todo corazón. \bibleverse{15} Antes
el nombre de Hebrón era Quiriat Arba, en honor al hombre más grande de
los anakim. Entonces la tierra descansó de la guerra. \footnote{\textbf{14:15}
  Jos 11,23}

\hypertarget{el-territorio-de-la-tribu-de-juduxe1}{%
\subsection{El territorio de la tribu de
Judá}\label{el-territorio-de-la-tribu-de-juduxe1}}

\hypertarget{section-14}{%
\section{15}\label{section-14}}

\bibleverse{1} La suerte de la tribu de los hijos de Judá, según sus
familias, fue hasta el límite de Edom, hasta el desierto de Zin hacia el
sur, en el extremo del sur. \footnote{\textbf{15:1} Núm 34,3-5}
\bibleverse{2} Su límite sur era desde el extremo del Mar Salado, desde
la bahía que mira hacia el sur; \bibleverse{3} y salía hacia el sur de
la subida de Akrabbim, y pasaba por Zin, y subía por el sur de Cades
Barnea, y pasaba por Hezrón, subía por Addar, y se volvía hacia Karka;
\bibleverse{4} y pasaba por Azmón, salía por el arroyo de Egipto; y el
límite terminaba en el mar. Esta será su frontera sur. \bibleverse{5} El
límite oriental era el Mar Salado, hasta el final del Jordán. El límite
del norte era desde la bahía del mar hasta el final del Jordán.
\bibleverse{6} El límite subía hasta Bet Hogá, y pasaba por el norte de
Bet Araba; y el límite subía hasta la piedra de Bohán, hijo de Rubén.
\footnote{\textbf{15:6} Jos 18,17} \bibleverse{7} La frontera subía
hasta Debir desde el valle de Acor, y así hacia el norte, mirando hacia
Gilgal, que está frente a la subida de Adummim, que está al lado sur del
río. La frontera pasaba hasta las aguas de En Shemesh, y terminaba en En
Rogel. \footnote{\textbf{15:7} 2Sam 17,17} \bibleverse{8} La frontera
subía por el valle del hijo de Hinom hasta el lado del jebuseo (también
llamado Jerusalén) hacia el sur; y la frontera subía hasta la cima del
monte que está frente al valle de Hinom hacia el oeste, que está en la
parte más lejana del valle de Refaim hacia el norte. \footnote{\textbf{15:8}
  2Cró 28,3} \bibleverse{9} La frontera se extendía desde la cima del
monte hasta el manantial de las aguas de Neftoa, y salía a las ciudades
del monte Efrón; y la frontera se extendía hasta Baalá (también llamada
Quiriat Jearim); \footnote{\textbf{15:9} Jos 15,60} \bibleverse{10} y la
frontera giraba desde Baalá hacia el oeste, hacia el monte Seir, y
pasaba al lado del monte Jearim (también llamado Cesalón), al norte, y
bajaba a Bet Semes, y pasaba junto a Timná; \bibleverse{11} y la
frontera salía al lado de Ecrón hacia el norte; y la frontera se
extendía hasta Siquerón, y pasaba por el monte Baalá, y salía por
Jabneel; y las salidas de la frontera estaban en el mar. \bibleverse{12}
El límite occidental llegaba hasta la orilla del gran mar. Esta es la
frontera de los hijos de Judá según sus familias.

\hypertarget{posesiuxf3n-de-caleb-y-actividad-exitosa}{%
\subsection{Posesión de Caleb y actividad
exitosa}\label{posesiuxf3n-de-caleb-y-actividad-exitosa}}

\bibleverse{13} Le dio a Caleb, hijo de Jefone, una porción entre los
hijos de Judá, según el mandato de Yahvé a Josué, hasta Quiriat Arba,
llamada así por el padre de Anac (también llamada Hebrón). \footnote{\textbf{15:13}
  Jos 14,6-15; Jue 1,10-15} \bibleverse{14} Caleb expulsó a los tres
hijos de Anac Sesai, Ahiman y Talmai, hijos de Anac. \bibleverse{15}
Subió contra los habitantes de Debir, que antes se llamaba Kiriath
Sepher. \bibleverse{16} Caleb dijo: ``Al que ataque a Quiriat-Sfer y lo
tome, le daré a mi hija Acsa como esposa''. \bibleverse{17} La tomó
Othniel, hijo de Kenaz, hermano de Caleb, y le dio a Acsa, su hija, como
esposa. \bibleverse{18} Cuando ella llegó, le hizo pedir a su padre un
campo. Ella se bajó del asno, y Caleb le dijo: ``¿Qué quieres?''

\bibleverse{19} Ella dijo: ``Dame una bendición. Ya que me has puesto en
la tierra del Sur, dame también manantiales de agua''. Así que le dio
los muelles superiores y los inferiores.

\hypertarget{las-ciudades-de-juduxe1}{%
\subsection{Las ciudades de Judá}\label{las-ciudades-de-juduxe1}}

\bibleverse{20} Esta es la herencia de la tribu de los hijos de Judá
según sus familias. \bibleverse{21} Las ciudades más lejanas de la tribu
de los hijos de Judá hacia la frontera de Edom, en el sur, fueron
Kabzeel, Eder, Jagur, \bibleverse{22} Kinah, Dimonah, Adadah,
\bibleverse{23} Kedesh, Hazor, Ithnan, \bibleverse{24} Ziph, Telem,
Bealoth, \bibleverse{25} Hazor Hadattah, Kerioth Hezron (también llamada
Hazor), \bibleverse{26} Amam, Shema, Moladah, \bibleverse{27} Hazar
Gaddah, Heshmon, Beth Pelet, \bibleverse{28} Hazar Shual, Beersheba,
Biziothiah, \bibleverse{29} Baalah, Iim, Ezem, \bibleverse{30} Eltolad,
Chesil, Hormah, \bibleverse{31} Ziklag, Madmannah, Sansannah,
\bibleverse{32} Lebaoth, Shilhim, Ain y Rimmon. Todas las ciudades son
veintinueve, con sus aldeas. \bibleverse{33} En la tierra baja, Eshtaol,
Zorah, Ashnah, \footnote{\textbf{15:33} Jue 13,25; Jue 16,31}
\bibleverse{34} Zanoah, En Gannim, Tappuah, Enam, \bibleverse{35}
Jarmuth, Adullam, Socoh, Azekah, \bibleverse{36} Shaaraim, Adithaim y
Gederah (o Gederothaim); catorce ciudades con sus aldeas.
\bibleverse{37} Zenan, Hadashah, Migdal Gad, \bibleverse{38} Dilean,
Mizpa, Joktheel, \bibleverse{39} Lachish, Bozkath, Eglon,
\bibleverse{40} Cabbon, Lahmam, Chitlish, \bibleverse{41} Gederoth, Beth
Dagon, Naamah, y Makkedah; dieciséis ciudades con sus aldeas.
\bibleverse{42} Libná, Éter, Asán, \bibleverse{43} Ifá, Asná, Nezib,
\bibleverse{44} Keilá, Achzib y Maresá; nueve ciudades con sus aldeas.
\footnote{\textbf{15:44} Jos 19,29} \bibleverse{45} Ecrón, con sus
ciudades y sus aldeas; \footnote{\textbf{15:45} 1Sam 5,10}
\bibleverse{46} desde Ecrón hasta el mar, todos los que estaban junto a
Asdod, con sus aldeas. \bibleverse{47} Asdod, sus ciudades y sus aldeas;
Gaza, sus ciudades y sus aldeas; hasta el arroyo de Egipto, y el gran
mar con su costa. \footnote{\textbf{15:47} 1Sam 5,1; Jue 1,18; Núm 34,6}
\bibleverse{48} En la región de las colinas, Shamir, Jattir, Socoh,
\bibleverse{49} Dannah, Kiriath Sannah (que es Debir), \bibleverse{50}
Anab, Eshtemoh, Anim, \bibleverse{51} Goshen, Holon y Giloh; once
ciudades con sus aldeas. \bibleverse{52} Arab, Dumah, Eshan,
\bibleverse{53} Janim, Beth Tappuah, Aphekah, \bibleverse{54} Humtah,
Kiriath Arba (también llamada Hebrón) y Zior; nueve ciudades con sus
aldeas. \bibleverse{55} Maón, Carmelo, Zif, Jutah, \bibleverse{56}
Jezreel, Jocdeam, Zanoa, \bibleverse{57} Caín, Guibeá y Timná; diez
ciudades con sus aldeas. \bibleverse{58} Halhul, Beth Zur, Gedor,
\bibleverse{59} Maarath, Beth Anoth y Eltekon; seis ciudades con sus
aldeas. \bibleverse{60} Kiriath Baal (también llamada Kiriath Jearim), y
Rabbah; dos ciudades con sus aldeas. \footnote{\textbf{15:60} Jos 9,17;
  Jos 18,14} \bibleverse{61} En el desierto, Bet Araba, Middin, Secacah,
\bibleverse{62} Nibshan, la Ciudad de la Sal y En Gedi; seis ciudades
con sus aldeas.

\bibleverse{63} En cuanto a los jebuseos, habitantes de Jerusalén, los
hijos de Judá no pudieron expulsarlos; pero los jebuseos viven con los
hijos de Judá en Jerusalén hasta el día de hoy. \footnote{\textbf{15:63}
  Jos 18,18; 2Sam 5,6}

\hypertarget{el-territorio-de-los-descendientes-de-josuxe9}{%
\subsection{El territorio de los descendientes de
José}\label{el-territorio-de-los-descendientes-de-josuxe9}}

\hypertarget{section-15}{%
\section{16}\label{section-15}}

\bibleverse{1} La suerte salió para los hijos de José desde el Jordán en
Jericó, en las aguas de Jericó al oriente, hasta el desierto, subiendo
desde Jericó por la región montañosa hasta Betel. \bibleverse{2} Salía
de Betel a Luz, y pasaba por el límite de los arquitas hasta Atarot;
\bibleverse{3} y descendía hacia el oeste hasta el límite de los
jafletitas, hasta el límite de Bet Horón el inferior, y seguía hasta
Gezer; y terminaba en el mar.

\bibleverse{4} Los hijos de José, Manasés y Efraín, tomaron su herencia.

\hypertarget{territorio-de-la-tribu-de-efrauxedn}{%
\subsection{Territorio de la tribu de
Efraín}\label{territorio-de-la-tribu-de-efrauxedn}}

\bibleverse{5} Este fue el límite de los hijos de Efraín según sus
familias. El límite de su herencia hacia el este era Atarot Addar, hasta
Bet Horón el superior. \bibleverse{6} La frontera salía hacia el oeste
en Micmetat, al norte. La frontera giraba hacia el este hasta Taanat
Silo, y pasaba por ella al este de Janoa. \bibleverse{7} Bajaba de Janoa
a Atarot, a Naarah, llegaba a Jericó y salía al Jordán. \bibleverse{8}
Desde Tappua, el límite se extendía hacia el oeste hasta el arroyo de
Caná, y terminaba en el mar. Esta es la herencia de la tribu de los
hijos de Efraín según sus familias; \bibleverse{9} junto con las
ciudades que fueron apartadas para los hijos de Efraín en medio de la
herencia de los hijos de Manasés, todas las ciudades con sus aldeas.
\footnote{\textbf{16:9} Jos 17,9} \bibleverse{10} No expulsaron a los
cananeos que vivían en Gezer; pero los cananeos habitan en el territorio
de Efraín hasta el día de hoy, y se han convertido en siervos para
realizar trabajos forzados. \footnote{\textbf{16:10} 1Re 9,16}

\hypertarget{territorio-de-la-tribu-manasuxe9s}{%
\subsection{Territorio de la tribu
Manasés}\label{territorio-de-la-tribu-manasuxe9s}}

\hypertarget{section-16}{%
\section{17}\label{section-16}}

\bibleverse{1} Esta fue la suerte de la tribu de Manasés, pues era el
primogénito de José. En cuanto a Maquir, primogénito de Manasés, padre
de Galaad, por ser hombre de guerra, le tocó Galaad y Basán. \footnote{\textbf{17:1}
  Núm 26,29; Jos 13,31} \bibleverse{2} Así fue para el resto de los
hijos de Manasés según sus familias: para los hijos de Abiezer, para los
hijos de Helek, para los hijos de Asriel, para los hijos de Siquem, para
los hijos de Hefer y para los hijos de Semida. Estos fueron los hijos
varones de Manasés hijo de José según sus familias.

\hypertarget{las-hijas-de-zelofhad-son-herederas}{%
\subsection{Las hijas de Zelofhad son
herederas}\label{las-hijas-de-zelofhad-son-herederas}}

\bibleverse{3} Pero Zelofehad, hijo de Hefer, hijo de Galaad, hijo de
Maquir, hijo de Manasés, no tuvo hijos, sino hijas. Estos son los
nombres de sus hijas: Mahá, Noé, Hogá, Milca y Tirsa. \footnote{\textbf{17:3}
  Núm 26,33; Núm 27,1} \bibleverse{4} Vinieron al sacerdote Eleazar, a
Josué hijo de Nun y a los príncipes, diciendo: ``Yahvé ordenó a Moisés
que nos diera una herencia entre nuestros hermanos''. Por lo tanto,
según el mandato de Yahvé, les dio una herencia entre los hermanos de su
padre. \bibleverse{5} Diez partes le correspondieron a Manasés, además
de la tierra de Galaad y de Basán, que está al otro lado del Jordán;
\bibleverse{6} porque las hijas de Manasés tenían herencia entre sus
hijos. La tierra de Galaad pertenecía al resto de los hijos de Manasés.

\hypertarget{fronteras-y-ciudades-de-la-tribu-manasuxe9s}{%
\subsection{Fronteras y ciudades de la tribu
Manasés}\label{fronteras-y-ciudades-de-la-tribu-manasuxe9s}}

\bibleverse{7} El límite de Manasés era desde Aser hasta Micmetat, que
está frente a Siquem. La frontera se extendía a la derecha, hasta los
habitantes de En Tappuah. \bibleverse{8} La tierra de Tappuá pertenecía
a Manasés; pero Tappuá, en la frontera de Manasés, pertenecía a los
hijos de Efraín. \bibleverse{9} La frontera bajaba hasta el arroyo de
Caná, al sur del arroyo. Estas ciudades pertenecían a Efraín entre las
ciudades de Manasés. La frontera de Manasés estaba al norte del arroyo y
terminaba en el mar. \footnote{\textbf{17:9} Jos 16,9} \bibleverse{10}
Al sur era de Efraín y al norte de Manasés, y el mar era su frontera.
Llegaban hasta Aser por el norte y hasta Isacar por el este.
\bibleverse{11} Manasés tenía tres alturas en Isacar, en Aser Bet Shean
y sus pueblos, e Ibleam y sus pueblos, y los habitantes de Dor y sus
pueblos, y los habitantes de Endor y sus pueblos, y los habitantes de
Taanac y sus pueblos, y los habitantes de Meguido y sus pueblos.
\footnote{\textbf{17:11} Jue 1,27; 1Sam 28,7} \bibleverse{12} Sin
embargo, los hijos de Manasés no pudieron expulsar a los habitantes de
esas ciudades, sino que los cananeos habitarían en esa tierra.
\footnote{\textbf{17:12} Jos 15,63}

\bibleverse{13} Cuando los hijos de Israel se hicieron fuertes,
sometieron a los cananeos a trabajos forzados y no los expulsaron del
todo. \footnote{\textbf{17:13} Jos 16,10}

\hypertarget{josuuxe9-les-dice-a-las-tribus-de-josuxe9-que-se-quejan-que-limpien-el-bosque}{%
\subsection{Josué les dice a las tribus de José que se quejan que
limpien el
bosque}\label{josuuxe9-les-dice-a-las-tribus-de-josuxe9-que-se-quejan-que-limpien-el-bosque}}

\bibleverse{14} Los hijos de José hablaron con Josué y le dijeron:
``¿Por qué me has dado una sola parcela y una sola parte como herencia,
ya que somos un pueblo numeroso, porque Yahvé nos ha bendecido hasta
ahora?''

\bibleverse{15} Josué les dijo: ``Si sois un pueblo numeroso, subid al
bosque y haced tierra allí, en la tierra de los ferezeos y de los
refaítas, ya que la región montañosa de Efraín es demasiado estrecha
para vosotros.''

\bibleverse{16} Los hijos de José dijeron: ``La tierra de la colina no
es suficiente para nosotros. Todos los cananeos que habitan en la tierra
del valle tienen carros de hierro, tanto los que están en Bet Sheán y
sus ciudades, como los que están en el valle de Jezreel.''

\bibleverse{17} Josué habló a la casa de José, es decir, a Efraín y a
Manasés, diciendo: ``Vosotros sois un pueblo numeroso y tenéis un gran
poder. No tendréis un solo lote; \bibleverse{18} sino que la región
montañosa será vuestra. Aunque sea un bosque, lo cortaréis, y su máxima
extensión será vuestra; porque expulsaréis a los cananeos, aunque tengan
carros de hierro, y aunque sean fuertes.''

\hypertarget{la-tienda-de-la-revelaciuxf3n-se-instaluxf3-en-el-silo-admisiuxf3n-y-distribuciuxf3n-por-escrito-de-la-tierra-auxfan-desocupada}{%
\subsection{La tienda de la revelación se instaló en el silo; Admisión y
distribución por escrito de la tierra aún
desocupada}\label{la-tienda-de-la-revelaciuxf3n-se-instaluxf3-en-el-silo-admisiuxf3n-y-distribuciuxf3n-por-escrito-de-la-tierra-auxfan-desocupada}}

\hypertarget{section-17}{%
\section{18}\label{section-17}}

\bibleverse{1} Toda la congregación de los hijos de Israel se reunió en
Silo y levantó allí la Tienda del Encuentro. La tierra fue sometida ante
ellos. \footnote{\textbf{18:1} Jue 21,19; 1Sam 1,3; 1Sam 4,4}
\bibleverse{2} Quedaban siete tribus entre los hijos de Israel, que aún
no habían repartido su herencia. \bibleverse{3} Josué dijo a los hijos
de Israel: ``¿Hasta cuándo dejaréis de entrar a poseer la tierra que el
Señor, el Dios de vuestros padres, os ha dado? \bibleverse{4} Designen
para ustedes tres hombres de cada tribu. Yo los enviaré, y ellos se
levantarán, recorrerán la tierra y la describirán según su herencia;
luego vendrán a mí. \bibleverse{5} La dividirán en siete porciones. Judá
vivirá en sus límites al sur, y la casa de José vivirá en sus límites al
norte. \bibleverse{6} Ustedes harán un reconocimiento de la tierra en
siete partes, y me traerán la descripción aquí; y yo les echaré suertes
aquí, delante de Yavé, nuestro Dios. \bibleverse{7} Sin embargo, los
levitas no tienen parte entre ustedes, pues el sacerdocio de Yavé es su
herencia. Gad, Rubén y la media tribu de Manasés han recibido su
herencia al este del Jordán, que les dio Moisés, siervo de Yahvé.''
\footnote{\textbf{18:7} Jos 13,14-33}

\bibleverse{8} Los hombres se levantaron y se fueron. Josué ordenó a los
que fueron a reconocer la tierra, diciendo: ``Vayan a recorrer la
tierra, a reconocerla, y vuelvan a mí. Yo les echaré suertes aquí, ante
Yavé, en Silo''.

\bibleverse{9} Los hombres recorrieron la tierra, y la inspeccionaron
por ciudades en siete porciones en un libro. Llegaron a Josué al
campamento de Silo. \bibleverse{10} Josué les echó suertes en Silo ante
el Señor. Allí Josué repartió la tierra a los hijos de Israel según sus
divisiones.

\hypertarget{el-territorio-de-la-tribu-de-benjamuxedn}{%
\subsection{El territorio de la tribu de
Benjamín}\label{el-territorio-de-la-tribu-de-benjamuxedn}}

\bibleverse{11} La suerte de la tribu de los hijos de Benjamín salió
según sus familias. El límite de su suerte salió entre los hijos de Judá
y los hijos de José. \bibleverse{12} Su límite en la parte norte era
desde el Jordán. La frontera llegaba hasta el lado de Jericó, al norte,
y subía por la región montañosa hacia el oeste. Terminaba en el desierto
de Bet-Aven. \footnote{\textbf{18:12} Jos 7,2} \bibleverse{13} La
frontera pasaba desde allí hasta Luz, al lado de Luz (también llamada
Betel), hacia el sur. La frontera bajaba hasta Atarot Addar, junto al
monte que está al sur de Bet Horón el de abajo. \footnote{\textbf{18:13}
  Gén 12,8; Gén 28,19} \bibleverse{14} La frontera se extendía y daba la
vuelta por el barrio occidental hacia el sur, desde el monte que está
frente a Bet Horón hacia el sur, y terminaba en Quiriat Baal (también
llamada Quiriat Jearim), ciudad de los hijos de Judá. Este era el barrio
oeste. \footnote{\textbf{18:14} Jos 15,6-9} \bibleverse{15} El barrio
sur se extendía desde la parte más lejana de Quiriat Jearim. La frontera
salía hacia el oeste y llegaba hasta el manantial de las aguas de
Neftoa. \bibleverse{16} El límite descendía hasta la parte más lejana
del monte que está frente al valle del hijo de Hinom, que está en el
valle de Refaim hacia el norte. Bajaba hasta el valle de Hinom, al lado
del jebuseo hacia el sur, y descendía hasta En Rogel. \bibleverse{17} Se
extendía hacia el norte, salía a En Shemesh y salía a Geliloth, que está
frente a la subida de Adummim. Bajaba hasta la piedra de Bohán, hijo de
Rubén. \bibleverse{18} Pasaba por el lado opuesto al Arabá, hacia el
norte, y bajaba hasta el Arabá. \bibleverse{19} La frontera pasaba por
el lado de Bet Hogá hacia el norte, y terminaba en la bahía norte del
Mar Salado, en el extremo sur del Jordán. Esta era la frontera sur.
\bibleverse{20} El Jordán era su frontera por la parte oriental. Esta
era la herencia de los hijos de Benjamín, por los límites que la
rodeaban, según sus familias. \bibleverse{21} Las ciudades de la tribu
de los hijos de Benjamín, según sus familias, eran Jericó, Bet Hoglah,
Emek Keziz, \bibleverse{22} Bet Araba, Zemaraim, Betel, \bibleverse{23}
Avvim, Pará, Ofra, \bibleverse{24} Chefar Ammoni, Ofni y Geba; doce
ciudades con sus aldeas. \bibleverse{25} Gabaón, Ramá, Beerot,
\bibleverse{26} Mizpá, Quifira, Moza, \bibleverse{27} Rekem, Irpeel,
Taralá, \bibleverse{28} Zelá, Elef, la Jebusita (también llamada
Jerusalén), Gibeat y Quiriat; catorce ciudades con sus aldeas. Esta es
la herencia de los hijos de Benjamín según sus familias. \footnote{\textbf{18:28}
  Jos 15,63; Jue 1,21}

\hypertarget{el-territorio-de-la-tribu-simeuxf3n}{%
\subsection{El territorio de la tribu
Simeón}\label{el-territorio-de-la-tribu-simeuxf3n}}

\hypertarget{section-18}{%
\section{19}\label{section-18}}

\bibleverse{1} La segunda suerte salió para Simeón, para la tribu de los
hijos de Simeón según sus familias. Su herencia estaba en medio de la
herencia de los hijos de Judá. \bibleverse{2} Tuvieron por herencia
Beerseba (o Seba), Molada, \bibleverse{3} Hazar Sual, Balá, Ezem,
\bibleverse{4} Eltolad, Betul, Horma, \bibleverse{5} Siclag, Bet
Marcabot, Hazar Susa, \bibleverse{6} Bet Lebaot y Sharuhen; trece
ciudades con sus aldeas; \bibleverse{7} Ain, Rimmón, Éter y Asán; cuatro
ciudades con sus aldeas; \bibleverse{8} y todas las aldeas que estaban
alrededor de estas ciudades hasta Baalat Beer, Ramá del Sur. Esta es la
herencia de la tribu de los hijos de Simeón según sus familias.
\bibleverse{9} De la parte de los hijos de Judá fue la herencia de los
hijos de Simeón; porque la parte de los hijos de Judá era demasiado para
ellos. Por lo tanto, los hijos de Simeón tuvieron herencia en medio de
su heredad.

\hypertarget{el-territorio-de-la-tribu-zabuluxf3n}{%
\subsection{El territorio de la tribu
Zabulón}\label{el-territorio-de-la-tribu-zabuluxf3n}}

\bibleverse{10} La tercera suerte correspondió a los hijos de Zabulón
según sus familias. El límite de su herencia fue hasta Sarid.
\bibleverse{11} Su límite subía hacia el oeste, hasta Maralá, y llegaba
hasta Dabbeshet. Llegaba hasta el arroyo que está frente a Jocneam.
\bibleverse{12} Desde Sarid giraba hacia el este, hacia la salida del
sol, hasta el límite de Chisloth Tabor. Salió a Daberat, y subió a Jafa.
\bibleverse{13} De allí pasaba hacia el oriente hasta Gat Hefer, hasta
Etkazin; y salía en Rimón que se extiende hasta Neah. \bibleverse{14} La
frontera la rodeaba por el norte hasta Hannatón; y terminaba en el valle
de Iphtah El; \bibleverse{15} Kattath, Nahalal, Shimron, Idalah y Belén:
doce ciudades con sus aldeas. \footnote{\textbf{19:15} Jue 1,30}
\bibleverse{16} Esta es la herencia de los hijos de Zabulón según sus
familias, estas ciudades con sus aldeas.

\hypertarget{el-territorio-de-la-tribu-isacar}{%
\subsection{El territorio de la tribu
Isacar}\label{el-territorio-de-la-tribu-isacar}}

\bibleverse{17} La cuarta suerte salió para Isacar, para los hijos de
Isacar según sus familias. \bibleverse{18} Su límite era Jezreel,
Cesulot, Sunem, \footnote{\textbf{19:18} 2Re 4,8} \bibleverse{19}
Hafaraim, Shion, Anaharat, \bibleverse{20} Rabbith, Kishion, Ebez,
\bibleverse{21} Remeth, Engannim, En Haddah y Bet Pazzez.
\bibleverse{22} La frontera llegaba hasta Tabor, Shahazumah y Beth
Shemesh. Su frontera terminaba en el Jordán: dieciséis ciudades con sus
aldeas. \bibleverse{23} Esta es la herencia de la tribu de los hijos de
Isacar según sus familias, las ciudades con sus aldeas.

\hypertarget{el-territorio-de-la-tribu-asser}{%
\subsection{El territorio de la tribu
Asser}\label{el-territorio-de-la-tribu-asser}}

\bibleverse{24} La quinta suerte salió para la tribu de los hijos de
Aser según sus familias. \bibleverse{25} Su frontera era Helkath, Hali,
Beten, Achshaph, \bibleverse{26} Allammelech, Amad, Mishal. Llegaba
hasta el Carmelo, al oeste, y hasta Shihorlibnath. \bibleverse{27} Se
volvió hacia la salida del sol hasta Bet Dagón, y llegó hasta Zabulón, y
hasta el valle de Iftá El hacia el norte, hasta Bet Emek y Neiel. Salía
a Cabul por la izquierda, \bibleverse{28} y a Ebrón, Rehob, Hamón y
Caná, hasta la gran Sidón. \bibleverse{29} La frontera daba vuelta a
Rama, a la ciudad fortificada de Tiro; y la frontera daba vuelta a Hosa.
Termina en el mar, junto a la región de Aczib; \footnote{\textbf{19:29}
  Jos 15,44; Jue 1,31} \bibleverse{30} También Umma, Afec y Rehob:
veintidós ciudades con sus aldeas. \bibleverse{31} Esta es la herencia
de la tribu de los hijos de Aser según sus familias, estas ciudades con
sus aldeas.

\hypertarget{el-territorio-de-la-tribu-naftaluxed}{%
\subsection{El territorio de la tribu
Naftalí}\label{el-territorio-de-la-tribu-naftaluxed}}

\bibleverse{32} La sexta suerte salió para los hijos de Neftalí, para
los hijos de Neftalí según sus familias. \bibleverse{33} Su frontera era
desde Helef, desde la encina de Zaanannim, Adami-nekeb y Jabneel, hasta
Lakkum. Terminaba en el Jordán. \bibleverse{34} La frontera giraba hacia
el oeste hasta Aznoth Tabor, y salía de allí hasta Hukkok. Llegaba hasta
Zabulón por el sur, y llegaba hasta Aser por el oeste, y hasta Judá en
el Jordán, hacia la salida del sol. \bibleverse{35} Las ciudades
fortificadas eran Ziddim, Zer, Hamat, Rakkat, Chinnereth,
\bibleverse{36} Adamah, Ramah, Hazor, \bibleverse{37} Kedesh, Edrei, En
Hazor, \bibleverse{38} Hierro, Migdal El, Horem, Beth Anath y Beth
Shemesh; diecinueve ciudades con sus aldeas. \footnote{\textbf{19:38}
  Jue 1,33} \bibleverse{39} Esta es la herencia de la tribu de los hijos
de Neftalí según sus familias, las ciudades con sus aldeas.

\hypertarget{el-territorio-de-la-tribu-dan}{%
\subsection{El territorio de la tribu
Dan}\label{el-territorio-de-la-tribu-dan}}

\bibleverse{40} La séptima suerte correspondió a la tribu de los hijos
de Dan según sus familias. \bibleverse{41} El límite de su heredad fue
Zora, Eshtaol, Irshemesh, \bibleverse{42} Shaalabbin, Aijalon, Ithlah,
\footnote{\textbf{19:42} Jue 1,35} \bibleverse{43} Elon, Timnah, Ecron,
\bibleverse{44} Eltekeh, Gibbethon, Baalath, \bibleverse{45} Jehud, Bene
Berak, Gath Rimmon, \bibleverse{46} Me Jarkon, y Rakkon, con el límite
frente a Joppa. \footnote{\textbf{19:46} Jon 1,3} \bibleverse{47} El
límite de los hijos de Dan iba más allá de ellos, pues los hijos de Dan
subieron y combatieron contra Leshem, la tomaron y la hirieron a filo de
espada, la poseyeron y vivieron en ella, y llamaron a Leshem, Dan, por
el nombre de Dan, su antepasado. \footnote{\textbf{19:47} Jue 18,27; Jue
  18,29} \bibleverse{48} Esta es la herencia de la tribu de los hijos de
Dan según sus familias, estas ciudades con sus aldeas.

\hypertarget{la-propiedad-de-joshua-finalizaciuxf3n-del-informe}{%
\subsection{La propiedad de Joshua; Finalización del
informe}\label{la-propiedad-de-joshua-finalizaciuxf3n-del-informe}}

\bibleverse{49} Así terminaron de distribuir la tierra en herencia por
sus fronteras. Los hijos de Israel dieron en herencia a Josué, hijo de
Nun, entre ellos. \bibleverse{50} De acuerdo con el mandato de Yahvé, le
dieron la ciudad que pidió, es decir, Timnathserah, en la región
montañosa de Efraín; y él edificó la ciudad y vivió allí. \footnote{\textbf{19:50}
  Jos 24,30} \bibleverse{51} Estas son las herencias que el sacerdote
Eleazar, Josué hijo de Nun y los jefes de las casas paternas de las
tribus de los hijos de Israel, repartieron por sorteo en Silo, delante
de Yavé, a la puerta de la Tienda del Encuentro. Así terminaron de
repartir la tierra. \footnote{\textbf{19:51} Jos 14,1; Jos 18,1}

\hypertarget{die-sechs-zufluchts--oder-freistuxe4dte-el-mandato-divino}{%
\subsection{Die sechs Zufluchts- oder Freistädte; El mandato
divino}\label{die-sechs-zufluchts--oder-freistuxe4dte-el-mandato-divino}}

\hypertarget{section-19}{%
\section{20}\label{section-19}}

\bibleverse{1} Yahvé habló a Josué, diciendo: \bibleverse{2} ``Habla a
los hijos de Israel, diciendo: `Asigna las ciudades de refugio, de las
que te hablé por medio de Moisés, \footnote{\textbf{20:2} Núm 35,6-29}
\bibleverse{3} para que el homicida que mate a cualquier persona
accidentalmente o sin intención pueda huir allí. Le servirán de refugio
contra el vengador de la sangre. \bibleverse{4} Huirá a una de esas
ciudades, se pondrá a la entrada de la puerta de la ciudad y declarará
su caso a los oídos de los ancianos de esa ciudad. Ellos lo llevarán a
la ciudad con ellos, y le darán un lugar, para que viva entre ellos.
\bibleverse{5} Si el vengador de la sangre lo persigue, no entregarán al
homicida en su mano, porque golpeó a su prójimo sin querer y no lo
odiaba antes. \bibleverse{6} El habitará en esa ciudad hasta que se
presente ante la congregación para el juicio, hasta la muerte del sumo
sacerdote que habrá en esos días. Entonces el homicida regresará y
volverá a su ciudad y a su casa, a la ciudad de la que huyó''.

\hypertarget{ejecuciuxf3n-del-mandato}{%
\subsection{Ejecución del mandato}\label{ejecuciuxf3n-del-mandato}}

\bibleverse{7} Asignaron Cedes en Galilea en la región montañosa de
Neftalí, Siquem en la región montañosa de Efraín, y Quiriat Arba
(también llamada Hebrón) en la región montañosa de Judá. \footnote{\textbf{20:7}
  Jos 19,37; Jos 15,13} \bibleverse{8} Más allá del Jordán, en Jericó,
hacia el este, asignaron a Beser en el desierto, en la llanura, de la
tribu de Rubén, a Ramot en Galaad, de la tribu de Gad, y a Golán en
Basán, de la tribu de Manasés. \footnote{\textbf{20:8} Deut 4,43}
\bibleverse{9} Estas fueron las ciudades designadas para todos los hijos
de Israel, y para el extranjero que vive entre ellos, para que el que
matara a cualquier persona sin querer pudiera huir allí, y no muriera
por la mano del vengador de la sangre, hasta que fuera juzgado ante la
congregación.

\hypertarget{las-cuarenta-y-ocho-ciudades-sacerdotales-y-levitas}{%
\subsection{Las cuarenta y ocho ciudades sacerdotales y
levitas}\label{las-cuarenta-y-ocho-ciudades-sacerdotales-y-levitas}}

\hypertarget{section-20}{%
\section{21}\label{section-20}}

\bibleverse{1} Los jefes de familia de los levitas se acercaron al
sacerdote Eleazar, a Josué hijo de Nun y a los jefes de familia de las
tribus de los hijos de Israel. \footnote{\textbf{21:1} Jos 14,1}
\bibleverse{2} Les hablaron en Silo, en la tierra de Canaán, diciendo:
``Yahvé ordenó por medio de Moisés que nos dieran ciudades para habitar,
con sus tierras de pastoreo para nuestro ganado.'' \footnote{\textbf{21:2}
  Núm 35,2-8}

\bibleverse{3} Los hijos de Israel dieron a los levitas, de su herencia,
según el mandato de Yavé, estas ciudades con sus tierras de pastoreo.
\bibleverse{4} La suerte salió para las familias de los coatitas. Los
hijos del sacerdote Aarón, que eran de los levitas, tuvieron trece
ciudades por sorteo de la tribu de Judá, de la tribu de los simeonitas y
de la tribu de Benjamín. \footnote{\textbf{21:4} 1Cró 6,54-81}
\bibleverse{5} Los demás hijos de Coat tenían diez ciudades por sorteo
de las familias de la tribu de Efraín, de la tribu de Dan y de la media
tribu de Manasés. \bibleverse{6} Los hijos de Gersón tuvieron trece
ciudades por sorteo de las familias de la tribu de Isacar, de la tribu
de Aser, de la tribu de Neftalí y de la media tribu de Manasés en Basán.
\bibleverse{7} Los hijos de Merari, según sus familias, tenían doce
ciudades de la tribu de Rubén, de la tribu de Gad y de la tribu de
Zabulón. \bibleverse{8} Los hijos de Israel dieron estas ciudades con
sus tierras de pastoreo por sorteo a los levitas, como lo ordenó el
Señor por medio de Moisés. \bibleverse{9} Dieron de la tribu de los
hijos de Judá, y de la tribu de los hijos de Simeón, estas ciudades que
se mencionan por su nombre: \bibleverse{10} y fueron para los hijos de
Aarón, de las familias de los coatitas, que eran de los hijos de Leví;
porque de ellos fue la primera suerte. \bibleverse{11} Les dieron
Quiriat Arba, llamada así por el padre de Anac (también llamada Hebrón),
en la región montañosa de Judá, con sus tierras de pastoreo alrededor.
\footnote{\textbf{21:11} Jos 20,7} \bibleverse{12} Pero dieron los
campos de la ciudad y sus aldeas a Caleb, hijo de Jefone, para su
posesión. \footnote{\textbf{21:12} Jos 14,14; Jos 15,13} \bibleverse{13}
A los hijos del sacerdote Aarón les dieron Hebrón con sus tierras de
pastoreo, la ciudad de refugio para el matador de hombres, Libna con sus
tierras de pastoreo, \bibleverse{14} Jattir con sus tierras de pastoreo,
Estemoa con sus tierras de pastoreo, \bibleverse{15} Holón con sus
tierras de pastoreo, Debir con sus tierras de pastoreo, \bibleverse{16}
Ain con sus tierras de pastoreo, Jutá con sus tierras de pastoreo y Bet
Semes con sus tierras de pastoreo: nueve ciudades de esas dos tribus.
\footnote{\textbf{21:16} 1Sam 6,12; 1Sam 6,15} \bibleverse{17} De la
tribu de Benjamín, Gabaón con sus tierras de pastoreo, Geba con sus
tierras de pastoreo, \bibleverse{18} Anatot con sus tierras de pastoreo
y Almón con sus tierras de pastoreo: cuatro ciudades. \footnote{\textbf{21:18}
  Jer 1,1} \bibleverse{19} Todas las ciudades de los hijos de Aarón, los
sacerdotes, eran trece ciudades con sus tierras de pastoreo.

\bibleverse{20} Las familias de los hijos de Coat, los levitas, el resto
de los hijos de Coat, tuvieron las ciudades de su lote de la tribu de
Efraín. \bibleverse{21} Les dieron Siquem con sus tierras de pastoreo en
la región montañosa de Efraín, la ciudad de refugio para el matador de
hombres, y Gezer con sus tierras de pastoreo, \footnote{\textbf{21:21}
  Jos 20,7} \bibleverse{22} Kibzaim con sus tierras de pastoreo, y Bet
Horón con sus tierras de pastoreo: cuatro ciudades. \bibleverse{23} De
la tribu de Dan, Elteke con sus tierras de pastoreo, Gibetón con sus
tierras de pastoreo, \bibleverse{24} Ajalón con sus tierras de pastoreo,
Gat Rimón con sus tierras de pastoreo: cuatro ciudades. \bibleverse{25}
De la media tribu de Manasés, Taanac con sus tierras de pastoreo y Gat
Rimón con sus tierras de pastoreo: dos ciudades. \bibleverse{26} Todas
las ciudades de las familias del resto de los hijos de Coat fueron diez
con sus tierras de pastoreo.

\bibleverse{27} Dieron a los hijos de Gersón, de las familias de los
levitas, de la media tribu de Manasés, Golán en Basán con sus tierras de
pastoreo, la ciudad de refugio para el matador de hombres, y Be Esterá
con sus tierras de pastoreo: dos ciudades. \footnote{\textbf{21:27} Jos
  20,8} \bibleverse{28} De la tribu de Isacar, Kishion con sus tierras
de pastoreo, Daberat con sus tierras de pastoreo, \bibleverse{29} Jarmut
con sus tierras de pastoreo, En Gannim con sus tierras de pastoreo:
cuatro ciudades. \bibleverse{30} De la tribu de Aser, Mishal con sus
tierras de pastoreo, Abdón con sus tierras de pastoreo, \bibleverse{31}
Helkat con sus tierras de pastoreo, y Rehob con sus tierras de pastoreo:
cuatro ciudades. \bibleverse{32} De la tribu de Neftalí, Cedes en
Galilea con sus tierras de pastoreo, la ciudad de refugio para el
matador de hombres, Hamotdor con sus tierras de pastoreo y Kartán con
sus tierras de pastoreo: tres ciudades. \footnote{\textbf{21:32} Jos
  20,7} \bibleverse{33} Todas las ciudades de los gersonitas, según sus
familias, eran trece ciudades con sus tierras de pastoreo.

\bibleverse{34} A las familias de los hijos de Merari, el resto de los
levitas, de la tribu de Zabulón, Jocneam con sus tierras de pastoreo,
Kartah con sus tierras de pastoreo, \bibleverse{35} Dimna con sus
tierras de pastoreo y Nahalal con sus tierras de pastoreo: cuatro
ciudades. \bibleverse{36} De la tribu de Rubén, Beser con sus tierras de
pastoreo, Jahaz con sus tierras de pastoreo, \footnote{\textbf{21:36}
  Jos 20,8} \bibleverse{37} Cedemot con sus tierras de pastoreo, y Mefat
con sus tierras de pastoreo: cuatro ciudades. \bibleverse{38} De la
tribu de Gad, Ramot en Galaad con sus tierras de pastoreo, la ciudad de
refugio para el matador de hombres, y Mahanaim con sus tierras de
pastoreo, \footnote{\textbf{21:38} Jos 20,8} \bibleverse{39} Hesbón con
sus tierras de pastoreo, Jazer con sus tierras de pastoreo: cuatro
ciudades en total. \bibleverse{40} Todas estas fueron las ciudades de
los hijos de Merari según sus familias, el resto de las familias de los
levitas. Su suerte fue de doce ciudades.

\bibleverse{41} Todas las ciudades de los levitas entre las posesiones
de los hijos de Israel eran cuarenta y ocho ciudades con sus tierras de
pastoreo. \bibleverse{42} Cada una de estas ciudades incluía sus tierras
de pastoreo alrededor de ellas. Así fue con todas estas ciudades.

\hypertarget{revisiuxf3n-final}{%
\subsection{Revisión final}\label{revisiuxf3n-final}}

\bibleverse{43} El Señor dio a Israel toda la tierra que había jurado
dar a sus padres. La poseyeron y vivieron en ella. \footnote{\textbf{21:43}
  Gén 12,7} \bibleverse{44} El Señor les dio descanso en todo el
territorio, según lo que había jurado a sus padres. Ni un solo hombre de
todos sus enemigos se presentó ante ellos. El Señor entregó a todos sus
enemigos en sus manos. \bibleverse{45} No faltó nada de lo bueno que el
Señor había dicho a la casa de Israel. Todo se cumplió. \footnote{\textbf{21:45}
  Jos 23,14}

\hypertarget{josuuxe9-despide-a-las-tribus-con-palabras-de-aprobaciuxf3n-amonestaciuxf3n-y-bendiciuxf3n}{%
\subsection{Josué despide a las tribus con palabras de aprobación,
amonestación y
bendición}\label{josuuxe9-despide-a-las-tribus-con-palabras-de-aprobaciuxf3n-amonestaciuxf3n-y-bendiciuxf3n}}

\hypertarget{section-21}{%
\section{22}\label{section-21}}

\bibleverse{1} Entonces Josué llamó a los rubenitas, a los gaditas y a
la media tribu de Manasés, \bibleverse{2} y les dijo: ``Habéis guardado
todo lo que Moisés, siervo de Yavé, os ha ordenado, y habéis escuchado
mi voz en todo lo que os he mandado. \footnote{\textbf{22:2} Núm
  32,20-22; Deut 3,18-20} \bibleverse{3} No habéis dejado a vuestros
hermanos en estos muchos días hasta hoy, sino que habéis cumplido el
deber del mandamiento de Yahvé vuestro Dios. \bibleverse{4} Ahora Yahvé
tu Dios ha dado descanso a tus hermanos, tal como les habló. Por lo
tanto, regresen ahora y vayan a sus tiendas, a la tierra de su posesión,
que Moisés, siervo de Yavé, les dio al otro lado del Jordán.
\bibleverse{5} Sólo cuida de poner en práctica el mandamiento y la ley
que Moisés, siervo de Yavé, te ordenó: amar a Yavé tu Dios, andar por
todos sus caminos, guardar sus mandamientos, aferrarte a él y servirle
con todo tu corazón y con toda tu alma.''

\bibleverse{6} Josué los bendijo y los despidió, y se fueron a sus
tiendas. \bibleverse{7} A la mitad de la tribu de Manasés Moisés le
había dado herencia en Basán, pero Josué le dio a la otra mitad entre
sus hermanos al otro lado del Jordán, hacia el oeste. Además, cuando
Josué los despidió a sus tiendas, los bendijo, \bibleverse{8} y les
habló diciendo: ``Volved con mucha riqueza a vuestras tiendas, con mucho
ganado, con plata, con oro, con bronce, con hierro y con mucha ropa.
Repartan el botín de sus enemigos con sus hermanos''. \footnote{\textbf{22:8}
  Núm 31,27}

\hypertarget{la-construcciuxf3n-del-altar-de-las-tribus-cisjordanias-en-gilgal-y-sus-malvadas-consecuencias-discurso-del-sacerdote-phinees}{%
\subsection{La construcción del altar de las tribus Cisjordanias en
Gilgal y sus malvadas consecuencias; Discurso del sacerdote
Phinees}\label{la-construcciuxf3n-del-altar-de-las-tribus-cisjordanias-en-gilgal-y-sus-malvadas-consecuencias-discurso-del-sacerdote-phinees}}

\bibleverse{9} Los hijos de Rubén y los hijos de Gad y la media tribu de
Manasés volvieron y se apartaron de los hijos de Israel desde Silo, que
está en la tierra de Canaán, para ir a la tierra de Galaad, a la tierra
de su posesión, que poseían, según el mandato de Yahvé por medio de
Moisés. \bibleverse{10} Cuando llegaron a la región cercana al Jordán,
que está en la tierra de Canaán, los hijos de Rubén y los hijos de Gad y
la media tribu de Manasés construyeron allí un altar junto al Jordán, un
gran altar para mirar. \bibleverse{11} Los hijos de Israel oyeron esto:
``He aquí que los hijos de Rubén y los hijos de Gad y la media tribu de
Manasés han edificado un altar a lo largo de la frontera de la tierra de
Canaán, en la región alrededor del Jordán, del lado que pertenece a los
hijos de Israel.'' \bibleverse{12} Cuando los hijos de Israel se
enteraron de esto, toda la congregación de los hijos de Israel se reunió
en Silo para subir contra ellos a la guerra. \bibleverse{13} Los hijos
de Israel enviaron a los hijos de Rubén, a los hijos de Gad y a la media
tribu de Manasés a la tierra de Galaad, a Finees, hijo del sacerdote
Eleazar. \footnote{\textbf{22:13} Núm 25,7} \bibleverse{14} Con él había
diez príncipes, un príncipe de una casa paterna por cada una de las
tribus de Israel, y cada uno era jefe de su casa paterna entre los
millares de Israel. \bibleverse{15} Vinieron a los hijos de Rubén, a los
hijos de Gad y a la media tribu de Manasés, a la tierra de Galaad, y
hablaron con ellos diciendo: \bibleverse{16} ``Toda la congregación de
Yavé dice: ``¿Qué transgresión es ésta que habéis cometido contra el
Dios de Israel, al apartaros hoy de seguir a Yavé, al construiros un
altar, para rebelaros hoy contra Yavé? \footnote{\textbf{22:16} Deut
  12,13-14; Lev 17,8-9} \bibleverse{17} ¿Acaso es poca la iniquidad de
Peor, de la cual no nos hemos limpiado hasta el día de hoy, a pesar de
que vino una plaga sobre la congregación de Yahvé, \footnote{\textbf{22:17}
  Núm 25,1} \bibleverse{18} para que ustedes se aparten hoy de seguir a
Yahvé? Será que, puesto que hoy os rebeláis contra Yahvé, mañana él se
enojará con toda la congregación de Israel. \bibleverse{19} Sin embargo,
si la tierra de tu posesión es impura, pasa a la tierra de la posesión
de Yavé, en la que habita el tabernáculo de Yavé, y toma posesión entre
nosotros; pero no te rebeles contra Yavé, ni te rebeles contra nosotros,
construyendo un altar que no sea el altar de Yavé, nuestro Dios.
\bibleverse{20} ¿No cometió Acán, hijo de Zéraj, una transgresión en lo
consagrado, y la ira cayó sobre toda la congregación de Israel? Ese
hombre no pereció solo en su iniquidad''. \footnote{\textbf{22:20} Jos
  7,1}

\hypertarget{las-tribus-de-cisjordania-se-justifican-con-uxe9xito}{%
\subsection{Las tribus de Cisjordania se justifican con
éxito}\label{las-tribus-de-cisjordania-se-justifican-con-uxe9xito}}

\bibleverse{21} Entonces los hijos de Rubén y los hijos de Gad y la
media tribu de Manasés respondieron y hablaron a los jefes de los
millares de Israel: \footnote{\textbf{22:21} Núm 1,16; Núm 10,4}
\bibleverse{22} ``El Poderoso, Dios, Yahvé, el Poderoso, Dios, Yahvé,
sabe; e Israel sabrá si fue por rebelión, o si por transgresión contra
Yahvé (no nos salve hoy), \bibleverse{23} que nos hemos construido un
altar para apartarnos de seguir a Yahvé; o si para ofrecer holocausto u
ofrenda, o si para ofrecer sacrificios de ofrendas de paz, que Yahvé
mismo lo exija.

\bibleverse{24} ``Si no hemos hecho esto por preocupación y por una
razón, diciendo: ``En el futuro, vuestros hijos podrían hablar a los
nuestros, diciendo: ``¿Qué tenéis vosotros que ver con Yahvé, el Dios de
Israel? \bibleverse{25} Porque Yahvé ha puesto el Jordán como frontera
entre nosotros y vosotros, hijos de Rubén e hijos de Gad. Vosotros no
tenéis parte en Yahvé''. Para que vuestros hijos hagan que los nuestros
dejen de temer a Yahvé.

\bibleverse{26} ``Por eso dijimos: `Preparemos ahora para construirnos
un altar, no para holocaustos ni para sacrificios; \bibleverse{27} sino
que será un testimonio entre nosotros y ustedes, y entre nuestras
generaciones después de nosotros, para que realicemos el servicio de
Yahvé ante él con nuestros holocaustos, con nuestros sacrificios y con
nuestras ofrendas de paz;' para que sus hijos no digan a los nuestros en
el futuro: `Ustedes no tienen parte en Yahvé'. \footnote{\textbf{22:27}
  Jos 24,27}

\bibleverse{28} ``Por eso dijimos: ``Cuando nos cuenten esto a nosotros
o a nuestras generaciones en el futuro, diremos: ``He aquí el modelo del
altar de Yahvé, que hicieron nuestros padres, no para holocausto ni para
sacrificio, sino que es un testigo entre nosotros y vosotros''\,''.

\bibleverse{29} ``¡Lejos de nosotros que nos rebelemos contra Yahvé y
nos apartemos hoy de seguir a Yahvé, para construir un altar para
holocausto, para ofrenda o para sacrificio, además del altar de Yahvé
nuestro Dios que está delante de su tabernáculo!''

\bibleverse{30} Cuando el sacerdote Finees y los jefes de la
congregación, los jefes de los millares de Israel que estaban con él,
oyeron las palabras que decían los hijos de Rubén, los hijos de Gad y
los hijos de Manasés, les pareció bien. \bibleverse{31} Finees, hijo del
sacerdote Eleazar, dijo a los hijos de Rubén, a los hijos de Gad y a los
hijos de Manasés: ``Hoy sabemos que Yavé está entre nosotros, porque
ustedes no han cometido esta transgresión contra Yavé. Ahora habéis
librado a los hijos de Israel de la mano de Yahvé''. \bibleverse{32}
Finees, hijo del sacerdote Eleazar, y los príncipes, volvieron de los
hijos de Rubén y de los hijos de Gad, de la tierra de Galaad, a la
tierra de Canaán, a los hijos de Israel, y les trajeron la noticia.
\bibleverse{33} Esto agradó a los hijos de Israel; y los hijos de Israel
bendijeron a Dios, y no hablaron más de subir contra ellos a la guerra,
para destruir la tierra en que vivían los hijos de Rubén y los hijos de
Gad. \bibleverse{34} Los hijos de Rubén y los hijos de Gad llamaron al
altar ``Testigo entre nosotros de que Yahvé es Dios''.

\hypertarget{el-primer-discurso-de-amonestaciuxf3n-de-josuuxe9-a-los-representantes-de-israel}{%
\subsection{El primer discurso de amonestación de Josué a los
representantes de
Israel}\label{el-primer-discurso-de-amonestaciuxf3n-de-josuuxe9-a-los-representantes-de-israel}}

\hypertarget{section-22}{%
\section{23}\label{section-22}}

\bibleverse{1} Después de muchos días, cuando Yahvé había dado descanso
a Israel de sus enemigos de alrededor, y Josué era viejo y bien avanzado
en años, \footnote{\textbf{23:1} Jos 21,44} \bibleverse{2} Josué convocó
a todo Israel, a sus ancianos y a sus jefes, a sus jueces y a sus
oficiales, y les dijo: ``Soy viejo y bien avanzado en años.
\bibleverse{3} Ustedes han visto todo lo que Yahvé su Dios ha hecho a
todas estas naciones por causa de ustedes; porque es Yahvé su Dios quien
ha luchado por ustedes. \bibleverse{4} He aquí que te he asignado estas
naciones que quedan, para que sean una herencia para tus tribus, desde
el Jordán, con todas las naciones que he cortado, hasta el gran mar
hacia la puesta del sol. \bibleverse{5} El Señor, tu Dios, las echará de
delante de ti y las expulsará de tu vista. Poseerás su tierra, tal como
te habló el Señor tu Dios.

\bibleverse{6} ``Por lo tanto, tened mucho ánimo para guardar y hacer
todo lo que está escrito en el libro de la ley de Moisés, para que no os
apartéis de él ni a la derecha ni a la izquierda; \footnote{\textbf{23:6}
  Deut 5,29} \bibleverse{7} para que no os acerquéis a esas naciones que
quedan entre vosotros, ni hagáis mención del nombre de sus dioses, ni
hagáis jurar por ellos, ni les sirváis, ni os inclinéis ante ellos;
\footnote{\textbf{23:7} Éxod 23,13; Éxod 23,24} \bibleverse{8} sino que
os aferréis a Yahvé, vuestro Dios, como lo habéis hecho hasta hoy.

\bibleverse{9} ``Porque el Señor ha expulsado de delante de ti a
naciones grandes y fuertes. Pero en cuanto a ti, ningún hombre se ha
enfrentado a ti hasta el día de hoy. \footnote{\textbf{23:9} Lev 26,7-8;
  Deut 28,7} \bibleverse{10} Un solo hombre de vosotros perseguirá a
mil, porque es Yahvé vuestro Dios quien lucha por vosotros, como os ha
dicho. \bibleverse{11} Por lo tanto, cuidaos bien de amar a Yahvé,
vuestro Dios.

\bibleverse{12} ``Pero si en algún momento retrocedes y te aferras a los
restos de estas naciones, a los que quedan en medio de ti, y contraes
matrimonio con ellos, y te acercas a ellos, y ellos a ti;
\bibleverse{13} ten por seguro que el Señor, tu Dios, ya no echará a
estas naciones de tu vista, sino que serán para ti un lazo y una trampa,
un azote en tus costados y espinas en tus ojos, hasta que perezcas de
esta buena tierra que el Señor, tu Dios, te ha dado. \footnote{\textbf{23:13}
  Núm 33,55; Deut 7,16; Jue 2,3}

\bibleverse{14} ``He aquí que hoy voy a recorrer el camino de toda la
tierra. Vosotros sabéis en todo vuestro corazón y en toda vuestra alma
que no ha faltado ni una sola cosa de todas las buenas que el Señor,
vuestro Dios, habló de vosotros. Todo os ha sucedido. No ha faltado ni
una sola cosa. \footnote{\textbf{23:14} 1Re 2,2; Jos 21,45}
\bibleverse{15} Sucederá que así como os han sucedido todas las cosas
buenas de las que os habló Yahvé vuestro Dios, así también Yahvé traerá
sobre vosotros todas las cosas malas, hasta que os haya destruido de
esta buena tierra que Yahvé vuestro Dios os ha dado, \bibleverse{16}
cuando desobedezcáis el pacto de Yahvé vuestro Dios, que él os mandó, y
vayáis a servir a otros dioses, y os inclinéis ante ellos. Entonces la
ira de Yahvé se encenderá contra vosotros, y pereceréis rápidamente de
la buena tierra que os ha dado.''

\hypertarget{josuuxe9-se-despide-de-la-gente-en-la-dieta-de-siquem}{%
\subsection{Josué se despide de la gente en la Dieta de
Siquem}\label{josuuxe9-se-despide-de-la-gente-en-la-dieta-de-siquem}}

\hypertarget{section-23}{%
\section{24}\label{section-23}}

\bibleverse{1} Josué reunió a todas las tribus de Israel en Siquem, y
llamó a los ancianos de Israel, a sus jefes, a sus jueces y a sus
oficiales, y se presentaron ante Dios. \bibleverse{2} Josué dijo a todo
el pueblo: ``Yahvé, el Dios de Israel, dice: `Vuestros padres vivieron
antiguamente al otro lado del río, Téraj, padre de Abraham, y padre de
Nacor. Ellos sirvieron a otros dioses. \footnote{\textbf{24:2} Gén
  11,26; Gén 31,19; Gén 35,2} \bibleverse{3} Yo tomé a vuestro padre
Abraham del otro lado del río y lo conduje por toda la tierra de Canaán,
y multipliqué su descendencia,\footnote{\textbf{24:3} o, semilla} y le
di a Isaac. \bibleverse{4} A Isaac le di Jacob y Esaú, y a Esaú le di el
monte Seír para que lo poseyera. Jacob y sus hijos descendieron a
Egipto. \footnote{\textbf{24:4} Gén 32,3; Gén 46,6}

\bibleverse{5} ``\,`Yo envié a Moisés y a Aarón, y plagué a Egipto,
según lo que hice entre ellos; y después os saqué. \footnote{\textbf{24:5}
  Éxod 3,10} \bibleverse{6} Yo saqué a vuestros padres de Egipto, y
llegasteis al mar. Los egipcios persiguieron a vuestros padres con
carros y con jinetes hasta el Mar Rojo. \footnote{\textbf{24:6} Éxod
  12,33} \bibleverse{7} Cuando clamaron a Yahvé, él puso tinieblas entre
vosotros y los egipcios, e hizo que el mar los cubriera; y vuestros ojos
vieron lo que hice en Egipto. Ustedes vivieron en el desierto muchos
días. \footnote{\textbf{24:7} Éxod 14,10}

\bibleverse{8} ``\,`Yo te llevé a la tierra de los amorreos, que vivían
al otro lado del Jordán. Ellos pelearon contigo, y yo los entregué en tu
mano. Tú poseíste su tierra, y yo los destruí delante de ti. \footnote{\textbf{24:8}
  Núm 21,25; Núm 21,31} \bibleverse{9} Entonces Balac, hijo de Zipor,
rey de Moab, se levantó y luchó contra Israel. Envió y llamó a Balaam
hijo de Beor para que te maldijera, \footnote{\textbf{24:9} Núm 22,1}
\bibleverse{10} pero yo no quise escuchar a Balaam, por lo que te siguió
bendiciendo. Así que te libré de su mano. \footnote{\textbf{24:10} Núm
  23,11; Núm 23,20}

\bibleverse{11} ``\,`Pasasteis el Jordán y llegasteis a Jericó. Los
hombres de Jericó pelearon contra ti, el amorreo, el ferezeo, el
cananeo, el hitita, el gergeseo, el heveo y el jebuseo; y yo los
entregué en tu mano. \footnote{\textbf{24:11} Jos 3,14; Jos 6,1}
\bibleverse{12} Envié el avispero delante de ti, que los expulsó de tu
presencia, a los dos reyes de los amorreos; no con tu espada ni con tu
arco. \footnote{\textbf{24:12} Éxod 23,28} \bibleverse{13} Te di una
tierra en la que no habías trabajado, y ciudades que no habías
construido, y vives en ellas. Comes de viñas y olivares que no
plantaste'. \footnote{\textbf{24:13} Deut 6,10-11}

\bibleverse{14} ``Ahora, pues, temed a Yahvé y servidle con sinceridad y
con verdad. Dejad los dioses a los que vuestros padres sirvieron al otro
lado del río, en Egipto, y servid a Yahvé. \footnote{\textbf{24:14} Jos
  24,2; Éxod 32,1} \bibleverse{15} Si os parece mal servir a Yahvé,
elegid hoy a quién serviréis; si a los dioses a los que sirvieron
vuestros padres que estaban al otro lado del río, o a los dioses de los
amorreos, en cuya tierra habitáis; pero en cuanto a mí y a mi casa,
serviremos a Yahvé.'' \footnote{\textbf{24:15} Mat 6,24}

\hypertarget{el-pueblo-promete-obediencia-leal-y-josuuxe9-se-compromete-de-nuevo-solemnemente-con-dios}{%
\subsection{El pueblo promete obediencia leal y Josué se compromete de
nuevo solemnemente con
Dios}\label{el-pueblo-promete-obediencia-leal-y-josuuxe9-se-compromete-de-nuevo-solemnemente-con-dios}}

\bibleverse{16} El pueblo respondió: ``Lejos de nosotros abandonar a
Yahvé para servir a otros dioses; \bibleverse{17} porque es Yahvé
nuestro Dios quien nos sacó a nosotros y a nuestros padres de la tierra
de Egipto, de la casa de servidumbre, y quien hizo esas grandes señales
ante nuestros ojos, y nos preservó en todo el camino por el que fuimos,
y en medio de todos los pueblos por los que pasamos. \bibleverse{18}
Yahvé expulsó de delante de nosotros a todos los pueblos, incluso a los
amorreos que vivían en la tierra. Por eso también nosotros serviremos a
Yahvé, porque él es nuestro Dios''.

\bibleverse{19} Josué dijo al pueblo: ``No podéis servir a Yahvé, porque
es un Dios santo. Es un Dios celoso. No perdonará vuestra desobediencia
ni vuestros pecados. \footnote{\textbf{24:19} Deut 5,26; Éxod 20,5}
\bibleverse{20} Si abandonáis a Yahvé y servís a dioses extranjeros, él
se volverá y os hará el mal, y os consumirá, después de haberos hecho el
bien.''

\bibleverse{21} El pueblo dijo a Josué: ``No, sino que serviremos a
Yahvé''. \bibleverse{22} Josué dijo al pueblo: ``Ustedes son testigos
contra sí mismos de que ustedes mismos han elegido a Yahvé para
servirle.'' Dijeron: ``Somos testigos''.

\bibleverse{23} ``Ahora, pues, dejad los dioses extranjeros que hay
entre vosotros, e inclinad vuestro corazón hacia Yahvé, el Dios de
Israel.'' \footnote{\textbf{24:23} Gén 35,2}

\bibleverse{24} El pueblo dijo a Josué: ``Serviremos a Yavé, nuestro
Dios, y escucharemos su voz''.

\bibleverse{25} Aquel día Josué hizo un pacto con el pueblo, y
estableció para ellos un estatuto y una ordenanza en Siquem. \footnote{\textbf{24:25}
  2Re 23,3} \bibleverse{26} Josué escribió estas palabras en el libro de
la ley de Dios, y tomó una gran piedra y la colocó allí, debajo de la
encina que estaba junto al santuario de Yavé. \footnote{\textbf{24:26}
  Gén 35,4; Jue 9,6} \bibleverse{27} Josué dijo a todo el pueblo:
``Miren, esta piedra será testigo contra nosotros, porque ha escuchado
todas las palabras de Yavé que nos ha dicho. Será, pues, un testigo
contra ustedes, para que no renieguen de su Dios''. \footnote{\textbf{24:27}
  Jos 22,27; Gén 31,48} \bibleverse{28} Entonces Josué despidió al
pueblo, cada uno a su heredad.

\hypertarget{la-muerte-y-el-entierro-de-josuuxe9-entierro-de-los-huesos-de-josuxe9-muerte-de-eleazar}{%
\subsection{La muerte y el entierro de Josué; Entierro de los huesos de
José; Muerte de
Eleazar}\label{la-muerte-y-el-entierro-de-josuuxe9-entierro-de-los-huesos-de-josuxe9-muerte-de-eleazar}}

\bibleverse{29} Después de estas cosas, murió Josué hijo de Nun, siervo
del Señor, siendo de ciento diez años. \bibleverse{30} Lo enterraron en
el límite de su heredad, en Timnat-sera, que está en la región montañosa
de Efraín, al norte de la montaña de Gaas. \footnote{\textbf{24:30} Jos
  19,50} \bibleverse{31} Israel sirvió a Yavé todos los días de Josué, y
todos los días de los ancianos que sobrevivieron a Josué, y había
conocido toda la obra de Yavé, que él había hecho por Israel.
\footnote{\textbf{24:31} Jue 2,7} \bibleverse{32} Los huesos de José,
que los hijos de Israel sacaron de Egipto, los enterraron en Siquem, en
la parcela que Jacob compró a los hijos de Hamor, padre de Siquem, por
cien monedas de plata.\footnote{\textbf{24:32} Hebreo: kesitahs. Una
  kesitah era una especie de moneda de plata.} Pasaron a ser la herencia
de los hijos de José. \footnote{\textbf{24:32} Gén 50,25; Gén 33,19}
\bibleverse{33} Eleazar, hijo de Aarón, murió. Lo enterraron en el monte
de su hijo Finees, que le fue dado en la región montañosa de Efraín.
