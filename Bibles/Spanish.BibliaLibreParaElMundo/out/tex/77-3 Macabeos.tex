\hypertarget{section}{%
\section{1}\label{section}}

\bibleverse{1} Filopáter, al enterarse por los que volvían de que
Antíoco se había hecho dueño de los lugares que le pertenecían, envió
órdenes a toda su infantería y caballería, tomó consigo a su hermana
Arsinoe y marchó hasta las partes de Rafia, donde Antíoco y sus fuerzas
acampaban. \bibleverse{2} Y un tal Teodoto, con la intención de llevar a
cabo su designio, tomó consigo a los más valientes de los hombres
armados que antes le habían sido confiados por Ptolomeo, y llegó de
noche a la tienda de Ptolomeo, para matarlo bajo su propia
responsabilidad, y así poner fin a la guerra. \bibleverse{3} Pero
Dositeo, llamado hijo de Drimulo, judío de nacimiento, después renegado
de las leyes y costumbres de su país, alejó a Ptolomeo e hizo que una
persona desconocida se acostara en su lugar en la tienda. Resultó que
este hombre recibió el destino que estaba previsto para el otro.
\bibleverse{4} Entonces tuvo lugar una feroz batalla. Los hombres de
Antíoco se imponían. Arsinoe subía y bajaba continuamente de las filas
y, con los cabellos revueltos, con lágrimas y súplicas, rogaba a los
soldados que luchasen con valentía por ellos mismos, por sus hijos y por
sus esposas, y les prometía que, si resultaban vencedores, les daría dos
minas de oro a cada uno. \bibleverse{5} Así fue como sus enemigos fueron
derrotados en un encuentro cuerpo a cuerpo y muchos de ellos fueron
hechos prisioneros. \bibleverse{6} Una vez vencido este intento, el rey
decidió entonces dirigirse a las ciudades vecinas y animarlas.
\bibleverse{7} De este modo, y haciendo donaciones a sus templos,
inspiró confianza a sus súbditos.

\bibleverse{8} Los judíos le enviaron a algunos de su consejo y de sus
ancianos. Los saludos, los regalos de bienvenida y las felicitaciones de
antaño que le hicieron, lo llenaron de un mayor deseo de visitar su
ciudad. \bibleverse{9} Después de llegar a Jerusalén, sacrificar y
ofrecer ofrendas de agradecimiento al Dios más grande, y hacer todo lo
que convenía a la santidad del lugar, y entrar en el atrio interior,
\bibleverse{10} quedó tan impresionado por la magnificencia del lugar, y
se asombró tanto de la ordenada disposición del templo, que pensó en
entrar en el propio santuario. \bibleverse{11} Cuando le dijeron que eso
no estaba permitido, que nadie de la nación, ni siquiera los sacerdotes
en general, sino sólo el sumo sacerdote supremo de todos, y él sólo una
vez al año, podía entrar, no quiso de ninguna manera ceder.
\bibleverse{12} Entonces le leyeron la ley, pero él insistió en
entrometerse, exclamando que se le debía permitir. Dijo: ``Aunque se les
privara de este honor, yo no lo haría''. \bibleverse{13} Preguntó por
qué, si había entrado en todos los demás templos, ninguno de los
sacerdotes presentes se lo había prohibido. \bibleverse{14} Alguien le
contestó con todo detalle que había hecho mal en jactarse de ello.
\bibleverse{15} ``Pues bien, ya que he hecho esto'', dijo, ``sea cual
sea la causa, ¿no entraré con o sin vuestro consentimiento?''

\bibleverse{16} Cuando los sacerdotes se postraron con sus ornamentos
sagrados implorando al Dios más grande que viniera a ayudar en el
momento de necesidad y a evitar la violencia del feroz agresor, y cuando
llenaron el templo de lamentos y lágrimas, \bibleverse{17} entonces los
que se habían quedado en la ciudad se asustaron y salieron corriendo,
sin saber qué iba a pasar. \bibleverse{18} Las vírgenes, que habían
estado encerradas en sus habitaciones, salieron con sus madres,
esparciendo polvo y ceniza sobre sus cabezas, y llenando las calles de
gritos. \bibleverse{19} Las mujeres que acababan de prepararse para el
matrimonio salieron de sus cámaras nupciales, abandonaron la reserva que
les correspondía y corrieron desordenadamente por la ciudad.
\bibleverse{20} Los niños recién nacidos fueron abandonados por las
madres o las nodrizas que los atendían, unos por aquí, otros por allá,
en las casas o en los campos; éstos, ahora, con un ardor que no podía
ser frenado, entraban en tropel en el templo del Altísimo.
\bibleverse{21} Los que se reunieron en este lugar ofrecieron diversas
oraciones a causa del impío intento del rey. \bibleverse{22} Junto a
ellos había algunos de los ciudadanos que se animaron y no se sometieron
a su obstinación y a su intención de llevar a cabo su propósito.
\bibleverse{23} Llamando a las armas y a morir con valentía en defensa
de la ley de sus padres, crearon un gran alboroto en el lugar, y con
dificultad fueron devueltos por los ancianos y las ancianas al puesto de
oración que habían ocupado antes. \bibleverse{24} Durante este tiempo,
la multitud siguió orando. \bibleverse{25} Los ancianos que rodeaban al
rey trataron de muchas maneras de desviar su mente arrogante del
designio que había formado. \bibleverse{26} Él, en su ánimo endurecido,
insensible a toda persuasión, seguía adelante con el propósito de llevar
a cabo este designio. \bibleverse{27} Sin embargo, hasta sus propios
oficiales, al ver esto, se unieron a los judíos en una apelación a Aquel
que tiene todo el poder para ayudar en la presente crisis, y no guiñar
el ojo ante tan altanera anarquía. \bibleverse{28} La frecuencia y la
vehemencia del grito de la muchedumbre reunida fue tal, que se produjo
un ruido indescriptible. \bibleverse{29} No sólo los hombres, sino las
mismas paredes y el suelo parecían resonar, pues todo prefería la muerte
antes que ver el lugar profanado.

\hypertarget{section-1}{%
\section{2}\label{section-1}}

\bibleverse{1} Sucedió que el sumo sacerdote Simón dobló las rodillas
cerca del lugar santo, extendió las manos en forma reverente y pronunció
la siguiente oración: \bibleverse{2} ``Señor, Señor, Rey de los cielos y
Gobernante de toda la creación, Santo entre los santos, único
Gobernador, Todopoderoso, préstanos atención a nosotros que estamos
oprimidos por un impío y profano, que celebra en su confianza y fuerza.
\bibleverse{3} Eres tú, el Creador de todo, el Señor del universo, que
eres un Gobernador justo, y juzgas a todos los que actúan con orgullo e
insolencia. \bibleverse{4} Fuiste tú quien destruyó a los antiguos
obreros de la injusticia, entre los que se encontraban los gigantes, que
confiaban en su fuerza y atrevimiento, cubriéndolos con un diluvio sin
medida. \bibleverse{5} Fuiste tú quien hizo de los sodomitas, esos
obreros de la iniquidad extrema, hombres notorios por sus vicios, un
ejemplo para las generaciones posteriores, cuando los cubriste con fuego
y azufre.\footnote{\textbf{2:5} o, azufre} \bibleverse{6} Diste a
conocer tu poder cuando hiciste que el audaz Faraón, el esclavizador de
tu pueblo, pasara por la prueba de muchos y diversos castigos.
\bibleverse{7} Hiciste rodar las profundidades del mar sobre él cuando
lo persiguió con carros y con una multitud de seguidores, y diste un
paso seguro a los que pusieron su confianza en ti, el Señor de toda la
creación. \bibleverse{8} Estos vieron y sintieron las obras de tus
manos, y te alabaron, el Todopoderoso. \bibleverse{9} Tú, oh Rey, cuando
creaste la tierra inconmensurable y sin medida, elegiste esta ciudad.
Hiciste que este lugar fuera sagrado para tu nombre, a pesar de no
necesitar nada. La glorificaste con tu ilustre presencia, después de
construirla para gloria de tu grande y honorable nombre. \bibleverse{10}
Prometiste, por amor al pueblo de Israel, que si nos alejamos de ti, nos
afligimos y luego venimos a esta casa a orar, escucharías nuestra
oración. \bibleverse{11} En verdad eres fiel y verdadero.
\bibleverse{12} Cuando a menudo ayudaste a nuestros padres cuando
estaban presionados y humillados, y los libraste de grandes peligros,
\bibleverse{13} mira ahora, santo Rey, cómo por nuestros muchos y
grandes pecados estamos aplastados y sometidos a nuestros enemigos, y
nos hemos vuelto débiles e impotentes. \bibleverse{14} En nuestra baja
condición, este hombre audaz y profano trata de deshonrar este tu santo
lugar, consagrado desde la tierra al nombre de tu Majestad.
\bibleverse{15} Tu morada, el cielo de los cielos, es ciertamente
inaccesible a los hombres. \bibleverse{16} Pero como te pareció bien
exhibir tu gloria en medio de tu pueblo Israel, santificaste este lugar.
\bibleverse{17} No nos castigues por medio de la impureza de sus
hombres, y no nos castigues por medio de su profanidad, no sea que los
inicuos se jacten en su furia, y se regocijen en su exuberante orgullo
de hablar, y digan: \bibleverse{18} Hemos pisoteado la casa santa, como
se pisotean las casas de los idólatras.' \bibleverse{19} Borra nuestras
iniquidades, elimina nuestros errores y muestra tu compasión en esta
hora. \bibleverse{20} Haz que tus misericordias vayan rápidamente
delante de nosotros. Concédenos la paz, para que los abatidos y los
quebrantados de corazón te alaben con su boca''.

\bibleverse{21} En aquel momento, Dios, que todo lo ve, que está más
allá de todo lo santo entre los santos, escuchó aquella oración, tan
adecuada, y azotó al hombre que estaba muy levantado por el desprecio y
la insolencia. \bibleverse{22} Sacudiéndolo de un lado a otro como se
sacude una caña con el viento, lo arrojó al pavimento, impotente, con
los miembros paralizados, y por un justo juicio privado de la capacidad
de hablar. \bibleverse{23} Sus amigos y guardaespaldas, al ver la rápida
recompensa que le había alcanzado repentinamente, aterrorizados en
extremo, y temiendo que muriera, lo sacaron rápidamente. \bibleverse{24}
Cuando con el tiempo volvió en sí, este severo castigo no provocó en él
ningún arrepentimiento, sino que se marchó con amargas amenazas.

\bibleverse{25} Se dirigió a Egipto, empeoró en su maldad por medio de
sus compañeros de vino antes mencionados, que estaban perdidos de toda
bondad, \bibleverse{26} y no satisfecho con innumerables actos de
impiedad, su audacia aumentó tanto que levantó malas noticias allí, y
muchos de sus amigos, observando atentamente su propósito, se unieron
para favorecer su voluntad. \bibleverse{27} Su propósito era infligir un
estigma público a nuestra raza. Por lo tanto, erigió un pilar de piedra
en el patio, e hizo que se grabara en él la siguiente inscripción:
\bibleverse{28} ``Se debe negar la entrada a este templo a todos los que
no quieran sacrificar. Todos los judíos debían ser registrados entre los
esclavos. Aquellos que se resistieran debían ser apresados por la fuerza
y condenados a muerte. \bibleverse{29} Aquellos que sean registrados de
esta manera serán marcados en sus personas con el símbolo de la hoja de
hiedra de Dionisio, y serán reducidos a estos derechos limitados''.
\bibleverse{30} Para que no pareciera que los odiaba a todos, mandó
escribir debajo que si alguno de ellos decidía entrar en la comunidad de
los iniciados en los ritos, éstos tendrían los mismos derechos que los
alejandrinos.

\bibleverse{31} Algunos de los que estaban sobre la ciudad, por lo
tanto, aborreciendo cualquier acercamiento a la ciudad de la piedad, se
entregaron sin vacilar al rey, y esperaban obtener algún gran honor de
una futura conexión con él. \bibleverse{32} Un espíritu más noble, sin
embargo, impulsó a la mayoría a aferrarse a sus observancias religiosas,
y pagando dinero para poder vivir sin ser molestados, éstos trataron de
escapar del registro, \bibleverse{33} esperando alegremente la ayuda
futura, aborrecieron a sus propios apóstatas, considerándolos como
enemigos nacionales, y privándolos de la comunión y la ayuda mutua.

\hypertarget{section-2}{%
\section{3}\label{section-2}}

\bibleverse{1} Al descubrir esto, el malvado rey se enfureció tanto que
ya no limitó su ira a los judíos de Alejandría. Poniendo la mano más
dura sobre los que vivían en el campo, dio órdenes de que se les
reuniera rápidamente en un lugar, y se les privara de la vida con la
mayor crueldad. \bibleverse{2} Mientras esto ocurría, se difundió un
rumor hostil por parte de hombres que se habían unido para perjudicar a
la raza judía. El pretexto de su acusación era que los judíos los
alejaban de las ordenanzas de la ley. \bibleverse{3} Ahora bien, los
judíos siempre mantuvieron un sentimiento de lealtad inquebrantable
hacia los reyes, \bibleverse{4} sin embargo, como adoraban a Dios y
observaban su ley, hacían ciertas distinciones y evitaban ciertas cosas.
De ahí que parecieran odiosos a algunas personas, \bibleverse{5} aunque,
como adornaban su conversación con obras de justicia, se habían
establecido en la buena opinión del mundo. \bibleverse{6} Sin embargo,
los extranjeros hacían caso omiso de lo que decía el resto de la
humanidad, \bibleverse{7} y hablaban mucho de la exclusividad de los
judíos con respecto a su culto y sus comidas. Alegaron que eran hombres
insociables, hostiles a los intereses del rey, negándose a asociarse con
él o con sus tropas. Con esta forma de hablar, atrajeron sobre ellos
mucho odio. \bibleverse{8} Este inesperado alboroto y la repentina
reunión de gente fue observada por los griegos que vivían en la ciudad,
en relación con hombres que nunca les habían hecho daño. Sin embargo, no
estaba en su mano ayudarles, ya que todo era opresión alrededor, pero
les animaron en sus problemas, y esperaron un giro favorable de los
acontecimientos. \bibleverse{9} El que lo sabe todo no se desentenderá,
decían, de un pueblo tan grande. \bibleverse{10} Algunos vecinos, amigos
y socios comerciales de los judíos llegaron a convocarlos en secreto a
una entrevista, les prometieron su ayuda y se comprometieron a hacer
todo lo posible por ellos.

\bibleverse{11} Ahora bien, el rey, eufórico por su próspera fortuna, y
sin considerar el poder superior de Dios, sino pensando en perseverar en
su actual propósito, escribió la siguiente carta al prejuicio de los
judíos: \bibleverse{12} ``Rey Ptolomeo Filopáter, a los comandantes y
soldados de Egipto y de todos los lugares, ¡salud y felicidad!
\bibleverse{13} Me va bien, y también mis asuntos. \bibleverse{14} Desde
que nuestra campaña asiática, cuyos pormenores conoces, y que por la
ayuda de los dioses, no concedida a la ligera, y por nuestro propio
vigor, ha sido llevada a buen término según nuestras expectativas,
\bibleverse{15} decidimos, no con fuerza de lanza, sino con dulzura y
mucha humanidad, por así decirlo, atender a los habitantes de
Coele-Siria y Fenicia, y ser sus voluntariosos benefactores.
\bibleverse{16} Así pues, después de haber repartido considerables sumas
de dinero en los templos de las distintas ciudades, nos dirigimos hasta
Jerusalén y subimos a honrar el templo de esos miserables que no cesan
en su locura. \bibleverse{17} En apariencia nos recibieron de buen
grado, pero desmintieron esa apariencia con sus actos. Cuando estábamos
ansiosos por entrar en su templo y honrarlo con los más bellos y
exquisitos regalos, \bibleverse{18} se dejaron llevar por su antigua
arrogancia hasta el punto de prohibirnos la entrada, mientras que
nosotros, por nuestra tolerancia hacia todos los hombres, nos abstuvimos
de ejercer nuestro poder sobre ellos. \bibleverse{19} Así, exhibiendo su
enemistad contra nosotros, son los únicos entre las naciones que
levantan la cabeza contra reyes y benefactores, como hombres no
dispuestos a someterse a nada razonable. \bibleverse{20} Nosotros, pues,
habiéndonos esforzado por tener en cuenta la locura de estas gentes, y a
nuestro regreso victorioso tratando cortésmente a todo el pueblo de
Egipto, actuamos de forma adecuada. \bibleverse{21} En consecuencia, no
guardando ninguna mala voluntad contra sus parientes, sino más bien
recordando nuestra conexión con ellos, y los numerosos asuntos con
corazón sincero desde un período remoto confiados a ellos, quisimos
aventurar una alteración total de su estado, dándoles los derechos de
ciudadanos de Alejandría, y admitirlos a los ritos eternos de nuestras
solemnidades. \bibleverse{22} Todo esto, sin embargo, lo han tomado con
un espíritu muy diferente. Con su malignidad innata, han despreciado la
oferta justa, e inclinándose constantemente hacia el mal,
\bibleverse{23} han rechazado los derechos inestimables. No sólo eso,
sino que mediante el uso de la palabra, y absteniéndose de hablar,
aborrecen a los pocos de entre ellos que están dispuestos de corazón
hacia nosotros, considerando siempre que su infame forma de vida nos
obligará a prescindir de nuestra reforma. \bibleverse{24} Habiendo
recibido, pues, ciertas pruebas de que estos judíos nos guardan toda
clase de mala voluntad, debemos esperar la posibilidad de que se
produzca algún tumulto repentino entre nosotros cuando estos impíos se
conviertan en traidores y bárbaros enemigos. \bibleverse{25} Por lo
tanto, tan pronto como el contenido de esta carta sea conocido por
vosotros, en esa misma hora ordenamos que esos judíos que habitan entre
vosotros, con esposas e hijos, sean enviados a nosotros, vilipendiados y
maltratados, con cadenas de hierro, para que sufran una muerte cruel y
vergonzosa, adecuada a los enemigos. \bibleverse{26} Porque con el
castigo de ellos en un solo cuerpo percibimos que hemos encontrado el
único medio de establecer nuestros asuntos para el futuro sobre una base
firme y satisfactoria. \bibleverse{27} Quien proteja a un judío, ya sea
anciano, niño o lactante, será torturado con toda su casa hasta la
muerte. \bibleverse{28} Quien informe contra los judíos, además de
recibir los bienes de la persona acusada, será obsequiado con dos mil
dracmas\footnote{\textbf{3:28} El siríaco.} del tesoro real, será puesto
en libertad y será coronado. \bibleverse{29} Cualquier lugar que acoja a
un judío se convertirá en inaccesible y será puesto bajo la prohibición
del fuego, y quedará inutilizado para todo ser viviente por todos los
tiempos.'' \bibleverse{30} La carta del rey fue escrita en la forma
anterior.

\hypertarget{section-3}{%
\section{4}\label{section-3}}

\bibleverse{1} Dondequiera que se recibiera este decreto, el pueblo
mantenía un jolgorio de alegría y gritos, como si su odio, largamente
reprimido y endurecido, se manifestara ahora abiertamente.
\bibleverse{2} Los judíos sufrían grandes penas y lloraban mucho,
mientras sus corazones, lamentándose de todo lo que les rodeaba, se
encendían al lamentar la repentina destrucción que se había decretado
contra ellos. \bibleverse{3} ¿Qué casa, o ciudad, o lugar habitado, o
qué calles había, que su estado no llenara de lamentos y lamentaciones?
\bibleverse{4} Fueron enviados unánimemente por los generales de varias
ciudades, con un sentimiento tan severo y despiadado que la
excepcionalidad de la inflicción conmovió incluso a algunos de sus
enemigos. Estos, influenciados por sentimientos de común humanidad, y
reflexionando sobre el incierto asunto de la vida, derramaron lágrimas
ante su miserable expulsión. \bibleverse{5} Una multitud de ancianos de
pelo canoso eran conducidos con los pies doblados y vacilantes, urgidos
por el impulso de una fuerza violenta y desvergonzada a una rápida
velocidad. \bibleverse{6} Las muchachas que habían entrado en la cámara
nupcial hacía poco tiempo, para disfrutar de la asociación del
matrimonio, cambiaron el placer por la miseria; y con el polvo esparcido
sobre sus cabezas ungidas por la mirra, fueron apresuradas a lo largo de
la marcha, sin que se les descubriera nada; y, en medio de extraños
insultos, lanzaron de común acuerdo un grito lamentable en lugar del
himno matrimonial. \bibleverse{7} Atadas y expuestas a las miradas del
público, fueron apresuradas violentamente a bordo del barco.
\bibleverse{8} Los maridos de éstas, en la plenitud de su vigor juvenil,
en lugar de coronas, llevaban sogas al cuello. En lugar de festejos y
celebraciones juveniles, pasaban el resto de sus días nupciales
lamentándose, y sólo veían la tumba a la mano. \bibleverse{9} Eran
arrastrados por cadenas inflexibles, como animales salvajes. De ellos,
algunos tenían el cuello clavado en los bancos de los remeros, mientras
que los pies de otros estaban encerrados en duros grilletes.
\bibleverse{10} Los tablones de la cubierta sobre ellos bloqueaban la
luz y cerraban el día por todos lados, para que fueran tratados como
traidores durante todo el viaje.

\bibleverse{11} Fueron transportados así en esta nave, y al final de la
misma llegaron a Schedia. El rey había ordenado que los arrojaran en el
vasto hipódromo que se había construido frente a la ciudad. Este lugar
estaba bien adaptado por su situación para exponerlos a la mirada de
todos los que entraban en la ciudad, y de los que iban de la ciudad al
campo. Así, no podían mantener ninguna comunicación con sus fuerzas. No
fueron considerados dignos de ningún alojamiento civilizado.
\bibleverse{12} Cuando se hizo esto, el rey, al oír que sus parientes en
la ciudad salían a menudo y se lamentaban de la melancólica angustia de
estas víctimas, \bibleverse{13} se llenó de rabia, y ordenó que se les
sometiera cuidadosamente al mismo --- y ni un poco más suave ---
tratamiento. \bibleverse{14} Toda la nación debía ser registrada. Cada
individuo debía ser especificado por su nombre, no para esa dura
servidumbre de trabajo que hemos mencionado un poco antes, sino para
poder exponerlos a las torturas antes mencionadas; y finalmente, en el
corto espacio de un día, podría exterminarlos con sus crueldades.
\bibleverse{15} El registro de estos hombres se llevó a cabo con
crueldad, celo y asiduidad, desde la salida del sol hasta su puesta, y
no se terminó en cuarenta días. \bibleverse{16} El rey se llenaba de
gran y constante alegría, y celebraba banquetes ante los ídolos del
templo. Su corazón descarriado, alejado de la verdad, y su boca profana
daban gloria a los ídolos, sorda e incapaz de hablar o de ayudar, y
pronunciaba un discurso indigno contra el Dios más grande.
\bibleverse{17} Al final del mencionado intervalo de tiempo, los
encargados del registro llevaron la noticia al rey de que la multitud de
los judíos era demasiado grande para el registro, \bibleverse{18} ya que
aún quedaban muchos en la tierra, de los cuales algunos estaban en casas
habitadas y otros estaban dispersos en diversos lugares, de modo que
todos los encargados en Egipto eran insuficientes para el trabajo.
\bibleverse{19} El rey los amenazó y los acusó de haber aceptado
sobornos para tramitar la fuga de los judíos, pero se convenció
claramente de la verdad de lo dicho. \bibleverse{20} Dijeron, y lo
demostraron, que el papel y las plumas les habían fallado para llevar a
cabo su propósito. \bibleverse{21} Ahora bien, esto fue una activa
interferencia de la inconquistable Providencia que asistió a los judíos
desde el cielo.

\hypertarget{section-4}{%
\section{5}\label{section-4}}

\bibleverse{1} Entonces llamó a Hermón, que estaba a cargo de los
elefantes. Lleno de rabia, totalmente fijado en su furioso designio,
\bibleverse{2} le ordenó que, con una cantidad de vino sin mezclar con
puñados de incienso infundido, drogase a los elefantes a primera hora
del día siguiente. Estos quinientos elefantes, enfurecidos por las
copiosas bebidas de incienso, debían ser conducidos a la ejecución de la
muerte sobre los judíos. \bibleverse{3} El rey, después de dar estas
órdenes, se dirigió a su banquete y reunió a todos aquellos de sus
amigos y del ejército que más odiaban a los judíos. \bibleverse{4} El
jefe de los elefantes, Hermón, cumplió puntualmente su encargo.
\bibleverse{5} Los criados designados al efecto salieron al anochecer y
ataron las manos de las miserables víctimas, y tomaron otras
precauciones para su seguridad durante la noche, pensando que toda la
raza perecería junta. \bibleverse{6} Los paganos creían que los judíos
estaban desprovistos de toda protección, pues las cadenas los ataban.
\bibleverse{7} Invocaron al Señor Todopoderoso, e imploraron
incesantemente con lágrimas a su Dios y Padre misericordioso, Gobernante
de todo, Señor de todo poder, \bibleverse{8} que derribara el mal
propósito que había salido contra ellos, y que los librara mediante una
manifestación extraordinaria de esa muerte que les estaba reservada.
\bibleverse{9} Su ferviente súplica subió al cielo. \bibleverse{10}
Entonces Hermón, que había llenado a sus despiadados elefantes con
copiosas bebidas de vino mezclado con incienso, llegó temprano al
palacio para informar sobre estos preparativos. \bibleverse{11} Pero él,
que desde siempre ha enviado su buen sueño de criatura de noche o de día
gratificando así a quien quiere, difundió ahora una porción de él sobre
el rey. \bibleverse{12} Por este dulce y profundo influjo del Señor, fue
retenido, y así su injusto propósito quedó bastante frustrado, y su
inquebrantable resolución, muy falseada. \bibleverse{13} Pero los
judíos, habiendo escapado a la hora fijada, alabaron a su santo Dios, y
volvieron a rogarle a aquel que se reconcilia fácilmente que desplegara
el poder de su poderosa mano ante los arrogantes gentiles.
\bibleverse{14} Casi había llegado la mitad de la hora décima, cuando el
que había enviado las invitaciones, al ver presentes a los invitados, se
acercó y sacudió al rey. \bibleverse{15} Ganó su atención con
dificultad, e insinuando que la hora de la comida estaba pasando, habló
con él del asunto. \bibleverse{16} El rey lo escuchó y, apartándose para
beber, ordenó a los invitados que se sentaran ante él. \bibleverse{17}
Hecho esto, les pidió que se divirtieran y se entregaran a la alegría a
esta hora tan tardía del banquete. \bibleverse{18} La conversación se
prolongó, y el rey mandó llamar a Hermón y le preguntó, con feroces
denuncias, por qué se había permitido a los judíos sobrevivir aquel día.
\bibleverse{19} Hermón le explicó que había hecho su voluntad durante la
noche, y en esto fue confirmado por sus amigos. \bibleverse{20} El rey,
entonces, con una barbaridad que superaba a la de Falaris, dijo:
``Podrían agradecer su sueño de ese día. No pierdas tiempo y prepara los
elefantes contra mañana, como lo hiciste antes, para la destrucción de
estos malditos judíos.'' \bibleverse{21} Cuando el rey dijo esto, los
presentes se alegraron y lo aprobaron. Entonces cada uno se fue a su
casa. \bibleverse{22} No emplearon la noche en dormir, sino en urdir
crueles burlas para los considerados miserables. \bibleverse{23} El
gallo de la mañana acababa de cantar, y Hermón, habiendo enjaezado a los
brutos, los estimulaba en la gran columnata. \bibleverse{24} La
muchedumbre de la ciudad se reunía para ver el espantoso espectáculo, y
esperaba con impaciencia el amanecer. \bibleverse{25} Los judíos, sin
aliento por el suspenso momentáneo, extendían las manos y rogaban al
Dios más grande, con afligidos lamentos, que los ayudara pronto.
\bibleverse{26} Los rayos del sol aún no brillaban y el rey esperaba a
sus amigos cuando Hermón se acercó a él, llamándole y diciéndole que sus
deseos podían realizarse ahora. \bibleverse{27} El rey, al recibirlo, se
asombró de su insólita invitación. Abrumado por un espíritu de olvido de
todo, preguntó por el objeto de esta ferviente preparación.
\bibleverse{28} Pero esto era obra de aquel Dios todopoderoso que le
había hecho olvidar todo su propósito. \bibleverse{29} Hermón y todos
sus amigos le señalaron la preparación de los animales. Están listos, oh
rey, según tu propia y estricta orden. \bibleverse{30} El rey se llenó
de feroz cólera ante estas palabras, pues, por la Providencia de Dios
respecto a estas cosas, su mente se había confundido por completo. Miró
con dureza a Hermón, y lo amenazó de la siguiente manera \bibleverse{31}
``Tus padres o tus hijos, si estuvieran aquí, habrían dado una gran
comida a estos animales salvajes, no a estos judíos inocentes, que me
han servido lealmente a mí y a mis antepasados. \bibleverse{32} Si no
fuera por la amistad familiar y por las exigencias de tu cargo, tu vida
habría ido a parar a la de ellos.''

\bibleverse{33} Hermón, al verse amenazado de esta manera tan inesperada
y alarmante, se turbó en sus ojos, y su rostro cayó. \bibleverse{34}
También los amigos salieron uno por uno y despidieron a las multitudes
reunidas a sus respectivas ocupaciones. \bibleverse{35} Los judíos, al
enterarse de estos acontecimientos, alabaron al glorioso Dios y Rey de
reyes, porque también habían obtenido de él esta ayuda. \bibleverse{36}
El rey organizó otro banquete de la misma manera, y proclamó una
invitación a la alegría. \bibleverse{37} Llamó a Hermón a su presencia y
le dijo con amenazas: ``¿Cuántas veces, oh desgraciado, he de repetirte
mis órdenes sobre estas mismas personas? \bibleverse{38} ¡Una vez más,
arma los elefantes para el exterminio de los judíos mañana!''
\bibleverse{39} Sus parientes, que estaban reclinados con él, se
asombraron de su inestabilidad, y se expresaron así \bibleverse{40} ``Oh
rey, ¿hasta cuándo nos pones a prueba, como a los hombres privados de
razón? Es la tercera vez que ordenas su destrucción. Cuando la cosa está
por hacer, cambias de opinión y recuerdas tus instrucciones.
\bibleverse{41} Por eso, la expectación provoca un tumulto en la ciudad.
Se llena de facciones, y está continuamente a punto de ser saqueada''.

\bibleverse{42} El rey, al igual que otro Falaris, presa de la
irreflexión, no tuvo en cuenta los cambios que había sufrido su propia
mente, que se tradujeron en la liberación de los judíos. Hizo un
juramento infructuoso, y determinó enviarlos inmediatamente al hades,
aplastados por las rodillas y los pies de los elefantes. \bibleverse{43}
También invadiría Judea, arrasaría sus ciudades con el fuego y la
espada, destruiría el templo en el que los paganos no podían entrar e
impediría que se ofrecieran sacrificios en él. \bibleverse{44}
Alegremente sus amigos se separaron, junto con sus parientes; y,
confiando en su determinación, dispusieron sus fuerzas en guardia en los
lugares más convenientes de la ciudad. \bibleverse{45} El dueño de los
elefantes incitó a los animales a un estado casi maníaco, los empapó de
incienso y vino, y los engalanó con espantosos dispositivos.
\bibleverse{46} Hacia la madrugada, cuando la ciudad estaba llena de un
inmenso número de personas en el hipódromo, entró en el palacio y llamó
al rey para que se ocupara del asunto. \bibleverse{47} El corazón del
rey bullía de impía rabia, y salió corriendo con la masa, junto con los
elefantes. Con sentimientos insensibles y ojos despiadados, anhelaba
contemplar la dura y miserable condena de los judíos antes mencionados.
\bibleverse{48} Pero los judíos, cuando los elefantes salieron por la
puerta, seguidos por la fuerza armada. Al ver la polvareda levantada por
la muchedumbre, y al oír los fuertes gritos de la misma, \bibleverse{49}
pensaron que habían llegado al último momento de sus vidas, al final de
lo que temblorosamente habían esperado. Por ello, se entregaron a los
lamentos y a los gemidos. Se besaron unos a otros. Los parientes más
cercanos se echaron al cuello unos a otros: los padres abrazando a sus
hijos y las madres a sus hijas. Otras mujeres sostenían a sus bebés
contra sus pechos, que extraían lo que parecía su última leche.
\bibleverse{50} Sin embargo, cuando reflexionaron sobre la ayuda que se
les había concedido anteriormente desde el cielo, se postraron
unánimemente, retiraron de los pechos incluso a los niños que mamaban, y
\bibleverse{51} lanzaron un grito extremadamente grande pidiendo al
Señor de todo poder que se revelara y tuviera piedad de los que ahora
yacían a las puertas del hades.

\hypertarget{section-5}{%
\section{6}\label{section-5}}

\bibleverse{1} Entonces Eleazar, ilustre sacerdote del país, que había
alcanzado la duración de sus días y cuya vida había sido adornada con
virtud, hizo que los ancianos que lo rodeaban dejaran de clamar al Dios
santo, y rezó lo siguiente \bibleverse{2} ``Oh rey, poderoso en poder,
altísimo, Dios todopoderoso, que regulas toda la creación con tu tierna
misericordia, \bibleverse{3} mira a la descendencia de Abraham, a los
hijos del santificado Jacob, tu santificada herencia, oh Padre, que
ahora son destruidos injustamente como extranjeros en una tierra
extranjera. \bibleverse{4} Tú destruiste al Faraón con su ejército de
carros cuando ese señor de este mismo Egipto se alzó con una osadía sin
ley y una lengua estridente. Derramando los rayos de tu misericordia
sobre la raza de Israel, lo abrumaste a él y a su orgulloso ejército.
\bibleverse{5} Cuando Senaquerim, el rey de los asirios, exultante con
su innumerable ejército, había sometido a toda la tierra con su lanza y
se alzaba contra tu ciudad sagrada con jactancias insoportables, tú,
Señor, lo derribaste y mostraste tu poderío a muchas naciones.
\bibleverse{6} Cuando los tres amigos en la tierra de Babilonia, por su
propia voluntad, expusieron sus vidas al fuego antes que servir a las
cosas vanas, tú enviaste un húmedo frescor a través del horno de fuego,
e hiciste caer el fuego sobre todos sus adversarios. \bibleverse{7}
Fuiste tú quien, cuando Daniel fue arrojado, por la calumnia y la
envidia, como presa de los leones de abajo, lo devolviste ileso a la
luz. \bibleverse{8} Cuando Jonás se consumía en el vientre del monstruo
marino, tú lo miraste, oh Padre, y lo recuperaste a la vista de los
suyos. \bibleverse{9} Ahora, tú que odias la insolencia, tú que abundas
en misericordia, tú que eres el protector de todas las cosas, muéstrate
pronto a los de la raza de Israel, que son insultados por gentiles
aborrecidos y sin ley. \bibleverse{10} Si nuestra vida durante el
destierro se ha manchado de iniquidad, líbranos de la mano del enemigo y
destrúyenos, Señor, con la muerte que prefieras. \bibleverse{11} No
permitas que los vanidosos feliciten a los ídolos vanos por la
destrucción de tus amados, diciendo: ``Su dios no los libró''.
\bibleverse{12} Tú que eres todopoderoso y omnipotente, oh Eterno,
¡mira! Ten piedad de nosotros que estamos siendo retirados de la vida,
como traidores, por la insolencia irracional de los hombres sin ley.
\bibleverse{13} Deja que los paganos se postren hoy ante tu invencible
poder, oh glorioso, que tienes todo el poder para salvar a la raza de
Jacob. \bibleverse{14} Todo el grupo de infantes y sus padres te piden
con lágrimas. \bibleverse{15} Que se muestre a todas las naciones que
estás con nosotros, Señor, y que no has apartado tu rostro de nosotros,
sino que, como dijiste que no los olvidarías ni siquiera en la tierra de
sus enemigos, cumple este dicho, Señor.''

\bibleverse{16} En el momento en que Eleazar había terminado su oración,
el rey se acercó al hipódromo con los animales salvajes y con su fuerza
tumultuosa. \bibleverse{17} Al ver esto, los judíos lanzaron un fuerte
grito al cielo, de modo que los valles adyacentes resonaron y provocaron
un lamento irreprimible en todo el ejército. \bibleverse{18} Entonces el
Dios todoglorioso, todopoderoso y verdadero, mostró su santo semblante y
abrió las puertas del cielo, de las que descendieron dos ángeles, de
espantosa forma, que fueron visibles para todos, excepto para los
judíos. \bibleverse{19} Se colocaron enfrente y llenaron de confusión y
cobardía al ejército de los enemigos, y los ataron con grilletes
inamovibles. \bibleverse{20} Un frío escalofrío se apoderó de la persona
del rey, y el olvido paralizó la vehemencia de su espíritu.
\bibleverse{21} Hicieron retroceder a los animales sobre las fuerzas
armadas que los seguían, y los animales los pisotearon y los
destruyeron. \bibleverse{22} La ira del rey se convirtió en compasión, y
lloró por lo que había ideado. \bibleverse{23} Porque al oír el clamor y
verlos a todos al borde de la destrucción, con lágrimas amenazó
airadamente a sus amigos, diciendo: \bibleverse{24} ``Habéis gobernado
mal y habéis superado a los tiranos en crueldad. Habéis trabajado para
privarme a mí, vuestro benefactor, a la vez de mi dominio y de mi vida,
ideando en secreto medidas perjudiciales para el reino. \bibleverse{25}
¿Quién ha reunido aquí, apartando injustificadamente a cada uno de su
casa, a los que, por fidelidad a nosotros, habían mantenido las
fortalezas del país? \bibleverse{26} ¿Quién ha consignado a castigos
inmerecidos a los que en su buena voluntad hacia nosotros desde el
principio han superado en todo a todas las naciones, y que a menudo se
han comprometido en las empresas más peligrosas? \bibleverse{27}
¡Suelta, suelta las ataduras injustas! Enviadlos a sus casas en paz,
pidiendo perdón por lo que se ha hecho. \bibleverse{28} Soltad a los
hijos del todopoderoso Dios vivo del cielo, que desde los tiempos de
nuestros antepasados hasta ahora ha concedido una gloriosa e
ininterrumpida prosperidad a nuestros asuntos.'' \bibleverse{29} Dijo
estas cosas, y ellos, liberados en el mismo momento, habiendo escapado
ya de la muerte, alabaron a Dios su santo Salvador.

\bibleverse{30} El rey se dirigió entonces a la ciudad, llamó a su
financiero y le pidió que proporcionara una cantidad de vino y otros
materiales para el banquete de los judíos para siete días. Decidió que
debían celebrar una alegre fiesta de liberación en el mismo lugar en el
que esperaban encontrar su destrucción. \bibleverse{31} Entonces los que
antes eran despreciados y estaban cerca del hades, sí, más bien
avanzaban hacia él, participaron de la copa de la salvación, en lugar de
una muerte penosa y lamentable. Llenos de júbilo, convirtieron el lugar
destinado a su caída y sepultura en cabinas de banquetes.
\bibleverse{32} Dejando de lado su miserable tensión de aflicción,
retomaron el tema de su patria, cantando en alabanza a Dios su
maravilloso Salvador. Dejaron de lado todos los gemidos y todos los
lamentos. Formaron danzas en señal de pacífica alegría. \bibleverse{33}
También el rey reunió a varios invitados para la ocasión, y agradeció
sin cesar con mucha magnificencia la inesperada liberación que se le
había concedido. \bibleverse{34} Los que los habían señalado como para
la muerte y para la carroña, y los habían registrado con alegría,
aullaron en voz alta, y fueron revestidos de vergüenza, y se les apagó
el fuego de su rabia con ignominia. \bibleverse{35} Pero los judíos,
como acabamos de decir, instituyeron una danza, y luego se entregaron a
la fiesta, a la acción de gracias y a los salmos. \bibleverse{36}
Hicieron una ordenanza pública para conmemorar estas cosas en las
generaciones venideras, mientras fueran residentes. Así establecieron
estos días como días de alegría, no con el propósito de beber o de
lujos, sino porque Dios los había salvado. \bibleverse{37} Pidieron al
rey que los enviara de vuelta a sus hogares. \bibleverse{38} Fueron
enrolados desde el veinticinco de Pachón hasta el cuatro de Epiphi, un
período de cuarenta días. Las medidas tomadas para su destrucción
duraron desde el quinto de Epiphi hasta el séptimo, es decir, tres días.
\bibleverse{39} Durante este tiempo, el soberano de todos manifestó
gloriosamente su misericordia y los liberó a todos juntos sin daño
alguno. \bibleverse{40} Hasta el decimocuarto día, los hombres se
alimentaron con las provisiones del rey, y luego pidieron que se les
despidiera. \bibleverse{41} El rey los elogió y escribió la siguiente
carta, de magnánima importancia para ellos, a los comandantes de cada
ciudad:

\hypertarget{section-6}{%
\section{7}\label{section-6}}

\bibleverse{1} ``Rey Ptolomeo Filopator a los comandantes de todo
Egipto, y a todos los que están al frente de los asuntos, alegría y
fuerza. \bibleverse{2} También nosotros y nuestros hijos estamos bien.
Dios ha dirigido nuestros asuntos como deseamos. \bibleverse{3} Algunos
de nuestros amigos, por malicia, nos instaron con vehemencia a castigar
a los judíos de nuestro reino en masa, con la imposición de un castigo
monstruoso. \bibleverse{4} Pretendían que nuestros asuntos nunca
estarían en buen estado hasta que esto tuviera lugar. Tal era, decían,
el odio que los judíos profesaban a todos los demás pueblos.
\bibleverse{5} Los trajeron encadenados como esclavos, no, como
traidores. Sin indagar ni examinar, se esforzaron por aniquilarlos. Se
abroquelaron con una crueldad salvaje, peor que la costumbre escita.
\bibleverse{6} Por esta causa los amenazamos severamente; sin embargo,
con la clemencia que solemos tener con todos los hombres, al final les
permitimos vivir. Al comprobar que el Dios del cielo arrojó un escudo de
protección sobre los judíos para preservarlos, y que luchó por ellos
como un padre lucha siempre por sus hijos, \bibleverse{7} y teniendo en
cuenta su constancia y fidelidad hacia nosotros y hacia nuestros
antepasados, los hemos absuelto, como es debido, de toda clase de
cargos. \bibleverse{8} Los hemos despedido a sus diferentes hogares,
diciendo a todos los hombres en todas partes que no les hagan ningún
mal, ni los injurien injustamente sobre el pasado. \bibleverse{9} Porque
sabed que si concebimos algún mal designio, o los agraviamos de alguna
manera, tendremos siempre como adversario, no al hombre, sino al Dios
supremo, el gobernante de todo poder. De Él no habrá escapatoria, como
vengador de tales hechos. Adiós''.

\bibleverse{10} Cuando recibieron esta carta, no se apresuraron a partir
inmediatamente. Pidieron al rey que se les permitiera infligir un
castigo adecuado a los de su raza que habían transgredido
voluntariamente al dios santo y a la ley de Dios. \bibleverse{11}
Alegaron que los hombres que habían transgredido por su vientre las
ordenanzas de Dios, nunca serían fieles a los intereses del rey.
\bibleverse{12} El rey admitió la verdad de este razonamiento y los
elogió. Se les dio pleno poder, sin orden ni comisión especial, para
destruir a los que habían transgredido la ley de Dios audazmente en
todas las partes de los dominios del rey. \bibleverse{13} Sus
sacerdotes, entonces, como correspondía, lo saludaron con buenos deseos,
y todo el pueblo resonó con el ``¡Aleluya!'' Luego partieron
alegremente. \bibleverse{14} Entonces castigaron y destruyeron
vergonzosamente a todo judío contaminado que caía en su camino,
\bibleverse{15} matando así, en aquel día, a más de trescientos hombres,
y estimando esta destrucción de los impíos como una temporada de
alegría. \bibleverse{16} Ellos mismos, habiéndose aferrado a su Dios
hasta la muerte, y habiendo gozado de una plena liberación, partieron de
la ciudad adornados con coronas de flores dulces de todo tipo.
Pronunciando exclamaciones de alegría, con cantos de alabanza e himnos
melodiosos, dieron gracias al Dios de sus padres, el eterno Salvador de
Israel. \bibleverse{17} Habiendo llegado a Tolemaida, llamada por la
especialidad de ese distrito ``Rosaleda'', donde la flota, de acuerdo
con el deseo general, los esperó siete días, \bibleverse{18}
participaron de un banquete de liberación, pues el rey les concedió
generosamente todos los medios para asegurar el regreso a casa.
\bibleverse{19} Por lo tanto, fueron llevados en paz, mientras daban las
gracias correspondientes, y decidieron observar estos días durante su
estancia como días de alegría. \bibleverse{20} Estos días los
inscribieron como sagrados en una columna, después de haber dedicado el
lugar de su fiesta a la oración. Partieron ilesos, libres, llenos de
alegría, preservados por la orden del rey, por tierra, por mar y por
río, cada uno a su casa. \bibleverse{21} Tenían más peso que antes entre
sus enemigos, y eran honrados y temidos. Nadie les robó sus bienes de
ninguna manera. \bibleverse{22} Cada uno recibió lo suyo, según el
inventario, los que habían obtenido sus bienes, entregándolos con el
mayor terror. Porque el Dios más grande hizo maravillas perfectas para
su salvación. \bibleverse{23} ¡Bendito sea el Redentor de Israel para
siempre! Amén.
