\hypertarget{nombre-del-remitente-y-destinatario-de-la-carta-y-bendiciuxf3n-apostuxf3lica-a-la-congregaciuxf3n}{%
\subsection{Nombre del remitente y destinatario de la carta y bendición
apostólica a la
congregación}\label{nombre-del-remitente-y-destinatario-de-la-carta-y-bendiciuxf3n-apostuxf3lica-a-la-congregaciuxf3n}}

\hypertarget{section}{%
\section{1}\label{section}}

\bibleverse{1} Pablo, siervo de Jesucristo, llamado a ser apóstol,
apartado para la Buena Nueva de Dios, \footnote{\textbf{1:1} Hech 9,15;
  Hech 13,2; Gal 1,15} \bibleverse{2} que él prometió antes por medio de
sus profetas en las santas Escrituras, \footnote{\textbf{1:2} Rom
  14,24-25; Tit 1,2; Luc 1,70} \bibleverse{3} acerca de su Hijo, que
nació de la descendencia de David según la carne, \footnote{\textbf{1:3}
  2Sam 7,12; Mat 22,42; Rom 9,5} \bibleverse{4} que fue declarado Hijo
de Dios con poder según el Espíritu de santidad, por la resurrección de
entre los muertos, Jesucristo nuestro Señor, \footnote{\textbf{1:4} Hech
  13,33; Mat 28,18} \bibleverse{5} por quien recibimos la gracia y el
apostolado para la obediencia de la fe entre todas las naciones por
causa de su nombre; \footnote{\textbf{1:5} Rom 15,18; Gal 2,7; Gal 2,9;
  Hech 26,16-18} \bibleverse{6} entre los cuales también estáis llamados
a pertenecer a Jesucristo; \bibleverse{7} a todos los que están en Roma,
amados de Dios, llamados a ser santos: Gracia a vosotros y paz de parte
de Dios nuestro Padre y del Señor Jesucristo. \footnote{\textbf{1:7}
  1Cor 1,2; 2Cor 1,1-2; Efes 1,1; Núm 6,24-26}

\hypertarget{acciuxf3n-de-gracias-del-apuxf3stol-a-dios-por-el-estado-de-fe-de-la-comunidad-y-expresiuxf3n-del-deseo-de-poder-predicar-el-mensaje-de-salvaciuxf3n-tambiuxe9n-en-roma}{%
\subsection{Acción de gracias del Apóstol a Dios por el estado de fe de
la comunidad y expresión del deseo de poder predicar el mensaje de
salvación también en
Roma}\label{acciuxf3n-de-gracias-del-apuxf3stol-a-dios-por-el-estado-de-fe-de-la-comunidad-y-expresiuxf3n-del-deseo-de-poder-predicar-el-mensaje-de-salvaciuxf3n-tambiuxe9n-en-roma}}

\bibleverse{8} En primer lugar, doy gracias a mi Dios, por medio de
Jesucristo, por todos vosotros, porque vuestra fe es proclamada en todo
el mundo. \footnote{\textbf{1:8} Rom 16,19} \bibleverse{9} Porque Dios
es mi testigo, a quien sirvo en mi espíritu en la Buena Nueva de su
Hijo, de cómo incesantemente hago mención de vosotros siempre en mis
oraciones, \footnote{\textbf{1:9} Efes 1,16} \bibleverse{10}
solicitando, si de alguna manera ahora por fin, me sea prosperada la
voluntad de Dios para ir a vosotros. \footnote{\textbf{1:10} Rom 15,23;
  Rom 15,32; Hech 19,21} \bibleverse{11} Porque anhelo veros, para poder
impartiros algún don espiritual, con el fin de que seáis firmes;
\footnote{\textbf{1:11} Rom 15,29} \bibleverse{12} es decir, para que yo
con vosotros me anime en vosotros, cada uno por la fe del otro, tanto la
vuestra como la mía. \footnote{\textbf{1:12} 2Pe 1,1}

\bibleverse{13} Ahora bien, no quiero que ignoréis, hermanos, que muchas
veces planeé ir a vosotros (y me lo impidieron hasta ahora), para tener
algún fruto también entre vosotros, como entre los demás gentiles.
\bibleverse{14} Soy deudor tanto de griegos como de extranjeros, tanto
de sabios como de necios. \bibleverse{15} Así que, en la medida en que
está en mí, estoy deseoso de predicar la Buena Nueva también a vosotros
que estáis en Roma.

\hypertarget{indicaciuxf3n-del-tema-la-justificaciuxf3n}{%
\subsection{Indicación del tema: La
justificación}\label{indicaciuxf3n-del-tema-la-justificaciuxf3n}}

\bibleverse{16} Porque no me avergüenzo de la Buena Nueva de Cristo,
porque es poder de Dios para la salvación de todo el que cree, primero
para el judío y también para el griego. \footnote{\textbf{1:16} Sal
  119,46; 1Cor 1,18; 1Cor 1,24; 2Tim 1,8} \bibleverse{17} Porque en ella
se revela la justicia de Dios de fe en fe. Como está escrito: ``Pero el
justo vivirá por la fe''. \footnote{\textbf{1:17} Rom 3,21-22}

\hypertarget{la-culpa-del-pecado-de-todo-paganismo}{%
\subsection{La culpa del pecado de todo
paganismo}\label{la-culpa-del-pecado-de-todo-paganismo}}

\bibleverse{18} Porque la ira de Dios se revela desde el cielo contra
toda impiedad e injusticia de los hombres que reprimen la verdad con
injusticia, \bibleverse{19} porque lo que se conoce de Dios se revela en
ellos, pues Dios se lo reveló. \footnote{\textbf{1:19} Hech 14,15-17;
  Hech 17,24-28} \bibleverse{20} Porque las cosas invisibles de él,
desde la creación del mundo, se ven claramente, percibiéndose por medio
de las cosas hechas, su eterno poder y su divinidad, para que no tengan
excusa. \footnote{\textbf{1:20} Sal 19,1; Heb 11,3} \bibleverse{21}
Porque conociendo a Dios, no lo glorificaron como a Dios, ni le dieron
gracias, sino que se envanecieron en sus razonamientos, y su corazón
insensato se oscureció. \footnote{\textbf{1:21} Efes 4,18}

\bibleverse{22} Profesando ser sabios, se hicieron necios, \footnote{\textbf{1:22}
  Jer 10,14; 1Cor 1,20} \bibleverse{23} y cambiaron la gloria del Dios
incorruptible por la semejanza de una imagen de hombre corruptible, y de
aves, cuadrúpedos y reptiles. \footnote{\textbf{1:23} Deut 4,15-19}

\hypertarget{el-juicio-divino-sobre-el-mundo-pagano-debido-a-su-ruina}{%
\subsection{El juicio divino sobre el mundo pagano debido a su
ruina}\label{el-juicio-divino-sobre-el-mundo-pagano-debido-a-su-ruina}}

\bibleverse{24} Por eso, Dios también los entregó a la impureza en los
deseos de sus corazones, para que sus cuerpos fueran deshonrados entre
ellos; \footnote{\textbf{1:24} Hech 14,16} \bibleverse{25} que cambiaron
la verdad de Dios por la mentira, y adoraron y sirvieron a la criatura
antes que al Creador, que es bendito por los siglos. Amén.

\bibleverse{26} Por esta razón, Dios los entregó a pasiones viles.
Porque sus mujeres cambiaron la función natural por lo que es contrario
a la naturaleza. \bibleverse{27} Así también los hombres, dejando la
función natural de la mujer, ardieron en su lujuria mutua, haciendo los
hombres lo que es inapropiado con los hombres, y recibiendo en sí mismos
el debido castigo de su error. \footnote{\textbf{1:27} Lev 18,22; Lev
  20,13; 1Cor 6,9} \bibleverse{28} Así como se negaron a tener a Dios en
su conocimiento, Dios los entregó a una mente reprobada, para hacer las
cosas que no convienen; \bibleverse{29} llenos de toda injusticia,
inmoralidad sexual, maldad, codicia, malicia llenos de envidia, de
homicidios, de contiendas, de engaños, de malas costumbres, de
calumniadores secretos, \bibleverse{30} aborrecedores de Dios, de
insolencia, de arrogancia, de jactancia, de invención de cosas malas, de
desobediencia a los padres, \bibleverse{31} de falta de entendimiento,
de ruptura de la alianza, de falta de afecto natural, de falta de
perdón, de falta de misericordia; \bibleverse{32} que, conociendo la
ordenanza de Dios, de que los que practican tales cosas son dignos de
muerte, no sólo hacen lo mismo, sino que aprueban a los que las
practican.

\hypertarget{el-juicio-de-la-ira-tambiuxe9n-estuxe1-ante-los-juduxedos-juzgar-a-los-demuxe1s-no-los-libera-del-juicio-de-dios}{%
\subsection{El juicio de la ira también está ante los judíos; juzgar a
los demás no los libera del juicio de
Dios}\label{el-juicio-de-la-ira-tambiuxe9n-estuxe1-ante-los-juduxedos-juzgar-a-los-demuxe1s-no-los-libera-del-juicio-de-dios}}

\hypertarget{section-1}{%
\section{2}\label{section-1}}

\bibleverse{1} Por lo tanto, no tienes excusa, oh hombre, quienquiera
que seas el que juzga. Porque en lo que juzgas a otro, te condenas a ti
mismo. Porque tú, que juzgas, practicas las mismas cosas. \footnote{\textbf{2:1}
  Mat 7,2; Juan 8,7; Sant 4,12} \bibleverse{2} Sabemos que el juicio de
Dios es según la verdad contra los que practican tales cosas.
\bibleverse{3} ¿Piensas esto, oh hombre que juzgas a los que practican
tales cosas, y haces lo mismo, que escaparás del juicio de Dios?
\bibleverse{4} ¿O acaso desprecias las riquezas de su bondad, su
tolerancia y su paciencia, sin saber que la bondad de Dios te lleva al
arrepentimiento? \footnote{\textbf{2:4} 2Pe 3,9; 2Pe 3,15}
\bibleverse{5} Pero según tu dureza y tu corazón impenitente estás
atesorando para ti la ira en el día de la ira, de la revelación y del
justo juicio de Dios, \bibleverse{6} que ``pagará a cada uno según sus
obras''. \footnote{\textbf{2:6} Mat 16,27; 2Cor 5,10} \bibleverse{7} a
los que por la perseverancia en el bien obrar buscan la gloria, el honor
y la incorruptibilidad, la vida eterna; \bibleverse{8} pero a los que
son egoístas y no obedecen a la verdad, sino que obedecen a la
injusticia, será la ira, la indignación, \footnote{\textbf{2:8} 2Tes 1,8}
\bibleverse{9} la opresión y la angustia sobre toda alma de hombre que
hace el mal, al judío primero, y también al griego.

\bibleverse{10} Pero la gloria, el honor y la paz van a todo hombre que
hace el bien, al judío primero y también al griego. \bibleverse{11}
Porque para Dios no hay parcialidad. \footnote{\textbf{2:11} Hech 10,34;
  1Pe 1,17; Col 3,25}

\hypertarget{el-juicio-de-dios-es-el-mismo-para-los-juduxedos-que-para-los-gentiles-determinado-uxfanicamente-por-el-cumplimiento-de-la-ley}{%
\subsection{El juicio de Dios es el mismo para los judíos que para los
gentiles, determinado únicamente por el cumplimiento de la
ley}\label{el-juicio-de-dios-es-el-mismo-para-los-juduxedos-que-para-los-gentiles-determinado-uxfanicamente-por-el-cumplimiento-de-la-ley}}

\bibleverse{12} Porque todos los que han pecado sin la ley, también
perecerán sin la ley. Todos los que han pecado bajo la ley serán
juzgados por la ley. \bibleverse{13} Porque no son los oidores de la ley
los que son justos ante Dios, sino que los hacedores de la ley serán
justificados \footnote{\textbf{2:13} Mat 7,21; Sant 1,22}
\bibleverse{14} (porque cuando los gentiles que no tienen la ley hacen
por naturaleza las cosas de la ley, éstos, no teniendo la ley, son una
ley para sí mismos, \footnote{\textbf{2:14} Hech 10,35} \bibleverse{15}
en cuanto muestran la obra de la ley escrita en sus corazones,
testificando con ellos su conciencia, y sus pensamientos entre sí
acusándolos o bien excusándolos) \footnote{\textbf{2:15} Rom 1,32}
\bibleverse{16} en el día en que Dios juzgará los secretos de los
hombres, según mi Buena Nueva, por Jesucristo. \footnote{\textbf{2:16}
  Luc 8,17}

\hypertarget{un-mejor-conocimiento-moral-y-la-capacidad-de-enseuxf1ar-no-hacen-que-los-juduxedos-sean-justos-ante-dios-su-fama-por-la-ley-es-nula-porque-la-transgrede}{%
\subsection{Un mejor conocimiento moral y la capacidad de enseñar no
hacen que los judíos sean justos ante Dios; su fama por la ley es nula
porque la
transgrede}\label{un-mejor-conocimiento-moral-y-la-capacidad-de-enseuxf1ar-no-hacen-que-los-juduxedos-sean-justos-ante-dios-su-fama-por-la-ley-es-nula-porque-la-transgrede}}

\bibleverse{17} En efecto, tú llevas el nombre de judío, te apoyas en la
ley, te glorías en Dios, \bibleverse{18} conoces su voluntad y apruebas
las cosas excelentes, siendo instruido por la ley, \bibleverse{19} y
estás seguro de que tú mismo eres guía de ciegos, luz para los que están
en tinieblas, \footnote{\textbf{2:19} Mat 15,14} \bibleverse{20}
corrector de necios, maestro de niños, teniendo en la ley la forma del
conocimiento y de la verdad. \bibleverse{21} Tú, pues, que enseñas a
otro, ¿no te enseñas a ti mismo? Tú que predicas que el hombre no debe
robar, ¿no robas tú? \footnote{\textbf{2:21} Sal 50,16-21; Mat 23,3-4}
\bibleverse{22} Tú que dices que el hombre no debe cometer adulterio,
¿cometes adulterio? Tú que aborreces los ídolos, ¿robas los templos?
\bibleverse{23} Ustedes que se glorían en la ley, ¿deshonran a Dios
desobedeciendo la ley? \bibleverse{24} Porque ``el nombre de Dios es
blasfemado entre los gentiles a causa de vosotros'', tal como está
escrito.

\hypertarget{la-circuncisiuxf3n-no-tiene-valor-para-el-juduxedo-si-infringe-la-ley-la-circuncisiuxf3n-del-corazuxf3n-es-necesaria}{%
\subsection{La circuncisión no tiene valor para el judío si infringe la
ley; La circuncisión del ``corazón'' es
necesaria}\label{la-circuncisiuxf3n-no-tiene-valor-para-el-juduxedo-si-infringe-la-ley-la-circuncisiuxf3n-del-corazuxf3n-es-necesaria}}

\bibleverse{25} Porque la circuncisión, en efecto, es provechosa, si
eres hacedor de la ley; pero si eres transgresor de la ley, tu
circuncisión se ha convertido en incircuncisión. \footnote{\textbf{2:25}
  Jer 4,4} \bibleverse{26} Por lo tanto, si el incircunciso guarda las
ordenanzas de la ley, ¿no se considerará su incircuncisión como
circuncisión? \footnote{\textbf{2:26} Gal 5,6} \bibleverse{27} ¿No te
juzgarán los que son físicamente incircuncisos, pero cumplen la ley, que
con la letra y la circuncisión son transgresores de la ley?
\bibleverse{28} Porque no es judío el que lo es exteriormente, ni la
circuncisión que es exterior en la carne; \bibleverse{29} sino que es
judío el que lo es interiormente, y la circuncisión es la del corazón,
en el espíritu, no en la letra; cuya alabanza no proviene de los
hombres, sino de Dios. \footnote{\textbf{2:29} Deut 30,6; Fil 3,3; Col
  2,11}

\hypertarget{sin-embargo-la-posiciuxf3n-privilegiada-de-los-juduxedos-permanece-su-infidelidad-pone-la-fidelidad-de-dios-en-una-luz-muxe1s-brillante}{%
\subsection{Sin embargo, la posición privilegiada de los judíos
permanece; su infidelidad pone la fidelidad de Dios en una luz más
brillante}\label{sin-embargo-la-posiciuxf3n-privilegiada-de-los-juduxedos-permanece-su-infidelidad-pone-la-fidelidad-de-dios-en-una-luz-muxe1s-brillante}}

\hypertarget{section-2}{%
\section{3}\label{section-2}}

\bibleverse{1} Entonces, ¿qué ventaja tiene el judío? ¿O cuál es el
beneficio de la circuncisión? \bibleverse{2} ¡Mucho en todos los
sentidos! Porque, en primer lugar, se les confiaron las revelaciones de
Dios. \footnote{\textbf{3:2} Rom 9,4; Deut 4,7-8; Sal 147,19-20}
\bibleverse{3} Pues, ¿qué pasa si algunos carecen de fe? ¿Acaso su falta
de fe anularía la fidelidad de Dios? \footnote{\textbf{3:3} Rom 9,6; Rom
  11,29; 2Tim 2,13} \bibleverse{4} ¡Que no sea así! Sí, que Dios sea
encontrado verdadero, pero todo hombre sea mentiroso. Como está escrito,
``para que se justifiquen sus palabras, y pueda prevalecer cuando entre
en juicio''. \footnote{\textbf{3:4} Sal 116,11}

\bibleverse{5} Pero si nuestra injusticia alaba la justicia de Dios,
¿qué diremos? ¿Es injusto el Dios que inflige la ira? Hablo como los
hombres. \bibleverse{6} ¡Que nunca lo sea! Porque entonces, ¿cómo
juzgará Dios al mundo? \bibleverse{7} Pues si la verdad de Dios por mi
mentira abundó para su gloria, ¿por qué también yo sigo siendo juzgado
como pecador? \bibleverse{8} ¿Por qué no (como se nos denuncia
calumniosamente, y como algunos afirman que decimos), ``Hagamos el mal,
para que venga el bien?'' Los que así dicen son justamente condenados.
\footnote{\textbf{3:8} Rom 6,1}

\hypertarget{resultado-la-corrupciuxf3n-del-pecado-se-extiende-a-gentiles-y-juduxedos-y-es-confirmada-por-numerosas-escrituras}{%
\subsection{Resultado: la corrupción del pecado se extiende a gentiles y
judíos y es confirmada por numerosas
escrituras}\label{resultado-la-corrupciuxf3n-del-pecado-se-extiende-a-gentiles-y-juduxedos-y-es-confirmada-por-numerosas-escrituras}}

\bibleverse{9} ¿Qué pasa entonces? ¿Somos mejores que ellos? No, de
ninguna manera. Porque ya hemos advertido tanto a los judíos como a los
griegos que todos están bajo el pecado. \footnote{\textbf{3:9} Rom
  1,18-999} \bibleverse{10} Como está escrito, ``No hay nadie justo; No,
no uno. \footnote{\textbf{3:10} Sal 14,1-3; Sal 53,1-3} \bibleverse{11}
No hay nadie que lo entienda. No hay nadie que busque a Dios.
\bibleverse{12} Todos se han alejado. Juntos han dejado de ser
rentables. No hay nadie que haga el bien, no, ni siquiera uno''.
\bibleverse{13} ``Su garganta es una tumba abierta. Con sus lenguas han
usado el engaño''. ``El veneno de las víboras está bajo sus labios''.
\footnote{\textbf{3:13} Sal 5,9; Sal 140,3} \bibleverse{14} ``Su boca
está llena de maldiciones y amargura''. \footnote{\textbf{3:14} Sal 10,7}
\bibleverse{15} ``Sus pies son rápidos para derramar sangre. \footnote{\textbf{3:15}
  Is 59,7-8} \bibleverse{16} La destrucción y la miseria están en sus
caminos. \bibleverse{17} El camino de la paz, no lo han conocido''.
\footnote{\textbf{3:17} Luc 1,79} \bibleverse{18} ``No hay temor de Dios
ante sus ojos''. \footnote{\textbf{3:18} Sal 36,1}

\bibleverse{19} Ahora bien, sabemos que todo lo que la ley dice, lo dice
a los que están bajo la ley, para que toda boca se cierre y todo el
mundo quede bajo el juicio de Dios. \footnote{\textbf{3:19} Rom 2,12;
  Gal 3,22} \bibleverse{20} Porque por las obras de la ley, ninguna
carne será justificada ante él; porque por la ley viene el conocimiento
del pecado. \footnote{\textbf{3:20} Sal 143,2; Gal 2,16; Rom 7,7}

\hypertarget{la-justicia-de-dios-se-otorga-a-los-que-creen-en-jesuxfas}{%
\subsection{La justicia de Dios se otorga a los que creen en
Jesús}\label{la-justicia-de-dios-se-otorga-a-los-que-creen-en-jesuxfas}}

\bibleverse{21} Pero ahora, aparte de la ley, se ha revelado una
justicia de Dios, testificada por la ley y los profetas; \footnote{\textbf{3:21}
  Rom 1,17; Hech 10,43} \bibleverse{22} la justicia de Dios por medio de
la fe en Jesucristo, para todos y sobre todos los que creen. Porque no
hay distinción, \footnote{\textbf{3:22} Fil 3,9} \bibleverse{23} pues
todos pecaron y están destituidos de la gloria de Dios; \footnote{\textbf{3:23}
  Rom 5,2; Juan 5,44; Sal 84,11} \bibleverse{24} siendo justificados
gratuitamente por su gracia, mediante la redención que es en Cristo
Jesús, \footnote{\textbf{3:24} Rom 5,1; 2Cor 5,19; Efes 2,8}
\bibleverse{25} a quien Dios envió como sacrificio expiatorio por medio
de la fe en su sangre, para demostración de su justicia mediante la
anulación de los pecados anteriores, en la tolerancia de Dios;
\footnote{\textbf{3:25} Lev 16,12-15; Heb 4,16} \bibleverse{26} para
demostrar su justicia en este tiempo, a fin de que él mismo sea justo y
justificador del que tiene fe en Jesús.

\hypertarget{la-justicia-de-dios-por-la-fe-excluye-toda-fama-propia-y-se-aplica-tanto-a-los-gentiles-como-a-los-juduxedos}{%
\subsection{La justicia de Dios por la fe excluye toda fama propia y se
aplica tanto a los gentiles como a los
judíos}\label{la-justicia-de-dios-por-la-fe-excluye-toda-fama-propia-y-se-aplica-tanto-a-los-gentiles-como-a-los-juduxedos}}

\bibleverse{27} ¿Dónde está entonces la jactancia? Está excluida. ¿Por
qué tipo de ley? ¿De obras? No, sino por una ley de fe. \footnote{\textbf{3:27}
  1Cor 1,29; 1Cor 1,31} \bibleverse{28} Sostenemos, pues, que el hombre
es justificado por la fe sin las obras de la ley. \footnote{\textbf{3:28}
  Gal 2,16} \bibleverse{29} ¿O acaso Dios es sólo el Dios de los judíos?
¿No es también el Dios de los gentiles? Sí, también de los gentiles,
\footnote{\textbf{3:29} Rom 10,12} \bibleverse{30} pues ciertamente hay
un solo Dios que justifica por la fe a los circuncisos y por la fe a los
incircuncisos. \footnote{\textbf{3:30} Rom 4,11-12}

\bibleverse{31} ¿Anulamos entonces la ley por la fe? ¡Que nunca sea así!
No, nosotros establecemos la ley. \footnote{\textbf{3:31} Mat 5,17}

\hypertarget{evidencia-de-la-justicia-de-la-fe-en-abraham-y-mediante-un-testimonio-de-david}{%
\subsection{Evidencia de la justicia de la fe en Abraham y mediante un
testimonio de
David}\label{evidencia-de-la-justicia-de-la-fe-en-abraham-y-mediante-un-testimonio-de-david}}

\hypertarget{section-3}{%
\section{4}\label{section-3}}

\bibleverse{1} ¿Qué diremos, pues, que ha encontrado Abraham, nuestro
antepasado, según la carne? \bibleverse{2} Porque si Abraham fue
justificado por las obras, tiene de qué jactarse, pero no ante Dios.
\bibleverse{3} Porque ¿qué dice la Escritura? ``Abraham creyó a Dios, y
le fue contado por justicia''. \footnote{\textbf{4:3} Gal 3,6}
\bibleverse{4} Ahora bien, al que trabaja, la recompensa no se le cuenta
como gracia, sino como algo debido. \footnote{\textbf{4:4} Rom 11,6}
\bibleverse{5} Pero al que no trabaja, sino que cree en el que justifica
al impío, su fe le es contada por justicia. \footnote{\textbf{4:5} Rom
  3,26} \bibleverse{6} Así como David también pronuncia la bendición
sobre el hombre a quien Dios le cuenta la justicia aparte de las obras:
\bibleverse{7} ``Bienaventurados aquellos cuyas iniquidades son
perdonadas, cuyos pecados están cubiertos. \bibleverse{8} Dichoso el
hombre al que el Señor no acusa de pecado''.

\hypertarget{abraham-como-el-padre-de-todos-los-creyentes-incluidos-los-gentiles}{%
\subsection{Abraham como el padre de todos los creyentes, incluidos los
gentiles}\label{abraham-como-el-padre-de-todos-los-creyentes-incluidos-los-gentiles}}

\bibleverse{9} Entonces, ¿se pronuncia esta bendición sólo sobre los
circuncisos, o también sobre los incircuncisos? Porque decimos que la fe
le fue contada a Abraham por justicia. \bibleverse{10} ¿Cómo, pues, le
fue contada? ¿En la circuncisión o en la incircuncisión? No en la
circuncisión, sino en la incircuncisión. \bibleverse{11} Recibió la
señal de la circuncisión, sello de la justicia de la fe que tenía
mientras estaba en la incircuncisión, para ser padre de todos los que
creen, aunque estén en la incircuncisión, a fin de que también les sea
contada la justicia. \footnote{\textbf{4:11} Gén 17,10-11}
\bibleverse{12} Él es el padre de la circuncisión para aquellos que no
sólo son de la circuncisión, sino que también caminan en los pasos de
esa fe de nuestro padre Abraham, que tuvo en la incircuncisión.
\footnote{\textbf{4:12} Mat 3,9}

\hypertarget{la-promesa-de-salvaciuxf3n-no-le-lleguxf3-a-abraham-por-la-ley-sino-por-la-fe}{%
\subsection{La promesa de salvación no le llegó a Abraham por la ley,
sino por la
fe}\label{la-promesa-de-salvaciuxf3n-no-le-lleguxf3-a-abraham-por-la-ley-sino-por-la-fe}}

\bibleverse{13} Porque la promesa hecha a Abraham y a su descendencia de
que sería heredero del mundo no fue por la ley, sino por la justicia de
la fe. \footnote{\textbf{4:13} Gén 22,17-18} \bibleverse{14} Porque si
los que son de la ley son herederos, la fe queda anulada, y la promesa
queda sin efecto. \bibleverse{15} Porque la ley produce ira; pues donde
no hay ley, tampoco hay desobediencia. \footnote{\textbf{4:15} Rom 3,20;
  Rom 5,13; Rom 7,8; Rom 7,10}

\bibleverse{16} Por eso es de fe, para que sea según la gracia, a fin de
que la promesa sea segura para toda la descendencia, no sólo para la que
es de la ley, sino también para la que es de la fe de Abraham, que es el
padre de todos nosotros. \bibleverse{17} Como está escrito: ``Te he
hecho padre de muchas naciones''. Esto es en presencia de aquel a quien
creyó: Dios, que da vida a los muertos, y llama a las cosas que no son,
como si fueran. \footnote{\textbf{4:17} Heb 11,19; 2Cor 1,9}

\hypertarget{la-fe-ejemplar-de-abraham}{%
\subsection{La fe ejemplar de Abraham}\label{la-fe-ejemplar-de-abraham}}

\bibleverse{18} En contra de la esperanza, Abraham creyó con esperanza,
a fin de llegar a ser padre de muchas naciones, según lo que se había
dicho: ``Así será tu descendencia.'' \bibleverse{19} Sin debilitarse en
la fe, no tuvo en cuenta su propio cuerpo, ya desgastado, (siendo él de
unos cien años de edad), y la mortandad del vientre de Sara. \footnote{\textbf{4:19}
  Gén 17,17} \bibleverse{20} Sin embargo, mirando la promesa de Dios, no
vaciló por la incredulidad, sino que se fortaleció por la fe, dando
gloria a Dios, \footnote{\textbf{4:20} Heb 11,11} \bibleverse{21} y
estando plenamente seguro de que lo que había prometido, también podía
cumplirlo. \bibleverse{22} Por eso también se le ``acreditó por
justicia''.

\hypertarget{tal-fe-tambiuxe9n-nos-trae-justicia-y-felicidad}{%
\subsection{Tal fe también nos trae justicia y
felicidad}\label{tal-fe-tambiuxe9n-nos-trae-justicia-y-felicidad}}

\bibleverse{23} Ahora bien, no está escrito que se le haya atribuido
sólo a él, \bibleverse{24} sino también a nosotros, a quienes se nos
atribuirá, que creemos en el que resucitó a Jesús, nuestro Señor, de
entre los muertos, \bibleverse{25} que fue entregado por nuestros
delitos y resucitó para nuestra justificación. \footnote{\textbf{4:25}
  Is 53,4-5; Rom 8,32; Rom 8,34}

\hypertarget{la-salvaciuxf3n-futura-estuxe1-garantizada-para-los-justificados-a-pesar-de-todas-las-tribulaciones-debido-al-amor-de-dios-demostrado-por-la-muerte-sacrificial-de-cristo}{%
\subsection{La salvación futura está garantizada para los justificados a
pesar de todas las tribulaciones debido al amor de Dios demostrado por
la muerte sacrificial de
Cristo}\label{la-salvaciuxf3n-futura-estuxe1-garantizada-para-los-justificados-a-pesar-de-todas-las-tribulaciones-debido-al-amor-de-dios-demostrado-por-la-muerte-sacrificial-de-cristo}}

\hypertarget{section-4}{%
\section{5}\label{section-4}}

\bibleverse{1} Justificados, pues, por la fe, tenemos paz para con Dios
por medio de nuestro Señor Jesucristo; \footnote{\textbf{5:1} Rom 3,24;
  Rom 3,28; Is 53,5} \bibleverse{2} por quien también tenemos acceso por
la fe a esta gracia en la que estamos. Nos alegramos en la esperanza de
la gloria de Dios. \footnote{\textbf{5:2} Juan 14,6; Efes 3,12}
\bibleverse{3} No sólo esto, sino que también nos alegramos de nuestros
sufrimientos, sabiendo que el sufrimiento produce perseverancia;
\footnote{\textbf{5:3} Sant 1,2; Sant 1,1-3} \bibleverse{4} y la
perseverancia, carácter probado; y el carácter probado, esperanza;
\footnote{\textbf{5:4} Sant 1,12} \bibleverse{5} y la esperanza no nos
defrauda, porque el amor de Dios ha sido derramado en nuestros corazones
por medio del Espíritu Santo que nos fue dado. \footnote{\textbf{5:5}
  Heb 6,18-19; Sal 22,5; Sal 25,3; Sal 25,20}

\bibleverse{6} Porque cuando aún éramos débiles, a su tiempo Cristo
murió por los impíos. \bibleverse{7} Porque difícilmente se morirá por
un justo. Sin embargo, tal vez por una persona buena alguien se atreva a
morir. \bibleverse{8} Pero Dios nos encomienda su propio amor, pues
siendo aún pecadores, Cristo murió por nosotros. \footnote{\textbf{5:8}
  Juan 3,16; 1Jn 4,10}

\bibleverse{9} Mucho más, pues, estando ahora justificados por su
sangre, seremos salvados de la ira de Dios por medio de él. \footnote{\textbf{5:9}
  Rom 1,18; Rom 2,5; Rom 2,8} \bibleverse{10} Porque si siendo enemigos,
fuimos reconciliados con Dios por la muerte de su Hijo, mucho más,
estando reconciliados, seremos salvados por su vida. \footnote{\textbf{5:10}
  Rom 8,7; Col 1,21; 2Cor 5,18}

\bibleverse{11} No sólo eso, sino que también nos alegramos en Dios por
medio de nuestro Señor Jesucristo, por quien ahora hemos recibido la
reconciliación.

\hypertarget{cristo-como-lo-opuesto-a-aduxe1n-la-gracia-que-trae-vida-inmortal-es-muxe1s-poderosa-que-el-pecado-mortal}{%
\subsection{Cristo como lo opuesto a Adán; la gracia que trae vida
inmortal es más poderosa que el pecado
mortal}\label{cristo-como-lo-opuesto-a-aduxe1n-la-gracia-que-trae-vida-inmortal-es-muxe1s-poderosa-que-el-pecado-mortal}}

\bibleverse{12} Por tanto, como el pecado entró en el mundo por un
hombre, y la muerte por el pecado, así la muerte pasó a todos los
hombres, porque todos pecaron. \footnote{\textbf{5:12} Gén 2,17; Gén
  3,19; Rom 6,23} \bibleverse{13} Porque hasta la ley, el pecado estaba
en el mundo; pero el pecado no es acusado cuando no hay ley. \footnote{\textbf{5:13}
  Rom 4,15} \bibleverse{14} Sin embargo, la muerte reinó desde Adán
hasta Moisés, incluso sobre aquellos cuyos pecados no fueron como la
desobediencia de Adán, que es una prefiguración del que había de venir.

\bibleverse{15} Pero el don gratuito no es como la transgresión. Porque
si por la transgresión de uno murieron los muchos, mucho más abundó la
gracia de Dios y el don por la gracia de un solo hombre, Jesucristo,
para los muchos. \bibleverse{16} El don no es como por uno solo que
pecó; porque el juicio vino por uno solo para condenación, pero el don
gratuito siguió a muchas transgresiones para justificación.
\bibleverse{17} Porque si por la transgresión de uno reinó la muerte por
medio de uno, mucho más reinarán en vida por medio de uno, Jesucristo,
los que reciben la abundancia de la gracia y del don de la justicia.

\bibleverse{18} Así, pues, como por una sola transgresión fueron
condenados todos los hombres, así por una sola acción de justicia fueron
justificados todos los hombres para la vida. \footnote{\textbf{5:18}
  1Cor 15,21-22} \bibleverse{19} Porque así como por la desobediencia de
un solo hombre muchos fueron hechos pecadores, así también por la
obediencia de uno, muchos serán hechos justos. \footnote{\textbf{5:19}
  Rom 3,26; Is 53,11} \bibleverse{20} La ley entró para que abundara la
transgresión; pero donde abundó el pecado, sobreabundó la gracia,
\footnote{\textbf{5:20} Rom 7,8; Rom 7,13; Gal 3,19} \bibleverse{21}
para que así como el pecado reinó en la muerte, así la gracia reine por
la justicia para vida eterna por medio de Jesucristo nuestro Señor.
\footnote{\textbf{5:21} Rom 6,23}

\hypertarget{fuimos-crucificados-con-ellos-morimos-con-ellos-sepultados-con-ellos-y-resucitamos-con-cristo-jesuxfas}{%
\subsection{Fuimos crucificados con ellos, morimos con ellos, sepultados
con ellos y resucitamos con Cristo
Jesús}\label{fuimos-crucificados-con-ellos-morimos-con-ellos-sepultados-con-ellos-y-resucitamos-con-cristo-jesuxfas}}

\hypertarget{section-5}{%
\section{6}\label{section-5}}

\bibleverse{1} ¿Qué diremos entonces? ¿Seguiremos en el pecado, para que
la gracia abunde? \footnote{\textbf{6:1} Rom 3,5-8} \bibleverse{2} ¡Que
no sea nunca! Nosotros, que hemos muerto al pecado, ¿cómo podríamos
seguir viviendo en él? \bibleverse{3} ¿O no sabéis que todos los que
fuimos bautizados en Cristo Jesús fuimos bautizados en su muerte?
\footnote{\textbf{6:3} Gal 3,27} \bibleverse{4} Fuimos, pues, sepultados
con él por el bautismo en la muerte, para que así como Cristo resucitó
de entre los muertos por la gloria del Padre, así también nosotros
andemos en una vida nueva. \footnote{\textbf{6:4} Col 2,12; 1Pe 3,21}

\bibleverse{5} Porque si nos hemos unido a él en la semejanza de su
muerte, seremos también partícipes de su resurrección; \bibleverse{6}
sabiendo esto, que nuestro viejo hombre fue crucificado con él, para que
el cuerpo del pecado fuera eliminado, a fin de que ya no fuéramos
esclavos del pecado. \footnote{\textbf{6:6} Gal 5,24} \bibleverse{7}
Porque el que ha muerto ha sido liberado del pecado.

\hypertarget{viviendo-con-cristo-resucitado}{%
\subsection{Viviendo con Cristo
resucitado}\label{viviendo-con-cristo-resucitado}}

\bibleverse{8} Pero si hemos muerto con Cristo, creemos que también
viviremos con él, \bibleverse{9} sabiendo que Cristo, resucitado de
entre los muertos, ya no muere. La muerte ya no se enseñorea más de él.
\bibleverse{10} Porque la muerte que murió, murió para el pecado una
vez; pero la vida que vive, la vive para Dios. \footnote{\textbf{6:10}
  Heb 9,26-28} \bibleverse{11} Así pues, considérense también muertos al
pecado, pero vivos para Dios en Cristo Jesús, nuestro Señor. \footnote{\textbf{6:11}
  2Cor 5,15; 1Pe 2,24}

\hypertarget{la-amonestaciuxf3n-del-apuxf3stol-a-los-fieles-de-permanecer-en-este-conocimiento-de-la-salvaciuxf3n-y-no-seguir-sirviendo-al-pecado}{%
\subsection{La amonestación del apóstol a los fieles de permanecer en
este conocimiento de la salvación y no seguir sirviendo al
pecado}\label{la-amonestaciuxf3n-del-apuxf3stol-a-los-fieles-de-permanecer-en-este-conocimiento-de-la-salvaciuxf3n-y-no-seguir-sirviendo-al-pecado}}

\bibleverse{12} Por tanto, no dejéis que el pecado reine en vuestro
cuerpo mortal, para que lo obedezcáis en sus concupiscencias.
\footnote{\textbf{6:12} Gén 4,7} \bibleverse{13} Asimismo, no presentéis
vuestros miembros al pecado como instrumentos de injusticia, sino
presentaos a Dios como vivos de entre los muertos, y vuestros miembros
como instrumentos de justicia para Dios. \footnote{\textbf{6:13} Rom
  12,1} \bibleverse{14} Porque el pecado no se enseñoreará de vosotros,
pues no estáis bajo la ley, sino bajo la gracia. \footnote{\textbf{6:14}
  Rom 7,4-6; 1Jn 3,6}

\hypertarget{el-servicio-del-pecado-ha-dado-paso-a-la-justicia}{%
\subsection{El servicio del pecado ha dado paso a la
justicia}\label{el-servicio-del-pecado-ha-dado-paso-a-la-justicia}}

\bibleverse{15} ¿Qué, pues? ¿Pecaremos porque no estamos bajo la ley
sino bajo la gracia? ¡Que nunca sea así! \footnote{\textbf{6:15} Rom
  5,17; Rom 5,21} \bibleverse{16} ¿No sabéis que cuando os presentáis
como siervos y obedecéis a alguien, sois siervos de aquel a quien
obedecéis, ya sea del pecado a la muerte o de la obediencia a la
justicia? \footnote{\textbf{6:16} Juan 8,34} \bibleverse{17} Pero
gracias a Dios que, mientras erais siervos del pecado, os hicisteis
obedientes de corazón a esa forma de enseñanza a la que fuisteis
entregados. \bibleverse{18} Liberados del pecado, os hicisteis siervos
de la justicia. \footnote{\textbf{6:18} Juan 8,32}

\bibleverse{19} Hablo en términos humanos a causa de la debilidad de
vuestra carne; pues así como presentasteis vuestros miembros como
siervos de la inmundicia y de la maldad sobre la maldad, así ahora
presentad vuestros miembros como siervos de la justicia para la
santificación. \bibleverse{20} Porque cuando erais siervos del pecado,
estabais libres de la justicia. \bibleverse{21} ¿Qué fruto teníais
entonces en las cosas de las que ahora os avergonzáis? Porque el fin de
esas cosas es la muerte. \footnote{\textbf{6:21} Rom 8,6; Rom 8,13}
\bibleverse{22} Pero ahora, liberados del pecado y convertidos en
siervos de Dios, tenéis el fruto de la santificación y el resultado de
la vida eterna. \bibleverse{23} Porque la paga del pecado es la muerte,
pero el don gratuito de Dios es la vida eterna en Cristo Jesús, nuestro
Señor. \footnote{\textbf{6:23} Rom 5,12; Sant 1,15}

\hypertarget{cuando-hemos-muerto-y-resucitado-con-cristo-estamos-leguxedtimamente-libres-de-la-ley-y-estamos-obligados-a-servir-al-cristo-resucitado-creyuxe9ndonos-muertos-al-pecado}{%
\subsection{Cuando hemos muerto y resucitado con Cristo, estamos
legítimamente libres de la ley y estamos obligados a servir al Cristo
resucitado creyéndonos muertos al
pecado}\label{cuando-hemos-muerto-y-resucitado-con-cristo-estamos-leguxedtimamente-libres-de-la-ley-y-estamos-obligados-a-servir-al-cristo-resucitado-creyuxe9ndonos-muertos-al-pecado}}

\hypertarget{section-6}{%
\section{7}\label{section-6}}

\bibleverse{1} ¿O acaso no sabéis, hermanos (pues hablo con hombres que
conocen la ley), que la ley se impone al hombre mientras vive?
\bibleverse{2} Porque la mujer que tiene marido está unida por la ley al
marido mientras éste vive; pero si el marido muere, queda liberada de la
ley del marido. \footnote{\textbf{7:2} 1Cor 7,39} \bibleverse{3} Así
pues, si mientras vive el marido se une a otro hombre, se la llamará
adúltera. Pero si el marido muere, ella queda libre de la ley, de modo
que no es adúltera, aunque esté unida a otro hombre. \bibleverse{4} Por
tanto, hermanos míos, también vosotros habéis sido muertos a la ley por
el cuerpo de Cristo, para que os unáis a otro, al que resucitó de entre
los muertos, a fin de que produzcamos fruto para Dios. \bibleverse{5}
Porque cuando estábamos en la carne, las pasiones pecaminosas que eran
por la ley obraban en nuestros miembros para producir frutos para la
muerte. \footnote{\textbf{7:5} Rom 6,21} \bibleverse{6} Pero ahora hemos
sido liberados de la ley, habiendo muerto a aquello en lo que estábamos
sujetos; de modo que servimos en la novedad del espíritu, y no en la
antigüedad de la letra. \footnote{\textbf{7:6} Rom 8,1-2; Rom 6,2; Rom
  6,4}

\hypertarget{el-efecto-calamitoso-de-la-ley-que-familiariza-al-hombre-con-el-pecado-y-le-da-vida-al-pecado-en-la-carne}{%
\subsection{El efecto calamitoso de la ley, que familiariza al hombre
con el pecado y le da vida al pecado en la
carne}\label{el-efecto-calamitoso-de-la-ley-que-familiariza-al-hombre-con-el-pecado-y-le-da-vida-al-pecado-en-la-carne}}

\bibleverse{7} ¿Qué diremos entonces? ¿Es la ley pecado? ¡Que nunca lo
sea! Sin embargo, yo no habría conocido el pecado si no fuera por la
ley. Pues no habría conocido la codicia si la ley no hubiera dicho: ``No
codiciarás''. \bibleverse{8} Pero el pecado, encontrando ocasión a
través del mandamiento, produjo en mí toda clase de codicia. Porque sin
la ley, el pecado está muerto. \footnote{\textbf{7:8} Rom 5,13; 1Cor
  15,56} \bibleverse{9} En otro tiempo vivía fuera de la ley, pero
cuando llegó el mandamiento, el pecado revivió y yo morí.
\bibleverse{10} El mandamiento que era para la vida, lo encontré para la
muerte; \footnote{\textbf{7:10} Sant 1,15; Lev 18,5} \bibleverse{11}
porque el pecado, encontrando ocasión por el mandamiento, me engañó, y
por él me mató. \footnote{\textbf{7:11} Heb 3,13} \bibleverse{12} Por
tanto, la ley es verdaderamente santa, y el mandamiento santo, justo y
bueno. \footnote{\textbf{7:12} 1Tim 1,8}

\bibleverse{13} ¿Acaso lo que es bueno se convirtió en muerte para mí?
¡Que nunca lo sea! Pero el pecado, para que se demuestre que es pecado,
estaba produciendo la muerte en mí por medio de lo que es bueno; para
que por medio del mandamiento el pecado se vuelva excesivamente
pecaminoso. \footnote{\textbf{7:13} Rom 5,20}

\hypertarget{la-impotencia-de-la-ley-y-de-la-buena-voluntad-ante-el-pecado-como-poder-en-la-carne}{%
\subsection{La impotencia de la ley y de la buena voluntad ante el
pecado como poder en la
carne}\label{la-impotencia-de-la-ley-y-de-la-buena-voluntad-ante-el-pecado-como-poder-en-la-carne}}

\bibleverse{14} Porque sabemos que la ley es espiritual, pero yo soy
carnal, vendido al pecado. \footnote{\textbf{7:14} Juan 3,6}
\bibleverse{15} Porque no entiendo lo que hago. Pues no practico lo que
deseo hacer; pero lo que aborrezco, eso hago. \bibleverse{16} Pero si lo
que no deseo, eso hago, consiento a la ley que sea bueno.
\bibleverse{17} Así que ya no soy yo quien lo hace, sino el pecado que
mora en mí. \bibleverse{18} Porque sé que en mí, es decir, en mi carne,
no mora nada bueno. Porque el deseo está presente en mí, pero no lo
encuentro haciendo lo que es bueno. \footnote{\textbf{7:18} Gén 6,5; Gén
  8,21} \bibleverse{19} Porque el bien que deseo, no lo hago; pero el
mal que no deseo, ese sí lo practico. \bibleverse{20} Pero si lo que no
deseo, eso hago, ya no soy yo quien lo hace, sino el pecado que mora en
mí. \bibleverse{21} Encuentro, pues, la ley de que, mientras deseo hacer
el bien, el mal está presente. \bibleverse{22} Porque me deleito en la
ley de Dios según la persona interior, \bibleverse{23} pero veo una ley
diferente en mis miembros, que se opone a la ley de mi mente, y me lleva
cautivo bajo la ley del pecado que está en mis miembros. \footnote{\textbf{7:23}
  Gal 5,17} \bibleverse{24} ¡Qué miserable soy! ¿Quién me librará del
cuerpo de esta muerte? \bibleverse{25} ¡Doy gracias a Dios por
Jesucristo, nuestro Señor! Así que con la mente, yo mismo sirvo a la ley
de Dios, pero con la carne, a la ley del pecado. \footnote{\textbf{7:25}
  1Cor 15,57}

\hypertarget{el-cristiano-estuxe1-bajo-la-ley-del-espuxedritu}{%
\subsection{El cristiano está bajo la ley del
Espíritu}\label{el-cristiano-estuxe1-bajo-la-ley-del-espuxedritu}}

\hypertarget{section-7}{%
\section{8}\label{section-7}}

\bibleverse{1} Ahora, pues, no hay condenación para los que están en
Cristo Jesús, que no andan según la carne, sino según el Espíritu.
\footnote{\textbf{8:1} Rom 8,33-34} \bibleverse{2} Porque la ley del
Espíritu de vida en Cristo Jesús me hizo libre de la ley del pecado y de
la muerte. \bibleverse{3} Porque lo que la ley no pudo hacer, por cuanto
era débil por la carne, Dios lo hizo, enviando a su propio Hijo en
semejanza de carne de pecado y por el pecado, condenó al pecado en la
carne, \footnote{\textbf{8:3} Hech 13,38; Hech 15,10; Heb 2,17}
\bibleverse{4} para que la ordenanza de la ley se cumpliera en nosotros,
que no andamos según la carne, sino según el Espíritu. \footnote{\textbf{8:4}
  Gal 5,16; Gal 5,25}

\hypertarget{el-contraste-entre-los-que-sirven-a-dios-en-el-espuxedritu-y-los-que-viven-por-los-instintos-de-la-carne}{%
\subsection{El contraste entre los que sirven a Dios en el Espíritu y
los que viven por los instintos de la
carne}\label{el-contraste-entre-los-que-sirven-a-dios-en-el-espuxedritu-y-los-que-viven-por-los-instintos-de-la-carne}}

\bibleverse{5} Porque los que viven según la carne ponen su mente en las
cosas de la carne, pero los que viven según el Espíritu, en las cosas
del Espíritu. \bibleverse{6} Porque la mente de la carne es muerte, pero
la mente del Espíritu es vida y paz; \footnote{\textbf{8:6} Rom 6,21;
  Gal 6,8} \bibleverse{7} porque la mente de la carne es hostil a Dios,
pues no se sujeta a la ley de Dios, ni tampoco puede hacerlo.
\footnote{\textbf{8:7} Sant 4,4} \bibleverse{8} Los que están en la
carne no pueden agradar a Dios.

\hypertarget{el-cristiano-como-morada-del-espuxedritu}{%
\subsection{El cristiano como morada del
Espíritu}\label{el-cristiano-como-morada-del-espuxedritu}}

\bibleverse{9} Pero no estáis en la carne, sino en el Espíritu, si es
que el Espíritu de Dios habita en vosotros. Pero si alguno no tiene el
Espíritu de Cristo, no es suyo. \bibleverse{10} Si Cristo está en
vosotros, el cuerpo está muerto a causa del pecado, pero el espíritu
está vivo a causa de la justicia. \footnote{\textbf{8:10} Gal 2,20}
\bibleverse{11} Pero si el Espíritu del que resucitó a Jesús de entre
los muertos habita en vosotros, el que resucitó a Cristo Jesús de entre
los muertos también dará vida a vuestros cuerpos mortales por medio de
su Espíritu que habita en vosotros.

\hypertarget{la-posesiuxf3n-del-espuxedritu-garantiza-la-redenciuxf3n-fuxedsica-de-los-hijos-de-dios-si-soportan-los-sufrimientos-de-este-tiempo}{%
\subsection{La posesión del espíritu garantiza la redención física de
los hijos de Dios si soportan los sufrimientos de este
tiempo}\label{la-posesiuxf3n-del-espuxedritu-garantiza-la-redenciuxf3n-fuxedsica-de-los-hijos-de-dios-si-soportan-los-sufrimientos-de-este-tiempo}}

\bibleverse{12} Así que, hermanos, somos deudores, no de la carne, para
vivir según la carne. \footnote{\textbf{8:12} Rom 6,7; Rom 6,18}
\bibleverse{13} Porque si vivís según la carne, debéis morir; pero si
por el Espíritu hacéis morir las obras del cuerpo, viviréis. \footnote{\textbf{8:13}
  Rom 7,24; Gal 6,8; Efes 4,22-24} \bibleverse{14} Porque todos los que
son guiados por el Espíritu de Dios, éstos son hijos de Dios.
\bibleverse{15} Porque no habéis recibido el espíritu de esclavitud para
el temor, sino que habéis recibido el Espíritu de adopción, por el cual
clamamos: ``¡Abba! Padre!'' \footnote{\textbf{8:15} 2Tim 1,7; Gal 4,5-6}

\bibleverse{16} El Espíritu mismo da testimonio a nuestro espíritu de
que somos hijos de Dios; \footnote{\textbf{8:16} 2Cor 1,22}
\bibleverse{17} y si hijos, también herederos, herederos de Dios y
coherederos con Cristo, si es que sufrimos con él, para que también
seamos glorificados con él. \footnote{\textbf{8:17} Gal 4,7; Apoc 21,7}

\bibleverse{18} Porque considero que los sufrimientos de este tiempo no
son dignos de compararse con la gloria que se nos revelará. \footnote{\textbf{8:18}
  2Cor 4,17} \bibleverse{19} Porque la creación espera con ansia que se
manifiesten los hijos de Dios. \footnote{\textbf{8:19} Col 3,4; 1Jn 3,2}
\bibleverse{20} Porque la creación fue sometida a la vanidad, no por su
propia voluntad, sino por causa del que la sometió, en la esperanza
\footnote{\textbf{8:20} Gén 3,17; Ecl 1,2} \bibleverse{21} de que
también la creación misma será liberada de la esclavitud de la
decadencia a la libertad de la gloria de los hijos de Dios. \footnote{\textbf{8:21}
  2Pe 3,13} \bibleverse{22} Porque sabemos que toda la creación gime y
sufre dolores hasta ahora. \bibleverse{23} No sólo eso, sino que
nosotros mismos, que tenemos las primicias del Espíritu, también gemimos
en nuestro interior, esperando la adopción, la redención de nuestro
cuerpo. \footnote{\textbf{8:23} 2Cor 5,2} \bibleverse{24} Porque fuimos
salvados en la esperanza, pero la esperanza que se ve no es esperanza.
Porque ¿quién espera lo que ve? \footnote{\textbf{8:24} 2Cor 5,7}
\bibleverse{25} Pero si esperamos lo que no vemos, lo esperamos con
paciencia. \footnote{\textbf{8:25} Gal 5,5}

\bibleverse{26} Del mismo modo, el Espíritu también ayuda a nuestras
debilidades, pues no sabemos orar como es debido. Pero el Espíritu mismo
intercede por nosotros con gemidos indecibles. \bibleverse{27} El que
escudriña los corazones sabe lo que piensa el Espíritu, porque intercede
por los santos según Dios.

\hypertarget{el-comienzo-de-nuestra-comuniuxf3n-con-dios-obra-de-dios-garantiza-su-finalizaciuxf3n-final}{%
\subsection{El comienzo de nuestra comunión con Dios, obra de Dios,
garantiza su finalización
final}\label{el-comienzo-de-nuestra-comuniuxf3n-con-dios-obra-de-dios-garantiza-su-finalizaciuxf3n-final}}

\bibleverse{28} Sabemos que todas las cosas cooperan para el bien de los
que aman a Dios, de los que son llamados según su propósito. \footnote{\textbf{8:28}
  Efes 1,11} \bibleverse{29} Porque a los que conoció de antemano,
también los predestinó a ser conformes a la imagen de su Hijo, para que
fuera el primogénito entre muchos hermanos. \footnote{\textbf{8:29} Col
  1,18; Heb 1,6} \bibleverse{30} A los que predestinó, también los
llamó. A los que llamó, también los justificó. A los que justificó,
también los glorificó. \footnote{\textbf{8:30} Rom 3,26; 2Tes 2,13-14}

\hypertarget{por-tanto-nuestro-estado-de-salvaciuxf3n-estuxe1-divinamente-asegurado-contra-todos-los-poderes-y-nuestra-certeza-de-fe-y-seguridad-de-la-salvaciuxf3n-estuxe1-justificada}{%
\subsection{Por tanto, nuestro estado de salvación está divinamente
asegurado contra todos los poderes y nuestra certeza de fe y seguridad
de la salvación está
justificada}\label{por-tanto-nuestro-estado-de-salvaciuxf3n-estuxe1-divinamente-asegurado-contra-todos-los-poderes-y-nuestra-certeza-de-fe-y-seguridad-de-la-salvaciuxf3n-estuxe1-justificada}}

\bibleverse{31} ¿Qué diremos, pues, de estas cosas? Si Dios está a favor
de nosotros, ¿quién puede estar en contra? \footnote{\textbf{8:31} Sal
  118,6} \bibleverse{32} El que no perdonó a su propio Hijo, sino que lo
entregó por todos nosotros, ¿cómo no va a darnos también con él todas
las cosas? \footnote{\textbf{8:32} Juan 3,16} \bibleverse{33} ¿Quién
podría acusar a los elegidos de Dios? Es Dios quien justifica.
\bibleverse{34} ¿Quién es el que condena? Es Cristo que murió, más aún,
que resucitó de entre los muertos, que está a la derecha de Dios, que
también intercede por nosotros. \footnote{\textbf{8:34} 1Jn 2,1; Heb
  7,25}

\bibleverse{35} ¿Quién nos separará del amor de Cristo? ¿Podrá la
opresión, o la angustia, o la persecución, o el hambre, o la desnudez, o
el peligro, o la espada? \bibleverse{36} Como está escrito, ``Por tu
causa nos matan todo el día. Fuimos contados como ovejas para el
matadero''. \footnote{\textbf{8:36} 2Cor 4,11} \bibleverse{37} No, en
todas estas cosas somos más que vencedores por medio de aquel que nos
amó. \footnote{\textbf{8:37} 1Jn 5,4} \bibleverse{38} Porque estoy
convencido de que ni la muerte, ni la vida, ni los ángeles, ni los
principados, ni lo presente, ni lo futuro, ni las potencias, \footnote{\textbf{8:38}
  Efes 6,12} \bibleverse{39} ni la altura, ni la profundidad, ni ninguna
otra cosa creada podrá separarnos del amor de Dios que está en Cristo
Jesús, nuestro Señor.

\hypertarget{introducciuxf3n-el-profundo-dolor-del-apuxf3stol-por-la-exclusiuxf3n-temporal-de-su-pueblo-de-la-salvaciuxf3n}{%
\subsection{Introducción: El profundo dolor del apóstol por la exclusión
temporal de su pueblo de la
salvación}\label{introducciuxf3n-el-profundo-dolor-del-apuxf3stol-por-la-exclusiuxf3n-temporal-de-su-pueblo-de-la-salvaciuxf3n}}

\hypertarget{section-8}{%
\section{9}\label{section-8}}

\bibleverse{1} Digo la verdad en Cristo. No miento, pues mi conciencia
testifica conmigo en el Espíritu Santo \bibleverse{2} que tengo una gran
pena y un dolor incesante en mi corazón. \bibleverse{3} Porque desearía
ser yo mismo separado de Cristo por mis hermanos, mis parientes según la
carne \footnote{\textbf{9:3} Éxod 32,32} \bibleverse{4} que son
israelitas; de los cuales es la adopción, la gloria, las alianzas, la
entrega de la ley, el servicio y las promesas; \footnote{\textbf{9:4}
  Éxod 4,22; Deut 7,6; Gén 17,7; Éxod 20,1; Éxod 40,34} \bibleverse{5}
de los cuales son los padres, y de los cuales es Cristo en cuanto a la
carne, que es sobre todo, Dios, bendito por siempre. Amén. \footnote{\textbf{9:5}
  Mat 1,1; Luc 3,23-34; Juan 1,1; Rom 1,3}

\hypertarget{las-promesas-de-dios-a-israel-son-inquebrantables-pero-no-se-aplican-a-todo-el-cuerpo-sino-solo-a-los-descendientes-espirituales-de-abraham}{%
\subsection{Las promesas de Dios a Israel son inquebrantables, pero no
se aplican a todo el cuerpo, sino solo a los descendientes espirituales
de
Abraham}\label{las-promesas-de-dios-a-israel-son-inquebrantables-pero-no-se-aplican-a-todo-el-cuerpo-sino-solo-a-los-descendientes-espirituales-de-abraham}}

\bibleverse{6} Pero no es que la palabra de Dios haya quedado en nada.
Porque no todos los que son de Israel son de Israel. \footnote{\textbf{9:6}
  Núm 23,19; Rom 2,28} \bibleverse{7} Tampoco, por ser descendientes de
Abraham, son todos hijos. Pero, ``su descendencia será contada como de
Isaac''. \bibleverse{8} Es decir, no son los hijos de la carne los que
son hijos de Dios, sino que son contados como herederos los hijos de la
promesa. \footnote{\textbf{9:8} Gal 4,23} \bibleverse{9} Porque esta es
una palabra de promesa: ``Al tiempo señalado vendré, y Sara tendrá un
hijo.'' \bibleverse{10} No sólo eso, sino que Rebeca también concibió
por uno, por nuestro padre Isaac. \bibleverse{11} Porque no habiendo
nacido aún, ni habiendo hecho nada bueno o malo, para que el propósito
de Dios según la elección se mantenga, no por las obras, sino por el que
llama, \bibleverse{12} se le dijo: ``El mayor servirá al menor.''
\bibleverse{13} Como está escrito: ``A Jacob lo amé, pero a Esaú lo
aborrecí''.

\hypertarget{la-elecciuxf3n-para-la-salvaciuxf3n-es-obra-gratuita-de-la-gracia-de-dios-la-negaciuxf3n-de-la-salvaciuxf3n-y-la-gracia-no-permite-al-hombre-pelear-con-dios}{%
\subsection{La elección para la salvación es obra gratuita de la gracia
de Dios; la negación de la salvación y la gracia no permite al hombre
pelear con
Dios}\label{la-elecciuxf3n-para-la-salvaciuxf3n-es-obra-gratuita-de-la-gracia-de-dios-la-negaciuxf3n-de-la-salvaciuxf3n-y-la-gracia-no-permite-al-hombre-pelear-con-dios}}

\bibleverse{14} ¿Qué diremos entonces? ¿Hay injusticia con Dios? ¡Que
nunca la haya! \bibleverse{15} Porque dijo a Moisés: ``Tendré
misericordia del que tenga misericordia, y me compadeceré del que me
compadezca''. \bibleverse{16} Así que no es del que quiere, ni del que
corre, sino de Dios que tiene misericordia. \footnote{\textbf{9:16} Efes
  2,8} \bibleverse{17} Porque la Escritura dice al Faraón: ``Para esto
mismo te hice levantar, para mostrar en ti mi poder, y para que mi
nombre sea proclamado en toda la tierra.'' \bibleverse{18} Así, pues,
tiene misericordia de quien quiere, y endurece a quien quiere.
\footnote{\textbf{9:18} Éxod 4,21; 1Pe 2,8}

\bibleverse{19} Me diréis entonces: ``¿Por qué sigue encontrando fallos?
Porque ¿quién resiste su voluntad?'' \bibleverse{20} Pero en verdad, oh
hombre, ¿quién eres tú para replicar contra Dios? ¿Acaso la cosa formada
le preguntará a quien la formó: ``Por qué me hiciste así''? \footnote{\textbf{9:20}
  Is 45,9} \bibleverse{21} ¿O acaso el alfarero no tiene derecho sobre
el barro, para hacer de la misma masa una parte para la honra y otra
para la deshonra? \bibleverse{22} ¿Y si Dios, queriendo mostrar su ira y
dar a conocer su poder, soportó con mucha paciencia vasos de ira
preparados para la destrucción, \footnote{\textbf{9:22} Rom 2,4; Prov
  16,4} \bibleverse{23} y para dar a conocer las riquezas de su gloria
en vasos de misericordia, que preparó de antemano para la gloria,
\footnote{\textbf{9:23} Rom 8,29; Efes 1,3-12} \bibleverse{24} nosotros,
a quienes también llamó, no sólo de los judíos, sino también de los
gentiles? \bibleverse{25} Como dice también en Oseas, ``Los llamaré `mi
pueblo', que no era mi pueblo; y su ``amado'', que no era amado''.
\bibleverse{26} ``Será que en el lugar donde se les dijo: `Ustedes no
son mi pueblo'allí serán llamados `hijos del Dios vivo'\,''.

\bibleverse{27} Isaías clama por Israel, ``Si el número de los hijos de
Israel es como la arena del mar, es el remanente el que se salvará;
\footnote{\textbf{9:27} Rom 11,5} \bibleverse{28} porque él terminará la
obra y la cortará en justicia, porque el Señor hará una obra corta sobre
la tierra''.

\bibleverse{29} Como ya dijo Isaías, ``A menos que el Señor de los
Ejércitos nos haya dejado una semilla, nos habríamos vuelto como Sodoma,
y se hubiera hecho como Gomorra''.

\hypertarget{la-culpa-de-los-juduxedos-consistiuxf3-en-el-rechazo-de-la-justicia-de-la-fe-y-en-la-persecuciuxf3n-excesiva-de-la-justicia-de-las-obras}{%
\subsection{La culpa de los judíos consistió en el rechazo de la
justicia de la fe y en la persecución excesiva de la justicia de las
obras}\label{la-culpa-de-los-juduxedos-consistiuxf3-en-el-rechazo-de-la-justicia-de-la-fe-y-en-la-persecuciuxf3n-excesiva-de-la-justicia-de-las-obras}}

\bibleverse{30} ¿Qué diremos entonces? Que los gentiles, que no seguían
la justicia, alcanzaron la justicia, la justicia que es de la fe;
\footnote{\textbf{9:30} Rom 10,20} \bibleverse{31} pero Israel,
siguiendo una ley de justicia, no llegó a la ley de justicia.
\footnote{\textbf{9:31} Rom 10,2-3} \bibleverse{32} ¿Por qué? Porque no
la buscaron por la fe, sino como por las obras de la ley. Tropezaron con
la piedra de tropiezo, \bibleverse{33} como está escrito, ``He aquí que
pongo en Sión una piedra de tropiezo y una roca de ofensa; y nadie que
crea en él quedará decepcionado''. \footnote{\textbf{9:33} Mat 21,42;
  Mat 21,44; 1Pe 2,8}

\hypertarget{section-9}{%
\section{10}\label{section-9}}

\bibleverse{1} Hermanos, el deseo de mi corazón y mi oración a Dios es
por Israel, para que se salve. \bibleverse{2} Porque doy testimonio de
ellos de que tienen celo por Dios, pero no según el conocimiento.
\bibleverse{3} Porque ignorando la justicia de Dios, y tratando de
establecer su propia justicia, no se sometieron a la justicia de Dios.

\hypertarget{la-falta-de-israel-es-auxfan-muxe1s-grave-ya-que-dios-no-ha-descuidado-nada-para-llevar-a-israel-a-la-justicia-de-la-fe-desde-la-uxe9poca-de-moisuxe9s}{%
\subsection{La falta de Israel es aún más grave ya que Dios no ha
descuidado nada para llevar a Israel a la justicia de la fe desde la
época de
Moisés}\label{la-falta-de-israel-es-auxfan-muxe1s-grave-ya-que-dios-no-ha-descuidado-nada-para-llevar-a-israel-a-la-justicia-de-la-fe-desde-la-uxe9poca-de-moisuxe9s}}

\bibleverse{4} Porque Cristo es el cumplimiento de la ley para la
justicia de todo el que cree. \footnote{\textbf{10:4} Mat 5,17; Heb
  8,13; Juan 3,18; Gal 3,24-25}

\bibleverse{5} Porque Moisés escribe sobre la justicia de la ley: ``El
que las cumpla vivirá por ellas''. \footnote{\textbf{10:5} Levítico 18:5}
\bibleverse{6} Pero la justicia que es de la fe dice esto: ``No digas en
tu corazón: ``¿Quién subirá al cielo?\footnote{\textbf{10:6}
  Deuteronomio 30:12} (es decir, hacer bajar a Cristo); \bibleverse{7}
o, `¿Quién bajará al abismo?\footnote{\textbf{10:7} Deuteronomio 30:13}
(es decir, hacer subir a Cristo de entre los muertos)''. \bibleverse{8}
Pero, ¿qué dice? ``La palabra está cerca de ti, en tu boca y en tu
corazón''\footnote{\textbf{10:8} Deuteronomio 30:14} , es decir, la
palabra de fe que predicamos: \bibleverse{9} que si confiesas con tu
boca que Jesús es el Señor y crees en tu corazón que Dios lo resucitó de
entre los muertos, te salvarás. \footnote{\textbf{10:9} Mat 10,32; 2Cor
  4,5} \bibleverse{10} Porque con el corazón se cree para obtener la
justicia, y con la boca se confiesa para obtener la salvación.
\bibleverse{11} Porque la Escritura dice: ``El que cree en él no quedará
defraudado''. \footnote{\textbf{10:11} Isaías 28:16}

\bibleverse{12} Porque no hay distinción entre judío y griego, pues el
mismo Señor es Señor de todos, y es rico para todos los que le invocan.
\footnote{\textbf{10:12} Hech 10,34-35; Hech 15,9} \bibleverse{13}
Porque ``Todo el que invoque el nombre del Señor se salvará''.
\footnote{\textbf{10:13} Joel 2:32} \bibleverse{14} ¿Cómo, pues,
invocarán a aquel en quien no han creído? ¿Cómo creerán en él si no han
oído? ¿Cómo oirán sin un predicador? \bibleverse{15} ¿Y cómo van a
predicar si no son enviados? Como está escrito: ``Qué hermosos son los
pies de los que anuncian la Buena Nueva de la paz, que traen buenas
noticias''. \footnote{\textbf{10:15} Isaías 52:7}

\hypertarget{la-inexcusabilidad-de-la-parte-incruxe9dula-de-israel-que-ha-rechazado-la-salvaciuxf3n-que-tambiuxe9n-le-fue-ofrecida}{%
\subsection{La inexcusabilidad de la parte incrédula de Israel, que ha
rechazado la salvación que también le fue
ofrecida}\label{la-inexcusabilidad-de-la-parte-incruxe9dula-de-israel-que-ha-rechazado-la-salvaciuxf3n-que-tambiuxe9n-le-fue-ofrecida}}

\bibleverse{16} Pero no todos escucharon las buenas noticias. Porque
Isaías dice: ``Señor, ¿quién ha creído en nuestro informe?'' \footnote{\textbf{10:16}
  Isaías 53:1} \bibleverse{17} Así que la fe viene por el oír, y el oír
por la palabra de Dios. \footnote{\textbf{10:17} Juan 17,20}
\bibleverse{18} Pero yo digo, ¿no escucharon? Sí, ciertamente, ``Su
sonido se extendió por toda la tierra, sus palabras hasta los confines
del mundo\footnote{\textbf{10:18} Salmo 19:4} ''. \footnote{\textbf{10:18}
  Rom 15,19}

\bibleverse{19} Pero yo pregunto, ¿no lo sabía Israel? Primero dice
Moisés, ``Te provocaré a los celos con lo que no es una nación. Te haré
enfadar con una nación vacía de entendimiento\footnote{\textbf{10:19}
  Deuteronomio 32:21} ''.

\bibleverse{20} Isaías es muy audaz y dice, ``Me encontraron los que no
me buscaron. Me revelé a los que no preguntaron por mí\footnote{\textbf{10:20}
  Isaías 65:1} ''. \footnote{\textbf{10:20} Rom 9,30}

\bibleverse{21} Pero sobre Israel dice: ``Todo el día extendí mis manos
a un pueblo desobediente y contrario''. \footnote{\textbf{10:21} Isaías
  65:2}

\hypertarget{la-mayor-parte-de-los-juduxedos-es-terca-y-rechazada-por-dios-pero-incluso-ahora-una-pequeuxf1a-parte-estuxe1-destinada-a-la-salvaciuxf3n-a-travuxe9s-de-la-gracia-de-dios}{%
\subsection{La mayor parte de los judíos es terca y rechazada por Dios,
pero incluso ahora una pequeña parte está destinada a la salvación a
través de la gracia de
Dios}\label{la-mayor-parte-de-los-juduxedos-es-terca-y-rechazada-por-dios-pero-incluso-ahora-una-pequeuxf1a-parte-estuxe1-destinada-a-la-salvaciuxf3n-a-travuxe9s-de-la-gracia-de-dios}}

\hypertarget{section-10}{%
\section{11}\label{section-10}}

\bibleverse{1} Pregunto entonces, ¿rechazó Dios a su pueblo? ¡Que nunca
lo haga! Porque yo también soy israelita, descendiente de Abraham, de la
tribu de Benjamín. \footnote{\textbf{11:1} Sal 94,14; Jer 31,37; Fil 3,5}
\bibleverse{2} Dios no rechazó a su pueblo, al que conoció de antemano.
¿O no sabes lo que dice la Escritura sobre Elías? Cómo suplica a Dios
contra Israel: \bibleverse{3} ``Señor, han matado a tus profetas. Han
derribado tus altares. Me han dejado solo, y buscan mi vida''.
\footnote{\textbf{11:3} 1 Reyes 19:10,14} \bibleverse{4} ¿Pero cómo le
responde Dios? ``Me he reservado siete mil hombres que no han doblado la
rodilla ante Baal''. \footnote{\textbf{11:4} 1 Reyes 19:18}
\bibleverse{5} Así también en este tiempo hay un remanente según la
elección de la gracia. \footnote{\textbf{11:5} Rom 9,27} \bibleverse{6}
Y si es por gracia, ya no es por obras; de lo contrario, la gracia ya no
es gracia. Pero si es por obras, ya no es gracia; de lo contrario, la
obra ya no es obra.

\bibleverse{7} ¿Qué es entonces? Lo que Israel busca, eso no lo obtuvo,
pero los elegidos lo obtuvieron, y los demás se endurecieron.
\footnote{\textbf{11:7} Rom 9,31} \bibleverse{8} Como está escrito:
``Dios les dio un espíritu de estupor, ojos para no ver y oídos para no
oír, hasta el día de hoy.'' \footnote{\textbf{11:8} Deuteronomio 29:4;
  Isaías 29:10} \footnote{\textbf{11:8} Deut 29,4}

\bibleverse{9} David dice, ``Que su mesa se convierta en un lazo, en una
trampa, un tropiezo, y una retribución para ellos. \bibleverse{10} Que
se les oscurezcan los ojos para que no vean. Mantengan siempre la
espalda doblada''.\footnote{\textbf{11:10} Salmo 69:22,23}

\hypertarget{el-propuxf3sito-divino-de-la-salvaciuxf3n-en-el-llamado-de-los-gentiles-era-vencer-la-incredulidad-de-los-juduxedos-estimuluxe1ndolos-a-emularlos-su-rechazo-no-es-definitivo}{%
\subsection{El propósito divino de la salvación en el llamado de los
gentiles era vencer la incredulidad de los judíos estimulándolos a
emularlos; su rechazo no es
definitivo}\label{el-propuxf3sito-divino-de-la-salvaciuxf3n-en-el-llamado-de-los-gentiles-era-vencer-la-incredulidad-de-los-juduxedos-estimuluxe1ndolos-a-emularlos-su-rechazo-no-es-definitivo}}

\bibleverse{11} Pregunto entonces, ¿acaso tropezaron para caer? ¡Que
nunca sea así! Pero por su caída ha llegado la salvación a los gentiles,
para provocarles celos. \footnote{\textbf{11:11} Hech 13,46; Rom 10,19;
  Deut 32,21} \bibleverse{12} Ahora bien, si su caída es la riqueza del
mundo, y su pérdida la riqueza de los gentiles, ¡cuánto más su plenitud!

\bibleverse{13} Porque a vosotros, que sois gentiles, os hablo. Pues
como soy apóstol de los gentiles, glorifico mi ministerio,
\bibleverse{14} por si de algún modo provoco celos a los que son de mi
carne, y puedo salvar a algunos de ellos. \footnote{\textbf{11:14} 1Tim
  4,16; 1Cor 9,20-22} \bibleverse{15} Porque si el rechazo de ellos es
la reconciliación del mundo, ¿qué sería su aceptación, sino la vida de
entre los muertos?

\bibleverse{16} Si las primicias son santas, también lo es la masa. Si
la raíz es santa, también lo son las ramas. \bibleverse{17} Pero si
algunas de las ramas fueron desgajadas, y tú, siendo un olivo silvestre,
fuiste injertado entre ellas y te hiciste partícipe con ellas de la raíz
y de la riqueza del olivo, \footnote{\textbf{11:17} Efes 2,11-14}
\bibleverse{18} no te jactes de las ramas. Pero si te jactas, recuerda
que no eres tú quien sostiene la raíz, sino que la raíz te sostiene a
ti. \footnote{\textbf{11:18} Juan 4,22} \bibleverse{19} Entonces dirás:
``Las ramas fueron cortadas para que yo fuera injertado''.
\bibleverse{20} Es cierto; por su incredulidad fueron desgajados, y tú
te mantienes por tu fe. No te envanezcas, sino teme; \footnote{\textbf{11:20}
  1Cor 10,12} \bibleverse{21} porque si Dios no perdonó a las ramas
naturales, tampoco te perdonará a ti. \bibleverse{22} Ved, pues, la
bondad y la severidad de Dios. Con los que cayeron, la severidad; pero
con ustedes, la bondad, si continúan en su bondad; de lo contrario,
también ustedes serán cortados. \footnote{\textbf{11:22} Juan 15,2; Juan
  15,4; Heb 3,14} \bibleverse{23} También ellos, si no continúan en su
incredulidad, serán injertados, pues Dios puede volver a injertarlos.
\bibleverse{24} Porque si vosotros fuisteis cortados de lo que es por
naturaleza un olivo silvestre, y fuisteis injertados contra natura en un
buen olivo, ¿cuánto más éstos, que son las ramas naturales, serán
injertados en su propio olivo?

\hypertarget{todo-el-resto-del-pueblo-de-israel-eventualmente-llegaruxe1-a-la-fe-despuuxe9s-de-que-las-elecciones-gentiles-se-conviertan-y-todo-seruxe1-usado-para-la-justificaciuxf3n-y-glorificaciuxf3n-de-dios}{%
\subsection{Todo el resto del pueblo de Israel eventualmente llegará a
la fe después de que las elecciones gentiles se conviertan, y todo será
usado para la justificación y glorificación de
Dios}\label{todo-el-resto-del-pueblo-de-israel-eventualmente-llegaruxe1-a-la-fe-despuuxe9s-de-que-las-elecciones-gentiles-se-conviertan-y-todo-seruxe1-usado-para-la-justificaciuxf3n-y-glorificaciuxf3n-de-dios}}

\bibleverse{25} Porque no quiero que ignoréis, hermanos, este misterio,
para que no seáis sabios en vuestra propia opinión, de que a Israel le
ha sucedido un endurecimiento parcial, hasta que haya entrado la
plenitud de los gentiles, \footnote{\textbf{11:25} Juan 10,16}
\bibleverse{26} y así se salve todo Israel. Como está escrito, ``Saldrá
de Sión el Libertador, y apartará la impiedad de Jacob. \footnote{\textbf{11:26}
  Mat 23,39; Sal 14,7} \bibleverse{27} Este es mi pacto con ellos,
cuando les quite sus pecados\footnote{\textbf{11:27} Isaías 59:20-21;
  27:9; Jeremías 31:33-34} ''.

\bibleverse{28} En cuanto a la Buena Nueva, son enemigos por causa de
ustedes. Pero en cuanto a la elección, son amados por causa de los
padres. \bibleverse{29} Porque los dones y la llamada de Dios son
irrevocables. \footnote{\textbf{11:29} Núm 23,19} \bibleverse{30} Porque
así como vosotros en el pasado fuisteis desobedientes a Dios, pero ahora
habéis obtenido misericordia por su desobediencia, \bibleverse{31} así
también éstos han sido ahora desobedientes, para que por la misericordia
que se os ha mostrado, obtengan también misericordia. \bibleverse{32}
Porque Dios ha obligado a todos a la desobediencia, para tener
misericordia de todos. \footnote{\textbf{11:32} Gal 3,22; 1Tim 2,4}

\bibleverse{33} ¡Oh, la profundidad de las riquezas de la sabiduría y
del conocimiento de Dios! ¡Cuán inescrutable son sus juicios, y sus
caminos que no pueden ser trazados! \footnote{\textbf{11:33} Is 45,15;
  Is 55,8-9} \bibleverse{34} ``Porque ¿quién ha conocido la mente del
Señor? ¿O quién ha sido su consejero?'' \footnote{\textbf{11:34} Isaías
  40:13} \footnote{\textbf{11:34} Jer 23,18; 1Cor 2,16} \bibleverse{35}
``O quien le ha dado primero, y le será devuelto de nuevo?'' \footnote{\textbf{11:35}
  Job 41:11}

\bibleverse{36} Porque de él, por él y para él son todas las cosas. A él
sea la gloria por los siglos de los siglos. Amén.

\hypertarget{advertencia-general-como-entrada-santificaciuxf3n-de-la-vida-personal-a-travuxe9s-de-la-entrega-completa-a-dios}{%
\subsection{Advertencia general como entrada: santificación de la vida
personal a través de la entrega completa a
Dios}\label{advertencia-general-como-entrada-santificaciuxf3n-de-la-vida-personal-a-travuxe9s-de-la-entrega-completa-a-dios}}

\hypertarget{section-11}{%
\section{12}\label{section-11}}

\bibleverse{1} Por lo tanto, os exhorto, hermanos, por la misericordia
de Dios, a que presentéis vuestros cuerpos en sacrificio vivo, santo,
agradable a Dios, que es vuestro servicio espiritual. \footnote{\textbf{12:1}
  Rom 6,13} \bibleverse{2} No os conforméis a este mundo, sino
transformaos mediante la renovación de vuestra mente, para que podáis
comprobar cuál es la buena, agradable y perfecta voluntad de Dios.
\footnote{\textbf{12:2} Efes 4,23; Efes 5,10; Efes 5,17}

\hypertarget{exhortaciuxf3n-a-la-humildad-del-individuo-y-al-uso-fiel-de-los-dones-recibidos-al-servicio-de-la-comunidad}{%
\subsection{Exhortación a la humildad del individuo y al uso fiel de los
dones recibidos al servicio de la
comunidad}\label{exhortaciuxf3n-a-la-humildad-del-individuo-y-al-uso-fiel-de-los-dones-recibidos-al-servicio-de-la-comunidad}}

\bibleverse{3} Pues digo, por la gracia que me ha sido dada, a todos los
que están entre vosotros, que no tengan más alto concepto de sí mismos
que el que deben tener, sino que piensen razonablemente, según la medida
de fe que Dios ha repartido a cada uno. \footnote{\textbf{12:3} 1Cor
  4,6; 1Cor 12,11; Efes 4,7; Mat 20,26} \bibleverse{4} Porque así como
tenemos muchos miembros en un solo cuerpo, y no todos los miembros
tienen la misma función, \footnote{\textbf{12:4} 1Cor 12,12}
\bibleverse{5} así nosotros, que somos muchos, somos un solo cuerpo en
Cristo, y cada uno es miembro del otro, \footnote{\textbf{12:5} 1Cor
  12,27; Efes 4,4; Efes 4,25} \bibleverse{6} teniendo dones diferentes
según la gracia que nos fue dada: si de profecía, profeticemos según la
proporción de nuestra fe; \footnote{\textbf{12:6} 1Cor 4,7; 1Cor 12,4}
\bibleverse{7} o de servicio, entreguémonos al servicio; o el que
enseña, a su enseñanza; \footnote{\textbf{12:7} 1Pe 4,10-11}
\bibleverse{8} o el que exhorta, a su exhortación; el que da, que lo
haga con generosidad; el que gobierna, con diligencia; el que hace
misericordia, con alegría. \footnote{\textbf{12:8} Mat 6,3; 2Cor 8,2;
  2Cor 9,7}

\hypertarget{exhortaciuxf3n-a-amar-fraternalmente-y-a-ejercitar-sentimientos-cristianos-contra-amigos-y-enemigos}{%
\subsection{Exhortación a amar fraternalmente y a ejercitar sentimientos
cristianos contra amigos y
enemigos}\label{exhortaciuxf3n-a-amar-fraternalmente-y-a-ejercitar-sentimientos-cristianos-contra-amigos-y-enemigos}}

\bibleverse{9} Que el amor sea sin hipocresía. Aborrece lo que es malo.
Aferraos a lo que es bueno. \footnote{\textbf{12:9} 1Tim 1,5; Am 5,15}
\bibleverse{10} En el amor a los hermanos, sed tiernos los unos con los
otros; en la honra, preferíos los unos a los otros, \footnote{\textbf{12:10}
  Juan 13,4-15; Fil 2,3} \bibleverse{11} no dejéis de ser diligentes,
fervientes en el espíritu, sirviendo al Señor, \footnote{\textbf{12:11}
  Apoc 3,15; Hech 18,25; Col 3,23} \bibleverse{12} alegrándoos en la
esperanza, soportando en las tribulaciones, perseverando en la oración,
\footnote{\textbf{12:12} 1Tes 5,17; Luc 18,1-8; Col 4,2} \bibleverse{13}
contribuyendo a las necesidades de los santos, y dados a la
hospitalidad. \footnote{\textbf{12:13} Heb 13,2; 3Jn 1,5-8}

\bibleverse{14} Bendice a los que te persiguen; bendice y no maldigas.
\footnote{\textbf{12:14} Mat 5,44; 1Cor 4,12; Hech 7,59} \bibleverse{15}
Alegraos con los que se alegran. Llorad con los que lloran. \footnote{\textbf{12:15}
  Sal 35,13-14; 2Cor 11,29} \bibleverse{16} Tened los mismos
sentimientos los unos hacia los otros. No sean altivos en su pensar,
sino asociaros con los humildes. No seáis sabios en vuestras propias
ideas. \footnote{\textbf{12:16} Rom 15,5; Fil 2,2} \bibleverse{17} No
paguéis a nadie mal por mal. Respetad lo que es honorable a los ojos de
todos los hombres. \footnote{\textbf{12:17} Is 5,21; 1Tes 5,15; Prov
  20,22; 2Cor 8,21} \bibleverse{18} Si es posible, en la medida en que
dependa de vosotros, estad en paz con todos los hombres. \footnote{\textbf{12:18}
  Mar 9,50; Heb 12,14} \bibleverse{19} No busquéis la venganza vosotros
mismos, amados, sino dad lugar a la ira de Dios. Porque está escrito:
``La venganza me pertenece; yo pagaré, dice el Señor''. \footnote{\textbf{12:19}
  Deuteronomio 32:35} \footnote{\textbf{12:19} Lev 19,18; Mat 5,38-44}
\bibleverse{20} Por eso``Si tu enemigo tiene hambre, aliméntalo. Si
tiene sed, dale de beber; porque al hacerlo, amontonarás carbones de
fuego sobre su cabeza''. \footnote{\textbf{12:20} Proverbios 25:21-22}
\footnote{\textbf{12:20} 2Re 6,22}

\bibleverse{21} No te dejes vencer por el mal, sino vence el mal con el
bien.

\hypertarget{exhortaciuxf3n-a-obedecer-a-las-autoridades-designadas-por-dios}{%
\subsection{Exhortación a obedecer a las autoridades designadas por
Dios}\label{exhortaciuxf3n-a-obedecer-a-las-autoridades-designadas-por-dios}}

\hypertarget{section-12}{%
\section{13}\label{section-12}}

\bibleverse{1} Que toda alma se someta a las autoridades superiores,
porque no hay autoridad sino de Dios, y las que hay son ordenadas por
Dios. \footnote{\textbf{13:1} Tit 3,1; Juan 19,11; Prov 8,15}
\bibleverse{2} Por lo tanto, el que resiste a la autoridad resiste la
ordenanza de Dios; y los que resisten recibirán para sí el juicio.
\bibleverse{3} Porque los gobernantes no son un terror para la buena
obra, sino para la mala. ¿Deseas no tener miedo a la autoridad? Haced lo
que es bueno, y tendréis la alabanza de la autoridad, \footnote{\textbf{13:3}
  1Pe 2,13-14} \bibleverse{4} porque es un servidor de Dios para
vosotros para el bien. Pero si hacéis lo que es malo, temed, porque no
lleva la espada en vano, pues es un servidor de Dios, vengador para la
ira del que hace el mal. \footnote{\textbf{13:4} 2Cró 19,6-7}
\bibleverse{5} Por tanto, es necesario que estéis sometidos, no sólo por
la ira, sino también por la conciencia. \bibleverse{6} Por eso también
pagas los impuestos, pues son servidores del servicio de Dios, haciendo
continuamente esto mismo.

\hypertarget{exhortaciones-al-cumplimiento-integral-de-los-deberes-especialmente-a-la-caridad-como-cumplimiento-de-la-ley}{%
\subsection{Exhortaciones al cumplimiento integral de los deberes,
especialmente a la caridad como cumplimiento de la
ley}\label{exhortaciones-al-cumplimiento-integral-de-los-deberes-especialmente-a-la-caridad-como-cumplimiento-de-la-ley}}

\bibleverse{7} Por tanto, dad a cada uno lo que debéis: si debéis
impuestos, pagad impuestos; si tributo, tributo; si respeto, respeto; si
honor, honor. \footnote{\textbf{13:7} Mat 22,21}

\bibleverse{8} No debáis a nadie nada, sino amaros unos a otros; porque
el que ama a su prójimo ha cumplido la ley. \footnote{\textbf{13:8} Gal
  5,14; 1Tim 1,5} \bibleverse{9} Porque los mandamientos: ``No cometerás
adulterio'', ``No asesinarás'', ``No robarás\footnote{\textbf{13:9} TR
  añade ``No darás falso testimonio''.} '', ``No codiciarás''\footnote{\textbf{13:9}
  Éxodo 20:13-15,17; Deuteronomio 5:17-19,21} y cualquier otro que haya,
se resumen en esta frase: ``Amarás a tu prójimo como a ti
mismo''.\footnote{\textbf{13:9} Levítico 19:18} \bibleverse{10} El amor
no hace daño al prójimo. Por tanto, el amor es el cumplimiento de la
ley. \footnote{\textbf{13:10} 1Cor 13,4; Mat 22,40}

\hypertarget{el-fin-cercano-del-mundo-advierte-caminar-en-luz-y-santificar-la-vida-personal}{%
\subsection{El fin cercano del mundo advierte caminar en luz y
santificar la vida
personal}\label{el-fin-cercano-del-mundo-advierte-caminar-en-luz-y-santificar-la-vida-personal}}

\bibleverse{11} Haced esto, conociendo el tiempo, que ya es hora de que
os despertéis del sueño, porque la salvación está ahora más cerca de
nosotros que cuando creímos por primera vez. \footnote{\textbf{13:11}
  Efes 5,14; 1Tes 5,6-8} \bibleverse{12} La noche está lejos, y el día
está cerca. Despojémonos, pues, de las obras de las tinieblas y
pongámonos la armadura de la luz. \footnote{\textbf{13:12} 1Jn 2,8; Efes
  5,11} \bibleverse{13} Caminemos correctamente, como de día; no en
juergas y borracheras, no en promiscuidades sexuales y actos lujuriosos,
y no en contiendas y envidias. \footnote{\textbf{13:13} Luc 21,34; Efes
  5,18} \bibleverse{14} Sino revestíos del Señor Jesucristo, y no
proveáis para la carne, para sus concupiscencias. \footnote{\textbf{13:14}
  Gal 3,27; 1Cor 9,27; Col 2,23}

\hypertarget{juicio-sobre-el-tema-que-conmueve-a-la-comunidad-y-advierte-contra-la-condena-sin-amor-del-modo-de-vida-externo-del-pruxf3jimo}{%
\subsection{Juicio sobre el tema que conmueve a la comunidad y advierte
contra la condena sin amor del modo de vida externo del
prójimo}\label{juicio-sobre-el-tema-que-conmueve-a-la-comunidad-y-advierte-contra-la-condena-sin-amor-del-modo-de-vida-externo-del-pruxf3jimo}}

\hypertarget{section-13}{%
\section{14}\label{section-13}}

\bibleverse{1} Ahora bien, acepta al que es débil en la fe, pero no por
disputas de opiniones. \footnote{\textbf{14:1} Rom 15,1; 1Cor 8,9}
\bibleverse{2} Un hombre tiene fe para comer de todo, pero el que es
débil sólo come verduras. \footnote{\textbf{14:2} Gén 1,29; Gén 9,3}
\bibleverse{3} Que el que come no desprecie al que no come. Que el que
no come no juzgue al que come, porque Dios lo ha aceptado. \footnote{\textbf{14:3}
  Col 2,16} \bibleverse{4} ¿Quiénes sois vosotros para juzgar al siervo
de otro? A su propio señor le hace frente o le hace caer. Sí, se pondrá
en pie, pues Dios tiene poder para hacerlo. \footnote{\textbf{14:4} Mat
  7,1; Sant 4,11; Sant 1,4-12}

\bibleverse{5} Un hombre considera que un día es más importante. Otro
considera que todos los días son iguales. Que cada uno esté bien seguro
en su propia mente. \footnote{\textbf{14:5} Gal 4,10} \bibleverse{6} El
que observa el día, lo observa para el Señor; y el que no lo observa,
para el Señor no lo observa. El que come, come para el Señor, porque da
gracias a Dios. El que no come, para el Señor no come, y da gracias a
Dios. \bibleverse{7} Porque ninguno de nosotros vive para sí mismo, y
ninguno muere para sí mismo. \bibleverse{8} Pues si vivimos, vivimos
para el Señor. O si morimos, morimos para el Señor. Por lo tanto, si
vivimos o morimos, somos del Señor. \footnote{\textbf{14:8} Gal 2,20;
  2Cor 5,15} \bibleverse{9} Porque para ello Cristo murió, resucitó y
volvió a vivir, para ser Señor tanto de los muertos como de los vivos.

\bibleverse{10} Pero tú, ¿por qué juzgas a tu hermano? O tú, ¿por qué
desprecias a tu hermano? Porque todos compareceremos ante el tribunal de
Cristo. \footnote{\textbf{14:10} Mat 25,31-32; Hech 17,31; 2Cor 5,10}
\bibleverse{11} Porque está escrito, ``\,`Vivo yo', dice el Señor, `ante
mí se doblará toda rodilla'. Toda lengua confesará a Dios''. \footnote{\textbf{14:11}
  Isaías 45:23} \footnote{\textbf{14:11} Fil 2,10-11}

\bibleverse{12} Así pues, cada uno de nosotros dará cuenta de sí mismo a
Dios. \footnote{\textbf{14:12} Gal 6,5}

\hypertarget{exhortaciuxf3n-a-los-que-tienen-una-fe-fuerte-a-no-ofender-a-los-que-tienen-una-fe-duxe9bil-y-a-esforzarse-por-tener-una-conciencia-segura-en-todo-lo-que-hacen}{%
\subsection{Exhortación a los que tienen una fe fuerte a no ofender a
los que tienen una fe débil y a esforzarse por tener una conciencia
segura en todo lo que
hacen}\label{exhortaciuxf3n-a-los-que-tienen-una-fe-fuerte-a-no-ofender-a-los-que-tienen-una-fe-duxe9bil-y-a-esforzarse-por-tener-una-conciencia-segura-en-todo-lo-que-hacen}}

\bibleverse{13} Por lo tanto, no juzguemos más los unos a los otros,
sino juzguemos más bien esto: que ninguno ponga tropiezo a su hermano,
ni sea ocasión de caer. \footnote{\textbf{14:13} 1Cor 10,33}
\bibleverse{14} Yo sé y estoy persuadido en el Señor Jesús de que nada
es inmundo por sí mismo, sino que para el que considera que algo es
inmundo, para él es inmundo. \footnote{\textbf{14:14} Mat 15,11; Hech
  10,15; Tit 1,15} \bibleverse{15} Pero si por causa de la comida tu
hermano se entristece, ya no andas con amor. No destruyas con tu comida
a aquel por quien murió Cristo. \footnote{\textbf{14:15} 1Cor 8,11-13}
\bibleverse{16} Entonces no permitas que se calumnie tu bien,
\bibleverse{17} porque el Reino de Dios no es comer ni beber, sino
justicia, paz y alegría en el Espíritu Santo. \footnote{\textbf{14:17}
  1Cor 8,8; Heb 13,9} \bibleverse{18} Porque el que sirve a Cristo en
estas cosas es agradable a Dios y aprobado por los hombres.
\bibleverse{19} Sigamos, pues, las cosas que contribuyen a la paz y a la
edificación mutua. \footnote{\textbf{14:19} Rom 12,18; Rom 15,2}
\bibleverse{20} No echéis por tierra la obra de Dios por causa de la
comida. Todas las cosas, en efecto, son limpias; sin embargo, es malo el
hombre que crea un tropiezo al comer. \bibleverse{21} Es bueno no comer
carne, ni beber vino, ni hacer nada por lo que tu hermano tropiece, se
ofenda o se debilite.

\bibleverse{22} ¿Tienes fe? Tenla para ti mismo ante Dios. Dichoso el
que no se juzga a sí mismo en lo que aprueba. \footnote{\textbf{14:22}
  Rom 14,2; 1Cor 10,25-27} \bibleverse{23} Pero el que duda se condena
si come, porque no es de fe; y todo lo que no es de fe es pecado.

\bibleverse{24} Ahora bien, a aquel que es capaz de afianzaros según mi
Buena Nueva y la predicación de Jesucristo, según la revelación del
misterio que se ha mantenido en secreto durante largos siglos,
\footnote{\textbf{14:24} Efes 1,9; Efes 3,4-9; 1Cor 2,7} \bibleverse{25}
pero que ahora se revela, y por las Escrituras de los profetas, según el
mandamiento del Dios eterno, se da a conocer para la obediencia de la fe
a todas las naciones; \footnote{\textbf{14:25} Rom 1,5} \bibleverse{26}
al único Dios sabio, por medio de Jesucristo, a quien sea la gloria por
los siglos. Amén. \footnote{\textbf{14:26} TR coloca los versos 24-26
  después de Romanos 16:24 como versos 25-27.}

\hypertarget{exhortaciuxf3n-a-ser-pacientes-con-los-duxe9biles-y-a-la-unidad-de-los-cristianos-basada-en-el-ejemplo-de-cristo}{%
\subsection{Exhortación a ser pacientes con los débiles y a la unidad de
los cristianos basada en el ejemplo de
Cristo}\label{exhortaciuxf3n-a-ser-pacientes-con-los-duxe9biles-y-a-la-unidad-de-los-cristianos-basada-en-el-ejemplo-de-cristo}}

\hypertarget{section-14}{%
\section{15}\label{section-14}}

\bibleverse{1} Ahora bien, los que somos fuertes debemos soportar las
debilidades de los débiles, y no complacernos a nosotros mismos.
\footnote{\textbf{15:1} Rom 14,1} \bibleverse{2} Cada uno de nosotros
debe complacer a su prójimo en lo que es bueno, para ir edificándolo.
\footnote{\textbf{15:2} 1Cor 9,19; 1Cor 10,24; 1Cor 10,33}
\bibleverse{3} Porque ni siquiera Cristo se complació a sí mismo. Sino
que, como está escrito, ``los reproches de los que os reprochaban
cayeron sobre mí''. \footnote{\textbf{15:3} Salmo 69:9} \bibleverse{4}
Porque todo lo que se ha escrito antes, se ha escrito para que
aprendamos, a fin de que, mediante la perseverancia y el estímulo de las
Escrituras, tengamos esperanza. \footnote{\textbf{15:4} 1Cor 10,11}
\bibleverse{5} Ahora bien, el Dios de la perseverancia y del estímulo os
conceda que tengáis un mismo sentir los unos con los otros según Cristo
Jesús, \footnote{\textbf{15:5} Fil 2,2} \bibleverse{6} para que unánimes
glorifiquéis con una sola boca al Dios y Padre de nuestro Señor
Jesucristo.

\hypertarget{un-recordatorio-para-que-ambas-partes-de-la-comunidad-estuxe9n-unidas-y-tengan-una-fe-gozosa}{%
\subsection{Un recordatorio para que ambas partes de la comunidad estén
unidas y tengan una fe
gozosa}\label{un-recordatorio-para-que-ambas-partes-de-la-comunidad-estuxe9n-unidas-y-tengan-una-fe-gozosa}}

\bibleverse{7} Por tanto, aceptaos los unos a los otros, como también
Cristo os aceptó a vosotros,\footnote{\textbf{15:7} TR lee ``nosotros''
  en lugar de ``vosotros''} para gloria de Dios. \bibleverse{8} Ahora
bien, digo que Cristo se ha hecho siervo de la circuncisión por la
verdad de Dios, para confirmar las promesas dadas a los padres,
\footnote{\textbf{15:8} Mat 15,24; Hech 3,25} \bibleverse{9} y para que
los gentiles glorifiquen a Dios por su misericordia. Como está escrito,
``Por eso te alabaré entre los gentilesy cantar a tu nombre\footnote{\textbf{15:9}
  2 Samuel 22:50; Salmo 18:49} ''.

\bibleverse{10} De nuevo dice, ``Alegraos, gentiles, con su pueblo''.
\footnote{\textbf{15:10} Deuteronomio 32:43}

\bibleverse{11} de nuevo, ``¡Alabad al Señor, todos los gentiles! Que
todos los pueblos lo alaben\footnote{\textbf{15:11} Salmo 117:1} ''.

\bibleverse{12} De nuevo, Isaías dice, ``Habrá la raíz de Jesse, el que
se levanta para gobernar a los gentiles; en él esperarán los
gentiles\footnote{\textbf{15:12} Isaías 11:10} ''. \footnote{\textbf{15:12}
  Apoc 5,5}

\bibleverse{13} Que el Dios de la esperanza os llene de toda alegría y
paz en la fe, para que abundéis en la esperanza con la fuerza del
Espíritu Santo.

\hypertarget{revisiuxf3n-justificativa-del-apuxf3stol-de-la-carta-y-referencia-a-su-oficio-apostuxf3lico-para-los-gentiles}{%
\subsection{Revisión justificativa del apóstol de la carta y referencia
a su oficio apostólico para los
gentiles}\label{revisiuxf3n-justificativa-del-apuxf3stol-de-la-carta-y-referencia-a-su-oficio-apostuxf3lico-para-los-gentiles}}

\bibleverse{14} Yo mismo estoy persuadido de vosotros, hermanos míos, de
que vosotros mismos estáis llenos de bondad, llenos de todo
conocimiento, capaces también de amonestar a los demás. \bibleverse{15}
Pero os escribo con mayor audacia, en parte como recordatorio, por la
gracia que me ha sido concedida por Dios, \footnote{\textbf{15:15} Rom
  1,5; Rom 12,3} \bibleverse{16} para ser siervo de Cristo Jesús para
los gentiles, sirviendo como sacerdote de la Buena Nueva de Dios, para
que la ofrenda de los gentiles sea aceptable, santificada por el
Espíritu Santo. \footnote{\textbf{15:16} Rom 11,13} \bibleverse{17}
Tengo, pues, mi jactancia en Cristo Jesús en las cosas que pertenecen a
Dios. \bibleverse{18} Porque no me atreveré a hablar de ninguna cosa,
sino de las que Cristo ha obrado por medio de mí para la obediencia de
los gentiles, con palabras y con hechos, \footnote{\textbf{15:18} 2Cor
  3,5; Rom 1,5} \bibleverse{19} con el poder de las señales y de los
prodigios, con el poder del Espíritu de Dios; de modo que desde
Jerusalén y alrededor hasta Ilírico, he predicado plenamente la Buena
Nueva de Cristo; \footnote{\textbf{15:19} Mar 16,17; 2Cor 12,12}
\bibleverse{20} sí, poniendo como objetivo predicar la Buena Nueva, no
donde ya se nombraba a Cristo, para no edificar sobre fundamento ajeno.
\footnote{\textbf{15:20} 2Cor 10,15-16} \bibleverse{21} Pero, como está
escrito, ``Verán, a quienes no les llegó ninguna noticia de él. Los que
no han oído lo entenderán\footnote{\textbf{15:21} Isaías 52:15} ''.

\hypertarget{anuncio-de-los-pruxf3ximos-planes-de-viaje-del-apuxf3stol}{%
\subsection{Anuncio de los próximos planes de viaje del
apóstol}\label{anuncio-de-los-pruxf3ximos-planes-de-viaje-del-apuxf3stol}}

\bibleverse{22} Por eso también me han impedido estas muchas veces ir a
ti, \footnote{\textbf{15:22} Rom 1,13} \bibleverse{23} pero ahora, no
teniendo ya lugar en estas regiones, y teniendo estos muchos años el
anhelo de ir a ti, \footnote{\textbf{15:23} Rom 1,10-11} \bibleverse{24}
siempre que viaje a España, iré a ti. Porque espero veros en mi viaje y
que me ayudéis en mi camino, si antes puedo disfrutar de vuestra
compañía durante un tiempo. \bibleverse{25} Pero ahora, digo, me voy a
Jerusalén, a servir a los santos. \footnote{\textbf{15:25} Hech 18,21;
  Hech 19,21; Hech 20,22; Hech 24,17} \bibleverse{26} Porque a Macedonia
y Acaya les ha parecido bien hacer una contribución para los pobres de
los santos que están en Jerusalén. \footnote{\textbf{15:26} 1Cor 16,1;
  2Cor 8,1-4; 2Cor 8,9} \bibleverse{27} Sí, les ha parecido bien, y son
sus deudores. Porque si los gentiles han sido hechos partícipes de sus
cosas espirituales, también les deben servir en las cosas materiales.
\footnote{\textbf{15:27} 1Cor 9,11; Gal 6,6} \bibleverse{28} Así pues,
cuando haya cumplido esto y les haya sellado este fruto, seguiré por tu
camino hacia España. \bibleverse{29} Sé que cuando vaya a ustedes, iré
en la plenitud de la bendición de la Buena Nueva de Cristo.

\hypertarget{la-amonestaciuxf3n-del-apuxf3stol-a-la-iglesia-de-que-interceda-por-uxe9l}{%
\subsection{La amonestación del apóstol a la iglesia de que interceda
por
él}\label{la-amonestaciuxf3n-del-apuxf3stol-a-la-iglesia-de-que-interceda-por-uxe9l}}

\bibleverse{30} Ahora os ruego, hermanos, por nuestro Señor Jesucristo y
por el amor del Espíritu, que os esforcéis junto conmigo en vuestras
oraciones a Dios por mí, \footnote{\textbf{15:30} 2Cor 1,11; 2Tes 3,1}
\bibleverse{31} para que sea librado de los desobedientes de Judea, y
para que mi servicio que tengo para Jerusalén sea aceptable a los
santos, \footnote{\textbf{15:31} 1Tes 2,15} \bibleverse{32} para que
pueda llegar a vosotros con alegría por la voluntad de Dios, y junto con
vosotros, encontrar el descanso. \bibleverse{33} El Dios de la paz esté
con todos vosotros. Amén.

\hypertarget{recomendaciuxf3n-de-phuxf6be-portador-de-la-carta-saludos-del-apuxf3stol-a-los-hermanos-en-roma}{%
\subsection{Recomendación de Phöbe, portador de la carta; Saludos del
Apóstol a los hermanos en
Roma}\label{recomendaciuxf3n-de-phuxf6be-portador-de-la-carta-saludos-del-apuxf3stol-a-los-hermanos-en-roma}}

\hypertarget{section-15}{%
\section{16}\label{section-15}}

\bibleverse{1} Os encomiendo a nuestra hermana Febe, que es
sierva\footnote{\textbf{16:1} o, diácono} de la asamblea que está en
Cencreas, \bibleverse{2} para que la recibáis en el Señor de manera
digna de los santos, y la ayudéis en todo lo que necesite de vosotros,
pues ella misma también ha sido ayudante de muchos, y de mí mismo.

\bibleverse{3} Saludad a Prisca y a Aquila, mis colaboradores en Cristo
Jesús, \footnote{\textbf{16:3} Hech 18,2; Hech 18,18; Hech 18,26}
\bibleverse{4} que arriesgaron sus propios cuellos por mi vida, a
quienes no sólo doy gracias, sino también a todas las asambleas de los
gentiles. \bibleverse{5} Saludad a la asamblea que está en su casa.
Saludad a Epaeneto, mi amado, que es la primicia de Acaya para Cristo.
\bibleverse{6} Saludad a María, que ha trabajado mucho por nosotros.
\bibleverse{7} Saludad a Andrónico y a Junia, mis parientes y compañeros
de prisión, que son notables entre los apóstoles, que también estuvieron
en Cristo antes que yo. \bibleverse{8} Saludad a Amplias, mi amado en el
Señor. \bibleverse{9} Saludad a Urbano, nuestro colaborador en Cristo, y
a Estaquis, mi amado. \bibleverse{10} Saludad a Apeles, el aprobado en
Cristo. Saludad a los de la casa de Aristóbulo. \bibleverse{11} Saludad
a Herodión, mi pariente. Saludad a los de la casa de Narciso, que están
en el Señor. \bibleverse{12} Saludad a Trifena y a Trifosa, que trabajan
en el Señor. Saludad a Persis, la amada, que trabaja mucho en el Señor.
\bibleverse{13} Saludad a Rufo, el elegido en el Señor, y a su madre y a
la mía. \footnote{\textbf{16:13} Mar 15,21} \bibleverse{14} Saludad a
Asíncrito, a Flegón, a Hermes, a Patrobas, a Hermas y a los hermanos que
están con ellos. \bibleverse{15} Saludad a Filólogo y a Julia, a Nereo y
a su hermana, a Olimpas y a todos los santos que están con ellos.
\bibleverse{16} Saludaos unos a otros con un beso sagrado. Las asambleas
de Cristo os saludan. \footnote{\textbf{16:16} 1Cor 16,20}

\hypertarget{advertencia-a-los-engauxf1adores-que-causan-divisiones-y-errores-en-la-iglesia}{%
\subsection{Advertencia a los engañadores que causan divisiones y
errores en la
iglesia}\label{advertencia-a-los-engauxf1adores-que-causan-divisiones-y-errores-en-la-iglesia}}

\bibleverse{17} Os ruego, hermanos, que estéis atentos a los que causan
divisiones y ocasiones de tropiezo, en contra de la doctrina que habéis
aprendido, y que os apartéis de ellos. \footnote{\textbf{16:17} Mat
  7,15; Tit 3,10; 2Tes 3,6} \bibleverse{18} Porque los tales no sirven a
nuestro Señor Jesucristo, sino a su propio vientre; y con su discurso
suave y lisonjero engañan los corazones de los inocentes. \footnote{\textbf{16:18}
  Fil 3,19; Col 2,4} \bibleverse{19} Porque tu obediencia ha llegado a
ser conocida por todos. Me alegro, pues, por vosotros. Pero deseo que
seáis sabios en lo que es bueno, pero inocentes en lo que es malo.
\footnote{\textbf{16:19} Rom 1,8; 1Cor 14,20} \bibleverse{20} Y el Dios
de la paz aplastará pronto a Satanás bajo vuestros pies. La gracia de
nuestro Señor Jesucristo esté con vosotros.

\hypertarget{saludos-de-los-amigos-de-pablo-a-roma-y-finalmente-alabanza-a-dios}{%
\subsection{Saludos de los amigos de Pablo a Roma y finalmente alabanza
a
Dios}\label{saludos-de-los-amigos-de-pablo-a-roma-y-finalmente-alabanza-a-dios}}

\bibleverse{21} Os saludan Timoteo, mi colaborador, así como Lucio,
Jasón y Sosípater, mis parientes. \footnote{\textbf{16:21} Hech 16,1-3;
  Hech 17,6; Hech 19,22; Hech 20,4; Fil 2,19-22} \bibleverse{22} Yo,
Tercio, que escribo la carta, os saludo en el Señor. \bibleverse{23} Os
saluda Gayo, mi anfitrión y anfitrión de toda la asamblea. Os saluda
Erasto, el tesorero de la ciudad, y el hermano Cuarto. \footnote{\textbf{16:23}
  1Cor 1,14} \bibleverse{24} ¡La gracia de nuestro Señor Jesucristo esté
con todos vosotros! Amén. \bibleverse{25} \footnote{\textbf{16:25} El TR
  coloca Romanos 14:24-26 al final de Romanos en lugar de al final del
  capítulo 14, y numera estos versículos 16:25-27.}
