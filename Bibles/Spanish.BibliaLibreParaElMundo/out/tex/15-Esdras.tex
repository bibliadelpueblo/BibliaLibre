\hypertarget{el-permiso-de-ciro-para-el-regreso-de-los-juduxedos-restantes-y-para-la-construcciuxf3n-del-templo}{%
\subsection{El permiso de Ciro para el regreso de los judíos restantes y
para la construcción del
templo}\label{el-permiso-de-ciro-para-el-regreso-de-los-juduxedos-restantes-y-para-la-construcciuxf3n-del-templo}}

\hypertarget{section}{%
\section{1}\label{section}}

\bibleverse{1} En el primer año de Ciro, rey de Persia, para que se
cumpliera la palabra de Yavé\footnote{\textbf{1:1} ``Yahvé'' es el
  nombre propio de Dios, a veces traducido como ``\textsc{Señor}'' (en
  mayúsculas) en otras traducciones.} por boca de Jeremías, Yavé
despertó el espíritu de Ciro, rey de Persia, de modo que hizo un anuncio
por todo su reino, y lo puso también por escrito, diciendo, \footnote{\textbf{1:1}
  2Cró 36,22-23; Jer 25,11; Jer 29,10} \bibleverse{2} ``Ciro, rey de
Persia, dice: ``Yahvé, el Dios\footnote{\textbf{1:2} La palabra hebrea
  traducida como ``Dios'' es ``\hebrew{אֱלֹהִ֑ים}'' (Elohim).} del cielo, me
ha dado todos los reinos de la tierra; y me ha ordenado que le construya
una casa en Jerusalén, que está en Judá. \footnote{\textbf{1:2} Is
  44,28; Is 45,1} \bibleverse{3} Quienquiera que haya entre ustedes de
todo su pueblo, que su Dios esté con él, y que suba a Jerusalén, que
está en Judá, y edifique la casa de Yavé, el Dios de Israel (él es
Dios), que está en Jerusalén. \bibleverse{4} El que quede, en cualquier
lugar donde viva, que los hombres de su lugar lo ayuden con plata, con
oro, con bienes y con animales, además de la ofrenda voluntaria para la
casa de Dios que está en Jerusalén''.

\hypertarget{el-efecto-y-ejecuciuxf3n-de-la-disposiciuxf3n}{%
\subsection{El efecto y ejecución de la
disposición}\label{el-efecto-y-ejecuciuxf3n-de-la-disposiciuxf3n}}

\bibleverse{5} Entonces los jefes de familia de Judá y de Benjamín, los
sacerdotes y los levitas, todos cuyos espíritus Dios había incitado a
subir, se levantaron para edificar la casa de Yahvé que está en
Jerusalén. \bibleverse{6} Todos los que los rodeaban reforzaron sus
manos con vasos de plata, con oro, con bienes, con animales y con cosas
preciosas, además de todo lo que se ofrecía voluntariamente.

\hypertarget{publicaciuxf3n-y-listado-de-los-implementos-del-templo-entregados-a-sesbassar-zorobabel}{%
\subsection{Publicación y listado de los implementos del templo
entregados a Sesbassar
(Zorobabel)}\label{publicaciuxf3n-y-listado-de-los-implementos-del-templo-entregados-a-sesbassar-zorobabel}}

\bibleverse{7} También el rey Ciro sacó los utensilios de la casa de
Yavé, que Nabucodonosor había sacado de Jerusalén y había puesto en la
casa de sus dioses; \bibleverse{8} esos los sacó Ciro, rey de Persia,
por mano del tesorero Mitrídates, y los contó a Sesbasar, príncipe de
Judá. \footnote{\textbf{1:8} Esd 2,2; Esd 2,63; Esd 5,14} \bibleverse{9}
Este es el número de ellos: treinta fuentes de oro, mil fuentes de
plata, veintinueve cuchillos, \bibleverse{10} treinta tazones de oro,
cuatrocientos diez tazones de plata de segunda clase, y otros mil
recipientes. \bibleverse{11} Todos los vasos de oro y de plata eran
cinco mil cuatrocientos. Todo esto lo trajo Sesbasar cuando los cautivos
fueron llevados de Babilonia a Jerusalén.

\hypertarget{directorio-de-juduxedos-que-regresan}{%
\subsection{Directorio de judíos que
regresan}\label{directorio-de-juduxedos-que-regresan}}

\hypertarget{section-1}{%
\section{2}\label{section-1}}

\bibleverse{1} Estos son los hijos de la provincia que subieron del
cautiverio de los deportados, que Nabucodonosor, rey de Babilonia, había
llevado a Babilonia, y que volvieron a Jerusalén y a Judá, cada uno a su
ciudad;

\hypertarget{lista-de-repatriados}{%
\subsection{Lista de repatriados}\label{lista-de-repatriados}}

\bibleverse{2} que vinieron con Zorobabel, Jesúa, Nehemías, Seraías,
Reelaías, Mardoqueo, Bilsán, Mispar, Bigvai, Rehum y Baana. El número de
los hombres del pueblo de Israel: \footnote{\textbf{2:2} Esd 1,8; Esd
  2,63} \bibleverse{3} Los hijos de Paros, dos mil ciento setenta y dos.
\bibleverse{4} Los hijos de Sefatías, trescientos setenta y dos.
\bibleverse{5} Los hijos de Ara, setecientos setenta y cinco.
\bibleverse{6} Los hijos de Pahatmoab, de los hijos de Jesúa y de Joab,
dos mil ochocientos doce. \bibleverse{7} Los hijos de Elam, mil
doscientos cincuenta y cuatro. \bibleverse{8} Los hijos de Zattu,
novecientos cuarenta y cinco. \bibleverse{9} Los hijos de Zacarías,
setecientos sesenta. \bibleverse{10} Los hijos de Bani, seiscientos
cuarenta y dos. \bibleverse{11} Los hijos de Bebai, seiscientos
veintitrés. \bibleverse{12} Los hijos de Azgad, mil doscientos
veintidós. \bibleverse{13} Los hijos de Adonikam, seiscientos sesenta y
seis. \bibleverse{14} Los hijos de Bigvai, dos mil cincuenta y seis.
\bibleverse{15} Los hijos de Adin, cuatrocientos cincuenta y cuatro.
\bibleverse{16} Los hijos de Ater, de Ezequías, noventa y ocho.
\bibleverse{17} Los hijos de Bezai, trescientos veintitrés.
\bibleverse{18} Los hijos de Jorah, ciento doce. \bibleverse{19} Los
hijos de Hasum, doscientos veintitrés. \bibleverse{20} Los hijos de
Gibbar, noventa y cinco. \bibleverse{21} Los hijos de Belén, ciento
veintitrés. \bibleverse{22} Los de Netofa, cincuenta y seis.
\bibleverse{23} Los de Anatot, ciento veintiocho. \bibleverse{24} Los
hijos de Azmavet, cuarenta y dos. \bibleverse{25} Los hijos de Quiriat
Arim, Chefira y Beerot, setecientos cuarenta y tres. \bibleverse{26} Los
hijos de Ramá y Geba, seiscientos veintiuno. \bibleverse{27} Los varones
de Micmas, ciento veintidós. \bibleverse{28} Los varones de Betel y de
Hai, doscientos veintitrés. \bibleverse{29} Los hijos de Nebo, cincuenta
y dos. \bibleverse{30} Los hijos de Magbis, ciento cincuenta y seis.
\bibleverse{31} Los hijos del otro Elam, mil doscientos cincuenta y
cuatro. \bibleverse{32} Los hijos de Harim, trescientos veinte.
\bibleverse{33} Los hijos de Lod, Hadid y Ono, setecientos veinticinco.
\bibleverse{34} Los hijos de Jericó, trescientos cuarenta y cinco.
\bibleverse{35} Los hijos de Senaa, tres mil seiscientos treinta.

\bibleverse{36} Los sacerdotes: los hijos de Jedaías, de la casa de
Jesúa, novecientos setenta y tres. \bibleverse{37} Los hijos de Immer,
mil cincuenta y dos. \bibleverse{38} Los hijos de Pashur, mil doscientos
cuarenta y siete. \bibleverse{39} Los hijos de Harim, mil diecisiete.

\bibleverse{40} Los levitas: los hijos de Jesúa y de Cadmiel, de los
hijos de Hodavías, setenta y cuatro. \footnote{\textbf{2:40} Neh 12,8}
\bibleverse{41} Los cantores: los hijos de Asaf, ciento veintiocho.
\bibleverse{42} Los hijos de los porteros: los hijos de Salum, los hijos
de Ater, los hijos de Talmón, los hijos de Acub, los hijos de Hatita,
los hijos de Sobai, en total ciento treinta y nueve.

\bibleverse{43} Los servidores del templo: los hijos de Ziha, los hijos
de Hasupha, los hijos de Tabbaoth, \footnote{\textbf{2:43} Esd 8,20}
\bibleverse{44} los hijos de Keros, los hijos de Siaha, los hijos de
Padon, \bibleverse{45} los hijos de Lebanah, los hijos de Hagabah, los
hijos de Akkub, \bibleverse{46} los hijos de Hagab, los hijos de
Shamlai, los hijos de Hanan, \bibleverse{47} los hijos de Giddel, los
hijos de Gahar, los hijos de Reaiah, \bibleverse{48} los hijos de Rezin,
los hijos de Nekoda, los hijos de Gazzam, \bibleverse{49} los hijos de
Uzza, los hijos de Paseah, los hijos de Besai, \bibleverse{50} los hijos
de Asna, los hijos de Meunim, los hijos de Nefisim, \bibleverse{51} los
hijos de Bakbuk, los hijos de Hakupha, los hijos de Harhur,
\bibleverse{52} los hijos de Bazluth, los hijos de Mehida, los hijos de
Harsha, \bibleverse{53} los hijos de Barkos, los hijos de Sisera, los
hijos de Temah, \bibleverse{54} los hijos de Neziah, los hijos de
Hatipha.

\bibleverse{55} Los hijos de los siervos de Salomón: los hijos de Sotai,
los hijos de Hassophereth, los hijos de Peruda, \footnote{\textbf{2:55}
  1Re 9,21} \bibleverse{56} los hijos de Jaalah, los hijos de Darkon,
los hijos de Giddel, \bibleverse{57} los hijos de Sefatías, los hijos de
Hattil, los hijos de Pochereth Hazzebaim, los hijos de Ami.
\bibleverse{58} Todos los servidores del templo, y los hijos de los
servidores de Salomón, fueron trescientos noventa y dos. \footnote{\textbf{2:58}
  Jos 9,23}

\bibleverse{59} Estos fueron los que subieron de Tel Melá, Tel Harsa,
Querubín, Addán e Immer; pero no pudieron mostrar las casas de sus
padres ni su descendencia,\footnote{\textbf{2:59} o, semilla} si eran de
Israel: \bibleverse{60} los hijos de Delaía, los hijos de Tobías, los
hijos de Necoda, seiscientos cincuenta y dos. \bibleverse{61} De los
hijos de los sacerdotes: los hijos de Habaía, los hijos de Hakkoz, y los
hijos de Barzilai, que tomó mujer de las hijas de Barzilai Galaadita, y
se llamó como ellas. \footnote{\textbf{2:61} 2Sam 19,31} \bibleverse{62}
Estos buscaron su lugar entre los que estaban registrados por
genealogía, pero no fueron encontrados; por lo tanto, fueron
considerados descalificados y apartados del sacerdocio. \bibleverse{63}
El gobernador les dijo que no debían comer de las cosas más santas hasta
que se levantara un sacerdote para servir con Urim y con Tumim.
\footnote{\textbf{2:63} Esd 2,2; Éxod 28,30}

\hypertarget{nuxfamero-total-de-personas-y-animales-de-carga-en-el-municipio}{%
\subsection{Número total de personas y animales de carga en el
municipio}\label{nuxfamero-total-de-personas-y-animales-de-carga-en-el-municipio}}

\bibleverse{64} Toda la asamblea reunida era de cuarenta y dos mil
trescientos sesenta, \bibleverse{65} además de sus siervos y siervas,
que eran siete mil trescientos treinta y siete; y tenían doscientos
cantores y cantoras. \bibleverse{66} Sus caballos eran setecientos
treinta y seis; sus mulos, doscientos cuarenta y cinco; \bibleverse{67}
sus camellos, cuatrocientos treinta y cinco; sus asnos, seis mil
setecientos veinte.

\hypertarget{contribuciones-a-la-construcciuxf3n-del-templo-en-jerusaluxe9n-palabra-final}{%
\subsection{Contribuciones a la construcción del templo en Jerusalén;
Palabra
final}\label{contribuciones-a-la-construcciuxf3n-del-templo-en-jerusaluxe9n-palabra-final}}

\bibleverse{68} Algunos de los jefes de familia de los padres, cuando
llegaron a la casa de Yahvé que está en Jerusalén, ofrecieron
voluntariamente por la casa de Dios para levantarla en su lugar.
\bibleverse{69} Dieron, según su capacidad, para el tesoro de la obra,
sesenta y un mil dáricos de oro,\footnote{\textbf{2:69} un dárico era
  una moneda de oro emitida por un rey persa, que pesaba unos 8,4 gramos
  o unas 0,27 onzas troy cada una.} cinco mil minas\footnote{\textbf{2:69}
  El Seol es el lugar de los muertos.} de plata, y cien vestidos
sacerdotales.

\bibleverse{70} Así que los sacerdotes y los levitas, con parte del
pueblo, los cantores, los porteros y los servidores del templo, vivían
en sus ciudades, y todo Israel en sus ciudades.

\hypertarget{construcciuxf3n-del-altar-de-las-ofrendas-quemadas-y-establecimiento-del-servicio-de-sacrificios-regular-celebraciuxf3n-de-la-fiesta-de-los-tabernuxe1culos}{%
\subsection{Construcción del altar de las ofrendas quemadas y
establecimiento del servicio de sacrificios regular; Celebración de la
Fiesta de los
Tabernáculos}\label{construcciuxf3n-del-altar-de-las-ofrendas-quemadas-y-establecimiento-del-servicio-de-sacrificios-regular-celebraciuxf3n-de-la-fiesta-de-los-tabernuxe1culos}}

\hypertarget{section-2}{%
\section{3}\label{section-2}}

\bibleverse{1} Cuando llegó el séptimo mes, y los hijos de Israel
estaban en las ciudades, el pueblo se reunió como un solo hombre en
Jerusalén. \footnote{\textbf{3:1} Esd 2,64} \bibleverse{2} Entonces
Jesúa, hijo de Josadac, se levantó con sus hermanos los sacerdotes y con
Zorobabel, hijo de Salatiel, y sus parientes, y edificaron el altar del
Dios de Israel para ofrecer sobre él holocaustos, como está escrito en
la ley de Moisés, el hombre de Dios. \footnote{\textbf{3:2} Esd 2,2;
  1Cró 3,17-19; Éxod 27,1; Lev 6,9} \bibleverse{3} A pesar del temor que
sentían por los pueblos de las tierras circundantes, colocaron el altar
sobre su base, y ofrecieron sobre él holocaustos a Yavé, holocaustos por
la mañana y por la tarde. \bibleverse{4} Celebraron la fiesta de las
cabañas, como está escrito, y ofrecieron los holocaustos diarios por
número, según la ordenanza, como lo exigía el deber de cada día;
\footnote{\textbf{3:4} Lev 23,34; Núm 29,12-38} \bibleverse{5} y después
el holocausto continuo, las ofrendas de las lunas nuevas, de todas las
fiestas fijas de Yahvé que estaban consagradas, y de todos los que
voluntariamente ofrecían una ofrenda voluntaria a Yahvé. \bibleverse{6}
Desde el primer día del séptimo mes comenzaron a ofrecer holocaustos a
Yavé, pero aún no se habían puesto los cimientos del templo de Yavé.

\hypertarget{preparaciones-para-la-construcciuxf3n-de-templos-colocaciuxf3n-ceremonial-de-la-primera-piedra}{%
\subsection{Preparaciones para la construcción de templos; Colocación
ceremonial de la primera
piedra}\label{preparaciones-para-la-construcciuxf3n-de-templos-colocaciuxf3n-ceremonial-de-la-primera-piedra}}

\bibleverse{7} También dieron dinero a los albañiles y a los
carpinteros. También dieron comida, bebida y aceite a la gente de Sidón
y Tiro para que trajeran cedros del Líbano al mar, a Jope, según la
concesión que tenían de Ciro, rey de Persia.

\bibleverse{8} En el segundo año de su llegada a la casa de Dios en
Jerusalén, en el segundo mes, Zorobabel hijo de Salatiel, Jesúa hijo de
Josadac, y los demás hermanos de ellos, los sacerdotes y los levitas, y
todos los que habían venido de la cautividad a Jerusalén, comenzaron la
obra y designaron a los levitas, de veinte años para arriba, para que
tuvieran la dirección de la obra de la casa de Yahvé. \bibleverse{9}
Entonces Jesúa se puso de pie con sus hijos y sus hermanos, Cadmiel y
sus hijos, los hijos de Judá, juntos para tener la supervisión de los
obreros en la casa de Dios: los hijos de Henadad, con sus hijos y sus
hermanos los levitas. \footnote{\textbf{3:9} Neh 10,9}

\bibleverse{10} Cuando los constructores pusieron los cimientos del
templo de Yavé, pusieron a los sacerdotes con sus vestimentas y con
trompetas, y a los levitas hijos de Asaf con címbalos, para alabar a
Yavé, según las indicaciones de David, rey de Israel. \bibleverse{11}
Cantaron entre sí alabando y dando gracias a Yavé: ``Porque es bueno,
porque su bondad es eterna para con Israel''. Todo el pueblo gritaba con
gran júbilo cuando alababa a Yahvé, porque se habían puesto los
cimientos de la casa de Yahvé. \footnote{\textbf{3:11} 2Cró 5,13; 2Cró
  7,3; Sal 118,1}

\bibleverse{12} Pero muchos de los sacerdotes, de los levitas y de los
jefes de familia, de los ancianos que habían visto la primera casa,
cuando se pusieron los cimientos de esta casa ante sus ojos, lloraron a
gritos. También muchos gritaban de alegría, \footnote{\textbf{3:12} Hag
  2,3} \bibleverse{13} de modo que el pueblo no podía distinguir el
ruido del grito de alegría del ruido del llanto del pueblo; porque el
pueblo gritaba con fuerza, y el ruido se oía lejos.

\hypertarget{rechazo-de-los-samaritanos-de-participar-en-la-construcciuxf3n-del-templo}{%
\subsection{Rechazo de los samaritanos de participar en la construcción
del
templo}\label{rechazo-de-los-samaritanos-de-participar-en-la-construcciuxf3n-del-templo}}

\hypertarget{section-3}{%
\section{4}\label{section-3}}

\bibleverse{1} Cuando los adversarios de Judá y Benjamín se enteraron de
que los hijos del cautiverio estaban construyendo un templo a Yavé, el
Dios de Israel, \bibleverse{2} se acercaron a Zorobabel y a los jefes de
familia de los padres y les dijeron: ``Permítannos construir con
ustedes, pues buscamos a su Dios como ustedes, y le ofrecemos
sacrificios desde los días de Esar Haddón, rey de Asiria, que nos hizo
subir aquí.'' \footnote{\textbf{4:2} 2Re 17,24; 2Re 17,33; 2Re 19,37}

\bibleverse{3} Pero Zorobabel, Jesúa y el resto de los jefes de familia
de Israel les dijeron: ``Vosotros no tenéis nada que ver con nosotros en
la construcción de una casa a nuestro Dios, sino que nosotros mismos
construiremos juntos a Yahvé, el Dios de Israel, como nos ha ordenado el
rey Ciro, rey de Persia.'' \footnote{\textbf{4:3} Esd 1,3}

\bibleverse{4} Entonces la gente del país debilitó las manos del pueblo
de Judá y lo perturbó en la construcción. \bibleverse{5} Contrataron
consejeros contra ellos para frustrar su propósito todos los días de
Ciro, rey de Persia, hasta el reinado de Darío, rey de Persia.
\footnote{\textbf{4:5} Esd 4,24}

\hypertarget{varias-acusaciones-contra-los-juduxedos-y-su-templo-y-la-construcciuxf3n-de-muros-bajo-el-gobierno-de-jerjes-y-artajerjes}{%
\subsection{Varias acusaciones contra los judíos y su templo y la
construcción de muros bajo el gobierno de Jerjes y
Artajerjes}\label{varias-acusaciones-contra-los-juduxedos-y-su-templo-y-la-construcciuxf3n-de-muros-bajo-el-gobierno-de-jerjes-y-artajerjes}}

\bibleverse{6} En el reinado de Asuero, al principio de su reinado,
escribieron una acusación contra los habitantes de Judá y de Jerusalén.

\bibleverse{7} En los días de Artajerjes, Bislam, Mitrídates, Tabeel y
el resto de sus compañeros escribieron a Artajerjes, rey de Persia, y la
redacción de la carta fue escrita en sirio y entregada en lengua siria.
\bibleverse{8} Rehum, el canciller, y Simsai, el escriba, escribieron
una carta contra Jerusalén al rey Artajerjes, de la siguiente manera
\bibleverse{9} Entonces Rehum, el canciller, Simsai, el escriba, y el
resto de sus compañeros, los dinaítas, los afarsatitas, los tarpelitas,
los afarsitas, los archevitas, los babilonios, los susanquitas, los
dehaitas, los elamitas, \bibleverse{10} y el resto de las naciones que
el grande y noble Osnappar trajo y estableció en la ciudad de Samaria, y
en el resto del país más allá del río, etc., escribieron. \footnote{\textbf{4:10}
  Esd 4,2}

\bibleverse{11} Esta es la copia de la carta que enviaron: Al rey
Artajerjes, de parte de tus siervos, el pueblo de allende el río.
\bibleverse{12} Hazle saber al rey que los judíos que subieron de ti han
venido a nosotros a Jerusalén. Están construyendo la ciudad rebelde y
mala, y han terminado las murallas y reparado los cimientos.
\bibleverse{13} Sepa ahora el rey que si se construye esta ciudad y se
terminan las murallas, no pagarán tributo, ni costumbre, ni peaje, y al
final será perjudicial para los reyes. \bibleverse{14} Ahora bien, como
nosotros comemos la sal del palacio y no es conveniente que veamos la
deshonra del rey, hemos enviado a informar al rey, \bibleverse{15} para
que se busque en el libro de los registros de tus padres. Verás en el
libro de los registros, y sabrás que esta ciudad es una ciudad rebelde y
perjudicial para los reyes y las provincias, y que en el pasado han
iniciado rebeliones en su interior. Por eso esta ciudad fue destruida.
\bibleverse{16} Informamos al rey que si se construye esta ciudad y se
terminan las murallas, no tendrás ninguna posesión más allá del río.

\hypertarget{la-construcciuxf3n-del-templo-se-paralizuxf3-como-consecuencia-de-un-real-decreto}{%
\subsection{La construcción del templo se paralizó como consecuencia de
un real
decreto}\label{la-construcciuxf3n-del-templo-se-paralizuxf3-como-consecuencia-de-un-real-decreto}}

\bibleverse{17} Entonces el rey envió una respuesta a Rehum, el
canciller, y a Simsai, el escriba, y al resto de sus compañeros que
viven en Samaria y en el resto del país al otro lado del río: La paz.
\bibleverse{18} La carta que nos enviasteis ha sido leída claramente
ante mí. \bibleverse{19} He decretado, y se ha hecho una búsqueda, y se
ha encontrado que esta ciudad ha hecho insurrección contra los reyes en
el pasado, y que se han hecho en ella rebeliones y revueltas.
\bibleverse{20} También ha habido reyes poderosos sobre Jerusalén que
han gobernado todo el país más allá del río, y se les pagaba tributo,
costumbre y peaje. \bibleverse{21} Haz ahora un decreto para que cesen
estos hombres, y que no se construya esta ciudad hasta que yo lo
decrete. \bibleverse{22} Tengan cuidado de no ser negligentes al
hacerlo. ¿Por qué ha de crecer el daño en perjuicio de los reyes?

\bibleverse{23} Entonces, cuando la copia de la carta del rey Artajerjes
fue leída ante Rehum, el escriba Simsai y sus compañeros, se dirigieron
apresuradamente a Jerusalén a los judíos y los hicieron cesar por la
fuerza de las armas. \bibleverse{24} Entonces se detuvo el trabajo en la
casa de Dios que está en Jerusalén. Se detuvo hasta el segundo año del
reinado de Darío, rey de Persia. \footnote{\textbf{4:24} Esd 4,5; Esd
  6,15}

\hypertarget{profecuxedas-favorables-de-dos-profetas-permiso-del-gobernador-para-reanudar-la-construcciuxf3n}{%
\subsection{Profecías favorables de dos profetas; Permiso del gobernador
para reanudar la
construcción}\label{profecuxedas-favorables-de-dos-profetas-permiso-del-gobernador-para-reanudar-la-construcciuxf3n}}

\hypertarget{section-4}{%
\section{5}\label{section-4}}

\bibleverse{1} Los profetas Hageo y Zacarías, hijo de Iddo, profetizaron
a los judíos que estaban en Judá y Jerusalén. Les profetizaron en nombre
del Dios de Israel. \footnote{\textbf{5:1} Hag 1,1; Zac 1,1}
\bibleverse{2} Entonces Zorobabel, hijo de Salatiel, y Jesúa, hijo de
Josadac, se levantaron y comenzaron a edificar la casa de Dios que está
en Jerusalén; y con ellos estaban los profetas de Dios, ayudándoles.

\bibleverse{3} Al mismo tiempo Tattenai, el gobernador del otro lado del
río, se acercó a ellos, con Shetharbozenai y sus compañeros, y les
preguntó: ``¿Quién les dio un decreto para construir esta casa y
terminar este muro?'' \bibleverse{4} También preguntaron por los nombres
de los hombres que estaban haciendo este edificio. \bibleverse{5} Pero
el ojo de su Dios estaba sobre los ancianos de los judíos, y no los
hicieron cesar hasta que el asunto llegara a Darío y se les respondiera
por carta al respecto. \footnote{\textbf{5:5} Deut 11,12; 1Re 8,29}

\hypertarget{informe-e-investigaciuxf3n-del-gobernador-al-rey-daruxedo-sobre-la-construcciuxf3n-del-templo}{%
\subsection{Informe e investigación del gobernador al rey Darío sobre la
construcción del
templo}\label{informe-e-investigaciuxf3n-del-gobernador-al-rey-daruxedo-sobre-la-construcciuxf3n-del-templo}}

\bibleverse{6} A continuación se presenta la copia de la carta que
Tattenai, el gobernador del otro lado del río, y Shetharbozenai, y sus
compañeros los afarsacianos que estaban al otro lado del río, enviaron
al rey Darío. \bibleverse{7} Le enviaron una carta en la que estaba
escrito: Al rey Darío, toda la paz. \bibleverse{8} Sepa el rey que
fuimos a la provincia de Judá, a la casa del gran Dios, que se está
construyendo con grandes piedras y se colocan maderas en las paredes.
Esta obra avanza con diligencia y prospera en sus manos. \bibleverse{9}
Entonces preguntamos a esos ancianos y les dijimos así: ``¿Quién os ha
dado el decreto de construir esta casa y de terminar este muro?''
\bibleverse{10} Les preguntamos también sus nombres, para informarles de
que podíamos escribir los nombres de los hombres que estaban a su
cabeza. \bibleverse{11} Ellos nos respondieron diciendo: ``Nosotros
somos los siervos del Dios del cielo y de la tierra y estamos
construyendo la casa que se edificó hace tantos años, que un gran rey de
Israel construyó y terminó. \bibleverse{12} Pero después que nuestros
padres provocaron la ira del Dios del cielo, él los entregó en manos de
Nabucodonosor, rey de Babilonia, el caldeo, quien destruyó esta casa y
llevó al pueblo a Babilonia. \footnote{\textbf{5:12} 2Re 25,9}
\bibleverse{13} Pero en el primer año de Ciro, rey de Babilonia, el rey
Ciro hizo un decreto para construir esta casa de Dios. \footnote{\textbf{5:13}
  Esd 1,1} \bibleverse{14} Los utensilios de oro y plata de la casa de
Dios, que Nabucodonosor sacó del templo que estaba en Jerusalén y los
llevó al templo de Babilonia, los sacó también el rey Ciro del templo de
Babilonia, y fueron entregados a uno que se llamaba Sesbasar, a quien
había nombrado gobernador. \footnote{\textbf{5:14} Esd 1,8}
\bibleverse{15} Este le dijo: ``Toma estos utensilios, ve y ponlos en el
templo que está en Jerusalén, y que se construya la casa de Dios en su
lugar''. \bibleverse{16} Entonces vino el mismo Sesbasar y puso los
cimientos de la casa de Dios que está en Jerusalén. Desde entonces hasta
ahora se ha estado construyendo, y aún no se ha terminado.
\bibleverse{17} Ahora, pues, si al rey le parece bien, que se investigue
en la casa del tesoro del rey, que está allí en Babilonia, si es cierto
que el rey Ciro decretó la construcción de esta casa de Dios en
Jerusalén; y que el rey nos envíe su beneplácito sobre este asunto.''

\hypertarget{encontrar-el-decreto-de-cyrus-en-ekbatana-e-informaciuxf3n-de-uxe9l}{%
\subsection{Encontrar el decreto de Cyrus en Ekbatana e información de
él}\label{encontrar-el-decreto-de-cyrus-en-ekbatana-e-informaciuxf3n-de-uxe9l}}

\hypertarget{section-5}{%
\section{6}\label{section-5}}

\bibleverse{1} Entonces el rey Darío dictó un decreto, y se registró la
casa de los archivos, donde se guardaban los tesoros en Babilonia.
\bibleverse{2} Se encontró un pergamino en Acmetá, en el palacio que
está en la provincia de Media, y en él se escribió esto para que quede
constancia: \bibleverse{3} En el primer año del rey Ciro, éste dictó un
decreto: En cuanto a la casa de Dios en Jerusalén, que se construya la
casa, el lugar donde se ofrecen los sacrificios, y que se pongan sus
cimientos con fuerza, con su altura de sesenta codos y su anchura de
sesenta codos; \footnote{\textbf{6:3} Esd 1,1} \bibleverse{4} con tres
hileras de grandes piedras y una hilera de madera nueva. Que los gastos
sean dados de la casa del rey. \bibleverse{5} También que los utensilios
de oro y plata de la casa de Dios, que Nabucodonosor sacó del templo que
está en Jerusalén y llevó a Babilonia, sean restaurados y llevados de
nuevo al templo que está en Jerusalén, cada cosa a su lugar. Los pondrás
en la casa de Dios.

\hypertarget{decreto-de-daruxedo-para-continuar-sin-trabas-y-promover-la-construcciuxf3n-del-templo}{%
\subsection{Decreto de Darío para continuar sin trabas y promover la
construcción del
templo}\label{decreto-de-daruxedo-para-continuar-sin-trabas-y-promover-la-construcciuxf3n-del-templo}}

\bibleverse{6} Ahora, pues, Tattenai, gobernador del otro lado del río,
Shetarbozenai, y tus compañeros los afarsaquitas, que están al otro lado
del río, debes permanecer lejos de allí. \bibleverse{7} Dejad la obra de
esta casa de Dios; dejad que el gobernador de los judíos y los ancianos
de los judíos construyan esta casa de Dios en su lugar. \bibleverse{8}
Además, yo dicto lo que haréis por estos ancianos de los judíos para la
construcción de esta casa de Dios: que de los bienes del rey, incluso
del tributo más allá del Río, se den gastos con toda diligencia a estos
hombres, para que no sean estorbados. \bibleverse{9} Lo que necesiten,
incluyendo novillos, carneros y corderos, para los holocaustos al Dios
del cielo; también trigo, sal, vino y aceite, según la palabra de los
sacerdotes que están en Jerusalén, que se les dé día a día sin falta,
\bibleverse{10} para que ofrezcan sacrificios de agradable aroma al Dios
del cielo, y oren por la vida del rey y de sus hijos. \bibleverse{11}
También he decretado que al que altere este mensaje, se le arranque una
viga de su casa y se le sujete a ella, y que su casa se convierta en un
estercolero por esto. \bibleverse{12} Que el Dios que ha hecho habitar
su nombre derribe a todos los reyes y pueblos que extiendan su mano para
alterar esto, para destruir esta casa de Dios que está en Jerusalén. Yo,
Darío, he hecho un decreto. Que se haga con toda diligencia.

\hypertarget{terminaciuxf3n-y-dedicaciuxf3n-solemne-del-templo}{%
\subsection{Terminación y dedicación solemne del
templo}\label{terminaciuxf3n-y-dedicaciuxf3n-solemne-del-templo}}

\bibleverse{13} Entonces Tattenai, el gobernador del otro lado del río,
Shetharbozenai y sus compañeros hicieron lo correspondiente con toda
diligencia, porque el rey Darío había enviado un decreto.

\bibleverse{14} Los ancianos de los judíos construyeron y prosperaron,
por la profecía del profeta Ageo y de Zacarías, hijo de Iddo. La
edificaron y la terminaron, según el mandato del Dios de Israel, y según
el decreto de Ciro, de Darío y de Artajerjes, rey de Persia.
\bibleverse{15} Esta casa fue terminada el tercer día del mes de Adar,
que fue en el sexto año del reinado del rey Darío. \footnote{\textbf{6:15}
  Esd 4,24}

\bibleverse{16} Los hijos de Israel, los sacerdotes, los levitas y el
resto de los hijos del cautiverio, celebraron la dedicación de esta casa
de Dios con alegría. \footnote{\textbf{6:16} Núm 7,10; 1Re 8,62-66}
\bibleverse{17} Ofrecieron en la dedicación de esta casa de Dios cien
toros, doscientos carneros y cuatrocientos corderos; y como ofrenda por
el pecado para todo Israel, doce machos cabríos, según el número de las
tribus de Israel. \footnote{\textbf{6:17} Esd 8,35} \bibleverse{18}
Pusieron a los sacerdotes en sus divisiones y a los levitas en sus
turnos, para el servicio de Dios que está en Jerusalén, como está
escrito en el libro de Moisés. \footnote{\textbf{6:18} Núm 3,6; Núm 8,24}

\hypertarget{celebraciuxf3n-de-la-pascua}{%
\subsection{Celebración de la
Pascua}\label{celebraciuxf3n-de-la-pascua}}

\bibleverse{19} Los hijos del cautiverio celebraron la Pascua el día
catorce del primer mes. \footnote{\textbf{6:19} Éxod 12,6}
\bibleverse{20} Como los sacerdotes y los levitas se habían purificado
juntos, todos ellos estaban puros. Mataron la Pascua por todos los hijos
del cautiverio, por sus hermanos los sacerdotes y por ellos mismos.
\bibleverse{21} Los hijos de Israel que habían regresado del cautiverio,
y todos los que se habían separado de la inmundicia de las naciones del
país para buscar a Yavé, el Dios de Israel, comieron, \bibleverse{22} y
celebraron con alegría la fiesta de los panes sin levadura durante siete
días, porque Yavé los había alegrado y había hecho volver el corazón del
rey de Asiria hacia ellos, para fortalecer sus manos en la obra de Dios,
el Dios de la casa de Israel.

\hypertarget{el-regreso-de-esdras-y-su-banda-de-babilonia-a-jerusaluxe9n}{%
\subsection{El regreso de Esdras y su banda de Babilonia a
Jerusalén}\label{el-regreso-de-esdras-y-su-banda-de-babilonia-a-jerusaluxe9n}}

\hypertarget{section-6}{%
\section{7}\label{section-6}}

\bibleverse{1} Después de esto, en el reinado de Artajerjes, rey de
Persia, Esdras, hijo de Seraías, hijo de Azarías, hijo de Hilcías,
\footnote{\textbf{7:1} 1Cró 6,14} \bibleverse{2} hijo de Salum, hijo de
Sadoc, hijo de Ajitub, \bibleverse{3} hijo de Amarías hijo de Azarías,
hijo de Meraiot, \bibleverse{4} hijo de Zerahías, hijo de Uzi, hijo de
Bukki, \bibleverse{5} hijo de Abisúa, hijo de Finehas, hijo de Eleazar,
hijo de Aarón, el sumo sacerdote --- \bibleverse{6} este Esdras subió de
Babilonia. Era un escriba experto en la ley de Moisés, que Yahvé, el
Dios de Israel, había dado; y el rey le concedió toda su petición, según
la mano de Yahvé, su Dios, sobre él. \footnote{\textbf{7:6} Esd 7,9; Esd
  7,28; Esd 8,18; Esd 8,22; Neh 2,8} \bibleverse{7} Algunos de los hijos
de Israel, entre ellos algunos de los sacerdotes, los levitas, los
cantores, los porteros y los servidores del templo, subieron a Jerusalén
en el séptimo año del rey Artajerjes. \footnote{\textbf{7:7} Esd 2,43}
\bibleverse{8} Llegó a Jerusalén en el quinto mes, que era el séptimo
año del rey. \bibleverse{9} Porque el primer día del primer mes comenzó
a subir de Babilonia, y el primer día del quinto mes llegó a Jerusalén,
según la buena mano de su Dios sobre él. \footnote{\textbf{7:9} Esd 7,6}
\bibleverse{10} Porque Esdras había puesto su corazón en buscar la ley
de Yahvé y en ponerla en práctica, y en enseñar los estatutos y los
reglamentos en Israel.

\hypertarget{redacciuxf3n-de-la-carta-real-carta-de-salvoconducto-con-detalles-de-los-poderes-otorgados-a-ezra}{%
\subsection{Redacción de la carta real (= carta de salvoconducto) con
detalles de los poderes otorgados a
Ezra}\label{redacciuxf3n-de-la-carta-real-carta-de-salvoconducto-con-detalles-de-los-poderes-otorgados-a-ezra}}

\bibleverse{11} Esta es la copia de la carta que el rey Artajerjes dio
al sacerdote Esdras, el escriba de las palabras de los mandamientos de
Yahvé y de sus estatutos para Israel: \bibleverse{12} Artajerjes, rey de
reyes, Al sacerdote Esdras, el escriba de la ley del Dios perfecto del
cielo. Ahora bien, \footnote{\textbf{7:12} Ezeq 26,7} \bibleverse{13} yo
decreto que todos los del pueblo de Israel y sus sacerdotes y los
levitas de mi reino, que tengan la intención de ir por su propia
voluntad a Jerusalén, vayan con vosotros. \bibleverse{14} Porque habéis
sido enviados por el rey y sus siete consejeros para investigar sobre
Judá y Jerusalén, según la ley de vuestro Dios que está en vuestra mano,
\bibleverse{15} y para llevar la plata y el oro que el rey y sus
consejeros han ofrecido voluntariamente al Dios de Israel, cuya morada
está en Jerusalén, \bibleverse{16} y toda la plata y el oro que
encontraréis en toda la provincia de Babilonia, con la ofrenda
voluntaria del pueblo y de los sacerdotes, ofreciendo voluntariamente
para la casa de su Dios que está en Jerusalén. \bibleverse{17} Por lo
tanto, con toda diligencia comprarás con este dinero toros, carneros y
corderos con sus ofrendas de comida y sus libaciones, y los ofrecerás en
el altar de la casa de tu Dios que está en Jerusalén. \bibleverse{18} Lo
que os parezca bien a vosotros y a vuestros hermanos hacer con el resto
de la plata y del oro, hacedlo según la voluntad de vuestro Dios.
\bibleverse{19} Los utensilios que se te den para el servicio de la casa
de tu Dios, entrégalos ante el Dios de Jerusalén. \bibleverse{20} Todo
lo que se necesite para la casa de tu Dios, y que tengas ocasión de dar,
dalo de la casa del tesoro del rey. \bibleverse{21} Yo, el rey
Artajerjes, decreto a todos los tesoreros que están al otro lado del
río, que todo lo que el sacerdote Esdras, escriba de la ley del Dios del
cielo, os pida, lo hagáis con toda diligencia, \bibleverse{22} hasta
cien talentos de plata, y hasta cien cors de trigo, y hasta cien baños
de vino, y hasta cien baños de aceite, y sal sin prescribir cuánto.
\bibleverse{23} Todo lo que sea ordenado por el Dios del cielo, hágase
exactamente para la casa del Dios del cielo; porque ¿por qué habría de
haber ira contra el reino del rey y de sus hijos? \bibleverse{24}
También les informamos que no será lícito imponer tributo, costumbre o
peaje a ninguno de los sacerdotes, levitas, cantores, porteros,
servidores del templo o trabajadores de esta casa de Dios.
\bibleverse{25} Tú, Esdras, según la sabiduría de tu Dios que está en tu
mano, nombra magistrados y jueces que puedan juzgar a todo el pueblo que
está al otro lado del río, que todos conozcan las leyes de tu Dios; y
enseña al que no las conozca. \bibleverse{26} El que no cumpla la ley de
tu Dios y la ley del rey, que se ejecute sobre él el juicio con toda
diligencia, ya sea a muerte, ya sea a destierro, ya sea a confiscación
de bienes, ya sea a prisión.

\hypertarget{oraciuxf3n-de-acciuxf3n-de-gracias-de-esdras-e-inicio-de-su-actividad}{%
\subsection{Oración de acción de gracias de Esdras e inicio de su
actividad}\label{oraciuxf3n-de-acciuxf3n-de-gracias-de-esdras-e-inicio-de-su-actividad}}

\bibleverse{27} Bendito sea Yahvé, el Dios de nuestros padres, que ha
puesto algo así en el corazón del rey, para embellecer la casa de Yahvé
que está en Jerusalén; \bibleverse{28} y ha extendido su bondad conmigo
ante el rey y sus consejeros, y ante todos los poderosos príncipes del
rey. Me he fortalecido según la mano del Señor, mi Dios, y he reunido a
los jefes de Israel para que suban conmigo. \footnote{\textbf{7:28} Esd
  7,6}

\hypertarget{directorio-de-los-jefes-de-las-familias-de-judea-que-regresan-con-esdras}{%
\subsection{Directorio de los jefes de las familias de Judea que
regresan con
Esdras}\label{directorio-de-los-jefes-de-las-familias-de-judea-que-regresan-con-esdras}}

\hypertarget{section-7}{%
\section{8}\label{section-7}}

\bibleverse{1} Estos son los jefes de familia de sus padres, y esta es
la genealogía de los que subieron conmigo desde Babilonia, en el reinado
del rey Artajerjes: \footnote{\textbf{8:1} Esd 7,1; Esd 7,7}
\bibleverse{2} De los hijos de Finehas, Gershom. De los hijos de
Ithamar, Daniel. De los hijos de David, Hattush. \bibleverse{3} De los
hijos de Secanías, de los hijos de Paros, Zacarías; y con él se
enumeraron por genealogía de los varones ciento cincuenta. \footnote{\textbf{8:3}
  1Cró 3,22} \bibleverse{4} De los hijos de Pahatmoab, Eliehoenai, hijo
de Zerahiah, y con él doscientos varones. \footnote{\textbf{8:4} Esd 2,6}
\bibleverse{5} De los hijos de Secanías, hijo de Jahaziel, y con él
trescientos varones. \footnote{\textbf{8:5} Esd 2,8} \bibleverse{6} De
los hijos de Adín, Ebed, hijo de Jonatán, y con él cincuenta varones.
\bibleverse{7} De los hijos de Elam, Jesaías, hijo de Atalía, y con él
setenta varones. \bibleverse{8} De los hijos de Sefatías, Zebadías, hijo
de Miguel, y con él ochenta varones. \bibleverse{9} De los hijos de
Joab, Obadías hijo de Jehiel, y con él doscientos dieciocho varones.
\bibleverse{10} De los hijos de Selomit, hijo de Josifa, y con él ciento
sesenta varones. \footnote{\textbf{8:10} Esd 2,10} \bibleverse{11} De
los hijos de Bebai, Zacarías, hijo de Bebai, y con él veintiocho
varones. \bibleverse{12} De los hijos de Azgad, Johanan hijo de
Hakkatan, y con él ciento diez varones. \bibleverse{13} De los hijos de
Adonikam, que fueron los últimos, sus nombres son: Eliphelet, Jeuel y
Semaías; y con ellos sesenta varones. \bibleverse{14} De los hijos de
Bigvai, Uthai y Zabbud, y con ellos setenta varones.

\hypertarget{los-preparativos-finales-para-la-salida}{%
\subsection{Los preparativos finales para la
salida}\label{los-preparativos-finales-para-la-salida}}

\bibleverse{15} Los reuní hasta el río que corre hacia Ahava, y allí
acampamos tres días. Entonces miré alrededor del pueblo y de los
sacerdotes, y encontré que no había ninguno de los hijos de Leví.
\bibleverse{16} Entonces mandé llamar a Eliezer, a Ariel, a Semaías, a
Elnatán, a Jarib, a Elnatán, a Natán, a Zacarías y a Mesulam, hombres
principales; también a Joiarib y a Elnatán, que eran maestros.
\bibleverse{17} Los envié a Iddo, el jefe, al lugar de Casifia, y les
dije lo que debían decir a Iddo y a sus hermanos, los servidores del
templo en el lugar de Casifia, para que nos trajeran ministros para la
casa de nuestro Dios. \footnote{\textbf{8:17} Esd 2,43} \bibleverse{18}
Conforme a la buena mano de nuestro Dios sobre nosotros, nos trajeron un
hombre discreto, de los hijos de Mahli, hijo de Leví, hijo de Israel, a
saber, Serebías, con sus hijos y sus hermanos, dieciocho; \footnote{\textbf{8:18}
  Esd 7,6} \bibleverse{19} y Hasabías, y con él Jesaías, de los hijos de
Merari, sus hermanos y sus hijos, veinte; \bibleverse{20} y de los
servidores del templo, que David y los príncipes habían dado para el
servicio de los levitas, doscientos veinte servidores del templo. Todos
ellos fueron mencionados por su nombre. \footnote{\textbf{8:20} 1Cró 9,2}

\hypertarget{ayuno-y-oraciuxf3n-de-los-que-regresan-a-casa-entrega-de-los-dones-del-templo-a-hombres-confiables}{%
\subsection{Ayuno y oración de los que regresan a casa; Entrega de los
dones del templo a hombres
confiables}\label{ayuno-y-oraciuxf3n-de-los-que-regresan-a-casa-entrega-de-los-dones-del-templo-a-hombres-confiables}}

\bibleverse{21} Entonces proclamé un ayuno allí, en el río Ahava, para
humillarnos ante nuestro Dios y buscar de él un camino recto para
nosotros, para nuestros pequeños y para todas nuestras posesiones.
\bibleverse{22} Porque me daba vergüenza pedir al rey una banda de
soldados y jinetes que nos ayudara contra el enemigo en el camino,
porque habíamos hablado con el rey diciendo: ``La mano de nuestro Dios
está sobre todos los que lo buscan, para bien; pero su poder y su ira
están contra todos los que lo abandonan.'' \footnote{\textbf{8:22} Esd
  7,6} \bibleverse{23} Así que ayunamos y rogamos a nuestro Dios por
esto, y él nos concedió nuestra petición.

\bibleverse{24} Entonces aparté a doce de los jefes de los sacerdotes, a
Serebías, a Hasabías y a diez de sus hermanos con ellos, \bibleverse{25}
y les pesé la plata, el oro y los utensilios, la ofrenda para la casa de
nuestro Dios, que habían ofrecido el rey, sus consejeros, sus príncipes
y todo Israel allí presente. \bibleverse{26} Pesé en su mano seiscientos
cincuenta talentos de plata, cien talentos de recipientes de plata, cien
talentos de oro, \bibleverse{27} veinte copas de oro que pesaban mil
dracmas, y dos recipientes de bronce fino y brillante, preciosos como el
oro. \bibleverse{28} Les dije: ``Vosotros sois santos para Yahvé, y los
vasos son santos. La plata y el oro son una ofrenda voluntaria a Yavé,
el Dios de vuestros padres. \bibleverse{29} Velen y guárdenlos hasta que
los pesen ante los jefes de los sacerdotes, los levitas y los príncipes
de las casas paternas de Israel en Jerusalén, en las salas de la casa de
Yavé.''

\bibleverse{30} Los sacerdotes y los levitas recibieron el peso de la
plata, el oro y los utensilios, para llevarlos a Jerusalén, a la casa de
nuestro Dios.

\hypertarget{llegada-a-jerusaluxe9n-entrega-de-los-obsequios-votivos-hacer-ofrendas-apoyo-de-funcionarios-reales}{%
\subsection{Llegada a Jerusalén; Entrega de los obsequios votivos; Hacer
ofrendas; Apoyo de funcionarios
reales}\label{llegada-a-jerusaluxe9n-entrega-de-los-obsequios-votivos-hacer-ofrendas-apoyo-de-funcionarios-reales}}

\bibleverse{31} Entonces partimos del río Ahava el duodécimo día del
primer mes, para ir a Jerusalén. La mano de nuestro Dios estaba sobre
nosotros, y nos libró de la mano del enemigo y de los bandidos en el
camino. \bibleverse{32} Llegamos a Jerusalén y nos quedamos allí tres
días. \bibleverse{33} Al cuarto día se pesó la plata, el oro y los
utensilios en la casa de nuestro Dios, en manos de Meremot, hijo del
sacerdote Urías; con él estaba Eleazar, hijo de Finees, y con ellos
estaban Jozabad, hijo de Jesúa, y Noadías, hijo de Binúi, los levitas.
\bibleverse{34} Todo fue contado y pesado, y todo el peso fue escrito en
ese momento.

\bibleverse{35} Los hijos del cautiverio, que habían salido del exilio,
ofrecieron holocaustos al Dios de Israel: doce toros por todo Israel,
noventa y seis carneros, setenta y siete corderos y doce machos cabríos
como ofrenda por el pecado. Todo esto fue un holocausto para Yahvé.
\bibleverse{36} Entregaron los encargos del rey a los gobernadores
locales del rey y a los gobernadores del otro lado del río. Así
mantenían al pueblo y a la casa de Dios. \footnote{\textbf{8:36} Esd
  7,12-26}

\hypertarget{esra-se-da-cuenta-de-los-matrimonios-mixtos-su-consternaciuxf3n-por-estos-funcionarios}{%
\subsection{Esra se da cuenta de los matrimonios mixtos; su
consternación por estos
funcionarios}\label{esra-se-da-cuenta-de-los-matrimonios-mixtos-su-consternaciuxf3n-por-estos-funcionarios}}

\hypertarget{section-8}{%
\section{9}\label{section-8}}

\bibleverse{1} Cuando se hicieron estas cosas, los príncipes se
acercaron a mí, diciendo: ``El pueblo de Israel, los sacerdotes y los
levitas no se han separado de los pueblos de las tierras, siguiendo sus
abominaciones, las de los cananeos, los hititas, los ferezeos, los
jebuseos, los amonitas, los moabitas, los egipcios y los amorreos.
\bibleverse{2} Porque han tomado de sus hijas para sí y para sus hijos,
de modo que la santa descendencia se ha mezclado con los pueblos de las
tierras. Sí, la mano de los príncipes y gobernantes ha sido la principal
en esta transgresión''. \footnote{\textbf{9:2} Esd 9,11-12; Neh 13,23}

\bibleverse{3} Cuando oí esto, rasgué mi vestido y mi túnica, me
arranqué el pelo de la cabeza y de la barba, y me senté confundido.
\footnote{\textbf{9:3} Gén 37,34} \bibleverse{4} Entonces se reunieron
conmigo todos los que temían las palabras del Dios de Israel a causa de
la transgresión de los desterrados, y me senté confundido hasta la
ofrenda de la tarde.

\hypertarget{la-oraciuxf3n-penitencial-de-esdras}{%
\subsection{La oración penitencial de
Esdras}\label{la-oraciuxf3n-penitencial-de-esdras}}

\bibleverse{5} En la ofrenda de la tarde me levanté de mi humillación,
con mi manto y mi túnica rasgados; caí de rodillas y extendí mis manos a
Yahvé, mi Dios; \bibleverse{6} y dije: ``Dios mío, me avergüenzo y me
sonrojo al levantar mi rostro hacia ti, mi Dios, porque nuestras
iniquidades han aumentado sobre nuestra cabeza, y nuestra culpa ha
crecido hasta el cielo. \footnote{\textbf{9:6} Dan 9,7-8; Sal 38,4}
\bibleverse{7} Desde los días de nuestros padres hemos sido sumamente
culpables hasta el día de hoy; y por nuestras iniquidades nosotros,
nuestros reyes y nuestros sacerdotes hemos sido entregados en manos de
los reyes de las tierras, a la espada, al cautiverio, al saqueo y a la
confusión de rostro, como sucede en este día. \bibleverse{8} Ahora bien,
por un momento se ha manifestado la gracia de Yahvé, nuestro Dios, de
dejarnos un remanente para que escapemos, y de darnos una estaca en su
lugar santo, para que nuestro Dios ilumine nuestros ojos, y nos reanime
un poco en nuestra esclavitud. \footnote{\textbf{9:8} Is 22,23}
\bibleverse{9} Porque somos siervos de la esclavitud, pero nuestro Dios
no nos ha abandonado en nuestra esclavitud, sino que nos ha extendido su
bondad a los ojos de los reyes de Persia, para revivirnos, para levantar
la casa de nuestro Dios y reparar sus ruinas, y para darnos un muro en
Judá y en Jerusalén. \footnote{\textbf{9:9} Is 5,5}

\bibleverse{10} ``Ahora, Dios nuestro, ¿qué diremos después de esto?
Porque hemos abandonado tus mandamientos, \bibleverse{11} que has
ordenado por medio de tus siervos los profetas, diciendo: `La tierra a
la que vais a poseer es una tierra impura por la impureza de los pueblos
de las tierras, por sus abominaciones, que la han llenado de un extremo
a otro con su inmundicia. \footnote{\textbf{9:11} Lev 18,24-25}
\bibleverse{12} Ahora, pues, no des tus hijas a sus hijos. No tomes sus
hijas para tus hijos, ni busques su paz o su prosperidad para siempre,
para que seas fuerte y comas el bien de la tierra, y la dejes en
herencia a tus hijos para siempre.' \footnote{\textbf{9:12} Deut 7,2-3}

\bibleverse{13} ``Después de todo lo que ha caído sobre nosotros por
nuestras malas acciones y por nuestra gran culpa, ya que tú, nuestro
Dios, nos has castigado menos de lo que merecen nuestras iniquidades, y
nos has dado tal remanente, \bibleverse{14} ¿volveremos a quebrantar tus
mandamientos y a unirnos a los pueblos que hacen estas abominaciones?
¿No te enojarías con nosotros hasta consumirnos, para que no quedara
ningún remanente, ni ninguno que pudiera escapar? \bibleverse{15} Yahvé,
el Dios de Israel, tú eres justo; porque nos ha quedado un remanente que
ha escapado, como ocurre hoy. He aquí que estamos ante ti en nuestra
culpabilidad; pues nadie puede permanecer ante ti a causa de esto''.
\footnote{\textbf{9:15} Neh 9,33}

\hypertarget{la-acciuxf3n-contra-los-matrimonios-mixtos}{%
\subsection{La acción contra los matrimonios
mixtos}\label{la-acciuxf3n-contra-los-matrimonios-mixtos}}

\hypertarget{section-9}{%
\section{10}\label{section-9}}

\bibleverse{1} Mientras Esdras oraba y se confesaba, llorando y
postrándose ante la casa de Dios, se reunió con él, de parte de Israel,
una asamblea muy numerosa de hombres, mujeres y niños, pues el pueblo
lloraba muy amargamente. \bibleverse{2} Secanías, hijo de Jehiel, uno de
los hijos de Elam, respondió a Esdras: ``Nos hemos rebelado contra
nuestro Dios y nos hemos casado con mujeres extranjeras de los pueblos
de la tierra. Sin embargo, ahora hay esperanza para Israel en cuanto a
esto. \bibleverse{3} Ahora, pues, hagamos un pacto con nuestro Dios para
repudiar a todas las mujeres y a los nacidos de ellas, según el consejo
de mi señor y de los que temen el mandamiento de nuestro Dios. Que se
haga según la ley. \bibleverse{4} Levántate, pues el asunto te pertenece
y nosotros estamos contigo. Sé valiente y hazlo''.

\bibleverse{5} Entonces Esdras se levantó e hizo jurar a los jefes de
los sacerdotes, a los levitas y a todo Israel que harían lo que se les
dijera. Así lo juraron. \bibleverse{6} Entonces Esdras se levantó de
delante de la casa de Dios y entró en la habitación de Johanán, hijo de
Eliasib. Cuando llegó allí, no comió pan ni bebió agua, pues se lamentó
por la transgresión de los exiliados. \bibleverse{7} Hicieron un pregón
por todo Judá y Jerusalén a todos los hijos del cautiverio, para que se
reunieran en Jerusalén; \bibleverse{8} y que el que no viniera dentro de
tres días, según el consejo de los príncipes y de los ancianos, perdiera
todos sus bienes, y él mismo se separara de la asamblea del cautiverio.

\bibleverse{9} Entonces todos los hombres de Judá y de Benjamín se
reunieron en Jerusalén dentro de los tres días. Era el mes noveno, a los
veinte días del mes; y todo el pueblo se sentó en el amplio lugar frente
a la casa de Dios, temblando por este asunto y por la gran lluvia.

\bibleverse{10} El sacerdote Esdras se levantó y les dijo: ``Ustedes han
cometido una infracción y se han casado con mujeres extranjeras,
aumentando la culpa de Israel. \bibleverse{11} Ahora, pues, confesad a
Yavé, el Dios de vuestros padres, y haced su voluntad. Sepárense de los
pueblos de la tierra y de las mujeres extranjeras''.

\bibleverse{12} Entonces toda la asamblea respondió en voz alta:
``Debemos hacer lo que has dicho sobre nosotros. \bibleverse{13} Pero el
pueblo es numeroso, y es tiempo de mucha lluvia, y no podemos quedarnos
afuera. Esta no es una obra de un día ni de dos, pues hemos transgredido
mucho en este asunto. \bibleverse{14} Ahora bien, que se designen
nuestros príncipes para toda la asamblea, y que vengan a horas señaladas
todos los que están en nuestras ciudades que se han casado con mujeres
extranjeras, y con ellos los ancianos de cada ciudad y sus jueces, hasta
que se aparte de nosotros la feroz ira de nuestro Dios, hasta que se
resuelva este asunto.''

\bibleverse{15} Sólo Jonatán, hijo de Asahel, y Jahzé, hijo de Ticva, se
opusieron a esto; y Mesulam y el levita Sabetai los ayudaron.

\bibleverse{16} Así lo hicieron los hijos del cautiverio. El sacerdote
Esdras, con algunos jefes de familia, según sus casas paternas, y todos
ellos por sus nombres, fueron apartados; y se sentaron el primer día del
décimo mes para examinar el asunto. \bibleverse{17} Terminaron con todos
los hombres que se habían casado con mujeres extranjeras para el primer
día del primer mes.

\hypertarget{lista-de-sacerdotes-levitas-y-laicos-que-se-casaron-con-mujeres-extrauxf1as}{%
\subsection{Lista de sacerdotes, levitas y laicos que se casaron con
mujeres
extrañas}\label{lista-de-sacerdotes-levitas-y-laicos-que-se-casaron-con-mujeres-extrauxf1as}}

\bibleverse{18} Entre los hijos de los sacerdotes se encontraron algunos
que se habían casado con mujeres extranjeras: de los hijos de Jesúa,
hijo de Josadac, y sus hermanos: Maasías, Eliezer, Jarib y Gedalías.
\footnote{\textbf{10:18} Esd 3,2; Esd 9,2} \bibleverse{19} Ellos dieron
su mano para que repudiaran a sus mujeres; y siendo culpables,
ofrecieron un carnero del rebaño por su culpa. \bibleverse{20} De los
hijos de Immer Hanani y Zebadiah. \bibleverse{21} De los hijos de Harim:
Maasías, Elías, Semaías, Jehiel y Uzías. \bibleverse{22} De los hijos de
Pashur: Elioenai, Maaseiah, Ismael, Natanel, Jozabad y Elasah.
\bibleverse{23} De los levitas: Jozabad, Simei, Kelaiah (también llamado
Kelita), Pethahiah, Judah y Eliezer. \bibleverse{24} De los cantantes:
Eliashib. De los guardianes de la puerta: Shallum, Telem y Uri.
\bibleverse{25} De Israel: De los hijos de Paros: Ramías, Izzías,
Malquías, Mijamín, Eleazar, Malquías y Benaías. \bibleverse{26} De los
hijos de Elam: Matanías, Zacarías, Jehiel, Abdi, Jeremot y Elías.
\bibleverse{27} De los hijos de Zattu Elioenai, Eliashib, Mattaniah,
Jeremoth, Zabad y Aziza. \bibleverse{28} De los hijos de Bebai Johanán,
Hananías, Zabbai y Atilái. \bibleverse{29} De los hijos de Bani:
Meshullam, Malluch, Adaiah, Jashub, Sheal y Jeremoth. \bibleverse{30} De
los hijos de Pahatmoab Adna, Quelal, Benaía, Maasías, Matanías, Bezalel,
Binúi y Manasés. \bibleverse{31} De los hijos de Harim Eliezer, Ishijá,
Malquías, Semaías, Simeón, \bibleverse{32} Benjamín, Malluch y Semarías.
\bibleverse{33} De los hijos de Hasum: Mattenai, Matattah, Zabad,
Eliphelet, Jeremai, Manasés y Simei. \bibleverse{34} De los hijos de
Bani: Maadai, Amram, Uel, \bibleverse{35} Benaiah, Bedeiah, Cheluhi,
\bibleverse{36} Vaniah, Meremoth, Eliashib, \bibleverse{37} Mattaniah,
Mattenai, Jaasu, \bibleverse{38} Bani, Binnui, Shimei, \bibleverse{39}
Shelemiah, Nathan, Adaiah, \bibleverse{40} Machnadebai, Shashai, Sharai,
\bibleverse{41} Azarel, Shelemiah, Shemariah, \bibleverse{42} Shallum,
Amariah, y Joseph. \bibleverse{43} De los hijos de Nebo: Jeiel,
Mattithiah, Zabad, Zebina, Iddo, Joel y Benaiah.

\bibleverse{44} Todos ellos habían tomado esposas extranjeras. Algunos
de ellos tenían esposas con las que tenían hijos.
