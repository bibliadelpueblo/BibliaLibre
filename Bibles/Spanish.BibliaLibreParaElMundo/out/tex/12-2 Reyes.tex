\hypertarget{eluxedas-anuncia-la-muerte-del-rey-enfermo-e-iduxf3latra-ochuxf4zuxedas}{%
\subsection{Elías anuncia la muerte del rey enfermo e idólatra
Ochôzías}\label{eluxedas-anuncia-la-muerte-del-rey-enfermo-e-iduxf3latra-ochuxf4zuxedas}}

\hypertarget{section}{%
\section{1}\label{section}}

\bibleverse{1} Moab se rebeló contra Israel tras la muerte de Ajab.
\footnote{\textbf{1:1} 2Re 3,5}

\bibleverse{2} Ocozías se cayó por la celosía de su habitación superior
que estaba en Samaria, y se enfermó. Envió entonces mensajeros y les
dijo: ``Vayan a consultar a Baal Zebub, el dios de Ecrón, si me
recuperaré de esta enfermedad.'' \footnote{\textbf{1:2} 1Re 22,51; Is
  19,3}

\bibleverse{3} Pero el ángel de Yahvé\footnote{\textbf{1:3} ``Yahvé'' es
  el nombre propio de Dios, a veces traducido como ``\textsc{Señor}''
  (en mayúsculas) en otras traducciones.} dijo a Elías tisbita:
``Levántate, sube a recibir a los mensajeros del rey de Samaria y diles:
`¿Es porque no hay un Dios\footnote{\textbf{1:3} La palabra hebrea
  traducida como ``Dios'' es ``\hebrew{אֱלֹהִ֑ים}'' (Elohim).} en Israel que
vas a consultar a Baal Zebub, el dios de Ecrón? \footnote{\textbf{1:3}
  Is 8,19} \bibleverse{4} Ahora, pues, Yahvé dice: ``No bajarás del
lecho al que has subido, sino que ciertamente morirás''\,''. Entonces
Elías partió.

\bibleverse{5} Los mensajeros volvieron a él y les dijo: ``¿Por qué
habéis vuelto?''.

\bibleverse{6} Le dijeron: ``Un hombre subió a nuestro encuentro y nos
dijo: ``Id, volved al rey que os ha enviado y decidle: ``Yahvé dice:
`¿Es porque no hay Dios en Israel que enviáis a consultar a Baal Zebub,
el dios de Ecrón? Por lo tanto, no bajarás del lecho al que has subido,
sino que ciertamente morirás'\,''.

\bibleverse{7} Les dijo: ``¿Qué clase de hombre es el que ha subido a
vuestro encuentro y os ha dicho estas palabras?''

\bibleverse{8} Le respondieron: ``Era un hombre velludo y llevaba un
cinturón de cuero en la cintura''. Dijo: ``Es Elías el tisbita''.
\footnote{\textbf{1:8} Zac 13,4; Mat 3,4}

\hypertarget{elijah-y-los-tres-capitanes}{%
\subsection{Elijah y los tres
capitanes}\label{elijah-y-los-tres-capitanes}}

\bibleverse{9} Entonces el rey envió a un capitán de cincuenta con sus
cincuenta a él. Subió hasta él; y he aquí que\footnote{\textbf{1:9} ``He
  aquí'', de ``\hebrew{הִנֵּה}'', significa mirar, fijarse, observar, ver o
  contemplar. Se utiliza a menudo como interjección.} estaba sentado en
la cima del monte. Le dijo: ``Hombre de Dios, el rey ha dicho que
bajes''.

\bibleverse{10} Elías respondió al capitán de los cincuenta: ``¡Si soy
un hombre de Dios, que baje fuego del cielo y te consuma a ti y a tus
cincuenta!'' Entonces bajó fuego del cielo y lo consumió a él y a sus
cincuenta. \footnote{\textbf{1:10} Luc 9,54; Apoc 11,5}

\bibleverse{11} De nuevo le envió otro capitán de cincuenta con sus
cincuenta. Él le respondió: ``Hombre de Dios, el rey ha dicho: ``¡Baja
rápido!''.

\bibleverse{12} Elías les respondió: ``¡Si soy un hombre de Dios, que
baje fuego del cielo y os consuma a vosotros y a vuestros cincuenta!''
Entonces el fuego de Dios bajó del cielo y lo consumió a él y a sus
cincuenta.

\bibleverse{13} Volvió a enviar al capitán de un tercer grupo de
cincuenta con sus cincuenta. El tercer capitán de los cincuenta subió, y
vino y se arrodilló ante Elías, y le rogó, y le dijo: ``Hombre de Dios,
por favor, haz que mi vida y la vida de estos cincuenta de tus siervos
sea preciosa ante tus ojos. \bibleverse{14} He aquí que ha bajado fuego
del cielo y ha consumido a los dos últimos capitanes de cincuenta con
sus cincuenta. Pero ahora haz que mi vida sea preciosa a tus ojos''.

\hypertarget{eluxedas-con-ochuxf4zuxedas-muerte-del-rey}{%
\subsection{Elías con Ochôzías; Muerte del
rey}\label{eluxedas-con-ochuxf4zuxedas-muerte-del-rey}}

\bibleverse{15} El ángel de Yahvé dijo a Elías: ``Baja con él. No le
tengas miedo''. Entonces se levantó y bajó con él al rey.
\bibleverse{16} Este le dijo: ``Yahvé dice: `Porque has enviado
mensajeros a consultar a Baal Zebub, el dios de Ecrón, ¿es que no hay
Dios en Israel para consultar su palabra? Por eso no bajarás del lecho
al que has subido, sino que morirás sin duda'\,''. \footnote{\textbf{1:16}
  2Re 1,3-4}

\bibleverse{17} Murió, pues, según la palabra de Yahvé que Elías había
pronunciado. Joram comenzó a reinar en su lugar en el segundo año de
Joram hijo de Josafat, rey de Judá, porque no tenía hijo. \footnote{\textbf{1:17}
  2Re 3,1} \bibleverse{18} El resto de los hechos de Ocozías, ¿no están
escritos en el libro de las crónicas de los reyes de Israel?

\hypertarget{eluxedas-en-la-caminata-con-su-fiel-sirvienta-elisa}{%
\subsection{Elías en la caminata con su fiel sirvienta
Elisa}\label{eluxedas-en-la-caminata-con-su-fiel-sirvienta-elisa}}

\hypertarget{section-1}{%
\section{2}\label{section-1}}

\bibleverse{1} Cuando Yahvé estaba a punto de llevarse a Elías en un
torbellino al cielo, Elías fue con Eliseo desde Gilgal. \bibleverse{2}
Elías le dijo a Eliseo: ``Por favor, espera aquí, porque Yahvé me ha
enviado hasta Betel''. Eliseo dijo: ``Vive Yahvé y vive tu alma, no te
dejaré''. Así que bajaron a Betel.

\bibleverse{3} Los hijos de los profetas que estaban en Betel salieron a
ver a Eliseo y le dijeron: ``¿Sabes que Yahvé te quitará hoy a tu
maestro de encima?'' Dijo: ``Sí, lo sé. No te metas en líos''.

\bibleverse{4} Elías le dijo: ``Eliseo, por favor, espera aquí, porque
Yahvé me ha enviado a Jericó''. Dijo: ``Vive Yahvé y vive tu alma, no te
dejaré''. Así que llegaron a Jericó.

\bibleverse{5} Los hijos de los profetas que estaban en Jericó se
acercaron a Eliseo y le dijeron: ``¿Sabes que Yahvé te quitará hoy a tu
maestro de encima?'' Él respondió: ``Sí, lo sé. Cállate''.

\bibleverse{6} Elías le dijo: ``Por favor, espera aquí, porque Yahvé me
ha enviado al Jordán''. Dijo: ``Vive Yahvé y vive tu alma, no te
dejaré''. Entonces ambos siguieron adelante. \bibleverse{7} Cincuenta
hombres de los hijos de los profetas fueron y se colocaron frente a
ellos a cierta distancia; y ambos se quedaron junto al Jordán.
\bibleverse{8} Elías tomó su manto, lo enrolló y golpeó las aguas, que
se dividieron aquí y allá, de modo que ambos pasaron en seco.
\footnote{\textbf{2:8} Éxod 14,21-22; Jos 3,16}

\hypertarget{eluxedas-se-despide-de-eliseo-su-ascensiuxf3n}{%
\subsection{Elías se despide de Eliseo; su
ascensión}\label{eluxedas-se-despide-de-eliseo-su-ascensiuxf3n}}

\bibleverse{9} Cuando hubieron pasado, Elías dijo a Eliseo: ``Pregunta
qué debo hacer por ti, antes de que me quiten.'' Eliseo dijo: ``Por
favor, que una doble porción de tu espíritu esté sobre mí''. \footnote{\textbf{2:9}
  Deut 21,17}

\bibleverse{10} Él dijo: ``Has pedido algo difícil. Si me ves cuando me
quiten, será así para ti; pero si no, no será así''.

\bibleverse{11} Mientras seguían hablando, he aquí que un carro de fuego
y caballos de fuego los separaban, y Elías subió al cielo en un
torbellino. \footnote{\textbf{2:11} Gén 5,24} \bibleverse{12} Eliseo lo
vio y gritó: ``¡Padre mío, padre mío, los carros de Israel y su
caballería!'' No lo vio más. Entonces tomó su propia ropa y la rompió en
dos pedazos. \footnote{\textbf{2:12} 2Re 13,14}

\hypertarget{el-regreso-de-eliseo-a-travuxe9s-del-jorduxe1n-a-jericuxf3-elijah-se-ha-ido}{%
\subsection{El regreso de Eliseo a través del Jordán a Jericó; Elijah se
ha
ido}\label{el-regreso-de-eliseo-a-travuxe9s-del-jorduxe1n-a-jericuxf3-elijah-se-ha-ido}}

\bibleverse{13} Tomó también el manto de Elías que se le había caído, y
regresó y se quedó a la orilla del Jordán. \footnote{\textbf{2:13} 2Re
  2,8} \bibleverse{14} Tomó el manto de Elías que se le había caído,
golpeó las aguas y dijo: ``¿Dónde está Yahvé, el Dios de Elías?'' Cuando
él también golpeó las aguas, se separaron, y Eliseo pasó.

\bibleverse{15} Cuando los hijos de los profetas que estaban en Jericó
frente a él lo vieron, dijeron: ``El espíritu de Elías reposa sobre
Eliseo''. Salieron a su encuentro y se postraron en tierra ante él.
\footnote{\textbf{2:15} 2Re 2,5; 2Re 2,7; Luc 1,17} \bibleverse{16} Le
dijeron: ``Mira ahora, hay con tus siervos cincuenta hombres fuertes.
Por favor, deja que vayan a buscar a tu amo. Tal vez el Espíritu de
Yahvé se lo haya llevado y lo haya puesto en alguna montaña o en algún
valle''. Dijo: ``No los envíes''.

\bibleverse{17} Cuando le insistieron hasta que se avergonzó, dijo:
``Envíalos''. Por eso enviaron a cincuenta hombres; y lo buscaron
durante tres días, pero no lo encontraron. \bibleverse{18} Volvieron a
buscarlo mientras se quedaba en Jericó, y él les dijo: ``¿No os he dicho
que no vayáis?''

\hypertarget{primera-apariciuxf3n-de-eliseo-el-milagro-del-agua-malsana-en-jericuxf3}{%
\subsection{Primera aparición de Eliseo: El milagro del agua malsana en
Jericó}\label{primera-apariciuxf3n-de-eliseo-el-milagro-del-agua-malsana-en-jericuxf3}}

\bibleverse{19} Los hombres de la ciudad dijeron a Eliseo: ``Mira, por
favor, la situación de esta ciudad es agradable, como ve mi señor; pero
el agua es mala y la tierra es estéril.''

\bibleverse{20} Dijo: ``Tráiganme un frasco nuevo y pongan sal en él''.
Entonces se la trajeron. \bibleverse{21} Salió al manantial de las
aguas, echó sal en él y dijo: ``Yahvé dice: `He sanado estas aguas. Ya
no habrá más muerte ni tierra estéril''. \bibleverse{22} Así quedaron
curadas las aguas hasta el día de hoy, según la palabra que Eliseo
pronunció.

\hypertarget{eliseo-y-los-chicos-malos-de-betel}{%
\subsection{Eliseo y los chicos malos de
Betel}\label{eliseo-y-los-chicos-malos-de-betel}}

\bibleverse{23} De allí subió a Betel. Mientras subía por el camino,
salieron de la ciudad unos jóvenes que se burlaban de él y le decían:
``¡Sube, calvo! Sube, calvo!'' \bibleverse{24} Él miró detrás de sí y
los vio, y los maldijo en nombre de Yavé. Entonces salieron del bosque
dos hembras de oso y mutilaron a cuarenta y dos de aquellos jóvenes.
\bibleverse{25} Se dirigió desde allí al monte Carmelo, y desde allí
regresó a Samaria. \footnote{\textbf{2:25} 2Re 4,25}

\hypertarget{rey-joram-de-israel}{%
\subsection{Rey Joram de Israel}\label{rey-joram-de-israel}}

\hypertarget{section-2}{%
\section{3}\label{section-2}}

\bibleverse{1} Joram, hijo de Ajab, comenzó a reinar sobre Israel en
Samaria en el año dieciocho de Josafat, rey de Judá, y reinó doce años.
\footnote{\textbf{3:1} 2Re 1,17} \bibleverse{2} Hizo lo que era malo a
los ojos de Yavé, pero no como su padre y como su madre, pues quitó la
columna de Baal que había hecho su padre. \footnote{\textbf{3:2} 1Re
  16,32} \bibleverse{3} Sin embargo, se aferró a los pecados de Jeroboam
hijo de Nabat, con los que hizo pecar a Israel. No se apartó de ellos.
\footnote{\textbf{3:3} 1Re 12,30}

\hypertarget{estallido-de-la-guerra-con-los-moabitas-el-pacto-de-joram-con-josafat-marcha-hacia-la-estepa-de-edom}{%
\subsection{Estallido de la guerra con los moabitas; El pacto de Joram
con Josafat; Marcha hacia la estepa de
Edom}\label{estallido-de-la-guerra-con-los-moabitas-el-pacto-de-joram-con-josafat-marcha-hacia-la-estepa-de-edom}}

\bibleverse{4} Mesá, rey de Moab, era criador de ovejas, y suministraba
al rey de Israel cien mil corderos y la lana de cien mil carneros.
\bibleverse{5} Pero cuando murió Acab, el rey de Moab se rebeló contra
el rey de Israel. \bibleverse{6} El rey Joram salió entonces de Samaria
y reunió a todo Israel. \bibleverse{7} Fue y envió a Josafat, rey de
Judá, diciendo: ``El rey de Moab se ha rebelado contra mí. ¿Vas a ir
conmigo contra Moab a la batalla?'' Dijo: ``Subiré. Yo soy como tú, mi
pueblo como tu pueblo, mis caballos como tus caballos''. \footnote{\textbf{3:7}
  1Re 22,4} \bibleverse{8} Entonces dijo: ``¿Por dónde subiremos?''
Joram respondió: ``El camino del desierto de Edom''.

\hypertarget{mala-situaciuxf3n-del-ejuxe9rcito-por-falta-de-agua-la-auspiciosa-profecuxeda-de-eliseo}{%
\subsection{Mala situación del ejército por falta de agua; La auspiciosa
profecía de
Eliseo}\label{mala-situaciuxf3n-del-ejuxe9rcito-por-falta-de-agua-la-auspiciosa-profecuxeda-de-eliseo}}

\bibleverse{9} El rey de Israel fue con el rey de Judá y el rey de Edom,
y marcharon durante siete días por una ruta tortuosa. No había agua para
el ejército ni para los animales que los seguían. \bibleverse{10} El rey
de Israel dijo: ``¡Ay! Porque Yahvé ha convocado a estos tres reyes para
entregarlos en manos de Moab''.

\bibleverse{11} Pero Josafat dijo: ``¿No hay aquí un profeta de Yahvé,
para que podamos consultar a Yahvé por medio de él?'' Uno de los siervos
del rey de Israel respondió: ``Eliseo, hijo de Safat, que derramó agua
sobre las manos de Elías, está aquí''. \footnote{\textbf{3:11} 1Re 22,5;
  1Re 22,7; 1Re 19,19; 1Re 19,21}

\bibleverse{12} Josafat dijo: ``La palabra de Yahvé está con él''.
Entonces el rey de Israel, Josafat y el rey de Edom bajaron hacia él.

\bibleverse{13} Eliseo dijo al rey de Israel: ``¿Qué tengo que hacer
contigo? Ve a los profetas de tu padre y a los profetas de tu madre''.
El rey de Israel le dijo: ``No, porque Yahvé ha convocado a estos tres
reyes para entregarlos a la mano de Moab''.

\bibleverse{14} Eliseo dijo: ``Vive Yahvé de los Ejércitos, ante quien
estoy, ciertamente, si no fuera porque respeto la presencia de Josafat,
rey de Judá, no miraría hacia ti ni te vería. \footnote{\textbf{3:14}
  1Re 18,15; Sal 15,4} \bibleverse{15} Pero ahora tráeme un músico''.
Cuando el músico tocó, la mano de Yavé se posó sobre él. \bibleverse{16}
Dijo: ``Yahvé dice: `Haz que este valle se llene de trincheras'.
\bibleverse{17} Porque Yahvé dice: `No verás viento, ni verás lluvia,
pero ese valle se llenará de agua y beberás, tanto tú como tu ganado y
tus otros animales. \bibleverse{18} Esto es algo fácil a los ojos del
Señor. También entregará a los moabitas en tu mano. \bibleverse{19}
Golpearás toda ciudad fortificada y toda ciudad selecta, y derribarás
todo árbol bueno, y detendrás todos los manantiales de agua, y
estropearás con piedras todo terreno bueno''.

\bibleverse{20} Por la mañana, a la hora de ofrecer el sacrificio, he
aquí que llegaron aguas por el camino de Edom, y el país se llenó de
agua.

\hypertarget{victoria-de-los-israelitas-mesa-sacrifica-a-su-hijo-primoguxe9nito-lo-que-hace-que-los-israelitas-se-vayan}{%
\subsection{Victoria de los israelitas; Mesa sacrifica a su hijo
primogénito, lo que hace que los israelitas se
vayan}\label{victoria-de-los-israelitas-mesa-sacrifica-a-su-hijo-primoguxe9nito-lo-que-hace-que-los-israelitas-se-vayan}}

\bibleverse{21} Cuando todos los moabitas se enteraron de que los reyes
habían subido a luchar contra ellos, se reunieron, todos los que podían
ponerse una armadura, jóvenes y viejos, y se pusieron en la frontera.
\bibleverse{22} Se levantaron de madrugada, y el sol brilló sobre el
agua, y los moabitas vieron el agua frente a ellos roja como la sangre.
\bibleverse{23} Dijeron: ``Esto es sangre. Los reyes están seguramente
destruidos, y se han golpeado mutuamente. Ahora, pues, Moab, ¡al
saqueo!''

\bibleverse{24} Cuando llegaron al campamento de Israel, los israelitas
se levantaron e hirieron a los moabitas, de modo que huyeron ante ellos;
y avanzaron por la tierra atacando a los moabitas. \bibleverse{25}
Derribaron las ciudades, y en cada pedazo de tierra buena cada uno echó
su piedra y la llenó. También detuvieron todos los manantiales de agua y
cortaron todos los árboles buenos, hasta que en Kir Hareset sólo
quedaron sus piedras; sin embargo, los hombres armados con hondas la
rodearon y la atacaron. \footnote{\textbf{3:25} 2Re 3,19}
\bibleverse{26} Cuando el rey de Moab vio que la batalla era demasiado
dura para él, tomó consigo a setecientos hombres que sacaban una espada,
para abrirse paso hasta el rey de Edom; pero no pudieron.
\bibleverse{27} Entonces tomó a su hijo mayor, que habría reinado en su
lugar, y lo ofreció en holocausto sobre el muro. Hubo gran ira contra
Israel, y se apartaron de él, y volvieron a su tierra.

\hypertarget{la-historia-del-cuxe1ntaro-de-aceite-de-la-viuda}{%
\subsection{La historia del cántaro de aceite de la
viuda}\label{la-historia-del-cuxe1ntaro-de-aceite-de-la-viuda}}

\hypertarget{section-3}{%
\section{4}\label{section-3}}

\bibleverse{1} Una mujer de las esposas de los hijos de los profetas
clamó a Eliseo diciendo: ``Tu siervo, mi esposo, ha muerto. Tú sabes que
tu siervo temía a Yahvé. Ahora el acreedor ha venido a tomar para sí a
mis dos hijos como esclavos''.

\bibleverse{2} Eliseo le dijo: ``¿Qué debo hacer por ti? Dime, ¿qué
tienes en la casa?'' Ella dijo: ``Tu siervo no tiene nada en la casa,
excepto una olla de aceite''. \footnote{\textbf{4:2} 1Re 17,12}

\bibleverse{3} Luego les dijo: ``Vayan y pidan prestados recipientes
vacíos a todos sus vecinos. No pidas prestados sólo algunos recipientes.
\bibleverse{4} Entra y cierra la puerta para ti y para tus hijos, y echa
aceite en todos esos recipientes; y aparta los que estén llenos.''

\bibleverse{5} Se separó de él y cerró la puerta para sí misma y para
sus hijos. Le trajeron los recipientes y ella echó aceite.
\bibleverse{6} Cuando los recipientes se llenaron, dijo a su hijo:
``Tráeme otro recipiente''. Le dijo: ``No hay otro recipiente''.
Entonces el aceite dejó de fluir.

\bibleverse{7} Entonces ella vino y se lo contó al hombre de Dios. Él le
dijo: ``Ve, vende el aceite y paga tu deuda; y tú y tus hijos vivid del
resto''.

\hypertarget{eliseo-y-sunamitin-eliseo-le-promete-un-hijo-al-sunamitin}{%
\subsection{Eliseo y Sunamitin; Eliseo le promete un hijo al
Sunamitin}\label{eliseo-y-sunamitin-eliseo-le-promete-un-hijo-al-sunamitin}}

\bibleverse{8} Un día Eliseo fue a Sunem, donde había una mujer
prominente, y ella lo convenció de que comiera pan. Así fue, que cada
vez que pasaba por allí, se volvía para comer pan. \footnote{\textbf{4:8}
  Jos 19,18} \bibleverse{9} Ella dijo a su marido: ``Mira ahora, percibo
que éste es un santo varón de Dios que pasa por delante de nosotros
continuamente. \bibleverse{10} Por favor, hagamos una pequeña habitación
en el techo. Pongamos allí una cama, una mesa, una silla y un candelabro
para él. Cuando venga a nosotros, podrá quedarse allí''.

\bibleverse{11} Un día llegó allí, entró en la habitación y se acostó.
\bibleverse{12} Dijo a Guejazi, su criado: ``Llama a esta sunamita''.
Cuando la llamó, ella se puso delante de él. \bibleverse{13} Él le dijo:
``Dile ahora: `Mira que nos has atendido con todos estos cuidados. ¿Qué
hay que hacer por ti? ¿Quieres que te hablen al rey o al capitán del
ejército?'' Ella respondió: ``Habito entre mi propia gente''.

\bibleverse{14} Dijo: ``¿Qué hay que hacer entonces por ella?'' Giezi
respondió: ``Ciertamente no tiene hijo, y su marido es viejo''.

\bibleverse{15} Él dijo: ``Llámala''. Cuando la llamó, ella se puso en
la puerta. \bibleverse{16} Le dijo: ``El año que viene, en esta época,
abrazarás un hijo''. Ella dijo: ``No, señor mío, hombre de Dios, no
mientas a tu siervo''. \footnote{\textbf{4:16} Gén 18,10; Gén 18,14}

\bibleverse{17} La mujer concibió y dio a luz un hijo en aquel tiempo,
como le había dicho Eliseo.

\hypertarget{la-muerte-del-niuxf1o-caminata-de-la-madre-a-elisa}{%
\subsection{La muerte del niño; Caminata de la madre a
Elisa}\label{la-muerte-del-niuxf1o-caminata-de-la-madre-a-elisa}}

\bibleverse{18} Cuando el niño creció, un día salió a ver a su padre a
los segadores. \bibleverse{19} Le dijo a su padre: ``¡Mi cabeza! Mi
cabeza!'' Dijo a su criado: ``Llévalo a su madre''.

\bibleverse{20} Cuando lo tomó y lo llevó a su madre, se sentó en sus
rodillas hasta el mediodía, y luego murió. \bibleverse{21} Ella subió,
lo puso en la cama del hombre de Dios, le cerró la puerta y salió.
\bibleverse{22} Llamó a su marido y le dijo: ``Te ruego que me envíes
uno de los criados y uno de los asnos, para que corra al hombre de Dios
y vuelva.''

\bibleverse{23} Él dijo: ``¿Por qué quieres ir a él hoy? No es luna
nueva ni sábado''. Ella dijo: ``Está bien''.

\bibleverse{24} Entonces ensilló un asno y dijo a su criado: ``¡Conduce
y avanza! No frenes por mí, si no te lo pido''.

\hypertarget{elisa-va-a-la-casa-de-la-madre}{%
\subsection{Elisa va a la casa de la
madre}\label{elisa-va-a-la-casa-de-la-madre}}

\bibleverse{25} Ella se fue y vino al hombre de Dios en el monte
Carmelo. Cuando el varón de Dios la vio de lejos, dijo a Giezi, su
siervo: ``Ahí está la sunamita. \footnote{\textbf{4:25} 2Re 2,25}
\bibleverse{26} Por favor, corre ahora a su encuentro y pregúntale:
``¿Te va bien? ¿Está bien tu marido? ¿Está bien tu hijo?'' Ella
respondió: ``Está bien''.

\bibleverse{27} Cuando se acercó al hombre de Dios en la colina, se
agarró a sus pies. Guejazi se acercó para empujarla; pero el hombre de
Dios dijo: ``Déjala, porque su alma está turbada dentro de ella, y Yahvé
me lo ha ocultado y no me lo ha dicho.''

\bibleverse{28} Entonces ella dijo: ``¿Acaso te pedí un hijo, mi señor?
¿No te dije que no me engañaras?'' \footnote{\textbf{4:28} 2Re 4,16}

\bibleverse{29} Entonces dijo a Guejazi: ``Mete tu capa en tu cinturón,
toma mi bastón en tu mano y sigue tu camino. Si te encuentras con algún
hombre, no lo saludes; y si alguien te saluda, no le vuelvas a
responder. Luego pon mi bastón en la cara del niño''. \footnote{\textbf{4:29}
  Luc 10,4}

\bibleverse{30} La madre del niño dijo: ``Vive Yahvé y vive tu alma, no
te dejaré''. Así que se levantó y la siguió.

\bibleverse{31} Gehazi se adelantó a ellos y puso el bastón sobre el
rostro del niño, pero no había voz ni oído. Por eso volvió a su
encuentro y le dijo: ``El niño no ha despertado''.

\bibleverse{32} Cuando Eliseo entró en la casa, he aquí que el niño
estaba muerto y acostado en su cama.

\hypertarget{reanimaciuxf3n-del-niuxf1o}{%
\subsection{Reanimación del niño}\label{reanimaciuxf3n-del-niuxf1o}}

\bibleverse{33} Entró, pues, y cerró la puerta a los dos, y oró a Yahvé.
\footnote{\textbf{4:33} Hech 9,40} \bibleverse{34} Subió y se acostó
sobre el niño, y puso su boca sobre su boca, y sus ojos sobre sus ojos,
y sus manos sobre sus manos. Se tendió sobre él, y la carne del niño se
calentó. \footnote{\textbf{4:34} 1Re 17,21} \bibleverse{35} Luego
regresó y se paseó por la casa una vez de un lado a otro, después subió
y se tendió sobre él. Entonces el niño estornudó siete veces, y el niño
abrió los ojos. \bibleverse{36} Llamó a Giezi y le dijo: ``¡Llama a esta
sunamita!'' Y la llamó. Cuando ella se acercó a él, le dijo: ``Toma a tu
hijo''. \footnote{\textbf{4:36} Luc 7,15; Heb 11,35}

\bibleverse{37} Entonces entró, se postró a sus pies y se inclinó hasta
el suelo; luego tomó a su hijo y salió.

\hypertarget{muerte-comida-venenosa-en-la-olla-y-la-maravillosa-alimentaciuxf3n-de-los-cien}{%
\subsection{Muerte (comida venenosa) en la olla y la maravillosa
alimentación de los
cien}\label{muerte-comida-venenosa-en-la-olla-y-la-maravillosa-alimentaciuxf3n-de-los-cien}}

\bibleverse{38} Eliseo llegó de nuevo a Gilgal. Había hambre en el país,
y los hijos de los profetas estaban sentados ante él; y dijo a su
criado: ``Trae la olla grande y hierve un guiso para los hijos de los
profetas.''

\bibleverse{39} Uno de ellos salió al campo a recoger hierbas, y
encontró una parra silvestre, de la que recogió un regazo lleno de
calabazas silvestres, y vino y las cortó en la olla del guiso, porque no
las reconocían. \bibleverse{40} Así echaron para que los hombres
comieran. Mientras comían un poco del guiso, gritaron y dijeron:
``¡Hombre de Dios, hay muerte en la olla!''. Y no pudieron comerlo.

\bibleverse{41} Pero él dijo: ``Entonces trae comida''. La echó en la
olla, y dijo: ``Sírvela al pueblo, para que coma''. Y no había nada malo
en la olla.

\bibleverse{42} Vino un hombre de Baal Salishah y le trajo al hombre de
Dios un poco de pan de las primicias: veinte panes de cebada y espigas
frescas en su saco. Eliseo le dijo: ``Dale al pueblo para que coma''.

\bibleverse{43} Su siervo dijo: ``¿Qué, debo exponer esto ante cien
hombres?'' Pero él dijo: ``Dáselo al pueblo, para que coma; porque Yahvé
dice: `Comerán y les sobrará'\,''. \footnote{\textbf{4:43} Juan 6,9; Mat
  15,33}

\bibleverse{44} Así que lo puso delante de ellos y comieron y sobró
algo, según la palabra de Yahvé. \footnote{\textbf{4:44} Mat 16,9-10}

\hypertarget{naeman-el-leproso-busca-sanidad-en-samaria}{%
\subsection{Naeman el leproso busca sanidad en
Samaria}\label{naeman-el-leproso-busca-sanidad-en-samaria}}

\hypertarget{section-4}{%
\section{5}\label{section-4}}

\bibleverse{1} Naamán, capitán del ejército del rey de Siria, era un
gran hombre con su amo, y honorable, porque por él Yahvé había dado la
victoria a Siria; era también un hombre valiente, pero era leproso.
\bibleverse{2} Los sirios habían salido en grupos y habían llevado
cautiva de la tierra de Israel a una niña, que atendía a la mujer de
Naamán. \bibleverse{3} Ella le dijo a su ama: ``¡Ojalá mi señor
estuviera con el profeta que está en Samaria! Entonces lo sanaría de su
lepra''.

\bibleverse{4} Alguien entró y se lo contó a su señor, diciendo: ``La
chica que es de la tierra de Israel dijo esto''.

\bibleverse{5} El rey de Siria dijo: ``Ve ahora y enviaré una carta al
rey de Israel''. Partió, y tomó consigo diez talentos\footnote{\textbf{5:5}
  Un talento son unos 30 kilogramos o 66 libras} de plata, seis mil
piezas de oro y diez mudas de ropa. \bibleverse{6} Llevó la carta al rey
de Israel, diciendo: ``Cuando te llegue esta carta, he aquí que he
enviado a mi siervo Naamán a ti, para que lo cures de su lepra.''

\bibleverse{7} Cuando el rey de Israel leyó la carta, se rasgó las
vestiduras y dijo: ``¿Soy yo Dios, para matar y dar vida, para que este
hombre me envíe a curar a un hombre de su lepra? Pero, por favor,
considera y ve cómo busca un pleito contra mí''. \footnote{\textbf{5:7}
  1Re 20,7}

\hypertarget{la-curaciuxf3n-de-naeman-a-travuxe9s-de-eliseo}{%
\subsection{La curación de Naeman a través de
Eliseo}\label{la-curaciuxf3n-de-naeman-a-travuxe9s-de-eliseo}}

\bibleverse{8} Cuando Eliseo, el hombre de Dios, oyó que el rey de
Israel se había rasgado las vestiduras, envió a decir al rey: ``¿Por qué
te has rasgado las vestiduras? Que venga ahora a mí, y sabrá que hay un
profeta en Israel''.

\bibleverse{9} Entonces Naamán vino con sus caballos y con sus carros, y
se paró a la puerta de la casa de Eliseo. \bibleverse{10} Eliseo le
envió un mensajero, diciendo: ``Ve y lávate en el Jordán siete veces, y
tu carne volverá a ti y quedarás limpio''.

\bibleverse{11} Pero Naamán se enojó, y se fue diciendo: ``He aquí, yo
pensaba: `Seguramente saldrá a mí, y se pondrá de pie, e invocará el
nombre de Yahvé su Dios, y agitará su mano sobre el lugar, y sanará al
leproso'. \bibleverse{12} ¿No son Abaná y Farfar, los ríos de Damasco,
mejores que todas las aguas de Israel? ¿No podría yo lavarme en ellos y
quedar limpio?''. Así que se dio la vuelta y se marchó furioso.

\bibleverse{13} Sus criados se acercaron y le hablaron diciendo: ``Padre
mío, si el profeta te hubiera pedido que hicieras alguna cosa grande,
¿no la habrías hecho? ¿Cuánto más cuando te dice: `Lávate y queda
limpio'?''

\bibleverse{14} Entonces descendió y se sumergió siete veces en el
Jordán, según el dicho del hombre de Dios; y su carne se restauró como
la carne de un niño pequeño, y quedó limpio. \footnote{\textbf{5:14} Luc
  4,27}

\hypertarget{acciuxf3n-de-gracias-y-alabanza-de-naeman-a-dios}{%
\subsection{Acción de gracias y alabanza de Naeman a
Dios}\label{acciuxf3n-de-gracias-y-alabanza-de-naeman-a-dios}}

\bibleverse{15} Volvió al hombre de Dios, él y toda su compañía, y vino
y se puso de pie ante él, y dijo: ``Mira ahora, yo sé que no hay Dios en
toda la tierra, sino en Israel. Ahora, pues, te ruego que aceptes un
regalo de tu siervo''. \footnote{\textbf{5:15} 2Re 5,5}

\bibleverse{16} Pero él dijo: ``Vive Yahvé, ante quien estoy, no
recibiré a ninguno''. Le instó a que lo tomara, pero él se negó.
\bibleverse{17} Naamán dijo: ``Si no es así, por favor, dale a tu siervo
dos mulas de tierra, porque tu siervo no ofrecerá de ahora en adelante
ni holocaustos ni sacrificios a otros dioses, sino a Yavé.
\bibleverse{18} Que Yahvé perdone a tu siervo en esto: cuando mi amo
entre en la casa de Rimón para adorar allí, y se apoye en mi mano, y yo
me incline en la casa de Rimón. Cuando me inclino en la casa de Rimón,
que el Señor perdone a tu siervo en esto''. \footnote{\textbf{5:18} 2Re
  7,2}

\bibleverse{19} Le dijo: ``Ve en paz''. Y se alejó de él un poco.
\bibleverse{20} Pero Giezi, siervo de Eliseo, el hombre de Dios, dijo:
``He aquí que mi amo ha perdonado a este Naamán el sirio, al no recibir
de sus manos lo que ha traído. Vive Yahvé, que correré tras él y tomaré
algo de él''.

\bibleverse{21} Entonces Giezi siguió a Naamán. Cuando Naamán vio que
uno corría detrás de él, bajó del carro a su encuentro y le dijo:
``¿Está todo bien?''.

\bibleverse{22} Él dijo: ``Todo está bien. Mi amo me ha enviado
diciendo: `He aquí que ahora mismo han venido a mí, de la región
montañosa de Efraín, dos jóvenes de los hijos de los profetas. Por
favor, dales un talento\footnote{\textbf{5:22} Un codo es la longitud
  desde la punta del dedo corazón hasta el codo del brazo de un hombre,
  es decir, unas 18 pulgadas o 46 centímetros.} de plata y dos mudas de
ropa'\,''.

\bibleverse{23} Naamán dijo: ``Tengan a bien tomar dos talentos''. Él lo
instó, y ató dos talentos de plata en dos bolsas, con dos mudas de ropa,
y se los puso a dos de sus siervos; y ellos los llevaron delante de él.
\bibleverse{24} Cuando llegó al monte, se los quitó de las manos y los
guardó en la casa. Luego dejó ir a los hombres y se marcharon.
\bibleverse{25} Pero él entró y se puso delante de su amo. Eliseo le
dijo: ``¿De dónde vienes, Guejazi?'' Dijo: ``Su servidor no fue a
ninguna parte''.

\bibleverse{26} Le dijo: ``¿No te acompañó mi corazón cuando el hombre
se apartó de su carro para salir a tu encuentro? ¿Acaso es tiempo de
recibir dinero, y de recibir vestidos, y olivares y viñas, y ovejas y
ganado, y siervos y siervas? \bibleverse{27} Por eso, la lepra de Naamán
se pegará a ti y a tu descendencia para siempre''. Salió de su presencia
como un leproso, blanco como la nieve.

\hypertarget{el-hierro-flotante}{%
\subsection{El hierro flotante}\label{el-hierro-flotante}}

\hypertarget{section-5}{%
\section{6}\label{section-5}}

\bibleverse{1} Los hijos de los profetas dijeron a Eliseo: ``Mira ahora,
el lugar donde vivimos y nos reunimos contigo es demasiado pequeño para
nosotros. \bibleverse{2} Por favor, vayamos al Jordán, y cada uno tome
una viga de allí, y hagamos allí un lugar donde podamos vivir.'' Él
respondió: ``¡Vete!''

\bibleverse{3} Uno dijo: ``Por favor, tened el gusto de ir con vuestros
siervos''. Él respondió: ``Iré''. \bibleverse{4} Así que se fue con
ellos. Cuando llegaron al Jordán, cortaron leña. \bibleverse{5} Pero
cuando uno estaba cortando un árbol, la cabeza del hacha cayó al agua.
Entonces gritó y dijo: ``¡Ay, señor mío! Porque era prestada''.

\bibleverse{6} El hombre de Dios preguntó: ``¿Dónde cayó?''. Le mostró
el lugar. Cortó un palo, lo arrojó allí e hizo flotar el hierro.
\bibleverse{7} Le dijo: ``Tómalo''. Así que alargó la mano y lo cogió.

\hypertarget{la-emboscada-traicionada-varias-veces}{%
\subsection{La emboscada traicionada varias
veces}\label{la-emboscada-traicionada-varias-veces}}

\bibleverse{8} El rey de Siria estaba en guerra contra Israel, y se
aconsejó con sus siervos, diciendo: ``Mi campamento estará en tal y tal
lugar''.

\bibleverse{9} El hombre de Dios envió a decir al rey de Israel: ``Ten
cuidado de no pasar por este lugar, porque los sirios bajan por allí''.
\bibleverse{10} El rey de Israel envió al lugar que el hombre de Dios le
había dicho y advertido, y se salvó allí, ni una ni dos veces.
\bibleverse{11} El corazón del rey de Siria se turbó mucho por esto.
Llamó a sus siervos y les dijo: ``¿No queréis mostrarme cuál de los dos
es para el rey de Israel?''

\bibleverse{12} Uno de sus siervos dijo: ``No, mi señor, oh rey; pero
Eliseo, el profeta que está en Israel, le cuenta al rey de Israel las
palabras que habla en su alcoba''.

\hypertarget{el-cegamiento-de-los-sirios}{%
\subsection{El cegamiento de los
sirios}\label{el-cegamiento-de-los-sirios}}

\bibleverse{13} Dijo: ``Ve a ver dónde está, para que envíe a
buscarlo''. Se le dijo: ``He aquí que está en Dotán''.

\bibleverse{14} Por eso envió allí caballos, carros y un gran ejército.
Llegaron de noche y rodearon la ciudad. \bibleverse{15} Cuando el siervo
del hombre de Dios se levantó de madrugada y salió, he aquí que un
ejército con caballos y carros rodeaba la ciudad. Su siervo le dijo:
``¡Ay, señor mío! ¿Qué haremos?''

\bibleverse{16} Él respondió: ``No temas, porque los que están con
nosotros son más que los que están con ellos''. \footnote{\textbf{6:16}
  2Cró 32,7} \bibleverse{17} Eliseo oró y dijo: ``Yahvé, por favor, abre
sus ojos para que pueda ver.'' Yahvé abrió los ojos del joven y vio; y
he aquí que la montaña estaba llena de caballos y carros de fuego
alrededor de Eliseo. \bibleverse{18} Cuando bajaron hacia él, Eliseo oró
a Yahvé y dijo: ``Por favor, hiere a este pueblo con ceguera''. Los
golpeó con ceguera según la palabra de Eliseo. \footnote{\textbf{6:18}
  Gén 19,11}

\bibleverse{19} Eliseo les dijo: ``Este no es el camino, ni esta es la
ciudad. Seguidme, y os llevaré al hombre que buscáis''. Los condujo a
Samaria. \bibleverse{20} Cuando llegaron a Samaria, Eliseo dijo:
``Yahvé, abre los ojos de estos hombres para que vean''. El Señor les
abrió los ojos, y vieron; y he aquí que estaban en medio de Samaria.

\bibleverse{21} El rey de Israel dijo a Eliseo, al verlos: ``Padre mío,
¿los golpearé? ¿Los golpeo?''

\bibleverse{22} Él respondió: ``No los golpearás. ¿Golpearías a los que
has llevado cautivos con tu espada y con tu arco? Pon delante de ellos
pan y agua, para que coman y beban, y luego vayan a su amo''.
\footnote{\textbf{6:22} Prov 25,21; 2Cró 28,15}

\bibleverse{23} Les preparó un gran banquete. Después de que comieron y
bebieron, los despidió y se fueron con su amo. Entonces las bandas de
Siria dejaron de asaltar la tierra de Israel.

\hypertarget{asedio-de-samaria-y-hambre}{%
\subsection{Asedio de Samaria y
hambre}\label{asedio-de-samaria-y-hambre}}

\bibleverse{24} Después de esto, Benhadad, rey de Siria, reunió a todo
su ejército y subió a sitiar Samaria. \bibleverse{25} Hubo una gran
hambruna en Samaria. La sitiaron hasta que una cabeza de asno se vendió
por ochenta monedas de plata, y la cuarta parte de un kab de estiércol
de paloma por cinco monedas de plata. \bibleverse{26} Cuando el rey de
Israel pasaba por la muralla, una mujer le gritó diciendo: ``¡Socorro,
mi señor, oh rey!''

\bibleverse{27} Dijo: ``Si Yahvé no te ayuda, ¿de dónde podría sacar
ayuda para ti? ¿De la era, o del lagar?'' \bibleverse{28} Entonces el
rey le preguntó: ``¿Cuál es tu problema?'' Ella respondió: ``Esta mujer
me dijo: `Entrega a tu hijo, para que lo comamos hoy, y mañana comeremos
a mi hijo'. \bibleverse{29} Así que hervimos a mi hijo y nos lo comimos;
y al día siguiente le dije a ella: `Entrega a tu hijo, para que nos lo
comamos'; y ella ha escondido a su hijo.'' \footnote{\textbf{6:29} Deut
  28,53}

\bibleverse{30} Cuando el rey oyó las palabras de la mujer, rasgó sus
vestidos. Pasaba por el muro, y la gente miró, y he aquí que tenía un
saco debajo de su cuerpo. \bibleverse{31} Entonces dijo: ``Que Dios me
haga así, y más aún, si la cabeza de Eliseo, hijo de Safat, permanece
hoy sobre él.''

\hypertarget{la-promesa-de-suerte-de-eliseo-para-la-ciudad}{%
\subsection{La promesa de suerte de Eliseo para la
ciudad}\label{la-promesa-de-suerte-de-eliseo-para-la-ciudad}}

\bibleverse{32} Pero Eliseo estaba sentado en su casa, y los ancianos
estaban sentados con él. Entonces el rey envió a un hombre de su parte;
pero antes de que el mensajero llegara a él, dijo a los ancianos:
``¿Veis cómo este hijo de un asesino ha enviado a quitarme la cabeza?
Mirad, cuando venga el mensajero, cerrad la puerta y mantenedla cerrada
contra él. ¿No se oye el ruido de los pies de su amo detrás de él?''

\bibleverse{33} Mientras aún hablaba con ellos, he aquí que el mensajero
descendió hacia él. Entonces dijo: ``He aquí que este mal viene de
Yahvé. ¿Por qué he de esperar más a Yahvé?''. \footnote{\textbf{6:33} Am
  3,6}

\hypertarget{section-6}{%
\section{7}\label{section-6}}

\bibleverse{1} Eliseo dijo: ``Escuchen la palabra de Yahvé. Yahvé dice:
`Mañana a esta hora se venderá un seah de harina fina por un siclo, y
dos seah de cebada por un siclo, en la puerta de Samaria'\,''.
\footnote{\textbf{7:1} 2Re 7,16}

\bibleverse{2} Entonces el capitán en cuya mano se apoyaba el rey
respondió al hombre de Dios y dijo: ``He aquí que si Yahvé hizo ventanas
en el cielo, ¿podría ser esto?'' Dijo: ``He aquí que lo veréis con
vuestros ojos, pero no comeréis de él''. \footnote{\textbf{7:2} 2Re
  7,17; 2Re 5,18}

\hypertarget{experiencias-de-los-cuatro-leprosos-en-el-campamento-sirio}{%
\subsection{Experiencias de los cuatro leprosos en el campamento
sirio}\label{experiencias-de-los-cuatro-leprosos-en-el-campamento-sirio}}

\bibleverse{3} Había cuatro leprosos a la entrada de la puerta. Se
dijeron unos a otros: ``¿Por qué nos sentamos aquí hasta que muramos?
\footnote{\textbf{7:3} Lev 13,46} \bibleverse{4} Si decimos: `Vamos a
entrar en la ciudad', entonces el hambre está en la ciudad y moriremos
allí. Si nos quedamos aquí sentados, también moriremos. Ahora, pues,
venid y entreguémonos al ejército de los sirios. Si nos salvan con vida,
viviremos; y si nos matan, sólo moriremos''. \footnote{\textbf{7:4} Est
  4,16}

\bibleverse{5} Se levantaron en el crepúsculo para ir al campamento de
los sirios. Cuando llegaron a la parte más alejada del campamento de los
sirios, he aquí que no había nadie allí. \bibleverse{6} Porque el Señor
había hecho oír al ejército de los sirios el ruido de los carros y el
ruido de los caballos, el ruido de un gran ejército; y se dijeron unos a
otros: ``He aquí que el rey de Israel ha contratado contra nosotros a
los reyes de los hititas y a los reyes de los egipcios para que nos
ataquen.'' \footnote{\textbf{7:6} 2Re 19,7} \bibleverse{7} Se
levantaron, pues, y huyeron en el crepúsculo, y dejaron sus tiendas, sus
caballos y sus asnos, y el campamento tal como estaba, y huyeron por su
vida. \bibleverse{8} Cuando estos leprosos llegaron a la parte más
alejada del campamento, entraron en una tienda y comieron y bebieron;
luego se llevaron plata, oro y ropa y fueron a esconderlos. Luego
volvieron, entraron en otra tienda y también se llevaron cosas de allí,
y fueron a esconderlas. \bibleverse{9} Entonces se dijeron unos a otros:
``No estamos haciendo bien las cosas. Hoy es un día de buenas noticias,
y guardamos silencio. Si esperamos hasta la luz de la mañana, el castigo
nos alcanzará. Ahora, pues, venid, vamos a contárselo a la casa del
rey''.

\hypertarget{reporta-los-leprosos-en-la-ciudad-y-sus-efectos}{%
\subsection{Reporta los leprosos en la ciudad y sus
efectos}\label{reporta-los-leprosos-en-la-ciudad-y-sus-efectos}}

\bibleverse{10} Vinieron, pues, y llamaron a los porteros de la ciudad,
y les dijeron: ``Hemos llegado al campamento de los sirios, y he aquí
que no había allí ningún hombre, ni siquiera una voz de hombre, sino los
caballos atados, los asnos atados y las tiendas tal como estaban.''

\bibleverse{11} Entonces los porteros dieron la voz de alarma y se lo
contaron a la casa del rey que estaba dentro.

\bibleverse{12} El rey se levantó por la noche y dijo a sus siervos:
``Ahora les mostraré lo que nos han hecho los sirios. Saben que tenemos
hambre. Por eso han salido del campamento para esconderse en el campo,
diciendo: `Cuando salgan de la ciudad, los tomaremos vivos y entraremos
en la ciudad'.''

\bibleverse{13} Uno de sus siervos respondió: ``Por favor, deja que
algunas personas tomen cinco de los caballos que quedan, que han quedado
en la ciudad. He aquí que son como toda la multitud de Israel que ha
quedado en ella. He aquí que son como toda la multitud de Israel que ha
sido consumida. Enviemos y veamos''.

\bibleverse{14} Por lo tanto, tomaron dos carros con caballos, y el rey
los envió al ejército sirio, diciendo: ``Vayan y vean''.

\hypertarget{la-profecuxeda-de-eliseo-se-hace-realidad}{%
\subsection{La profecía de Eliseo se hace
realidad}\label{la-profecuxeda-de-eliseo-se-hace-realidad}}

\bibleverse{15} Fueron tras ellos hasta el Jordán, y he aquí que todo el
camino estaba lleno de ropas y equipos que los sirios habían arrojado en
su apuro. Los mensajeros volvieron y se lo comunicaron al rey.
\bibleverse{16} El pueblo salió y saqueó el campamento de los sirios.
Así, un seah de harina fina se vendió por un siclo, y dos medidas de
cebada por un siclo, según la palabra de Yavé. \footnote{\textbf{7:16}
  2Re 7,1} \bibleverse{17} El rey había designado al capitán en cuya
mano se apoyó para que estuviera a cargo de la puerta; y el pueblo lo
pisoteó en la puerta, y murió como había dicho el hombre de Dios, que
habló cuando el rey bajó a él. \footnote{\textbf{7:17} 2Re 7,2}
\bibleverse{18} Sucedió así como el hombre de Dios había hablado al rey,
diciendo: ``Dos seahs de cebada por un siclo, y un seah de harina fina
por un siclo, estarán mañana a esta hora en la puerta de Samaria;''
\bibleverse{19} y aquel capitán respondió al hombre de Dios, y dijo:
``Ahora bien, si Yahvé hiciera ventanas en el cielo, ¿podría ser tal
cosa?'' y dijo: ``He aquí, lo verás con tus ojos, pero no comerás de
él.'' \bibleverse{20} Así le sucedió, pues el pueblo lo pisoteó en la
puerta, y murió.

\hypertarget{elisa-y-la-sunamita}{%
\subsection{Elisa y la Sunamita}\label{elisa-y-la-sunamita}}

\hypertarget{section-7}{%
\section{8}\label{section-7}}

\bibleverse{1} Eliseo había hablado con la mujer a cuyo hijo había
devuelto la vida, diciéndole: ``Levántate y vete, tú y tu familia, y
quédate por un tiempo donde puedas; porque Yahvé ha convocado una
hambruna. También vendrá sobre la tierra durante siete años''.
\footnote{\textbf{8:1} 2Re 4,35}

\bibleverse{2} La mujer se levantó e hizo lo que le dijo el hombre de
Dios. Se fue con su familia y vivió en la tierra de los filisteos
durante siete años. \bibleverse{3} Al cabo de los siete años, la mujer
regresó de la tierra de los filisteos. Entonces salió a rogar al rey por
su casa y por su tierra. \bibleverse{4} El rey estaba hablando con
Giezi, el siervo del hombre de Dios, diciendo: ``Por favor, cuéntame
todas las grandes cosas que ha hecho Eliseo.'' \bibleverse{5} Mientras
él le contaba al rey cómo había devuelto la vida al que estaba muerto,
he aquí que la mujer a cuyo hijo había devuelto la vida le rogó al rey
por su casa y por su tierra. Giezi dijo: ``Señor mío, oh rey, ésta es la
mujer y éste es su hijo, al que Eliseo devolvió la vida.''

\bibleverse{6} Cuando el rey preguntó a la mujer, ella se lo contó.
Entonces el rey le asignó un oficial, diciendo: ``Devuélvele todo lo que
era suyo, y todos los frutos del campo desde el día en que dejó la
tierra, hasta ahora''.

\hypertarget{eliseo-en-damasco-preguntado-por-hazael-sobre-el-rey-ben-adad-enfermo}{%
\subsection{Eliseo en Damasco preguntado por Hazael sobre el rey
Ben-adad
enfermo}\label{eliseo-en-damasco-preguntado-por-hazael-sobre-el-rey-ben-adad-enfermo}}

\bibleverse{7} Eliseo llegó a Damasco, y Benhadad, rey de Siria, estaba
enfermo. Se le dijo: ``El hombre de Dios ha venido aquí''.

\bibleverse{8} El rey dijo a Hazael: ``Toma un regalo en tu mano y ve a
encontrarte con el hombre de Dios y consulta a Yahvé por él, diciendo:
``¿Me recuperaré de esta enfermedad?''\,''.

\bibleverse{9} Entonces Hazael salió a su encuentro y tomó un regalo de
todo lo bueno de Damasco, cuarenta camellos de carga, y vino y se puso
delante de él y le dijo: ``Tu hijo Benhadad, rey de Siria, me ha enviado
a ti, diciendo: ``¿Me recuperaré de esta enfermedad?''\,''.

\hypertarget{la-apertura-de-eliseo-a-hazael-el-asesinato-de-benhadad-hasael-asumiuxf3-el-cargo}{%
\subsection{La apertura de Eliseo a Hazael; El asesinato de Benhadad;
Hasael asumió el
cargo}\label{la-apertura-de-eliseo-a-hazael-el-asesinato-de-benhadad-hasael-asumiuxf3-el-cargo}}

\bibleverse{10} Eliseo le dijo: ``Ve y dile: `Seguramente te
recuperarás'; sin embargo, Yahvé me ha mostrado que seguramente
morirá''. \bibleverse{11} Y fijó su mirada en él, hasta que se
avergonzó. Entonces el hombre de Dios lloró. \footnote{\textbf{8:11} Luc
  19,41}

\bibleverse{12} Hazael dijo: ``¿Por qué lloras, mi señor?'' Él
respondió: ``Porque sé el mal que harás a los hijos de Israel. Prenderás
fuego a sus fortalezas, y matarás a sus jóvenes a espada, y despedazarás
a sus pequeños, y desgarrarás a sus mujeres embarazadas.'' \footnote{\textbf{8:12}
  2Re 10,32}

\bibleverse{13} Hazael dijo: ``¿Pero qué es tu siervo, que no es más que
un perro, para que pueda hacer esta gran cosa?'' Eliseo respondió:
``Yahvé me ha mostrado que serás rey de Siria''. \footnote{\textbf{8:13}
  1Sam 24,14; 1Re 19,15}

\bibleverse{14} Entonces se apartó de Eliseo y se acercó a su amo, que
le dijo: ``¿Qué te ha dicho Eliseo?''. Respondió: ``Me dijo que
seguramente te recuperarías''.

\bibleverse{15} Al día siguiente, tomó un paño grueso, lo mojó en agua y
lo extendió sobre el rostro del rey, de modo que éste murió. Entonces
Hazael reinó en su lugar.

\hypertarget{joram-y-ocozuxedas-su-hijo-reyes-de-juduxe1}{%
\subsection{Joram y Ocozías su hijo, reyes de
Judá}\label{joram-y-ocozuxedas-su-hijo-reyes-de-juduxe1}}

\bibleverse{16} En el quinto año de Joram hijo de Acab, rey de Israel,
siendo entonces Josafat rey de Judá, comenzó a reinar Joram hijo de
Josafat, rey de Judá. \footnote{\textbf{8:16} 1Re 22,50; 2Cró 21,1; 2Cró
  21,5-10} \bibleverse{17} Tenía treinta y dos años cuando comenzó a
reinar. Reinó ocho años en Jerusalén. \bibleverse{18} Siguió el camino
de los reyes de Israel, al igual que la casa de Ajab, pues se casó con
la hija de éste. Hizo lo que era malo a los ojos del Señor. \footnote{\textbf{8:18}
  2Re 8,26} \bibleverse{19} Sin embargo, el Señor no quiso destruir a
Judá por amor a David, su siervo, pues le prometió que le daría siempre
una lámpara para sus hijos. \footnote{\textbf{8:19} 2Sam 7,11-16; 1Re
  11,36}

\hypertarget{la-cauxedda-de-los-edomitas-y-la-muerte-de-joram}{%
\subsection{La caída de los edomitas y la muerte de
Joram}\label{la-cauxedda-de-los-edomitas-y-la-muerte-de-joram}}

\bibleverse{20} En sus días Edom se rebeló de la mano de Judá y se hizo
un rey sobre ellos. \bibleverse{21} Entonces Joram cruzó a Zair, y todos
sus carros con él; se levantó de noche e hirió a los edomitas que lo
rodeaban con los capitanes de los carros; y el pueblo huyó a sus
tiendas. \bibleverse{22} Así Edom se rebeló de la mano de Judá hasta el
día de hoy. También Libna se rebeló al mismo tiempo. \bibleverse{23} Los
demás hechos de Joram y todo lo que hizo, ¿no están escritos en el libro
de las crónicas de los reyes de Judá? \bibleverse{24} Joram durmió con
sus padres y fue enterrado con ellos en la ciudad de David, y su hijo
Ocozías reinó en su lugar.

\hypertarget{ochuxf4zuxedas-de-juduxe1-guerra-con-hazael}{%
\subsection{Ochôzías de Judá; Guerra con
Hazael}\label{ochuxf4zuxedas-de-juduxe1-guerra-con-hazael}}

\bibleverse{25} En el duodécimo año de Joram hijo de Acab, rey de
Israel, comenzó a reinar Ocozías hijo de Joram, rey de Judá.
\bibleverse{26} Ocozías tenía veintidós años cuando comenzó a reinar, y
reinó un año en Jerusalén. Su madre se llamaba Atalía, hija de Omri, rey
de Israel. \footnote{\textbf{8:26} 2Re 8,18; 2Re 11,1} \bibleverse{27}
Anduvo en el camino de la casa de Acab e hizo lo que era malo a los ojos
de Yahvé, al igual que la casa de Acab, pues era yerno de la casa de
Acab.

\bibleverse{28} Fue con Joram, hijo de Ajab, a la guerra contra Hazael,
rey de Siria, en Ramot de Galaad, y los sirios hirieron a Joram.
\bibleverse{29} El rey Joram regresó para curarse en Jezreel de las
heridas que los sirios le habían hecho en Ramá, cuando luchó contra
Hazael, rey de Siria. Ocozías hijo de Joram, rey de Judá, bajó a ver a
Joram hijo de Acab en Jezreel, porque estaba enfermo. \footnote{\textbf{8:29}
  2Re 9,15-16; 2Re 9,21}

\hypertarget{jehuxfa-ungiuxf3-rey-por-instigaciuxf3n-de-eliseo}{%
\subsection{Jehú ungió rey por instigación de
Eliseo}\label{jehuxfa-ungiuxf3-rey-por-instigaciuxf3n-de-eliseo}}

\hypertarget{section-8}{%
\section{9}\label{section-8}}

\bibleverse{1} El profeta Eliseo llamó a uno de los hijos de los
profetas y le dijo: ``Ponte el cinturón en la cintura, toma esta vasija
de aceite en tu mano y ve a Ramot de Galaad. \bibleverse{2} Cuando
llegues allí, busca a Jehú, hijo de Josafat, hijo de Nimsí, y entra y
haz que se levante de entre sus hermanos, y llévalo a una habitación
interior. \bibleverse{3} Luego toma la vasija de aceite y derrámala
sobre su cabeza, y di: ``Yahvé dice: ``Te he ungido como rey sobre
Israel''\,''. Entonces abre la puerta, huye y no esperes''. \footnote{\textbf{9:3}
  1Re 19,16}

\bibleverse{4} El joven profeta fue a Ramot de Galaad. \bibleverse{5}
Cuando llegó, he aquí que los capitanes del ejército estaban sentados.
Entonces dijo: ``Tengo un mensaje para ti, capitán''. Jehú dijo: ``¿A
quién de nosotros?'' Dijo: ``A ti, oh capitán''. \bibleverse{6} Se
levantó y entró en la casa. Luego derramó el aceite sobre su cabeza y le
dijo: ``Yahvé, el Dios de Israel, dice: `Te he ungido rey sobre el
pueblo de Yahvé, sobre Israel. \bibleverse{7} Debes golpear la casa de
tu amo Ajab, para que yo vengue la sangre de mis siervos los profetas, y
la sangre de todos los siervos de Yavé, a manos de Jezabel. \footnote{\textbf{9:7}
  1Re 21,22} \bibleverse{8} Porque toda la casa de Ajab perecerá.
Cortaré de Ajab a todo el que orine contra una pared, tanto al que está
encerrado como al que queda suelto en Israel. \footnote{\textbf{9:8} 1Re
  14,10} \bibleverse{9} Haré que la casa de Acab sea como la casa de
Jeroboam, hijo de Nabat, y como la casa de Baasa, hijo de Ahías.
\footnote{\textbf{9:9} 1Re 15,29; 1Re 16,3; 1Re 16,11} \bibleverse{10}
Los perros se comerán a Jezabel en la parcela de Jezreel, y no habrá
quien la entierre.'\,'' Entonces abrió la puerta y huyó.

\hypertarget{jehuxfa-reconocido-como-rey-por-los-luxedderes-militares}{%
\subsection{Jehú reconocido como rey por los líderes
militares}\label{jehuxfa-reconocido-como-rey-por-los-luxedderes-militares}}

\bibleverse{11} Cuando Jehú salió a ver a los siervos de su señor y uno
le dijo: ``¿Está todo bien? ¿Por qué ha venido a ti este loco?'' Les
dijo: ``Ya conocéis al hombre y su forma de hablar''.

\bibleverse{12} Ellos dijeron: ``Eso es mentira. Dinos ahora''. Me dijo:
``Dice Yahvé que te he ungido como rey de Israel''.

\bibleverse{13} Entonces se apresuraron, y cada uno tomó su manto y lo
puso debajo de él en lo alto de la escalera, y tocaron la trompeta,
diciendo: ``Jehú es rey.'' \footnote{\textbf{9:13} Mat 21,7}

\bibleverse{14} Entonces Jehú, hijo de Josafat, hijo de Nimsí, conspiró
contra Joram. (Joram estaba defendiendo Ramot de Galaad, él y todo
Israel, a causa de Hazael, rey de Siria; \bibleverse{15} pero el rey
Joram había regresado para curarse en Jezreel de las heridas que los
sirios le habían hecho cuando luchó con Hazael, rey de Siria). Jehú
dijo: ``Si este es tu pensamiento, que nadie se escape y salga de la
ciudad para ir a contarlo en Jezreel''. \footnote{\textbf{9:15} 2Re
  8,28-29}

\hypertarget{jehuxfa-mata-a-joram-y-ochuxf4zuxedas}{%
\subsection{Jehú mata a Joram y
Ochôzías}\label{jehuxfa-mata-a-joram-y-ochuxf4zuxedas}}

\bibleverse{16} Así que Jehú montó en un carro y fue a Jezreel, pues
Joram yacía allí. Ocozías, rey de Judá, había bajado a ver a Joram.
\footnote{\textbf{9:16} 2Re 8,29} \bibleverse{17} El centinela estaba en
la torre de Jezreel, y al ver que llegaba la compañía de Jehú, dijo:
``Veo una compañía.'' Joram dijo: ``Toma un jinete y envía a recibirlos,
y que diga: ``¿Hay paz?''.

\bibleverse{18} Entonces uno fue a caballo a su encuentro y dijo: ``El
rey dice: ``¿Es la paz?'' Jehú dijo: ``¿Qué tienes que ver con la paz?
Ponte detrás de mí''. El vigilante dijo: ``El mensajero vino a ellos,
pero no vuelve''.

\bibleverse{19} Entonces envió a un segundo a caballo, que se acercó a
ellos y les dijo: ``El rey dice: ``¿Hay paz?''. Jehú respondió: ``¿Qué
tienes que ver con la paz? Ponte detrás de mí''.

\bibleverse{20} El vigilante dijo: ``Ha venido hacia ellos y no vuelve.
La conducción es como la de Jehú, hijo de Nimsí, pues conduce con
furia''.

\bibleverse{21} Joram dijo: ``¡Prepárate!'' Prepararon su carro.
Entonces salieron Joram, rey de Israel, y Ocozías, rey de Judá, cada uno
en su carro; y salieron al encuentro de Jehú, y lo encontraron en la
tierra de Nabot el jezreelita. \footnote{\textbf{9:21} 1Re 21,1}
\bibleverse{22} Cuando Joram vio a Jehú, le dijo: ``¿Hay paz, Jehú?''
Respondió: ``¿Qué paz, mientras abunden la prostitución de tu madre
Jezabel y sus brujerías?''

\bibleverse{23} Joram volvió las manos y huyó, y dijo a Ocozías: ``¡Esto
es traición, Ocozías!''

\bibleverse{24} Jehú tensó su arco con todas sus fuerzas, e hirió a
Joram entre sus brazos; la flecha le salió al corazón, y se hundió en su
carro. \bibleverse{25} Entonces Jehú dijo a Bidkar, su capitán:
``Recógelo y arrójalo en la parcela del campo de Nabot el jezreelita;
pues recuerda que cuando tú y yo cabalgábamos juntos tras su padre Ajab,
Yahvé le impuso esta carga: \footnote{\textbf{9:25} 1Re 21,19}
\bibleverse{26} `Ciertamente he visto ayer la sangre de Nabot y la
sangre de sus hijos', dice Yahvé; `y te pagaré en esta parcela', dice
Yahvé. Ahora, pues, tómalo y échalo en la parcela, según la palabra de
Yahvé''.

\bibleverse{27} Al ver esto, Ocozías, rey de Judá, huyó por el camino de
la casa del jardín. Jehú lo siguió, y dijo: ``¡Hiéranlo también en el
carro!'' Lo hirieron en la subida de Gur, que está junto a Ibleam. Huyó
a Meguido, y allí murió. \footnote{\textbf{9:27} 2Cró 22,7-9}
\bibleverse{28} Sus servidores lo llevaron en un carro a Jerusalén, y lo
enterraron en su tumba con sus padres en la ciudad de David. \footnote{\textbf{9:28}
  2Re 14,2; 2Re 23,30} \bibleverse{29} En el undécimo año de Joram, hijo
de Ajab, Ocozías comenzó a reinar sobre Judá.

\hypertarget{el-espantoso-final-de-jezabel}{%
\subsection{El espantoso final de
Jezabel}\label{el-espantoso-final-de-jezabel}}

\bibleverse{30} Cuando Jehú llegó a Jezreel, Jezabel se enteró, y se
pintó los ojos, se adornó la cabeza y se asomó a la ventana.
\bibleverse{31} Cuando Jehú entró por la puerta, ella dijo: ``¿Vienes en
paz, Zimri, asesino de tu señor?'' \footnote{\textbf{9:31} 1Re 16,10;
  1Re 16,18}

\bibleverse{32} Levantó el rostro hacia la ventana y dijo: ``¿Quién está
de mi lado? ¿Quién?'' Dos o tres eunucos le miraron.

\bibleverse{33} Él dijo: ``¡Tírala!'' Entonces la arrojaron al suelo, y
parte de su sangre fue rociada sobre el muro y sobre los caballos.
Entonces él la pisoteó. \bibleverse{34} Cuando entró, comió y bebió.
Luego dijo: ``Encargaos ahora de esta mujer maldita y enterradla, porque
es hija de un rey''.

\bibleverse{35} Fueron a enterrarla, pero no encontraron de ella más que
el cráneo, los pies y las palmas de las manos. \bibleverse{36}
Volvieron, pues, y le contaron. Dijo: ``Esta es la palabra de Yahvé, que
habló por medio de su siervo Elías el tisbita, diciendo: `Los perros
comerán la carne de Jezabel en la parcela de Jezreel, \footnote{\textbf{9:36}
  2Re 9,10; 1Re 21,23} \bibleverse{37} y el cuerpo de Jezebel será como
estiércol en la superficie del campo en la tierra de Jezreel, para que
no digan: ``Esta es Jezebel''\,''.

\hypertarget{jehuxfa-asesinuxf3-a-los-setenta-pruxedncipes-reales-y-exterminuxf3-a-toda-la-casa-de-acab}{%
\subsection{Jehú asesinó a los setenta príncipes reales y exterminó a
toda la casa de
Acab}\label{jehuxfa-asesinuxf3-a-los-setenta-pruxedncipes-reales-y-exterminuxf3-a-toda-la-casa-de-acab}}

\hypertarget{section-9}{%
\section{10}\label{section-9}}

\bibleverse{1} Ajab tenía setenta hijos en Samaria. Jehú escribió cartas
y las envió a Samaria, a los gobernantes de Jezreel, a los ancianos y a
los que criaban a los hijos de Acab, diciendo: \bibleverse{2} ``Ahora
bien, en cuanto te llegue esta carta, ya que los hijos de tu amo están
contigo, y tienes carros y caballos, una ciudad fortificada también, y
armaduras, \bibleverse{3} elige al mejor y más apto de los hijos de tu
amo, ponlo en el trono de su padre y pelea por la casa de tu amo.''

\bibleverse{4} Pero ellos tuvieron mucho miedo y dijeron: ``¡Mira que
los dos reyes no se han puesto en pie delante de él! ¿Cómo, pues, nos
pondremos en pie?'' \bibleverse{5} El que estaba a cargo de la casa, y
el que estaba a cargo de la ciudad, los ancianos también y los que
criaban a los niños, enviaron a decir a Jehú: ``Somos tus servidores y
haremos todo lo que nos pidas. No haremos rey a ningún hombre. Haz tú lo
que te parezca bien''.

\bibleverse{6} Entonces les escribió por segunda vez una carta en la que
les decía: ``Si estáis de mi parte, y si escucháis mi voz, tomad las
cabezas de los hombres que son hijos de vuestro amo y venid a mí a
Jezreel mañana a esta hora.'' Los hijos del rey, que eran setenta
personas, estaban con los grandes de la ciudad, quienes los hicieron
subir. \bibleverse{7} Cuando les llegó la carta, tomaron a los hijos del
rey y los mataron, siendo setenta personas, y pusieron sus cabezas en
canastas, y se las enviaron a Jezreel. \bibleverse{8} Vino un mensajero
y le dijo: ``Han traído las cabezas de los hijos del rey''. Dijo:
``Ponedlos en dos montones a la entrada de la puerta hasta la mañana''.
\bibleverse{9} Por la mañana, salió, se puso en pie y dijo a todo el
pueblo: ``Sois justos. He aquí que yo he conspirado contra mi amo y lo
he matado, pero ¿quién ha matado a todos estos? \bibleverse{10} Sepan
ahora que nada caerá a la tierra de la palabra de Yahvé, que Yahvé habló
sobre la casa de Ajab. Porque Yahvé ha hecho lo que habló por medio de
su siervo Elías''. \footnote{\textbf{10:10} 1Re 21,22}

\bibleverse{11} Así que Jehú hirió a todo lo que quedaba de la casa de
Acab en Jezreel, con todos sus grandes hombres, sus amigos familiares y
sus sacerdotes, hasta que no le dejó nadie.

\hypertarget{jehuxfa-mata-a-los-pruxedncipes-de-judea}{%
\subsection{Jehú mata a los príncipes de
Judea}\label{jehuxfa-mata-a-los-pruxedncipes-de-judea}}

\bibleverse{12} Se levantó y partió, y se dirigió a Samaria. Mientras
estaba en la casa de esquila de los pastores en el camino,
\bibleverse{13} Jehú se encontró con los hermanos de Ocozías, rey de
Judá, y les dijo: ``¿Quiénes son ustedes?'' Ellos respondieron: ``Somos
los hermanos de Ocozías. Bajamos a saludar a los hijos del rey y a los
hijos de la reina''. \footnote{\textbf{10:13} 2Cró 22,8}

\bibleverse{14} Él dijo: ``¡Tómenlos vivos!'' Los cogieron vivos y los
mataron en la fosa de la esquila, hasta cuarenta y dos hombres. No dejó
a ninguno de ellos.

\hypertarget{jehuxfa-lleva-al-recabita-jonadab-a-su-amistad}{%
\subsection{Jehú lleva al recabita Jonadab a su
amistad}\label{jehuxfa-lleva-al-recabita-jonadab-a-su-amistad}}

\bibleverse{15} Cuando partió de allí, encontró a Jonadab, hijo de
Recab, que venía a su encuentro. Lo saludó y le dijo: ``¿Está bien tu
corazón, como está mi corazón con el tuyo?''. Jehonadab respondió: ``Lo
es''. ``Si es así, dame la mano''. Le dio la mano y lo subió al carro.
\footnote{\textbf{10:15} Jer 35,6} \bibleverse{16} Le dijo: ``Acompáñame
y mira mi celo por Yahvé''. Así lo hicieron subir a su carro.
\bibleverse{17} Cuando llegó a Samaria, golpeó a todos los que le
quedaban a Ajab en Samaria, hasta que los destruyó, según la palabra de
Yavé que le habló a Elías. \footnote{\textbf{10:17} 1Re 21,21-22}

\hypertarget{jehuxfa-extermina-a-los-adoradores-de-baals-en-samaria}{%
\subsection{Jehú extermina a los adoradores de Baals en
Samaria}\label{jehuxfa-extermina-a-los-adoradores-de-baals-en-samaria}}

\bibleverse{18} Jehú reunió a todo el pueblo y les dijo: ``Acab sirvió
poco a Baal, pero Jehú le servirá mucho. \footnote{\textbf{10:18} 1Re
  16,31-33} \bibleverse{19} Ahora, pues, llama a todos los profetas de
Baal, a todos sus adoradores y a todos sus sacerdotes. Que no falte
ninguno, porque tengo un gran sacrificio a Baal. El que esté ausente, no
vivirá''. Pero Jehú actuó con engaño, con la intención de destruir a los
adoradores de Baal.

\bibleverse{20} Jehú dijo: ``¡Santificad una asamblea solemne para
Baal!'' Así lo proclamaron. \bibleverse{21} Jehú envió por todo Israel,
y todos los adoradores de Baal vinieron, de modo que no quedó ninguno
que no viniera. Entraron en la casa de Baal, y la casa de Baal se llenó
de un extremo a otro. \bibleverse{22} Le dijo al que guardaba el
guardarropa: ``¡Saca túnicas para todos los adoradores de Baal!'' Y les
sacó las túnicas. \bibleverse{23} Jehú fue con Jonadab, hijo de Recab, a
la casa de Baal. Entonces dijo a los adoradores de Baal: ``Busquen y
vean que ninguno de los siervos de Yavé está aquí con ustedes, sino sólo
los adoradores de Baal.'' \footnote{\textbf{10:23} 2Re 10,15}

\bibleverse{24} Entraron, pues, a ofrecer sacrificios y holocaustos. Y
Jehú había designado para sí ochenta hombres afuera, y dijo: ``Si alguno
de los hombres que traigo en sus manos se escapa, el que lo deje ir, su
vida será para él''. \footnote{\textbf{10:24} 1Re 20,39}

\bibleverse{25} En cuanto terminó de ofrecer el holocausto, Jehú dijo a
la guardia y a los capitanes: ``¡Entren y mátenlos! Que no escape
ninguno''. Así que los hirieron con el filo de la espada. La guardia y
los capitanes arrojaron los cadáveres y se dirigieron al santuario
interior de la casa de Baal. \footnote{\textbf{10:25} 1Re 18,40}
\bibleverse{26} Sacaron las columnas que había en el templo de Baal y
las quemaron. \footnote{\textbf{10:26} 2Re 11,18} \bibleverse{27}
Derribaron la columna de Baal, y derribaron la casa de Baal y la
convirtieron en una letrina, hasta el día de hoy. \footnote{\textbf{10:27}
  2Re 3,2}

\hypertarget{la-predicaciuxf3n-de-dios-a-jehuxfa-fracasos-de-jehuxfa-conclusiuxf3n-de-la-historia-de-jehuxfa}{%
\subsection{La predicación de Dios a Jehú; Fracasos de Jehú; Conclusión
de la historia de
Jehú}\label{la-predicaciuxf3n-de-dios-a-jehuxfa-fracasos-de-jehuxfa-conclusiuxf3n-de-la-historia-de-jehuxfa}}

\bibleverse{28} Así destruyó Jehú a Baal de Israel.

\bibleverse{29} Sin embargo, Jehú no se apartó de los pecados de
Jeroboam hijo de Nabat, con los que hizo pecar a Israel: los becerros de
oro que estaban en Betel y que estaban en Dan. \footnote{\textbf{10:29}
  1Re 12,26-33} \bibleverse{30} Yahvé dijo a Jehú: ``Porque has hecho
bien en ejecutar lo que es justo a mis ojos, y has hecho a la casa de
Ajab según todo lo que estaba en mi corazón, tus descendientes se
sentarán en el trono de Israel hasta la cuarta generación.'' \footnote{\textbf{10:30}
  2Re 15,12}

\bibleverse{31} Pero Jehú no se cuidó de andar en la ley de Yavé, el
Dios de Israel, con todo su corazón. No se apartó de los pecados de
Jeroboam, con los que hizo pecar a Israel.

\bibleverse{32} En aquellos días Yahvé comenzó a cortar partes de
Israel; y Hazael los hirió en todos los límites de Israel \footnote{\textbf{10:32}
  2Re 8,12} \bibleverse{33} desde el Jordán hacia el oriente, toda la
tierra de Galaad, los gaditas, los rubenitas y los manasitas, desde
Aroer, que está junto al valle de Arnón, hasta Galaad y Basán.
\bibleverse{34} El resto de los hechos de Jehú, y todo lo que hizo, y
todo su poderío, ¿no están escritos en el libro de las crónicas de los
reyes de Israel? \bibleverse{35} Jehú durmió con sus padres, y lo
enterraron en Samaria. Su hijo Joacaz reinó en su lugar. \footnote{\textbf{10:35}
  2Re 13,1} \bibleverse{36} El tiempo que Jehú reinó sobre Israel en
Samaria fue de veintiocho años.

\hypertarget{el-robo-y-el-asesinato-de-ataluxeda-rescate-de-jouxe1s}{%
\subsection{El robo y el asesinato de Atalía; Rescate de
Joás}\label{el-robo-y-el-asesinato-de-ataluxeda-rescate-de-jouxe1s}}

\hypertarget{section-10}{%
\section{11}\label{section-10}}

\bibleverse{1} Cuando Atalía, madre de Ocozías, vio que su hijo había
muerto, se levantó y destruyó toda la descendencia real. \footnote{\textbf{11:1}
  2Re 8,26; 2Re 9,27} \bibleverse{2} Pero Josheba, hija del rey Joram,
hermana de Ocozías, tomó a Joás, hijo de Ocozías, y lo robó de entre los
hijos del rey que habían sido asesinados, a él y a su nodriza, y los
puso en la alcoba; y lo escondieron de Atalía, para que no lo mataran.
\bibleverse{3} Estuvo con ella escondido en la casa de Yahvé seis años,
mientras Atalía reinaba sobre el país.

\hypertarget{la-conspiraciuxf3n-de-joiada}{%
\subsection{La conspiración de
Joiada}\label{la-conspiraciuxf3n-de-joiada}}

\bibleverse{4} En el séptimo año, Joiada envió a buscar a los capitanes
de centenares de caritas y de la guardia, y los trajo a él a la casa de
Yavé; e hizo con ellos un pacto, y les mostró al hijo del rey.
\bibleverse{5} Les ordenó, diciendo: ``Esto es lo que debéis hacer: un
tercio de vosotros, los que entréis en sábado, seréis guardianes de la
guardia de la casa del rey; \bibleverse{6} un tercio de vosotros estará
en la puerta Sur, y un tercio de vosotros en la puerta detrás de la
guardia. Así vigilaréis la casa, y seréis una barrera. \bibleverse{7}
Los dos grupos de ustedes, todos los que salen el sábado, mantendrán la
guardia de la casa de Yahvé alrededor del rey. \bibleverse{8} Rodead al
rey, cada uno con sus armas en la mano; y el que se acerque a las filas,
que lo maten. Estad con el rey cuando salga y cuando entre''.

\bibleverse{9} Los capitanes de centenas hicieron todo lo que ordenó el
sacerdote Joiada, y cada uno tomó a sus hombres, los que debían entrar
en sábado con los que debían salir en sábado, y vinieron al sacerdote
Joiada. \bibleverse{10} El sacerdote entregó a los capitanes de más de
cien años las lanzas y los escudos que habían sido del rey David y que
estaban en la casa de Yahvé. \footnote{\textbf{11:10} 2Sam 8,7}
\bibleverse{11} La guardia se puso en pie, cada uno con sus armas en la
mano, desde el lado derecho de la casa hasta el lado izquierdo, junto al
altar y la casa, alrededor del rey. \bibleverse{12} Entonces sacó al
hijo del rey, le puso la corona y le dio la alianza; lo proclamaron rey
y lo ungieron, y aplaudieron y dijeron: ``¡Viva el rey!'' \footnote{\textbf{11:12}
  Deut 17,18-19}

\hypertarget{captura-y-asesinato-de-athaluxeda}{%
\subsection{Captura y asesinato de
Athalía}\label{captura-y-asesinato-de-athaluxeda}}

\bibleverse{13} Cuando Atalía oyó el ruido de la guardia y del pueblo,
se acercó al pueblo a la casa de Yahvé; \bibleverse{14} y miró, y he
aquí que el rey estaba de pie junto a la columna, como era la tradición,
con los capitanes y las trompetas junto al rey; y todo el pueblo del
país se alegró y tocó las trompetas. Entonces Atalía se rasgó las
vestiduras y gritó: ``¡Traición! Traición!''

\bibleverse{15} El sacerdote Joiada ordenó a los capitanes de centenas
que estaban al mando del ejército y les dijo: ``Sáquenla entre las
filas. Maten a espada a cualquiera que la siga''. Porque el sacerdote
dijo: ``No dejen que la maten en la casa de Yavé''. \bibleverse{16} Así
que la apresaron, y se fue por el camino de la entrada de los caballos a
la casa del rey, y allí la mataron. \footnote{\textbf{11:16} Neh 3,28}

\hypertarget{medidas-de-joiada-para-la-gloria-de-dios-coronaciuxf3n-de-jouxe1s}{%
\subsection{Medidas de Joiada para la gloria de Dios; Coronación de
Joás}\label{medidas-de-joiada-para-la-gloria-de-dios-coronaciuxf3n-de-jouxe1s}}

\bibleverse{17} Joiada hizo un pacto entre Yahvé y el rey y el pueblo,
para que fueran pueblo de Yahvé; también entre el rey y el pueblo.
\bibleverse{18} Todo el pueblo del país fue a la casa de Baal y la
derribó. Rompieron a fondo sus altares y sus imágenes, y mataron a
Matán, el sacerdote de Baal, delante de los altares. El sacerdote nombró
oficiales sobre la casa de Yahvé. \footnote{\textbf{11:18} 2Re 10,26-27;
  Jue 6,25} \bibleverse{19} Tomó a los capitanes de centenas, a los
caritas, a la guardia y a todo el pueblo del país, y bajaron al rey de
la casa de Yavé, y llegaron por el camino de la puerta de la guardia a
la casa del rey. Él se sentó en el trono de los reyes. \bibleverse{20}
Entonces todo el pueblo del país se alegró, y la ciudad se tranquilizó.
Habían matado a Atalía con la espada en la casa del rey.

\bibleverse{21} Joás tenía siete años cuando comenzó a reinar.

\hypertarget{jouxe1s-rey-de-juduxe1}{%
\subsection{Joás rey de Judá}\label{jouxe1s-rey-de-juduxe1}}

\hypertarget{section-11}{%
\section{12}\label{section-11}}

\bibleverse{1} Joás comenzó a reinar en el séptimo año de Jehú, y reinó
cuarenta años en Jerusalén. Su madre se llamaba Sibías de Beerseba.
\bibleverse{2} Joás hizo lo que era justo a los ojos de Yavé durante
todos sus días, en lo que le instruyó el sacerdote Joiada.
\bibleverse{3} Sin embargo, los lugares altos no fueron quitados. El
pueblo seguía sacrificando y quemando incienso en los lugares altos.
\footnote{\textbf{12:3} 2Re 14,4; 1Re 22,43}

\bibleverse{4} Yahvé dijo a los sacerdotes: ``Todo el dinero de las
cosas sagradas que se traiga a la casa de Yavé, en dinero corriente, el
dinero del pueblo por el que se evalúa cada uno, y todo el dinero que se
le ocurra a cualquier hombre traer a la casa de Yavé,

\hypertarget{ordenanza-del-rey-sobre-la-reparaciuxf3n-del-templo-y-sobre-la-administraciuxf3n-y-uso-del-dinero-del-templo}{%
\subsection{Ordenanza del rey sobre la reparación del templo y sobre la
administración y uso del dinero del
templo}\label{ordenanza-del-rey-sobre-la-reparaciuxf3n-del-templo-y-sobre-la-administraciuxf3n-y-uso-del-dinero-del-templo}}

\bibleverse{5} que los sacerdotes se lo lleven, cada uno de su donante;
y ellos repararán el daño de la casa, dondequiera que se encuentre
cualquier daño.''

\bibleverse{6} Pero sucedió que en el año veintitrés del rey Joás los
sacerdotes no habían reparado los daños de la casa. \bibleverse{7}
Entonces el rey Joás llamó al sacerdote Joiada y a los demás sacerdotes
y les dijo: ``¿Por qué no reparáis los daños de la casa? Ahora, pues, no
toméis más dinero de vuestros tesoros, sino entregadlo para reparar los
daños de la casa.''

\bibleverse{8} Los sacerdotes consintieron en no tomar más dinero del
pueblo y en no reparar el daño causado a la casa. \bibleverse{9} Pero el
sacerdote Joiada tomó un cofre, le hizo un agujero en la tapa y lo puso
al lado del altar, a la derecha, como se entra en la casa de Yavé; y los
sacerdotes que guardaban el umbral pusieron en él todo el dinero que se
traía a la casa de Yavé. \bibleverse{10} Cuando vieron que había mucho
dinero en el cofre, subieron el escriba del rey y el sumo sacerdote, y
lo pusieron en bolsas y contaron el dinero que se encontraba en la casa
de Yavé. \bibleverse{11} Dieron el dinero que se había pesado en manos
de los que hacían la obra, que tenían la supervisión de la casa de Yavé;
y lo pagaron a los carpinteros y a los constructores que trabajaban en
la casa de Yavé, \bibleverse{12} y a los albañiles y a los canteros, y
para comprar madera y piedra cortada para reparar los daños de la casa
de Yavé, y para todo lo que se hizo en la casa para repararla.
\bibleverse{13} Pero no se hicieron para la casa de Yavé copas de plata,
tazas, jofainas, trompetas, ningún vaso de oro ni de plata, del dinero
que se traía a la casa de Yavé; \bibleverse{14} porque eso se lo daban a
los que hacían la obra, y con eso reparaban la casa de Yavé.
\bibleverse{15} Además, no pidieron cuentas a los hombres en cuya mano
entregaron el dinero para que lo entregaran a los que hacían la obra,
pues actuaron con fidelidad. \footnote{\textbf{12:15} 2Re 22,7}
\bibleverse{16} El dinero para las ofrendas por la culpa y el dinero
para las ofrendas por el pecado no entraba en la casa de Yavé. Era de
los sacerdotes.

\hypertarget{jouxe1s-salvuxf3-a-jerusaluxe9n-del-ataque-de-hazael-pagando-dinero-su-asesinato}{%
\subsection{Joás salvó a Jerusalén del ataque de Hazael pagando dinero;
su
asesinato}\label{jouxe1s-salvuxf3-a-jerusaluxe9n-del-ataque-de-hazael-pagando-dinero-su-asesinato}}

\bibleverse{17} Entonces subió Hazael, rey de Siria, y combatió contra
Gat, y la tomó; y Hazael se dispuso a subir a Jerusalén. \footnote{\textbf{12:17}
  2Re 10,32} \bibleverse{18} Joás, rey de Judá, tomó todas las cosas
sagradas que Josafat, Joram y Ocozías, sus padres, reyes de Judá, habían
dedicado, y sus propias cosas sagradas, y todo el oro que se encontraba
en los tesoros de la casa de Yahvé y de la casa real, y lo envió a
Hazael, rey de Siria; y se fue de Jerusalén. \footnote{\textbf{12:18}
  1Re 15,18}

\bibleverse{19} El resto de los hechos de Joás, y todo lo que hizo, ¿no
está escrito en el libro de las crónicas de los reyes de Judá?
\bibleverse{20} Sus siervos se levantaron e hicieron una conspiración, y
golpearon a Joás en la casa de Millo, en el camino que desciende a
Silla. \footnote{\textbf{12:20} 2Re 14,5} \bibleverse{21} Pues Jozacar,
hijo de Simeat, y Jozabad, hijo de Shomer, sus servidores, lo hirieron,
y murió; y lo enterraron con sus padres en la ciudad de David; y su hijo
Amasías reinó en su lugar. \footnote{\textbf{12:21} 2Re 14,1}

\hypertarget{joachuxe2z-rey-de-israel}{%
\subsection{Joachâz rey de Israel}\label{joachuxe2z-rey-de-israel}}

\hypertarget{section-12}{%
\section{13}\label{section-12}}

\bibleverse{1} En el año veintitrés de Joás hijo de Ocozías, rey de
Judá, Joacaz hijo de Jehú comenzó a reinar sobre Israel en Samaria
durante diecisiete años. \footnote{\textbf{13:1} 2Re 10,35}
\bibleverse{2} Hizo lo que era malo a los ojos de Yavé, y siguió los
pecados de Jeroboam hijo de Nabat, con los que hizo pecar a Israel. No
se apartó de él. \footnote{\textbf{13:2} 1Re 12,26-33} \bibleverse{3} La
ira de Yavé se encendió contra Israel y lo entregó continuamente en
manos de Hazael, rey de Siria, y en manos de Benhadad, hijo de Hazael.
\footnote{\textbf{13:3} 2Re 10,32} \bibleverse{4} Joacaz le rogó a Yavé,
y Yavé lo escuchó, porque vio la opresión de Israel, cómo lo oprimía el
rey de Siria. \bibleverse{5} (Yahvé dio a Israel un salvador, de modo
que salieron de la mano de los sirios; y los hijos de Israel vivieron en
sus tiendas como antes. \footnote{\textbf{13:5} 2Re 14,27}
\bibleverse{6} Sin embargo, no se apartaron de los pecados de la casa de
Jeroboam, con los que él hizo pecar a Israel, sino que anduvieron en
ellos; y también la Asera permaneció en Samaria). \footnote{\textbf{13:6}
  1Re 16,33} \bibleverse{7} Porque no dejó a Joacaz del pueblo más que
cincuenta jinetes, diez carros y diez mil hombres de a pie, porque el
rey de Siria los destruyó y los hizo como el polvo en la trilla.
\bibleverse{8} Los demás hechos de Joacaz, y todo lo que hizo, y su
poderío, ¿no están escritos en el libro de las crónicas de los reyes de
Israel? \bibleverse{9} Joacaz durmió con sus padres, y lo enterraron en
Samaria; y su hijo Joás reinó en su lugar.

\hypertarget{joas-kuxf6nig-von-israel}{%
\subsection{Joas König von Israel}\label{joas-kuxf6nig-von-israel}}

\bibleverse{10} En el año treinta y siete de Joás, rey de Judá, Joás,
hijo de Joacaz, comenzó a reinar sobre Israel en Samaria durante
dieciséis años. \bibleverse{11} Hizo lo que era malo a los ojos de Yavé.
No se apartó de todos los pecados de Jeroboam hijo de Nabat, con los que
hizo pecar a Israel, sino que anduvo en ellos. \footnote{\textbf{13:11}
  2Re 13,2} \bibleverse{12} El resto de los hechos de Joás, y todo lo
que hizo, y su poderío con el que luchó contra Amasías, rey de Judá, ¿no
están escritos en el libro de las crónicas de los reyes de Israel?
\footnote{\textbf{13:12} 2Re 14,8-16} \bibleverse{13} Joás durmió con
sus padres, y Jeroboam se sentó en su trono. Joás fue enterrado en
Samaria con los reyes de Israel. \footnote{\textbf{13:13} 2Re 14,23}

\hypertarget{jouxe1s-con-el-enfermo-eliseo-la-muerte-de-eliseo}{%
\subsection{Joás con el enfermo Eliseo; La muerte de
Eliseo}\label{jouxe1s-con-el-enfermo-eliseo-la-muerte-de-eliseo}}

\bibleverse{14} Eliseo enfermó de la enfermedad de la que murió, y Joás,
el rey de Israel, bajó a verlo y lloró por él, diciendo: ``¡Padre mío,
padre mío, los carros de Israel y su caballería!'' \footnote{\textbf{13:14}
  2Re 2,12}

\bibleverse{15} Eliseo le dijo: ``Toma el arco y las flechas''; y él
tomó el arco y las flechas para sí. \bibleverse{16} Le dijo al rey de
Israel: ``Pon tu mano sobre el arco''; y él puso su mano sobre él.
Eliseo puso sus manos sobre las manos del rey. \bibleverse{17} Dijo:
``Abre la ventana hacia el este'', y él la abrió. Entonces Eliseo dijo:
``Dispara'', y disparó. Dijo: ``La flecha de la victoria de Yahvé, la
flecha de la victoria sobre Siria; porque herirás a los sirios en Afec
hasta consumirlos''.

\bibleverse{18} Dijo: ``Toma las flechas'', y él las tomó. Dijo al rey
de Israel: ``Golpea el suelo'', y él golpeó tres veces, y se detuvo.
\bibleverse{19} El hombre de Dios se enojó con él y dijo: ``Deberías
haber golpeado cinco o seis veces. Entonces habrías golpeado a Siria
hasta consumirla, pero ahora sólo golpearás a Siria tres veces''.

\hypertarget{elisa-sigue-trabajando-milagrosamente-en-su-tumba}{%
\subsection{Elisa sigue trabajando milagrosamente en su
tumba}\label{elisa-sigue-trabajando-milagrosamente-en-su-tumba}}

\bibleverse{20} Eliseo murió y lo enterraron. Las bandas de moabitas
invadieron el país al llegar el año. \bibleverse{21} Mientras enterraban
a un hombre, vieron una banda de asaltantes, y arrojaron al hombre en la
tumba de Eliseo. En cuanto el hombre tocó los huesos de Eliseo, revivió
y se puso de pie.

\hypertarget{las-tres-victorias-de-jouxe1s-sobre-los-sirios}{%
\subsection{Las tres victorias de Joás sobre los
sirios}\label{las-tres-victorias-de-jouxe1s-sobre-los-sirios}}

\bibleverse{22} Hazael, rey de Siria, oprimió a Israel durante todos los
días de Joacaz. \bibleverse{23} Pero Yahvé tuvo misericordia de ellos y
se compadeció de ellos, y los favoreció a causa de su pacto con Abraham,
Isaac y Jacob, y no los destruyó y no los echó de su presencia todavía.
\footnote{\textbf{13:23} Lev 26,42}

\bibleverse{24} Murió Hazael, rey de Siria, y reinó en su lugar
Benhadad, su hijo. \bibleverse{25} Joás, hijo de Joacaz, volvió a tomar
de la mano de Benhadad, hijo de Hazael, las ciudades que había tomado de
la mano de su padre Joacaz mediante la guerra. Joás lo golpeó tres
veces, y recuperó las ciudades de Israel. \footnote{\textbf{13:25} 2Re
  13,19}

\hypertarget{amasuxedas-rey-de-juduxe1-buen-comienzo-para-el-gobierno}{%
\subsection{Amasías rey de Judá; Buen comienzo para el
gobierno}\label{amasuxedas-rey-de-juduxe1-buen-comienzo-para-el-gobierno}}

\hypertarget{section-13}{%
\section{14}\label{section-13}}

\bibleverse{1} En el segundo año de Joás, hijo de Joacaz, rey de Israel,
comenzó a reinar Amasías, hijo de Joás, rey de Judá. \footnote{\textbf{14:1}
  2Re 12,21} \bibleverse{2} Tenía veinticinco años cuando comenzó a
reinar, y reinó veintinueve años en Jerusalén. Su madre se llamaba
Joaquín de Jerusalén. \bibleverse{3} Hizo lo que era justo a los ojos de
Yavé, pero no como David, su padre. Hizo conforme a todo lo que había
hecho su padre Joás. \footnote{\textbf{14:3} 2Re 12,2-3} \bibleverse{4}
Sin embargo, los lugares altos no fueron quitados. El pueblo seguía
sacrificando y quemando incienso en los lugares altos. \footnote{\textbf{14:4}
  2Re 15,4} \bibleverse{5} Tan pronto como el reino fue establecido en
su mano, mató a sus siervos que habían matado al rey su padre,
\footnote{\textbf{14:5} 2Re 12,20-21} \bibleverse{6} pero a los hijos de
los asesinos no los mató, según lo que está escrito en el libro de la
ley de Moisés, como lo ordenó Yavé, diciendo: ``Los padres no morirán
por los hijos, ni los hijos morirán por los padres; sino que cada uno
morirá por su propio pecado.'' \footnote{\textbf{14:6} Deut 24,16}

\bibleverse{7} Mató a diez mil edomitas en el Valle de la Sal, y tomó
Sela por la guerra, y llamó su nombre Joktheel, hasta el día de hoy.

\hypertarget{la-desafortunada-guerra-de-amasuxedas-con-jouxe1s-de-israel}{%
\subsection{La desafortunada guerra de Amasías con Joás de
Israel}\label{la-desafortunada-guerra-de-amasuxedas-con-jouxe1s-de-israel}}

\bibleverse{8} Entonces Amasías envió mensajeros a Joás, hijo de Joacaz,
hijo de Jehú, rey de Israel, diciéndole: ``Ven, mirémonos a la cara''.

\bibleverse{9} Joás, rey de Israel, envió a Amasías, rey de Judá,
diciendo: ``El cardo que estaba en el Líbano envió al cedro que estaba
en el Líbano, diciendo: `Dale tu hija a mi hijo como esposa'. Entonces
pasó un animal salvaje que estaba en el Líbano y pisoteó el cardo.
\footnote{\textbf{14:9} Jue 9,14} \bibleverse{10} Ciertamente has
golpeado a Edom, y tu corazón te ha levantado. Disfruta de su gloria, y
quédate en casa; pues, ¿por qué has de entrometerte para tu mal, para
que caigas tú, y Judá contigo?''

\bibleverse{11} Pero Amasías no quiso escuchar. Entonces subió Joás, rey
de Israel, y él y Amasías, rey de Judá, se miraron a la cara en Bet
Semes, que es de Judá. \bibleverse{12} Judá fue derrotado por Israel, y
cada uno huyó a su tienda. \bibleverse{13} Joás, rey de Israel, apresó a
Amasías, rey de Judá, hijo de Joás, hijo de Ocozías, en Bet Semes, y
llegó a Jerusalén; luego derribó el muro de Jerusalén desde la puerta de
Efraín hasta la puerta de la esquina, cuatrocientos codos.
\bibleverse{14} Tomó todo el oro y la plata y todos los utensilios que
se encontraban en la casa de Yavé y en los tesoros de la casa del rey,
también los rehenes, y regresó a Samaria.

\hypertarget{palabras-de-clausura-sobre-jouxe1s-de-israel}{%
\subsection{Palabras de clausura sobre Joás de
Israel}\label{palabras-de-clausura-sobre-jouxe1s-de-israel}}

\bibleverse{15} Los demás hechos de Joás, y su poderío, y cómo luchó con
Amasías, rey de Judá, ¿no están escritos en el libro de las crónicas de
los reyes de Israel? \bibleverse{16} Joás durmió con sus padres y fue
sepultado en Samaria con los reyes de Israel; y su hijo Jeroboam reinó
en su lugar. \footnote{\textbf{14:16} 2Re 13,13}

\hypertarget{palabras-finales-sobre-amasuxedas-de-juduxe1-su-asesinato}{%
\subsection{Palabras finales sobre Amasías de Judá; su
asesinato}\label{palabras-finales-sobre-amasuxedas-de-juduxe1-su-asesinato}}

\bibleverse{17} Amasías hijo de Joás, rey de Judá, vivió después de la
muerte de Joás hijo de Joacaz, rey de Israel, quince años.
\bibleverse{18} Los demás hechos de Amasías, ¿no están escritos en el
libro de las crónicas de los reyes de Judá? \bibleverse{19} Hicieron una
conspiración contra él en Jerusalén, y él huyó a Laquis; pero enviaron
tras él a Laquis y lo mataron allí. \footnote{\textbf{14:19} 2Re 12,20}
\bibleverse{20} Lo llevaron a caballo, y fue enterrado en Jerusalén con
sus padres, en la ciudad de David. \footnote{\textbf{14:20} 2Re 9,28}

\hypertarget{azaruxedas-asume-el-cargo}{%
\subsection{Azarías asume el cargo}\label{azaruxedas-asume-el-cargo}}

\bibleverse{21} Todo el pueblo de Judá tomó a Azarías, que tenía
dieciséis años, y lo nombró rey en lugar de su padre Amasías.
\footnote{\textbf{14:21} 2Re 15,1-2} \bibleverse{22} Él edificó Elat y
se la devolvió a Judá. Después de eso el rey durmió con sus padres.
\footnote{\textbf{14:22} 2Re 16,6}

\hypertarget{jeroboam-ii-rey-de-israel}{%
\subsection{Jeroboam II Rey de Israel}\label{jeroboam-ii-rey-de-israel}}

\bibleverse{23} En el año quince de Amasías, hijo de Joás, rey de Judá,
Jeroboam, hijo de Joás, rey de Israel, comenzó a reinar en Samaria
durante cuarenta y un años. \footnote{\textbf{14:23} 2Re 14,16; Os 1,1;
  Am 1,1} \bibleverse{24} Hizo lo que era malo a los ojos de Yavé. No se
apartó de todos los pecados de Jeroboam hijo de Nabat, con los que hizo
pecar a Israel. \footnote{\textbf{14:24} 1Re 12,26-33} \bibleverse{25}
Restauró la frontera de Israel desde la entrada de Hamat hasta el mar
del Arabá, según la palabra de Yavé, el Dios de Israel, que habló por
medio de su siervo Jonás, hijo de Amittai, el profeta, que era de Gat
Hefer. \footnote{\textbf{14:25} Jon 1,1} \bibleverse{26} Porque Yahvé
vio la aflicción de Israel, que era muy amarga para todos, esclavos y
libres; y no había quien ayudara a Israel. \footnote{\textbf{14:26} Deut
  32,36} \bibleverse{27} El Señor no dijo que borraría el nombre de
Israel de debajo del cielo, sino que lo salvó por la mano de Jeroboam
hijo de Joás. \footnote{\textbf{14:27} 2Re 13,5} \bibleverse{28} El
resto de los hechos de Jeroboam, y todo lo que hizo, y su poderío, cómo
luchó y cómo recuperó para Israel Damasco y Hamat, que habían
pertenecido a Judá, ¿no están escritos en el libro de las crónicas de
los reyes de Israel? \bibleverse{29} Jeroboam durmió con sus padres, con
los reyes de Israel, y su hijo Zacarías reinó en su lugar. \footnote{\textbf{14:29}
  2Re 15,8}

\hypertarget{azaruxedas-rey-de-juduxe1}{%
\subsection{Azarías rey de Judá}\label{azaruxedas-rey-de-juduxe1}}

\hypertarget{section-14}{%
\section{15}\label{section-14}}

\bibleverse{1} En el año veintisiete de Jeroboam, rey de Israel, comenzó
a reinar Azarías hijo de Amasías, rey de Judá. \footnote{\textbf{15:1}
  2Re 14,21} \bibleverse{2} Tenía dieciséis años cuando comenzó a
reinar, y reinó cincuenta y dos años en Jerusalén. Su madre se llamaba
Jecolías, de Jerusalén. \bibleverse{3} Hizo lo que era justo a los ojos
de Yavé, conforme a todo lo que había hecho su padre Amasías.
\bibleverse{4} Sin embargo, los lugares altos no fueron quitados. El
pueblo seguía sacrificando y quemando incienso en los lugares altos.
\footnote{\textbf{15:4} 2Re 14,3-4} \bibleverse{5} El Señor hirió al
rey, de modo que fue leproso hasta el día de su muerte, y vivió en una
casa separada. Jotam, el hijo del rey, estaba al frente de la casa,
juzgando al pueblo del país. \footnote{\textbf{15:5} Lev 13,46}
\bibleverse{6} El resto de los hechos de Azarías y todo lo que hizo, ¿no
está escrito en el libro de las crónicas de los reyes de Judá?
\bibleverse{7} Azarías durmió con sus padres, y lo enterraron con sus
padres en la ciudad de David; y su hijo Jotam reinó en su lugar.
\footnote{\textbf{15:7} 2Re 15,32}

\hypertarget{zacaruxedas-rey-de-israel}{%
\subsection{Zacarías rey de Israel}\label{zacaruxedas-rey-de-israel}}

\bibleverse{8} En el año treinta y ocho de Azarías, rey de Judá,
Zacarías, hijo de Jeroboam, reinó sobre Israel en Samaria durante seis
meses. \footnote{\textbf{15:8} 2Re 14,29} \bibleverse{9} Hizo lo que era
malo a los ojos de Yavé, como habían hecho sus padres. No se apartó de
los pecados de Jeroboam hijo de Nabat, con los que hizo pecar a Israel.
\footnote{\textbf{15:9} 1Re 12,26-33} \bibleverse{10} Salum hijo de
Jabes conspiró contra él, lo golpeó ante el pueblo y lo mató, y reinó en
su lugar. \footnote{\textbf{15:10} 2Re 15,14; Am 7,9} \bibleverse{11}
Los demás hechos de Zacarías están escritos en el libro de las crónicas
de los reyes de Israel. \bibleverse{12} Esta fue la palabra de Yahvé que
habló a Jehú, diciendo: ``Tus hijos hasta la cuarta generación se
sentarán en el trono de Israel.'' Así sucedió. \footnote{\textbf{15:12}
  2Re 10,30}

\hypertarget{sallum-kuxf6nig-von-israel}{%
\subsection{Sallum König von Israel}\label{sallum-kuxf6nig-von-israel}}

\bibleverse{13} Salum hijo de Jabes comenzó a reinar en el año treinta y
nueve de Uzías, rey de Judá, y reinó durante un mes en Samaria.
\bibleverse{14} Menahem hijo de Gadi subió de Tirsa, llegó a Samaria,
hirió a Salum hijo de Jabes en Samaria, lo mató y reinó en su lugar.
\footnote{\textbf{15:14} 1Re 16,17} \bibleverse{15} El resto de los
hechos de Salum y la conspiración que hizo, están escritos en el libro
de las crónicas de los reyes de Israel.

\bibleverse{16} Entonces Menahem atacó a Tiphsah y a todos los que
estaban en ella y en sus zonas fronterizas, desde Tirsa. La atacó porque
no le abrieron sus puertas, y despedazó a todas sus mujeres embarazadas.

\hypertarget{menahem-rey-de-israel}{%
\subsection{Menahem Rey de Israel}\label{menahem-rey-de-israel}}

\bibleverse{17} En el año treinta y nueve de Azarías, rey de Judá,
Menahem, hijo de Gadi, comenzó a reinar sobre Israel durante diez años
en Samaria. \bibleverse{18} Hizo lo que era malo a los ojos de Yavé. No
se apartó en todos sus días de los pecados de Jeroboam hijo de Nabat,
con los que hizo pecar a Israel. \footnote{\textbf{15:18} 2Re 15,9}
\bibleverse{19} Pul, el rey de Asiria, vino contra el país, y Menahem le
dio a Pul mil talentos\footnote{\textbf{15:19} Un talento equivale a
  unos 30 kilogramos o 66 libras, por lo que 1000 talentos son unas 30
  toneladas métricas} de plata, para que su mano estuviera con él para
confirmar el reino en su mano. \bibleverse{20} Menahem exigió el dinero
a Israel, a todos los hombres poderosos y ricos, a cada uno cincuenta
siclos\footnote{\textbf{15:20} Un siclo equivale a unos 10 gramos o a
  unas 0,35 onzas, por lo que 50 siclos eran unos 0,5 kilogramos o 1,1
  libras.} de plata, para dárselos al rey de Asiria. Y el rey de Asiria
se volvió, y no se quedó allí en la tierra. \footnote{\textbf{15:20} 2Re
  23,35} \bibleverse{21} El resto de los hechos de Menahem y todo lo que
hizo, ¿no está escrito en el libro de las crónicas de los reyes de
Israel? \bibleverse{22} Menahem durmió con sus padres, y su hijo Pekaía
reinó en su lugar.

\hypertarget{pekaja-rey-de-israel}{%
\subsection{Pekaja, rey de Israel}\label{pekaja-rey-de-israel}}

\bibleverse{23} En el año cincuenta de Azarías, rey de Judá, Pekaías,
hijo de Menahem, comenzó a reinar sobre Israel en Samaria durante dos
años. \bibleverse{24} Hizo lo que era malo a los ojos de Yavé. No se
apartó de los pecados de Jeroboam hijo de Nabat, con los que hizo pecar
a Israel. \footnote{\textbf{15:24} 2Re 15,9} \bibleverse{25} Peka, hijo
de Remalías, su capitán, conspiró contra él y lo atacó en Samaria, en la
fortaleza de la casa del rey, con Argob y Arieh; y con él había
cincuenta hombres de los galaaditas. Lo mató y reinó en su lugar.
\footnote{\textbf{15:25} 2Re 15,10; 2Re 15,14; 2Re 15,30}
\bibleverse{26} El resto de los hechos de Pekahiah, y todo lo que hizo,
están escritos en el libro de las crónicas de los reyes de Israel.

\hypertarget{peka-rey-de-israel}{%
\subsection{Peka, rey de Israel}\label{peka-rey-de-israel}}

\bibleverse{27} En el año cincuenta y dos de Azarías, rey de Judá, Peka,
hijo de Remalías, comenzó a reinar sobre Israel en Samaria durante
veinte años. \bibleverse{28} Hizo lo que era malo a los ojos de Yavé. No
se apartó de los pecados de Jeroboam hijo de Nabat, con los que hizo
pecar a Israel. \footnote{\textbf{15:28} 2Re 15,9} \bibleverse{29} En
los días de Peka, rey de Israel, vino Tiglat Pileser, rey de Asiria, y
tomó Ijón, Abel Bet Maaca, Janoa, Cedes, Hazor, Galaad y Galilea, toda
la tierra de Neftalí; y los llevó cautivos a Asiria. \footnote{\textbf{15:29}
  1Cró 5,26} \bibleverse{30} Oseas hijo de Ela conspiró contra Peka hijo
de Remalías, lo atacó, lo mató y reinó en su lugar, en el año veinte de
Jotam hijo de Uzías. \footnote{\textbf{15:30} 2Re 17,1; 2Re 15,25}
\bibleverse{31} El resto de los hechos de Peka, y todo lo que hizo, está
escrito en el libro de las crónicas de los reyes de Israel.

\hypertarget{jotam-rey-de-juduxe1}{%
\subsection{Jotam rey de Judá}\label{jotam-rey-de-juduxe1}}

\bibleverse{32} En el segundo año de Peka, hijo de Remalías, rey de
Israel, comenzó a reinar Jotam, hijo de Uzías, rey de Judá. \footnote{\textbf{15:32}
  2Re 15,5; 2Re 15,7; 2Cró 27,1} \bibleverse{33} Tenía veinticinco años
cuando comenzó a reinar, y reinó dieciséis años en Jerusalén. Su madre
se llamaba Jerusha, hija de Sadoc. \bibleverse{34} Hizo lo que era justo
a los ojos del Señor. Hizo conforme a todo lo que había hecho su padre
Uzías. \footnote{\textbf{15:34} 2Re 15,3-4} \bibleverse{35} Sin embargo,
los lugares altos no fueron quitados. El pueblo seguía sacrificando y
quemando incienso en los lugares altos. Él construyó la puerta superior
de la casa de Yavé. \bibleverse{36} El resto de los actos de Jotam y
todo lo que hizo, ¿no está escrito en el libro de las crónicas de los
reyes de Judá? \bibleverse{37} En aquellos días, el Señor comenzó a
enviar a Rezín, rey de Siria, y a Peka, hijo de Remalías, contra Judá.
\footnote{\textbf{15:37} 2Re 16,5} \bibleverse{38} Jotam durmió con sus
padres y fue enterrado con ellos en la ciudad de su padre David; y su
hijo Acaz reinó en su lugar.

\hypertarget{las-abominaciones-paganas-de-acaz}{%
\subsection{Las abominaciones paganas de
Acaz}\label{las-abominaciones-paganas-de-acaz}}

\hypertarget{section-15}{%
\section{16}\label{section-15}}

\bibleverse{1} En el año diecisiete de Peka, hijo de Remalías, comenzó a
reinar Acaz, hijo de Jotam, rey de Judá. \footnote{\textbf{16:1} 2Re
  15,38} \bibleverse{2} Acaz tenía veinte años cuando comenzó a reinar,
y reinó dieciséis años en Jerusalén. No hizo lo que era correcto a los
ojos de Yavé, su Dios, como David, su padre. \bibleverse{3} Sino que
siguió el camino de los reyes de Israel, e incluso hizo pasar a su hijo
por el fuego, según las abominaciones de las naciones que Yahvé echó de
delante de los hijos de Israel. \footnote{\textbf{16:3} 2Re 21,6; Lev
  18,21} \bibleverse{4} Sacrificó y quemó incienso en los lugares altos,
en las colinas y debajo de todo árbol verde.

\hypertarget{su-guerra-con-siria-e-israel-acaz-se-convierte-en-tributo-a-los-asirios}{%
\subsection{Su guerra con Siria e Israel; Acaz se convierte en tributo a
los
asirios}\label{su-guerra-con-siria-e-israel-acaz-se-convierte-en-tributo-a-los-asirios}}

\bibleverse{5} Entonces Rezín, rey de Siria, y Peka, hijo de Remalías,
rey de Israel, subieron a Jerusalén para hacer la guerra. Asediaron a
Acaz, pero no pudieron vencerlo. \footnote{\textbf{16:5} Is 7,1-9}
\bibleverse{6} En aquel tiempo, Rezín, rey de Siria, recuperó Elat para
Siria y expulsó a los judíos de Elat; y los sirios llegaron a Elat y
vivieron allí hasta el día de hoy. \footnote{\textbf{16:6} 2Re 14,22}
\bibleverse{7} Entonces Acaz envió mensajeros a Tiglat Pileser, rey de
Asiria, diciendo: ``Soy tu siervo y tu hijo. Sube y sálvame de la mano
del rey de Siria y de la mano del rey de Israel, que se levantan contra
mí''. \footnote{\textbf{16:7} 2Re 15,29} \bibleverse{8} Ajaz tomó la
plata y el oro que se encontraba en la casa de Yavé y en los tesoros de
la casa real, y lo envió como regalo al rey de Asiria. \footnote{\textbf{16:8}
  1Re 15,18} \bibleverse{9} El rey de Asiria lo escuchó; y el rey de
Asiria subió contra Damasco y la tomó, y llevó a su pueblo cautivo a
Kir, y mató a Rezín.

\hypertarget{acaz-tiene-un-nuevo-altar-para-los-holocaustos-construido-emite-una-nueva-orden-de-sacrificio-e-interviene-en-la-propiedad-del-templo}{%
\subsection{Acaz tiene un nuevo altar para los holocaustos construido,
emite una nueva orden de sacrificio e interviene en la propiedad del
templo}\label{acaz-tiene-un-nuevo-altar-para-los-holocaustos-construido-emite-una-nueva-orden-de-sacrificio-e-interviene-en-la-propiedad-del-templo}}

\bibleverse{10} El rey Acaz fue a Damasco para reunirse con Tiglat
Pileser, rey de Asiria, y vio el altar que estaba en Damasco; y el rey
Acaz envió al sacerdote Urías un dibujo del altar y los planos para
construirlo. \bibleverse{11} El sacerdote Urías construyó el altar.
Según todo lo que el rey Acaz había enviado desde Damasco, así lo hizo
el sacerdote Urías para la venida del rey Acaz desde Damasco.
\bibleverse{12} Cuando el rey llegó de Damasco, vio el altar; y el rey
se acercó al altar y ofreció sobre él. \bibleverse{13} Quemó su
holocausto y su ofrenda, derramó su libación y roció la sangre de sus
ofrendas de paz sobre el altar. \bibleverse{14} El altar de bronce que
estaba delante de Yavé lo trajo de la parte delantera de la casa, de
entre su altar y la casa de Yavé, y lo puso al lado norte de su altar.
\bibleverse{15} El rey Acaz ordenó al sacerdote Urías que dijera: ``En
el gran altar quema el holocausto de la mañana, la ofrenda de la tarde,
el holocausto del rey y su ofrenda de comida, con el holocausto de todo
el pueblo del país, su ofrenda de comida y sus ofrendas de bebida; y
rocía sobre él toda la sangre del holocausto y toda la sangre del
sacrificio; pero el altar de bronce será para que yo lo consulte.''
\bibleverse{16} El sacerdote Urías lo hizo así, conforme a todo lo que
el rey Acaz había ordenado.

\bibleverse{17} El rey Acaz cortó los paneles de las bases y quitó la
pila de ellas, y quitó el mar de los bueyes de bronce que estaban
debajo, y lo puso sobre un pavimento de piedra. \footnote{\textbf{16:17}
  1Re 7,23-39} \bibleverse{18} Quitó el camino cubierto para el sábado
que habían construido en la casa, y la entrada exterior del rey a la
casa de Yavé, a causa del rey de Asiria. \bibleverse{19} El resto de los
actos de Acaz que hizo, ¿no están escritos en el libro de las crónicas
de los reyes de Judá? \bibleverse{20} Acaz durmió con sus padres y fue
enterrado con ellos en la ciudad de David; y su hijo Ezequías reinó en
su lugar. \footnote{\textbf{16:20} 2Re 18,1}

\hypertarget{oseas-rey-de-israel-cauxedda-del-imperio-cautiverio-asirio}{%
\subsection{Oseas rey de Israel; Caída del imperio; Cautiverio
asirio}\label{oseas-rey-de-israel-cauxedda-del-imperio-cautiverio-asirio}}

\hypertarget{section-16}{%
\section{17}\label{section-16}}

\bibleverse{1} En el duodécimo año de Acaz, rey de Judá, Oseas, hijo de
Ela, comenzó a reinar en Samaria sobre Israel durante nueve años.
\footnote{\textbf{17:1} 2Re 15,30} \bibleverse{2} Hizo lo que era malo a
los ojos de Yavé, pero no como los reyes de Israel que lo precedieron.
\bibleverse{3} Salmanasar, rey de Asiria, subió contra él, y Oseas se
convirtió en su siervo y le trajo tributo. \footnote{\textbf{17:3} 2Re
  18,9-12} \bibleverse{4} El rey de Asiria descubrió una conspiración en
Oseas, pues éste había enviado mensajeros a So, rey de Egipto, y no
ofrecía tributo al rey de Asiria, como lo había hecho año tras año. Por
lo tanto, el rey de Asiria lo apresó y lo encarceló. \footnote{\textbf{17:4}
  Os 12,1} \bibleverse{5} Entonces el rey de Asiria recorrió todo el
país, subió a Samaria y la sitió durante tres años. \bibleverse{6} En el
noveno año de Oseas, el rey de Asiria tomó Samaria y se llevó a Israel a
Asiria, y los puso en Halah y en el Habor, el río de Gozán, y en las
ciudades de los medos.

\bibleverse{7} Fue así porque los hijos de Israel habían pecado contra
Yahvé su Dios, que los sacó de la tierra de Egipto de la mano del
faraón, rey de Egipto, y habían temido a otros dioses, \bibleverse{8} y
anduvieron en los estatutos de las naciones que Yahvé echó de delante de
los hijos de Israel, y de los reyes de Israel, que ellos hicieron.
\footnote{\textbf{17:8} 2Re 16,3} \bibleverse{9} Los hijos de Israel
hicieron en secreto cosas que no eran rectas contra Yahvé su Dios; y se
construyeron lugares altos en todas sus ciudades, desde la torre de los
vigías hasta la ciudad fortificada; \bibleverse{10} y se erigieron
columnas y postes de Asera en todo cerro alto y debajo de todo árbol
verde \footnote{\textbf{17:10} 2Re 16,4; 1Re 14,23} \bibleverse{11} y
quemaron incienso en todos los lugares altos, como lo hicieron las
naciones que Yahvé transportó antes de ellos; e hicieron cosas perversas
para provocar la ira de Yahvé; \footnote{\textbf{17:11} 2Re 17,8}
\bibleverse{12} y sirvieron a ídolos, de los cuales Yahvé les había
dicho: ``No harás esto.'' \footnote{\textbf{17:12} Éxod 20,2-3; Éxod
  23,13} \bibleverse{13} Sin embargo, Yahvé dio testimonio a Israel y a
Judá, por medio de todo profeta y todo vidente, diciendo: ``Convertíos
de vuestros malos caminos, y guardad mis mandamientos y mis estatutos,
conforme a toda la ley que ordené a vuestros padres, y que os envié por
medio de mis siervos los profetas.'' \bibleverse{14} Sin embargo, no
quisieron escuchar, sino que endurecieron su cuello como el de sus
padres, que no creyeron en Yavé, su Dios. \bibleverse{15} Rechazaron sus
estatutos y su pacto que había hecho con sus padres, y sus testimonios
que les había atestiguado; siguieron la vanidad y se envanecieron, y
siguieron a las naciones que estaban a su alrededor, acerca de las
cuales Yahvé les había ordenado que no hicieran como ellas. \footnote{\textbf{17:15}
  Éxod 23,24} \bibleverse{16} Abandonaron todos los mandatos de Yahvé,
su Dios, y se hicieron imágenes fundidas, dos becerros, e hicieron una
Asera, y adoraron a todo el ejército del cielo, y sirvieron a Baal.
\footnote{\textbf{17:16} 1Re 12,28; 1Re 16,33} \bibleverse{17} Hicieron
pasar por el fuego a sus hijos y a sus hijas, usaron la adivinación y
los encantamientos, y se vendieron para hacer lo que era malo a los ojos
de Yahvé, para provocarlo a la ira. \footnote{\textbf{17:17} 2Re 16,3}
\bibleverse{18} Por eso Yahvé se enojó mucho con Israel y los quitó de
su vista. No quedó más que la tribu de Judá.

\bibleverse{19} También Judá no guardó los mandamientos de Yavé, su
Dios, sino que anduvo en los estatutos de Israel que ellos hicieron.
\bibleverse{20} Yahvé rechazó a toda la descendencia de Israel, la
afligió y la entregó en manos de salteadores, hasta echarla de su vista.

\hypertarget{las-causas-que-provocaron-el-rechazo-y-la-cauxedda-del-reino-del-norte}{%
\subsection{Las causas que provocaron el rechazo y la caída del reino
del
norte}\label{las-causas-que-provocaron-el-rechazo-y-la-cauxedda-del-reino-del-norte}}

\bibleverse{21} Porque arrancó a Israel de la casa de David, e hicieron
rey a Jeroboam hijo de Nabat; y Jeroboam apartó a Israel de seguir a
Yavé y les hizo cometer un gran pecado. \footnote{\textbf{17:21} 1Re
  12,20} \bibleverse{22} Los hijos de Israel anduvieron en todos los
pecados de Jeroboam que él cometió; no se apartaron de ellos
\bibleverse{23} hasta que Yahvé quitó a Israel de su vista, como lo dijo
por medio de todos sus siervos los profetas. Así que Israel fue llevado
de su propia tierra a Asiria hasta el día de hoy. \footnote{\textbf{17:23}
  Deut 28,63-64}

\hypertarget{repoblaciuxf3n-del-pauxeds-origen-de-los-samaritanos-y-su-religiuxf3n}{%
\subsection{Repoblación del país; Origen de los samaritanos y su
religión}\label{repoblaciuxf3n-del-pauxeds-origen-de-los-samaritanos-y-su-religiuxf3n}}

\bibleverse{24} El rey de Asiria trajo gente de Babilonia, de Cuta, de
Avva, de Hamat y de Sefarvaim, y los colocó en las ciudades de Samaria
en lugar de los hijos de Israel; y se apoderaron de Samaria y vivieron
en sus ciudades. \bibleverse{25} Así fue que, al principio de su
permanencia allí, no temieron a Yavé. Por eso Yahvé envió leones entre
ellos, que mataron a algunos de ellos. \bibleverse{26} Por eso hablaron
con el rey de Asiria, diciendo: ``Las naciones que has transportado y
colocado en las ciudades de Samaria no conocen la ley del dios de la
tierra. Por eso ha enviado leones entre ellos; y he aquí que los matan,
porque no conocen la ley del dios de la tierra.''

\bibleverse{27} Entonces el rey de Asiria ordenó: ``Lleva allí a uno de
los sacerdotes que trajiste de allí, y que\footnote{\textbf{17:27}
  Hebreo: ellos} vaya y habite allí, y que les enseñe la ley del dios de
la tierra.''

\bibleverse{28} Entonces uno de los sacerdotes que habían llevado de
Samaria vino y vivió en Betel, y les enseñó cómo debían temer a Yavé.

\bibleverse{29} Sin embargo, cada nación hizo sus propios dioses y los
puso en las casas de los lugares altos que los samaritanos habían hecho,
cada nación en sus ciudades en las que vivían. \bibleverse{30} Los
hombres de Babilonia hicieron a Succoth Benot, y los hombres de Cut
hicieron a Nergal, y los hombres de Hamat hicieron a Ashima,
\bibleverse{31} y los avvitas hicieron a Nibhaz y a Tartak; y los
sefarvitas quemaron a sus hijos en el fuego a Adrammelech y a
Anammelech, los dioses de Sefarvaim. \footnote{\textbf{17:31} 2Re 17,17}
\bibleverse{32} Temían, pues, a Yavé, y también hacían de entre ellos
sacerdotes de los lugares altos, que sacrificaban para ellos en las
casas de los lugares altos. \bibleverse{33} Temían a Yahvé, y también
servían a sus propios dioses, según los caminos de las naciones de las
que habían sido transportados. \bibleverse{34} Hasta el día de hoy hacen
lo mismo que antes. No temen a Yahvé, y no siguen los estatutos, ni las
ordenanzas, ni la ley, ni el mandamiento que Yahvé ordenó a los hijos de
Jacob, a quienes llamó Israel; \bibleverse{35} con quienes Yahvé había
hecho un pacto y les había ordenado, diciendo: ``No temeréis a otros
dioses, ni os inclinaréis ante ellos, ni les serviréis, ni les
sacrificaréis \footnote{\textbf{17:35} Éxod 23,24} \bibleverse{36} sino
que temeréis a Yahvé, que os sacó de la tierra de Egipto con gran poder
y con brazo extendido, y a él os inclinaréis y a él sacrificaréis.
\bibleverse{37} Los estatutos y las ordenanzas, la ley y el mandamiento
que él escribió para vosotros, los cumpliréis para siempre. No temeréis
a otros dioses. \bibleverse{38} No olvidarás el pacto que he hecho
contigo. No temerán a otros dioses. \bibleverse{39} Sino que temerás a
Yahvé, tu Dios, y él te librará de la mano de todos tus enemigos.''
\footnote{\textbf{17:39} Deut 6,12-19}

\bibleverse{40} Sin embargo, no escucharon, sino que hicieron lo mismo
que antes. \bibleverse{41} Así que estas naciones temieron a Yavé, y
también sirvieron a sus imágenes grabadas. Sus hijos hicieron lo mismo,
y también los hijos de sus hijos. Hacen lo mismo que hicieron sus padres
hasta el día de hoy.

\hypertarget{ezequuxedas-asumiuxf3-el-cargo-su-piedad-y-sus-servicios-al-culto-y-al-bien-puxfablico}{%
\subsection{Ezequías asumió el cargo, su piedad y sus servicios al culto
y al bien
público}\label{ezequuxedas-asumiuxf3-el-cargo-su-piedad-y-sus-servicios-al-culto-y-al-bien-puxfablico}}

\hypertarget{section-17}{%
\section{18}\label{section-17}}

\bibleverse{1} En el tercer año de Oseas hijo de Ela, rey de Israel,
comenzó a reinar Ezequías hijo de Acaz, rey de Judá. \footnote{\textbf{18:1}
  2Re 16,20} \bibleverse{2} Tenía veinticinco años cuando comenzó a
reinar, y reinó veintinueve años en Jerusalén. Su madre se llamaba Abi,
hija de Zacarías. \footnote{\textbf{18:2} 2Cró 29,1-2} \bibleverse{3}
Hizo lo que era justo a los ojos del Señor, conforme a todo lo que había
hecho su padre David. \footnote{\textbf{18:3} 2Re 20,3} \bibleverse{4}
Quitó los lugares altos, rompió las columnas y derribó la Asera. También
hizo pedazos la serpiente de bronce que había hecho Moisés, porque en
aquellos días los hijos de Israel le quemaban incienso; y la llamó
Nehustán. \footnote{\textbf{18:4} 2Cró 31,1; 2Re 15,35; Núm 21,8-9}
\bibleverse{5} Confió en Yahvé, el Dios de Israel, de modo que después
de él no hubo nadie como él entre todos los reyes de Judá, ni entre los
que le precedieron. \footnote{\textbf{18:5} 2Re 23,25} \bibleverse{6}
Porque se unió a Yavé. No se apartó de seguirlo, sino que guardó sus
mandamientos, que Yahvé le ordenó a Moisés. \bibleverse{7} El Señor
estaba con él. Dondequiera que iba, prosperaba. Se rebeló contra el rey
de Asiria y no le sirvió. \bibleverse{8} Golpeó a los filisteos hasta
Gaza y sus fronteras, desde la torre de los vigías hasta la ciudad
fortificada.

\hypertarget{la-cauxedda-de-samaria}{%
\subsection{La caída de Samaria}\label{la-cauxedda-de-samaria}}

\bibleverse{9} En el cuarto año del rey Ezequías, que era el séptimo año
de Oseas hijo de Ela, rey de Israel, Salmanasar, rey de Asiria, subió
contra Samaria y la sitió. \footnote{\textbf{18:9} 2Re 17,3-6}
\bibleverse{10} Al cabo de tres años la tomaron. En el sexto año de
Ezequías, que era el noveno año de Oseas, rey de Israel, Samaria fue
tomada. \bibleverse{11} El rey de Asiria se llevó a Israel a Asiria, y
los puso en Halah y en el Habor, el río de Gozán, y en las ciudades de
los medos, \bibleverse{12} porque no obedecieron la voz de Yavé, su
Dios, sino que transgredieron su pacto, todo lo que mandó Moisés, siervo
de Yavé, y no quisieron oírlo ni hacerlo.

\hypertarget{ezequuxedas-envuxeda-sin-uxe9xito-el-tributo-exigido-por-senaquerib}{%
\subsection{Ezequías envía sin éxito el tributo exigido por
Senaquerib}\label{ezequuxedas-envuxeda-sin-uxe9xito-el-tributo-exigido-por-senaquerib}}

\bibleverse{13} En el año catorce del rey Ezequías, Senaquerib, rey de
Asiria, subió contra todas las ciudades fortificadas de Judá y las tomó.
\bibleverse{14} Ezequías, rey de Judá, envió al rey de Asiria a Laquis,
diciendo: ``Te he ofendido. Retírate de mí. Lo que me impongas, lo
soportaré''. El rey de Asiria asignó a Ezequías, rey de Judá,
trescientos talentos de plata y treinta talentos\footnote{\textbf{18:14}
  Un talento equivale a unos 30 kilogramos o 66 libras o 965 onzas
  troyanas} de oro. \footnote{\textbf{18:14} 2Re 18,7} \bibleverse{15}
Ezequías le dio toda la plata que se encontraba en la casa de Yavé y en
los tesoros de la casa real. \footnote{\textbf{18:15} 2Re 16,8}
\bibleverse{16} En aquel tiempo, Ezequías cortó el oro de las puertas
del templo de Yavé y de las columnas que Ezequías, rey de Judá, había
recubierto, y se lo dio al rey de Asiria.

\hypertarget{desde-laquis-senaquerib-hace-que-la-ciudad-de-jerusaluxe9n-sea-convocada-desdeuxf1osamente-a-rendirse-por-su-gran-visir}{%
\subsection{Desde Laquis, Senaquerib hace que la ciudad de Jerusalén sea
convocada desdeñosamente a rendirse por su gran
visir}\label{desde-laquis-senaquerib-hace-que-la-ciudad-de-jerusaluxe9n-sea-convocada-desdeuxf1osamente-a-rendirse-por-su-gran-visir}}

\bibleverse{17} El rey de Asiria envió a Tartán, a Rabsaris y a Rabsaces
desde Laquis al rey Ezequías con un gran ejército a Jerusalén. Subieron
y llegaron a Jerusalén. Cuando subieron, vinieron y se pararon junto al
conducto del estanque superior, que está en el camino del campo del
batán. \bibleverse{18} Cuando llamaron al rey, salieron hacia ellos
Eliaquim, hijo de Hilcías, que estaba al frente de la casa, y Sebnah, el
escriba, y Joah, hijo de Asaf, el registrador. \bibleverse{19} Rabsaces
les dijo: ``Di ahora a Ezequías: ``El gran rey, el rey de Asiria, dice:
``¿Qué confianza es ésta en la que confías? \bibleverse{20} Ustedes
dicen (pero no son más que palabras vanas): `Hay consejo y fuerza para
la guerra'. Ahora bien, ¿en quién confían ustedes, que se han rebelado
contra mí? \bibleverse{21} Ahora bien, he aquí que confiáis en el bastón
de esta caña magullada, incluso en Egipto. Si un hombre se apoya en
ella, se le meterá en la mano y la atravesará. Así es el Faraón, rey de
Egipto, para todos los que confían en él. \bibleverse{22} Pero si me
decís: ``Confiamos en el Señor, nuestro Dios'', ¿no es aquel cuyos
lugares altos y cuyos altares ha quitado Ezequías, y ha dicho a Judá y a
Jerusalén: ``Adoraréis ante este altar en Jerusalén''? \footnote{\textbf{18:22}
  Éxod 20,24; Deut 12,14} \bibleverse{23} Ahora, pues, por favor, dad
prendas a mi amo el rey de Asiria, y yo os daré dos mil caballos si sois
capaces de poner jinetes en ellos. \bibleverse{24} ¿Cómo, pues, puedes
rechazar el rostro de un capitán del más pequeño de los siervos de mi
amo, y poner tu confianza en Egipto para carros y jinetes?
\bibleverse{25} ¿Acaso he subido sin Yahvé contra este lugar para
destruirlo? Yahvé me dijo: ``Sube contra esta tierra y destrúyela''\,''.

\hypertarget{senaquerib-y-la-arrogancia-de-sus-embajadores}{%
\subsection{Senaquerib y la arrogancia de sus
embajadores}\label{senaquerib-y-la-arrogancia-de-sus-embajadores}}

\bibleverse{26} Entonces Eliaquim, hijo de Jilquías, Sebna y Joá,
dijeron a Rabsaces: ``Por favor, habla a tus siervos en lengua siria,
porque nosotros la entendemos. No hables con nosotros en la lengua de
los judíos, a oídos del pueblo que está en el muro''.

\bibleverse{27} Pero Rabsaces les dijo: ``¿Acaso mi amo me ha enviado a
su amo y a ustedes para decirles estas palabras? ¿No me ha enviado a los
hombres que se sientan en el muro, para que coman su propio estiércol y
beban su propia orina con ustedes?''

\bibleverse{28} Entonces Rabsaces se puso de pie y gritó con gran voz en
el idioma de los judíos, y habló diciendo: ``Oigan la palabra del gran
rey, el rey de Asiria. \bibleverse{29} El rey dice: `No dejes que
Ezequías te engañe, porque no podrá librarte de su mano. \bibleverse{30}
No dejen que Ezequías los haga confiar en Yavé, diciendo: ``Seguramente
Yavé nos librará, y esta ciudad no será entregada en manos del rey de
Asiria.'' \bibleverse{31} No escuchen a Ezequías'. Porque el rey de
Asiria dice: `Hagan las paces conmigo y salgan a mi encuentro; y cada
uno de ustedes coma de su propia vid, y cada uno de su propia higuera, y
cada uno beba agua de su propia cisterna; \footnote{\textbf{18:31} 1Re
  4,25} \bibleverse{32} hasta que yo venga y los lleve a una tierra como
la suya, una tierra de grano y de vino nuevo, una tierra de pan y de
viñas, una tierra de olivos y de miel, para que vivan y no mueran. No
escuchen a Ezequías cuando los convenza diciendo: ``El Señor nos
librará''. \bibleverse{33} ¿Acaso alguno de los dioses de las naciones
ha librado su tierra de la mano del rey de Asiria? \footnote{\textbf{18:33}
  Is 10,10-11} \bibleverse{34} ¿Dónde están los dioses de Hamat y de
Arpad? ¿Dónde están los dioses de Sefarvaim, de Hena y de Ivva? ¿Han
librado a Samaria de mi mano? \bibleverse{35} ¿Quiénes son, entre todos
los dioses de los países, los que han librado a su país de mi mano, para
que Yahvé libere a Jerusalén de mi mano?''

\bibleverse{36} Pero el pueblo se quedó callado y no le respondió ni una
sola palabra, porque la orden del rey era: ``No le respondan''.
\bibleverse{37} Entonces Eliaquim, hijo de Hilcías, que estaba al frente
de la casa, vino con Sebna, el escriba, y con Joah, hijo de Asaf, el
registrador, a Ezequías con las ropas rasgadas, y le contaron las
palabras de Rabsaces.

\hypertarget{el-estuxedmulo-de-ezequuxedas-de-isauxedas}{%
\subsection{El estímulo de Ezequías de
Isaías}\label{el-estuxedmulo-de-ezequuxedas-de-isauxedas}}

\hypertarget{section-18}{%
\section{19}\label{section-18}}

\bibleverse{1} Cuando el rey Ezequías lo oyó, se rasgó las vestiduras,
se cubrió de cilicio y entró en la casa de Yahvé. \bibleverse{2} Envió a
Eliaquim, que estaba al frente de la casa, a Sebna, el escriba, y a los
ancianos de los sacerdotes, cubiertos de cilicio, a ver al profeta
Isaías, hijo de Amoz. \bibleverse{3} Le dijeron: ``Ezequías dice: `Hoy
es un día de angustia, de reprimenda y de rechazo; porque los niños han
llegado al punto de nacer, y no hay fuerza para librarlos.
\bibleverse{4} Puede ser que Yavé tu Dios escuche todas las palabras de
Rabsaces, a quien el rey de Asiria, su amo, ha enviado para desafiar al
Dios vivo, y reprenda las palabras que Yavé tu Dios ha escuchado. Por lo
tanto, levanta tu oración por el remanente que queda'\,''. \footnote{\textbf{19:4}
  2Re 18,35}

\bibleverse{5} Los siervos del rey Ezequías acudieron a Isaías.
\bibleverse{6} Isaías les dijo: ``Díganle esto a su amo: El Señor dice:
``No temas las palabras que has oído, con las que los servidores del rey
de Asiria me han blasfemado. \bibleverse{7} He aquí que yo pondré un
espíritu en él, y oirá noticias, y volverá a su tierra. Haré que caiga a
espada en su propia tierra''. \footnote{\textbf{19:7} 2Re 9,35-37}

\hypertarget{la-segunda-solicitud-de-sennacherib-a-travuxe9s-de-una-carta-amenazante-de-libna}{%
\subsection{La segunda solicitud de Sennacherib a través de una carta
amenazante de
Libna}\label{la-segunda-solicitud-de-sennacherib-a-travuxe9s-de-una-carta-amenazante-de-libna}}

\bibleverse{8} Volvió, pues, Rabsaces y encontró al rey de Asiria
guerreando contra Libna, pues había oído que había salido de Laquis.
\bibleverse{9} Cuando oyó decir de Tirhakah, rey de Etiopía: ``He aquí
que ha salido a pelear contra ti'', volvió a enviar mensajeros a
Ezequías, diciendo: \bibleverse{10} ``Dile a Ezequías, rey de Judá, lo
siguiente: `No permitas que tu Dios, en quien confías, te engañe
diciendo que Jerusalén no será entregada en manos del rey de Asiria.
\footnote{\textbf{19:10} 2Re 18,30} \bibleverse{11} He aquí, tú has oído
lo que los reyes de Asiria han hecho a todas las tierras, destruyéndolas
por completo. ¿Serás liberado? \bibleverse{12} ¿Los dioses de las
naciones los han librado, a los que mis padres han destruido: Gozán,
Harán, Rezef y los hijos de Edén que estaban en Telasar? \footnote{\textbf{19:12}
  2Re 18,33-34} \bibleverse{13} ¿Dónde está el rey de Hamat, el rey de
Arpad y el rey de la ciudad de Sefarvaim, de Hena y de Ivva?''

\hypertarget{la-suxfaplica-de-ezequuxedas-en-el-templo}{%
\subsection{La súplica de Ezequías en el
templo}\label{la-suxfaplica-de-ezequuxedas-en-el-templo}}

\bibleverse{14} Ezequías recibió la carta de manos de los mensajeros y
la leyó. Entonces Ezequías subió a la casa de Yavé y la extendió ante
Yavé. \bibleverse{15} Ezequías oró ante Yavé y dijo: ``Yavé, Dios de
Israel, que estás entronizado sobre los querubines, tú eres el Dios,
sólo tú, de todos los reinos de la tierra. Tú has hecho el cielo y la
tierra. \footnote{\textbf{19:15} Éxod 25,22; Sal 80,1} \bibleverse{16}
Inclina tu oído, Yahvé, y escucha. Abre tus ojos, Yahvé, y mira. Escucha
las palabras de Senaquerib, que ha enviado para desafiar al Dios vivo.
\footnote{\textbf{19:16} 2Re 19,4; 1Sam 17,10} \bibleverse{17} En
verdad, Yahvé, los reyes de Asiria han asolado a las naciones y a sus
tierras, \bibleverse{18} y han echado al fuego a sus dioses, pues no
eran dioses, sino obra de manos de hombres, madera y piedra. Por eso los
han destruido. \bibleverse{19} Ahora, pues, Yahvé, nuestro Dios,
sálvanos, te lo ruego, de su mano, para que todos los reinos de la
tierra sepan que tú, Yahvé, eres el único Dios.''

\hypertarget{isauxedas-envuxeda-notificaciuxf3n-de-su-oraciuxf3n-al-rey-ezequuxedas-en-el-nombre-de-dios}{%
\subsection{Isaías envía notificación de su oración al rey Ezequías en
el nombre de
Dios}\label{isauxedas-envuxeda-notificaciuxf3n-de-su-oraciuxf3n-al-rey-ezequuxedas-en-el-nombre-de-dios}}

\bibleverse{20} Entonces Isaías, hijo de Amoz, envió a decir a Ezequías:
``Yahvé, el Dios de Israel, dice: `Me has orado contra Senaquerib, rey
de Asiria, y te he escuchado. \bibleverse{21} Esta es la palabra que
Yahvé ha pronunciado sobre él: `La virgen hija de Sión te ha despreciado
y se ha burlado de ti. La hija de Jerusalén ha sacudido la cabeza ante
ti. \bibleverse{22} ¿A quién has desafiado y blasfemado? ¿Contra quién
has alzado tu voz y levantado tus ojos en alto? ¡Contra el Santo de
Israel! \bibleverse{23} Por medio de tus mensajeros, has desafiado al
Señor y has dicho: ``Con la multitud de mis carros, he subido a la
altura de los montes, a lo más recóndito del Líbano, y cortaré sus altos
cedros y sus selectos cipreses; y entraré en su más lejana morada, en el
bosque de su campo fructífero. \bibleverse{24} He cavado y bebido aguas
extrañas, y secaré todos los ríos de Egipto con la planta de mis pies''.
\bibleverse{25} ¿No has oído cómo lo he hecho hace mucho tiempo, y lo he
formado de antiguo? Ahora he hecho que sea tuyo el arrasar las ciudades
fortificadas hasta convertirlas en montones ruinosos. \bibleverse{26}
Por eso sus habitantes tenían poco poder. Estaban consternados y
confundidos. Eran como la hierba del campo y como la hierba verde, como
la hierba de los tejados y como el grano desgastado antes de crecer.
\bibleverse{27} Pero yo sé que te sientas, que sales, que entras y que
te enfureces contra mí. \bibleverse{28} A causa de tu furia contra mí, y
porque tu arrogancia ha subido a mis oídos, pondré mi garfio en tu nariz
y mi freno en tus labios, y te haré volver por el camino por el que
viniste.

\bibleverse{29} ``Esta será la señal para vosotros: Este año comeréis lo
que crezca por sí mismo, y el segundo año lo que brote de él; y el
tercer año sembraréis y segaréis, y plantaréis viñas y comeréis su
fruto. \bibleverse{30} El remanente que ha escapado de la casa de Judá
volverá a echar raíces hacia abajo y a dar fruto hacia arriba.
\bibleverse{31} Porque de Jerusalén saldrá un remanente, y del monte
Sión los que escapen. El celo de Yahvé lo realizará. \footnote{\textbf{19:31}
  Is 9,7}

\bibleverse{32} ``Por eso dice el Señor sobre el rey de Asiria: `No
vendrá a esta ciudad, ni lanzará una flecha contra ella. No vendrá ante
ella con escudo, ni levantará un montículo contra ella. \bibleverse{33}
Volverá por el mismo camino por el que vino, y no vendrá a esta ciudad',
dice el Señor. \bibleverse{34} `Porque yo defenderé esta ciudad para
salvarla, por mí y por mi siervo David'\,''. \footnote{\textbf{19:34}
  2Re 20,6}

\hypertarget{el-cumplimiento-de-la-promesa-la-partida-y-el-asesinato-de-senaquerib}{%
\subsection{El cumplimiento de la promesa: la partida y el asesinato de
Senaquerib}\label{el-cumplimiento-de-la-promesa-la-partida-y-el-asesinato-de-senaquerib}}

\bibleverse{35} Aquella noche, el ángel de Yavé salió e hirió a ciento
ochenta y cinco mil en el campamento de los asirios. Cuando los hombres
se levantaron de madrugada, he aquí que todos ellos eran cadáveres.
\bibleverse{36} Entonces Senaquerib, rey de Asiria, partió, se fue a su
casa y vivió en Nínive. \bibleverse{37} Mientras adoraba en la casa de
Nisroc, su dios, Adramelec y Sharezer lo hirieron con la espada, y
escaparon a la tierra de Ararat. Su hijo Esar Haddón reinó en su lugar.
\footnote{\textbf{19:37} 2Re 19,7}

\hypertarget{la-enfermedad-y-la-recuperaciuxf3n-de-ezequuxedas-la-embajada-de-babilonia}{%
\subsection{La enfermedad y la recuperación de Ezequías; la embajada de
babilonia}\label{la-enfermedad-y-la-recuperaciuxf3n-de-ezequuxedas-la-embajada-de-babilonia}}

\hypertarget{section-19}{%
\section{20}\label{section-19}}

\bibleverse{1} En aquellos días Ezequías estaba enfermo y moribundo. El
profeta Isaías, hijo de Amoz, se acercó a él y le dijo: ``Yahvé dice:
`Pon en orden tu casa, porque morirás y no vivirás'\,''.

\bibleverse{2} Entonces volvió su rostro hacia la pared y oró a Yavé,
diciendo: \bibleverse{3} ``Acuérdate ahora, Yavé, te lo ruego, de cómo
he andado delante de ti con verdad y con un corazón perfecto, y he hecho
lo que es bueno ante tus ojos.'' Y Ezequías lloró amargamente.

\bibleverse{4} Antes de que Isaías saliera a la parte central de la
ciudad, le llegó la palabra de Yavé, que decía: \bibleverse{5} ``Vuelve
y dile a Ezequías, príncipe de mi pueblo: `Yavé, el Dios de David, tu
padre, dice: ``He oído tu oración. He visto tus lágrimas. He aquí que yo
te curaré. Al tercer día subirás a la casa de Yavé. \bibleverse{6}
Añadiré a tus días quince años. Te libraré a ti y a esta ciudad de la
mano del rey de Asiria. Defenderé esta ciudad por mí y por mi siervo
David''. \footnote{\textbf{20:6} 2Re 19,34}

\bibleverse{7} Isaías dijo: ``Toma una torta de higos''. Lo cogieron y
lo pusieron a hervir, y se recuperó.

\hypertarget{el-signo-del-milagro-divino-en-el-reloj-de-sol}{%
\subsection{El signo del milagro divino en el reloj de
sol}\label{el-signo-del-milagro-divino-en-el-reloj-de-sol}}

\bibleverse{8} Ezequías dijo a Isaías: ``¿Cuál será la señal de que
Yahvé me sanará y de que subiré a la casa de Yahvé al tercer día?''

\bibleverse{9} Isaías dijo: ``Esta será la señal para ustedes de parte
de Yahvé, de que Yahvé hará lo que ha dicho: ¿debe la sombra avanzar
diez pasos, o retroceder diez pasos?''

\bibleverse{10} Ezequías respondió: ``Es cosa ligera que la sombra
avance diez pasos. No, sino que la sombra vuelva atrás diez pasos''.

\bibleverse{11} El profeta Isaías clamó a Yahvé, y éste hizo retroceder
diez pasos la sombra que había descendido en el reloj de sol de Acaz.

\hypertarget{embajada-de-merodac-baladan-de-babilonia}{%
\subsection{Embajada de Merodac-Baladan de
Babilonia}\label{embajada-de-merodac-baladan-de-babilonia}}

\bibleverse{12} En aquel tiempo Berodac Baladán, hijo de Baladán, rey de
Babilonia, envió cartas y un regalo a Ezequías, pues había oído que
Ezequías había estado enfermo. \bibleverse{13} Ezequías los escuchó y
les mostró todo el depósito de sus cosas preciosas: la plata, el oro,
las especias y el aceite precioso, y la casa de su armadura, y todo lo
que se encontraba en sus tesoros. No había nada en su casa, ni en todo
su dominio, que Ezequías no les mostrara.

\hypertarget{el-discurso-de-castigo-de-isauxedas-sobre-la-pompa-descuidada-del-rey-y-su-profecuxeda-sobre-el-cautiverio-en-babilonia}{%
\subsection{El discurso de castigo de Isaías sobre la pompa descuidada
del rey y su profecía sobre el cautiverio en
Babilonia}\label{el-discurso-de-castigo-de-isauxedas-sobre-la-pompa-descuidada-del-rey-y-su-profecuxeda-sobre-el-cautiverio-en-babilonia}}

\bibleverse{14} Entonces el profeta Isaías se acercó al rey Ezequías y
le dijo: ``¿Qué han dicho estos hombres? ¿De dónde han venido a ti?''
Ezequías dijo: ``Han venido de un país lejano, incluso de Babilonia''.

\bibleverse{15} Él dijo: ``¿Qué han visto en tu casa?'' Ezequías
respondió: ``Han visto todo lo que hay en mi casa. No hay nada entre mis
tesoros que no les haya mostrado''.

\bibleverse{16} Isaías dijo a Ezequías: ``Escucha la palabra de Yahvé.
\bibleverse{17} `He aquí que vienen días en que todo lo que hay en tu
casa, y lo que tus padres han almacenado hasta hoy, será llevado a
Babilonia. No quedará nada', dice el Señor. \footnote{\textbf{20:17} 2Re
  24,13-14} \bibleverse{18} `Se llevarán a algunos de tus hijos que
saldrán de ti, a los que engendrarás, y serán eunucos en el palacio del
rey de Babilonia'\,''. \footnote{\textbf{20:18} Dan 1,3-4}

\hypertarget{la-respuesta-devota-pero-impenitente-de-ezequuxedas}{%
\subsection{La respuesta devota pero impenitente de
Ezequías}\label{la-respuesta-devota-pero-impenitente-de-ezequuxedas}}

\bibleverse{19} Entonces Ezequías dijo a Isaías: ``La palabra de Yahvé
que has pronunciado es buena''. Dijo además: ``¿No es así, si la paz y
la verdad estarán en mis días?'' \footnote{\textbf{20:19} 1Sam 3,18}

\bibleverse{20} El resto de los hechos de Ezequías, y toda su fuerza, y
cómo hizo el estanque y el conducto, y cómo introdujo el agua en la
ciudad, ¿no están escritos en el libro de las crónicas de los reyes de
Judá? \bibleverse{21} Ezequías durmió con sus padres, y su hijo Manasés
reinó en su lugar.

\hypertarget{idolatruxeda-manasuxe9s}{%
\subsection{Idolatría manasés}\label{idolatruxeda-manasuxe9s}}

\hypertarget{section-20}{%
\section{21}\label{section-20}}

\bibleverse{1} Manasés tenía doce años cuando comenzó a reinar, y reinó
cincuenta y cinco años en Jerusalén. El nombre de su madre era Hefzibá.
\bibleverse{2} Hizo lo que era malo a los ojos de Yavé, según las
abominaciones de las naciones que Yavé arrojó delante de los hijos de
Israel. \bibleverse{3} Porque volvió a edificar los lugares altos que
Ezequías, su padre, había destruido; levantó altares para Baal e hizo
una Asera, como hizo Ajab, rey de Israel, y adoró a todo el ejército del
cielo, y les sirvió. \footnote{\textbf{21:3} 1Re 16,33} \bibleverse{4}
Edificó altares en la casa de Yahvé, de la cual Yahvé dijo: ``Pondré mi
nombre en Jerusalén''. \footnote{\textbf{21:4} 2Re 21,7} \bibleverse{5}
Construyó altares para todo el ejército del cielo en los dos atrios de
la casa de Yavé. \footnote{\textbf{21:5} 2Re 23,12} \bibleverse{6} Hizo
pasar a su hijo por el fuego, practicó la hechicería, usó encantamientos
y trató con los que tenían espíritus familiares y con los magos. Hizo
mucho mal a los ojos de Yahvé, para provocarlo a la ira. \footnote{\textbf{21:6}
  2Re 16,3} \bibleverse{7} Puso la imagen grabada de Asera que había
hecho en la casa de la que Yahvé dijo a David y a Salomón su hijo: ``En
esta casa y en Jerusalén, que he elegido de entre todas las tribus de
Israel, pondré mi nombre para siempre; \footnote{\textbf{21:7} 1Re 8,29;
  1Re 9,3} \bibleverse{8} no haré que los pies de Israel vuelvan a errar
fuera de la tierra que di a sus padres, con tal de que observen hacer
todo lo que les he mandado, y toda la ley que mi siervo Moisés les
mandó.'' \bibleverse{9} Pero ellos no escucharon, y Manasés los sedujo
para que hicieran lo que es malo, más de lo que hicieron las naciones
que Yahvé destruyó antes de los hijos de Israel.

\hypertarget{la-amenaza-de-dios-a-manasuxe9s-la-crueldad-de-manasuxe9s-y-las-palabras-finales-sobre-uxe9l}{%
\subsection{La amenaza de Dios a Manasés; La crueldad de Manasés y las
palabras finales sobre
él}\label{la-amenaza-de-dios-a-manasuxe9s-la-crueldad-de-manasuxe9s-y-las-palabras-finales-sobre-uxe9l}}

\bibleverse{10} Yahvé habló por medio de sus siervos los profetas,
diciendo: \bibleverse{11} ``Por cuanto Manasés, rey de Judá, ha hecho
estas abominaciones, y ha hecho maldades mayores que las que hicieron
los amorreos que fueron antes de él, y también ha hecho pecar a Judá con
sus ídolos; \bibleverse{12} por tanto, Yahvé, el Dios de Israel, dice:
He aquí que yo traigo tal mal sobre Jerusalén y sobre Judá, que al que
lo oiga le hormiguearán ambos oídos. \footnote{\textbf{21:12} 1Sam 3,11;
  Jer 19,3} \bibleverse{13} Extenderé sobre Jerusalén la línea de
Samaria y la plomada de la casa de Acab; y limpiaré a Jerusalén como se
limpia un plato, limpiándolo y poniéndolo boca abajo. \bibleverse{14}
Desecharé el remanente de mi herencia y lo entregaré en manos de sus
enemigos. Se convertirán en presa y botín de todos sus enemigos,
\bibleverse{15} porque han hecho lo que es malo ante mis ojos, y me han
provocado a la ira desde el día en que sus padres salieron de Egipto,
hasta hoy.'\,''

\bibleverse{16} Además, Manasés derramó mucha sangre inocente, hasta
llenar Jerusalén de un extremo a otro; además de su pecado con el que
hizo pecar a Judá, al hacer lo que era malo a los ojos de Yahvé.
\footnote{\textbf{21:16} 2Re 24,4}

\bibleverse{17} El resto de los hechos de Manasés, y todo lo que hizo, y
su pecado que cometió, ¿no están escritos en el libro de las crónicas de
los reyes de Judá? \bibleverse{18} Manasés durmió con sus padres y fue
enterrado en el jardín de su casa, en el jardín de Uza; y su hijo Amón
reinó en su lugar.

\hypertarget{amon-von-juda}{%
\subsection{Amon von Juda}\label{amon-von-juda}}

\bibleverse{19} Amón tenía veintidós años cuando comenzó a reinar, y
reinó dos años en Jerusalén. Su madre se llamaba Mesulmet, hija de Haruz
de Jotba. \footnote{\textbf{21:19} 2Cró 33,21-22; 2Cró 33,24-25}
\bibleverse{20} Hizo lo que era malo a los ojos de Yahvé, como lo hizo
su padre Manasés. \bibleverse{21} Anduvo en todos los caminos en que
anduvo su padre, y sirvió a los ídolos que su padre servía, y los adoró;
\bibleverse{22} y abandonó a Yavé, el Dios de sus padres, y no anduvo en
el camino de Yavé. \bibleverse{23} Los servidores de Amón conspiraron
contra él y mataron al rey en su propia casa. \footnote{\textbf{21:23}
  2Re 14,19} \bibleverse{24} Pero el pueblo del país mató a todos los
que habían conspirado contra el rey Amón, y el pueblo del país hizo rey
a su hijo Josías en su lugar. \bibleverse{25} Los demás hechos de Amón,
¿no están escritos en el libro de las crónicas de los reyes de Judá?
\bibleverse{26} Fue enterrado en su tumba en el jardín de Uza, y su hijo
Josías reinó en su lugar. \footnote{\textbf{21:26} 2Re 21,18}

\hypertarget{el-rey-josuxedas-encontrar-el-cuxf3digo-legal-y-limpiar-la-adoraciuxf3n}{%
\subsection{El rey Josías; Encontrar el código legal y limpiar la
adoración}\label{el-rey-josuxedas-encontrar-el-cuxf3digo-legal-y-limpiar-la-adoraciuxf3n}}

\hypertarget{section-21}{%
\section{22}\label{section-21}}

\bibleverse{1} Josías tenía ocho años cuando comenzó a reinar, y reinó
treinta y un años en Jerusalén. Su madre se llamaba Yedida, hija de
Adaías de Bozcat. \footnote{\textbf{22:1} 2Cró 34,1-2; 2Cró 34,8-11}
\bibleverse{2} Hizo lo que era justo a los ojos de Yavé, y siguió todo
el camino de David, su padre, sin apartarse ni a la derecha ni a la
izquierda. \footnote{\textbf{22:2} 2Re 18,3; Deut 5,29}

\hypertarget{josuxedas-se-encarga-de-la-reparaciuxf3n-del-templo-informe-sobre-el-descubrimiento-del-cuxf3digo-y-su-primer-efecto}{%
\subsection{Josías se encarga de la reparación del templo; Informe sobre
el descubrimiento del código y su primer
efecto}\label{josuxedas-se-encarga-de-la-reparaciuxf3n-del-templo-informe-sobre-el-descubrimiento-del-cuxf3digo-y-su-primer-efecto}}

\bibleverse{3} En el año dieciocho del rey Josías, el rey envió a Safán,
hijo de Azalías, hijo de Mesulam, el escriba, a la casa de Yavé,
diciendo: \bibleverse{4} ``Sube al sumo sacerdote Hilcías, para que
cuente el dinero que se trae a la casa de Yavé, que los guardianes del
umbral han reunido del pueblo. \bibleverse{5} Que lo entreguen en mano
de los obreros que tienen la vigilancia de la casa de Yavé; y que lo den
a los obreros que están en la casa de Yavé, para reparar los daños de la
casa, \bibleverse{6} a los carpinteros, a los constructores y a los
albañiles, y para comprar madera y piedra cortada para reparar la casa.
\bibleverse{7} Sin embargo, no se les pedirá cuenta del dinero entregado
en sus manos, porque ellos actúan con fidelidad.'' \footnote{\textbf{22:7}
  2Re 12,15}

\bibleverse{8} El sumo sacerdote Hilcías dijo al escriba Safán: ``He
encontrado el libro de la ley en la casa de Yahvé''. Hilcías entregó el
libro a Safán, y éste lo leyó. \bibleverse{9} El escriba Safán fue a ver
al rey y le trajo de nuevo la noticia, diciendo: ``Tus servidores han
vaciado el dinero que se encontró en la casa y lo han entregado en manos
de los obreros que tienen la supervisión de la casa de Yavé.''
\bibleverse{10} El escriba Safán informó al rey diciendo: ``El sacerdote
Hilcías me ha entregado un libro''. Entonces Safán lo leyó ante el rey.

\bibleverse{11} Cuando el rey escuchó las palabras del libro de la ley,
se rasgó las vestiduras. \bibleverse{12} El rey ordenó al sacerdote
Hilcías, a Ajicam hijo de Safán, a Acbor hijo de Micaías, al escriba
Safán y a Asaías, siervo del rey, diciendo: \bibleverse{13} ``Vayan a
consultar a Yavé por mí, por el pueblo y por todo Judá, sobre las
palabras de este libro que se ha encontrado pues es grande la ira del
Señor que se ha encendido contra nosotros, porque nuestros padres no han
escuchado las palabras de este libro, para hacer conforme a todo lo que
está escrito acerca de nosotros.''

\hypertarget{interrogatorio-y-respuesta-de-la-profetisa-hulda}{%
\subsection{Interrogatorio y respuesta de la profetisa
Hulda}\label{interrogatorio-y-respuesta-de-la-profetisa-hulda}}

\bibleverse{14} Entonces el sacerdote Hilcías, Ajicam, Acbor, Safán y
Asaías fueron a ver a la profetisa Hulda, esposa de Salum hijo de Ticva,
hijo de Harhas, guardián del armario (que vivía en Jerusalén en el
segundo barrio), y hablaron con ella. \bibleverse{15} Ella les dijo:
``Yahvé, el Dios de Israel, dice: `Díganle al hombre que los ha enviado
a mí: \bibleverse{16} ``Yahvé dice: `He aquí que yo traigo el mal sobre
este lugar y sobre sus habitantes, incluso todas las palabras del libro
que ha leído el rey de Judá. \bibleverse{17} Porque me han abandonado y
han quemado incienso a otros dioses, para provocarme a la ira con toda
la obra de sus manos, por eso mi ira se encenderá contra este lugar, y
no se apagará.'\,'' \footnote{\textbf{22:17} Deut 31,29; Deut 32,21-23}
\bibleverse{18} Pero al rey de Judá, que te envió a consultar a Yavé,
dile: ``Dice Yavé, el Dios de Israel: `En cuanto a las palabras que has
oído, \bibleverse{19} porque tu corazón se enterneció y te humillaste
ante Yavé cuando oíste lo que hablé contra este lugar y contra sus
habitantes, para que se convirtieran en desolación y maldición, y has
rasgado tus vestiduras y llorado ante mí, yo también te he oído', dice
Yavé. \bibleverse{20} `Por lo tanto, he aquí que te reuniré con tus
padres, y serás reunido a tu tumba en paz. Tus ojos no verán todo el mal
que traeré a este lugar'\,''\,'. Así que llevaron este mensaje al rey.
\footnote{\textbf{22:20} Is 57,1-2}

\hypertarget{josuxedas-concluye-el-nuevo-pacto-de-dios-en-asociaciuxf3n-con-los-ancianos-del-pueblo}{%
\subsection{Josías concluye el nuevo pacto de Dios en asociación con los
ancianos del
pueblo}\label{josuxedas-concluye-el-nuevo-pacto-de-dios-en-asociaciuxf3n-con-los-ancianos-del-pueblo}}

\hypertarget{section-22}{%
\section{23}\label{section-22}}

\bibleverse{1} El rey envió, y se reunieron con él todos los ancianos de
Judá y de Jerusalén. \bibleverse{2} El rey subió a la casa de Yavé, y
con él todos los hombres de Judá y todos los habitantes de Jerusalén,
con los sacerdotes, los profetas y todo el pueblo, tanto el pequeño como
el grande; y leyó en su presencia todas las palabras del libro de la
alianza que se encontraba en la casa de Yavé. \bibleverse{3} El rey se
puso de pie junto a la columna e hizo un pacto ante Yavé de caminar en
pos de Yavé y de guardar sus mandamientos, sus testimonios y sus
estatutos con todo su corazón y toda su alma, para confirmar las
palabras de este pacto que estaban escritas en este libro; y todo el
pueblo estuvo de acuerdo con el pacto. \footnote{\textbf{23:3} 2Re
  11,14; Jos 24,25}

\hypertarget{josuxedas-limpia-el-templo-y-todo-el-culto-puxfablico}{%
\subsection{Josías limpia el templo y todo el culto
público}\label{josuxedas-limpia-el-templo-y-todo-el-culto-puxfablico}}

\bibleverse{4} El rey ordenó al sumo sacerdote Jilquías, a los
sacerdotes del segundo orden y a los guardianes del umbral, que sacaran
del templo de Yavé todos los recipientes que habían sido fabricados para
Baal, para Asera y para todo el ejército del cielo; y los quemó fuera de
Jerusalén, en los campos del Cedrón, y llevó sus cenizas a Betel.
\footnote{\textbf{23:4} 2Re 21,3} \bibleverse{5} Se deshizo de los
sacerdotes idólatras que los reyes de Judá habían ordenado que quemaran
incienso en los lugares altos de las ciudades de Judá y en los lugares
de los alrededores de Jerusalén; los que también quemaban incienso a
Baal, al sol, a la luna, a los planetas y a todo el ejército del cielo.
\bibleverse{6} Sacó la Asera de la casa de Yavé, fuera de Jerusalén,
hasta el arroyo Cedrón, y la quemó en el arroyo Cedrón, la redujo a
polvo y arrojó su polvo sobre las tumbas de la gente común.
\bibleverse{7} Derribó las casas de las prostitutas masculinas que
estaban en la casa de Yavé, donde las mujeres tejían colgaduras para el
Ashera. \footnote{\textbf{23:7} 1Re 14,24} \bibleverse{8} Sacó a todos
los sacerdotes de las ciudades de Judá, y profanó los lugares altos
donde los sacerdotes habían quemado incienso, desde Geba hasta Beerseba;
y derribó los lugares altos de las puertas que estaban a la entrada de
la puerta de Josué, el gobernador de la ciudad, que estaban a la mano
izquierda del hombre en la puerta de la ciudad. \bibleverse{9} Sin
embargo, los sacerdotes de los lugares altos no subían al altar de Yavé
en Jerusalén, sino que comían panes sin levadura entre sus hermanos.
\bibleverse{10} Profanó a Tofet, que está en el valle de los hijos de
Hinom, para que nadie hiciera pasar a su hijo o a su hija por el fuego a
Moloc. \footnote{\textbf{23:10} 2Re 17,17; Lev 18,21} \bibleverse{11}
Quitó los caballos que los reyes de Judá habían dedicado al sol, a la
entrada de la casa de Yavé, junto a la habitación de Natán Melec, el
oficial que estaba en el atrio, y quemó con fuego los carros del sol.
\bibleverse{12} El rey derribó los altares que estaban en el techo de la
habitación superior de Acaz, que habían hecho los reyes de Judá, y los
altares que había hecho Manasés en los dos atrios de la Casa de Yavé, y
los derribó de allí, y arrojó su polvo al arroyo Cedrón. \footnote{\textbf{23:12}
  2Re 16,10-11; 2Re 21,4-5; 2Cró 28,24} \bibleverse{13} El rey profanó
los lugares altos que estaban delante de Jerusalén, a la derecha del
monte de la corrupción, que Salomón, rey de Israel, había edificado para
Astoret, abominación de los sidonios, para Quemos, abominación de Moab,
y para Milcom, abominación de los hijos de Amón. \footnote{\textbf{23:13}
  1Re 11,7} \bibleverse{14} Rompió las columnas, cortó los postes de
Asera y llenó sus lugares con huesos de hombres.

\hypertarget{el-juicio-de-josuxedas-en-betel-y-contra-el-servicio-en-las-alturas-en-samaria}{%
\subsection{El juicio de Josías en Betel y contra el servicio en las
alturas en
Samaria}\label{el-juicio-de-josuxedas-en-betel-y-contra-el-servicio-en-las-alturas-en-samaria}}

\bibleverse{15} Además, el altar que estaba en Betel y el lugar alto que
había hecho Jeroboam hijo de Nabat, el que hizo pecar a Israel, ese
altar y el lugar alto los derribó; y quemó el lugar alto y lo redujo a
polvo, y quemó la Asera. \footnote{\textbf{23:15} 1Re 12,32}
\bibleverse{16} Al volverse Josías, divisó los sepulcros que estaban
allí en el monte; y mandó sacar los huesos de los sepulcros, y los quemó
sobre el altar, y lo profanó, conforme a la palabra de Yahvé que
proclamaba el hombre de Dios que anunciaba estas cosas. \footnote{\textbf{23:16}
  1Re 13,2} \bibleverse{17} Entonces dijo: ``¿Qué monumento es el que
veo?'' Los hombres de la ciudad le dijeron: ``Es la tumba del hombre de
Dios que vino de Judá y proclamó estas cosas que has hecho contra el
altar de Betel''. \footnote{\textbf{23:17} 1Re 13,30}

\bibleverse{18} Dijo: ``¡Dejadle! Que nadie mueva sus huesos''. Así que
dejaron en paz sus huesos, con los del profeta que salió de Samaria.
\bibleverse{19} También tomó Josías todas las casas de los lugares altos
que había en las ciudades de Samaria, que los reyes de Israel habían
hecho para provocar la ira de Yavé, e hizo con ellas lo mismo que había
hecho en Betel. \bibleverse{20} Mató a todos los sacerdotes de los
lugares altos que estaban allí, sobre los altares, y quemó huesos de
hombres sobre ellos; y volvió a Jerusalén.

\hypertarget{celebraciuxf3n-estricta-de-la-pascua}{%
\subsection{Celebración estricta de la
Pascua}\label{celebraciuxf3n-estricta-de-la-pascua}}

\bibleverse{21} El rey ordenó a todo el pueblo diciendo: ``Celebrad la
Pascua a Yahvé, vuestro Dios, como está escrito en este libro de la
Alianza.'' \footnote{\textbf{23:21} Éxod 12,1; 2Cró 35,1-19}
\bibleverse{22} Ciertamente no se celebró una Pascua así desde los días
de los jueces que juzgaban a Israel, ni en todos los días de los reyes
de Israel, ni de los reyes de Judá; \bibleverse{23} pero en el año
dieciocho del rey Josías, se celebró esta Pascua a Yavé en Jerusalén.

\hypertarget{actuar-contra-la-idolatruxeda-en-la-vida-privada-persistencia-de-la-ira-divina-contra-juduxe1}{%
\subsection{Actuar contra la idolatría en la vida privada; Persistencia
de la ira divina contra
Judá}\label{actuar-contra-la-idolatruxeda-en-la-vida-privada-persistencia-de-la-ira-divina-contra-juduxe1}}

\bibleverse{24} Además, Josías eliminó a los que tenían espíritus
familiares, a los magos y a los terafines,\footnote{\textbf{23:24} Los
  terafines eran ídolos domésticos.} y a los ídolos, y todas las
abominaciones que se veían en la tierra de Judá y en Jerusalén, para
confirmar las palabras de la ley que estaban escritas en el libro que el
sacerdote Hilcías encontró en la casa de Yavé. \footnote{\textbf{23:24}
  Lev 20,27; Deut 29,17-18} \bibleverse{25} No hubo antes de él ningún
rey que se convirtiera a Yavé con todo su corazón, con toda su alma y
con todas sus fuerzas, según toda la ley de Moisés; y no hubo ninguno
como él que se levantara después de él. \footnote{\textbf{23:25} 2Re
  18,5} \bibleverse{26} Sin embargo, Yahvé no se apartó del ardor de su
gran ira, con la que ardía su enojo contra Judá, a causa de toda la
provocación con que Manasés lo había provocado. \footnote{\textbf{23:26}
  2Re 21,11-16} \bibleverse{27} Yahvé dijo: ``También quitaré a Judá de
mi vista, como he quitado a Israel; y desecharé esta ciudad que he
elegido, Jerusalén, y la casa de la que dije: `Mi nombre estará allí'.''
\footnote{\textbf{23:27} 2Re 17,18; 1Re 8,29}

\hypertarget{palabra-final-necao-de-egipto-y-la-muerte-de-josuxedas}{%
\subsection{Palabra final; Necao de Egipto y la muerte de
Josías}\label{palabra-final-necao-de-egipto-y-la-muerte-de-josuxedas}}

\bibleverse{28} Los demás hechos de Josías, y todo lo que hizo, ¿no
están escritos en el libro de las crónicas de los reyes de Judá?
\bibleverse{29} En sus días el faraón Necoh, rey de Egipto, subió contra
el rey de Asiria hasta el río Éufrates; y el rey Josías fue contra él,
pero el faraón Necoh lo mató en Meguido cuando lo vio. \bibleverse{30}
Sus servidores lo llevaron muerto en un carro desde Meguido, lo trajeron
a Jerusalén y lo enterraron en su propia tumba. El pueblo del país tomó
a Joacaz, hijo de Josías, lo ungió y lo hizo rey en lugar de su padre.
\footnote{\textbf{23:30} 2Re 9,28}

\hypertarget{los-hijos-de-josuxedas-y-su-nieto-reyes-de-juduxe1-joachuxe2z}{%
\subsection{Los hijos de Josías y su nieto reyes de Judá;
Joachâz}\label{los-hijos-de-josuxedas-y-su-nieto-reyes-de-juduxe1-joachuxe2z}}

\bibleverse{31} Joacaz tenía veintitrés años cuando comenzó a reinar, y
reinó tres meses en Jerusalén. Su madre se llamaba Hamutal, hija de
Jeremías de Libna. \bibleverse{32} Hizo lo que era malo a los ojos de
Yavé, según todo lo que habían hecho sus padres. \bibleverse{33} El
faraón Necoh lo puso en prisión en Ribla, en la tierra de Hamat, para
que no reinara en Jerusalén, y le impuso un tributo de cien talentos de
plata y un talento\footnote{\textbf{23:33} Un talento es de unos 30
  kilogramos o 66 libras o 965 onzas troyanas} de oro. \footnote{\textbf{23:33}
  Ezeq 19,4} \bibleverse{34} El faraón Necoh hizo rey a Eliaquim, hijo
de Josías, en lugar de Josías, su padre, y le cambió el nombre por el de
Joacim; pero se llevó a Joacaz, que vino a Egipto y murió allí.
\bibleverse{35} Joacim entregó la plata y el oro al faraón, pero gravó
la tierra para dar el dinero según el mandato del faraón. Exigió la
plata y el oro del pueblo de la tierra, a cada uno según su valoración,
para dárselo al faraón Necó. \footnote{\textbf{23:35} 2Re 15,20}

\hypertarget{joacim-de-juduxe1}{%
\subsection{Joacim de Judá}\label{joacim-de-juduxe1}}

\bibleverse{36} Joacim tenía veinticinco años cuando comenzó a reinar, y
reinó once años en Jerusalén. Su madre se llamaba Zebida, hija de
Pedaías de Rumah. \bibleverse{37} Hizo lo que era malo a los ojos de
Yahvé, según todo lo que habían hecho sus padres.

\hypertarget{section-23}{%
\section{24}\label{section-23}}

\bibleverse{1} En sus días subió Nabucodonosor, rey de Babilonia, y
Joacim fue su siervo durante tres años. Luego se volvió y se rebeló
contra él. \bibleverse{2} Yahvé envió contra él grupos de caldeos,
grupos de sirios, grupos de moabitas y grupos de hijos de Amón, y los
envió contra Judá para destruirla, según la palabra de Yahvé que habló
por medio de sus siervos los profetas. \bibleverse{3} Ciertamente por
mandato de Yahvé esto vino sobre Judá, para quitarlos de su vista por
los pecados de Manasés, según todo lo que hizo, \footnote{\textbf{24:3}
  2Re 21,10-16; 2Re 23,26-27} \bibleverse{4} y también por la sangre
inocente que derramó; porque llenó a Jerusalén de sangre inocente, y
Yahvé no quiso perdonar. \bibleverse{5} Los demás hechos de Joacim y
todo lo que hizo, ¿no están escritos en el libro de las crónicas de los
reyes de Judá? \bibleverse{6} Y Joacim durmió con sus padres, y su hijo
Joaquín reinó en su lugar.

\bibleverse{7} El rey de Egipto no salió más de su tierra, porque el rey
de Babilonia había tomado, desde el arroyo de Egipto hasta el río
Éufrates, todo lo que pertenecía al rey de Egipto.

\hypertarget{joaquuxedn-de-juduxe1-la-primera-conquista-de-jerusaluxe9n-y-la-primera-ruta-a-babilonia}{%
\subsection{Joaquín de Judá; la primera conquista de Jerusalén y la
primera ruta a
Babilonia}\label{joaquuxedn-de-juduxe1-la-primera-conquista-de-jerusaluxe9n-y-la-primera-ruta-a-babilonia}}

\bibleverse{8} Joaquín tenía dieciocho años cuando comenzó a reinar, y
reinó en Jerusalén tres meses. Su madre se llamaba Nehushta, hija de
Elnatán de Jerusalén. \bibleverse{9} Hizo lo que era malo a los ojos de
Yavé, según todo lo que había hecho su padre. \footnote{\textbf{24:9}
  2Re 23,37} \bibleverse{10} En aquel tiempo los servidores de
Nabucodonosor, rey de Babilonia, subieron a Jerusalén, y la ciudad fue
sitiada. \bibleverse{11} Nabucodonosor, rey de Babilonia, llegó a la
ciudad mientras sus siervos la sitiaban, \bibleverse{12} y Joaquín, rey
de Judá, salió hacia el rey de Babilonia: él, su madre, sus siervos, sus
príncipes y sus oficiales; y el rey de Babilonia lo capturó en el octavo
año de su reinado. \bibleverse{13} Sacó de allí todos los tesoros de la
casa de Yavé y los tesoros de la casa real, y cortó en pedazos todos los
objetos de oro que Salomón, rey de Israel, había hecho en el templo de
Yavé, como había dicho Yavé. \footnote{\textbf{24:13} 2Re 20,17}
\bibleverse{14} Se llevó a toda Jerusalén, a todos los príncipes y a
todos los hombres valientes, hasta diez mil cautivos, y a todos los
artesanos y herreros. No quedó nadie más que los más pobres del país.
\bibleverse{15} Llevó a Joaquín a Babilonia, con la madre del rey, las
mujeres del rey, sus oficiales y los principales hombres del país. Los
llevó en cautiverio desde Jerusalén hasta Babilonia. \footnote{\textbf{24:15}
  Jer 22,26; Jer 24,1; 2Re 25,27} \bibleverse{16} El rey de Babilonia
llevó cautivos a Babilonia a todos los hombres poderosos, siete mil, y a
los artesanos y herreros, mil, todos ellos fuertes y aptos para la
guerra. \bibleverse{17} El rey de Babilonia hizo rey en su lugar a
Matanías, hermano del padre de Joaquín, y le cambió el nombre por el de
Sedequías.

\hypertarget{sedequuxedas-rey-de-juduxe1-fin-del-reino-de-juduxe1}{%
\subsection{Sedequías, rey de Judá; Fin del reino de
Judá}\label{sedequuxedas-rey-de-juduxe1-fin-del-reino-de-juduxe1}}

\bibleverse{18} Sedequías tenía veintiún años cuando comenzó a reinar, y
reinó once años en Jerusalén. Su madre se llamaba Hamutal, hija de
Jeremías de Libna. \footnote{\textbf{24:18} Jer 52,1-3} \bibleverse{19}
Hizo lo que era malo a los ojos de Yavé, según todo lo que había hecho
Joacim. \bibleverse{20} Porque por la ira de Yavé, esto sucedió en
Jerusalén y en Judá, hasta que los expulsó de su presencia. Entonces
Sedequías se rebeló contra el rey de Babilonia. \footnote{\textbf{24:20}
  2Re 23,27}

\hypertarget{los-desperdicios-de-sedequuxedas-asedio-de-jerusaluxe9n-escape-y-captura-del-rey-juzgado-penal-de-ribla}{%
\subsection{Los desperdicios de Sedequías; Asedio de Jerusalén; Escape y
captura del rey; Juzgado penal de
Ribla}\label{los-desperdicios-de-sedequuxedas-asedio-de-jerusaluxe9n-escape-y-captura-del-rey-juzgado-penal-de-ribla}}

\hypertarget{section-24}{%
\section{25}\label{section-24}}

\bibleverse{1} En el noveno año de su reinado, en el décimo mes, a los
diez días del mes, vino Nabucodonosor, rey de Babilonia, él y todo su
ejército, contra Jerusalén, y acampó contra ella; y construyeron contra
ella fortalezas alrededor. \bibleverse{2} Así estuvo sitiada la ciudad
hasta el undécimo año del rey Sedequías. \bibleverse{3} En el noveno día
del cuarto mes, la hambruna fue severa en la ciudad, de modo que no hubo
pan para el pueblo de la tierra. \bibleverse{4} Entonces se abrió una
brecha en la ciudad, y todos los hombres de guerra huyeron de noche por
el camino de la puerta entre las dos murallas, que estaba junto al
jardín del rey (ahora los caldeos estaban contra la ciudad alrededor); y
el rey se fue por el camino del Arabá. \bibleverse{5} Pero el ejército
caldeo persiguió al rey y lo alcanzó en las llanuras de Jericó, y todo
su ejército se dispersó de él. \bibleverse{6} Entonces capturaron al rey
y lo llevaron al rey de Babilonia, a Ribla, y lo juzgaron.
\bibleverse{7} Mataron a los hijos de Sedequías ante sus ojos, luego le
sacaron los ojos, lo ataron con grilletes y lo llevaron a Babilonia.

\hypertarget{conquista-y-destrucciuxf3n-de-jerusaluxe9n-saqueo-e-incendio-del-templo-traslado-de-habitantes-a-babilonia-ejecuciones-en-ribla}{%
\subsection{Conquista y destrucción de Jerusalén; Saqueo e incendio del
templo; Traslado de habitantes a Babilonia; Ejecuciones en
Ribla}\label{conquista-y-destrucciuxf3n-de-jerusaluxe9n-saqueo-e-incendio-del-templo-traslado-de-habitantes-a-babilonia-ejecuciones-en-ribla}}

\bibleverse{8} En el mes quinto, a los siete días del mes, que era el
año decimonoveno del rey Nabucodonosor, rey de Babilonia, llegó a
Jerusalén Nabuzaradán, capitán de la guardia, siervo del rey de
Babilonia. \bibleverse{9} Quemó la casa de Yahvé, la casa del rey y
todas las casas de Jerusalén. Quemó con fuego todas las casas grandes.
\bibleverse{10} Todo el ejército de los caldeos, que estaba con el
capitán de la guardia, derribó los muros alrededor de Jerusalén.
\bibleverse{11} Nabuzaradán, el capitán de la guardia, se llevó cautivo
al resto del pueblo que había quedado en la ciudad y a los que habían
desertado al rey de Babilonia; todo el resto de la multitud.
\bibleverse{12} Pero el capitán de la guardia dejó a algunos de los más
pobres de la tierra para que trabajaran las viñas y los campos.

\bibleverse{13} Los caldeos rompieron las columnas de bronce que había
en la casa de Yavé, así como las bases y el mar de bronce que había en
la casa de Yavé, y llevaron los pedazos de bronce a Babilonia.
\footnote{\textbf{25:13} Jer 27,19-22} \bibleverse{14} Se llevaron las
ollas, las palas, los apagadores, las cucharas y todos los recipientes
de bronce con los que servían. \bibleverse{15} El capitán de la guardia
se llevó las sartenes para el fuego, las palanganas, lo que era de oro,
por oro, y lo que era de plata, por plata. \bibleverse{16} Los dos
pilares, el único mar y las bases, que Salomón había hecho para la casa
de Yavé, el bronce de todos estos recipientes no se pesaba. \footnote{\textbf{25:16}
  1Re 7,15; 1Re 7,23; 1Re 7,27} \bibleverse{17} La altura de la única
columna era de dieciocho codos,\footnote{\textbf{25:17} Un codo es la
  longitud desde la punta del dedo corazón hasta el codo del brazo de un
  hombre, es decir, unas 18 pulgadas o 46 centímetros.} y sobre ella
había un capitel de bronce. La altura del capitel era de tres codos, con
red y granadas en el capitel alrededor, todo de bronce; y la segunda
columna con su red era como éstas.

\bibleverse{18} El capitán de la guardia tomó a Seraías, el sumo
sacerdote, a Sofonías, el segundo sacerdote, y a los tres guardianes del
umbral; \bibleverse{19} y de la ciudad tomó a un oficial que estaba al
frente de los hombres de guerra, y a cinco hombres de los que habían
visto la cara del rey, que se encontraban en la ciudad, y al escriba, al
capitán del ejército que reunía al pueblo del país, y a sesenta hombres
del pueblo del país que se encontraban en la ciudad. \bibleverse{20}
Nabuzaradán, capitán de la guardia, los tomó y los llevó al rey de
Babilonia, a Ribla. \bibleverse{21} El rey de Babilonia los atacó y los
mató en Ribla, en la tierra de Hamat. Y Judá fue llevado cautivo fuera
de su tierra. \footnote{\textbf{25:21} 2Re 23,33}

\hypertarget{gedalja-gobernador-designado-reuxfane-a-los-juduxedos-en-una-colonia-en-mizpa.-despuuxe9s-de-su-asesinato-los-juduxedos-emigran-a-egipto}{%
\subsection{Gedalja, gobernador designado, reúne a los judíos en una
colonia en Mizpa. Después de su asesinato, los judíos emigran a
Egipto}\label{gedalja-gobernador-designado-reuxfane-a-los-juduxedos-en-una-colonia-en-mizpa.-despuuxe9s-de-su-asesinato-los-juduxedos-emigran-a-egipto}}

\bibleverse{22} En cuanto al pueblo que había quedado en la tierra de
Judá y que Nabucodonosor, rey de Babilonia, había dejado como gobernador
a Gedalías, hijo de Ajicam, hijo de Safán. \bibleverse{23} Cuando todos
los capitanes de las fuerzas, ellos y sus hombres, oyeron que el rey de
Babilonia había nombrado gobernador a Gedalías, vinieron a Gedalías a
Mizpa, Ismael hijo de Netanías, Johanán hijo de Carea, Seraías hijo de
Tanhumet el netofita, y Jaazanías hijo del maacateo, ellos y sus
hombres. \bibleverse{24} Gedalías les juró a ellos y a sus hombres, y
les dijo: ``No teman por los siervos de los caldeos. Moren en la tierra
y sirvan al rey de Babilonia, y les irá bien''.

\bibleverse{25} Pero en el séptimo mes vino Ismael, hijo de Netanías,
hijo de Elisama, de la estirpe real, y diez hombres con él, e hirieron a
Gedalías de tal manera que murió, con los judíos y los caldeos que
estaban con él en Mizpa. \bibleverse{26} Todo el pueblo, tanto el
pequeño como el grande, y los capitanes de las fuerzas se levantaron y
fueron a Egipto, porque tenían miedo de los caldeos.

\hypertarget{jojachuxedn-indultado-tras-treinta-y-siete-auxf1os-de-prisiuxf3n}{%
\subsection{Jojachín indultado tras treinta y siete años de
prisión}\label{jojachuxedn-indultado-tras-treinta-y-siete-auxf1os-de-prisiuxf3n}}

\bibleverse{27} En el año treinta y siete del cautiverio de Joaquín, rey
de Judá, en el mes duodécimo, a los veintisiete días del mes,
Evilmerodac, rey de Babilonia, en el año en que comenzó a reinar, liberó
a Joaquín, rey de Judá, de la prisión, \footnote{\textbf{25:27} 2Re
  24,15} \bibleverse{28} y le habló amablemente y puso su trono por
encima del trono de los reyes que estaban con él en Babilonia,
\bibleverse{29} y le cambió sus ropas de prisión. Joaquín comió pan
delante de él continuamente todos los días de su vida; \bibleverse{30} y
para su manutención, se le dio continuamente una ración de parte del
rey, cada día una porción, todos los días de su vida.
