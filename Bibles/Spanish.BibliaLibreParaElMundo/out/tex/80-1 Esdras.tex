\hypertarget{section}{%
\section{1}\label{section}}

\bibleverse{1} Josías celebró la Pascua en Jerusalén para su Señor, y
ofreció la Pascua el día catorce del primer mes, \bibleverse{2} habiendo
puesto a los sacerdotes según sus turnos diarios, vestidos con sus
ropas, en el templo del Señor. \bibleverse{3} Habló a los levitas, los
servidores del templo de Israel, para que se santiguaran ante el Señor,
a fin de colocar el arca sagrada del Señor en la casa que había
construido el rey Salomón, hijo de David. \bibleverse{4} Les dijo: ``Ya
no es necesario que la lleven sobre sus hombros. Ahora, pues, servid al
Señor, vuestro Dios, y servid a su pueblo Israel, y preparaos junto a
las casas de vuestros padres y familiares, \bibleverse{5} según la
escritura del rey David de Israel, y según la magnificencia de Salomón,
su hijo. Poneos en el lugar santo según las divisiones de vuestras
familias de levitas que ministran en presencia de vuestros parientes los
descendientes de Israel. \bibleverse{6} Ofreced la Pascua en orden,
preparad los sacrificios para vuestra parentela y celebrad la Pascua
según el mandamiento del Señor, que fue dado a Moisés.

\bibleverse{7} Al pueblo presente, Josías le dio treinta mil corderos y
cabritos, y tres mil terneros. Estas cosas fueron dadas de los bienes
del rey, como él lo había prometido, al pueblo y a los sacerdotes y
levitas. \bibleverse{8} Helkias, Zacarías y Esyelus, los jefes del
templo, dieron a los sacerdotes para la Pascua dos mil seiscientas
ovejas y trescientos terneros. \bibleverse{9} Jeconias, Samaias,
Natanael su hermano, Sabias, Ochielus y Joram, capitanes de millares,
dieron a los levitas para la Pascua cinco mil ovejas y setecientos
terneros.

\bibleverse{10} Una vez hechas estas cosas, los sacerdotes y los
levitas, con los panes sin levadura, se ponían en pie en el orden
adecuado según la parentela, \bibleverse{11} y según las distintas
divisiones por casas paternas, ante el pueblo, para ofrecer al Señor
como está escrito en el libro de Moisés. Esto lo hacían por la mañana.
\bibleverse{12} Asaban al fuego el cordero de la Pascua, tal como estaba
previsto. Hirvieron los sacrificios en las vasijas de bronce y en los
calderos de olor agradable, \bibleverse{13} y los pusieron delante de
todo el pueblo. Después prepararon para ellos y para sus parientes los
sacerdotes, los hijos de Aarón. \bibleverse{14} Los sacerdotes ofrecían
la grasa hasta la noche. Los levitas preparaban para sí y para su
parentela a los sacerdotes, hijos de Aarón. \bibleverse{15} También los
cantores sagrados, los hijos de Asaf, estaban en su orden, según la
designación de David: Asaf, Zacarías y Edino, que representaban al rey.
\bibleverse{16} Además, los porteros estaban en cada puerta. Ninguno
tenía que apartarse de sus deberes diarios, pues sus parientes los
levitas les preparaban.

\bibleverse{17} Así se cumplió en aquel día lo que correspondía a los
sacrificios del Señor, celebrando la Pascua, \bibleverse{18} y
ofreciendo sacrificios en el altar del Señor, según el mandato del rey
Josías. \bibleverse{19} Los hijos de Israel que estaban presentes en
aquel tiempo celebraron la Pascua y la fiesta de los panes sin levadura
durante siete días. \bibleverse{20} No se había celebrado una Pascua así
en Israel desde los tiempos del profeta Samuel. \bibleverse{21} En
efecto, ninguno de los reyes de Israel celebró una Pascua como la que
Josías, con los sacerdotes, los levitas y los judíos, celebró con todo
Israel presente en su morada de Jerusalén. \bibleverse{22} Esta Pascua
se celebró en el año dieciocho del reinado de Josías. \bibleverse{23}
Las obras de Josías fueron rectas ante su Señor, con un corazón lleno de
piedad. \bibleverse{24} Además, en tiempos pasados se escribieron las
cosas que sucedieron en sus días, acerca de los que pecaron e hicieron
maldad contra el Señor más que ningún otro pueblo o reino, y cómo lo
contrariaron en extremo, de modo que las palabras del Señor se
confirmaron contra Israel.

\bibleverse{25} Después de todos estos actos de Josías, sucedió que el
faraón, rey de Egipto, vino a hacer la guerra a Carquemis, en el
Éufrates, y Josías salió contra él. \bibleverse{26} Pero el rey de
Egipto le envió a decir: ``¿Qué tengo yo que ver contigo, oh rey de
Judea? \bibleverse{27} No fui enviado por el Señor Dios contra ti, pues
mi guerra es contra el Éufrates. Ahora el Señor está conmigo, sí, el
Señor está conmigo apresurándome a avanzar. Apártate de mí y no te
opongas al Señor''.

\bibleverse{28} Sin embargo, Josías no regresó a su carro, sino que
trató de luchar con él, sin tener en cuenta las palabras del profeta
Jeremías de boca del Señor, \bibleverse{29} sino que se unió a la
batalla con él en la llanura de Meguido, y los comandantes descendieron
contra el rey Josías. \bibleverse{30} Entonces el rey dijo a sus
siervos: ``¡Sacadme de la batalla, porque estoy muy débil!''
Inmediatamente sus servidores lo sacaron del ejército. \bibleverse{31}
Luego subió a su segundo carro. Después de ser llevado de vuelta a
Jerusalén, murió y fue enterrado en la tumba de sus antepasados.
\bibleverse{32} Toda Judea se lamentó por Josías. El profeta Jeremías se
lamentó por Josías, y los jefes con las mujeres se lamentaron por él
hasta el día de hoy. Esto fue dado por ordenanza para que se hiciera
continuamente en toda la nación de Israel. \bibleverse{33} Estas cosas
están escritas en el libro de las historias de los reyes de Judea, y
cada uno de los actos que hizo Josías, y su gloria, y su entendimiento
en la ley del Señor, y las cosas que había hecho antes, y las cosas que
ahora se cuentan, están relatadas en el libro de los reyes de Israel y
de Judá.

\bibleverse{34} El pueblo tomó a Joacaz, hijo de Josías, y lo hizo rey
en lugar de su padre Josías, cuando tenía veintitrés años.
\bibleverse{35} Reinó en Judá y en Jerusalén durante tres meses. Luego
el rey de Egipto lo destituyó de su reinado en Jerusalén.
\bibleverse{36} Estableció un impuesto sobre el pueblo de cien talentos
de plata y un talento de oro. \bibleverse{37} El rey de Egipto también
nombró al rey Joaquín, su hermano, rey de Judea y Jerusalén.
\bibleverse{38} Joakim encarceló a los nobles y apresó a su hermano
Zarakes, y lo sacó de Egipto.

\bibleverse{39} Joakim tenía veinticinco años cuando comenzó a reinar en
Judea y Jerusalén. Hizo lo que era malo a los ojos del Señor.
\bibleverse{40} El rey Nabucodonosor de Babilonia subió contra él, lo
ató con una cadena de bronce y lo llevó a Babilonia. \bibleverse{41}
Nabucodonosor también tomó algunos de los vasos sagrados del Señor, se
los llevó y los guardó en su propio templo en Babilonia. \bibleverse{42}
Pero las cosas que se cuentan de él, de su impureza y de su falta de
ética, están escritas en las crónicas de los reyes. \bibleverse{43}
Luego reinó en su lugar Joaquín, su hijo. Cuando fue nombrado rey, tenía
dieciocho años. \bibleverse{44} Reinó tres meses y diez días en
Jerusalén. Hizo lo que era malo ante el Señor.

\bibleverse{45} Así que al cabo de un año Nabucodonosor envió e hizo que
lo llevaran a Babilonia con los objetos sagrados del Señor,
\bibleverse{46} y nombró a Sedecías rey de Judea y Jerusalén cuando
tenía veintiún años. Reinó once años. \bibleverse{47} Él también hizo lo
que era malo a los ojos del Señor, y no hizo caso de las palabras que
había dicho el profeta Jeremías de boca del Señor. \bibleverse{48}
Después de que el rey Nabucodonosor le hizo jurar por el nombre del
Señor, rompió su juramento y se rebeló. Endureciendo su cuello y su
corazón, transgredió las leyes del Señor, el Dios de Israel.
\bibleverse{49} Además, los gobernantes del pueblo y de los sacerdotes
hicieron muchas maldades, excediendo todas las contaminaciones de todas
las naciones, y profanaron el templo del Señor, que estaba santificado
en Jerusalén. \bibleverse{50} El Dios de sus antepasados envió por medio
de su mensajero a llamarlos para que volvieran, porque tenía compasión
de ellos y de su morada. \bibleverse{51} Pero ellos se burlaron de sus
mensajeros. En el día en que el Señor habló, se burlaron de sus profetas
\bibleverse{52} hasta que él, enojado con su pueblo por su gran
impiedad, mandó traer contra ellos a los reyes de los caldeos.
\bibleverse{53} Mataron a sus jóvenes a espada alrededor de su santo
templo, y no perdonaron ni a jóvenes ni a mujeres, ni a ancianos ni a
niños; pero él los entregó a todos en sus manos. \bibleverse{54} Tomaron
todos los vasos sagrados del Señor, grandes y pequeños, con los cofres
del arca del Señor y los tesoros del rey, y los llevaron a Babilonia.
\bibleverse{55} Quemaron la casa del Señor, derribaron los muros de
Jerusalén y quemaron sus torres con fuego. \bibleverse{56} En cuanto a
sus cosas gloriosas, no se detuvieron hasta que las redujeron a la nada.
Llevó a Babilonia a la gente que no fue asesinada con la espada.
\bibleverse{57} Fueron siervos de él y de sus hijos hasta que reinaron
los persas, para que se cumpliera la palabra del Señor por boca de
Jeremías: \bibleverse{58} ``Hasta que la tierra haya disfrutado de sus
sábados, todo el tiempo de su desolación guardará sábados, para cumplir
setenta años.

\hypertarget{section-1}{%
\section{2}\label{section-1}}

\bibleverse{1} En el primer año del rey Ciro de los persas, para que se
cumpliera la palabra del Señor por boca de Jeremías, \bibleverse{2} el
Señor despertó el espíritu del rey Ciro de los persas, y éste hizo una
proclamación por todo su reino, y también por escrito, \bibleverse{3}
diciendo: ``Dice Ciro, rey de los persas: El Señor de Israel, el Señor
Altísimo, me ha hecho rey de todo el mundo, \bibleverse{4} y me ha
mandado construirle una casa en Jerusalén, que está en Judea.
\bibleverse{5} Por tanto, si hay alguno de vosotros que sea de su
pueblo, que el Señor, su Señor, esté con él, y que suba a Jerusalén que
está en Judea, y edifique la casa del Señor de Israel. Él es el Señor
que habita en Jerusalén. \bibleverse{6} Por lo tanto, de los que habitan
en diversos lugares, que los que están en su propio lugar ayuden cada
uno con oro, con plata, \bibleverse{7} con regalos, con caballos y
ganado, además de las otras cosas que han sido añadidas por voto para el
templo del Señor que está en Jerusalén.

\bibleverse{8} Entonces se levantaron los jefes de las familias de Judá
y de la tribu de Benjamín, con los sacerdotes, los levitas y todos
aquellos a quienes el Señor había movido el espíritu para que subieran a
construir la casa del Señor que está en Jerusalén. \bibleverse{9} Los
que vivían alrededor de ellos les ayudaron en todo, con plata y oro, con
caballos y ganado, y con muchísimos dones que fueron ofrecidos por un
gran número de personas cuyo espíritu estaba tan conmovido.

\bibleverse{10} El rey Ciro también sacó los vasos sagrados del Señor,
que Nabucodonosor había llevado de Jerusalén y había guardado en su
templo de los ídolos. \bibleverse{11} Cuando el rey Ciro de los persas
los sacó, los entregó a Mitrídates, su tesorero, \bibleverse{12} y por
él fueron entregados a Sanabassar, gobernador de Judea. \bibleverse{13}
Este fue el número de ellos: mil copas de oro, mil copas de plata,
veintinueve incensarios de plata, treinta copas de oro, dos mil
cuatrocientas diez copas de plata y otros mil vasos. \bibleverse{14} Así
fueron subidos todos los vasos de oro y de plata, cinco mil
cuatrocientos setenta y nueve, \bibleverse{15} y fueron llevados por
Sanabassar, junto con los exiliados que regresaban, de Babilonia a
Jerusalén.

\bibleverse{16} En tiempos del rey Artajerjes de los persas, Belemus,
Mitrídates, Tabelio, Rathumus, Beeltethmus y Samelio el escriba, con sus
otros asociados, que vivían en Samaria y otros lugares, le escribieron
contra los que vivían en Judea y Jerusalén la siguiente carta:

\bibleverse{17} ``Al rey Artajerjes, nuestro Señor, de parte de tus
siervos, Rathumus el registrador, Samellius el escriba, y el resto de su
consejo, y los jueces que están en Coelesyria y Fenicia: \bibleverse{18}
Sepa ahora nuestro señor el rey, que los judíos que han subido de ti a
nosotros, habiendo llegado a Jerusalén, están construyendo esa ciudad
rebelde y perversa, y están reparando sus plazas y muros, y están
poniendo los cimientos de un templo. \bibleverse{19} Ahora bien, si se
construye esta ciudad y se terminan sus muros, no sólo se negarán a dar
tributo, sino que incluso se levantarán contra los reyes.
\bibleverse{20} Puesto que las cosas relativas al templo están ya en
marcha, nos parece oportuno no descuidar tal asunto, \bibleverse{21}
sino hablar a nuestro señor el rey, con el fin de que, si le parece
bien, se busque en los libros de sus antepasados. \bibleverse{22}
Encontrarás en las crónicas lo que está escrito acerca de estas cosas, y
comprenderás que esa ciudad era rebelde, que perturbaba a los reyes y a
las ciudades, \bibleverse{23} y que los judíos eran rebeldes y seguían
iniciando guerras allí en el pasado. Por esta causa, esta ciudad fue
asolada. \bibleverse{24} Por lo tanto, ahora te declaramos, oh señor
rey, que si esta ciudad se construye de nuevo y se levantan sus
murallas, desde entonces no tendrás paso a Coelesyria y Fenicia.''

\bibleverse{25} Entonces el rey volvió a escribir a Rathumus, el
registrador, a Beeltethmus, a Samellius, el escriba, y al resto de sus
asociados que vivían en Samaria, Siria y Fenicia, lo siguiente:

\bibleverse{26} ``He leído la carta que me has enviado. Por lo tanto, he
ordenado que se haga una búsqueda, y se ha encontrado que esa ciudad
desde tiempos antiguos ha luchado contra los reyes, \bibleverse{27} y
que los hombres eran dados a la rebelión y a la guerra en ella, y que
había reyes poderosos y feroces en Jerusalén, que reinaban y exigían
tributo en Coelesyria y Fenicia. \bibleverse{28} Ahora, pues, he
ordenado que se impida a esos hombres edificar la ciudad, y que se cuide
de que no se haga nada en contra de esta orden, \bibleverse{29} y que
esas perversas acciones no prosigan para molestia de los reyes.''
\bibleverse{30} Entonces el rey Artajerjes, una vez leídas sus cartas,
Rathumus y el escriba Samelio, y el resto de sus asociados, se
dirigieron apresuradamente a Jerusalén con caballería y una multitud de
gente en formación de batalla, y comenzaron a impedir a los
constructores. Así que la construcción del templo de Jerusalén cesó
hasta el segundo año del reinado del rey Darío de los persas.

\hypertarget{section-2}{%
\section{3}\label{section-2}}

\bibleverse{1} El rey Darío hizo un gran banquete para todos sus
súbditos, para todos los nacidos en su casa, para todos los príncipes de
Media y de Persia, \bibleverse{2} y para todos los gobernadores locales
y capitanes y gobernadores que estaban bajo su mando, desde la India
hasta Etiopía, en las ciento veintisiete provincias. \bibleverse{3}
Comieron y bebieron, y cuando estuvieron satisfechos se fueron a sus
casas. Entonces el rey Darío entró en su alcoba y durmió, pero se
despertó de su sueño.

\bibleverse{4} Entonces los tres jóvenes de la guardia que custodiaban
al rey hablaron entre sí: \bibleverse{5} ``Que cada uno de nosotros
declare lo que es más fuerte. El rey Darío dará a aquel cuya declaración
parezca más sabia que las demás grandes regalos y grandes honores en
señal de victoria. \bibleverse{6} Se vestirá de púrpura, beberá en copas
de oro, dormirá en un lecho de oro y tendrá un carro con bridas de oro,
un turbante de lino fino y una cadena al cuello. \bibleverse{7} Se
sentará junto a Darío por su sabiduría y se le llamará primo de Darío''.

\bibleverse{8} Entonces cada uno de ellos escribió su sentencia, la
selló y la puso bajo la almohada del rey Darío, \bibleverse{9} y dijo:
``Cuando el rey se despierte, alguien le entregará el escrito. A quien
el rey y los tres príncipes de Persia juzguen que su sentencia es la más
sabia, se le dará la victoria, como está escrito.'' \bibleverse{10} El
primero escribió: ``El vino es el más fuerte''. \bibleverse{11} El
segundo escribió: ``El rey es el más fuerte''. \bibleverse{12} El
tercero escribió: ``Las mujeres son más fuertes, pero sobre todo la
Verdad es la vencedora''.

\bibleverse{13} Cuando el rey se despertó, tomaron el escrito y se lo
dieron, y él lo leyó. \bibleverse{14} Enviando, llamó a todos los
príncipes de Persia y de Media, a los gobernadores locales, a los
capitanes, a los gobernadores y a los oficiales principales
\bibleverse{15} y se sentó en la sede real del juicio; y el escrito fue
leído ante ellos. \bibleverse{16} Dijo: ``Llamad a los jóvenes, y ellos
explicarán sus propias sentencias. Así que los llamaron y entraron.
\bibleverse{17} Les dijeron: ``Explicad lo que habéis escrito''.

Entonces el primero, que había hablado de la fuerza del vino, comenzó
\bibleverse{18} y dijo esto ``¡Oh, señores, qué fuerte es el vino! Hace
que todos los hombres que lo beben se extravíen. \bibleverse{19} Hace
que la mente del rey y la del huérfano sean iguales, así como la del
siervo y la del libre, la del pobre y la del rico. \bibleverse{20}
Convierte todo pensamiento en alegría y gozo, de modo que el hombre no
se acuerda de las penas ni de las deudas. \bibleverse{21} Hace que todo
corazón se enriquezca, de modo que el hombre no se acuerda ni del rey ni
del gobernador local. Hace que la gente diga cosas en grandes
cantidades. \bibleverse{22} Cuando están en sus copas, se olvidan del
amor a los amigos y a la parentela, y no tardan en desenvainar la
espada. \bibleverse{23} Pero cuando despiertan de su vino, no recuerdan
lo que han hecho. \bibleverse{24} Oh, señores, ¿no es el vino el más
fuerte, ya que obliga a la gente a hacer esto?'' Y cuando hubo dicho
esto, dejó de hablar.

\hypertarget{section-3}{%
\section{4}\label{section-3}}

\bibleverse{1} Entonces el segundo, que había hablado de la fuerza del
rey, comenzó a decir: \bibleverse{2} ``Oh, señores, ¿no sobresalen en
fuerza los hombres que dominan el mar y la tierra, y todas las cosas que
hay en ellos? \bibleverse{3} Pero, sin embargo, el rey es más fuerte. Él
es su señor y tiene dominio sobre ellos. En todo lo que les manda, le
obedecen. \bibleverse{4} Si les dice que hagan la guerra los unos contra
los otros, lo hacen. Si los envía contra los enemigos, van y conquistan
montañas, muros y torres. \bibleverse{5} Matan y son matados, y no
desobedecen el mandato del rey. Si ganan la victoria, le llevan todo al
rey: todo el botín y todo lo demás. \bibleverse{6} Asimismo, los que no
son soldados y no tienen nada que ver con las guerras, sino que
cultivan, cuando han vuelto a cosechar lo que habían sembrado, traen una
parte al rey y se obligan unos a otros a pagar tributo al rey.
\bibleverse{7} ¡Él es sólo un hombre! Si manda matar, matan. Si les
ordena que perdonen, perdonan. \bibleverse{8} Si les ordena golpear,
golpean. Si les manda asolar, asolan. Si les manda construir,
construyen. \bibleverse{9} Si les manda talar, talan. Si les manda
plantar, plantan. \bibleverse{10} Así todo su pueblo y sus ejércitos le
obedecen. Además, se acuesta, come y bebe, y descansa; \bibleverse{11} y
éstos vigilan a su alrededor. Ninguno de ellos puede apartarse y hacer
sus propios negocios. No le desobedecen en nada. \bibleverse{12} Oh
señores, ¿cómo no va a ser el rey el más fuerte, viendo que se le
obedece así?'' Entonces dejó de hablar.

\bibleverse{13} Entonces el tercero, que había hablado de las mujeres y
de la verdad, (éste era Zorobabel) comenzó a hablar: \bibleverse{14}
``Oh señores, ¿no es grande el rey, y los hombres son muchos, y no es
fuerte el vino? ¿Quién es, pues, el que los gobierna, o tiene el señorío
sobre ellos? ¿Acaso no son mujeres? \bibleverse{15} Las mujeres han dado
a luz al rey y a toda la gente que gobierna sobre el mar y la tierra.
\bibleverse{16} Ellos vinieron de las mujeres. Las mujeres nutrieron a
los que plantaron las viñas, de donde sale el vino. \bibleverse{17} Las
mujeres también confeccionan prendas de vestir para los hombres. Éstas
dan gloria a los hombres. Sin las mujeres, los hombres no pueden
existir. \bibleverse{18} Sí, y si los hombres han reunido oro y plata y
cualquier otra cosa hermosa, y ven a una mujer que es hermosa en
apariencia y belleza, \bibleverse{19} dejan todas esas cosas y se quedan
boquiabiertos al verla, y con la boca abierta la miran. Todos tienen más
deseo de ella que de oro, o de plata, o de cualquier otra cosa hermosa.
\bibleverse{20} Un hombre deja a su propio padre que lo crió, deja su
propio país y se une a su esposa. \bibleverse{21} Con su mujer termina
sus días, sin pensar en su padre, en su madre ni en su patria.
\bibleverse{22} En esto también debéis saber que las mujeres os dominan.
¿No trabajáis y os afanáis, y lo lleváis todo para dárselo a las
mujeres? \bibleverse{23} Sí, el hombre toma su espada y sale a viajar, a
robar, a hurtar y a navegar por el mar y por los ríos. \bibleverse{24}
Ve un león y camina en la oscuridad. Cuando ha robado, saqueado y
desvalijado, se lo lleva a la mujer que ama. \bibleverse{25} Por eso el
hombre ama a su mujer más que a su padre o a su madre. \bibleverse{26}
Sí, hay muchos que han perdido la cabeza por las mujeres y se han hecho
esclavos por ellas. \bibleverse{27} También muchos han perecido, han
tropezado y han pecado por causa de las mujeres. \bibleverse{28} Ahora
bien, ¿no me creéis? ¿No es grande el rey en su poder? ¿No temen todas
las regiones tocarlo? \bibleverse{29} Sin embargo, lo vi a él y a Apame,
la concubina del rey, hija del ilustre Barticus, sentada a la derecha
del rey, \bibleverse{30} y tomando la corona de la cabeza del rey, la
puso sobre su propia cabeza. Sí, ella golpeó al rey con su mano
izquierda. \bibleverse{31} Al ver esto, el rey se quedó boquiabierto y
la miró con la boca abierta. Si ella le sonríe, él se ríe. Pero si ella
se disgusta con él, la halaga, para que se reconcilie de nuevo con él.
\bibleverse{32} Oh, señores, ¿cómo no puede ser que las mujeres sean
fuertes, viendo que hacen esto?''

\bibleverse{33} Entonces el rey y los nobles se miraron entre sí.
Entonces él comenzó a hablar sobre la verdad. \bibleverse{34} ``Oh,
señores, ¿no son fuertes las mujeres? La tierra es grande. El cielo es
alto. El sol es rápido en su curso, pues da vueltas alrededor del cielo
y vuelve a su curso en un día. \bibleverse{35} ¿No es grande el que hace
estas cosas? Por eso la verdad es grande, y más fuerte que todas las
cosas. \bibleverse{36} Toda la tierra invoca la verdad, y el cielo la
bendice. Todas las obras se estremecen y tiemblan, pero con la verdad no
hay nada injusto. \bibleverse{37} El vino es injusto. El rey es injusto.
Las mujeres son injustas. Todos los hijos de los hombres son injustos, y
todas sus obras son injustas. No hay verdad en ellos. También ellos
perecerán en su injusticia. \bibleverse{38} Pero la verdad permanece y
es fuerte para siempre. La verdad vive y vence para siempre.
\bibleverse{39} Con la verdad no hay parcialidad hacia las personas o
las recompensas, sino que la verdad hace las cosas que son justas, en
lugar de cualquier cosa injusta o malvada. Todos los hombres aprueban
las obras de la verdad. \bibleverse{40} En el juicio de la verdad no hay
ninguna injusticia. La verdad es la fuerza, el reino, el poder y la
majestad de todos los tiempos. Bendito sea el Dios de la verdad''.

\bibleverse{41} Con esto, dejó de hablar. Entonces todo el pueblo gritó
y dijo: ``¡Grande es la verdad, y fuerte sobre todas las cosas!''

\bibleverse{42} Entonces el rey le dijo: ``Pide lo que desees, incluso
más de lo que está señalado por escrito, y te lo concederemos, porque
eres hallado el más sabio. Te sentarás a mi lado y serás llamado mi
primo''.

\bibleverse{43} Entonces dijo al rey: ``Acuérdate de tu voto, que
hiciste para edificar Jerusalén, el día en que llegaste a tu reino,
\bibleverse{44} y para devolver todos los utensilios que fueron sacados
de Jerusalén, que Ciro apartó cuando juró destruir a Babilonia, y juró
devolverlos allí. \bibleverse{45} También prometiste construir el templo
que los edomitas quemaron cuando Judea fue desolada por los caldeos.
\bibleverse{46} Ahora, oh señor rey, esto es lo que pido y lo que deseo
de ti, y esta es la generosidad principesca que puede proceder de ti: Te
pido, pues, que cumplas el voto cuyo cumplimiento has prometido al Rey
del Cielo con tu propia boca.''

\bibleverse{47} Entonces el rey Darío se levantó, lo besó y escribió
cartas para él a todos los tesoreros y gobernadores y capitanes y
gobernadores locales, para que lo trajeran sano y salvo a él y a todos
los que subieran con él a edificar Jerusalén. \bibleverse{48} Escribió
también cartas a todos los gobernadores que estaban en Coelesiria y
Fenicia, y a los que estaban en Libano, para que trajeran madera de
cedro de Libano a Jerusalén, y para que le ayudaran a edificar la
ciudad. \bibleverse{49} Además, escribió para todos los judíos que
salieran de su reino hacia Judea en relación con su libertad, que ningún
oficial, ningún gobernador, ningún gobernador local, ni tesorero,
entrara por la fuerza en sus puertas, \bibleverse{50} y que todo el país
que ocuparan fuera libre para ellos sin tributo, y que los edomitas
entregaran las aldeas de los judíos que tenían en ese momento,
\bibleverse{51} y que se dieran veinte talentos anuales para la
construcción del templo, hasta el momento en que se construyera,
\bibleverse{52} y otros diez talentos anuales para los holocaustos que
se presentarían sobre el altar cada día, ya que tenían el mandamiento de
hacer diecisiete ofrendas, \bibleverse{53} y que todos los que vinieran
de Babilonia para construir la ciudad tuvieran su libertad, ellos y sus
descendientes, y todos los sacerdotes que vinieran. \bibleverse{54}
También escribió que se les diera su sustento y las vestimentas
sacerdotales con las que ejercen su ministerio. \bibleverse{55} Para los
levitas escribió que se les diera su sustento hasta el día en que se
terminara la casa y se edificara Jerusalén. \bibleverse{56} Ordenó que
se diera tierra y salario a todos los que custodiaban la ciudad.
\bibleverse{57} También mandó traer de Babilonia todos los utensilios
que Ciro había apartado, y todo lo que Ciro había ordenado que se
hiciera y se enviara a Jerusalén.

\bibleverse{58} Cuando este joven salió, levantó su rostro al cielo,
hacia Jerusalén, y alabó al Rey del cielo, \bibleverse{59} y dijo: ``De
ti viene la victoria. De ti viene la sabiduría. Tuya es la gloria, y yo
soy tu servidor. \bibleverse{60} Bendito seas, que me has dado la
sabiduría. Te doy gracias, Señor de nuestros padres. \bibleverse{61} Así
que tomó las cartas, salió, llegó a Babilonia y lo contó a toda su
parentela. \bibleverse{62} Alabaron al Dios de sus antepasados, porque
les había dado libertad y libertad \bibleverse{63} para subir y
construir Jerusalén y el templo que lleva su nombre. Hicieron una fiesta
con instrumentos de música y alegría durante siete días.

\hypertarget{section-4}{%
\section{5}\label{section-4}}

\bibleverse{1} Después de esto, los jefes de las casas paternas fueron
elegidos para subir según sus tribus, con sus esposas, hijos e hijas,
con sus siervos y siervas, y sus ganados. \bibleverse{2} Darío envió con
ellos mil soldados de caballería para llevarlos de vuelta a Jerusalén
con paz, con instrumentos musicales, tambores y flautas. \bibleverse{3}
Toda su parentela se alegraba, y los hizo subir con ellos.

\bibleverse{4} Estos son los nombres de los hombres que subieron, según
sus familias entre sus tribus, según sus diversas divisiones.
\bibleverse{5} Los sacerdotes, hijos de Finees, hijos de Aarón: Jesús,
hijo de Josedec, hijo de Saraías, y Joaquín, hijo de Zorobabel, hijo de
Salatiel, de la casa de David, del linaje de Fares, de la tribu de Judá,
\bibleverse{6} que hablaron palabras sabias ante Darío, rey de Persia,
en el segundo año de su reinado, en el mes de Nisán, que es el primer
mes.

\bibleverse{7} Estos son los de Judea que subieron del cautiverio, donde
vivían como extranjeros, a quienes Nabucodonosor, rey de Babilonia,
había llevado a Babilonia. \bibleverse{8} Volvieron a Jerusalén y a las
otras partes de Judea, cada uno a su ciudad, los que vinieron con
Zorobabel, con Jesús, Nehemías, Zaraias, Resaias, Eneneus, Mardocheus,
Beelsarus, Aspharsus, Reelias, Roimus y Baana, sus líderes.

\bibleverse{9} El número de ellos de la nación y sus jefes los hijos de
Foros, dos mil ciento setenta y dos; los hijos de Saphat, cuatrocientos
setenta y dos; \bibleverse{10} los hijos de Ares, setecientos cincuenta
y seis; \bibleverse{11} los hijos de Phaath Moab, de los hijos de Jesús
y Joab, dos mil ochocientos doce; \bibleverse{12} los hijos de Elam, mil
doscientos cincuenta y cuatro; los hijos de Zathui, novecientos cuarenta
y cinco; los hijos de Chorbe, setecientos cinco; los hijos de Bani,
seiscientos cuarenta y ocho; \bibleverse{13} los hijos de Bebai,
seiscientos veintitrés; los hijos de Astad, mil trescientos veintidós;
\bibleverse{14} los hijos de Adonikam, seiscientos sesenta y siete; los
hijos de Bagoi, dos mil sesenta y seis; los hijos de Adinu,
cuatrocientos cincuenta y cuatro; \bibleverse{15} los hijos de Ater, de
Ezequías, noventa y dos; los hijos de Kilan y Azetas, sesenta y siete;
los hijos de Azaru, cuatrocientos treinta y dos; \bibleverse{16} los
hijos de Annis, ciento uno; los hijos de Arom, los hijos de Bassai,
trescientos veintitrés; los hijos de Arsifurit, ciento doce;
\bibleverse{17} los hijos de Baitero, tres mil cinco; los hijos de
Betlomón, ciento veintitrés; \bibleverse{18} los de Netofas, cincuenta y
cinco; los de Anatot, ciento cincuenta y ocho; los de Betasmot, cuarenta
y dos; \bibleverse{19} los de Kariathiarius, veinticinco: los de Caphira
y Beroth, setecientos cuarenta y tres; \bibleverse{20} los de Chadiasai
y Ammidioi, cuatrocientos veintidós; los de Kirama y Gabbe, seiscientos
veintiuno; \bibleverse{21} los de Macalon, ciento veintidós; los de
Betolion, cincuenta y dos; los hijos de Niphis, ciento cincuenta y seis;
\bibleverse{22} los hijos de Calamolalus y Onus, setecientos
veinticinco; los hijos de Jerechu, trescientos cuarenta y cinco;
\bibleverse{23} y los hijos de Sanaas, tres mil trescientos treinta.

\bibleverse{24} Los sacerdotes: los hijos de Jeddu, hijo de Jesús, entre
los hijos de Sanasib, novecientos setenta y dos; los hijos de Emmeruth,
mil cincuenta y dos; \bibleverse{25} los hijos de Phassurus, mil
doscientos cuarenta y siete; y los hijos de Charme, mil diecisiete.
\bibleverse{26} Los levitas: los hijos de Jesús, Kadmiel, Bannas y
Sudias, setenta y cuatro. \bibleverse{27} Los cantores sagrados: los
hijos de Asaf, ciento veintiocho. \bibleverse{28} Los porteros: los
hijos de Salum, los hijos de Atar, los hijos de Tolman, los hijos de
Dacubi, los hijos de Ateta, los hijos de Sabi, en total ciento treinta y
nueve.

\bibleverse{29} Los servidores del templo: los hijos de Esaú, los hijos
de Asifa, los hijos de Tabaot, los hijos de Keras, los hijos de Sua, los
hijos de Phaleas, los hijos de Labana, los hijos de Aggaba.
\bibleverse{30} los hijos de Acud, los hijos de Uta, los hijos de Ketab,
los hijos de Accaba, los hijos de Subai, los hijos de Anan, los hijos de
Cathua, los hijos de Geddur, \bibleverse{31} los hijos de Jairus, los
hijos de Daisan, los hijos de Noeba, los hijos de Chaseba, los hijos de
Gazera, los hijos de Ozias, los hijos de Phinoe, los hijos de Asara, los
hijos de Basthai, los hijos de Asana, los hijos de Maani, los hijos de
Naphisi, los hijos de Acub, los hijos de Achipha, los hijos de Asur, los
hijos de Pharakim, los hijos de Basaloth, \bibleverse{32} los hijos de
Meedda, los hijos de Cutha, los hijos de Charea, los hijos de Barchus,
los hijos de Serar, los hijos de Thomei, los hijos de Nasi, los hijos de
Atipha.

\bibleverse{33} Los hijos de los siervos de Salomón: los hijos de
Assaphioth, los hijos de Pharida, los hijos de Jeeli, los hijos de
Lozon, los hijos de Isdael, los hijos de Saphuthi, \bibleverse{34} los
hijos de Agia, los hijos de Phacareth, los hijos de Sabie, los hijos de
Sarothie, los hijos de Masias, los hijos de Gas, los hijos de Addus, los
hijos de Subas, los hijos de Apherra, los hijos de Barodis, los hijos de
Saphat, los hijos de Allon.

\bibleverse{35} Todos los servidores del templo y los hijos de los
servidores de Salomón eran trescientos setenta y dos. \bibleverse{36}
Estos subieron de Thermeleth, y Thelersas, Charaathalan dirigiéndolos, y
Allar; \bibleverse{37} y no pudieron mostrar sus familias, ni su linaje,
como eran de Israel: los hijos de Dalan hijo de Ban, los hijos de
Nekodan, seiscientos cincuenta y dos.

\bibleverse{38} De los sacerdotes, los que usurparon el oficio del
sacerdocio y no fueron encontrados: los hijos de Obdia, los hijos de
Akkos, los hijos de Jaddus, quien se casó con Augia una de las hijas de
Zorzelleus, y fue llamado con su nombre. \bibleverse{39} Cuando se buscó
en el registro la descripción de la parentela de estos hombres y no se
encontró, se les impidió ejercer el oficio del sacerdocio;
\bibleverse{40} porque Nehemías y Attarías les dijeron que no debían
participar de las cosas santas hasta que surgiera un sumo sacerdote que
llevara Urim y Tumim.

\bibleverse{41} Así que todos los de Israel, de doce años para arriba,
además de los siervos y las siervas, eran en número de cuarenta y dos
mil trescientos sesenta. \bibleverse{42} Sus siervos y siervas eran
siete mil trescientos treinta y siete; los juglares y cantores,
doscientos cuarenta y cinco; \bibleverse{43} cuatrocientos treinta y
cinco camellos, siete mil treinta y seis caballos, doscientos cuarenta y
cinco mulos, y cinco mil quinientos veinticinco animales de carga.

\bibleverse{44} Y algunos de los jefes de sus familias, al llegar al
templo de Dios que está en Jerusalén, hicieron el voto de volver a
levantar la casa en su propio lugar, según su capacidad, \bibleverse{45}
y de dar al santo tesoro de las obras mil minas de oro, cinco mil minas
de plata y cien vestiduras sacerdotales.

\bibleverse{46} Los sacerdotes y los levitas y parte del pueblo vivían
en Jerusalén y en el campo. También los cantores sagrados y los porteros
y todo Israel vivían en sus aldeas.

\bibleverse{47} Pero cuando se acercó el séptimo mes, y cuando los hijos
de Israel estaban cada uno en su lugar, se reunieron todos al unísono en
el lugar amplio delante del primer pórtico que está hacia el oriente.
\bibleverse{48} Entonces Jesús, hijo de Josedec, sus parientes los
sacerdotes, Zorobabel, hijo de Salatiel, y sus parientes se levantaron y
prepararon el altar del Dios de Israel \bibleverse{49} para ofrecer
sobre él sacrificios quemados, conforme a los mandatos expresos del
libro de Moisés, el hombre de Dios. \bibleverse{50} Se les unió gente de
las otras naciones del país, y erigieron el altar en su propio lugar,
porque todas las naciones del país les eran hostiles y los oprimían; y
ofrecieron sacrificios a las horas apropiadas y holocaustos al Señor
tanto por la mañana como por la tarde. \bibleverse{51} También
celebraban la fiesta de los tabernáculos, como está mandado en la ley, y
ofrecían sacrificios diariamente, según el caso. \bibleverse{52} Después
ofrecían las oblaciones continuas y los sacrificios de los sábados, de
las lunas nuevas y de todas las fiestas consagradas. \bibleverse{53}
Todos los que habían hecho algún voto a Dios comenzaron a ofrecer
sacrificios a Dios a partir de la luna nueva del séptimo mes, aunque el
templo de Dios aún no estaba construido. \bibleverse{54} Daban dinero,
comida y bebida a los albañiles y carpinteros. \bibleverse{55} También
dieron carros a la gente de Sidón y de Tiro, para que trajeran cedros de
Libano y los llevaran en balsas al puerto de Jope, según el mandamiento
que les había escrito Ciro, rey de los persas.

\bibleverse{56} En el segundo año después de su llegada al templo de
Dios en Jerusalén, en el segundo mes, Zorobabel hijo de Salatiel, Jesús
hijo de Josedec, sus parientes, los sacerdotes levitas y todos los que
habían llegado a Jerusalén de la cautividad comenzaron a trabajar.
\bibleverse{57} Colocaron los cimientos del templo de Dios en la luna
nueva del segundo mes, en el segundo año después de haber llegado a
Judea y a Jerusalén. \bibleverse{58} Nombraron a los levitas de al menos
veinte años de edad para que se encargaran de las obras del Señor.
Entonces Jesús, con sus hijos y su parentela, Kadmiel su hermano, los
hijos de Jesús, Emadabun, y los hijos de Joda hijo de Iliadun, y sus
hijos y su parentela, todos los levitas, de común acuerdo se levantaron
y comenzaron la empresa, trabajando para hacer avanzar las obras en la
casa de Dios. Así los constructores edificaron el templo del Señor.

\bibleverse{59} Los sacerdotes estaban vestidos con sus ornamentos, con
instrumentos musicales y trompetas, y los levitas hijos de Asaf con sus
címbalos, \bibleverse{60} cantando canciones de acción de gracias y
alabando al Señor, según las indicaciones del rey David de Israel.
\bibleverse{61} Cantaron en voz alta, alabando al Señor con cantos de
acción de gracias, porque su bondad y su gloria son eternas en todo
Israel. \bibleverse{62} Todo el pueblo tocaba las trompetas y gritaba a
gran voz, entonando cantos de acción de gracias al Señor por la
elevación de la casa del Señor. \bibleverse{63} Algunos de los
sacerdotes levitas y de los jefes de sus familias, los ancianos que
habían visto la casa anterior, vinieron a la construcción de ésta con
lamentaciones y grandes llantos. \bibleverse{64} Pero muchos, con
trompetas y alegría, gritaban con gran voz, \bibleverse{65} de modo que
el pueblo no podía oír las trompetas por el llanto del pueblo, pues la
multitud sonaba con fuerza, de modo que se oía a lo lejos.

\bibleverse{66} Por eso, cuando los enemigos de la tribu de Judá y de
Benjamín lo oyeron, llegaron a saber lo que significaba aquel ruido de
trompetas. \bibleverse{67} Se enteraron de que los que habían regresado
del cautiverio construían el templo para el Señor, el Dios de Israel.
\bibleverse{68} Así que fueron a Zorobabel y a Jesús, y a los jefes de
las familias, y les dijeron: ``Construiremos junto con ustedes.
\bibleverse{69} Porque nosotros, al igual que ustedes, obedecemos a su
Señor y le ofrecemos sacrificios desde los días del rey Asbasaret de los
asirios, que nos trajo aquí.''

\bibleverse{70} Entonces Zorobabel, Jesús y los jefes de las familias de
Israel les dijeron: ``No les corresponde a ustedes construir la casa
para el Señor nuestro Dios. \bibleverse{71} Nosotros solos construiremos
para el Señor de Israel, como nos ha mandado el rey Ciro de los
persas''. \bibleverse{72} Pero los paganos del país presionaron
duramente a los habitantes de Judea, les cortaron los suministros y les
impidieron construir. \bibleverse{73} Con sus conspiraciones secretas, y
sus persuasiones y conmociones populares, impidieron que se terminara la
construcción todo el tiempo que vivió el rey Ciro. Así que se les
impidió construir durante dos años, hasta el reinado de Darío.

\hypertarget{section-5}{%
\section{6}\label{section-5}}

\bibleverse{1} En el segundo año del reinado de Darío, los profetas
Aggeo y Zacarías, hijo de Addo, profetizaron a los judíos de Judea y
Jerusalén en nombre del Señor, el Dios de Israel. \bibleverse{2}
Entonces Zorobabel, hijo de Salatiel, y Jesús, hijo de Josedec, se
levantaron y comenzaron a edificar la casa del Señor en Jerusalén,
estando los profetas del Señor con ellos y ayudándoles.

\bibleverse{3} Al mismo tiempo, Sisinnes, el gobernador de Siria y
Fenicia, se acercó a ellos, con Sathrabuzanes y sus compañeros, y les
dijo: \bibleverse{4} ``¿Con qué autoridad construís esta casa y este
techo, y realizáis todas las demás cosas? ¿Quiénes son los constructores
que hacen estas cosas?'' \bibleverse{5} Sin embargo, los ancianos de los
judíos obtuvieron el favor, porque el Señor había visitado a los
cautivos; \bibleverse{6} y no se les impidió construir hasta que se le
comunicó a Darío acerca de ellos, y se recibió su respuesta.

\bibleverse{7} Una copia de la carta que Sisinnes, gobernador de Siria y
Fenicia, y Sathrabuzanes, con sus compañeros, los gobernantes en Siria y
Fenicia, escribieron y enviaron a Darío:

\bibleverse{8} ``Al rey Darío, saludos. Sea notorio a nuestro señor el
rey, que habiendo llegado al país de Judea, y entrado en la ciudad de
Jerusalén, encontramos en la ciudad de Jerusalén a los ancianos de los
judíos que eran del cautiverio \bibleverse{9} construyendo una gran casa
nueva para el Señor, de piedras labradas y costosas, con madera puesta
en las paredes. \bibleverse{10} Esas obras se están realizando con gran
rapidez. La obra prosigue prósperamente en sus manos, y se está llevando
a cabo con toda gloria y diligencia. \bibleverse{11} Entonces
preguntamos a estos ancianos, diciendo: ``¿Con qué autoridad estáis
construyendo esta casa y echando los cimientos de estas obras?''
\bibleverse{12} Por lo tanto, con el fin de daros a conocer por escrito
quiénes eran los dirigentes, los interrogamos, y les exigimos los
nombres por escrito de sus principales hombres. \bibleverse{13} Y nos
respondieron lo siguiente: ``Somos los siervos del Señor, que hizo el
cielo y la tierra. \bibleverse{14} En cuanto a esta casa, fue construida
hace muchos años por un rey grande y fuerte de Israel, y fue terminada.
\bibleverse{15} Pero cuando nuestros padres pecaron contra el Señor de
Israel, que está en el cielo, y lo provocaron a la ira, él los entregó
en manos del rey Nabucodonosor de Babilonia, rey de los caldeos.
\bibleverse{16} Derribaron la casa, la quemaron y llevaron al pueblo
cautivo a Babilonia. \bibleverse{17} Pero en el primer año en que Ciro
reinó sobre el país de Babilonia, el rey Ciro escribió que esta casa
debía ser reconstruida. \bibleverse{18} Los utensilios sagrados de oro y
de plata que Nabucodonosor había sacado de la casa de Jerusalén y que
había instalado en su propio templo, los sacó el rey Ciro del templo de
Babilonia, y fueron entregados a Zorobabel y a Sanabassarus, el
gobernador, \bibleverse{19} con la orden de que se llevara todos estos
utensilios y los pusiera en el templo de Jerusalén, y que se construyera
el templo del Señor en su lugar. \bibleverse{20} Entonces Sanabassarus,
habiendo llegado aquí, puso los cimientos de la casa del Señor que está
en Jerusalén. Desde entonces hasta ahora seguimos construyendo. Todavía
no está totalmente terminada''. \bibleverse{21} Ahora, pues, si te
parece bien, oh rey, que se haga una búsqueda entre los archivos reales
de nuestro señor el rey que están en Babilonia. \bibleverse{22} Si se
encuentra que la construcción de la casa del Señor que está en Jerusalén
se ha hecho con el consentimiento del rey Ciro, y le parece bien a
nuestro señor el rey, que nos envíe instrucciones sobre estas cosas.''

\bibleverse{23} Entonces el rey Darío ordenó que se buscara entre los
archivos que estaban guardados en Babilonia. Y en el palacio de
Ekbatana, que está en el país de Media, se encontró un pergamino en el
que estaban registradas estas cosas: \bibleverse{24} ``En el primer año
del reinado de Ciro, el rey Ciro mandó edificar la casa del Señor que
está en Jerusalén, donde se sacrifica con fuego continuo.
\bibleverse{25} Su altura será de sesenta codos y su anchura de sesenta
codos, con tres hileras de piedras talladas y una hilera de madera nueva
de aquel país. Sus gastos serán dados de la casa del rey Ciro.
\bibleverse{26} Los vasos sagrados de la casa del Señor, tanto de oro
como de plata, que Nabucodonosor sacó de la casa de Jerusalén y se llevó
a Babilonia, deben ser devueltos a la casa de Jerusalén y puestos en el
lugar donde estaban antes.'' \bibleverse{27} También ordenó que Sisinés,
gobernador de Siria y Fenicia, y Sathrabuzanes, y sus compañeros, y los
que habían sido nombrados gobernantes en Siria y Fenicia, tuvieran
cuidado de no entrometerse en el lugar, sino que permitieran a
Zorobabel, siervo del Señor, y gobernador de Judea, y a los ancianos de
los judíos, edificar esa casa del Señor en su lugar. \bibleverse{28}
``También ordeno que la reconstruyan íntegramente y que miren con
diligencia para ayudar a los que son de la cautividad de Judea, hasta
que la casa del Señor esté terminada, \bibleverse{29} y que del tributo
de Coelesyria y de Fenicia se dé cuidadosamente una parte a estos
hombres para los sacrificios del Señor, es decir, a Zorobabel el
gobernador para toros, carneros y corderos, \bibleverse{30} y también
maíz, sal, vino y aceite, y que continuamente cada año, sin más, según
los sacerdotes que están en Jerusalén ordenen que se gaste diariamente,
\bibleverse{31} para que se hagan libaciones al Dios Altísimo por el rey
y por sus hijos, y para que oren por sus vidas.'' \bibleverse{32} Mandó
que al que transgrediera, sí, o descuidara algo de lo aquí escrito, se
le quitara una viga de su propia casa y se le colgara en ella, y se le
embargaran todos sus bienes para el rey. \bibleverse{33} ``Por tanto,
que el Señor, cuyo nombre se invoca allí, destruya por completo a todo
rey y nación que extienda su mano para impedir o dañar esa casa del
Señor en Jerusalén. \bibleverse{34} Yo, el rey Darío, he ordenado que
estas cosas se hagan con diligencia''.

\hypertarget{section-6}{%
\section{7}\label{section-6}}

\bibleverse{1} Entonces Sisinnes, gobernador de Coelesyria y Fenicia, y
Sathrabuzanes, con sus compañeros, siguiendo los mandatos del rey Darío,
\bibleverse{2} supervisaron con mucho cuidado la obra santa, ayudando a
los ancianos de los judíos y a los administradores del templo.
\bibleverse{3} Así prosperó la obra santa, mientras los profetas Aggeo y
Zacarías profetizaban. \bibleverse{4} Ellos terminaron estas cosas por
mandato del Señor, el Dios de Israel, y con el consentimiento de Ciro,
Darío y Artajerjes, reyes de los persas. \bibleverse{5} La Casa Sagrada
fue terminada el día veintitrés del mes de Adar, en el sexto año del rey
Darío. \bibleverse{6} Los hijos de Israel, los sacerdotes, los levitas y
los demás que volvieron del cautiverio, que se unieron a ellos, hicieron
lo que estaba escrito en el libro de Moisés. \bibleverse{7} Para la
dedicación del templo del Señor, ofrecieron cien toros, doscientos
carneros, cuatrocientos corderos, \bibleverse{8} y doce machos cabríos
por el pecado de todo Israel, según el número de los doce príncipes de
las tribus de Israel. \bibleverse{9} Los sacerdotes y los levitas
estaban vestidos con sus vestimentas, según su parentela, para los
servicios del Señor, el Dios de Israel, según el libro de Moisés. Los
porteros estaban en cada puerta.

\bibleverse{10} Los hijos de Israel que salieron del cautiverio
celebraron la Pascua el día catorce del primer mes, cuando los
sacerdotes y los levitas se santificaron juntos, \bibleverse{11} con
todos los que volvieron del cautiverio; porque fueron santificados. Pues
los levitas se santificaban todos juntos, \bibleverse{12} y ofrecían la
Pascua por todos los que volvían del cautiverio, por sus parientes los
sacerdotes y por ellos mismos. \bibleverse{13} Comían los hijos de
Israel que habían salido del cautiverio, todos los que se habían
apartado de las abominaciones de las naciones del país y buscaban al
Señor. \bibleverse{14} Celebraron la fiesta de los panes sin levadura
durante siete días, regocijándose ante el Señor, \bibleverse{15} porque
él había cambiado el consejo del rey de Asiria hacia ellos, para
fortalecer sus manos en las obras del Señor, el Dios de Israel.

\hypertarget{section-7}{%
\section{8}\label{section-7}}

\bibleverse{1} Después de estas cosas, cuando reinaba Artajerjes, rey de
los persas, vino Esdras, que era hijo de Azaraias, hijo de Zacrias, hijo
de Helkias, hijo de Salem, \bibleverse{2} hijo de Sadduk, hijo de
Ahitob, hijo de Amarias, hijo de Ozias, hijo de Memeroth, hijo de
Zaraias, hijo de Boccas, hijo de Abisne, hijo de Phinees, hijo de
Eleazar, hijo de Aarón, el sumo sacerdote. \bibleverse{3} Este Esdras
subió de Babilonia como escriba experto en la ley de Moisés, que fue
dada por el Dios de Israel. \bibleverse{4} El rey lo honró, pues halló
gracia ante sus ojos en todas sus peticiones. \bibleverse{5} También
subieron con él algunos de los hijos de Israel y de los sacerdotes,
levitas, cantores sagrados, porteros y servidores del templo a Jerusalén
\bibleverse{6} en el séptimo año del reinado de Artajerjes, en el quinto
mes (éste era el séptimo año del rey); pues salieron de Babilonia en la
luna nueva del primer mes y llegaron a Jerusalén, por el próspero viaje
que el Señor les dio por su causa. \bibleverse{7} Porque Esdras tenía
una gran habilidad, de modo que no omitió nada de la ley y de los
mandamientos del Señor, sino que enseñó a todo Israel las ordenanzas y
los juicios.

\bibleverse{8} Ahora bien, el encargo, que fue escrito por el rey
Artajerjes, llegó a Esdras, sacerdote y lector de la ley del Señor, fue
el siguiente

\bibleverse{9} ``Rey Artajerjes a Esdras, sacerdote y lector de la ley
del Señor, saludos. \bibleverse{10} Habiendo decidido tratar con gracia,
he dado órdenes de que vayan contigo a Jerusalén los de la nación de los
judíos, los sacerdotes y los levitas, y los de nuestro reino que estén
dispuestos y lo decidan libremente. \bibleverse{11} Por lo tanto, todos
los que estén dispuestos, que partan con vosotros, como nos ha parecido
bien tanto a mí como a mis siete amigos los consejeros, \bibleverse{12}
para que se ocupen de los asuntos de Judea y Jerusalén, de acuerdo con
lo que está en la ley del Señor, \bibleverse{13} y lleven a Jerusalén
las ofrendas al Señor de Israel que yo y mis amigos hemos prometido, y
que se recoja todo el oro y la plata que se pueda encontrar en el país
de Babilonia para el Señor en Jerusalén, \bibleverse{14} con lo que
también se dé del pueblo para el templo del Señor su Dios que está en
Jerusalén: el oro y la plata de los toros, los carneros y los corderos,
y lo que vaya con ellos, \bibleverse{15} a fin de que ofrezcan
sacrificios al Señor sobre el altar del Señor su Dios, que está en
Jerusalén. \bibleverse{16} Todo lo que tú y tu parentela decidan hacer
con el oro y la plata, háganlo según la voluntad de su Dios.
\bibleverse{17} Los vasos sagrados del Señor, que se te han dado para el
uso del templo de tu Dios, que está en Jerusalén, \bibleverse{18} y todo
lo demás que recuerdes para el uso del templo de tu Dios, lo darás del
tesoro del rey. \bibleverse{19} Yo, el rey Artajerjes, he ordenado
también a los guardianes de los tesoros en Siria y Fenicia, que todo lo
que mande pedir Esdras, sacerdote y lector de la ley del Dios Altísimo,
se lo den con toda diligencia, \bibleverse{20} hasta la suma de cien
talentos de plata, así como de trigo hasta cien cors, y cien odres de
vino, y de sal en abundancia. \bibleverse{21} Hacedlo todo según la ley
de Dios con diligencia para el Dios altísimo, para que no venga la ira
sobre el reino del rey y de sus hijos. \bibleverse{22} Te ordeno también
que no se imponga ningún impuesto, ni ninguna otra carga, a ninguno de
los sacerdotes, ni a los levitas, ni a los cantores sagrados, ni a los
porteros, ni a los servidores del templo, ni a ninguno de los que tienen
empleo en este templo, y que ningún hombre tenga autoridad para
imponerles ningún impuesto. \bibleverse{23} Tú, Esdras, según la
sabiduría de Dios, ordena jueces y magistrados para que juzguen en toda
Siria y Fenicia a todos los que conocen la ley de tu Dios; y a los que
no la conocen, les enseñarás. \bibleverse{24} El que transgreda la ley
de tu Dios y del rey será castigado con diligencia, ya sea con la muerte
o con otro castigo, con pena de dinero o con prisión.''

\bibleverse{25} Entonces el escriba Esdras dijo: ``Bendito sea el único
Señor, el Dios de mis padres, que ha puesto estas cosas en el corazón
del rey, para glorificar su casa que está en Jerusalén, \bibleverse{26}
y me ha honrado a los ojos del rey, de sus consejeros y de todos sus
amigos y nobles. \bibleverse{27} Por eso me animé con la ayuda del
Señor, mi Dios, y reuní de Israel hombres que subieran conmigo.

\bibleverse{28} Estos son los principales, según sus familias y sus
diversas divisiones, que subieron conmigo desde Babilonia en el reinado
del rey Artajerjes \bibleverse{29} de los hijos de Finees, Gerson; de
los hijos de Itamar, Gamael; de los hijos de David, Attus hijo de
Sechenias; \bibleverse{30} de los hijos de Phoros, Zacharais; y con él
se contaban ciento cincuenta hombres; \bibleverse{31} de los hijos de
Phaath Moab, Eliaonias hijo de Zaraias, y con él doscientos hombres;
\bibleverse{32} de los hijos de Zato, Sechenias hijo de Jezelus, y con
él trescientos hombres; de los hijos de Adin, Obeth hijo de Jonathan, y
con él doscientos cincuenta hombres; \bibleverse{33} de los hijos de
Elam, Jesias hijo de Gotholias, y con él setenta hombres;
\bibleverse{34} de los hijos de Saphatias, Zaraias hijo de Michael, y
con él setenta hombres; \bibleverse{35} de los hijos de Joab, Abadias
hijo de Jezelo, y con él doscientos doce hombres; \bibleverse{36} de los
hijos de Banias, Salimot hijo de Josafías, y con él ciento sesenta
hombres; \bibleverse{37} de los hijos de Babi, Zacarías hijo de Bebai, y
con él veintiocho hombres; \bibleverse{38} de los hijos de Azgad Astath,
Joannes hijo de Hakkatan Akatan, y con él ciento diez hombres;
\bibleverse{39} de los hijos de Adonikam, los últimos, y estos son sus
nombres, Eliphalat, Jeuel y Samaias, y con ellos setenta hombres;
\bibleverse{40} de los hijos de Bago, Uthi hijo de Istalcurus, y con él
setenta hombres.

\bibleverse{41} Los reuní hasta el río llamado Theras. Allí acampamos
tres días, y los inspeccioné. \bibleverse{42} Cuando no encontré allí a
ninguno de los sacerdotes y levitas, \bibleverse{43} envié entonces a
Eleazar, Iduel, Maasmas, \bibleverse{44} Elnathan, Samaias, Joribus,
Natán, Ennatan, Zacarías y Mosolamus, hombres principales y de
entendimiento. \bibleverse{45} Les pedí que fueran a ver al capitán
Loddeo, que estaba en el lugar de la tesorería, \bibleverse{46} y les
ordené que hablaran con Loddeo, con su parentela y con los tesoreros de
aquel lugar, para que nos enviaran hombres que pudieran desempeñar el
oficio de sacerdotes en la casa de nuestro Señor. \bibleverse{47} Por la
poderosa mano de nuestro Señor, nos trajeron hombres de entendimiento de
los hijos de Mooli, hijo de Leví, hijo de Israel, Asebebias, y sus
hijos, y su parentela, que eran dieciocho, \bibleverse{48} y Asebias,
Annuus, y Osaias su hermano, de los hijos de Chanuneus, y sus hijos eran
veinte hombres; \bibleverse{49} y de los sirvientes del templo que David
y los principales habían designado para los sirvientes de los levitas,
doscientos veinte sirvientes del templo. La lista de todos sus nombres
fue reportada.

\bibleverse{50} Allí hice un voto de ayuno por los jóvenes ante nuestro
Señor, para pedirle un viaje próspero tanto para nosotros como para
nuestros hijos y el ganado que estaba con nosotros; \bibleverse{51} pues
me daba vergüenza pedir al rey infantería, caballería y una escolta para
protegernos de nuestros adversarios. \bibleverse{52} Porque habíamos
dicho al rey que el poder de nuestro Señor estaría con los que lo
buscan, para apoyarlos en todo. \bibleverse{53} Volvimos a orar a
nuestro señor sobre estas cosas, y lo encontramos misericordioso.

\bibleverse{54} Entonces aparté a doce hombres de los jefes de los
sacerdotes, Eserebias, Assamias y diez hombres de su parentela con
ellos. \bibleverse{55} Les pesé la plata, el oro y los utensilios
sagrados de la casa de nuestro Señor, que el rey, sus consejeros, los
nobles y todo Israel habían dado. \bibleverse{56} Cuando lo hube pesado,
les entregué seiscientos cincuenta talentos de plata, vasos de plata que
pesaban cien talentos, cien talentos de oro, \bibleverse{57} veinte
vasos de oro y doce vasos de bronce, de bronce fino, brillantes como el
oro. \bibleverse{58} Les dije: ``Vosotros sois santos para el Señor, los
vasos son santos, y el oro y la plata son un voto para el Señor, el
Señor de nuestros padres. \bibleverse{59} Velad y guardadlos hasta que
los entreguéis a los jefes de los sacerdotes y de los levitas, y a los
principales hombres de las familias de Israel en Jerusalén, en las
cámaras de la casa de nuestro Señor. \bibleverse{60} Los sacerdotes y
los levitas que recibieron la plata, el oro y los utensilios que estaban
en Jerusalén, los llevaron al templo del Señor.

\bibleverse{61} Salimos del río Theras el duodécimo día del primer mes.
Llegamos a Jerusalén por la poderosa mano de nuestro Señor que estaba
sobre nosotros. El Señor nos libró de todo enemigo en el camino, y así
llegamos a Jerusalén. \bibleverse{62} Cuando estuvimos allí tres días,
la plata y el oro fueron pesados y entregados en la casa de nuestro
Señor al cuarto día a Marmot, el sacerdote hijo de Urias.
\bibleverse{63} Con él estaba Eleazar, hijo de Finees, y con ellos
estaban Josabdus, hijo de Jesús, y Moeth, hijo de Sabannus, los levitas.
Todo les fue entregado por número y peso. \bibleverse{64} Todo el peso
de ellos fue registrado a la misma hora. \bibleverse{65} Además, los que
habían salido del cautiverio ofrecieron sacrificios al Señor, el Dios de
Israel, doce toros por todo Israel, noventa y seis carneros,
\bibleverse{66} setenta y dos corderos y doce machos cabríos como
ofrenda de paz, todo ello como sacrificio al Señor. \bibleverse{67}
Entregaron los mandatos del rey a los administradores del rey y a los
gobernadores de Coelesyria y Fenicia, y honraron al pueblo y al templo
del Señor.

\bibleverse{68} Una vez hechas estas cosas, vinieron a mí los
principales y me dijeron: \bibleverse{69} ``La nación de Israel, los
príncipes, los sacerdotes y los levitas no han apartado de sí a los
pueblos extranjeros de la tierra ni las impurezas de los gentiles:
cananeos, hititas, ferezeos, jebuseos, moabitas, egipcios y edomitas.
\bibleverse{70} Porque tanto ellos como sus hijos se han casado con sus
hijas, y la semilla santa se ha mezclado con los pueblos extranjeros de
la tierra. Desde el principio de este asunto los gobernantes y los
nobles han sido partícipes de esta iniquidad.''

\bibleverse{71} Tan pronto como oí estas cosas, rasgué mis vestidos y mi
vestimenta sagrada, y me arranqué el pelo de la cabeza y de la barba, y
me senté triste y lleno de tristeza. \bibleverse{72} Y todos los que se
conmovieron por la palabra del Señor, el Dios de Israel, se reunieron
conmigo mientras yo lloraba por la iniquidad, pero me quedé sentado
lleno de tristeza hasta el sacrificio de la tarde. \bibleverse{73}
Entonces, levantándome del ayuno con mis ropas y mi vestimenta sagrada
rasgada, e inclinando mis rodillas y extendiendo mis manos hacia el
Señor, \bibleverse{74} dije: ``Señor, estoy avergonzado y confundido
ante tu rostro, \bibleverse{75} porque nuestros pecados se han
multiplicado sobre nuestras cabezas, y nuestros errores han llegado
hasta el cielo \bibleverse{76} desde el tiempo de nuestros padres.
Estamos en gran pecado, hasta el día de hoy. \bibleverse{77} Por
nuestros pecados y los de nuestros padres, nosotros, con nuestra
parentela, nuestros reyes y nuestros sacerdotes, fuimos entregados a los
reyes de la tierra, a la espada y al cautiverio, y como presa con
vergüenza, hasta el día de hoy. \bibleverse{78} Ahora bien, en alguna
medida se nos ha mostrado misericordia de tu parte, Señor, para dejarnos
una raíz y un nombre en el lugar de tu santuario, \bibleverse{79} y para
descubrir una luz en la casa del Señor, nuestro Dios, y para darnos
alimento en el tiempo de nuestra servidumbre. \bibleverse{80} Sí, cuando
estábamos en la esclavitud, no fuimos abandonados por nuestro Señor,
sino que nos favoreció ante los reyes de Persia, de modo que nos dieron
alimento, \bibleverse{81} glorificaron el templo de nuestro Señor, y
levantaron la desolada Sión, para darnos una morada segura en Judea y
Jerusalén.

\bibleverse{82} ``Ahora, Señor, ¿qué diremos, teniendo en cuenta estas
cosas? Porque hemos transgredido tus mandamientos que diste por medio de
tus siervos los profetas, diciendo: \bibleverse{83} `La tierra, en la
que entráis para poseerla como herencia, es una tierra contaminada por
las contaminaciones de los extranjeros del país, y ellos la han llenado
de su impureza. \bibleverse{84} Por tanto, ahora no unirás a tus hijas
con sus hijos, ni tomarás a sus hijas para tus hijos. \bibleverse{85} No
buscarás nunca la paz con ellos, para que te fortalezcas y comas los
bienes de la tierra, y para que la dejes en herencia a tus hijos para
siempre.' \bibleverse{86} Todo lo que ha sucedido nos ha sido hecho por
nuestras malas obras y grandes pecados, pues tú, Señor, hiciste livianos
nuestros pecados, \bibleverse{87} y nos diste tal raíz; pero nos hemos
vuelto de nuevo para transgredir tu ley mezclándonos con la impureza de
las naciones de la tierra. \bibleverse{88} No te enojaste con nosotros
para destruirnos hasta que no nos dejaste ni raíz, ni semilla, ni
nombre. \bibleverse{89} Señor de Israel, tú eres veraz, pues hoy nos has
dejado una raíz. \bibleverse{90} He aquí que ahora estamos ante ti en
nuestras iniquidades, pues ya no podemos permanecer ante ti a causa de
estas cosas.''

\bibleverse{91} Mientras Esdras, en su oración, hacía su confesión,
llorando y tendido en el suelo ante el templo, se reunió hacia él una
muchedumbre muy grande de hombres, mujeres y niños de Jerusalén, pues
había un gran llanto entre la multitud. \bibleverse{92} Entonces
Jechonias, hijo de Jeelus, uno de los hijos de Israel, gritó y dijo:
``Oh Esdras, hemos pecado contra el Señor Dios, nos hemos casado con
mujeres extranjeras de los paganos del país, pero todavía hay esperanza
para Israel. \bibleverse{93} Hagamos un juramento al Señor sobre esto,
de que repudiaremos a todas nuestras mujeres extranjeras con sus hijos,
\bibleverse{94} como te parezca bien, y a todos los que obedezcan la Ley
del Señor. \bibleverse{95} Levántate y ponte en acción, porque ésta es
tu tarea, y nosotros estaremos contigo para hacerla valientemente''.
\bibleverse{96} Se levantó, pues, Esdras y tomó juramento a los jefes de
los sacerdotes y levitas de todo Israel para que hicieran estas cosas, y
lo juraron.

\hypertarget{section-8}{%
\section{9}\label{section-8}}

\bibleverse{1} Entonces Esdras se levantó del atrio del templo y fue a
la cámara de Jonás, hijo de Eliasib, \bibleverse{2} y se alojó allí, y
no comió pan ni bebió agua, lamentándose por las grandes iniquidades de
la multitud. \bibleverse{3} Se hizo una proclama en toda Judea y
Jerusalén a todos los que volvían de la cautividad, para que se
reunieran en Jerusalén, \bibleverse{4} y para que quien no se reuniera
allí en el plazo de dos o tres días, de acuerdo con la decisión de los
ancianos, se le confiscara su ganado para el uso del templo, y fuera
expulsado de la multitud de los que volvían de la cautividad.

\bibleverse{5} A los tres días, todos los de la tribu de Judá y de
Benjamín se reunieron en Jerusalén. Era el mes noveno, a los veinte días
del mes. \bibleverse{6} Toda la multitud se sentó junta, temblando, en
el amplio lugar delante del templo, a causa del mal tiempo que hacía.
\bibleverse{7} Entonces Esdras se levantó y les dijo: ``Habéis
transgredido la ley y os habéis casado con mujeres extranjeras,
aumentando los pecados de Israel. \bibleverse{8} Ahora confesad y dad
gloria al Señor, el Dios de nuestros padres, \bibleverse{9} y haced su
voluntad, y apartaos de los paganos del país y de las mujeres
extranjeras.''

\bibleverse{10} Entonces toda la multitud gritó y dijo en voz alta:
``Tal como has dicho, así haremos. \bibleverse{11} Pero como la multitud
es grande, y hace mal tiempo, de modo que no podemos estar afuera, y
esto no es obra de un día ni de dos, viendo que nuestro pecado en estas
cosas se ha extendido mucho, \bibleverse{12} por lo tanto, que los jefes
de la multitud se queden, y que todos los de nuestros asentamientos que
tienen esposas extranjeras vengan a la hora señalada, \bibleverse{13} y
con ellos los jefes y los jueces de cada lugar, hasta que alejemos de
nosotros la ira del Señor por este asunto.''

\bibleverse{14} Así que Jonatán, hijo de Azael, y Ezequías, hijo de
Tocano, se encargaron del asunto. Mosollamus, Levis y Sabbateus fueron
jueces con ellos. \bibleverse{15} Los que volvieron del cautiverio
hicieron conforme a todo esto.

\bibleverse{16} El sacerdote Esdras eligió para sí a los principales
hombres de sus familias, todos por su nombre. En la luna nueva del
décimo mes se reunieron para examinar el asunto. \bibleverse{17} Así que
sus casos de hombres que tenían esposas extranjeras llegaron a su fin en
la luna nueva del primer mes.

\bibleverse{18} De los sacerdotes que se habían reunido y tenían esposas
extranjeras, se encontró \bibleverse{19} de los hijos de Jesús, hijo de
Josedec, y de su parentela, Mathelas, Eleazar, y Joribus, y Joadanus.
\bibleverse{20} Dieron sus manos para repudiar a sus mujeres, y para
ofrecer carneros para reconciliarse con su error. \bibleverse{21} De los
hijos de Emmer Ananías, Zabdeus, Manes, Sameus, Hiereel, y Azarias.
\bibleverse{22} De los hijos de Faisur: Elionas, Massias, Ismael,
Natanael, Ocidelus, y Saloas.

\bibleverse{23} De los levitas: Jozabdus, Semeis, Colius que se llamaba
Calitas, Patheus, Judas y Jonas. \bibleverse{24} De los cantores
sagrados: Eliasibus y Bacchurus. \bibleverse{25} De los porteros:
Sallumus y Tolbanes.

\bibleverse{26} De Israel, de los hijos de Foros: Hiermas, Ieddias,
Melchias, Maelus, Eleazar, Asibas, y Banneas. \bibleverse{27} De los
hijos de Ela Matanías, Zacarías, Jezrielus, Oabdius, Hieremoth, y
Aedias. \bibleverse{28} De los hijos de Zamot: Eliadas, Eliasimus,
Othonias, Jarimoth, Sabathus, y Zardeus. \bibleverse{29} De los hijos de
Bebai Joannes, Ananías, Jozabdus, y Ematheis. \bibleverse{30} De los
hijos de Mani: Olamus, Mamuchus, Jedeus, Jasubas, Jasaelus, y Hieremoth.
\bibleverse{31} De los hijos de Addi: Naato, Moisés, Lacunio, Naido,
Matanías, Sestiel, Balnuus y Manasés. \bibleverse{32} De los hijos de
Anás Elionas, Aseas, Melchias, Sabbeus y Simón Chosameus.
\bibleverse{33} De los hijos de Asom Maltaneo, Matatías, Sabaneo,
Elifalato, Manasés y Semei. \bibleverse{34} De los hijos de Baani
Jeremías, Momdis, Ismaerus, Juel, Mamdai, Pedias, Anos, Carabasion,
Enasibus, Mamnitamenus, Eliasis, Bannus, Eliali, Someis, Selemias y
Nathanias. De los hijos de Ezora: Sesis, Ezril, Azaelus, Samatus, Zambri
y Josephus. \bibleverse{35} De los hijos de Nooma: Mazitias, Zabadeas,
Edos, Juel y Banaias. \bibleverse{36} Todos estos habían tomado esposas
extranjeras, y las despidieron con sus hijos.

\bibleverse{37} Los sacerdotes y levitas, y los que eran de Israel,
vivían en Jerusalén y en el campo, en la luna nueva del mes séptimo, y
los hijos de Israel en sus asentamientos.

\bibleverse{38} Toda la multitud se reunió al unísono en el lugar amplio
delante del pórtico del templo, hacia el este. \bibleverse{39} Dijeron
al sacerdote y lector Esdras: ``Trae la ley de Moisés que fue dada por
el Señor, el Dios de Israel''.

\bibleverse{40} Entonces Esdras, el sumo sacerdote, llevó la ley a toda
la multitud, tanto de hombres como de mujeres, y a todos los sacerdotes,
para que escucharan la ley en la luna nueva del mes séptimo.
\bibleverse{41} Leyó en el lugar amplio, delante del pórtico del templo,
desde la mañana hasta el mediodía, ante hombres y mujeres; y toda la
multitud prestó atención a la ley. \bibleverse{42} El sacerdote Esdras,
lector de la ley, estaba de pie sobre el púlpito de madera que se había
preparado. \bibleverse{43} Junto a él estaban Matatías, Samús, Ananías,
Azarías, Urías, Ezequías y Baalsamo, a la derecha, \bibleverse{44} y a
la izquierda, Faldeo, Misael, Melquías, Lotásubo, Nabarias y Zacarías.
\bibleverse{45} Entonces Esdras tomó el libro de la ley ante la
multitud, y se sentó honorablemente en el primer lugar ante todos.
\bibleverse{46} Cuando abrió la ley, todos se pusieron de pie. Entonces
Esdras bendijo al Señor Dios Altísimo, el Dios de los ejércitos, el
Todopoderoso. \bibleverse{47} Todo el pueblo respondió: ``Amén''.
Levantando las manos, se postraron en el suelo y adoraron al Señor.
\bibleverse{48} También Jesús, Annus, Sarabias, Iadinus, Jacubus,
Sabateus, Auteas, Maiannas, Calitas, Azarias, Jozabdus, Ananias y
Phalias, los levitas, enseñaban la ley del Señor, y leían a la multitud
la ley del Señor, explicando lo leído.

\bibleverse{49} Entonces Attharates dijo a Esdras, el sumo sacerdote y
lector, y a los levitas que enseñaban a la multitud, a todos,
\bibleverse{50} ``Este día es sagrado para el Señor --- todos lloraron
al oír la ley --- \bibleverse{51} Id, pues, a comer lo gordo, a beber lo
dulce y a enviar porciones a los que no tienen nada; \bibleverse{52}
porque el día es sagrado para el Señor. No os entristezcáis, porque el
Señor os honrará''. \bibleverse{53} Así que los levitas ordenaron todo
al pueblo, diciendo: ``Este día es santo. No os entristezcáis''.
\bibleverse{54} Entonces se pusieron en camino, cada uno a comer, a
beber, a divertirse, a dar porciones a los que no tenían nada, y a
alegrarse mucho, \bibleverse{55} porque entendían las palabras con las
que habían sido instruidos, y para las cuales se habían reunido.
