\hypertarget{el-destino-de-noemuxed-en-la-tierra-de-los-moabitas}{%
\subsection{El destino de Noemí en la tierra de los
moabitas}\label{el-destino-de-noemuxed-en-la-tierra-de-los-moabitas}}

\hypertarget{section}{%
\section{1}\label{section}}

\bibleverse{1} En los días en que los jueces juzgaban, hubo hambre en la
tierra. Un hombre de Belén de Judá se fue a vivir al país de Moab con su
mujer y sus dos hijos. \bibleverse{2} El nombre de aquel hombre era
Elimelec, y el de su mujer Noemí. Los nombres de sus dos hijos eran
Mahlón y Quelión, efrateos de Belén de Judá. Llegaron al país de Moab y
vivieron allí. \bibleverse{3} Elimelec, el esposo de Noemí, murió, y
ella quedó con sus dos hijos. \bibleverse{4} Ellos tomaron para sí
esposas de las mujeres de Moab. El nombre de una era Orfa, y el de la
otra era Rut. Vivieron allí unos diez años. \bibleverse{5} Mahlón y
Quelión murieron, y la mujer quedó despojada de sus dos hijos y de su
marido. \bibleverse{6} Entonces se levantó con sus nueras para volver
del país de Moab, porque había oído en el país de Moab cómo el
Señor\footnote{\textbf{1:6} ``Yahvé'' es el nombre propio de Dios, a
  veces traducido como ``\textsc{Señor}'' (en mayúsculas) en otras
  traducciones.} había visitado a su pueblo dándole pan.

\hypertarget{partida-de-noemuxed-y-sus-dos-nueras-para-regresar-a-beluxe9n-la-despedida-de-orpa-la-lealtad-de-ruth}{%
\subsection{Partida de Noemí y sus dos nueras para regresar a Belén; La
despedida de Orpa, la lealtad de
Ruth}\label{partida-de-noemuxed-y-sus-dos-nueras-para-regresar-a-beluxe9n-la-despedida-de-orpa-la-lealtad-de-ruth}}

\bibleverse{7} Salió del lugar donde estaba, y sus dos nueras con ella.
Siguieron el camino para regresar a la tierra de Judá. \bibleverse{8}
Noemí dijo a sus dos nueras: ``Vayan, vuelvan cada una a la casa de su
madre. Que el Señor las trate con bondad, como ha tratado a los muertos
y a mí. \bibleverse{9} Que Yahvé les conceda que encuentren descanso,
cada una en la casa de su marido''. Entonces ella los besó, y ellos
alzaron la voz y lloraron. \footnote{\textbf{1:9} Rut 3,1}
\bibleverse{10} Le dijeron: ``No, pero volveremos contigo a tu pueblo''.

\bibleverse{11} Noemí dijo: ``Volved, hijas mías. ¿Por qué queréis ir
conmigo? ¿Aún tengo hijos en mi vientre, para que sean vuestros maridos?
\bibleverse{12} Volved, hijas mías, seguid vuestro camino, porque soy
demasiado vieja para tener marido. Si dijera: ``Tengo esperanza'', si
incluso tuviera un marido esta noche, y también diera a luz hijos,
\bibleverse{13} ¿esperaríais entonces a que crecieran? ¿Os abstendríais
entonces de tener maridos? No, hijas mías, porque me duele mucho por
vosotras, pues la mano de Yahvé ha salido contra mí''. \footnote{\textbf{1:13}
  Job 19,21}

\bibleverse{14} Levantaron la voz y volvieron a llorar; entonces Orfa
besó a su suegra, pero Rut se quedó con ella. \bibleverse{15} Ella dijo:
``Mira,\footnote{\textbf{1:15} ``He aquí'', de ``\hebrew{הִנֵּה}'',
  significa mirar, fijarse, observar, ver o contemplar. Se utiliza a
  menudo como interjección.} tu cuñada ha vuelto a su pueblo y a su
dios. Sigue a tu cuñada''.

\bibleverse{16} Rut le dijo: ``No me insistas en que te deje y en que
deje de seguirte, porque adonde tú vayas, iré yo; y donde tú te quedes,
me quedaré yo. Tu pueblo será mi pueblo, y tu Dios\footnote{\textbf{1:16}
  La palabra hebrea traducida como ``Dios'' es ``\hebrew{אֱלֹהִ֑ים}''
  (Elohim).} mi Dios. \footnote{\textbf{1:16} 2Sam 15,21}
\bibleverse{17} Donde tú mueras, moriré yo, y allí seré enterrado. Que
Yahvé haga así conmigo, y más también, si algo más que la muerte nos
separa a ti y a mí''.

\bibleverse{18} Cuando Noemí vio que estaba decidida a ir con ella, dejó
de insistirle.

\hypertarget{llegada-y-recepciuxf3n-de-las-dos-mujeres-en-beluxe9n}{%
\subsection{Llegada y recepción de las dos mujeres en
Belén}\label{llegada-y-recepciuxf3n-de-las-dos-mujeres-en-beluxe9n}}

\bibleverse{19} Así fueron las dos hasta que llegaron a Belén. Cuando
llegaron a Belén, toda la ciudad se entusiasmó con ellas, y preguntaron:
``¿Esta es Noemí?''.

\bibleverse{20} Ella les dijo: ``No me llaméis Noemí.\footnote{\textbf{1:20}
  ``Noemí'' significa ``agradable''.} Llámenme Mara,\footnote{\textbf{1:20}
  Un payim (o pim) era 2/3 siclos de plata, es decir, 0,26 onzas o 7,6
  gramos} porque el Todopoderoso me ha tratado con mucha amargura.
\footnote{\textbf{1:20} Éxod 15,23} \bibleverse{21} Salí llena, y el
Señor me ha hecho volver a casa vacía. ¿Por qué me llamas Noemí, ya que
Yahvé ha testificado contra mí, y el Todopoderoso me ha afligido?''
\bibleverse{22} Volvió, pues, Noemí, y con ella Rut la moabita, su
nuera, que había vuelto del país de Moab. Llegaron a Belén al comienzo
de la cosecha de cebada.

\hypertarget{rut-viene-a-recoger-espigas-en-el-campo-del-booz-quien-pregunta-por-ella-y-la-recibe-amablemente}{%
\subsection{Rut viene a recoger espigas en el campo del booz, quien
pregunta por ella y la recibe
amablemente}\label{rut-viene-a-recoger-espigas-en-el-campo-del-booz-quien-pregunta-por-ella-y-la-recibe-amablemente}}

\hypertarget{section-1}{%
\section{2}\label{section-1}}

\bibleverse{1} Noemí tenía un pariente de su marido, un hombre poderoso
y rico, de la familia de Elimelec, que se llamaba Booz. \bibleverse{2}
Rut, la moabita, dijo a Noemí: ``Déjame ir ahora al campo y espigar
entre las espigas en pos de aquel a cuya vista encuentro gracia.'' Le
dijo: ``Ve, hija mía''. \bibleverse{3} Ella fue, y vino a espigar en el
campo tras los segadores; y por casualidad llegó a la parte del campo
que pertenecía a Booz, que era de la familia de Elimelec.

\bibleverse{4} He aquí que Booz vino de Belén y dijo a los segadores:
``Que Yahvé esté con vosotros''. Le respondieron: ``Que Yahvé te
bendiga''.

\bibleverse{5} Entonces Booz dijo a su criado, que estaba al frente de
los segadores: ``¿De quién es esta joven?''

\bibleverse{6} El criado que estaba a cargo de los segadores respondió:
``Es la dama moabita que regresó con Noemí del país de Moab.
\bibleverse{7} Ella dijo: `Por favor, déjame espigar y recoger después
de los segadores entre las gavillas'. Así que vino, y ha continuado
desde la mañana hasta ahora, excepto que descansó un poco en la casa.''

\bibleverse{8} Entonces Booz dijo a Rut: ``Escucha, hija mía. No vayas a
espigar a otro campo, ni te vayas de aquí, sino que quédate aquí cerca
de mis doncellas. \bibleverse{9} Pon tus ojos en el campo que cosechan,
y ve tras ellas. ¿No he ordenado a los jóvenes que no te toquen? Cuando
tengas sed, ve a las vasijas y bebe de lo que los jóvenes han sacado''.

\bibleverse{10} Entonces ella se postró en el suelo y le dijo: ``¿Por
qué he hallado gracia ante tus ojos para que me conozcas, ya que soy
extranjera?''

\bibleverse{11} Booz le respondió: ``Me han contado todo lo que has
hecho por tu suegra desde la muerte de tu marido, y cómo has dejado a tu
padre, a tu madre y la tierra donde naciste, y has llegado a un pueblo
que no conocías. \footnote{\textbf{2:11} Rut 1,16-17} \bibleverse{12}
Que Yahvé te pague tu trabajo y te dé una recompensa completa de parte
de Yahvé, el Dios de Israel, bajo cuyas alas has venido a refugiarte.''

\bibleverse{13} Entonces ella dijo: ``Halle yo gracia ante tus ojos,
señor mío, porque me has consolado y porque has hablado con bondad a tu
sierva, aunque no soy como una de tus siervas.''

\hypertarget{rut-sigue-siendo-tratada-amablemente-por-booz-llega-a-casa-con-una-rica-cosecha-y-recibe-informaciuxf3n-sobre-booz-de-su-suegra}{%
\subsection{Rut sigue siendo tratada amablemente por Booz, llega a casa
con una rica cosecha y recibe información sobre Booz de su
suegra}\label{rut-sigue-siendo-tratada-amablemente-por-booz-llega-a-casa-con-una-rica-cosecha-y-recibe-informaciuxf3n-sobre-booz-de-su-suegra}}

\bibleverse{14} A la hora de comer, Booz le dijo: ``Ven aquí, come un
poco de pan y moja tu bocado en el vinagre''. Se sentó junto a los
segadores, y éstos le pasaron el grano reseco. Ella comió, quedó
satisfecha y dejó un poco. \bibleverse{15} Cuando se levantó a espigar,
Booz ordenó a sus criados, diciendo: ``Dejadla espigar incluso entre las
gavillas, y no la reprochéis. \bibleverse{16} También saquen algo para
ella de los manojos y déjenlo. Dejadla espigar, y no la reprendáis''.
\footnote{\textbf{2:16} Lev 19,9}

\bibleverse{17} Así que espigó en el campo hasta la noche; y sacó lo que
había espigado, que era como un efa de cebada. \bibleverse{18} Lo
recogió y se fue a la ciudad. Entonces su suegra vio lo que había
espigado, y sacó y le dio lo que le había sobrado.

\bibleverse{19} Su suegra le dijo: ``¿Dónde has espigado hoy? ¿Dónde has
trabajado? Bendito sea el que se fijó en ti''. Le dijo a su suegra con
quien había trabajado: ``El nombre del hombre con quien he trabajado hoy
es Booz''. \bibleverse{20} Noemí dijo a su nuera: ``Que sea bendecido
por Yahvé, que no ha abandonado su bondad con los vivos y con los
muertos.'' Noemí le dijo: ``Ese hombre es un pariente cercano a
nosotros, uno de nuestros parientes cercanos.''

\bibleverse{21} Rut la moabita dijo: ``Sí, él me dijo: ``Te quedarás
cerca de mis jóvenes hasta que terminen toda mi cosecha''\,''.

\bibleverse{22} Noemí dijo a Rut, su nuera: ``Es bueno, hija mía, que
salgas con sus doncellas y que no te encuentren en ningún otro campo.''
\bibleverse{23} Así que se quedó cerca de las doncellas de Booz, para
espigar hasta el final de la cosecha de cebada y de trigo; y vivió con
su suegra.

\hypertarget{siguiendo-el-consejo-de-noemuxed-rut-va-a-la-era-de-booz-y-se-acuesta-a-sus-pies}{%
\subsection{Siguiendo el consejo de Noemí, Rut va a la era de Booz y se
acuesta a sus
pies}\label{siguiendo-el-consejo-de-noemuxed-rut-va-a-la-era-de-booz-y-se-acuesta-a-sus-pies}}

\hypertarget{section-2}{%
\section{3}\label{section-2}}

\bibleverse{1} Noemí, su suegra, le dijo: ``Hija mía, ¿no he de buscar
tu descanso para que te vaya bien? \footnote{\textbf{3:1} Rut 1,9}
\bibleverse{2} ¿No es Booz nuestro pariente, con cuyas doncellas
estabas? He aquí que esta noche estará aventando cebada en la era.
\bibleverse{3} Por tanto, lávate, úntate, vístete y baja a la era; pero
no te des a conocer al hombre hasta que haya terminado de comer y beber.
\bibleverse{4} Cuando se acueste, te fijarás en el lugar donde está
acostado. Entonces entrarás, descubrirás sus pies y te acostarás.
Entonces él te dirá lo que debes hacer''.

\bibleverse{5} Ella le dijo: ``Todo lo que digas, lo haré''.
\bibleverse{6} Bajó a la era e hizo todo lo que su suegra le dijo.
\bibleverse{7} Cuando Booz hubo comido y bebido, y su corazón estaba
alegre, fue a acostarse al final del montón de grano. Ella se acercó
suavemente, le descubrió los pies y se acostó.

\hypertarget{rut-habla-con-booz-recibe-la-confirmaciuxf3n-solicitada-y-regresa-a-noemuxed-con-un-regalo}{%
\subsection{Rut habla con Booz, recibe la confirmación solicitada y
regresa a Noemí con un
regalo}\label{rut-habla-con-booz-recibe-la-confirmaciuxf3n-solicitada-y-regresa-a-noemuxed-con-un-regalo}}

\bibleverse{8} A medianoche, el hombre se asustó y se volvió; y he aquí
que una mujer yacía a sus pies. \bibleverse{9} Le dijo: ``¿Quién
eres?''. Ella respondió: ``Yo soy Rut, tu sierva. Extiende, pues, la
esquina de tu manto sobre tu sierva, porque eres pariente cercano''.
\footnote{\textbf{3:9} Deut 25,5; Ezeq 16,8}

\bibleverse{10} Él dijo: ``Has sido bendecida por Yahvé, hija mía. Has
mostrado más bondad al final que al principio, porque no seguiste a los
jóvenes, sean pobres o ricos. \footnote{\textbf{3:10} Rut 2,11}
\bibleverse{11} Ahora, hija mía, no tengas miedo. Haré contigo todo lo
que digas; porque toda la ciudad de mi pueblo sabe que eres una mujer
digna. \bibleverse{12} Es cierto que soy un pariente cercano. Sin
embargo, hay un pariente más cercano que yo. \bibleverse{13} Quédate
esta noche, y por la mañana, si él hace por ti la parte de un pariente,
bien. Que cumpla con el deber de pariente. Pero si no cumple con el
deber de un pariente para ti, entonces yo haré el deber de un pariente
para ti, vive Yahvé. Acuéstate hasta la mañana''.

\bibleverse{14} La mujer se echó a sus pies hasta la mañana, y luego se
levantó antes de que se pudiera discernir otra cosa. Porque dijo: ``Que
no se sepa que la mujer vino a la era''. \bibleverse{15} Le dijo: ``Trae
el manto que llevas puesto y sostenlo''. Ella lo sostuvo; y él midió
seis medidas de cebada, y se las puso encima; luego se fue a la ciudad.

\bibleverse{16} Cuando llegó a su suegra, le dijo: ``¿Cómo te fue, hija
mía?''. Ella le contó todo lo que el hombre había hecho por ella.
\bibleverse{17} Ella dijo: ``Me dio estas seis medidas de cebada, porque
me dijo: ``No vayas con las manos vacías a casa de tu suegra''\,''.

\bibleverse{18} Entonces ella dijo: ``Espera, hija mía, hasta que sepas
lo que va a pasar; porque el hombre no descansará hasta que haya
resuelto esto hoy.''

\hypertarget{la-negociaciuxf3n-puxfablica-entre-booz-y-el-solver}{%
\subsection{La negociación pública entre Booz y el
Solver}\label{la-negociaciuxf3n-puxfablica-entre-booz-y-el-solver}}

\hypertarget{section-3}{%
\section{4}\label{section-3}}

\bibleverse{1} Booz subió a la puerta y se sentó allí. He aquí que el
pariente cercano del que hablaba Booz pasaba por allí. Booz le dijo:
``¡Ven aquí, amigo, y siéntate!''. Se acercó y se sentó. \bibleverse{2}
Booz tomó a diez hombres de los ancianos de la ciudad y les dijo:
``Siéntate aquí'', y se sentaron. \bibleverse{3} Le dijo al pariente
cercano: ``Noemí, que ha vuelto del país de Moab, está vendiendo la
parcela que era de nuestro hermano Elimelec. \bibleverse{4} Pensé que
debía decírtelo, diciendo: `Cómpralo ante los que están aquí sentados y
ante los ancianos de mi pueblo'. Si quieres redimirla, redímela; pero si
no quieres redimirla, dímelo para que lo sepa. Porque no hay nadie que
la redima aparte de ti; y yo estoy detrás de ti''. Dijo: ``Lo
redimiré''. \footnote{\textbf{4:4} Lev 25,25}

\bibleverse{5} Entonces Booz dijo: ``El día que compres el campo de la
mano de Noemí, deberás comprárselo también a Rut la moabita, la mujer
del muerto, para levantar el nombre del muerto sobre su herencia.''
\footnote{\textbf{4:5} Deut 25,5-6}

\bibleverse{6} El pariente cercano dijo: ``No puedo redimirlo por mí
mismo, para no poner en peligro mi propia herencia. Toma para ti mi
derecho de redención, pues no puedo redimirlo''.

\bibleverse{7} Esta era la costumbre de antaño en Israel en cuanto al
rescate y al intercambio, para confirmar todas las cosas: un hombre se
quitaba la sandalia y se la daba a su vecino; y esta era la manera de
formalizar las transacciones en Israel. \footnote{\textbf{4:7} Deut
  25,7-10} \bibleverse{8} Entonces el pariente cercano dijo a Booz:
``Cómpralo para ti'', y se quitó la sandalia.

\bibleverse{9} Booz dijo a los ancianos y a todo el pueblo: ``Vosotros
sois testigos hoy de que he comprado todo lo que era de Elimelec, y todo
lo que era de Quelión y de Mahlón, de la mano de Noemí. \bibleverse{10}
Además, a Rut la moabita, esposa de Mahlón, la he comprado para que sea
mi esposa, para levantar el nombre del muerto en su herencia, para que
el nombre del muerto no sea cortado de entre sus hermanos y de la puerta
de su lugar. Vosotros sois hoy testigos''.

\bibleverse{11} Todo el pueblo que estaba en la puerta, y los ancianos,
dijeron: ``Somos testigos. Que el Señor haga que la mujer que ha entrado
en tu casa sea como Raquel y como Lea, que ambas edificaron la casa de
Israel; y que te trate dignamente en Efrata, y que seas famosa en Belén.
\bibleverse{12} Que tu casa sea como la casa de Pérez, que Tamar dio a
Judá, de la descendencia que Yahvé te dará por esta joven.'' \footnote{\textbf{4:12}
  Gén 38,29}

\hypertarget{el-matrimonio-de-des-booz-con-rut-se-completuxf3-y-fue-bendecido-con-el-nacimiento-de-obed-uxedndice-de-guxe9nero-de-puxe9rez-a-david}{%
\subsection{El matrimonio de Des Booz con Rut se completó y fue
bendecido con el nacimiento de Obed; Índice de género de Pérez a
David}\label{el-matrimonio-de-des-booz-con-rut-se-completuxf3-y-fue-bendecido-con-el-nacimiento-de-obed-uxedndice-de-guxe9nero-de-puxe9rez-a-david}}

\bibleverse{13} Booz tomó a Rut y ella se convirtió en su esposa; se
acercó a ella, y el Señor le permitió concebir, y dio a luz un hijo.
\footnote{\textbf{4:13} Sal 127,3} \bibleverse{14} Las mujeres dijeron a
Noemí: ``Bendito sea Yahvé, que no te ha dejado hoy sin pariente
cercano. Que su nombre sea famoso en Israel. \bibleverse{15} Él será
para ti un restaurador de la vida y te sostendrá en tu vejez; porque tu
nuera, que te ama, que es mejor para ti que siete hijos, lo ha dado a
luz.'' \bibleverse{16} Noemí tomó al niño, lo puso en su seno y lo
amamantó. \bibleverse{17} Las mujeres, sus vecinas, le pusieron un
nombre, diciendo: ``Le ha nacido un hijo a Noemí''. Le pusieron el
nombre de Obed. Es el padre de Jesé, el padre de David. \footnote{\textbf{4:17}
  Mat 1,5-6; Luc 3,32}

\bibleverse{18} Esta es la historia de las generaciones de Pérez: Pérez
fue padre de Hezrón, \footnote{\textbf{4:18} Gén 46,12; 1Cró 2,5}
\bibleverse{19} y Hezrón fue padre de Rama, y Rama fue padre de
Aminadab, \footnote{\textbf{4:19} 1Cró 2,9-15} \bibleverse{20} y
Aminadab fue padre de Nashón, y Nashón fue padre de Salmón, \footnote{\textbf{4:20}
  Núm 1,7} \bibleverse{21} y Salmón fue padre de Booz, y Booz fue padre
de Obed, \bibleverse{22} y Obed fue padre de Jesé, y Jesé fue padre de
David. \footnote{\textbf{4:22} 1Sam 16,1; 1Sam 16,11-13}
