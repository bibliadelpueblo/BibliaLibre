\hypertarget{el-nacimiento-y-ordenaciuxf3n-de-samuel-como-siervo-del-seuxf1or-en-silo-canciuxf3n-de-alabanza-de-hanna}{%
\subsection{El nacimiento y ordenación de Samuel como siervo del Señor
en Silo; Canción de alabanza de
Hanna}\label{el-nacimiento-y-ordenaciuxf3n-de-samuel-como-siervo-del-seuxf1or-en-silo-canciuxf3n-de-alabanza-de-hanna}}

\hypertarget{section}{%
\section{1}\label{section}}

\bibleverse{1} Había un hombre de Ramathaim Zophim, de la región
montañosa de Ephraim, que se llamaba Elkanah, hijo de Jeroham, hijo de
Elihu, hijo de Tohu, hijo de Zuph, un Ephraimita. \footnote{\textbf{1:1}
  1Cró 6,26-27; 1Cró 6,34-35} \bibleverse{2} Tenía dos esposas. Una se
llamaba Ana y la otra Penina. Penina tuvo hijos, pero Ana no tuvo hijos.
\footnote{\textbf{1:2} Gén 29,31} \bibleverse{3} Este hombre subía de su
ciudad de año en año para adorar y sacrificar a Yavé de los Ejércitos en
Silo. Los dos hijos de Elí, Ofni y Finees, sacerdotes de Yavé, estaban
allí. \footnote{\textbf{1:3} Jos 18,1} \bibleverse{4} Cuando llegó el
día en que Elcana sacrificó, dio porciones a Penina, su esposa, y a
todos sus hijos e hijas; \bibleverse{5} pero dio una porción doble a
Ana, porque amaba a Ana, pero Yavé había cerrado su vientre.
\bibleverse{6} Su rival la provocaba duramente, para irritarla, porque
Yahvé había cerrado su vientre. \bibleverse{7} Así, año tras año, cuando
subía a la casa de Yavé, su rival la provocaba. Por eso lloraba y no
comía. \bibleverse{8} Su esposo Elcana le dijo: ``Ana, ¿por qué lloras?
¿Por qué no comes? ¿Por qué está afligido tu corazón? ¿No soy yo mejor
para ti que diez hijos?''

\hypertarget{los-votos-de-hanna-en-silo-y-su-conversaciuxf3n-con-eli}{%
\subsection{Los votos de Hanna en Silo y su conversación con
Eli}\label{los-votos-de-hanna-en-silo-y-su-conversaciuxf3n-con-eli}}

\bibleverse{9} Ana se levantó cuando terminaron de comer y beber en
Silo. El sacerdote Elí estaba sentado en su silla junto al umbral del
templo de Yahvé. \bibleverse{10} Ana, amargada de alma, oró a Yahvé,
llorando amargamente. \bibleverse{11} Hizo un voto y dijo: ``Yahvé de
los Ejércitos, si en verdad miras la aflicción de tu siervo y te
acuerdas de mí, y no te olvidas de tu siervo, sino que le das a tu
siervo un hijo varón, yo se lo daré a Yahvé todos los días de su vida, y
ninguna navaja pasará por su cabeza.'' \footnote{\textbf{1:11} Núm
  6,2-21}

\bibleverse{12} Mientras ella seguía orando ante el Señor, Elí vio su
boca. \bibleverse{13} Ana hablaba en su corazón. Sólo sus labios se
movían, pero su voz no se oía. Por eso Elí pensó que estaba borracha.
\bibleverse{14} Elí le dijo: ``¿Hasta cuándo estarás borracha? Deshazte
de tu vino''.

\bibleverse{15} Ana respondió: ``No, señor mío, soy una mujer de
espíritu afligido. No he estado bebiendo vino ni bebida fuerte, sino que
he derramado mi alma ante Yahvé. \footnote{\textbf{1:15} Sal 62,8}
\bibleverse{16} No consideres a tu sierva una mujer malvada, pues he
estado hablando por la abundancia de mi queja y mi provocación.''

\bibleverse{17} Entonces Elí respondió: ``Ve en paz, y que el Dios de
Israel te conceda la petición que le has hecho.''

\bibleverse{18} Ella dijo: ``Que tu sierva encuentre gracia ante tus
ojos''. Entonces la mujer se fue y comió; y la expresión de su rostro ya
no era triste.

\hypertarget{nacimiento-de-samuel-primera-infancia-y-consagraciuxf3n-en-silo}{%
\subsection{Nacimiento de Samuel, primera infancia y consagración en
Silo}\label{nacimiento-de-samuel-primera-infancia-y-consagraciuxf3n-en-silo}}

\bibleverse{19} Se levantaron temprano por la mañana y adoraron a Yavé,
y luego regresaron y llegaron a su casa en Ramá. Entonces Elcana conoció
a su esposa Ana, y el Señor se acordó de ella. \footnote{\textbf{1:19}
  Gén 30,22}

\bibleverse{20} Llegado el momento, Ana concibió y dio a luz un hijo, al
que puso por nombre Samuel, diciendo: ``Porque se lo he pedido a
Yahvé''.

\bibleverse{21} El hombre Elcana y toda su casa subieron a ofrecer a
Yavé el sacrificio anual y su voto. \bibleverse{22} Pero Ana no subió,
porque le dijo a su marido: ``No hasta que el niño sea destetado;
entonces lo llevaré para que se presente ante Yavé y se quede allí para
siempre.''

\bibleverse{23} Su esposo Elcana le dijo: ``Haz lo que te parezca bien.
Espera hasta que lo hayas destetado; sólo que Yahvé confirme su
palabra''. La mujer esperó y amamantó a su hijo hasta que lo destetó.
\bibleverse{24} Cuando lo destetó, lo subió con ella, con tres toros, un
efa de harina y un recipiente de vino, y lo llevó a la casa de Yahvé en
Silo. El niño era pequeño. \bibleverse{25} Mataron el toro y llevaron al
niño a Elí. \bibleverse{26} Ella dijo: ``Oh, señor mío, vive tu alma,
señor mío, yo soy la mujer que estuvo aquí junto a ti, orando a Yavé.
\bibleverse{27} He rogado por este niño, y el Señor me ha concedido la
petición que le hice. \footnote{\textbf{1:27} 1Sam 1,17} \bibleverse{28}
Por eso también se lo he entregado a Yavé. Mientras viva está entregado
a Yahvé''. Allí adoró a Yahvé. \footnote{\textbf{1:28} 1Sam 1,11}

\hypertarget{himno-de-alabanza-a-hanna-inicio-de-servicio-de-samuel-en-silo}{%
\subsection{Himno de alabanza a Hanna; Inicio de servicio de Samuel en
Silo}\label{himno-de-alabanza-a-hanna-inicio-de-servicio-de-samuel-en-silo}}

\hypertarget{section-1}{%
\section{2}\label{section-1}}

\bibleverse{1} Ana oró y dijo, ``¡Mi corazón se regocija en Yahvé! Mi
cuerno está exaltado en Yahvé. Mi boca se ensancha sobre mis enemigos,
porque me alegro de tu salvación. \footnote{\textbf{2:1} Luc 1,46-55}
\bibleverse{2} No hay nadie tan santo como Yahvé, porque no hay nadie
más que tú, ni hay ninguna roca como nuestro Dios. \bibleverse{3} ``No
sigas hablando con tanto orgullo. No dejes que la arrogancia salga de tu
boca, porque Yahvé es un Dios de conocimiento. Por él se pesan las
acciones. \bibleverse{4} ``Los arcos de los poderosos están rotos. Los
que tropezaron están armados de fuerza. \bibleverse{5} Los que estaban
llenos se han alquilado por el pan. Los que tenían hambre están
satisfechos. Sí, la estéril ha dado a luz a siete. La que tiene muchos
hijos languidece. \bibleverse{6} ``Yahvé mata y da vida. Baja al Seol y
sube. \footnote{\textbf{2:6} Deut 32,39} \bibleverse{7} Yahvé empobrece
y enriquece. Él baja, también levanta. \footnote{\textbf{2:7} Sal 75,7}
\bibleverse{8} Él levanta a los pobres del polvo. Él levanta al
necesitado del estercoleropara que se sienten con los príncipesy heredar
el trono de la gloria. Porque las columnas de la tierra son de Yahvé. Ha
puesto el mundo sobre ellos. \footnote{\textbf{2:8} Sal 113,7-8}
\bibleverse{9} Él guardará los pies de sus santos, pero los malvados
serán silenciados en la oscuridad; porque ningún hombre prevalecerá por
su fuerza. \footnote{\textbf{2:9} Sal 33,16} \bibleverse{10} Los que
luchan contra Yahvé serán despedazados. Él tronará contra ellos en el
cielo. ``Yahvé juzgará los confines de la tierra. Dará fuerza a su rey,
y exaltar el cuerno de su ungido''. \footnote{\textbf{2:10} Sal 132,17}

\bibleverse{11} Elcana se fue a Ramá, a su casa. El niño sirvió a Yahvé
ante el sacerdote Elí.

\hypertarget{la-maldad-de-los-hijos-de-eluxed-anuncio-del-juicio-divino}{%
\subsection{La maldad de los hijos de Elí; Anuncio del juicio
divino}\label{la-maldad-de-los-hijos-de-eluxed-anuncio-del-juicio-divino}}

\bibleverse{12} Los hijos de Elí eran hombres malvados. No conocían a
Yavé. \bibleverse{13} La costumbre de los sacerdotes con el pueblo era
que cuando alguien ofrecía un sacrificio, el siervo del sacerdote se
acercaba, mientras la carne estaba hirviendo, con un tenedor de tres
dientes en la mano; \footnote{\textbf{2:13} Éxod 27,3} \bibleverse{14} y
lo clavaba en la sartén, o caldera, o caldero. El sacerdote tomaba para
sí todo lo que el tenedor sacaba. Esto lo hacían con todos los
israelitas que llegaban allí a Silo. \bibleverse{15} Antes de quemar la
grasa, se acercaba el criado del sacerdote y le decía al hombre que
sacrificaba: ``Da carne para asar para el sacerdote, porque no aceptará
de ti carne hervida, sino cruda.'' \footnote{\textbf{2:15} Lev 3,3-5}

\bibleverse{16} Si el hombre le decía: ``Que se queme primero la grasa,
y luego toma la cantidad que desee tu alma'', entonces le decía: ``No,
pero me la darás ahora; y si no, la tomaré por la fuerza''.
\bibleverse{17} El pecado de los jóvenes fue muy grande ante Yavé, pues
los hombres despreciaron la ofrenda de Yavé.

\hypertarget{hanna-y-el-niuxf1o-del-coro-samuel}{%
\subsection{Hanna y el niño del coro
Samuel}\label{hanna-y-el-niuxf1o-del-coro-samuel}}

\bibleverse{18} Pero Samuel ministraba ante Yavé, siendo un niño,
vestido con un efod de lino. \bibleverse{19} Además, su madre le hizo un
pequeño manto, y se lo traía de año en año cuando subía con su esposo a
ofrecer el sacrificio anual. \bibleverse{20} Elí bendijo a Elcana y a su
esposa, y dijo: ``Que Yavé les dé descendencia de esta mujer por la
petición que se le hizo a Yavé.'' Luego se fueron a su casa.
\bibleverse{21} Yahvé visitó a Ana, y ella concibió y dio a luz tres
hijos y dos hijas. El niño Samuel creció ante Yavé. \footnote{\textbf{2:21}
  Luc 1,80}

\hypertarget{las-suaves-amonestaciones-de-eluxed-a-sus-hijos-degenerados}{%
\subsection{Las suaves amonestaciones de Elí a sus hijos
degenerados}\label{las-suaves-amonestaciones-de-eluxed-a-sus-hijos-degenerados}}

\bibleverse{22} Elí era ya muy viejo, y oyó todo lo que sus hijos hacían
a todo Israel, y cómo se acostaban con las mujeres que servían a la
puerta de la Tienda de las Reuniones. \footnote{\textbf{2:22} Éxod 38,8}
\bibleverse{23} Les dijo: ``¿Por qué hacéis tales cosas? Porque me he
enterado de vuestros malos tratos por parte de todo este pueblo.
\bibleverse{24} ¡No, hijos míos, porque no es una buena noticia lo que
oigo! Ustedes hacen desobedecer al pueblo de Yavé. \bibleverse{25} Si un
hombre peca contra otro, Dios lo juzgará; pero si un hombre peca contra
Yavé, ¿quién intercederá por él?'' No obstante, no escucharon la voz de
su padre, porque Yahvé pretendía matarlos.

\bibleverse{26} El niño Samuel crecía y aumentaba el favor de Yahvé y de
los hombres. \footnote{\textbf{2:26} Luc 2,52}

\hypertarget{refruxe1n-del-profeta-anuncio-de-la-cauxedda-de-eli-y-su-casa}{%
\subsection{Refrán del Profeta: Anuncio de la caída de Eli y su
casa}\label{refruxe1n-del-profeta-anuncio-de-la-cauxedda-de-eli-y-su-casa}}

\bibleverse{27} Un hombre de Dios se acercó a Elí y le dijo: ``Yahvé
dice: `¿Acaso me revelé a la casa de tu padre cuando estaban en Egipto
en la esclavitud de la casa del faraón? \bibleverse{28} ¿Acaso no lo
elegí de entre todas las tribus de Israel para que fuera mi sacerdote,
para que subiera a mi altar, quemara incienso y llevara un efod ante mí?
¿No le di a la casa de su padre todas las ofrendas de los hijos de
Israel hechas por el fuego? \footnote{\textbf{2:28} Núm 18,8}
\bibleverse{29} ¿Por qué pateáis mi sacrificio y mi ofrenda, que yo he
ordenado en mi morada, y honráis a vuestros hijos por encima de mí, para
engordaros con lo mejor de todas las ofrendas de Israel, mi pueblo?'
\bibleverse{30} ``Por tanto, Yahvé, el Dios de Israel, dice:
`Ciertamente dije que tu casa y la casa de tu padre andarían delante de
mí para siempre'. Pero ahora Yahvé dice: `Lejos de mí; porque a los que
me honran los honraré, y a los que me desprecian los maldeciré.
\footnote{\textbf{2:30} Éxod 28,1} \bibleverse{31} He aquí, vienen los
días en que cortaré tu brazo y el brazo de la casa de tu padre, para que
no haya un anciano en tu casa. \footnote{\textbf{2:31} 1Re 2,27}
\bibleverse{32} Verás la aflicción de mi morada, en toda la riqueza que
daré a Israel. No habrá un anciano en tu casa para siempre.
\bibleverse{33} El hombre tuyo que no corte de mi altar consumirá tus
ojos y entristecerá tu corazón. Todo el aumento de tu casa morirá en la
flor de su edad. \footnote{\textbf{2:33} 1Sam 22,20} \bibleverse{34}
Esta será la señal que te llegará sobre tus dos hijos, sobre Ofni y
Finees: en un solo día morirán los dos. \footnote{\textbf{2:34} 1Sam
  4,11} \bibleverse{35} Yo me levantaré un sacerdote fiel que hará lo
que está en mi corazón y en mi mente. Le construiré una casa segura. Él
caminará delante de mi ungido para siempre. \bibleverse{36} Sucederá que
todos los que queden en tu casa vendrán y se inclinarán ante él por una
pieza de plata y un pan, y dirán: ``Por favor, ponme en uno de los
oficios de los sacerdotes, para que pueda comer un bocado de pan''\,''.

\hypertarget{dios-se-revela-a-samuel-y-anuncia-la-cauxedda-de-la-casa-de-eluxed}{%
\subsection{Dios se revela a Samuel y anuncia la caída de la casa de
Elí}\label{dios-se-revela-a-samuel-y-anuncia-la-cauxedda-de-la-casa-de-eluxed}}

\hypertarget{section-2}{%
\section{3}\label{section-2}}

\bibleverse{1} El niño Samuel ministraba a Yahvé ante Elí. La palabra de
Yahvé era rara en aquellos días. No había entonces muchas visiones.
\footnote{\textbf{3:1} Am 8,11} \bibleverse{2} En aquel tiempo, cuando
Elí estaba acostado en su lugar (ahora sus ojos habían comenzado a
oscurecerse, de modo que no podía ver), \bibleverse{3} y la lámpara de
Dios aún no se había apagado, y Samuel se había acostado en el templo de
Yavé donde estaba el arca de Dios, \bibleverse{4} Yavé llamó a Samuel.
Le dijo: ``Aquí estoy''.

\bibleverse{5} Corrió hacia Elí y le dijo: ``Aquí estoy, porque me has
llamado''. Dijo: ``No he llamado. Acuéstate de nuevo''. Fue y se acostó.
\bibleverse{6} Yahvé volvió a llamar: ``¡Samuel!'' Samuel se levantó y
se dirigió a Elí y le dijo: ``Aquí estoy, porque me has llamado''.
Respondió: ``No he llamado, hijo mío. Vuelve a acostarte''.
\bibleverse{7} Samuel aún no conocía a Yavé ni se le había revelado la
palabra de Yavé. \bibleverse{8} Yahvé volvió a llamar a Samuel por
tercera vez. Se levantó y fue a Elí y le dijo: ``Aquí estoy, porque me
has llamado''. Elí percibió que Yahvé había llamado al niño.
\bibleverse{9} Por eso Elí dijo a Samuel: ``Ve, acuéstate. Si te llama,
dirás: `Habla, Yahvé, porque tu siervo escucha'\,''. Entonces Samuel fue
y se acostó en su lugar. \bibleverse{10} Llegó Yahvé, se puso de pie y
llamó como otras veces: ``¡Samuel! Samuel!'' Entonces Samuel dijo:
``Habla, que tu siervo oye''.

\bibleverse{11} El Señor dijo a Samuel: ``He aquí que yo haré en Israel
una cosa que hará vibrar los oídos de todo el que la oiga.
\bibleverse{12} En aquel día cumpliré contra Elí todo lo que he dicho
sobre su casa, desde el principio hasta el fin. \bibleverse{13} Porque
le he dicho que juzgaré a su casa para siempre por la iniquidad que él
conoció, porque sus hijos trajeron una maldición sobre sí mismos, y él
no los refrenó. \footnote{\textbf{3:13} 1Sam 2,27-36} \bibleverse{14}
Por eso he jurado a la casa de Elí que la iniquidad de la casa de Elí no
se quitará con sacrificio ni con ofrenda para siempre.''

\hypertarget{samuel-comparte-la-revelaciuxf3n-con-eluxed-y-comienza-su-trabajo-como-profeta-para-todo-israel}{%
\subsection{Samuel comparte la revelación con Elí y comienza su trabajo
como profeta para todo
Israel}\label{samuel-comparte-la-revelaciuxf3n-con-eluxed-y-comienza-su-trabajo-como-profeta-para-todo-israel}}

\bibleverse{15} Samuel se acostó hasta la mañana y abrió las puertas de
la casa de Yahvé. Samuel tenía miedo de mostrarle a Elí la visión.
\bibleverse{16} Entonces Elí llamó a Samuel y le dijo: ``¡Samuel, hijo
mío!'' Dijo: ``Aquí estoy''.

\bibleverse{17} Él dijo: ``¿Qué es lo que te ha dicho? Por favor, no me
lo ocultes. Dios te lo haga, y más aún, si me ocultas algo de todo lo
que te ha hablado''.

\bibleverse{18} Samuel le contó todo y no le ocultó nada. Dijo: ``Es
Yahvé. Que haga lo que le parezca bien''. \footnote{\textbf{3:18} 2Sam
  15,26}

\bibleverse{19} Samuel crecía, y el Señor estaba con él y no dejaba que
ninguna de sus palabras cayera en tierra. \bibleverse{20} Todo Israel,
desde Dan hasta Beerseba, sabía que Samuel había sido establecido como
profeta de Yavé. \bibleverse{21} Yahvé volvió a aparecer en Silo; porque
Yahvé se reveló a Samuel en Silo por palabra de Yahvé.

\hypertarget{die-bundeslade-ins-lager-der-israeliten-geholt}{%
\subsection{Die Bundeslade ins Lager der Israeliten
geholt}\label{die-bundeslade-ins-lager-der-israeliten-geholt}}

\hypertarget{section-3}{%
\section{4}\label{section-3}}

\bibleverse{1} La palabra de Samuel llegó a todo Israel. Salió Israel
contra los filisteos para combatir, y acampó junto a Ebenezer, y los
filisteos acamparon en Afec. \footnote{\textbf{4:1} Jos 15,53}
\bibleverse{2} Los filisteos se pusieron en fila contra Israel. Cuando
entraron en combate, Israel fue derrotado por los filisteos, que mataron
en el campo a unos cuatro mil hombres del ejército. \bibleverse{3}
Cuando el pueblo entró en el campamento, los ancianos de Israel dijeron:
``¿Por qué el Señor nos ha derrotado hoy ante los filisteos? Saquemos de
Silo el arca de la alianza de Yavé y traigámosla, para que venga entre
nosotros y nos salve de la mano de nuestros enemigos.'' \footnote{\textbf{4:3}
  1Sam 14,18}

\bibleverse{4} Entonces el pueblo envió a Silo, y trajeron de allí el
arca de la alianza de Yavé de los Ejércitos, que está sentada sobre los
querubines; y los dos hijos de Elí, Ofni y Finees, estaban allí con el
arca de la alianza de Dios. \footnote{\textbf{4:4} 2Sam 6,2}

\hypertarget{el-efecto-de-este-evento-en-las-partes-en-conflicto-derrota-de-los-israelitas-y-puxe9rdida-del-arca}{%
\subsection{El efecto de este evento en las partes en conflicto; Derrota
de los israelitas y pérdida del
arca}\label{el-efecto-de-este-evento-en-las-partes-en-conflicto-derrota-de-los-israelitas-y-puxe9rdida-del-arca}}

\bibleverse{5} Cuando el arca de la alianza de Yavé entró en el
campamento, todo Israel gritó con un gran alarido, de modo que la tierra
resonó. \bibleverse{6} Cuando los filisteos oyeron el ruido del grito,
dijeron: ``¿Qué significa el ruido de este gran grito en el campamento
de los hebreos?'' Comprendieron que el arca de Yavé había entrado en el
campamento. \bibleverse{7} Los filisteos se asustaron, pues dijeron:
``Dios ha entrado en el campamento''. Dijeron: ``¡Ay de nosotros! Porque
nunca antes había ocurrido algo semejante. \bibleverse{8} ¡Ay de
nosotros! ¿Quién nos librará de la mano de estos poderosos dioses? Estos
son los dioses que golpearon a los egipcios con toda clase de plagas en
el desierto. \bibleverse{9} Fortaleceos y comportaos como hombres, oh
filisteos, para que no seáis siervos de los hebreos, como ellos lo han
sido de vosotros. Fortalézcanse como hombres y luchen''. \footnote{\textbf{4:9}
  Jue 13,1} \bibleverse{10} Los filisteos lucharon, e Israel fue
derrotado, y cada uno huyó a su tienda. Hubo una matanza muy grande,
pues cayeron treinta mil hombres de a pie de Israel. \bibleverse{11} El
arca de Dios fue tomada, y los dos hijos de Elí, Ofni y Finees, fueron
asesinados.

\hypertarget{los-tristes-efectos-del-mensaje-en-shiloh-la-muerte-de-eli-y-su-nuera}{%
\subsection{Los tristes efectos del mensaje en Shiloh; la muerte de Eli
y su
nuera}\label{los-tristes-efectos-del-mensaje-en-shiloh-la-muerte-de-eli-y-su-nuera}}

\bibleverse{12} Un hombre de Benjamín salió corriendo del ejército y
llegó a Silo ese mismo día, con la ropa rota y con tierra en la cabeza.
\bibleverse{13} Cuando llegó, he aquí que Elí estaba sentado en su
asiento junto al camino, vigilando, porque su corazón temía el arca de
Dios. Cuando el hombre llegó a la ciudad y lo contó, toda la ciudad
gritó. \bibleverse{14} Cuando Elí oyó el ruido del clamor, dijo: ``¿Qué
significa el ruido de este tumulto?'' El hombre se apresuró y vino a
contárselo a Elí. \bibleverse{15} Elí tenía noventa y ocho años. Sus
ojos estaban entornados, de modo que no podía ver. \footnote{\textbf{4:15}
  1Sam 3,2} \bibleverse{16} El hombre le dijo a Elí: ``Yo soy el que
salió del ejército, y hoy he huido del ejército''. Dijo: ``¿Cómo fue el
asunto, hijo mío?''

\bibleverse{17} El que trajo la noticia respondió: ``Israel ha huido
ante los filisteos, y también ha habido una gran matanza entre el
pueblo. También tus dos hijos, Ofni y Finees, han muerto, y el arca de
Dios ha sido capturada.''

\bibleverse{18} Cuando mencionó el arca de Dios, Elí se cayó de su
asiento hacia atrás, al lado de la puerta, y su cuello se quebró y
murió, pues era un hombre viejo y pesado. Había juzgado a Israel durante
cuarenta años.

\bibleverse{19} Su nuera, la mujer de Finees, estaba encinta, a punto de
dar a luz. Cuando oyó la noticia de que el arca de Dios había sido
tomada y que su suegro y su marido habían muerto, se inclinó y dio a
luz, pues le sobrevinieron los dolores. \bibleverse{20} A punto de
morir, las mujeres que estaban junto a ella le dijeron: ``No temas,
porque has dado a luz un hijo''. Pero ella no respondió, ni lo
consideró. \footnote{\textbf{4:20} Gén 35,17} \bibleverse{21} Le puso al
niño el nombre de Icabod, diciendo: ``¡La gloria se ha ido de Israel!'',
porque el arca de Dios fue tomada, y por su suegro y su marido.
\footnote{\textbf{4:21} Sal 78,61} \bibleverse{22} Ella dijo: ``La
gloria se ha alejado de Israel, porque el arca de Dios ha sido tomada''.

\hypertarget{en-la-tierra-de-los-filisteos-el-arca-estuxe1-causando-estragos-en-varias-ciudades}{%
\subsection{En la tierra de los filisteos, el arca está causando
estragos en varias
ciudades}\label{en-la-tierra-de-los-filisteos-el-arca-estuxe1-causando-estragos-en-varias-ciudades}}

\hypertarget{section-4}{%
\section{5}\label{section-4}}

\bibleverse{1} Los filisteos tomaron el arca de Dios y la llevaron de
Ebenezer a Asdod. \bibleverse{2} Los filisteos tomaron el arca de Dios,
la llevaron a la casa de Dagón y la pusieron junto a Dagón. \footnote{\textbf{5:2}
  Jue 16,23} \bibleverse{3} Cuando el pueblo de Asdod se levantó
temprano al día siguiente, he aquí que Dagón había caído de bruces al
suelo ante el arca de Dios. Tomaron a Dagón y lo volvieron a colocar en
su lugar. \bibleverse{4} Al día siguiente, cuando se levantaron de
madrugada, vieron que Dagón había caído de bruces al suelo ante el arca
de Yahvé, y que la cabeza de Dagón y las dos palmas de sus manos estaban
cortadas en el umbral. Sólo el torso de Dagón estaba intacto.
\bibleverse{5} Por eso, ni los sacerdotes de Dagón ni los que entran en
la casa de Dagón pisan el umbral de Dagón en Asdod hasta el día de hoy.
\bibleverse{6} Pero la mano de Yavé se ensañó con el pueblo de Asdod, y
lo destruyó y lo golpeó con tumores, incluso a Asdod y sus fronteras.
\footnote{\textbf{5:6} Sal 78,66}

\bibleverse{7} Cuando los hombres de Asdod vieron que era así, dijeron:
``El arca del Dios de Israel no se quedará con nosotros, porque su mano
es severa con nosotros y con Dagón, nuestro dios.'' \bibleverse{8}
Enviaron, pues, a reunir a todos los señores de los filisteos y dijeron:
``¿Qué haremos con el arca del Dios de Israel?'' Respondieron: ``Que el
arca del Dios de Israel sea llevada a Gat''. Llevaron allí el arca del
Dios de Israel. \bibleverse{9} Y cuando la llevaron allí, la mano de
Yavé se abatió sobre la ciudad con una gran confusión, e hirió a los
hombres de la ciudad, tanto a los pequeños como a los grandes, de modo
que los tumores estallaron sobre ellos. \bibleverse{10} Entonces
enviaron el arca de Dios a Ecrón. Cuando el arca de Dios llegó a Ecrón,
los ecronitas gritaron diciendo: ``Han traído aquí el arca del Dios de
Israel para matarnos a nosotros y a nuestro pueblo.'' \bibleverse{11}
Enviaron, pues, y reunieron a todos los señores de los filisteos, y
dijeron: ``Despide el arca del Dios de Israel y que vuelva a su lugar,
para que no nos mate a nosotros y a nuestro pueblo.'' Porque hubo un
pánico mortal en toda la ciudad. La mano de Dios estaba muy pesada allí.
\bibleverse{12} Los hombres que no murieron fueron alcanzados por los
tumores, y el clamor de la ciudad subió al cielo.

\hypertarget{resoluciuxf3n-de-los-filisteos-sobre-el-regreso-del-arca}{%
\subsection{Resolución de los filisteos sobre el regreso del
arca}\label{resoluciuxf3n-de-los-filisteos-sobre-el-regreso-del-arca}}

\hypertarget{section-5}{%
\section{6}\label{section-5}}

\bibleverse{1} El arca de Yavé estuvo siete meses en el país de los
filisteos. \bibleverse{2} Los filisteos llamaron a los sacerdotes y a
los adivinos, diciendo: ``¿Qué haremos con el arca de Yavé? Muéstranos
cómo debemos enviarla a su lugar''.

\bibleverse{3} Ellos dijeron: ``Si envías el arca del Dios de Israel, no
la envíes vacía, sino que por todos los medios devuélvele una ofrenda
por la culpa. Entonces quedarás curado, y se sabrá por qué su mano no se
aparta de ti''.

\bibleverse{4} Entonces dijeron: ``¿Cuál debe ser la ofrenda por la
culpa que le devolveremos?'' Dijeron: ``Cinco tumores de oro y cinco
ratones de oro, por el número de los señores de los filisteos; porque
una sola plaga fue sobre todos vosotros y sobre vuestros señores.
\footnote{\textbf{6:4} Jos 13,3} \bibleverse{5} Por tanto, haréis
imágenes de vuestros tumores e imágenes de vuestros ratones que
estropean la tierra, y daréis gloria al Dios de Israel. Tal vez él
libere su mano de vosotros, de vuestros dioses y de vuestra tierra.
\bibleverse{6} ¿Por qué, pues, endurecéis vuestros corazones como
endurecieron los egipcios y el Faraón? Cuando había obrado
maravillosamente entre ellos, ¿no dejaron ir al pueblo y se marcharon?
\footnote{\textbf{6:6} Éxod 8,15; Éxod 12,31}

\bibleverse{7} ``Ahora, pues, tomad y preparad un carro nuevo y dos
vacas lecheras en las que no haya yugo; atad las vacas al carro y llevad
de ellas a casa sus terneros; \bibleverse{8} y tomad el arca de Yahvé y
ponedla sobre el carro. Pon las joyas de oro, que le devuelves como
ofrenda por la culpa, en una caja a su lado; y envíala para que se vaya.
\bibleverse{9} He aquí, si sube por el camino de su propia frontera
hasta Bet Semes, entonces él nos ha hecho este gran mal; pero si no,
entonces sabremos que no es su mano la que nos golpeó. Fue una
casualidad que nos sucedió''.

\hypertarget{ejecuciuxf3n-de-la-resoluciuxf3n-llegada-y-recepciuxf3n-del-arca-en-bet-semes}{%
\subsection{Ejecución de la resolución; Llegada y recepción del arca en
Bet-semes}\label{ejecuciuxf3n-de-la-resoluciuxf3n-llegada-y-recepciuxf3n-del-arca-en-bet-semes}}

\bibleverse{10} Así lo hicieron los hombres, quienes tomaron dos vacas
lecheras, las ataron al carro y encerraron a sus terneros en casa.
\bibleverse{11} Pusieron el arca de Yahvé en el carro, y la caja con los
ratones de oro y las imágenes de sus tumores. \bibleverse{12} Las vacas
tomaron el camino recto por la vía de Bet Shemesh. Iban por el camino,
mugiendo a su paso, y no se apartaban ni a la derecha ni a la izquierda;
y los señores de los filisteos iban tras ellas hasta el límite de Bet
Semes. \bibleverse{13} Los habitantes de Bet Semes estaban segando su
cosecha de trigo en el valle, y alzando los ojos vieron el arca y se
alegraron de verla. \bibleverse{14} El carro llegó al campo de Josué de
Bet Semes, y se detuvo allí, donde había una gran piedra. Entonces
partieron la madera del carro y ofrecieron las vacas en holocausto a
Yavé. \bibleverse{15} Los levitas bajaron el arca de Yavé y el cofre que
la acompañaba, en el que estaban las joyas de oro, y los pusieron sobre
la gran piedra; y los hombres de Bet Shemesh ofrecieron ese mismo día
holocaustos y sacrificios a Yavé. \bibleverse{16} Cuando los cinco
señores de los filisteos lo vieron, volvieron a Ecrón el mismo día.
\bibleverse{17} Estas son las tumbas de oro que los filisteos
devolvieron como ofrenda por la culpa a Yavé: por Asdod una, por Gaza
una, por Ascalón una, por Gat una, por Ecrón una; \bibleverse{18} y los
ratones de oro, según el número de todas las ciudades de los filisteos
que pertenecían a los cinco señores, tanto de las ciudades fortificadas
como de las aldeas rurales, hasta la gran piedra sobre la que
depositaron el arca de Yavé. Esa piedra permanece hasta hoy en el campo
de Josué de Bet Semes.

\hypertarget{se-instala-el-arca-en-quiriat-jearim}{%
\subsection{Se instala el arca en
Quiriat-Jearim}\label{se-instala-el-arca-en-quiriat-jearim}}

\bibleverse{19} Hirió a los hombres de Bet Semes, porque habían mirado
el arca de Yavé, hirió a cincuenta mil setenta de los hombres. Entonces
el pueblo se lamentó, porque Yavé había herido al pueblo con una gran
matanza. \footnote{\textbf{6:19} Núm 4,20; 2Sam 6,6-7} \bibleverse{20}
Los hombres de Bet Semes dijeron: ``¿Quién podrá estar frente a Yavé,
este Dios santo? ¿A quién subirá de nosotros?''

\bibleverse{21} Enviaron mensajeros a los habitantes de Quiriat Jearim,
diciendo: ``Los filisteos han traído de vuelta el arca de Yahvé. Bajen y
llévenla a ustedes''.

\hypertarget{section-6}{%
\section{7}\label{section-6}}

\bibleverse{1} Los hombres de Quiriat Jearim vinieron y tomaron el arca
de Yavé, y la llevaron a la casa de Abinadab en la colina, y consagraron
a Eleazar, su hijo, para que guardara el arca de Yavé.

\hypertarget{los-israelitas-se-vuelven-arrepentidos-a-dios}{%
\subsection{Los israelitas se vuelven arrepentidos a
Dios}\label{los-israelitas-se-vuelven-arrepentidos-a-dios}}

\bibleverse{2} Desde el día en que el arca permaneció en Quiriat Jearim,
el tiempo se prolongó, pues fueron veinte años; y toda la casa de Israel
se lamentaba en pos de Yavé. \footnote{\textbf{7:2} 1Cró 13,6}
\bibleverse{3} Samuel habló a toda la casa de Israel, diciendo: ``Si
volvéis a Yahvé de todo corazón, quitad de en medio los dioses
extranjeros y el Astarot, y dirigid vuestro corazón a Yahvé, y servidle
sólo a él; y él os librará de la mano de los filisteos.'' \footnote{\textbf{7:3}
  Gén 35,2; Jos 24,23} \bibleverse{4} Entonces los hijos de Israel
eliminaron a los baales y a Astarot, y sólo sirvieron a Yavé.
\footnote{\textbf{7:4} Jue 10,6; Jue 10,16}

\hypertarget{la-intercesiuxf3n-y-el-sacrificio-de-samuel-por-israel-en-mizpa-derrota-de-los-filisteos-la-piedra-eben-eser}{%
\subsection{La intercesión y el sacrificio de Samuel por Israel en
Mizpa; Derrota de los filisteos; la piedra
Eben-Eser}\label{la-intercesiuxf3n-y-el-sacrificio-de-samuel-por-israel-en-mizpa-derrota-de-los-filisteos-la-piedra-eben-eser}}

\bibleverse{5} Samuel dijo: ``Reúnan a todo Israel en Mizpa, y yo oraré
a Yavé por ustedes.'' \footnote{\textbf{7:5} 1Sam 10,17; Jue 11,11; Jue
  20,1} \bibleverse{6} Se reunieron en Mizpa, sacaron agua y la
derramaron ante Yavé, y ese día ayunaron y dijeron allí: ``Hemos pecado
contra Yavé.'' Samuel juzgó a los hijos de Israel en Mizpa.

\bibleverse{7} Cuando los filisteos oyeron que los hijos de Israel
estaban reunidos en Mizpa, los señores de los filisteos subieron contra
Israel. Cuando los hijos de Israel lo oyeron, tuvieron miedo de los
filisteos. \bibleverse{8} Los hijos de Israel dijeron a Samuel: ``No
dejes de clamar por nosotros a Yavé, nuestro Dios, para que nos salve de
la mano de los filisteos.'' \footnote{\textbf{7:8} 1Sam 12,23}
\bibleverse{9} Samuel tomó un cordero lechal y lo ofreció en holocausto
a Yavé. Samuel clamó a Yavé por Israel, y Yavé le respondió.
\bibleverse{10} Mientras Samuel ofrecía el holocausto, los filisteos se
acercaron para combatir contra Israel; pero aquel día Yavé tronó con
gran estruendo sobre los filisteos y los confundió, y fueron derribados
ante Israel. \bibleverse{11} Los hombres de Israel salieron de Mizpa y
persiguieron a los filisteos, y los golpearon hasta que llegaron debajo
de Bet Kar.

\bibleverse{12} Entonces Samuel tomó una piedra y la puso entre Mizpa y
Shen, y la llamó Ebenezer, diciendo: ``El Señor nos ha ayudado hasta
ahora.''

\hypertarget{estado-de-paz-en-el-pauxeds-la-eficacia-de-samuel-como-juez}{%
\subsection{Estado de paz en el país; La eficacia de Samuel como
juez}\label{estado-de-paz-en-el-pauxeds-la-eficacia-de-samuel-como-juez}}

\bibleverse{13} Así los filisteos fueron sometidos y dejaron de entrar
en la frontera de Israel. La mano de Yavé estuvo contra los filisteos
todos los días de Samuel.

\bibleverse{14} Las ciudades que los filisteos habían arrebatado a
Israel fueron devueltas a éste, desde Ecrón hasta Gat, e Israel recuperó
su frontera de manos de los filisteos. Hubo paz entre Israel y los
amorreos.

\bibleverse{15} Samuel juzgó a Israel todos los días de su vida.
\bibleverse{16} Iba de año en año en un circuito a Betel, Gilgal y
Mizpa, y juzgaba a Israel en todos esos lugares. \bibleverse{17} Su
regreso fue a Ramá, porque allí estaba su casa, y allí juzgó a Israel; y
allí construyó un altar a Yahvé.

\hypertarget{el-deseo-de-israel-por-un-rey-la-demanda-del-pueblo-despierta-el-disgusto-de-samuel-pero-encuentra-la-aprobaciuxf3n-de-dios}{%
\subsection{El deseo de Israel por un rey; La demanda del pueblo
despierta el disgusto de Samuel, pero encuentra la aprobación de
Dios}\label{el-deseo-de-israel-por-un-rey-la-demanda-del-pueblo-despierta-el-disgusto-de-samuel-pero-encuentra-la-aprobaciuxf3n-de-dios}}

\hypertarget{section-7}{%
\section{8}\label{section-7}}

\bibleverse{1} Cuando Samuel envejeció, puso a sus hijos como jueces de
Israel. \footnote{\textbf{8:1} 1Cró 6,28} \bibleverse{2} El nombre de su
primogénito fue Joel, y el del segundo, Abías. Fueron jueces en
Beerseba. \bibleverse{3} Sus hijos no siguieron sus caminos, sino que se
apartaron en pos de ganancias deshonestas, aceptaron sobornos y
pervirtieron la justicia. \footnote{\textbf{8:3} Deut 16,19}

\bibleverse{4} Entonces se reunieron todos los ancianos de Israel y
vinieron a Samuel a Ramá. \footnote{\textbf{8:4} 1Sam 7,17}
\bibleverse{5} Le dijeron: ``Mira que eres viejo, y tus hijos no andan
por tus caminos. Haznos ahora un rey que nos juzgue como a todas las
naciones''. \footnote{\textbf{8:5} Deut 17,14; Os 13,10; Hech 13,21}
\bibleverse{6} Pero a Samuel le disgustó que dijeran: ``Danos un rey que
nos juzgue''. Samuel oró a Yahvé. \bibleverse{7} Yahvé le dijo a Samuel:
``Escucha la voz del pueblo en todo lo que te diga; porque no te han
rechazado a ti, sino que me han rechazado a mí como rey sobre ellos.
\bibleverse{8} Según todas las obras que han hecho desde el día en que
los saqué de Egipto hasta hoy, en que me han abandonado y han servido a
otros dioses, así hacen también contigo. \bibleverse{9} Ahora, pues,
escucha su voz. Sin embargo, protestarás solemnemente ante ellos, y les
mostrarás el camino del rey que reinará sobre ellos.''

\hypertarget{samuel-le-dice-a-la-gente-los-derechos-de-un-rey}{%
\subsection{Samuel le dice a la gente los derechos de un
rey}\label{samuel-le-dice-a-la-gente-los-derechos-de-un-rey}}

\bibleverse{10} Samuel contó todas las palabras de Yahvé al pueblo que
le pedía un rey. \bibleverse{11} Dijo: ``Este será el camino del rey que
reinará sobre ustedes: tomará a sus hijos y los designará como sus
servidores, para sus carros y para ser sus jinetes; y correrán delante
de sus carros. \bibleverse{12} Los nombrará para él como capitanes de
millares y capitanes de cincuenta; y asignará a algunos para arar su
tierra y segar su cosecha, y para hacer sus instrumentos de guerra y los
instrumentos de sus carros. \bibleverse{13} Tomará a vuestras hijas para
que sean perfumistas, cocineras y panaderas. \bibleverse{14} Tomará tus
campos, tus viñedos y tus olivares, incluso los mejores, y los dará a
sus siervos. \bibleverse{15} Tomará la décima parte de tus semillas y de
tus viñedos, y se la dará a sus funcionarios y a sus siervos.
\bibleverse{16} Tomará tus siervos, tus siervas, tus mejores jóvenes y
tus asnos, y los destinará a su propio trabajo. \bibleverse{17} Tomará
la décima parte de vuestros rebaños, y vosotros seréis sus siervos.
\bibleverse{18} Aquel día gritaréis a causa de vuestro rey que habréis
elegido para vosotros, y el Señor no os responderá en aquel día.''

\hypertarget{la-gente-persiste-en-su-demanda-la-aprobaciuxf3n-de-dios}{%
\subsection{La gente persiste en su demanda; La aprobación de
dios}\label{la-gente-persiste-en-su-demanda-la-aprobaciuxf3n-de-dios}}

\bibleverse{19} Pero el pueblo se negó a escuchar la voz de Samuel y
dijo: ``No, sino que tendremos un rey sobre nosotros, \bibleverse{20}
para que también seamos como todas las naciones, y para que nuestro rey
nos juzgue y salga delante de nosotros y pelee nuestras batallas.''

\bibleverse{21} Samuel escuchó todas las palabras del pueblo y las
ensayó en los oídos de Yavé. \bibleverse{22} Yahvé dijo a Samuel:
``Escucha su voz y hazles un rey''. Samuel dijo a los hombres de Israel:
``Que cada uno se vaya a su ciudad''. \footnote{\textbf{8:22} 1Sam 8,7;
  1Sam 8,9}

\hypertarget{sauxfal-llega-a-la-casa-de-samuel-en-busca-de-los-asnos-de-su-padre}{%
\subsection{Saúl llega a la casa de Samuel en busca de los asnos de su
padre}\label{sauxfal-llega-a-la-casa-de-samuel-en-busca-de-los-asnos-de-su-padre}}

\hypertarget{section-8}{%
\section{9}\label{section-8}}

\bibleverse{1} Había un hombre de Benjamín que se llamaba Cis, hijo de
Abiel, hijo de Zeror, hijo de Becorat, hijo de Afía, hijo de un
benjamita, hombre valiente. \bibleverse{2} Tenía un hijo que se llamaba
Saúl, un joven impresionante, y no había entre los hijos de Israel una
persona más hermosa que él. Desde los hombros hacia arriba era más alto
que cualquiera del pueblo.

\bibleverse{3} Los asnos de Cis, padre de Saúl, se perdieron. Kish dijo
a su hijo Saúl: ``Toma ahora uno de los criados contigo, y levántate, ve
a buscar los asnos''. \bibleverse{4} Atravesó la región montañosa de
Efraín y pasó por la tierra de Salisá, pero no los encontraron. Luego
pasaron por la tierra de Shaalim, y no estaban allí. Luego pasó por la
tierra de los benjamitas, pero no los encontraron. \footnote{\textbf{9:4}
  Juan 3,23}

\bibleverse{5} Cuando llegaron a la tierra de Zuf, Saúl dijo a su criado
que estaba con él: ``¡Ven! Volvamos, no sea que mi padre deje de
preocuparse por los asnos y se inquiete por nosotros''. \footnote{\textbf{9:5}
  1Sam 10,2}

\bibleverse{6} El criado le dijo: ``He aquí que hay un hombre de Dios en
esta ciudad, y es un hombre al que se le tiene en honor. Todo lo que él
dice, ciertamente sucede. Ahora vayamos allí. Tal vez él pueda
indicarnos el camino a seguir''.

\bibleverse{7} Entonces Saúl dijo a su criado: ``Pero si vamos, ¿qué le
vamos a llevar al hombre? Porque el pan se ha gastado en nuestros sacos,
y no hay regalo que llevar al hombre de Dios. ¿Qué tenemos?''

\bibleverse{8} El criado volvió a responder a Saúl y le dijo: ``Mira,
tengo en mi mano la cuarta parte de un siclo de plata. Se lo daré al
hombre de Dios, para que nos indique nuestro camino''. \bibleverse{9}
(En tiempos anteriores en Israel, cuando un hombre iba a consultar a
Dios, decía: ``¡Ven! Vayamos al vidente''; pues el que ahora se llama
profeta, antes se llamaba vidente). \footnote{\textbf{9:9} 2Re 17,13;
  1Cró 9,22; Núm 24,3}

\hypertarget{la-cuxe1lida-bienvenida-de-sauxfal-y-el-trato-honorable-de-parte-de-samuel}{%
\subsection{La cálida bienvenida de Saúl y el trato honorable de parte
de
Samuel}\label{la-cuxe1lida-bienvenida-de-sauxfal-y-el-trato-honorable-de-parte-de-samuel}}

\bibleverse{10} Entonces Saúl dijo a su criado: ``Bien dicho. Ven.
Vamos''. Y fueron a la ciudad donde estaba el hombre de Dios.
\bibleverse{11} Mientras subían la cuesta de la ciudad, encontraron a
unas jóvenes que salían a sacar agua, y les dijeron: ``¿Está aquí el
vidente?''

\bibleverse{12} Ellos les respondieron y dijeron: ``Es él. Mirad, está
delante de vosotros. Daos prisa, porque ha venido hoy a la ciudad;
porque el pueblo tiene hoy un sacrificio en el lugar alto.
\bibleverse{13} En cuanto hayáis entrado en la ciudad, lo encontraréis
inmediatamente antes de que suba al lugar alto a comer; porque el pueblo
no comerá hasta que él llegue, porque él bendice el sacrificio. Después
comen los invitados. Ahora, pues, sube; porque a esta hora lo
encontrarás''.

\bibleverse{14} Subieron a la ciudad. Cuando llegaron a la ciudad, he
aquí que Samuel salió hacia ellos para subir al lugar alto.

\bibleverse{15} Un día antes de que llegara Saúl, Yahvé se lo había
revelado a Samuel, diciendo: \bibleverse{16} ``Mañana a esta hora te
enviaré un hombre de la tierra de Benjamín, y lo ungirás para que sea
príncipe de mi pueblo Israel. Él salvará a mi pueblo de la mano de los
filisteos; porque yo he mirado a mi pueblo, porque su clamor ha llegado
hasta mí.''

\bibleverse{17} Cuando Samuel vio a Saúl, Yahvé le dijo: ``¡Mira al
hombre del que te hablé! Él tendrá autoridad sobre mi pueblo''.

\bibleverse{18} Entonces Saúl se acercó a Samuel en la puerta y le dijo:
``Por favor, dime dónde está la casa del vidente''.

\bibleverse{19} Samuel respondió a Saúl y le dijo: ``Yo soy el vidente.
Sube delante de mí al lugar alto, porque hoy vas a comer conmigo. Por la
mañana te dejaré ir y te diré todo lo que hay en tu corazón.
\bibleverse{20} En cuanto a tus asnos que se perdieron hace tres días,
no te preocupes por ellos, pues han sido encontrados. ¿Por quién desea
todo Israel? ¿No eres tú y toda la casa de tu padre?''

\bibleverse{21} Saúl respondió: ``¿No soy yo un benjamita, de la más
pequeña de las tribus de Israel? ¿Y mi familia la más pequeña de todas
las familias de la tribu de Benjamín? ¿Por qué, pues, me hablas así?''
\footnote{\textbf{9:21} 1Sam 15,17}

\bibleverse{22} Samuel tomó a Saúl y a su criado y los llevó a la sala
de invitados, y los hizo sentarse en el mejor lugar entre los invitados,
que eran unas treinta personas. \bibleverse{23} Samuel dijo al cocinero:
``Trae la porción que te di, de la que te dije: ``Apártala''\,''.
\bibleverse{24} El cocinero tomó el muslo y lo que había en él, y lo
puso delante de Saúl. Samuel dijo: ``¡Mira lo que se ha reservado! Ponlo
delante de ti y come; porque te ha sido reservado para el tiempo
señalado, pues yo dije: `He invitado al pueblo'.'' Así que Saúl comió
con Samuel aquel día.

\bibleverse{25} Cuando bajaron del lugar alto a la ciudad, habló con
Saúl en el terrado.

\hypertarget{sauxfal-ungido-rey-por-samuel-su-regreso-a-guibeuxe1}{%
\subsection{Saúl ungido rey por Samuel; su regreso a
Guibeá}\label{sauxfal-ungido-rey-por-samuel-su-regreso-a-guibeuxe1}}

\bibleverse{26} Se levantaron temprano, y cerca del amanecer, Samuel
llamó a Saúl en el terrado, diciendo: ``Levántate, para que te
despache''. Saúl se levantó, y ambos salieron fuera, él y Samuel,
juntos. \bibleverse{27} Cuando bajaban al final de la ciudad, Samuel
dijo a Saúl: ``Dile al criado que se adelante a nosotros.'' Él se
adelantó, y entonces Samuel le dijo: ``Pero quédate quieto primero, para
que te haga escuchar el mensaje de Dios''.

\hypertarget{section-9}{%
\section{10}\label{section-9}}

\bibleverse{1} Entonces Samuel tomó la vasija de aceite y la derramó
sobre su cabeza; luego lo besó y le dijo: ``¿No te ha ungido Yahvé para
que seas príncipe sobre su heredad?

\hypertarget{samuel-profetiza-tres-seuxf1ales-que-sauxfal-recibiruxe1-de-camino-a-casa-y-lo-envuxeda-a-gilgal}{%
\subsection{Samuel profetiza tres señales que Saúl recibirá de camino a
casa y lo envía a
Gilgal}\label{samuel-profetiza-tres-seuxf1ales-que-sauxfal-recibiruxe1-de-camino-a-casa-y-lo-envuxeda-a-gilgal}}

\bibleverse{2} Cuando hoy te hayas alejado de mí, encontrarás a dos
hombres junto a la tumba de Raquel, en la frontera de Benjamín, en
Zelza. Ellos te dirán: `Los asnos que fuiste a buscar han sido
encontrados; y he aquí que tu padre ha dejado de preocuparse por los
asnos y está ansioso por ti, diciendo: ``¿Qué haré por mi hijo?''\,'
\footnote{\textbf{10:2} Gén 35,19}

\bibleverse{3} ``Luego seguirás adelante desde allí, y llegarás a la
encina de Tabor. Allí te saldrán al encuentro tres hombres que suben a
Dios a Betel: uno que llevará tres cabritos, otro que llevará tres panes
y otro que llevará un recipiente de vino. \bibleverse{4} Ellos te
saludarán y te darán dos panes, que recibirás de su mano.

\bibleverse{5} ``Después llegarás a la colina de Dios, donde está la
guarnición de los filisteos; y sucederá que cuando hayas llegado allí a
la ciudad, te encontrarás con una banda de profetas que bajarán del
lugar alto con un laúd, un pandero, una flauta y un arpa delante de
ellos; y estarán profetizando. \bibleverse{6} Entonces el Espíritu de
Yahvé vendrá poderosamente sobre ti, entonces profetizarás con ellos y
te convertirás en otro hombre. \footnote{\textbf{10:6} 1Sam 10,10}
\bibleverse{7} Cuando te lleguen estas señales, haz lo que sea apropiado
para la ocasión, porque Dios está contigo.

\bibleverse{8} ``Desciende delante de mí a Gilgal; y he aquí que yo
bajaré a ti para ofrecer holocaustos y sacrificar ofrendas de paz.
Espera siete días, hasta que venga a ti y te muestre lo que debes
hacer''. \footnote{\textbf{10:8} 1Sam 13,8}

\hypertarget{la-llegada-de-los-carteles-anunciados-saulo-entre-los-profetas}{%
\subsection{La llegada de los carteles anunciados; Saulo entre los
profetas}\label{la-llegada-de-los-carteles-anunciados-saulo-entre-los-profetas}}

\bibleverse{9} Fue así que cuando le dio la espalda a Samuel para irse,
Dios le dio otro corazón; y todas esas señales sucedieron ese día.
\bibleverse{10} Cuando llegaron al monte, he aquí que un grupo de
profetas le salió al encuentro; y el Espíritu de Dios vino poderosamente
sobre él, y profetizó entre ellos. \footnote{\textbf{10:10} 1Sam
  19,20-24} \bibleverse{11} Cuando todos los que le conocían de antes
vieron que profetizaba con los profetas, el pueblo se dijo: ``¿Qué es
esto que le ha sucedido al hijo de Cis? ¿Está Saúl también entre los
profetas?''

\bibleverse{12} Uno del mismo lugar respondió: ``¿Quién es su padre?''
Por eso se convirtió en un proverbio: ``¿También Saúl está entre los
profetas?''

\hypertarget{saul-de-regreso-a-casa-su-conversaciuxf3n-reservada-con-su-prima}{%
\subsection{Saul de regreso a casa; su conversación reservada con su
prima}\label{saul-de-regreso-a-casa-su-conversaciuxf3n-reservada-con-su-prima}}

\bibleverse{13} Cuando terminó de profetizar, llegó al lugar alto.

\bibleverse{14} El tío de Saúl les dijo a él y a su criado: ``¿Adónde
habéis ido?'' Dijo: ``Para buscar los burros. Cuando vimos que no se
encontraban, vinimos a Samuel''.

\bibleverse{15} El tío de Saúl le dijo: ``Por favor, cuéntame lo que te
dijo Samuel''.

\bibleverse{16} Saúl dijo a su tío: ``Nos dijo claramente que los asnos
habían sido encontrados''. Pero en cuanto al asunto del reino, del que
habló Samuel, no se lo dijo.

\hypertarget{sauxfal-estuxe1-decidido-a-ser-rey-en-mizpa-por-la-santa-suerte}{%
\subsection{Saúl está decidido a ser rey en Mizpa por la santa
suerte}\label{sauxfal-estuxe1-decidido-a-ser-rey-en-mizpa-por-la-santa-suerte}}

\bibleverse{17} Samuel convocó al pueblo ante Yavé en Mizpa;
\bibleverse{18} y dijo a los hijos de Israel: ``Yavé, el Dios de Israel,
dice: `Yo saqué a Israel de Egipto y os libré de la mano de los
egipcios, y de la mano de todos los reinos que os oprimían'.
\bibleverse{19} Pero hoy habéis rechazado a vuestro Dios, que os salva
de todas vuestras calamidades y angustias, y le habéis dicho: ``¡No,
poned un rey sobre nosotros!''. Ahora, pues, preséntense ante Yahvé por
sus tribus y por sus miles''. \footnote{\textbf{10:19} 1Sam 8,7}

\bibleverse{20} Entonces Samuel acercó a todas las tribus de Israel, y
fue elegida la tribu de Benjamín. \footnote{\textbf{10:20} 1Sam
  14,41-42; Jos 7,16} \bibleverse{21} Hizo acercar a la tribu de
Benjamín por sus familias, y fue elegida la familia de los matritenses.
Luego se eligió a Saúl, hijo de Cis, pero cuando lo buscaron, no lo
encontraron. \bibleverse{22} Por lo tanto, preguntaron más a Yahvé:
``¿Hay todavía un hombre que venga aquí?'' Yahvé respondió: ``He aquí
que se ha escondido entre el equipaje''.

\bibleverse{23} Corrieron y lo llevaron allí. Cuando se puso de pie en
medio del pueblo, era más alto que cualquiera de ellos desde los hombros
hacia arriba. \bibleverse{24} Samuel dijo a todo el pueblo: ``¿Ven al
que Yahvé ha elegido, que no hay nadie como él en todo el pueblo?'' Todo
el pueblo gritó y dijo: ``¡Viva el rey!'' \footnote{\textbf{10:24} 1Re
  1,25}

\bibleverse{25} Entonces Samuel comunicó al pueblo el reglamento del
reino, lo escribió en un libro y lo puso delante de Yavé. Samuel
despidió a todo el pueblo, cada uno a su casa. \footnote{\textbf{10:25}
  1Sam 8,11; Deut 17,14-20} \bibleverse{26} Saúl también se fue a su
casa en Gabaa, y con él iba el ejército, cuyos corazones había tocado
Dios. \bibleverse{27} Pero algunos despreciables dijeron: ``¿Cómo podría
salvarnos este hombre?'' Lo despreciaron, y no le trajeron ningún
tributo. Pero él calló. \footnote{\textbf{10:27} 1Sam 11,12}

\hypertarget{la-ciudad-de-jabuxe9s-que-estuxe1-en-apuros-por-los-amonitas-nahas-pide-la-ayuda-de-los-israelitas}{%
\subsection{La ciudad de Jabés, que está en apuros por los amonitas
Nahas, pide la ayuda de los
israelitas}\label{la-ciudad-de-jabuxe9s-que-estuxe1-en-apuros-por-los-amonitas-nahas-pide-la-ayuda-de-los-israelitas}}

\hypertarget{section-10}{%
\section{11}\label{section-10}}

\bibleverse{1} Entonces Nahas, el amonita, subió y acampó contra Jabes
de Galaad; y todos los hombres de Jabes dijeron a Nahas: ``Haz un pacto
con nosotros y te serviremos.'' \footnote{\textbf{11:1} 1Sam 31,11}

\bibleverse{2} Nahas, el amonita, les dijo: ``Con esta condición lo haré
con ustedes: que les saquen todos los ojos derechos. Haré que esto
deshonre a todo Israel''. \footnote{\textbf{11:2} Jer 39,7}

\bibleverse{3} Los ancianos de Jabes le dijeron: ``Danos siete días,
para que enviemos mensajeros a todos los confines de Israel; y entonces,
si no hay nadie que nos salve, saldremos hacia ti.''

\hypertarget{la-conducta-decidida-de-sauxfal-y-su-espluxe9ndida-victoria}{%
\subsection{La conducta decidida de Saúl y su espléndida
victoria}\label{la-conducta-decidida-de-sauxfal-y-su-espluxe9ndida-victoria}}

\bibleverse{4} Los mensajeros llegaron a Gabaa de Saúl y dijeron estas
palabras a los oídos del pueblo; entonces todo el pueblo alzó la voz y
lloró.

\bibleverse{5} He aquí que Saúl venía siguiendo a los bueyes del campo,
y dijo: ``¿Qué le pasa al pueblo que llora?'' Ellos le contaron las
palabras de los hombres de Jabes. \bibleverse{6} El Espíritu de Dios se
apoderó de Saúl al oír esas palabras, y su ira se encendió. \footnote{\textbf{11:6}
  Jue 14,6} \bibleverse{7} Tomó una yunta de bueyes y los cortó en
pedazos, y los envió por todos los límites de Israel por medio de
mensajeros, diciendo: ``El que no salga en pos de Saúl y en pos de
Samuel, así se hará con sus bueyes.'' El temor de Yavé cayó sobre el
pueblo, y salieron como un solo hombre. \footnote{\textbf{11:7} Jue
  19,29} \bibleverse{8} Los contó en Bezec, y los hijos de Israel eran
trescientos mil, y los hombres de Judá treinta mil. \bibleverse{9}
Dijeron a los mensajeros que vinieron: ``Digan a los hombres de Jabes de
Galaad: `Mañana, cuando el sol esté caliente, serán rescatados'\,''. Los
mensajeros vinieron y se lo dijeron a los hombres de Jabes; y se
alegraron. \bibleverse{10} Por lo tanto, los hombres de Jabes dijeron:
``Mañana saldremos hacia ti, y harás con nosotros todo lo que te parezca
bien.'' \bibleverse{11} Al día siguiente, Saúl puso a la gente en tres
compañías, y llegaron al centro del campamento en la guardia de la
mañana, y golpearon a los amonitas hasta el calor del día. Los que
quedaron se dispersaron, de modo que no quedaron dos juntos.

\hypertarget{la-generosidad-de-sauxfal-hacia-sus-despreciadores-celebraciuxf3n-de-la-alegruxeda-en-gilgal}{%
\subsection{La generosidad de Saúl hacia sus despreciadores; Celebración
de la alegría en
Gilgal}\label{la-generosidad-de-sauxfal-hacia-sus-despreciadores-celebraciuxf3n-de-la-alegruxeda-en-gilgal}}

\bibleverse{12} El pueblo dijo a Samuel: ``¿Quién es el que ha dicho:
``Saúl reinará sobre nosotros''? Traigan a esos hombres, para que los
matemos''. \footnote{\textbf{11:12} 1Sam 10,27}

\bibleverse{13} Saúl dijo: ``Ningún hombre será ejecutado hoy, porque
hoy Yahvé ha rescatado a Israel''. \footnote{\textbf{11:13} 1Sam 14,45}

\bibleverse{14} Entonces Samuel dijo al pueblo: ``¡Vengan! Vayamos a
Gilgal y renovemos allí el reino''. \footnote{\textbf{11:14} 1Sam 10,8}
\bibleverse{15} Todo el pueblo fue a Gilgal, y allí hicieron a Saúl rey
ante Yavé en Gilgal. Allí ofrecieron sacrificios de ofrendas de paz ante
Yahvé; y allí se alegraron mucho Saúl y todos los hombres de Israel.

\hypertarget{la-renuncia-voluntaria-de-samuel-y-la-solemne-despedida-del-pueblo}{%
\subsection{La renuncia voluntaria de Samuel y la solemne despedida del
pueblo}\label{la-renuncia-voluntaria-de-samuel-y-la-solemne-despedida-del-pueblo}}

\hypertarget{section-11}{%
\section{12}\label{section-11}}

\bibleverse{1} Samuel dijo a todo Israel: ``He aquí que he escuchado
vuestra voz en todo lo que me habéis dicho, y he puesto un rey sobre
vosotros. \footnote{\textbf{12:1} 1Sam 8,7; 1Sam 8,22; 1Sam 11,15}
\bibleverse{2} Ahora, he aquí que el rey camina delante de ustedes. Yo
soy viejo y canoso. He aquí que mis hijos están contigo. He caminado
delante de ti desde mi juventud hasta hoy. \bibleverse{3} Aquí estoy.
Atestigüen contra mí ante el Señor y ante su ungido. ¿De quién es el
buey que he tomado? ¿De quién he tomado el asno? ¿A quién he defraudado?
¿A quién he oprimido? ¿De quién he tomado un soborno para que me ciegue
los ojos? Te lo devolveré''. \footnote{\textbf{12:3} Núm 16,15}

\bibleverse{4} Dijeron: ``No nos has defraudado, ni nos has oprimido, ni
has tomado nada de la mano de nadie''.

\bibleverse{5} Les dijo: ``Yahvé es testigo contra vosotros, y su ungido
es testigo hoy, de que no habéis encontrado nada en mi mano.'' Dijeron:
``Él es testigo''.

\hypertarget{samuel-le-recuerda-al-pueblo-los-muchos-beneficios-de-dios}{%
\subsection{Samuel le recuerda al pueblo los muchos beneficios de
Dios}\label{samuel-le-recuerda-al-pueblo-los-muchos-beneficios-de-dios}}

\bibleverse{6} Samuel dijo al pueblo: ``Es Yahvé quien designó a Moisés
y a Aarón, y quien sacó a vuestros padres de la tierra de Egipto.
\bibleverse{7} Ahora, pues, quédense quietos, para que yo pueda alegar
ante Yavé todos los actos justos de Yavé, que hizo con ustedes y con sus
padres.

\bibleverse{8} ``Cuando Jacob entró en Egipto, y vuestros padres
clamaron a Yavé, entonces Yavé envió a Moisés y a Aarón, quienes sacaron
a vuestros padres de Egipto y los hicieron habitar en este lugar.
\footnote{\textbf{12:8} Éxod 3,7} \bibleverse{9} Pero ellos se olvidaron
de Yavé, su Dios, y él los vendió en manos de Sísara, capitán del
ejército de Hazor, y en manos de los filisteos, y en manos del rey de
Moab; y pelearon contra ellos. \footnote{\textbf{12:9} Jue 4,2; Jue
  10,7; Jue 13,1; Jue 3,12} \bibleverse{10} Ellos clamaron a Yavé y
dijeron: ``Hemos pecado, porque hemos abandonado a Yavé y hemos servido
a los baales y a Astarot; pero líbranos ahora de la mano de nuestros
enemigos, y te serviremos''. \bibleverse{11} Yahvé envió a Jerobaal, a
Bedán, a Jefté y a Samuel, y os libró de la mano de vuestros enemigos de
todas partes, y vivisteis seguros. \footnote{\textbf{12:11} Jue 6,14;
  Jue 11,29; 1Sam 7,3}

\hypertarget{samuel-demuestra-al-pueblo-a-travuxe9s-de-una-maravillosa-seuxf1al-divina-que-han-pecado-al-elegir-un-rey}{%
\subsection{Samuel demuestra al pueblo a través de una maravillosa señal
divina que han pecado al elegir un
rey}\label{samuel-demuestra-al-pueblo-a-travuxe9s-de-una-maravillosa-seuxf1al-divina-que-han-pecado-al-elegir-un-rey}}

\bibleverse{12} ``Cuando viste que Nahas, el rey de los hijos de Amón,
venía contra ti, me dijiste: `No, sino que un rey reinará sobre
nosotros', cuando Yahvé, tu Dios, era tu rey. \footnote{\textbf{12:12}
  1Sam 8,19} \bibleverse{13} Ahora, pues, vean al rey que han elegido y
al que han pedido. He aquí que el Señor ha puesto un rey sobre ustedes.
\bibleverse{14} Si temes a Yavé, le sirves y escuchas su voz, y no te
rebelas contra el mandamiento de Yavé, tanto tú como el rey que reina
sobre ti son seguidores de Yavé tu Dios. \bibleverse{15} Pero si no
escuchan la voz de Yahvé y se rebelan contra el mandamiento de Yahvé, la
mano de Yahvé estará contra ustedes, como lo estuvo contra sus padres.

\bibleverse{16} ``Ahora, pues, quédense quietos y vean esta gran cosa
que el Señor va a hacer ante sus ojos. \bibleverse{17} ¿No es hoy la
cosecha de trigo? Invocaré a Yavé, para que envíe truenos y lluvia; y
ustedes sabrán y verán que es grande la maldad que han hecho ante los
ojos de Yavé, al pedir un rey.''

\bibleverse{18} Entonces Samuel invocó a Yahvé, y Yahvé envió truenos y
lluvia aquel día. Entonces todo el pueblo temió mucho a Yavé y a Samuel.

\hypertarget{samuel-anima-al-pueblo-les-exhorta-a-temer-a-dios-y-les-manda-recibir-bendiciones-divinas}{%
\subsection{Samuel anima al pueblo, les exhorta a temer a Dios y les
manda recibir bendiciones
divinas}\label{samuel-anima-al-pueblo-les-exhorta-a-temer-a-dios-y-les-manda-recibir-bendiciones-divinas}}

\bibleverse{19} Todo el pueblo dijo a Samuel: ``Ruega por tus siervos a
Yahvé, tu Dios, para que no muramos, pues hemos añadido a todos nuestros
pecados esta maldad de pedir un rey.''

\bibleverse{20} Samuel dijo al pueblo: ``No tengan miedo. Ciertamente
han hecho todo este mal; pero no se aparten de seguir a Yavé, sino que
sirvan a Yavé con todo su corazón. \bibleverse{21} No se aparten para ir
en pos de cosas vanas que no pueden aprovechar ni liberar, porque son
vanas. \footnote{\textbf{12:21} Deut 32,37-38} \bibleverse{22} Porque
Yahvé no abandonará a su pueblo por causa de su gran nombre, porque a
Yahvé le ha gustado hacer de ustedes un pueblo para sí mismo.
\footnote{\textbf{12:22} Éxod 19,6} \bibleverse{23} Además, en cuanto a
mí, lejos de pecar contra Yahvé al dejar de orar por ustedes, los
instruiré en el camino bueno y correcto. \footnote{\textbf{12:23} 1Sam
  7,8} \bibleverse{24} Sólo temed a Yahvé y servidle de verdad con todo
vuestro corazón, pues considerad las grandes cosas que ha hecho por
vosotros. \footnote{\textbf{12:24} 2Re 17,39} \bibleverse{25} Pero si
sigues haciendo el mal, serás consumido, tanto tú como tu rey''.

\hypertarget{estallido-de-la-guerra-filistea-la-primera-desobediencia-de-sauxfal-mediante-un-sacrificio-apresurado}{%
\subsection{Estallido de la guerra filistea; La primera desobediencia de
Saúl mediante un sacrificio
apresurado}\label{estallido-de-la-guerra-filistea-la-primera-desobediencia-de-sauxfal-mediante-un-sacrificio-apresurado}}

\hypertarget{section-12}{%
\section{13}\label{section-12}}

\bibleverse{1} Saúl tenía treinta años cuando llegó a ser rey, y reinó
sobre Israel cuarenta y dos años.

\bibleverse{2} Saúl escogió para sí tres mil hombres de Israel, de los
cuales dos mil estaban con Saúl en Micmas y en el monte de Betel, y mil
estaban con Jonatán en Guibeá de Benjamín. Envió al resto del pueblo a
sus propias tiendas. \bibleverse{3} Jonatán atacó la guarnición de los
filisteos que estaba en Geba, y los filisteos se enteraron. Saúl hizo
sonar la trompeta por todo el país, diciendo: ``¡Que se enteren los
hebreos!''. \footnote{\textbf{13:3} 1Sam 14,49} \bibleverse{4} Todo
Israel se enteró de que Saúl había golpeado a la guarnición de los
filisteos, y también de que Israel era considerado una abominación para
los filisteos. El pueblo se reunió tras Saúl en Gilgal. \bibleverse{5}
Los filisteos se reunieron para luchar contra Israel: treinta mil
carros, seis mil jinetes y gente como la arena que está a la orilla del
mar en multitud. Subieron y acamparon en Micmas, al este de Bet Aven.
\bibleverse{6} Cuando los hombres de Israel vieron que estaban en apuros
(pues el pueblo estaba angustiado), el pueblo se escondió en cuevas, en
matorrales, en rocas, en tumbas y en fosas. \bibleverse{7} Algunos de
los hebreos habían pasado el Jordán a la tierra de Gad y de Galaad; pero
Saúl estaba todavía en Gilgal, y todo el pueblo lo seguía temblando.

\hypertarget{el-sacrificio-apresurado-y-arbitrario-de-sauxfal-en-gilgal-rompe-entre-samuel-y-el-rey-el-rechazo-de-sauxfal}{%
\subsection{El sacrificio apresurado y arbitrario de Saúl en Gilgal;
Rompe entre Samuel y el rey; El rechazo de
Saúl}\label{el-sacrificio-apresurado-y-arbitrario-de-sauxfal-en-gilgal-rompe-entre-samuel-y-el-rey-el-rechazo-de-sauxfal}}

\bibleverse{8} Se quedó siete días, según el tiempo fijado por Samuel;
pero éste no llegó a Gilgal, y el pueblo se dispersó de él. \footnote{\textbf{13:8}
  1Sam 10,8} \bibleverse{9} Saúl le dijo: ``Tráeme aquí el holocausto y
las ofrendas de paz''. Ofreció el holocausto.

\bibleverse{10} Sucedió que en cuanto terminó de ofrecer el holocausto,
he aquí que llegó Samuel; y Saúl salió a recibirlo para saludarlo.
\bibleverse{11} Samuel le dijo: ``¿Qué has hecho?'' Saúl dijo: ``Como vi
que el pueblo se dispersaba de mí, y que tú no venías en los días
señalados, y que los filisteos se reunían en Micmas, \bibleverse{12}
dije: ``Ahora los filisteos bajarán sobre mí a Gilgal, y yo no he
suplicado el favor de Yahvé. Me obligué, pues, a ofrecer el
holocausto''.

\bibleverse{13} Samuel le dijo a Saúl: ``Has hecho una tontería. No has
cumplido el mandamiento del Señor, tu Dios, que él te ordenó; porque
ahora el Señor habría establecido tu reino en Israel para siempre.
\bibleverse{14} Pero ahora tu reino no continuará. Yahvé se ha buscado
un hombre según su propio corazón, y Yahvé lo ha designado como príncipe
de su pueblo, porque tú no has guardado lo que Yahvé te mandó.''
\footnote{\textbf{13:14} Hech 13,22}

\hypertarget{el-pequeuxf1o-ejuxe9rcito-de-sauxfal-el-pillaje-de-los-filisteos-indefensiuxf3n-de-los-israelitas}{%
\subsection{El pequeño ejército de Saúl; el pillaje de los filisteos;
Indefensión de los
israelitas}\label{el-pequeuxf1o-ejuxe9rcito-de-sauxfal-el-pillaje-de-los-filisteos-indefensiuxf3n-de-los-israelitas}}

\bibleverse{15} Samuel se levantó y se dirigió de Gilgal a Gabaa de
Benjamín. Saúl contó el pueblo que estaba presente con él, unos
seiscientos hombres. \bibleverse{16} Saúl, su hijo Jonatán y el pueblo
que estaba con ellos se quedaron en Gabaa de Benjamín, pero los
filisteos acamparon en Micmas. \bibleverse{17} Los asaltantes salieron
del campamento de los filisteos en tres compañías: una compañía se
dirigió al camino que lleva a Ofra, a la tierra de Shual;
\bibleverse{18} otra compañía se dirigió al camino de Bet Horón; y otra
compañía se dirigió al camino de la frontera que da al valle de Zeboim,
hacia el desierto. \bibleverse{19} No se encontró ningún herrero en toda
la tierra de Israel, porque los filisteos dijeron: ``No sea que los
hebreos se hagan espadas o lanzas''; \bibleverse{20} pero todos los
israelitas bajaron a los filisteos, cada uno para afilar su propia reja
de arado, su azadón, su hacha y su hoz. \bibleverse{21} El precio era de
un payim cada uno para afilar azadones, rejas de arado, horcas, hachas y
picos. \bibleverse{22} Y sucedió que el día de la batalla no se encontró
espada ni lanza en manos de ninguno de los que estaban con Saúl y
Jonatán, sino que Saúl y su hijo las tenían.

\bibleverse{23} La guarnición de los filisteos salió al paso de Micmas.

\hypertarget{el-herouxedsmo-de-jonathan-la-victoria-de-sauxfal-sobre-los-filisteos}{%
\subsection{El heroísmo de Jonathan; La victoria de Saúl sobre los
filisteos}\label{el-herouxedsmo-de-jonathan-la-victoria-de-sauxfal-sobre-los-filisteos}}

\hypertarget{section-13}{%
\section{14}\label{section-13}}

\bibleverse{1} Sucedió un día que Jonatán, hijo de Saúl, dijo al joven
que llevaba su armadura: ``¡Ven! Vamos a la guarnición de los filisteos
que está al otro lado''. Pero no se lo dijo a su padre. \bibleverse{2}
Saúl se quedó en el extremo de Guibeá, bajo el granado que está en
Migrón; y la gente que estaba con él era como seiscientos hombres,
\bibleverse{3} incluyendo a Ajías, hijo de Ajitub, hermano de Icabod,
hijo de Finehas, hijo de Elí, sacerdote de Yavé en Silo, que llevaba un
efod. El pueblo no sabía que Jonatán se había ido. \footnote{\textbf{14:3}
  1Sam 4,19; 1Sam 4,21}

\bibleverse{4} Entre los pasos por los que Jonatán pretendía pasar a la
guarnición de los filisteos, había un peñasco a un lado y otro peñasco
al otro lado; el nombre del uno era Bozez, y el del otro Seneh.
\bibleverse{5} Un peñasco se levantaba al norte, frente a Micmas, y el
otro al sur, frente a Geba. \bibleverse{6} Jonatán dijo al joven que
llevaba su armadura: ``¡Ven! Vamos a la guarnición de estos
incircuncisos. Puede ser que Yavé actúe a nuestro favor, pues no hay
freno para que Yavé salve por muchos o por pocos.'' \footnote{\textbf{14:6}
  Jue 7,7; 2Cró 14,11}

\bibleverse{7} El portador de su armadura le dijo: ``Haz todo lo que
está en tu corazón. Ve, y he aquí que yo estoy contigo según tu
corazón''.

\bibleverse{8} Entonces Jonatán dijo: ``He aquí que pasaremos a los
hombres y nos revelaremos a ellos. \bibleverse{9} Si nos dicen esto:
`Esperen hasta que lleguemos a ustedes', nos quedaremos quietos en
nuestro lugar y no subiremos a ellos. \bibleverse{10} Pero si nos dicen
esto: ``¡Suban a nosotros!'', entonces subiremos, porque el Señor los ha
entregado en nuestra mano. Esta será la señal para nosotros''.

\bibleverse{11} Ambos se revelaron ante la guarnición de los filisteos,
y éstos dijeron: ``¡Mira que los hebreos salen de los agujeros donde se
habían escondido!'' \bibleverse{12} Los hombres de la guarnición
respondieron a Jonatán y a su portador de armadura y les dijeron:
``¡Suban a nosotros y les mostraremos algo!'' Jonatán dijo a su portador
de armadura: ``Sube detrás de mí, porque Yahvé los ha entregado en manos
de Israel''. \bibleverse{13} Jonatán subió sobre sus manos y sus pies, y
su escudero tras él, y ellos cayeron ante Jonatán; y su escudero los
mató tras él. \footnote{\textbf{14:13} Lev 26,7-8} \bibleverse{14}
Aquella primera matanza que hicieron Jonatán y su portador de armadura
fue de unos veinte hombres, dentro de la longitud de medio surco en un
acre de tierra.

\bibleverse{15} Hubo un temblor en el campamento, en el campo y en todo
el pueblo; la guarnición y los asaltantes también temblaron, y la tierra
se estremeció, por lo que hubo un temblor sumamente grande.

\hypertarget{sauxfal-interviene-y-obtiene-una-brillante-victoria}{%
\subsection{Saúl interviene y obtiene una brillante
victoria}\label{sauxfal-interviene-y-obtiene-una-brillante-victoria}}

\bibleverse{16} Los centinelas de Saúl en Gabaa de Benjamín miraron, y
he aquí que la multitud se desvaneció y se dispersó. \bibleverse{17}
Entonces Saúl dijo al pueblo que estaba con él: ``Contad ahora y ved
quién falta de nosotros.'' Cuando hubieron contado, he aquí que Jonatán
y su portador de armadura no estaban allí.

\bibleverse{18} Saúl dijo a Ahías: ``Trae aquí el arca de Dios''. Pues
el arca de Dios estaba con los hijos de Israel en ese momento.
\footnote{\textbf{14:18} 1Sam 4,3} \bibleverse{19} Mientras Saúl hablaba
con el sacerdote, el tumulto que había en el campamento de los filisteos
continuaba y aumentaba, y Saúl le dijo al sacerdote: ``¡Retírate!''

\bibleverse{20} Saúl y todo el pueblo que estaba con él se reunieron y
vinieron a la batalla; y he aquí que todos se golpeaban con sus espadas
en una gran confusión. \footnote{\textbf{14:20} Jue 7,22; 2Cró 20,23}
\bibleverse{21} Los hebreos que antes estaban con los filisteos y que
subieron con ellos al campamento desde todos los alrededores, también se
volvieron para estar con los israelitas que estaban con Saúl y Jonatán.
\bibleverse{22} Asimismo, todos los hombres de Israel que se habían
escondido en la región montañosa de Efraín, cuando oyeron que los
filisteos habían huido, también los siguieron con ahínco en la batalla.
\bibleverse{23} Así salvó Yahvé a Israel aquel día, y la batalla pasó
junto a Bet-Aven.

\hypertarget{el-celo-intempestivo-de-sauxfal-jonathan-estuxe1-amenazado-de-muerte-las-guerras-de-sauxfal-y-su-familia}{%
\subsection{El celo intempestivo de Saúl; Jonathan está amenazado de
muerte; Las guerras de Saúl y su
familia}\label{el-celo-intempestivo-de-sauxfal-jonathan-estuxe1-amenazado-de-muerte-las-guerras-de-sauxfal-y-su-familia}}

\bibleverse{24} Aquel día los hombres de Israel estaban angustiados,
pues Saúl había conjurado al pueblo diciendo: ``Maldito el hombre que
coma cualquier alimento hasta que anochezca, y me vengue de mis
enemigos.'' Así que nadie del pueblo probó alimento.

\bibleverse{25} Todo el pueblo entró en el bosque, y había miel en el
suelo. \bibleverse{26} Cuando el pueblo llegó al bosque, he aquí que la
miel goteaba, pero nadie se llevó la mano a la boca, porque el pueblo
temía el juramento. \bibleverse{27} Pero Jonatán no escuchó cuando su
padre ordenó al pueblo con el juramento. Por eso sacó la punta de la
vara que tenía en la mano y la mojó en el panal, y se llevó la mano a la
boca; y sus ojos se iluminaron. \bibleverse{28} Entonces respondió uno
del pueblo y dijo: ``Tu padre ordenó directamente al pueblo con un
juramento, diciendo: ``Maldito el hombre que hoy coma comida''\,''.
Entonces el pueblo se desmayó.

\bibleverse{29} Entonces Jonatán dijo: ``Mi padre ha turbado la tierra.
Por favor, mira cómo se han iluminado mis ojos porque he probado un poco
de esta miel. \bibleverse{30} ¿Cuánto más, si acaso el pueblo hubiera
comido hoy libremente del botín de sus enemigos que encontró? Porque
ahora no ha habido gran matanza entre los filisteos''. \bibleverse{31}
Aquel día atacaron a los filisteos desde Micmas hasta Ajalón. El pueblo
estaba muy desmayado; \bibleverse{32} y el pueblo se abalanzó sobre el
botín, y tomó ovejas, vacas y terneros, y los mató en el suelo; y el
pueblo se los comió con la sangre. \footnote{\textbf{14:32} Lev 3,17}
\bibleverse{33} Entonces se lo comunicaron a Saúl, diciendo: ``He aquí
que el pueblo peca contra Yavé, pues come carne con la sangre.'' Dijo:
``Has hecho un trato traicionero. Hazme rodar hoy una gran piedra''.
\bibleverse{34} Saúl dijo: ``Dispérsense entre el pueblo y díganle:
``Cada uno traiga aquí su buey y cada uno su oveja, y mátenlos aquí y
coman, y no pequen contra Yavé comiendo carne con la sangre''.'' Todo el
pueblo trajo aquella noche cada uno su buey, y los mató allí.

\bibleverse{35} Saúl construyó un altar a Yahvé. Este fue el primer
altar que construyó a Yahvé.

\hypertarget{jonatuxe1n-amenazado-de-muerte-por-el-celo-ciego-de-sauxfal-es-salvado-por-el-ejuxe9rcito}{%
\subsection{Jonatán, amenazado de muerte por el celo ciego de Saúl, es
salvado por el
ejército}\label{jonatuxe1n-amenazado-de-muerte-por-el-celo-ciego-de-sauxfal-es-salvado-por-el-ejuxe9rcito}}

\bibleverse{36} Saúl dijo: ``Descendamos tras los filisteos de noche y
saquemos provecho entre ellos hasta la luz de la mañana. No dejemos a
ningún hombre de ellos''. Dijeron: ``Haz lo que te parezca bien''.
Entonces el sacerdote dijo: ``Acerquémonos aquí a Dios''.

\bibleverse{37} Saúl pidió consejo a Dios: ``¿Debo bajar tras los
filisteos? ¿Los entregarás en manos de Israel?'' Pero aquel día no le
respondió. \footnote{\textbf{14:37} 1Sam 14,18; 1Sam 23,9}
\bibleverse{38} Saúl dijo: ``Acercaos aquí todos los jefes del pueblo, y
sabed y ved en quién ha estado hoy este pecado. \bibleverse{39} Porque
vive Yahvé, que salva a Israel, aunque sea en Jonatán, mi hijo, sin duda
morirá.'' Pero no hubo un solo hombre de todo el pueblo que le
respondiera. \bibleverse{40} Entonces dijo a todo Israel: ``Vosotros
estáis de un lado, y yo y Jonatán mi hijo estaremos del otro''. El
pueblo le dijo a Saúl: ``Haz lo que te parezca bien''.

\bibleverse{41} Por eso Saúl dijo a Yahvé, el Dios de Israel: ``Muestra
la derecha''. Jonatán y Saúl fueron elegidos, pero el pueblo escapó.
\footnote{\textbf{14:41} 1Sam 10,20}

\bibleverse{42} Saúl dijo: ``Echad suertes entre mi hijo y yo''.
Jonathan fue seleccionado.

\bibleverse{43} Entonces Saúl dijo a Jonatán: ``¡Dime qué has hecho!''
Jonatán se lo contó y dijo: ``Ciertamente probé un poco de miel con la
punta de la vara que tenía en la mano, y he de morir''. \footnote{\textbf{14:43}
  Jos 7,19}

\bibleverse{44} Saúl dijo: ``Que Dios haga eso y más, porque seguramente
morirás, Jonatán''.

\bibleverse{45} El pueblo dijo a Saúl: ``¿Ha de morir Jonatán, que ha
obrado esta gran salvación en Israel? ¡Lejos de eso! Vive Yahvé, que no
se le caerá ni un pelo de la cabeza, ¡porque hoy ha trabajado con
Dios!'' Así que el pueblo rescató a Jonatán, para que no muriera.
\bibleverse{46} Entonces Saúl subió de seguir a los filisteos, y los
filisteos se fueron a su lugar.

\hypertarget{los-otros-actos-de-guerra-de-sauxfal-y-su-familia}{%
\subsection{Los otros actos de guerra de Saúl y su
familia}\label{los-otros-actos-de-guerra-de-sauxfal-y-su-familia}}

\bibleverse{47} Cuando Saúl tomó el reino de Israel, luchó contra todos
sus enemigos de todas partes: contra Moab, contra los hijos de Amón,
contra Edom, contra los reyes de Soba y contra los filisteos. A
dondequiera que se dirigía, los derrotaba. \bibleverse{48} Hizo valentía
e hirió a los amalecitas, y libró a Israel de las manos de los que lo
saqueaban. \bibleverse{49} Los hijos de Saúl fueron Jonatán, Ishvi y
Malquisúa, y los nombres de sus dos hijas fueron estos: el nombre de la
primogénita, Merab, y el de la menor, Mical. \footnote{\textbf{14:49}
  1Cró 9,39} \bibleverse{50} El nombre de la esposa de Saúl era Ahinoam,
hija de Ahimaas. El nombre del capitán de su ejército era Abner, hijo de
Ner, tío de Saúl. \footnote{\textbf{14:50} 1Sam 17,55} \bibleverse{51}
Cis era el padre de Saúl, y Ner el padre de Abner era hijo de Abiel.

\bibleverse{52} Hubo una severa guerra contra los filisteos durante
todos los días de Saúl; y cuando éste veía a algún hombre poderoso o
valiente, lo tomaba a su servicio.

\hypertarget{la-campauxf1a-de-sauxfal-contra-los-amalecitas-su-desobediencia-a-dios-y-su-rechazo}{%
\subsection{La campaña de Saúl contra los amalecitas; su desobediencia a
Dios y su
rechazo}\label{la-campauxf1a-de-sauxfal-contra-los-amalecitas-su-desobediencia-a-dios-y-su-rechazo}}

\hypertarget{section-14}{%
\section{15}\label{section-14}}

\bibleverse{1} Samuel dijo a Saúl: ``Yahvé me ha enviado para ungirte
como rey de su pueblo, de Israel. Ahora, pues, escucha la voz de las
palabras de Yahvé. \footnote{\textbf{15:1} 1Sam 10,1} \bibleverse{2} El
Señor de los Ejércitos dice: `Me acuerdo de lo que Amalec hizo a Israel,
de cómo se puso en su contra en el camino cuando subió de Egipto.
\footnote{\textbf{15:2} Éxod 17,8-16; Deut 25,17-19} \bibleverse{3}
Ahora ve y ataca a Amalec, y destruye por completo todo lo que tiene, y
no lo perdones; mata tanto al hombre como a la mujer, al niño y al
lactante, al buey y a la oveja, al camello y al asno'\,''. \footnote{\textbf{15:3}
  Núm 21,2}

\bibleverse{4} Saúl convocó al pueblo y lo contó en Telaim, doscientos
mil hombres de a pie y diez mil hombres de Judá. \bibleverse{5} Saúl
llegó a la ciudad de Amalec y puso una emboscada en el valle.
\bibleverse{6} Saúl dijo a los ceneos: ``Vayan, váyanse, desciendan de
entre los amalecitas, para que no los destruya con ellos, pues ustedes
mostraron bondad con todos los hijos de Israel cuando subieron de
Egipto.'' Así que los ceneos se marcharon de entre los amalecitas.
\footnote{\textbf{15:6} Jue 1,16}

\bibleverse{7} Saúl hirió a los amalecitas, desde Havila, como vas a
Shur, que está frente a Egipto. \bibleverse{8} Tomó vivo a Agag, rey de
los amalecitas, y destruyó a todo el pueblo a filo de espada.
\bibleverse{9} Pero Saúl y el pueblo perdonaron a Agag y a lo mejor de
las ovejas, del ganado, de los terneros gordos, de los corderos, y de
todo lo bueno, y no quisieron destruirlo del todo; pero todo lo vil y
desecho lo destruyeron del todo.

\hypertarget{saulo-rechazado-por-dios-a-causa-de-su-desobediencia-el-discurso-de-samuel-y-la-admisiuxf3n-de-culpabilidad-de-sauxfal}{%
\subsection{Saulo rechazado por Dios a causa de su desobediencia; El
discurso de Samuel y la admisión de culpabilidad de
Saúl}\label{saulo-rechazado-por-dios-a-causa-de-su-desobediencia-el-discurso-de-samuel-y-la-admisiuxf3n-de-culpabilidad-de-sauxfal}}

\bibleverse{10} Entonces llegó la palabra de Yahvé a Samuel, diciendo:
\bibleverse{11} ``Me apena haber puesto a Saúl como rey, pues se ha
apartado de seguirme y no ha cumplido mis mandatos.'' Samuel se
enfureció y clamó a Yahvé toda la noche.

\bibleverse{12} Samuel se levantó temprano para encontrarse con Saúl por
la mañana, y le dijeron: ``Saúl llegó al Carmelo, y he aquí que se
levantó un monumento, se volvió, pasó y bajó a Gilgal.''

\bibleverse{13} Samuel se acercó a Saúl, y éste le dijo: ``¡Bendito seas
por Yahvé! He cumplido el mandamiento de Yahvé''.

\bibleverse{14} Samuel dijo: ``Entonces, ¿qué significa este balido de
las ovejas en mis oídos y el mugido del ganado que oigo?''

\bibleverse{15} Saúl dijo: ``Los han traído de los amalecitas, pues el
pueblo perdonó lo mejor de las ovejas y del ganado, para sacrificar a
Yavé tu Dios. El resto lo hemos destruido por completo''.

\bibleverse{16} Entonces Samuel dijo a Saúl: ``Quédate, y te contaré lo
que me dijo Yahvé anoche''. Le dijo: ``Diga''.

\bibleverse{17} Samuel dijo: ``Aunque eras pequeño a tus ojos, ¿no
fuiste hecho jefe de las tribus de Israel? Yahvé te ungió como rey de
Israel; \footnote{\textbf{15:17} 1Sam 9,21} \bibleverse{18} y Yahvé te
envió de viaje y te dijo: `Ve y destruye por completo a los pecadores
amalecitas, y lucha contra ellos hasta consumirlos'. \bibleverse{19}
¿Por qué entonces no obedeciste la voz de Yavé, sino que tomaste el
botín e hiciste lo que era malo a los ojos de Yavé?''

\bibleverse{20} Saúl dijo a Samuel: ``Pero yo he obedecido la voz de
Yavé y he seguido el camino que Yavé me envió, y he traído a Agag, rey
de Amalec, y he destruido por completo a los amalecitas. \bibleverse{21}
Pero el pueblo tomó del botín, ovejas y ganado, lo mejor de lo
consagrado, para sacrificar a Yavé tu Dios en Gilgal.''

\bibleverse{22} Samuel dijo: ``¿Se complace tanto Yahvé en los
holocaustos y sacrificios como en obedecer la voz de Yahvé? He aquí que
obedecer es mejor que los sacrificios, y escuchar que la grasa de los
carneros. \footnote{\textbf{15:22} Os 6,6; Is 1,11; Mat 9,13; Mat 12,7}
\bibleverse{23} Porque la rebeldía es como el pecado de brujería, y la
obstinación es como la idolatría y los terafines.\footnote{\textbf{15:23}
  Los terafines eran ídolos domésticos que podían estar asociados a los
  derechos de herencia de los bienes del hogar.} Porque has rechazado la
palabra de Yahvé, él también te ha rechazado para ser rey''. \footnote{\textbf{15:23}
  1Sam 16,1}

\bibleverse{24} Saúl dijo a Samuel: ``He pecado, pues he transgredido el
mandamiento de Yavé y tus palabras, porque temí al pueblo y obedecí su
voz. \bibleverse{25} Ahora, pues, te ruego que perdones mi pecado y
vuelvas conmigo para que pueda adorar a Yavé''.

\bibleverse{26} Samuel le dijo a Saúl: ``No volveré contigo, porque has
rechazado la palabra de Yahvé, y Yahvé te ha rechazado para ser rey de
Israel''. \bibleverse{27} Cuando Samuel se dio la vuelta para marcharse,
Saúl se agarró a la falda de su túnica y ésta se rasgó. \bibleverse{28}
Samuel le dijo: ``Yahvé te ha arrancado hoy el reino de Israel y se lo
ha dado a un vecino tuyo que es mejor que tú. \footnote{\textbf{15:28}
  1Sam 28,17} \bibleverse{29} También la Fuerza de Israel no mentirá ni
se arrepentirá, porque no es hombre para arrepentirse.'' \footnote{\textbf{15:29}
  Núm 23,19}

\bibleverse{30} Entonces dijo: ``He pecado; pero te ruego que me honres
ahora ante los ancianos de mi pueblo y ante Israel, y que vuelvas
conmigo para que pueda adorar a Yahvé, tu Dios.''

\bibleverse{31} Entonces Samuel regresó con Saúl, y éste adoró a Yahvé.

\hypertarget{samuel-realiza-la-proscripciuxf3n-del-rey-agag-y-se-separa-de-sauxfal-para-no-volver-a-ser-visto}{%
\subsection{Samuel realiza la proscripción del rey Agag y se separa de
Saúl para no volver a ser
visto}\label{samuel-realiza-la-proscripciuxf3n-del-rey-agag-y-se-separa-de-sauxfal-para-no-volver-a-ser-visto}}

\bibleverse{32} Entonces Samuel dijo: ``¡Trae aquí a Agag, rey de los
amalecitas!'' Agag se acercó a él alegremente. Agag dijo: ``Ciertamente
la amargura de la muerte ha pasado''.

\bibleverse{33} Samuel dijo: ``¡Como tu espada ha dejado sin hijos a las
mujeres, así tu madre quedará sin hijos entre las mujeres!'' Entonces
Samuel cortó en pedazos a Agag ante Yahvé en Gilgal.

\bibleverse{34} Entonces Samuel se fue a Ramá, y Saúl subió a su casa en
Gabaa de Saúl. \bibleverse{35} Samuel no volvió a ver a Saúl hasta el
día de su muerte, pero Samuel hizo duelo por Saúl. El Señor se afligió
por haber hecho a Saúl rey de Israel.

\hypertarget{el-llamado-y-unciuxf3n-de-david-por-samuel}{%
\subsection{El llamado y unción de David por
Samuel}\label{el-llamado-y-unciuxf3n-de-david-por-samuel}}

\hypertarget{section-15}{%
\section{16}\label{section-15}}

\bibleverse{1} El Señor le dijo a Samuel: ``¿Hasta cuándo llorarás por
Saúl, ya que lo he rechazado como rey de Israel? Llena tu cuerno de
aceite y vete. Te enviaré a Jesé, el betlemita, porque me he provisto de
un rey entre sus hijos''. \footnote{\textbf{16:1} 1Sam 15,23; 1Sam 15,35}

\bibleverse{2} Samuel dijo: ``¿Cómo puedo ir? Si Saúl se entera, me
matará''. Yahvé dijo: ``Toma una novilla contigo y di: `He venido a
sacrificar a Yahvé'. \bibleverse{3} Llama a Isaí al sacrificio, y yo te
mostraré lo que debes hacer. Me ungirás al que yo te nombre''.

\bibleverse{4} Samuel hizo lo que Yahvé había dicho y llegó a Belén. Los
ancianos de la ciudad salieron a su encuentro temblando, y le dijeron:
``¿Vienes en paz?''. \footnote{\textbf{16:4} 2Re 9,18}

\bibleverse{5} Dijo: ``Tranquilos; he venido a sacrificar a Yahvé.
Santificaos y venid conmigo al sacrificio''. Santificó a Jesé y a sus
hijos, y los llamó al sacrificio.

\hypertarget{samuel-unge-como-rey-al-hijo-menor-de-isauxed-david}{%
\subsection{Samuel unge como rey al hijo menor de Isaí,
David}\label{samuel-unge-como-rey-al-hijo-menor-de-isauxed-david}}

\bibleverse{6} Cuando llegaron, miró a Eliab y dijo: ``Ciertamente el
ungido de Yavé está delante de él.''

\bibleverse{7} Pero Yahvé dijo a Samuel: ``No te fijes en su rostro ni
en la altura de su estatura, porque lo he rechazado; porque yo no veo
como ve el hombre. Porque el hombre mira la apariencia exterior, pero
Yahvé mira el corazón''. \footnote{\textbf{16:7} Hech 10,34; Sal 7,9}

\bibleverse{8} Entonces Isaí llamó a Abinadab y lo hizo pasar ante
Samuel. Éste le dijo: ``El Señor tampoco ha elegido a éste''.
\bibleverse{9} Luego Isaí hizo pasar a Shammah. Dijo: ``Tampoco a éste
lo ha elegido Yahvé''. \bibleverse{10} Isaí hizo pasar a siete de sus
hijos ante Samuel. Samuel dijo a Isaí: ``Yahvé no ha elegido a éstos''.
\footnote{\textbf{16:10} 1Cró 2,13-15} \bibleverse{11} Samuel dijo a
Isaí: ``¿Están todos tus hijos aquí?'' Dijo: ``Todavía queda el más
joven. He aquí que está guardando las ovejas''. Samuel dijo a Isaí:
``Envía a buscarlo, porque no nos sentaremos hasta que venga''.
\footnote{\textbf{16:11} 1Sam 17,14}

\bibleverse{12} Envió y lo hizo entrar. Ahora era rubicundo, de rostro
apuesto y buena apariencia. Yahvé dijo: ``¡Levántate! Ungidlo, porque
éste es''.

\bibleverse{13} Entonces Samuel tomó el cuerno de aceite y lo ungió en
medio de sus hermanos. Entonces el Espíritu de Yahvé vino poderosamente
sobre David desde aquel día. Samuel se levantó y se fue a Ramá.
\footnote{\textbf{16:13} 2Sam 2,4; 2Sam 5,3}

\hypertarget{david-es-llamado-a-tocar-el-arpa-en-la-corte-de-sauxfal-y-entra-al-servicio-real}{%
\subsection{David es llamado a tocar el arpa en la corte de Saúl y entra
al servicio
real}\label{david-es-llamado-a-tocar-el-arpa-en-la-corte-de-sauxfal-y-entra-al-servicio-real}}

\bibleverse{14} El Espíritu de Yahvé se apartó de Saúl, y un espíritu
maligno de Yahvé lo perturbó. \footnote{\textbf{16:14} 1Sam 18,10}
\bibleverse{15} Los servidores de Saúl le dijeron: ``Mira, un espíritu
maligno de parte de Dios te perturba. \bibleverse{16} Que nuestro señor
ordene ahora a sus siervos que están delante de usted que busquen a un
hombre que sepa tocar el arpa. Entonces, cuando el espíritu maligno de
Dios esté sobre ti, él tocará con su mano, y tú estarás bien''.
\footnote{\textbf{16:16} 2Re 3,15}

\bibleverse{17} Saúl dijo a sus siervos: ``Proporciónenme ahora un
hombre que sepa tocar bien y tráiganmelo''.

\bibleverse{18} Uno de los jóvenes respondió y dijo: ``He aquí que he
visto a un hijo de Jesé, el de Belén, que es hábil en el juego,
valiente, hombre de guerra, prudente en la palabra y apuesto; y Yahvé
está con él.''

\bibleverse{19} Por eso Saúl envió mensajeros a Jesé y le dijo:
``Envíame a tu hijo David, que está con las ovejas''.

\bibleverse{20} Jesé tomó un asno cargado de pan, un recipiente de vino
y un cabrito, y los envió por medio de su hijo David a Saúl.
\bibleverse{21} David llegó a Saúl y se presentó ante él. Lo amaba
mucho, y se convirtió en su portador de la armadura. \bibleverse{22}
Saúl envió a decir a Isaí: ``Por favor, deja que David se presente ante
mí, porque ha encontrado gracia ante mis ojos''. \bibleverse{23} Cuando
el espíritu de Dios estaba sobre Saúl, David tomó el arpa y tocó con su
mano; así Saúl se refrescó y se puso bien, y el espíritu malo se alejó
de él. \footnote{\textbf{16:23} 1Sam 16,14}

\hypertarget{david-y-el-campeuxf3n-enemigo-goliat}{%
\subsection{David y el campeón enemigo
Goliat}\label{david-y-el-campeuxf3n-enemigo-goliat}}

\hypertarget{section-16}{%
\section{17}\label{section-16}}

\bibleverse{1} Los filisteos reunieron sus ejércitos para combatir, y se
reunieron en Soco, que pertenece a Judá, y acamparon entre Soco y Azeca,
en Efesdammim. \bibleverse{2} Saúl y los hombres de Israel se reunieron
y acamparon en el valle de Ela, y prepararon la batalla contra los
filisteos. \bibleverse{3} Los filisteos estaban en la montaña de un
lado, e Israel estaba en la montaña del otro lado; y había un valle
entre ellos. \bibleverse{4} Del campamento de los filisteos salió un
campeón llamado Goliat de Gat, cuya altura era de seis codos y un
palmo.\footnote{\textbf{17:4} Un codo es la longitud desde la punta del
  dedo corazón hasta el codo del brazo de un hombre, es decir, unas 18
  pulgadas o 46 centímetros. Un palmo es la longitud desde la punta del
  pulgar de un hombre hasta la punta de su dedo meñique cuando su mano
  está extendida (aproximadamente medio codo, o 9 pulgadas, o 22,8 cm.)
  Por lo tanto, Goliat medía aproximadamente 9 pies y 9 pulgadas o 2,97
  metros de altura.} \footnote{\textbf{17:4} Jos 11,22} \bibleverse{5}
Tenía un casco de bronce en la cabeza y llevaba una cota de malla, cuyo
peso era de cinco mil siclos\footnote{\textbf{17:5} Un siclo equivale a
  unos 10 gramos o a unas 0,35 onzas, por lo que 5000 siclos equivalen a
  unos 50 kilogramos o 110 libras.} de bronce. \bibleverse{6} Tenía
espinilleras de bronce en las piernas y una jabalina de bronce entre los
hombros. \bibleverse{7} El asta de su lanza era como una viga de
tejedor, y la punta de su lanza pesaba seiscientos siclos de
hierro.\footnote{\textbf{17:7} Un siclo equivale a unos 10 gramos o a
  unas 0,35 onzas, por lo que 600 siclos son unos 6 kilogramos o unas 13
  libras.} Su escudero iba delante de él. \bibleverse{8} Se puso de pie
y gritó a los ejércitos de Israel, y les dijo: ``¿Por qué habéis salido
a preparar vuestra batalla? ¿Acaso no soy yo un filisteo, y vosotros
siervos de Saúl? Escoged a un hombre para vosotros, y que baje hacia mí.
\bibleverse{9} Si es capaz de luchar conmigo y de matarme, entonces
seremos tus siervos; pero si yo venzo contra él y lo mato, entonces
seréis nuestros siervos y nos serviréis.'' \bibleverse{10} El filisteo
dijo: ``¡Desafío hoy a los ejércitos de Israel! Dame un hombre, para que
luchemos juntos''. \footnote{\textbf{17:10} 2Re 19,4; 2Re 19,16}

\bibleverse{11} Cuando Saúl y todo Israel oyeron estas palabras del
filisteo, se espantaron y tuvieron mucho miedo.

\hypertarget{david-enviado-por-su-padre-a-sus-hermanos-en-el-campamento-estuxe1-indignado-por-la-arrogancia-de-goliat-y-se-siente-llamado-a-pelear-con-uxe9l}{%
\subsection{David, enviado por su padre a sus hermanos en el campamento,
está indignado por la arrogancia de Goliat y se siente llamado a pelear
con
él}\label{david-enviado-por-su-padre-a-sus-hermanos-en-el-campamento-estuxe1-indignado-por-la-arrogancia-de-goliat-y-se-siente-llamado-a-pelear-con-uxe9l}}

\bibleverse{12} David era hijo de aquel efrateo de Belén de Judá, que se
llamaba Isaí, y tenía ocho hijos. El hombre era un anciano en los días
de Saúl. \footnote{\textbf{17:12} 1Sam 16,1} \bibleverse{13} Los tres
hijos mayores de Isaí habían ido en pos de Saúl a la batalla; y los
nombres de sus tres hijos que fueron a la batalla eran Eliab, el
primogénito, y junto a él Abinadab, y el tercero Samá. \footnote{\textbf{17:13}
  1Sam 16,6; 1Sam 16,8-9} \bibleverse{14} David era el menor, y los tres
mayores siguieron a Saúl. \bibleverse{15} David iba y venía de Saúl para
apacentar las ovejas de su padre en Belén.

\bibleverse{16} El filisteo se acercó por la mañana y por la tarde, y se
presentó durante cuarenta días.

\bibleverse{17} Jesé dijo a su hijo David: ``Toma ahora para tus
hermanos un efa\footnote{\textbf{17:17} 1 efa equivale a unos 22 litros
  o a 2/3 de una fanega} de este grano tostado y estos diez panes, y
llévalos rápidamente al campamento para tus hermanos; \bibleverse{18} y
lleva estos diez quesos al capitán de sus mil; y mira cómo están tus
hermanos, y trae noticias.'' \bibleverse{19} Saúl, ellos y todos los
hombres de Israel estaban en el valle de Elah, luchando contra los
filisteos.

\bibleverse{20} David se levantó de madrugada, dejó las ovejas con un
cuidador, tomó las provisiones y se fue, como le había ordenado Isaí.
Llegó al lugar de los carros cuando el ejército que salía a la lucha
gritaba para la batalla. \bibleverse{21} Israel y los filisteos se
prepararon para la batalla, ejército contra ejército. \bibleverse{22}
David dejó su equipaje en manos del guardián del equipaje y corrió hacia
el ejército, y llegó a saludar a sus hermanos. \bibleverse{23} Mientras
hablaba con ellos, he aquí que el campeón, el filisteo de Gat, de nombre
Goliat, salió de las filas de los filisteos y dijo las mismas palabras;
y David las oyó. \bibleverse{24} Todos los hombres de Israel, al ver al
hombre, huyeron de él y se aterrorizaron. \bibleverse{25} Los hombres de
Israel dijeron: ``¿Habéis visto a este hombre que ha subido? Seguramente
ha subido para desafiar a Israel. El rey dará grandes riquezas al hombre
que lo mate, y le dará su hija, y hará que la casa de su padre quede
libre de impuestos en Israel.''

\bibleverse{26} David habló con los hombres que estaban a su lado,
diciendo: ``¿Qué se hará con el hombre que mate a este filisteo y quite
el oprobio a Israel? Porque ¿quién es este filisteo incircunciso, para
que desafíe a los ejércitos del Dios vivo?''

\bibleverse{27} La gente le respondió así: ``Así se hará con el que lo
mate''.

\bibleverse{28} Eliab, su hermano mayor, oyó cuando hablaba con los
hombres; y la ira de Eliab ardió contra David, y dijo: ``¿Por qué has
bajado? ¿Con quién has dejado esas pocas ovejas en el desierto? Conozco
tu orgullo y la maldad de tu corazón; porque has descendido para ver la
batalla''.

\bibleverse{29} David dijo: ``¿Qué he hecho ahora? ¿No hay una causa?''
\bibleverse{30} Se apartó de él hacia otro, y volvió a hablar así; y el
pueblo volvió a responderle de la misma manera.

\hypertarget{david-se-ofrece-a-duelo-pero-rechaza-la-armadura-de-sauxfal-y-solo-usa-su-honda-como-arma}{%
\subsection{David se ofrece a duelo, pero rechaza la armadura de Saúl y
solo usa su honda como
arma}\label{david-se-ofrece-a-duelo-pero-rechaza-la-armadura-de-sauxfal-y-solo-usa-su-honda-como-arma}}

\bibleverse{31} Al oír las palabras que David había dicho, las
repitieron ante Saúl, y éste mandó a buscarlo. \bibleverse{32} David
dijo a Saúl: ``Que no desfallezca el corazón de nadie a causa de él. Tu
siervo irá a pelear con este filisteo''.

\bibleverse{33} Saúl dijo a David: ``No eres capaz de ir contra ese
filisteo para luchar con él, pues tú eres sólo un joven, y él un hombre
de guerra desde su juventud.''

\bibleverse{34} David dijo a Saúl: ``Tu siervo estaba cuidando las
ovejas de su padre; y cuando vino un león o un oso y se llevó un cordero
del rebaño, \bibleverse{35} salí tras él, lo golpeé y lo rescaté de su
boca. Cuando se levantó contra mí, lo agarré por la barba, lo golpeé y
lo maté. \bibleverse{36} Tu siervo golpeó al león y al oso. Este
filisteo incircunciso será como uno de ellos, pues ha desafiado a los
ejércitos del Dios vivo.'' \bibleverse{37} David dijo: ``El Señor, que
me libró de la zarpa del león y de la zarpa del oso, me librará de la
mano de este filisteo.'' Saúl le dijo a David: ``Ve, Yahvé estará
contigo''.

\bibleverse{38} Saúl vistió a David con sus ropas. Le puso un casco de
bronce en la cabeza y lo vistió con una cota de malla. \bibleverse{39}
David se ató la espada a la ropa y trató de moverse, pues no la había
probado. David le dijo a Saúl: ``No puedo ir con esto, pues no lo he
probado''. Entonces David se los quitó.

\bibleverse{40} Tomó su bastón en la mano, y escogió para sí cinco
piedras lisas del arroyo, y las puso en el zurrón de su bolsa de pastor
que tenía. Tenía su honda en la mano, y se acercó al filisteo.
\footnote{\textbf{17:40} 1Cró 11,23}

\hypertarget{la-lucha-victoriosa-de-david-con-goliat}{%
\subsection{La lucha victoriosa de David con
Goliat}\label{la-lucha-victoriosa-de-david-con-goliat}}

\bibleverse{41} El filisteo caminó y se acercó a David, y el hombre que
llevaba el escudo iba delante de él. \bibleverse{42} Cuando el filisteo
miró a su alrededor y vio a David, lo menospreció, pues era un joven
rubio y de buen aspecto. \footnote{\textbf{17:42} 1Sam 16,12}
\bibleverse{43} El filisteo dijo a David: ``¿Soy un perro, para que
vengas a mí con palos?'' El filisteo maldijo a David por sus dioses.
\bibleverse{44} El filisteo dijo a David: ``Ven a mí, y daré tu carne a
las aves del cielo y a los animales del campo''. \footnote{\textbf{17:44}
  Ezeq 29,5}

\bibleverse{45} Entonces David dijo al filisteo: ``Tú vienes a mí con
espada, con lanza y con jabalina; pero yo vengo a ti en nombre del Señor
de los Ejércitos, el Dios de los ejércitos de Israel, a quien has
desafiado. \bibleverse{46} Hoy, Yahvé te entregará en mi mano. Te
golpearé y te quitaré la cabeza. Entregaré hoy los cadáveres del
ejército de los filisteos a las aves del cielo y a las fieras de la
tierra, para que toda la tierra sepa que hay un Dios en Israel,
\bibleverse{47} y para que toda esta asamblea sepa que Yahvé no salva
con espada y lanza; porque la batalla es de Yahvé, y él te entregará en
nuestra mano.''

\bibleverse{48} Cuando el filisteo se levantó y caminó y se acercó para
encontrarse con David, éste se apresuró y corrió hacia el ejército para
encontrarse con el filisteo. \bibleverse{49} David metió la mano en su
bolsa, tomó una piedra y la lanzó, e hirió al filisteo en la frente. La
piedra se hundió en su frente, y él cayó de bruces a la tierra.
\bibleverse{50} Entonces David se impuso al filisteo con la honda y con
la piedra, e hirió al filisteo y lo mató; pero David no tenía espada en
la mano. \bibleverse{51} Entonces David corrió, se paró sobre el
filisteo, tomó su espada, la sacó de su vaina, lo mató y le cortó la
cabeza con ella. Cuando los filisteos vieron que su campeón había
muerto, huyeron. \bibleverse{52} Los hombres de Israel y de Judá se
levantaron y gritaron, y persiguieron a los filisteos hasta Gai y hasta
las puertas de Ecrón. Los heridos de los filisteos cayeron por el camino
de Shaaraim, hasta Gat y Ecrón. \bibleverse{53} Los hijos de Israel
volvieron de perseguir a los filisteos, y saquearon su campamento.
\bibleverse{54} David tomó la cabeza del filisteo y la llevó a
Jerusalén, pero puso su armadura en su tienda.

\hypertarget{sauxfal-pregunta-por-david}{%
\subsection{Saúl pregunta por David}\label{sauxfal-pregunta-por-david}}

\bibleverse{55} Cuando Saúl vio que David salía contra el filisteo, le
dijo a Abner, el capitán del ejército: ``Abner, ¿de quién es hijo este
joven?'' Abner dijo: ``Como vive tu alma, oh rey, no puedo saberlo''.
\footnote{\textbf{17:55} 1Sam 14,50}

\bibleverse{56} El rey dijo: ``¡Investiga de quién es hijo el joven!''

\bibleverse{57} Cuando David regresó de la matanza del filisteo, Abner
lo tomó y lo llevó ante Saúl con la cabeza del filisteo en la mano.
\bibleverse{58} Saúl le dijo: ``¿De quién eres hijo, joven?'' David
respondió: ``Soy hijo de tu siervo Jesé, el de Belén''.

\hypertarget{david-llega-a-la-corte-de-sauxfal-su-relaciuxf3n-con-sauxfal-y-jonatuxe1n}{%
\subsection{David llega a la corte de Saúl; su relación con Saúl y
Jonatán}\label{david-llega-a-la-corte-de-sauxfal-su-relaciuxf3n-con-sauxfal-y-jonatuxe1n}}

\hypertarget{section-17}{%
\section{18}\label{section-17}}

\bibleverse{1} Cuando terminó de hablar con Saúl, el alma de Jonatán se
unió al alma de David, y Jonatán lo amó como a su propia alma.
\bibleverse{2} Ese día Saúl lo apresó y no lo dejó volver a la casa de
su padre. \footnote{\textbf{18:2} 1Sam 16,22; 1Sam 17,15} \bibleverse{3}
Entonces Jonatán y David hicieron un pacto, porque él lo amaba como a su
propia alma. \footnote{\textbf{18:3} 1Sam 19,1; 1Sam 20,17; 1Sam 23,18;
  2Sam 1,26; 2Sam 21,7} \bibleverse{4} Jonatán se despojó de la túnica
que llevaba puesta y se la dio a David con sus ropas, incluyendo su
espada, su arco y su faja.

\bibleverse{5} David salía a dondequiera que Saúl lo enviaba, y se
comportaba con sabiduría; y Saúl lo puso al frente de los hombres de
guerra. Fue bueno a los ojos de todo el pueblo, y también a los ojos de
los servidores de Saúl. \footnote{\textbf{18:5} 1Sam 18,14}

\hypertarget{regreso-festivo-de-los-guerreros-david-fue-celebrado-como-el-vencedor-por-la-gente}{%
\subsection{Regreso festivo de los guerreros; David fue celebrado como
el vencedor por la
gente}\label{regreso-festivo-de-los-guerreros-david-fue-celebrado-como-el-vencedor-por-la-gente}}

\bibleverse{6} Cuando David regresó de la matanza del filisteo, las
mujeres salieron de todas las ciudades de Israel, cantando y bailando, a
recibir al rey Saúl con panderetas, con alegría y con instrumentos de
música. \footnote{\textbf{18:6} Jue 11,34} \bibleverse{7} Las mujeres
cantaban entre sí mientras tocaban, y decían``Saúl ha matado a sus
miles, y David sus diez mil''. \footnote{\textbf{18:7} 1Sam 21,11; 1Sam
  29,5}

\bibleverse{8} Saúl se enojó mucho, y este dicho le desagradó. Dijo: ``A
David le han acreditado diez mil, y a mí sólo me han acreditado mil.
¿Qué más puede tener él sino el reino?''. \bibleverse{9} Saúl vigiló a
David desde ese día en adelante.

\hypertarget{david-odiado-mortalmente-por-sauxfal-demuestra-ser-un-huxe9roe-de-guerra}{%
\subsection{David, odiado mortalmente por Saúl, demuestra ser un héroe
de
guerra}\label{david-odiado-mortalmente-por-sauxfal-demuestra-ser-un-huxe9roe-de-guerra}}

\bibleverse{10} Al día siguiente, un espíritu maligno de Dios se apoderó
poderosamente de Saúl, y profetizó en medio de la casa. David jugaba con
su mano, como lo hacía cada día. Saúl tenía su lanza en la mano;
\footnote{\textbf{18:10} 1Sam 16,14} \bibleverse{11} y Saúl arrojó la
lanza, pues dijo: ``¡Clavaré a David contra la pared!'' David escapó de
su presencia dos veces. \footnote{\textbf{18:11} 1Sam 19,10; 1Sam 20,33}
\bibleverse{12} Saúl tenía miedo de David, porque Yahvé estaba con él y
se había alejado de Saúl. \bibleverse{13} Por eso Saúl lo apartó de su
presencia y lo puso como jefe de mil, y salió y entró delante del
pueblo.

\bibleverse{14} David se comportó sabiamente en todos sus caminos, y el
Señor estaba con él. \footnote{\textbf{18:14} 1Sam 18,5} \bibleverse{15}
Cuando Saúl vio que se comportaba con mucha sabiduría, le tuvo miedo.
\bibleverse{16} Pero todo Israel y Judá amaban a David, porque salía y
entraba delante de ellos.

\hypertarget{david-engauxf1ado-para-casarse-con-la-hija-mayor-de-sauxfal-se-casuxf3-con-su-hermana-menor-michal}{%
\subsection{David, engañado para casarse con la hija mayor de Saúl, se
casó con su hermana menor,
Michal}\label{david-engauxf1ado-para-casarse-con-la-hija-mayor-de-sauxfal-se-casuxf3-con-su-hermana-menor-michal}}

\bibleverse{17} Saúl dijo a David: ``He aquí mi hija mayor Merab. Te la
daré como esposa. Sólo sé valiente para mí y lucha en las batallas de
Yahvé''. Porque Saúl dijo: ``No dejes que mi mano esté sobre él, sino
que la mano de los filisteos esté sobre él''.

\bibleverse{18} David dijo a Saúl: ``¿Quién soy yo, y qué es mi vida, o
la familia de mi padre en Israel, para que sea yerno del rey?''

\bibleverse{19} Pero en el momento en que Merab, la hija de Saúl, debía
ser entregada a David, fue dada como esposa a Adriel el meholatí.
\footnote{\textbf{18:19} Jue 15,2}

\hypertarget{el-servicio-militar-de-david-para-la-novia}{%
\subsection{El servicio militar de David para la
novia}\label{el-servicio-militar-de-david-para-la-novia}}

\bibleverse{20} Mical, hija de Saúl, amaba a David; se lo contaron a
Saúl, y el asunto le agradó. \bibleverse{21} Saúl dijo: ``Se la
entregaré para que le sirva de lazo y para que la mano de los filisteos
esté contra él''. Por eso Saúl dijo a David por segunda vez: ``Hoy serás
mi yerno''.

\bibleverse{22} Saúl ordenó a sus siervos: ``Hablen con David en secreto
y díganle: `He aquí que el rey se complace en ti, y todos sus siervos te
aman. Ahora, pues, sé el yerno del rey'\,''. \footnote{\textbf{18:22}
  1Sam 22,14}

\bibleverse{23} Los siervos de Saúl dijeron esas palabras a los oídos de
David. David dijo: ``¿Os parece poca cosa ser yerno del rey, ya que soy
un hombre pobre y poco conocido?''.

\bibleverse{24} Los sirvientes de Saúl le dijeron: ``David habló así''.

\bibleverse{25} Saúl dijo: ``Dile a David que el rey no desea otra dote
que cien prepucios de los filisteos, para vengarse de los enemigos del
rey''. Entonces Saúl pensó que haría caer a David por mano de los
filisteos. \bibleverse{26} Cuando sus siervos le dijeron a David estas
palabras, le pareció bien ser el yerno del rey. Antes del plazo,
\bibleverse{27} David se levantó y fue, él y sus hombres, y mató a
doscientos hombres de los filisteos. Entonces David trajo sus prepucios,
y se los dieron en número completo al rey, para que fuera yerno del rey.
Entonces Saúl le dio a su hija Mical como esposa. \bibleverse{28} Saúl
vio y supo que el Señor estaba con David, y Mical, la hija de Saúl, lo
amaba. \bibleverse{29} Saúl tenía aún más miedo de David, y Saúl era
continuamente enemigo de David. \footnote{\textbf{18:29} 1Sam 18,12}

\bibleverse{30} Entonces salieron los príncipes de los filisteos; y cada
vez que salían, David se comportaba con más sabiduría que todos los
siervos de Saúl, de modo que su nombre era muy estimado.

\hypertarget{la-reconciliaciuxf3n-de-sauxfal-con-david-como-resultado-de-la-intercesiuxf3n-de-jonatuxe1n-despuuxe9s-de-los-repetidos-asesinatos-de-sauxfal-david-huye-a-samuel}{%
\subsection{La reconciliación de Saúl con David como resultado de la
intercesión de Jonatán; Después de los repetidos asesinatos de Saúl,
David huye a
Samuel}\label{la-reconciliaciuxf3n-de-sauxfal-con-david-como-resultado-de-la-intercesiuxf3n-de-jonatuxe1n-despuuxe9s-de-los-repetidos-asesinatos-de-sauxfal-david-huye-a-samuel}}

\hypertarget{section-18}{%
\section{19}\label{section-18}}

\bibleverse{1} Saúl habló con su hijo Jonatán y con todos sus servidores
para que mataran a David. Pero Jonatán, hijo de Saúl, se alegró mucho de
David. \footnote{\textbf{19:1} 1Sam 18,3} \bibleverse{2} Jonatán le dijo
a David: ``Mi padre Saúl quiere matarte. Ahora, pues, cuídate por la
mañana, vive en un lugar secreto y escóndete. \bibleverse{3} Yo saldré y
me pondré al lado de mi padre en el campo donde estás, y hablaré con mi
padre sobre ti; y si veo algo, te lo diré.''

\bibleverse{4} Jonatán habló bien de David a Saúl, su padre, y le dijo:
``No permitas que el rey peque contra su siervo, contra David, porque él
no ha pecado contra ti, y porque sus obras han sido muy buenas para
contigo; \bibleverse{5} porque él puso su vida en su mano e hirió al
filisteo, y Yahvé obró una gran victoria para todo Israel. Tú lo viste y
te alegraste. ¿Por qué, pues, pecarás contra la sangre inocente, matando
a David sin causa?'' \footnote{\textbf{19:5} 1Sam 17,50}

\bibleverse{6} Saúl escuchó la voz de Jonatán y juró: ``Vive Yahvé que
no lo matarán''.

\bibleverse{7} Jonatán llamó a David, y Jonatán le mostró todas esas
cosas. Entonces Jonatán llevó a David ante Saúl, y éste estuvo en su
presencia como antes.

\hypertarget{la-nueva-fortuna-de-david-en-la-guerra-el-repetido-intento-de-asesinato-de-sauxfal}{%
\subsection{La nueva fortuna de David en la guerra; El repetido intento
de asesinato de
Saúl}\label{la-nueva-fortuna-de-david-en-la-guerra-el-repetido-intento-de-asesinato-de-sauxfal}}

\bibleverse{8} Volvió a haber guerra. David salió y luchó con los
filisteos, y los mató con gran mortandad; y ellos huyeron ante él.

\bibleverse{9} Un espíritu maligno de parte de Yahvé estaba sobre Saúl
mientras éste estaba sentado en su casa con su lanza en la mano, y David
tocaba música con su mano. \footnote{\textbf{19:9} 1Sam 18,10-11}
\bibleverse{10} Saúl trató de clavar a David en la pared con la lanza,
pero éste se escabulló de la presencia de Saúl, y clavó la lanza en la
pared. David huyó y escapó esa noche.

\hypertarget{el-escape-de-david-a-su-hogar-y-su-salvaciuxf3n-a-travuxe9s-de-la-astucia-de-michal}{%
\subsection{El escape de David a su hogar y su salvación a través de la
astucia de
Michal}\label{el-escape-de-david-a-su-hogar-y-su-salvaciuxf3n-a-travuxe9s-de-la-astucia-de-michal}}

\bibleverse{11} Saúl envió mensajeros a la casa de David para vigilarlo
y matarlo por la mañana. Mical, la esposa de David, le dijo: ``Si no
salvas tu vida esta noche, mañana te matarán''. \footnote{\textbf{19:11}
  Sal 59,1} \bibleverse{12} Entonces Mical hizo bajar a David por la
ventana. Él se alejó, huyó y escapó. \bibleverse{13} Mical tomó el
terafín\footnote{\textbf{19:13} Los terafines eran ídolos domésticos que
  podían estar asociados a los derechos de herencia de los bienes del
  hogar.} y lo puso en la cama, y le puso una almohada de pelo de cabra
en la cabeza y lo cubrió con ropa. \bibleverse{14} Cuando Saúl envió
mensajeros para llevarse a David, ella dijo: ``Está enfermo''.

\bibleverse{15} Saúl envió a los mensajeros a ver a David, diciendo:
``Tráiganlo a la cama, para que lo mate''. \bibleverse{16} Cuando los
mensajeros entraron, he aquí que el terafín estaba en la cama, con la
almohada de pelo de cabra a la cabeza.

\bibleverse{17} Saúl dijo a Mical: ``¿Por qué me has engañado así y has
dejado ir a mi enemigo, de modo que ha escapado?'' Mical respondió a
Saúl: ``Me dijo: `¡Déjame ir! ¿Por qué he de matarte?''

\hypertarget{david-con-samuel-en-rama-el-rapto-profuxe9tico-de-sauxfal-en-la-casa-profuxe9tica-alluxed}{%
\subsection{David con Samuel en Rama; El rapto profético de Saúl en la
casa profética
allí}\label{david-con-samuel-en-rama-el-rapto-profuxe9tico-de-sauxfal-en-la-casa-profuxe9tica-alluxed}}

\bibleverse{18} David huyó y escapó, y vino a Samuel en Ramá, y le contó
todo lo que Saúl le había hecho. Él y Samuel se fueron a vivir a Naiot.
\bibleverse{19} Le avisaron a Saúl diciendo: ``He aquí que David está en
Naiot, en Ramá''.

\bibleverse{20} Saúl envió mensajeros para apresar a David; y cuando
vieron a la compañía de los profetas profetizando, y a Samuel de pie
como jefe sobre ellos, el Espíritu de Dios vino sobre los mensajeros de
Saúl, y ellos también profetizaron. \footnote{\textbf{19:20} 1Sam
  10,10-12} \bibleverse{21} Cuando se le informó a Saúl, envió otros
mensajeros, y ellos también profetizaron. Saúl volvió a enviar
mensajeros la tercera vez, y también profetizaron. \bibleverse{22}
También fue a Ramá y llegó al gran pozo que está en Secu, y preguntó:
``¿Dónde están Samuel y David?'' Uno dijo: ``He aquí que están en
Naioth, en Ramá''.

\bibleverse{23} Allí fue a Naiot en Ramá. Entonces el Espíritu de Dios
vino también sobre él, y siguió profetizando hasta llegar a Naiot en
Ramá. \bibleverse{24} También se despojó de sus ropas. También profetizó
ante Samuel y se acostó desnudo todo aquel día y toda aquella noche. Por
eso dicen: ``¿También Saúl está entre los profetas?''

\hypertarget{la-reuniuxf3n-de-david-y-la-discusiuxf3n-con-jonatuxe1n-renovaciuxf3n-de-su-alianza-de-amistad}{%
\subsection{La reunión de David y la discusión con Jonatán; Renovación
de su alianza de
amistad}\label{la-reuniuxf3n-de-david-y-la-discusiuxf3n-con-jonatuxe1n-renovaciuxf3n-de-su-alianza-de-amistad}}

\hypertarget{section-19}{%
\section{20}\label{section-19}}

\bibleverse{1} David huyó de Naiot, en Ramá, y vino a decir a Jonatán:
``¿Qué he hecho? ¿Cuál es mi iniquidad? ¿Cuál es mi pecado ante tu
padre, para que busque mi vida?''

\bibleverse{2} Él le dijo: ``Ni mucho menos; no morirás. He aquí que mi
padre no hace nada, ni grande ni pequeño, sino que me lo revela. ¿Por
qué iba mi padre a ocultarme esto? No es así''.

\bibleverse{3} Además, David juró y dijo: ``Tu padre sabe bien que he
hallado gracia ante tus ojos, y dice: `No dejes que Jonatán lo sepa,
para que no se aflija'; pero en verdad, vive Yahvé y vive tu alma, que
sólo hay un paso entre yo y la muerte.''

\bibleverse{4} Entonces Jonatán dijo a David: ``Todo lo que tu alma
desee, lo haré por ti''.

\hypertarget{la-sugerencia-de-david}{%
\subsection{La sugerencia de David}\label{la-sugerencia-de-david}}

\bibleverse{5} David dijo a Jonatán: ``He aquí que mañana es luna nueva,
y no debo dejar de cenar con el rey; pero déjame que me esconda en el
campo hasta el tercer día por la tarde. \bibleverse{6} Si tu padre me
echa de menos, dile: `David me ha pedido encarecidamente que le deje ir
a Belén, su ciudad, porque allí se celebra el sacrificio anual para toda
la familia'. \bibleverse{7} Si él dice: ``Está bien'', tu siervo tendrá
paz; pero si se enoja, debes saber que el mal está determinado por él.
\bibleverse{8} Trata, pues, con benevolencia a tu siervo, porque lo has
llevado a un pacto de Yahvé contigo; pero si hay iniquidad en mí, mátame
tú mismo, pues ¿para qué me has de llevar a tu padre?'' \footnote{\textbf{20:8}
  1Sam 18,3}

\bibleverse{9} Jonatán dijo: ``Lejos de ti, pues si yo supiera que el
mal está determinado por mi padre a venir sobre ti, ¿no te lo diría?''

\bibleverse{10} Entonces David dijo a Jonatán: ``¿Quién me dirá si tu
padre te responde con rudeza?''

\bibleverse{11} Jonatán dijo a David: ``¡Ven! Salgamos al campo''. Ambos
salieron al campo.

\hypertarget{el-juramento-mutuo}{%
\subsection{El juramento mutuo}\label{el-juramento-mutuo}}

\bibleverse{12} Jonatán dijo a David: ``Por Yahvé, el Dios de Israel,
cuando haya sondeado a mi padre mañana a esta hora, o al tercer día, he
aquí que si hay bien hacia David, ¿no enviaré entonces a ti y te lo
revelaré? \bibleverse{13} Que Yahvé haga así con Jonatán y más aún, si a
mi padre le agrada haceros mal, si no os lo revelo y os envío, para que
vayáis en paz. Que Yahvé esté contigo como ha estado con mi padre.
\bibleverse{14} No sólo me mostrarás la bondad amorosa de Yavé mientras
viva, para que no muera; \bibleverse{15} sino que tampoco cortarás tu
bondad de mi casa para siempre, no, cuando Yavé haya cortado a cada uno
de los enemigos de David de la superficie de la tierra.''
\bibleverse{16} Entonces Jonatán hizo un pacto con la casa de David,
diciendo: ``Yahvé lo exigirá de la mano de los enemigos de David.''

\bibleverse{17} Jonatán hizo que David volviera a jurar, por el amor que
le tenía, pues lo amaba como a su propia alma. \footnote{\textbf{20:17}
  1Sam 18,3}

\hypertarget{acordar-el-procedimiento-a-seguir-para-la-comunicaciuxf3n-de-la-informaciuxf3n}{%
\subsection{Acordar el procedimiento a seguir para la comunicación de la
información}\label{acordar-el-procedimiento-a-seguir-para-la-comunicaciuxf3n-de-la-informaciuxf3n}}

\bibleverse{18} Entonces Jonatán le dijo: ``Mañana es luna nueva, y se
te echará de menos, porque tu asiento estará vacío. \bibleverse{19}
Cuando hayas permanecido tres días, baja rápidamente y ven al lugar
donde te escondiste cuando esto empezó, y quédate junto a la piedra
Ezel. \bibleverse{20} Yo lanzaré tres flechas a su lado, como si
disparara a una marca. \bibleverse{21} He aquí que yo enviaré al
muchacho, diciendo: ``¡Ve, busca las flechas! Si le digo al muchacho:
`Mira, las flechas están a este lado tuyo. Tómalas'; entonces ven,
porque hay paz para ti y no hay peligro, vive Yahvé. \bibleverse{22}
Pero si le digo al muchacho: `He aquí que las flechas están más allá de
ti', entonces vete, porque Yahvé te ha enviado. \bibleverse{23} En
cuanto al asunto del que tú y yo hemos hablado, he aquí que Yahvé está
entre tú y yo para siempre.''

\hypertarget{curso-de-las-dos-comidas-del-medioduxeda-en-casa-de-sauxfal-en-la-luna-nueva-y-al-duxeda-siguiente}{%
\subsection{Curso de las dos comidas del mediodía en casa de Saúl en la
luna nueva y al día
siguiente}\label{curso-de-las-dos-comidas-del-medioduxeda-en-casa-de-sauxfal-en-la-luna-nueva-y-al-duxeda-siguiente}}

\bibleverse{24} Entonces David se escondió en el campo. Cuando llegó la
luna nueva, el rey se sentó a comer. \bibleverse{25} El rey se sentó en
su silla, como otras veces, incluso en el asiento junto a la pared; y
Jonatán se puso de pie, y Abner se sentó al lado de Saúl, pero el lugar
de David estaba vacío. \bibleverse{26} Sin embargo, Saúl no dijo nada
ese día, pues pensó: ``Algo le ha sucedido. No está limpio. Seguramente
no está limpio''. \footnote{\textbf{20:26} Lev 15,1; Deut 23,10}

\bibleverse{27} Al día siguiente de la luna nueva, el segundo día, el
lugar de David estaba vacío. Saúl le dijo a su hijo Jonatán: ``¿Por qué
no vino a comer el hijo de Isaí, ni ayer ni hoy?''.

\bibleverse{28} Jonatán respondió a Saúl: ``David me pidió
encarecidamente permiso para ir a Belén. \bibleverse{29} Dijo: `Por
favor, déjame ir, porque nuestra familia tiene un sacrificio en la
ciudad. Mi hermano me ha ordenado que esté allí. Ahora, si he encontrado
gracia ante tus ojos, por favor déjame ir a ver a mis hermanos'. Por eso
no ha venido a la mesa del rey''.

\bibleverse{30} Entonces la ira de Saúl ardió contra Jonatán, y le dijo:
``Hijo de una perversa rebelde, ¿no sé que has elegido al hijo de Isaí
para vergüenza tuya y de tu madre? \bibleverse{31} Porque mientras el
hijo de Isaí viva en la tierra, tú no serás establecido, ni tu reino.
Por lo tanto, ¡envía ahora y tráemelo, porque seguramente morirá!''

\bibleverse{32} Jonatán respondió a su padre Saúl y le dijo: ``¿Por qué
ha de morir? ¿Qué ha hecho?''

\bibleverse{33} Saúl le arrojó su lanza para herirlo. Con esto, Jonatán
supo que su padre estaba decidido a dar muerte a David. \footnote{\textbf{20:33}
  1Sam 18,11} \bibleverse{34} Así que Jonatán se levantó de la mesa con
una furia terrible y no comió nada el segundo día del mes, pues estaba
afligido por David, porque su padre lo había tratado de manera
vergonzosa.

\hypertarget{jonatuxe1n-informa-a-david-de-la-situaciuxf3n-desfavorable-y-se-despide-de-uxe9l}{%
\subsection{Jonatán informa a David de la situación desfavorable y se
despide de
él}\label{jonatuxe1n-informa-a-david-de-la-situaciuxf3n-desfavorable-y-se-despide-de-uxe9l}}

\bibleverse{35} Por la mañana, Jonatán salió al campo a la hora señalada
con David, y un niño pequeño con él. \bibleverse{36} Le dijo a su niño:
``Corre, encuentra ahora las flechas que yo tiro''. Mientras el niño
corría, disparó una flecha más allá de él. \bibleverse{37} Cuando el
niño llegó al lugar de la flecha que Jonatán había disparado, Jonatán
gritó tras el niño y le dijo: ``¿No está la flecha más allá de ti?''
\bibleverse{38} Jonatán gritó tras el muchacho: ``¡Ve rápido!
¡Apresúrate! No te demores''. El muchacho de Jonatán recogió las flechas
y se acercó a su amo. \bibleverse{39} Pero el muchacho no sabía nada.
Sólo Jonatán y David sabían el asunto. \bibleverse{40} Jonatán le dio
las armas a su muchacho y le dijo: ``Ve, llévalas a la ciudad''.

\bibleverse{41} En cuanto el muchacho se fue, David se levantó del sur,
se postró en tierra y se inclinó tres veces. Se besaron y lloraron
mutuamente, y David fue el que más lloró. \footnote{\textbf{20:41} Gén
  33,3-4} \bibleverse{42} Jonatán dijo a David: ``Vete en paz, porque
ambos hemos jurado en nombre de Yavé, diciendo: ``Yavé está entre tú y
yo, y entre mi descendencia y tu descendencia, para siempre''\,''. Él se
levantó y partió; y Jonatán entró en la ciudad.

\hypertarget{david-como-refugiado-en-nob-y-gat-el-asesinato-del-sacerdote-por-parte-de-sauxfal}{%
\subsection{David como refugiado en Nob y Gat; El asesinato del
sacerdote por parte de
Saúl}\label{david-como-refugiado-en-nob-y-gat-el-asesinato-del-sacerdote-por-parte-de-sauxfal}}

\hypertarget{section-20}{%
\section{21}\label{section-20}}

\bibleverse{1} Entonces David vino a Nob a ver al sacerdote Ahimelec.
Ahimelec salió al encuentro de David temblando y le dijo: ``¿Por qué
estás solo y no hay nadie contigo?''. \bibleverse{2} David respondió al
sacerdote Ahimelec: ``El rey me ha mandado hacer algo y me ha dicho:
`Que nadie sepa nada del asunto sobre el que te envío y de lo que te he
mandado. He enviado a los jóvenes a un lugar determinado'.
\bibleverse{3} Ahora, pues, ¿qué hay bajo tu mano? Por favor, dame cinco
panes en la mano, o lo que haya''.

\bibleverse{4} El sacerdote respondió a David y le dijo: ``No tengo pan
común, pero hay pan sagrado; si tan sólo los jóvenes se hubieran
apartado de las mujeres.'' \footnote{\textbf{21:4} Lev 24,5-9; Lev
  22,3-7; Éxod 19,15}

\bibleverse{5} David respondió al sacerdote y le dijo: ``En verdad, las
mujeres han sido apartadas de nosotros como de costumbre estos tres
días. Cuando yo salí, los vasos de los jóvenes eran santos, aunque sólo
era un viaje común. ¿Cuánto más entonces hoy serán santos sus vasos?''
\bibleverse{6} Entonces el sacerdote le dio pan sagrado, pues allí no
había más pan que el pan de muestra que se tomaba delante de Yavé, para
sustituirlo por pan caliente el día en que se quitaba. \footnote{\textbf{21:6}
  Mat 12,3}

\bibleverse{7} Aquel día, un hombre de los siervos de Saúl estaba
detenido ante el Señor, y se llamaba Doeg el edomita, el mejor de los
pastores que pertenecían a Saúl. \footnote{\textbf{21:7} 1Sam 22,9; 1Sam
  22,18}

\bibleverse{8} David dijo a Ajimelec: ``¿No hay aquí bajo tu mano lanza
o espada? Porque no he traído mi espada ni mis armas, porque el asunto
del rey requería premura''.

\bibleverse{9} El sacerdote dijo: ``Mira, la espada de Goliat el
filisteo, a quien mataste en el valle de Elah, está aquí envuelta en un
paño detrás del efod. Si quieres tomarla, tómala, pues aquí no hay otra
más que esa''. David dijo: ``No hay ninguno así. Dámelo''. \footnote{\textbf{21:9}
  1Sam 17,50-51}

\hypertarget{david-se-vuelve-loco-con-el-rey-achis-en-gat}{%
\subsection{David se vuelve loco con el rey Achis en
Gat}\label{david-se-vuelve-loco-con-el-rey-achis-en-gat}}

\bibleverse{10} David se levantó y huyó aquel día por miedo a Saúl, y se
dirigió a Aquis, rey de Gat. \footnote{\textbf{21:10} Sal 56,1}
\bibleverse{11} Los servidores de Aquis le dijeron: ``¿No es éste David
el rey del país? ¿No se cantaban unos a otros sobre él en las danzas,
diciendo,`Saúl ha matado a sus miles, ¿y David sus diez mil?''
\footnote{\textbf{21:11} 1Sam 18,7; 1Sam 29,5}

\bibleverse{12} David guardó estas palabras en su corazón, y tuvo mucho
miedo de Aquis, rey de Gat. \bibleverse{13} Cambió su conducta ante
ellos y se hizo pasar por loco en sus manos, y garabateó en las puertas
de la puerta, y dejó que su saliva cayera sobre su barba. \footnote{\textbf{21:13}
  Sal 34,1} \bibleverse{14} Entonces Aquis dijo a sus siervos: ``Mirad,
veis que el hombre está loco. ¿Por qué, pues, me lo habéis traído?
\bibleverse{15} ¿Acaso me faltan locos, para que hayáis traído a este
sujeto a hacer de loco en mi presencia? ¿Debe entrar este sujeto en mi
casa?''

\hypertarget{la-posterior-huida-de-david-a-adullam-mizpe-en-moab-y-jaar-hereth-en-juduxe1-su-cuidado-por-sus-padres}{%
\subsection{La posterior huida de David a Adullam, Mizpe en Moab y
Jaar-Hereth en Judá; su cuidado por sus
padres}\label{la-posterior-huida-de-david-a-adullam-mizpe-en-moab-y-jaar-hereth-en-juduxe1-su-cuidado-por-sus-padres}}

\hypertarget{section-21}{%
\section{22}\label{section-21}}

\bibleverse{1} David, pues, salió de allí y se escapó a la cueva de
Adulam. Cuando lo oyeron sus hermanos y toda la casa de su padre,
bajaron allí hacia él. \footnote{\textbf{22:1} Sal 57,1} \bibleverse{2}
Todos los que estaban en apuros, todos los que estaban endeudados y
todos los que estaban descontentos se reunieron con él, y él se
convirtió en capitán de ellos. Había con él unos cuatrocientos hombres.
\footnote{\textbf{22:2} Jue 11,3} \bibleverse{3} David fue de allí a
Mizpa de Moab, y le dijo al rey de Moab: ``Te ruego que dejes salir a mi
padre y a mi madre, hasta que sepa lo que Dios hará por mí.''
\bibleverse{4} Los llevó ante el rey de Moab, y vivieron con él todo el
tiempo que David estuvo en la fortaleza. \bibleverse{5} El profeta Gad
le dijo a David: ``No te quedes en la fortaleza. Vete y entra en la
tierra de Judá''. Entonces David partió y llegó al bosque de Heret.
\footnote{\textbf{22:5} 1Sam 23,14; Sal 63,1}

\hypertarget{la-queja-de-sauxfal-a-los-que-lo-rodeaban-en-guibeuxe1-traiciuxf3n-del-edomita-doeg-la-sangrienta-venganza-de-sauxfal-contra-los-sacerdotes-de-nob}{%
\subsection{La queja de Saúl a los que lo rodeaban en Guibeá; Traición
del edomita Doeg; La sangrienta venganza de Saúl contra los sacerdotes
de
Nob}\label{la-queja-de-sauxfal-a-los-que-lo-rodeaban-en-guibeuxe1-traiciuxf3n-del-edomita-doeg-la-sangrienta-venganza-de-sauxfal-contra-los-sacerdotes-de-nob}}

\bibleverse{6} Saúl oyó que David había sido descubierto, con los
hombres que lo acompañaban. Saúl estaba sentado en Guibeá, bajo el
tamarisco de Ramá, con su lanza en la mano, y todos sus siervos estaban
a su alrededor. \bibleverse{7} Saúl dijo a sus siervos que estaban a su
alrededor: ``¡Oigan ahora, benjamitas! ¿Acaso el hijo de Isaí les dará a
todos ustedes campos y viñas? ¿Os hará a todos capitanes de millares y
de centenas? \bibleverse{8} ¿Es por eso que todos ustedes han conspirado
contra mí, y no hay nadie que me revele cuando mi hijo hace un tratado
con el hijo de Isaí, y no hay ninguno de ustedes que se apene por mí, o
que me revele que mi hijo ha incitado a mi siervo contra mí, para
acechar, como sucede hoy?'' \footnote{\textbf{22:8} 1Sam 18,3}

\bibleverse{9} Entonces Doeg el edomita, que estaba junto a los
servidores de Saúl, respondió y dijo: ``He visto al hijo de Jesé llegar
a Nob, a Ajimelec, hijo de Ajitub. \footnote{\textbf{22:9} 1Sam 22,22;
  Sal 52,1} \bibleverse{10} Él consultó a Yavé por él, le dio comida y
le entregó la espada de Goliat el filisteo''. \footnote{\textbf{22:10}
  1Sam 21,6-9}

\hypertarget{el-plato-de-sangre-en-guibeuxe1}{%
\subsection{El plato de sangre en
Guibeá}\label{el-plato-de-sangre-en-guibeuxe1}}

\bibleverse{11} Entonces el rey envió a llamar al sacerdote Ajimelec,
hijo de Ajitub, y a toda la casa de su padre, los sacerdotes que estaban
en Nob; y todos ellos vinieron al rey. \bibleverse{12} Saúl dijo:
``Escucha ahora, hijo de Ajitub''. Él respondió: ``Aquí estoy, mi
señor''.

\bibleverse{13} Saúl le dijo: ``¿Por qué has conspirado contra mí, tú y
el hijo de Jesé, en que le has dado pan y espada, y has consultado a
Dios por él, para que se levante contra mí, para acechar, como hoy?''

\bibleverse{14} Entonces Ahimelec respondió al rey y dijo: ``¿Quién de
todos tus siervos es tan fiel como David, que es yerno del rey, capitán
de tu guardia y honrado en tu casa? \footnote{\textbf{22:14} 1Sam 18,22;
  1Sam 18,27} \bibleverse{15} ¿Acaso he comenzado hoy a preguntar a Dios
por él? ¡Lejos de mí! Que el rey no impute nada a su siervo, ni a toda
la casa de mi padre; porque tu siervo no sabía nada de todo esto, ni
menos ni más.''

\bibleverse{16} El rey dijo: ``Sin duda morirás, Ajimelec, tú y toda la
casa de tu padre''. \bibleverse{17} El rey dijo a la guardia que lo
rodeaba: ``Vuélvanse y maten a los sacerdotes de Yavé, porque también su
mano está con David, y porque sabían que había huido y no me lo
revelaron.'' Pero los servidores del rey no quisieron extender su mano
para caer sobre los sacerdotes de Yahvé.

\bibleverse{18} El rey le dijo a Doeg: ``¡Vuelve y ataca a los
sacerdotes!'' Doeg el edomita se volvió y atacó a los sacerdotes, y ese
día mató a ochenta y cinco personas que llevaban un efod de lino.
\bibleverse{19} Hirió a Nob, la ciudad de los sacerdotes, a filo de
espada, tanto a hombres como a mujeres, niños y lactantes, y ganado,
asnos y ovejas, a filo de espada. \footnote{\textbf{22:19} 1Sam 21,1}

\hypertarget{el-sacerdote-fugitivo-abjathar-encuentra-una-recepciuxf3n-amistosa-con-david}{%
\subsection{El sacerdote fugitivo Abjathar encuentra una recepción
amistosa con
David}\label{el-sacerdote-fugitivo-abjathar-encuentra-una-recepciuxf3n-amistosa-con-david}}

\bibleverse{20} Uno de los hijos de Ajimelec, hijo de Ajitub, llamado
Abiatar, escapó y huyó tras David. \bibleverse{21} Abiatar le dijo a
David que Saúl había matado a los sacerdotes de Yahvé.

\bibleverse{22} David le dijo a Abiatar: ``Yo sabía que ese día, cuando
Doeg el edomita estaba allí, seguramente se lo diría a Saúl. Soy
responsable de la muerte de todas las personas de la casa de tu padre.
\footnote{\textbf{22:22} 1Sam 22,9} \bibleverse{23} Quédate conmigo. No
tengas miedo, porque el que busca mi vida busca la tuya. Estarás a salvo
conmigo''.

\hypertarget{david-en-el-desierto-de-juduxe1-en-kegila-y-maon-su-uxfaltimo-encuentro-con-jonathan-traiciuxf3n-de-los-sifitas}{%
\subsection{David en el desierto de Judá (en Kegila y Maon); su último
encuentro con Jonathan; Traición de los
sifitas}\label{david-en-el-desierto-de-juduxe1-en-kegila-y-maon-su-uxfaltimo-encuentro-con-jonathan-traiciuxf3n-de-los-sifitas}}

\hypertarget{section-22}{%
\section{23}\label{section-22}}

\bibleverse{1} Le dijeron a David: ``He aquí que los filisteos combaten
contra Keila y roban las eras''. \footnote{\textbf{23:1} Jos 15,44}

\bibleverse{2} Por lo tanto, David consultó a Yahvé, diciendo: ``¿Debo
ir a golpear a estos filisteos?'' Yahvé dijo a David: ``Ve a golpear a
los filisteos y salva a Keilah''.

\bibleverse{3} Los hombres de David le dijeron: ``He aquí que tenemos
miedo aquí en Judá. ¿Cuánto más si vamos a Keila contra los ejércitos de
los filisteos?''

\bibleverse{4} Entonces David volvió a consultar a Yavé. Yahvé le
respondió y le dijo: ``Levántate, baja a Keila, porque entregaré a los
filisteos en tu mano''.

\bibleverse{5} David y sus hombres fueron a Keila y lucharon contra los
filisteos, y se llevaron su ganado, y los mataron con una gran matanza.
Así David salvó a los habitantes de Keila. \footnote{\textbf{23:5} 1Sam
  19,8}

\bibleverse{6} Cuando Abiatar, hijo de Ajimelec, huyó con David a Keilá,
bajó con un efod en la mano. \footnote{\textbf{23:6} 1Sam 22,20}

\bibleverse{7} Saúl fue informado de que David había llegado a Keila.
Saúl dijo: ``Dios lo ha entregado en mi mano, pues está encerrado al
entrar en una ciudad que tiene puertas y rejas''. \bibleverse{8} Saúl
convocó a todo el pueblo a la guerra, para bajar a Keila a sitiar a
David y a sus hombres. \bibleverse{9} David sabía que Saúl estaba
tramando una travesura contra él. Le dijo al sacerdote Abiatar: ``Trae
el efod aquí''. \footnote{\textbf{23:9} 1Sam 30,7} \bibleverse{10}
Entonces David dijo: ``Oh Yahvé, Dios de Israel, tu siervo ha oído
ciertamente que Saúl pretende venir a Keila para destruir la ciudad por
mi causa. \bibleverse{11} ¿Me entregarán los hombres de Keila en sus
manos? ¿Bajará Saúl, como ha oído tu siervo? Yahvé, el Dios de Israel,
te ruego que se lo digas a tu siervo''. Yahvé dijo: ``Bajará''.

\bibleverse{12} Entonces David dijo: ``¿Me entregarán los hombres de
Keila a mí y a mis hombres en manos de Saúl?'' Yahvé dijo: ``Te
entregarán''.

\bibleverse{13} Entonces David y sus hombres, que eran como seiscientos,
se levantaron y salieron de Keila y se fueron a donde pudieron. Saúl se
enteró de que David había escapado de Keila, y renunció a ir allí.

\hypertarget{david-perseguido-por-sauxfal-en-el-desierto-de-siph-su-entrevista-con-jonathan-en-horesa}{%
\subsection{David perseguido por Saúl en el desierto de Siph; su
entrevista con Jonathan en
Horesa}\label{david-perseguido-por-sauxfal-en-el-desierto-de-siph-su-entrevista-con-jonathan-en-horesa}}

\bibleverse{14} David se quedó en el desierto, en las fortalezas, y
permaneció en la región montañosa, en el desierto de Zif. Saúl lo
buscaba todos los días, pero Dios no lo entregó en su mano. \footnote{\textbf{23:14}
  1Sam 23,19; 1Sam 23,29} \bibleverse{15} David vio que Saúl había
salido a buscar su vida. David estaba en el desierto de Zif, en el
bosque.

\bibleverse{16} Jonatán, hijo de Saúl, se levantó y fue a ver a David al
bosque, y fortaleció su mano en Dios. \bibleverse{17} Le dijo: ``No
temas, porque la mano de mi padre Saúl no te encontrará; tú serás rey de
Israel y yo estaré a tu lado, y eso también lo sabe mi padre Saúl.''
\bibleverse{18} Ambos hicieron un pacto ante el Señor. Luego David se
quedó en el bosque y Jonatán se fue a su casa. \footnote{\textbf{23:18}
  1Sam 18,3}

\hypertarget{david-traicionado-por-los-sifitas-y-maravillosamente-salvado-de-sauxfal-en-el-desierto-de-mauxf3n}{%
\subsection{David traicionado por los sifitas y maravillosamente salvado
de Saúl en el desierto de
Maón}\label{david-traicionado-por-los-sifitas-y-maravillosamente-salvado-de-sauxfal-en-el-desierto-de-mauxf3n}}

\bibleverse{19} Entonces los zifitas subieron a Saúl a Gabaa, diciendo:
``¿No se esconde David con nosotros en las fortalezas del bosque, en la
colina de Haquila, que está al sur del desierto? \footnote{\textbf{23:19}
  1Sam 26,1; Sal 54,1} \bibleverse{20} Ahora, pues, oh rey, baja. Según
todo el deseo de tu alma desciende; y nuestra parte será entregarlo en
mano del rey''.

\bibleverse{21} Saúl dijo: ``Bendito seas por Yahvé, pues te has
compadecido de mí. \bibleverse{22} Te ruego que vayas a asegurarte aún
más, y que conozcas y veas dónde está su guarida, y quién lo ha visto
allí; porque me han dicho que es muy astuto. \bibleverse{23} Ve, pues, y
conoce todos los lugares donde se esconde; y vuelve a mí con seguridad,
y yo iré contigo. Si él está en la tierra, yo lo buscaré entre todos los
millares de Judá''.

\bibleverse{24} Se levantaron y se dirigieron a Zif delante de Saúl,
pero David y sus hombres estaban en el desierto de Maón, en el Arabá, al
sur del desierto. \bibleverse{25} Saúl y sus hombres fueron a buscarlo.
Cuando le avisaron a David, bajó a la roca y se quedó en el desierto de
Maón. Cuando Saúl se enteró, persiguió a David en el desierto de Maón.
\bibleverse{26} Saúl iba por este lado de la montaña, y David y sus
hombres por aquel lado; y David se apresuraba a huir por miedo a Saúl,
pues éste y sus hombres rodeaban a David y a los suyos para apresarlos.
\bibleverse{27} Pero llegó un mensajero a Saúl, diciendo: ``¡Apúrate y
ven, porque los filisteos han hecho una incursión en la tierra!''
\bibleverse{28} Así que Saúl regresó de perseguir a David y fue contra
los filisteos. Por eso llamaron a ese lugar Sela Hammahlekoth.
\footnote{\textbf{23:28} ``Sela Hammahlekoth'' significa ``roca de
  separación''.}

\bibleverse{29} David subió de allí y vivió en las fortalezas de En
Gedi.

\hypertarget{la-generosidad-de-david-hacia-sauxfal-en-la-cueva-cerca-de-engedi}{%
\subsection{La generosidad de David hacia Saúl en la cueva cerca de
Engedi}\label{la-generosidad-de-david-hacia-sauxfal-en-la-cueva-cerca-de-engedi}}

\hypertarget{section-23}{%
\section{24}\label{section-23}}

\bibleverse{1} Cuando Saúl volvió de seguir a los filisteos, le dijeron:
``He aquí que David está en el desierto de En Gedi''. \bibleverse{2}
Entonces Saúl tomó a tres mil hombres escogidos de todo Israel y fue a
buscar a David y a sus hombres a las rocas de las cabras salvajes.
\bibleverse{3} Llegó a los corrales de las ovejas junto al camino, donde
había una cueva, y Saúl entró a hacer sus necesidades. David y sus
hombres se encontraban en lo más recóndito de la cueva. \footnote{\textbf{24:3}
  Sal 142,1} \bibleverse{4} Los hombres de David le dijeron: ``He aquí
el día del que Yahvé te dijo: `He aquí que entregaré a tu enemigo en tu
mano, y harás con él lo que te parezca'\,''. Entonces David se levantó y
cortó en secreto la falda del manto de Saúl. \bibleverse{5} Después, el
corazón de David se conmovió porque había cortado la falda de Saúl.
\bibleverse{6} Dijo a sus hombres: ``No permita Yahvé que le haga esto a
mi señor, el ungido de Yahvé, que extienda mi mano contra él, ya que es
el ungido de Yahvé.'' \footnote{\textbf{24:6} 2Sam 1,14; Sal 105,15}
\bibleverse{7} Así que David controló a sus hombres con estas palabras,
y no permitió que se levantaran contra Saúl. Saúl se levantó de la cueva
y siguió su camino. \bibleverse{8} David también se levantó después,
salió de la cueva y gritó tras Saúl, diciendo: ``¡Mi señor el rey!''
Cuando Saúl miró hacia atrás, David se inclinó con el rostro hacia la
tierra, y mostró respeto.

\hypertarget{los-discursos-intercambiados-entre-sauxfal-y-david-su-despedida}{%
\subsection{Los discursos intercambiados entre Saúl y David; su
despedida}\label{los-discursos-intercambiados-entre-sauxfal-y-david-su-despedida}}

\bibleverse{9} David dijo a Saúl: ``¿Por qué escuchas las palabras de
los hombres, que dicen: `He aquí que David busca hacerte daño'?
\bibleverse{10} He aquí que tus ojos han visto cómo el Señor te ha
entregado hoy en mi mano en la cueva. Algunos me instaron a matarte,
pero te perdoné. Dije: `No extenderé mi mano contra mi señor, porque es
el ungido de Yavé'. \bibleverse{11} Además, padre mío, mira, sí, mira la
falda de tu manto en mi mano; porque en que corté la falda de tu manto y
no te maté, conoce y ve que no hay maldad ni desobediencia en mi mano.
No he pecado contra ti, aunque persigas mi vida para quitármela.
\bibleverse{12} Que Yahvé juzgue entre tú y yo, y que Yahvé me vengue de
ti; pero mi mano no estará sobre ti. \footnote{\textbf{24:12} Rom 12,19;
  1Pe 2,23} \bibleverse{13} Como dice el proverbio de los antiguos: ``De
los impíos sale la maldad''; pero mi mano no estará sobre ti.
\bibleverse{14} ¿Contra quién ha salido el rey de Israel? ¿A quién
persigue? ¿A un perro muerto? ¿Una pulga? \bibleverse{15} Sea, pues,
Yahvé el juez, y dicte sentencia entre tú y yo, y vea, y defienda mi
causa, y me libre de tu mano.''

\bibleverse{16} Cuando David terminó de decir estas palabras a Saúl,
éste dijo: ``¿Es esa tu voz, hijo mío David?'' Saúl alzó la voz y lloró.
\bibleverse{17} Dijo a David: ``Tú eres más justo que yo, pues me has
hecho bien, mientras que yo te he hecho mal. \bibleverse{18} Hoy has
declarado cómo me has tratado bien, porque cuando Yahvé me entregó en tu
mano, no me mataste. \bibleverse{19} Porque si un hombre encuentra a su
enemigo, ¿lo dejará ir ileso? Por eso, que el Señor te recompense bien
por lo que has hecho hoy conmigo. \bibleverse{20} Ahora bien, yo sé que
ciertamente serás rey, y que el reino de Israel será establecido en tu
mano. \footnote{\textbf{24:20} 1Sam 23,17} \bibleverse{21} Júrame, pues,
por Yahvé que no cortarás mi descendencia después de mí, y que no
destruirás mi nombre de la casa de mi padre.''

\bibleverse{22} David juró a Saúl. Saúl se fue a su casa, pero David y
sus hombres subieron a la fortaleza.

\hypertarget{la-muerte-de-samuel-la-locura-de-nabal-david-y-abigail}{%
\subsection{La muerte de Samuel; La locura de Nabal; David y
Abigail}\label{la-muerte-de-samuel-la-locura-de-nabal-david-y-abigail}}

\hypertarget{section-24}{%
\section{25}\label{section-24}}

\bibleverse{1} Samuel murió, y todo Israel se reunió y lo lloró, y lo
enterró en su casa de Ramá. Entonces David se levantó y descendió al
desierto de Parán. \footnote{\textbf{25:1} 1Sam 28,3}

\hypertarget{el-comportamiento-necio-de-nabal-hacia-la-peticiuxf3n-de-david}{%
\subsection{El comportamiento necio de Nabal hacia la petición de
David}\label{el-comportamiento-necio-de-nabal-hacia-la-peticiuxf3n-de-david}}

\bibleverse{2} Había un hombre en Maón cuyas posesiones estaban en el
Carmelo; el hombre era muy grande. Tenía tres mil ovejas y mil cabras, y
estaba esquilando sus ovejas en el Carmelo. \bibleverse{3} El nombre de
aquel hombre era Nabal, y el de su mujer, Abigail. Esta mujer era
inteligente y tenía un rostro hermoso; pero el hombre era huraño y
malvado en sus acciones. Era de la casa de Caleb. \bibleverse{4} David
oyó en el desierto que Nabal estaba esquilando sus ovejas.
\bibleverse{5} David envió a diez jóvenes, y les dijo: ``Suban al
Carmelo y vayan a Nabal y salúdenlo en mi nombre. \bibleverse{6}
Díganle: ``¡Que te vaya bien! ¡La paz sea contigo! ¡La paz sea con tu
casa! ¡La paz sea con todo lo que tienes! \bibleverse{7} He oído que
tienes esquiladores. Tus pastores han estado ahora con nosotros, y no
les hemos hecho ningún daño. Nada les faltó en todo el tiempo que
estuvieron en el Carmelo. \bibleverse{8} Pregunta a tus jóvenes, y ellos
te lo dirán. Por lo tanto, que los jóvenes encuentren favor ante tus
ojos, porque venimos en un buen día. Por favor, da lo que venga a tu
mano a tus siervos y a tu hijo David''.

\bibleverse{9} Cuando llegaron los jóvenes de David, le dijeron a Nabal
todas esas palabras en nombre de David, y esperaron.

\bibleverse{10} Nabal respondió a los siervos de David y dijo: ``¿Quién
es David? ¿Quién es el hijo de Isaí? Hay muchos siervos que se separan
de sus amos en estos días. \bibleverse{11} ¿Debo, pues, tomar mi pan, mi
agua y mi carne que he matado para mis esquiladores, y dárselos a
hombres que no sé de dónde vienen?''

\bibleverse{12} Entonces los jóvenes de David se pusieron en camino y
volvieron, y vinieron a contarle todas estas palabras.

\hypertarget{david-se-lanza-a-la-venganza-abigail-se-entera-de-la-erupciuxf3n-de-su-marido}{%
\subsection{David se lanza a la venganza; Abigail se entera de la
erupción de su
marido}\label{david-se-lanza-a-la-venganza-abigail-se-entera-de-la-erupciuxf3n-de-su-marido}}

\bibleverse{13} David dijo a sus hombres: ``¡Cada uno ponga su espada!''
Cada hombre se puso su espada. David también se puso su espada. Unos
cuatrocientos hombres siguieron a David, y doscientos se quedaron junto
al equipaje.

\bibleverse{14} Pero uno de los jóvenes se lo contó a Abigail, la mujer
de Nabal, diciendo: ``He aquí que David envió mensajeros desde el
desierto para saludar a nuestro amo, y él los insultó. \bibleverse{15}
Pero los hombres se portaron muy bien con nosotros, y no nos hicieron
ningún daño, y no nos faltó nada mientras íbamos con ellos, cuando
estábamos en el campo. \bibleverse{16} Fueron un muro para nosotros
tanto de noche como de día, todo el tiempo que estuvimos con ellos
cuidando las ovejas. \bibleverse{17} Ahora, pues, sabed y considerad lo
que vais a hacer, porque el mal está decidido contra nuestro amo y
contra toda su casa, pues es un tipo tan inútil que no se puede hablar
con él.''

\hypertarget{abigail-usa-muxe9todos-inteligentes-para-evitar-que-david-tome-su-venganza}{%
\subsection{Abigail usa métodos inteligentes para evitar que David tome
su
venganza}\label{abigail-usa-muxe9todos-inteligentes-para-evitar-que-david-tome-su-venganza}}

\bibleverse{18} Entonces Abigail se apresuró a tomar doscientos panes,
dos cántaros de vino, cinco ovejas preparadas, cinco seahs\footnote{\textbf{25:18}
  1 marino equivale a unos 7 litros o 1,9 galones o 0,8 picotazos} de
grano tostado, cien racimos de pasas y doscientos pasteles de higos, y
los puso sobre los asnos. \bibleverse{19} Dijo a sus jóvenes: ``Id
delante de mí. Mirad, voy detrás de vosotros''. Pero no se lo dijo a su
marido, Nabal. \bibleverse{20} Mientras montaba en su asno y bajaba
escondida por el monte, he aquí que David y sus hombres bajaban hacia
ella, y ella les salió al encuentro.

\bibleverse{21} Ahora bien, David había dicho: ``Ciertamente en vano he
guardado todo lo que este hombre tiene en el desierto, para que no le
falte nada de todo lo que le pertenece. Me ha devuelto mal por bien.
\bibleverse{22} Que Dios haga lo mismo con los enemigos de David, y más
aún, si dejo de todo lo que le pertenece a la luz de la mañana tanto
como a uno que orina en una pared''. \footnote{\textbf{25:22}
  ``Jedidiah'' significa ``amado por Yahvé''.} \footnote{\textbf{25:22}
  1Re 14,10}

\bibleverse{23} Cuando Abigail vio a David, se apresuró a bajar de su
asno, y se postró ante David de bruces y se postró en el suelo.
\bibleverse{24} Se postró a sus pies y le dijo: ``¡A mí, señor mío, a mí
me corresponde la culpa! Por favor, deja que tu siervo hable en tus
oídos. Escucha las palabras de tu siervo. \bibleverse{25} Por favor, no
permitas que mi señor preste atención a este inútil de Nabal, pues como
su nombre es, así es él. Nabal es su nombre, y la insensatez está con
él; pero yo, tu siervo, no vi a los jóvenes de mi señor que tú enviaste.
\bibleverse{26} Ahora, pues, señor mío, vive Yahvé y vive tu alma, ya
que Yahvé te ha impedido culparte de la sangre y vengarte con tu propia
mano, ahora, pues, que tus enemigos y los que buscan el mal para mi
señor sean como Nabal. \bibleverse{27} Ahora bien, este presente que tu
siervo ha traído a mi señor, dáselo a los jóvenes que siguen a mi señor.
\bibleverse{28} Por favor, perdona la falta de tu siervo. Porque
ciertamente Yahvé hará de mi señor una casa segura, porque mi señor
pelea las batallas de Yahvé. El mal no se encontrará en ti en todos tus
días. \bibleverse{29} Aunque los hombres se levanten para perseguirte y
buscar tu alma, el alma de mi señor estará atada en el fardo de la vida
con Yavé, tu Dios. Él sacará las almas de tus enemigos como del bolsillo
de una honda. \bibleverse{30} Sucederá que cuando Yahvé haya hecho a mi
señor conforme a todo el bien que ha hablado de ti, y te haya nombrado
príncipe de Israel, \footnote{\textbf{25:30} 2Sam 5,2} \bibleverse{31}
esto no te supondrá ninguna pena, ni ofensa de corazón para mi señor, ni
que hayas derramado sangre sin causa, ni que mi señor se haya vengado.
Cuando Yahvé haya tratado bien a mi señor, entonces acuérdate de tu
siervo''.

\bibleverse{32} David dijo a Abigail: ``¡Bendito sea Yahvé, el Dios de
Israel, que te ha enviado hoy a mi encuentro! \bibleverse{33} Bendita
sea tu discreción, y bendita seas tú, que me has librado hoy de la culpa
de la sangre y de vengarme con mi propia mano. \bibleverse{34} Porque,
ciertamente, vive Yahvé, el Dios de Israel, que me ha impedido hacerte
daño, si no te hubieras apresurado a venir a mi encuentro, ciertamente
no le habría quedado a Nabal, al amanecer, tanto como el que orina en
una pared.''

\bibleverse{35} Entonces David recibió de su mano lo que ella le había
traído. Luego le dijo: ``Sube en paz a tu casa. He aquí que he escuchado
tu voz y he concedido tu petición''.

\hypertarget{la-muerte-repentina-de-nabal-el-matrimonio-de-david-con-abigail-y-ahinoam}{%
\subsection{La muerte repentina de Nabal; El matrimonio de David con
Abigail (y
Ahinoam)}\label{la-muerte-repentina-de-nabal-el-matrimonio-de-david-con-abigail-y-ahinoam}}

\bibleverse{36} Abigail fue a ver a Nabal, y he aquí que él celebraba
una fiesta en su casa como la fiesta de un rey. El corazón de Nabal
estaba alegre en su interior, pues estaba muy borracho. Por eso no le
dijo nada hasta la luz de la mañana. \bibleverse{37} Por la mañana,
cuando el vino se le fue a Nabal, su mujer le contó estas cosas; y su
corazón se apagó dentro de él, y quedó como una piedra. \bibleverse{38}
Unos diez días después, el Señor hirió a Nabal, de modo que murió.
\bibleverse{39} Cuando David se enteró de que Nabal había muerto, dijo:
``Bendito sea Yahvé, que ha defendido la causa de mi afrenta de la mano
de Nabal, y que ha apartado a su siervo del mal. Yahvé ha devuelto la
maldad de Nabal sobre su propia cabeza''. David envió a hablar sobre
Abigail, para tomarla como esposa. \bibleverse{40} Cuando los siervos de
David fueron a buscar a Abigail al Carmelo, le hablaron diciendo:
``David nos ha enviado a ti para tomarte como esposa.''

\bibleverse{41} Ella se levantó y se inclinó con el rostro hacia la
tierra, y dijo: ``He aquí que tu sierva es una sierva para lavar los
pies de los siervos de mi señor.'' \bibleverse{42} Abigail se levantó
apresuradamente y montó en un asno con sus cinco criadas que la seguían;
y fue tras los mensajeros de David y se convirtió en su esposa.
\footnote{\textbf{25:42} 1Sam 27,3; 1Sam 30,5} \bibleverse{43} David
también tomó a Ahinoam de Jezreel, y ambas fueron sus esposas.

\bibleverse{44} Saúl había dado a su hija Mical, esposa de David, a
Palti, hijo de Lais, que era de Galim. \footnote{\textbf{25:44} 2Sam
  3,15}

\hypertarget{la-renovada-generosidad-de-david-hacia-sauxfal-en-el-desierto-de-siph}{%
\subsection{La renovada generosidad de David hacia Saúl en el desierto
de
Siph}\label{la-renovada-generosidad-de-david-hacia-sauxfal-en-el-desierto-de-siph}}

\hypertarget{section-25}{%
\section{26}\label{section-25}}

\bibleverse{1} Los zifitas fueron a ver a Saúl a Gabaa, diciendo: ``¿No
se esconde David en la colina de Hachilá, que está delante del
desierto?'' \footnote{\textbf{26:1} 1Sam 23,19; Sal 54,1} \bibleverse{2}
Entonces Saúl se levantó y bajó al desierto de Zif, llevando consigo a
tres mil hombres escogidos de Israel, para buscar a David en el desierto
de Zif. \bibleverse{3} Saúl acampó en la colina de Hachilá, que está
antes del desierto, junto al camino. Pero David se quedó en el desierto,
y vio que Saúl iba tras él por el desierto. \bibleverse{4} David, pues,
envió espías, y comprendió que Saúl había venido ciertamente.
\bibleverse{5} Entonces David se levantó y llegó al lugar donde Saúl
había acampado, y vio el lugar donde yacía Saúl, con Abner hijo de Ner,
jefe de su ejército. Saúl estaba acostado en el lugar de los carros, y
el pueblo estaba acampado alrededor de él. \footnote{\textbf{26:5} 1Sam
  14,50; 1Sam 17,55}

\bibleverse{6} Entonces David respondió y dijo a Ahimelec el hitita y a
Abisai, hijo de Sarvia, hermano de Joab: ``¿Quién bajará conmigo a Saúl
al campamento?''. Abisai dijo: ``Bajaré con vosotros''. \bibleverse{7}
David y Abisai llegaron al pueblo de noche, y he aquí que Saúl yacía
durmiendo en el lugar de los carros, con su lanza clavada en el suelo a
la altura de su cabeza, y Abner y el pueblo yacían a su alrededor.
\bibleverse{8} Entonces Abisai dijo a David: ``Dios ha entregado hoy a
tu enemigo en tu mano. Ahora, pues, déjame herirlo con la lanza hasta la
tierra de un solo golpe, y no lo heriré la segunda vez''. \footnote{\textbf{26:8}
  2Sam 16,9}

\bibleverse{9} David dijo a Abisai: ``No lo destruyas, porque ¿quién
puede extender su mano contra el ungido de Yavé y quedar libre de
culpa?'' \bibleverse{10} David respondió: ``Vive Yahvé, que Yahvé lo
golpeará; o le llegará el día de morir, o bajará a la batalla y
perecerá. \footnote{\textbf{26:10} 1Sam 28,10; 1Sam 24,12}
\bibleverse{11} Yahvé no permita que yo extienda mi mano contra el
ungido de Yahvé; pero ahora, por favor, toma la lanza que está en su
cabeza y la vasija de agua, y vámonos.''

\bibleverse{12} Entonces David tomó la lanza y la jarra de agua de la
cabeza de Saúl, y se fueron. Nadie lo vio, ni lo supo, ni se despertó,
pues todos estaban dormidos, porque un profundo sueño de Yahvé había
caído sobre ellos. \footnote{\textbf{26:12} Gén 2,21; Gén 15,12}

\hypertarget{la-aclamaciuxf3n-burlona-de-david-a-abner}{%
\subsection{La aclamación burlona de David a
Abner}\label{la-aclamaciuxf3n-burlona-de-david-a-abner}}

\bibleverse{13} Entonces David se pasó al otro lado y se puso en la cima
del monte, muy lejos, habiendo un gran espacio entre ellos;
\bibleverse{14} y David gritó al pueblo y a Abner, hijo de Ner,
diciendo: ``¿No respondes, Abner?'' Entonces Abner respondió: ``¿Quién
eres tú que llamas al rey?''

\bibleverse{15} David dijo a Abner: ``¿No eres un hombre? ¿Quién es como
tú en Israel? ¿Por qué, pues, no has velado por tu señor el rey? Porque
uno del pueblo entró para destruir a tu señor el rey. \bibleverse{16} No
es bueno lo que has hecho. Vive Yahvé, que eres digno de morir, porque
no has velado por tu señor, el ungido de Yahvé. Mira ahora dónde está la
lanza del rey y la jarra de agua que estaba a su cabeza''.

\hypertarget{los-discursos-intercambiados-entre-sauxfal-y-david-la-divergencia-de-ambos}{%
\subsection{Los discursos intercambiados entre Saúl y David; la
divergencia de
ambos}\label{los-discursos-intercambiados-entre-sauxfal-y-david-la-divergencia-de-ambos}}

\bibleverse{17} Saúl reconoció la voz de David y dijo: ``¿Es ésta tu
voz, hijo mío David?'' David dijo: ``Es mi voz, mi señor, oh rey''.
\bibleverse{18} Dijo: ``¿Por qué persigue mi señor a su siervo? ¿Qué he
hecho yo? ¿Qué mal hay en mi mano? \bibleverse{19} Ahora, pues, te ruego
que mi señor el rey escuche las palabras de su siervo. Si es así que el
Señor los ha incitado contra mí, que acepte una ofrenda. Pero si son los
hijos de los hombres, malditos sean ante Yavé, pues me han expulsado hoy
para que no me aferre a la herencia de Yavé, diciendo: ``¡Vete a servir
a otros dioses!'' \bibleverse{20} Ahora, pues, no dejes que mi sangre
caiga a la tierra lejos de la presencia de Yavé, porque el rey de Israel
ha salido a buscar una pulga, como cuando se caza una perdiz en el
monte.''

\bibleverse{21} Entonces Saúl dijo: ``He pecado. Vuélvete, hijo mío
David, porque ya no te haré más daño, ya que mi vida era preciosa a tus
ojos hoy. He aquí que me he hecho el loco, y he errado mucho''.

\bibleverse{22} David respondió: ``¡Mira la lanza, oh rey! Que uno de
los jóvenes venga a buscarla. \bibleverse{23} El Señor pagará a cada uno
su justicia y su fidelidad, porque el Señor te ha entregado hoy en mi
mano, y yo no he querido extender mi mano contra el ungido del Señor.
\bibleverse{24} Así como tu vida fue respetada hoy a mis ojos, que mi
vida sea respetada a los ojos de Yahvé, y que él me libre de toda
opresión.''

\bibleverse{25} Entonces Saúl dijo a David: ``Bendito seas, hijo mío
David. Harás lo que quieras, y seguro que vencerás''. Entonces David se
fue, y Saúl volvió a su lugar.

\hypertarget{la-conversiuxf3n-de-david-a-los-filisteos-su-estancia-con-el-pruxedncipe-filisteo-achis-en-gat-y-en-siclag}{%
\subsection{La conversión de David a los filisteos; su estancia con el
príncipe filisteo Achis en Gat y en
Siclag}\label{la-conversiuxf3n-de-david-a-los-filisteos-su-estancia-con-el-pruxedncipe-filisteo-achis-en-gat-y-en-siclag}}

\hypertarget{section-26}{%
\section{27}\label{section-26}}

\bibleverse{1} David dijo en su corazón: ``Ahora pereceré un día por la
mano de Saúl. No hay nada mejor para mí que escapar a la tierra de los
filisteos; y Saúl se desesperará por buscarme más en todos los límites
de Israel. Así escaparé de su mano''. \bibleverse{2} David se levantó y
pasó, él y los seiscientos hombres que estaban con él, a Aquis hijo de
Maoc, rey de Gat. \footnote{\textbf{27:2} 1Sam 21,10; 1Re 2,39}
\bibleverse{3} David vivía con Aquis en Gat, él y sus hombres, cada uno
con su casa, incluso David con sus dos esposas, Ahinoam la jezreelita y
Abigail la carmelita, esposa de Nabal. \footnote{\textbf{27:3} 1Sam
  25,40-43} \bibleverse{4} Cuando Saúl supo que David había huido a Gat,
dejó de buscarlo.

\bibleverse{5} David dijo a Aquis: ``Si ahora he hallado gracia ante tus
ojos, que me den un lugar en una de las ciudades del país, para que
habite allí. Porque, ¿por qué habría de habitar tu siervo en la ciudad
real contigo?''. \bibleverse{6} Entonces Aquis le dio aquel día Siclag;
por eso Siclag pertenece a los reyes de Judá hasta el día de hoy.
\footnote{\textbf{27:6} Jos 15,31; Jue 1,19} \bibleverse{7} El número de
días que David vivió en el país de los filisteos fue un año completo y
cuatro meses.

\hypertarget{la-vida-privada-de-david-su-engauxf1o-a-los-filisteos}{%
\subsection{La vida privada de David; su engaño a los
filisteos}\label{la-vida-privada-de-david-su-engauxf1o-a-los-filisteos}}

\bibleverse{8} David y sus hombres subieron y asaltaron a los
guesuritas, a los girzitas y a los amalecitas, pues esos eran los
habitantes de la tierra de antaño, en el camino hacia Shur, hasta la
tierra de Egipto. \bibleverse{9} David atacó la tierra y no salvó a
ningún hombre ni a ninguna mujer con vida, y se llevó las ovejas, el
ganado, los asnos, los camellos y la ropa. Luego regresó y llegó a
Aquis.

\bibleverse{10} Aquis dijo: ``¿Contra quién has hecho hoy una
incursión?'' David dijo: ``Contra el sur de Judá, contra el sur de los
jeraelitas y contra el sur de los ceneos''. \bibleverse{11} David no
salvó a ningún hombre ni a ninguna mujer con vida para llevarlos a Gat,
diciendo: ``No sea que cuenten de nosotros, diciendo: ``David hizo esto,
y este ha sido su camino todo el tiempo que ha vivido en el país de los
filisteos.''\,''

\bibleverse{12} Aquis creyó a David, diciendo: ``Ha hecho que su pueblo
Israel lo aborrezca por completo. Por eso será mi siervo para siempre''.
\footnote{\textbf{27:12} Gén 34,30; Éxod 5,21}

\hypertarget{la-guerra-con-los-filisteos-saul-con-el-nigromante-en-endor}{%
\subsection{La guerra con los filisteos; Saul con el nigromante en
Endor}\label{la-guerra-con-los-filisteos-saul-con-el-nigromante-en-endor}}

\hypertarget{section-27}{%
\section{28}\label{section-27}}

\bibleverse{1} En aquellos días, los filisteos reunieron sus ejércitos
para la guerra, para luchar contra Israel. Aquis dijo a David: ``Ten por
seguro que saldrás conmigo en el ejército, tú y tus hombres''.

\bibleverse{2} David dijo a Aquis: ``Así sabrás lo que puede hacer tu
siervo''. Aquis dijo a David: ``Por eso te haré mi guardaespaldas para
siempre''.

\hypertarget{comienzo-de-la-guerra-en-su-perplejidad-sauxfal-decide-cuestionar-un-oruxe1culo-de-los-muertos}{%
\subsection{Comienzo de la guerra; En su perplejidad, Saúl decide
cuestionar un oráculo de los
muertos}\label{comienzo-de-la-guerra-en-su-perplejidad-sauxfal-decide-cuestionar-un-oruxe1culo-de-los-muertos}}

\bibleverse{3} Samuel había muerto, y todo Israel lo había llorado y
enterrado en Ramá, en su propia ciudad. Saúl había expulsado del país a
los que tenían espíritus familiares y a los magos. \footnote{\textbf{28:3}
  1Sam 25,1; Éxod 22,18}

\bibleverse{4} Los filisteos se reunieron y vinieron a acampar en Sunem,
y Saúl reunió a todo Israel y acamparon en Gilboa. \bibleverse{5} Cuando
Saúl vio el ejército de los filisteos, tuvo miedo y su corazón se
estremeció mucho. \bibleverse{6} Cuando Saúl consultó a Yavé, éste no le
respondió ni por sueños, ni por Urim, ni por profetas. \footnote{\textbf{28:6}
  Éxod 28,30; 1Sam 14,37; 1Sam 23,9} \bibleverse{7} Entonces Saúl dijo a
sus servidores: ``Buscadme una mujer que tenga un espíritu familiar,
para que vaya a ella y le pregunte.'' Sus sirvientes le dijeron: ``Mira,
hay una mujer que tiene un espíritu familiar en Endor''. \footnote{\textbf{28:7}
  Hech 16,16}

\hypertarget{saul-con-el-nigromante-en-endor-la-apariciuxf3n-y-profecuxeda-de-la-desgracia-del-espuxedritu-de-samuel}{%
\subsection{Saul con el nigromante en Endor; la aparición y profecía de
la desgracia del espíritu de
Samuel}\label{saul-con-el-nigromante-en-endor-la-apariciuxf3n-y-profecuxeda-de-la-desgracia-del-espuxedritu-de-samuel}}

\bibleverse{8} Saúl se disfrazó y se puso otra ropa, y fue, él y dos
hombres con él, y llegaron a la mujer de noche. Entonces les dijo: ``Por
favor, consulta por mí por el espíritu familiar, y hazme subir a quien
yo te nombre''.

\bibleverse{9} La mujer le dijo: ``Mira, tú sabes lo que ha hecho Saúl,
cómo ha eliminado del país a los que tienen espíritus familiares y a los
magos. ¿Por qué, pues, pones una trampa a mi vida, para causarme la
muerte?''

\bibleverse{10} Saúl le juró por Yahvé, diciendo: ``Vive Yahvé, que no
te sucederá ningún castigo por esto''.

\bibleverse{11} Entonces la mujer dijo: ``¿A quién te voy a subir?''
Dijo: ``Trae a Samuel por mí''.

\bibleverse{12} Cuando la mujer vio a Samuel, gritó con fuerza; y la
mujer habló a Saúl diciendo: ``¿Por qué me has engañado? Porque tú eres
Saúl''.

\bibleverse{13} El rey le dijo: ``¡No tengas miedo! ¿Qué ves?'' La mujer
le dijo a Saúl: ``Veo un dios que sale de la tierra''.

\bibleverse{14} Le dijo: ``¿Qué aspecto tiene?'' Ella dijo: ``Un anciano
se acerca. Está cubierto con un manto''. Saúl percibió que era Samuel, y
se inclinó con el rostro hacia el suelo, mostrando respeto.

\bibleverse{15} Samuel dijo a Saúl: ``¿Por qué me has molestado para
hacerme subir?'' Saúl respondió: ``Estoy muy angustiado, porque los
filisteos me hacen la guerra, y Dios se ha alejado de mí y no me
responde más, ni por profetas ni por sueños. Por eso te he llamado, para
que me des a conocer lo que debo hacer''.

\bibleverse{16} Samuel dijo: ``¿Por qué me preguntas, pues Yahvé se ha
alejado de ti y se ha convertido en tu adversario? \bibleverse{17} El
Señor ha hecho contigo lo que dijo por mí. El Señor ha arrancado el
reino de tus manos y se lo ha dado a tu vecino, a David. \bibleverse{18}
Porque no obedeciste la voz del Señor y no ejecutaste su furia contra
Amalec, por eso el Señor te ha hecho esto hoy. \footnote{\textbf{28:18}
  1Sam 15,18-19} \bibleverse{19} Además, Yahvé entregará a Israel
también con ustedes en manos de los filisteos, y mañana tú y tus hijos
estarán conmigo. El Señor entregará también el ejército de Israel en
manos de los filisteos''. \footnote{\textbf{28:19} 1Sam 31,6}

\hypertarget{efecto-de-la-profecuxeda-sobre-saulo}{%
\subsection{Efecto de la profecía sobre
Saulo}\label{efecto-de-la-profecuxeda-sobre-saulo}}

\bibleverse{20} Entonces Saúl cayó inmediatamente en tierra en toda su
extensión, y se aterrorizó a causa de las palabras de Samuel. No había
fuerzas en él, pues no había comido pan en todo el día ni en toda la
noche.

\bibleverse{21} La mujer se acercó a Saúl y, viendo que estaba muy
turbado, le dijo: ``Mira, tu siervo ha escuchado tu voz, y yo he puesto
mi vida en mi mano, y he escuchado tus palabras que me has dicho.
\bibleverse{22} Ahora, pues, te ruego que escuches también la voz de tu
siervo, y permíteme poner ante ti un bocado de pan. Come, para que
tengas fuerzas cuando sigas tu camino''.

\bibleverse{23} Pero él se negó y dijo: ``No quiero comer''. Pero sus
siervos, junto con la mujer, lo obligaron; y él escuchó su voz. Entonces
se levantó de la tierra y se sentó en la cama. \bibleverse{24} La mujer
tenía en casa un ternero cebado. Se apresuró a matarlo, tomó harina, la
amasó y coció con ella panes sin levadura. \bibleverse{25} Lo llevó ante
Saúl y ante sus sirvientes, y comieron. Luego se levantaron y se fueron
aquella noche.

\hypertarget{el-envuxedo-de-david-a-casa-a-instancias-de-los-sospechosos-pruxedncipes-filisteos}{%
\subsection{El envío de David a casa a instancias de los sospechosos
príncipes
filisteos}\label{el-envuxedo-de-david-a-casa-a-instancias-de-los-sospechosos-pruxedncipes-filisteos}}

\hypertarget{section-28}{%
\section{29}\label{section-28}}

\bibleverse{1} Los filisteos reunieron todos sus ejércitos en Afec, y
los israelitas acamparon junto a la fuente que está en Jezreel.
\footnote{\textbf{29:1} 1Sam 4,1} \bibleverse{2} Los jefes de los
filisteos pasaban de a cientos y de a miles, y David y sus hombres
pasaban en la retaguardia con Aquis.

\bibleverse{3} Entonces los príncipes de los filisteos dijeron: ``¿Y
estos hebreos?'' Aquis dijo a los príncipes de los filisteos: ``¿No es
éste David, el siervo de Saúl, rey de Israel, que ha estado conmigo
estos días, o más bien estos años? No he encontrado ningún defecto en él
desde que cayó hasta hoy''.

\bibleverse{4} Pero los príncipes de los filisteos se enojaron con él, y
le dijeron: ``Haz volver a ese hombre a su lugar donde lo has destinado,
y que no baje con nosotros a la batalla, no sea que en la batalla se
convierta en un adversario nuestro. Porque, ¿con qué debería
reconciliarse este hombre con su señor? ¿No debería ser con las cabezas
de estos hombres? \bibleverse{5} ¿No es éste David, de quien el pueblo
cantaba entre sí en las danzas, diciendo,`Saúl ha matado a sus miles, y
David sus diez mil''. \footnote{\textbf{29:5} 1Sam 18,7}

\bibleverse{6} Entonces Aquis llamó a David y le dijo: ``Vive Yahvé que
has sido recto, y que tu salida y tu entrada conmigo en el ejército es
buena a mis ojos, pues no he encontrado maldad en ti desde el día de tu
llegada a mí hasta hoy. Sin embargo, los señores no te favorecen.
\bibleverse{7} Por tanto, vuelve ahora y vete en paz, para que no
disgustes a los señores de los filisteos.''

\bibleverse{8} David dijo a Aquis: ``¿Pero qué he hecho yo? ¿Qué has
encontrado en tu siervo desde que estoy ante ti hasta hoy, para que no
vaya a luchar contra los enemigos de mi señor el rey?''

\bibleverse{9} Aquis respondió a David: ``Sé que eres bueno ante mis
ojos, como un ángel de Dios. Sin embargo, los príncipes de los filisteos
han dicho: `No subirá con nosotros a la batalla'. \footnote{\textbf{29:9}
  2Sam 19,27} \bibleverse{10} Por tanto, levántate ahora de madrugada
con los siervos de tu señor que han venido contigo; y en cuanto
madrugues y tengas luz, parte.''

\bibleverse{11} Así que David se levantó temprano, él y sus hombres,
para partir por la mañana, para volver a la tierra de los filisteos; y
los filisteos subieron a Jezreel.

\hypertarget{david-encuentra-siclag-devastada-por-los-amalecitas-su-consternaciuxf3n-y-aliento}{%
\subsection{David encuentra Siclag devastada por los amalecitas; su
consternación y
aliento}\label{david-encuentra-siclag-devastada-por-los-amalecitas-su-consternaciuxf3n-y-aliento}}

\hypertarget{section-29}{%
\section{30}\label{section-29}}

\bibleverse{1} Cuando David y sus hombres habían llegado a Siclag al
tercer día, los amalecitas habían hecho una incursión en el sur y en
Siclag, y habían atacado a Siclag y la habían quemado con fuego,
\bibleverse{2} y habían llevado cautivas a las mujeres y a todos los que
estaban en ella, tanto pequeños como grandes. No mataron a ninguno, sino
que se los llevaron y siguieron su camino. \bibleverse{3} Cuando David y
sus hombres llegaron a la ciudad, he aquí que ésta había sido
incendiada, y sus mujeres, sus hijos y sus hijas habían sido llevados
cautivos. \bibleverse{4} Entonces David y el pueblo que estaba con él
alzaron la voz y lloraron hasta que ya no tuvieron fuerzas para llorar.
\bibleverse{5} Las dos esposas de David fueron llevadas cautivas,
Ahinoam la jezreelita y Abigail la esposa de Nabal el carmelita.
\footnote{\textbf{30:5} 1Sam 25,42-43} \bibleverse{6} David estaba muy
afligido, pues el pueblo hablaba de apedrearlo, porque las almas de todo
el pueblo estaban afligidas, cada una por sus hijos y por sus hijas;
pero David se fortaleció en Yahvé, su Dios. \bibleverse{7} David dijo al
sacerdote Abiatar, hijo de Ajimelec: ``Por favor, tráeme el efod''.
Abiatar llevó el efod a David. \footnote{\textbf{30:7} 1Sam 23,9}
\bibleverse{8} David consultó a Yahvé, diciendo: ``Si persigo a esta
tropa, ¿la alcanzaré?''. Él le respondió: ``Persigue, porque seguramente
los alcanzarás, y sin falta lo recuperarás todo''.

\bibleverse{9} Así que David se fue, él y los seiscientos hombres que
estaban con él, y llegaron al arroyo Besor, donde se quedaron los que
quedaron atrás. \bibleverse{10} Pero David siguió, él y cuatrocientos
hombres; porque se quedaron atrás doscientos, que estaban tan débiles
que no podían pasar el arroyo de Besor.

\hypertarget{la-persecuciuxf3n-y-destrucciuxf3n-de-david-de-la-banda-de-ladrones-de-amalecita}{%
\subsection{La persecución y destrucción de David de la banda de
ladrones de
Amalecita}\label{la-persecuciuxf3n-y-destrucciuxf3n-de-david-de-la-banda-de-ladrones-de-amalecita}}

\bibleverse{11} Encontraron a un egipcio en el campo, lo llevaron a
David y le dieron pan, y él comió, y le dieron de beber agua.
\bibleverse{12} Le dieron un trozo de una torta de higos y dos racimos
de pasas. Cuando hubo comido, su espíritu volvió a él, pues no había
comido pan ni bebido agua durante tres días y tres noches. \footnote{\textbf{30:12}
  Jue 15,19} \bibleverse{13} David le preguntó: ``¿A quién perteneces?
¿De dónde eres?'' Dijo: ``Soy un joven egipcio, siervo de un amalecita;
y mi amo me dejó, porque hace tres días me enfermé. \bibleverse{14}
Hicimos una incursión en el sur de los queretanos, en el que pertenece a
Judá y en el sur de Caleb, y quemamos Ziklag con fuego.'' \footnote{\textbf{30:14}
  2Sam 8,18; Jos 14,13}

\bibleverse{15} David le dijo: ``¿Me harás bajar a esta tropa?'' Dijo:
``Júrame por Dios que no me matarás ni me entregarás en manos de mi amo,
y te haré bajar a esta tropa''.

\bibleverse{16} Cuando lo hizo descender, he aquí que estaban esparcidos
por toda la tierra, comiendo, bebiendo y bailando, a causa de todo el
gran botín que habían sacado de la tierra de los filisteos y de la
tierra de Judá. \bibleverse{17} David los golpeó desde el crepúsculo
hasta la tarde del día siguiente. Ningún hombre de ellos escapó de allí,
excepto cuatrocientos jóvenes que montaron en camellos y huyeron.
\bibleverse{18} David recuperó todo lo que los amalecitas habían tomado,
y David rescató a sus dos esposas. \bibleverse{19} No les faltó nada, ni
pequeño ni grande, ni hijos ni hijas, ni botín, ni nada de lo que habían
tomado. David los devolvió a todos. \bibleverse{20} David tomó todos los
rebaños y las vacas, que condujeron delante de los otros ganados, y
dijo: ``Este es el botín de David''.

\hypertarget{david-hace-que-su-pueblo-lleve-ante-la-justicia-a-sus-camaradas}{%
\subsection{David hace que su pueblo lleve ante la justicia a sus
camaradas}\label{david-hace-que-su-pueblo-lleve-ante-la-justicia-a-sus-camaradas}}

\bibleverse{21} David se acercó a los doscientos hombres, que estaban
tan desanimados que no podían seguir a David, a quien también habían
hecho quedarse en el arroyo de Besor, y salieron a recibir a David y al
pueblo que estaba con él. Cuando David se acercó al pueblo, lo saludó.
\bibleverse{22} Entonces todos los hombres malvados y despreciables de
los que iban con David respondieron y dijeron: ``Por no haber ido con
nosotros, no les daremos nada del botín que hemos recuperado, salvo a
cada uno su mujer y sus hijos, para que los lleve y se vaya.''

\bibleverse{23} Entonces David dijo: ``No hagáis eso, hermanos míos, con
lo que nos ha dado Yahvé, que nos ha preservado y ha entregado en
nuestra mano a la tropa que venía contra nosotros. \bibleverse{24}
¿Quién os escuchará en este asunto? Porque así como su parte es el que
baja a la batalla, así será su parte el que se quede con el bagaje. Se
repartirán por igual''. \footnote{\textbf{30:24} Núm 31,27}
\bibleverse{25} Así fue desde aquel día y lo convirtió en estatuto y
ordenanza para Israel hasta el día de hoy.

\hypertarget{david-envuxeda-regalos-a-los-ancianos-en-numerosas-ciudades-de-juduxe1}{%
\subsection{David envía regalos a los ancianos en numerosas ciudades de
Judá}\label{david-envuxeda-regalos-a-los-ancianos-en-numerosas-ciudades-de-juduxe1}}

\bibleverse{26} Cuando David llegó a Siclag, envió parte del botín a los
ancianos de Judá, a sus amigos, diciendo: ``He aquí un regalo para
ustedes del botín de los enemigos de Yavé.'' \bibleverse{27} Lo envió a
los que estaban en Betel, a los que estaban en Ramot del Sur, a los que
estaban en Jattir, \bibleverse{28} a los que estaban en Aroer, a los que
estaban en Sifmot, a los que estaban en Estemoa, \bibleverse{29} a los
que estaban en Racal a los que estaban en las ciudades de los
jeraelitas, a los que estaban en las ciudades de los ceneos,
\bibleverse{30} a los que estaban en Horma, a los que estaban en
Borashan, a los que estaban en Athach, \bibleverse{31} a los que estaban
en Hebrón, y a todos los lugares donde David mismo y sus hombres solían
quedarse.

\hypertarget{la-derrota-de-israel-y-el-desastre-de-sauxfal-y-su-casa}{%
\subsection{La derrota de Israel y el desastre de Saúl y su
casa}\label{la-derrota-de-israel-y-el-desastre-de-sauxfal-y-su-casa}}

\hypertarget{section-30}{%
\section{31}\label{section-30}}

\bibleverse{1} Los filisteos lucharon contra Israel, y los hombres de
Israel huyeron de la presencia de los filisteos y cayeron muertos en el
monte Gilboa. \bibleverse{2} Los filisteos alcanzaron a Saúl y a sus
hijos, y los filisteos mataron a Jonatán, Abinadab y Malquisúa, hijos de
Saúl. \bibleverse{3} La batalla fue dura contra Saúl, y los arqueros lo
alcanzaron; y él se angustió mucho a causa de los arqueros.
\bibleverse{4} Entonces Saúl dijo a su escudero: ``¡Saca tu espada y
traspásame con ella, no sea que vengan estos incircuncisos y me
traspasen y abusen de mí!'' Pero su escudero no quiso, porque estaba
aterrorizado. Por lo tanto, Saúl tomó su espada y cayó sobre ella.
\footnote{\textbf{31:4} Jue 9,54} \bibleverse{5} Cuando su escudero vio
que Saúl estaba muerto, también cayó sobre su espada y murió con él.
\bibleverse{6} Así murió Saúl con sus tres hijos, su escudero y todos
sus hombres aquel mismo día.

\bibleverse{7} Cuando los hombres de Israel que estaban al otro lado del
valle, y los que estaban al otro lado del Jordán, vieron que los hombres
de Israel huían y que Saúl y sus hijos estaban muertos, abandonaron las
ciudades y huyeron, y los filisteos vinieron y vivieron en ellas.

\hypertarget{el-destino-de-los-caduxe1veres-de-sauxfal-y-sus-hijos}{%
\subsection{El destino de los cadáveres de Saúl y sus
hijos}\label{el-destino-de-los-caduxe1veres-de-sauxfal-y-sus-hijos}}

\bibleverse{8} Al día siguiente, cuando los filisteos fueron a despojar
a los muertos, encontraron a Saúl y a sus tres hijos caídos en el monte
Gilboa. \bibleverse{9} Le cortaron la cabeza, le despojaron de su
armadura y enviaron a la tierra de los filisteos a todos los
alrededores, para llevar la noticia a la casa de sus ídolos y al pueblo.
\bibleverse{10} Pusieron su armadura en la casa de Astarot, y sujetaron
su cuerpo al muro de Bet Shan. \bibleverse{11} Cuando los habitantes de
Jabes de Galaad se enteraron de lo que los filisteos le habían hecho a
Saúl, \footnote{\textbf{31:11} 1Sam 11,1-11} \bibleverse{12} todos los
hombres valientes se levantaron, fueron toda la noche y tomaron el
cuerpo de Saúl y los cuerpos de sus hijos de la muralla de Bet Shan;
llegaron a Jabes y los quemaron allí. \bibleverse{13} Tomaron sus huesos
y los enterraron bajo el árbol de tamarisco en Jabes, y ayunaron siete
días. \footnote{\textbf{31:13} 2Sam 1,12}
