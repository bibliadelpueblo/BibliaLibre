\hypertarget{nehemuxedas-como-copero-del-rey-artajerjes-en-susa-su-dolor-por-la-desgracia-de-su-pauxeds}{%
\subsection{Nehemías como copero del rey Artajerjes en Susa; su dolor
por la desgracia de su
país}\label{nehemuxedas-como-copero-del-rey-artajerjes-en-susa-su-dolor-por-la-desgracia-de-su-pauxeds}}

\hypertarget{section}{%
\section{1}\label{section}}

\bibleverse{1} Las palabras de Nehemías, hijo de Hacalías. En el mes de
Chislev, en el año veinte, estando yo en el palacio de Susa,
\bibleverse{2} vinieron Hanani, uno de mis hermanos, él y algunos
hombres de Judá, y les pregunté sobre los judíos que habían escapado,
que habían quedado del cautiverio, y sobre Jerusalén. \bibleverse{3}
Ellos me dijeron: ``El remanente que queda del cautiverio allí en la
provincia está en gran aflicción y reproche. También el muro de
Jerusalén está derrumbado, y sus puertas están quemadas por el fuego''.
\footnote{\textbf{1:3} 2Cró 36,19}

\bibleverse{4} Cuando oí estas palabras, me senté y lloré, y me lamenté
durante varios días y ayuné y oré ante el Dios del cielo, \footnote{\textbf{1:4}
  Neh 9,1; Esd 9,3}

\hypertarget{la-penitencia-y-la-suxfaplica-de-nehemuxedas}{%
\subsection{La penitencia y la súplica de
Nehemías}\label{la-penitencia-y-la-suxfaplica-de-nehemuxedas}}

\bibleverse{5} y dije: ``Te ruego, Yahvé, el Dios del cielo, el Dios
grande y temible que guarda el pacto y la bondad amorosa con los que lo
aman y guardan sus mandamientos, \footnote{\textbf{1:5} Neh 4,14; Dan
  9,4} \bibleverse{6} que tu oído esté ahora atento y tus ojos abiertos,
para que escuches la oración de tu siervo que hago ante ti en este
momento, de día y de noche, por los hijos de Israel tus siervos,
mientras confieso los pecados de los hijos de Israel que hemos cometido
contra ti. Sí, yo y la casa de mi padre hemos pecado. \bibleverse{7}
Hemos actuado muy corruptamente contra ti, y no hemos guardado los
mandamientos, ni los estatutos, ni las ordenanzas, que ordenaste a tu
siervo Moisés.

\bibleverse{8} ``Acuérdate, te lo ruego, de la palabra que ordenaste a
tu siervo Moisés, diciendo: `Si os desviáis, os dispersaré entre los
pueblos; \bibleverse{9} pero si os volvéis a mí, y guardáis mis
mandamientos y los ponéis en práctica, aunque vuestros desterrados estén
en el extremo de los cielos, yo los recogeré de allí y los traeré al
lugar que he elegido, para hacer habitar allí mi nombre'. \footnote{\textbf{1:9}
  Deut 30,4}

\bibleverse{10} ``Ahora bien, estos son tus siervos y tu pueblo, a
quienes has redimido con tu gran poder y con tu mano fuerte.
\bibleverse{11} Señor, te ruego que tu oído esté atento ahora a la
oración de tu siervo, y a la oración de tus siervos que se deleitan en
temer tu nombre; y por favor, prospera a tu siervo hoy, y concédele
misericordia ante este hombre.'' Ahora yo era portador de la copa del
rey.

\hypertarget{nehemuxedas-recibe-permiso-y-autoridad-del-rey-persa-artajerjes-para-restaurar-jerusaluxe9n}{%
\subsection{Nehemías recibe permiso y autoridad del rey persa Artajerjes
para restaurar
Jerusalén}\label{nehemuxedas-recibe-permiso-y-autoridad-del-rey-persa-artajerjes-para-restaurar-jerusaluxe9n}}

\hypertarget{section-1}{%
\section{2}\label{section-1}}

\bibleverse{1} En el mes de Nisán, en el vigésimo año del rey
Artajerjes, cuando el vino estaba delante de él, recogí el vino y se lo
di al rey. Nunca antes había estado triste en su presencia.
\bibleverse{2} El rey me dijo: ``¿Por qué tienes el rostro triste, ya
que no estás enfermo? Esto no es más que tristeza de corazón''. Entonces
tuve mucho miedo. \bibleverse{3} Le dije al rey: ``¡Que el rey viva para
siempre! ¿Por qué no ha de estar triste mi rostro, cuando la ciudad, el
lugar de las tumbas de mis padres, yace desolada, y sus puertas han sido
consumidas por el fuego?''

\bibleverse{4} Entonces el rey me dijo: ``¿Cuál es tu petición?''
Entonces oré al Dios del cielo. \bibleverse{5} Dije al rey: ``Si al rey
le parece bien, y si tu siervo ha hallado gracia ante tus ojos, te pido
que me envíes a Judá, a la ciudad de las tumbas de mis padres, para que
la construya.''

\bibleverse{6} El rey me dijo (la reina también estaba sentada junto a
él): ``¿Cuánto durará tu viaje? ¿Cuándo volverás?'' Así pues, el rey
tuvo a bien enviarme, y yo le fijé un plazo. \bibleverse{7} Además, dije
al rey: ``Si al rey le parece bien, que se me den cartas a los
gobernadores del otro lado del río, para que me dejen pasar hasta que
llegue a Judá; \bibleverse{8} y una carta a Asaf, guardián del bosque
del rey, para que me dé madera para hacer vigas para las puertas de la
ciudadela junto al templo, para el muro de la ciudad y para la casa que
voy a ocupar.'' El rey accedió a mis peticiones, por la buena mano de mi
Dios sobre mí. \bibleverse{9} Entonces llegué a los gobernadores del
otro lado del río y les entregué las cartas del rey. El rey había
enviado conmigo a los capitanes del ejército y a los jinetes.
\bibleverse{10} Cuando Sanbalat, el horonita, y Tobías, el siervo
amonita, se enteraron de esto, se entristecieron mucho, porque un hombre
había venido a buscar el bienestar de los hijos de Israel.

\hypertarget{el-recorrido-nocturno-de-nehemuxedas-por-las-murallas-de-la-ciudad-su-llamado-a-los-camaradas-nacionales-para-restaurar-el-muro}{%
\subsection{El recorrido nocturno de Nehemías por las murallas de la
ciudad; su llamado a los camaradas nacionales para restaurar el
muro}\label{el-recorrido-nocturno-de-nehemuxedas-por-las-murallas-de-la-ciudad-su-llamado-a-los-camaradas-nacionales-para-restaurar-el-muro}}

\bibleverse{11} Llegué, pues, a Jerusalén y estuve allí tres días.
\bibleverse{12} Me levanté de noche, yo y algunos hombres conmigo. No
dije a nadie lo que mi Dios puso en mi corazón para hacer por Jerusalén.
No me acompañaba ningún animal, excepto el que yo montaba.
\bibleverse{13} Salí de noche por la puerta del valle hacia el pozo del
chacal, y luego hacia la puerta del estiércol; e inspeccioné los muros
de Jerusalén, que estaban derrumbados, y sus puertas consumidas por el
fuego. \bibleverse{14} Luego seguí hasta la puerta del manantial y hasta
el estanque del rey, pero no había lugar para que pasara el animal que
estaba debajo de mí. \footnote{\textbf{2:14} Neh 3,15} \bibleverse{15}
Luego subí de noche por el arroyo e inspeccioné la muralla; me volví y
entré por la puerta del valle, y así regresé. \bibleverse{16} Los jefes
no sabían adónde había ido ni lo que había hecho. Todavía no lo había
contado a los judíos, ni a los sacerdotes, ni a los nobles, ni a los
gobernantes, ni a los demás que hacían la obra.

\bibleverse{17} Entonces les dije: ``Vosotros veis la mala situación en
que nos encontramos, cómo Jerusalén yace desolada y sus puertas están
quemadas por el fuego. Vengan, construyamos el muro de Jerusalén, para
que no seamos deshonrados''.

\hypertarget{compromiso-del-jefe-de-comunidad-el-riduxedculo-de-los-tres-oponentes-paganos-rechazados-por-nehemuxedas}{%
\subsection{Compromiso del jefe de comunidad; el ridículo de los tres
oponentes paganos rechazados por
Nehemías}\label{compromiso-del-jefe-de-comunidad-el-riduxedculo-de-los-tres-oponentes-paganos-rechazados-por-nehemuxedas}}

\bibleverse{18} Les hablé de la mano de mi Dios, que era buena conmigo,
y también de las palabras del rey que me había dicho. Dijeron:
``Levantémonos y construyamos''. Así que fortalecieron sus manos para la
buena obra.

\bibleverse{19} Pero cuando lo oyeron Sanbalat el horonita, Tobías el
siervo amonita y Gesem el árabe, se burlaron de nosotros y nos
despreciaron, y dijeron: ``¿Qué es esto que estáis haciendo? ¿Os vais a
rebelar contra el rey?''

\bibleverse{20} Entonces les respondí y les dije: ``El Dios del cielo
nos prosperará. Por eso nosotros, sus siervos, nos levantaremos y
construiremos; pero vosotros no tenéis parte, ni derecho, ni memoria en
Jerusalén.'' \footnote{\textbf{2:20} Efes 2,12}

\hypertarget{construcciuxf3n-pieza-a-pieza-del-muro-lista-de-los-involucrados-en-la-construcciuxf3n-del-muro}{%
\subsection{Construcción pieza a pieza del muro; Lista de los
involucrados en la construcción del
muro}\label{construcciuxf3n-pieza-a-pieza-del-muro-lista-de-los-involucrados-en-la-construcciuxf3n-del-muro}}

\hypertarget{section-2}{%
\section{3}\label{section-2}}

\bibleverse{1} Entonces el sumo sacerdote Eliasib se levantó con sus
hermanos sacerdotes, y construyeron la puerta de las ovejas. La
santificaron y levantaron sus puertas. La santificaron hasta la torre de
Hammeah, hasta la torre de Hananel. \bibleverse{2} Junto a él edificaron
los hombres de Jericó. Junto a ellos edificó Zaccur, hijo de Imri.

\bibleverse{3} Los hijos de Hassenaah construyeron la puerta del
pescado. Colocaron sus vigas, y colocaron sus puertas, sus cerrojos y
sus barras. \bibleverse{4} Junto a ellos, Meremot, hijo de Urías, hijo
de Hakkoz, hizo las reparaciones. Junto a ellos, Mesulam hijo de
Berequías, hijo de Meshezabel, hizo reparaciones. Junto a ellos, Sadoc
hijo de Baana hizo reparaciones. \bibleverse{5} Junto a ellos, los
tecoítas hacían reparaciones; pero sus nobles no ponían el cuello en la
obra del Señor.

\bibleverse{6} Joiada, hijo de Paseah, y Meshullam, hijo de Besodeiah,
repararon la vieja puerta. Colocaron sus vigas y levantaron sus puertas,
sus cerrojos y sus barras. \bibleverse{7} Junto a ellos, Melatiá el
gabaonita y Jadón el meronita, hombres de Gabaón y de Mizpa, repararon
la residencia del gobernador al otro lado del río. \bibleverse{8} Junto
a él, Uziel hijo de Harhaiah, orfebres, hizo reparaciones. Junto a él,
Hananías, uno de los perfumistas, hizo reparaciones, y fortificaron
Jerusalén hasta el ancho muro. \bibleverse{9} Junto a ellos, Refaías
hijo de Hur, jefe de la mitad del distrito de Jerusalén, hizo
reparaciones. \bibleverse{10} Junto a ellos, Jedaías hijo de Harumaf
hizo reparaciones frente a su casa. Junto a él, Hattush hijo de
Hasabneías hizo reparaciones. \bibleverse{11} Malquías hijo de Harim y
Hasub hijo de Pahatmoab repararon otra parte y la torre de los hornos.
\bibleverse{12} Junto a él, Salum hijo de Hallohesh, jefe de la mitad
del distrito de Jerusalén, él y sus hijas hicieron reparaciones.

\bibleverse{13} Hanun y los habitantes de Zanoa repararon la puerta del
valle. La construyeron y colocaron sus puertas, sus cerrojos y sus
barras, y mil codos de la muralla hasta la puerta del estiércol.

\bibleverse{14} Malquías, hijo de Recab, jefe del distrito de Bet
Haccherem, reparó la puerta del estiércol. La construyó y colocó sus
puertas, sus cerrojos y sus barras.

\bibleverse{15} Salún hijo de Colhoze, jefe del distrito de Mizpa,
reparó la puerta del manantial. La edificó, la cubrió y levantó sus
puertas, sus cerrojos y sus rejas; y reparó el muro del estanque de
Sela, junto al jardín del rey, hasta la escalera que baja de la ciudad
de David. \footnote{\textbf{3:15} Juan 9,7} \bibleverse{16} Después de
él, Nehemías hijo de Azbuk, jefe de la mitad del distrito de Bet Zur,
hizo reparaciones en el lugar frente a las tumbas de David, en el
estanque que se hizo y en la casa de los valientes. \bibleverse{17}
Después de él, los levitas-Rehum, hijo de Bani, hicieron reparaciones.
Después de él, Hasabías, jefe de la mitad del distrito de Keila, hizo
reparaciones en su distrito. \bibleverse{18} Después de él, sus
hermanos, Bavvai hijo de Henadad, gobernante de la mitad del distrito de
Keila, hizo reparaciones. \bibleverse{19} Después de él, Ezer hijo de
Jesúa, gobernante de Mizpa, reparó otra porción al otro lado de la
subida a la armería, en el recodo de la muralla. \bibleverse{20} Después
de él, Baruc hijo de Zabbai reparó con empeño otra parte, desde la
vuelta de la muralla hasta la puerta de la casa del sumo sacerdote
Eliasib. \footnote{\textbf{3:20} Neh 3,1} \bibleverse{21} Después de él,
Meremot, hijo de Urías, hijo de Hakkoz, reparó otra parte, desde la
puerta de la casa de Eliasib hasta el final de la casa de Eliashib.
\footnote{\textbf{3:21} Esd 8,33} \bibleverse{22} Después de él, los
sacerdotes, los hombres de los alrededores hicieron las reparaciones.
\bibleverse{23} Después de ellos, Benjamín y Jasub hicieron reparaciones
al otro lado de su casa. Después de ellos, Azarías, hijo de Maasías,
hijo de Ananías, hizo reparaciones junto a su propia casa.
\bibleverse{24} Después de él, Binnui hijo de Henadad reparó otra parte,
desde la casa de Azarías hasta la vuelta del muro y hasta la esquina.
\bibleverse{25} Palal hijo de Uzai reparó frente a la curva del muro y
la torre que sobresale de la casa superior del rey, que está junto al
patio de la guardia. Después de él, Pedaías, hijo de Paros, hizo las
reparaciones. \footnote{\textbf{3:25} Jer 32,2; Jer 33,1}
\bibleverse{26} (Los servidores del templo vivían en Ofel, en el lugar
que está frente a la puerta de las aguas, hacia el oriente, y en la
torre que sobresale). \bibleverse{27} Después de él, los tecoítas
repararon otra parte, frente a la gran torre que sobresale, y hasta el
muro de Ofel.

\bibleverse{28} Encima de la puerta de los caballos, los sacerdotes
hacían reparaciones, cada uno frente a su propia casa. \footnote{\textbf{3:28}
  2Re 11,16} \bibleverse{29} Después de ellos, Sadoc hijo de Immer hacía
las reparaciones frente a su propia casa. Después de él, Semaías hijo de
Secanías, guardián de la puerta oriental, hizo reparaciones.
\bibleverse{30} Después de él, Hananías hijo de Selemías y Hanún, sexto
hijo de Zalaf, repararon otra parte. Después de él, Mesulam hijo de
Berequías hizo reparaciones frente a su habitación. \bibleverse{31}
Después de él, Malquías, uno de los orfebres de la casa de los
servidores del templo, y de los mercaderes, hizo reparaciones frente a
la puerta de Hamifcad y a la subida de la esquina. \bibleverse{32} Entre
la subida de la esquina y la puerta de las ovejas, los orfebres y los
mercaderes hicieron reparaciones.

\hypertarget{continuaciuxf3n-de-la-construcciuxf3n-del-muro-a-pesar-del-riduxedculo-y-la-hostilidad-de-los-oponentes-paganos}{%
\subsection{Continuación de la construcción del muro a pesar del
ridículo y la hostilidad de los oponentes
paganos}\label{continuaciuxf3n-de-la-construcciuxf3n-del-muro-a-pesar-del-riduxedculo-y-la-hostilidad-de-los-oponentes-paganos}}

\hypertarget{section-3}{%
\section{4}\label{section-3}}

\bibleverse{1} Pero cuando Sanbalat se enteró de que estábamos
construyendo el muro, se enojó y se indignó mucho, y se burló de los
judíos. \footnote{\textbf{4:1} Neh 2,19} \bibleverse{2} Habló ante sus
hermanos y el ejército de Samaria y dijo: ``¿Qué hacen estos débiles
judíos? ¿Se van a fortificar? ¿Sacrificarán? ¿Acabarán en un día?
¿Revivirán las piedras de los montones de basura, ya que están
quemadas?''

\bibleverse{3} Junto a él estaba Tobías, el amonita, quien dijo: ``Lo
que están construyendo, si una zorra trepara por él, derribaría su muro
de piedra.''

\bibleverse{4} ``Escucha, Dios nuestro, porque somos despreciados.
Vuelve su reproche sobre su propia cabeza. Entrégalos como botín en
tierra de cautiverio. \footnote{\textbf{4:4} Sal 7,16} \bibleverse{5} No
cubras su iniquidad. No permitas que su pecado sea borrado ante ti;
porque han insultado a los constructores.''

\bibleverse{6} Construimos, pues, el muro, y toda la muralla se unió
hasta la mitad de su altura, porque el pueblo tenía ganas de trabajar.

\hypertarget{nuevos-ataques-de-los-oponentes-al-edificio-las-medidas-exitosas-de-nehemia-en-su-contra}{%
\subsection{Nuevos ataques de los oponentes al edificio; Las medidas
exitosas de Nehemia en su
contra}\label{nuevos-ataques-de-los-oponentes-al-edificio-las-medidas-exitosas-de-nehemia-en-su-contra}}

\bibleverse{7} Pero cuando Sanbalat, Tobías, los árabes, los amonitas y
los asdoditas se enteraron de que se había avanzado en la reparación de
los muros de Jerusalén y de que se empezaban a rellenar las brechas, se
enfurecieron mucho; \bibleverse{8} y todos ellos se conjuraron para
venir a pelear contra Jerusalén y causar confusión entre nosotros.
\bibleverse{9} Pero nosotros hicimos nuestra oración a nuestro Dios, y
pusimos guardia contra ellos día y noche a causa de ellos.

\bibleverse{10} Judá dijo: ``La fuerza de los portadores de cargas se
desvanece y hay muchos escombros, de modo que no podemos construir el
muro''. \bibleverse{11} Nuestros adversarios dijeron: ``No lo sabrán ni
lo verán, hasta que entremos en medio de ellos y los matemos, y hagamos
cesar la obra.''

\bibleverse{12} Cuando llegaron los judíos que vivían junto a ellos, nos
dijeron diez veces desde todos los lugares: ``Dondequiera que os
volváis, nos atacarán''.

\bibleverse{13} Por eso puse guardias en las partes más bajas del
espacio detrás de la muralla, en los lugares abiertos. Puse al pueblo
por grupos familiares con sus espadas, sus lanzas y sus arcos.
\bibleverse{14} Miré, me levanté y dije a los nobles, a los gobernantes
y al resto del pueblo: ``¡No tengan miedo de ellos! Acuérdense del
Señor, que es grande y temible, y luchen por sus hermanos, sus hijos,
sus hijas, sus esposas y sus casas''. \footnote{\textbf{4:14} Neh 1,5}

\bibleverse{15} Cuando nuestros enemigos se enteraron de que habíamos
sido informados, y de que Dios había hecho fracasar su consejo, todos
nosotros volvimos al muro, cada uno a su trabajo. \footnote{\textbf{4:15}
  Job 5,12} \bibleverse{16} Desde entonces, la mitad de mis siervos
hacía el trabajo, y la otra mitad tenía las lanzas, los escudos, los
arcos y las cotas de malla; y los jefes estaban detrás de toda la casa
de Judá. \bibleverse{17} Los que construían el muro y los que llevaban
cargas se cargaban; cada uno con una de sus manos hacía el trabajo y con
la otra sostenía su arma. \bibleverse{18} Entre los constructores, todos
llevaban su espada al costado, y así construían. El que tocaba la
trompeta estaba junto a mí. \bibleverse{19} Dije a los nobles, a los
jefes y al resto del pueblo: ``La obra es grande y está muy extendida, y
nosotros estamos separados en el muro, lejos unos de otros.
\bibleverse{20} Dondequiera que oigáis el sonido de la trompeta, reuníos
allí con nosotros. Nuestro Dios luchará por nosotros''.

\bibleverse{21} Así hicimos el trabajo. La mitad del pueblo sostuvo las
lanzas desde el amanecer hasta que aparecieron las estrellas.
\bibleverse{22} Asimismo, al mismo tiempo dije al pueblo: ``Que cada uno
con su siervo se aloje dentro de Jerusalén, para que durante la noche
nos sirvan de guardia y para que trabajen durante el día.''
\bibleverse{23} Así que ni yo, ni mis hermanos, ni mis siervos, ni los
hombres de la guardia que me seguían nos quitamos la ropa. Cada uno
llevó su arma al agua.

\hypertarget{alivio-de-la-difuxedcil-situaciuxf3n-de-la-gente-comuxfan-mediante-el-alivio-de-la-deuda-el-gobierno-desinteresado-de-nehemuxedas}{%
\subsection{Alivio de la difícil situación de la gente común mediante el
alivio de la deuda; El gobierno desinteresado de
Nehemías}\label{alivio-de-la-difuxedcil-situaciuxf3n-de-la-gente-comuxfan-mediante-el-alivio-de-la-deuda-el-gobierno-desinteresado-de-nehemuxedas}}

\hypertarget{section-4}{%
\section{5}\label{section-4}}

\bibleverse{1} Entonces se levantó un gran clamor del pueblo y de sus
mujeres contra sus hermanos los judíos. \bibleverse{2} Porque había
algunos que decían: ``Nosotros, nuestros hijos y nuestras hijas, somos
muchos. Consigamos grano, para comer y vivir''. \bibleverse{3} También
había algunos que decían: ``Estamos hipotecando nuestros campos,
nuestras viñas y nuestras casas. Consigamos grano, a causa del hambre''.
\bibleverse{4} Hubo también algunos que dijeron: ``Hemos pedido dinero
prestado para el tributo del rey usando nuestros campos y nuestras viñas
como garantía. \bibleverse{5} Pero ahora nuestra carne es como la carne
de nuestros hermanos, nuestros hijos como sus hijos. He aquí, traemos a
nuestros hijos y a nuestras hijas a la esclavitud para ser siervos, y
algunas de nuestras hijas han sido traídas a la esclavitud. Tampoco está
en nuestro poder evitarlo, porque otros hombres tienen nuestros campos y
nuestras viñas.''

\hypertarget{eliminaciuxf3n-de-los-males-mediante-las-resoluciones-de-la-asamblea-popular}{%
\subsection{Eliminación de los males mediante las resoluciones de la
asamblea
popular}\label{eliminaciuxf3n-de-los-males-mediante-las-resoluciones-de-la-asamblea-popular}}

\bibleverse{6} Me enojé mucho al oír su clamor y estas palabras.
\bibleverse{7} Entonces consulté conmigo mismo y discutí con los nobles
y los gobernantes, y les dije: ``Ustedes exigen usura, cada uno a su
hermano''. Celebré una gran asamblea contra ellos. \footnote{\textbf{5:7}
  Éxod 22,25} \bibleverse{8} Les dije: ``Nosotros, según nuestra
capacidad, hemos redimido a nuestros hermanos los judíos que fueron
vendidos a las naciones; ¿y vosotros queréis incluso vender a vuestros
hermanos, y que se nos vendan a nosotros?'' Entonces callaron, y no
hallaron palabra que decir. \bibleverse{9} También dije: ``Lo que hacéis
no es bueno. ¿No debéis andar en el temor de nuestro Dios a causa del
oprobio de las naciones, nuestros enemigos? \bibleverse{10} Asimismo,
mis hermanos y mis siervos les prestan dinero y grano. Por favor,
detengamos esta usura. \bibleverse{11} Por favor, devuélveles hoy mismo
sus campos, sus viñedos, sus olivares y sus casas, también la centésima
parte del dinero y del grano, del vino nuevo y del aceite que les estás
cobrando.''

\bibleverse{12} Entonces dijeron: ``Los restauraremos y no les
exigiremos nada. Lo haremos así, como tú dices''. Entonces llamé a los
sacerdotes y les tomé juramento de que cumplirían esta promesa.
\bibleverse{13} También sacudí mi regazo, y dije: ``Así sacuda Dios a
todo hombre de su casa y de su trabajo que no cumpla esta promesa;
incluso que sea sacudido y vaciado así.'' Toda la asamblea dijo:
``Amén'', y alabó a Yahvé. El pueblo cumplió esta promesa.

\hypertarget{el-altruismo-de-nehemuxedas-mientras-estaba-en-el-cargo}{%
\subsection{El altruismo de Nehemías mientras estaba en el
cargo}\label{el-altruismo-de-nehemuxedas-mientras-estaba-en-el-cargo}}

\bibleverse{14} Además, desde que fui designado para ser su gobernador
en la tierra de Judá, desde el año veinte hasta el año treinta y dos del
rey Artajerjes, es decir, doce años, ni yo ni mis hermanos hemos comido
el pan del gobernador. \bibleverse{15} Pero los anteriores gobernadores
que me precedieron eran mantenidos por el pueblo, y tomaban de ellos pan
y vino, más cuarenta siclos de plata; sí, incluso sus siervos gobernaban
al pueblo, pero yo no lo hice, por temor a Dios. \bibleverse{16} Sí, yo
también continué en la obra de este muro. No compramos ninguna tierra.
Todos mis siervos se reunieron allí para la obra. \bibleverse{17}
Además, había en mi mesa, de los judíos y de los gobernantes, ciento
cincuenta hombres, además de los que vinieron a nosotros de entre las
naciones que estaban alrededor. \bibleverse{18} Lo que se preparó para
un día fue un buey y seis ovejas selectas. También se me prepararon aves
de corral, y una vez cada diez días una reserva de toda clase de vino.
Pero por todo esto no exigí la paga del gobernador, porque la esclavitud
era pesada para este pueblo. \bibleverse{19} Acuérdate de mí, Dios mío,
por todo el bien que he hecho a este pueblo. \footnote{\textbf{5:19} Neh
  13,14; Neh 13,22; Neh 13,31}

\hypertarget{esquemas-y-asesinatos-de-sanbalat-y-sus-camaradas-su-rechazo-por-parte-de-nehemuxedas}{%
\subsection{Esquemas (y asesinatos) de Sanbalat y sus camaradas; su
rechazo por parte de
Nehemías}\label{esquemas-y-asesinatos-de-sanbalat-y-sus-camaradas-su-rechazo-por-parte-de-nehemuxedas}}

\hypertarget{section-5}{%
\section{6}\label{section-5}}

\bibleverse{1} Cuando se informó a Sanbalat, a Tobías, a Gesem el árabe
y al resto de nuestros enemigos que yo había construido el muro y que no
quedaba ninguna brecha en él (aunque hasta ese momento no había colocado
las puertas en los portones), \bibleverse{2} Sanbalat y Gesem me
enviaron a decir: ``¡Ven! Reunámonos en las aldeas de la llanura de
Ono''. Pero ellos pretendían hacerme daño.

\bibleverse{3} Les envié mensajeros diciendo: ``Estoy haciendo una gran
obra, de modo que no puedo bajar. ¿Por qué ha de cesar la obra mientras
yo la dejo y bajo a vosotros?''

\bibleverse{4} Me enviaron cuatro veces de esta manera, y yo les
respondí de la misma manera. \bibleverse{5} Entonces Sanbalat me envió a
su siervo de la misma manera la quinta vez, con una carta abierta en la
mano, \bibleverse{6} en la que estaba escrito: ``Se ha informado entre
las naciones, y lo dice Gashmu, que tú y los judíos tienen la intención
de rebelarse. Por ello, estáis construyendo el muro. Tú serías su rey,
según estas palabras. \footnote{\textbf{6:6} Esd 4,12} \bibleverse{7}
También has nombrado profetas para que proclamen de ti en Jerusalén,
diciendo: ``¡Hay un rey en Judá! Ahora se informará al rey según estas
palabras. Ven, pues, ahora, y tomemos consejo juntos''.

\bibleverse{8} Entonces le envié a decir: ``No se hacen tales cosas como
tú dices, sino que las imaginas de tu propio corazón''. \bibleverse{9}
Porque todos nos habrían hecho temer, diciendo: ``Sus manos se
debilitarán por la obra, para que no se haga''. Pero ahora, fortalece
mis manos.

\hypertarget{exponiendo-a-un-falso-profeta}{%
\subsection{Exponiendo a un falso
profeta}\label{exponiendo-a-un-falso-profeta}}

\bibleverse{10} Fui a casa de Semaías, hijo de Delaías, hijo de
Mehetabel, que estaba encerrado en su casa, y me dijo: ``Reunámonos en
la casa de Dios, dentro del templo, y cerremos las puertas del templo,
porque vendrán a matarte. Sí, en la noche vendrán a matarte''.

\bibleverse{11} Dije: ``¿Debe huir un hombre como yo? ¿Quién hay que,
siendo como yo, quiera entrar en el templo para salvar su vida? No
entraré''. \bibleverse{12} Discerní, y he aquí que Dios no lo había
enviado, sino que él pronunció esta profecía contra mí. Tobías y
Sanbalat lo habían contratado. \bibleverse{13} Lo contrataron para que
yo tuviera miedo, lo hiciera y pecara, y para que ellos tuvieran
material para un informe malo, para que me reprocharan. \footnote{\textbf{6:13}
  Núm 18,7} \bibleverse{14} ``Acuérdate, Dios mío, de Tobías y de
Sanbalat según estas sus obras, y también de la profetisa Noadías y del
resto de los profetas que me habrían hecho temer''. \footnote{\textbf{6:14}
  Neh 4,4-5}

\hypertarget{finalizaciuxf3n-de-la-construcciuxf3n-del-muro-correspondencia-sospechosa-entre-tobija-y-muchos-juduxedos-dedicados-a-uxe9l}{%
\subsection{Finalización de la construcción del muro; correspondencia
sospechosa entre Tobija y muchos judíos dedicados a
él}\label{finalizaciuxf3n-de-la-construcciuxf3n-del-muro-correspondencia-sospechosa-entre-tobija-y-muchos-juduxedos-dedicados-a-uxe9l}}

\bibleverse{15} Así que el muro fue terminado el día veinticinco de
Elul, en cincuenta y dos días. \bibleverse{16} Cuando todos nuestros
enemigos se enteraron de ello, todas las naciones que nos rodeaban
tuvieron miedo y perdieron su confianza, porque se dieron cuenta de que
esta obra era hecha por nuestro Dios. \bibleverse{17} Además, en
aquellos días los nobles de Judá enviaron muchas cartas a Tobías, y las
cartas de Tobías llegaron a ellos. \bibleverse{18} Porque había muchos
en Judá que le habían jurado por ser yerno de Secanías, hijo de Ara; y
su hijo Johanán había tomado por mujer a la hija de Mesulam, hijo de
Berequías. \bibleverse{19} También hablaron de sus buenas acciones
delante de mí, y le informaron de mis palabras. Tobías envió cartas para
atemorizarme.

\hypertarget{la-preocupaciuxf3n-de-nehemuxedas-por-la-seguridad-de-la-ciudad}{%
\subsection{La preocupación de Nehemías por la seguridad de la
ciudad}\label{la-preocupaciuxf3n-de-nehemuxedas-por-la-seguridad-de-la-ciudad}}

\hypertarget{section-6}{%
\section{7}\label{section-6}}

\bibleverse{1} Cuando la muralla estaba construida y yo había levantado
las puertas, y los guardias de las puertas y los cantores y los levitas
estaban designados, \bibleverse{2} puse a mi hermano Hanani, y a
Hananías, el gobernador de la fortaleza, al frente de Jerusalén, porque
era un hombre fiel y temía a Dios por encima de muchos. \bibleverse{3}
Les dije: ``No dejen que se abran las puertas de Jerusalén hasta que el
sol esté caliente; y mientras hacen guardia, que cierren las puertas y
ustedes las atranquen; y designen guardias de los habitantes de
Jerusalén, cada uno en su guardia, con cada uno cerca de su casa.''

\hypertarget{la-preocupaciuxf3n-de-nehemuxedas-por-aumentar-la-poblaciuxf3n-de-jerusaluxe9n-lista-de-los-israelitas-que-anteriormente-regresaron-del-cautiverio-con-zorobabel}{%
\subsection{La preocupación de Nehemías por aumentar la población de
Jerusalén; Lista de los israelitas que anteriormente regresaron del
cautiverio con
Zorobabel}\label{la-preocupaciuxf3n-de-nehemuxedas-por-aumentar-la-poblaciuxf3n-de-jerusaluxe9n-lista-de-los-israelitas-que-anteriormente-regresaron-del-cautiverio-con-zorobabel}}

\bibleverse{4} La ciudad era amplia y grande, pero la gente era poca y
las casas no estaban construidas.

\bibleverse{5} Mi Dios puso en mi corazón reunir a los nobles, a los
gobernantes y al pueblo, para que fueran listados por genealogía.
Encontré el libro de la genealogía de los que subieron al principio, y
encontré esto escrito en él:

\bibleverse{6} Estos son los hijos de la provincia que subieron del
cautiverio de los deportados, que Nabucodonosor, rey de Babilonia, había
llevado, y que volvieron a Jerusalén y a Judá, cada uno a su ciudad,
\footnote{\textbf{7:6} Esd 2,1} \bibleverse{7} que vinieron con
Zorobabel, Jesúa, Nehemías, Azarías, Raamías, Nahamani, Mardoqueo,
Bilsán, Misperet, Bigvai, Nehum y Baana. El número de los hombres del
pueblo de Israel: \bibleverse{8} Los hijos de Paros: dos mil ciento
setenta y dos. \bibleverse{9} Los hijos de Sefatías: trescientos setenta
y dos. \bibleverse{10} Los hijos de Arah: seiscientos cincuenta y dos.
\bibleverse{11} Los hijos de Pahatmoab, de los hijos de Jesúa y de Joab:
dos mil ochocientos dieciocho. \bibleverse{12} Los hijos de Elam: mil
doscientos cincuenta y cuatro. \bibleverse{13} Los hijos de Zattu:
ochocientos cuarenta y cinco. \bibleverse{14} Los hijos de Zaccai:
setecientos sesenta. \bibleverse{15} Los hijos de Binnui: seiscientos
cuarenta y ocho. \bibleverse{16} Los hijos de Bebai: seiscientos
veintiocho. \bibleverse{17} Los hijos de Azgad: dos mil trescientos
veintidós. \bibleverse{18} Los hijos de Adonikam: seiscientos sesenta y
siete. \bibleverse{19} Los hijos de Bigvai: dos mil sesenta y siete.
\bibleverse{20} Los hijos de Adin: seiscientos cincuenta y cinco.
\bibleverse{21} Los hijos de Ater: de Ezequías, noventa y ocho.
\bibleverse{22} Los hijos de Hashum: trescientos veintiocho.
\bibleverse{23} Los hijos de Bezai: trescientos veinticuatro.
\bibleverse{24} Los hijos de Hariph: ciento doce. \bibleverse{25} Los
hijos de Gabaón: noventa y cinco. \bibleverse{26} Los hombres de Belén y
Netofa: ciento ochenta y ocho. \bibleverse{27} Los hombres de Anatot:
ciento veintiocho. \bibleverse{28} Los hombres de Bet Azmavet: cuarenta
y dos. \bibleverse{29} Los hombres de Kiriath Jearim, Chephirah y
Beeroth: setecientos cuarenta y tres. \bibleverse{30} Los hombres de
Rama y Geba: seiscientos veintiuno. \bibleverse{31} Los hombres de
Micmas: ciento veintidós. \bibleverse{32} Los hombres de Betel y Hai:
ciento veintitrés. \bibleverse{33} Los hombres del otro Nebo: cincuenta
y dos. \bibleverse{34} Los hijos del otro Elam: mil doscientos cincuenta
y cuatro. \bibleverse{35} Los hijos de Harim: trescientos veinte.
\bibleverse{36} Los hijos de Jericó: trescientos cuarenta y cinco.
\bibleverse{37} Los hijos de Lod, Hadid y Ono: setecientos veintiuno.
\bibleverse{38} Los hijos de Senaah: tres mil novecientos treinta.
\bibleverse{39} Los sacerdotes: Los hijos de Jedaías, de la casa de
Jesúa: novecientos setenta y tres. \bibleverse{40} Los hijos de Immer:
mil cincuenta y dos. \bibleverse{41} Los hijos de Pashur: mil doscientos
cuarenta y siete. \bibleverse{42} Los hijos de Harim: mil diecisiete.
\bibleverse{43} Los levitas: los hijos de Jesúa, de Cadmiel, de los
hijos de Hodevah: setenta y cuatro. \bibleverse{44} Los cantores: los
hijos de Asaf: ciento cuarenta y ocho. \bibleverse{45} Los porteros: los
hijos de Salum, los hijos de Ater, los hijos de Talmón, los hijos de
Acub, los hijos de Hatita, los hijos de Sobai: ciento treinta y ocho.

\bibleverse{46} Los servidores del templo: los hijos de Ziha, los hijos
de Hasupha, los hijos de Tabbaoth, \bibleverse{47} los hijos de Keros,
los hijos de Sia, los hijos de Padon, \bibleverse{48} los hijos de
Lebana, los hijos de Hagaba, los hijos de Salmai, \bibleverse{49} los
hijos de Hanan, los hijos de Giddel, los hijos de Gahar, \bibleverse{50}
los hijos de Reaiah, los hijos de Rezin, los hijos de Nekoda,
\bibleverse{51} los hijos de Gazzam, los hijos de Uzza, los hijos de
Paseah, \bibleverse{52} los hijos de Besai, los hijos de Meunim, los
hijos de Nephushesim, \bibleverse{53} los hijos de Bakbuk, los hijos de
Hakupha, los hijos de Harhur, \bibleverse{54} los hijos de Bazlith, los
hijos de Mehida, los hijos de Harsha, \bibleverse{55} los hijos de
Barkos, los hijos de Sisera, los hijos de Temah, \bibleverse{56} los
hijos de Neziah, y los hijos de Hatipha.

\bibleverse{57} Los hijos de los siervos de Salomón: los hijos de Sotai,
los hijos de Soferet, los hijos de Perida, \bibleverse{58} los hijos de
Jaala, los hijos de Darkon, los hijos de Giddel, \bibleverse{59} los
hijos de Sefatías, los hijos de Hattil, los hijos de Poqueret Hazzebaim
y los hijos de Amón. \bibleverse{60} Todos los siervos del templo y los
hijos de los siervos de Salomón eran trescientos noventa y dos.

\bibleverse{61} Estos fueron los que subieron de Tel Melah, Tel Harsha,
Querubín, Addón e Immer; pero no pudieron mostrar las casas de sus
padres, ni su descendencia, si eran de Israel: \bibleverse{62} Los hijos
de Delaía, los hijos de Tobías y los hijos de Necoda: seiscientos
cuarenta y dos. \bibleverse{63} De los sacerdotes: los hijos de Hobaiah,
los hijos de Hakkoz, los hijos de Barzillai, el cual tomó mujer de las
hijas de Barzillai Galaadita, y fue llamado según su nombre.

\bibleverse{64} Estos buscaron sus registros genealógicos, pero no
pudieron encontrarlos. Por lo tanto, fueron considerados descalificados
y apartados del sacerdocio. \bibleverse{65} El gobernador les dijo que
no comieran de las cosas más sagradas hasta que un sacerdote se
levantara para ministrar con el Urim y el Tumim.

\bibleverse{66} Toda la asamblea reunida era de cuarenta y dos mil
trescientos sesenta, \bibleverse{67} además de sus siervos y siervas,
que eran siete mil trescientos treinta y siete. Tenían doscientos
cuarenta y cinco hombres cantores y mujeres cantoras. \bibleverse{68}
Sus caballos eran setecientos treinta y seis; sus mulos, doscientos
cuarenta y cinco; \bibleverse{69} sus camellos, cuatrocientos treinta y
cinco; sus asnos, seis mil setecientos veinte.

\bibleverse{70} Algunos de entre los jefes de familia dieron para la
obra. El gobernador dio para el tesoro mil dáricos de oro, cincuenta
cuencas, y quinientos treinta vestidos sacerdotales. \footnote{\textbf{7:70}
  Neh 7,65} \bibleverse{71} Algunos de los jefes de familia dieron para
el tesoro de la obra veinte mil dáricos de oro, y dos mil doscientas
minas de plata. \bibleverse{72} Lo que dio el resto del pueblo fue
veinte mil dáricos de oro, más dos mil minas de plata, y sesenta y siete
vestiduras sacerdotales.

\bibleverse{73} Así, los sacerdotes, los levitas, los porteros, los
cantores, parte del pueblo, los servidores del templo y todo Israel
vivían en sus ciudades. Cuando llegó el séptimo mes, los hijos de Israel
estaban en sus ciudades.

\hypertarget{lectura-de-la-ley-por-esdras-y-celebraciuxf3n-de-la-fiesta-de-los-tabernuxe1culos}{%
\subsection{Lectura de la ley por Esdras y celebración de la Fiesta de
los
Tabernáculos}\label{lectura-de-la-ley-por-esdras-y-celebraciuxf3n-de-la-fiesta-de-los-tabernuxe1culos}}

\hypertarget{section-7}{%
\section{8}\label{section-7}}

\bibleverse{1} Todo el pueblo se reunió como un solo hombre en el lugar
amplio que estaba frente a la puerta de las aguas, y hablaron a Esdras
el escriba para que trajera el libro de la ley de Moisés, que Yahvé
había ordenado a Israel. \footnote{\textbf{8:1} Esd 7,6} \bibleverse{2}
El sacerdote Esdras trajo la ley ante la asamblea, tanto de hombres como
de mujeres, y de todos los que podían oír con entendimiento, el primer
día del mes séptimo. \footnote{\textbf{8:2} Deut 31,10-13}
\bibleverse{3} Leyó de ella ante el lugar amplio que estaba frente a la
puerta de las aguas, desde la mañana hasta el mediodía, en presencia de
los hombres y de las mujeres, y de los que podían entender. Los oídos de
todo el pueblo estaban atentos al libro de la ley. \bibleverse{4}
Esdras, el escriba, estaba de pie sobre un púlpito de madera que habían
hecho al efecto; y junto a él estaban Matatías, Sema, Anáías, Urías,
Hilcías y Maasías, a su derecha; y a su izquierda, Pedaías, Misael,
Malquías, Hasum, Hasbaddana, Zacarías y Mesulam. \bibleverse{5} Esdras
abrió el libro a la vista de todo el pueblo (porque él estaba por encima
de todo el pueblo), y cuando lo abrió, todo el pueblo se puso de pie.
\bibleverse{6} Entonces Esdras bendijo a Yavé, el gran Dios. Todo el
pueblo respondió: ``Amén, Amén'', levantando las manos. Inclinaron la
cabeza y adoraron a Yavé con el rostro en tierra. \bibleverse{7} También
Jesúa, Baní, Serebías, Jamín, Acub, Sabetai, Hodías, Maasías, Kelita,
Azarías, Jozabad, Hanán, Pelaías y los levitas, hicieron que el pueblo
entendiera la ley; y el pueblo permaneció en su lugar. \bibleverse{8}
Leían en el libro, en la ley de Dios, claramente; y daban el sentido, de
modo que entendían la lectura.

\hypertarget{la-invitaciuxf3n-de-nehemuxedas-y-esdras-a-las-personas-afligidas-a-celebrar-el-duxeda-con-alegruxeda-festiva}{%
\subsection{La invitación de Nehemías y Esdras a las personas afligidas
a celebrar el día con alegría
festiva}\label{la-invitaciuxf3n-de-nehemuxedas-y-esdras-a-las-personas-afligidas-a-celebrar-el-duxeda-con-alegruxeda-festiva}}

\bibleverse{9} Nehemías, que era el gobernador, el sacerdote y escriba
Esdras, y los levitas que enseñaban al pueblo, dijeron a todo el pueblo:
``Hoy es un día santo para Yahvé, vuestro Dios. No os lamentéis ni
lloréis''. Porque todo el pueblo lloró al oír las palabras de la ley.
\footnote{\textbf{8:9} Neh 5,14} \bibleverse{10} Luego les dijo:
``Vayan. Coman la grasa, beban lo dulce y envíen porciones a quien no
tiene nada preparado, porque hoy es santo para nuestro Señor. No os
entristezcáis, porque la alegría de Yahvé es vuestra fuerza''.

\bibleverse{11} Entonces los levitas calmaron a todo el pueblo,
diciendo: ``Callad, porque el día es sagrado. No se aflijan''.

\bibleverse{12} Todo el pueblo siguió su camino para comer, beber,
enviar porciones y celebrar, porque habían entendido las palabras que se
les habían declarado.

\hypertarget{celebraciuxf3n-de-la-fiesta-de-los-tabernuxe1culos-con-lectura-constante-de-la-ley}{%
\subsection{Celebración de la Fiesta de los Tabernáculos con lectura
constante de la
ley}\label{celebraciuxf3n-de-la-fiesta-de-los-tabernuxe1culos-con-lectura-constante-de-la-ley}}

\bibleverse{13} Al segundo día, los jefes de familia de todo el pueblo,
los sacerdotes y los levitas se reunieron con Esdras, el escriba, para
estudiar las palabras de la ley. \bibleverse{14} Encontraron escrito en
la ley cómo Yahvé había ordenado por medio de Moisés que los hijos de
Israel debían habitar en cabañas en la fiesta del séptimo mes;
\footnote{\textbf{8:14} Lev 23,42} \bibleverse{15} y que debían publicar
y proclamar en todas sus ciudades y en Jerusalén, diciendo: ``Salgan al
monte y traigan ramas de olivo, ramas de olivo silvestre, ramas de
mirto, ramas de palmeras y ramas de árboles frondosos, para hacer
refugios temporales, como está escrito.''

\bibleverse{16} Así que el pueblo salió y los trajo, y se hicieron
refugios temporales, cada uno en el techo de su casa, en sus patios, en
los patios de la casa de Dios, en el lugar ancho de la puerta de las
aguas y en el lugar ancho de la puerta de Efraín. \footnote{\textbf{8:16}
  Neh 8,1} \bibleverse{17} Toda la asamblea de los que habían regresado
del cautiverio hizo refugios temporales y vivió en los refugios
temporales, pues desde los días de Josué hijo de Nun hasta ese día los
hijos de Israel no lo habían hecho. Hubo una alegría muy grande.
\bibleverse{18} Además, cada día, desde el primero hasta el último, se
leía en el libro de la ley de Dios. Celebraron la fiesta durante siete
días; y el octavo día hubo una asamblea solemne, según la ordenanza.

\hypertarget{celebraciuxf3n-del-duxeda-de-la-penitencia-con-varias-horas-de-lectura-de-la-ley-y-varias-horas-de-confesiuxf3n}{%
\subsection{Celebración del día de la penitencia con varias horas de
lectura de la ley y varias horas de
confesión}\label{celebraciuxf3n-del-duxeda-de-la-penitencia-con-varias-horas-de-lectura-de-la-ley-y-varias-horas-de-confesiuxf3n}}

\hypertarget{section-8}{%
\section{9}\label{section-8}}

\bibleverse{1} El día veinticuatro de este mes se reunieron los hijos de
Israel con ayuno, con cilicio y con tierra sobre ellos. \bibleverse{2}
Los descendientes de Israel se separaron de todos los extranjeros y se
pusieron de pie y confesaron sus pecados y las iniquidades de sus
padres. \bibleverse{3} Se pusieron de pie en su lugar y leyeron en el
libro de la ley de Yavé su Dios la cuarta parte del día; y la cuarta
parte la confesaron y adoraron a Yavé su Dios. \bibleverse{4} Entonces
Jesúa, Baní, Cadmiel, Sebanías, Buní, Serebías, Baní y Quenaní, de los
levitas, se pusieron de pie en la escalera y clamaron en voz alta a Yavé
su Dios.

\hypertarget{invitaciuxf3n-a-alabar-a-dios-referencia-a-los-maravillosos-actos-de-poder-y-gracia-de-dios-en-tiempos-prehistuxf3ricos-hasta-la-introducciuxf3n-de-su-pueblo-en-la-tierra-prometida}{%
\subsection{Invitación a alabar a Dios; Referencia a los maravillosos
actos de poder y gracia de Dios en tiempos prehistóricos hasta la
introducción de su pueblo en la tierra
prometida}\label{invitaciuxf3n-a-alabar-a-dios-referencia-a-los-maravillosos-actos-de-poder-y-gracia-de-dios-en-tiempos-prehistuxf3ricos-hasta-la-introducciuxf3n-de-su-pueblo-en-la-tierra-prometida}}

\bibleverse{5} Entonces los levitas, Jesúa y Cadmiel, Baní, Hasabneías,
Serebías, Hodías, Sebanías y Petaías, dijeron: ``¡Ponte de pie y bendice
a Yahvé, tu Dios, desde la eternidad hasta la eternidad! ¡Bendito sea tu
nombre glorioso, que es exaltado sobre toda bendición y alabanza!
\bibleverse{6} Tú eres Yahvé, tú solo. Tú has hecho el cielo, el cielo
de los cielos, con todo su ejército, la tierra y todo lo que hay en
ella, los mares y todo lo que hay en ellos, y tú lo conservas todo. El
ejército de los cielos te adora. \bibleverse{7} Tú eres Yahvé, el Dios
que eligió a Abram, lo sacó de Ur de los Caldeos, le dio el nombre de
Abraham, \footnote{\textbf{9:7} Gén 11,31; Gén 17,5} \bibleverse{8}
encontró su corazón fiel ante ti, e hizo un pacto con él para darle la
tierra del cananeo, del hitita, del amorreo, del ferezeo, del jebuseo y
del gergeseo, para dársela a su descendencia, y has cumplido tus
palabras, porque eres justo. \footnote{\textbf{9:8} Gén 15,18-21}

\bibleverse{9} ``Viste la aflicción de nuestros padres en Egipto, y
oíste su clamor junto al Mar Rojo, \footnote{\textbf{9:9} Éxod 3,7}
\bibleverse{10} y mostraste señales y prodigios contra el Faraón, contra
todos sus siervos y contra todo el pueblo de su tierra, porque sabías
que se burlaban de ellos, y te hiciste un nombre, como lo es hoy.
\bibleverse{11} Dividiste el mar delante de ellos, de modo que pasaron
por el medio del mar en seco; y arrojaste a sus perseguidores a las
profundidades, como una piedra a las aguas impetuosas. \footnote{\textbf{9:11}
  Éxod 14,21; Éxod 15,5; Éxod 15,10} \bibleverse{12} Además, en una
columna de nube los guiaste de día, y en una columna de fuego de noche,
para alumbrarles el camino que debían seguir. \footnote{\textbf{9:12}
  Éxod 13,21}

\bibleverse{13} ``También bajaste al monte Sinaí, y hablaste con ellos
desde el cielo, y les diste ordenanzas rectas y leyes verdaderas, buenos
estatutos y mandamientos, \footnote{\textbf{9:13} Éxod 19,18; Éxod 20,1}
\bibleverse{14} y les diste a conocer tu santo sábado, y les ordenaste
mandamientos, estatutos y una ley, por medio de Moisés, tu siervo,
\bibleverse{15} y les diste pan del cielo para su hambre, y les sacaste
agua de la roca para su sed, y les ordenaste que entraran a poseer la
tierra que habías jurado darles. \footnote{\textbf{9:15} Éxod 16,4; Éxod
  16,14; Éxod 17,6}

\bibleverse{16} ``Pero ellos y nuestros padres se comportaron con
soberbia, endurecieron su cerviz, no escucharon tus mandamientos,
\footnote{\textbf{9:16} Éxod 32,9} \bibleverse{17} y se negaron a
obedecer. No tuvieron en cuenta tus maravillas que hiciste entre ellos,
sino que endurecieron su cerviz, y en su rebeldía nombraron un capitán
para volver a su esclavitud. Pero tú eres un Dios dispuesto a perdonar,
clemente y misericordioso, lento para la ira y abundante en bondades, y
no los abandonaste. \footnote{\textbf{9:17} Núm 14,4; Éxod 34,6}
\bibleverse{18} Sí, cuando se hicieron un becerro moldeado y dijeron:
``Este es vuestro Dios, que os hizo subir de Egipto'', y cometieron
horribles blasfemias, \footnote{\textbf{9:18} Éxod 32,4} \bibleverse{19}
pero tú, en tus múltiples misericordias, no los abandonaste en el
desierto. La columna de nube no se apartó de ellos durante el día para
guiarlos por el camino, ni la columna de fuego durante la noche para
mostrarles la luz y el camino que debían seguir. \bibleverse{20} También
diste tu buen Espíritu para instruirlos, y no retuviste tu maná de su
boca, y les diste agua para su sed. \footnote{\textbf{9:20} Núm 11,25}

\bibleverse{21} ``Sí, cuarenta años los sostuviste en el desierto. Nada
les faltó. Sus vestidos no envejecieron y sus pies no se hincharon.
\footnote{\textbf{9:21} Deut 8,4} \bibleverse{22} Además, les diste
reinos y pueblos, que asignaste según sus porciones. Así poseyeron la
tierra de Sehón, la tierra del rey de Hesbón y la tierra de Og, rey de
Basán. \footnote{\textbf{9:22} Núm 21,24; Núm 21,35} \bibleverse{23}
También multiplicaste sus hijos como las estrellas del cielo, y los
introdujiste en la tierra sobre la que habías dicho a sus padres que
entrarían a poseerla.

\bibleverse{24} ``Así que los hijos entraron y poseyeron la tierra; y tú
sometiste ante ellos a los habitantes de la tierra, los cananeos, y los
entregaste en sus manos, con sus reyes y los pueblos de la tierra, para
que hicieran con ellos lo que quisieran. \footnote{\textbf{9:24} Jos
  12,1} \bibleverse{25} Tomaron ciudades fortificadas y una tierra rica,
y poseyeron casas llenas de todos los bienes, cisternas excavadas,
viñas, olivares y árboles frutales en abundancia. Así comieron, se
saciaron, engordaron y se deleitaron en tu gran bondad. \footnote{\textbf{9:25}
  Deut 6,10-11; Deut 32,15}

\hypertarget{en-posesiuxf3n-de-la-tierra-el-pueblo-con-desprecio-por-los-profetas-y-la-divina-paciencia-continuxfaa-con-su-conducta-pecaminosa-hasta-que-dios-los-entrega-en-manos-de-los-gentiles}{%
\subsection{En posesión de la tierra, el pueblo, con desprecio por los
profetas y la divina paciencia, continúa con su conducta pecaminosa
hasta que Dios los entrega en manos de los
gentiles}\label{en-posesiuxf3n-de-la-tierra-el-pueblo-con-desprecio-por-los-profetas-y-la-divina-paciencia-continuxfaa-con-su-conducta-pecaminosa-hasta-que-dios-los-entrega-en-manos-de-los-gentiles}}

\bibleverse{26} ``Sin embargo, fueron desobedientes y se rebelaron
contra ti, echaron tu ley a sus espaldas, mataron a tus profetas que
testificaban contra ellos para que se volvieran a ti, y cometieron
horribles blasfemias. \bibleverse{27} Por eso los entregaste en manos de
sus adversarios, que los angustiaron. En el tiempo de su angustia,
cuando clamaron a ti, tú oíste desde el cielo; y según tus múltiples
misericordias les diste salvadores que los salvaron de las manos de sus
adversarios. \footnote{\textbf{9:27} Jue 3,9; Jue 3,15} \bibleverse{28}
Pero después de haber descansado, volvieron a hacer el mal ante ti; por
eso los dejaste en manos de sus enemigos, para que se enseñorearan de
ellos; sin embargo, cuando volvieron y clamaron a ti, tú oíste desde el
cielo; y muchas veces los libraste según tus misericordias, \footnote{\textbf{9:28}
  Jue 2,18-21} \bibleverse{29} y diste testimonio contra ellos, para que
volvieran a tu ley. Sin embargo, fueron arrogantes y no escucharon tus
mandamientos, sino que pecaron contra tus ordenanzas (que si el hombre
hace, vivirá en ellas), volvieron la espalda, endurecieron su cerviz y
no quisieron escuchar. \footnote{\textbf{9:29} Lev 18,5} \bibleverse{30}
Sin embargo, durante muchos años los aguantaste y les diste testimonio
con tu Espíritu por medio de tus profetas. Sin embargo, no quisieron
escuchar. Por eso los entregaste en manos de los pueblos de las tierras.
\footnote{\textbf{9:30} Jer 7,25-26; Jer 44,4-6}

\bibleverse{31} ``Sin embargo, en tus múltiples misericordias, no les
pusiste fin ni los abandonaste, porque eres un Dios clemente y
misericordioso. \footnote{\textbf{9:31} Lam 3,22}

\hypertarget{pida-nueva-gracia-y-lealtad-y-el-alivio-del-bien-merecido-sufrimiento-desde-el-dominio-asirio-hasta-el-presente}{%
\subsection{Pida nueva gracia y lealtad y el alivio del bien merecido
sufrimiento desde el dominio asirio hasta el
presente}\label{pida-nueva-gracia-y-lealtad-y-el-alivio-del-bien-merecido-sufrimiento-desde-el-dominio-asirio-hasta-el-presente}}

\bibleverse{32} Ahora, pues, Dios nuestro, el grande, el poderoso y el
imponente, que guarda el pacto y la bondad amorosa, no dejes que te
parezcan pequeños todos los trabajos que nos han sobrevenido, a nuestros
reyes, a nuestros príncipes, a nuestros sacerdotes, a nuestros profetas,
a nuestros padres y a todo tu pueblo, desde el tiempo de los reyes de
Asiria hasta hoy. \footnote{\textbf{9:32} Neh 1,5} \bibleverse{33} Sin
embargo, tú eres justo en todo lo que ha recaído sobre nosotros; porque
tú has actuado con verdad, pero nosotros hemos hecho maldad. \footnote{\textbf{9:33}
  Esd 9,15; Dan 9,5; Dan 9,7} \bibleverse{34} También nuestros reyes,
nuestros príncipes, nuestros sacerdotes y nuestros padres no han
guardado tu ley, ni han escuchado tus mandamientos y tus testimonios con
los que testificaste contra ellos. \bibleverse{35} Porque no te han
servido en su reino y en tu gran bondad que les diste, y en la tierra
grande y rica que les diste. No se convirtieron de sus obras malvadas.

\bibleverse{36} ``He aquí que hoy somos siervos, y en cuanto a la tierra
que diste a nuestros padres para comer su fruto y su bien, he aquí que
somos siervos en ella. \bibleverse{37} La tierra da muchos frutos a los
reyes que has puesto sobre nosotros a causa de nuestros pecados. También
tienen poder sobre nuestros cuerpos y sobre nuestro ganado, a su antojo,
y estamos en gran aflicción. \bibleverse{38} Sin embargo, por todo esto,
hacemos un pacto seguro y lo escribimos; y nuestros príncipes, nuestros
levitas y nuestros sacerdotes lo sellan.''

\hypertarget{renovaciuxf3n-federal-mediante-contrato-escrito-y-firmado-de-los-jefes-especialmente-jefes-de-familia-del-pueblo}{%
\subsection{Renovación federal mediante contrato escrito y firmado de
los jefes (especialmente jefes de familia) del
pueblo}\label{renovaciuxf3n-federal-mediante-contrato-escrito-y-firmado-de-los-jefes-especialmente-jefes-de-familia-del-pueblo}}

\hypertarget{section-9}{%
\section{10}\label{section-9}}

\bibleverse{1} Los que sellaron fueron: Nehemías, el gobernador, hijo de
Hacalías, y Sedequías, \bibleverse{2} Seraías, Azarías, Jeremías,
\bibleverse{3} Pashur, Amarías, Malquías, \bibleverse{4} Hattush,
Sebanías, Malluch, \bibleverse{5} Harim, Meremot, Abdías, \bibleverse{6}
Daniel, Ginetón, Baruc, \bibleverse{7} Mesulam, Abías, Mijamín,
\bibleverse{8} Maazías, Bilgai y Semaías. Estos eran los sacerdotes.
\bibleverse{9} Los levitas: Jesúa hijo de Azanías, Binúi de los hijos de
Henadad, Cadmiel; \bibleverse{10} y sus hermanos, Sebanías, Hodías,
Kelita, Pelaías, Hanán, \bibleverse{11} Mica, Rehob, Hasabías,
\bibleverse{12} Zaccur, Serebías, Sebanías, \bibleverse{13} Hodías, Baní
y Beninú. \bibleverse{14} Los jefes del pueblo: Parosh, Pahathmoab,
Elam, Zattu, Bani, \footnote{\textbf{10:14} Esd 2,3; Esd 2,6}
\bibleverse{15} Bunni, Azgad, Bebai, \bibleverse{16} Adonías, Bigvai,
Adin, \bibleverse{17} Ater, Ezequías, Azzur, \bibleverse{18} Hodías,
Hashum, Bezai, \bibleverse{19} Hariph, Anathoth, Nobai, \bibleverse{20}
Magpiash, Meshullam, Hezir, \bibleverse{21} Meshezabel, Zadok, Jaddua,
\bibleverse{22} Pelatiah, Hanan, Anaiah, \bibleverse{23} Hoshea,
Hananiah, Hasshub, \bibleverse{24} Hallohesh, Pilha, Shobek,
\bibleverse{25} Rehum, Hashabnah, Maaseiah, \bibleverse{26} Ahiah,
Hanan, Anan, \bibleverse{27} Malluch, Harim y Baanah.

\hypertarget{evitar-los-matrimonios-mixtos-y-no-tomar-el-suxe1bado}{%
\subsection{Evitar los matrimonios mixtos y no tomar el
sábado}\label{evitar-los-matrimonios-mixtos-y-no-tomar-el-suxe1bado}}

\bibleverse{28} El resto del pueblo, los sacerdotes, los levitas, los
porteros, los cantores, los servidores del templo, y todos los que se
habían apartado de los pueblos de las tierras a la ley de Dios, sus
esposas, sus hijos y sus hijas --- todos los que tenían conocimiento y
entendimiento --- \footnote{\textbf{10:28} Esd 2,70} \bibleverse{29} se
unieron a sus hermanos sus nobles, y entraron en una maldición y en un
juramento, para andar en la ley de Dios, que fue dada por Moisés el
siervo de Dios, y para observar y hacer todos los mandamientos de Yahvé
nuestro Señor, y sus ordenanzas y sus estatutos; \bibleverse{30} y que
no daríamos nuestras hijas a los pueblos de la tierra, ni tomaríamos sus
hijas para nuestros hijos; \bibleverse{31} y que si los pueblos de la
tierra trajeran mercancías o algún grano en el día de reposo para
vender, no les compraríamos en el día de reposo, ni en un día santo; y
que renunciaríamos a las cosechas del séptimo año y a la exacción de
toda deuda. \footnote{\textbf{10:31} Neh 13,15-16; Am 8,5}

\hypertarget{pago-oportuno-y-abundante-de-todos-los-deberes-y-obligaciones-relacionados-con-la-adoraciuxf3n-y-el-sacerdocio}{%
\subsection{Pago oportuno y abundante de todos los deberes y
obligaciones relacionados con la adoración y el
sacerdocio}\label{pago-oportuno-y-abundante-de-todos-los-deberes-y-obligaciones-relacionados-con-la-adoraciuxf3n-y-el-sacerdocio}}

\bibleverse{32} Además, nos hemos impuesto la obligación de cobrar
anualmente la tercera parte de un siclo para el servicio de la casa de
nuestro Dios: \bibleverse{33} para el pan de la proposición, para la
ofrenda continua, para el holocausto continuo, para los sábados, para
las lunas nuevas, para las fiestas, para las cosas santas, para las
ofrendas por el pecado para hacer expiación por Israel, y para toda la
obra de la casa de nuestro Dios. \bibleverse{34} Nosotros, los
sacerdotes, los levitas y el pueblo, echamos suertes sobre la ofrenda de
leña, para traerla a la casa de nuestro Dios, según las casas de
nuestros padres, en los tiempos señalados de año en año, para quemarla
en el altar de Yahvé nuestro Dios, como está escrito en la ley;
\footnote{\textbf{10:34} Lev 6,12} \bibleverse{35} y para traer las
primicias de nuestra tierra y las primicias de todos los frutos de toda
clase de árboles, de año en año, a la casa de Yahvé; \footnote{\textbf{10:35}
  Éxod 23,19} \bibleverse{36} también los primogénitos de nuestros hijos
y de nuestros ganados, como está escrito en la ley, y los primogénitos
de nuestras vacas y de nuestros rebaños, para llevarlos a la casa de
nuestro Dios, a los sacerdotes que sirven en la casa de nuestro Dios
\footnote{\textbf{10:36} Éxod 13,2} \bibleverse{37} y que traigamos las
primicias de nuestra masa, nuestras ofrendas de ondas, el fruto de toda
clase de árboles, el vino nuevo y el aceite, a los sacerdotes, a las
salas de la casa de nuestro Dios; y los diezmos de nuestra tierra a los
levitas; porque ellos, los levitas, toman los diezmos en todas nuestras
aldeas agrícolas. \footnote{\textbf{10:37} Núm 18,21} \bibleverse{38} El
sacerdote, descendiente de Aarón, estará con los levitas cuando éstos
tomen los diezmos. Los levitas llevarán el diezmo de los diezmos a la
casa de nuestro Dios, a las habitaciones, a la casa del tesoro.
\footnote{\textbf{10:38} Núm 18,26; Núm 18,28} \bibleverse{39} Porque
los hijos de Israel y los hijos de Leví llevarán la ofrenda mecida del
grano, del vino nuevo y del aceite, a las habitaciones donde están los
utensilios del santuario y los sacerdotes que ministran, con los
porteros y los cantores. No abandonaremos la casa de nuestro Dios.

\hypertarget{una-duxe9cima-parte-de-la-poblaciuxf3n-rural-estuxe1-determinada-por-sorteo-a-mudarse-a-jerusaluxe9n}{%
\subsection{Una décima parte de la población rural está determinada por
sorteo a mudarse a
Jerusalén}\label{una-duxe9cima-parte-de-la-poblaciuxf3n-rural-estuxe1-determinada-por-sorteo-a-mudarse-a-jerusaluxe9n}}

\hypertarget{section-10}{%
\section{11}\label{section-10}}

\bibleverse{1} Los príncipes del pueblo vivían en Jerusalén. El resto
del pueblo también echó suertes para que uno de los diez habitara en
Jerusalén, la ciudad santa, y nueve partes en las demás ciudades.
\footnote{\textbf{11:1} Neh 7,5} \bibleverse{2} El pueblo bendijo a
todos los hombres que se ofrecieron voluntariamente para habitar en
Jerusalén.

\hypertarget{listas-de-los-jefes-de-los-juduxedos-y-benjaminitas-que-vivuxedan-en-jerusaluxe9n-incluidos-sacerdotes-porteros-etc.}{%
\subsection{Listas de los jefes de los judíos y benjaminitas que vivían
en Jerusalén (incluidos sacerdotes, porteros,
etc.)}\label{listas-de-los-jefes-de-los-juduxedos-y-benjaminitas-que-vivuxedan-en-jerusaluxe9n-incluidos-sacerdotes-porteros-etc.}}

\bibleverse{3} Estos son los jefes de la provincia que vivían en
Jerusalén; pero en las ciudades de Judá, cada uno vivía en su posesión
en sus ciudades: los sacerdotes, los levitas, los servidores del templo
y los hijos de los servidores de Salomón. \footnote{\textbf{11:3} Neh
  7,57; 1Cró 9,2-17} \bibleverse{4} Algunos de los hijos de Judá y de
los hijos de Benjamín vivían en Jerusalén. De los hijos de Judá Ataías
hijo de Uzías, hijo de Zacarías, hijo de Amarías, hijo de Sefatías, hijo
de Mahalalel, de los hijos de Pérez; \bibleverse{5} y Maasías hijo de
Baruc, hijo de Colhoze, hijo de Hazaías, hijo de Adaías, hijo de
Joiarib, hijo de Zacarías, hijo del silonita. \bibleverse{6} Todos los
hijos de Pérez que vivían en Jerusalén eran cuatrocientos sesenta y ocho
hombres valientes.

\bibleverse{7} Estos son los hijos de Benjamín: Salú, hijo de Mesulam,
hijo de Joed, hijo de Pedaías, hijo de Colaías, hijo de Maasías, hijo de
Itiel, hijo de Jesaías. \bibleverse{8} Después de él, Gabbai y Sallai,
novecientos veintiocho. \bibleverse{9} Joel hijo de Zicri era su
supervisor; y Judá hijo de Hasenúa era el segundo sobre la ciudad.

\bibleverse{10} De los sacerdotes: Jedaías hijo de Joiarib, Jacín,
\bibleverse{11} Seraías hijo de Hilcías, hijo de Mesulam, hijo de Sadoc,
hijo de Meraiot, hijo de Ahitub, jefe de la casa de Dios,
\bibleverse{12} y sus hermanos que hacían la obra de la casa,
ochocientos veintidós; y Adaías hijo de Jeroham, hijo de Pelalías, hijo
de Amzi, hijo de Zacarías, hijo de Pashur, hijo de Malquías,
\bibleverse{13} y sus hermanos, jefes de las casas paternas, doscientos
cuarenta y dos y Amashsai hijo de Azarel, hijo de Ahzai, hijo de
Meshillemoth, hijo de Immer, \bibleverse{14} y sus hermanos, hombres
valientes, ciento veintiocho; y su jefe era Zabdiel, hijo de Haggedolim.

\bibleverse{15} De los levitas Semaías hijo de Hasub, hijo de Azricam,
hijo de Hasabías, hijo de Bunni; \bibleverse{16} y Sabetai y Jozabad, de
los jefes de los levitas, que tenían a su cargo los asuntos externos de
la casa de Dios; \bibleverse{17} y Matanías hijo de Mica, hijo de Zabdi,
hijo de Asaf, que era el jefe que iniciaba la acción de gracias en la
oración, y Bacbuquías, el segundo entre sus hermanos; y Abda hijo de
Sammúa, hijo de Galal, hijo de Jedutún. \bibleverse{18} Todos los
levitas de la ciudad santa eran doscientos ochenta y cuatro.

\bibleverse{19} Además, los guardianes de las puertas, Acub, Talmón y
sus hermanos, que vigilaban las puertas, eran ciento setenta y dos.
\bibleverse{20} El resto de Israel, de los sacerdotes y de los levitas,
estaban en todas las ciudades de Judá, cada uno en su heredad.
\bibleverse{21} Pero los servidores del templo vivían en Ofel, y Ziha y
Gishpa estaban al frente de los servidores del templo.

\bibleverse{22} El supervisor de los levitas en Jerusalén era Uzi hijo
de Bani, hijo de Hasabías, hijo de Matanías, hijo de Mica, de los hijos
de Asaf, los cantores, estaba a cargo de los asuntos de la casa de Dios.
\bibleverse{23} Porque había un mandamiento del rey acerca de ellos, y
una provisión establecida para los cantores, como cada día lo requería.
\bibleverse{24} Petaías, hijo de Meshezabel, de los hijos de Zera, hijo
de Judá, estaba al lado del rey en todos los asuntos del pueblo.

\hypertarget{lista-de-lugares-que-luego-fueron-poblados-por-juduxedos-benjaminitas-y-levitas}{%
\subsection{Lista de lugares que luego fueron poblados por judíos,
benjaminitas y
levitas}\label{lista-de-lugares-que-luego-fueron-poblados-por-juduxedos-benjaminitas-y-levitas}}

\bibleverse{25} En cuanto a las aldeas con sus campos, algunos de los
hijos de Judá vivían en Quiriat Arba y sus pueblos, en Dibón y sus
pueblos, en Jekabzeel y sus aldeas, \footnote{\textbf{11:25} Jos 20,7;
  Jos 21,11} \bibleverse{26} en Jesúa, en Moladah, Bet Pelet,
\bibleverse{27} en Hazar Shual en Beersheba y sus aldeas,
\bibleverse{28} en Ziklag, en Meconah y sus aldeas, \footnote{\textbf{11:28}
  Jos 15,31} \bibleverse{29} en En Rimmon, en Zorah, en Jarmuth,
\bibleverse{30} Zanoah, Adullam y sus aldeas, Lachish y sus campos, y
Azekah y sus aldeas. Y acamparon desde Beerseba hasta el valle de Hinom.
\bibleverse{31} Los hijos de Benjamín también vivían desde Geba, en
Micmas y Aija, y en Betel y sus pueblos, \footnote{\textbf{11:31} Jos
  18,22} \bibleverse{32} en Anatot, Nob, Ananías, \bibleverse{33} Hazor,
Ramá, Gittaim, \bibleverse{34} Hadid, Zeboim, Neballat, \bibleverse{35}
Lod y Ono, el valle de los artesanos. \bibleverse{36} De los levitas,
ciertas divisiones de Judá se establecieron en el territorio de
Benjamín.

\hypertarget{clases-de-sacerdotes-y-levitas-que-regresaron-con-zorobabel-y-jesuxfas}{%
\subsection{Clases de sacerdotes y levitas que regresaron con Zorobabel
y
Jesús}\label{clases-de-sacerdotes-y-levitas-que-regresaron-con-zorobabel-y-jesuxfas}}

\hypertarget{section-11}{%
\section{12}\label{section-11}}

\bibleverse{1} Estos son los sacerdotes y levitas que subieron con
Zorobabel, hijo de Salatiel, y con Jesúa: Seraías, Jeremías, Esdras,
\footnote{\textbf{12:1} Esd 2,2} \bibleverse{2} Amarías, Malluch,
Hattush, \bibleverse{3} Secanías, Rehum, Meremot, \bibleverse{4} Iddo,
Ginnethoi, Abías, \footnote{\textbf{12:4} Luc 1,5} \bibleverse{5}
Mijamín, Maadia, Bilgah, \bibleverse{6} Semaías, Joiarib, Jedaías,
\bibleverse{7} Sallu, Amok, Hilkiah y Jedaías. Estos eran los jefes de
los sacerdotes y de sus hermanos en los días de Jesúa.

\bibleverse{8} Además, los levitas eran Jesúa, Binúi, Cadmiel, Serebías,
Judá y Matanías, que estaba a cargo de los cantos de acción de gracias,
él y sus hermanos. \footnote{\textbf{12:8} Neh 11,17} \bibleverse{9}
También Bakbucías y Unno, sus hermanos, estaban cerca de ellos según sus
cargos.

\hypertarget{la-luxednea-de-sumo-sacerdote}{%
\subsection{La línea de sumo
sacerdote}\label{la-luxednea-de-sumo-sacerdote}}

\bibleverse{10} Jesúa fue padre de Joiaquim, y Joiaquim fue padre de
Eliasib, y Eliasib fue padre de Joiada, \footnote{\textbf{12:10} Neh
  12,1; Neh 12,26; Neh 3,1; Neh 3,20} \bibleverse{11} y Joiada fue padre
de Jonatán, y Jonatán fue padre de Jaddua.

\hypertarget{jefes-de-familia-sacerdotal-desde-la-uxe9poca-del-sumo-sacerdote-joiacim}{%
\subsection{Jefes de familia sacerdotal desde la época del sumo
sacerdote
Joiacim}\label{jefes-de-familia-sacerdotal-desde-la-uxe9poca-del-sumo-sacerdote-joiacim}}

\bibleverse{12} En los días de Joiakim había sacerdotes, jefes de
familia de los padres: de Seraías, Meraías; de Jeremías, Hananías;
\bibleverse{13} de Esdras, Mesulam; de Amarías, Johanán; \bibleverse{14}
de Malluchi, Jonatán; de Sebanías, José; \bibleverse{15} de Harim, Adna;
de Meraiot, Helkai; \bibleverse{16} de Iddo, Zacarías; de Ginetón,
Mesulam; \bibleverse{17} de Abías, Zicri; de Miniamin, de Moadías,
Piltai; \bibleverse{18} de Bilgah, Sammua; de Semaías, Jonatán;
\bibleverse{19} de Joiarib, Mattenai; de Jedaías, Uzzi; \bibleverse{20}
de Sallai, Kallai; de Amok, Eber; \bibleverse{21} de Hilcías, Hasabías;
de Jedaías, Netanel.

\hypertarget{lista-de-los-levitas-hasta-la-uxe9poca-del-sumo-sacerdote-johanuxe1n}{%
\subsection{Lista de los levitas hasta la época del sumo sacerdote
Johanán}\label{lista-de-los-levitas-hasta-la-uxe9poca-del-sumo-sacerdote-johanuxe1n}}

\bibleverse{22} En cuanto a los levitas, en los días de Eliasib, Joiada,
Johanán y Jaddua, se registraron los jefes de familia; también los
sacerdotes, en el reinado de Darío el Persa. \footnote{\textbf{12:22}
  Neh 12,10-11} \bibleverse{23} Los hijos de Leví, jefes de familia,
fueron inscritos en el libro de las crónicas, hasta los días de Johanán
hijo de Eliasib. \bibleverse{24} Los jefes de los levitas: Hasabías,
Serebías y Jesúa hijo de Cadmiel, con sus hermanos cerca de ellos, para
alabar y dar gracias según el mandato de David, el hombre de Dios,
sección tras sección. \footnote{\textbf{12:24} 1Cró 25,1; 2Cró 29,25}
\bibleverse{25} Matanías, Bacbuquías, Abdías, Mesulam, Talmón y Acub
eran guardianes de las puertas, que vigilaban los almacenes de las
puertas. \footnote{\textbf{12:25} Neh 11,17; Neh 11,19; 2Cró 8,14; 1Cró
  26,15; 1Cró 26,17} \bibleverse{26} Estos eran en los días de Joiaquim
hijo de Jesúa, hijo de Josadac, y en los días del gobernador Nehemías y
del sacerdote y escriba Esdras. \footnote{\textbf{12:26} Neh 12,10; 1Cró
  6,14-15; Neh 5,14; Esd 7,1-6}

\hypertarget{inauguraciuxf3n-de-la-muralla-de-la-ciudad}{%
\subsection{Inauguración de la muralla de la
ciudad}\label{inauguraciuxf3n-de-la-muralla-de-la-ciudad}}

\bibleverse{27} En la dedicación del muro de Jerusalén, buscaron a los
levitas de todos sus lugares para traerlos a Jerusalén a celebrar la
dedicación con alegría, dando gracias y cantando, con címbalos,
instrumentos de cuerda y arpas. \bibleverse{28} Los hijos de los
cantores se reunieron, tanto de la llanura que rodea a Jerusalén como de
las aldeas de los netofatitas; \bibleverse{29} también de Bet Gilgal y
de los campos de Geba y Azmavet, porque los cantores se habían
construido aldeas alrededor de Jerusalén. \bibleverse{30} Los sacerdotes
y los levitas se purificaron, y purificaron el pueblo, las puertas y el
muro.

\bibleverse{31} Entonces hice subir a los príncipes de Judá al muro, y
designé dos grandes compañías que dieron gracias y fueron en procesión.
Una iba a la derecha en el muro, hacia la puerta del estiércol;
\footnote{\textbf{12:31} Neh 2,13; Neh 3,13} \bibleverse{32} y tras
ellos iban Oseas, con la mitad de los príncipes de Judá, \bibleverse{33}
y Azarías, Esdras y Mesulam, \bibleverse{34} Judá, Benjamín, Semaías,
Jeremías, \bibleverse{35} y algunos de los hijos de los sacerdotes con
trompetas: Zacarías hijo de Jonatán, hijo de Semaías, hijo de Matanías,
hijo de Micaías, hijo de Zacur, hijo de Asaf; \bibleverse{36} y sus
hermanos, Semaías, Azarel, Milalai, Gilalai, Maai, Netanel, Judá y
Hanani, con los instrumentos musicales de David, el hombre de Dios; y el
escriba Esdras estaba delante de ellos. \bibleverse{37} Por la puerta
del manantial, y en línea recta delante de ellos, subieron por las
escaleras de la ciudad de David, en la subida del muro, por encima de la
casa de David, hasta la puerta de las aguas, hacia el este. \footnote{\textbf{12:37}
  Neh 3,26}

\bibleverse{38} La otra compañía de los que daban gracias salió a su
encuentro, y yo tras ellos, con la mitad del pueblo en el muro, por
encima de la torre de los hornos, hasta el muro ancho, \footnote{\textbf{12:38}
  Neh 3,11} \bibleverse{39} y por encima de la puerta de Efraín, y por
la puerta vieja, y por la puerta del pescado, la torre de Hananel, y la
torre de Hammea, hasta la puerta de las ovejas; y se detuvieron en la
puerta de la guardia. \bibleverse{40} Las dos compañías de los que daban
gracias en la casa de Dios estaban en pie, y yo y la mitad de los jefes
conmigo; \bibleverse{41} y los sacerdotes, Eliaquim, Maasías, Miniamin,
Micaías, Elioenai, Zacarías y Hananías, con trompetas; \bibleverse{42} y
Maasías, Semaías, Eleazar, Uzi, Johanán, Malquías, Elam y Ezer. Los
cantores cantaban en voz alta, con Jezrahías, su supervisor.
\bibleverse{43} Aquel día ofrecieron grandes sacrificios y se
regocijaron, porque Dios los había hecho gozar con gran alegría; y
también se regocijaron las mujeres y los niños, de modo que la alegría
de Jerusalén se oyó hasta muy lejos.

\hypertarget{empleo-de-funcionarios-para-supervisar-los-ingresos-de-los-sacerdotes-y-levitas}{%
\subsection{Empleo de funcionarios para supervisar los ingresos de los
sacerdotes y
levitas}\label{empleo-de-funcionarios-para-supervisar-los-ingresos-de-los-sacerdotes-y-levitas}}

\bibleverse{44} Aquel día se designaron hombres sobre las salas para los
tesoros, para las ofrendas de las olas, para las primicias y para los
diezmos, a fin de recoger en ellas, según los campos de las ciudades,
las porciones señaladas por la ley para los sacerdotes y los levitas;
porque Judá se alegró por los sacerdotes y por los levitas que servían.
\footnote{\textbf{12:44} Neh 10,36; Neh 13,5} \bibleverse{45} Ellos
cumplían el deber de su Dios y el deber de la purificación, y lo mismo
hacían los cantores y los porteros, según el mandato de David y de
Salomón su hijo. \bibleverse{46} Porque en los días de David y de Asaf
de antaño había un jefe de los cantores, y cantos de alabanza y de
acción de gracias a Dios. \footnote{\textbf{12:46} 1Cró 25,1}
\bibleverse{47} Todo Israel, en los días de Zorobabel y en los días de
Nehemías, dio las porciones de los cantores y de los porteros, según
cada día; y apartaron lo que era para los levitas; y los levitas
apartaron lo que era para los hijos de Aarón. \footnote{\textbf{12:47}
  Neh 10,38}

\hypertarget{eliminaciuxf3n-de-los-componentes-paganos-especialmente-amonitas-y-moabitas-de-la-comunidad}{%
\subsection{Eliminación de los componentes paganos (especialmente
amonitas y moabitas) de la
comunidad}\label{eliminaciuxf3n-de-los-componentes-paganos-especialmente-amonitas-y-moabitas-de-la-comunidad}}

\hypertarget{section-12}{%
\section{13}\label{section-12}}

\bibleverse{1} Aquel día leyeron en el libro de Moisés a oídos del
pueblo, y se encontró escrito en él que un amonita y un moabita no
debían entrar en la asamblea de Dios para siempre, \footnote{\textbf{13:1}
  Deut 23,3-5} \bibleverse{2} porque no salieron al encuentro de los
hijos de Israel con pan y agua, sino que contrataron a Balaam contra
ellos para maldecirlos; sin embargo, nuestro Dios convirtió la maldición
en bendición. \footnote{\textbf{13:2} Núm 22,5-6} \bibleverse{3} Sucedió
que cuando escucharon la ley, separaron de Israel a toda la multitud
mixta.

\hypertarget{eliminaciuxf3n-de-la-celda-de-tobija-en-el-templo}{%
\subsection{Eliminación de la celda de Tobija en el
templo}\label{eliminaciuxf3n-de-la-celda-de-tobija-en-el-templo}}

\bibleverse{4} Antes de esto, el sacerdote Eliasib, que había sido
designado como encargado de las habitaciones de la casa de nuestro Dios,
siendo aliado de Tobías, \bibleverse{5} había preparado para él una gran
sala, en la que antes se depositaban las ofrendas de comida, el
incienso, los vasos y los diezmos del grano, el vino nuevo y el aceite,
que se daban por mandato a los levitas, a los cantores y a los porteros;
y las ofrendas onduladas para los sacerdotes. \bibleverse{6} Pero en
todo esto no estuve en Jerusalén, porque en el año treinta y dos de
Artajerjes, rey de Babilonia, fui al rey; y después de algunos días pedí
permiso al rey, \bibleverse{7} y llegué a Jerusalén, y comprendí el mal
que Eliasib había hecho a Tobías, al prepararle una habitación en los
atrios de la casa de Dios. \bibleverse{8} Esto me afligió mucho. Por eso
eché de la habitación todos los enseres de Tobías. \bibleverse{9} Luego
ordené, y ellos limpiaron las habitaciones. Llevé a ellas los utensilios
de la casa de Dios, con las ofrendas de comida y el incienso de nuevo.
\footnote{\textbf{13:9} Neh 10,39}

\hypertarget{asegurar-la-correcta-entrega-de-los-tributos-a-los-levitas}{%
\subsection{Asegurar la correcta entrega de los tributos a los
levitas}\label{asegurar-la-correcta-entrega-de-los-tributos-a-los-levitas}}

\bibleverse{10} Me di cuenta de que las porciones de los levitas no se
les habían dado, de modo que los levitas y los cantores, que hacían el
trabajo, habían huido cada uno a su campo. \bibleverse{11} Entonces
discutí con los jefes y dije: ``¿Por qué está abandonada la casa de
Dios?'' Los reuní y los puse en su lugar. \footnote{\textbf{13:11} Neh
  10,39} \bibleverse{12} Entonces todo Judá trajo el diezmo del grano,
del vino nuevo y del aceite a las arcas. \footnote{\textbf{13:12} Núm
  18,21} \bibleverse{13} Puse como tesoreros sobre los tesoros al
sacerdote Selemías y al escriba Sadoc, y de los levitas a Pedaías; y
junto a ellos a Hanán hijo de Zaccur, hijo de Matanías; porque eran
tenidos por fieles, y su oficio era repartir a sus hermanos.

\bibleverse{14} Acuérdate de mí, Dios mío, en cuanto a esto, y no borres
mis buenas obras que he hecho por la casa de mi Dios y por sus
celebraciones. \footnote{\textbf{13:14} Neh 13,31; Neh 5,19}

\hypertarget{eliminar-la-profanaciuxf3n-del-suxe1bado-por-parte-de-empresarios-y-comerciantes}{%
\subsection{Eliminar la profanación del sábado por parte de empresarios
y
comerciantes}\label{eliminar-la-profanaciuxf3n-del-suxe1bado-por-parte-de-empresarios-y-comerciantes}}

\bibleverse{15} En aquellos días vi a algunos hombres que pisaban
lagares en sábado en Judá, que traían gavillas y cargaban asnos con
vino, uvas, higos y toda clase de cargas que llevaban a Jerusalén en día
de reposo; y testifiqué contra ellos en el día en que vendían alimentos.
\footnote{\textbf{13:15} Neh 10,31; Jer 17,21-27} \bibleverse{16}
También vivían allí algunos hombres de Tiro, que traían pescado y toda
clase de mercancías, y vendían en sábado a los hijos de Judá y en
Jerusalén. \bibleverse{17} Entonces discutí con los nobles de Judá y les
dije: ``¿Qué maldad es ésta que hacéis, profanando el día de reposo?
\bibleverse{18} ¿No hicieron esto vuestros padres, y no trajo nuestro
Dios todo este mal sobre nosotros y sobre esta ciudad? Sin embargo,
vosotros traéis más ira sobre Israel al profanar el sábado''.

\bibleverse{19} Sucedió que cuando las puertas de Jerusalén comenzaron a
oscurecerse antes del sábado, mandé cerrar las puertas y ordené que no
se abrieran hasta después del sábado. Puse a algunos de mis siervos a
cargo de las puertas, para que no se introdujera ninguna carga en el día
de reposo. \bibleverse{20} Entonces los mercaderes y vendedores de toda
clase de mercancías acamparon fuera de Jerusalén una o dos veces.
\bibleverse{21} Entonces yo testifiqué contra ellos y les dije: ``¿Por
qué os quedáis alrededor del muro? Si volvéis a hacerlo, os echaré
mano''. Desde entonces, no vinieron en sábado. \bibleverse{22} Mandé a
los levitas que se purificaran, y que vinieran a guardar las puertas,
para santificar el día de reposo. Acuérdate de mí también por esto, Dios
mío, y perdóname según la grandeza de tu amorosa bondad. \footnote{\textbf{13:22}
  Neh 13,14}

\hypertarget{medidas-contra-los-matrimonios-mixtos-rechazo-del-hijo-de-un-sumo-sacerdote}{%
\subsection{Medidas contra los matrimonios mixtos; Rechazo del hijo de
un sumo
sacerdote}\label{medidas-contra-los-matrimonios-mixtos-rechazo-del-hijo-de-un-sumo-sacerdote}}

\bibleverse{23} En aquellos días vi también a los judíos que se habían
casado con mujeres de Asdod, de Amón y de Moab; \bibleverse{24} y sus
hijos hablaban la mitad en el idioma de Asdod, y no podían hablar en la
lengua de los judíos, sino según la lengua de cada pueblo.
\bibleverse{25} Yo discutí con ellos, los maldije, golpeé a algunos de
ellos, les arranqué el cabello y les hice jurar por Dios: ``No daréis
vuestras hijas a sus hijos, ni tomaréis sus hijas para vuestros hijos,
ni para vosotros. \footnote{\textbf{13:25} Deut 7,3} \bibleverse{26} ¿No
pecó Salomón, rey de Israel, con estas cosas? Sin embargo, entre muchas
naciones no hubo un rey como él, y fue amado por su Dios, y Dios lo hizo
rey de todo Israel. Sin embargo, las mujeres extranjeras lo hicieron
pecar. \footnote{\textbf{13:26} 1Re 11,3-8} \bibleverse{27} ¿Debemos,
pues, escucharte para hacer todo este gran mal, para transgredir a
nuestro Dios casándonos con mujeres extranjeras?''

\bibleverse{28} Uno de los hijos de Joiada, hijo del sumo sacerdote
Eliasib, era yerno de Sanbalat el horonita; por eso lo eché de mí.
\footnote{\textbf{13:28} Neh 11,10; Neh 2,19} \bibleverse{29} Acuérdate
de ellos, Dios mío, porque han profanado el sacerdocio y la alianza del
sacerdocio y de los levitas.

\hypertarget{fin-del-memorando}{%
\subsection{Fin del memorando}\label{fin-del-memorando}}

\bibleverse{30} Así los limpié de todos los extranjeros y señalé los
deberes para los sacerdotes y para los levitas, cada uno en su trabajo;
\bibleverse{31} y para la ofrenda de leña, en los tiempos señalados, y
para las primicias. Acuérdate de mí, Dios mío, para bien. \footnote{\textbf{13:31}
  Neh 13,14; Neh 13,22; Neh 5,19}
