\hypertarget{la-queja-de-david-por-sauxfal-y-jonatuxe1n-ante-la-noticia-de-sus-muertes}{%
\subsection{La queja de David por Saúl y Jonatán ante la noticia de sus
muertes}\label{la-queja-de-david-por-sauxfal-y-jonatuxe1n-ante-la-noticia-de-sus-muertes}}

\hypertarget{section}{%
\section{1}\label{section}}

\bibleverse{1} Después de la muerte de Saúl, cuando David regresó de la
matanza de los amalecitas, y David había permanecido dos días en Siclag,
\bibleverse{2} al tercer día, he aquí que un hombre salió del campamento
de Saúl, con sus ropas rasgadas y tierra en la cabeza. Cuando llegó a
David, se postró en tierra y le mostró respeto.

\hypertarget{el-informe-del-mensajero-sobre-los-momentos-finales-de-sauxfal}{%
\subsection{El informe del mensajero sobre los momentos finales de
Saúl}\label{el-informe-del-mensajero-sobre-los-momentos-finales-de-sauxfal}}

\bibleverse{3} David le dijo: ``¿De dónde vienes?'' Le dijo: ``He
escapado del campamento de Israel''.

\bibleverse{4} David le dijo: ``¿Cómo te fue? Por favor, cuéntame''. Él
respondió: ``El pueblo ha huido de la batalla, y también muchos del
pueblo han caído y están muertos. También han muerto Saúl y su hijo
Jonatán''.

\bibleverse{5} David dijo al joven que se lo contó: ``¿Cómo sabes que
Saúl y su hijo Jonatán han muerto?''.

\bibleverse{6} El joven que se lo contó dijo: ``Cuando pasé por
casualidad por el monte Gilboa, he aquí que Saúl estaba apoyado en su
lanza, y he aquí que los carros y la caballería le seguían de cerca.
\footnote{\textbf{1:6} 1Sam 31,1-3} \bibleverse{7} Cuando miró detrás de
él, me vio y me llamó. Yo respondí: ``Aquí estoy''. \bibleverse{8} Me
dijo: ``¿Quién eres tú? Yo le respondí: ``Soy amalecita''.
\bibleverse{9} Me dijo: `Por favor, ponte a mi lado y mátame, pues la
angustia se ha apoderado de mí porque mi vida perdura'. \bibleverse{10}
Así que me puse a su lado y lo maté, porque estaba seguro de que no
podría vivir después de haber caído. Tomé la corona que llevaba en la
cabeza y el brazalete que tenía en el brazo, y se los he traído a mi
señor''.

\hypertarget{el-dolor-de-david-matando-al-mensajero}{%
\subsection{El dolor de David; Matando al
mensajero}\label{el-dolor-de-david-matando-al-mensajero}}

\bibleverse{11} Entonces David se agarró a sus ropas y las rasgó; y
todos los hombres que estaban con él hicieron lo mismo. \footnote{\textbf{1:11}
  Gén 37,29} \bibleverse{12} Hicieron duelo, lloraron y ayunaron hasta
la noche por Saúl y por su hijo Jonatán, y por el pueblo de Yahvé, y por
la casa de Israel, porque habían caído a espada. \footnote{\textbf{1:12}
  1Sam 31,13}

\bibleverse{13} David dijo al joven que se lo contó: ``¿De dónde
eres?''. Respondió: ``Soy hijo de un extranjero, un amalecita''.

\bibleverse{14} David le dijo: ``¿Por qué no tuviste miedo de extender
tu mano para destruir al ungido de Yahvé?'' \footnote{\textbf{1:14} 1Sam
  24,6} \bibleverse{15} David llamó a uno de los jóvenes y le dijo:
``¡Acércate y derríbalo!'' Lo golpeó de tal manera que murió.
\footnote{\textbf{1:15} 2Sam 4,10; 2Sam 4,12} \bibleverse{16} David le
dijo: ``Que tu sangre caiga sobre tu cabeza, porque tu boca ha dado
testimonio contra ti, diciendo: ``He matado al ungido de Yahvé''.''
\footnote{\textbf{1:16} 1Re 2,23; 1Re 2,33}

\hypertarget{lamentaciuxf3n-de-david-por-sauxfal-y-jonatuxe1n}{%
\subsection{Lamentación de David por Saúl y
Jonatán}\label{lamentaciuxf3n-de-david-por-sauxfal-y-jonatuxe1n}}

\bibleverse{17} David se lamentó con este lamento por Saúl y por
Jonatán, su hijo \bibleverse{18} (y les ordenó que enseñaran a los hijos
de Judá el canto del arco; he aquí que está escrito en el libro de
Jasar): \footnote{\textbf{1:18} 2Sam 1,22; Jos 10,13} \bibleverse{19}
``¡Tu gloria, Israel, fue asesinada en tus lugares altos! ¡Cómo han
caído los poderosos! \bibleverse{20} No lo cuentes en Gat. No lo
publiques en las calles de Ashkelon, para que las hijas de los filisteos
no se alegren, para que no triunfen las hijas de los incircuncisos.
\footnote{\textbf{1:20} Miq 1,10; 1Sam 18,6} \bibleverse{21} Montes de
Gilboa, que no haya rocío ni lluvia sobre ti, ni campos de ofrendas;
porque allí el escudo de los poderosos fue profanado y desechado, el
escudo de Saúl no fue ungido con aceite. \footnote{\textbf{1:21} Núm
  15,18-21} \bibleverse{22} De la sangre de los muertos, de la grasa de
los poderosos, El arco de Jonathan no se volvió. La espada de Saúl no
volvió vacía. \bibleverse{23} Saúl y Jonatán fueron encantadores y
agradables en sus vidas. En su muerte, no fueron divididos. Eran más
veloces que las águilas. Eran más fuertes que los leones.
\bibleverse{24} Hijas de Israel, llorad a Saúl, que te vistió
delicadamente de escarlata, que ponen adornos de oro en su ropa.
\bibleverse{25} ¡Cómo han caído los poderosos en medio de la batalla!
Jonathan fue asesinado en sus lugares altos. \bibleverse{26} Estoy
angustiado por ti, hermano Jonatán. Has sido muy agradable conmigo. Su
amor hacia mí fue maravilloso, superando el amor de las mujeres.
\bibleverse{27} Cómo han caído los poderosos, y las armas de guerra han
perecido''.

\hypertarget{david-llega-a-ser-rey-sobre-la-tribu-de-juduxe1-isboseth-sobre-israel}{%
\subsection{David llega a ser rey sobre la tribu de Judá, Isboseth sobre
Israel}\label{david-llega-a-ser-rey-sobre-la-tribu-de-juduxe1-isboseth-sobre-israel}}

\hypertarget{section-1}{%
\section{2}\label{section-1}}

\bibleverse{1} Después de esto, David consultó a Yahvé, diciendo:
``¿Debo subir a alguna de las ciudades de Judá?''. Yahvé le dijo:
``Sube''. David dijo: ``¿Adónde subiré?'' Dijo: ``A Hebrón''.
\footnote{\textbf{2:1} 1Sam 30,8}

\bibleverse{2} David subió allí con sus dos mujeres, Ahinoam la
jezreelita y Abigail la mujer de Nabal el carmelita. \footnote{\textbf{2:2}
  1Sam 25,42-43} \bibleverse{3} David hizo subir a sus hombres que
estaban con él, cada uno con su familia. Vivían en las ciudades de
Hebrón.

\hypertarget{el-mensaje-de-david-al-pueblo-de-jabes}{%
\subsection{El mensaje de David al pueblo de
Jabes}\label{el-mensaje-de-david-al-pueblo-de-jabes}}

\bibleverse{4} Vinieron los hombres de Judá y allí ungieron a David como
rey de la casa de Judá. Le dijeron a David: ``Los hombres de Jabes de
Galaad fueron los que enterraron a Saúl''. \footnote{\textbf{2:4} 2Sam
  5,3; 1Sam 16,13; 1Sam 31,12} \bibleverse{5} David envió mensajeros a
los hombres de Jabes de Galaad y les dijo: ``Benditos seáis por Yahvé,
porque habéis mostrado esta bondad con vuestro señor, con Saúl, y lo
habéis enterrado. \bibleverse{6} Que el Señor les muestre su bondad y su
verdad. Yo también te recompensaré por esta bondad, porque has hecho
esto. \bibleverse{7} Ahora, pues, fortalece tus manos y sé valiente,
porque Saúl, tu señor, ha muerto, y también la casa de Judá me ha ungido
como rey sobre ellos.''

\hypertarget{isboseth-hijo-de-sauxfal-se-convierte-en-rey-de-israel}{%
\subsection{Isboseth, hijo de Saúl, se convierte en rey de
Israel}\label{isboseth-hijo-de-sauxfal-se-convierte-en-rey-de-israel}}

\bibleverse{8} Abner, hijo de Ner, capitán del ejército de Saúl, había
capturado a Isboset, hijo de Saúl, y lo había llevado a Mahanaim.
\footnote{\textbf{2:8} 1Sam 14,50} \bibleverse{9} Lo hizo rey de Galaad,
de los asuritas, de Jezreel, de Efraín, de Benjamín y de todo Israel.
\bibleverse{10} Isboset, hijo de Saúl, tenía cuarenta años cuando
comenzó a reinar sobre Israel, y reinó dos años. Pero la casa de Judá
siguió a David. \bibleverse{11} El tiempo que David fue rey en Hebrón
sobre la casa de Judá fue de siete años y seis meses.

\hypertarget{juego-de-lucha-y-batalla-en-gabauxf3n-la-victoria-de-joab}{%
\subsection{Juego de lucha y batalla en Gabaón; La victoria de
Joab}\label{juego-de-lucha-y-batalla-en-gabauxf3n-la-victoria-de-joab}}

\bibleverse{12} Abner hijo de Ner, y los siervos de Isboset hijo de
Saúl, salieron de Mahanaim a Gabaón. \bibleverse{13} Joab, hijo de
Sarvia, y los siervos de David salieron a su encuentro junto al estanque
de Gabaón, y se sentaron, el uno a un lado del estanque y el otro al
otro. \bibleverse{14} Abner dijo a Joab: ``¡Por favor, que los jóvenes
se levanten y compitan ante nosotros!'' Joab dijo: ``¡Que se levanten!''
\bibleverse{15} Entonces se levantaron y pasaron por número: doce por
Benjamín y por Isboset, hijo de Saúl, y doce de los siervos de David.
\bibleverse{16} Cada uno de ellos agarró a su adversario por la cabeza y
le clavó la espada en el costado a su compañero; así cayeron juntos. Por
eso aquel lugar de Gabaón se llamó Helkath Hazzurim. \bibleverse{17} La
batalla fue muy dura aquel día, y Abner y los hombres de Israel fueron
derrotados ante los servidores de David.

\hypertarget{asael-el-hermano-menor-de-joab-muerto-en-persecuciuxf3n-de-abner}{%
\subsection{Asael, el hermano menor de Joab, muerto en persecución de
Abner}\label{asael-el-hermano-menor-de-joab-muerto-en-persecuciuxf3n-de-abner}}

\bibleverse{18} Los tres hijos de Sarvia estaban allí: Joab, Abisai y
Asael. Asael era ligero de pies como una gacela salvaje. \footnote{\textbf{2:18}
  1Cró 2,16} \bibleverse{19} Asael persiguió a Abner. No se volvió ni a
la derecha ni a la izquierda de seguir a Abner.

\bibleverse{20} Entonces Abner miró detrás de él y dijo: ``¿Eres tú,
Asahel?'' Respondió: ``Lo es''.

\bibleverse{21} Abner le dijo: ``Vuélvete a tu derecha o a tu izquierda,
agarra a uno de los jóvenes y toma su armadura''. Pero Asahel no quiso
dejar de seguirlo. \bibleverse{22} Abner le dijo de nuevo a Asael:
``Apártate de seguirme. ¿Por qué habría de tirarte al suelo? ¿Cómo
podría entonces mirar a la cara a tu hermano Joab?'' \bibleverse{23} Sin
embargo, él se negó a apartarse. Entonces Abner, con el extremo
posterior de la lanza, lo golpeó en el cuerpo, de modo que la lanza
salió por detrás de él; y allí cayó y murió en el mismo lugar. Todos los
que llegaron al lugar donde cayó y murió Asael se detuvieron.
\footnote{\textbf{2:23} 2Sam 3,27}

\hypertarget{fin-de-la-persecuciuxf3n-continuaciuxf3n-de-la-guerra}{%
\subsection{Fin de la persecución; Continuación de la
guerra}\label{fin-de-la-persecuciuxf3n-continuaciuxf3n-de-la-guerra}}

\bibleverse{24} Pero Joab y Abisai persiguieron a Abner. El sol se puso
cuando llegaron a la colina de Amma, que está frente a Giah por el
camino del desierto de Gabaón. \bibleverse{25} Los hijos de Benjamín se
reunieron en pos de Abner y se convirtieron en un solo grupo, y se
pusieron en la cima de la colina. \bibleverse{26} Entonces Abner llamó a
Joab y le dijo: ``¿La espada va a devorar para siempre? ¿No sabes que al
final será amarga? ¿Cuánto tiempo pasará entonces, antes de que pidas al
pueblo que vuelva de seguir a sus hermanos?''

\bibleverse{27} Joab dijo: ``Vive Dios, si no hubieras hablado,
seguramente por la mañana el pueblo se habría ido, y no habría seguido
cada uno a su hermano''. \bibleverse{28} Así que Joab tocó la trompeta,
y todo el pueblo se detuvo y no persiguió más a Israel, y no lucharon
más. \bibleverse{29} Abner y sus hombres recorrieron toda aquella noche
el Arabá, y pasaron el Jordán, atravesaron todo Bitrón y llegaron a
Mahanaim.

\bibleverse{30} Joab regresó de seguir a Abner, y cuando reunió a todo
el pueblo, faltaban diecinueve hombres de David y Asael. \bibleverse{31}
Pero los siervos de David habían herido a los hombres de Abner, de modo
que murieron trescientos sesenta hombres. \bibleverse{32} Recogieron a
Asael y lo enterraron en la tumba de su padre, que estaba en Belén. Joab
y sus hombres pasaron toda la noche, y el día amaneció en Hebrón.

\hypertarget{section-2}{%
\section{3}\label{section-2}}

\bibleverse{1} Hubo una larga guerra entre la casa de Saúl y la de
David. David se hacía cada vez más fuerte, pero la casa de Saúl se
debilitaba cada vez más. \footnote{\textbf{3:1} 2Sam 5,10}

\hypertarget{la-familia-de-david-en-hebruxf3n}{%
\subsection{La familia de David en
Hebrón}\label{la-familia-de-david-en-hebruxf3n}}

\bibleverse{2} A David le nacieron hijos en Hebrón. Su primogénito fue
Amnón, de Ahinoam la jezreelita; \footnote{\textbf{3:2} 1Cró 3,1-4; 2Sam
  13,1} \bibleverse{3} y su segundo, Chileab, de Abigail la mujer de
Nabal el carmelita; y el tercero, Absalón, hijo de Maaca la hija de
Talmai, rey de Guesur; \bibleverse{4} y el cuarto, Adonías, hijo de
Haggit; y el quinto, Sefatías, hijo de Abital; \footnote{\textbf{3:4}
  1Re 1,5} \bibleverse{5} y el sexto, Itream, de Egla, mujer de David.
Estos le nacieron a David en Hebrón.

\hypertarget{abner-se-estuxe1-peleando-con-isboseth}{%
\subsection{Abner se está peleando con
Isboseth}\label{abner-se-estuxe1-peleando-con-isboseth}}

\bibleverse{6} Mientras había guerra entre la casa de Saúl y la de
David, Abner se hizo fuerte en la casa de Saúl. \bibleverse{7} Saúl
tenía una concubina que se llamaba Rizpa, hija de Aja; e Ishboset le
dijo a Abner: ``¿Por qué te has metido con la concubina de mi padre?''
\footnote{\textbf{3:7} 2Sam 21,8}

\bibleverse{8} Entonces Abner se enojó mucho por las palabras de
Ishboshet, y dijo: ``¿Soy yo una cabeza de perro que pertenece a Judá?
Hoy me muestro bondadoso con la casa de tu padre Saúl, con sus hermanos
y con sus amigos, y no te he entregado en manos de David; ¡y sin embargo
me acusas hoy de una falta con respecto a esta mujer! \bibleverse{9} Que
Dios haga lo mismo con Abner, y más aún, si, como Yahvé ha jurado a
David, no hago lo mismo con él: \bibleverse{10} transferir el reino de
la casa de Saúl y establecer el trono de David sobre Israel y sobre
Judá, desde Dan hasta Beerseba.''

\bibleverse{11} No pudo responder a Abner ni una palabra más, porque le
tenía miedo.

\hypertarget{las-negociaciones-de-abner-con-david-y-los-jefes-de-israel}{%
\subsection{Las negociaciones de Abner con David y los jefes de
Israel}\label{las-negociaciones-de-abner-con-david-y-los-jefes-de-israel}}

\bibleverse{12} Abner envió mensajeros a David en su nombre, diciéndole:
``¿De quién es la tierra?'' y diciendo: ``Haz tu alianza conmigo, y he
aquí que mi mano estará contigo para traer a todo Israel a tu
alrededor.''

\bibleverse{13} David dijo: ``Bien. Haré un tratado contigo, pero te
pido una cosa. Esto es, que no verás mi rostro a menos que primero
traigas a Mical, la hija de Saúl, cuando vengas a ver mi rostro''.
\bibleverse{14} David envió mensajeros a Isboset, hijo de Saúl,
diciéndole: ``Entrégame a mi esposa Mical, a quien me dieron en
matrimonio por cien prepucios de los filisteos.'' \footnote{\textbf{3:14}
  1Sam 18,25-27}

\bibleverse{15} Ishbosheth envió y la separó de su marido, Paltiel hijo
de Laish. \bibleverse{16} Su marido la acompañó, llorando, y la siguió
hasta Bahurim. Entonces Abner le dijo: ``¡Vete, vuelve!'', y regresó.

\bibleverse{17} Abner se comunicó con los ancianos de Israel, diciendo:
``En tiempos pasados, ustedes buscaban que David fuera rey sobre
ustedes. \bibleverse{18} ¡Ahora, pues, háganlo! Porque Yahvé ha hablado
de David, diciendo: `Por la mano de mi siervo David, salvaré a mi pueblo
Israel de la mano de los filisteos y de la mano de todos sus
enemigos'.''

\bibleverse{19} Abner también habló en los oídos de Benjamín; y Abner
también fue a hablar en los oídos de David en Hebrón todo lo que le
parecía bien a Israel y a toda la casa de Benjamín.

\hypertarget{el-encuentro-de-abner-con-david-en-hebruxf3n-su-asesinato-por-joab}{%
\subsection{El encuentro de Abner con David en Hebrón; su asesinato por
Joab}\label{el-encuentro-de-abner-con-david-en-hebruxf3n-su-asesinato-por-joab}}

\bibleverse{20} Y Abner vino a David a Hebrón, y veinte hombres con él.
David hizo un banquete a Abner y a los hombres que estaban con él.
\bibleverse{21} Abner dijo a David: ``Me levantaré e iré y reuniré a
todo Israel ante mi señor el rey, para que hagan un pacto contigo y para
que reines sobre todo lo que tu alma desee.'' David despidió a Abner, y
éste se fue en paz.

\bibleverse{22} He aquí que los siervos de David y Joab venían de una
incursión y traían consigo un gran botín; pero Abner no estaba con David
en Hebrón, pues éste lo había despedido y se había ido en paz.
\bibleverse{23} Cuando llegó Joab y todo el ejército que estaba con él,
le dijeron a Joab: ``Abner hijo de Ner vino al rey, y él lo ha
despedido, y se ha ido en paz.''

\bibleverse{24} Entonces Joab se acercó al rey y le dijo: ``¿Qué has
hecho? He aquí que Abner ha venido a ti. ¿Por qué lo has despedido, y ya
se ha ido? \bibleverse{25} Tú conoces a Abner, hijo de Ner. Vino a
engañarte y a conocer tu salida y tu entrada, y a saber todo lo que
haces''.

\bibleverse{26} Cuando Joab salió de David, envió mensajeros en busca de
Abner, y lo trajeron de vuelta del pozo de Sira; pero David no lo sabía.
\bibleverse{27} Cuando Abner regresó a Hebrón, Joab lo apartó en medio
de la puerta para hablar con él en voz baja, y lo golpeó allí en el
cuerpo, de modo que murió por la sangre de su hermano Asael. \footnote{\textbf{3:27}
  1Re 2,5; 2Sam 2,23}

\hypertarget{abner-hizo-duelo-por-david-y-fue-sepultado-con-honor-la-declaraciuxf3n-de-inocencia-de-david-graduaciuxf3n}{%
\subsection{Abner hizo duelo por David y fue sepultado con honor; La
declaración de inocencia de David;
Graduación}\label{abner-hizo-duelo-por-david-y-fue-sepultado-con-honor-la-declaraciuxf3n-de-inocencia-de-david-graduaciuxf3n}}

\bibleverse{28} Después, cuando David lo oyó, dijo: ``Yo y mi reino
quedamos libres de culpa ante el Señor para siempre por la sangre de
Abner, hijo de Ner. \bibleverse{29} Que caiga sobre la cabeza de Joab y
sobre toda la casa de su padre. Que no falte de la casa de Joab ninguno
que tenga una baja, o que sea leproso, o que se apoye en un bastón, o
que caiga por la espada, o que le falte el pan.'' \bibleverse{30}
Entonces Joab y su hermano Abisai mataron a Abner, porque éste había
matado a su hermano Asael en Gabaón en la batalla.

\bibleverse{31} David dijo a Joab y a todo el pueblo que estaba con él:
``Rasguen sus ropas, vístanse de saco y hagan duelo frente a Abner''. El
rey David siguió el féretro. \bibleverse{32} Enterraron a Abner en
Hebrón; el rey alzó la voz y lloró ante la tumba de Abner, y todo el
pueblo lloró. \footnote{\textbf{3:32} 1Sam 30,4} \bibleverse{33} El rey
se lamentó por Abner y dijo: ``¿Debe morir Abner como muere un tonto?
\bibleverse{34} Sus manos no fueron atadas, ni sus pies fueron puestos
en grilletes. Como un hombre cae ante los hijos de la iniquidad, así
caíste tú''. Todo el pueblo volvió a llorar por él. \bibleverse{35} Todo
el pueblo vino a exhortar a David a que comiera pan mientras fuera de
día; pero David juró diciendo: ``Que Dios me haga así, y más, si pruebo
el pan o cualquier otra cosa, hasta que se ponga el sol.''

\bibleverse{36} Todo el pueblo se dio por enterado, y les pareció bien,
pues todo lo que hacía el rey le parecía bien a todo el pueblo.
\bibleverse{37} Así que todo el pueblo y todo Israel comprendieron aquel
día que no era del rey matar a Abner hijo de Ner. \bibleverse{38} El rey
dijo a sus siervos: ``¿No sabéis que hoy ha caído un príncipe y un gran
hombre en Israel? \footnote{\textbf{3:38} 1Sam 26,15} \bibleverse{39}
Hoy soy débil, aunque he sido ungido rey. Estos hombres, los hijos de
Sarvia, son demasiado duros para mí. Que el Señor recompense al
malhechor según su maldad''.

\hypertarget{asesinato-de-isboseth-coronaciuxf3n-de-david-como-rey-de-todo-israel}{%
\subsection{Asesinato de Isboseth; Coronación de David como Rey de todo
Israel}\label{asesinato-de-isboseth-coronaciuxf3n-de-david-como-rey-de-todo-israel}}

\hypertarget{section-3}{%
\section{4}\label{section-3}}

\bibleverse{1} Cuando el hijo de Saúl se enteró de que Abner había
muerto en Hebrón, sus manos se debilitaron, y todos los israelitas se
preocuparon. \bibleverse{2} El hijo de Saúl tenía dos hombres que eran
capitanes de bandas de asalto. Uno se llamaba Baana y el otro Recab,
hijos de Rimón el beerotita, de los hijos de Benjamín (pues Beerot
también se considera parte de Benjamín; \bibleverse{3} y los beerotitas
huyeron a Gittaim, y han vivido allí como extranjeros hasta hoy).

\bibleverse{4} Jonatán, hijo de Saúl, tenía un hijo que era cojo de los
pies. Tenía cinco años cuando llegó la noticia de que Saúl y Jonatán
habían salido de Jezreel; y su nodriza lo recogió y huyó. Mientras se
apresuraba a huir, él se cayó y quedó cojo. Se llamaba Mefiboset.
\footnote{\textbf{4:4} 2Sam 9,3}

\bibleverse{5} Los hijos de Rimón el beerotita, Recab y Baana, salieron
y llegaron al filo del calor del día a la casa de Ishboshet cuando éste
descansaba al mediodía. \bibleverse{6} Entraron allí, en medio de la
casa, como si quisieran recoger trigo, y lo hirieron en el cuerpo; pero
Recab y su hermano Baana escaparon.

\hypertarget{david-castiga-a-los-asesinos-y-honra-al-muerto-isboseth}{%
\subsection{David castiga a los asesinos y honra al muerto
Isboseth}\label{david-castiga-a-los-asesinos-y-honra-al-muerto-isboseth}}

\bibleverse{7} Cuando entraron en la casa mientras él estaba acostado en
su cama, en su dormitorio, lo golpearon, lo mataron, lo decapitaron y
tomaron su cabeza, y se fueron por el camino del Arabá toda la noche.
\bibleverse{8} Llevaron la cabeza de Isboset a David, a Hebrón, y le
dijeron al rey: ``¡He aquí la cabeza de Isboset, hijo de Saúl, tu
enemigo, que buscaba tu vida! El Señor ha vengado hoy a mi señor el rey
de Saúl y de su descendencia.''

\bibleverse{9} David respondió a Recab y a su hermano Baana, hijos de
Rimón el beerothita, y les dijo: ``Vive Yahvé, que ha redimido mi alma
de toda adversidad, \bibleverse{10} cuando alguien me dijo: `He aquí que
Saúl ha muerto', pensando que traía buenas noticias, lo agarré y lo maté
en Siclag, que fue la recompensa que le di por sus noticias. \footnote{\textbf{4:10}
  2Sam 1,15} \bibleverse{11} ¿Cuánto más, si los malvados han matado a
un justo en su propia casa, en su lecho, no he de exigir ahora su sangre
de tu mano, y librar la tierra de ti?'' \bibleverse{12} David ordenó a
sus jóvenes, y los mataron, les cortaron las manos y los pies y los
colgaron junto al estanque de Hebrón. Pero tomaron la cabeza de Isboset
y la enterraron en la tumba de Abner en Hebrón.

\hypertarget{david-ungido-rey-por-todos-los-israelitas-en-hebruxf3n}{%
\subsection{David ungido rey por todos los israelitas en
Hebrón}\label{david-ungido-rey-por-todos-los-israelitas-en-hebruxf3n}}

\hypertarget{section-4}{%
\section{5}\label{section-4}}

\bibleverse{1} Entonces todas las tribus de Israel acudieron a David en
Hebrón y hablaron diciendo: ``He aquí que somos tu hueso y tu carne.
\footnote{\textbf{5:1} 2Sam 19,12} \bibleverse{2} En tiempos pasados,
cuando Saúl era rey sobre nosotros, eras tú quien conducía a Israel
hacia afuera y hacia adentro. Yahvé te dijo: `Tú serás pastor de mi
pueblo Israel, y serás príncipe sobre Israel'\,''. \footnote{\textbf{5:2}
  1Sam 13,14; 1Sam 25,30} \bibleverse{3} Así que todos los ancianos de
Israel vinieron al rey a Hebrón, y el rey David hizo un pacto con ellos
en Hebrón ante Yavé; y ungieron a David como rey de Israel. \footnote{\textbf{5:3}
  2Sam 2,4; 1Sam 16,13}

\bibleverse{4} David tenía treinta años cuando comenzó a reinar, y reinó
cuarenta años. \footnote{\textbf{5:4} 1Re 2,11; 1Cró 29,27}
\bibleverse{5} En Hebrón reinó sobre Judá siete años y seis meses, y en
Jerusalén reinó treinta y tres años sobre todo Israel y Judá.

\hypertarget{david-conquista-jerusaluxe9n-y-la-convierte-en-su-capital-y-su-residencia}{%
\subsection{David conquista Jerusalén y la convierte en su capital y su
residencia}\label{david-conquista-jerusaluxe9n-y-la-convierte-en-su-capital-y-su-residencia}}

\bibleverse{6} El rey y sus hombres se dirigieron a Jerusalén contra los
jebuseos, habitantes del país, que hablaron a David diciendo: ``Los
ciegos y los cojos te mantendrán fuera de aquí'', pensando que ``David
no puede entrar aquí''. \bibleverse{7} Sin embargo, David tomó la
fortaleza de Sión. Esta es la ciudad de David. \bibleverse{8} Aquel día
David dijo: ``El que golpee a los jebuseos, que suba a la rambla y
golpee a los cojos y a los ciegos, que son odiados por el alma de
David.'' Por eso dicen: ``Los ciegos y los cojos no pueden entrar en la
casa''.

\bibleverse{9} David vivió en la fortaleza y la llamó ciudad de David.
David construyó alrededor de Millo y hacia adentro. \bibleverse{10}
David crecía cada vez más, porque Yahvé, el Dios de los Ejércitos,
estaba con él. \footnote{\textbf{5:10} 2Sam 3,1}

\hypertarget{sus-construcciones-con-la-ayuda-de-hiram-de-tiro-aumentando-el-nuxfamero-de-sus-esposas-sus-hijos-nacidos-en-jerusaluxe9n}{%
\subsection{Sus construcciones (con la ayuda de Hiram de Tiro);
Aumentando el número de sus esposas; sus hijos nacidos en
Jerusalén}\label{sus-construcciones-con-la-ayuda-de-hiram-de-tiro-aumentando-el-nuxfamero-de-sus-esposas-sus-hijos-nacidos-en-jerusaluxe9n}}

\bibleverse{11} Hiram, rey de Tiro, envió mensajeros a David con cedros,
carpinteros y albañiles, y le construyeron una casa. \bibleverse{12}
David se dio cuenta de que Yavé lo había establecido como rey sobre
Israel y que había exaltado su reino por causa de su pueblo Israel.

\bibleverse{13} David tomó para sí más concubinas y esposas fuera de
Jerusalén, después de haber venido de Hebrón; y le nacieron más hijos e
hijas. \bibleverse{14} Estos son los nombres de los que le nacieron en
Jerusalén Shammua, Shobab, Natán, Salomón, \footnote{\textbf{5:14} Luc
  3,31; Mat 1,6} \bibleverse{15} Ibhar, Elishua, Nepheg, Japhia,
\bibleverse{16} Elishama, Eliada y Eliphelet.

\hypertarget{sus-dos-batallas-victoriosas-con-los-filisteos}{%
\subsection{Sus dos batallas victoriosas con los
filisteos}\label{sus-dos-batallas-victoriosas-con-los-filisteos}}

\bibleverse{17} Cuando los filisteos se enteraron de que habían ungido a
David como rey de Israel, todos los filisteos subieron a buscar a David,
pero éste se enteró y bajó a la fortaleza. \bibleverse{18} Los filisteos
habían llegado y se habían extendido en el valle de Refaim.
\bibleverse{19} David consultó al Señor, diciendo: ``¿Debo subir contra
los filisteos? ¿Los entregarás en mi mano?'' Yahvé dijo a David: ``Sube,
porque ciertamente entregaré a los filisteos en tu mano''. \footnote{\textbf{5:19}
  1Sam 30,8}

\bibleverse{20} David llegó a Baal Perazim, y allí los golpeó. Entonces
dijo: ``Yahvé ha quebrado a mis enemigos ante mí, como la brecha de las
aguas''. Por eso llamó a ese lugar Baal Perazim. \bibleverse{21} Dejaron
allí sus imágenes, y David y sus hombres se las llevaron.

\bibleverse{22} Los filisteos volvieron a subir y se extendieron por el
valle de Refaim. \bibleverse{23} Cuando David consultó al Señor, éste le
dijo: ``No subas. Da la vuelta por detrás de ellos y atácalos frente a
las moreras. \bibleverse{24} Cuando oigas el ruido de la marcha en las
copas de las moreras, entonces revuélvete, porque entonces Yahvé ha
salido delante de ti para atacar al ejército de los filisteos.''

\bibleverse{25} David lo hizo así, tal como se lo había ordenado Yahvé,
y atacó a los filisteos desde Geba hasta Gezer.

\hypertarget{traslado-del-arca-a-sion-en-jerusaluxe9n-fracaso-del-primer-intento}{%
\subsection{Traslado del arca a Sion en Jerusalén; Fracaso del primer
intento}\label{traslado-del-arca-a-sion-en-jerusaluxe9n-fracaso-del-primer-intento}}

\hypertarget{section-5}{%
\section{6}\label{section-5}}

\bibleverse{1} David volvió a reunir a todos los hombres elegidos de
Israel, treinta mil. \footnote{\textbf{6:1} 1Cró 13,1} \bibleverse{2}
David se levantó y fue con todo el pueblo que lo acompañaba desde Baale
Judá, para hacer subir desde allí el arca de Dios, que se llama con el
Nombre, el nombre de Yavé de los Ejércitos que se sienta encima de los
querubines. \footnote{\textbf{6:2} Jos 15,9; Éxod 25,22} \bibleverse{3}
Pusieron el arca de Dios en un carro nuevo y la sacaron de la casa de
Abinadab, que estaba en la colina; Uza y Ahio, hijos de Abinadab,
conducían el carro nuevo. \footnote{\textbf{6:3} 1Sam 7,1}
\bibleverse{4} Lo sacaron de la casa de Abinadab que estaba en la
colina, con el arca de Dios; y Ahio iba delante del arca. \bibleverse{5}
David y toda la casa de Israel tocaban delante de Yavé con toda clase de
instrumentos de madera de ciprés, con arpas, con instrumentos de cuerda,
con panderetas, con castañuelas y con címbalos.

\bibleverse{6} Cuando llegaron a la era de Nacón, Uza alcanzó el arca de
Dios y se aferró a ella, pues el ganado tropezó. \bibleverse{7} La ira
de Yavé ardió contra Uza, y Dios lo hirió allí por su error; y murió
allí junto al arca de Dios. \footnote{\textbf{6:7} Núm 4,15; 1Sam 6,19}
\bibleverse{8} David se disgustó porque Yavé había arremetido contra
Uza; y llamó a ese lugar Pérez Uza hasta el día de hoy. \bibleverse{9}
David tuvo miedo de Yavé aquel día, y dijo: ``¿Cómo podría venir a mí el
arca de Yavé?''. \bibleverse{10} Así que David no quiso trasladar el
arca de Yavé para que estuviera con él en la ciudad de David, sino que
la llevó a un lado, a la casa de Obed-Edom el geteo. \bibleverse{11} El
arca de Yahvé permaneció tres meses en la casa de Obed-Edom el geteo, y
Yahvé bendijo a Obed-Edom y a toda su casa.

\hypertarget{traslado-solemne-del-arca-a-jerusaluxe9n-fiesta-del-sacrificio-y-acciuxf3n-de-gracias-del-pueblo}{%
\subsection{Traslado solemne del arca a Jerusalén; Fiesta del sacrificio
y acción de gracias del
pueblo}\label{traslado-solemne-del-arca-a-jerusaluxe9n-fiesta-del-sacrificio-y-acciuxf3n-de-gracias-del-pueblo}}

\bibleverse{12} Se le dijo al rey David: ``El Señor ha bendecido la casa
de Obed-Edom y todo lo que le pertenece, a causa del arca de Dios.''
Entonces David fue y subió con alegría el arca de Dios desde la casa de
Obed-Edom a la ciudad de David. \footnote{\textbf{6:12} 1Cró 15,1}
\bibleverse{13} Cuando los que llevaban el arca de Yavé habían recorrido
seis pasos, sacrificó un buey y un ternero cebado. \footnote{\textbf{6:13}
  1Re 8,5} \bibleverse{14} David danzó ante Yavé con todas sus fuerzas,
y se vistió con un efod de lino. \bibleverse{15} Entonces David y toda
la casa de Israel subieron el arca de Yavé con gritos y con el sonido de
la trompeta.

\bibleverse{16} Cuando el arca de Yahvé llegó a la ciudad de David,
Mical, hija de Saúl, se asomó a la ventana y vio al rey David saltando y
danzando ante Yahvé; y lo despreció en su corazón. \bibleverse{17}
Hicieron entrar el arca de Yavé y la colocaron en su lugar, en medio de
la tienda que David había montado para ella; y David ofreció holocaustos
y ofrendas de paz ante Yavé. \footnote{\textbf{6:17} 1Cró 16,1}
\bibleverse{18} Cuando David terminó de ofrecer los holocaustos y las
ofrendas de paz, bendijo al pueblo en nombre de Yavé de los Ejércitos.
\footnote{\textbf{6:18} 1Re 8,55} \bibleverse{19} Dio a todo el pueblo,
de entre toda la multitud de Israel, tanto a los hombres como a las
mujeres, a cada uno una porción de pan, dátiles y pasas. Y todo el
pueblo se fue, cada uno a su casa.

\hypertarget{la-noble-conducta-de-david-y-su-humilde-declaraciuxf3n-contra-mical}{%
\subsection{La noble conducta de David y su humilde declaración contra
Mical}\label{la-noble-conducta-de-david-y-su-humilde-declaraciuxf3n-contra-mical}}

\bibleverse{20} Entonces David volvió para bendecir a su familia. Mical,
la hija de Saúl, salió al encuentro de David y dijo: ``¡Qué glorioso ha
sido hoy el rey de Israel, que se ha descubierto a los ojos de las
criadas de sus siervos, como se descubre descaradamente uno de los
vanidosos!''

\bibleverse{21} David dijo a Mical: ``Fue ante Yahvé, que me eligió por
encima de tu padre y de toda su casa, para nombrarme príncipe del pueblo
de Yahvé, de Israel. Por eso celebraré ante Yahvé. \footnote{\textbf{6:21}
  2Sam 5,2} \bibleverse{22} Todavía seré más indigno que esto, y no
tendré ningún valor a mis ojos. Pero las doncellas de las que has
hablado me honrarán''.

\bibleverse{23} Mical, hija de Saúl, no tuvo hijos hasta el día de su
muerte.

\hypertarget{natuxe1n-aprueba-el-plan-de-david-para-construir-el-templo}{%
\subsection{Natán aprueba el plan de David para construir el
templo}\label{natuxe1n-aprueba-el-plan-de-david-para-construir-el-templo}}

\hypertarget{section-6}{%
\section{7}\label{section-6}}

\bibleverse{1} Cuando el rey vivía en su casa, y Yahvé le había dado
descanso de todos sus enemigos alrededor, \bibleverse{2} el rey dijo al
profeta Natán: ``Mira ahora, yo habito en una casa de cedro, pero el
arca de Dios habita dentro de las cortinas.'' \footnote{\textbf{7:2} Sal
  132,1}

\bibleverse{3} Natán dijo al rey: ``Ve, haz todo lo que está en tu
corazón, porque Yahvé está contigo''.

\hypertarget{rechazo-de-dios-del-plan-el-discurso-profuxe9tico-de-nathan-el-templo-seruxe1-construido-por-el-hijo-de-david}{%
\subsection{Rechazo de Dios del plan; El discurso profético de Nathan;
el templo será construido por el hijo de
David}\label{rechazo-de-dios-del-plan-el-discurso-profuxe9tico-de-nathan-el-templo-seruxe1-construido-por-el-hijo-de-david}}

\bibleverse{4} Esa misma noche, la palabra de Yahvé llegó a Natán,
diciendo: \bibleverse{5} ``Ve y dile a mi siervo David: ``Dice Yahvé:
``¿Debes construirme una casa para que habite en ella? \footnote{\textbf{7:5}
  1Cró 22,8; 1Re 5,3} \bibleverse{6} Porque no he vivido en una casa
desde el día en que saqué a los hijos de Israel de Egipto, hasta hoy,
sino que me he movido en una tienda y en un tabernáculo. \footnote{\textbf{7:6}
  1Re 8,16; 1Re 8,27; Is 66,1} \bibleverse{7} En todos los lugares por
donde he andado con todos los hijos de Israel, ¿he dicho alguna palabra
a alguno de las tribus de Israel a quien mandé que fuera pastor de mi
pueblo Israel, diciendo: ``¿Por qué no me habéis construido una casa de
cedro?''\,' \bibleverse{8} Ahora, pues, dile esto a mi siervo David:
`Dice el Señor de los Ejércitos: ``Te tomé del corral de las ovejas, de
seguir a las ovejas, para ser príncipe de mi pueblo, de Israel.
\footnote{\textbf{7:8} 1Sam 16,11-13} \bibleverse{9} Yo he estado
contigo dondequiera que hayas ido, y he eliminado a todos tus enemigos
de delante de ti. Te haré un nombre grande, como el nombre de los
grandes que hay en la tierra. \bibleverse{10} Designaré un lugar para mi
pueblo Israel, y lo plantaré, para que habite en su propio lugar y no se
mueva más. Los hijos de la maldad no los afligirán más, como al
principio, \bibleverse{11} y como desde el día en que ordené que hubiera
jueces sobre mi pueblo Israel. Les haré descansar de todos sus enemigos.
Además, Yahvé te dice que Yahvé te hará una casa. \bibleverse{12} Cuando
se cumplan tus días y duermas con tus padres, levantaré después de ti a
tu descendiente, que saldrá de tu cuerpo, y estableceré su reino.
\footnote{\textbf{7:12} 1Re 8,20; Is 9,7} \bibleverse{13} Él construirá
una casa a mi nombre, y yo estableceré el trono de su reino para
siempre. \footnote{\textbf{7:13} 1Re 5,5; 1Re 6,12; Sal 89,3-4}

\hypertarget{la-gran-proclamaciuxf3n-de-salvaciuxf3n-de-dios-a-david-con-respecto-a-la-eternidad-de-su-casa}{%
\subsection{La gran proclamación de salvación de Dios a David con
respecto a la eternidad de su
casa}\label{la-gran-proclamaciuxf3n-de-salvaciuxf3n-de-dios-a-david-con-respecto-a-la-eternidad-de-su-casa}}

\bibleverse{14} Yo seré su padre, y él será mi hijo. Si comete una
iniquidad, lo castigaré con vara de hombres y con azotes de hijos de
hombres; \footnote{\textbf{7:14} Sal 89,26; Heb 1,5; Luc 1,32}
\bibleverse{15} pero mi bondad no se apartará de él, como la aparté de
Saúl, a quien expulsé antes de ti. \footnote{\textbf{7:15} 1Sam 15,23;
  1Sam 15,26} \bibleverse{16} Tu casa y tu reino serán asegurados para
siempre delante de ti. Tu trono será establecido para siempre''\,''.
\footnote{\textbf{7:16} Sal 72,17; Is 55,3}

\hypertarget{acciuxf3n-de-gracias-y-suxfaplica-de-david}{%
\subsection{Acción de gracias y súplica de
David}\label{acciuxf3n-de-gracias-y-suxfaplica-de-david}}

\bibleverse{17} Natán le dijo a David todas estas palabras y según toda
esta visión.

\bibleverse{18} Entonces el rey David entró y se sentó delante de Yahvé,
y dijo: ``¿Quién soy yo, Señor Yahvé, y cuál es mi casa, para que me
hayas traído hasta aquí? \footnote{\textbf{7:18} Gén 32,10}
\bibleverse{19} Esto era aún poco a tus ojos, Señor Yahvé, pero también
has hablado de la casa de tu siervo por mucho tiempo; y esto entre los
hombres, Señor Yahvé. \bibleverse{20} ¿Qué más puede decirte David?
Porque tú conoces a tu siervo, Señor Yahvé. \bibleverse{21} Por tu
palabra, y según tu propio corazón, has obrado toda esta grandeza, para
que tu siervo la conozca. \bibleverse{22} Por eso eres grande, Yahvé
Dios. Porque no hay nadie como tú, ni hay otro Dios fuera de ti, según
todo lo que hemos oído con nuestros oídos. \bibleverse{23} ¿Qué nación
hay en la tierra que sea como tu pueblo, como Israel, al que Dios fue a
redimir para sí como pueblo, y a hacerse un nombre, y a hacer cosas
grandes para ti, y cosas impresionantes para tu tierra, ante tu pueblo,
al que redimiste para ti de Egipto, de las naciones y de sus dioses?
\footnote{\textbf{7:23} Deut 4,7} \bibleverse{24} Estableciste para ti a
tu pueblo Israel para que fuera tu pueblo para siempre; y tú, Yahvé, te
convertiste en su Dios.

\bibleverse{25} ``Ahora bien, Yahvé Dios, la palabra que has pronunciado
sobre tu siervo y sobre su casa, confírmala para siempre y haz lo que
has dicho. \bibleverse{26} Que tu nombre sea engrandecido para siempre,
diciendo: `El Señor de los Ejércitos es Dios sobre Israel; y la casa de
tu siervo David será establecida ante ti'. \bibleverse{27} Porque tú,
Señor de los Ejércitos, Dios de Israel, has revelado a tu siervo,
diciendo: `Yo te construiré una casa'. Por eso tu siervo ha encontrado
en su corazón el rezarte esta oración. \footnote{\textbf{7:27} Is 50,5}

\bibleverse{28} ``Ahora bien, Señor Yahvé, tú eres Dios, y tus palabras
son verdaderas, y has prometido este bien a tu siervo. \footnote{\textbf{7:28}
  1Re 8,26} \bibleverse{29} Ahora, pues, te conviene bendecir la casa de
tu siervo, para que permanezca para siempre ante ti; porque tú, Señor
Yahvé, lo has dicho. Que la casa de tu siervo sea bendecida para siempre
con tu bendición''.

\hypertarget{las-victorias-de-david-sobre-los-filisteos-moabitas-y-sirios}{%
\subsection{Las victorias de David sobre los filisteos, moabitas y
sirios}\label{las-victorias-de-david-sobre-los-filisteos-moabitas-y-sirios}}

\hypertarget{section-7}{%
\section{8}\label{section-7}}

\bibleverse{1} Después de esto, David golpeó a los filisteos y los
sometió; y David tomó el freno de la ciudad madre de la mano de los
filisteos. \bibleverse{2} Derrotó a Moab, y los midió con el cordel,
haciéndolos acostar en el suelo; y midió dos cordeles para darles
muerte, y un cordel completo para mantenerlos vivos. Los moabitas se
convirtieron en siervos de David, y trajeron tributo.

\bibleverse{3} David también hirió a Hadadézer hijo de Rehob, rey de
Soba, cuando iba a recuperar su dominio en el río. \bibleverse{4} David
le quitó mil setecientos jinetes y veinte mil hombres de a pie. David
ató los caballos de los carros, pero reservó los suficientes para cien
carros. \footnote{\textbf{8:4} Jos 11,9} \bibleverse{5} Cuando los
sirios de Damasco vinieron a ayudar a Hadadézer, rey de Soba, David
hirió a veintidós mil hombres de los sirios. \bibleverse{6} Entonces
David puso guarniciones en Siria de Damasco, y los sirios se
convirtieron en siervos de David y le trajeron tributo. El Señor le dio
la victoria a David dondequiera que fuera.

\hypertarget{el-botuxedn-y-sus-usos-felicitaciones-del-rey-thoi}{%
\subsection{El botín y sus usos; Felicitaciones del Rey
Thoi}\label{el-botuxedn-y-sus-usos-felicitaciones-del-rey-thoi}}

\bibleverse{7} David tomó los escudos de oro que tenían los siervos de
Hadadézer y los llevó a Jerusalén. \bibleverse{8} De Betah y de
Berothai, ciudades de Hadadzer, el rey David tomó una gran cantidad de
bronce.

\bibleverse{9} Cuando Toi, rey de Hamat, se enteró de que David había
golpeado a todo el ejército de Hadadézer, \bibleverse{10} entonces Toi
envió a Joram, su hijo, a saludar al rey David y a bendecirlo, porque
había luchado contra Hadadézer y lo había golpeado; pues Hadadézer tenía
guerras con Toi. Joram trajo consigo vasos de plata, vasos de oro y
vasos de bronce. \bibleverse{11} El rey David también los dedicó a Yavé,
con la plata y el oro que dedicó de todas las naciones que sometió:
\bibleverse{12} de Siria, de Moab, de los hijos de Amón, de los
filisteos, de Amalec y del botín de Hadadézer, hijo de Rehob, rey de
Soba.

\hypertarget{derrota-y-subyugaciuxf3n-de-los-edomitas}{%
\subsection{Derrota y subyugación de los
edomitas}\label{derrota-y-subyugaciuxf3n-de-los-edomitas}}

\bibleverse{13} David se ganó una reputación cuando volvió de abatir a
dieciocho mil hombres de los sirios en el Valle de la Sal. \footnote{\textbf{8:13}
  Sal 60,1} \bibleverse{14} Puso guarniciones en Edom. En todo Edom puso
guarniciones, y todos los edomitas se convirtieron en siervos de David.
El Señor le dio la victoria a David dondequiera que fuera. \footnote{\textbf{8:14}
  Gén 27,40}

\hypertarget{directorio-de-los-principales-oficiales-de-david}{%
\subsection{Directorio de los principales oficiales de
David}\label{directorio-de-los-principales-oficiales-de-david}}

\bibleverse{15} David reinó sobre todo Israel; y David hizo justicia y
rectitud a todo su pueblo. \bibleverse{16} Joab, hijo de Sarvia, estaba
al frente del ejército; Josafat, hijo de Ahilud, era secretario;
\footnote{\textbf{8:16} 2Sam 20,23-26} \bibleverse{17} Sadoc, hijo de
Ajitub, y Ajimelec, hijo de Abiatar, eran sacerdotes; Seraías era
escriba; \bibleverse{18} Benaía, hijo de Joiada, estaba al frente de los
cereteos y de los peleteos; y los hijos de David eran ministros
principales. \footnote{\textbf{8:18} 2Sam 15,18; 1Sam 30,14}

\hypertarget{la-generosidad-de-david-hacia-el-hijo-de-jonatuxe1n-mefiboset}{%
\subsection{La generosidad de David hacia el hijo de Jonatán,
Mefiboset}\label{la-generosidad-de-david-hacia-el-hijo-de-jonatuxe1n-mefiboset}}

\hypertarget{section-8}{%
\section{9}\label{section-8}}

\bibleverse{1} David dijo: ``¿Queda todavía alguien de la casa de Saúl,
para que le haga un favor en favor de Jonatán?'' \bibleverse{2} Había de
la casa de Saúl un siervo que se llamaba Siba, y lo llamaron a David, y
el rey le dijo: ``¿Eres tú Siba?'' Dijo: ``Soy tu siervo''. \footnote{\textbf{9:2}
  2Sam 16,1}

\bibleverse{3} El rey dijo: ``¿No hay todavía nadie de la casa de Saúl,
para que le muestre la bondad de Dios?'' Siba dijo al rey: ``Jonatán aún
tiene un hijo, que es cojo de los pies''. \footnote{\textbf{9:3} 2Sam
  4,4}

\bibleverse{4} El rey le dijo: ``¿Dónde está?''. Siba dijo al rey: ``He
aquí que está en casa de Maquir, hijo de Ammiel, en Lo Debar''.
\footnote{\textbf{9:4} 2Sam 17,27}

\hypertarget{las-magnuxe1nimas-disposiciones-de-david-con-respecto-a-mefiboset}{%
\subsection{Las magnánimas disposiciones de David con respecto a
Mefiboset}\label{las-magnuxe1nimas-disposiciones-de-david-con-respecto-a-mefiboset}}

\bibleverse{5} Entonces el rey David envió y lo sacó de la casa de
Maquir, hijo de Amiel, de Lo Debar. \bibleverse{6} Mefiboset, hijo de
Jonatán, hijo de Saúl, se acercó a David, se postró sobre su rostro y le
mostró respeto. David dijo: ``¿Mefiboset?'' Él respondió: ``He aquí tu
siervo''.

\bibleverse{7} David le dijo: ``No temas, porque seguramente te mostraré
bondad por amor a Jonatán, tu padre, y te devolveré toda la tierra de
Saúl, tu padre. Comerás continuamente el pan en mi mesa''.

\bibleverse{8} Se inclinó y dijo: ``¿Qué es tu siervo, para que mires a
un perro muerto como yo?''. \footnote{\textbf{9:8} 1Sam 24,14}

\bibleverse{9} Entonces el rey llamó a Siba, siervo de Saúl, y le dijo:
``Todo lo que era de Saúl y de toda su casa se lo he dado al hijo de tu
amo. \bibleverse{10} Labra la tierra para él: tú, tus hijos y tus
siervos. Trae la cosecha, para que el hijo de tu amo tenga pan que
comer; pero Mefiboset, el hijo de tu amo, siempre comerá pan en mi
mesa.'' Y Siba tenía quince hijos y veinte siervos. \bibleverse{11}
Entonces Siba dijo al rey: ``Según todo lo que mi señor el rey ordena a
su siervo, así lo hará tu siervo''. Así que Mefiboset comía en la mesa
del rey como uno de sus hijos. \footnote{\textbf{9:11} 2Sam 19,28}
\bibleverse{12} Mefiboset tenía un hijo pequeño que se llamaba Mica.
Todos los que vivían en la casa de Siba eran siervos de Mefiboset.
\bibleverse{13} Así que Mefiboset vivía en Jerusalén, pues comía
continuamente en la mesa del rey. Era cojo de ambos pies. \footnote{\textbf{9:13}
  2Sam 9,3}

\hypertarget{el-vergonzoso-crimen-de-los-amonitas-contra-el-mensajero-de-david}{%
\subsection{El vergonzoso crimen de los amonitas contra el mensajero de
David}\label{el-vergonzoso-crimen-de-los-amonitas-contra-el-mensajero-de-david}}

\hypertarget{section-9}{%
\section{10}\label{section-9}}

\bibleverse{1} Después de esto, el rey de los hijos de Amón murió, y su
hijo Hanún reinó en su lugar. \bibleverse{2} David dijo: ``Me mostraré
bondadoso con Hanún, hijo de Nahas, como su padre se mostró bondadoso
conmigo''. Así que David envió por medio de sus siervos a consolarlo en
lo que respecta a su padre. Los siervos de David llegaron a la tierra de
los hijos de Amón.

\bibleverse{3} Pero los príncipes de los hijos de Amón dijeron a Hanún,
su señor: ``¿Piensas que David honra a tu padre, pues te ha enviado
consoladores? ¿Acaso no ha enviado David a sus siervos para que
registren la ciudad, la espíen y la derriben?''

\bibleverse{4} Entonces Hanún tomó a los siervos de David, les afeitó la
mitad de la barba y les cortó los vestidos por la mitad, hasta las
nalgas, y los despidió. \bibleverse{5} Cuando le contaron esto a David,
éste envió a recibirlos, pues los hombres estaban muy avergonzados. El
rey les dijo: ``Esperen en Jericó hasta que les crezca la barba, y luego
vuelvan''.

\hypertarget{comienzo-de-la-guerra-primeros-trabajos-ganados}{%
\subsection{Comienzo de la guerra; primeros trabajos
ganados}\label{comienzo-de-la-guerra-primeros-trabajos-ganados}}

\bibleverse{6} Cuando los hijos de Amón vieron que se habían vuelto
odiosos para David, los hijos de Amón enviaron y contrataron a los
sirios de Bet Rehob y a los sirios de Soba, veinte mil hombres de a pie,
y al rey de Maaca con mil hombres, y a los hombres de Tob doce mil
hombres. \bibleverse{7} Cuando David se enteró, envió a Joab y a todo el
ejército de valientes. \bibleverse{8} Los hijos de Amón salieron y
pusieron la batalla en orden a la entrada de la puerta. Los sirios de
Soba y de Rehob y los hombres de Tob y de Maaca estaban solos en el
campo. \bibleverse{9} Cuando Joab vio que la batalla estaba en su contra
por delante y por detrás, eligió a todos los hombres selectos de Israel
y los puso en orden de batalla contra los sirios. \bibleverse{10} El
resto del pueblo lo puso en manos de Abisai, su hermano, y lo alineó
contra los amonitas. \bibleverse{11} Dijo: ``Si los sirios son demasiado
fuertes para mí, tú me ayudarás; pero si los hijos de Amón son demasiado
fuertes para ti, yo iré a ayudarte. \bibleverse{12} Sé valiente y seamos
fuertes por nuestro pueblo y por las ciudades de nuestro Dios; y que
Yahvé haga lo que le parezca bien.'' \bibleverse{13} Así que Joab y la
gente que estaba con él se acercaron a la batalla contra los sirios, y
huyeron ante él. \bibleverse{14} Cuando los hijos de Amón vieron que los
sirios habían huido, también ellos huyeron ante Abisai y entraron en la
ciudad. Entonces Joab regresó de los hijos de Amón y llegó a Jerusalén.

\hypertarget{el-mismo-david-en-el-campo-su-victoria-sobre-los-sirios-aliados-con-los-amonitas}{%
\subsection{El mismo David en el campo; su victoria sobre los sirios
aliados con los
amonitas}\label{el-mismo-david-en-el-campo-su-victoria-sobre-los-sirios-aliados-con-los-amonitas}}

\bibleverse{15} Cuando los sirios vieron que habían sido derrotados por
Israel, se reunieron. \bibleverse{16} Hadadzer envió y sacó a los sirios
que estaban al otro lado del río; y llegaron a Helam, con Sobac, el
capitán del ejército de Hadadzer, a la cabeza. \bibleverse{17} David fue
informado de esto, y reunió a todo Israel, pasó el Jordán y llegó a
Helam. Los sirios se pusieron en guardia contra David y lucharon contra
él. \bibleverse{18} Los sirios huyeron ante Israel, y David mató a
setecientos aurigas de los sirios y a cuarenta mil jinetes, e hirió a
Sobac, jefe de su ejército, que murió allí. \bibleverse{19} Cuando todos
los reyes que estaban al servicio de Hadadézer vieron que habían sido
derrotados ante Israel, hicieron la paz con Israel y les sirvieron.
Entonces los sirios tuvieron miedo de seguir ayudando a los hijos de
Amón.

\hypertarget{el-adulterio-de-david-con-betsabuxe9}{%
\subsection{El adulterio de David con
Betsabé}\label{el-adulterio-de-david-con-betsabuxe9}}

\hypertarget{section-10}{%
\section{11}\label{section-10}}

\bibleverse{1} A la vuelta del año, en el tiempo en que salen los reyes,
David envió con él a Joab y a sus siervos, y a todo Israel; y
destruyeron a los hijos de Amón y sitiaron Rabá. Pero David se quedó en
Jerusalén. \footnote{\textbf{11:1} 1Cró 20,1} \bibleverse{2} Al
anochecer, David se levantó de su cama y se paseó por el tejado de la
casa real. Desde el tejado vio a una mujer bañándose, y la mujer era muy
hermosa de ver. \footnote{\textbf{11:2} Mat 5,28-29} \bibleverse{3}
David envió a preguntar por la mujer. Le dijo: ``¿No es ésta Betsabé, la
hija de Eliam, esposa de Urías el hitita?''. \footnote{\textbf{11:3}
  2Sam 23,29}

\bibleverse{4} David envió mensajeros, y la tomó; ella entró a él, y él
se acostó con ella (pues estaba purificada de su impureza); y ella
volvió a su casa. \footnote{\textbf{11:4} Lev 15,18} \bibleverse{5} La
mujer concibió, y envió a avisar a David, diciendo: ``Estoy
embarazada''.

\hypertarget{el-comportamiento-ejemplar-de-uria-durante-su-estancia-en-el-palacio-de-david}{%
\subsection{El comportamiento ejemplar de Uria durante su estancia en el
Palacio de
David}\label{el-comportamiento-ejemplar-de-uria-durante-su-estancia-en-el-palacio-de-david}}

\bibleverse{6} David envió a Joab: ``Envíame a Urías el hitita''. Joab
envió a Urías a David. \bibleverse{7} Cuando Urías llegó a él, David le
preguntó cómo le había ido a Joab, cómo le había ido al pueblo y cómo
había prosperado la guerra. \bibleverse{8} David le dijo a Urías: ``Baja
a tu casa y lávate los pies''. Urías salió de la casa del rey, y se
envió tras él un regalo del rey. \bibleverse{9} Pero Urías durmió a la
puerta de la casa del rey con todos los servidores de su señor, y no
bajó a su casa. \bibleverse{10} Cuando se lo contaron a David, diciendo:
``Urías no bajó a su casa'', David le dijo a Urías: ``¿No vienes de
viaje? ¿Por qué no has bajado a tu casa?''

\bibleverse{11} Urías dijo a David: ``El arca, Israel y Judá están en
tiendas, y mi señor Joab y los servidores de mi señor están acampados en
el campo. ¿Debo, pues, entrar en mi casa para comer y beber, y acostarme
con mi mujer? Mientras vivas tú y tu alma, no haré esto''. \footnote{\textbf{11:11}
  1Sam 4,4}

\bibleverse{12} David le dijo a Urías: ``Quédate aquí también hoy, y
mañana te dejaré partir''. Así que Urías se quedó en Jerusalén ese día y
el siguiente. \bibleverse{13} Cuando David lo llamó, comió y bebió
delante de él, y lo embriagó. Al anochecer, salió a acostarse en su cama
con los servidores de su señor, pero no bajó a su casa.

\hypertarget{la-carta-de-urias-la-muerte-de-uruxedas-el-mensaje-de-joab-a-david-aviso-del-rey}{%
\subsection{La carta de Urias; La muerte de Urías; El mensaje de Joab a
David; Aviso del
rey}\label{la-carta-de-urias-la-muerte-de-uruxedas-el-mensaje-de-joab-a-david-aviso-del-rey}}

\bibleverse{14} Por la mañana, David escribió una carta a Joab y la
envió por mano de Urías. \bibleverse{15} En la carta le decía: ``Envía a
Urías a la vanguardia de la batalla más caliente, y retírate de él, para
que sea golpeado y muera''.

\bibleverse{16} Cuando Joab vigilaba la ciudad, destinó a Urías al lugar
donde sabía que había hombres valientes. \bibleverse{17} Los hombres de
la ciudad salieron y lucharon con Joab. Algunos de ellos cayeron,
incluso de los siervos de David, y también murió Urías el hitita.
\bibleverse{18} Entonces Joab envió a contarle a David todo lo
concerniente a la guerra; \bibleverse{19} y le ordenó al mensajero que
dijera: ``Cuando termines de contarle al rey todo lo concerniente a la
guerra, \bibleverse{20} sucederá que, si se levanta la ira del rey y te
pregunta: `¿Por qué te acercaste tanto a la ciudad para pelear? ¿No
sabías que iban a disparar desde la muralla? \bibleverse{21} ¿Quién
golpeó a Abimelec, hijo de Jerubbeshet? ¿No le arrojó una mujer una
piedra de molino desde el muro, para que muriera en Tebas? ¿Por qué te
acercaste tanto a la muralla?'' Entonces dirás: ``Tu siervo Urías el
hitita también ha muerto''. \footnote{\textbf{11:21} Jue 9,53}

\bibleverse{22} El mensajero fue, y vino y le mostró a David todo lo que
Joab le había enviado. \bibleverse{23} El mensajero dijo a David: ``Los
hombres se impusieron a nosotros, y salieron al campo; y estuvimos sobre
ellos hasta la entrada de la puerta. \bibleverse{24} Los tiradores
dispararon contra tus siervos desde el muro; y algunos de los siervos
del rey han muerto, y tu siervo Urías el hitita también ha muerto.''

\bibleverse{25} Entonces David dijo al mensajero: ``Dile a Joab: `No
dejes que esto te disguste, pues la espada devora a uno como a otro. Haz
que tu batalla sea más fuerte contra la ciudad, y derríbala'. Anímalo''.

\hypertarget{el-duelo-de-betsabuxe9-por-su-marido-su-matrimonio-con-david}{%
\subsection{El duelo de Betsabé por su marido; su matrimonio con
David}\label{el-duelo-de-betsabuxe9-por-su-marido-su-matrimonio-con-david}}

\bibleverse{26} Cuando la mujer de Urías se enteró de que su marido
había muerto, hizo duelo por su marido. \bibleverse{27} Cuando pasó el
luto, David la envió y la llevó a su casa, y ella se convirtió en su
esposa y le dio un hijo. Pero lo que David había hecho desagradó a
Yahvé. \footnote{\textbf{11:27} Éxod 20,13-14}

\hypertarget{el-discurso-de-nathan-y-el-anuncio-de-la-perdiciuxf3n-la-confesiuxf3n-de-culpa-y-arrepentimiento-de-david}{%
\subsection{El discurso de Nathan y el anuncio de la perdición; La
confesión de culpa y arrepentimiento de
David}\label{el-discurso-de-nathan-y-el-anuncio-de-la-perdiciuxf3n-la-confesiuxf3n-de-culpa-y-arrepentimiento-de-david}}

\hypertarget{section-11}{%
\section{12}\label{section-11}}

\bibleverse{1} Yahvé envió a Natán a David. Se acercó a él y le dijo:
``Había dos hombres en una ciudad: uno rico y otro pobre. \bibleverse{2}
El rico tenía muchos rebaños y manadas, \bibleverse{3} pero el pobre no
tenía nada, excepto una ovejita que había comprado y criado. Creció
junto a él y a sus hijos. Comía de su comida, bebía de su copa, se
acostaba en su seno y era como una hija para él. \bibleverse{4} Llegó un
viajero al hombre rico, y éste no quiso tomar de su propio rebaño y de
su propia manada para preparar al caminante que había venido a él, sino
que tomó el cordero del pobre y lo preparó para el hombre que había
venido a él.''

\bibleverse{5} La ira de David se encendió contra el hombre, y dijo a
Natán: ``¡Vive Yahvé, que el hombre que ha hecho esto merece morir!
\bibleverse{6} ¡Debe restituir el cordero cuatro veces, porque ha hecho
esto y porque no ha tenido piedad!'' \footnote{\textbf{12:6} Éxod 22,1}

\bibleverse{7} Natán le dijo a David: ``¡Tú eres el hombre! Esto es lo
que dice Yahvé, el Dios de Israel: `Yo te ungí como rey de Israel, y te
libré de la mano de Saúl. \footnote{\textbf{12:7} 1Re 20,40}
\bibleverse{8} Te di la casa de tu amo y las mujeres de tu amo en tu
seno, y te di la casa de Israel y de Judá; y si eso hubiera sido poco,
te habría añadido muchas cosas más. \bibleverse{9} ¿Por qué has
despreciado la palabra de Yahvé, para hacer lo que es malo a sus ojos?
Has herido con la espada a Urías el hitita, has tomado a su mujer para
que sea tu esposa, y lo has matado con la espada de los hijos de Amón.
\footnote{\textbf{12:9} 2Sam 11,1; 1Re 15,5} \bibleverse{10} Ahora,
pues, la espada nunca se apartará de tu casa, porque me has despreciado
y has tomado a la mujer de Urías el hitita como esposa.' \footnote{\textbf{12:10}
  2Sam 13,28-29; 2Sam 18,14; 2Re 25,7}

\bibleverse{11} ``Esto es lo que dice el Señor: `He aquí que yo suscito
contra ti el mal de tu propia casa; y tomaré tus mujeres ante tus ojos y
se las daré a tu prójimo, y él se acostará con tus mujeres a la vista de
este sol. \bibleverse{12} Porque tú hiciste esto en secreto, pero yo
haré esto delante de todo Israel y delante del sol'\,''.

\bibleverse{13} David dijo a Natán: ``He pecado contra Yahvé''. Natán le
dijo a David: ``También Yahvé ha quitado tu pecado. No morirás.
\footnote{\textbf{12:13} 2Sam 24,10; Sal 51,1} \bibleverse{14} Sin
embargo, como con esta acción has dado gran ocasión a los enemigos de
Yavé para blasfemar, también el niño que te nazca morirá.'' \footnote{\textbf{12:14}
  2Sam 11,27}

\hypertarget{enfermedad-y-muerte-del-niuxf1o-betsabuxe9-el-dolor-y-la-renovada-valentuxeda-de-david-nacimiento-y-educaciuxf3n-de-salomuxf3n}{%
\subsection{Enfermedad y muerte del niño Betsabé; El dolor y la renovada
valentía de David; Nacimiento y educación de
Salomón}\label{enfermedad-y-muerte-del-niuxf1o-betsabuxe9-el-dolor-y-la-renovada-valentuxeda-de-david-nacimiento-y-educaciuxf3n-de-salomuxf3n}}

\bibleverse{15} Entonces Natán se fue a su casa. El Señor golpeó al niño
que la mujer de Urías le dio a David, y éste quedó muy enfermo.
\bibleverse{16} David, pues, rogó a Dios por el niño; y David ayunó, y
entró y se acostó toda la noche en el suelo. \bibleverse{17} Los
ancianos de su casa se levantaron junto a él para levantarlo de la
tierra; pero él no quiso, y no comió pan con ellos. \bibleverse{18} Al
séptimo día, el niño murió. Los siervos de David tuvieron miedo de
decirle que el niño había muerto, pues dijeron: ``He aquí que, mientras
el niño vivía, le hablamos y no escuchó nuestra voz. ¿Cómo se va a
perjudicar entonces si le decimos que el niño ha muerto?''

\bibleverse{19} Pero cuando David vio que sus sirvientes cuchicheaban
juntos, se dio cuenta de que el niño estaba muerto; y David dijo a sus
sirvientes: ``¿Ha muerto el niño?'' Dijeron: ``Está muerto''.

\bibleverse{20} Entonces David se levantó de la tierra, se lavó y se
ungió, y se cambió de ropa; entró en la casa de Yahvé y adoró. Luego
llegó a su casa; y cuando pidió, le pusieron pan delante y comió.
\bibleverse{21} Entonces sus servidores le dijeron: ``¿Qué es lo que has
hecho? Ayunaste y lloraste por el niño mientras vivía, pero cuando el
niño murió, te levantaste y comiste pan.''

\bibleverse{22} Dijo: ``Mientras el niño vivía, yo ayunaba y lloraba,
porque decía: ``¿Quién sabe si Yahvé no tendrá piedad de mí para que el
niño viva?'' \bibleverse{23} Pero ahora ha muerto. ¿Por qué he de
ayunar? ¿Puedo hacer que vuelva a nacer? Iré hacia él, pero no volverá a
mí''.

\bibleverse{24} David consoló a su mujer Betsabé, se acercó a ella y se
acostó con ella. Ella dio a luz un hijo, al que llamó Salomón. Yavé lo
amó; \bibleverse{25} y envió por medio del profeta Natán, y lo llamó
Jedidías, por amor a Yavé.

\hypertarget{joab-conquista-rabuxe1-castigo-de-los-amonitas}{%
\subsection{Joab conquista Rabá; Castigo de los
amonitas}\label{joab-conquista-rabuxe1-castigo-de-los-amonitas}}

\bibleverse{26} Joab luchó contra Rabá, de los hijos de Amón, y tomó la
ciudad real. \footnote{\textbf{12:26} Jer 49,2} \bibleverse{27} Joab
envió mensajeros a David y le dijo: ``He luchado contra Rabá. Sí, he
tomado la ciudad de las aguas. \bibleverse{28} Reúne, pues, ahora al
resto del pueblo y acampa contra la ciudad y tómala; no sea que yo tome
la ciudad y sea llamada con mi nombre.''

\bibleverse{29} David reunió a todo el pueblo y fue a Rabá, luchó contra
ella y la tomó. \bibleverse{30} Tomó la corona de su rey de la cabeza;
su peso era de un talento\footnote{\textbf{12:30} Un talento equivale a
  unos 30 kilogramos o 66 libras o 965 onzas troy.} de oro, y en ella
había piedras preciosas; y fue puesta sobre la cabeza de David. Sacó de
la ciudad una gran cantidad de botín. \bibleverse{31} Sacó al pueblo que
estaba en ella y lo puso a trabajar bajo sierras, bajo picos de hierro,
bajo hachas de hierro, y lo hizo ir al horno de ladrillos; y así hizo
con todas las ciudades de los hijos de Amón. Luego David y todo el
pueblo regresaron a Jerusalén.

\hypertarget{el-amor-apasionado-de-amnuxf3n-su-indignaciuxf3n-hacia-su-media-hermana-thamar}{%
\subsection{El amor apasionado de Amnón; su indignación hacia su media
hermana
Thamar}\label{el-amor-apasionado-de-amnuxf3n-su-indignaciuxf3n-hacia-su-media-hermana-thamar}}

\hypertarget{section-12}{%
\section{13}\label{section-12}}

\bibleverse{1} Después de esto, Absalón, hijo de David, tenía una
hermosa hermana que se llamaba Tamar, y Amnón, hijo de David, la amaba.
\footnote{\textbf{13:1} 2Sam 3,2-3} \bibleverse{2} Amnón se preocupó
tanto que enfermó a causa de su hermana Tamar, pues ella era virgen, y a
Amnón le parecía difícil hacerle algo. \bibleverse{3} Pero Amnón tenía
un amigo que se llamaba Jonadab, hijo de Simea, hermano de David; y
Jonadab era un hombre muy sutil. \footnote{\textbf{13:3} 1Sam 16,9}
\bibleverse{4} Le dijo: ``¿Por qué, hijo del rey, estás tan triste de un
día para otro? ¿No quieres decírmelo?'' Amnón le dijo: ``Amo a Tamar, la
hermana de mi hermano Absalón''.

\bibleverse{5} Jonadab le dijo: ``Acuéstate en tu cama y finge estar
enfermo. Cuando tu padre venga a verte, dile: `Por favor, que venga mi
hermana Tamar y me dé pan para comer, y prepara la comida a mi vista,
para que la vea y la coma de su mano'.''

\hypertarget{ejecuciuxf3n-del-infame-ataque}{%
\subsection{Ejecución del infame
ataque}\label{ejecuciuxf3n-del-infame-ataque}}

\bibleverse{6} Entonces Amnón se acostó y fingió estar enfermo. Cuando
el rey vino a verlo, Amnón le dijo al rey: ``Por favor, que venga mi
hermana Tamar y me haga un par de pasteles en mi presencia, para que
coma de su mano.''

\bibleverse{7} Entonces David envió a casa a Tamar, diciendo: ``Ve ahora
a casa de tu hermano Amnón y prepárale comida''. \bibleverse{8} Así que
Tamar fue a la casa de su hermano Amnón, que estaba acostado. Ella tomó
masa, la amasó, hizo pasteles a la vista de él, y horneó los pasteles.
\bibleverse{9} Tomó la sartén y las sirvió delante de él, pero éste se
negó a comer. Amnón dijo: ``Que todos los hombres me dejen''. Entonces
todos los hombres se alejaron de él. \bibleverse{10} Amnón dijo a Tamar:
``Trae la comida a la habitación, para que coma de tu mano''. Tamar tomó
las tortas que había hecho y se las llevó a la habitación a su hermano
Amnón. \bibleverse{11} Cuando se las acercó para que comiera, él la
agarró y le dijo: ``¡Ven, acuéstate conmigo, hermana mía!'' \footnote{\textbf{13:11}
  Lev 18,9}

\bibleverse{12} Ella le respondió: ``¡No, hermano mío, no me obligues!
Porque no se debe hacer tal cosa en Israel. ¡No hagas esta locura!
\footnote{\textbf{13:12} Deut 22,21} \bibleverse{13} En cuanto a mí,
¿dónde voy a llevar mi vergüenza? Y en cuanto a ti, serás como uno de
los necios de Israel. Ahora, pues, por favor, habla con el rey; porque
él no me negará nada''.

\bibleverse{14} Pero él no quiso escuchar su voz, sino que, siendo más
fuerte que ella, la forzó y se acostó con ella.

\hypertarget{otro-pecado-vergonzoso-de-amnuxf3n-a-thamar}{%
\subsection{Otro pecado vergonzoso de Amnón a
Thamar}\label{otro-pecado-vergonzoso-de-amnuxf3n-a-thamar}}

\bibleverse{15} Entonces Amnón la odió con un odio muy grande, pues el
odio con que la odiaba era mayor que el amor con que la había amado.
Amnón le dijo: ``¡Levántate, vete!''

\bibleverse{16} Ella le dijo: ``¡No, porque este gran agravio al
despedirme es peor que el otro que me hiciste!'' Pero él no quiso
escucharla. \bibleverse{17} Entonces llamó a su criado, que le servía, y
le dijo: ``Aparta ahora a esta mujer de mí y echa el cerrojo tras
ella.''

\bibleverse{18} Llevaba un vestido de varios colores, pues las hijas del
rey que eran vírgenes se vestían con tales ropas. Entonces su criado la
sacó y cerró la puerta tras ella. \bibleverse{19} Tamar se puso ceniza
en la cabeza y se rasgó el vestido de varios colores que llevaba puesto;
se puso la mano en la cabeza y se fue, llorando en voz alta mientras se
iba. \footnote{\textbf{13:19} Job 2,12}

\hypertarget{el-comportamiento-de-absaluxf3n-y-el-rey-despuuxe9s-del-ultraje}{%
\subsection{El comportamiento de Absalón y el rey después del
ultraje}\label{el-comportamiento-de-absaluxf3n-y-el-rey-despuuxe9s-del-ultraje}}

\bibleverse{20} Su hermano Absalón le dijo: ``¿Amnón, tu hermano, ha
estado contigo? Pero ahora calla, hermana mía. Él es tu hermano. No te
tomes esto a pecho''. Así que Tamar se quedó desolada en casa de su
hermano Absalón. \bibleverse{21} Pero cuando el rey David se enteró de
todas estas cosas, se enojó mucho. \bibleverse{22} Absalón no hablaba
con Amnón ni bien ni mal, porque Absalón odiaba a Amnón porque había
forzado a su hermana Tamar.

\hypertarget{la-venganza-de-absaluxf3n-contra-amnuxf3n}{%
\subsection{La venganza de Absalón contra
Amnón}\label{la-venganza-de-absaluxf3n-contra-amnuxf3n}}

\bibleverse{23} Después de dos años completos, Absalón tenía
esquiladores de ovejas en Baal Hazor, que está junto a Efraín; y Absalón
invitó a todos los hijos del rey. \bibleverse{24} Absalón fue a ver al
rey y le dijo: ``Mira ahora, tu siervo tiene esquiladores de ovejas. Por
favor, deja que el rey y sus siervos vayan con tu siervo''.

\bibleverse{25} El rey dijo a Absalón: ``No, hijo mío, no vayamos todos,
no vaya a ser que seamos una carga para ti''. Lo presionó; sin embargo,
no quiso ir, sino que lo bendijo.

\bibleverse{26} Entonces Absalón dijo: ``Si no, deja que mi hermano
Amnón venga con nosotros''. El rey le dijo: ``¿Por qué ha de ir
contigo?''.

\bibleverse{27} Pero Absalón lo presionó, y dejó que Amnón y todos los
hijos del rey se fueran con él. \bibleverse{28} Absalón ordenó a sus
siervos, diciendo: ``Fíjense ahora, cuando el corazón de Amnón esté
alegre por el vino; y cuando yo les diga: `Golpeen a Amnón', entonces
mátenlo. No tengáis miedo. ¿No te lo he ordenado? Sé valiente, y sé
valeroso''. \footnote{\textbf{13:28} Lev 20,17}

\bibleverse{29} Los siervos de Absalón hicieron con Amnón lo que éste
les había ordenado. Entonces todos los hijos del rey se levantaron, y
cada uno subió a su mula y huyó.

\hypertarget{los-eventos-en-el-palacio-de-david-cuando-llegaron-las-terribles-noticias}{%
\subsection{Los eventos en el palacio de David cuando llegaron las
terribles
noticias}\label{los-eventos-en-el-palacio-de-david-cuando-llegaron-las-terribles-noticias}}

\bibleverse{30} Mientras iban de camino, llegó a David la noticia:
``¡Absalón ha matado a todos los hijos del rey, y no queda ni uno de
ellos!''

\bibleverse{31} Entonces el rey se levantó, se rasgó las vestiduras y se
echó en tierra; y todos sus servidores estaban con las vestiduras
rasgadas. \bibleverse{32} Jonadab hijo de Simea, hermano de David,
respondió: ``No deje mi señor suponer que han matado a todos los
jóvenes, hijos del rey, pues sólo Amnón ha muerto; porque por
designación de Absalón esto ha sido determinado desde el día en que
forzó a su hermana Tamar. \bibleverse{33} Ahora, pues, que mi señor el
rey no se tome el asunto a pecho, para pensar que todos los hijos del
rey han muerto, pues sólo Amnón ha muerto.'' \bibleverse{34} Pero
Absalón huyó. El joven que vigilaba levantó los ojos y miró, y he aquí
que por la ladera del monte venía mucha gente detrás de él.
\bibleverse{35} Jonadab dijo al rey: ``¡Mira que vienen los hijos del
rey! Es como dijo tu siervo''. \bibleverse{36} Tan pronto como terminó
de hablar, he aquí que los hijos del rey venían, y alzaban la voz y
lloraban. También el rey y todos sus servidores lloraron amargamente.

\hypertarget{la-fuga-de-absaluxf3n-a-gesur-a-su-abuelo}{%
\subsection{La fuga de Absalón a Gesur a su
abuelo}\label{la-fuga-de-absaluxf3n-a-gesur-a-su-abuelo}}

\bibleverse{37} Pero Absalón huyó y se fue a Talmai, hijo de Ammihur,
rey de Gesur. David lloraba a su hijo todos los días. \footnote{\textbf{13:37}
  2Sam 3,3; 2Sam 14,23} \bibleverse{38} Así que Absalón huyó y se fue a
Guesur, y estuvo allí tres años.

\hypertarget{la-intervenciuxf3n-de-joab-la-conversaciuxf3n-de-la-sabia-esposa-de-thecoa-con-david}{%
\subsection{La intervención de Joab; la conversación de la sabia esposa
de Thecoa con
David}\label{la-intervenciuxf3n-de-joab-la-conversaciuxf3n-de-la-sabia-esposa-de-thecoa-con-david}}

\bibleverse{39} El rey David anhelaba salir a ver a Absalón, pues estaba
consolado por Amnón, ya que había muerto.

\hypertarget{section-13}{%
\section{14}\label{section-13}}

\bibleverse{1} Joab, hijo de Sarvia, se dio cuenta de que el corazón del
rey estaba inclinado hacia Absalón. \bibleverse{2} Joab envió a Tecoa y
trajo de allí a una mujer sabia, y le dijo: ``Por favor, actúa como una
mujer de luto, y ponte ropa de luto, por favor, y no te unjas con
aceite, sino sé como una mujer que ha llorado mucho tiempo a un muerto.
\bibleverse{3} Entra al rey y háblale así''. Entonces Joab puso las
palabras en su boca.

\hypertarget{el-primer-discurso-de-la-sabia}{%
\subsection{El primer discurso de la
sabia}\label{el-primer-discurso-de-la-sabia}}

\bibleverse{4} Cuando la mujer de Tecoa se dirigió al rey, se postró en
el suelo, mostró respeto y dijo: ``¡Ayuda, oh rey!''

\bibleverse{5} El rey le dijo: ``¿Qué te pasa?'' Ella respondió:
``Verdaderamente soy viuda, y mi marido ha muerto. \bibleverse{6} Tu
siervo tenía dos hijos, y ambos peleaban juntos en el campo, y no había
quien los separara, pero el uno hirió al otro y lo mató. \bibleverse{7}
He aquí que toda la familia se ha levantado contra tu siervo, y dicen:
``Entrega al que hirió a su hermano, para que lo matemos por la vida de
su hermano al que mató, y así destruir también al heredero. Así
apagarían mi carbón que queda, y no dejarían a mi marido ni nombre ni
resto en la superficie de la tierra.'' \footnote{\textbf{14:7} Deut
  19,11-13}

\bibleverse{8} El rey dijo a la mujer: ``Vete a tu casa, y yo daré una
orden sobre ti''.

\bibleverse{9} La mujer de Tecoa dijo al rey: ``Rey, señor mío, que la
iniquidad caiga sobre mí y sobre la casa de mi padre, y que el rey y su
trono queden libres de culpa.''

\bibleverse{10} El rey dijo: ``Quien te diga algo, tráemelo y no te
molestará más''.

\bibleverse{11} Entonces ella dijo: ``Por favor, que el rey se acuerde
de Yahvé, tu Dios, para que el vengador de la sangre no destruya más,
para que no destruyan a mi hijo''. Dijo: ``Vive Yahvé, que ni un pelo de
tu hijo caerá a la tierra''. \footnote{\textbf{14:11} 1Sam 14,45; 1Re
  1,52}

\hypertarget{nuevo-discurso-de-mujer-sabia}{%
\subsection{Nuevo discurso de mujer
sabia}\label{nuevo-discurso-de-mujer-sabia}}

\bibleverse{12} Entonces la mujer dijo: ``Por favor, deja que tu siervo
hable una palabra a mi señor el rey''. Dijo: ``Diga''.

\bibleverse{13} La mujer dijo: ``¿Por qué, pues, has ideado tal cosa
contra el pueblo de Dios? Porque al decir esta palabra el rey es como
uno que es culpable, ya que el rey no hace volver a casa a su
desterrado. \bibleverse{14} Porque es necesario que muramos, y somos
como el agua derramada en la tierra, que no puede volver a recogerse; y
Dios no quita la vida, sino que inventa medios para que el desterrado no
sea desterrado de él. \footnote{\textbf{14:14} Ezeq 18,23}
\bibleverse{15} Ahora, pues, viendo que he venido a decir esta palabra a
mi señor el rey, es porque el pueblo me ha hecho temer. Tu siervo dijo:
`Ahora hablaré al rey; puede ser que el rey cumpla la petición de su
siervo'. \bibleverse{16} Porque el rey escuchará, para librar a su
siervo de la mano del hombre que quiere destruirnos a mí y a mi hijo
juntos de la herencia de Dios. \bibleverse{17} Entonces su siervo dijo:
`Por favor, que la palabra de mi señor el rey traiga descanso; porque
como un ángel de Dios, así es mi señor el rey para discernir lo bueno y
lo malo. Que Yahvé, tu Dios, esté contigo'\,''. \footnote{\textbf{14:17}
  2Sam 19,27}

\hypertarget{el-rey-ve-a-travuxe9s-del-astuto-plan}{%
\subsection{El rey ve a través del astuto
plan}\label{el-rey-ve-a-travuxe9s-del-astuto-plan}}

\bibleverse{18} El rey respondió a la mujer: ``Por favor, no me ocultes
nada de lo que te pido''. La mujer dijo: ``Que hable ahora mi señor el
rey''.

\bibleverse{19} El rey dijo: ``¿Está la mano de Joab contigo en todo
esto?'' La mujer respondió: ``Vive tu alma, mi señor el rey, que nadie
puede volverse a la derecha o a la izquierda de nada de lo que mi señor
el rey ha dicho; porque tu siervo Joab me urgió, y puso todas estas
palabras en boca de tu siervo. \bibleverse{20} Tu siervo Joab ha hecho
esto para cambiar la cara del asunto. Mi señor es sabio, según la
sabiduría de un ángel de Dios, para conocer todas las cosas que hay en
la tierra.'' \footnote{\textbf{14:20} 2Sam 14,17}

\hypertarget{la-promesa-de-david-joab-agradece-al-rey-por-haber-cumplido-su-pedido-y-trae-a-absaluxf3n}{%
\subsection{La promesa de David; Joab agradece al rey por haber cumplido
su pedido y trae a
Absalón}\label{la-promesa-de-david-joab-agradece-al-rey-por-haber-cumplido-su-pedido-y-trae-a-absaluxf3n}}

\bibleverse{21} El rey dijo a Joab: ``Mira ahora, he concedido esto. Ve,
pues, y haz volver al joven Absalón''.

\bibleverse{22} Joab se postró en el suelo sobre su rostro, mostró
respeto y bendijo al rey. Joab dijo: ``Hoy sabe tu siervo que he hallado
gracia ante tus ojos, mi señor, oh rey, pues el rey ha cumplido la
petición de su siervo''.

\bibleverse{23} Entonces Joab se levantó y fue a Gesur, y trajo a
Absalón a Jerusalén. \footnote{\textbf{14:23} 2Sam 13,37}
\bibleverse{24} El rey dijo: ``Que vuelva a su casa, pero que no vea mi
rostro''. Así que Absalón volvió a su casa y no vio el rostro del rey.

\hypertarget{la-belleza-de-absaluxf3n-sus-hijos}{%
\subsection{La belleza de Absalón; sus
hijos}\label{la-belleza-de-absaluxf3n-sus-hijos}}

\bibleverse{25} En todo Israel no había nadie que fuera tan alabado como
Absalón por su belleza. Desde la planta de su pie hasta la coronilla de
su cabeza no había en él ningún defecto. \bibleverse{26} Cuando se
cortaba el pelo de la cabeza (ahora era al final de cada año que se lo
cortaba; porque le pesaba, por eso se lo cortaba), pesaba el pelo de su
cabeza en doscientos siclos,\footnote{\textbf{14:26} Un siclo equivale a
  unos 10 gramos o a unas 0,35 onzas, por lo que 200 siclos equivalen a
  unos 2 kilogramos o a unas 4,4 libras.} según el peso del rey.
\bibleverse{27} A Absalón le nacieron tres hijos y una hija que se
llamaba Tamar. Era una mujer de rostro hermoso. \footnote{\textbf{14:27}
  2Sam 13,1}

\hypertarget{absaluxf3n-hace-que-joab-lo-reconcilie-formalmente-con-su-padre}{%
\subsection{Absalón hace que Joab lo reconcilie formalmente con su
padre}\label{absaluxf3n-hace-que-joab-lo-reconcilie-formalmente-con-su-padre}}

\bibleverse{28} Absalón vivió dos años enteros en Jerusalén, y no vio el
rostro del rey. \bibleverse{29} Entonces Absalón mandó llamar a Joab
para que lo enviara al rey, pero éste no quiso acudir a él. Volvió a
enviar por segunda vez, pero no quiso venir. \bibleverse{30} Entonces
dijo a sus siervos: ``He aquí que el campo de Joab está cerca del mío, y
tiene allí cebada. Vayan y préndanle fuego''. Así que los siervos de
Absalón prendieron fuego al campo.

\bibleverse{31} Entonces Joab se levantó y vino a Absalón a su casa y le
dijo: ``¿Por qué tus siervos han incendiado mi campo?''

\bibleverse{32} Absalón respondió a Joab: ``He aquí que yo te envié a
decir: ``Ven aquí, para que te envíe al rey a decir: ``¿Por qué he
venido de Guesur? Sería mejor para mí estar todavía allí. Ahora, pues,
déjame ver la cara del rey; y si hay iniquidad en mí, que me mate''\,''.

\bibleverse{33} Entonces Joab vino al rey y se lo comunicó; y cuando
llamó a Absalón, éste vino al rey y se postró en tierra ante el rey; y
el rey besó a Absalón.

\hypertarget{actividades-ambiciosas-y-favorables-de-absalom}{%
\subsection{Actividades ambiciosas y favorables de
Absalom}\label{actividades-ambiciosas-y-favorables-de-absalom}}

\hypertarget{section-14}{%
\section{15}\label{section-14}}

\bibleverse{1} Después de esto, Absalón preparó para sí un carro y
caballos, y cincuenta hombres para que corrieran delante de él.
\footnote{\textbf{15:1} 1Re 1,5} \bibleverse{2} Absalón se levantó
temprano y se puso junto al camino de la puerta. Cuando alguno tenía un
pleito que debía presentarse ante el rey para ser juzgado, Absalón lo
llamaba y le decía: ``¿De qué ciudad eres?'' Dijo: ``Tu siervo es de una
de las tribus de Israel''.

\bibleverse{3} Absalón le dijo: ``He aquí que tus asuntos son buenos y
correctos, pero no hay nadie nombrado por el rey para oírte.''
\bibleverse{4} Absalón dijo además: ``¡Oh, si me nombraran juez en el
país, para que todo hombre que tuviera algún pleito o causa viniera a mí
y yo le hiciera justicia!'' \bibleverse{5} Era así, que cuando algún
hombre se acercaba a inclinarse ante él, extendía la mano, lo tomaba y
lo besaba. \bibleverse{6} Absalón hacía este tipo de cosas con todo
Israel que se acercaba al rey para pedirle justicia. Así, Absalón robó
el corazón de los hombres de Israel.

\hypertarget{la-conspiraciuxf3n-y-la-indignaciuxf3n-de-absaluxf3n-en-hebruxf3n}{%
\subsection{La conspiración y la indignación de Absalón en
Hebrón}\label{la-conspiraciuxf3n-y-la-indignaciuxf3n-de-absaluxf3n-en-hebruxf3n}}

\bibleverse{7} Al cabo de cuarenta años, Absalón dijo al rey: ``Por
favor, déjame ir a pagar mi voto, que he hecho a Yavé, en Hebrón.
\bibleverse{8} Porque tu siervo hizo un voto mientras estaba en Guesur,
en Siria, diciendo: ``Si Yahvé me hace volver a Jerusalén, entonces
serviré a Yahvé.'' \footnote{\textbf{15:8} Gén 28,20; 2Sam 13,38}

\bibleverse{9} El rey le dijo: ``Ve en paz''. Así que se levantó y se
dirigió a Hebrón. \bibleverse{10} Pero Absalón envió espías por todas
las tribus de Israel, diciendo: ``En cuanto oigan el sonido de la
trompeta, dirán: ``¡Absalón es rey en Hebrón!''\,''

\bibleverse{11} Doscientos hombres salieron con Absalón de Jerusalén,
que fueron invitados, y fueron en su sencillez; y no sabían nada.
\bibleverse{12} Absalón mandó llamar a Ajitofel el gilonita, consejero
de David, desde su ciudad, desde Giloh, mientras ofrecía los
sacrificios. La conspiración era fuerte, pues el pueblo aumentaba
continuamente con Absalón. \footnote{\textbf{15:12} 2Sam 23,34}

\hypertarget{david-huye-apresuradamente-de-jerusaluxe9n-despuuxe9s-de-dejar-atruxe1s-algunas-concubinas}{%
\subsection{David huye apresuradamente de Jerusalén después de dejar
atrás algunas
concubinas}\label{david-huye-apresuradamente-de-jerusaluxe9n-despuuxe9s-de-dejar-atruxe1s-algunas-concubinas}}

\bibleverse{13} Un mensajero llegó a David diciendo: ``El corazón de los
hombres de Israel está en pos de Absalón''.

\bibleverse{14} David dijo a todos sus servidores que estaban con él en
Jerusalén: ``¡Levántate! Huyamos, o ninguno de nosotros escapará de
Absalón. Apresúrense a partir, no sea que nos alcance rápidamente y haga
caer el mal sobre nosotros, y golpee la ciudad con el filo de la
espada.'' \footnote{\textbf{15:14} Sal 3,1}

\bibleverse{15} Los siervos del rey dijeron al rey: ``He aquí que tus
siervos están dispuestos a hacer lo que mi señor el rey quiera''.

\bibleverse{16} El rey salió, y toda su casa tras él. El rey dejó a diez
mujeres, que eran concubinas, para que cuidaran la casa.

\hypertarget{la-gente-de-guerra-marchando-frente-al-rey-la-lealtad-de-itthai}{%
\subsection{La gente de guerra marchando frente al rey; la lealtad de
Itthai}\label{la-gente-de-guerra-marchando-frente-al-rey-la-lealtad-de-itthai}}

\bibleverse{17} El rey salió, y todo el pueblo tras él; y se quedaron en
Bet Merac. \bibleverse{18} Todos sus siervos pasaron junto a él, y todos
los cereteos, los peleteos y los gitanos, seiscientos hombres que
vinieron tras él desde Gat, pasaron delante del rey.

\bibleverse{19} Entonces el rey le dijo a Ittai el gita: ``¿Por qué
también tú vas con nosotros? Regresa y quédate con el rey, pues eres
extranjero y también desterrado. Vuelve a tu lugar. \footnote{\textbf{15:19}
  2Sam 18,2} \bibleverse{20} Ya que viniste ayer, ¿he de hacerte subir y
bajar hoy con nosotros, ya que yo voy donde puedo? Vuelve y recupera a
tus hermanos. La misericordia y la verdad sean contigo''.

\bibleverse{21} Ittai respondió al rey y dijo: ``Vive Yahvé y vive mi
señor el rey, ciertamente en el lugar en que esté mi señor el rey, ya
sea para la muerte o para la vida, tu siervo estará también allí.''

\bibleverse{22} David dijo a Ittai: ``Ve y pasa''. Pasó Ittai, el
getita, y todos sus hombres, y todos los pequeños que estaban con él.
\bibleverse{23} Todo el país lloró a gritos, y todo el pueblo pasó.
También el rey pasó el arroyo de Cedrón, y todo el pueblo pasó hacia el
camino del desierto. \footnote{\textbf{15:23} Juan 18,1}

\hypertarget{la-comisiuxf3n-de-david-a-sadoc-y-abiatar}{%
\subsection{La comisión de David a Sadoc y
Abiatar}\label{la-comisiuxf3n-de-david-a-sadoc-y-abiatar}}

\bibleverse{24} También vino Sadoc, y con él todos los levitas, llevando
el arca del pacto de Dios, y depositaron el arca de Dios; y Abiatar
subió hasta que todo el pueblo terminó de salir de la ciudad.
\bibleverse{25} El rey dijo a Sadoc: ``Lleva el arca de Dios de vuelta a
la ciudad. Si hallo gracia ante los ojos de Yavé, él me hará volver, y
me mostrará tanto ella como su morada; \bibleverse{26} pero si dice: `No
me complaces', aquí estoy. Que haga conmigo lo que le parezca bien''.
\footnote{\textbf{15:26} 2Sam 10,12; 1Sam 3,18} \bibleverse{27} El rey
dijo también al sacerdote Sadoc: ``¿No eres vidente? Vuelve a la ciudad
en paz, y tus dos hijos contigo, Ahimaas tu hijo y Jonatán el hijo de
Abiatar. \footnote{\textbf{15:27} 1Re 1,42} \bibleverse{28} Yo me
quedaré en los vados del desierto hasta que me llegue una noticia tuya
para informarme.'' \bibleverse{29} Así pues, Sadoc y Abiatar volvieron a
llevar el arca de Dios a Jerusalén, y se quedaron allí.

\hypertarget{marcha-de-david-en-el-monte-de-los-olivos-su-orden-a-husai}{%
\subsection{Marcha de David en el monte de los Olivos; su orden a
Husai}\label{marcha-de-david-en-el-monte-de-los-olivos-su-orden-a-husai}}

\bibleverse{30} David subió por la cuesta del monte de los Olivos, y
lloró al subir; se cubrió la cabeza y fue descalzo. Todo el pueblo que
estaba con él se cubrió la cabeza, y subieron llorando.

\bibleverse{31} Alguien le dijo a David: ``Ajitófel está entre los
conspiradores con Absalón''. David dijo: ``Yahvé, por favor, convierte
el consejo de Ajitófel en una tontería''.

\bibleverse{32} Cuando David llegó a la cima, donde se adoraba a Dios,
he aquí que Husai el arquita salió a su encuentro con la túnica rota y
tierra en la cabeza. \bibleverse{33} David le dijo: ``Si pasas conmigo,
serás una carga para mí; \bibleverse{34} pero si vuelves a la ciudad y
le dices a Absalón: ``Seré tu siervo, oh rey. Como he sido siervo de tu
padre en el pasado, así seré ahora tu siervo; entonces derrotarás para
mí el consejo de Ajitófel.' \footnote{\textbf{15:34} 2Sam 17,7}
\bibleverse{35} ¿No tienes allí contigo a los sacerdotes Sadoc y
Abiatar? Por tanto, todo lo que oigas de la casa del rey, dilo a los
sacerdotes Sadoc y Abiatar. \bibleverse{36} He aquí que tienen allí con
ellos a sus dos hijos, Ahimaas, hijo de Sadoc, y Jonatán, hijo de
Abiatar. Envíame todo lo que oigas por ellos''. \footnote{\textbf{15:36}
  2Sam 17,15-17}

\bibleverse{37} Entonces Husai, amigo de David, entró en la ciudad; y
Absalón entró en Jerusalén. \footnote{\textbf{15:37} 1Cró 27,33}

\hypertarget{siba-el-siervo-de-mefiboseth-da-presentes-al-rey-su-informe-de-mentiras-sobre-mefiboset}{%
\subsection{Siba, el siervo de Mefiboseth, da presentes al rey; su
informe de mentiras sobre
Mefiboset}\label{siba-el-siervo-de-mefiboseth-da-presentes-al-rey-su-informe-de-mentiras-sobre-mefiboset}}

\hypertarget{section-15}{%
\section{16}\label{section-15}}

\bibleverse{1} Cuando David estaba un poco más arriba, he aquí que Siba,
el siervo de Mefiboset, le salió al encuentro con un par de asnos
ensillados, y sobre ellos doscientos panes, cien racimos de pasas, cien
frutos de verano y un recipiente de vino. \footnote{\textbf{16:1} 2Sam
  9,2} \bibleverse{2} El rey dijo a Siba: ``¿Qué quieres decir con
esto?'' Siba dijo: ``Los asnos son para que los monte la casa del rey; y
el pan y la fruta de verano para que los coman los jóvenes; y el vino,
para que lo beban los que están cansados en el desierto.''

\bibleverse{3} El rey dijo: ``¿Dónde está el hijo de tu amo?'' Siba dijo
al rey: ``He aquí que él se queda en Jerusalén, porque ha dicho: ``Hoy
la casa de Israel me devolverá el reino de mi padre''\,''. \footnote{\textbf{16:3}
  2Sam 19,26}

\bibleverse{4} Entonces el rey dijo a Siba: ``Mira, todo lo que
pertenece a Mefiboset es tuyo''. Ziba dijo: ``Me inclino. Haz que
encuentre favor ante tus ojos, mi señor, oh rey''.

\hypertarget{comportamiento-indigno-de-simei-hacia-el-rey}{%
\subsection{Comportamiento indigno de Simei hacia el
rey}\label{comportamiento-indigno-de-simei-hacia-el-rey}}

\bibleverse{5} Cuando el rey David llegó a Bahurim, he aquí que salió un
hombre de la familia de la casa de Saúl, cuyo nombre era Simei, hijo de
Gera. Salió y maldijo al llegar. \footnote{\textbf{16:5} 1Re 2,8; Éxod
  22,28} \bibleverse{6} Arrojó piedras contra David y contra todos los
servidores del rey David, y todo el pueblo y todos los valientes estaban
a su derecha y a su izquierda. \bibleverse{7} Al maldecir, Simei dijo:
``¡Vete, vete, hombre de sangre y malvado! \bibleverse{8} ¡El Señor ha
hecho recaer sobre ti toda la sangre de la casa de Saúl, en cuyo lugar
has reinado! El Señor ha entregado el reino en manos de tu hijo Absalón.
He aquí que has sido atrapado por tu propia maldad, porque eres un
hombre de sangre!''

\bibleverse{9} Entonces Abisai, hijo de Sarvia, dijo al rey: ``¿Por qué
ha de maldecir este perro muerto a mi señor el rey? Por favor, déjame ir
y quitarle la cabeza''. \footnote{\textbf{16:9} 1Sam 26,8}
\bibleverse{10} El rey respondió: ``¿Qué tengo que ver con ustedes,
hijos de Sarvia? Porque él maldice, y porque Yahvé le ha dicho: `Maldice
a David', ¿quién dirá entonces: `Por qué lo has hecho'?''

\bibleverse{11} David dijo a Abisai y a todos sus siervos: ``He aquí que
mi hijo, que salió de mis entrañas, busca mi vida. ¿Cuánto más este
benjamita, ahora? Déjenlo en paz y que maldiga, porque Yahvé lo ha
invitado. \bibleverse{12} Puede ser que Yavé se fije en el mal que se me
ha hecho, y que Yavé me pague bien la maldición que hoy se me hace.''
\bibleverse{13} Así que David y sus hombres se fueron por el camino, y
Simei iba por la ladera opuesta a él y maldecía a su paso, le arrojaba
piedras y tiraba polvo. \bibleverse{14} El rey y todo el pueblo que lo
acompañaba llegaron cansados, y él se refrescó allí.

\hypertarget{absaluxf3n-engauxf1ado-por-husai}{%
\subsection{Absalón engañado por
Husai}\label{absaluxf3n-engauxf1ado-por-husai}}

\bibleverse{15} Absalón y todo el pueblo, los hombres de Israel,
llegaron a Jerusalén, y Ajitófel con él. \bibleverse{16} Cuando Husai el
arquita, amigo de David, se acercó a Absalón, Husai le dijo: ``¡Viva el
rey! Viva el rey!'' \footnote{\textbf{16:16} 2Sam 15,37; 1Sam 10,24}

\bibleverse{17} Absalón dijo a Husai: ``¿Esta es tu bondad con tu amigo?
¿Por qué no te has ido con tu amigo?''

\bibleverse{18} Husai dijo a Absalón: ``No; pero a quien Yahvé y este
pueblo y todos los hombres de Israel hayan elegido, yo seré suyo y me
quedaré con él. \bibleverse{19} Además, ¿a quién debo servir? ¿No debo
servir en presencia de su hijo? Como he servido en presencia de tu
padre, así estaré en tu presencia''.

\hypertarget{se-siguiuxf3-el-primer-consejo-de-ahitofel-de-absaluxf3n}{%
\subsection{Se siguió el primer consejo de Ahitofel de
Absalón}\label{se-siguiuxf3-el-primer-consejo-de-ahitofel-de-absaluxf3n}}

\bibleverse{20} Entonces Absalón dijo a Ajitófel: ``Aconseja lo que
debemos hacer''.

\bibleverse{21} Ajitófel le dijo a Absalón: ``Entra con las concubinas
de tu padre que él ha dejado para cuidar la casa. Entonces todo Israel
se enterará de que tu padre te aborrece. Entonces las manos de todos los
que están contigo serán fuertes''. \footnote{\textbf{16:21} 2Sam 15,16}

\bibleverse{22} Así que extendieron una tienda para Absalón en lo alto
de la casa, y Absalón entró a las concubinas de su padre a la vista de
todo Israel. \footnote{\textbf{16:22} 2Sam 12,11; Lev 18,8}

\bibleverse{23} El consejo de Ajitofel, que dio en aquellos días, fue
como si un hombre preguntara en el santuario interior de Dios. Todo el
consejo de Ajitofel fue así tanto con David como con Absalón.

\hypertarget{el-segundo-excelente-consejo-de-ahitofel-fue-rechazado-por-husai-y-rechazado-por-absaluxf3n}{%
\subsection{El segundo excelente consejo de Ahitofel fue rechazado por
Husai y rechazado por
Absalón}\label{el-segundo-excelente-consejo-de-ahitofel-fue-rechazado-por-husai-y-rechazado-por-absaluxf3n}}

\hypertarget{section-16}{%
\section{17}\label{section-16}}

\bibleverse{1} Además, Ajitófel dijo a Absalón: ``Déjame elegir ahora
doce mil hombres, y me levantaré y perseguiré a David esta noche.
\footnote{\textbf{17:1} Sal 71,11} \bibleverse{2} Lo atacaré cuando esté
cansado y agotado, y lo asustaré. Todo el pueblo que está con él huirá.
Golpearé sólo al rey, \bibleverse{3} y haré volver a todo el pueblo
hacia ti. El hombre que buscas es como si todos regresaran. Todo el
pueblo estará en paz''.

\bibleverse{4} El dicho agradó a Absalón y a todos los ancianos de
Israel. \bibleverse{5} Entonces Absalón dijo: ``Llama ahora también a
Husai el arquita, y oigamos igualmente lo que dice.'' \footnote{\textbf{17:5}
  2Sam 16,16}

\bibleverse{6} Cuando Husai se acercó a Absalón, éste le habló diciendo:
``Ajitófel ha hablado así. ¿Haremos lo que él dice? Si no, habla''.

\bibleverse{7} Husai dijo a Absalón: ``El consejo que Ajitófel ha dado
esta vez no es bueno''. \bibleverse{8} Husai dijo además: ``Tú conoces a
tu padre y a sus hombres, que son hombres poderosos, y son fieros de
mente, como una osa despojada de sus cachorros en el campo. Tu padre es
un hombre de guerra, y no se alojará con el pueblo. \bibleverse{9} He
aquí que ahora está escondido en algún pozo o en otro lugar. Sucederá
que cuando algunos de ellos hayan caído al principio, quien lo oiga
dirá: ``¡Hay una matanza entre el pueblo que sigue a Absalón!''
\bibleverse{10} Incluso el que es valiente, cuyo corazón es como el de
un león, se derretirá por completo; porque todo Israel sabe que tu padre
es un hombre valiente, y que los que están con él son hombres valientes.
\bibleverse{11} Pero yo aconsejo que se reúna contigo todo Israel, desde
Dan hasta Beerseba, como la arena que está junto al mar para la
multitud; y que vayas a la batalla en tu propia persona. \bibleverse{12}
Así llegaremos a él en algún lugar donde se encuentre, y lo iluminaremos
como cae el rocío en la tierra, y no dejaremos ni uno solo de él y de
todos los hombres que están con él. \bibleverse{13} Además, si se ha
metido en una ciudad, todo Israel llevará cuerdas a esa ciudad, y la
arrastraremos al río, hasta que no se encuentre allí ni una sola piedra
pequeña.''

\bibleverse{14} Absalón y todos los hombres de Israel dijeron: ``El
consejo de Husai el arquita es mejor que el consejo de Ajitófel.''
Porque Yahvé había ordenado derrotar el buen consejo de Ajitofel, con el
propósito de que Yahvé trajera el mal a Absalón. \footnote{\textbf{17:14}
  2Sam 15,31; 2Sam 15,34}

\hypertarget{husai-y-los-sacerdotes-envuxedan-mensajes-en-secreto-al-rey-david-pone-sobre-el-jorduxe1n}{%
\subsection{Husai y los sacerdotes envían mensajes en secreto al rey;
David pone sobre el
Jordán}\label{husai-y-los-sacerdotes-envuxedan-mensajes-en-secreto-al-rey-david-pone-sobre-el-jorduxe1n}}

\bibleverse{15} Entonces Husai dijo a Sadoc y a los sacerdotes Abiatar:
``Ajitófel aconsejó a Absalón y a los ancianos de Israel de esta manera,
y yo he aconsejado de esta otra. \bibleverse{16} Ahora, pues, enviad
rápidamente a decir a David: ``No te alojes esta noche en los vados del
desierto, sino pasa de una vez, no sea que el rey sea devorado, y todo
el pueblo que está con él.''

\bibleverse{17} Jonatán y Ahimaas se hospedaban junto a En Rogel, y una
sirvienta iba a informarles, y ellos iban y se lo contaban al rey David,
pues no podían arriesgarse a que los vieran entrar en la ciudad.
\footnote{\textbf{17:17} 1Re 1,9} \bibleverse{18} Pero un muchacho los
vio y se lo dijo a Absalón. Entonces ambos se fueron rápidamente y
llegaron a la casa de un hombre en Bahurim, que tenía un pozo en su
patio; y bajaron allí. \bibleverse{19} La mujer tomó y extendió la
cubierta sobre la boca del pozo, y esparció sobre ella grano molido; y
no se supo nada. \bibleverse{20} Los siervos de Absalón fueron a la casa
de la mujer y le dijeron: ``¿Dónde están Ajimaas y Jonatán?'' La mujer
les dijo: ``Han pasado el arroyo de las aguas''. Cuando los buscaron y
no pudieron encontrarlos, volvieron a Jerusalén. \bibleverse{21} Después
de partir, salieron del pozo y fueron a avisar al rey David, y le
dijeron: ``Levántate y pasa rápido por encima del agua, porque así ha
aconsejado Ajitófel contra ti.''

\bibleverse{22} Entonces David se levantó, y todo el pueblo que estaba
con él, y pasaron el Jordán. A la luz de la mañana no faltaba ninguno de
ellos que no hubiera pasado el Jordán.

\hypertarget{el-suicidio-de-ahitofel}{%
\subsection{El suicidio de Ahitofel}\label{el-suicidio-de-ahitofel}}

\bibleverse{23} Cuando Ajitófel vio que su consejo no era seguido,
ensilló su asno, se levantó y se fue a su ciudad, puso en orden su casa
y se ahorcó; murió y fue enterrado en la tumba de su padre. \footnote{\textbf{17:23}
  Mat 27,5}

\hypertarget{absaluxf3n-comienza-la-persecuciuxf3n-de-david-y-le-da-a-amasa-el-mando-supremo-david-en-mahanaim}{%
\subsection{Absalón comienza la persecución de David y le da a Amasa el
mando supremo; David en
Mahanaim}\label{absaluxf3n-comienza-la-persecuciuxf3n-de-david-y-le-da-a-amasa-el-mando-supremo-david-en-mahanaim}}

\bibleverse{24} Entonces David llegó a Mahanaim. Absalón pasó el Jordán,
él y todos los hombres de Israel con él. \bibleverse{25} Absalón puso a
Amasa al frente del ejército en lugar de Joab. Amasa era hijo de un
hombre que se llamaba Ithra, el israelita, que había entrado con
Abigail, la hija de Nahas, hermana de Zeruiah, la madre de Joab.
\footnote{\textbf{17:25} 2Sam 19,13} \bibleverse{26} Israel y Absalón
acamparon en la tierra de Galaad.

\bibleverse{27} Cuando David llegó a Mahanaim, Sobi, hijo de Nahas, de
Rabá, de los hijos de Amón, y Maquir, hijo de Amiel, de Lodebar, y
Barzilai, galaadita, de Rogelim, \footnote{\textbf{17:27} 2Sam 9,4; 1Re
  2,7} \bibleverse{28} trajeron camas, cuencos, vasijas de barro, trigo,
cebada, harina, grano tostado, frijoles, lentejas, grano tostado,
\bibleverse{29} miel, mantequilla, ovejas y queso del rebaño, para que
David y el pueblo que estaba con él comieran; porque decían: ``El pueblo
está hambriento, cansado y sediento en el desierto.'' \footnote{\textbf{17:29}
  2Sam 16,2}

\hypertarget{las-uxf3rdenes-militares-de-david-salida-de-su-ejuxe9rcito}{%
\subsection{Las órdenes militares de David; Salida de su
ejército}\label{las-uxf3rdenes-militares-de-david-salida-de-su-ejuxe9rcito}}

\hypertarget{section-17}{%
\section{18}\label{section-17}}

\bibleverse{1} David contó el pueblo que estaba con él, y puso al frente
de él a capitanes de millares y a capitanes de centenas. \bibleverse{2}
David envió al pueblo, una tercera parte bajo la mano de Joab, y una
tercera parte bajo la mano de Abisai hijo de Sarvia, hermano de Joab, y
una tercera parte bajo la mano de Ittai el geteo. El rey dijo al pueblo:
``Seguramente yo también saldré con ustedes''. \footnote{\textbf{18:2}
  2Sam 15,19}

\bibleverse{3} Pero el pueblo dijo: ``No salgas, porque si huimos, no se
ocuparán de nosotros, ni si la mitad de nosotros muere, se ocuparán de
nosotros. Pero tú vales por diez mil de nosotros. Por eso, ahora es
mejor que estés dispuesto a ayudarnos a salir de la ciudad''.

\bibleverse{4} El rey les dijo: ``Haré lo que os parezca mejor''. El rey
se paró junto a la puerta, y todo el pueblo salió por cientos y por
miles. \bibleverse{5} El rey ordenó a Joab, a Abisai y a Ittai,
diciendo: ``Traten con delicadeza por mi causa al joven Absalón''. Todo
el pueblo escuchó cuando el rey ordenó a todos los capitanes acerca de
Absalón. \footnote{\textbf{18:5} 2Sam 18,12}

\hypertarget{absaluxf3n-es-derrotado-y-asesinado-por-el-mismo-joab-su-tumba}{%
\subsection{Absalón es derrotado y asesinado por el mismo Joab; su
tumba}\label{absaluxf3n-es-derrotado-y-asesinado-por-el-mismo-joab-su-tumba}}

\bibleverse{6} El pueblo salió al campo contra Israel, y la batalla fue
en el bosque de Efraín. \bibleverse{7} El pueblo de Israel fue golpeado
allí ante los siervos de David, y hubo allí una gran matanza aquel día
de veinte mil hombres. \bibleverse{8} Porque la batalla se extendió allí
sobre la superficie de todo el país, y el bosque devoró aquel día más
gente que la espada.

\bibleverse{9} Absalón se encontró con los siervos de David. Absalón iba
montado en su mula, y la mula pasó por debajo de las gruesas ramas de
una gran encina; su cabeza se enganchó en la encina, y quedó colgando
entre el cielo y la tierra; y la mula que estaba debajo de él siguió
adelante. \bibleverse{10} Un hombre lo vio y se lo comunicó a Joab,
diciendo: ``He aquí que he visto a Absalón colgado en una encina''.

\bibleverse{11} Joab dijo al hombre que se lo había contado: ``Mira que
lo has visto, ¿y por qué no lo has derribado allí mismo? Te habría dado
diez piezas de plata y una faja''.

\bibleverse{12} El hombre le dijo a Joab: ``Aunque recibiera en mi mano
mil monedas de plata, no extendería mi mano contra el hijo del rey,
porque a nuestro entender el rey os ordenó a ti, a Abisai y a Ittai,
diciendo: `Cuidad de que nadie toque al joven Absalón'. \footnote{\textbf{18:12}
  2Sam 18,5} \bibleverse{13} De lo contrario, si yo hubiera atentado
contra su vida (y no hay asunto que se le oculte al rey), tú mismo te
habrías puesto en mi contra.''

\bibleverse{14} Entonces Joab dijo: ``No voy a esperar así contigo''.
Tomó tres dardos en la mano y los clavó en el corazón de Absalón, cuando
aún estaba vivo en medio de la encina. \bibleverse{15} Diez jóvenes que
llevaban la armadura de Joab rodearon y golpearon a Absalón y lo
mataron. \bibleverse{16} Joab tocó la trompeta, y el pueblo volvió de
perseguir a Israel, pues Joab retuvo al pueblo. \bibleverse{17} Tomaron
a Absalón y lo arrojaron a un gran pozo en el bosque, y levantaron sobre
él un montón de piedras muy grande. Entonces todo Israel huyó, cada uno
a su tienda.

\bibleverse{18} Absalón, en vida, había tomado y levantado para sí la
columna que está en el valle del rey, pues dijo: ``No tengo hijo que
guarde mi nombre en la memoria''. Llamó a la columna con su propio
nombre. Hasta hoy se llama el monumento de Absalón.

\hypertarget{david-recibe-la-noticia-de-la-muerte-de-absaluxf3n-su-dolor}{%
\subsection{David recibe la noticia de la muerte de Absalón; su
dolor}\label{david-recibe-la-noticia-de-la-muerte-de-absaluxf3n-su-dolor}}

\bibleverse{19} Entonces Ahimaas, hijo de Sadoc, dijo: ``Permítanme
correr y llevarle al rey noticias de cómo Yahvé lo ha vengado de sus
enemigos.'' \footnote{\textbf{18:19} 2Sam 15,36; 2Sam 17,1}

\bibleverse{20} Joab le dijo: ``Hoy no debes ser portador de noticias,
sino que deberás llevarlas otro día. Pero hoy no debes llevar noticias,
porque el hijo del rey ha muerto''.

\bibleverse{21} Entonces Joab le dijo al cusita: ``¡Ve y dile al rey lo
que has visto!'' El cusita se inclinó ante Joab y corrió.

\bibleverse{22} Entonces Ahimaas, hijo de Sadoc, volvió a decir a Joab:
``Pero pase lo que pase, por favor, déjame también correr tras el
cusita.'' Joab dijo: ``¿Por qué quieres huir, hijo mío, ya que no
tendrás recompensa por la noticia?''

\bibleverse{23} ``Pero pase lo que pase'', dijo, ``correré''. Le dijo:
``¡Corre!'' Entonces Ahimaas corrió por el camino de la Llanura, y
superó al cusita.

\hypertarget{david-en-la-puerta-de-mahanaim-su-dolor-por-la-muerte-de-absaluxf3n}{%
\subsection{David en la puerta de Mahanaim; su dolor por la muerte de
Absalón}\label{david-en-la-puerta-de-mahanaim-su-dolor-por-la-muerte-de-absaluxf3n}}

\bibleverse{24} David estaba sentado entre las dos puertas, y el
centinela subió al techo de la puerta que da a la muralla, y alzó los
ojos y miró, y he aquí un hombre que corría solo. \bibleverse{25} El
vigilante dio un grito y se lo comunicó al rey. El rey dijo: ``Si está
solo, hay noticias en su boca''. Se acercó más y más.

\bibleverse{26} El vigilante vio a otro hombre que corría; y el
vigilante llamó al portero y le dijo: ``¡Mira, un hombre que corre
solo!'' El rey dijo: ``Él también trae noticias''.

\bibleverse{27} El vigilante dijo: ``Creo que la carrera del primero es
como la de Ajimaas, hijo de Sadoc''. El rey dijo: ``Es un buen hombre y
viene con buenas noticias''.

\bibleverse{28} Ahimaas llamó y dijo al rey: ``Todo está bien''. Se
inclinó ante el rey con el rostro hacia la tierra, y dijo: ``¡Bendito
sea Yavé, tu Dios, que ha entregado a los hombres que levantaron su mano
contra mi señor el rey!''

\bibleverse{29} El rey dijo: ``¿Está bien el joven Absalón?'' Ahimaas
respondió: ``Cuando Joab envió al siervo del rey, yo también tu siervo,
vi un gran alboroto, pero no sé qué era''.

\bibleverse{30} El rey le dijo: ``Ven y párate aquí''. Vino y se quedó
quieto.

\bibleverse{31} He aquí que vino el cusita. El cusita dijo: ``Buenas
noticias para mi señor el rey, porque Yahvé te ha vengado hoy de todos
los que se levantaron contra ti.''

\bibleverse{32} El rey dijo al cusita: ``¿Está bien el joven Absalón?''
El cusita respondió: ``Que los enemigos de mi señor el rey, y todos los
que se levanten contra ti para hacerte daño, sean como ese joven''.

\bibleverse{33} El rey, muy conmovido, subió a la sala de la puerta y
lloró. Mientras iba, dijo: ``¡Hijo mío Absalón! ¡Hijo mío, hijo mío
Absalón! Ojalá hubiera muerto yo en tu lugar, Absalón, hijo mío, hijo
mío''.

\hypertarget{section-18}{%
\section{19}\label{section-18}}

\bibleverse{1} Le dijeron a Joab: ``He aquí que el rey llora y se
lamenta por Absalón''.

\hypertarget{efecto-maligno-del-dolor-de-david-en-el-ejuxe9rcito-la-reprensiuxf3n-de-joab-david-se-levanta}{%
\subsection{Efecto maligno del dolor de David en el ejército; La
reprensión de Joab; David se
levanta}\label{efecto-maligno-del-dolor-de-david-en-el-ejuxe9rcito-la-reprensiuxf3n-de-joab-david-se-levanta}}

\bibleverse{2} La victoria de ese día se convirtió en luto en todo el
pueblo, pues el pueblo oyó decir ese día: ``El rey llora por su hijo.''

\bibleverse{3} Aquel día el pueblo se escabulló en la ciudad, como se
escabulle la gente avergonzada cuando huye en la batalla. \bibleverse{4}
El rey se cubrió el rostro, y el rey gritó en voz alta: ``¡Hijo mío
Absalón, Absalón, hijo mío, hijo mío!''

\bibleverse{5} Joab entró en la casa del rey y le dijo: ``Hoy has
avergonzado los rostros de todos tus siervos que hoy han salvado tu
vida, la de tus hijos y la de tus hijas, la de tus esposas y la de tus
concubinas; \bibleverse{6} porque amas a los que te odian y odias a los
que te aman. Porque hoy has declarado que los príncipes y los siervos no
son nada para ti. Pues hoy percibo que si Absalón hubiera vivido y todos
nosotros hubiéramos muerto hoy, entonces te habría complacido.
\bibleverse{7} Ahora, pues, levántate, sal y habla para consolar a tus
siervos; porque te juro por Yavé que si no sales, ni un solo hombre se
quedará contigo esta noche. Eso sería peor para ti que todo el mal que
te ha ocurrido desde tu juventud hasta ahora''.

\bibleverse{8} Entonces el rey se levantó y se sentó en la puerta. A
todo el pueblo se le dijo: ``He aquí que el rey está sentado en la
puerta''. Todo el pueblo se presentó ante el rey. Israel había huido
cada uno a su tienda.

\hypertarget{sobre-de-sentimiento-popular-por-david-las-negociaciones-de-david-con-los-ancianos-de-juduxe1-y-con-amasa}{%
\subsection{Sobre de sentimiento popular por David; Las negociaciones de
David con los ancianos de Judá y con
Amasa}\label{sobre-de-sentimiento-popular-por-david-las-negociaciones-de-david-con-los-ancianos-de-juduxe1-y-con-amasa}}

\bibleverse{9} Todo el pueblo estaba en lucha por todas las tribus de
Israel, diciendo: ``El rey nos libró de la mano de nuestros enemigos, y
nos salvó de la mano de los filisteos; y ahora ha huido del país de
Absalón. \bibleverse{10} Absalón, a quien ungimos sobre nosotros, ha
muerto en la batalla. Ahora, pues, ¿por qué no dices una palabra para
hacer volver al rey?''

\bibleverse{11} El rey David envió a los sacerdotes Sadoc y Abiatar,
diciendo: ``Hablad a los ancianos de Judá, diciendo: ``¿Por qué sois los
últimos en hacer volver al rey a su casa, ya que el discurso de todo
Israel ha llegado al rey, para hacerlo volver a su casa? \bibleverse{12}
Ustedes son mis hermanos. Sois mi hueso y mi carne. ¿Por qué, pues, sois
los últimos en hacer volver al rey?' \bibleverse{13} Di a Amasa: `¿No
eres tú mi hueso y mi carne? Que Dios me lo haga, y más aún, si no eres
tú el capitán del ejército delante de mí continuamente en lugar de
Joab'\,''. \footnote{\textbf{19:13} 2Sam 17,25; 1Cró 2,16-17}
\bibleverse{14} El corazón de todos los hombres de Judá se inclinó como
un solo hombre, de modo que enviaron al rey diciendo: ``Vuelve tú y
todos tus servidores.''

\bibleverse{15} El rey regresó y llegó al Jordán. Judá vino a Gilgal,
para ir al encuentro del rey, para hacer pasar al rey al otro lado del
Jordán.

\hypertarget{david-regresa-y-es-alcanzado-por-los-juduxedos-su-dulzura-hacia-simei}{%
\subsection{David regresa y es alcanzado por los judíos; su dulzura
hacia
Simei}\label{david-regresa-y-es-alcanzado-por-los-juduxedos-su-dulzura-hacia-simei}}

\bibleverse{16} Simei hijo de Gera, el benjamita, que era de Bahurim, se
apresuró a bajar con los hombres de Judá a recibir al rey David.
\footnote{\textbf{19:16} 1Re 2,8} \bibleverse{17} Lo acompañaban mil
hombres de Benjamín, y Siba, siervo de la casa de Saúl, con sus quince
hijos y sus veinte siervos; y pasaron el Jordán en presencia del rey.
\footnote{\textbf{19:17} 2Sam 16,1-4; 2Sam 9,2; 2Sam 9,10}
\bibleverse{18} Una barca fue a pasar la casa del rey, y a hacer lo que
le pareciera bien. Simei, hijo de Gera, se postró ante el rey cuando
hubo pasado el Jordán. \bibleverse{19} Le dijo al rey: ``No permita mi
señor que me impute iniquidad, ni recuerde lo que su siervo hizo
perversamente el día en que mi señor el rey salió de Jerusalén, para que
el rey lo tome en cuenta. \footnote{\textbf{19:19} 2Sam 16,5}
\bibleverse{20} Porque tu siervo sabe que he pecado. Por eso, he venido
hoy como el primero de toda la casa de José para bajar a recibir a mi
señor el rey.''

\bibleverse{21} Pero Abisai, hijo de Sarvia, respondió: ``¿No debería
morir Simei por esto, por haber maldecido al ungido de Yahvé?''

\bibleverse{22} David dijo: ``¿Qué tengo que hacer con vosotros, hijos
de Sarvia, para que seáis hoy adversarios míos? ¿Habrá que matar hoy a
alguien en Israel? ¿Acaso no sé que hoy soy rey de Israel?'' \footnote{\textbf{19:22}
  2Sam 16,10} \bibleverse{23} El rey dijo a Simei: ``No morirás''. El
rey le juró.

\bibleverse{24} Mefiboset, hijo de Saúl, bajó a recibir al rey, y no se
había aseado los pies, ni se había recortado la barba, ni se había
lavado la ropa, desde el día en que el rey partió hasta el día en que
volvió a casa en paz. \footnote{\textbf{19:24} 2Sam 9,6}

\hypertarget{mefiboset-se-justifica-a-suxed-mismo-contra-david}{%
\subsection{Mefiboset se justifica a sí mismo contra
David}\label{mefiboset-se-justifica-a-suxed-mismo-contra-david}}

\bibleverse{25} Cuando llegó a Jerusalén para recibir al rey, éste le
dijo: ``¿Por qué no has ido conmigo, Mefiboset?''

\bibleverse{26} Él respondió: ``Señor mío, oh rey, mi siervo me engañó.
Porque tu siervo dijo: `Voy a ensillar un asno para mí, para montar en
él e ir con el rey', porque tu siervo es cojo. \bibleverse{27} Él ha
calumniado a tu siervo ante mi señor el rey, pero mi señor el rey es
como un ángel de Dios. Haz, pues, lo que te parezca bien. \footnote{\textbf{19:27}
  2Sam 16,3; 2Sam 14,17} \bibleverse{28} Porque toda la casa de mi padre
no era más que hombres muertos ante mi señor el rey; sin embargo, tú
pusiste a tu siervo entre los que comían a tu mesa. ¿Qué derecho, pues,
tengo todavía para apelar más al rey?'' \footnote{\textbf{19:28} 2Sam
  9,11}

\bibleverse{29} El rey le dijo: ``¿Por qué hablas más de tus asuntos? Yo
digo que tú y Siba se repartan la tierra''. \footnote{\textbf{19:29}
  2Sam 9,9-10; 2Sam 16,4}

\bibleverse{30} Mefiboset dijo al rey: ``Sí, que se lo lleve todo,
porque mi señor el rey ha venido en paz a su casa''.

\bibleverse{31} Barzilai, el Galaadita, descendió de Rogelim, y pasó el
Jordán con el rey para conducirlo al otro lado del Jordán. \footnote{\textbf{19:31}
  1Re 2,7}

\hypertarget{cordial-conversaciuxf3n-de-barsillai-con-david-cruzando-el-jorduxe1n}{%
\subsection{Cordial conversación de Barsillai con David; Cruzando el
Jordán}\label{cordial-conversaciuxf3n-de-barsillai-con-david-cruzando-el-jorduxe1n}}

\bibleverse{32} Barzilai era un hombre muy anciano, de ochenta años. Él
le había proporcionado el sustento al rey mientras estuvo en Mahanaim,
pues era un hombre muy grande. \footnote{\textbf{19:32} 2Sam 17,27}
\bibleverse{33} El rey le dijo a Barzilai: ``Pasa conmigo, y te
mantendré conmigo en Jerusalén''.

\bibleverse{34} Barzilai dijo al rey: ``¿Cuántos son los días de los
años de mi vida, para que suba con el rey a Jerusalén? \bibleverse{35}
Hoy tengo ochenta años. ¿Puedo discernir entre lo bueno y lo malo?
¿Puede tu siervo probar lo que como o lo que bebo? ¿Acaso puedo oír ya
la voz de los hombres que cantan y de las mujeres que cantan? ¿Por qué
entonces tu siervo ha de ser una carga para mi señor el rey?
\bibleverse{36} Tu siervo sólo pasará el Jordán con el rey. ¿Por qué ha
de pagarme el rey con semejante recompensa? \bibleverse{37} Por favor,
deja que tu siervo regrese, para que yo muera en mi ciudad, junto a la
tumba de mi padre y de mi madre. Pero he aquí a tu siervo Quimam; déjalo
pasar con mi señor el rey; y haz con él lo que te parezca bien.''

\bibleverse{38} El rey respondió: ``Chimham irá conmigo, y yo haré con
él lo que te parezca bien. Todo lo que me pidas, eso haré por ti''.

\bibleverse{39} Todo el pueblo pasó el Jordán, y el rey también.
Entonces el rey besó a Barzilai y lo bendijo, y se volvió a su lugar.
\bibleverse{40} Entonces el rey pasó a Gilgal, y Quimam pasó con él.
Todo el pueblo de Judá hizo pasar al rey, y también la mitad del pueblo
de Israel. \bibleverse{41} He aquí que todos los hombres de Israel
vinieron al rey y le dijeron: ``¿Por qué nuestros hermanos, los hombres
de Judá, te han robado y han hecho pasar el Jordán al rey y a su
familia, y a todos los hombres de David con él?''

\hypertarget{celos-y-amarga-disputa-entre-israel-y-juduxe1-para-alcanzar-a-david}{%
\subsection{Celos y amarga disputa entre Israel y Judá para alcanzar a
David}\label{celos-y-amarga-disputa-entre-israel-y-juduxe1-para-alcanzar-a-david}}

\bibleverse{42} Todos los hombres de Judá respondieron a los de Israel:
``Porque el rey es un pariente cercano a nosotros. ¿Por qué, pues, os
enfadáis por este asunto? ¿Acaso hemos comido a costa del rey? ¿O nos ha
dado él algún regalo?'' \footnote{\textbf{19:42} 2Sam 19,11-12}

\bibleverse{43} Los hombres de Israel respondieron a los de Judá y
dijeron: ``Nosotros tenemos diez partes en el rey, y también tenemos más
derecho a David que ustedes. ¿Por qué, pues, nos habéis despreciado,
para que nuestro consejo no sea el primero en hacer volver a nuestro
rey?'' Las palabras de los hombres de Judá fueron más feroces que las de
los hombres de Israel.

\hypertarget{uxf3rdenes-de-david-en-jerusaluxe9n}{%
\subsection{Órdenes de David en
Jerusalén}\label{uxf3rdenes-de-david-en-jerusaluxe9n}}

\hypertarget{section-19}{%
\section{20}\label{section-19}}

\bibleverse{1} Sucedió que estaba allí un malvado, que se llamaba Seba,
hijo de Bichri, benjamita, y tocó la trompeta y dijo: ``No tenemos parte
en David, ni tenemos herencia en el hijo de Isaí. Cada uno a sus
tiendas, Israel''.

\bibleverse{2} Así que todos los hombres de Israel dejaron de seguir a
David y siguieron a Seba hijo de Bichri; pero los hombres de Judá se
unieron a su rey, desde el Jordán hasta Jerusalén.

\bibleverse{3} David llegó a su casa en Jerusalén, y el rey tomó a las
diez mujeres sus concubinas, que había dejado para guardar la casa, y
las puso en custodia y les dio sustento, pero no entró en ellas. Así
quedaron encerradas hasta el día de su muerte, viviendo en la viudez.
\footnote{\textbf{20:3} 2Sam 16,21}

\bibleverse{4} Entonces el rey le dijo a Amasa: ``Convoca a los hombres
de Judá dentro de tres días y que estén aquí presentes''.

\bibleverse{5} Entonces Amasa fue a convocar a los hombres de Judá, pero
se quedó más tiempo del que se le había señalado. \bibleverse{6} David
dijo a Abisai: ``Ahora Seba, hijo de Bichri, nos hará más daño que
Absalón. Toma a los siervos de tu señor y persíguelo, no sea que se haga
de ciudades fortificadas y se escape de nuestra vista''.

\bibleverse{7} Los hombres de Joab salieron tras él con los cereteos,
los peleteos y todos los hombres fuertes, y salieron de Jerusalén para
perseguir a Seba hijo de Bicri.

\hypertarget{el-asesinato-de-amasa-por-joab}{%
\subsection{El asesinato de Amasa por
Joab}\label{el-asesinato-de-amasa-por-joab}}

\bibleverse{8} Cuando llegaron a la gran piedra que está en Gabaón,
Amasa salió a su encuentro. Joab estaba vestido con su ropa de guerra
que se había puesto, y sobre ella tenía un fajín con una espada sujeta a
su cintura en su vaina; y mientras avanzaba se le cayó. \bibleverse{9}
Joab dijo a Amasa: ``¿Estás bien, hermano mío?'' Joab tomó a Amasa por
la barba con su mano derecha para besarlo. \footnote{\textbf{20:9} Sal
  28,3} \bibleverse{10} Pero Amasa no hizo caso de la espada que estaba
en la mano de Joab. Así que lo golpeó con ella en el cuerpo y derramó
sus entrañas en el suelo, y no lo volvió a golpear; y murió. Joab y su
hermano Abisai persiguieron a Sabá, hijo de Bicri. \footnote{\textbf{20:10}
  1Re 2,5} \bibleverse{11} Uno de los jóvenes de Joab se puso a su lado
y dijo: ``El que esté a favor de Joab y el que esté a favor de David,
que siga a Joab''.

\bibleverse{12} Amasa yacía revolcándose en su sangre en medio del
camino. Cuando el hombre vio que todo el pueblo se detenía, sacó a Amasa
de la calzada al campo, y echó un manto sobre él al ver que todos los
que pasaban por allí se detenían. \bibleverse{13} Cuando lo sacaron del
camino, todo el pueblo siguió a Joab para perseguir a Seba, hijo de
Bicri.

\hypertarget{seba-de-joab-guerrea-y-asesina-a-instigaciuxf3n-de-una-mujer-inteligente-el-regreso-de-joab-a-jerusaluxe9n}{%
\subsection{Seba de Joab guerrea y asesina a instigación de una mujer
inteligente; El regreso de Joab a
Jerusalén}\label{seba-de-joab-guerrea-y-asesina-a-instigaciuxf3n-de-una-mujer-inteligente-el-regreso-de-joab-a-jerusaluxe9n}}

\bibleverse{14} Atravesó todas las tribus de Israel hasta Abel, hasta
Bet Maaca y todos los beritas. Se reunieron y también fueron tras él.
\bibleverse{15} Llegaron y lo sitiaron en Abel de Bet Maaca, y
levantaron un montículo contra la ciudad, el cual se mantuvo contra la
muralla; y todo el pueblo que estaba con Joab golpeó la muralla para
derribarla.

\bibleverse{16} Entonces una mujer sabia gritó desde la ciudad: ``¡Oye,
oye! Di a Joab: ``Acércate, para que pueda hablar contigo''\,''.
\bibleverse{17} Él se acercó a ella, y la mujer le dijo: ``¿Eres Joab?''
Él contestó: ``Lo soy''. Entonces le dijo: ``Escucha las palabras de tu
siervo''. Respondió: ``Te escucho''.

\bibleverse{18} Entonces ella habló diciendo: ``Antiguamente decían:
`Seguramente pedirán consejo a Abel', y así resolvieron un asunto.
\bibleverse{19} Yo estoy entre los pacíficos y fieles de Israel. Ustedes
pretenden destruir una ciudad y una madre en Israel. ¿Por qué quieres
tragar la herencia de Yahvé?''

\bibleverse{20} Joab respondió: ``Lejos de mí, lejos de mí, que yo
trague o destruya. \bibleverse{21} El asunto no es así. Pero un hombre
de la región montañosa de Efraín, de nombre Seba, hijo de Bichri, ha
levantado su mano contra el rey, incluso contra David. Libéralo, y me
iré de la ciudad''. La mujer dijo a Joab: ``He aquí que su cabeza te
será arrojada por encima del muro''.

\bibleverse{22} Entonces la mujer acudió a todo el pueblo en su
sabiduría. Cortaron la cabeza de Seba, hijo de Bichri, y la arrojaron a
Joab. Este tocó la trompeta, y se dispersaron de la ciudad, cada uno a
su tienda. Entonces Joab regresó a Jerusalén ante el rey.

\hypertarget{los-altos-funcionarios-de-david}{%
\subsection{Los altos funcionarios de
David}\label{los-altos-funcionarios-de-david}}

\bibleverse{23} Joab estaba al frente de todo el ejército de Israel;
Benaía, hijo de Joiada, estaba al frente de los cereteos y de los
peleteos; \bibleverse{24} Adoram estaba al frente de los hombres
sometidos a trabajos forzados; Josafat, hijo de Ahilud, era el
registrador; \footnote{\textbf{20:24} 1Re 4,6} \bibleverse{25} Sheva era
el escriba; Sadoc y Abiatar eran los sacerdotes; \bibleverse{26} e Ira,
el jairita, era el ministro principal de David.

\hypertarget{declaraciuxf3n-de-la-deuda-de-sauxfal-el-requisito-de-los-gabaonitas}{%
\subsection{Declaración de la deuda de Saúl; el requisito de los
gabaonitas}\label{declaraciuxf3n-de-la-deuda-de-sauxfal-el-requisito-de-los-gabaonitas}}

\hypertarget{section-20}{%
\section{21}\label{section-20}}

\bibleverse{1} Hubo hambre en los días de David durante tres años, año
tras año; y David buscó el rostro de Yavé. Yavé dijo: ``Es por Saúl y
por su sangrienta casa, porque dio muerte a los gabaonitas''.

\bibleverse{2} El rey llamó a los gabaonitas y les dijo (ahora bien, los
gabaonitas no eran de los hijos de Israel, sino del remanente de los
amorreos, y los hijos de Israel les habían jurado; y Saúl trató de
matarlos en su celo por los hijos de Israel y de Judá); \footnote{\textbf{21:2}
  Jos 9,15; Jos 9,19} \bibleverse{3} y David dijo a los gabaonitas:
``¿Qué debo hacer por ustedes? ¿Y con qué debo hacer expiación, para que
bendigáis la heredad de Yahvé?''

\bibleverse{4} Los gabaonitas le dijeron: ``No es cuestión de plata ni
de oro entre nosotros y Saúl o su casa; tampoco nos corresponde dar
muerte a ningún hombre en Israel.'' Dijo: ``Haré por ti lo que digas''.

\bibleverse{5} Dijeron al rey: ``El hombre que nos consumió y que
conspiró contra nosotros para que no permaneciéramos en ninguna de las
fronteras de Israel, \bibleverse{6} que nos entreguen a siete hombres de
sus hijos, y los colgaremos a Yahvé en Guibeá de Saúl, el elegido de
Yahvé.'' El rey dijo: ``Se los daré''. \footnote{\textbf{21:6} Núm 25,4}

\hypertarget{la-promesa-de-david-y-su-ejecuciuxf3n-a-la-familia-de-sauxfal}{%
\subsection{La promesa de David y su ejecución a la familia de
Saúl}\label{la-promesa-de-david-y-su-ejecuciuxf3n-a-la-familia-de-sauxfal}}

\bibleverse{7} Pero el rey perdonó a Mefiboset, hijo de Jonatán, hijo de
Saúl, a causa del juramento de Yahvé que había entre ellos, entre David
y Jonatán, hijo de Saúl. \footnote{\textbf{21:7} 1Sam 20,15-17}
\bibleverse{8} Pero el rey tomó a los dos hijos de Rizpa, hija de Aja,
que ella había dado a luz a Saúl, Armoní y Mefiboset, y a los cinco
hijos de Merab, hija de Saúl, que ella había dado a luz a Adriel, hijo
de Barzilái el meholatí. \footnote{\textbf{21:8} 2Sam 3,7; 1Sam 18,19}
\bibleverse{9} Los entregó en manos de los gabaonitas, y los colgaron en
el monte delante de Yavé, y los siete cayeron juntos. Los mataron en los
días de la cosecha, en los primeros días, al comienzo de la cosecha de
cebada.

\hypertarget{la-maravillosa-muestra-de-amor-de-rizpa-sepultura-de-los-huesos-de-sauxfal-y-sus-descendientes}{%
\subsection{La maravillosa muestra de amor de Rizpa; Sepultura de los
huesos de Saúl y sus
descendientes}\label{la-maravillosa-muestra-de-amor-de-rizpa-sepultura-de-los-huesos-de-sauxfal-y-sus-descendientes}}

\bibleverse{10} Rizpa, hija de Aja, tomó un saco y lo extendió para sí
misma sobre la roca, desde el comienzo de la cosecha hasta que el agua
se derramó sobre ellos desde el cielo. No permitió que las aves del
cielo se posaran sobre ellos de día, ni los animales del campo de noche.
\bibleverse{11} A David le contaron lo que había hecho Rizpa, hija de
Aia, la concubina de Saúl. \bibleverse{12} Entonces David fue y tomó los
huesos de Saúl y los huesos de su hijo de los hombres de Jabes de
Galaad, que los habían robado de la calle de Bet Shan, donde los
filisteos los habían colgado el día que los filisteos mataron a Saúl en
Gilboa; \footnote{\textbf{21:12} 1Sam 31,12} \bibleverse{13} y sacó de
allí los huesos de Saúl y los huesos de su hijo. También recogieron los
huesos de los ahorcados. \bibleverse{14} Enterraron los huesos de Saúl y
de su hijo en el país de Benjamín, en Zela, en la tumba de Cis, su
padre; y cumplieron todo lo que el rey les ordenó. Después de eso, Dios
respondió a la oración por la tierra. \footnote{\textbf{21:14} 2Sam
  24,25}

\hypertarget{algunas-hazauxf1as-de-los-guerreros-de-david-en-las-guerras-filisteas}{%
\subsection{Algunas hazañas de los guerreros de David en las guerras
filisteas}\label{algunas-hazauxf1as-de-los-guerreros-de-david-en-las-guerras-filisteas}}

\bibleverse{15} Los filisteos volvieron a hacer la guerra a Israel; y
David descendió, y sus siervos con él, y lucharon contra los filisteos.
David desfallecía; \bibleverse{16} e Isbibenob, que era de los hijos del
gigante, cuyo peso de la lanza era de trescientos siclos de bronce,
estando armado con una espada nueva, pensó en matar a David.
\bibleverse{17} Pero Abisai, hijo de Sarvia, lo ayudó, e hirió al
filisteo y lo mató. Entonces los hombres de David le juraron: ``No
salgas más con nosotros a combatir, para que no apagues la lámpara de
Israel''. \footnote{\textbf{21:17} 2Sam 23,18}

\bibleverse{18} Después de esto, volvió a haber guerra con los filisteos
en Gob. Entonces Sibbecai, el husatita, mató a Saf, que era de los hijos
del gigante. \footnote{\textbf{21:18} 1Cró 20,4-8} \bibleverse{19}
Volvió a haber guerra con los filisteos en Gob, y Elhanán, hijo de
Jaare-Oregim, betlemita, mató al hermano de Goliat, el gitita, cuyo asta
era como un haz de telar. \footnote{\textbf{21:19} 1Sam 17,7}
\bibleverse{20} Volvió a haber guerra en Gat, donde había un hombre de
gran estatura, que tenía seis dedos en cada mano y seis dedos en cada
pie, veinticuatro en total, y también era hijo del gigante.
\bibleverse{21} Cuando desafió a Israel, lo mató Jonatán, hijo de Simei,
hermano de David. \footnote{\textbf{21:21} 1Sam 17,10} \bibleverse{22}
Estos cuatro le nacieron al gigante en Gat, y cayeron por la mano de
David y por la de sus servidores.

\hypertarget{el-cuxe1ntico-de-acciuxf3n-de-gracias-y-victoria-de-david-despuuxe9s-de-derrotar-a-sus-enemigos}{%
\subsection{El cántico de acción de gracias y victoria de David después
de derrotar a sus
enemigos}\label{el-cuxe1ntico-de-acciuxf3n-de-gracias-y-victoria-de-david-despuuxe9s-de-derrotar-a-sus-enemigos}}

\hypertarget{section-21}{%
\section{22}\label{section-21}}

\bibleverse{1} David dirigió a Yahvé las palabras de este cántico el día
en que Yahvé lo libró de la mano de todos sus enemigos y de la mano de
Saúl, \bibleverse{2} y dijo: ``Yahvé es mi roca, mi fortaleza, y mi
libertador, incluso el mío; \bibleverse{3} Dios es mi roca en la que me
refugio; mi escudo, y el cuerno de mi salvación, mi alta torre, y mi
refugio. Mi salvador, me salvas de la violencia. \bibleverse{4} Invoco a
Yahvé, que es digno de ser alabado; Así me salvaré de mis enemigos.
\bibleverse{5} Porque las olas de la muerte me rodearon. Las
inundaciones de la impiedad me dieron miedo. \bibleverse{6} Las cuerdas
del Seol\footnote{\textbf{22:6} El Seol es el lugar de los muertos.} me
rodeaban. Las trampas de la muerte me atraparon. \bibleverse{7} En mi
angustia, invoqué a Yahvé. Sí, llamé a mi Dios. Escuchó mi voz fuera de
su templo. Mi grito llegó a sus oídos. \bibleverse{8} Entonces la tierra
se estremeció y tembló. Los cimientos del cielo temblaron y fueron
sacudidos, porque estaba enfadado. \bibleverse{9} Salió humo de sus
fosas nasales. De su boca salió fuego consumidor. Las brasas se
encendieron con él. \bibleverse{10} También inclinó los cielos y
descendió. La espesa oscuridad estaba bajo sus pies. \bibleverse{11}
Montó en un querubín y voló. Sí, fue visto en las alas del viento.
\bibleverse{12} Hizo de las tinieblas un refugio a su alrededor, la
reunión de las aguas, y las espesas nubes de los cielos. \bibleverse{13}
Ante el resplandor de la luz, se encendieron las brasas del fuego.
\bibleverse{14} Yahvé tronó desde el cielo. El Altísimo emitió su voz.
\bibleverse{15} Envió flechas y los dispersó, rayos y los confundió.
\bibleverse{16} Entonces aparecieron los canales del mar. Los cimientos
del mundo quedaron al descubierto por la reprimenda de Yahvé, al soplo
de sus fosas nasales. \bibleverse{17} Envió desde lo alto y me llevó. Me
sacó de muchas aguas. \bibleverse{18} Me libró de mi fuerte enemigo, de
los que me odiaban, porque eran demasiado poderosos para mí.
\bibleverse{19} Vinieron sobre mí en el día de mi calamidad, pero Yahvé
fue mi apoyo. \bibleverse{20} También me llevó a un lugar grande. Me
liberó, porque se deleitó en mí. \bibleverse{21} El Señor me recompensó
según mi justicia. Me recompensó según la limpieza de mis manos.
\bibleverse{22} Porque he guardado los caminos de Yahvé, y no me he
alejado impíamente de mi Dios. \bibleverse{23} Porque todas sus
ordenanzas estaban delante de mí. En cuanto a sus estatutos, no me
aparté de ellos. \bibleverse{24} Yo también fui perfecto con él. Me
guardé de mi iniquidad. \bibleverse{25} Por lo tanto, Yahvé me ha
recompensado según mi justicia, Según mi limpieza en la vista.
\bibleverse{26} Con los misericordiosos te mostrarás misericordioso. Con
el hombre perfecto te mostrarás perfecta. \bibleverse{27} Con los puros
te mostrarás puro. Con lo torcido te mostrarás astuto. \bibleverse{28}
Tú salvarás al pueblo afligido, pero tus ojos están puestos en los
arrogantes, para derribarlos. \bibleverse{29} Porque tú eres mi lámpara,
Yahvé. Yahvé iluminará mis tinieblas. \bibleverse{30} Por ti, corro
contra una tropa. Por Dios, salto un muro. \bibleverse{31} En cuanto a
Dios, su camino es perfecto. La palabra de Yahvé se pone a prueba. Es un
escudo para todos los que se refugian en él. \bibleverse{32} Porque
¿quién es Dios, además de Yahvé? ¿Quién es una roca, además de nuestro
Dios? \bibleverse{33} Dios es mi fortaleza. Él hace que mi camino sea
perfecto. \bibleverse{34} Hace que sus pies sean como los de las
ciervas, y me pone en mis alturas. \bibleverse{35} Enseña mis manos a la
guerra, para que mis brazos doblen un arco de bronce. \bibleverse{36}
También me has dado el escudo de tu salvación. Tu gentileza me ha hecho
grande. \bibleverse{37} Has ensanchado mis pasos debajo de mí. Mis pies
no han resbalado. \bibleverse{38} He perseguido a mis enemigos y los he
destruido. No volví a girar hasta que se consumieron. \bibleverse{39}
Los he consumido, y los atravesó, para que no puedan surgir. Sí, han
caído bajo mis pies. \bibleverse{40} Porque me has armado de fuerza para
la batalla. Has sometido bajo mi mando a los que se levantaron contra
mí. \bibleverse{41} También has hecho que mis enemigos me den la
espalda, para cortar a los que me odian. \bibleverse{42} Miraron, pero
no había nadie a quien salvar; incluso a Yahvé, pero no les respondió.
\bibleverse{43} Entonces los hice tan pequeños como el polvo de la
tierra. Los aplasté como el fango de las calles, y los esparcí por todas
partes. \bibleverse{44} Tú también me has librado de los esfuerzos de mi
pueblo. Me has guardado para ser la cabeza de las naciones. Un pueblo
que no he conocido me servirá. \bibleverse{45} Los extranjeros se
someterán a mí. En cuanto oigan hablar de mí, me obedecerán.
\bibleverse{46} Los extranjeros se desvanecerán, y saldrán temblando de
sus lugares cerrados. \bibleverse{47} ¡Yahvé vive! ¡Bendita sea mi roca!
Exaltado sea Dios, la roca de mi salvación, \bibleverse{48} incluso el
Dios que ejecuta la venganza por mí, que hace caer a los pueblos debajo
de mí, \bibleverse{49} que me aleja de mis enemigos. Sí, me elevas por
encima de los que se levantan contra mí. Líbrame del hombre violento.
\bibleverse{50} Por eso te daré gracias, Yahvé, entre las naciones, y
cantarán alabanzas a tu nombre. \bibleverse{51} Da una gran liberación a
su rey, y muestra una bondad amorosa a su ungido, a David y a su
descendencia, para siempre''.

\hypertarget{las-uxfaltimas-palabras-de-david}{%
\subsection{Las últimas palabras de
David}\label{las-uxfaltimas-palabras-de-david}}

\hypertarget{section-22}{%
\section{23}\label{section-22}}

\bibleverse{1} Estas son las últimas palabras de David. David el hijo de
Jesé dice, el hombre que fue elevado a lo alto dice, el ungido del Dios
de Jacob, el dulce salmista de Israel: \bibleverse{2} ``El Espíritu de
Yahvé habló por mí. Su palabra estaba en mi lengua. \bibleverse{3} El
Dios de Israel dijo, la Roca de Israel me habló, El que gobierna a los
hombres con rectitud, que gobierna en el temor de Dios, \bibleverse{4}
será como la luz de la mañana cuando sale el sol, una mañana sin nubes,
cuando la hierba tierna brota de la tierra, a través de un claro
resplandor después de la lluvia''. \bibleverse{5} ¿No es así mi casa con
Dios? Sin embargo, ha hecho conmigo un pacto eterno, ordenado en todas
las cosas, y seguro, porque es toda mi salvación y todo mi deseo. ¿No lo
hará crecer? \bibleverse{6} Pero todos los impíos serán como espinas que
hay que arrancar, porque no se pueden coger con la mano. \bibleverse{7}
El hombre que los toque debe estar armado con hierro y el bastón de una
lanza. Serán totalmente quemados con fuego en su lugar''.

\hypertarget{directorio-y-hazauxf1as-de-los-guerreros-de-david}{%
\subsection{Directorio y hazañas de los guerreros de
David}\label{directorio-y-hazauxf1as-de-los-guerreros-de-david}}

\bibleverse{8} Estos son los nombres de los valientes que tuvo David
Josheb Basshebeth, tahchemonita, jefe de los capitanes; se llamaba Adino
el eznita, que mató a ochocientos de una vez. \bibleverse{9} Después de
él estaba Eleazar hijo de Dodai, hijo de un ahohita, uno de los tres
valientes que estaban con David cuando desafiaron a los filisteos que
estaban allí reunidos para la batalla, y los hombres de Israel se habían
marchado. \bibleverse{10} Se levantó y golpeó a los filisteos hasta que
su mano se cansó, y su mano se congeló a la espada; y Yahvé obró una
gran victoria aquel día, y el pueblo volvió tras él sólo para tomar
botín. \bibleverse{11} Después de él fue Samma hijo de Agee, un harareo.
Los filisteos se habían reunido en tropa donde había un terreno lleno de
lentejas; y el pueblo huyó de los filisteos. \bibleverse{12} Pero él se
puso en medio de la parcela y la defendió, y mató a los filisteos; y el
Señor obtuvo una gran victoria.

\hypertarget{riesgo-de-tres-huxe9roes}{%
\subsection{Riesgo de tres héroes}\label{riesgo-de-tres-huxe9roes}}

\bibleverse{13} Tres de los treinta jefes descendieron y vinieron a
David en el tiempo de la cosecha, a la cueva de Adulam; y la tropa de
los filisteos estaba acampada en el valle de Refaim. \bibleverse{14}
David estaba entonces en la fortaleza, y la guarnición de los filisteos
estaba entonces en Belén. \bibleverse{15} David decía con nostalgia:
``¡Oh, si alguien me diera de beber agua del pozo de Belén, que está
junto a la puerta!''

\bibleverse{16} Los tres valientes irrumpieron en el ejército de los
filisteos y sacaron agua del pozo de Belén que estaba junto a la puerta,
la tomaron y se la llevaron a David; pero éste no quiso beber de ella,
sino que la derramó a Yahvé. \bibleverse{17} Él dijo: ``¡Lejos de mí,
Yahvé, que yo haga esto! ¿No es ésta la sangre de los hombres que
arriesgaron su vida para ir?'' Por eso no quiso beberla. Los tres
hombres poderosos hicieron estas cosas.

\hypertarget{abisai-y-benaja}{%
\subsection{Abisai y Benaja}\label{abisai-y-benaja}}

\bibleverse{18} Abisai, hermano de Joab, hijo de Sarvia, era el jefe de
los tres. Levantó su lanza contra trescientos y los mató, y tuvo un
nombre entre los tres. \footnote{\textbf{23:18} 2Sam 21,17}
\bibleverse{19} ¿No era él el más honrado de los tres? Por eso fue
nombrado su capitán. Sin embargo, no fue incluido como uno de los tres.

\bibleverse{20} Benaía, hijo de Joiada, hijo de un valiente de Kabzeel,
que había hecho obras poderosas, mató a los dos hijos de Ariel de Moab.
También bajó y mató a un león en medio de un pozo en tiempo de nieve.
\footnote{\textbf{23:20} Jos 15,21; Neh 11,25} \bibleverse{21} Mató a un
egipcio enorme, y el egipcio tenía una lanza en la mano; pero él bajó
hacia él con un bastón y arrancó la lanza de la mano del egipcio, y lo
mató con su propia lanza. \bibleverse{22} Benaía, hijo de Joiada, hizo
estas cosas y tuvo un nombre entre los tres valientes. \bibleverse{23}
Era más honorable que los treinta, pero no llegó a los tres. David lo
puso al frente de su guardia.

\hypertarget{una-lista-de-otros-huxe9roes-de-david}{%
\subsection{Una lista de otros héroes de
David}\label{una-lista-de-otros-huxe9roes-de-david}}

\bibleverse{24} Asael hermano de Joab era uno de los treinta: Elhanan
hijo de Dodo de Belén, \footnote{\textbf{23:24} 2Sam 2,18}
\bibleverse{25} Samma de Harod, Elika de Harod, \bibleverse{26} Helez de
Palti, Ira hijo de Ikkesh de Teko, \footnote{\textbf{23:26} 1Cró 27,9-10}
\bibleverse{27} Abiezer de Anatot, Mebunnai de Hushat, \bibleverse{28}
Zalmón ahohita, Maharai netofatita, \bibleverse{29} Heleb hijo de Baana
netofatita, Ittai hijo de Ribai de Gabaa de los hijos de Benjamín,
\bibleverse{30} Benaía piratonita, Hiddai de los arroyos de Gaas.
\bibleverse{31} Abialbón arbateo, Azmaveth barhumita, \bibleverse{32}
Eliahba saalbonita, los hijos de Jasén, Jonatán, \bibleverse{33} Shammah
hararita, Ahiam hijo de Sharar ararita, \bibleverse{34} Eliphelet hijo
de Ahasbai, hijo del maacateo, Eliam hijo de Ajitofel gilonita,
\footnote{\textbf{23:34} 2Sam 15,12} \bibleverse{35} Hezro el carmelita,
Paarai el arbita, \bibleverse{36} Igal hijo de Natán de Soba, Bani el
gadita, \bibleverse{37} Zelek el amonita, Naharai el beerotita,
portadores de armaduras de Joab hijo de Sarvia, \bibleverse{38} Ira el
itrita, Gareb el itrita, \bibleverse{39} y Urías el hitita: treinta y
siete en total. \footnote{\textbf{23:39} 2Sam 11,3}

\hypertarget{david-decide-el-censo-a-pesar-de-la-advertencia-de-joab}{%
\subsection{David decide el censo a pesar de la advertencia de
Joab}\label{david-decide-el-censo-a-pesar-de-la-advertencia-de-joab}}

\hypertarget{section-23}{%
\section{24}\label{section-23}}

\bibleverse{1} Nuevamente ardió la ira de Yahvé contra Israel, y movió a
David contra ellos, diciendo: ``Ve, cuenta a Israel y a Judá''.
\footnote{\textbf{24:1} 2Sam 21,1} \bibleverse{2} El rey dijo a Joab, el
capitán del ejército, que estaba con él: ``Ve ahora de un lado a otro
por todas las tribus de Israel, desde Dan hasta Beerseba, y cuenta el
pueblo, para que yo sepa la suma del pueblo.''

\bibleverse{3} Joab dijo al rey: ``Ahora, que el Señor, tu Dios, añada
al pueblo, por más que sea, cien veces, y que los ojos de mi señor el
rey lo vean. Pero, ¿por qué se complace mi señor el rey en esto?''.

\bibleverse{4} No obstante, la palabra del rey prevaleció contra Joab y
contra los capitanes del ejército. Joab y los capitanes del ejército
salieron de la presencia del rey para contar al pueblo de Israel.

\hypertarget{ejecuciuxf3n-del-censo-y-su-resultado}{%
\subsection{Ejecución del censo y su
resultado}\label{ejecuciuxf3n-del-censo-y-su-resultado}}

\bibleverse{5} Pasaron el Jordán y acamparon en Aroer, a la derecha de
la ciudad que está en medio del valle de Gad, y hasta Jazer;
\bibleverse{6} luego llegaron a Galaad y a la tierra de Tahtim Hodshi; y
llegaron a Dan Jaan y alrededor de Sidón, \bibleverse{7} y llegaron a la
fortaleza de Tiro, y a todas las ciudades de los heveos y de los
cananeos; y salieron al sur de Judá, en Beerseba. \bibleverse{8} Después
de recorrer todo el país, llegaron a Jerusalén al cabo de nueve meses y
veinte días. \footnote{\textbf{24:8} Jos 18,9} \bibleverse{9} Joab
entregó al rey la suma del recuento del pueblo; y había en Israel
ochocientos mil hombres valientes que sacaban la espada, y los de Judá
eran quinientos mil hombres.

\hypertarget{el-arrepentimiento-de-david-intervenciuxf3n-del-profeta-gad-david-elige-una-muerte-popular-para-expiar-su-culpa-la-penitencia-y-la-suxfaplica-de-david}{%
\subsection{El arrepentimiento de David; Intervención del profeta Gad;
David elige una muerte popular para expiar su culpa; La penitencia y la
súplica de
David}\label{el-arrepentimiento-de-david-intervenciuxf3n-del-profeta-gad-david-elige-una-muerte-popular-para-expiar-su-culpa-la-penitencia-y-la-suxfaplica-de-david}}

\bibleverse{10} El corazón de David se conmovió después de haber contado
al pueblo. David dijo a Yahvé: ``He pecado mucho en lo que he hecho.
Pero ahora, Yahvé, quita, te lo ruego, la iniquidad de tu siervo; porque
he actuado con mucha insensatez.''

\bibleverse{11} Cuando David se levantó por la mañana, llegó la palabra
de Yahvé al profeta Gad, vidente de David, diciendo: \bibleverse{12}
``Ve y habla a David: ``Yahvé dice: ``Te ofrezco tres cosas. Elige una
de ellas, para que te la haga''\,''.

\bibleverse{13} Gad vino a David y le dijo: ``¿Te vendrán siete años de
hambre en tu tierra? ¿O huirás tres meses ante tus enemigos mientras te
persiguen? ¿O habrá tres días de pestilencia en tu tierra? Responde
ahora, y considera qué respuesta daré al que me ha enviado''.
\footnote{\textbf{24:13} Jer 24,10; Jer 29,17; Ezeq 6,12}

\bibleverse{14} David dijo a Gad: ``Estoy en apuros. Caigamos ahora en
la mano de Yahvé, porque sus misericordias son grandes. No caiga en la
mano del hombre''.

\bibleverse{15} El Señor envió una peste sobre Israel desde la mañana
hasta la hora señalada, y murieron setenta mil hombres del pueblo, desde
Dan hasta Beerseba. \bibleverse{16} Cuando el ángel extendió su mano
hacia Jerusalén para destruirla, Yahvé se despreocupó del desastre y
dijo al ángel que destruía al pueblo: ``Es suficiente. Ahora retira tu
mano''. El ángel de Yahvé estaba junto a la era de Arauná el jebuseo.

\bibleverse{17} David habló a Yahvé cuando vio al ángel que golpeaba al
pueblo, y dijo: ``He aquí que he pecado y he obrado perversamente; pero
estas ovejas, ¿qué han hecho? Por favor, que tu mano esté contra mí y
contra la casa de mi padre''. \footnote{\textbf{24:17} Núm 16,22}

\hypertarget{montaje-de-un-altar-en-la-era-de-arawnas-fin-de-la-plaga}{%
\subsection{Montaje de un altar en la era de Arawnas; Fin de la
plaga}\label{montaje-de-un-altar-en-la-era-de-arawnas-fin-de-la-plaga}}

\bibleverse{18} Gad vino aquel día a David y le dijo: ``Sube y construye
un altar a Yahvé en la era de Arauna el jebuseo.''

\bibleverse{19} David subió según el dicho de Gad, tal como lo había
ordenado el Señor. \bibleverse{20} Arauna se asomó y vio que el rey y
sus servidores se acercaban a él. Entonces Arauna salió y se inclinó
ante el rey con el rostro en tierra. \bibleverse{21} Arauna dijo: ``¿Por
qué ha venido mi señor el rey a su siervo?'' David dijo: ``Para comprar
tu era, para construir un altar a Yahvé, para que la plaga deje de
afligir al pueblo''.

\bibleverse{22} Arauna dijo a David: ``Que mi señor el rey tome y
ofrezca lo que le parezca bien. He aquí el ganado para el holocausto, y
los trillos y los yugos de los bueyes para la leña. \bibleverse{23} Todo
esto, oh rey, lo da Arauna al rey''. Arauna dijo al rey: ``Que Yahvé, tu
Dios, te acepte''.

\bibleverse{24} El rey le dijo a Arauna: ``No, pero ciertamente te lo
compraré por un precio. No ofreceré a Yahvé mi Dios holocaustos que no
me cuestan nada''. Así que David compró la era y los bueyes por
cincuenta siclos\footnote{\textbf{24:24} Un siclo equivale a unos 10
  gramos o a unas 0,35 onzas, por lo que 50 siclos equivalen a unos 0,5
  kilogramos o 1,1 libras.} de plata. \bibleverse{25} David construyó
allí un altar a Yavé, y ofreció holocaustos y ofrendas de paz. Así se
suplicó a Yahvé por la tierra, y la plaga se alejó de Israel.
\footnote{\textbf{24:25} 2Sam 21,14}
