\hypertarget{bendiciones}{%
\subsection{Bendiciones}\label{bendiciones}}

\hypertarget{section}{%
\section{1}\label{section}}

\bibleverse{1} Pablo, llamado a ser apóstol de Jesucristo\footnote{\textbf{1:1}
  ``Cristo'' significa ``Ungido''.} por la voluntad de Dios, y nuestro
hermano Sóstenes, \bibleverse{2} a la asamblea de Dios que está en
Corinto: los santificados en Cristo Jesús, llamados santos, con todos
los que invocan el nombre de nuestro Señor Jesucristo en todo lugar,
tanto de ellos como de nosotros: \footnote{\textbf{1:2} 1Cor 6,11; Hech
  9,14; Hech 18,1-7} \bibleverse{3} Gracia a vosotros y paz de parte de
Dios nuestro Padre y del Señor Jesucristo.

\hypertarget{acciuxf3n-de-gracias-del-apuxf3stol-por-la-rica-gracia-de-dios-que-cayuxf3-sobre-los-corintios-esperanza-segura-para-el-futuro}{%
\subsection{Acción de gracias del apóstol por la rica gracia de Dios que
cayó sobre los corintios; esperanza segura para el
futuro}\label{acciuxf3n-de-gracias-del-apuxf3stol-por-la-rica-gracia-de-dios-que-cayuxf3-sobre-los-corintios-esperanza-segura-para-el-futuro}}

\bibleverse{4} Siempre doy gracias a mi Dios respecto a vosotros por la
gracia de Dios que os fue dada en Cristo Jesús, \bibleverse{5} que en
todo fuisteis enriquecidos en él, en toda palabra y en toda sabiduria
--- \bibleverse{6} así como el testimonio de Cristo fue confirmado en
vosotros --- \bibleverse{7} para que no os quedéis atrás en ningún don,
esperando la revelación de nuestro Señor Jesucristo, \footnote{\textbf{1:7}
  Tit 2,13; 2Pe 3,13-14} \bibleverse{8} que también os confirmará hasta
el fin, irreprochables en el día de nuestro Señor Jesucristo.
\footnote{\textbf{1:8} Fil 1,6; 1Tes 3,13} \bibleverse{9} Fiel es Dios,
por quien fuisteis llamados a la comunión de su Hijo Jesucristo, nuestro
Señor. \footnote{\textbf{1:9} 1Tes 5,24}

\hypertarget{contiendas-en-la-iglesia}{%
\subsection{Contiendas en la iglesia}\label{contiendas-en-la-iglesia}}

\bibleverse{10} Ahora os ruego, hermanos, por el nombre de nuestro Señor
Jesucristo, que habléis todos una misma cosa, y que no haya divisiones
entre vosotros, sino que os perfeccionéis juntos en una misma mente y en
un mismo juicio. \footnote{\textbf{1:10} 1Cor 11,18; Rom 15,5; Fil 2,2}
\bibleverse{11} Porque se me ha informado acerca de vosotros, hermanos
míos, por parte de los que son de la casa de Cloe, que hay disputas
entre vosotros. \bibleverse{12} Quiero decir que cada uno de vosotros
dice: ``Yo sigo a Pablo'', ``Yo sigo a Apolos'', ``Yo sigo a Cefas'' y
``Yo sigo a Cristo''. \footnote{\textbf{1:12} 1Cor 3,4; Hech 18,24-27;
  Juan 1,42} \bibleverse{13} ¿Está dividido Cristo? ¿Fue Pablo
crucificado por vosotros? ¿O fuisteis bautizados en el nombre de Pablo?
\bibleverse{14} Doy gracias a Dios porque no bauticé a ninguno de
vosotros, excepto a Crispo y a Gayo, \footnote{\textbf{1:14} Hech 18,8;
  Rom 16,23} \bibleverse{15} para que nadie diga que os bauticé en mi
propio nombre. \bibleverse{16} (También bauticé a la casa de Estéfanas;
además de ellos, no sé si bauticé a algún otro). \footnote{\textbf{1:16}
  1Cor 16,15} \bibleverse{17} Porque Cristo no me ha enviado a bautizar,
sino a predicar la Buena Nueva, no con sabiduría de palabras, para que
la cruz de Cristo no sea anulada.

\hypertarget{la-palabra-de-la-cruz-es-un-poder-divino-opuesto-a-la-sabiduruxeda-mundial-y-respetado-por-el-mundo-como-una-locura}{%
\subsection{La palabra de la cruz es un poder divino, opuesto a la
sabiduría mundial y respetado por el mundo como una
locura}\label{la-palabra-de-la-cruz-es-un-poder-divino-opuesto-a-la-sabiduruxeda-mundial-y-respetado-por-el-mundo-como-una-locura}}

\bibleverse{18} Porque la palabra de la cruz es una tontería para los
que mueren, pero para los que se salvan es poder de Dios. \footnote{\textbf{1:18}
  2Cor 4,3; Rom 1,16} \bibleverse{19} Porque está escrito, ``Destruiré
la sabiduría de los sabios. Haré que el discernimiento de los
perspicaces quede en nada\footnote{\textbf{1:19} Isaías 29:14} ''.

\bibleverse{20} ¿Dónde está el sabio? ¿Dónde está el escriba? ¿Dónde
está el polemista de este siglo? ¿Acaso Dios no ha hecho insensata la
sabiduría de este mundo? \footnote{\textbf{1:20} Rom 1,22; Mat 11,25}
\bibleverse{21} Pues viendo que en la sabiduría de Dios, el mundo por su
sabiduría no conoció a Dios, a Dios le agradó salvar a los creyentes por
medio de la locura de la predicación. \bibleverse{22} Porque los judíos
piden señales, los griegos buscan sabiduría, \footnote{\textbf{1:22} Mat
  12,38; Juan 4,48; Hech 17,18-21} \bibleverse{23} pero nosotros
predicamos a Cristo crucificado, escándalo para los judíos y necedad
para los griegos, \footnote{\textbf{1:23} 1Cor 2,14; Gal 5,11; Hech
  17,32} \bibleverse{24} pero para los llamados, tanto judíos como
griegos, Cristo es poder de Dios y sabiduría de Dios; \footnote{\textbf{1:24}
  Col 2,3} \bibleverse{25} porque la necedad de Dios es más sabia que
los hombres, y la debilidad de Dios es más fuerte que los hombres.

\hypertarget{prueba-de-la-existencia-real-de-la-comunidad-cristiana-llamada-por-dios-en-corinto}{%
\subsection{Prueba de la existencia real de la comunidad cristiana
llamada por Dios en
Corinto}\label{prueba-de-la-existencia-real-de-la-comunidad-cristiana-llamada-por-dios-en-corinto}}

\bibleverse{26} Porque ya veis vuestra vocación, hermanos, que no hay
muchos sabios según la carne, ni muchos poderosos, ni muchos nobles;
\footnote{\textbf{1:26} Juan 7,48; Sant 2,1-5} \bibleverse{27} sino que
Dios eligió lo necio del mundo para avergonzar a los sabios. Dios eligió
a los débiles del mundo para avergonzar a los fuertes. \bibleverse{28}
Dios eligió lo humilde del mundo, lo despreciable y lo que no existe,
para reducir a la nada lo que existe, \bibleverse{29} a fin de que nadie
se jacte ante Dios. \footnote{\textbf{1:29} Rom 3,27; Efes 2,9}
\bibleverse{30} Porque por él estáis en Cristo Jesús, que nos fue hecho
sabiduría de Dios, y justicia y santificación, y redención, \footnote{\textbf{1:30}
  Jer 23,5-6; 2Cor 5,21; Juan 17,19; Mat 20,28} \bibleverse{31} para
que, como está escrito: ``El que se gloríe, que se gloríe en el Señor''.
\footnote{\textbf{1:31} Jeremías 9:24} \footnote{\textbf{1:31} 2Cor
  10,17}

\hypertarget{la-manera-de-predicar-de-pablo-cuando-se-funduxf3-la-iglesia-era-poco-exigente-y-carecuxeda-de-sabiduruxeda-mundana}{%
\subsection{La manera de predicar de Pablo cuando se fundó la iglesia
era poco exigente y carecía de sabiduría
mundana}\label{la-manera-de-predicar-de-pablo-cuando-se-funduxf3-la-iglesia-era-poco-exigente-y-carecuxeda-de-sabiduruxeda-mundana}}

\hypertarget{section-1}{%
\section{2}\label{section-1}}

\bibleverse{1} Cuando fui a vosotros, hermanos, no fui con excelencia de
palabra o de sabiduría, anunciándoos el testimonio de Dios.
\bibleverse{2} Porque me propuse no conocer nada entre vosotros, sino a
Jesucristo y a éste crucificado. \footnote{\textbf{2:2} Gal 6,14}
\bibleverse{3} Estuve con vosotros con debilidad, con temor y con mucho
temblor. \footnote{\textbf{2:3} Hech 18,9; 2Cor 10,1; Gal 4,13}
\bibleverse{4} Mi discurso y mi predicación no fueron con palabras
persuasivas de sabiduría humana, sino con la demostración del Espíritu y
del poder, \footnote{\textbf{2:4} Mat 10,20} \bibleverse{5} para que
vuestra fe no permaneciera en la sabiduría de los hombres, sino en el
poder de Dios. \footnote{\textbf{2:5} 1Tes 1,5}

\hypertarget{la-misteriosa-sabiduruxeda-de-dios-para-los-perfectos}{%
\subsection{La misteriosa sabiduría de Dios para los
perfectos}\label{la-misteriosa-sabiduruxeda-de-dios-para-los-perfectos}}

\bibleverse{6} Sin embargo, hablamos de la sabiduría de los que ya han
crecido, pero una sabiduría que no es de este mundo ni de los
gobernantes de este mundo que están llegando a la nada. \bibleverse{7}
Pero hablamos la sabiduría de Dios en un misterio, la sabiduría que ha
estado oculta, que Dios preordenó antes de los mundos para nuestra
gloria, \footnote{\textbf{2:7} Rom 14,24; Mat 11,24} \bibleverse{8} que
ninguno de los gobernantes de este mundo ha conocido. Porque si la
hubieran conocido, no habrían crucificado al Señor de la gloria.
\bibleverse{9} Pero como está escrito, ``Cosas que un ojo no vio, y un
oído no oyó, que no entró en el corazón del hombre, que Dios ha
preparado para los que le aman\footnote{\textbf{2:9} Isaías 64:4} ''.

\hypertarget{la-exploraciuxf3n-y-absorciuxf3n-de-esta-sabiduruxeda-solo-es-posible-para-personas-espirituales}{%
\subsection{La exploración y absorción de esta sabiduría solo es posible
para personas
espirituales}\label{la-exploraciuxf3n-y-absorciuxf3n-de-esta-sabiduruxeda-solo-es-posible-para-personas-espirituales}}

\bibleverse{10} Pero a nosotros, Dios nos las reveló por medio del
Espíritu. Porque el Espíritu escudriña todas las cosas, sí, las cosas
profundas de Dios. \footnote{\textbf{2:10} Mat 13,11; Col 1,26}
\bibleverse{11} Porque ¿quién de los hombres conoce las cosas del hombre
sino el espíritu del hombre que está en él? Así, nadie conoce las cosas
de Dios sino el Espíritu de Dios. \bibleverse{12} Pero nosotros no hemos
recibido el espíritu del mundo, sino el Espíritu que viene de Dios, para
conocer las cosas que nos han sido dadas gratuitamente por Dios.
\footnote{\textbf{2:12} Juan 14,16-17} \bibleverse{13} También hablamos
estas cosas, no con las palabras que enseña la sabiduría de los hombres,
sino con las que enseña el Espíritu Santo, comparando las cosas
espirituales con las espirituales. \bibleverse{14} Ahora bien, el hombre
natural no recibe las cosas del Espíritu de Dios, porque para él son
locura, y no puede conocerlas, porque se disciernen espiritualmente.
\footnote{\textbf{2:14} 1Cor 1,23; Juan 8,47} \bibleverse{15} Pero el
que es espiritual discierne todas las cosas, y no debe ser juzgado por
nadie. \bibleverse{16} ``Porque ¿quién ha conocido la mente del Señor
para instruirlo?'' \footnote{\textbf{2:16} Isaías 40:13} Pero nosotros
tenemos la mente de Cristo. \footnote{\textbf{2:16} Rom 11,34}

\hypertarget{hasta-ahora-pablo-no-ha-podido-proclamar-plena-sabiduruxeda-a-los-corintios-debido-a-su-inmadurez-que-ha-sido-demostrada-por-la-picarduxeda-del-partido}{%
\subsection{Hasta ahora Pablo no ha podido proclamar plena sabiduría a
los corintios debido a su inmadurez, que ha sido demostrada por la
picardía del
partido}\label{hasta-ahora-pablo-no-ha-podido-proclamar-plena-sabiduruxeda-a-los-corintios-debido-a-su-inmadurez-que-ha-sido-demostrada-por-la-picarduxeda-del-partido}}

\hypertarget{section-2}{%
\section{3}\label{section-2}}

\bibleverse{1} Hermanos, no podía hablaros como a espirituales, sino
como a carnales, como a bebés en Cristo. \footnote{\textbf{3:1} Juan
  16,12} \bibleverse{2} Os he alimentado con leche, no con alimentos
sólidos, porque aún no estáis preparados. De hecho, no estáis preparados
ni siquiera ahora, \footnote{\textbf{3:2} 1Pe 2,2} \bibleverse{3} porque
todavía sois carnales. Porque en cuanto a los celos, las disputas y las
facciones entre vosotros, ¿no sois carnales y no andáis por los caminos
de los hombres? \footnote{\textbf{3:3} 1Cor 1,10-11; 1Cor 11,18}
\bibleverse{4} Porque cuando uno dice: ``Yo sigo a Pablo'', y otro: ``Yo
sigo a Apolos'', ¿no sois carnales? \footnote{\textbf{3:4} 1Cor 1,12}

\hypertarget{son-siervos-y-colaboradores-de-dios}{%
\subsection{Son siervos y colaboradores de
Dios}\label{son-siervos-y-colaboradores-de-dios}}

\bibleverse{5} ¿Quién es, pues, Apolos y quién Pablo, sino servidores
por medio de los cuales creísteis, y cada uno según le dio el Señor?
\bibleverse{6} Yo planté. Apolos regó. Pero el crecimiento lo dio Dios.
\footnote{\textbf{3:6} Hech 18,24-28} \bibleverse{7} Así que ni el que
planta es algo, ni el que riega, sino Dios que da el crecimiento.
\bibleverse{8} Ahora bien, el que planta y el que riega son lo mismo,
pero cada uno recibirá su propia recompensa según su trabajo.
\footnote{\textbf{3:8} 1Cor 4,5} \bibleverse{9} Porque nosotros somos
colaboradores de Dios. Vosotros sois labradores de Dios, constructores
de Dios. \footnote{\textbf{3:9} Mat 13,3-9; Efes 2,20}

\hypertarget{cada-maestro-procura-que-su-obra-consista-en-el-fuego-del-juicio-divino-de-un-duxeda}{%
\subsection{¡Cada maestro procura que su obra consista en el fuego del
juicio divino de un
día!}\label{cada-maestro-procura-que-su-obra-consista-en-el-fuego-del-juicio-divino-de-un-duxeda}}

\bibleverse{10} Según la gracia de Dios que me fue concedida, como sabio
maestro de obras puse un fundamento, y otro construye sobre él. Pero que
cada uno tenga cuidado de cómo construye sobre él. \bibleverse{11}
Porque nadie puede poner otro fundamento que el que está puesto, que es
Jesucristo. \footnote{\textbf{3:11} 1Pe 2,4-6} \bibleverse{12} Pero si
alguien construye sobre el fundamento con oro, plata, piedras preciosas,
madera, heno o paja, \bibleverse{13} la obra de cada uno será revelada.
Porque el Día lo declarará, porque se revela en el fuego; y el fuego
mismo probará qué clase de obra es la de cada uno. \footnote{\textbf{3:13}
  1Cor 4,5} \bibleverse{14} Si la obra de algún hombre permanece lo que
construyó, recibirá una recompensa. \bibleverse{15} Si la obra de alguno
se quema, sufrirá pérdida, pero él mismo se salvará, pero como a través
del fuego.

\bibleverse{16} ¿No saben que ustedes son el templo de Dios y que el
Espíritu de Dios vive en ustedes? \footnote{\textbf{3:16} 1Cor 6,19;
  2Cor 6,16} \bibleverse{17} Si alguien destruye el templo de Dios, Dios
lo destruirá a él, porque el templo de Dios, que ustedes son, es santo.

\bibleverse{18} Que nadie se engañe a sí mismo. Si alguno se cree sabio
entre vosotros en este mundo, que se haga tonto para llegar a ser sabio.
\footnote{\textbf{3:18} Apoc 3,17-18} \bibleverse{19} Porque la
sabiduría de este mundo es una tontería para Dios. Porque está escrito:
``Él ha tomado a los sabios en su astucia''. \footnote{\textbf{3:19} Job
  5:13} \bibleverse{20} Y también: ``El Señor conoce el razonamiento de
los sabios, que es inútil''. \footnote{\textbf{3:20} Salmo 94:11}
\bibleverse{21} Por tanto, que nadie se jacte en los hombres. Porque
todas las cosas son vuestras, \bibleverse{22} ya sea Pablo, o Apolos, o
Cefas, o el mundo, o la vida, o la muerte, o las cosas presentes, o las
cosas por venir. Todo es vuestro, \bibleverse{23} y vosotros sois de
Cristo, y Cristo es de Dios. \footnote{\textbf{3:23} 1Cor 11,3}

\hypertarget{pablo-sabe-que-es-responsable-solo-ante-el-seuxf1or}{%
\subsection{Pablo sabe que es responsable solo ante el
Señor}\label{pablo-sabe-que-es-responsable-solo-ante-el-seuxf1or}}

\hypertarget{section-3}{%
\section{4}\label{section-3}}

\bibleverse{1} Así pues, que el hombre piense en nosotros como
servidores de Cristo y administradores de los misterios de Dios.
\footnote{\textbf{4:1} Tit 1,7} \bibleverse{2} Aquí, además, se exige a
los administradores que sean hallados fieles. \footnote{\textbf{4:2} Luc
  12,42} \bibleverse{3} Pero para mí es una cosa muy pequeña que me
juzguen ustedes o un tribunal humano. Sí, ni siquiera me juzgo a mí
mismo. \bibleverse{4} Porque nada sé contra mí mismo. Pero no me
justifico por esto, sino que el que me juzga es el Señor. \bibleverse{5}
Por tanto, no juzgues nada antes de tiempo, hasta que venga el Señor,
que sacará a la luz lo oculto de las tinieblas y revelará los designios
de los corazones. Entonces cada uno recibirá su alabanza de Dios.
\footnote{\textbf{4:5} 1Cor 3,8}

\hypertarget{pablo-reprocha-a-los-corintios-su-arrogancia-hacia-el-sufrimiento-de-los-apuxf3stoles}{%
\subsection{Pablo reprocha a los corintios su arrogancia hacia el
sufrimiento de los
apóstoles}\label{pablo-reprocha-a-los-corintios-su-arrogancia-hacia-el-sufrimiento-de-los-apuxf3stoles}}

\bibleverse{6} Ahora bien, estas cosas, hermanos, las he transferido en
figura a mí mismo y a Apolos por vosotros, para que en nosotros
aprendáis a no pensar más allá de lo que está escrito, para que ninguno
de vosotros se ensoberbezca contra el otro. \footnote{\textbf{4:6} Rom
  12,3} \bibleverse{7} Porque ¿quién os hace diferentes? ¿Y qué tenéis
que no hayáis recibido? Pero si lo habéis recibido, ¿por qué os jactáis
como si no lo hubierais recibido?

\bibleverse{8} Ya estás lleno. Ya te has enriquecido. Has venido a
reinar sin nosotros. Sí, ¡y yo quisiera que reinarais, para que también
nosotros reináramos con vosotros! \footnote{\textbf{4:8} Apoc 3,17; Apoc
  3,21} \bibleverse{9} Porque pienso que Dios nos ha exhibido a
nosotros, los apóstoles, los últimos, como hombres condenados a muerte.
Porque somos un espectáculo para el mundo, tanto para los ángeles como
para los hombres. \footnote{\textbf{4:9} Rom 8,36; Heb 10,33}
\bibleverse{10} Nosotros somos tontos por causa de Cristo, pero vosotros
sois sabios en Cristo. Nosotros somos débiles, pero vosotros sois
fuertes. Vosotros tenéis honor, pero nosotros tenemos deshonra.
\footnote{\textbf{4:10} 1Cor 3,18} \bibleverse{11} Hasta esta hora
tenemos hambre, sed, estamos desnudos, somos golpeados y no tenemos una
morada segura. \footnote{\textbf{4:11} 2Cor 11,23-27} \bibleverse{12}
Nos esforzamos, trabajando con nuestras propias manos. Cuando la gente
nos maldice, nosotros bendecimos. Si nos persiguen, aguantamos.
\footnote{\textbf{4:12} 1Cor 9,15; Hech 18,3; Mat 5,44; Rom 12,14}
\bibleverse{13} Cuando nos difaman, suplicamos. Estamos hechos como la
inmundicia del mundo, la suciedad limpiada por todos, incluso hasta
ahora.

\hypertarget{la-referencia-de-pablo-a-su-relaciuxf3n-personal-con-la-iglesia}{%
\subsection{La referencia de Pablo a su relación personal con la
iglesia}\label{la-referencia-de-pablo-a-su-relaciuxf3n-personal-con-la-iglesia}}

\bibleverse{14} No escribo estas cosas para avergonzaros, sino para
amonestaros como a mis hijos amados. \bibleverse{15} Porque aunque
tengáis diez mil tutores en Cristo, no tenéis muchos padres. Porque en
Cristo Jesús me convertí en vuestro padre por la Buena Nueva.
\footnote{\textbf{4:15} 1Cor 9,2; Gal 4,19} \bibleverse{16} Os ruego,
pues, que seáis imitadores míos. \footnote{\textbf{4:16} 1Cor 11,1}
\bibleverse{17} Por eso os he enviado a Timoteo, que es mi hijo amado y
fiel en el Señor, el cual os recordará mis caminos que son en Cristo,
así como yo enseño en todas las asambleas. \footnote{\textbf{4:17} Hech
  16,1-3} \bibleverse{18} Ahora bien, algunos se envanecen, como si yo
no fuera a vosotros. \bibleverse{19} Pero iré pronto a vosotros, si el
Señor quiere. Y conoceré, no la palabra de los engreídos, sino el poder.
\bibleverse{20} Porque el Reino de Dios no es de palabra, sino de poder.
\footnote{\textbf{4:20} 1Cor 2,4} \bibleverse{21} ¿Qué queréis? ¿Voy a
ir a vosotros con vara, o con amor y espíritu de mansedumbre?
\footnote{\textbf{4:21} 2Cor 10,2}

\hypertarget{grave-reprimenda-por-la-tolerancia-mostrada-por-la-comunidad-a-un-fornicario}{%
\subsection{Grave reprimenda por la tolerancia mostrada por la comunidad
a un
fornicario}\label{grave-reprimenda-por-la-tolerancia-mostrada-por-la-comunidad-a-un-fornicario}}

\hypertarget{section-4}{%
\section{5}\label{section-4}}

\bibleverse{1} En realidad, se dice que hay inmoralidad sexual entre
vosotros, y una inmoralidad sexual como no se nombra entre los gentiles,
que uno tiene la mujer de su padre. \bibleverse{2} Vosotros sois
arrogantes y no os habéis lamentado, en cambio, de que el que ha hecho
este acto sea eliminado de entre vosotros. \footnote{\textbf{5:2} 1Cor
  4,6-8} \bibleverse{3} Porque ciertamente, como ausente en cuerpo pero
presente en espíritu, ya he juzgado, como si estuviera presente, al que
ha hecho esto. \footnote{\textbf{5:3} Col 2,5} \bibleverse{4} En el
nombre de nuestro Señor Jesucristo, cuando os reunáis con mi espíritu
con el poder de nuestro Señor Jesucristo, \footnote{\textbf{5:4} Mat
  16,19; Mat 18,18; 2Cor 13,10} \bibleverse{5} debéis entregar al tal a
Satanás para la destrucción de la carne, a fin de que el espíritu se
salve en el día del Señor Jesús. \footnote{\textbf{5:5} 1Tim 1,20}

\hypertarget{amonestaciuxf3n-general-a-la-pureza-moral-con-referencia-a-la-muerte-en-sacrificio-de-jesuxfas-el-cordero-pascual}{%
\subsection{Amonestación general a la pureza moral con referencia a la
muerte en sacrificio de Jesús, ``el cordero
pascual''}\label{amonestaciuxf3n-general-a-la-pureza-moral-con-referencia-a-la-muerte-en-sacrificio-de-jesuxfas-el-cordero-pascual}}

\bibleverse{6} Tu jactancia no es buena. ¿No sabéis que un poco de
levadura leuda toda la masa? \footnote{\textbf{5:6} Gal 5,9}
\bibleverse{7} Limpiad la levadura vieja, para que seáis una masa nueva,
así como sin levadura. Porque, en efecto, Cristo, nuestra Pascua, ha
sido sacrificado en nuestro lugar. \footnote{\textbf{5:7} Éxod 12,3-20;
  Éxod 13,7; Is 53,7; 1Pe 1,19} \bibleverse{8} Por tanto, celebremos la
fiesta, no con la levadura vieja, ni con la levadura de la malicia y de
la maldad, sino con el pan sin levadura de la sinceridad y de la verdad.

\hypertarget{correcciuxf3n-de-un-malentendido-corintio-sobre-la-advertencia-contra-los-fornicadores}{%
\subsection{Corrección de un malentendido corintio sobre la advertencia
contra los
fornicadores}\label{correcciuxf3n-de-un-malentendido-corintio-sobre-la-advertencia-contra-los-fornicadores}}

\bibleverse{9} Os escribí en mi carta que no os juntarais con los
pecadores sexuales; \bibleverse{10} pero no me refiero en absoluto a los
pecadores sexuales de este mundo, ni a los avaros y extorsionadores, ni
a los idólatras, porque entonces tendríais que dejar el mundo.
\bibleverse{11} Pero tal como es, os escribí que no os juntéis con
ninguno de los llamados hermanos que sean pecadores sexuales, o
codiciosos, o idólatras, o calumniadores, o borrachos, o extorsionistas.
Ni siquiera comas con una persona así. \footnote{\textbf{5:11} 2Tes 3,6}
\bibleverse{12} Porque, ¿qué tengo yo que ver con juzgar también a los
que están fuera? ¿No juzgas tú a los que están dentro? \bibleverse{13}
Pero a los que están fuera, Dios los juzga. ``Quitad al malvado de entre
vosotros''. \footnote{\textbf{5:13} Deuteronomio 17:7; 19:19; 21:21;
  22:21; 24:7} \footnote{\textbf{5:13} Deut 13,5; Mat 18,17}

\hypertarget{denuncia-de-litigio-en-tribunales-paganos-y-litigio-en-general}{%
\subsection{Denuncia de litigio en tribunales paganos y litigio en
general}\label{denuncia-de-litigio-en-tribunales-paganos-y-litigio-en-general}}

\hypertarget{section-5}{%
\section{6}\label{section-5}}

\bibleverse{1} ¿Se atreve alguno de vosotros, teniendo un asunto contra
su prójimo, a acudir a la justicia ante los injustos, y no ante los
santos? \bibleverse{2} ¿No sabéis que los santos juzgarán al mundo? Y si
el mundo es juzgado por vosotros, ¿sois indignos de juzgar los asuntos
más pequeños? \footnote{\textbf{6:2} Mat 19,28} \bibleverse{3} ¿No
sabéis que nosotros juzgaremos a los ángeles? ¿Cuánto más las cosas que
pertenecen a esta vida? \bibleverse{4} Si, pues, tenéis que juzgar las
cosas que pertenecen a esta vida, ¿los ponéis a juzgar a los que no
tienen importancia en la asamblea? \bibleverse{5} Digo esto para
avergonzaros. ¿No hay entre vosotros ni siquiera un sabio que pueda
decidir entre sus hermanos? \bibleverse{6} ¡Pero el hermano va a juicio
con el hermano, y eso ante los incrédulos! \bibleverse{7} Por lo tanto,
ya es un defecto en vosotros que tengáis pleitos unos con otros. ¿Por
qué no ser más bien agraviados? ¿Por qué no ser más bien defraudados?
\footnote{\textbf{6:7} Mat 5,38-41; 1Tes 5,15; 1Pe 3,9} \bibleverse{8}
No, sino que vosotros mismos hacéis mal y defraudáis, y eso contra
vuestros hermanos.

\bibleverse{9} ¿O es que no sabéis que los injustos no heredarán el
Reino de Dios? No os engañéis. Ni los inmorales, ni los idólatras, ni
los adúlteros, ni las prostitutas, ni los homosexuales, \footnote{\textbf{6:9}
  1Tim 1,9-11; Gal 5,19-21} \bibleverse{10} ni los ladrones, ni los
avaros, ni los borrachos, ni los calumniadores, ni los extorsionistas,
heredarán el Reino de Dios. \bibleverse{11} Algunos de ustedes eran así,
pero fueron lavados. Fuisteis santificados. Fuisteis justificados en el
nombre del Señor Jesús y en el Espíritu de nuestro Dios. \footnote{\textbf{6:11}
  1Cor 1,2; Rom 3,26; Tit 3,3-7}

\hypertarget{los-pecados-de-fornicaciuxf3n-no-tienen-nada-que-ver-con-la-libertad-cristiana-advertencia-de-fornicaciuxf3n}{%
\subsection{Los pecados de fornicación no tienen nada que ver con la
libertad cristiana; Advertencia de
fornicación}\label{los-pecados-de-fornicaciuxf3n-no-tienen-nada-que-ver-con-la-libertad-cristiana-advertencia-de-fornicaciuxf3n}}

\bibleverse{12} ``Todo me es lícito'', pero no todo es conveniente.
``Todas las cosas me son lícitas'', pero no me someteré al poder de
nada. \footnote{\textbf{6:12} 1Cor 10,23} \bibleverse{13} ``Alimentos
para el vientre, y el vientre para los alimentos'', pero Dios hará
desaparecer tanto a él como a ellos. Pero el cuerpo no es para la
inmoralidad sexual, sino para el Señor, y el Señor para el cuerpo.
\footnote{\textbf{6:13} 1Cor 1,3-5} \bibleverse{14} Ahora bien, Dios
resucitó al Señor, y también nos resucitará a nosotros con su poder.
\footnote{\textbf{6:14} 1Cor 15,20; 2Cor 4,14} \bibleverse{15} ¿No
sabéis que vuestros cuerpos son miembros de Cristo? ¿Acaso voy a tomar
los miembros de Cristo para hacerlos miembros de una prostituta? ¡Que
nunca sea así! \bibleverse{16} ¿Acaso no sabéis que el que se une a una
prostituta es un solo cuerpo? Porque, ``Los dos'', dice, ``se
convertirán en una sola carne''. \footnote{\textbf{6:16} Génesis 2:24}
\bibleverse{17} Pero el que se une al Señor es un solo espíritu.
\footnote{\textbf{6:17} Juan 17,21-22} \bibleverse{18} ¡Huye de la
inmoralidad sexual! ``Todo pecado que el hombre hace está fuera del
cuerpo'', pero el que comete inmoralidad sexual peca contra su propio
cuerpo. \bibleverse{19} ¿O no sabéis que vuestro cuerpo es templo del
Espíritu Santo que está en vosotros, el cual tenéis de Dios? No sois
vuestros, \footnote{\textbf{6:19} 1Cor 3,16} \bibleverse{20} porque
habéis sido comprados por un precio. Por tanto, glorificad a Dios en
vuestro cuerpo y en vuestro espíritu, que son de Dios. \footnote{\textbf{6:20}
  1Cor 7,23; 1Pe 1,18-19; Fil 1,20}

\hypertarget{el-valor-y-las-necesidades-del-matrimonio-y-la-vida-conyugal-en-general}{%
\subsection{El valor y las necesidades del matrimonio y la vida conyugal
en
general}\label{el-valor-y-las-necesidades-del-matrimonio-y-la-vida-conyugal-en-general}}

\hypertarget{section-6}{%
\section{7}\label{section-6}}

\bibleverse{1} En cuanto a lo que me escribisteis, es bueno que el
hombre no toque a la mujer. \bibleverse{2} Pero, a causa de las
inmoralidades sexuales, que cada hombre tenga su propia esposa, y que
cada mujer tenga su propio marido. \bibleverse{3} Que el marido dé a su
mujer el afecto que se le debe, \footnote{\textbf{7:3} NU y TR tienen
  ``lo que se le debe'' en lugar de ``el afecto que se le debe''.} y así
también la mujer a su marido. \bibleverse{4} La mujer no tiene autoridad
sobre su propio cuerpo, sino el marido. Así también el marido no tiene
autoridad sobre su propio cuerpo, sino la mujer. \bibleverse{5} No os
privéis los unos a los otros, a no ser que sea de común acuerdo por un
tiempo, para que os dediquéis al ayuno y a la oración, y estéis de nuevo
juntos, para que Satanás no os tiente por vuestra falta de dominio
propio.

\bibleverse{6} Pero esto lo digo a modo de concesión, no de mandamiento.
\bibleverse{7} Sin embargo, quisiera que todos los hombres fueran como
yo. Sin embargo, cada hombre tiene su propio don de Dios, uno de este
tipo y otro de aquel. \footnote{\textbf{7:7} Mat 19,22}

\hypertarget{sobre-el-comportamiento-de-las-personas-solteras-y-sobre-el-divorcio-en-los-matrimonios-cristianos}{%
\subsection{Sobre el comportamiento de las personas solteras y sobre el
divorcio en los matrimonios
cristianos}\label{sobre-el-comportamiento-de-las-personas-solteras-y-sobre-el-divorcio-en-los-matrimonios-cristianos}}

\bibleverse{8} Pero a los solteros y a las viudas les digo que es bueno
que se queden como yo. \bibleverse{9} Pero si no tienen dominio propio,
que se casen. Porque es mejor casarse que arder de pasión. \footnote{\textbf{7:9}
  1Tim 5,14} \bibleverse{10} Pero a los casados les ordeno --- no yo,
sino el Señor --- que la mujer no deje a su marido \footnote{\textbf{7:10}
  Mat 5,32} \bibleverse{11} (pero si se separa, que se quede soltera, o
que se reconcilie con su marido), y que el marido no deje a su mujer.

\hypertarget{comportamiento-en-el-matrimonio-mixto}{%
\subsection{Comportamiento en el matrimonio
mixto}\label{comportamiento-en-el-matrimonio-mixto}}

\bibleverse{12} Pero a los demás, yo --- no el Señor --- les digo: Si
algún hermano tiene una esposa incrédula, y ella se contenta con vivir
con él, que no la deje. \bibleverse{13} La mujer que tiene un marido
incrédulo, y éste se contenta con vivir con ella, que no deje a su
marido. \bibleverse{14} Porque el marido incrédulo se santifica en la
mujer, y la mujer incrédula se santifica en el marido. De lo contrario,
sus hijos serían impuros, pero ahora son santos. \footnote{\textbf{7:14}
  Rom 11,16} \bibleverse{15} Pero si el incrédulo se aparta, que haya
separación. El hermano o la hermana no están sometidos en tales casos,
sino que Dios nos ha llamado en paz. \footnote{\textbf{7:15} Rom 14,19}
\bibleverse{16} Pues ¿cómo sabes, esposa, si salvarás a tu marido? ¿O
cómo sabes, esposo, si salvarás a tu esposa? \footnote{\textbf{7:16} 1Pe
  3,1}

\hypertarget{regla-general-sobre-la-posiciuxf3n-del-cristiano-a-las-condiciones-externas-existentes-todo-creyente-permanece-en-la-posiciuxf3n-en-la-que-fue-llamado}{%
\subsection{Regla general sobre la posición del cristiano a las
condiciones externas existentes: ¡Todo creyente permanece en la posición
en la que fue
llamado!}\label{regla-general-sobre-la-posiciuxf3n-del-cristiano-a-las-condiciones-externas-existentes-todo-creyente-permanece-en-la-posiciuxf3n-en-la-que-fue-llamado}}

\bibleverse{17} Solamente, como el Señor ha distribuido a cada hombre,
como Dios ha llamado a cada uno, así debe caminar. Así lo ordeno en
todas las asambleas.

\bibleverse{18} ¿Se llamó a alguien habiendo sido circuncidado? Que no
se vuelva incircunciso. ¿Ha sido llamado alguien en la incircuncisión?
Que no se circuncide. \bibleverse{19} La circuncisión no es nada, y la
incircuncisión no es nada, pero lo que importa es guardar los
mandamientos de Dios. \footnote{\textbf{7:19} Gal 5,6; Gal 6,15}
\bibleverse{20} Que cada uno permanezca en la vocación a la que fue
llamado. \bibleverse{21} ¿Fuiste llamado siendo siervo? No dejes que eso
te moleste, pero si tienes la oportunidad de ser libre, aprovéchala.
\bibleverse{22} Porque el que fue llamado en el Señor siendo siervo, es
el hombre libre del Señor. Asimismo, el que fue llamado siendo libre es
siervo de Cristo. \footnote{\textbf{7:22} Efes 6,6; Filem 1,16}
\bibleverse{23} Ustedes fueron comprados por un precio. No os hagáis
siervos de los hombres. \footnote{\textbf{7:23} 1Cor 6,20}
\bibleverse{24} Hermanos, que cada uno, en la condición en que fue
llamado, permanezca en esa condición con Dios.

\hypertarget{sobre-el-celibato-de-ambos-sexos-consejos-para-casarse-con-mujeres-solteras-y-volver-a-casarse-con-viudas}{%
\subsection{Sobre el celibato de ambos sexos; Consejos para casarse con
mujeres solteras y volver a casarse con
viudas}\label{sobre-el-celibato-de-ambos-sexos-consejos-para-casarse-con-mujeres-solteras-y-volver-a-casarse-con-viudas}}

\bibleverse{25} En cuanto a las vírgenes, no tengo ningún mandamiento
del Señor, sino que doy mi juicio como alguien que ha obtenido la
misericordia del Señor para ser digno de confianza. \bibleverse{26} Por
lo tanto, creo que a causa de la angustia que nos invade, es bueno que
el hombre permanezca como está. \footnote{\textbf{7:26} 1Cor 10,11}
\bibleverse{27} ¿Estás atado a una esposa? No busques liberarte. ¿Estás
libre de una esposa? No busques esposa. \bibleverse{28} Pero si te
casas, no has pecado. Si una virgen se casa, no ha pecado. Sin embargo,
los tales tendrán opresión en la carne, y yo quiero librarlos.
\bibleverse{29} Pero os digo esto, hermanos: el tiempo es corto. A
partir de ahora, tanto los que tienen esposa como los que no la tienen;
\footnote{\textbf{7:29} Rom 13,11; Luc 14,26} \bibleverse{30} y los que
lloran, como si no lloraran; y los que se alegran, como si no se
alegraran; y los que compran, como si no poseyeran; \bibleverse{31} y
los que usan el mundo, como si no lo usaran al máximo. Porque el modo de
este mundo pasa. \footnote{\textbf{7:31} 1Jn 2,15-17}

\bibleverse{32} Pero yo quiero que estéis libres de preocupaciones. El
que no está casado se preocupa de las cosas del Señor, de cómo puede
agradar al Señor; \bibleverse{33} pero el que está casado se preocupa de
las cosas del mundo, de cómo puede agradar a su mujer. \footnote{\textbf{7:33}
  Luc 14,20} \bibleverse{34} También hay una diferencia entre una esposa
y una virgen. La mujer soltera se preocupa por las cosas del Señor, para
ser santa tanto en cuerpo como en espíritu. Pero la que está casada se
preocupa por las cosas del mundo: por complacer a su marido.
\bibleverse{35} Esto lo digo por tu propio bien, no para que te atrape,
sino por lo que conviene, y para que atiendas al Señor sin distracción.

\bibleverse{36} Pero si algún hombre piensa que se comporta de manera
inapropiada con su virgen, si ella ha pasado la flor de la edad, y si la
necesidad lo requiere, que haga lo que quiera. No peca. Que se casen.
\bibleverse{37} Pero el que se mantiene firme en su corazón, sin tener
urgencia, sino que tiene poder sobre su propia voluntad, y ha decidido
en su propio corazón conservar su propia virgen, hace bien.
\bibleverse{38} Así pues, tanto el que da su propia virgen en matrimonio
hace bien, como el que no la da en matrimonio hace mejor.

\hypertarget{sobre-el-nuevo-matrimonio-de-las-viudas}{%
\subsection{Sobre el nuevo matrimonio de las
viudas}\label{sobre-el-nuevo-matrimonio-de-las-viudas}}

\bibleverse{39} La mujer está obligada por la ley mientras viva su
marido; pero si el marido ha muerto, es libre de casarse con quien
quiera, sólo en el Señor. \footnote{\textbf{7:39} Rom 7,2}
\bibleverse{40} Pero ella es más feliz si se queda como está, a mi
juicio, y creo que también tengo el Espíritu de Dios.

\hypertarget{el-conocimiento-en-suxed-mismo-tiene-menos-valor-que-el-amor}{%
\subsection{El conocimiento en sí mismo tiene menos valor que el
amor}\label{el-conocimiento-en-suxed-mismo-tiene-menos-valor-que-el-amor}}

\hypertarget{section-7}{%
\section{8}\label{section-7}}

\bibleverse{1} En cuanto a las cosas sacrificadas a los ídolos, sabemos
que todos tenemos conocimiento. El conocimiento infla, pero el amor
edifica. \footnote{\textbf{8:1} Hech 15,29} \bibleverse{2} Pero si
alguien piensa que sabe algo, todavía no sabe como debe saber.
\footnote{\textbf{8:2} Gal 6,3} \bibleverse{3} Pero el que ama a Dios es
conocido por él. \footnote{\textbf{8:3} Gal 4,9; 1Cor 13,12}

\hypertarget{no-todo-el-mundo-tiene-un-conocimiento-perfecto}{%
\subsection{No todo el mundo tiene un conocimiento
perfecto}\label{no-todo-el-mundo-tiene-un-conocimiento-perfecto}}

\bibleverse{4} Por lo tanto, en cuanto a comer cosas sacrificadas a los
ídolos, sabemos que no hay ningún ídolo en el mundo, y que no hay más
Dios que uno. \footnote{\textbf{8:4} Deut 6,4} \bibleverse{5} Porque
aunque hay cosas que se llaman ``dioses'', ya sea en los cielos o en la
tierra --- como hay muchos ``dioses'' y muchos ``señores'' ---,
\footnote{\textbf{8:5} 1Cor 10,19-20; Sal 136,2-3; Rom 8,38-39}
\bibleverse{6} sin embargo, para nosotros hay un solo Dios, el Padre,
del cual proceden todas las cosas, y nosotros para él; y un solo Señor,
Jesucristo, por el cual son todas las cosas, y nosotros vivimos por él.
\footnote{\textbf{8:6} 1Cor 12,5-6; Col 1,16; Efes 4,5-6; Mal 2,10; Juan
  1,3}

\bibleverse{7} Sin embargo, ese conocimiento no está en todos los
hombres. Pero algunos, con la conciencia de un ídolo hasta ahora, comen
como de una cosa sacrificada a un ídolo, y su conciencia, siendo débil,
se contamina. \footnote{\textbf{8:7} 1Cor 10,28}

\hypertarget{para-el-uso-de-la-libertad-cristiana-la-consideraciuxf3n-amorosa-por-los-duxe9biles-es-decisiva}{%
\subsection{Para el uso de la libertad cristiana, la consideración
amorosa por los débiles es
decisiva}\label{para-el-uso-de-la-libertad-cristiana-la-consideraciuxf3n-amorosa-por-los-duxe9biles-es-decisiva}}

\bibleverse{8} Pero la comida no nos recomendará a Dios. Pues ni si no
comemos somos peores, ni si comemos somos mejores. \footnote{\textbf{8:8}
  Rom 14,17} \bibleverse{9} Pero tened cuidado de que esta libertad
vuestra no se convierta en un tropiezo para los débiles. \footnote{\textbf{8:9}
  Gal 5,13} \bibleverse{10} Porque si un hombre os ve a vosotros, que
tenéis conocimiento, sentados en el templo de un ídolo, ¿no se
envalentonará su conciencia, si es débil, para comer cosas sacrificadas
a los ídolos? \bibleverse{11} Y por vuestro conocimiento perece el que
es débil, el hermano por el que murió Cristo. \footnote{\textbf{8:11}
  Rom 14,15} \bibleverse{12} Así, pecando contra los hermanos e hiriendo
su conciencia cuando es débil, pecas contra Cristo. \bibleverse{13} Por
tanto, si la comida hace tropezar a mi hermano, no comeré más carne
jamas, para no hacer tropezar a mi hermano. \footnote{\textbf{8:13} Rom
  14,21}

\hypertarget{explicaciuxf3n-y-justificaciuxf3n-de-los-derechos-debidos-a-pablo-como-apuxf3stol}{%
\subsection{Explicación y justificación de los derechos debidos a Pablo
como
apóstol}\label{explicaciuxf3n-y-justificaciuxf3n-de-los-derechos-debidos-a-pablo-como-apuxf3stol}}

\hypertarget{section-8}{%
\section{9}\label{section-8}}

\bibleverse{1} ¿No soy libre? ¿No soy un apóstol? ¿No he visto a
Jesucristo, nuestro Señor? ¿No sois vosotros mi obra en el Señor?
\footnote{\textbf{9:1} 1Cor 15,8; Hech 9,3-5; Hech 9,15} \bibleverse{2}
Si para los demás no soy apóstol, al menos lo soy para vosotros, pues
vosotros sois el sello de mi apostolado en el Señor. \footnote{\textbf{9:2}
  1Cor 4,15; 2Cor 3,2-3}

\bibleverse{3} Mi defensa ante los que me examinan es ésta:
\bibleverse{4} ¿No tenemos derecho a comer y beber? \footnote{\textbf{9:4}
  Luc 10,8} \bibleverse{5} ¿No tenemos derecho a llevar una esposa
creyente, como los demás apóstoles, los hermanos del Señor y Cefas?
\footnote{\textbf{9:5} Juan 1,42; Mat 8,14} \bibleverse{6} ¿O es que
Bernabé y yo no tenemos derecho a no trabajar? \footnote{\textbf{9:6}
  Hech 4,36; 2Tes 3,7-9} \bibleverse{7} ¿Qué soldado sirve a sus
expensas? ¿Quién planta una viña, y no come de su fruto? ¿O quién
apacienta un rebaño, y no bebe de la leche del rebaño?

\bibleverse{8} ¿Digo estas cosas según las costumbres de los hombres? ¿O
no dice también la ley lo mismo? \bibleverse{9} Porque está escrito en
la ley de Moisés: ``No pondrás bozal al buey mientras pisa el
grano''.\footnote{\textbf{9:9} Deuteronomio 25:4} ¿Es por los bueyes que
Dios se preocupa, \footnote{\textbf{9:9} 1Tim 5,18} \bibleverse{10} o lo
dice seguramente por nosotros? Sí, fue escrito por nuestro bien, porque
el que ara debe arar con esperanza, y el que trilla con esperanza debe
participar de su esperanza. \bibleverse{11} Si hemos sembrado para
vosotros cosas espirituales, ¿es gran cosa si cosechamos vuestras cosas
carnales? \footnote{\textbf{9:11} Rom 15,27} \bibleverse{12} Si otros
participan de este derecho sobre vosotros, ¿no lo hacemos nosotros aún
más? Sin embargo, no usamos este derecho, sino que lo soportamos todo,
para no causar ningún obstáculo a la Buena Nueva de Cristo. \footnote{\textbf{9:12}
  Hech 20,33-35; 2Cor 11,9}

\hypertarget{explique-las-razones-por-las-que-pablo-renuncia-a-sus-derechos}{%
\subsection{Explique las razones por las que Pablo renuncia a sus
derechos}\label{explique-las-razones-por-las-que-pablo-renuncia-a-sus-derechos}}

\bibleverse{13} ¿No sabéis que los que sirven en torno a las cosas
sagradas comen de las cosas del templo, y los que sirven en el altar
tienen su parte con el altar? \footnote{\textbf{9:13} Núm 18,18-19; Núm
  18,31; Deut 18,1-3} \bibleverse{14} Así ordenó el Señor que los que
anuncian la Buena Nueva vivan de ella. \footnote{\textbf{9:14} Gal 6,6;
  Luc 10,7}

\bibleverse{15} Pero yo no me he servido de nada de esto, ni escribo
estas cosas para que se haga así en mi caso; porque prefiero morir,
antes de que alguien haga nula mi jactancia. \footnote{\textbf{9:15}
  Hech 18,3} \bibleverse{16} Porque si predico la Buena Nueva, no tengo
nada de qué jactarme, pues la necesidad me obliga a ello; pero ¡ay de mí
si no predico la Buena Nueva! \footnote{\textbf{9:16} Jer 20,9}
\bibleverse{17} Porque si lo hago por mi propia voluntad, tengo una
recompensa. Pero si no lo hago por mi propia voluntad, tengo una
administración que se me ha confiado. \footnote{\textbf{9:17} 1Cor 4,1}
\bibleverse{18} ¿Cuál es, pues, mi recompensa? Que cuando predique la
Buena Nueva, pueda presentar la Buena Nueva de Cristo gratuitamente,
para no abusar de mi autoridad en la Buena Nueva.

\hypertarget{pablo-aunque-exteriormente-es-completamente-libre-es-sin-embargo-un-servidor-de-todos-los-hombres}{%
\subsection{Pablo, aunque exteriormente es completamente libre, es sin
embargo un servidor de todos los
hombres}\label{pablo-aunque-exteriormente-es-completamente-libre-es-sin-embargo-un-servidor-de-todos-los-hombres}}

\bibleverse{19} Porque siendo libre de todo, me sometí a todos para
ganar más. \footnote{\textbf{9:19} Mat 20,27; Rom 15,2} \bibleverse{20}
Para los judíos me hice como judío, para ganar a los judíos; para los
que están bajo la ley, como bajo la ley,\footnote{\textbf{9:20} NU
  añade: aunque yo mismo no estoy bajo la ley} para ganar a los que
están bajo la ley; \footnote{\textbf{9:20} 1Cor 10,33; Hech 16,3; Hech
  21,20-26} \bibleverse{21} para los que están sin ley, como sin ley (no
estando sin ley para con Dios, sino bajo la ley para con Cristo), para
ganar a los que están sin ley. \footnote{\textbf{9:21} Gal 2,3}
\bibleverse{22} A los débiles me hice como débil, para ganar a los
débiles. Me he hecho todo para todos, a fin de salvar a algunos por
todos los medios. \footnote{\textbf{9:22} Rom 11,14} \bibleverse{23}
Esto lo hago por la Buena Nueva, para ser partícipe de ella.

\hypertarget{el-apuxf3stol-como-competidor-por-el-premio-celestial}{%
\subsection{El apóstol como competidor por el premio
celestial}\label{el-apuxf3stol-como-competidor-por-el-premio-celestial}}

\bibleverse{24} ¿No sabéis que los que corren en una carrera corren
todos, pero uno recibe el premio? Corred así, para que podáis ganar.
\footnote{\textbf{9:24} Fil 3,14; 2Tim 4,7} \bibleverse{25} Todo hombre
que se esfuerza en los juegos ejerce el autocontrol en todas las cosas.
Ellos lo hacen para recibir una corona corruptible, pero nosotros una
incorruptible. \footnote{\textbf{9:25} 2Tim 2,4-5; 1Pe 5,4}
\bibleverse{26} Yo, pues, corro así, no sin rumbo. Lucho así, no
golpeando el aire, \bibleverse{27} sino que golpeo mi cuerpo y lo
someto, no sea que, después de haber predicado a otros, yo mismo quede
descalificado. \footnote{\textbf{9:27} Rom 13,14}

\hypertarget{das-durch-guxf6ttliche-gnadenerweise-in-der-wuxfcste-gesegnete-und-zur-rettung-ins-heilige-land-berufene-israel}{%
\subsection{Das durch göttliche Gnadenerweise in der Wüste gesegnete und
zur Rettung ins heilige Land berufene
Israel}\label{das-durch-guxf6ttliche-gnadenerweise-in-der-wuxfcste-gesegnete-und-zur-rettung-ins-heilige-land-berufene-israel}}

\hypertarget{section-9}{%
\section{10}\label{section-9}}

\bibleverse{1} Ahora bien, no quiero que ignoréis, hermanos, que
nuestros padres estuvieron todos bajo la nube, y todos pasaron por el
mar; \footnote{\textbf{10:1} Éxod 13,21; Éxod 14,22} \bibleverse{2} y
todos fueron bautizados en Moisés en la nube y en el mar; \footnote{\textbf{10:2}
  Éxod 16,4; Éxod 16,35; Deut 8,3} \bibleverse{3} y todos comieron el
mismo alimento espiritual; \bibleverse{4} y todos bebieron la misma
bebida espiritual. Porque bebieron de una roca espiritual que los
seguía, y la roca era Cristo. \footnote{\textbf{10:4} Éxod 17,6}

\hypertarget{a-pesar-de-esto-debido-a-que-voluntariamente-sirvieron-su-lujuria-por-la-carne-fueron-rechazados-como-un-ejemplo-de-advertencia-para-nosotros}{%
\subsection{A pesar de esto, debido a que voluntariamente sirvieron su
lujuria por la carne, fueron rechazados como un ejemplo de advertencia
para
nosotros}\label{a-pesar-de-esto-debido-a-que-voluntariamente-sirvieron-su-lujuria-por-la-carne-fueron-rechazados-como-un-ejemplo-de-advertencia-para-nosotros}}

\bibleverse{5} Sin embargo, con la mayoría de ellos, Dios no se
complació, pues fueron derribados en el desierto. \footnote{\textbf{10:5}
  Núm 14,22-32}

\bibleverse{6} Estos fueron nuestros ejemplos, para que no codiciemos
cosas malas como ellos también codiciaron. \footnote{\textbf{10:6} Núm
  11,4} \bibleverse{7} No seáis idólatras, como lo fueron algunos de
ellos. Como está escrito: ``El pueblo se sentaba a comer y beber, y se
levantaba a jugar''. \footnote{\textbf{10:7} Éxodo 32:6} \bibleverse{8}
No cometamos inmoralidad sexual, como algunos de ellos, y en un día
cayeron veintitrés mil. \footnote{\textbf{10:8} Núm 25,1; Núm 25,9}
\bibleverse{9} No pongamos a prueba a Cristo, \footnote{\textbf{10:9} NU
  lee ``el Señor'' en lugar de ``Cristo''.} como algunos de ellos lo
hicieron, y perecieron a causa de las serpientes. \footnote{\textbf{10:9}
  Núm 21,4-6} \bibleverse{10} No refunfuñen, como también refunfuñaron
algunos de ellos, y perecieron a manos del destructor. \footnote{\textbf{10:10}
  Núm 14,2; Núm 14,35-36; Heb 3,11; Heb 3,17} \bibleverse{11} Ahora
bien, todas estas cosas les sucedieron a modo de ejemplo, y fueron
escritas para nuestra amonestación, sobre la cual ha llegado el fin de
los tiempos. \footnote{\textbf{10:11} 1Pe 4,7} \bibleverse{12} Por lo
tanto, el que piensa que está en pie, tenga cuidado de no caer.

\bibleverse{13} Ninguna tentación os ha cogido sino la que es común al
hombre. Fiel es Dios, que no permitirá que seáis tentados por encima de
vuestras posibilidades, sino que junto con la tentación os dará la vía
de escape, para que podáis soportarla. \footnote{\textbf{10:13} 2Pe 2,9}

\hypertarget{la-participaciuxf3n-en-idolatruxeda-y-comidas-de-sacrificio-es-incompatible-con-la-celebraciuxf3n-de-la-cena-del-seuxf1or-cristiano-y-por-lo-tanto-debe-evitarse}{%
\subsection{La participación en idolatría y comidas de sacrificio es
incompatible con la celebración de la Cena del Señor cristiano y, por lo
tanto, debe
evitarse}\label{la-participaciuxf3n-en-idolatruxeda-y-comidas-de-sacrificio-es-incompatible-con-la-celebraciuxf3n-de-la-cena-del-seuxf1or-cristiano-y-por-lo-tanto-debe-evitarse}}

\bibleverse{14} Por tanto, amado mío, huye de la idolatría. \footnote{\textbf{10:14}
  1Jn 5,21} \bibleverse{15} Hablo como a los sabios. Juzgad lo que digo.
\bibleverse{16} La copa de bendición que bendecimos, ¿no es una
participación de la sangre de Cristo? El pan que partimos, ¿no es una
participación del cuerpo de Cristo? \footnote{\textbf{10:16} 1Cor
  11,23-26; Mat 26,27; Hech 2,42} \bibleverse{17} Porque hay un solo
pan, nosotros, que somos muchos, somos un solo cuerpo, pues todos
participamos de un solo pan. \footnote{\textbf{10:17} 1Cor 12,27; Rom
  12,5} \bibleverse{18} Considerad a Israel según la carne. ¿Acaso los
que comen los sacrificios no participan en el altar? \footnote{\textbf{10:18}
  Lev 7,6}

\bibleverse{19} ¿Qué estoy diciendo entonces? ¿Que una cosa sacrificada
a los ídolos es algo, o que un ídolo es algo? \footnote{\textbf{10:19}
  1Cor 8,4} \bibleverse{20} Pero yo digo que lo que los gentiles
sacrifican, lo sacrifican a los demonios y no a Dios, y no deseo que
tengáis comunión con los demonios. \bibleverse{21} No podéis beber a la
vez la copa del Señor y la copa de los demonios. No podéis participar a
la vez en la mesa del Señor y en la de los demonios. \footnote{\textbf{10:21}
  Mat 6,24; 2Cor 6,15-16} \bibleverse{22} ¿O acaso provocamos los celos
del Señor? ¿Somos más fuertes que él?

\hypertarget{cuuxe1ndo-es-seguro-el-consumo-de-carne-sacrificada-a-los-uxeddolos-restricciuxf3n-de-la-libertad-cristiana-por-consideraciuxf3n-al-amor-fraternal}{%
\subsection{¿Cuándo es seguro el consumo de carne sacrificada a los
ídolos? Restricción de la libertad cristiana por consideración al amor
fraternal}\label{cuuxe1ndo-es-seguro-el-consumo-de-carne-sacrificada-a-los-uxeddolos-restricciuxf3n-de-la-libertad-cristiana-por-consideraciuxf3n-al-amor-fraternal}}

\bibleverse{23} ``Todo me es lícito,'' pero no todo es provechoso.
``Todo me es lícito,'' pero no todo edifica. \footnote{\textbf{10:23}
  1Cor 6,12} \bibleverse{24} Que nadie busque lo suyo, sino que cada uno
busque el bien de su prójimo. \footnote{\textbf{10:24} Rom 15,2; Fil 2,4}
\bibleverse{25} Todo lo que se vende en la carnicería, cómelo, sin
preguntar por la conciencia, \footnote{\textbf{10:25} Rom 14,2-10; Rom
  14,22} \bibleverse{26} porque ``del Señor es la tierra y su
plenitud''. \footnote{\textbf{10:26} Salmo 24:1} \bibleverse{27} Pero si
alguno de los que no creen os invita a comer y os apetece ir, comed lo
que os pongan delante, sin preguntar nada por motivos de conciencia.
\bibleverse{28} Pero si alguien te dice: ``Esto ha sido ofrecido a los
ídolos'', no lo comas por el bien de quien te lo dijo y por el bien de
la conciencia. Porque ``la tierra es del Señor, con toda su plenitud''.
\footnote{\textbf{10:28} 1Cor 8,7} \bibleverse{29} Conciencia, digo, no
la tuya, sino la de los demás. Pues, ¿por qué mi libertad es juzgada por
otra conciencia? \bibleverse{30} Si participo con agradecimiento, ¿por
qué se me denuncia por algo por lo que doy gracias? \footnote{\textbf{10:30}
  1Tim 4,4}

\hypertarget{amonestaciuxf3n-final-para-el-correcto-caminar-cristiano-en-todo-momento}{%
\subsection{Amonestación final para el correcto caminar cristiano en
todo
momento}\label{amonestaciuxf3n-final-para-el-correcto-caminar-cristiano-en-todo-momento}}

\bibleverse{31} Así que, ya sea que comas o bebas, o hagas lo que hagas,
hazlo todo para la gloria de Dios. \footnote{\textbf{10:31} Col 3,17}
\bibleverse{32} No deis ocasión de tropiezo, ni a los judíos, ni a los
griegos, ni a la asamblea de Dios; \footnote{\textbf{10:32} Rom 14,13}
\bibleverse{33} así como yo también complazco a todos en todo, no
buscando mi propio provecho, sino el de muchos, para que se salven.
\footnote{\textbf{10:33} 1Cor 9,20-22}

\hypertarget{section-10}{%
\section{11}\label{section-10}}

\bibleverse{1} Sed imitadores de mí, como yo también lo soy de Cristo.

\hypertarget{sobre-el-comportamiento-decente-de-los-hombres-y-el-velo-de-las-mujeres-durante-la-oraciuxf3n-y-el-culto}{%
\subsection{Sobre el comportamiento decente de los hombres y el velo de
las mujeres durante la oración y el
culto}\label{sobre-el-comportamiento-decente-de-los-hombres-y-el-velo-de-las-mujeres-durante-la-oraciuxf3n-y-el-culto}}

\bibleverse{2} Ahora bien, os alabo, hermanos, porque os acordáis de mí
en todo y mantenéis firmes las tradiciones, tal como os las entregué.
\bibleverse{3} Pero quiero que sepáis que la cabeza \footnote{\textbf{11:3}
  o, origen} de todo hombre es Cristo, y la cabeza\footnote{\textbf{11:3}
  o, origen} de la mujer es el hombre, y la cabeza\footnote{\textbf{11:3}
  o, origen} de Cristo es Dios. \footnote{\textbf{11:3} Gén 3,16; Efes
  5,23; 1Cor 3,23} \bibleverse{4} Todo hombre que ora o profetiza con la
cabeza cubierta, deshonra su cabeza. \bibleverse{5} Pero toda mujer que
ora o profetiza con la cabeza descubierta, deshonra su cabeza. Porque es
lo mismo que si se afeitara. \bibleverse{6} Porque si la mujer no se
cubre, que se le corte también el cabello. Pero si es vergonzoso que la
mujer se corte el pelo o se afeite, que se cubra. \bibleverse{7} Porque
el hombre no debe cubrirse la cabeza, porque es imagen y gloria de Dios,
pero la mujer es la gloria del hombre. \bibleverse{8} Porque el hombre
no procede de la mujer, sino la mujer del hombre; \footnote{\textbf{11:8}
  Gén 2,21-23} \bibleverse{9} pues el hombre no fue creado para la
mujer, sino la mujer para el hombre. \footnote{\textbf{11:9} Gén 2,18}
\bibleverse{10} Por eso la mujer debe tener autoridad sobre su propia
cabeza, a causa de los ángeles.

\hypertarget{rechazo-del-desduxe9n-por-la-mujer-y-todas-las-discusiones-sobre-el-tema}{%
\subsection{Rechazo del desdén por la mujer y todas las discusiones
sobre el
tema}\label{rechazo-del-desduxe9n-por-la-mujer-y-todas-las-discusiones-sobre-el-tema}}

\bibleverse{11} Sin embargo, ni la mujer es independiente del hombre, ni
el hombre es independiente de la mujer, en el Señor. \bibleverse{12}
Porque así como la mujer procede del hombre, también el hombre procede
de la mujer; pero todo procede de Dios. \bibleverse{13} Juzguen ustedes
mismos. ¿Es apropiado que una mujer ore a Dios sin velo? \bibleverse{14}
¿Acaso no os enseña la misma naturaleza que si un hombre tiene el pelo
largo, es una deshonra para él? \bibleverse{15} Pero si una mujer tiene
el cabello largo, es una gloria para ella, pues su cabello le es dado
para cubrirse. \bibleverse{16} Pero si alguno parece ser pendenciero, no
tenemos esa costumbre, ni tampoco las asambleas de Dios.

\hypertarget{seria-reprimenda-por-los-agravios-en-las-comidas-comunes-e-instrucciones-para-la-celebraciuxf3n-digna-de-la-cena-del-seuxf1or}{%
\subsection{Seria reprimenda por los agravios en las comidas comunes e
instrucciones para la celebración digna de la Cena del
Señor}\label{seria-reprimenda-por-los-agravios-en-las-comidas-comunes-e-instrucciones-para-la-celebraciuxf3n-digna-de-la-cena-del-seuxf1or}}

\bibleverse{17} Pero al daros esta orden no os alabo, porque os reunís
no para lo mejor, sino para lo peor. \bibleverse{18} Porque, en primer
lugar, cuando os reunís en la asamblea, oigo que existen divisiones
entre vosotros, y en parte lo creo. \footnote{\textbf{11:18} 1Cor 1,12;
  1Cor 3,3} \bibleverse{19} Porque también es necesario que haya
divisiones entre vosotros, para que se manifiesten entre vosotros los
que son aprobados. \footnote{\textbf{11:19} Mat 18,7; 1Jn 2,19}
\bibleverse{20} Por tanto, cuando os reunís, no es la cena del Señor lo
que coméis. \bibleverse{21} Porque en vuestra comida cada uno toma
primero su propia cena. Uno tiene hambre, y otro está borracho.
\footnote{\textbf{11:21} Jds 1,12} \bibleverse{22} ¿Acaso no tenéis
casas donde comer y beber? ¿O acaso despreciáis la asamblea de Dios y
avergonzáis a los que no tienen suficiente? ¿Qué debo decirles? ¿Debo
alabarte? En esto no te alabo. \footnote{\textbf{11:22} Sant 2,5; Sant
  1,2-6}

\hypertarget{la-celebraciuxf3n-correcta-de-la-cena-del-seuxf1or-y-las-malas-consecuencias-de-un-disfrute-indigno-recordatorio-final}{%
\subsection{La celebración correcta de la Cena del Señor y las malas
consecuencias de un disfrute indigno; recordatorio
final}\label{la-celebraciuxf3n-correcta-de-la-cena-del-seuxf1or-y-las-malas-consecuencias-de-un-disfrute-indigno-recordatorio-final}}

\bibleverse{23} Porque he recibido del Señor lo que también os he
transmitido: que el Señor Jesús, la noche en que fue entregado, tomó
pan. \footnote{\textbf{11:23} Mat 26,26-28; Mar 14,22-24; Luc 22,19-20}
\bibleverse{24} Después de dar gracias, lo partió y dijo: ``Tomad,
comed. Esto es mi cuerpo, que es partido por vosotros. Haced esto en
memoria mía''. \bibleverse{25} De la misma manera tomó también la copa
después de la cena, diciendo: ``Esta copa es la nueva alianza en mi
sangre. Haced esto, cuantas veces que bebáis, en memoria mía''.
\bibleverse{26} Porque todas las veces que comáis este pan y bebáis esta
copa, proclamaréis la muerte del Señor hasta que venga. \footnote{\textbf{11:26}
  Mat 26,29}

\bibleverse{27} Por tanto, quien coma este pan o beba la copa del Señor
de manera indigna, será culpable del cuerpo y de la sangre del Señor.
\footnote{\textbf{11:27} 1Cor 11,21-22} \bibleverse{28} Pero que el
hombre se examine a sí mismo, y así coma del pan y beba de la copa.
\footnote{\textbf{11:28} Mat 26,22} \bibleverse{29} Porque el que come y
bebe de manera indigna, come y bebe juicio para sí mismo, si no
discierne el cuerpo del Señor. \footnote{\textbf{11:29} 1Cor 10,16-17}
\bibleverse{30} Por eso muchos de vosotros están débiles y enfermos, y
no pocos duermen. \bibleverse{31} Porque si nos discernimos a nosotros
mismos, no seríamos juzgados. \bibleverse{32} Pero cuando somos
juzgados, somos disciplinados por el Señor, para que no seamos
condenados con el mundo. \footnote{\textbf{11:32} Prov 3,11-12}
\bibleverse{33} Por tanto, hermanos míos, cuando os reunáis para comer,
esperaos unos a otros. \bibleverse{34} Pero si alguno tiene hambre, que
coma en su casa, para que vuestra reunión no sea para ser juzgada. Lo
demás lo pondré en orden cuando venga.

\hypertarget{la-marca-de-los-dones-espirituales-divinamente-forjados}{%
\subsection{La marca de los dones espirituales divinamente
forjados}\label{la-marca-de-los-dones-espirituales-divinamente-forjados}}

\hypertarget{section-11}{%
\section{12}\label{section-11}}

\bibleverse{1} Ahora bien, respecto a las cosas espirituales, hermanos,
no quiero que seáis ignorantes. \bibleverse{2} Sabéis que cuando erais
paganos,\footnote{\textbf{12:2} o gentiles} os dejasteis llevar por
aquellos ídolos mudos, como quiera que fueseis. \footnote{\textbf{12:2}
  Hab 2,18-19} \bibleverse{3} Por eso os hago saber que ningún hombre
que hable por el Espíritu de Dios dice: ``Jesús es maldito''. Nadie
puede decir: ``Jesús es el Señor'', sino por el Espíritu Santo.
\footnote{\textbf{12:3} Mar 9,39; 1Jn 4,2; 1Jn 1,4-3}

\hypertarget{diversidad-de-dones-espirituales-pero-solo-un-espuxedritu-activo-y-un-propuxf3sito}{%
\subsection{Diversidad de dones espirituales, pero solo un espíritu
activo y un
propósito}\label{diversidad-de-dones-espirituales-pero-solo-un-espuxedritu-activo-y-un-propuxf3sito}}

\bibleverse{4} Hay diversas clases de dones, pero el Espíritu es el
mismo. \footnote{\textbf{12:4} Rom 12,6; Efes 4,4-999} \bibleverse{5}
Hay diversas clases de servicio, pero el mismo Señor. \footnote{\textbf{12:5}
  1Cor 12,28} \bibleverse{6} Hay diversas clases de obras, pero un mismo
Dios que hace todas las cosas en todos. \bibleverse{7} Pero a cada uno
se le da la manifestación del Espíritu para beneficio de todos.
\footnote{\textbf{12:7} 1Cor 14,26} \bibleverse{8} Porque a uno se le da
por medio del Espíritu la palabra de sabiduría, y a otro la palabra de
conocimiento según el mismo Espíritu, \bibleverse{9} a otro la fe por el
mismo Espíritu, y a otro los dones de sanidad por el mismo Espíritu,
\bibleverse{10} y a otro la realización de milagros, y a otro la
profecía, y a otro el discernimiento de espíritus, a otro las diversas
clases de lenguas, y a otro la interpretación de lenguas. \footnote{\textbf{12:10}
  1Cor 14,1; Hech 2,4} \bibleverse{11} Pero el mismo Espíritu produce
todo esto, distribuyendo a cada uno por separado como quiera.
\footnote{\textbf{12:11} Rom 12,3; Efes 4,7}

\hypertarget{ilustrado-por-la-paruxe1bola-del-cuerpo-humano-y-sus-muchos-miembros}{%
\subsection{Ilustrado por la parábola del cuerpo humano y sus muchos
miembros}\label{ilustrado-por-la-paruxe1bola-del-cuerpo-humano-y-sus-muchos-miembros}}

\bibleverse{12} Porque así como el cuerpo es uno y tiene muchos
miembros, y todos los miembros del cuerpo, siendo muchos, son un solo
cuerpo, así también es Cristo. \bibleverse{13} Porque en un solo
Espíritu fuimos todos bautizados en un solo cuerpo, sean judíos o
griegos, sean siervos o libres; y a todos se nos dio a beber en un solo
Espíritu. \footnote{\textbf{12:13} Gal 3,28}

\bibleverse{14} Porque el cuerpo no es un solo miembro, sino muchos.
\bibleverse{15} Si el pie dijera: ``Como no soy la mano, no soy parte
del cuerpo'', no es por tanto parte del cuerpo. \bibleverse{16} Si la
oreja dijera: ``Porque no soy el ojo, no soy parte del cuerpo'', no es
por tanto parte del cuerpo. \bibleverse{17} Si todo el cuerpo fuera ojo,
¿dónde estaría el oído? Si todo el cuerpo fuera oído, ¿dónde estaría el
olfato? \bibleverse{18} Pero ahora Dios ha puesto los miembros, cada uno
de ellos, en el cuerpo, tal y como él quería. \bibleverse{19} Si todos
fueran un solo miembro, ¿dónde estaría el cuerpo? \bibleverse{20} Pero
ahora son muchos miembros, pero un solo cuerpo. \bibleverse{21} El ojo
no puede decir a la mano: ``No te necesito'', ni tampoco la cabeza a los
pies: ``No te necesito''. \bibleverse{22} No, mucho más bien, los
miembros del cuerpo que parecen más débiles son necesarios.
\bibleverse{23} Aquellas partes del cuerpo que nos parecen menos
honrosas, a esas les concedemos más abundante honor; y nuestras partes
impresentables tienen más abundante modestia, \bibleverse{24} mientras
que nuestras partes presentables no tienen tal necesidad. Pero Dios
compuso el cuerpo en conjunto, dando más abundante honor a la parte
inferior, \bibleverse{25} para que no haya división en el cuerpo, sino
que los miembros tengan el mismo cuidado unos de otros. \bibleverse{26}
Cuando un miembro sufre, todos los miembros sufren con él. Cuando un
miembro es honrado, todos los miembros se alegran con él.

\hypertarget{aplicaciuxf3n-de-la-imagen-a-la-estructura-divina-de-la-iglesia}{%
\subsection{Aplicación de la imagen a la estructura divina de la
iglesia}\label{aplicaciuxf3n-de-la-imagen-a-la-estructura-divina-de-la-iglesia}}

\bibleverse{27} Ahora bien, vosotros sois el cuerpo de Cristo, y los
miembros individualmente. \footnote{\textbf{12:27} Rom 12,5}
\bibleverse{28} Dios ha puesto a algunos en la asamblea: primero,
apóstoles; segundo, profetas; tercero, maestros; luego, obradores de
milagros; después, dones de sanidad, de ayuda, de gobierno y de diversas
clases de lenguas. \footnote{\textbf{12:28} Efes 4,11-12}
\bibleverse{29} ¿Son todos apóstoles? ¿Son todos profetas? ¿Son todos
maestros? ¿Son todos taumaturgos? \bibleverse{30} ¿Tienen todos dones de
curación? ¿Hablan todos varios idiomas? ¿Todos interpretan?

\hypertarget{sin-amor-incluso-los-dones-espirituales-muxe1s-elevados-no-valen-nada}{%
\subsection{Sin amor, incluso los dones espirituales más elevados no
valen
nada}\label{sin-amor-incluso-los-dones-espirituales-muxe1s-elevados-no-valen-nada}}

\bibleverse{31} Pero desead seriamente los mejores dones. Además, os
muestro un camino muy excelente. \footnote{\textbf{12:31} 1Cor 14,1;
  1Cor 14,12}

\hypertarget{section-12}{%
\section{13}\label{section-12}}

\bibleverse{1} Si hablo con las lenguas de los hombres y de los ángeles,
pero no tengo amor, me he convertido en bronce que resuena o en címbalo
que retiñe. \bibleverse{2} Si tengo el don de profecía, y conozco todos
los misterios y toda la ciencia, y si tengo toda la fe, como para
remover montañas, pero no tengo amor, no soy nada. \footnote{\textbf{13:2}
  Mat 7,22; Mat 17,20} \bibleverse{3} Si doy todos mis bienes para
alimentar a los pobres, y si entrego mi cuerpo para que lo quemen, pero
no tengo amor, de nada me sirve. \footnote{\textbf{13:3} Mat 6,2}

\hypertarget{la-esencia-del-amor}{%
\subsection{La esencia del amor}\label{la-esencia-del-amor}}

\bibleverse{4} El amor es paciente y bondadoso. El amor no tiene
envidia. El amor no se jacta, no es orgulloso, \bibleverse{5} no se
comporta de forma inadecuada, no busca su propio camino, no se provoca,
no tiene en cuenta el mal; \footnote{\textbf{13:5} Fil 2,4}
\bibleverse{6} no se alegra de la injusticia, sino que se alegra con la
verdad; \footnote{\textbf{13:6} Rom 12,9} \bibleverse{7} lo soporta
todo, lo cree todo, lo espera todo y lo soporta todo. \footnote{\textbf{13:7}
  Mat 18,21-22; Prov 10,12; Rom 15,1}

\hypertarget{la-perfecciuxf3n-del-amor-eterno-contra-el-fragmento-de-otras-gracias}{%
\subsection{La perfección del amor eterno contra el fragmento de otras
gracias}\label{la-perfecciuxf3n-del-amor-eterno-contra-el-fragmento-de-otras-gracias}}

\bibleverse{8} El amor nunca falla. Pero donde hay profecías, se
acabarán. Donde hay varias lenguas, cesarán. Donde hay conocimiento, se
acabará. \bibleverse{9} Porque sabemos en parte y profetizamos en parte;
\bibleverse{10} pero cuando llegue lo que es completo, entonces lo que
es parcial será eliminado. \bibleverse{11} Cuando era niño, hablaba como
niño, sentía como niño, pensaba como niño. Ahora que me he hecho hombre,
he dejado de lado las cosas de niño. \bibleverse{12} Porque ahora vemos
en un espejo, tenuemente, pero luego cara a cara. Ahora conozco en
parte, pero entonces conoceré plenamente, como también fui conocido
plenamente. \footnote{\textbf{13:12} 1Cor 8,3; Núm 12,8; 2Cor 5,7}
\bibleverse{13} Pero ahora quedan la fe, la esperanza y el amor, estos
tres. El mayor de ellos es el amor. \footnote{\textbf{13:13} 1Tes 1,3;
  1Jn 4,16}

\hypertarget{section-13}{%
\section{14}\label{section-13}}

\bibleverse{1} Seguid el amor y desead fervientemente los dones
espirituales, pero sobre todo que profeticéis.

\hypertarget{la-diferencia-entre-el-habla-profuxe9tica-y-el-hablar-en-lenguas}{%
\subsection{La diferencia entre el habla profética y el hablar en
lenguas}\label{la-diferencia-entre-el-habla-profuxe9tica-y-el-hablar-en-lenguas}}

\bibleverse{2} Porque el que habla en otra lengua no habla a los
hombres, sino a Dios, pues nadie entiende, pero en el Espíritu habla
misterios. \footnote{\textbf{14:2} Hech 2,4; Hech 10,46} \bibleverse{3}
Pero el que profetiza habla a los hombres para su edificación,
exhortación y consuelo. \bibleverse{4} El que habla en otra lengua se
edifica a sí mismo, pero el que profetiza edifica a la asamblea.
\bibleverse{5} Ahora bien, deseo que todos vosotros habléis con otras
lenguas, pero más aún que profeticéis. Porque es mayor el que profetiza
que el que habla con otras lenguas, si no interpreta, para que la
asamblea sea edificada. \footnote{\textbf{14:5} Núm 11,29; 1Cor 12,10}

\bibleverse{6} Pero ahora, hermanos, si voy a vosotros hablando con
otras lenguas, ¿de qué os serviría si no os hablara por medio de la
revelación, o del conocimiento, o de la profecía, o de la enseñanza?
\footnote{\textbf{14:6} 1Cor 12,8}

\hypertarget{la-inutilidad-e-inadecuaciuxf3n-de-todo-sonido-y-habla-incomprensibles}{%
\subsection{La inutilidad e inadecuación de todo sonido y habla
incomprensibles}\label{la-inutilidad-e-inadecuaciuxf3n-de-todo-sonido-y-habla-incomprensibles}}

\bibleverse{7} Incluso las cosas sin vida que hacen ruido, ya sea pipa o
arpa, si no dieran una distinción en los sonidos, ¿cómo se sabría lo que
se toca con pipa o con arpa? \bibleverse{8} Porque si la trompeta diera
un sonido incierto, ¿quién se prepararía para la guerra? \bibleverse{9}
Así también vosotros, si no pronunciarais por la lengua palabras fáciles
de entender, ¿cómo se sabría lo que se habla? Porque estarías hablando
en el aire. \bibleverse{10} Es posible que haya tantas clases de lenguas
en el mundo, y ninguna de ellas carece de significado. \bibleverse{11}
Si, pues, no conozco el significado de la lengua, sería para el que
habla un extranjero, y el que habla sería un extranjero para mí.
\bibleverse{12} Así también vosotros, ya que sois celosos de los dones
espirituales, procurad abundar para la edificación de la asamblea.

\bibleverse{13} Por tanto, el que habla en otra lengua, ore para que
pueda interpretar. \footnote{\textbf{14:13} 1Cor 12,10} \bibleverse{14}
Porque si oro en otra lengua, mi espíritu ora, pero mi entendimiento es
infructuoso.

\bibleverse{15} ¿Qué debo hacer? Oraré con el espíritu, y oraré también
con el entendimiento. Cantaré con el espíritu, y cantaré también con el
entendimiento. \footnote{\textbf{14:15} Efes 5,19} \bibleverse{16} De lo
contrario, si bendices con el espíritu, ¿cómo dirá el que ocupa el lugar
de los indoctos el ``Amén'' a tu acción de gracias, ya que no sabe lo
que dices? \bibleverse{17} Porque ciertamente tú das las gracias bien,
pero el otro no está edificado. \bibleverse{18} Doy gracias a mi Dios
porque hablo con otras lenguas más que todos vosotros. \bibleverse{19}
Sin embargo, en la asamblea prefiero hablar cinco palabras con mi
entendimiento, para instruir también a los demás, que diez mil palabras
en otra lengua.

\hypertarget{el-antiguo-testamento-y-el-mundo-exterior-no-cristiano-tambiuxe9n-condenan-este-incomprensible-discurso}{%
\subsection{El Antiguo Testamento y el mundo exterior no cristiano
también condenan este incomprensible
discurso}\label{el-antiguo-testamento-y-el-mundo-exterior-no-cristiano-tambiuxe9n-condenan-este-incomprensible-discurso}}

\bibleverse{20} Hermanos, no seáis niños en los pensamientos, pero en la
malicia sed bebés, pero en los pensamientos sed maduros. \footnote{\textbf{14:20}
  Efes 4,14} \bibleverse{21} En la ley está escrito: ``Por hombres de
lenguas extrañas y por labios de extraños hablaré a este pueblo. Ni
siquiera me escucharán así, dice el Señor''. \footnote{\textbf{14:21}
  Isaías 28:11-12} \bibleverse{22} Por tanto, las lenguas extrañas
sirven de señal, no para los que creen, sino para los incrédulos; pero
la profecía sirve de señal, no para los incrédulos, sino para los que
creen. \bibleverse{23} Por tanto, si toda la asamblea está reunida y
todos hablan con otras lenguas, y entran personas indoctas o incrédulas,
¿no dirán que estáis locos? \bibleverse{24} Pero si todos profetizan, y
entra alguien incrédulo o indocto, es reprendido por todos, y es juzgado
por todos. \bibleverse{25} Y así se revelan los secretos de su corazón.
Entonces se postrará sobre su rostro y adorará a Dios, declarando que
Dios está realmente entre vosotros. \footnote{\textbf{14:25} Juan 16,8}

\hypertarget{orden-de-los-altavoces}{%
\subsection{Orden de los altavoces}\label{orden-de-los-altavoces}}

\bibleverse{26} ¿Qué es, pues, hermanos? Cuando os reunís, cada uno de
vosotros tiene un salmo, tiene una enseñanza, tiene una revelación,
tiene otra lengua o tiene una interpretación. Hacedlo todo para
edificaros mutuamente. \footnote{\textbf{14:26} 1Cor 12,8-10; Efes 4,12}
\bibleverse{27} Si alguno habla en otra lengua, que sean dos, o a lo
sumo tres, y por turno, y que uno interprete. \bibleverse{28} Pero si no
hay intérprete, que guarde silencio en la asamblea, y que hable para sí
mismo y para Dios. \bibleverse{29} Que hablen dos o tres de los
profetas, y que los demás disciernan. \footnote{\textbf{14:29} 1Tes
  5,21; Hech 17,11} \bibleverse{30} Pero si se hace una revelación a
otro que esté sentado, que el primero guarde silencio. \bibleverse{31}
Porque todos pueden profetizar uno por uno, para que todos aprendan y
todos sean exhortados. \bibleverse{32} Los espíritus de los profetas
están sujetos a los profetas, \bibleverse{33} porque Dios no es un Dios
de confusión, sino de paz, como en todas las asambleas de los santos.
\footnote{\textbf{14:33} 1Cor 14,40}

\hypertarget{contra-los-discursos-inapropiados-de-mujeres-en-reuniones}{%
\subsection{Contra los discursos inapropiados de mujeres en
reuniones}\label{contra-los-discursos-inapropiados-de-mujeres-en-reuniones}}

\bibleverse{34} Que las esposas guarden silencio en las asambleas, pues
no se les ha permitido hablar sino con sumisión, como dice también la
ley, \footnote{\textbf{14:34} Deuteronomio 27:9} \footnote{\textbf{14:34}
  1Tim 2,11-12; Gén 3,16} \bibleverse{35} si desean aprender algo. ``Que
pregunten a sus propios maridos en casa, porque es vergonzoso que una
esposa esté hablando en la asamblea.'' \bibleverse{36} ¿Qué? ¿Salió de
ti la palabra de Dios? ¿O solo a vosotros ha llegado?

\bibleverse{37} Si alguno se cree profeta o espiritual, que reconozca
las cosas que os escribo, que son mandamiento del Señor. \footnote{\textbf{14:37}
  1Jn 4,6} \bibleverse{38} Pero si alguien es ignorante, que sea
ignorante.

\bibleverse{39} Por lo tanto, hermanos, desead con ahínco profetizar, y
no prohibáis hablar con otras lenguas. \bibleverse{40} Que todo se haga
decentemente y en orden. \footnote{\textbf{14:40} 1Cor 14,33; Col 2,5}

\hypertarget{de-los-hechos-y-testigos-por-los-que-se-certifica-la-resurrecciuxf3n-de-cristo}{%
\subsection{De los hechos y testigos por los que se certifica la
resurrección de
Cristo}\label{de-los-hechos-y-testigos-por-los-que-se-certifica-la-resurrecciuxf3n-de-cristo}}

\hypertarget{section-14}{%
\section{15}\label{section-14}}

\bibleverse{1} Ahora os anuncio, hermanos, la Buena Nueva que os he
predicado, que también habéis recibido, en la que también estáis firmes,
\bibleverse{2} por la que también os salváis, si retenéis firmemente la
palabra que os he predicado, a menos que hayáis creído en vano.

\bibleverse{3} Porque os he transmitido en primer lugar lo que yo
también recibí: que Cristo murió por nuestros pecados según las
Escrituras, \footnote{\textbf{15:3} Is 53,8-9} \bibleverse{4} que fue
sepultado, que resucitó al tercer día según las Escrituras, \footnote{\textbf{15:4}
  Luc 24,27; Luc 24,44-46} \bibleverse{5} y que se apareció a Cefas y
luego a los doce. \footnote{\textbf{15:5} Juan 20,19; Juan 20,26; Luc
  23,34} \bibleverse{6} Luego se apareció a más de quinientos hermanos a
la vez, la mayoría de los cuales permanecen hasta ahora, pero algunos
también se han dormido. \bibleverse{7} Luego se apareció a Santiago,
después a todos los apóstoles, \footnote{\textbf{15:7} Luc 24,50}
\bibleverse{8} y por último, como al niño nacido a destiempo, se me
apareció a mí también. \footnote{\textbf{15:8} 1Cor 9,1; Hech 9,3-6}
\bibleverse{9} Porque yo soy el más pequeño de los apóstoles, que no es
digno de ser llamado apóstol, porque perseguí a la asamblea de Dios.
\footnote{\textbf{15:9} Hech 8,3; Efes 3,8} \bibleverse{10} Pero por la
gracia de Dios soy lo que soy. Su gracia que me fue dada no fue inútil,
sino que trabajé más que todos ellos; pero no yo, sino la gracia de Dios
que estaba conmigo. \footnote{\textbf{15:10} 2Cor 11,5; 2Cor 11,23}
\bibleverse{11} Sea, pues, yo o ellos, así lo predicamos, y así lo
habéis creído.

\hypertarget{la-fe-y-la-firme-esperanza-de-todos-los-cristianos-se-basan-en-la-resurrecciuxf3n-de-cristo-de-entre-los-muertos}{%
\subsection{La fe y la firme esperanza de todos los cristianos se basan
en la resurrección de Cristo de entre los
muertos}\label{la-fe-y-la-firme-esperanza-de-todos-los-cristianos-se-basan-en-la-resurrecciuxf3n-de-cristo-de-entre-los-muertos}}

\bibleverse{12} Ahora bien, si se predica que Cristo ha resucitado de
entre los muertos, ¿cómo dicen algunos de vosotros que no hay
resurrección de los muertos? \bibleverse{13} Pero si no hay resurrección
de los muertos, tampoco Cristo ha resucitado. \bibleverse{14} Si Cristo
no ha resucitado, vana es nuestra predicación y vana es también vuestra
fe. \bibleverse{15} Sí, también nosotros somos hallados falsos testigos
de Dios, porque testificamos de Dios que resucitó a Cristo, a quien no
resucitó si es verdad que los muertos no resucitan. \footnote{\textbf{15:15}
  Hech 1,22} \bibleverse{16} Porque si los muertos no han resucitado,
tampoco Cristo ha resucitado. \bibleverse{17} Si Cristo no ha
resucitado, vuestra fe es vana; todavía estáis en vuestros pecados.
\bibleverse{18} Entonces también los que duermen en Cristo han perecido.
\bibleverse{19} Si sólo hemos esperado en Cristo en esta vida, somos los
más lamentables de todos los hombres.

\hypertarget{exposiciuxf3n-de-las-consecuencias-de-la-resurrecciuxf3n-de-cristo-los-procesos-en-los-que-la-resurrecciuxf3n-tiene-lugar-hasta-su-finalizaciuxf3n}{%
\subsection{Exposición de las consecuencias de la resurrección de
Cristo; los procesos en los que la resurrección tiene lugar hasta su
finalización}\label{exposiciuxf3n-de-las-consecuencias-de-la-resurrecciuxf3n-de-cristo-los-procesos-en-los-que-la-resurrecciuxf3n-tiene-lugar-hasta-su-finalizaciuxf3n}}

\bibleverse{20} Pero ahora Cristo ha resucitado de entre los muertos. Se
convirtió en la primicia de los que duermen. \footnote{\textbf{15:20}
  1Cor 6,14; Col 1,18} \bibleverse{21} Porque como la muerte vino por el
hombre, también la resurrección de los muertos vino por el hombre.
\footnote{\textbf{15:21} Gén 3,17-19; Rom 5,18} \bibleverse{22} Porque
así como en Adán todos mueren, también en Cristo todos serán
vivificados. \bibleverse{23} Pero cada uno en su orden: Cristo las
primicias, luego los que son de Cristo en su venida. \footnote{\textbf{15:23}
  1Tes 4,16-17} \bibleverse{24} Luego vendrá el fin, cuando entregue el
Reino a Dios Padre, cuando haya abolido todo gobierno y toda autoridad y
poder. \footnote{\textbf{15:24} Rom 8,38} \bibleverse{25} Porque es
necesario que reine hasta que haya puesto a todos sus enemigos bajo sus
pies. \footnote{\textbf{15:25} Mat 22,44} \bibleverse{26} El último
enemigo que será abolido es la muerte. \footnote{\textbf{15:26} Apoc
  20,14; Apoc 21,4} \bibleverse{27} Porque ``Todo lo sometió bajo sus
pies''.\footnote{\textbf{15:27} Salmo 8:6} Pero cuando dice: ``Todas las
cosas están sometidas'', es evidente que se exceptúa al que sometió
todas las cosas a él. \bibleverse{28} Cuando todas las cosas le hayan
sido sometidas, entonces también el Hijo se someterá al que le sometió
todas las cosas, para que Dios sea todo en todos.

\hypertarget{mucho-de-lo-que-los-cristianos-hacen-y-sufren-solo-es-justificado-y-comprensible-cuando-creen-en-la-resurrecciuxf3n}{%
\subsection{Mucho de lo que los cristianos hacen y sufren solo es
justificado y comprensible cuando creen en la
resurrección}\label{mucho-de-lo-que-los-cristianos-hacen-y-sufren-solo-es-justificado-y-comprensible-cuando-creen-en-la-resurrecciuxf3n}}

\bibleverse{29} ¿O qué harán los que se bautizan por los muertos? Si los
muertos no resucitan en absoluto, ¿por qué entonces se bautizan por los
muertos? \bibleverse{30} ¿Por qué también nosotros estamos en peligro
cada hora? \footnote{\textbf{15:30} Rom 8,36; Gal 5,11} \bibleverse{31}
Afirmo que por la jactancia que tengo en Cristo Jesús, nuestro Señor,
muero cada día. \footnote{\textbf{15:31} 2Cor 4,10} \bibleverse{32} Si
como hombre luche en Éfeso contra bestias, ¿de qué me sirve? Si los
muertos no resucitan, entonces ``comamos y bebamos, porque mañana
moriremos''. \footnote{\textbf{15:32} Isaías 22:13} \bibleverse{33} ¡No
te engañes! ``Las malas compañías corrompen las buenas costumbres''.
\bibleverse{34} Despierta con rectitud y no peques, porque algunos no
conocen a Dios. Digo esto para su vergüenza. \footnote{\textbf{15:34}
  1Tes 5,8}

\hypertarget{la-imagen-de-la-semilla}{%
\subsection{La imagen de la semilla}\label{la-imagen-de-la-semilla}}

\bibleverse{35} Pero alguien dirá: ``¿Cómo resucitan los muertos?'' y:
``¿Con qué clase de cuerpo vienen?'' \bibleverse{36} Necio, lo que tú
mismo siembras no se vivifica si no muere. \footnote{\textbf{15:36} Juan
  12,24} \bibleverse{37} Lo que tú siembras, no siembras el cuerpo que
será, sino un grano desnudo, tal vez de trigo, o de otra clase.
\bibleverse{38} Pero Dios le da un cuerpo tal como le ha gustado, y a
cada semilla un cuerpo propio. \footnote{\textbf{15:38} Gén 1,11}

\hypertarget{toda-la-creaciuxf3n-muestra-la-mayor-diversidad-de-materia-forma-y-naturaleza-de-las-cosas}{%
\subsection{Toda la creación muestra la mayor diversidad de materia,
forma y naturaleza de las
cosas}\label{toda-la-creaciuxf3n-muestra-la-mayor-diversidad-de-materia-forma-y-naturaleza-de-las-cosas}}

\bibleverse{39} No toda la carne es la misma, sino que hay una carne de
hombres, otra de animales, otra de peces y otra de aves. \bibleverse{40}
Hay también cuerpos celestes y cuerpos terrestres; pero la gloria de los
celestes difiere de la de los terrestres. \bibleverse{41} Hay una gloria
del sol, otra gloria de la luna, y otra gloria de las estrellas; porque
una estrella difiere de otra en su gloria.

\bibleverse{42} Así es también la resurrección de los muertos. El cuerpo
se siembra perecedero; resucita imperecedero. \bibleverse{43} Se siembra
en la deshonra, pero resucita en la gloria. Se siembra en la debilidad;
resucita en el poder. \footnote{\textbf{15:43} Fil 3,21; Col 3,4}
\bibleverse{44} Se siembra un cuerpo natural; se resucita un cuerpo
espiritual. Hay un cuerpo natural y hay también un cuerpo espiritual.

\hypertarget{la-realidad-de-un-cuerpo-celestial-incorruptible}{%
\subsection{La realidad de un cuerpo celestial,
incorruptible}\label{la-realidad-de-un-cuerpo-celestial-incorruptible}}

\bibleverse{45} Así también está escrito: ``El primer hombre Adán se
convirtió en un alma viviente''. \footnote{\textbf{15:45} Génesis 2:7}
El último Adán se convirtió en un espíritu viviente. \footnote{\textbf{15:45}
  2Cor 3,17} \bibleverse{46} Sin embargo, lo que es espiritual no es lo
primero, sino lo que es natural, y luego lo que es espiritual.
\bibleverse{47} El primer hombre es de la tierra, hecho de polvo. El
segundo hombre es el Señor del cielo. \bibleverse{48} Como el que está
hecho de polvo, así son los que también están hechos de polvo; y como el
celestial, así son también los celestiales. \bibleverse{49} Así como
hemos llevado la imagen de los que están hechos de polvo, llevemos
\footnote{\textbf{15:49} NU, TR dice ``vamos a'' en lugar de ``vamos a''}
también la imagen de los celestiales. \footnote{\textbf{15:49} Gén 5,3}

\hypertarget{la-transformaciuxf3n-final-en-la-consumaciuxf3n-de-los-creyentes}{%
\subsection{La transformación final en la consumación de los
creyentes}\label{la-transformaciuxf3n-final-en-la-consumaciuxf3n-de-los-creyentes}}

\bibleverse{50} Ahora bien, hermanos, digo que la carne y la sangre no
pueden heredar el Reino de Dios; ni lo perecedero hereda lo
imperecedero.

\bibleverse{51} He aquí,\footnote{\textbf{15:51} TR añade ``y estando
  convencido de''} os digo un misterio. No todos dormiremos, sino que
todos seremos transformados, \footnote{\textbf{15:51} 1Tes 4,15-17}
\bibleverse{52} en un momento, en un abrir y cerrar de ojos, a la última
trompeta. Porque sonará la trompeta y los muertos resucitarán
incorruptibles, y nosotros seremos transformados. \footnote{\textbf{15:52}
  Mat 24,31} \bibleverse{53} Porque es necesario que este cuerpo
perecedero se convierta en incorruptible, y que este mortal se vista de
inmortalidad. \footnote{\textbf{15:53} 2Cor 5,4} \bibleverse{54} Pero
cuando este cuerpo perecedero se convierta en incorruptible, y este
mortal se vista de inmortalidad, entonces sucederá lo que está escrito:
``La muerte es absorbida por la victoria''. \bibleverse{55} ``Muerte,
¿dónde está tu aguijón? Hades, ¿dónde está tu victoria?''

\bibleverse{56} El aguijón de la muerte es el pecado, y el poder del
pecado es la ley. \footnote{\textbf{15:56} Rom 7,8; Rom 7,11; Rom 7,13}
\bibleverse{57} Pero gracias a Dios, que nos da la victoria por medio de
nuestro Señor Jesucristo. \footnote{\textbf{15:57} 1Jn 5,4}
\bibleverse{58} Por lo tanto, mis amados hermanos, estad firmes,
inamovibles, abundando siempre en la obra del Señor, porque sabéis que
vuestro trabajo no es en vano en el Señor. \footnote{\textbf{15:58} 2Cró
  15,7}

\hypertarget{invitaciuxf3n-a-participar-en-la-recaudaciuxf3n-de-fondos-para-jerusaluxe9n}{%
\subsection{Invitación a participar en la recaudación de fondos para
Jerusalén}\label{invitaciuxf3n-a-participar-en-la-recaudaciuxf3n-de-fondos-para-jerusaluxe9n}}

\hypertarget{section-15}{%
\section{16}\label{section-15}}

\bibleverse{1} En cuanto a la colecta para los santos: como ordené a las
asambleas de Galacia, haced vosotros lo mismo. \footnote{\textbf{16:1}
  2Cor 8,9; Gal 2,10} \bibleverse{2} El primer día de cada semana, que
cada uno de vosotros ahorre como pueda prosperar, para que no se hagan
colectas cuando yo llegue. \footnote{\textbf{16:2} Hech 20,7}
\bibleverse{3} Cuando llegue, enviaré a quien tú apruebes con cartas
para que lleve a Jerusalén tu donativo. \bibleverse{4} Si es conveniente
que yo vaya también, irán conmigo.

\hypertarget{los-planes-de-viaje-de-pablo-y-las-noticias-de-la-venida-de-timoteo-y-apolos}{%
\subsection{Los planes de viaje de Pablo y las noticias de la venida de
Timoteo y
Apolos}\label{los-planes-de-viaje-de-pablo-y-las-noticias-de-la-venida-de-timoteo-y-apolos}}

\bibleverse{5} Iré a vosotros cuando haya pasado por Macedonia, pues
estoy pasando por Macedonia. \footnote{\textbf{16:5} Hech 19,21}
\bibleverse{6} Pero puede ser que me quede con vosotros, o incluso que
pase el invierno con vosotros, para que me enviéis de viaje a donde
quiera que vaya. \bibleverse{7} Porque no quiero veros ahora de paso,
sino que espero quedarme un tiempo con vosotros, si el Señor lo permite.
\footnote{\textbf{16:7} Hech 20,2} \bibleverse{8} Pero me quedaré en
Éfeso hasta Pentecostés, \footnote{\textbf{16:8} Hech 19,1; Hech 19,10}
\bibleverse{9} porque se me ha abierto una puerta grande y eficaz, y hay
muchos adversarios. \footnote{\textbf{16:9} 2Cor 2,12; Col 4,3}

\bibleverse{10} Ahora bien, si viene Timoteo, procurad que esté con
vosotros sin temor, porque hace la obra del Señor, como yo también.
\footnote{\textbf{16:10} 1Cor 4,17; Fil 2,19-22} \bibleverse{11} Por
tanto, que nadie lo desprecie. Antes bien, ponedlo en camino en paz,
para que venga a verme; porque lo espero con los hermanos.

\bibleverse{12} En cuanto al hermano Apolos, le insté encarecidamente a
que fuera a vosotros con los hermanos, pero no quiso en absoluto ir
ahora; pero irá cuando tenga ocasión. \footnote{\textbf{16:12} 1Cor 1,12}

\hypertarget{advertencias-finales-recomendaciones-personales-saludos-y-bendiciones}{%
\subsection{Advertencias finales, recomendaciones personales, saludos y
bendiciones}\label{advertencias-finales-recomendaciones-personales-saludos-y-bendiciones}}

\bibleverse{13} ¡Observa! ¡Manténganse firmes en la fe! ¡Sé valiente!
Sed fuertes. \footnote{\textbf{16:13} Efes 6,10} \bibleverse{14} Que
todo lo que hagáis lo hagáis con amor.

\bibleverse{15} Os ruego, hermanos, que conozcáis la casa de Estéfanas,
que es la primicia de Acaya, y que se han puesto al servicio de los
santos, \footnote{\textbf{16:15} 1Cor 1,16} \bibleverse{16} que os
sometáis también a ellos, y a todos los que ayudan en la obra y
trabajan. \footnote{\textbf{16:16} Fil 2,29} \bibleverse{17} Me alegro
de la venida de Estéfanas, Fortunato y Acáico, pues lo que os faltaba,
lo han suplido ellos. \bibleverse{18} Pues ellos refrescaron mi espíritu
y el vuestro. Reconoced, pues, a los que son así. \footnote{\textbf{16:18}
  1Tes 5,12}

\bibleverse{19} Las asambleas de Asia os saludan. Aquila y Priscila os
saludan cordialmente en el Señor, junto con la asamblea que está en su
casa. \footnote{\textbf{16:19} Hech 18,2; Rom 16,3; Rom 16,5}
\bibleverse{20} Os saludan todos los hermanos. Saludaos los unos a los
otros con un beso sagrado.

\bibleverse{21} Este saludo es de mi parte, Pablo, con mi propia mano.
\footnote{\textbf{16:21} Gal 6,11; Col 4,18; 2Tes 3,17} \bibleverse{22}
El que no ame al Señor Jesucristo, que se maldiga. Ven, Señor.
\footnote{\textbf{16:22} Apoc 22,20} \bibleverse{23} La gracia del Señor
Jesucristo esté con vosotros. \bibleverse{24} Mi amor a todos ustedes en
Cristo Jesús. Amén.
