\hypertarget{uxe1rbol-genealuxf3gico-de-jesuxfas-como-descendiente-de-abraham-y-david}{%
\subsection{Árbol genealógico de Jesús como descendiente de Abraham y
David}\label{uxe1rbol-genealuxf3gico-de-jesuxfas-como-descendiente-de-abraham-y-david}}

\hypertarget{section}{%
\section{1}\label{section}}

\bibleverse{1} El libro de la genealogía de Jesucristo,\footnote{\textbf{1:1}
  Mesías (hebreo) y Cristo (griego) significan ambos ``Ungido''} hijo de
David, hijo de Abraham. \footnote{\textbf{1:1} 1Cró 17,11; Gén 22,18}

\bibleverse{2} Abraham se convirtió en el padre de Isaac. Isaac se
convirtió en el padre de Jacob. Jacob se convirtió en el padre de Judá y
sus hermanos. \footnote{\textbf{1:2} Gén 21,3; Gén 21,12; Gén 25,26; Gén
  29,35; Gén 49,10} \bibleverse{3} Judá se convirtió en el padre de
Fares y Zara por Tamar. Fares fue el padre de Esrom. Esrom fue el padre
de Aram. \footnote{\textbf{1:3} Gén 38,29-30; Rut 4,18-22}
\bibleverse{4} Aram fue el padre de Aminadab. Aminadab fue el padre de
Naasón. Naasón fue el padre de Salmón. \bibleverse{5} Salmón fue el
padre de Booz, de Rahab. Booz fue el padre de Obed por Rut. Obed fue el
padre de Isaí. \footnote{\textbf{1:5} Jos 2,1; Rut 4,13-17}
\bibleverse{6} Isaí fue el padre del rey David. El rey\footnote{\textbf{1:6}
  NU omite ``el rey''.} David fue padre de Salomón por la que había sido
esposa de Urías. \footnote{\textbf{1:6} 2Sam 12,24} \bibleverse{7}
Salomón fue padre de Roboam. Roboam fue padre de Abías. Abías fue el
padre de Asa. \footnote{\textbf{1:7} 1Cró 3,10-16} \bibleverse{8} Asa
fue el padre de Josafat. Josafat fue el padre de Joram. Joram fue el
padre de Uzías. \bibleverse{9} Uzías fue el padre de Jotam. Jotam fue el
padre de Acaz. Acaz fue el padre de Ezequías. \bibleverse{10} Ezequías
fue padre de Manasés. Manasés fue el padre de Amón. Amón fue el padre de
Josías. \bibleverse{11} Josías se convirtió en el padre de Jechoniah y
sus hermanos en el momento del exilio a Babilonia. \footnote{\textbf{1:11}
  2Re 25,1}

\bibleverse{12} Después del exilio a Babilonia, Jechoniah se convirtió
en el padre de Salatiel. Salatiel fue el padre de Zorobabel. \footnote{\textbf{1:12}
  1Cró 3,17; Esd 3,2} \bibleverse{13} Zorobabel fue el padre de Abiud.
Abiud fue el padre de Eliaquim. Eliaquim fue el padre de Azor.
\bibleverse{14} Azor fue el padre de Sadoc. Sadoc fue el padre de Aquim.
Aquim fue el padre de Eliud. \bibleverse{15} Eliud fue el padre de
Eleazar. Eleazar fue el padre de Matán. Matán fue el padre de Jacob.
\bibleverse{16} Jacob fue el padre de José, el esposo de María, de quien
nació Jesús,\footnote{\textbf{1:16} ``Jesús'' significa ``Salvación''.}
llamado Cristo. \footnote{\textbf{1:16} Luc 1,27}

\bibleverse{17} Así que todas las generaciones desde Abraham hasta David
son catorce generaciones; desde David hasta el exilio a Babilonia,
catorce generaciones; y desde el traslado a Babilonia hasta el Cristo,
catorce generaciones.

\hypertarget{nacimiento-y-nombre-de-jesuxfas}{%
\subsection{Nacimiento y nombre de
Jesús}\label{nacimiento-y-nombre-de-jesuxfas}}

\bibleverse{18} El nacimiento de Jesucristo fue así: Después de que su
madre, María, se comprometiera con José, antes de que se juntasen, fue
hallada embarazada por el Espíritu Santo. \footnote{\textbf{1:18} Luc
  1,35} \bibleverse{19} José, su marido, siendo un hombre justo, y no
queriendo hacer de ella un ejemplo público, pensaba repudiarla en
secreto. \bibleverse{20} Pero cuando pensaba en estas cosas, he aquí que
\footnote{\textbf{1:20} ``Contemplar'', de ``\greek{ἰδοὺ}'', significa
  mirar, fijarse, observar, ver o contemplar. Se utiliza a menudo como
  interjección.} un ángel del Señor se le apareció en sueños, diciendo:
``José, hijo de David, no temas recibir a María como esposa, porque lo
que ha sido concebido en ella es del Espíritu Santo. \bibleverse{21}
Ella dará a luz un hijo. Le pondrás el nombre de Jesús, \footnote{\textbf{1:21}
  ``Jesús'' significa ``Salvación''.} porque es él quien salvará a su
pueblo de sus pecados''. \footnote{\textbf{1:21} Sal 130,8; Luc 2,21;
  Hech 4,12}

\bibleverse{22} Todo esto ha sucedido para que se cumpla lo dicho por el
Señor por medio del profeta, que dijo \bibleverse{23} ``He aquí que la
virgen quedará encinta, y dará a luz un hijo. Llamarán su nombre
Emanuel''. que es, interpretado, ``Dios con nosotros''. \footnote{\textbf{1:23}
  Isaías 7:14}

\bibleverse{24} José se levantó de su sueño e hizo lo que el ángel del
Señor le había ordenado, y tomó a su mujer para sí; \bibleverse{25} y no
la conoció sexualmente hasta que dio a luz a su hijo primogénito. Le
puso el nombre de Jesús. \footnote{\textbf{1:25} Luc 2,7}

\hypertarget{los-magos-del-oriente-vienen-al-niuxf1o-jesuxfas-y-le-rinden-homenaje}{%
\subsection{Los magos del oriente vienen al niño Jesús y le rinden
homenaje}\label{los-magos-del-oriente-vienen-al-niuxf1o-jesuxfas-y-le-rinden-homenaje}}

\hypertarget{section-1}{%
\section{2}\label{section-1}}

\bibleverse{1} Cuando Jesús nació en Belén de Judea, en tiempos del rey
Herodes, vinieron a Jerusalén unos\footnote{\textbf{2:1} La palabra
  ``sabios'' (magoi) también puede significar maestros, científicos,
  médicos, astrólogos, videntes, intérpretes de sueños o hechiceros.}
magos de Oriente, diciendo: \footnote{\textbf{2:1} Luc 2,1-7}
\bibleverse{2} ``¿Dónde está el que ha nacido como Rey de los judíos?
Porque hemos visto su estrella en el oriente y hemos venido a
adorarle''. \footnote{\textbf{2:2} Núm 24,17} \bibleverse{3} Al oírlo,
el rey Herodes se turbó, y toda Jerusalén con él. \bibleverse{4}
Reuniendo a todos los jefes de los sacerdotes y a los escribas del
pueblo, les preguntó dónde iba a nacer el Cristo. \bibleverse{5} Ellos
le respondieron: ``En Belén de Judea, porque así está escrito por el
profeta, \footnote{\textbf{2:5} Juan 7,42} \bibleverse{6} `Tú Belén,
tierra de Judá, no son en absoluto los menos importantes entre los
príncipes de Judá; porque de ti saldrá un gobernadorque pastoreará a mi
pueblo, Israel''. \footnote{\textbf{2:6} Miqueas 5:2}

\bibleverse{7} Entonces Herodes llamó en secreto a los sabios y se
enteró por ellos de la hora exacta en que apareció la estrella.
\bibleverse{8} Los envió a Belén y les dijo: ``Vayan y busquen
diligentemente al niño. Cuando lo hayáis encontrado, traedme la noticia,
para que yo también vaya a adorarlo''.

\bibleverse{9} Ellos, habiendo oído al rey, se pusieron en camino; y he
aquí que la estrella que habían visto en el oriente, iba delante de
ellos hasta que llegó y se paró sobre donde estaba el niño.
\bibleverse{10} Al ver la estrella, se alegraron mucho. \bibleverse{11}
Entraron en la casa y vieron al niño con María, su madre, y se postraron
y lo adoraron. Abriendo sus tesoros, le ofrecieron regalos: oro,
incienso y mirra. \footnote{\textbf{2:11} Sal 72,10; Sal 72,15; Is 60,6}
\bibleverse{12} Al ser advertidos en sueños de que no debían volver a
Herodes, regresaron a su país por otro camino.

\hypertarget{la-huida-de-josuxe9-a-egipto}{%
\subsection{La huida de José a
Egipto}\label{la-huida-de-josuxe9-a-egipto}}

\bibleverse{13} Cuando se fueron, he aquí que un ángel del Señor se le
apareció a José en sueños, diciendo: ``Levántate y toma al niño y a su
madre, y huye a Egipto, y quédate allí hasta que yo te diga, porque
Herodes buscará al niño para destruirlo.''

\bibleverse{14} Se levantó, tomó al niño y a su madre de noche y se
marchó a Egipto, \bibleverse{15} y estuvo allí hasta la muerte de
Herodes, para que se cumpliera lo que había dicho el Señor por medio del
profeta: ``De Egipto llamé a mi hijo.'' \footnote{\textbf{2:15} Oseas
  11:1}

\hypertarget{el-asesinato-del-niuxf1o-de-herodes-en-beluxe9n}{%
\subsection{El asesinato del niño de Herodes en
Belén}\label{el-asesinato-del-niuxf1o-de-herodes-en-beluxe9n}}

\bibleverse{16} Entonces Herodes, cuando se vio burlado por los magos,
se enojó mucho y mandó matar a todos los niños varones que había en
Belén y en toda la campiña de los alrededores, de dos años para abajo,
según el tiempo exacto que había aprendido de los sabios.
\bibleverse{17} Entonces se cumplió lo dicho por el profeta Jeremías,
que dijo \bibleverse{18} ``Se oyó una voz en Ramá, lamento, llanto y
gran luto, Raquel llorando por sus hijos; no se consolaría, porque ya no
existen\footnote{\textbf{2:18} Jeremías 31:15} ''. \footnote{\textbf{2:18}
  Gén 35,19}

\hypertarget{el-regreso-de-josuxe9-de-egipto-y-su-asentamiento-en-nazaret}{%
\subsection{El regreso de José de Egipto y su asentamiento en
Nazaret}\label{el-regreso-de-josuxe9-de-egipto-y-su-asentamiento-en-nazaret}}

\bibleverse{19} Pero cuando Herodes murió, he aquí que un ángel del
Señor se le apareció en sueños a José en Egipto, diciendo:
\bibleverse{20} ``Levántate y toma al niño y a su madre, y vete a la
tierra de Israel, porque los que buscaban la vida del niño han muerto.''
\footnote{\textbf{2:20} Éxod 4,19}

\bibleverse{21} Se levantó, tomó al niño y a su madre y se fue a la
tierra de Israel. \bibleverse{22} Pero cuando se enteró de que Arquelao
reinaba en Judea en lugar de su padre, Herodes, tuvo miedo de ir allí.
Advertido en sueños, se retiró a la región de Galilea, \bibleverse{23} y
vino a vivir a una ciudad llamada Nazaret, para que se cumpliera lo
dicho por los profetas de que sería llamado nazareno. \footnote{\textbf{2:23}
  Luc 2,39; Juan 1,46}

\hypertarget{apariciuxf3n-y-sermuxf3n-penitencial-de-juan-el-bautista}{%
\subsection{Aparición y sermón penitencial de Juan el
Bautista}\label{apariciuxf3n-y-sermuxf3n-penitencial-de-juan-el-bautista}}

\hypertarget{section-2}{%
\section{3}\label{section-2}}

\bibleverse{1} En aquellos días, vino Juan el Bautista predicando en el
desierto de Judea, diciendo: \footnote{\textbf{3:1} Luc 1,13}
\bibleverse{2} ``¡Arrepentíos, porque el Reino de los Cielos está
cerca!'' \footnote{\textbf{3:2} Mat 4,17; Rom 12,2} \bibleverse{3}
Porque éste es el que fue anunciado por el profeta Isaías, diciendo,
``La voz de uno que clama en el desierto, ¡preparen el camino del Señor!
Endereza sus caminos''. \footnote{\textbf{3:3} Isaías 40:3} \footnote{\textbf{3:3}
  Juan 1,23}

\bibleverse{4} El mismo Juan llevaba ropa de pelo de camello y un
cinturón de cuero alrededor de la cintura. Su comida era chapulines y
miel silvestre. \footnote{\textbf{3:4} 2Re 1,8} \bibleverse{5} Entonces
la gente de Jerusalén, de toda Judea y de toda la región del Jordán
salía hacia él. \bibleverse{6} Se dejaban bautizar por él en el Jordán,
confesando sus pecados.

\bibleverse{7} Pero al ver que muchos de los fariseos y saduceos venían
a su bautismo, les dijo: ``Hijos de víboras, ¿quién os ha advertido que
huyáis de la ira que ha de venir? \footnote{\textbf{3:7} Mat 23,33}
\bibleverse{8} Por lo tanto, ¡produzcan un fruto digno de
arrepentimiento! \bibleverse{9} No penséis para vosotros mismos:
``Tenemos a Abraham por padre'', porque os digo que Dios puede levantar
hijos a Abraham de estas piedras. \footnote{\textbf{3:9} Juan 8,33; Juan
  8,39; Rom 2,28-29; Rom 4,12} \bibleverse{10} Incluso ahora el hacha
está a la raíz de los árboles. Por eso, todo árbol que no da buen fruto
es cortado y echado al fuego. \footnote{\textbf{3:10} Luc 13,6-9}

\bibleverse{11} ``Yo sí os bautizo en agua para que os arrepintáis, pero
el que viene detrás de mí es más poderoso que yo, cuyas sandalias no soy
digno de llevar. Él os bautizará en el Espíritu Santo. \footnote{\textbf{3:11}
  TR y NU añaden ``y con fuego''} \footnote{\textbf{3:11} Juan 1,26-27;
  Juan 1,33; Hech 1,5; Hech 2,3-4} \bibleverse{12} Tiene en la mano su
aventador, y limpiará a fondo su era. Recogerá su trigo en el granero,
pero la paja la quemará con fuego inextinguible.'' \footnote{\textbf{3:12}
  Mat 13,30}

\hypertarget{el-bautismo-y-la-consagraciuxf3n-del-mesuxedas-de-jesuxfas}{%
\subsection{El bautismo y la consagración del Mesías de
Jesús}\label{el-bautismo-y-la-consagraciuxf3n-del-mesuxedas-de-jesuxfas}}

\bibleverse{13} Entonces Jesús vino de Galilea al Jordán\footnote{\textbf{3:13}
  decir, el río Jordán} , a Juan, para ser bautizado por él.
\bibleverse{14} Pero Juan se lo impedía, diciendo: ``Tengo necesidad de
ser bautizado por ti, ¿y tú vienes a mí?'' \footnote{\textbf{3:14} Juan
  13,6}

\bibleverse{15} Pero Jesús, respondiendo, le dijo: ``Permítelo ahora,
porque éste es el camino adecuado para cumplir toda justicia.'' Entonces
se lo permitió.

\bibleverse{16} Jesús, después de ser bautizado, subió directamente del
agua; y he aquí que se le abrieron los cielos. Vio que el Espíritu de
Dios descendía como una paloma y venía sobre él. \footnote{\textbf{3:16}
  Is 11,2} \bibleverse{17} He aquí que una voz de los cielos decía:
``Este es mi Hijo amado, en quien me complazco.'' \footnote{\textbf{3:17}
  Mat 17,5; Is 42,1}

\hypertarget{la-tentaciuxf3n-de-jesuxfas-como-prueba-de-mesuxedas}{%
\subsection{La tentación de Jesús como prueba de
Mesías}\label{la-tentaciuxf3n-de-jesuxfas-como-prueba-de-mesuxedas}}

\hypertarget{section-3}{%
\section{4}\label{section-3}}

\bibleverse{1} Entonces Jesús fue llevado por el Espíritu al desierto
para ser tentado por el diablo. \footnote{\textbf{4:1} Heb 4,15}
\bibleverse{2} Después de haber ayunado cuarenta días y cuarenta noches,
tuvo hambre. \footnote{\textbf{4:2} Éxod 34,28; 1Re 19,8} \bibleverse{3}
Se acercó el tentador y le dijo: ``Si eres el Hijo de Dios, ordena que
estas piedras se conviertan en pan''. \footnote{\textbf{4:3} Gén 3,1-7}

\bibleverse{4} Pero él respondió: ``Está escrito que no sólo de pan vive
el hombre, sino de toda palabra que sale de la boca de Dios''.
\footnote{\textbf{4:4} Deuteronomio 8:3}

\bibleverse{5} Entonces el diablo lo llevó a la ciudad santa. Lo puso en
el pináculo del templo, \bibleverse{6} y le dijo: ``Si eres el Hijo de
Dios, tírate al suelo, porque está escrito,`Él ordenará a sus ángeles
con respecto a ti,' y, En sus manos te llevarán, para no tropezar con
una piedra''. \footnote{\textbf{4:6} Salmo 91:11-12}

\bibleverse{7} Jesús le dijo: ``También está escrito: ``No pondrás a
prueba al Señor, tu Dios''\,''. \footnote{\textbf{4:7} Deuteronomio 6:16}

\bibleverse{8} De nuevo, el diablo lo llevó a un monte muy alto, y le
mostró todos los reinos del mundo y su gloria. \bibleverse{9} Le dijo:
``Te daré todas estas cosas, si te postras y me adoras''. \footnote{\textbf{4:9}
  Mat 16,26}

\bibleverse{10} Entonces Jesús le dijo: ``¡Quítate de encima,
\footnote{\textbf{4:10} TR y NU leen ``Vete'' en lugar de ``Ponte detrás
  de mí''} Satanás! Porque está escrito: `Al Señor tu Dios adorarás y a
él sólo servirás'\,''. \footnote{\textbf{4:10} Deuteronomio 6:13}

\bibleverse{11} Entonces el diablo lo dejó, y he aquí que vinieron
ángeles y le sirvieron. \footnote{\textbf{4:11} Juan 1,51; Heb 1,6; Heb
  1,14}

\hypertarget{jesuxfas-asume-enseuxf1ar-en-capernaum}{%
\subsection{Jesús asume enseñar en
Capernaum}\label{jesuxfas-asume-enseuxf1ar-en-capernaum}}

\bibleverse{12} Cuando Jesús oyó que Juan había sido entregado, se
retiró a Galilea. \footnote{\textbf{4:12} Mat 14,3} \bibleverse{13}
Dejando a Nazaret, vino a vivir a Capernaum, que está junto al mar, en
la región de Zabulón y Neftalí, \bibleverse{14} para que se cumpliera lo
que se había dicho por medio del profeta Isaías, que decía
\bibleverse{15} ``La tierra de Zabulón y la tierra de Neftalí, hacia el
mar, más allá del Jordán, Galilea de los Gentiles, \bibleverse{16} el
pueblo que estaba sentado en la oscuridad vio una gran luz; a los que
estaban sentados en la región y la sombra de la muerte, para ellos ha
amanecido la luz\footnote{\textbf{4:16} Isaías 9:1-2} ''. \footnote{\textbf{4:16}
  Juan 8,12}

\bibleverse{17} Desde entonces, Jesús comenzó a predicar y a decir:
``¡Arrepiéntanse! Porque el Reino de los Cielos está cerca''.
\footnote{\textbf{4:17} Mat 3,2}

\hypertarget{jesuxfas-llama-a-los-dos-primeros-pares-de-discuxedpulos}{%
\subsection{Jesús llama a los dos primeros pares de
discípulos}\label{jesuxfas-llama-a-los-dos-primeros-pares-de-discuxedpulos}}

\bibleverse{18} Caminando junto al mar de Galilea, \footnote{\textbf{4:18}
  TR lee ``Jesús'' en lugar de ``él''} vio a dos hermanos: Simón, que se
llama Pedro, y Andrés, su hermano, echando la red en el mar, pues eran
pescadores. \bibleverse{19} Les dijo: ``Venid en pos de mí, y os haré
pescadores de hombres''. \footnote{\textbf{4:19} Mat 28,19-20}

\bibleverse{20} Al instante dejaron las redes y le siguieron.
\footnote{\textbf{4:20} Mat 19,27} \bibleverse{21} Al salir de allí, vio
a otros dos hermanos, Santiago, hijo de Zebedeo, y Juan, su hermano, en
la barca con el padre de Zebedeo, remendando las redes. Los llamó.
\bibleverse{22} Ellos dejaron inmediatamente la barca y a su padre, y le
siguieron.

\hypertarget{descripciuxf3n-de-los-efectos-de-enseuxf1anza-y-curaciuxf3n-de-jesuxfas-y-su-uxe9xito}{%
\subsection{Descripción de los efectos de enseñanza y curación de Jesús
y su
éxito}\label{descripciuxf3n-de-los-efectos-de-enseuxf1anza-y-curaciuxf3n-de-jesuxfas-y-su-uxe9xito}}

\bibleverse{23} Jesús recorría toda Galilea, enseñando en sus sinagogas,
predicando la Buena Nueva del Reino y curando toda enfermedad y toda
dolencia en el pueblo. \footnote{\textbf{4:23} Mar 1,39; Luc 4,44}
\bibleverse{24} La noticia sobre él llegó a toda Siria. Le llevaban a
todos los enfermos, aquejados de diversas enfermedades y tormentos,
endemoniados, epilépticos y paralíticos; y los curaba. \footnote{\textbf{4:24}
  Mar 6,55} \bibleverse{25} Le seguían grandes multitudes de Galilea,
Decápolis, Jerusalén, Judea y del otro lado del Jordán. \footnote{\textbf{4:25}
  Mar 3,7-8; Luc 6,17-19}

\hypertarget{el-sermuxf3n-del-monte}{%
\subsection{El sermón del monte}\label{el-sermuxf3n-del-monte}}

\hypertarget{section-4}{%
\section{5}\label{section-4}}

\bibleverse{1} Al ver las multitudes, subió al monte. Cuando se sentó,
sus discípulos se acercaron a él. \bibleverse{2} Abrió la boca y les
enseñó, diciendo,

\hypertarget{las-bienaventuranzas}{%
\subsection{Las Bienaventuranzas}\label{las-bienaventuranzas}}

\bibleverse{3} ``Bienaventurados los pobres de espíritu, porque de ellos
es el Reino de los Cielos. \footnote{\textbf{5:3} Isaías 57:15; 66:2}
\footnote{\textbf{5:3} Sal 51,17; Is 57,15} \bibleverse{4}
Bienaventurados los que lloran, porque serán consolados. \footnote{\textbf{5:4}
  Isaías 61:2; 66:10,13} \footnote{\textbf{5:4} Sal 126,5; Apoc 7,17}
\bibleverse{5} Benditos sean los gentiles, porque ellos heredarán la
tierra. \footnote{\textbf{5:5} o, tierra.} \footnote{\textbf{5:5} Salmo
  37:11} \footnote{\textbf{5:5} Mat 11,29; Sal 37,11} \bibleverse{6}
Dichosos los que tienen hambre y sed de justicia, porque se llenarán.
\footnote{\textbf{5:6} Luc 18,9-14; Juan 6,35} \bibleverse{7} Benditos
sean los misericordiosos, porque obtendrán misericordia. \footnote{\textbf{5:7}
  Mat 25,35-46; Sant 2,13} \bibleverse{8} Bienaventurados los puros de
corazón, porque verán a Dios. \footnote{\textbf{5:8} Sal 24,3-5; Sal
  51,10; 1Jn 3,2; 1Jn 1,3} \bibleverse{9} Dichosos los pacificadores,
porque serán llamados hijos de Dios. \footnote{\textbf{5:9} Heb 12,14}
\bibleverse{10} Bienaventurados los que han sido perseguidos por causa
de la justicia, porque de ellos es el Reino de los Cielos. \footnote{\textbf{5:10}
  1Pe 3,14}

\bibleverse{11} ``Bienaventurados sois cuando os reprochen, os persigan
y digan toda clase de mal contra vosotros con falsedad, por mi causa.
\footnote{\textbf{5:11} Mat 10,22; Hech 5,41; 1Pe 4,14} \bibleverse{12}
Alegraos y regocijaos, porque vuestra recompensa es grande en el cielo.
Porque así persiguieron a los profetas que os precedieron. \footnote{\textbf{5:12}
  Sant 5,10; Heb 11,33-38}

\hypertarget{sal-de-la-tierra-luz-del-mundo}{%
\subsection{Sal de la tierra, luz del
mundo}\label{sal-de-la-tierra-luz-del-mundo}}

\bibleverse{13} ``Vosotros sois la sal de la tierra; pero si la sal ha
perdido su sabor, ¿con qué se salará? Entonces no sirve para nada, sino
para ser arrojada y pisoteada por los hombres. \footnote{\textbf{5:13}
  Mar 9,50; Luc 14,34-35}

\bibleverse{14} Tú eres la luz del mundo. Una ciudad situada en una
colina no se puede ocultar. \footnote{\textbf{5:14} Juan 8,12}
\bibleverse{15} Tampoco se enciende una lámpara y se pone debajo de una
cesta de medir, sino sobre un candelero; y brilla para todos los que
están en la casa. \footnote{\textbf{5:15} Mar 4,21; Luc 8,16}
\bibleverse{16} Así brille vuestra luz delante de los hombres, para que
vean vuestras buenas obras y glorifiquen a vuestro Padre que está en los
cielos. \footnote{\textbf{5:16} Juan 15,8; Efes 5,8-9; Fil 2,14-15}

\hypertarget{perfecciuxf3n-comparada-con-las-exigencias-del-antiguo-pacto}{%
\subsection{Perfección comparada con las exigencias del antiguo
pacto}\label{perfecciuxf3n-comparada-con-las-exigencias-del-antiguo-pacto}}

\bibleverse{17} ``No penséis que he venido a destruir la ley o los
profetas. No he venido a destruir, sino a cumplir. \footnote{\textbf{5:17}
  Mat 3,15; Rom 3,31; 1Jn 2,7} \bibleverse{18} Porque de cierto os digo
que hasta que pasen el cielo y la tierra, ni una letra\footnote{\textbf{5:18}
  literalmente, iota} mínima ni un trazo\footnote{\textbf{5:18} o, serif}
de pluma pasarán de la ley, hasta que todo se cumpla. \footnote{\textbf{5:18}
  Luc 16,17} \bibleverse{19} Por lo tanto, el que quebrante uno de estos
mandamientos más pequeños y enseñe a otros a hacerlo, será llamado el
más pequeño en el Reino de los Cielos; pero el que los cumpla y los
enseñe será llamado grande en el Reino de los Cielos. \footnote{\textbf{5:19}
  Sant 2,10} \bibleverse{20} Porque os digo que si vuestra justicia no
es mayor que la de los escribas y fariseos, no entraréis en el Reino de
los Cielos. \footnote{\textbf{5:20} Mat 23,2-33}

\hypertarget{acerca-de-matar-y-juzgar}{%
\subsection{Acerca de matar y juzgar}\label{acerca-de-matar-y-juzgar}}

\bibleverse{21} ``Habéis oído que a los antiguos se les dijo: ``No
matarás'',\footnote{\textbf{5:21} Éxodo 20:13} y que ``quien asesine
correrá peligro de ser juzgado''. \bibleverse{22} Pero yo os digo que
todo el que se enoje con su hermano sin causa,\footnote{\textbf{5:22} NU
  omite ``sin causa''.} estará en peligro del juicio. El que diga a su
hermano: ``¡Raca!\footnote{\textbf{5:22} ``Raca'' es un insulto arameo,
  relacionado con la palabra ``vacío'' y que transmite la idea de cabeza
  hueca.} '', correrá el peligro del consejo. El que diga: ``¡Necio!'',
correrá el peligro del fuego de la Gehena. \footnote{\textbf{5:22} o,
  Infierno} \footnote{\textbf{5:22} 1Jn 3,15}

\bibleverse{23} ``Por tanto, si estás ofreciendo tu ofrenda en el altar,
y allí te acuerdas de que tu hermano tiene algo contra ti,
\bibleverse{24} deja tu ofrenda allí, ante el altar, y sigue tu camino.
Primero reconcíliate con tu hermano, y luego ven a ofrecer tu ofrenda.
\footnote{\textbf{5:24} Mar 11,25} \bibleverse{25} Ponte de acuerdo con
tu adversario rápidamente mientras estás con él en el camino; no sea que
el fiscal te entregue al juez, y el juez te entregue al oficial, y seas
echado a la cárcel. \footnote{\textbf{5:25} Mat 18,23-35; Luc 12,58-59}
\bibleverse{26} De cierto te digo que no saldrás de allí hasta que hayas
pagado el último centavo. \footnote{\textbf{5:26} literalmente,
  kodrantes. Un kodrante era una pequeña moneda de cobre que valía
  alrededor de 2 leptas (ácaros de viuda), lo cual no era suficiente
  para comprar mucho.}

\hypertarget{sobre-adulteria}{%
\subsection{Sobre adulteria}\label{sobre-adulteria}}

\bibleverse{27} ``Habéis oído que se dijo: ``\footnote{\textbf{5:27} El
  TR añade ``a los antiguos''.} No cometerás adulterio''\footnote{\textbf{5:27}
  Éxodo 20:14} ; \bibleverse{28} pero yo os digo que todo el que mira a
una mujer para codiciarla, ya ha cometido adulterio con ella en su
corazón. \footnote{\textbf{5:28} 2Sam 11,2; Job 31,1; 2Pe 2,14}
\bibleverse{29} Si tu ojo derecho te hace tropezar, sácalo y arrójalo
lejos de ti. Porque más te vale que perezca uno de tus miembros que todo
tu cuerpo sea arrojado a la Gehenna. \footnote{\textbf{5:29} o, Infierno}
\footnote{\textbf{5:29} Mat 18,8-9; Mar 9,43; Mar 9,47; Col 3,5}
\bibleverse{30} Si tu mano derecha te hace tropezar, córtala y arrójala
lejos de ti. Porque más te conviene que perezca uno de tus miembros, que
no que todo tu cuerpo sea arrojado a la Gehenna. \footnote{\textbf{5:30}
  o, el infierno}

\bibleverse{31} ``También se dijo: ``El que repudie a su mujer, que le
dé carta de divorcio'', \footnote{\textbf{5:31} Deuteronomio 24:1}
\footnote{\textbf{5:31} Mat 19,3-9; Mar 10,4-12} \bibleverse{32} pero yo
os digo que el que repudia a su mujer, salvo por causa de inmoralidad
sexual, la convierte en adúltera; y el que se casa con ella estando
repudiada, comete adulterio. \footnote{\textbf{5:32} Luc 16,18; 1Cor
  7,10-11}

\hypertarget{sobre-jurar}{%
\subsection{Sobre jurar}\label{sobre-jurar}}

\bibleverse{33} ``Habéis oído que se dijo a los antiguos: `No
perjurarás, sino que cumplirás al Señor tus juramentos"\footnote{\textbf{5:33}
  Números 30:2; Deuteronomio 23:21; Eclesiastés 5:4} , \bibleverse{34}
pero yo os digo que no juréis en absoluto: ni por el cielo, porque es el
trono de Dios; \footnote{\textbf{5:34} Mat 2,16-22; Is 66,1}
\bibleverse{35} ni por la tierra, porque es el escabel de sus pies; ni
por Jerusalén, porque es la ciudad del gran Rey. \footnote{\textbf{5:35}
  Sal 48,2} \bibleverse{36} Tampoco jurarás por tu cabeza, porque no
puedes hacer blanco ni negro un solo cabello. \bibleverse{37} Pero que
tu ``Sí'' sea ``Sí'' y tu ``No'' sea ``No''. Todo lo que sea más que
esto es del maligno. \footnote{\textbf{5:37} Sant 5,12}

\hypertarget{sobre-lamor-al-projimo-y-al-enemigo}{%
\subsection{Sobre l'amor al projimo y al
enemigo}\label{sobre-lamor-al-projimo-y-al-enemigo}}

\bibleverse{38} ``Habéis oído que se dijo: ``Ojo por ojo y diente por
diente''. \footnote{\textbf{5:38} Éxodo 21:24; Levítico 24:20;
  Deuteronomio 19:21} \bibleverse{39} Pero yo os digo que no resistáis
al que es malo, sino que al que te golpee en tu mejilla derecha,
vuélvele también la otra. \footnote{\textbf{5:39} Lam 3,27-32; Juan
  18,22-23; Rom 12,19; Rom 12,21; 1Pe 2,20-23} \bibleverse{40} Si
alguien te demanda para quitarte la túnica, déjale también el manto.
\footnote{\textbf{5:40} 1Cor 6,7; Heb 10,34} \bibleverse{41} El que te
obligue a recorrer una milla, ve con él dos. \bibleverse{42} Da al que
te pida, y no rechaces al que quiera pedirte prestado.

\bibleverse{43} ``Habéis oído que se dijo: `Amarás a tu prójimo y
\footnote{\textbf{5:43} Levítico 19:18} odiarás a tu enemigo'.
\footnote{\textbf{5:43} no aparece en la Biblia, pero véase el Manual de
  Disciplina de Qumrán Ix, 21-26} \bibleverse{44} Pero yo os digo: amad
a vuestros enemigos, bendecid a los que os maldicen, haced el bien a los
que os odian y orad por los que os maltratan y os persiguen, \footnote{\textbf{5:44}
  Éxod 23,4-5; Luc 6,27-28; Luc 23,34; Rom 12,14; Rom 12,20; Hech 7,59}
\bibleverse{45} para que seáis hijos de vuestro Padre que está en los
cielos. Porque él hace salir su sol sobre malos y buenos, y hace llover
sobre justos e injustos. \footnote{\textbf{5:45} Efes 5,1}
\bibleverse{46} Porque si amáis a los que os aman, ¿qué recompensa
tendréis? ¿Acaso no hacen lo mismo los recaudadores de impuestos?
\bibleverse{47} Si sólo saludáis a vuestros amigos, ¿qué más hacéis
vosotros que los demás? ¿Acaso no \footnote{\textbf{5:47} NU lee
  ``gentiles'' en lugar de ``recaudadores de impuestos''.} hacen lo
mismo los recaudadores de impuestos? \bibleverse{48} Por eso seréis
perfectos, como vuestro Padre que está en los cielos es perfecto.
\footnote{\textbf{5:48} Lev 19,2}

\hypertarget{ten-cuidado-al-dar-limosna}{%
\subsection{Ten cuidado al dar
limosna}\label{ten-cuidado-al-dar-limosna}}

\hypertarget{section-5}{%
\section{6}\label{section-5}}

\bibleverse{1} ``Tened cuidado de no hacer vuestras obras de caridad
\footnote{\textbf{6:1} NU lee ``actos de justicia'' en lugar de
  ``donaciones caritativas''} delante de los hombres, para ser vistos
por ellos, pues de lo contrario no tendréis recompensa de vuestro Padre
que está en los cielos. \bibleverse{2} Por eso, cuando hagáis obras de
caridad, no hagáis sonar la trompeta delante de vosotros, como hacen los
hipócritas en las sinagogas y en las calles, para obtener la gloria de
los hombres. Ciertamente os digo que ya han recibido su recompensa.
\footnote{\textbf{6:2} 1Cor 13,3} \bibleverse{3} Pero cuando hagas obras
de misericordia, no dejes que tu mano izquierda sepa lo que hace tu mano
derecha, \footnote{\textbf{6:3} Mat 25,37-40; Rom 12,8} \bibleverse{4}
para que tus obras de misericordia estén en secreto, entonces tu Padre
que ve en secreto te recompensará abiertamente.

\hypertarget{ten-cuidado-cuando-oras}{%
\subsection{Ten cuidado cuando oras}\label{ten-cuidado-cuando-oras}}

\bibleverse{5} ``Cuando oréis, no seáis como los hipócritas, pues les
gusta estar de pie y orar en las sinagogas y en las esquinas de las
calles, para ser vistos por los hombres. Ciertamente, os digo que han
recibido su recompensa. \bibleverse{6} Pero tú, cuando ores, entra en tu
cuarto interior, y habiendo cerrado la puerta, ora a tu Padre que está
en secreto; y tu Padre, que ve en secreto, te recompensará abiertamente.
\bibleverse{7} Al orar, no utilices vanas repeticiones, como hacen los
gentiles, pues piensan que serán escuchados por su mucho hablar.
\footnote{\textbf{6:7} Is 1,15} \bibleverse{8} No seáis, pues, como
ellos, porque vuestro Padre sabe lo que necesitáis antes de que se lo
pidáis. \bibleverse{9} Orad así: ``\,`Padre nuestro que estás en el
cielo, que tu nombre sea santificado'. \footnote{\textbf{6:9} Ezeq
  36,23; Luc 11,2-4} \bibleverse{10} Que venga tu Reino. Que se haga tu
voluntad en la tierra como en el cielo. \footnote{\textbf{6:10} Luc
  22,42} \bibleverse{11} Danos hoy el pan de cada día. \bibleverse{12}
Perdona nuestras deudas, así como nosotros perdonamos a nuestros
deudores. \footnote{\textbf{6:12} Mat 18,21-35} \bibleverse{13} No nos
dejes caer en la tentación, pero líbranos del maligno. Porque tuyo es el
Reino, el poder y la gloria por siempre. Amén.''\footnote{\textbf{6:13}
  NU omite ``Porque tuyo es el Reino, el poder y la gloria por siempre.
  Amén''.} \footnote{\textbf{6:13} 1Cró 29,11-13; Juan 17,15}

\bibleverse{14} ``Porque si perdonáis a los hombres sus ofensas, también
vuestro Padre celestial os perdonará a vosotros. \bibleverse{15} Pero si
no perdonáis a los hombres sus ofensas, tampoco vuestro Padre os
perdonará vuestras ofensas. \footnote{\textbf{6:15} Mar 11,25-26}

\hypertarget{ten-cuidado-cuando-ayunas}{%
\subsection{Ten cuidado cuando ayunas}\label{ten-cuidado-cuando-ayunas}}

\bibleverse{16} ``Además, cuando ayunéis, no seáis como los hipócritas,
con rostros tristes. Porque ellos desfiguran sus rostros para que los
hombres vean que están ayunando. Ciertamente os digo que han recibido su
recompensa. \footnote{\textbf{6:16} Is 58,5-9} \bibleverse{17} Pero
vosotros, cuando ayunéis, ungid vuestra cabeza y lavad vuestra cara,
\bibleverse{18} para que no os vean los hombres ayunando, sino vuestro
Padre que está en secreto; y vuestro Padre, que ve en secreto, os
recompensará.

\hypertarget{recoge-tesoros-en-el-cielo}{%
\subsection{Recoge tesoros en el
cielo}\label{recoge-tesoros-en-el-cielo}}

\bibleverse{19} ``No os hagáis tesoros en la tierra, donde la polilla y
el orín se consumen, y donde los ladrones se cuelan y roban;
\bibleverse{20} sino haceos tesoros en el cielo, donde ni la polilla ni
el orín se consumen, y donde los ladrones no se cuelan ni roban;
\footnote{\textbf{6:20} Mat 19,21; Luc 12,33-34; Col 3,1-2}
\bibleverse{21} porque donde esté vuestro tesoro, allí estará también
vuestro corazón.

\bibleverse{22} ``La lámpara del cuerpo es el ojo. Por tanto, si tu ojo
es sano, todo tu cuerpo estará lleno de luz. \footnote{\textbf{6:22} Luc
  11,34-36} \bibleverse{23} Pero si tu ojo es malo, todo tu cuerpo
estará lleno de tinieblas. Por tanto, si la luz que hay en ti es
oscuridad, ¡qué grandes son las tinieblas! \footnote{\textbf{6:23} Juan
  11,10}

\bibleverse{24} ``Nadie puede servir a dos señores, porque o bien odiará
a uno y amará al otro, o bien se dedicará a uno y despreciará al otro.
No se puede servir a la vez a Dios y a Mammón. \footnote{\textbf{6:24}
  Luc 16,9; Luc 16,13; Sant 4,4}

\hypertarget{busque-el-reino-de-dios-primero}{%
\subsection{Busque el reino de Dios
primero}\label{busque-el-reino-de-dios-primero}}

\bibleverse{25} Por eso os digo que no os preocupéis por vuestra vida:
qué vais a comer o qué vais a beber; ni tampoco por vuestro cuerpo, qué
vais a vestir. ¿No es la vida más que el alimento, y el cuerpo más que
el vestido? \footnote{\textbf{6:25} Fil 4,6; 1Pe 5,7; Luc 12,22-31}
\bibleverse{26} Mirad las aves del cielo, que no siembran, ni cosechan,
ni recogen en graneros. Vuestro Padre celestial las alimenta. ¿No tienes
tú mucho más valor que ellas? \footnote{\textbf{6:26} Mat 10,29-31; Luc
  12,6-7}

\bibleverse{27} ``¿Quién de vosotros, estando ansioso, puede añadir un
momento\footnote{\textbf{6:27} literalmente, cúbito} a su vida?
\bibleverse{28} ¿Por qué os preocupáis por la ropa? Considerad los
lirios del campo, cómo crecen. No se afanan, ni hilan, \bibleverse{29}
pero os digo que ni siquiera Salomón, con toda su gloria, se vistió como
uno de ellos. \footnote{\textbf{6:29} 1Re 10,1} \bibleverse{30} Pero si
Dios viste así a la hierba del campo, que hoy existe y mañana es
arrojada al horno, ¿no os vestirá mucho más a vosotros, hombres de poca
fe?

\bibleverse{31} ``Por tanto, no os preocupéis diciendo: ``¿Qué
comeremos?'', ``¿Qué beberemos?'' o ``¿Con qué nos vestiremos?''
\bibleverse{32} Porque los gentiles buscan todas estas cosas; pues
vuestro Padre celestial sabe que necesitáis todas estas cosas.
\bibleverse{33} Pero buscad primero el Reino de Dios y su justicia, y
todas estas cosas se os darán también a vosotros. \footnote{\textbf{6:33}
  Rom 14,17; 1Re 3,13-14; Sal 37,4; Sal 37,25} \bibleverse{34} Por
tanto, no os preocupéis por el día de mañana, porque el día de mañana se
preocupará por sí mismo. El mal de cada día es suficiente. \footnote{\textbf{6:34}
  Éxod 16,19}

\hypertarget{no-juzguuxe9is}{%
\subsection{No juzguéis}\label{no-juzguuxe9is}}

\hypertarget{section-6}{%
\section{7}\label{section-6}}

\bibleverse{1} ``No juzguéis, para que no seáis juzgados. \footnote{\textbf{7:1}
  Rom 2,1; 1Cor 4,5} \bibleverse{2} Porque con el juicio que juzgues,
serás juzgado; y con la medida que midas, te será medido. \footnote{\textbf{7:2}
  Mar 4,24} \bibleverse{3} ¿Por qué ves la paja que está en el ojo de tu
hermano, pero no consideras la viga que está en tu propio ojo?
\bibleverse{4} ¿O cómo vas a decir a tu hermano: ``Déjame sacar la paja
de tu ojo'', y he aquí que la viga está en tu propio ojo? \bibleverse{5}
¡Hipócrita! Saca primero la viga de tu propio ojo, y entonces podrás ver
con claridad para sacar la paja del ojo de tu hermano.

\bibleverse{6} ``No des lo santo a los perros, ni eches tus perlas a los
cerdos, no sea que las pisoteen y se vuelvan y te hagan pedazos.
\footnote{\textbf{7:6} Mat 10,11; Luc 23,9}

\hypertarget{pedid-buscad-llaman}{%
\subsection{Pedid, buscad, llaman}\label{pedid-buscad-llaman}}

\bibleverse{7} ``Pedid y se os dará. Buscad y encontraréis. Llamad, y se
os abrirá. \footnote{\textbf{7:7} Jer 29,13-14; Mar 11,24; Luc 11,5-13;
  Juan 14,13} \bibleverse{8} Porque todo el que pide recibe. El que
busca, encuentra. Al que llama se le abrirá. \bibleverse{9} ¿O quién hay
entre vosotros que, si su hijo le pide pan, le dé una piedra?
\bibleverse{10} O si le pide un pescado, ¿quién le dará una serpiente?
\bibleverse{11} Pues si vosotros, siendo malos, sabéis dar buenas
dádivas a vuestros hijos, ¡cuánto más vuestro Padre, que está en los
cielos, dará cosas buenas a los que le pidan! \footnote{\textbf{7:11}
  Sant 1,17}

\hypertarget{la-regla-de-oro-de-la-caridad}{%
\subsection{La regla de oro de la
caridad}\label{la-regla-de-oro-de-la-caridad}}

\bibleverse{12} Por tanto, todo lo que queráis que os hagan los hombres,
también se lo haréis vosotros a ellos; porque esto es la ley y los
profetas. \footnote{\textbf{7:12} Mat 22,36-40; Rom 13,8-10; Gal 5,14}

\bibleverse{13} ``Entrad por la puerta estrecha; porque ancha es la
puerta y ancho el camino que lleva a la perdición, y son muchos los que
entran por ella. \footnote{\textbf{7:13} Luc 13,24} \bibleverse{14}
¡Qué\footnote{\textbf{7:14} TR dice ``Porque'' en lugar de ``Como''}
estrecha es la puerta y qué estrecho el camino que lleva a la vida! Son
pocos los que la encuentran. \footnote{\textbf{7:14} Mat 19,24; Hech
  14,22}

\hypertarget{guardaos-de-los-falsos-profetas}{%
\subsection{Guardaos de los falsos
profetas}\label{guardaos-de-los-falsos-profetas}}

\bibleverse{15} ``Guardaos de los falsos profetas, que vienen a vosotros
con piel de oveja, pero por dentro son lobos rapaces. \footnote{\textbf{7:15}
  Mat 24,4-5; Mat 24,24; 2Cor 11,13-15} \bibleverse{16} Por sus frutos
los conoceréis. ¿Acaso recogéis uvas de los espinos o higos de los
cardos? \footnote{\textbf{7:16} Gal 5,19-22; Sant 3,12} \bibleverse{17}
Así, todo árbol bueno produce frutos buenos, pero el árbol corrompido
produce frutos malos. \footnote{\textbf{7:17} Mat 12,33} \bibleverse{18}
Un árbol bueno no puede producir frutos malos, ni un árbol corrompido
puede producir frutos buenos. \bibleverse{19} Todo árbol que no da
buenos frutos es cortado y arrojado al fuego. \footnote{\textbf{7:19}
  Mat 3,10; Juan 15,2; Juan 15,6} \bibleverse{20} Por tanto, por sus
frutos los conoceréis.

\hypertarget{sea-el-hacedor-de-la-palabra-no-solo-un-oyente}{%
\subsection{Sea el hacedor de la palabra, no solo un
oyente}\label{sea-el-hacedor-de-la-palabra-no-solo-un-oyente}}

\bibleverse{21} ``No todo el que me dice: `Señor, Señor', entrará en el
Reino de los Cielos, sino el que hace la voluntad de mi Padre que está
en los cielos. \footnote{\textbf{7:21} Rom 2,13; Sant 1,22}
\bibleverse{22} Muchos me dirán en aquel día: ``Señor, Señor, ¿no
profetizamos en tu nombre, en tu nombre expulsamos demonios y en tu
nombre hicimos muchas obras poderosas? \footnote{\textbf{7:22} Jer
  27,13; Luc 13,25-27} \bibleverse{23} Entonces diré: Nunca os conocí.
Apartaos de mí, obradores de iniquidad'. \footnote{\textbf{7:23} Mat
  25,12; 2Tim 2,19}

\bibleverse{24} ``Por tanto, todo el que oiga estas palabras mías y las
ponga en práctica, lo compararé a un hombre prudente que construyó su
casa sobre una roca. \bibleverse{25} Cayó la lluvia, vinieron las
inundaciones y los vientos soplaron y golpearon esa casa; y no se cayó,
porque estaba fundada sobre la roca. \bibleverse{26} Todo el que oiga
estas palabras mías y no las ponga en práctica será como un insensato
que construyó su casa sobre la arena. \bibleverse{27} Cayó la lluvia,
vinieron las inundaciones y los vientos soplaron y golpearon esa casa; y
se cayó, y su caída fue grande.'' \footnote{\textbf{7:27} Ezeq 13,10-11}

\bibleverse{28} Cuando Jesús terminó de decir estas cosas, las
multitudes se asombraron de su enseñanza, \footnote{\textbf{7:28} Hech
  2,12} \bibleverse{29} porque les enseñaba con autoridad, y no como los
escribas. \footnote{\textbf{7:29} Juan 7,16; Juan 7,46}

\hypertarget{sanando-a-un-leproso}{%
\subsection{Sanando a un leproso}\label{sanando-a-un-leproso}}

\hypertarget{section-7}{%
\section{8}\label{section-7}}

\bibleverse{1} Cuando bajó del monte, le siguieron grandes multitudes.
\bibleverse{2} He aquí que un leproso se le acercó y le adoró diciendo:
``Señor, si quieres, puedes limpiarme''.

\bibleverse{3} Jesús extendió la mano y lo tocó, diciendo: ``Quiero.
Queda limpio''. Al instante su lepra quedó limpia. \bibleverse{4} Jesús
le dijo: ``Mira que no se lo digas a nadie; pero ve, muéstrate al
sacerdote y ofrece la ofrenda que mandó Moisés, como testimonio para
ellos.'' \footnote{\textbf{8:4} Mar 8,30; Lev 14,2-32}

\hypertarget{sanaciuxf3n-del-siervo-del-centuriuxf3n-de-capernaum}{%
\subsection{Sanación del siervo del centurión de
Capernaum}\label{sanaciuxf3n-del-siervo-del-centuriuxf3n-de-capernaum}}

\bibleverse{5} Cuando llegó a Capernaúm, se le acercó un centurión
pidiéndole ayuda, \bibleverse{6} diciendo: ``Señor, mi siervo yace en la
casa paralizado, gravemente atormentado.''

\bibleverse{7} Jesús le dijo: ``Iré y lo curaré''.

\bibleverse{8} El centurión respondió: ``Señor, no soy digno de que
entres bajo mi techo. Basta con que digas la palabra, y mi siervo
quedará curado. \bibleverse{9} Porque también yo soy un hombre con
autoridad, que tiene soldados a mi cargo. Digo a éste: ``Ve'', y va; y
digo a otro: ``Ven'', y viene; y digo a mi siervo: ``Haz esto'', y lo
hace.''

\bibleverse{10} Al oírlo, Jesús se maravilló y dijo a los que le
seguían: ``Os aseguro que no he encontrado una fe tan grande, ni
siquiera en Israel. \footnote{\textbf{8:10} Mar 6,6; Luc 18,8}
\bibleverse{11} Os digo que vendrán muchos del este y del oeste y se
sentarán con Abraham, Isaac y Jacob en el Reino de los Cielos,
\footnote{\textbf{8:11} Luc 13,28-29} \bibleverse{12} pero los hijos del
Reino serán arrojados a las tinieblas exteriores. Allí será el llanto y
el crujir de dientes''. \bibleverse{13} Jesús dijo al centurión: ``Vete.
Que se haga contigo lo que has creído''. Su siervo quedó sanado en
aquella hora. \footnote{\textbf{8:13} Mat 9,29; Mat 15,28}

\hypertarget{sanaciuxf3n-de-la-suegra-de-pedro-y-de-muchos-otros-enfermos-en-cafarnauxfam}{%
\subsection{Sanación de la suegra de Pedro y de muchos otros enfermos en
Cafarnaúm}\label{sanaciuxf3n-de-la-suegra-de-pedro-y-de-muchos-otros-enfermos-en-cafarnauxfam}}

\bibleverse{14} Cuando Jesús entró en la casa de Pedro, vio a la madre
de éste, enferma de fiebre. \footnote{\textbf{8:14} 1Cor 9,5}
\bibleverse{15} Le tocó la mano, y la fiebre la dejó. Ella se levantó y
le sirvió. \footnote{\textbf{8:15} TR lee ``ellos'' en lugar de ``él''}
\bibleverse{16} Cuando llegó la noche, le trajeron muchos endemoniados.
Él expulsó a los espíritus con una palabra, y sanó a todos los enfermos,
\bibleverse{17} para que se cumpliera lo que se dijo por medio del
profeta Isaías, que dijo: ``Tomó nuestras dolencias y cargó con nuestras
enfermedades.'' \footnote{\textbf{8:17} Isaías 53:4}

\hypertarget{jesuxfas-escapa-a-la-otra-orilla-del-lago-proverbios-sobre-seguir-a-jesuxfas}{%
\subsection{Jesús escapa a la otra orilla del lago; Proverbios sobre
seguir a
Jesús}\label{jesuxfas-escapa-a-la-otra-orilla-del-lago-proverbios-sobre-seguir-a-jesuxfas}}

\bibleverse{18} Al ver que lo rodeaba una gran multitud, Jesús dio la
orden de marcharse al otro lado.

\bibleverse{19} Se acercó un escriba y le dijo: ``Maestro, te seguiré a
donde vayas''.

\bibleverse{20} Jesús le dijo: ``Las zorras tienen madrigueras y las
aves del cielo nidos, pero el Hijo del Hombre no tiene dónde reclinar la
cabeza''. \footnote{\textbf{8:20} 2Cor 8,9}

\bibleverse{21} Otro de sus discípulos le dijo: ``Señor, permíteme ir
primero a enterrar a mi padre''. \footnote{\textbf{8:21} Mat 10,37}

\bibleverse{22} Pero Jesús le dijo: ``Sígueme y deja que los muertos
entierren a sus propios muertos''.

\hypertarget{jesuxfas-apacigua-la-tormenta-del-mar}{%
\subsection{Jesús apacigua la tormenta del
mar}\label{jesuxfas-apacigua-la-tormenta-del-mar}}

\bibleverse{23} Cuando subió a una barca, sus discípulos le siguieron.
\bibleverse{24} Se levantó una violenta tormenta en el mar, tanto que la
barca quedó cubierta por las olas; pero él dormía. \bibleverse{25} Los
discípulos se acercaron a él y le despertaron diciendo: ``¡Sálvanos,
Señor! Nos estamos muriendo''.

\bibleverse{26} Les dijo: ``¿Por qué tenéis miedo, hombres de poca
fe?''. Entonces se levantó, reprendió al viento y al mar, y se produjo
una gran calma. \footnote{\textbf{8:26} Sal 89,9; Hech 27,22; Hech 27,34}

\bibleverse{27} Los hombres se maravillaron diciendo: ``¿Qué clase de
hombre es éste, que hasta el viento y el mar le obedecen?''

\hypertarget{curaciuxf3n-de-dos-poseuxeddos-en-la-tierra-de-los-gadarenos}{%
\subsection{Curación de dos poseídos en la tierra de los
gadarenos}\label{curaciuxf3n-de-dos-poseuxeddos-en-la-tierra-de-los-gadarenos}}

\bibleverse{28} Cuando llegó a la otra orilla, al país de los
gergesenos,\footnote{\textbf{8:28} NU lee ``gadarenos''} le salieron al
encuentro dos endemoniados que salían de los sepulcros, con gran
ferocidad, de modo que nadie podía pasar por allí. \footnote{\textbf{8:28}
  Luc 4,41; 2Pe 2,4; Sant 2,19} \bibleverse{29} Y gritaban diciendo:
``¿Qué tenemos que ver contigo, Jesús, Hijo de Dios? ¿Has venido a
atormentarnos antes de tiempo?'' \bibleverse{30} Había una piara de
muchos cerdos que se alimentaba lejos de ellos. \bibleverse{31} Los
demonios le rogaron, diciendo: ``Si nos echas, permítenos ir a la piara
de cerdos''.

\bibleverse{32} Les dijo: ``¡Vayan!'' Salieron y entraron en la piara de
cerdos; y he aquí que toda la piara de cerdos se precipitó por el
acantilado al mar y murió en el agua. \bibleverse{33} Los que les daban
de comer huyeron y se fueron a la ciudad y contaron todo, incluso lo que
les había pasado a los endemoniados. \bibleverse{34} Toda la ciudad
salió a recibir a Jesús. Cuando lo vieron, le rogaron que se fuera de
sus fronteras.

\hypertarget{curaciuxf3n-de-un-paraluxedtico-en-capernaum-jesuxfas-perdona-los-pecados}{%
\subsection{Curación de un paralítico en Capernaum; Jesús perdona los
pecados}\label{curaciuxf3n-de-un-paraluxedtico-en-capernaum-jesuxfas-perdona-los-pecados}}

\hypertarget{section-8}{%
\section{9}\label{section-8}}

\bibleverse{1} Entró en una barca, cruzó y llegó a su ciudad.
\footnote{\textbf{9:1} Mat 4,13} \bibleverse{2} Le trajeron un
paralítico que estaba tendido en una cama. Jesús, al ver su fe, dijo al
paralítico: ``¡Hijo, anímate! Tus pecados te son perdonados''.
\footnote{\textbf{9:2} Éxod 34,6-7; Sal 103,3}

\bibleverse{3} He aquí que algunos de los escribas se decían: ``Este
hombre blasfema''. \footnote{\textbf{9:3} Mat 26,65}

\bibleverse{4} Jesús, conociendo sus pensamientos, les dijo: ``¿Por qué
pensáis mal en vuestros corazones? \footnote{\textbf{9:4} Juan 2,25}
\bibleverse{5} Porque, ¿qué es más fácil, decir: ``Tus pecados quedan
perdonados'', o decir: ``Levántate y anda''? \bibleverse{6} Pero para
que sepáis que el Hijo del Hombre tiene autoridad en la tierra para
perdonar los pecados, le dijo al paralítico: ``Levántate, toma tu
camilla y vete a tu casa''. \footnote{\textbf{9:6} Juan 17,2}

\bibleverse{7} Se levantó y se fue a su casa. \bibleverse{8} Pero cuando
las multitudes lo vieron, se maravillaron y glorificaron a Dios, que
había dado tal autoridad a los hombres.

\hypertarget{llamada-del-recaudador-de-impuestos-mateo-jesuxfas-como-compauxf1ero-de-mesa-para-recaudadores-de-impuestos-y-pecadores}{%
\subsection{Llamada del recaudador de impuestos Mateo; Jesús como
compañero de mesa para recaudadores de impuestos y
pecadores}\label{llamada-del-recaudador-de-impuestos-mateo-jesuxfas-como-compauxf1ero-de-mesa-para-recaudadores-de-impuestos-y-pecadores}}

\bibleverse{9} Al pasar por allí, Jesús vio a un hombre llamado Mateo,
sentado en la oficina de recaudación de impuestos. Le dijo: ``Sígueme''.
Él se levantó y le siguió. \footnote{\textbf{9:9} Mat 10,3}
\bibleverse{10} Mientras estaba sentado en la casa, he aquí que muchos
recaudadores de impuestos y pecadores vinieron y se sentaron con Jesús y
sus discípulos. \bibleverse{11} Al ver esto, los fariseos dijeron a sus
discípulos: ``¿Por qué come vuestro maestro con recaudadores de
impuestos y pecadores?''

\bibleverse{12} Al oírlo, Jesús les dijo: ``Los sanos no tienen
necesidad de médico, pero los enfermos sí. \footnote{\textbf{9:12} Ezeq
  34,16} \bibleverse{13} Pero ustedes vayan y aprendan lo que significa:
``Quiero misericordia y no sacrificios,''\footnote{\textbf{9:13} Oseas
  6:6} porque no he venido a llamar a los justos, sino a los pecadores
al arrepentimiento.'' \footnote{\textbf{9:13} NU omite ``al
  arrepentimiento''.} \footnote{\textbf{9:13} 1Sam 15,22; Mat 18,11}

\hypertarget{la-pregunta-del-ayuno-de-los-discuxedpulos-de-juan}{%
\subsection{La pregunta del ayuno de los discípulos de
Juan}\label{la-pregunta-del-ayuno-de-los-discuxedpulos-de-juan}}

\bibleverse{14} Entonces los discípulos de Juan se acercaron a él,
diciendo: ``¿Por qué nosotros y los fariseos ayunamos a menudo, pero tus
discípulos no ayunan?'' \footnote{\textbf{9:14} Luc 18,12}

\bibleverse{15} Jesús les dijo: ``¿Pueden los amigos del novio llorar
mientras el novio esté con ellos? Pero vendrán días en que el novio les
será quitado, y entonces ayunarán. \footnote{\textbf{9:15} Juan 3,29}
\bibleverse{16} Nadie pone un trozo de tela sin remendar en una prenda
vieja, porque el remiendo se desprende de la prenda y se hace un agujero
peor. \footnote{\textbf{9:16} Rom 7,6} \bibleverse{17} Tampoco se pone
vino nuevo en odres viejos, porque se reventarían los odres, se
derramaría el vino y se arruinarían los odres. No, ponen vino nuevo en
odres frescos, y ambos se conservan''.

\hypertarget{resucitar-a-la-hija-de-jairo-y-curar-a-la-mujer-asolada-por-la-sangre}{%
\subsection{Resucitar a la hija de Jairo y curar a la mujer asolada por
la
sangre}\label{resucitar-a-la-hija-de-jairo-y-curar-a-la-mujer-asolada-por-la-sangre}}

\bibleverse{18} Mientras les contaba estas cosas, se acercó un
gobernante y le adoró diciendo: ``Mi hija acaba de morir, pero ven y pon
tu mano sobre ella, y vivirá.''

\bibleverse{19} Jesús se levantó y le siguió, al igual que sus
discípulos. \bibleverse{20} He aquí que una mujer que tenía flujo de
sangre desde hacía doce años se acercó detrás de él y tocó los
flecos\footnote{\textbf{9:20} o, borla} de su manto; \bibleverse{21}
porque decía en su interior: ``Si toco su manto, quedaré sana.''
\footnote{\textbf{9:21} Mat 14,36}

\bibleverse{22} Pero Jesús, al volverse y verla, le dijo: ``¡Hija,
anímate! Tu fe te ha curado''. Y la mujer quedó sana desde aquella hora.

\bibleverse{23} Cuando Jesús entró en la casa del gobernante y vio a los
flautistas y a la multitud en ruidoso desorden, \bibleverse{24} les
dijo: ``Haced sitio, porque la muchacha no está muerta, sino dormida.''
Se burlaban de él. \footnote{\textbf{9:24} Juan 11,11; Juan 11,14; Juan
  11,25} \bibleverse{25} Pero cuando la multitud fue despedida, él
entró, la tomó de la mano y la muchacha se levantó. \bibleverse{26} La
noticia de esto se difundió por toda aquella tierra.

\hypertarget{curaciuxf3n-de-dos-ciegos-y-un-mudo-endemoniado-graduaciuxf3n}{%
\subsection{Curación de dos ciegos y un mudo endemoniado;
Graduación}\label{curaciuxf3n-de-dos-ciegos-y-un-mudo-endemoniado-graduaciuxf3n}}

\bibleverse{27} Al pasar Jesús de allí, le siguieron dos ciegos,
gritando y diciendo: ``¡Ten piedad de nosotros, hijo de David!''
\footnote{\textbf{9:27} Mat 20,29-34} \bibleverse{28} Cuando entró en la
casa, los ciegos se acercaron a él. Jesús les dijo: ``¿Creéis que soy
capaz de hacer esto?'' Le dijeron: ``Sí, Señor''. \footnote{\textbf{9:28}
  Hech 14,9}

\bibleverse{29} Entonces les tocó los ojos, diciendo: ``Conforme a
vuestra fe os sea hecho''. \footnote{\textbf{9:29} Mat 8,13}
\bibleverse{30} Entonces se les abrieron los ojos. Jesús les ordenó
estrictamente, diciendo: ``Mirad que nadie sepa esto''. \footnote{\textbf{9:30}
  Mat 8,4} \bibleverse{31} Pero ellos salieron y difundieron su fama en
toda aquella tierra.

\bibleverse{32} Cuando salieron, le trajeron a un mudo endemoniado.
\bibleverse{33} Cuando el demonio fue expulsado, el mudo habló. Las
multitudes se maravillaron, diciendo: ``¡Nunca se ha visto nada
semejante en Israel!''

\bibleverse{34} Pero los fariseos decían: ``Por el príncipe de los
demonios, expulsa a los demonios''. \footnote{\textbf{9:34} Mat 12,24-32}

\bibleverse{35} Jesús recorría todas las ciudades y aldeas, enseñando en
sus sinagogas y predicando la Buena Nueva del Reino, y curando toda
enfermedad y toda dolencia en el pueblo.

\hypertarget{la-compasiuxf3n-de-jesuxfas-a-la-vista-de-la-gente-la-palabra-de-la-cosecha}{%
\subsection{La compasión de Jesús a la vista de la gente; la palabra de
la
cosecha}\label{la-compasiuxf3n-de-jesuxfas-a-la-vista-de-la-gente-la-palabra-de-la-cosecha}}

\bibleverse{36} Pero al ver las multitudes, se compadeció de ellas,
porque estaban acosadas\footnote{\textbf{9:36} TR lee ``cansado'' en
  lugar de ``acosado''} y dispersas, como ovejas sin pastor. \footnote{\textbf{9:36}
  Mar 6,34; Ezeq 34,5} \bibleverse{37} Entonces dijo a sus discípulos:
``La mies es abundante, pero los obreros son pocos. \footnote{\textbf{9:37}
  Luc 10,2} \bibleverse{38} Orad, pues, para que el Señor de la mies
envíe obreros a su mies''.

\hypertarget{llamadas-y-nombres-de-los-doce-discuxedpulos}{%
\subsection{Llamadas y nombres de los doce
discípulos}\label{llamadas-y-nombres-de-los-doce-discuxedpulos}}

\hypertarget{section-9}{%
\section{10}\label{section-9}}

\bibleverse{1} Llamó a sus doce discípulos y les dio autoridad sobre los
espíritus inmundos, para expulsarlos y para sanar toda enfermedad y toda
dolencia. \bibleverse{2} Los nombres de los doce apóstoles son estos El
primero, Simón, llamado Pedro; Andrés, su hermano; Santiago, hijo de
Zebedeo; Juan, su hermano; \footnote{\textbf{10:2} Mar 3,16-19; Luc
  6,14-16; Hech 1,13} \bibleverse{3} Felipe; Bartolomé; Tomás; Mateo, el
recaudador de impuestos; Santiago, hijo de Alfeo; Lebeo, que también se
llamaba \footnote{\textbf{10:3} NU omite ``Lebbaeus, que también se
  llamaba''} Tadeo; \bibleverse{4} Simón el Zelote; y Judas Iscariote,
que también lo traicionó.

\hypertarget{el-mensaje-enviado-a-los-doce-discuxedpulos}{%
\subsection{El mensaje enviado a los doce
discípulos}\label{el-mensaje-enviado-a-los-doce-discuxedpulos}}

\bibleverse{5} Jesús envió a estos doce y les ordenó: ``No vayan entre
los gentiles, ni entren en ninguna ciudad de los samaritanos.
\bibleverse{6} Id más bien a las ovejas perdidas de la casa de Israel.
\footnote{\textbf{10:6} Mat 15,24; Hech 13,46} \bibleverse{7} Mientras
vais, predicad diciendo: ``El Reino de los Cielos está cerca''
\footnote{\textbf{10:7} Mat 4,17; Luc 10,9} \bibleverse{8} Curad a los
enfermos, limpiad a los leprosos \footnote{\textbf{10:8} TR añade
  ``resucitar a los muertos''.} y expulsad a los demonios. Si habéis
recibido gratuitamente, dad gratuitamente. \footnote{\textbf{10:8} Mar
  16,17; Hech 20,33} \bibleverse{9} No llevéis oro, ni plata, ni latón
en vuestros cinturones. \bibleverse{10} No llevéis bolsa para vuestro
viaje, ni dos túnicas, ni sandalias, ni bastón; porque el trabajador es
digno de su alimento. \footnote{\textbf{10:10} Luc 10,4; 1Cor 9,14; 1Tim
  5,18; Núm 18,31} \bibleverse{11} En cualquier ciudad o aldea en que
entréis, averiguad quién es digno en ella, y quedaos allí hasta que
sigáis. \bibleverse{12} Cuando entres en la casa, salúdala. \footnote{\textbf{10:12}
  Luc 10,5-6} \bibleverse{13} Si la casa es digna, que tu paz llegue a
ella, pero si no es digna, que tu paz vuelva a ti. \bibleverse{14} El
que no te reciba ni escuche tus palabras, al salir de esa casa o de esa
ciudad, sacude el polvo de tus pies. \footnote{\textbf{10:14} Luc
  10,10-12; Hech 13,51} \bibleverse{15} De cierto os digo que será más
tolerable para la tierra de Sodoma y Gomorra en el día del juicio que
para esa ciudad. \footnote{\textbf{10:15} Gén 19,1-29}

\bibleverse{16} ``He aquí que os envío como ovejas en medio de lobos.
Por tanto, sed prudentes como serpientes y sencillos como palomas.
\footnote{\textbf{10:16} Rom 16,19; Efes 5,15}

\hypertarget{anuncio-de-los-sufrimientos-que-vendruxe1n-a-los-discuxedpulos}{%
\subsection{Anuncio de los sufrimientos que vendrán a los
discípulos}\label{anuncio-de-los-sufrimientos-que-vendruxe1n-a-los-discuxedpulos}}

\bibleverse{17} Pero tened cuidado con los hombres, porque os entregarán
a los concilios, y en sus sinagogas os azotarán. \footnote{\textbf{10:17}
  Hech 5,40; 2Cor 11,24} \bibleverse{18} Sí, y seréis llevados ante
gobernadores y reyes por causa de mí, para testimonio a ellos y a las
naciones. \footnote{\textbf{10:18} Hech 25,23; Hech 27,24}
\bibleverse{19} Pero cuando os entreguen, no os preocupéis por cómo o
qué vais a decir, porque se os dará en esa hora lo que vais a decir.
\footnote{\textbf{10:19} Luc 12,11-12; Hech 4,8} \bibleverse{20} Porque
no sois vosotros los que habláis, sino el Espíritu de vuestro Padre que
habla en vosotros. \footnote{\textbf{10:20} 1Cor 2,4}

\bibleverse{21} ``El hermano entregará al hermano a la muerte, y el
padre a su hijo. Los hijos se levantarán contra los padres y los harán
morir. \footnote{\textbf{10:21} Miq 7,6} \bibleverse{22} Seréis odiados
por todos los hombres por causa de mi nombre, pero el que aguante hasta
el final se salvará. \footnote{\textbf{10:22} Mat 24,9-13; 2Tim 2,12}
\bibleverse{23} Pero cuando os persigan en esta ciudad, huid a la
siguiente, porque de cierto os digo que no habréis pasado por las
ciudades de Israel hasta que venga el Hijo del Hombre. \footnote{\textbf{10:23}
  Mat 16,28; Hech 8,1}

\bibleverse{24} ``El discípulo no está por encima de su maestro, ni el
siervo por encima de su señor. \footnote{\textbf{10:24} Luc 6,40; Juan
  13,16; Juan 15,20} \bibleverse{25} Al discípulo le basta con ser como
su maestro, y al siervo como su señor. Si han llamado Beelzebul al dueño
de la casa,\footnote{\textbf{10:25} Literalmente, el Señor de las
  Moscas, o el diablo} ¡cuánto más a los de su casa! \footnote{\textbf{10:25}
  Mat 12,24}

\hypertarget{aliento-para-perseverar-fielmente-y-consuelo-en-tiempos-de-tribulaciuxf3n}{%
\subsection{Aliento para perseverar fielmente y consuelo en tiempos de
tribulación}\label{aliento-para-perseverar-fielmente-y-consuelo-en-tiempos-de-tribulaciuxf3n}}

\bibleverse{26} Por lo tanto, no tengan miedo de ellos, porque no hay
nada encubierto que no se revele, ni oculto que no se sepa. \footnote{\textbf{10:26}
  Mar 4,22; Luc 8,17} \bibleverse{27} Lo que os diga en la oscuridad,
habladlo en la luz; y lo que oigáis susurrar al oído, proclamadlo en los
tejados. \bibleverse{28} No temáis a los que matan el cuerpo, pero no
pueden matar el alma. Temed más bien a aquel que es capaz de destruir
tanto el alma como el cuerpo en la Gehena. \footnote{\textbf{10:28} o,
  el infierno.} \footnote{\textbf{10:28} Heb 10,31; Sant 4,12}

\bibleverse{29} ``¿No se venden dos gorriones por una moneda de
asarion?\footnote{\textbf{10:29} es una pequeña moneda que vale la
  décima parte de un dracma o la decimosexta parte de un denario. Un
  asarion es aproximadamente el salario de una media hora de trabajo
  agrícola.} Ni uno solo de ellos cae al suelo si no es por la voluntad
de tu Padre. \bibleverse{30} Pero los cabellos de tu cabeza están todos
contados. \footnote{\textbf{10:30} Hech 27,34} \bibleverse{31} Por eso,
no tengáis miedo. Vosotros tenéis más valor que muchos gorriones.
\footnote{\textbf{10:31} Mat 6,26} \bibleverse{32} Por eso, todo el que
me confiese ante los hombres, yo también lo confesaré ante mi Padre que
está en los cielos. \footnote{\textbf{10:32} Apoc 3,5} \bibleverse{33}
Pero el que me niegue ante los hombres, yo también lo negaré ante mi
Padre que está en los cielos. \footnote{\textbf{10:33} Mar 8,38; Luc
  9,26; 2Tim 2,12}

\hypertarget{paz-y-espada-puxe9rdida-y-ganancia}{%
\subsection{Paz y espada, pérdida y
ganancia}\label{paz-y-espada-puxe9rdida-y-ganancia}}

\bibleverse{34} ``No penséis que he venido a traer la paz a la tierra.
No he venido a traer la paz, sino la espada. \footnote{\textbf{10:34}
  Luc 12,51-53} \bibleverse{35} Porque he venido a enfrentar al hombre
con su padre, a la hija con su madre y a la nuera con su suegra.
\bibleverse{36} Los enemigos del hombre serán los de su propia casa.
\footnote{\textbf{10:36} Miqueas 7:6} \footnote{\textbf{10:36} Miq 7,6}
\bibleverse{37} El que ama al padre o a la madre más que a mí, no es
digno de mí; y el que ama al hijo o a la hija más que a mí, no es digno
de mí. \footnote{\textbf{10:37} Deut 13,6-11; Deut 33,9; Luc 14,26-27}
\bibleverse{38} El que no toma su cruz y sigue en pos de mí, no es digno
de mí. \footnote{\textbf{10:38} Mat 16,24-25} \bibleverse{39} El que
busca su vida, la perderá; y el que pierde su vida por mí, la
encontrará. \footnote{\textbf{10:39} Luc 9,24; Juan 12,25}

\hypertarget{promesas-de-ayuda-fraternal}{%
\subsection{Promesas de ayuda
fraternal}\label{promesas-de-ayuda-fraternal}}

\bibleverse{40} ``El que os recibe a vosotros me recibe a mí, y el que
me recibe a mí recibe al que me ha enviado. \footnote{\textbf{10:40} Luc
  9,48; Juan 13,20; Gal 4,14} \bibleverse{41} El que recibe a un profeta
en nombre de un profeta, recibirá la recompensa de un profeta. El que
recibe a un justo en nombre de un justo, recibirá la recompensa de un
justo. \footnote{\textbf{10:41} 1Re 17,9-24} \bibleverse{42} El que dé
de beber a uno de estos pequeños un vaso de agua fría en nombre de un
discípulo, de cierto os digo que no perderá su recompensa.'' \footnote{\textbf{10:42}
  Mat 25,40; Mar 9,41}

\hypertarget{section-10}{%
\section{11}\label{section-10}}

\bibleverse{1} Cuando Jesús terminó de dirigir a sus doce discípulos,
partió de allí para enseñar y predicar en sus ciudades.

\hypertarget{embajada-de-juan-el-bautista-desde-la-prisiuxf3n-la-respuesta-y-el-testimonio-de-jesuxfas-sobre-juan}{%
\subsection{Embajada de Juan el Bautista desde la prisión; La respuesta
y el testimonio de Jesús sobre
Juan}\label{embajada-de-juan-el-bautista-desde-la-prisiuxf3n-la-respuesta-y-el-testimonio-de-jesuxfas-sobre-juan}}

\bibleverse{2} Cuando Juan oyó en la cárcel las obras de Cristo, envió a
dos de sus discípulos \footnote{\textbf{11:2} Mat 14,3} \bibleverse{3} y
le dijeron: ``¿Eres tú el que viene, o tenemos que buscar a otro?''
\footnote{\textbf{11:3} Mat 3,11; Mal 3,1}

\bibleverse{4} Jesús les respondió: ``Id y contad a Juan lo que oís y
veis: \bibleverse{5} los ciegos ven, los cojos andan, los leprosos
quedan limpios, los sordos oyen,\footnote{\textbf{11:5} Isaías 35:5} los
muertos resucitan y a los pobres se les anuncia la buena nueva.
\footnote{\textbf{11:5} Isaías 61:1-4} \footnote{\textbf{11:5} Is
  35,5-6; Is 61,1} \bibleverse{6} Dichoso el que no encuentra en mí
ocasión de tropezar''. \footnote{\textbf{11:6} Mat 13,57; Mat 26,31}

\bibleverse{7} Mientras éstos se iban, Jesús comenzó a decir a las
multitudes acerca de Juan: ``¿Qué salisteis a ver al desierto? ¿Una caña
sacudida por el viento? \footnote{\textbf{11:7} Mat 3,1; Mat 3,5}
\bibleverse{8} ¿Y qué salisteis a ver? ¿A un hombre con ropa elegante?
He aquí que los que llevan ropa elegante están en las casas de los
reyes. \bibleverse{9} Pero, ¿por qué salisteis? ¿Para ver a un profeta?
Sí, os digo, y mucho más que un profeta. \footnote{\textbf{11:9} Luc
  1,76; Luc 20,6} \bibleverse{10} Porque éste es aquel de quien está
escrito: `He aquí que yo envío mi mensajero delante de ti, el cual
preparará tu camino delante de ti'. \footnote{\textbf{11:10} Malaquías
  3:1} \bibleverse{11} De cierto os digo que entre los nacidos de mujer
no se ha levantado nadie más grande que Juan el Bautista; pero el más
pequeño en el Reino de los Cielos es más grande que él. \footnote{\textbf{11:11}
  Mat 13,17} \bibleverse{12} Desde los días de Juan el Bautista hasta
ahora, el Reino de los Cielos sufre violencia, y los violentos lo toman
por la fuerza. \footnote{\textbf{11:12} o, saquearlo.} \footnote{\textbf{11:12}
  Luc 16,16} \bibleverse{13} Porque todos los profetas y la ley
profetizaron hasta Juan. \bibleverse{14} Si estáis dispuestos a
recibirlo, éste es Elías, que ha de venir. \footnote{\textbf{11:14} Mal
  4,5; Mat 17,10-13} \bibleverse{15} El que tenga oídos para oír, que
oiga.

\bibleverse{16} ``¿Pero con qué compararé a esta generación? Es como los
niños sentados en las plazas, que llaman a sus compañeros
\bibleverse{17} y dicen: `Tocamos la flauta por ti, y no bailaste.
Nosotros nos lamentamos por ti, y tú no te lamentaste'. \footnote{\textbf{11:17}
  Prov 29,9; Juan 5,35} \bibleverse{18} Porque Juan no vino ni a comer
ni a beber, y dicen: `Tiene un demonio.' \footnote{\textbf{11:18} Mat
  3,4; Juan 10,20} \bibleverse{19} El Hijo del Hombre vino comiendo y
bebiendo, y dicen: `He aquí un glotón y un borracho, amigo de
recaudadores y pecadores.' Pero la sabiduría se justifica por sus
hijos.''\footnote{\textbf{11:19} NU lee ``acciones'' en lugar de
  ``niños''} \footnote{\textbf{11:19} Mat 9,10-15; Juan 2,2; 1Cor
  1,24-30}

\hypertarget{lamento-de-jesuxfas-por-las-ciudades-galileas-impenitentes}{%
\subsection{Lamento de Jesús por las ciudades galileas
impenitentes}\label{lamento-de-jesuxfas-por-las-ciudades-galileas-impenitentes}}

\bibleverse{20} Entonces comenzó a denunciar a las ciudades en las que
se habían realizado la mayoría de sus obras poderosas, porque no se
arrepentían. \bibleverse{21} ``¡Ay de ti, Corazin! ¡Ay de ti, Betsaida!
Porque si en Tiro y en Sidón se hubieran hecho las obras poderosas que
se hicieron en ustedes, hace tiempo que se habrían arrepentido en saco y
ceniza. \footnote{\textbf{11:21} Jon 3,6} \bibleverse{22} Pero os digo
que el día del juicio será más tolerable para Tiro y Sidón que para
vosotros. \bibleverse{23} Tú, Capernaúm, que estás exaltada hasta el
cielo, descenderás al Hades. \footnote{\textbf{11:23} o el infierno}
Porque si en Sodoma se hubieran hecho las obras poderosas que se
hicieron en ti, habría permanecido hasta hoy. \footnote{\textbf{11:23}
  Mat 4,13; Mat 8,5; Mat 9,1; Is 14,13; Is 14,15} \bibleverse{24} Pero
os digo que será más tolerable para la tierra de Sodoma en el día del
juicio, que para vosotros.'' \footnote{\textbf{11:24} Mat 10,15}

\hypertarget{el-juxfabilo-y-la-alabanza-de-jesuxfas-por-el-padre}{%
\subsection{El júbilo y la alabanza de Jesús por el
Padre}\label{el-juxfabilo-y-la-alabanza-de-jesuxfas-por-el-padre}}

\bibleverse{25} En aquel momento, Jesús respondió: ``Te doy gracias,
Padre, Señor del cielo y de la tierra, porque has ocultado estas cosas a
los sabios y entendidos, y las has revelado a los niños. \footnote{\textbf{11:25}
  1Cor 1,19-29; Is 29,14; Luc 10,21-22; Juan 17,25} \bibleverse{26} Sí,
Padre, porque así fue agradable a tus ojos. \bibleverse{27} Todas las
cosas me han sido entregadas por mi Padre. Nadie conoce al Hijo, sino el
Padre; ni nadie conoce al Padre, sino el Hijo y aquel a quien el Hijo
quiera revelarlo. \footnote{\textbf{11:27} Mat 28,18; Juan 3,35; Juan
  17,2; Fil 2,9}

\hypertarget{el-llamado-del-salvador-a-los-cansados-y-agobiados}{%
\subsection{El llamado del Salvador a los cansados
\hspace{0pt}\hspace{0pt}y
agobiados}\label{el-llamado-del-salvador-a-los-cansados-y-agobiados}}

\bibleverse{28} ``Venid a mí todos los que estáis fatigados y agobiados,
y yo os haré descansar. \footnote{\textbf{11:28} Mat 12,20; Mat 23,4;
  Jer 31,25} \bibleverse{29} Llevad mi yugo y aprended de mí, que soy
manso y humilde de corazón; y encontraréis descanso para vuestras almas.
\footnote{\textbf{11:29} Is 28,12; Jer 6,16; Zac 9,9} \bibleverse{30}
Porque mi yugo es fácil, y mi carga es ligera''. \footnote{\textbf{11:30}
  Luc 11,46; 1Jn 5,3}

\hypertarget{los-ouxeddos-de-los-discuxedpulos-en-suxe1bado-la-primera-disputa-de-jesuxfas-con-los-fariseos-sobre-la-santificaciuxf3n-del-duxeda-de-reposo}{%
\subsection{Los oídos de los discípulos en sábado; La primera disputa de
Jesús con los fariseos sobre la santificación del día de
reposo}\label{los-ouxeddos-de-los-discuxedpulos-en-suxe1bado-la-primera-disputa-de-jesuxfas-con-los-fariseos-sobre-la-santificaciuxf3n-del-duxeda-de-reposo}}

\hypertarget{section-11}{%
\section{12}\label{section-11}}

\bibleverse{1} En aquel tiempo, Jesús pasó el día de reposo por los
campos de cereales. Sus discípulos tenían hambre y se pusieron a
arrancar espigas y a comer. \footnote{\textbf{12:1} Deut 23,25}
\bibleverse{2} Pero los fariseos, al verlo, le dijeron: ``He aquí que
tus discípulos hacen lo que no es lícito hacer en sábado.'' \footnote{\textbf{12:2}
  Éxod 20,10}

\bibleverse{3} Pero él les dijo: ``¿No habéis leído lo que hizo David
cuando tuvo hambre, y los que estaban con él \footnote{\textbf{12:3}
  1Sam 21,6} \bibleverse{4} cómo entró en la casa de Dios y comió el pan
de la feria, que no le era lícito comer a él ni a los que estaban con
él, sino sólo a los sacerdotes? \footnote{\textbf{12:4} 1 Samuel 21:3-6}
\footnote{\textbf{12:4} Lev 24,9} \bibleverse{5} ¿Acaso no habéis leído
en la ley que en el día de reposo los sacerdotes en el templo profanan
el sábado y son inocentes? \footnote{\textbf{12:5} Núm 28,9}
\bibleverse{6} Pero yo os digo que aquí hay uno más grande que el
templo. \bibleverse{7} Pero si hubierais sabido lo que significa esto:
``Quiero misericordia y no sacrificios,''\footnote{\textbf{12:7} Oseas
  6:6} no habríais condenado a los inocentes. \footnote{\textbf{12:7}
  Mat 9,13} \bibleverse{8} Porque el Hijo del Hombre es el Señor del
sábado''.

\hypertarget{sanaciuxf3n-del-hombre-con-el-brazo-paralizado-en-suxe1bado-el-segundo-argumento-sobre-la-observancia-del-suxe1bado}{%
\subsection{Sanación del hombre con el brazo paralizado en sábado; el
segundo argumento sobre la observancia del
sábado}\label{sanaciuxf3n-del-hombre-con-el-brazo-paralizado-en-suxe1bado-el-segundo-argumento-sobre-la-observancia-del-suxe1bado}}

\bibleverse{9} Salió de allí y entró en la sinagoga de ellos.
\bibleverse{10} Y he aquí que había un hombre con una mano seca. Le
preguntaron: ``¿Es lícito curar en día de reposo?'', para acusarle.

\bibleverse{11} Les dijo: ``¿Qué hombre hay entre vosotros que tenga una
sola oveja, y si ésta cae en un pozo en día de sábado, no se agarra a
ella y la saca? \footnote{\textbf{12:11} Luc 14,3-5} \bibleverse{12}
¡Cuánto más vale un hombre que una oveja! Por eso es lícito hacer el
bien en el día de reposo''. \bibleverse{13} Entonces le dijo al hombre:
``Extiende tu mano''. Él la extendió; y se la devolvió restaurada, igual
que la otra. \bibleverse{14} Pero los fariseos salieron y conspiraron
contra él para destruirlo. \footnote{\textbf{12:14} Juan 5,16}

\hypertarget{jesuxfas-evade-la-persecuciuxf3n-su-actividad-sanadora-piadosa}{%
\subsection{Jesús evade la persecución; su actividad sanadora
piadosa}\label{jesuxfas-evade-la-persecuciuxf3n-su-actividad-sanadora-piadosa}}

\bibleverse{15} Jesús, al darse cuenta, se retiró de allí. Le siguieron
grandes multitudes; y los curó a todos, \bibleverse{16} y les ordenó que
no le dieran a conocer, \footnote{\textbf{12:16} Mat 8,4}
\bibleverse{17} para que se cumpliera lo que se había dicho por medio
del profeta Isaías, que decía \bibleverse{18} ``He aquí a mi siervo que
he elegido, mi amado en quien mi alma se complace. Pondré mi Espíritu
sobre él. Anunciará la justicia a las naciones. \footnote{\textbf{12:18}
  Mat 3,17; Hech 3,13; Hech 3,26} \bibleverse{19} No se esforzará, ni
gritará, ni nadie escuchará su voz en las calles. \bibleverse{20} No
romperá una caña magullada. No apagará un lino humeante, hasta que lleve
la justicia a la victoria. \bibleverse{21} En su nombre esperarán las
naciones\footnote{\textbf{12:21} Isaías 42:1-4} ''.

\hypertarget{jesuxfas-se-defiende-de-la-blasfemia-de-los-fariseos-contra-beelzebul}{%
\subsection{Jesús se defiende de la blasfemia de los fariseos contra
Beelzebul}\label{jesuxfas-se-defiende-de-la-blasfemia-de-los-fariseos-contra-beelzebul}}

\bibleverse{22} Entonces le trajeron a uno poseído por un demonio, ciego
y mudo, y lo curó, de modo que el ciego y el mudo hablaba y veía.
\bibleverse{23} Todas las multitudes estaban asombradas y decían:
``¿Puede ser éste el hijo de David?'' \bibleverse{24} Pero cuando los
fariseos lo oyeron, dijeron: ``Este hombre no expulsa los demonios sino
por Beelzebul, el príncipe de los demonios.'' \footnote{\textbf{12:24}
  Mat 9,34}

\bibleverse{25} Conociendo sus pensamientos, Jesús les dijo: ``Todo
reino dividido contra sí mismo es desolado, y toda ciudad o casa
dividida contra sí misma no permanecerá. \bibleverse{26} Si Satanás
expulsa a Satanás, está dividido contra sí mismo. ¿Cómo, pues, se
mantendrá su reino? \bibleverse{27} Si yo, por medio de Beelzebul,
expulso los demonios, ¿por quién los expulsan vuestros hijos? Por tanto,
ellos serán vuestros jueces. \bibleverse{28} Pero si yo por el Espíritu
de Dios expulso los demonios, entonces el Reino de Dios ha llegado a
vosotros. \footnote{\textbf{12:28} 1Jn 3,8} \bibleverse{29} ¿Cómo puede
uno entrar en la casa del hombre fuerte y saquear sus bienes, si antes
no ata al hombre fuerte? Entonces saqueará su casa. \footnote{\textbf{12:29}
  Mat 4,1-11; Is 49,24}

\bibleverse{30} ``El que no está conmigo está contra mí, y el que no se
reúne conmigo, se dispersa. \footnote{\textbf{12:30} Mar 9,40}

\hypertarget{advertencia-de-la-blasfemia-del-espuxedritu-del-uxe1rbol-y-los-frutos}{%
\subsection{Advertencia de la blasfemia del espíritu; del árbol y los
frutos}\label{advertencia-de-la-blasfemia-del-espuxedritu-del-uxe1rbol-y-los-frutos}}

\bibleverse{31} Por eso os digo que todo pecado y toda blasfemia serán
perdonados a los hombres, pero la blasfemia contra el Espíritu no será
perdonada a los hombres. \footnote{\textbf{12:31} Heb 6,4-6; Heb 10,26;
  1Jn 5,16} \bibleverse{32} Al que hable una palabra contra el Hijo del
Hombre, se le perdonará; pero al que hable contra el Espíritu Santo, no
se le perdonará, ni en este tiempo ni en el venidero. \footnote{\textbf{12:32}
  1Tim 1,13}

\bibleverse{33} ``O haced el árbol bueno y su fruto bueno, o haced el
árbol corrompido y su fruto corrompido; porque por su fruto se conoce el
árbol. \footnote{\textbf{12:33} Mat 7,17} \bibleverse{34} Vástagos de
víboras, ¿cómo podéis, siendo malos, hablar cosas buenas? Porque de la
abundancia del corazón habla la boca. \footnote{\textbf{12:34} Mat 3,7}
\bibleverse{35} El hombre bueno de su buen tesoro\footnote{\textbf{12:35}
  TR añade ``del corazón''} saca cosas buenas, y el hombre malo de su
mal tesoro saca cosas malas. \bibleverse{36} Os digo que de toda palabra
ociosa que los hombres hablen, darán cuenta en el día del juicio.
\footnote{\textbf{12:36} Sant 3,6; Jds 1,15} \bibleverse{37} Porque por
sus palabras serán justificados, y por sus palabras serán condenados.''

\hypertarget{el-rechazo-de-jesuxfas-a-la-demanda-de-seuxf1ales-la-seuxf1al-de-jonuxe1s-la-paruxe1bola-de-la-recauxedda}{%
\subsection{El rechazo de Jesús a la demanda de señales; la señal de
Jonás; la parábola de la
recaída}\label{el-rechazo-de-jesuxfas-a-la-demanda-de-seuxf1ales-la-seuxf1al-de-jonuxe1s-la-paruxe1bola-de-la-recauxedda}}

\bibleverse{38} Entonces algunos de los escribas y fariseos
respondieron: ``Maestro, queremos ver una señal tuya''.

\bibleverse{39} Pero él les respondió: ``Una generación mala y adúltera
busca una señal, pero no se le dará otra señal que la del profeta Jonás.
\bibleverse{40} Porque como Jonás estuvo tres días y tres noches en el
vientre del gran pez, así el Hijo del Hombre estará tres días y tres
noches en el corazón de la tierra. \footnote{\textbf{12:40} Jon 1,17;
  Efes 4,9; 1Pe 3,19} \bibleverse{41} Los hombres de Nínive se
levantarán en el juicio con esta generación y la condenarán, porque se
arrepintieron ante la predicación de Jonás; y he aquí que hay alguien
más grande que Jonás. \footnote{\textbf{12:41} Jon 3,5} \bibleverse{42}
La Reina del Sur se levantará en el juicio con esta generación y la
condenará, porque vino desde los confines de la tierra para escuchar la
sabiduría de Salomón; y he aquí que hay alguien más grande que Salomón.
\footnote{\textbf{12:42} 1Re 10,1-10}

\bibleverse{43} ``Cuando un espíritu inmundo ha salido del hombre, pasa
por lugares sin agua buscando descanso, y no lo encuentra.
\bibleverse{44} Entonces dice: `Volveré a mi casa de donde salí'; y
cuando ha vuelto, la encuentra vacía, barrida y ordenada.
\bibleverse{45} Entonces va y toma consigo otros siete espíritus más
malos que él, y entran y habitan allí. El último estado de ese hombre
llega a ser peor que el primero. Así será también para esta generación
malvada''. \footnote{\textbf{12:45} 2Pe 2,20}

\hypertarget{los-verdaderos-parientes-de-jesuxfas}{%
\subsection{Los verdaderos parientes de
Jesús}\label{los-verdaderos-parientes-de-jesuxfas}}

\bibleverse{46} Mientras aún hablaba a las multitudes, he aquí que su
madre y sus hermanos estaban afuera, buscando hablar con él. \footnote{\textbf{12:46}
  Mat 13,55} \bibleverse{47} Uno le dijo: ``He aquí, tu madre y tus
hermanos están afuera, buscando hablar contigo''.

\bibleverse{48} Pero él respondió al que le hablaba: ``¿Quién es mi
madre? ¿Quiénes son mis hermanos?'' \footnote{\textbf{12:48} Mat 10,37;
  Luc 2,49} \bibleverse{49} Extendió la mano hacia sus discípulos y
dijo: ``¡Mira, mi madre y mis hermanos! \footnote{\textbf{12:49} Heb
  2,11} \bibleverse{50} Porque todo aquel que hace la voluntad de mi
Padre que está en los cielos, ése es mi hermano, mi hermana y mi
madre''. \footnote{\textbf{12:50} Rom 8,29}

\hypertarget{la-paruxe1bola-del-sembrador-y-el-campo-cuuxe1druple}{%
\subsection{la parábola del sembrador y el campo
cuádruple}\label{la-paruxe1bola-del-sembrador-y-el-campo-cuuxe1druple}}

\hypertarget{section-12}{%
\section{13}\label{section-12}}

\bibleverse{1} Aquel día, Jesús salió de casa y se sentó a la orilla del
mar. \bibleverse{2} Se reunió con él una gran multitud, de modo que
entró en una barca y se sentó; y toda la multitud se quedó de pie en la
playa. \bibleverse{3} Les hablaba de muchas cosas en parábolas,
diciendo: ``He aquí que un agricultor salió a sembrar. \bibleverse{4}
Mientras sembraba, algunas semillas cayeron al borde del camino, y
vinieron los pájaros y las devoraron. \bibleverse{5} Otras cayeron en un
terreno rocoso, donde no había mucha tierra, y enseguida brotaron,
porque no tenían profundidad de tierra. \bibleverse{6} Cuando salió el
sol, se quemaron. Como no tenían raíz, se marchitaron. \bibleverse{7}
Otras cayeron entre espinas. Los espinos crecieron y los ahogaron.
\bibleverse{8} Otras cayeron en buena tierra y dieron fruto: unas cien
veces más, otras sesenta y otras treinta. \bibleverse{9} El que tenga
oídos para oír, que oiga''.

\hypertarget{explicaciuxf3n-de-jesuxfas-de-la-razuxf3n-y-el-propuxf3sito-de-sus-paruxe1bolas}{%
\subsection{Explicación de Jesús de la razón y el propósito de sus
parábolas}\label{explicaciuxf3n-de-jesuxfas-de-la-razuxf3n-y-el-propuxf3sito-de-sus-paruxe1bolas}}

\bibleverse{10} Los discípulos se acercaron y le dijeron: ``¿Por qué les
hablas en parábolas?''

\bibleverse{11} Les respondió: ``A vosotros se os ha dado conocer los
misterios del Reino de los Cielos, pero a ellos no se les ha dado.
\footnote{\textbf{13:11} 1Cor 2,10} \bibleverse{12} Porque al que tiene,
se le dará y tendrá en abundancia; pero al que no tiene, se le quitará
hasta lo que tiene. \footnote{\textbf{13:12} Mat 25,28-29; Mar 4,25; Luc
  8,18} \bibleverse{13} Por eso les hablo en parábolas, porque viendo no
ven, y oyendo no oyen, ni entienden. \footnote{\textbf{13:13} Deut 29,4;
  Juan 16,25} \bibleverse{14} En ellos se cumple la profecía de Isaías,
que dice,`Oyendo escucharás', y no lo entenderá de ninguna manera;
Viendo verás, y no percibirá de ninguna manera; \bibleverse{15} porque
el corazón de este pueblo se ha vuelto insensible, sus oídos están
embotados, y han cerrado los ojos; para que no vean con los ojos, oigan
con sus oídos, entienden con el corazón, y se conviertan, y yo los
sane.'\footnote{\textbf{13:15} Isaías 6:9-10} \footnote{\textbf{13:15}
  Juan 9,39}

\bibleverse{16} ``Pero benditos sean vuestros ojos, porque ven; y
vuestros oídos, porque oyen. \footnote{\textbf{13:16} Luc 10,23-24}
\bibleverse{17} Porque ciertamente os digo que muchos profetas y justos
desearon ver lo que vosotros veis, y no lo vieron; y oír lo que oís, y
no lo oyeron. \footnote{\textbf{13:17} 1Pe 1,10}

\hypertarget{interpretaciuxf3n-de-la-paruxe1bola-del-sembrador}{%
\subsection{Interpretación de la parábola del
sembrador}\label{interpretaciuxf3n-de-la-paruxe1bola-del-sembrador}}

\bibleverse{18} ``Oíd, pues, la parábola del sembrador. \bibleverse{19}
Cuando alguien oye la palabra del Reino y no la entiende, viene el
maligno y arrebata lo que se ha sembrado en su corazón. Esto es lo que
se sembró junto al camino. \bibleverse{20} Lo que fue sembrado en los
pedregales, éste es el que oye la palabra y enseguida la recibe con
alegría; \bibleverse{21} pero no tiene raíz en sí mismo, sino que
aguanta un tiempo. Cuando surge la opresión o la persecución a causa de
la palabra, inmediatamente tropieza. \bibleverse{22} Lo que se sembró
entre espinos, éste es el que oye la palabra, pero los afanes de este
siglo y el engaño de las riquezas ahogan la palabra, y queda sin fruto.
\footnote{\textbf{13:22} Mat 6,19-34; 1Tim 6,9} \bibleverse{23} Lo que
se sembró en buena tierra, éste es el que oye la palabra y la entiende,
que ciertamente da fruto y produce, unos cien veces más, otros sesenta y
otros treinta.''

\hypertarget{la-paruxe1bola-de-la-cizauxf1a-entre-el-trigo}{%
\subsection{La parábola de la cizaña entre el
trigo}\label{la-paruxe1bola-de-la-cizauxf1a-entre-el-trigo}}

\bibleverse{24} Les expuso otra parábola, diciendo: ``El Reino de los
Cielos es semejante a un hombre que sembró buena semilla en su campo,
\bibleverse{25} pero mientras la gente dormía, vino su enemigo y sembró
\footnote{\textbf{13:25} La cizaña es una hierba (probablemente la
  cizaña barbuda o lolium temulentum) que se parece mucho al trigo hasta
  que madura, cuando la diferencia se hace muy evidente.} también cizaña
entre el trigo, y se fue. \bibleverse{26} Pero cuando la hoja brotó y
produjo grano, entonces apareció también la cizaña. \bibleverse{27} Se
acercaron los criados del dueño de casa y le dijeron: ``Señor, ¿no
sembraste buena semilla en tu campo? ¿De dónde ha salido esta cizaña?

\bibleverse{28} ``Les dijo: `Un enemigo ha hecho esto'. ``Los sirvientes
le preguntaron: `¿Quieres que vayamos a recogerlos?

\bibleverse{29} Pero él dijo: ``No, no sea que mientras recogéis la
cizaña, arranquéis con ella el trigo. \bibleverse{30} Dejad que ambos
crezcan juntos hasta la cosecha, y en el tiempo de la cosecha diré a los
segadores: ``Recoged primero la cizaña y atadla en manojos para
quemarla; pero recoged el trigo en mi granero''\,''. \footnote{\textbf{13:30}
  Mat 3,12; Mat 15,13; Apoc 14,15}

\hypertarget{las-dos-paruxe1bolas-de-la-semilla-de-mostaza-y-la-levadura}{%
\subsection{Las dos parábolas de la semilla de mostaza y la
levadura}\label{las-dos-paruxe1bolas-de-la-semilla-de-mostaza-y-la-levadura}}

\bibleverse{31} Les expuso otra parábola, diciendo: ``El Reino de los
Cielos es semejante a un grano de mostaza que un hombre tomó y sembró en
su campo, \bibleverse{32} que a la verdad es más pequeño que todas las
semillas. Pero cuando crece, es más grande que las hierbas y se
convierte en un árbol, de modo que las aves del cielo vienen y se alojan
en sus ramas.'' \footnote{\textbf{13:32} Ezeq 17,23}

\bibleverse{33} Les dijo otra parábola. ``El Reino de los Cielos es como
la levadura que una mujer tomó y escondió en tres medidas\footnote{\textbf{13:33}
  literalmente, tres sata. Tres sata son unos 39 litros o un poco más de
  una fanega} de harina, hasta que todo quedó leudado''. \footnote{\textbf{13:33}
  Luc 13,20-21}

\hypertarget{interpretaciuxf3n-de-la-paruxe1bola-de-la-cizauxf1a-del-trigo}{%
\subsection{Interpretación de la parábola de la cizaña del
trigo}\label{interpretaciuxf3n-de-la-paruxe1bola-de-la-cizauxf1a-del-trigo}}

\bibleverse{34} Jesús hablaba todas estas cosas en parábolas a las
multitudes; y sin parábola, no les hablaba, \footnote{\textbf{13:34} Mar
  4,33-34} \bibleverse{35} para que se cumpliera lo que se dijo por
medio del profeta, diciendo, ``Abriré mi boca en parábolas; Voy a decir
cosas ocultas desde la fundación del mundo\footnote{\textbf{13:35} Salmo
  78:2} ''.

\bibleverse{36} Entonces Jesús despidió a las multitudes y entró en la
casa. Sus discípulos se acercaron a él, diciendo: ``Explícanos la
parábola de la cizaña del campo''.

\bibleverse{37} Él les respondió: ``El que siembra la buena semilla es
el Hijo del Hombre, \bibleverse{38} el campo es el mundo, las buenas
semillas son los hijos del Reino y la cizaña son los hijos del maligno.
\footnote{\textbf{13:38} Juan 8,44; 1Cor 3,9} \bibleverse{39} El enemigo
que las sembró es el diablo. La cosecha es el fin de los tiempos, y los
segadores son los ángeles. \bibleverse{40} Así como la cizaña es
recogida y quemada en el fuego, así será al final de este siglo.
\bibleverse{41} El Hijo del Hombre enviará a sus ángeles, y recogerán de
su reino a todos los que causan tropiezo y a los que hacen iniquidad,
\footnote{\textbf{13:41} Mat 25,31-46} \bibleverse{42} y los echarán en
el horno de fuego. Allí será el llanto y el crujir de dientes.
\bibleverse{43} Entonces los justos brillarán como el sol en el Reino de
su Padre. El que tenga oídos para oír, que oiga. \footnote{\textbf{13:43}
  Dan 12,3}

\hypertarget{las-uxfaltimas-tres-paruxe1bolas-tesoro-en-el-campo-perla-preciosa-red-de-pesca-conclusiuxf3n-de-la-paruxe1bola}{%
\subsection{Las últimas tres parábolas (tesoro en el campo; perla
preciosa; red de pesca); Conclusión de la
parábola}\label{las-uxfaltimas-tres-paruxe1bolas-tesoro-en-el-campo-perla-preciosa-red-de-pesca-conclusiuxf3n-de-la-paruxe1bola}}

\bibleverse{44} ``Además, el Reino de los Cielos es como un tesoro
escondido en el campo, que un hombre encontró y escondió. En su alegría,
va y vende todo lo que tiene y compra ese campo. \footnote{\textbf{13:44}
  Mat 19,29; Luc 14,33; Fil 3,7}

\bibleverse{45} ``Además, el Reino de los Cielos se parece a un hombre
que es un mercader que busca perlas finas, \bibleverse{46} que habiendo
encontrado una perla de gran valor, fue y vendió todo lo que tenía y la
compró.

\bibleverse{47} ``Además, el Reino de los Cielos es como una red de
arrastre que se echó al mar y recogió peces de toda clase, \footnote{\textbf{13:47}
  Mat 22,9-10} \bibleverse{48} y que, cuando se llenó, los pescadores
sacaron a la playa. Se sentaron y recogieron lo bueno en recipientes,
pero lo malo lo tiraron. \bibleverse{49} Así será al fin del
mundo.\footnote{\textbf{13:49} El nombre de Pedro, Petros en griego, es
  la palabra para una roca o piedra específica.} Los ángeles vendrán y
separarán a los malos de entre los justos, \footnote{\textbf{13:49} Mat
  25,32} \bibleverse{50} y los echarán al horno de fuego. Allí será el
llanto y el crujir de dientes''. \bibleverse{51} Jesús les dijo:
``¿Habéis entendido todo esto?'' Le respondieron: ``Sí, Señor''.

\bibleverse{52} Les dijo: ``Por eso todo escriba que ha sido hecho
discípulo en el Reino de los Cielos es como un hombre que es dueño de
casa, que saca de su tesoro cosas nuevas y viejas.''

\hypertarget{rechazo-y-fracaso-de-jesuxfas-en-su-natal-nazaret}{%
\subsection{Rechazo y fracaso de Jesús en su natal
Nazaret}\label{rechazo-y-fracaso-de-jesuxfas-en-su-natal-nazaret}}

\bibleverse{53} Cuando Jesús terminó estas parábolas, se fue de allí.
\bibleverse{54} Al llegar a su tierra, les enseñaba en la sinagoga de
ellos, de modo que se asombraban y decían: ``¿De dónde ha sacado este
hombre esta sabiduría y estas maravillas? \bibleverse{55} ¿No es éste el
hijo del carpintero? ¿No se llama su madre María, y sus hermanos
Santiago, José, Simón y Judas? \bibleverse{56} ¿No están todas sus
hermanas con nosotros? ¿De dónde, pues, ha sacado este hombre todas
estas cosas?'' \footnote{\textbf{13:56} Juan 6,42; Juan 7,15; Juan 7,52}
\bibleverse{57} Se sintieron ofendidos por él. Pero Jesús les dijo: ``Un
profeta no carece de honor, sino en su propio país y en su propia
casa.'' \footnote{\textbf{13:57} Juan 4,44} \bibleverse{58} No hizo
muchas obras poderosas allí a causa de la incredulidad de ellos.

\hypertarget{jesuxfas-y-herodes-el-fin-de-juan-el-bautista}{%
\subsection{Jesús y Herodes; el fin de Juan el
Bautista}\label{jesuxfas-y-herodes-el-fin-de-juan-el-bautista}}

\hypertarget{section-13}{%
\section{14}\label{section-13}}

\bibleverse{1} En aquel tiempo, Herodes el tetrarca oyó la noticia sobre
Jesús, \bibleverse{2} y dijo a sus servidores: ``Este es Juan el
Bautista. Ha resucitado de entre los muertos. Por eso actúan en él estos
poderes''. \bibleverse{3} Porque Herodes había arrestado a Juan, lo
había atado y lo había encarcelado por causa de Herodías, la mujer de su
hermano Felipe. \footnote{\textbf{14:3} Mat 11,2} \bibleverse{4} Porque
Juan le dijo: ``No te es lícito tenerla''. \footnote{\textbf{14:4} Mat
  19,9; Lev 18,16} \bibleverse{5} Cuando quiso matarlo, temió a la
multitud, porque lo tenían por profeta. \footnote{\textbf{14:5} Mat
  21,26} \bibleverse{6} Pero cuando llegó el cumpleaños de Herodes, la
hija de Herodías bailó en medio de ellos y agradó a Herodes.
\bibleverse{7} Por eso prometió con juramento darle todo lo que pidiera.
\bibleverse{8} Ella, incitada por su madre, dijo: ``Dadme aquí en
bandeja la cabeza de Juan el Bautista.''

\bibleverse{9} El rey se afligió, pero por el bien de sus juramentos y
de los que se sentaban a la mesa con él, ordenó que se le diera,
\bibleverse{10} y mandó decapitar a Juan en la cárcel. \bibleverse{11}
Su cabeza fue traída en una bandeja y entregada a la joven; y ella la
llevó a su madre. \bibleverse{12} Vinieron sus discípulos, tomaron el
cuerpo y lo enterraron. Luego fueron a avisar a Jesús.

\hypertarget{alimentando-a-los-cinco-mil}{%
\subsection{Alimentando a los cinco
mil}\label{alimentando-a-los-cinco-mil}}

\bibleverse{13} Al oír esto, Jesús se retiró de allí en una barca a un
lugar desierto y apartado. Cuando las multitudes lo oyeron, lo siguieron
a pie desde las ciudades.

\bibleverse{14} Jesús salió y vio una gran multitud. Se compadeció de
ellos y sanó a los enfermos. \bibleverse{15} Al anochecer, sus
discípulos se acercaron a él, diciendo: ``Este lugar está desierto, y la
hora ya es tardía. Despide a las multitudes para que vayan a las aldeas
y se compren comida''.

\bibleverse{16} Pero Jesús les dijo: ``No hace falta que se vayan.
Denles ustedes algo de comer''.

\bibleverse{17} Le dijeron: ``Sólo tenemos aquí cinco panes y dos
peces''.

\bibleverse{18} Dijo: ``Tráiganmelos''. \bibleverse{19} Mandó a las
multitudes que se sentaran sobre la hierba; tomó los cinco panes y los
dos peces, y mirando al cielo, bendijo, partió y dio los panes a los
discípulos; y los discípulos dieron a las multitudes. \bibleverse{20}
Todos comieron y se saciaron. Tomaron doce cestas llenas de lo que había
sobrado de los trozos. \footnote{\textbf{14:20} 2Re 4,44}
\bibleverse{21} Los que comieron fueron unos cinco mil hombres, además
de las mujeres y los niños.

\hypertarget{regreso-de-los-discuxedpulos-al-otro-lado-del-lago-por-la-noche-el-caminar-de-jesuxfas-sobre-el-lago-el-desembarco-en-gennesaret}{%
\subsection{Regreso de los discípulos al otro lado del lago por la
noche; el caminar de Jesús sobre el lago; el desembarco en
Gennesaret}\label{regreso-de-los-discuxedpulos-al-otro-lado-del-lago-por-la-noche-el-caminar-de-jesuxfas-sobre-el-lago-el-desembarco-en-gennesaret}}

\bibleverse{22} En seguida, Jesús hizo que los discípulos subieran a la
barca y fueran delante de él a la otra orilla, mientras despedía a la
multitud. \bibleverse{23} Después de despedir a las multitudes, subió al
monte a orar. Al anochecer, estaba allí solo. \footnote{\textbf{14:23}
  Luc 6,12; Luc 9,18} \bibleverse{24} Pero la barca estaba ahora en
medio del mar, angustiada por las olas, pues el viento era contrario.
\bibleverse{25} En la cuarta vigilia de la noche, Jesús se acercó a
ellos, caminando sobre el mar. \bibleverse{26} Cuando los discípulos le
vieron caminar sobre el mar, se turbaron, diciendo: ``¡Es un fantasma!''
Y gritaron de miedo. \footnote{\textbf{14:26} Luc 24,37} \bibleverse{27}
Pero enseguida Jesús les habló diciendo: ``¡Anímense! ¡Soy yo! No
tengáis miedo''.

\bibleverse{28} Pedro le respondió: ``Señor, si eres tú, mándame ir a ti
sobre las aguas''.

\bibleverse{29} Dijo: ``¡Ven!'' Pedro bajó de la barca y caminó sobre
las aguas para acercarse a Jesús. \bibleverse{30} Pero al ver que el
viento era fuerte, tuvo miedo, y empezando a hundirse, gritó diciendo:
``¡Señor, sálvame!''.

\bibleverse{31} Inmediatamente, Jesús extendió la mano, lo agarró y le
dijo: ``Hombre de poca fe, ¿por qué dudaste?'' \bibleverse{32} Cuando
subieron a la barca, cesó el viento. \bibleverse{33} Los que estaban en
la barca se acercaron y le adoraron, diciendo: ``¡Verdaderamente eres el
Hijo de Dios!'' \footnote{\textbf{14:33} Mat 16,16; Juan 1,49}

\hypertarget{la-reuniuxf3n-de-personas-y-la-curaciuxf3n-de-los-enfermos-en-gennesaret}{%
\subsection{La reunión de personas y la curación de los enfermos en
Gennesaret}\label{la-reuniuxf3n-de-personas-y-la-curaciuxf3n-de-los-enfermos-en-gennesaret}}

\bibleverse{34} Después de cruzar, llegaron a la tierra de Genesaret.
\bibleverse{35} Cuando los habitantes de aquel lugar lo reconocieron,
enviaron a toda la región circundante y le trajeron a todos los
enfermos; \bibleverse{36} y le rogaron que sólo tocaran el fleco de su
manto. Todos los que lo tocaban quedaban sanos. \footnote{\textbf{14:36}
  Mat 9,21; Luc 6,19}

\hypertarget{la-disputa-de-jesuxfas-con-sus-oponentes-por-lavarse-las-manos-su-advertencia-de-los-estatutos-humanos-y-la-marca-de-la-verdadera-impureza}{%
\subsection{La disputa de Jesús con sus oponentes por lavarse las manos;
su advertencia de los estatutos humanos y la marca de la verdadera
impureza}\label{la-disputa-de-jesuxfas-con-sus-oponentes-por-lavarse-las-manos-su-advertencia-de-los-estatutos-humanos-y-la-marca-de-la-verdadera-impureza}}

\hypertarget{section-14}{%
\section{15}\label{section-14}}

\bibleverse{1} Entonces los fariseos y los escribas vinieron a Jesús
desde Jerusalén, diciendo: \bibleverse{2} ``¿Por qué tus discípulos
desobedecen la tradición de los ancianos? Porque no se lavan las manos
cuando comen el pan''. \footnote{\textbf{15:2} Luc 11,38}

\bibleverse{3} Él les respondió: ``¿Por qué también vosotros
desobedecéis el mandamiento de Dios por vuestra tradición?
\bibleverse{4} Porque Dios mandó: `Honra a tu padre y a tu madre,' y `El
que hable mal del padre o de la madre, que muera'. \bibleverse{5} Pero
vosotros decís: `El que diga a su padre o a su madre: ``La ayuda que de
otro modo hubieras recibido es un don dedicado a Dios'', \footnote{\textbf{15:5}
  Prov 28,24} \bibleverse{6} no honrará a su padre ni a su madre.'
Habéis anulado el mandamiento de Dios por vuestra tradición. \footnote{\textbf{15:6}
  1Tim 5,8} \bibleverse{7} ¡Hipócritas! Bien profetizó Isaías sobre
vosotros, diciendo, \bibleverse{8} `Esta gente se acerca a mí con su
boca, y me honran con sus labios; pero su corazón está lejos de mí.
\bibleverse{9} Y me adoran en vano, enseñando como doctrina reglas
hechas por los hombres''.

\bibleverse{10} Convocó a la multitud y les dijo: ``Oíd y entended.
\bibleverse{11} Lo que entra en la boca no contamina al hombre; pero lo
que sale de la boca, esto contamina al hombre.'' \footnote{\textbf{15:11}
  Hech 10,15; 1Tim 4,4; Tit 1,15}

\bibleverse{12} Entonces se acercaron los discípulos y le dijeron:
``¿Sabes que los fariseos se ofendieron al oír esta frase?''

\bibleverse{13} Pero él respondió: ``Toda planta que mi Padre celestial
no haya plantado será desarraigada. \footnote{\textbf{15:13} Hech 5,38}
\bibleverse{14} Déjenlos en paz. Son guías ciegos de los ciegos. Si los
ciegos guían a los ciegos, ambos caerán en un pozo''. \footnote{\textbf{15:14}
  Mat 23,24; Luc 6,39; Rom 2,19}

\bibleverse{15} Pedro le respondió: ``Explícanos la parábola''.

\bibleverse{16} Entonces Jesús dijo: ``¿Tampoco vosotros entendéis
todavía? \bibleverse{17} ¿No entendéis que todo lo que entra en la boca
pasa al vientre y luego sale del cuerpo? \bibleverse{18} Pero lo que
sale de la boca, sale del corazón y contamina al hombre. \bibleverse{19}
Porque del corazón salen los malos pensamientos, los asesinatos, los
adulterios, los pecados sexuales, los robos, los falsos testimonios y
las blasfemias. \footnote{\textbf{15:19} Gén 8,21} \bibleverse{20} Estas
son las cosas que contaminan al hombre; pero comer con las manos sin
lavar no contamina al hombre.''

\hypertarget{jesuxfas-y-la-mujer-cananea-en-el-uxe1rea-de-tiro-y-siduxf3n}{%
\subsection{Jesús y la mujer cananea en el área de Tiro y
Sidón}\label{jesuxfas-y-la-mujer-cananea-en-el-uxe1rea-de-tiro-y-siduxf3n}}

\bibleverse{21} Jesús salió de allí y se retiró a la región de Tiro y
Sidón. \bibleverse{22} He aquí que una mujer cananea salió de aquellos
confines y clamó diciendo: ``¡Ten piedad de mí, Señor, hijo de David! Mi
hija está gravemente poseída por un demonio''.

\bibleverse{23} Pero él no le respondió ni una palabra. Sus discípulos
se acercaron y le rogaron, diciendo: ``Despídela, porque clama tras
nosotros''.

\bibleverse{24} Pero él respondió: ``No he sido enviado sino a las
ovejas perdidas de la casa de Israel''. \footnote{\textbf{15:24} Mat
  10,5-6; Rom 15,8}

\bibleverse{25} Pero ella se acercó y le adoró diciendo: ``Señor,
ayúdame''.

\bibleverse{26} Pero él respondió: ``No conviene tomar el pan de los
niños y echarlo a los perros''.

\bibleverse{27} Pero ella dijo: ``Sí, Señor, pero hasta los perros comen
las migajas que caen de la mesa de sus amos''.

\bibleverse{28} Entonces Jesús le respondió: ``Mujer, ¡qué grande es tu
fe! Hágase en ti lo que deseas''. Y su hija quedó curada desde aquella
hora. \footnote{\textbf{15:28} Mat 8,10; Mat 8,13}

\hypertarget{actividad-curativa-de-jesuxfas-en-galilea-en-la-orilla-oriental-del-lago-alimentando-a-los-cuatro-mil}{%
\subsection{Actividad curativa de Jesús en Galilea en la orilla oriental
del lago; Alimentando a los cuatro
mil}\label{actividad-curativa-de-jesuxfas-en-galilea-en-la-orilla-oriental-del-lago-alimentando-a-los-cuatro-mil}}

\bibleverse{29} Jesús salió de allí y se acercó al mar de Galilea; subió
al monte y se sentó allí. \bibleverse{30} Acudieron a él grandes
multitudes, llevando consigo cojos, ciegos, mudos, mutilados y muchos
otros, y los pusieron a sus pies. Él los curó, \bibleverse{31} de modo
que la multitud se maravillaba al ver que los mudos hablaban, los
heridos se curaban, los cojos caminaban y los ciegos veían, y
glorificaban al Dios de Israel. \footnote{\textbf{15:31} Mar 7,37}

\bibleverse{32} Jesús llamó a sus discípulos y les dijo: ``Tengo
compasión de la multitud, porque ya llevan tres días conmigo y no tienen
nada que comer. No quiero despedirlos en ayunas, o podrían desmayarse en
el camino''. \footnote{\textbf{15:32} Mat 14,13-21}

\bibleverse{33} Los discípulos le dijeron: ``¿De dónde podríamos sacar
tantos panes en un lugar desierto como para satisfacer a una multitud
tan grande?''

\bibleverse{34} Jesús les dijo: ``¿Cuántos panes tienen?'' Dijeron:
``Siete, y unos pocos peces pequeños''.

\bibleverse{35} Mandó a la multitud que se sentara en el suelo;
\bibleverse{36} y tomó los siete panes y los peces. Dio gracias y los
partió, y dio a los discípulos, y los discípulos a la multitud.
\bibleverse{37} Todos comieron y se saciaron. Tomaron siete cestas
llenas de los trozos que sobraron. \bibleverse{38} Los que comieron
fueron cuatro mil hombres, además de las mujeres y los niños.
\bibleverse{39} Luego despidió a las multitudes, subió a la barca y
llegó a los límites de Magdala.

\hypertarget{rechazo-de-la-demanda-de-los-oponentes-de-seuxf1ales-y-advertencia-contra-la-levadura-de-los-fariseos}{%
\subsection{Rechazo de la demanda de los oponentes de señales y
advertencia contra la levadura de los
fariseos}\label{rechazo-de-la-demanda-de-los-oponentes-de-seuxf1ales-y-advertencia-contra-la-levadura-de-los-fariseos}}

\hypertarget{section-15}{%
\section{16}\label{section-15}}

\bibleverse{1} Se acercaron los fariseos y los saduceos y, poniéndole a
prueba, le pidieron que les mostrara una señal del cielo. \footnote{\textbf{16:1}
  Mat 12,38} \bibleverse{2} Pero él les contestó: ``Cuando cae la tarde,
decís: `Va a hacer buen tiempo, porque el cielo está rojo'.
\bibleverse{3} Por la mañana, decís: ``Hoy hará mal tiempo, porque el
cielo está rojo y amenazante''. ¡Hipócritas! Sabéis discernir el aspecto
del cielo, ¡pero no sabéis discernir los signos de los tiempos!
\footnote{\textbf{16:3} Mat 11,4} \bibleverse{4} Una generación malvada
y adúltera busca una señal, y no se le dará ninguna señal, sino la del
profeta Jonás.'' Los dejó y se fue. \footnote{\textbf{16:4} Mat 12,39-40}
\bibleverse{5} Los discípulos llegaron al otro lado y se habían olvidado
de tomar el pan. \bibleverse{6} Jesús les dijo: ``Mirad y guardaos de la
levadura de los fariseos y saduceos.'' \footnote{\textbf{16:6} Luc 12,1}

\bibleverse{7} Razonaban entre ellos, diciendo: ``No hemos traído pan''.

\bibleverse{8} Jesús, al darse cuenta, dijo: ``¿Por qué discutís entre
vosotros, hombres de poca fe, porque no habéis traído pan?
\bibleverse{9} ¿Aún no percibís ni os acordáis de los cinco panes para
los cinco mil, y de cuántas cestas recogisteis, \footnote{\textbf{16:9}
  Mat 14,17-21} \bibleverse{10} o de los siete panes para los cuatro
mil, y de cuántas cestas recogisteis? \footnote{\textbf{16:10} Mat
  15,34-38} \bibleverse{11} ¿Cómo es que no percibís que no os hablé del
pan? Pero tened cuidado con la levadura de los fariseos y saduceos''.

\bibleverse{12} Entonces comprendieron que no les decía que se cuidaran
de la levadura del pan, sino de la enseñanza de los fariseos y saduceos.

\hypertarget{la-confesiuxf3n-del-mesuxedas-de-pedro-en-cesarea-de-filipo-llamando-a-pedro-a-ser-el-fundador-y-luxedder-de-la-iglesia}{%
\subsection{La Confesión del Mesías de Pedro en Cesarea de Filipo;
Llamando a Pedro a ser el fundador y líder de la
iglesia}\label{la-confesiuxf3n-del-mesuxedas-de-pedro-en-cesarea-de-filipo-llamando-a-pedro-a-ser-el-fundador-y-luxedder-de-la-iglesia}}

\bibleverse{13} Cuando Jesús llegó a las partes de Cesarea de Filipo,
preguntó a sus discípulos: ``¿Quién dicen los hombres que soy yo, el
Hijo del Hombre?''

\bibleverse{14} Dijeron: ``Unos dicen que Juan el Bautista, otros que
Elías y otros que Jeremías o alguno de los profetas''. \footnote{\textbf{16:14}
  Mat 14,2; Mat 17,10; Luc 7,16}

\bibleverse{15} Les dijo: ``¿Pero quién decís que soy yo?''.

\bibleverse{16} Simón Pedro respondió: ``Tú eres el Cristo, el Hijo de
Dios vivo''. \footnote{\textbf{16:16} Juan 6,69}

\bibleverse{17} Jesús le respondió: ``Bendito seas, Simón Bar Jonás,
porque no te lo ha revelado la carne ni la sangre, sino mi Padre que
está en los cielos. \footnote{\textbf{16:17} Mat 11,27; Gal 1,15-16}
\bibleverse{18} También te digo que tú eres Pedro, y sobre esta piedra
edificaré mi iglesia,\footnote{\textbf{16:18} Griego, petra, masa rocosa
  o lecho de roca.} y las puertas del Hades\footnote{\textbf{16:18} o,
  Infierno} no prevalecerán contra ella. \footnote{\textbf{16:18} Juan
  1,42; Efes 2,20} \bibleverse{19} Te daré las llaves del Reino de los
Cielos, y todo lo que ates en la tierra habrá sido atado en el cielo; y
todo lo que sueltes en la tierra habrá sido soltado en el cielo.''
\footnote{\textbf{16:19} Mat 18,18} \bibleverse{20} Entonces mandó a los
discípulos que no dijeran a nadie que él era Jesús el Cristo.
\footnote{\textbf{16:20} Mat 17,9}

\hypertarget{primer-anuncio-de-sufrimiento}{%
\subsection{Primer anuncio de
sufrimiento}\label{primer-anuncio-de-sufrimiento}}

\bibleverse{21} Desde entonces, Jesús comenzó a mostrar a sus discípulos
que debía ir a Jerusalén y sufrir muchas cosas por parte de los
ancianos, los jefes de los sacerdotes y los escribas, y ser muerto, y al
tercer día resucitar. \footnote{\textbf{16:21} Mat 12,40; Juan 2,19}

\bibleverse{22} Pedro lo tomó aparte y comenzó a reprenderlo, diciendo:
``¡Lejos de ti, Señor! Esto no se te hará nunca''.

\bibleverse{23} Pero él, volviéndose, dijo a Pedro: ``¡Apártate de mí,
Satanás! Eres una piedra de tropiezo para mí, porque no pones tu mente
en las cosas de Dios, sino en las de los hombres.''

\hypertarget{proverbios-sobre-el-seguimiento-de-los-discuxedpulos-en-el-sufrimiento}{%
\subsection{Proverbios sobre el seguimiento de los discípulos en el
sufrimiento}\label{proverbios-sobre-el-seguimiento-de-los-discuxedpulos-en-el-sufrimiento}}

\bibleverse{24} Entonces Jesús dijo a sus discípulos: ``Si alguno quiere
venir en pos de mí, niéguese a sí mismo, tome su cruz y sígame.
\footnote{\textbf{16:24} Mat 10,38-39; 1Pe 2,21} \bibleverse{25} Porque
el que quiera salvar su vida, la perderá, y el que pierda su vida por
mí, la encontrará. \footnote{\textbf{16:25} Apoc 12,11} \bibleverse{26}
Porque ¿de qué le servirá al hombre ganar el mundo entero si pierde su
vida? ¿O qué dará el hombre a cambio de su vida? \footnote{\textbf{16:26}
  Luc 12,20} \bibleverse{27} Porque el Hijo del Hombre vendrá en la
gloria de su Padre con sus ángeles, y entonces pagará a cada uno según
sus obras. \footnote{\textbf{16:27} Rom 2,6} \bibleverse{28} De cierto
os digo que hay algunos de los que están aquí que no probarán la muerte
hasta que vean al Hijo del Hombre venir en su Reino.'' \footnote{\textbf{16:28}
  Mat 10,23}

\hypertarget{la-transfiguraciuxf3n-de-jesuxfas-en-la-montauxf1a}{%
\subsection{La transfiguración de Jesús en la
montaña}\label{la-transfiguraciuxf3n-de-jesuxfas-en-la-montauxf1a}}

\hypertarget{section-16}{%
\section{17}\label{section-16}}

\bibleverse{1} Al cabo de seis días, Jesús tomó consigo a Pedro,
Santiago y Juan, su hermano, y los llevó solos a un monte alto.
\footnote{\textbf{17:1} Mat 26,37; Mar 5,37; Mar 13,3; Mar 14,33; Luc
  8,51} \bibleverse{2} Se transformó \footnote{\textbf{17:2} NU omite el
  versículo 21.} ante ellos. Su rostro brillaba como el sol, y sus
vestidos se volvieron blancos como la luz. \footnote{\textbf{17:2} 2Pe
  1,16-18; Apoc 1,16} \bibleverse{3} Se les aparecieron Moisés y Elías
hablando con él.

\bibleverse{4} Pedro respondió y dijo a Jesús: ``Señor, es bueno que
estemos aquí. Si quieres, hagamos aquí tres tiendas: una para ti, otra
para Moisés y otra para Elías''.

\bibleverse{5} Mientras aún hablaba, he aquí que una nube brillante los
cubrió con su sombra. De la nube salió una voz que decía: ``Este es mi
Hijo amado, en quien me complazco. Escuchadle''. \footnote{\textbf{17:5}
  Mat 3,17}

\bibleverse{6} Cuando los discípulos lo oyeron, cayeron de bruces y
tuvieron mucho miedo. \bibleverse{7} Jesús se acercó, los tocó y les
dijo: ``Levántense y no tengan miedo''. \bibleverse{8} Levantando los
ojos, no vieron a nadie, excepto a Jesús solo.

\bibleverse{9} Mientras bajaban del monte, Jesús les mandó decir: ``No
contéis a nadie lo que habéis visto, hasta que el Hijo del Hombre haya
resucitado.'' \footnote{\textbf{17:9} Mat 16,20}

\bibleverse{10} Sus discípulos le preguntaron: ``Entonces, ¿por qué
dicen los escribas que Elías debe venir primero?'' \footnote{\textbf{17:10}
  Mat 11,14}

\bibleverse{11} Jesús les contestó: ``En efecto, Elías viene primero y
restaurará todas las cosas; \bibleverse{12} pero yo os digo que Elías ya
ha venido, y no lo reconocieron, sino que le hicieron lo que quisieron.
Así también el Hijo del Hombre sufrirá por ellos''. \footnote{\textbf{17:12}
  Mat 14,9-10} \bibleverse{13} Entonces los discípulos comprendieron que
les hablaba de Juan el Bautista. \footnote{\textbf{17:13} Luc 1,17}

\hypertarget{curaciuxf3n-de-un-niuxf1o-epiluxe9ptico-enseuxf1ando-sobre-el-fracaso-de-los-discuxedpulos}{%
\subsection{Curación de un niño epiléptico; Enseñando sobre el fracaso
de los
discípulos}\label{curaciuxf3n-de-un-niuxf1o-epiluxe9ptico-enseuxf1ando-sobre-el-fracaso-de-los-discuxedpulos}}

\bibleverse{14} Cuando llegaron a la multitud, se le acercó un hombre
que se arrodilló ante él y le dijo: \bibleverse{15} ``Señor, ten
compasión de mi hijo, porque es epiléptico y sufre gravemente; pues
muchas veces cae en el fuego y otras en el agua. \bibleverse{16} Lo
llevé a tus discípulos, y no pudieron sanarlo''.

\bibleverse{17} Jesús respondió: ``¡Generación infiel y perversa! ¿Hasta
cuándo estaré con vosotros? ¿Hasta cuándo los soportaré? Tráiganlo a
mí''. \bibleverse{18} Jesús reprendió al demonio, y salió de él, y el
muchacho quedó sano desde aquella hora.

\bibleverse{19} Entonces los discípulos se acercaron a Jesús en privado
y le dijeron: ``¿Por qué no pudimos expulsarlo?'' \footnote{\textbf{17:19}
  Mat 10,1}

\bibleverse{20} Les dijo: ``Por vuestra incredulidad. Porque ciertamente
os digo que si tenéis fe como un grano de mostaza, le diréis a este
monte: ``Muévete de aquí para allá'', y se moverá; y nada os será
imposible. \footnote{\textbf{17:20} Mat 21,21; Luc 17,6; 1Cor 13,2}
\bibleverse{21} Pero esta clase no sale sino con oración y ayuno''.
\footnote{\textbf{17:21} Mar 9,29}

\hypertarget{segundo-anuncio-del-sufrimiento-en-galilea}{%
\subsection{Segundo anuncio del sufrimiento en
Galilea}\label{segundo-anuncio-del-sufrimiento-en-galilea}}

\bibleverse{22} Mientras estaban en Galilea, Jesús les dijo: ``El Hijo
del Hombre va a ser entregado en manos de los hombres, \footnote{\textbf{17:22}
  Mat 16,21; Mat 20,18-19} \bibleverse{23} y lo matarán, y al tercer día
resucitará.'' Lo sentían mucho.

\hypertarget{el-impuesto-del-templo-y-su-maravilloso-pago-en-capernaum}{%
\subsection{El impuesto del templo y su maravilloso pago en
Capernaum}\label{el-impuesto-del-templo-y-su-maravilloso-pago-en-capernaum}}

\bibleverse{24} Cuando llegaron a Capernaúm, los que recogían las
monedas\footnote{\textbf{17:24} Una didracma es una moneda de plata
  griega que vale 2 dracmas, más o menos lo mismo que 2 denarios
  romanos, o sea, el salario de 2 días. Se utilizaba comúnmente para
  pagar el impuesto del templo de medio shekel, porque 2 dracmas valían
  un medio shekel de plata. Un siclo equivale a unos 10 gramos o a unas
  0,35 onzas.} de la didracma se acercaron a Pedro y le dijeron: ``¿Tu
maestro no paga la didracma?'' \footnote{\textbf{17:24} Éxod 30,13}
\bibleverse{25} Él respondió: ``Sí''. Cuando entró en la casa, Jesús se
le anticipó diciendo: ``¿Qué te parece, Simón? ¿De quién reciben peaje o
tributo los reyes de la tierra? ¿De sus hijos, o de los extranjeros?''

\bibleverse{26} Pedro le dijo: ``De extraños''. Jesús le dijo: ``Por lo
tanto, los niños están exentos. \bibleverse{27} Pero, para no hacerlos
tropezar, ve al mar, echa el anzuelo y recoge el primer pez que salga.
Cuando le hayas abierto la boca, encontrarás una moneda de
plata.\footnote{\textbf{17:27} Un stater es una moneda de plata
  equivalente a cuatro dracmas áticas o dos alejandrinas, o a un siclo
  judío: lo suficiente para cubrir el impuesto de medio siclo del templo
  para dos personas. Un siclo equivale a unos 10 gramos o unas 0,35
  onzas, generalmente en forma de moneda de plata.} Tómala y dásela por
mí y por ti''.

\hypertarget{controversia-entre-discuxedpulos-la-exhortaciuxf3n-de-jesuxfas-a-la-humildad}{%
\subsection{Controversia entre discípulos; La exhortación de Jesús a la
humildad}\label{controversia-entre-discuxedpulos-la-exhortaciuxf3n-de-jesuxfas-a-la-humildad}}

\hypertarget{section-17}{%
\section{18}\label{section-17}}

\bibleverse{1} En aquella hora, los discípulos se acercaron a Jesús y le
dijeron: ``¿Quién es el mayor en el Reino de los Cielos?''

\bibleverse{2} Jesús llamó a un niño, lo puso en medio de ellos
\bibleverse{3} y les dijo: ``Os aseguro que si no os convertís y os
hacéis como niños, no entraréis en el Reino de los Cielos. \footnote{\textbf{18:3}
  Mat 19,14} \bibleverse{4} Por tanto, el que se humille como este niño
es el mayor en el Reino de los Cielos. \bibleverse{5} El que recibe a un
niño como éste en mi nombre, me recibe a mí; \footnote{\textbf{18:5} Mat
  10,40}

\hypertarget{la-preocupaciuxf3n-de-jesuxfas-por-los-pequeuxf1os-y-los-duxe9biles-advertencia-contra-los-seductores-del-mal}{%
\subsection{La preocupación de Jesús por los pequeños y los débiles;
Advertencia contra los seductores del
mal}\label{la-preocupaciuxf3n-de-jesuxfas-por-los-pequeuxf1os-y-los-duxe9biles-advertencia-contra-los-seductores-del-mal}}

\bibleverse{6} pero el que hace tropezar a uno de estos pequeños que
creen en mí, más le valdría que le colgaran al cuello una enorme piedra
de molino y lo hundieran en el fondo del mar. \footnote{\textbf{18:6}
  Luc 17,1-2}

\bibleverse{7} ``¡Ay del mundo por los tropiezos! Porque es necesario
que las ocasiones vengan, pero ¡ay de la persona por la que viene la
ocasión! \bibleverse{8} Si tu mano o tu pie te hacen tropezar, córtalo y
apártalo de ti. Es mejor que entres en la vida manco o lisiado, antes
que tener dos manos o dos pies para ser arrojado al fuego eterno.
\footnote{\textbf{18:8} Mat 5,29-30} \bibleverse{9} Si tu ojo te hace
tropezar, arráncalo y échalo de ti. Es mejor que entres en la vida con
un solo ojo, en lugar de tener dos ojos para ser arrojado a la
Gehenna\footnote{\textbf{18:9} NU omite el versículo 11.} del fuego.
\bibleverse{10} Mirad que no despreciéis a uno de estos pequeños, porque
os digo que en el cielo sus ángeles ven siempre el rostro de mi Padre
que está en el cielo. \footnote{\textbf{18:10} Heb 1,14} \bibleverse{11}
Porque el Hijo del Hombre ha venido a salvar lo que se había perdido.
\footnote{\textbf{18:11} Mat 9,13; Luc 19,10}

\hypertarget{la-paruxe1bola-de-la-oveja-perdida}{%
\subsection{La parábola de la oveja
perdida}\label{la-paruxe1bola-de-la-oveja-perdida}}

\bibleverse{12} ``¿Qué os parece? Si un hombre tiene cien ovejas, y una
de ellas se extravía, ¿no deja las noventa y nueve, va a los montes y
busca la que se ha extraviado? \bibleverse{13} Si la encuentra, os
aseguro que se alegra más por ella que por las noventa y nueve que no se
han descarriado. \bibleverse{14} Así pues, no es la voluntad de vuestro
Padre que está en los cielos que se pierda uno de estos pequeños.

\hypertarget{de-comportamiento-hacia-el-hermano-pecador-sobre-el-efecto-del-juicio-y-la-oraciuxf3n-de-la-iglesia}{%
\subsection{De comportamiento hacia el hermano pecador; sobre el efecto
del juicio y la oración de la
iglesia}\label{de-comportamiento-hacia-el-hermano-pecador-sobre-el-efecto-del-juicio-y-la-oraciuxf3n-de-la-iglesia}}

\bibleverse{15} ``Si tu hermano peca contra ti, ve, muéstrale su falta
entre tú y él solo. Si te escucha, habrás recuperado a tu hermano.
\footnote{\textbf{18:15} Lev 19,17; Luc 17,3; Gal 6,1} \bibleverse{16}
Pero si no te escucha, llévate a uno o dos más contigo, para que en boca
de dos o tres testigos quede establecida toda palabra. \footnote{\textbf{18:16}
  Deuteronomio 19:15} \footnote{\textbf{18:16} Deut 19,15}
\bibleverse{17} Si se niega a escucharles, díselo a la asamblea. Si
también se niega a escuchar a la asamblea, que sea para ustedes como un
gentil o un recaudador de impuestos. \footnote{\textbf{18:17} 1Cor 5,13;
  2Tes 3,6; Tit 3,10} \bibleverse{18} De cierto os digo que todo lo que
atéis en la tierra habrá sido atado en el cielo, y todo lo que soltéis
en la tierra habrá sido soltado en el cielo. \footnote{\textbf{18:18}
  Mat 16,19; Juan 20,23} \bibleverse{19} Además, os aseguro que si dos
de vosotros se ponen de acuerdo en la tierra sobre cualquier cosa que
pidan, les será hecho por mi Padre que está en el cielo. \footnote{\textbf{18:19}
  Mar 11,24} \bibleverse{20} Porque donde hay dos o tres reunidos en mi
nombre, allí estoy yo en medio de ellos''. \footnote{\textbf{18:20} Mat
  28,20}

\hypertarget{del-perduxf3n-la-paruxe1bola-del-sinverguxfcenza}{%
\subsection{Del perdón; la parábola del
sinvergüenza}\label{del-perduxf3n-la-paruxe1bola-del-sinverguxfcenza}}

\bibleverse{21} Entonces Pedro se acercó y le dijo: ``Señor, ¿cuántas
veces va a pecar mi hermano contra mí, y le perdono? ¿Hasta siete
veces?''

\bibleverse{22} Jesús le dijo: ``No te digo hasta siete veces, sino
hasta setenta veces siete. \footnote{\textbf{18:22} Gén 4,24; Luc 17,4;
  Efes 4,32} \bibleverse{23} Por eso, el Reino de los Cielos se parece a
cierto rey que quería ajustar cuentas con sus siervos. \bibleverse{24}
Cuando empezó a ajustar cuentas, le presentaron a uno que le debía diez
mil talentos. \footnote{\textbf{18:24} Diez mil talentos (unas 300
  toneladas de plata) representan una suma de dinero extremadamente
  grande, equivalente a unos 60.000.000 denarios, donde un denario era
  el típico salario de un día de trabajo agrícola.} \bibleverse{25} Pero
como no podía pagar, su señor mandó venderlo, con su mujer, sus hijos y
todo lo que tenía, y que se le pagara. \bibleverse{26} El siervo, pues,
se postró y se arrodilló ante él, diciendo: ``Señor, ten paciencia
conmigo y te lo pagaré todo''. \bibleverse{27} El señor de aquel siervo,
compadecido, lo liberó y le perdonó la deuda.

\bibleverse{28} ``Pero aquel siervo salió y encontró a uno de sus
compañeros que le debía cien denarios,\footnote{\textbf{18:28} 100
  denarios eran aproximadamente la sexagésima parte de un talento, es
  decir, unos 500 gramos (1,1 libras) de plata.} lo agarró y lo tomó por
el cuello, diciendo: ``¡Págame lo que me debes!''.

\bibleverse{29} ``Entonces su consiervo se postró a sus pies y le rogó,
diciendo: ``Ten paciencia conmigo, y te lo pagaré''. \bibleverse{30}
Pero él no quiso, sino que fue y lo echó en la cárcel hasta que le
devolviera lo que le debía. \bibleverse{31} Cuando sus compañeros de
servicio vieron lo que se había hecho, se entristecieron mucho, y
vinieron a contarle a su señor todo lo que se había hecho.
\bibleverse{32} Entonces su señor lo llamó y le dijo: ``¡Siervo malvado!
Te perdoné toda esa deuda porque me lo rogaste. \footnote{\textbf{18:32}
  Luc 6,36} \bibleverse{33} ¿No debías tú también tener misericordia de
tu consiervo, como yo tuve misericordia de ti?' \footnote{\textbf{18:33}
  1Jn 4,11} \bibleverse{34} Su señor se enfureció y lo entregó a los
verdugos hasta que pagara todo lo que se le debía. \footnote{\textbf{18:34}
  Mat 5,26} \bibleverse{35} Así hará también mi Padre celestial con
vosotros, si no perdonáis cada uno a vuestro hermano de corazón por sus
fechorías.'' \footnote{\textbf{18:35} Mat 6,14-15; Sant 2,13}

\hypertarget{salida-hacia-jerusaluxe9n-y-caminata-por-la-ribera-oriental-conversaciones-sobre-el-matrimonio-sobre-el-divorcio-y-la-renuncia-al-matrimonio}{%
\subsection{Salida hacia Jerusalén y caminata por la Ribera Oriental;
Conversaciones sobre el matrimonio, sobre el divorcio y la renuncia al
matrimonio}\label{salida-hacia-jerusaluxe9n-y-caminata-por-la-ribera-oriental-conversaciones-sobre-el-matrimonio-sobre-el-divorcio-y-la-renuncia-al-matrimonio}}

\hypertarget{section-18}{%
\section{19}\label{section-18}}

\bibleverse{1} Cuando Jesús terminó estas palabras, salió de Galilea y
llegó a los límites de Judea, al otro lado del Jordán. \bibleverse{2} Le
siguieron grandes multitudes, y allí los curó.

\bibleverse{3} Los fariseos se acercaron a él para ponerle a prueba y
decirle: ``¿Es lícito que un hombre se divorcie de su mujer por
cualquier motivo?'' \footnote{\textbf{19:3} Mat 5,31-32}

\bibleverse{4} Él respondió: ``¿No has leído que el que los hizo desde
el principio los hizo varón y mujer, \footnote{\textbf{19:4} Génesis
  1:27} \footnote{\textbf{19:4} Gén 1,27} \bibleverse{5} y dijo: ``Por
eso el hombre dejará a su padre y a su madre, y se unirá a su mujer, y
los dos se convertirán en una sola carne''? \footnote{\textbf{19:5}
  Génesis 2:24} \bibleverse{6} De modo que ya no son dos, sino una sola
carne. Por tanto, lo que Dios ha unido, que no lo separe el hombre''.
\footnote{\textbf{19:6} 1Cor 7,10-11}

\bibleverse{7} Le preguntaron: ``¿Por qué, entonces, Moisés nos ordenó
que le diéramos un certificado de divorcio y nos divorciáramos de
ella?'' \footnote{\textbf{19:7} Deut 24,1}

\bibleverse{8} Les dijo: ``Moisés, a causa de la dureza de vuestros
corazones, os permitió divorciaros de vuestras mujeres, pero desde el
principio no ha sido así. \bibleverse{9} Os digo que el que se divorcia
de su mujer, salvo por inmoralidad sexual, y se casa con otra, comete
adulterio; y el que se casa con ella estando divorciada, comete
adulterio.'' \footnote{\textbf{19:9} Luc 16,18}

\bibleverse{10} Sus discípulos le dijeron: ``Si este es el caso del
hombre con su mujer, no conviene casarse''.

\bibleverse{11} Pero él les dijo: ``No todos los hombres pueden recibir
esta palabra, sino aquellos a quienes se les ha dado. \footnote{\textbf{19:11}
  1Cor 7,7; 1Cor 7,17} \bibleverse{12} Porque hay eunucos que nacieron
así desde el vientre de su madre, y hay eunucos que fueron hechos
eunucos por los hombres; y hay eunucos que se hicieron a sí mismos
eunucos por el Reino de los Cielos. El que pueda recibirlo, que lo
reciba''.

\hypertarget{jesuxfas-bendice-a-los-niuxf1os}{%
\subsection{Jesús bendice a los
niños}\label{jesuxfas-bendice-a-los-niuxf1os}}

\bibleverse{13} Entonces le trajeron niños pequeños para que les
impusiera las manos y orara; y los discípulos les reprendieron.
\bibleverse{14} Pero Jesús les dijo: ``Dejad a los niños y no les
prohibáis que vengan a mí, porque el Reino de los Cielos es de los que
son como ellos.'' \footnote{\textbf{19:14} Mat 18,2-3} \bibleverse{15}
Les impuso las manos y se fue de allí.

\hypertarget{la-conversaciuxf3n-de-jesuxfas-con-el-joven-rico-el-peligro-de-la-riqueza}{%
\subsection{La conversación de Jesús con el joven rico; el peligro de la
riqueza}\label{la-conversaciuxf3n-de-jesuxfas-con-el-joven-rico-el-peligro-de-la-riqueza}}

\bibleverse{16} He aquí que uno se acercó a él y le dijo: ``Maestro
bueno, ¿qué debo hacer para tener la vida eterna?''

\bibleverse{17} Le dijo: ``¿Por qué me llamas bueno?\footnote{\textbf{19:17}
  Así que MT y TR. NU dice ``¿Por qué me preguntas sobre lo que es
  bueno?''} Nadie es bueno sino uno, es decir, Dios. Pero si quieres
entrar en la vida, guarda los mandamientos''.

\bibleverse{18} Le dijo: ``¿Cuáles?'' Jesús dijo: ``\,`No asesinarás'.
No cometerás adulterio''. `No robarás'. No darás falso testimonio''.
\footnote{\textbf{19:18} Éxod 20,12-16} \bibleverse{19} `Honra a tu
padre y a tu madre'.\footnote{\textbf{19:19} Éxodo 20:12-16;
  Deuteronomio 5:16-20} Y, `Amarás a tu prójimo como a ti mismo'\,''.
\footnote{\textbf{19:19} Levítico 19:18} \footnote{\textbf{19:19} Lev
  19,18}

\bibleverse{20} El joven le dijo: ``Todo esto lo he observado desde mi
juventud. ¿Qué me falta todavía?''

\bibleverse{21} Jesús le dijo: ``Si quieres ser perfecto, anda, vende lo
que tienes y dalo a los pobres, y tendrás un tesoro en el cielo; y ven,
sígueme.'' \footnote{\textbf{19:21} Mat 6,20; Luc 12,33} \bibleverse{22}
Pero el joven, al oír esto, se fue triste, porque era uno de los que
tenía grandes posesiones. \footnote{\textbf{19:22} Sal 62,10}

\bibleverse{23} Jesús dijo a sus discípulos: ``Os aseguro que un rico
entrará con dificultad en el Reino de los Cielos. \bibleverse{24}
También os digo que es más fácil que un camello pase por el ojo de una
aguja que un rico entre en el Reino de Dios.'' \footnote{\textbf{19:24}
  Mat 7,14}

\bibleverse{25} Cuando los discípulos lo oyeron, se asombraron mucho,
diciendo: ``¿Quién, pues, podrá salvarse?''

\bibleverse{26} Mirándolos, Jesús dijo: ``Para los hombres esto es
imposible, pero para Dios todo es posible''. \footnote{\textbf{19:26}
  Job 42,2}

\hypertarget{la-recompensa-de-seguir-a-jesuxfas-y-la-renuncia}{%
\subsection{La recompensa de seguir a Jesús y la
renuncia}\label{la-recompensa-de-seguir-a-jesuxfas-y-la-renuncia}}

\bibleverse{27} Entonces Pedro respondió: ``He aquí que lo hemos dejado
todo y te hemos seguido. ¿Qué tendremos entonces?'' \footnote{\textbf{19:27}
  Mat 4,20; Mat 4,22}

\bibleverse{28} Jesús les dijo: ``De cierto os digo que vosotros, los
que me habéis seguido, en la regeneración, cuando el Hijo del hombre se
siente en el trono de su gloria, os sentaréis también vosotros en doce
tronos, para juzgar a las doce tribus de Israel. \footnote{\textbf{19:28}
  Luc 22,30; 1Cor 6,2; Apoc 3,21} \bibleverse{29} Todo el que haya
dejado casas, o hermanos, o hermanas, o padre, o madre, o esposa, o
hijos, o tierras, por mi nombre, recibirá cien veces, y heredará la vida
eterna. \bibleverse{30} Pero serán últimos los que sean primeros, y
primeros los que sean últimos.

\hypertarget{paruxe1bola-de-los-trabajadores-de-la-viuxf1a}{%
\subsection{Parábola de los trabajadores de la
viña}\label{paruxe1bola-de-los-trabajadores-de-la-viuxf1a}}

\hypertarget{section-19}{%
\section{20}\label{section-19}}

\bibleverse{1} ``Porque el Reino de los Cielos es semejante a un hombre,
dueño de una casa, que salió de madrugada a contratar obreros para su
viña. \bibleverse{2} Cuando se puso de acuerdo con los obreros por un
denario\footnote{\textbf{20:2} Un denario es una moneda romana de plata
  que vale 1/25 de un aureus romano. Este era el salario común para un
  día de trabajo agrícola.} al día, los envió a su viña. \bibleverse{3}
Salió a eso de la tercera hora, \footnote{\textbf{20:3} El tiempo se
  medía desde la salida hasta la puesta del sol, por lo que la tercera
  hora sería alrededor de las 9:00 de la mañana.} y vio a otros que
estaban ociosos en la plaza. \bibleverse{4} Les dijo: ``Id también
vosotros a la viña, y os daré lo que sea justo. Y ellos se fueron.
\bibleverse{5} Volvió a salir hacia la hora sexta y la novena,
\footnote{\textbf{20:5} mediodía y 15:00 h.} e hizo lo mismo.
\bibleverse{6} A la hora undécima\footnote{\textbf{20:6} 17:00 h.} salió
y encontró a otros que estaban parados. Les dijo: ``¿Por qué estáis aquí
todo el día sin hacer nada?

\bibleverse{7} ``Le dijeron: `Porque nadie nos ha contratado'. ``Les
dijo: `Id también vosotros a la viña, y recibiréis lo que sea justo'.

\bibleverse{8} ``Cuando llegó la noche, el señor de la viña dijo a su
administrador: ``Llama a los obreros y págales su salario, empezando por
los últimos hasta los primeros''. \bibleverse{9} ``Cuando llegaron los
que habían sido contratados hacia la hora undécima, recibieron un
denario cada uno. \bibleverse{10} Cuando llegaron los primeros,
supusieron que iban a recibir más; y también ellos recibieron cada uno
un denario. \bibleverse{11} Cuando lo recibieron, murmuraron contra el
dueño de la casa, \bibleverse{12} diciendo: `¡Estos últimos han gastado
una hora, y los has hecho iguales a nosotros, que hemos soportado la
carga del día y el calor abrasador!'

\bibleverse{13} ``Pero él respondió a uno de ellos: `Amigo, no te hago
ningún mal. ¿No te has puesto de acuerdo conmigo por un denario?
\bibleverse{14} Toma lo que es tuyo y sigue tu camino. Es mi deseo dar a
este último tanto como a ti. \bibleverse{15} ¿No me es lícito hacer lo
que quiero con lo que poseo? ¿O acaso tu ojo es malicioso, porque yo soy
bueno?' \footnote{\textbf{20:15} Rom 9,16; Rom 9,21} \bibleverse{16}
Así, los últimos serán los primeros, y los primeros los últimos. Porque
muchos son los llamados, pero pocos los elegidos''.

\hypertarget{salida-hacia-jerusaluxe9n-tercer-anuncio-del-sufrimiento-de-jesuxfas}{%
\subsection{Salida hacia Jerusalén; tercer anuncio del sufrimiento de
Jesús}\label{salida-hacia-jerusaluxe9n-tercer-anuncio-del-sufrimiento-de-jesuxfas}}

\bibleverse{17} Mientras Jesús subía a Jerusalén, tomó aparte a los doce
discípulos, y en el camino les dijo: \bibleverse{18} ``He aquí que
subimos a Jerusalén, y el Hijo del Hombre será entregado a los sumos
sacerdotes y a los escribas, y lo condenarán a muerte, \footnote{\textbf{20:18}
  Mat 16,21; Mat 17,22-23; Juan 2,13} \bibleverse{19} y lo entregarán a
los gentiles para que lo escarnezcan, lo azoten y lo crucifiquen; y al
tercer día resucitará.''

\hypertarget{solicitud-ambiciosa-de-salomuxe9-para-sus-hijos-santiago-y-juan}{%
\subsection{Solicitud ambiciosa de Salomé para sus hijos Santiago y
Juan}\label{solicitud-ambiciosa-de-salomuxe9-para-sus-hijos-santiago-y-juan}}

\bibleverse{20} Entonces la madre de los hijos de Zebedeo se acercó a él
con sus hijos, arrodillándose y pidiéndole una cosa. \footnote{\textbf{20:20}
  Mat 10,2} \bibleverse{21} Él le dijo: ``¿Qué quieres?'' Ella le dijo:
``Ordena que estos dos hijos míos se sienten, uno a tu derecha y otro a
tu izquierda, en tu Reino''. \footnote{\textbf{20:21} Mat 19,28}

\bibleverse{22} Pero Jesús respondió: ``No sabes lo que pides. ¿Eres
capaz de beber el cáliz que yo voy a beber, y ser bautizado con el
bautismo con el que yo soy bautizado?'' Le dijeron: ``Podemos''.
\footnote{\textbf{20:22} Mat 26,39; Luc 12,50}

\bibleverse{23} Les dijo: ``Ciertamente, beberéis mi copa y seréis
bautizados con el bautismo con el que yo soy bautizado; pero sentarse a
mi derecha y a mi izquierda no me corresponde a mí, sino a quien ha sido
preparado por mi Padre.'' \footnote{\textbf{20:23} Hech 12,2; Apoc 1,9}

\hypertarget{del-deber-de-servir-por-el-reino-de-los-cielos}{%
\subsection{Del deber de servir por el reino de los
cielos}\label{del-deber-de-servir-por-el-reino-de-los-cielos}}

\bibleverse{24} Cuando los diez lo oyeron, se indignaron con los dos
hermanos. \footnote{\textbf{20:24} Luc 22,24-26}

\bibleverse{25} Pero Jesús los convocó y les dijo: ``Sabéis que los
jefes de las naciones se enseñorean de ellas, y sus grandes ejercen su
autoridad sobre ellas. \bibleverse{26} No será así entre ustedes, sino
que el que quiera hacerse grande entre ustedes será\footnote{\textbf{20:26}
  TR lee ``déjalo ser'' en lugar de ``será''} su servidor. \footnote{\textbf{20:26}
  Mat 23,11; 1Cor 9,19} \bibleverse{27} El que quiera ser el primero
entre vosotros será vuestro siervo, \footnote{\textbf{20:27} Mar 9,35}
\bibleverse{28} así como el Hijo del Hombre no ha venido a ser servido,
sino a servir, y a dar su vida en rescate por muchos.'' \footnote{\textbf{20:28}
  Luc 22,27; Fil 2,7; 1Pe 1,18-19}

\hypertarget{curaciuxf3n-de-dos-ciegos-cerca-de-jericuxf3}{%
\subsection{Curación de dos ciegos cerca de
Jericó}\label{curaciuxf3n-de-dos-ciegos-cerca-de-jericuxf3}}

\bibleverse{29} Al salir de Jericó, le seguía una gran multitud.
\bibleverse{30} He aquí que dos ciegos sentados junto al camino, al oír
que pasaba Jesús, gritaron: ``¡Señor, ten piedad de nosotros, hijo de
David!'' \bibleverse{31} La multitud los reprendió, diciéndoles que se
callaran, pero ellos gritaron aún más: ``¡Señor, ten piedad de nosotros,
hijo de David!''

\bibleverse{32} Jesús se detuvo, los llamó y les preguntó: ``¿Qué
quieren que haga por ustedes?''

\bibleverse{33} Le dijeron: ``Señor, que se nos abran los ojos''.

\bibleverse{34} Jesús, compadecido, les tocó los ojos; y al instante sus
ojos recibieron la vista, y le siguieron.

\hypertarget{entrada-de-jesuxfas-a-jerusaluxe9n}{%
\subsection{Entrada de Jesús a
Jerusalén}\label{entrada-de-jesuxfas-a-jerusaluxe9n}}

\hypertarget{section-20}{%
\section{21}\label{section-20}}

\bibleverse{1} Cuando se acercaron a Jerusalén y llegaron a
Betfagé,\footnote{\textbf{21:1} TR y NU leen ``Bethphage'' en lugar de
  ``Bethsphage''} al Monte de los Olivos, Jesús envió a dos discípulos,
\bibleverse{2} diciéndoles: ``Id a la aldea que está enfrente de
vosotros, y enseguida encontraréis una asna atada, y un pollino con
ella. Desátenlos y tráiganlos a mí. \bibleverse{3} Si alguien os dice
algo, le diréis: ``El Señor los necesita'', e inmediatamente los
enviará.'' \footnote{\textbf{21:3} Mat 26,18}

\bibleverse{4} Todo esto se hizo para que se cumpliera lo que se dijo
por medio del profeta, diciendo, \bibleverse{5} ``Dile a la hija de
Sion, He aquí que tu Rey viene a ti, humilde, y montado \footnote{\textbf{21:5}
  Zacarías 9:9} sobre una asna, sobre un pollino, hijo de animal de
carga''.

\bibleverse{6} Los discípulos fueron e hicieron lo que Jesús les había
mandado, \bibleverse{7} y trajeron el asno y el pollino, y pusieron
sobre ellos sus ropas; y él se sentó sobre ellos. \bibleverse{8} Una
multitud muy numerosa tendió sus ropas en el camino. Otros cortaban
ramas de los árboles y las extendían sobre el camino. \footnote{\textbf{21:8}
  2Re 9,13} \bibleverse{9} Las multitudes que iban delante de él, y las
que le seguían, no dejaban de gritar: ``¡Hosanna\footnote{\textbf{21:9}
  ``Hosanna'' significa ``sálvanos'' o ``ayúdanos, te rogamos''.} al
hijo de David! ¡Bendito el que viene en nombre del Señor! Hosanna en las
alturas!'' \footnote{\textbf{21:9} Salmo 118:26} \footnote{\textbf{21:9}
  Sal 118,25-26}

\bibleverse{10} Cuando llegó a Jerusalén, toda la ciudad se agitó
diciendo: ``¿Quién es éste?''.

\bibleverse{11} Las multitudes decían: ``Este es el profeta Jesús, de
Nazaret de Galilea''.

\hypertarget{la-limpieza-del-templo}{%
\subsection{La limpieza del templo}\label{la-limpieza-del-templo}}

\bibleverse{12} Jesús entró en el templo de Dios y expulsó a todos los
que vendían y compraban en el templo, y derribó las mesas de los
cambistas y los asientos de los que vendían palomas. \bibleverse{13} Les
dijo: ``Está escrito: ``Mi casa será llamada casa de
oración'',\footnote{\textbf{21:13} Isaías 56:7} pero vosotros la habéis
convertido en una cueva de ladrones.'' \footnote{\textbf{21:13} Jeremías
  7:11} \footnote{\textbf{21:13} Jer 7,11}

\hypertarget{sanaciones-en-el-templo-y-homenaje-a-los-niuxf1os}{%
\subsection{Sanaciones en el templo y homenaje a los
niños}\label{sanaciones-en-el-templo-y-homenaje-a-los-niuxf1os}}

\bibleverse{14} Los cojos y los ciegos acudían a él en el templo, y los
curaba. \bibleverse{15} Pero cuando los jefes de los sacerdotes y los
escribas vieron las maravillas que hacía, y a los niños que gritaban en
el templo y decían: ``¡Hosanna al hijo de David!'', se indignaron,
\bibleverse{16} y le dijeron: ``¿Oyes lo que dicen éstos?''. Jesús les
dijo: ``Sí. ¿Nunca habéis leído: ``De la boca de los niños y de los
lactantes has perfeccionado la alabanza''?'' \footnote{\textbf{21:16}
  Salmo 8:2}

\bibleverse{17} Los dejó y salió de la ciudad hacia Betania, y acampó
allí.

\hypertarget{la-maldiciuxf3n-de-la-higuera-estuxe9ril}{%
\subsection{La maldición de la higuera
estéril}\label{la-maldiciuxf3n-de-la-higuera-estuxe9ril}}

\bibleverse{18} Por la mañana, al volver a la ciudad, tuvo hambre.
\bibleverse{19} Al ver una higuera junto al camino, se acercó a ella y
no encontró en ella más que hojas. Le dijo: ``Que no haya fruto de ti
para siempre''. Inmediatamente la higuera se secó. \footnote{\textbf{21:19}
  Luc 13,6}

\bibleverse{20} Cuando los discípulos lo vieron, se maravillaron
diciendo: ``¿Cómo es que la higuera se marchitó inmediatamente?''

\bibleverse{21} Jesús les contestó: ``De cierto os digo que, si tenéis
fe y no dudáis, no sólo se hará lo que se hizo con la higuera, sino que
incluso si le dijerais a este monte: ``Tómalo y échalo al mar'', se
haría. \footnote{\textbf{21:21} Mat 17,20} \bibleverse{22} Todo lo que
pidáis en la oración, creyendo, lo recibiréis''.

\hypertarget{la-pregunta-del-sumo-consejo-sobre-la-autoridad-de-jesuxfas}{%
\subsection{La pregunta del sumo consejo sobre la autoridad de
Jesús}\label{la-pregunta-del-sumo-consejo-sobre-la-autoridad-de-jesuxfas}}

\bibleverse{23} Cuando entró en el templo, los jefes de los sacerdotes y
los ancianos del pueblo se acercaron a él mientras enseñaba y le
dijeron: ``¿Con qué autoridad haces estas cosas? ¿Quién te ha dado esta
autoridad?'' \footnote{\textbf{21:23} Juan 2,18; Hech 4,7}

\bibleverse{24} Jesús les respondió: ``Yo también os haré una pregunta,
que si me la decís, yo también os diré con qué autoridad hago estas
cosas. \bibleverse{25} El bautismo de Juan, ¿de dónde procede? ¿Del
cielo o de los hombres?'' Razonaban entre sí, diciendo: ``Si decimos:
``Del cielo'', nos preguntará: ``¿Por qué, pues, no le habéis creído?''
\bibleverse{26} Pero si decimos: ``De los hombres'', tememos a la
multitud, porque todos tienen a Juan por profeta.'' \footnote{\textbf{21:26}
  Mat 14,5} \bibleverse{27} Ellos respondieron a Jesús y dijeron: ``No
sabemos''. También les dijo: ``Tampoco os diré con qué autoridad hago
estas cosas.

\hypertarget{la-paruxe1bola-de-los-dos-hijos-desiguales}{%
\subsection{La parábola de los dos hijos
desiguales}\label{la-paruxe1bola-de-los-dos-hijos-desiguales}}

\bibleverse{28} Pero, ¿qué os parece? Un hombre tenía dos hijos y,
acercándose al primero, le dijo: ``Hijo, ve a trabajar hoy en mi viña''.
\bibleverse{29} Él respondió: ``No quiero'', pero después cambió de
opinión y fue. \bibleverse{30} Llegó al segundo y le dijo lo mismo. Él
respondió: `Voy, señor', pero no fue. \footnote{\textbf{21:30} Mat 7,21}
\bibleverse{31} ¿Cuál de los dos hizo la voluntad de su padre?'' Le
dijeron: ``El primero''. Jesús les dijo: ``Os aseguro que los
recaudadores de impuestos y las prostitutas entran en el Reino de Dios
antes que vosotros. \footnote{\textbf{21:31} Luc 18,14} \bibleverse{32}
Porque Juan vino a vosotros por el camino de la justicia, y no le
creísteis; pero los recaudadores de impuestos y las prostitutas le
creyeron. Cuando lo visteis, ni siquiera os arrepentisteis después, para
creerle. \footnote{\textbf{21:32} Luc 7,29}

\hypertarget{la-paruxe1bola-de-los-viticultores-infieles}{%
\subsection{La parábola de los viticultores
infieles}\label{la-paruxe1bola-de-los-viticultores-infieles}}

\bibleverse{33} ``Escuchad otra parábola. Había un hombre que era amo de
casa, que plantó una viña, la rodeó de un seto, cavó en ella un lagar,
construyó una torre, la arrendó a los agricultores y se fue a otro país.
\footnote{\textbf{21:33} Is 5,1-2} \bibleverse{34} Cuando se acercó la
temporada de los frutos, envió a sus siervos a los agricultores para que
recibieran sus frutos. \bibleverse{35} Los campesinos tomaron a sus
siervos, golpearon a uno, mataron a otro y apedrearon a otro.
\bibleverse{36} Volvió a enviar otros siervos más que los primeros, y
los trataron de la misma manera. \bibleverse{37} Pero después les envió
a su hijo, diciendo: ``Respetarán a mi hijo''. \bibleverse{38} Pero los
campesinos, al ver al hijo, dijeron entre sí: `Este es el heredero.
Vamos, matémoslo y apoderémonos de su herencia'. \footnote{\textbf{21:38}
  Mat 26,3-5; Juan 1,11} \bibleverse{39} Así que lo tomaron y lo echaron
de la viña, y luego lo mataron. \bibleverse{40} Por tanto, cuando venga
el señor de la viña, ¿qué hará con esos labradores?''

\bibleverse{41} Le dijeron: ``Destruirá sin misericordia a los malos y
arrendará la viña a otros agricultores que le darán el fruto en su
temporada.''

\bibleverse{42} Jesús les dijo: ``¿Nunca habéis leído en las Escrituras,
La piedra que desecharon los constructores ha venido a ser cabeza de
esquina. Él Señor ha hecho esto. Es maravilloso a nuestros ojos'?
\footnote{\textbf{21:42} Salmo 118:22-23} \footnote{\textbf{21:42} Hech
  4,11; 1Pe 2,4-8}

\bibleverse{43} ``Por eso os digo que el Reino de Dios os será quitado y
será dado a una nación que produzca su fruto. \bibleverse{44} El que
caiga sobre esta piedra se hará pedazos, pero sobre el que caiga, lo
esparcirá como polvo.'' \footnote{\textbf{21:44} Dan 2,34-35; Dan
  2,44-45}

\bibleverse{45} Cuando los jefes de los sacerdotes y los fariseos oyeron
sus parábolas, se dieron cuenta de que hablaba de ellos. \bibleverse{46}
Cuando trataron de apresarlo, temieron a las multitudes, porque lo
consideraban un profeta.

\hypertarget{la-paruxe1bola-de-la-cena-de-bodas-real}{%
\subsection{La parábola de la cena de bodas
real}\label{la-paruxe1bola-de-la-cena-de-bodas-real}}

\hypertarget{section-21}{%
\section{22}\label{section-21}}

\bibleverse{1} Respondiendo Jesús, les habló otra vez en parábolas,
diciendo: \bibleverse{2} El Reino de los Cielos es semejante a cierto
rey que hizo una fiesta de bodas para su hijo, \footnote{\textbf{22:2}
  Juan 3,29} \bibleverse{3} y envió a sus siervos a llamar a los
invitados a la fiesta de bodas, pero no quisieron venir. \bibleverse{4}
Volvió a enviar a otros siervos, diciendo: ``Decid a los invitados: ``He
aquí que he preparado mi cena. Mi ganado y mis animales cebados han sido
sacrificados, y todo está preparado. Venid al banquete de bodas''.
\bibleverse{5} Pero ellos no le dieron importancia y se fueron, uno a su
finca y otro a su mercancía; \bibleverse{6} y los demás agarraron a sus
siervos, los trataron vergonzosamente y los mataron. \footnote{\textbf{22:6}
  Mat 21,35} \bibleverse{7} Cuando el rey se enteró, se enfureció y
envió sus ejércitos, destruyó a esos asesinos y quemó su ciudad.
\footnote{\textbf{22:7} Mat 24,2}

\bibleverse{8} ``Entonces dijo a sus siervos: `Las bodas están
preparadas, pero los invitados no eran dignos. \bibleverse{9} Id, pues,
a los cruces de los caminos y, a cuantos encontréis, invitad al banquete
de bodas.' \footnote{\textbf{22:9} Mat 13,47} \bibleverse{10} Aquellos
servidores salieron a los caminos y reunieron a cuantos encontraron,
tanto malos como buenos. La boda se llenó de invitados.

\bibleverse{11} ``Pero cuando el rey entró a ver a los invitados, vio
allí a un hombre que no tenía puesto el traje de boda, \footnote{\textbf{22:11}
  Apoc 19,8} \bibleverse{12} y le dijo: `Amigo, ¿cómo has entrado aquí
sin llevar el traje de boda?' Se quedó sin palabras. \bibleverse{13}
Entonces el rey dijo a los sirvientes: ``Atadle de pies y manos,
llevadle y echadle a las tinieblas exteriores. Allí será el llanto y el
rechinar de dientes'. \bibleverse{14} Porque muchos son los llamados,
pero pocos los elegidos''.

\hypertarget{la-cuestiuxf3n-fiscal-de-los-fariseos}{%
\subsection{La cuestión fiscal de los
fariseos}\label{la-cuestiuxf3n-fiscal-de-los-fariseos}}

\bibleverse{15} Entonces los fariseos fueron y aconsejaron cómo podrían
atraparlo en su charla. \bibleverse{16} Enviaron a sus discípulos, junto
con los herodianos, diciendo: ``Maestro, sabemos que eres honesto y que
enseñas el camino de Dios con verdad, sin importar a quién enseñes; pues
no eres parcial con nadie. \footnote{\textbf{22:16} Juan 3,2}
\bibleverse{17} Díganos, pues, ¿qué piensa usted? ¿Es lícito pagar
impuestos al César, o no?''

\bibleverse{18} Pero Jesús se dio cuenta de su maldad y les dijo: ``¿Por
qué me ponéis a prueba, hipócritas? \bibleverse{19} Muéstrenme el dinero
de los impuestos''. Le trajeron un denario.

\bibleverse{20} Les preguntó: ``¿De quién es esta imagen y esta
inscripción?''

\bibleverse{21} Le dijeron: ``Del César''. Entonces les dijo: ``Dad al
César lo que es del César y a Dios lo que es de Dios''. \footnote{\textbf{22:21}
  Luc 23,2; Rom 13,7}

\bibleverse{22} Al oírlo, se maravillaron, lo dejaron y se fueron.

\hypertarget{sobre-la-resurrecciuxf3n-de-los-muertos}{%
\subsection{Sobre la resurrección de los
muertos}\label{sobre-la-resurrecciuxf3n-de-los-muertos}}

\bibleverse{23} Aquel día se le acercaron los saduceos (los que dicen
que no hay resurrección). Le preguntaron, \footnote{\textbf{22:23} Hech
  4,2; Hech 23,6; Hech 23,8} \bibleverse{24} diciendo: ``Maestro, Moisés
dijo: ``Si un hombre muere sin tener hijos, su hermano se casará con su
mujer y levantará descendencia\footnote{\textbf{22:24} o, semilla} para
su hermano''. \bibleverse{25} Había entre nosotros siete hermanos. El
primero se casó y murió, y al no tener descendencia dejó su mujer a su
hermano. \bibleverse{26} De la misma manera, el segundo también, y el
tercero, al séptimo. \bibleverse{27} Después de todos ellos, murió la
mujer. \bibleverse{28} En la resurrección, pues, ¿de quién será la mujer
de los siete? Porque todos la tuvieron''.

\bibleverse{29} Pero Jesús les respondió: ``Estáis equivocados, pues no
conocéis las Escrituras ni el poder de Dios. \bibleverse{30} Porque en
la resurrección no se casan ni se dan en matrimonio, sino que son como
los ángeles de Dios en el cielo. \bibleverse{31} Pero en cuanto a la
resurrección de los muertos, ¿no habéis leído lo que os ha dicho Dios,
\bibleverse{32} ``Yo soy el Dios de Abraham, el Dios de Isaac y el Dios
de Jacob''?\footnote{\textbf{22:32} Éxodo 3:6} Dios no es el Dios de los
muertos, sino de los vivos''.

\bibleverse{33} Cuando las multitudes lo oyeron, se asombraron de su
enseñanza.

\hypertarget{la-pregunta-de-un-intuxe9rprete-de-la-ley-sobre-el-mandamiento-muxe1s-noble}{%
\subsection{La pregunta de un intérprete de la ley sobre el mandamiento
más
noble}\label{la-pregunta-de-un-intuxe9rprete-de-la-ley-sobre-el-mandamiento-muxe1s-noble}}

\bibleverse{34} Pero los fariseos, al oír que había hecho callar a los
saduceos, se reunieron. \bibleverse{35} Uno de ellos, un abogado, le
hizo una pregunta para ponerlo a prueba. \bibleverse{36} ``Maestro,
¿cuál es el mayor mandamiento de la ley?''

\bibleverse{37} Jesús le dijo: ``Amarás al Señor tu Dios con todo tu
corazón, con toda tu alma y con toda tu mente''. \footnote{\textbf{22:37}
  Deuteronomio 6:5} \bibleverse{38} Este es el primer y gran
mandamiento. \bibleverse{39} El segundo también es éste: ``Amarás a tu
prójimo como a ti mismo''. \footnote{\textbf{22:39} Levítico 19:18}
\bibleverse{40} Toda la ley y los profetas dependen de estos dos
mandamientos''. \footnote{\textbf{22:40} Rom 13,9-10}

\hypertarget{la-contrapregunta-de-jesuxfas-sobre-el-mesuxedas-como-hijo-de-david}{%
\subsection{La contrapregunta de Jesús sobre el Mesías como hijo de
David}\label{la-contrapregunta-de-jesuxfas-sobre-el-mesuxedas-como-hijo-de-david}}

\bibleverse{41} Mientras los fariseos estaban reunidos, Jesús les hizo
una pregunta, \bibleverse{42} diciendo: ``¿Qué pensáis del Cristo? ¿De
quién es hijo?'' Le dijeron: ``De David''. \footnote{\textbf{22:42} Is
  11,1; Juan 7,42}

\bibleverse{43} Les dijo: ``¿Cómo, pues, David en el Espíritu le llama
Señor, diciendo, \bibleverse{44} `El Señor dijo a mi Señor, siéntate en
mi diestra, hasta que haga de tus enemigos estrado para tus pies'?
\footnote{\textbf{22:44} Salmo 110:1} \footnote{\textbf{22:44} Mat 26,64}

\bibleverse{45} ``Si entonces David lo llama Señor, ¿cómo es su hijo?''

\bibleverse{46} Nadie pudo responderle una palabra, ni nadie se atrevió
a hacerle más preguntas desde aquel día.

\hypertarget{el-gran-discurso-de-castigo-de-jesuxfas-contra-los-escribas-y-fariseos}{%
\subsection{El gran discurso de castigo de Jesús contra los escribas y
fariseos}\label{el-gran-discurso-de-castigo-de-jesuxfas-contra-los-escribas-y-fariseos}}

\hypertarget{section-22}{%
\section{23}\label{section-22}}

\bibleverse{1} Entonces Jesús habló a las multitudes y a sus discípulos,

\hypertarget{reprimenda-por-el-comportamiento-reprobable-de-los-luxedderes-espirituales-del-pueblo-en-su-alto-cargo}{%
\subsection{Reprimenda por el comportamiento reprobable de los líderes
espirituales del pueblo en su alto
cargo}\label{reprimenda-por-el-comportamiento-reprobable-de-los-luxedderes-espirituales-del-pueblo-en-su-alto-cargo}}

\bibleverse{2} diciendo: ``Los escribas y los fariseos se sientan en la
cátedra de Moisés. \bibleverse{3} Por tanto, todo lo que os digan que
observéis, observadlo y hacedlo, pero no hagáis sus obras; porque ellos
dicen y no hacen. \footnote{\textbf{23:3} Mal 2,7-8; Rom 2,21-23}
\bibleverse{4} Porque atan cargas pesadas y difíciles de llevar, y las
ponen sobre los hombros de los hombres; pero ellos mismos no mueven un
dedo para ayudarlos. \footnote{\textbf{23:4} Mat 11,28-30; Hech 15,10;
  Hech 15,28} \bibleverse{5} Pero hacen todas sus obras para ser vistos
por los hombres. Ensanchan sus filacterias\footnote{\textbf{23:5} NU
  omite el segundo ``Rabí''.} y agrandan los flecos de sus vestidos,
\footnote{\textbf{23:5} Mat 6,1; Éxod 13,9; Núm 15,38-39} \bibleverse{6}
y aman el lugar de honor en las fiestas, los mejores asientos en las
sinagogas, \footnote{\textbf{23:6} Luc 14,7} \bibleverse{7} las
salutaciones en las plazas, y que los hombres los llamen ``Rabí, Rabí''.
\bibleverse{8} Pero a vosotros no se os debe llamar ``Rabí'', porque uno
es vuestro maestro, el Cristo, y todos vosotros sois hermanos.
\bibleverse{9} No llaméis padre a nadie en la tierra, porque uno es
vuestro Padre, el que está en el cielo. \bibleverse{10} Ni os llaméis
maestros, porque uno es vuestro maestro, el Cristo. \bibleverse{11} Pero
el que es más grande entre vosotros será vuestro servidor. \footnote{\textbf{23:11}
  Mat 20,26-27} \bibleverse{12} El que se enaltece será humillado, y el
que se humilla será enaltecido. \footnote{\textbf{23:12} Prov 29,23; Job
  22,29; Ezeq 21,26; Luc 18,14; 1Pe 5,5}

\hypertarget{los-siete-ayes-de-los-escribas-y-fariseos}{%
\subsection{Los siete ayes de los escribas y
fariseos}\label{los-siete-ayes-de-los-escribas-y-fariseos}}

\bibleverse{13} ``¡Ay de vosotros, escribas y fariseos, hipócritas!
Porque devoráis las casas de las viudas, y como pretexto hacéis largas
oraciones. Por eso recibiréis mayor condena.

\bibleverse{14} ``Pero ¡ay de vosotros, escribas y fariseos, hipócritas!
porque cerráis el Reino de los Cielos a los hombres; porque no entráis
vosotros mismos, ni dejáis entrar a los que están entrando. \footnote{\textbf{23:14}
  TR lee ``autoindulgencia'' en lugar de ``injusticia''} \footnote{\textbf{23:14}
  Ezeq 22,25} \bibleverse{15} ¡Ay de vosotros, escribas y fariseos,
hipócritas! Porque recorréis mar y tierra para hacer un prosélito; y
cuando lo es, lo hacéis dos veces más hijo de la Gehena que vosotros.

\bibleverse{16} ``¡Ay de vosotros, guías ciegos, que decís: `Quien jura
por el templo, no es nada; pero quien jura por el oro del templo, está
obligado'! \footnote{\textbf{23:16} Mat 15,4; Mat 5,34-37}
\bibleverse{17} ¡Necios ciegos! Porque, ¿qué es más grande, el oro o el
templo que santifica el oro? \bibleverse{18} Y el que jura por el altar,
no es nada; pero el que jura por la ofrenda que está sobre él, está
obligado. \bibleverse{19} ¡Ustedes tontos ciegos! Porque, ¿qué es más
grande, el don o el altar que santifica el don? \footnote{\textbf{23:19}
  Éxod 29,37} \bibleverse{20} El que jura por el altar, jura por él y
por todo lo que hay en él. \bibleverse{21} El que jura por el templo,
jura por él y por el que ha vivido en él. \bibleverse{22} El que jura
por el cielo, jura por el trono de Dios y por el que está sentado en él.

\bibleverse{23} ``¡Ay de vosotros, escribas y fariseos, hipócritas!
Porque diezmáis la menta, el eneldo y el comino, y habéis dejado de
hacer las cosas más importantes de la ley: la justicia, la misericordia
y la fe. Pero deberíais haber hecho éstas, y no haber dejado de hacer
las otras. \footnote{\textbf{23:23} Lev 27,30; Miq 6,8; Luc 18,12}
\bibleverse{24} ¡Guías ciegos, que coláis un mosquito y os tragáis un
camello!

\bibleverse{25} ``¡Ay de vosotros, escribas y fariseos, hipócritas!
Porque limpiáis el exterior de la copa y del plato, pero por dentro
están llenos de extorsión e injusticia. \footnote{\textbf{23:25} Mar
  7,4; Mar 7,8} \bibleverse{26} Fariseo ciego, limpia primero el
interior de la copa y del plato, para que también se limpie su exterior.
\footnote{\textbf{23:26} Juan 9,40; Tit 1,15}

\bibleverse{27} ``¡Ay de vosotros, escribas y fariseos, hipócritas!
Porque sois como sepulcros blanqueados, que por fuera parecen hermosos,
pero por dentro están llenos de huesos de muertos y de toda inmundicia.
\bibleverse{28} Así también vosotros por fuera parecéis justos a los
hombres, pero por dentro estáis llenos de hipocresía e iniquidad.

\bibleverse{29} ``¡Ay de vosotros, escribas y fariseos, hipócritas!
Porque edificáis los sepulcros de los profetas y adornáis los sepulcros
de los justos, \bibleverse{30} y decís: `Si hubiéramos vivido en los
días de nuestros padres, no habríamos participado con ellos en la sangre
de los profetas'. \bibleverse{31} Por lo tanto, vosotros mismos
atestiguáis que sois hijos de los que mataron a los profetas.
\footnote{\textbf{23:31} Mat 5,12; Hech 7,52} \bibleverse{32} Llenad,
pues, la medida de vuestros padres. \bibleverse{33} Vosotros,
serpientes, descendientes de víboras, ¿cómo podréis escapar del juicio
de la Gehena? \footnote{\textbf{23:33} o, Infierno}

\hypertarget{amenaza-contra-las-personas-manchadas-de-sangre-que-se-resisten-a-su-salvaciuxf3n}{%
\subsection{Amenaza contra las personas manchadas de sangre que se
resisten a su
salvación}\label{amenaza-contra-las-personas-manchadas-de-sangre-que-se-resisten-a-su-salvaciuxf3n}}

\bibleverse{34} Por tanto, he aquí que yo os envío profetas, sabios y
escribas. A algunos de ellos los mataréis y crucificaréis, y a otros los
azotaréis en vuestras sinagogas y los perseguiréis de ciudad en ciudad,
\bibleverse{35} para que caiga sobre vosotros toda la sangre justa
derramada en la tierra, desde la sangre del justo Abel hasta la sangre
de Zacarías hijo de Baracía, a quien matasteis entre el santuario y el
altar. \footnote{\textbf{23:35} Gén 4,8; 2Cró 24,20-21} \bibleverse{36}
De cierto os digo que todas estas cosas vendrán sobre esta generación.

\hypertarget{salida-de-jesuxfas-de-la-ciudad-de-jerusaluxe9n-y-anuncio-de-su-regreso}{%
\subsection{Salida de Jesús de la ciudad de Jerusalén y anuncio de su
regreso}\label{salida-de-jesuxfas-de-la-ciudad-de-jerusaluxe9n-y-anuncio-de-su-regreso}}

\bibleverse{37} ``¡Jerusalén, Jerusalén, que matas a los profetas y
apedreas a los que te son envíados ¡Cuántas veces quise reunir a tus
hijos, como la gallina reúne a sus polluelos bajo sus alas, y no
quisiste! \bibleverse{38} He aquí que tu casa te ha quedado desolada.
\footnote{\textbf{23:38} 1Re 9,7-8} \bibleverse{39} Porque os digo que
desde ahora no me veréis, hasta que digáis: ``¡Bendito el que viene en
nombre del Señor!'' \footnote{\textbf{23:39} Salmo 118:26} \footnote{\textbf{23:39}
  Mat 21,9; Mat 26,64}

\hypertarget{el-monte-de-los-olivos-de-jesuxfas-a-sus-discuxedpulos-desde-la-destrucciuxf3n-del-templo-desde-el-fin-de-este-mundo-desde-su-segunda-venida-y-desde-el-juicio-sobre-los-pueblos}{%
\subsection{El monte de los Olivos de Jesús a sus discípulos desde la
destrucción del templo, desde el fin de este mundo, desde su segunda
venida y desde el juicio sobre los
pueblos}\label{el-monte-de-los-olivos-de-jesuxfas-a-sus-discuxedpulos-desde-la-destrucciuxf3n-del-templo-desde-el-fin-de-este-mundo-desde-su-segunda-venida-y-desde-el-juicio-sobre-los-pueblos}}

\hypertarget{section-23}{%
\section{24}\label{section-23}}

\bibleverse{1} Jesús salió del templo y siguió su camino. Sus discípulos
se acercaron a él para mostrarle los edificios del templo.
\bibleverse{2} Pero él les respondió: ``¿Veis todo esto, verdad? Os
aseguro que no quedará aquí una piedra sobre otra que no sea
derribada''. \footnote{\textbf{24:2} Luc 19,44}

\bibleverse{3} Mientras estaba sentado en el Monte de los Olivos, los
discípulos se acercaron a él en privado, diciendo: ``Dinos, ¿cuándo
serán estas cosas? ¿Cuál es la señal de tu venida y del fin de los
tiempos?'' \footnote{\textbf{24:3} Hech 1,6-8}

\hypertarget{el-fin-de-este-tiempo-mundial-las-primeras-seuxf1ales}{%
\subsection{El fin de este tiempo mundial; Las primeras
señales}\label{el-fin-de-este-tiempo-mundial-las-primeras-seuxf1ales}}

\bibleverse{4} Jesús les contestó: ``Tened cuidado de que nadie os
engañe. \bibleverse{5} Porque vendrán muchos en mi nombre, diciendo:
``Yo soy el Cristo'', y engañarán a muchos. \footnote{\textbf{24:5} Juan
  5,43; 1Jn 2,18} \bibleverse{6} Oiréis hablar de guerras y rumores de
guerras. Mirad que no os turbéis, porque es necesario que todo esto
ocurra, pero aún no es el fin. \bibleverse{7} Porque se levantará nación
contra nación, y reino contra reino; y habrá hambres, plagas y
terremotos en diversos lugares. \bibleverse{8} Pero todas estas cosas
son el principio de los dolores de parto.

\hypertarget{las-persecuciones-de-los-discuxedpulos}{%
\subsection{Las persecuciones de los
discípulos}\label{las-persecuciones-de-los-discuxedpulos}}

\bibleverse{9} ``Entonces te entregarán a la tribulación y te matarán.
Seréis odiados por todas las naciones por causa de mi nombre.
\footnote{\textbf{24:9} Juan 16,2; Mat 10,17-22} \bibleverse{10}
Entonces muchos tropezarán, se entregarán unos a otros y se odiarán.
\bibleverse{11} Se levantarán muchos falsos profetas y llevarán a muchos
por el mal camino. \footnote{\textbf{24:11} 2Pe 2,1; 1Jn 4,1}
\bibleverse{12} Porque se multiplicará la iniquidad, el amor de muchos
se enfriará. \footnote{\textbf{24:12} 2Tim 3,1-5} \bibleverse{13} Pero
el que aguante hasta el final se salvará. \footnote{\textbf{24:13} Apoc
  13,10} \bibleverse{14} Esta Buena Nueva del Reino será predicada en
todo el mundo para testimonio de todas las naciones, y entonces vendrá
el fin. \footnote{\textbf{24:14} Mat 28,19}

\hypertarget{el-cluxedmax-de-la-tribulaciuxf3n-en-judea}{%
\subsection{El clímax de la tribulación en
Judea}\label{el-cluxedmax-de-la-tribulaciuxf3n-en-judea}}

\bibleverse{15} ``Por tanto, cuando veáis la abominación de la
desolación,\footnote{\textbf{24:15} Daniel 9:27; 11:31; 12:11} de la que
se habló por medio del profeta Daniel, de pie en el lugar santo (que el
lector entienda), \bibleverse{16} entonces los que estén en Judea huyan
a las montañas. \bibleverse{17} Que el que esté en la azotea no baje a
sacar las cosas que están en su casa. \footnote{\textbf{24:17} Luc 17,31}
\bibleverse{18} Que el que esté en el campo no regrese a buscar su ropa.
\bibleverse{19} Pero ¡ay de las que estén embarazadas y de las madres
lactantes en esos días! \footnote{\textbf{24:19} Luc 23,29}
\bibleverse{20} Rogad que vuestra huida no sea en invierno ni en sábado,
\bibleverse{21} porque entonces habrá un gran sufrimiento,\footnote{\textbf{24:21}
  o, opresión} como no lo ha habido desde el principio del mundo hasta
ahora, ni lo habrá jamás. \footnote{\textbf{24:21} Dan 12,1}
\bibleverse{22} Si no se acortaran esos días, ninguna carne se habría
salvado. Pero por el bien de los elegidos, esos días serán acortados.

\hypertarget{profecuxeda-de-los-falsos-profetas}{%
\subsection{Profecía de los falsos
profetas}\label{profecuxeda-de-los-falsos-profetas}}

\bibleverse{23} ``Entonces, si alguien les dice: `He aquí el Cristo' o
`Allí', no lo crean. \bibleverse{24} Porque se levantarán falsos cristos
y falsos profetas, y harán grandes señales y prodigios, para extraviar,
si es posible, incluso a los elegidos. \footnote{\textbf{24:24} Deut
  13,1-3; 2Tes 2,8-9; Apoc 13,13}

\bibleverse{25} ``He aquí que os lo he dicho de antemano.

\bibleverse{26} ``Por tanto, si os dicen: `He aquí que está en el
desierto', no salgáis; o `He aquí que está en las habitaciones
interiores', no lo creáis. \bibleverse{27} Porque como el rayo que sale
del oriente y se ve hasta el occidente, así será la venida del Hijo del
Hombre. \footnote{\textbf{24:27} Luc 17,23-24} \bibleverse{28} Porque
donde está el cadáver, allí se reúnen los buitres.\footnote{\textbf{24:28}
  o, águilas} \footnote{\textbf{24:28} Job 39,30; Luc 17,37; Apoc
  19,17-18}

\hypertarget{los-uxfaltimos-augurios-y-la-segunda-venida-del-hijo-del-hombre-con-los-fenuxf3menos-que-los-acompauxf1an}{%
\subsection{Los últimos augurios y la segunda venida del Hijo del Hombre
con los fenómenos que los
acompañan}\label{los-uxfaltimos-augurios-y-la-segunda-venida-del-hijo-del-hombre-con-los-fenuxf3menos-que-los-acompauxf1an}}

\bibleverse{29} ``Pero inmediatamente después la tribulación\footnote{\textbf{24:29}
  o, opresión} de esos días, el sol se oscurecerá, la luna no dará su
luz, las estrellas caerán del cielo y las potencias de los cielos serán
sacudidas; \footnote{\textbf{24:29} Isaías 13:10; 34:4} \footnote{\textbf{24:29}
  Is 13,10; 2Pe 3,10; Apoc 6,12-13} \bibleverse{30} y entonces aparecerá
en el cielo la señal del Hijo del Hombre. Entonces todas las tribus de
la tierra se lamentarán, y verán al Hijo del Hombre venir sobre las
nubes del cielo con poder y gran gloria. \footnote{\textbf{24:30} Mat
  26,64; Dan 7,13-14; Apoc 1,7; Apoc 19,11-13} \bibleverse{31} Enviará a
sus ángeles con gran sonido de trompeta, y reunirán a sus elegidos de
los cuatro vientos, desde un extremo del cielo hasta el otro.
\footnote{\textbf{24:31} 1Cor 15,52; Apoc 8,1-2}

\bibleverse{32} ``Aprended ahora de la higuera esta parábola: Cuando su
rama ya está tierna y produce sus hojas, sabéis que el verano está
cerca. \bibleverse{33} Así también vosotros, cuando veáis todas estas
cosas, sabed que está cerca, incluso a las puertas. \bibleverse{34} De
cierto os digo que no pasará esta generación\footnote{\textbf{24:34} La
  palabra ``generación'' (genea) también puede traducirse como ``raza''.}
hasta que se cumplan todas estas cosas. \bibleverse{35} El cielo y la
tierra pasarán, pero mis palabras no pasarán. \footnote{\textbf{24:35}
  Mat 5,18}

\bibleverse{36} ``Pero nadie sabe de ese día y de esa hora, ni siquiera
los ángeles del cielo, \footnote{\textbf{24:36} NU añade ``ni el hijo''}
sino sólo mi Padre. \footnote{\textbf{24:36} Hech 1,7} \bibleverse{37}
Como los días de Noé, así será la venida del Hijo del Hombre.
\footnote{\textbf{24:37} Gén 6,11-13; Luc 17,26-27} \bibleverse{38}
Porque como en los días anteriores al diluvio estaban comiendo y
bebiendo, casándose y dando en matrimonio, hasta el día en que Noé entró
en la nave, \bibleverse{39} y no lo supieron hasta que vino el diluvio y
se los llevó a todos, así será la venida del Hijo del Hombre.
\bibleverse{40} Entonces dos hombres estarán en el campo: uno será
tomado y otro será dejado. \footnote{\textbf{24:40} Luc 17,35-36}
\bibleverse{41} Dos mujeres estarán moliendo en el molino: una será
tomada y la otra será dejada.

\hypertarget{advertencia-de-estar-alerta-en-general}{%
\subsection{Advertencia de estar alerta en
general}\label{advertencia-de-estar-alerta-en-general}}

\bibleverse{42} Velad, pues, porque no sabéis a qué hora vendrá vuestro
Señor. \footnote{\textbf{24:42} Mat 25,13} \bibleverse{43} Pero sabed
esto, que si el dueño de la casa hubiera sabido a qué hora de la noche
iba a venir el ladrón, habría velado y no habría permitido que entraran
en su casa. \bibleverse{44} Por tanto, estad también preparados, porque
a una hora que no esperáis, vendrá el Hijo del Hombre. \footnote{\textbf{24:44}
  1Tes 5,2}

\hypertarget{paruxe1bola-del-siervo-fiel-y-del-infiel}{%
\subsection{Parábola del siervo fiel y del
infiel}\label{paruxe1bola-del-siervo-fiel-y-del-infiel}}

\bibleverse{45} ``¿Quién es, pues, el siervo fiel y prudente al que su
señor ha puesto al frente de su casa para que les dé el alimento a su
debido tiempo? \bibleverse{46} Dichoso aquel siervo al que su señor
encuentre haciendo eso cuando venga. \bibleverse{47} Ciertamente os digo
que lo pondrá sobre todo lo que tiene. \footnote{\textbf{24:47} Mat
  25,21; Mat 25,23} \bibleverse{48} Pero si ese siervo malo dice en su
corazón: ``Mi señor se demora en venir'', \footnote{\textbf{24:48} 2Pe
  3,4} \bibleverse{49} y comienza a golpear a sus consiervos, y a comer
y beber con los borrachos, \bibleverse{50} el señor de ese siervo vendrá
en un día en que no lo espera y en una hora en que no lo sabe,
\bibleverse{51} y lo despedazará y pondrá su parte con los hipócritas.
Allí será el llanto y el rechinar de dientes.

\hypertarget{la-paruxe1bola-de-las-diez-vuxedrgenes-prudentes-y-necias}{%
\subsection{La parábola de las diez vírgenes prudentes y
necias}\label{la-paruxe1bola-de-las-diez-vuxedrgenes-prudentes-y-necias}}

\hypertarget{section-24}{%
\section{25}\label{section-24}}

\bibleverse{1} ``Entonces el Reino de los Cielos será como diez vírgenes
que, tomando sus lámparas, salieron a recibir al novio. \footnote{\textbf{25:1}
  Luc 12,35-36; Apoc 19,7} \bibleverse{2} Cinco de ellas eran insensatas
y cinco prudentes. \bibleverse{3} Las insensatas, al tomar sus lámparas,
no tomaron aceite con ellas, \bibleverse{4} pero las prudentes tomaron
aceite en sus vasos con sus lámparas. \bibleverse{5} Mientras el novio
se demoraba, todas adormecieron y se quedaron dormidas. \bibleverse{6}
Pero a medianoche se oyó un grito: ``¡Mira! ¡Viene el novio! Salid a
recibirlo''. \bibleverse{7} Entonces todas aquellas vírgenes se
levantaron y arreglaron sus lámparas. \footnote{\textbf{25:7} El extremo
  de la mecha de una lámpara de aceite debe cortarse periódicamente para
  evitar que se obstruya con depósitos de carbón. La altura de la mecha
  también se ajusta para que la llama arda uniformemente y dé buena luz
  sin producir mucho humo.} \bibleverse{8} Las insensatas dijeron a las
prudentes: ``Dadnos un poco de vuestro aceite, porque nuestras lámparas
se apagan''. \bibleverse{9} Pero las prudentes respondieron diciendo:
``¿Y si no hay suficiente para nosotras y para vosotras? Id más bien a
los que venden y comprad para vosotros'. \bibleverse{10} Mientras ellas
iban a comprar, llegó el novio, y las que estaban preparadas entraron
con él al banquete de bodas, y se cerró la puerta. \bibleverse{11}
Después vinieron también las otras vírgenes, diciendo: ``Señor, Señor,
ábrenos''. \footnote{\textbf{25:11} Luc 13,25; Luc 13,27}
\bibleverse{12} Pero él les respondió: ``Os aseguro que no os conozco''.
\footnote{\textbf{25:12} Mat 7,23} \bibleverse{13} Velad, pues, porque
no sabéis el día ni la hora en que vendrá el Hijo del Hombre.
\footnote{\textbf{25:13} Mat 24,42}

\hypertarget{paruxe1bola-de-los-talentos-confiados}{%
\subsection{Parábola de los talentos
confiados}\label{paruxe1bola-de-los-talentos-confiados}}

\bibleverse{14} ``Pues es como un hombre que, al ir a otro país, llamó a
sus propios siervos y les confió sus bienes. \bibleverse{15} A uno le
dio cinco talentos,\footnote{\textbf{25:15} Un talento equivale a unos
  30 kilogramos o 66 libras (normalmente se utiliza para pesar la plata,
  a menos que se especifique lo contrario)} a otro dos, a otro uno, a
cada uno según su capacidad. Luego siguió su camino. \footnote{\textbf{25:15}
  Rom 12,6} \bibleverse{16} Enseguida, el que recibió los cinco talentos
fue a comerciar con ellos y ganó otros cinco talentos. \bibleverse{17}
De la misma manera, el que recibió los dos ganó otros dos.
\bibleverse{18} Pero el que recibió el único talento se fue, cavó en la
tierra y escondió el dinero de su señor.

\bibleverse{19} ``Después de mucho tiempo, vino el señor de aquellos
siervos y ajustó cuentas con ellos. \bibleverse{20} El que recibió los
cinco talentos vino y trajo otros cinco talentos, diciendo: `Señor, me
entregaste cinco talentos. He aquí que he ganado otros cinco talentos
además de ellos'.

\bibleverse{21} ``Su señor le dijo: `Bien hecho, siervo bueno y fiel.
Has sido fiel en pocas cosas, yo te pondré al frente de muchas. Entra en
la alegría de tu señor'. \footnote{\textbf{25:21} Mat 24,45-47}

\bibleverse{22} ``También el que recibió los dos talentos se acercó y
dijo: `Señor, me entregaste dos talentos. He aquí que he ganado otros
dos talentos además de ellos'.

\bibleverse{23} ``Su señor le dijo: `Bien hecho, siervo bueno y fiel.
Has sido fiel en algunas cosas. Yo te pondré al frente de muchas cosas.
Entra en la alegría de tu señor'.

\bibleverse{24} ``También el que había recibido el único talento se
acercó y dijo: ``Señor, te conozco que eres un hombre duro, que cosechas
donde no sembraste y recoges donde no esparciste. \bibleverse{25} Tuve
miedo, me fui y escondí tu talento en la tierra. He aquí que tienes lo
que es tuyo'.

\bibleverse{26} ``Pero su señor le respondió: `Siervo malo y perezoso.
Sabías que cosecho donde no sembré, y recojo donde no esparcí.
\bibleverse{27} Por lo tanto, deberías haber depositado mi dinero en los
banqueros, y a mi llegada debería haber recibido lo mío con intereses.
\bibleverse{28} Quítale, pues, el talento y dáselo al que tiene los diez
talentos. \bibleverse{29} Porque a todo el que tiene se le dará y tendrá
en abundancia, pero al que no tiene se le quitará hasta lo que tiene.
\footnote{\textbf{25:29} Mat 13,12} \bibleverse{30} Echad al siervo
inútil a las tinieblas exteriores, donde habrá llanto y crujir de
dientes''.

\hypertarget{el-juicio-de-jesuxfas-sobre-los-pueblos-y-las-personas-la-separaciuxf3n-de-las-ovejas-de-las-cabras}{%
\subsection{El juicio de Jesús sobre los pueblos y las personas; la
separación de las ovejas de las
cabras}\label{el-juicio-de-jesuxfas-sobre-los-pueblos-y-las-personas-la-separaciuxf3n-de-las-ovejas-de-las-cabras}}

\bibleverse{31} ``Pero cuando el Hijo del Hombre venga en su gloria, y
todos los santos ángeles con él, se sentará en el trono de su gloria.
\footnote{\textbf{25:31} Mat 16,27; Apoc 20,11-13} \bibleverse{32} Ante
él se reunirán todas las naciones, y las separará unas de otras, como el
pastor separa las ovejas de los cabritos. \footnote{\textbf{25:32} Mat
  13,49; Rom 14,10} \bibleverse{33} Pondrá las ovejas a su derecha, pero
los cabritos a la izquierda. \footnote{\textbf{25:33} Ezeq 34,17}
\bibleverse{34} Entonces el Rey dirá a los de su derecha: ``Venid,
benditos de mi Padre, heredad el Reino preparado para vosotros desde la
fundación del mundo; \bibleverse{35} porque tuve hambre y me disteis de
comer. Tuve sed y me disteis de beber. Fui forastero y me acogisteis.
\footnote{\textbf{25:35} Is 58,7} \bibleverse{36} Estuve desnudo y me
vestisteis. Estuve enfermo y me visitasteis. Estuve en la cárcel y
vinisteis a verme''.

\bibleverse{37} ``Entonces los justos le responderán diciendo: ``Señor,
¿cuándo te vimos hambriento y te dimos de comer, o sediento y te dimos
de beber? \footnote{\textbf{25:37} Mat 6,3} \bibleverse{38} ¿Cuándo te
vimos como forastero y te acogimos, o desnudo y te vestimos?
\bibleverse{39} ¿Cuándo te vimos enfermo o en la cárcel y acudimos a ti?

\bibleverse{40} ``El Rey les responderá: `Os aseguro que porque lo
hicisteis con uno de estos mis hermanos más pequeños, conmigo lo
hicisteis'. \footnote{\textbf{25:40} Mat 10,42; Prov 19,17; Heb 2,11}
\bibleverse{41} Entonces dirá también a los de la izquierda: `Apartaos
de mí, malditos, al fuego eterno que está preparado para el diablo y sus
ángeles; \footnote{\textbf{25:41} Apoc 20,10; Apoc 20,15}
\bibleverse{42} porque tuve hambre, y no me disteis de comer; tuve sed,
y no me disteis de beber; \bibleverse{43} fui forastero, y no me
acogisteis; estuve desnudo, y no me vestisteis; enfermo, y en la cárcel,
y no me visitasteis.'

\bibleverse{44} ``Entonces también responderán diciendo: `Señor, ¿cuándo
te vimos hambriento, o sediento, o forastero, o desnudo, o enfermo, o en
la cárcel, y no te ayudamos?

\bibleverse{45} ``Entonces les responderá diciendo: ``Os aseguro que
porque no lo hicisteis con uno de estos más pequeños, no lo hicisteis
conmigo. \bibleverse{46} Estos irán al castigo eterno, pero los justos a
la vida eterna.'' \footnote{\textbf{25:46} Juan 5,29; Sant 2,13}

\hypertarget{uxfaltimo-anuncio-del-sufrimiento-de-jesuxfas-intento-de-asesinato-por-parte-de-los-luxedderes-del-pueblo}{%
\subsection{Último anuncio del sufrimiento de Jesús; Intento de
asesinato por parte de los líderes del
pueblo}\label{uxfaltimo-anuncio-del-sufrimiento-de-jesuxfas-intento-de-asesinato-por-parte-de-los-luxedderes-del-pueblo}}

\hypertarget{section-25}{%
\section{26}\label{section-25}}

\bibleverse{1} Cuando Jesús terminó todas estas palabras, dijo a sus
discípulos: \bibleverse{2} ``Sabéis que dentro de dos días viene la
Pascua, y el Hijo del Hombre será entregado para ser crucificado.''
\footnote{\textbf{26:2} Mat 20,18; Éxod 12,1-20}

\bibleverse{3} Entonces los jefes de los sacerdotes, los escribas y los
ancianos del pueblo se reunieron en el patio del sumo sacerdote, que se
llamaba Caifás. \footnote{\textbf{26:3} Luc 3,1-2} \bibleverse{4} Se
pusieron de acuerdo para prender a Jesús con engaño y matarlo.
\bibleverse{5} Pero dijeron: ``No durante la fiesta, para que no se
produzca un motín en el pueblo''.

\hypertarget{unciuxf3n-de-jesuxfas-en-betania}{%
\subsection{Unción de Jesús en
Betania}\label{unciuxf3n-de-jesuxfas-en-betania}}

\bibleverse{6} Estando Jesús en Betania, en casa de Simón el leproso,
\bibleverse{7} se le acercó una mujer con un frasco de alabastro de
ungüento muy caro, y se lo derramó sobre la cabeza mientras estaba
sentado a la mesa. \bibleverse{8} Al ver esto, sus discípulos se
indignaron diciendo: ``¿Por qué este derroche? \bibleverse{9} Porque
este ungüento podría haberse vendido por mucho y haberse dado a los
pobres''.

\bibleverse{10} Sin embargo, sabiendo esto, Jesús les dijo: ``¿Por qué
molestáis a la mujer? Ella ha hecho una buena obra para mí.
\bibleverse{11} Porque siempre tenéis a los pobres con vosotros, pero a
mí no me tenéis siempre. \footnote{\textbf{26:11} Deut 15,11}
\bibleverse{12} Porque al derramar este ungüento sobre mi cuerpo, lo
hizo para prepararme para la sepultura. \bibleverse{13} Os aseguro que
dondequiera que se predique esta Buena Noticia en todo el mundo, también
se hablará de lo que ha hecho esta mujer como un recuerdo de ella.''

\hypertarget{traiciuxf3n-de-judas}{%
\subsection{Traición de Judas}\label{traiciuxf3n-de-judas}}

\bibleverse{14} Entonces uno de los doce, que se llamaba Judas
Iscariote, fue a los jefes de los sacerdotes \bibleverse{15} y les dijo:
``¿Cuánto estáis dispuestos a darme si os lo entrego?'' Y le pesaron
treinta monedas de plata. \footnote{\textbf{26:15} Juan 11,57; Zac 11,12}
\bibleverse{16} Desde entonces buscó la oportunidad de traicionarlo.

\hypertarget{preparaciuxf3n-de-la-comida-pascual}{%
\subsection{Preparación de la comida
pascual}\label{preparaciuxf3n-de-la-comida-pascual}}

\bibleverse{17} El primer día de los panes sin levadura, los discípulos
se acercaron a Jesús y le dijeron: ``¿Dónde quieres que te preparemos
para comer la Pascua?'' \footnote{\textbf{26:17} Éxod 12,18-20}

\bibleverse{18} Dijo: ``Ve a la ciudad a cierta persona y dile: ``El
Maestro dice: ``Se acerca mi hora. Celebraré la Pascua en tu casa con
mis discípulos''\,''. \footnote{\textbf{26:18} Mat 21,3}

\bibleverse{19} Los discípulos hicieron lo que Jesús les mandó y
prepararon la Pascua.

\hypertarget{la-uxfaltima-cena-de-jesuxfas-en-el-cuxedrculo-de-los-discuxedpulos-exposiciuxf3n-de-la-traiciuxf3n-de-judas-instituciuxf3n-de-la-santa-comuniuxf3n}{%
\subsection{La última cena de Jesús en el círculo de los discípulos;
Exposición de la traición de Judas; Institución de la santa
comunión}\label{la-uxfaltima-cena-de-jesuxfas-en-el-cuxedrculo-de-los-discuxedpulos-exposiciuxf3n-de-la-traiciuxf3n-de-judas-instituciuxf3n-de-la-santa-comuniuxf3n}}

\bibleverse{20} Cuando llegó la noche, estaba sentado a la mesa con los
doce discípulos. \bibleverse{21} Mientras comían, dijo: ``Os aseguro que
uno de vosotros me va a traicionar''.

\bibleverse{22} Estaban muy apenados y cada uno comenzó a preguntarle:
``No soy yo, ¿verdad, Señor?''.

\bibleverse{23} Él respondió: ``El que mojó su mano conmigo en el plato
me entregará. \bibleverse{24} El Hijo del Hombre va como está escrito de
él, pero ¡ay de aquel hombre por el que el Hijo del Hombre es entregado!
Más le valdría a ese hombre no haber nacido''. \footnote{\textbf{26:24}
  Luc 17,1}

\bibleverse{25} Judas, el que lo traicionó, respondió: ``No soy yo,
¿verdad, rabino?'' Le dijo: ``Tú lo has dicho''.

\bibleverse{26} Mientras comían, Jesús tomó el pan, dio gracias
por\footnote{\textbf{26:26} TR lee ``bendecido'' en lugar de ``dio
  gracias por''} él y lo partió. Se lo dio a los discípulos y les dijo:
``Tomad, comed; esto es mi cuerpo''. \footnote{\textbf{26:26} 1Cor
  10,16; 1Cor 11,23-25} \bibleverse{27} Tomó la copa, dio gracias y se
la dio a ellos, diciendo: ``Bebed todos de ella, \bibleverse{28} porque
ésta es mi sangre de la nueva alianza, que se derrama por muchos para la
remisión de los pecados. \footnote{\textbf{26:28} Éxod 24,8; Jer 31,31;
  Heb 9,15-16} \bibleverse{29} Pero os digo que desde ahora no beberé de
este fruto de la vid, hasta aquel día en que lo beba de nuevo con
vosotros en el Reino de mi Padre.''

\hypertarget{camina-a-getsemanuxed}{%
\subsection{Camina a Getsemaní}\label{camina-a-getsemanuxed}}

\bibleverse{30} Cuando cantaron un himno, salieron al Monte de los
Olivos. \footnote{\textbf{26:30} Sal 113,1-118}

\bibleverse{31} Entonces Jesús les dijo: ``Esta noche todos vosotros
tropezaréis por mi causa, porque está escrito: ``Heriré al pastor, y las
ovejas del rebaño se dispersarán. \footnote{\textbf{26:31} 26:31
  Zacarías 13:7} \footnote{\textbf{26:31} Juan 16,32} \bibleverse{32}
Pero cuando haya resucitado, iré delante de vosotros a Galilea''.
\footnote{\textbf{26:32} Mat 28,7}

\bibleverse{33} Pero Pedro le contestó: ``Aunque todos tropiecen por tu
culpa, yo no tropezaré jamás''.

\bibleverse{34} Jesús le dijo: ``Te aseguro que esta noche, antes de que
cante el gallo, me negarás tres veces''. \footnote{\textbf{26:34} Juan
  13,18}

\bibleverse{35} Pedro le dijo: ``Aunque tenga que morir contigo, no te
negaré''. Todos los discípulos también dijeron lo mismo.

\hypertarget{el-conflicto-y-la-oraciuxf3n-de-jesuxfas-en-getsemanuxed-debilidad-de-los-discuxedpulos}{%
\subsection{El conflicto y la oración de Jesús en Getsemaní; Debilidad
de los
discípulos}\label{el-conflicto-y-la-oraciuxf3n-de-jesuxfas-en-getsemanuxed-debilidad-de-los-discuxedpulos}}

\bibleverse{36} Entonces Jesús vino con ellos a un lugar llamado
Getsemaní, y dijo a sus discípulos: ``Sentaos aquí, mientras voy allí a
orar.'' \bibleverse{37} Tomó consigo a Pedro y a los dos hijos de
Zebedeo, y comenzó a entristecerse y a angustiarse gravemente.
\footnote{\textbf{26:37} Mat 17,1; Heb 5,7} \bibleverse{38} Entonces les
dijo: ``Mi alma está muy triste, hasta la muerte. Quedaos aquí y velad
conmigo''. \footnote{\textbf{26:38} Juan 12,27}

\bibleverse{39} Se adelantó un poco, se postró sobre su rostro y oró
diciendo: ``Padre mío, si es posible, haz que pase de mí esta copa; pero
no lo que yo quiero, sino lo que tú quieres.'' \footnote{\textbf{26:39}
  Juan 6,38; Juan 18,11; Heb 5,8}

\bibleverse{40} Vino a los discípulos y los encontró durmiendo, y dijo a
Pedro: ``¿Qué, no habéis podido velar conmigo una hora? \bibleverse{41}
Velad y orad, para que no entréis en tentación. El espíritu, en efecto,
está dispuesto, pero la carne es débil''. \footnote{\textbf{26:41} Efes
  6,18; Heb 2,18}

\bibleverse{42} Otra vez se fue y oró diciendo: ``Padre mío, si esta
copa no puede pasar de mí si no la bebo, hágase tu voluntad''.

\bibleverse{43} Volvió y los encontró durmiendo, pues los ojos de ellos
estaban cargados. \bibleverse{44} Los dejó de nuevo, se fue y oró por
tercera vez, diciendo las mismas palabras. \footnote{\textbf{26:44} 2Cor
  12,8} \bibleverse{45} Entonces se acercó a sus discípulos y les dijo:
``¿Todavía estáis durmiendo y descansando? He aquí que se acerca la
hora, y el Hijo del Hombre es entregado en manos de los pecadores.
\bibleverse{46} Levantaos, vamos. He aquí que se acerca el que me
traiciona''.

\hypertarget{encarcelamiento-de-jesuxfas-escape-de-los-discuxedpulos}{%
\subsection{Encarcelamiento de Jesús; Escape de los
discípulos}\label{encarcelamiento-de-jesuxfas-escape-de-los-discuxedpulos}}

\bibleverse{47} Mientras aún hablaba, he aquí que vino Judas, uno de los
doce, y con él una gran multitud con espadas y palos, de parte de los
sumos sacerdotes y de los ancianos del pueblo. \bibleverse{48} El que le
entregaba les había dado una señal, diciendo: ``Al que yo bese, ése es.
Apresadle''. \bibleverse{49} Inmediatamente se acercó a Jesús y le dijo:
``¡Saludos, Rabí!'', y le besó.

\bibleverse{50} Jesús le dijo: ``Amigo, ¿qué haces aquí?'' Entonces
vinieron y le echaron mano a Jesús, y le prendieron. \bibleverse{51} He
aquí que uno de los que estaban con Jesús extendió la mano y sacó la
espada, e hirió al siervo del sumo sacerdote y le cortó la oreja.

\bibleverse{52} Entonces Jesús le dijo: ``Vuelve a poner tu espada en su
sitio, porque todos los que toman la espada morirán a espada.
\footnote{\textbf{26:52} Gén 9,6} \bibleverse{53} ¿O acaso crees que no
podría pedirle a mi Padre, y que incluso ahora me enviaría más de doce
legiones de ángeles? \footnote{\textbf{26:53} Mat 4,11} \bibleverse{54}
¿Cómo, pues, se cumplirían las Escrituras que deben ser así?''

\bibleverse{55} En aquella hora, Jesús dijo a las multitudes: ``¿Habéis
salido como contra un ladrón con espadas y palos para prenderme? Yo me
sentaba todos los días en el templo a enseñar, y no me habéis arrestado.
\bibleverse{56} Pero todo esto ha sucedido para que se cumplan las
Escrituras de los profetas.'' Entonces todos los discípulos le dejaron y
huyeron.

\hypertarget{el-interrogatorio-y-la-condena-de-jesuxfas-ante-el-sumo-sacerdote-y-el-concilio}{%
\subsection{El interrogatorio y la condena de Jesús ante el sumo
sacerdote y el
concilio}\label{el-interrogatorio-y-la-condena-de-jesuxfas-ante-el-sumo-sacerdote-y-el-concilio}}

\bibleverse{57} Los que habían prendido a Jesús lo llevaron al sumo
sacerdote Caifás, donde estaban reunidos los escribas y los ancianos.
\bibleverse{58} Pero Pedro le siguió de lejos hasta el patio del sumo
sacerdote, y entró y se sentó con los oficiales para ver el final.

\bibleverse{59} Los jefes de los sacerdotes, los ancianos y todo el
consejo buscaban falsos testimonios contra Jesús para condenarlo a
muerte, \bibleverse{60} y no los encontraron. Aunque se presentaron
muchos testigos falsos, no encontraron ninguno. Pero al fin se
presentaron dos testigos falsos \bibleverse{61} y dijeron: ``Este hombre
dijo: `Puedo destruir el templo de Dios y reconstruirlo en tres días'.''
\footnote{\textbf{26:61} Juan 2,19-21; Hech 6,14}

\bibleverse{62} El sumo sacerdote se levantó y le dijo: ``¿No tienes
respuesta? ¿Qué es esto que estos testifican contra ti?''
\bibleverse{63} Pero Jesús guardó silencio. El sumo sacerdote le
respondió: ``Te conjuro por el Dios vivo que nos digas si eres el
Cristo, el Hijo de Dios.'' \footnote{\textbf{26:63} Mat 27,12; Juan
  10,24}

\bibleverse{64} Jesús le dijo: ``Tú lo has dicho. Sin embargo, te digo
que después de esto verás al Hijo del Hombre sentado a la derecha del
Poder, y viniendo sobre las nubes del cielo.'' \footnote{\textbf{26:64}
  Sal 110,1; Mat 16,27; Mat 24,30; 2Cor 13,4}

\bibleverse{65} Entonces el sumo sacerdote se rasgó las vestiduras,
diciendo: ``¡Ha dicho una blasfemia! ¿Para qué necesitamos más testigos?
Mirad, ahora habéis oído su blasfemia. \footnote{\textbf{26:65} Juan
  10,33} \bibleverse{66} ¿Qué os parece?'' Ellos respondieron: ``¡Es
digno de muerte!'' \footnote{\textbf{26:66} Juan 19,7; Lev 24,16}
\bibleverse{67} Entonces le escupieron en la cara y le golpearon con los
puños, y algunos le abofetearon, \footnote{\textbf{26:67} Is 50,6}
\bibleverse{68} diciendo: ``¡Profetízanos, Cristo! ¿Quién te ha
pegado?''

\hypertarget{negaciuxf3n-y-arrepentimiento-de-pedro}{%
\subsection{Negación y arrepentimiento de
Pedro}\label{negaciuxf3n-y-arrepentimiento-de-pedro}}

\bibleverse{69} Pedro estaba sentado fuera, en el patio, y se le acercó
una criada diciendo: ``¡También tú estabas con Jesús, el galileo!''

\bibleverse{70} Pero él lo negó ante todos, diciendo: ``No sé de qué
estáis hablando''.

\bibleverse{71} Cuando salió al pórtico, otro lo vio y dijo a los que
estaban allí: ``Este también estuvo con Jesús de Nazaret.''

\bibleverse{72} De nuevo lo negó con un juramento: ``No conozco al
hombre''.

\bibleverse{73} Al cabo de un rato, los que estaban allí se acercaron y
dijeron a Pedro: ``Seguramente tú también eres uno de ellos, pues tu
discurso te da a conocer.''

\bibleverse{74} Entonces empezó a maldecir y a jurar: ``¡No conozco a
ese hombre!''. Inmediatamente cantó el gallo. \bibleverse{75} Pedro se
acordó de la palabra que Jesús le había dicho: ``Antes de que cante el
gallo, me negarás tres veces''. Entonces salió y lloró amargamente.

\hypertarget{uxfaltima-deliberaciuxf3n-del-sumo-consejo-extradiciuxf3n-de-los-condenados-al-gobernador-romano-pilato}{%
\subsection{Última deliberación del sumo consejo; Extradición de los
condenados al gobernador romano
Pilato}\label{uxfaltima-deliberaciuxf3n-del-sumo-consejo-extradiciuxf3n-de-los-condenados-al-gobernador-romano-pilato}}

\hypertarget{section-26}{%
\section{27}\label{section-26}}

\bibleverse{1} Al amanecer, todos los jefes de los sacerdotes y los
ancianos del pueblo se pusieron de acuerdo contra Jesús para matarlo.
\bibleverse{2} Lo ataron, lo llevaron y lo entregaron a Poncio Pilato,
el gobernador.

\bibleverse{3} Entonces Judas, el que lo traicionó, al ver que Jesús era
condenado, sintió remordimiento y devolvió las treinta monedas de plata
a los sumos sacerdotes y a los ancianos, \footnote{\textbf{27:3} Mat
  26,15} \bibleverse{4} diciendo: ``He pecado al entregar sangre
inocente.'' Pero ellos dijeron: ``¿Qué es eso para nosotros? Vosotros os
ocupáis de ello''.

\bibleverse{5} Arrojó las piezas de plata en el santuario y se marchó.
Luego se fue y se ahorcó. \footnote{\textbf{27:5} Hech 1,18-19}

\bibleverse{6} Los jefes de los sacerdotes tomaron las piezas de plata y
dijeron: ``No es lícito ponerlas en el tesoro, pues es el precio de la
sangre.'' \footnote{\textbf{27:6} Deut 23,18} \bibleverse{7} Se
asesoraron y compraron con ellas el campo del alfarero para enterrar a
los extranjeros. \bibleverse{8} Por eso ese campo ha sido llamado ``El
campo de la sangre'' hasta el día de hoy. \bibleverse{9} Entonces se
cumplió lo que se había dicho por medio del profeta Jeremías\footnote{\textbf{27:9}
  algunos manuscritos omiten ``Jeremías''} , que decía``Tomaron las
treinta piezas de plata, el precio de aquel sobre el que se había fijado
un precio, al que algunos de los hijos de Israel le dieron precio,
\bibleverse{10} y los dieron para el campo del alfarero, como el Señor
me ordenó\footnote{\textbf{27:10} Zacarías 11:12-13; Jeremías 19:1-13;
  32:6-9} ''.

\hypertarget{interrogatorio-de-jesuxfas-ante-pilato-jesuxfas-rechazado-por-la-gente-su-condenaciuxf3n-y-flagelaciuxf3n}{%
\subsection{Interrogatorio de Jesús ante Pilato; Jesús rechazado por la
gente; su condenación y
flagelación}\label{interrogatorio-de-jesuxfas-ante-pilato-jesuxfas-rechazado-por-la-gente-su-condenaciuxf3n-y-flagelaciuxf3n}}

\bibleverse{11} Jesús se presentó ante el gobernador y éste le preguntó:
``¿Eres tú el rey de los judíos?'' Jesús le dijo: ``Tú lo dices''.

\bibleverse{12} Cuando fue acusado por los sumos sacerdotes y los
ancianos, no respondió nada. \footnote{\textbf{27:12} Mat 26,63; Is 53,7}
\bibleverse{13} Entonces Pilato le dijo: ``¿No oyes cuántas cosas
declaran contra ti?''.

\bibleverse{14} No le respondió, ni siquiera una palabra, de modo que el
gobernador se maravilló mucho. \footnote{\textbf{27:14} Juan 19,9}

\hypertarget{jesuxfas-y-barrabuxe1s}{%
\subsection{Jesús y Barrabás}\label{jesuxfas-y-barrabuxe1s}}

\bibleverse{15} En la fiesta, el gobernador acostumbraba a liberar a la
multitud un prisionero que ellos deseaban. \bibleverse{16} Tenían
entonces un preso notable llamado Barrabás. \bibleverse{17} Así pues,
cuando se reunieron, Pilato les dijo: ``¿A quién queréis que os suelte?
¿A Barrabás, o a Jesús, que se llama Cristo?'' \bibleverse{18} Porque
sabía que por envidia le habían entregado. \footnote{\textbf{27:18} Juan
  12,19}

\bibleverse{19} Mientras estaba sentado en el tribunal, su mujer le
mandó decir: ``No tengas nada que ver con ese justo, porque hoy he
sufrido muchas cosas en sueños por su culpa.''

\bibleverse{20} Los jefes de los sacerdotes y los ancianos persuadieron
a las multitudes para que pidieran a Barrabás y destruyeran a Jesús.
\bibleverse{21} Pero el gobernador les respondió: ``¿A cuál de los dos
queréis que os suelte?'' Dijeron: ``¡Barabbas!''

\bibleverse{22} Pilato les dijo: ``¿Qué haré, pues, a Jesús, que se
llama Cristo?'' Todos le decían: ``¡Que lo crucifiquen!''

\bibleverse{23} Pero el gobernador dijo: ``¿Por qué? ¿Qué mal ha
hecho?'' Pero ellos gritaban mucho, diciendo: ``¡Que lo crucifiquen!''.

\bibleverse{24} Al ver Pilato que no se ganaba nada, sino que se
iniciaba un alboroto, tomó agua y se lavó las manos ante la multitud,
diciendo: ``Yo soy inocente de la sangre de este justo. Vosotros os
encargáis de ello''. \footnote{\textbf{27:24} Deut 21,6}

\bibleverse{25} Todo el pueblo respondió: ``¡Que su sangre sea sobre
nosotros y sobre nuestros hijos!'' \footnote{\textbf{27:25} Hech 5,28}

\bibleverse{26} Entonces les soltó a Barrabás, pero a Jesús lo azotó y
lo entregó para que lo crucificaran.

\hypertarget{la-burla-y-el-maltrato-de-jesuxfas-por-parte-de-los-soldados-romanos}{%
\subsection{La burla y el maltrato de Jesús por parte de los soldados
romanos}\label{la-burla-y-el-maltrato-de-jesuxfas-por-parte-de-los-soldados-romanos}}

\bibleverse{27} Entonces los soldados del gobernador llevaron a Jesús al
pretorio y reunieron a toda la guarnición contra él. \bibleverse{28} Lo
desnudaron y le pusieron un manto escarlata. \bibleverse{29} Trenzaron
una corona de espinas y se la pusieron en la cabeza, y una caña en la
mano derecha; se arrodillaron ante él y se burlaron, diciendo: ``¡Salve,
Rey de los judíos!'' \bibleverse{30} Le escupían, tomaban la caña y le
golpeaban en la cabeza. \footnote{\textbf{27:30} Is 50,6}
\bibleverse{31} Después de burlarse de él, le quitaron el manto, le
pusieron su ropa y lo llevaron a crucificar.

\hypertarget{el-curso-de-la-muerte-de-jesuxfas-despuuxe9s-del-guxf3lgota-su-crucifixiuxf3n-y-su-muerte}{%
\subsection{El curso de la muerte de Jesús después del Gólgota, su
crucifixión y su
muerte}\label{el-curso-de-la-muerte-de-jesuxfas-despuuxe9s-del-guxf3lgota-su-crucifixiuxf3n-y-su-muerte}}

\bibleverse{32} Al salir, encontraron a un hombre de Cirene, de nombre
Simón, y le obligaron a ir con ellos para que llevara su cruz.
\bibleverse{33} Cuando llegaron a un lugar llamado ``Gólgota'', es
decir, ``El lugar de la calavera'', \bibleverse{34} le dieron a beber
vino\footnote{\textbf{27:34} TR añade ``para que se cumpla lo dicho por
  el profeta: `Se repartieron mis vestidos, y para mi ropa echaron
  suertes;'\,'' {[}ver Salmo 22:18 y Juan 19:24{]}} agrio mezclado con
hiel. Cuando lo probó, no quiso beber. \footnote{\textbf{27:34} Sal
  69,21} \bibleverse{35} Cuando lo crucificaron, se repartieron su ropa
echando suertes, \footnote{\textbf{27:35} Juan 19,24} \bibleverse{36} y
se sentaron a velarlo allí. \bibleverse{37} Colocaron sobre su cabeza la
acusación escrita: ``ESTE ES JESÚS, EL REY DE LOS JUDÍOS''.

\bibleverse{38} Entonces había dos ladrones crucificados con él, uno a
su derecha y otro a la izquierda. \footnote{\textbf{27:38} Is 53,12}

\bibleverse{39} Los que pasaban le blasfemaban, moviendo la cabeza
\footnote{\textbf{27:39} Sal 22,7} \bibleverse{40} y diciendo: ``Tú, que
destruyes el templo y lo construyes en tres días, sálvate a ti mismo. Si
eres el Hijo de Dios, baja de la cruz''. \footnote{\textbf{27:40} Mat
  26,61; Juan 2,19}

\bibleverse{41} Asimismo, los jefes de los sacerdotes, burlándose con
los escribas, los fariseos \footnote{\textbf{27:41} TR omite ``los
  fariseos''} y los ancianos, decían: \bibleverse{42} ``Ha salvado a
otros, pero no puede salvarse a sí mismo. Si es el Rey de Israel, que
baje ahora de la cruz, y creeremos en él. \bibleverse{43} Él confía en
Dios. Que Dios lo libere ahora, si lo quiere; porque ha dicho: ``Yo soy
el Hijo de Dios''\,''. \footnote{\textbf{27:43} Sal 22,8}
\bibleverse{44} También los ladrones que estaban crucificados con él le
lanzaron el mismo reproche.

\hypertarget{la-muerte-de-jesuxfas-las-seuxf1ales-milagrosas-de-su-muerte}{%
\subsection{La muerte de Jesús; las señales milagrosas de su
muerte}\label{la-muerte-de-jesuxfas-las-seuxf1ales-milagrosas-de-su-muerte}}

\bibleverse{45} Desde la hora\footnote{\textbf{27:45} mediodía} sexta
hubo oscuridad sobre toda la tierra hasta la hora novena.
\bibleverse{46} Hacia la hora novena, Jesús gritó con gran voz,
diciendo: ``Elí, Elí, ¿lama sabactani?'' Es decir, ``Dios mío, Dios mío,
¿por qué me has abandonado?'' \footnote{\textbf{27:46} Sal 22,1}

\bibleverse{47} Algunos de los que estaban allí, al oírlo, dijeron:
``Este hombre llama a Elías''.

\bibleverse{48} Inmediatamente, uno de ellos corrió y tomó una esponja,
la llenó de vinagre, la puso en una caña y le dio de beber. \footnote{\textbf{27:48}
  Sal 69,21} \bibleverse{49} Los demás dijeron: ``Déjenlo. Vamos a ver
si Elías viene a salvarlo''.

\bibleverse{50} Jesús volvió a gritar con fuerza y entregó su espíritu.

\bibleverse{51} He aquí que el velo del templo se rasgó en dos desde
arriba hasta abajo. La tierra tembló y las rocas se partieron.
\footnote{\textbf{27:51} Éxod 26,31} \bibleverse{52} Se abrieron los
sepulcros y resucitaron muchos cuerpos de los santos que habían dormido;
\bibleverse{53} y saliendo de los sepulcros después de su resurrección,
entraron en la ciudad santa y se aparecieron a muchos.

\bibleverse{54} El centurión y los que estaban con él observando a
Jesús, al ver el terremoto y las cosas que se hacían, se espantaron,
diciendo: ``¡Verdaderamente éste era el Hijo de Dios!''

\bibleverse{55} Estaban allí mirando desde lejos muchas mujeres que
habían seguido a Jesús desde Galilea, sirviéndole. \footnote{\textbf{27:55}
  Luc 8,2-3} \bibleverse{56} Entre ellas estaban María Magdalena, María
la madre de Santiago y de José, y la madre de los hijos de Zebedeo.

\hypertarget{entierro-de-jesuxfas-orden-de-los-guardias-de-la-tumba}{%
\subsection{Entierro de Jesús; Orden de los guardias de la
tumba}\label{entierro-de-jesuxfas-orden-de-los-guardias-de-la-tumba}}

\bibleverse{57} Cuando llegó la noche, vino un hombre rico de Arimatea
llamado José, que también era discípulo de Jesús. \footnote{\textbf{27:57}
  Deut 21,22-23} \bibleverse{58} Este hombre fue a Pilato y pidió el
cuerpo de Jesús. Entonces Pilato ordenó que se entregara el cuerpo.
\bibleverse{59} José tomó el cuerpo, lo envolvió en una tela de lino
limpia \bibleverse{60} y lo puso en su propio sepulcro nuevo, que había
excavado en la roca. Luego hizo rodar una gran piedra contra la puerta
del sepulcro y se fue. \footnote{\textbf{27:60} Is 53,9} \bibleverse{61}
María Magdalena estaba allí, y la otra María, sentadas frente al
sepulcro.

\bibleverse{62} Al día siguiente, que era el día siguiente al de la
preparación, se reunieron los jefes de los sacerdotes y los fariseos
ante Pilato, \bibleverse{63} diciendo: ``Señor, nos acordamos de lo que
dijo aquel engañador cuando aún vivía: `Después de tres días
resucitaré'. \footnote{\textbf{27:63} Mat 20,19; 2Cor 6,8}
\bibleverse{64} Manda, pues, que se asegure el sepulcro hasta el tercer
día, no sea que vengan sus discípulos de noche y lo roben, y digan al
pueblo: `Ha resucitado de entre los muertos'; y el último engaño será
peor que el primero.''

\bibleverse{65} Pilato les dijo: ``Tenéis una guardia. Vayan y
asegúrenlo todo lo que puedan''. \bibleverse{66} Así que fueron con la
guardia y aseguraron el sepulcro, sellando la piedra.

\hypertarget{las-dos-mujeres-junto-a-la-tumba-vacuxeda-en-la-mauxf1ana-de-pascua-la-primera-apariciuxf3n-de-jesuxfas-engauxf1ar-al-luxedder-del-pueblo}{%
\subsection{Las dos mujeres junto a la tumba vacía en la mañana de
Pascua; La primera aparición de Jesús; Engañar al líder del
pueblo}\label{las-dos-mujeres-junto-a-la-tumba-vacuxeda-en-la-mauxf1ana-de-pascua-la-primera-apariciuxf3n-de-jesuxfas-engauxf1ar-al-luxedder-del-pueblo}}

\hypertarget{section-27}{%
\section{28}\label{section-27}}

\bibleverse{1} Después del sábado, al amanecer del primer día de la
semana, María Magdalena y la otra María fueron a ver el sepulcro.
\footnote{\textbf{28:1} Hech 20,7; 1Cor 16,2; Apoc 1,10} \bibleverse{2}
Se produjo un gran terremoto, porque un ángel del Señor descendió del
cielo, vino, removió la piedra de la puerta y se sentó sobre ella.
\bibleverse{3} Su aspecto era como un relámpago, y su ropa blanca como
la nieve. \footnote{\textbf{28:3} Mat 17,2; Hech 1,10} \bibleverse{4}
Por miedo a él, los guardias se estremecieron y quedaron como muertos.
\bibleverse{5} El ángel respondió a las mujeres: ``No temáis, porque sé
que buscáis a Jesús, que ha sido crucificado. \bibleverse{6} No está
aquí, porque ha resucitado, tal como dijo. Venid a ver el lugar donde
yacía el Señor. \footnote{\textbf{28:6} Mat 12,40; Mat 16,21; Mat 17,23;
  Mat 20,19} \bibleverse{7} Id pronto a decir a sus discípulos: ``Ha
resucitado de entre los muertos, y he aquí que va delante de vosotros a
Galilea; allí le veréis''. He aquí que os lo he dicho''. \footnote{\textbf{28:7}
  Mat 26,32}

\bibleverse{8} Salieron rápidamente del sepulcro con miedo y gran
alegría, y corrieron a avisar a sus discípulos. \bibleverse{9} Mientras
iban a avisar a sus discípulos, he aquí que Jesús les salió al
encuentro, diciendo: ``¡Alégrense!'' Se acercaron, se agarraron a sus
pies y le adoraron.

\bibleverse{10} Entonces Jesús les dijo: ``No tengan miedo. Id a decir a
mis hermanos que vayan a Galilea, y allí me verán''. \footnote{\textbf{28:10}
  Heb 2,11}

\hypertarget{la-falsa-afirmaciuxf3n-de-los-luxedderes-del-pueblo-del-cuerpo-robado-de-jesuxfas}{%
\subsection{La falsa afirmación de los líderes del pueblo del cuerpo
robado de
Jesús}\label{la-falsa-afirmaciuxf3n-de-los-luxedderes-del-pueblo-del-cuerpo-robado-de-jesuxfas}}

\bibleverse{11} Mientras iban, he aquí que algunos de los guardias
entraron en la ciudad y contaron a los sumos sacerdotes todo lo que
había sucedido. \bibleverse{12} Cuando se reunieron con los ancianos y
tomaron consejo, dieron una gran cantidad de plata a los soldados,
\bibleverse{13} diciendo: ``Decid que sus discípulos vinieron de noche y
lo robaron mientras dormíamos. \footnote{\textbf{28:13} Mat 27,64}
\bibleverse{14} Si esto llega a oídos del gobernador, le convenceremos y
os libraremos de preocupaciones.'' \bibleverse{15} Así que tomaron el
dinero e hicieron lo que se les dijo. Este dicho se difundió entre los
judíos, y continúa hasta hoy.

\hypertarget{jesuxfas-apareciuxf3-en-la-montauxf1a-de-galilea-su-uxfaltimo-mandato-a-los-once-discuxedpulos}{%
\subsection{Jesús apareció en la montaña de Galilea; su último mandato a
los once
discípulos}\label{jesuxfas-apareciuxf3-en-la-montauxf1a-de-galilea-su-uxfaltimo-mandato-a-los-once-discuxedpulos}}

\bibleverse{16} Pero los once discípulos fueron a Galilea, al monte
donde Jesús los había enviado. \bibleverse{17} Cuando le vieron, se
postraron ante él; pero algunos dudaban. \bibleverse{18} Jesús se acercó
a ellos y les habló diciendo: ``Se me ha dado toda la autoridad en el
cielo y en la tierra. \footnote{\textbf{28:18} Mat 11,27; Efes 1,20-22}
\bibleverse{19} Id\footnote{\textbf{28:19} TR y NU añaden ``por tanto''}
y haced discípulos a todas las naciones, bautizándolas en el nombre del
Padre y del Hijo y del Espíritu Santo, \footnote{\textbf{28:19} Mat
  24,14; Mar 16,15-16; 2Cor 5,20} \bibleverse{20} enseñándoles a
observar todo lo que os he mandado. He aquí que yo estoy con vosotros
todos los días, hasta el fin del mundo''. Amén. \footnote{\textbf{28:20}
  Mat 18,20}
