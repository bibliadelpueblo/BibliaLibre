\hypertarget{los-hijos-de-israel-multiplicaron}{%
\subsection{Los hijos de Israel
multiplicaron}\label{los-hijos-de-israel-multiplicaron}}

\hypertarget{section}{%
\section{1}\label{section}}

\bibleverse{1} Estos son los nombres de los hijos de Israel que vinieron
a Egipto (cada hombre y su familia vinieron con Jacob): \footnote{\textbf{1:1}
  Gén 46,8} \bibleverse{2} Rubén, Simeón, Leví y Judá, \bibleverse{3}
Isacar, Zabulón y Benjamín, \bibleverse{4} Dan y Neftalí, Gad y Aser.
\bibleverse{5} Todas las almas que salieron del cuerpo de Jacob fueron
setenta almas, y José ya estaba en Egipto. \footnote{\textbf{1:5} Gén
  46,27} \bibleverse{6} José murió, al igual que todos sus hermanos y
toda aquella generación. \footnote{\textbf{1:6} Gén 50,26}
\bibleverse{7} Los hijos de Israel fructificaron, se multiplicaron y se
hicieron muy poderosos, y la tierra se llenó de ellos. \footnote{\textbf{1:7}
  Hech 7,17}

\bibleverse{8} Se levantó un nuevo rey sobre Egipto, que no conocía a
José. \bibleverse{9} Dijo a su pueblo: ``He aquí,\footnote{\textbf{1:9}
  ``He aquí'', de ``\hebrew{הִנֵּה}'', significa mirar, fijarse, observar,
  ver o contemplar. Se utiliza a menudo como interjección.} el pueblo de
los hijos de Israel es más y más poderoso que nosotros. \bibleverse{10}
Vengan, tratemos con sabiduría con ellos, no sea que se multipliquen, y
suceda que cuando estalle alguna guerra, ellos también se unan a
nuestros enemigos y luchen contra nosotros, y escapen del país.''
\bibleverse{11} Por lo tanto, pusieron sobre ellos a los capataces para
que los afligieran con sus cargas. Construyeron ciudades de
almacenamiento para el Faraón: Pitón y Raamsés. \footnote{\textbf{1:11}
  Gén 15,13; Gén 47,11} \bibleverse{12} Pero cuanto más los afligían,
más se multiplicaban y más se extendían. Comenzaron a temer a los hijos
de Israel. \bibleverse{13} Los egipcios hicieron servir sin piedad a los
hijos de Israel, \bibleverse{14} y les amargaron la vida con un duro
servicio en la argamasa y en el ladrillo, y en todo tipo de servicio en
el campo, todo su servicio, en el que los hicieron servir sin piedad.

\hypertarget{el-temor-de-dios-de-los-dos-parteras}{%
\subsection{El temor de Dios de los dos
parteras}\label{el-temor-de-dios-de-los-dos-parteras}}

\bibleverse{15} El rey de Egipto habló con las parteras hebreas, de las
cuales una se llamaba Sifra y la otra Puah, \bibleverse{16} y les dijo:
``Cuando cumpláis con el deber de partera a las mujeres hebreas y las
veáis en el taburete de parto, si es un hijo, lo mataréis; pero si es
una hija, vivirá.'' \bibleverse{17} Pero las parteras temían a
Dios,\footnote{\textbf{1:17} ``Maná'' significa ``¿Qué es?''} y no
hicieron lo que el rey de Egipto les ordenaba, sino que salvaron a los
niños vivos. \bibleverse{18} El rey de Egipto llamó a las parteras y les
dijo: ``¿Por qué habéis hecho esto y habéis salvado vivos a los niños?''

\bibleverse{19} Las comadronas dijeron al faraón: ``Porque las mujeres
hebreas no son como las egipcias, pues son vigorosas y dan a luz antes
de que la comadrona llegue a ellas.''

\bibleverse{20} Dios trató bien a las parteras, y el pueblo se
multiplicó y se hizo muy poderoso. \bibleverse{21} Como las parteras
temían a Dios, él les dio familias. \bibleverse{22} El faraón ordenó a
todo su pueblo, diciendo: ``Echaréis al río a todo hijo que nazca, y a
toda hija la salvaréis con vida.''

\hypertarget{nacimiento-y-abandono-salvaciuxf3n-y-educaciuxf3n-de-moisuxe9s}{%
\subsection{Nacimiento y abandono, salvación y educación de
Moisés}\label{nacimiento-y-abandono-salvaciuxf3n-y-educaciuxf3n-de-moisuxe9s}}

\hypertarget{section-1}{%
\section{2}\label{section-1}}

\bibleverse{1} Un hombre de la casa de Leví fue y tomó como esposa a una
hija de Leví. \footnote{\textbf{2:1} Éxod 6,20} \bibleverse{2} La mujer
concibió y dio a luz un hijo. Al ver que era un buen niño, lo escondió
durante tres meses. \footnote{\textbf{2:2} Hech 7,20; Heb 11,23}
\bibleverse{3} Cuando ya no pudo esconderlo, tomó un cesto de papiro
para él, y lo cubrió con brea y alquitrán. Puso al niño en ella y lo
depositó en los juncos de la orilla del río. \bibleverse{4} Su hermana
se quedó lejos, para ver qué se hacía con él. \footnote{\textbf{2:4}
  Éxod 15,20} \bibleverse{5} La hija del faraón bajó a bañarse en el
río. Sus doncellas se paseaban por la orilla del río. Vio el cesto entre
los juncos y envió a su criado a buscarlo. \bibleverse{6} La abrió y vio
al niño, y he aquí que el niño lloraba. Se compadeció de él y dijo:
``Este es uno de los hijos de los hebreos''.

\bibleverse{7} Entonces su hermana dijo a la hija del faraón: ``¿Debo ir
a llamar a una nodriza de las mujeres hebreas para que te amamante al
niño?''

\bibleverse{8} La hija del faraón le dijo: ``Ve''. La joven fue y llamó
a la madre del niño. \bibleverse{9} La hija del faraón le dijo:
``Llévate a este niño y cuídalo por mí, y te daré tu salario''. La mujer
tomó al niño y lo amamantó. \bibleverse{10} El niño creció, lo llevó a
la hija del faraón y se convirtió en su hijo. Ella le puso el nombre de
Moisés, y dijo: ``Porque lo saqué del agua''.

\hypertarget{moisuxe9s-matuxf3-al-egipcio}{%
\subsection{Moisés mató al Egipcio}\label{moisuxe9s-matuxf3-al-egipcio}}

\bibleverse{11} En aquellos días, cuando Moisés había crecido, salió a
ver a sus hermanos y vio sus cargas. Vio que un egipcio golpeaba a un
hebreo, uno de sus hermanos. \footnote{\textbf{2:11} Heb 11,24-25}
\bibleverse{12} Miró a un lado y a otro, y al ver que no había nadie,
mató al egipcio y lo escondió en la arena. \footnote{\textbf{2:12} Hech
  7,24}

\bibleverse{13} Salió el segundo día, y he aquí que dos hombres de los
hebreos estaban peleando entre sí. Le dijo al que había hecho el mal:
``¿Por qué golpeas a tu compañero?''.

\bibleverse{14} Él dijo: ``¿Quién te ha hecho príncipe y juez sobre
nosotros? ¿Piensas matarme, como mataste al egipcio?'' Moisés tuvo miedo
y dijo: ``Ciertamente esto se sabe''. \footnote{\textbf{2:14} Hech
  7,27-28; Hech 7,35}

\hypertarget{moisuxe9s-huyuxf3-a-madiuxe1n-y-se-casa-con-suxe9phora}{%
\subsection{Moisés huyó a Madián y se casa con
Séphora}\label{moisuxe9s-huyuxf3-a-madiuxe1n-y-se-casa-con-suxe9phora}}

\bibleverse{15} Cuando el Faraón se enteró de esto, trató de matar a
Moisés. Pero Moisés huyó de la presencia del Faraón, y vivió en la
tierra de Madián, y se sentó junto a un pozo. \footnote{\textbf{2:15}
  Heb 11,27}

\bibleverse{16} El sacerdote de Madián tenía siete hijas. Ellas vinieron
y sacaron agua, y llenaron los abrevaderos para abrevar el rebaño de su
padre. \footnote{\textbf{2:16} Éxod 3,1} \bibleverse{17} Los pastores
vinieron y las ahuyentaron; pero Moisés se levantó y las ayudó, y abrevó
su rebaño. \footnote{\textbf{2:17} Gén 29,10} \bibleverse{18} Cuando
llegaron a Reuel, su padre, éste les dijo: ``¿Cómo es que habéis vuelto
hoy tan temprano?''

\bibleverse{19} Dijeron: ``Un egipcio nos libró de la mano de los
pastores, y además nos sacó agua y abrevó el rebaño.''

\bibleverse{20} Dijo a sus hijas: ``¿Dónde está? ¿Por qué habéis dejado
al hombre? Llamadle, para que coma pan''.

\bibleverse{21} Moisés se contentó con habitar con el hombre. Le dio a
Moisés a Séfora, su hija. \bibleverse{22} Ella dio a luz un hijo, y él
le puso el nombre de Gershom, porque dijo: ``He vivido como extranjero
en tierra extranjera''. \footnote{\textbf{2:22} Éxod 18,3}

\hypertarget{dios-escucha-las-aflicciones-de-los-israelitas-oprimidos}{%
\subsection{Dios escucha las aflicciones de los israelitas
oprimidos}\label{dios-escucha-las-aflicciones-de-los-israelitas-oprimidos}}

\bibleverse{23} En el transcurso de esos muchos días, el rey de Egipto
murió, y los hijos de Israel suspiraron a causa de la esclavitud, y
lloraron, y su clamor subió a Dios a causa de la esclavitud. \footnote{\textbf{2:23}
  Éxod 3,7} \bibleverse{24} Dios oyó su gemido, y se acordó de su pacto
con Abraham, con Isaac y con Jacob. \footnote{\textbf{2:24} Gén 15,18;
  Gén 26,3; Gén 28,13-14} \bibleverse{25} Dios vio a los hijos de
Israel, y Dios comprendió.

\hypertarget{dios-se-revela-a-moisuxe9s-en-la-zarza-como-el-yo-soy}{%
\subsection{Dios se revela a Moisés en la zarza como el ``Yo
soy''}\label{dios-se-revela-a-moisuxe9s-en-la-zarza-como-el-yo-soy}}

\hypertarget{section-2}{%
\section{3}\label{section-2}}

\bibleverse{1} Moisés guardaba el rebaño de Jetro, su suegro, el
sacerdote de Madián, y condujo el rebaño al fondo del desierto, y llegó
al monte de Dios, a Horeb. \bibleverse{2} El ángel de Yahvé se le
apareció en una llama de fuego en medio de un arbusto. Miró, y he aquí
que la zarza ardía en fuego, y la zarza no se consumía. \footnote{\textbf{3:2}
  Hech 7,30; Deut 33,16} \bibleverse{3} Moisés dijo: ``Iré ahora a ver
este gran espectáculo, por qué la zarza no se quema''.

\bibleverse{4} Cuando Yahvé vio que se acercaba a ver, Dios le llamó
desde el centro de la zarza y le dijo: ``¡Moisés! Moisés!'' Dijo: ``Aquí
estoy''.

\bibleverse{5} Él dijo: ``No te acerques. Quítate las sandalias, porque
el lugar que pisas es tierra santa''. \footnote{\textbf{3:5} Jos 5,15;
  Gén 28,17} \bibleverse{6} Además, dijo: ``Yo soy el Dios de tu padre,
el Dios de Abraham, el Dios de Isaac y el Dios de Jacob''. Moisés ocultó
su rostro porque tenía miedo de mirar a Dios. \footnote{\textbf{3:6} Mat
  22,32}

\bibleverse{7} Yahvé dijo: ``Ciertamente he visto la aflicción de mi
pueblo que está en Egipto, y he oído su clamor a causa de sus capataces,
pues conozco sus penas. \footnote{\textbf{3:7} Éxod 2,23} \bibleverse{8}
He descendido para librarlos de la mano de los egipcios, y para hacerlos
subir de esa tierra a una tierra buena y extensa, a una tierra que fluye
leche y miel; al lugar del cananeo, del hitita, del amorreo, del
ferezeo, del heveo y del jebuseo. \bibleverse{9} Ahora bien, he aquí que
el clamor de los hijos de Israel ha llegado hasta mí. Además, he visto
la opresión con que los egipcios los oprimen. \bibleverse{10} Ven, pues,
ahora y te enviaré al Faraón para que saques a mi pueblo, los hijos de
Israel, de Egipto''.

\bibleverse{11} Moisés dijo a Dios: ``¿Quién soy yo para ir al Faraón y
sacar a los hijos de Israel de Egipto?'' \footnote{\textbf{3:11} Éxod
  4,10; Is 6,5; Is 6,8; Jer 1,6}

\bibleverse{12} Dijo: ``Ciertamente yo estaré con vosotros. Esta será la
señal para ti, de que te he enviado: cuando hayas sacado al pueblo de
Egipto, servirás a Dios en este monte.''

\hypertarget{la-revelacion-del-nombre-de-dios}{%
\subsection{La revelacion del nombre de
Dios}\label{la-revelacion-del-nombre-de-dios}}

\bibleverse{13} Moisés dijo a Dios: ``He aquí que cuando llegue a los
hijos de Israel y les diga: ``El Dios de vuestros padres me ha enviado a
vosotros'', y ellos me pregunten: ``¿Cuál es su nombre?'', ¿qué debo
decirles?''

\bibleverse{14} Dios dijo a Moisés: ``YO SOY EL QUE SOY'', y dijo:
``Dirás a los hijos de Israel esto: `YO SOY me ha enviado a
ustedes'\,''. \footnote{\textbf{3:14} Apoc 1,4; Apoc 1,8}
\bibleverse{15} Dios dijo además a Moisés: ``Dirás a los hijos de Israel
esto: `Yahvé, el Dios de vuestros padres, el Dios de Abraham, el Dios de
Isaac y el Dios de Jacob, me ha enviado a vosotros'. Este es mi nombre
para siempre, y este es mi memorial para todas las generaciones.
\footnote{\textbf{3:15} Éxod 6,2-3; Is 42,8}

\hypertarget{el-llamado-de-dios-y-su-promesa-por-moisuxe9s}{%
\subsection{El llamado de Dios y su promesa por
Moisés}\label{el-llamado-de-dios-y-su-promesa-por-moisuxe9s}}

\bibleverse{16} Ve y reúne a los ancianos de Israel y diles: `El Señor,
el Dios de tus padres, el Dios de Abraham, de Isaac y de Jacob, se me ha
aparecido diciendo: ``Ciertamente te he visitado y he visto lo que te
han hecho en Egipto. \bibleverse{17} He dicho que te haré subir de la
aflicción de Egipto a la tierra del cananeo, del hitita, del amorreo,
del ferezeo, del heveo y del jebuseo, a una tierra que mana leche y
miel''. \bibleverse{18} Ellos escucharán tu voz. Vendrás, tú y los
ancianos de Israel, al rey de Egipto, y le dirás: `El Señor, el Dios de
los hebreos, se ha reunido con nosotros. Ahora, por favor, vayamos tres
días de camino al desierto, para que ofrezcamos sacrificios a Yavé,
nuestro Dios'. \footnote{\textbf{3:18} Éxod 5,1; Éxod 5,3}
\bibleverse{19} Yo sé que el rey de Egipto no te dará permiso para ir,
ni siquiera con una mano poderosa. \bibleverse{20} Extenderé mi mano y
golpearé a Egipto con todas mis maravillas que haré entre ellos, y
después de eso los dejará ir. \bibleverse{21} Yo le daré a este pueblo
el favor a los ojos de los egipcios, y sucederá que cuando te vayas, no
te irás con las manos vacías. \footnote{\textbf{3:21} Éxod 11,2-3; Éxod
  12,35-36; Gén 15,14} \bibleverse{22} Pero cada mujer pedirá a su
vecina, y a la que visite su casa, joyas de plata, joyas de oro y ropa.
Las pondréis sobre vuestros hijos y sobre vuestras hijas. Saquearás a
los egipcios''.

\hypertarget{los-milagros-de-la-autenticaciuxf3n}{%
\subsection{Los milagros de la
autenticación}\label{los-milagros-de-la-autenticaciuxf3n}}

\hypertarget{section-3}{%
\section{4}\label{section-3}}

\bibleverse{1} Moisés respondió: ``Pero he aquí que no me creerán ni
escucharán mi voz, porque dirán: ``Yahvé no se te ha aparecido''\,''.

\bibleverse{2} Yahvé le dijo: ``¿Qué es eso que tienes en la mano?''
Dijo: ``Una vara''.

\bibleverse{3} Dijo: ``Tíralo al suelo''. La arrojó al suelo, y se
convirtió en una serpiente; y Moisés huyó de ella. \footnote{\textbf{4:3}
  Éxod 7,10}

\bibleverse{4} Yahvé dijo a Moisés: ``Extiende tu mano y tómalo por la
cola''. Extendió la mano y la agarró, y se convirtió en una vara en su
mano.

\bibleverse{5} ``Esto es para que crean que Yahvé, el Dios de sus
padres, el Dios de Abraham, el Dios de Isaac y el Dios de Jacob, se te
ha aparecido''. \bibleverse{6} Yahvé le dijo además: ``Ahora pon tu mano
dentro de tu manto''. Metió la mano dentro de su manto, y cuando la
sacó, he aquí que su mano estaba leprosa, blanca como la nieve.

\bibleverse{7} Dijo: ``Vuelve a meter la mano en el manto''. Volvió a
meter la mano dentro de su manto, y cuando la sacó de su manto, he aquí
que se había vuelto de nuevo como su otra carne.

\bibleverse{8} ``Sucederá que si no te creen ni escuchan la voz de la
primera señal, creerán la voz de la segunda señal. \bibleverse{9}
Sucederá, si no creen ni siquiera en estas dos señales ni escuchan tu
voz, que tomarás del agua del río y la derramarás sobre la tierra seca.
El agua que saques del río se convertirá en sangre sobre la tierra
seca''. \footnote{\textbf{4:9} Éxod 7,17}

\hypertarget{nuevas-objeciones-de-moisuxe9s-nombramiento-de-aaruxf3n-como-orador}{%
\subsection{Nuevas objeciones de Moisés; Nombramiento de Aarón como
orador}\label{nuevas-objeciones-de-moisuxe9s-nombramiento-de-aaruxf3n-como-orador}}

\bibleverse{10} Moisés dijo a Yahvé: ``Oh, Señor, no soy elocuente, ni
antes, ni desde que has hablado a tu siervo; porque soy lento de palabra
y de lengua lenta.'' \footnote{\textbf{4:10} Éxod 3,11; Éxod 6,12; Éxod
  6,30}

\bibleverse{11} El Señor le dijo: ``¿Quién hizo la boca del hombre? ¿O
quién hace que uno sea mudo, o sordo, o que vea, o ciego? ¿No soy yo,
Yahvé? \footnote{\textbf{4:11} Sal 94,9} \bibleverse{12} Ahora, pues,
vete, y yo estaré con tu boca y te enseñaré lo que debes hablar''.
\footnote{\textbf{4:12} Mat 10,19}

\bibleverse{13} Moisés dijo: ``Oh, Señor, por favor, envía a otro''.

\bibleverse{14} La ira de Yahvé ardió contra Moisés y le dijo: ``¿Y
Aarón, tu hermano, el levita? Sé que sabe hablar bien. Además, he aquí
que él sale a recibirte. Cuando te vea, se alegrará en su corazón.
\bibleverse{15} Tú le hablarás y pondrás las palabras en su boca. Yo
estaré con tu boca y con la suya, y te enseñaré lo que debes hacer.
\bibleverse{16} Él será tu portavoz ante el pueblo. Sucederá que él será
para ti una boca, y tú serás para él como Dios. \footnote{\textbf{4:16}
  Éxod 7,1-2} \bibleverse{17} Tomarás esta vara en tu mano, con la que
harás las señales''.

\hypertarget{moisuxe9s-despidiuxe9ndose-de-su-suegro-jetro-instrucciuxf3n-de-dios}{%
\subsection{Moisés despidiéndose de su suegro Jetro; Instrucción de
Dios}\label{moisuxe9s-despidiuxe9ndose-de-su-suegro-jetro-instrucciuxf3n-de-dios}}

\bibleverse{18} Moisés fue y regresó a Jetro, su suegro, y le dijo:
``Por favor, déjame ir y regresar a mis hermanos que están en Egipto, y
ver si todavía están vivos.'' Jetro dijo a Moisés: ``Ve en paz''.
\footnote{\textbf{4:18} Éxod 3,1}

\bibleverse{19} Yahvé dijo a Moisés en Madián: ``Ve, vuelve a Egipto,
porque todos los hombres que buscaban tu vida han muerto''. \footnote{\textbf{4:19}
  Mat 2,20}

\bibleverse{20} Moisés tomó a su mujer y a sus hijos, los montó en un
asno y volvió a la tierra de Egipto. Moisés tomó la vara de Dios en su
mano. \footnote{\textbf{4:20} Éxod 18,3-4} \bibleverse{21} El Señor le
dijo a Moisés: ``Cuando vuelvas a Egipto, procura hacer ante el Faraón
todas las maravillas que he puesto en tu mano, pero yo endureceré su
corazón y no dejará ir al pueblo. \footnote{\textbf{4:21} Éxod 7,3; Éxod
  7,13; Éxod 8,15; Éxod 8,19; Éxod 8,32; Éxod 9,12; Éxod 9,23; Éxod
  10,1; Éxod 10,20; Éxod 10,27; Éxod 11,10; Éxod 14,4; Éxod 14,17}
\bibleverse{22} Le dirás al faraón: ``Yahvé dice: Israel es mi hijo, mi
primogénito, \footnote{\textbf{4:22} Jer 31,9; Os 11,1} \bibleverse{23}
y yo te he dicho: ``Deja ir a mi hijo para que me sirva'', y tú te has
negado a dejarlo ir. He aquí que voy a matar a tu hijo primogénito'\,''.
\footnote{\textbf{4:23} Éxod 11,5; Éxod 12,29}

\hypertarget{dios-quiso-matar-a-moisuxe9s-en-la-noche}{%
\subsection{Dios quiso matar a Moisés en la
noche}\label{dios-quiso-matar-a-moisuxe9s-en-la-noche}}

\bibleverse{24} En el camino, en un lugar de alojamiento, Yahvé se
encontró con Moisés y quiso matarlo. \footnote{\textbf{4:24} Gén 17,14}
\bibleverse{25} Entonces Séfora tomó un pedernal, cortó el prepucio de
su hijo y lo arrojó a sus pies, y dijo: ``Ciertamente eres un novio de
sangre para mí''. \footnote{\textbf{4:25} Jos 5,2}

\bibleverse{26} Así que lo dejó en paz. Entonces le dijo: ``Eres un
novio de sangre'', a causa de la circuncisión.

\hypertarget{moisuxe9s-y-aaruxf3n-encontraron-fe-entre-los-israelitas-en-egipto}{%
\subsection{Moisés y Aarón encontraron fe entre los israelitas en
Egipto}\label{moisuxe9s-y-aaruxf3n-encontraron-fe-entre-los-israelitas-en-egipto}}

\bibleverse{27} Yahvé dijo a Aarón: ``Ve al desierto a recibir a
Moisés''. Fue, y se encontró con él en el monte de Dios, y lo besó.
\bibleverse{28} Moisés le contó a Aarón todas las palabras de Yahvé con
las que lo había enviado, y todas las señales con las que lo había
instruido. \bibleverse{29} Moisés y Aarón fueron y reunieron a todos los
ancianos de los hijos de Israel. \bibleverse{30} Aarón pronunció todas
las palabras que Yahvé había dicho a Moisés, e hizo las señales a la
vista del pueblo. \bibleverse{31} El pueblo creyó, y al oír que el Señor
había visitado a los hijos de Israel y que había visto su aflicción,
inclinaron la cabeza y adoraron. \footnote{\textbf{4:31} Éxod 3,16}

\hypertarget{la-primera-negociaciuxf3n-fallida-con-el-farauxf3n}{%
\subsection{La primera negociación fallida con el
faraón}\label{la-primera-negociaciuxf3n-fallida-con-el-farauxf3n}}

\hypertarget{section-4}{%
\section{5}\label{section-4}}

\bibleverse{1} Después vinieron Moisés y Aarón y le dijeron al Faraón:
``Esto es lo que dice Yahvé, el Dios de Israel: `Deja ir a mi pueblo
para que me celebre una fiesta en el desierto'\,''. \footnote{\textbf{5:1}
  Éxod 3,18; Éxod 7,16; Éxod 8,1; Éxod 8,20; Éxod 9,1; Éxod 9,13}

\bibleverse{2} El faraón dijo: ``¿Quién es Yahvé, para que yo escuche su
voz para dejar ir a Israel? No conozco a Yahvé, y además no dejaré ir a
Israel''. \footnote{\textbf{5:2} Dan 3,15}

\bibleverse{3} Dijeron: ``El Dios de los hebreos se ha reunido con
nosotros. Por favor, vayamos tres días de camino al desierto y
ofrezcamos sacrificios a Yahvé, nuestro Dios, no sea que caiga sobre
nosotros la peste o la espada.''

\bibleverse{4} El rey de Egipto les dijo: ``¿Por qué ustedes, Moisés y
Aarón, sacan al pueblo de su trabajo? Volved a vuestras cargas''.
\bibleverse{5} El faraón dijo: ``He aquí que el pueblo de la tierra es
ahora numeroso, y ustedes lo hacen descansar de sus cargas''.
\footnote{\textbf{5:5} Éxod 1,7; Éxod 1,12}

\hypertarget{el-pueblo-estuxe1-auxfan-muxe1s-oprimida-los-israelitas-reprochan-amargamente-a-moisuxe9s-y-a-aaruxf3n}{%
\subsection{El pueblo está aún más oprimida; los israelitas reprochan
amargamente a Moisés y a
Aarón}\label{el-pueblo-estuxe1-auxfan-muxe1s-oprimida-los-israelitas-reprochan-amargamente-a-moisuxe9s-y-a-aaruxf3n}}

\bibleverse{6} Ese mismo día el faraón ordenó a los capataces del pueblo
y a sus oficiales, diciendo: \bibleverse{7} ``Ya no le daréis al pueblo
paja para hacer ladrillos, como antes. Que vayan a recoger paja por sí
mismos. \bibleverse{8} Les exigiréis el número de ladrillos que antes
hacían. No disminuirás nada de ella, porque son ociosos. Por eso claman
diciendo: `Vamos a sacrificar a nuestro Dios'. \bibleverse{9} Deja que
el trabajo más pesado recaiga sobre los hombres, para que trabajen en
él. Que no presten atención a las palabras mentirosas''.

\bibleverse{10} Los capataces del pueblo salieron con sus oficiales y
hablaron al pueblo diciendo: ``Esto es lo que dice el Faraón: `No les
daré paja. \bibleverse{11} Vayan ustedes mismos, consigan paja donde
puedan encontrarla, porque nada de su trabajo será disminuido'\,''.
\bibleverse{12} Así que el pueblo se dispersó por toda la tierra de
Egipto para recoger rastrojos para hacer paja. \bibleverse{13} Los
capataces urgían diciendo: ``¡Cumplan su cuota de trabajo diariamente,
como cuando había paja!'' \bibleverse{14} Los oficiales de los hijos de
Israel, que los capataces del faraón habían puesto sobre ellos, fueron
golpeados y se les preguntó: ``¿Por qué no habéis cumplido vuestra cuota
tanto ayer como hoy, haciendo ladrillos como antes?''

\bibleverse{15} Entonces los oficiales de los hijos de Israel vinieron y
gritaron al Faraón, diciendo: ``¿Por qué tratas así a tus siervos?
\bibleverse{16} No se les da paja a tus siervos, y nos dicen: `¡Haz
ladrillo!' y he aquí que tus siervos son golpeados; pero la culpa es de
tu propio pueblo.'' \footnote{\textbf{5:16} 1Re 1,21}

\bibleverse{17} Pero el Faraón dijo: ``¡Estás ocioso! ¡Estás ocioso! Por
eso dices: `Vamos a sacrificar a Yavé'. \bibleverse{18} ¡Vayan, pues,
ahora y trabajen, porque no se les dará paja, pero entregarán el mismo
número de ladrillos!''

\bibleverse{19} Los oficiales de los hijos de Israel vieron que estaban
en problemas cuando se les dijo: ``¡No disminuirás nada de tu cuota
diaria de ladrillos!''

\bibleverse{20} Se encontraron con Moisés y Aarón, que estaban en el
camino, cuando salían del Faraón. \bibleverse{21} Les dijeron: ``¡Que
Yahvé los mire y los juzgue, porque ustedes nos han convertido en un
hedor abominable a los ojos del Faraón y a los ojos de sus siervos, para
poner una espada en su mano para matarnos!'' \footnote{\textbf{5:21} Gén
  34,30}

\hypertarget{el-lamento-de-moisuxe9s-y-la-promesa-de-dios}{%
\subsection{El lamento de Moisés y la promesa de
Dios}\label{el-lamento-de-moisuxe9s-y-la-promesa-de-dios}}

\bibleverse{22} Moisés volvió a Yahvé y le dijo: ``Señor, ¿por qué has
traído problemas a este pueblo? ¿Por qué me has enviado? \bibleverse{23}
Porque desde que vine al Faraón a hablar en tu nombre, él ha traído
problemas a este pueblo. No has rescatado a tu pueblo en absoluto''.

\hypertarget{section-5}{%
\section{6}\label{section-5}}

\bibleverse{1} Yahvé dijo a Moisés: ``Ahora verás lo que haré con el
Faraón, porque con mano fuerte los dejará ir, y con mano fuerte los
expulsará de su tierra.'' \footnote{\textbf{6:1} Éxod 11,1; Éxod 12,33}

\hypertarget{dios-se-revela-a-suxed-mismo-a-moisuxe9s-de-nuevo-y-promete-la-salvaciuxf3n-del-pueblo}{%
\subsection{Dios se revela a sí mismo a Moisés de nuevo y promete la
salvación del
pueblo}\label{dios-se-revela-a-suxed-mismo-a-moisuxe9s-de-nuevo-y-promete-la-salvaciuxf3n-del-pueblo}}

\bibleverse{2} Dios habló a Moisés y le dijo: ``Yo soy Yavé.
\bibleverse{3} Me presenté a Abraham, a Isaac y a Jacob como Dios
Todopoderoso; pero por mi nombre Yahvé no me conocieron. \footnote{\textbf{6:3}
  Gén 17,1; Éxod 3,14-15} \bibleverse{4} También he establecido mi pacto
con ellos, para darles la tierra de Canaán, la tierra de sus viajes, en
la que vivían como extranjeros. \footnote{\textbf{6:4} Gén 12,7}
\bibleverse{5} Además, he oído el gemido de los hijos de Israel, a
quienes los egipcios mantienen en esclavitud, y me he acordado de mi
pacto. \bibleverse{6} Por tanto, di a los hijos de Israel: ``Yo soy
Yahvé, y os sacaré de las cargas de los egipcios, y os libraré de su
esclavitud, y os redimiré con brazo extendido y con grandes juicios.
\bibleverse{7} Os tomaré para mí como pueblo. Yo seré tu Dios; y sabrás
que yo soy Yahvé, tu Dios, que te saca de las cargas de los egipcios.
\bibleverse{8} Os llevaré a la tierra que juré dar a Abraham, a Isaac y
a Jacob, y os la daré en herencia: yo soy Yahvé''. \footnote{\textbf{6:8}
  Gén 22,16; Deut 32,40}

\hypertarget{moisuxe9s-rechazado-por-su-pueblo-desesperado-recibe-nuevas-instrucciones-de-dios}{%
\subsection{Moisés, rechazado por su pueblo desesperado, recibe nuevas
instrucciones de
Dios}\label{moisuxe9s-rechazado-por-su-pueblo-desesperado-recibe-nuevas-instrucciones-de-dios}}

\bibleverse{9} Moisés habló así a los hijos de Israel, pero ellos no
escucharon a Moisés por la angustia de espíritu y la cruel esclavitud.

\bibleverse{10} Yahvé habló a Moisés diciendo: \bibleverse{11} ``Entra y
habla con el faraón, rey de Egipto, para que deje salir a los hijos de
Israel de su tierra.''

\bibleverse{12} Moisés habló ante el Señor diciendo: ``He aquí que los
hijos de Israel no me han escuchado. ¿Cómo, pues, me va a escuchar el
faraón, si tengo los labios incircuncisos?'' \footnote{\textbf{6:12}
  Éxod 6,30; Éxod 4,10} \bibleverse{13} Yahvé habló a Moisés y a Aarón y
les dio una orden a los hijos de Israel y al faraón, rey de Egipto, para
que sacaran a los hijos de Israel de la tierra de Egipto.

\hypertarget{uxe1rbol-genealuxf3gico-de-aarons-y-moisuxe9s}{%
\subsection{Árbol genealógico de Aarons y
Moisés}\label{uxe1rbol-genealuxf3gico-de-aarons-y-moisuxe9s}}

\bibleverse{14} Estos son los jefes de las casas de sus padres. Los
hijos de Rubén, primogénito de Israel: Hanoc, Pallu, Hezrón y Carmi;
estas son las familias de Rubén. \footnote{\textbf{6:14} Gén 46,9-11}
\bibleverse{15} Los hijos de Simeón: Jemuel, Jamín, Ohad, Jacín, Zohar y
Shaúl, hijo de una cananea; estas son las familias de Simeón.
\bibleverse{16} Estos son los nombres de los hijos de Leví según sus
generaciones Gersón, Coat y Merari; y los años de la vida de Leví fueron
ciento treinta y siete años. \footnote{\textbf{6:16} 1Cró 6,1-4; 1Cró
  6,16-19} \bibleverse{17} Los hijos de Gersón: Libni y Simei, según sus
familias. \bibleverse{18} Los hijos de Coat Amram, Izhar, Hebrón y
Uziel; y los años de la vida de Coat fueron ciento treinta y tres años.
\bibleverse{19} Los hijos de Merari: Mahli y Mushi. Estas son las
familias de los levitas según sus generaciones. \bibleverse{20} Amram
tomó como esposa a Jocabed, hermana de su padre, y ella le dio a luz a
Aarón y a Moisés. Los años de la vida de Amram fueron ciento treinta y
siete años. \footnote{\textbf{6:20} Éxod 2,1} \bibleverse{21} Los hijos
de Izhar Coré, Nefeg y Zicri. \footnote{\textbf{6:21} Núm 16,1}
\bibleverse{22} Los hijos de Uziel: Misael, Elzafán y Sithri.
\footnote{\textbf{6:22} Lev 10,4} \bibleverse{23} Aarón tomó por esposa
a Elisaba, hija de Aminadab, hermana de Naasón, y ella le dio a luz a
Nadab y Abiú, Eleazar e Itamar. \footnote{\textbf{6:23} Éxod 28,1}
\bibleverse{24} Los hijos de Coré: Asir, Elcana y Abiasaf; estas son las
familias de los corasitas. \footnote{\textbf{6:24} Núm 25,7}
\bibleverse{25} Eleazar, hijo de Aarón, tomó por mujer a una de las
hijas de Putiel, y ella le dio a luz a Finees. Estos son los jefes de
las casas paternas de los levitas según sus familias. \bibleverse{26}
Estos son aquel Aarón y aquel Moisés a quienes Yahvé dijo: ``Sacad a los
hijos de Israel de la tierra de Egipto según sus ejércitos.''
\bibleverse{27} Estos son los que hablaron con el faraón, rey de Egipto,
para sacar a los hijos de Israel de Egipto. Estos son ese Moisés y ese
Aarón.

\hypertarget{la-nueva-misiuxf3n-de-moisuxe9s-y-aaruxf3n-ante-el-farauxf3n}{%
\subsection{La nueva misión de Moisés y Aarón ante el
faraón}\label{la-nueva-misiuxf3n-de-moisuxe9s-y-aaruxf3n-ante-el-farauxf3n}}

\bibleverse{28} El día en que Yahvé habló con Moisés en la tierra de
Egipto, \bibleverse{29} Yahvé dijo a Moisés: ``Yo soy Yahvé. Dile al
Faraón, rey de Egipto, todo lo que te digo''.

\bibleverse{30} Moisés dijo ante Yahvé: ``He aquí que soy de labios
incircuncisos, ¿y cómo me escuchará el Faraón?'' \footnote{\textbf{6:30}
  Éxod 6,12}

\hypertarget{section-6}{%
\section{7}\label{section-6}}

\bibleverse{1} Yahvé dijo a Moisés: ``He aquí que te he puesto como Dios
ante el Faraón, y Aarón tu hermano será tu profeta. \footnote{\textbf{7:1}
  Éxod 4,16} \bibleverse{2} Tú hablarás todo lo que yo te mande; y Aarón
tu hermano hablará al Faraón para que deje salir a los hijos de Israel
de su tierra. \bibleverse{3} Yo endureceré el corazón del Faraón, y
multiplicaré mis señales y mis prodigios en la tierra de Egipto.
\footnote{\textbf{7:3} Éxod 4,21} \bibleverse{4} Pero el Faraón no te
escuchará, así que pondré mi mano sobre Egipto y sacaré a mis ejércitos,
a mi pueblo los hijos de Israel, de la tierra de Egipto con grandes
juicios. \bibleverse{5} Los egipcios sabrán que yo soy Yavé cuando
extienda mi mano sobre Egipto y saque a los hijos de Israel de entre
ellos.'' \footnote{\textbf{7:5} Éxod 8,22; Éxod 9,14; Éxod 9,29}

\bibleverse{6} Moisés y Aarón lo hicieron. Como el Señor les ordenó, así
lo hicieron. \bibleverse{7} Moisés tenía ochenta años, y Aarón ochenta y
tres, cuando hablaron con el faraón.

\hypertarget{transformaciuxf3n-del-bastuxf3n-en-serpiente}{%
\subsection{Transformación del bastón en
serpiente}\label{transformaciuxf3n-del-bastuxf3n-en-serpiente}}

\bibleverse{8} Yahvé habló a Moisés y a Aarón, diciendo: \bibleverse{9}
``Cuando el Faraón les hable diciendo: `Hagan un milagro', entonces le
dirán a Aarón: `Toma tu vara y arrójala ante el Faraón, y se convertirá
en una serpiente'\,''. \footnote{\textbf{7:9} Éxod 4,3}

\bibleverse{10} Moisés y Aarón se presentaron ante el faraón, y así lo
hicieron, tal como lo había ordenado el Señor. Aarón arrojó su vara ante
el Faraón y ante sus servidores, y se convirtió en una serpiente.
\bibleverse{11} Entonces el faraón llamó también a los sabios y a los
hechiceros. Ellos también, los magos de Egipto, hicieron lo mismo con
sus encantamientos. \footnote{\textbf{7:11} Éxod 7,22; Éxod 8,7; Éxod
  8,18-19; 2Tim 3,8} \bibleverse{12} Cada uno de ellos arrojó su vara y
se convirtió en serpiente; pero la vara de Aarón se tragó sus varas.
\bibleverse{13} El corazón del faraón se endureció y no los escuchó,
como había dicho el Señor. \footnote{\textbf{7:13} Éxod 4,21}

\hypertarget{la-primer-plaga-transformaciuxf3n-de-las-aguas-del-nilo-en-sangre}{%
\subsection{La primer plaga: Transformación de las aguas del Nilo en
sangre}\label{la-primer-plaga-transformaciuxf3n-de-las-aguas-del-nilo-en-sangre}}

\bibleverse{14} El Señor dijo a Moisés: ``El corazón del faraón es
obstinado. Se niega a dejar ir al pueblo. \bibleverse{15} Ve a ver al
faraón por la mañana. He aquí que él va a salir al agua. Tú estarás a la
orilla del río para recibirlo. Tomarás en tu mano la vara convertida en
serpiente. \bibleverse{16} Le dirás: ``Yahvé, el Dios de los hebreos, me
ha enviado a ti, diciendo: ``Deja ir a mi pueblo, para que me sirva en
el desierto. He aquí que hasta ahora no has hecho caso''. \footnote{\textbf{7:16}
  Éxod 5,1} \bibleverse{17} Yahvé dice: ``En esto sabrán que yo soy
Yahvé. Miren: Golpearé con la vara que tengo en mi mano las aguas que
están en el río, y se convertirán en sangre. \footnote{\textbf{7:17}
  Éxod 4,9} \bibleverse{18} Los peces que están en el río morirán y el
río se ensuciará. Los egipcios detestarán beber agua del río''.
\bibleverse{19} Yahvé dijo a Moisés: ``Dile a Aarón: `Toma tu vara y
extiende tu mano sobre las aguas de Egipto, sobre sus ríos, sobre sus
arroyos y sobre sus estanques, y sobre todos sus estanques de agua, para
que se conviertan en sangre. Habrá sangre en toda la tierra de Egipto,
tanto en los recipientes de madera como en los de piedra'\,''.
\footnote{\textbf{7:19} Apoc 11,6}

\bibleverse{20} Moisés y Aarón lo hicieron así, tal como lo había
ordenado el Señor, y él levantó la vara y golpeó las aguas que estaban
en el río, a la vista del Faraón y de sus servidores, y todas las aguas
que estaban en el río se convirtieron en sangre. \bibleverse{21} Los
peces que había en el río murieron. El río se volvió fétido. Los
egipcios no podían beber agua del río. La sangre se extendió por toda la
tierra de Egipto. \bibleverse{22} Los magos de Egipto hicieron lo mismo
con sus encantamientos. Así que el corazón del faraón se endureció y no
les hizo caso, como había dicho el Señor. \footnote{\textbf{7:22} Éxod
  7,11} \bibleverse{23} El faraón se volvió y entró en su casa, y ni
siquiera lo tomó en cuenta. \bibleverse{24} Todos los egipcios cavaron
alrededor del río en busca de agua para beber, porque no podían beber el
agua del río. \bibleverse{25} Se cumplieron siete días, después de que
el Señor golpeó el río.

\hypertarget{la-segunda-plaga-de-las-ranas}{%
\subsection{La segunda plaga de las
ranas}\label{la-segunda-plaga-de-las-ranas}}

\hypertarget{section-7}{%
\section{8}\label{section-7}}

\bibleverse{1} Yahvé le habló a Moisés: ``Ve a Faraón y dile: `Esto es
lo que dice Yahvé: ``Deja ir a mi pueblo para que me sirva. \footnote{\textbf{8:1}
  Éxod 5,1} \bibleverse{2} Si te niegas a dejarlos ir, he aquí que yo
plagaré de ranas todas tus fronteras. \bibleverse{3} El río se llenará
de ranas, que subirán y entrarán en tu casa, en tu dormitorio, en tu
cama, en la casa de tus siervos y en tu pueblo, en tus hornos y en tus
amasadoras. \bibleverse{4} Las ranas subirán sobre ti, sobre tu pueblo y
sobre todos tus servidores''. \bibleverse{5} Yahvé dijo a Moisés: ``Dile
a Aarón: `Extiende tu mano con tu vara sobre los ríos, sobre los arroyos
y sobre los estanques, y haz que las ranas suban sobre la tierra de
Egipto'\,''. \bibleverse{6} Aarón extendió su mano sobre las aguas de
Egipto, y las ranas subieron y cubrieron la tierra de Egipto.
\bibleverse{7} Los magos hicieron lo mismo con sus encantamientos, e
hicieron subir ranas sobre la tierra de Egipto. \footnote{\textbf{8:7}
  Éxod 7,11}

\bibleverse{8} Entonces el faraón llamó a Moisés y a Aarón y les dijo:
``Rogad a Yahvé que quite las ranas de mí y de mi pueblo, y dejaré ir al
pueblo para que ofrezca sacrificios a Yahvé.'' \footnote{\textbf{8:8}
  Éxod 8,28; Éxod 9,28; Éxod 10,17}

\bibleverse{9} Moisés dijo al Faraón: ``Te concedo el honor de fijar el
tiempo en que debo orar por ti, por tus siervos y por tu pueblo, para
que las ranas desaparezcan de ti y de tus casas, y se queden sólo en el
río.''

\bibleverse{10} El faraón dijo: ``Mañana''. Moisés dijo: ``Que sea según
tu palabra, para que sepas que no hay nadie como Yahvé, nuestro Dios.
\footnote{\textbf{8:10} Éxod 9,14; Éxod 15,11} \bibleverse{11} Las ranas
se apartarán de ti, de tus casas, de tus siervos y de tu pueblo. Se
quedarán sólo en el río''.

\bibleverse{12} Moisés y Aarón salieron del Faraón, y Moisés clamó a
Yahvé por las ranas que había traído sobre el Faraón. \bibleverse{13} El
Señor hizo lo que le dijo Moisés, y las ranas murieron en las casas, en
los patios y en los campos. \bibleverse{14} Las juntaron en montones, y
la tierra apestaba. \bibleverse{15} Pero cuando el faraón vio que había
un respiro, endureció su corazón y no les hizo caso, como había dicho el
Señor. \footnote{\textbf{8:15} Éxod 4,21}

\hypertarget{la-tercera-plaga-mosquitos}{%
\subsection{La tercera plaga:
Mosquitos}\label{la-tercera-plaga-mosquitos}}

\bibleverse{16} Yahvé dijo a Moisés: ``Dile a Aarón: `Extiende tu vara y
golpea el polvo de la tierra para que se convierta en piojos en toda la
tierra de Egipto'\,''. \bibleverse{17} Así lo hicieron; y Aarón extendió
su mano con su vara y golpeó el polvo de la tierra, y hubo piojos en los
hombres y en los animales; todo el polvo de la tierra se convirtió en
piojos en toda la tierra de Egipto. \bibleverse{18} Los magos intentaron
con sus encantamientos producir piojos, pero no pudieron. Había piojos
en los hombres y en los animales. \footnote{\textbf{8:18} Éxod 9,11}
\bibleverse{19} Entonces los magos le dijeron al faraón: ``Este es el
dedo de Dios''; pero el corazón del faraón se endureció y no los
escuchó, tal como lo había dicho el Señor. \footnote{\textbf{8:19} Éxod
  14,25; Éxod 4,21}

\hypertarget{cuarto-plaga-moscas-del-perro}{%
\subsection{Cuarto plaga: moscas del
perro}\label{cuarto-plaga-moscas-del-perro}}

\bibleverse{20} El Señor dijo a Moisés: ``Levántate de madrugada y
preséntate ante el Faraón; he aquí que él sale al agua, y dile: ``Esto
es lo que dice el Señor: ``Deja ir a mi pueblo para que me sirva.
\footnote{\textbf{8:20} Éxod 5,1} \bibleverse{21} De lo contrario, si no
dejas ir a mi pueblo, he aquí que enviaré enjambres de moscas sobre ti,
sobre tus siervos y sobre tu pueblo, y a tus casas. Las casas de los
egipcios se llenarán de enjambres de moscas, y también el suelo sobre el
que están. \bibleverse{22} En ese día apartaré la tierra de Gosén, en la
que habita mi pueblo, para que no haya enjambres de moscas, a fin de que
se sepa que yo soy Yavé en la tierra. \footnote{\textbf{8:22} Éxod 7,5}
\bibleverse{23} Pondré una división entre mi pueblo y el tuyo. Esta
señal se producirá para mañana''\,''. \bibleverse{24} Así lo hizo Yavé,
y entraron graves enjambres de moscas en la casa del Faraón y en las
casas de sus servidores. En todo el territorio de Egipto la tierra se
corrompió a causa de los enjambres de moscas.

\bibleverse{25} El faraón llamó a Moisés y a Aarón y les dijo: ``¡Vayan
a sacrificar a su Dios en la tierra!''

\bibleverse{26} Moisés dijo: ``No es conveniente hacerlo, porque vamos a
sacrificar la abominación de los egipcios a Yahvé, nuestro Dios. He aquí
que si sacrificamos la abominación de los egipcios ante sus ojos, ¿no
nos apedrearán? \footnote{\textbf{8:26} Gén 43,32} \bibleverse{27}
Iremos tres días de camino al desierto y sacrificaremos a Yavé, nuestro
Dios, como él nos mande.'' \footnote{\textbf{8:27} Éxod 3,18}

\bibleverse{28} El faraón dijo: ``Te dejaré ir para que ofrezcas
sacrificios a Yavé, tu Dios, en el desierto, sólo que no te irás muy
lejos. Reza por mí''. \footnote{\textbf{8:28} Éxod 8,8}

\bibleverse{29} Moisés dijo: ``He aquí que yo salgo de ti. Rezaré a Yavé
para que los enjambres de moscas se alejen mañana del Faraón, de sus
servidores y de su pueblo; sólo que no permitas que el Faraón siga
actuando con engaño al no dejar que el pueblo vaya a sacrificar a
Yavé.'' \bibleverse{30} Moisés salió del Faraón y oró a Yavé.
\bibleverse{31} El Señor hizo lo que le dijo Moisés, y eliminó los
enjambres de moscas del Faraón, de sus servidores y de su pueblo. No
quedó ni una. \bibleverse{32} El faraón también endureció su corazón
esta vez y no dejó ir al pueblo. \footnote{\textbf{8:32} Éxod 4,21}

\hypertarget{quinta-plaga-plaga-del-ganado}{%
\subsection{Quinta plaga: plaga del
ganado}\label{quinta-plaga-plaga-del-ganado}}

\hypertarget{section-8}{%
\section{9}\label{section-8}}

\bibleverse{1} Entonces Yahvé dijo a Moisés: ``Ve a Faraón y dile: `Esto
es lo que dice Yahvé, el Dios de los hebreos: ``Deja ir a mi pueblo para
que me sirva. \footnote{\textbf{9:1} Éxod 5,1} \bibleverse{2} Porque si
te niegas a dejarlos ir y los retienes, \bibleverse{3} he aquí que la
mano de Yavé está sobre tu ganado que está en el campo, sobre los
caballos, sobre los asnos, sobre los camellos, sobre los rebaños y sobre
las manadas con una peste muy grave. \footnote{\textbf{9:3} Éxod 3,20}
\bibleverse{4} El Señor hará una distinción entre el ganado de Israel y
el de Egipto, y no morirá nada de todo lo que pertenece a los hijos de
Israel''. \bibleverse{5} Yahvé fijó un tiempo determinado, diciendo:
``Mañana Yahvé hará esta cosa en la tierra''. \bibleverse{6} Yahvé hizo
esa cosa al día siguiente; y todo el ganado de Egipto murió, pero del
ganado de los hijos de Israel no murió ni uno. \bibleverse{7} El faraón
envió, y he aquí que no había muerto ni uno solo de los ganados de los
israelitas. Pero el corazón del Faraón era obstinado, y no dejó ir al
pueblo. \footnote{\textbf{9:7} Éxod 4,21}

\hypertarget{la-sexta-plaga-tumores}{%
\subsection{La sexta plaga: tumores}\label{la-sexta-plaga-tumores}}

\bibleverse{8} El Señor dijo a Moisés y a Aarón: ``Tomen puñados de
ceniza del horno y que Moisés la esparza hacia el cielo a la vista del
Faraón. \bibleverse{9} Se convertirá en polvo pequeño sobre toda la
tierra de Egipto, y habrá forúnculos y ampollas que brotarán en el
hombre y en el animal, en toda la tierra de Egipto.'' \footnote{\textbf{9:9}
  Deut 28,27}

\bibleverse{10} Tomaron cenizas del horno y se presentaron ante el
Faraón; Moisés las roció hacia el cielo, y se convirtieron en forúnculos
y ampollas que brotaron en el hombre y en el animal. \footnote{\textbf{9:10}
  Apoc 16,2} \bibleverse{11} Los magos no podían estar delante de Moisés
a causa de los forúnculos, pues los forúnculos estaban en los magos y en
todos los egipcios. \bibleverse{12} El Señor endureció el corazón del
faraón y no les hizo caso, como el Señor le había dicho a Moisés.
\footnote{\textbf{9:12} Éxod 4,21}

\hypertarget{la-suxe9ptima-plaga-el-granizo}{%
\subsection{La séptima plaga: el
granizo}\label{la-suxe9ptima-plaga-el-granizo}}

\bibleverse{13} Yahvé dijo a Moisés: ``Levántate de madrugada y
preséntate ante el faraón y dile: ``Esto es lo que dice Yahvé, el Dios
de los hebreos: ``Deja ir a mi pueblo para que me sirva. \footnote{\textbf{9:13}
  Éxod 5,1} \bibleverse{14} Por esta vez enviaré todas mis plagas contra
tu corazón, contra tus funcionarios y contra tu pueblo, para que sepas
que no hay nadie como yo en toda la tierra. \footnote{\textbf{9:14} Éxod
  8,10} \bibleverse{15} Porque ahora habría extendido mi mano y te
habría herido a ti y a tu pueblo con la peste, y habrías sido eliminado
de la tierra; \bibleverse{16} pero, en verdad, por esta causa te he
puesto en pie: para mostrarte mi poder, y para que mi nombre sea
declarado en toda la tierra, \footnote{\textbf{9:16} Éxod 7,3; Éxod
  14,4; Rom 9,17} \bibleverse{17} porque todavía te exaltas contra mi
pueblo, que no lo dejas ir. \bibleverse{18} He aquí que mañana a esta
hora haré llover un granizo muy fuerte, como no ha habido en Egipto
desde el día de su fundación hasta ahora. \footnote{\textbf{9:18} Job
  38,22} \bibleverse{19} Ordena, pues, que todo tu ganado y todo lo que
tengas en el campo se ponga a resguardo. El granizo caerá sobre todos
los hombres y animales que se encuentren en el campo y no sean llevados
a casa, y morirán''\,''.

\bibleverse{20} Los que temían la palabra de Yavé entre los siervos del
Faraón hicieron huir a sus siervos y a sus ganados a las casas.
\bibleverse{21} Los que no respetaron la palabra de Yavé dejaron a sus
siervos y a su ganado en el campo.

\bibleverse{22} Yahvé dijo a Moisés: ``Extiende tu mano hacia el cielo,
para que haya granizo en toda la tierra de Egipto, sobre el hombre,
sobre el animal y sobre toda hierba del campo, en toda la tierra de
Egipto.''

\bibleverse{23} Moisés extendió su vara hacia el cielo, y el Señor envió
truenos y granizo, y los relámpagos cayeron sobre la tierra. El Señor
hizo llover granizo sobre la tierra de Egipto. \footnote{\textbf{9:23}
  Apoc 16,21} \bibleverse{24} Hubo un granizo muy fuerte, y los
relámpagos se mezclaron con el granizo, como no había habido en toda la
tierra de Egipto desde que se convirtió en una nación. \bibleverse{25}
El granizo hirió en toda la tierra de Egipto a todo lo que estaba en el
campo, tanto a los hombres como a los animales; y el granizo hirió toda
hierba del campo, y quebró todo árbol del campo. \bibleverse{26} Sólo en
la tierra de Gosén, donde estaban los hijos de Israel, no hubo granizo.

\bibleverse{27} El faraón mandó llamar a Moisés y a Aarón y les dijo:
``Esta vez he pecado. El Señor es justo, y yo y mi pueblo somos impíos.
\footnote{\textbf{9:27} Éxod 10,16} \bibleverse{28} Recen a Yavé, porque
ya ha habido bastantes truenos y granizo. Los dejaré ir, y no se
quedarán por más tiempo''. \footnote{\textbf{9:28} Éxod 8,8}

\bibleverse{29} Moisés le dijo: ``En cuanto salga de la ciudad,
extenderé mis manos a Yahvé. Cesarán los truenos y no habrá más granizo,
para que sepas que la tierra es de Yavé. \bibleverse{30} Pero en cuanto
a ti y a tus siervos, sé que aún no temes a Yavé Dios''.

\bibleverse{31} El lino y la cebada fueron golpeados, porque la cebada
había madurado y el lino estaba floreciendo. \bibleverse{32} Pero el
trigo y la escanda no fueron golpeados, porque no habían crecido.
\bibleverse{33} Moisés salió de la ciudad del Faraón y extendió sus
manos a Yahvé; y cesaron los truenos y el granizo, y no se derramó la
lluvia sobre la tierra. \bibleverse{34} Cuando el Faraón vio que la
lluvia, el granizo y los truenos habían cesado, pecó aún más, y
endureció su corazón, él y sus siervos. \bibleverse{35} El corazón del
faraón se endureció, y no dejó ir a los hijos de Israel, tal como Yahvé
había dicho por medio de Moisés. \footnote{\textbf{9:35} Éxod 4,21}

\hypertarget{la-octava-plaga-langostas}{%
\subsection{La octava plaga:
langostas}\label{la-octava-plaga-langostas}}

\hypertarget{section-9}{%
\section{10}\label{section-9}}

\bibleverse{1} Yahvé dijo a Moisés: ``Entra a Faraón, porque he
endurecido su corazón y el de sus siervos, para mostrar estas mis
señales entre ellos; \bibleverse{2} y para que cuentes a tu hijo y al
hijo de tu hijo las cosas que he hecho a Egipto y mis señales que he
realizado entre ellos, para que sepas que yo soy Yahvé.'' \footnote{\textbf{10:2}
  Éxod 6,2-7}

\bibleverse{3} Moisés y Aarón fueron a ver al faraón y le dijeron:
``Esto es lo que dice el Señor, el Dios de los hebreos: `¿Hasta cuándo
te negarás a humillarte ante mí? Deja ir a mi pueblo para que me sirva.
\footnote{\textbf{10:3} Éxod 5,3} \bibleverse{4} O bien, si te niegas a
dejar ir a mi pueblo, he aquí que mañana traeré langostas a tu país,
\bibleverse{5} y cubrirán la superficie de la tierra, de modo que no se
podrá ver la tierra. Se comerán el residuo de lo que se haya escapado,
lo que os quede del granizo, y se comerán todo árbol que crezca para
vosotros del campo. \bibleverse{6} Tus casas se llenarán, y las casas de
todos tus siervos, y las casas de todos los egipcios, como nunca vieron
tus padres ni los padres de tus padres, desde el día en que estuvieron
en la tierra hasta hoy.'\,'' Se volvió y salió del Faraón.

\bibleverse{7} Los siervos del faraón le dijeron: ``¿Hasta cuándo será
este hombre una trampa para nosotros? Deja ir a los hombres para que
sirvan a Yavé, su Dios. ¿Aún no sabes que Egipto está destruido?''

\bibleverse{8} Moisés y Aarón fueron llevados de nuevo ante el Faraón, y
éste les dijo: ``Id, servid a Yahvé vuestro Dios; pero ¿quiénes son los
que irán?''

\bibleverse{9} Moisés dijo: ``Iremos con nuestros jóvenes y nuestros
ancianos. Iremos con nuestros hijos y con nuestras hijas, con nuestros
rebaños y con nuestras manadas; porque debemos celebrar una fiesta a
Yahvé''. \footnote{\textbf{10:9} Éxod 5,1}

\bibleverse{10} Les dijo: ``¡Que el Señor esté con vosotros si os dejo
ir con vuestros pequeños! Ved que el mal está claramente ante vuestras
caras. \bibleverse{11} ¡No es así! Vayan ahora ustedes, que son hombres,
y sirvan a Yahvé; ¡pues eso es lo que desean!'' Entonces fueron
expulsados de la presencia del faraón.

\bibleverse{12} Yahvé dijo a Moisés: ``Extiende tu mano sobre la tierra
de Egipto para que suban las langostas sobre la tierra de Egipto y coman
toda la hierba de la tierra, todo lo que el granizo ha dejado.''
\footnote{\textbf{10:12} Éxod 9,32} \bibleverse{13} Moisés extendió su
vara sobre la tierra de Egipto, y el Señor trajo un viento del este
sobre la tierra durante todo ese día y toda la noche; y cuando amaneció,
el viento del este trajo las langostas. \bibleverse{14} Las langostas
subieron por toda la tierra de Egipto y se posaron en todos los límites
de Egipto. Eran muy graves. Antes de ellas no hubo langostas como ellas,
ni las habrá jamás. \bibleverse{15} Porque cubrieron la superficie de
toda la tierra, de modo que la tierra se oscureció, y se comieron toda
la hierba de la tierra y todo el fruto de los árboles que había dejado
el granizo. No quedó nada verde, ni árbol ni hierba del campo, en toda
la tierra de Egipto. \bibleverse{16} Entonces el faraón llamó a Moisés y
a Aarón a toda prisa, y dijo: ``He pecado contra el Señor, tu Dios, y
contra ti. \footnote{\textbf{10:16} Éxod 9,27} \bibleverse{17} Ahora,
por favor, perdonen de nuevo mi pecado, y rueguen a Yavé su Dios, para
que también me quite esta muerte.'' \footnote{\textbf{10:17} Éxod 8,8;
  1Sam 12,19}

\bibleverse{18} Moisés salió del Faraón y oró al Señor. \footnote{\textbf{10:18}
  Núm 11,2} \bibleverse{19} El Señor envió un fortísimo viento del oeste
que arrastró a las langostas y las arrojó al Mar Rojo. No quedó ni una
sola langosta en todos los límites de Egipto. \bibleverse{20} Pero Yavé
endureció el corazón del faraón, y no dejó ir a los hijos de Israel.
\footnote{\textbf{10:20} Éxod 4,21}

\hypertarget{la-novena-plaga-oscuridad}{%
\subsection{La novena plaga:
oscuridad}\label{la-novena-plaga-oscuridad}}

\bibleverse{21} Yahvé dijo a Moisés: ``Extiende tu mano hacia el cielo,
para que haya tinieblas sobre la tierra de Egipto, tinieblas que se
puedan sentir.'' \bibleverse{22} Moisés extendió su mano hacia el cielo,
y hubo una densa oscuridad en toda la tierra de Egipto durante tres
días. \bibleverse{23} No se veían unos a otros, y nadie se levantó de su
lugar durante tres días; pero todos los hijos de Israel tenían luz en
sus moradas.

\bibleverse{24} El faraón llamó a Moisés y le dijo: ``Ve y sirve a
Yahvé. Sólo deja que tus rebaños y tus manadas se queden atrás. Que tus
pequeños también vayan contigo''. \footnote{\textbf{10:24} Éxod 10,10}

\bibleverse{25} Moisés dijo: ``También debes entregar en nuestras manos
sacrificios y holocaustos, para que ofrezcamos sacrificios a Yahvé,
nuestro Dios. \bibleverse{26} Nuestro ganado también irá con nosotros.
No se dejará ni una pezuña, pues de ella debemos tomar para servir a
Yavé nuestro Dios; y no sabemos con qué debemos servir a Yavé, hasta que
lleguemos allí.''

\bibleverse{27} Pero Yahvé endureció el corazón del faraón y no los dejó
ir. \footnote{\textbf{10:27} Éxod 4,21} \bibleverse{28} El faraón le
dijo: ``¡Aléjate de mí! Cuídate de no ver más mi rostro, porque el día
que veas mi rostro morirás''.

\bibleverse{29} Moisés dijo: ``Has hablado bien. No volveré a ver tu
rostro''.

\hypertarget{anuncio-de-la-duxe9cima-plaga-la-muerte-del-primoguxe9nito}{%
\subsection{Anuncio de la décima plaga, la muerte del
primogénito}\label{anuncio-de-la-duxe9cima-plaga-la-muerte-del-primoguxe9nito}}

\hypertarget{section-10}{%
\section{11}\label{section-10}}

\bibleverse{1} El Señor le dijo a Moisés: ``Traeré una plaga más sobre
el Faraón y sobre Egipto; después te dejará ir. Cuando los deje ir,
seguramente los expulsará por completo. \bibleverse{2} Habla ahora en
los oídos del pueblo, y que cada hombre pida a su prójimo, y cada mujer
a su prójimo, joyas de plata y joyas de oro.'' \footnote{\textbf{11:2}
  Éxod 3,21-22} \bibleverse{3} Yahvé le dio al pueblo el favor a los
ojos de los egipcios. Además, el hombre Moisés era muy grande en la
tierra de Egipto, a los ojos de los siervos del faraón y del pueblo.

\bibleverse{4} Moisés dijo: ``Esto es lo que dice el Señor: `Hacia la
medianoche saldré al centro de Egipto, \bibleverse{5} y morirán todos
los primogénitos en la tierra de Egipto, desde el primogénito del Faraón
que se sienta en su trono, hasta el primogénito de la sierva que está
detrás del molino, y todos los primogénitos del ganado. \footnote{\textbf{11:5}
  Éxod 4,23} \bibleverse{6} Habrá un gran clamor en toda la tierra de
Egipto, como no lo ha habido ni lo habrá nunca. \bibleverse{7} Pero
contra cualquiera de los hijos de Israel ni siquiera ladrará un perro ni
moverá la lengua, ni contra el hombre ni contra el animal, para que
sepas que el Señor hace distinción entre los egipcios e Israel.
\footnote{\textbf{11:7} Éxod 9,4; Éxod 9,26} \bibleverse{8} Todos estos
siervos tuyos bajarán a mí y se inclinarán ante mí, diciendo: ``Sal, con
todo el pueblo que te sigue''; y después de eso saldré yo''. Salió del
Faraón con gran ira.

\bibleverse{9} Yahvé dijo a Moisés: ``El faraón no te escuchará, para
que mis maravillas se multipliquen en la tierra de Egipto''.
\bibleverse{10} Moisés y Aarón hicieron todos estos prodigios ante el
faraón, pero Yahvé endureció el corazón del faraón, y no dejó salir a
los hijos de Israel de su tierra. \footnote{\textbf{11:10} Éxod 4,21}

\hypertarget{instituciuxf3n-de-pascua}{%
\subsection{Institución de Pascua}\label{instituciuxf3n-de-pascua}}

\hypertarget{section-11}{%
\section{12}\label{section-11}}

\bibleverse{1} Yahvé habló a Moisés y a Aarón en la tierra de Egipto,
diciendo: \bibleverse{2} ``Este mes será para vosotros el principio de
los meses. Será para ustedes el primer mes del año. \footnote{\textbf{12:2}
  Éxod 13,4} \bibleverse{3} Hablad a toda la congregación de Israel,
diciendo: ``El día diez de este mes, cada uno tomará un cordero, según
las casas de sus padres, un cordero por familia; \bibleverse{4} y si la
familia es demasiado pequeña para un cordero, entonces él y su vecino de
al lado tomarán uno según el número de las almas. Harás la cuenta del
cordero según lo que cada uno pueda comer. \bibleverse{5} Tu cordero
será sin defecto, un macho de un año. Lo tomarás de las ovejas o de las
cabras. \footnote{\textbf{12:5} Lev 22,20} \bibleverse{6} Lo guardarás
hasta el día catorce del mismo mes; y toda la asamblea de la
congregación de Israel lo matará al atardecer. \bibleverse{7} Tomarán un
poco de la sangre y la pondrán en los dos postes de la puerta y en el
dintel, en las casas en las que la comerán. \footnote{\textbf{12:7} Éxod
  12,13; Éxod 12,22} \bibleverse{8} Esa noche comerán la carne, asada al
fuego, con panes sin levadura. La comerán con hierbas amargas.
\bibleverse{9} No la comerán cruda, ni hervida en absoluto con agua,
sino asada al fuego; con su cabeza, sus patas y sus partes interiores.
\bibleverse{10} No dejarás que quede nada de él hasta la mañana; pero lo
que quede de él hasta la mañana lo quemarás al fuego. \bibleverse{11}
Así lo comerás: con tu cinturón en la cintura, tus sandalias en los pies
y tu bastón en la mano; y lo comerás de prisa: es la Pascua de Yahvé.
\footnote{\textbf{12:11} Is 52,12} \bibleverse{12} Porque esa noche
pasaré por la tierra de Egipto y heriré a todos los primogénitos de la
tierra de Egipto, tanto a los hombres como a los animales. Ejecutaré
juicios contra todos los dioses de Egipto. Yo soy Yahvé. \footnote{\textbf{12:12}
  Núm 33,4} \bibleverse{13} La sangre les servirá de señal en las casas
donde estén. Cuando vea la sangre, pasaré por encima de ustedes, y no
habrá ninguna plaga que los destruya cuando golpee la tierra de Egipto.
\footnote{\textbf{12:13} Heb 11,28}

\hypertarget{arreglos-para-la-fiesta-de-los-panes-sin-levadura}{%
\subsection{Arreglos para la fiesta de los panes sin
levadura}\label{arreglos-para-la-fiesta-de-los-panes-sin-levadura}}

\bibleverse{14} Este día será un memorial para ustedes. Lo celebrarán
como una fiesta para Yahvé. Lo celebraréis como una fiesta a lo largo de
vuestras generaciones, como una ordenanza para siempre.

\bibleverse{15} ``\,`Siete días comeréis panes sin levadura; el primer
día quitaréis la levadura de vuestras casas, porque cualquiera que coma
pan con levadura desde el primer día hasta el séptimo, esa persona será
cortada de Israel. \footnote{\textbf{12:15} Éxod 13,7} \bibleverse{16}
El primer día tendréis una santa convocación, y el séptimo día una santa
convocación; no se hará en ellos ninguna clase de trabajo, sino el que
cada uno deba comer, sólo el que pueda ser hecho por vosotros.
\bibleverse{17} Observaréis la fiesta de los panes sin levadura, porque
en este mismo día he sacado vuestros ejércitos de la tierra de Egipto.
Por lo tanto, observaréis este día a lo largo de vuestras generaciones
como una ordenanza para siempre. \bibleverse{18} En el primer mes, el
día catorce del mes por la tarde, comeréis panes sin levadura, hasta el
día veintiuno del mes por la tarde. \bibleverse{19} No se hallará
levadura en vuestras casas durante siete días, porque el que coma algo
leudado será eliminado de la congregación de Israel, ya sea extranjero o
nacido en el país. \bibleverse{20} No comeréis nada con levadura. En
todas vuestras moradas comeréis panes sin levadura''.

\hypertarget{moisuxe9s-enseuxf1a-a-los-ancianos-los-preceptos-sobre-la-pascua}{%
\subsection{Moisés enseña a los ancianos los preceptos sobre la
Pascua}\label{moisuxe9s-enseuxf1a-a-los-ancianos-los-preceptos-sobre-la-pascua}}

\bibleverse{21} Entonces Moisés llamó a todos los ancianos de Israel y
les dijo: ``Sacad y tomad corderos según vuestras familias y matad la
Pascua. \bibleverse{22} Tomaréis un manojo de hisopo y lo mojaréis en la
sangre que está en la pila, y golpearéis el dintel y los dos postes de
la puerta con la sangre que está en la pila. Ninguno de ustedes saldrá
de la puerta de su casa hasta la mañana. \bibleverse{23} Porque Yahvé
pasará para herir a los egipcios; y cuando vea la sangre en el dintel y
en los dos postes de la puerta, Yahvé pasará por encima de la puerta, y
no permitirá que el destructor entre en vuestras casas para heriros.
\bibleverse{24} Observaréis esto como una ordenanza para vosotros y para
vuestros hijos para siempre. \bibleverse{25} Sucederá que cuando
lleguéis a la tierra que Yahvé os dará, como ha prometido, guardaréis
este servicio. \bibleverse{26} Sucederá que cuando vuestros hijos os
pregunten: ``¿Qué queréis decir con este servicio?'' \footnote{\textbf{12:26}
  Gén 18,19; Deut 6,7; Deut 6,20} \bibleverse{27} Diréis: ``Es el
sacrificio de la Pascua de Yahvé, que pasó por encima de las casas de
los hijos de Israel en Egipto, cuando hirió a los egipcios y perdonó
nuestras casas.'' El pueblo inclinó la cabeza y adoró. \bibleverse{28}
Los hijos de Israel fueron y lo hicieron; como Yahvé había ordenado a
Moisés y a Aarón, así lo hicieron.

\hypertarget{la-duxe9cima-plaga-la-muerte-del-primoguxe9nito-egipcio}{%
\subsection{La décima plaga: la muerte del primogénito
egipcio}\label{la-duxe9cima-plaga-la-muerte-del-primoguxe9nito-egipcio}}

\bibleverse{29} A medianoche, el Señor hirió a todos los primogénitos en
la tierra de Egipto, desde el primogénito del faraón que se sentaba en
su trono hasta el primogénito del cautivo que estaba en el calabozo, y a
todos los primogénitos del ganado. \footnote{\textbf{12:29} Éxod 4,23}
\bibleverse{30} El faraón se levantó por la noche, él y todos sus
siervos, y todos los egipcios; y hubo un gran clamor en Egipto, pues no
había casa donde no hubiera un muerto. \bibleverse{31} Llamó de noche a
Moisés y a Aarón y les dijo: ``¡Levántense, salgan de entre mi pueblo,
ustedes y los hijos de Israel, y vayan a servir a Yavé, como han dicho!
\bibleverse{32} Tomen sus rebaños y sus manadas, como han dicho, y
váyanse; ¡y bendíganme también a mí!'' \footnote{\textbf{12:32} Éxod
  10,26}

\bibleverse{33} Los egipcios estaban urgidos con el pueblo, para
enviarlo fuera de la tierra a toda prisa, pues decían: ``Todos somos
hombres muertos''. \footnote{\textbf{12:33} Éxod 6,1} \bibleverse{34} El
pueblo tomó su masa antes de que fuera leudada, con sus amasadoras
atadas a sus ropas sobre los hombros. \bibleverse{35} Los hijos de
Israel hicieron conforme a la palabra de Moisés, y pidieron a los
egipcios joyas de plata, joyas de oro y ropa. \footnote{\textbf{12:35}
  Éxod 11,2} \bibleverse{36} El Señor le concedió al pueblo el favor de
los egipcios, de modo que les permitieron tener lo que pedían. Saquearon
a los egipcios. \footnote{\textbf{12:36} Éxod 3,21}

\hypertarget{el-uxe9xodo-de-israel-de-egipto}{%
\subsection{El éxodo de Israel de
Egipto}\label{el-uxe9xodo-de-israel-de-egipto}}

\bibleverse{37} Los hijos de Israel viajaron de Ramsés a Sucot, unos
seiscientos mil a pie que eran hombres, además de los niños.
\bibleverse{38} También subió con ellos una multitud mixta, con rebaños,
manadas y mucho ganado. \bibleverse{39} Con la masa que habían sacado de
Egipto cocinaban tortas sin levadura, pues no estaba leudada, porque
habían sido expulsados de Egipto y no podían esperar, y no habían
preparado ningún alimento para ellos. \bibleverse{40} El tiempo que los
hijos de Israel vivieron en Egipto fue de cuatrocientos treinta años.
\footnote{\textbf{12:40} Gén 15,13} \bibleverse{41} Al final de los
cuatrocientos treinta años, hasta el día de hoy, todos los ejércitos de
Yahvé salieron de la tierra de Egipto. \bibleverse{42} Es una noche que
hay que observar mucho a Yavé por haberlos sacado de la tierra de
Egipto. Esta es la noche de Yahvé, que debe ser muy observada por todos
los hijos de Israel a lo largo de sus generaciones.

\hypertarget{ordenanza-de-la-pascua-santificaciuxf3n-del-primoguxe9nito}{%
\subsection{Ordenanza de la Pascua; Santificación del
primogénito}\label{ordenanza-de-la-pascua-santificaciuxf3n-del-primoguxe9nito}}

\bibleverse{43} El Señor dijo a Moisés y a Aarón: ``Esta es la ordenanza
de la Pascua. Ningún extranjero comerá de ella, \bibleverse{44} pero el
siervo de todo hombre comprado por dinero, cuando lo hayas circuncidado,
entonces comerá de ella. \bibleverse{45} El extranjero y el jornalero no
comerán de ella. \bibleverse{46} Debe comerse en una sola casa. No
llevarás nada de la carne fuera de la casa. No rompas ninguno de sus
huesos. \footnote{\textbf{12:46} Juan 19,36} \bibleverse{47} Toda la
congregación de Israel la guardará. \bibleverse{48} Cuando un extranjero
viva con vosotros como forastero y quiera celebrar la Pascua a Yahvé,
que se circunciden todos sus varones, y entonces que se acerque y la
celebre. Será como uno de los nacidos en la tierra; pero ningún
incircunciso podrá comer de ella. \bibleverse{49} Una misma ley será
para el nacido en casa, y para el extranjero que vive como forastero
entre vosotros.'' \footnote{\textbf{12:49} Lev 24,22} \bibleverse{50}
Así lo hicieron todos los hijos de Israel. Como Yahvé les ordenó a
Moisés y a Aarón, así lo hicieron. \bibleverse{51} Ese mismo día, Yahvé
sacó a los hijos de Israel de la tierra de Egipto con sus ejércitos.

\hypertarget{section-12}{%
\section{13}\label{section-12}}

\bibleverse{1} Yahvé habló a Moisés, diciendo: \bibleverse{2}
``Santifícame a todos los primogénitos, todo lo que abre el vientre
entre los hijos de Israel, tanto de los hombres como de los animales. Es
mío''. \footnote{\textbf{13:2} Núm 8,17-18; Núm 18,15; Luc 2,23}

\hypertarget{el-reglamento-sobre-la-celebraciuxf3n-de-la-fiesta-de-los-panes-sin-levadura}{%
\subsection{El reglamento sobre la celebración de la Fiesta de los Panes
sin
Levadura}\label{el-reglamento-sobre-la-celebraciuxf3n-de-la-fiesta-de-los-panes-sin-levadura}}

\bibleverse{3} Moisés dijo al pueblo: ``Acuérdate de este día en que
saliste de Egipto, de la casa de servidumbre, porque con la fuerza de tu
mano Yahvé te sacó de este lugar. No se comerá pan con levadura.
\bibleverse{4} Hoy salís en el mes de Abib. \footnote{\textbf{13:4} Éxod
  12,2} \bibleverse{5} Cuando Yahvé os lleve a la tierra del cananeo,
del hitita, del amorreo, del heveo y del jebuseo, que juró a vuestros
padres que os daría, una tierra que mana leche y miel, celebraréis este
servicio en este mes. \footnote{\textbf{13:5} Gén 17,8} \bibleverse{6}
Durante siete días comeréis panes sin levadura, y el séptimo día será
una fiesta para Yahvé. \footnote{\textbf{13:6} Éxod 12,15-16}
\bibleverse{7} Durante los siete días comeréis panes sin levadura, y no
se verá con vosotros ningún pan con levadura. No se verá levadura con
vosotros, dentro de todas vuestras fronteras. \footnote{\textbf{13:7}
  1Cor 5,8} \bibleverse{8} Ese día le dirás a tu hijo: `Es por lo que
hizo el Señor por mí cuando salí de Egipto'. \bibleverse{9} Te servirá
de señal en tu mano, y de recuerdo entre tus ojos, para que la ley de
Yahvé esté en tu boca; porque con mano fuerte Yahvé te sacó de Egipto.
\footnote{\textbf{13:9} Deut 6,8; Deut 11,18} \bibleverse{10} Por lo
tanto, guardarás esta ordenanza en su temporada de año en año.

\hypertarget{santificaciuxf3n-del-primoguxe9nito}{%
\subsection{Santificación del
primogénito}\label{santificaciuxf3n-del-primoguxe9nito}}

\bibleverse{11} ``Cuando Yahvé os introduzca en la tierra de los
cananeos, como os juró a vosotros y a vuestros padres, y os la entregue,
\bibleverse{12} apartaréis para Yahvé todo lo que abra el vientre, y
todo primogénito que proceda de un animal que tengáis. Los machos serán
de Yahvé. \bibleverse{13} Todo primogénito de asno lo redimirás con un
cordero; y si no lo quieres redimir, le romperás el cuello; y redimirás
todo primogénito de hombre entre tus hijos. \bibleverse{14} Cuando tu
hijo te pregunte en el futuro, diciendo: ``¿Qué es esto?'', le dirás:
``Con la fuerza de la mano, Yahvé nos sacó de Egipto, de la casa de
servidumbre. \footnote{\textbf{13:14} Éxod 12,26} \bibleverse{15} Cuando
el faraón se negó obstinadamente a dejarnos ir, el Señor mató a todos
los primogénitos en la tierra de Egipto, tanto a los primogénitos de los
hombres como a los primogénitos de los animales. Por eso sacrifico a
Yavé todo lo que abre el vientre, siendo varones; pero a todos los
primogénitos de mis hijos los redimo'. \footnote{\textbf{13:15} Éxod
  12,29} \bibleverse{16} Será como una señal en tu mano y como un
símbolo entre tus ojos, porque con la fuerza de la mano Yahvé nos sacó
de Egipto.''

\hypertarget{el-tren-al-desierto-y-el-mar-rojo-a-etham}{%
\subsection{El tren al desierto y el Mar Rojo a
Etham}\label{el-tren-al-desierto-y-el-mar-rojo-a-etham}}

\bibleverse{17} Cuando el faraón dejó ir al pueblo, Dios no lo condujo
por el camino de la tierra de los filisteos, aunque estaba cerca; porque
Dios dijo: ``No sea que el pueblo cambie de opinión al ver la guerra, y
vuelva a Egipto''; \bibleverse{18} sino que Dios condujo al pueblo por
el camino del desierto, junto al Mar Rojo; y los hijos de Israel
subieron armados de la tierra de Egipto. \bibleverse{19} Moisés llevó
consigo los huesos de José, porque había hecho jurar a los hijos de
Israel, diciendo: ``Ciertamente Dios os visitará, y llevaréis con
vosotros mis huesos.'' \footnote{\textbf{13:19} Gén 50,25; Jos 24,32}
\bibleverse{20} Partieron de Succoth y acamparon en Etham, en el límite
del desierto. \bibleverse{21} Yahvé iba delante de ellos de día en una
columna de nube, para guiarlos en su camino, y de noche en una columna
de fuego, para alumbrarles, para que pudieran ir de día y de noche:
\footnote{\textbf{13:21} Éxod 40,34; Núm 9,15-23; 1Cor 10,1}
\bibleverse{22} la columna de nube de día, y la columna de fuego de
noche, no se apartaban de delante del pueblo.

\hypertarget{dios-ordena-el-cambio-de-direcciuxf3n}{%
\subsection{Dios ordena el cambio de
dirección}\label{dios-ordena-el-cambio-de-direcciuxf3n}}

\hypertarget{section-13}{%
\section{14}\label{section-13}}

\bibleverse{1} Yahvé habló a Moisés, diciendo: \bibleverse{2} ``Habla a
los hijos de Israel para que regresen y acampen frente a Pihahiroth,
entre Migdol y el mar, frente a Baal Zephon. Acamparán frente a ella,
junto al mar. \bibleverse{3} El Faraón dirá de los hijos de Israel:
`Están enredados en la tierra. El desierto los ha encerrado'.
\bibleverse{4} Endureceré el corazón del faraón y los seguirá, y
obtendré honor sobre el faraón y sobre todos sus ejércitos, y los
egipcios sabrán que yo soy Yavé.'' Así lo hicieron. \footnote{\textbf{14:4}
  Éxod 4,21; Éxod 9,16; Ezeq 28,22}

\hypertarget{el-farauxf3n-persigue-a-los-israelitas-y-los-alcanza}{%
\subsection{El faraón persigue a los israelitas y los
alcanza}\label{el-farauxf3n-persigue-a-los-israelitas-y-los-alcanza}}

\bibleverse{5} El rey de Egipto recibió la noticia de que el pueblo
había huido; y el corazón del faraón y de sus siervos se transformó
hacia el pueblo, y dijeron: ``¿Qué es lo que hemos hecho, que hemos
dejado que Israel deje de servirnos?'' \bibleverse{6} Preparó su carro,
y tomó su ejército con él; \bibleverse{7} y tomó seiscientos carros
escogidos, y todos los carros de Egipto, con capitanes sobre todos
ellos. \bibleverse{8} El Señor endureció el corazón del faraón, rey de
Egipto, y persiguió a los hijos de Israel; porque los hijos de Israel
salieron con la mano en alto. \footnote{\textbf{14:8} Éxod 13,9}
\bibleverse{9} Los egipcios los persiguieron. Todos los caballos y
carros del Faraón, su caballería y su ejército los alcanzaron acampando
junto al mar, junto a Pihahiroth, ante Baal Zephon.

\hypertarget{moisuxe9s-anima-a-los-israelitas-desanimados-la-intervenciuxf3n-de-dios}{%
\subsection{Moisés anima a los israelitas desanimados; La intervención
de
Dios}\label{moisuxe9s-anima-a-los-israelitas-desanimados-la-intervenciuxf3n-de-dios}}

\bibleverse{10} Cuando el faraón se acercó, los hijos de Israel alzaron
los ojos y vieron que los egipcios marchaban tras ellos, y tuvieron
mucho miedo. Los hijos de Israel clamaron a Yahvé. \bibleverse{11} Le
dijeron a Moisés: ``Porque no había tumbas en Egipto, ¿nos has llevado a
morir al desierto? ¿Por qué nos has tratado así, para sacarnos de
Egipto? \bibleverse{12} ¿No es ésta la palabra que te dijimos en Egipto:
`Déjanos en paz, para que sirvamos a los egipcios'? Porque hubiera sido
mejor para nosotros servir a los egipcios que morir en el desierto''.

\bibleverse{13} Moisés dijo al pueblo: ``No tengan miedo. Quédense
quietos y vean la salvación de Yavé, que él obrará hoy en favor de
ustedes; porque nunca más verán a los egipcios que han visto hoy.
\bibleverse{14} Yahvé luchará por ustedes, y ustedes se quedarán
quietos''. \footnote{\textbf{14:14} Deut 1,30; 2Cró 20,15; Is 30,15}

\bibleverse{15} El Señor dijo a Moisés: ``¿Por qué clamas a mí? Habla a
los hijos de Israel para que avancen. \bibleverse{16} Levanta tu vara,
extiende tu mano sobre el mar y divídelo. Entonces los hijos de Israel
entrarán en medio del mar sobre tierra seca. \bibleverse{17} He aquí que
yo mismo endureceré el corazón de los egipcios, y entrarán tras ellos.
Yo mismo obtendré honor sobre el Faraón y sobre todos sus ejércitos,
sobre sus carros y sobre su caballería. \footnote{\textbf{14:17} Éxod
  14,4} \bibleverse{18} Los egipcios sabrán que yo soy Yavé cuando me
haya ganado el honor sobre el Faraón, sobre sus carros y sobre su
caballería.'' \bibleverse{19} El ángel de Dios, que iba delante del
campamento de Israel, se movió y fue detrás de ellos; y la columna de
nube se movió de delante de ellos y se puso detrás de ellos. \footnote{\textbf{14:19}
  Éxod 13,21} \bibleverse{20} Se interpuso entre el campamento de Egipto
y el campamento de Israel. Allí estaba la nube y las tinieblas, pero
daba luz de noche. Uno no se acercó al otro en toda la noche.

\hypertarget{paso-de-los-israelitas-por-el-mar-rojo-cauxedda-de-los-egipcios}{%
\subsection{Paso de los israelitas por el Mar Rojo; Caída de los
egipcios}\label{paso-de-los-israelitas-por-el-mar-rojo-cauxedda-de-los-egipcios}}

\bibleverse{21} Moisés extendió su mano sobre el mar, y el Señor hizo
retroceder el mar con un fuerte viento del este durante toda la noche, e
hizo que el mar se secara, y las aguas se dividieron. \bibleverse{22}
Los hijos de Israel entraron en medio del mar, en seco, y las aguas les
sirvieron de muro a su derecha y a su izquierda. \footnote{\textbf{14:22}
  Jos 4,23; Is 11,15-16; 1Cor 10,1; Heb 11,29} \bibleverse{23} Los
egipcios los persiguieron y entraron tras ellos en medio del mar: todos
los caballos del faraón, sus carros y su caballería. \footnote{\textbf{14:23}
  Éxod 15,19} \bibleverse{24} En la vigilia de la mañana, el Señor miró
al ejército egipcio a través de la columna de fuego y de nube, y
confundió al ejército egipcio. \footnote{\textbf{14:24} Sal 34,16; Sal
  104,32} \bibleverse{25} Les quitó las ruedas de los carros, y los hizo
caer pesadamente, de modo que los egipcios dijeron: ``¡Huyamos de la faz
de Israel, porque Yavé lucha por ellos contra los egipcios!''
\footnote{\textbf{14:25} Éxod 14,14; Sal 64,9}

\bibleverse{26} Yahvé dijo a Moisés: ``Extiende tu mano sobre el mar,
para que las aguas vuelvan a caer sobre los egipcios, sobre sus carros y
sobre su caballería.'' \bibleverse{27} Moisés extendió su mano sobre el
mar, y el mar volvió a su fuerza cuando apareció la mañana, y los
egipcios huyeron contra él. El Señor derrotó a los egipcios en medio del
mar. \bibleverse{28} Las aguas volvieron a cubrir los carros y la gente
de a caballo, así como todo el ejército del faraón que entró tras ellos
en el mar. No quedó ni uno solo de ellos. \bibleverse{29} Pero los hijos
de Israel caminaban en seco en medio del mar, y las aguas eran un muro
para ellos a su derecha y a su izquierda. \footnote{\textbf{14:29} Éxod
  14,22} \bibleverse{30} Así salvó Yahvé a Israel aquel día de la mano
de los egipcios; e Israel vio a los egipcios muertos en la orilla del
mar. \bibleverse{31} Israel vio la gran obra que Yahvé hizo a los
egipcios, y el pueblo temió a Yahvé; y creyeron en Yahvé y en su siervo
Moisés. \footnote{\textbf{14:31} Éxod 19,9; 2Cró 20,20}

\hypertarget{canciuxf3n-de-victoria-de-los-israelitas-en-el-mar-rojo}{%
\subsection{Canción de victoria de los israelitas en el Mar
Rojo}\label{canciuxf3n-de-victoria-de-los-israelitas-en-el-mar-rojo}}

\hypertarget{section-14}{%
\section{15}\label{section-14}}

\bibleverse{1} Entonces Moisés y los hijos de Israel entonaron este
cántico a Yahvé, y dijeron``Cantaré a Yahvé, porque ha triunfado
gloriosamente. Ha arrojado al mar al caballo y a su jinete. \footnote{\textbf{15:1}
  Apoc 15,3} \bibleverse{2} Yah es mi fuerza y mi canción. Se ha
convertido en mi salvación. Este es mi Dios, y lo alabaré; el Dios de mi
padre, y lo exaltaré. \footnote{\textbf{15:2} Sal 118,14; Is 12,2}
\bibleverse{3} Yahvé es un hombre de guerra. Yavé es su nombre.
\footnote{\textbf{15:3} Éxod 14,14; Sal 46,9; Éxod 3,15} \bibleverse{4}
Ha arrojado al mar los carros del Faraón y su ejército. Sus capitanes
elegidos se hunden en el Mar Rojo. \bibleverse{5} Las profundidades los
cubren. Bajaron a las profundidades como una piedra. \bibleverse{6} Tu
mano derecha, Yahvé, es gloriosa en poder. Tu mano derecha, Yahvé, hace
pedazos al enemigo. \bibleverse{7} En la grandeza de tu excelencia,
derrotas a los que se levantan contra ti. Envías tu ira. Los consume
como rastrojo. \footnote{\textbf{15:7} Is 47,14} \bibleverse{8} Con el
soplo de tus narices, las aguas se amontonaron. Las inundaciones se
levantaron como un montón. Las profundidades se congelaron en el corazón
del mar. \bibleverse{9} El enemigo dijo: ``Voy a perseguir. Voy a
alcanzarlo. Repartiré el botín. Mi deseo será satisfecho en ellos.
Sacaré mi espada. Mi mano los destruirá''. \bibleverse{10} Soplaste con
tu viento. El mar los cubrió. Se hundieron como el plomo en las
poderosas aguas. \bibleverse{11} ¿Quién es como tú, Yahvé, entre los
dioses? Que es como tú, glorioso en santidad, temeroso en las alabanzas,
haciendo maravillas? \footnote{\textbf{15:11} Éxod 18,11; Sal 72,18-19}
\bibleverse{12} Extendiste tu mano derecha. La tierra se los tragó.
\bibleverse{13} ``Tú, en tu amorosa bondad, has guiado al pueblo que has
redimido. Los has guiado con tu fuerza hacia tu santa morada.
\bibleverse{14} Los pueblos han oído. Tiemblan. Los dolores se han
apoderado de los habitantes de Filistea. \footnote{\textbf{15:14} Jos
  2,9-11} \bibleverse{15} Entonces los jefes de Edom quedaron
consternados. El temblor se apodera de los poderosos hombres de Moab.
Todos los habitantes de Canaán se han derretido. \bibleverse{16} El
terror y el pavor caen sobre ellos. Por la grandeza de tu brazo están
tan quietos como una piedra, hasta que tu pueblo pase, Yahvé, hasta que
pasen las personas que has comprado. \bibleverse{17} Los traerás y los
plantarás en el monte de tu heredad, el lugar, Yahvé, que te has hecho
para habitar: el santuario, Señor, que tus manos han establecido.
\bibleverse{18} Yahvé reinará por los siglos de los siglos''.
\footnote{\textbf{15:18} Sal 93,1}

\bibleverse{19} Porque los caballos del faraón entraron con sus carros y
con su gente de a caballo en el mar, y el Señor hizo volver las aguas
del mar sobre ellos; pero los hijos de Israel caminaron en seco en medio
del mar. \footnote{\textbf{15:19} Éxod 14,22-29}

\hypertarget{participaciuxf3n-de-mujeres-en-la-alabanza-del-seuxf1or}{%
\subsection{Participación de mujeres en la alabanza del
Señor}\label{participaciuxf3n-de-mujeres-en-la-alabanza-del-seuxf1or}}

\bibleverse{20} La profetisa Miriam, hermana de Aarón, tomó un pandero
en su mano, y todas las mujeres salieron tras ella con panderos y
danzas. \footnote{\textbf{15:20} Sal 68,25} \bibleverse{21} Miriam les
respondió, ``Cantad a Yahvé, porque ha triunfado gloriosamente. Ha
arrojado al mar al caballo y a su jinete''. \footnote{\textbf{15:21}
  Éxod 15,1}

\hypertarget{el-agua-amarga-de-mara-se-volviuxf3-apetecible-la-llegada-a-la-encantadora-elim}{%
\subsection{El agua amarga de Mara se volvió apetecible; la llegada a la
encantadora
Elim}\label{el-agua-amarga-de-mara-se-volviuxf3-apetecible-la-llegada-a-la-encantadora-elim}}

\bibleverse{22} Moisés condujo a Israel desde el Mar Rojo, y salieron al
desierto de Shur; anduvieron tres días por el desierto y no encontraron
agua. \bibleverse{23} Cuando llegaron a Mara, no pudieron beber de las
aguas de Mara, porque eran amargas. Por eso su nombre fue llamado Mara.
\bibleverse{24} El pueblo murmuró contra Moisés, diciendo: ``¿Qué vamos
a beber?'' \bibleverse{25} Entonces él clamó a Yavé. Yavé le mostró un
árbol, y él lo arrojó a las aguas, y las aguas se endulzaron. Allí les
hizo un estatuto y una ordenanza, y allí los puso a prueba.
\bibleverse{26} Les dijo: ``Si escucháis con diligencia la voz del
Señor, vuestro Dios, y hacéis lo que es justo a sus ojos, y prestáis
atención a sus mandamientos y guardáis todos sus estatutos, no pondré
sobre vosotros ninguna de las enfermedades que puse sobre los egipcios,
porque yo soy el Señor que os cura.'' \footnote{\textbf{15:26} Deut
  7,15; Deut 32,39; Mat 9,12}

\bibleverse{27} Llegaron a Elim, donde había doce fuentes de agua y
setenta palmeras. Allí acamparon junto a las aguas.

\hypertarget{el-murmullo-del-pueblo-la-exaltaciuxf3n-divina-mediante-la-donaciuxf3n-de-codornices-y-manuxe1}{%
\subsection{El murmullo del pueblo; la exaltación divina mediante la
donación de codornices y
maná}\label{el-murmullo-del-pueblo-la-exaltaciuxf3n-divina-mediante-la-donaciuxf3n-de-codornices-y-manuxe1}}

\hypertarget{section-15}{%
\section{16}\label{section-15}}

\bibleverse{1} Partieron de Elim, y toda la congregación de los hijos de
Israel llegó al desierto de Sin, que está entre Elim y Sinaí, el día
quince del segundo mes después de su salida de la tierra de Egipto.
\bibleverse{2} Toda la congregación de los hijos de Israel murmuró
contra Moisés y contra Aarón en el desierto; \footnote{\textbf{16:2}
  Éxod 17,2} \bibleverse{3} y los hijos de Israel les dijeron: ``Ojalá
hubiéramos muerto por mano de Yavé en la tierra de Egipto, cuando nos
sentábamos junto a las ollas de carne, cuando nos saciábamos de pan,
porque ustedes nos han traído a este desierto para matar de hambre a
toda esta asamblea.'' \footnote{\textbf{16:3} Éxod 14,11}

\bibleverse{4} Entonces Yahvé dijo a Moisés: ``He aquí que yo haré
llover pan del cielo para ustedes, y el pueblo saldrá a recoger la
porción de un día cada día, para que yo los pruebe, si andan o no en mi
ley. \footnote{\textbf{16:4} Juan 6,31; 1Cor 10,3} \bibleverse{5} Al
sexto día prepararán lo que traigan, y será el doble de lo que recojan
cada día.''

\bibleverse{6} Moisés y Aarón dijeron a todos los hijos de Israel: ``Al
anochecer, sabrán que el Señor los ha sacado de la tierra de Egipto.
\bibleverse{7} Por la mañana, verán la gloria de Yavé, porque él escucha
sus murmuraciones contra Yavé. ¿Quiénes somos nosotros, para que
murmuren contra nosotros?'' \bibleverse{8} Moisés dijo: ``Ahora Yahvé os
dará de comer por la tarde, y por la mañana pan para saciaros, porque
Yahvé oye vuestras murmuraciones, que vosotros murmuráis contra él. ¿Y
quiénes somos nosotros? Sus murmuraciones no son contra nosotros, sino
contra Yavé''. \bibleverse{9} Moisés le dijo a Aarón: ``Dile a toda la
congregación de los hijos de Israel que se acerque a Yavé, porque él ha
escuchado sus murmuraciones''. \bibleverse{10} Mientras Aarón hablaba a
toda la congregación de los hijos de Israel, miraron hacia el desierto,
y he aquí que la gloria de Yavé apareció en la nube. \footnote{\textbf{16:10}
  Núm 12,5; Núm 14,10; Núm 16,19} \bibleverse{11} Yahvé habló a Moisés
diciendo: \bibleverse{12} ``He oído las murmuraciones de los hijos de
Israel. Háblales diciendo: `Al atardecer comeréis carne, y por la mañana
os saciaréis de pan. Entonces sabrán que yo soy Yahvé, su Dios'\,''.

\bibleverse{13} Al atardecer, las codornices subieron y cubrieron el
campamento; y por la mañana el rocío se posó alrededor del campamento.
\footnote{\textbf{16:13} Núm 11,31} \bibleverse{14} Cuando el rocío que
había caído se desvaneció, he aquí que en la superficie del desierto
había una cosa pequeña y redonda, pequeña como la escarcha del suelo.
\bibleverse{15} Cuando los hijos de Israel lo vieron, se dijeron unos a
otros: ``¿Qué es?'' Porque no sabían lo que era. Moisés les dijo: ``Es
el pan que el Señor les ha dado para comer. \footnote{\textbf{16:15}
  Éxod 16,4}

\hypertarget{reglas-sobre-recolecciuxf3n-de-manuxe1-moisuxe9s-explica-un-fenuxf3meno-milagroso-que-ocurriuxf3}{%
\subsection{Reglas sobre recolección de maná; Moisés explica un fenómeno
milagroso que
ocurrió}\label{reglas-sobre-recolecciuxf3n-de-manuxe1-moisuxe9s-explica-un-fenuxf3meno-milagroso-que-ocurriuxf3}}

\bibleverse{16} Esto es lo que Yahvé ha ordenado: `Recoged de él cada
uno según su consumo; un omer por cabeza, según el número de vuestras
personas, lo tomaréis, cada uno para los que están en su tienda'.''
\bibleverse{17} Los hijos de Israel lo hicieron así, y unos recogieron
más y otros menos. \bibleverse{18} Cuando lo midieron con un omer, al
que recogió mucho no le sobró nada, y al que recogió poco no le faltó.
Cada uno recogió según lo que comía. \footnote{\textbf{16:18} 2Cor 8,15}
\bibleverse{19} Moisés les dijo: ``Que nadie deje de comer hasta la
mañana''. \footnote{\textbf{16:19} Mat 6,34; Luc 11,3} \bibleverse{20}
Pero no escucharon a Moisés, sino que algunos lo dejaron hasta la
mañana, por lo que crió gusanos y se ensució; y Moisés se enojó con
ellos. \bibleverse{21} Lo recogieron de mañana en mañana, cada uno según
su consumo. Cuando el sol se calentó, se derritió. \bibleverse{22} Al
sexto día, recogieron el doble de pan, dos omers para cada uno; y todos
los jefes de la congregación vinieron a decírselo a Moisés.
\bibleverse{23} Este les dijo: ``Esto es lo que ha dicho el Señor:
`Mañana es un descanso solemne, un sábado sagrado para el Señor. Horneen
lo que quieran hornear, y hiervan lo que quieran hervir; y todo lo que
sobre lo guarden hasta la mañana'\,''. \footnote{\textbf{16:23} Gén
  2,2-3; Éxod 20,8} \bibleverse{24} Lo guardaron hasta la mañana, tal
como lo ordenó Moisés, y no se ensució, ni hubo gusanos en él.
\bibleverse{25} Moisés dijo: ``Coman eso hoy, porque hoy es sábado para
Yavé. Hoy no lo encontrarás en el campo. \bibleverse{26} Seis días lo
recogerás, pero el séptimo día es sábado. En él no habrá nada''.
\bibleverse{27} El séptimo día, algunos del pueblo salieron a recoger, y
no encontraron nada. \bibleverse{28} Yahvé dijo a Moisés: ``¿Hasta
cuándo os negáis a cumplir mis mandamientos y mis leyes? \bibleverse{29}
Miren, porque Yahvé les ha dado el sábado, por eso les da en el sexto
día el pan de dos días. Que cada uno se quede en su lugar. Que nadie
salga de su lugar en el séptimo día''. \bibleverse{30} Así que el pueblo
descansó el séptimo día.

\bibleverse{31} La casa de Israel lo llamó ``Maná'', y era como semilla
de cilantro, blanco; y su sabor era como de obleas con miel.
\bibleverse{32} Moisés dijo: ``Esto es lo que Yahvé ha ordenado:
`Guarden un omer-lleno de él a lo largo de sus generaciones, para que
vean el pan con el que los alimenté en el desierto, cuando los saqué de
la tierra de Egipto'\,''. \bibleverse{33} Moisés le dijo a Aarón: ``Toma
una vasija y pon en ella un omer-lleno de maná, y guárdalo delante de
Yavé, para que lo guarden a lo largo de sus generaciones.'' \footnote{\textbf{16:33}
  Heb 9,4} \bibleverse{34} Tal como Yahvé le ordenó a Moisés, Aarón lo
depositó ante el Testimonio, para que se conservara. \bibleverse{35} Los
hijos de Israel comieron el maná durante cuarenta años, hasta que
llegaron a una tierra habitada. Comieron el maná hasta que llegaron a
los límites de la tierra de Canaán. \footnote{\textbf{16:35} Jos 5,12}
\bibleverse{36} Un omeres la décima parte de un efa. \footnote{\textbf{16:36}
  1 efa equivale a unos 22 litros o a 2/3 de una fanega}

\hypertarget{la-maravillosa-donaciuxf3n-de-agua-de-la-roca-cerca-de-massa-y-meriba}{%
\subsection{La maravillosa donación de agua de la roca cerca de Massa y
Meriba}\label{la-maravillosa-donaciuxf3n-de-agua-de-la-roca-cerca-de-massa-y-meriba}}

\hypertarget{section-16}{%
\section{17}\label{section-16}}

\bibleverse{1} Toda la congregación de los hijos de Israel partió del
desierto de Sin, siguiendo el mandato de Yavé, y acampó en Refidim; pero
no había agua para que el pueblo bebiera. \bibleverse{2} Por eso el
pueblo discutió con Moisés y le dijo: ``Danos agua para beber''. Moisés
les dijo: ``¿Por qué os peleáis conmigo? ¿Por qué ponéis a prueba a
Yahvé?'' \footnote{\textbf{17:2} Deut 6,16; 1Cor 10,9}

\bibleverse{3} El pueblo estaba sediento de agua allí; por eso el pueblo
murmuró contra Moisés y dijo: ``¿Por qué nos has hecho subir de Egipto
para matarnos de sed a nosotros, a nuestros hijos y a nuestro ganado?''

\bibleverse{4} Moisés clamó a Yavé diciendo: ``¿Qué debo hacer con este
pueblo? Están casi listos para apedrearme''. \footnote{\textbf{17:4} Núm
  14,10}

\bibleverse{5} Yahvé dijo a Moisés: ``Camina delante del pueblo, y lleva
contigo a los ancianos de Israel, y toma en tu mano la vara con la que
golpeaste el Nilo, y vete. \footnote{\textbf{17:5} Éxod 7,20}
\bibleverse{6} He aquí que yo me pondré delante de ti allí, en la roca
de Horeb. Golpearás la roca y saldrá agua de ella, para que el pueblo
pueda beber''. Así lo hizo Moisés a la vista de los ancianos de Israel.
\footnote{\textbf{17:6} Núm 20,11; 1Cor 10,4} \bibleverse{7} Llamó el
nombre del lugar Massah,\footnote{\textbf{17:7} Massah significa prueba.}
y Meribah,\footnote{\textbf{17:7} Meribah significa riña.} porque los
hijos de Israel se peleaban y porque ponían a prueba a Yahvé, diciendo:
``¿Está Yahvé entre nosotros o no?'' \footnote{\textbf{17:7} Sal 95,8-9}

\hypertarget{la-victoria-sobre-los-amalecitas-en-refidim}{%
\subsection{La victoria sobre los amalecitas en
Refidim}\label{la-victoria-sobre-los-amalecitas-en-refidim}}

\bibleverse{8} Entonces vino Amalec y luchó contra Israel en Refidim.
\bibleverse{9} Moisés dijo a Josué: ``Escoge hombres para nosotros y sal
a pelear con Amalec. Mañana estaré en la cima de la colina con la vara
de Dios en la mano''. \footnote{\textbf{17:9} Núm 13,8; Núm 13,16}
\bibleverse{10} Así que Josué hizo lo que Moisés le había dicho, y luchó
contra Amalec; y Moisés, Aarón y Hur subieron a la cima de la colina.
\bibleverse{11} Cuando Moisés levantó la mano, Israel venció. Cuando
bajó su mano, Amalec prevaleció. \bibleverse{12} Pero a Moisés le
pesaban las manos, así que tomaron una piedra y la pusieron debajo de
él, y se sentó sobre ella. Aarón y Hur le sostuvieron las manos, el uno
de un lado y el otro del otro. Sus manos estuvieron firmes hasta la
puesta del sol. \bibleverse{13} Josué derrotó a Amalec y a su pueblo a
filo de espada. \bibleverse{14} Yahvé dijo a Moisés: ``Escribe esto como
recuerdo en un libro, y recuérdalo en los oídos de Josué: que borraré
por completo la memoria de Amalec de debajo del cielo.'' \footnote{\textbf{17:14}
  Deut 25,17-19; 1Sam 15,2-3} \bibleverse{15} Moisés construyó un altar
y lo llamó ``Yahvé, nuestro estandarte''. \footnote{\textbf{17:15}
  Hebreo, Yahvé Nissi} \bibleverse{16} Dijo: ``Yah ha jurado: `Yahvé
tendrá guerra con Amalec de generación en generación'.''

\hypertarget{la-visita-de-jetro-a-moisuxe9s-en-el-monte-de-dios-nombramiento-de-jueces}{%
\subsection{La visita de Jetro a Moisés en el monte de Dios;
Nombramiento de
jueces}\label{la-visita-de-jetro-a-moisuxe9s-en-el-monte-de-dios-nombramiento-de-jueces}}

\hypertarget{section-17}{%
\section{18}\label{section-17}}

\bibleverse{1} Jetro, el sacerdote de Madián, suegro de Moisés, se
enteró de todo lo que Dios había hecho por Moisés y por su pueblo
Israel, de cómo Yahvé había sacado a Israel de Egipto. \footnote{\textbf{18:1}
  Éxod 3,1} \bibleverse{2} Jetro, suegro de Moisés, recibió a Séfora, la
esposa de Moisés, después de haberla despedido, \footnote{\textbf{18:2}
  Éxod 4,20} \bibleverse{3} y a sus dos hijos. El nombre de un hijo era
Gershom,\footnote{\textbf{18:3} ``Gershom'' suena como el hebreo para
  ``un extranjero allí''.} porque Moisés dijo: ``He vivido como
extranjero en tierra extranjera''. \footnote{\textbf{18:3} Éxod 2,22}
\bibleverse{4} El nombre del otro fue Eliezer,\footnote{\textbf{18:4}
  Eliezer significa ``Dios es mi ayudante''.} pues dijo: ``El Dios de mi
padre fue mi ayuda y me libró de la espada del Faraón''. \bibleverse{5}
Jetro, el suegro de Moisés, vino con los hijos de Moisés y su esposa a
Moisés al desierto, donde estaba acampado, en la Montaña de Dios.
\bibleverse{6} Le dijo a Moisés: ``Yo, tu suegro Jetro, he venido a ti
con tu mujer y sus dos hijos con ella.''

\bibleverse{7} Moisés salió al encuentro de su suegro, se inclinó y lo
besó. Se preguntaron mutuamente por su bienestar, y entraron en la
tienda. \bibleverse{8} Moisés le contó a su suegro todo lo que Yahvé
había hecho al faraón y a los egipcios por causa de Israel, todas las
dificultades que les habían sobrevenido en el camino, y cómo Yahvé los
había librado. \bibleverse{9} Jetro se alegró de toda la bondad que
Yahvé había hecho con Israel, al librarlo de la mano de los egipcios.
\bibleverse{10} Jetro dijo: ``Bendito sea Yahvé, que te ha librado de la
mano de los egipcios y de la mano del faraón; que ha librado al pueblo
de la mano de los egipcios. \bibleverse{11} Ahora sé que Yavé es más
grande que todos los dioses, por la forma en que trataron al pueblo con
arrogancia.'' \footnote{\textbf{18:11} Neh 9,10} \bibleverse{12} Jetro,
suegro de Moisés, llevó un holocausto y sacrificios para Dios. Aarón
vino con todos los ancianos de Israel, para comer el pan con el suegro
de Moisés ante Dios.

\hypertarget{la-reorganizaciuxf3n-del-poder-judicial}{%
\subsection{La reorganización del poder
judicial}\label{la-reorganizaciuxf3n-del-poder-judicial}}

\bibleverse{13} Al día siguiente, Moisés se sentó a juzgar al pueblo, y
el pueblo estuvo de pie alrededor de Moisés desde la mañana hasta la
noche. \bibleverse{14} Cuando el suegro de Moisés vio todo lo que hacía
con el pueblo, le dijo: ``¿Qué es esto que haces por el pueblo? ¿Por qué
te sientas solo, y todo el pueblo está de pie a tu alrededor desde la
mañana hasta la noche?''

\bibleverse{15} Moisés dijo a su suegro: ``Porque el pueblo viene a mí
para consultar a Dios. \bibleverse{16} Cuando tienen un asunto, vienen a
mí, y yo juzgo entre el hombre y su prójimo, y les hago conocer los
estatutos de Dios y sus leyes.'' \bibleverse{17} El suegro de Moisés le
dijo: ``Lo que haces no es bueno. \bibleverse{18} Seguramente te
desgastarás, tanto tú como este pueblo que está contigo, porque la cosa
es demasiado pesada para ti. No eres capaz de realizarlo tú solo.
\footnote{\textbf{18:18} Núm 11,14; Deut 1,9} \bibleverse{19} Escucha
ahora mi voz. Yo te aconsejaré, y Dios estará contigo. Tú representas al
pueblo ante Dios, y llevas las causas a Dios. \bibleverse{20} Les
enseñarás los estatutos y las leyes, y les mostrarás el camino por el
que deben andar y el trabajo que deben hacer. \bibleverse{21} Además,
proveerás de todo el pueblo hombres capaces y temerosos de Dios; hombres
de verdad, que odien la ganancia injusta; y los pondrás al frente de
ellos, para que sean jefes de millares, jefes de centenas, jefes de
cincuenta y jefes de decenas. \bibleverse{22} Que juzguen al pueblo en
todo momento. Todo asunto grande te lo traerán a ti, pero todo asunto
pequeño lo juzgarán ellos mismos. Así te será más fácil, y ellos
compartirán la carga contigo. \bibleverse{23} Si haces esto, y Dios te
lo ordena, entonces podrás soportar, y toda esta gente también irá a su
lugar en paz.''

\bibleverse{24} Entonces Moisés escuchó la voz de su suegro, e hizo todo
lo que él había dicho. \bibleverse{25} Moisés eligió a hombres capaces
de todo Israel, y los nombró jefes del pueblo, jefes de millares, jefes
de centenas, jefes de cincuenta y jefes de decenas. \bibleverse{26}
Ellos juzgaban al pueblo en todo momento. Traían los casos difíciles a
Moisés, pero todo asunto menor lo juzgaban ellos mismos. \bibleverse{27}
Moisés dejó partir a su suegro y se fue a su tierra.

\hypertarget{llegada-del-pueblo-al-sinauxed-elaboraciuxf3n-de-legislaciuxf3n}{%
\subsection{Llegada del pueblo al Sinaí; Elaboración de
legislación}\label{llegada-del-pueblo-al-sinauxed-elaboraciuxf3n-de-legislaciuxf3n}}

\hypertarget{section-18}{%
\section{19}\label{section-18}}

\bibleverse{1} En el tercer mes después que los hijos de Israel salieron
de la tierra de Egipto, ese mismo día llegaron al desierto de Sinaí.
\bibleverse{2} Cuando partieron de Refidim y llegaron al desierto de
Sinaí, acamparon en el desierto, y allí acampó Israel ante el monte.
\bibleverse{3} Moisés subió a Dios, y Yahvé lo llamó desde el monte,
diciendo: ``Esto es lo que dirás a la casa de Jacob y a los hijos de
Israel \bibleverse{4} `Habéis visto lo que hice a los egipcios, y cómo
os llevé en alas de águila y os traje a mí. \footnote{\textbf{19:4} Deut
  32,11} \bibleverse{5} Ahora, pues, si de verdad obedecéis mi voz y
guardáis mi pacto, seréis mi posesión de entre todos los pueblos, porque
toda la tierra es mía; \footnote{\textbf{19:5} Deut 7,6} \bibleverse{6}
y seréis para mí un reino de sacerdotes y una nación santa.' Estas son
las palabras que dirás a los hijos de Israel''. \footnote{\textbf{19:6}
  1Pe 2,9; Apoc 1,6; Lev 19,2}

\bibleverse{7} Moisés vino y llamó a los ancianos del pueblo, y les
expuso todas estas palabras que Yahvé le había ordenado. \bibleverse{8}
Todo el pueblo respondió en conjunto y dijo: ``Haremos todo lo que Yahvé
ha dicho''. Moisés informó a Yahvé de las palabras del pueblo.
\bibleverse{9} Yahvé dijo a Moisés: ``He aquí que vengo a ti en una nube
espesa, para que el pueblo oiga cuando hablo contigo y te crea para
siempre''. Moisés le contó a Yavé las palabras del pueblo.
\bibleverse{10} Yahvé dijo a Moisés: ``Ve al pueblo y santifícalo hoy y
mañana, y que lave sus vestidos, \bibleverse{11} y esté preparado para
el tercer día; porque al tercer día Yahvé bajará a la vista de todo el
pueblo al monte Sinaí. \bibleverse{12} Pondrás límites a todo el pueblo,
diciendo: `Tengan cuidado de no subir al monte ni tocar su borde. El que
toque el monte será condenado a muerte. \footnote{\textbf{19:12} Éxod
  34,3} \bibleverse{13} Ninguna mano lo tocará, sino que será apedreado
o atravesado; sea animal u hombre, no vivirá.' Cuando la trompeta suene
largamente, subirán al monte''. \footnote{\textbf{19:13} Heb 12,18-20}

\bibleverse{14} Moisés bajó del monte hacia el pueblo y lo santificó, y
ellos lavaron sus ropas. \bibleverse{15} Le dijo al pueblo: ``Prepárense
para el tercer día. No tengan relaciones sexuales con una mujer''.
\footnote{\textbf{19:15} 1Cor 7,5}

\hypertarget{la-aterradora-apariciuxf3n-de-dios-en-el-sinauxed}{%
\subsection{La aterradora aparición de Dios en el
Sinaí}\label{la-aterradora-apariciuxf3n-de-dios-en-el-sinauxed}}

\bibleverse{16} Al tercer día, al amanecer, hubo truenos y relámpagos, y
una espesa nube sobre la montaña, y el sonido de una trompeta muy
fuerte; y todo el pueblo que estaba en el campamento tembló. \footnote{\textbf{19:16}
  Heb 12,21} \bibleverse{17} Moisés sacó al pueblo del campamento para
ir al encuentro de Dios, y se paró en la parte baja del monte.
\bibleverse{18} Todo el monte Sinaí humeaba, porque Yahvé descendía
sobre él en fuego; y su humo subía como el humo de un horno, y todo el
monte temblaba en gran manera. \bibleverse{19} Cuando el sonido de la
trompeta se hizo más y más fuerte, Moisés habló, y Dios le respondió con
una voz. \footnote{\textbf{19:19} Hech 7,38} \bibleverse{20} Yahvé bajó
al monte Sinaí, a la cima de la montaña. Yahvé llamó a Moisés a la cima
del monte, y Moisés subió.

\bibleverse{21} Yahvé dijo a Moisés: ``Baja y advierte al pueblo, no sea
que se abran paso hacia Yahvé para mirar, y muchos de ellos perezcan.
\bibleverse{22} Que también los sacerdotes que se acercan a Yahvé se
santifiquen, no sea que Yahvé irrumpa sobre ellos.''

\bibleverse{23} Moisés dijo a Yahvé: ``El pueblo no puede subir al monte
Sinaí, porque tú nos advertiste diciendo: ``Poned límites alrededor del
monte y santificadlo''\,''.

\bibleverse{24} Yahvé le dijo: ``¡Baja! Harás subir a Aarón contigo,
pero no permitas que los sacerdotes y el pueblo se abran paso para subir
a Yahvé, no sea que él se revele contra ellos.''

\bibleverse{25} Entonces Moisés bajó al pueblo y les dijo.

\hypertarget{la-proclamaciuxf3n-de-los-diez-mandamientos}{%
\subsection{La proclamación de los Diez
Mandamientos}\label{la-proclamaciuxf3n-de-los-diez-mandamientos}}

\hypertarget{section-19}{%
\section{20}\label{section-19}}

\bibleverse{1} Dios\footnote{\textbf{20:1} Después de ``Dios'', el
  hebreo tiene las dos letras ``Aleph Tav'' (la primera y la última del
  alfabeto hebreo), no como una palabra, sino como un marcador
  gramatical.} pronunció todas estas palabras, diciendo: \footnote{\textbf{20:1}
  Deut 5,6-19; Mat 5,17-48} \bibleverse{2} ``Yo soy Yahvé, tu Dios, que
te sacó de la tierra de Egipto, de la casa de servidumbre.

\bibleverse{3} ``No tendréis otros dioses delante de mí. \footnote{\textbf{20:3}
  Deut 6,4-5; 1Cor 8,6}

\bibleverse{4} ``No os haréis ningún ídolo, ni ninguna imagen de lo que
está arriba en los cielos, ni abajo en la tierra, ni en las aguas debajo
de la tierra: \footnote{\textbf{20:4} Lev 26,1; Deut 27,15; Sal 97,7; Is
  40,18-26; Rom 1,23} \bibleverse{5} no os inclinaréis ante ellos, ni
los serviréis, porque yo, Yahvé, vuestro Dios, soy un Dios celoso, que
visita la iniquidad de los padres en los hijos, en la tercera y en la
cuarta generación de los que me odian, \footnote{\textbf{20:5} Éxod
  34,7; Jer 31,29-30; Ezeq 18,2-3; Ezeq 18,20} \bibleverse{6} y que
muestra bondad amorosa a miles de los que me aman y guardan mis
mandamientos.

\bibleverse{7} ``No harás mal uso del nombre de Yahvé, tu
Dios,\footnote{\textbf{20:7} o ``No tomarás el nombre de Yahvé, tu Dios,
  en vano} porque Yahvé no declarará inocente al que haga mal uso de su
nombre. \footnote{\textbf{20:7} Lev 24,16}

\bibleverse{8} ``Acuérdate del día de reposo para santificarlo.
\footnote{\textbf{20:8} Éxod 16,25; Ezeq 20,12; Mar 2,27-28; Col 2,16-17}
\bibleverse{9} Trabajarás seis días y harás todo tu trabajo,
\bibleverse{10} pero el séptimo día es sábado para Yahvé tu Dios. No
harás ningún trabajo en él, ni tú, ni tu hijo, ni tu hija, ni tu siervo,
ni tu sierva, ni tu ganado, ni tu extranjero que esté dentro de tus
puertas; \bibleverse{11} porque en seis días Yahvé hizo los cielos y la
tierra, el mar y todo lo que hay en ellos, y descansó el séptimo día;
por eso Yahvé bendijo el día de reposo y lo santificó. \footnote{\textbf{20:11}
  Gén 2,2-3}

\bibleverse{12} ``Honra a tu padre y a tu madre, para que tus días se
alarguen en la tierra que el Señor tu Dios te da. \footnote{\textbf{20:12}
  Mat 15,4; Efes 6,2-3}

\bibleverse{13} ``No matarás. \footnote{\textbf{20:13} Éxod 21,12; Gén
  9,5-6; Sant 2,11}

\bibleverse{14} ``No cometerás adulterio. \footnote{\textbf{20:14} Lev
  20,10; Efes 5,3-5}

\bibleverse{15} ``No robarás. \footnote{\textbf{20:15} Lev 19,11; Efes
  4,28}

\bibleverse{16} ``No darás falso testimonio contra tu prójimo.
\footnote{\textbf{20:16} Éxod 23,1; Efes 4,25}

\bibleverse{17} ``No codiciarás la casa de tu prójimo. No codiciarás la
mujer de tu prójimo, ni su siervo, ni su sierva, ni su buey, ni su asno,
ni nada que sea de tu prójimo''. \footnote{\textbf{20:17} Rom 7,7; Rom
  13,9}

\hypertarget{el-efecto-de-la-apariciuxf3n-divina-en-la-gente-discurso-tranquilizador-de-moisuxe9s}{%
\subsection{El efecto de la aparición divina en la gente; Discurso
tranquilizador de
Moisés}\label{el-efecto-de-la-apariciuxf3n-divina-en-la-gente-discurso-tranquilizador-de-moisuxe9s}}

\bibleverse{18} Todo el pueblo percibió los truenos, los relámpagos, el
sonido de la trompeta y la montaña humeante. Al verlo, el pueblo tembló
y se mantuvo a distancia. \bibleverse{19} Dijeron a Moisés: ``Habla tú
con nosotros, y te escucharemos; pero no dejes que Dios hable con
nosotros, no sea que muramos.''

\bibleverse{20} Moisés dijo al pueblo: ``No tengan miedo, porque Dios ha
venido a probarlos, y para que su temor esté ante ustedes, para que no
pequen''. \bibleverse{21} El pueblo se mantuvo a distancia, y Moisés se
acercó a la espesa oscuridad donde estaba Dios. \footnote{\textbf{20:21}
  Heb 12,18}

\hypertarget{orden-provisional-de-culto}{%
\subsection{Orden provisional de
culto}\label{orden-provisional-de-culto}}

\bibleverse{22} El Señor dijo a Moisés: ``Esto es lo que les dirás a los
hijos de Israel: `Ustedes mismos han visto que yo he hablado con ustedes
desde el cielo. \bibleverse{23} No os haréis dioses de plata ni dioses
de oro para estar junto a mí. \bibleverse{24} Haréis un altar de tierra
para mí, y sacrificaréis en él vuestros holocaustos y vuestras ofrendas
de paz, vuestras ovejas y vuestros ganados. En todo lugar donde registre
mi nombre vendré a ti y te bendeciré. \footnote{\textbf{20:24} Éxod
  27,1; Éxod 27,8; Éxod 29,42-43; Deut 12,5} \bibleverse{25} Si me haces
un altar de piedra, no lo construirás de piedras cortadas; porque si
alzas tu herramienta sobre él, lo habrás contaminado. \footnote{\textbf{20:25}
  Deut 27,5; Jos 8,31} \bibleverse{26} No subirás por las escaleras a mi
altar, para que tu desnudez no quede expuesta a él'.

\hypertarget{los-derechos-de-los-esclavos-hebreos}{%
\subsection{Los derechos de los esclavos
hebreos}\label{los-derechos-de-los-esclavos-hebreos}}

\hypertarget{section-20}{%
\section{21}\label{section-20}}

\bibleverse{1} ``Éstas son las ordenanzas que les pondrás delante:

\bibleverse{2} ``Si compras un siervo hebreo, servirá seis años, y al
séptimo saldrá libre sin pagar nada. \footnote{\textbf{21:2} Lev
  25,39-40; Deut 15,12-17; Jer 34,14} \bibleverse{3} Si entra solo,
saldrá solo. Si está casado, su mujer saldrá con él. \bibleverse{4} Si
su amo le da una esposa y ella le da hijos o hijas, la esposa y sus
hijos serán de su amo, y él saldrá solo. \bibleverse{5} Pero si el
siervo dice claramente: ``Amo a mi amo, a mi mujer y a mis hijos. No
saldré libre;' \bibleverse{6} entonces su amo lo llevará a Dios, y lo
traerá a la puerta o al poste de la puerta, y su amo le perforará la
oreja con un punzón, y le servirá para siempre. \footnote{\textbf{21:6}
  Éxod 22,8-9; Éxod 22,28; Deut 1,17}

\bibleverse{7} ``Si un hombre vende a su hija para que sea sierva, no
saldrá como los siervos. \footnote{\textbf{21:7} Éxod 21,2}
\bibleverse{8} Si no le gusta a su amo, que la ha casado consigo,
entonces la dejará rescatar. No tendrá derecho a venderla a un pueblo
extranjero, ya que ha actuado con engaño con ella. \bibleverse{9} Si la
casa con su hijo, la tratará como a una hija. \bibleverse{10} Si toma
otra esposa para sí, no disminuirá su comida, su ropa ni sus derechos
matrimoniales. \bibleverse{11} Si no hace estas tres cosas por ella,
podrá quedar libre sin pagar nada.

\hypertarget{disposiciones-para-la-protecciuxf3n-de-la-vida-humana}{%
\subsection{Disposiciones para la protección de la vida
humana}\label{disposiciones-para-la-protecciuxf3n-de-la-vida-humana}}

\bibleverse{12} ``El que golpee a un hombre de modo que muera, será
ciertamente condenado a muerte, \footnote{\textbf{21:12} Gén 9,6; Éxod
  20,13; Mat 5,21-22} \bibleverse{13} pero no si es involuntario, sino
que Dios permite que ocurra; entonces te designaré un lugar donde huirá.
\footnote{\textbf{21:13} Núm 35,6-29; Deut 19,4-13} \bibleverse{14} Si
un hombre trama y se acerca presuntuosamente a su prójimo para matarlo,
lo sacarás de mi altar para que muera. \footnote{\textbf{21:14} 1Re
  2,29; 1Re 2,31}

\bibleverse{15} ``Cualquiera que ataque a su padre o a su madre será
condenado a muerte.

\bibleverse{16} ``Cualquiera que secuestre a alguien y lo venda, o si lo
encuentra en su mano, será condenado a muerte. \footnote{\textbf{21:16}
  Deut 24,7; 1Tim 1,10}

\bibleverse{17} ``Cualquiera que maldiga a su padre o a su madre será
condenado a muerte. \footnote{\textbf{21:17} Deut 27,16; Prov 20,20; Mat
  15,4}

\bibleverse{18} ``Si los hombres riñen y uno golpea al otro con una
piedra o con su puño, y no muere, sino que queda confinado en la cama;
\bibleverse{19} si se levanta de nuevo y camina con su bastón, entonces
el que lo golpeó quedará libre de culpa; sólo que pagará por la pérdida
de su tiempo, y proveerá a su curación hasta que esté completamente
curado.

\bibleverse{20} ``Si un hombre golpea a su siervo o a su sierva con una
vara, y éste muere bajo su mano, el hombre será castigado.
\bibleverse{21} Sin embargo, si su siervo se levanta después de uno o
dos días, no será castigado, porque el siervo es de su propiedad.

\bibleverse{22} ``Si los hombres pelean y hieren a una mujer embarazada
de modo que dé a luz prematuramente, y sin embargo no se produce ningún
daño, se le impondrá la multa que el marido de la mujer exija y los
jueces permitan. \bibleverse{23} Pero si se produce algún daño, entonces
hay que quitar vida por vida, \footnote{\textbf{21:23} Lev 24,19-20;
  Deut 19,21; Mat 5,38} \bibleverse{24} ojo por ojo, diente por diente,
mano por mano, pie por pie, \bibleverse{25} quemadura por quemadura,
herida por herida y contusión por contusión.

\bibleverse{26} ``Si un hombre golpea el ojo de su siervo o de su sierva
y lo destruye, lo dejará libre por su ojo. \bibleverse{27} Si golpea el
diente de su siervo o de su sierva, lo dejará libre por su diente.

\hypertarget{indemnizaciuxf3n-por-muerte-o-lesiones-a-una-persona-por-animales}{%
\subsection{Indemnización por muerte o lesiones a una persona por
animales}\label{indemnizaciuxf3n-por-muerte-o-lesiones-a-una-persona-por-animales}}

\bibleverse{28} ``Si un toro mata a un hombre o a una mujer de una
cornada, el toro será apedreado y su carne no se comerá; pero el dueño
del toro no será responsable. \bibleverse{29} Pero si el toro ha tenido
la costumbre de cornear en el pasado, y esto ha sido atestiguado a su
dueño, y éste no lo ha guardado, pero ha matado a un hombre o a una
mujer, el toro será apedreado, y su dueño también será condenado a
muerte. \footnote{\textbf{21:29} Gén 9,5} \bibleverse{30} Si se le
impone un rescate, deberá dar por la redención de su vida lo que se le
imponga. \bibleverse{31} Tanto si ha corneado a un hijo como si ha
corneado a una hija, según este juicio se hará con él. \bibleverse{32}
Si el toro cornea a un siervo o a una sierva, se darán treinta
siclos\footnote{\textbf{21:32} Un siclo equivale a unos 10 gramos o a
  unas 0,35 onzas, por lo que 30 siclos son unos 300 gramos o unas 10,6
  onzas.} de plata a su amo, y el buey será apedreado.

\hypertarget{disposiciones-para-la-protecciuxf3n-de-la-propiedad}{%
\subsection{Disposiciones para la protección de la
propiedad}\label{disposiciones-para-la-protecciuxf3n-de-la-propiedad}}

\bibleverse{33} ``Si un hombre abre una fosa, o si un hombre cava una
fosa y no la cubre, y un toro o un asno cae en ella, \bibleverse{34} el
dueño de la fosa deberá repararla. Dará dinero a su dueño, y el animal
muerto será suyo.

\bibleverse{35} ``Si el toro de un hombre hiere al de otro, de modo que
muera, entonces venderán el toro vivo y dividirán su precio; y también
dividirán el animal muerto. \bibleverse{36} O si se sabe que el toro
tenía la costumbre de corneado en el pasado, y su dueño no lo ha
guardado, pagará ciertamente toro por toro, y el animal muerto será
suyo.

\hypertarget{section-21}{%
\section{22}\label{section-21}}

\bibleverse{1} ``Si un hombre roba un buey o una oveja, y lo mata o lo
vende, deberá pagar cinco bueyes por un buey, y cuatro ovejas por una
oveja. \footnote{\textbf{22:1} Luc 19,8} \bibleverse{2} Si el ladrón es
encontrado entrando a la fuerza, y es golpeado de tal manera que muere,
no habrá culpa de sangre para él. \bibleverse{3} Si el sol ha salido
sobre él, es culpable de derramamiento de sangre. Deberá restituirlo. Si
no tiene nada, será vendido por su robo. \bibleverse{4} Si la propiedad
robada se encuentra en su mano con vida, ya sea buey, burro u oveja,
deberá pagar el doble.

\bibleverse{5} ``Si un hombre hace comer un campo o una viña dejando
suelto a su animal, y éste pasta en el campo de otro, deberá restituirlo
con lo mejor de su campo y con lo mejor de su viña.

\bibleverse{6} ``Si se produce un incendio y se prende en las espinas,
de modo que se consuman las mazorcas, el grano en pie o el campo, el que
encendió el fuego deberá restituirlo.

\hypertarget{apropiaciuxf3n-indebida-puxe9rdida-o-dauxf1o-de-bienes-que-se-le-hayan-confiado}{%
\subsection{Apropiación indebida, pérdida o daño de bienes que se le
hayan
confiado}\label{apropiaciuxf3n-indebida-puxe9rdida-o-dauxf1o-de-bienes-que-se-le-hayan-confiado}}

\bibleverse{7} ``Si un hombre entrega a su prójimo dinero o cosas para
que las guarde, y se las roban en su casa, si el ladrón es encontrado,
deberá pagar el doble. \bibleverse{8} Si no se encuentra al ladrón, el
dueño de la casa se acercará a Dios para averiguar si ha metido la mano
en los bienes de su prójimo. \bibleverse{9} En todo asunto de
transgresión, ya sea por buey, por asno, por oveja, por ropa o por
cualquier cosa perdida, sobre la que uno diga: ``Esto es mío'', la causa
de ambas partes se presentará ante Dios. Aquel a quien Dios condene
pagará el doble a su prójimo.

\bibleverse{10} ``Si un hombre entrega a su prójimo un asno, un buey,
una oveja o cualquier otro animal para que lo guarde, y éste muere o se
daña, o se aleja, sin que nadie lo vea; \bibleverse{11} el juramento de
Yahvé será entre ambos, no ha puesto su mano en los bienes de su
prójimo; y su dueño lo aceptará, y no hará restitución. \bibleverse{12}
Pero si se lo roban, el que lo robó deberá restituirlo a su dueño.
\bibleverse{13} Si se ha roto en pedazos, que lo traiga como prueba. No
deberá restituir lo que fue roto. \footnote{\textbf{22:13} Gén 31,39}

\bibleverse{14} ``Si un hombre toma prestado algo de su prójimo, y se
daña o muere, sin que su dueño esté con él, deberá restituirlo.
\bibleverse{15} Si su dueño está con ella, no deberá restituirla. Si se
trata de una cosa alquilada, vendrá por su alquiler.

\hypertarget{varias-normas-relativas-a-dios-la-moral-y-la-caridad}{%
\subsection{Varias normas relativas a Dios, la moral y la
caridad}\label{varias-normas-relativas-a-dios-la-moral-y-la-caridad}}

\bibleverse{16} ``Si un hombre seduce a una virgen que no está
comprometida para casarse y se acuesta con ella, deberá pagar una dote
para que sea su esposa. \footnote{\textbf{22:16} Deut 22,28-29}
\bibleverse{17} Si el padre de ella se niega rotundamente a dársela,
deberá pagar el dinero correspondiente a la dote de las vírgenes.

\bibleverse{18} ``No permitirás que viva una hechicera. \footnote{\textbf{22:18}
  Lev 20,6; Lev 20,27; Deut 18,10; 1Sam 28,9}

\bibleverse{19} ``El que tenga relaciones sexuales con un animal será
condenado a muerte. \footnote{\textbf{22:19} Lev 18,23; Deut 27,21}

\bibleverse{20} ``El que ofrezca sacrificios a cualquier dios, excepto a
Yahvé solamente, será destruido por completo. \footnote{\textbf{22:20}
  Deut 13,6-18; Deut 17,2-7}

\bibleverse{21} ``No agraviarás al extranjero ni lo oprimirás, pues
fuisteis extranjeros en la tierra de Egipto. \footnote{\textbf{22:21}
  Éxod 23,9; Lev 19,33-34; Deut 10,18-19; Deut 24,17-18; Deut 27,19}

\bibleverse{22} ``No te aprovecharás de ninguna viuda ni de ningún
huérfano. \footnote{\textbf{22:22} Is 1,17} \bibleverse{23} Si os
aprovecháis de ellos, y ellos claman a mí, ciertamente oiré su clamor;
\bibleverse{24} y mi ira se encenderá, y os mataré a espada; y vuestras
mujeres serán viudas, y vuestros hijos huérfanos.

\bibleverse{25} ``Si prestas dinero a alguno de mi pueblo que esté
contigo y sea pobre, no serás para él como un acreedor. No le cobrarás
intereses. \footnote{\textbf{22:25} Lev 25,36; Deut 23,19; Deut 24,10}
\bibleverse{26} Si tomas el vestido de tu prójimo como garantía, se lo
devolverás antes de que se ponga el sol, \footnote{\textbf{22:26} Deut
  24,12-13} \bibleverse{27} porque es su única cobertura, es su vestido
para su piel. ¿Con qué va a dormir? Sucederá que cuando clame a mí, yo
lo escucharé, porque soy clemente.

\bibleverse{28} ``No blasfemarás a Dios, ni maldecirás a un gobernante
de tu pueblo. \footnote{\textbf{22:28} Éxod 21,6; Ecl 10,20; Hech 23,5}

\bibleverse{29} ``No te demorarás en ofrecer de tu cosecha y de la
salida de tus lagares. ``Me darás el primogénito de tus hijos.
\footnote{\textbf{22:29} Deut 18,4; Éxod 13,2; Éxod 13,13}
\bibleverse{30} Lo mismo harás con tu ganado y con tus ovejas. Estará
con su madre siete días, y al octavo día me lo darás. \footnote{\textbf{22:30}
  Lev 22,27}

\bibleverse{31} ``Seréis hombres santos para mí, por lo que no comeréis
ninguna carne desgarrada por los animales en el campo. Se la echarán a
los perros. \footnote{\textbf{22:31} Lev 7,24; Lev 11,40; Lev 17,15; Lev
  22,8; Deut 14,21; Ezeq 44,31}

\hypertarget{comportamiento-veraz-y-honesto-especialmente-en-la-corte}{%
\subsection{Comportamiento veraz y honesto, especialmente en la
corte}\label{comportamiento-veraz-y-honesto-especialmente-en-la-corte}}

\hypertarget{section-22}{%
\section{23}\label{section-22}}

\bibleverse{1} ``No difundirás una noticia falsa. No juntes tu mano con
la del malvado para ser un testigo malicioso. \footnote{\textbf{23:1}
  Éxod 20,16}

\bibleverse{2} ``No seguirás a una multitud para hacer el mal. No
testificarás en la corte para ponerte del lado de una multitud para
pervertir la justicia. \bibleverse{3} No favorecerás al pobre en su
causa. \footnote{\textbf{23:3} Lev 19,15}

\bibleverse{4} ``Si encuentras al buey de tu enemigo o a su asno
extraviado, lo harás volver a él. \footnote{\textbf{23:4} Luc 6,27}
\bibleverse{5} Si ves que el asno del que te odia se ha caído bajo su
carga, no lo dejes. Lo ayudarás con toda seguridad.

\bibleverse{6} ``No negarás la justicia a tu pueblo pobre en sus
pleitos. \footnote{\textbf{23:6} Deut 27,19}

\bibleverse{7} ``Aléjate de una acusación falsa y no mates al inocente y
al justo, porque no justificaré al impío.

\bibleverse{8} ``No aceptarás soborno, porque el soborno ciega a los que
tienen vista y pervierte las palabras de los justos. \footnote{\textbf{23:8}
  Deut 16,19; Deut 27,25}

\bibleverse{9} ``No oprimirás al extranjero, pues conoces el corazón del
extranjero, ya que fuisteis extranjeros en la tierra de Egipto.
\footnote{\textbf{23:9} Éxod 22,21}

\hypertarget{disposiciones-para-los-auxf1os-sabuxe1ticos-las-fiestas-y-los-sacrificios}{%
\subsection{Disposiciones para los años sabáticos, las fiestas y los
sacrificios}\label{disposiciones-para-los-auxf1os-sabuxe1ticos-las-fiestas-y-los-sacrificios}}

\bibleverse{10} ``Durante seis años sembrarás tu tierra y recogerás sus
frutos, \footnote{\textbf{23:10} Lev 25,1; Deut 15,1-11} \bibleverse{11}
pero el séptimo año la dejarás descansar y en barbecho, para que coman
los pobres de tu pueblo; y lo que dejen lo comerá el animal del campo.
Del mismo modo, te ocuparás de tu viña y de tu olivar.

\bibleverse{12} ``Seis días harás tu trabajo, y el séptimo día
descansarás, para que tu buey y tu asno tengan descanso, y el hijo de tu
siervo y el extranjero se refresquen. \footnote{\textbf{23:12} Éxod
  20,8-11}

\bibleverse{13} ``Cuida de hacer todo lo que te he dicho; y no invoques
el nombre de otros dioses ni dejes que se oiga de tu boca. \footnote{\textbf{23:13}
  Jos 23,7}

\bibleverse{14} ``Celebrarás una fiesta para mí tres veces al año.
\bibleverse{15} Celebrarás la fiesta de los panes sin levadura. Siete
días comeréis panes sin levadura, como os he mandado, en el tiempo
señalado en el mes de Abib (porque en él salisteis de Egipto), y nadie
se presentará vacío ante mí. \footnote{\textbf{23:15} Éxod 12,15}
\bibleverse{16} Y la fiesta de la cosecha, con los primeros frutos de
vuestras labores, que sembréis en el campo; y la fiesta de la
recolección, al final del año, cuando recojáis vuestras labores del
campo. \bibleverse{17} Tres veces al año se presentarán todos vuestros
varones ante el Señor Yahvé.

\bibleverse{18} ``No ofrecerás la sangre de mi sacrificio con pan
leudado. La grasa de mi fiesta no permanecerá toda la noche hasta la
mañana. \footnote{\textbf{23:18} Éxod 12,10}

\bibleverse{19} Traerás las primicias de tu tierra a la casa de Yahvé,
tu Dios. ``No hervirás un cabrito en la leche de su madre. \footnote{\textbf{23:19}
  Deut 26,1-11; Éxod 22,30; Deut 14,21}

\hypertarget{advertencia-final-sobre-la-expulsiuxf3n-de-los-cananeos-promesa-de-asistencia-y-bendiciones-en-caso-de-obediencia-fiel}{%
\subsection{Advertencia final sobre la expulsión de los cananeos;
Promesa de asistencia y bendiciones en caso de obediencia
fiel}\label{advertencia-final-sobre-la-expulsiuxf3n-de-los-cananeos-promesa-de-asistencia-y-bendiciones-en-caso-de-obediencia-fiel}}

\bibleverse{20} ``He aquí que yo envío un ángel delante de ti, para que
te guarde en el camino y te lleve al lugar que he preparado. \footnote{\textbf{23:20}
  Éxod 14,19} \bibleverse{21} Presta atención a él y escucha su voz. No
lo provoques, porque no perdonará tu desobediencia, pues mi nombre está
en él. \footnote{\textbf{23:21} Is 63,9-10} \bibleverse{22} Pero si en
verdad escuchas su voz y haces todo lo que yo digo, seré enemigo de tus
enemigos y adversario de tus adversarios. \bibleverse{23} Porque mi
ángel irá delante de ti y te llevará ante el amorreo, el hitita, el
ferezeo, el cananeo, el heveo y el jebuseo, y los eliminaré.
\bibleverse{24} No te inclinarás ante sus dioses, ni los servirás, ni
seguirás sus prácticas, sino que los derrocarás por completo y demolerás
sus pilares. \footnote{\textbf{23:24} Éxod 20,5; Lev 18,3}
\bibleverse{25} Servirás a Yahvé, tu Dios, y él bendecirá tu pan y tu
agua, y quitaré la enfermedad de en medio de ti. \footnote{\textbf{23:25}
  Éxod 15,26} \bibleverse{26} Nadie abortará ni será estéril en tu
tierra. Cumpliré el número de tus días. \bibleverse{27} Enviaré mi
terror delante de ti y confundiré a todos los pueblos a los que vayas, y
haré que todos tus enemigos te den la espalda. \bibleverse{28} Enviaré
el avispón delante de ti, que expulsará al heveo, al cananeo y al hitita
de tu presencia. \footnote{\textbf{23:28} Deut 1,44; Deut 7,20; Jos
  24,12} \bibleverse{29} No los expulsaré de delante de ti en un año, no
sea que la tierra quede desolada y los animales del campo se
multipliquen contra ti. \bibleverse{30} Poco a poco los expulsaré de
delante de ti, hasta que te hayas multiplicado y hayas heredado la
tierra. \bibleverse{31} Fijaré tu frontera desde el Mar Rojo hasta el
mar de los filisteos, y desde el desierto hasta el río; porque entregaré
en tu mano a los habitantes de la tierra, y los expulsarás delante de
ti. \footnote{\textbf{23:31} Gén 15,18} \bibleverse{32} No harás ningún
pacto con ellos, ni con sus dioses. \footnote{\textbf{23:32} Éxod 34,12;
  Deut 7,2} \bibleverse{33} No habrán de habitar en tu tierra, para que
no te hagan pecar contra mí, pues si sirves a sus dioses, ciertamente
será una trampa para ti.'' \footnote{\textbf{23:33} Jue 2,3}

\hypertarget{conclusiuxf3n-solemne-del-pacto-redacciuxf3n-y-lectura-del-libro-federal-el-sacrificio-del-pacto-y-la-aspersiuxf3n-de-sangre}{%
\subsection{Conclusión solemne del pacto; Redacción y lectura del libro
federal; el sacrificio del pacto y la aspersión de
sangre}\label{conclusiuxf3n-solemne-del-pacto-redacciuxf3n-y-lectura-del-libro-federal-el-sacrificio-del-pacto-y-la-aspersiuxf3n-de-sangre}}

\hypertarget{section-23}{%
\section{24}\label{section-23}}

\bibleverse{1} Le dijo a Moisés: ``Sube a Yahvé, tú, y Aarón, Nadab y
Abiú, y setenta de los ancianos de Israel; y adora desde lejos.
\footnote{\textbf{24:1} Núm 11,16} \bibleverse{2} Sólo Moisés se
acercará a Yavé, pero ellos no se acercarán. El pueblo no subirá con
él''.

\bibleverse{3} Llegó Moisés y contó al pueblo todas las palabras de
Yahvé y todas las ordenanzas; y todo el pueblo respondió a una sola voz
y dijo: ``Todas las palabras que Yahvé ha dicho las pondremos en
práctica.'' \footnote{\textbf{24:3} Éxod 19,8}

\bibleverse{4} Moisés escribió todas las palabras de Yavé, luego se
levantó de madrugada y construyó un altar al pie de la montaña, con doce
pilares para las doce tribus de Israel. \footnote{\textbf{24:4} Éxod
  34,27; 1Re 18,31} \bibleverse{5} Envió a jóvenes de los hijos de
Israel, que ofrecieron holocaustos y sacrificaron ofrendas de paz de
ganado a Yavé. \footnote{\textbf{24:5} Éxod 3,12} \bibleverse{6} Moisés
tomó la mitad de la sangre y la puso en cuencos, y la otra mitad la
roció sobre el altar. \bibleverse{7} Tomó el libro de la alianza y lo
leyó a la vista del pueblo, que dijo: ``Haremos todo lo que Yahvé ha
dicho y seremos obedientes''. \footnote{\textbf{24:7} Éxod 24,4}

\bibleverse{8} Moisés tomó la sangre, la roció sobre el pueblo y dijo:
``Miren, ésta es la sangre de la alianza que Yahvé ha hecho con ustedes
sobre todas estas palabras.'' \footnote{\textbf{24:8} Heb 9,19-22}

\hypertarget{los-setenta-ancianos-de-los-israelitas-en-el-sinauxed-ante-dios}{%
\subsection{Los setenta ancianos de los israelitas en el Sinaí ante
Dios}\label{los-setenta-ancianos-de-los-israelitas-en-el-sinauxed-ante-dios}}

\bibleverse{9} Entonces subieron Moisés, Aarón, Nadab, Abiú y setenta de
los ancianos de Israel. \bibleverse{10} Vieron al Dios de Israel. Bajo
sus pies había como una obra de piedra de zafiro,\footnote{\textbf{24:10}
  o, lapislázuli} como los cielos por su claridad. \footnote{\textbf{24:10}
  Ezeq 1,26} \bibleverse{11} No puso su mano sobre los nobles de los
hijos de Israel. Vieron a Dios, y comieron y bebieron. \footnote{\textbf{24:11}
  Éxod 33,20-23}

\hypertarget{moisuxe9s-permanece-en-el-sinauxed-durante-cuarenta-duxedas}{%
\subsection{Moisés permanece en el Sinaí durante cuarenta
días}\label{moisuxe9s-permanece-en-el-sinauxed-durante-cuarenta-duxedas}}

\bibleverse{12} Yahvé dijo a Moisés: ``Sube a mí en la montaña y quédate
aquí, y te daré las tablas de piedra con la ley y los mandamientos que
he escrito, para que los enseñes.'' \footnote{\textbf{24:12} Éxod 31,18}

\bibleverse{13} Moisés se levantó con Josué, su siervo, y subió a la
Montaña de Dios. \bibleverse{14} Dijo a los ancianos: ``Esperadnos aquí,
hasta que volvamos a vosotros. He aquí que Aarón y Hur están con
ustedes. El que esté involucrado en una disputa puede acudir a ellos''.

\bibleverse{15} Moisés subió al monte, y la nube cubrió la montaña.
\bibleverse{16} La gloria de Yahvé se posó en el monte Sinaí, y la nube
lo cubrió durante seis días. Al séptimo día llamó a Moisés desde el
centro de la nube. \footnote{\textbf{24:16} Éxod 16,10} \bibleverse{17}
La apariencia de la gloria de Yahvé era como un fuego devorador en la
cima de la montaña a los ojos de los hijos de Israel. \footnote{\textbf{24:17}
  Deut 4,24; Deut 9,3; Heb 12,29} \bibleverse{18} Moisés entró en medio
de la nube y subió a la montaña; y Moisés estuvo en la montaña cuarenta
días y cuarenta noches. \footnote{\textbf{24:18} Éxod 34,28}

\hypertarget{regulaciones-sobre-la-construcciuxf3n-y-equipamiento-del-tabernuxe1culo}{%
\subsection{Regulaciones sobre la construcción y equipamiento del
tabernáculo}\label{regulaciones-sobre-la-construcciuxf3n-y-equipamiento-del-tabernuxe1culo}}

\hypertarget{section-24}{%
\section{25}\label{section-24}}

\bibleverse{1} Yahvé habló a Moisés diciendo: \bibleverse{2} ``Habla a
los hijos de Israel para que tomen una ofrenda para mí. De todo aquel
cuyo corazón lo haga querer, tomarás mi ofrenda. \footnote{\textbf{25:2}
  Éxod 35,5; Éxod 35,22} \bibleverse{3} Esta es la ofrenda que tomarás
de ellos: oro, plata, bronce, \bibleverse{4} azul, púrpura, escarlata,
lino fino, pelo de cabra, \bibleverse{5} pieles de carnero teñidas de
rojo, cueros de vacas marinas,\footnote{\textbf{25:5} Un codo es la
  longitud desde la punta del dedo corazón hasta el codo del brazo de un
  hombre, es decir, unas 18 pulgadas o 46 centímetros.} madera de
acacia, \bibleverse{6} aceite para la luz, especias para el aceite de la
unción y para el incienso aromático, \bibleverse{7} piedras de ónice, y
piedras de engaste para el efod y para el pectoral. \bibleverse{8} Que
me hagan un santuario, para que yo habite en medio de ellos.
\bibleverse{9} Conforme a todo lo que te muestre, el modelo del
tabernáculo y el modelo de todos sus muebles, así lo harás. \footnote{\textbf{25:9}
  Éxod 25,40}

\hypertarget{instrucciones-para-hacer-los-implementos-sagrados}{%
\subsection{Instrucciones para hacer los implementos
sagrados}\label{instrucciones-para-hacer-los-implementos-sagrados}}

\bibleverse{10} ``Harán un arca de madera de acacia. Su longitud será de
dos codos y medio, su anchura un codo y medio, y un codo y medio su
altura. \bibleverse{11} La recubrirás de oro puro. Lo recubrirás por
dentro y por fuera, y harás una moldura de oro alrededor.
\bibleverse{12} Le fundirás cuatro anillos de oro y los pondrás en sus
cuatro pies. Dos anillos estarán a un lado de él, y dos anillos al otro
lado. \bibleverse{13} Harás varas de madera de acacia y las recubrirás
de oro. \bibleverse{14} Pondrás las varas en las argollas a los lados
del arca para transportarla. \bibleverse{15} Las varas estarán en los
anillos del arca. No se sacarán de ella. \bibleverse{16} Pondrás en el
arca el pacto que yo te daré. \footnote{\textbf{25:16} Éxod 25,21}
\bibleverse{17} Harás un propiciatorio de oro puro. Su longitud será de
dos codos y medio, y su anchura de codo y medio. \footnote{\textbf{25:17}
  Heb 4,16} \bibleverse{18} Harás dos querubines de oro martillado. Los
harás en los dos extremos del propiciatorio. \bibleverse{19} Haz un
querubín en un extremo y un querubín en el otro. Harás los querubines en
sus dos extremos de una sola pieza con el propiciatorio. \bibleverse{20}
Los querubines extenderán sus alas hacia arriba, cubriendo el
propiciatorio con sus alas, con sus rostros uno hacia el otro. Los
rostros de los querubines estarán hacia el propiciatorio.
\bibleverse{21} Pondrás el propiciatorio encima del arca, y en el arca
pondrás el pacto que yo te daré. \footnote{\textbf{25:21} Éxod 34,29;
  1Re 8,9; Heb 9,4} \bibleverse{22} Allí me reuniré contigo, y te diré
desde arriba del propiciatorio, desde entre los dos querubines que están
sobre el arca de la alianza, todo lo que te mando para los hijos de
Israel. \footnote{\textbf{25:22} Núm 7,89}

\bibleverse{23} ``Harás una mesa de madera de acacia. Su longitud será
de dos codos, su anchura de un codo y su altura de un codo y medio.
\bibleverse{24} La recubrirás de oro puro y le harás una moldura de oro
alrededor. \bibleverse{25} Harás un borde de un palmo de ancho
alrededor. Harás una moldura de oro en su borde alrededor.
\bibleverse{26} Le harás cuatro anillos de oro y los pondrás en las
cuatro esquinas que están sobre sus cuatro pies. \bibleverse{27} Los
anillos estarán cerca del borde, como lugares para las varas para llevar
la mesa. \bibleverse{28} Harás las varas de madera de acacia y las
recubrirás de oro, para que la mesa pueda ser transportada con ellas.
\bibleverse{29} Harás sus platos, sus cucharas, sus cucharones y sus
tazones con los que se vierten las ofrendas. Los harás de oro puro.
\bibleverse{30} En la mesa pondrás siempre el pan de la presencia
delante de mí. \footnote{\textbf{25:30} Lev 24,5-6}

\bibleverse{31} ``Harás un candelabro de oro puro. El candelabro se hará
de obra martillada. Su base, su fuste, sus copas, sus capullos y sus
flores serán de una sola pieza con él. \bibleverse{32} De sus lados
saldrán seis ramas: tres brazos del candelabro salen de un lado, y tres
brazos del candelabro salen del otro lado; \bibleverse{33} tres copas
hechas como flores de almendro en un brazo, un capullo y una flor; y
tres copas hechas como flores de almendro en el otro brazo, un capullo y
una flor, así para los seis brazos que salen del candelabro;
\bibleverse{34} y en el candelabro cuatro copas hechas como flores de
almendro, sus capullos y sus flores; \bibleverse{35} y un capullo debajo
de dos ramas de una pieza con él, y un capullo debajo de dos ramas de
una pieza con él, y un capullo debajo de dos ramas de una pieza con él,
para las seis ramas que salen del candelabro. \bibleverse{36} Sus
capullos y sus ramas serán de una sola pieza con ella, toda ella de una
sola pieza batida de oro puro. \bibleverse{37} Harás sus lámparas de
siete, y ellas encenderán sus lámparas para alumbrar el espacio que está
frente a ella. \bibleverse{38} Sus apagadores y sus tabaqueras serán de
oro puro. \bibleverse{39} Se hará de un talento de oro puro, con todos
estos accesorios. \bibleverse{40} Procura hacerlos según su modelo, que
te ha sido mostrado en la montaña. \footnote{\textbf{25:40} Éxod 26,30;
  Hech 7,44; Heb 8,5}

\hypertarget{instrucciones-para-hacer-el-apartamento-los-cuatro-techos}{%
\subsection{Instrucciones para hacer el apartamento: Los cuatro
techos}\label{instrucciones-para-hacer-el-apartamento-los-cuatro-techos}}

\hypertarget{section-25}{%
\section{26}\label{section-25}}

\bibleverse{1} ``Además, harás el tabernáculo con diez cortinas de lino
fino, azul, púrpura y escarlata, con querubines. Las harás con el
trabajo de un obrero hábil. \bibleverse{2} La longitud de cada cortina
será de veintiocho codos, y la anchura de cada cortina de cuatro codos;
todas las cortinas tendrán una misma medida. \bibleverse{3} Cinco
cortinas estarán unidas entre sí, y las otras cinco cortinas estarán
unidas entre sí. \bibleverse{4} Harás lazos de color azul en el borde de
una de las cortinas desde el borde en el acoplamiento, y harás lo mismo
en el borde de la cortina que está más afuera en el segundo
acoplamiento. \bibleverse{5} Harás cincuenta lazos en la primera
cortina, y harás cincuenta lazos en el borde de la cortina que está en
el segundo acoplamiento. Los lazos estarán uno frente al otro.
\bibleverse{6} Harás cincuenta corchetes de oro y unirás las cortinas
entre sí con los corchetes. El tabernáculo será una unidad.

\bibleverse{7} ``Harás cortinas de pelo de cabra para cubrir el
tabernáculo. Harás once cortinas. \bibleverse{8} La longitud de cada
cortina será de treinta codos, y la anchura de cada cortina de cuatro
codos; las once cortinas tendrán una sola medida. \bibleverse{9}
Acoplarás cinco cortinas solas y seis cortinas solas, y doblarás la
sexta cortina en la parte delantera de la tienda. \bibleverse{10} Harás
cincuenta lazos en el borde de la cortina que está más afuera en el
acople, y cincuenta lazos en el borde de la cortina que está más afuera
en el segundo acople. \bibleverse{11} Harás cincuenta broches de bronce,
los pondrás en las presillas y unirás la tienda para que sea una sola.
\bibleverse{12} La parte que sobresale de las cortinas de la tienda ---
la mitad de la cortina que queda --- colgará sobre la parte posterior
del tabernáculo. \bibleverse{13} El codo de un lado y el codo del otro
lado, de lo que queda de la longitud de las cortinas de la tienda,
colgará sobre los lados del tabernáculo de este lado y del otro, para
cubrirlo. \bibleverse{14} Harás una cubierta para la tienda de pieles de
carnero teñidas de rojo, y una cubierta de pieles de vaca marina por
encima.

\hypertarget{el-marco-de-madera-que-consta-de-tablas-y-cinco-barras}{%
\subsection{El marco de madera, que consta de tablas y cinco
barras}\label{el-marco-de-madera-que-consta-de-tablas-y-cinco-barras}}

\bibleverse{15} ``Harás las tablas para el tabernáculo de madera de
acacia, de pie. \bibleverse{16} La longitud de una tabla será de diez
codos, y la anchura de cada tabla de un codo y medio. \bibleverse{17} En
cada tabla habrá dos espigas unidas entre sí; así harás todas las tablas
del tabernáculo. \bibleverse{18} Harás veinte tablas para el
tabernáculo, para el lado sur, hacia el sur. \bibleverse{19} Harás
cuarenta basas de plata debajo de las veinte tablas; dos basas debajo de
una tabla para sus dos espigas, y dos basas debajo de otra tabla para
sus dos espigas. \bibleverse{20} Para el segundo lado del tabernáculo,
en el lado norte, veinte tablas, \bibleverse{21} y sus cuarenta basas de
plata; dos basas debajo de una tabla, y dos basas debajo de otra tabla.
\bibleverse{22} Para el lado opuesto del tabernáculo, hacia el oeste,
harás seis tablas. \bibleverse{23} Harás dos tablas para las esquinas
del tabernáculo en el lado opuesto. \bibleverse{24} Serán dobles por
debajo, y de la misma manera serán enteras hasta su parte superior a un
anillo; así será para ambas; serán para las dos esquinas.
\bibleverse{25} Habrá ocho tablas, y sus basas de plata, dieciséis
basas; dos basas debajo de una tabla, y dos basas debajo de otra tabla.

\bibleverse{26} ``Harás barras de madera de acacia: cinco para las
tablas de un lado del tabernáculo, \bibleverse{27} y cinco barras para
las tablas del otro lado del tabernáculo, y cinco barras para las tablas
del lado del tabernáculo, para el lado opuesto, hacia el oeste.
\bibleverse{28} La barra del medio de las tablas pasará de extremo a
extremo. \bibleverse{29} Recubrirás las tablas con oro, y harás sus
anillos de oro para colocar las barras. Recubrirás de oro las barras.
\bibleverse{30} Montarás el tabernáculo de acuerdo con la forma en que
se te mostró en la montaña. \footnote{\textbf{26:30} Éxod 25,9}

\hypertarget{las-dos-cortinas-y-el-interior-del-apartamento}{%
\subsection{Las dos cortinas y el interior del
apartamento}\label{las-dos-cortinas-y-el-interior-del-apartamento}}

\bibleverse{31} ``Harás un velo de azul, púrpura, escarlata y lino fino,
con querubines. Será obra de un hábil obrero. \footnote{\textbf{26:31}
  Mat 27,51} \bibleverse{32} Lo colgarás en cuatro columnas de acacia
recubiertas de oro; sus ganchos serán de oro, sobre cuatro bases de
plata. \bibleverse{33} Colgarás el velo debajo de los corchetes, y
meterás allí el arca de la alianza dentro del velo. El velo separará
para ti el lugar santo del santísimo. \footnote{\textbf{26:33} Éxod
  26,6; Éxod 26,11; Heb 9,3-12} \bibleverse{34} Pondrás el propiciatorio
sobre el arca de la alianza en el lugar santísimo. \footnote{\textbf{26:34}
  Éxod 25,21} \bibleverse{35} Pondrás la mesa fuera del velo, y el
candelabro frente a la mesa, del lado del tabernáculo hacia el sur.
Pondrás la mesa en el lado norte. \footnote{\textbf{26:35} Éxod 40,22}

\bibleverse{36} ``Harás un biombo para la puerta de la Tienda, de azul,
púrpura, escarlata y lino torcido, obra del bordador. \bibleverse{37}
Harás para el biombo cinco columnas de acacia, y las recubrirás de oro.
Sus ganchos serán de oro. Fundirás para ellas cinco bases de bronce.

\hypertarget{instrucciones-sobre-el-altar-de-los-holocaustos-la-explanada-y-la-entrega-de-aceite-para-el-candelero}{%
\subsection{Instrucciones sobre el altar de los holocaustos, la
explanada y la entrega de aceite para el
candelero}\label{instrucciones-sobre-el-altar-de-los-holocaustos-la-explanada-y-la-entrega-de-aceite-para-el-candelero}}

\hypertarget{section-26}{%
\section{27}\label{section-26}}

\bibleverse{1} ``Harás el altar de madera de acacia, de cinco
codos\footnote{\textbf{27:1} Un codo es la longitud desde la punta del
  dedo corazón hasta el codo del brazo de un hombre, es decir, unas 18
  pulgadas o 46 centímetros.} de largo y cinco codos de ancho. El altar
será cuadrado. Su altura será de tres codos. \footnote{\textbf{27:1} El
  altar debía tener unos 2,3×2,3×1,4 metros o unos 7½×7½×4½ pies.}
\bibleverse{2} Harás sus cuernos en sus cuatro esquinas. Sus cuernos
serán de una sola pieza con él. Lo recubrirás de bronce. \bibleverse{3}
Harás sus ollas para recoger sus cenizas, sus palas, sus cuencos, sus
ganchos para la carne y sus sartenes para el fuego. Harás todos sus
recipientes de bronce. \bibleverse{4} Le harás una rejilla de red de
bronce. En la red harás cuatro anillos de bronce en sus cuatro esquinas.
\bibleverse{5} La pondrás debajo de la cornisa que rodea el altar, para
que la red llegue hasta la mitad del altar. \bibleverse{6} Harás varas
para el altar, varas de madera de acacia, y las recubrirás de bronce.
\bibleverse{7} Sus varas se pondrán en los anillos, y las varas estarán
a los dos lados del altar cuando lo lleves. \bibleverse{8} Lo harás
hueco con tablas. Lo harán como se te ha mostrado en la montaña.
\footnote{\textbf{27:8} Éxod 26,30}

\bibleverse{9} ``Harás el atrio del tabernáculo: para el lado sur, hacia
el sur, habrá cortinas para el atrio de lino fino de cien codos de largo
por un lado. \bibleverse{10} Sus columnas serán veinte, y sus bases
veinte, de bronce. Los ganchos de las columnas y sus filetes serán de
plata. \bibleverse{11} Asimismo, para la longitud del lado norte, habrá
cortinas de cien codos, y sus columnas serán veinte, y sus bases veinte,
de bronce; los ganchos de las columnas y sus filetes, de plata.
\bibleverse{12} La anchura del atrio del lado occidental tendrá cortinas
de cincuenta codos; sus columnas, diez, y sus basas, diez.
\bibleverse{13} La anchura del atrio del lado oriental será de cincuenta
codos. \bibleverse{14} Las cortinas de un lado de la puerta serán de
quince codos, sus columnas de tres y sus bases de tres. \bibleverse{15}
Las cortinas del otro lado serán de quince codos; sus columnas, tres, y
sus bases, tres. \bibleverse{16} Para la puerta del atrio habrá una
cortina de veinte codos, de azul, púrpura, carmesí y lino torcido, obra
del bordador; sus columnas cuatro, y sus bases cuatro. \bibleverse{17}
Todas las columnas del atrio alrededor estarán forradas de plata; sus
ganchos, de plata, y sus bases, de bronce. \bibleverse{18} La longitud
del atrio será de cien codos, la anchura de cincuenta y la altura de
cinco codos, de lino fino, y sus bases de bronce. \bibleverse{19} Todos
los instrumentos del tabernáculo en todo su servicio, y todos sus
pasadores, y todos los pasadores del atrio, serán de bronce.

\bibleverse{20} ``Mandarás a los hijos de Israel que te traigan aceite
de oliva puro batido para la luz, para hacer arder continuamente una
lámpara. \footnote{\textbf{27:20} Lev 24,2} \bibleverse{21} En la Tienda
del Encuentro, fuera del velo que está delante del pacto, Aarón y sus
hijos la mantendrán en orden desde la tarde hasta la mañana delante de
Yahvé; será un estatuto para siempre a través de sus generaciones a
favor de los hijos de Israel.

\hypertarget{instrucciones-sobre-la-vestimenta-sacerdotal-de-aaruxf3n-y-sus-hijos}{%
\subsection{Instrucciones sobre la vestimenta sacerdotal de Aarón y sus
hijos}\label{instrucciones-sobre-la-vestimenta-sacerdotal-de-aaruxf3n-y-sus-hijos}}

\hypertarget{section-27}{%
\section{28}\label{section-27}}

\bibleverse{1} ``Trae a Aarón, tu hermano, y a sus hijos con él, cerca
de ti, de entre los hijos de Israel, para que me sirva en el oficio de
sacerdote: Aarón, con Nadab, Abiú, Eleazar e Itamar, hijos de Aarón.
\footnote{\textbf{28:1} 1Cró 23,13; Éxod 6,23} \bibleverse{2} Harás
vestiduras sagradas para Aarón, tu hermano, para gloria y belleza.
\bibleverse{3} Hablarás a todos los sabios de corazón, a quienes he
llenado de espíritu de sabiduría, para que hagan las vestiduras de Aarón
para santificarlo, a fin de que me sirva en el oficio de sacerdote.
\footnote{\textbf{28:3} Éxod 31,3} \bibleverse{4} Estas son las
vestimentas que harán: un pectoral, un efod, un manto, una túnica
ajustada, un turbante y un fajín. Harán las vestiduras sagradas para
Aarón tu hermano y sus hijos, para que me sirvan en el oficio
sacerdotal. \bibleverse{5} Usarán el oro, el azul, la púrpura, la
escarlata y el lino fino.

\hypertarget{el-vestido-de-hombro-ephod}{%
\subsection{El vestido de hombro
(ephod)}\label{el-vestido-de-hombro-ephod}}

\bibleverse{6} ``Harán el efod de oro, azul, púrpura, escarlata y lino
torcido, obra del obrero hábil. \bibleverse{7} Tendrá dos correas para
los hombros, unidas a los dos extremos del mismo, para que se pueda
unir. \bibleverse{8} La banda tejida con destreza, que está sobre él,
será como su obra y de la misma pieza; de oro, azul, púrpura, escarlata
y lino fino torcido. \bibleverse{9} Tomarás dos piedras de ónice y
grabarás en ellas los nombres de los hijos de Israel. \bibleverse{10}
Seis de sus nombres en una piedra, y los nombres de los seis que quedan
en la otra piedra, en el orden de su nacimiento. \bibleverse{11} Con el
trabajo de un grabador en piedra, como los grabados de un sello,
grabarás las dos piedras, según los nombres de los hijos de Israel. Las
harás encerrar en engastes de oro. \bibleverse{12} Pondrás las dos
piedras en los tirantes del efod, para que sean piedras conmemorativas
de los hijos de Israel. Aarón llevará sus nombres ante el Señor en sus
dos hombros como recuerdo. \bibleverse{13} Harás engastes de oro,
\bibleverse{14} y dos cadenas de oro puro; las harás como cordones
trenzados. Pondrás las cadenas trenzadas en los engastes.

\hypertarget{el-peto-con-accesorios}{%
\subsection{El peto con accesorios}\label{el-peto-con-accesorios}}

\bibleverse{15} ``Harás un pectoral de juicio, obra de obrero experto;
como la obra del efod lo harás; de oro, de azul, de púrpura, de carmesí
y de lino torcido, lo harás. \bibleverse{16} Será cuadrado y con doble
pliegue; un palmo\footnote{\textbf{28:16} Un palmo es la longitud desde
  la punta del pulgar de un hombre hasta la punta de su dedo meñique
  cuando su mano está extendida (aproximadamente medio codo, o 9
  pulgadas, o 22,8 cm.)} será su longitud, y un palmo su anchura.
\bibleverse{17} Pondrás en él engastes de piedras, cuatro hileras de
piedras: una hilera de rubíes, topacios y berilos será la primera
hilera; \bibleverse{18} y la segunda hilera una turquesa, un
zafiro,\footnote{\textbf{28:18} o, lapislázuli} y una esmeralda;
\bibleverse{19} y la tercera hilera un jacinto, un ágata y una amatista;
\bibleverse{20} y la cuarta hilera un crisolito, un ónice y un jaspe.
Estarán encerrados en oro en sus engastes. \bibleverse{21} Las piedras
serán según los nombres de los hijos de Israel, doce, según sus nombres;
como los grabados de un sello, cada uno según su nombre, serán para las
doce tribus. \bibleverse{22} Harás en el pectoral cadenas como cordones,
de oro puro trenzado. \bibleverse{23} Harás en el pectoral dos anillos
de oro, y pondrás los dos anillos en los dos extremos del pectoral.
\bibleverse{24} Pondrás las dos cadenas trenzadas de oro en los dos
anillos de los extremos del pectoral. \bibleverse{25} Los otros dos
extremos de las dos cadenas trenzadas los pondrás en los dos engastes, y
los pondrás en los tirantes del efod en su parte delantera.
\bibleverse{26} Harás dos anillos de oro y los pondrás en los dos
extremos del pectoral, en su borde, que está hacia el lado del efod,
hacia adentro. \bibleverse{27} Harás dos anillos de oro y los pondrás en
los dos tirantes del efod por debajo, en su parte delantera, cerca de su
acoplamiento, por encima de la banda hábilmente tejida del efod.
\bibleverse{28} El pectoral lo unirán por sus anillos a los anillos del
efod con un cordón de color azul, para que quede sobre la banda
hábilmente tejida del efod, y para que el pectoral no se salga del efod.
\bibleverse{29} Aarón llevará los nombres de los hijos de Israel en el
pectoral del juicio sobre su corazón, cuando entre en el lugar santo,
como recuerdo ante Yahvé siempre. \bibleverse{30} En el pectoral del
juicio pondrás el Urim y el Tumim, y estarán en el corazón de Aarón
cuando entre delante de Yavé. Aarón llevará el juicio de los hijos de
Israel en su corazón ante el Señor continuamente. \footnote{\textbf{28:30}
  Lev 8,8; Núm 27,21; Deut 33,8}

\hypertarget{la-prenda-superior-para-el-vestido-de-hombros}{%
\subsection{La prenda superior para el vestido de
hombros}\label{la-prenda-superior-para-el-vestido-de-hombros}}

\bibleverse{31} ``Harás el manto del efod todo de color azul.
\bibleverse{32} Tendrá un orificio para la cabeza en el centro. Tendrá
un cordón de tejido alrededor de su orificio, como el orificio de una
cota de malla, para que no se rompa. \bibleverse{33} En su dobladillo
harás granadas de azul, de púrpura y de escarlata, alrededor de su
dobladillo; con campanillas de oro entre ellas y alrededor de ellas:
\bibleverse{34} una campanilla de oro y una granada, una campanilla de
oro y una granada, alrededor del dobladillo del manto. \bibleverse{35}
Estará sobre Aarón para ministrar; y su sonido se oirá cuando entre al
lugar santo delante de Yahvé, y cuando salga, para que no muera.
\footnote{\textbf{28:35} Éxod 30,21; Lev 16,2; Lev 16,13}

\hypertarget{frente-ropa-interior-diadema-y-cinturuxf3n}{%
\subsection{Frente, ropa interior, diadema y
cinturón}\label{frente-ropa-interior-diadema-y-cinturuxf3n}}

\bibleverse{36} ``Harás una placa de oro puro y grabarás en ella, como
los grabados de un sello, `SANTO A YAHWEH'. \bibleverse{37} La pondrás
sobre un cordón de color azul, y estará en el fajín. Estará en la parte
delantera del fajín. \bibleverse{38} Estará en la frente de Aarón, y
Aarón llevará la iniquidad de las cosas sagradas que los hijos de Israel
santifican en todos sus dones sagrados; y estará siempre en su frente,
para que sean aceptados ante Yahvé. \bibleverse{39} Tejerás la túnica
con lino fino. Harás un turbante de lino fino. Harás un fajín, obra del
bordador.

\hypertarget{la-ropa-de-los-hijos-de-aaruxf3n}{%
\subsection{La ropa de los hijos de
Aarón}\label{la-ropa-de-los-hijos-de-aaruxf3n}}

\bibleverse{40} ``Harás túnicas para los hijos de Aarón. Les harás
fajas. Les harás cintillos, para gloria y belleza. \bibleverse{41} Se
las pondrás a Aarón, tu hermano, y a sus hijos con él, y los ungirás,
los consagrarás y los santificarás, para que me sirvan en el oficio de
sacerdote. \footnote{\textbf{28:41} Lev 8,12; Éxod 29,9; Éxod 29,24}
\bibleverse{42} Les harás pantalones de lino para cubrir su carne
desnuda. Llegarán desde la cintura hasta los muslos. \bibleverse{43}
Estarán sobre Aarón y sobre sus hijos, cuando entren en la Tienda de
Reunión, o cuando se acerquen al altar para ministrar en el lugar santo,
para que no lleven iniquidad y mueran. Esto será un estatuto para
siempre para él y para su descendencia después de él.

\hypertarget{instrucciuxf3n-para-la-ordenaciuxf3n-de-sacerdotes}{%
\subsection{Instrucción para la ordenación de
sacerdotes}\label{instrucciuxf3n-para-la-ordenaciuxf3n-de-sacerdotes}}

\hypertarget{section-28}{%
\section{29}\label{section-28}}

\bibleverse{1} ``Esto es lo que les harás para santificarlos, para que
me sirvan en el oficio sacerdotal: toma un novillo y dos carneros sin
defecto, \footnote{\textbf{29:1} Lev 8,1-32} \bibleverse{2} panes sin
levadura, tortas sin levadura mezcladas con aceite y obleas sin levadura
untadas con aceite. Las harás de harina de trigo fina. \bibleverse{3}
Los pondrás en un canasto y los traerás en el canasto, con el toro y los
dos carneros. \bibleverse{4} Llevarás a Aarón y a sus hijos a la puerta
de la Tienda del Encuentro, y los lavarás con agua. \bibleverse{5}
Tomarás las vestimentas y le pondrás a Aarón la túnica, el manto del
efod, el efod y el pectoral, y lo vestirás con la banda hábilmente
tejida del efod. \bibleverse{6} Pondrás el turbante sobre su cabeza y
pondrás la corona sagrada sobre el turbante. \footnote{\textbf{29:6}
  Éxod 28,36; Éxod 39,30} \bibleverse{7} Luego tomarás el aceite de la
unción, lo derramarás sobre su cabeza y lo ungirás. \footnote{\textbf{29:7}
  Éxod 30,25} \bibleverse{8} Traerás a sus hijos y les pondrás túnicas.
\bibleverse{9} Los vestirás con cinturones, a Aarón y a sus hijos, y les
atarás cintillos. Ellos tendrán el sacerdocio por estatuto perpetuo.
Consagrarás a Aarón y a sus hijos. \footnote{\textbf{29:9} Éxod 28,41}

\bibleverse{10} ``Llevarás el toro ante la Tienda del Encuentro, y Aarón
y sus hijos pondrán sus manos sobre la cabeza del toro. \bibleverse{11}
Matarás el toro ante el Señor, a la puerta de la Tienda del Encuentro.
\bibleverse{12} Tomarás de la sangre del toro y la pondrás con tu dedo
sobre los cuernos del altar, y derramarás toda la sangre al pie del
altar. \bibleverse{13} Tomarás toda la grasa que cubre las vísceras, la
cubierta del hígado, los dos riñones y la grasa que hay sobre ellos, y
los quemarás sobre el altar. \footnote{\textbf{29:13} Éxod 29,22}
\bibleverse{14} Pero la carne del toro, su piel y su estiércol los
quemarás al fuego fuera del campamento. Es una ofrenda por el pecado.
\footnote{\textbf{29:14} Lev 4,11-12}

\bibleverse{15} ``También tomarás el único carnero, y Aarón y sus hijos
pondrán sus manos sobre la cabeza del carnero. \bibleverse{16} Matarás
el carnero, tomarás su sangre y la rociarás alrededor del altar.
\bibleverse{17} Cortarás el carnero en pedazos, y lavarás sus entrañas y
sus patas, y las pondrás con sus pedazos y con su cabeza.
\bibleverse{18} Quemarás todo el carnero sobre el altar: es un
holocausto para Yahvé; es un aroma agradable, una ofrenda hecha por
fuego para Yahvé.

\bibleverse{19} ``Tomarás el otro carnero, y Aarón y sus hijos pondrán
sus manos sobre la cabeza del carnero. \bibleverse{20} Luego matarás el
carnero, tomarás un poco de su sangre y la pondrás en el lóbulo de la
oreja derecha de Aarón y en el lóbulo de la oreja derecha de sus hijos,
en el pulgar de su mano derecha y en el dedo gordo de su pie derecho, y
rociarás la sangre alrededor del altar. \bibleverse{21} Tomarás de la
sangre que está sobre el altar, y del aceite de la unción, y la rociarás
sobre Aarón, y sobre sus vestiduras, y sobre sus hijos, y sobre las
vestiduras de sus hijos con él; y él será santificado, y sus vestiduras,
y sus hijos, y las vestiduras de sus hijos con él. \bibleverse{22}
También tomarás parte de la grasa del carnero, la cola gorda, la grasa
que cubre las vísceras, la cubierta del hígado, los dos riñones, la
grasa que hay en ellos y el muslo derecho (porque es un carnero de
consagración), \footnote{\textbf{29:22} Lev 3,3-4} \bibleverse{23} y una
hogaza de pan, una torta de pan engrasado y una oblea del canasto de los
panes sin levadura que están delante de Yahvé. \bibleverse{24} Pondrás
todo esto en las manos de Aarón y en las manos de sus hijos, y los
agitarás como ofrenda mecida ante Yavé. \bibleverse{25} Los tomarás de
sus manos y los harás arder en el altar, sobre el holocausto, como aroma
agradable ante Yavé; es una ofrenda encendida para Yavé.

\bibleverse{26} ``Tomarás el pecho del carnero de las consagraciones de
Aarón y lo mecerás como ofrenda mecida ante Yahvé. Será tu porción.
\bibleverse{27} Santificarás el pecho de la ofrenda mecida y el muslo de
la ofrenda mecida, que se eleva, del carnero de las consagraciones, del
que es para Aarón y del que es para sus hijos. \footnote{\textbf{29:27}
  Núm 18,18} \bibleverse{28} Será para Aarón y sus hijos como su porción
para siempre de los hijos de Israel; porque es una ofrenda mecida. Será
una ofrenda mecida de los hijos de Israel de los sacrificios de sus
ofrendas de paz, su ofrenda mecida a Yahvé.

\bibleverse{29} ``Las vestiduras sagradas de Aarón serán para sus hijos
después de él, para ser ungidos con ellas y para ser consagrados con
ellas. \footnote{\textbf{29:29} Éxod 29,9} \bibleverse{30} Siete días se
las pondrá el hijo que sea sacerdote en su lugar, cuando entre en la
Tienda de Reunión para ministrar en el lugar santo.

\bibleverse{31} ``Tomarás el carnero de las consagraciones y cocerás su
carne en un lugar sagrado. \bibleverse{32} Aarón y sus hijos comerán la
carne del carnero y el pan que esté en el canasto, a la puerta de la
Tienda de Reunión. \bibleverse{33} Comerán esas cosas con las que se
hizo expiación, para consagrarlas y santificarlas; pero un extraño no
comerá de ellas, porque son sagradas. \bibleverse{34} Si algo de la
carne de la consagración, o del pan, queda hasta la mañana, entonces
quemarás el resto con fuego. No se comerá, porque es sagrado.

\bibleverse{35} ``Así harás con Aarón y con sus hijos, según todo lo que
te he mandado. Los consagrarás durante siete días.

\hypertarget{la-santificaciuxf3n-y-unciuxf3n-del-altar-del-holocausto}{%
\subsection{La santificación y unción del altar del
holocausto}\label{la-santificaciuxf3n-y-unciuxf3n-del-altar-del-holocausto}}

\bibleverse{36} Cada día ofrecerás el toro de la ofrenda por el pecado
para la expiación. Limpiarás el altar cuando hagas la expiación por él.
Lo ungirás para santificarlo. \bibleverse{37} Siete días expiarás el
altar y lo santificarás, y el altar será santísimo. Todo lo que toque el
altar será santo.

\hypertarget{la-ofrenda-diaria-de-quema-bebida-y-comida-por-la-mauxf1ana-y-por-la-noche}{%
\subsection{La ofrenda diaria de quema, bebida y comida por la mañana y
por la
noche}\label{la-ofrenda-diaria-de-quema-bebida-y-comida-por-la-mauxf1ana-y-por-la-noche}}

\bibleverse{38} ``Esto es lo que ofrecerás sobre el altar: dos corderos
de un año, de día en día, continuamente. \bibleverse{39} El primer
cordero lo ofrecerás por la mañana, y el otro cordero lo ofrecerás al
atardecer; \footnote{\textbf{29:39} Sal 141,2} \bibleverse{40} y con el
primer cordero la décima parte de un efa\footnote{\textbf{29:40} Un codo
  es la longitud desde la punta del dedo corazón hasta el codo del brazo
  de un hombre, es decir, unas 18 pulgadas o 46 centímetros.} de harina
fina mezclada con la cuarta parte de un hin de aceite batido, y la
cuarta parte de un hin de vino como libación. \bibleverse{41} El otro
cordero lo ofrecerás al atardecer, y harás con él lo mismo que con la
ofrenda de la mañana y con su libación, como aroma agradable, ofrenda
encendida a Yahvé. \bibleverse{42} Será un holocausto continuo a lo
largo de vuestras generaciones, a la puerta de la Tienda del Encuentro,
delante de Yahvé, donde me reuniré con vosotros para hablaros allí.
\bibleverse{43} Allí me reuniré con los hijos de Israel, y el lugar será
santificado por mi gloria. \footnote{\textbf{29:43} Éxod 20,24}
\bibleverse{44} Santificaré la Carpa del Encuentro y el altar. También
santificaré a Aarón y a sus hijos para que me sirvan en el oficio de
sacerdote. \bibleverse{45} Habitaré entre los hijos de Israel y seré su
Dios. \bibleverse{46} Sabrán que yo soy el Señor, su Dios, que los sacó
de la tierra de Egipto para que yo habitara en medio de ellos: Yo soy el
Señor, su Dios.

\hypertarget{regulaciones-sobre-el-altar-humeante}{%
\subsection{Regulaciones sobre el altar
humeante}\label{regulaciones-sobre-el-altar-humeante}}

\hypertarget{section-29}{%
\section{30}\label{section-29}}

\bibleverse{1} ``Harás un altar para quemar incienso. Lo harás de madera
de acacia. \footnote{\textbf{30:1} Éxod 37,25-28} \bibleverse{2} Su
longitud será de un codo, y su anchura de un codo. Será cuadrado, y su
altura será de dos codos. Sus cuernos serán de una sola pieza con él.
\bibleverse{3} Lo recubrirás de oro puro, su parte superior, sus lados
alrededor y sus cuernos; y harás una moldura de oro alrededor.
\bibleverse{4} Le harás dos anillos de oro debajo de su moldura; en sus
dos costillas, en sus dos lados los harás; y servirán de lugares para
las varas con las que se soportará. \bibleverse{5} Harás las varas de
madera de acacia y las recubrirás de oro. \bibleverse{6} Lo pondrás
delante del velo que está junto al arca de la alianza, delante del
propiciatorio que está sobre la alianza, donde me reuniré contigo.
\footnote{\textbf{30:6} Éxod 25,22} \bibleverse{7} Aarón quemará sobre
él incienso de especias dulces cada mañana. Cuando atienda las lámparas,
lo quemará. \footnote{\textbf{30:7} Sal 141,2; Apoc 5,8} \bibleverse{8}
Cuando Aarón encienda las lámparas al atardecer, lo quemará, un incienso
perpetuo ante el Señor por vuestras generaciones. \bibleverse{9} No
ofrecerás sobre él ningún incienso extraño, ni holocausto, ni ofrenda; y
no derramarás sobre él ninguna libación. \footnote{\textbf{30:9} Lev
  10,1} \bibleverse{10} Aarón hará expiación sobre sus cuernos una vez
al año; con la sangre del sacrificio por el pecado de la expiación, una
vez al año, hará expiación por él a lo largo de vuestras generaciones.
Es muy sagrado para Yahvé''. \footnote{\textbf{30:10} Lev 16,18; Éxod
  29,37}

\hypertarget{regulaciones-relativas-a-la-recaudaciuxf3n-de-un-impuesto-de-capitaciuxf3n-en-el-santuario-en-la-reuniuxf3n-del-pueblo}{%
\subsection{Regulaciones relativas a la recaudación de un impuesto de
capitación en el santuario en la reunión del
pueblo}\label{regulaciones-relativas-a-la-recaudaciuxf3n-de-un-impuesto-de-capitaciuxf3n-en-el-santuario-en-la-reuniuxf3n-del-pueblo}}

\bibleverse{11} Yahvé habló a Moisés diciendo: \bibleverse{12} ``Cuando
hagas el censo de los hijos de Israel, según los que se cuenten entre
ellos, cada uno dará un rescate por su alma a Yahvé cuando los cuentes,
para que no haya plaga entre ellos cuando los cuentes. \bibleverse{13}
Todo el que pase a los contados dará medio siclo según el
siclo\footnote{\textbf{30:13} Un siclo equivale a unos 10 gramos o a
  unas 0,35 onzas.} del santuario (el siclo es de veinte
gerahs\footnote{\textbf{30:13} una gerah son unos 0,5 gramos o unos 7,7
  granos} ); medio siclo como ofrenda a Yavé. \bibleverse{14} Todo el
que pase a los contados, de veinte años para arriba, dará la ofrenda a
Yahvé. \bibleverse{15} El rico no dará más, y el pobre no dará menos,
que el medio siclo,\footnote{\textbf{30:15} Un siclo equivale a unos 10
  gramos o a unas 0,35 onzas.} cuando den la ofrenda a Yahvé, para hacer
expiación por vuestras almas. \bibleverse{16} Tomarás el dinero de la
expiación de los hijos de Israel y lo destinarás al servicio de la
Tienda de Reunión, para que sea un memorial de los hijos de Israel ante
Yahvé, para hacer expiación por vuestras almas.''

\hypertarget{normativa-sobre-el-fregadero-de-cobre-para-los-sacerdotes}{%
\subsection{Normativa sobre el fregadero de cobre para los
sacerdotes}\label{normativa-sobre-el-fregadero-de-cobre-para-los-sacerdotes}}

\bibleverse{17} Yahvé habló a Moisés diciendo: \bibleverse{18} ``Harás
también una pila de bronce, con su base de bronce, en la que se lavará.
La pondrás entre la Tienda de Reunión y el altar, y pondrás agua en
ella. \footnote{\textbf{30:18} Éxod 38,8} \bibleverse{19} Aarón y sus
hijos se lavarán las manos y los pies en ella. \bibleverse{20} Cuando
entren en la Tienda del Encuentro, se lavarán con agua, para no morir; o
cuando se acerquen al altar para ministrar, para quemar una ofrenda
encendida a Yahvé. \bibleverse{21} Así se lavarán las manos y los pies
para no morir. Esto les servirá de estatuto para siempre, a él y a sus
descendientes por sus generaciones.''

\hypertarget{preparaciuxf3n-y-uso-del-aceite-de-la-unciuxf3n-sagrada}{%
\subsection{Preparación y uso del aceite de la unción
sagrada}\label{preparaciuxf3n-y-uso-del-aceite-de-la-unciuxf3n-sagrada}}

\bibleverse{22} Además, Yahvé habló a Moisés, diciendo: \bibleverse{23}
``Toma también especias finas: de mirra líquida, quinientos
siclos;\footnote{\textbf{30:23} Un siclo equivale a unos 10 gramos o a
  unas 0,35 onzas, por lo que 500 siclos equivalen a unos 5 kilogramos o
  a unas 11 libras.} y de canela aromática la mitad, doscientos
cincuenta; y de caña aromática, doscientos cincuenta; \bibleverse{24} y
de casia quinientos, según el siclo del santuario; y un hin\footnote{\textbf{30:24}
  Un codo es la longitud desde la punta del dedo corazón hasta el codo
  del brazo de un hombre, es decir, unas 18 pulgadas o 46 centímetros.}
de aceite de oliva. \bibleverse{25} Lo convertirás en un aceite santo
para la unción, un perfume compuesto según el arte del perfumista; será
un aceite santo para la unción. \footnote{\textbf{30:25} Éxod 37,29}
\bibleverse{26} Lo usarás para ungir la Tienda de reunión, el arca de la
alianza, \bibleverse{27} la mesa y todos sus artículos, el candelabro y
sus accesorios, el altar del incienso, \bibleverse{28} el altar del
holocausto con todos sus utensilios, y la pila con su base.
\bibleverse{29} Los santificarás para que sean santos. Todo lo que los
toque será santo. \footnote{\textbf{30:29} Éxod 30,10} \bibleverse{30}
Ungirás a Aarón y a sus hijos, y los santificarás para que me sirvan en
el oficio de sacerdote. \footnote{\textbf{30:30} Éxod 29,7}
\bibleverse{31} Hablarás a los hijos de Israel diciendo: ``Este será un
aceite de unción santo para mí a través de vuestras generaciones.
\bibleverse{32} No se derramará sobre la carne del hombre, y no hagas
nada semejante a él, según su composición. Es santo. Será santo para
vosotros. \bibleverse{33} El que componga algo semejante, o el que ponga
algo de él sobre un extraño, será cortado de su pueblo''.

\hypertarget{preparaciuxf3n-y-uso-del-incienso-sagrado}{%
\subsection{Preparación y uso del incienso
sagrado}\label{preparaciuxf3n-y-uso-del-incienso-sagrado}}

\bibleverse{34} Yahvé dijo a Moisés: ``Toma para ti especias dulces,
resina de goma, onycha y gálbano: especias dulces con incienso puro.
Habrá un peso igual de cada una. \bibleverse{35} Harás con ello
incienso, un perfume según el arte del perfumista, sazonado con sal,
puro y santo. \footnote{\textbf{30:35} Éxod 37,29} \bibleverse{36}
Machacaréis una parte muy pequeña y pondréis otra delante del pacto en
la Tienda de reunión, donde me reuniré con vosotros. Será para ti algo
muy sagrado. \footnote{\textbf{30:36} Éxod 30,6} \bibleverse{37} No
haréis este incienso, según su composición, para vosotros; será para
vosotros santo para Yahvé. \bibleverse{38} El que haga algo semejante,
para olerlo, será cortado de su pueblo.''

\hypertarget{nombramiento-de-dos-capataces-y-sus-ayudantes}{%
\subsection{Nombramiento de dos capataces y sus
ayudantes}\label{nombramiento-de-dos-capataces-y-sus-ayudantes}}

\hypertarget{section-30}{%
\section{31}\label{section-30}}

\bibleverse{1} Yahvé habló a Moisés, diciendo: \footnote{\textbf{31:1}
  Éxod 35,30-35} \bibleverse{2} ``He aquí que he llamado por nombre a
Bezalel, hijo de Uri, hijo de Hur, de la tribu de Judá. \bibleverse{3}
Lo he llenado con el Espíritu de Dios, en sabiduría, en inteligencia y
en conocimiento, y en toda clase de trabajos, \footnote{\textbf{31:3}
  1Re 7,14} \bibleverse{4} para idear obras de arte, para trabajar en
oro, en plata y en bronce, \bibleverse{5} y en el corte de piedras para
engastar, y en la talla de madera, para trabajar en toda clase de
trabajos. \bibleverse{6} He aquí que yo mismo he puesto con él a
Oholiab, hijo de Ahisamac, de la tribu de Dan; y en el corazón de todos
los sabios de corazón he puesto la sabiduría, para que hagan todo lo que
os he mandado: \bibleverse{7} la Tienda de reunión, el arca de la
alianza, el propiciatorio que está sobre ella, todo el mobiliario de la
Tienda, \footnote{\textbf{31:7} Éxod 35,11-19} \bibleverse{8} la mesa y
sus recipientes, el candelabro puro con todos sus recipientes, el altar
del incienso, \bibleverse{9} el altar del holocausto con todos sus
recipientes, la pila y su base, \bibleverse{10} las vestiduras finamente
trabajadas --- las vestiduras sagradas para el sacerdote Aarón, las
vestiduras de sus hijos para servir en el oficio sacerdotal ---
\bibleverse{11} el aceite de la unción, y el incienso de especias dulces
para el lugar santo: conforme a todo lo que te he mandado, lo harán.''

\hypertarget{promulgaciuxf3n-del-mandamiento-del-suxe1bado}{%
\subsection{Promulgación del mandamiento del
sábado}\label{promulgaciuxf3n-del-mandamiento-del-suxe1bado}}

\bibleverse{12} Yahvé habló a Moisés, diciendo: \bibleverse{13} ``Habla
también a los hijos de Israel, diciendo: `Ciertamente guardaréis mis
sábados, porque es una señal entre yo y vosotros por vuestras
generaciones, para que sepáis que yo soy Yahvé, que os santifico.
\footnote{\textbf{31:13} Éxod 20,8} \bibleverse{14} Por lo tanto,
guardarán el sábado, porque es sagrado para ustedes. Todo el que lo
profane será condenado a muerte, pues el que haga algún trabajo en él,
esa persona será cortada de entre su pueblo. \footnote{\textbf{31:14}
  Núm 15,32-35} \bibleverse{15} Seis días se trabajará, pero el séptimo
día es un día de descanso solemne, santo para Yahvé. El que haga algún
trabajo en el día de reposo será condenado a muerte. \bibleverse{16} Por
lo tanto, los hijos de Israel guardarán el sábado, para observar el
sábado a través de sus generaciones, como un pacto perpetuo.
\bibleverse{17} Es una señal entre yo y los hijos de Israel para
siempre; porque en seis días Yahvé hizo el cielo y la tierra, y en el
séptimo día descansó y se refrescó''. \footnote{\textbf{31:17} Gén 2,2}

\hypertarget{moisuxe9s-recibe-las-tablas-de-la-ley}{%
\subsection{Moisés recibe las tablas de la
ley}\label{moisuxe9s-recibe-las-tablas-de-la-ley}}

\bibleverse{18} Cuando terminó de hablar con él en el monte Sinaí, le
dio a Moisés las dos tablas de la alianza, tablas de piedra, escritas
con el dedo de Dios. \footnote{\textbf{31:18} Éxod 32,15-16; Éxod 34,28;
  Deut 4,13; Deut 5,19; Deut 9,10; Deut 10,4}

\hypertarget{hacer-y-adorar-la-imagen-dorada-del-toro}{%
\subsection{Hacer y adorar la imagen dorada del
toro}\label{hacer-y-adorar-la-imagen-dorada-del-toro}}

\hypertarget{section-31}{%
\section{32}\label{section-31}}

\bibleverse{1} Cuando el pueblo vio que Moisés se demoraba en bajar del
monte, se reunió con Aarón y le dijo: ``Ven, haznos dioses que vayan
delante de nosotros, porque en cuanto a este Moisés, el hombre que nos
sacó de la tierra de Egipto, no sabemos qué ha sido de él.''

\bibleverse{2} Aarón les dijo: ``Quítense los anillos de oro que están
en las orejas de sus esposas, de sus hijos y de sus hijas, y
tráiganmelos''.

\bibleverse{3} Todo el pueblo se quitó los anillos de oro que tenía en
sus orejas y se los llevó a Aarón. \bibleverse{4} El recibió lo que le
entregaron, lo modeló con un instrumento de grabado y lo convirtió en un
becerro moldeado. Luego le dijeron: ``Estos son tus dioses, Israel, que
te sacaron de la tierra de Egipto''. \footnote{\textbf{32:4} Sal
  106,19-20; 1Re 12,28; Hech 7,41}

\bibleverse{5} Al ver esto, Aarón construyó un altar delante de él; y
Aarón hizo una proclama y dijo: ``Mañana será una fiesta para Yahvé.''

\bibleverse{6} Al día siguiente se levantaron temprano, ofrecieron
holocaustos y trajeron ofrendas de paz; el pueblo se sentó a comer y a
beber, y se levantó a jugar. \footnote{\textbf{32:6} 1Cor 10,7}

\hypertarget{moisuxe9s-informado-por-dios-de-la-apostasuxeda-del-pueblo-desciende-del-monte}{%
\subsection{Moisés, informado por Dios de la apostasía del pueblo,
desciende del
monte}\label{moisuxe9s-informado-por-dios-de-la-apostasuxeda-del-pueblo-desciende-del-monte}}

\bibleverse{7} El Señor le dijo a Moisés: ``Ve, baja, porque tu pueblo,
al que sacaste de la tierra de Egipto, se ha corrompido. \bibleverse{8}
Se han desviado rápidamente del camino que les ordené. Se han hecho un
becerro moldeado, lo han adorado y le han ofrecido sacrificios, y han
dicho: `Estos son tus dioses, Israel, que te hicieron subir de la tierra
de Egipto'.'' \footnote{\textbf{32:8} Éxod 20,4; Éxod 20,23; Éxod 32,4}

\bibleverse{9} Yahvé dijo a Moisés: ``He visto a este pueblo, y he aquí
que es un pueblo de dura cerviz. \bibleverse{10} Ahora, pues, déjame en
paz, para que arda mi ira contra ellos y los consuma; y haré de ti una
gran nación.'' \footnote{\textbf{32:10} Núm 14,11-20}

\bibleverse{11} Moisés suplicó a su Dios y le dijo: ``Señor, ¿por qué
arde tu ira contra tu pueblo, que sacaste de la tierra de Egipto con
gran poder y con mano poderosa? \bibleverse{12} ¿Por qué han de hablar
los egipcios diciendo: `Los sacó para mal, para matarlos en los montes y
consumirlos de la superficie de la tierra'? Vuélvete de tu feroz ira, y
aléjate de este mal contra tu pueblo. \bibleverse{13} Acuérdate de
Abraham, de Isaac y de Israel, tus siervos, a quienes juraste por ti
mismo y les dijiste: `Multiplicaré tu descendencia como las estrellas
del cielo, y toda esta tierra de la que he hablado se la daré a tu
descendencia, y la heredarán para siempre'.'' \footnote{\textbf{32:13}
  Gén 22,16-17; Gén 26,4; Gén 28,14}

\bibleverse{14} Entonces Yahvé se apartó del mal que dijo que haría a su
pueblo.

\bibleverse{15} Moisés se volvió y bajó del monte con las dos tablas de
la alianza en la mano, tablas que estaban escritas por ambos lados.
Estaban escritas de un lado y del otro. \bibleverse{16} Las tablas eran
obra de Dios, y la escritura era la escritura de Dios, grabada en las
tablas. \footnote{\textbf{32:16} Éxod 31,18}

\bibleverse{17} Cuando Josué oyó el ruido del pueblo al gritar, dijo a
Moisés: ``Hay ruido de guerra en el campamento''.

\bibleverse{18} Dijo: ``No es la voz de los que gritan por la victoria.
No es la voz de los que gritan por ser vencidos, sino el ruido de los
que cantan lo que oigo''.

\hypertarget{el-celo-de-moisuxe9s-por-dios-castiga-al-pueblo-a-travuxe9s-de-los-levitas}{%
\subsection{El celo de Moisés por Dios; castiga al pueblo a través de
los
levitas}\label{el-celo-de-moisuxe9s-por-dios-castiga-al-pueblo-a-travuxe9s-de-los-levitas}}

\bibleverse{19} En cuanto se acercó al campamento, vio el becerro y las
danzas. Entonces la ira de Moisés se encendió, y arrojó las tablas de
sus manos, y las rompió debajo de la montaña. \bibleverse{20} Tomó el
becerro que habían hecho, lo quemó con fuego, lo molió hasta hacerlo
polvo y lo esparció sobre el agua, e hizo que los hijos de Israel lo
bebieran.

\bibleverse{21} Moisés le dijo a Aarón: ``¿Qué te ha hecho esta gente
para que les hayas provocado un gran pecado?''

\bibleverse{22} Aarón dijo: ``No dejes que se caliente la ira de mi
señor. Tú conoces al pueblo, que está empeñado en el mal.
\bibleverse{23} Porque me han dicho: `Haznos dioses que vayan delante de
nosotros'. En cuanto a ese Moisés, el hombre que nos sacó de la tierra
de Egipto, no sabemos qué ha sido de él'. \bibleverse{24} Les dije: `El
que tenga oro, que lo saque'. Y me lo dieron; lo eché al fuego, y salió
este becerro''.

\bibleverse{25} Cuando Moisés vio que el pueblo estaba fuera de control,
(pues Aarón los había dejado perder el control, causando la burla de sus
enemigos), \bibleverse{26} entonces Moisés se paró en la puerta del
campamento y dijo: ``¡Quien esté del lado de Yahvé, venga a mí!'' Todos
los hijos de Leví se reunieron con él. \bibleverse{27} Él les dijo:
``Yahvé, el Dios de Israel, dice: `Cada uno ponga su espada en el muslo
y vaya de puerta en puerta por todo el campamento, y cada uno mate a su
hermano, a su compañero y a su vecino'\,''. \bibleverse{28} Los hijos de
Leví hicieron lo que dijo Moisés. Ese día cayeron unos tres mil hombres
del pueblo. \bibleverse{29} Moisés dijo: ``Conságrense hoy a Yavé,
porque cada hombre estaba en contra de su hijo y de su hermano, para que
él les dé hoy una bendición.'' \footnote{\textbf{32:29} Éxod 28,41; Núm
  3,6-10; Deut 33,8-11}

\hypertarget{intercesiuxf3n-de-moisuxe9s-por-el-pueblo-el-respiro-de-dios}{%
\subsection{Intercesión de Moisés por el pueblo; El respiro de
dios}\label{intercesiuxf3n-de-moisuxe9s-por-el-pueblo-el-respiro-de-dios}}

\bibleverse{30} Al día siguiente, Moisés dijo al pueblo: ``Habéis
cometido un gran pecado. Ahora subiré a Yahvé. Tal vez haga expiación
por su pecado''.

\bibleverse{31} Moisés volvió a Yahvé y dijo: ``Oh, este pueblo ha
cometido un gran pecado y se ha hecho dioses de oro. \bibleverse{32}
Pero ahora, si quieres, perdona su pecado; y si no, por favor, bórrame
de tu libro que has escrito.'' \footnote{\textbf{32:32} Sal 69,28; Dan
  12,1; Luc 10,20; Rom 9,3}

\bibleverse{33} Yahvé dijo a Moisés: ``A quien haya pecado contra mí, lo
borraré de mi libro. \bibleverse{34} Ahora ve, conduce al pueblo al
lugar del que te he hablado. He aquí que mi ángel irá delante de ti. Sin
embargo, el día en que yo castigue, los castigaré por su pecado''.
\footnote{\textbf{32:34} Éxod 33,2; Éxod 33,12; Éxod 33,14}
\bibleverse{35} El Señor golpeó al pueblo por lo que hicieron con el
becerro que hizo Aarón.

\hypertarget{el-mandato-de-dios-de-ir-a-la-tierra-prometida-el-dolor-de-la-gente-por-el-rechazo-de-dios}{%
\subsection{El mandato de Dios de ir a la tierra prometida; El dolor de
la gente por el rechazo de
Dios}\label{el-mandato-de-dios-de-ir-a-la-tierra-prometida-el-dolor-de-la-gente-por-el-rechazo-de-dios}}

\hypertarget{section-32}{%
\section{33}\label{section-32}}

\bibleverse{1} Yahvé habló a Moisés: ``Vete, sube de aquí, tú y el
pueblo que has sacado de la tierra de Egipto, a la tierra que juré a
Abraham, a Isaac y a Jacob, diciendo: `La daré a tu descendencia'.
\footnote{\textbf{33:1} Éxod 32,13; Gén 12,7} \bibleverse{2} Enviaré un
ángel delante de ti, y expulsaré al cananeo, al amorreo, al hitita, al
ferezeo, al heveo y al jebuseo. \footnote{\textbf{33:2} Éxod 32,34}
\bibleverse{3} Vayan a una tierra que fluye leche y miel; pero yo no
subiré en medio de ustedes, porque son un pueblo de cuello duro, no sea
que los consuma en el camino.'' \footnote{\textbf{33:3} Éxod 32,9-10}

\bibleverse{4} Al oír esta mala noticia, el pueblo se puso de luto y
nadie se puso sus joyas.

\bibleverse{5} Yahvé había dicho a Moisés: ``Di a los hijos de Israel:
`Sois un pueblo de cuello duro. Si subiera entre vosotros un momento, os
consumiría. Por lo tanto, quítense ahora sus joyas, para que yo sepa qué
hacer con ustedes'\,''.

\bibleverse{6} Los hijos de Israel se despojaron de sus joyas a partir
del monte Horeb. \footnote{\textbf{33:6} Jon 3,6}

\hypertarget{la-fundaciuxf3n-y-uso-de-la-tienda-de-revelaciuxf3n-frente-al-campamento}{%
\subsection{La fundación y uso de la tienda de revelación frente al
campamento}\label{la-fundaciuxf3n-y-uso-de-la-tienda-de-revelaciuxf3n-frente-al-campamento}}

\bibleverse{7} Moisés acostumbraba a tomar la tienda y a armarla fuera
del campamento, lejos de él, y la llamaba ``Tienda del Encuentro''.
Todos los que buscaban a Yahvé salían a la Tienda del Encuentro, que
estaba fuera del campamento. \footnote{\textbf{33:7} Éxod 29,42}
\bibleverse{8} Cuando Moisés salió a la Tienda, todo el pueblo se
levantó y se puso de pie, cada uno a la puerta de su tienda, y observó a
Moisés hasta que éste entró en la Tienda. \bibleverse{9} Cuando Moisés
entró en la Carpa, la columna de nube descendió, se puso a la puerta de
la Carpa, y Yahvé habló con Moisés. \footnote{\textbf{33:9} Éxod 13,21}
\bibleverse{10} Todo el pueblo vio que la columna de nube estaba a la
puerta de la Carpa, y todo el pueblo se levantó y adoró, cada uno a la
puerta de su carpa. \bibleverse{11} Yahvé habló con Moisés cara a cara,
como un hombre habla con su amigo. Volvió a entrar en el campamento,
pero su siervo Josué, hijo de Nun, un joven, no salió de la Tienda.
\footnote{\textbf{33:11} Éxod 33,20; Núm 12,8; Deut 34,10}

\hypertarget{nuevas-negociaciones-entre-moisuxe9s-y-dios-sobre-la-direcciuxf3n-divina-adicional-del-pueblo}{%
\subsection{Nuevas negociaciones entre Moisés y Dios sobre la dirección
divina adicional del
pueblo}\label{nuevas-negociaciones-entre-moisuxe9s-y-dios-sobre-la-direcciuxf3n-divina-adicional-del-pueblo}}

\bibleverse{12} Moisés dijo a Yahvé: ``He aquí que tú me dices: `Haz
subir a este pueblo', y no me has hecho saber a quién enviarás conmigo.
Sin embargo, has dicho: `Te conozco por tu nombre, y también has hallado
gracia ante mis ojos'. \footnote{\textbf{33:12} Éxod 33,2-5}
\bibleverse{13} Ahora, pues, si he hallado gracia ante tus ojos,
muéstrame ahora tu camino, para que te conozca y pueda hallar gracia
ante tus ojos; y considera que esta nación es tu pueblo.'' \footnote{\textbf{33:13}
  Sal 103,7}

\bibleverse{14} Dijo: ``Mi presencia irá contigo y te daré descanso''.

\bibleverse{15} Moisés le dijo: ``Si tu presencia no va conmigo, no nos
subas de aquí. \bibleverse{16} Pues ¿cómo sabrá la gente que he hallado
gracia ante tus ojos, yo y tu pueblo? ¿No es que tú vas con nosotros,
para que estemos separados, yo y tu pueblo, de todos los pueblos que
están sobre la superficie de la tierra?'' \footnote{\textbf{33:16} Deut
  4,6-8}

\bibleverse{17} Yahvé dijo a Moisés: ``También haré esto que has dicho,
porque has hallado gracia ante mis ojos y te conozco por tu nombre.''
\footnote{\textbf{33:17} Éxod 33,12; 2Tim 2,19}

\hypertarget{dios-le-promete-a-moisuxe9s-que-veruxe1-su-gloria-como-una-muestra-de-gracia}{%
\subsection{Dios le promete a Moisés que verá su gloria como una muestra
de
gracia}\label{dios-le-promete-a-moisuxe9s-que-veruxe1-su-gloria-como-una-muestra-de-gracia}}

\bibleverse{18} Moisés dijo: ``Por favor, muéstrame tu gloria''.

\bibleverse{19} Dijo: ``Haré pasar ante ti toda mi bondad, y proclamaré
el nombre de Yahvé ante ti. Tendré piedad con quien tenga piedad, y
mostraré misericordia con quien tenga piedad''. \footnote{\textbf{33:19}
  Rom 9,15} \bibleverse{20} Dijo: ``No puedes ver mi rostro, porque el
hombre no puede verme y vivir.'' \footnote{\textbf{33:20} Gén 32,30; Is
  6,5; 1Tim 6,16} \bibleverse{21} Yahvé también dijo: ``He aquí que hay
un lugar junto a mí, y tú estarás sobre la roca. \footnote{\textbf{33:21}
  1Re 19,8-13} \bibleverse{22} Sucederá que, mientras pasa mi gloria, te
pondré en una hendidura de la roca y te cubriré con mi mano hasta que
haya pasado; \footnote{\textbf{33:22} Éxod 34,5-6; Éxod 24,11}
\bibleverse{23} entonces quitaré mi mano, y verás mi espalda; pero mi
rostro no se verá.''

\hypertarget{por-orden-de-dios-moisuxe9s-sube-al-sinauxed-con-dos-tablas-de-piedra-en-blanco}{%
\subsection{Por orden de Dios, Moisés sube al Sinaí con dos tablas de
piedra en
blanco}\label{por-orden-de-dios-moisuxe9s-sube-al-sinauxed-con-dos-tablas-de-piedra-en-blanco}}

\hypertarget{section-33}{%
\section{34}\label{section-33}}

\bibleverse{1} Yahvé dijo a Moisés: ``Talla dos tablas de piedra como
las primeras. Yo escribiré en las tablas las palabras que estaban en las
primeras tablas, que tú rompiste. \footnote{\textbf{34:1} Éxod 32,19}
\bibleverse{2} Prepárate para la mañana, y sube por la mañana al monte
Sinaí, y preséntate allí ante mí en la cima de la montaña.
\bibleverse{3} Nadie subirá contigo ni se te verá en ningún lugar del
monte. No dejes que los rebaños o las manadas pasten frente a ese
monte''. \footnote{\textbf{34:3} Éxod 19,12-13}

\bibleverse{4} Cinceló dos tablas de piedra como la primera; entonces
Moisés se levantó de madrugada y subió al monte Sinaí, como Yahvé le
había ordenado, y tomó en su mano dos tablas de piedra.

\hypertarget{la-apariciuxf3n-de-dios-y-la-intercesiuxf3n-de-moisuxe9s}{%
\subsection{La aparición de Dios y la intercesión de
Moisés}\label{la-apariciuxf3n-de-dios-y-la-intercesiuxf3n-de-moisuxe9s}}

\bibleverse{5} Yahvé descendió en la nube y se quedó allí con él, y
proclamó el nombre de Yahvé. \footnote{\textbf{34:5} Éxod 33,19}
\bibleverse{6} Yahvé pasó por delante de él y proclamó: ``¡Yahvé! Yahvé,
Dios misericordioso y clemente, lento a la cólera y abundante en
bondades y verdades, \footnote{\textbf{34:6} Núm 14,18; Sal 103,8; 1Jn
  4,16} \bibleverse{7} que guarda la bondad amorosa durante miles de
años, que perdona la iniquidad, la desobediencia y el pecado, y que no
exculpa a los culpables, visitando la iniquidad de los padres en los
hijos y en los hijos de los hijos, en la tercera y en la cuarta
generación.'' \footnote{\textbf{34:7} Éxod 20,5-6}

\bibleverse{8} Moisés se apresuró a inclinar la cabeza hacia la tierra y
adoró. \bibleverse{9} Dijo: ``Si ahora he hallado gracia ante tus ojos,
Señor, por favor, deja que el Señor vaya entre nosotros, aunque éste sea
un pueblo de dura cerviz; perdona nuestra iniquidad y nuestro pecado, y
tómanos como herencia.''

\hypertarget{dios-consiente-la-renovaciuxf3n-de-la-relaciuxf3n-del-pacto-con-advertencias-serias}{%
\subsection{Dios consiente la renovación de la relación del pacto con
advertencias
serias}\label{dios-consiente-la-renovaciuxf3n-de-la-relaciuxf3n-del-pacto-con-advertencias-serias}}

\bibleverse{10} Dijo: ``He aquí que hago un pacto: ante todo tu pueblo
haré maravillas, como no se han hecho en toda la tierra, ni en ninguna
nación; y todos los pueblos en medio de los cuales te encuentras verán
la obra de Yahvé, porque es algo impresionante lo que hago contigo.
\bibleverse{11} Observen lo que hoy les ordeno. He aquí que voy a
expulsar ante vosotros al amorreo, al cananeo, al hitita, al ferezeo, al
heveo y al jebuseo. \bibleverse{12} Tened cuidado, no sea que hagáis un
pacto con los habitantes de la tierra a la que vais, para que no os
sirva de lazo; \footnote{\textbf{34:12} Éxod 23,32-33} \bibleverse{13}
sino que derrumbéis sus altares, y hagáis pedazos sus columnas, y
cortéis sus postes de Asera; \footnote{\textbf{34:13} Éxod 23,24}
\bibleverse{14} porque no adoraréis a ningún otro dios; porque Yahvé,
cuyo nombre es Celoso, es un Dios celoso. \footnote{\textbf{34:14} Éxod
  20,3; Éxod 20,5}

\bibleverse{15} ``No hagas un pacto con los habitantes de la tierra, no
sea que ellos se prostituyan según sus dioses y sacrifiquen a sus
dioses, y uno te llame y comas de su sacrificio; \bibleverse{16} y tomes
de sus hijas a tus hijos, y sus hijas se prostituyan según sus dioses, y
hagas que tus hijos se prostituyan según sus dioses. \footnote{\textbf{34:16}
  Deut 7,3; Jue 3,6; 1Re 11,2}

\hypertarget{las-nuevas-regulaciones-federales-sobre-el-debido-culto-a-dios}{%
\subsection{Las nuevas regulaciones federales sobre el debido culto a
Dios}\label{las-nuevas-regulaciones-federales-sobre-el-debido-culto-a-dios}}

\bibleverse{17} ``No os haréis ídolos de fundición. \footnote{\textbf{34:17}
  Éxod 20,23}

\bibleverse{18} ``Celebrarás la fiesta de los panes sin levadura. Siete
días comeréis panes sin levadura, como os he mandado, en el tiempo
señalado del mes de Abib; porque en el mes de Abib salisteis de Egipto.
\footnote{\textbf{34:18} Éxod 23,14-19}

\bibleverse{19} ``Todo lo que abre el vientre es mío; y todo tu ganado
que sea macho, el primogénito de vaca y de oveja. \bibleverse{20} El
primogénito del asno lo redimirás con un cordero. Si no quieres
redimirlo, le romperás el cuello. Redimirás a todos los primogénitos de
tus hijos. Nadie se presentará ante mí con las manos vacías. \footnote{\textbf{34:20}
  Éxod 13,12-16}

\bibleverse{21} ``Seis días trabajarás, pero el séptimo día descansarás:
en el tiempo de arar y en el de cosechar descansarás.

\bibleverse{22} ``Celebrarás la fiesta de las semanas con las primicias
de la cosecha de trigo, y la fiesta de la cosecha al final del año.
\bibleverse{23} Tres veces al año se presentarán todos vuestros varones
ante el Señor Yahvé, el Dios de Israel. \bibleverse{24} Porque expulsaré
a las naciones delante de ti y ampliaré tus fronteras; nadie deseará tu
tierra cuando subas a presentarte ante Yavé, tu Dios, tres veces al año.

\bibleverse{25} ``No ofrecerás la sangre de mi sacrificio con pan
leudado. El sacrificio de la fiesta de la Pascua no se dejará para la
mañana.

\bibleverse{26} ``Traerás las primicias de los primeros frutos de tu
tierra a la casa de Yahvé, tu Dios. ``No hervirás un cabrito en la leche
de su madre''.

\hypertarget{moisuxe9s-escribe-los-mandamientos-del-pacto-dios-renueva-las-tablas-de-la-ley}{%
\subsection{Moisés escribe los mandamientos del pacto; Dios renueva las
tablas de la
ley}\label{moisuxe9s-escribe-los-mandamientos-del-pacto-dios-renueva-las-tablas-de-la-ley}}

\bibleverse{27} Yahvé dijo a Moisés: ``Escribe estas palabras, porque de
acuerdo con ellas he hecho un pacto contigo y con Israel''. \footnote{\textbf{34:27}
  Éxod 24,4}

\bibleverse{28} Estuvo allí con Yahvé cuarenta días y cuarenta noches;
no comió pan ni bebió agua. Escribió en las tablas las palabras de la
alianza, los diez mandamientos. \footnote{\textbf{34:28} Éxod 24,18; Mat
  4,2; Éxod 31,18}

\hypertarget{el-descenso-de-moisuxe9s-el-brillo-de-la-piel-de-su-rostro}{%
\subsection{El descenso de Moisés; el brillo de la piel de su
rostro}\label{el-descenso-de-moisuxe9s-el-brillo-de-la-piel-de-su-rostro}}

\bibleverse{29} Cuando Moisés bajó del monte Sinaí con las dos tablas de
la alianza en la mano, cuando bajó del monte, Moisés no sabía que la
piel de su rostro brillaba por haber hablado con él. \bibleverse{30}
Cuando Aarón y todos los hijos de Israel vieron a Moisés, he aquí que la
piel de su rostro brillaba, y tuvieron miedo de acercarse a él.
\footnote{\textbf{34:30} 2Cor 3,7-18} \bibleverse{31} Moisés los llamó,
y Aarón y todos los jefes de la congregación volvieron a él; y Moisés
les habló. \bibleverse{32} Después se acercaron todos los hijos de
Israel, y él les dio todos los mandamientos que Yahvé había hablado con
él en el monte Sinaí. \bibleverse{33} Cuando Moisés terminó de hablar
con ellos, se puso un velo sobre el rostro. \bibleverse{34} Pero cuando
Moisés entró delante de Yahvé para hablar con él, se quitó el velo hasta
que salió; y salió y habló a los hijos de Israel lo que se le había
ordenado. \footnote{\textbf{34:34} Éxod 33,8-9} \bibleverse{35} Los
hijos de Israel vieron el rostro de Moisés, que la piel del rostro de
Moisés resplandecía; entonces Moisés volvió a poner el velo sobre su
rostro, hasta que entró a hablar con él.

\hypertarget{comunicaciuxf3n-del-mandamiento-del-suxe1bado-invitaciuxf3n-a-contribuir-al-tabernuxe1culo}{%
\subsection{Comunicación del mandamiento del sábado; Invitación a
contribuir al
tabernáculo}\label{comunicaciuxf3n-del-mandamiento-del-suxe1bado-invitaciuxf3n-a-contribuir-al-tabernuxe1culo}}

\hypertarget{section-34}{%
\section{35}\label{section-34}}

\bibleverse{1} Moisés reunió a toda la congregación de los hijos de
Israel y les dijo: ``Estas son las palabras que Yahvé ha ordenado, para
que las pongáis en práctica. \bibleverse{2} `Seis días se trabajará,
pero el séptimo día será un día sagrado para ustedes, un día de descanso
solemne para Yahvé; cualquiera que haga algún trabajo en él será
condenado a muerte. \footnote{\textbf{35:2} Éxod 20,8-11; Éxod 31,12-17}
\bibleverse{3} No encenderéis fuego en vuestras moradas en el día de
reposo'\,''.

\bibleverse{4} Moisés habló a toda la congregación de los hijos de
Israel, diciendo: ``Esto es lo que mandó Yahvé, diciendo: \bibleverse{5}
`Tomad de entre vosotros una ofrenda para Yahvé. El que tenga el corazón
dispuesto, que lo traiga como ofrenda a Yahvé: oro, plata, bronce,
\footnote{\textbf{35:5} Éxod 25,2} \bibleverse{6} azul, púrpura,
escarlata, lino fino, pelo de cabra, \bibleverse{7} pieles de carnero
teñidas de rojo, cueros de vaca marina, madera de acacia, \bibleverse{8}
aceite para la luz, especias para el aceite de la unción y para el
incienso aromático, \bibleverse{9} piedras de ónice, y piedras para
engastar para el efod y para el pectoral.

\bibleverse{10} ``\,`Que venga todo sabio de corazón de entre vosotros y
haga todo lo que Yahvé ha mandado: \bibleverse{11} el tabernáculo, su
cubierta exterior, su techo, sus corchetes, sus tablas, sus barras, sus
pilares y sus bases; \footnote{\textbf{35:11} Éxod 31,7-11}
\bibleverse{12} el arca y sus postes, el propiciatorio, el velo de la
pantalla; \bibleverse{13} la mesa con sus postes y todos sus
recipientes, y el pan de la función; \bibleverse{14} el candelabro para
la luz, con sus vasos, sus lámparas y el aceite para la luz;
\bibleverse{15} y el altar del incienso con sus varas, el aceite de la
unción, el incienso aromático, la cortina de la puerta, a la entrada del
tabernáculo \bibleverse{16} el altar del holocausto, con su reja de
bronce, sus varas y todos sus utensilios, la pila y su base;
\bibleverse{17} las cortinas del atrio, sus columnas, sus bases y la
cortina para la puerta del atrio; \bibleverse{18} las clavijas del
tabernáculo, las clavijas del atrio y sus cuerdas; \bibleverse{19} las
vestimentas finamente trabajadas para ministrar en el lugar santo: las
vestimentas sagradas para Aarón, el sacerdote, y las vestimentas de sus
hijos, para ministrar en el oficio del sacerdote.'\,''

\hypertarget{la-gente-muestra-su-disposiciuxf3n}{%
\subsection{La gente muestra su
disposición}\label{la-gente-muestra-su-disposiciuxf3n}}

\bibleverse{20} Toda la congregación de los hijos de Israel partió de la
presencia de Moisés. \bibleverse{21} Vinieron, todos aquellos cuyo
corazón los animó, y todos aquellos a quienes su espíritu los hizo
dispuestos, y trajeron la ofrenda de Yahvé para la obra de la Tienda del
Encuentro, y para todo su servicio, y para las vestiduras sagradas.
\footnote{\textbf{35:21} Éxod 36,3; 1Cró 29,5; 1Cró 29,9; 2Cor 9,7}
\bibleverse{22} Vinieron, tanto hombres como mujeres, todos los que
estaban dispuestos, y trajeron broches, pendientes, anillos de sello y
brazaletes, todas las joyas de oro; todo hombre que ofreció una ofrenda
de oro a Yavé. \bibleverse{23} Todo el que tenía azul, púrpura,
escarlata, lino fino, pelo de cabra, pieles de carnero teñidas de rojo y
cueros de vaca marina, los traía. \bibleverse{24} Todo el que ofrecía
una ofrenda de plata y de bronce traía la ofrenda de Yahvé; y todo el
que tenía madera de acacia para cualquier obra del servicio, la traía.
\bibleverse{25} Todas las mujeres de corazón sabio hilaron con sus
manos, y trajeron lo que habían hilado: el azul, la púrpura, la
escarlata y el lino fino. \bibleverse{26} Todas las mujeres cuyo corazón
las movía a la sabiduría hilaron el pelo de las cabras. \bibleverse{27}
Los jefes trajeron las piedras de ónice y las piedras de engaste para el
efod y para el pectoral; \bibleverse{28} con la especia y el aceite para
la luz, para el aceite de la unción y para el incienso aromático.
\bibleverse{29} Los hijos de Israel trajeron una ofrenda voluntaria a
Yavé; cada hombre y cada mujer cuyo corazón los hizo traer para toda la
obra, que Yavé había mandado hacer por medio de Moisés.

\hypertarget{nombramiento-de-los-capataces-y-demuxe1s-artesanos-abundantes-donaciones-y-servicios-voluntarios-por-parte-del-pueblo}{%
\subsection{Nombramiento de los capataces y demás artesanos; abundantes
donaciones y servicios voluntarios por parte del
pueblo}\label{nombramiento-de-los-capataces-y-demuxe1s-artesanos-abundantes-donaciones-y-servicios-voluntarios-por-parte-del-pueblo}}

\bibleverse{30} Moisés dijo a los hijos de Israel: ``He aquí que Yahvé
ha llamado por nombre a Bezalel, hijo de Uri, hijo de Hur, de la tribu
de Judá. \bibleverse{31} Lo ha llenado con el Espíritu de Dios, en
sabiduría, en inteligencia, en conocimiento y en toda clase de trabajos;
\bibleverse{32} y para hacer obras de arte, para trabajar en oro, en
plata, en bronce, \bibleverse{33} en el corte de piedras para engastar,
y en el tallado de madera, para trabajar en toda clase de trabajos de
habilidad. \bibleverse{34} Ha puesto en su corazón que puede enseñar,
tanto él como Oholiab, hijo de Ahisamac, de la tribu de Dan.
\bibleverse{35} Los ha llenado de sabiduría de corazón para que trabajen
en toda clase de obra, del grabador, del obrero hábil y del bordador, en
azul, en púrpura, en escarlata y en lino fino, y del tejedor, incluso de
los que hacen cualquier obra, y de los que hacen obras hábiles.

\hypertarget{section-35}{%
\section{36}\label{section-35}}

\bibleverse{1} ``Bezalel y Oholiab trabajarán con todo hombre de corazón
sabio, en quien Yahvé haya puesto sabiduría y entendimiento para saber
hacer toda la obra para el servicio del santuario, según todo lo que
Yahvé ha ordenado.''

\bibleverse{2} Moisés llamó a Bezalel y a Oholiab, y a todo hombre de
corazón sabio, en cuyo corazón Yahvé había puesto la sabiduría, a todo
aquel cuyo corazón lo movía a venir a la obra para hacerla.
\bibleverse{3} Ellos recibían de Moisés toda la ofrenda que los hijos de
Israel habían traído para la obra del servicio del santuario, con la
cual la hacían. Cada mañana le traían ofrendas voluntarias.
\bibleverse{4} Todos los sabios, que realizaban toda la obra del
santuario, venían cada uno de su trabajo que hacía. \bibleverse{5}
Hablaron con Moisés, diciendo: ``El pueblo ha traído mucho más de lo
necesario para el servicio de la obra que Yahvé mandó hacer.''

\bibleverse{6} Moisés dio un mandamiento, y lo hicieron proclamar por
todo el campamento, diciendo: ``Que ni el hombre ni la mujer hagan otra
cosa para la ofrenda para el santuario''. Así el pueblo se abstuvo de
traer. \bibleverse{7} Porque lo que tenían era suficiente para hacer
toda la obra, y demasiado.

\hypertarget{la-fabricaciuxf3n-de-las-cuatro-cubiertas-de-la-tienda}{%
\subsection{La fabricación de las cuatro cubiertas de la
tienda}\label{la-fabricaciuxf3n-de-las-cuatro-cubiertas-de-la-tienda}}

\bibleverse{8} Todos los sabios de corazón entre los que hacían la obra
hicieron el tabernáculo con diez cortinas de lino fino torcido, azul,
púrpura y escarlata. Las hicieron con querubines, obra de un hábil
obrero. \footnote{\textbf{36:8} Éxod 26,1-14} \bibleverse{9} La longitud
de cada cortina era de veintiocho codos, y el ancho de cada cortina de
cuatro codos. Todas las cortinas tenían una misma medida.
\bibleverse{10} Unió cinco cortinas entre sí, y las otras cinco cortinas
las unió entre sí. \bibleverse{11} Hizo lazos de color azul en el borde
de una de las cortinas, desde el borde en la unión. Igualmente hizo en
el borde de la cortina que estaba más afuera en el segundo acoplamiento.
\bibleverse{12} Hizo cincuenta lazos en la primera cortina y cincuenta
lazos en el borde de la cortina que estaba en el segundo acoplamiento.
Los lazos estaban opuestos entre sí. \bibleverse{13} Hizo cincuenta
corchetes de oro y unió las cortinas entre sí con los corchetes; así el
tabernáculo formaba una unidad.

\bibleverse{14} Hizo cortinas de pelo de cabra para cubrir el
tabernáculo. Les hizo once cortinas. \bibleverse{15} La longitud de cada
cortina era de treinta codos, y el ancho de cada cortina era de cuatro
codos. Las once cortinas tenían una sola medida. \bibleverse{16} Unió
cinco cortinas solas y seis cortinas solas. \bibleverse{17} Hizo
cincuenta lazos en el borde de la cortina que estaba más afuera en el
acople, e hizo cincuenta lazos en el borde de la cortina que estaba más
afuera en el segundo acople. \bibleverse{18} Hizo cincuenta ganchos de
bronce para unir la tienda, a fin de que fuera una unidad.
\bibleverse{19} Hizo una cubierta para la tienda de pieles de carnero
teñidas de rojo, y una cubierta de pieles de vaca marina encima.

\hypertarget{fabricaciuxf3n-del-marco-de-madera}{%
\subsection{Fabricación del marco de
madera}\label{fabricaciuxf3n-del-marco-de-madera}}

\bibleverse{20} Hizo las tablas para el tabernáculo de madera de acacia,
de pie. \footnote{\textbf{36:20} Éxod 26,15-25} \bibleverse{21} Diez
codos era la longitud de una tabla, y codo y medio la anchura de cada
tabla. \bibleverse{22} Cada tabla tenía dos espigas unidas entre sí. Así
hizo todas las tablas del tabernáculo. \bibleverse{23} Hizo las tablas
del tabernáculo, veinte tablas para el lado sur hacia el sur.
\bibleverse{24} Hizo cuarenta basas de plata debajo de las veinte
tablas: dos basas debajo de una tabla para sus dos espigas, y dos basas
debajo de otra tabla para sus dos espigas. \bibleverse{25} Para el
segundo lado del tabernáculo, en el lado norte, hizo veinte tablas
\bibleverse{26} y sus cuarenta basas de plata: dos basas debajo de una
tabla y dos basas debajo de otra tabla. \bibleverse{27} Para la parte
más alejada del tabernáculo, al oeste, hizo seis tablas. \bibleverse{28}
Hizo dos tablas para las esquinas del tabernáculo en la parte más
alejada. \bibleverse{29} Eran dobles por debajo, y de la misma manera
llegaban hasta su parte superior a un anillo. Hizo esto en las dos
esquinas. \bibleverse{30} Había ocho tablas y sus bases de plata,
dieciséis bases; debajo de cada tabla había dos bases.

\bibleverse{31} Hizo barras de madera de acacia: cinco para las tablas
de un lado del tabernáculo, \footnote{\textbf{36:31} Éxod 26,26-30}
\bibleverse{32} y cinco barras para las tablas del otro lado del
tabernáculo, y cinco barras para las tablas del tabernáculo de la parte
posterior hacia el oeste. \bibleverse{33} Hizo que la barra del medio
pasara por en medio de las tablas, desde un extremo hasta el otro.
\bibleverse{34} Recubrió las tablas de oro, e hizo sus anillos de oro
como lugares para las barras, y recubrió las barras de oro.

\hypertarget{fabricaciuxf3n-de-las-dos-cortinas}{%
\subsection{Fabricación de las dos
cortinas}\label{fabricaciuxf3n-de-las-dos-cortinas}}

\bibleverse{35} Hizo el velo de azul, púrpura, escarlata y lino fino,
con querubines. Lo hizo obra de un hábil obrero. \footnote{\textbf{36:35}
  Éxod 26,31-37} \bibleverse{36} Hizo para él cuatro columnas de acacia
y las recubrió de oro. Sus ganchos eran de oro. Les fundió cuatro bases
de plata. \bibleverse{37} Hizo una cortina para la puerta de la tienda,
de azul, púrpura, escarlata y lino fino, obra de un bordador;
\bibleverse{38} y sus cinco columnas con sus ganchos. Recubrió de oro
sus capiteles y sus filetes, y sus cinco bases eran de bronce.

\hypertarget{el-cajuxf3n-con-la-placa-de-cubierta}{%
\subsection{El cajón con la placa de
cubierta}\label{el-cajuxf3n-con-la-placa-de-cubierta}}

\hypertarget{section-36}{%
\section{37}\label{section-36}}

\bibleverse{1} Bezalel hizo el arca de madera de acacia. Su longitud era
de dos codos y medio, y su anchura de codo y medio, y su altura de codo
y medio. \footnote{\textbf{37:1} Éxod 25,10-22} \bibleverse{2} La
recubrió de oro puro por dentro y por fuera, y le hizo una moldura de
oro alrededor. \bibleverse{3} Le fundió cuatro anillos de oro en sus
cuatro pies: dos anillos en un lado y dos anillos en el otro.
\bibleverse{4} Hizo varas de madera de acacia y las recubrió de oro.
\bibleverse{5} Colocó las varas en los anillos a los lados del arca,
para que la soportaran. \bibleverse{6} Hizo un propiciatorio de oro
puro. Su longitud era de dos codos y medio, y su anchura de codo y
medio. \bibleverse{7} Hizo dos querubines de oro. Los hizo de obra
batida, en los dos extremos del propiciatorio: \bibleverse{8} un
querubín en un extremo, y un querubín en el otro extremo. Hizo los
querubines de una sola pieza con el propiciatorio en sus dos extremos.
\bibleverse{9} Los querubines extendían sus alas por encima, cubriendo
el propiciatorio con sus alas, con sus rostros uno hacia el otro. Los
rostros de los querubines estaban hacia el propiciatorio.

\hypertarget{la-mesa-para-los-panes-de-la-proposiciuxf3n-y-las-libaciones}{%
\subsection{La mesa para los panes de la proposición y las
libaciones}\label{la-mesa-para-los-panes-de-la-proposiciuxf3n-y-las-libaciones}}

\bibleverse{10} Hizo la mesa de madera de acacia. Su longitud era de dos
codos, su anchura de un codo y su altura de codo y medio. \footnote{\textbf{37:10}
  Éxod 25,23-29} \bibleverse{11} La recubrió de oro puro y le hizo una
moldura de oro alrededor. \bibleverse{12} Hizo un borde de un palmo de
ancho a su alrededor, e hizo una moldura de oro a su alrededor.
\bibleverse{13} Le fundió cuatro anillos de oro y los puso en las cuatro
esquinas que estaban sobre sus cuatro pies. \bibleverse{14} Los anillos
estaban cerca del borde, los lugares para las varas para llevar la mesa.
\bibleverse{15} Hizo las varas de madera de acacia y las recubrió de oro
para transportar la mesa. \bibleverse{16} Hizo los recipientes que
estaban sobre la mesa, sus platos, sus cucharas, sus tazones y sus
cántaros para servir, de oro puro.

\hypertarget{el-candelero-de-oro}{%
\subsection{El candelero de oro}\label{el-candelero-de-oro}}

\bibleverse{17} Hizo el candelabro de oro puro. Hizo el candelabro de
obra batida. Su base, su fuste, sus copas, sus capullos y sus flores
eran de una sola pieza. \footnote{\textbf{37:17} Éxod 25,31-39}
\bibleverse{18} Había seis brazos que salían de sus lados: tres brazos
del candelabro salían de un lado, y tres brazos del candelabro salían
del otro lado: \bibleverse{19} tres copas hechas como flores de almendro
en un brazo, un capullo y una flor, y tres copas hechas como flores de
almendro en el otro brazo, un capullo y una flor; así para los seis
brazos que salían del candelabro. \bibleverse{20} En el candelabro había
cuatro copas hechas como flores de almendro, sus capullos y sus flores;
\bibleverse{21} y un capullo bajo dos ramas de una pieza con él, y un
capullo bajo dos ramas de una pieza con él, y un capullo bajo dos ramas
de una pieza con él, para las seis ramas que salían de él.
\bibleverse{22} Sus brotes y sus ramas eran de una sola pieza con él.
Todo el conjunto era una sola pieza batida de oro puro. \bibleverse{23}
Hizo sus siete lámparas, sus apagadores y sus tabaqueras de oro puro.
\bibleverse{24} Lo hizo de un talento de oro puro, con todos sus
recipientes.

\hypertarget{el-altar-del-incienso}{%
\subsection{El altar del incienso}\label{el-altar-del-incienso}}

\bibleverse{25} Hizo el altar del incienso de madera de acacia. Era
cuadrado: su longitud era de un codo, y su anchura de un codo. Su altura
era de dos codos. Sus cuernos eran de una sola pieza. \footnote{\textbf{37:25}
  Éxod 30,1-5} \bibleverse{26} Lo recubrió de oro puro: su parte
superior, sus lados alrededor y sus cuernos. Hizo una moldura de oro a
su alrededor. \bibleverse{27} Le hizo dos anillos de oro debajo de la
corona de la moldura, en sus dos costillas, en sus dos lados, para los
lugares de las varas con las que se transportaba. \bibleverse{28} Hizo
las varas de madera de acacia y las recubrió de oro. \bibleverse{29}
Hizo el aceite santo de la unción y el incienso puro de especias dulces,
según el arte del perfumista. \footnote{\textbf{37:29} Éxod 30,25-35}

\hypertarget{el-altar-de-los-holocaustos.-la-cuenca-de-cobre}{%
\subsection{El altar de los holocaustos. La cuenca de
cobre}\label{el-altar-de-los-holocaustos.-la-cuenca-de-cobre}}

\hypertarget{section-37}{%
\section{38}\label{section-37}}

\bibleverse{1} Hizo el altar del holocausto de madera de acacia. Era
cuadrado. Su longitud era de cinco codos, su anchura era de cinco codos,
y su altura de tres codos. \footnote{\textbf{38:1} Éxod 27,1-8}
\bibleverse{2} Hizo sus cuernos en sus cuatro esquinas. Sus cuernos eran
de una sola pieza con él, y lo recubrió de bronce. \bibleverse{3} Hizo
todos los utensilios del altar: las ollas, las palas, las pilas, los
tenedores y las sartenes. Hizo todos sus recipientes de bronce.
\bibleverse{4} Hizo para el altar una rejilla de red de bronce, debajo
de la cornisa que lo rodeaba por debajo, que llegaba hasta la mitad.
\bibleverse{5} Fundió cuatro anillos para las cuatro esquinas de la reja
de bronce, para que fueran lugares para los postes. \bibleverse{6} Hizo
los postes de madera de acacia y los recubrió de bronce. \bibleverse{7}
Colocó las varas en los anillos a los lados del altar, con los que se
podía transportar. Lo hizo hueco con tablas.

\bibleverse{8} Hizo la pila de bronce, y su base de bronce, de los
espejos de las mujeres que servían a la puerta de la Tienda de Reunión.
\footnote{\textbf{38:8} Éxod 30,18-21}

\hypertarget{el-atrio}{%
\subsection{El atrio}\label{el-atrio}}

\bibleverse{9} Hizo el atrio: para el lado sur, las cortinas del atrio
eran de lino fino torcido, de cien codos; \footnote{\textbf{38:9} Éxod
  27,9-19} \bibleverse{10} sus columnas eran veinte, y sus basas veinte,
de bronce; los ganchos de las columnas y sus filetes eran de plata.
\bibleverse{11} Para el lado norte, cien codos; sus columnas, veinte, y
sus basas, veinte, de bronce; los ganchos de las columnas, y sus
filetes, de plata. \bibleverse{12} Para el lado del oeste había cortinas
de cincuenta codos, sus columnas de diez, y sus bases de diez; los
ganchos de las columnas, y sus filetes, de plata. \bibleverse{13} Para
el lado del oriente, cincuenta codos, \bibleverse{14} las cortinas de un
lado eran de quince codos; sus columnas, tres, y sus bases, tres;
\bibleverse{15} y lo mismo para el otro lado: a un lado y a otro de la
puerta del atrio había cortinas de quince codos; sus columnas, tres, y
sus bases, tres. \bibleverse{16} Todas las cortinas alrededor del atrio
eran de lino fino. \bibleverse{17} Las bases de las columnas eran de
bronce. Los ganchos de las columnas y sus filetes eran de plata. Sus
capiteles estaban recubiertos de plata. Todas las columnas del atrio
tenían bandas de plata. \bibleverse{18} La cortina de la puerta del
atrio era obra del bordador, de azul, púrpura, escarlata y lino fino. Su
longitud era de veinte codos, y su altura a lo ancho era de cinco codos,
como las cortinas del atrio. \bibleverse{19} Sus columnas eran cuatro, y
sus bases cuatro, de bronce; sus ganchos de plata, y el revestimiento de
sus capiteles y sus filetes, de plata. \bibleverse{20} Todos los
pasadores del tabernáculo, y alrededor del atrio, eran de bronce.

\hypertarget{el-cuxe1lculo-del-costo-de-los-metales-utilizados-para-el-santuario}{%
\subsection{El cálculo del costo de los metales utilizados para el
santuario}\label{el-cuxe1lculo-del-costo-de-los-metales-utilizados-para-el-santuario}}

\bibleverse{21} Estas son las cantidades de materiales que se usaron
para el tabernáculo, el Tabernáculo del Testimonio, tal como fueron
contadas, según el mandato de Moisés, para el servicio de los levitas,
por mano de Itamar, hijo del sacerdote Aarón. \footnote{\textbf{38:21}
  Núm 4,28} \bibleverse{22} Bezalel hijo de Uri, hijo de Hur, de la
tribu de Judá, hizo todo lo que Yahvé mandó a Moisés. \footnote{\textbf{38:22}
  Éxod 31,1-11} \bibleverse{23} Con él estaba Oholiab, hijo de Ahisamac,
de la tribu de Dan, grabador y hábil obrero, y bordador en azul, en
púrpura, en escarlata y en lino fino.

\bibleverse{24} Todo el oro que se usó para la obra en toda la obra del
santuario, el oro de la ofrenda, fue de veintinueve talentos\footnote{\textbf{38:24}
  Un talento es de unos 30 kilogramos o 66 libras o 965 onzas troy.} y
setecientos treinta siclos, según el siclo del santuario. \footnote{\textbf{38:24}
  Un siclo equivale a unos 10 gramos o a unas 0,32 onzas troy.}
\footnote{\textbf{38:24} Éxod 30,13} \bibleverse{25} La plata de los
contados de la congregación era de cien talentos\footnote{\textbf{38:25}
  ``He aquí'', de ``\hebrew{הִנֵּה}'', significa mirar, fijarse, observar,
  ver o contemplar. Se utiliza a menudo como interjección.} y mil
setecientos setenta y cinco siclos, según el siclo del santuario:
\bibleverse{26} un beka por cabeza, es decir, medio siclo, según el
siclo del santuario, por todos los que pasaron a los contados, de veinte
años para arriba, por seiscientos tres mil quinientos cincuenta hombres.
\bibleverse{27} Los cien talentos de plata fueron para fundir las basas
del santuario y las basas del velo: cien basas para los cien talentos,
un talento por basas. \bibleverse{28} De los mil setecientos setenta y
cinco siclos hizo ganchos para las columnas, recubrió sus capiteles e
hizo filetes para ellos. \bibleverse{29} El bronce de la ofrenda era de
setenta talentos y dos mil cuatrocientos siclos. \bibleverse{30} Con
esto hizo los zócalos de la puerta de la Tienda de reunión, el altar de
bronce, la reja de bronce para él, todos los utensilios del altar,
\bibleverse{31} los zócalos alrededor del atrio, los zócalos de la
puerta del atrio, todos los pasadores del tabernáculo y todos los
pasadores alrededor del atrio.

\hypertarget{confecciuxf3n-de-la-ropa-sacerdotal}{%
\subsection{Confección de la ropa
sacerdotal}\label{confecciuxf3n-de-la-ropa-sacerdotal}}

\hypertarget{section-38}{%
\section{39}\label{section-38}}

\bibleverse{1} De azul, púrpura y escarlata, hicieron las prendas de
vestir finamente trabajadas para ministrar en el lugar santo, e hicieron
las vestiduras sagradas para Aarón, como Yahvé le ordenó a Moisés.
\footnote{\textbf{39:1} Éxod 28,4-30}

\hypertarget{el-vestido-de-hombro-ephod-1}{%
\subsection{El vestido de hombro
(ephod)}\label{el-vestido-de-hombro-ephod-1}}

\bibleverse{2} Hizo el efod de oro, azul, púrpura, escarlata y lino
fino. \bibleverse{3} El oro lo batieron en láminas finas y lo cortaron
en hilos, para trabajarlo con el azul, la púrpura, la escarlata y el
lino fino, obra del obrero hábil. \bibleverse{4} Le hicieron correas
para los hombros, unidas entre sí. Se unió por los dos extremos.
\bibleverse{5} La banda tejida con destreza que lo cubría, con la cual
se sujetaba, era de la misma pieza, como su obra: de oro, de azul, de
púrpura, de escarlata y de lino fino torcido, como Yahvé le ordenó a
Moisés.

\bibleverse{6} Trabajaron las piedras de ónice, encerradas en engastes
de oro, grabadas con los grabados de un sello, según los nombres de los
hijos de Israel. \bibleverse{7} Las puso en los tirantes del efod, para
que fueran piedras conmemorativas de los hijos de Israel, tal como el
Señor se lo había ordenado a Moisés.

\hypertarget{el-adorno-del-pecho}{%
\subsection{El adorno del pecho}\label{el-adorno-del-pecho}}

\bibleverse{8} Hizo el pectoral, obra de un hábil obrero, como la obra
del efod: de oro, de azul, de púrpura, de escarlata y de lino torcido.
\bibleverse{9} Era cuadrado. Hicieron el pectoral doble. Su longitud era
de un palmo, y su anchura de un palmo, siendo doble. \bibleverse{10}
Colocaron en él cuatro hileras de piedras. Una hilera de rubí, topacio y
berilo era la primera hilera; \bibleverse{11} y la segunda hilera, una
turquesa, un zafiro, y una esmeralda; \bibleverse{12} y la tercera
hilera, un jacinto, una ágata y una amatista; \bibleverse{13} y la
cuarta hilera, un crisolito, un ónice y un jaspe. Estaban encerradas en
engastes de oro. \bibleverse{14} Las piedras eran según los nombres de
los hijos de Israel, doce, según sus nombres; como los grabados de un
sello, cada uno según su nombre, para las doce tribus. \bibleverse{15}
Hicieron en el pectoral cadenas como cordones, de oro puro trenzado.
\bibleverse{16} Hicieron dos engastes de oro y dos anillos de oro, y
pusieron los dos anillos en los dos extremos del pectoral.
\bibleverse{17} Pusieron las dos cadenas trenzadas de oro en los dos
anillos de los extremos del pectoral. \bibleverse{18} Los otros dos
extremos de las dos cadenas trenzadas los pusieron en los dos engastes,
y los pusieron en los tirantes del efod, en su parte delantera.
\bibleverse{19} Hicieron dos anillos de oro y los pusieron en los dos
extremos del pectoral, en su borde, que estaba hacia el lado del efod,
hacia adentro. \bibleverse{20} Hicieron otros dos anillos de oro y los
pusieron en los dos tirantes del efod por debajo, en su parte delantera,
cerca de su acoplamiento, por encima de la banda hábilmente tejida del
efod. \bibleverse{21} Luego unieron el pectoral por sus anillos a los
anillos del efod con un cordón de color azul, para que quedara sobre la
banda hábilmente tejida del efod, y para que el pectoral no se
desprendiera del efod, tal como el Señor lo había ordenado a Moisés.

\hypertarget{la-prenda-superior-para-el-vestido-de-hombros-1}{%
\subsection{La prenda superior para el vestido de
hombros}\label{la-prenda-superior-para-el-vestido-de-hombros-1}}

\bibleverse{22} Hizo el manto del efod de tela, todo de color azul.
\footnote{\textbf{39:22} Éxod 28,31-35} \bibleverse{23} La abertura del
manto en el centro era como la abertura de una cota de malla, con una
cinta alrededor de la abertura, para que no se rompiera. \bibleverse{24}
Hicieron en las faldas del manto granadas de color azul, púrpura,
escarlata y lino torcido. \bibleverse{25} Hicieron campanas de oro puro,
y pusieron las campanas entre las granadas alrededor de los faldones del
manto, entre las granadas; \bibleverse{26} una campana y una granada,
una campana y una granada, alrededor de los faldones del manto, para
ministrar, como Yahvé le ordenó a Moisés.

\hypertarget{la-ropa-oficial-restante-de-los-sacerdotes}{%
\subsection{La ropa oficial restante de los
sacerdotes}\label{la-ropa-oficial-restante-de-los-sacerdotes}}

\bibleverse{27} Hicieron las túnicas de lino fino de obra tejida para
Aarón y para sus hijos, \footnote{\textbf{39:27} Éxod 28,39-42}
\bibleverse{28} el turbante de lino fino, las cintillos de lino fino,
los pantalones de lino fino, \bibleverse{29} el fajín de lino fino,
azul, púrpura y escarlata, obra del bordador, como Yahvé mandó a Moisés.

\hypertarget{el-rostro-del-sumo-sacerdote}{%
\subsection{El rostro del sumo
sacerdote}\label{el-rostro-del-sumo-sacerdote}}

\bibleverse{30} Hicieron la placa de la corona sagrada de oro puro, y
escribieron en ella una inscripción, como los grabados de un sello:
``SANTO A YAHWEH''. \footnote{\textbf{39:30} Éxod 29,6; Lev 8,9}
\bibleverse{31} Le ataron un cordón de color azul, para sujetarlo al
turbante de arriba, como Yahvé le ordenó a Moisés. \footnote{\textbf{39:31}
  Éxod 28,36-38}

\hypertarget{entrega-de-los-artuxedculos-terminados-a-moisuxe9s}{%
\subsection{Entrega de los artículos terminados a
Moisés}\label{entrega-de-los-artuxedculos-terminados-a-moisuxe9s}}

\bibleverse{32} Así quedó terminada toda la obra del tabernáculo de la
Tienda de Reunión. Los hijos de Israel hicieron conforme a todo lo que
Yahvé ordenó a Moisés; así lo hicieron. \bibleverse{33} Llevaron el
tabernáculo a Moisés la tienda, con todos sus muebles, sus broches, sus
tablas, sus barras, sus pilares, sus bases, \bibleverse{34} la cubierta
de pieles de carnero teñidas de rojo, la cubierta de pieles de vaca
marina, el velo de la pantalla, \bibleverse{35} el arca de la alianza
con sus postes, el propiciatorio, \bibleverse{36} la mesa, todos sus
utensilios, el pan de la proposición, \bibleverse{37} el candelabro
puro, sus lámparas, todos sus utensilios, el aceite para la luz,
\bibleverse{38} el altar de oro, el aceite de la unción, el incienso
aromático, la cortina para la puerta de la Tienda, \bibleverse{39} el
altar de bronce, su reja de bronce, sus varas, todos sus vasos, la pila
y su base, \bibleverse{40} las cortinas del atrio, sus columnas, sus
bases, la cortina para la puerta del atrio, sus cuerdas, sus clavijas, y
todos los instrumentos del servicio del tabernáculo, para la Tienda de
Reunión, \bibleverse{41} las vestimentas finamente trabajadas para
ministrar en el lugar santo, las vestimentas sagradas para el sacerdote
Aarón y las vestimentas de sus hijos, para ministrar en el oficio del
sacerdote. \bibleverse{42} Conforme a todo lo que Yahvé mandó a Moisés,
así hicieron los hijos de Israel todo el trabajo. \bibleverse{43} Moisés
vio toda la obra, y he aquí que la habían hecho como Yahvé había
ordenado. Así lo habían hecho; y Moisés los bendijo.

\hypertarget{establecimiento-y-dedicaciuxf3n-del-santuario}{%
\subsection{Establecimiento y dedicación del
santuario}\label{establecimiento-y-dedicaciuxf3n-del-santuario}}

\hypertarget{section-39}{%
\section{40}\label{section-39}}

\bibleverse{1} Yahvé habló a Moisés, diciendo: \footnote{\textbf{40:1}
  Éxod 25,1-31} \bibleverse{2} ``El primer día del primer mes levantarás
el tabernáculo de la Tienda de Reunión. \bibleverse{3} Pondrás en él el
Arca de la Alianza, y cubrirás el Arca con el velo. \bibleverse{4}
Traerás la mesa y pondrás en orden las cosas que están sobre ella.
Traerás el candelabro y encenderás sus lámparas. \bibleverse{5} Pondrás
el altar de oro para el incienso delante del arca de la alianza, y
pondrás la cortina de la puerta del tabernáculo.

\bibleverse{6} ``Pondrás el altar del holocausto delante de la puerta de
la Carpa del Encuentro. \bibleverse{7} Pondrás la pila entre la Carpa
del Encuentro y el altar, y pondrás agua en ella. \bibleverse{8} Armarás
el atrio alrededor, y colgarás la cortina de la puerta del atrio.

\bibleverse{9} ``Tomarás el aceite de la unción y ungirás el tabernáculo
y todo lo que hay en él, y lo santificarás junto con todo su mobiliario,
y será santo. \bibleverse{10} Ungirás el altar del holocausto con todos
sus utensilios, y santificarás el altar, y el altar será santísimo.
\bibleverse{11} Ungirás la pila y su base, y la santificarás.

\bibleverse{12} ``Llevarás a Aarón y a sus hijos a la puerta de la
Tienda de Reunión, y los lavarás con agua. \bibleverse{13} Pondrás a
Aarón las vestiduras sagradas, lo ungirás y lo santificarás para que me
sirva en el oficio de sacerdote. \bibleverse{14} Traerás a sus hijos y
les pondrás túnicas. \bibleverse{15} Los ungirás, como ungiste a su
padre, para que me sirvan en el oficio sacerdotal. Su unción será para
ellos un sacerdocio eterno por sus generaciones''.

\hypertarget{la-ejecuciuxf3n-del-mandato-divino}{%
\subsection{La ejecución del mandato
divino}\label{la-ejecuciuxf3n-del-mandato-divino}}

\bibleverse{16} Así lo hizo Moisés. Conforme a todo lo que Yahvé le
ordenó, así lo hizo.

\bibleverse{17} En el primer mes del segundo año, el primer día del mes,
se levantó el tabernáculo. \bibleverse{18} Moisés levantó el
tabernáculo, colocó sus bases, puso sus tablas, colocó sus barras y
levantó sus columnas. \bibleverse{19} Extendió la cubierta sobre la
tienda, y puso encima el techo del tabernáculo, como Yahvé le había
ordenado a Moisés. \bibleverse{20} Tomó y puso el pacto en el arca,
colocó las varas sobre el arca y puso el propiciatorio encima del arca.
\bibleverse{21} Llevó el arca al tabernáculo, colocó el velo de la
cortina y cubrió el arca de la alianza, tal como el Señor se lo había
ordenado a Moisés. \bibleverse{22} Puso la mesa en la Tienda de Reunión,
en el lado norte de la Morada, fuera del velo. \bibleverse{23} Sobre
ella puso el pan en orden ante el Señor, como el Señor le había ordenado
a Moisés. \bibleverse{24} Puso el candelabro en la Tienda de Reunión,
frente a la mesa, en el lado sur de la Morada. \bibleverse{25} Encendió
las lámparas delante de Yavé, como Yavé le había ordenado a Moisés.
\bibleverse{26} Puso el altar de oro en la Tienda del Encuentro, delante
del velo; \bibleverse{27} y quemó en él incienso de especias dulces,
como el Señor le había ordenado a Moisés. \bibleverse{28} Colocó la
cortina de la puerta del tabernáculo. \bibleverse{29} Puso el altar del
holocausto a la puerta de la Carpa del Encuentro, y ofreció sobre él el
holocausto y la ofrenda, como Yahvé le había ordenado a Moisés.
\bibleverse{30} Puso la pila entre la Tienda del Encuentro y el altar, y
puso en ella agua para lavarse. \bibleverse{31} Moisés, Aarón y sus
hijos se lavaron allí las manos y los pies. \bibleverse{32} Cuando
entraban en la Tienda del Encuentro, y cuando se acercaban al altar, se
lavaban, como Yahvé le había ordenado a Moisés. \bibleverse{33} El
levantó el atrio alrededor del tabernáculo y del altar, y colocó la
cortina de la puerta del atrio. Y Moisés terminó la obra.

\hypertarget{la-gloria-del-seuxf1or-se-apodera-de-la-morada}{%
\subsection{La gloria del Señor se apodera de la
morada}\label{la-gloria-del-seuxf1or-se-apodera-de-la-morada}}

\bibleverse{34} Entonces la nube cubrió la Tienda del Encuentro, y la
gloria de Yahvé llenó el tabernáculo. \footnote{\textbf{40:34} Éxod
  13,21; Núm 9,15-23; 1Re 8,10-11; Is 4,5; Ezeq 43,5} \bibleverse{35}
Moisés no pudo entrar en la Tienda del Encuentro, porque la nube
permanecía sobre ella, y la gloria de Yahvé llenaba el tabernáculo.
\bibleverse{36} Cuando la nube se alzaba sobre el tabernáculo, los hijos
de Israel seguían adelante en todos sus viajes; \footnote{\textbf{40:36}
  Núm 10,34-36} \bibleverse{37} pero si la nube no se alzaba, entonces
no viajaban hasta el día en que se alzaba. \bibleverse{38} Porque la
nube de Yahvé estaba sobre el tabernáculo de día, y había fuego en la
nube de noche, a la vista de toda la casa de Israel, durante todos sus
viajes.
