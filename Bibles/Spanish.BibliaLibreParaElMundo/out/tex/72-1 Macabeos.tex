\hypertarget{section}{%
\section{1}\label{section}}

\bibleverse{1} Después de Alejandro el Macedonio, hijo de Filipo, que
salió de la tierra de Quitim, e hirió a Darío, rey de los persas y de
los medos, sucedió que, después de haberlo herido, reinó en su lugar, en
tiempos anteriores, sobre\footnote{\textbf{1:1} decir, el Imperio
  griego. Compárese con 1 Macabeos 1:10 y 1 Macabeos 6:2.} Grecia.
\bibleverse{2} Libró muchas batallas, ganó muchas fortalezas, mató a los
reyes de la tierra, \bibleverse{3} recorrió los confines de la tierra y
tomó botín de una multitud de naciones. La tierra estaba tranquila ante
él. Fue exaltado. Su corazón se enalteció. \bibleverse{4} Reunió un
ejército muy fuerte y dominó países, naciones y principados, y le
pagaron tributo.

\bibleverse{5} Después de estas cosas, cayó enfermo y comprendió que iba
a morir. \bibleverse{6} Llamó a sus honorables servidores, que se habían
criado con él desde su juventud, y les repartió su reino en vida.
\bibleverse{7} Alejandro reinó doce años y luego murió. \bibleverse{8}
Entonces sus siervos gobernaron, cada uno en su lugar. \bibleverse{9}
Todos ellos se pusieron coronas después de su muerte, y lo mismo
hicieron sus hijos después de ellos muchos años; y multiplicaron los
males en la tierra.

\bibleverse{10} De ellos salió una raíz pecadora, Antíoco Epífanes, hijo
del rey Antíoco, que había sido rehén en Roma, y reinó en\footnote{\textbf{1:10}
  hacia el año 176 a. C.} el año ciento treinta y siete del reino de los
griegos.

\bibleverse{11} En aquellos días salieron de Israel transgresores de la
ley y persuadieron a muchos, diciendo: Vamos a hacer un pacto con
los\footnote{\textbf{1:11} O, naciones: y así en todo este libro.}
gentiles que nos rodean, porque desde que nos separamos de ellos nos han
sucedido muchos males. \bibleverse{12} Esta propuesta les pareció buena.
\bibleverse{13} Algunos del pueblo acudieron con entusiasmo al rey, y
éste les autorizó a observar las ordenanzas de los\footnote{\textbf{1:13}
  O, naciones: y así en todo este libro.} gentiles. \bibleverse{14} Así
que\footnote{\textbf{1:14} Véase 2 Macabeos 4:9:12.} construyeron un
gimnasio en Jerusalén según las leyes de los\footnote{\textbf{1:14} O,
  naciones: y así en todo este libro.} gentiles. \bibleverse{15} Se
hicieron incircuncisos, abandonaron la santa alianza, se unieron a
los\footnote{\textbf{1:15} O, naciones: y así en todo este libro}
gentiles y se vendieron para hacer el mal.

\bibleverse{16} El reino se estableció a la vista de Antíoco, y planeó
reinar sobre Egipto, para poder reinar sobre ambos reinos.
\bibleverse{17} Entró en Egipto con una\footnote{\textbf{1:17} 2:18.
  Compárese 1 Macabeos 10:65.} gran multitud, con carros, con elefantes,
con caballería y con una gran armada. \bibleverse{18} Hizo la guerra a
Ptolomeo, rey de Egipto. Ptolomeo fue avergonzado ante él, y huyó; y
muchos cayeron heridos de muerte. \bibleverse{19} Se apoderó de las
ciudades fuertes en la tierra de Egipto, y tomó el botín de Egipto.

\bibleverse{20} Antíoco, después de haber derrotado a Egipto, regresó en
el año ciento cuarenta y tres, y subió contra Israel y Jerusalén con una
gran multitud, \bibleverse{21} y entró presuntuosamente en el santuario,
y tomó el altar de oro, el candelabro para la luz y todos sus
utensilios. \bibleverse{22} Tomó la mesa del pan de la proposición, las
copas para las libaciones, las copas, los incensarios de oro, el velo,
las coronas y la decoración de oro de la fachada del templo. Lo quitó
todo. \bibleverse{23} Se llevó la plata, el oro y los objetos preciosos.
Tomó los tesoros escondidos que encontró. \bibleverse{24} Cuando se lo
llevó todo, se marchó a su tierra. Hizo una gran matanza y habló con
mucha arrogancia. \bibleverse{25} Un gran luto invadió a Israel, en
todos los lugares donde se encontraban. \bibleverse{26} Los gobernantes
y los ancianos gimieron. Las vírgenes y los jóvenes se debilitaron. La
belleza de las mujeres cambió. \bibleverse{27} Todos los novios se
lamentaron. La que se sentaba en la cámara nupcial se lamentaba.
\bibleverse{28} La tierra se conmovió por sus habitantes, y toda la casa
de Jacob se vistió de vergüenza.

\bibleverse{29} Después de dos años completos, el rey envió a un jefe de
recaudación de tributos a las ciudades de Judá, y llegó a Jerusalén con
una gran multitud. \bibleverse{30} Les dijo palabras de paz con
sutileza, y ellos le creyeron. Entonces cayó sobre la ciudad de manera
repentina, la golpeó muy duramente y destruyó a mucha gente de Israel.
\bibleverse{31} Tomó los despojos de la ciudad, le prendió fuego y
derribó sus casas y sus murallas por todos lados. \bibleverse{32}
Llevaron cautivos a las mujeres y a los niños, y se apoderaron del
ganado. \bibleverse{33} Luego fortificaron la ciudad de David con una
muralla grande y fuerte, y con torres sólidas, y se convirtió en su
ciudadela. \bibleverse{34} Pusieron allí a una nación pecadora,
transgresora de la ley, y se fortalecieron en ella. \bibleverse{35}
Acumularon armas y alimentos, y reuniendo los despojos de Jerusalén, los
almacenaron allí, y se convirtieron en una gran amenaza. \bibleverse{36}
Se convirtió en un lugar de acecho contra el santuario, y en un
adversario maligno para Israel continuamente. \bibleverse{37} Derramaron
sangre inocente por todos los lados del santuario, y profanaron el
santuario. \bibleverse{38} Los habitantes de Jerusalén huyeron a causa
de ellos. Se convirtió en una morada de extranjeros. Se convirtió en
extraña para los que habían nacido en ella, y sus hijos la abandonaron.
\bibleverse{39} Su santuario fue asolado como un desierto, sus fiestas
se convirtieron en luto, sus sábados en oprobio y su honor en desprecio.
\bibleverse{40} Según su gloria, así se multiplicó su deshonra, y su
exaltación se convirtió en luto.

\bibleverse{41} El rey Antíoco escribió a todo su reino que todos debían
ser un solo pueblo, \bibleverse{42} y que cada uno debía abandonar sus
propias leyes. Todas las naciones estuvieron de acuerdo con la palabra
del rey. \bibleverse{43} Muchos de Israel consintieron en su culto,
sacrificaron a los ídolos y profanaron el sábado. \bibleverse{44} El rey
envió cartas por medio de mensajeros a Jerusalén y a las ciudades de
Judá, para que siguieran leyes extrañas al país, \bibleverse{45} y para
que prohibieran los holocaustos completos y los sacrificios y las
libaciones en el santuario, y para que profanaran los sábados y las
fiestas, \bibleverse{46} y contaminaran el santuario y a los que eran
santos; \bibleverse{47} que construyeran altares, templos y santuarios
para los ídolos, y que sacrificaran carne de cerdo y animales inmundos;
\bibleverse{48} y que dejaran a sus hijos incircuncisos, que hicieran
abominables sus almas con toda clase de impurezas y profanaciones;
\bibleverse{49} para que se olvidaran de la ley y cambiaran todas las
ordenanzas. \bibleverse{50} El que no haga conforme a la palabra del
rey, morirá. \bibleverse{51} Conforme a todas estas palabras escribió a
todo su reino. Nombró supervisores sobre todo el pueblo, y ordenó a las
ciudades de Judá que sacrificaran, ciudad por ciudad. \bibleverse{52}
Del pueblo se juntaron a ellos muchos, todos los que habían abandonado
la ley; e hicieron cosas malas en la tierra. \bibleverse{53} Hicieron
que Israel se escondiera en todos los lugares de refugio que tenía.

\bibleverse{54} El día quince de Chislev, en el año ciento cuarenta y
cinco, edificaron una abominación de desolación sobre el altar, y en las
ciudades de Judá, por todas partes, edificaron altares de ídolos.
\bibleverse{55} A las puertas de las casas y en las calles quemaron
incienso. \bibleverse{56} Rompieron los libros de la ley que encontraron
en pedazos y les prendieron fuego. \bibleverse{57} A cualquiera que se
le encontrara con algún libro de la alianza, y si alguno consentía en la
ley, la sentencia del rey lo entregaba a la muerte. \bibleverse{58} Así
hacían en su poder a Israel, a los que se encontraban mes a mes en las
ciudades. \bibleverse{59} El día veinticinco del mes sacrificaban sobre
el altar de los ídolos que estaba encima del altar de los holocaustos.
\bibleverse{60} Mataron a las mujeres que habían circuncidado a sus
hijos, según el mandamiento. \bibleverse{61} Colgaron al cuello a sus
hijos, a sus casas y a las que los habían circuncidado. \bibleverse{62}
Muchos en Israel estaban plenamente resueltos y confirmados en sí mismos
a no comer cosas inmundas. \bibleverse{63} Decidieron morir para no
contaminarse con la comida y para no profanar el santo pacto; y
murieron. \bibleverse{64} Una ira muy grande cayó sobre Israel.

\hypertarget{section-1}{%
\section{2}\label{section-1}}

\bibleverse{1} En aquellos días se levantó Matatías, hijo de Juan, hijo
de Simeón, sacerdote de los hijos de Joarib, de Jerusalén, y vivió en
Modín. \bibleverse{2} Y tuvo cinco hijos: Juan, que se apellidaba
Gaddis; \bibleverse{3} Simón, que se llamaba Thassi; \bibleverse{4}
Judas, que se llamaba Maccabaeus; \bibleverse{5} Eleazar, que se llamaba
Avaran; y Jonatán, que se llamaba Apphus.

\bibleverse{6} Vio las blasfemias que se cometían en Judá y en
Jerusalén, \bibleverse{7} y dijo: ``¡Ay de mí! ¿Por qué he nacido para
ver la destrucción de mi pueblo y la destrucción de la ciudad santa, y
para habitar en ella cuando ha sido entregada a la mano del enemigo, el
santuario a la mano de los extranjeros? \bibleverse{8} Su templo se ha
vuelto como un hombre glorioso. \bibleverse{9} Sus objetos de gloria han
sido llevados al cautiverio. Sus niños son asesinados en sus calles. Sus
jóvenes son asesinados con la espada del enemigo. \bibleverse{10} ¿Qué
nación no ha heredado sus palacios y se ha apoderado de sus despojos?
\bibleverse{11} Su adorno le ha sido arrebatado. En lugar de ser una
mujer libre, se ha convertido en una esclava. \bibleverse{12} He aquí
que nuestras cosas santas, nuestra belleza y nuestra gloria han sido
asoladas. Los gentiles las han profanado. \bibleverse{13} ¿Por qué hemos
de seguir viviendo?''

\bibleverse{14} Matatías y sus hijos se rasgaron las vestiduras, se
vistieron de cilicio y se lamentaron mucho.

\bibleverse{15} Los oficiales del rey que imponían la apostasía entraron
en la ciudad de Modín para sacrificar. \bibleverse{16} Muchos de Israel
acudieron a ellos, y Matatías y sus hijos estaban reunidos.
\bibleverse{17} Los oficiales del rey respondieron y hablaron con
Matatías, diciendo: ``Tú eres un gobernante y un hombre honorable y
grande en esta ciudad, y fortalecido con hijos y parentela.
\bibleverse{18} Ahora, pues, ven primero y cumple el mandamiento del
rey, como lo han hecho todas las naciones, incluso los hombres de Judá y
los que permanecen en Jerusalén. Tú y tu casa serán contados entre los
amigos del rey, y tú y tus hijos serán honrados con plata y oro y muchos
regalos.''

\bibleverse{19} Respondió Matatías y dijo en voz alta: ``Si todas las
naciones que están en la casa del dominio del rey le escuchan, para
abandonar cada una el culto de sus padres, y han optado por seguir sus
mandamientos, \bibleverse{20} sin embargo, yo y mis hijos y mi parentela
andaremos en el pacto de nuestros padres. \bibleverse{21} Lejos de
nosotros el abandono de la ley y de los preceptos. \bibleverse{22} No
escucharemos las palabras del rey para apartarnos de nuestro culto, ni a
la derecha ni a la izquierda''.

\bibleverse{23} Cuando terminó de decir estas palabras, un judío vino, a
la vista de todos, a sacrificar en el altar que estaba en Modín, según
la orden del rey. \bibleverse{24} Al verlo Matatías, se le encendió el
celo y le temblaron las tripas, y descargó su ira según el juicio, y
corrió a matarlo sobre el altar. \bibleverse{25} Al mismo tiempo mató al
oficial del rey que obligaba a los hombres a sacrificar, y derribó el
altar. \bibleverse{26} Fue celoso de la ley, como lo hizo Finehas con
Zimri, hijo de Salú.

\bibleverse{27} Matatías gritó en la ciudad con gran voz, diciendo:
``¡El que sea celoso de la ley y mantenga el pacto, que me siga!''
\bibleverse{28} Él y sus hijos huyeron a las montañas y dejaron todo lo
que tenían en la ciudad.

\bibleverse{29} Entonces muchos de los que buscaban la justicia y el
juicio descendieron al desierto para vivir allí --- \bibleverse{30}
ellos, sus hijos, sus mujeres y sus ganados --- porque los males se
multiplicaban sobre ellos. \bibleverse{31} Se informó a los oficiales
del rey y a las fuerzas que estaban en Jerusalén, la ciudad de David,
que algunos hombres que habían quebrantado el mandamiento del rey habían
descendido a los lugares secretos del desierto; \bibleverse{32} y muchos
los persiguieron, y habiéndolos alcanzado, acamparon contra ellos y
prepararon la batalla contra ellos en el día de reposo. \bibleverse{33}
Les dijeron: ``¡Basta ya! Salid y haced según la palabra del rey, y
todos viviréis''.

\bibleverse{34} Dijeron: ``No saldremos. No cumpliremos la palabra del
rey de profanar el día de reposo''.

\bibleverse{35} Entonces el enemigo se apresuró a atacarlos.

\bibleverse{36} No les respondieron. No les arrojaron una piedra, ni
bloquearon sus lugares secretos, \bibleverse{37} diciendo: ``Muramos
todos en nuestra inocencia. El cielo y la tierra atestiguan por nosotros
que nos matasteis injustamente''.

\bibleverse{38} Así que los atacaron en sábado, y murieron --- ellos,
sus mujeres, sus hijos y su ganado --- en número de mil almas de
hombres.

\bibleverse{39} Cuando Matatías y sus amigos se enteraron, se lamentaron
mucho por ellos. \bibleverse{40} Uno dijo a otro: ``Si todos hacemos
como nuestros parientes y no luchamos contra los gentiles por nuestras
vidas y nuestras ordenanzas, pronto nos destruirán de la tierra.''
\bibleverse{41} Así que decidieron aquel día, diciendo: ``Cualquiera que
venga contra nosotros a luchar en el día de reposo, luchemos contra él,
y de ninguna manera moriremos todos, como murieron nuestros parientes en
los lugares secretos.''

\bibleverse{42} Entonces se reunió con ellos una compañía de asiduos,
hombres poderosos de Israel, todos los que se ofrecían voluntariamente
por la ley. \bibleverse{43} Todos los que huyeron de los males se
sumaron a ellos, y los apoyaron. \bibleverse{44} Reunieron un ejército,
e hirieron a los pecadores en su cólera, y a los hombres sin ley en su
ira. Los demás huyeron a los gentiles para ponerse a salvo.
\bibleverse{45} Y Matatías y sus amigos fueron y derribaron los altares.
\bibleverse{46} Circuncidaron por la fuerza a los muchachos
incircuncisos, a todos los que encontraron en las costas de Israel.
\bibleverse{47} Persiguieron a los arrogantes, y la obra prosperó en sus
manos. \bibleverse{48} Rescataron la ley de la mano de los gentiles y de
la mano de los reyes. Nunca permitieron que el pecador triunfara.

\bibleverse{49} Se acercaban los días de Matatías para morir, y dijo a
sus hijos: ``Ahora el orgullo y el desprecio han cobrado fuerza. Es una
época de derrocamiento y de ira indignada. \bibleverse{50} Ahora, hijos
míos, sed celosos de la ley y dad la vida por la alianza de vuestros
padres. \bibleverse{51} Recordad las obras de nuestros padres que
hicieron en sus generaciones, y recibid gran gloria y un nombre eterno.
\bibleverse{52} ¿No fue hallado Abraham fiel en la tentación, y le fue
contado por justicia? \bibleverse{53} José, en el tiempo de su angustia,
guardó el mandamiento, y llegó a ser señor de Egipto. \bibleverse{54}
Finees, nuestro padre, por su gran celo, obtuvo el pacto de un
sacerdocio eterno. \bibleverse{55} Josué llegó a ser juez en Israel por
cumplir la palabra. \bibleverse{56} Caleb obtuvo una herencia en la
tierra por testificar en la congregación. \bibleverse{57} David heredó
el trono de un reino para siempre por ser misericordioso.
\bibleverse{58} Elías fue llevado al cielo por ser muy celoso de la ley.
\bibleverse{59} Hananías, Azarías y Misael creyeron y se salvaron de las
llamas. \bibleverse{60} Daniel fue librado de la boca de los leones por
su inocencia.

\bibleverse{61} ``Considera así, de generación en generación, que a
nadie que ponga su confianza en él le faltará fuerza. \bibleverse{62} No
temas las palabras de un hombre pecador, porque su gloria será estiércol
y gusanos. \bibleverse{63} Hoy será levantado, y mañana no será
encontrado, porque ha vuelto al polvo, y su pensamiento ha perecido.
\bibleverse{64} Vosotros, hijos míos, sed fuertes y mostraos como
hombres a favor de la ley, porque en ella obtendréis la gloria.
\bibleverse{65} He aquí a Simón, vuestro hermano, de quien sé que es
hombre de consejo. Escuchadle siempre. Será un padre para vosotros.
\bibleverse{66} Judas Macabeo ha sido fuerte y poderoso desde su
juventud. Él será vuestro capitán y luchará la batalla del pueblo.
\bibleverse{67} Reúne a todos los cumplidores de la ley, y véngate del
mal hecho a tu pueblo. \bibleverse{68} Repara a los gentiles y obedece
los mandamientos de la ley''.

\bibleverse{69} Los bendijo y se reunió con sus antepasados.
\bibleverse{70} Murió en el año ciento cuarenta y seis, y sus hijos lo
enterraron en las tumbas de sus antepasados en Modín. Todo Israel hizo
grandes lamentaciones por él.

\hypertarget{section-2}{%
\section{3}\label{section-2}}

\bibleverse{1} Su hijo Judas, llamado Macabeo, se levantó en su lugar.
\bibleverse{2} Toda su parentela le ayudó, y también todos los que se
unieron a su padre, y combatieron con alegría la batalla de Israel.
\bibleverse{3} Consiguió para su pueblo una gran gloria, se puso una
coraza como la de un gigante, se ató los arreos de guerra y puso en
orden las batallas, protegiendo al ejército con su espada.
\bibleverse{4} Era como un león en sus acciones, y como un cachorro de
león que ruge por la presa. \bibleverse{5} Cazaba y perseguía a los
infractores, y quemaba a los que perturbaban a su pueblo. \bibleverse{6}
Los injustos retrocedieron por temor a él, y todos los obreros de la
iniquidad se vieron muy perturbados, y la liberación prosperó en su
mano. \bibleverse{7} Enfureció a muchos reyes y alegró a Jacob con sus
actos. Su memoria es bendita para siempre. \bibleverse{8} Recorrió las
ciudades de Judá, destruyó del país a los impíos y alejó la ira de
Israel. \bibleverse{9} Tuvo fama hasta el último rincón de la tierra.
Reunió a los que estaban dispuestos a perecer.

\bibleverse{10} Apolonio reunió a los gentiles con un gran ejército de
Samaria para luchar contra Israel. \bibleverse{11} Judas se enteró y
salió a su encuentro, lo golpeó y lo mató. Muchos cayeron heridos de
muerte, y los demás huyeron. \bibleverse{12} Tomaron su botín, y Judas
tomó la espada de Apolonio, y luchó con ella todos sus días.

\bibleverse{13} Serón, el comandante del ejército de Siria, se enteró de
que Judas había reunido una gran compañía, incluyendo un cuerpo de
hombres fieles que se quedaron con él, salió a la guerra.
\bibleverse{14} Dijo: ``Me haré un nombre y conseguiré gloria en el
reino. Lucharé contra Judas y los que están con él, que desprecian el
mandato del rey. \bibleverse{15} Un poderoso ejército de impíos subió
con él para ayudarle, para vengarse de los hijos de Israel.

\bibleverse{16} Se acercó a la subida de Bethoron, y Judas salió a su
encuentro con una pequeña compañía. \bibleverse{17} Pero al ver que el
ejército les salía al encuentro, dijeron a Judas: ``¿Qué? ¿Podremos,
siendo una pequeña compañía, luchar contra una multitud tan grande y
fuerte? Nosotros, por nuestra parte, estamos desfallecidos, pues no
hemos probado alimento en este día''.

\bibleverse{18} Judas dijo: ``Es cosa fácil que muchos sean acorralados
por las manos de unos pocos. Con el cielo es todo uno, para salvar por
muchos o por pocos; \bibleverse{19} porque la victoria en la batalla no
está en la multitud de un ejército, sino que la fuerza viene del cielo.
\bibleverse{20} Vienen a nosotros en plena insolencia y anarquía, para
destruirnos a nosotros y a nuestras esposas e hijos, y para saquearnos,
\bibleverse{21} pero nosotros luchamos por nuestras vidas y nuestras
leyes. \bibleverse{22} Él mismo los aplastará ante nuestra cara; pero en
cuanto a vosotros, no les tengáis miedo.

\bibleverse{23} Cuando terminó de hablar, se abalanzó repentinamente
contra Serón y su ejército, y fueron derrotados ante él. \bibleverse{24}
Los persiguieron por la bajada de Bethorón hasta la llanura, y cayeron
unos ochocientos hombres de ellos; pero los demás huyeron al país de los
filisteos.

\bibleverse{25} El temor a Judas y a su parentela, y el miedo a ellos,
comenzó a caer sobre las naciones de alrededor. \bibleverse{26} Su fama
llegó hasta el rey, y todas las naciones contaban las batallas de Judas.

\bibleverse{27} Pero cuando el rey Antíoco oyó estas palabras, se llenó
de indignación, y envió y reunió todas las fuerzas de su reino, un
ejército sumamente fuerte. \bibleverse{28} Abrió su tesorería y dio a
sus fuerzas la paga de un año, y les ordenó que estuvieran listas para
cualquier necesidad. \bibleverse{29} Vio que el dinero había
desaparecido de sus tesoros y que los tributos del país eran escasos, a
causa de la disensión y el desastre que había provocado en el país, con
el fin de quitar las leyes que habían sido desde los primeros días.
\bibleverse{30} Tenía miedo de no tener suficiente como en otros tiempos
para los cargos y los regalos que solía dar con mano liberal, más
abundantemente que los reyes que le precedieron. \bibleverse{31} Y se
quedó muy perplejo en su mente, y determinó ir a Persia y tomar los
tributos de esos países, y reunir mucho dinero. \bibleverse{32} Dejó a
Lisias, hombre honorable y de linaje real, para que se ocupara de los
asuntos del rey desde el río Éufrates hasta los límites de Egipto,
\bibleverse{33} y para que educara a su hijo Antíoco, hasta que
volviera. \bibleverse{34} Entregó a Lisias la mitad de sus fuerzas y los
elefantes, y le encargó todo lo que quería hacer, y en cuanto a los que
vivían en Judea y en Jerusalén, \bibleverse{35} que enviara un ejército
contra ellos para desarraigar y destruir la fuerza de Israel y el resto
de Jerusalén, y para quitar su memoria del lugar, \bibleverse{36} y que
hiciera vivir a los extranjeros en todo su territorio, y les repartiera
su tierra por sorteo. \bibleverse{37} El rey tomó la mitad que quedaba
de las fuerzas y salió de Antioquía, su ciudad real, en el año ciento
cuarenta y siete; y pasó el río Éufrates y atravesó las tierras altas.

\bibleverse{38} Lisias eligió a Tolomeo hijo de Dorimenes, a Nicanor y a
Gorgias, hombres poderosos de los amigos del rey; \bibleverse{39} y con
ellos envió cuarenta mil soldados de infantería y siete mil de
caballería para ir a la tierra de Judá y destruirla, según la palabra
del rey. \bibleverse{40} Partieron con todo su ejército, y llegaron y
acamparon cerca de Emaús, en la llanura. \bibleverse{41} Los mercaderes
del país se enteraron de su fama, y tomaron plata y oro en grandes
cantidades, y grilletes, y entraron en el campamento para tomar a los
hijos de Israel como esclavos. Fuerzas de Siria y del país de los
filisteos se unieron a ellos.

\bibleverse{42} Judas y su parentela vieron que los males se
multiplicaban y que las fuerzas acampaban en sus fronteras. Se enteraron
de las palabras del rey que había ordenado destruir al pueblo y acabar
con él. \bibleverse{43} Entonces cada uno dijo a su vecino: ``Reparemos
las ruinas de nuestro pueblo. Luchemos por nuestro pueblo y por el lugar
santo''. \bibleverse{44} La congregación se reunió para estar preparada
para la batalla, y para orar y pedir misericordia y compasión.
\bibleverse{45} Jerusalén estaba sin habitantes como un desierto. No
había ninguno de sus descendientes que entrara o saliera. El santuario
estaba pisoteado. Los hijos de los extranjeros estaban en la ciudadela.
Los gentiles vivían allí. La alegría fue quitada a Jacob, y la flauta y
el arpa cesaron. \bibleverse{46} Se reunieron y vinieron a Mizpa, cerca
de Jerusalén, pues en Mizpa solía haber un lugar de oración para Israel.
\bibleverse{47} Aquel día ayunaron, se vistieron de saco, se pusieron
ceniza en la cabeza, se rasgaron las vestiduras, \bibleverse{48} y
abrieron el libro de la ley, para enterarse de las cosas por las que los
gentiles consultaban las imágenes de sus ídolos. \bibleverse{49}
Trajeron las vestimentas de los sacerdotes, las primicias y los diezmos.
Incitaron a los nazareos, que habían cumplido sus días. \bibleverse{50}
Pusieron el grito en el cielo, diciendo: ``¿Qué debemos hacer con estos
hombres? ¿Adónde debemos llevarlos? \bibleverse{51} Tu lugar santo está
pisoteado y profanado. Tus sacerdotes se lamentan con humillación.
\bibleverse{52} He aquí que los gentiles se han reunido contra nosotros
para destruirnos. Tú sabes qué cosas imaginan contra nosotros.
\bibleverse{53} ¿Cómo podremos enfrentarnos a ellos, si no nos ayudas?''
\bibleverse{54} Tocaron las trompetas y dieron un fuerte grito.

\bibleverse{55} Después de esto, Judas nombró jefes del pueblo:
capitanes de millares, capitanes de centenas, capitanes de cincuenta y
capitanes de diez. \bibleverse{56} Dijo a los que estaban construyendo
casas, desposando mujeres, plantando viñas y temiendo, que se volvieran,
cada uno a su casa, según la ley. \bibleverse{57} El ejército salió y
acampó en el lado sur de Emaús. \bibleverse{58} Judas dijo: ``¡Armaos y
sed hombres valientes! Estad preparados por la mañana para luchar contra
estos gentiles que se han reunido contra nosotros para destruirnos a
nosotros y a nuestro lugar santo. \bibleverse{59} Porque es mejor para
nosotros morir en la batalla que ver las calamidades de nuestra nación y
del lugar santo. \bibleverse{60} Sin embargo, como sea la voluntad en el
cielo, así se hará.

\hypertarget{section-3}{%
\section{4}\label{section-3}}

\bibleverse{1} Gorgias tomó cinco mil soldados de infantería y mil de
caballería elegidos, y el ejército salió de noche, \bibleverse{2} para
poder caer sobre el ejército de los judíos y golpearlos de repente. Los
hombres de la ciudadela fueron sus guías. \bibleverse{3} Judas se enteró
de esto, y él y los hombres valientes se movieron para poder golpear al
ejército del rey que estaba en Emaús \bibleverse{4} mientras las fuerzas
aún estaban dispersas del campamento. \bibleverse{5} Gorgias entró de
noche en el campamento de Judas y no encontró a nadie. Los buscó en las
montañas, pues dijo: ``Estos hombres huyen de nosotros''.

\bibleverse{6} Tan pronto como se hizo de día, Judas apareció en la
llanura con tres mil hombres. Sin embargo, no tenían la armadura ni las
espadas que deseaban. \bibleverse{7} Vieron el campamento de los
gentiles fuerte y fortificado, con caballería a su alrededor; y éstos
eran expertos en la guerra. \bibleverse{8} Judas dijo a los hombres que
estaban con él: ``No tengáis miedo de su número, ni de cuando carguen.
\bibleverse{9} Recordad cómo se salvaron nuestros padres en el mar Rojo,
cuando el faraón los persiguió con un ejército. \bibleverse{10} Ahora
clamemos al cielo, si es que nos acepta, y se acuerda del pacto de
nuestros padres, y destruye hoy este ejército ante nuestra vista.
\bibleverse{11} Entonces todos los gentiles sabrán que hay uno que
redime y salva a Israel.

\bibleverse{12} Los extranjeros alzaron sus ojos y los vieron acercarse
a ellos. \bibleverse{13} Salieron del campamento para combatir. Los que
estaban con Judas tocaron sus trompetas \bibleverse{14} y se unieron a
la batalla. Los gentiles fueron derrotados y huyeron a la llanura.
\bibleverse{15} Pero todos los de la retaguardia cayeron a espada. Los
persiguieron hasta Gazara, y hasta las llanuras de Idumea, Azoto y
Jamnia. Cayeron unos tres mil de esos hombres. \bibleverse{16} Entonces
Judas y su ejército volvieron de perseguirlos; \bibleverse{17} y dijo al
pueblo: ``No seáis codiciosos del botín, porque tenemos una batalla por
delante. \bibleverse{18} Gorgias y su ejército están cerca de nosotros
en la montaña. Pero enfréntate ahora a nuestros enemigos y lucha contra
ellos, y después toma el botín con audacia.'' \bibleverse{19} Mientras
Judas terminaba este discurso, una parte de ellos apareció mirando desde
la montaña. \bibleverse{20} Vieron que su ejército había sido puesto en
fuga y que los judíos estaban quemando el campamento, pues el humo que
se veía declaraba lo que se había hecho. \bibleverse{21} Pero al
percibir estas cosas, tuvieron mucho miedo. Percibiendo también el
ejército de Judas en la llanura, listo para la batalla, \bibleverse{22}
todos huyeron a la tierra de los filisteos. \bibleverse{23} Judas volvió
a saquear el campamento, y tomaron mucho oro, plata, azul, púrpura
marina y grandes riquezas. \bibleverse{24} Luego volvieron a su casa y
entonaron un cántico de acción de gracias, y alabaron al cielo, porque
él es bueno, porque su misericordia es eterna. \bibleverse{25} Ese día
Israel tuvo una gran liberación.

\bibleverse{26} Los extranjeros que habían escapado vinieron y contaron
a Lisias todo lo que había sucedido. \bibleverse{27} Cuando se enteró,
se sintió confundido y desanimado, porque no se habían hecho las cosas
que él deseaba para Israel, ni habían sucedido las cosas que el rey le
había ordenado.

\bibleverse{28} Al año siguiente, reunió sesenta mil soldados de
infantería escogidos y cinco mil de caballería, para someterlos.
\bibleverse{29} Llegaron a Idumea y acamparon en Betsura. Judas les
salió al encuentro con diez mil hombres. \bibleverse{30} Al ver que el
ejército era fuerte, oró y dijo: ``Bendito seas, oh Salvador de Israel,
que derrotaste el ataque del poderoso guerrero por la mano de tu siervo
David, y entregaste el ejército de los filisteos en manos de Jonatán,
hijo de Saúl, y de su portador de armas. \bibleverse{31} Pon este
ejército en manos de tu pueblo Israel, y que se avergüence de su
ejército y de su caballería. \bibleverse{32} Haz que su corazón
desfallezca. Haz que se desvanezca la audacia de su fuerza, y que
tiemblen ante su destrucción. \bibleverse{33} Derríbalos con la espada
de los que te aman, y que todos los que conocen tu nombre te alaben con
acción de gracias.''

\bibleverse{34} Se unieron en la batalla, y cayeron unos cinco mil
hombres del ejército de Lisias. Cayeron cerca de ellos. \bibleverse{35}
Pero cuando Lisias vio que sus tropas habían sido puestas en fuga, y la
audacia que se había apoderado de los que estaban con Judas, y cómo
estaban dispuestos a vivir o a morir noblemente, se retiró a Antioquía y
reunió soldados a sueldo, para volver a entrar en Judea con un ejército
aún mayor.

\bibleverse{36} Pero Judas y los suyos dijeron: ``He aquí que nuestros
enemigos han sido derrotados. Subamos a limpiar el lugar santo y a
rededificarlo''. \bibleverse{37} Se reunió todo el ejército y subieron
al monte Sión. \bibleverse{38} Vieron el santuario desolado, el altar
profanado, las puertas quemadas, los arbustos que crecían en los atrios
como en un bosque o como en uno de los montes, y las cámaras de los
sacerdotes derribadas; \bibleverse{39} y se rasgaron las vestiduras,
hicieron grandes lamentaciones, pusieron ceniza sobre sus cabezas,
\bibleverse{40} se postraron en tierra, tocaron las trompetas solemnes,
y clamaron hacia el cielo. \bibleverse{41} Entonces Judas designó a
algunos hombres para que lucharan contra los que estaban en la ciudadela
hasta que él hubiera limpiado el lugar santo.

\bibleverse{42} Escogió a sacerdotes intachables y devotos de la ley;
\bibleverse{43} y limpiaron el lugar santo y sacaron las piedras
profanadas a un lugar impuro. \bibleverse{44} Deliberaron sobre qué
hacer con el altar de los holocaustos, que había sido profanado.
\bibleverse{45} Se les ocurrió un buen plan: derribarlo, para que no les
sirviera de oprobio, porque los gentiles lo habían profanado. Así que
derribaron el altar \bibleverse{46} y colocaron las piedras en el monte
del templo en un lugar conveniente, hasta que viniera un profeta a dar
una respuesta sobre ellas. \bibleverse{47} Tomaron piedras enteras según
la ley, y construyeron un nuevo altar como el anterior. \bibleverse{48}
Edificaron el lugar santo y las partes interiores de la casa, y
consagraron los atrios. \bibleverse{49} Hicieron nuevos vasos sagrados,
y llevaron al templo el candelabro, el altar del incienso y la mesa.
\bibleverse{50} Quemaron incienso sobre el altar, y encendieron las
lámparas que estaban sobre el candelabro, y alumbraron el templo.
\bibleverse{51} Pusieron panes sobre la mesa, colgaron las cortinas y
terminaron toda la obra que habían hecho.

\bibleverse{52} Se levantaron de madrugada, el día veinticinco del
noveno mes, que es el mes de Chislev, en el año ciento cuarenta y ocho,
\bibleverse{53} y ofrecieron sacrificios según la ley en el nuevo altar
de los holocaustos que habían hecho. \bibleverse{54} A la hora y en el
día en que los gentiles lo habían profanado, fue dedicado con cantos,
arpas, laúdes y címbalos. \bibleverse{55} Todo el pueblo se postró sobre
sus rostros, adoró y alabó al cielo, que les había dado buen éxito.
\bibleverse{56} Celebraron la dedicación del altar durante ocho días, y
ofrecieron holocaustos con alegría, y sacrificaron un sacrificio de
liberación y de alabanza. \bibleverse{57} Adornaron la fachada del
templo con coronas de oro y pequeños escudos. Dedicaron las puertas y
las cámaras de los sacerdotes, y les hicieron puertas. \bibleverse{58}
Hubo una alegría muy grande en el pueblo, y se apartó el oprobio de los
gentiles.

\bibleverse{59} Judas y su parentela y toda la congregación de Israel
ordenaron que los días de la dedicación del altar se celebraran en sus
estaciones de año en año durante ocho días, a partir del día veinticinco
del mes de Chislev, con alegría y gozo.

\bibleverse{60} En aquel tiempo, fortificaron el monte de Sión con altos
muros y fuertes torres a su alrededor, para que no vinieran los gentiles
y los pisotearan, como habían hecho antes. \bibleverse{61} Judas puso
una guarnición para custodiarla. Fortificaron Betsura para guardarla, a
fin de que el pueblo tuviera una fortaleza cerca de Idumea.

\hypertarget{section-4}{%
\section{5}\label{section-4}}

\bibleverse{1} Sucedió que cuando los gentiles de alrededor oyeron que
el altar había sido reconstruido y el santuario dedicado como antes, se
enojaron mucho. \bibleverse{2} Tomaron consejo para destruir la raza de
Jacob que estaba en medio de ellos, y comenzaron a matar y a destruir en
medio del pueblo. \bibleverse{3} Judas luchó contra los hijos de Esaú en
Idumea, en Akrabattine, porque asediaban a Israel. Los golpeó con una
gran matanza, los humilló y tomó sus despojos. \bibleverse{4} Se acordó
de la maldad de los hijos de Baean, que eran una trampa y un tropiezo
para el pueblo, acechándolos en los caminos. \bibleverse{5} Fueron
encerrados por él en las torres. Acampó contra ellos y los destruyó por
completo, y quemó con fuego las torres del lugar con todos los que
estaban en ellas. \bibleverse{6} Pasó a los hijos de Amón, y encontró
una poderosa banda y mucha gente, con Timoteo como jefe. \bibleverse{7}
Libró muchas batallas con ellos, y fueron derrotados ante su rostro. Los
golpeó, \bibleverse{8} y se apoderó de Jazer y sus aldeas, y regresó de
nuevo a Judea.

\bibleverse{9} Los gentiles que estaban en Galaad se reunieron contra
los israelitas que estaban en sus fronteras, para destruirlos. Huyeron a
la fortaleza de Datema, \bibleverse{10} y enviaron cartas a Judas y a su
parentela, diciendo: ``Los gentiles que están alrededor nuestro se han
reunido contra nosotros para destruirnos. \bibleverse{11} Se preparan
para venir a tomar posesión de la fortaleza donde nos refugiamos, y
Timoteo es el jefe de su ejército. \bibleverse{12} Ahora, pues, ven y
líbranos de su mano, porque muchos de nosotros hemos caído.
\bibleverse{13} Todos nuestros parientes que estaban en la tierra de
Tubias han sido ejecutados. Han llevado al cautiverio a sus esposas, a
sus hijos y a sus cosas. Han destruido allí a unos mil hombres''.

\bibleverse{14} Mientras aún se leían las cartas, he aquí que otros
mensajeros vinieron de Galilea con sus ropas rasgadas, trayendo un
informe similar, \bibleverse{15} diciendo: ``Gente de Tolemaida, de
Tiro, de Sidón y de toda la Galilea de los gentiles se han reunido para
destruirnos.''

\bibleverse{16} Cuando Judas y el pueblo escucharon estas palabras, se
reunió una gran congregación para determinar qué debían hacer por sus
parientes que estaban en peligro y bajo ataque. \bibleverse{17} Judas
dijo a su hermano Simón: ``Escoge hombres y ve a socorrer a tus
parientes que están en Galilea, pero Jonatán, mi hermano, y yo iremos a
la tierra de Galaad.'' \bibleverse{18} Dejó a José, hijo de Zacarías, y
a Azarías, como jefes del pueblo, con el resto del ejército, en Judea,
para que la custodiaran. \bibleverse{19} Les ordenó diciendo:
``Encárguense de este pueblo y no peleen con los gentiles hasta que
regresemos.'' \bibleverse{20} Entonces se asignaron tres mil hombres
para ir a Galilea con Simón, pero se asignaron ocho mil hombres a Judas
para ir a la tierra de Galaad.

\bibleverse{21} Simón fue a Galilea y libró muchas batallas con los
gentiles, y éstos fueron derrotados ante él. \bibleverse{22} Los
persiguió hasta la puerta de Tolemaida. Cayeron unos tres mil hombres de
los gentiles, y él tomó su botín. \bibleverse{23} Se llevó a los que
estaban en Galilea y en Arbatta, con sus mujeres, sus hijos y todo lo
que tenían, y los llevó a Judea con gran alegría. \bibleverse{24} Judas
Macabeo y su hermano Jonatán pasaron el Jordán y recorrieron tres días
de camino en el desierto. \bibleverse{25} Se encontraron con los
nabateos, y éstos los recibieron de manera pacífica y les contaron todo
lo que les había sucedido a sus parientes en la tierra de Galaad,
\bibleverse{26} y cómo muchos de ellos estaban encerrados en Bosora,
Bosor, Alema, Casphor, Maked y Carnaim --- todas estas ciudades son
fuertes y grandes --- \bibleverse{27} y cómo estaban encerrados en el
resto de las ciudades de la tierra de Galaad, y que mañana planeaban
acampar contra las fortalezas y tomarlas, y destruir a todos estos
hombres en un solo día.

\bibleverse{28} Judas y su ejército se desviaron repentinamente por el
camino del desierto hacia Bosora; tomó la ciudad, mató a todos los
varones a filo de espada, tomó todos sus despojos y quemó la ciudad.
\bibleverse{29} Salió de allí por la noche, y fue hasta llegar a la
fortaleza. \bibleverse{30} Cuando llegó la mañana, alzó los ojos y vio a
mucha gente que no se podía contar, llevando escaleras y máquinas de
guerra, para tomar la fortaleza; y estaban luchando contra ellos.
\bibleverse{31} Judas vio que la batalla había comenzado y que el grito
de la ciudad subía al cielo, con trompetas y un gran estruendo,
\bibleverse{32} y dijo a los hombres de su ejército: ``¡Luchad hoy por
vuestra parentela!'' \bibleverse{33} Entonces salió detrás de ellos en
tres compañías. Tocaron con sus trompetas y gritaron en oración.
\bibleverse{34} El ejército de Timoteo se dio cuenta de que era Macabeo,
y huyeron ante él. Los atacó con una gran matanza. Aquel día cayeron
unos ocho mil hombres de ellos.

\bibleverse{35} Se dirigió a Mizpa y luchó contra ella, la tomó, mató a
todos sus varones, tomó sus despojos y la quemó. \bibleverse{36} Desde
allí marchó y tomó Casphor, Maked, Bosor y las demás ciudades del país
de Galaad.

\bibleverse{37} Después de esto, Timoteo reunió otro ejército y acampó
cerca de Raphon, al otro lado del arroyo. \bibleverse{38} Judas envió
hombres a espiar al ejército, y le comunicaron lo siguiente: ``Todos los
gentiles que nos rodean se han reunido con ellos, un ejército muy
numeroso. \bibleverse{39} Han contratado a los árabes para que les
ayuden, y están acampados al otro lado del arroyo, dispuestos a venir
contra vosotros a la batalla.'' Así que Judas salió a su encuentro.

\bibleverse{40} Timoteo dijo a los capitanes de su ejército cuando Judas
y su ejército se acercaron al arroyo de agua: ``Si él cruza hacia
nosotros primero, no podremos resistirlo, pues ciertamente nos vencerá;
\bibleverse{41} pero si tiene miedo y acampa más allá del río,
cruzaremos hacia él y lo venceremos.'' \bibleverse{42} Cuando Judas se
acercó al arroyo, hizo que los escribas del pueblo se quedaran junto al
arroyo y les ordenó: ``Que nadie acampe, sino que todos vengan a la
batalla.'' \bibleverse{43} Entonces cruzó el primero contra ellos, y
todo el pueblo tras él; y todos los gentiles fueron derrotados ante su
rostro, y arrojaron sus armas y huyeron al templo de Carnaim.
\bibleverse{44} Tomaron la ciudad y quemaron el templo con fuego, junto
con todos los que estaban en él. Carnaim fue sometida. No pudieron
resistir más ante el rostro de Judas.

\bibleverse{45} Judas reunió a todo Israel, a los que estaban en la
tierra de Galaad, desde el más pequeño hasta el más grande, con sus
mujeres, sus hijos y sus enseres, un ejército muy numeroso, para entrar
en la tierra de Judá. \bibleverse{46} Llegaron hasta Efrón, y esta misma
ciudad era grande y muy fuerte. Estaba en el camino por donde iban. No
podían apartarse de ella ni a la derecha ni a la izquierda, sino que
tenían que pasar por el medio. \bibleverse{47} Los habitantes de la
ciudad les cerraron el paso y bloquearon las puertas con piedras.
\bibleverse{48} Judas les envió palabras de paz, diciendo: ``Pasaremos
por vuestra tierra para ir a la nuestra, y nadie os hará daño. Sólo
pasaremos de pie''. Pero no le abrieron. \bibleverse{49} Entonces Judas
ordenó que se proclamara en el ejército, que cada uno acampara en el
lugar donde estaba. \bibleverse{50} Así que los hombres del ejército
acamparon y lucharon contra la ciudad todo aquel día y toda aquella
noche, y la ciudad fue entregada en sus manos. \bibleverse{51} Destruyó
a todos los varones a filo de espada, arrasó la ciudad, tomó su botín y
pasó por la ciudad sobre los muertos. \bibleverse{52} Pasaron el Jordán
a la gran llanura cercana a Bet-sán. \bibleverse{53} Judas reunió a los
rezagados y animó al pueblo durante todo el camino, hasta llegar a la
tierra de Judá. \bibleverse{54} Subieron al monte de Sión con alegría y
regocijo, y ofrecieron holocaustos enteros, porque no había muerto ni
uno solo de ellos hasta que regresaron en paz.

\bibleverse{55} En los días en que Judas y Jonatán estaban en la tierra
de Galaad, y Simón, su hermano, en Galilea, ante Tolemaida,
\bibleverse{56} José, hijo de Zacarías, y Azarías, jefes del ejército,
oyeron hablar de sus hazañas y de la guerra, y de las cosas que habían
hecho. \bibleverse{57} Dijeron: ``Consigamos también nosotros un nombre,
y vayamos a luchar contra los gentiles que nos rodean.'' \bibleverse{58}
Así que dieron órdenes a los hombres del ejército que estaba con ellos,
y se dirigieron hacia Jamnia. \bibleverse{59} Gorgias y sus hombres
salieron de la ciudad para enfrentarlos en la batalla. \bibleverse{60}
José y Azarías fueron puestos en fuga, y fueron perseguidos hasta las
fronteras de Judea. Aquel día cayeron unos dos mil hombres de Israel.
\bibleverse{61} Hubo un gran descalabro en el pueblo, porque no
escucharon a Judas y a su parentela, pensando en hacer alguna hazaña.
\bibleverse{62} Pero ellos no eran de la familia de aquellos hombres por
cuya mano se dio la liberación a Israel.

\bibleverse{63} Aquel hombre, Judas, y su parentela, fueron glorificados
en extremo a los ojos de todo Israel y de todos los gentiles,
dondequiera que se oyera su nombre. \bibleverse{64} Los hombres se
reunían con ellos, aclamándolos.

\bibleverse{65} Judas y su parentela salieron a luchar contra los hijos
de Esaú en la tierra hacia el sur. Golpeó a Hebrón y a sus aldeas,
derribó sus fortalezas y quemó sus torres por todas partes.
\bibleverse{66} Marchó para entrar en la tierra de los filisteos, y pasó
por Samaria. \bibleverse{67} Aquel día, algunos sacerdotes que querían
hacer hazañas allí, fueron muertos en la batalla, cuando salieron a
combatir imprudentemente. \bibleverse{68} Pero Judas se volvió hacia
Azoto, a la tierra de los filisteos, derribó sus altares, quemó con
fuego las imágenes talladas de sus dioses, tomó el botín de sus ciudades
y regresó a la tierra de Judá.

\hypertarget{section-5}{%
\section{6}\label{section-5}}

\bibleverse{1} El rey Antíoco estaba viajando por los países altos; y
oyó que en Elymais, en Persia, había una ciudad famosa por sus riquezas,
por la plata y el oro, \bibleverse{2} y que el templo que había en ella
era sumamente rico, y que en él había escudos de oro, corazas y armas
que Alejandro, hijo de Filipo, el rey macedonio, que reinó primero entre
los griegos, dejó allí. \bibleverse{3} Llegó, pues, y trató de tomar la
ciudad y de saquearla; pero no pudo, porque su plan era conocido por los
de la ciudad, \bibleverse{4} y se levantaron contra él en la batalla.
Huyó y regresó a Babilonia con gran decepción.

\bibleverse{5} Entonces llegó alguien a Persia trayéndole noticias de
que los ejércitos que iban contra la tierra de Judá habían sido puestos
en fuga, \bibleverse{6} y que Lisias fue el primero con un fuerte
ejército y fue avergonzado ante ellos, y que se habían hecho fuertes
gracias a las armas, el poder y el suministro de botín que tomaron de
los ejércitos que habían eliminado, \bibleverse{7} y que habían
derribado la abominación que él había construido sobre el altar que
estaba en Jerusalén, y que habían rodeado el santuario con altos muros,
como antes, y también Betsura, su ciudad.

\bibleverse{8} Sucedió que cuando el rey escuchó estas palabras, se
asombró y se conmovió mucho. Se acostó en su cama y cayó enfermo de
dolor, porque no le había salido como había planeado. \bibleverse{9}
Estuvo allí muchos días, porque una gran pena se apoderaba de él
continuamente, y se dio cuenta de que iba a morir. \bibleverse{10} Llamó
a todos sus amigos de y les dijo: ``El sueño se va de mis ojos, y mi
corazón desfallece por la preocupación. \bibleverse{11} Dije en mi
corazón: `¡A qué sufrimiento he llegado! ¡Qué grande es el diluvio en
que me encuentro ahora! Pues yo era bondadoso y amado en mi poder'.
\bibleverse{12} Pero ahora me acuerdo de los males que hice en
Jerusalén, y de que tomé todos los objetos de plata y oro que había en
ella, y envié a destruir a los habitantes de Judá sin causa alguna.
\bibleverse{13} Me doy cuenta de que es por esto que me han sobrevenido
estos males. He aquí que estoy pereciendo con gran dolor en tierra
extraña''.

\bibleverse{14} Entonces llamó a Filipo, uno de sus amigos de, y lo puso
al frente de todo su reino. \bibleverse{15} Le dio su corona, su manto y
su anillo de sello, para que guiara a su hijo Antíoco y lo alimentara
para que fuera rey. \bibleverse{16} El rey Antíoco murió allí en el año
ciento cuarenta y nueve. \bibleverse{17} Cuando Lisias se enteró de que
el rey había muerto, puso a reinar a su hijo Antíoco, al que había
alimentado siendo joven, y lo llamó Eupator.

\bibleverse{18} Los que estaban en la ciudadela seguían acorralando a
Israel en torno al santuario, y siempre trataban de perjudicarlos y de
fortalecer a los gentiles. \bibleverse{19} Judas planeó destruirlos, y
convocó a todo el pueblo para asediarlos. \bibleverse{20} Se reunieron y
los sitiaron en el año ciento cincuenta, y él hizo montículos para
disparar y máquinas de guerra. \bibleverse{21} Algunos de los sitiados
salieron, y algunos de los impíos de Israel se unieron a ellos.
\bibleverse{22} Se dirigieron al rey y le dijeron: ``¿Hasta cuándo no
harás juicio y vengarás a nuestra parentela? \bibleverse{23} Estábamos
dispuestos a servir a tu padre, a vivir según sus palabras y a seguir
sus mandamientos. \bibleverse{24} A causa de esto, los hijos de nuestro
pueblo asediaron la ciudadela y se alejaron de nosotros; pero a cuantos
de nosotros pudieron atrapar, los mataron, y saquearon nuestras
herencias. \bibleverse{25} No sólo contra nosotros extendieron su mano,
sino también contra todas sus fronteras. \bibleverse{26} He aquí que hoy
acampan contra la ciudadela de Jerusalén para tomarla. Han fortificado
el santuario y Betsura. \bibleverse{27} Si no te apresuras a impedirlo,
harán cosas mayores que éstas, y no podrás controlarlas.

\bibleverse{28} Al oír esto, el rey se enfureció y reunió a todos sus
amigos de, a los jefes de su ejército y a los que estaban al mando de la
caballería. \bibleverse{29} Vinieron a él bandas de soldados contratados
de otros reinos y de las islas del mar. \bibleverse{30} El número de sus
fuerzas era de cien mil soldados de infantería, veinte mil de caballería
y treinta y dos elefantes entrenados para la guerra. \bibleverse{31}
Atravesaron Idumea y acamparon frente a Betsura, contra la que
combatieron muchos días y fabricaron máquinas de guerra. Los judíos
salieron y los quemaron con fuego, y lucharon valientemente.

\bibleverse{32} Judas se alejó de la ciudadela y acampó en Betzacarías,
cerca del campamento del rey. \bibleverse{33} El rey se levantó de
madrugada y puso en marcha su ejército a toda velocidad por el camino de
Betzacarías. Sus fuerzas se prepararon para la batalla y tocaron las
trompetas. \bibleverse{34} Ofrecieron a los elefantes el jugo de las
uvas y de las moras, a fin de prepararlos para la batalla.
\bibleverse{35} Distribuyeron los animales entre las falanges. Pusieron
junto a cada elefante mil hombres armados con cota de malla y cascos de
bronce en la cabeza. Para cada elefante se designaron quinientos hombres
de caballería elegidos. \bibleverse{36} Estos estaban preparados de
antemano, dondequiera que estuviera el elefante. Dondequiera que fuera
el elefante, iban con él. No lo dejaban. \bibleverse{37} Sobre ellos
había fuertes torres de madera cubiertas, una sobre cada elefante,
sujetas a él con arneses seguros. Sobre cada uno había cuatro hombres
valientes que luchaban sobre ellos, junto a su conductor indio.
\bibleverse{38} El resto de la caballería la colocó a un lado y a otro
en los dos flancos del ejército, infundiendo terror al enemigo y
protegidos por las falanges. \bibleverse{39} Cuando el sol brillaba
sobre los escudos de oro y bronce, las montañas se iluminaban y ardían
como antorchas.

\bibleverse{40} Una parte del ejército del rey estaba extendida sobre
las colinas altas y otra sobre el terreno bajo, y avanzaban con firmeza
y en orden. \bibleverse{41} Todos los que oían el ruido de la multitud,
el paso de la gente y el ruido de las armas, temblaban, porque el
ejército era muy grande y fuerte. \bibleverse{42} Judas y su ejército se
acercaron a la batalla, y cayeron seiscientos hombres del ejército del
rey. \bibleverse{43} Eleazar, que se llamaba Avaran, vio uno de los
animales armados con corazas reales, y era más alto que todos los
animales, y parecía que el rey estaba sobre él. \bibleverse{44} Dio su
vida para liberar a su pueblo y conseguir un nombre eterno.
\bibleverse{45} Corrió sobre él valientemente en medio de la falange, y
mató a derecha e izquierda, y se separaron de él a un lado y a otro.
\bibleverse{46} Se arrastró bajo el elefante, lo apuñaló desde abajo y
lo mató. El elefante cayó a tierra sobre él, y allí murió.
\bibleverse{47} Al ver la fuerza del reino y el feroz ataque del
ejército, se apartaron de ellos.

\bibleverse{48} Pero los soldados del ejército del rey subieron a
Jerusalén a recibirlos, y el rey acampó hacia Judea y hacia el monte
Sión. \bibleverse{49} Hizo la paz con el pueblo de Betsura. Salió de la
ciudad porque allí no tenían comida para soportar el asedio, porque era
un día de reposo para la tierra. \bibleverse{50} El rey tomó Betsura y
designó allí una guarnición para guardarla. \bibleverse{51} Acampó
contra el santuario muchos días, y puso allí montículos para disparar, y
máquinas de guerra, y máquinas para lanzar fuego y piedras, y armas para
lanzar dardos y hondas. \bibleverse{52} También los judíos hicieron
máquinas de guerra contra sus máquinas, y lucharon durante muchos días.
\bibleverse{53} Pero no había comida en el santuario, porque era el
séptimo año, y los que habían huido a Judea de entre los gentiles para
ponerse a salvo habían consumido el resto de las provisiones.
\bibleverse{54} Quedaron pocos en el santuario, porque el hambre
prevaleció contra ellos, y fueron dispersados, cada uno a su lugar.

\bibleverse{55} Lisias se enteró de que Filipo, a quien el rey Antíoco,
en vida, había designado para elevar a su hijo Antíoco a la categoría de
rey, \bibleverse{56} había regresado de Persia y de Media, y con él las
fuerzas que acompañaban al rey, y que pretendía apoderarse del gobierno.
\bibleverse{57} Se apresuró y dio órdenes de partir. Dijo al rey, a los
jefes del ejército y a los hombres: ``Cada día estamos más débiles,
nuestra comida es escasa, el lugar donde acampamos es fuerte, y los
asuntos del reino recaen sobre nosotros. \bibleverse{58} Ahora, pues,
negociemos con estos hombres y hagamos la paz con ellos y con toda su
nación, \bibleverse{59} y hagamos un pacto con ellos, para que anden
según sus propias leyes, como antes; porque a causa de sus leyes, que
nosotros abolimos, se enojaron e hicieron todas estas cosas.''

\bibleverse{60} El discurso agradó al rey y a los príncipes, y envió a
ellos para hacer la paz; y ellos la aceptaron. \bibleverse{61} El rey y
los príncipes les juraron. Con estas condiciones, salieron de la
fortaleza. \bibleverse{62} Entonces el rey entró en el monte Sión. Vio
la fortaleza del lugar, y rompió el juramento que había hecho, y dio
órdenes de derribar la muralla por todas partes. \bibleverse{63} Luego
partió apresuradamente y volvió a Antioquía, y encontró a Filipo dueño
de la ciudad. Luchó contra él y tomó la ciudad por la fuerza.

\hypertarget{section-6}{%
\section{7}\label{section-6}}

\bibleverse{1} En el año ciento cincuenta y uno, Demetrio, hijo de
Seleuco, salió de Roma y subió con algunos hombres a una ciudad junto al
mar, y reinó allí. \bibleverse{2} Sucedió que cuando quiso entrar en la
casa del reino de sus padres, el ejército echó mano de Antíoco y Lisias
para llevarlos a él. \bibleverse{3} El asunto se le dio a conocer, y
dijo: ``¡No me muestren sus rostros!'' \bibleverse{4} Entonces el
ejército los mató. Entonces Demetrio se sentó en el trono de su reino.

\bibleverse{5} Todos los hombres sin ley e impíos de Israel acudieron a
él. Alcimo era su jefe, que deseaba ser sumo sacerdote. \bibleverse{6}
Acusaron al pueblo ante el rey, diciendo: ``Judas y los suyos han
destruido a todos tus amigos, y nos han dispersado de nuestra propia
tierra. \bibleverse{7} Envía, pues, ahora a un hombre de tu confianza, y
que vaya a ver toda la destrucción que ha provocado en nosotros y en el
país del rey, y cómo los ha castigado a ellos y a todos los que los
ayudaron.'' \bibleverse{8} Así que el rey eligió a Báquides, uno de los
amigos del rey, que gobernaba en el país más allá del río, y era un gran
hombre en el reino, y fiel al rey. \bibleverse{9} Lo envió a él y a
aquel impío Alcimo, a quien nombró sumo sacerdote, y le ordenó que se
vengara de los hijos de Israel.

\bibleverse{10} Marcharon y llegaron con un gran ejército a la tierra de
Judá. Envió mensajeros a Judas y a su parentela con palabras de paz
engañosas. \bibleverse{11} Ellos no prestaron atención a sus palabras,
pues vieron que habían venido con un gran ejército. \bibleverse{12} Un
grupo de escribas se reunió ante Alcimo y Báquides para buscar términos
justos. \bibleverse{13} Los Hasidaeans fueron los primeros entre los
hijos de Israel que les pidieron la paz, \bibleverse{14} pues dijeron:
``Ha venido con el ejército uno que es sacerdote de la descendencia de
Aarón, y no nos hará ningún mal.'' \bibleverse{15} Él habló con ellos
palabras de paz y les juró: ``No buscaremos haceros daño ni a vosotros
ni a vuestros amigos.'' \bibleverse{16} Ellos confiaron en él. Entonces
apresó a sesenta hombres de ellos, y los mató en un día, según la
palabra que estaba escrita, \bibleverse{17} La carne de tus santosy su
sangre fue derramada alrededor de Jerusalén, y no había nadie para
enterrarlos.

\bibleverse{18} El temor y el miedo a ellos cayeron sobre todo el
pueblo, pues dijeron: ``No hay en ellos ni verdad ni justicia, pues han
roto el pacto y el juramento que habían hecho.'' \bibleverse{19}
Báquides se retiró de Jerusalén y acampó en Bezet. Envió y apresó a
muchos de los desertores que estaban con él, y a algunos del pueblo, y
los mató, arrojándolos a un gran pozo. \bibleverse{20} Puso a Alcimo al
frente del país y dejó con él una fuerza que le ayudara. Entonces
Báquides se fue con el rey.

\bibleverse{21} Alcimo luchó por mantener su alto sacerdocio.
\bibleverse{22} Todos los que molestaban a su pueblo se unieron a él, y
se apoderaron de la tierra de Judá e hicieron gran daño en Israel.
\bibleverse{23} Judas vio todos los males que Alcimo y su compañía
habían hecho entre los hijos de Israel, incluso más que los gentiles.
\bibleverse{24} Salió a todas las fronteras de Judea y se vengó de los
hombres que le habían abandonado, y se les impidió salir al campo.
\bibleverse{25} Pero cuando Alcimo vio que Judas y su compañía se habían
hecho fuertes, y supo que no era capaz de resistirlos, regresó al rey y
presentó malas acusaciones contra ellos.

\bibleverse{26} Entonces el rey envió a Nicanor, uno de sus honorables
príncipes, un hombre que odiaba a Israel y era su enemigo, y le ordenó
que destruyera al pueblo. \bibleverse{27} Nicanor llegó a Jerusalén con
un gran ejército. Envió a Judas y a su parentela con engaño, con
palabras de paz, diciendo: \bibleverse{28} ``Que no haya batalla entre
ustedes y yo; vendré con unos pocos hombres, para ver sus rostros en
paz.'' \bibleverse{29} Se acercó a Judas, y se saludaron pacíficamente.
Los enemigos estaban dispuestos a apresar a Judas con violencia.
\bibleverse{30} Esto lo supo Judas, que vino a él con engaño, y le tuvo
mucho miedo y no quiso ver más su rostro. \bibleverse{31} Nicanor se
enteró de que su plan había sido revelado, y salió al encuentro de Judas
en la batalla junto a Cafarsalama. \bibleverse{32} Cayeron unos
quinientos hombres del ejército de Nicanor, y el resto huyó a la ciudad
de David.

\bibleverse{33} Después de estas cosas, Nicanor subió al monte Sión.
Algunos de los sacerdotes salieron del santuario, con algunos de los
ancianos del pueblo, para saludarlo pacíficamente y mostrarle todo el
sacrificio quemado que se ofrecía para el rey. \bibleverse{34} Él se
burló de ellos, se rió, se mofó vergonzosamente de ellos, habló con
arrogancia, \bibleverse{35} y juró con furia, diciendo: ``¡Si no se
entrega ahora a Judas y a su ejército en mis manos, será que, si regreso
sano y salvo, quemaré esta casa!'' Y salió con gran rabia.
\bibleverse{36} Los sacerdotes entraron y se pusieron de pie ante el
altar y el templo; lloraron y dijeron: \bibleverse{37} ``Tú elegiste
esta casa para que fuera llamada con tu nombre, para que fuera casa de
oración y súplica para tu pueblo. \bibleverse{38} Véngate de este hombre
y de su ejército, y que caigan a espada. Acuérdate de sus blasfemias y
no les permitas seguir viviendo''.

\bibleverse{39} Entonces Nicanor salió de Jerusalén y acampó en
Bethorón, y allí le salió al encuentro el ejército sirio.
\bibleverse{40} Judas acampó en Adasa con tres mil hombres. Judas oró y
dijo: \bibleverse{41} ``Cuando los que venían de parte del rey
blasfemaron, salió tu ángel e hirió entre ellos a ciento ochenta y cinco
mil. \bibleverse{42} Así, aplasta hoy a este ejército ante nosotros, y
que todos los demás sepan que ha hablado con maldad contra tu santuario.
Júzgalo según su maldad''. \bibleverse{43} El día trece del mes de Adar,
los ejércitos se enfrentaron en la batalla. El ejército de Nicanor fue
derrotado, y él mismo fue el primero en caer en la batalla.
\bibleverse{44} Cuando su ejército vio que Nicanor había caído, tiraron
sus armas y huyeron. \bibleverse{45} Los persiguieron un día de camino
desde Adasa hasta llegar a Gazara, y dieron la alarma tras ellos con las
trompetas de señales. \bibleverse{46} Salieron hombres de todos los
pueblos de los alrededores de Judea y los flanquearon. Estos les
hicieron retroceder, y todos cayeron a espada. No quedó ni uno de ellos.
\bibleverse{47} Los judíos tomaron el botín y el despojo, y le cortaron
a Nicanor la cabeza y la mano derecha que había extendido con tanta
arrogancia, las trajeron y las colgaron junto a Jerusalén.
\bibleverse{48} El pueblo se alegró mucho, y celebró ese día como un día
de gran alegría. \bibleverse{49} Y ordenaron que ese día se celebrara
cada año el día trece de Adar. \bibleverse{50} Así la tierra de Judá
descansó unos días.

\hypertarget{section-7}{%
\section{8}\label{section-7}}

\bibleverse{1} Judas oyó hablar de la fama de los romanos, de que son
hombres valientes, y que se complacen con todos los que se unen a ellos,
y hacen amistad con todos los que acuden a ellos, \bibleverse{2} y que
son hombres valientes. Le contaron sus guerras y hazañas que hacen entre
los galos, y cómo los conquistaron y los obligaron a pagar tributo;
\bibleverse{3} y qué cosas hicieron en la tierra de España, para
apoderarse de las minas de plata y oro que allí había; \bibleverse{4} y
cómo con su política y perseverancia conquistaron todo el lugar (y el
lugar estaba muy alejado de ellos), y a los reyes que vinieron contra
ellos desde el extremo de la tierra, hasta derrotarlos y golpearlos
severamente; y cómo los demás les dan tributo año tras año.
\bibleverse{5} A Filipo y a Perseo, rey de Quitim, y a los que se
alzaron contra ellos, los derrotaron en batalla y los conquistaron.
\bibleverse{6} También Antíoco, el gran rey de Asia, vino contra ellos a
la batalla, teniendo ciento veinte elefantes, con caballería, carros y
un ejército sumamente grande, y fue derrotado por ellos. \bibleverse{7}
Lo apresaron con vida y decretaron que tanto él como los que reinaron
después de él les dieran un gran tributo, y que entregaran rehenes y una
parcela de tierra de las mejores de sus provincias: \bibleverse{8} los
países de la India, Media y Lidia. Se los quitaron y se los dieron al
rey Eumenes. \bibleverse{9} Judas se enteró de que los griegos planeaban
venir a destruirlos, \bibleverse{10} pero esto se les dio a conocer, y
enviaron contra ellos un general que los combatió, y muchos de ellos
cayeron heridos de muerte, y los hicieron cautivos de sus mujeres y de
sus hijos, y los saquearon, y conquistaron sus tierras, y derribaron sus
fortalezas, y los saquearon, y los pusieron en servidumbre hasta el día
de hoy. \bibleverse{11} A los demás reinos e islas, a todos los que se
levantaron contra ellos en cualquier momento, los destruyeron y los
convirtieron en sus siervos; \bibleverse{12} pero con sus amigos y los
que se apoyaron en ellos se mantuvieron amigos. Conquistaron los reinos
cercanos y los lejanos, y todos los que oían su fama les temían.
\bibleverse{13} Además, a quienes quisieron ayudar y hacer reyes, a
éstos los hicieron reyes; y a quienes quisieron, los depusieron. Son
exaltados en extremo. \bibleverse{14} Por todo esto, ninguno de ellos se
puso jamás una corona, ni se vistió de púrpura, como muestra de
grandeza. \bibleverse{15} Judas oyó cómo se habían hecho una casa
senatorial, y día a día se sentaban en consejo trescientos veinte
hombres, consultando siempre por el pueblo, a fin de ser bien
gobernados, \bibleverse{16} y cómo encomiendan su gobierno a un solo
hombre año a año, para que los gobierne y controle todo su país, y todos
le obedecen a ése, y no hay entre ellos ni envidia ni emulación.

\bibleverse{17} Entonces Judas eligió a Eupolemo, hijo de Juan, hijo de
Accos, y a Jasón, hijo de Eleazar, y los envió a Roma, para que
establecieran amistad y alianza con ellos, \bibleverse{18} y para que se
libraran del yugo, pues veían que el reino de los griegos mantenía a
Israel en la esclavitud. \bibleverse{19} Entonces fueron a Roma, un
viaje muy largo, y entraron en la casa del Senado, y dijeron:
\bibleverse{20} ``Judas, que también se llama Macabeo, y su parentela, y
el pueblo de los judíos, nos han enviado a vosotros para establecer una
alianza y la paz con vosotros, y para que seamos registrados como
vuestros aliados y amigos.''

\bibleverse{21} Esto les agradó. \bibleverse{22} Esta es la copia del
escrito que volvieron a escribir en tablas de bronce y que enviaron a
Jerusalén, para que estuviera allí como recuerdo de paz y alianza:

\bibleverse{23} ``Buen éxito tengan los romanos y la nación de los
judíos, por mar y por tierra, para siempre. Que la espada y el enemigo
estén lejos de ellos. \bibleverse{24} Pero si la guerra surge para Roma
primero, o para cualquiera de sus aliados en todo su dominio,
\bibleverse{25} la nación de los judíos los ayudará como aliados, según
les indique la ocasión, de todo corazón. \bibleverse{26} A los que les
hagan la guerra, no les darán provisiones, alimentos, armas, dinero ni
barcos, como le ha parecido bien a Roma, y guardarán sus ordenanzas sin
tomar nada a cambio. \bibleverse{27} Del mismo modo, además, si la
guerra llega primero a la nación de los judíos, los romanos los ayudarán
de buen grado como aliados, según les indique la ocasión;
\bibleverse{28} y a los que combaten con ellos, no se les dará comida,
armas, dinero o barcos, como le ha parecido bien a Roma. Deberán guardar
estas ordenanzas, y eso sin engaño. \bibleverse{29} Según estos
términos, los romanos hicieron un tratado con el pueblo judío.
\bibleverse{30} Pero si en lo sucesivo una parte y la otra deciden
añadir o disminuir algo, lo harán a su gusto, y lo que añadan o quiten
será ratificado.

\bibleverse{31} Con respecto a los males que el rey Demetrio les está
haciendo, le hemos escrito diciendo: ``¿Por qué has hecho pesado tu yugo
sobre nuestros amigos y aliados los judíos? \bibleverse{32} Por tanto,
si vuelven a alegar contra vosotros, les haremos justicia y lucharemos
con vosotros en mar y en tierra'\,''.

\hypertarget{section-8}{%
\section{9}\label{section-8}}

\bibleverse{1} Demetrio se enteró de que Nicanor había caído con sus
fuerzas en la batalla, y envió por segunda vez a Báquides y a Alcimo al
país de Judá, y con ellos el ala derecha de su ejército. \bibleverse{2}
Fueron por el camino que lleva a Gilgal, y acamparon contra Mesalot, que
está en Arbela, y se apoderaron de ella, y mataron a mucha gente.
\bibleverse{3} El primer mes del año ciento cincuenta y dos, acamparon
contra Jerusalén. \bibleverse{4} Luego marcharon y se dirigieron a Berea
con veinte mil soldados de infantería y dos mil de caballería.
\bibleverse{5} Judas estaba acampado en Elasa con tres mil hombres
elegidos. \bibleverse{6} Vieron la multitud de las fuerzas, que eran
muchas, y se aterraron. Muchos se escabulleron del ejército. No quedaron
de ellos más que ochocientos hombres.

\bibleverse{7} Judas vio que su ejército se alejaba y que la batalla le
apremiaba, y se turbó mucho de espíritu, porque no tenía tiempo de
reunirlos, y se desmayó. \bibleverse{8} Dijo a los que quedaban:
``Levantémonos y subamos contra nuestros adversarios, si acaso podemos
luchar con ellos.''

\bibleverse{9} Trataron de disuadirlo, diciendo: ``No hay manera de que
seamos capaces; pero mejor salvemos nuestras vidas ahora. Volvamos de
nuevo con nuestra parentela y luchemos contra ellos; pero somos
demasiado pocos''.

\bibleverse{10} Judas dijo: ``No sea que yo haga esto, para huir de
ellos. Si ha llegado nuestra hora, muramos varonilmente por el bien de
nuestra parentela, y no dejemos una causa de reproche contra nuestro
honor.''

\bibleverse{11} El ejército salió del campamento y se dispuso a
enfrentarlos. La caballería se dividió en dos compañías, y los honderos
y los arqueros iban delante del ejército, y todos los hombres poderosos
que luchaban en el frente de la batalla. \bibleverse{12} Báquides estaba
en el ala derecha. La falange avanzó sobre las dos partes, y tocaron con
sus trompetas. \bibleverse{13} Los hombres que estaban al lado de Judas
tocaron con sus trompetas, y la tierra tembló con el grito de los
ejércitos, y la batalla se unió, y continuó desde la mañana hasta la
noche.

\bibleverse{14} Judas vio que Báquides y la fuerza de su ejército
estaban en el lado derecho, y todos los valientes de corazón se fueron
con él, \bibleverse{15} y el ala derecha fue derrotada por ellos, y los
persiguió hasta el monte Azoto. \bibleverse{16} Los que estaban en el
ala izquierda vieron que el ala derecha había sido derrotada, y se
volvieron y siguieron los pasos de Judas y de los que estaban con él.
\bibleverse{17} La batalla se volvió desesperada, y muchos de ambos
bandos cayeron heridos de muerte. \bibleverse{18} Judas cayó y los demás
huyeron.

\bibleverse{19} Jonatán y Simón tomaron a Judas, su hermano, y lo
enterraron en la tumba de sus antepasados en Modín. \bibleverse{20} Lo
lloraron. Todo Israel se lamentó mucho por él, y se lamentó durante
muchos días, diciendo: \bibleverse{21} ``¡Cómo ha caído el poderoso, el
salvador de Israel!'' \bibleverse{22} El resto de los hechos de Judas,
sus guerras, las hazañas que hizo y su grandeza, no están escritos,
porque fueron muchísimos.

\bibleverse{23} Después de la muerte de Judas, surgieron los anárquicos
en todos los límites de Israel. Se levantaron todos los que hacían
iniquidad. \bibleverse{24} En aquellos días hubo una gran hambruna, y el
país se pasó a su lado. \bibleverse{25} Báquides eligió a los impíos y
los nombró gobernantes del país. \bibleverse{26} Ellos indagaron y
buscaron a los amigos de Judas y los llevaron a Báquides, y él se vengó
de ellos y los utilizó despectivamente. \bibleverse{27} Hubo un gran
sufrimiento en Israel, como no lo hubo desde el tiempo en que los
profetas dejaron de aparecérseles.

\bibleverse{28} Se reunieron todos los amigos de Judas y le dijeron a
Jonatán: \bibleverse{29} ``Desde que murió tu hermano Judas, no tenemos
ningún hombre como él para salir contra nuestros enemigos y báquidos, y
entre los de nuestra nación que nos odian. \bibleverse{30} Ahora, pues,
te hemos elegido hoy para que seas nuestro príncipe y jefe en su lugar,
para que luches en nuestras batallas.'' \bibleverse{31} Así que Jonatán
tomó el gobierno en ese momento y se levantó en lugar de su hermano
Judas.

\bibleverse{32} Cuando Báquides se enteró, trató de matarlo.
\bibleverse{33} Lo supieron Jonatán, su hermano Simón y todos los que
estaban con él, y huyeron al desierto de Tecoa y acamparon junto a las
aguas del estanque de Asfar. \bibleverse{34} Báquides lo descubrió el
día sábado, y vino --- él y todo su ejército --- a cruzar el Jordán.

\bibleverse{35} Jonatán envió a su hermano, jefe de la multitud, y
suplicó a sus amigos los nabateos que guardaran con ellos su equipaje,
que era mucho. \bibleverse{36} Los hijos de Jambri salieron de Medaba,
se apoderaron de Juan y de todo lo que tenía y se fueron con él.

\bibleverse{37} Pero después de estas cosas, llevaron la noticia a
Jonatán y a su hermano Simón de que los hijos de Jambri estaban
celebrando una gran boda, y que traían a la novia, hija de uno de los
grandes nobles de Canaán, desde Nadabat con una gran escolta.
\bibleverse{38} Se acordaron de Juan, su hermano, y subieron y se
escondieron al amparo del monte. \bibleverse{39} Levantaron los ojos y
miraron, y vieron un gran cortejo con mucho equipaje. El novio salía con
sus amigos y su parentela a su encuentro con timbales, músicos y muchas
armas. \bibleverse{40} Se levantaron contra ellos desde su emboscada y
los mataron, y muchos cayeron heridos de muerte. El resto huyó al monte,
y los judíos se llevaron todo su botín. \bibleverse{41} Entonces las
bodas se convirtieron en luto, y la voz de sus músicos en lamento.
\bibleverse{42} Vengaron plenamente la sangre de su hermano y se
volvieron a los pantanos del Jordán.

\bibleverse{43} Lo oyó Báquides, y vino el sábado a las orillas del
Jordán con un gran ejército. \bibleverse{44} Jonatán dijo a su compañía:
``Levantémonos ahora y luchemos por nuestras vidas, pues las cosas son
diferentes hoy de lo que fueron ayer y anteayer. \bibleverse{45} Porque
he aquí que la batalla está delante y detrás de nosotros. Además, las
aguas del Jordán están a un lado y al otro, y el pantano y la espesura.
No hay lugar para escapar. \bibleverse{46} Ahora, pues, clamad al cielo,
para que seáis librados de la mano de vuestros enemigos.''
\bibleverse{47} Así se unió la batalla, y Jonatán extendió su mano para
golpear a Báquides, y éste se apartó de él. \bibleverse{48} Jonatán y
los que estaban con él saltaron al Jordán y pasaron a nado al otro lado.
El enemigo no pasó el Jordán contra ellos. \bibleverse{49} Aquel día
cayeron unos mil hombres de la compañía de Báquides; \bibleverse{50} y
volvió a Jerusalén. Construyeron ciudades fuertes en Judea, la fortaleza
que estaba en Jericó, y Emaús, Bethorón, Betel, Timnat, Faratón y Tefón,
con altos muros, puertas y rejas. \bibleverse{51} En ellas puso
guarniciones para acosar a Israel. \bibleverse{52} Fortificó la ciudad
de Betsura, Gázara y la ciudadela, y puso en ellas tropas y provisiones.
\bibleverse{53} Tomó como rehenes a los hijos de los jefes del país y
los puso de guardia en la ciudadela de Jerusalén.

\bibleverse{54} En el año ciento cincuenta y tres, en el segundo mes,
Alcimo dio órdenes de derribar el muro del patio interior del santuario.
También derribó las obras de los profetas. \bibleverse{55} Comenzó a
derribar. En aquel tiempo, Alcimo fue golpeado, y sus obras fueron
obstaculizadas; y su boca se detuvo, y fue tomado por una parálisis, y
ya no pudo hablar nada ni dar órdenes acerca de su casa. \bibleverse{56}
Alcimo murió en aquel momento con gran tormento. \bibleverse{57}
Báquides vio que Alcimo había muerto, y volvió al rey. Entonces la
tierra de Judá tuvo descanso durante dos años.

\bibleverse{58} Entonces todos los hombres sin ley tomaron consejo,
diciendo: ``He aquí que Jonatán y sus hombres habitan tranquilos y
seguros. Ahora, pues, traeremos a Báquides, y él los capturará a todos
en una sola noche. \bibleverse{59} Fueron y consultaron con él.
\bibleverse{60} Marchó y vino con un gran ejército, y envió cartas en
secreto a todos sus aliados que estaban en Judea, para que apresaran a
Jonatán y a los que estaban con él; pero no pudieron, porque su plan era
conocido por ellos. \bibleverse{61} Los hombres de Jonatán apresaron a
unos cincuenta de los hombres del país que eran autores de la maldad, y
él los mató. \bibleverse{62} Jonatán, Simón y los que estaban con él, se
fueron a Bet-bási, que está en el desierto, y él reconstruyó lo que
había sido derribado, y lo hicieron fuerte. \bibleverse{63} Báquides se
enteró de ello, y reunió a toda su multitud, y envió órdenes a los que
eran de Judea. \bibleverse{64} Fue y acampó contra Bet-bási y luchó
contra ella muchos días, e hizo máquinas de guerra.

\bibleverse{65} Jonatán dejó a su hermano Simón en la ciudad, y salió al
campo, y fue con algunos hombres. \bibleverse{66} Golpeó a Odomera y a
su parentela, y a los hijos de Fasirón en sus tiendas. \bibleverse{67}
Comenzaron a golpearlos y a subir con sus fuerzas. Entonces Simón y los
que estaban con él salieron de la ciudad, e incendiaron las máquinas de
guerra, \bibleverse{68} y lucharon contra Báquides, y éste fue derrotado
por ellos. Lo afligieron severamente, pues su consejo y su expedición
fueron en vano. \bibleverse{69} Se enfadaron mucho con los hombres sin
ley que le aconsejaron entrar en el país, y mataron a muchos de ellos.
Entonces decidió partir hacia su propia tierra.

\bibleverse{70} Jonatán se enteró de esto y le envió embajadores, con el
fin de que hicieran las paces con él y les devolviera a los cautivos.
\bibleverse{71} El aceptó la cosa, e hizo lo que había dicho, y le juró
que no buscaría su mal en todos los días de su vida. \bibleverse{72} Le
devolvió los cautivos que había tomado antes de la tierra de Judá, y él
regresó y se marchó a su tierra, y no volvió a entrar en sus fronteras.
\bibleverse{73} Así cesó la espada en Israel. Jonatán vivió en Micmas.
Jonatán comenzó a juzgar al pueblo, y destruyó a los impíos de Israel.

\hypertarget{section-9}{%
\section{10}\label{section-9}}

\bibleverse{1} En el año ciento sesenta, Alejandro Epífanes, hijo de
Antíoco, subió y tomó posesión de Tolemaida. Lo recibieron y reinó allí.
\bibleverse{2} El rey Demetrio se enteró de esto, y reunió fuerzas muy
grandes y salió a enfrentarse con él en la batalla.

\bibleverse{3} Demetrio envió una carta a Jonatán con palabras de paz,
para honrarlo. \bibleverse{4} Pues dijo: ``Vayamos de antemano a hacer
la paz con ellos, antes de que él haga la paz con Alejandro contra
nosotros; \bibleverse{5} pues se acordará de todos los males que hemos
hecho contra él, y a su parentela y a su nación.'' \bibleverse{6} Así
que le dio autoridad para reunir fuerzas y proporcionar armas, y que
fuera su aliado. También ordenó que le soltaran los rehenes que había en
la ciudadela.

\bibleverse{7} Jonatán llegó a Jerusalén y leyó la carta a la vista de
todo el pueblo y de los que estaban en la ciudadela. \bibleverse{8} Se
asustaron mucho al oír que el rey le había dado autoridad para reunir un
ejército. \bibleverse{9} Los que estaban en la ciudadela entregaron los
rehenes a Jonatán, y él los devolvió a sus padres.

\bibleverse{10} Jonatán vivía en Jerusalén y comenzó a construir y
renovar la ciudad. \bibleverse{11} Mandó a los que hacían la obra que
construyeran las murallas y rodearan el monte Sión con piedras cuadradas
para la defensa; y así lo hicieron. \bibleverse{12} Los extranjeros que
estaban en las fortalezas que había construido Báquides huyeron.
\bibleverse{13} Cada uno dejó su lugar y se marchó a su tierra.
\bibleverse{14} Sólo en Betsura quedaron algunos de los que habían
abandonado la ley y los mandamientos, pues era un lugar de refugio para
ellos.

\bibleverse{15} El rey Alejandro escuchó todas las promesas que Demetrio
había enviado a Jonatán. Le contaron las batallas y las hazañas que él y
los suyos habían hecho, y los problemas que habían soportado.
\bibleverse{16} Entonces dijo: ``¿Podríamos encontrar otro hombre como
él? Ahora lo haremos nuestro amigo y aliado''. \bibleverse{17} Escribió
una carta y se la envió con estas palabras, diciendo: \bibleverse{18}
``Rey Alejandro a su hermano Jonatán, saludos. \bibleverse{19} Hemos
oído hablar de ti, que eres un hombre valiente y digno de ser nuestro
amigo. \bibleverse{20} Hoy te hemos nombrado sumo sacerdote de tu
nación, y te llamaremos amigo del rey, y te pondrás de nuestra parte y
mantendrás la amistad con nosotros.'' También le envió un manto de
púrpura y una corona de oro.

\bibleverse{21} Y Jonatán se vistió con las vestiduras sagradas en el
séptimo mes del año ciento sesenta, en la fiesta de los tabernáculos; y
reunió fuerzas y se proveyó de armas en abundancia.

\bibleverse{22} Al oír estas cosas, Demetrio se afligió y dijo:
\bibleverse{23} ``¿Qué es lo que hemos hecho, que Alejandro se ha
adelantado a nosotros en establecer amistad con los judíos para
fortalecerse? \bibleverse{24} Yo también les escribiré palabras de
aliento, de honor y de regalos, para que estén conmigo para ayudarme.''
\bibleverse{25} Así que les envió este mensaje:

``Rey Demetrio a la nación de los judíos, saludos. \bibleverse{26}
Puesto que habéis mantenido vuestros pactos con nosotros y habéis
continuado con nuestra amistad, y no os habéis unido a nuestros
enemigos, nos hemos enterado de ello y nos alegramos. \bibleverse{27}
Ahora seguid manteniendo la fe con nosotros, y os devolveremos el bien a
cambio de vuestros tratos con nosotros. \bibleverse{28} Os concederemos
muchas inmunidades y os daremos regalos.

\bibleverse{29} ``Ahora te libero y libero a todos los judíos de los
tributos, del impuesto de la sal y de los gravámenes de la corona.
\bibleverse{30} En lugar de la tercera parte de la semilla, y en lugar
de la mitad del fruto de los árboles, que me corresponde recibir, la
libero desde hoy y en adelante, para que no la tome de la tierra de Judá
y de los tres distritos que se le agregan del país de Samaria y de
Galilea, desde hoy y para siempre. \bibleverse{31} Que Jerusalén sea
santa y libre, con sus fronteras, diezmos e impuestos. \bibleverse{32}
También cedo mi autoridad sobre la ciudadela que está en Jerusalén, y se
la doy al sumo sacerdote, para que designe en ella a los hombres que él
quiera para que la guarden. \bibleverse{33} A toda alma de los judíos
que haya sido llevada cautiva desde la tierra de Judá a cualquier parte
de mi reino, la pongo en libertad sin pago alguno. Que todos los
funcionarios cancelen también los impuestos sobre su ganado.

\bibleverse{34} ``Todas las fiestas, los sábados, las lunas nuevas, los
días señalados, los tres días antes de una fiesta y los tres días
después de una fiesta, sean todos ellos días de inmunidad y liberación
para todos los judíos que están en mi reino. \bibleverse{35} Nadie
tendrá autoridad para exigir nada a ninguno de ellos, ni para
molestarlos en ningún asunto.

\bibleverse{36} ``Que se inscriban entre las fuerzas del rey unos
treinta mil hombres de los judíos, y se les dará la paga que corresponde
a todas las fuerzas del rey. \bibleverse{37} De ellos, algunos serán
colocados en las grandes fortalezas del rey, y algunos de ellos serán
colocados sobre los asuntos del reino, que son cargos de confianza. Que
los que estén sobre ellos y sus gobernantes sean de ellos mismos, y que
anden según sus propias leyes, tal como lo ha ordenado el rey en la
tierra de Judá.

\bibleverse{38} ``Los tres distritos que se han añadido a Judea desde el
país de Samaria, que se anexen a Judea, para que se consideren bajo un
solo gobernante, para que no obedezcan a otra autoridad que la del sumo
sacerdote. \bibleverse{39} En cuanto a Tolemaida y su tierra, la he dado
como regalo al santuario que está en Jerusalén, para los gastos del
santuario. \bibleverse{40} También doy cada año quince mil siclos de
plata de las rentas del rey de los lugares que son apropiados.
\bibleverse{41} Y todos los fondos adicionales que los que administran
los asuntos del rey no pagaron como en los primeros años, los darán
desde ahora para las obras del templo. \bibleverse{42} Además de esto,
también se liberan los cinco mil siclos de plata que recibían de los
usos del santuario de los ingresos de año en año, porque pertenecen a
los sacerdotes que ministran allí. \bibleverse{43} Todo el que huya al
templo que está en Jerusalén, y dentro de todos sus límites, ya sea que
deba dinero al rey o cualquier otro asunto, que quede libre, junto con
todo lo que tenga en mi reino. \bibleverse{44} Para la construcción y
renovación de las estructuras del santuario, el gasto se dará también de
los ingresos del rey. \bibleverse{45} Para la construcción de los muros
de Jerusalén y la fortificación de todo su contorno, el gasto se dará
también con los ingresos del rey, así como para la construcción de los
muros en Judea.''

\bibleverse{46} Cuando Jonatán y el pueblo oyeron estas palabras, no les
dieron crédito y no las aceptaron, porque se acordaron del gran mal que
había hecho en Israel y de que los había afligido muy severamente.
\bibleverse{47} Ellos estaban muy contentos con Alejandro, porque fue el
primero que les habló palabras de paz, y siempre fueron aliados de él.

\bibleverse{48} El rey Alejandro reunió grandes fuerzas y acampó cerca
de Demetrio. \bibleverse{49} Los dos reyes entablaron batalla, y el
ejército de Alejandro huyó; y Demetrio lo siguió, y se impuso sobre
ellos. \bibleverse{50} El rey Alejandro intensificó mucho la batalla
hasta que se puso el sol, y Demetrio cayó aquel día.

\bibleverse{51} Alejandro envió embajadores a Ptolomeo, rey de Egipto,
con este mensaje \bibleverse{52} ``Ya que he regresado a mi reino, y me
he sentado en el trono de mis padres, y he establecido mi dominio, y he
derrocado a Demetrio, y he tomado posesión de nuestro país ---
\bibleverse{53} sí, me uní a la batalla con él, y él y su ejército
fueron derrotados por nosotros, y nos sentamos en el trono de su reino
--- \bibleverse{54} ahora también hagamos amistad entre nosotros. Dame
ahora a tu hija como esposa. Me uniré a ti, y te daré a ti y a ella
regalos dignos de ti''.

\bibleverse{55} El rey Ptolomeo respondió diciendo: ``Feliz es el día en
que has vuelto a la tierra de tus antepasados y te has sentado en el
trono de su reino. \bibleverse{56} Ahora haré contigo lo que has
escrito, pero reúnete conmigo en Tolemaida, para que nos veamos; y me
uniré a ti, tal como has dicho.'' \bibleverse{57} Así que Ptolomeo salió
de Egipto, él y su hija Cleopatra, y llegó a Tolemaida en el año ciento
sesenta y dos. \bibleverse{58} El rey Alejandro salió a su encuentro y
le entregó a su hija Cleopatra, y celebró su boda en Tolemaida con gran
pompa, como hacen los reyes.

\bibleverse{59} El rey Alejandro escribió a Jonatán para que fuera a su
encuentro. \bibleverse{60} Fue con pompa a Tolemaida y se encontró con
los dos reyes. Les dio a ellos y a sus amigos plata y oro, y muchos
regalos, y halló gracia ante ellos. \bibleverse{61} Algunos descontentos
de Israel, hombres transgresores de la ley, se reunieron contra él para
quejarse, pero el rey no les hizo caso. \bibleverse{62} El rey ordenó
que le quitaran los vestidos a Jonatán y lo vistieran de púrpura, y así
lo hicieron. \bibleverse{63} El rey hizo que se sentara con él, y dijo a
sus príncipes: ``Salgan con él al centro de la ciudad y proclamen que
nadie se queje contra él de ningún asunto, y que nadie lo moleste por
ningún motivo.'' \bibleverse{64} Sucedió que cuando los que se quejaban
contra él vieron su honor según la proclama, y lo vieron vestido de
púrpura, todos huyeron. \bibleverse{65} El rey le concedió honores y lo
inscribió entre sus principales amigos,\footnote{\textbf{10:65} 11:27; 2
  Macabeos 8:9. Compárese 1 Macabeos 2:18; 10:16, etc.} y lo nombró
capitán y gobernador de una provincia. \bibleverse{66} Entonces Jonatán
volvió a Jerusalén con paz y alegría.

\bibleverse{67} En el año ciento sesenta y cinco,\footnote{\textbf{10:67}
  hacia el año 148 a. C.} Demetrio, hijo de Demetrio, salió de Creta a
la tierra de sus antepasados. \bibleverse{68} Cuando el rey Alejandro se
enteró, se entristeció mucho y volvió a Antioquía. \bibleverse{69}
Demetrio nombró a Apolonio, que estaba sobre Coelesyria, y reunió un
gran ejército y acampó contra Jamnia, y envió al sumo sacerdote Jonatán,
diciendo

\bibleverse{70} ``Sólo tú te alzas contra nosotros, pero yo me siento
ridiculizado y en reproche por tu culpa. ¿Por qué te arrogas autoridad
contra nosotros en las montañas? \bibleverse{71} Ahora, pues, si confías
en tus fuerzas, baja a nosotros a la llanura, y comparemos allí nuestras
fuerzas; porque el poder de las ciudades está conmigo. \bibleverse{72}
Pregunta y aprende quién soy yo y los demás que nos ayudan. Ellos dicen:
`Tu pie no puede estar ante nuestra cara; porque tus antepasados han
sido puestos en fuga dos veces en su propia tierra.' \bibleverse{73}
Ahora no podrás resistir a la caballería y a un ejército como éste en la
llanura, donde no hay piedra ni guijarro, ni lugar para huir.''

\bibleverse{74} Cuando Jonatán oyó las palabras de Apolonio, se
conmovió, y escogió diez mil hombres y salió de Jerusalén; y su hermano
Simón le salió al encuentro para ayudarle. \bibleverse{75} Entonces
acampó frente a Jope. Los habitantes de la ciudad le cerraron el paso,
porque Apolonio tenía una guarnición en Jope. \bibleverse{76} Así que
lucharon contra él. Los habitantes de la ciudad tuvieron miedo y le
abrieron; y Jonatán se hizo dueño de Jope.

\bibleverse{77} Apolonio se enteró de ello, y reunió un ejército de tres
mil soldados de caballería, y un gran ejército, y se dirigió a Azoto
como si estuviera de viaje, y al mismo tiempo avanzó hacia la llanura,
porque tenía una multitud de caballería en la que confiaba.
\bibleverse{78} Le persiguió hasta Azoto, y los ejércitos se unieron en
la batalla. \footnote{\textbf{10:78} La mayoría de las autoridades
  repiten después de él.} \bibleverse{79} Apolonio había dejado en
secreto un millar de caballería detrás de ellos. \bibleverse{80} Jonatán
se enteró de que había una emboscada detrás de él. Rodearon a su
ejército, y dispararon sus flechas contra el pueblo, desde la mañana
hasta la noche; \bibleverse{81} pero el pueblo se mantuvo firme, como
Jonatán les ordenó; y los caballos del enemigo se cansaron.

\bibleverse{82} Entonces Simón adelantó su ejército y se unió a la
falange (pues la caballería estaba agotada), y fueron derrotados por él
y huyeron. \bibleverse{83} La caballería se dispersó por la llanura.
Huyeron a Azoto y entraron en Bet-dagón, el templo de su ídolo, para
salvarse. \bibleverse{84} Jonatán quemó Azoto y las ciudades de los
alrededores y tomó sus despojos. Quemó con fuego el templo de Dagón y a
los que huyeron a él. \bibleverse{85} Los que cayeron a espada más los
que fueron quemados fueron unos ocho mil hombres.

\bibleverse{86} Desde allí, Jonatán partió y acampó frente a Ascalón. La
gente de la ciudad salió a recibirlo con gran pompa. \bibleverse{87}
Jonatán, con los que estaban de su lado, regresó a Jerusalén, teniendo
muchos despojos. \bibleverse{88} Cuando el rey Alejandro se enteró de
estas cosas, honró aún más a Jonatán. \bibleverse{89} Le envió una
hebilla de oro, como se acostumbra a dar a los parientes del rey. Le dio
Ecrón y toda su tierra como posesión.

\hypertarget{section-10}{%
\section{11}\label{section-10}}

\bibleverse{1} Entonces el rey de Egipto reunió grandes fuerzas, como la
arena que está a la orilla del mar, y muchas naves, y trató de hacerse
dueño del reino de Alejandro con engaños, y añadirlo a su propio reino.
\bibleverse{2} Salió a Siria con palabras de paz, y los habitantes de
las ciudades le abrieron las puertas y le salieron al encuentro, pues el
rey Alejandro había ordenado que le salieran al encuentro, porque era su
suegro. \bibleverse{3} Al entrar en las ciudades de Tolemaida, dispuso
sus fuerzas para una guarnición en cada ciudad.

\bibleverse{4} Pero cuando se acercó a Azoto, le mostraron el templo de
Dagón quemado por el fuego, y Azoto y sus tierras de pastoreo
destruidas, y los cadáveres arrojados, y los que habían sido quemados,
que él había quemado en la guerra, pues habían hecho montones de ellos
en su camino. \bibleverse{5} Le contaron al rey lo que había hecho
Jonatán, para echarle la culpa, pero el rey guardó silencio.
\bibleverse{6} Jonatán recibió al rey con pompa en Jope, se saludaron y
durmieron allí. \bibleverse{7} Jonatán acompañó al rey hasta el río que
se llama Eleutero, y luego regresó a Jerusalén.

\bibleverse{8} Pero el rey Ptolomeo se apoderó de las ciudades de la
costa del mar, hasta Selucia, que está junto al mar, e ideó malvados
planes respecto a Alejandro. \bibleverse{9} Envió embajadores al rey
Demetrio, diciendo: ``¡Ven! Hagamos un pacto entre nosotros, y yo te
daré a mi hija que tiene Alejandro, y tú reinarás sobre el reino de tu
padre; \bibleverse{10} pues me arrepiento de haberle dado a mi hija, ya
que intentó matarme. \bibleverse{11} Lo acusó porque codiciaba su reino.
\bibleverse{12} Quitándole a su hija, se la dio a Demetrio, y se alejó
de Alejandro, y su enemistad se vio abiertamente.

\bibleverse{13} Ptolomeo entró en Antioquía y se puso la corona de Asia.
Puso sobre su cabeza dos coronas, la de Egipto y la de Asia.
\bibleverse{14} Pero el rey Alejandro estaba entonces en Cilicia, porque
los pueblos de esa región estaban revueltos. \bibleverse{15} Cuando
Alejandro se enteró, vino contra él en guerra. Ptolomeo salió a su
encuentro con una fuerte fuerza y lo puso en fuga. \bibleverse{16}
Alejandro huyó a Arabia para refugiarse allí, pero el rey Tolomeo salió
triunfante. \bibleverse{17} Zabdiel, el árabe, le quitó la cabeza a
Alejandro y se la envió a Tolomeo. \bibleverse{18} El rey Tolomeo murió
al tercer día, y los habitantes de sus fortalezas mataron a los que
estaban en ellas. \bibleverse{19} Demetrio se convirtió en rey el año
ciento sesenta y siete. \footnote{\textbf{11:19} hacia el año 146 a. C.}

\bibleverse{20} En aquellos días Jonatán reunió a los judaítas para
tomar la ciudadela que estaba en Jerusalén. Hizo muchas máquinas de
guerra para usarlas contra ella. \bibleverse{21} Algunos hombres sin ley
que odiaban a su propia nación fueron al rey y le informaron que Jonatán
estaba sitiando la ciudadela. \bibleverse{22} El rey se enteró y se
enfureció, pero al oírlo se puso en marcha inmediatamente, llegó a
Tolemaida y le escribió a Jonatán para que no la sitiara y para que se
reuniera con él y hablara con él en Tolemaida a toda prisa.

\bibleverse{23} Pero cuando Jonatán oyó esto, dio órdenes de continuar
el asedio. Escogió a algunos de los ancianos de Israel y de los
sacerdotes, y se puso en peligro \bibleverse{24} tomando plata, oro,
ropa y varios otros regalos, y fue a Tolemaida a ver al rey. Entonces
halló gracia ante sus ojos. \bibleverse{25} Algunos hombres sin ley de
los que eran de la nación presentaron quejas contra él, \bibleverse{26}
y el rey hizo con él lo mismo que habían hecho sus predecesores, y lo
exaltó a la vista de todos sus\footnote{\textbf{11:26} Véase 1 Macabeos
  2:18.} amigos, \bibleverse{27} y le confirmó el sumo sacerdocio y
todos los demás honores que antes tenía, y le dio preeminencia entre
sus\footnote{\textbf{11:27} Véase 1 Macabeos 10:65.} principales amigos.
\bibleverse{28} Jonatán pidió al rey que liberara a Judea de los
tributos, junto con las tres\footnote{\textbf{11:28} Gr. toparchies}
provincias y el país de Samaria, y le prometió trescientos talentos.
\bibleverse{29} El rey consintió y escribió cartas a Jonatán acerca de
todas estas cosas, como sigue:

\bibleverse{30} ``Rey Demetrio a su hermano Jonatán y a la nación de los
judíos, saludos. \bibleverse{31} La copia de la carta que escribimos a
Lóstenes, nuestro pariente, acerca de ti, te la hemos escrito también a
ti, para que la veas. \bibleverse{32} ``Rey Demetrio a su padre
Lóstenes, saludos. \bibleverse{33} Hemos decidido hacer el bien a la
nación de los judíos, que son nuestros amigos, y observar lo que es
justo para con nosotros, a causa de su buena voluntad para con ellos.
\bibleverse{34} Por lo tanto, les hemos confirmado los límites de Judea,
y también los tres gobiernos de Afaerema, Lida y Ramataim (estos fueron
agregados a Judea desde el país de Samaria), y todo su territorio a
ellos, para todos los que hacen sacrificios en Jerusalén, en lugar de
las cuotas reales que el rey recibía de ellos anualmente antes del
producto de la tierra y de los frutos de los árboles. \bibleverse{35} En
cuanto a los demás pagos que nos corresponden de ahora en adelante, de
los diezmos y de los impuestos que nos corresponden, y de las salinas y
de los impuestos de la corona que nos corresponden, todo esto se lo
devolveremos. \bibleverse{36} Ninguna de estas concesiones será anulada
desde ahora y para siempre. \bibleverse{37} Ahora, pues, cuida de hacer
una copia de estas cosas y entrégala a Jonatán, y colócala en el monte
sagrado en un lugar adecuado y visible''.

\bibleverse{38} Cuando el rey Demetrio vio que la tierra estaba
tranquila ante él y que no se le oponía ninguna resistencia, despidió a
todas sus tropas, cada una a su lugar, excepto a las tropas extranjeras
que había levantado de las islas de los gentiles. Así que todas las
tropas de sus padres lo odiaban. \bibleverse{39} Trifón era uno de los
que antes habían estado del lado de Alejandro, y vio que todas las
fuerzas murmuraban contra Demetrio. Así que fue a ver a Imalcue el
árabe, que estaba criando a Antíoco, el hijo pequeño de Alejandro,
\bibleverse{40} y le insistió urgentemente en que se lo entregara, para
que reinara en lugar de su padre. Le contó todo lo que había hecho
Demetrio, y el odio con que sus fuerzas lo odiaban; y se quedó allí
muchos días.

\bibleverse{41} Jonatán envió al rey Demetrio para que retirara de
Jerusalén las tropas de la ciudadela y las que estaban en las
fortalezas, porque luchaban continuamente contra Israel. \bibleverse{42}
Demetrio envió a decir a Jonatán: ``No sólo haré esto por ti y por tu
nación, sino que te honraré mucho a ti y a tu nación, si encuentro la
oportunidad. \bibleverse{43} Ahora, pues, harás bien si me envías
hombres que luchen por mí, porque todas mis fuerzas se han sublevado.''
\bibleverse{44} Así que Jonatán le envió tres mil hombres valientes a
Antioquía. Llegaron al rey, y éste se alegró de su llegada.

\bibleverse{45} El pueblo de la ciudad se reunió en medio de la ciudad,
en número de ciento veinte mil hombres, y querían matar al rey.
\bibleverse{46} El rey huyó al patio del palacio, y el pueblo de la
ciudad se apoderó de las calles principales de la ciudad y comenzó a
luchar. \bibleverse{47} El rey llamó a los judíos para que le ayudaran,
y se reunieron todos a la vez, y se dispersaron por la ciudad, y mataron
aquel día a unos cien mil. \bibleverse{48} Prendieron fuego a la ciudad
y se apoderaron de muchos despojos aquel día, y salvaron al rey.
\bibleverse{49} La gente de la ciudad vio que los judíos se habían
apoderado de la ciudad a su antojo, y desfallecieron de corazón, y
clamaron al rey con súplicas, diciendo: \bibleverse{50} ``Danos tu mano
derecha, y que los judíos dejen de luchar contra nosotros y la ciudad.''
\bibleverse{51} Tiraron las armas e hicieron la paz. Los judíos fueron
glorificados a los ojos del rey y ante todos los que estaban en su
reino. Luego regresaron a Jerusalén, con mucho botín. \bibleverse{52} El
rey Demetrio se sentó en el trono de su reino, y la tierra estaba
tranquila ante él. \bibleverse{53} Mintió en todo lo que dijo, y se
alejó de Jonatán, y no le pagó conforme a los beneficios con que le
había retribuido, y lo trató muy duramente.

\bibleverse{54} Después de esto, regresó Trifón y con él el niño
Antíoco, que reinó y se puso una corona. \bibleverse{55} Se reunieron
con él todas las fuerzas que Demetrio había despedido con deshonra, y
lucharon contra él, y huyó y fue derrotado. \bibleverse{56} Trifón tomó
los elefantes y se apoderó de Antioquía. \bibleverse{57} El joven
Antíoco escribió a Jonatán, diciendo: ``Te confirmo el sumo sacerdocio y
te nombro sobre los cuatro distritos, y para que seas uno de los amigos
del rey''.\footnote{\textbf{11:57} Ver 1 Macabeos 2:18.} \bibleverse{58}
Le envió vasos de oro y muebles para la mesa, y le dio permiso para
beber en vasos de oro, y para vestirse de púrpura y tener una hebilla de
oro. \bibleverse{59} Hizo gobernador a su hermano Simón desde la
Escalera de Tiro hasta los límites de Egipto.

\bibleverse{60} Jonatán salió y emprendió su viaje más allá del río y a
través de las ciudades. Todas las fuerzas de Siria se reunieron con él
para ser sus aliados. Llegó a Ascalón, y el pueblo de la ciudad lo
recibió honorablemente. \bibleverse{61} De allí partió hacia Gaza, y el
pueblo de Gaza le cerró el paso. Entonces la sitió y quemó sus tierras
de pastoreo con fuego, y las saqueó. \bibleverse{62} El pueblo de Gaza
suplicó a Jonatán, y él les dio su mano derecha, y tomó a los hijos de
sus príncipes como rehenes, y los envió a Jerusalén. Luego pasó por el
país hasta Damasco.

\bibleverse{63} Entonces Jonatán se enteró de que los príncipes de
Demetrio habían llegado a Cedes, que está en Galilea, con un gran
ejército, con la intención de destituirlo. \bibleverse{64} Salió a su
encuentro, pero dejó a su hermano Simón en el campo. \bibleverse{65}
Simón acampó contra Betsura, y luchó contra ella muchos días, y la
cercó. \bibleverse{66} Le pidieron que les diera su mano derecha, y él
se la dio. Los sacó de allí, tomó posesión de la ciudad y puso una
guarnición sobre ella.

\bibleverse{67} Jonatán y su ejército acamparon junto a las aguas de
Genesaret, y de madrugada marcharon a la llanura de Hazor.
\bibleverse{68} He aquí que un ejército de extranjeros le salió al
encuentro en la llanura. Le tendieron una emboscada en las montañas,
pero ellos mismos se encontraron con él cara a cara. \bibleverse{69}
Pero los que estaban en la emboscada se levantaron de sus lugares y se
unieron a la batalla. Todos los que estaban del lado de Jonatán huyeron.
\bibleverse{70} No quedó ni uno de ellos, excepto Matatías, hijo de
Absalón, y Judas, hijo de Calfo, capitanes de las fuerzas.
\bibleverse{71} Jonatán se rasgó las vestiduras, se puso tierra en la
cabeza y oró. \bibleverse{72} Volvió a enfrentarse a ellos en la batalla
y los derrotó, y ellos huyeron. \bibleverse{73} Cuando los hombres de su
bando que habían huido vieron esto, volvieron a él y siguieron con él
hasta Cedes, hasta su campamento, y acamparon allí. \bibleverse{74} Ese
día cayeron unos tres mil hombres de los extranjeros. Entonces Jonatán
regresó a Jerusalén.

\hypertarget{section-11}{%
\section{12}\label{section-11}}

\bibleverse{1} Jonatán vio que el momento era favorable para él, así que
eligió hombres y los envió a Roma para confirmar y renovar la amistad
que tenían con ellos. \bibleverse{2} También envió cartas similares a
los espartanos y a otros lugares. \bibleverse{3} Fueron a Roma, entraron
en la casa del Senado y dijeron: ``El sumo sacerdote Jonatán y la nación
de los judíos nos han enviado para renovarles la amistad y la alianza,
como en tiempos anteriores.'' \bibleverse{4} Les entregaron cartas a los
hombres de cada lugar, para que les proporcionaran un salvoconducto en
su camino hacia la tierra de Judá. \bibleverse{5} Esta es la copia de
las cartas que Jonatán escribió a los espartanos:

\bibleverse{6} ``El sumo sacerdote Jonatán, el senado de la nación, los
sacerdotes y el resto del pueblo de los judíos, a sus parientes los
espartanos, saludos. \bibleverse{7} Ya antes de este tiempo se enviaron
cartas al sumo sacerdote Onías de parte de Arrio,\footnote{\textbf{12:7}
  Así lo indican las antiguas versiones latinas y Josefo: compárese
  también el ver. 20. Todas las demás autoridades leen Darío en este
  lugar.} que reinaba entre ustedes, para significar que ustedes son
nuestros parientes, como lo muestra la copia escrita a continuación.
\bibleverse{8} Onías acogió honorablemente al enviado y recibió las
cartas, en las que se declaraba la alianza y la amistad. \bibleverse{9}
Por lo tanto, también nosotros, aunque no necesitamos nada de esto,
teniendo como estímulo los libros sagrados que están en nuestras manos,
\bibleverse{10} nos hemos comprometido a enviar para renovar nuestra
hermandad y amistad con vosotros, con el fin de no alejarnos del todo de
vosotros; pues ha pasado mucho tiempo desde que nos enviasteis vuestra
carta. \bibleverse{11} Por lo tanto, en todo momento y sin cesar, tanto
en nuestras fiestas como en los demás días convenientes, nos acordamos
de vosotros en los sacrificios que ofrecemos y en nuestras oraciones,
como es justo y apropiado tener en cuenta a los parientes.
\bibleverse{12} Además, nos alegramos por tu gloria. \bibleverse{13}
Pero en cuanto a nosotros, nos han rodeado muchas aflicciones y muchas
guerras, y los reyes que nos rodean han luchado contra nosotros.
\bibleverse{14} No quisimos ser molestos para ti, ni para el resto de
nuestros aliados y amigos, en estas guerras; \bibleverse{15} porque
tenemos la ayuda que viene del cielo para socorrernos, y hemos sido
librados de nuestros enemigos, y nuestros enemigos han sido humillados.
\bibleverse{16} Elegimos, pues, a Numenio, hijo de Antíoco, y a
Antípatro, hijo de Jasón, y los enviamos a los romanos para renovar la
amistad que teníamos con ellos y la antigua alianza. \bibleverse{17} Les
hemos ordenado, pues, que vayan también a vosotros y os saluden, y os
entreguen nuestras cartas relativas a la renovación de la amistad y de
nuestra hermandad. \bibleverse{18} Y ahora haréis bien si nos dais una
respuesta''.

\bibleverse{19} Esta es la copia de las cartas que enviaron a Onías:

\bibleverse{20} ``Arrio, rey de los espartanos, a Onías, el sumo
sacerdote, saludos. \bibleverse{21} Se ha encontrado por escrito,
respecto a los espartanos y los judíos, que son parientes y que son de
la descendencia de Abraham. \bibleverse{22} Ya que esto ha llegado a
nuestro conocimiento, harás bien en escribirnos de tu
prosperidad.\footnote{\textbf{12:22} Gr. paz} \bibleverse{23} Además,
les escribimos de nuestra parte que su ganado y sus bienes son nuestros,
y los nuestros son suyos. Ordenamos, pues, que os informen de ello''.

\bibleverse{24} Jonatán se enteró de que los príncipes de Demetrio
habían regresado a luchar contra él con un ejército más numeroso que el
anterior, \bibleverse{25} así que marchó desde Jerusalén y les salió al
encuentro en el país de Hamat, pues no les dio oportunidad de poner el
pie en su país. \bibleverse{26} Envió espías a su campamento, los cuales
volvieron y le informaron que se preparaban para atacarlos por la noche.
\bibleverse{27} Pero en cuanto se puso el sol, Jonatán ordenó a sus
hombres que vigilaran y se armaran, para que durante toda la noche
estuvieran listos para la batalla. Colocó centinelas alrededor del
campamento. \bibleverse{28} Los adversarios oyeron que Jonatán y sus
hombres estaban listos para la batalla, y temieron y temblaron en sus
corazones, y encendieron fuego en su campamento y se retiraron.
\bibleverse{29} Pero Jonatán y sus hombres no lo supieron hasta la
mañana, pues vieron los fuegos encendidos. \bibleverse{30} Jonatán los
persiguió, pero no los alcanzó, porque habían pasado el río Eleutero.
\bibleverse{31} Entonces Jonatán se dirigió hacia los árabes, que se
llaman zabadeos, y los atacó y tomó su botín. \bibleverse{32} Salió de
allí, llegó a Damasco y recorrió todo el país.

\bibleverse{33} Salió Simón y emprendió su viaje hasta Ascalón y las
fortalezas que estaban cerca de ella. Luego se dirigió a Jope y se
apoderó de ella; \bibleverse{34} pues había oído que planeaban entregar
la fortaleza a los hombres de Demetrio. Puso allí una guarnición para
vigilarla.

\bibleverse{35} Entonces Jonatán regresó y convocó a los ancianos del
pueblo. Planeó con ellos construir fortalezas en Judea, \bibleverse{36}
y hacer más altas las murallas de Jerusalén, y levantar un gran
montículo entre la ciudadela y la ciudad, para separarla de la ciudad,
de modo que quedara aislada, y su guarnición no pudiera comprar ni
vender. \bibleverse{37} Se reunieron para construir la ciudad. Se
derrumbó parte del muro del arroyo que está en el lado oriental, y
reparó la parte llamada Chaphenatha. \bibleverse{38} Simón también
construyó Adida en la llanura de\footnote{\textbf{12:38} Gr. Sephela.} ,
la hizo fuerte y puso puertas y rejas.

\bibleverse{39} Trifón pretendía reinar sobre Asia y coronarse, y
extender su mano contra el rey Antíoco. \bibleverse{40} Temía que
Jonatán no se lo permitiera y que luchara contra él, y buscaba la manera
de apoderarse de él para destruirlo. Así que marchó y llegó a Betsán.
\bibleverse{41} Jonatán salió a su encuentro con cuarenta mil hombres
escogidos para la batalla, y llegó a Bet-sán. \bibleverse{42} Trifón vio
que venía con un gran ejército, y tuvo miedo de extender su mano contra
él. \bibleverse{43} Lo recibió honorablemente, y lo encomendó a todos
sus amigos de\footnote{\textbf{12:43} Ver 1 Macabeos 2:18.} , y le dio
regalos, y ordenó a sus fuerzas que le fueran obedientes, como a él
mismo. \bibleverse{44} Le dijo a Jonatán: ``¿Por qué has hecho sufrir
tanto a todo este pueblo, ya que no hay guerra entre nosotros?
\bibleverse{45} Ahora mándalos a sus casas, pero escoge para ti unos
cuantos hombres que te acompañen, y ven conmigo a Tolemaida, y te la
entregaré a ti, y al resto de las fortalezas y al resto de las fuerzas,
y a todos los oficiales del rey. Luego me daré la vuelta y partiré;
porque para eso he venido''.

\bibleverse{46} Puso su confianza en él, e hizo lo que había dicho, y
envió sus fuerzas, y partieron a la tierra de Judá. \bibleverse{47} Pero
se reservó tres mil hombres, de los cuales dejó dos mil en Galilea, pero
mil se fueron con él. \bibleverse{48} En cuanto Jonatán entró en
Tolemaida, el pueblo de Tolemaida cerró las puertas y lo apresó. Mataron
a espada a todos los que entraron con él.

\bibleverse{49} Trifón envió tropas y caballería a Galilea y a la Gran
Llanura para destruir a todos los hombres de Jonatán. \bibleverse{50} Se
dieron cuenta de que había sido apresado y había perecido, junto con los
que estaban con él. Se animaron mutuamente y siguieron su camino muy
juntos, dispuestos a luchar. \bibleverse{51} Los que los seguían vieron
que estaban dispuestos a luchar por sus vidas, y se volvieron de nuevo.
\bibleverse{52} Todos llegaron en paz a la tierra de Judá, y se
lamentaron por Jonatán y los que estaban con él, y tuvieron mucho miedo.
Todo Israel se enlutó con un gran duelo. \bibleverse{53} Y todos los
gentiles que estaban alrededor de ellos trataban de destruirlos por
completo, pues decían: ``No tienen jefe ni nadie que los ayude. Ahora,
pues, luchemos contra ellos, y quitemos su memoria de entre los
hombres''.

\hypertarget{section-12}{%
\section{13}\label{section-12}}

\bibleverse{1} Simón oyó que Trifón había reunido un poderoso ejército
para entrar en la tierra de Judá y destruirla por completo.
\bibleverse{2} Vio que el pueblo temblaba de miedo. Así que subió a
Jerusalén y reunió al pueblo. \bibleverse{3} Los animó y les dijo:
``Vosotros mismos sabéis todo lo que yo, mi familia y la casa de mi
padre hemos hecho por las leyes y el santuario, y las batallas y las
angustias que hemos visto. \bibleverse{4} Por eso, todos mis hermanos
han perecido por causa de Israel, y yo he quedado solo. \bibleverse{5}
Lejos de mí, pues, el perdonar mi propia vida en cualquier momento de
aflicción, ya que no soy mejor que mi parentela. \bibleverse{6} Sin
embargo, me vengaré por mi nación, por el santuario y por nuestras
esposas e hijos, porque todos los gentiles se han reunido por odio para
destruirnos.''

\bibleverse{7} El espíritu del pueblo revivió al oír estas palabras.
\bibleverse{8} Respondieron en voz alta, diciendo: ``Tú eres nuestro
jefe en lugar de Judas y de Jonatán, tus hermanos. \bibleverse{9} Libra
nuestras batallas y haremos todo lo que nos digas''.

\bibleverse{10} Reunió a todos los hombres de guerra y se apresuró a
terminar los muros de Jerusalén. La fortificó por todas partes.
\bibleverse{11} Envió a Jonatán, hijo de Absalón, y con él un gran
ejército, a Jope. Expulsó a los que estaban en ella, y vivió allí.

\bibleverse{12} Trifón salió de Tolemaida con un poderoso ejército para
entrar en la tierra de Judá, y Jonatán iba con él de guardia.
\bibleverse{13} Pero Simón acampó en Adida, cerca de la llanura.
\bibleverse{14} Trifón supo que Simón se había levantado en lugar de su
hermano Jonatán, y que tenía la intención de unirse a la batalla con él,
por lo que le envió embajadores, diciendo: \bibleverse{15} ``Es por el
dinero que Jonatán, tu hermano, debía al tesoro del rey, en razón de los
cargos que tenía, que lo estamos deteniendo. \bibleverse{16} Envía ahora
cien talentos de plata y dos de sus hijos como rehenes, para que cuando
sea liberado no se subleve contra nosotros, y lo liberaremos.''

\bibleverse{17} Simón sabía que le hablaban con engaño, pero envió a
buscar el dinero y los niños, para no despertar una gran hostilidad
entre el pueblo, \bibleverse{18} que dijera: ``Por no haberle enviado el
dinero y los niños, ha perecido.'' \bibleverse{19} Así que envió los
niños y los cien talentos, pero Trifón mintió y no liberó a Jonatán.

\bibleverse{20} Después de esto, Trifón vino a invadir la tierra y a
destruirla, y dio la vuelta por el camino que lleva a Adora. Simón y su
ejército marcharon cerca de él a todos los lugares, dondequiera que
fuera. \bibleverse{21} Los habitantes de la ciudadela enviaron a Trifón
embajadores, instándole a que se acercara a ellos por el desierto y les
enviara alimentos. \bibleverse{22} Así que Trifón preparó toda su
caballería para venir, pero esa noche cayó una nieve muy fuerte, y no
vino a causa de la nieve. Marchó y se internó en la tierra de Galaad.
\bibleverse{23} Cuando llegó cerca de Bascama, mató a Jonatán, y lo
enterraron allí. \bibleverse{24} Luego Trifón dio la vuelta y se fue a
su tierra.

\bibleverse{25} Simón envió y tomó los huesos de su hermano Jonatán y lo
enterró en Modín, la ciudad de sus padres. \bibleverse{26} Todo Israel
se lamentó mucho por él y lo lloró durante muchos días. \bibleverse{27}
Simón construyó un monumento sobre la tumba de su padre y de su familia,
y lo levantó en alto para que pudiera verse, con piedra pulida en el
frente y en la parte posterior. \bibleverse{28} También levantó siete
pirámides, una cerca de la otra, para su padre, su madre y sus cuatro
hermanos. \bibleverse{29} Para éstas, hizo un elaborado decorado,
erigiendo grandes pilares alrededor de ellas, y sobre los pilares hizo
trajes de armadura para un recuerdo perpetuo, y junto a los trajes de
armadura, talló barcos, para que pudieran ser vistos por todos los que
navegan por el mar. \bibleverse{30} Esta es la tumba que hizo en Modin.
Se ha conservado hasta el día de hoy.

\bibleverse{31} Trifón engañó al joven rey Antíoco y lo mató,
\bibleverse{32} y reinó en su lugar. Se puso la corona de Asia y trajo
una gran calamidad sobre el país. \bibleverse{33} Simón construyó las
fortalezas de Judea y las amuralló por todas partes con altas torres,
grandes muros, puertas y rejas; y almacenó alimentos en las fortalezas.
\bibleverse{34} Simón eligió hombres y envió al rey Demetrio con la
petición de que concediera al país una inmunidad, porque todo lo que
hacía Trifón era saquear. \bibleverse{35} El rey Demetrio le envió según
estas palabras, le respondió y le escribió una carta como sigue

\bibleverse{36} ``Rey Demetrio a Simón el sumo sacerdote y
amigo\footnote{\textbf{13:36} Ver 1 Macabeos 2:18.} de los reyes, y a
los ancianos y a la nación de los judíos, saludos. \bibleverse{37} Hemos
recibido la corona de oro y la rama de palma que nos enviaste. Estamos
dispuestos a hacer una paz firme con vosotros, sí, y a escribir a
nuestros oficiales para que os liberen de los tributos. \bibleverse{38}
Todo lo que os confirmamos, está confirmado. Las fortalezas que habéis
construido, que sean vuestras. \bibleverse{39} En cuanto a los descuidos
y faltas cometidos hasta hoy, los perdonamos, así como el impuesto de la
corona que nos debías. Si se cobraba algún otro impuesto en Jerusalén,
que no se cobre más. \bibleverse{40} Si alguno de vosotros está
capacitado para inscribirse en nuestro tribunal, que se inscriba, y que
haya paz entre nosotros.''

\bibleverse{41} En el año ciento setenta,\footnote{\textbf{13:41} hacia
  el año 143 a. C.} el yugo de los gentiles fue quitado de Israel.
\bibleverse{42} El pueblo comenzó a escribir en sus instrumentos y
contratos: ``En el primer año de Simón, el gran sumo sacerdote y capitán
y líder de los judíos''.

\bibleverse{43} En aquellos días, Simón acampó contra\footnote{\textbf{13:43}
  Ver 1 Macabeos 13:53 (comparar 1 Macabeos 13:48); 1 Macabeos 14:7:34;
  15:28; 16:1: también Josefo. Todas las autoridades leen Gaza en este
  versículo.} Gazara, y la rodeó con tropas. Hizo una máquina de asedio,
la llevó hasta la ciudad, golpeó una torre y la capturó. \bibleverse{44}
Los que estaban en la máquina saltaron a la ciudad, y hubo un gran
alboroto en la ciudad. \bibleverse{45} Los habitantes de la ciudad se
rasgaron las vestiduras y subieron a las murallas con sus mujeres e
hijos, y gritaron a viva voz, pidiendo a Simón que les diera\footnote{\textbf{13:45}
  Gr. Manos derechas.} su mano derecha. \bibleverse{46} Dijeron: ``No
nos trates según nuestras maldades, sino según tu misericordia''.
\bibleverse{47} Así que Simón se reconcilió con ellos y no luchó contra
ellos, sino que los expulsó de la ciudad y limpió las casas donde
estaban los ídolos, y entró en ella cantando y dando alabanzas.
\bibleverse{48} Quitó de ella toda impureza, puso en ella hombres que
guardaran la ley, la fortaleció más que antes y se edificó en ella una
morada.

\bibleverse{49} Pero a la gente de la ciudadela de Jerusalén se le
impidió salir e ir al campo, y comprar y vender. Así que pasaron mucha
hambre, y un gran número de ellos pereció de hambre. \bibleverse{50}
Entonces clamaron a Simón para que les diera su mano derecha, y él se la
dio; pero los expulsó de allí, y limpió la ciudadela de sus
contaminaciones. \bibleverse{51} Entró en ella el día veintitrés del
segundo mes, en el año ciento setenta y uno,\footnote{\textbf{13:51}
  hacia el año 142 a. C.} con alabanzas y ramas de palma, con arpas, con
címbalos y con instrumentos de cuerda, con himnos y con canciones,
porque un gran enemigo había sido destruido de Israel. \bibleverse{52}
Simón ordenó que celebraran ese día todos los años con alegría. Hizo más
fuerte que antes la colina del templo que estaba junto a la ciudadela, y
vivió allí con sus hombres. \bibleverse{53} Simón vio que su hijo Juan
era un hombre, así que lo nombró jefe de todas sus fuerzas; y vivió en
Gázara.

\hypertarget{section-13}{%
\section{14}\label{section-13}}

\bibleverse{1} En el año ciento setenta y dos,\footnote{\textbf{14:1}
  hacia el año 141 a. C.} el rey Demetrio reunió sus fuerzas y fue a
Media a buscar ayuda para luchar contra Trifón. \bibleverse{2} Cuando
Arsaces, rey de Persia y de Media, se enteró de que Demetrio había
entrado en sus fronteras, envió a uno de sus príncipes para que lo
capturara vivo. \bibleverse{3} Este fue a atacar al ejército de
Demetrio, lo apresó y lo llevó a Arsaces, quien lo puso bajo vigilancia.

\bibleverse{4} La tierra tuvo descanso todos los días de Simón. Buscó el
bien de su nación. Su autoridad y su honor le fueron gratos todos sus
días. \bibleverse{5} En medio de todos sus honores, tomó Jope como
puerto y lo convirtió en una entrada para las islas del mar.
\bibleverse{6} Amplió las fronteras de su nación y tomó posesión del
país. \bibleverse{7} Reunió un gran número de cautivos y se apoderó de
Gázara, Betsura y la ciudadela, y eliminó de ella sus impurezas. No hubo
nadie que se le resistiera. \bibleverse{8} Cultivaron su tierra en paz,
y la tierra dio su cosecha, y los árboles de las llanuras dieron su
fruto. \bibleverse{9} Los ancianos se sentaban en las calles; todos
conversaban juntos sobre cosas buenas. Los jóvenes se vistieron con
ropas gloriosas y guerreras. \bibleverse{10} Él proveyó de alimentos a
las ciudades y las dotó de medios de defensa, hasta que la gloria de su
nombre fue conocida hasta el fin de la tierra. \bibleverse{11} Hizo la
paz en la tierra, e Israel se regocijó con gran alegría. \bibleverse{12}
Cada uno se sentaba bajo su vid y su higuera, y no había nadie que los
atemorizara. \bibleverse{13} No quedó nadie en la tierra que luchara
contra ellos. Los reyes fueron derrotados en aquellos días.
\bibleverse{14} Fortaleció a todos los de su pueblo que eran humildes.
Buscó la ley, y a todo inicuo y malvado lo eliminó. \bibleverse{15}
Glorificó el santuario y aumentó los vasos del templo.

\bibleverse{16} Se oyó en Roma que Jonatán había muerto, e incluso en
Esparta, y se entristecieron mucho. \bibleverse{17} Pero en cuanto se
enteraron de que su hermano Simón había sido nombrado sumo sacerdote en
su lugar, y gobernaba el país y las ciudades en él, \bibleverse{18} le
escribieron en tablas de bronce para renovar con él la amistad y la
alianza que habían confirmado con sus hermanos Judas y Jonatán.
\bibleverse{19} Éstas fueron leídas ante la congregación de Jerusalén.

\bibleverse{20} Esta es la copia de las cartas que enviaron los
espartanos:

``a Simón, el sumo sacerdote, a los ancianos, a los sacerdotes y al
resto del pueblo de los judíos, nuestros parientes, saludos.
\bibleverse{21} Los embajadores que fueron enviados a nuestro pueblo nos
informaron de vuestra gloria y honor. Nos alegramos de su venida,
\bibleverse{22} y registramos lo dicho por ellos en los registros
públicos\footnote{\textbf{14:22} Gr. consejos del pueblo.} como sigue:
`Numenio hijo de Antíoco y Antípatro hijo de Jasón, embajadores de los
judíos, vinieron a nosotros para renovar la amistad que tenían con
nosotros. \bibleverse{23} El pueblo se complació en agasajar a los
hombres honorablemente y en poner la copia de sus palabras en los
registros públicos,\footnote{\textbf{14:23} Gr. los libros que se
  designan para el pueblo.} para que el pueblo de los espartanos tuviera
constancia de ellas. Además, escribieron una copia de estas cosas a
Simón, el sumo sacerdote''.

\bibleverse{24} Después de esto, Simón envió a Numenio a Roma con un
gran escudo de oro de mil minas de peso,\footnote{\textbf{14:24} 1.000
  minas son unos 499 kg o 1.098 libras.} para confirmar la alianza con
ellos.

\bibleverse{25} Pero cuando el pueblo oyó estas cosas, dijo: ``¿Qué
agradecimiento debemos dar a Simón y a sus hijos? \bibleverse{26} Porque
él y sus hermanos y la casa de su padre se han hecho fuertes, y han
combatido y ahuyentado a los enemigos de Israel, y han confirmado la
libertad de Israel.\footnote{\textbf{14:26} Ver 1 Macabeos 2:18.} ''
\bibleverse{27} Y escribieron en tablas de bronce, y las pusieron sobre
columnas en el monte Sión. Esta es la copia de la escritura:

``El día dieciocho de Elul, en el año ciento setenta y dos, que es el
tercer año del sumo sacerdote Simón, \bibleverse{28} en Asaramel, en una
gran congregación de sacerdotes y pueblo y príncipes de la nación, y de
los ancianos del país, se nos proclamó \bibleverse{29} `Como a menudo
había guerras en el país, Simón, hijo de Matatías, hijo de los hijos de
Joarib, y sus hermanos, se pusieron en peligro y resistieron a los
enemigos de su nación, para que su santuario y la ley fueran
establecidos, y glorificaron a su nación con gran gloria.
\bibleverse{30} Jonatán reunió a la nación, se convirtió en su sumo
sacerdote y se reunió con su pueblo. \bibleverse{31} Sus enemigos
planeaban invadir su país, para destruirlo por completo y extender sus
manos contra su santuario. \bibleverse{32} Entonces Simón se levantó y
luchó por su nación. Gastó mucho de su propio dinero para armar a los
valientes de su nación y darles un salario. \bibleverse{33} Fortificó
las ciudades de Judea y Betsurá, que está en las fronteras de Judea,
donde se habían almacenado las armas de los enemigos, y allí puso una
guarnición de judíos. \bibleverse{34} Fortificó Jope, que está a orillas
del mar, y Gázara, que está en los límites de Azoto, donde vivían los
enemigos, y colocó allí judíos y puso todo lo necesario para su
restauración. \bibleverse{35} El pueblo vio la fe de Simón, y la gloria
que resolvió dar a su nación, y lo hicieron su jefe y sumo sacerdote,
por haber hecho todas estas cosas, y por la justicia y la fe que
guardaba para su nación, y porque procuraba por todos los medios
enaltecer a su pueblo. \bibleverse{36} En sus días, las cosas
prosperaron en sus manos, de modo que los gentiles fueron sacados de su
país, y también los que estaban en la ciudad de David, los que estaban
en Jerusalén, que se habían hecho una ciudadela, de la que solían salir,
y contaminaban todo lo que rodeaba al santuario, y hacían gran daño a su
pureza. \bibleverse{37} Colocó a los judíos en ella y la fortificó para
la seguridad del país y de la ciudad, e hizo altas las murallas de
Jerusalén. \bibleverse{38} El rey Demetrio le confirmó el sumo
sacerdocio de acuerdo con estas cosas, \bibleverse{39} y lo hizo uno de
sus amigos, y lo honró con grandes honores; \bibleverse{40} pues había
oído que los judíos habían sido llamados por los romanos amigos, aliados
y afines, y que habían recibido honorablemente a los embajadores de
Simón \bibleverse{41} y que a los judíos y a los sacerdotes les parecía
bien que Simón fuera su jefe y sumo sacerdote para siempre, hasta que
surgiera un profeta fiel; \bibleverse{42} y que fuera gobernador sobre
ellos, y que se hiciera cargo del santuario, para ponerlos al frente de
sus obras, y del país, y de las armas, y de las fortalezas y que se
hiciera cargo del santuario, \bibleverse{43} y que fuera obedecido por
todos, y que todos los contratos en el país se escribieran a su nombre,
y que se vistiera de púrpura y llevara oro; \bibleverse{44} y que no le
sea lícito a ningún miembro del pueblo o de los sacerdotes anular
ninguna de estas cosas, ni oponerse a las palabras que él pronuncie, ni
reunir una asamblea en el país sin él, ni vestirse de púrpura, ni llevar
una hebilla de oro; \bibleverse{45} pero quien haga lo contrario, o
anule cualquiera de estas cosas, será castigado.'\,'' \bibleverse{46}
Todo el pueblo consintió en ordenar a Simón que hiciera lo que decían
estas palabras. \bibleverse{47} Así que Simón aceptó esto, y consintió
en ser sumo sacerdote, y en ser capitán y gobernador\footnote{\textbf{14:47}
  Gr. etnarca.} de los judíos y de los sacerdotes, y en ser protector de
todos.

\bibleverse{48} Mandaron poner este escrito en tablas de bronce, y
colocarlo en el recinto del santuario en un lugar visible,
\bibleverse{49} y además poner copias de ellas en el tesoro, para que
Simón y sus hijos las tuvieran.

\hypertarget{section-14}{%
\section{15}\label{section-14}}

\bibleverse{1} Antíoco, hijo del rey Demetrio, envió una carta desde las
islas del mar al sacerdote Simón y a\footnote{\textbf{15:1} Gr. etnarca.}
, gobernador de los judíos, y a toda la nación. \bibleverse{2} Su
contenido es el siguiente:

``Rey Antíoco a Simón el sumo sacerdote y al gobernador\footnote{\textbf{15:2}
  Véase 1 Macabeos 2:18.} , y a la nación de los judíos, saludos.
\bibleverse{3} Considerando que algunos alborotadores se han hecho
dueños del reino de nuestros padres, pero mi propósito es reclamar el
reino, para restaurarlo como era antes; y además he levantado una
multitud de soldados extranjeros, y he preparado barcos de guerra;
\bibleverse{4} además pienso desembarcar en el país, para castigar a los
que han destruido nuestro país, y a los que han hecho desoladas muchas
ciudades del reino; \bibleverse{5} ahora, por tanto, te confirmo todas
las remisiones de impuestos que te remitieron los reyes que me
precedieron, y cualquier otro regalo que te hayan remitido,
\bibleverse{6} y te permito acuñar moneda para tu país con tu propio
sello, \bibleverse{7} pero que Jerusalén y el santuario sean libres.
Todas las armas que habéis preparado y las fortalezas que habéis
construido, que tenéis en vuestro poder, que sigan siendo vuestras.
\bibleverse{8} Todas las deudas contraídas con el rey, y las que se
contraigan con el rey desde ahora y para siempre, que se os condonen.
\bibleverse{9} Además, cuando hayamos establecido nuestro reino, te
glorificaremos a ti, a tu nación y al templo con gran gloria, para que
tu gloria se manifieste en toda la tierra.

\bibleverse{10} En el año ciento setenta y cuatro, Antíoco entró en la
tierra de sus padres; y todas las fuerzas se reunieron con él, de modo
que había pocos hombres con Trifón. \bibleverse{11} El rey Antíoco lo
persiguió y llegó, en su huida, a Dor, que está junto al mar;
\bibleverse{12} pues sabía que los problemas le habían sobrevenido de
golpe, y que sus fuerzas lo habían abandonado. \bibleverse{13} Antíoco
acampó contra Dor, y con él ciento veinte mil hombres de guerra y ocho
mil de caballería. \bibleverse{14} Rodeó la ciudad y los barcos se
unieron al ataque desde el mar. Acosó la ciudad por tierra y por mar, y
no permitió a nadie salir ni entrar.

\bibleverse{15} Numenio y su compañía vinieron de Roma con cartas para
los reyes y los países, en las que estaban escritas estas cosas:

\bibleverse{16} ``Lucio, cónsul de los romanos, al rey Ptolomeo,
saludos. \bibleverse{17} Los embajadores de los judíos vinieron a
nosotros como amigos y aliados nuestros, para renovar la antigua amistad
y alianza, enviados por el sumo sacerdote Simón y por el pueblo de los
judíos. \bibleverse{18} Además, trajeron un escudo de oro que pesaba mil
minas. \bibleverse{19} Nos pareció bien, pues, escribir a los reyes y a
los países para que no buscasen su daño ni luchasen contra ellos, sus
ciudades y su país, ni se aliasen con los que luchan contra ellos.
\bibleverse{20} Además, nos pareció bien recibir el escudo de ellos.
\bibleverse{21} Por lo tanto, si algunos alborotadores han huido de su
país hacia ustedes, entréguenlos al sumo sacerdote Simón, para que se
vengue de ellos según su ley.''

\bibleverse{22} Lo mismo escribió al rey Demetrio, a Atalo, a Aratos, a
Arsaces, \bibleverse{23} a todos los países, a Sampsames, a los
espartanos, a Delos, a Myndos, a Sicíone, a Caria, a Samos, a Panfilia,
a Licia, a Halicarnaso, a Rodas, a Phaselis, a Cos, a Side, a Arado, a
Gortyna, a Cnidus, a Chipre y a Cirene. \bibleverse{24} También
escribieron esta copia al sumo sacerdote Simón.

\bibleverse{25} Pero el rey Antíoco acampó contra Dor el segundo día,
trayendo continuamente sus fuerzas y haciendo máquinas de guerra; y le
impidió a Trifón entrar o salir. \bibleverse{26} Simón le envió dos mil
hombres escogidos para luchar de su lado, con plata, oro e instrumentos
de guerra en abundancia. \bibleverse{27} No quiso recibirlos, sino que
anuló todos los pactos que había hecho antes con él y se alejó de él.
\bibleverse{28} Le envió a Ateneo, uno de sus amigos, para que
consultara con él, diciéndole: ``Estás en posesión de Jope, Gázara y la
ciudadela que está en Jerusalén, ciudades de mi reino. \bibleverse{29}
Has devastado su territorio y has hecho un gran daño en la tierra y el
control de muchos lugares de mi reino. \bibleverse{30} Ahora, pues,
entregadme las ciudades que habéis tomado y los tributos de los lugares
que habéis dominado fuera de los límites de Judea; \bibleverse{31} o
bien dadme por ellos quinientos talentos de plata; y por el daño que
habéis hecho y los tributos de las ciudades, otros quinientos talentos.
De lo contrario, iremos y os someteremos''.

\bibleverse{32} Ateneo, amigo del rey,\footnote{\textbf{15:32} Ver 1
  Macabeos 2:18.} llegó a Jerusalén. Cuando vio la gloria de Simón, la
alacena de vasos de oro y plata, y su gran asistencia, se quedó
asombrado. Le informó de las palabras del rey. \bibleverse{33} Simón
respondió y le dijo: ``No hemos tomado tierras ajenas ni tenemos
posesión de lo que es de otros, sino de la herencia de nuestros padres.
Sin embargo, había estado en posesión de nuestros enemigos injustamente
durante un tiempo. \bibleverse{34} Pero nosotros, teniendo la
oportunidad, mantenemos firmemente la herencia de nuestros padres.
\bibleverse{35} En cuanto a Jope y Gázara, que ustedes exigen, hicieron
gran daño entre el pueblo de todo nuestro país. Daremos cien talentos
por ellas''.

Ateneo no respondió ni una sola palabra, \bibleverse{36} sino que
regresó furioso al rey, y le informó de estas palabras, de la gloria de
Simón y de todo lo que había visto; y el rey se enfadó mucho.
\bibleverse{37} Mientras tanto, Trifón se embarcó en una nave y huyó a
Ortosia.

\bibleverse{38} El rey nombró a Cendebaeus jefe de la costa del mar, y
le dio tropas de infantería y de caballería. \bibleverse{39} Le mandó
acampar contra Judea, y le ordenó que construyera el Cedrón y
fortificara las puertas, y que luchara contra el pueblo; pero el rey
persiguió a Trifón. \bibleverse{40} Entonces Cendebao llegó a Jamnia y
comenzó a provocar al pueblo, a invadir Judea y a llevar cautivo al
pueblo y a matarlo. \bibleverse{41} Edificó Cedrón y estacionó allí la
caballería y la infantería, a fin de que al salir hicieran incursiones
en los caminos de Judea, como el rey le había ordenado.

\hypertarget{section-15}{%
\section{16}\label{section-15}}

\bibleverse{1} Juan subió de Gázara y le contó a Simón, su padre, lo que
hacía Cendebaeus. \bibleverse{2} Simón llamó a sus dos hijos mayores,
Judas y Juan, y les dijo: ``Yo y mis hermanos y la casa de mi padre
hemos librado las batallas de Israel desde nuestra juventud, hasta el
día de hoy; y las cosas han prosperado en nuestras manos, que muchas
veces hemos librado a Israel. \bibleverse{3} Pero ahora yo soy viejo, y
tú además, por su misericordia, tienes edad suficiente. Tomad mi lugar y
el de mi hermano, y salid a luchar por nuestra nación; y que la ayuda
que viene del cielo os acompañe.

\bibleverse{4} Escogió del país veinte mil hombres de guerra y de
caballería, y fueron contra Cendebaeus, y durmieron en Modin.
\bibleverse{5} Al levantarse por la mañana, salieron a la llanura, y he
aquí que un gran ejército de infantería y de caballería les salió al
encuentro. Había un arroyo entre ellos. \bibleverse{6} Acampó cerca de
ellos, él y su gente. Vio que la gente tenía miedo de pasar por el
arroyo, y él pasó primero. Cuando los hombres lo vieron, pasaron tras
él. \bibleverse{7} Dividió al pueblo y colocó a la caballería en medio
de la infantería; pero la caballería de los enemigos era muy numerosa.
\bibleverse{8} Hicieron sonar las trompetas, y Cendebaeus y su ejército
fueron puestos en fuga, y muchos de ellos cayeron heridos de muerte,
pero los que quedaron huyeron a la fortaleza. \bibleverse{9} En ese
momento, Judas, el hermano de Juan, fue herido; pero Juan los persiguió
hasta llegar al Cedrón, que Cendebaeus había construido. \bibleverse{10}
Ellos huyeron a las torres que están en los campos de Azoto, y él las
quemó con fuego. Cayeron unos dos mil hombres de ellos. Luego regresó a
Judea en paz.

\bibleverse{11} Tolomeo, hijo de Abubo, había sido nombrado gobernador
de la llanura de Jericó, y tenía mucha plata y oro; \bibleverse{12} pues
era yerno del sumo sacerdote. \bibleverse{13} Su corazón se enalteció, y
planeó hacerse dueño del país, e hizo planes engañosos contra Simón y
sus hijos, para acabar con ellos. \bibleverse{14} Simón visitaba las
ciudades del país y atendía sus necesidades. Bajó a Jericó --- con
Matatías y Judas, sus hijos --- en el año ciento setenta y
siete,\footnote{\textbf{16:14} hacia el año 136 a. C.} en el mes
undécimo, que es el mes de Sebat. \bibleverse{15} El hijo de Abubus los
recibió con engaño en la pequeña fortaleza que se llama Dok, que él
había construido, y les hizo un gran banquete, y escondió allí a los
hombres. \bibleverse{16} Cuando Simón y sus hijos bebieron libremente,
Tolomeo y sus hombres se levantaron, tomaron sus armas, se abalanzaron
contra Simón en el lugar del banquete y lo mataron a él, a sus dos hijos
y a algunos de sus servidores. \bibleverse{17} Cometió una gran
iniquidad y pagó mal por bien.

\bibleverse{18} Ptolomeo escribió estas cosas y envió al rey para que le
enviara fuerzas para ayudarle y le entregara su país y las ciudades.
\bibleverse{19} Envió a otros a Gázara para que acabaran con Juan. A los
capitanes de millares les envió cartas para que vinieran a él, a fin de
darles plata, oro y regalos. \bibleverse{20} Envió a otros a tomar
posesión de Jerusalén y del monte del templo. \bibleverse{21} Uno corrió
antes a Gázara y le dijo a Juan que su padre y su parentela habían
perecido, y que él había enviado a matarte también a ti. \bibleverse{22}
Cuando se enteró, se escandalizó mucho. Agarró a los hombres que venían
a destruirlo y los mató, porque se dio cuenta de que querían destruirlo.

\bibleverse{23} Y el resto de los hechos de Juan y de sus guerras y de
sus hechos valerosos que hizo, y de la construcción de los muros que
edificó, y de sus logros, \bibleverse{24} he aquí, están escritos en las
crónicas\footnote{\textbf{16:24} Gr. libro de los días.} de su sumo
sacerdocio, desde el tiempo en que fue hecho sumo sacerdote después de
su padre.
