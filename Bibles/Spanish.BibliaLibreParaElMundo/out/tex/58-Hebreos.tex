\hypertarget{la-soberanuxeda-uxfanica-del-hijo-de-dios-sobre-los-mensajeros-de-dios-del-antiguo-testamento}{%
\subsection{La soberanía única del Hijo de Dios sobre los mensajeros de
Dios del Antiguo
Testamento}\label{la-soberanuxeda-uxfanica-del-hijo-de-dios-sobre-los-mensajeros-de-dios-del-antiguo-testamento}}

\hypertarget{section}{%
\section{1}\label{section}}

\bibleverse{1} Dios, habiendo hablado en el pasado a los padres por
medio de los profetas en muchas ocasiones y de diversas maneras,
\bibleverse{2} al final de estos días nos ha hablado por medio de su
Hijo, a quien nombró heredero de todas las cosas, por quien también hizo
los mundos. \footnote{\textbf{1:2} Sal 2,8; Juan 1,3; Col 1,16}
\bibleverse{3} Su Hijo es el resplandor de su gloria, la imagen misma de
su sustancia, y sostiene todas las cosas con la palabra de su poder, el
cual, después de habernos purificado por sí mismo de nuestros pecados,
se sentó a la derecha de la Majestad en las alturas, \footnote{\textbf{1:3}
  2Cor 4,4; Col 1,15; Heb 9,14; Heb 9,26; Mar 16,19} \bibleverse{4}
habiendo llegado a ser tan mejor que los ángeles como el nombre más
excelente que ha heredado es mejor que el de ellos. \footnote{\textbf{1:4}
  Fil 2,9; 1Pe 3,22}

\hypertarget{evidencia-del-antiguo-testamento-de-la-exaltaciuxf3n-del-hijo-de-dios-sobre-los-uxe1ngeles}{%
\subsection{Evidencia del Antiguo Testamento de la exaltación del Hijo
de Dios sobre los
ángeles}\label{evidencia-del-antiguo-testamento-de-la-exaltaciuxf3n-del-hijo-de-dios-sobre-los-uxe1ngeles}}

\bibleverse{5} Porque ¿a cuál de los ángeles dijo en algún momento, ``Tú
eres mi Hijo. ¿Hoy me he convertido en tu padre?''y otra vez, ``Seré
para él un Padre, y será para mí un Hijo?''

\bibleverse{6} Cuando vuelve a traer al primogénito al mundo dice: ``Que
todos los ángeles de Dios lo adoren''. \footnote{\textbf{1:6} Rom 8,29}
\bibleverse{7} De los ángeles dice, ``Hace vientos a sus ángeles, y sus
siervos una llama de fuego''.

\bibleverse{8} Pero del Hijo dice, ``Tu trono, oh Dios, es por los
siglos de los siglos. El cetro de la rectitud es el cetro de tu Reino.
\bibleverse{9} Has amado la justicia y odiado la iniquidad; por eso
Dios, tu Dios, te ha ungido con el aceite de la alegría por encima de
tus compañeros''.

\bibleverse{10} Y, ``Tú, Señor, en el principio, pusiste los cimientos
de la tierra. Los cielos son obra de tus manos. \bibleverse{11} Ellos
perecerán, pero tú continúas. Todos ellos envejecerán como lo hace una
prenda de vestir. \bibleverse{12} Los enrollarás como un manto, y serán
cambiados; pero tú eres el mismo. Tus años no fallarán''.

\bibleverse{13} Pero a cuál de los ángeles le ha dicho en algún momento,
``Siéntate a mi derecha, hasta que haga de tus enemigos el escabel de
tus pies?''

\bibleverse{14} ¿No son todos ellos espíritus servidores, enviados a
hacer un servicio por el bien de los que heredarán la salvación?
\footnote{\textbf{1:14} Sal 34,7; Sal 91,11-12}

\hypertarget{de-ahuxed-surge-la-obligaciuxf3n-de-que-obedezcamos-voluntariamente-las-palabras-que-nos-ha-dicho-este-hijo}{%
\subsection{De ahí surge la obligación de que obedezcamos
voluntariamente las palabras que nos ha dicho este
Hijo}\label{de-ahuxed-surge-la-obligaciuxf3n-de-que-obedezcamos-voluntariamente-las-palabras-que-nos-ha-dicho-este-hijo}}

\hypertarget{section-1}{%
\section{2}\label{section-1}}

\bibleverse{1} Por lo tanto, debemos prestar más atención a las cosas
que se escucharon, para que no nos desviemos. \bibleverse{2} Porque si
la palabra hablada por medio de los ángeles resultó firme, y toda
transgresión y desobediencia recibió un justo castigo, \footnote{\textbf{2:2}
  Hech 7,53; Gal 3,19} \bibleverse{3} ¿cómo escaparemos nosotros si
descuidamos una salvación tan grande, la cual, habiendo sido hablada al
principio por medio del Señor, nos fue confirmada por los que oyeron,
\footnote{\textbf{2:3} Heb 10,29; Heb 12,25} \bibleverse{4} testificando
Dios también con ellos, tanto por señales como por prodigios, por
diversas obras de poder y por dones del Espíritu Santo, según su propia
voluntad? \footnote{\textbf{2:4} Mar 16,20; 1Cor 12,4-11; 2Cor 12,12;
  Hech 1,2-13; Hech 10,44-45}

\hypertarget{su-humillaciuxf3n-encarnaciuxf3n-y-sufrimiento-de-muerte-no-limita-su-sublimidad}{%
\subsection{Su humillación, encarnación y sufrimiento de muerte, no
limita su
sublimidad}\label{su-humillaciuxf3n-encarnaciuxf3n-y-sufrimiento-de-muerte-no-limita-su-sublimidad}}

\bibleverse{5} Porque no sometió a los ángeles el mundo venidero, del
que hablamos. \bibleverse{6} Pero uno ha testificado en alguna parte,
diciendo, ``¿Qué es el hombre, para que pienses en él? ¿O el hijo del
hombre, que se preocupa por él? \bibleverse{7} Lo hiciste un poco más
bajo que los ángeles. Lo coronaste de gloria y honor. \bibleverse{8} Has
sometido todas las cosas bajo sus pies''. Porque al someter todas las
cosas a él, no dejó nada que no le estuviera sometido. Pero ahora
todavía no vemos todas las cosas sometidas a él. \bibleverse{9} Pero
vemos al que ha sido hecho un poco más bajo que los ángeles, Jesús, a
causa del sufrimiento de la muerte, coronado de gloria y honor, para que
por la gracia de Dios probara la muerte por todos. \footnote{\textbf{2:9}
  Fil 2,8-9}

\hypertarget{la-necesidad-de-la-humillaciuxf3n-especialmente-el-sufrimiento-de-la-muerte}{%
\subsection{La necesidad de la humillación, especialmente el sufrimiento
de la
muerte}\label{la-necesidad-de-la-humillaciuxf3n-especialmente-el-sufrimiento-de-la-muerte}}

\bibleverse{10} Porque convenía a aquel por quien son todas las cosas y
por quien son todas las cosas, al llevar a muchos niños a la gloria,
perfeccionar por aflicciones al autor de la salvación de ellos.
\footnote{\textbf{2:10} Heb 12,2} \bibleverse{11} Porque tanto el que
santifica como los santificados proceden todos de uno, por lo que no se
avergüenza de llamarlos hermanos, \footnote{\textbf{2:11} Mar 3,34-35;
  Juan 17,19; Juan 20,17} \bibleverse{12} diciendo, ``Declararé tu
nombre a mis hermanos. Entre la congregación cantaré tu alabanza''.

\bibleverse{13} De nuevo: ``Pondré mi confianza en él''. De nuevo: ``He
aquí que estoy con los hijos que Dios me ha dado''.

\hypertarget{las-beneficiosas-consecuencias-de-la-humillaciuxf3n}{%
\subsection{Las beneficiosas consecuencias de la
humillación}\label{las-beneficiosas-consecuencias-de-la-humillaciuxf3n}}

\bibleverse{14} Puesto que los hijos participaron de la carne y de la
sangre, también él participó de lo mismo, para anular por medio de la
muerte al que tenía el poder de la muerte, es decir, al diablo,
\footnote{\textbf{2:14} 2Tim 1,10; 1Jn 3,8} \bibleverse{15} y liberar a
todos los que, por temor a la muerte, estaban durante toda su vida
sujetos a esclavitud. \bibleverse{16} Porque ciertamente, no da ayuda a
los ángeles, sino que da ayuda a la descendencia de Abraham.
\bibleverse{17} Por eso estaba obligado en todo a hacerse semejante a
sus hermanos, para llegar a ser un sumo sacerdote misericordioso y fiel
en las cosas de Dios, para expiar los pecados del pueblo. \footnote{\textbf{2:17}
  Fil 2,7} \bibleverse{18} Porque habiendo sufrido él mismo la
tentación, puede ayudar a los que son tentados. \footnote{\textbf{2:18}
  Heb 4,15}

\hypertarget{el-hijo-de-dios-jesuxfas-en-su-majestad-sobre-el-ministro-de-dios-moisuxe9s}{%
\subsection{El Hijo de Dios Jesús en su majestad sobre el ministro de
Dios
Moisés}\label{el-hijo-de-dios-jesuxfas-en-su-majestad-sobre-el-ministro-de-dios-moisuxe9s}}

\hypertarget{section-2}{%
\section{3}\label{section-2}}

\bibleverse{1} Por tanto, santos hermanos, partícipes de una vocación
celestial, considerad al Apóstol y Sumo Sacerdote de nuestra confesión:
Jesús, \footnote{\textbf{3:1} Heb 4,14} \bibleverse{2} que fue fiel al
que lo designó, como también lo fue Moisés en toda su casa. \footnote{\textbf{3:2}
  Núm 12,7} \bibleverse{3} Pues ha sido considerado digno de más gloria
que Moisés, porque el que construyó la casa tiene más honor que la casa.
\bibleverse{4} Porque toda casa es construida por alguien; pero el que
construyó todas las cosas es Dios. \bibleverse{5} Moisés, en efecto, fue
fiel en toda su casa como siervo, para dar testimonio de lo que después
se iba a decir, \bibleverse{6} pero Cristo es fiel como Hijo sobre su
casa. Nosotros somos su casa, si mantenemos firme nuestra confianza y la
gloria de nuestra esperanza hasta el fin. \footnote{\textbf{3:6} 1Pe
  2,5; Efes 2,19}

\hypertarget{la-advertencia-del-salmista-contra-la-incredulidad-y-la-apostasuxeda}{%
\subsection{La advertencia del salmista contra la incredulidad y la
apostasía}\label{la-advertencia-del-salmista-contra-la-incredulidad-y-la-apostasuxeda}}

\bibleverse{7} Por tanto, como dice el Espíritu Santo, ``Hoy, si
escuchas su voz, \footnote{\textbf{3:7} Heb 4,7} \bibleverse{8} no
endurezcáis vuestros corazones como en la rebelión, en el día de la
prueba en el desierto, \footnote{\textbf{3:8} Éxod 17,7; Núm 20,2-5}
\bibleverse{9} donde tus padres me pusieron a prueba y me probaron, y
vio mis actos durante cuarenta años. \bibleverse{10} Por eso me disgusté
con esa generación, y dijo: ``Siempre se equivocan en su corazón, pero
no conocían mis costumbres'. \bibleverse{11} Como juré en mi ira,`No
entrarán en mi descanso'\,''. \footnote{\textbf{3:11} Heb 4,3; Núm
  14,21-23}

\bibleverse{12} Cuidado, hermanos, no sea que haya en alguno de vosotros
un mal corazón de incredulidad, apartándose del Dios vivo;
\bibleverse{13} sino que os exhortéis unos a otros de día en día,
mientras se llame ``hoy'', no sea que alguno de vosotros se endurezca
por el engaño del pecado. \footnote{\textbf{3:13} 1Tes 5,11}

\hypertarget{el-ejemplo-de-advertencia-de-los-israelitas-en-el-desierto}{%
\subsection{El ejemplo de advertencia de los israelitas en el
desierto}\label{el-ejemplo-de-advertencia-de-los-israelitas-en-el-desierto}}

\bibleverse{14} Porque hemos llegado a ser partícipes de Cristo, si
mantenemos firme el principio de nuestra confianza hasta el fin,
\footnote{\textbf{3:14} Heb 6,11} \bibleverse{15} mientras se dice,
``Hoy, si escuchas su voz, no endurezcáis vuestros corazones, como en la
rebelión''.

\bibleverse{16} Porque ¿quiénes, al oírlo, se rebelaron? ¿No fueron
todos los que salieron de Egipto guiados por Moisés? \bibleverse{17}
¿Con quiénes se disgustó durante cuarenta años? ¿No fue con los que
pecaron, cuyos cuerpos cayeron en el desierto? \footnote{\textbf{3:17}
  Núm 14,29; 1Cor 10,10} \bibleverse{18} ¿A quiénes juró que no
entrarían en su reposo, sino a los desobedientes? \bibleverse{19} Vemos
que no pudieron entrar a causa de la incredulidad.

\hypertarget{interpretaciuxf3n-de-la-promesa-del-salmo-sobre-el-resto-del-pueblo-de-dios}{%
\subsection{Interpretación de la promesa del salmo sobre el resto del
pueblo de
Dios}\label{interpretaciuxf3n-de-la-promesa-del-salmo-sobre-el-resto-del-pueblo-de-dios}}

\hypertarget{section-3}{%
\section{4}\label{section-3}}

\bibleverse{1} Temamos, pues, que no parezca que alguno de vosotros se
ha quedado sin la promesa de entrar en su descanso. \bibleverse{2}
Porque ciertamente se nos ha anunciado la buena noticia, como a ellos
también, pero la palabra que oyeron no les aprovechó, porque no se
mezcló con la fe de los que oyeron. \bibleverse{3} Pues nosotros, los
que hemos creído, entramos en ese reposo, como él ha dicho: ``Como juré
en mi ira, no entrarán en mi reposo'', aunque las obras estaban acabadas
desde la fundación del mundo. \bibleverse{4} Porque él ha dicho esto en
algún lugar acerca del séptimo día: ``Dios descansó en el séptimo día de
todas sus obras''; \bibleverse{5} y en este lugar otra vez: ``No
entrarán en mi reposo''.

\bibleverse{6} Viendo, pues, que falta que algunos entren en ella, y que
aquellos a los que antes se les había predicado la buena nueva no
entraron por desobediencia, \bibleverse{7} vuelve a definir un día
determinado, ``hoy'', diciendo por medio de David tanto tiempo después
(tal como se ha dicho), ``Hoy, si escuchas su voz, no endurezcáis
vuestros corazones''. \footnote{\textbf{4:7} Heb 3,7}

\bibleverse{8} Porque si Josué les hubiera dado descanso, no habría
hablado después de otro día. \footnote{\textbf{4:8} Deut 31,7; Jos 22,4}
\bibleverse{9} Queda, pues, un descanso sabático para el pueblo de Dios.
\bibleverse{10} Porque el que ha entrado en su reposo ha descansado
también de sus obras, como Dios lo hizo de las suyas. \footnote{\textbf{4:10}
  Apoc 14,13}

\hypertarget{exhortaciuxf3n-final-en-referencia-a-la-seriedad-y-el-poder-de-la-palabra-de-dios}{%
\subsection{Exhortación final en referencia a la seriedad y el poder de
la palabra de
Dios}\label{exhortaciuxf3n-final-en-referencia-a-la-seriedad-y-el-poder-de-la-palabra-de-dios}}

\bibleverse{11} Procuremos, pues, entrar en ese reposo, para que nadie
caiga en el mismo ejemplo de desobediencia. \footnote{\textbf{4:11} Heb
  3,16-19} \bibleverse{12} Porque la palabra de Dios es viva y eficaz, y
más cortante que toda espada de dos filos, pues penetra hasta la
división del alma y del espíritu, de las articulaciones y de los
tuétanos, y es capaz de discernir los pensamientos y las intenciones del
corazón. \footnote{\textbf{4:12} Jer 23,29; Apoc 2,12} \bibleverse{13}
No hay criatura que se oculte a su vista, sino que todas las cosas están
desnudas y expuestas ante los ojos de aquel a quien debemos rendir
cuentas.

\hypertarget{jesuxfas-conoce-las-debilidades-humanas-por-experiencia-personal}{%
\subsection{Jesús conoce las debilidades humanas por experiencia
personal}\label{jesuxfas-conoce-las-debilidades-humanas-por-experiencia-personal}}

\bibleverse{14} Teniendo, pues, un gran sumo sacerdote que ha atravesado
los cielos, Jesús, el Hijo de Dios, aferrémonos a nuestra confesión.
\footnote{\textbf{4:14} Heb 3,1; Heb 9,11-12; Heb 10,23} \bibleverse{15}
Porque no tenemos un sumo sacerdote que no pueda compadecerse de
nuestras debilidades, sino uno que ha sido tentado en todo como
nosotros, pero sin pecado. \footnote{\textbf{4:15} Heb 2,18; Juan 8,46}
\bibleverse{16} Acerquémonos, pues, con confianza al trono de la gracia,
para recibir misericordia y hallar gracia para el auxilio en el momento
de necesidad. \footnote{\textbf{4:16} Rom 3,25; Rom 5,2}

\hypertarget{con-cristo-se-encuentran-los-requisitos-necesarios-del-sumo-sacerdote-sugeridos-en-melquisedec}{%
\subsection{Con Cristo se encuentran los requisitos necesarios del sumo
sacerdote sugeridos en
Melquisedec}\label{con-cristo-se-encuentran-los-requisitos-necesarios-del-sumo-sacerdote-sugeridos-en-melquisedec}}

\hypertarget{section-4}{%
\section{5}\label{section-4}}

\bibleverse{1} Porque todo sumo sacerdote, tomado de entre los hombres,
es designado para los hombres en lo que respecta a Dios, para que
ofrezca tanto dones como sacrificios por los pecados. \bibleverse{2} El
sumo sacerdote puede tratar con dulzura a los que son ignorantes y se
extravían, porque él mismo está también rodeado de debilidad.
\bibleverse{3} Por eso debe ofrecer sacrificios por los pecados, tanto
por el pueblo como por él mismo. \footnote{\textbf{5:3} Lev 9,7}
\bibleverse{4} Nadie se arroga este honor, sino que es llamado por Dios,
como lo fue Aarón. \footnote{\textbf{5:4} Éxod 28,1} \bibleverse{5} Así
también Cristo no se glorificó para ser hecho sumo sacerdote, sino que
fue él quien le dijo, ``Tú eres mi Hijo. Hoy me he convertido en tu
padre''.

\bibleverse{6} Como dice también en otro lugar, ``Eres un sacerdote para
siempre, según el orden de Melquisedec''. \footnote{\textbf{5:6} Heb
  6,20}

\bibleverse{7} Él, en los días de su carne, habiendo ofrecido oraciones
y peticiones con fuerte clamor y lágrimas al que podía salvarlo de la
muerte, y habiendo sido escuchado por su temor piadoso, \footnote{\textbf{5:7}
  Mat 26,39-46} \bibleverse{8} aunque era un Hijo, aprendió la
obediencia por las cosas que sufrió. \footnote{\textbf{5:8} Fil 2,8}
\bibleverse{9} Habiendo sido perfeccionado, llegó a ser para todos los
que le obedecen el autor de la salvación eterna, \bibleverse{10}
nombrado por Dios sumo sacerdote según el orden de Melquisedec.
\footnote{\textbf{5:10} Heb 7,1}

\hypertarget{quejarse-de-la-inmadurez-la-indolencia-intelectual-y-el-atraso-de-los-lectores}{%
\subsection{Quejarse de la inmadurez, la indolencia intelectual y el
atraso de los
lectores}\label{quejarse-de-la-inmadurez-la-indolencia-intelectual-y-el-atraso-de-los-lectores}}

\bibleverse{11} Acerca de él tenemos muchas palabras que decir, y
difíciles de interpretar, ya que os habéis vuelto torpes de oído.
\bibleverse{12} Pues aunque ya deberíais ser maestros, necesitáis de
nuevo que alguien os enseñe los rudimentos de los primeros principios de
las revelaciones de Dios. Habéis llegado a necesitar leche, y no
alimento sólido. \footnote{\textbf{5:12} 1Cor 3,1-3; 1Pe 2,2}
\bibleverse{13} Porque todo el que vive de leche no tiene experiencia en
la palabra de justicia, pues es un bebé. \footnote{\textbf{5:13} Efes
  4,14} \bibleverse{14} Pero el alimento sólido es para los que ya han
crecido, que por el uso tienen sus sentidos ejercitados para discernir
el bien y el mal.

\hypertarget{es-una-cuestiuxf3n-de-progreso-la-recauxedda-es-peligrosa-y-puede-provocar-dauxf1os-incurables}{%
\subsection{Es una cuestión de progreso; La recaída es peligrosa y puede
provocar daños
incurables}\label{es-una-cuestiuxf3n-de-progreso-la-recauxedda-es-peligrosa-y-puede-provocar-dauxf1os-incurables}}

\hypertarget{section-5}{%
\section{6}\label{section-5}}

\bibleverse{1} Por lo tanto, dejando la enseñanza de los primeros
principios de Cristo, prosigamos hacia la perfección, volviendo a poner
el fundamento del arrepentimiento de las obras muertas, de la fe hacia
Dios, \bibleverse{2} de la enseñanza de los bautismos, de la imposición
de manos, de la resurrección de los muertos y del juicio eterno.
\bibleverse{3} Esto haremos, si Dios lo permite. \bibleverse{4} Porque
en cuanto a los que una vez fueron iluminados y gustaron del don
celestial, y fueron hechos partícipes del Espíritu Santo, \footnote{\textbf{6:4}
  Heb 10,26-29; 2Pe 2,20} \bibleverse{5} y gustaron de la buena palabra
de Dios y de los poderes del siglo venidero, \bibleverse{6} y luego
recayeron, es imposible renovarlos de nuevo al arrepentimiento, ya que
crucifican de nuevo al Hijo de Dios para sí mismos, y lo exponen a la
vergüenza. \bibleverse{7} Porque la tierra que ha bebido la lluvia que
viene a menudo sobre ella y produce una cosecha adecuada para los que la
cultivan, recibe la bendición de Dios; \bibleverse{8} pero si produce
espinas y cardos, es rechazada y está a punto de ser maldecida, cuyo fin
es ser quemada.

\hypertarget{confiada-esperanza-de-superar-este-angustioso-estado-de-los-lectores-y-el-peligro-que-los-amenaza}{%
\subsection{Confiada esperanza de superar este angustioso estado de los
lectores y el peligro que los
amenaza}\label{confiada-esperanza-de-superar-este-angustioso-estado-de-los-lectores-y-el-peligro-que-los-amenaza}}

\bibleverse{9} Pero, amados, estamos persuadidos de cosas mejores para
vosotros, y de cosas que acompañan a la salvación, aunque hablemos así.
\bibleverse{10} Porque Dios no es injusto, como para olvidar vuestra
obra y el trabajo de amor que habéis mostrado hacia su nombre, al servir
a los santos, y al servirlos todavía. \footnote{\textbf{6:10} Heb
  10,32-34} \bibleverse{11} Deseamos que cada uno de vosotros muestre la
misma diligencia en la plenitud de la esperanza hasta el final,
\footnote{\textbf{6:11} Heb 3,14; Fil 1,6} \bibleverse{12} para que no
seáis perezosos, sino imitadores de los que por la fe y la perseverancia
heredaron las promesas.

\hypertarget{el-fundamento-firme-de-la-esperanza-en-la-gloria-que-seguramente-se-espera-radica-en-las-confiables-promesas-de-dios}{%
\subsection{El fundamento firme de la esperanza en la gloria que
seguramente se espera radica en las confiables promesas de
Dios}\label{el-fundamento-firme-de-la-esperanza-en-la-gloria-que-seguramente-se-espera-radica-en-las-confiables-promesas-de-dios}}

\bibleverse{13} Porque cuando Dios hizo una promesa a Abraham, como no
podía jurar por nadie más grande, juró por sí mismo, \bibleverse{14}
diciendo: ``Ciertamente te bendeciré y te multiplicaré''.
\bibleverse{15} Así, habiendo soportado pacientemente, obtuvo la
promesa. \bibleverse{16} Porque los hombres ciertamente juran por uno
mayor, y en toda disputa suya el juramento es definitivo para la
confirmación. \footnote{\textbf{6:16} Éxod 22,11} \bibleverse{17} De
este modo, Dios, decidido a mostrar más abundantemente a los herederos
de la promesa la inmutabilidad de su consejo, se interpuso con un
juramento, \bibleverse{18} para que por dos cosas inmutables, en las que
es imposible que Dios mienta, tengamos un fuerte estímulo, los que hemos
huido para refugiarnos en la esperanza puesta ante nosotros.
\bibleverse{19} Esta esperanza la tenemos como ancla del alma, una
esperanza segura y firme que entra en lo que está dentro del velo,
\footnote{\textbf{6:19} Lev 16,2; Lev 16,12} \bibleverse{20} donde como
precursor entró Jesús por nosotros, convertido en sumo sacerdote para
siempre según el orden de Melquisedec. \footnote{\textbf{6:20} Heb 5,6}

\hypertarget{jesuxfas-el-sumo-sacerdote-perfecto-para-siempre-seguxfan-el-orden-de-melquisedec}{%
\subsection{Jesús, el sumo sacerdote perfecto para siempre según el
orden de
Melquisedec}\label{jesuxfas-el-sumo-sacerdote-perfecto-para-siempre-seguxfan-el-orden-de-melquisedec}}

\hypertarget{section-6}{%
\section{7}\label{section-6}}

\bibleverse{1} Porque este Melquisedec, rey de Salem, sacerdote del Dios
Altísimo, que salió al encuentro de Abraham al volver de la matanza de
los reyes y lo bendijo, \footnote{\textbf{7:1} Gén 14,18-20}
\bibleverse{2} a quien también Abraham repartió la décima parte de todo
(siendo primero, por interpretación, ``rey de la justicia'', y luego
también ``rey de Salem'', que significa ``rey de la paz'',
\bibleverse{3} sin padre, sin madre, sin genealogía, no teniendo
principio de días ni fin de vida, sino hecho como el Hijo de Dios),
permanece sacerdote continuamente. \footnote{\textbf{7:3} Juan 7,27}

\hypertarget{melquisedec-es-muxe1s-digno-que-los-sacerdotes-levitas}{%
\subsection{Melquisedec es más digno que los sacerdotes
levitas}\label{melquisedec-es-muxe1s-digno-que-los-sacerdotes-levitas}}

\bibleverse{4} Ahora bien, considera cuán grande era este hombre, a
quien incluso Abraham el patriarca dio un décimo del mejor botín.
\bibleverse{5} Ciertamente, los hijos de Leví que reciben el oficio de
sacerdote tienen el mandato de tomar los diezmos del pueblo según la
ley, es decir, de sus hermanos, aunque éstos hayan salido del cuerpo de
Abraham, \footnote{\textbf{7:5} Núm 18,21} \bibleverse{6} pero aquel
cuya genealogía no se cuenta a partir de ellos ha aceptado los diezmos
de Abraham, y ha bendecido al que tiene las promesas. \bibleverse{7}
Pero sin ninguna disputa el menor es bendecido por el mayor.
\bibleverse{8} Aquí reciben los diezmos los que mueren, pero allí recibe
los diezmos aquel de quien se da testimonio de que vive. \bibleverse{9}
Podemos decir que, por medio de Abraham, incluso Leví, que recibe los
diezmos, ha pagado los diezmos, \bibleverse{10} pues todavía estaba en
el cuerpo de su padre cuando Melquisedec lo conoció.

\hypertarget{el-cambio-y-aboliciuxf3n-del-sacerdocio-provocado-por-el-sacerdocio-peculiar-de-jesuxfas}{%
\subsection{El cambio y abolición del sacerdocio provocado por el
sacerdocio peculiar de
Jesús}\label{el-cambio-y-aboliciuxf3n-del-sacerdocio-provocado-por-el-sacerdocio-peculiar-de-jesuxfas}}

\bibleverse{11} Ahora bien, si la perfección fue por medio del
sacerdocio levítico (porque bajo él el pueblo ha recibido la ley), ¿qué
necesidad había de que se levantara otro sacerdote según el orden de
Melquisedec, y no fuera llamado según el orden de Aarón? \bibleverse{12}
Porque siendo cambiado el sacerdocio, es necesario que se haga también
un cambio en la ley. \bibleverse{13} Porque aquel de quien se dicen
estas cosas pertenece a otra tribu, de la cual nadie ha oficiado en el
altar. \bibleverse{14} Porque es evidente que nuestro Señor ha salido de
Judá, de cuya tribu Moisés no habló nada respecto al sacerdocio.
\footnote{\textbf{7:14} Gén 49,10; Is 11,1; Mat 1,1-3} \bibleverse{15}
Esto es aún más abundantemente evidente, si a semejanza de Melquisedec
se levanta otro sacerdote, \bibleverse{16} que ha sido hecho, no según
la ley de un mandamiento carnal, sino según el poder de una vida sin
fin; \bibleverse{17} porque está atestiguado, ``Eres un sacerdote para
siempre, según el orden de Melquisedec''. \footnote{\textbf{7:17} Heb
  5,6}

\hypertarget{la-razuxf3n-del-cambio-en-el-orden-de-los-sacerdotes-es-que-jesuxfas-deberuxeda-ser-el-garante-de-un-pacto-superior}{%
\subsection{La razón del cambio en el orden de los sacerdotes es que
Jesús debería ser el garante de un pacto
superior}\label{la-razuxf3n-del-cambio-en-el-orden-de-los-sacerdotes-es-que-jesuxfas-deberuxeda-ser-el-garante-de-un-pacto-superior}}

\bibleverse{18} Porque hay una anulación de un mandamiento anterior a
causa de su debilidad e inutilidad \bibleverse{19} (porque la ley no
hizo nada perfecto), y una introducción de una esperanza mejor, por la
cual nos acercamos a Dios. \bibleverse{20} Ya que no fue hecho sacerdote
sin prestar juramento \bibleverse{21} (pues ciertamente fueron hechos
sacerdotes sin juramento), sino con juramento por el que dice de él,
``El Señor juró y no cambiará de opinión,`Eres un sacerdote para
siempre, según el orden de Melquisedec''.

\bibleverse{22} Por tanto, Jesús se ha convertido en la garantía de un
pacto mejor. \footnote{\textbf{7:22} Heb 8,6; Heb 12,24}

\bibleverse{23} Muchos, en efecto, han sido hechos sacerdotes, porque la
muerte les impide continuar. \bibleverse{24} Pero él, por vivir
eternamente, tiene su sacerdocio inmutable. \bibleverse{25} Por eso
también puede salvar hasta el extremo a los que se acercan a Dios por
medio de él, ya que vive eternamente para interceder por ellos.
\footnote{\textbf{7:25} Rom 8,34; 1Jn 2,1}

\hypertarget{jesuxfas-como-el-sumo-sacerdote-perfecto-y-eterno}{%
\subsection{Jesús como el sumo sacerdote perfecto y
eterno}\label{jesuxfas-como-el-sumo-sacerdote-perfecto-y-eterno}}

\bibleverse{26} Porque tal sumo sacerdote nos convenía: santo, sin
culpa, sin mancha, apartado de los pecadores y hecho más alto que los
cielos; \bibleverse{27} que no tiene necesidad, como aquellos sumos
sacerdotes, de ofrecer sacrificios cada día, primero por sus propios
pecados y luego por los del pueblo. Porque esto lo hizo una vez para
siempre, al ofrecerse a sí mismo. \footnote{\textbf{7:27} Lev 16,6; Lev
  16,15} \bibleverse{28} Porque la ley nombra como sumos sacerdotes a
hombres que tienen debilidad, pero la palabra del juramento, que vino
después de la ley, nombra para siempre a un Hijo que ha sido
perfeccionado.

\hypertarget{la-superioridad-del-ministerio-sumo-sacerdotal-celestial-de-jesuxfas-y-el-nuevo-pacto-del-que-uxe9l-es-mediador}{%
\subsection{La superioridad del ministerio sumo sacerdotal celestial de
Jesús y el nuevo pacto del que él es
mediador}\label{la-superioridad-del-ministerio-sumo-sacerdotal-celestial-de-jesuxfas-y-el-nuevo-pacto-del-que-uxe9l-es-mediador}}

\hypertarget{section-7}{%
\section{8}\label{section-7}}

\bibleverse{1} Ahora bien, en las cosas que estamos diciendo, el punto
principal es éste: tenemos tal sumo sacerdote, que se sentó a la derecha
del trono de la Majestad en los cielos, \footnote{\textbf{8:1} Heb 4,14}
\bibleverse{2} un servidor del santuario y del verdadero tabernáculo que
el Señor levantó, no el hombre. \bibleverse{3} Porque todo sumo
sacerdote está destinado a ofrecer tanto ofrendas como sacrificios. Por
lo tanto, es necesario que este sumo sacerdote también tenga algo que
ofrecer. \bibleverse{4} Porque si estuviera en la tierra, no sería
sacerdote en absoluto, ya que hay sacerdotes que ofrecen las ofrendas
según la ley, \bibleverse{5} que sirven de copia y sombra de las cosas
celestiales, tal como Moisés fue advertido por Dios cuando iba a hacer
el tabernáculo, pues le dijo: ``Mira, todo lo harás según el modelo que
se te mostró en la montaña.'' \footnote{\textbf{8:5} Col 2,17}
\bibleverse{6} Pero ahora ha obtenido un ministerio más excelente, por
cuanto es también el mediador de un pacto mejor, que sobre mejores
promesas ha sido dado como ley. \footnote{\textbf{8:6} Heb 7,22}

\bibleverse{7} Porque si aquel primer pacto hubiera sido impecable, no
se habría buscado lugar para un segundo. \bibleverse{8} Porque
encontrando faltas en ellos, dijo, ``He aquí que vienen los días'', dice
el Señor, ``que haré un nuevo pacto con la casa de Israel y con la casa
de Judá; \footnote{\textbf{8:8} Heb 10,16-17} \bibleverse{9} no según el
pacto que hice con sus padresel día en que los tomé de la mano para
sacarlos de la tierra de Egipto; porque no continuaron en mi pacto, y no
les hice caso'', dice el Señor. \footnote{\textbf{8:9} Éxod 19,5-6}
\bibleverse{10} ``Porque éste es el pacto que haré con la casa de
Israeldespués de esos días'', dice el Señor: ``Pondré mis leyes en su
mente; También los escribiré en su corazón. Yo seré su Dios, y serán mi
pueblo. \bibleverse{11} No enseñarán a cada hombre a su conciudadanoy
cada uno a su hermano, diciendo: ``Conoce al Señor''. porque todos me
conocerán, desde lo más pequeño hasta lo más grande. \bibleverse{12}
Porque seré misericordioso con su injusticia. No me acordaré más de sus
pecados y de sus actos ilícitos''.

\bibleverse{13} Al decir: ``Un nuevo pacto'', ha dejado obsoleto el
primero. Pero lo que se vuelve obsoleto y envejece está a punto de
desaparecer. \footnote{\textbf{8:13} Rom 10,4}

\hypertarget{la-imperfecciuxf3n-del-ministerio-sacerdotal-levuxedtico-y-la-perfecciuxf3n-o-superioridad-del-ministerio-sumo-sacerdotal-de-cristo}{%
\subsection{La imperfección del ministerio sacerdotal levítico y la
perfección (o superioridad) del ministerio sumo sacerdotal de
Cristo}\label{la-imperfecciuxf3n-del-ministerio-sacerdotal-levuxedtico-y-la-perfecciuxf3n-o-superioridad-del-ministerio-sumo-sacerdotal-de-cristo}}

\hypertarget{section-8}{%
\section{9}\label{section-8}}

\bibleverse{1} Ciertamente, incluso el primer pacto tenía ordenanzas de
servicio divino y un santuario terrenal. \bibleverse{2} Se preparó un
tabernáculo. En la primera parte estaban el candelabro, la mesa y el pan
de muestra, que se llama el Lugar Santo. \footnote{\textbf{9:2} Éxod
  25,23; Éxod 25,30-31} \bibleverse{3} Después del segundo velo estaba
el tabernáculo que se llama el Santo de los Santos, \footnote{\textbf{9:3}
  Éxod 26,33} \bibleverse{4} que tenía un altar de oro para el incienso
y el arca de la alianza recubierta de oro por todos lados, en la que
había una vasija de oro que contenía el maná, la vara de Aarón que
brotaba y las tablas de la alianza; \footnote{\textbf{9:4} Éxod 16,33;
  Éxod 25,10-22; Núm 17,8-10} \bibleverse{5} y encima querubines de
gloria que cubrían el propiciatorio, de lo cual no podemos hablar ahora
en detalle.

\bibleverse{6} Así preparadas estas cosas, los sacerdotes entraban
continuamente en el primer tabernáculo, cumpliendo los servicios,
\footnote{\textbf{9:6} Núm 18,3-4; Éxod 30,10; Lev 16,2; Lev 16,14-15}
\bibleverse{7} pero en el segundo sólo entraba el sumo sacerdote, una
vez al año, no sin sangre, que ofrecía por sí mismo y por los errores
del pueblo. \bibleverse{8} El Espíritu Santo está indicando esto, que el
camino hacia el Lugar Santo no fue revelado todavía mientras el primer
tabernáculo estaba en pie. \footnote{\textbf{9:8} Heb 10,20}
\bibleverse{9} Esto es un símbolo de la época actual, en la que se
ofrecen dones y sacrificios que son incapaces, en lo que respecta a la
conciencia, de hacer perfecto al adorador, \footnote{\textbf{9:9} Heb
  7,19; Heb 10,1-2} \bibleverse{10} siendo sólo (con comidas y bebidas y
lavados diversos) ordenanzas carnales, impuestas hasta un tiempo de
reforma. \footnote{\textbf{9:10} Lev 11,1; Núm 19,1}

\bibleverse{11} Pero Cristo, habiendo venido como sumo sacerdote de los
bienes venideros, a través del mayor y más perfecto tabernáculo, no
hecho de manos, es decir, no de esta creación, \bibleverse{12} ni por la
sangre de machos cabríos y terneros, sino por su propia sangre, entró
una vez por todas en el Lugar Santo, habiendo obtenido la redención
eterna. \bibleverse{13} Porque si la sangre de los machos cabríos y de
los toros, y la ceniza de una vaquilla que rocía a los contaminados,
santifican para la limpieza de la carne, \footnote{\textbf{9:13} Núm
  19,2; Núm 19,9; Núm 19,17} \bibleverse{14} ¿cuánto más la sangre de
Cristo, que por el Espíritu eterno se ofreció a sí mismo sin defecto a
Dios, limpiará vuestra conciencia de las obras muertas para servir al
Dios vivo? \footnote{\textbf{9:14} Heb 1,3; 1Pe 1,18-19; 1Jn 1,7; Apoc
  1,5}

\hypertarget{cristo-como-mediador-de-un-nuevo-pacto-y-su-muerte-sacrificial-uxfanica-como-medio-eterno-de-su-servicio-celestial-como-sumo-sacerdote}{%
\subsection{Cristo como mediador de un nuevo pacto y su muerte
sacrificial única como medio eterno de su servicio celestial como sumo
sacerdote}\label{cristo-como-mediador-de-un-nuevo-pacto-y-su-muerte-sacrificial-uxfanica-como-medio-eterno-de-su-servicio-celestial-como-sumo-sacerdote}}

\bibleverse{15} Por eso es mediador de una nueva alianza, ya que se ha
producido una muerte para la redención de las transgresiones que había
bajo la primera alianza, a fin de que los llamados reciban la promesa de
la herencia eterna. \footnote{\textbf{9:15} Heb 12,24; 1Tim 2,5}
\bibleverse{16} Porque donde hay un testamento, necesariamente tiene que
haber la muerte del que lo hizo. \bibleverse{17} Porque el testamento
está en vigor donde ha habido muerte, pues nunca está en vigor mientras
vive el que lo hizo. \bibleverse{18} Por lo tanto, ni siquiera el primer
pacto ha sido dedicado sin sangre. \bibleverse{19} Pues cuando Moisés
pronunció todos los mandamientos para todo el pueblo según la ley, tomó
la sangre de los terneros y de los machos cabríos, con agua, lana
escarlata e hisopo, y roció tanto el libro como a todo el pueblo,
\bibleverse{20} diciendo: ``Esta es la sangre de la alianza que Dios os
ha ordenado.'' \footnote{\textbf{9:20} Núm 19,6}

\bibleverse{21} De la misma manera roció con sangre el tabernáculo y
todos los utensilios del ministerio. \footnote{\textbf{9:21} Lev 8,15;
  Lev 8,19} \bibleverse{22} Según la ley, casi todo se limpia con
sangre, y sin derramamiento de sangre no hay remisión. \footnote{\textbf{9:22}
  Lev 17,11}

\hypertarget{el-autosacrificio-uxfanico-y-sangriento-de-cristo-y-su-tremendo-significado-de-salvaciuxf3n-para-los-creyentes}{%
\subsection{El autosacrificio único y sangriento de Cristo y su tremendo
significado de salvación para los
creyentes}\label{el-autosacrificio-uxfanico-y-sangriento-de-cristo-y-su-tremendo-significado-de-salvaciuxf3n-para-los-creyentes}}

\bibleverse{23} Era, pues, necesario que las copias de las cosas
celestiales fueran purificadas con éstas, pero las cosas celestiales
mismas con mejores sacrificios que éstos. \bibleverse{24} Porque Cristo
no ha entrado en los lugares santos hechos de mano, que son
representaciones de los verdaderos, sino en el cielo mismo, para
presentarse ahora en la presencia de Dios por nosotros; \footnote{\textbf{9:24}
  Heb 7,25; 1Jn 2,1} \bibleverse{25} ni tampoco que se ofrezca a sí
mismo con frecuencia, como el sumo sacerdote entra en el lugar santo año
tras año con sangre que no es suya, \bibleverse{26} pues de lo contrario
tendría que haber sufrido con frecuencia desde la fundación del mundo.
Pero ahora, al final de los tiempos, se ha manifestado para quitar el
pecado con el sacrificio de sí mismo. \footnote{\textbf{9:26} Heb 1,3;
  1Cor 10,11; Gal 4,4} \bibleverse{27} Así como está establecido que los
hombres mueran una vez, y después de esto, el juicio, \footnote{\textbf{9:27}
  Gén 3,19} \bibleverse{28} así también Cristo, habiendo sido ofrecido
una vez para llevar los pecados de muchos, aparecerá por segunda vez, no
para ocuparse del pecado, sino para salvar a los que lo esperan
ansiosamente. \footnote{\textbf{9:28} Heb 10,10; Heb 10,12; Heb 10,14}

\hypertarget{el-ejemplo-sombruxedo-y-la-insuficiencia-del-sacrificio-anual-de-reconciliaciuxf3n-del-sumo-sacerdote-levuxedtico-la-perfecciuxf3n-del-sacrificio-de-jesuxfas}{%
\subsection{El ejemplo sombrío y la insuficiencia del sacrificio anual
de reconciliación del sumo sacerdote levítico; la perfección del
sacrificio de
Jesús}\label{el-ejemplo-sombruxedo-y-la-insuficiencia-del-sacrificio-anual-de-reconciliaciuxf3n-del-sumo-sacerdote-levuxedtico-la-perfecciuxf3n-del-sacrificio-de-jesuxfas}}

\hypertarget{section-9}{%
\section{10}\label{section-9}}

\bibleverse{1} Porque la ley, teniendo una sombra del bien que ha de
venir, y no la imagen misma de las cosas, no puede con los mismos
sacrificios de año en año, que ofrecen continuamente, hacer perfectos a
los que se acercan. \footnote{\textbf{10:1} Heb 8,5} \bibleverse{2} De
lo contrario, ¿no habrían dejado de ofrecerse, porque los adoradores,
una vez purificados, ya no tendrían conciencia de los pecados?
\bibleverse{3} Pero en esos sacrificios hay un recuerdo anual de los
pecados. \footnote{\textbf{10:3} Lev 16,34} \bibleverse{4} Porque es
imposible que la sangre de los toros y de los machos cabríos quite los
pecados. \bibleverse{5} Por eso, cuando viene al mundo, dice, ``No
deseabas sacrificios ni ofrendas, pero preparaste un cuerpo para mí.
\bibleverse{6} No te agradaron los holocaustos completos ni los
sacrificios por el pecado. \bibleverse{7} Entonces dije: ``He aquí que
he venido (en el rollo del libro está escrito de mí)para hacer tu
voluntad, oh Dios''.

\bibleverse{8} Antes de decir: ``Sacrificios y ofrendas y holocaustos
completos y sacrificios por el pecado no quisiste, ni te agradaron''
(los que se ofrecen según la ley), \bibleverse{9} entonces ha dicho:
``He venido a hacer tu voluntad''. Quita lo primero para establecer lo
segundo, \bibleverse{10} por cuya voluntad hemos sido santificados
mediante la ofrenda del cuerpo de Jesucristo hecha una vez para siempre.
\footnote{\textbf{10:10} Juan 17,19}

\hypertarget{el-autosacrificio-uxfanico-y-perfectamente-vuxe1lido-de-jesuxfas-hace-que-todos-los-demuxe1s-sacrificios-por-el-pecado-sean-innecesarios-porque-hizo-que-los-creyentes-fueran-completamente-perfectos-ante-dios}{%
\subsection{El autosacrificio único y perfectamente válido de Jesús hace
que todos los demás sacrificios por el pecado sean innecesarios porque
hizo que los creyentes fueran completamente perfectos ante
Dios}\label{el-autosacrificio-uxfanico-y-perfectamente-vuxe1lido-de-jesuxfas-hace-que-todos-los-demuxe1s-sacrificios-por-el-pecado-sean-innecesarios-porque-hizo-que-los-creyentes-fueran-completamente-perfectos-ante-dios}}

\bibleverse{11} En efecto, todos los sacerdotes están de pie día tras
día, sirviendo y ofreciendo a menudo los mismos sacrificios, que nunca
pueden quitar los pecados, \footnote{\textbf{10:11} Éxod 29,38}
\bibleverse{12} pero él, después de haber ofrecido un solo sacrificio
por los pecados para siempre, se sentó a la derecha de Dios,
\bibleverse{13} esperando desde entonces hasta que sus enemigos sean
puestos como escabel de sus pies. \footnote{\textbf{10:13} Sal 110,1}
\bibleverse{14} Porque con una sola ofrenda ha perfeccionado para
siempre a los santificados. \bibleverse{15} El Espíritu Santo también
nos da testimonio, pues después de decir, \bibleverse{16} ``Este es el
pacto que haré con ellosdespués de esos días'', dice el Señor, ``Pondré
mis leyes en su corazón, También los escribiré en su mente''. entonces
dice, \footnote{\textbf{10:16} Heb 8,10} \bibleverse{17} ``No me
acordaré más de sus pecados e iniquidades''. \footnote{\textbf{10:17}
  Heb 8,12}

\bibleverse{18} Ahora bien, donde está la remisión de éstos, no hay más
ofrenda por el pecado.

\hypertarget{amonestaciuxf3n-general-para-perseverar-en-la-fe-la-esperanza-y-el-amor-en-comunidad-con-toda-la-comunidad}{%
\subsection{Amonestación general para perseverar en la fe, la esperanza
y el amor, en comunidad con toda la
comunidad}\label{amonestaciuxf3n-general-para-perseverar-en-la-fe-la-esperanza-y-el-amor-en-comunidad-con-toda-la-comunidad}}

\bibleverse{19} Teniendo, pues, hermanos, la seguridad de entrar en el
lugar santo por la sangre de Jesús, \footnote{\textbf{10:19} Mat 27,51;
  Rom 5,2} \bibleverse{20} por el camino que él nos dedicó, un camino
nuevo y vivo, a través del velo, es decir, de su carne, \footnote{\textbf{10:20}
  Heb 9,8} \bibleverse{21} y teniendo un gran sacerdote sobre la casa de
Dios, \bibleverse{22} acerquémonos con un corazón verdadero en la
plenitud de la fe, teniendo nuestros corazones rociados de una mala
conciencia y teniendo nuestro cuerpo lavado con agua pura, \footnote{\textbf{10:22}
  Heb 4,16; Efes 5,26; 1Pe 3,21} \bibleverse{23} mantengamos firme la
confesión de nuestra esperanza sin vacilar, porque el que prometió es
fiel. \footnote{\textbf{10:23} Heb 4,14}

\bibleverse{24} Consideremos cómo provocarnos unos a otros al amor y a
las buenas obras, \bibleverse{25} no dejando de reunirnos, como
acostumbran algunos, sino exhortándonos unos a otros, y tanto más cuanto
veis que el Día se acerca. \footnote{\textbf{10:25} Heb 3,13; Rom
  13,11-12}

\hypertarget{advertencia-de-apostasuxeda-y-del-juicio-divino-que-golpearuxe1-a-los-que-se-burlan-de-la-gracia}{%
\subsection{Advertencia de apostasía y del juicio divino que golpeará a
los que se burlan de la
gracia}\label{advertencia-de-apostasuxeda-y-del-juicio-divino-que-golpearuxe1-a-los-que-se-burlan-de-la-gracia}}

\bibleverse{26} Porque si pecamos voluntariamente después de haber
recibido el conocimiento de la verdad, ya no queda un sacrificio por los
pecados, \footnote{\textbf{10:26} Heb 6,4-8} \bibleverse{27} sino una
temible expectativa de juicio y una ferocidad de fuego que devorará a
los adversarios. \bibleverse{28} El hombre que hace caso omiso de la ley
de Moisés muere sin compasión por la palabra de dos o tres testigos.
\footnote{\textbf{10:28} Núm 15,30; Deut 17,6} \bibleverse{29} ¿De qué
peor castigo creéis que será juzgado el que ha pisoteado al Hijo de
Dios, y ha considerado impía la sangre de la alianza con la que fue
santificado, y ha insultado al Espíritu de gracia? \footnote{\textbf{10:29}
  Heb 2,3; Heb 12,25} \bibleverse{30} Porque conocemos al que dijo: ``La
venganza me pertenece. Yo pagaré'', dice el Señor. Otra vez: ``El Señor
juzgará a su pueblo''. \bibleverse{31} Es una cosa temible caer en las
manos del Dios vivo. \footnote{\textbf{10:31} Heb 12,29}

\hypertarget{recordatorio-para-ser-fiel-y-tener-confianza-en-la-esperanza-frente-al-sufrimiento-creciente-en-vista-de-la-recompensa-prometida}{%
\subsection{Recordatorio para ser fiel y tener confianza en la esperanza
frente al sufrimiento creciente en vista de la recompensa
prometida}\label{recordatorio-para-ser-fiel-y-tener-confianza-en-la-esperanza-frente-al-sufrimiento-creciente-en-vista-de-la-recompensa-prometida}}

\bibleverse{32} Pero recordad los días anteriores, en los que, después
de ser iluminados, soportasteis una gran lucha con sufrimientos:
\footnote{\textbf{10:32} Heb 6,4} \bibleverse{33} en parte, estando
expuestos tanto a los reproches como a las opresiones, y en parte,
haciéndoos partícipes de los que eran tratados así. \footnote{\textbf{10:33}
  1Cor 4,9} \bibleverse{34} Pues ambos os compadecisteis de mí en mis
cadenas y aceptasteis con alegría el despojo de vuestros bienes,
sabiendo que tenéis para vosotros una posesión mejor y duradera en los
cielos. \footnote{\textbf{10:34} Mat 6,20; Mat 19,21; Mat 19,29}
\bibleverse{35} Por lo tanto, no desperdiciéis vuestra audacia, que
tiene una gran recompensa. \bibleverse{36} Porque necesitáis la
resistencia para que, habiendo hecho la voluntad de Dios, recibáis la
promesa. \footnote{\textbf{10:36} Luc 21,19; Sant 5,7} \bibleverse{37}
``Dentro de muy poco, el que venga, vendrá y no esperará.
\bibleverse{38} Pero el justo vivirá por la fe. Si se encoge, mi alma no
se complace en él''. \footnote{\textbf{10:38} Rom 1,17}

\bibleverse{39} Pero no somos de los que retroceden a la destrucción,
sino de los que tienen fe para la salvación del alma. \footnote{\textbf{10:39}
  1Tes 3,3}

\hypertarget{section-10}{%
\section{11}\label{section-10}}

\bibleverse{1} Ahora bien, la fe es la certeza de lo que se espera, la
prueba de lo que no se ve. \footnote{\textbf{11:1} 2Cor 5,7}

\hypertarget{modelos-del-antiguo-testamento-de-tal-fe}{%
\subsection{Modelos del Antiguo Testamento de tal
fe}\label{modelos-del-antiguo-testamento-de-tal-fe}}

\bibleverse{2} Pues con esto, los ancianos obtuvieron la aprobación.
\bibleverse{3} Por la fe entendemos que el universo ha sido creado por
la palabra de Dios, de modo que lo que se ve no ha sido hecho de cosas
visibles. \footnote{\textbf{11:3} Gén 1,1}

\hypertarget{tres-ejemplos-de-huxe9roes-de-la-fe-de-la-uxe9poca-de-los-antepasados-de-abel-a-nouxe9}{%
\subsection{Tres ejemplos de héroes de la fe de la época de los
antepasados \hspace{0pt}\hspace{0pt}de Abel a
Noé}\label{tres-ejemplos-de-huxe9roes-de-la-fe-de-la-uxe9poca-de-los-antepasados-de-abel-a-nouxe9}}

\bibleverse{4} Por la fe, Abel ofreció a Dios un sacrificio más
excelente que el de Caín, por el cual se le dio testimonio de que era
justo, dando Dios testimonio con respecto a sus dones; y por él, estando
muerto, todavía habla. \footnote{\textbf{11:4} Gén 4,4}

\bibleverse{5} Por la fe, Enoc fue trasladado para no ver la muerte, y
no fue encontrado, porque Dios lo trasladó. Pues se le ha dado
testimonio de que antes de su traslado había sido agradable a Dios.
\footnote{\textbf{11:5} Gén 5,24} \bibleverse{6} Sin fe es imposible
agradar a Dios, pues el que se acerca a él debe creer que existe y que
es remunerador de los que lo buscan.

\bibleverse{7} Por la fe, Noé, advertido de cosas que aún no se veían,
movido por un temor piadoso, preparó una nave para la salvación de su
casa, mediante la cual condenó al mundo y se hizo heredero de la
justicia que es según la fe. \footnote{\textbf{11:7} Gén 6,8-9; Gén
  6,13-22}

\hypertarget{ejemplos-de-la-uxe9poca-de-abraham-y-su-familia}{%
\subsection{Ejemplos de la época de Abraham y su
familia}\label{ejemplos-de-la-uxe9poca-de-abraham-y-su-familia}}

\bibleverse{8} Por la fe, Abraham, cuando fue llamado, obedeció para
salir al lugar que iba a recibir como herencia. Salió sin saber a dónde
iba. \footnote{\textbf{11:8} Gén 12,1-21} \bibleverse{9} Por la fe vivió
como un extranjero en la tierra prometida, como en una tierra que no era
la suya, habitando en tiendas con Isaac y Jacob, herederos con él de la
misma promesa. \bibleverse{10} Porque buscaba la ciudad que tiene
fundamentos, cuyo constructor y artífice es Dios.

\bibleverse{11} Por la fe, hasta la misma Sara recibió poder para
concebir, y dio a luz a un niño cuando ya había pasado la edad, ya que
consideraba fiel al que había prometido. \bibleverse{12} Por tanto,
tantos como las estrellas del cielo en multitud, y tan innumerables como
la arena que está a la orilla del mar, fueron engendrados por un solo
hombre, y él como muerto.

\bibleverse{13} Todos estos murieron en la fe, sin haber recibido las
promesas, pero habiéndolas visto y abrazado de lejos, y habiendo
confesado que eran extranjeros y peregrinos en la tierra. \footnote{\textbf{11:13}
  Gén 23,4; Gén 47,9} \bibleverse{14} Porque los que dicen tales cosas
dejan claro que buscan un país propio. \bibleverse{15} Si en verdad
hubieran pensado en la patria de la que salieron, habrían tenido tiempo
suficiente para regresar. \bibleverse{16} Pero ahora desean un país
mejor, es decir, uno celestial. Por eso Dios no se avergüenza de ellos,
para ser llamado su Dios, pues les ha preparado una ciudad. \footnote{\textbf{11:16}
  Éxod 3,6}

\bibleverse{17} Por la fe, Abraham, siendo probado, ofreció a Isaac. Sí,
el que había recibido gustosamente las promesas ofrecía a su \footnote{\textbf{11:17}
  TR lee ``Él'' en lugar de ``Ellos''} hijo unigénito, \footnote{\textbf{11:17}
  Gén 22,1; Sant 2,21} \bibleverse{18} al que se le dijo: ``Tu
descendencia será contada como de Isaac'', \bibleverse{19} concluyendo
que Dios es capaz de resucitar incluso de entre los muertos. En sentido
figurado, también lo recibió de entre los muertos.

\bibleverse{20} Por la fe, Isaac bendijo a Jacob y a Esaú, incluso en lo
que respecta a las cosas por venir. \footnote{\textbf{11:20} Gén 27,1;
  Gén 48,1-48; Gén 50,1-50}

\bibleverse{21} Por la fe, Jacob, cuando estaba muriendo, bendijo a cada
uno de los hijos de José, y adoró apoyándose en la punta de su bastón.

\bibleverse{22} Por la fe, José, cuando se acercaba su fin, hizo mención
de la partida de los hijos de Israel y dio instrucciones sobre sus
huesos.

\hypertarget{ejemplos-de-la-uxe9poca-de-moisuxe9s-y-josuuxe9}{%
\subsection{Ejemplos de la época de Moisés y
Josué}\label{ejemplos-de-la-uxe9poca-de-moisuxe9s-y-josuuxe9}}

\bibleverse{23} Por la fe, Moisés, cuando nació, fue escondido durante
tres meses por sus padres, porque vieron que era un niño hermoso; y no
tuvieron miedo del mandato del rey. \footnote{\textbf{11:23} Éxod 2,1;
  Éxod 12,1-12; Éxod 14,1-14}

\bibleverse{24} Por la fe, Moisés, una vez crecido, rehusó ser llamado
hijo de la hija del Faraón, \bibleverse{25} prefiriendo compartir los
malos tratos con el pueblo de Dios que gozar por un tiempo de los
placeres del pecado, \bibleverse{26} considerando que el oprobio de
Cristo era mayor riqueza que los tesoros de Egipto, pues esperaba la
recompensa. \bibleverse{27} Por la fe salió de Egipto, sin temer la ira
del rey; pues aguantó como quien ve al que es invisible. \bibleverse{28}
Por la fe guardó la Pascua y la aspersión de la sangre, para que el
destructor de los primogénitos no los tocara.

\bibleverse{29} Por la fe pasaron el Mar Rojo como por tierra firme.
Cuando los egipcios intentaron hacerlo, fueron tragados.

\bibleverse{30} Por la fe, las murallas de Jericó se derrumbaron después
de haber sido rodeadas durante siete días.

\bibleverse{31} Por la fe, Rahab la prostituta no pereció con los
desobedientes, habiendo recibido a los espías en paz.

\hypertarget{ejemplos-de-huxe9roes-de-la-fe-de-la-historia-posterior-de-israel}{%
\subsection{Ejemplos de héroes de la fe de la historia posterior de
Israel}\label{ejemplos-de-huxe9roes-de-la-fe-de-la-historia-posterior-de-israel}}

\bibleverse{32} ¿Qué más puedo decir? Porque me faltaría tiempo si
contara lo de Gedeón, Barac, Sansón, Jefté, David, Samuel y los
profetas, \bibleverse{33} que por la fe sometieron reinos, obraron la
justicia, obtuvieron promesas, taparon la boca de los leones,
\bibleverse{34} apagaron el poder del fuego, escaparon del filo de la
espada, de la debilidad se hicieron fuertes, se hicieron poderosos en la
guerra e hicieron huir a los ejércitos extranjeros. \bibleverse{35} Las
mujeres recibieron a sus muertos por resurrección. Otros fueron
torturados, no aceptando su liberación, para obtener una mejor
resurrección. \bibleverse{36} Otros fueron juzgados por medio de burlas
y azotes, sí, más aún, por medio de prisiones y encarcelamientos.
\bibleverse{37} Fueron apedreados. Fueron aserrados. Fueron tentados.
Fueron asesinados con la espada. Anduvieron por ahí con pieles de oveja
y de cabra, desamparados, afligidos, maltratados, \bibleverse{38} de los
que el mundo no era digno, vagando por los desiertos, los montes, las
cuevas y los agujeros de la tierra.

\bibleverse{39} Todos estos, habiendo sido alabados por su fe, no
recibieron la promesa, \bibleverse{40} habiendo provisto Dios algo mejor
respecto a nosotros, para que sin nosotros no fueran perfeccionados.

\hypertarget{exhortaciuxf3n-a-mantener-la-fidelidad-especialmente-con-respecto-al-ejemplo-de-jesuxfas}{%
\subsection{Exhortación a mantener la fidelidad, especialmente con
respecto al ejemplo de
Jesús}\label{exhortaciuxf3n-a-mantener-la-fidelidad-especialmente-con-respecto-al-ejemplo-de-jesuxfas}}

\hypertarget{section-11}{%
\section{12}\label{section-11}}

\bibleverse{1} Por tanto, nosotros también, viéndonos rodeados de una
nube tan grande de testigos, despojémonos de todo peso y del pecado que
tan fácilmente nos enreda, y corramos con perseverancia la carrera que
tenemos por delante, \footnote{\textbf{12:1} 1Cor 9,24} \bibleverse{2}
mirando a Jesús, el autor y el perfeccionador de la fe, que por el gozo
que le fue propuesto soportó la cruz, despreciando su vergüenza, y se ha
sentado a la derecha del trono de Dios. \footnote{\textbf{12:2} Heb
  5,8-9; Fil 2,8; Fil 2,10}

\bibleverse{3} Porque considerad al que ha soportado tal contradicción
de los pecadores contra sí mismo, para que no os canséis, desfalleciendo
en vuestras almas. \footnote{\textbf{12:3} Luc 2,34; Mat 26,67}

\hypertarget{recordatorio-para-permitir-que-los-desafuxedos-del-sufrimiento-sirvan-como-medio-para-promover-la-vida-de-fe}{%
\subsection{Recordatorio para permitir que los desafíos del sufrimiento
sirvan como medio para promover la vida de
fe}\label{recordatorio-para-permitir-que-los-desafuxedos-del-sufrimiento-sirvan-como-medio-para-promover-la-vida-de-fe}}

\bibleverse{4} Todavía no habéis resistido hasta la sangre, luchando
contra el pecado. \bibleverse{5} Habéis olvidado la exhortación que
razona con vosotros como con los niños, ``Hijo mío, no tomes a la ligera
el castigo del Señor, ni desmayes cuando seas reprendido por él;
\bibleverse{6} porque al que el Señor ama, lo disciplina, y castiga a
todo hijo que recibe''. \footnote{\textbf{12:6} Apoc 3,19}

\bibleverse{7} Es por la disciplina que ustedes soportan. Dios os trata
como a hijos, pues ¿qué hijo hay al que su padre no disciplina?
\bibleverse{8} Pero si no tenéis disciplina, de la que todos habéis sido
hechos partícipes, entonces sois ilegítimos, y no hijos. \bibleverse{9}
Además, tuvimos a los padres de nuestra carne para que nos castigaran, y
les hicimos caso. ¿No será mejor que nos sometamos al Padre de los
espíritus y vivamos? \bibleverse{10} Porque ciertamente ellos nos
disciplinaron por unos días como les pareció bien, pero él para nuestro
provecho, para que seamos partícipes de su santidad. \bibleverse{11}
Todo castigo parece al presente no ser alegre sino penoso; sin embargo,
después da el fruto apacible de la justicia a los que han sido
entrenados por él. \footnote{\textbf{12:11} 2Cor 4,17-18}

\hypertarget{una-advertencia-a-la-comunidad-para-que-se-levante-y-cuide-a-los-miembros-duxe9biles-y-vulnerables}{%
\subsection{Una advertencia a la comunidad para que se levante y cuide a
los miembros débiles y
vulnerables}\label{una-advertencia-a-la-comunidad-para-que-se-levante-y-cuide-a-los-miembros-duxe9biles-y-vulnerables}}

\bibleverse{12} Por tanto, levantad las manos que cuelgan y las rodillas
débiles, \footnote{\textbf{12:12} Is 35,3} \bibleverse{13} y haced
caminos rectos para vuestros pies, para que lo que está cojo no se
disloque, sino que se cure. \footnote{\textbf{12:13} Prov 4,26-27}

\bibleverse{14} Seguid la paz con todos los hombres, y la santificación
sin la cual nadie verá al Señor, \footnote{\textbf{12:14} Rom 12,18;
  2Tim 2,22} \bibleverse{15} mirando atentamente para que no haya
ninguno que esté desprovisto de la gracia de Dios, para que ninguna raíz
de amargura que brote os moleste y muchos sean contaminados por ella,
\footnote{\textbf{12:15} Deut 29,18} \bibleverse{16} para que no haya
ningún inmoral sexual o profano, como Esaú, que vendió su primogenitura
por una sola comida. \footnote{\textbf{12:16} Gén 25,33-34}
\bibleverse{17} Porque sabéis que aun cuando después deseó heredar la
bendición, fue rechazado, pues no encontró lugar para cambiar de
opinión, aunque lo buscó diligentemente con lágrimas. \footnote{\textbf{12:17}
  Gén 27,30-40}

\hypertarget{otra-referencia-a-la-soberanuxeda-del-nuevo-pacto-y-la-inminente-decisiuxf3n-final}{%
\subsection{Otra referencia a la soberanía del nuevo pacto y la
inminente decisión
final}\label{otra-referencia-a-la-soberanuxeda-del-nuevo-pacto-y-la-inminente-decisiuxf3n-final}}

\bibleverse{18} Porque no has venido a un monte que se puede tocar y que
arde con fuego, y a la negrura, a la oscuridad, a la tormenta,
\footnote{\textbf{12:18} Éxod 19,12; Éxod 19,16; Éxod 19,18; Deut 4,11}
\bibleverse{19} al sonido de una trompeta y a la voz de las palabras,
que los que lo oyeron rogaron que no se les dijera ni una palabra más,
\footnote{\textbf{12:19} Éxod 20,19} \bibleverse{20} porque no podían
soportar lo que se había ordenado: ``Si hasta un animal toca el monte,
será apedreado''. \bibleverse{21} Tan temible fue la aparición que
Moisés dijo: ``Estoy aterrado y temblando''.

\bibleverse{22} Pero tú has venido al monte Sión y a la ciudad del Dios
vivo, la Jerusalén celestial, y a innumerables multitudes de ángeles,
\footnote{\textbf{12:22} Gal 4,26; Efes 2,6; Fil 3,20; Apoc 5,11; Apoc
  21,2} \bibleverse{23} a la reunión festiva y a la asamblea de los
primogénitos que están inscritos en el cielo, a Dios el Juez de todos, a
los espíritus de los justos hechos perfectos, \footnote{\textbf{12:23}
  Luc 10,20} \bibleverse{24} a Jesús, el mediador de un nuevo pacto, y a
la sangre de la aspersión que habla mejor que la de Abel. \footnote{\textbf{12:24}
  Heb 9,15; Gén 4,10}

\hypertarget{la-gloria-del-fin-de-los-tiempos-aterradora-para-los-reacios-y-dichosa-para-los-obedientes}{%
\subsection{La gloria del fin de los tiempos, aterradora para los
reacios y dichosa para los
obedientes}\label{la-gloria-del-fin-de-los-tiempos-aterradora-para-los-reacios-y-dichosa-para-los-obedientes}}

\bibleverse{25} Procurad no rechazar al que habla. Porque si no
escaparon cuando rechazaron al que advertía en la tierra, cuánto más no
escaparemos los que nos apartamos del que advierte desde el cielo,
\footnote{\textbf{12:25} Heb 2,2; Heb 10,28-29} \bibleverse{26} cuya voz
hizo temblar la tierra entonces, pero que ahora ha prometido, diciendo:
``Todavía una vez más haré temblar no sólo la tierra, sino también los
cielos.'' \bibleverse{27} Esta frase, ``Todavía una vez más'', significa
la remoción de las cosas que son sacudidas, como de las cosas que han
sido hechas, para que las cosas que no son sacudidas puedan permanecer.
\bibleverse{28} Por lo tanto, recibiendo un Reino que no puede ser
sacudido, tengamos gracia, a través de la cual servimos a Dios
aceptablemente, con reverencia y temor, \bibleverse{29} porque nuestro
Dios es un fuego consumidor. \footnote{\textbf{12:29} Heb 10,31; Deut
  4,24}

\hypertarget{advertencias-individuales-por-el-amor-fraterno-la-pureza-moral-y-la-promociuxf3n-de-la-vida-comunitaria}{%
\subsection{Advertencias individuales por el amor fraterno, la pureza
moral y la promoción de la vida
comunitaria}\label{advertencias-individuales-por-el-amor-fraterno-la-pureza-moral-y-la-promociuxf3n-de-la-vida-comunitaria}}

\hypertarget{section-12}{%
\section{13}\label{section-12}}

\bibleverse{1} Que continúe el amor fraternal. \footnote{\textbf{13:1}
  Juan 13,34; 2Pe 1,7} \bibleverse{2} No os olvidéis de dar hospitalidad
a los extraños, pues al hacerlo, algunos han hospedado a los ángeles sin
saberlo. \footnote{\textbf{13:2} Gén 18,3; Gén 19,2-3; Rom 12,13; 1Pe
  4,9; 3Jn 1,5-8} \bibleverse{3} Acuérdate de los presos, como si
estuvieras atado a ellos, y de los maltratados, ya que tú también estás
en el cuerpo. \footnote{\textbf{13:3} Mat 25,36} \bibleverse{4} Que el
matrimonio sea honrado entre todos, y que el lecho sea incontaminado;
pero Dios juzgará a los inmorales y a los adúlteros.

\bibleverse{5} Sed libres del amor al dinero, contentos con lo que
tenéis, porque él ha dicho: ``No os dejaré en absoluto, ni os
abandonaré''. \footnote{\textbf{13:5} 1Tim 6,6} \bibleverse{6} Para que
con buen ánimo digamos, ``El Señor es mi ayudante. No temeré. ¿Qué puede
hacerme el hombre?''

\hypertarget{amonestaciuxf3n-principal-de-ser-fieles-a-los-gobernantes-y-a-jesuxfas-el-que-permanece-en-la-eternidad-y-el-fin-del-servicio-del-sacrificio-por-el-pecado-juduxedo}{%
\subsection{Amonestación principal de ser fieles a los gobernantes y a
Jesús, el que permanece en la eternidad y el fin del servicio del
sacrificio por el pecado
judío}\label{amonestaciuxf3n-principal-de-ser-fieles-a-los-gobernantes-y-a-jesuxfas-el-que-permanece-en-la-eternidad-y-el-fin-del-servicio-del-sacrificio-por-el-pecado-juduxedo}}

\bibleverse{7} Recordad a vuestros líderes, hombres que os hablaron de
la palabra de Dios, y considerando los resultados de su conducta, imitad
su fe. \bibleverse{8} Jesucristo es el mismo ayer, hoy y siempre.
\footnote{\textbf{13:8} Is 41,4; Apoc 1,17-18; Apoc 22,13; 1Cor 3,11}
\bibleverse{9} No os dejéis llevar por enseñanzas diversas y extrañas,
pues es bueno que el corazón se establezca por la gracia, no por las
comidas, por las que no se beneficiaron los que se ocuparon de esa
manera. \footnote{\textbf{13:9} 2Cor 1,21; 1Tim 4,8; Rom 14,17; Efes
  4,14}

\bibleverse{10} Tenemos un altar del que no tienen derecho a comer los
que sirven al sagrado tabernáculo. \bibleverse{11} Porque los cuerpos de
esos animales, cuya sangre es introducida en el lugar santo por el sumo
sacerdote como ofrenda por el pecado, son quemados fuera del campamento.
\footnote{\textbf{13:11} Lev 7,6; Lev 16,27} \bibleverse{12} Por eso
también Jesús, para santificar al pueblo con su propia sangre, padeció
fuera de la puerta. \footnote{\textbf{13:12} Juan 19,17; Mat 21,39}
\bibleverse{13} Salgamos, pues, hacia él fuera del campamento, llevando
su vituperio. \footnote{\textbf{13:13} Heb 11,26; Heb 12,2}
\bibleverse{14} Porque no tenemos aquí una ciudad duradera, sino que
buscamos la que ha de venir. \footnote{\textbf{13:14} Heb 11,10; Heb
  12,22} \bibleverse{15} Por lo tanto, ofrezcamos continuamente a Dios
un sacrificio de alabanza, es decir, el fruto de los labios que
proclaman la fidelidad a su nombre. \footnote{\textbf{13:15} Os 14,2;
  Sal 50,14; Sal 50,23}

\hypertarget{advertencias-individuales-repetidas-especialmente-con-respecto-al-comportamiento-contra-los-luxedderes-comunitarios}{%
\subsection{Advertencias individuales repetidas, especialmente con
respecto al comportamiento contra los líderes
comunitarios}\label{advertencias-individuales-repetidas-especialmente-con-respecto-al-comportamiento-contra-los-luxedderes-comunitarios}}

\bibleverse{16} Pero no se olviden de hacer el bien y de compartir,
porque con tales sacrificios Dios se complace.

\bibleverse{17} Obedezcan a sus jefes y sométanse a ellos, pues velan
por sus almas, como quienes han de dar cuenta, para que lo hagan con
alegría y no con gemidos, pues eso sería inútil para ustedes.
\footnote{\textbf{13:17} 1Tes 5,12; Ezeq 3,17-19}

\bibleverse{18} Ruega por nosotros, pues estamos persuadidos de que
tenemos buena conciencia, deseando vivir honradamente en todo.
\footnote{\textbf{13:18} Rom 15,30; 2Cor 1,11-12} \bibleverse{19} Os
ruego encarecidamente que lo hagáis, para que yo me restablezca antes.

\hypertarget{clausura-de-la-carta-bendiciuxf3n-mensajes-personales-saludos}{%
\subsection{Clausura de la carta, bendición, mensajes personales,
saludos}\label{clausura-de-la-carta-bendiciuxf3n-mensajes-personales-saludos}}

\bibleverse{20} Que el Dios de la paz, que resucitó de entre los muertos
al gran pastor de las ovejas con la sangre de un pacto eterno, nuestro
Señor Jesús, \footnote{\textbf{13:20} Juan 10,12; 1Pe 2,25}
\bibleverse{21} os haga completos en toda obra buena para que hagáis su
voluntad, obrando en vosotros lo que es agradable a sus ojos, por
Jesucristo, a quien sea la gloria por los siglos de los siglos. Amén.

\bibleverse{22} Pero os exhorto, hermanos, a que soportéis la palabra de
exhortación, pues os he escrito con pocas palabras. \bibleverse{23}
Sabed que nuestro hermano Timoteo ha sido liberado, con el cual, si
viene pronto, os veré.

\bibleverse{24} Saludad a todos vuestros jefes y a todos los santos. Los
italianos te saludan.

\bibleverse{25} La gracia sea con todos vosotros. Amén.
