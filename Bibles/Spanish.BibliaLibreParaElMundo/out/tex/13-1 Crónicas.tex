\hypertarget{los-antepasados-hasta-el-diluvio}{%
\subsection{Los antepasados \hspace{0pt}\hspace{0pt}hasta el
diluvio}\label{los-antepasados-hasta-el-diluvio}}

\hypertarget{section}{%
\section{1}\label{section}}

\bibleverse{1} Adán, Seth, Enosh, \footnote{\textbf{1:1} Gén 5,1}
\bibleverse{2} Kenan, Mahalalel, Jared, \bibleverse{3} Enoc, Matusalén,
Lamec, \bibleverse{4} Noé, Sem, Cam y Jafet.

\hypertarget{los-descendientes-de-nouxe9-excepto-abraham-los-jafetitas}{%
\subsection{Los descendientes de Noé excepto Abraham; Los
jafetitas}\label{los-descendientes-de-nouxe9-excepto-abraham-los-jafetitas}}

\bibleverse{5} Los hijos de Jafet: Gomer, Magog, Madai, Javan, Tubal,
Meshech y Tiras. \footnote{\textbf{1:5} Gén 10,2-5} \bibleverse{6} Los
hijos de Gomer: Ashkenaz, Diphath y Togarmah. \bibleverse{7} Los hijos
de Javán: Elishah, Tarsis, Kittim y Rodanim.

\hypertarget{los-camitas}{%
\subsection{Los camitas}\label{los-camitas}}

\bibleverse{8} Los hijos de Cam: Cus, Mizraim, Put y Canaán. \footnote{\textbf{1:8}
  Gén 10,6-20} \bibleverse{9} Los hijos de Cus: Seba, Havilah, Sabta,
Raama y Sabteca. Los hijos de Raamah: Sabá y Dedán. \bibleverse{10} Cus
se convirtió en el padre de Nimrod. Él comenzó a ser un poderoso en la
tierra. \bibleverse{11} Mizraim se convirtió en el padre de Ludim,
Anamim, Lehabim, Naphtuhim, \bibleverse{12} Pathrusim, Casluhim (de
donde vinieron los filisteos) y Caphtorim. \bibleverse{13} Canaán se
convirtió en el padre de Sidón, su primogénito, Het, \bibleverse{14} el
jebuseo, el amorreo, el gergeseo, \bibleverse{15} el heveo, el arquita,
el sinita, \bibleverse{16} el arvadita, el zemarita y el hamateo.

\hypertarget{los-semitas}{%
\subsection{Los semitas}\label{los-semitas}}

\bibleverse{17} Los hijos de Sem: Elam, Asur, Arpachshad, Lud, Aram, Uz,
Hul, Gether y Meshech. \footnote{\textbf{1:17} Gén 10,21-31}
\bibleverse{18} Arpachshad fue padre de Shelah, y Shelah fue padre de
Eber. \bibleverse{19} A Eber le nacieron dos hijos: el nombre del uno
fue Peleg, porque en sus días la tierra fue dividida; y el nombre de su
hermano fue Joktán. \bibleverse{20} Joktán fue padre de Almodad, Shelef,
Hazarmaveth, Jerah, \bibleverse{21} Hadoram, Uzal, Diklah,
\bibleverse{22} Ebal, Abimael, Sheba, \bibleverse{23} Ophir, Havilah y
Jobab. Todos estos fueron hijos de Joktán.

\hypertarget{la-luxednea-recta-de-sem-a-abraham}{%
\subsection{La línea recta de Sem a
Abraham}\label{la-luxednea-recta-de-sem-a-abraham}}

\bibleverse{24} Sem, Arpachshad, Shelah, \footnote{\textbf{1:24} 1Cró
  1,17; Gén 11,10-26} \bibleverse{25} Eber, Peleg, Reu, \bibleverse{26}
Serug, Nahor, Terah, \bibleverse{27} Abram (también llamado Abraham).

\hypertarget{los-ismaelitas}{%
\subsection{Los ismaelitas}\label{los-ismaelitas}}

\bibleverse{28} Los hijos de Abraham: Isaac e Ismael. \footnote{\textbf{1:28}
  Gén 21,3; Gén 16,16} \bibleverse{29} Estas son sus generaciones: el
primogénito de Ismael, Nebaiot; luego Cedar, Adbeel, Mibsam, \footnote{\textbf{1:29}
  Gén 25,13-16} \bibleverse{30} Mishma, Dumah, Massa, Hadad, Tema,
\bibleverse{31} Jetur, Nafis y Cedemah. Estos son los hijos de Ismael.

\hypertarget{los-descendientes-de-ketura}{%
\subsection{Los descendientes de
Ketura}\label{los-descendientes-de-ketura}}

\bibleverse{32} Los hijos de Cetura, concubina de Abraham: dio a luz a
Zimran, Jokshan, Medan, Midian, Ishbak y Shuah. Los hijos de Joksán:
Seba y Dedán. \footnote{\textbf{1:32} Gén 25,1-3} \bibleverse{33} Los
hijos de Madián: Efá, Efer, Hanoc, Abida y Eldaá. Todos estos fueron
hijos de Cetura.

\hypertarget{los-descendientes-de-esauxfa}{%
\subsection{Los descendientes de
Esaú}\label{los-descendientes-de-esauxfa}}

\bibleverse{34} Abraham se convirtió en el padre de Isaac. Los hijos de
Isaac: Esaú e Israel. \footnote{\textbf{1:34} Gén 25,19-26}
\bibleverse{35} Los hijos de Esaú: Elifaz, Reuel, Jeús, Jalam y Coré.
\footnote{\textbf{1:35} Gén 36,10-19} \bibleverse{36} Los hijos de
Elifaz: Temán, Omar, Zefi, Gatam, Kenaz, Timna y Amalec. \bibleverse{37}
Los hijos de Reuel: Nahat, Zerah, Shammah y Mizzah.

\bibleverse{38} Los hijos de Seir: Lotán, Sobal, Zibeón, Aná, Disón,
Ezer y Disán. \footnote{\textbf{1:38} Gén 36,20-30} \bibleverse{39} Los
hijos de Lotán: Hori y Homam; y Timna era hermana de Lotán.
\bibleverse{40} Los hijos de Sobal: Alian, Manahath, Ebal, Shephi y
Onam. Los hijos de Zibeón: Aiah y Anah. \bibleverse{41} El hijo de Aná:
Disón. Los hijos de Disón: Hamrán, Eshbán, Itrán y Querán.
\bibleverse{42} Los hijos de Ezer: Bilhan, Zaavan y Jaakan. Los hijos de
Disán: Uz y Arán.

\hypertarget{los-reyes-y-jefes-edomitas}{%
\subsection{Los reyes y jefes
edomitas}\label{los-reyes-y-jefes-edomitas}}

\bibleverse{43} Estos son los reyes que reinaron en la tierra de Edom,
antes de que ningún rey reinara sobre los hijos de Israel Bela hijo de
Beor, y el nombre de su ciudad fue Dinhabah. \footnote{\textbf{1:43} Gén
  36,31-43} \bibleverse{44} Murió Bela, y en su lugar reinó Jobab, hijo
de Zera, de Bosra. \bibleverse{45} Murió Jobab, y reinó en su lugar
Husam, de la tierra de los temanitas. \bibleverse{46} Murió Husam, y
reinó en su lugar Hadad, hijo de Bedad, que hirió a Madián en el campo
de Moab, y el nombre de su ciudad fue Avit. \bibleverse{47} Murió Hadad,
y en su lugar reinó Samá de Masreca. \bibleverse{48} Murió Samá, y reinó
en su lugar Saúl, de Rehobot, junto al río. \bibleverse{49} Murió Saúl,
y en su lugar reinó Baal Hanán, hijo de Acbor. \bibleverse{50} Murió
Baal Hanán, y en su lugar reinó Hadad; el nombre de su ciudad fue Pai.
Su esposa se llamaba Mehetabel, hija de Matred, hija de Mezahab.
\bibleverse{51} Luego murió Hadad. Los jefes de Edom fueron: el jefe
Timna, el jefe Aliah, el jefe Jetheth, \bibleverse{52} el jefe
Oholibamah, el jefe Elah, el jefe Pinon, \bibleverse{53} el jefe Kenaz,
el jefe Teman, el jefe Mibzar, \bibleverse{54} el jefe Magdiel y el jefe
Iram. Estos son los jefes de Edom.

\hypertarget{los-hijos-de-jacob-israel-y-las-familias-de-la-tribu-de-juduxe1}{%
\subsection{Los hijos de Jacob Israel y las familias de la tribu de
Judá}\label{los-hijos-de-jacob-israel-y-las-familias-de-la-tribu-de-juduxe1}}

\hypertarget{section-1}{%
\section{2}\label{section-1}}

\bibleverse{1} Estos son los hijos de Israel: Rubén, Simeón, Leví, Judá,
Isacar, Zabulón, \footnote{\textbf{2:1} Gén 35,22-26} \bibleverse{2}
Dan, José, Benjamín, Neftalí, Gad y Aser.

\hypertarget{de-juduxe1-a-hezruxf3n}{%
\subsection{De Judá a Hezrón}\label{de-juduxe1-a-hezruxf3n}}

\bibleverse{3} Los hijos de Judá: Er, Onán y Sela, los tres que le
nacieron de la hija de Súa, la cananea. Er, el primogénito de Judá, fue
malvado a los ojos de Yahvé\footnote{\textbf{2:3} ``Yahvé'' es el nombre
  propio de Dios, a veces traducido como ``\textsc{Señor}'' (en
  mayúsculas) en otras traducciones.} ; y lo mató. \footnote{\textbf{2:3}
  Gén 38,1-7} \bibleverse{4} Tamar, su nuera, le dio a luz a Pérez y a
Zéraj. Todos los hijos de Judá fueron cinco. \footnote{\textbf{2:4} Gén
  38,29-30}

\bibleverse{5} Los hijos de Pérez: Hezrón y Hamul. \footnote{\textbf{2:5}
  Gén 46,12} \bibleverse{6} Los hijos de Zera: Zimri, Etán, Hemán,
Calcol y Dara: cinco en total. \bibleverse{7} Los hijos de Carmi: Acar,
el perturbador de Israel, que cometió una infracción en la cosa
consagrada. \footnote{\textbf{2:7} Jos 7,1} \bibleverse{8} El hijo de
Etán: Azarías.

\hypertarget{de-hezron-a-david-la-luxednea-ram}{%
\subsection{De Hezron a David (la línea
Ram)}\label{de-hezron-a-david-la-luxednea-ram}}

\bibleverse{9} También los hijos de Hezrón, que le nacieron: Jerajmeel,
Ram y Quelubai. \footnote{\textbf{2:9} Rut 4,19-22; Mat 1,3; 1Cró 2,18;
  1Cró 2,42} \bibleverse{10} Ram fue padre de Aminadab, y Aminadab fue
padre de Nahsón, príncipe de los hijos de Judá; \bibleverse{11} y Nahsón
fue padre de Salma, y Salma fue padre de Booz, \bibleverse{12} y Booz
fue padre de Obed, y Obed fue padre de Isaí \bibleverse{13} e Isaí fue
padre de su primogénito Eliab, Abinadab el segundo, Simea el tercero,
\footnote{\textbf{2:13} 1Sam 16,6-10} \bibleverse{14} Netanel el cuarto,
Raddai el quinto, \bibleverse{15} Ozem el sexto y David el séptimo;
\footnote{\textbf{2:15} 1Sam 17,12} \bibleverse{16} y sus hermanas
fueron Zeruiah y Abigail. Los hijos de Sarvia: Abisai, Joab y Asael,
tres. \footnote{\textbf{2:16} 2Sam 2,18} \bibleverse{17} Abigail dio a
luz a Amasa, y el padre de Amasa fue Jeter, el ismaelita. \footnote{\textbf{2:17}
  2Sam 17,25}

\hypertarget{la-luxednea-caleb}{%
\subsection{La línea Caleb}\label{la-luxednea-caleb}}

\bibleverse{18} Caleb, hijo de Hezrón, fue padre de hijos de Azubá, su
mujer, y de Jeriot; y estos fueron sus hijos Jesher, Shobab y Ardón.
\footnote{\textbf{2:18} 1Cró 2,9; 1Cró 2,42} \bibleverse{19} Murió
Azubá, y Caleb se casó con Efrat, que le dio a luz a Hur. \footnote{\textbf{2:19}
  1Cró 2,50} \bibleverse{20} Hur fue el padre de Uri, y Uri fue el padre
de Bezalel. \footnote{\textbf{2:20} Éxod 31,2}

\bibleverse{21} Después, Hezrón se acercó a la hija de Maquir, padre de
Galaad, a la que tomó como esposa cuando tenía sesenta años, y ella le
dio a luz a Segub. \bibleverse{22} Segub fue el padre de Jair, quien
tuvo veintitrés ciudades en la tierra de Galaad. \footnote{\textbf{2:22}
  Jue 10,3} \bibleverse{23} Gesur y Aram les arrebataron las ciudades de
Jair, con Kenat y sus aldeas, hasta sesenta ciudades. Todos estos fueron
los hijos de Maquir, padre de Galaad. \footnote{\textbf{2:23} 1Re 4,13}
\bibleverse{24} Después de la muerte de Hezrón en Caleb Efrata, Abías,
mujer de Hezrón, le dio a luz a Ashur, padre de Tecoa. \footnote{\textbf{2:24}
  1Cró 4,5}

\hypertarget{la-luxednea-jerameel}{%
\subsection{La línea Jerameel}\label{la-luxednea-jerameel}}

\bibleverse{25} Los hijos de Jerajmeel, primogénito de Hezrón, fueron
Rama el primogénito, Buna, Orén, Ozem y Ahías. \footnote{\textbf{2:25}
  1Cró 2,9} \bibleverse{26} Jerajmeel tuvo otra esposa que se llamaba
Atará. Ella fue la madre de Onam. \bibleverse{27} Los hijos de Ram,
primogénito de Jerajmeel, fueron Maaz, Jamín y Eker. \bibleverse{28} Los
hijos de Onam fueron Shammai y Jada. Los hijos de Shammai: Nadab y
Abisur. \bibleverse{29} La mujer de Abisur se llamaba Abihail, y dio a
luz a Ahban y a Molid. \bibleverse{30} Los hijos de Nadab: Seled y
Appaim; pero Seled murió sin hijos. \bibleverse{31} El hijo de Appaim:
Ishi. El hijo de Ishi: Sesán. El hijo de Sesán: Ahlai. \bibleverse{32}
Los hijos de Jada, hermano de Shammai: Jeter y Jonatán; pero Jeter murió
sin hijos. \bibleverse{33} Los hijos de Jonatán: Pelet y Zaza. Estos
fueron los hijos de Jerajmeel. \bibleverse{34} Sesán no tuvo hijos, sino
sólo hijas. Sesán tenía un sirviente, un egipcio, que se llamaba Jarha.
\bibleverse{35} Sesán dio su hija a Jarha, su siervo, como esposa, y
ella le dio a luz a Atai. \bibleverse{36} Atai fue padre de Natán, y
Natán fue padre de Zabad, \bibleverse{37} y Zabad fue padre de Eflal, y
Eflal fue padre de Obed, \bibleverse{38} y Obed fue padre de Jehú, y
Jehú fue padre de Azarías \bibleverse{39} y Azarías fue padre de Helez,
y Helez fue padre de Eleasah, \bibleverse{40} y Eleasah fue padre de
Sismai, y Sismai fue padre de Sallum, \bibleverse{41} y Sallum fue padre
de Jekamiah, y Jekamiah fue padre de Elishama.

\hypertarget{la-luxednea-caleb-1}{%
\subsection{La línea Caleb}\label{la-luxednea-caleb-1}}

\bibleverse{42} Los hijos de Caleb, hermano de Jerajmeel, fueron Mesá,
su primogénito, que fue padre de Zif, y los hijos de Maresá, padre de
Hebrón. \footnote{\textbf{2:42} 1Cró 2,18} \bibleverse{43} Los hijos de
Hebrón: Coré, Tapú, Recem y Sema. \bibleverse{44} Sema fue el padre de
Raham, el padre de Jorkeam; y Rekem fue el padre de Shammai.
\bibleverse{45} El hijo de Samai fue Maón; y Maón fue el padre de Bet
Zur. \bibleverse{46} Efá, concubina de Caleb, dio a luz a Harán, Moza y
Gazez; y Harán fue padre de Gazez. \bibleverse{47} Los hijos de Jahdai:
Regem, Jotán, Gesán, Pelet, Efá y Shaaf. \bibleverse{48} Maaca,
concubina de Caleb, dio a luz a Seber y a Tirana. \bibleverse{49}
También dio a luz a Shaaf, padre de Madmaná, a Sheva, padre de Macbena,
y al padre de Gbea; y la hija de Caleb fue Acsa. \footnote{\textbf{2:49}
  Jos 15,16; Jue 1,12}

\bibleverse{50} Estos fueron los hijos de Caleb, hijo de Hur,
primogénito de Efrata: Sobal, padre de Quiriat Jearim, \footnote{\textbf{2:50}
  1Cró 2,19} \bibleverse{51} Salma, padre de Belén, y Haref, padre de
Bet Gader. \bibleverse{52} Sobal, padre de Quiriat Jearim, tuvo hijos:
Haroeh, la mitad de los Menuhoth. \bibleverse{53} Las familias de
Quiriat Jearim: los itritas, los putitas, los shumatitas y los
misraítas; de ellos salieron los zoratitas y los eshtaolitas.
\footnote{\textbf{2:53} 1Cró 4,2} \bibleverse{54} Los hijos de Salma:
Belén, los netofatitas, Atrot Bet Joab, y la mitad de los manaítas, los
zoritas. \footnote{\textbf{2:54} 1Cró 9,16} \bibleverse{55} Las familias
de escribas que vivían en Jabes: los tiratitas, los simeatitas y los
sucatitas. Estos son los ceneos que vinieron de Hamat, el padre de la
casa de Recab. \footnote{\textbf{2:55} Jue 1,16; Jer 35,1}

\hypertarget{los-hijos-de-david}{%
\subsection{Los hijos de David}\label{los-hijos-de-david}}

\hypertarget{section-2}{%
\section{3}\label{section-2}}

\bibleverse{1} Estos fueron los hijos de David que le nacieron en
Hebrón: el primogénito, Amnón, de Ahinoam jezreelita; el segundo,
Daniel, de Abigail carmelita; \footnote{\textbf{3:1} 2Sam 3,2-5}
\bibleverse{2} el tercero, Absalón, hijo de Maaca, hija de Talmai, rey
de Gesur; el cuarto, Adonías, hijo de Haggit; \bibleverse{3} el quinto,
Sefatías, de Abital; el sexto, Itream, de Egla, su mujer: \bibleverse{4}
seis le nacieron en Hebrón; y reinó allí siete años y seis meses. Reinó
treinta y tres años en Jerusalén; \bibleverse{5} y estos le nacieron en
Jerusalén: Simea, Sobab, Natán y Salomón, cuatro, de Betsúa, hija de
Amiel; \footnote{\textbf{3:5} 1Cró 14,4-7; 2Sam 5,14-16} \bibleverse{6}
e Ibhar, Elisama, Elifelet, \bibleverse{7} Nogah, Nefeg, Jafía,
\bibleverse{8} Elisama, Eliada y Elifelet, nueve. \bibleverse{9} Todos
estos eran hijos de David, además de los hijos de las concubinas; y
Tamar era su hermana. \footnote{\textbf{3:9} 2Sam 13,1}

\hypertarget{los-reyes-davuxeddicos-desde-salomuxf3n-hasta-la-destrucciuxf3n-de-jerusaluxe9n}{%
\subsection{Los reyes davídicos desde Salomón hasta la destrucción de
Jerusalén}\label{los-reyes-davuxeddicos-desde-salomuxf3n-hasta-la-destrucciuxf3n-de-jerusaluxe9n}}

\bibleverse{10} El hijo de Salomón fue Roboam, su hijo Abías, su hijo
Asa, su hijo Josafat, \footnote{\textbf{3:10} Mat 1,7-12}
\bibleverse{11} Su hijo Joram, su hijo Ocozías, su hijo Joás,
\bibleverse{12} Su hijo Amasías, su hijo Azarías, su hijo Jotam,
\bibleverse{13} Su hijo Acaz, su hijo Ezequías, su hijo Manasés,
\bibleverse{14} Su hijo Amón, su hijo Josías. \bibleverse{15} Los hijos
de Josías: el primogénito Johanán, el segundo Joaquím, el tercero
Sedequías y el cuarto Salum. \bibleverse{16} Los hijos de Joacim
Jeconías, su hijo, y Sedequías, su hijo.

\hypertarget{los-otros-descendientes-de-david-desde-jechonja-en-adelante}{%
\subsection{Los otros descendientes de David (desde Jechonja en
adelante)}\label{los-otros-descendientes-de-david-desde-jechonja-en-adelante}}

\bibleverse{17} Los hijos de Jeconías, el cautivo: Saltiel su hijo,
\footnote{\textbf{3:17} 2Cró 36,10} \bibleverse{18} Malquiram, Pedaías,
Senazar, Jecamías, Hosama y Nedabías. \bibleverse{19} Los hijos de
Pedaías: Zorobabel y Simei. Los hijos de Zorobabel: Mesulam y Hananías;
y Selomit fue su hermana; \footnote{\textbf{3:19} Esd 3,2; Esd 3,8}
\bibleverse{20} y Hasubá, Ohel, Berequías, Hasadías y Jushab Hesed,
cinco. \bibleverse{21} Los hijos de Ananías: Pelatías y Jesaías; los
hijos de Refaías, los hijos de Arnán, los hijos de Abdías, los hijos de
Secanías. \bibleverse{22} El hijo de Secanías: Semaías. Los hijos de
Semaías: Hatús, Igal, Barías, Nearías y Safat, seis. \bibleverse{23} Los
hijos de Nearías: Elioenai, Hizkiah y Azrikam, tres. \bibleverse{24} Los
hijos de Elioenai: Hodaviah, Eliashib, Pelaiah, Akkub, Johanan, Delaiah,
y Anani, siete.

\hypertarget{muxe1s-informaciuxf3n-sobre-las-familias-de-la-tribu-de-juduxe1}{%
\subsection{Más información sobre las familias de la tribu de
Judá}\label{muxe1s-informaciuxf3n-sobre-las-familias-de-la-tribu-de-juduxe1}}

\hypertarget{section-3}{%
\section{4}\label{section-3}}

\bibleverse{1} Los hijos de Judá: Pérez, Hezrón, Carmi, Hur y Sobal.
\footnote{\textbf{4:1} 1Cró 2,4-5; 1Cró 2,7; 1Cró 2,19; 1Cró 2,50}
\bibleverse{2} Reaías, hijo de Sobal, fue padre de Jahat, y Jahat fue
padre de Ahumai y Lahad. Estas son las familias de los zoratíes.
\footnote{\textbf{4:2} 1Cró 2,53} \bibleverse{3} Estos fueron los hijos
del padre de Etam Jezreel, Isma e Idbash. El nombre de su hermana era
Hazzelelponi. \bibleverse{4} Penuel fue el padre de Gedor y Ezer el
padre de Hushah. Estos son los hijos de Hur, primogénito de Efrata,
padre de Belén. \footnote{\textbf{4:4} 1Cró 2,19; 1Cró 2,50}
\bibleverse{5} Ashur, padre de Tecoa, tuvo dos esposas, Helá y Naara.
\bibleverse{6} Naara le dio a luz a Ahuzzam, Hefer, Temeni y
Haahashtari. Estos fueron los hijos de Naara. \bibleverse{7} Los hijos
de Hela fueron Zeret, Izhar y Etnán. \bibleverse{8} Hakkoz fue el padre
de Anub, Zobebah y las familias de Aharhel, hijo de Harum.

\bibleverse{9} Jabes era más honorable que sus hermanos. Su madre le
puso el nombre de Jabes,\footnote{\textbf{4:9} ``Jabes'' suena similar a
  la palabra hebrea para ``dolor''.} diciendo: ``Porque lo parí con
dolor''.

\bibleverse{10} Jabes invocó al Dios\footnote{\textbf{4:10} La palabra
  hebrea traducida como ``Dios'' es ``\hebrew{אֱלֹהִ֑ים}'' (Elohim).} de
Israel, diciendo: ``¡Oh, que me bendigas de verdad y amplíes mi
frontera! Que tu mano esté conmigo, y que me guardes del mal, para que
no cause dolor''. Dios le concedió lo que pidió. \footnote{\textbf{4:10}
  Gén 28,20}

\bibleverse{11} Quelub, hermano de Shuhah, se convirtió en el padre de
Mehir, quien fue el padre de Eshton. \bibleverse{12} Eshton llegó a ser
el padre de Beth Rapha, Paseah, y Tehinnah el padre de Ir Nahash. Estos
son los hombres de Recah. \bibleverse{13} Los hijos de Kenaz: Othniel y
Seraiah. Los hijos de Othniel: Hathath. \footnote{\textbf{4:13} El
  griego y la Vulgata añaden ``y Meonothai''} \footnote{\textbf{4:13}
  Jos 15,17; Jue 1,13} \bibleverse{14} Meonothai fue el padre de Ofra; y
Seraiah fue el padre de Joab, el padre de Ge Harashim, porque eran
artesanos. \bibleverse{15} Los hijos de Caleb, hijo de Jefone: Iru, Ela
y Naam. El hijo de Ela: Kenaz. \footnote{\textbf{4:15} Núm 13,6; Núm
  14,6} \bibleverse{16} Los hijos de Jehallelel: Zif, Zifa, Tiria y
Asarel. \bibleverse{17} Los hijos de Esdras: Jeter, Mered, Efer y Jalón;
y la mujer de Mered dio a luz a Miriam, a Shammai y a Ishbah, padre de
Eshtemoa. \bibleverse{18} Su mujer, la judía, dio a luz a Jered, padre
de Gedor, a Heber, padre de Soco, y a Jekutiel, padre de Zanoa. Estos
son los hijos de Bitías, hija del faraón, que tomó Mered.
\bibleverse{19} Los hijos de la mujer de Hodías, hermana de Naham,
fueron los padres de Keilá el garmita y de Estemoa el maacateo.
\bibleverse{20} Los hijos de Simón: Amnón, Rinna, Ben Hanán y Tilón. Los
hijos de Ishi: Zohet y Ben Zohet. \bibleverse{21} Los hijos de Sela,
hijo de Judá: Er padre de Leca, Laada padre de Maresa, y las familias de
la casa de los que trabajaban el lino fino, de la casa de Asbea;
\footnote{\textbf{4:21} 1Cró 2,3} \bibleverse{22} y Jokim, y los hombres
de Cozeba, y Joás, y Saraf, que tenían dominio en Moab, y Jasubilehem.
Estos registros son antiguos. \bibleverse{23} Estos eran los alfareros y
los habitantes de Netaim y Gedera; vivían allí con el rey para su
trabajo.

\hypertarget{informaciuxf3n-sobre-los-descendientes-de-simeuxf3n}{%
\subsection{Información sobre los descendientes de
Simeón}\label{informaciuxf3n-sobre-los-descendientes-de-simeuxf3n}}

\bibleverse{24} Los hijos de Simeón: Nemuel, Jamín, Jarib, Zera, Shaúl;
\footnote{\textbf{4:24} Gén 46,10} \bibleverse{25} Su hijo Salum, su
hijo Mibsam y su hijo Misma. \bibleverse{26} Los hijos de Misma: Hamuel
su hijo, Zaccur su hijo, Simei su hijo. \bibleverse{27} Simei tuvo
dieciséis hijos y seis hijas; pero sus hermanos no tuvieron muchos
hijos, y toda su familia no se multiplicó como los hijos de Judá.

\hypertarget{las-residencias-muxe1s-antiguas-de-la-tribu}{%
\subsection{Las residencias más antiguas de la
tribu}\label{las-residencias-muxe1s-antiguas-de-la-tribu}}

\bibleverse{28} Vivían en Beerseba, Molada, Hazarshual, \footnote{\textbf{4:28}
  Jos 19,2-8} \bibleverse{29} en Bilhá, en Ezem, en Tolad,
\bibleverse{30} en Betuel, en Horma, en Siclag, \bibleverse{31} en Bet
Marcabot, Hazar Susim, en Bet Biri y en Shaaraim. Estas fueron sus
ciudades hasta el reinado de David. \bibleverse{32} Sus aldeas eran
Etam, Ain, Rimmon, Tochen y Ashan, cinco ciudades; \bibleverse{33} y
todas sus aldeas que estaban alrededor de las mismas ciudades, hasta
Baal. Estos fueron sus asentamientos, y conservaron su genealogía.

\hypertarget{indicaciuxf3n-de-otros-jefes-de-familia-simeonitas-las-dos-conquistas-de-los-simeonitas}{%
\subsection{Indicación de otros jefes de familia simeonitas; las dos
conquistas de los
simeonitas}\label{indicaciuxf3n-de-otros-jefes-de-familia-simeonitas-las-dos-conquistas-de-los-simeonitas}}

\bibleverse{34} Meshobab, Jamlec, Josá hijo de Amasías, \bibleverse{35}
Joel, Jehú hijo de Joshibías, hijo de Seraías, hijo de Asiel,
\bibleverse{36} Elioenai, Jaakobah, Jeshohaiah, Asaiah, Adiel, Jesimiel,
Benaía, \bibleverse{37} y Ziza hijo de Sifí, hijo de Allón, hijo de
Jedaías, hijo de Simri, hijo de Semaías --- \bibleverse{38} estos
mencionados por su nombre eran príncipes en sus familias. Las casas de
sus padres aumentaron mucho.

\bibleverse{39} Fueron a la entrada de Gedor, al lado oriental del
valle, para buscar pastos para sus rebaños. \bibleverse{40} Encontraron
ricos y buenos pastos, y la tierra era amplia, tranquila y apacible,
porque los que vivían allí antes eran descendientes de Cam. \footnote{\textbf{4:40}
  Jue 18,7} \bibleverse{41} Estos escritos por su nombre vinieron en los
días de Ezequías, rey de Judá, y atacaron sus tiendas y a los meuníes
que allí se encontraban, y los destruyeron por completo hasta el día de
hoy, y vivieron en su lugar, porque allí había pastos para sus rebaños.
\footnote{\textbf{4:41} 2Re 18,1} \bibleverse{42} Algunos de ellos, de
los hijos de Simeón, quinientos hombres, fueron al monte de Seir,
teniendo por capitanes a Pelatías, Nearías, Refaías y Uziel, hijos de
Ishi. \bibleverse{43} Hirieron al resto de los amalecitas que escaparon,
y han vivido allí hasta el día de hoy. \footnote{\textbf{4:43} 1Sam
  15,3; 1Sam 15,8}

\hypertarget{informaciuxf3n-sobre-rubuxe9n-y-sus-descendientes}{%
\subsection{Información sobre Rubén y sus
descendientes}\label{informaciuxf3n-sobre-rubuxe9n-y-sus-descendientes}}

\hypertarget{section-4}{%
\section{5}\label{section-4}}

\bibleverse{1} Los hijos de Rubén, primogénito de Israel (pues él era el
primogénito, pero por haber profanado el lecho de su padre, su
primogenitura fue entregada a los hijos de José, hijo de Israel; y la
genealogía no debe enumerarse según la primogenitura. \footnote{\textbf{5:1}
  Gén 35,22; Gén 49,4} \bibleverse{2} Porque Judá prevaleció sobre sus
hermanos, y de él salió el príncipe; pero la primogenitura fue de José)
--- \footnote{\textbf{5:2} Gén 49,8; Gén 49,10; Gén 49,22; Deut 33,7;
  Deut 33,13-17} \bibleverse{3} los hijos de Rubén, primogénito de
Israel: Hanoch, Pallu, Hezron y Carmi. \footnote{\textbf{5:3} Éxod 6,14}
\bibleverse{4} Los hijos de Joel: Semaías su hijo, Gog su hijo, Simei su
hijo, \bibleverse{5} Miqueas su hijo, Reaías su hijo, Baal su hijo,
\bibleverse{6} y Beera su hijo, a quien Tilgath Pilneser, rey de Asiria,
llevó cautivo. Era príncipe de los rubenitas. \footnote{\textbf{5:6}
  1Cró 5,26} \bibleverse{7} Sus hermanos por sus familias, cuando se
enumeró la genealogía de sus generaciones: el jefe, Jeiel, y Zacarías,
\bibleverse{8} y Bela hijo de Azaz, hijo de Sema, hijo de Joel, que
vivía en Aroer, hasta Nebo y Baal Meón;

\hypertarget{informaciuxf3n-histuxf3rica-sobre-bela}{%
\subsection{Información histórica sobre
Bela}\label{informaciuxf3n-histuxf3rica-sobre-bela}}

\bibleverse{9} y vivía hacia el este hasta la entrada del desierto desde
el río Éufrates, porque sus ganados se multiplicaban en la tierra de
Galaad.

\bibleverse{10} En los días de Saúl, hicieron la guerra a los hagritas,
que cayeron por su mano, y vivieron en sus tiendas por toda la tierra al
este de Galaad.

\hypertarget{informaciuxf3n-sobre-la-estirpe-y-lugares-de-residencia-asuxed-como-sobre-la-valoraciuxf3n-de-los-gaditas}{%
\subsection{Información sobre la estirpe y lugares de residencia, así
como sobre la valoración de los
gaditas}\label{informaciuxf3n-sobre-la-estirpe-y-lugares-de-residencia-asuxed-como-sobre-la-valoraciuxf3n-de-los-gaditas}}

\bibleverse{11} Los hijos de Gad vivieron junto a ellos en la tierra de
Basán hasta Salecá: \bibleverse{12} Joel el principal, Safam el segundo,
Janai y Safat en Basán. \bibleverse{13} Sus hermanos de las casas
paternas: Miguel, Mesulam, Seba, Jorai, Jacan, Zia y Eber, siete.
\bibleverse{14} Estos fueron los hijos de Abihail, hijo de Huri, hijo de
Jaroa, hijo de Galaad, hijo de Micael, hijo de Jeshishai, hijo de Jahdo,
hijo de Buz; \bibleverse{15} Ahi hijo de Abdiel, hijo de Guni, jefe de
las casas de sus padres. \bibleverse{16} Vivían en Galaad, en Basán, y
en sus ciudades, y en todas las tierras de pastoreo de Sarón hasta sus
fronteras. \bibleverse{17} Todos estos fueron enumerados por genealogías
en los días de Jotam, rey de Judá, y en los días de Jeroboam, rey de
Israel. \footnote{\textbf{5:17} 2Re 15,32; 2Re 14,23}

\hypertarget{la-lucha-de-las-tres-tribus-de-transjordania-con-los-agaritascon-los-agaritas}{%
\subsection{La lucha de las tres tribus de Transjordania con los
agaritascon los
agaritas}\label{la-lucha-de-las-tres-tribus-de-transjordania-con-los-agaritascon-los-agaritas}}

\bibleverse{18} Los hijos de Rubén, los gaditas y la media tribu de
Manasés, de hombres valientes, capaces de llevar broquel y espada, de
disparar con arco y hábiles en la guerra, eran cuarenta y cuatro mil
setecientos sesenta que podían salir a la guerra. \bibleverse{19}
Hicieron la guerra a los hagritas, a Jetur, a Nafis y a Nodab.
\bibleverse{20} Fueron ayudados contra ellos, y los hagritas fueron
entregados en su mano, y todos los que estaban con ellos; porque
clamaron a Dios en la batalla, y él les respondió porque pusieron su
confianza en él. \bibleverse{21} Les quitaron el ganado: de sus camellos
cincuenta mil, de las ovejas doscientas cincuenta mil, de los asnos dos
mil y de los hombres cien mil. \bibleverse{22} Porque muchos cayeron
muertos, porque la guerra era de Dios. Vivieron en su lugar hasta el
cautiverio.

\hypertarget{las-residencias-y-la-divisiuxf3n-de-guxe9nero-de-los-manasitas}{%
\subsection{Las residencias y la división de género de los
manasitas}\label{las-residencias-y-la-divisiuxf3n-de-guxe9nero-de-los-manasitas}}

\bibleverse{23} Los hijos de la media tribu de Manasés vivían en la
tierra. Se multiplicaron desde Basán hasta Baal Hermón, Senir y el monte
Hermón. \footnote{\textbf{5:23} Deut 3,9} \bibleverse{24} Estos eran los
jefes de las casas de sus padres: Efer, Ishi, Eliel, Azriel, Jeremías,
Hodavías y Jahdiel: hombres valientes y famosos, jefes de sus casas
paternas.

\hypertarget{castigo-por-la-apostasuxeda-de-las-tres-tribus-de-jordania-oriental-por-los-reyes-asirios}{%
\subsection{Castigo por la apostasía de las tres tribus de Jordania
Oriental por los reyes
asirios}\label{castigo-por-la-apostasuxeda-de-las-tres-tribus-de-jordania-oriental-por-los-reyes-asirios}}

\bibleverse{25} Se rebelaron contra el Dios de sus padres y se
prostituyeron en pos de los dioses de los pueblos de la tierra que Dios
había destruido antes que ellos. \bibleverse{26} Entonces el Dios de
Israel despertó el espíritu de Pul, rey de Asiria, y el espíritu de
Tilgat Pilneser, rey de Asiria, y se llevó a los rubenitas, a los
gaditas y a la media tribu de Manasés, y los llevó a Halah, Habor, Hara
y al río de Gozán, hasta el día de hoy. \footnote{\textbf{5:26} 2Re
  15,19; 2Re 15,29}

\hypertarget{de-levi-a-los-hijos-de-aaruxf3n}{%
\subsection{De Levi a los hijos de
Aarón}\label{de-levi-a-los-hijos-de-aaruxf3n}}

\hypertarget{section-5}{%
\section{6}\label{section-5}}

\bibleverse{1} Los hijos de Leví: Gersón, Coat y Merari. \footnote{\textbf{6:1}
  1Cró 6,16; 1Cró 6,18} \bibleverse{2} Los hijos de Coat: Amram, Izhar,
Hebrón y Uziel. \bibleverse{3} Los hijos de Amram: Aarón, Moisés y
Miriam. Los hijos de Aarón: Nadab, Abiú, Eleazar e Itamar. \footnote{\textbf{6:3}
  Éxod 6,20; Éxod 6,23; Éxod 6,25}

\hypertarget{la-luxednea-de-sumo-sacerdote-desde-eleazar-hasta-el-cautiverio-babiluxf3nico}{%
\subsection{La línea de sumo sacerdote desde Eleazar hasta el cautiverio
babilónico}\label{la-luxednea-de-sumo-sacerdote-desde-eleazar-hasta-el-cautiverio-babiluxf3nico}}

\bibleverse{4} Eleazar fue el padre de Finehas, Finehas fue el padre de
Abisua, \bibleverse{5} Abisua fue el padre de Bukki. Bukki fue el padre
de Uzzi. \bibleverse{6} Uzí fue el padre de Zerahia. Zerahiah fue el
padre de Meraioth. \bibleverse{7} Meraioth fue el padre de Amarías.
Amarías fue el padre de Ajitub. \bibleverse{8} Ajitub fue el padre de
Sadoc. Sadoc fue el padre de Ahimaas. \footnote{\textbf{6:8} 2Sam 8,17;
  2Sam 15,27; 2Sam 15,36} \bibleverse{9} Ahimaas fue el padre de
Azarías. Azarías fue el padre de Johanán. \bibleverse{10} Johanán fue el
padre de Azarías, quien ejerció el oficio de sacerdote en la casa que
Salomón construyó en Jerusalén. \bibleverse{11} Azarías fue el padre de
Amarías. Amarías fue el padre de Ajitub. \bibleverse{12} Ajitub fue el
padre de Sadoc. Sadoc fue el padre de Salum. \bibleverse{13} Salum fue
el padre de Jilquías. Hilcías fue el padre de Azarías. \footnote{\textbf{6:13}
  2Re 22,4} \bibleverse{14} Azarías fue el padre de Seraías. Seraías fue
el padre de Josadac. \footnote{\textbf{6:14} 2Re 25,18; Esd 7,1; Neh
  12,26} \bibleverse{15} Josadac fue al cautiverio cuando Yahvé se llevó
a Judá y a Jerusalén de la mano de Nabucodonosor. \footnote{\textbf{6:15}
  2Re 25,21}

\hypertarget{los-descendientes-de-levi}{%
\subsection{Los descendientes de Levi}\label{los-descendientes-de-levi}}

\bibleverse{16} Los hijos de Leví: Gersón, Coat y Merari. \footnote{\textbf{6:16}
  1Cró 6,1; Éxod 6,16-19} \bibleverse{17} Estos son los nombres de los
hijos de Gersón Libni y Simei. \bibleverse{18} Los hijos de Coat fueron
Amram, Izhar, Hebrón y Uziel. \bibleverse{19} Los hijos de Merari: Mahli
y Mushi. Estas son las familias de los levitas según las familias de sus
padres. \bibleverse{20} De Gersón: Libni su hijo, Jahat su hijo, Zimma
su hijo, \bibleverse{21} Joah su hijo, Iddo su hijo, Zera su hijo y
Jeatherai su hijo. \bibleverse{22} Los hijos de Coat: Aminadab su hijo,
Coré su hijo, Asir su hijo, \footnote{\textbf{6:22} Éxod 6,24}
\bibleverse{23} Elcaná su hijo, Ebiasaf su hijo, Asir su hijo,
\bibleverse{24} Tahat su hijo, Uriel su hijo, Uzías su hijo y Shaúl su
hijo. \bibleverse{25} Los hijos de Elcana: Amasai y Ahimoth.
\bibleverse{26} En cuanto a Elcana, los hijos de Elcana: Zophai su hijo,
Nahath su hijo, \bibleverse{27} Eliab su hijo, Jeroham su hijo, y Elcana
su hijo. \footnote{\textbf{6:27} 1Sam 1,1} \bibleverse{28} Los hijos de
Samuel: el primogénito, Joel, y el segundo, Abías. \footnote{\textbf{6:28}
  1Sam 8,2} \bibleverse{29} Los hijos de Merari: Mahli, Libni su hijo,
Simei su hijo, Uza su hijo, \bibleverse{30} Simea su hijo, Haggia su
hijo, Asaías su hijo.

\hypertarget{las-tres-familias-de-cantantes-levuxedticos-hemuxe1n-asaf-y-etuxe1n}{%
\subsection{Las tres familias de cantantes Levíticos, Hemán, Asaf y
Etán}\label{las-tres-familias-de-cantantes-levuxedticos-hemuxe1n-asaf-y-etuxe1n}}

\bibleverse{31} Estos son los que David puso al frente del servicio del
canto en la casa de Yahvé, después de que el arca vino a descansar allí.
\bibleverse{32} Ellos ministraron con el canto ante el tabernáculo de la
Tienda de Reunión hasta que Salomón edificó la casa de Yavé en
Jerusalén. Desempeñaron los deberes de su cargo según su orden.
\bibleverse{33} Estos son los que servían, y sus hijos. De los hijos de
los coatitas Hemán el cantor, hijo de Joel, hijo de Samuel, \footnote{\textbf{6:33}
  1Cró 15,17} \bibleverse{34} hijo de Elcana, hijo de Jeroham, hijo de
Eliel, hijo de Toah, \bibleverse{35} hijo de Zuph, hijo de Elcana, hijo
de Mahat, hijo de Amasai, \bibleverse{36} hijo de Elcana, hijo de Joel,
hijo de Azarías, hijo de Sofonías, \bibleverse{37} hijo de Tahat, hijo
de Asir, hijo de Ebiasaf, hijo de Coré, \bibleverse{38} hijo de Izhar,
hijo de Coat, hijo de Leví, hijo de Israel. \bibleverse{39} Su hermano
Asaf, que estaba a su derecha, Asaf hijo de Berequías, hijo de Simea,
\footnote{\textbf{6:39} 1Cró 15,17} \bibleverse{40} hijo de Micael, hijo
de Baasías, hijo de Malquías, \bibleverse{41} hijo de Etni, hijo de
Zera, hijo de Adaías, \bibleverse{42} hijo de Etán, hijo de Zimma, hijo
de Simei, \bibleverse{43} hijo de Jahat, hijo de Gersón, hijo de Leví.
\bibleverse{44} A la izquierda sus hermanos los hijos de Merari: Etán
hijo de Cisí, hijo de Abdi, hijo de Malluch, \footnote{\textbf{6:44}
  1Cró 15,17} \bibleverse{45} hijo de Hasabías, hijo de Amasías, hijo de
Hilcías, \bibleverse{46} hijo de Amzi, hijo de Baní, hijo de Semer,
\bibleverse{47} hijo de Mahli, hijo de Musí, hijo de Merari, hijo de
Leví.

\hypertarget{los-levitas-y-los-aaronitas-en-el-servicio-del-templo}{%
\subsection{Los levitas y los aaronitas en el servicio del
templo}\label{los-levitas-y-los-aaronitas-en-el-servicio-del-templo}}

\bibleverse{48} Sus hermanos los levitas fueron designados para todo el
servicio del tabernáculo de la casa de Dios. \bibleverse{49} Pero Aarón
y sus hijos ofrecían en el altar de los holocaustos y en el altar del
incienso, para toda la obra del lugar santísimo y para hacer la
expiación por Israel, conforme a todo lo que había mandado Moisés,
siervo de Dios. \footnote{\textbf{6:49} Éxod 28,1; Lev 16,1}

\hypertarget{segunda-luxednea-de-sumos-sacerdotes-desde-aaruxf3n-hasta-ahimaas}{%
\subsection{Segunda línea de sumos sacerdotes desde Aarón hasta
Ahimaas}\label{segunda-luxednea-de-sumos-sacerdotes-desde-aaruxf3n-hasta-ahimaas}}

\bibleverse{50} Estos son los hijos de Aarón: Eleazar su hijo, Finees su
hijo, Abisua su hijo, \footnote{\textbf{6:50} 1Cró 6,3-8}
\bibleverse{51} Buki su hijo, Uzi su hijo, Zerahiah su hijo,
\bibleverse{52} Meraiot su hijo, Amarías su hijo, Ahitub su hijo,
\bibleverse{53} Sadoc su hijo, y Ahimaas su hijo.

\hypertarget{las-ciudades-levitas}{%
\subsection{Las ciudades levitas}\label{las-ciudades-levitas}}

\bibleverse{54} Estos son sus lugares de residencia según sus
campamentos en sus fronteras: a los hijos de Aarón, de las familias de
los coatitas (porque la suya fue la primera suerte), \bibleverse{55} les
dieron Hebrón en la tierra de Judá, y sus tierras de pastoreo alrededor
de ella; \bibleverse{56} pero los campos de la ciudad y sus aldeas, se
los dieron a Caleb hijo de Jefone. \bibleverse{57} A los hijos de Aarón
les dieron las ciudades de refugio, Hebrón, Libna con sus tierras de
pastoreo, Jattir, Estemoa con sus tierras de pastoreo, \bibleverse{58}
Hilen con sus tierras de pastoreo, Debir con sus tierras de pastoreo,
\bibleverse{59} Asán con sus tierras de pastoreo y Bet Semes con sus
tierras de pastoreo; \bibleverse{60} y de la tribu de Benjamín, Geba con
sus tierras de pastoreo, Allemeth con sus tierras de pastoreo y Anatot
con sus tierras de pastoreo. Todas las ciudades de sus familias eran
trece ciudades.

\bibleverse{61} A los demás hijos de Coat se les dio por sorteo, de la
familia de la tribu, de la media tribu, la mitad de Manasés, diez
ciudades. \footnote{\textbf{6:61} 1Cró 6,66-70} \bibleverse{62} A los
hijos de Gersón, según sus familias, de la tribu de Isacar, de la tribu
de Aser, de la tribu de Neftalí y de la tribu de Manasés en Basán, trece
ciudades. \footnote{\textbf{6:62} 1Cró 6,71-76} \bibleverse{63} A los
hijos de Merari se les dio por sorteo, según sus familias, de la tribu
de Rubén, de la tribu de Gad y de la tribu de Zabulón, doce ciudades.
\footnote{\textbf{6:63} 1Cró 6,77-81} \bibleverse{64} Los hijos de
Israel dieron a los levitas las ciudades con sus tierras de pastoreo.
\bibleverse{65} De la tribu de los hijos de Judá, de la tribu de los
hijos de Simeón y de la tribu de los hijos de Benjamín, dieron por
sorteo estas ciudades que se mencionan por su nombre. \footnote{\textbf{6:65}
  1Cró 6,55-60}

\bibleverse{66} Algunas de las familias de los hijos de Coat tenían
ciudades de sus fronteras fuera de la tribu de Efraín. \bibleverse{67}
Les dieron las ciudades de refugio, Siquem en la región montañosa de
Efraín con sus tierras de pastoreo y Gezer con sus tierras de pastoreo,
\bibleverse{68} Jokmeam con sus tierras de pastoreo, Bet Horón con sus
tierras de pastoreo, \bibleverse{69} Ajalón con sus tierras de pastoreo,
Gat Rimmón con sus tierras de pastoreo; \bibleverse{70} y de la media
tribu de Manasés, Aner con sus tierras de pastoreo y Bileam con sus
tierras de pastoreo, para el resto de la familia de los hijos de Coat.

\bibleverse{71} A los hijos de Gersón se les dio, de la familia de la
media tribu de Manasés, Golán en Basán con sus tierras de pastoreo, y
Astarot con sus tierras de pastoreo; \bibleverse{72} y de la tribu de
Isacar, Cedes con sus tierras de pastoreo, Daberat con sus tierras de
pastoreo, \bibleverse{73} Ramot con sus tierras de pastoreo, y Anem con
sus tierras de pastoreo; \bibleverse{74} y de la tribu de Aser, Mashal
con sus tierras de pastoreo, Abdón con sus tierras de pastoreo,
\bibleverse{75} Hukok con sus tierras de pastoreo, y Rehob con sus
tierras de pastoreo; \bibleverse{76} y de la tribu de Neftalí, Cedes en
Galilea con sus tierras de pastoreo, Hamón con sus tierras de pastoreo,
y Quiriatáim con sus tierras de pastoreo.

\bibleverse{77} Al resto de los levitas, hijos de Merari, se les dio, de
la tribu de Zabulón, Rimmono con sus tierras de pastoreo, y Tabor con
sus tierras de pastoreo; \bibleverse{78} y al otro lado del Jordán, en
Jericó, al lado oriental del Jordán, se les dio de la tribu de Rubén:
Beser en el desierto con sus tierras de pastoreo, Jahza con sus tierras
de pastoreo, \bibleverse{79} Cedemot con sus tierras de pastoreo y Mefat
con sus tierras de pastoreo; \bibleverse{80} y de la tribu de Gad, Ramot
en Galaad con sus tierras de pastoreo, Mahanaim con sus tierras de
pastoreo, \bibleverse{81} Hesbón con sus tierras de pastoreo y Jazer con
sus tierras de pastoreo.

\hypertarget{la-tribu-de-isacar}{%
\subsection{La tribu de Isacar}\label{la-tribu-de-isacar}}

\hypertarget{section-6}{%
\section{7}\label{section-6}}

\bibleverse{1} De los hijos de Isacar: Tola, Puah, Jasub y Simrón,
cuatro. \footnote{\textbf{7:1} Gén 46,13; Núm 26,23-24} \bibleverse{2}
Los hijos de Tola: Uzzi, Refaías, Jeriel, Jahmai, Ibsam y Semuel, jefes
de las casas paternas de Tola; hombres valientes en sus generaciones. Su
número en los días de David era de veintidós mil seiscientos.
\bibleverse{3} El hijo de Uzí: Izrahías. Los hijos de Izrahía: Miguel,
Obadías, Joel e Isías, cinco; todos ellos hombres principales.
\bibleverse{4} Con ellos, por sus generaciones, según las casas de sus
padres, había grupos del ejército para la guerra, treinta y seis mil;
porque tenían muchas mujeres e hijos. \bibleverse{5} Sus hermanos de
todas las familias de Isacar, hombres valientes, enumerados en su
totalidad por genealogía, eran ochenta y siete mil.

\hypertarget{la-tribu-de-benjamuxedn}{%
\subsection{La tribu de Benjamín}\label{la-tribu-de-benjamuxedn}}

\bibleverse{6} Los hijos de Benjamín: Bela, Becher y Jediael, tres.
\footnote{\textbf{7:6} 1Cró 8,1-2; Gén 46,21} \bibleverse{7} Los hijos
de Bela: Ezbón, Uzí, Uziel, Jerimot e Iri, cinco; jefes de familia,
hombres valientes; y fueron enumerados por genealogía veintidós mil
treinta y cuatro. \bibleverse{8} Los hijos de Becher: Zemira, Joás,
Eliezer, Elioenai, Omrí, Jeremot, Abías, Anatot y Alemet. Todos estos
fueron los hijos de Becher. \bibleverse{9} Fueron listados por
genealogía, según sus generaciones, jefes de las casas de sus padres,
hombres valientes, veinte mil doscientos. \bibleverse{10} El hijo de
Jediael: Bilhán. Los hijos de Bilhán: Jeús, Benjamín, Ehud, Quená,
Zetán, Tarsis y Ahishahar. \bibleverse{11} Todos estos fueron hijos de
Jediael, según los jefes de familia de sus padres, hombres valientes,
diecisiete mil doscientos, capaces de salir en el ejército para la
guerra. \bibleverse{12} También estaban Suppim, Huppim, los hijos de Ir,
Husim y los hijos de Aher.

\hypertarget{la-tribu-de-neftaluxed}{%
\subsection{La tribu de Neftalí}\label{la-tribu-de-neftaluxed}}

\bibleverse{13} Los hijos de Neftalí: Jahziel, Guni, Jezer, Salum y los
hijos de Bilhá. \footnote{\textbf{7:13} Gén 46,24}

\hypertarget{la-tribu-de-manasuxe9s}{%
\subsection{La tribu de Manasés}\label{la-tribu-de-manasuxe9s}}

\bibleverse{14} Los hijos de Manasés: Asriel, a quien dio a luz su
concubina la aramea. Ella dio a luz a Maquir, padre de Galaad.
\footnote{\textbf{7:14} Núm 26,29-33} \bibleverse{15} Maquir tomó una
esposa de Huppim y Suppim, cuya hermana se llamaba Maaca. El nombre de
la segunda era Zelofehad; y Zelofehad tuvo hijas. \footnote{\textbf{7:15}
  Núm 27,1} \bibleverse{16} Maaca, la mujer de Maquir, dio a luz un
hijo, al que llamó Peres. El nombre de su hermano fue Sheres, y sus
hijos fueron Ulam y Rakem. \bibleverse{17} Los hijos de Ulam: Bedán.
Estos fueron los hijos de Galaad, hijo de Maquir, hijo de Manasés.
\bibleverse{18} Su hermana Hamolecet dio a luz a Ishod, Abiezer y Mahá.
\bibleverse{19} Los hijos de Semida fueron Ahian, Siquem, Likhi y Aniam.

\hypertarget{la-tribu-de-ephraim}{%
\subsection{La tribu de Ephraim}\label{la-tribu-de-ephraim}}

\bibleverse{20} Los hijos de Efraín: Sutela, Bered su hijo, Tahat su
hijo, Eleada su hijo, Tahat su hijo, \footnote{\textbf{7:20} Núm 26,35}
\bibleverse{21} Zabad su hijo, Sutela su hijo, Ezer y Elead, a quienes
mataron los hombres de Gat que habían nacido en el país, porque bajaron
a quitarles el ganado. \bibleverse{22} Efraín, su padre, estuvo de luto
muchos días, y sus hermanos fueron a consolarlo. \bibleverse{23} Se
acercó a su mujer, y ella concibió y dio a luz un hijo, al que puso el
nombre de Beriá,\footnote{\textbf{7:23} ``Beriah'' es similar a la
  palabra hebrea para ``desgracia''.} porque había problemas con su
casa. \bibleverse{24} Su hija fue Sheerah, que construyó Beth Horon el
inferior y el superior, y Uzzen Sheerah. \bibleverse{25} Su hijo fue
Refa, su hijo Resef, su hijo Tela, su hijo Tahan, \bibleverse{26} Su
hijo Ladán, su hijo Ammihud, su hijo Elishama, \footnote{\textbf{7:26}
  Núm 1,10} \bibleverse{27} Su hijo Nun, y su hijo Josué. \footnote{\textbf{7:27}
  Núm 13,8}

\hypertarget{residencias-de-la-tribu}{%
\subsection{Residencias de la tribu}\label{residencias-de-la-tribu}}

\bibleverse{28} Sus posesiones y asentamientos fueron Betel y sus
poblaciones, al este Naarán, y al oeste Gezer con sus poblaciones;
también Siquem y sus poblaciones, hasta Azza y sus poblaciones;
\footnote{\textbf{7:28} Jos 16,1; Jos 16,10} \bibleverse{29} y por los
límites de los hijos de Manasés, Bet Sheán y sus poblaciones, Taanac y
sus poblaciones, Meguido y sus poblaciones, y Dor y sus poblaciones. En
ellas vivieron los hijos de José, hijo de Israel. \footnote{\textbf{7:29}
  Jos 17,11}

\hypertarget{la-tribu-de-asser}{%
\subsection{La tribu de Asser}\label{la-tribu-de-asser}}

\bibleverse{30} Los hijos de Aser: Imnah, Ishvah, Ishvi y Beriah. Serah
era su hermana. \footnote{\textbf{7:30} Gén 46,17} \bibleverse{31} Los
hijos de Beriá: Heber y Malquiel, que fue el padre de Birzait.
\bibleverse{32} Heber fue el padre de Jafet, de Shomer, de Hotham y de
su hermana Shua. \bibleverse{33} Los hijos de Jafet: Pasach, Bimhal y
Ashvath. Estos son los hijos de Jafet. \bibleverse{34} Los hijos de
Semer: Ahi, Rohgah, Jehubbah y Aram. \bibleverse{35} Los hijos de Helem,
su hermano: Zofa, Imna, Seles y Amal. \bibleverse{36} Los hijos de Zofa:
Suah, Harnefer, Shual, Beri, Imra, \bibleverse{37} Bezer, Hod, Shamma,
Shilshah, Ithran y Beera. \bibleverse{38} Los hijos de Jeter: Jephunneh,
Pispa y Ara. \bibleverse{39} Los hijos de Ulla: Ara, Hanniel y Rizia.
\bibleverse{40} Todos estos fueron los hijos de Aser, jefes de las casas
paternas, hombres selectos y valientes, jefes de los príncipes. El
número de ellos inscritos por genealogía para el servicio en la guerra
era de veintiséis mil hombres.

\hypertarget{hijos-y-descendientes-de-benjamuxedn-a-travuxe9s-de-bela}{%
\subsection{Hijos y descendientes de Benjamín a través de
Bela}\label{hijos-y-descendientes-de-benjamuxedn-a-travuxe9s-de-bela}}

\hypertarget{section-7}{%
\section{8}\label{section-7}}

\bibleverse{1} Benjamín fue el padre de Bela, su primogénito; Ashbel, el
segundo; Aharah, el tercero; \footnote{\textbf{8:1} Gén 46,21}
\bibleverse{2} Nohah, el cuarto, y Rapha, el quinto. \bibleverse{3} Bela
tuvo hijos: Addar, Gera, Abihud, \bibleverse{4} Abisua, Naamán, Ahoá,
\bibleverse{5} Gera, Sefufán y Huram.

\hypertarget{los-hijos-de-ehud}{%
\subsection{Los hijos de Ehud}\label{los-hijos-de-ehud}}

\bibleverse{6} Estos son los hijos de Ehud. Estos son los jefes de
familia de los habitantes de Geba, que fueron llevados cautivos a
Manahath: \bibleverse{7} Naamán, Ahijá y Gera, que los llevó cautivos; y
fue padre de Uza y Ahijud.

\hypertarget{la-familia-de-saharaim}{%
\subsection{La familia de Saharaim}\label{la-familia-de-saharaim}}

\bibleverse{8} Shaharaim fue padre de hijos en el campo de Moab, después
de haberlos despedido. Hushim y Baara fueron sus esposas. \bibleverse{9}
De Hodesh, su mujer, fue padre de Jobab, Zibia, Mesha, Malcam,
\bibleverse{10} Jeuz, Shachia y Mirmah. Estos fueron sus hijos, jefes de
familia de sus padres. \bibleverse{11} Por Hushim fue padre de Abitub y
Elpaal. \bibleverse{12} Los hijos de Elpaal: Eber, Misham y Shemed, que
edificaron Ono y Lod, con sus ciudades;

\hypertarget{cinco-familias-benjaminitas-en-ajaluxf3n-y-jerusaluxe9n}{%
\subsection{Cinco familias benjaminitas en Ajalón y
Jerusalén}\label{cinco-familias-benjaminitas-en-ajaluxf3n-y-jerusaluxe9n}}

\bibleverse{13} y Beriá y Sema, que fueron jefes de familia de los
habitantes de Ajalón, que pusieron en fuga a los habitantes de Gat;
\bibleverse{14} y Ahio, Sasac, Jeremot, \bibleverse{15} Zebadías, Arad,
Eder, \bibleverse{16} Miguel, Ispah, Joha, hijos de Beriá,
\bibleverse{17} Zebadías, Meshullam, Hizki, Heber, \bibleverse{18}
Ishmerai, Izliah, Jobab, hijos de Elpaal, \bibleverse{19} Jakim, Zichri,
Zabdi, \bibleverse{20} Elienai, Zillethai, Eliel, \bibleverse{21}
Adaiah, Beraiah, Shimrath, los hijos de Shimei, \bibleverse{22} Ishpan,
Eber, Eliel, \bibleverse{23} Abdon, Zichri, Hanan, \bibleverse{24}
Hananiah, Elam, Anthothijah, \bibleverse{25} Iphdeiah, Penuel, los hijos
de Shashak, \bibleverse{26} Shamsherai, Shehariah, Athaliah,
\bibleverse{27} Jaareshiah, Elijah, Zichri, y los hijos de Jeroham.
\bibleverse{28} Estos eran jefes de familia por sus generaciones,
hombres principales. Estos vivían en Jerusalén.

\hypertarget{la-familia-del-rey-sauxfal}{%
\subsection{La familia del rey Saúl}\label{la-familia-del-rey-sauxfal}}

\bibleverse{29} El padre de Gabaón, cuya mujer se llamaba Maaca, vivía
en Gabaón \footnote{\textbf{8:29} 1Cró 9,35-44} \bibleverse{30} con su
hijo primogénito Abdón, Zur, Cis, Baal, Nadab, \bibleverse{31} Gedor,
Ahio, Zécher, \bibleverse{32} y Miklot, que fue el padre de Simeá.
También vivían con sus familias en Jerusalén, cerca de sus parientes.
\bibleverse{33} Ner fue el padre de Kish. Cis fue el padre de Saúl. Saúl
fue el padre de Jonatán, Malquisúa, Abinadab y Eshbaal. \footnote{\textbf{8:33}
  1Sam 14,51} \bibleverse{34} El hijo de Jonatán fue Merib-baal.
Merib-baal fue el padre de Miqueas. \bibleverse{35} Los hijos de
Miqueas: Pitón, Melej, Tarea y Acaz. \bibleverse{36} Acaz fue el padre
de Joaddah. Y Joaddah fue padre de Alemeth, Azmaveth y Zimri. Zimri fue
el padre de Moza. \bibleverse{37} Moza fue el padre de Binea. Raphah fue
su hijo, Eleasah su hijo, y Azel su hijo. \bibleverse{38} Azel tuvo seis
hijos, cuyos nombres son estos Azricam, Boquerú, Ismael, Searías, Abdías
y Hanán. Todos estos fueron hijos de Azel. \bibleverse{39} Los hijos de
su hermano Eshek: Ulam su primogénito, Jeús el segundo y Elifelet el
tercero. \bibleverse{40} Los hijos de Ulam fueron hombres valientes,
arqueros, y tuvieron muchos hijos y nietos, ciento cincuenta. Todos
ellos eran de los hijos de Benjamín. \footnote{\textbf{8:40} 1Cró 12,2}

\hypertarget{directorio-de-residentes-destacados-de-jerusaluxe9n-en-el-peruxedodo-posterior-al-cautiverio}{%
\subsection{Directorio de residentes destacados de Jerusalén (en el
período posterior al
cautiverio)}\label{directorio-de-residentes-destacados-de-jerusaluxe9n-en-el-peruxedodo-posterior-al-cautiverio}}

\hypertarget{section-8}{%
\section{9}\label{section-8}}

\bibleverse{1} Así que todo Israel fue enumerado por genealogías; y he
aquí que\footnote{\textbf{9:1} ``He aquí'', de ``\hebrew{הִנֵּה}'',
  significa mirar, fijarse, observar, ver o contemplar. Se utiliza a
  menudo como interjección.} están escritas en el libro de los reyes de
Israel. Judá fue llevado cautivo a Babilonia por su desobediencia.
\footnote{\textbf{9:1} 2Re 24,15-16} \bibleverse{2} Los primeros
habitantes que vivían en sus posesiones, en sus ciudades, eran los
israelitas, los sacerdotes, los levitas y los servidores del templo.
\footnote{\textbf{9:2} Jos 9,23; Esd 8,20}

\hypertarget{el-pueblo-de-jerusaluxe9n}{%
\subsection{El pueblo de Jerusalén}\label{el-pueblo-de-jerusaluxe9n}}

\bibleverse{3} En Jerusalén vivían de los hijos de Judá, de los hijos de
Benjamín y de los hijos de Efraín y Manasés: \footnote{\textbf{9:3} Neh
  11,3-19} \bibleverse{4} Utaí hijo de Ammihud, hijo de Omrí, hijo de
Imrí, hijo de Baní, de los hijos de Pérez hijo de Judá. \bibleverse{5}
De los silonitas Asaías el primogénito y sus hijos. \bibleverse{6} De
los hijos de Zera: Jeuel y sus hermanos, seiscientos noventa.
\bibleverse{7} De los hijos de Benjamín: Salú, hijo de Mesulam, hijo de
Hodavías, hijo de Hasenúa; \bibleverse{8} e Ibneías, hijo de Jeroham, y
Elá, hijo de Uzi, hijo de Micrí; y Mesulam, hijo de Sefatías, hijo de
Reuel, hijo de Ibniá; \bibleverse{9} y sus hermanos, según sus
generaciones, novecientos cincuenta y seis. Todos estos hombres eran
jefes de familia por las casas de sus padres.

\bibleverse{10} De los sacerdotes: Jedaías, Joiarib, Jacín,
\bibleverse{11} y Azarías hijo de Hilcías, hijo de Mesulam, hijo de
Sadoc, hijo de Meraiot, hijo de Ajitub, jefe de la casa de Dios;
\footnote{\textbf{9:11} 1Cró 6,13} \bibleverse{12} y Adaías hijo de
Jeroham, hijo de Pashur, hijo de Malquías y Maasai hijo de Adiel, hijo
de Jahzerah, hijo de Meshullam, hijo de Meshillemith, hijo de Immer;
\bibleverse{13} y sus hermanos, jefes de las casas de sus padres, mil
setecientos sesenta; eran hombres muy capaces para la obra del servicio
de la casa de Dios.

\bibleverse{14} De los levitas Semaías hijo de Hasub, hijo de Azricam,
hijo de Hasabías, de los hijos de Merari; \bibleverse{15} y Bacbacar,
Heres, Galal y Mattanías hijo de Mica, hijo de Zicri, hijo de Asaf,
\bibleverse{16} y Abdías hijo de Semaías, hijo de Galal, hijo de
Jedutún; y Berequías hijo de Asá, hijo de Elcana, que vivían en las
aldeas de los netofatitas. \footnote{\textbf{9:16} 1Cró 2,54}

\hypertarget{los-porteros-y-sus-servicios}{%
\subsection{Los porteros y sus
servicios}\label{los-porteros-y-sus-servicios}}

\bibleverse{17} Los porteros: Salum, Acub, Talmón, Ahimán y sus hermanos
(Salum era el jefe), \bibleverse{18} que antes servían en la puerta del
rey hacia el este. Eran los porteros del campamento de los hijos de
Leví. \bibleverse{19} Salum era hijo de Coré, hijo de Ebiasaf, hijo de
Coré, y sus hermanos, de la casa de su padre, los corasitas, estaban a
cargo del trabajo del servicio, guardianes de los umbrales de la tienda.
Sus padres habían estado sobre el campamento de Yahvé, guardianes de la
entrada. \footnote{\textbf{9:19} Núm 4,18-20} \bibleverse{20} Finees,
hijo de Eleazar, era el jefe de ellos en el pasado, y el Señor estaba
con él. \footnote{\textbf{9:20} Núm 25,7-13} \bibleverse{21} Zacarías,
hijo de Meselemías, era el guardián de la puerta de la Tienda del
Encuentro. \bibleverse{22} Todos estos que fueron elegidos para ser
porteros en los umbrales fueron doscientos doce. Estos fueron listados
por genealogía en sus pueblos, a quienes David y Samuel el vidente
ordenaron en su cargo de confianza. \footnote{\textbf{9:22} 1Sam 9,9;
  1Sam 9,11} \bibleverse{23} Ellos y sus hijos tenían la vigilancia de
las puertas de la casa de Yavé, la casa de la tienda, como guardianes.
\bibleverse{24} En los cuatro lados estaban los guardianes de las
puertas, hacia el este, el oeste, el norte y el sur. \bibleverse{25} Sus
hermanos, en sus aldeas, debían entrar cada siete días para estar con
ellos,

\hypertarget{informaciuxf3n-sobre-los-deberes-oficiales-de-los-levitas}{%
\subsection{Información sobre los deberes oficiales de los
levitas}\label{informaciuxf3n-sobre-los-deberes-oficiales-de-los-levitas}}

\bibleverse{26} porque los cuatro principales porteros, que eran
levitas, tenían un cargo de confianza y estaban a cargo de las
habitaciones y de los tesoros en la casa de Dios. \bibleverse{27} Ellos
permanecían alrededor de la casa de Dios, porque ese era su deber; y era
su deber abrirla de mañana en mañana.

\bibleverse{28} Algunos de ellos estaban a cargo de los utensilios del
servicio, pues éstos se traían por cuenta, y éstos se sacaban por
cuenta. \bibleverse{29} Algunos de ellos también estaban encargados de
los muebles y de todos los utensilios del santuario, de la harina fina,
del vino, del aceite, del incienso y de las especias.

\bibleverse{30} Algunos de los hijos de los sacerdotes preparaban la
mezcla de las especias. \footnote{\textbf{9:30} Éxod 30,23-25}
\bibleverse{31} Matatías, uno de los levitas, que era primogénito de
Salum el coreíta, tenía el cargo de confianza sobre las cosas que se
cocían en las ollas. \bibleverse{32} Algunos de sus hermanos, de los
hijos de los coatitas, estaban sobre el pan de la feria, para prepararlo
cada sábado. \footnote{\textbf{9:32} Lev 24,5; Lev 24,8}

\hypertarget{informaciuxf3n-sobre-los-cantantes-del-templo-palabra-final}{%
\subsection{Información sobre los cantantes del templo; Palabra
final}\label{informaciuxf3n-sobre-los-cantantes-del-templo-palabra-final}}

\bibleverse{33} Estos son los cantores, jefes de familia de los levitas,
que vivían en las habitaciones y estaban libres de cualquier otro
servicio, pues se empleaban en su trabajo de día y de noche. \footnote{\textbf{9:33}
  1Cró 9,14-16} \bibleverse{34} Estos eran jefes de familia de los
levitas, por sus generaciones, hombres principales. Vivían en Jerusalén.

\hypertarget{apuxe9ndice-los-habitantes-de-gabauxf3n-y-una-segunda-genealoguxeda-de-la-casa-de-sauxfal}{%
\subsection{Apéndice: Los habitantes de Gabaón y una segunda genealogía
de la casa de
Saúl}\label{apuxe9ndice-los-habitantes-de-gabauxf3n-y-una-segunda-genealoguxeda-de-la-casa-de-sauxfal}}

\bibleverse{35} Jeiel, padre de Gabaón, cuya mujer se llamaba Maaca,
vivía en Gabaón. \footnote{\textbf{9:35} 1Cró 8,29-38} \bibleverse{36}
Su hijo primogénito fue Abdón, luego Zur, Cis, Baal, Ner, Nadab,
\bibleverse{37} Gedor, Ahio, Zacarías y Miklot. \bibleverse{38} Mikloth
fue el padre de Shimeam. Ellos también vivieron con sus parientes en
Jerusalén, cerca de sus parientes. \bibleverse{39} Ner fue el padre de
Cis. Cis fue el padre de Saúl. Saúl fue el padre de Jonatán, Malquisúa,
Abinadab y Eshbaal. \bibleverse{40} El hijo de Jonatán fue Merib-baal.
Merib-baal fue el padre de Miqueas. \bibleverse{41} Los hijos de
Miqueas: Pitón, Melec, Tahrea y Acaz. \bibleverse{42} Acaz fue el padre
de Jarah. Jarah fue el padre de Alemeth, Azmaveth y Zimri. Zimri fue el
padre de Moza. \bibleverse{43} Moza fue padre de Binea, de su hijo
Refaías, de su hijo Eleasá y de su hijo Azel. \bibleverse{44} Azel tuvo
seis hijos, cuyos nombres son Azrikam, Boquerú, Ismael, Searías, Obadías
y Hanán. Estos fueron los hijos de Azel.

\hypertarget{israel-derrotado-por-los-filisteos-en-el-monte-gilboa-muerte-de-sauxfal-y-sus-tres-hijos}{%
\subsection{Israel derrotado por los filisteos en el monte Gilboa;
Muerte de Saúl y sus tres
hijos}\label{israel-derrotado-por-los-filisteos-en-el-monte-gilboa-muerte-de-sauxfal-y-sus-tres-hijos}}

\hypertarget{section-9}{%
\section{10}\label{section-9}}

\bibleverse{1} Los filisteos lucharon contra Israel, y los hombres de
Israel huyeron de la presencia de los filisteos y cayeron muertos en el
monte Gilboa. \bibleverse{2} Los filisteos siguieron con fuerza a Saúl y
a sus hijos, y los filisteos mataron a Jonatán, a Abinadab y a
Malquisúa, hijos de Saúl. \bibleverse{3} La batalla fue dura contra
Saúl, y los arqueros lo alcanzaron; y él estaba angustiado a causa de
los arqueros. \bibleverse{4} Entonces Saúl le dijo a su armero: ``Saca
tu espada y traspásame con ella, no sea que estos incircuncisos vengan a
abusar de mí.'' Pero su portador de armadura no quiso, porque estaba
aterrorizado. Entonces Saúl tomó su espada y cayó sobre ella.
\bibleverse{5} Cuando el portador de su armadura vio que Saúl estaba
muerto, él también cayó sobre su espada y murió. \bibleverse{6} Así
murió Saúl con sus tres hijos, y toda su casa murió junta.
\bibleverse{7} Cuando todos los hombres de Israel que estaban en el
valle vieron que huían, y que Saúl y sus hijos estaban muertos,
abandonaron sus ciudades y huyeron, y los filisteos vinieron y habitaron
en ellas.

\hypertarget{el-destino-de-los-caduxe1veres-de-sauxfal-y-sus-hijos}{%
\subsection{El destino de los cadáveres de Saúl y sus
hijos}\label{el-destino-de-los-caduxe1veres-de-sauxfal-y-sus-hijos}}

\bibleverse{8} Al día siguiente, cuando los filisteos fueron a despojar
a los muertos, encontraron a Saúl y a sus hijos caídos en el monte
Gilboa. \bibleverse{9} Lo despojaron y tomaron su cabeza y su armadura,
y luego enviaron a la tierra de los filisteos por todos lados para
llevar la noticia a sus ídolos y al pueblo. \bibleverse{10} Pusieron su
armadura en la casa de sus dioses, y fijaron su cabeza en la casa de
Dagón. \bibleverse{11} Cuando todo Jabes de Galaad se enteró de todo lo
que los filisteos le habían hecho a Saúl, \bibleverse{12} todos los
hombres valientes se levantaron y se llevaron el cuerpo de Saúl y los
cuerpos de sus hijos, y los llevaron a Jabes, y enterraron sus huesos
bajo la encina en Jabes, y ayunaron siete días. \footnote{\textbf{10:12}
  2Sam 2,5}

\hypertarget{revisiuxf3n-de-la-deuda-de-sauxfal-con-dios}{%
\subsection{Revisión de la deuda de Saúl con
Dios}\label{revisiuxf3n-de-la-deuda-de-sauxfal-con-dios}}

\bibleverse{13} Así pues, Saúl murió por la infracción que cometió
contra Yavé, a causa de la palabra de Yavé, que no cumplió, y también
porque pidió consejo a uno que tenía un espíritu familiar, para
consultar, \footnote{\textbf{10:13} 1Sam 15,11; 1Sam 28,8}
\bibleverse{14} y no consultó a Yavé. Por eso lo mató, y entregó el
reino a David, hijo de Jesé.

\hypertarget{la-unciuxf3n-de-david-en-hebruxf3n-y-la-conquista-de-jerusaluxe9n}{%
\subsection{La unción de David en Hebrón y la conquista de
Jerusalén}\label{la-unciuxf3n-de-david-en-hebruxf3n-y-la-conquista-de-jerusaluxe9n}}

\hypertarget{section-10}{%
\section{11}\label{section-10}}

\bibleverse{1} Entonces todo Israel se reunió con David en Hebrón,
diciendo: ``He aquí que somos tu hueso y tu carne. \footnote{\textbf{11:1}
  Gén 29,14} \bibleverse{2} En tiempos pasados, cuando Saúl era rey,
fuiste tú quien sacó y trajo a Israel. El Señor, tu Dios, te dijo: `Tú
serás el pastor de mi pueblo Israel, y tú serás el príncipe de mi pueblo
Israel'\,''.

\bibleverse{3} Así que todos los ancianos de Israel vinieron a ver al
rey a Hebrón, y David hizo un pacto con ellos en Hebrón ante Yavé.
Ungieron a David como rey de Israel, según la palabra de Yahvé por medio
de Samuel. \footnote{\textbf{11:3} 1Sam 16,1; 1Sam 16,3; 1Sam 16,12}

\bibleverse{4} David y todo Israel se dirigieron a Jerusalén (también
llamada Jebús), y los jebuseos, habitantes de la tierra, estaban allí.
\bibleverse{5} Los habitantes de Jebús dijeron a David: ``¡No entrarás
aquí!'' Sin embargo, David tomó la fortaleza de Sión. La misma es la
ciudad de David. \bibleverse{6} David había dicho: ``El que ataque
primero a los jebuseos será jefe y capitán''. Joab, hijo de Sarvia,
subió primero y fue nombrado jefe. \bibleverse{7} David vivía en la
fortaleza; por eso la llamaban la ciudad de David. \bibleverse{8} Él
edificó la ciudad por todas partes, desde Millo hasta los alrededores; y
Joab reparó el resto de la ciudad. \bibleverse{9} David crecía cada vez
más, porque el Señor de los Ejércitos estaba con él.

\hypertarget{directorio-y-hazauxf1as-de-los-guerreros-de-david}{%
\subsection{Directorio y hazañas de los guerreros de
David}\label{directorio-y-hazauxf1as-de-los-guerreros-de-david}}

\bibleverse{10} Estos son los principales de los valientes que tenía
David, que se mostraron fuertes con él en su reino, junto con todo
Israel, para hacerlo rey, según la palabra de Yahvé sobre Israel.
\footnote{\textbf{11:10} 2Sam 23,8-39}

\bibleverse{11} Este es el número de los valientes que tenía David
Jashobeam, hijo de un hakmonita, jefe de los treinta; alzó su lanza
contra trescientos y los mató a la vez. \footnote{\textbf{11:11} 1Cró
  27,2} \bibleverse{12} Después de él estaba Eleazar, hijo de Dodo,
ahohita, que era uno de los tres valientes. \footnote{\textbf{11:12}
  1Cró 27,4} \bibleverse{13} Estaba con David en Pasdammim, y allí se
reunieron los filisteos para combatir, donde había un terreno lleno de
cebada; y el pueblo huyó de delante de los filisteos. \bibleverse{14} Se
pusieron en medio de la parcela, la defendieron y mataron a los
filisteos; y el Señor los salvó con una gran victoria.

\hypertarget{wagnis-dreier-helden}{%
\subsection{Wagnis dreier Helden}\label{wagnis-dreier-helden}}

\bibleverse{15} Tres de los treinta jefes bajaron a la roca a David, a
la cueva de Adulam; y el ejército de los filisteos estaba acampado en el
valle de Refaim. \footnote{\textbf{11:15} 1Sam 22,1} \bibleverse{16}
David estaba entonces en la fortaleza, y la guarnición de los filisteos
estaba en Belén en ese momento. \bibleverse{17} David anhelaba y decía:
``¡Oh, si alguien me diera de beber agua del pozo de Belén, que está
junto a la puerta!''

\bibleverse{18} Los tres atravesaron el ejército de los filisteos y
sacaron agua del pozo de Belén que estaba junto a la puerta, la tomaron
y se la llevaron a David; pero éste no quiso beberla, sino que la
derramó a Yahvé, \bibleverse{19} y dijo: ``¡Mi Dios me prohíbe que haga
esto! ¿He de beber la sangre de estos hombres que han puesto su vida en
peligro?'' Pues arriesgaron sus vidas para traerla. Por eso no quiso
beberla. Los tres hombres poderosos hicieron estas cosas.

\hypertarget{abisai-y-benauxedas}{%
\subsection{Abisai y Benaías}\label{abisai-y-benauxedas}}

\bibleverse{20} Abisai, hermano de Joab, era el jefe de los tres, pues
levantó su lanza contra trescientos y los mató, y tenía un nombre entre
los tres. \bibleverse{21} De los tres, él era más honorable que los dos,
y fue nombrado su capitán; sin embargo, no fue incluido entre los tres.

\bibleverse{22} Benaía, hijo de Joiada, hijo de un valiente de Kabzeel,
que había hecho obras poderosas, mató a los dos hijos de Ariel de Moab.
También bajó y mató a un león en medio de un pozo en un día de nieve.
\bibleverse{23} Mató a un egipcio, un hombre de gran estatura, de cinco
codos\footnote{\textbf{11:23} Un codo es la longitud desde la punta del
  dedo corazón hasta el codo del brazo de un hombre, o sea, unas 18
  pulgadas o 46 centímetros. Por lo tanto, este egipcio medía alrededor
  de 7 pies y 6 pulgadas o 2,28 metros de altura.} de altura. En la mano
del egipcio había una lanza como la de un telar; y él bajó hacia él con
un bastón, arrancó la lanza de la mano del egipcio y lo mató con su
propia lanza. \footnote{\textbf{11:23} 1Sam 17,7; 1Sam 17,40; 1Sam 17,51}
\bibleverse{24} Benaía, hijo de Joiada, hizo estas cosas y tuvo un
nombre entre los tres valientes. \footnote{\textbf{11:24} 1Cró 27,5-6}
\bibleverse{25} Era más honorable que los treinta, pero no llegó a los
tres; y David lo puso al frente de su guardia.

\hypertarget{una-lista-de-otros-huxe9roes-de-david}{%
\subsection{Una lista de otros héroes de
David}\label{una-lista-de-otros-huxe9roes-de-david}}

\bibleverse{26} Entre los valientes de los ejércitos se encuentran
también Asahel hermano de Joab, Elhanán hijo de Dodo de Belén,
\bibleverse{27} Sammot harorita, Helez pelonita, \footnote{\textbf{11:27}
  1Cró 27,8; 1Cró 27,10} \bibleverse{28} Ira hijo de Ikkesh tekoita,
Abiezer anatotita, \bibleverse{29} Sibecai husatita, Ilai ahohita,
\bibleverse{30} Maharai netofatita, Heled hijo de Baana netofatita,
\bibleverse{31} Itai hijo de Ribai de Gabaa de los hijos de Benjamín,
Benaía el Piratonita, \bibleverse{32} Hurai de los arroyos de Gaas,
Abiel el Arbateo, \bibleverse{33} Azmaveth el Baharumita, Eliahba el
Shaalbonita, \bibleverse{34} los hijos de Hashem el Gizonita, Jonatán
hijo de Shagee el Hararita, \bibleverse{35} Ahiam hijo de Sacar el
Hararita, Elifal hijo de Ur, \bibleverse{36} Hepher el Mequeratita,
Ahijah el Pelonita, \bibleverse{37} Hezro el carmelita, Naarai el hijo
de Ezbai, \bibleverse{38} Joel el hermano de Natán, Mibhar el hijo de
Hagri, \bibleverse{39} Zelek el amonita, Naharai el berotita (el
portador de la armadura de Joab el hijo de Zeruiah), \bibleverse{40} Ira
el itrita, Gareb el itrita, \bibleverse{41} Urías el hitita, Zabad el
hijo de Ahlai, \footnote{\textbf{11:41} 2Sam 11,3} \bibleverse{42} Adina
el hijo de Shiza el rubenita (un jefe de los rubenitas), y treinta con
él, \bibleverse{43} Hanan hijo de Maacah, Josafat mitnita,
\bibleverse{44} Uzia asterita, Shama y Jeiel hijos de Hotham aroerita,
\bibleverse{45} Jediael hijo de Shimri, y Joha su hermano tizita,
\bibleverse{46} Eliel mahavita, Jeribai y Josavita, hijos de Elnaam, e
Ithmah moabita, \bibleverse{47} Eliel, Obed y Jaasiel mezobaita.

\hypertarget{los-seguidores-de-david-en-siclag-y-adullam-mientras-sauxfal-todavuxeda-estaba-vivo}{%
\subsection{Los seguidores de David en Siclag y Adullam mientras Saúl
todavía estaba
vivo}\label{los-seguidores-de-david-en-siclag-y-adullam-mientras-sauxfal-todavuxeda-estaba-vivo}}

\hypertarget{section-11}{%
\section{12}\label{section-11}}

\bibleverse{1} Estos son los que vinieron a David a Siclag cuando estaba
fugitivo de Saúl, hijo de Cis. Estaban entre los hombres poderosos, sus
ayudantes en la guerra. \footnote{\textbf{12:1} 1Sam 27,6}
\bibleverse{2} Estaban armados con arcos, y podían usar tanto la mano
derecha como la izquierda para lanzar piedras y tirar flechas con el
arco. Eran de los parientes de Saúl de la tribu de Benjamín. \footnote{\textbf{12:2}
  1Cró 8,40} \bibleverse{3} El jefe era Ahiezer, luego Joás, hijos de
Semaá el gabatita; Jeziel y Pelet, hijos de Azmavet; Beracá; Jehú el
anatotita; \bibleverse{4} Ismaías el gabatita, hombre poderoso entre los
treinta y jefe de los treinta; Jeremías; Jahaziel; Johanán; Jozabad el
gederatita; \footnote{\textbf{12:4} 1Cró 25,18} \bibleverse{5} Eluzai;
Jerimot; Bealías; Semarías; Sefatías el harupita; \bibleverse{6} Elcaná,
Isías, Azarel, Joezer y Jashobeam, los corasitas; \bibleverse{7} y Joelá
y Zebadías, hijos de Jeroham de Gedor. \footnote{\textbf{12:7} 2Sam 2,18}

\bibleverse{8} Algunos gaditas se unieron a David en la fortaleza del
desierto, hombres valientes y entrenados para la guerra, que sabían
manejar el escudo y la lanza; sus rostros eran como los de los leones, y
eran tan veloces como las gacelas de los montes: \bibleverse{9} Ezer, el
principal; Obadías, el segundo; Eliab, el tercero; \bibleverse{10}
Mismaná, el cuarto; Jeremías, el quinto; \bibleverse{11} Atai, el sexto;
Eliel, el séptimo; \bibleverse{12} Johanán, el octavo; Elzabad, el
noveno; \bibleverse{13} Jeremías, el décimo; y Maqubannai, el undécimo.
\bibleverse{14} Estos de los hijos de Gad eran capitanes del ejército.
El menor era igual a cien, y el mayor a mil. \bibleverse{15} Estos son
los que pasaron el Jordán en el primer mes, cuando se desbordó por todas
sus orillas; y pusieron en fuga a todos los que vivían en los valles,
tanto hacia el oriente como hacia el occidente.

\bibleverse{16} Algunos de los hijos de Benjamín y de Judá vinieron a la
fortaleza a David. \bibleverse{17} David salió a recibirlos y les
respondió: ``Si habéis venido pacíficamente a ayudarme, mi corazón se
unirá a vosotros; pero si habéis venido a entregarme a mis adversarios,
ya que no hay mal en mis manos, que el Dios de nuestros padres lo vea y
lo reprenda.'' \bibleverse{18} Entonces el Espíritu vino sobre Amasai,
que era el jefe de los treinta, y dijo: ``Somos tuyos, David, y de tu
parte, hijo de Jesé. Paz, paz a ti, y paz a tus ayudantes, porque tu
Dios te ayuda''. Entonces David los recibió y los nombró capitanes de la
banda. \footnote{\textbf{12:18} 1Sam 29,4}

\bibleverse{19} Algunos de Manasés también se unieron a David cuando
vino con los filisteos contra Saúl a la batalla, pero no los ayudaron,
pues los señores de los filisteos lo despidieron después de consultarlo,
diciendo: ``Desertará con su amo Saúl con peligro de nuestras cabezas.''

\bibleverse{20} Cuando se dirigía a Siclag, se le unieron algunos de
Manasés: Adná, Jozabad, Jediael, Miguel, Jozabad, Eliú y Zilletai,
capitanes de millares que eran de Manasés. \bibleverse{21} Ellos
ayudaron a David contra la banda de asaltantes, pues todos eran hombres
valientes y capitanes del ejército. \bibleverse{22} Porque de día en día
venían hombres a ayudar a David, hasta que hubo un gran ejército, como
el ejército de Dios.

\bibleverse{23} Estos son los números de los jefes de los que estaban
armados para la guerra, que vinieron a David a Hebrón para entregarle el
reino de Saúl, según la palabra de Yahvé.

\hypertarget{nuxfamero-de-guerreros-en-la-elecciuxf3n-de-david-como-rey-en-hebruxf3n}{%
\subsection{Número de guerreros en la elección de David como rey en
Hebrón}\label{nuxfamero-de-guerreros-en-la-elecciuxf3n-de-david-como-rey-en-hebruxf3n}}

\bibleverse{24} Los hijos de Judá que llevaban escudo y lanza eran seis
mil ochocientos, armados para la guerra. \bibleverse{25} De los hijos de
Simeón, hombres valientes para la guerra: siete mil cien.
\bibleverse{26} De los hijos de Leví, cuatro mil seiscientos.
\bibleverse{27} Joiada era el jefe de la casa de Aarón, y con él había
tres mil setecientos, \footnote{\textbf{12:27} 2Sam 15,24; 1Cró 6,8}
\bibleverse{28} y Sadoc, joven valiente, y de la casa de su padre
veintidós capitanes. \bibleverse{29} De los hijos de Benjamín, parientes
de Saúl, tres mil, pues hasta entonces la mayor parte de ellos había
mantenido su fidelidad a la casa de Saúl. \bibleverse{30} De los hijos
de Efraín: veinte mil ochocientos, hombres valientes y famosos en las
casas de sus padres. \bibleverse{31} De la media tribu de Manasés:
dieciocho mil, que fueron mencionados por su nombre, para venir a hacer
rey a David. \bibleverse{32} De los hijos de Isacar, hombres entendidos
en los tiempos, para saber lo que debía hacer Israel, sus jefes eran
doscientos; y todos sus hermanos estaban a sus órdenes. \bibleverse{33}
De Zabulón, los que podían salir en el ejército, que podían preparar la
batalla con toda clase de instrumentos de guerra: cincuenta mil que
podían mandar y no tenían doblez de corazón. \bibleverse{34} De Neftalí:
mil capitanes, y con ellos, con escudo y lanza, treinta y siete mil.
\bibleverse{35} De los danitas que sabían preparar la batalla:
veintiocho mil seiscientos. \bibleverse{36} De Aser, los que podían
salir en el ejército, los que podían preparar la batalla: cuarenta mil.
\bibleverse{37} Del otro lado del Jordán, de los rubenitas, gaditas y de
la media tribu de Manasés, con toda clase de instrumentos de guerra para
la batalla: ciento veinte mil.

\bibleverse{38} Todos estos eran hombres de guerra que sabían ordenar la
formación de la batalla, y vinieron con un corazón perfecto a Hebrón
para hacer a David rey de todo Israel; y también todos los demás de
Israel tenían un mismo corazón para hacer a David rey. \bibleverse{39}
Estuvieron allí con David tres días, comiendo y bebiendo, pues sus
hermanos les habían proporcionado provisiones. \bibleverse{40} Además,
los que estaban cerca de ellos, hasta Isacar, Zabulón y Neftalí,
trajeron pan en burros, en camellos, en mulos y en bueyes: provisiones
de harina, tortas de higos, racimos de pasas, vino, aceite, ganado y
ovejas en abundancia; porque había alegría en Israel.

\hypertarget{movilizaciuxf3n-de-todo-el-pueblo-con-fines-de-recuperaciuxf3n}{%
\subsection{Movilización de todo el pueblo con fines de
recuperación}\label{movilizaciuxf3n-de-todo-el-pueblo-con-fines-de-recuperaciuxf3n}}

\hypertarget{section-12}{%
\section{13}\label{section-12}}

\bibleverse{1} David consultó con los capitanes de millares y de
centenas, incluso con cada jefe. \bibleverse{2} David dijo a toda la
asamblea de Israel: ``Si les parece bien, y si es de Yahvé nuestro Dios,
mandemos a decir a nuestros hermanos que han quedado en toda la tierra
de Israel, con los sacerdotes y levitas que están en sus ciudades que
tienen tierras de pastoreo, que se reúnan con nosotros. \bibleverse{3}
Además, traigamos de nuevo el arca de nuestro Dios, pues no la buscamos
en los días de Saúl.''

\bibleverse{4} Toda la asamblea dijo que lo harían, porque la cosa era
justa a los ojos de todo el pueblo. \bibleverse{5} Entonces David reunió
a todo Israel, desde el río Shihor de Egipto hasta la entrada de Hamat,
para traer el arca de Dios desde Quiriat Jearim.

\hypertarget{fracaso-del-plan}{%
\subsection{Fracaso del plan}\label{fracaso-del-plan}}

\bibleverse{6} David subió con todo Israel a Baalá, es decir, a Quiriat
Jearim, que pertenecía a Judá, para hacer subir desde allí el arca de
Dios que está sentada encima de los querubines, que se llama por el
Nombre. \footnote{\textbf{13:6} Jos 15,9} \bibleverse{7} Llevaron el
arca de Dios en un carro nuevo, y la sacaron de la casa de Abinadab;
Uzza y Ahio conducían el carro. \bibleverse{8} David y todo Israel
tocaron ante Dios con toda su fuerza, con cantos, con arpas, con
instrumentos de cuerda, con panderetas, con címbalos y con trompetas.

\bibleverse{9} Cuando llegaron a la era de Chidón, Uza extendió su mano
para sostener el arca, pues los bueyes tropezaron. \bibleverse{10} La
ira de Yavé se encendió contra Uza, y lo hirió porque había puesto su
mano sobre el arca; y allí murió ante Dios. \bibleverse{11} David se
disgustó porque Yavé se había ensañado con Uza. Llamó a ese lugar Pérez
Uzza, hasta el día de hoy.

\hypertarget{el-cajuxf3n-se-encuentra-en-la-casa-de-obed-edom}{%
\subsection{El cajón se encuentra en la casa de
Obed-Edom}\label{el-cajuxf3n-se-encuentra-en-la-casa-de-obed-edom}}

\bibleverse{12} Ese día David tuvo miedo de Dios, diciendo: ``¿Cómo voy
a llevar el arca de Dios a mi casa?'' \bibleverse{13} Así que David no
trasladó el arca con él a la ciudad de David, sino que la llevó a un
lado, a la casa de Obed-Edom el gitano. \bibleverse{14} El arca de Dios
permaneció con la familia de Obed-Edom en su casa durante tres meses; y
el Señor bendijo la casa de Obed-Edom y todo lo que tenía.

\hypertarget{el-edificio-del-palacio-de-david-y-los-nuevos-matrimonios-sus-guerras-victoriosas-con-los-filisteos}{%
\subsection{El edificio del palacio de David y los nuevos matrimonios;
sus guerras victoriosas con los
filisteos}\label{el-edificio-del-palacio-de-david-y-los-nuevos-matrimonios-sus-guerras-victoriosas-con-los-filisteos}}

\hypertarget{section-13}{%
\section{14}\label{section-13}}

\bibleverse{1} Hiram, rey de Tiro, envió mensajeros a David con cedros,
albañiles y carpinteros para que le construyeran una casa.
\bibleverse{2} David se dio cuenta de que Yavé lo había establecido como
rey de Israel, pues su reino era muy exaltado, por causa de su pueblo
Israel.

\hypertarget{los-hijos-de-david-nacidos-en-jerusaluxe9n}{%
\subsection{Los hijos de David nacidos en
Jerusalén}\label{los-hijos-de-david-nacidos-en-jerusaluxe9n}}

\bibleverse{3} David tomó más esposas en Jerusalén, y fue padre de más
hijos e hijas. \bibleverse{4} Estos son los nombres de los hijos que
tuvo en Jerusalén Shammua, Shobab, Natán, Salomón, \bibleverse{5} Ibhar,
Elishua, Elpelet, \footnote{\textbf{14:5} 2Sam 5,17-25} \bibleverse{6}
Nogah, Nepheg, Japhia, \bibleverse{7} Elishama, Beeliada y Eliphelet.

\hypertarget{dos-batallas-victoriosas-entre-david-y-los-filisteos}{%
\subsection{Dos batallas victoriosas entre David y los
filisteos}\label{dos-batallas-victoriosas-entre-david-y-los-filisteos}}

\bibleverse{8} Cuando los filisteos oyeron que David había sido ungido
rey sobre todo Israel, todos los filisteos subieron a buscar a David;
pero David lo oyó y salió contra ellos. \bibleverse{9} Los filisteos
habían llegado y hecho una incursión en el valle de Refaim.
\bibleverse{10} David consultó a Dios, diciendo: ``¿Subiré contra los
filisteos? ¿Los entregarás en mi mano?'' Yahvé le dijo: ``Sube, porque
los entregaré en tu mano''.

\bibleverse{11} Así que subieron a Baal Perazim, y David los derrotó
allí. David dijo: ``Dios ha roto a mis enemigos por mi mano, como las
aguas que brotan. Por eso llamaron a ese lugar Baal Perazim. \footnote{\textbf{14:11}
  ``Baal Perazim'' significa ``El Señor que irrumpe''.} \bibleverse{12}
Dejaron allí sus dioses, y David dio una orden, y fueron quemados con
fuego. \footnote{\textbf{14:12} Deut 7,5; Deut 7,25}

\bibleverse{13} Los filisteos hicieron otra incursión en el valle.
\bibleverse{14} David volvió a consultar a Dios, y éste le dijo: ``No
subirás tras ellos. Aléjate de ellos, y acércate a ellos frente a las
moreras. \bibleverse{15} Cuando oigas el ruido de la marcha en las copas
de las moreras, sal a combatir, porque Dios ha salido delante de ti para
atacar al ejército de los filisteos.''

\bibleverse{16} David hizo lo que Dios le ordenó, y atacaron al ejército
de los filisteos desde Gabaón hasta Gezer. \bibleverse{17} La fama de
David se extendió por todas las tierras, y el Señor hizo que todas las
naciones lo temieran.

\hypertarget{preparativos-para-el-traslado-del-arca-sagrada-designaciuxf3n-e-instrucciuxf3n-de-los-levitas-a-cargo}{%
\subsection{Preparativos para el traslado del arca sagrada; Designación
e instrucción de los levitas a
cargo}\label{preparativos-para-el-traslado-del-arca-sagrada-designaciuxf3n-e-instrucciuxf3n-de-los-levitas-a-cargo}}

\hypertarget{section-14}{%
\section{15}\label{section-14}}

\bibleverse{1} David se hizo casas en la ciudad de David, y preparó un
lugar para el arca de Dios, y levantó una tienda para ella.
\bibleverse{2} Entonces David dijo: ``Nadie debe llevar el arca de Dios
sino los levitas. Porque Yahvé los ha escogido para que lleven el arca
de Dios y le sirvan para siempre''.

\bibleverse{3} David reunió a todo Israel en Jerusalén, para llevar el
arca de Yahvé a su lugar, que él había preparado para ella.
\bibleverse{4} David reunió a los hijos de Aarón y a los levitas
\bibleverse{5} de los hijos de Coat, Uriel el principal, y sus hermanos,
ciento veinte; \bibleverse{6} de los hijos de Merari, Asaías el
principal, y sus hermanos, doscientos veinte; \bibleverse{7} de los
hijos de Gersón, Joel el principal, y sus hermanos, ciento treinta;
\bibleverse{8} de los hijos de Elizafán, Semaías el principal, y sus
hermanos doscientos; \bibleverse{9} de los hijos de Hebrón, Eliel el
principal, y sus hermanos ochenta; \bibleverse{10} de los hijos de
Uziel, Aminadab el principal, y sus hermanos ciento doce.

\bibleverse{11} David llamó a los sacerdotes Sadoc y Abiatar, y a los
levitas: a Uriel, Asaías, Joel, Semaías, Eliel y Aminadab, \footnote{\textbf{15:11}
  2Sam 15,29} \bibleverse{12} y les dijo: ``Ustedes son los jefes de
familia de los levitas. Santificaos, vosotros y vuestros hermanos, para
que podáis llevar el arca de Yavé, el Dios de Israel, hasta el lugar que
le he preparado. \bibleverse{13} Porque como no la llevasteis al
principio, Yahvé, nuestro Dios, estalló en cólera contra nosotros,
porque no lo buscamos según la ordenanza.'' \footnote{\textbf{15:13}
  1Cró 13,9-10}

\bibleverse{14} Los sacerdotes y los levitas se santificaron para subir
el arca de Yavé, el Dios de Israel. \bibleverse{15} Los hijos de los
levitas llevaban el arca de Dios sobre sus hombros con sus varas, como
lo había ordenado Moisés según la palabra de Yavé. \footnote{\textbf{15:15}
  Éxod 25,14; Núm 4,15}

\hypertarget{orden-de-los-cantantes-muxfasicos-y-porteadores-levuxedticos}{%
\subsection{Orden de los cantantes, músicos y porteadores
levíticos}\label{orden-de-los-cantantes-muxfasicos-y-porteadores-levuxedticos}}

\bibleverse{16} David habló a los jefes de los levitas para que
designaran a sus hermanos como cantantes con instrumentos de música,
instrumentos de cuerda, arpas y címbalos, que tocaran en voz alta y
alzaran la voz con alegría. \bibleverse{17} Los levitas nombraron a
Hemán hijo de Joel, y de sus hermanos a Asaf hijo de Berequías, y de los
hijos de Merari a sus hermanos, a Etán hijo de Cushaías; \bibleverse{18}
y con ellos a sus hermanos de segundo grado: Zacarías, Ben, Jaaziel,
Semiramot, Jehiel, Unni, Eliab, Benaía, Maasías, Matatías, Elifelehu,
Micneías, Obed-Edom y Jeiel, los porteros. \bibleverse{19} A los
cantores, Hemán, Asaf y Etán, se les dieron címbalos de bronce para que
los hicieran sonar en voz alta; \footnote{\textbf{15:19} 1Cró 6,33; 1Cró
  6,39; 1Cró 6,44; 1Cró 25,1} \bibleverse{20} y a Zacarías, Aziel,
Semiramot, Jehiel, Unni, Eliab, Maasías y Benaía, con instrumentos de
cuerda afinados con Alamot; \bibleverse{21} y a Matatías, Elifelehu,
Micneías, Obed-Edom, Jeiel y Azazías, con arpas afinadas con lira de
ocho cuerdas, para que los dirigieran. \bibleverse{22} Quenanías, jefe
de los levitas, estaba a cargo del canto. Él enseñaba a los cantantes,
porque era hábil. \bibleverse{23} Berequías y Elcana eran porteros del
arca. \bibleverse{24} Sebanías, Josafat, Natanel, Amasai, Zacarías,
Benaía y Eliezer, los sacerdotes, tocaban las trompetas delante del arca
de Dios; y Obed-Edom y Jehías eran porteros del arca.

\hypertarget{la-participaciuxf3n-personal-de-david-en-la-transferencia-la-fiesta-del-sacrificio-y-la-acciuxf3n-de-gracias}{%
\subsection{La participación personal de David en la transferencia; la
fiesta del sacrificio y la acción de
gracias}\label{la-participaciuxf3n-personal-de-david-en-la-transferencia-la-fiesta-del-sacrificio-y-la-acciuxf3n-de-gracias}}

\bibleverse{25} Entonces David, los ancianos de Israel y los capitanes
de millares fueron a sacar con alegría el arca de la alianza de Yavé de
la casa de Obed-Edom. \footnote{\textbf{15:25} 2Sam 6,12-16}
\bibleverse{26} Cuando Dios ayudó a los levitas que llevaban el arca de
la alianza de Yavé, éstos sacrificaron siete toros y siete carneros.
\bibleverse{27} David estaba vestido con una túnica de lino fino, al
igual que todos los levitas que llevaban el arca, los cantores y
Quenanías, el director del coro, con los cantores; y David llevaba un
efod de lino. \bibleverse{28} Así subió todo Israel el arca de la
alianza de Yahvé con gritos, con sonido de corneta, de trompetas y de
címbalos, tocando en voz alta con instrumentos de cuerda y arpas.
\bibleverse{29} Cuando el arca de la alianza de Yavé llegó a la ciudad
de David, Mical, hija de Saúl, se asomó a la ventana y vio al rey David
bailando y tocando, y lo despreció en su corazón.

\hypertarget{section-15}{%
\section{16}\label{section-15}}

\bibleverse{1} Trajeron el arca de Dios y la pusieron en medio de la
tienda que David había levantado para ella; y ofrecieron holocaustos y
ofrendas de paz ante Dios. \footnote{\textbf{16:1} 2Sam 6,17-19}
\bibleverse{2} Cuando David terminó de ofrecer el holocausto y las
ofrendas de paz, bendijo al pueblo en nombre de Yavé. \bibleverse{3} Dio
a todos los israelíes, hombres y mujeres, a cada uno una hogaza de pan,
una porción de carne y una torta de pasas.

\hypertarget{orden-del-servicio-de-canto-y-muxfasica-en-el-arca}{%
\subsection{Orden del servicio de canto y música en el
Arca}\label{orden-del-servicio-de-canto-y-muxfasica-en-el-arca}}

\bibleverse{4} Nombró a algunos de los levitas para que sirvieran ante
el arca de Yahvé, y para que conmemoraran, dieran gracias y alabaran a
Yahvé, el Dios de Israel: \bibleverse{5} Asaf, el principal, y tras él
Zacarías, luego Jeiel, Semiramot, Jehiel, Mattithiah, Eliab, Benaiah,
Obed-Edom y Jeiel, con instrumentos de cuerda y con arpas; y Asaf con
címbalos, tocando en voz alta; \bibleverse{6} con Benaiah y Jahaziel,
los sacerdotes, con trompetas continuamente, ante el arca de la alianza
de Dios.

\hypertarget{canto-de-agradecimiento-y-alabanza-de-david}{%
\subsection{Canto de agradecimiento y alabanza de
David}\label{canto-de-agradecimiento-y-alabanza-de-david}}

\bibleverse{7} Aquel día, David ordenó por primera vez dar gracias a
Yahvé de la mano de Asaf y sus hermanos. \bibleverse{8} Dad gracias a
Yahvé. Invoca su nombre. Haz que lo que ha hecho se conozca entre los
pueblos. \footnote{\textbf{16:8} Sal 105,1-15} \bibleverse{9} Cántale.
Cántale alabanzas. Cuenta todas sus maravillosas obras. \bibleverse{10}
Gloria a su santo nombre. Que se alegre el corazón de los que buscan a
Yahvé. \bibleverse{11} Busca a Yahvé y su fuerza. Busca su rostro para
siempre. \bibleverse{12} Acuérdate de las maravillas que ha hecho, sus
maravillas, y los juicios de su boca, \bibleverse{13} tú,
descendiente\footnote{\textbf{16:13} o, semilla} de Israel, su siervo,
vosotros, hijos de Jacob, sus elegidos. \bibleverse{14} Él es Yahvé,
nuestro Dios. Sus juicios están en toda la tierra. \bibleverse{15}
Recuerda su pacto para siempre, la palabra que ordenó a mil
generaciones, \bibleverse{16} el pacto que hizo con Abraham, su
juramento a Isaac. \bibleverse{17} Se lo confirmó a Jacob por un
estatuto, y a Israel por un pacto eterno, \bibleverse{18} diciendo: ``Te
daré la tierra de Canaán, El lote de tu herencia''. \bibleverse{19}
cuando no erais más que unos pocos hombres, sí, muy pocos, y extranjeros
en ella. \bibleverse{20} Iban de nación en nación, de un reino a otro
pueblo. \bibleverse{21} No permitió que nadie les hiciera mal. Sí,
reprendió a los reyes por su bien, \footnote{\textbf{16:21} Gén 12,17;
  Gén 20,3; Gén 26,9} \bibleverse{22} ``¡No toquen a mis ungidos! No
hagas daño a mis profetas''. \bibleverse{23} ¡Cantad a Yahvé, toda la
tierra! Mostrar su salvación de día en día. \footnote{\textbf{16:23} Sal
  96,1} \bibleverse{24} Anuncia su gloria entre las naciones, y sus
obras maravillosas entre todos los pueblos. \bibleverse{25} Porque
grande es Yahvé, y muy digno de alabanza. También debe ser temido por
encima de todos los dioses. \bibleverse{26} Porque todos los dioses de
los pueblos son ídolos, pero Yahvé hizo los cielos. \bibleverse{27} El
honor y la majestad están ante él. La fuerza y la alegría están en su
lugar. \bibleverse{28} Atribuid a Yahvé, familias de los pueblos,
¡atribuir a Yahvé la gloria y la fuerza! \footnote{\textbf{16:28} Sal
  29,1-2} \bibleverse{29} Atribuid a Yahvé la gloria debida a su nombre.
Trae una ofrenda y preséntate ante él. Adoren a Yahvé en forma sagrada.
\bibleverse{30} Temblad ante él, toda la tierra. El mundo también está
establecido que no se puede mover. \bibleverse{31} Que se alegren los
cielos, ¡y que la tierra se alegre! Que digan entre las naciones:
``¡Yahvé reina!'' \bibleverse{32} ¡Que ruja el mar y su plenitud! ¡Que
el campo se regocije, y todo lo que hay en él! \bibleverse{33} Entonces
los árboles del bosque cantarán de alegría ante Yahvé, porque viene a
juzgar la tierra. \bibleverse{34} Dad gracias a Yahvé, porque es bueno,
porque su bondad es eterna. \footnote{\textbf{16:34} Sal 106,47-48}
\bibleverse{35} Di: ``¡Sálvanos, Dios de nuestra salvación! Reúnenos y
líbranos de las naciones, para dar gracias a tu santo nombre, para
triunfar en tu alabanza''. \bibleverse{36} Bendito sea Yahvé, el Dios de
Israel, desde la eternidad hasta la eternidad. Todo el pueblo dijo:
``Amén'', y alabó a Yahvé. \footnote{\textbf{16:36} Sal 41,13}

\hypertarget{establecimiento-del-servicio-de-portero-sacerdote-y-cantante-en-el-arca-fin-del-festival}{%
\subsection{Establecimiento del servicio de portero, sacerdote y
cantante en el arca; Fin del
festival}\label{establecimiento-del-servicio-de-portero-sacerdote-y-cantante-en-el-arca-fin-del-festival}}

\bibleverse{37} Dejó allí a Asaf y a sus hermanos, delante del arca de
la alianza de Yahvé, para que sirvieran continuamente delante del arca,
según el trabajo de cada día; \bibleverse{38} y a Obed-Edom con sus
sesenta y ocho parientes; a Obed-Edom también, hijo de Jedutún, y a Hosa
para que fueran porteros; \bibleverse{39} y el sacerdote Sadoc y sus
hermanos sacerdotes, ante el tabernáculo de Yahvé en el lugar alto que
estaba en Gabaón, \footnote{\textbf{16:39} 1Cró 21,29} \bibleverse{40}
para ofrecer holocaustos a Yahvé en el altar de los holocaustos
continuamente por la mañana y por la tarde, conforme a todo lo que está
escrito en la ley de Yahvé, que él ordenó a Israel; \footnote{\textbf{16:40}
  Éxod 29,38-39} \bibleverse{41} y con ellos Hemán y Jedutún y los demás
elegidos, mencionados por su nombre, para dar gracias a Yahvé, porque es
eterna su misericordia; \bibleverse{42} y con ellos Hemán y Jedutún con
trompetas y címbalos para los que debían tocar en voz alta, y con
instrumentos para los cánticos de Dios, y los hijos de Jedutún para
estar en la puerta. \bibleverse{43} Todo el pueblo se fue, cada uno a su
casa; y David volvió a bendecir su casa.

\hypertarget{natuxe1n-aprueba-el-plan-de-david-para-construir-el-templo}{%
\subsection{Natán aprueba el plan de David para construir el
templo}\label{natuxe1n-aprueba-el-plan-de-david-para-construir-el-templo}}

\hypertarget{section-16}{%
\section{17}\label{section-16}}

\bibleverse{1} Cuando David vivía en su casa, le dijo al profeta Natán:
``Mira, yo vivo en una casa de cedro, pero el arca de la alianza de
Yahvé está en una tienda.''

\bibleverse{2} Natán dijo a David: ``Haz todo lo que está en tu corazón,
porque Dios está contigo''.

\hypertarget{dios-rechaza-el-plan-el-discurso-profuxe9tico-de-nathan-el-templo-seruxe1-construido-por-el-hijo-de-david}{%
\subsection{Dios rechaza el plan; El discurso profético de Nathan; el
templo será construido por el hijo de
David}\label{dios-rechaza-el-plan-el-discurso-profuxe9tico-de-nathan-el-templo-seruxe1-construido-por-el-hijo-de-david}}

\bibleverse{3} Aquella misma noche vino la palabra de Dios a Natán,
diciendo: \bibleverse{4} ``Ve y dile a David, mi siervo, que Yahvé dice:
``No me construirás una casa para habitarla; \bibleverse{5} porque no he
vivido en una casa desde el día en que hice surgir a Israel hasta hoy,
sino que he ido de tienda en tienda, y de tienda en tienda.
\bibleverse{6} En todos los lugares en que he andado con todo Israel,
¿hablé una palabra con alguno de los jueces de Israel, a quienes mandé
que fueran pastores de mi pueblo, diciendo: ``¿Por qué no me habéis
construido una casa de cedro?''\,'

\bibleverse{7} ``Ahora, pues, le dirás a mi siervo David: ``El Señor de
los Ejércitos dice: ``Te tomé del corral de las ovejas, de seguir a las
ovejas, para ser príncipe de mi pueblo Israel. \bibleverse{8} He estado
contigo dondequiera que has ido, y he cortado a todos tus enemigos de
delante de ti. Te haré un nombre como el de los grandes que hay en la
tierra. \bibleverse{9} Yo designaré un lugar para mi pueblo Israel, y lo
plantaré, para que habite en su propio lugar y no se mueva más. Los
hijos de la maldad no los asolarán más, como al principio,
\bibleverse{10} y desde el día en que ordené que hubiera jueces sobre mi
pueblo Israel. Someteré a todos sus enemigos. Además, te digo que el
Señor te construirá una casa. \bibleverse{11} Sucederá que cuando se
cumplan tus días en que debes ir a estar con tus padres, yo estableceré
a tu descendiente después de ti, que será de tus hijos, y estableceré su
reino. \bibleverse{12} Él me construirá una casa, y yo estableceré su
trono para siempre. \footnote{\textbf{17:12} 1Cró 22,10; 1Cró 28,6}

\hypertarget{la-gran-proclamaciuxf3n-de-salvaciuxf3n-de-dios-a-david-con-respecto-a-la-duraciuxf3n-eterna-de-su-casa}{%
\subsection{La gran proclamación de salvación de Dios a David con
respecto a la duración eterna de su
casa}\label{la-gran-proclamaciuxf3n-de-salvaciuxf3n-de-dios-a-david-con-respecto-a-la-duraciuxf3n-eterna-de-su-casa}}

\bibleverse{13} Yo seré su padre, y él será mi hijo. No le quitaré mi
bondad, como se la quité al que fue antes de ti; \bibleverse{14} sino
que lo estableceré en mi casa y en mi reino para siempre. Su trono
quedará establecido para siempre''\,''.

\hypertarget{acciuxf3n-de-gracias-y-suxfaplica-de-david}{%
\subsection{Acción de gracias y súplica de
David}\label{acciuxf3n-de-gracias-y-suxfaplica-de-david}}

\bibleverse{15} Según todas estas palabras y según toda esta visión, así
habló Natán a David.

\bibleverse{16} El rey David entró y se sentó delante de Yahvé, y dijo:
``¿Quién soy yo, Yahvé Dios, y cuál es mi casa, para que me hayas traído
hasta aquí? \footnote{\textbf{17:16} Gén 32,10} \bibleverse{17} Esto era
una pequeñez a tus ojos, oh Dios, pero has hablado de la casa de tu
siervo por mucho tiempo, y me has respetado según la norma de un hombre
de alto rango, Yahvé Dios. \bibleverse{18} ¿Qué más puede decirte David
acerca del honor que se le hace a tu siervo? Porque tú conoces a tu
siervo. \bibleverse{19} Yahvé, por causa de tu siervo, y según tu propio
corazón, has hecho toda esta grandeza, para dar a conocer todas estas
grandes cosas. \bibleverse{20} Yahvé, no hay nadie como tú, ni hay otro
Dios fuera de ti, según todo lo que hemos oído con nuestros oídos.
\bibleverse{21} ¿Qué nación hay en la tierra que se parezca a tu pueblo
Israel, al que Dios fue a redimir para sí como pueblo, para hacerte un
nombre con cosas grandes y asombrosas, al expulsar a las naciones de
delante de tu pueblo que redimiste de Egipto? \bibleverse{22} Porque
hiciste de tu pueblo Israel tu propio pueblo para siempre; y tú, Yahvé,
te convertiste en su Dios. \bibleverse{23} Ahora bien, Yahvé, que la
palabra que has pronunciado respecto a tu siervo y a su casa quede
establecida para siempre, y haz lo que has dicho. \bibleverse{24} Que tu
nombre sea establecido y engrandecido para siempre, diciendo: ``El Señor
de los Ejércitos es el Dios de Israel, un Dios para Israel. La casa de
David, tu siervo, está establecida ante ti.' \bibleverse{25} Porque tú,
Dios mío, has revelado a tu siervo que le construirás una casa. Por eso
tu siervo ha encontrado valor para orar ante ti. \bibleverse{26} Ahora
bien, Yahvé, tú eres Dios y has prometido este bien a tu siervo.
\bibleverse{27} Ahora te ha parecido bien bendecir la casa de tu siervo,
para que permanezca para siempre ante ti; porque tú, Yahvé, la has
bendecido, y es bendita para siempre.''

\hypertarget{las-victorias-de-david-sobre-los-filisteos-moabitas-sirios-y-edomitas}{%
\subsection{Las victorias de David sobre los filisteos, moabitas, sirios
y
edomitas}\label{las-victorias-de-david-sobre-los-filisteos-moabitas-sirios-y-edomitas}}

\hypertarget{section-17}{%
\section{18}\label{section-17}}

\bibleverse{1} Después de esto, David derrotó a los filisteos y los
sometió, y tomó Gat y sus ciudades de manos de los filisteos.
\bibleverse{2} Derrotó a Moab, y los moabitas se convirtieron en siervos
de David y le trajeron tributo.

\hypertarget{las-victorias-de-david-sobre-los-sirios-el-uso-del-botuxedn-felicitaciones-del-rey-tou}{%
\subsection{Las victorias de David sobre los sirios; el uso del botín;
Felicitaciones del rey
Tou}\label{las-victorias-de-david-sobre-los-sirios-el-uso-del-botuxedn-felicitaciones-del-rey-tou}}

\bibleverse{3} David derrotó a Hadadézer, rey de Soba, hacia Hamat,
cuando iba a establecer su dominio junto al río Éufrates. \bibleverse{4}
David le arrebató mil carros, siete mil jinetes y veinte mil hombres de
a pie; y a todos los caballos de los carros David les quitó la cuerda,
pero les reservó lo suficiente para cien carros. \bibleverse{5} Cuando
los sirios de Damasco vinieron a ayudar a Hadadézer, rey de Soba, David
hirió a veintidós mil hombres de los sirios. \bibleverse{6} Luego David
puso guarniciones en Siria de Damasco, y los sirios se convirtieron en
servidores de David y le trajeron tributo. El Señor le dio la victoria a
David dondequiera que fuera. \bibleverse{7} David tomó los escudos de
oro que tenían los siervos de Hadadézer y los llevó a Jerusalén.
\bibleverse{8} De Tibhat y de Cun, ciudades de Hadadzer, David tomó
mucho bronce, con el cual Salomón hizo el mar de bronce, las columnas y
los utensilios de bronce. \footnote{\textbf{18:8} 1Re 7,23; 1Re 7,15}

\bibleverse{9} Cuando Tou, rey de Hamat, se enteró de que David había
derrotado a todo el ejército de Hadadzer, rey de Soba, \bibleverse{10}
envió a su hijo Hadoram a saludar al rey David y a bendecirlo, porque
había luchado contra Hadadzer y lo había derrotado (pues Hadadzer tenía
guerras con Tou); y llevaba consigo toda clase de objetos de oro, plata
y bronce. \bibleverse{11} El rey David también los dedicó a Yavé, junto
con la plata y el oro que se llevó de todas las naciones: de Edom, de
Moab, de los hijos de Amón, de los filisteos y de Amalec.

\hypertarget{derrota-y-subyugaciuxf3n-de-los-edomitas}{%
\subsection{Derrota y subyugación de los
edomitas}\label{derrota-y-subyugaciuxf3n-de-los-edomitas}}

\bibleverse{12} Además, Abisai, hijo de Sarvia, hirió a dieciocho mil
edomitas en el Valle de la Sal. \bibleverse{13} Puso guarniciones en
Edom, y todos los edomitas se convirtieron en servidores de David. El
Señor le dio la victoria a David dondequiera que fuera.

\hypertarget{los-altos-funcionarios-de-david}{%
\subsection{Los altos funcionarios de
David}\label{los-altos-funcionarios-de-david}}

\bibleverse{14} David reinó sobre todo Israel, y ejecutó justicia y
rectitud para todo su pueblo. \bibleverse{15} Joab, hijo de Sarvia,
estaba al frente del ejército; Josafat, hijo de Ahilud, era secretario;
\bibleverse{16} Sadoc, hijo de Ajitub, y Abimelec, hijo de Abiatar, eran
sacerdotes; Shavsha era escriba; \footnote{\textbf{18:16} 1Cró 24,6}
\bibleverse{17} y Benaía, hijo de Joiada, estaba al frente de los
queretanos y de los peletanos; y los hijos de David eran funcionarios
principales al servicio del rey.

\hypertarget{el-vergonzoso-crimen-de-los-amonitas-contra-el-mensajero-de-david}{%
\subsection{El vergonzoso crimen de los amonitas contra el mensajero de
David}\label{el-vergonzoso-crimen-de-los-amonitas-contra-el-mensajero-de-david}}

\hypertarget{section-18}{%
\section{19}\label{section-18}}

\bibleverse{1} Después de esto, murió Nahas, rey de los hijos de Amón, y
su hijo reinó en su lugar. \bibleverse{2} David dijo: ``Mostraré bondad
con Hanún, hijo de Nahas, porque su padre mostró bondad conmigo''.
Entonces David envió mensajeros para consolarle respecto a su padre. Los
servidores de David fueron a la tierra de los hijos de Amón a Hanún para
consolarlo. \bibleverse{3} Pero los príncipes de los hijos de Amón
dijeron a Hanún: ``¿Acaso crees que David honra a tu padre, pues te ha
enviado consoladores? ¿No han venido a ti sus siervos para buscar,
derrocar y espiar la tierra?'' \footnote{\textbf{19:3} 1Sam 3,18}
\bibleverse{4} Entonces Hanún tomó a los siervos de David, los afeitó y
les cortó los vestidos por la mitad a la altura de las nalgas, y los
despidió. \bibleverse{5} Luego, algunas personas fueron a contarle a
David cómo habían sido tratados los hombres. Él envió a recibirlos, pues
los hombres estaban muy humillados. El rey les dijo: ``Quédense en
Jericó hasta que les crezca la barba, y luego vuelvan''.

\hypertarget{comienzo-de-la-guerra-primeros-trabajos-ganados}{%
\subsection{Comienzo de la guerra; primeros trabajos
ganados}\label{comienzo-de-la-guerra-primeros-trabajos-ganados}}

\bibleverse{6} Cuando los hijos de Amón vieron que se habían hecho
odiosos para David, Hanún y los hijos de Amón enviaron mil
talentos\footnote{\textbf{19:6} Un talento equivale a unos 30 kilogramos
  o 66 libras, por lo que 1000 talentos son unas 30 toneladas métricas}
de plata para contratar carros y jinetes de Mesopotamia, de Aram-maacá y
de Zoba. \bibleverse{7} Y contrataron para sí treinta y dos mil carros,
y al rey de Maaca con su gente, que vino y acampó cerca de Medeba. Los
hijos de Amón se reunieron desde sus ciudades y vinieron a la batalla.
\bibleverse{8} Cuando David se enteró, envió a Joab con todo el ejército
de valientes. \bibleverse{9} Los hijos de Amón salieron y prepararon la
batalla a la puerta de la ciudad, y los reyes que habían venido estaban
solos en el campo.

\bibleverse{10} Cuando Joab vio que la batalla estaba preparada contra
él por delante y por detrás, escogió a algunos de todos los hombres
selectos de Israel y los puso en orden de batalla contra los sirios.
\bibleverse{11} El resto del pueblo lo puso en manos de Abisai, su
hermano, y se puso en orden de batalla contra los amonitas.
\bibleverse{12} Él dijo: ``Si los sirios son demasiado fuertes para mí,
tú me ayudarás; pero si los hijos de Amón son demasiado fuertes para ti,
yo te ayudaré. \bibleverse{13} Sé valiente, y seamos fuertes por nuestro
pueblo y por las ciudades de nuestro Dios. Que Yahvé haga lo que le
parece bien''.

\bibleverse{14} Entonces Joab y el pueblo que estaba con él se acercaron
al frente de los sirios a la batalla, y huyeron ante él. \bibleverse{15}
Cuando los hijos de Amón vieron que los sirios habían huido, huyeron
igualmente ante Abisai, su hermano, y entraron en la ciudad. Entonces
Joab llegó a Jerusalén.

\hypertarget{david-personalmente-en-el-campo-su-victoria-sobre-los-sirios-aliados-con-los-amonitas}{%
\subsection{David personalmente en el campo; su victoria sobre los
sirios aliados con los
amonitas}\label{david-personalmente-en-el-campo-su-victoria-sobre-los-sirios-aliados-con-los-amonitas}}

\bibleverse{16} Cuando los sirios vieron que habían sido derrotados por
Israel, enviaron mensajeros y convocaron a los sirios que estaban al
otro lado del río,\footnote{\textbf{19:16} o, el río Éufrates} con
Shophach, el capitán del ejército de Hadadezer, al frente.
\bibleverse{17} David recibió la noticia, así que reunió a todo Israel,
pasó el Jordán, llegó hasta ellos y preparó la batalla contra ellos.
Cuando David preparó la batalla contra los sirios, éstos lucharon con
él. \bibleverse{18} Los sirios huyeron ante Israel, y David mató de los
sirios a siete mil cuadrillas y cuarenta mil hombres de a pie, y también
mató a Shophach, el capitán del ejército. \bibleverse{19} Cuando los
servidores de Hadadézer vieron que habían sido derrotados por Israel,
hicieron la paz con David y le sirvieron. Los sirios no quisieron ayudar
más a los hijos de Amón.

\hypertarget{joab-conquista-rabuxe1-el-triunfo-de-david-y-el-castigo-de-los-amonitas}{%
\subsection{Joab conquista Rabá; El triunfo de David y el castigo de los
amonitas}\label{joab-conquista-rabuxe1-el-triunfo-de-david-y-el-castigo-de-los-amonitas}}

\hypertarget{section-19}{%
\section{20}\label{section-19}}

\bibleverse{1} A la vuelta del año, en la época en que salen los reyes,
Joab sacó el ejército y asoló el país de los hijos de Amón, y llegó a
sitiar Rabá. Pero David se quedó en Jerusalén. Joab atacó a Rabá y la
derrocó. \bibleverse{2} David se quitó la corona de su rey de la cabeza,
y encontró que pesaba un talento de oro,\footnote{\textbf{20:2} Un
  talento es de unos 30 kilogramos o 66 libras o 965 onzas troy, por lo
  que 3000 talentos son unas 90 toneladas métricas} y que había piedras
preciosas en ella. La puso en la cabeza de David, y sacó mucho botín de
la ciudad. \bibleverse{3} Sacó a la gente que estaba en ella y la hizo
cortar con sierras, con picos de hierro y con hachas. Así hizo David con
todas las ciudades de los hijos de Amón. Luego David y todo el pueblo
regresaron a Jerusalén.

\hypertarget{algunas-hazauxf1as-de-los-guerreros-de-david-en-las-guerras-filisteas}{%
\subsection{Algunas hazañas de los guerreros de David en las guerras
filisteas}\label{algunas-hazauxf1as-de-los-guerreros-de-david-en-las-guerras-filisteas}}

\bibleverse{4} Después de esto, surgió la guerra en Gezer con los
filisteos. Entonces Sibecai el husatita mató a Sippai, de los hijos del
gigante, y fueron sometidos. \footnote{\textbf{20:4} 1Cró 27,11}

\bibleverse{5} Nuevamente hubo guerra con los filisteos, y Elhanán, hijo
de Jair, mató a Lahmi, hermano de Goliat, el giteo, cuyo asta de lanza
era como una viga de tejedor. \bibleverse{6} Volvió a haber guerra en
Gat, donde había un hombre de gran estatura que tenía veinticuatro dedos
en las manos y en los pies, seis en cada mano y seis en cada pie.
\bibleverse{7} Cuando desafió a Israel, lo mató Jonatán, hijo de Simea,
hermano de David. \footnote{\textbf{20:7} 1Sam 17,10} \bibleverse{8}
Estos le nacieron al gigante en Gat; y cayeron por mano de David y por
mano de sus siervos.

\hypertarget{david-a-instigaciuxf3n-de-satanuxe1s-completa-el-censo-a-pesar-de-la-advertencia-de-joab-resultado-del-recuento}{%
\subsection{David, a instigación de Satanás, completa el censo a pesar
de la advertencia de Joab; Resultado del
recuento}\label{david-a-instigaciuxf3n-de-satanuxe1s-completa-el-censo-a-pesar-de-la-advertencia-de-joab-resultado-del-recuento}}

\hypertarget{section-20}{%
\section{21}\label{section-20}}

\bibleverse{1} Satanás se levantó contra Israel e incitó a David a hacer
un censo de Israel. \bibleverse{2} David dijo a Joab y a los jefes del
pueblo: ``Vayan a contar a Israel desde Beerseba hasta Dan, y tráiganme
la noticia para que yo sepa cuántos son.''

\bibleverse{3} Joab dijo: ``Que Yahvé haga que su pueblo sea cien veces
mayor que ellos. Pero, mi señor el rey, ¿no son todos ellos siervos de
mi señor? ¿Por qué mi señor exige esto? ¿Por qué será causa de culpa
para Israel?'' \footnote{\textbf{21:3} Éxod 30,12}

\bibleverse{4} Sin embargo, la palabra del rey prevaleció contra Joab.
Por eso Joab partió y recorrió todo Israel, y luego llegó a Jerusalén.
\bibleverse{5} Joab dio a David la suma del censo del pueblo. Todos los
de Israel eran un millón cien mil hombres que sacaban espada; y en Judá
había cuatrocientos setenta mil hombres que sacaban espada.
\bibleverse{6} Pero no contó a Leví y a Benjamín entre ellos, porque la
palabra del rey era abominable para Joab.

\hypertarget{el-arrepentimiento-de-david-intervenciuxf3n-del-profeta-gad-david-elige-una-muerte-popular-para-expiar-su-culpa}{%
\subsection{El arrepentimiento de David; Intervención del profeta Gad;
David elige una muerte popular para expiar su
culpa}\label{el-arrepentimiento-de-david-intervenciuxf3n-del-profeta-gad-david-elige-una-muerte-popular-para-expiar-su-culpa}}

\bibleverse{7} A Dios le disgustó este hecho, por lo que golpeó a
Israel. \footnote{\textbf{21:7} 1Cró 27,24} \bibleverse{8} David dijo a
Dios: ``He pecado mucho, pues he hecho esto. Pero ahora quita, te ruego,
la iniquidad de tu siervo, porque he hecho una gran locura''.

\bibleverse{9} Yahvé habló a Gad, el vidente de David, diciendo:
\bibleverse{10} ``Ve y habla a David, diciendo: ``Yahvé dice: ``Te
ofrezco tres cosas. Escoge una de ellas, para que te la haga''\,''.

\bibleverse{11} Gad se acercó a David y le dijo: ``Yahvé dice: `Elige:
\bibleverse{12} o tres años de hambre; o tres meses para ser consumido
ante tus enemigos, mientras la espada de tus enemigos te alcanza; o bien
tres días de la espada de Yahvé, con pestilencia en la tierra, y el
ángel de Yahvé destruyendo por todos los límites de Israel. Ahora, pues,
considera qué respuesta daré al que me envió''.

\bibleverse{13} David dijo a Gad: ``Estoy en apuros. Te ruego que me
dejes caer en la mano de Yahvé, porque sus misericordias son muy
grandes. No me dejes caer en la mano del hombre''.

\hypertarget{el-juicio-divino-la-penitencia-y-la-suxfaplica-de-david}{%
\subsection{El juicio divino; La penitencia y la súplica de
David}\label{el-juicio-divino-la-penitencia-y-la-suxfaplica-de-david}}

\bibleverse{14} Entonces Yahvé envió una peste sobre Israel, y cayeron
setenta mil hombres de Israel. \bibleverse{15} Dios envió un ángel a
Jerusalén para destruirla. Cuando estaba a punto de destruirla, Yahvé lo
vio, y cedió ante el desastre, y le dijo al ángel destructor: ``Es
suficiente. Ahora retira tu mano''. El ángel de Yavé estaba junto a la
era de Ornán el jebuseo. \bibleverse{16} David alzó los ojos y vio al
ángel de Yavé de pie entre la tierra y el cielo, con una espada
desenvainada en la mano extendida sobre Jerusalén. Entonces David y los
ancianos, vestidos de saco, se postraron sobre sus rostros.
\bibleverse{17} David dijo a Dios: ``¿No fui yo quien mandó contar al
pueblo? Soy yo quien ha pecado y ha hecho mucha maldad; pero estas
ovejas, ¿qué han hecho? Por favor, que tu mano, oh Yahvé, mi Dios, sea
contra mí y contra la casa de mi padre; pero no contra tu pueblo, para
que sea azotado.''

\hypertarget{david-adquiere-la-era-de-ornuxe1n-y-la-dedica-a-un-lugar-de-sacrificio-y-templo-fin-de-la-plaga}{%
\subsection{David adquiere la era de Ornán y la dedica a un lugar de
sacrificio y templo; Fin de la
plaga}\label{david-adquiere-la-era-de-ornuxe1n-y-la-dedica-a-un-lugar-de-sacrificio-y-templo-fin-de-la-plaga}}

\bibleverse{18} Entonces el ángel de Yavé ordenó a Gad que dijera a
David que subiera a levantar un altar a Yavé en la era de Ornán el
jebuseo. \bibleverse{19} David subió por la palabra de Gad, que habló en
nombre de Yavé.

\bibleverse{20} Ornán se volvió y vio al ángel, y sus cuatro hijos que
estaban con él se escondieron. Ornán estaba trillando trigo.
\bibleverse{21} Cuando David se acercó a Ornán, éste miró y vio a David,
salió de la era y se inclinó ante David con el rostro en tierra.

\bibleverse{22} Entonces David le dijo a Ornán: ``Véndeme el lugar de
esta era, para que construya en él un altar a Yavé. Me lo venderás por
el precio completo, para que la peste deje de afligir al pueblo''.
\footnote{\textbf{21:22} Núm 25,8}

\bibleverse{23} Ornán dijo a David: ``Tómalo para ti, y deja que mi
señor el rey haga lo que es bueno a sus ojos. He aquí que yo doy los
bueyes para los holocaustos, y los trillos para la leña, y el trigo para
la ofrenda. Lo doy todo''.

\bibleverse{24} El rey David le dijo a Ornán: ``No, pero ciertamente lo
compraré por el precio completo. Porque no tomaré lo que es tuyo para
Yahvé, ni ofreceré un holocausto que no me cueste nada''.

\bibleverse{25} Entonces David le dio a Ornán seiscientos siclos de oro
en peso para el lugar. \bibleverse{26} David edificó allí un altar a
Yavé, y ofreció holocaustos y ofrendas de paz, e invocó a Yavé; y éste
le respondió desde el cielo con fuego sobre el altar de los holocaustos.
\footnote{\textbf{21:26} 1Re 18,24}

\bibleverse{27} Entonces Yahvé ordenó al ángel, y éste volvió a enfundar
su espada.

\bibleverse{28} En aquel tiempo, cuando David vio que Yahvé le había
respondido en la era de Ornán el jebuseo, sacrificó allí.
\bibleverse{29} Porque el tabernáculo de Yavé, que Moisés hizo en el
desierto, y el altar de los holocaustos, estaban entonces en el lugar
alto de Gabaón. \footnote{\textbf{21:29} 1Cró 16,39} \bibleverse{30}
Pero David no pudo presentarse ante él para consultar a Dios, pues tenía
miedo a causa de la espada del ángel de Yavé. \footnote{\textbf{21:30}
  1Cró 21,16}

\hypertarget{section-21}{%
\section{22}\label{section-21}}

\bibleverse{1} Entonces David dijo: ``Esta es la casa de Yahvé Dios, y
este es el altar del holocausto para Israel''. \footnote{\textbf{22:1}
  2Cró 3,1}

\hypertarget{los-preparativos-de-david-para-la-construcciuxf3n-del-templo-colecciuxf3n-de-materiales-de-construcciuxf3n}{%
\subsection{Los preparativos de David para la construcción del templo;
Colección de materiales de
construcción}\label{los-preparativos-de-david-para-la-construcciuxf3n-del-templo-colecciuxf3n-de-materiales-de-construcciuxf3n}}

\bibleverse{2} David dio órdenes de reunir a los extranjeros que estaban
en la tierra de Israel, y puso a los albañiles a cortar piedras labradas
para construir la casa de Dios. \footnote{\textbf{22:2} 2Cró 2,17}
\bibleverse{3} David preparó hierro en abundancia para los clavos de las
puertas y para los enganches, y bronce en abundancia sin peso,
\bibleverse{4} y cedros en abundancia, porque los sidonios y la gente de
Tiro le trajeron cedros en abundancia a David. \bibleverse{5} David
dijo: ``Salomón, mi hijo, es joven y tierno, y la casa que se va a
construir para Yavé debe ser sumamente magnífica, de fama y de gloria en
todos los países. Por lo tanto, haré los preparativos para ello''. Así
que David se preparó abundantemente antes de su muerte. \footnote{\textbf{22:5}
  1Cró 29,1}

\hypertarget{instrucciones-de-david-a-su-hijo-salomuxf3n}{%
\subsection{Instrucciones de David a su hijo
Salomón}\label{instrucciones-de-david-a-su-hijo-salomuxf3n}}

\bibleverse{6} Luego llamó a su hijo Salomón y le ordenó que construyera
una casa para Yavé, el Dios de Israel. \bibleverse{7} David le dijo a su
hijo Salomón: ``En cuanto a mí, tenía en mi corazón construir una casa
al nombre de Yavé, mi Dios. \footnote{\textbf{22:7} 1Cró 17,1-14; 1Cró
  28,2-7} \bibleverse{8} Pero vino a mí la palabra de Yavé, diciendo:
`Has derramado mucha sangre y has hecho grandes guerras. No construirás
una casa a mi nombre, porque has derramado mucha sangre en la tierra a
mis ojos. \bibleverse{9} He aquí que te nacerá un hijo, que será un
hombre de paz. Le daré descanso de todos sus enemigos alrededor; porque
su nombre será Salomón, y daré paz y tranquilidad a Israel en sus días.
\bibleverse{10} Él edificará una casa a mi nombre, y él será mi hijo, y
yo seré su padre; y estableceré el trono de su reino sobre Israel para
siempre.' \bibleverse{11} Ahora, hijo mío, que el Señor te acompañe y te
haga prosperar, y que construyas la casa del Señor, tu Dios, como él ha
hablado de ti. \bibleverse{12} Que Yahvé te dé discreción y
entendimiento, y te ponga al frente de Israel, para que cumplas la ley
de Yahvé tu Dios. \bibleverse{13} Entonces prosperarás, si cumples con
los estatutos y las ordenanzas que el Señor le dio a Moisés acerca de
Israel. Sé fuerte y valiente. No tengas miedo ni te desanimes.
\footnote{\textbf{22:13} 1Re 2,2-3} \bibleverse{14} Ahora bien, he aquí
que en mi aflicción he preparado para la casa de Yavé cien mil talentos
de oro, un millón de talentos de plata, y bronce y hierro sin peso, pues
hay en abundancia. También he preparado madera y piedra; y tú puedes
añadirlas. \footnote{\textbf{22:14} 1Cró 29,2} \bibleverse{15} También
hay con vosotros obreros en abundancia --- cortadores y trabajadores de
la piedra y de la madera, y toda clase de hombres hábiles en toda clase
de trabajos; \bibleverse{16} del oro, de la plata, del bronce y del
hierro, no hay número. Levantaos y haced, y que el Señor esté con
vosotros''.

\hypertarget{la-amonestaciuxf3n-de-david-a-los-pruxedncipes-de-israel}{%
\subsection{La amonestación de David a los príncipes de
Israel}\label{la-amonestaciuxf3n-de-david-a-los-pruxedncipes-de-israel}}

\bibleverse{17} David también ordenó a todos los príncipes de Israel que
ayudaran a su hijo Salomón, diciendo: \bibleverse{18} ``¿No está Yahvé,
tu Dios, contigo? ¿No te ha dado descanso por todos lados? Porque él ha
entregado a los habitantes de la tierra en mi mano; y la tierra está
sometida ante el Señor y ante su pueblo. \footnote{\textbf{22:18} 1Cró
  22,9; 1Cró 23,25} \bibleverse{19} Ahora pon tu corazón y tu alma para
seguir a Yahvé, tu Dios. Levántate, pues, y construye el santuario de
Yahvé Dios, para llevar el arca de la alianza de Yahvé y los utensilios
sagrados de Dios a la casa que se va a construir para el nombre de
Yahvé.''

\hypertarget{contando-y-ejecutando-los-levitas}{%
\subsection{Contando y ejecutando los
levitas}\label{contando-y-ejecutando-los-levitas}}

\hypertarget{section-22}{%
\section{23}\label{section-22}}

\bibleverse{1} Ya David era viejo y lleno de días, y puso a Salomón, su
hijo, como rey de Israel. \footnote{\textbf{23:1} 1Re 1,28-40}
\bibleverse{2} Reunió a todos los príncipes de Israel, con los
sacerdotes y los levitas. \bibleverse{3} Los levitas fueron contados de
treinta años en adelante, y su número por sus encuestas, hombre por
hombre, era de treinta y ocho mil. \bibleverse{4} David dijo: ``De
ellos, veinticuatro mil estaban para supervisar la obra de la casa de
Yavé, seis mil eran oficiales y jueces, \bibleverse{5} cuatro mil eran
porteros, y cuatro mil alababan a Yavé con los instrumentos que yo hacía
para dar alabanza.''

\hypertarget{clasificaciuxf3n-de-los-levitas-seguxfan-gerson-kehath-y-merari}{%
\subsection{Clasificación de los levitas según Gerson, Kehath y
Merari}\label{clasificaciuxf3n-de-los-levitas-seguxfan-gerson-kehath-y-merari}}

\bibleverse{6} David los dividió según los hijos de Leví: Gersón, Coat y
Merari. \footnote{\textbf{23:6} 1Cró 6,16-17}

\bibleverse{7} De los gersonitas: Ladán y Simei. \bibleverse{8} Los
hijos de Ladán: Jehiel el principal, Zetham y Joel, tres. \footnote{\textbf{23:8}
  1Cró 26,21} \bibleverse{9} Los hijos de Simei: Selomot, Haziel y
Harán, tres. Estos fueron los jefes de familia de Ladán. \bibleverse{10}
Los hijos de Simei: Jahat, Zina, Jeús y Beriá. Estos cuatro fueron los
hijos de Simei. \bibleverse{11} Jahat fue el jefe, y Zina el segundo;
pero Jeús y Beriá no tuvieron muchos hijos, por lo que se convirtieron
en una casa paterna en un conteo.

\bibleverse{12} Los hijos de Coat: Amram, Izhar, Hebrón y Uziel, cuatro.
\footnote{\textbf{23:12} 1Cró 6,2-3} \bibleverse{13} Los hijos de Amram:
Aarón y Moisés; y Aarón fue separado para que santificara las cosas más
santas, él y sus hijos para siempre, para quemar incienso ante Yahvé,
para servirle y para bendecir en su nombre para siempre. \footnote{\textbf{23:13}
  1Cró 6,49; Heb 5,4; Deut 10,8} \bibleverse{14} En cuanto a Moisés, el
hombre de Dios, sus hijos fueron nombrados en la tribu de Leví.
\footnote{\textbf{23:14} Deut 33,1} \bibleverse{15} Los hijos de Moisés:
Gersón y Eliezer. \footnote{\textbf{23:15} Éxod 18,3-4} \bibleverse{16}
Los hijos de Gersón: Sebuel, el jefe. \footnote{\textbf{23:16} 1Cró
  26,24} \bibleverse{17} El hijo de Eliezer fue Rehabía, el jefe; y
Eliezer no tuvo más hijos, pero los hijos de Rehabía fueron muy
numerosos. \footnote{\textbf{23:17} 1Cró 24,21-30} \bibleverse{18} El
hijo de Izhar: Selomit, el jefe. \bibleverse{19} Los hijos de Hebrón:
Jeria el principal, Amarías el segundo, Jahaziel el tercero y Jekam el
cuarto. \bibleverse{20} Los hijos de Uziel: Miqueas, el primero, e
Isías, el segundo.

\bibleverse{21} Los hijos de Merari: Mahli y Mushi. Los hijos de Mahli:
Eleazar y Kish. \footnote{\textbf{23:21} 1Cró 6,19} \bibleverse{22}
Eleazar murió y no tuvo hijos, sino sólo hijas; y sus parientes, los
hijos de Cis, las tomaron como esposas. \bibleverse{23} Los hijos de
Mushi: Mahli, Eder y Jeremot, tres.

\hypertarget{instrucciones-oficiales-para-los-levitas}{%
\subsection{Instrucciones oficiales para los
levitas}\label{instrucciones-oficiales-para-los-levitas}}

\bibleverse{24} Estos fueron los hijos de Leví según las casas paternas,
los jefes de las casas paternas de los que fueron contados
individualmente, en el número de nombres por sus encuestas, que hicieron
el trabajo para el servicio de la casa de Yahvé, de veinte años en
adelante. \bibleverse{25} Porque David dijo: ``Yahvé, el Dios de Israel,
ha dado descanso a su pueblo; y él habita en Jerusalén para siempre.
\footnote{\textbf{23:25} Jl 3,21} \bibleverse{26} También los levitas ya
no necesitarán llevar el tabernáculo y todos sus utensilios para su
servicio.'' \bibleverse{27} Porque por las últimas palabras de David
fueron contados los hijos de Leví, de veinte años para arriba.
\bibleverse{28} Porque el deber de ellos era servir a los hijos de Aarón
para el servicio de la casa de Yavé: en los atrios, en las habitaciones
y en la purificación de todas las cosas santas, en la obra del servicio
de la casa de Dios; \bibleverse{29} también para el pan de la
proposición y para la harina fina para la ofrenda, ya sea de obleas sin
levadura, o de la que se cuece en la sartén, o de la que se remoja, y
para todas las medidas de cantidad y tamaño; \bibleverse{30} y que se
pusieran de pie cada mañana para dar gracias y alabar a Yahvé, y lo
mismo al atardecer; \footnote{\textbf{23:30} Sal 92,2} \bibleverse{31} y
que ofrecieran todos los holocaustos a Yahvé en los sábados, en las
lunas nuevas y en las fiestas señaladas, en número conforme a la
ordenanza relativa a ellas, continuamente ante Yahvé; \bibleverse{32} y
que cumplieran con el deber de la Tienda de Reunión, el deber del lugar
santo y el deber de los hijos de Aarón sus hermanos para el servicio de
la casa de Yahvé.

\hypertarget{el-dibujo-de-las-24-clases-sacerdotales}{%
\subsection{El dibujo de las 24 clases
sacerdotales}\label{el-dibujo-de-las-24-clases-sacerdotales}}

\hypertarget{section-23}{%
\section{24}\label{section-23}}

\bibleverse{1} Estas fueron las divisiones de los hijos de Aarón. Los
hijos de Aarón: Nadab, Abiú, Eleazar e Itamar. \footnote{\textbf{24:1}
  1Cró 23,6; 1Cró 6,3} \bibleverse{2} Pero Nadab y Abiú murieron antes
que su padre y no tuvieron hijos, por lo que Eleazar e Itamar sirvieron
como sacerdotes. \footnote{\textbf{24:2} Lev 10,1-2; Lev 10,12}
\bibleverse{3} David, con Sadoc de los hijos de Eleazar y Ahimelec de
los hijos de Itamar, los repartió según su ordenamiento en su servicio.
\footnote{\textbf{24:3} 2Cró 8,14} \bibleverse{4} Se encontraron más
jefes de los hijos de Eleazar que de los hijos de Itamar, y fueron
repartidos así: de los hijos de Eleazar había dieciséis, jefes de casas
paternas; y de los hijos de Itamar, según las casas paternas, ocho.
\bibleverse{5} Así fueron repartidos imparcialmente por sorteo; porque
había príncipes del santuario y príncipes de Dios, tanto de los hijos de
Eleazar como de los hijos de Itamar. \bibleverse{6} Semaías hijo de
Netanel, escriba, que era de los levitas, los escribió en presencia del
rey, de los príncipes, del sacerdote Sadoc, de Ajimelec hijo de Abiatar,
y de los jefes de las casas paternas de los sacerdotes y de los levitas;
una casa paterna fue tomada para Eleazar, y otra para Itamar.
\footnote{\textbf{24:6} 1Cró 18,16}

\bibleverse{7} La primera suerte correspondió a Joiarib, la segunda a
Jedaías, \bibleverse{8} la tercera a Harim, la cuarta a Seorim,
\bibleverse{9} la quinta a Malquías, la sexta a Mijamín, \bibleverse{10}
la séptima a Hakkoz, la octava a Abías, \footnote{\textbf{24:10} Luc 1,5}
\bibleverse{11} la novena a Jesúa, la décima a Secanías, \bibleverse{12}
la undécima a Eliasib, la duodécima a Jakim, \bibleverse{13} la
decimotercera a Huppah, el decimocuarto a Jeshebeab, \bibleverse{14} el
decimoquinto a Bilgah, el decimosexto a Immer, \bibleverse{15} el
decimoséptimo a Hezir, el decimoctavo a Happizzez, \bibleverse{16} el
decimonoveno a Pethahiah, el vigésimo a Jehezkel, \bibleverse{17} el
vigésimo primero a Jachin, el vigésimo segundo a Gamul, \bibleverse{18}
el vigésimo tercero a Delaiah, y el vigésimo cuarto a Maaziah.
\bibleverse{19} Esta era la ordenación de su servicio, para entrar en la
casa de Yavé según la ordenanza que les había dado su padre Aarón, como
le había ordenado Yavé, el Dios de Israel.

\hypertarget{las-clases-levitas-y-sus-luxedderes}{%
\subsection{Las clases levitas y sus
líderes}\label{las-clases-levitas-y-sus-luxedderes}}

\bibleverse{20} De los demás hijos de Leví: de los hijos de Amram,
Subael; de los hijos de Subael, Jehdeías. \bibleverse{21} De Rehabía: de
los hijos de Rehabía, Isías el principal. \footnote{\textbf{24:21} 1Cró
  23,17-23; 1Cró 26,25} \bibleverse{22} De los izharitas, Selomot; de
los hijos de Selomot, Jahat. \bibleverse{23} Los hijos de Hebrón: Jeria,
Amarías el segundo, Jahaziel el tercero y Jecamé el cuarto.
\bibleverse{24} Los hijos de Uziel Miqueas; de los hijos de Miqueas,
Samir. \bibleverse{25} El hermano de Miqueas: Isisías; de los hijos de
Isisías, Zacarías. \bibleverse{26} Los hijos de Merari: Mahli y Mushi.
El hijo de Jaazías: Beno. \bibleverse{27} Los hijos de Merari, por parte
de Jaazías: Beno, Shoham, Zaccur e Ibri. \bibleverse{28} De Mahli:
Eleazar, que no tuvo hijos. \bibleverse{29} De Cis, hijo de Cis:
Jerajmeel. \bibleverse{30} Los hijos de Mushi: Mahli, Eder y Jerimot.
Estos fueron los hijos de los levitas según las casas de sus padres.
\bibleverse{31} Estos también echaron suertes como sus hermanos los
hijos de Aarón en presencia del rey David, Sadoc, Ajimelec y los jefes
de las casas paternas de los sacerdotes y de los levitas, las casas
paternas del jefe como las de su hermano menor. \footnote{\textbf{24:31}
  1Cró 25,8}

\hypertarget{el-sorteo-de-las-24-divisiones-de-los-cantantes-y-muxfasicos-sagrados}{%
\subsection{El sorteo de las 24 divisiones de los cantantes y músicos
sagrados}\label{el-sorteo-de-las-24-divisiones-de-los-cantantes-y-muxfasicos-sagrados}}

\hypertarget{section-24}{%
\section{25}\label{section-24}}

\bibleverse{1} Además, David y los capitanes del ejército apartaron para
el servicio a algunos de los hijos de Asaf, de Hemán y de Jedutún, que
debían profetizar con arpas, con instrumentos de cuerda y con címbalos.
El número de los que hicieron la obra según su servicio fue: \footnote{\textbf{25:1}
  1Cró 15,19} \bibleverse{2} de los hijos de Asaf: Zaccur, José,
Netanías y Asarela. Los hijos de Asaf estaban bajo la mano de Asaf,
quien profetizaba por orden del rey. \bibleverse{3} De Jedutún, los
hijos de Jedutún: Gedalías, Zeri, Jesaías, Simei, Hasabías y Matatías,
seis, bajo la mano de su padre Jedutún, que profetizaban dando gracias y
alabando a Yahvé con el arpa. \bibleverse{4} De Hemán, los hijos de
Hemán: Buquías, Matanías, Uziel, Sebuel, Jerimot, Hananías, Hanani,
Eliathah, Giddalti, Romamti-Ezer, Josbekashah, Mallothi, Hothir y
Mahazioth. \bibleverse{5} Todos estos fueron los hijos de Hemán, el
vidente del rey, en las palabras de Dios, para levantar el cuerno. Dios
le dio a Hemán catorce hijos y tres hijas. \footnote{\textbf{25:5} 1Cró
  21,9; 2Cró 35,15} \bibleverse{6} Todos ellos estaban bajo las manos de
su padre para cantar en la casa de Yahvé, con címbalos, instrumentos de
cuerda y arpas, para el servicio de la casa de Dios: Asaf, Jedutún y
Hemán estaban bajo la orden del rey. \bibleverse{7} El número de ellos,
con sus hermanos instruidos en el canto a Yahvé, todos los que eran
hábiles, era de doscientos ochenta y ocho. \bibleverse{8} Echaron
suertes para sus cargos, todos por igual, tanto el pequeño como el
grande, tanto el maestro como el alumno. \footnote{\textbf{25:8} 1Cró
  24,31}

\bibleverse{9} La primera suerte le tocó a Asaf, a José; la segunda a
Gedalías, él y sus hermanos e hijos, doce; \bibleverse{10} la tercera a
Zacur, sus hijos y sus hermanos, doce; \bibleverse{11} la cuarta a Izri,
sus hijos y sus hermanos, doce; \bibleverse{12} la quinta a Netanías,
sus hijos y sus hermanos, doce; \bibleverse{13} la sexta a Bucías, sus
hijos y sus hermanos, doce; \bibleverse{14} la séptima a Jesharela, sus
hijos y sus hermanos, doce; \bibleverse{15} la octava a Jesaías, sus
hijos y sus hermanos, doce; \bibleverse{16} la novena a Matanías, sus
hijos y sus hermanos, doce; \bibleverse{17} la décima a Simei, sus hijos
y sus hermanos, doce; \bibleverse{18} la undécima a Azarel, sus hijos y
sus hermanos, doce; \bibleverse{19} la duodécima a Hasabías, sus hijos y
sus hermanos, doce; \bibleverse{20} la decimotercera a Subael, sus hijos
y sus hermanos, doce; \bibleverse{21} el decimocuarto, a Mattithiah, sus
hijos y sus hermanos, doce; \bibleverse{22} el decimoquinto, a Jeremoth,
sus hijos y sus hermanos, doce; \bibleverse{23} el decimosexto, a
Hananiah, sus hijos y sus hermanos, doce; \bibleverse{24} el
decimoséptimo, a Joshbekashah, sus hijos y sus hermanos, doce;
\bibleverse{25} el decimoctavo, a Hanani, sus hijos y sus hermanos,
doce; \bibleverse{26} el decimonoveno, a Mallothi, sus hijos y sus
hermanos, doce; \bibleverse{27} por el vigésimo a Eliathah, sus hijos y
sus hermanos, doce; \bibleverse{28} por el vigésimo primero a Hothir,
sus hijos y sus hermanos, doce; \bibleverse{29} por el vigésimo segundo
a Giddalti, sus hijos y sus hermanos, doce; \bibleverse{30} por el
vigésimo tercero a Mahazioth, sus hijos y sus hermanos, doce;
\bibleverse{31} por el vigésimo cuarto a Romamti-Ezer, sus hijos y sus
hermanos, doce.

\hypertarget{divisiones-de-los-porteros-levuxedticos}{%
\subsection{Divisiones de los porteros
levíticos}\label{divisiones-de-los-porteros-levuxedticos}}

\hypertarget{section-25}{%
\section{26}\label{section-25}}

\bibleverse{1} Por las divisiones de los porteros: de los corasitas,
Meselemías hijo de Coré, de los hijos de Asaf. \footnote{\textbf{26:1}
  2Cró 8,14; 2Cró 35,15} \bibleverse{2} Meshelemías tuvo hijos: Zacarías
el primogénito, Jediael el segundo, Zebadías el tercero, Jatniel el
cuarto, \bibleverse{3} Elam el quinto, Johanán el sexto y Eliehoenai el
séptimo. \bibleverse{4} Obed-Edom tuvo hijos: Semaías el primogénito,
Jozabad el segundo, Joah el tercero, Sacar el cuarto, Natanel el quinto,
\bibleverse{5} Ammiel el sexto, Isacar el séptimo y Peullethai el
octavo; porque Dios lo bendijo. \bibleverse{6} También le nacieron hijos
a Semaías, su hijo, que gobernaban la casa de su padre; porque eran
hombres valientes. \bibleverse{7} Los hijos de Semaías: Otni, Rafael,
Obed y Elzabad, cuyos parientes eran hombres valientes, Eliú y
Semachiah. \bibleverse{8} Todos estos fueron de los hijos de Obed-Edom
con sus hijos y sus hermanos, hombres capaces en fuerza para el
servicio: sesenta y dos de Obed-Edom. \bibleverse{9} Meselemías tenía
hijos y hermanos, dieciocho hombres valientes. \bibleverse{10} También
Hosa, de los hijos de Merari, tenía hijos: Simri, el principal (pues
aunque no era el primogénito, su padre lo hizo jefe), \bibleverse{11}
Hilcías, el segundo, Tebalías, el tercero, y Zacarías, el cuarto. Todos
los hijos y hermanos de Hosah eran trece.

\hypertarget{la-distribuciuxf3n-de-los-porteros-a-las-diferentes-localizaciones}{%
\subsection{La distribución de los porteros a las diferentes
localizaciones}\label{la-distribuciuxf3n-de-los-porteros-a-las-diferentes-localizaciones}}

\bibleverse{12} De éstos eran las divisiones de los porteros, de los
principales, que tenían cargos como sus hermanos, para servir en la casa
de Yahvé. \bibleverse{13} Echaron suertes, tanto los pequeños como los
grandes, según las casas de sus padres, para cada puerta. \footnote{\textbf{26:13}
  1Cró 25,8} \bibleverse{14} La suerte del este correspondió a Selemías.
Luego echaron suertes para Zacarías, su hijo, sabio consejero, y su
suerte salió hacia el norte. \bibleverse{15} A Obed-Edom al sur; y a sus
hijos el almacén. \bibleverse{16} A Suppim y a Hosa hacia el oeste,
junto a la puerta de Salecet, en la calzada que sube, vigilante frente a
vigilante. \bibleverse{17} Al este, seis levitas, al norte cuatro por
día, al sur cuatro por día, y para el depósito dos y dos.
\bibleverse{18} Para Parbar al oeste, cuatro en la calzada y dos en
Parbar. \bibleverse{19} Estas fueron las divisiones de los porteros: de
los hijos de los corasitas y de los hijos de Merari.

\hypertarget{los-tesoreros-levuxedticos-y-los-funcionarios-de-la-administraciuxf3n}{%
\subsection{Los tesoreros levíticos y los funcionarios de la
administración}\label{los-tesoreros-levuxedticos-y-los-funcionarios-de-la-administraciuxf3n}}

\bibleverse{20} De los levitas, Ahías estaba a cargo de los tesoros de
la casa de Dios y de los tesoros de las cosas consagradas.
\bibleverse{21} Los hijos de Ladán, los hijos de los gersonitas que
pertenecían a Ladán, los jefes de familia que pertenecían a Ladán el
gersonita: Jehieli. \footnote{\textbf{26:21} 1Cró 23,8} \bibleverse{22}
Los hijos de Jehieli Zetam, y Joel su hermano, sobre los tesoros de la
casa de Yahvé. \bibleverse{23} De los amramitas, de los izharitas, de
los hebronitas y de los uzielitas: \bibleverse{24} Shebuel hijo de
Gersón, hijo de Moisés, era el encargado de los tesoros. \footnote{\textbf{26:24}
  1Cró 23,16} \bibleverse{25} Sus hermanos: de Eliezer, su hijo Rehabía,
su hijo Jesaías, su hijo Joram, su hijo Zicri y su hijo Selomot.
\footnote{\textbf{26:25} 1Cró 23,17} \bibleverse{26} Este Selomot y sus
hermanos estaban a cargo de todos los tesoros de las cosas dedicadas,
que el rey David y los jefes de las casas paternas, los capitanes de
millares y de centenas, y los capitanes del ejército, habían dedicado.
\bibleverse{27} Dedicaron parte del botín ganado en las batallas para
reparar la casa de Yavé. \bibleverse{28} Todo lo que habían dedicado el
vidente Samuel, Saúl hijo de Cis, Abner hijo de Ner y Joab hijo de
Sarvia, quienquiera que hubiera dedicado algo, estaba bajo la mano de
Selomot y de sus hermanos.

\bibleverse{29} De los izharitas, Quenanías y sus hijos fueron
designados para los asuntos exteriores de Israel, como oficiales y
jueces. \bibleverse{30} De los hebronitas, Hasabías y sus hermanos, mil
setecientos hombres valientes, tenían la administración de Israel al
otro lado del Jordán, hacia el oeste, para todos los asuntos de Yahvé y
para el servicio del rey. \bibleverse{31} De los hebronitas, Jerías era
el jefe de los hebronitas, según sus generaciones por casas paternas.
Fueron buscados en el año cuarenta del reinado de David, y se
encontraron entre ellos hombres valientes en Jazer de Galaad.
\bibleverse{32} Sus parientes, hombres de valor, eran dos mil
setecientos, jefes de familias paternas, a quienes el rey David nombró
supervisores de los rubenitas, de los gaditas y de la media tribu de los
manasitas, para todo asunto relacionado con Dios y con los asuntos del
rey.

\hypertarget{los-doce-jefes-militares-los-caudillos-y-los-demuxe1s-altos-funcionarios-de-david-la-divisiuxf3n-del-ejuxe9rcito-en-doce}{%
\subsection{Los doce jefes militares, los caudillos y los demás altos
funcionarios de David; La división del ejército en
doce}\label{los-doce-jefes-militares-los-caudillos-y-los-demuxe1s-altos-funcionarios-de-david-la-divisiuxf3n-del-ejuxe9rcito-en-doce}}

\hypertarget{section-26}{%
\section{27}\label{section-26}}

\bibleverse{1} Los hijos de Israel, según su número, los jefes de
familia y los capitanes de millares y de centenas, y sus oficiales que
servían al rey en cualquier asunto de las divisiones que entraban y
salían mes a mes durante todos los meses del año, eran veinticuatro mil
en cada división.

\bibleverse{2} Al frente de la primera división del primer mes estaba
Jashobeam, hijo de Zabdiel. En su división había veinticuatro mil.
\bibleverse{3} Era de los hijos de Pérez, el jefe de todos los capitanes
del ejército del primer mes. \bibleverse{4} El jefe de la división del
segundo mes era Dodai el ahohita y su división, y Mikloth el jefe; en su
división había veinticuatro mil. \bibleverse{5} El tercer jefe del
ejército para el tercer mes era Benaía, hijo del sumo sacerdote Joiada.
En su división había veinticuatro mil. \bibleverse{6} Este es el Benaía
que era el hombre fuerte de los treinta y sobre los treinta. De su
división era Amizabad su hijo. \bibleverse{7} El cuarto jefe del cuarto
mes era Asael, hermano de Joab, y después de él Zebadías, su hijo. En su
división había veinticuatro mil. \bibleverse{8} El quinto jefe del
quinto mes era Samhut el izraíta. En su división había veinticuatro mil.
\bibleverse{9} El sexto capitán para el sexto mes era Ira, hijo de
Ikkesh el Tekoíta. En su división había veinticuatro mil.
\bibleverse{10} El séptimo jefe del séptimo mes era Helez pelonita, de
los hijos de Efraín. En su división había veinticuatro mil.
\bibleverse{11} El octavo jefe del octavo mes era Sibecai husatita, de
los zeraítas. En su división había veinticuatro mil. \footnote{\textbf{27:11}
  1Cró 20,4} \bibleverse{12} El noveno jefe del noveno mes era Abiezer
anatotita, de los benjamitas. En su división había veinticuatro mil.
\bibleverse{13} El décimo jefe del décimo mes era Maharai netofatita, de
los zeraítas. En su división había veinticuatro mil. \bibleverse{14} El
undécimo jefe del undécimo mes era Benaía Piratonita, de los hijos de
Efraín. En su división había veinticuatro mil. \bibleverse{15} El
duodécimo jefe del duodécimo mes era Heldai netofatita, de Otoniel. En
su división había veinticuatro mil.

\hypertarget{los-doce-pruxedncipes-tribales-de-israel}{%
\subsection{Los doce príncipes tribales de
Israel}\label{los-doce-pruxedncipes-tribales-de-israel}}

\bibleverse{16} Además, sobre las tribus de Israel de los rubenitas,
Eliezer hijo de Zicri era el jefe; de los simeonitas, Sefatías hijo de
Maaca; \bibleverse{17} de Leví, Hasabías hijo de Kemuel; de Aarón,
Sadoc; \bibleverse{18} de Judá, Elihú, uno de los hermanos de David; de
Isacar, Omrí hijo de Miguel; \bibleverse{19} de Zabulón, Ismaías hijo de
Abdías; de Neftalí, Jeremot hijo de Azriel; \bibleverse{20} de los hijos
de Efraín, Oseas hijo de Azazías; de la media tribu de Manasés, Joel
hijo de Pedaías; \bibleverse{21} de la media tribu de Manasés en Galaad,
Iddo hijo de Zacarías; de Benjamín, Jaasiel hijo de Abner;
\bibleverse{22} de Dan, Azarel hijo de Jeroham. Estos eran los capitanes
de las tribus de Israel.

\hypertarget{comentar-el-censo-incompleto}{%
\subsection{Comentar el censo
incompleto}\label{comentar-el-censo-incompleto}}

\bibleverse{23} Pero David no tomó el número de ellos de veinte años
para abajo, porque Yahvé había dicho que aumentaría a Israel como las
estrellas del cielo. \footnote{\textbf{27:23} Gén 22,17} \bibleverse{24}
Joab, hijo de Sarvia, comenzó a hacer el censo, pero no lo terminó; y la
ira cayó sobre Israel por esto. El número no fue puesto en la cuenta en
las crónicas del rey David. \footnote{\textbf{27:24} 1Cró 21,14}

\hypertarget{los-administradores-de-la-propiedad-real-tesorero-y-maestro-de-alquileres}{%
\subsection{Los administradores de la propiedad real (tesorero y maestro
de
alquileres)}\label{los-administradores-de-la-propiedad-real-tesorero-y-maestro-de-alquileres}}

\bibleverse{25} Sobre los tesoros del rey estaba Azmavet hijo de Adiel.
Sobre los tesoros en los campos, en las ciudades, en las aldeas y en las
torres estaba Jonatán hijo de Uzías; \bibleverse{26} Sobre los que
hacían el trabajo del campo para labrar la tierra estaba Ezri hijo de
Quelub. \bibleverse{27} Sobre las viñas estaba Simei ramatita. Sobre el
aumento de las viñas para las bodegas estaba Zabdi el sifmita.
\bibleverse{28} Sobre los olivos y los sicómoros que había en la tierra
baja estaba Baal Hanan gederita. Sobre las bodegas de aceite estaba
Joás. \bibleverse{29} Sobre los rebaños que se alimentaban en Sarón
estaba Sita, el sharonita. Sobre los rebaños que estaban en los valles
estaba Safat, hijo de Adlai. \bibleverse{30} Sobre los camellos estaba
Obil el ismaelita. Sobre los asnos estaba Jehdeiah el meronita. Sobre
los rebaños estaba Jaziz el hagrita. \bibleverse{31} Todos ellos eran
los jefes de la propiedad del rey David.

\hypertarget{los-muxe1s-altos-funcionarios-imperiales-consejeros-del-rey}{%
\subsection{Los más altos funcionarios imperiales (consejeros del
rey)}\label{los-muxe1s-altos-funcionarios-imperiales-consejeros-del-rey}}

\bibleverse{32} También Jonatán, tío de David, era consejero, hombre de
entendimiento y escriba. Jehiel hijo de Hacmoni estaba con los hijos del
rey. \bibleverse{33} Ajitófel era el consejero del rey. Husai el arquita
era amigo del rey. \footnote{\textbf{27:33} 2Sam 15,12; 2Sam 15,37}
\bibleverse{34} Después de Ajitófel estaban Joiada, hijo de Benaía, y
Abiatar. Joab era el capitán del ejército del rey. \footnote{\textbf{27:34}
  2Sam 8,16}

\hypertarget{el-discurso-de-david-a-los-jefes-de-israel}{%
\subsection{El discurso de David a los jefes de
Israel}\label{el-discurso-de-david-a-los-jefes-de-israel}}

\hypertarget{section-27}{%
\section{28}\label{section-27}}

\bibleverse{1} David reunió en Jerusalén a todos los príncipes de
Israel, a los príncipes de las tribus, a los capitanes de las compañías
que servían al rey por divisiones, a los capitanes de millares, a los
capitanes de centenas y a los jefes de toda la hacienda y las posesiones
del rey y de sus hijos, con los oficiales y los valientes, todos los
valientes.

\hypertarget{david-presenta-al-superior-del-pueblo-a-salomuxf3n-como-su-sucesor}{%
\subsection{David presenta al superior del pueblo a Salomón como su
sucesor}\label{david-presenta-al-superior-del-pueblo-a-salomuxf3n-como-su-sucesor}}

\bibleverse{2} Entonces el rey David se puso en pie y dijo: ``¡Oídme,
hermanos míos y pueblo mío! En cuanto a mí, estaba en mi corazón
construir una casa de reposo para el arca de la alianza de Yavé, y para
el escabel de nuestro Dios; y había preparado la construcción.
\footnote{\textbf{28:2} 1Cró 22,7-10} \bibleverse{3} Pero Dios me dijo:
`No construirás una casa a mi nombre, porque eres un hombre de guerra y
has derramado sangre'. \footnote{\textbf{28:3} 2Sam 7,5} \bibleverse{4}
Sin embargo, Yahvé, el Dios de Israel, me eligió de entre toda la casa
de mi padre para ser rey de Israel para siempre. Porque ha elegido a
Judá como príncipe; y en la casa de Judá, la casa de mi padre; y entre
los hijos de mi padre se complació en mí para hacerme rey sobre todo
Israel. \footnote{\textbf{28:4} Gén 49,10; 1Sam 16,1; 1Sam 16,12}
\bibleverse{5} De todos mis hijos (pues Yahvé me ha dado muchos hijos),
ha elegido a mi hijo Salomón para que se siente en el trono del reino de
Yahvé sobre Israel. \bibleverse{6} Me dijo: ``Salomón, tu hijo,
edificará mi casa y mis atrios; porque lo he escogido para que sea mi
hijo, y yo seré su padre. \footnote{\textbf{28:6} 1Cró 17,11-14}
\bibleverse{7} Estableceré su reino para siempre si sigue cumpliendo mis
mandamientos y mis ordenanzas, como hoy.'

\bibleverse{8} Ahora, pues, a la vista de todo Israel, de la asamblea de
Yahvé, y en audiencia de nuestro Dios, observa y busca todos los
mandamientos de Yahvé, tu Dios, para que poseas esta buena tierra y la
dejes en herencia a tus hijos después de ti para siempre.

\hypertarget{las-instrucciones-de-david-y-su-contribuciuxf3n-a-salomuxf3n}{%
\subsection{Las instrucciones de David y su contribución a
Salomón}\label{las-instrucciones-de-david-y-su-contribuciuxf3n-a-salomuxf3n}}

\bibleverse{9} Tú, Salomón, hijo mío, conoce al Dios de tu padre y
sírvele con un corazón perfecto y una mente dispuesta; porque el Señor
escudriña todos los corazones y entiende todas las imaginaciones de los
pensamientos. Si lo buscas, lo encontrarás; pero si lo abandonas, te
desechará para siempre. \footnote{\textbf{28:9} Sal 7,9} \bibleverse{10}
Presta atención ahora, porque Yahvé te ha elegido para construir una
casa para el santuario. Sé fuerte y hazlo''.

\hypertarget{david-le-da-a-salomuxf3n-el-modelo-de-la-casa-del-templo-y-los-tesoros-recolectados-para-su-construcciuxf3n}{%
\subsection{David le da a Salomón el modelo de la casa del templo y los
tesoros recolectados para su
construcción}\label{david-le-da-a-salomuxf3n-el-modelo-de-la-casa-del-templo-y-los-tesoros-recolectados-para-su-construcciuxf3n}}

\bibleverse{11} Entonces David dio a Salomón su hijo los planos del
pórtico del templo, de sus casas, de sus tesoros, de sus habitaciones
superiores, de sus habitaciones interiores, del lugar del propiciatorio;
\footnote{\textbf{28:11} Éxod 25,9} \bibleverse{12} y los planos de todo
lo que tenía por el Espíritu, para los atrios de la casa de Yahvé, para
todas las habitaciones circundantes, para los tesoros de la casa de Dios
y para los tesoros de las cosas dedicadas \bibleverse{13} también para
las divisiones de los sacerdotes y de los levitas, para toda la obra del
servicio de la casa de Yahvé, y para todos los utensilios del servicio
de la casa de Yahvé--- \bibleverse{14} de oro por peso para el oro de
todos los utensilios de toda clase de servicio, para todos los
utensilios de plata por peso, para todos los utensilios de toda clase de
servicio; \bibleverse{15} por peso también para los candelabros de oro y
para sus lámparas, de oro, por peso para cada candelabro y para sus
lámparas; y para los candelabros de plata, por peso para cada candelabro
y para sus lámparas, según el uso de cada candelabro; \bibleverse{16} y
el oro por peso para las mesas de pan de muestra, para cada mesa; y la
plata para las mesas de plata \bibleverse{17} y los tenedores, los
cuencos y las copas, de oro puro; y para los cuencos de oro, por peso,
para cada cuenco; y para los cuencos de plata, por peso, para cada
cuenco; \bibleverse{18} y para el altar del incienso, oro refinado por
peso; y oro para los planos del carro, y los querubines que se extienden
y cubren el arca del pacto de Yahvé. \bibleverse{19} ``Todo esto'', dijo
David, ``se me ha hecho entender por escrito de la mano de Yahvé, todas
las obras de este modelo.''

\bibleverse{20} David dijo a su hijo Salomón: ``Sé fuerte y valiente, y
hazlo. No temas ni te desanimes, porque el Dios de Yahvé, mi Dios, está
contigo. Él no te fallará ni te abandonará, hasta que toda la obra para
el servicio de la casa de Yahvé esté terminada. \footnote{\textbf{28:20}
  1Cró 22,13; Deut 31,6} \bibleverse{21} He aquí que hay divisiones de
los sacerdotes y de los levitas para todo el servicio de la casa de
Dios. Todo hombre dispuesto que tenga habilidad para cualquier clase de
servicio estará con ustedes en toda clase de trabajo. También los
capitanes y todo el pueblo estarán enteramente a tus órdenes''.

\hypertarget{la-contribuciuxf3n-de-los-pruxedncipes-a-la-construcciuxf3n-del-templo-siguiendo-la-amonestaciuxf3n-de-david}{%
\subsection{La contribución de los príncipes a la construcción del
templo siguiendo la amonestación de
David}\label{la-contribuciuxf3n-de-los-pruxedncipes-a-la-construcciuxf3n-del-templo-siguiendo-la-amonestaciuxf3n-de-david}}

\hypertarget{section-28}{%
\section{29}\label{section-28}}

\bibleverse{1} El rey David dijo a toda la asamblea: ``Salomón, mi hijo,
a quien sólo Dios ha elegido, es todavía joven y tierno, y la obra es
grande; porque el palacio no es para el hombre, sino para Yahvé Dios.
\footnote{\textbf{29:1} 1Cró 22,5} \bibleverse{2} He preparado con todas
mis fuerzas para la casa de mi Dios el oro para las cosas de oro, la
plata para las cosas de plata, el bronce para las cosas de bronce, el
hierro para las cosas de hierro, y la madera para las cosas de madera,
también piedras de ónice, piedras para engastar, piedras para
incrustaciones de diversos colores, toda clase de piedras preciosas, y
piedras de mármol en abundancia. \bibleverse{3} Además, como he puesto
mi afecto en la casa de mi Dios, ya que tengo un tesoro propio de oro y
plata, lo doy a la casa de mi Dios, además de todo lo que he preparado
para la casa santa \bibleverse{4} tres mil talentos de oro, del oro de
Ofir, y siete mil talentos\footnote{\textbf{29:4} Un talento es de unos
  30 kilogramos o 66 libras o 965 onzas troy, por lo que 5000 talentos
  son unas 150 toneladas métricas} de plata refinada, con los que se
recubrirán las paredes de las casas; \bibleverse{5} de oro para las
cosas de oro, y de plata para las cosas de plata, y para toda clase de
trabajos que se hagan por manos de artesanos. ¿Quién, pues, se ofrece
voluntariamente a consagrarse hoy a Yahvé?'' \footnote{\textbf{29:5}
  Éxod 35,5}

\bibleverse{6} Entonces los príncipes de las casas paternas, los
príncipes de las tribus de Israel y los capitanes de millares y de
centenas, con los jefes de la obra del rey, ofrecieron voluntariamente;
\bibleverse{7} y dieron para el servicio de la casa de Dios de oro cinco
mil talentos y diez mil dáricos,\footnote{\textbf{29:7} un dárico era
  una moneda de oro emitida por un rey persa, que pesaba unos 8,4 gramos
  o unas 0,27 onzas troy cada una.} de plata diez mil talentos, de
bronce dieciocho mil talentos y de hierro cien mil talentos.
\bibleverse{8} Las personas que encontraron piedras preciosas las
entregaron al tesoro de la casa de Yavé, bajo la mano de Jehiel el
gersonita. \footnote{\textbf{29:8} Éxod 35,27} \bibleverse{9} Entonces
el pueblo se alegró, porque ofrecía de buena gana, porque con un corazón
perfecto ofrecía de buena gana a Yavé; y el rey David también se alegró
mucho.

\hypertarget{oraciuxf3n-final-de-david}{%
\subsection{Oración final de David}\label{oraciuxf3n-final-de-david}}

\bibleverse{10} Por eso David bendijo a Yavé en presencia de toda la
asamblea, y dijo: ``Bendito seas, Yavé, el Dios de Israel, nuestro
padre, por los siglos de los siglos. \bibleverse{11} Tuya es, Yahvé, la
grandeza, el poder, la gloria, la victoria y la majestad. Porque todo lo
que hay en los cielos y en la tierra es tuyo. Tuyo es el reino, Yahvé, y
tú eres exaltado como cabeza de todo. \footnote{\textbf{29:11} Apoc
  4,11; Apoc 5,13} \bibleverse{12} ¡Las riquezas y el honor provienen de
ti, y tú gobiernas sobre todo! En tu mano está el poder y la fuerza. En
tu mano está engrandecer y dar fuerza a todos. \footnote{\textbf{29:12}
  2Cró 20,6} \bibleverse{13} Por eso, Dios nuestro, te damos gracias y
alabamos tu glorioso nombre. \bibleverse{14} Pero, ¿quién soy yo, y qué
es mi pueblo, para que podamos ofrecer tan voluntariamente como esto?
Porque todo viene de ti, y nosotros te hemos dado de lo tuyo.
\bibleverse{15} Porque somos extranjeros ante vosotros y forasteros,
como lo fueron todos nuestros padres. Nuestros días en la tierra son
como una sombra, y no queda nada. \footnote{\textbf{29:15} Sal 39,12;
  Heb 11,13; Job 14,2} \bibleverse{16} Yahvé, nuestro Dios, todo este
depósito que hemos preparado para construirte una casa para tu santo
nombre viene de tu mano, y es todo tuyo. \bibleverse{17} Sé también,
Dios mío, que tú pruebas el corazón y te complaces en la rectitud. En
cuanto a mí, en la rectitud de mi corazón he ofrecido voluntariamente
todas estas cosas. Ahora he visto con alegría a tu pueblo, que está aquí
presente, ofrecerte voluntariamente. \footnote{\textbf{29:17} 1Cró 28,9}
\bibleverse{18} Yahvé, el Dios de Abraham, de Isaac y de Israel,
nuestros padres, mantén este deseo para siempre en el pensamiento del
corazón de tu pueblo, y prepara su corazón para ti; \bibleverse{19} y
dale a Salomón, mi hijo, un corazón perfecto, para que guarde tus
mandamientos, tus testimonios y tus estatutos, y para que haga todas
estas cosas, y para que construya el palacio, para el cual he hecho
provisión.''

\hypertarget{final-solemne-de-la-reuniuxf3n-la-unciuxf3n-de-salomuxf3n-como-rey-fin-del-reinado-de-david}{%
\subsection{Final solemne de la reunión; La unción de Salomón como rey;
Fin del reinado de
David}\label{final-solemne-de-la-reuniuxf3n-la-unciuxf3n-de-salomuxf3n-como-rey-fin-del-reinado-de-david}}

\bibleverse{20} Entonces David dijo a toda la asamblea: ``¡Bendigan
ahora a Yahvé, su Dios!'' Toda la asamblea bendijo a Yavé, el Dios de
sus padres, e inclinaron sus cabezas y se postraron ante Yavé y el rey.
\bibleverse{21} Al día siguiente de ese día sacrificaron a Yavé y
ofrecieron holocaustos a Yavé, mil toros, mil carneros y mil corderos,
con sus libaciones y sacrificios en abundancia para todo Israel,
\bibleverse{22} y aquel día comieron y bebieron ante Yavé con gran
alegría. Hicieron rey por segunda vez a Salomón, hijo de David, y lo
ungieron ante Yavé como príncipe, y a Sadoc como sacerdote. \footnote{\textbf{29:22}
  1Cró 23,1}

\bibleverse{23} Entonces Salomón se sentó en el trono de Yahvé como rey
en lugar de David, su padre, y prosperó; y todo Israel le obedeció.
\footnote{\textbf{29:23} 1Cró 28,5; 1Re 1,35; 1Re 1,39} \bibleverse{24}
Todos los príncipes, los valientes y también todos los hijos del rey
David se sometieron al rey Salomón. \bibleverse{25} El Señor engrandeció
mucho a Salomón a los ojos de todo Israel, y le dio una majestad real
como no la había tenido ningún rey antes de él en Israel. \footnote{\textbf{29:25}
  2Cró 1,1}

\hypertarget{el-final-de-david-y-las-fuentes-de-su-historia}{%
\subsection{El final de David y las fuentes de su
historia}\label{el-final-de-david-y-las-fuentes-de-su-historia}}

\bibleverse{26} David, hijo de Isaí, reinó sobre todo Israel.
\bibleverse{27} El tiempo que reinó sobre Israel fue de cuarenta años;
reinó siete años en Hebrón, y treinta y tres años en Jerusalén.
\footnote{\textbf{29:27} 1Re 2,11} \bibleverse{28} Murió en buena vejez,
lleno de días, de riquezas y de honores; y en su lugar reinó su hijo
Salomón. \bibleverse{29} Los hechos del rey David, primeros y últimos,
están escritos en la historia del vidente Samuel, en la historia del
profeta Natán y en la historia del vidente Gad, \footnote{\textbf{29:29}
  1Cró 21,9} \bibleverse{30} con todo su reinado y su poderío, y los
sucesos que lo involucraron a él, a Israel y a todos los reinos de las
tierras.
