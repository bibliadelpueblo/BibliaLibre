\hypertarget{section}{%
\section{1}\label{section}}

\bibleverse{1} La parentela, los judíos que están en Jerusalén y los que
están en el país de Judea, envían saludos y buena paz a la parentela,
los judíos que están en todo Egipto. \bibleverse{2} Que Dios os haga el
bien y se acuerde de su alianza con Abraham, Isaac y Jacob, sus fieles
servidores, \bibleverse{3} y os dé a todos un corazón para adorarle y
hacer su voluntad con un corazón fuerte y un alma dispuesta.
\bibleverse{4} Que Dios abra vuestro corazón a su ley y a sus estatutos,
y haga la paz, \bibleverse{5} y escuche vuestras peticiones, y se
reconcilie con vosotros, y no os abandone en los malos tiempos.
\bibleverse{6} Ahora rezamos aquí por ti.

\bibleverse{7} En el reinado de Demetrio, en el año ciento sesenta y
nueve, los judíos ya te escribimos por el sufrimiento y la angustia que
nos ha sobrevenido en estos años, desde que Jasón y su compañía se
rebelaron de la tierra santa y del reino, \bibleverse{8} e incendiaron
la puerta, y derramaron sangre inocente. Oramos al Señor, y fuimos
escuchados. Ofrecimos sacrificios y ofrendas de comida. Encendimos las
lámparas. Pusimos el pan de la feria. \footnote{\textbf{1:8} Gr. panes}
\bibleverse{9} Ahora bien, procurad celebrar los días de la fiesta de
los tabernáculos en el mes de Chislev del año ciento ochenta y ocho.

\bibleverse{10} El pueblo de Jerusalén y los que están en Judea, con el
senado y Judas, a Aristóbulo, maestro del rey Tolomeo, que también es de
la estirpe de los sacerdotes ungidos, y a los judíos que están en
Egipto, les enviamos saludos y salud.

\bibleverse{11} Habiendo sido salvados por Dios de grandes peligros,
como hombres que se enfrentan a un rey, le damos muchas gracias.
\bibleverse{12} Porque arrojó a Persia a los que nos combatían en la
ciudad santa. \bibleverse{13} Pues cuando el príncipe llegó allí, con un
ejército que parecía irresistible, fueron despedazados en el templo de
Nanaea por la traición de los sacerdotes de Nanaea. \bibleverse{14}
Porque Antíoco, con el pretexto de que iba a casarse con ella, entró en
el lugar, él y sus amigos que estaban con él, para tomar una gran parte
de los tesoros como dote. \bibleverse{15} Cuando los sacerdotes del
templo de Nanaea habían dispuesto los tesoros y él había llegado allí
con una pequeña compañía dentro del muro del recinto sagrado, cerraron
el templo cuando Antíoco entró. \bibleverse{16} Abriendo la puerta
secreta del techo de paneles, lanzaron piedras y abatieron al príncipe.
Lo despedazaron a él y a su compañía, les cortaron la cabeza y la
arrojaron al pueblo que estaba fuera. \bibleverse{17} Bendito sea
nuestro Dios en todo, que entregó a los que habían cometido impiedad.

\bibleverse{18} Como estamos a punto de celebrar la purificación del
templo en el mes de Chislev, el día veinticinco, hemos creído necesario
avisaros, para que también celebréis la fiesta de los tabernáculos y
recordéis el fuego que se dio cuando Nehemías ofreció sacrificios,
después de haber construido el templo y el altar.

\bibleverse{19} En efecto, cuando nuestros padres estaban a punto de ser
conducidos a la tierra de Persia, los sacerdotes piadosos de aquel
tiempo tomaron parte del fuego del altar y lo escondieron secretamente
en el hueco de un pozo sin agua, donde se aseguraron de que el lugar
fuera desconocido para cualquiera. \bibleverse{20} Después de muchos
años, cuando Dios quiso, Nehemías, habiendo recibido un encargo del rey
de Persia, envió en busca del fuego a los descendientes de los
sacerdotes que lo habían escondido. Cuando le declararon que no habían
encontrado fuego, sino un líquido espeso, \bibleverse{21} les ordenó que
sacaran parte de él y se lo trajeran. Una vez ofrecidos los sacrificios,
Nehemías ordenó a los sacerdotes que rociaran con ese líquido tanto la
madera como las cosas puestas sobre ella. \bibleverse{22} Una vez hecho
esto y transcurrido algún tiempo, cuando salió el sol, que antes estaba
oculto por las nubes, se encendió un gran resplandor, de modo que todos
los hombres se maravillaron. \bibleverse{23} Los sacerdotes hicieron una
oración mientras se consumía el sacrificio, tanto los sacerdotes como
todos los demás. Jonatán dirigió y los demás respondieron, al igual que
Nehemías.

\bibleverse{24} La oración era así ``Oh, Señor, Señor Dios, creador de
todas las cosas, que eres imponente, fuerte, justo y misericordioso, que
eres el único rey y misericordioso, \bibleverse{25} que eres el único
que suple toda necesidad, que eres el único justo, todopoderoso y
eterno, tú que salvas a Israel de todo mal, que elegiste a los
antepasados y los santificaste, \bibleverse{26} acepta el sacrificio por
todo tu pueblo Israel, y conserva tu propia porción, y conságrala.
\bibleverse{27} Reúne a nuestro pueblo disperso, libera a los
esclavizados entre las naciones, mira a los despreciados y aborrecidos,
y haz saber a las naciones que tú eres nuestro Dios. \bibleverse{28}
Castiga a los que nos oprimen y, con arrogancia, nos agreden.
\bibleverse{29} Planta a tu pueblo en tu lugar santo, como dijo
Moisés''.

\bibleverse{30} Entonces los sacerdotes cantaron los himnos.
\bibleverse{31} En cuanto se consumió el sacrificio, Nehemías ordenó que
el resto del líquido se vertiera sobre grandes piedras. \bibleverse{32}
Hecho esto, se encendió una llama; pero cuando la luz del altar volvió a
brillar, se apagó. \bibleverse{33} Cuando se dio a conocer el asunto, y
se le dijo al rey de los persas que en el lugar donde los sacerdotes que
fueron llevados habían escondido el fuego, apareció el líquido con el
que Nehemías y los que estaban con él purificaron el sacrificio,
\bibleverse{34} entonces el rey cercó el lugar y lo hizo sagrado después
de haber investigado el asunto. \bibleverse{35} Cuando el rey se
mostraba favorable a alguno, le cambiaba muchos regalos y le daba un
poco de este líquido. \bibleverse{36} Nehemías y los que estaban con él
llamaron a esta cosa ``Neftar'', que es por interpretación,
``Purificación''; pero la mayoría de los hombres lo llaman Neftai.

\hypertarget{section-1}{%
\section{2}\label{section-1}}

\bibleverse{1} También se encuentra en los registros que el profeta
Jeremías ordenó a los que fueron llevados a tomar un poco del fuego,
como se ha mencionado, \bibleverse{2} y cómo el profeta ordenó a los que
fueron llevados, habiéndoles dado la ley, que no se olvidaran de los
estatutos del Señor ni se extraviaran en sus mentes cuando vieran
imágenes de oro y plata, y su adorno. \bibleverse{3} Con otras palabras
semejantes les exhortó a que la ley no se apartara de sus corazones.

\bibleverse{4} El profeta, advertido por Dios, ordenó que el tabernáculo
y el arca le siguieran,\footnote{\textbf{2:4} Gr. y cuando. El texto
  griego aquí es probablemente corrupto.} cuando salió al monte donde
Moisés había subido y visto la heredad de Dios. \bibleverse{5} Jeremías
llegó y encontró una cueva, introdujo en ella el tabernáculo, el arca y
el altar del incienso, y selló la entrada. \bibleverse{6} Algunos de los
que le seguían llegaron allí para marcar el camino, y no pudieron
encontrarlo. \bibleverse{7} Pero cuando Jeremías se enteró de eso, los
reprendió diciendo: ``El lugar será desconocido hasta que Dios vuelva a
reunir al pueblo y se apiade de él. \bibleverse{8} Entonces el Señor
revelará estas cosas, y la gloria del Señor se verá con la nube, como
también se mostró a Moisés, también como Salomón imploró que el lugar
fuera consagrado en gran medida, \bibleverse{9} y también se declaró que
él, teniendo sabiduría, ofreció un sacrificio de dedicación, y de
acabado del templo. \bibleverse{10} Así como Moisés oró al Señor y el
fuego descendió del cielo y consumió el sacrificio, así también Salomón
oró, y el fuego descendió y consumió los holocaustos. \bibleverse{11}
\footnote{\textbf{2:11} Ver Levítico 10:16 y 9:24.} Moisés dijo: ``Como
la ofrenda por el pecado no se había comido, se consumió de la misma
manera''. \bibleverse{12} Así también Salomón guardó los ocho días''.

\bibleverse{13} Lo mismo se cuenta en los archivos públicos y en los
registros de Nehemías, y también cómo éste, fundando una biblioteca,
reunió los libros sobre los reyes y los profetas, y los escritos de
David, y las cartas de los reyes sobre los dones sagrados.
\bibleverse{14} De la misma manera, Judas también reunió para nosotros
todos los libros que se habían dispersado a causa de la guerra, y
todavía están con nosotros. \bibleverse{15} Por tanto, si tenéis
necesidad de ellos, enviad a alguien que os los traiga.

\bibleverse{16} Viendo, pues, que estamos a punto de celebrar la
purificación, os escribimos. Haréis, pues, bien en celebrar los días.
\bibleverse{17} Ahora bien, Dios, que salvó a todo su pueblo y restituyó
la herencia a todos, con el reino, el sacerdocio y la consagración,
\bibleverse{18} tal como lo prometió por medio de la ley, en Dios
tenemos la esperanza de que pronto tendrá misericordia de nosotros y nos
reunirá de todas partes bajo el cielo en su lugar santo, pues nos libró
de grandes males y purificó el lugar.

c ------ ---

\bibleverse{19} Ahora bien, las cosas relativas a Judas Macabeo y a sus
hermanos, la purificación del templo mayor, la dedicación del altar,
\bibleverse{20} y además las guerras contra Antíoco Epífanes y Eupátor
su hijo, \bibleverse{21} y las manifestaciones que vinieron del cielo a
los que lucharon entre sí en acciones valientes por la religión de los
judíos; de modo que, siendo sólo unos pocos, se apoderaron de todo el
país, persiguieron a las multitudes bárbaras, \bibleverse{22}
recuperaron el templo reconocido en todo el mundo, liberaron la ciudad y
restablecieron las leyes que estaban a punto de ser derrocadas, viendo
que el Señor se apiadaba de ellos con toda bondad. \bibleverse{23} Estas
cosas que han sido declaradas por Jasón de Cirene en cinco libros,
intentaremos resumirlas en un solo libro. \bibleverse{24} Porque
teniendo en cuenta la masa confusa de los números, y la
dificultad\footnote{\textbf{2:24} O, cansancio} que espera a los que
quieren entrar en las narraciones de la historia, a causa de la
abundancia de la materia, \bibleverse{25} hemos tenido cuidado de que
los que quieren leer se sientan atraídos, y que los que nos desean lo
encuentren fácil de recordar, y que todos los lectores se beneficien.
\bibleverse{26} Aunque para nosotros, que hemos asumido la penosa labor
de la compilación, la tarea no es fácil, sino una cuestión de sudor y
desvelo, \bibleverse{27} así como no es cosa ligera para quien prepara
un banquete y busca el beneficio de los demás. Sin embargo, en aras de
la gratitud de muchos, soportaremos con gusto la penosa labor,
\bibleverse{28} dejando al historiador el manejo exacto de cada detalle,
y no teniendo fuerzas para rellenar los contornos de nuestro resumen.
\bibleverse{29} Porque como el maestro de obras de una casa nueva debe
cuidar toda la estructura, y también el que se encarga de decorarla y
pintarla debe buscar las cosas adecuadas para su adorno; así creo que
ocurre también con nosotros. \bibleverse{30} Ocuparse del terreno, y
entregarse a largas discusiones, y ser curioso en los detalles, es
apropiado para el primer autor de la historia; \bibleverse{31} pero
esforzarse por la brevedad de la expresión, y evitar una laboriosa
plenitud en el tratamiento, debe ser concedido a quien quiere llevar un
escrito a una nueva forma. \bibleverse{32} Comencemos, pues, la
narración, añadiendo sólo esto a lo que ya se ha dicho; porque es una
tontería hacer un largo prólogo a la historia y abreviar la historia
misma.

\hypertarget{section-2}{%
\section{3}\label{section-2}}

\bibleverse{1} Cuando la ciudad santa estaba habitada con una paz
ininterrumpida y las leyes se cumplían muy bien a causa de la piedad del
sumo sacerdote Onías y su odio a la maldad, \bibleverse{2} sucedió que
hasta los mismos reyes honraban el lugar y glorificaban el templo con
los más nobles regalos, \bibleverse{3} de modo que hasta el rey Seleuco
de Asia sufragó con sus propios ingresos todos los gastos
correspondientes a los servicios de los sacrificios. \bibleverse{4} Pero
un hombre llamado Simón, de la tribu de Benjamín, habiendo sido nombrado
guardián del templo, discrepó con el sumo sacerdote sobre la regulación
del mercado en la ciudad. \bibleverse{5} Al no poder vencer a Onías,
acudió a Apolonio de\footnote{\textbf{3:5} Griego Thraseas} Tarso, que
en aquel tiempo era gobernador de Coelesiria y Fenicia. \bibleverse{6}
Le hizo saber que el tesoro de Jerusalén estaba lleno de sumas
incalculables de dinero, de modo que la cantidad de fondos era
innumerable, y que no pertenecían a la cuenta de los sacrificios, sino
que era posible que éstos cayeran bajo el poder del rey. \bibleverse{7}
Cuando Apolonio se reunió con el rey, le informó del dinero del que se
le había hablado. Entonces el rey nombró a Heliodoro, que era su
canciller, y lo envió con la orden de llevar a cabo la retirada del
dinero denunciado. \bibleverse{8} Heliodoro se puso en marcha de
inmediato, aparentemente para visitar las ciudades de Coelesyria y
Fenicia, pero en realidad para ejecutar el propósito del rey.

\bibleverse{9} Cuando llegó a Jerusalén y fue recibido cortésmente por
el sumo sacerdote de la ciudad, le contó la información que se le había
dado y le declaró por qué había venido; y le preguntó si en verdad eran
esas cosas. \bibleverse{10} El sumo sacerdote le explicó que había en el
tesoro depósitos de viudas y huérfanos, \bibleverse{11} y además algún
dinero que pertenecía a Hircano, hijo de Tobías, hombre de muy alto
rango, no como aquel impío Simón alegaba falsamente; y que en total
había cuatrocientos talentos de plata y doscientos de oro,
\bibleverse{12} y que era del todo imposible que se hiciera mal a
quienes habían puesto su confianza en la santidad del lugar, y en la
majestad e inviolable santidad del templo, honrado sobre todo el mundo.
\bibleverse{13} Pero Heliodoro, a causa de la orden que le había dado el
rey, dijo que en cualquier caso ese dinero debía ser confiscado para el
tesoro del rey.

\bibleverse{14} Así que, habiendo fijado un día, entró para dirigir la
investigación sobre estos asuntos; y hubo no poca angustia en toda la
ciudad. \bibleverse{15} Los sacerdotes, postrados ante el altar con sus
vestiduras sacerdotales, clamaban al cielo por el que había dado la ley
sobre los depósitos, para que preservara estos tesoros a salvo para los
que los habían depositado. \bibleverse{16} Quien veía el aspecto del
sumo sacerdote quedaba herido en su ánimo, pues su semblante y el cambio
de su color delataban la angustia de su alma. \bibleverse{17} Porque un
terror y un estremecimiento del cuerpo se habían apoderado de aquel
hombre, por lo cual el dolor que había en su corazón se manifestaba
claramente a los que lo miraban. \bibleverse{18} Los que estaban en las
casas salieron en tropel para hacer una súplica universal, porque el
lugar estaba a punto de caer en la deshonra. \bibleverse{19} Las
mujeres, ceñidas de cilicio bajo el pecho, se agolpaban en las calles.
Las vírgenes que estaban encerradas corrían juntas, unas hacia las
puertas, otras hacia los muros, y algunas se asomaban a las ventanas.
\bibleverse{20} Todas, extendiendo las manos hacia el cielo, hacían su
solemne súplica. \bibleverse{21} Entonces fue lamentable ver a la
multitud postrada toda junta, y la ansiedad del sumo sacerdote en su
gran angustia.

\bibleverse{22} Mientras, por tanto, invocaban al Señor Todopoderoso
para que mantuviera a salvo las cosas que les habían sido
confiadas\footnote{\textbf{3:22} Gr. seguro con toda seguridad.} y
asegurara a los que las habían confiado, \bibleverse{23} Heliodoro pasó
a ejecutar lo que se había decretado. \bibleverse{24} Pero cuando ya
estaba presente allí con sus guardias cerca del tesoro, el Soberano de
los espíritus y de toda autoridad provocó una gran manifestación, de
modo que todos los que habían presumido de venir con él, espantados por
el poder de Dios, se desmayaron de terror. \bibleverse{25} Pues vieron
un caballo con un jinete espantoso, adornado con hermosos atavíos, que
se abalanzó con furia y golpeó a Heliodoro con sus patas delanteras.
Parecía que el que iba sentado en el caballo tenía una armadura completa
de oro. \bibleverse{26} También se le aparecieron otros dos, jóvenes
notables por su fuerza, hermosos por su gloria y espléndidos por su
vestimenta, que se colocaron junto a él a ambos lados y lo azotaron sin
cesar, infligiéndole muchos y dolorosos azotes. \bibleverse{27} Cuando
cayó repentinamente al suelo, y una gran oscuridad se apoderó de él, sus
guardias lo levantaron y lo pusieron en una camilla, \bibleverse{28} y
lo llevaron: este hombre, que acababa de entrar con un gran séquito y
toda su guardia en el citado tesoro, se encontraba ahora en una
situación de total impotencia, y manifiestamente se veía obligado a
reconocer la soberanía de Dios. \bibleverse{29} Así, mientras él, por
obra de Dios, sin palabras y desprovisto de toda esperanza y liberación,
yacía postrado, \bibleverse{30} bendecían al Señor que actuaba
maravillosamente por su propio lugar. El templo, que poco antes estaba
lleno de terror y alarma, se llenó de alegría y gozo tras la aparición
del Señor Todopoderoso.

\bibleverse{31} Pero rápidamente algunos de los amigos familiares de
Heliodoro imploraron a Onías que invocara al Altísimo para que
concediera la vida a quien yacía en el último suspiro. \bibleverse{32}
El sumo sacerdote, temiendo secretamente que el rey llegara a pensar que
los judíos habían perpetrado alguna traición a Heliodoro, trajo un
sacrificio para la recuperación del hombre. \bibleverse{33} Pero
mientras el sumo sacerdote hacía el sacrificio expiatorio, los mismos
jóvenes se presentaron de nuevo ante Heliodoro, vestidos con las mismas
ropas. Se pusieron de pie y le dijeron: ``Da las gracias al sumo
sacerdote Onías, porque por él el Señor te ha concedido la vida.
\bibleverse{34} Procura que, ya que has sido azotado desde el cielo,
anuncies a todos los hombres la soberana majestad de Dios''. Cuando
hubieron pronunciado estas palabras, desaparecieron de la vista.
\bibleverse{35} Entonces Heliodoro, después de haber ofrecido un
sacrificio al Señor y de haber hecho\footnote{\textbf{3:35} Gr. mayor.}
grandes votos al que le había salvado la vida, y de haber despedido a
Onías, volvió con su ejército al rey. \bibleverse{36} Dio testimonio a
todos los hombres de las obras del Dios más grande, que había visto con
sus ojos.

\bibleverse{37} Cuando el rey preguntó a Heliodoro qué clase de hombre
era apto para ser enviado una vez más a Jerusalén, dijo: \bibleverse{38}
``Si tienes algún enemigo o conspirador contra el Estado, envíalo allí,
y lo recibirás de vuelta bien azotado, si es que escapa con vida; porque
verdaderamente hay algún poder de Dios en ese lugar. \bibleverse{39}
Porque el mismo que tiene su morada en el cielo tiene sus ojos puestos
en ese lugar y lo ayuda. A los que vienen a hacerle daño, los golpea y
los destruye''.

\bibleverse{40} Esta fue la historia de Heliodoro y la custodia del
tesoro.

\hypertarget{section-3}{%
\section{4}\label{section-3}}

\bibleverse{1} El ya mencionado Simón, que había dado información sobre
el dinero contra su país, calumnió a Onías, diciendo que era él quien
había incitado a Heliodoro y había sido el verdadero causante de estos
males. \bibleverse{2} Se atrevió a llamar conspirador contra el Estado a
quien en realidad era el benefactor de la ciudad, el guardián de sus
compatriotas y un celoso de las leyes. \bibleverse{3} Cuando su odio
creció tanto que incluso se perpetraron asesinatos a través de uno de
los agentes aprobados por Simón, \bibleverse{4} Onías, viendo el peligro
de la contienda, y que\footnote{\textbf{4:4} Compárese con 2 Macabeos
  4:21. Véase también 2 Macabeos 3:5. El griego, tal como se lee
  comúnmente, significa que Apolonio, como gobernador... de Fenicia, se
  enfureció y aumentó, etc.} Apolonio hijo de Menesteo, gobernador de
Coelesiria y Fenicia, estaba aumentando la malicia de Simón,
\bibleverse{5} apeló al rey, para que no fuera acusador de sus
conciudadanos, sino que mirara por el bien de todo el pueblo\footnote{\textbf{4:5}
  Gr. multitud.} , tanto público como privado; \bibleverse{6} pues vio
que sin la participación del rey era imposible que el Estado obtuviera
más paz, y que Simón no cesaría en su locura.

\bibleverse{7} Cuando murió Seleuco y Antíoco, que se llamaba Epífanes,
sucedió en el reino, Jasón, hermano de Onías, suplantó a su hermano en
el sumo sacerdocio, \bibleverse{8} habiendo prometido al rey en una
audiencia trescientos sesenta talentos de plata, y de otro fondo ochenta
talentos. \bibleverse{9} Además de esto, se comprometió a asignar otros
ciento cincuenta, si se le permitía\footnote{\textbf{4:9} Gr. a través
  de su.} por medio de la autoridad del rey, establecer un gimnasio y un
cuerpo de jóvenes que se formaran en él, y registrar a los habitantes de
Jerusalén como ciudadanos de Antioquía. \bibleverse{10} Cuando el rey
accedió y Jasón tomó posesión del cargo, inmediatamente cambió a los de
su raza al estilo de vida griego. \bibleverse{11} Dejando a un lado las
ordenanzas reales de especial favor a los judíos, concedidas por medio
de Juan el padre de Eupolemo, que fue en misión a los romanos para
establecer amistad y alianza, y tratando de derribar las formas de vida
lícitas, introdujo nuevas costumbres prohibidas por la ley.
\bibleverse{12} Pues estableció con ahínco un gimnasio bajo la propia
ciudadela, e hizo que los más nobles de los jóvenes llevaran el sombrero
griego. \bibleverse{13} De este modo se llegó a un extremo de
helenización y a un avance de una religión extranjera, a causa de la
excesiva profanidad de Jasón, que era un hombre impío y no un sumo
sacerdote; \bibleverse{14} de modo que los sacerdotes ya no tenían
ningún celo por los servicios del altar, sino que, despreciando el
santuario y descuidando los sacrificios, se apresuraban a disfrutar de
lo que se les proporcionaba ilícitamente en la arena de lucha, después
de la convocatoria al lanzamiento de discos. \bibleverse{15}
Despreciaron los honores de sus padres y valoraron más el prestigio de
los griegos. \bibleverse{16} Por eso les sobrevino una grave calamidad.
Los hombres cuya forma de vida seguían con ahínco, y a los que deseaban
parecerse en todo, se convirtieron en sus enemigos y los castigaron.
\bibleverse{17} Porque no es cosa ligera mostrar irreverencia a las
leyes de Dios, pero los acontecimientos posteriores lo pondrán de
manifiesto.

\bibleverse{18} Cuando se celebraban en Tiro ciertos juegos que venían
cada cinco años, y el rey estaba presente, \bibleverse{19} el vil Jasón
envió enviados sagrados,\footnote{\textbf{4:19} Ver ver. 9.} como si
fuesen antioquenos de Jerusalén, llevando trescientos dracmas de plata
para el sacrificio de Hércules, que incluso los portadores de los mismos
consideraron que no era correcto utilizar para ningún sacrificio, porque
no era adecuado, sino que lo gastaron para otro fin. \bibleverse{20}
Aunque el propósito del remitente de este dinero era para el sacrificio
de Hércules, sin embargo, debido a las circunstancias actuales
de\footnote{\textbf{4:20} Algunas autoridades leen los portadores.} , se
destinó a la construcción de barcos de guerra trímeros.

\bibleverse{21} Cuando Apolonio, hijo de Menesteo, fue enviado a Egipto
para la entronización de Filométor como rey\footnote{\textbf{4:21} El
  significado exacto de la palabra griega es incierto.} , Antíoco, al
saber que Filométor se había mostrado hostil al gobierno, tomó
precauciones para la seguridad de su reino. Por ello, dirigiéndose a
Jope, se dirigió a Jerusalén. \bibleverse{22} Al ser recibido
magníficamente por Jasón y la ciudad, fue introducido con antorchas y
gritos. Luego condujo a su ejército hasta Fenicia.

\bibleverse{23} Al cabo de tres años, Jasón envió a Menelao, el hermano
de Simón antes mencionado, para que llevara el dinero al rey y le
informara sobre algunos asuntos necesarios. \bibleverse{24} Pero éste,
al ser encomendado al rey, y habiendo sido glorificado por la exhibición
de su autoridad, se aseguró el sumo sacerdocio para sí mismo, superando
a Jasón en trescientos talentos de plata. \bibleverse{25} Después de
recibir los mandatos reales, volvió sin traer nada digno del sumo
sacerdocio, sino con la pasión de un tirano cruel y la furia de un
animal salvaje. \bibleverse{26} Así, Jasón, que había suplantado a su
propio hermano, fue suplantado por otro y expulsado como fugitivo al
país de los amonitas, \bibleverse{27} Menelao tomó posesión del cargo;
pero del dinero que se había prometido al rey no se pagó nada
regularmente, aunque Sostrato, el gobernador de la ciudadela, lo exigió
--- \bibleverse{28} pues su oficio era la recaudación de los ingresos
--- por lo que ambos fueron llamados por el rey a su presencia.
\bibleverse{29} Menelao dejó a su propio hermano Lisímaco por
su\footnote{\textbf{4:29} Gr. sucesor.} adjunto en el sumo sacerdocio; y
Sóstrato dejó a Crates, que estaba sobre los chipriotas.

\bibleverse{30} Mientras esto ocurría, sucedió que los habitantes de
Tarso y de Mallus se rebelaron porque iban a ser regalados a Antíoco, la
concubina del rey. \bibleverse{31} El rey, por lo tanto, acudió
rápidamente a arreglar los asuntos, dejando para su\footnote{\textbf{4:31}
  O bien, no mucho antes celebraban la fiesta de los tabernáculos
  vagando} a Andrónico, un hombre de alto rango. \bibleverse{32}
Entonces Menelao, suponiendo que había conseguido una oportunidad
favorable, presentó a Andrónico algunos vasos de oro pertenecientes al
templo, que había robado. Ya había vendido otros en Tiro y en las
ciudades vecinas. \bibleverse{33} Cuando Onías tuvo conocimiento seguro
de esto, lo reprendió duramente, habiéndose retirado a un santuario en
Dafne, que está junto a Antioquía. \bibleverse{34} Entonces Menelao,
llevando a Andrónico aparte, le pidió que matara a Onías. Acudiendo a
Onías, y siendo persuadido de utilizar la traición, y siendo recibido
como amigo, Andrónico le dio su mano derecha con juramentos y, aunque
receloso, le persuadió para que saliera del santuario. Luego, sin tener
en cuenta la justicia, le dio muerte inmediatamente. \bibleverse{35} Por
esta razón, no sólo los judíos, sino también muchos de las demás
naciones, se indignaron y disgustaron por el injusto asesinato de aquel
hombre. \bibleverse{36} Cuando el rey regresó de los lugares de Cilicia,
los judíos que estaban en la ciudad apelaron a él contra Andrónico (los
griegos también se unieron a ellos en el odio a la maldad), alegando que
Onías había sido asesinado injustamente. \bibleverse{37} Antíoco, pues,
se apenó de corazón, se compadeció y lloró por la vida sobria y ordenada
del muerto. \bibleverse{38} Enardecido por la cólera, despojó
inmediatamente a Andrónico de su manto de púrpura y le arrancó los
vestidos interiores, y después de conducirlo por toda la ciudad hasta el
mismo lugar donde había cometido el ultraje contra Onías, quitó de en
medio al asesino, dándole el castigo que merecía.

\bibleverse{39} Cuando Lisímaco, con el consentimiento de Menelao,
cometió muchos sacrilegios en la ciudad, y cuando la noticia de los
mismos se extendió al exterior, el pueblo se reunió contra Lisímaco,
después de que ya habían sido robados muchos recipientes de oro.
\bibleverse{40} Cuando las multitudes se alzaron contra él y se llenaron
de ira, Lisímaco armó a unos tres mil hombres y, con injusta violencia,
comenzó el ataque bajo el mando de Hauran, hombre entrado en años y no
menos también en locura. \bibleverse{41} Pero cuando percibieron el
asalto de Lisímaco, unos cogieron piedras, otros troncos de madera y
otros tomaron puñados de la ceniza que había cerca, y los arrojaron
todos en salvaje confusión contra Lisímaco y los que estaban con él.
\bibleverse{42} Como resultado, hirieron a muchos de ellos, mataron a
algunos y obligaron a los demás a huir, pero al autor del sacrilegio lo
mataron junto al tesoro.

\bibleverse{43} Pero a propósito de estos asuntos, se presentó una
acusación contra Menelao. \bibleverse{44} Cuando el rey llegó a Tiro,
los tres hombres enviados por el senado defendieron la causa ante él.
\bibleverse{45} Pero Menelao, viéndose ya derrotado, prometió mucho
dinero a Ptolomeo, hijo de Dorímenes, para que ganara al rey.
\bibleverse{46} Entonces Tolomeo, llevándose al rey a un claustro, como
para que tomara aire fresco, lo convenció de que cambiara de opinión.
\bibleverse{47} Al causante de todos los males, Menelao, lo eximió de
las acusaciones; pero a estos desventurados, que si hubieran alegado
incluso ante los escitas, los habría eximido sin condena, los condenó a
muerte. \bibleverse{48} Los que eran portavoces de la ciudad y de las
familias de Israel y de los vasos sagrados, pronto sufrieron esa injusta
pena. \bibleverse{49} Por eso, incluso algunos tirios, movidos por el
odio a la maldad, proveyeron magníficamente para su entierro.
\bibleverse{50} Pero Menelao, por los manejos codiciosos de los que
estaban en el poder, permaneció todavía en su cargo, creciendo en
maldad, establecido como gran conspirador contra sus conciudadanos.

\hypertarget{section-4}{%
\section{5}\label{section-4}}

\bibleverse{1} Por esta época, Antíoco realizó su segunda invasión a
Egipto. \bibleverse{2} Sucedió que por toda la ciudad, durante casi
cuarenta días, apareció en medio del cielo una caballería en rápido
movimiento, vistiendo túnicas tejidas con oro y portando lanzas,
equipada con tropas para la batalla, \bibleverse{3} desenvainando
espadas, escuadrones de caballería en formación, encuentros y
persecuciones de ambos ejércitos, escudos agitados, multitud de lanzas,
lanzamiento de proyectiles, destellos de adornos de oro y puesta de toda
clase de armaduras. \bibleverse{4} Por lo tanto, todos oraron para que
la manifestación se diera para bien.

\bibleverse{5} Cuando surgió el falso rumor de que Antíoco había muerto,
Jasón tomó no menos de mil hombres y asaltó repentinamente la ciudad.
Cuando los que estaban en la muralla fueron derrotados, y la ciudad
estuvo a punto de ser tomada, Menelao se refugió en la ciudadela.
\bibleverse{6} Pero Jasón masacró sin piedad a sus propios ciudadanos,
sin considerar que el buen éxito contra los parientes es la mayor de las
desgracias, sino suponiendo que se erigía en trofeo sobre los enemigos y
no sobre los compatriotas. \bibleverse{7} No consiguió el control del
gobierno, pero al recibir la vergüenza como resultado de su
conspiración, huyó de nuevo como fugitivo al país de los amonitas.
\bibleverse{8} Por lo tanto, finalmente tuvo un final miserable.
Habiendo sido encarcelado en la corte de Aretas, el príncipe de los
árabes, huyendo de ciudad en ciudad, perseguido por todos los hombres,
odiado como rebelde contra las leyes y aborrecido como verdugo de su
país y de sus conciudadanos, fue arrojado a Egipto. \bibleverse{9} El
que había expulsado a muchos de su propio país al exilio, pereció en el
destierro, habiendo cruzado el mar hacia los lacedemonios, esperando
encontrar allí refugio porque eran parientes cercanos. \bibleverse{10}
El que había echado a una multitud sin enterrar no tenía a nadie que lo
llorara. No tuvo ningún funeral ni lugar en la tumba de sus antepasados.

\bibleverse{11} Cuando llegó al rey la noticia de lo que había sucedido,
pensó que Judea se había sublevado. Así que, partiendo de Egipto con
furia, tomó la ciudad por la fuerza de las armas, \bibleverse{12} y
ordenó a sus soldados que cortaran sin piedad a los que se cruzaran en
su camino, y que mataran a los que entraran en sus casas.
\bibleverse{13} Entonces hubo matanza de jóvenes y ancianos, destrucción
de muchachos, mujeres y niños, y matanza de vírgenes y niños.
\bibleverse{14} En un total de tres días, fueron destruidos ochenta mil,
de los cuales cuarenta mil fueron muertos en combate cuerpo a cuerpo, y
no fueron menos los vendidos como esclavos que los muertos.

\bibleverse{15} No contento con esto, presumió de entrar en el templo
más sagrado de toda la tierra, teniendo a Menelao por guía (que había
demostrado ser un traidor tanto a las leyes como a su país),
\bibleverse{16} incluso tomando los vasos sagrados con sus manos
contaminadas, y arrastrando con sus manos profanas las ofrendas que
habían sido dedicadas por otros reyes para aumentar la gloria y el honor
del lugar. \bibleverse{17} Antíoco se ensoberbeció, sin ver que a causa
de los pecados de los que vivían en la ciudad, el Señor Soberano había
sido provocado a la ira por un tiempo, y por eso su mirada se apartó del
lugar. \bibleverse{18} Pero si no fuera porque ya estaban atados por
muchos pecados, este hombre, al igual que Heliodoro, que fue enviado por
el rey Seleuco para ver el tesoro, habría sido azotado en cuanto se
presentara y se habría apartado de su atrevimiento. \bibleverse{19} Sin
embargo, el Señor no eligió la nación por el lugar, sino el lugar por la
nación. \bibleverse{20} Por lo tanto, también el lugar mismo, habiendo
participado en las calamidades que le sucedieron a la nación, participó
después en sus beneficios; y el lugar que fue abandonado en la ira del
Todopoderoso fue, en la reconciliación del gran Soberano, restaurado de
nuevo con toda gloria.

\bibleverse{21} En cuanto a Antíoco, cuando sacó del templo mil
ochocientos talentos, se apresuró a irse a Antioquía, pensando en su
arrogancia que podía navegar por tierra y caminar por el mar, porque su
corazón estaba enaltecido. \bibleverse{22} Además, dejó gobernadores
para afligir a la raza: en Jerusalén, Filipo, de raza frigia y de
carácter más bárbaro que el que lo puso allí; \bibleverse{23} y en
Gerizim, Andrónico; y además de éstos, Menelao, que peor que todos los
demás, se exaltó contra sus conciudadanos. Teniendo una mente maliciosa
hacia los judíos a quienes había convertido en sus ciudadanos,
\bibleverse{24} envió a ese señor de las contaminaciones, Apolonio, con
un ejército de veintidós mil personas, ordenándole que matara a todos
los mayores de edad, y que vendiera a las mujeres y a los niños como
esclavos. \bibleverse{25} Llegó a Jerusalén, y fingiendo ser un hombre
de paz, esperó hasta el día sagrado del sábado, y encontrando a los
judíos en reposo del trabajo, ordenó a sus hombres que desfilaran
completamente armados. \bibleverse{26} Pasó a cuchillo a todos los que
salieron al espectáculo. Corriendo hacia la ciudad con los hombres
armados, mató a grandes multitudes. \bibleverse{27} Pero Judas, que
también se llama Macabeo, con unos nueve más, se retiró y con su
compañía se mantuvo vivo en los montes como lo hacen los animales
salvajes. Siguieron alimentándose de lo que crecía en estado salvaje,
para no ser partícipes de la inmundicia.

\hypertarget{section-5}{%
\section{6}\label{section-5}}

\bibleverse{1} No mucho después de esto, el rey envió a, un anciano de
Atenas, para obligar a los judíos a apartarse de las leyes de sus padres
y a no vivir según las leyes de Dios, \bibleverse{2} y también para
contaminar el santuario de Jerusalén y llamarlo con el nombre de Zeus
Olímpico, y para llamar al santuario de Gerizim con el nombre de Zeus
Protector de los extranjeros, tal como lo hacía la gente que vivía en
ese lugar.

\bibleverse{3} La visita de este mal fue dura y totalmente penosa.
\bibleverse{4} Porque el templo se llenó de libertinaje y de juergas por
parte de los paganos, que se prostituían y tenían relaciones sexuales
con mujeres dentro del recinto sagrado, y además traían dentro cosas que
no eran apropiadas. \bibleverse{5} El altar estaba lleno de esas cosas
abominables que habían sido prohibidas por las leyes. \bibleverse{6} El
hombre no podía guardar el sábado, ni observar las fiestas de sus
antepasados, ni siquiera confesarse judío.

\bibleverse{7} El día del nacimiento del rey, cada mes, eran conducidos
con amarga coacción a comer de los sacrificios. Cuando llegaba la fiesta
de la Dionisia, se les obligaba a ir en procesión en honor de Dionisio,
llevando coronas de hiedra. \bibleverse{8} Por sugerencia de Ptolomeo,
se emitió un decreto para que las ciudades griegas vecinas observaran la
misma conducta contra los judíos y les hicieran comer de los
sacrificios, \bibleverse{9} y que mataran a los que no eligieran pasarse
a los ritos griegos. Así que la miseria presente estaba a la vista de
todos. \bibleverse{10} Por ejemplo, trajeron a dos mujeres por haber
circuncidado a sus hijos. A éstas, cuando las llevaron públicamente por
la ciudad con los bebés colgados de sus pechos, las arrojaron de cabeza
desde el muro. \bibleverse{11} Otras que habían corrido juntas a las
cuevas cercanas para celebrar el séptimo día en secreto, fueron
delatadas a Filipo y fueron quemadas todas juntas, porque su piedad les
impedía defenderse, en vista del honor de ese día tan solemne.

\bibleverse{12} Exhorto a los que lean este libro a que no se desanimen
a causa de las calamidades, sino que reconozcan que estos castigos no
fueron para la destrucción, sino para el escarmiento de nuestra raza.
\bibleverse{13} Porque, en efecto, es una señal de gran bondad que no se
deje en paz a los que actúan impíamente durante mucho tiempo, sino que
se les castigue inmediatamente. \bibleverse{14} Porque en el caso de las
otras naciones, el Señor Soberano espera pacientemente para castigarlas
hasta que hayan llegado a la medida completa de sus pecados; pero no con
nosotros, \bibleverse{15} para no vengarse de nosotros después, cuando
hayamos llegado a la altura de nuestros pecados. \bibleverse{16} Por eso
nunca retira de nosotros su misericordia, sino que, aunque castiga con
calamidades, no abandona a su pueblo. \bibleverse{17} Sin embargo, baste
esto que hemos dicho para recordarlo; pero después de algunas palabras,
debemos llegar a la narración.

\bibleverse{18} Eleazar, uno de los principales escribas, hombre ya muy
avanzado en edad y de noble semblante, fue obligado a abrir la boca para
comer carne de cerdo. \bibleverse{19} Pero él, prefiriendo la muerte con
el honor que la vida con la inmundicia, avanzó por su propia voluntad
hacia el instrumento de tortura, pero primero escupió la carne,
\bibleverse{20} como deben venir los hombres que están decididos a
rechazar cosas que ni siquiera por el amor natural a la vida es lícito
probar.

\bibleverse{21} Pero los que tenían a su cargo aquel banquete de
sacrificios prohibidos tomaron al hombre aparte, por la amistad que de
antiguo tenían con él, y le suplicaron en privado que trajera carne de
su propia provisión, tal como era propio de él, y que hiciera como si
comiera de la carne del sacrificio, tal como había sido ordenado por el
rey; \bibleverse{22} para que al hacerlo se librara de la muerte, y así
su antigua amistad con ellos fuera tratada con benevolencia.
\bibleverse{23} Pero él, habiendo tomado una decisión elevada y acorde
con sus años, la dignidad de la vejez y las canas que había alcanzado
con honor, y su excelente educación desde niño, o más bien las santas
leyes de ordenación de Dios, declaró su mente en consecuencia,
ordenándoles que lo enviaran rápidamente al Hades.

\bibleverse{24} ``Porque no es propio de nuestros años disimular'',
dijo, ``que muchos de los jóvenes supongan que Eleazar, el hombre de
noventa años, se ha pasado a una religión ajena; \bibleverse{25} y así
ellos, a causa de mi engaño, y en aras de esta vida breve y momentánea,
se extraviarían por mi culpa, y yo me mancharía y deshonraría en mi
vejez. \bibleverse{26} Pues, aunque por el momento me quitara el castigo
de los hombres, sin embargo, viva o muera, no escaparía de las manos del
Todopoderoso. \bibleverse{27} Por lo tanto, separando valientemente mi
vida ahora, me mostraré digno de mi vejez, \bibleverse{28} y dejaré un
noble ejemplo a los jóvenes para que mueran voluntaria y noblemente una
muerte gloriosa por las veneradas y santas leyes.''

Cuando hubo dicho estas palabras, se dirigió inmediatamente al
instrumento de tortura. \bibleverse{29} Cuando cambiaron la buena
voluntad que tenían hacia él un poco antes en mala voluntad porque estas
palabras suyas eran, según pensaban, una pura locura, \bibleverse{30} y
cuando estaba a punto de morir con los golpes, gimió en voz alta y dijo:
``Al Señor, que tiene el santo conocimiento, se le manifiesta que,
aunque podría haberme librado de la muerte, soporto fuertes dolores en
mi cuerpo al ser azotado; pero en el alma sufro de buen grado estas
cosas por mi temor a él.''

\bibleverse{31} Así murió también este hombre, dejando su muerte como
ejemplo de nobleza y monumento a la virtud, no sólo para los jóvenes
sino también para el gran cuerpo de su nación.

\hypertarget{section-6}{%
\section{7}\label{section-6}}

\bibleverse{1} Sucedió que, por orden del rey, siete hermanos y su madre
fueron apresados y manipulados vergonzosamente con azotes y cuerdas,
para obligarlos a probar la abominable carne de cerdo. \bibleverse{2}
Uno de ellos se hizo portavoz y dijo: ``¿Qué queréis pedir y aprender de
nosotros? Porque estamos dispuestos a morir antes que transgredir las
leyes de nuestros antepasados''.

\bibleverse{3} El rey montó en cólera y ordenó que se calentaran
sartenes y calderas. \bibleverse{4} Cuando estos se calentaron
inmediatamente, dio órdenes de cortar la lengua del que había sido su
portavoz, y de arrancarle la cabellera y cortarle las extremidades, con
el resto de sus hermanos y su madre mirando. \bibleverse{5} Y cuando
estuvo completamente mutilado, el rey dio órdenes de llevarlo al fuego,
estando aún vivo, y de freírlo en la sartén. Y mientras el humo de la
sartén se extendía a lo lejos, ellos y su madre también se exhortaban
mutuamente a morir noblemente, diciendo esto \bibleverse{6} ``El Señor
Dios ve, y en verdad es suplicado por nosotros, como declaró Moisés en
su canción, que atestigua contra el pueblo en su cara, diciendo: `Y
tendrá compasión de sus siervos'.''

\bibleverse{7} Cuando el primero murió así, llevaron al segundo al
escarnio; le arrancaron la piel de la cabeza con el pelo y le
preguntaron: ``¿Quieres comer, antes de que tu cuerpo sea castigado en
todos sus miembros?''

\bibleverse{8} Pero él respondió en la lengua de sus antepasados y les
dijo: ``No''. Por lo tanto, también él se sometió a la siguiente tortura
sucesivamente, como lo había hecho el primero. \bibleverse{9} Cuando
estaba en el último suspiro, dijo: ``Tú, malhechor, libéranos de esta
vida presente, pero el Rey del mundo nos resucitará a los que hemos
muerto por sus leyes para una renovación eterna de la vida.''

\bibleverse{10} Después de él, el tercero fue víctima de sus burlas.
Cuando se le requirió, sacó rápidamente la lengua y extendió las manos
con valentía, \bibleverse{11} y dijo noblemente: ``A mí me llegaron del
cielo. Por sus leyes las trato con desprecio. De él espero recibirlas de
nuevo''. \bibleverse{12} Como resultado, el propio rey y los que estaban
con él se asombraron del alma del joven, pues consideraba las penas como
nada.

\bibleverse{13} Cuando él también murió, manipularon y torturaron al
cuarto de la misma manera. \bibleverse{14} Estando a punto de morir dijo
esto: ``Es bueno morir a manos de los hombres y esperar la esperanza que
nos da Dios, de que seremos resucitados por él. Pues en cuanto a
vosotros, no tendréis resurrección a la vida''.

\bibleverse{15} Después de él, trajeron al quinto y lo manipularon
vergonzosamente. \bibleverse{16} Pero él miró hacia, el rey, y dijo:
``Porque tienes autoridad entre los hombres, aunque eres corruptible,
haces lo que quieres. Pero no pienses que nuestra raza ha sido
abandonada por Dios. \bibleverse{17} ¡Pero mantén tus costumbres, y
verás cómo su soberana majestad te torturará a ti y a tus
descendientes!''

\bibleverse{18} Después de él trajeron al sexto. Cuando estaba a punto
de morir, dijo: ``No os engañéis en vano, pues sufrimos estas cosas por
nuestras propias acciones, como si pecáramos contra nuestro propio Dios.
Han sucedido cosas asombrosas; \bibleverse{19} ¡pero no penséis que
quedaréis impunes, habiendo intentado luchar contra Dios!''

\bibleverse{20} Pero, sobre todo, la madre fue maravillosa y digna de
honrosa memoria, pues cuando vio perecer a siete hijos en el espacio de
un día, soportó el espectáculo con buen ánimo a causa de su esperanza en
el Señor. \bibleverse{21} Exhortó a cada uno de ellos en la lengua de
sus padres, llena de un espíritu noble y avivando sus pensamientos de
mujer con valor varonil, diciéndoles: \bibleverse{22} ``No sé cómo
habéis venido a mi vientre. No fui yo quien te dio tu espíritu y tu
vida. No fui yo quien puso en orden los primeros elementos de cada uno
de vosotros. \bibleverse{23} Por eso, el Creador del mundo, que dio
forma al primer origen del hombre e ideó el primer origen de todas las
cosas, por misericordia os devuelve de nuevo tanto vuestro espíritu como
vuestra vida, ya que ahora os tratáis con desprecio por sus leyes.''

\bibleverse{24} Pero Antíoco, creyéndose despreciado, y sospechando la
voz de reproche, mientras el más joven aún vivía no sólo le hizo su
llamamiento con palabras, sino que al mismo tiempo le prometió con
juramentos que lo enriquecería y lo elevaría a altos honores si se
apartaba de los caminos de sus antepasados, y que lo tomaría por su
amigo y le confiaría los asuntos públicos. \bibleverse{25} Pero como el
joven no quiso escuchar, el rey llamó a su madre y la instó a aconsejar
al joven que se salvara. \bibleverse{26} Cuando él la instó con muchas
palabras, ella se empeñó en persuadir a su hijo. \bibleverse{27} Pero
inclinándose hacia él, burlándose del cruel tirano, dijo esto en la
lengua de sus padres ``Hijo mío, ten compasión de mí, que te llevé nueve
meses en mi vientre, y te amamanté tres años, y te alimenté y te traje
hasta esta edad, y te sostuve. \bibleverse{28} Te ruego, hijo mío, que
levantes tus ojos al cielo y a la tierra y veas todas las cosas que hay
en ella, y así reconozcas que Dios no las hizo de las cosas que eran, y
que la raza de los hombres de esta manera llega a existir.
\bibleverse{29} No tengas miedo de este carnicero, sino que, demostrando
ser digno de tus hermanos, acepta tu muerte, para que en la misericordia
de Dios pueda recibirte de nuevo con tus hermanos.''

\bibleverse{30} Pero antes de que terminara de hablar, el joven dijo:
``¿A qué esperáis todos? Yo no obedezco el mandamiento del rey, sino que
escucho el mandamiento de la ley que fue dada a nuestros padres por
medio de Moisés. \bibleverse{31} Pero ustedes, que han ideado toda clase
de maldades contra los hebreos, no escaparán de las manos de Dios.
\bibleverse{32} Porque nosotros sufrimos a causa de nuestros propios
pecados. \bibleverse{33} Si para reprender y castigar, nuestro Señor
vivo se ha enojado un poco, sin embargo, volverá a reconciliarse con sus
propios siervos. \bibleverse{34} Pero tú, oh hombre impío y de lo más
vil, no te envanezcas en tu salvaje orgullo con inciertas esperanzas,
levantando tu mano contra los hijos celestiales. \bibleverse{35} Porque
aún no te has librado del juicio del Dios Todopoderoso, que todo lo ve.
\bibleverse{36} Porque estos hermanos nuestros, habiendo soportado un
corto dolor que trae la vida eterna, ahora han muerto bajo el pacto de
Dios. Pero ustedes, por el juicio de Dios, recibirán en justa medida las
penas de su arrogancia. \bibleverse{37} Pero yo, como mis hermanos,
entrego mi cuerpo y mi alma por las leyes de nuestros padres, invocando
a Dios para que se apresure a ser clemente con la nación, y para que
vosotros, en medio de las pruebas y de las plagas, confeséis que sólo él
es Dios, \bibleverse{38} y para que en mí y en mis hermanos pongáis fin
a la ira del Todopoderoso que se ha abatido justamente sobre toda
nuestra raza.''

\bibleverse{39} Pero el rey, al caer en cólera, lo trató peor que a
todos los demás, exasperado por sus burlas. \bibleverse{40} Así que
también él murió puro, poniendo toda su confianza en el Señor.

\bibleverse{41} Por último, después de sus hijos, murió la madre.

\bibleverse{42} Baste, pues, con haber dicho esto sobre las fiestas de
los sacrificios y las torturas extremas.

\hypertarget{section-7}{%
\section{8}\label{section-7}}

\bibleverse{1} Pero Judas, que también se llama Macabeo, y los que
estaban con él, dirigiéndose en secreto a las aldeas, convocaron a su
parentela. Tomando a los que habían continuado en la religión de los
judíos, reunieron a unos seis mil. \bibleverse{2} Invocaron al Señor
para que mirara al pueblo oprimido por todos, y para que se compadeciera
del santuario profanado por los impíos, \bibleverse{3} y para que se
apiadara de la ciudad que sufría la ruina y estaba a punto de ser
arrasada, y para que escuchara la sangre que clamaba a él,
\bibleverse{4} y para que se acordara de la destrucción sin ley de los
niños inocentes, y de las blasfemias que se habían cometido contra su
nombre, y para que mostrara su odio a la maldad.

\bibleverse{5} Cuando Maccabaeus hubo entrenado a sus hombres para el
servicio, los paganos enseguida lo encontraron irresistible, pues la ira
del Señor se convirtió en misericordia. \bibleverse{6} Llegando sin
avisar, incendió ciudades y aldeas. Y al recuperar las posiciones más
importantes, poniendo en fuga a no pocos enemigos, \bibleverse{7}
aprovechó especialmente las noches para tales asaltos. En todas partes
se hablaba de su valor.

\bibleverse{8} Pero cuando Filipo vio que el hombre ganaba terreno poco
a poco, y que aumentaba cada vez más su éxito, escribió a Ptolomeo, el
gobernador de Coelesyria y Fenicia, para que apoyara la causa del rey.
\bibleverse{9} Ptolomeo no tardó en nombrar a Nicanor, hijo de Patroclo,
uno de los principales amigos del rey, y lo envió, al mando de no menos
de veinte mil personas de todas las naciones, a destruir toda la raza de
Judea. Con él se unió también Gorgias, un capitán y alguien que tenía
experiencia en asuntos de guerra. \bibleverse{10} Nicanor resolvió
compensar al rey, mediante la venta de los judíos cautivos, el tributo
de dos mil talentos que debía pagar a los romanos. \bibleverse{11}
Inmediatamente envió a las ciudades de la costa del mar, invitándolas a
comprar esclavos judíos, prometiendo entregar setenta esclavos por un
talento, sin esperar el juicio que le esperaba del Todopoderoso.

\bibleverse{12} A Judas le llegaron noticias sobre la invasión de
Nicanor. Cuando comunicó a los que estaban con él la presencia del
ejército, \bibleverse{13} los que eran cobardes y desconfiaban del
juicio de Dios huyeron y abandonaron el país. \bibleverse{14} Otros
vendieron todo lo que les quedaba, y al mismo tiempo imploraron al Señor
que liberara a los que habían sido vendidos como esclavos por el impío
Nicanor antes de que los conociera, \bibleverse{15} si no por su propio
bien, sí por los pactos hechos con sus antepasados, y porque los había
llamado por su santo y glorioso nombre. \bibleverse{16} Así pues,
Macabeo reunió a sus hombres, seis mil, y les exhortó a no dejarse
atemorizar por el enemigo, ni a temer a la gran multitud de paganos que
venían injustamente contra ellos, sino a luchar noblemente,
\bibleverse{17} poniendo ante sus ojos el ultraje que se había
perpetrado injustamente contra el lugar santo, y la tortura de la ciudad
que se había convertido en burla, y además el derrocamiento de la forma
de vida recibida de sus antepasados. \bibleverse{18} ``Porque ellos ---
dijo --- confían en sus armas y en sus actos audaces, pero nosotros
confiamos en el Dios todopoderoso, ya que él es capaz de derribar con un
gesto a los que vienen contra nosotros, e incluso al mundo entero.''
\bibleverse{19} Además, les contó la ayuda prestada de vez en cuando en
los días de sus antepasados, tanto en los días de Senaquerib, cuando
perecieron ciento ochenta y cinco mil, \bibleverse{20} como en la tierra
de Babilonia, en la batalla que se libró contra los galos de, cómo
llegaron a la batalla con ocho mil en total, con cuatro mil macedonios,
y cómo, estando los macedonios muy presionados, los seis mil destruyeron
a los ciento veinte mil a causa de la ayuda que tenían del cielo, y se
llevaron un gran botín.

\bibleverse{21} Y cuando con estas palabras los llenó de valor y los
dispuso a morir por las leyes y por su patria, dividió su ejército en
cuatro partes. \bibleverse{22} Nombró a sus hermanos Simón, José y
Jonatán como jefes de las divisiones con él, dándole a cada uno el mando
de mil quinientos hombres. \bibleverse{23} También Eleazer, habiendo
leído en voz alta el libro sagrado y habiendo dado como consigna ``LA
AYUDA DE DIOS'', encabezando él mismo la primera banda, se unió a la
batalla con Nicanor.

\bibleverse{24} Como el Todopoderoso luchó de su lado, mataron a más de
nueve mil enemigos, e hirieron y inutilizaron a la mayor parte del
ejército de Nicanor, y los obligaron a todos a huir. \bibleverse{25}
Tomaron el dinero de los que habían llegado allí para comprarlos como
esclavos. Después de haberlos perseguido a cierta distancia, regresaron,
obligados por la hora del día; \bibleverse{26} pues era la víspera del
sábado, y por esta razón no se esforzaron en perseguirlos lejos.
\bibleverse{27} Cuando reunieron las armas del enemigo y despojaron sus
despojos, guardaron el sábado, bendiciendo y dando gracias en gran
manera al Señor que los había salvado hasta hoy, porque había comenzado
a tener misericordia de ellos. \bibleverse{28} Después del sábado,
cuando dieron parte del botín a los mutilados de y a las viudas y
huérfanos, repartieron el resto entre ellos y sus hijos. \bibleverse{29}
Una vez realizadas estas cosas, y habiendo hecho una súplica común,
imploraron al Señor misericordioso que se reconciliara totalmente con
sus siervos.

\bibleverse{30} Habiendo tenido un encuentro con las fuerzas de Timoteo
y Báquides, mataron a más de veinte mil de ellos, y se hicieron dueños
de fortalezas muy altas, y repartieron mucho botín, dando a los
mutilados, huérfanos, viudas y ancianos una parte igual a la de ellos.
\bibleverse{31} Cuando reunieron las armas del enemigo, las almacenaron
todas cuidadosamente en los lugares más estratégicos, y llevaron el
resto del botín a Jerusalén. \bibleverse{32} Mataron al filárquico de
las fuerzas de Timoteo, un hombre muy impío y que había hecho mucho daño
a los judíos. \bibleverse{33} Mientras celebraban la fiesta de la
victoria en la ciudad de sus padres, quemaron a los que habían
incendiado las puertas sagradas, incluido Calístenes, que había huido a
una pequeña casa. Así recibieron la debida recompensa por su impiedad.

\bibleverse{34} El tres veces maldito Nicanor, que había traído a los
mil mercaderes para comprar a los judíos como esclavos, \bibleverse{35}
siendo por la ayuda del Señor humillado por los que a sus ojos eran
considerados de menor importancia, se quitó su gloriosa vestimenta, y
pasando por el país, rehuyendo toda compañía como un esclavo fugitivo,
llegó a Antioquía, habiendo, como él pensaba, tenido la mayor fortuna
posible, aunque su ejército fue destruido. \bibleverse{36} El que se
había encargado de asegurar el tributo a los romanos por el cautiverio
de los hombres de Jerusalén, publicó por todas partes que los judíos
tenían a Uno que luchaba por ellos, y que porque esto era así, los
judíos eran invulnerables, porque seguían las leyes ordenadas por él.

\hypertarget{section-8}{%
\section{9}\label{section-8}}

\bibleverse{1} Por aquel entonces, Antíoco se retiró desordenadamente de
la región de Persia. \bibleverse{2} Porque había entrado en la ciudad
llamada Persépolis, e intentó robar un templo y controlar la ciudad. Por
lo tanto, las multitudes se precipitaron y la gente del país se volcó
para defenderse con las armas; y sucedió que Antíoco fue puesto en fuga
por la gente del país y rompió su campamento con la desgracia.
\bibleverse{3} Mientras se encontraba en Ecbatana, le llegó la noticia
de lo ocurrido a Nicanor y a las fuerzas de Timoteo. \bibleverse{4}
Dominado por su cólera, planeó hacer sufrir a los judíos por las malas
acciones de los que lo habían puesto en fuga. Por lo tanto, con el
juicio del cielo que lo acompañaba, ordenó a su auriga que condujera sin
cesar hasta que completara el viaje; pues dijo con arrogancia lo
siguiente ``Haré de Jerusalén un cementerio común de judíos cuando
llegue allí''.

\bibleverse{5} Pero el Señor que todo lo ve, el Dios de Israel, lo
golpeó con un golpe mortal e invisible. En cuanto terminó de pronunciar
esta palabra, se apoderó de él un dolor incurable de las entrañas, con
amargos tormentos de las partes internas --- \bibleverse{6} y eso con
mucha justicia, pues había atormentado las entrañas de otros hombres con
muchos y extraños sufrimientos. \bibleverse{7} Pero él no cesó en
absoluto de su grosera insolencia. No, se llenó de más arrogancia aún,
exhalando fuego en su pasión contra los judíos, y dando órdenes de
acelerar el viaje. Pero sucedió además que se cayó de su carro mientras
éste se precipitaba, y al sufrir una grave caída fue torturado en todos
los miembros de su cuerpo. \bibleverse{8} El que acababa de suponer que
las olas del mar estaban a su disposición por su arrogancia sobrehumana,
y que pensaba sopesar las alturas de las montañas en una balanza, fue
ahora derribado y llevado en una litera, mostrando a todos que el poder
era evidentemente de Dios, \bibleverse{9} de modo que los gusanos
salieron del cuerpo del impío, y mientras aún vivía en la angustia y los
dolores, su carne se desprendió, y a causa del hedor todo el ejército se
apartó con repugnancia de su descomposición. \bibleverse{10} El hombre
que poco antes suponía tocar las estrellas del cielo, nadie podía
soportar llevarlo a causa de su intolerable hedor. \bibleverse{11} Por
lo tanto, comenzó en gran parte a dejar su arrogancia, quebrantada en su
espíritu, y a llegar al conocimiento bajo el azote de Dios, aumentando
sus dolores a cada momento. \bibleverse{12} Cuando él mismo no pudo
soportar su propio olor, dijo estas palabras: ``Es justo estar sujeto a
Dios, y que quien es mortal no se crea igual a Dios''.

\bibleverse{13} El vil hombre juró al soberano Señor, que ahora ya no se
apiadaría de él, diciendo \bibleverse{14} que la ciudad santa, a la que
se dirigía apresuradamente para ponerla a ras de suelo y convertirla en
un cementerio común, la declararía libre. \bibleverse{15} En cuanto a
los judíos, a quienes había decidido ni siquiera considerar dignos de
sepultura, sino arrojarlos a los animales con sus hijos para que los
devoraran las aves, los haría a todos iguales a los ciudadanos de
Atenas. \bibleverse{16} El santuario sagrado, que antes había saqueado,
lo adornaría con las mejores ofrendas, y restauraría todos los vasos
sagrados multiplicados, y con sus propios ingresos sufragaría los gastos
que exigían los sacrificios. \bibleverse{17} Además de todo esto, dijo
que se convertiría en judío y visitaría todo lugar habitado, proclamando
el poder de Dios. \bibleverse{18} Pero como sus sufrimientos no cesaban,
pues el juicio de Dios había caído sobre él en justicia, habiendo
abandonado toda esperanza para sí mismo, escribió a los judíos la carta
que se escribe a continuación, con carácter de súplica, a este efecto:

\bibleverse{19} ``A los dignos ciudadanos judíos, Antíoco, rey y
general, les desea mucha alegría, salud y prosperidad. \bibleverse{20}
Que os vaya bien a vosotros y a vuestros hijos, y que vuestros asuntos
sean como deseáis. Teniendo mi esperanza en el cielo, \bibleverse{21}
recordé con afecto vuestro honor y buena voluntad. Volviendo de la
región de Persia, y siendo presa de una molesta enfermedad, consideré
necesario pensar en la seguridad común de todos, \bibleverse{22} no
desesperando de mí mismo, sino teniendo gran esperanza de escapar de la
enfermedad. \bibleverse{23} Pero considerando que también mi padre, en
el momento en que condujo un ejército a la región superior, nombró a su
sucesor, \bibleverse{24} con el fin de que, si ocurría algo contrario a
lo esperado, o si se traía alguna noticia inoportuna, la gente del país,
sabiendo a quién se le había dejado el estado, no se preocupara,
\bibleverse{25} y, además, observando cómo los príncipes que están a lo
largo de las fronteras y vecinos de mi reino velan por las oportunidades
y esperan el acontecimiento futuro, he nombrado rey a mi hijo Antíoco, a
quien a menudo confié y encomendé a la mayoría de vosotros cuando me
apresuraba a las provincias superiores. Le he escrito lo que está
escrito a continuación. \bibleverse{26} Por tanto, os exhorto y os ruego
que, teniendo en cuenta los beneficios que se os han hecho en común y
por separado, conservéis vuestra buena voluntad actual, cada uno de
vosotros, hacia mí y hacia mi hijo. \bibleverse{27} Porque estoy
persuadido de que él, con gentileza y bondad, seguirá mi propósito y os
tratará con moderación y amabilidad.

\bibleverse{28} Así, el asesino y blasfemo, habiendo soportado los más
intensos sufrimientos, tal como había tratado a otros hombres, terminó
su vida entre las montañas con un destino muy lastimoso en una tierra
extraña. \bibleverse{29} Filipo, su hermano adoptivo, llevó el cadáver a
su casa y luego, temiendo al hijo de Antíoco, se retiró a Ptolomeo
Filometor en Egipto.

\hypertarget{section-9}{%
\section{10}\label{section-9}}

\bibleverse{1} Entonces Macabeo y los que estaban con él, guiados por el
Señor, recuperaron el templo y la ciudad. \bibleverse{2} Derribaron los
altares que los extranjeros habían construido en la plaza, así como los
recintos sagrados. \bibleverse{3} Después de limpiar el santuario,
hicieron otro altar de sacrificios. Golpeando el pedernal y encendiendo
el fuego, ofrecieron sacrificios después de haber cesado durante dos
años, quemaron incienso, encendieron lámparas y pusieron el pan de la
feria. \bibleverse{4} Una vez hechas estas cosas, se postraron e
imploraron al Señor que no volvieran a caer en tales males, sino que, si
alguna vez pecaban, fueran castigados por él con indulgencia, y no
fueran entregados a paganos blasfemos y bárbaros. \bibleverse{5} El
mismo día en que el santuario fue profanado por los extranjeros, ese
mismo día se limpió el santuario, el día veinticinco del mismo mes, que
es Chislev. \bibleverse{6} Celebraron ocho días con alegría a la manera
de la fiesta de los tabernáculos, recordando cómo no mucho antes,
durante la fiesta de los tabernáculos, andaban errantes por los montes y
en las cuevas como animales salvajes. \bibleverse{7} Llevando, pues,
varas de flores, ramas hermosas y hojas de palmera, elevaron himnos de
acción de gracias a aquel que había logrado la purificación de su propio
lugar. \bibleverse{8} También ordenaron con un estatuto y un decreto
públicos, para toda la nación de los judíos, que observaran estos días
cada año.

\bibleverse{9} Tales fueron los acontecimientos del fin de Antíoco, que
fue llamado Epífanes.

\bibleverse{10} Ahora declararemos lo que sucedió bajo
Antíoco\footnote{\textbf{10:10} decir, hijo de un buen padre.} Eupator,
que resultó ser hijo de aquel impío, y resumiremos los principales males
de las guerras. \bibleverse{11} Porque este hombre, cuando sucedió en el
reino, nombró a un tal Lisias como canciller y gobernador supremo de
Coelesiria y Fenicia. \bibleverse{12} Porque Ptolomeo, que se llamaba
Macrón, dando ejemplo de observar la justicia hacia los judíos a causa
del mal que se les había hecho, se esforzó por tratar con ellos en
términos pacíficos. \bibleverse{13} Entonces, siendo acusado por los
amigos del rey\footnote{\textbf{10:13} Ver 2 Macabeos 8:9} ante Eupator,
y oyendo que se le llamaba traidor en todo momento porque había
abandonado Chipre que le había confiado Filométor, y se había retirado a
Antíoco Epífanes, y\footnote{\textbf{10:13} El texto griego está
  corrupto.} faltando al honor de su cargo, tomó veneno y se suicidó.

\bibleverse{14} Pero cuando Gorgias fue nombrado gobernador del
distrito, mantuvo una fuerza de mercenarios, y en todo momento mantuvo
la guerra contra los judíos. \bibleverse{15} Junto con él, también los
idumeos, dueños de importantes fortalezas, hostigaban a los judíos; y
recibiendo a los que se habían refugiado de Jerusalén, se esforzaban por
mantener la guerra. \bibleverse{16} Pero Macabeo y sus hombres, habiendo
hecho una súplica solemne y habiendo implorado a Dios que luchara de su
parte, se abalanzaron sobre las fortalezas de los idumeos.
\bibleverse{17} Asaltándolas enérgicamente, se apoderaron de las
posiciones, impidieron el paso a todos los que luchaban en la muralla y
mataron a los que encontraron, matando no menos de veinte mil.

\bibleverse{18} Como no menos de nueve mil habían huido a dos torres muy
fuertes teniendo todo lo necesario para un asedio, \bibleverse{19}
Macabeo, habiendo dejado a Simón y a José, y también a Zaqueo y a los
que estaban con él, una fuerza suficiente para asediarlos, se marchó él
mismo a los lugares donde era más necesario. \bibleverse{20} Pero Simón
y los que estaban con él, cediendo a la codicia, fueron sobornados por
algunos de los que estaban en las torres, y recibiendo setenta mil
dracmas, dejaron escapar a algunos de ellos. \bibleverse{21} Pero cuando
se le informó a Maccabeo de lo que se había hecho, reunió a los líderes
del pueblo y acusó a esos hombres de haber vendido a sus parientes por
dinero, liberando a sus enemigos para que lucharan contra ellos.
\bibleverse{22} Así que mató a esos hombres por haberse convertido en
traidores, e inmediatamente tomó posesión de las dos torres.
\bibleverse{23} Prosperando con sus armas en todo lo que emprendía,
destruyó a más de veinte mil en las dos fortalezas.

\bibleverse{24} Ahora bien, Timoteo, que ya había sido derrotado por los
judíos, habiendo reunido fuerzas extranjeras en gran cantidad, y
habiendo reunido la caballería que pertenecía a Asia, no poca, vino como
si fuera a tomar Judea por la fuerza de las armas. \bibleverse{25} Pero
cuando se acercó, Macabeo y sus hombres se rociaron la cabeza con tierra
y se ciñeron el lomo con un saco, en señal de súplica a Dios,
\bibleverse{26} y, postrándose en el escalón frente al altar, le
imploraron que se hiciera\footnote{\textbf{10:26} Gr. propicio.}
clemente con ellos, y\footnote{\textbf{10:26} Ver Éxodo 23:22.} fuera
enemigo de sus enemigos y adversario de sus adversarios, como declara la
ley. \bibleverse{27} Levantándose de su oración, tomaron sus armas y
avanzaron a cierta distancia de la ciudad. Cuando se acercaron a sus
enemigos, se detuvieron en\footnote{\textbf{10:27} Gr. estaban solos.} .
\bibleverse{28} Al despuntar el alba, los dos ejércitos se unieron en la
batalla, teniendo los unos, además de la virtud, como prenda de éxito y
victoria, el haber huido al Señor para refugiarse, y los otros haciendo
de su pasión su líder en la lucha.

\bibleverse{29} Cuando la batalla se hizo fuerte, aparecieron desde el
cielo a sus adversarios cinco espléndidos hombres montados en caballos
con bridas de oro,\footnote{\textbf{10:29} Algunas autoridades leen y
  dirigen a los judíos, quienes también, tomando.} y dos de ellos,
dirigiendo a los judíos, \bibleverse{30} y tomando a Macabeo en medio de
ellos, y cubriéndolo con su propia armadura, lo protegieron de las
heridas, mientras disparaban flechas y rayos a los enemigos. Por esta
razón, fueron cegados y sumidos en la confusión, y quedaron destrozados,
llenos de desconcierto. \bibleverse{31} Veinte mil quinientos fueron
muertos, además de seiscientos de caballería.

\bibleverse{32} El mismo Timoteo huyó a una fortaleza llamada Gázara,
una fortaleza de gran fuerza,\footnote{\textbf{10:32} Ver ver. 37.}
donde Chaereas estaba al mando. \bibleverse{33} Entonces Macabeo y sus
hombres se alegraron y sitiaron la fortaleza durante cuatro días.
\bibleverse{34} Los que estaban dentro, confiando en la fuerza del
lugar, blasfemaban mucho y lanzaban palabras impías. \bibleverse{35}
Pero al amanecer del quinto día, algunos jóvenes de la compañía de
Maccabaeus, inflamados de cólera a causa de las blasfemias, asaltaron la
muralla con fuerza masculina y con\footnote{\textbf{10:35} Gr. pasión
  como la de los animales salvajes.} cólera furiosa, y derribaron a todo
el que se interpuso en su camino. \bibleverse{36} Otros subieron de la
misma manera, mientras los enemigos estaban distraídos con los que se
habían abierto paso dentro, prendieron fuego a las torres y encendieron
hogueras que quemaron vivos a los blasfemos, mientras que otros
rompieron las puertas y, tras dar entrada al resto de la banda, ocuparon
la ciudad. \bibleverse{37} Mataron a Timoteo, que estaba escondido en
una cisterna, y a su hermano Quereas, y a Apolófanes. \bibleverse{38}
Una vez realizadas estas acciones, bendijeron al Señor con himnos y
acciones de gracias, bendiciendo al que proporciona grandes beneficios a
Israel y le da la victoria.

\hypertarget{section-10}{%
\section{11}\label{section-10}}

\bibleverse{1} Al cabo de muy poco tiempo, Lisias, tutor, pariente y
canciller del rey, muy disgustado por lo sucedido, \bibleverse{2} reunió
unos ochenta mil soldados de infantería y toda su caballería y vino
contra los judíos, planeando hacer de la ciudad un hogar para los
griegos, \bibleverse{3} y cobrar tributo en el templo, como\footnote{\textbf{11:3}
  O, en todos los lugares sagrados de los paganos} en los demás lugares
sagrados de las naciones, y poner en venta el sumo sacerdocio cada año.
\bibleverse{4} No tuvo en cuenta el poder de Dios, sino que se envaneció
con sus diez mil soldados de infantería, sus miles de soldados de
caballería y sus ochenta elefantes. \bibleverse{5} Entrando en Judea y
acercándose a Betsurón, que era un lugar fuerte y estaba a unos cinco
estadios\footnote{\textbf{11:5} Un estadio era aproximadamente 189
  metros o 618 pies, por lo que 5 estadios eran algo menos de 1 km o
  algo más de media milla.} de Jerusalén, la presionó con fuerza.

\bibleverse{6} Cuando Macabeo y sus hombres se enteraron de que estaba
sitiando las fortalezas, ellos y todo el pueblo, con lamentos y
lágrimas, suplicaron al Señor que enviara un ángel bueno para salvar a
Israel. \bibleverse{7} El mismo Macabeo tomó las armas primero, y
exhortó a los demás a que se pusieran en peligro junto con él y ayudaran
a su parentela; y salieron con él de muy buena gana. \bibleverse{8}
Cuando estaban allí, cerca de Jerusalén, apareció a su cabeza un jinete
vestido de blanco, blandiendo\footnote{\textbf{11:8} Gr. una panoplia.}
armas de oro. \bibleverse{9} Todos juntos alabaron al Dios
misericordioso, y se fortalecieron aún más en su corazón, estando
dispuestos a\footnote{\textbf{11:9} Gr. herida.} asaltar no sólo a los
hombres, sino también a los animales más salvajes y a los muros de
hierro, \bibleverse{10} avanzaron en formación, teniendo al que está en
los cielos para luchar de su lado, porque el Señor tuvo misericordia de
ellos. \bibleverse{11} Lanzándose como leones contra el enemigo, mataron
a once mil soldados de infantería y a mil seiscientos de caballería, y
obligaron a huir a todos los demás. \bibleverse{12} La mayoría de ellos
escaparon heridos y desnudos. El propio Lisias también escapó con una
huida vergonzosa. \bibleverse{13} Pero como era un hombre no falto de
entendimiento, reflexionando sobre la derrota que le había sobrevenido,
y considerando que los hebreos no podían ser vencidos porque el Dios
Todopoderoso luchaba de su parte, envió de nuevo \bibleverse{14} y les
persuadió de que llegaran a un acuerdo con la condición de que se
reconocieran todos sus derechos, y\footnote{\textbf{11:14} El texto
  griego está corrupto.} prometió que también persuadiría al rey para
que se hiciera amigo suyo. \bibleverse{15} Macabeo consintió en todas
las condiciones que Lisias le propuso, cuidando el bien común; pues
todas las peticiones que Macabeo entregó por escrito a Lisias en
relación con los judíos, el rey las aceptó.

\bibleverse{16} La carta escrita a los judíos por Lisias era en este
sentido:

``Lisias a la gente\footnote{\textbf{11:16} Gr. multitudinario.} de los
judíos, saludos. \bibleverse{17} Juan y Absalón, que fueron enviados por
ti, habiendo entregado el documento que se escribe a continuación,
hicieron una petición sobre las cosas que en él se escriben.
\bibleverse{18} Por lo tanto, le declaré todo lo que era necesario
llevar ante el rey, y lo que era posible lo permitió. \bibleverse{19}
Si, pues, todos conserváis vuestra buena voluntad hacia el gobierno, yo
también me esforzaré en el futuro por contribuir a vuestro bien.
\bibleverse{20} Con respecto a esto, he dado orden en detalle, tanto a
estos hombres como a los que han sido enviados por mí, para que
consulten con vosotros. \bibleverse{21} Adiós. Escrito en el año ciento
cuarenta y ocho, el día veinticuatro del mes\footnote{\textbf{11:21}
  nombre de este mes no se encuentra en ninguna otra parte, y quizás
  esté corrupto.} Dioscorinthius''.

\bibleverse{22} La carta del rey contenía estas palabras:

``Rey Antíoco a su hermano Lisias, saludos. \bibleverse{23} Viendo que
nuestro padre pasó a los dioses teniendo el deseo de que los súbditos de
su reino\footnote{\textbf{11:23} O bien, no hay que inquietarse sino} no
sean perturbados y se dediquen al cuidado de sus propios asuntos,
\bibleverse{24} nosotros, habiendo oído que los judíos no consienten el
propósito de nuestro padre de convertirlos a las costumbres de los
griegos, sino que eligen más bien su propia manera de vivir, y pedimos
que se les permitan las costumbres de su ley --- \bibleverse{25}
eligiendo, por tanto, que también esta nación esté libre de disturbios,
determinamos que se les devuelva su templo, y que vivan según las
costumbres que había en los días de sus antepasados. \bibleverse{26} Por
lo tanto, harás bien en enviarles mensajeros y darles la mano derecha de
la amistad, para que, conociendo nuestro parecer, tengan buen corazón y
se ocupen con gusto de la dirección de sus propios asuntos.''

\bibleverse{27} Y para la nación, la carta del rey fue la siguiente:

``Rey Antíoco al senado de los judíos y a los demás judíos, saludos.
\bibleverse{28} Si todos ustedes están bien, es como lo deseamos.
Nosotros también gozamos de buena salud. \bibleverse{29} Menelao nos
informó de que tu deseo era volver a casa y seguir tus propios asuntos.
\bibleverse{30} Por lo tanto, los que partan de casa hasta el día
treinta de Xanthicus tendrán nuestra amistad\footnote{\textbf{11:30} Gr.
  mano derecha.} , con pleno permiso \bibleverse{31} de que los judíos
usen sus propios alimentos y observen sus propias leyes, igual que
antes. Ninguno de ellos será molestado en modo alguno por las cosas que
se han hecho por ignorancia. \bibleverse{32} También he enviado a
Menelao, para que os anime. \bibleverse{33} Adiós. Escrito en el año
ciento cuarenta y ocho, el día quince de Xanthicus''.

\bibleverse{34} Los romanos también les enviaron una carta con estas
palabras:

``Quinto Memmio y Tito Manio, embajadores de los romanos, al pueblo de
los judíos, saludos. \bibleverse{35} En cuanto a las cosas que os
concedió Lisias, el pariente del rey, también damos nuestro
consentimiento. \bibleverse{36} Pero en cuanto a las cosas que él juzgó
que debían remitirse al rey, enviad prontamente a alguien, después de
que las hayáis considerado, para que publiquemos los decretos que
convengan a vuestro caso; pues estamos de camino a Antioquía.
\bibleverse{37} Envía, pues, a alguien con prontitud, para que también
nosotros sepamos lo que piensas. \bibleverse{38} \footnote{\textbf{11:38}
  Gr. Gozar de buena salud.} Despedida. Escrito en el año ciento
cuarenta y ocho, el día quince de Xanthicus.

\hypertarget{section-11}{%
\section{12}\label{section-11}}

\bibleverse{1} Una vez hecho este acuerdo, Lisias partió hacia el rey, y
los judíos se dedicaron a sus labores agrícolas.

\bibleverse{2} Pero algunos de los gobernadores de los distritos,
Timoteo y Apolonio, hijo de Genneo, y también Jerónimo y Demofón, y
junto a ellos Nicanor, gobernador de Chipre, no les permitían disfrutar
de la tranquilidad y vivir en paz. \bibleverse{3} Los hombres de Jope
perpetraron esta gran impiedad: invitaron a los judíos que vivían entre
ellos a subir con sus esposas e hijos a las barcas que habían
proporcionado, como si no tuvieran mala voluntad hacia ellos.
\bibleverse{4} Cuando\footnote{\textbf{12:4} Gr. ellos también.} los
judíos,\footnote{\textbf{12:4} Gr. después.} confiando en el voto
público de la ciudad, aceptaron la invitación, como hombres deseosos de
vivir en paz y sin sospechar nada, los llevaron al mar y ahogaron a no
menos de doscientos de ellos. \bibleverse{5} Cuando Judas se enteró de
la crueldad cometida contra sus compatriotas, dando orden a los hombres
que estaban con él \bibleverse{6} e invocando a Dios, el justo Juez,
vino contra los asesinos de su parentela, e incendió el puerto por la
noche, quemó las barcas y pasó a cuchillo a los que habían huido de
allí. \bibleverse{7} Pero cuando se cerraron las puertas de la ciudad,
se retiró, con la intención de venir de nuevo a desarraigar a toda la
comunidad de los hombres de Jope. \bibleverse{8} Pero al enterarse de
que los hombres de Jamnia tenían la intención de hacer lo mismo con los
judíos que vivían entre ellos, \bibleverse{9} atacó a los jamnitas por
la noche, e incendió el puerto junto con la flota, de modo que el
resplandor de la luz se vio en Jerusalén, a doscientos cuarenta estadios
de distancia\footnote{\textbf{12:9} un furlong es de unos 201 metros o
  220 yardas, por lo que 240 furlongs son unos 48 km. o 30 millas} .

\bibleverse{10} Cuando se habían alejado nueve estadios\footnote{\textbf{12:10}
  un tramo es de unos 201 metros o 220 yardas, por lo que 9 tramos son
  unos 1,8 km. o 1 1/8 millas} de allí, al marchar contra Timoteo, lo
atacó un ejército de árabes, no menos de cinco mil de infantería y
quinientos de caballería. \bibleverse{11} Y cuando se libró una dura
batalla, y Judas y su compañía, con la ayuda de Dios, tuvieron buen
éxito, los nómadas, al ser vencidos, imploraron a Judas que les
concediera amistad, prometiendo darle ganado y ayudar a\footnote{\textbf{12:11}
  Gr. ellos.} su pueblo en todo lo demás. \bibleverse{12} Así que Judas,
pensando que en verdad serían provechosos en muchas cosas, accedió a
vivir en paz con ellos; y recibiendo las promesas de amistad se
marcharon a sus tiendas.

\bibleverse{13} También atacó cierta ciudad, fuerte y cercada con
terraplenes y murallas, y habitada por una multitud mixta de varias
naciones. Se llamaba Caspín. \bibleverse{14} Los que estaban dentro,
confiando en la fuerza de las murallas y en sus provisiones, se
comportaron con rudeza contra Judas y los que estaban con él,
injuriando, y además blasfemando y diciendo palabras impías.
\bibleverse{15} Pero Judas y su compañía, invocando al gran Soberano del
mundo, que sin carneros y con astutas máquinas de guerra derribó Jericó
en tiempos de Josué, se abalanzaron salvajemente contra la muralla.
\bibleverse{16} Habiendo tomado la ciudad por voluntad de Dios, hicieron
una matanza indecible, tanto que el lago contiguo, que tenía dos
estadios\footnote{\textbf{12:16} un tramo es de unos 201 metros o 220
  yardas, por lo que 2 tramos son unos 402 metros o 1/4 de milla} de
ancho, pareció llenarse con el diluvio de sangre.

\bibleverse{17} Cuando hubieron recorrido setecientos cincuenta
estadios\footnote{\textbf{12:17} un tramo es de unos 201 metros o 220
  yardas, por lo que 750 tramos son unos 151 km. o 94 millas} desde
allí, se dirigieron a Charax, a los judíos que se llaman\footnote{\textbf{12:17}
  hombres de Tob: véanse Jueces 11:3:2 Samuel 10:6, y compárese 1
  Macabeos 5:13.} Tubieni. \bibleverse{18} No encontraron a Timoteo en
ese distrito, pues para entonces se había marchado de él sin lograr
nada, pero había dejado una guarnición muy fuerte en un lugar.
\bibleverse{19} Pero Dosite y Sosípater, que eran capitanes a las
órdenes de Macabao, salieron y destruyeron a los que había dejado
Timoteo en la fortaleza, más de diez mil hombres. \bibleverse{20}
Macabeo, organizando su propio ejército en divisiones, puso\footnote{\textbf{12:20}
  Gr. ellos.} a estos dos sobre las bandas, y marchó a toda prisa contra
Timoteo, que tenía con él ciento veinte mil de infantería y dos mil
quinientos de caballería. \bibleverse{21} Cuando Timoteo se enteró de
que se acercaba Judas, enseguida envió a las mujeres y a los niños con
el equipaje a la fortaleza llamada\footnote{\textbf{12:21} Compárese con
  Carnain, 1 Macabeos 5:26:43, 44.} Carnion, pues el lugar era difícil
de sitiar y de acceder por la estrechez de los accesos por todos lados.
\bibleverse{22} Cuando la banda de Judas, que encabezaba la primera
división, apareció a la vista, y cuando el terror y el miedo se
apoderaron del enemigo, porque la manifestación de aquel que todo lo ve
se apoderó de ellos, huyeron en todas direcciones, llevados de un lado a
otro, de modo que a menudo eran heridos por sus propios hombres y
atravesados con las puntas de sus propias espadas. \bibleverse{23} Judas
continuó la persecución con más vigor, pasando a cuchillo a los
malvados, y destruyó hasta treinta mil hombres.

\bibleverse{24} El mismo Timoteo, al caer en la compañía de Dositeo y
Sosípater, les imploró con mucha astucia que le dejaran ir con su vida,
porque tenía en su poder a los padres de muchos de ellos y a la
parentela de algunos. \footnote{\textbf{12:24} Gr. y el resultado será
  que éstos no serán tenidos en cuenta. El texto griego aquí está quizás
  corrupto.} ``De lo contrario, dijo, poca consideración se tendrá con
estos''.\footnote{\textbf{12:24} Compárese con Carnain, 1 Macabeos
  5:26:43, 44.} \bibleverse{25} Así que, después de haber confirmado con
muchas palabras el acuerdo de restituirlos sin daño, lo dejaron ir para
poder salvar a su parentela.

\bibleverse{26} Entonces Judas, marchando contra Carnion y el templo de
Atergatis, mató a veinticinco mil personas. \bibleverse{27} Después de
haber puesto en fuga a éstos y de haberlos destruido, marchó también
contra Efrón, una ciudad fuerte,\footnote{\textbf{12:27} El texto griego
  aquí está quizás corrompido.} en la que había multitud de personas de
todas las naciones. Jóvenes robustos colocados\footnote{\textbf{12:27}
  Gr. frente a.} en las murallas hicieron una vigorosa defensa. Había
allí grandes reservas de máquinas de guerra y de dardos. \bibleverse{28}
\footnote{\textbf{12:28} O, sus enemigos} Pero invocando al Soberano que
con su poderío destroza la\footnote{\textbf{12:28} Algunas autoridades
  leen el peso.} fuerza del enemigo, tomaron la ciudad en sus manos y
mataron hasta veinticinco mil de los que estaban en ella.

\bibleverse{29} Partiendo de allí, marcharon a toda prisa contra
Escitópolis, que está a seiscientos estadios\footnote{\textbf{12:29} un
  tramo es de unos 201 metros o 220 yardas, por lo que 600 tramos son
  unos 121 km. o 75 millas} de Jerusalén. \bibleverse{30} Pero cuando
los judíos que estaban allí asentados dieron testimonio de la buena
voluntad que los escitopolitas habían mostrado hacia ellos, y del buen
trato que les habían dispensado en los momentos de su desgracia,
\bibleverse{31} les dieron las gracias, y además les exhortaron a seguir
teniendo buena disposición hacia la raza en el futuro. Luego subieron a
Jerusalén, estando próxima la fiesta de las semanas.

\bibleverse{32} Pero después de la fiesta llamada Pentecostés, marcharon
a toda prisa contra Gorgias, el gobernador de Idumea. \bibleverse{33}
Este salió con tres mil soldados de infantería y cuatrocientos de
caballería. \bibleverse{34} Cuando se pusieron en orden, sucedió que
cayeron algunos de los judíos. \bibleverse{35} Un tal Dositeo, uno de
los\footnote{\textbf{12:35} El texto griego es incierto.} de la compañía
de Bacenor, que iba a caballo y era un hombre fuerte, presionó
fuertemente a Gorgias y, agarrando su manto, lo arrastró con fuerza.
Mientras planeaba atrapar al maldito con vida, uno de los miembros de la
caballería tracia se abalanzó sobre él y le inutilizó el hombro, por lo
que Gorgias escapó a\footnote{\textbf{12:35} Compárese con 1 Macabeos
  5:65.} Marisa.

\bibleverse{36} Cuando los que estaban con Esdris llevaban mucho tiempo
luchando y estaban cansados, Judas invocó al Señor para que se mostrara,
luchando de su lado y dirigiendo la batalla. \bibleverse{37} Entonces,
en la lengua de sus antepasados, lanzó el grito de guerra unido a los
himnos. Luego se abalanzó contra las tropas de Gorgias, cuando éstas no
lo esperaban, y las puso en fuga.

\bibleverse{38} Judas reunió a su ejército y llegó a la ciudad
de\footnote{\textbf{12:38} Gr. Odollam.} Adulam. Como se acercaba el
séptimo día, se purificaron según la costumbre, y guardaron allí el
sábado.

\bibleverse{39} Al día siguiente,\footnote{\textbf{12:39} El texto
  griego aquí es incierto.} cuando fue necesario, Judas y su compañía
vinieron a recoger los cuerpos de los caídos,\footnote{\textbf{12:39} O
  bien, para hacerlos volver a estar con sus parientes en los sepulcros}
y en compañía de sus parientes para llevarlos a los sepulcros de sus
antepasados. \bibleverse{40} Pero bajo las vestiduras de cada uno de los
muertos encontraron\footnote{\textbf{12:40} Tal vez se trate de imágenes
  consagradas de los ídolos.} señales consagradas de los ídolos de
Jamnia, con los que la ley prohíbe a los judíos tener nada que ver. A
todos les quedó claro que era por esta causa que habían caído.
\bibleverse{41} Por lo tanto, todos, bendiciendo los caminos del Señor,
el Juez justo, que hace manifiestas las cosas ocultas, \bibleverse{42}
se volvieron a la súplica, rogando que el pecado cometido fuera
totalmente borrado. El noble Judas exhortó a la multitud a que se
guardara del pecado, pues había visto con sus propios ojos lo que
sucedía por el pecado de los que habían caído. \bibleverse{43} Cuando
hubo hecho una colecta hombre por hombre por la suma de dos mil dracmas
de plata, envió a Jerusalén a ofrecer un sacrificio por el pecado,
haciendo muy bien y honradamente en esto, en cuanto que pensó en la
resurrección. \bibleverse{44} Porque si no esperaba que los que habían
caído resucitaran, sería superfluo y ocioso orar por los muertos.
\bibleverse{45} Pero si esperaba un memorial honorable de gratitud
guardado para los que\footnote{\textbf{12:45} Gr. se duerme.}
mueren\footnote{\textbf{12:45} O, del lado de la piedad} en la piedad,
entonces el pensamiento era santo y piadoso. Por eso hizo el sacrificio
expiatorio por los que habían muerto, para que fueran liberados de su
pecado.

\hypertarget{section-12}{%
\section{13}\label{section-12}}

\bibleverse{1} En el año ciento cuarenta y nueve, se trajo a Judas y a
su compañía la noticia de que Antíoco Eupátor venía con multitudes
contra Judea, \bibleverse{2} y con él Lisias, su tutor y canciller, cada
uno con una fuerza griega de ciento diez mil infantes, cinco mil
trescientos de caballería, veintidós elefantes y trescientos carros
armados con guadañas.

\bibleverse{3} También Menelao se unió a ellos y, con gran hipocresía,
alentó a Antíoco, no para salvar a su país, sino porque pensaba que
sería puesto al frente del gobierno. \bibleverse{4} Pero el Rey de los
reyes despertó la ira de Antíoco contra el malvado. Cuando Lisias le
informó de que este hombre era el causante de todos los males, el rey
ordenó que lo llevaran a Berea y que lo mataran de la manera
acostumbrada en ese lugar. \bibleverse{5} En ese lugar hay una torre de
cincuenta codos de altura, llena de cenizas, y tenía a su alrededor un
borde circular\footnote{\textbf{13:5} Gr. sobre.} que se inclinaba por
todos lados hacia las cenizas. \bibleverse{6} Aquí se empuja a la
destrucción a quien es culpable de sacrilegio o notorio por otros
crímenes. \bibleverse{7} Por este destino sucedió que el infractor de la
ley, Menelao, murió sin obtener ni siquiera una tumba en la tierra, y
eso justamente; \bibleverse{8} pues como había perpetrado muchos pecados
contra el altar, cuyo fuego y cuyas cenizas eran sagrados, recibió su
muerte en cenizas.

\bibleverse{9} Ahora bien, el rey,\footnote{\textbf{13:9} Algunas
  autoridades leen indignado.} enfurecido de espíritu, venía con la
intención de infligir a los judíos los peores sufrimientos que se habían
hecho en tiempos de su padre. \bibleverse{10} Pero cuando Judas se
enteró de estas cosas, ordenó a la multitud que invocara al Señor de día
y de noche, si es que lo hacía en algún otro momento, para que ahora
ayudara a los que estaban a punto de ser privados de la ley, de su
patria y del templo sagrado, \bibleverse{11} y para que no permitiera
que el pueblo que acababa de empezar a revivir cayera en manos de
aquellos paganos profanos. \bibleverse{12} Así que cuando todos juntos
hicieron lo mismo,\footnote{\textbf{13:12} Gr. e imploró.} rogando al
Señor misericordioso con llantos, ayunos y postraciones durante tres
días sin cesar, Judas los exhortó y ordenó que se unieran a él.

\bibleverse{13} Habiendo consultado en privado con los ancianos,
resolvió que antes de que el ejército del rey entrara en Judea y se
hiciera dueño de la ciudad, salieran a decidir el asunto con la ayuda
de\footnote{\textbf{13:13} Algunas autoridades leen el Señor.} Dios.
\bibleverse{14} Y encomendando la decisión al\footnote{\textbf{13:14}
  Algunas autoridades leen Creador.} Señor del mundo, y exhortando a los
que estaban con él a contender noblemente hasta la muerte por las leyes,
el templo, la ciudad, el país y el modo de vida, acampó junto a Modín.
\bibleverse{15} Dio a sus hombres la consigna: ``LA VICTORIA ES DE
DIOS'', con una fuerza escogida de los más valientes jóvenes atacó de
noche junto al pabellón del rey, y mató de su ejército hasta dos mil
hombres, y\footnote{\textbf{13:15} El texto griego aquí es probablemente
  corrupto.} derribó sobre él al elefante principal que estaba en la
torre.\footnote{\textbf{13:15} Algunas autoridades leen una segunda vez.}
\bibleverse{16} Al final llenaron de terror y alarma al ejército de y
partieron con buen éxito. \bibleverse{17} Esto se logró cuando apenas
amanecía, gracias a la protección del Señor que dio ayuda a Judas.

\bibleverse{18} Pero el rey, habiendo probado la excesiva audacia de los
judíos, realizó ataques estratégicos contra sus posiciones,
\bibleverse{19} y contra una fuerte fortaleza de los judíos en Betsura.
Avanzó, fue rechazado, fracasó y fue derrotado. \bibleverse{20} Judas
envió lo necesario a los que estaban dentro. \bibleverse{21} Pero
Rodoco, de las filas judías, dio a conocer los secretos al enemigo. Lo
buscaron, lo arrestaron y lo encerraron en la cárcel. \bibleverse{22} El
rey negoció con ellos en Betsura por segunda vez, dio su mano, tomó la
de ellos, partió, atacó a las fuerzas de Judas, fue puesto en lo peor,
\bibleverse{23} oyó que Filipo, que había quedado como canciller en
Antioquía, se había vuelto imprudente, se confundió, hizo a los judíos
una proposición de paz, se sometió y juró reconocer todos sus derechos,
llegó a un acuerdo con ellos y ofreció sacrificios, honró el santuario y
el lugar, \bibleverse{24} mostró amabilidad y recibió amablemente a
Maccabaeus, dejó a Hegemónides como gobernador desde Ptolemais hasta los
gerenios, \bibleverse{25} y llegó a Ptolemais. Los hombres de Tolemaida
estaban disgustados por el tratado, pues estaban muy indignados con los
judíos. Deseaban anular los artículos del acuerdo. \bibleverse{26}
Lisias se adelantó a hablar, hizo la mejor defensa posible, persuadió,
pacificó, se ganó su buena voluntad y partió hacia Antioquía. Este fue
el asunto del ataque y la partida del rey.

\hypertarget{section-13}{%
\section{14}\label{section-13}}

\bibleverse{1} Tres años más tarde, llegó a Judas y a su compañía la
noticia de que Demetrio, hijo de Seleuco, había entrado en el puerto de
Trípoli con un poderoso ejército y una flota, \bibleverse{2} y se había
apoderado del país, habiendo hecho desaparecer a Antíoco y a su tutor
Lisias.

\bibleverse{3} Pero un tal Alcimo, que antes había sido sumo sacerdote y
se había contaminado voluntariamente en los tiempos en que no había
mezcla con los gentiles, considerando que no había liberación para él de
ninguna manera, ni más acceso al altar sagrado, \bibleverse{4} vino al
rey Demetrio alrededor del año ciento cincuenta y uno, presentándole una
corona de oro y una palma, y junto a éstas algunas de las ramas de olivo
festivas del templo. Durante ese día, guardó silencio; \bibleverse{5}
pero habiendo tenido la oportunidad de promover su propia locura, al ser
llamado por Demetrio a una reunión de su consejo, y al preguntarle cómo
estaban los judíos afectados y qué pretendían, respondió:

\bibleverse{6} ``Aquellos de los judíos llamados Hasidaeans, cuyo líder
es Judas Maccabaeus, mantienen la guerra y son sediciosos, no
permitiendo que el reino encuentre tranquilidad. \bibleverse{7} Por lo
tanto, habiendo dejado de lado mi gloria ancestral --- me refiero al
sumo sacerdocio --- he venido ahora aquí \bibleverse{8} primero por el
genuino cuidado que tengo por las cosas que conciernen al rey, y segundo
porque también tengo consideración por mis propios conciudadanos.
Porque, por culpa de la imprudencia de aquellos de los que he hablado
antes, toda nuestra raza se encuentra en una desgracia no pequeña.
\bibleverse{9} Oh rey, después de haberte informado de estas cosas,
piensa tanto en nuestro país como en nuestra raza, que está rodeada de
enemigos, según la bondadosa bondad con que recibes a todos.
\bibleverse{10} Porque mientras Judas siga vivo, es imposible que el
gobierno encuentre la paz. \bibleverse{11} Cuando hubo pronunciado estas
palabras, en seguida\footnote{\textbf{14:11} O también los amigos del
  rey} el resto de los amigos del rey\footnote{\textbf{14:11} Véase 2
  Macabeos 8:9.} , teniendo mala voluntad contra Judas, enardeció aún
más a Demetrio. \bibleverse{12} Inmediatamente nombró a Nicanor, que
había sido maestro de los elefantes, y lo hizo gobernador de Judea. Lo
envió, \bibleverse{13} dándole instrucciones por escrito para que matara
al propio Judas y dispersara a los que estaban con él, y para que
pusiera a Alcimo como sumo sacerdote del\footnote{\textbf{14:13} Gr.
  mayor.} gran templo. \bibleverse{14} Los que en Judea\footnote{\textbf{14:14}
  Véase 2 Macabeos 5:27.} habían expulsado a Judas al exilio acudieron
en tropel a Nicanor, suponiendo que las desgracias y calamidades de los
judíos serían éxitos para ellos.

\bibleverse{15} Pero cuando los judíos se enteraron del avance de
Nicanor y del asalto de los paganos, se rociaron la cabeza con tierra e
hicieron oraciones solemnes a aquel que había establecido a su propio
pueblo para siempre, y que siempre, manifestando su presencia, sostiene
a los que son su propia herencia. \bibleverse{16} \footnote{\textbf{14:16}
  El texto griego de este verso y del siguiente está corrupto.} Cuando
el jefe dio las órdenes, partió inmediatamente de allí y se unió a ellos
en una aldea llamada Lessau. \bibleverse{17} Pero Simón, el hermano de
Judas, había encontrado a Nicanor, aunque no hasta tarde, pues se había
retrasado a causa de la repentina consternación causada por sus
adversarios.

\bibleverse{18} Sin embargo, Nicanor, al oír el valor de los que estaban
con Judas, y su coraje al luchar por su país, rehuyó llevar el asunto a
la decisión de la espada. \bibleverse{19} Por lo tanto, envió a
Posidonio, Teodoto y Matatías para que dieran y recibieran promesas de
amistad. \bibleverse{20} Así pues, después de haber considerado
largamente estas propuestas, y de que el jefe las pusiera en
conocimiento de las tropas de\footnote{\textbf{14:20} O, gente Gr.
  multitudes.} , y de que pareciera que todos estaban de acuerdo,
consintieron en los pactos. \bibleverse{21} Designaron un día para
reunirse por su cuenta. Se presentó un carro de cada ejército. Colocaron
asientos de honor. \bibleverse{22} Judas dispuso hombres armados en
lugares convenientes, para que no hubiera de repente una traición por
parte del enemigo. Celebraron una conferencia como era de rigor.
\bibleverse{23} Nicanor esperó en Jerusalén y no hizo nada para causar
disturbios, sino que despidió a los rebaños de gente que se habían
reunido. \bibleverse{24} Mantenía a Judas siempre en su presencia. Se
había ganado un sincero afecto por el hombre. \bibleverse{25} Lo instó a
casarse y a tener hijos. Se casó, se instaló tranquilamente y participó
en la vida común.

\bibleverse{26} Pero Alcimo, percibiendo la buena voluntad que había
entre ellos,\footnote{\textbf{14:26} O bien, los pactos que se habían
  hecho, aprovecharon la ocasión y vinieron} y habiéndose apoderado de
los pactos que se habían hecho, vino a Demetrio y le dijo que Nicanor
era desleal al gobierno, pues había nombrado sucesor a ese conspirador
contra su reino, Judas. \bibleverse{27} El rey, enfurecido y exasperado
por las falsas acusaciones de aquel malvado, escribió a Nicanor
indicándole que estaba disgustado por los pactos y ordenándole que
enviara a toda prisa a Maccabaeus prisionero a Antioquía.
\bibleverse{28} Cuando este mensaje llegó a Nicanor, se confundió y se
turbó mucho al pensar en anular los artículos que se habían acordado, ya
que el hombre no había hecho ningún mal; \bibleverse{29} pero como no
había quien se opusiera al rey, vigiló su tiempo para ejecutar este
propósito mediante una estrategia. \bibleverse{30} Pero Macabeo, al
percibir que Nicanor se comportaba con más dureza en su trato y que se
había convertido en gobernante con su porte habitual, comprendiendo que
esta dureza no venía de bien, reunió a no pocos de sus hombres y se
ocultó de Nicanor.

\bibleverse{31} Pero el otro,\footnote{\textbf{14:31} O, aunque era
  consciente de que había sido derrotado noblemente por} cuando se dio
cuenta de que había sido derrotado valientemente por la estrategia de
Judas,\footnote{\textbf{14:31} Gr. el hombre} se acercó al
gran\footnote{\textbf{14:31} Gr. más grande.} y santo templo, mientras
los sacerdotes ofrecían los sacrificios habituales, y les ordenó que le
entregaran al hombre. \bibleverse{32} Cuando declararon con juramentos
que no tenían conocimiento de dónde estaba el hombre que buscaba,
\bibleverse{33} extendió su mano derecha hacia el santuario, e hizo este
juramento ``Si no me entregáis a Judas como prisionero, arrasaré
este\footnote{\textbf{14:33} O, capilla Gr. recinto.} templo de Dios
hasta el suelo, derribaré el altar y erigiré aquí un templo a Dionisio
para que todos lo vean.

\bibleverse{34} Y habiendo dicho esto, se marchó. Pero los sacerdotes,
extendiendo sus manos hacia el cielo, invocaron al que siempre lucha por
nuestra nación, con estas palabras \bibleverse{35} ``Tú, Señor del
universo, que en ti no tienes necesidad de nada, has querido que se
establezca entre nosotros un santuario de tu morada\footnote{\textbf{14:35}
  Gr. tabernáculo.} . \bibleverse{36} Así que ahora, Señor de toda
santidad, mantén impoluta para siempre esta casa que ha sido
recientemente limpiada.''

\bibleverse{37} Se informó a Nicanor sobre un tal Razis, anciano de
Jerusalén, amante de sus compatriotas y hombre de muy buena reputación,
al que llamaban Padre de los Judíos por su buena voluntad.
\bibleverse{38} Porque en los tiempos anteriores, cuando no había mezcla
con los gentiles, se le había acusado de seguir la religión de los
judíos, y había arriesgado su cuerpo y su vida con todo empeño por la
religión de los judíos. \bibleverse{39} Nicanor, queriendo hacer
evidente la mala voluntad que tenía contra los judíos, envió más de
quinientos soldados para apresarlo; \bibleverse{40} pues pensaba que
apresándolo les causaría un perjuicio. \bibleverse{41} Pero cuando las
tropas de\footnote{\textbf{14:41} O, el lugar vacío} estaban a punto de
tomar la torre, y forzaban la puerta del patio, y pedían fuego para
quemar las puertas, él, estando rodeado por todas partes, cayó sobre su
espada, \bibleverse{42} prefiriendo morir noblemente antes que caer en
manos de los malvados infelices, y sufrir un ultraje indigno de su
propia nobleza. \bibleverse{43} Pero como perdió el golpe por la
excitación de la lucha, y la multitud se precipitaba ahora dentro de la
puerta, corrió valientemente hasta el muro y se arrojó con valentía
entre la multitud. \bibleverse{44} Pero como ellos retrocedieron
rápidamente, se hizo un espacio, y él cayó en medio de su lado.
\bibleverse{45} Todavía con aliento, y encendido de ira, se levantó, y
aunque su sangre brotaba a borbotones y sus heridas eran graves, corrió
a través de la multitud, y de pie sobre una roca escarpada,
\bibleverse{46} como su sangre estaba ya bien gastada, sacó sus
intestinos a través de la herida, y tomándolos con ambas manos los
sacudió contra la multitud. Invocando a aquel que es el Señor de la vida
y del espíritu para que le devolviera\footnote{\textbf{14:46} Algunas
  autoridades leen lo mismo.} éstos de nuevo, murió así.

\hypertarget{section-14}{%
\section{15}\label{section-14}}

\bibleverse{1} Pero Nicanor, al oír que Judas y su compañía estaban en
la región de Samaria, resolvió atacarlos con toda seguridad en el día de
descanso. \bibleverse{2} Cuando los judíos que se vieron obligados a
seguirlo le dijeron: ``No destruyas tan salvaje y bárbaramente, sino da
la debida gloria al día que el que ve todas las cosas ha honrado y
santificado por encima de los demás días''.

\bibleverse{3} Entonces el infeliz tres veces maldito preguntó si había
un soberano en el cielo que hubiera ordenado guardar el día de reposo.

\bibleverse{4} Cuando declararon: ``Ahí está el Señor, viviendo él mismo
como Soberano en el cielo, que nos dijo que observáramos el séptimo
día''.

\bibleverse{5} Él respondió: ``Yo también soy un soberano en la tierra,
que te ordena tomar las armas y ejecutar los asuntos del rey''. Sin
embargo, no prevaleció para ejecutar su cruel plan.

\bibleverse{6} Y Nicanor,\footnote{\textbf{15:6} Gr. llevando el cuello
  en alto.} en su total jactancia y arrogancia, había decidido erigir un
monumento de completa victoria sobre Judas y todos los que estaban con
él. \bibleverse{7} Pero Macabeo seguía confiando incesantemente, con
toda la esperanza de obtener la ayuda del Señor. \bibleverse{8} Exhortó
a los suyos a que no tuvieran miedo ante el asalto de los paganos, sino
que, teniendo en cuenta la ayuda que en tiempos anteriores habían
recibido a menudo del cielo, esperaran también ahora la victoria que les
llegaría del Todopoderoso, \bibleverse{9} y alentándoles con la ley y
los profetas, y recordándoles las contiendas que habían ganado, los hizo
más ávidos. \bibleverse{10} Y cuando hubo despertado su valor, les dio
órdenes, señalando al mismo tiempo la falta de fe de los paganos y el
incumplimiento de sus juramentos. \bibleverse{11} Armando a cada uno de
ellos, no tanto con la segura defensa de escudos y lanzas como con el
estímulo de las buenas palabras, y además relatándoles un sueño digno de
ser creído, los alegró a todos en extremo.

\bibleverse{12} La visión de aquel sueño fue ésta: Onías, el que había
sido sumo sacerdote, un hombre noble y bueno, de porte modesto, pero de
modales amables y bien hablados, y formado desde niño en todos los
puntos de la virtud, con las manos extendidas invocando bendiciones
sobre todo el cuerpo de los judíos. \bibleverse{13} Entonces vio
aparecer a un hombre de edad venerable y de gran gloria, y la dignidad
que le rodeaba era maravillosa y muy majestuosa. \bibleverse{14} Onías
respondió y dijo: ``Este es el amante de la parentela, el que ora mucho
por el pueblo y la ciudad santa: Jeremías, el profeta de Dios.
\bibleverse{15} Jeremías extendió su mano derecha y entregó a Judas una
espada de oro, y al dársela se dirigió así \bibleverse{16} ``Toma esta
espada sagrada, regalo de Dios, con la que abatirás a los adversarios''.

\bibleverse{17} Alentados por las palabras de Judas, que eran nobles y
eficaces, y capaces de incitar a la virtud y de mover las almas de los
jóvenes a la valentía varonil, decidieron no llevar a cabo una campaña,
sino enfrentarse noblemente al enemigo, y luchando cuerpo a cuerpo con
todo el valor llevar el asunto a su conclusión, porque la ciudad, el
santuario y el templo estaban en peligro. \bibleverse{18} Porque su
temor por las esposas y los hijos, y además por la familia y los
parientes, era menos importante para ellos; pero lo más grande y lo
primero era su temor por el santuario consagrado. \bibleverse{19}
También los que estaban encerrados en la ciudad no tenían una angustia
ligera, pues estaban preocupados por el encuentro en campo abierto.

\bibleverse{20} Cuando todos esperaban la decisión de la cuestión, y el
enemigo ya se había incorporado a la batalla, y el ejército había sido
puesto en orden de batalla, y los elefantes\footnote{\textbf{15:20} o,
  supernumerario} llevados a un puesto conveniente, y la caballería
desplegada en los flancos, \bibleverse{21} Macabeo, percibiendo la
presencia de las tropas y las diversas armas con las que estaban
equipadas, y el salvajismo de los elefantes, levantando las manos al
cielo invocó al Señor que hace maravillas, sabiendo que el éxito no
viene por las armas, sino que, según como el Señor juzga, obtiene la
victoria para los que son dignos. \bibleverse{22} E invocando a Dios,
dijo esto ``Tú, Señor Soberano, enviaste a tu ángel en tiempos del rey
Ezequías de Judea, y él mató del ejército de Senaquerib hasta ciento
ochenta y cinco mil. \bibleverse{23} Así también ahora, oh soberano de
los cielos, envía un ángel bueno delante de nosotros para infundir
terror y temblor. \bibleverse{24} Por la grandeza de tu brazo, haz que
sean golpeados con espanto los que con blasfemia han venido aquí contra
tu santo pueblo''. Al terminar estas palabras, \bibleverse{25} Nicanor y
su compañía avanzaron con trompetas y cantos de victoria;
\bibleverse{26} pero Judas y su compañía se unieron a la batalla con el
enemigo con invocaciones y oraciones. \bibleverse{27} Luchando con las
manos y orando a Dios con el corazón, mataron no menos de treinta y
cinco mil hombres, alegrándose sobremanera por la manifestación de Dios.

\bibleverse{28} Cuando terminó el combate y regresaron con alegría,
reconocieron a Nicanor muerto y con su armadura completa.
\bibleverse{29} Entonces hubo gritos y ruido, y bendijeron al Señor
Soberano en la lengua de sus antepasados. \bibleverse{30} El que en todo
era en cuerpo y alma el principal campeón de sus conciudadanos, el que
mantuvo durante toda la vida la buena voluntad de su juventud hacia sus
compatriotas, ordenó que le cortaran la cabeza a Nicanor junto con la
mano y el brazo, y que los llevaran a Jerusalén. \bibleverse{31} Cuando
llegó allí y reunió a sus compatriotas y puso a los sacerdotes ante el
altar, mandó llamar a los que estaban en la ciudadela. \bibleverse{32}
Mostrando la cabeza del vil Nicanor y la mano de aquel profano, que con
altanería había extendido contra la santa casa del Todopoderoso,
\bibleverse{33} y cortando la lengua del impío Nicanor, dijo que la
daría en pedazos a los pájaros, y que colgaría estas recompensas de su
locura cerca del santuario. \bibleverse{34} Todos, mirando al cielo,
bendijeron al Señor que se había manifestado, diciendo: ``¡Bendito el
que ha conservado su propio lugar sin mancilla!'' \bibleverse{35} Colgó
de la ciudadela la cabeza y el hombro de Nicanor, señal evidente para
todos de la ayuda del Señor. \bibleverse{36} Todos ordenaron con un
decreto común no dejar pasar este día sin distinguirlo, sino marcar con
honor el día trece del mes doce (se llama Adar en el idioma sirio), el
día anterior al día de Mardoqueo.

\bibleverse{37} Habiendo sido éste el relato de la tentativa de Nicanor,
y habiendo estado la ciudad desde aquellos tiempos en poder de los
hebreos, también aquí pondré fin a mi libro. \bibleverse{38} Si he
escrito bien y al punto en mi relato, esto es lo que yo mismo deseaba;
pero si está mal hecho y es mediocre, esto es lo mejor que pude hacer.
\bibleverse{39} Porque como es desagradable beber vino solo y también
beber agua sola, mientras que la mezcla del vino con el agua da de una
vez todo el sabor agradable; así también la forma del lenguaje deleita
los oídos de los que leen la historia.

Aquí está el final.
