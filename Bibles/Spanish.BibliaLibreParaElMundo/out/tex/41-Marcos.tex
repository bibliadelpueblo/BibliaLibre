\hypertarget{apariciuxf3n-y-eficacia-de-juan-el-bautista}{%
\subsection{Aparición y eficacia de Juan el
Bautista}\label{apariciuxf3n-y-eficacia-de-juan-el-bautista}}

\hypertarget{section}{%
\section{1}\label{section}}

\bibleverse{1} El comienzo de la Buena Nueva de Jesucristo, el Hijo de
Dios.

\bibleverse{2} Como está escrito en los profetas, ``He aquí
que\footnote{\textbf{1:2} ``Contemplar'', de ``\greek{ἰδοὺ}'', significa
  mirar, fijarse, observar, ver o contemplar. Se utiliza a menudo como
  interjección.} envío a mi mensajero ante tu faz, que te preparará el
camino delante de ti: \footnote{\textbf{1:2} Malaquías 3:1} \footnote{\textbf{1:2}
  Mat 11,10} \bibleverse{3} la voz de uno que clama en el
desierto,`¡Preparen el camino del Señor! Endereza sus caminos''.
\footnote{\textbf{1:3} Isaías 40:3}

\bibleverse{4} Juan vino bautizando\footnote{\textbf{1:4} o, sumergiendo}
en el desierto y predicando el bautismo del arrepentimiento para el
perdón de los pecados. \bibleverse{5} Toda la región de Judea y todos
los de Jerusalén salieron a su encuentro. Fueron bautizados por él en el
río Jordán, confesando sus pecados. \bibleverse{6} Juan estaba vestido
con pelo de camello y un cinturón de cuero alrededor de la cintura.
Comía chapulines y miel silvestre. \bibleverse{7} Predicaba diciendo:
``Después de mí viene el que es más poderoso que yo, la correa de cuyas
sandalias no soy digno de agacharme y desatar. \bibleverse{8} Yo os he
bautizado en \footnote{\textbf{1:8} La palabra griega (en) traducida
  aquí como ``en'' podría traducirse también como ``con'' en algunos
  contextos.} agua, pero él os bautizará en el Espíritu Santo''.

\hypertarget{el-bautismo-y-la-tentaciuxf3n-de-jesuxfas}{%
\subsection{El bautismo y la tentación de
Jesús}\label{el-bautismo-y-la-tentaciuxf3n-de-jesuxfas}}

\bibleverse{9} En aquellos días, Jesús vino de Nazaret de Galilea y fue
bautizado por Juan en el Jordán. \footnote{\textbf{1:9} Luc 2,51}
\bibleverse{10} Al salir del agua, vio que los cielos se abrían y que el
Espíritu descendía sobre él como una paloma. \bibleverse{11} Una voz
salió del cielo: ``Tú eres mi Hijo amado, en quien me complazco''.
\footnote{\textbf{1:11} Mar 9,7}

\bibleverse{12} Inmediatamente, el Espíritu lo condujo al desierto.
\bibleverse{13} Estuvo allí en el desierto cuarenta días, tentado por
Satanás. Estaba con los animales salvajes, y los ángeles le servían.

\hypertarget{primera-apariciuxf3n-de-jesuxfas-en-galilea}{%
\subsection{Primera aparición de Jesús en
Galilea}\label{primera-apariciuxf3n-de-jesuxfas-en-galilea}}

\bibleverse{14} Después de que Juan fue detenido, Jesús vino a Galilea
predicando la Buena Nueva del Reino de Dios, \bibleverse{15} y diciendo:
``¡El tiempo se ha cumplido y el Reino de Dios está cerca! Arrepiéntanse
y crean en la Buena Nueva''. \footnote{\textbf{1:15} Gal 4,4}

\hypertarget{llamando-a-los-primeros-cuatro-discuxedpulos}{%
\subsection{Llamando a los primeros cuatro
discípulos}\label{llamando-a-los-primeros-cuatro-discuxedpulos}}

\bibleverse{16} Pasando junto al mar de Galilea, vio a Simón y a Andrés,
hermano de Simón, echando la red en el mar, pues eran pescadores.
\bibleverse{17} Jesús les dijo: ``Venid en pos de mí, y os haré
pescadores de hombres''.

\bibleverse{18} Inmediatamente dejaron las redes y le siguieron.

\bibleverse{19} Al alejarse un poco de allí, vio a Santiago, hijo de
Zebedeo, y a Juan, su hermano, que también estaban en la barca
remendando las redes. \bibleverse{20} Inmediatamente los llamó, y ellos
dejaron a su padre, Zebedeo, en la barca con los jornaleros, y fueron
tras él.

\hypertarget{el-primer-sermuxf3n-de-jesuxfas-y-la-curaciuxf3n-de-un-hombre-poseuxeddo-en-la-sinagoga-de-capernaum}{%
\subsection{El primer sermón de Jesús y la curación de un hombre poseído
en la sinagoga de
Capernaum}\label{el-primer-sermuxf3n-de-jesuxfas-y-la-curaciuxf3n-de-un-hombre-poseuxeddo-en-la-sinagoga-de-capernaum}}

\bibleverse{21} Fueron a Capernaúm, y en seguida, el día de reposo,
entró en la sinagoga y enseñó. \bibleverse{22} Se asombraban de su
enseñanza, porque les enseñaba como quien tiene autoridad, y no como los
escribas. \footnote{\textbf{1:22} Mat 7,28-29} \bibleverse{23} En
seguida se presentó en la sinagoga de ellos un hombre con un espíritu
impuro, que gritaba, \bibleverse{24} diciendo: ``¡Ja! ¿Qué tenemos que
ver contigo, Jesús, el nazareno? ¿Has venido a destruirnos? Yo sé quién
eres: el Santo de Dios''. \footnote{\textbf{1:24} Mar 5,7}

\bibleverse{25} Jesús le reprendió diciendo: ``¡Cállate y sal de él!''

\bibleverse{26} El espíritu inmundo, que lo convulsionaba y gritaba con
fuerza, salió de él. \footnote{\textbf{1:26} Mar 9,26} \bibleverse{27}
Todos estaban asombrados, y se preguntaban entre sí, diciendo: ``¿Qué es
esto? ¿Una nueva enseñanza? Porque con autoridad manda hasta a los
espíritus inmundos, y le obedecen''. \bibleverse{28} Inmediatamente se
difundió su fama por toda la región de Galilea y sus alrededores.

\hypertarget{sanaciuxf3n-de-la-suegra-de-simuxf3n-y-otros-enfermos-en-capernaum}{%
\subsection{Sanación de la suegra de Simón y otros enfermos en
Capernaum}\label{sanaciuxf3n-de-la-suegra-de-simuxf3n-y-otros-enfermos-en-capernaum}}

\bibleverse{29} En seguida, cuando salieron de la sinagoga, entraron en
casa de Simón y Andrés, con Santiago y Juan. \bibleverse{30} La madre de
la mujer de Simón estaba enferma de fiebre, y enseguida le hablaron de
ella. \bibleverse{31} Él se acercó, la tomó de la mano y la levantó. La
fiebre se le quitó enseguida, \footnote{\textbf{1:31} NU omite
  ``inmediatamente''.} y les sirvió.

\bibleverse{32} Al atardecer, cuando se puso el sol, le llevaron a todos
los enfermos y endemoniados. \bibleverse{33} Toda la ciudad estaba
reunida a la puerta. \bibleverse{34} El curó a muchos enfermos de
diversas enfermedades y expulsó a muchos demonios. No dejaba hablar a
los demonios, porque le conocían. \footnote{\textbf{1:34} Hech 16,17-18}

\hypertarget{jesuxfas-deja-capernaum-su-sermuxf3n-errante-y-actividad-curativa-en-galilea}{%
\subsection{Jesús deja Capernaum; su sermón errante y actividad curativa
en
Galilea}\label{jesuxfas-deja-capernaum-su-sermuxf3n-errante-y-actividad-curativa-en-galilea}}

\bibleverse{35} De madrugada, cuando aún estaba oscuro, se levantó y
salió, y se fue a un lugar desierto, y allí oró. \footnote{\textbf{1:35}
  Mat 14,23; Mat 26,36; Luc 5,16; Luc 11,1} \bibleverse{36} Simón y los
que estaban con él lo buscaron. \bibleverse{37} Lo encontraron y le
dijeron: ``Todos te buscan''.

\bibleverse{38} Les dijo: ``Vayamos a otra parte, a las ciudades
vecinas, para que predique también allí, porque he salido por este
motivo.'' \bibleverse{39} Y entró en las sinagogas de ellos por toda
Galilea, predicando y expulsando los demonios.

\hypertarget{jesuxfas-sana-a-un-leproso-y-escapa-a-la-soledad}{%
\subsection{Jesús sana a un leproso y escapa a la
soledad}\label{jesuxfas-sana-a-un-leproso-y-escapa-a-la-soledad}}

\bibleverse{40} Un leproso se acercó a él rogándole, arrodillándose ante
él y diciéndole: ``Si quieres, puedes limpiarme''.

\bibleverse{41} Conmovido por la compasión, extendió la mano, lo tocó y
le dijo: ``Quiero. Queda limpio''. \bibleverse{42} Al decir esto,
inmediatamente la lepra se apartó de él y quedó limpio. \bibleverse{43}
Lo amonestó estrictamente e inmediatamente lo envió fuera, \footnote{\textbf{1:43}
  Mar 3,12; Mar 7,36} \bibleverse{44} y le dijo: ``Mira que no digas
nada a nadie, sino ve a presentarte al sacerdote y ofrece por tu
limpieza lo que Moisés mandó, para que les sirva de testimonio.''
\footnote{\textbf{1:44} Lev 14,2-32}

\bibleverse{45} Pero él salió, y comenzó a proclamarlo mucho, y a
difundir el hecho, de modo que Jesús ya no podía entrar abiertamente en
una ciudad, sino que estaba fuera, en lugares desiertos. La gente acudía
a él de todas partes.

\hypertarget{curaciuxf3n-de-un-paraluxedtico-en-capernaum-jesuxfas-perdona-los-pecados}{%
\subsection{Curación de un paralítico en Capernaum; Jesús perdona los
pecados}\label{curaciuxf3n-de-un-paraluxedtico-en-capernaum-jesuxfas-perdona-los-pecados}}

\hypertarget{section-1}{%
\section{2}\label{section-1}}

\bibleverse{1} Cuando volvió a entrar en Capernaúm después de algunos
días, se oyó que estaba en casa. \bibleverse{2} Inmediatamente se
reunieron muchos, de modo que ya no cabían ni siquiera alrededor de la
puerta; y él les habló. \bibleverse{3} Se acercaron cuatro personas
llevando a un paralítico. \bibleverse{4} Como no podían acercarse a él
por la multitud, quitaron el techo donde estaba. Después de romperlo,
bajaron la estera en la que estaba acostado el paralítico.
\bibleverse{5} Jesús, al ver su fe, dijo al paralítico: ``Hijo, tus
pecados te son perdonados''.

\bibleverse{6} Pero había algunos de los escribas que estaban sentados y
razonaban en sus corazones: \bibleverse{7} ``¿Por qué este hombre dice
blasfemias así? ¿Quién puede perdonar los pecados sino sólo Dios?''
\footnote{\textbf{2:7} Sal 130,4; Is 43,25}

\bibleverse{8} En seguida Jesús, percibiendo en su espíritu que así
razonaban en su interior, les dijo: ``¿Por qué razonáis así en vuestros
corazones? \bibleverse{9} ¿Qué es más fácil, decir al paralítico ``Tus
pecados quedan perdonados'', o decirle: ``Levántate, toma tu cama y
anda''? \bibleverse{10} Pero para que sepáis que el Hijo del Hombre
tiene autoridad en la tierra para perdonar los pecados'' --- dijo al
paralítico --- \bibleverse{11} ``Te digo que te levantes, toma tu
camilla y vete a tu casa.''

\bibleverse{12} Se levantó, y en seguida tomó la estera y salió delante
de todos, de modo que todos se asombraron y glorificaron a Dios,
diciendo: ``¡Nunca vimos nada semejante!''

\hypertarget{llamando-al-recaudador-de-impuestos-levi-jesuxfas-como-compauxf1ero-de-mesa-para-recaudadores-de-impuestos-y-pecadores}{%
\subsection{Llamando al recaudador de impuestos Levi; Jesús como
compañero de mesa para recaudadores de impuestos y
pecadores}\label{llamando-al-recaudador-de-impuestos-levi-jesuxfas-como-compauxf1ero-de-mesa-para-recaudadores-de-impuestos-y-pecadores}}

\bibleverse{13} Volvió a salir a la orilla del mar. Toda la multitud se
acercaba a él, y él les enseñaba. \bibleverse{14} Al pasar, vio a Leví,
hijo de Alfeo, sentado en la oficina de impuestos. Le dijo: ``Sígueme''.
Y él se levantó y le siguió.

\bibleverse{15} Estaba sentado a la mesa en su casa, y muchos
recaudadores de impuestos y pecadores se sentaron con Jesús y sus
discípulos, pues eran muchos, y le seguían. \bibleverse{16} Los escribas
y los fariseos, al ver que comía con los pecadores y los recaudadores de
impuestos, dijeron a sus discípulos: ``¿Por qué come y bebe con los
recaudadores de impuestos y los pecadores?''

\bibleverse{17} Al oírlo, Jesús les dijo: ``Los sanos no tienen
necesidad de médico, sino los enfermos. No he venido a llamar a los
justos, sino a los pecadores al arrepentimiento''.

\hypertarget{la-pregunta-del-ayuno-de-los-discuxedpulos-de-juan-y-los-fariseos}{%
\subsection{La pregunta del ayuno de los discípulos de Juan y los
fariseos}\label{la-pregunta-del-ayuno-de-los-discuxedpulos-de-juan-y-los-fariseos}}

\bibleverse{18} Los discípulos de Juan y los fariseos estaban ayunando,
y se acercaron a preguntarle: ``¿Por qué los discípulos de Juan y los de
los fariseos ayunan, pero tus discípulos no ayunan?''

\bibleverse{19} Jesús les dijo: ``¿Pueden los padrinos ayunar mientras
el novio está con ellos? Mientras tengan al novio con ellos, no pueden
ayunar. \bibleverse{20} Pero vendrán días en que el novio les será
quitado, y entonces ayunarán en ese día. \bibleverse{21} Nadie cose un
trozo de tela sin remendar en una prenda vieja, porque si no el remiendo
se encoge y lo nuevo se desprende de lo viejo, y se hace un agujero
peor. \bibleverse{22} Nadie pone vino nuevo en odres viejos; de lo
contrario, el vino nuevo revienta los odres, y el vino se derrama, y los
odres se destruyen; pero ponen vino nuevo en odres nuevos.''

\hypertarget{el-arranco-de-espigas-de-los-discuxedpulos-en-suxe1bado-la-primera-disputa-de-jesuxfas-con-los-fariseos-sobre-la-santificaciuxf3n-del-duxeda-de-reposo}{%
\subsection{El arranco de espigas de los discípulos en sábado; La
primera disputa de Jesús con los fariseos sobre la santificación del día
de
reposo}\label{el-arranco-de-espigas-de-los-discuxedpulos-en-suxe1bado-la-primera-disputa-de-jesuxfas-con-los-fariseos-sobre-la-santificaciuxf3n-del-duxeda-de-reposo}}

\bibleverse{23} Iba el sábado por los campos de trigo, y sus discípulos
empezaron, mientras iban, a arrancar espigas. \bibleverse{24} Los
fariseos le dijeron: ``He aquí, ¿por qué hacen lo que no es lícito en el
día de reposo?''

\bibleverse{25} Les dijo ``¿Nunca leísteis lo que hizo David cuando tuvo
necesidad y hambre, él y los que estaban con él? \bibleverse{26} ¿Cómo
entró en la casa de Dios en el tiempo del sumo sacerdote Abiatar, y
comió el pan de la feria, que no es lícito comer sino a los sacerdotes,
y dio también a los que estaban con él?'' \footnote{\textbf{2:26} 1Sam
  21,6; Lev 24,9}

\bibleverse{27} Les dijo: ``El sábado fue hecho para el hombre, no el
hombre para el sábado. \footnote{\textbf{2:27} Deut 5,14}
\bibleverse{28} Por lotanto, el Hijo del Hombre es señor incluso del
sábado''.

\hypertarget{sanaciuxf3n-del-hombre-con-el-brazo-paralizado-en-suxe1bado-el-segundo-argumento-sobre-la-observancia-del-suxe1bado}{%
\subsection{Sanación del hombre con el brazo paralizado en sábado; el
segundo argumento sobre la observancia del
sábado}\label{sanaciuxf3n-del-hombre-con-el-brazo-paralizado-en-suxe1bado-el-segundo-argumento-sobre-la-observancia-del-suxe1bado}}

\hypertarget{section-2}{%
\section{3}\label{section-2}}

\bibleverse{1} Volvió a entrar en la sinagoga, y allí había un hombre
que tenía la mano seca. \bibleverse{2} Le vigilaban para ver si le
curaba en día de sábado, a fin de acusarle. \bibleverse{3} Dijo al
hombre que tenía la mano seca: ``Levántate''. \bibleverse{4} Les dijo:
``¿Es lícito en día de sábado hacer el bien o el mal? ¿Salvar una vida o
matar?'' Pero ellos guardaron silencio. \bibleverse{5} Cuando los miró
con ira, apenado por el endurecimiento de sus corazones, dijo al hombre:
``Extiende tu mano''. La extendió, y su mano quedó tan sana como la
otra. \bibleverse{6} Los fariseos salieron y enseguida conspiraron con
los herodianos contra él para destruirlo.

\hypertarget{afluencia-de-personas-muchas-curaciones-en-el-lago}{%
\subsection{Afluencia de personas; muchas curaciones en el
lago}\label{afluencia-de-personas-muchas-curaciones-en-el-lago}}

\bibleverse{7} Jesús se retiró al mar con sus discípulos; y le siguió
una gran multitud de Galilea, de Judea, \bibleverse{8} de Jerusalén, de
Idumea, del otro lado del Jordán, y los de los alrededores de Tiro y
Sidón. Una gran multitud, al oír las grandes cosas que hacía, se acercó
a él. \footnote{\textbf{3:8} Mat 4,25} \bibleverse{9} Él dijo a sus
discípulos que, a causa de la muchedumbre, le tuvieran cerca de él una
pequeña barca, para que no le presionaran. \bibleverse{10} Porque había
curado a muchos, de modo que todos los que tenían enfermedades le
apretaban para tocarle. \bibleverse{11} Los espíritus inmundos, al
verlo, se postraron ante él y gritaron: ``¡Tú eres el Hijo de Dios!''
\footnote{\textbf{3:11} Luc 4,41} \bibleverse{12} Él les advertía con
severidad que no debían darlo a conocer. \footnote{\textbf{3:12} Mar
  1,43}

\hypertarget{berufung-und-namen-der-zwuxf6lf-juxfcnger}{%
\subsection{Berufung und Namen der zwölf
Jünger}\label{berufung-und-namen-der-zwuxf6lf-juxfcnger}}

\bibleverse{13} Subió al monte y llamó a los que quería, y ellos fueron
a él. \bibleverse{14} Nombró a doce, para que estuvieran con él, y para
enviarlos a predicar \bibleverse{15} y a tener autoridad para sanar
enfermedades y expulsar demonios: \bibleverse{16} Simón (al que dio el
nombre de Pedro); \bibleverse{17} Santiago, hijo de Zebedeo; y Juan,
hermano de Santiago, (al que llamó Boanerges, que significa, Hijos del
Trueno); \footnote{\textbf{3:17} Luc 9,54} \bibleverse{18} Andrés;
Felipe; Bartolomé; Mateo; Tomás; Santiago, hijo de Alfeo; Tadeo; Simón
el Zelote; \bibleverse{19} y Judas Iscariote, que también lo traicionó.
Entonces entró en una casa.

\hypertarget{el-crecimiento-del-movimiento}{%
\subsection{El crecimiento del
movimiento}\label{el-crecimiento-del-movimiento}}

\bibleverse{20} La multitud se reunió de nuevo, de modo que no podían ni
comer pan. \bibleverse{21} Cuando lo oyeron sus amigos, salieron a
prenderlo, porque decían: ``Está loco''. \footnote{\textbf{3:21} Mar
  6,4; Juan 7,5; Juan 8,48}

\hypertarget{jesuxfas-se-defiende-de-la-blasfemia-de-beelzebul-de-los-escribas.-del-pecado-contra-el-espuxedritu-santo}{%
\subsection{Jesús se defiende de la blasfemia de Beelzebul de los
escribas. Del pecado contra el espíritu
santo}\label{jesuxfas-se-defiende-de-la-blasfemia-de-beelzebul-de-los-escribas.-del-pecado-contra-el-espuxedritu-santo}}

\bibleverse{22} Los escribas que bajaron de Jerusalén decían: ``Tiene a
Beelzebul'', y ``Por el príncipe de los demonios expulsa a los
demonios''. \footnote{\textbf{3:22} Mat 9,34}

\bibleverse{23} Los convocó y les dijo en parábolas: ``¿Cómo puede
Satanás expulsar a Satanás? \bibleverse{24} Si un reino está dividido
contra sí mismo, ese reino no puede permanecer. \bibleverse{25} Si una
casa está dividida contra sí misma, esa casa no puede permanecer.
\bibleverse{26} Si Satanás se ha levantado contra sí mismo y está
dividido, no puede mantenerse en pie, sino que tiene un fin.
\bibleverse{27} Pero nadie puede entrar en la casa del hombre fuerte
para saquear, si antes no ata al hombre fuerte; entonces saqueará su
casa.

\bibleverse{28} ``Ciertamente os digo que todos los pecados de los
descendientes del hombre serán perdonados, incluso las blasfemias con
las que puedan blasfemar; \bibleverse{29} pero el que blasfeme contra el
Espíritu Santo nunca tiene perdón, sino que está sujeto a la condenación
eterna.'' \footnote{\textbf{3:29} NU lee, culpable de un pecado eterno.}
\footnote{\textbf{3:29} Heb 6,4-6} \bibleverse{30} --- porque dijeron:
``Tiene un espíritu impuro''. \footnote{\textbf{3:30} Juan 10,20}

\hypertarget{los-verdaderos-parientes-de-jesuxfas}{%
\subsection{Los verdaderos parientes de
Jesús}\label{los-verdaderos-parientes-de-jesuxfas}}

\bibleverse{31} Llegaron su madre y sus hermanos y, estando fuera, le
mandaron llamar. \bibleverse{32} Una multitud estaba sentada a su
alrededor, y le dijeron: ``Mira, tu madre, tus hermanos y tus
hermanas\footnote{\textbf{3:32} TR omite ``sus hermanas''} están afuera
buscándote''.

\bibleverse{33} Él les respondió: ``¿Quiénes son mi madre y mis
hermanos?'' \bibleverse{34} Mirando a los que estaban sentados a su
alrededor, dijo: ``¡Mira, mi madre y mis hermanos! \bibleverse{35}
Porque todo el que hace la voluntad de Dios es mi hermano, mi hermana y
mi madre''.

\hypertarget{paruxe1bola-del-sembrador-y-cuatro-tipos-de-campos}{%
\subsection{Parábola del sembrador y cuatro tipos de
campos}\label{paruxe1bola-del-sembrador-y-cuatro-tipos-de-campos}}

\hypertarget{section-3}{%
\section{4}\label{section-3}}

\bibleverse{1} De nuevo se puso a enseñar a la orilla del mar. Se reunió
con él una gran multitud, de modo que entró en una barca en el mar y se
sentó. Toda la multitud estaba en tierra firme junto al mar.
\bibleverse{2} Les enseñaba muchas cosas en parábolas, y les decía en su
enseñanza: \bibleverse{3} ``¡Escuchad! He aquí que el agricultor salió a
sembrar. \bibleverse{4} Mientras sembraba, una parte de la semilla cayó
en el camino, y \footnote{\textbf{4:4} TR añade ``del aire''} vinieron
los pájaros y la devoraron. \bibleverse{5} Otras cayeron en el suelo
rocoso, donde tenía poca tierra, y enseguida brotaron, porque no tenían
profundidad de tierra. \bibleverse{6} Cuando salió el sol, se quemó; y
como no tenía raíz, se secó. \bibleverse{7} Otra cayó entre los espinos,
y los espinos crecieron y la ahogaron, y no dio fruto. \bibleverse{8}
Otras cayeron en buena tierra y dieron fruto, creciendo y aumentando.
Algunos produjeron treinta veces, otros sesenta veces y otros cien veces
más''. \bibleverse{9} Dijo: ``El que tenga oídos para oír, que oiga''.

\hypertarget{analice-el-significado-y-el-propuxf3sito-de-las-paruxe1bolas}{%
\subsection{Analice el significado y el propósito de las
parábolas}\label{analice-el-significado-y-el-propuxf3sito-de-las-paruxe1bolas}}

\bibleverse{10} Cuando se quedó solo, los que estaban a su alrededor con
los doce le preguntaron por las parábolas. \bibleverse{11} Él les dijo:
``A vosotros se os ha dado el misterio del Reino de Dios, pero a los que
están fuera, todas las cosas se hacen en parábolas, \bibleverse{12} para
que ``viendo vean y no perciban, y oyendo, no entiendan, no sea que se
vuelvan y se les perdonen los pecados.''\footnote{\textbf{4:12} Isaías
  6:9-10} \footnote{\textbf{4:12} Is 6,9-10}

\bibleverse{13} Les dijo: ``¿No entendéis esta parábola? ¿Cómo vais a
entender todas las parábolas?

\hypertarget{interpretaciuxf3n-de-la-paruxe1bola-del-sembrador}{%
\subsection{Interpretación de la parábola del
sembrador}\label{interpretaciuxf3n-de-la-paruxe1bola-del-sembrador}}

\bibleverse{14} El agricultor siembra la palabra. \bibleverse{15} Los
que están junto al camino son aquellos en los que se siembra la palabra;
y cuando han oído, enseguida viene Satanás y les quita la palabra que se
ha sembrado en ellos. \bibleverse{16} Estos, de la misma manera, son los
que están sembrados en los pedregales, los cuales, cuando han oído la
palabra, inmediatamente la reciben con alegría. \bibleverse{17} No
tienen raíz en sí mismos, sino que duran poco. Cuando surge la opresión
o la persecución a causa de la palabra, enseguida tropiezan.
\bibleverse{18} Otros son los que están sembrados entre las espinas.
Estos son los que han oído la palabra, \bibleverse{19} y los afanes de
este siglo, y el engaño de las riquezas, y los deseos de otras cosas que
entran, ahogan la palabra, y se hace infructuosa. \bibleverse{20} Los
que fueron sembrados en buena tierra son los que oyen la palabra, la
aceptan y dan fruto, unos treinta veces, otros sesenta y otros cien.''

\bibleverse{21} Les dijo: ``¿Acaso se trae una lámpara para ponerla
debajo de un cesto o\footnote{\textbf{4:21} literalmente, un modión, una
  cesta de medición seca que contiene aproximadamente un pico (unos 9
  litros)} de una cama? ¿No se pone sobre un candelero? \footnote{\textbf{4:21}
  Mat 5,15} \bibleverse{22} Porque no hay nada oculto si no es para que
se conozca, ni se ha hecho nada secreto si no es para que salga a la
luz. \footnote{\textbf{4:22} Mat 10,26-27; Luc 12,2} \bibleverse{23} El
que tenga oídos para oír, que oiga''.

\bibleverse{24} Les dijo: ``Prestad atención a lo que oís. Con cualquier
medida que midáis, se os medirá; y se os dará más a los que oís.
\footnote{\textbf{4:24} Mat 7,2} \bibleverse{25} Porque al que tiene, se
le dará más; y al que no tiene, se le quitará hasta lo que tiene.''
\footnote{\textbf{4:25} Mat 13,12-13}

\hypertarget{paruxe1bolas-de-la-semilla-que-crece-tranquilamente-por-suxed-misma-y-de-la-semilla-de-mostaza}{%
\subsection{Parábolas de la semilla que crece tranquilamente por sí
misma y de la semilla de
mostaza}\label{paruxe1bolas-de-la-semilla-que-crece-tranquilamente-por-suxed-misma-y-de-la-semilla-de-mostaza}}

\bibleverse{26} Dijo ``El Reino de Dios es como si un hombre echara la
semilla en la tierra, \bibleverse{27} y durmiera y se levantara de noche
y de día, y la semilla brotara y creciera, aunque no supiera cómo.
\footnote{\textbf{4:27} Sant 5,7} \bibleverse{28} Porque la tierra da
fruto por sí misma: primero la hoja, luego la espiga, después el grano
completo en la espiga. \bibleverse{29} Pero cuando el fruto está maduro,
enseguida se mete la hoz, porque ha llegado la cosecha.''

\bibleverse{30} Dijo: ``¿Cómo compararemos el Reino de Dios? ¿O con qué
parábola lo ilustraremos? \bibleverse{31} Es como un grano de mostaza,
que, cuando se siembra en la tierra, aunque es menor que todas las
semillas que hay en la tierra, \bibleverse{32} sin embargo, cuando se
siembra, crece y se hace más grande que todas las hierbas, y echa
grandes ramas, de modo que las aves del cielo pueden alojarse bajo su
sombra.''

\bibleverse{33} Con muchas parábolas de este tipo les hablaba la
palabra, según podían oírla. \bibleverse{34} Sin parábola no les
hablaba, sino que en privado a sus propios discípulos les explicaba
todo.

\hypertarget{jesuxfas-apacigua-la-tormenta-del-mar}{%
\subsection{Jesús apacigua la tormenta del
mar}\label{jesuxfas-apacigua-la-tormenta-del-mar}}

\bibleverse{35} Aquel día, al atardecer, les dijo: ``Pasemos a la otra
orilla''. \bibleverse{36} Dejando a la multitud, lo llevaron con ellos,
tal como estaba, en la barca. También iban con él otras barcas pequeñas.
\bibleverse{37} Se levantó una gran tormenta de viento, y las olas
golpeaban la barca, tanto que ésta ya estaba llena. \bibleverse{38} Él
mismo estaba en la popa, dormido sobre el cojín; y le despertaron y le
preguntaron: ``Maestro, ¿no te importa que nos estemos muriendo?''

\bibleverse{39} Se despertó y reprendió al viento, y dijo al mar:
``¡Paz! Quédate quieto!'' El viento cesó y se produjo una gran calma.
\bibleverse{40} Les dijo: ``¿Por qué tenéis tanto miedo? ¿Cómo es que no
tenéis fe?''

\bibleverse{41} Se asustaron mucho y se dijeron unos a otros: ``¿Quién
es, pues, éste, que hasta el viento y el mar le obedecen?''

\hypertarget{jesuxfas-sana-a-los-poseuxeddos-en-la-tierra-de-los-gerasenos}{%
\subsection{Jesús sana a los poseídos en la tierra de los
gerasenos}\label{jesuxfas-sana-a-los-poseuxeddos-en-la-tierra-de-los-gerasenos}}

\hypertarget{section-4}{%
\section{5}\label{section-4}}

\bibleverse{1} Llegaron al otro lado del mar, al región de los
gadarenos. \bibleverse{2} Cuando bajó de la barca, enseguida le salió al
encuentro un hombre con un espíritu impuro que salía de los sepulcros.
\bibleverse{3} Vivía en los sepulcros. Ya nadie podía atarlo, ni
siquiera con cadenas, \bibleverse{4} porque muchas veces había sido
atado con grilletes y cadenas, y las cadenas habían sido destrozadas por
él, y los grilletes hechos pedazos. Nadie tenía la fuerza para domarlo.
\bibleverse{5} Siempre, de noche y de día, en los sepulcros y en los
montes, gritaba y se cortaba con piedras. \bibleverse{6} Cuando vio a
Jesús de lejos, corrió y se postró ante él, \bibleverse{7} y gritando a
gran voz, dijo: ``¿Qué tengo que ver contigo, Jesús, Hijo del Dios
Altísimo? Te conjuro por Dios, no me atormentes''. \footnote{\textbf{5:7}
  Mar 1,24} \bibleverse{8} Pues le dijo: ``¡Sal del hombre, espíritu
inmundo!''

\bibleverse{9} Le preguntó: ``¿Cómo te llamas?''. Le dijo: ``Me llamo
Legión, porque somos muchos''. \bibleverse{10} Le rogó mucho que no los
echara del región. \bibleverse{11} En la ladera del monte había una gran
piara de cerdos alimentándose. \bibleverse{12} Todos los demonios le
rogaron, diciendo: ``Envíanos a los cerdos, para que entremos en
ellos''.

\bibleverse{13} En seguida Jesús les dio permiso. Los espíritus inmundos
salieron y entraron en los cerdos. La piara, de unos dos mil ejemplares,
se precipitó al mar por la empinada orilla, y se ahogaron en el mar.
\bibleverse{14} Los que alimentaban a los cerdos huyeron y lo contaron
en la ciudad y en el campo. La gente vino a ver qué era lo que había
sucedido. \bibleverse{15} Se acercaron a Jesús y vieron al endemoniado
sentado, vestido y en su sano juicio, al que tenía la legión, y se
asustaron. \bibleverse{16} Los que lo vieron les contaron lo que le
había sucedido al endemoniado y lo de los cerdos. \bibleverse{17}
Comenzaron a rogarle que se fuera de su región.

\bibleverse{18} Cuando entraba en la barca, el que había sido poseído
por los demonios le rogó que lo dejara ir con él. \bibleverse{19} No se
lo permitió, sino que le dijo: ``Vete a tu casa, a tus amigos, y
cuéntales las grandes cosas que el Señor ha hecho por ti y cómo ha
tenido misericordia de ti.''

\bibleverse{20} Se puso en camino y comenzó a proclamar en Decápolis
cómo Jesús había hecho grandes cosas por él, y todos se maravillaban.
\footnote{\textbf{5:20} Mar 7,31}

\hypertarget{jesuxfas-sana-a-la-mujer-ensangrentada-en-capernaum-y-despierta-a-la-hija-de-jairo}{%
\subsection{Jesús sana a la mujer ensangrentada en Capernaum y despierta
a la hija de
Jairo}\label{jesuxfas-sana-a-la-mujer-ensangrentada-en-capernaum-y-despierta-a-la-hija-de-jairo}}

\bibleverse{21} Cuando Jesús volvió a pasar en la barca a la otra
orilla, se reunió con él una gran multitud; y estaba junto al mar.
\bibleverse{22} He aquí que vino uno de los jefes de la sinagoga,
llamado Jairo, y viéndole, se echó a sus pies \bibleverse{23} y le rogó
mucho, diciendo: ``Mi hijita está a punto de morir. Te ruego que vengas
y pongas tus manos sobre ella, para que quede sana y viva''.

\bibleverse{24} Se fue con él, y le seguía una gran multitud que le
apretaba por todas partes. \bibleverse{25} Una mujer que tenía flujo de
sangre desde hacía doce años, \bibleverse{26} y que había padecido
muchas cosas por parte de muchos médicos, y que había gastado todo lo
que tenía, y no mejoraba, sino que empeoraba, \bibleverse{27} habiendo
oído las cosas que se referían a Jesús, se acercó por detrás de él entre
la multitud y tocó sus vestidos. \bibleverse{28} Porque decía: ``Con
sólo tocar sus vestidos, quedaré sana''. \bibleverse{29} Al instante se
le secó el flujo de sangre, y sintió en su cuerpo que estaba curada de
su aflicción.

\bibleverse{30} En seguida, Jesús, percibiendo en sí mismo que el poder
había salido de él, se volvió entre la multitud y preguntó: ``¿Quién ha
tocado mis vestidos?'' \footnote{\textbf{5:30} Luc 6,19}

\bibleverse{31} Sus discípulos le dijeron: ``Ves que la multitud te
aprieta, y dices: ``¿Quién me ha tocado?''

\bibleverse{32} Él miró a su alrededor para ver quién había hecho esto.
\bibleverse{33} Pero la mujer, temerosa y temblorosa, sabiendo lo que le
habían hecho, vino y se postró ante él y le contó toda la verdad.

\bibleverse{34} Él le dijo: ``Hija, tu fe te ha curado. Ve en paz y
cúrate de tu enfermedad''.

\bibleverse{35} Mientras aún hablaba, vino gente de la casa del jefe de
la sinagoga, diciendo: ``Tu hija ha muerto. ¿Para qué molestar más al
Maestro?''

\bibleverse{36} Pero Jesús, al oír el mensaje pronunciado, dijo
inmediatamente al jefe de la sinagoga: ``No tengas miedo, sólo cree''.
\bibleverse{37} No permitió que nadie le siguiera, sino Pedro, Santiago
y Juan, el hermano de Santiago. \footnote{\textbf{5:37} Mat 17,1}
\bibleverse{38} Llegó a la casa del jefe de la sinagoga, y vio un
alboroto, llantos y grandes lamentos. \bibleverse{39} Cuando entró, les
dijo: ``¿Por qué alborotáis y lloráis? La niña no está muerta, sino que
duerme''. \footnote{\textbf{5:39} Juan 11,11}

\bibleverse{40} Se burlaron de él. Pero él, después de echarlos a todos,
tomó al padre de la niña, a su madre y a los que estaban con él, y entró
donde estaba la niña. \bibleverse{41} Tomando a la niña de la mano, le
dijo: ``¡Talitha cumi!'', que significa, interpretándose, ``Muchacha, te
digo, levántate''. \footnote{\textbf{5:41} Luc 7,14; Hech 9,40}
\bibleverse{42} Inmediatamente la niña se levantó y caminó, pues tenía
doce años. Quedaron asombrados con gran asombro. \bibleverse{43} Les
ordenó estrictamente que nadie lo supiera, y mandó que le dieran algo de
comer.

\hypertarget{rechazo-y-fracaso-de-jesuxfas-en-su-natal-nazaret}{%
\subsection{Rechazo y fracaso de Jesús en su natal
Nazaret}\label{rechazo-y-fracaso-de-jesuxfas-en-su-natal-nazaret}}

\hypertarget{section-5}{%
\section{6}\label{section-5}}

\bibleverse{1} Salió de allí. Vino a su tierra, y sus discípulos le
siguieron. \bibleverse{2} Cuando llegó el sábado, se puso a enseñar en
la sinagoga, y muchos que le oían se asombraban, diciendo: ``¿De dónde
ha sacado éste estas cosas?'' y ``¿Qué sabiduría se le ha dado a éste,
para que por sus manos se realicen obras tan grandes? \footnote{\textbf{6:2}
  Juan 7,15} \bibleverse{3} ¿No es éste el carpintero, hijo de María y
hermano de Santiago, José, Judá y Simón? ¿No están sus hermanas aquí con
nosotros?'' Así que se ofendieron con él. \footnote{\textbf{6:3} Juan
  6,42}

\bibleverse{4} Jesús les dijo: ``Un profeta no carece de honor, sino en
su propio país, entre sus parientes y en su propia casa.''
\bibleverse{5} No pudo hacer allí ninguna obra poderosa, salvo que
impuso las manos sobre algunos enfermos y los sanó. \bibleverse{6} Se
asombraba de la incredulidad de ellos. Recorría las aldeas enseñando.

\hypertarget{enviar-e-instruir-a-los-doce-discuxedpulos}{%
\subsection{Enviar e instruir a los doce
discípulos}\label{enviar-e-instruir-a-los-doce-discuxedpulos}}

\bibleverse{7} Llamó a los doce, y comenzó a enviarlos de dos en dos; y
les dio autoridad sobre los espíritus inmundos. \footnote{\textbf{6:7}
  Luc 10,1} \bibleverse{8} Les ordenó que no llevaran nada para el
camino, sino sólo un bastón: ni pan, ni cartera, ni dinero en la bolsa,
\bibleverse{9} sino que llevaran sandalias y no se pusieran dos túnicas.
\bibleverse{10} Les dijo: ``Dondequiera que entréis en una casa, quedaos
allí hasta que salgáis de ella. \bibleverse{11} A quien no os reciba ni
os escuche, cuando salgáis de allí, sacudid el polvo que está bajo
vuestros pies como testimonio contra él. Os aseguro que el día del
juicio será más tolerable para Sodoma y Gomorra que para esa ciudad''.

\bibleverse{12} Salieron y predicaron que la gente debía arrepentirse.
\bibleverse{13} Expulsaron a muchos demonios y ungieron con aceite a
muchos enfermos y los sanaron. \footnote{\textbf{6:13} Sant 5,14; Sant
  1,5-15}

\hypertarget{el-juicio-de-herodes-sobre-jesuxfas-el-fin-de-juan-el-bautista}{%
\subsection{El juicio de Herodes sobre Jesús; el fin de Juan el
Bautista}\label{el-juicio-de-herodes-sobre-jesuxfas-el-fin-de-juan-el-bautista}}

\bibleverse{14} El rey Herodes oyó esto, pues su nombre se había hecho
conocido, y dijo: ``Juan el Bautista ha resucitado de entre los muertos,
y por eso actúan en él estos poderes.'' \bibleverse{15} Pero otros
decían: ``Es Elías''. Otros decían: ``Es un profeta, o como uno de los
profetas''. \bibleverse{16} Pero Herodes, al oír esto, dijo: ``Este es
Juan, a quien yo decapité. Ha resucitado de entre los muertos''.
\bibleverse{17} Porque el mismo Herodes había enviado y arrestado a Juan
y lo había encerrado en la cárcel por causa de Herodías, la mujer de su
hermano Felipe, pues se había casado con ella. \bibleverse{18} Porque
Juan había dicho a Herodes: ``No te es lícito tener la mujer de tu
hermano.'' \footnote{\textbf{6:18} Lev 18,16} \bibleverse{19} Herodías
se puso en contra de él y deseaba matarlo, pero no pudo, \bibleverse{20}
porque Herodes temía a Juan, sabiendo que era un hombre justo y santo, y
lo mantenía a salvo. Cuando lo escuchó, hizo muchas cosas, y lo escuchó
con gusto.

\bibleverse{21} Llegó un día oportuno en que Herodes, en su cumpleaños,
hizo una cena para sus nobles, los altos funcionarios y los principales
hombres de Galilea. \bibleverse{22} Cuando la hija de Herodías entró y
bailó, agradó a Herodes y a los que estaban sentados con él. El rey dijo
a la joven: ``Pídeme lo que quieras y te lo daré''. \bibleverse{23} Le
juró: ``Todo lo que me pidas, te lo daré, hasta la mitad de mi reino''.
\footnote{\textbf{6:23} Est 5,3; Est 5,6}

\bibleverse{24} Salió y le dijo a su madre: ``¿Qué voy a pedir?''. Ella
dijo: ``La cabeza de Juan el Bautista''.

\bibleverse{25} Ella entró inmediatamente con premura al rey y le pidió:
``Quiero que me des ahora mismo la cabeza de Juan el Bautista en una
bandeja''.

\bibleverse{26} El rey lo lamentó mucho, pero por el bien de sus
juramentos y de sus invitados a cenar, no quiso rechazarla.
\bibleverse{27} Inmediatamente el rey envió a un soldado de su guardia y
ordenó que trajera la cabeza de Juan; éste fue y lo decapitó en la
cárcel, \bibleverse{28} y trajo su cabeza en una bandeja y se la dio a
la joven; y la joven se la dio a su madre.

\bibleverse{29} Cuando sus discípulos se enteraron de esto, vinieron,
tomaron su cadáver y lo pusieron en un sepulcro.

\hypertarget{regreso-de-los-doce-apuxf3stoles-jesuxfas-escapa-a-la-soledad-alimentando-a-los-cinco-mil}{%
\subsection{Regreso de los doce apóstoles; Jesús escapa a la soledad;
Alimentando a los cinco
mil}\label{regreso-de-los-doce-apuxf3stoles-jesuxfas-escapa-a-la-soledad-alimentando-a-los-cinco-mil}}

\bibleverse{30} Los apóstoles se reunieron con Jesús y le contaron todo
lo que habían hecho y lo que habían enseñado. \footnote{\textbf{6:30}
  Luc 9,10; Luc 10,17} \bibleverse{31} Él les dijo: ``Venid a un lugar
desierto y descansad un poco''. Porque eran muchos los que iban y
venían, y no tenían tiempo ni para comer. \bibleverse{32} Se fueron en
la barca a un lugar desierto, solos. \bibleverse{33} Los \footnote{\textbf{6:33}
  TR lee ``Las multitudes'' en lugar de ``Ellos''} vieron ir, y muchos
lo reconocieron y corrieron allí a pie desde todas las ciudades.
Llegaron antes que ellos y se acercaron a él. \bibleverse{34} Salió
Jesús, vio una gran multitud y se compadeció de ellos porque eran como
ovejas sin pastor; y se puso a enseñarles muchas cosas. \footnote{\textbf{6:34}
  Mat 9,36} \bibleverse{35} Cuando se hizo tarde, sus discípulos se
acercaron a él y le dijeron: ``Este lugar está desierto, y ya es tarde.
\footnote{\textbf{6:35} Mar 8,1-9} \bibleverse{36} Despídelos para que
vayan al campo y a las aldeas de los alrededores y se compren el pan,
porque no tienen qué comer.''

\bibleverse{37} Pero él les respondió: ``Dadles vosotros de comer''. Le
preguntaron: ``¿Vamos a comprar doscientos denarios\footnote{\textbf{6:37}
  200 denarios eran unos 7 u 8 meses de salario para un trabajador
  agrícola.} de pan y les damos de comer?''.

\bibleverse{38} Les dijo: ``¿Cuántos panes tienen? Id a ver''. Cuando lo
supieron, dijeron: ``Cinco y dos peces''.

\bibleverse{39} Les ordenó que todos se sentaran en grupos sobre la
hierba verde. \bibleverse{40} Se sentaron en filas, de cien en cien y de
cincuenta en cincuenta. \bibleverse{41} Tomó los cinco panes y los dos
peces, y mirando al cielo, bendijo y partió los panes, y los dio a sus
discípulos para que los pusieran delante, y repartió los dos peces entre
todos. \footnote{\textbf{6:41} Mar 7,34} \bibleverse{42} Todos comieron
y se saciaron. \bibleverse{43} Recogieron doce cestas llenas de trozos y
también de los peces. \bibleverse{44} Los que comieron los panes fueron
\footnote{\textbf{6:44} TR añade ``sobre''} cinco mil hombres.

\hypertarget{regrese-a-travuxe9s-del-lago-por-la-noche-el-caminar-de-jesuxfas-sobre-el-lago-el-desembarco-en-gennesaret}{%
\subsection{Regrese a través del lago por la noche; el caminar de Jesús
sobre el lago; el desembarco en
Gennesaret}\label{regrese-a-travuxe9s-del-lago-por-la-noche-el-caminar-de-jesuxfas-sobre-el-lago-el-desembarco-en-gennesaret}}

\bibleverse{45} Inmediatamente hizo que sus discípulos subieran a la
barca y se adelantaran a la otra orilla, a Betsaida, mientras él mismo
despedía a la multitud. \bibleverse{46} Después de despedirse de ellos,
subió al monte a orar.

\bibleverse{47} Cuando llegó la noche, la barca estaba en medio del mar,
y él estaba solo en tierra. \bibleverse{48} Viendo que se afanaban en
remar, pues el viento les era contrario, hacia la cuarta vigilia de la
noche se acercó a ellos, caminando sobre el mar; y\footnote{\textbf{6:48}
  Ver Job 9:8} hubiera querido pasar junto a ellos, \bibleverse{49} pero
ellos, al verlo caminar sobre el mar, supusieron que era un fantasma, y
gritaron; \bibleverse{50} pues todos lo vieron y se turbaron. Pero él
habló enseguida con ellos y les dijo: ``¡Anímense! ¡Soy yo!\footnote{\textbf{6:50}
  o, ``¡Yo soy!''} No tengáis miedo''. \bibleverse{51} Subió a la barca
con ellos, y el viento cesó, y ellos se asombraron mucho entre sí, y se
maravillaron; \footnote{\textbf{6:51} Mar 4,39} \bibleverse{52} porque
no habían entendido lo de los panes, sino que tenían el corazón
endurecido. \footnote{\textbf{6:52} Mar 8,17}

\bibleverse{53} Cuando hubieron cruzado, llegaron a tierra en Genesaret
y atracaron en la orilla. \bibleverse{54} Cuando bajaron de la barca, la
gente lo reconoció inmediatamente, \bibleverse{55} y corrió por toda
aquella región, y comenzó a llevar a los enfermos sobre sus esteras a
donde oían que estaba. \bibleverse{56} Dondequiera que entraba --- en
las aldeas, o en las ciudades, o en el campo ---, ponían a los enfermos
en las plazas y le rogaban que sólo les dejara tocar los
flecos\footnote{\textbf{6:56} o, borla} de su manto; y todos los que lo
tocaban quedaban sanos. \footnote{\textbf{6:56} Mar 5,27-28; Hech 5,15;
  Hech 19,11-12}

\hypertarget{pelea-con-los-oponentes-sobre-el-lavado-de-manos-advertencia-de-estatutos-humanos-y-marcado-de-verdadera-impureza}{%
\subsection{Pelea con los oponentes sobre el lavado de manos;
Advertencia de estatutos humanos y marcado de verdadera
impureza}\label{pelea-con-los-oponentes-sobre-el-lavado-de-manos-advertencia-de-estatutos-humanos-y-marcado-de-verdadera-impureza}}

\hypertarget{section-6}{%
\section{7}\label{section-6}}

\bibleverse{1} Entonces se reunieron con él los fariseos y algunos de
los escribas, que habían venido de Jerusalén. \bibleverse{2} Al ver que
algunos de sus discípulos comían el pan con las manos manchadas, es
decir, sin lavar, se quejaron. \bibleverse{3} (Porque los fariseos y
todos los judíos no comen si no se lavan las manos y los antebrazos,
siguiendo la tradición de los ancianos. \bibleverse{4} No comen cuando
vienen de la plaza si no se bañan, y hay otras muchas cosas que han
recibido para aferrarse a ellas: lavados de copas, cántaros, vasos de
bronce y camillas). \footnote{\textbf{7:4} Mat 23,25} \bibleverse{5} Los
fariseos y los escribas le preguntaron: ``¿Por qué tus discípulos no
andan según la tradición de los ancianos, sino que comen el pan con las
manos sin lavar?''

\bibleverse{6} Les respondió: ``Bien profetizó Isaías de vosotros,
hipócritas, como está escrito, `Este pueblo me honra con sus labios,
pero su corazón está lejos de mí. \bibleverse{7} Me adoran en vano,
enseñando como doctrinas los mandamientos de los hombres.''\footnote{\textbf{7:7}
  Isaías 29:13}

\bibleverse{8} ``Porque dejáis de lado el mandamiento de Dios, y os
aferráis a la tradición de los hombres: el lavado de los cántaros y de
las copas, y hacéis otras muchas cosas semejantes.'' \bibleverse{9} Él
les dijo: ``Bien rechazáis el mandamiento de Dios para mantener vuestra
tradición. \bibleverse{10} Porque Moisés dijo: `Honra a tu padre y a tu
madre,'\footnote{\textbf{7:10} Éxodo 20:12; Deuteronomio 5:16} y `El que
hable mal del padre o de la madre, que muera'. \footnote{\textbf{7:10}
  Éxodo 21:17; Levítico 20:9} \bibleverse{11} Pero vosotros decís: ``Si
un hombre dice a su padre o a su madre: ``Cualquier beneficio que hayas
recibido de mí es ``corbán''\footnote{\textbf{7:11} Corbán es una
  palabra hebrea que designa una ofrenda dedicada a Dios.} , es decir,
entregado a Dios, \bibleverse{12} ``entonces ya no le permitís hacer
nada por su padre o por su madre, \bibleverse{13} anulando la palabra de
Dios por vuestra tradición que habéis transmitido. Vosotros hacéis
muchas cosas así''.

\bibleverse{14} Llamó a toda la multitud y les dijo: ``Oídme todos y
entended. \bibleverse{15} Nada de lo que sale del hombre puede
contaminarle; pero lo que sale del hombre es lo que le contamina.
\footnote{\textbf{7:15} Hech 10,14-15} \bibleverse{16} Si alguien tiene
oídos para oír, que oiga''. \footnote{\textbf{7:16} NU omite el
  versículo 16.}

\bibleverse{17} Cuando entró en una casa lejos de la multitud, sus
discípulos le preguntaron por la parábola. \bibleverse{18} Él les dijo:
``¿También vosotros estáis sin entendimiento? ¿No os dais cuenta de que
todo lo que entra en el hombre desde fuera no puede contaminarle,
\bibleverse{19} porque no entra en su corazón, sino en su estómago, y
luego en la letrina, con lo que todos los alimentos quedan limpios?''
\footnote{\textbf{7:19} NU termina la cita directa y la pregunta de
  Jesús después de ``letrina'', terminando el verso con ``Así declaró
  limpios todos los alimentos''.} \bibleverse{20} El dijo: ``Lo que sale
del hombre, eso contamina al hombre. \bibleverse{21} Porque de dentro,
del corazón de los hombres, salen los malos pensamientos, los
adulterios, los pecados sexuales, los asesinatos, los robos,
\bibleverse{22} las codicias, la maldad, el engaño, los deseos
lujuriosos, el mal de ojo, la blasfemia, la soberbia y la necedad.
\bibleverse{23} Todas estas cosas malas salen de dentro y contaminan al
hombre''.

\hypertarget{jesuxfas-y-la-sirofenicia-en-el-uxe1rea-de-tiro-y-siduxf3n}{%
\subsection{Jesús y la sirofenicia en el área de Tiro y
Sidón}\label{jesuxfas-y-la-sirofenicia-en-el-uxe1rea-de-tiro-y-siduxf3n}}

\bibleverse{24} De allí se levantó y se fue a los límites de Tiro y
Sidón. Entró en una casa y no quiso que nadie lo supiera, pero no pudo
pasar desapercibido. \bibleverse{25} Porque una mujer cuya hija pequeña
tenía un espíritu impuro, al oír hablar de él, vino y se postró a sus
pies. \bibleverse{26} La mujer era griega, de raza sirofenicia. Le rogó
que expulsara el demonio de su hija. \bibleverse{27} Pero Jesús le dijo:
``Deja que se sacien primero los niños, porque no conviene tomar el pan
de los niños y echarlo a los perros.''

\bibleverse{28} Pero ella le respondió: ``Sí, Señor. Pero hasta los
perros que están debajo de la mesa se comen las migajas de los niños''.

\bibleverse{29} Le dijo: ``Por este dicho, vete. El demonio ha salido de
tu hija''.

\bibleverse{30} Se fue a su casa y encontró al niño acostado en la cama,
con el demonio fuera.

\hypertarget{el-regreso-de-jesuxfas-a-galilea-en-la-orilla-oriental-del-lago-sanando-a-un-sordomudo}{%
\subsection{El regreso de Jesús a Galilea en la orilla oriental del
lago; Sanando a un
sordomudo}\label{el-regreso-de-jesuxfas-a-galilea-en-la-orilla-oriental-del-lago-sanando-a-un-sordomudo}}

\bibleverse{31} Volvió a salir de los límites de Tiro y Sidón, y llegó
al mar de Galilea por el centro de la región de Decápolis. \footnote{\textbf{7:31}
  Mar 5,20; Mat 15,29-31} \bibleverse{32} Le trajeron a uno que era
sordo y tenía un impedimento en el habla. Le rogaron que le pusiera la
mano encima. \bibleverse{33} Lo apartó de la multitud en privado y le
metió los dedos en los oídos, y escupiendo le tocó la lengua.
\bibleverse{34} Mirando al cielo, suspiró y le dijo: ``¡Efatá!'', es
decir, ``¡Ábrete!''. \bibleverse{35} Al instante se le abrieron los
oídos y se le soltó el impedimento de la lengua, y habló con claridad.
\bibleverse{36} Les ordenó que no se lo dijeran a nadie, pero cuanto más
les ordenaba, tanto más lo proclamaban. \footnote{\textbf{7:36} Mar
  1,43-45} \bibleverse{37} Ellos se asombraron mucho, diciendo: ``Todo
lo ha hecho bien. Hace que hasta los sordos oigan y los mudos hablen''.

\hypertarget{alimentando-a-los-cuatro-mil}{%
\subsection{Alimentando a los cuatro
mil}\label{alimentando-a-los-cuatro-mil}}

\hypertarget{section-7}{%
\section{8}\label{section-7}}

\bibleverse{1} En aquellos días, cuando había una multitud muy grande y
no tenían nada que comer, Jesús llamó a sus discípulos y les dijo:
\bibleverse{2} ``Tengo compasión de la multitud, porque ya llevan tres
días conmigo y no tienen nada que comer. \footnote{\textbf{8:2} Mar
  6,34-44} \bibleverse{3} Si los despido en ayunas para que se vayan a
su casa, se desmayarán en el camino, porque algunos de ellos han hecho
un largo recorrido.''

\bibleverse{4} Sus discípulos le respondieron: ``¿De dónde se podría
saciar a esta gente con pan aquí en un lugar desierto?''

\bibleverse{5} Les preguntó: ``¿Cuántos panes tenéis?''. Dijeron:
``Siete''.

\bibleverse{6} Mandó a la multitud que se sentara en el suelo, y tomó
los siete panes. Después de dar gracias, los partió y los dio a sus
discípulos para que los sirvieran, y ellos sirvieron a la multitud.
\bibleverse{7} También tenían unos cuantos pececillos. Después de
bendecirlos, dijo que los sirvieran también. \bibleverse{8} Comieron y
se saciaron. Recogieron siete cestas con los trozos que habían sobrado.
\bibleverse{9} Los que habían comido eran unos cuatro mil. Luego los
despidió.

\hypertarget{el-rechazo-de-jesuxfas-a-la-demanda-de-seuxf1ales-de-los-fariseos}{%
\subsection{El rechazo de Jesús a la demanda de señales de los
fariseos}\label{el-rechazo-de-jesuxfas-a-la-demanda-de-seuxf1ales-de-los-fariseos}}

\bibleverse{10} En seguida entró en la barca con sus discípulos y llegó
a la región de Dalmanutha. \bibleverse{11} Los fariseos salieron y
empezaron a interrogarle, pidiéndole una señal del cielo y poniéndole a
prueba. \footnote{\textbf{8:11} Juan 6,30} \bibleverse{12} El suspiró
profundamente en su espíritu y dijo: ``¿Por qué esta generación
\footnote{\textbf{8:12} La palabra traducida aquí como ``generación''
  (genea) también podría traducirse como ``pueblo'', ``raza'' o
  ``familia''.} busca una señal? Os aseguro que a esta generación no se
le dará ninguna señal''.

\bibleverse{13} Los dejó, y entrando de nuevo en la barca, se fue a la
otra orilla.

\hypertarget{advertencia-de-la-levadura-de-los-fariseos-y-la-de-herodes}{%
\subsection{Advertencia de la levadura de los fariseos y la de
Herodes}\label{advertencia-de-la-levadura-de-los-fariseos-y-la-de-herodes}}

\bibleverse{14} Se olvidaron de tomar pan, y no llevaban más que un pan
en la barca. \bibleverse{15} Les advirtió diciendo: ``Tened cuidado:
guardaos de la levadura de los fariseos y de la levadura de Herodes.''
\footnote{\textbf{8:15} Luc 12,1; Mar 3,6}

\bibleverse{16} Razonaban entre sí, diciendo: ``Es porque no tenemos
pan''.

\bibleverse{17} Jesús, al darse cuenta, les dijo: ``¿Por qué razonáis
que es porque no tenéis pan? ¿Aún no lo percibís o no lo entendéis? ¿Aún
está endurecido vuestro corazón? \footnote{\textbf{8:17} Mar 6,52}
\bibleverse{18} Teniendo ojos, ¿no veis? Teniendo oídos, ¿no oís? ¿No os
acordáis? \footnote{\textbf{8:18} Mat 13,13; Mat 13,16} \bibleverse{19}
Cuando partí los cinco panes entre los cinco mil, ¿cuántas cestas llenas
de trozos recogisteis?'' Le dijeron: ``Doce''. \footnote{\textbf{8:19}
  Mar 6,41-44}

\bibleverse{20} ``Cuando los siete panes alimentaron a los cuatro mil,
¿cuántas cestas llenas de trozos recogisteis?'' Le dijeron: ``Siete''.

\bibleverse{21} Les preguntó: ``¿Aún no lo habéis entendido?''.

\hypertarget{curaciuxf3n-de-ciegos-en-betsaida}{%
\subsection{Curación de ciegos en
Betsaida}\label{curaciuxf3n-de-ciegos-en-betsaida}}

\bibleverse{22} Llegó a Betsaida. Le trajeron un ciego y le rogaron que
lo tocara. \footnote{\textbf{8:22} Mar 6,56} \bibleverse{23} Tomó al
ciego de la mano y lo sacó de la aldea. Cuando le escupió en los ojos y
le puso las manos encima, le preguntó si veía algo. \footnote{\textbf{8:23}
  Juan 9,6}

\bibleverse{24} Levantó la vista y dijo: ``Veo hombres, pero los veo
como árboles que caminan''.

\bibleverse{25} Entonces volvió a poner las manos sobre sus ojos. Él
miró atentamente, y quedó restablecido, y vio a todos con claridad.
\bibleverse{26} Lo despidió a su casa, diciéndole: ``No entres en el
pueblo, ni se lo digas a nadie en el pueblo''. \footnote{\textbf{8:26}
  Mar 7,36}

\hypertarget{la-confesiuxf3n-de-pedro-del-mesuxedas}{%
\subsection{La confesión de Pedro del
Mesías}\label{la-confesiuxf3n-de-pedro-del-mesuxedas}}

\bibleverse{27} Jesús salió, con sus discípulos, a las aldeas de Cesarea
de Filipo. En el camino preguntó a sus discípulos: ``¿Quién dicen los
hombres que soy yo?''

\bibleverse{28} Le dijeron: ``Juan el Bautista, y otros dicen que Elías,
pero otros, uno de los profetas''. \footnote{\textbf{8:28} Mar 6,15}

\bibleverse{29} Les dijo: ``¿Pero quién decís que soy yo?''. Pedro
respondió: ``Tú eres el Cristo''.

\bibleverse{30} Les mandó que no hablaran a nadie de él. \footnote{\textbf{8:30}
  Mar 9,9}

\hypertarget{el-primer-anuncio-del-sufrimiento-de-jesuxfas}{%
\subsection{El primer anuncio del sufrimiento de
Jesús}\label{el-primer-anuncio-del-sufrimiento-de-jesuxfas}}

\bibleverse{31} Comenzó a enseñarles que era necesario que el Hijo del
Hombre padeciera muchas cosas, y que fuera rechazado por los ancianos,
los sumos sacerdotes y los escribas, y que fuera matado, y que después
de tres días resucitara. \bibleverse{32} Les hablaba abiertamente. Pedro
lo tomó y comenzó a reprenderlo. \bibleverse{33} Pero él, volviéndose y
viendo a sus discípulos, reprendió a Pedro y le dijo: ``¡Quítate de
encima, Satanás! Porque no piensas en las cosas de Dios, sino en las de
los hombres''.

\hypertarget{proverbios-sobre-el-seguimiento-de-los-discuxedpulos-en-el-sufrimiento}{%
\subsection{Proverbios sobre el seguimiento de los discípulos en el
sufrimiento}\label{proverbios-sobre-el-seguimiento-de-los-discuxedpulos-en-el-sufrimiento}}

\bibleverse{34} Llamó a la multitud con sus discípulos y les dijo: ``El
que quiera venir en pos de mí, que se niegue a sí mismo, tome su cruz y
me siga. \bibleverse{35} Porque el que quiera salvar su vida, la
perderá; y el que pierda su vida por mí y por la Buena Nueva, la
salvará. \footnote{\textbf{8:35} Mat 10,39} \bibleverse{36} Porque ¿de
qué le sirve al hombre ganar el mundo entero y perder su vida?
\bibleverse{37} Porque ¿qué dará el hombre a cambio de su vida?
\bibleverse{38} Porque el que se avergüence de mí y de mis palabras en
esta generación adúltera y pecadora, también el Hijo del Hombre se
avergonzará de él cuando venga en la gloria de su Padre con los santos
ángeles.'' \footnote{\textbf{8:38} Mat 10,33}

\hypertarget{section-8}{%
\section{9}\label{section-8}}

\bibleverse{1} Les dijo: ``Os aseguro que hay algunos de los que están
aquí que no probarán la muerte hasta que vean llegar el Reino de Dios
con poder.''

\hypertarget{la-transfiguraciuxf3n-de-jesuxfas-en-la-montauxf1a-y-su-conversaciuxf3n-con-los-discuxedpulos-en-el-descenso}{%
\subsection{La transfiguración de Jesús en la montaña y su conversación
con los discípulos en el
descenso}\label{la-transfiguraciuxf3n-de-jesuxfas-en-la-montauxf1a-y-su-conversaciuxf3n-con-los-discuxedpulos-en-el-descenso}}

\bibleverse{2} Al cabo de seis días, Jesús tomó consigo a Pedro,
Santiago y Juan, y los llevó a un monte alto en privado, y se transformó
en otra forma delante de ellos. \bibleverse{3} Sus vestidos se volvieron
relucientes, sumamente blancos, como la nieve, como ningún lavandero en
la tierra puede blanquearlos. \bibleverse{4} Se les aparecieron Elías y
Moisés, que hablaban con Jesús.

\bibleverse{5} Pedro respondió a Jesús: ``Rabí, es bueno que estemos
aquí. Hagamos tres tiendas: una para ti, otra para Moisés y otra para
Elías''. \bibleverse{6} Pues no sabía qué decir, ya que tenían mucho
miedo.

\bibleverse{7} Llegó una nube que los cubría, y una voz salió de la
nube: ``Este es mi Hijo amado. Escuchadle''. \footnote{\textbf{9:7} Mar
  1,11; 2Pe 1,17}

\bibleverse{8} De repente, al mirar a su alrededor, ya no vieron a nadie
con ellos, sino sólo a Jesús.

\bibleverse{9} Mientras bajaban del monte, les ordenó que no contaran a
nadie lo que habían visto, hasta que el Hijo del Hombre hubiera
resucitado de entre los muertos. \footnote{\textbf{9:9} Mar 8,30}
\bibleverse{10} Ellos guardaron esta frase para sí mismos, preguntándose
qué significaba eso de ``resucitar de entre los muertos''.

\bibleverse{11} Le preguntaron: ``¿Por qué dicen los escribas que Elías
debe venir primero?''

\bibleverse{12} Les dijo: ``En efecto, Elías viene primero y restaura
todas las cosas. ¿Cómo está escrito acerca del Hijo del Hombre, que ha
de padecer muchas cosas y ser despreciado? \footnote{\textbf{9:12} Mal
  4,5; Is 53,3} \bibleverse{13} Pero yo os digo que Elías ha venido, y
también han hecho con él lo que han querido, tal como está escrito de
él.'' \footnote{\textbf{9:13} Mat 11,14; 1Re 19,2; 1Re 19,10}

\hypertarget{curaciuxf3n-de-un-niuxf1o-epiluxe9ptico-la-incapacidad-de-los-discuxedpulos}{%
\subsection{Curación de un niño epiléptico; la incapacidad de los
discípulos}\label{curaciuxf3n-de-un-niuxf1o-epiluxe9ptico-la-incapacidad-de-los-discuxedpulos}}

\bibleverse{14} Al llegar a los discípulos, vio que los rodeaba una gran
multitud y que los escribas los interrogaban. \bibleverse{15} En
seguida, toda la multitud, al verle, se asombró mucho, y corriendo hacia
él, le saludó. \bibleverse{16} Él preguntó a los escribas: ``¿Qué les
preguntas?''

\bibleverse{17} Uno de la multitud respondió: ``Maestro, te he traído a
mi hijo, que tiene un espíritu mudo; \bibleverse{18} y dondequiera que
se apodera de él, lo derriba, y echa espuma por la boca, rechina los
dientes y se pone rígido. He pedido a tus discípulos que lo expulsen, y
no han podido''.

\bibleverse{19} Le respondió: ``Generación incrédula, ¿hasta cuándo
estaré con vosotros? ¿Hasta cuándo habré de soportaros? Traedlo a mí''.

\bibleverse{20} Lo llevaron hasta él, y cuando lo vio, inmediatamente el
espíritu lo convulsionó y cayó al suelo, revolcándose y echando espuma
por la boca.

\bibleverse{21} Le preguntó a su padre: ``¿Cuánto tiempo hace que le
pasa esto?''. Dijo: ``Desde la infancia. \bibleverse{22} Muchas veces lo
ha echado al fuego y al agua para destruirlo. Pero si puedes hacer algo,
ten compasión de nosotros y ayúdanos''.

\bibleverse{23} Jesús le dijo: ``Si puedes creer, todo es posible para
el que cree''.

\bibleverse{24} Inmediatamente el padre del niño gritó con lágrimas:
``¡Creo! Ayuda a mi incredulidad''.

\bibleverse{25} Al ver Jesús que una multitud venía corriendo, reprendió
al espíritu impuro, diciéndole: ``¡Espíritu mudo y sordo, te ordeno que
salgas de él y no vuelvas a entrar!''

\bibleverse{26} Después de gritar y convulsionar mucho, salió de él. El
muchacho quedó como muerto, tanto que la mayoría decía: ``Está muerto''.
\footnote{\textbf{9:26} Mar 1,26} \bibleverse{27} Pero Jesús lo tomó de
la mano y lo resucitó; y se levantó.

\bibleverse{28} Cuando entró en la casa, sus discípulos le preguntaron
en privado: ``¿Por qué no pudimos expulsarlo?''

\bibleverse{29} Les dijo: ``Este tipo no puede salir sino con oración y
ayuno''.

\hypertarget{segundo-anuncio-de-sufrimiento}{%
\subsection{Segundo anuncio de
sufrimiento}\label{segundo-anuncio-de-sufrimiento}}

\bibleverse{30} Salieron de allí y pasaron por Galilea. No quería que
nadie lo supiera, \bibleverse{31} porque estaba enseñando a sus
discípulos, y les decía: ``El Hijo del Hombre va a ser entregado a manos
de los hombres, y lo matarán; y cuando lo maten, al tercer día
resucitará.'' \footnote{\textbf{9:31} Mar 8,31; Mar 10,32-34}

\bibleverse{32} Pero no entendieron el dicho y tuvieron miedo de
preguntarle. \footnote{\textbf{9:32} Luc 18,34}

\hypertarget{controversia-entre-discuxedpulos-la-exhortaciuxf3n-de-jesuxfas-a-la-humildad}{%
\subsection{Controversia entre discípulos; La exhortación de Jesús a la
humildad}\label{controversia-entre-discuxedpulos-la-exhortaciuxf3n-de-jesuxfas-a-la-humildad}}

\bibleverse{33} Llegó a Capernaúm y, estando en la casa, les preguntó:
``¿Qué discutíais entre vosotros por el camino?''

\bibleverse{34} Pero ellos guardaron silencio, porque habían discutido
entre sí en el camino sobre quién era el más grande.

\bibleverse{35} Se sentó y llamó a los doce, y les dijo: ``Si alguno
quiere ser el primero, será el último de todos y el servidor de todos''.
\footnote{\textbf{9:35} Mar 10,44; Mat 20,27} \bibleverse{36} Tomó a un
niño pequeño y lo puso en medio de ellos. Tomándolo en brazos, les dijo:
\bibleverse{37} ``El que recibe a un niño como éste en mi nombre, me
recibe a mí; y el que me recibe a mí, no me recibe a mí, sino al que me
ha enviado.'' \footnote{\textbf{9:37} Mat 10,40}

\hypertarget{enseuxf1ar-sobre-la-tolerancia}{%
\subsection{Enseñar sobre la
tolerancia}\label{enseuxf1ar-sobre-la-tolerancia}}

\bibleverse{38} Juan le dijo: ``Maestro, hemos visto a uno que no nos
sigue expulsando demonios en tu nombre, y se lo prohibimos porque no nos
sigue.'' \footnote{\textbf{9:38} Núm 11,27-28}

\bibleverse{39} Pero Jesús dijo: ``No se lo prohíbas, porque no hay
nadie que haga una obra poderosa en mi nombre y pueda rápidamente hablar
mal de mí. \footnote{\textbf{9:39} 1Cor 12,3} \bibleverse{40} Porque el
que no está contra nosotros, está de nuestra parte. \footnote{\textbf{9:40}
  Mat 12,30; Luc 11,23} \bibleverse{41} Porque cualquiera que os dé a
beber un vaso de agua en mi nombre porque sois de Cristo, de cierto os
digo que no perderá su recompensa. \footnote{\textbf{9:41} Mat 10,42}

\hypertarget{advertencia-de-engauxf1o-a-la-incredulidad-y-al-pecado-dichos-de-sal}{%
\subsection{Advertencia de engaño (a la incredulidad y al pecado);
Dichos de
sal}\label{advertencia-de-engauxf1o-a-la-incredulidad-y-al-pecado-dichos-de-sal}}

\bibleverse{42} ``El que haga tropezar a uno de estos pequeños que creen
en mí, más le valdría ser arrojado al mar con una piedra de molino
colgada al cuello. \bibleverse{43} Si tu mano te hace tropezar, córtala.
Es mejor que entres en la vida mutilado, en lugar de que tus dos manos
vayan a la Gehenna, \footnote{\textbf{9:43} o, el infierno} al fuego
inextinguible, \footnote{\textbf{9:43} Mat 5,30} \bibleverse{44} `donde
su gusano no muere, y el fuego no se apaga.' \footnote{\textbf{9:44}
  Isaías 66:24} \footnote{\textbf{9:44} NU omite el versículo 44.}
\bibleverse{45} Si tu pie te hace tropezar, córtalo. Es mejor que entres
cojo en la vida, antes que tus dos pies sean arrojados a la Gehenna,
\footnote{\textbf{9:45} o, el infierno} al fuego que nunca se apagará,
\bibleverse{46} `donde su gusano no muere, y el fuego no se
apaga.'\footnote{\textbf{9:46} NU omite el verso 46.} \bibleverse{47} Si
tu ojo te hace tropezar, arrójalo. Es mejor que entres en el Reino de
Dios con un solo ojo, en lugar de tener dos ojos para ser arrojado a la
Gehenna\footnote{\textbf{9:47} o el infierno} del fuego, \footnote{\textbf{9:47}
  Mat 5,29} \bibleverse{48} `donde su gusano no muere, y el fuego no se
apaga.' \footnote{\textbf{9:48} Isaías 66:24} \bibleverse{49} Porque
todos serán salados con fuego, y todo sacrificio será sazonado con sal.
\footnote{\textbf{9:49} Lev 2,13} \bibleverse{50} La sal es buena, pero
si la sal ha perdido su salinidad, ¿con qué la sazonaréis? Tened sal en
vosotros mismos, y estad en paz unos con otros''. \footnote{\textbf{9:50}
  Mat 5,13; Luc 14,34; Col 4,6}

\hypertarget{jesuxfas-en-judea-y-transjordania-conversaciones-sobre-matrimonio-y-divorcio}{%
\subsection{Jesús en Judea y Transjordania; Conversaciones sobre
matrimonio y
divorcio}\label{jesuxfas-en-judea-y-transjordania-conversaciones-sobre-matrimonio-y-divorcio}}

\hypertarget{section-9}{%
\section{10}\label{section-9}}

\bibleverse{1} Se levantó de allí y llegó a las fronteras de Judea y al
otro lado del Jordán. Las multitudes volvieron a reunirse con él. Como
solía hacer, volvía a enseñarles.

\bibleverse{2} Los fariseos se acercaron a él para ponerle a prueba y le
preguntaron: ``¿Es lícito que un hombre se divorcie de su mujer?''

\bibleverse{3} Él respondió: ``¿Qué te ordenó Moisés?''

\bibleverse{4} Dijeron: ``Moisés permitió que se escribiera un
certificado de divorcio y que se divorciara''. \footnote{\textbf{10:4}
  Deut 24,1; Mat 5,31-32}

\bibleverse{5} Pero Jesús les dijo: ``Por vuestra dureza de corazón, os
escribió este mandamiento. \bibleverse{6} Pero desde el principio de la
creación, Dios los hizo hombre y mujer. \footnote{\textbf{10:6} Génesis
  1:27} \footnote{\textbf{10:6} Gén 1,27} \bibleverse{7} Por eso el
hombre dejará a su padre y a su madre y se unirá a su mujer, \footnote{\textbf{10:7}
  Gén 2,24} \bibleverse{8} y los dos se convertirán en una sola
carne,\footnote{\textbf{10:8} Génesis 2:24} de modo que ya no son dos,
sino una sola carne. \bibleverse{9} Lo que Dios ha unido, que no lo
separe el hombre''.

\bibleverse{10} En la casa, sus discípulos le volvieron a preguntar
sobre el mismo asunto. \bibleverse{11} Él les dijo: ``El que se divorcia
de su mujer y se casa con otra, comete adulterio contra ella.
\footnote{\textbf{10:11} Luc 16,18; 1Cor 7,10-11} \bibleverse{12} Si una
mujer se divorcia de su marido y se casa con otro, comete adulterio''.

\hypertarget{jesuxfas-bendice-a-los-niuxf1os}{%
\subsection{Jesús bendice a los
niños}\label{jesuxfas-bendice-a-los-niuxf1os}}

\bibleverse{13} Le traían niños para que los tocara, pero los discípulos
reprendieron a los que los traían. \bibleverse{14} Al ver esto, Jesús se
indignó y les dijo: ``Dejad que los niños se acerquen a mí. No se lo
prohibáis, porque el Reino de Dios es de los que son como ellos.
\bibleverse{15} Os aseguro que quien no quiera recibir el Reino de Dios
como un niño, no entrará en él.'' \footnote{\textbf{10:15} Mat 18,3}
\bibleverse{16} Los tomó en sus brazos y los bendijo, imponiéndoles las
manos. \footnote{\textbf{10:16} Mar 9,36}

\hypertarget{la-conversaciuxf3n-de-jesuxfas-con-los-ricos-y-su-referencia-al-peligro-de-las-riquezas}{%
\subsection{La conversación de Jesús con los ricos y su referencia al
peligro de las
riquezas}\label{la-conversaciuxf3n-de-jesuxfas-con-los-ricos-y-su-referencia-al-peligro-de-las-riquezas}}

\bibleverse{17} Al salir al camino, uno corrió hacia él, se arrodilló
ante él y le preguntó: ``Maestro bueno, ¿qué debo hacer para heredar la
vida eterna?''

\bibleverse{18} Jesús le dijo: ``¿Por qué me llamas bueno? Nadie es
bueno sino uno: Dios. \bibleverse{19} Tú conoces los mandamientos: `No
matar', `No cometer adulterio', `No robar', `No dar falso testimonio',
`No defraudar', `Honrar a tu padre y a tu madre'\,''. \footnote{\textbf{10:19}
  Éxodo 20:12-16; Deuteronomio 5:16-20} \footnote{\textbf{10:19} Éxod
  20,12-17}

\bibleverse{20} Le dijo: ``Maestro, todo esto lo he observado desde mi
juventud''.

\bibleverse{21} Jesús, mirándolo, lo amó y le dijo: ``Una cosa te falta.
Vete, vende todo lo que tienes y dalo a los pobres, y tendrás un tesoro
en el cielo; y ven, sígueme, tomando la cruz.'' \footnote{\textbf{10:21}
  Mar 8,34; Mat 10,38}

\bibleverse{22} Pero su rostro se abatió al oír estas palabras y se
marchó apenado, porque era alguien que tenía grandes posesiones.

\bibleverse{23} Jesús miró a su alrededor y dijo a sus discípulos:
``¡Qué difícil es para los que tienen riquezas entrar en el Reino de
Dios!''

\bibleverse{24} Los discípulos se asombraron de sus palabras. Pero Jesús
volvió a responder: ``Hijos, ¡qué difícil es entrar en el Reino de Dios
para los que confían en las riquezas! \footnote{\textbf{10:24} Sal
  62,10; 1Tim 6,17} \bibleverse{25} Es más fácil que un camello pase por
el ojo de una aguja que un rico entre en el Reino de Dios.''

\bibleverse{26} Estaban muy asombrados y le decían: ``Entonces, ¿quién
puede salvarse?''.

\bibleverse{27} Jesús, mirándolos, dijo: ``Para los hombres es
imposible, pero no para Dios, porque para Dios todo es posible.''

\hypertarget{la-recompensa-de-seguir-a-jesuxfas-y-la-renuncia}{%
\subsection{La recompensa de seguir a Jesús y la
renuncia}\label{la-recompensa-de-seguir-a-jesuxfas-y-la-renuncia}}

\bibleverse{28} Pedro comenzó a decirle: ``Mira, lo hemos dejado todo y
te hemos seguido''.

\bibleverse{29} Jesús dijo: ``Os aseguro que no hay nadie que haya
dejado casa, ni hermanos, ni hermanas, ni padre, ni madre, ni mujer, ni
hijos, ni tierra, por mí y por la Buena Noticia, \bibleverse{30} sino
que recibirá cien veces más ahora en este tiempo: casas, hermanos,
hermanas, madres, hijos y tierra, con persecuciones; y en el siglo
venidero la vida eterna. \bibleverse{31} Pero muchos de los primeros
serán los últimos, y los últimos los primeros''.

\hypertarget{salida-hacia-jerusaluxe9n-tercer-anuncio-del-sufrimiento-de-jesuxfas}{%
\subsection{Salida hacia Jerusalén; tercer anuncio del sufrimiento de
Jesús}\label{salida-hacia-jerusaluxe9n-tercer-anuncio-del-sufrimiento-de-jesuxfas}}

\bibleverse{32} Iban por el camino, subiendo a Jerusalén, y Jesús iba
delante de ellos, y estaban asombrados; y los que le seguían tenían
miedo. Volvió a tomar a los doce, y comenzó a contarles las cosas que le
iban a suceder. \footnote{\textbf{10:32} Mar 9,31} \bibleverse{33} ``He
aquí que subimos a Jerusalén. El Hijo del Hombre será entregado a los
sumos sacerdotes y a los escribas. Lo condenarán a muerte y lo
entregarán a los gentiles. \bibleverse{34} Se burlarán de él, lo
escupirán, lo azotarán y lo matarán. Al tercer día resucitará''.

\hypertarget{solicitud-ambiciosa-de-los-dos-hijos-de-zebedeo}{%
\subsection{Solicitud ambiciosa de los dos hijos de
Zebedeo}\label{solicitud-ambiciosa-de-los-dos-hijos-de-zebedeo}}

\bibleverse{35} Santiago y Juan, los hijos de Zebedeo, se acercaron a él
diciendo: ``Maestro, queremos que hagas por nosotros todo lo que te
pidamos.''

\bibleverse{36} Les dijo: ``¿Qué queréis que haga por vosotros?''.

\bibleverse{37} Le dijeron: ``Concédenos que nos sentemos, uno a tu
derecha y otro a tu izquierda, en tu gloria''.

\bibleverse{38} Pero Jesús les dijo: ``No sabéis lo que pedís. ¿Sois
capaces de beber el cáliz que yo bebo, y de ser bautizados con el
bautismo con el que yo soy bautizado?'' \footnote{\textbf{10:38} Mar
  14,36; Luc 12,50}

\bibleverse{39} Le dijeron: ``Podemos''. Jesús les dijo: ``Ciertamente
beberéis el cáliz que yo bebo, y seréis bautizados con el bautismo con
el que yo soy bautizado; \footnote{\textbf{10:39} Hech 12,2; Apoc 1,9}
\bibleverse{40} pero sentarse a mi derecha y a mi izquierda no me
corresponde a mí, sino a quien ha sido preparado.''

\bibleverse{41} Cuando los diez lo oyeron, comenzaron a indignarse
contra Santiago y Juan.

\bibleverse{42} Jesús los convocó y les dijo ``Ustedes saben que los que
son reconocidos como gobernantes de las naciones se enseñorean de ellas,
y sus grandes ejercen autoridad sobre ellas. \footnote{\textbf{10:42}
  Luc 22,25-27} \bibleverse{43} Pero entre ustedes no será así, sino que
el que quiera hacerse grande entre ustedes será su servidor. \footnote{\textbf{10:43}
  Mar 9,35; 1Pe 5,3} \bibleverse{44} El que de vosotros quiera llegar a
ser el primero, será siervo de todos. \bibleverse{45} Porque también el
Hijo del Hombre no ha venido a ser servido, sino a servir, y a dar su
vida en rescate por muchos.''

\hypertarget{curaciuxf3n-del-ciego-bartimeo-cerca-de-jericuxf3}{%
\subsection{Curación del ciego Bartimeo cerca de
Jericó}\label{curaciuxf3n-del-ciego-bartimeo-cerca-de-jericuxf3}}

\bibleverse{46} Llegaron a Jericó. Al salir de Jericó con sus discípulos
y una gran multitud, el hijo de Timeo, Bartimeo, un mendigo ciego,
estaba sentado junto al camino. \bibleverse{47} Al oír que era Jesús el
Nazareno, se puso a gritar y a decir: ``¡Jesús, hijo de David, ten
piedad de mí!'' \bibleverse{48} Muchos le reprendían para que se
callara, pero él gritaba mucho más: ``¡Hijo de David, ten piedad de
mí!''

\bibleverse{49} Jesús se detuvo y dijo: ``Llámalo''. Llamaron al ciego,
diciéndole: ``¡Anímate! Levántate. Te está llamando''.

\bibleverse{50} Él, arrojando su manto, se levantó y se acercó a Jesús.

\bibleverse{51} Jesús le preguntó: ``¿Qué quieres que haga por ti?''. El
ciego le dijo: ``Rabboni,\footnote{\textbf{10:51} Rabboni es una
  transliteración de la palabra hebrea ``gran maestro''.} que vuelva a
ver''.

\bibleverse{52} Jesús le dijo: ``Vete. Tu fe te ha curado''.
Inmediatamente recibió la vista y siguió a Jesús por el camino.

\hypertarget{la-entrada-de-jesuxfas-a-jerusaluxe9n}{%
\subsection{La entrada de Jesús a
Jerusalén}\label{la-entrada-de-jesuxfas-a-jerusaluxe9n}}

\hypertarget{section-10}{%
\section{11}\label{section-10}}

\bibleverse{1} Cuando se acercaron a Jerusalén, a Betfagé\footnote{\textbf{11:1}
  TR y NU leen ``Bethphage'' en lugar de ``Bethsphage''} y Betania, en
el Monte de los Olivos, envió a dos de sus discípulos \footnote{\textbf{11:1}
  Juan 2,13} \bibleverse{2} y les dijo: ``Id a la aldea que está
enfrente de vosotros. En cuanto entréis en ella, encontraréis un pollino
atado, en el que nadie se ha sentado. Desátenlo y tráiganlo.
\bibleverse{3} Si alguien os pregunta: ``¿Por qué hacéis esto?'',
decidle: ``El Señor lo necesita'', e inmediatamente lo enviará de vuelta
aquí.''

\bibleverse{4} Se fueron y encontraron un pollino atado a la puerta, en
la calle, y lo desataron. \bibleverse{5} Algunos de los que estaban allí
les preguntaron: ``¿Qué hacéis desatando el pollino?''. \bibleverse{6}
Ellos les dijeron lo mismo que Jesús, y los dejaron ir.

\bibleverse{7} Trajeron a Jesús el pollino y echaron sobre él sus
vestidos, y Jesús se sentó en él. \bibleverse{8} Muchos extendían sus
vestidos por el camino, y otros cortaban ramas de los árboles y las
esparcían por el camino. \bibleverse{9} Los que iban delante y los que
les seguían gritaban: ``¡Hosanna!\footnote{\textbf{11:9} ``Hosanna''
  significa ``sálvanos'' o ``ayúdanos, te rogamos''.} ¡Bendito el que
viene en nombre del Señor! \footnote{\textbf{11:9} Salmo 118:25-26}
\footnote{\textbf{11:9} Sal 118,25-26} \bibleverse{10} ¡Bendito el reino
de nuestro padre David que viene en el nombre del Señor! Hosanna en las
alturas''.

\bibleverse{11} Jesús entró en el templo de Jerusalén. Después de haber
observado todo, siendo ya de noche, salió a Betania con los doce.

\hypertarget{la-maldiciuxf3n-de-una-higuera-estuxe9ril}{%
\subsection{La maldición de una higuera
estéril}\label{la-maldiciuxf3n-de-una-higuera-estuxe9ril}}

\bibleverse{12} Al día siguiente, cuando salieron de Betania, tuvo
hambre. \bibleverse{13} Al ver una higuera lejana que tenía hojas, se
acercó para ver si acaso podía encontrar algo en ella. Cuando llegó a
ella, no encontró más que hojas, pues no era la época de los higos.
\bibleverse{14} Jesús le dijo: ``Que nadie vuelva a comer fruto de ti'',
y sus discípulos lo oyeron.

\hypertarget{la-limpieza-del-templo}{%
\subsection{La limpieza del templo}\label{la-limpieza-del-templo}}

\bibleverse{15} Llegaron a Jerusalén, y Jesús entró en el templo y
comenzó a echar a los que vendían y a los que compraban en el templo, y
derribó las mesas de los cambistas y los asientos de los que vendían
palomas. \footnote{\textbf{11:15} Juan 2,14-16} \bibleverse{16} No
permitía que nadie llevara un recipiente por el templo. \bibleverse{17}
Les enseñaba diciendo ``¿No está escrito que mi casa será llamada casa
de oración para todas las naciones?\footnote{\textbf{11:17} Isaías 56:7}
Pero vosotros la habéis convertido en una cueva de ladrones''.
\footnote{\textbf{11:17} Jeremías 7:11} \footnote{\textbf{11:17} Jer
  7,11}

\bibleverse{18} Los jefes de los sacerdotes y los escribas lo oyeron, y
buscaban cómo destruirlo. Porque le temían, pues toda la multitud se
asombraba de su enseñanza.

\bibleverse{19} Al caer la tarde, salió de la ciudad.

\hypertarget{repaso-de-la-higuera-seca-con-posterior-referencia-al-poder-de-la-fe-y-la-oraciuxf3n-advertencia}{%
\subsection{Repaso de la higuera seca con posterior referencia al poder
de la fe y la oración;
advertencia}\label{repaso-de-la-higuera-seca-con-posterior-referencia-al-poder-de-la-fe-y-la-oraciuxf3n-advertencia}}

\bibleverse{20} Al pasar por la mañana, vieron la higuera seca de raíz.
\bibleverse{21} Pedro, acordándose, le dijo: ``¡Rabí, mira! La higuera
que maldijiste se ha secado''.

\bibleverse{22} Jesús les respondió: ``Tened fe en Dios. \bibleverse{23}
Porque de cierto os digo que cualquiera que diga a este monte: ``Tómalo
y arrójalo al mar'', y no dude en su corazón, sino que crea que lo que
dice sucede, tendrá lo que dice. \footnote{\textbf{11:23} Mar 9,23; Mat
  17,20} \bibleverse{24} Por eso os digo que todo lo que pidáis y oréis,
creed que lo habéis recibido, y lo tendréis. \footnote{\textbf{11:24}
  Mat 7,7; Juan 14,13; 1Jn 5,14; 1Jn 1,5-15} \bibleverse{25} Siempre que
estéis orando, perdonad, si tenéis algo contra alguien, para que vuestro
Padre, que está en los cielos, os perdone también vuestras
transgresiones. \footnote{\textbf{11:25} Mat 5,23} \bibleverse{26} Pero
si no perdonáis, tampoco vuestro Padre que está en los cielos os
perdonará vuestras transgresiones.'' \footnote{\textbf{11:26} NU omite
  el versículo 26.} \footnote{\textbf{11:26} Mat 6,14-15}

\hypertarget{la-pregunta-del-sumo-consejo-sobre-la-autoridad-de-jesuxfas}{%
\subsection{La pregunta del sumo consejo sobre la autoridad de
Jesús}\label{la-pregunta-del-sumo-consejo-sobre-la-autoridad-de-jesuxfas}}

\bibleverse{27} Llegaron de nuevo a Jerusalén y, mientras caminaba por
el templo, se le acercaron los jefes de los sacerdotes, los escribas y
los ancianos, \bibleverse{28} y comenzaron a decirle: ``¿Con qué
autoridad haces estas cosas? ¿O quién te ha dado esta autoridad para
hacer estas cosas?''

\bibleverse{29} Jesús les dijo: ``Les voy a hacer una pregunta.
Respóndanme, y les diré con qué autoridad hago estas cosas.
\bibleverse{30} El bautismo de Juan, ¿es del cielo o de los hombres?
Respondedme''.

\bibleverse{31} Razonaban entre sí, diciendo: ``Si decimos: ``Del
cielo'', dirá: ``¿Por qué, pues, no le habéis creído?'' \bibleverse{32}
Si decimos: ``De los hombres'', temían a la gente, pues todos
consideraban que Juan era realmente un profeta. \footnote{\textbf{11:32}
  Luc 7,29-30} \bibleverse{33} Ellos respondieron a Jesús: ``No lo
sabemos''. Jesús les dijo: ``Tampoco os diré con qué autoridad hago
estas cosas''.

\hypertarget{paruxe1bola-de-los-viticultores-infieles}{%
\subsection{Parábola de los viticultores
infieles}\label{paruxe1bola-de-los-viticultores-infieles}}

\hypertarget{section-11}{%
\section{12}\label{section-11}}

\bibleverse{1} Se puso a hablarles en parábolas. ``Un hombre plantó una
viña, la rodeó de un seto, cavó un pozo para el lagar, construyó una
torre, la alquiló a un agricultor y se fue a otro país. \footnote{\textbf{12:1}
  Is 5,1-2} \bibleverse{2} Cuando llegó el momento, envió a un siervo al
agricultor para que le diera su parte del fruto de la viña.
\bibleverse{3} Lo tomaron, lo golpearon y lo despidieron vacío.
\bibleverse{4} Volvió a enviar a otro siervo, y le tiraron piedras, lo
hirieron en la cabeza y lo despidieron maltratado. \bibleverse{5} Volvió
a enviar a otro, y lo mataron a él y a otros muchos, golpeando a unos y
matando a otros. \bibleverse{6} Por eso, teniendo todavía uno, su hijo
amado, lo envió el último a ellos, diciendo: ``Respetarán a mi hijo''.
\bibleverse{7} Pero aquellos campesinos dijeron entre sí: `Este es el
heredero. Vengan, matémoslo, y la herencia será nuestra'. \bibleverse{8}
Lo tomaron, lo mataron y lo echaron de la viña. \footnote{\textbf{12:8}
  Heb 13,12} \bibleverse{9} ¿Qué hará, pues, el señor de la viña? Vendrá
y destruirá a los labradores, y dará la viña a otros. \bibleverse{10}
¿Acaso no has leído esta Escritura? La piedra que desecharon los
constructores fue nombrado jefe de la esquina. \bibleverse{11} Esto era
del Señor. Es maravilloso a nuestros ojos'\,''. \footnote{\textbf{12:11}
  Salmo 118:22-23}

\bibleverse{12} Intentaron apoderarse de él, pero temían a la multitud,
pues se dieron cuenta de que decía la parábola contra ellos. Lo dejaron
y se fueron.

\hypertarget{la-cuestiuxf3n-fiscal-de-los-fariseos}{%
\subsection{La cuestión fiscal de los
fariseos}\label{la-cuestiuxf3n-fiscal-de-los-fariseos}}

\bibleverse{13} Enviaron a algunos de los fariseos y de los herodianos
hacia él, para atraparlo con palabras. \bibleverse{14} Cuando llegaron,
le preguntaron: ``Maestro, sabemos que eres honesto y que no te inclinas
por nadie, pues no eres parcial con nadie, sino que enseñas
verdaderamente el camino de Dios. ¿Es lícito pagar impuestos al César, o
no? \bibleverse{15} ¿Debemos dar, o no debemos dar?'' Pero él,
conociendo su hipocresía, les dijo: ``¿Por qué me ponéis a prueba?
Traedme un denario, para que lo vea''.

\bibleverse{16} Lo trajeron. Les dijo: ``¿De quién es esta imagen y esta
inscripción?'' Le dijeron: ``Del César''.

\bibleverse{17} Jesús les respondió: ``Dad al César lo que es del César
y a Dios lo que es de Dios''. Se maravillaron mucho con él. \footnote{\textbf{12:17}
  Rom 13,7}

\hypertarget{la-pregunta-sobre-la-resurrecciuxf3n-de-los-muertos}{%
\subsection{La pregunta sobre la resurrección de los
muertos}\label{la-pregunta-sobre-la-resurrecciuxf3n-de-los-muertos}}

\bibleverse{18} Algunos saduceos, que dicen que no hay resurrección, se
acercaron a él. Le preguntaron, diciendo: \bibleverse{19} ``Maestro,
Moisés nos escribió: ``Si el hermano de un hombre muere y deja esposa, y
no deja hijos, que su hermano tome a su esposa y levante descendencia
para su hermano''. \bibleverse{20} Había siete hermanos. El primero tomó
una esposa, y al morir no dejó descendencia. \bibleverse{21} El segundo
la tomó y murió sin dejar descendencia. El tercero hizo lo mismo;
\bibleverse{22} y los siete la tomaron y no dejaron hijos. El último de
todos murió también la mujer. \bibleverse{23} En la resurrección, cuando
resuciten, ¿de quién será ella la esposa de ellos? Porque los siete la
tuvieron como esposa''.

\bibleverse{24} Jesús les contestó: ``¿No es porque estáis equivocados,
al no conocer las Escrituras ni el poder de Dios? \bibleverse{25} Porque
cuando resuciten de entre los muertos, ni se casan ni se dan en
matrimonio, sino que son como ángeles en el cielo. \bibleverse{26} Pero
sobre los muertos, que resucitan, ¿no habéis leído en el libro de Moisés
sobre la Zarza, cómo Dios le habló diciendo: ``Yo soy el Dios de
Abraham, el Dios de Isaac y el Dios de Jacob''?\footnote{\textbf{12:26}
  Éxodo 3:6} \bibleverse{27} No es el Dios de los muertos, sino de los
vivos. Por tanto, estáis muy equivocados''.

\hypertarget{la-pregunta-de-un-escriba-sobre-el-mandamiento-muxe1s-noble}{%
\subsection{La pregunta de un escriba sobre el mandamiento más
noble}\label{la-pregunta-de-un-escriba-sobre-el-mandamiento-muxe1s-noble}}

\bibleverse{28} Uno de los escribas se acercó y los oyó interrogar
juntos, y sabiendo que les había respondido bien, le preguntó: ``¿Cuál
es el mayor de los mandamientos?''

\bibleverse{29} Jesús respondió: ``El más grande es: `Escucha, Israel,
el Señor nuestro Dios, el Señor es uno. \bibleverse{30} Amarás al Señor
tu Dios con todo tu corazón, con toda tu alma, con toda tu mente y con
todas tus fuerzas.\footnote{\textbf{12:30} Deuteronomio 6:4-5} Este es
el primer mandamiento. \bibleverse{31} El segundo es así: `Amarás a tu
prójimo como a ti mismo'.\footnote{\textbf{12:31} Levítico 19:18} No hay
otro mandamiento mayor que éstos''.

\bibleverse{32} El escriba le dijo: ``En verdad, maestro, has dicho bien
que él es uno, y no hay otro sino él; \bibleverse{33} y amarlo con todo
el corazón, con todo el entendimiento, con toda el alma y con todas las
fuerzas, y amar al prójimo como a sí mismo, es más importante que todos
los holocaustos y sacrificios.'' \footnote{\textbf{12:33} 1Sam 15,22; Os
  6,6}

\bibleverse{34} Al ver que respondía con sabiduría, Jesús le dijo: ``No
estás lejos del Reino de Dios''. Después nadie se atrevió a preguntarle
nada. \footnote{\textbf{12:34} Hech 26,27-29}

\hypertarget{la-contrapregunta-de-jesuxfas-sobre-el-mesuxedas-como-hijo-de-david}{%
\subsection{La contrapregunta de Jesús sobre el Mesías como hijo de
David}\label{la-contrapregunta-de-jesuxfas-sobre-el-mesuxedas-como-hijo-de-david}}

\bibleverse{35} Jesús respondió, mientras enseñaba en el templo: ``¿Cómo
es que los escribas dicen que el Cristo es hijo de David? \footnote{\textbf{12:35}
  Is 11,1; Rom 1,3} \bibleverse{36} Porque el mismo David dijo en el
Espíritu Santo `El Señor dijo a mi Señor, ``Siéntate a mi derecha, hasta
que haga de tus enemigos el escabel de tus pies''. \footnote{\textbf{12:36}
  Salmo 110:1} \footnote{\textbf{12:36} 2Sam 23,2}

\bibleverse{37} Por lo tanto, el mismo David lo llama Señor, ¿cómo puede
ser su hijo?'' La gente común le escuchaba con gusto.

\hypertarget{la-advertencia-de-jesuxfas-sobre-la-ambiciuxf3n-y-la-codicia-de-los-escribas}{%
\subsection{La advertencia de Jesús sobre la ambición y la codicia de
los
escribas}\label{la-advertencia-de-jesuxfas-sobre-la-ambiciuxf3n-y-la-codicia-de-los-escribas}}

\bibleverse{38} En su enseñanza les decía: ``Cuídense de los escribas, a
quienes les gusta andar con ropas largas, y recibir saludos en las
plazas, \bibleverse{39} y obtener los mejores asientos en las sinagogas
y los mejores lugares en las fiestas, \bibleverse{40} los que devoran
las casas de las viudas, y por un pretexto hacen largas oraciones. Estos
recibirán mayor condena''. \footnote{\textbf{12:40} Sant 1,27}

\hypertarget{jesuxfas-alaba-las-dos-blancas-de-la-viuda-pobre}{%
\subsection{Jesús alaba las dos blancas de la viuda
pobre}\label{jesuxfas-alaba-las-dos-blancas-de-la-viuda-pobre}}

\bibleverse{41} Jesús se sentó frente al tesoro y vio cómo la multitud
echaba dinero en el tesoro. Muchos ricos echaban mucho. \footnote{\textbf{12:41}
  2Re 12,9} \bibleverse{42} Vino una viuda pobre y echó dos moneditas de
bronce, \footnote{\textbf{12:42} literalmente, lepta (o ácaros de
  viuda). Los lepta son monedas de latón muy pequeñas que valen medio
  cuadrante cada una, que es una cuarta parte del asarion de cobre. Los
  lepta valen menos del 1\% del salario diario de un trabajador
  agrícola.} que equivalen a una moneda de cuadrante. \footnote{\textbf{12:42}
  o, ``¡Yo soy!''} \bibleverse{43} Llamó a sus discípulos y les dijo:
``Os aseguro que esta viuda pobre ha echado más que todos los que echan
en el tesoro, \bibleverse{44} porquetodos han echado de su abundancia,
pero ella, de su pobreza, ha echado todo lo que tenía para vivir.''

\hypertarget{los-primeros-signos-del-fin-de-los-tiempos}{%
\subsection{Los primeros signos del fin de los
tiempos}\label{los-primeros-signos-del-fin-de-los-tiempos}}

\hypertarget{section-12}{%
\section{13}\label{section-12}}

\bibleverse{1} Al salir del templo, uno de sus discípulos le dijo:
``¡Maestro, mira qué piedras y qué edificios!''

\bibleverse{2} Jesús le dijo: ``¿Ves estos grandes edificios? No quedará
aquí una piedra sobre otra que no sea derribada''.

\bibleverse{3} Mientras estaba sentado en el Monte de los Olivos, frente
al templo, Pedro, Santiago, Juan y Andrés le preguntaron en privado:
\footnote{\textbf{13:3} Mat 17,1} \bibleverse{4} ``Dinos, ¿cuándo serán
estas cosas? ¿Cuál es la señal de que todas estas cosas están por
cumplirse?''

\hypertarget{los-primeros-signos-del-fin-de-los-tiempos-1}{%
\subsection{Los primeros signos del fin de los
tiempos}\label{los-primeros-signos-del-fin-de-los-tiempos-1}}

\bibleverse{5} Respondiendo Jesús, comenzó a decirles: ``Tened cuidado
de que nadie os extravíe. \bibleverse{6} Porque vendrán muchos en mi
nombre, diciendo: ``Yo soy'' y engañarán a muchos. \footnote{\textbf{13:6}
  Juan 5,43}

\bibleverse{7} ``Cuando oigáis hablar de guerras y rumores de guerras,
no os preocupéis. Porque es necesario que se produzcan, pero aún no es
el fin. \bibleverse{8} Porque se levantará nación contra nación, y reino
contra reino. Habrá terremotos en varios lugares. Habrá hambres y
problemas. Estas cosas son el comienzo de los dolores de parto.

\hypertarget{la-persecuciuxf3n-de-los-discuxedpulos}{%
\subsection{La persecución de los
discípulos}\label{la-persecuciuxf3n-de-los-discuxedpulos}}

\bibleverse{9} ``Pero vigilad, porque os entregarán a los concilios.
Seréis golpeados en las sinagogas. Estaréis ante gobernantes y reyes por
mi causa, para darles testimonio. \bibleverse{10} Primero hay que
predicar la Buena Nueva a todas las naciones. \footnote{\textbf{13:10}
  Mar 16,15} \bibleverse{11} Cuando os lleven y os entreguen, no os
preocupéis de antemano ni premeditéis lo que vais a decir, sino que
decid lo que se os dé en esa hora. Porque no sois vosotros los que
habláis, sino el Espíritu Santo.

\bibleverse{12} ``El hermano entregará al hermano a la muerte, y el
padre a su hijo. Los hijos se levantarán contra los padres y los harán
morir. \bibleverse{13} Seréis odiados por todos los hombres por causa de
mi nombre, pero el que aguante hasta el final se salvará. \footnote{\textbf{13:13}
  Juan 15,18; Juan 15,21}

\hypertarget{el-cluxedmax-de-la-tribulaciuxf3n-en-judea}{%
\subsection{El clímax de la tribulación en
Judea}\label{el-cluxedmax-de-la-tribulaciuxf3n-en-judea}}

\bibleverse{14} ``Pero cuando veáis que la abominación de la
desolación,\footnote{\textbf{13:14} Daniel 9:17; 11:31; 12:11} de la que
habló el profeta Daniel, está donde no debe estar'' (que el lector
entienda), ``entonces los que estén en Judea huyan a las montañas,
\footnote{\textbf{13:14} Dan 9,27; Dan 11,31} \bibleverse{15} y el que
esté en la azotea no baje ni entre para tomar algo de su casa.
\bibleverse{16} Que el que esté en el campo no regrese para tomar su
manto. \bibleverse{17} Pero ¡ay de las que están embarazadas y de las
que amamantan en esos días! \bibleverse{18} Orad para que su huida no
sea en el invierno. \bibleverse{19} Porque en esos días habrá opresión,
como no la ha habido desde el principio de la creación que Dios creó
hasta ahora, ni la habrá jamás. \footnote{\textbf{13:19} Dan 12,1}
\bibleverse{20} Si el Señor no hubiera acortado los días, ninguna carne
se habría salvado; pero por amor a los elegidos, a quienes escogió,
acortó los días.

\hypertarget{profecuxeda-sobre-los-falsos-profetas}{%
\subsection{Profecía sobre los falsos
profetas}\label{profecuxeda-sobre-los-falsos-profetas}}

\bibleverse{21} Entonces, si alguien les dice: ``Miren, aquí está el
Cristo'' o ``Miren, allí'', no lo crean. \bibleverse{22} Porque se
levantarán falsos cristos y falsos profetas que harán señales y
prodigios, para extraviar, si es posible, incluso a los elegidos.
\bibleverse{23} Pero ustedes vigilen. ``He aquí, os he dicho todas las
cosas de antemano.

\hypertarget{los-uxfaltimos-augurios-y-la-apariciuxf3n-del-hijo-del-hombre-en-el-uxfaltimo-duxeda}{%
\subsection{Los últimos augurios y la aparición del Hijo del Hombre en
el último
día}\label{los-uxfaltimos-augurios-y-la-apariciuxf3n-del-hijo-del-hombre-en-el-uxfaltimo-duxeda}}

\bibleverse{24} Pero en esos días, después de esa opresión, el sol se
oscurecerá, la luna no dará su luz, \bibleverse{25} las estrellas caerán
del cielo, y las potencias que están en los cielos serán sacudidas.
\footnote{\textbf{13:25} Isaías 13:10; 34:4} \footnote{\textbf{13:25}
  Heb 12,26} \bibleverse{26} Entonces verán al Hijo del Hombre venir en
las nubes con gran poder y gloria. \bibleverse{27} Entonces enviará a
sus ángeles y reunirá a sus elegidos de los cuatro vientos, desde los
confines de la tierra hasta los confines del cielo. \footnote{\textbf{13:27}
  Mat 13,41}

\bibleverse{28} ``Ahora, de la higuera, aprended esta parábola. Cuando
la rama ya está tierna y produce sus hojas, sabéis que el verano está
cerca; \bibleverse{29} así también vosotros, cuando veáis que suceden
estas cosas, sabed que está cerca, a las puertas. \bibleverse{30} De
cierto os digo que esta generación\footnote{\textbf{13:30} La palabra
  traducida ``generación'' (genea) también podría traducirse como
  ``raza'', ``familia'' o ``pueblo''.} no pasará hasta que sucedan todas
estas cosas. \bibleverse{31} El cielo y la tierra pasarán, pero mis
palabras no pasarán.

\bibleverse{32} ``Pero de ese día o de esa hora nadie sabe, ni siquiera
los ángeles del cielo, ni el Hijo, sino sólo el Padre.

\hypertarget{exhortaciuxf3n-final-a-los-discuxedpulos-a-estar-alerta}{%
\subsection{Exhortación final a los discípulos a estar
alerta}\label{exhortaciuxf3n-final-a-los-discuxedpulos-a-estar-alerta}}

\bibleverse{33} Velad, estad atentos y orad, porque no sabéis cuándo es
el momento. \footnote{\textbf{13:33} Luc 12,35-40}

\bibleverse{34} ``Es como si un hombre que viaja a otro país, dejara su
casa y diera autoridad a sus siervos, y a cada uno su trabajo, y
ordenara también al portero que vigilara. \bibleverse{35} Velad, pues,
porque no sabéis cuándo vendrá el señor de la casa, si al atardecer, o a
medianoche, o cuando cante el gallo, o por la mañana; \footnote{\textbf{13:35}
  Luc 12,38} \bibleverse{36} no sea que, viniendo de repente, os
encuentre durmiendo. \bibleverse{37} Loque os digo, lo digo a todos:
Velad''.

\hypertarget{intento-de-asesinato-por-parte-de-los-luxedderes-del-pueblo}{%
\subsection{Intento de asesinato por parte de los líderes del
pueblo}\label{intento-de-asesinato-por-parte-de-los-luxedderes-del-pueblo}}

\hypertarget{section-13}{%
\section{14}\label{section-13}}

\bibleverse{1} Faltaban dos días para la Pascua y la Fiesta de los Panes
sin Levadura, y los jefes de los sacerdotes y los escribas buscaban la
manera de apoderarse de él con engaños y matarlo. \bibleverse{2} Pues
decían: ``No durante la fiesta, porque podría haber un disturbio entre
el pueblo''.

\hypertarget{unciuxf3n-de-jesuxfas-en-betania}{%
\subsection{Unción de Jesús en
Betania}\label{unciuxf3n-de-jesuxfas-en-betania}}

\bibleverse{3} Estando en Betania, en casa de Simón el leproso, mientras
estaba sentado a la mesa, llegó una mujer con un frasco de alabastro con
ungüento de nardo puro, muy costoso. Rompió el frasco y lo derramó sobre
su cabeza. \footnote{\textbf{14:3} Juan 12,1-8} \bibleverse{4} Pero
algunos se indignaron entre sí, diciendo: ``¿Por qué se ha desperdiciado
este ungüento? \bibleverse{5} Porque podría haberse vendido por más de
trescientos denarios\footnote{\textbf{14:5} 300 denarios era
  aproximadamente el salario de un año para un trabajador agrícola.} y
haberse dado a los pobres''. Así que refunfuñaron contra ella.

\bibleverse{6} Pero Jesús le dijo: ``Déjala en paz. ¿Por qué la
molestas? Ella ha hecho una buena obra para mí. \bibleverse{7} Porque tú
siempre tienes a los pobres contigo, y cuando quieres les haces un bien;
pero a mí no siempre me tienes. \footnote{\textbf{14:7} Deut 15,11}
\bibleverse{8} Ella ha hecho lo que ha podido. Ha ungido mi cuerpo de
antemano para el entierro. \bibleverse{9} Os aseguro que dondequiera que
se predique esta Buena Noticia en todo el mundo, se hablará también de
lo que ha hecho esta mujer para que quede constancia de ella.''

\hypertarget{traiciuxf3n-de-judas}{%
\subsection{Traición de Judas}\label{traiciuxf3n-de-judas}}

\bibleverse{10} Judas Iscariote, que era uno de los doce, se fue a los
sumos sacerdotes para entregárselo. \bibleverse{11} Ellos, al oírlo, se
alegraron y prometieron darle dinero. Él buscó la manera de entregarlo
convenientemente.

\hypertarget{preparaciuxf3n-de-la-comida-pascual}{%
\subsection{Preparación de la comida
pascual}\label{preparaciuxf3n-de-la-comida-pascual}}

\bibleverse{12} El primer día de los panes sin levadura, cuando
sacrificaban la Pascua, sus discípulos le preguntaron: ``¿Dónde quieres
que vayamos a preparar para que comáis la Pascua?''

\bibleverse{13} Envió a dos de sus discípulos y les dijo: ``Id a la
ciudad, y allí os saldrá al encuentro un hombre con un cántaro de agua.
Seguidle, \bibleverse{14} y dondequiera que entre, decid al dueño de la
casa: ``El Maestro dice: ``¿Dónde está la sala de invitados, donde pueda
comer la Pascua con mis discípulos?''\,''. \footnote{\textbf{14:14} Mar
  11,3} \bibleverse{15} Él mismo te mostrará una gran habitación
superior amueblada y preparada. Prepáranos allí''.

\bibleverse{16} Sus discípulos salieron y entraron en la ciudad, y
encontraron las cosas como él les había dicho, y prepararon la Pascua.

\hypertarget{la-uxfaltima-cena-de-jesuxfas-en-el-cuxedrculo-de-los-discuxedpulos-anuncio-de-la-traiciuxf3n-de-judas-instituciuxf3n-de-la-santa-comuniuxf3n}{%
\subsection{La última cena de Jesús en el círculo de los discípulos;
Anuncio de la traición de Judas; Institución de la santa
comunión}\label{la-uxfaltima-cena-de-jesuxfas-en-el-cuxedrculo-de-los-discuxedpulos-anuncio-de-la-traiciuxf3n-de-judas-instituciuxf3n-de-la-santa-comuniuxf3n}}

\bibleverse{17} Al anochecer llegó con los doce. \bibleverse{18}
Mientras estaban sentados y comiendo, Jesús dijo: ``Os aseguro que uno
de vosotros me va a traicionar: el que come conmigo.'' \footnote{\textbf{14:18}
  Juan 13,21-26}

\bibleverse{19} Comenzaron a entristecerse y a preguntarle uno por uno:
``¿Seguro que no soy yo?''. Y otro decía: ``¿Seguro que no soy yo?''

\bibleverse{20} Él les respondió: ``Es uno de los doce, el que moja
conmigo en el plato. \bibleverse{21} Porque el Hijo del Hombre va como
está escrito de él, pero ¡ay de aquel hombre por quien el Hijo del
Hombre es entregado! Más le valdría a ese hombre no haber nacido''.

\bibleverse{22} Mientras comían, Jesús tomó el pan y, después de
bendecirlo, lo partió y les dijo: ``Tomad, comed. Esto es mi cuerpo''.
\footnote{\textbf{14:22} 1Cor 11,23-25}

\bibleverse{23} Tomó el cáliz y, después de dar gracias, se lo dio a
ellos. Todos bebieron de ella. \bibleverse{24} Les dijo: ``Esta es mi
sangre del nuevo pacto, que se derrama por muchos. \footnote{\textbf{14:24}
  Heb 9,15-16} \bibleverse{25} De cierto os digo que no beberé más del
fruto de la vid hasta el día en que lo beba de nuevo en el Reino de
Dios.''

\hypertarget{camina-a-getsemanuxed}{%
\subsection{Camina a Getsemaní}\label{camina-a-getsemanuxed}}

\bibleverse{26} Después de cantar un himno, salieron al Monte de los
Olivos. \footnote{\textbf{14:26} Sal 113,1-118}

\bibleverse{27} Jesús les dijo: ``Esta noche todos vosotros tropezaréis
por mi culpa, porque está escrito: ``Heriré al pastor y las ovejas se
dispersarán''. \footnote{\textbf{14:27} Zacarías 13:7} \footnote{\textbf{14:27}
  Juan 16,32} \bibleverse{28} Sin embargo, cuando haya resucitado, iré
delante de vosotros a Galilea''. \footnote{\textbf{14:28} Mar 16,7}

\bibleverse{29} Pero Pedro le dijo: ``Aunque todos se ofendan, yo no''.

\bibleverse{30} Jesús le dijo: ``Muy ciertamente te digo que hoy,
incluso esta noche, antes de que el gallo cante dos veces, me negarás
tres veces.'' \footnote{\textbf{14:30} Juan 13,36-38}

\bibleverse{31} Pero él habló aún más: ``Si tengo que morir con
vosotros, no os negaré''. Todos dijeron lo mismo.

\hypertarget{el-conflicto-y-la-oraciuxf3n-de-jesuxfas-en-getsemanuxed-debilidad-de-los-discuxedpulos}{%
\subsection{El conflicto y la oración de Jesús en Getsemaní; Debilidad
de los
discípulos}\label{el-conflicto-y-la-oraciuxf3n-de-jesuxfas-en-getsemanuxed-debilidad-de-los-discuxedpulos}}

\bibleverse{32} Llegaron a un lugar que se llama Getsemaní. Dijo a sus
discípulos: ``Sentaos aquí mientras oro''. \bibleverse{33} Tomó consigo
a Pedro, a Santiago y a Juan, y comenzó a estar muy preocupado y
angustiado. \footnote{\textbf{14:33} Mat 17,1} \bibleverse{34} Les dijo:
``Mi alma está muy triste, hasta la muerte. Quedaos aquí y velad''.
\footnote{\textbf{14:34} Juan 12,27}

\bibleverse{35} Se adelantó un poco, se postró en el suelo y oró para
que, si era posible, la hora pasara de largo. \bibleverse{36} Dijo:
``Abba, \footnote{\textbf{14:36} Abba es una grafía griega de la palabra
  aramea que significa ``Padre'' o ``Papá'', utilizada de forma
  familiar, respetuosa y cariñosa.} Padre, todo es posible para ti. Por
favor, aparta de mí esta copa. Pero no lo que yo deseo, sino lo que tú
deseas''. \footnote{\textbf{14:36} Mar 10,38}

\bibleverse{37} Llegó y los encontró durmiendo, y dijo a Pedro: ``Simón,
¿duermes? ¿No podías velar una hora? \bibleverse{38} Velad y orad, para
que no entréis en tentación. El espíritu, en efecto, está dispuesto,
pero la carne es débil''.

\bibleverse{39} De nuevo se fue y oró diciendo las mismas palabras.
\bibleverse{40} Volvió y los encontró durmiendo, pues sus ojos estaban
muy cargados; y no sabían qué responderle. \bibleverse{41} Llegó por
tercera vez y les dijo: ``Dormid ya y descansad. Ya es suficiente. La
hora ha llegado. He aquí que el Hijo del Hombre ha sido entregado en
manos de los pecadores. \bibleverse{42} ¡Levántate! Pongámonos en
marcha. He aquí, el que me traiciona está cerca''.

\hypertarget{encarcelamiento-de-jesuxfas-escape-de-los-discuxedpulos}{%
\subsection{Encarcelamiento de Jesús; Escape de los
discípulos}\label{encarcelamiento-de-jesuxfas-escape-de-los-discuxedpulos}}

\bibleverse{43} En seguida, mientras aún hablaba, vino Judas, uno de los
doce, y con él una multitud con espadas y palos, de parte de los sumos
sacerdotes, de los escribas y de los ancianos. \bibleverse{44} Y el que
le entregaba les había dado una señal, diciendo: ``Al que yo bese, ése
es. Agarradle y llevadle con seguridad''. \bibleverse{45} Cuando llegó,
enseguida se acercó a él y le dijo: ``¡Rabí! Rabí!'' y le besó.
\bibleverse{46} Le pusieron las manos encima y le agarraron.
\bibleverse{47} Pero uno de los que estaban allí sacó su espada e hirió
al siervo del sumo sacerdote y le cortó la oreja.

\bibleverse{48} Jesús les respondió: ``¿Habéis salido, como contra un
ladrón, con espadas y palos para prenderme? \bibleverse{49} Cada día
estaba con vosotros en el templo enseñando, y no me habéis arrestado.
Pero esto es para que se cumplan las Escrituras''.

\bibleverse{50} Todos le dejaron y huyeron. \bibleverse{51} Cierto joven
lo siguió, teniendo una tela de lino echada alrededor de su cuerpo
desnudo. Los jóvenes lo agarraron, \bibleverse{52} pero él dejó el
lienzo y huyó de ellos desnudo.

\hypertarget{el-interrogatorio-la-confesiuxf3n-y-la-condena-de-jesuxfas-ante-el-sumo-sacerdote-y-el-concilio}{%
\subsection{El interrogatorio, la confesión y la condena de Jesús ante
el sumo sacerdote y el
concilio}\label{el-interrogatorio-la-confesiuxf3n-y-la-condena-de-jesuxfas-ante-el-sumo-sacerdote-y-el-concilio}}

\bibleverse{53} Llevaron a Jesús ante el sumo sacerdote. Todos los jefes
de los sacerdotes, los ancianos y los escribas se reunieron con él.

\bibleverse{54} Pedro le había seguido de lejos, hasta que llegó al
patio del sumo sacerdote. Estaba sentado con los oficiales, y se
calentaba a la luz del fuego. \bibleverse{55} Los jefes de los
sacerdotes y todo el consejo buscaban testigos contra Jesús para
condenarlo a muerte, pero no los encontraron. \bibleverse{56} Porque
muchos daban falso testimonio contra él, y sus testimonios no
concordaban entre sí. \bibleverse{57} Algunos se levantaron y dieron
falso testimonio contra él, diciendo: \bibleverse{58} ``Le oímos decir:
`Destruiré este templo hecho a mano, y en tres días construiré otro
hecho sin manos'.'' \footnote{\textbf{14:58} Juan 2,19-21}
\bibleverse{59} Aun así, su testimonio no concordaba.

\bibleverse{60} El sumo sacerdote se levantó en medio y preguntó a
Jesús: ``¿No tienes respuesta? ¿Qué es lo que éstos testifican contra
ti?'' \bibleverse{61} Pero él se quedó callado y no respondió nada. De
nuevo el sumo sacerdote le preguntó: ``¿Eres tú el Cristo, el Hijo del
Bendito?'' \footnote{\textbf{14:61} Mar 15,5; Is 53,7}

\bibleverse{62} Jesús dijo: ``Yo soy. Veréis al Hijo del Hombre sentado
a la derecha del Poder, y viniendo con las nubes del cielo''.
\footnote{\textbf{14:62} Dan 7,13-14}

\bibleverse{63} El sumo sacerdote se rasgó las vestiduras y dijo: ``¿Qué
más necesidad tenemos de testigos? \bibleverse{64} ¡Habéis oído la
blasfemia! ¿Qué os parece?'' Todos le condenaron a ser digno de muerte.
\footnote{\textbf{14:64} Juan 19,7} \bibleverse{65} Algunos empezaron a
escupirle, a cubrirle la cara, a golpearle con los puños y a decirle:
``¡Profeta!''. Los oficiales le golpearon con las palmas de las manos.

\hypertarget{negaciuxf3n-y-arrepentimiento-de-pedro}{%
\subsection{Negación y arrepentimiento de
Pedro}\label{negaciuxf3n-y-arrepentimiento-de-pedro}}

\bibleverse{66} Mientras Pedro estaba en el patio de abajo, se acercó
una de las criadas del sumo sacerdote, \bibleverse{67} y al ver que
Pedro se calentaba, lo miró y le dijo: ``¡Tú también estabas con el
nazareno, Jesús!''

\bibleverse{68} Pero él lo negó, diciendo: ``No sé ni entiendo lo que
dices''. Salió a la entrada; y cantó el gallo.

\bibleverse{69} La criada lo vio y comenzó a decir de nuevo a los que
estaban allí: ``Este es uno de ellos''. \bibleverse{70} Pero él volvió a
negarlo. Al cabo de un rato, los que estaban allí volvieron a decir a
Pedro: ``Verdaderamente eres uno de ellos, pues eres galileo, y tu forma
de hablar lo demuestra.'' \bibleverse{71} Pero él comenzó a maldecir y a
jurar: ``¡No conozco a ese hombre del que habláis!''

\bibleverse{72} El gallo cantó por segunda vez. Pedro recordó las
palabras que le dijo Jesús: ``Antes de que cante el gallo dos veces, me
negarás tres''. Cuando pensó en eso, lloró.

\hypertarget{el-interrogatorio-de-jesuxfas-ante-el-gobernador-romano-poncio-pilato-su-condenaciuxf3n-y-flagelaciuxf3n}{%
\subsection{El interrogatorio de Jesús ante el gobernador romano Poncio
Pilato; su condenación y
flagelación}\label{el-interrogatorio-de-jesuxfas-ante-el-gobernador-romano-poncio-pilato-su-condenaciuxf3n-y-flagelaciuxf3n}}

\hypertarget{section-14}{%
\section{15}\label{section-14}}

\bibleverse{1} Por la mañana, los jefes de los sacerdotes, con los
ancianos, los escribas y todo el consejo, celebraron una consulta,
ataron a Jesús, lo llevaron y lo entregaron a Pilato. \bibleverse{2}
Pilato le preguntó: ``¿Eres tú el Rey de los judíos?'' Respondió: ``Eso
dices tú''.

\bibleverse{3} Los jefes de los sacerdotes le acusaron de muchas cosas.
\bibleverse{4} Pilato volvió a preguntarle: ``¿No tienes respuesta? Mira
cuántas cosas declaran contra ti''.

\bibleverse{5} Pero Jesús no respondió más, por lo que Pilato se
maravilló. \footnote{\textbf{15:5} Mar 14,61; Is 53,7}

\bibleverse{6} En la fiesta solía liberar a un prisionero, cualquiera
que pidiesen. \bibleverse{7} Había uno llamado Barrabás, atado con sus
compañeros de insurrección, hombres que en la insurrección habían
cometido un asesinato. \bibleverse{8} La multitud, gritando, comenzó a
pedirle que hiciera lo que siempre hacía por ellos. \bibleverse{9}
Pilato les respondió diciendo: ``¿Queréis que os suelte al Rey de los
judíos?'' \bibleverse{10} Porque se dio cuenta de que por envidia los
jefes de los sacerdotes lo habían entregado. \footnote{\textbf{15:10}
  Juan 11,48} \bibleverse{11} Pero los jefes de los sacerdotes incitaron
a la multitud para que les soltara a Barrabás en su lugar.
\bibleverse{12} Pilato volvió a preguntarles: ``¿Qué debo hacer, pues,
con el que llamáis Rey de los judíos?''

\bibleverse{13} Volvieron a gritar: ``¡Crucifícalo!''

\bibleverse{14} Pilato les dijo: ``¿Qué mal ha hecho?'' Pero ellos
gritaron con fuerza: ``¡Crucifícalo!''

\bibleverse{15} Pilato, queriendo complacer a la multitud, les soltó a
Barrabás y entregó a Jesús, después de haberlo azotado, para que fuera
crucificado.

\hypertarget{la-burla-y-el-maltrato-de-jesuxfas-por-parte-de-los-soldados-romanos}{%
\subsection{La burla y el maltrato de Jesús por parte de los soldados
romanos}\label{la-burla-y-el-maltrato-de-jesuxfas-por-parte-de-los-soldados-romanos}}

\bibleverse{16} Los soldados lo llevaron dentro del patio, que es el
pretorio, y convocaron a toda la cohorte. \bibleverse{17} Lo vistieron
de púrpura y le pusieron una corona de espinas. \bibleverse{18}
Comenzaron a saludarlo: ``¡Salve, rey de los judíos!'' \bibleverse{19}
Le golpearon la cabeza con una caña, le escupieron y, doblando las
rodillas, le rindieron homenaje.

\hypertarget{el-curso-de-la-muerte-de-jesuxfas-despuuxe9s-del-guxf3lgota-su-crucifixiuxf3n-y-su-muerte}{%
\subsection{El curso de la muerte de Jesús después del Gólgota, su
crucifixión y su
muerte}\label{el-curso-de-la-muerte-de-jesuxfas-despuuxe9s-del-guxf3lgota-su-crucifixiuxf3n-y-su-muerte}}

\bibleverse{20} Cuando se burlaron de él, le quitaron el manto de
púrpura y le pusieron sus propios vestidos. Lo llevaron para
crucificarlo.

\bibleverse{21} Obligaron a uno que pasaba por allí, procedente del
campo, Simón de Cirene, padre de Alejandro y de Rufo, a ir con ellos
para que llevara su cruz. \footnote{\textbf{15:21} Rom 16,13}
\bibleverse{22} Le llevaron al lugar llamado Gólgota, que es, según la
interpretación, ``El lugar de la calavera''. \bibleverse{23} Le
ofrecieron de beber vino mezclado con mirra, pero no lo tomó.
\footnote{\textbf{15:23} Sal 69,21}

\bibleverse{24} Al crucificarlo, se repartieron sus vestidos, echando a
suertes lo que debía tomar cada uno. \footnote{\textbf{15:24} Sal 22,18}
\bibleverse{25} Era la hora\footnote{\textbf{15:25} 09:00 h.} tercera
cuando lo crucificaron. \bibleverse{26} Sobre él estaba escrita la
superposición de su acusación: ``EL REY DE LOS JUDÍOS''. \bibleverse{27}
Con él crucificaron a dos ladrones, uno a su derecha y otro a su
izquierda. \bibleverse{28} Se cumplió la Escritura que dice: ``Fue
contado con los transgresores''. \footnote{\textbf{15:28} NU omite el
  versículo 28.}

\bibleverse{29} Los que pasaban por allí le blasfemaban, moviendo la
cabeza y diciendo: ``¡Ja! Tú que destruyes el templo y lo construyes en
tres días, \footnote{\textbf{15:29} Mar 14,58} \bibleverse{30} sálvate a
ti mismo y baja de la cruz''.

\bibleverse{31} Asimismo, también los jefes de los sacerdotes,
burlándose entre ellos con los escribas, decían: ``Ha salvado a otros.
No puede salvarse a sí mismo. \bibleverse{32} Que baje ahora de la cruz
el Cristo, el Rey de Israel, para que le veamos y le
creamos.''\footnote{\textbf{15:32} TR omite ``él''} Los que estaban
crucificados con él también le insultaban. \footnote{\textbf{15:32} Mat
  16,1; Mat 16,4}

\hypertarget{la-muerte-de-jesuxfas-el-signo-milagroso-de-su-muerte}{%
\subsection{La muerte de Jesús; el signo milagroso de su
muerte}\label{la-muerte-de-jesuxfas-el-signo-milagroso-de-su-muerte}}

\bibleverse{33} Cuando llegó la hora\footnote{\textbf{15:33} o, mediodía}
sexta, hubo oscuridad sobre toda la tierra hasta la hora novena.
\footnote{\textbf{15:33} 15:00 h.} \bibleverse{34} A la hora novena,
Jesús clamó a gran voz, diciendo: ``Eloi, Eloi, lama sabachthani?'', que
es, interpretado, ``Dios mío, Dios mío, ¿por qué me has abandonado?''
\footnote{\textbf{15:34} Salmo 22:1} \footnote{\textbf{15:34} Sal 22,1}

\bibleverse{35} Algunos de los que estaban allí, al oírlo, dijeron: ``He
aquí que llama a Elías''.

\bibleverse{36} Uno corrió y, llenando una esponja de vinagre, la puso
en una caña y se la dio a beber, diciendo: ``Déjalo. A ver si viene
Elías a bajarlo''.

\bibleverse{37} Jesús gritó con fuerza y entregó el espíritu.
\bibleverse{38} El velo del templo se rasgó en dos desde arriba hasta
abajo. \bibleverse{39} Cuando el centurión, que estaba frente a él, vio
que gritaba así y exhalaba, dijo: ``¡Verdaderamente este hombre era el
Hijo de Dios!''

\bibleverse{40} Había también mujeres que miraban desde lejos, entre las
cuales estaban María Magdalena y María la madre de Santiago el Menor y
de José, y Salomé; \footnote{\textbf{15:40} Luc 8,2-3} \bibleverse{41}
las cuales, estando él en Galilea, le seguían y le servían; y otras
muchas que subieron con él a Jerusalén.

\hypertarget{entierro-de-jesuxfas}{%
\subsection{Entierro de Jesús}\label{entierro-de-jesuxfas}}

\bibleverse{42} Cuando llegó la noche, por ser el día de la preparación,
es decir, la víspera del sábado, \bibleverse{43} vino José de Arimatea,
miembro destacado del consejo, que también buscaba el Reino de Dios.
Entró audazmente a Pilato y pidió el cuerpo de Jesús. \bibleverse{44}
Pilato se sorprendió al oír que ya estaba muerto; y llamando al
centurión, le preguntó si llevaba mucho tiempo muerto. \bibleverse{45}
Al enterarse por el centurión, concedió el cuerpo a José.
\bibleverse{46} Compró un lienzo y, bajándolo, lo envolvió en el lienzo
y lo depositó en un sepulcro excavado en una roca. Hizo rodar una piedra
contra la puerta del sepulcro. \bibleverse{47} María Magdalena y María,
la madre de Josés, vieron dónde estaba depositado.

\hypertarget{descubrimiento-de-la-tumba-vacuxeda-en-la-mauxf1ana-de-pascua-la-revelaciuxf3n-del-uxe1ngel-a-las-mujeres}{%
\subsection{Descubrimiento de la tumba vacía en la mañana de Pascua; la
revelación del ángel a las
mujeres}\label{descubrimiento-de-la-tumba-vacuxeda-en-la-mauxf1ana-de-pascua-la-revelaciuxf3n-del-uxe1ngel-a-las-mujeres}}

\hypertarget{section-15}{%
\section{16}\label{section-15}}

\bibleverse{1} Cuando pasó el sábado, María Magdalena, María la madre de
Santiago y Salomé compraron especias para ir a ungirlo. \bibleverse{2}
El primer día de la semana, muy temprano, llegaron al sepulcro cuando ya
había salido el sol. \bibleverse{3} Decían entre ellas: ``¿Quién nos
quitará la piedra de la puerta del sepulcro?'' \bibleverse{4} porque era
muy grande. Al levantar la vista, vieron que la piedra había sido
removida.

\bibleverse{5} Al entrar en el sepulcro, vieron a un joven sentado a la
derecha, vestido con una túnica blanca; y se asombraron. \bibleverse{6}
Él les dijo: ``No os asombréis. Buscáis a Jesús, el Nazareno, que ha
sido crucificado. Ha resucitado. El no está aquí. Ved el lugar donde lo
han puesto. \bibleverse{7} Pero id y decid a sus discípulos y a Pedro:
``Va delante de vosotros a Galilea. Allí le veréis, como os ha
dicho'\,''. \footnote{\textbf{16:7} Mar 14,28}

\bibleverse{8} Salieron \footnote{\textbf{16:8} TR añade ``rápidamente''}
y huyeron del sepulcro, porque les había invadido el temor y el asombro.
No dijeron nada a nadie, porque tenían miedo. \footnote{\textbf{16:8} Un
  manuscrito aislado omite los versículos 9-20, pero añade este ``breve
  final de Marcos'' al final del versículo 8: Contaron brevemente todo
  lo que se les había ordenado a los que estaban alrededor de Pedro.
  Después, Jesús mismo los envió, de este a oeste, con el sagrado e
  imperecedero anuncio de la salvación eterna.}

\hypertarget{jesuxfas-se-aparece-a-maruxeda-magdalena-y-a-los-dos-discuxedpulos-de-emauxfas}{%
\subsection{Jesús se aparece a María Magdalena y a los dos discípulos de
Emaús}\label{jesuxfas-se-aparece-a-maruxeda-magdalena-y-a-los-dos-discuxedpulos-de-emauxfas}}

\bibleverse{9} \footnote{\textbf{16:9} NU incluye el texto de los
  versículos 9-20, pero menciona en una nota a pie de página que algunos
  manuscritos lo omitieron. Los traductores de la World English Bible
  consideran que Marcos 16:9-20 es fiable, basándose en una abrumadora
  mayoría de pruebas textuales, entre las que se incluyen no sólo el
  autorizado Nuevo Testamento del Texto Griego Mayoritario, sino también
  el TR y muchos de los manuscritos citados en el texto NU.} El primer
día de la semana, cuando se levantó temprano, se apareció primero a
María Magdalena, de quien había expulsado siete demonios. \footnote{\textbf{16:9}
  Luc 8,2; Juan 20,11-18} \bibleverse{10} Ella fue a contárselo a los
que habían estado con él, mientras se lamentaban y lloraban.
\bibleverse{11} Cuando oyeron que estaba vivo y que había sido visto por
ella, no creyeron.

\bibleverse{12} Después de estas cosas, se les reveló en otra forma a
dos de ellos mientras caminaban, de camino al campo. \footnote{\textbf{16:12}
  Luc 24,13-35} \bibleverse{13} Se fueron y lo contaron a los demás.
Ellos tampoco les creyeron.

\hypertarget{la-apariciuxf3n-de-jesuxfas-a-los-once-apuxf3stoles-y-su-mandato-misional}{%
\subsection{La aparición de Jesús a los once apóstoles y su mandato
misional}\label{la-apariciuxf3n-de-jesuxfas-a-los-once-apuxf3stoles-y-su-mandato-misional}}

\bibleverse{14} Después se reveló a los mismos once, mientras estaban
sentados a la mesa; y les reprendió por su incredulidad y dureza de
corazón, porque no creían a los que le habían visto después de
resucitado. \footnote{\textbf{16:14} 1Cor 15,5} \bibleverse{15} Les
dijo: ``Id por todo el mundo y predicad la Buena Nueva a toda la
creación. \footnote{\textbf{16:15} Mar 13,10; Mat 28,18-20}
\bibleverse{16} El que crea y se bautice se salvará; pero el que no crea
se condenará. \footnote{\textbf{16:16} Hech 2,38; Hech 16,31; Hech 16,33}
\bibleverse{17} Estas señales acompañarán a los que crean: en mi nombre
expulsarán a los demonios; hablarán con nuevas lenguas; \footnote{\textbf{16:17}
  Hech 16,18; Hech 10,46; Hech 19,6} \bibleverse{18} cogerán serpientes;
y si beben cualquier cosa mortífera, no les hará ningún daño; impondrán
las manos a los enfermos, y sanarán.'' \footnote{\textbf{16:18} Luc
  10,19; Hech 28,3-6; Sant 5,14; Sant 1,5-15}

\hypertarget{ascensiuxf3n-de-jesuxfas}{%
\subsection{Ascensión de Jesús}\label{ascensiuxf3n-de-jesuxfas}}

\bibleverse{19} Entonces el Señor, \footnote{\textbf{16:19} NU añade
  ``Jesús''} después de hablarles, fue recibido en el cielo y se sentó a
la derecha de Dios. \footnote{\textbf{16:19} Sal 110,1; Hech 1,2; Hech
  7,55} \bibleverse{20} Ellos salieron y predicaron por todas partes,
colaborando el Señor con ellos y confirmando la palabra con las señales
que se producían. Amén. \footnote{\textbf{16:20} Hech 14,3; Heb 2,4}
