\hypertarget{jesuxfas-como-el-verbo-hecho-hombre}{%
\subsection{Jesús como el ``Verbo'' hecho
hombre}\label{jesuxfas-como-el-verbo-hecho-hombre}}

\hypertarget{section}{%
\section{1}\label{section}}

\bibleverse{1} En el principio era el Verbo, y el Verbo estaba con Dios,
y el Verbo era Dios. \footnote{\textbf{1:1} Gén 1,1; 1Jn 1,1-2; Juan
  17,5; Apoc 19,13} \bibleverse{2} El mismo estaba en el principio con
Dios. \bibleverse{3} Todas las cosas fueron hechas por medio de él. Sin
él no se hizo nada de lo que se ha hecho. \footnote{\textbf{1:3} 1Cor
  8,6; Col 1,16-17; Heb 1,2} \bibleverse{4} En él estaba la vida, y la
vida era la luz de los hombres. \footnote{\textbf{1:4} Juan 8,12}
\bibleverse{5} La luz brilla en las tinieblas, y las tinieblas no la han
vencido. \footnote{\textbf{1:5} Juan 3,19}

\bibleverse{6} Vino un hombre enviado por Dios, que se llamaba Juan.
\footnote{\textbf{1:6} Mat 3,1; Mar 1,4} \bibleverse{7} Este vino como
testigo, para dar testimonio de la luz, a fin de que todos creyeran por
medio de él. \footnote{\textbf{1:7} Hech 19,4} \bibleverse{8} Él no era
la luz, sino que fue enviado para dar testimonio de la luz.
\bibleverse{9} Existía la verdadera luz que ilumina a todo hombre, venía
a este mundo.

\bibleverse{10} Estaba en el mundo, y el mundo fue hecho por medio de
él, y el mundo no lo reconoció. \bibleverse{11} Vino a los suyos, y los
suyos no le recibieron. \bibleverse{12} Pero a todos los que le
recibieron, les dio el derecho de ser hijos de Dios, a los que creen en
su nombre: \footnote{\textbf{1:12} Gal 3,26} \bibleverse{13} que no
nacieron de la sangre, ni de la voluntad de la carne, ni de la voluntad
del hombre, sino de Dios. \footnote{\textbf{1:13} Juan 3,5-6}

\bibleverse{14} El Verbo se hizo carne y vivió entre nosotros. Vimos su
gloria, una gloria como la del Hijo unigénito del Padre, lleno de gracia
y de verdad. \footnote{\textbf{1:14} Is 7,14; Is 60,1; 2Pe 1,16-17}
\bibleverse{15} Juan dio testimonio de él. Gritó diciendo: ``Este era
aquel de quien dije: ``El que viene después de mí me ha superado, porque
era antes que yo''\,''. \bibleverse{16} De su plenitud todos hemos
recibido gracia sobre gracia. \footnote{\textbf{1:16} Juan 3,34; Col
  1,19} \bibleverse{17} Porque la ley fue dada por medio de Moisés. La
gracia y la verdad se realizaron por medio de Jesucristo. \footnote{\textbf{1:17}
  Rom 10,4} \bibleverse{18} Nadie ha visto a Dios en ningún momento. El
Hijo único, que está en el seno del Padre, lo ha declarado. \footnote{\textbf{1:18}
  Juan 6,46; Mat 11,27}

\hypertarget{el-testimonio-de-suxed-mismo-del-bautista}{%
\subsection{El testimonio de sí mismo del
Bautista}\label{el-testimonio-de-suxed-mismo-del-bautista}}

\bibleverse{19} Este es el testimonio de Juan, cuando los judíos
enviaron sacerdotes y levitas de Jerusalén para preguntarle: ``¿Quién
eres tú?''

\bibleverse{20} Declaró, y no negó, pero declaró: ``Yo no soy el
Cristo''.

\bibleverse{21} Le preguntaron: ``¿Entonces qué? ¿Eres tú Elías?'' Él
dijo: ``No lo soy''. ``¿Eres el profeta?'' Él respondió: ``No''.
\footnote{\textbf{1:21} Mat 17,10-13; Deut 18,15; Mal 4,5}

\bibleverse{22} Le dijeron entonces: ``¿Quién eres tú? Danos una
respuesta para llevarla a los que nos han enviado. ¿Qué dices de ti
mismo?''

\bibleverse{23} Dijo: ``Soy la voz del que clama en el desierto:
``Enderezad el camino del Señor'', como dijo el profeta Isaías''.

\bibleverse{24} Los enviados eran de los fariseos. \bibleverse{25} Le
preguntaron: ``¿Por qué, pues, bautizas si no eres el Cristo, ni Elías,
ni el profeta?''.

\bibleverse{26} Juan les respondió: ``Yo bautizo en agua, pero entre
vosotros hay uno que no conocéis. \footnote{\textbf{1:26} Luc 17,21}
\bibleverse{27} Él es el que viene después de mí, el que es preferido
antes que yo, cuya correa de sandalia no soy digno de desatar.''
\bibleverse{28} Estas cosas sucedieron en Betania, al otro lado del
Jordán, donde Juan bautizaba.

\hypertarget{el-testimonio-del-bautista-acerca-de-jesuxfas}{%
\subsection{El testimonio del Bautista acerca de
Jesús}\label{el-testimonio-del-bautista-acerca-de-jesuxfas}}

\bibleverse{29} Al día siguiente, vio a Jesús que se acercaba a él, y
dijo: ``¡He aquí el Cordero de Dios, que quita el pecado del mundo!
\footnote{\textbf{1:29} Is 53,7} \bibleverse{30} Este es aquel de quien
dije: ``Después de mí viene un hombre que es preferido antes que yo,
porque fue antes que yo''. \bibleverse{31} Yo no lo conocía, pero por
eso vine a bautizar en agua, para que fuera revelado a Israel.''
\bibleverse{32} Juan dio testimonio diciendo: ``He visto al Espíritu
descender del cielo como una paloma, y permaneció sobre él.
\bibleverse{33} Yo no lo reconocí, pero el que me envió a bautizar en
agua me dijo: `Sobre quien veas descender el Espíritu y permanecer sobre
él es el que bautiza en el Espíritu Santo'. \bibleverse{34} He visto y
he dado testimonio de que éste es el Hijo de Dios''.

\bibleverse{35} Al día siguiente, Juan estaba de pie con dos de sus
discípulos, \bibleverse{36} y mirando a Jesús mientras caminaba, dijo:
``¡He aquí el Cordero de Dios!'' \bibleverse{37} Los dos discípulos le
oyeron hablar y siguieron a Jesús. \bibleverse{38} Jesús se volvió y, al
ver que le seguían, les dijo: ``¿Qué buscáis?'' Le dijeron: ``Rabí''
(que se interpreta como Maestro), ``¿dónde te alojas?''.

\bibleverse{39} Les dijo: ``Vengan y vean''. Vinieron y vieron dónde se
alojaba, y se quedaron con él ese día. Era como la hora décima.
\bibleverse{40} Uno de los que oyeron a Juan y le siguieron fue Andrés,
hermano de Simón Pedro. \footnote{\textbf{1:40} Mat 4,18-20}
\bibleverse{41} Este encontró primero a su propio hermano, Simón, y le
dijo: ``¡Hemos encontrado al Mesías!'' (que es, interpretado, Cristo).
\bibleverse{42} Lo llevó a Jesús. Jesús lo miró y le dijo: ``Tú eres
Simón, hijo de Jonás. Seras llamado Cefas'' (que es, por interpretación,
Pedro). \footnote{\textbf{1:42} Mat 16,18}

\bibleverse{43} Al día siguiente, decidido a salir a Galilea, encontró a
Felipe. Jesús le dijo: ``Sígueme''. \bibleverse{44} Felipe era de
Betsaida, la ciudad de Andrés y Pedro. \bibleverse{45} Felipe encontró a
Natanael y le dijo: ``Hemos encontrado a aquel de quien escribió Moisés
en la ley y también los profetas: Jesús de Nazaret, hijo de José''.
\footnote{\textbf{1:45} Deut 18,18; Is 53,2; Jer 23,5; Ezeq 34,23}

\bibleverse{46} Natanael le dijo: ``¿Puede salir algo bueno de
Nazaret?'' Felipe le dijo: ``Ven a ver''. \footnote{\textbf{1:46} Juan
  7,41}

\bibleverse{47} Jesús vio que Natanael se acercaba a él, y dijo de él:
``¡He aquí un verdadero israelita, en quien no hay engaño!''

\bibleverse{48} Natanael le dijo: ``¿De qué me conoces?'' Jesús le
respondió: ``Antes de que Felipe te llamara, cuando estabas debajo de la
higuera, te vi''.

\bibleverse{49} Natanael le respondió: ``¡Rabí, tú eres el Hijo de Dios!
Tú eres el Rey de Israel''. \footnote{\textbf{1:49} Sal 2,7; Jer 23,5;
  Juan 6,69; Mat 14,33; Mat 16,16}

\bibleverse{50} Jesús le respondió: ``Porque te he dicho que te he visto
debajo de la higuera, ¿crees? Verás cosas más grandes que éstas''.
\bibleverse{51} Le dijo: ``Te aseguro que después verás el cielo abierto
y a los ángeles de Dios subiendo y bajando sobre el Hijo del Hombre.''
\footnote{\textbf{1:51} Gén 28,12; Mat 4,11}

\hypertarget{la-primera-seuxf1al-milagrosa-de-jesuxfas-en-las-bodas-de-canuxe1}{%
\subsection{La primera señal milagrosa de Jesús en las bodas de
Caná}\label{la-primera-seuxf1al-milagrosa-de-jesuxfas-en-las-bodas-de-canuxe1}}

\hypertarget{section-1}{%
\section{2}\label{section-1}}

\bibleverse{1} Al tercer día, hubo una boda en Caná de Galilea. La madre
de Jesús estaba allí. \bibleverse{2} También Jesús fue invitado, con sus
discípulos, a la boda. \bibleverse{3} Cuando se acabó el vino, la madre
de Jesús le dijo: ``No tienen vino''.

\bibleverse{4} Jesús le dijo: ``Mujer, ¿qué tiene que ver eso contigo y
conmigo? Todavía no ha llegado mi hora''. \footnote{\textbf{2:4} Juan
  19,26}

\bibleverse{5} Su madre dijo a los criados: ``Haced lo que os diga''.

\bibleverse{6} Había allí seis vasijas de piedra, colocadas según la
costumbre judía de purificar, que contenían dos o tres metretes cada
una. \footnote{\textbf{2:6} Mar 7,3-4} \bibleverse{7} Jesús les dijo:
``Llenen de agua las tinajas''. Así que las llenaron hasta el borde.
\bibleverse{8} Les dijo: ``Sacad ahora un poco y llevadlo al jefe de la
fiesta.'' Así que lo tomaron. \bibleverse{9} Cuando el dueño del
banquete probó el agua convertida en vino, y no sabía de dónde procedía
(pero los criados que habían sacado el agua sí lo sabían), el dueño del
banquete llamó al novio \bibleverse{10} y le dijo: ``Servid todos
primero el vino bueno, y cuando los invitados hayan bebido libremente,
entonces el que es peor. Tú has guardado el vino bueno hasta ahora''.
\bibleverse{11} Este principio de sus milagros lo hizo Jesús en Caná de
Galilea, y reveló su gloria; y sus discípulos creyeron en él.
\footnote{\textbf{2:11} Juan 1,14}

\hypertarget{jesuxfas-por-primera-vez-en-jerusaluxe9n-en-la-pascua}{%
\subsection{Jesús por primera vez en Jerusalén en la
Pascua}\label{jesuxfas-por-primera-vez-en-jerusaluxe9n-en-la-pascua}}

\bibleverse{12} Después de esto, bajó a Capernaúm, él y su madre, sus
hermanos y sus discípulos; y se quedaron allí unos días. \footnote{\textbf{2:12}
  Juan 7,3; Mat 13,55}

\bibleverse{13} Se acercaba la Pascua de los judíos, y Jesús subió a
Jerusalén. \footnote{\textbf{2:13} Mat 20,18; Mar 11,1; Luc 19,28; Juan
  5,1} \bibleverse{14} Encontró en el templo a los que vendían bueyes,
ovejas y palomas, y a los cambistas sentados. \bibleverse{15} Hizo un
látigo de cuerdas y expulsó a todos del templo, tanto a las ovejas como
a los bueyes; y a los cambistas les derramó el dinero y derribó sus
mesas. \bibleverse{16} A los que vendían las palomas les dijo: ``¡Sacad
esto de aquí! No hagáis de la casa de mi Padre un mercado''.
\bibleverse{17} Sus discípulos recordaron que estaba escrito: ``El celo
por tu casa me consumirá''.

\bibleverse{18} Los judíos le respondieron: ``¿Qué señal nos muestras,
ya que haces estas cosas?'' \footnote{\textbf{2:18} Mat 21,3}

\bibleverse{19} Jesús les respondió: ``Destruyan este templo y en tres
días lo levantaré''. \footnote{\textbf{2:19} Mat 26,61; Mat 27,40}

\bibleverse{20} Los judíos, por tanto, dijeron: ``¡Se necesitaron
cuarenta y seis años para construir este templo! ¿Lo levantarás en tres
días?'' \bibleverse{21} Pero él hablaba del templo de su cuerpo.
\footnote{\textbf{2:21} 1Cor 6,19} \bibleverse{22} Por eso, cuando
resucitó de entre los muertos, sus discípulos se acordaron de que había
dicho esto, y creyeron en la Escritura y en la palabra que Jesús había
dicho. \footnote{\textbf{2:22} Os 6,2}

\bibleverse{23} Estando en Jerusalén en la Pascua, durante la fiesta,
muchos creyeron en su nombre, observando las señales que hacía.
\bibleverse{24} Pero Jesús no se confió a ellos, porque conocía a todos,
\bibleverse{25} y porque no necesitaba que nadie diera testimonio acerca
del hombre, pues él mismo sabía lo que había en el hombre. \footnote{\textbf{2:25}
  Mar 2,8}

\hypertarget{jesuxfas-y-nicodemo}{%
\subsection{Jesús y Nicodemo}\label{jesuxfas-y-nicodemo}}

\hypertarget{section-2}{%
\section{3}\label{section-2}}

\bibleverse{1} Había un hombre de los fariseos que se llamaba Nicodemo,
jefe de los judíos. \footnote{\textbf{3:1} Juan 7,50; Juan 19,39}
\bibleverse{2} Se acercó a Jesús de noche y le dijo: ``Rabí, sabemos que
eres un maestro venido de Dios, porque nadie puede hacer estas señales
que tú haces, si no está Dios con él.''

\bibleverse{3} Jesús le contestó: ``Te aseguro que si uno no nace de
nuevo, no puede ver el Reino de Dios''. \footnote{\textbf{3:3} 1Pe 1,23}

\bibleverse{4} Nicodemo le dijo: ``¿Cómo puede un hombre nacer siendo
viejo? ¿Puede entrar por segunda vez en el vientre de su madre y
nacer?''

\bibleverse{5} Jesús respondió: ``En verdad os digo que el que no nazca
del agua y del Espíritu, no puede entrar en el Reino de Dios.
\footnote{\textbf{3:5} Ezeq 36,25-27; Mat 3,11; Tit 3,5} \bibleverse{6}
Lo que nace de la carne es carne. Lo que nace del Espíritu es espíritu.
\footnote{\textbf{3:6} Juan 1,13; Rom 8,5-9} \bibleverse{7} No os
extrañéis de que os haya dicho: ``Tenéis que nacer de nuevo''.
\bibleverse{8} El viento sopla donde quiere, y vosotros oís su sonido,
pero no sabéis de dónde viene ni a dónde va. Así es todo el que nace del
Espíritu''.

\bibleverse{9} Nicodemo le respondió: ``¿Cómo puede ser esto?''

\bibleverse{10} Jesús le respondió: ``¿Eres tú el maestro de Israel y no
entiendes estas cosas? \bibleverse{11} De cierto te digo que hablamos lo
que sabemos y damos testimonio de lo que hemos visto, y no recibís
nuestro testimonio. \bibleverse{12} Si os he dicho cosas terrenales y no
creéis, ¿cómo creeréis si os digo cosas celestiales? \bibleverse{13}
Nadie ha subido al cielo sino el que descendió del cielo, el Hijo del
Hombre, que está en el cielo. \bibleverse{14} Como Moisés levantó la
serpiente en el desierto, así debe ser levantado el Hijo del Hombre,
\footnote{\textbf{3:14} Núm 21,8-9} \bibleverse{15} para que todo el que
crea en él no perezca, sino que tenga vida eterna. \bibleverse{16}
Porque tanto amó Dios al mundo, que dio a su Hijo unigénito, para que
todo el que crea en él no perezca, sino que tenga vida eterna.
\footnote{\textbf{3:16} Rom 5,8; Rom 8,32; 1Jn 4,9} \bibleverse{17}
Porque Dios no envió a su Hijo al mundo para juzgar al mundo, sino para
que el mundo se salve por él. \footnote{\textbf{3:17} Luc 19,10}
\bibleverse{18} El que cree en él no es juzgado. El que no cree ya ha
sido juzgado, porque no ha creído en el nombre del Hijo único de Dios.
\footnote{\textbf{3:18} Juan 5,24} \bibleverse{19} Esta es la sentencia:
la luz vino al mundo, y los hombres amaron más las tinieblas que la luz,
porque sus obras eran malas. \footnote{\textbf{3:19} Juan 1,5; Juan
  1,9-11} \bibleverse{20} Porque todo el que hace el mal odia la luz y
no viene a la luz, para que sus obras no sean expuestas. \footnote{\textbf{3:20}
  Efes 5,13} \bibleverse{21} Pero el que hace la verdad viene a la luz,
para que se revelen sus obras, que han sido hechas en Dios.''
\footnote{\textbf{3:21} 1Jn 1,6}

\hypertarget{jesuxfas-en-judea-y-el-testimonio-final-del-bautista}{%
\subsection{Jesús en Judea y el testimonio final del
Bautista}\label{jesuxfas-en-judea-y-el-testimonio-final-del-bautista}}

\bibleverse{22} Después de estas cosas, Jesús vino con sus discípulos a
la tierra de Judea. Se quedó allí con ellos y bautizó. \footnote{\textbf{3:22}
  Juan 4,1-2} \bibleverse{23} También Juan bautizaba en Enón, cerca de
Salim, porque allí había mucha agua. Vinieron y se bautizaron;
\bibleverse{24} porque Juan no había sido aún encarcelado. \footnote{\textbf{3:24}
  Mar 1,14} \bibleverse{25} Entonces surgió una disputa por parte de los
discípulos de Juan con algunos judíos sobre la purificación.
\bibleverse{26} Se acercaron a Juan y le dijeron: ``Rabí, el que estaba
contigo al otro lado del Jordán, del que has dado testimonio, he aquí
que bautiza, y todo el mundo acude a él.'' \footnote{\textbf{3:26} Juan
  1,26-34}

\bibleverse{27} Juan respondió: ``El hombre no puede recibir nada si no
le ha sido dado del cielo. \footnote{\textbf{3:27} Heb 5,4}
\bibleverse{28} Vosotros mismos dais testimonio de que yo he dicho: ``Yo
no soy el Cristo'', sino: ``He sido enviado antes que él''. \footnote{\textbf{3:28}
  Juan 1,20; Juan 1,23; Juan 1,27} \bibleverse{29} El que tiene la novia
es el novio; pero el amigo del novio, que está de pie y lo escucha, se
alegra mucho por la voz del novio. Por eso mi alegría es plena.
\footnote{\textbf{3:29} Mat 9,15} \bibleverse{30} Él debe aumentar, pero
yo debo disminuir.

\bibleverse{31} ``El que viene de arriba está por encima de todo. El que
es de la tierra pertenece a la tierra y habla de la tierra. El que viene
del cielo está por encima de todo. \footnote{\textbf{3:31} Juan 8,23}
\bibleverse{32} Lo que ha visto y oído, de eso da testimonio; y nadie
recibe su testimonio. \bibleverse{33} El que ha recibido su testimonio
ha puesto su sello en esto: que Dios es verdadero. \bibleverse{34}
Porque el que Dios ha enviado habla las palabras de Dios; pues Dios da
el Espíritu sin medida. \footnote{\textbf{3:34} Juan 1,16}
\bibleverse{35} El Padre ama al Hijo y ha entregado todas las cosas en
su mano. \footnote{\textbf{3:35} Juan 5,20; Mat 11,27} \bibleverse{36}
El que cree en el Hijo tiene vida eterna, pero el que desobedece al Hijo
no verá la vida, sino que la ira de Dios permanece sobre él.''

\hypertarget{jesuxfas-habla-con-la-mujer-samaritana-junto-al-pozo-de-jacob}{%
\subsection{Jesús habla con la mujer samaritana junto al pozo de
Jacob}\label{jesuxfas-habla-con-la-mujer-samaritana-junto-al-pozo-de-jacob}}

\hypertarget{section-3}{%
\section{4}\label{section-3}}

\bibleverse{1} Por eso, cuando el Señor supo que los fariseos habían
oído que Jesús hacía y bautizaba más discípulos que Juan \footnote{\textbf{4:1}
  Juan 3,22; Juan 3,26} \bibleverse{2} (aunque Jesús mismo no bautizaba,
sino sus discípulos), \bibleverse{3} abandonó Judea y partió hacia
Galilea. \bibleverse{4} Tenía que pasar por Samaria. \bibleverse{5} Y
llegó a una ciudad de Samaria llamada Sicar, cerca de la parcela que
Jacob dio a su hijo José. \footnote{\textbf{4:5} Gén 48,22; Jos 24,32}
\bibleverse{6} Allí estaba el pozo de Jacob. Jesús, cansado del viaje,
se sentó junto al pozo. Era como la hora sexta.

\bibleverse{7} Una mujer de Samaria vino a sacar agua. Jesús le dijo:
``Dame de beber''. \bibleverse{8} Porque sus discípulos habían ido a la
ciudad a comprar comida.

\bibleverse{9} La samaritana le dijo entonces: ``¿Cómo es que tú, siendo
judío, me pides de beber a una samaritana?'' (Porque los judíos no
tienen trato con los samaritanos). \footnote{\textbf{4:9} Luc 9,52-53}

\bibleverse{10} Jesús le contestó: ``Si conocieras el don de Dios y
quién es el que te dice: ``Dame de beber'', se lo habrías pedido a él y
te habría dado agua viva.'' \footnote{\textbf{4:10} Juan 7,38-39}

\bibleverse{11} La mujer le dijo: ``Señor, no tienes con qué sacar, y el
pozo es profundo. ¿De dónde sacas esa agua viva? \bibleverse{12} ¿Acaso
eres más grande que nuestro padre Jacob, que nos dio el pozo y él mismo
bebió de él, al igual que sus hijos y su ganado?''

\bibleverse{13} Jesús le contestó: ``Todo el que beba de esta agua
volverá a tener sed, \footnote{\textbf{4:13} Juan 6,58} \bibleverse{14}
pero el que beba del agua que yo le daré no volverá a tener sed, sino
que el agua que yo le daré se convertirá en él en una fuente de agua que
salta hasta la vida eterna.'' \footnote{\textbf{4:14} Juan 6,35; Juan
  7,38-39}

\bibleverse{15} La mujer le dijo: ``Señor, dame esta agua, para que no
tenga sed ni venga hasta aquí a sacar''.

\bibleverse{16} Jesús le dijo: ``Ve, llama a tu marido y ven aquí''.

\bibleverse{17} La mujer respondió: ``No tengo marido''. Jesús le dijo:
``Has dicho bien: ``No tengo marido'', \bibleverse{18} porque has tenido
cinco maridos; y el que ahora tienes no es tu marido. Esto lo has dicho
con verdad''.

\bibleverse{19} La mujer le dijo: ``Señor, me doy cuenta de que eres un
profeta. \bibleverse{20} Nuestros padres adoraban en este monte, y
vosotros los judíos decís que en Jerusalén es el lugar donde se debe
adorar.'' \footnote{\textbf{4:20} Deut 12,5; Sal 122,1}

\bibleverse{21} Jesús le dijo: ``Mujer, créeme, que llega la hora en que
ni en este monte ni en Jerusalén adoraréis al Padre. \bibleverse{22}
Vosotros adoráis lo que no conocéis. Nosotros adoramos lo que conocemos,
porque la salvación viene de los judíos. \footnote{\textbf{4:22} 2Re
  17,29-41; Is 2,3} \bibleverse{23} Pero viene la hora, y ahora es,
cuando los verdaderos adoradores adorarán al Padre en espíritu y en
verdad, porque el Padre busca a los tales para que sean sus adoradores.
\bibleverse{24} Dios es espíritu, y los que lo adoran deben hacerlo en
espíritu y en verdad.'' \footnote{\textbf{4:24} Rom 12,1; 2Cor 3,17}

\bibleverse{25} La mujer le dijo: ``Sé que viene el Mesías, el que se
llama Cristo. Cuando haya venido, nos declarará todas las cosas''.
\footnote{\textbf{4:25} Juan 1,41}

\bibleverse{26} Jesús le dijo: ``Yo soy el que te habla''.

\hypertarget{jesuxfas-y-los-discuxedpulos}{%
\subsection{Jesús y los discípulos}\label{jesuxfas-y-los-discuxedpulos}}

\bibleverse{27} En ese momento llegaron sus discípulos. Se maravillaron
de que hablara con una mujer; pero nadie dijo: ``¿Qué buscas?'' o ``¿Por
qué hablas con ella?''. \bibleverse{28} Entonces la mujer dejó su
cántaro, se fue a la ciudad y dijo a la gente: \bibleverse{29} ``Venid a
ver a un hombre que me ha contado todo lo que he hecho. ¿Será éste el
Cristo?'' \bibleverse{30} Salieron de la ciudad y se acercaron a él.

\bibleverse{31} Mientras tanto, los discípulos le urgían diciendo:
``Rabí, come''.

\bibleverse{32} Pero él les dijo: ``Tengo que comer algo que vosotros no
sabéis''.

\bibleverse{33} Entonces los discípulos se dijeron unos a otros:
``¿Alguien le ha traído algo de comer?''

\bibleverse{34} Jesús les dijo: ``Mi comida es hacer la voluntad del que
me envió y cumplir su obra. \footnote{\textbf{4:34} Juan 6,38; Juan 17,4}
\bibleverse{35} ¿No decís que aún faltan cuatro meses para la cosecha?
Pues os digo que levantéis los ojos y miréis los campos, que ya están
blancos para la cosecha. \footnote{\textbf{4:35} Mat 9,37}
\bibleverse{36} El que cosecha recibe el salario y recoge el fruto para
la vida eterna, para que tanto el que siembra como el que cosecha se
alegren juntos. \bibleverse{37} Porque en esto es cierto el dicho: ``Uno
siembra y otro cosecha''. \bibleverse{38} Yo os he enviado a cosechar lo
que no habéis trabajado. Otros han trabajado, y tú has entrado en su
trabajo''.

\bibleverse{39} De aquella ciudad muchos samaritanos creyeron en él por
la palabra de la mujer, que testificó: ``Me ha dicho todo lo que he
hecho.'' \bibleverse{40} Así que los samaritanos se acercaron a él y le
rogaron que se quedara con ellos. Se quedó allí dos días.
\bibleverse{41} Muchos más creyeron gracias a su palabra.
\bibleverse{42} Dijeron a la mujer: ``Ahora creemos, no por lo que tú
dices; porque hemos oído por nosotros mismos, y sabemos que éste es
verdaderamente el Cristo, el Salvador del mundo.'' \footnote{\textbf{4:42}
  Hech 8,5-8}

\hypertarget{curaciuxf3n-del-hijo-de-un-funcionario-real-en-cafarnauxfam}{%
\subsection{Curación del hijo de un funcionario real en
Cafarnaúm}\label{curaciuxf3n-del-hijo-de-un-funcionario-real-en-cafarnauxfam}}

\bibleverse{43} Después de los dos días, salió de allí y se fue a
Galilea. \footnote{\textbf{4:43} Mat 4,12} \bibleverse{44} Porque el
mismo Jesús dio testimonio de que un profeta no tiene honor en su propia
tierra. \footnote{\textbf{4:44} Mat 13,57} \bibleverse{45} Cuando llegó
a Galilea, los galileos le recibieron, habiendo visto todo lo que hacía
en Jerusalén en la fiesta, pues también ellos habian ido a la fiesta.
\footnote{\textbf{4:45} Juan 2,23} \bibleverse{46} Vino, pues, Jesús de
nuevo a Caná de Galilea, donde convirtió el agua en vino. Había un noble
cuyo hijo estaba enfermo en Capernaúm. \footnote{\textbf{4:46} Juan 2,1;
  Juan 2,9} \bibleverse{47} Cuando se enteró de que Jesús había salido
de Judea a Galilea, fue a él y le rogó que bajara a curar a su hijo,
porque estaba a punto de morir. \bibleverse{48} Entonces Jesús le dijo:
``Si no ves señales y prodigios, de ninguna manera creerás''.
\footnote{\textbf{4:48} Juan 2,18; 1Cor 1,22}

\bibleverse{49} El noble le dijo: ``Señor, baja antes de que muera mi
hijo''.

\bibleverse{50} Jesús le dijo: ``Vete. Tu hijo vive''. El hombre creyó
en la palabra que Jesús le había dicho, y se fue. \bibleverse{51}
Mientras bajaba, sus siervos le salieron al encuentro y le informaron
diciendo: ``¡Tu hijo vive!'' \bibleverse{52} Entonces les preguntó a qué
hora había empezado a mejorar. Ellos le dijeron: ``Ayer, a la hora
séptima, le dejó la fiebre''. \bibleverse{53} Así que el padre supo que
fue a esa hora cuando Jesús le dijo: ``Tu hijo vive''. Creyó, al igual
que toda su casa. \bibleverse{54} Esta es también la segunda señal que
hizo Jesús, habiendo salido de Judea a Galilea. \footnote{\textbf{4:54}
  Juan 2,11}

\hypertarget{sanaciuxf3n-de-los-enfermos-en-el-estanque-de-betesda-cerca-de-jerusaluxe9n-y-concurso-del-suxe1bado}{%
\subsection{Sanación de los enfermos en el estanque de Betesda cerca de
Jerusalén y concurso del
sábado}\label{sanaciuxf3n-de-los-enfermos-en-el-estanque-de-betesda-cerca-de-jerusaluxe9n-y-concurso-del-suxe1bado}}

\hypertarget{section-4}{%
\section{5}\label{section-4}}

\bibleverse{1} Después de estas cosas, hubo una fiesta de los judíos, y
Jesús subió a Jerusalén. \footnote{\textbf{5:1} Juan 2,13}
\bibleverse{2} En Jerusalén, junto a la puerta de las ovejas, hay un
estanque llamado en hebreo ``Betesda'', que tiene cinco pórticos.
\footnote{\textbf{5:2} Neh 3,1} \bibleverse{3} En ellos yacía una gran
multitud de enfermos, ciegos, cojos o paralíticos, esperando que se
moviera el agua; \bibleverse{4} porque un ángel bajaba a ciertas horas
al estanque y agitaba el agua. El que entraba primero después de agitar
el agua quedaba curado de cualquier enfermedad que tuviera.
\bibleverse{5} Estaba allí un hombre que llevaba treinta y ocho años
enfermo. \bibleverse{6} Cuando Jesús lo vio allí tendido, y supo que
llevaba mucho tiempo enfermo, le preguntó: ``¿Quieres curarte?''

\bibleverse{7} El enfermo le respondió: ``Señor, no tengo a nadie que me
meta en la piscina cuando se agita el agua, pero mientras vengo, otro
baja antes que yo.''

\bibleverse{8} Jesús le dijo: ``Levántate, toma tu estera y anda''.

\bibleverse{9} Al instante, el hombre quedó sano, tomó su estera y
caminó. Ese día era sábado. \bibleverse{10} Así que los judíos le
dijeron al que estaba curado: ``Es sábado. No te es lícito llevar la
estera''. \footnote{\textbf{5:10} Jer 17,21-22}

\bibleverse{11} Él les contestó: ``El que me curó me dijo: ``Toma tu
estera y camina''\,''.

\bibleverse{12} Entonces le preguntaron: ``¿Quién es el hombre que te ha
dicho: ``Coge tu estera y anda''?''

\bibleverse{13} Pero el que había sido curado no sabía quién era, porque
Jesús se había retirado, ya que había una multitud en el lugar.

\bibleverse{14} Después, Jesús lo encontró en el templo y le dijo:
``Mira, has quedado bien. No peques más, para que no te ocurra nada
peor''. \footnote{\textbf{5:14} Juan 8,11}

\bibleverse{15} El hombre se fue y contó a los judíos que era Jesús
quien lo había curado. \bibleverse{16} Por eso los judíos persiguieron a
Jesús y trataron de matarlo, porque hacía estas cosas en sábado.
\footnote{\textbf{5:16} Mat 12,14} \bibleverse{17} Pero Jesús les
respondió: ``Mi Padre sigue trabajando, así que yo también trabajo''.
\footnote{\textbf{5:17} Juan 9,4}

\bibleverse{18} Por eso los judíos procuraban matarlo aún más, porque no
sólo quebrantaba el sábado, sino que llamaba a Dios su propio Padre,
haciéndose igual a Dios. \footnote{\textbf{5:18} Juan 7,30; Juan 10,33}

\hypertarget{el-testimonio-de-jesuxfas-de-su-obra-divina-y-de-su-filiaciuxf3n-divina-jesuxfas-como-juez-y-dador-de-vida}{%
\subsection{El testimonio de Jesús de su obra divina y de su filiación
divina; Jesús como juez y dador de
vida}\label{el-testimonio-de-jesuxfas-de-su-obra-divina-y-de-su-filiaciuxf3n-divina-jesuxfas-como-juez-y-dador-de-vida}}

\bibleverse{19} Entonces Jesús les respondió: ``Os aseguro que el Hijo
no puede hacer nada por sí mismo, sino lo que ve hacer al Padre. Porque
todo lo que él hace, también lo hace el Hijo. \footnote{\textbf{5:19}
  Juan 3,11; Juan 3,32} \bibleverse{20} Porque el Padre tiene afecto por
el Hijo, y le muestra todas las cosas que él mismo hace. Le mostrará
obras mayores que éstas, para que os maravilléis. \footnote{\textbf{5:20}
  Juan 3,35} \bibleverse{21} Porque como el Padre resucita a los muertos
y les da vida, así también el Hijo da vida a quien quiere.
\bibleverse{22} Porque el Padre no juzga a nadie, sino que ha dado todo
el juicio al Hijo, \footnote{\textbf{5:22} Dan 7,12; Dan 7,14; Hech
  10,42} \bibleverse{23} para que todos honren al Hijo como honran al
Padre. El que no honra al Hijo no honra al Padre que lo envió.
\footnote{\textbf{5:23} Fil 2,10-11; 1Jn 2,23}

\bibleverse{24} ``De cierto os digo que el que oye mi palabra y cree al
que me ha enviado tiene vida eterna, y no entra en juicio, sino que ha
pasado de la muerte a la vida. \footnote{\textbf{5:24} Juan 3,16; Juan
  3,18} \bibleverse{25} De cierto os digo que llega la hora, y ya es, en
que los muertos oirán la voz del Hijo de Dios, y los que la oigan
vivirán. \footnote{\textbf{5:25} Efes 2,5-6} \bibleverse{26} Porque como
el Padre tiene vida en sí mismo, así también le dio al Hijo que tenga
vida en sí mismo. \footnote{\textbf{5:26} Juan 1,1-4} \bibleverse{27}
También le dio autoridad para ejecutar el juicio, porque es el Hijo del
hombre. \footnote{\textbf{5:27} Dan 7,13-14} \bibleverse{28} No os
maravilléis de esto, porque llega la hora en que todos los que están en
los sepulcros oirán su voz \bibleverse{29} y saldrán; los que han hecho
el bien, a la resurrección de la vida; y los que han hecho el mal, a la
resurrección del juicio. \footnote{\textbf{5:29} Dan 12,2; Mat 25,46;
  2Cor 5,10} \bibleverse{30} Yo no puedo hacer nada por mí mismo. Según
oigo, juzgo; y mi juicio es justo, porque no busco mi propia voluntad,
sino la voluntad de mi Padre que me ha enviado. \footnote{\textbf{5:30}
  Juan 6,38}

\hypertarget{el-testimonio-de-juan}{%
\subsection{El testimonio de Juan}\label{el-testimonio-de-juan}}

\bibleverse{31} ``Si yo testifico de mí mismo, mi testimonio no es
válido. \bibleverse{32} Es otro el que testifica sobre mí. Sé que el
testimonio que da sobre mí es verdadero. \bibleverse{33} Tú has enviado
a Juan, y él ha dado testimonio de la verdad. \footnote{\textbf{5:33}
  Juan 1,19-34} \bibleverse{34} Pero el testimonio que yo recibo no
proviene de un hombre. Sin embargo, digo estas cosas para que os
salvéis. \bibleverse{35} Él era la lámpara que ardía y brillaba, y
vosotros quisisteis regocijaros por un tiempo en su luz.

\hypertarget{el-testimonio-del-padre}{%
\subsection{El testimonio del padre}\label{el-testimonio-del-padre}}

\bibleverse{36} Pero el testimonio que yo tengo es mayor que el de Juan;
porque las obras que el Padre me dio para realizar, las mismas obras que
yo hago, dan testimonio de mí, de que el Padre me ha enviado.
\footnote{\textbf{5:36} Juan 3,2; Juan 10,25; Juan 10,38}
\bibleverse{37} El Padre mismo, que me ha enviado, ha dado testimonio de
mí. Vosotros no habéis oído su voz en ningún momento, ni habéis visto su
forma. \footnote{\textbf{5:37} Mat 3,17} \bibleverse{38} No tenéis su
palabra viviendo en vosotros, porque no creéis al que él ha enviado.

\bibleverse{39} ``Escudriñáis las Escrituras, porque pensáis que en
ellas tenéis la vida eterna; y éstas son las que dan testimonio de mí.
\footnote{\textbf{5:39} Luc 24,27; Luc 24,44; 2Tim 3,15-17}
\bibleverse{40} Pero no queréis venir a mí para tener vida.

\hypertarget{ataque-a-la-incredulidad-y-ambiciuxf3n-de-los-juduxedos-testimonio-de-moisuxe9s}{%
\subsection{Ataque a la incredulidad y ambición de los judíos;
Testimonio de
moisés}\label{ataque-a-la-incredulidad-y-ambiciuxf3n-de-los-juduxedos-testimonio-de-moisuxe9s}}

\bibleverse{41} Yo no recibo la gloria de los hombres. \bibleverse{42}
Pero yo os conozco, que no tenéis el amor de Dios en vosotros mismos.
\bibleverse{43} Yo he venido en nombre de mi Padre, y no me recibís. Si
otro viene en su propio nombre, lo recibiréis. \footnote{\textbf{5:43}
  Mat 24,5} \bibleverse{44} ¿Cómo podéis creer, que recibís la gloria
unos de otros, y no buscáis la gloria que viene del único Dios?
\footnote{\textbf{5:44} Juan 12,42-43; 1Tes 2,6}

\bibleverse{45} ``No creas que te voy a acusar ante el Padre. Hay uno
que os acusa, incluso Moisés, en quien habéis puesto vuestra esperanza.
\footnote{\textbf{5:45} Deut 31,26-27} \bibleverse{46} Porque si
creyerais a Moisés, me creeríais a mí, pues él escribió sobre mí.
\footnote{\textbf{5:46} Gén 3,15; Gén 49,10; Deut 18,15} \bibleverse{47}
Perosi no creéis en sus escritos, ¿cómo vais a creer en mis palabras?''
\footnote{\textbf{5:47} Luc 16,31}

\hypertarget{jesuxfas-alimenta-a-los-cinco-mil}{%
\subsection{Jesús alimenta a los cinco
mil}\label{jesuxfas-alimenta-a-los-cinco-mil}}

\hypertarget{section-5}{%
\section{6}\label{section-5}}

\bibleverse{1} Después de estas cosas, Jesús se fue al otro lado del mar
de Galilea, que también se llama mar de Tiberíades. \bibleverse{2} Le
seguía una gran multitud, porque veían las señales que hacía con los
enfermos. \bibleverse{3} Jesús subió al monte y se sentó allí con sus
discípulos. \bibleverse{4} Se acercaba la Pascua, la fiesta de los
judíos. \footnote{\textbf{6:4} Juan 2,13; Juan 11,55} \bibleverse{5}
Entonces Jesús, alzando los ojos y viendo que se acercaba a él una gran
multitud, dijo a Felipe: ``¿Dónde vamos a comprar pan para que estos
coman?'' \bibleverse{6} Decía esto para ponerle a prueba, pues él mismo
sabía lo que iba a hacer.

\bibleverse{7} Felipe le respondió: ``No les basta con doscientos
denarios de pan, para que cada uno reciba un poco.''

\bibleverse{8} Uno de sus discípulos, Andrés, hermano de Simón Pedro, le
dijo: \bibleverse{9} ``Hay aquí un muchacho que tiene cinco panes de
cebada y dos peces, pero ¿qué son éstos entre tantos?''

\bibleverse{10} Jesús dijo: ``Que la gente se siente''. Había mucha
hierba en aquel lugar. Así que los hombres se sentaron, en número de
unos cinco mil. \bibleverse{11} Jesús tomó los panes, y habiendo dado
gracias, repartió a los discípulos, y los discípulos a los que estaban
sentados, asimismo de los peces cuanto quisieron. \bibleverse{12} Cuando
se saciaron, dijo a sus discípulos: ``Recoged los trozos que han
sobrado, para que no se pierda nada.'' \bibleverse{13} Así que los
recogieron y llenaron doce cestas con los trozos de los cinco panes de
cebada que habían sobrado a los que habían comido. \bibleverse{14} Al
ver la gente la señal que Jesús había hecho, dijeron: ``Este es
verdaderamente el profeta que viene al mundo.'' \footnote{\textbf{6:14}
  Deut 18,15} \bibleverse{15} Jesús, pues, percibiendo que iban a venir
a prenderle por la fuerza para hacerle rey, se retiró de nuevo al monte,
a solas. \footnote{\textbf{6:15} Juan 18,36}

\hypertarget{jesuxfas-camina-sobre-el-lago}{%
\subsection{Jesús camina sobre el
lago}\label{jesuxfas-camina-sobre-el-lago}}

\bibleverse{16} Al atardecer, sus discípulos bajaron al mar.
\bibleverse{17} Entraron en la barca y atravesaron el mar hacia
Capernaum. Ya había oscurecido, y Jesús no había llegado hasta ellos.
\bibleverse{18} El mar estaba agitado por un gran viento que soplaba.
\bibleverse{19} Por lo tanto, cuando habían remado unos veinticinco o
treinta estadios, vieron a Jesús que caminaba sobre el mar y se acercaba
a la barca; y tuvieron miedo. \bibleverse{20} Pero él les dijo: ``Soy
yo, no tengáis miedo''. \bibleverse{21} Por lo tanto, estaban dispuestos
a recibirlo en la barca. Inmediatamente la barca llegó a la tierra a la
que se dirigían.

\hypertarget{el-reencuentro-con-el-pueblo-y-la-demanda-de-seuxf1al-del-pueblo}{%
\subsection{El reencuentro con el pueblo y la demanda de señal del
pueblo}\label{el-reencuentro-con-el-pueblo-y-la-demanda-de-seuxf1al-del-pueblo}}

\bibleverse{22} Al día siguiente, la multitud que estaba al otro lado
del mar vio que no había allí ninguna otra barca, sino aquella en la que
se habían embarcado sus discípulos, y que Jesús no había entrado con sus
discípulos en la barca, sino que sus discípulos se habían ido solos.
\bibleverse{23} Sin embargo, unas barcas procedentes de Tiberíades se
acercaron al lugar donde comieron el pan después de que el Señor diera
las gracias. \bibleverse{24} Al ver, pues, la multitud que Jesús no
estaba allí, ni sus discípulos, subieron ellos mismos a las barcas y
vinieron a Capernaum, buscando a Jesús. \bibleverse{25} Cuando lo
encontraron al otro lado del mar, le preguntaron: ``Rabí, ¿cuándo has
venido aquí?''

\bibleverse{26} Jesús les respondió: ``Os aseguro que me buscáis, no
porque hayáis visto señales, sino porque habéis comido de los panes y os
habéis saciado. \bibleverse{27} No trabajéis por el alimento que perece,
sino por el que permanece para la vida eterna, que os dará el Hijo del
Hombre. Porque Dios Padre lo ha sellado''. \footnote{\textbf{6:27} Juan
  5,36}

\bibleverse{28} Entonces le dijeron: ``¿Qué debemos hacer, para que
podamos obrar las obras de Dios?''

\bibleverse{29} Jesús les respondió: ``Esta es la obra de Dios, que
creáis en el que él ha enviado''.

\bibleverse{30} Por eso le dijeron: ``¿Qué señal haces, pues, para que
te veamos y te creamos? ¿Qué obra haces? \bibleverse{31} Nuestros padres
comieron el maná en el desierto. Como está escrito: `Les dio a comer pan
del cielo'\,''. \footnote{\textbf{6:31} Éxod 16,13-14}

\hypertarget{el-discurso-de-jesuxfas-sobre-el-pan-de-vida}{%
\subsection{El discurso de Jesús sobre el pan de
vida}\label{el-discurso-de-jesuxfas-sobre-el-pan-de-vida}}

\bibleverse{32} Entonces Jesús les dijo: ``Os aseguro que no fue Moisés
quien os dio el pan del cielo, sino que mi Padre os da el verdadero pan
del cielo. \bibleverse{33} Porque el pan de Dios es el que baja del
cielo y da vida al mundo.''

\bibleverse{34} Por eso le dijeron: ``Señor, danos siempre este pan''.

\bibleverse{35} Jesús les dijo: ``Yo soy el pan de vida. El que viene a
mí no tendrá hambre, y el que cree en mí nunca tendrá sed. \footnote{\textbf{6:35}
  Juan 4,14; Juan 7,37} \bibleverse{36} Pero os he dicho que me habéis
visto, y sin embargo no creéis. \bibleverse{37} Todos los que el Padre
me dé vendrán a mí. Al que venga a mí no lo echaré de ninguna manera.
\footnote{\textbf{6:37} Mat 11,28} \bibleverse{38} Porque he bajado del
cielo, no para hacer mi propia voluntad, sino la voluntad del que me ha
enviado. \footnote{\textbf{6:38} Juan 4,34} \bibleverse{39} Esta es la
voluntad de mi Padre que me ha enviado: que de todo lo que me ha dado no
pierda nada, sino que lo resucite en el último día. \footnote{\textbf{6:39}
  Juan 10,28-29; Juan 17,12} \bibleverse{40} Esta es la voluntad del que
me ha enviado: que todo el que vea al Hijo y crea en él tenga vida
eterna; y yo lo resucitaré en el último día.'' \footnote{\textbf{6:40}
  Juan 5,29; Juan 11,24}

\bibleverse{41} Los judíos, pues, murmuraban de él, porque decía: ``Yo
soy el pan bajado del cielo''. \bibleverse{42} Dijeron: ``¿No es éste
Jesús, el hijo de José, cuyo padre y madre conocemos? ¿Cómo, pues, dice:
``He bajado del cielo''?'' \footnote{\textbf{6:42} Luc 4,22}

\bibleverse{43} Por eso Jesús les respondió: ``No murmuren entre
ustedes. \bibleverse{44} Nadie puede venir a mí si el Padre que me envió
no lo atrae; y yo lo resucitaré en el último día. \bibleverse{45} Está
escrito en los profetas: `Todos serán enseñados por Dios'. Por eso, todo
el que oye del Padre y ha aprendido, viene a mí. \bibleverse{46} No es
que alguien haya visto al Padre, sino el que viene de Dios. Él ha visto
al Padre. \footnote{\textbf{6:46} Juan 1,18} \bibleverse{47} De cierto
os digo que el que cree en mí tiene vida eterna. \footnote{\textbf{6:47}
  Juan 3,16} \bibleverse{48} Yo soy el pan de vida. \footnote{\textbf{6:48}
  Juan 6,35} \bibleverse{49} Vuestros padres comieron el maná en el
desierto y murieron. \footnote{\textbf{6:49} 1Cor 10,3-5}
\bibleverse{50} Este es el pan que baja del cielo, para que cualquiera
coma de él y no muera. \bibleverse{51} Yo soy el pan vivo que ha bajado
del cielo. Si alguien come de este pan, vivirá para siempre. Sí, el pan
que daré para la vida del mundo es mi carne''.

\bibleverse{52} Los judíos, pues, discutían entre sí, diciendo: ``¿Cómo
puede éste darnos a comer su carne?''

\bibleverse{53} Por eso Jesús les dijo: ``Os aseguro que si no coméis la
carne del Hijo del Hombre y no bebéis su sangre, no tenéis vida en
vosotros mismos. \bibleverse{54} El que come mi carne y bebe mi sangre
tiene vida eterna, y yo lo resucitaré en el último día. \footnote{\textbf{6:54}
  Mat 26,26-28} \bibleverse{55} Porque mi carne es verdadera comida y mi
sangre es verdadera bebida. \bibleverse{56} El que come mi carne y bebe
mi sangre vive en mí, y yo en él. \footnote{\textbf{6:56} Juan 15,4; 1Jn
  3,24} \bibleverse{57} Como el Padre viviente me envió, y yo vivo por
el Padre, así el que se alimenta de mí también vivirá por mí.
\bibleverse{58} Este es el pan que bajó del cielo, no como nuestros
padres que comieron el maná y murieron. El que come este pan vivirá para
siempre''. \bibleverse{59} Estas cosas las decía en la sinagoga,
mientras enseñaba en Capernaum.

\hypertarget{el-divorcio-de-los-discuxedpulos-de-jesuxfas-como-efecto-del-habla}{%
\subsection{El divorcio de los discípulos de Jesús como efecto del
habla}\label{el-divorcio-de-los-discuxedpulos-de-jesuxfas-como-efecto-del-habla}}

\bibleverse{60} Por eso, muchos de sus discípulos, al oír esto, dijeron:
``¡Qué dura es esta frase! ¿Quién puede escucharlo?''

\bibleverse{61} Pero Jesús, sabiendo en sí mismo que sus discípulos
murmuraban de esto, les dijo: ``¿Esto os hace tropezar? \bibleverse{62}
¿Y si vierais al Hijo del Hombre subir adonde estaba antes? \footnote{\textbf{6:62}
  Luc 24,50-51} \bibleverse{63} El espíritu es el que da la vida. La
carne no aprovecha nada. Las palabras que yo os digo son espíritu y son
vida. \footnote{\textbf{6:63} 2Cor 3,6} \bibleverse{64} Pero hay algunos
de vosotros que no creen''. Porque Jesús sabía desde el principio
quiénes eran los que no creían, y quiénes eran los que lo iban a
traicionar. \bibleverse{65} Dijo: ``Por eso os he dicho que nadie puede
venir a mí, si no le es dado por mi Padre.''

\bibleverse{66} Al oír esto, muchos de sus discípulos volvieron atrás y
ya no andaban con él. \bibleverse{67} Entonces Jesús dijo a los doce:
``¿No queréis iros también vosotros, verdad?''

\bibleverse{68} Simón Pedro le respondió: ``Señor, ¿a quién vamos a ir?
Tú tienes palabras de vida eterna. \bibleverse{69} Hemos llegado a creer
y conocer que tú eres el Cristo, el Hijo de Dios vivo''. \footnote{\textbf{6:69}
  Mat 16,16}

\bibleverse{70} Jesús les respondió: ``¿No os he elegido a vosotros, los
doce, y uno de vosotros es un demonio?'' \bibleverse{71} Ahora bien,
hablaba de Judas, hijo de Simón Iscariote, porque era él quien lo iba a
traicionar, siendo uno de los doce.

\hypertarget{jesuxfas-viaja-a-jerusaluxe9n-para-la-fiesta-de-los-tabernuxe1culos}{%
\subsection{Jesús viaja a Jerusalén para la Fiesta de los
Tabernáculos}\label{jesuxfas-viaja-a-jerusaluxe9n-para-la-fiesta-de-los-tabernuxe1culos}}

\hypertarget{section-6}{%
\section{7}\label{section-6}}

\bibleverse{1} Después de estas cosas, Jesús andaba por Galilea, pues no
quería andar por Judea, porque los judíos buscaban matarlo. \footnote{\textbf{7:1}
  Juan 4,43} \bibleverse{2} Se acercaba la fiesta de los judíos, la
Fiesta de los Tabernáculos. \footnote{\textbf{7:2} Lev 23,34-36}
\bibleverse{3} Entonces sus hermanos le dijeron: ``Sal de aquí y vete a
Judea, para que también tus discípulos vean las obras que haces.
\footnote{\textbf{7:3} Juan 2,12; Mat 12,46; Hech 1,14} \bibleverse{4}
Porque nadie hace nada en secreto mientras busca ser conocido
abiertamente. Si haces estas cosas, date a conocer al mundo''.
\bibleverse{5} Porque ni siquiera sus hermanos creían en él.

\bibleverse{6} Por eso, Jesús les dijo: ``Todavía no ha llegado mi hora,
pero vuestra hora está siempre lista. \footnote{\textbf{7:6} Juan 2,4}
\bibleverse{7} El mundo no puede odiaros, pero me odia a mí, porque yo
doy testimonio de él, de que sus obras son malas. \footnote{\textbf{7:7}
  Juan 15,18} \bibleverse{8} Vosotros subid a la fiesta. Yo todavía no
subo a esta fiesta, porque mi tiempo aún no se ha cumplido.''

\bibleverse{9} Habiéndoles dicho estas cosas, se quedó en Galilea.
\bibleverse{10} Pero cuando sus hermanos subieron a la fiesta, él
también subió, no en público, sino como en secreto. \footnote{\textbf{7:10}
  Juan 2,13} \bibleverse{11} Los judíos, pues, le buscaban en la fiesta
y decían: ``¿Dónde está?''. \bibleverse{12} Había mucha murmuración
entre las multitudes acerca de él. Algunos decían: ``Es un buen
hombre''. Otros decían: ``No es así, sino que extravía a la multitud''.
\bibleverse{13} Pero nadie hablaba abiertamente de él por miedo a los
judíos. \footnote{\textbf{7:13} Juan 9,22; Juan 12,42; Juan 19,38}

\hypertarget{la-apariciuxf3n-y-el-testimonio-de-suxed-mismo-de-jesuxfas-en-la-fiesta-de-los-tabernuxe1culos}{%
\subsection{La aparición y el testimonio de sí mismo de Jesús en la
Fiesta de los
Tabernáculos}\label{la-apariciuxf3n-y-el-testimonio-de-suxed-mismo-de-jesuxfas-en-la-fiesta-de-los-tabernuxe1culos}}

\bibleverse{14} Pero cuando ya era la mitad de la fiesta, Jesús subió al
templo y enseñó. \bibleverse{15} Entonces los judíos se maravillaron,
diciendo: ``¿Cómo sabe éste las letras, no habiendo sido educado?''
\footnote{\textbf{7:15} Mat 13,56}

\bibleverse{16} Por eso Jesús les respondió: ``Mi enseñanza no es mía,
sino de quien me ha enviado. \bibleverse{17} Si alguien quiere hacer su
voluntad, conocerá la enseñanza, si viene de Dios o si hablo por mi
cuenta. \bibleverse{18} El que habla por su cuenta busca su propia
gloria, pero el que busca la gloria del que lo envió es veraz, y no hay
en él ninguna injusticia. \footnote{\textbf{7:18} Juan 5,41; Juan 5,44}
\bibleverse{19} ¿No os dio Moisés la ley, y sin embargo ninguno de
vosotros la cumple? ¿Por qué buscáis matarme?'' \footnote{\textbf{7:19}
  Juan 5,16; Juan 5,18; Rom 2,17-24}

\bibleverse{20} La multitud respondió: ``¡Tienes un demonio! ¿Quién
busca matarte?'' \footnote{\textbf{7:20} Juan 10,20}

\bibleverse{21} Jesús les respondió: ``Yo hice una obra y todos ustedes
se maravillan por ella. \footnote{\textbf{7:21} Juan 5,16}
\bibleverse{22} Moisés os ha dado la circuncisión (no es de Moisés, sino
de los padres), y en sábado circuncidáis a un muchacho. \footnote{\textbf{7:22}
  Gén 17,10-12; Lev 12,3} \bibleverse{23} Si un muchacho recibe la
circuncisión en sábado, para que no se infrinja la ley de Moisés, ¿os
enfadáis conmigo porque he hecho a un hombre completamente sano en
sábado? \bibleverse{24} No juzguéis según las apariencias, sino juzgad
con rectitud.''

\hypertarget{jesuxfas-viene-de-dios}{%
\subsection{Jesús viene de Dios}\label{jesuxfas-viene-de-dios}}

\bibleverse{25} Por eso algunos de los de Jerusalén dijeron: ``¿No es
éste al que quieren matar? \bibleverse{26} He aquí que habla
abiertamente, y no le dicen nada. ¿Es posible que los gobernantes sepan
que éste es verdaderamente el Cristo? \bibleverse{27} Sin embargo,
nosotros sabemos de dónde viene este hombre, pero cuando venga el
Cristo, nadie sabrá de dónde viene.'' \footnote{\textbf{7:27} Heb 7,3}

\bibleverse{28} Por eso Jesús alzó la voz en el templo, enseñando y
diciendo: ``Vosotros me conocéis y sabéis de dónde vengo. No he venido
por mí mismo, sino que es verdadero el que me ha enviado, a quien
vosotros no conocéis. \bibleverse{29} Yo lo conozco, porque vengo de él,
y él me ha enviado''. \footnote{\textbf{7:29} Mat 11,27}

\bibleverse{30} Buscaban, pues, prenderle; pero nadie le echó mano,
porque aún no había llegado su hora. \footnote{\textbf{7:30} Juan 8,20;
  Luc 22,53} \bibleverse{31} Pero de la multitud, muchos creyeron en él.
Decían: ``Cuando venga el Cristo, no hará más señales que las que ha
hecho este hombre, ¿verdad?'' \bibleverse{32} Los fariseos oyeron que la
multitud murmuraba estas cosas acerca de él, y los jefes de los
sacerdotes y los fariseos enviaron oficiales para arrestarlo.

\hypertarget{jesuxfas-anuncia-su-regressa-a-dios}{%
\subsection{Jesús anuncia su regressa a
Dios}\label{jesuxfas-anuncia-su-regressa-a-dios}}

\bibleverse{33} Entonces Jesús dijo: ``Estaré con vosotros un poco más,
y luego me iré con el que me ha enviado. \footnote{\textbf{7:33} Juan
  13,33} \bibleverse{34} Me buscaréis y no me encontraréis. No podéis
venir donde yo estoy''. \footnote{\textbf{7:34} Juan 8,21}

\bibleverse{35} Los judíos, pues, decían entre sí: ``¿Adónde irá este
hombre para que no lo encontremos? ¿Irá a la Dispersión entre los
griegos y enseñará a los griegos? \bibleverse{36} ¿Qué es esto que ha
dicho: ``Me buscaréis y no me encontraréis'', y ``Donde yo esté,
vosotros no podréis venir''?''

\hypertarget{jesuxfas-en-el-apogeo-de-la-fiesta-como-dador-del-agua-de-vida}{%
\subsection{Jesús en el apogeo de la fiesta como dador del agua de
vida}\label{jesuxfas-en-el-apogeo-de-la-fiesta-como-dador-del-agua-de-vida}}

\bibleverse{37} El último y más importante día de la fiesta, Jesús se
puso en pie y alzó la voz: ``Si alguien tiene sed, que venga a mí y
beba. \footnote{\textbf{7:37} Lev 23,36; Juan 4,10; Is 55,1; Apoc 22,17}
\bibleverse{38} El que cree en mí, como dice la Escritura, de su
interior brotarán ríos de agua viva.'' \footnote{\textbf{7:38} Is 58,11}
\bibleverse{39} Pero esto lo dijo a propósito del Espíritu, que iban a
recibir los que creyeran en él. Porque el Espíritu Santo no se había
dado aún, porque Jesús no estaba todavía glorificado. \footnote{\textbf{7:39}
  Juan 16,7}

\bibleverse{40} Por lo tanto, muchos de la multitud, al oír estas
palabras, dijeron: ``Este es verdaderamente el profeta''. \footnote{\textbf{7:40}
  Juan 6,14} \bibleverse{41} Otros decían: ``Este es el Cristo''. Pero
algunos decían: ``¿Qué, el Cristo sale de Galilea? \footnote{\textbf{7:41}
  Juan 1,46} \bibleverse{42} ¿No ha dicho la Escritura que el Cristo
viene de la estirpe de David y de Belén, la aldea donde estuvo David?''
\footnote{\textbf{7:42} Miq 5,2; Mat 2,5-6; Mat 22,42} \bibleverse{43}
Así que surgió una división en la multitud a causa de él. \footnote{\textbf{7:43}
  Juan 9,16} \bibleverse{44} Algunos querían prenderle, pero nadie le
echó mano.

\hypertarget{fracaso-del-plan-de-arresto-de-los-luxedderes-divisiuxf3n-entre-los-miembros-del-sumo-consejo-amonestaciuxf3n-de-nicodemo}{%
\subsection{Fracaso del plan de arresto de los líderes; División entre
los miembros del sumo consejo; Amonestación de
Nicodemo}\label{fracaso-del-plan-de-arresto-de-los-luxedderes-divisiuxf3n-entre-los-miembros-del-sumo-consejo-amonestaciuxf3n-de-nicodemo}}

\bibleverse{45} Los oficiales, pues, acudieron a los sumos sacerdotes y
a los fariseos, y les dijeron: ``¿Por qué no le habéis traído?''

\bibleverse{46} Los oficiales respondieron: ``¡Nunca nadie habló como
este hombre!'' \footnote{\textbf{7:46} Mat 7,28-29}

\bibleverse{47} Los fariseos, por tanto, les respondieron: ``¿No
estaréis también vosotros engañados, verdad? \bibleverse{48} ¿Acaso ha
creído en él alguno de los gobernantes o alguno de los fariseos?
\bibleverse{49} Pero esta multitud que no conoce la ley es maldita''.

\bibleverse{50} Nicodemo (el que vino a él de noche, siendo uno de
ellos) les dijo: \footnote{\textbf{7:50} Juan 3,1-2} \bibleverse{51}
``¿Acaso nuestra ley juzga a un hombre si antes no lo oye personalmente
y sabe lo que hace?'' \footnote{\textbf{7:51} Deut 1,16-17}

\bibleverse{52} Le respondieron: ``¿Tú también eres de Galilea? Busca y
ve que no ha surgido ningún profeta de Galilea''.

\bibleverse{53} Cada uno se fue a su casa,

\hypertarget{jesuxfas-y-la-aduxfaltera}{%
\subsection{Jesús y la adúltera}\label{jesuxfas-y-la-aduxfaltera}}

\hypertarget{section-7}{%
\section{8}\label{section-7}}

\bibleverse{1} pero Jesús fue al Monte de los Olivos.

\bibleverse{2} Por la mañana, muy temprano, entró de nuevo en el templo,
y toda la gente acudió a él. Se sentó y les enseñó. \bibleverse{3} Los
escribas y los fariseos trajeron a una mujer sorprendida por el
adulterio. Tras ponerla en medio, \bibleverse{4} le dijeron: ``Maestro,
hemos encontrado a esta mujer en adulterio, en el acto mismo.
\bibleverse{5} Ahora bien, en nuestra ley, Moisés nos ordenó apedrear a
tales mujeres. ¿Qué dices, pues, de ella?'' \footnote{\textbf{8:5} Lev
  20,10} \bibleverse{6} Dijeron esto poniéndole a prueba, para tener de
qué acusarle. Pero Jesús se inclinó y escribió en el suelo con el dedo.
\bibleverse{7} Pero como le seguían preguntando, levantó la vista y les
dijo: ``El que esté libre de pecado entre vosotros, que tire la primera
piedra contra ella.'' \footnote{\textbf{8:7} Rom 2,1} \bibleverse{8} De
nuevo se agachó y escribió en el suelo con el dedo.

\bibleverse{9} Ellos, al oírlo, condenados por su conciencia, salieron
uno por uno, empezando por el más viejo hasta el último. Jesús se quedó
solo con la mujer donde estaba, en medio. \bibleverse{10} Jesús,
levantándose, la vio y le dijo: ``Mujer, ¿dónde están tus acusadores?
¿Nadie te ha condenado?''

\bibleverse{11} Ella dijo: ``Nadie, Señor''. Jesús dijo: ``Tampoco yo te
condeno. Sigue tu camino. Desde ahora, no peques más.'' \footnote{\textbf{8:11}
  Juan 5,14}

\hypertarget{el-testimonio-de-suxed-mismo-de-jesuxfas-como-la-luz-del-mundo-y-el-hijo-de-dios}{%
\subsection{El testimonio de sí mismo de Jesús como la luz del mundo y
el Hijo de
Dios}\label{el-testimonio-de-suxed-mismo-de-jesuxfas-como-la-luz-del-mundo-y-el-hijo-de-dios}}

\bibleverse{12} Por eso, Jesús les habló de nuevo, diciendo: ``Yo soy la
luz del mundo. El que me sigue no caminará en la oscuridad, sino que
tendrá la luz de la vida''. \footnote{\textbf{8:12} Is 49,6; Juan 1,5;
  Juan 1,9; Mat 5,14-16}

\bibleverse{13} Los fariseos, por tanto, le dijeron: ``Das testimonio de
ti mismo. Tu testimonio no es válido''.

\bibleverse{14} Jesús les respondió: ``Aunque yo dé testimonio de mí
mismo, mi testimonio es verdadero, porque sé de dónde vengo y a dónde
voy; pero ustedes no saben de dónde vengo ni a dónde voy. \footnote{\textbf{8:14}
  Juan 5,31; Juan 7,28} \bibleverse{15} Ustedes juzgan según la carne.
Yo no juzgo a nadie. \footnote{\textbf{8:15} Juan 3,17} \bibleverse{16}
Aunque juzgue, mi juicio es verdadero, porque no estoy solo, sino que
estoy con el Padre que me envió. \bibleverse{17} También está escrito en
tu ley que el testimonio de dos personas es válido. \footnote{\textbf{8:17}
  Deut 19,15} \bibleverse{18} Yo soy uno que da testimonio de mí mismo,
y el Padre que me envió da testimonio de mí''.

\bibleverse{19} Por eso le dijeron: ``¿Dónde está tu Padre?''. Jesús
respondió: ``No me conocéis ni a mí ni a mi Padre. Si me conocieran,
conocerían también a mi Padre''. \footnote{\textbf{8:19} Juan 14,7}
\bibleverse{20} Jesús dijo estas palabras en el tesoro, mientras
enseñaba en el templo. Pero nadie lo arrestó, porque aún no había
llegado su hora. \footnote{\textbf{8:20} Juan 7,30}

\hypertarget{jesuxfas-da-testimonio-del-profundo-abismo-que-lo-separa-de-los-juduxedos-seguxfan-sus-oruxedgenes}{%
\subsection{Jesús da testimonio del profundo abismo que lo separa de los
judíos según sus
orígenes}\label{jesuxfas-da-testimonio-del-profundo-abismo-que-lo-separa-de-los-juduxedos-seguxfan-sus-oruxedgenes}}

\bibleverse{21} Por eso, Jesús les dijo de nuevo: ``Me voy, y me
buscaréis, y moriréis en vuestros pecados. Donde yo voy, vosotros no
podéis venir''. \footnote{\textbf{8:21} Juan 7,34-35; Juan 13,33}

\bibleverse{22} Los judíos, por tanto, dijeron: ``¿Se va a matar, porque
dice: ``A donde yo voy, tú no puedes venir''?''

\bibleverse{23} Les dijo: ``Vosotros sois de abajo. Yo soy de arriba.
Vosotros sois de este mundo. Yo no soy de este mundo. \footnote{\textbf{8:23}
  Juan 3,31} \bibleverse{24} Por eso os he dicho que moriréis en
vuestros pecados; porque si no creéis que yo soy, moriréis en vuestros
pecados.''

\bibleverse{25} Le dijeron, pues, ``¿Quién eres tú?''. Jesús les dijo:
``Justo lo que os he estado diciendo desde el principio. \bibleverse{26}
Tengo muchas cosas que decir y juzgar sobre vosotros. Sin embargo, el
que me ha enviado es veraz; y lo que he oído de él, eso digo al mundo.''

\bibleverse{27} No entendían que les hablaba del Padre. \bibleverse{28}
Por eso Jesús les dijo: ``Cuando hayáis levantado al Hijo del Hombre,
entonces sabréis que yo soy, y que no hago nada por mí mismo, sino que,
como me enseñó mi Padre, digo estas cosas. \footnote{\textbf{8:28} Juan
  3,14; Juan 12,32} \bibleverse{29} El que me ha enviado está conmigo.
El Padre no me ha dejado solo, porque siempre hago las cosas que le
agradan.''

\hypertarget{el-testimonio-de-jesuxfas-de-su-filiaciuxf3n-de-dios-y-de-la-esclavitud-del-pecado-de-los-juduxedos-a-pesar-de-su-descendencia-de-abraham}{%
\subsection{El testimonio de Jesús de su filiación de Dios y de la
esclavitud del pecado de los judíos a pesar de su descendencia de
Abraham}\label{el-testimonio-de-jesuxfas-de-su-filiaciuxf3n-de-dios-y-de-la-esclavitud-del-pecado-de-los-juduxedos-a-pesar-de-su-descendencia-de-abraham}}

\bibleverse{30} Mientras decía estas cosas, muchos creían en él.
\bibleverse{31} Entonces Jesús dijo a los judíos que habían creído en
él: ``Si permanecéis en mi palabra, entonces sois verdaderamente mis
discípulos. \footnote{\textbf{8:31} Juan 15,7} \bibleverse{32}
Conoceréis la verdad, y la verdad os hará libres''.

\bibleverse{33} Ellos le respondieron: ``Somos descendientes de Abraham,
y nunca hemos sido esclavos de nadie. ¿Cómo dices que serás libre?''
\footnote{\textbf{8:33} Mat 3,9}

\bibleverse{34} Jesús les contestó: ``De cierto os digo que todo el que
comete pecado es siervo del pecado. \bibleverse{35} Un siervo no vive en
la casa para siempre. Un hijo permanece para siempre. \bibleverse{36}
Por eso, si el Hijo os hace libres, seréis verdaderamente libres.
\footnote{\textbf{8:36} Rom 6,16; Rom 6,18; Rom 6,22}

\hypertarget{los-juduxedos-incruxe9dulos-no-son-hijos-de-abraham-ni-de-dios-sino-hijos-del-diablo}{%
\subsection{Los judíos incrédulos no son hijos de Abraham ni de Dios,
sino hijos del
diablo}\label{los-juduxedos-incruxe9dulos-no-son-hijos-de-abraham-ni-de-dios-sino-hijos-del-diablo}}

\bibleverse{37} Yo sé que sois descendientes de Abraham, y sin embargo
buscáis matarme, porque mi palabra no encuentra lugar en vosotros.
\bibleverse{38} Yo digo lo que he visto con mi Padre; y vosotros también
hacéis lo que habéis visto con vuestro padre.''

\bibleverse{39} Ellos le respondieron: ``Nuestro padre es Abraham''.
Jesús les dijo: ``Si fuerais hijos de Abraham, haríais las obras de
Abraham. \bibleverse{40} Pero ahora buscáis matarme a mí, un hombre que
os ha dicho la verdad que he oído de Dios. Abraham no hizo esto.
\bibleverse{41} Vosotros hacéis las obras de vuestro padre''. Le
dijeron: ``No hemos nacido de la inmoralidad sexual. Tenemos un solo
Padre, Dios''.

\bibleverse{42} Por eso Jesús les dijo: ``Si Dios fuera vuestro padre,
me amaríais, porque he salido y vengo de Dios. Pues no he venido por mí
mismo, sino que él me ha enviado. \bibleverse{43} ¿Por qué no entendéis
mi discurso? Porque no puedes escuchar mi palabra. \footnote{\textbf{8:43}
  1Cor 2,14} \bibleverse{44} Vosotros sois de vuestro padre el diablo, y
queréis hacer los deseos de vuestro padre. Él es un asesino desde el
principio, y no se mantiene en la verdad, porque no hay verdad en él.
Cuando habla una mentira, habla por su cuenta; porque es un mentiroso y
el padre de la mentira. \footnote{\textbf{8:44} 1Jn 3,8-10; Gén 3,4; Gén
  3,19} \bibleverse{45} Pero porque digo la verdad, no me creéis.
\bibleverse{46} ¿Quién de vosotros me convence de pecado? Si digo la
verdad, ¿por qué no me creéis? \footnote{\textbf{8:46} 2Cor 5,21; 1Pe
  2,22; 1Jn 3,5; Heb 4,15} \bibleverse{47} El que es de Dios escucha las
palabras de Dios. Por eso no oís, porque no sois de Dios''. \footnote{\textbf{8:47}
  Juan 18,37}

\hypertarget{el-testimonio-de-jesuxfas-de-la-majestad-de-suxed-mismo-y-de-su-superioridad-sobre-abraham}{%
\subsection{El testimonio de Jesús de la majestad de sí mismo y de su
superioridad sobre
Abraham}\label{el-testimonio-de-jesuxfas-de-la-majestad-de-suxed-mismo-y-de-su-superioridad-sobre-abraham}}

\bibleverse{48} Entonces los judíos le respondieron: ``¿No decimos bien
que eres samaritano y tienes un demonio?'' \footnote{\textbf{8:48} Juan
  7,20}

\bibleverse{49} Jesús respondió: ``Yo no tengo un demonio, pero honro a
mi Padre y ustedes me deshonran. \bibleverse{50} Pero yo no busco mi
propia gloria. Hay uno que busca y juzga. \bibleverse{51} Ciertamente,
les digo que si una persona cumple mi palabra, nunca verá la muerte''.
\footnote{\textbf{8:51} Juan 6,40; Juan 6,47}

\bibleverse{52} Entonces los judíos le dijeron: ``Ahora sabemos que
tienes un demonio. Abraham murió, así como los profetas; y tú dices: `Si
un hombre guarda mi palabra, no probará jamás la muerte'.
\bibleverse{53} ¿Eres tú mayor que nuestro padre Abraham, que murió? Los
profetas murieron. ¿Quién te crees que eres?''

\bibleverse{54} Jesús respondió: ``Si me glorifico a mí mismo, mi gloria
no es nada. Quien me glorifica es mi Padre, del que decís que es nuestro
Dios. \bibleverse{55} Ustedes no lo han conocido, pero yo sí lo conozco.
Si dijera: ``No lo conozco'', sería como vosotros, un mentiroso. Pero yo
lo conozco y cumplo su palabra. \footnote{\textbf{8:55} Juan 7,28-29}
\bibleverse{56} Vuestro padre Abraham se alegró al ver mi día. Lo vio y
se alegró''.

\bibleverse{57} Los judíos le dijeron: ``¡Todavía no tienes cincuenta
años! ¿Has visto a Abraham?''

\bibleverse{58} Jesús les dijo: ``Os aseguro que antes de que Abraham
llegara a existir, YO SOY.'' \footnote{\textbf{8:58} Juan 1,1-2}

\bibleverse{59} Por eso tomaron piedras para arrojárselas, pero Jesús se
escondió y salió del templo, pasando por en medio de ellos, y así pasó
de largo. \footnote{\textbf{8:59} Juan 10,31}

\hypertarget{la-curaciuxf3n-del-ciego-de-nacimiento-en-suxe1bado}{%
\subsection{La curación del ciego de nacimiento en
sábado}\label{la-curaciuxf3n-del-ciego-de-nacimiento-en-suxe1bado}}

\hypertarget{section-8}{%
\section{9}\label{section-8}}

\bibleverse{1} Al pasar, vio a un hombre ciego de nacimiento.
\bibleverse{2} Sus discípulos le preguntaron: ``Rabí, ¿quién pecó, este
hombre o sus padres, para que naciera ciego?'' \footnote{\textbf{9:2}
  Luc 13,2}

\bibleverse{3} Jesús respondió: ``Este hombre no pecó, ni tampoco sus
padres, sino para que las obras de Dios se manifiesten en él.
\footnote{\textbf{9:3} Juan 11,4} \bibleverse{4} Yo debo hacer las obras
del que me envió mientras es de día. Se acerca la noche, cuando nadie
puede trabajar. \footnote{\textbf{9:4} Juan 5,17; Jer 13,16}
\bibleverse{5} Mientras estoy en el mundo, soy la luz del mundo''.
\footnote{\textbf{9:5} Juan 12,35; Juan 8,12} \bibleverse{6} Dicho esto,
escupió en el suelo, hizo lodo con la saliva, ungió los ojos del ciego
con el lodo, \footnote{\textbf{9:6} Mar 8,23} \bibleverse{7} y le dijo:
``Ve, lávate en el estanque de Siloé'' (que significa ``Enviado''). Así
que se fue, se lavó y volvió viendo.

\bibleverse{8} Por eso, los vecinos y los que habían visto que era ciego
antes decían: ``¿No es éste el que se sentaba a pedir limosna?''
\bibleverse{9} Otros decían: ``Es él''. Y otros decían: ``Se parece a
él''. Dijo: ``Yo soy''.

\bibleverse{10} Por eso le preguntaban: ``¿Cómo se te abrieron los
ojos?''.

\bibleverse{11} Respondió: ``Un hombre llamado Jesús hizo lodo, me untó
los ojos y me dijo: ``Ve al estanque de Siloé y lávate''. Así que fui y
me lavé, y recibí la vista''.

\bibleverse{12} Entonces le preguntaron: ``¿Dónde está?''. Dijo: ``No lo
sé''.

\hypertarget{el-primer-interrogatorio-de-los-fariseos}{%
\subsection{El primer interrogatorio de los
fariseos}\label{el-primer-interrogatorio-de-los-fariseos}}

\bibleverse{13} Llevaron al que había sido ciego a los fariseos.
\bibleverse{14} Era sábado cuando Jesús hizo el lodo y le abrió los
ojos. \bibleverse{15} También los fariseos le preguntaron cómo había
recibido la vista. Él les dijo: ``Me puso barro en los ojos, me lavé y
veo''.

\bibleverse{16} Por eso algunos de los fariseos decían: ``Este hombre no
es de Dios, porque no guarda el sábado''. Otros decían: ``¿Cómo puede
hacer tales señales un hombre que es pecador?''. Así que hubo división
entre ellos.

\bibleverse{17} Por eso volvieron a preguntar al ciego: ``¿Qué dices de
él, porque te ha abierto los ojos?'' Dijo: ``Es un profeta''.

\hypertarget{el-interrogatorio-de-los-padres}{%
\subsection{El interrogatorio de los
padres}\label{el-interrogatorio-de-los-padres}}

\bibleverse{18} Los judíos, por tanto, no creían respecto a él que había
sido ciego y que había recibido la vista, hasta que llamaron a los
padres del que había recibido la vista, \bibleverse{19} y les
preguntaron: ``¿Es éste vuestro hijo, del que decís que nació ciego?
¿Cómo es que ahora ve?''

\bibleverse{20} Sus padres les respondieron: ``Sabemos que éste es
nuestro hijo y que nació ciego; \bibleverse{21} pero cómo ve ahora, no
lo sabemos; o quién le abrió los ojos, no lo sabemos. Es mayor de edad.
Pregúntale a él. Él hablará por sí mismo''. \bibleverse{22} Sus padres
decían estas cosas porque temían a los judíos, pues éstos ya habían
acordado que si alguno lo confesaba como Cristo, sería expulsado de la
sinagoga. \footnote{\textbf{9:22} Juan 7,13; Juan 12,42} \bibleverse{23}
Por eso sus padres dijeron: ``Es mayor de edad. Pregúntale a él''.

\hypertarget{el-segundo-interrogatorio-del-curado}{%
\subsection{El segundo interrogatorio del
curado}\label{el-segundo-interrogatorio-del-curado}}

\bibleverse{24} Entonces llamaron por segunda vez al ciego y le dijeron:
``Da gloria a Dios. Sabemos que este hombre es un pecador''.

\bibleverse{25} Por eso respondió: ``No sé si es pecador. Una cosa sí
sé: que aunque estaba ciego, ahora veo''.

\bibleverse{26} Le volvieron a decir: ``¿Qué te ha hecho? ¿Cómo te ha
abierto los ojos?''

\bibleverse{27} Él les respondió: ``Ya os lo he dicho, y no me habéis
escuchado. ¿Por qué queréis oírlo otra vez? No queréis también haceros
sus discípulos, ¿verdad?''.

\bibleverse{28} Le insultaron y le dijeron: ``Tú eres su discípulo, pero
nosotros somos discípulos de Moisés. \bibleverse{29} Sabemos que Dios ha
hablado con Moisés. Pero en cuanto a este hombre, no sabemos de dónde
viene''.

\bibleverse{30} El hombre les respondió: ``¡Qué maravilla! No sabéis de
dónde viene, y sin embargo me ha abierto los ojos. \bibleverse{31}
Sabemos que Dios no escucha a los pecadores, pero si alguien es adorador
de Dios y hace su voluntad, le escucha. \footnote{\textbf{9:31} Sal
  66,18; Prov 15,29; Is 1,15} \bibleverse{32} Desde el principio del
mundo no se ha oído decir que alguien haya abierto los ojos a un ciego
de nacimiento. \bibleverse{33} Si este hombre no viniera de Dios, no
podría hacer nada''.

\bibleverse{34} Le respondieron: ``Tú, que has nacido en pecado, ¿nos
enseñas?''. Entonces le echaron.

\hypertarget{la-fe-del-sanado-en-jesuxfas-jesuxfas-como-la-luz-de-los-que-no-ven-y-como-la-ceguera-de-los-que-ven}{%
\subsection{La fe del sanado en Jesús; Jesús como la luz de los que no
ven y como la ceguera de los que
ven}\label{la-fe-del-sanado-en-jesuxfas-jesuxfas-como-la-luz-de-los-que-no-ven-y-como-la-ceguera-de-los-que-ven}}

\bibleverse{35} Jesús oyó que lo habían echado, y encontrándolo, le
dijo: ``¿Crees en el Hijo de Dios?''

\bibleverse{36} Él respondió: ``¿Quién es, Señor, para que crea en él?''

\bibleverse{37} Jesús le dijo: ``Pues lo has visto, y es él quien habla
contigo.'' \footnote{\textbf{9:37} Juan 4,26}

\bibleverse{38} Dijo: ``¡Señor, creo!'' y lo adoró.

\bibleverse{39} Jesús dijo: ``He venido a este mundo para juzgar, para
que los que no ven vean y para que los que ven se vuelvan ciegos''.
\footnote{\textbf{9:39} Mat 13,11-15}

\bibleverse{40} Los fariseos que estaban con él oyeron estas cosas y le
dijeron: ``¿También nosotros somos ciegos?''

\bibleverse{41} Jesús les dijo: ``Si fuerais ciegos, no tendríais
pecado; pero ahora decís: ``Vemos''. Por eso vuestro pecado permanece.
\footnote{\textbf{9:41} Prov 26,12; Juan 15,22}

\hypertarget{el-lenguaje-figurado-del-pastor-y-ladruxf3n-y-del-buen-pastor-y-asalariado}{%
\subsection{El lenguaje figurado del pastor y ladrón y del buen pastor y
asalariado}\label{el-lenguaje-figurado-del-pastor-y-ladruxf3n-y-del-buen-pastor-y-asalariado}}

\hypertarget{section-9}{%
\section{10}\label{section-9}}

\bibleverse{1} ``Os aseguro que el que no entra por la puerta en el
redil de las ovejas, sino que sube por otro camino, es un ladrón y un
salteador. \bibleverse{2} Pero el que entra por la puerta es el pastor
de las ovejas. \bibleverse{3} El guardián le abre la puerta, y las
ovejas escuchan su voz. Llama a sus ovejas por su nombre y las saca.
\bibleverse{4} Cada vez que saca a sus ovejas, va delante de ellas; y
las ovejas le siguen, porque conocen su voz. \bibleverse{5} No seguirán
en absoluto a un extraño, sino que huirán de él, porque no conocen la
voz de los extraños.'' \bibleverse{6} Jesús les dijo esta parábola, pero
no entendieron lo que les decía.

\hypertarget{yo-soy-la-puerta-para-las-ovejas}{%
\subsection{¡Yo soy la puerta para las
ovejas!}\label{yo-soy-la-puerta-para-las-ovejas}}

\bibleverse{7} Por eso Jesús les volvió a decir: ``Os aseguro que yo soy
la puerta de las ovejas. \bibleverse{8} Todos los que vinieron antes que
yo son ladrones y salteadores, pero las ovejas no les hicieron caso.
\bibleverse{9} Yo soy la puerta. Si alguien entra por mí, se salvará, y
entrará y saldrá y hallará pastos. \footnote{\textbf{10:9} Juan 14,6}
\bibleverse{10} El ladrón sólo viene a robar, matar y destruir. Yo he
venido para que tengan vida y la tengan en abundancia.

\hypertarget{jesuxfas-como-el-buen-pastor}{%
\subsection{Jesús como el buen
pastor}\label{jesuxfas-como-el-buen-pastor}}

\bibleverse{11} ``Yo soy el buen pastor. El buen pastor da su vida por
las ovejas. \bibleverse{12} El que es asalariado y no pastor, que no es
dueño de las ovejas, ve venir al lobo, deja las ovejas y huye. El lobo
arrebata las ovejas y las dispersa. \footnote{\textbf{10:12} Sal 23,1;
  Is 40,11; Ezeq 34,11-23; Juan 15,13; Heb 13,20} \bibleverse{13} El
jornalero huye porque es jornalero y no cuida de las ovejas.
\bibleverse{14} Yo soy el buen pastor. Conozco a las mías, y soy
conocido por las mías; \footnote{\textbf{10:14} 2Tim 2,19}
\bibleverse{15} así como el Padre me conoce, y yo conozco al Padre. Yo
doy mi vida por las ovejas. \bibleverse{16} Tengo otras ovejas que no
son de este redil. Debo traerlas también, y oirán mi voz. Serán un solo
rebaño con un solo pastor. \footnote{\textbf{10:16} Juan 11,52; Hech
  10,34-35} \bibleverse{17} Por eso el Padre me ama, porque doy mi vida
para volver a tomarla. \bibleverse{18} Nadie me la quita, sino que yo
mismo la pongo. Tengo poder para ponerla, y tengo poder para volver a
tomarla. Este mandamiento lo recibí de mi Padre''. \footnote{\textbf{10:18}
  Juan 5,26}

\bibleverse{19} Por eso volvió a surgir una división entre los judíos a
causa de estas palabras. \footnote{\textbf{10:19} Juan 7,43; Juan 9,16}
\bibleverse{20} Muchos de ellos decían: ``¡Tiene un demonio y está loco!
¿Por qué le escucháis?'' \footnote{\textbf{10:20} Juan 7,20; Mar 3,21}
\bibleverse{21} Otros decían: ``Estos no son los dichos de un poseído
por un demonio. No es posible que un demonio abra los ojos de los
ciegos, ¿verdad?''

\hypertarget{la-uxfaltima-justificaciuxf3n-de-jesuxfas-a-los-juduxedos-en-la-fiesta-de-la-dedicaciuxf3n-del-templo}{%
\subsection{La última justificación de Jesús a los judíos en la fiesta
de la dedicación del
templo}\label{la-uxfaltima-justificaciuxf3n-de-jesuxfas-a-los-juduxedos-en-la-fiesta-de-la-dedicaciuxf3n-del-templo}}

\bibleverse{22} Era la fiesta de la Dedicación en Jerusalén.
\bibleverse{23} Era invierno, y Jesús andaba por el templo, en el
pórtico de Salomón. \footnote{\textbf{10:23} Hech 3,11} \bibleverse{24}
Los judíos se acercaron a él y le dijeron: ``¿Hasta cuándo nos vas a
tener en suspenso? Si eres el Cristo, dínoslo claramente''.

\bibleverse{25} Jesús les respondió: ``Os lo he dicho, y no creéis. Las
obras que hago en nombre de mi Padre, éstas dan testimonio de mí.
\footnote{\textbf{10:25} Juan 5,36} \bibleverse{26} Pero vosotros no
creéis, porque no sois de mis ovejas, como os he dicho. \footnote{\textbf{10:26}
  Juan 8,45; Juan 8,47} \bibleverse{27} Mis ovejas oyen mi voz, y yo las
conozco, y me siguen. \bibleverse{28} Yo les doy vida eterna. Nunca
perecerán, y nadie las arrebatará de mi mano. \bibleverse{29} Mi Padre,
que me las ha dado, es más grande que todos. Nadie puede arrebatarlos de
la mano de mi Padre. \bibleverse{30} Yo y el Padre somos uno''.

\bibleverse{31} Por eso los judíos volvieron a tomar piedras para
apedrearlo. \footnote{\textbf{10:31} Juan 8,59} \bibleverse{32} Jesús
les respondió: ``Os he mostrado muchas obras buenas de mi Padre. ¿Por
cuál de esas obras me apedreáis?''

\bibleverse{33} Los judíos le respondieron: ``No te apedreamos por una
obra buena, sino por blasfemia, porque tú, siendo hombre, te haces
Dios''. \footnote{\textbf{10:33} Juan 5,18; Mat 9,3; Mat 26,65}

\bibleverse{34} Jesús les contestó: ``¿No está escrito en vuestra ley:
``Yo dije que sois dioses''? \footnote{\textbf{10:34} Salmo 82:6}
\bibleverse{35} Si los llamó dioses, a los que vino la palabra de Dios
(y la Escritura no puede ser quebrantada), \bibleverse{36} ¿decís de
aquel a quien el Padre santificó y envió al mundo: ``Tú blasfemas'',
porque yo dije: ``Yo soy el Hijo de Dios''? \footnote{\textbf{10:36}
  Juan 5,17-20} \bibleverse{37} Si no hago las obras de mi Padre, no me
creáis. \bibleverse{38} Pero si las hago, aunque no me creáis, creed en
las obras, para que sepáis y creáis que el Padre está en mí, y yo en el
Padre.''

\bibleverse{39} Volvieron a buscarlo para apresarlo, pero se les escapó
de las manos. \footnote{\textbf{10:39} Juan 8,59; Luc 4,30}

\hypertarget{jesuxfas-y-luxe1zaro-jesuxfas-como-la-resurrecciuxf3n-y-la-vida}{%
\subsection{Jesús y Lázaro; Jesús como la resurrección y la
vida}\label{jesuxfas-y-luxe1zaro-jesuxfas-como-la-resurrecciuxf3n-y-la-vida}}

\bibleverse{40} Volvió a pasar el Jordán, al lugar donde Juan bautizaba
al principio, y se quedó allí. \footnote{\textbf{10:40} Juan 1,28}
\bibleverse{41} Muchos se acercaron a él. Decían: ``Ciertamente Juan no
hizo ninguna señal, pero todo lo que Juan dijo de este hombre es
verdad''. \bibleverse{42} Muchos creyeron allí en él.

\hypertarget{section-10}{%
\section{11}\label{section-10}}

\bibleverse{1} Un hombre estaba enfermo, Lázaro, de Betania, del pueblo
de María y de su hermana Marta. \footnote{\textbf{11:1} Luc 10,38-39}
\bibleverse{2} Era aquella María, que había ungido al Señor con ungüento
y enjugado sus pies con sus cabellos, cuyo hermano Lázaro estaba
enfermo. \footnote{\textbf{11:2} Juan 12,3} \bibleverse{3} Las hermanas,
pues, enviaron a decirle: ``Señor, he aquí que está enfermo aquel a
quien tienes gran afecto.''

\bibleverse{4} Pero Jesús, al oírlo, dijo: ``Esta enfermedad no es para
la muerte, sino para la gloria de Dios, para que el Hijo de Dios sea
glorificado por ella.'' \footnote{\textbf{11:4} Juan 9,3} \bibleverse{5}
Jesús amaba a Marta, a su hermana y a Lázaro. \bibleverse{6} Por eso, al
saber que estaba enfermo, se quedó dos días en el lugar donde estaba.
\bibleverse{7} Luego, después de esto, dijo a los discípulos: ``Vamos a
Judea de nuevo''.

\bibleverse{8} Los discípulos le preguntaron: ``Rabí, los judíos querían
apedrearte. ¿Vas a ir allí de nuevo?'' \footnote{\textbf{11:8} Juan
  10,31}

\bibleverse{9} Jesús respondió: ``¿No hay doce horas de luz? Si un
hombre camina de día, no tropieza, porque ve la luz de este mundo.
\footnote{\textbf{11:9} Juan 9,4-5} \bibleverse{10} Pero si un hombre
camina de noche, tropieza, porque la luz no está en él''. \footnote{\textbf{11:10}
  Juan 12,35} \bibleverse{11} Dijo estas cosas, y después les dijo:
``Nuestro amigo Lázaro se ha dormido, pero yo voy para despertarlo del
sueño.'' \footnote{\textbf{11:11} Mat 9,24}

\bibleverse{12} Entonces los discípulos dijeron: ``Señor, si se ha
dormido, se recuperará''.

\bibleverse{13} Ahora bien, Jesús había hablado de su muerte, pero ellos
pensaron que hablaba de descansar en el sueño. \bibleverse{14} Entonces
Jesús les dijo claramente: ``Lázaro ha muerto. \bibleverse{15} Me alegro
por vosotros de no haber estado allí, para que creáis. Sin embargo,
vayamos a verlo''.

\bibleverse{16} Entonces Tomás, que se llama Dídimo,\footnote{\textbf{11:16}
  ``Dídimo'' significa ``gemelo''.} dijo a sus condiscípulos: ``Vayamos
también nosotros, para morir con él.'' \footnote{\textbf{11:16} Juan
  20,24-28}

\hypertarget{el-regreso-de-jesuxfas-a-betania-su-encuentro-con-martha-y-maria}{%
\subsection{El regreso de Jesús a Betania; su encuentro con Martha y
Maria}\label{el-regreso-de-jesuxfas-a-betania-su-encuentro-con-martha-y-maria}}

\bibleverse{17} Cuando llegó Jesús, se dio cuenta de que ya llevaba
cuatro días en el sepulcro. \bibleverse{18} Betania estaba cerca de
Jerusalén, a unos quince pasos\footnote{\textbf{11:18} 15 estadios son
  unos 2,8 kilómetros o 1,7 millas} . \bibleverse{19} Muchos de los
judíos se habían reunido con las mujeres en torno a Marta y María, para
consolarlas por su hermano. \bibleverse{20} Cuando Marta se enteró de
que Jesús venía, fue a recibirlo, pero María se quedó en la casa.
\bibleverse{21} Entonces Marta dijo a Jesús: ``Señor, si hubieras estado
aquí, mi hermano no habría muerto. \bibleverse{22} Incluso ahora sé que
todo lo que pidas a Dios, Dios te lo dará''.

\bibleverse{23} Jesús le dijo: ``Tu hermano resucitará''.

\bibleverse{24} Marta le dijo: ``Sé que resucitará en la resurrección en
el último día''. \footnote{\textbf{11:24} Juan 5,28-29; Juan 6,40; Mat
  22,23-33}

\bibleverse{25} Jesús le dijo: ``Yo soy la resurrección y la vida. El
que cree en mí seguirá viviendo, aunque muera. \bibleverse{26} El que
vive y cree en mí no morirá jamás. ¿Crees en esto?'' \footnote{\textbf{11:26}
  Juan 8,51}

\bibleverse{27} Ella le dijo: ``Sí, Señor. He llegado a creer que tú
eres el Cristo, el Hijo de Dios, el que viene al mundo''. \footnote{\textbf{11:27}
  Mat 16,16}

\bibleverse{28} Cuando hubo dicho esto, se fue y llamó a María, su
hermana, en secreto, diciendo: ``El Maestro está aquí y te llama.''

\bibleverse{29} Al oír esto, se levantó rápidamente y fue hacia él.
\bibleverse{30} Pero Jesús no había entrado aún en la aldea, sino que
estaba en el lugar donde Marta lo había encontrado. \bibleverse{31}
Entonces los judíos que estaban con ella en la casa y la consolaban, al
ver que María se levantaba rápidamente y salía, la siguieron diciendo:
``Va al sepulcro a llorar allí.''

\bibleverse{32} Por eso, cuando María llegó a donde estaba Jesús y lo
vio, se postró a sus pies, diciéndole: ``Señor, si hubieras estado aquí,
mi hermano no habría muerto.''

\bibleverse{33} Cuando Jesús la vio llorar, y a los judíos que venían
con ella, gimió en el espíritu y se turbó, \footnote{\textbf{11:33} Juan
  13,21}

\hypertarget{jesuxfas-en-la-tumba-y-su-oraciuxf3n-la-resurrecciuxf3n-de-luxe1zaro-de-entre-los-muertos}{%
\subsection{Jesús en la tumba y su oración; la resurrección de Lázaro de
entre los
muertos}\label{jesuxfas-en-la-tumba-y-su-oraciuxf3n-la-resurrecciuxf3n-de-luxe1zaro-de-entre-los-muertos}}

\bibleverse{34} y dijo: ``¿Dónde lo habéis puesto?'' Le dijeron:
``Señor, ven a ver''.

\bibleverse{35} Jesús lloró.

\bibleverse{36} Por eso los judíos decían: ``¡Vean cuánto afecto le
tenía!''. \bibleverse{37} Algunos de ellos decían: ``¿No podía este
hombre, que abrió los ojos del ciego, evitar que éste muriera?''
\footnote{\textbf{11:37} Juan 9,7}

\bibleverse{38} Jesús, gimiendo de nuevo en su interior, llegó al
sepulcro. Era una cueva, y una piedra estaba apoyada en ella.
\footnote{\textbf{11:38} Mat 27,60} \bibleverse{39} Jesús dijo: ``Quita
la piedra''. Marta, la hermana del que había muerto, le dijo: ``Señor, a
estas alturas hay un hedor, pues lleva cuatro días muerto''.

\bibleverse{40} Jesús le dijo: ¿No te dije que, si crees, verás la
gloria de Dios?''

\bibleverse{41} Entonces quitaron la piedra del lugar donde yacía el
muerto.\footnote{\textbf{11:41} NU omite ``del lugar donde yacía el
  muerto''.} Jesús levantó los ojos y dijo: ``Padre, te agradezco que me
hayas escuchado. \bibleverse{42} Sé que siempre me escuchas, pero a
causa de la multitud que está alrededor he dicho esto, para que crean
que tú me has enviado.'' \footnote{\textbf{11:42} Juan 12,30}
\bibleverse{43} Cuando hubo dicho esto, gritó a gran voz: ``¡Lázaro, ven
afuera!''

\bibleverse{44} El que estaba muerto salió, atado de pies y manos con
vendas, y su rostro estaba envuelto con un paño. Jesús les dijo:
``Libéralo y déjalo ir''.

\hypertarget{los-efectos-del-milagro-resoluciuxf3n-de-muerte-del-sumo-consejo-jesuxfas-escapa-a-efrauxedn}{%
\subsection{Los efectos del milagro; Resolución de muerte del sumo
consejo; Jesús escapa a
Efraín}\label{los-efectos-del-milagro-resoluciuxf3n-de-muerte-del-sumo-consejo-jesuxfas-escapa-a-efrauxedn}}

\bibleverse{45} Por eso, muchos de los judíos que se acercaron a María y
vieron lo que hacía Jesús creyeron en él. \bibleverse{46} Pero algunos
de ellos se fueron a los fariseos y les contaron las cosas que Jesús
había hecho. \bibleverse{47} Entonces los jefes de los sacerdotes y los
fariseos reunieron un consejo y dijeron: ``¿Qué hacemos? Porque este
hombre hace muchas señales. \footnote{\textbf{11:47} Mat 26,3-4}
\bibleverse{48} Si lo dejamos así, todos creerán en él, y vendrán los
romanos y nos quitarán nuestro lugar y nuestra nación.''

\bibleverse{49} Pero uno de ellos, Caifás, siendo sumo sacerdote aquel
año, les dijo: ``Vosotros no sabéis nada en absoluto, \bibleverse{50} ni
consideráis que nos convenga que un hombre muera por el pueblo, y que no
perezca toda la nación.'' \footnote{\textbf{11:50} Juan 18,14}
\bibleverse{51} Pero él no dijo esto por sí mismo, sino que, siendo sumo
sacerdote aquel año, profetizó que Jesús moriría por la nación,
\footnote{\textbf{11:51} Éxod 28,30; Núm 27,21} \bibleverse{52} y no
sólo por la nación, sino también para reunir en uno a los hijos de Dios
que están dispersos. \footnote{\textbf{11:52} Juan 7,35; Juan 10,16; 1Jn
  2,2} \bibleverse{53} Así que desde aquel día tomaron consejo para
darle muerte. \bibleverse{54} Así que Jesús ya no andaba abiertamente
entre los judíos, sino que se fue de allí al campo, cerca del desierto,
a una ciudad llamada Efraín. Allí se quedó con sus discípulos.

\bibleverse{55} Se acercaba la Pascua de los judíos. Muchos subieron del
campo a Jerusalén antes de la Pascua, para purificarse. \footnote{\textbf{11:55}
  2Cró 30,17-18} \bibleverse{56} Entonces buscaban a Jesús y hablaban
entre sí, estando en el templo: ``¿Qué pensáis, que no viene a la
fiesta?'' \bibleverse{57} Ahora bien, los jefes de los sacerdotes y los
fariseos habían ordenado que si alguien sabía dónde estaba, lo
denunciara para poder apresarlo.

\hypertarget{la-unciuxf3n-de-jesuxfas-consagraciuxf3n-de-la-muerte-en-betania}{%
\subsection{La unción de Jesús (consagración de la muerte) en
Betania}\label{la-unciuxf3n-de-jesuxfas-consagraciuxf3n-de-la-muerte-en-betania}}

\hypertarget{section-11}{%
\section{12}\label{section-11}}

\bibleverse{1} Seis días antes de la Pascua, Jesús llegó a Betania,
donde estaba Lázaro, que había estado muerto, al que resucitó de entre
los muertos. \footnote{\textbf{12:1} Juan 11,1; Juan 11,43}
\bibleverse{2} Y le prepararon allí una cena. Marta servía, pero Lázaro
era uno de los que se sentaban a la mesa con él. \bibleverse{3} Entonces
María tomó una libra\footnote{\textbf{12:3} 300 denarios era el salario
  de un año para un trabajador agrícola.} de ungüento de nardo puro, muy
precioso, y ungió los pies de Jesús y le secó los pies con sus cabellos.
La casa se llenó de la fragancia del ungüento. \footnote{\textbf{12:3}
  Luc 7,38}

\bibleverse{4} Entonces Judas Iscariote, hijo de Simón, uno de sus
discípulos, que lo iba a traicionar, dijo: \bibleverse{5} ``¿Por qué no
se vendió este ungüento por trescientos denarios y se dio a los
pobres?'' \bibleverse{6} Esto lo dijo, no porque se preocupara por los
pobres, sino porque era un ladrón, y teniendo la bolsa, solía robar lo
que se echaba en ella. \footnote{\textbf{12:6} Luc 8,3}

\bibleverse{7} Pero Jesús dijo: ``Dejadla en paz. Ha guardado esto para
el día de mi entierro. \bibleverse{8} Porque siempre tenéis a los pobres
con vosotros, pero no siempre me tenéis a mí''. \footnote{\textbf{12:8}
  Deut 15,11}

\bibleverse{9} Se enteró, pues, una gran multitud de judíos de que
estaba allí; y vinieron, no sólo por causa de Jesús, sino también para
ver a Lázaro, a quien había resucitado de entre los muertos.
\bibleverse{10} Pero los jefes de los sacerdotes conspiraron para dar
muerte también a Lázaro, \bibleverse{11} porque a causa de él muchos de
los judíos se fueron y creyeron en Jesús.

\hypertarget{la-entrada-de-jesuxfas-a-jerusaluxe9n-el-domingo-de-ramos}{%
\subsection{La entrada de Jesús a Jerusalén el Domingo de
Ramos}\label{la-entrada-de-jesuxfas-a-jerusaluxe9n-el-domingo-de-ramos}}

\bibleverse{12} Al día siguiente, una gran multitud había acudido a la
fiesta. Al enterarse de que Jesús venía a Jerusalén, \bibleverse{13}
tomaron las ramas de las palmeras y salieron a recibirlo, y gritaron:
``¡Hosanna!\footnote{\textbf{12:13} ``Hosanna'' significa ``sálvanos'' o
  ``ayúdanos, te rogamos''.} Bendito el que viene en nombre del
Señor,\footnote{\textbf{12:13} Salmo 118:25-26} el Rey de Israel''.
\footnote{\textbf{12:13} Sal 118,25-26}

\bibleverse{14} Jesús, habiendo encontrado un asnillo, se sentó en él.
Como está escrito: \bibleverse{15} ``No temas, hija de Sión. He aquí que
viene tu Rey, sentado en un pollino de asna''. \footnote{\textbf{12:15}
  Zacarías 9:9} \bibleverse{16} Sus discípulos no entendían estas cosas
al principio, pero cuando Jesús fue glorificado, entonces se acordaron
de que estas cosas estaban escritas sobre él, y de que le habían hecho
estas cosas. \bibleverse{17} La multitud, pues, que estaba con él cuando
llamó a Lázaro del sepulcro y lo resucitó de entre los muertos, daba
testimonio de ello. \bibleverse{18} Por esta razón también la multitud
fue a su encuentro, porque oyeron que había hecho esta señal.
\bibleverse{19} Entonces los fariseos decían entre sí: ``Mirad cómo no
conseguís nada. He aquí que el mundo ha ido tras él''. \footnote{\textbf{12:19}
  Juan 11,48}

\hypertarget{jesuxfas-anuncia-su-sufrimiento-mortal-y-su-subsiguiente-glorificaciuxf3n-como-salvador-del-mundo}{%
\subsection{Jesús anuncia su sufrimiento mortal y su subsiguiente
glorificación como salvador del
mundo}\label{jesuxfas-anuncia-su-sufrimiento-mortal-y-su-subsiguiente-glorificaciuxf3n-como-salvador-del-mundo}}

\bibleverse{20} Había algunos griegos entre los que subían a adorar en
la fiesta. \bibleverse{21} Estos, pues, se acercaron a Felipe, que era
de Betsaida de Galilea, y le preguntaron: ``Señor, queremos ver a
Jesús.'' \footnote{\textbf{12:21} Juan 1,44} \bibleverse{22} Felipe vino
y se lo comunicó a Andrés, y a su vez, Andrés vino con Felipe, y se lo
comunicaron a Jesús.

\bibleverse{23} Jesús les respondió: ``Ha llegado el momento de que el
Hijo del Hombre sea glorificado. \bibleverse{24} De cierto os digo que
si el grano de trigo no cae en la tierra y muere, queda solo. Pero si
muere, da mucho fruto. \footnote{\textbf{12:24} Rom 14,9; 1Cor 15,36}
\bibleverse{25} El que ama su vida la perderá. El que odia su vida en
este mundo, la conservará para la vida eterna. \footnote{\textbf{12:25}
  Mat 10,39; Mat 16,25; Luc 17,33} \bibleverse{26} El que me sirve, que
me siga. Donde yo esté, allí estará también mi servidor. Si alguien me
sirve, el Padre lo honrará. \footnote{\textbf{12:26} Juan 17,24}

\bibleverse{27} ``Ahora mi alma está turbada. ¿Qué voy a decir? ¿Padre,
sálvame de está hora? Pero he venido a está hora por esta causa.
\footnote{\textbf{12:27} Mat 26,38} \bibleverse{28} ¡Padre, glorifica tu
nombre!'' Entonces salió una voz del cielo que decía: ``Lo he
glorificado y lo volveré a glorificar''. \footnote{\textbf{12:28} Mat
  3,17; Mat 17,5; Juan 13,31}

\bibleverse{29} Por eso, la multitud que estaba de pie y lo oyó, dijo
que había tronado. Otros decían: ``Un ángel le ha hablado''.

\bibleverse{30} Jesús respondió: ``Esta voz no ha venido por mí, sino
por vosotros. \footnote{\textbf{12:30} Juan 11,42} \bibleverse{31} Ahora
es el juicio de este mundo. Ahora el príncipe de este mundo será echado
fuera. \footnote{\textbf{12:31} Juan 14,30; Juan 16,11; Luc 10,18}
\bibleverse{32} Y yo, si soy levantado de la tierra, atraeré a todos
hacia mí''. \footnote{\textbf{12:32} Juan 8,28} \bibleverse{33} Pero él
dijo esto, dando a entender con qué clase de muerte debía morir.

\bibleverse{34} La multitud le respondió: ``Hemos oído por la ley que el
Cristo permanece para siempre.\footnote{\textbf{12:34} Isaías 9:7;
  Daniel 2:44; Véase Isaías 53:8} ¿Cómo dices que el Hijo del Hombre
debe ser levantado? ¿Quién es ese Hijo del Hombre?'' \footnote{\textbf{12:34}
  Sal 110,4; Dan 7,14}

\bibleverse{35} Por eso Jesús les dijo: ``Todavía un poco de tiempo la
luz está con vosotros. Caminen mientras tienen la luz, para que las
tinieblas no los alcancen. El que camina en las tinieblas no sabe a
dónde va. \footnote{\textbf{12:35} Juan 11,10} \bibleverse{36} Mientras
tengáis la luz, creed en la luz, para que seáis hijos de la luz''. Jesús
dijo estas cosas, y se alejó y se escondió de ellos. \footnote{\textbf{12:36}
  Efes 5,9}

\hypertarget{la-revisiuxf3n-del-evangelista-de-la-actividad-puxfablica-de-jesuxfas}{%
\subsection{La revisión del evangelista de la actividad pública de
Jesús}\label{la-revisiuxf3n-del-evangelista-de-la-actividad-puxfablica-de-jesuxfas}}

\bibleverse{37} Pero aunque había hecho tantas señales delante de ellos,
no creían en él, \bibleverse{38} para que se cumpliera la palabra del
profeta Isaías que había dicho: ``Señor, ¿quién ha creído en nuestro
informe? ¿A quién se le ha revelado el brazo del Señor?'' \footnote{\textbf{12:38}
  Isaías 53:1}

\bibleverse{39} Por eso no podían creer, pues Isaías volvió a decir
\bibleverse{40} ``Ha cegado sus ojos y ha endurecido su corazón, para
que no vean con sus ojos, y entiendan con el corazón, y se conviertan, y
yo los sane\footnote{\textbf{12:40} Isaías 6:10} ''. \footnote{\textbf{12:40}
  Mat 13,14-15}

\bibleverse{41} Isaías dijo estas cosas al ver su gloria, y habló de él.
\footnote{\textbf{12:41} Isaías 6:1} \footnote{\textbf{12:41} Is 6,1}
\bibleverse{42} Sin embargo, incluso muchos de los gobernantes creyeron
en él, pero a causa de los fariseos no lo confesaron, para no ser
expulsados de la sinagoga, \footnote{\textbf{12:42} Juan 9,22}
\bibleverse{43} porque amaban más la alabanza de los hombres que la de
Dios. \footnote{\textbf{12:43} Juan 5,44}

\hypertarget{el-testimonio-de-jesuxfas-sobre-suxed-mismo-y-sobre-su-relaciuxf3n-con-dios}{%
\subsection{El testimonio de Jesús sobre sí mismo y sobre su relación
con
Dios}\label{el-testimonio-de-jesuxfas-sobre-suxed-mismo-y-sobre-su-relaciuxf3n-con-dios}}

\bibleverse{44} Jesús clamó y dijo: ``El que cree en mí, no cree en mí,
sino en el que me ha enviado. \bibleverse{45} El que me ve, ve al que me
ha enviado. \footnote{\textbf{12:45} Juan 14,9} \bibleverse{46} Yo he
venido al mundo como una luz, para que quien crea en mí no permanezca en
las tinieblas. \bibleverse{47} Si alguien escucha mis palabras y no
cree, yo no lo juzgo. Porque no he venido a juzgar al mundo, sino a
salvar al mundo. \footnote{\textbf{12:47} Juan 3,17; Luc 9,56}
\bibleverse{48} El que me rechaza y no recibe mis palabras, tiene quien
lo juzgue. La palabra que yo hablé lo juzgará en el último día.
\bibleverse{49} Porque no he hablado por mí mismo, sino que el Padre que
me ha enviado me ha dado un mandamiento sobre lo que debo decir y lo que
debo hablar. \bibleverse{50} Yosé que su mandamiento es la vida eterna.
Por lo tanto, las cosas que hablo, como el Padre me ha dicho, así las
hablo''.

\hypertarget{el-lavado-de-pies}{%
\subsection{El lavado de pies}\label{el-lavado-de-pies}}

\hypertarget{section-12}{%
\section{13}\label{section-12}}

\bibleverse{1} Antes de la fiesta de la Pascua, Jesús, sabiendo que
había llegado su hora de pasar de este mundo al Padre, habiendo amado a
los suyos que estaban en el mundo, los amó hasta el fin. \footnote{\textbf{13:1}
  Juan 7,30; Juan 17,1} \bibleverse{2} Durante la cena, habiendo metido
ya el diablo en el corazón de Judas Iscariote, hijo de Simón, para que
lo traicionara, \footnote{\textbf{13:2} Luc 22,3} \bibleverse{3} Jesús,
sabiendo que el Padre había entregado todas las cosas en sus manos, y
que venía de Dios y se iba a Dios, \footnote{\textbf{13:3} Juan 3,35;
  Juan 16,28} \bibleverse{4} se levantó de la cena y se despojó de sus
vestidos exteriores. Tomó una toalla y se la puso alrededor de la
cintura. \bibleverse{5} Luego echó agua en un lebrillo y se puso a lavar
los pies de los discípulos y a enjugarlos con la toalla que le envolvía.
\bibleverse{6} Luego se acercó a Simón Pedro. Le dijo: ``Señor, ¿me
lavas los pies?''.

\bibleverse{7} Jesús le contestó: ``No sabes lo que hago ahora, pero lo
entenderás después''.

\bibleverse{8} Pedro le dijo: ``¡Nunca me lavarás los pies!'' Jesús le
respondió: ``Si no te lavo, no tienes parte conmigo''.

\bibleverse{9} Simón Pedro le dijo: ``Señor, no sólo mis pies, sino
también mis manos y mi cabeza''.

\bibleverse{10} Jesús le dijo: ``Alguien que se ha bañado sólo necesita
que le laven los pies, pero está completamente limpio. Vosotros estáis
limpios, pero no todos''. \footnote{\textbf{13:10} Juan 15,3}
\bibleverse{11} Porque conocía al que lo iba a traicionar; por eso dijo:
``No estáis todos limpios''.

\hypertarget{la-interpretaciuxf3n-de-jesuxfas-de-su-humilde-servicio-de-amor}{%
\subsection{La interpretación de Jesús de su humilde servicio de
amor}\label{la-interpretaciuxf3n-de-jesuxfas-de-su-humilde-servicio-de-amor}}

\bibleverse{12} Así que, después de lavarles los pies, volver a ponerse
la ropa exterior y sentarse de nuevo, les dijo: ``¿Sabéis lo que os he
hecho? \bibleverse{13} Me llamáis ``Maestro'' y ``Señor''. Lo decís con
razón, porque así soy. \footnote{\textbf{13:13} Mat 23,8; Mat 23,10}
\bibleverse{14} Si yo, el Señor y el Maestro, os he lavado los pies,
también vosotros debéis lavaros los pies unos a otros. \footnote{\textbf{13:14}
  Luc 22,27} \bibleverse{15} Porque os he dado ejemplo, para que también
vosotros hagáis lo que yo he hecho con vosotros. \footnote{\textbf{13:15}
  Fil 2,5; 1Pe 2,21} \bibleverse{16} De cierto os digo que el siervo no
es mayor que su señor, ni el enviado es mayor que el que lo envió.
\footnote{\textbf{13:16} Mat 10,24} \bibleverse{17} Si sabéis estas
cosas, dichosos vosotros si las ponéis en práctica. \footnote{\textbf{13:17}
  Mat 7,24} \bibleverse{18} No hablo de todos vosotros. Yo sé a quién he
escogido; pero para que se cumpla la Escritura: `El que come pan
conmigo, ha levantado su talón contra mí'. \footnote{\textbf{13:18}
  Salmo 41:9} \bibleverse{19} Desde ahora os lo digo antes de que
ocurra, para que cuando ocurra, creáis que yo soy. \bibleverse{20} De
cierto os digo que el que recibe a quien yo envío, me recibe a mí; y el
que me recibe a mí, recibe al que me envió.'' \footnote{\textbf{13:20}
  Mat 10,40}

\hypertarget{identificaciuxf3n-y-remociuxf3n-del-traidor}{%
\subsection{Identificación y remoción del
traidor}\label{identificaciuxf3n-y-remociuxf3n-del-traidor}}

\bibleverse{21} Al decir esto, Jesús se turbó en su espíritu y declaró:
``Os aseguro que uno de vosotros me va a traicionar.'' \footnote{\textbf{13:21}
  Juan 12,27}

\bibleverse{22} Los discípulos se miraban unos a otros, perplejos sobre
quién hablaba. \bibleverse{23} Uno de sus discípulos, a quien Jesús
amaba, estaba en la mesa, apoyado en el pecho de Jesús. \footnote{\textbf{13:23}
  Juan 19,26; Juan 20,2; Juan 21,20} \bibleverse{24} Entonces Simón
Pedro le hizo señas y le dijo: ``Dinos de quién habla''.

\bibleverse{25} Él, recostado, como estaba, sobre el pecho de Jesús, le
preguntó: ``Señor, ¿quién es?''.

\bibleverse{26} Entonces Jesús respondió: ``Es a quien le daré este
pedazo de pan cuando lo haya mojado''. Y cuando hubo mojado el pedazo de
pan, se lo dio a Judas, hijo de Simón Iscariote. \bibleverse{27} Después
del trozo de pan, entró en él Satanás. Entonces Jesús le dijo: ``Lo que
hagas, hazlo rápido''.

\bibleverse{28} Nadie en la mesa sabía por qué le decía esto.
\bibleverse{29} Pues algunos pensaron, porque Judas tenía la bolsa, que
Jesús le había dicho: ``Compra lo que necesitamos para la fiesta'', o
que debía dar algo a los pobres. \bibleverse{30} Así que, habiendo
recibido aquel bocado, salió inmediatamente. Era de noche.

\hypertarget{el-anuncio-de-jesuxfas-de-su-glorificaciuxf3n}{%
\subsection{El anuncio de Jesús de su
glorificación}\label{el-anuncio-de-jesuxfas-de-su-glorificaciuxf3n}}

\bibleverse{31} Cuando salió, Jesús dijo: ``Ahora el Hijo del Hombre ha
sido glorificado, y Dios ha sido glorificado en él. \footnote{\textbf{13:31}
  Juan 12,23; Juan 12,28} \bibleverse{32} Si Dios ha sido glorificado en
él, Dios también lo glorificará en sí mismo, y lo glorificará
inmediatamente. \footnote{\textbf{13:32} Juan 17,1-5} \bibleverse{33}
Hijitos, estaré con vosotros un poco más de tiempo. Me buscaréis, y como
dije a los judíos: ``Donde yo voy, vosotros no podéis venir'', así os lo
digo ahora. \footnote{\textbf{13:33} Juan 9,21}

\hypertarget{el-nuevo-mandamiento-de-amar}{%
\subsection{El nuevo mandamiento de
amar}\label{el-nuevo-mandamiento-de-amar}}

\bibleverse{34} Un nuevo mandamiento os doy: que os améis unos a otros.
Como yo os he amado, amaos también vosotros unos a otros. \footnote{\textbf{13:34}
  Juan 15,12-13; Juan 15,17} \bibleverse{35} En esto reconocerán todos
que sois mis discípulos, si os amáis unos a otros.''

\hypertarget{anuncio-de-la-negaciuxf3n-de-pedro}{%
\subsection{Anuncio de la negación de
Pedro}\label{anuncio-de-la-negaciuxf3n-de-pedro}}

\bibleverse{36} Simón Pedro le dijo: ``Señor, ¿a dónde vas?''. Jesús
respondió: ``A donde voy, no puedes seguirme ahora, pero me seguirás
después''. \footnote{\textbf{13:36} Juan 21,18-19}

\bibleverse{37} Pedro le dijo: ``Señor, ¿por qué no puedo seguirte
ahora? Daré mi vida por ti''.

\bibleverse{38} Jesús le contestó: ``¿Vas a dar tu vida por mí? Te
aseguro que el gallo no cantará hasta que me hayas negado tres veces.

\hypertarget{jesuxfas-el-camino-a-dios-su-uniuxf3n-con-dios}{%
\subsection{Jesús el camino a Dios, su unión con
Dios}\label{jesuxfas-el-camino-a-dios-su-uniuxf3n-con-dios}}

\hypertarget{section-13}{%
\section{14}\label{section-13}}

\bibleverse{1} ``No dejes que tu corazón se turbe. Creéis en Dios. Creed
también en mí. \bibleverse{2} En la casa de mi Padre hay muchas casas.
Si no fuera así, os lo habría dicho. Voy a preparar un lugar para
vosotros. \footnote{\textbf{14:2} Mat 25,34} \bibleverse{3} Si me voy y
os preparo un lugar, volveré y os recibiré en mi casa; para que donde yo
esté, estéis también vosotros. \footnote{\textbf{14:3} Juan 12,26; Juan
  17,24} \bibleverse{4} Vosotros sabéis a dónde voy y conocéis el
camino''.

\bibleverse{5} Tomás le dijo: ``Señor, no sabemos a dónde vas. ¿Cómo
podemos saber el camino?''

\bibleverse{6} Jesús le dijo: ``Yo soy el camino, la verdad y la vida.
Nadie viene al Padre sino por mí. \footnote{\textbf{14:6} Heb 10,20; Mat
  11,27; Juan 10,9; Rom 5,1-2} \bibleverse{7} Si me hubieras conocido,
habrías conocido también a mi Padre. Desde ahora, lo conoces y lo has
visto''.

\bibleverse{8} Felipe le dijo: ``Señor, muéstranos al Padre, y eso nos
bastará''.

\bibleverse{9} Jesús le dijo: ``¿Tanto tiempo llevo con vosotros y no me
conoces, Felipe? El que me ha visto a mí ha visto al Padre. ¿Cómo dices:
``Muéstranos al Padre''? \footnote{\textbf{14:9} Juan 12,45; Heb 1,3}
\bibleverse{10} ¿No crees que yo estoy en el Padre y el Padre en mí? Las
palabras que os digo no las hablo por mí mismo, sino que el Padre que
vive en mí hace sus obras. \footnote{\textbf{14:10} Juan 12,49}
\bibleverse{11} Creedme que yo estoy en el Padre, y el Padre en mí; o
bien creedme por las mismas obras. \footnote{\textbf{14:11} Juan 10,25;
  Juan 10,38}

\hypertarget{promesa-del-espuxedritu-santo}{%
\subsection{Promesa del Espíritu
Santo}\label{promesa-del-espuxedritu-santo}}

\bibleverse{12} De cierto os digo que el que cree en mí, las obras que
yo hago, él también las hará; y hará obras mayores que éstas, porque yo
voy a mi Padre. \footnote{\textbf{14:12} Mat 28,19} \bibleverse{13} Todo
lo que pidáis en mi nombre, lo haré, para que el Padre sea glorificado
en el Hijo. \footnote{\textbf{14:13} Juan 15,7; Juan 16,24; Mar 11,24;
  1Jn 5,14; 1Jn 1,5-15} \bibleverse{14} Si pedís algo en mi nombre, yo
lo haré. \bibleverse{15} Si me amáis, guardad mis mandamientos.
\footnote{\textbf{14:15} Juan 15,10; 1Jn 5,3} \bibleverse{16} Yo rogaré
al Padre, y él os dará otro Consejero, para\footnote{\textbf{14:16}
  Griego \greek{παρακλητον}: Consejero, Ayudante, Intercesor, Abogado y
  Consolador.} que esté con vosotros para siempre: \footnote{\textbf{14:16}
  Juan 15,26; Juan 16,7} \bibleverse{17} el Espíritu de la verdad, al
que el mundo no puede recibir, porque no lo ve y no lo conoce. Vosotros
lo conocéis, porque vive con vosotros y estará en vosotros. \footnote{\textbf{14:17}
  Juan 16,13} \bibleverse{18} No os dejaré huérfanos. Vendré a vosotros.
\bibleverse{19} Todavía un poco, y el mundo no me verá más; pero
vosotros me veréis. Porque yo vivo, vosotros también viviréis.
\footnote{\textbf{14:19} Juan 20,20} \bibleverse{20} En aquel día
sabréis que yo estoy en mi Padre, y vosotros en mí, y yo en vosotros.

\hypertarget{promesa-de-la-muxe1s-uxedntima-comunidad-de-espuxedritu-y-amor-con-dios-y-jesuxfas}{%
\subsection{Promesa de la más íntima comunidad de espíritu y amor con
Dios y
Jesús}\label{promesa-de-la-muxe1s-uxedntima-comunidad-de-espuxedritu-y-amor-con-dios-y-jesuxfas}}

\bibleverse{21} El que tiene mis mandamientos y los cumple, ése es el
que me ama. El que me ama será amado por mi Padre, y yo le amaré y me
revelaré a él''. \footnote{\textbf{14:21} Juan 16,27; 1Jn 5,3}

\bibleverse{22} Judas (no Iscariote) le dijo: ``Señor, ¿qué ha pasado
para que te reveles a nosotros y no al mundo?'' \footnote{\textbf{14:22}
  Hech 10,40-41}

\bibleverse{23} Jesús le respondió: ``Si un hombre me ama, cumplirá mi
palabra. Mi Padre lo amará, y vendremos a él y haremos nuestra casa con
él. \footnote{\textbf{14:23} Prov 8,17; Efes 3,17} \bibleverse{24} El
que no me ama no guarda mis palabras. La palabra que oís no es mía, sino
del Padre que me ha enviado. \footnote{\textbf{14:24} Juan 7,16-17}

\hypertarget{promesa-de-enseuxf1ar-del-espuxedritu-santo}{%
\subsection{Promesa de enseñar del Espíritu
Santo}\label{promesa-de-enseuxf1ar-del-espuxedritu-santo}}

\bibleverse{25} ``Os he dicho estas cosas mientras vivía con vosotros.
\bibleverse{26} Pero el Consejero, el Espíritu Santo, que el Padre
enviará en mi nombre, os enseñará todas las cosas y os recordará todo lo
que os he dicho. \bibleverse{27} La paz os dejo. Mi paz os doy; no os la
doy como la da el mundo. No dejes que tu corazón se turbe, ni que tenga
miedo. \footnote{\textbf{14:27} Juan 16,33; Fil 4,7} \bibleverse{28}
Habéis oído que os dije: ``Me voy y volveré a vosotros''. Si me amarais,
os habríais alegrado porque dije: ``Me voy a mi Padre'', porque el Padre
es más grande que yo. \bibleverse{29} Ahora os lo he dicho antes de que
ocurra, para que, cuando ocurra, creáis. \bibleverse{30} Ya no hablaré
mucho con vosotros, porque viene el príncipe de este mundo y no tiene
nada en mí. \footnote{\textbf{14:30} Juan 12,31; Efes 2,2}
\bibleverse{31} Pero para que el mundo sepa que amo al Padre, y que como
el Padre me mandó, así hago yo. Levantaos, vámonos de aquí. \footnote{\textbf{14:31}
  Juan 10,18}

\hypertarget{paruxe1bola-de-la-vid-y-las-ramas}{%
\subsection{Parábola de la vid y las
ramas}\label{paruxe1bola-de-la-vid-y-las-ramas}}

\hypertarget{section-14}{%
\section{15}\label{section-14}}

\bibleverse{1} ``Yo soy la verdadera vid, y mi Padre es el viticultor.
\bibleverse{2} Todo sarmiento que en mí no da fruto, lo quita. Todo
sarmiento que da fruto, lo poda para que dé más fruto. \bibleverse{3}
Vosotros ya estáis limpios por la palabra que os he dicho. \footnote{\textbf{15:3}
  Juan 13,10; 1Pe 1,23} \bibleverse{4} Permaneced en mí, y yo en
vosotros. Como el sarmiento no puede dar fruto por sí mismo si no
permanece en la vid, así tampoco vosotros, si no permanecéis en mí.
\bibleverse{5} Yo soy la vid. Vosotros sois los sarmientos. El que
permanece en mí y yo en él da mucho fruto, porque sin mí no podéis hacer
nada. \footnote{\textbf{15:5} 2Cor 3,5-6} \bibleverse{6} El que no
permanece en mí, es arrojado como pámpano y se seca; los recogen, los
echan al fuego y se queman. \bibleverse{7} Si permanecéis en mí, y mis
palabras permanecen en vosotros, pediréis todo lo que queráis, y se os
hará. \footnote{\textbf{15:7} Mar 11,24}

\bibleverse{8} ``En esto es glorificado mi Padre, en que deis mucho
fruto; y así seréis mis discípulos. \footnote{\textbf{15:8} Mat 5,16}

\hypertarget{el-mandamiento-del-amor-permanezcan-en-la-comunidad-de-amor-conmigo-y-entre-nosotros}{%
\subsection{El mandamiento del amor: ¡Permanezcan en la comunidad de
amor conmigo y entre
nosotros!}\label{el-mandamiento-del-amor-permanezcan-en-la-comunidad-de-amor-conmigo-y-entre-nosotros}}

\bibleverse{9} Como el Padre me ha amado, yo también os he amado.
Permaneced en mi amor. \bibleverse{10} Si guardáis mis mandamientos,
permaneceréis en mi amor, como yo he guardado los mandamientos de mi
Padre y permanezco en su amor. \bibleverse{11} Os he dicho estas cosas
para que mi alegría permanezca en vosotros y vuestra alegría sea
cumplida. \footnote{\textbf{15:11} Juan 17,13}

\bibleverse{12} ``Este es mi mandamiento: que os améis unos a otros,
como yo os he amado. \footnote{\textbf{15:12} Juan 13,34}
\bibleverse{13} Nadie tiene mayor amor que el que da la vida por sus
amigos. \footnote{\textbf{15:13} Juan 10,12; 1Jn 3,16} \bibleverse{14}
Vosotros sois mis amigos si hacéis lo que yo os mando. \footnote{\textbf{15:14}
  Juan 8,31; Mat 12,50} \bibleverse{15} Ya no os llamo siervos, porque
el siervo no sabe lo que hace su señor. Pero os he llamado amigos,
porque todo lo que he oído a mi Padre os lo he dado a conocer.
\bibleverse{16} No me habéis elegido a mí, sino que yo os he elegido a
vosotros y os he designado para que vayáis y deis fruto, y vuestro fruto
permanezca; para que todo lo que pidáis al Padre en mi nombre os lo dé.

\bibleverse{17} ``Os mando estas cosas, para que os améis unos a otros.

\hypertarget{profecuxeda-del-destino-de-los-discuxedpulos-sufriendo-el-odio-del-mundo}{%
\subsection{Profecía del destino de los discípulos, sufriendo el odio
del
mundo}\label{profecuxeda-del-destino-de-los-discuxedpulos-sufriendo-el-odio-del-mundo}}

\bibleverse{18} Si el mundo os odia, sabed que me ha odiado a mí antes
que a vosotros. \footnote{\textbf{15:18} Juan 7,7} \bibleverse{19} Si
fuerais del mundo, el mundo amaría a los suyos. Pero como no sois del
mundo, puesto que yo os elegí del mundo, por eso el mundo os odia.
\footnote{\textbf{15:19} 1Jn 4,4; 1Jn 1,4-5; Juan 17,14} \bibleverse{20}
Recordad la palabra que os dije: `Un siervo no es mayor que su
señor'.\footnote{\textbf{15:20} Juan 13:16} Si me persiguieron a mí,
también os perseguirán a vosotros. Si ellos cumplieron mi palabra,
también cumplirán la vuestra. \footnote{\textbf{15:20} Juan 13,16; Mat
  10,24-25} \bibleverse{21} Pero todo esto os lo harán por mi nombre,
porque no conocen al que me ha enviado. \footnote{\textbf{15:21} Juan
  16,3} \bibleverse{22} Si yo no hubiera venido a hablarles, no tendrían
pecado; pero ahora no tienen excusa para su pecado. \footnote{\textbf{15:22}
  Juan 9,41} \bibleverse{23} El que me odia, odia también a mi Padre.
\footnote{\textbf{15:23} Luc 10,16} \bibleverse{24} Si yo no hubiera
hecho entre ellos las obras que nadie hizo, no tendrían pecado. Pero
ahora han visto y también me han odiado a mí y a mi Padre.
\bibleverse{25} Pero esto ha sucedido para que se cumpla la palabra que
estaba escrita en su ley: ``Me odiaron sin causa''. \footnote{\textbf{15:25}
  Salmo 35:19; 69:4}

\bibleverse{26} ``Cuando venga el Consejero\footnote{\textbf{15:26}
  Parakletos griego: Consejero, Ayudante, Abogado, Intercesor y
  Consolador.} que os enviaré de parte del Padre, el Espíritu de la
verdad, que procede del Padre, él dará testimonio de mí. \footnote{\textbf{15:26}
  Juan 14,16; Juan 14,26; Luc 24,49} \bibleverse{27} También
vosotrosdaréis testimonio, porque habéis estado conmigo desde el
principio. \footnote{\textbf{15:27} Hech 1,8; Hech 1,21-22; Hech 5,32}

\hypertarget{section-15}{%
\section{16}\label{section-15}}

\bibleverse{1} ``Os he dicho estas cosas para no haceros tropezar.
\bibleverse{2} Os expulsarán de las sinagogas. Sí, viene el tiempo en
que quien os mate pensará que ofrece un servicio a Dios. \footnote{\textbf{16:2}
  Mat 10,17; Mat 10,22; Mat 24,9} \bibleverse{3} Ellos harán estas cosas
\footnote{\textbf{16:3} TR añade ``a ti''} porque no han conocido al
Padre ni a mí. \footnote{\textbf{16:3} Juan 15,21} \bibleverse{4} Pero
os he dicho estas cosas para que, cuando llegue el momento, os acordéis
de que os las he contado. No os dije estas cosas desde el principio,
porque estaba con vosotros.

\hypertarget{promesa-del-espuxedritu-santo-y-su-obra-benuxe9fica-en-el-mundo-y-en-los-discuxedpulos}{%
\subsection{Promesa del Espíritu Santo y su obra benéfica en el mundo y
en los
discípulos}\label{promesa-del-espuxedritu-santo-y-su-obra-benuxe9fica-en-el-mundo-y-en-los-discuxedpulos}}

\bibleverse{5} Pero ahora me voy con el que me ha enviado, y ninguno de
vosotros me pregunta: ``¿Adónde vas?'' \bibleverse{6} Pero como os he
dicho estas cosas, la tristeza ha llenado vuestro corazón.
\bibleverse{7} Sin embargo, os digo la verdad: os conviene que me vaya,
porque si no me voy, el Consejero no vendrá a vosotros. Pero si me voy,
os lo enviaré. \footnote{\textbf{16:7} Juan 14,16; Juan 14,26}
\bibleverse{8} Cuando venga, convencerá al mundo de pecado, de justicia
y de juicio; \bibleverse{9} de pecado, porque no creen en mí;
\footnote{\textbf{16:9} Juan 15,22; Juan 15,24} \bibleverse{10} de
justicia, porque me voy a mi Padre y ya no me veréis; \footnote{\textbf{16:10}
  Hech 5,31; Rom 4,25} \bibleverse{11} de juicio, porque el príncipe de
este mundo ha sido juzgado. \footnote{\textbf{16:11} Juan 12,31}

\bibleverse{12} ``Todavía tengo muchas cosas que deciros, pero ahora no
podéis soportarlas. \footnote{\textbf{16:12} 1Cor 3,1} \bibleverse{13}
Sin embargo, cuando él, el Espíritu de la verdad, haya venido, os guiará
a toda la verdad, porque no hablará por su cuenta, sino que hablará todo
lo que oiga. Él os anunciará las cosas que se avecinan. \footnote{\textbf{16:13}
  Juan 14,26; 1Jn 2,27} \bibleverse{14} Él me glorificará, porque tomará
de lo mío y os lo declarará. \bibleverse{15} Todo lo que tiene el Padre
es mío; por eso he dicho que toma\footnote{\textbf{16:15} TR dice
  ``tomará'' en lugar de ``toma''} de lo mío y os lo anunciará.
\footnote{\textbf{16:15} Juan 3,35; Juan 17,10}

\hypertarget{promesa-de-una-reuniuxf3n-temprana-y-amonestaciuxf3n-de-orar-en-el-nombre-de-jesuxfas}{%
\subsection{Promesa de una reunión temprana y amonestación de orar en el
nombre de
Jesús}\label{promesa-de-una-reuniuxf3n-temprana-y-amonestaciuxf3n-de-orar-en-el-nombre-de-jesuxfas}}

\bibleverse{16} ``Un poco de tiempo, y no me verás. De nuevo un poco de
tiempo, y me verás''. \footnote{\textbf{16:16} Juan 14,19}

\bibleverse{17} Entonces algunos de sus discípulos se dijeron unos a
otros: ``¿Qué es eso que nos dice: ``Un poco de tiempo y no me veréis, y
de nuevo un poco de tiempo y me veréis'', y ``porque voy al Padre''?''
\bibleverse{18} Dijeron entonces: ``¿Qué es eso que dice: `Un poco de
tiempo'? No sabemos lo que dice''.

\bibleverse{19} Por lo tanto, Jesús se dio cuenta de que querían
preguntarle, y les dijo: ``¿Preguntáis entre vosotros acerca de esto que
he dicho: ``Un poco de tiempo y no me veréis, y de nuevo un poco de
tiempo y me veréis''? \bibleverse{20} Ciertamente os digo que lloraréis
y os lamentaréis, pero el mundo se alegrará. Estaréis tristes, pero
vuestra tristeza se convertirá en alegría. \footnote{\textbf{16:20} Mar
  16,10} \bibleverse{21} La mujer, cuando da a luz, se entristece porque
ha llegado su hora. Pero cuando ha dado a luz al niño, ya no se acuerda
de la angustia, por la alegría de que haya nacido un ser humano en el
mundo. \footnote{\textbf{16:21} Is 26,17} \bibleverse{22} Por eso ahora
tienes angustia, pero volveré a verte, y tu corazón se alegrará, y nadie
te quitará la alegría.

\bibleverse{23} ``En aquel día no me preguntaréis nada. Os aseguro que
todo lo que pidáis al Padre en mi nombre, os lo dará. \footnote{\textbf{16:23}
  Juan 14,13-14} \bibleverse{24} Hasta ahora no habéis pedido nada en mi
nombre. Pedid y recibiréis, para que vuestra alegría sea completa.
\footnote{\textbf{16:24} Juan 15,11}

\hypertarget{promesa-de-completar-la-comuniuxf3n-con-dios-para-los-discuxedpulos-conclusiuxf3n-de-los-discursos-de-despedida}{%
\subsection{Promesa de completar la comunión con Dios para los
discípulos; Conclusión de los discursos de
despedida}\label{promesa-de-completar-la-comuniuxf3n-con-dios-para-los-discuxedpulos-conclusiuxf3n-de-los-discursos-de-despedida}}

\bibleverse{25} ``Os he hablado de estas cosas en parábolas. Pero viene
el tiempo en que ya no os hablaré por parábolas, sino que os hablaré
claramente del Padre. \bibleverse{26} En aquel día pediréis en mi
nombre; y no os digo que yo rogaré al Padre por vosotros,
\bibleverse{27} pues el Padre mismo os ama, porque vosotros me habéis
amado y habéis creído que he venido de Dios. \footnote{\textbf{16:27}
  Juan 14,21} \bibleverse{28} Yo he salido del Padre y he venido al
mundo. De nuevo, dejo el mundo y voy al Padre''.

\bibleverse{29} Sus discípulos le dijeron: ``He aquí que ahora hablas
con claridad y no usas parábolas. \bibleverse{30} Ahora sabemos que lo
sabes todo y que no necesitas que nadie te cuestione. Por eso creemos
que has venido de Dios''.

\bibleverse{31} Jesús les respondió: ``¿Ahora creéis? \bibleverse{32} He
aquí que viene el tiempo, y ya ha llegado, en que seréis dispersados,
cada uno a su lugar, y me dejaréis solo. Pero no estoy solo, porque el
Padre está conmigo. \footnote{\textbf{16:32} Zac 13,7; Mat 26,31}
\bibleverse{33} Os he dicho estas cosas para que en mí tengáis paz. En
el mundo tenéis problemas; pero ¡anímense! Yo he vencido al mundo''.
\footnote{\textbf{16:33} Juan 14,27; Rom 5,1; 1Jn 5,4}

\hypertarget{oraciuxf3n-de-despedida-de-jesuxfas-con-los-suyos-y-para-los-suyos}{%
\subsection{Oración de despedida de Jesús con los suyos y para los
suyos}\label{oraciuxf3n-de-despedida-de-jesuxfas-con-los-suyos-y-para-los-suyos}}

\hypertarget{section-16}{%
\section{17}\label{section-16}}

\bibleverse{1} Jesús dijo estas cosas y, levantando los ojos al cielo,
dijo: ``Padre, ha llegado el momento. Glorifica a tu Hijo, para que tu
Hijo también te glorifique a ti; \bibleverse{2} así como le diste
autoridad sobre toda carne, así dará vida eterna a todos los que le has
dado. \footnote{\textbf{17:2} Mat 11,27} \bibleverse{3} Esta es la vida
eterna: que te conozcan a ti, el único Dios verdadero, y al que has
enviado, Jesucristo. \footnote{\textbf{17:3} 1Jn 5,20} \bibleverse{4} Yo
te he glorificado en la tierra. He cumplido la obra que me has
encomendado. \bibleverse{5} Ahora, Padre, glorifícame tú mismo con la
gloria que tenía contigo antes de que el mundo existiera. \footnote{\textbf{17:5}
  Juan 1,1; Fil 2,6}

\hypertarget{la-intercesiuxf3n-de-jesuxfas-por-el-mantenimiento-de-los-discuxedpulos-en-el-conocimiento-correcto-de-dios}{%
\subsection{La intercesión de Jesús por el mantenimiento de los
discípulos en el conocimiento correcto de
Dios}\label{la-intercesiuxf3n-de-jesuxfas-por-el-mantenimiento-de-los-discuxedpulos-en-el-conocimiento-correcto-de-dios}}

\bibleverse{6} ``He revelado tu nombre al pueblo que me has dado fuera
del mundo. Eran tuyos y me los has dado. Ellos han cumplido tu palabra.
\bibleverse{7} Ahora han sabido que todas las cosas que me has dado
vienen de ti, \bibleverse{8} porque las palabras que me has dado se las
he dado a ellos; y las han recibido, y han sabido con certeza que vengo
de ti. Han creído que tú me has enviado. \footnote{\textbf{17:8} Juan
  16,30} \bibleverse{9} Yo rezo por ellos. No ruego por el mundo, sino
por los que me has dado, porque son tuyos. \footnote{\textbf{17:9} Juan
  6,37; Juan 6,44} \bibleverse{10} Todas las cosas que son mías son
tuyas, y las tuyas son mías, y yo soy glorificado en ellas. \footnote{\textbf{17:10}
  Juan 16,15} \bibleverse{11} Yo ya no estoy en el mundo, pero éstos
están en el mundo, y yo voy a ti. Padre santo, guárdalos por tu nombre
que me has dado, para que sean uno, como nosotros. \bibleverse{12}
Mientras estuve con ellos en el mundo, los guardé en tu nombre. He
guardado a los que me has dado. Ninguno de ellos se ha perdido, sino el
hijo de la destrucción, para que se cumpla la Escritura. \footnote{\textbf{17:12}
  Juan 6,39; Sal 41,9} \bibleverse{13} Pero ahora vengo a ti, y digo
estas cosas en el mundo, para que tengan mi gozo pleno en ellos.
\footnote{\textbf{17:13} Juan 15,11} \bibleverse{14} Les he dado tu
palabra. El mundo los ha odiado porque no son del mundo, así como yo no
soy del mundo. \footnote{\textbf{17:14} Juan 15,19} \bibleverse{15} No
ruego que los quites del mundo, sino que los guardes del maligno.
\footnote{\textbf{17:15} Mat 6,13; 2Tes 3,3} \bibleverse{16} No son del
mundo, como tampoco yo soy del mundo. \bibleverse{17} Santifícalos en tu
verdad. Tu palabra es la verdad. \footnote{\textbf{17:17} 17:17 Salmo
  119:142} \footnote{\textbf{17:17} Sal 119,160} \bibleverse{18} Como me
enviaste al mundo, así los he enviado yo al mundo. \footnote{\textbf{17:18}
  Juan 20,21} \bibleverse{19} Por ellos me santifico, para que ellos
también sean santificados en la verdad. \footnote{\textbf{17:19} Heb
  10,10}

\hypertarget{intercesiuxf3n-por-todos-los-creyentes}{%
\subsection{Intercesión por todos los
creyentes}\label{intercesiuxf3n-por-todos-los-creyentes}}

\bibleverse{20} ``No ruego sólo por éstos, sino también por los que
crean en mí por medio de su palabra, \footnote{\textbf{17:20} Rom 10,17}
\bibleverse{21} para que todos sean uno, como tú, Padre, estás en mí, y
yo en ti, para que también ellos sean uno en nosotros; para que el mundo
crea que tú me has enviado. \footnote{\textbf{17:21} Gal 3,28}
\bibleverse{22} La gloria que me has dado, yo se la he dado a ellos,
para que sean uno, como nosotros somos uno, \footnote{\textbf{17:22}
  Hech 4,32} \bibleverse{23} yo en ellos y tú en mí, para que se
perfeccionen en uno, para que el mundo sepa que tú me has enviado y que
los has amado, como a mí. \footnote{\textbf{17:23} 1Cor 6,17}
\bibleverse{24} Padre, quiero que también los que me has dado estén
conmigo donde yo estoy, para que vean mi gloria que me has dado, porque
me has amado antes de la fundación del mundo. \footnote{\textbf{17:24}
  Juan 12,26} \bibleverse{25} Padre justo, el mundo no te ha conocido,
pero yo te he conocido, y éstos han sabido que tú me has enviado.
\bibleverse{26} Yoles he dado a conocer tu nombre, y lo daré a conocer,
para que el amor con que me has amado esté en ellos, y yo en ellos.''

\hypertarget{jesuxfas-en-getsemanuxed-judas-malco-arresto-de-jesuxfas}{%
\subsection{Jesús en Getsemaní: Judas, Malco, arresto de
Jesús}\label{jesuxfas-en-getsemanuxed-judas-malco-arresto-de-jesuxfas}}

\hypertarget{section-17}{%
\section{18}\label{section-17}}

\bibleverse{1} Cuando Jesús hubo dicho estas palabras, salió con sus
discípulos por el torrente Cedrón, donde había un huerto en el que
entraron él y sus discípulos. \bibleverse{2} También Judas, el que lo
traicionó, conocía el lugar, porque Jesús se reunía allí a menudo con
sus discípulos. \footnote{\textbf{18:2} Luc 21,37} \bibleverse{3}
Entonces Judas, habiendo tomado un destacamento de soldados y oficiales
de los sumos sacerdotes y de los fariseos, llegó allí con linternas,
antorchas y armas. \bibleverse{4} Jesús, pues, sabiendo todo lo que le
pasaba, salió y les dijo: ``¿A quién buscáis?''

\bibleverse{5} Le respondieron: ``Jesús de Nazaret''. Jesús les dijo:
``Yo soy''. También Judas, el que le traicionó, estaba con ellos.
\bibleverse{6} Por eso, cuando les dijo: ``Yo soy'', retrocedieron y
cayeron al suelo.

\bibleverse{7} Por eso les preguntó de nuevo: ``¿A quién buscáis?''.
Dijeron: ``Jesús de Nazaret''.

\bibleverse{8} Jesús respondió: ``Os he dicho que yo soy. Si, pues, me
buscáis, dejad que éstos se vayan'', \bibleverse{9} para que se cumpla
la palabra que dijo: ``De los que me has dado, no he perdido a
ninguno''. \footnote{\textbf{18:9} Juan 6:39} \footnote{\textbf{18:9}
  Juan 17,12}

\bibleverse{10} Entonces Simón Pedro, teniendo una espada, la sacó,
hirió al siervo del sumo sacerdote y le cortó la oreja derecha. El
siervo se llamaba Malco. \bibleverse{11} Entonces Jesús dijo a Pedro:
``Mete la espada en la vaina. El cáliz que el Padre me ha dado, ¿no lo
voy a beber?''

\bibleverse{12} Entonces el destacamento, el comandante y los oficiales
de los judíos prendieron a Jesús y lo ataron, \bibleverse{13} y lo
llevaron primero a Anás, porque era suegro de Caifás, que era sumo
sacerdote aquel año. \bibleverse{14} Fue Caifás quien aconsejó a los
judíos que era conveniente que un hombre pereciera por el pueblo.
\footnote{\textbf{18:14} Juan 11,49-50; Luc 3,1-2}

\hypertarget{primera-negaciuxf3n-de-pedro}{%
\subsection{Primera negación de
Pedro}\label{primera-negaciuxf3n-de-pedro}}

\bibleverse{15} Simón Pedro siguió a Jesús, al igual que otro discípulo.
Aquel discípulo era conocido del sumo sacerdote, y entró con Jesús en el
atrio del sumo sacerdote; \bibleverse{16} pero Pedro estaba fuera, a la
puerta. Entonces el otro discípulo, que era conocido del sumo sacerdote,
salió y habló a la que guardaba la puerta, e hizo entrar a Pedro.
\bibleverse{17} Entonces la criada que guardaba la puerta dijo a Pedro:
``¿Eres tú también uno de los discípulos de este hombre?'' Él dijo: ``No
lo soy''.

\bibleverse{18} Los sirvientes y los oficiales estaban allí de pie,
habiendo hecho un fuego de brasas, pues hacía frío. Se estaban
calentando. Pedro estaba con ellos, de pie y calentándose.

\hypertarget{jesuxfas-ante-los-sumos-sacerdotes-anuxe1s-y-caifuxe1s}{%
\subsection{Jesús ante los sumos sacerdotes Anás y
Caifás}\label{jesuxfas-ante-los-sumos-sacerdotes-anuxe1s-y-caifuxe1s}}

\bibleverse{19} El sumo sacerdote preguntó entonces a Jesús por sus
discípulos y por su enseñanza.

\bibleverse{20} Jesús le contestó: ``Yo hablé abiertamente al mundo.
Siempre enseñé en las sinagogas y en el templo, donde siempre se reúnen
los judíos. No dije nada en secreto. \footnote{\textbf{18:20} Juan 7,14;
  Juan 7,26} \bibleverse{21} ¿Por qué me preguntas? Preguntad a los que
me han oído lo que les he dicho. He aquí que ellos saben las cosas que
dije''.

\bibleverse{22} Cuando hubo dicho esto, uno de los oficiales que estaban
allí abofeteó a Jesús con la mano, diciendo: ``¿Así respondes al sumo
sacerdote?''

\bibleverse{23} Jesús le respondió: ``Si he hablado mal, testifica el
mal; pero si está bien, ¿por qué me golpeas?''

\bibleverse{24} Anás lo envió atado a Caifás, el sumo sacerdote.

\hypertarget{segunda-y-tercera-negaciuxf3n-de-pedro}{%
\subsection{Segunda y tercera negación de
Pedro}\label{segunda-y-tercera-negaciuxf3n-de-pedro}}

\bibleverse{25} Simón Pedro estaba de pie, calentándose. Entonces le
dijeron: ``¿No eres tú también uno de sus discípulos, verdad?'' Él lo
negó y dijo: ``No lo soy''.

\bibleverse{26} Uno de los siervos del sumo sacerdote, que era pariente
del que Pedro había cortado la oreja, le dijo: ``¿No te vi en el jardín
con él?''

\bibleverse{27} Pedro, pues, lo negó de nuevo, e inmediatamente el gallo
cantó.

\hypertarget{el-interrogatorio-y-la-confesiuxf3n-de-jesuxfas-ante-el-gobernador-romano-pilato-su-flagelaciuxf3n-burla-y-condena}{%
\subsection{El interrogatorio y la confesión de Jesús ante el gobernador
romano Pilato; su flagelación, burla y
condena}\label{el-interrogatorio-y-la-confesiuxf3n-de-jesuxfas-ante-el-gobernador-romano-pilato-su-flagelaciuxf3n-burla-y-condena}}

\bibleverse{28} Condujeron, pues, a Jesús desde Caifás al pretorio. Era
temprano, y ellos mismos no entraron en el pretorio para no
contaminarse, sino para comer la Pascua. \bibleverse{29} Salió, pues,
Pilato hacia ellos y les dijo: ``¿Qué acusación traéis contra este
hombre?''

\bibleverse{30} Le respondieron: ``Si este hombre no fuera un malhechor,
no te lo habríamos entregado''.

\bibleverse{31} Pilato, pues, les dijo: ``Tomadlo vosotros y juzgadlo
según vuestra ley''. Por eso los judíos le decían: ``Nos es ilícito dar
muerte a nadie'', \footnote{\textbf{18:31} Juan 19,6-7} \bibleverse{32}
para que se cumpliera la palabra de Jesús que había dicho, dando a
entender con qué clase de muerte debía morir. \footnote{\textbf{18:32}
  Juan 12,32-33; Mat 20,19}

\bibleverse{33} Entonces Pilato entró de nuevo en el pretorio, llamó a
Jesús y le dijo: ``¿Eres tú el Rey de los judíos?''

\bibleverse{34} Jesús le respondió: ``¿Dices esto por ti mismo, o te lo
han dicho otros?''

\bibleverse{35} Pilato respondió: ``No soy judío, ¿verdad? Tu propia
nación y los jefes de los sacerdotes te entregaron a mí. ¿Qué has
hecho?''

\bibleverse{36} Jesús respondió: ``Mi Reino no es de este mundo. Si mi
Reino fuera de este mundo, mis siervos lucharían para que yo no fuera
entregado a los judíos. Pero ahora mi Reino no es de aquí''.

\bibleverse{37} Pilato, pues, le dijo: ``¿Eres entonces un rey?'' Jesús
respondió: ``Vosotros decís que soy un rey. Para eso he nacido y para
eso he venido al mundo, para dar testimonio de la verdad. Todo el que es
de la verdad escucha mi voz''. \footnote{\textbf{18:37} 1Tim 6,13}

\bibleverse{38} Pilato le dijo: ``¿Qué es la verdad?'' Cuando hubo dicho
esto, salió de nuevo a los judíos y les dijo: ``No encuentro fundamento
para una acusación contra él. \bibleverse{39} Pero ustedes tienen la
costumbre de que les suelte a alguien en la Pascua. Por tanto, ¿queréis
que os suelte al Rey de los judíos?''

\bibleverse{40} Entonces todos volvieron a gritar, diciendo: ``Este no,
sino Barrabás''. Ahora bien, Barrabás era un ladrón.

\hypertarget{section-18}{%
\section{19}\label{section-18}}

\bibleverse{1} Entonces Pilato tomó a Jesús y lo azotó. \bibleverse{2}
Los soldados trenzaron espinas en la forma de una corona y se la
pusieron en la cabeza, y lo vistieron con un manto de púrpura.
\bibleverse{3} No dejaban de decir: ``¡Salve, Rey de los Judíos!'' y no
dejaban de abofetearle.

\bibleverse{4} Entonces Pilato volvió a salir y les dijo: ``He aquí que
os lo traigo, para que sepáis que no encuentro fundamento para una
acusación contra él.''

\bibleverse{5} Salió, pues, Jesús con la corona de espinas y el manto de
púrpura. Pilato les dijo: ``He aquí el hombre''.

\bibleverse{6} Al verlo, los jefes de los sacerdotes y los oficiales
gritaron diciendo: ``¡Crucifícalo! Crucifícalo!'' Pilato les dijo:
``Tomadlo vosotros y crucificadlo, porque no encuentro fundamento para
una acusación contra él''.

\bibleverse{7} Los judíos le respondieron: ``Nosotros tenemos una ley, y
según nuestra ley debe morir, porque se hizo Hijo de Dios''. \footnote{\textbf{19:7}
  Juan 10,33; Lev 24,16}

\bibleverse{8} Cuando Pilato oyó estas palabras, tuvo más miedo.
\bibleverse{9} Entró de nuevo en el pretorio y dijo a Jesús: ``¿De dónde
eres?''. Pero Jesús no le respondió. \bibleverse{10} Entonces Pilato le
dijo: ``¿No me hablas a mí? ¿No sabes que tengo poder para liberarte y
tengo poder para crucificarte?''

\bibleverse{11} Jesús respondió: ``No tendrías ningún poder contra mí,
si no te fuera dado de arriba. Por tanto, el que me ha entregado a
vosotros tiene un pecado mayor''.

\bibleverse{12} Ante esto, Pilato quiso ponerlo en libertad, pero los
judíos gritaron diciendo: ``¡Si sueltas a este hombre, no eres amigo del
César! Todo el que se hace rey habla contra el César''. \footnote{\textbf{19:12}
  Hech 17,7}

\bibleverse{13} Cuando Pilato oyó estas palabras, sacó a Jesús y se
sentó en el tribunal en un lugar llamado ``El Pavimento'', pero en
hebreo, ``Gabbatha.'' \bibleverse{14} Era el día de la preparación de la
Pascua, hacia la hora sexta.\footnote{\textbf{19:14} ``la hora sexta''
  habría sido las 06:00 h. según el sistema horario romano, o el
  mediodía para el sistema horario judío en uso, entonces.} Dijo a los
judíos: ``¡He aquí vuestro Rey!''

\bibleverse{15} Gritaron: ``¡Fuera de aquí! ¡Fuera de aquí!
Crucifíquenlo''. Pilato les dijo: ``¿Debo crucificar a vuestro Rey?''
Los jefes de los sacerdotes respondieron: ``No tenemos más rey que el
César''. \footnote{\textbf{19:15} Juan 18,37}

\hypertarget{la-crucifixiuxf3n-y-muerte-de-jesuxfas}{%
\subsection{La crucifixión y muerte de
Jesús}\label{la-crucifixiuxf3n-y-muerte-de-jesuxfas}}

\bibleverse{16} Entonces se lo entregó para que lo crucificaran.
Tomaron, pues, a Jesús y se lo llevaron. \bibleverse{17} Salió, llevando
su cruz, al lugar llamado ``Lugar de la Calavera'', que en hebreo se
llama ``Gólgota'', \bibleverse{18} donde lo crucificaron, y con él a
otros dos, uno a cada lado, y Jesús en medio. \bibleverse{19} Pilato
escribió también un título y lo puso en la cruz. Allí estaba escrito:
``JESÚS DE NAZARET, EL REY DE LOS JUDÍOS''. \bibleverse{20} Por lo
tanto, muchos de los judíos leyeron este título, porque el lugar donde
Jesús fue crucificado estaba cerca de la ciudad; y estaba escrito en
hebreo, en latín y en griego. \bibleverse{21} Los jefes de los judíos
dijeron, pues, a Pilato: ``No escribas: ``El Rey de los judíos'', sino:
``Dijo: ``Yo soy el Rey de los judíos''\,''.

\bibleverse{22} Pilato respondió: ``Lo que he escrito, lo he escrito''.

\bibleverse{23} Entonces los soldados, después de crucificar a Jesús,
tomaron sus vestidos e hicieron cuatro partes, a cada soldado una parte;
y también la túnica. La túnica era sin costura, tejida de arriba abajo.
\bibleverse{24} Entonces se dijeron unos a otros: ``No la rasguemos,
sino echemos suertes para decidir de quién será'', para que se cumpla la
Escritura que dice``Se repartieron mis ropas entre ellos. Echan a
suertes mi ropa\footnote{\textbf{19:24} Salmo 22:18} ''. Por eso los
soldados hicieron estas cosas.

\bibleverse{25} Pero junto a la cruz de Jesús estaban su madre, la
hermana de su madre, María la mujer de Cleofás y María Magdalena.
\bibleverse{26} Por eso, al ver Jesús a su madre y al discípulo al que
amaba que estaban allí, dijo a su madre: ``Mujer, ahí tienes a tu
hijo''. \footnote{\textbf{19:26} Juan 13,23} \bibleverse{27} Luego dijo
al discípulo: ``¡He ahí a tu madre! A partir de esa hora, el discípulo
se la llevó a su casa.

\bibleverse{28} Después de esto, Jesús, viendo\footnote{\textbf{19:28}
  NU, TR lee ``saber'' en lugar de ``ver''} que todo estaba ya
terminado, para que se cumpliera la Escritura, dijo: ``¡Tengo sed!''
\footnote{\textbf{19:28} Sal 22,15} \bibleverse{29} Se puso allí una
vasija llena de vinagre; entonces pusieron una esponja llena de vinagre
sobre un hisopo, y se la acercaron a la boca. \footnote{\textbf{19:29}
  Sal 69,21} \bibleverse{30} Así pues, cuando Jesús recibió el vinagre,
dijo: ``¡Se acabó!''. Entonces inclinó la cabeza y entregó su espíritu.

\bibleverse{31} Por lo tanto, los judíos, como era el día de la
preparación, para que los cuerpos no permanecieran en la cruz durante el
día de reposo (pues ese día de reposo era especial), pidieron a Pilato
que les quebraran las piernas y se los llevaran. \footnote{\textbf{19:31}
  Lev 23,7; Deut 21,23} \bibleverse{32} Vinieron, pues, los soldados y
rompieron las piernas del primero y del otro que estaba crucificado con
él; \bibleverse{33} pero cuando llegaron a Jesús y vieron que ya estaba
muerto, no le rompieron las piernas. \bibleverse{34} Sin embargo, uno de
los soldados le atravesó el costado con una lanza, e inmediatamente
salió sangre y agua. \bibleverse{35} El que ha visto ha dado testimonio,
y su testimonio es verdadero. Sabe que dice la verdad, para que creáis.
\bibleverse{36} Porque estas cosas sucedieron para que se cumpliera la
Escritura: ``Un hueso de él no será quebrado''. \footnote{\textbf{19:36}
  Éxodo 12:46; Números 9:12; Salmo 34:20} \bibleverse{37} Otra Escritura
dice: ``Mirarán al que traspasaron''. \footnote{\textbf{19:37} Zacarías
  12:10} \footnote{\textbf{19:37} Apoc 1,7}

\hypertarget{descenso-de-la-cruz-y-sepultura-de-jesuxfas}{%
\subsection{Descenso de la cruz y sepultura de
Jesús}\label{descenso-de-la-cruz-y-sepultura-de-jesuxfas}}

\bibleverse{38} Después de estas cosas, José de Arimatea, que era
discípulo de Jesús, pero en secreto por miedo a los judíos, pidió a
Pilato poder llevarse el cuerpo de Jesús. Pilato le dio permiso. Vino,
pues, y se llevó el cuerpo. \footnote{\textbf{19:38} Juan 7,13}
\bibleverse{39} Nicodemo, que al principio se acercó a Jesús de noche,
vino también trayendo una mezcla de mirra y áloes, como cien libras
romanas. \footnote{\textbf{19:39} 100 libras romanas de 12 onzas cada
  una, es decir, unas 72 libras o 33 kilogramos.} \footnote{\textbf{19:39}
  Juan 3,2} \bibleverse{40} Tomaron, pues, el cuerpo de Jesús y lo
envolvieron en telas de lino con las especias, según la costumbre de los
judíos de enterrarlo. \bibleverse{41} En el lugar donde fue crucificado
había un jardín. En el jardín había un sepulcro nuevo en el que nunca se
había puesto a nadie. \bibleverse{42} Entonces, a causa del día de
preparación de los judíos (pues el sepulcro estaba cerca), pusieron allí
a Jesús.

\hypertarget{maruxeda-magdalena-y-el-sepulcro-vacuxedo-pedro-y-juan-en-la-tumba}{%
\subsection{María Magdalena y el sepulcro vacío; Pedro y Juan en la
tumba}\label{maruxeda-magdalena-y-el-sepulcro-vacuxedo-pedro-y-juan-en-la-tumba}}

\hypertarget{section-19}{%
\section{20}\label{section-19}}

\bibleverse{1} El primer día de la semana, María Magdalena fue temprano,
cuando todavía estaba oscuro, al sepulcro, y vio que la piedra había
sido retirada del sepulcro. \bibleverse{2} Entonces corrió y vino a
Simón Pedro y al otro discípulo a quien Jesús amaba, y les dijo: ``¡Se
han llevado al Señor del sepulcro y no sabemos dónde lo han puesto!''
\footnote{\textbf{20:2} Juan 13,23}

\bibleverse{3} Salieron, pues, Pedro y el otro discípulo, y fueron hacia
el sepulcro. \bibleverse{4} Los dos corrieron juntos. El otro discípulo
se adelantó a Pedro y llegó primero al sepulcro. \bibleverse{5} Al
agacharse y mirar dentro, vio los lienzos tendidos; pero no entró.
\bibleverse{6} Entonces llegó Simón Pedro, siguiéndole, y entró en el
sepulcro. Vio los lienzos tendidos, \bibleverse{7} y el paño que había
estado sobre su cabeza, no tendido con los lienzos, sino enrollado en un
lugar aparte. \footnote{\textbf{20:7} Juan 11,44} \bibleverse{8}
Entonces entró también el otro discípulo que había llegado primero al
sepulcro, y vio y creyó. \bibleverse{9} Porque aún no entendían la
Escritura, que Él debía de resucitar de entre los muertos. \footnote{\textbf{20:9}
  Luc 24,25-27; Hech 2,24-32; 1Cor 15,4} \bibleverse{10} Entonces los
discípulos se fueron de nuevo a sus casas.

\hypertarget{apariciuxf3n-de-jesuxfas-a-maruxeda-magdalena}{%
\subsection{Aparición de Jesús a María
Magdalena}\label{apariciuxf3n-de-jesuxfas-a-maruxeda-magdalena}}

\bibleverse{11} Pero María estaba fuera, junto al sepulcro, llorando.
Mientras lloraba, se inclinó y miró dentro del sepulcro, \bibleverse{12}
y vio a dos ángeles vestidos de blanco sentados, uno a la cabecera y
otro a los pies, donde estaba el cuerpo de Jesús. \bibleverse{13} Le
preguntaron: ``Mujer, ¿por qué lloras?'' Ella les dijo: ``Porque se han
llevado a mi Señor, y no sé dónde lo han puesto''. \bibleverse{14}
Cuando dijo esto, se volvió y vio a Jesús de pie, y no sabía que era
Jesús.

\bibleverse{15} Jesús le dijo: ``Mujer, ¿por qué lloras? ¿A quién
buscas?'' Ella, suponiendo que era el jardinero, le dijo: ``Señor, si te
lo has llevado, dime dónde lo has puesto y me lo llevaré''.

\bibleverse{16} Jesús le dijo: ``María''. Se volvió y le dijo:
``¡Rabboni!'', \footnote{\textbf{20:16} Rabboni es una transliteración
  de la palabra hebrea ``gran maestro''.} que es como decir
``¡Maestro!''. \footnote{\textbf{20:16} o, Maestro}

\bibleverse{17} Jesús le dijo: ``No me retengas, porque todavía no he
subido a mi Padre; pero vete a mis hermanos y diles: ``Subo a mi Padre y
a vuestro Padre, a mi Dios y a vuestro Dios''\,''. \footnote{\textbf{20:17}
  Heb 2,11-12}

\bibleverse{18} Vino María Magdalena y contó a los discípulos que había
visto al Señor y que éste le había dicho estas cosas.

\hypertarget{jesuxfas-y-los-discuxedpulos-en-la-noche-del-domingo-de-pascua}{%
\subsection{Jesús y los discípulos en la noche del domingo de
Pascua}\label{jesuxfas-y-los-discuxedpulos-en-la-noche-del-domingo-de-pascua}}

\bibleverse{19} Así pues, al atardecer de aquel día, el primero de la
semana, y estando cerradas las puertas donde estaban reunidos los
discípulos, por miedo a los judíos, vino Jesús, se puso en medio y les
dijo: ``Paz a vosotros''.

\bibleverse{20} Cuando dijo esto, les mostró las manos y el costado. Los
discípulos se alegraron al ver al Señor. \footnote{\textbf{20:20} 1Jn
  1,1} \bibleverse{21} Entonces Jesús les dijo de nuevo: ``La paz sea
con vosotros. Como el Padre me ha enviado, así os envío yo''.
\footnote{\textbf{20:21} Juan 17,18} \bibleverse{22} Dicho esto, sopló
sobre ellos y les dijo: ``Recibid el Espíritu Santo. \bibleverse{23} Si
perdonáis los pecados a alguien, le serán perdonados. Si retienen los
pecados de alguien, les son retenido''. \footnote{\textbf{20:23} Mat
  18,16}

\hypertarget{los-discuxedpulos-con-tomuxe1s}{%
\subsection{Los discípulos con
Tomás}\label{los-discuxedpulos-con-tomuxe1s}}

\bibleverse{24} Pero Tomás, uno de los doce, llamado Dídimo,\footnote{\textbf{20:24}
  o, Twin} no estaba con ellos cuando vino Jesús. \footnote{\textbf{20:24}
  Juan 11,16; Juan 14,5; Juan 21,2} \bibleverse{25} Los demás discípulos
le dijeron: ``¡Hemos visto al Señor!'' Pero él les dijo: ``Si no veo en
sus manos la huella de los clavos, si no meto mi dedo en la huella de
los clavos y si no meto mi mano en su costado, no creeré''. \footnote{\textbf{20:25}
  Juan 19,34}

\bibleverse{26} Al cabo de ocho días, sus discípulos estaban de nuevo
dentro y Tomás estaba con ellos. Llegó Jesús, con las puertas cerradas,
se puso en medio y dijo: ``La paz sea con vosotros''. \bibleverse{27}
Luego dijo a Tomás: ``Alcanza aquí tu dedo y mira mis manos. Alcanza
aquí tu mano, y métela en mi costado. No seas incrédulo, sino
creyente''.

\bibleverse{28} Tomás le respondió: ``¡Señor mío y Dios mío!''
\footnote{\textbf{20:28} Juan 1,1}

\bibleverse{29} Jesús le dijo: ``Porque me has visto,\footnote{\textbf{20:29}
  TR añade ``Thomas,''} has creído. Dichosos los que no han visto y han
creído''. \footnote{\textbf{20:29} 1Pe 1,8; Heb 11,1}

\bibleverse{30} Por eso Jesús hizo otras muchas señales en presencia de
sus discípulos, que no están escritas en este libro; \footnote{\textbf{20:30}
  Juan 21,24-25} \bibleverse{31} pero éstas se han escrito para que
creáis que Jesús es el Cristo, el Hijo de Dios, y para que creyendo
tengáis vida en su nombre. \footnote{\textbf{20:31} 1Jn 5,13}

\hypertarget{jesuxfas-se-revela-a-sus-discuxedpulos-en-el-lago-de-tiberuxedades}{%
\subsection{Jesús se revela a sus discípulos en el lago de
Tiberíades}\label{jesuxfas-se-revela-a-sus-discuxedpulos-en-el-lago-de-tiberuxedades}}

\hypertarget{section-20}{%
\section{21}\label{section-20}}

\bibleverse{1} Después de estas cosas, Jesús se reveló de nuevo a los
discípulos en el mar de Tiberias. Se reveló así. \bibleverse{2} Estaban
juntos Simón Pedro, Tomás, llamado Dídimo, \footnote{\textbf{21:2} o,
  Twin} Natanael, de Caná de Galilea, y los hijos de Zebedeo, y otros
dos de sus discípulos. \footnote{\textbf{21:2} Juan 1,45} \bibleverse{3}
Simón Pedro les dijo: ``Voy a pescar''. Le dijeron: ``Nosotros también
vamos contigo''. Inmediatamente salieron y entraron en la barca. Aquella
noche no pescaron nada. \bibleverse{4} Pero cuando ya se hizo de día,
Jesús se paró en la playa; pero los discípulos no sabían que era Jesús.
\footnote{\textbf{21:4} Juan 20,14; Luc 24,16} \bibleverse{5} Entonces
Jesús les dijo: ``Hijos, ¿tenéis algo de comer?'' Le respondieron:
``No''. \footnote{\textbf{21:5} Luc 24,41}

\bibleverse{6} Les dijo: ``Echad la red a la derecha de la barca y
encontraréis algunos''. Así pues, lo echaron, y entonces no pudieron
sacarla por la multitud de peces. \footnote{\textbf{21:6} Luc 5,4-7}
\bibleverse{7} Aquel discípulo al que Jesús amaba dijo a Pedro: ``¡Es el
Señor!'' Cuando Simón Pedro oyó que era el Señor, se envolvió con su
capa (pues estaba desnudo) y se arrojó al mar. \footnote{\textbf{21:7}
  Juan 13,23} \bibleverse{8} Pero los demás discípulos venían en la
barca pequeña (pues no estaban lejos de la tierra, sino a unos
doscientos codos\footnote{\textbf{21:8} 200 codos son unas 100 yardas o
  unos 91 metros} ), arrastrando la red llena de peces. \bibleverse{9}
Cuando salieron a tierra, vieron allí un fuego de brasas, con peces y
panes puestos sobre él. \bibleverse{10} Jesús les dijo: ``Traed algunos
de los peces que acabáis de pescar''.

\bibleverse{11} Simón Pedro subió y sacó la red a tierra, llena de
ciento cincuenta y tres peces grandes. A pesar de ser tantos, la red no
se rompió.

\bibleverse{12} Jesús les dijo: ``¡Vengan a desayunar!'' Ninguno de los
discípulos se atrevió a preguntarle: ``¿Quién eres tú?'', sabiendo que
era el Señor.

\bibleverse{13} Entonces Jesús se acercó, tomó el pan y se lo dio, y el
pescado también. \footnote{\textbf{21:13} Juan 6,11} \bibleverse{14}
Esta es la tercera vez que Jesús se revela a sus discípulos después de
haber resucitado.

\hypertarget{trus-reinstalado-en-su-cargo-pastoral-profecuxeda-sobre-el-fin-de-la-vida-de-pedro-y-el-discuxedpulo-amado}{%
\subsection{Trus reinstalado en su cargo pastoral; Profecía sobre el fin
de la vida de Pedro y el discípulo
amado}\label{trus-reinstalado-en-su-cargo-pastoral-profecuxeda-sobre-el-fin-de-la-vida-de-pedro-y-el-discuxedpulo-amado}}

\bibleverse{15} Cuando hubieron desayunado, Jesús dijo a Simón Pedro:
``Simón, hijo de Jonás, ¿me amas más que éstos?'' Le dijo: ``Sí, Señor;
tú sabes que te tengo afecto''. Le dijo: ``Apacienta mis corderos''.
\footnote{\textbf{21:15} Juan 1,42} \bibleverse{16} Le volvió a decir
por segunda vez: ``Simón, hijo de Jonás, ¿me amas?'' Le dijo: ``Sí,
Señor; tú sabes que te tengo afecto''. Le dijo: ``Cuida mis ovejas''.
\footnote{\textbf{21:16} 1Pe 5,2; 1Pe 5,4} \bibleverse{17} Le dijo por
tercera vez: ``Simón, hijo de Jonás, ¿me tienes afecto?'' Pedro se
afligió porque le preguntó por tercera vez: ``¿Me tienes afecto?''. Él
le dijo: ``Señor, tú lo sabes todo. Sabes que te tengo afecto''. Jesús
le dijo: ``Apacienta mis ovejas. \footnote{\textbf{21:17} Juan 13,38;
  Juan 16,30} \bibleverse{18} De cierto te digo que cuando eras joven te
vestías solo y andabas por donde querías. Pero cuando seas viejo,
extenderás tus manos, y otro te vestirá y te llevará donde no quieras.''

\bibleverse{19} Y dijo esto, dando a entender con qué clase de muerte
glorificaría a Dios. Cuando hubo dicho esto, le dijo: ``Sígueme''.
\footnote{\textbf{21:19} Juan 13,36}

\bibleverse{20} Entonces Pedro, volviéndose, vio que le seguía un
discípulo. Este era el discípulo al que Jesús amaba, el que también se
había apoyado en el pecho de Jesús en la cena y había preguntado:
``Señor, ¿quién te va a entregar?'' \footnote{\textbf{21:20} Juan 13,23;
  Juan 13,25} \bibleverse{21} Pedro, al verlo, dijo a Jesús: ``Señor, ¿y
éste?''

\bibleverse{22} Jesús le dijo: ``Si quiero que se quede hasta que yo
venga, ¿qué te importa? Sígueme''. \bibleverse{23} Así pues, se difundió
entre los hermanos el dicho de que este discípulo no moriría. Pero Jesús
no le dijo que no moriría, sino: ``Si quiero que se quede hasta que yo
venga, ¿qué te importa?''

\bibleverse{24} Este es el discípulo que da testimonio de estas cosas, y
escribió estas cosas. Sabemos que su testimonio es verdadero.
\footnote{\textbf{21:24} Juan 15,27} \bibleverse{25} Hay también muchas
otras cosas que hizo Jesús, que si se escribieran todas, supongo que ni
el mundo mismo tendría espacio para los libros que se escribirían.
\footnote{\textbf{21:25} Juan 20,30}
