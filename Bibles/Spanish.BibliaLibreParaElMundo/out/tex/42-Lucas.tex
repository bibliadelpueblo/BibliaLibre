\hypertarget{introducciuxf3n}{%
\subsection{Introducción}\label{introducciuxf3n}}

\hypertarget{section}{%
\section{1}\label{section}}

\bibleverse{1} Puesto que muchos han emprendido la tarea de poner en
orden una narración relativa a los asuntos que se han cumplido entre
nosotros, \bibleverse{2} tal como nos lo transmitieron los que desde el
principio fueron testigos oculares y servidores de la palabra,
\footnote{\textbf{1:2} 1Jn 1,1-4} \bibleverse{3} también me pareció
bien, habiendo entendido el curso de todas las cosas con exactitud desde
el principio, escribirte en orden, excelentísimo Teófilo; \footnote{\textbf{1:3}
  Hech 1,1; Col 4,14} \bibleverse{4} para que conozcas la certeza
relativa a las cosas en las que fuiste instruido.

\hypertarget{anuncio-del-nacimiento-de-juan-el-bautista}{%
\subsection{Anuncio del nacimiento de Juan el
Bautista}\label{anuncio-del-nacimiento-de-juan-el-bautista}}

\bibleverse{5} Había en los días de Herodes, rey de Judea, un sacerdote
llamado Zacarías, de la división sacerdotal de Abías. Tenía una esposa
de las hijas de Aarón, que se llamaba Elisabet. \footnote{\textbf{1:5}
  1Cró 24,10; 1Cró 24,19} \bibleverse{6} Ambos eran justos ante Dios, y
andaban irreprochablemente en todos los mandamientos y ordenanzas del
Señor. \bibleverse{7} Pero no tuvieron hijos, porque Elisabet era
estéril, y ambos eran de edad avanzada.

\bibleverse{8} Mientras ejercía el oficio sacerdotal ante Dios en el
orden de su división \bibleverse{9} según la costumbre del oficio
sacerdotal, le tocaba entrar en el templo del Señor y quemar incienso.
\footnote{\textbf{1:9} Éxod 20,7} \bibleverse{10} Toda la multitud del
pueblo oraba fuera a la hora del incienso.

\bibleverse{11} Se le apareció un ángel del Señor, de pie a la derecha
del altar del incienso. \bibleverse{12} Zacarías se turbó al verlo y le
entró miedo. \bibleverse{13} Pero el ángel le dijo: ``No temas,
Zacarías, porque tu petición ha sido escuchada. Tu mujer, Elisabet, te
dará a luz un hijo, y le pondrás por nombre Juan. \bibleverse{14}
Tendrás alegría y gozo, y muchos se alegrarán de su nacimiento.
\bibleverse{15} Porque será grande a los ojos del Señor, y no beberá
vino ni bebida fuerte. Estará lleno del Espíritu Santo, incluso desde el
vientre de su madre. \footnote{\textbf{1:15} Jue 13,4-5} \bibleverse{16}
Hará que muchos de los hijos de Israel se conviertan al Señor, su Dios.
\bibleverse{17} Irá delante de él con el espíritu y el poder de Elías,
`para hacer volver el corazón de los padres a los hijos'\footnote{\textbf{1:17}
  Malaquías 4:6} , y a los desobedientes a la sabiduría de los justos;
para preparar un pueblo preparado para el Señor.'' \footnote{\textbf{1:17}
  Mat 17,11-13; Mal 3,1; Mal 4,5-6}

\bibleverse{18} Zacarías dijo al ángel: ``¿Cómo puedo estar seguro de
esto? Porque soy un anciano, y mi mujer está muy avanzada en años''.
\footnote{\textbf{1:18} Gén 18,11}

\bibleverse{19} El ángel le respondió: ``Soy Gabriel, que está en la
presencia de Dios. He sido enviado para hablarte y traerte esta buena
noticia. \footnote{\textbf{1:19} Dan 8,16} \bibleverse{20} He aquí
que\footnote{\textbf{1:20} ``Contemplar'', de ``\greek{ἰδοὺ}'',
  significa mirar, fijarse, observar, ver o contemplar. Se utiliza a
  menudo como interjección.} te quedarás callado y no podrás hablar
hasta el día en que sucedan estas cosas, porque no creíste en mis
palabras, que se cumplirán a su debido tiempo.''

\bibleverse{21} La gente esperaba a Zacarías y se maravillaba de que se
demorara en el templo. \bibleverse{22} Cuando salió, no pudo hablarles.
Se dieron cuenta de que había tenido una visión en el templo. Siguió
haciéndoles señales, y permaneció mudo. \bibleverse{23} Cuando se
cumplieron los días de su servicio, se fue a su casa. \bibleverse{24}
Después de estos días, concibió Elisabet, su mujer, y se escondió cinco
meses, diciendo: \bibleverse{25} ``Así me ha hecho el Señor en los días
en que me ha mirado, para quitar mi oprobio entre los hombres.''
\footnote{\textbf{1:25} Gén 30,23}

\hypertarget{anunciando-el-nacimiento-de-jesuxfas}{%
\subsection{Anunciando el nacimiento de
Jesús}\label{anunciando-el-nacimiento-de-jesuxfas}}

\bibleverse{26} En el sexto mes, el ángel Gabriel fue enviado por Dios a
una ciudad de Galilea llamada Nazaret, \bibleverse{27} a una virgen
comprometida a casarse con un hombre que se llamaba José, de la casa de
David. La virgen se llamaba María. \footnote{\textbf{1:27} Mat 1,16; Mat
  1,18} \bibleverse{28} Al entrar, el ángel le dijo: ``¡Alégrate, muy
favorecida! El Señor está contigo. Bendita eres entre las mujeres''.

\bibleverse{29} Pero cuando lo vio, se preocupó mucho por el dicho, y
pensó qué clase de saludo sería éste. \bibleverse{30} El ángel le dijo:
``No temas, María, porque has encontrado el favor de Dios.
\bibleverse{31} He aquí que concebirás en tu seno y darás a luz un hijo,
al que pondrás por nombre ``Jesús''. \footnote{\textbf{1:31} Is 7,14;
  Mat 1,21-23} \bibleverse{32} Será grande y se llamará Hijo del
Altísimo. El Señor Dios le dará el trono de su padre David, \footnote{\textbf{1:32}
  Is 9,7} \bibleverse{33} y reinará sobre la casa de Jacob para siempre.
Su Reino no tendrá fin''.

\bibleverse{34} María dijo al ángel: ``¿Cómo puede ser esto, siendo yo
virgen?''.

\bibleverse{35} El ángel le respondió: ``El Espíritu Santo vendrá sobre
ti, y el poder del Altísimo te cubrirá con su sombra. Por eso también el
santo que nazca de ti será llamado Hijo de Dios. \footnote{\textbf{1:35}
  Mat 1,18; Mat 1,20} \bibleverse{36} He aquí que también Elisabet, tu
pariente, ha concebido un hijo en su vejez; y éste es el sexto mes de la
que se llamaba estéril. \bibleverse{37} Porque nada de lo dicho por Dios
es imposible.''\footnote{\textbf{1:37} o ``Porque todo lo que Dios dice
  es posible''.} \footnote{\textbf{1:37} Gén 18,14}

\bibleverse{38} María dijo: ``He aquí la sierva del Señor; hágase en mí
según tu palabra''. Entonces el ángel se alejó de ella.

\hypertarget{el-encuentro-de-las-dos-madres-el-saludo-de-elisabeth}{%
\subsection{El encuentro de las dos madres; El saludo de
Elisabeth}\label{el-encuentro-de-las-dos-madres-el-saludo-de-elisabeth}}

\bibleverse{39} En aquellos días, María se levantó y se fue de prisa a
la región montañosa, a una ciudad de Judá, \bibleverse{40} entró en casa
de Zacarías y saludó a Isabel. \bibleverse{41} Cuando Isabel oyó el
saludo de María, el niño saltó en su seno; e Isabel quedó llena del
Espíritu Santo. \bibleverse{42} Gritó en voz alta y dijo: ``Bendita eres
entre las mujeres y bendito es el fruto de tu vientre. \bibleverse{43}
¿Por qué soy tan favorecida, para que la madre de mi Señor venga a mí?
\bibleverse{44} Porque cuando la voz de tu saludo llegó a mis oídos, el
niño saltó de alegría en mi vientre. \bibleverse{45} ¡Bienaventurada la
que ha creído, porque se cumplirán las cosas que se le han dicho de
parte del Señor!'' \footnote{\textbf{1:45} Luc 11,28}

\hypertarget{canciuxf3n-de-alabanza-de-maruxeda}{%
\subsection{Canción de alabanza de
María}\label{canciuxf3n-de-alabanza-de-maruxeda}}

\bibleverse{46} María dijo, ``Mi alma engrandece al Señor. \footnote{\textbf{1:46}
  1Sam 2,1-10} \bibleverse{47} Mi espíritu se ha alegrado en Dios, mi
Salvador, \bibleverse{48} pues ha mirado el humilde estado de su sierva.
Porque he aquí que, a partir de ahora, todas las generaciones me
llamarán dichosa. \bibleverse{49} Porque el que es poderoso ha hecho
grandes cosas por mí. Santo es su nombre. \bibleverse{50} Su
misericordia es por generaciones y generaciones sobre los que le temen.
\footnote{\textbf{1:50} Sal 103,13; Sal 103,17} \bibleverse{51} Ha
demostrado poder con su brazo. Ha dispersado a los orgullosos en la
imaginación de sus corazones. \bibleverse{52} Ha derribado a los
príncipes de sus tronos, y ha exaltado a los humildes. \footnote{\textbf{1:52}
  Sal 147,6} \bibleverse{53} Ha colmado de bienes a los hambrientos. Ha
enviado a los ricos con las manos vacías. \footnote{\textbf{1:53} Sal
  34,10; Sal 107,9} \bibleverse{54} Ha dado ayuda a Israel, su siervo,
para que se acuerde de la misericordia, \bibleverse{55} como habló con
nuestros padres, a Abraham y a su descendencia para \footnote{\textbf{1:55}
  o, semilla} siempre''. \footnote{\textbf{1:55} Gén 17,7; Gén 18,18}

\bibleverse{56} María se quedó con ella unos tres meses y luego volvió a
su casa.

\hypertarget{nacimiento-circuncisiuxf3n-y-juventud-de-juan-himno-de-zacaruxedas}{%
\subsection{Nacimiento, circuncisión y juventud de Juan; Himno de
Zacarías}\label{nacimiento-circuncisiuxf3n-y-juventud-de-juan-himno-de-zacaruxedas}}

\bibleverse{57} Se cumplió el tiempo en que Elisabet debía dar a luz, y
dio a luz un hijo. \bibleverse{58} Sus vecinos y sus parientes oyeron
que el Señor había engrandecido su misericordia con ella, y se alegraron
con ella. \bibleverse{59} Al octavo día vinieron a circuncidar al niño,
y quisieron llamarlo Zacarías, como el nombre de su padre. \footnote{\textbf{1:59}
  Gén 17,12} \bibleverse{60} Su madre respondió: ``No, sino que se
llamará Juan''.

\bibleverse{61} Le dijeron: ``No hay nadie entre tus parientes que se
llame así''. \bibleverse{62} Hicieron señas a su padre de cómo quería
que se llamara.

\bibleverse{63} Pidió una tablilla y escribió: ``Se llama Juan''. Todos
se maravillaron. \bibleverse{64} Al instante se le abrió la boca y se le
liberó la lengua, y habló bendiciendo a Dios. \bibleverse{65} El temor
se apoderó de todos los que vivían alrededor, y todos estos dichos
fueron comentados en toda la región montañosa de Judea. \bibleverse{66}
Todos los que los oían los guardaban en su corazón, diciendo: ``¿Qué
será entonces este niño?'' La mano del Señor estaba con él.

\hypertarget{himno-profuxe9tico-de-alabanza-de-zacaruxedas}{%
\subsection{Himno profético de alabanza de
Zacarías}\label{himno-profuxe9tico-de-alabanza-de-zacaruxedas}}

\bibleverse{67} Su padre Zacarías fue lleno del Espíritu Santo y
profetizó diciendo, \bibleverse{68} ``Bendito sea el Señor, el Dios de
Israel, porque ha visitado y redimido a su pueblo; \footnote{\textbf{1:68}
  Luc 7,16} \bibleverse{69} y nos ha levantado un cuerno de salvación en
la casa de su siervo David \footnote{\textbf{1:69} Sal 132,17}
\bibleverse{70} (como habló por boca de sus santos profetas que han sido
desde la antigüedad), \bibleverse{71} salvación de nuestros enemigos y
de la mano de todos los que nos odian; \bibleverse{72} para mostrar
misericordia hacia nuestros padres, para recordar su santa alianza,
\footnote{\textbf{1:72} Gén 17,7} \bibleverse{73} el juramento que hizo
a Abraham, nuestro padre, \footnote{\textbf{1:73} Gén 22,16-18; Miq 7,20}
\bibleverse{74} que nos conceda que, siendo liberados de la mano de
nuestros enemigos, debe servirle sin miedo, \footnote{\textbf{1:74} Tit
  2,12; Tit 1,2-14} \bibleverse{75} en santidad y justicia ante él todos
los días de nuestra vida. \bibleverse{76} Y tú, niño, serás llamado
profeta del Altísimo; porque irás delante de la cara del Señor para
preparar sus caminos, \footnote{\textbf{1:76} Mal 3,1} \bibleverse{77}
para dar conocimiento de la salvación a su pueblo por la remisión de sus
pecados, \footnote{\textbf{1:77} Jer 31,34} \bibleverse{78} por la
tierna misericordia de nuestro Dios, por la que nos visitará la aurora
de lo alto, \footnote{\textbf{1:78} Núm 24,17; Is 60,1-2; Mal 4,2}
\bibleverse{79} para iluminar a los que están en las tinieblas y en la
sombra de la muerte; para guiar nuestros pies por el camino de la paz''.
\footnote{\textbf{1:79} Is 9,2}

\bibleverse{80} El niño crecía y se fortalecía en espíritu, y estuvo en
el desierto hasta el día de su aparición pública ante Israel.
\footnote{\textbf{1:80} Mat 3,1}

\hypertarget{el-decreto-del-emperador-augusto-y-su-significado-para-el-nacimiento-de-jesuxfas}{%
\subsection{El decreto del emperador Augusto y su significado para el
nacimiento de
Jesús}\label{el-decreto-del-emperador-augusto-y-su-significado-para-el-nacimiento-de-jesuxfas}}

\hypertarget{section-1}{%
\section{2}\label{section-1}}

\bibleverse{1} En aquellos días, salió un decreto de César Augusto para
que se inscribiera todo el mundo. \bibleverse{2} Esta fue la primera
inscripción que se hizo cuando Quirinius era gobernador de Siria.
\bibleverse{3} Todos fueron a inscribirse, cada uno a su ciudad.
\bibleverse{4} También José subió de Galilea, de la ciudad de Nazaret, a
Judea, a la ciudad de David, que se llama Belén, porque era de la casa y
de la familia de David, \bibleverse{5} para inscribirse con María, que
estaba comprometida con él como esposa, estando embarazada.

\bibleverse{6} Mientras estaban allí, le llegó el día de dar a luz.
\bibleverse{7} Dio a luz a su hijo primogénito. Lo envolvió en cintas de
tela y lo puso en un pesebre, porque no había sitio para ellos en la
posada. \footnote{\textbf{2:7} Mat 1,25}

\hypertarget{los-pastores-en-el-campo-y-la-apariciuxf3n-angelical}{%
\subsection{Los pastores en el campo y la aparición
angelical}\label{los-pastores-en-el-campo-y-la-apariciuxf3n-angelical}}

\bibleverse{8} Había en la misma región unos pastores que permanecían en
el campo y velaban de noche por su rebaño. \bibleverse{9} He aquí que un
ángel del Señor se puso junto a ellos, y la gloria del Señor los rodeó,
y se asustaron. \bibleverse{10} El ángel les dijo: ``No temáis, porque
he aquí que os traigo una buena noticia de gran alegría que será para
todo el pueblo. \bibleverse{11} Porque os ha nacido hoy, en la ciudad de
David, un Salvador, que es Cristo\footnote{\textbf{2:11} ``Cristo''
  significa ``Ungido''.} el Señor. \bibleverse{12} Esta es la señal para
vosotros: encontraréis un niño envuelto en tiras de tela, acostado en un
comedero''. \bibleverse{13} De repente, apareció con el ángel una
multitud del ejército celestial que alababa a Dios y decía \footnote{\textbf{2:13}
  Sal 103,20-21; Dan 7,10} \bibleverse{14} ``Gloria a Dios en las
alturas, en la tierra la paz, la buena voluntad hacia los hombres''.
\footnote{\textbf{2:14} Luc 19,38; Is 57,19; Efes 2,14; Efes 2,17}

\hypertarget{los-pastores-con-el-niuxf1o-jesuxfas-en-beluxe9n}{%
\subsection{Los pastores con el niño Jesús en
Belén}\label{los-pastores-con-el-niuxf1o-jesuxfas-en-beluxe9n}}

\bibleverse{15} Cuando los ángeles se alejaron de ellos hacia el cielo,
los pastores se dijeron unos a otros: ``Vamos ahora a Belén a ver esto
que ha sucedido y que el Señor nos ha dado a conocer.'' \bibleverse{16}
Llegaron a toda prisa y encontraron a María y a José, y al niño acostado
en el pesebre. \bibleverse{17} Al verlo, difundieron ampliamente el
dicho que se les había dicho sobre este niño. \bibleverse{18} Todos los
que lo oían se asombraban de lo que les decían los pastores.
\bibleverse{19} Pero María guardaba todas estas palabras, meditándolas
en su corazón. \bibleverse{20} Los pastores volvieron glorificando y
alabando a Dios por todo lo que habían oído y visto, tal como se les
había dicho.

\hypertarget{circuncisiuxf3n-y-presentaciuxf3n-de-jesuxfas-en-el-templo}{%
\subsection{Circuncisión y presentación de Jesús en el
templo}\label{circuncisiuxf3n-y-presentaciuxf3n-de-jesuxfas-en-el-templo}}

\bibleverse{21} Cuando se cumplieron los ocho días para la circuncisión
del niño, se le puso el nombre de Jesús, que le fue dado por el ángel
antes de ser concebido en el vientre. \footnote{\textbf{2:21} Luc 1,31;
  Luc 1,59; Gén 17,12}

\bibleverse{22} Cuando se cumplieron los días de su purificación según
la ley de Moisés, lo llevaron a Jerusalén para presentarlo al Señor
\footnote{\textbf{2:22} Núm 18,15-16} \bibleverse{23} (como está escrito
en la ley del Señor: ``Todo varón que abra el vientre será llamado santo
para el Señor''), \footnote{\textbf{2:23} Éxodo 13:2,12} \bibleverse{24}
y para ofrecer un sacrificio según lo que se dice en la ley del Señor:
``Un par de tórtolas o dos pichones\footnote{\textbf{2:24} Levítico 12:8}
''.

\hypertarget{saludos-himnos-y-profecuxeda-de-simeuxf3n-envejecido}{%
\subsection{Saludos, himnos y profecía de Simeón
envejecido}\label{saludos-himnos-y-profecuxeda-de-simeuxf3n-envejecido}}

\bibleverse{25} He aquí que había en Jerusalén un hombre que se llamaba
Simeón. Este hombre era justo y piadoso, y buscaba la consolación de
Israel, y el Espíritu Santo estaba sobre él. \footnote{\textbf{2:25} Is
  40,1; Is 49,13} \bibleverse{26} Le había sido revelado por el Espíritu
Santo que no vería la muerte antes de ver al Cristo del Señor.
\footnote{\textbf{2:26} ``Cristo'' (griego) y ``Mesías'' (hebreo)
  significan ambos ``Ungido''} \bibleverse{27} Entró en el templo en el
Espíritu. Cuando los padres introdujeron al niño, Jesús, para que
hicieran con él lo que estaba previsto en la ley, \bibleverse{28}
entonces lo recibió en sus brazos, bendijo a Dios y dijo \bibleverse{29}
``Ahora, Señor, liberas a tu siervo, en paz, según tu palabra;
\bibleverse{30} porque mis ojos han visto tu salvación, \bibleverse{31}
que has preparado delante de todos los pueblos; \bibleverse{32} una luz
para la revelación a las naciones, y la gloria de tu pueblo Israel''.
\footnote{\textbf{2:32} Is 49,6}

\bibleverse{33} José y su madre se maravillaban de lo que se decía de
él. \bibleverse{34} Simeón los bendijo, y dijo a María, su madre: ``He
aquí que este niño está destinado a la caída y al levantamiento de
muchos en Israel, y a ser una señal de la que se habla. \footnote{\textbf{2:34}
  Is 8,13-14; Mat 21,42; Mat 21,44; Hech 28,22; 1Cor 1,23}
\bibleverse{35} Sí, una espada atravesará tu propia alma, para que se
revelen los pensamientos de muchos corazones.'' \footnote{\textbf{2:35}
  Juan 19,25}

\hypertarget{la-anciana-hanna-saluda-al-niuxf1o-regreso-de-la-sagrada-familia-a-nazaret}{%
\subsection{La anciana Hanna saluda al niño; Regreso de la Sagrada
Familia a
Nazaret}\label{la-anciana-hanna-saluda-al-niuxf1o-regreso-de-la-sagrada-familia-a-nazaret}}

\bibleverse{36} Había una tal Ana, profetisa, hija de Fanuel, de la
tribu de Aser (era de edad avanzada, pues había vivido con un marido
siete años desde su virginidad, \bibleverse{37} y llevaba como ochenta y
cuatro años de viuda), que no se apartaba del templo, adorando con
ayunos y peticiones noche y día. \footnote{\textbf{2:37} 1Tim 5,5}
\bibleverse{38} Subiendo a esa misma hora, dio gracias al Señor y habló
de él a todos los que buscaban la redención en Jerusalén. \footnote{\textbf{2:38}
  Is 52,9}

\bibleverse{39} Cuando cumplieron todo lo que estaba previsto en la ley
del Señor, volvieron a Galilea, a su ciudad, Nazaret. \bibleverse{40} El
niño crecía y se fortalecía en su espíritu, lleno de sabiduría, y la
gracia de Dios estaba sobre él.

\hypertarget{el-niuxf1o-jesuxfas-de-doce-auxf1os-en-el-templo}{%
\subsection{El niño Jesús de doce años en el
templo}\label{el-niuxf1o-jesuxfas-de-doce-auxf1os-en-el-templo}}

\bibleverse{41} Sus padres iban todos los años a Jerusalén en la fiesta
de la Pascua. \bibleverse{42} Cuando tenía doce años, subieron a
Jerusalén según la costumbre de la fiesta; \bibleverse{43} y cuando se
cumplieron los días, al regresar, el niño Jesús se quedó en Jerusalén.
José y su madre no lo sabían, \footnote{\textbf{2:43} Éxod 12,18}
\bibleverse{44} pero suponiendo que estaba en la compañía, se fueron de
viaje un día; y lo buscaron entre sus parientes y conocidos.
\bibleverse{45} Al no encontrarlo, volvieron a Jerusalén buscándolo.
\bibleverse{46} Al cabo de tres días lo encontraron en el templo,
sentado en medio de los maestros, escuchándolos y haciéndoles preguntas.
\bibleverse{47} Todos los que le oían se asombraban de su comprensión y
de sus respuestas. \bibleverse{48} Al verle, se asombraron; y su madre
le dijo: ``Hijo, ¿por qué nos has tratado así? He aquí que tu padre y yo
te buscábamos ansiosamente''.

\bibleverse{49} Él les dijo: ``¿Por qué me buscabais? ¿No sabíais que
debía estar en la casa de mi Padre?'' \footnote{\textbf{2:49} Juan 2,16}
\bibleverse{50} Ellos no entendían lo que les decía. \bibleverse{51}
Bajó con ellos y llegó a Nazaret. Se sometió a ellos, y su madre
guardaba todas estas palabras en su corazón. \bibleverse{52} Y Jesús
crecía en sabiduría y en estatura, y en gracia ante Dios y los hombres.
\footnote{\textbf{2:52} 1Sam 2,26}

\hypertarget{apariciuxf3n-sermuxf3n-penitencial-efectividad-y-captura-de-juan-bautista}{%
\subsection{Aparición, sermón penitencial, efectividad y captura de Juan
Bautista}\label{apariciuxf3n-sermuxf3n-penitencial-efectividad-y-captura-de-juan-bautista}}

\hypertarget{section-2}{%
\section{3}\label{section-2}}

\bibleverse{1} En el año quince del reinado de Tiberio César, siendo
Poncio Pilato gobernador de Judea, y Herodes tetrarca de Galilea, y su
hermano Felipe tetrarca de la región de Iturea y Traconite, y Lisanias
tetrarca de Abilinia, \bibleverse{2} durante el sumo sacerdocio de Anás
y Caifás, vino la palabra de Dios a Juan, hijo de Zacarías, en el
desierto. \bibleverse{3} Este llegó a toda la región alrededor del
Jordán, predicando el bautismo de arrepentimiento para la remisión de
los pecados. \bibleverse{4} Como está escrito en el libro de las
palabras del profeta Isaías``La voz de uno que clama en el
desierto,`\,'Preparad el camino del Señor'. Enderezad sus caminos.
\bibleverse{5} Todo valle se llenará. Toda montaña y colina será
abatida. Lo torcido se volverá recto, y los caminos ásperos allanados.
\bibleverse{6} Toda carne verá la salvación de Dios\footnote{\textbf{3:6}
  Isaías 40:3-5} '\,''.

\bibleverse{7} Por eso dijo a las multitudes que salían para ser
bautizadas por él: ``Vástagos de víboras, ¿quién os ha advertido que
huyáis de la ira que ha de venir? \bibleverse{8} Producid, pues, frutos
dignos de arrepentimiento, y no empecéis a decir entre vosotros:
``Tenemos a Abraham por padre'', porque os digo que Dios puede suscitar
hijos a Abraham de estas piedras. \bibleverse{9} También ahora el hacha
está a la raíz de los árboles. Por eso, todo árbol que no da buen fruto
es cortado y arrojado al fuego.''

\bibleverse{10} Las multitudes le preguntaron: ``¿Qué debemos hacer
entonces?''

\bibleverse{11} Les respondió: ``El que tenga dos túnicas, que se las dé
al que no tiene. El que tenga comida, que haga lo mismo''.

\bibleverse{12} También los recaudadores de impuestos vinieron a
bautizarse, y le dijeron: ``Maestro, ¿qué debemos hacer?''

\bibleverse{13} Les dijo: ``No recojan más de lo que les corresponde''.

\bibleverse{14} Los soldados también le preguntaron: ``¿Y nosotros? ¿Qué
debemos hacer?'' Les dijo: ``No extorsionéis a nadie con violencia, ni
acuséis a nadie injustamente. Contentaos con vuestro salario''.

\bibleverse{15} Mientras la gente estaba a la expectativa, y todos los
hombres discutían en sus corazones acerca de Juan, si acaso él era el
Cristo, \footnote{\textbf{3:15} Juan 1,19-28} \bibleverse{16} Juan les
respondió a todos: ``Yo, en efecto, os bautizo con agua, pero viene el
que es más poderoso que yo, la correa de cuyas sandalias no soy digno de
desatar. Él os bautizará en el Espíritu Santo y en el fuego.
\bibleverse{17} Tiene en la mano su aventador, y limpiará a fondo su
era, y recogerá el trigo en su granero; pero quemará la paja con fuego
inextinguible.''

\bibleverse{18} Entonces, con otras muchas exhortaciones, anunciaba al
pueblo la buena nueva, \bibleverse{19} pero Herodes el tetrarca, al
\footnote{\textbf{3:19} un tetrarca es uno de los cuatro gobernadores de
  una provincia} ser reprendido por él por Herodías, la \footnote{\textbf{3:19}
  TR lee ``del hermano Felipe'' en lugar de ``del hermano''} mujer de su
hermano, y por todas las cosas malas que Herodes había hecho,
\footnote{\textbf{3:19} Mat 14,3-4; Mar 6,17-18} \bibleverse{20} añadió
a todas ellas la de encerrar a Juan en la cárcel.

\hypertarget{bautismo-y-consagraciuxf3n-del-mesuxedas-de-jesuxfas}{%
\subsection{Bautismo y consagración del Mesías de
Jesús}\label{bautismo-y-consagraciuxf3n-del-mesuxedas-de-jesuxfas}}

\bibleverse{21} Cuando todo el pueblo se bautizaba, Jesús también se
había bautizado y estaba orando. El cielo se abrió, \bibleverse{22} y el
Espíritu Santo descendió en forma corporal como una paloma sobre él; y
una voz salió del cielo, diciendo: ``Tú eres mi Hijo amado. En ti me
complazco''. \footnote{\textbf{3:22} Luc 9,35; Juan 1,32}

\hypertarget{uxe1rbol-genealuxf3gico-de-jesuxfas}{%
\subsection{Árbol genealógico de
Jesús}\label{uxe1rbol-genealuxf3gico-de-jesuxfas}}

\bibleverse{23} El mismo Jesús, cuando comenzó a enseñar, tenía unos
treinta años, siendo hijo (como se suponía) de José, hijo de Eli,
\footnote{\textbf{3:23} Luc 4,22} \bibleverse{24} hijo de Matat, hijo de
Leví, hijo de Melchi, hijo de Jannai, hijo de José, \bibleverse{25} hijo
de Matatías, hijo de Amós, hijo de Nahum, hijo de Esli, hijo de Naggai,
\bibleverse{26} hijo de Maat, hijo de Matatías, hijo de Semein hijo de
José, hijo de Judá, \bibleverse{27} hijo de Joanán, hijo de Rhesa, hijo
de Zorobabel, hijo de Sealtiel, hijo de Neri, \bibleverse{28} hijo de
Melchi, hijo de Addi, hijo de Cosam, hijo de Elmodam, hijo de Er,
\bibleverse{29} hijo de José, hijo de Eliezer, hijo de Jorim, hijo de
Matat, hijo de Leví, \bibleverse{30} hijo de Simeón, hijo de Judá, hijo
de José, hijo de Jonán, hijo de Eliaquim, \bibleverse{31} hijo de Melea,
hijo de Menán, hijo de Matatá, hijo de Natán, hijo de David, \footnote{\textbf{3:31}
  2Sam 5,14} \bibleverse{32} hijo de Jesé, hijo de Obed, hijo de Booz,
hijo de Salmón, hijo de Nahsón, \footnote{\textbf{3:32} Rut 4,17-22}
\bibleverse{33} hijo de Aminadab, hijo de Aram,\footnote{\textbf{3:33}
  NU lee ``Admin, el hijo de Arni'' en lugar de ``Aram''} hijo de
Hezrón, hijo de Pérez, hijo de Judá, \footnote{\textbf{3:33} Gén 5,1-32;
  Gén 11,10-26; Gén 21,2-3; Gén 29,35} \bibleverse{34} hijo de Jacob,
hijo de Isaac, hijo de Abraham, el hijo de Taré, el hijo de Nacor,
\bibleverse{35} el hijo de Serug, el hijo de Reu, el hijo de Peleg, el
hijo de Eber, el hijo de Sela, \bibleverse{36} el hijo de Cainán, el
hijo de Arfaxad, el hijo de Sem, el hijo de Noé, hijo de Lamec,
\bibleverse{37} hijo de Matusalén, hijo de Enoc, hijo de Jared, hijo de
Mahalaleel, hijo de Cainán, \bibleverse{38} hijo de Enós, hijo de Set,
hijo de Adán, hijo de Dios.

\hypertarget{la-tentaciuxf3n-de-jesuxfas-como-prueba-de-mesuxedas}{%
\subsection{La tentación de Jesús como prueba de
Mesías}\label{la-tentaciuxf3n-de-jesuxfas-como-prueba-de-mesuxedas}}

\hypertarget{section-3}{%
\section{4}\label{section-3}}

\bibleverse{1} Jesús, lleno del Espíritu Santo, volvió del Jordán y fue
llevado por el Espíritu al desierto \bibleverse{2} durante cuarenta
días, siendo tentado por el diablo. No comió nada en esos días. Después,
cuando terminaron, tuvo hambre.

\bibleverse{3} El diablo le dijo: ``Si eres el Hijo de Dios, ordena que
esta piedra se convierta en pan''.

\bibleverse{4} Jesús le contestó diciendo: ``Está escrito que no sólo de
pan vivirá el hombre, sino de toda palabra de Dios''. \footnote{\textbf{4:4}
  Deuteronomio 8:3}

\bibleverse{5} El diablo, llevándolo a un monte alto, le mostró en un
momento todos los reinos del mundo. \bibleverse{6} El diablo le dijo:
``Te daré toda esta autoridad y su gloria, porque me ha sido entregada,
y la doy a quien quiero. \bibleverse{7} Por tanto, si adoras ante mí,
todo será tuyo''.

\bibleverse{8} Jesús le respondió: ``¡Quítate de encima, Satanás! Porque
está escrito: `Al Señor tu Dios adorarás y a él sólo servirás'\,''.
\footnote{\textbf{4:8} Deuteronomio 6:13}

\bibleverse{9} Lo condujo a Jerusalén, lo puso en el pináculo del templo
y le dijo: ``Si eres el Hijo de Dios, échate de aquí, \bibleverse{10}
porque está escrito, Pondrá a sus ángeles a cargo de ti, para que te
guarden;'

\bibleverse{11} y, En sus manos te llevarán, para que no tropieces con
una piedra''. \footnote{\textbf{4:11} Salmo 91:11-12}

\bibleverse{12} Respondiendo Jesús, le dijo: ``Se ha dicho que no
tentarás al Señor tu Dios''. \footnote{\textbf{4:12} Deuteronomio 6:16}

\bibleverse{13} Cuando el demonio hubo completado todas las tentaciones,
se alejó de él hasta otro momento. \footnote{\textbf{4:13} Heb 4,15}

\hypertarget{primera-apariciuxf3n-de-jesuxfas-en-galilea-su-predicaciuxf3n-y-rechazo-en-su-natal-nazaret}{%
\subsection{Primera aparición de Jesús en Galilea; su predicación y
rechazo en su natal
Nazaret}\label{primera-apariciuxf3n-de-jesuxfas-en-galilea-su-predicaciuxf3n-y-rechazo-en-su-natal-nazaret}}

\bibleverse{14} Jesús regresó con el poder del Espíritu a Galilea, y la
noticia sobre él se extendió por todos los alrededores. \bibleverse{15}
Enseñaba en sus sinagogas, siendo glorificado por todos.

\bibleverse{16} Llegó a Nazaret, donde se había criado. Entró, como era
su costumbre, en la sinagoga en el día de reposo, y se puso de pie para
leer. \bibleverse{17} Se le entregó el libro del profeta Isaías. Abrió
el libro y encontró el lugar donde estaba escrito, \bibleverse{18} ``El
Espíritu del Señor está sobre mí, porque me ha ungido para predicar la
buena nueva a los pobres. Me ha enviado a sana a los
corazones\footnote{\textbf{4:18} NU omite ``para curar a los corazones
  rotos''} rotos, para proclamar la liberación de los cautivos,
recuperar la vista de los ciegos, para liberar a los oprimidos,
\footnote{\textbf{4:18} Is 42,7} \bibleverse{19} y proclamar el año de
gracia del Señor.''\footnote{\textbf{4:19} Isaías 61:1-2} \footnote{\textbf{4:19}
  Lev 25,10}

\bibleverse{20} Cerró el libro, se lo devolvió al asistente y se sentó.
Los ojos de todos en la sinagoga estaban fijos en él. \bibleverse{21}
Comenzó a decirles: ``Hoy se ha cumplido esta Escritura ante vosotros''.

\bibleverse{22} Todos daban testimonio de él y se asombraban de las
palabras de gracia que salían de su boca, y decían: ``¿No es éste el
hijo de José?'' \footnote{\textbf{4:22} Luc 3,23}

\bibleverse{23} Les dijo: ``Seguramente me dirán este proverbio:
``¡Médico, cúrate a ti mismo! Todo lo que hemos oído hacer en Cafarnaúm,
hazlo también aquí en tu pueblo''. \bibleverse{24} Él dijo: ``De cierto
os digo que ningún profeta es aceptable en su ciudad natal. \footnote{\textbf{4:24}
  Juan 4,44} \bibleverse{25} Pero en verdad os digo que había muchas
viudas en Israel en los días de Elías, cuando el cielo estuvo cerrado
durante tres años y seis meses, cuando sobrevino una gran hambruna en
toda la tierra. \footnote{\textbf{4:25} 1Re 17,1; 1Re 17,9; 1Re 18,1}
\bibleverse{26} A ninguna de ellas fue enviado Elías, sino a Sarepta, en
la tierra de Sidón, a una mujer que era viuda. \bibleverse{27} Había
muchos leprosos en Israel en tiempos del profeta Eliseo, pero ninguno de
ellos fue limpiado, excepto Naamán, el sirio.'' \footnote{\textbf{4:27}
  2Re 5,14}

\bibleverse{28} Todos se llenaron de ira en la sinagoga al oír estas
cosas. \bibleverse{29} Se levantaron, le echaron fuera de la ciudad y le
llevaron a la cima del monte sobre el que estaba edificada su ciudad,
para arrojarle por el precipicio. \bibleverse{30} Pero él, pasando por
en medio de ellos, siguió su camino.

\hypertarget{jesuxfas-enseuxf1a-en-la-sinagoga-de-capernaum-y-sana-a-un-poseuxeddo}{%
\subsection{Jesús enseña en la sinagoga de Capernaum y sana a un
poseído}\label{jesuxfas-enseuxf1a-en-la-sinagoga-de-capernaum-y-sana-a-un-poseuxeddo}}

\bibleverse{31} Bajó a Capernaúm, una ciudad de Galilea. Les enseñaba en
sábado, \footnote{\textbf{4:31} Mat 4,13; Juan 2,12} \bibleverse{32} y
se asombraban de su enseñanza, porque su palabra era con autoridad.
\footnote{\textbf{4:32} Mat 7,28-29; Juan 7,46} \bibleverse{33} En la
sinagoga había un hombre que tenía un espíritu de demonio inmundo; y
gritaba a gran voz, \bibleverse{34} diciendo: ``¡Ah! ¿Qué tenemos que
ver contigo, Jesús de Nazaret? ¿Has venido a destruirnos? Yo sé quién
eres: el Santo de Dios''.

\bibleverse{35} Jesús le reprendió diciendo: ``¡Cállate y sal de él!''.
Cuando el demonio lo arrojó en medio de ellos, salió de él, sin hacerle
ningún daño.

\bibleverse{36} El asombro se apoderó de todos y hablaban entre sí,
diciendo: ``¿Qué es esta palabra? Porque con autoridad y poder manda a
los espíritus inmundos, y salen''. \bibleverse{37} La noticia sobre él
se difundió por todos los lugares de la región circundante.

\hypertarget{sanaciuxf3n-de-la-suegra-de-simuxf3n-pedro-y-otros-enfermos-en-capernaum}{%
\subsection{Sanación de la suegra de Simón Pedro y otros enfermos en
Capernaum}\label{sanaciuxf3n-de-la-suegra-de-simuxf3n-pedro-y-otros-enfermos-en-capernaum}}

\bibleverse{38} Se levantó de la sinagoga y entró en casa de Simón. La
suegra de Simón estaba afligida por una gran fiebre, y le rogaron que la
ayudara. \bibleverse{39} Él se puso al lado de ella, reprendió la fiebre
y la dejó. Al instante se levantó y les sirvió. \bibleverse{40} Cuando
se puso el sol, todos los que tenían algún enfermo de diversas
enfermedades se los trajeron, y él puso las manos sobre cada uno de
ellos y los curó. \bibleverse{41} También salieron demonios de muchos,
gritando y diciendo: ``¡Tú eres el Cristo, el Hijo de Dios!''
Reprendiéndolos, no les permitió hablar, porque sabían que él era el
Cristo. \footnote{\textbf{4:41} Mat 8,29; Mar 3,11-12}

\hypertarget{el-sermuxf3n-itinerante-de-jesuxfas-en-las-cercanuxedas-de-capernaum}{%
\subsection{El sermón itinerante de Jesús en las cercanías de
Capernaum}\label{el-sermuxf3n-itinerante-de-jesuxfas-en-las-cercanuxedas-de-capernaum}}

\bibleverse{42} Cuando se hizo de día, partió y se fue a un lugar
despoblado, y las multitudes lo buscaban y se acercaban a él, para que
no se alejara de ellos. \bibleverse{43} Pero él les dijo: ``Es necesario
que anuncie la buena noticia del Reino de Dios también en las demás
ciudades. Para esto he sido enviado''. \bibleverse{44} Estaba predicando
en las sinagogas de Galilea. \footnote{\textbf{4:44} Mat 4,23}

\hypertarget{el-sermuxf3n-de-jesuxfas-en-el-barco-maravilloso-viaje-de-pesca-de-peter-llamando-a-los-primeros-cuatro-discuxedpulos}{%
\subsection{El sermón de Jesús en el barco; maravilloso viaje de pesca
de Peter; Llamando a los primeros cuatro
discípulos}\label{el-sermuxf3n-de-jesuxfas-en-el-barco-maravilloso-viaje-de-pesca-de-peter-llamando-a-los-primeros-cuatro-discuxedpulos}}

\hypertarget{section-4}{%
\section{5}\label{section-4}}

\bibleverse{1} Mientras la multitud le apretaba y escuchaba la palabra
de Dios, él estaba de pie junto al lago de Genesaret. \bibleverse{2} Vio
dos barcas paradas junto al lago, pero los pescadores habían salido de
ellas y estaban lavando las redes. \bibleverse{3} Entró en una de las
barcas, que era la de Simón, y le pidió que se alejara un poco de la
tierra. Se sentó y enseñó a las multitudes desde la barca.

\bibleverse{4} Cuando terminó de hablar, le dijo a Simón: ``Rema mar
adentro y echa las redes para pescar''. \footnote{\textbf{5:4} Juan 21,6}

\bibleverse{5} Simón le respondió: ``Maestro, hemos trabajado toda la
noche y no hemos pescado nada; pero en tu palabra echaré la red''.
\bibleverse{6} Cuando hicieron esto, pescaron una gran cantidad de
peces, y su red se rompía. \bibleverse{7} Hicieron señas a sus
compañeros de la otra barca para que vinieran a ayudarlos. Vinieron y
llenaron las dos barcas, de modo que empezaron a hundirse.
\bibleverse{8} Pero Simón Pedro, al verlo, cayó de rodillas ante Jesús,
diciendo: ``Apártate de mí, porque soy un hombre pecador, Señor''.
\footnote{\textbf{5:8} Luc 18,13} \bibleverse{9} Porque estaba
asombrado, y todos los que estaban con él, de la pesca que habían hecho;
\bibleverse{10} y también Santiago y Juan, hijos de Zebedeo, que eran
compañeros de Simón. Jesús le dijo a Simón: ``No tengas miedo. A partir
de ahora cogerás gente viva''.

\bibleverse{11} Cuando llevaron sus barcas a tierra, lo dejaron todo y
le siguieron.

\hypertarget{jesuxfas-sana-a-un-leproso-y-escapa-a-la-soledad}{%
\subsection{Jesús sana a un leproso y escapa a la
soledad}\label{jesuxfas-sana-a-un-leproso-y-escapa-a-la-soledad}}

\bibleverse{12} Mientras estaba en una de las ciudades, he aquí que
había un hombre lleno de lepra. Al ver a Jesús, se postró sobre su
rostro y le rogó diciendo: ``Señor, si quieres, puedes limpiarme''.

\bibleverse{13} Extendió la mano y lo tocó, diciendo: ``Quiero. Queda
limpio''. Inmediatamente la lepra lo abandonó. \bibleverse{14} Le ordenó
que no se lo dijera a nadie: ``Pero vete y muéstrate al sacerdote, y
ofrece por tu purificación lo que ha mandado Moisés, para que les sirva
de testimonio.'' \footnote{\textbf{5:14} Lev 14,2-32}

\bibleverse{15} Pero la noticia sobre él se extendió mucho más, y se
reunieron grandes multitudes para escuchar y ser curados por él de sus
enfermedades. \bibleverse{16} Pero él se retiró al desierto y oró.
\footnote{\textbf{5:16} Mar 1,35}

\hypertarget{curaciuxf3n-de-un-paraluxedtico-jesuxfas-perdona-los-pecados}{%
\subsection{Curación de un paralítico; Jesús perdona los
pecados}\label{curaciuxf3n-de-un-paraluxedtico-jesuxfas-perdona-los-pecados}}

\bibleverse{17} Uno de esos días, estaba enseñando, y había fariseos y
maestros de la ley sentados que habían salido de todas las aldeas de
Galilea, Judea y Jerusalén. El poder del Señor estaba con él para
curarlos. \bibleverse{18} He aquí que unos hombres trajeron a un
paralítico en un catre, y trataron de traerlo para ponerlo delante de
Jesús. \bibleverse{19} Al no encontrar la manera de hacerlo entrar a
causa de la multitud, subieron a la azotea y lo hicieron bajar por las
tejas con su catre al centro, ante Jesús. \bibleverse{20} Al ver su fe,
le dijo: ``Hombre, tus pecados te son perdonados''.

\bibleverse{21} Los escribas y los fariseos se pusieron a razonar,
diciendo: ``¿Quién es éste que dice blasfemias? ¿Quién puede perdonar
los pecados, sino sólo Dios?'' \footnote{\textbf{5:21} Luc 7,49; Sal
  130,4; Is 43,25}

\bibleverse{22} Pero Jesús, percibiendo sus pensamientos, les respondió:
``¿Por qué razonáis así en vuestros corazones? \bibleverse{23} ¿Qué es
más fácil decir: ``Tus pecados te son perdonados'', o decir: ``Levántate
y anda''? \bibleverse{24} Pero para que sepáis que el Hijo del Hombre
tiene autoridad en la tierra para perdonar los pecados, dijo al
paralítico: ``Te digo que te levantes, toma tu camilla y te vete a tu
casa.'' \footnote{\textbf{5:24} Juan 5,36}

\bibleverse{25} Inmediatamente se levantó delante de ellos, tomó lo que
tenía puesto y se fue a su casa, glorificando a Dios. \bibleverse{26} El
asombro se apoderó de todos, y glorificaron a Dios. Se llenaron de
temor, diciendo: ``Hoy hemos visto cosas extrañas''.

\hypertarget{llamando-al-recaudador-de-impuestos-levi-jesuxfas-como-compauxf1ero-de-mesa-para-recaudadores-de-impuestos-y-pecadores}{%
\subsection{Llamando al recaudador de impuestos Levi; Jesús como
compañero de mesa para recaudadores de impuestos y
pecadores}\label{llamando-al-recaudador-de-impuestos-levi-jesuxfas-como-compauxf1ero-de-mesa-para-recaudadores-de-impuestos-y-pecadores}}

\bibleverse{27} Después de estas cosas, salió y vio a un recaudador de
impuestos llamado Leví, sentado en la oficina de impuestos, y le dijo:
``¡Sígueme!''

\bibleverse{28} Lo dejó todo, se levantó y le siguió. \bibleverse{29}
Leví hizo una gran fiesta para él en su casa. Había una gran multitud de
recaudadores de impuestos y otros que estaban reclinados con ellos.
\footnote{\textbf{5:29} Luc 15,1} \bibleverse{30} Sus escribas y los
fariseos murmuraban contra sus discípulos, diciendo: ``¿Por qué coméis y
bebéis con los recaudadores de impuestos y los pecadores?''

\bibleverse{31} Jesús les respondió: ``Los sanos no tienen necesidad de
médico, pero los enfermos sí. \bibleverse{32} No he venido a llamar a
los justos, sino a los pecadores, al arrepentimiento.''

\hypertarget{la-pregunta-del-ayuno-de-los-discuxedpulos-de-juan-y-los-fariseos-jesuxfas-justifica-lo-nuevo-en-su-comportamiento}{%
\subsection{La pregunta del ayuno de los discípulos de Juan y los
fariseos; Jesús justifica lo nuevo en su
comportamiento}\label{la-pregunta-del-ayuno-de-los-discuxedpulos-de-juan-y-los-fariseos-jesuxfas-justifica-lo-nuevo-en-su-comportamiento}}

\bibleverse{33} Le dijeron: ``¿Por qué los discípulos de Juan suelen
ayunar y orar, así como los discípulos de los fariseos, pero los tuyos
comen y beben?''

\bibleverse{34} Les dijo: ``¿Podéis hacer ayunar a los amigos del novio
mientras el novio está con ellos? \bibleverse{35} Pero vendrán días en
que el novio les será quitado. Entonces ayunarán en esos días''.

\bibleverse{36} También les contó una parábola. ``Nadie pone un trozo de
una prenda nueva en una prenda vieja, porque si no se romperá la nueva,
y además el trozo de la nueva no coincidirá con el de la vieja.
\bibleverse{37} Nadie pone vino nuevo en odres viejos, porque el vino
nuevo reventaría los odres, se derramaría y los odres se destruirían.
\bibleverse{38} Pero el vino nuevo debe ponerse en odres frescos, y
ambos se conservan. \bibleverse{39} Nadie que haya bebido vino viejo
desea inmediatamente el nuevo, porque dice: ``El viejo es mejor''.''

\hypertarget{el-arranco-de-espigas-de-los-discuxedpulos-en-suxe1bado-la-primera-disputa-de-jesuxfas-con-los-fariseos-sobre-la-santificaciuxf3n-del-duxeda-de-reposo}{%
\subsection{El arranco de espigas de los discípulos en sábado; La
primera disputa de Jesús con los fariseos sobre la santificación del día
de
reposo}\label{el-arranco-de-espigas-de-los-discuxedpulos-en-suxe1bado-la-primera-disputa-de-jesuxfas-con-los-fariseos-sobre-la-santificaciuxf3n-del-duxeda-de-reposo}}

\hypertarget{section-5}{%
\section{6}\label{section-5}}

\bibleverse{1} Y aconteció, que un día de reposo iba por los campos de
trigo. Sus discípulos arrancaban las espigas y comían, frotándolas en
sus manos. \footnote{\textbf{6:1} Luc 13,10-17; Luc 14,1-6}
\bibleverse{2} Pero algunos de los fariseos les dijeron: ``¿Por qué
hacéis lo que no es lícito hacer en día de reposo?''

\bibleverse{3} Jesús, respondiéndoles, dijo: ``¿No habéis leído lo que
hizo David cuando tuvo hambre, él y los que estaban con él, \footnote{\textbf{6:3}
  1Sam 21,6} \bibleverse{4} cómo entró en la casa de Dios, y tomó y
comió el pan de la feria, y dio también a los que estaban con él, lo que
no es lícito comer sino a los sacerdotes solos?'' \footnote{\textbf{6:4}
  Lev 24,9} \bibleverse{5} Él les dijo: ``El Hijo del Hombre es el señor
del sábado''.

\hypertarget{sanaciuxf3n-del-hombre-con-el-brazo-paralizado-en-suxe1bado-el-segundo-argumento-sobre-la-observancia-del-suxe1bado}{%
\subsection{Sanación del hombre con el brazo paralizado en sábado; el
segundo argumento sobre la observancia del
sábado}\label{sanaciuxf3n-del-hombre-con-el-brazo-paralizado-en-suxe1bado-el-segundo-argumento-sobre-la-observancia-del-suxe1bado}}

\bibleverse{6} Sucedió también otro sábado que entró en la sinagoga y
enseñó. Había allí un hombre que tenía la mano derecha seca.
\bibleverse{7} Los escribas y los fariseos le vigilaban para ver si
sanaba en sábado, a fin de encontrar una acusación contra él.
\bibleverse{8} Pero él conocía sus pensamientos, y dijo al hombre que
tenía la mano seca: ``Levántate y ponte en medio.'' Se levantó y se puso
en pie. \bibleverse{9} Entonces Jesús les dijo: ``Os voy a preguntar una
cosa: ¿Es lícito en sábado hacer el bien, o hacer el mal? ¿Salvar una
vida, o matar?'' \bibleverse{10} Miró a todos y le dijo al hombre:
``Extiende tu mano''. Lo hizo, y su mano quedó tan sana como la otra.
\bibleverse{11} Pero ellos, llenos de ira, hablaban entre sí sobre lo
que podrían hacer a Jesús.

\hypertarget{llamadas-y-nombres-de-los-doce-apuxf3stoles-afluencia-de-personas-muchas-curaciones}{%
\subsection{Llamadas y nombres de los doce apóstoles; Afluencia de
personas; muchas
curaciones}\label{llamadas-y-nombres-de-los-doce-apuxf3stoles-afluencia-de-personas-muchas-curaciones}}

\bibleverse{12} En esos días, salió al monte a orar, y pasó toda la
noche orando a Dios. \footnote{\textbf{6:12} Mar 1,35} \bibleverse{13}
Cuando se hizo de día, llamó a sus discípulos, y de entre ellos eligió a
doce, a los que también llamó apóstoles \footnote{\textbf{6:13} Mat
  10,2-4; Hech 1,13} \bibleverse{14} Simón, al que también llamó Pedro;
Andrés, su hermano; Santiago; Juan; Felipe; Bartolomé; \bibleverse{15}
Mateo; Tomás; Santiago, hijo de Alfeo; Simón, al que llamaban el Zelote;
\bibleverse{16} Judas, hijo de Santiago; y Judas Iscariote, que también
se hizo traidor.

\bibleverse{17} Bajó con ellos y se puso en un lugar llano, con una
multitud de sus discípulos y un gran número de la gente de toda Judea y
Jerusalén y de la costa de Tiro y Sidón, que venían a escucharle y a ser
curados de sus enfermedades, \bibleverse{18} así como los que estaban
turbados por espíritus inmundos; y eran curados. \bibleverse{19} Toda la
multitud procuraba tocarle, porque salía de él poder y los sanaba a
todos.

\hypertarget{el-sermuxf3n-del-monte}{%
\subsection{El sermón del monte}\label{el-sermuxf3n-del-monte}}

\bibleverse{20} Levantó los ojos hacia sus discípulos y dijo``Benditos
seáis los pobres, porque vuestro el Reino de Dios. \footnote{\textbf{6:20}
  Sant 2,5} \bibleverse{21} Dichosos los que ahora tienen hambre, porque
seréis saciados. Benditos seáis los que lloráis ahora, porque te reirás.
\footnote{\textbf{6:21} Apoc 7,16-17} \bibleverse{22} Bienaventurados
seréis cuando los hombres os odien, y cuando os excluyan y se burlen de
vosotros, y desechen vuestro nombre como malo, por causa del Hijo del
Hombre. \footnote{\textbf{6:22} Juan 15,18-19} \bibleverse{23} Alégrate
en ese día y da saltos de alegría, porque he aquí que tu recompensa es
grande en el cielo, ya que sus padres hicieron lo mismo con los
profetas. \bibleverse{24} ``Pero ¡ay de vosotros, los ricos! Porque has
recibido tu consuelo. \footnote{\textbf{6:24} Mat 19,23; Sant 5,1}
\bibleverse{25} Ay de ti, que estás lleno ahora, porque tendrás hambre.
Ay de ti que te ríes ahora, porque te lamentarás y llorarás.
\bibleverse{26} Ay,\footnote{\textbf{6:26} TR añade ``a ti''} cuando los
\footnote{\textbf{6:26} TR añade ``todos''} hombres hablan bien de ti,
porque sus padres hicieron lo mismo con los falsos profetas. \footnote{\textbf{6:26}
  Miq 2,11}

\hypertarget{mandamiento-de-amar-al-enemigo-renuncia-a-represalias}{%
\subsection{Mandamiento de amar al enemigo; Renuncia a
represalias}\label{mandamiento-de-amar-al-enemigo-renuncia-a-represalias}}

\bibleverse{27} ``Pero yo os digo a vosotros que escucháis: amad a
vuestros enemigos, haced el bien a los que os odian, \bibleverse{28}
bendecid a los que os maldicen y orad por los que os maltratan.
\footnote{\textbf{6:28} 1Cor 4,12} \bibleverse{29} Al que te golpee en
la mejilla, ofrécele también la otra; y al que te quite el manto, no le
quites también la túnica. \bibleverse{30} Da a todo el que te pida, y no
le pidas al que te quita tus bienes que te los devuelva.

\bibleverse{31} ``Como quieras que la gente te haga a ti, haz
exactamente lo mismo con ellos. \footnote{\textbf{6:31} Mat 7,12}

\bibleverse{32} ``Si amas a los que te aman, ¿qué mérito tienes? Porque
también los pecadores aman a los que los aman. \bibleverse{33} Si hacéis
bien a los que os hacen bien, ¿qué mérito tenéis? Porque también los
pecadores hacen lo mismo. \bibleverse{34} Si prestáis a aquellos de
quienes esperáis recibir, ¿qué mérito tenéis? Incluso los pecadores
prestan a los pecadores, para recibir lo mismo. \footnote{\textbf{6:34}
  Lev 25,35-36} \bibleverse{35} Pero amad a vuestros enemigos, haced el
bien y prestad sin esperar nada a cambio; y vuestra recompensa será
grande, y seréis hijos del Altísimo, porque él es bondadoso con los
ingratos y los malos. \bibleverse{36} ``Por lo tanto, sean
misericordiosos, así como tu Padre es también misericordioso.
\footnote{\textbf{6:36} Mat 6,14}

\hypertarget{advertencia-contra-el-juicio-y-la-hipuxf3crita-voluntad-de-mejorar}{%
\subsection{Advertencia contra el juicio y la hipócrita voluntad de
mejorar}\label{advertencia-contra-el-juicio-y-la-hipuxf3crita-voluntad-de-mejorar}}

\bibleverse{37} No juzgues, y no serás juzgado. No condenes, y no serás
condenado. Libérate, y serás liberado.

\bibleverse{38} ``Dad, y se os dará; medida buena, apretada, remecida y
rebosante, se os dará.\footnote{\textbf{6:38} literalmente, en su seno.}
Porque con la misma medida que midan se les devolverá''. \footnote{\textbf{6:38}
  Mar 4,24}

\bibleverse{39} Les dijo una parábola. ``¿Puede el ciego guiar al ciego?
¿No caerán ambos en un pozo? \footnote{\textbf{6:39} Mat 15,14}
\bibleverse{40} El discípulo no está por encima de su maestro, pero todo
el mundo, cuando esté completamente formado, será como su maestro.
\footnote{\textbf{6:40} Mat 10,24-25; Juan 15,20} \bibleverse{41} ¿Por
qué ves la paja que está en el ojo de tu hermano, pero no consideras la
viga que está en tu propio ojo? \bibleverse{42} ¿Cómo puedes decirle a
tu hermano: ``Hermano, déjame quitarte la paja que tienes en el ojo'',
cuando tú mismo no ves la viga que tienes en tu propio ojo? ¡Hipócrita!
Primero quita la viga de tu propio ojo, y entonces podrás ver con
claridad para quitar la paja que está en el ojo de tu hermano.

\hypertarget{tanto-la-obediencia-de-la-fe-como-la-incredulidad-de-la-gente-viene-del-corazuxf3n-como-el-fruto-de-la-especie-del-uxe1rbol}{%
\subsection{Tanto la obediencia de la fe como la incredulidad de la
gente viene del corazón, como el fruto de la especie del
árbol}\label{tanto-la-obediencia-de-la-fe-como-la-incredulidad-de-la-gente-viene-del-corazuxf3n-como-el-fruto-de-la-especie-del-uxe1rbol}}

\bibleverse{43} ``Porque no hay árbol bueno que produzca frutos
podridos, ni árbol podrido que produzca frutos buenos. \bibleverse{44}
Porque cada árbol se conoce por su propio fruto. Porque no se recogen
higos de los espinos, ni se recogen uvas de las zarzas. \bibleverse{45}
El hombre bueno del buen tesoro de su corazón saca lo bueno, y el hombre
malo del mal tesoro de su corazón saca lo malo, porque de la abundancia
del corazón habla su boca.

\hypertarget{sea-un-hacedor-de-la-palabra-no-solo-un-oyente}{%
\subsection{Sea un hacedor de la palabra, no solo un
oyente}\label{sea-un-hacedor-de-la-palabra-no-solo-un-oyente}}

\bibleverse{46} ``¿Por qué me llamáis ``Señor, Señor'' y no hacéis lo
que yo digo? \footnote{\textbf{6:46} Mal 1,6} \bibleverse{47} Todo el
que viene a mí, y escucha mis palabras y las pone en práctica, os
mostraré a quién se parece. \bibleverse{48} Es como un hombre que
construye una casa, que cavó y profundizó y puso los cimientos sobre la
roca. Cuando se produjo una inundación, la corriente rompió contra esa
casa, y no pudo sacudirla, porque estaba fundada sobre la roca.
\bibleverse{49} Peroel que oye y no hace, es como un hombre que
construyó una casa sobre la tierra sin cimientos, contra la cual rompió
la corriente, y enseguida cayó; y la ruina de aquella casa fue grande.''

\hypertarget{sanaciuxf3n-del-siervo-del-centuriuxf3n-de-capernaum}{%
\subsection{Sanación del siervo del centurión de
Capernaum}\label{sanaciuxf3n-del-siervo-del-centuriuxf3n-de-capernaum}}

\hypertarget{section-6}{%
\section{7}\label{section-6}}

\bibleverse{1} Cuando terminó de hablar a la gente, entró en Capernaum.
\bibleverse{2} El siervo de un centurión, que le era muy querido, estaba
enfermo y a punto de morir. \bibleverse{3} Cuando oyó hablar de Jesús,
le envió a los ancianos de los judíos, pidiéndole que viniera a sanar a
su siervo. \bibleverse{4} Cuando llegaron a Jesús, le rogaron
encarecidamente, diciendo: ``Es digno de que hagas esto por él,
\bibleverse{5} porque ama a nuestra nación y nos ha construido nuestra
sinagoga.'' \bibleverse{6} Jesús fue con ellos. Cuando ya no estaba
lejos de la casa, el centurión envió a sus amigos a decirle: ``Señor, no
te preocupes, porque no soy digno de que entres bajo mi techo.
\bibleverse{7} Por eso ni siquiera me he considerado digno de venir a
ti; pero di la palabra, y mi criado quedará sano. \bibleverse{8} Porque
también yo soy un hombre puesto bajo autoridad, que tiene bajo su mando
soldados. A éste le digo: ``Ve'', y va; a otro: ``Ven'', y viene; y a mi
siervo: ``Haz esto'', y lo hace''.

\bibleverse{9} Cuando Jesús oyó estas cosas, se maravilló de él y,
volviéndose, dijo a la multitud que le seguía: ``Os digo que no he
encontrado una fe tan grande, ni siquiera en Israel.'' \bibleverse{10}
Los enviados, al volver a la casa, encontraron que el siervo que había
estado enfermo estaba bien.

\hypertarget{criar-al-joven-en-nain}{%
\subsection{Criar al joven en Nain}\label{criar-al-joven-en-nain}}

\bibleverse{11} Poco después, fue a una ciudad llamada Naín. Muchos de
sus discípulos, junto con una gran multitud, iban con él.
\bibleverse{12} Cuando se acercó a la puerta de la ciudad, he aquí que
sacaban a un muerto, \footnote{\textbf{7:12} La frase ``unigénito''
  proviene de la palabra griega ``\greek{μονογενη}'', que a veces se
  traduce como ``unigénito'' o ``único''.} hijo único de su madre, que
era viuda. La acompañaba mucha gente de la ciudad. \footnote{\textbf{7:12}
  1Re 17,17} \bibleverse{13} Al verla, el Señor se compadeció de ella y
le dijo: ``No llores''. \bibleverse{14} Se acercó y tocó el féretro, y
los portadores se detuvieron. Dijo: ``Joven, te digo que te levantes''.
\footnote{\textbf{7:14} Mar 5,41} \bibleverse{15} El que estaba muerto
se sentó y empezó a hablar. Luego se lo entregó a su madre. \footnote{\textbf{7:15}
  1Re 17,23; 2Re 4,36}

\bibleverse{16} El temor se apoderó de todos, y glorificaron a Dios,
diciendo: ``¡Ha surgido un gran profeta entre nosotros!'' y ``¡Dios ha
visitado a su pueblo!'' \footnote{\textbf{7:16} Luc 1,68; Mat 16,14}
\bibleverse{17} Esta noticia se difundió sobre él en toda Judea y en
toda la región circundante.

\hypertarget{embajada-de-juan-el-bautista-la-respuesta-y-el-testimonio-de-jesuxfas-sobre-juan}{%
\subsection{Embajada de Juan el Bautista; La respuesta y el testimonio
de Jesús sobre
Juan}\label{embajada-de-juan-el-bautista-la-respuesta-y-el-testimonio-de-jesuxfas-sobre-juan}}

\bibleverse{18} Los discípulos de Juan le contaron todas estas cosas.
\bibleverse{19} Juan, llamando a dos de sus discípulos, los envió a
Jesús, diciendo: ``¿Eres tú el que viene, o debemos buscar a otro?''
\bibleverse{20} Cuando los hombres se acercaron a él, dijeron: ``Juan el
Bautista nos ha enviado a ti, diciendo: ``¿Eres tú el que viene, o
debemos buscar a otro?''

\bibleverse{21} En aquella hora curó a muchos de enfermedades y plagas y
espíritus malignos; y a muchos ciegos les dio la vista. \bibleverse{22}
Jesús les respondió: ``Id y contad a Juan lo que habéis visto y oído:
que los ciegos ven, los cojos andan, los leprosos quedan limpios, los
sordos oyen, los muertos resucitan y a los pobres se les anuncia la
buena nueva. \bibleverse{23} Dichoso el que no encuentra en mí ocasión
de tropezar''.

\bibleverse{24} Cuando los mensajeros de Juan se marcharon, comenzó a
decir a las multitudes sobre Juan: ``¿Qué salisteis a ver al desierto?
¿Una caña agitada por el viento? \bibleverse{25} Pero, ¿qué salisteis a
ver? ¿A un hombre vestido con ropas finas? He aquí que los que se visten
de forma elegante y viven con deleites están en las cortes de los reyes.
\bibleverse{26} Pero, ¿qué salisteis a ver? ¿A un profeta? Sí, os digo,
y mucho más que un profeta. \footnote{\textbf{7:26} Luc 1,76}
\bibleverse{27} Este es aquel de quien está escrito, `He aquí que envío
a mi mensajero ante tu rostro, que te preparará el camino delante de
ti.'\footnote{\textbf{7:27} Malaquías 3:1}

\bibleverse{28} ``Porque os digo que entre los nacidos de mujer no hay
mayor profeta que Juan el Bautista; pero el más pequeño en el Reino de
Dios es mayor que él.'' \footnote{\textbf{7:28} Luc 1,15}

\bibleverse{29} Al oír esto, todo el pueblo y los recaudadores de
impuestos declararon que Dios era justo, pues habían sido bautizados con
el bautismo de Juan. \footnote{\textbf{7:29} Luc 3,7; Luc 3,12; Mat
  21,32} \bibleverse{30} Pero los fariseos y los letrados rechazaron el
consejo de Dios, no siendo ellos mismos bautizados por él. \footnote{\textbf{7:30}
  Hech 13,46}

\bibleverse{31} ``\footnote{\textbf{7:31} TR añade ``Pero el Señor
  dijo''} ¿Con qué debo comparar a la gente de esta generación? ¿A qué
se parecen? \bibleverse{32} Son como niños que se sientan en el mercado
y se llaman unos a otros, diciendo: `Te cantamos, y no bailaste.
Nosotros nos lamentamos, y vosotros no llorasteis'. \bibleverse{33}
Porque Juan el Bautista no vino ni a comer pan ni a beber vino, y
vosotros decís: `Tiene un demonio'. \bibleverse{34} El Hijo del Hombre
ha venido comiendo y bebiendo, y vosotros decís: `He aquí un comilón y
un borracho, amigo de recaudadores y pecadores.' \footnote{\textbf{7:34}
  Luc 15,2} \bibleverse{35} La sabiduría es justificada por todos sus
hijos''. \footnote{\textbf{7:35} 1Cor 1,24-30}

\hypertarget{la-unciuxf3n-de-jesuxfas-por-la-gran-pecadora}{%
\subsection{La unción de Jesús por la gran
pecadora}\label{la-unciuxf3n-de-jesuxfas-por-la-gran-pecadora}}

\bibleverse{36} Uno de los fariseos le invitó a comer con él. Entró en
la casa del fariseo y se sentó a la mesa. \footnote{\textbf{7:36} Luc
  11,37} \bibleverse{37} He aquí que una mujer pecadora de la ciudad, al
saber que él estaba reclinado en casa del fariseo, trajo un frasco de
alabastro con ungüento. \footnote{\textbf{7:37} Mar 14,3-9}
\bibleverse{38} Se puso detrás, a sus pies, llorando, y comenzó a
mojarle los pies con sus lágrimas, y se los secó con los cabellos de su
cabeza, le besó los pies y se los untó con el ungüento. \bibleverse{39}
Al verla, el fariseo que le había invitado se dijo: ``Este hombre, si
fuera profeta, se habría dado cuenta de quién y qué clase de mujer es la
que le toca, que es una pecadora.''

\bibleverse{40} Jesús le respondió: ``Simón, tengo algo que decirte''.
Él dijo: ``Maestro, dígalo''.

\bibleverse{41} ``Un prestamista tenía dos deudores. Uno debía
quinientos denarios y el otro cincuenta. \bibleverse{42} Como no podían
pagar, les perdonó a los dos. ¿Cuál de ellos lo amará más?''

\bibleverse{43} Simón respondió: ``Aquel, supongo, al que más perdonó''.
Le dijo: ``Has juzgado correctamente''. \bibleverse{44} Volviéndose a la
mujer, dijo a Simón: ``¿Ves a esta mujer? Entré en tu casa y no me diste
agua para mis pies, pero ella ha mojado mis pies con sus lágrimas y los
ha enjugado con el pelo de su cabeza. \footnote{\textbf{7:44} Gén 18,4}
\bibleverse{45} No me diste ningún beso, pero ella, desde que entré, no
ha dejado de besar mis pies. \footnote{\textbf{7:45} Rom 16,16}
\bibleverse{46} Tú no ungiste mi cabeza con aceite, pero ella ha ungido
mis pies con ungüento. \bibleverse{47} Por eso os digo que sus pecados,
que son muchos, le han sido perdonados, porque ha amado mucho. Pero a
quien se le perdona poco, ama poco''. \bibleverse{48} Y le dijo: ``Tus
pecados están perdonados''.

\bibleverse{49} Los que se sentaban a la mesa con él empezaron a
decirse: ``¿Quién es éste que hasta perdona los pecados?'' \footnote{\textbf{7:49}
  Luc 5,21}

\bibleverse{50} Le dijo a la mujer: ``Tu fe te ha salvado. Ve en paz''.
\footnote{\textbf{7:50} Luc 8,48; Luc 17,19; Luc 18,42}

\hypertarget{el-compauxf1ero-constante-de-jesuxfas-en-sus-andanzas-las-sirvientas-galileas}{%
\subsection{El compañero constante de Jesús en sus andanzas; las
sirvientas
galileas}\label{el-compauxf1ero-constante-de-jesuxfas-en-sus-andanzas-las-sirvientas-galileas}}

\hypertarget{section-7}{%
\section{8}\label{section-7}}

\bibleverse{1} Poco después, recorrió ciudades y aldeas, predicando y
llevando la buena noticia del Reino de Dios. Con él iban los doce,
\bibleverse{2} y algunas mujeres que habían sido curadas de espíritus
malignos y enfermedades: María, que se llamaba Magdalena, de la que
habían salido siete demonios; \footnote{\textbf{8:2} Mar 15,40-41; Mar
  16,9} \bibleverse{3} y Juana, mujer de Chuzas, mayordomo de Herodes;
Susana, y muchas otras que les\footnote{\textbf{8:3} TR lee ``él'' en
  lugar de ``ellos''} servían de sus bienes.

\hypertarget{paruxe1bola-del-sembrador-y-el-campo-de-cuatro-tipos}{%
\subsection{Parábola del sembrador y el campo de cuatro
tipos}\label{paruxe1bola-del-sembrador-y-el-campo-de-cuatro-tipos}}

\bibleverse{4} Cuando se reunió una gran multitud y acudió a él gente de
todas las ciudades, habló con una parábola \bibleverse{5} ``El
agricultor salió a sembrar su semilla. Al sembrar, una parte cayó en el
camino y fue pisoteada, y las aves del cielo la devoraron.
\bibleverse{6} Otra semilla cayó en la roca, y en cuanto creció, se
secó, porque no tenía humedad. \bibleverse{7} Otra cayó en medio de los
espinos, y los espinos crecieron con ella y la ahogaron. \bibleverse{8}
Otra cayó en tierra buena y creció y produjo cien veces más fruto''.
Mientras decía estas cosas, gritó: ``El que tenga oídos para oír, que
oiga''.

\hypertarget{significado-y-propuxf3sito-de-las-paruxe1bolas-interpretaciuxf3n-de-la-paruxe1bola-del-sembrador}{%
\subsection{Significado y propósito de las parábolas; Interpretación de
la parábola del
sembrador}\label{significado-y-propuxf3sito-de-las-paruxe1bolas-interpretaciuxf3n-de-la-paruxe1bola-del-sembrador}}

\bibleverse{9} Entonces sus discípulos le preguntaron: ``¿Qué significa
esta parábola?''

\bibleverse{10} Dijo: ``A vosotros se os ha dado a conocer los misterios
del Reino de Dios, pero a los demás se les ha dado en parábolas, para
que ``viendo no vean y oyendo no entiendan''. \footnote{\textbf{8:10}
  Isaías 6:9} \footnote{\textbf{8:10} Is 6,9-10}

\bibleverse{11} ``La parábola es ésta: La semilla es la palabra de Dios.
\bibleverse{12} Los que están en el camino son los que oyen; luego viene
el diablo y les quita la palabra del corazón, para que no crean y se
salven. \bibleverse{13} Los que están sobre la roca son los que, al oír,
reciben la palabra con alegría; pero éstos no tienen raíz. Creen por un
tiempo, y luego caen en el tiempo de la tentación. \bibleverse{14} Los
que cayeron entre los espinos, éstos son los que han oído, y al seguir
su camino son ahogados por los afanes, las riquezas y los placeres de la
vida; y no dan fruto hasta la madurez. \bibleverse{15} Los que están en
la buena tierra, éstos son los que con corazón honesto y bueno, habiendo
oído la palabra, la retienen firmemente y producen fruto con
perseverancia. \footnote{\textbf{8:15} Hech 16,14}

\bibleverse{16} ``Nadie, cuando ha encendido una lámpara, la cubre con
un recipiente o la pone debajo de la cama, sino que la pone sobre un
soporte, para que los que entren puedan ver la luz. \footnote{\textbf{8:16}
  Mat 5,15} \bibleverse{17} Porque no hay nada oculto que no se revele,
ni nada secreto que no se conozca y salga a la luz. \footnote{\textbf{8:17}
  Mat 10,26; 1Cor 4,5} \bibleverse{18} Tened, pues, cuidado con lo que
oís. Porque al que tiene, se le dará; y al que no tiene, se le quitará
hasta lo que cree tener.'' \footnote{\textbf{8:18} Mat 25,29}

\hypertarget{los-verdaderos-parientes-de-jesuxfas}{%
\subsection{Los verdaderos parientes de
Jesús}\label{los-verdaderos-parientes-de-jesuxfas}}

\bibleverse{19} Su madre y sus hermanos se acercaron a él, pero no
podían acercarse por la multitud. \bibleverse{20} Algunas personas le
dijeron: ``Tu madre y tus hermanos están fuera, deseando verte''.

\bibleverse{21} Pero él les respondió: ``Mi madre y mis hermanos son los
que oyen la palabra de Dios y la cumplen.''

\hypertarget{jesuxfas-apacigua-la-tormenta-del-mar}{%
\subsection{Jesús apacigua la tormenta del
mar}\label{jesuxfas-apacigua-la-tormenta-del-mar}}

\bibleverse{22} Uno de esos días, entró en una barca, él y sus
discípulos, y les dijo: ``Vamos al otro lado del lago''. Y se
embarcaron. \bibleverse{23} Pero mientras navegaban, se quedó dormido.
Una tormenta de viento se abatió sobre el lago, y peligraban, pues se
estaban anegando de agua. \bibleverse{24} Se acercaron a él y lo
despertaron, diciendo: ``¡Maestro, Maestro, nos estamos muriendo!''. Él
se despertó y reprendió al viento y a la furia del agua; entonces
cesaron, y se calmó. \footnote{\textbf{8:24} Ver Salmo 107:29}
\bibleverse{25} Les dijo: ``¿Dónde está vuestra fe?''. Atemorizados, se
maravillaron, diciéndose unos a otros: ``¿Quién es éste, pues, que manda
incluso a los vientos y a las aguas, y le obedecen?''

\hypertarget{jesuxfas-sana-a-los-poseuxeddos-en-la-tierra-de-los-gergesen}{%
\subsection{Jesús sana a los poseídos en la tierra de los
Gergesen}\label{jesuxfas-sana-a-los-poseuxeddos-en-la-tierra-de-los-gergesen}}

\bibleverse{26} Luego llegaron al país de los gadarenos, que está frente
a Galilea. \bibleverse{27} Cuando Jesús desembarcó, le salió al
encuentro un hombre de la ciudad que tenía demonios desde hacía mucho
tiempo. No llevaba ropa y no vivía en una casa, sino en los sepulcros.
\bibleverse{28} Al ver a Jesús, dio un grito y se postró ante él, y con
gran voz dijo: ``¿Qué tengo que ver contigo, Jesús, Hijo del Dios
Altísimo? Te ruego que no me atormentes''. \bibleverse{29} Porque Jesús
ordenaba al espíritu inmundo que saliera del hombre. Porque el espíritu
inmundo se había apoderado muchas veces del hombre. Lo tenían vigilado y
atado con cadenas y grilletes. Al romper las ataduras, el demonio lo
condujo al desierto.

\bibleverse{30} Jesús le preguntó: ``¿Cuál es tu nombre?'' Dijo:
``Legión'', porque muchos demonios habían entrado en él. \bibleverse{31}
Le rogaron que no les ordenara ir al abismo.

\bibleverse{32} Había allí una piara de muchos cerdos alimentándose en
el monte, y le rogaron que les permitiera entrar en ellos. Entonces se
lo permitió. \bibleverse{33} Los demonios salieron del hombre y entraron
en los cerdos, y la piara se precipitó por un despeñadero al lago y se
ahogó.

\bibleverse{34} Cuando los que les daban de comer vieron lo que había
sucedido, huyeron y lo contaron en la ciudad y en el campo.

\bibleverse{35} La gente salió a ver qué había pasado. Se acercaron a
Jesús y encontraron al hombre del que habían salido los demonios,
sentado a los pies de Jesús, vestido y en su sano juicio; y se
asustaron. \bibleverse{36} Los que lo vieron les contaron cómo había
quedado curado el que había sido poseído por los demonios.
\bibleverse{37} Toda la gente de los alrededores de los gadarenos le
pedía que se alejara de ellos, porque tenían mucho miedo. Entonces él
entró en la barca y regresó. \bibleverse{38} Pero el hombre del que
habían salido los demonios le rogó que se fuera con él, pero Jesús lo
despidió diciendo: \bibleverse{39} ``Vuelve a tu casa y anuncia las
grandes cosas que Dios ha hecho contigo.'' Él se fue, proclamando por
toda la ciudad las grandes cosas que Jesús había hecho por él.

\hypertarget{jesuxfas-sana-a-la-mujer-ensangrentada-y-despierta-a-la-hija-de-jairo}{%
\subsection{Jesús sana a la mujer ensangrentada y despierta a la hija de
Jairo}\label{jesuxfas-sana-a-la-mujer-ensangrentada-y-despierta-a-la-hija-de-jairo}}

\bibleverse{40} Cuando Jesús regresó, la multitud le dio la bienvenida,
pues todos le esperaban. \bibleverse{41} Llegó un hombre llamado Jairo.
Era un jefe de la sinagoga. Se postró a los pies de Jesús y le rogó que
entrara en su casa, \bibleverse{42} porque tenía una \footnote{\textbf{8:42}
  La frase ``unigénito'' proviene de la palabra griega
  ``\greek{μονογενη}'', que a veces se traduce como ``unigénito'' o
  ``único''.} hija única, de unos doce años, que se estaba muriendo.
Pero mientras iba, las multitudes le apretaban. \bibleverse{43} Una
mujer que tenía un flujo de sangre desde hacía doce años, que había
gastado todo su sustento en médicos y no podía ser curada por ninguno,
\bibleverse{44} se acercó por detrás de él y tocó los flecos\footnote{\textbf{8:44}
  o, borla} de su manto. Al instante, el flujo de su sangre se detuvo.

\bibleverse{45} Jesús dijo: ``¿Quién me ha tocado?'' Cuando todos lo
negaron, Pedro y los que estaban con él dijeron: ``Maestro, las
multitudes te apretujan y empujan, y tú dices: ``¿Quién me ha
tocado?''\,''.

\bibleverse{46} Pero Jesús dijo: ``Alguien me ha tocado, porque he
percibido que el poder ha salido de mí''. \bibleverse{47} La mujer, al
ver que no se ocultaba, se acercó temblando y, postrándose ante él, le
declaró en presencia de todo el pueblo la razón por la que le había
tocado y cómo había quedado curada al instante. \bibleverse{48} Él le
dijo: ``Hija, anímate. Tu fe te ha sanado. Vete en paz''. \footnote{\textbf{8:48}
  Luc 7,50}

\bibleverse{49} Mientras aún hablaba, se acercó uno de la casa del jefe
de la sinagoga, diciéndole: ``Tu hija ha muerto. No molestes al
Maestro''.

\bibleverse{50} Pero Jesús, al oírlo, le respondió: ``No temas. Sólo
cree, y quedará sanada''.

\bibleverse{51} Cuando llegó a la casa, no dejó entrar a nadie, excepto
a Pedro, Juan, Santiago, el padre de la niña y su madre. \footnote{\textbf{8:51}
  Mat 17,1} \bibleverse{52} Todos lloraban y la lloraban, pero él dijo:
``No lloréis. No está muerta, sino que duerme''. \footnote{\textbf{8:52}
  Luc 7,13}

\bibleverse{53} Se burlaban de él, sabiendo que estaba muerta.
\bibleverse{54} Pero él los echó a todos fuera, y tomándola de la mano,
la llamó diciendo: ``¡Niña, levántate!'' \bibleverse{55} El espíritu de
la niña volvió y se levantó enseguida. Mandó que le dieran de comer.
\bibleverse{56} Sus padres se asombraron, pero él les ordenó que no
dijeran a nadie lo que había sucedido. \footnote{\textbf{8:56} Luc 5,14;
  Mar 7,36}

\hypertarget{enviar-a-los-doce-discuxedpulos-y-darles-instrucciones}{%
\subsection{Enviar a los doce discípulos y darles
instrucciones}\label{enviar-a-los-doce-discuxedpulos-y-darles-instrucciones}}

\hypertarget{section-8}{%
\section{9}\label{section-8}}

\bibleverse{1} Convocó a los doce\footnote{\textbf{9:1} TR dice ``sus
  doce discípulos'' en lugar de ``los doce''} y les dio poder y
autoridad sobre todos los demonios y para curar enfermedades.
\footnote{\textbf{9:1} Luc 10,1-12} \bibleverse{2} Los envió a predicar
el Reino de Dios y a curar a los enfermos. \bibleverse{3} Les dijo: ``No
tomen nada para su viaje: ni bastones, ni cartera, ni pan, ni dinero. Ni
tengáis dos túnicas cada uno. \bibleverse{4} En cualquier casa en la que
entréis, quedaos allí, y salid de allí. \bibleverse{5} A todos los que
no os reciban, cuando salgáis de esa ciudad, sacudid hasta el polvo de
vuestros pies como testimonio contra ellos.''

\bibleverse{6} Partieron y recorrieron las aldeas, predicando la Buena
Nueva y sanando por todas partes.

\hypertarget{conclusiuxf3n-de-la-obra-de-jesuxfas-en-galilea}{%
\subsection{Conclusión de la obra de Jesús en
Galilea}\label{conclusiuxf3n-de-la-obra-de-jesuxfas-en-galilea}}

\bibleverse{7} El tetrarca Herodes se enteró de todo lo que había hecho,
y se quedó muy perplejo, porque unos decían que Juan había resucitado de
entre los muertos, \bibleverse{8} y otros que Elías había aparecido, y
otros que uno de los antiguos profetas había resucitado. \bibleverse{9}
Herodes dijo: ``Yo decapité a Juan, pero ¿quién es éste del que oigo
tales cosas?'' Buscó verlo. \footnote{\textbf{9:9} Luc 23,8}

\hypertarget{regreso-de-los-apuxf3stoles-jesuxfas-se-retira-a-la-soledad-alimentando-a-los-cinco-mil}{%
\subsection{Regreso de los apóstoles; Jesús se retira a la soledad;
Alimentando a los cinco
mil}\label{regreso-de-los-apuxf3stoles-jesuxfas-se-retira-a-la-soledad-alimentando-a-los-cinco-mil}}

\bibleverse{10} Los apóstoles, al regresar, le contaron lo que habían
hecho. Los tomó y se retiró a una región desierta de \footnote{\textbf{9:10}
  NU omite ``una región desértica de''.} una ciudad llamada Betsaida.
\bibleverse{11} Pero las multitudes, al darse cuenta, le siguieron. Él
los acogió, les habló del Reino de Dios y curó a los que necesitaban
curación. \bibleverse{12} Empezaba a declinar el día, y los doce se
acercaron y le dijeron: ``Despide a la multitud para que vaya a las
aldeas y granjas de los alrededores y se aloje y consiga comida, porque
estamos aquí en un lugar desierto.''

\bibleverse{13} Pero él les dijo: ``Dadles vosotros de comer''. Dijeron:
``No tenemos más que cinco panes y dos peces, si no vamos a comprar
comida para toda esta gente.'' \bibleverse{14} Porque eran unos cinco
mil hombres. Dijo a sus discípulos: ``Haced que se sienten en grupos de
unos cincuenta cada uno''. \bibleverse{15} Así lo hicieron, y los hizo
sentar a todos. \bibleverse{16} Tomó los cinco panes y los dos peces y,
mirando al cielo, los bendijo, los partió y los dio a los discípulos
para que los pusieran delante de la multitud. \bibleverse{17} Comieron y
se saciaron. Recogieron doce cestas con los trozos que habían sobrado.

\hypertarget{la-confesiuxf3n-de-pedro-del-mesuxedas-y-el-primer-anuncio-del-sufrimiento-de-jesuxfas}{%
\subsection{La confesión de Pedro del Mesías y el primer anuncio del
sufrimiento de
Jesús}\label{la-confesiuxf3n-de-pedro-del-mesuxedas-y-el-primer-anuncio-del-sufrimiento-de-jesuxfas}}

\bibleverse{18} Mientras oraba a solas, los discípulos estaban cerca de
él y les preguntó: ``¿Quién dicen las multitudes que soy yo?''

\bibleverse{19} Ellos respondieron: ``Juan el Bautista'', pero otros
dicen: ``Elías'', y otros, que uno de los antiguos profetas ha
resucitado''.

\bibleverse{20} Les dijo: ``¿Pero quién decís que soy yo?''. Pedro
respondió: ``El Cristo de Dios''.

\bibleverse{21} Pero les advirtió y les ordenó que no contaran esto a
nadie, \bibleverse{22} diciendo: ``Es necesario que el Hijo del Hombre
padezca muchas cosas, y que sea rechazado por los ancianos, los sumos
sacerdotes y los escribas, y que sea muerto, y al tercer día resucite.''

\hypertarget{proverbios-sobre-seguir-por-los-discuxedpulos}{%
\subsection{Proverbios sobre seguir por los
discípulos}\label{proverbios-sobre-seguir-por-los-discuxedpulos}}

\bibleverse{23} Dijo a todos: ``Si alguno quiere venir en pos de mí,
niéguese a sí mismo, tome su cruz \footnote{\textbf{9:23} TR, NU añaden
  ``diariamente''} y sígame. \bibleverse{24} Porque el que quiera salvar
su vida, la perderá; pero el que pierda su vida por mí, la salvará.
\footnote{\textbf{9:24} Luc 17,33; Mat 10,39; Juan 12,25}
\bibleverse{25} Porque ¿de qué le sirve al hombre ganar el mundo entero,
si se pierde o pierde a sí mismo? \bibleverse{26} Porque el que se
avergüence de mí y de mis palabras, de él se avergonzará el Hijo del
Hombre cuando venga en su gloria, y la gloria del Padre y de los santos
ángeles. \footnote{\textbf{9:26} Mat 10,33; 2Tim 1,8} \bibleverse{27}
Pero os digo la verdad: hay algunos de los que están aquí que no
probarán la muerte hasta que vean el Reino de Dios.''

\hypertarget{la-transfiguraciuxf3n-de-jesuxfas-en-la-montauxf1a}{%
\subsection{La transfiguración de Jesús en la
montaña}\label{la-transfiguraciuxf3n-de-jesuxfas-en-la-montauxf1a}}

\bibleverse{28} Unos ocho días después de estas palabras, tomó consigo a
Pedro, Juan y Santiago, y subió al monte a orar. \bibleverse{29}
Mientras oraba, el aspecto de su rostro se alteró, y su ropa se volvió
blanca y deslumbrante. \bibleverse{30} He aquí que dos hombres hablaban
con él, que eran Moisés y Elías, \bibleverse{31} los cuales aparecieron
en gloria y hablaron de su partida, \footnote{\textbf{9:31}
  literalmente, ``éxodo''} que iba a cumplir en Jerusalén.

\bibleverse{32} Pedro y los que estaban con él estaban agobiados por el
sueño, pero cuando se despertaron del todo, vieron su gloria y a los dos
hombres que estaban con él. \bibleverse{33} Cuando se separaban de él,
Pedro dijo a Jesús: ``Maestro, es bueno que estemos aquí. Hagamos tres
tiendas: una para ti, otra para Moisés y otra para Elías'', sin saber lo
que decía.

\bibleverse{34} Mientras decía estas cosas, vino una nube y los cubrió,
y tuvieron miedo al entrar en la nube. \bibleverse{35} De la nube salió
una voz que decía: ``Este es mi Hijo amado. Escuchadle''. \footnote{\textbf{9:35}
  Luc 3,22; Deut 18,15; Deut 18,19; Sal 2,7} \bibleverse{36} Cuando
llegó la voz, Jesús se encontró solo. Ellos guardaron silencio y no
contaron a nadie en aquellos días nada de lo que habían visto.

\hypertarget{curaciuxf3n-de-un-niuxf1o-epiluxe9ptico}{%
\subsection{Curación de un niño
epiléptico}\label{curaciuxf3n-de-un-niuxf1o-epiluxe9ptico}}

\bibleverse{37} Al día siguiente, cuando bajaron del monte, le salió al
encuentro una gran multitud. \bibleverse{38} He aquí que un hombre de la
muchedumbre gritó diciendo: ``Maestro, te ruego que mires a mi hijo,
porque es mi único hijo nacido\footnote{\textbf{9:38} La frase
  ``unigénito'' proviene de la palabra griega ``\greek{μονογενη}'', que
  a veces se traduce como ``unigénito'' o ``único''.} . \bibleverse{39}
He aquí que un espíritu se apodera de él, grita repentinamente y lo
convulsiona de tal manera que hace espuma; y apenas se aparta de él, lo
hiere gravemente. \bibleverse{40} He rogado a tus discípulos que lo
expulsen, y no han podido''.

\bibleverse{41} Jesús respondió: ``Generación incrédula y perversa,
¿hasta cuándo estaré con vosotros y os soportaré? Traed a vuestro
hijo''.

\bibleverse{42} Mientras se acercaba, el demonio lo arrojó al suelo y lo
convulsionó violentamente. Pero Jesús reprendió al espíritu impuro, curó
al muchacho y se lo devolvió a su padre. \footnote{\textbf{9:42} Luc
  7,15} \bibleverse{43} Todos estaban asombrados de la majestad de Dios.
Pero mientras todos se maravillaban de todas las cosas que Jesús hacía,
dijo a sus discípulos: \footnote{\textbf{9:43} Luc 18,31-34}

\hypertarget{segundo-anuncio-del-sufrimiento-de-jesuxfas}{%
\subsection{Segundo anuncio del sufrimiento de
Jesús}\label{segundo-anuncio-del-sufrimiento-de-jesuxfas}}

\bibleverse{44} ``Que estas palabras se os graben en los oídos, porque
el Hijo del Hombre será entregado en manos de los hombres.''
\bibleverse{45} Pero ellos no entendieron este dicho. Se les ocultó,
para que no lo percibieran, y tuvieron miedo de preguntarle sobre este
dicho. \footnote{\textbf{9:45} Luc 24,45}

\hypertarget{la-arrogancia-de-los-discuxedpulos-enseuxf1anza-sobre-la-humildad-y-la-tolerancia}{%
\subsection{La arrogancia de los discípulos; Enseñanza sobre la humildad
y la
tolerancia}\label{la-arrogancia-de-los-discuxedpulos-enseuxf1anza-sobre-la-humildad-y-la-tolerancia}}

\bibleverse{46} Se suscitó una discusión entre ellos acerca de cuál de
ellos era el más grande. \bibleverse{47} Jesús, percibiendo el
razonamiento de sus corazones, tomó un niño pequeño y lo puso a su lado,
\bibleverse{48} y les dijo: ``El que recibe a este niño en mi nombre, me
recibe a mí. El que me recibe a mí, recibe al que me ha enviado. Porque
el que sea más pequeño entre todos vosotros, éste será grande''.
\footnote{\textbf{9:48} Mat 10,40}

\bibleverse{49} Juan respondió: ``Maestro, vimos a alguien que expulsaba
demonios en tu nombre, y se lo prohibimos, porque no sigue con
nosotros.''

\bibleverse{50} Jesús le dijo: ``No se lo prohíbas, porque el que no
está contra nosotros está a favor''. \footnote{\textbf{9:50} Luc 11,23;
  Fil 1,18}

\hypertarget{salida-para-el-viaje-el-inhuxf3spito-pueblo-samaritano}{%
\subsection{Salida para el viaje; el inhóspito pueblo
samaritano}\label{salida-para-el-viaje-el-inhuxf3spito-pueblo-samaritano}}

\bibleverse{51} Sucedió que, cuando se acercaban los días en que debía
ser llevado, se propuso intensamente ir a Jerusalén \footnote{\textbf{9:51}
  Mar 10,32} \bibleverse{52} y envió mensajeros delante de él. Ellos
fueron y entraron en una aldea de los samaritanos, para prepararse para
él. \bibleverse{53} No le recibieron, porque viajaba con el rostro
puesto en Jerusalén. \footnote{\textbf{9:53} Juan 4,9} \bibleverse{54}
Al ver esto, sus discípulos, Santiago y Juan, dijeron: ``Señor, ¿quieres
que mandemos bajar fuego del cielo y los destruyamos, como hizo Elías?''
\footnote{\textbf{9:54} 2Re 1,10-12}

\bibleverse{55} Pero él se volvió y les reprendió: ``No sabéis de qué
espíritu sois. \bibleverse{56} Porque el Hijo del Hombre no ha venido a
destruir la vida de los hombres, sino a salvarla''. Fueron a otra aldea.
\footnote{\textbf{9:56} Juan 12,47}

\hypertarget{tres-seguidores-diferentes-de-jesuxfas-proverbios-sobre-seguir}{%
\subsection{Tres seguidores diferentes de Jesús; Proverbios sobre
seguir}\label{tres-seguidores-diferentes-de-jesuxfas-proverbios-sobre-seguir}}

\bibleverse{57} Mientras iban por el camino, un hombre le dijo: ``Quiero
seguirte dondequiera que vayas, Señor''.

\bibleverse{58} Jesús le dijo: ``Las zorras tienen madrigueras y las
aves del cielo tienen nidos, pero el Hijo del Hombre no tiene dónde
reclinar la cabeza''.

\bibleverse{59} Le dijo a otro: ``¡Sígueme!'' Pero él dijo: ``Señor,
permíteme primero ir a enterrar a mi padre''.

\bibleverse{60} Pero Jesús le dijo: ``Deja que los muertos entierren a
sus propios muertos, pero tú ve a anunciar el Reino de Dios''.

\bibleverse{61} Otro también dijo: ``Quiero seguirte, Señor, pero
primero permíteme despedirme de los que están en mi casa''. \footnote{\textbf{9:61}
  1Re 19,20}

\bibleverse{62} Pero Jesús le dijo: ``Nadie que ponga la mano en el
arado y mire hacia atrás es apto para el Reino de Dios.'' \footnote{\textbf{9:62}
  Fil 3,13}

\hypertarget{enviar-a-los-setenta-discuxedpulos-y-darles-instrucciones-arrepentimiento-sobre-las-ciudades-galileas-impenitentes}{%
\subsection{Enviar a los setenta discípulos y darles instrucciones;
Arrepentimiento sobre las ciudades galileas
impenitentes}\label{enviar-a-los-setenta-discuxedpulos-y-darles-instrucciones-arrepentimiento-sobre-las-ciudades-galileas-impenitentes}}

\hypertarget{section-9}{%
\section{10}\label{section-9}}

\bibleverse{1} Después de esto, el Señor designó también a otros
setenta, y los envió de dos en dos delante de él\footnote{\textbf{10:1}
  literalmente, ``ante su rostro''} a todas las ciudades y lugares a los
que iba a llegar. \footnote{\textbf{10:1} Mar 6,7} \bibleverse{2} Y les
dijo: ``La mies es abundante, pero los obreros son pocos. Rogad, pues,
al Señor de la mies que envíe obreros a su mies. \footnote{\textbf{10:2}
  Juan 4,35; Mat 9,37-38} \bibleverse{3} Seguid vuestro camino. He aquí
que os envío como corderos en medio de lobos. \bibleverse{4} No lleven
bolso, ni cartera, ni sandalias. No saluden a nadie en el camino.
\footnote{\textbf{10:4} Luc 9,3-5; 2Re 4,29} \bibleverse{5} En cualquier
casa en la que entréis, decid primero: ``Paz a esta casa''. \footnote{\textbf{10:5}
  Juan 20,19} \bibleverse{6} Si hay un hijo de la paz, tu paz descansará
en él; pero si no, volverá a ti. \bibleverse{7} Quédate en esa misma
casa, comiendo y bebiendo lo que te den, porque el trabajador es digno
de su salario. No vayas de casa en casa. \footnote{\textbf{10:7} Éxod
  24,1} \bibleverse{8} En cualquier ciudad en la que entres y te
reciban, come lo que te pongan delante. \bibleverse{9} Sanad a los
enfermos que estén allí y decidles: ``El Reino de Dios se ha acercado a
vosotros''. \bibleverse{10} Pero en cualquier ciudad en la que entréis y
no os reciban, salid a sus calles y decid: \bibleverse{11} `Hasta el
polvo de vuestra ciudad que se nos pegue, lo limpiamos contra vosotros.
Sin embargo, sabed que el Reino de Dios se ha acercado a vosotros'.
\bibleverse{12} Os digo que aquel día será más tolerable para Sodoma que
para esa ciudad.

\bibleverse{13} ``¡Ay de ti, Corazín! ¡Ay de ti, Betsaida! Porque si en
Tiro y en Sidón se hubieran hecho las maravillas que se han hecho en
vosotros, hace tiempo que se habrían arrepentido, sentados en cilicio y
ceniza. \bibleverse{14} Pero será más tolerable para Tiro y Sidón en el
juicio que para vosotros. \bibleverse{15} Vosotros, Capernaum, que
estáis exaltados hasta el cielo, seréis descendida al Hades. \footnote{\textbf{10:15}
  El Hades es el reino inferior de los muertos, o el infierno.}
\bibleverse{16} El que os escucha a vosotros me escucha a mí, y el que
os rechaza a vosotros me rechaza a mí. El que me rechaza a mí, rechaza
al que me envió''. \footnote{\textbf{10:16} Mat 10,40; Juan 5,23}

\hypertarget{regreso-de-los-setenta-discuxedpulos-alegruxeda-de-jesuxfas-y-beatificaciuxf3n-de-los-discuxedpulos}{%
\subsection{Regreso de los setenta discípulos; Alegría de Jesús y
beatificación de los
discípulos}\label{regreso-de-los-setenta-discuxedpulos-alegruxeda-de-jesuxfas-y-beatificaciuxf3n-de-los-discuxedpulos}}

\bibleverse{17} Los setenta volvieron con alegría, diciendo: ``¡Señor,
hasta los demonios se nos someten en tu nombre!''

\bibleverse{18} Les dijo: ``He visto a Satanás caer del cielo como un
rayo. \footnote{\textbf{10:18} Juan 12,31; Apoc 12,8-9} \bibleverse{19}
He aquí que os doy autoridad para pisar serpientes y escorpiones, y
sobre todo el poder del enemigo. Nada podrá haceros daño. \footnote{\textbf{10:19}
  Mar 16,18; Sal 91,13} \bibleverse{20} Sin embargo, no os alegréis de
que los espíritus se os sometan, sino alegraos de que vuestros nombres
estén escritos en el cielo.'' \footnote{\textbf{10:20} Éxod 32,32; Is
  4,3; Fil 4,3; Apoc 3,5}

\bibleverse{21} En esa misma hora, Jesús se regocijó en el Espíritu
Santo y dijo: ``Te doy gracias, Padre, Señor del cielo y de la tierra,
porque has ocultado estas cosas a los sabios y entendidos y las has
revelado a los niños. Sí, Padre, porque así ha sido agradable a tus
ojos''. \footnote{\textbf{10:21} 1Cor 2,7}

\bibleverse{22} Volviéndose a los discípulos, dijo: ``Todo me ha sido
entregado por mi Padre. Nadie sabe quién es el Hijo, sino el Padre, y
quién es el Padre, sino el Hijo, y aquel a quien el Hijo quiera
revelarlo.''

\bibleverse{23} Volviéndose a los discípulos, les dijo en privado:
``Dichosos los ojos que ven lo que vosotros veis, \footnote{\textbf{10:23}
  Mat 13,16-17} \bibleverse{24} porque os digo que muchos profetas y
reyes desearon ver lo que vosotros veis, y no lo vieron, y oír lo que
vosotros oís, y no lo oyeron.'' \footnote{\textbf{10:24} 1Pe 1,10}

\hypertarget{jesuxfas-y-el-maestro-de-la-ley-seres-de-caridad-historia-del-buen-samaritano}{%
\subsection{Jesús y el maestro de la ley; Seres de caridad; Historia del
buen
samaritano}\label{jesuxfas-y-el-maestro-de-la-ley-seres-de-caridad-historia-del-buen-samaritano}}

\bibleverse{25} He aquí que un abogado se levantó y le puso a prueba,
diciendo: ``Maestro, ¿qué debo hacer para heredar la vida eterna?''
\footnote{\textbf{10:25} Luc 18,18-20}

\bibleverse{26} Le dijo: ``¿Qué está escrito en la ley? ¿Cómo la lees?''

\bibleverse{27} Respondió: ``Amarás al Señor tu Dios con todo tu
corazón, con toda tu alma, con todas tus fuerzas y con toda tu mente,
\footnote{\textbf{10:27} Deuteronomio 6:5} y a tu prójimo como a ti
mismo''. \footnote{\textbf{10:27} Levítico 19:18}

\bibleverse{28} Le dijo: ``Has respondido correctamente. Haz esto y
vivirás''. \footnote{\textbf{10:28} Lev 18,5; Mat 19,17}

\bibleverse{29} Pero él, queriendo justificarse, preguntó a Jesús:
``¿Quién es mi prójimo?''

\bibleverse{30} Jesús respondió: ``Un hombre bajaba de Jerusalén a
Jericó, y cayó en manos de unos ladrones, que lo despojaron y golpearon,
y se fueron dejándolo medio muerto. \bibleverse{31} Por casualidad, un
sacerdote bajaba por ese camino. Al verlo, pasó por el otro lado.
\bibleverse{32} Del mismo modo, un levita, al llegar al lugar y verlo,
pasó por el otro lado. \bibleverse{33} Pero un samaritano, que iba de
camino, llegó donde él estaba. Al verlo, se compadeció, \bibleverse{34}
se acercó a él y vendó sus heridas, echando aceite y vino. Lo montó en
su propio animal, lo llevó a una posada y lo cuidó. \bibleverse{35} Al
día siguiente, cuando se marchó, sacó dos denarios, se los dio al
anfitrión y le dijo: ``Cuida de él. Lo que gastes de más, te lo
devolveré cuando vuelva'. \bibleverse{36} Ahora bien, ¿cuál de estos
tres te parece que era prójimo del que cayó entre los ladrones?''

\bibleverse{37} Dijo: ``El que se apiadó de él''. Entonces Jesús le
dijo: ``Ve y haz lo mismo''. \footnote{\textbf{10:37} Juan 13,17}

\hypertarget{marta-y-maruxeda-en-betania}{%
\subsection{Marta y María en
Betania}\label{marta-y-maruxeda-en-betania}}

\bibleverse{38} Mientras iban de camino, entró en una aldea, y una mujer
llamada Marta le recibió en su casa. \footnote{\textbf{10:38} Juan 11,1;
  Juan 12,2-3} \bibleverse{39} Ella tenía una hermana llamada María, que
también se sentaba a los pies de Jesús y escuchaba su palabra.
\bibleverse{40} Pero Marta estaba distraída con muchos quehaceres, y se
acercó a él y le dijo: ``Señor, ¿no te importa que mi hermana me haya
dejado sola para servir? Pídele, pues, que me ayude''.

\bibleverse{41} Jesús le contestó: ``Marta, Marta, te afanas y te
preocupas por muchas cosas, \bibleverse{42} perouna cosa es necesaria.
María ha elegido la parte buena, que no le será quitada''.

\hypertarget{jesuxfas-enseuxf1a-a-los-discuxedpulos-a-orar-el-padre-nuestro}{%
\subsection{Jesús enseña a los discípulos a orar: el Padre
Nuestro}\label{jesuxfas-enseuxf1a-a-los-discuxedpulos-a-orar-el-padre-nuestro}}

\hypertarget{section-10}{%
\section{11}\label{section-10}}

\bibleverse{1} Cuando terminó de orar en un lugar, uno de sus discípulos
le dijo: ``Señor, enséñanos a orar, como también Juan enseñó a sus
discípulos.''

\bibleverse{2} Les dijo: ``Cuando oréis, decid, `Padre nuestro que estás
en el cielo', que tu nombre sea sagrado. Que venga tu Reino. Que se haga
tu voluntad en la tierra, como en el cielo. \bibleverse{3} Danos cada
día el pan de cada día. \bibleverse{4} Perdona nuestros pecados, porque
nosotros también perdonamos a todos los que están en deuda con nosotros.
No nos dejes caer en la tentación, pero líbranos del maligno''.

\hypertarget{la-paruxe1bola-del-amigo-suplicante-jesuxfas-anima-a-la-oraciuxf3n-persistente-y-persistente}{%
\subsection{La parábola del amigo suplicante; Jesús anima a la oración
persistente y
persistente}\label{la-paruxe1bola-del-amigo-suplicante-jesuxfas-anima-a-la-oraciuxf3n-persistente-y-persistente}}

\bibleverse{5} Les dijo: ``¿Quién de vosotros, si va a un amigo a
medianoche y le dice: ``Amigo, préstame tres panes, \bibleverse{6}
porque un amigo mío ha venido de viaje y no tengo nada que ponerle
delante'', \bibleverse{7} y él, desde dentro, le responde y le dice:
``No me molestes. La puerta está cerrada y mis hijos están conmigo en la
cama. No puedo levantarme y dárselo'? \bibleverse{8} Os digo que, aunque
no se levante a dárselo porque es su amigo, por su insistencia se
levantará y le dará todos los que necesite.

\bibleverse{9} ``Os digo que sigáis pidiendo y se os dará. Sigan
buscando y encontrarán. Seguid llamando, y se os abrirá. \bibleverse{10}
Porque todo el que pide recibe. El que busca encuentra. Al que llama se
le abrirá. \footnote{\textbf{11:10} Luc 13,25}

\bibleverse{11} ``¿Quién de vosotros, padres, si su hijo le pide pan, le
dará una piedra? O si le pide un pescado, acaso le dará una serpiente en
lugar de un pescado, ¿verdad? \bibleverse{12} O si le pide un huevo, no
le dará un escorpión, ¿verdad? \bibleverse{13} Si ustedes, siendo malos,
saben dar buenos regalos a sus hijos, ¿cuánto más su Padre celestial
dará el Espíritu Santo a los que se lo pidan?''

\hypertarget{jesuxfas-se-defiende-de-la-blasfemia-de-los-fariseos-contra-beelzebul-paruxe1bola-de-recauxedda}{%
\subsection{Jesús se defiende de la blasfemia de los fariseos contra
Beelzebul; Parábola de
recaída}\label{jesuxfas-se-defiende-de-la-blasfemia-de-los-fariseos-contra-beelzebul-paruxe1bola-de-recauxedda}}

\bibleverse{14} Estaba expulsando a un demonio, y éste era mudo. Cuando
el demonio salió, el mudo habló; y las multitudes se maravillaron.
\bibleverse{15} Pero algunos de ellos decían: ``Expulsa los demonios por
Beelzebul, el príncipe de los demonios.'' \bibleverse{16} Otros,
poniéndole a prueba, pedían de él una señal del cielo. \footnote{\textbf{11:16}
  Mar 8,11} \bibleverse{17} Pero él, conociendo sus pensamientos, les
dijo: ``Todo reino dividido contra sí mismo es asolado. Una casa
dividida contra sí misma cae. \bibleverse{18} Si también Satanás está
dividido contra sí mismo, ¿cómo permanecerá su reino? Porque decís que
yo expulso los demonios por Beelzebul. \bibleverse{19} Pero si yo
expulso los demonios por Beelzebul, ¿por quién los expulsan vuestros
hijos? Por tanto, ellos serán vuestros jueces. \bibleverse{20} Pero si
yo expulso los demonios por el dedo de Dios, entonces el Reino de Dios
ha llegado a vosotros. \footnote{\textbf{11:20} Éxod 8,19}

\bibleverse{21} ``Cuando el hombre fuerte, completamente armado, vigila
su propia morada, sus bienes están a salvo. \bibleverse{22} Pero cuando
alguien más fuerte lo ataca y lo vence, le quita toda la armadura en la
que confiaba y reparte su botín. \footnote{\textbf{11:22} Col 2,15; 1Jn
  4,4}

\bibleverse{23} ``El que no está conmigo está contra mí. El que no se
reúne conmigo se dispersa. \footnote{\textbf{11:23} Luc 9,50}

\bibleverse{24} El espíritu inmundo, cuando ha salido del hombre, pasa
por lugares secos, buscando descanso; y al no encontrarlo, dice: `Me
volveré a mi casa de donde salí'. \bibleverse{25} Cuando regresa, la
encuentra barrida y ordenada. \bibleverse{26} Entonces va y toma otros
siete espíritus más malos que él, y entran y habitan allí. El último
estado de ese hombre llega a ser peor que el primero''. \footnote{\textbf{11:26}
  Juan 5,14}

\hypertarget{beatificaciuxf3n-de-jesuxfas-de-los-verdaderos-hijos-de-dios}{%
\subsection{Beatificación de Jesús de los verdaderos hijos de
Dios}\label{beatificaciuxf3n-de-jesuxfas-de-los-verdaderos-hijos-de-dios}}

\bibleverse{27} Mientras decía estas cosas, una mujer de entre la
multitud alzó la voz y le dijo: ``¡Bendito sea el vientre que te llevó y
los pechos que te amamantaron!'' \footnote{\textbf{11:27} Luc 1,28; Luc
  1,48}

\bibleverse{28} Pero él dijo: ``Al contrario, bienaventurados los que
escuchan la palabra de Dios y la guardan''. \footnote{\textbf{11:28} Luc
  8,15; Luc 8,21}

\hypertarget{el-discurso-de-jesuxfas-contra-los-que-peduxedan-seuxf1ales-el-signo-de-jonuxe1s}{%
\subsection{El discurso de Jesús contra los que pedían señales; el signo
de
Jonás}\label{el-discurso-de-jesuxfas-contra-los-que-peduxedan-seuxf1ales-el-signo-de-jonuxe1s}}

\bibleverse{29} Cuando las multitudes se reunieron con él, comenzó a
decir: ``Esta es una generación malvada. Busca una señal. No se le dará
otra señal que la del profeta Jonás. \bibleverse{30} Porque así como
Jonás fue una señal para los ninivitas, así también lo será el Hijo del
Hombre para esta generación. \bibleverse{31} La Reina del Sur se
levantará en el juicio con los hombres de esta generación y los
condenará, porque ha venido desde los confines de la tierra para
escuchar la sabiduría de Salomón; y he aquí que uno más grande que
Salomón está aquí. \footnote{\textbf{11:31} 1Re 10,1} \bibleverse{32}
Los hombres de Nínive se levantarán en el juicio con esta generación y
la condenarán, porque se arrepintieron ante la predicación de Jonás; y
he aquí que uno más grande que Jonás está aquí. \footnote{\textbf{11:32}
  Jon 3,5}

\bibleverse{33} ``Nadie, cuando ha encendido una lámpara, la pone en un
sótano o debajo de un cesto, sino sobre un soporte, para que los que
entren puedan ver la luz. \footnote{\textbf{11:33} Luc 8,16}
\bibleverse{34} La lámpara del cuerpo es el ojo. Por eso, cuando tu ojo
es bueno, todo tu cuerpo está también lleno de luz; pero cuando es malo,
también tu cuerpo está lleno de oscuridad. \bibleverse{35} Mira, pues,
si la luz que hay en ti no es oscuridad. \bibleverse{36} Si, pues, todo
tu cuerpo está lleno de luz, sin que haya ninguna parte oscura, estará
totalmente lleno de luz, como cuando la lámpara con su resplandor te
alumbra.''

\hypertarget{contra-la-hipocresuxeda-de-los-fariseos}{%
\subsection{Contra la hipocresía de los
fariseos}\label{contra-la-hipocresuxeda-de-los-fariseos}}

\bibleverse{37} Mientras hablaba, un fariseo le pidió que cenara con él.
Entró y se sentó a la mesa. \footnote{\textbf{11:37} Luc 7,36; Luc 14,1}
\bibleverse{38} Cuando el fariseo lo vio, se maravilló de que no se
hubiera lavado antes de cenar. \footnote{\textbf{11:38} Mat 15,2}
\bibleverse{39} El Señor le dijo: ``Ahora bien, vosotros, fariseos,
limpiáis el exterior de la copa y del plato, pero vuestro interior está
lleno de extorsión y de maldad. \bibleverse{40} Vosotros, insensatos,
¿no hizo también lo de dentro el que hizo lo de fuera? \bibleverse{41}
Pero dad por regalos a los necesitados lo que hay dentro, y he aquí que
todo os quedará limpio. \bibleverse{42} Pero ¡ay de vosotros, fariseos!
Porque diezmáis la menta y la ruda y toda hierba, pero dejáis de lado la
justicia y el amor de Dios. Deberíais haber hecho esto, y no haber
dejado de hacer lo otro. \footnote{\textbf{11:42} Juan 5,42}
\bibleverse{43} ¡Ay de vosotros, fariseos! Porque amáis los mejores
asientos en las sinagogas y los saludos en las plazas. \footnote{\textbf{11:43}
  Luc 14,7} \bibleverse{44} ¡Ay de vosotros, escribas y fariseos,
hipócritas! Porque sois como sepulcros ocultos, y los hombres que andan
sobre ellos no lo saben''.

\hypertarget{gritos-de-defensa-sobre-la-malicia-de-los-maestros-de-la-ley}{%
\subsection{Gritos de defensa sobre la malicia de los maestros de la
ley}\label{gritos-de-defensa-sobre-la-malicia-de-los-maestros-de-la-ley}}

\bibleverse{45} Uno de los abogados le respondió: ``Maestro, al decir
esto también nos insultas''.

\bibleverse{46} Dijo: ``¡Ay de ustedes, los abogados! Porque cargáis a
los hombres con cargas difíciles de llevar, y vosotros mismos no
levantáis ni un dedo para ayudar a llevar esas cargas. \bibleverse{47}
¡Ay de ustedes! Porque construís las tumbas de los profetas, y vuestros
padres los mataron. \footnote{\textbf{11:47} Hech 7,52} \bibleverse{48}
Así pues, vosotros dais testimonio y consentid en las obras de vuestros
padres. Porque ellos los mataron, y vosotros construís sus tumbas.
\bibleverse{49} Por eso también la sabiduría de Dios dijo: `Les enviaré
profetas y apóstoles; y a algunos de ellos los matarán y perseguirán,
\bibleverse{50} para que la sangre de todos los profetas, que fue
derramada desde la fundación del mundo, sea requerida de esta
generación, \bibleverse{51} desde la sangre de Abel hasta la sangre de
Zacarías, que pereció entre el altar y el santuario.' Sí, os digo que se
exigirá a esta generación. \bibleverse{52} ¡Ay de vosotros, abogados!
Porque os habéis llevado la llave del conocimiento. Vosotros mismos no
entrasteis, y a los que entraban, se lo impedisteis''.

\bibleverse{53} Mientras les decía estas cosas, los escribas y los
fariseos empezaron a enojarse terriblemente, y con vehemencia le hacían
preguntas, \bibleverse{54} acechándole y buscando sorprenderle en algo
que pudiera decir, para acusarle. \footnote{\textbf{11:54} Luc 20,20}

\hypertarget{advertencia-de-hipocresuxeda-farisea}{%
\subsection{Advertencia de hipocresía
farisea}\label{advertencia-de-hipocresuxeda-farisea}}

\hypertarget{section-11}{%
\section{12}\label{section-11}}

\bibleverse{1} Mientras tanto, cuando se había reunido una multitud de
muchos miles de personas, tanto que se pisoteaban unos a otros, comenzó
a decir a sus discípulos, en primer lugar: ``Guardaos de la levadura de
los fariseos, que es la hipocresía. \footnote{\textbf{12:1} Mat 16,6;
  Mar 8,15} \bibleverse{2} Pero no hay nada encubierto que no se revele,
ni oculto que no se sepa. \footnote{\textbf{12:2} Luc 8,17}
\bibleverse{3} Por tanto, lo que habéis dicho en la oscuridad se oirá en
la luz. Lo que habéis dicho al oído en las habitaciones interiores se
proclamará en las azoteas.

\hypertarget{advertencia-de-miedo-al-hombre}{%
\subsection{Advertencia de miedo al
hombre}\label{advertencia-de-miedo-al-hombre}}

\bibleverse{4} ``Os digo, amigos míos, que no tengáis miedo de los que
matan el cuerpo, y después no tienen más que hacer. \bibleverse{5} Pero
os advertiré a quién debéis temer. Temed a aquel que, después de haber
matado, tiene poder para arrojar a la Gehena.\footnote{\textbf{12:5} TR
  lee ``asno'' en lugar de ``hijo''} Sí, os digo que le temáis.
\footnote{\textbf{12:5} Heb 12,29}

\bibleverse{6} ``¿No se venden cinco gorriones por dos monedas de
asaria? Ni uno solo de ellos es olvidado por Dios. \bibleverse{7} Pero
los cabellos de tu cabeza están todos contados. Por eso no tengas miedo.
Vosotros tenéis más valor que muchos gorriones. \footnote{\textbf{12:7}
  Luc 21,18}

\bibleverse{8} ``Os digo que todo el que me confiese ante los hombres,
el Hijo del Hombre lo hará también ante los ángeles de Dios;
\bibleverse{9} pero el que me niegue en presencia de los hombres, será
negado en presencia de los ángeles de Dios. \footnote{\textbf{12:9} Luc
  9,26; 1Sam 2,30}

\hypertarget{advertencia-de-blasfemia-contra-el-espuxedritu-santo-referencia-a-la-asistencia-del-espuxedritu}{%
\subsection{Advertencia de blasfemia contra el Espíritu Santo;
Referencia a la asistencia del
Espíritu}\label{advertencia-de-blasfemia-contra-el-espuxedritu-santo-referencia-a-la-asistencia-del-espuxedritu}}

\bibleverse{10} Todo el que diga una palabra contra el Hijo del Hombre
será perdonado, pero los que blasfemen contra el Espíritu Santo no serán
perdonados. \footnote{\textbf{12:10} Mat 12,32; Mar 3,28-29}
\bibleverse{11} Cuando os lleven ante las sinagogas, los gobernantes y
las autoridades, no os preocupéis por cómo o qué vais a responder o qué
vais a decir; \footnote{\textbf{12:11} Luc 21,14-15; Mat 10,19-20}
\bibleverse{12} porque el Espíritu Santo os enseñará en esa misma hora
lo que debéis decir.''

\bibleverse{13} Uno de la multitud le dijo: ``Maestro, dile a mi hermano
que reparta la herencia conmigo''.

\bibleverse{14} Pero él le dijo: ``Hombre, ¿quién me ha hecho juez o
árbitro sobre vosotros?''

\hypertarget{advertencia-de-codicia-paruxe1bola-del-rico-tonto}{%
\subsection{Advertencia de codicia; Parábola del rico
tonto}\label{advertencia-de-codicia-paruxe1bola-del-rico-tonto}}

\bibleverse{15} Él les dijo: ``¡Cuidado! Guardaos de la codicia, porque
la vida de un hombre no consiste en la abundancia de los bienes que
posee.'' \footnote{\textbf{12:15} 1Tim 6,9-10}

\bibleverse{16} Les contó una parábola, diciendo: ``La tierra de un
hombre rico producía en abundancia. \bibleverse{17} El hombre
reflexionaba sobre su situación, diciendo: ``¿Qué voy a hacer, porque no
tengo espacio para almacenar mis cosechas? \bibleverse{18} Derribaré mis
graneros y construiré otros más grandes, y allí almacenaré todo mi grano
y mis bienes. \bibleverse{19} Le diré a mi alma: ``Alma, tienes muchos
bienes acumulados para muchos años. Descansa, come, bebe y alégrate''.
\footnote{\textbf{12:19} Sal 49,16-19}

\bibleverse{20} ``Pero Dios le dijo: `Necio, esta noche tu alma es
requerida. Las cosas que has preparado, ¿de quién serán?' \footnote{\textbf{12:20}
  Heb 9,27} \bibleverse{21} Así es el que acumula tesoros para sí mismo
y no es rico para con Dios.'' \footnote{\textbf{12:21} Mat 6,20}

\hypertarget{procurad-el-reino-de-dios-y-todas-estas-cosas-os-seruxe1n-auxf1adidas}{%
\subsection{Procurad el reino de Dios, y todas estas cosas os serán
añadidas}\label{procurad-el-reino-de-dios-y-todas-estas-cosas-os-seruxe1n-auxf1adidas}}

\bibleverse{22} Dijo a sus discípulos: ``Por eso os digo que no os
preocupéis por vuestra vida, por lo que vais a comer, ni por vuestro
cuerpo, por lo que vais a vestir. \bibleverse{23} La vida es más que el
alimento, y el cuerpo más que el vestido. \bibleverse{24} Consideren a
los cuervos: no siembran, ni cosechan, no tienen almacén ni granero, y
Dios los alimenta. ¡Cuánto más valéis vosotros que las aves!
\bibleverse{25} ¿Quién de vosotros puede añadir un codo a su estatura
por estar ansioso? \bibleverse{26} Pues si no sois capaces de hacer ni
siquiera lo más mínimo, ¿por qué os preocupáis por lo demás?
\bibleverse{27} Considerad los lirios, cómo crecen. No trabajan, ni
hilan; pero os digo que ni siquiera Salomón, con toda su gloria, se
vistió como uno de ellos. \bibleverse{28} Pero si así viste Dios a la
hierba del campo, que hoy existe y mañana se echa en el horno, ¿cuánto
más os vestirá a vosotros, hombres de poca fe?

\bibleverse{29} ``No busquéis lo que vais a comer o lo que vais a beber,
ni os preocupéis. \bibleverse{30} Porque las naciones del mundo buscan
todas estas cosas, pero vuestro Padre sabe que necesitáis estas cosas.
\bibleverse{31} Pero buscad el Reino de Dios, y todas estas cosas os
serán añadidas.

\bibleverse{32} ``No tengáis miedo, pequeño rebaño, porque a vuestro
Padre le ha parecido bien daros el Reino. \footnote{\textbf{12:32} Luc
  22,29; Is 41,14} \bibleverse{33} Vendan lo que tienen y den regalos a
los necesitados. Haceos bolsas que no envejecen, un tesoro en los cielos
que no falla, donde ningún ladrón se acerca y ninguna polilla destruye.
\footnote{\textbf{12:33} Luc 18,22} \bibleverse{34} Porque donde esté
vuestro tesoro, allí estará también vuestro corazón.

\hypertarget{recordatorios-de-vigilancia-y-preparaciuxf3n-igualdad-de-disposiciuxf3n-de-los-siervos-fieles-y-del-maligno}{%
\subsection{Recordatorios de vigilancia y preparación; Igualdad de
disposición de los siervos fieles y del
maligno}\label{recordatorios-de-vigilancia-y-preparaciuxf3n-igualdad-de-disposiciuxf3n-de-los-siervos-fieles-y-del-maligno}}

\bibleverse{35} ``Tened el talle vestido y las lámparas encendidas.
\footnote{\textbf{12:35} Éxod 12,11; 1Pe 1,13; Mat 25,1-13}
\bibleverse{36} Sed como hombres que velan por su señor cuando vuelve
del banquete de bodas, para que cuando venga y llame, le abran
enseguida. \footnote{\textbf{12:36} Apoc 3,20} \bibleverse{37}
Bienaventurados los siervos a los que el Señor encuentre velando cuando
venga. Ciertamente os digo que se vestirá, los hará sentar y vendrá a
servirles. \bibleverse{38} Serán bienaventurados si viene en la segunda
o tercera vigilia y los encuentra así. \bibleverse{39} Pero sabed esto,
que si el dueño de la casa hubiera sabido a qué hora iba a venir el
ladrón, habría vigilado y no habría permitido que entraran en su casa.
\footnote{\textbf{12:39} 1Tes 5,2} \bibleverse{40} Por tanto, estad
también preparados, porque el Hijo del Hombre vendrá a una hora que no
esperáis.''

\bibleverse{41} Pedro le dijo: ``Señor, ¿nos cuentas esta parábola a
nosotros o a todo el mundo?''.

\bibleverse{42} El Señor dijo: ``¿Quién es, pues, el administrador fiel
y prudente, al que su señor pondrá al frente de su casa, para que les dé
su ración de comida en los momentos oportunos? \bibleverse{43} Dichoso
aquel siervo al que su señor encuentre haciendo eso cuando venga.
\bibleverse{44} En verdad os digo que le pondrá al frente de todo lo que
tiene. \bibleverse{45} Pero si ese siervo dice en su corazón: ``Mi señor
tarda en venir'', y comienza a golpear a los siervos y a las siervas, y
a comer y a beber y a embriagarse, \bibleverse{46} entonces el señor de
ese siervo vendrá en un día que no lo espera y a una hora que no conoce,
y lo partirá en dos, y pondrá su porción con los infieles.
\bibleverse{47} Aquel siervo que conocía la voluntad de su señor, y no
se preparó ni hizo lo que él quería, será azotado con muchos azotes,
\footnote{\textbf{12:47} Sant 4,17} \bibleverse{48} pero el que no
sabía, y hacía cosas dignas de azotes, será azotado con pocos azotes. A
quien se le dio mucho, se le exigirá mucho; y a quien se le confió
mucho, se le pedirá más.

\hypertarget{jesuxfas-trae-fuego-y-conflicto-a-la-tierra}{%
\subsection{Jesús trae fuego y conflicto a la
tierra}\label{jesuxfas-trae-fuego-y-conflicto-a-la-tierra}}

\bibleverse{49} ``He venido a arrojar fuego sobre la tierra. Ojalá
estuviera ya encendido. \bibleverse{50} Pero tengo un bautismo con el
que ser bautizado, ¡y qué angustia tengo hasta que se cumpla!
\footnote{\textbf{12:50} Luc 18,31; Mat 20,22; Mat 26,38}
\bibleverse{51} ¿Creéis que he venido a dar paz en la tierra? Os digo
que no, sino más bien para dividir. \bibleverse{52} Porque a partir de
ahora, en una casa habrá cinco divididos, tres contra dos, y dos contra
tres. \bibleverse{53} Estarán divididos, el padre contra el hijo, y el
hijo contra el padre; la madre contra la hija, y la hija contra su
madre; la suegra contra su nuera, y la nuera contra su suegra.''

\hypertarget{los-signos-de-los-tiempos-te-instan-a-tomar-en-serio-la-salvaciuxf3n-de-tu-alma}{%
\subsection{Los signos de los tiempos te instan a tomar en serio la
salvación de tu
alma}\label{los-signos-de-los-tiempos-te-instan-a-tomar-en-serio-la-salvaciuxf3n-de-tu-alma}}

\bibleverse{54} También dijo a las multitudes: ``Cuando veis una nube
que se levanta por el oeste, enseguida decís: `Va a llover', y así
sucede. \footnote{\textbf{12:54} Mat 16,2-3} \bibleverse{55} Cuando
sopla un viento del sur, decís: `Habrá un calor abrasador', y así
sucede. \bibleverse{56} ¡Hipócritas! Sabéis interpretar el aspecto de la
tierra y del cielo, pero ¿cómo es que no interpretáis este tiempo?

\bibleverse{57} ``¿Por qué no juzgáis por vosotros mismos lo que es
justo? \bibleverse{58} Porque cuando vayáis con vuestro adversario ante
el magistrado, procurad con diligencia en el camino libraros de él, no
sea que os arrastre al juez, y el juez os entregue al oficial, y el
oficial os meta en la cárcel. \footnote{\textbf{12:58} Mat 5,25-26}
\bibleverse{59} Te digo que de ninguna manera saldrás de allí hasta que
hayas pagado hasta el último centavo.''

\hypertarget{recordatorios-de-penitencia}{%
\subsection{Recordatorios de
penitencia}\label{recordatorios-de-penitencia}}

\hypertarget{section-12}{%
\section{13}\label{section-12}}

\bibleverse{1} Al mismo tiempo estaban presentes algunos que le hablaron
de los galileos cuya sangre Pilato había mezclado con sus sacrificios.
\bibleverse{2} Jesús les contestó: ``¿Pensáis que estos galileos eran
peores pecadores que todos los demás galileos, por haber sufrido tales
cosas? \footnote{\textbf{13:2} Juan 9,2} \bibleverse{3} Os digo que no,
pero si no os arrepentís, todos pereceréis de la misma manera.
\bibleverse{4} O aquellos dieciocho sobre los que cayó la torre en Siloé
y los mató: ¿pensáis que eran peores pecadores que todos los hombres que
habitan en Jerusalén? \bibleverse{5} Os digo que no, sino que, si no os
arrepentís, todos pereceréis de la misma manera.''

\bibleverse{6} Dijo esta parábola. ``Un hombre tenía una higuera
plantada en su viña, y vino a buscar fruto en ella y no lo encontró.
\footnote{\textbf{13:6} Mat 21,19} \bibleverse{7} Y dijo al viñador:
``Mira, estos tres años he venido a buscar fruto en esta higuera, y no
lo he encontrado. Córtala. \bibleverse{8} El viñador respondió: ``Señor,
déjala también este año, hasta que cave alrededor y la abone.
\footnote{\textbf{13:8} 2Pe 3,9; 2Pe 3,15} \bibleverse{9} Si da fruto,
bien; pero si no, después puedes cortarla'\,''. \footnote{\textbf{13:9}
  Luc 3,9}

\hypertarget{una-curaciuxf3n-de-los-enfermos-en-suxe1bado-disputa-por-guardar-el-duxeda-de-reposo}{%
\subsection{Una curación de los enfermos en sábado; Disputa por guardar
el día de
reposo}\label{una-curaciuxf3n-de-los-enfermos-en-suxe1bado-disputa-por-guardar-el-duxeda-de-reposo}}

\bibleverse{10} Estaba enseñando en una de las sinagogas en el día de
reposo. \footnote{\textbf{13:10} Luc 6,6-11} \bibleverse{11} He aquí que
había una mujer que tenía un espíritu de enfermedad de dieciocho años.
Estaba encorvada y no podía enderezarse. \bibleverse{12} Al verla, Jesús
la llamó y le dijo: ``Mujer, estás libre de tu enfermedad''.
\bibleverse{13} Le impuso las manos, y al instante ella se enderezó y
glorificaba a Dios.

\bibleverse{14} El jefe de la sinagoga, indignado porque Jesús había
curado en sábado, dijo a la multitud: ``Hay seis días en los que se debe
trabajar. Vengan, pues, en esos días y sean curados, y no en el día de
reposo''. \footnote{\textbf{13:14} Éxod 20,9-10}

\bibleverse{15} Por eso el Señor le respondió: ``¡Hipócritas! ¿No libera
cada uno de vosotros a su buey o a su asno del establo en sábado y lo
lleva al agua? \footnote{\textbf{13:15} Luc 14,5} \bibleverse{16} ¿No
debería esta mujer, que es hija de Abraham y que Satanás ha atado
durante dieciocho largos años, ser liberada de esta esclavitud en el día
de reposo?'' \footnote{\textbf{13:16} Luc 19,9}

\bibleverse{17} Al decir estas cosas, todos sus adversarios quedaron
decepcionados, y toda la multitud se alegró por todas las cosas
gloriosas que había hecho.

\hypertarget{paruxe1bolas-del-grano-de-mostaza-y-levadura}{%
\subsection{Parábolas del grano de mostaza y
levadura}\label{paruxe1bolas-del-grano-de-mostaza-y-levadura}}

\bibleverse{18} Dijo: ``¿Cómo es el Reino de Dios? ¿Con qué lo
compararé? \bibleverse{19} Es como un grano de mostaza que un hombre
tomó y puso en su jardín. Creció y se convirtió en un gran árbol, y las
aves del cielo viven en sus ramas''.

\bibleverse{20} Y volvió a decir: ``¿A qué voy a comparar el Reino de
Dios? \bibleverse{21} Es como la levadura que una mujer tomó y escondió
en tres medidas de harina, hasta que todo quedó leudado.''

\bibleverse{22} Siguió su camino por ciudades y aldeas, enseñando, y
viajando hacia Jerusalén. \bibleverse{23} Uno le dijo: ``Señor, ¿son
pocos los que se salvan?'' Les dijo:

\hypertarget{porfiad-uxe1-entrar-por-la-puerta-angosta}{%
\subsection{Porfiad á entrar por la puerta
angosta}\label{porfiad-uxe1-entrar-por-la-puerta-angosta}}

\bibleverse{24} ``Procurad entrar por la puerta estrecha, porque os digo
que muchos intentarán entrar y no podrán. \footnote{\textbf{13:24} Mat
  7,13-14} \bibleverse{25} Cuando el dueño de la casa se levante y
cierre la puerta, y vosotros empecéis a estar fuera y a llamar a la
puerta, diciendo: ``Señor, Señor, ábrenos'', entonces os responderá y os
dirá: ``No os conozco ni sabéis de dónde venís''. \footnote{\textbf{13:25}
  Mat 25,11-12} \bibleverse{26} Entonces comenzará a decir: `Comimos y
bebimos en tu presencia, y enseñaste en nuestras calles.' \footnote{\textbf{13:26}
  Mat 7,22-23} \bibleverse{27} Él dirá: ``Os digo que no sé de dónde
venís. Apartaos de mí, todos los obreros de la iniquidad'.
\bibleverse{28} Será el llanto y el crujir de dientes cuando vean a
Abraham, a Isaac, a Jacob y a todos los profetas en el Reino de Dios, y
a ustedes mismos arrojados fuera. \footnote{\textbf{13:28} Mat 8,11-12}
\bibleverse{29} Vendrán del este, del oeste, del norte y del sur, y se
sentarán en el Reino de Dios. \footnote{\textbf{13:29} Luc 14,15}
\bibleverse{30} He aquí que hay unos últimos que serán primeros, y hay
unos primeros que serán últimos.'' \footnote{\textbf{13:30} Mat 19,30}

\hypertarget{ay-de-jerusaluxe9n-que-apedrea-a-los-profetas}{%
\subsection{Ay de Jerusalén, que apedrea a los
profetas}\label{ay-de-jerusaluxe9n-que-apedrea-a-los-profetas}}

\bibleverse{31} Aquel mismo día vinieron unos fariseos y le dijeron:
``Sal de aquí y vete, porque Herodes quiere matarte''.

\bibleverse{32} Les dijo: ``Id y decidle a esa zorra: `He aquí que hoy y
mañana expulso demonios y hago curaciones, y al tercer día concluyo mi
misión. \bibleverse{33} Sin embargo, debo seguir mi camino hoy y mañana
y al día siguiente, pues no puede ser que un profeta perezca fuera de
Jerusalén.'

\bibleverse{34} ``¡Jerusalén, Jerusalén, la que mata a los profetas y
apedrea a los que le son enviados! Cuántas veces quise reunir a tus
hijos, como la gallina reúne a sus crías bajo sus alas, y te negaste.
\footnote{\textbf{13:34} Luc 19,41-44; Mat 23,37-39} \bibleverse{35} He
aquíque tu casa te ha quedado desolada. Os digo que no me veréis hasta
que digáis: ``¡Bendito el que viene en nombre del Señor!''. \footnote{\textbf{13:35}
  Sal 118,26}

\hypertarget{nueva-disputa-sobre-la-curaciuxf3n-de-los-enfermos-en-suxe1bado}{%
\subsection{Nueva disputa sobre la curación de los enfermos en
sábado}\label{nueva-disputa-sobre-la-curaciuxf3n-de-los-enfermos-en-suxe1bado}}

\hypertarget{section-13}{%
\section{14}\label{section-13}}

\bibleverse{1} Al entrar un sábado en casa de uno de los jefes de los
fariseos para comer pan, le estaban vigilando. \footnote{\textbf{14:1}
  Luc 6,6-11; Luc 11,37} \bibleverse{2} He aquí que un hombre que tenía
hidropesía estaba delante de él. \bibleverse{3} Respondiendo Jesús,
habló a los letrados y fariseos, diciendo: ``¿Es lícito curar en
sábado?''

\bibleverse{4} Pero ellos guardaron silencio. Lo tomó, lo curó y lo dejó
ir. \bibleverse{5} Les respondió: ``¿Quién de vosotros, si su hijo o su
buey cayera en un pozo, no lo sacaría inmediatamente en un día de
reposo?'' \footnote{\textbf{14:5} Luc 13,5; Mat 12,11}

\bibleverse{6} No pudieron responderle sobre estas cosas.

\hypertarget{cualquiera-que-se-ensalza-seruxe1-humillado-y-el-que-se-humilla-seruxe1-ensalzado}{%
\subsection{Cualquiera que se ensalza, será humillado; y el que se
humilla, será
ensalzado}\label{cualquiera-que-se-ensalza-seruxe1-humillado-y-el-que-se-humilla-seruxe1-ensalzado}}

\bibleverse{7} Dijo una parábola a los invitados, al notar que elegían
los mejores asientos, y les dijo: \footnote{\textbf{14:7} Mat 23,6}
\bibleverse{8} ``Cuando alguien os invite a un banquete de bodas, no os
sentéis en el mejor asiento, pues tal vez alguien más honorable que
vosotros sea invitado por él, \bibleverse{9} y el que os invitó a los
dos vendría y os diría: ``Haced sitio a esta persona''. Entonces
empezaríais, con vergüenza, a ocupar el lugar más bajo. \bibleverse{10}
Pero cuando te inviten, ve y siéntate en el lugar más bajo, para que
cuando venga el que te invitó, te diga: `Amigo, sube más arriba'.
Entonces serás honrado en presencia de todos los que se sienten a la
mesa contigo. \bibleverse{11} Porque todo el que se enaltece será
humillado, y el que se humilla será enaltecido''. \footnote{\textbf{14:11}
  Luc 18,14; Mat 23,12; Sant 4,6; Sant 1,4-10}

\bibleverse{12} También le dijo al que le había invitado: ``Cuando hagas
una cena o un banquete, no llames a tus amigos, ni a tus hermanos, ni a
tus parientes, ni a los vecinos ricos, porque tal vez ellos también te
devuelvan el favor y te lo paguen. \bibleverse{13} Pero cuando hagas un
banquete, pide a los pobres, a los mancos, a los cojos o a los ciegos;
\footnote{\textbf{14:13} Deut 14,29} \bibleverse{14} y serás bendecido,
porque ellos no tienen recursos para pagarte. Porque te lo pagarán en la
resurrección de los justos''. \footnote{\textbf{14:14} Juan 5,29}

\hypertarget{la-paruxe1bola-de-la-gran-cena}{%
\subsection{La parábola de la gran
cena}\label{la-paruxe1bola-de-la-gran-cena}}

\bibleverse{15} Cuando uno de los que se sentaba a la mesa con él oyó
estas cosas, le dijo: ``¡Bienaventurado el que festejará en el Reino de
Dios!'' \footnote{\textbf{14:15} Luc 13,29}

\bibleverse{16} Pero él le dijo: ``Un hombre hizo una gran cena, e
invitó a mucha gente. \bibleverse{17} A la hora de la cena mandó a su
criado a decir a los invitados: ``Venid, porque ya está todo
preparado''. \bibleverse{18} Todos a una comenzaron a excusarse. ``El
primero le dijo: `He comprado un campo y debo ir a verlo. Te ruego que
me disculpes'.

\bibleverse{19} ``Otro dijo: `He comprado cinco yuntas de bueyes y debo
ir a probarlos. Te ruego que me disculpes'.

\bibleverse{20} ``Otro dijo: `Me he casado con una mujer, y por eso no
puedo venir'. \footnote{\textbf{14:20} 1Cor 7,33}

\bibleverse{21} ``Llegó aquel siervo y le contó a su señor estas cosas.
Entonces el señor de la casa, enojado, dijo a su siervo: `Sal pronto a
las calles y a las callejuelas de la ciudad, y trae a los pobres, a los
mancos, a los ciegos y a los cojos'.

\bibleverse{22} ``El siervo dijo: `Señor, está hecho como lo has
mandado, y todavía hay lugar'.

\bibleverse{23} ``El señor dijo al criado: `Sal a los caminos y a los
setos y oblígalos a entrar, para que se llene mi casa. \bibleverse{24}
Porque te digo que ninguno de esos hombres invitados probará mi
cena'\,''.

\hypertarget{sobre-las-condiciones-del-discipulado-y-la-seriedad-de-seguir-a-jesuxfas}{%
\subsection{Sobre las condiciones del discipulado y la seriedad de
seguir a
Jesús}\label{sobre-las-condiciones-del-discipulado-y-la-seriedad-de-seguir-a-jesuxfas}}

\bibleverse{25} Iban con él grandes multitudes. Se volvió y les dijo:
\footnote{\textbf{14:25} Deut 33,9-10; Luc 18,29-30; 1Cor 7,29}
\bibleverse{26} ``Si alguien viene a mí y no se desentiende\footnote{\textbf{14:26}
  o, odio} de su padre, de su madre, de su mujer, de sus hijos, de sus
hermanos y hermanas, y también de su propia vida, no puede ser mi
discípulo. \bibleverse{27} El que no lleva su propia cruz y viene en pos
de mí, no puede ser mi discípulo. \footnote{\textbf{14:27} Luc 9,23}
\bibleverse{28} Porque ¿quién de vosotros, queriendo construir una
torre, no se sienta primero a contar lo que cuesta, para ver si tiene lo
suficiente para terminarla? \bibleverse{29} O acaso, cuando ha puesto
los cimientos y no puede terminar, todos los que lo ven comienzan a
burlarse de él, \bibleverse{30} diciendo: ``Este empezó a construir y no
pudo terminar''. \bibleverse{31} ¿O qué rey, cuando va a enfrentarse a
otro rey en la guerra, no se sienta primero a considerar si es capaz con
diez mil de enfrentarse al que viene contra él con veinte mil?
\bibleverse{32} O bien, estando el otro todavía muy lejos, envía un
enviado y pide condiciones de paz. \bibleverse{33} Así pues, cualquiera
de vosotros que no renuncie a todo lo que tiene, no puede ser mi
discípulo. \footnote{\textbf{14:33} Luc 9,62}

\bibleverse{34} ``La sal es buena, pero si la sal se vuelve plana e
insípida, ¿con qué la condimentas? \footnote{\textbf{14:34} Mat 5,13;
  Mar 9,50} \bibleverse{35} No sirve ni para la tierra ni para el montón
de estiércol. Se desecha. El que tenga oídos para oír, que oiga''.

\hypertarget{paruxe1bolas-de-la-oveja-perdida-y-la-dracma-perdida}{%
\subsection{Parábolas de la oveja perdida y la dracma
perdida}\label{paruxe1bolas-de-la-oveja-perdida-y-la-dracma-perdida}}

\hypertarget{section-14}{%
\section{15}\label{section-14}}

\bibleverse{1} Todos los recaudadores de impuestos y los pecadores se
acercaban a él para escucharle. \bibleverse{2} Los fariseos y los
escribas murmuraban diciendo: ``Este acoge a los pecadores y come con
ellos.'' \footnote{\textbf{15:2} Luc 5,30; Luc 19,7}

\bibleverse{3} Les contó esta parábola: \bibleverse{4} ``¿Quién de
vosotros, si tuviera cien ovejas y perdiera una de ellas, no dejaría las
noventa y nueve en el desierto e iría tras la que se perdió, hasta
encontrarla? \footnote{\textbf{15:4} Luc 19,10; Juan 10,11-12}
\bibleverse{5} Cuando la encuentra, la lleva sobre sus hombros,
alegrándose. \bibleverse{6} Cuando vuelve a casa, convoca a sus amigos y
a sus vecinos, diciéndoles: ``Alegraos conmigo, porque he encontrado mi
oveja que se había perdido''. \bibleverse{7} Os digo que así habrá más
alegría en el cielo por un solo pecador que se arrepienta, que por
noventa y nueve justos que no necesitan arrepentirse.

\bibleverse{8} ``¿O qué mujer, si tuviera diez \footnote{\textbf{15:8}
  Una moneda de dracma valía aproximadamente dos días de salario para un
  trabajador agrícola.} monedas de dracma, si perdiera una moneda de
dracma, no encendería una lámpara, barrería la casa y buscaría
diligentemente hasta encontrarla? \bibleverse{9} Cuando la encuentra,
convoca a sus amigos y vecinos, diciendo: ``¡Alégrense conmigo, porque
he encontrado la dracma que había perdido! \bibleverse{10} Así os digo
que hay alegría en presencia de los ángeles de Dios por un pecador que
se arrepiente.'' \footnote{\textbf{15:10} Efes 3,10}

\hypertarget{la-paruxe1bola-del-hijo-pruxf3digo}{%
\subsection{La parábola del hijo
pródigo}\label{la-paruxe1bola-del-hijo-pruxf3digo}}

\bibleverse{11} Dijo: ``Un hombre tenía dos hijos. \bibleverse{12} El
menor de ellos dijo a su padre: ``Padre, dame mi parte de tus bienes''.
Así que repartió su sustento entre ellos. \bibleverse{13} No muchos días
después, el hijo menor lo reunió todo y se fue a un país lejano. Allí
malgastó sus bienes con una vida desenfrenada. \footnote{\textbf{15:13}
  Prov 29,3} \bibleverse{14} Cuando lo hubo gastado todo, sobrevino una
gran hambruna en aquel país, y empezó a pasar necesidad. \bibleverse{15}
Fue y se unió a uno de los ciudadanos de aquel país, y éste lo envió a
sus campos para alimentar a los cerdos. \bibleverse{16} Quiso llenar su
vientre con las vainas que comían los cerdos, pero nadie le dio nada.
\footnote{\textbf{15:16} Prov 23,21} \bibleverse{17} Cuando volvió en
sí, dijo: ``¡Cuántos jornaleros de mi padre tienen pan de sobra, y yo me
muero de hambre! \bibleverse{18} Me levantaré, iré a ver a mi padre y le
diré: ``Padre, he pecado contra el cielo y ante tus ojos. \footnote{\textbf{15:18}
  Jer 3,12-13; Sal 51,4} \bibleverse{19} Ya no soy digno de ser llamado
hijo tuyo. Hazme como uno de tus jornaleros''.

\bibleverse{20} ``Se levantó y vino a su padre. Pero cuando aún estaba
lejos, su padre lo vio y se compadeció, corrió, se echó a su cuello y lo
besó. \bibleverse{21} El hijo le dijo: ``Padre, he pecado contra el
cielo y ante tus ojos. Ya no soy digno de ser llamado hijo tuyo''.

\bibleverse{22} ``Pero el padre dijo a sus siervos: ``Sacad el mejor
vestido y ponédselo. Ponedle un anillo en la mano y sandalias en los
pies. \bibleverse{23} Traed el ternero cebado, matadlo y comamos y
celebremos; \bibleverse{24} porque éste, mi hijo, estaba muerto y ha
vuelto a vivir. Se había perdido y se ha encontrado''. Entonces se
pusieron a celebrar. \footnote{\textbf{15:24} Efes 2,5}

\bibleverse{25} ``Su hijo mayor estaba en el campo. Al acercarse a la
casa, oyó música y danzas. \bibleverse{26} Llamó a uno de los criados y
le preguntó qué pasaba. \bibleverse{27} Este le dijo: ``Tu hermano ha
venido, y tu padre ha matado el ternero cebado, porque lo ha recibido
sano y salvo''. \bibleverse{28} Pero él se enfadó y no quiso entrar.
Entonces su padre salió y le rogó. \footnote{\textbf{15:28} Mat 20,15}
\bibleverse{29} Pero él respondió a su padre: `Mira, estos muchos años
te he servido, y nunca he desobedecido un mandamiento tuyo, pero nunca
me has dado un cabrito para que lo celebre con mis amigos.
\bibleverse{30} Pero cuando vino este hijo tuyo, que ha devorado tu
sustento con las prostitutas, mataste para él el ternero cebado'.

\bibleverse{31} ``Le dijo: `Hijo, tú estás siempre conmigo, y todo lo
mío es tuyo. \bibleverse{32} Peroera conveniente celebrar y alegrarse,
porque éste, tu hermano, estaba muerto y ha vuelto a vivir. Estaba
perdido y ha sido encontrado''.

\hypertarget{paruxe1bola-del-mayordomo-infiel-pero-sabio}{%
\subsection{Parábola del mayordomo infiel pero
sabio}\label{paruxe1bola-del-mayordomo-infiel-pero-sabio}}

\hypertarget{section-15}{%
\section{16}\label{section-15}}

\bibleverse{1} También dijo a sus discípulos: ``Había un hombre rico que
tenía un administrador. Se le acusó de que este hombre malgastaba sus
bienes. \bibleverse{2} Lo llamó y le dijo: ``¿Qué es lo que oigo de ti?
Da cuenta de tu gestión, porque ya no puedes ser administrador'.

\bibleverse{3} ``El gerente se dijo en su interior: `¿Qué voy a hacer,
viendo que mi señor me quita el puesto de gerente? No tengo fuerzas para
cavar. Me da vergüenza pedir limosna. \bibleverse{4} Ya sé lo que haré
para que, cuando me quiten la gerencia, me reciban en sus casas.'
\bibleverse{5} Llamando a cada uno de los deudores de su señor, le dijo
al primero: `¿Cuánto le debes a mi señor?' \bibleverse{6} Él respondió:
`Cien batos\footnote{\textbf{16:6} 100 batos son unos 395 litros o 104
  galones americanos.} de aceite.' Le dijo: ``Toma tu factura, siéntate
pronto y escribe cincuenta''. \bibleverse{7} Luego le dijo a otro:
``¿Cuánto debes? Le dijo: ``Cien cors \footnote{\textbf{16:7} 100 cors =
  unos 2.110 litros o 600 bushels.} de trigo''. Le dijo: ``Toma tu
cuenta y escribe ochenta''.

\bibleverse{8} ``Su señor elogió al administrador deshonesto porque
había actuado con sabiduría, pues los hijos de este mundo son, en su
propia generación, más sabios que los hijos de la luz. \footnote{\textbf{16:8}
  Efes 5,8-9} \bibleverse{9} Os digo que os hagáis amigos por medio de
las riquezas injustas, para que, cuando fracaséis, os reciban en las
tiendas eternas. \footnote{\textbf{16:9} Luc 14,14; Mat 6,20; Mat 19,21}

\hypertarget{de-lealtad}{%
\subsection{De lealtad}\label{de-lealtad}}

\bibleverse{10} El que es fiel en lo poco, lo es también en lo mucho. El
que es deshonesto en lo poco, también lo es en lo mucho. \footnote{\textbf{16:10}
  Luc 19,17} \bibleverse{11} Por tanto, si no habéis sido fieles en las
riquezas injustas, ¿quién os confiará las verdaderas? \bibleverse{12} Si
no has sido fiel en lo ajeno, ¿quién te dará lo propio? \bibleverse{13}
Ningún siervo puede servir a dos amos, pues o aborrece a uno y ama al
otro, o se aferra a uno y desprecia al otro. No puedes servir a Dios y a
Mammón.''\footnote{\textbf{16:13} ``Mamón'' se refiere a las riquezas o
  a un falso dios de la riqueza.} \footnote{\textbf{16:13} Mat 6,24}

\hypertarget{jesuxfas-reprende-a-los-fariseos-codiciosos-y-burladores}{%
\subsection{Jesús reprende a los fariseos codiciosos y
burladores}\label{jesuxfas-reprende-a-los-fariseos-codiciosos-y-burladores}}

\bibleverse{14} También los fariseos, amantes del dinero, oyeron todo
esto y se burlaron de él. \bibleverse{15} Él les dijo: ``Vosotros sois
los que os justificáis ante los hombres, pero Dios conoce vuestros
corazones. Porque lo que se enaltece entre los hombres es una
abominación a los ojos de Dios. \footnote{\textbf{16:15} Luc 18,9-14}

\bibleverse{16} ``La ley y los profetas eran hasta Juan. Desde entonces
se predica la Buena Nueva del Reino de Dios, y todo el mundo entra en él
a la fuerza. \footnote{\textbf{16:16} Mat 11,12-13} \bibleverse{17} Pero
es más fácil que desaparezcan el cielo y la tierra que un pequeño trazo
de la ley. \footnote{\textbf{16:17} Mat 5,18}

\bibleverse{18} ``Todo el que se divorcia de su mujer y se casa con otra
comete adulterio. El que se casa con una divorciada del marido comete
adulterio. \footnote{\textbf{16:18} Mat 5,32; Mat 19,9}

\hypertarget{historia-del-rico-y-el-pobre-luxe1zaro}{%
\subsection{Historia del rico y el pobre
Lázaro}\label{historia-del-rico-y-el-pobre-luxe1zaro}}

\bibleverse{19} ``Había un hombre rico, que se vestía de púrpura y de
lino fino, y vivía cada día con lujo. \bibleverse{20} Un mendigo,
llamado Lázaro, fue llevado a su puerta, lleno de llagas,
\bibleverse{21} y deseando ser alimentado con las migajas que caían de
la mesa del rico. Y hasta los perros vinieron a lamerle las llagas.
\bibleverse{22} El mendigo murió y fue llevado por los ángeles al seno
de Abraham. También el rico murió y fue enterrado. \bibleverse{23} En el
Hades,\footnote{\textbf{16:23} o, Infierno} levantó los ojos, estando
atormentado, y vio a Abraham a lo lejos, y a Lázaro a su lado.
\bibleverse{24} Y llorando dijo: ``Padre Abraham, ten piedad de mí y
envía a Lázaro, para que moje la punta de su dedo en agua y refresque mi
lengua. Porque estoy angustiado en esta llama'.

\bibleverse{25} ``Pero Abraham le dijo: `Hijo, acuérdate de que tú,
durante tu vida, recibiste tus cosas buenas, y Lázaro, del mismo modo,
cosas malas. Pero aquí está ahora consolado y tú estás angustiado.
\footnote{\textbf{16:25} Luc 6,24} \bibleverse{26} Además de todo esto,
entre nosotros y vosotros hay fijado un gran abismo, de modo que los que
quieren pasar de aquí a vosotros no pueden, y nadie puede cruzar de allí
a nosotros.'

\bibleverse{27} ``Dijo: `Te pido, pues, padre, que lo envíes a la casa
de mi padre --- \bibleverse{28} porque tengo cinco hermanos --- para que
les dé testimonio, y no vengan también a este lugar de tormento'.

\bibleverse{29} ``Pero Abraham le dijo: `Tienen a Moisés y a los
profetas. Que los escuchen'. \footnote{\textbf{16:29} 2Tim 3,16}

\bibleverse{30} ``Él dijo: `No, padre Abraham, pero si uno va a ellos de
entre los muertos, se arrepentirán'.

\bibleverse{31} ``Le dijo: ``Si no escuchan a Moisés y a los profetas,
tampoco se convencerán si uno se levanta de entre los muertos''\,''.

\hypertarget{advertencia-contra-la-seducciuxf3n-y-recordatorio-de-perdonar}{%
\subsection{Advertencia contra la seducción y recordatorio de
perdonar}\label{advertencia-contra-la-seducciuxf3n-y-recordatorio-de-perdonar}}

\hypertarget{section-16}{%
\section{17}\label{section-16}}

\bibleverse{1} Dijo a los discípulos: ``Es imposible que no vengan
ocasiones de tropiezo, pero ¡ay de aquel por quien vienen!
\bibleverse{2} Más le valdría que le colgaran al cuello una piedra de
molino y lo arrojaran al mar, que hacer tropezar a uno de estos
pequeños. \footnote{\textbf{17:2} Mat 18,6-7} \bibleverse{3} Tened
cuidado. Si tu hermano peca contra ti, repréndelo. Si se arrepiente,
perdónalo. \bibleverse{4} Si peca contra ti siete veces en el día, y
siete veces vuelve diciendo: ``Me arrepiento'', le perdonarás.''
\footnote{\textbf{17:4} Mat 18,15; Mat 18,21-22}

\hypertarget{del-poder-de-la-fe}{%
\subsection{Del poder de la fe}\label{del-poder-de-la-fe}}

\bibleverse{5} Los apóstoles dijeron al Señor: ``Aumenta nuestra fe''.

\bibleverse{6} El Señor dijo: ``Si tuvieras fe como un grano de mostaza,
le dirías a este sicómoro: `Arráncate y plántate en el mar', y te
obedecería. \footnote{\textbf{17:6} Mat 17,20; Mat 21,21}

\hypertarget{paruxe1bola-del-seuxf1or-y-su-siervo-comprometido-a-trabajar}{%
\subsection{Parábola del Señor y su siervo comprometido a
trabajar}\label{paruxe1bola-del-seuxf1or-y-su-siervo-comprometido-a-trabajar}}

\bibleverse{7} Pero, ¿quién hay entre vosotros que tenga un siervo
arando o guardando ovejas, que le diga al llegar del campo: ``Ven
enseguida y siéntate a la mesa''? \bibleverse{8} ¿No le dirá más bien:
`Prepara mi cena, vístete bien y sírveme mientras como y bebo. Después
comerás y beberás'? \bibleverse{9} ¿Acaso le da las gracias a ese siervo
porque hizo lo que se le ordenó? Creo que no. \bibleverse{10} Así
también vosotros, cuando hayáis hecho todo lo que se os ha mandado,
decid: `Somos siervos indignos. Hemos cumplido con nuestro deber'\,''.
\footnote{\textbf{17:10} 1Cor 9,16}

\hypertarget{curaciuxf3n-de-diez-leprosos-el-samaritano-agradecido}{%
\subsection{Curación de diez leprosos; el samaritano
agradecido}\label{curaciuxf3n-de-diez-leprosos-el-samaritano-agradecido}}

\bibleverse{11} Cuando se dirigía a Jerusalén, pasaba por los límites de
Samaria y Galilea. \footnote{\textbf{17:11} Luc 9,51; Luc 13,22}
\bibleverse{12} Al entrar en una aldea, le salieron al encuentro diez
hombres que eran leprosos y que estaban a distancia. \footnote{\textbf{17:12}
  Lev 13,45-46} \bibleverse{13} Levantaron la voz diciendo: ``Jesús,
Maestro, ten piedad de nosotros''.

\bibleverse{14} Al verlos, les dijo: ``Vayan y muéstrense a los
sacerdotes''. Mientras iban, quedaron limpios. \footnote{\textbf{17:14}
  Luc 5,14} \bibleverse{15} Uno de ellos, al ver que estaba sanado, se
volvió glorificando a Dios a gran voz. \bibleverse{16} Se postró a los
pies de Jesús dándole gracias; era un samaritano.

\bibleverse{17} Jesús respondió: ``¿No quedaron limpios los diez? Pero,
¿dónde están los nueve? \bibleverse{18} ¿No se encontró a ninguno que
volviera a dar gloria a Dios, sino a este extranjero?'' \bibleverse{19}
Entonces le dijo: ``Levántate y vete. Tu fe te ha sanado''. \footnote{\textbf{17:19}
  Luc 7,50}

\hypertarget{de-la-venida-del-reino-de-dios}{%
\subsection{De la venida del reino de
Dios}\label{de-la-venida-del-reino-de-dios}}

\bibleverse{20} Cuando los fariseos le preguntaron cuándo vendría el
Reino de Dios, les contestó: ``El Reino de Dios no viene con la
observación; \footnote{\textbf{17:20} Juan 18,36} \bibleverse{21}
tampoco dirán: ``¡Mira, aquí!'' o ``¡Mira, allí!'', porque he aquí que
el Reino de Dios está dentro de vosotros.''

\bibleverse{22} Dijo a los discípulos: ``Vendrán días en que desearéis
ver uno de los días del Hijo del Hombre, y no lo veréis. \bibleverse{23}
Os dirán: ``¡Mira, aquí!'' o ``¡Mira, allí!''. No os vayáis ni les
sigáis, \footnote{\textbf{17:23} Luc 21,8} \bibleverse{24} porque como
el rayo, cuando sale de una parte bajo el cielo, brilla hacia otra parte
bajo el cielo, así será el Hijo del Hombre en su día. \bibleverse{25}
Pero primero tiene que sufrir muchas cosas y ser rechazado por esta
generación. \footnote{\textbf{17:25} Luc 9,22} \bibleverse{26} Como fue
en los días de Noé, así será también en los días del Hijo del Hombre.
\bibleverse{27} Comían, bebían, se casaban y se daban en matrimonio
hasta el día en que Noé entró en la nave, y vino el diluvio y los
destruyó a todos. \footnote{\textbf{17:27} Gén 6,1-8} \bibleverse{28}
Asimismo, como en los días de Lot: comían, bebían, compraban, vendían,
plantaban y construían; \bibleverse{29} pero el día en que Lot salió de
Sodoma, llovió fuego y azufre del cielo y los destruyó a todos.
\footnote{\textbf{17:29} Gén 19,15; Gén 19,24-25} \bibleverse{30} Lo
mismo sucederá el día en que se manifieste el Hijo del Hombre.
\bibleverse{31} En aquel día, el que esté en la azotea y sus bienes en
la casa, que no baje a llevárselos. Que el que esté en el campo tampoco
se vuelva atrás. \bibleverse{32} ¡Acuérdate de la mujer de Lot!
\footnote{\textbf{17:32} Gén 19,26} \bibleverse{33} El que busca salvar
su vida la pierde, pero el que la pierde la conserva. \footnote{\textbf{17:33}
  Luc 9,24} \bibleverse{34} Os digo que en aquella noche habrá dos
personas en una cama. Uno será tomado y el otro será dejado.
\bibleverse{35} Habrá dos que molerán juntos el grano. Uno será tomado y
el otro será dejado''. \bibleverse{36} \footnote{\textbf{17:36} Algunos
  manuscritos griegos añaden: ``Dos estarán en el campo: el uno tomado y
  el otro dejado''.}

\bibleverse{37} Ellos, respondiendo, le preguntaron: ``¿Dónde, Señor?''.
Les dijo: ``Donde esté el cuerpo, allí se reunirán también los
buitres''.

\hypertarget{paruxe1bola-del-juez-impuxedo-y-la-viuda-suplicante}{%
\subsection{Parábola del juez impío y la viuda
suplicante}\label{paruxe1bola-del-juez-impuxedo-y-la-viuda-suplicante}}

\hypertarget{section-17}{%
\section{18}\label{section-17}}

\bibleverse{1} También les contó una parábola para que oraran siempre y
no se dieran por vencidos, \footnote{\textbf{18:1} 1Tes 5,17}
\bibleverse{2} diciendo: ``Había un juez en cierta ciudad que no temía a
Dios ni respetaba a los hombres. \bibleverse{3} En aquella ciudad había
una viuda que acudía a menudo a él diciendo: ``Defiéndeme de mi
adversario''. \bibleverse{4} Él no quiso hacerlo durante un tiempo; pero
después se dijo a sí mismo: `Aunque no temo a Dios ni respeto a los
hombres, \bibleverse{5} sin embargo, como esta viuda me molesta, la
defenderé, o de lo contrario me agotará con sus continuas visitas.'\,''
\footnote{\textbf{18:5} Luc 11,7-8}

\bibleverse{6} El Señor dijo: ``Escuchen lo que dice el juez injusto.
\bibleverse{7} ¿No va a vengar Dios a sus elegidos, que claman a él día
y noche, y sin embargo tiene paciencia con ellos? \bibleverse{8} Os digo
que los vengará pronto. Sin embargo, cuando venga el Hijo del Hombre,
¿encontrará fe en la tierra?''

\hypertarget{la-paruxe1bola-del-fariseo-y-el-recaudador-de-impuestos}{%
\subsection{La parábola del fariseo y el recaudador de
impuestos}\label{la-paruxe1bola-del-fariseo-y-el-recaudador-de-impuestos}}

\bibleverse{9} También dijo esta parábola a ciertas personas que estaban
convencidas de su propia justicia y que despreciaban a todos los demás:
\footnote{\textbf{18:9} Rom 10,3; Mat 5,6} \bibleverse{10} ``Dos hombres
subieron al templo a orar; uno era fariseo y el otro recaudador de
impuestos. \bibleverse{11} El fariseo se puso de pie y oró a solas así
`Dios, te doy gracias porque no soy como los demás hombres:
extorsionadores, injustos, adúlteros, ni tampoco como este recaudador de
impuestos. \footnote{\textbf{18:11} Is 58,2-3} \bibleverse{12} Ayuno dos
veces por semana. Doy el diezmo de todo lo que recibo''. \footnote{\textbf{18:12}
  Mat 23,23} \bibleverse{13} Pero el recaudador de impuestos, que estaba
lejos, ni siquiera alzaba los ojos al cielo, sino que se golpeaba el
pecho diciendo: ``¡Dios, ten piedad de mí, que soy un pecador!
\footnote{\textbf{18:13} Sal 51,17} \bibleverse{14} Os digo que éste
bajó a su casa justificado antes que el otro; porque todo el que se
enaltece será humillado, pero el que se humilla será enaltecido.''
\footnote{\textbf{18:14} Luc 14,11; Mat 21,3; Mat 23,12}

\hypertarget{jesuxfas-bendice-a-los-niuxf1os}{%
\subsection{Jesús bendice a los
niños}\label{jesuxfas-bendice-a-los-niuxf1os}}

\bibleverse{15} También le traían sus bebés para que los tocara. Pero
los discípulos, al verlo, los reprendieron. \bibleverse{16} Jesús los
llamó, diciendo: ``Dejad que los niños vengan a mí y no se lo impidáis,
porque el Reino de Dios es de los que son como ellos. \bibleverse{17} Os
aseguro que el que no reciba el Reino de Dios como un niño, no entrará
en él.''

\hypertarget{del-peligro-de-la-riqueza}{%
\subsection{Del peligro de la riqueza}\label{del-peligro-de-la-riqueza}}

\bibleverse{18} Un gobernante le preguntó: ``Maestro bueno, ¿qué debo
hacer para heredar la vida eterna?'' \footnote{\textbf{18:18} Luc
  10,25-28}

\bibleverse{19} Jesús le preguntó: ``¿Por qué me llamas bueno? Nadie es
bueno, sino uno: Dios. \bibleverse{20} Tú conoces los mandamientos: `No
cometerás adulterio', `No matarás', `No robarás', `No darás falso
testimonio', `Honra a tu padre y a tu madre'.'' \footnote{\textbf{18:20}
  Éxodo 20:12-16; Deuteronomio 5:16-20} \footnote{\textbf{18:20} Éxod
  20,12-16}

\bibleverse{21} Dijo: ``He observado todas estas cosas desde mi
juventud''.

\bibleverse{22} Al oír esto, Jesús le dijo: ``Todavía te falta una cosa.
Vende todo lo que tienes y repártelo entre los pobres. Así tendrás un
tesoro en el cielo; entonces ven y sígueme''. \footnote{\textbf{18:22}
  Mat 6,20}

\bibleverse{23} Pero al oír estas cosas, se puso muy triste, porque era
muy rico.

\bibleverse{24} Jesús, viendo que se ponía muy triste, dijo: ``¡Qué
difícil es para los que tienen riquezas entrar en el Reino de Dios!
\footnote{\textbf{18:24} Luc 19,9} \bibleverse{25} Porque es más fácil
que un camello entre por el ojo de una aguja que un rico entre en el
Reino de Dios.''

\bibleverse{26} Los que lo oyeron dijeron: ``Entonces, ¿quién puede
salvarse?''.

\bibleverse{27} Pero él dijo: ``Lo que es imposible para los hombres es
posible para Dios''.

\hypertarget{la-recompensa-de-seguir-a-jesuxfas}{%
\subsection{La recompensa de seguir a
Jesús}\label{la-recompensa-de-seguir-a-jesuxfas}}

\bibleverse{28} Pedro dijo: ``Mira, lo hemos dejado todo y te hemos
seguido''.

\bibleverse{29} Les dijo: ``Os aseguro que no hay nadie que haya dejado
casa, o mujer, o hermanos, o padres, o hijos, por el Reino de Dios,
\bibleverse{30} que no reciba muchas veces más en este tiempo, y en el
mundo venidero, la vida eterna.''

\hypertarget{huele-a-jerusaluxe9n-tercer-anuncio-del-sufrimiento-de-jesuxfas}{%
\subsection{Huele a Jerusalén; tercer anuncio del sufrimiento de
Jesús}\label{huele-a-jerusaluxe9n-tercer-anuncio-del-sufrimiento-de-jesuxfas}}

\bibleverse{31} Tomó aparte a los doce y les dijo: ``Mirad, vamos a
subir a Jerusalén, y se cumplirán todas las cosas que están escritas por
los profetas acerca del Hijo del Hombre. \footnote{\textbf{18:31} Luc
  9,22; Luc 9,44; Is 53,1} \bibleverse{32} Porque será entregado a los
gentiles, será escarnecido, tratado con vergüenza y escupido.
\bibleverse{33} Lo azotarán y lo matarán. Al tercer día resucitará''.

\bibleverse{34} No entendieron nada de esto. Este dicho se les ocultó, y
no entendieron las cosas que se decían. \footnote{\textbf{18:34} Luc
  9,45; Luc 24,45}

\hypertarget{la-curaciuxf3n-del-ciego-en-jericuxf3}{%
\subsection{La curación del ciego en
Jericó}\label{la-curaciuxf3n-del-ciego-en-jericuxf3}}

\bibleverse{35} Al llegar a Jericó, un ciego estaba sentado junto al
camino, pidiendo limosna. \bibleverse{36} Al oír pasar una multitud,
preguntó qué significaba aquello. \bibleverse{37} Le dijeron que pasaba
Jesús de Nazaret. \bibleverse{38} Él gritó: ``¡Jesús, hijo de David, ten
piedad de mí!''. \bibleverse{39} Los que iban delante le reprendieron
para que se callara; pero él gritó aún más: ``¡Hijo de David, ten
compasión de mí!''

\bibleverse{40} Parado, Jesús mandó que lo trajeran hacia él. Cuando se
hubo acercado, le preguntó: \bibleverse{41} ``¿Qué quieres que haga?''.
Dijo: ``Señor, que vuelva a ver''.

\bibleverse{42} Jesús le dijo: ``Recibe la vista. Tu fe te ha sanado ''.
\footnote{\textbf{18:42} Luc 17,19}

\bibleverse{43} Inmediatamente recibió la vista y lo siguió,
glorificando a Dios. Todo el pueblo, al verlo, alabó a Dios.

\hypertarget{la-visita-de-jesuxfas-al-principal-recaudador-de-impuestos-zaqueo-en-jericuxf3}{%
\subsection{La visita de Jesús al principal recaudador de impuestos
Zaqueo en
Jericó}\label{la-visita-de-jesuxfas-al-principal-recaudador-de-impuestos-zaqueo-en-jericuxf3}}

\hypertarget{section-18}{%
\section{19}\label{section-18}}

\bibleverse{1} Entró y pasó por Jericó. \bibleverse{2} Había un hombre
llamado Zaqueo. Era un jefe de los recaudadores de impuestos, y era
rico. \bibleverse{3} Trataba de ver quién era Jesús, y no podía a causa
de la multitud, porque era de baja estatura. \bibleverse{4} Se adelantó
corriendo y se subió a un sicómoro para verlo, pues iba a pasar por
allí. \bibleverse{5} Cuando Jesús llegó al lugar, levantó la vista y lo
vio, y le dijo: ``Zaqueo, date prisa en bajar, porque hoy tengo que
quedarme en tu casa.'' \bibleverse{6} Él se apresuró, bajó y lo recibió
con alegría. \bibleverse{7} Al verlo, todos murmuraron, diciendo: ``Ha
entrado a hospedarse con un hombre que es pecador.'' \footnote{\textbf{19:7}
  Luc 15,2}

\bibleverse{8} Zaqueo se puso en pie y dijo al Señor: ``Mira, Señor, la
mitad de mis bienes la doy a los pobres. Si a alguien le he exigido algo
injustamente, le devuelvo cuatro veces más''. \footnote{\textbf{19:8}
  Éxod 22,1; Ezeq 33,14-16}

\bibleverse{9} Jesús le dijo: ``Hoy ha llegado la salvación a esta casa,
porque también él es hijo de Abraham. \footnote{\textbf{19:9} Luc 13,16}
\bibleverse{10} Porque el Hijo del Hombre ha venido a buscar y a salvar
lo que se había perdido''. \footnote{\textbf{19:10} Luc 5,32; Ezeq
  34,16; 1Tim 1,15}

\hypertarget{la-paruxe1bola-de-las-minas-confiadas}{%
\subsection{La parábola de las minas
confiadas}\label{la-paruxe1bola-de-las-minas-confiadas}}

\bibleverse{11} Al oír estas cosas, prosiguió y contó una parábola,
porque estaba cerca de Jerusalén, y ellos suponían que el Reino de Dios
se revelaría inmediatamente. \bibleverse{12} Dijo, pues: ``Cierto noble
se fue a un país lejano para recibir para sí un reino y regresar.
\bibleverse{13} Llamó a diez siervos suyos y les dio diez monedas de
mina, y les \footnote{\textbf{19:13} 10 minas eran más de 3 años de
  salario para un trabajador agrícola.} dijo: ``Ocúpense de los negocios
hasta que yo llegue''. \bibleverse{14} Pero sus ciudadanos lo odiaban y
enviaron un enviado tras él, diciendo: `No queremos que este hombre
reine sobre nosotros.' \footnote{\textbf{19:14} Juan 1,11}

\bibleverse{15} ``Cuando regresó de nuevo, habiendo recibido el reino,
mandó llamar a estos siervos, a los que había dado el dinero, para que
supiera lo que habían ganado haciendo negocios. \bibleverse{16} El
primero se presentó ante él, diciendo: ``Señor, tu mina ha hecho diez
minas más''.

\bibleverse{17} ``Le dijo: ``¡Bien hecho, buen siervo! Porque has sido
hallado fiel con muy poco, tendrás autoridad sobre diez ciudades'.
\footnote{\textbf{19:17} Luc 16,10}

\bibleverse{18} ``El segundo vino diciendo: `Tu mina, Señor, ha hecho
cinco minas'.

\bibleverse{19} ``Entonces le dijo: `Y tú vas a estar sobre cinco
ciudades'.

\bibleverse{20} Vino otro diciendo: `Señor, he aquí tu mina, que yo
guardaba en un pañuelo, \bibleverse{21} pues te temía, porque eres un
hombre exigente. Recoges lo que no pusiste, y cosechas lo que no
sembraste'.

\bibleverse{22} ``Le dijo: `¡De tu propia boca te juzgaré, siervo
malvado! Sabías que soy un hombre exigente, que tomo lo que no dejé y
cosecho lo que no sembré. \bibleverse{23} Entonces, ¿por qué no
depositaste mi dinero en el banco, y al llegar yo, podría haber ganado
intereses por él?' \bibleverse{24} Y dijo a los que estaban allí:
`Quitadle la mina y dadla al que tiene las diez minas'.

\bibleverse{25} ``Le dijeron: `¡Señor, tiene diez minas! \bibleverse{26}
`Porque yo os digo que a todo el que tiene, se le dará más; pero al que
no tiene, se le quitará hasta lo que tiene. \footnote{\textbf{19:26} Luc
  8,18; Mat 13,12} \bibleverse{27} Pero traed aquí a esos enemigos míos
que no querían que reinara sobre ellos, y matadlos delante de mí.'\,''

\hypertarget{jesuxfas-a-las-puertas-de-jerusaluxe9n-su-entrada-en-jerusaluxe9n}{%
\subsection{Jesús a las puertas de Jerusalén; su entrada en
Jerusalén}\label{jesuxfas-a-las-puertas-de-jerusaluxe9n-su-entrada-en-jerusaluxe9n}}

\bibleverse{28} Dicho esto, siguió adelante, subiendo a Jerusalén.
\footnote{\textbf{19:28} Juan 2,13}

\bibleverse{29} Cuando se acercó a Betfagé\footnote{\textbf{19:29} TR,
  NU leer ``Bethpage'' en lugar de ``Bethsphage''} y a Betania, en el
monte que se llama del Olivar, envió a dos de sus discípulos,
\bibleverse{30} diciendo: ``Id a la aldea del otro lado, en la que, al
entrar, encontraréis un pollino atado, en el que nadie se ha sentado
jamás. Desátenlo y tráiganlo. \bibleverse{31} Si alguien os pregunta:
``¿Por qué lo desatáis el pollino?'', decidle: ``El Señor lo
necesita''.''

\bibleverse{32} Los enviados se fueron y encontraron las cosas tal como
él les había dicho. \bibleverse{33} Mientras desataban el potro, sus
dueños les dijeron: ``¿Por qué desatáis el potro?''. \bibleverse{34}
Ellos respondieron: ``El Señor lo necesita''. \bibleverse{35} Entonces
se lo llevaron a Jesús. Echaron sus mantos sobre el pollino y sentaron a
Jesús sobre ellos. \bibleverse{36} Mientras él iba, extendieron sus
mantos en el camino.

\bibleverse{37} Cuando ya se acercaba, al bajar del Monte de los Olivos,
toda la multitud de los discípulos comenzó a alegrarse y a alabar a Dios
a gran voz por todas las maravillas que habían visto, \bibleverse{38}
diciendo: ``¡Bendito el Rey que viene en nombre del Señor! \footnote{\textbf{19:38}
  Salmo 118:26} Paz en el cielo y gloria en las alturas''. \footnote{\textbf{19:38}
  Luc 2,14; Sal 118,26}

\bibleverse{39} Algunos fariseos de la multitud le dijeron: ``Maestro,
reprende a tus discípulos''.

\bibleverse{40} Él les respondió: ``Os digo que si éstos callaran, las
piedras gritarían''.

\hypertarget{jesuxfas-llora-por-jerusaluxe9n-y-profecuxeda-de-la-destrucciuxf3n-de-jerusaluxe9n}{%
\subsection{Jesús llora por Jerusalén y profecía de la destrucción de
Jerusalén}\label{jesuxfas-llora-por-jerusaluxe9n-y-profecuxeda-de-la-destrucciuxf3n-de-jerusaluxe9n}}

\bibleverse{41} Cuando se acercó, vio la ciudad y lloró por ella,
\bibleverse{42} diciendo: ``¡Si tú, incluso tú, hubieras sabido hoy las
cosas que pertenecen a tu paz! Pero ahora están ocultas a tus ojos.
\footnote{\textbf{19:42} Luc 13,34; Deut 32,29; Mat 13,14}
\bibleverse{43} Porque vendrán días en que tus enemigos levantarán una
barricada contra ti, te rodearán, te cercarán por todos lados,
\bibleverse{44} y te derribarán a ti y a tus hijos dentro de ti. No
dejarán en ti una piedra sobre otra, porque no conociste el tiempo de tu
visitación''. \footnote{\textbf{19:44} Luc 21,6}

\hypertarget{jesuxfas-limpiando-el-templo}{%
\subsection{Jesús limpiando el
templo}\label{jesuxfas-limpiando-el-templo}}

\bibleverse{45} Entró en el templo y comenzó a expulsar a los que
compraban y vendían en él, \bibleverse{46} diciéndoles: ``Está escrito:
``Mi casa es una casa de oración\footnote{\textbf{19:46} Isaías 56:7}
'', pero vosotros la habéis convertido en una ``cueva de ladrones''.''
\footnote{\textbf{19:46} Jeremías 7:11} \footnote{\textbf{19:46} Jer
  7,11}

\bibleverse{47} Cada día enseñaba en el templo, pero los jefes de los
sacerdotes, los escribas y los principales hombres del pueblo trataban
de matarle. \bibleverse{48} No hallaban como hacerlo, porque todo el
pueblo se aferraba a cada palabra que él decía.

\hypertarget{la-pregunta-del-sumo-consejo-sobre-la-autoridad-de-jesuxfas}{%
\subsection{La pregunta del sumo consejo sobre la autoridad de
Jesús}\label{la-pregunta-del-sumo-consejo-sobre-la-autoridad-de-jesuxfas}}

\hypertarget{section-19}{%
\section{20}\label{section-19}}

\bibleverse{1} Uno de esos días, mientras enseñaba al pueblo en el
templo y predicaba la Buena Nueva, se le acercaron los \footnote{\textbf{20:1}
  TR añade ``jefe''} sacerdotes y los escribas con los ancianos.
\bibleverse{2} Le preguntaron: ``Dinos: ¿con qué autoridad haces estas
cosas? ¿O quién te da esta autoridad?''

\bibleverse{3} Él les respondió: ``Yo también os haré una pregunta.
Decidme: \bibleverse{4} el bautismo de Juan, ¿era del cielo o de los
hombres?''

\bibleverse{5} Ellos razonaban entre sí, diciendo: ``Si decimos: ``Del
cielo'', nos dirá: ``¿Por qué no le habéis creído?'' \footnote{\textbf{20:5}
  Luc 7,29-30} \bibleverse{6} Pero si decimos: ``De los hombres'', todo
el pueblo nos apedreará, porque están persuadidos de que Juan era un
profeta.'' \bibleverse{7} Ellos respondieron que no sabían de dónde
venía.

\bibleverse{8} Jesús les dijo: ``Tampoco os diré con qué autoridad hago
estas cosas''.

\hypertarget{la-paruxe1bola-de-los-viticultores-infieles}{%
\subsection{La parábola de los viticultores
infieles}\label{la-paruxe1bola-de-los-viticultores-infieles}}

\bibleverse{9} Comenzó a contar a la gente esta parábola: ``\footnote{\textbf{20:9}
  NU (entre paréntesis) y TR añaden ``cierto''} Un hombre plantó una
viña, la alquiló a unos agricultores y se fue a otro país durante mucho
tiempo. \bibleverse{10} A su debido tiempo, envió un criado a los
agricultores para que recogiera su parte del fruto de la viña. Pero los
campesinos lo golpearon y lo despidieron con las manos vacías.
\footnote{\textbf{20:10} 2Cró 36,15-16} \bibleverse{11} Envió a otro
siervo, pero también lo golpearon y lo trataron de forma vergonzosa, y
lo despidieron con las manos vacías. \bibleverse{12} Envió a un tercero,
y también lo hirieron y lo echaron. \bibleverse{13} El señor de la viña
dijo: ``¿Qué voy a hacer? Enviaré a mi hijo amado. Puede ser que, al
verlo, lo respeten'.

\bibleverse{14} ``Pero cuando los campesinos lo vieron, razonaron entre
ellos, diciendo: `Este es el heredero. Vamos, matémoslo, para que la
herencia sea nuestra'. \bibleverse{15} Entonces lo echaron de la viña y
lo mataron. ¿Qué hará, pues, el señor de la viña con ellos?
\bibleverse{16} Vendrá y destruirá a estos labradores, y dará la viña a
otros''. Cuando lo oyeron, dijeron: ``¡Que nunca sea así!''.

\bibleverse{17} Pero él los miró y dijo: ``Entonces, ¿qué es esto que
está escrito, La piedra que desecharon los constructores se convirtió en
la principal piedra angular''. \footnote{\textbf{20:17} Salmo 118:22}
\bibleverse{18} Todo el que caiga sobre esa piedra se hará pedazos, pero
aplastará a quien caiga en polvo''.

\bibleverse{19} Los jefes de los sacerdotes y los escribas trataron de
echarle mano en aquella misma hora, pero temían al pueblo, pues sabían
que había dicho esta parábola contra ellos. \footnote{\textbf{20:19} Luc
  19,48}

\hypertarget{la-cuestiuxf3n-fiscal-de-los-fariseos}{%
\subsection{La cuestión fiscal de los
fariseos}\label{la-cuestiuxf3n-fiscal-de-los-fariseos}}

\bibleverse{20} Lo vigilaban y enviaron espías, que se hacían pasar por
justos, para atraparlo en algo que dijera, a fin de entregarlo al poder
y a la autoridad del gobernador. \footnote{\textbf{20:20} Luc 11,54}
\bibleverse{21} Le preguntaron: ``Maestro, sabemos que dices y enseñas
lo que es justo, y que no eres parcial con nadie, sino que enseñas
verdaderamente el camino de Dios. \bibleverse{22} ¿Nos es lícito pagar
impuestos al César, o no?''

\bibleverse{23} Pero él, al darse cuenta de su astucia, les dijo: ``¿Por
qué me ponéis a prueba? \bibleverse{24} Muéstrenme un denario. ¿De quién
es la imagen y la inscripción que lleva?'' Ellos respondieron: ``Del
César''.

\bibleverse{25} Les dijo: ``Dad al César lo que es del César y a Dios lo
que es de Dios''. \footnote{\textbf{20:25} Rom 13,1; Rom 13,7; Hech 5,29}

\bibleverse{26} No pudieron atraparlo en sus palabras ante el pueblo. Se
maravillaron de su respuesta y guardaron silencio.

\hypertarget{sobre-la-resurrecciuxf3n-de-los-muertos}{%
\subsection{Sobre la resurrección de los
muertos}\label{sobre-la-resurrecciuxf3n-de-los-muertos}}

\bibleverse{27} Se le acercaron algunos de los saduceos, los que niegan
que haya resurrección. \bibleverse{28} Le preguntaron: ``Maestro, Moisés
nos escribió que si el hermano de un hombre muere teniendo esposa y no
tiene hijos, su hermano debe tomar la esposa y criar hijos para su
hermano. \bibleverse{29} Había, pues, siete hermanos. El primero tomó
una esposa y murió sin hijos. \bibleverse{30} El segundo la tomó como
esposa, y murió sin hijos. \bibleverse{31} El tercero la tomó, e
igualmente los siete no dejaron hijos, y murieron. \bibleverse{32}
Después murió también la mujer. \bibleverse{33} Por tanto, en la
resurrección, ¿de quién será ella la esposa? Porque los siete la
tuvieron como esposa''.

\bibleverse{34} Jesús les dijo: ``Los hijos de este siglo se casan y se
dan en matrimonio. \bibleverse{35} Pero los que son considerados dignos
de llegar a esa edad y a la resurrección de los muertos ni se casan ni
se dan en matrimonio. \bibleverse{36} Porque ya no pueden morir, pues
son como los ángeles y son hijos de Dios, siendo hijos de la
resurrección. \footnote{\textbf{20:36} 1Jn 3,1; 1Jn 1,3-2}
\bibleverse{37} Pero que los muertos resucitan, lo demostró también
Moisés en la zarza, cuando llamó al Señor `El Dios de Abraham, el Dios
de Isaac y el Dios de Jacob'. \footnote{\textbf{20:37} Éxodo 3:6}
\bibleverse{38} Ahora bien, no es el Dios de los muertos, sino de los
vivos, pues todos están vivos para él.'' \footnote{\textbf{20:38} Rom
  14,8}

\bibleverse{39} Algunos de los escribas respondieron: ``Maestro, hablas
bien''. \bibleverse{40} No se atrevieron a hacerle más preguntas.

\hypertarget{la-contrapregunta-de-jesuxfas-sobre-el-mesuxedas-como-hijo-de-david}{%
\subsection{La contrapregunta de Jesús sobre el Mesías como hijo de
David}\label{la-contrapregunta-de-jesuxfas-sobre-el-mesuxedas-como-hijo-de-david}}

\bibleverse{41} Les dijo: ``¿Por qué dicen que el Cristo es hijo de
David? \bibleverse{42} El mismo David dice en el libro de los Salmos,
`El Señor dijo a mi Señor, ``Siéntate a mi derecha, \bibleverse{43}
hasta que haga de tus enemigos el escabel de tus pies''. \footnote{\textbf{20:43}
  Salmo 110:1}

\bibleverse{44} ``Por lo tanto, David lo llama Señor, ¿cómo es su
hijo?''

\hypertarget{advertencia-de-jesuxfas-sobre-la-ambiciuxf3n-y-la-codicia-de-los-escribas}{%
\subsection{Advertencia de Jesús sobre la ambición y la codicia de los
escribas}\label{advertencia-de-jesuxfas-sobre-la-ambiciuxf3n-y-la-codicia-de-los-escribas}}

\bibleverse{45} A la vista de todo el pueblo, dijo a sus discípulos:
\bibleverse{46} ``Cuídense de esos escribas que gustan de andar con
ropas largas, y aman los saludos en las plazas, los mejores asientos en
las sinagogas, y los mejores lugares en las fiestas; \footnote{\textbf{20:46}
  Luc 11,34} \bibleverse{47} que devoran las casas de las viudas, y por
un pretexto hacen largas oraciones. Estos recibirán mayor condena''.

\hypertarget{jesuxfas-alaba-las-dos-blancas-de-la-viuda-pobre}{%
\subsection{Jesús alaba las dos blancas de la viuda
pobre}\label{jesuxfas-alaba-las-dos-blancas-de-la-viuda-pobre}}

\hypertarget{section-20}{%
\section{21}\label{section-20}}

\bibleverse{1} Levantó la vista y vio a los ricos que echaban sus
donativos en el tesoro. \bibleverse{2} Vio a una viuda pobre que echaba
dos moneditas de bronce. \footnote{\textbf{21:2} literalmente, ``dos
  lepta''. 2 lepta era aproximadamente el 1\% del salario diario de un
  trabajador agrícola.} \bibleverse{3} Y dijo: ``En verdad os digo que
esta viuda pobre ha echado más que todos ellos, \footnote{\textbf{21:3}
  2Cor 8,12} \bibleverse{4} porque todos estos echan dones para Dios de
su abundancia, pero ella, de su pobreza, echó todo lo que tenía para
vivir.''

\hypertarget{el-discurso-de-jesuxfas-en-el-monte-de-los-olivos-a-los-apuxf3stoles-sobre-la-destrucciuxf3n-del-templo-y-jerusaluxe9n-el-fin-de-este-mundo-y-su-apariciuxf3n-en-el-uxfaltimo-duxeda}{%
\subsection{El discurso de Jesús en el Monte de los Olivos a los
apóstoles sobre la destrucción del templo y Jerusalén, el fin de este
mundo y su aparición en el último
día}\label{el-discurso-de-jesuxfas-en-el-monte-de-los-olivos-a-los-apuxf3stoles-sobre-la-destrucciuxf3n-del-templo-y-jerusaluxe9n-el-fin-de-este-mundo-y-su-apariciuxf3n-en-el-uxfaltimo-duxeda}}

\bibleverse{5} Mientras algunos hablaban del templo y de cómo estaba
decorado con hermosas piedras y regalos, dijo: \bibleverse{6} ``En
cuanto a estas cosas que veis, vendrán días en que no quedará aquí una
piedra sobre otra que no sea derribada.'' \footnote{\textbf{21:6} Luc
  19,44}

\bibleverse{7} Le preguntaron: ``Maestro, ¿cuándo ocurrirán estas cosas?
¿Cuál es la señal de que estas cosas van a suceder?''

\hypertarget{los-primeros-signos-del-fin}{%
\subsection{Los primeros signos del
fin}\label{los-primeros-signos-del-fin}}

\bibleverse{8} Dijo: ``Tened cuidado de no dejaros llevar por el mal
camino, porque vendrán muchos en mi nombre, diciendo: ``Yo
soy'',\footnote{\textbf{21:8} o, YO SOY} y ``El tiempo está cerca''. Por
tanto, no los sigáis. \bibleverse{9} Cuando oigáis hablar de guerras y
disturbios, no os asustéis, porque es necesario que estas cosas sucedan
primero, pero el fin no llegará inmediatamente.''

\bibleverse{10} Entonces les dijo: ``Se levantará nación contra nación,
y reino contra reino. \bibleverse{11} Habrá grandes terremotos, hambres
y plagas en varios lugares. Habrá terrores y grandes señales del cielo.

\hypertarget{las-persecuciones-de-los-discuxedpulos}{%
\subsection{Las persecuciones de los
discípulos}\label{las-persecuciones-de-los-discuxedpulos}}

\bibleverse{12} Pero antes de todas estas cosas, os echarán mano y os
perseguirán, entregándoos a las sinagogas y a las cárceles, llevándoos
ante los reyes y los gobernadores por causa de mi nombre. \footnote{\textbf{21:12}
  Mat 10,18-22; Mat 10,30} \bibleverse{13} Esto se convertirá en un
testimonio para ustedes. \bibleverse{14} Por tanto, no meditéis de
antemano cómo responder, \footnote{\textbf{21:14} Luc 12,11}
\bibleverse{15} porque yo os daré una boca y una sabiduría que todos
vuestros adversarios no podrán resistir ni contradecir. \footnote{\textbf{21:15}
  Hech 6,10} \bibleverse{16} Seréis entregados incluso por padres,
hermanos, parientes y amigos. Harán que algunos de vosotros sean
condenados a muerte. \bibleverse{17} Seréis odiados por todos los
hombres por causa de mi nombre. \bibleverse{18} Y no perecerá ni un pelo
de vuestra cabeza. \footnote{\textbf{21:18} Luc 12,7}

\bibleverse{19} ``Con vuestra perseverancia ganaréis vuestras vidas.
\footnote{\textbf{21:19} Heb 10,36}

\hypertarget{la-destrucciuxf3n-de-jerusaluxe9n-y-la-difuxedcil-situaciuxf3n-del-pueblo-juduxedo}{%
\subsection{La destrucción de Jerusalén y la difícil situación del
pueblo
judío}\label{la-destrucciuxf3n-de-jerusaluxe9n-y-la-difuxedcil-situaciuxf3n-del-pueblo-juduxedo}}

\bibleverse{20} ``Pero cuando vean a Jerusalén rodeada de ejércitos,
sepan que su desolación está cerca. \bibleverse{21} Entonces que los que
estén en Judea huyan a las montañas. Que los que están en medio de ella
se vayan. Que no entren en ella los que están en el campo.
\bibleverse{22} Porque estos son días de venganza, para que se cumplan
todas las cosas que están escritas. \footnote{\textbf{21:22} Jer 5,29}
\bibleverse{23} ¡Ay de las embarazadas y de las que amamantan en esos
días! Porque habrá gran angustia en la tierra e ira para este pueblo.
\bibleverse{24} Caerán a filo de espada y serán llevados cautivos a
todas las naciones. Jerusalén será pisoteada por los gentiles hasta que
se cumplan los tiempos de los gentiles. \footnote{\textbf{21:24} Is
  63,18; Rom 11,25; Apoc 11,2}

\hypertarget{las-uxfaltimas-seuxf1ales-del-fin-y-la-apariciuxf3n-del-hijo-del-hombre}{%
\subsection{Las últimas señales del fin y la aparición del Hijo del
Hombre}\label{las-uxfaltimas-seuxf1ales-del-fin-y-la-apariciuxf3n-del-hijo-del-hombre}}

\bibleverse{25} ``Habrá señales en el sol, la luna y las estrellas; y en
la tierra ansiedad de las naciones, en la perplejidad por el rugido del
mar y de las olas; \footnote{\textbf{21:25} Apoc 6,12-13}
\bibleverse{26} los hombres desmayando por el temor y la expectación de
las cosas que vienen sobre el mundo, porque las potencias de los cielos
serán sacudidas. \bibleverse{27} Entonces verán al Hijo del Hombre venir
en una nube con poder y gran gloria. \footnote{\textbf{21:27} Dan 7,13}
\bibleverse{28} Pero cuando estas cosas comiencen a suceder, miren y
levanten la cabeza, porque su redención está cerca.'' \footnote{\textbf{21:28}
  Fil 4,4-5}

\bibleverse{29} Les contó una parábola. ``Mirad la higuera y todos los
árboles. \bibleverse{30} Cuando ya están brotando, lo veis y sabéis por
vosotros mismos que el verano ya está cerca. \bibleverse{31} Así también
vosotros, cuando veáis que suceden estas cosas, sabed que el Reino de
Dios está cerca. \bibleverse{32} De cierto os digo que esta generación
no pasará hasta que todo se haya cumplido. \bibleverse{33} El cielo y la
tierra pasarán, pero mis palabras no pasarán.

\hypertarget{una-advertencia-final-sobre-la-sobriedad-y-la-vigilancia}{%
\subsection{Una advertencia final sobre la sobriedad y la
vigilancia}\label{una-advertencia-final-sobre-la-sobriedad-y-la-vigilancia}}

\bibleverse{34} ``Así que tened cuidado, o vuestros corazones se
cargarán de juergas, borracheras y preocupaciones de esta vida, y ese
día os llegará de repente. \footnote{\textbf{21:34} Mar 4,19}
\bibleverse{35} Porque vendrá como un lazo sobre todos los que habitan
en la superficie de toda la tierra. \footnote{\textbf{21:35} 1Tes 5,3}
\bibleverse{36} Por tanto, velad en todo momento, orando para que seáis
tenidos por dignos de escapar de todas estas cosas que van a suceder, y
de estar en pie ante el Hijo del Hombre.'' \footnote{\textbf{21:36} Mar
  13,33}

\bibleverse{37} Todos los días, Jesús enseñaba en el templo, y todas las
noches salía a pasar la noche en el monte que se llama del Olivar.
\bibleverse{38} Todo el pueblo acudía de madrugada a escucharle en el
templo.

\hypertarget{intento-de-asesinato-por-parte-de-los-luxedderes-del-pueblo}{%
\subsection{Intento de asesinato por parte de los líderes del
pueblo}\label{intento-de-asesinato-por-parte-de-los-luxedderes-del-pueblo}}

\hypertarget{section-21}{%
\section{22}\label{section-21}}

\bibleverse{1} Se acercaba la fiesta de los panes sin levadura, que se
llama la Pascua. \bibleverse{2} Los jefes de los sacerdotes y los
escribas buscaban la manera de condenarlo a muerte, porque temían al
pueblo. \footnote{\textbf{22:2} Luc 20,19}

\hypertarget{traiciuxf3n-de-judas}{%
\subsection{Traición de Judas}\label{traiciuxf3n-de-judas}}

\bibleverse{3} Satanás entró en Judas, que también se llamaba Iscariote,
que era contado con los doce. \footnote{\textbf{22:3} Juan 13,2; Juan
  13,27} \bibleverse{4} Se fue y habló con los jefes de los sacerdotes y
con los capitanes sobre cómo podría entregarlo a ellos. \bibleverse{5}
Ellos se alegraron y aceptaron darle dinero. \bibleverse{6} Él consintió
y buscó una oportunidad para entregárselo en ausencia de la multitud.

\hypertarget{preparaciuxf3n-de-la-cena-de-pascua}{%
\subsection{Preparación de la cena de
Pascua}\label{preparaciuxf3n-de-la-cena-de-pascua}}

\bibleverse{7} Llegó el día de los panes sin levadura, en el que debía
sacrificarse la Pascua. \footnote{\textbf{22:7} Éxod 12,18-20}
\bibleverse{8} Jesús envió a Pedro y a Juan, diciendo: ``Id y
preparadnos la Pascua para que comamos.''

\bibleverse{9} Le dijeron: ``¿Dónde quieres que nos preparemos?''

\bibleverse{10} Les dijo: ``Mirad, cuando hayáis entrado en la ciudad,
os saldrá al encuentro un hombre que lleva un cántaro de agua. Seguidle
hasta la casa en la que entre. \bibleverse{11} Decid al dueño de la
casa: ``El Maestro os dice: ``¿Dónde está la habitación de los
invitados, donde pueda comer la Pascua con mis discípulos?''\,''.
\bibleverse{12} Él te mostrará una habitación superior grande y
amueblada. Haz los preparativos allí''.

\bibleverse{13} Fueron, encontraron las cosas como Jesús les había
dicho, y prepararon la Pascua. \footnote{\textbf{22:13} Luc 19,32}

\hypertarget{la-uxfaltima-cena-de-jesuxfas-en-el-cuxedrculo-de-los-discuxedpulos-instituciuxf3n-de-la-santa-comuniuxf3n}{%
\subsection{La última cena de Jesús en el círculo de los discípulos;
Institución de la santa
comunión}\label{la-uxfaltima-cena-de-jesuxfas-en-el-cuxedrculo-de-los-discuxedpulos-instituciuxf3n-de-la-santa-comuniuxf3n}}

\bibleverse{14} Cuando llegó la hora, se sentó con los doce apóstoles.
\bibleverse{15} Les dijo: ``Con cuánto anhelo he deseado comer esta
Pascua con vosotros antes de sufrir, \bibleverse{16} porque os digo que
ya no comeré de ella hasta que se cumpla en el Reino de Dios.''
\footnote{\textbf{22:16} Luc 13,29} \bibleverse{17} Recibió una copa y,
después de dar gracias, dijo: ``Tomad y compartidla entre vosotros,
\bibleverse{18} porque os digo que no volveré a beber del fruto de la
vid hasta que venga el Reino de Dios.''

\bibleverse{19} Tomó el pan y, después de dar gracias, lo partió y les
dio diciendo: ``Esto es mi cuerpo que se entrega por vosotros. Haced
esto en memoria mía''. \footnote{\textbf{22:19} 1Cor 11,23-25}
\bibleverse{20} Asimismo, tomó la copa después de la cena, diciendo:
``Esta copa es él nuevo pacto en mi sangre, que se derrama por vosotros.
\bibleverse{21} Pero he aquí que la mano del que me traiciona está
conmigo sobre la mesa. \footnote{\textbf{22:21} Juan 13,21-22}
\bibleverse{22} El Hijo del Hombre, en efecto, se va como ha sido
determinado, pero ¡ay de aquel hombre por quien es entregado!''

\bibleverse{23} Empezaron a preguntarse entre ellos quién era el que iba
a hacer esto.

\hypertarget{palabras-de-despedida-a-los-discuxedpulos}{%
\subsection{Palabras de despedida a los
discípulos}\label{palabras-de-despedida-a-los-discuxedpulos}}

\bibleverse{24} También surgió una disputa entre ellos, sobre cuál de
ellos se consideraba más grande. \footnote{\textbf{22:24} Luc 9,46; Mat
  20,25-28; Mar 10,42-45} \bibleverse{25} Él les dijo: ``Los reyes de
las naciones se enseñorean de ellas, y los que tienen autoridad sobre
ellas son llamados ``benefactores''. \bibleverse{26} Pero no es así con
ustedes. Más bien, el que es mayor entre vosotros, que se haga como el
más joven, y el que gobierna, como el que sirve. \bibleverse{27} Porque
¿quién es mayor, el que se sienta a la mesa o el que sirve? ¿No es el
que se sienta a la mesa? Pero yo estoy entre vosotros como uno que
sirve. \footnote{\textbf{22:27} Juan 13,4-14}

\bibleverse{28} ``Pero vosotros sois los que habéis continuado conmigo
en mis pruebas. \footnote{\textbf{22:28} Juan 6,67-68} \bibleverse{29}
Yo os confiero un reino, como me lo confirió mi Padre, \bibleverse{30}
para que comáis y bebáis en mi mesa en mi Reino. Os sentaréis en tronos
para juzgar a las doce tribus de Israel''. \footnote{\textbf{22:30} Mat
  19,28}

\hypertarget{advertencia-al-pedro-seguro-de-suxed-mismo-y-profecuxeda-de-su-negaciuxf3n}{%
\subsection{Advertencia al Pedro seguro de sí mismo y profecía de su
negación}\label{advertencia-al-pedro-seguro-de-suxed-mismo-y-profecuxeda-de-su-negaciuxf3n}}

\bibleverse{31} El Señor dijo: ``Simón, Simón, he aquí que Satanás pedía
disponer de todos vosotros para zarandearos como el trigo, \footnote{\textbf{22:31}
  2Cor 2,11} \bibleverse{32} pero yo he rogado por ti, para que tu fe no
desfallezca. Tú, cuando te hayas convertido de nuevo, confirma a tus
hermanos''. \footnote{\textbf{22:32} Juan 17,11; Juan 17,15}

\bibleverse{33} Le dijo: ``Señor, estoy dispuesto a ir contigo a la
cárcel y a la muerte''.

\bibleverse{34} Él dijo: ``Te digo, Pedro, que el gallo no cantará hoy
hasta que niegues que me conoces tres veces''.

\hypertarget{referencia-al-tiempo-que-los-discuxedpulos-vivieron-con-seguridad-y-al-futuro-serio-y-difuxedcil}{%
\subsection{Referencia al tiempo que los discípulos vivieron con
seguridad y al futuro serio y
difícil}\label{referencia-al-tiempo-que-los-discuxedpulos-vivieron-con-seguridad-y-al-futuro-serio-y-difuxedcil}}

\bibleverse{35} Les dijo: ``Cuando os envié sin bolsa, sin alforja y sin
sandalias, ¿os faltó algo?'' Dijeron: ``Nada''. \footnote{\textbf{22:35}
  Luc 9,3; Luc 10,4}

\bibleverse{36} Entonces les dijo: ``Pero ahora, quien tenga una bolsa,
que la tome, y también una alforja. El que no tenga, que venda su manto
y compre una espada. \bibleverse{37} Porque os digo que aún debe
cumplirse en mí lo que está escrito: `Fue contado con los
transgresores'.\footnote{\textbf{22:37} Isaías 53:12} Porque lo que me
concierne se está cumpliendo''.

\bibleverse{38} Dijeron: ``Señor, he aquí dos espadas''. Les dijo: ``Es
suficiente''.

\hypertarget{la-lucha-del-alma-de-jesuxfas-y-la-oraciuxf3n-en-el-monte-de-los-olivos}{%
\subsection{La lucha del alma de Jesús y la oración en el Monte de los
Olivos}\label{la-lucha-del-alma-de-jesuxfas-y-la-oraciuxf3n-en-el-monte-de-los-olivos}}

\bibleverse{39} Salió y se dirigió, como era su costumbre, al Monte de
los Olivos. Sus discípulos también le siguieron. \bibleverse{40} Cuando
llegó al lugar, les dijo: ``Orad para que no entréis en tentación''.

\bibleverse{41} Se apartó de ellos como a un tiro de piedra, y se
arrodilló y oró, \bibleverse{42} diciendo: ``Padre, si quieres, aparta
de mí esta copa. Sin embargo, no se haga mi voluntad, sino la tuya''.
\footnote{\textbf{22:42} Mat 6,10}

\bibleverse{43} Se le apareció un ángel del cielo que lo fortaleció.
\bibleverse{44} Estando en agonía, oró con más ahínco. Su sudor se
convirtió en grandes gotas de sangre que caían al suelo.

\bibleverse{45} Cuando se levantó de su oración, se acercó a los
discípulos y los encontró durmiendo a causa del dolor, \bibleverse{46} y
les dijo: ``¿Por qué dormís? Levantaos y orad para no entrar en la
tentación''.

\hypertarget{captura-de-jesuxfas}{%
\subsection{Captura de Jesús}\label{captura-de-jesuxfas}}

\bibleverse{47} Mientras aún hablaba, apareció una multitud. El que se
llamaba Judas, uno de los doce, los guiaba. Se acercó a Jesús para
besarlo. \bibleverse{48} Pero Jesús le dijo: ``Judas, ¿traicionas al
Hijo del Hombre con un beso?''

\bibleverse{49} Cuando los que estaban a su alrededor vieron lo que iba
a suceder, le dijeron: ``Señor, ¿herimos con la espada?''
\bibleverse{50} Uno de ellos hirió al siervo del sumo sacerdote y le
cortó la oreja derecha.

\bibleverse{51} Pero Jesús respondió: ``Déjame al menos hacer esto'', y
tocando su oreja lo sanó. \bibleverse{52} Jesús dijo a los jefes de los
sacerdotes, a los capitanes del templo y a los ancianos que habían
venido contra él: ``¿Habéis salido como contra un ladrón, con espadas y
palos? \bibleverse{53} Cuando estaba con ustedes en el templo cada día,
no extendían sus manos contra mí. Pero esta es vuestra hora, y el poder
de las tinieblas''. \footnote{\textbf{22:53} Juan 7,30; Juan 8,20}

\hypertarget{negaciuxf3n-y-arrepentimiento-de-pedro}{%
\subsection{Negación y arrepentimiento de
Pedro}\label{negaciuxf3n-y-arrepentimiento-de-pedro}}

\bibleverse{54} Lo agarraron, lo llevaron y lo metieron en la casa del
sumo sacerdote. Pero Pedro lo seguía de lejos. \bibleverse{55} Cuando
encendieron el fuego en medio del patio y se sentaron juntos, Pedro se
sentó entre ellos. \bibleverse{56} Una sirvienta le vio sentado a la
luz, y mirándole fijamente, dijo: ``También éste estaba con él.''

\bibleverse{57} Negó a Jesús, diciendo: ``Mujer, no lo conozco''.

\bibleverse{58} Al cabo de un rato, otro le vio y le dijo: ``¡También tú
eres uno de ellos!'' Pero Pedro respondió: ``¡Hombre, no lo soy!''.

\bibleverse{59} Al cabo de una hora aproximadamente, otro afirmó con
confianza, diciendo: ``¡Verdaderamente este hombre también estaba con
él, pues es galileo!''

\bibleverse{60} Pero Pedro dijo: ``¡Hombre, no sé de qué estás
hablando!''. Inmediatamente, mientras aún hablaba, cantó un gallo.
\bibleverse{61} El Señor se volvió y miró a Pedro. Entonces Pedro se
acordó de la palabra del Señor, de cómo le había dicho: ``Antes de que
cante el gallo me negarás tres veces''. \bibleverse{62} Salió y lloró
amargamente. \footnote{\textbf{22:62} Sal 51,17}

\hypertarget{burlarse-y-maltratar-a-jesuxfas-interrogatorio-ante-el-sumo-consejo}{%
\subsection{Burlarse y maltratar a Jesús; Interrogatorio ante el sumo
consejo}\label{burlarse-y-maltratar-a-jesuxfas-interrogatorio-ante-el-sumo-consejo}}

\bibleverse{63} Los hombres que retenían a Jesús se burlaban de él y lo
golpeaban. \bibleverse{64} Después de vendarle los ojos, le golpearon en
la cara y le preguntaron: ``¡Profetiza! ¿Quién es el que te ha
golpeado?'' \bibleverse{65} Dijeron muchas otras cosas contra él,
insultándolo.

\bibleverse{66} Cuando se hizo de día, se reunió la asamblea de los
ancianos del pueblo, tanto de los sumos sacerdotes como de los escribas,
y le llevaron a su consejo, diciendo: \bibleverse{67} ``Si eres el
Cristo, dínoslo''. Pero él les dijo: ``Si os lo digo, no creeréis,
\footnote{\textbf{22:67} Juan 3,12} \bibleverse{68} y si os lo pido, no
me responderéis ni me dejaréis ir. \bibleverse{69} Desde ahora, el Hijo
del Hombre estará sentado a la derecha del poder de Dios.'' \footnote{\textbf{22:69}
  Sal 110,1}

\bibleverse{70} Todos dijeron: ``¿Eres entonces el Hijo de Dios?'' Les
dijo: ``Lo decís vosotros, porque yo lo soy''.

\bibleverse{71} Dijeron: ``¿Por qué necesitamos más testigos? Porque
nosotros mismos hemos oído de su propia boca''.

\hypertarget{la-acusaciuxf3n-de-los-juduxedos-y-el-interrogatorio-de-jesuxfas-ante-pilato}{%
\subsection{La acusación de los judíos y el interrogatorio de Jesús ante
Pilato}\label{la-acusaciuxf3n-de-los-juduxedos-y-el-interrogatorio-de-jesuxfas-ante-pilato}}

\hypertarget{section-22}{%
\section{23}\label{section-22}}

\bibleverse{1} Toda la compañía se levantó y le llevó ante Pilato.
\bibleverse{2} Comenzaron a acusarle, diciendo: ``Hemos encontrado a
este hombre pervirtiendo a la nación, prohibiendo pagar los impuestos al
César y diciendo que él mismo es el Cristo, un rey.'' \footnote{\textbf{23:2}
  Luc 20,25; Hech 24,5}

\bibleverse{3} Pilato le preguntó: ``¿Eres tú el rey de los judíos?'' Le
respondió: ``Eso dices tú''.

\bibleverse{4} Pilato dijo a los jefes de los sacerdotes y a la
multitud: ``No encuentro fundamento para una acusación contra este
hombre''.

\bibleverse{5} Pero ellos insistieron, diciendo: ``Él agita al pueblo,
enseñando en toda Judea, comenzando desde Galilea hasta este lugar.''

\bibleverse{6} Pero cuando Pilato oyó mencionar a Galilea, preguntó si
el hombre era galileo. \bibleverse{7} Al enterarse de que estaba en la
jurisdicción de Herodes, lo envió a Herodes, que también estaba en
Jerusalén en esos días. \footnote{\textbf{23:7} Luc 3,1}

\hypertarget{jesus-antes-herodes}{%
\subsection{Jesus antes Herodes}\label{jesus-antes-herodes}}

\bibleverse{8} Cuando Herodes vio a Jesús, se alegró mucho, pues hacía
tiempo que quería verlo, porque había oído hablar mucho de él. Esperaba
ver algún milagro hecho por él. \footnote{\textbf{23:8} Luc 9,9}
\bibleverse{9} Lo interrogó con muchas palabras, pero no le respondió.
\bibleverse{10} Los jefes de los sacerdotes y los escribas estaban de
pie, acusándolo con vehemencia. \bibleverse{11} Herodes y sus soldados
lo humillaron y se burlaron de él. Vistiéndolo con ropas lujosas, lo
enviaron de vuelta a Pilato. \bibleverse{12} Ese mismo día Herodes y
Pilato se hicieron amigos entre sí, pues antes eran enemigos entre sí.

\hypertarget{jesuxfas-de-nuevo-ante-pilato}{%
\subsection{Jesús de nuevo ante
Pilato}\label{jesuxfas-de-nuevo-ante-pilato}}

\bibleverse{13} Pilato convocó a los jefes de los sacerdotes, a los
gobernantes y al pueblo, \bibleverse{14} y les dijo: ``Me habéis traído
a este hombre como a uno que pervierte al pueblo, y he aquí, habiéndolo
examinado delante de vosotros, no he encontrado fundamento para acusar a
este hombre de las cosas de que le acusáis. \bibleverse{15} Tampoco lo
ha hecho Herodes, pues os he enviado a él, y ved que no ha hecho nada
digno de muerte. \bibleverse{16} Por lo tanto, lo castigaré y lo
liberaré''.

\hypertarget{jesuxfas-y-barrabuxe1s-la-condenacion}{%
\subsection{Jesús y Barrabás; la
condenacion}\label{jesuxfas-y-barrabuxe1s-la-condenacion}}

\bibleverse{17} Ahora bien, tenía que soltarles un prisionero en la
fiesta. \footnote{\textbf{23:17} NU omite el versículo 17.} \footnote{\textbf{23:17}
  Mat 27,15} \bibleverse{18} Pero todos gritaron juntos, diciendo:
``¡Quita a este hombre! Y suéltanos a Barrabás! \bibleverse{19} que
había sido encarcelado por una revuelta en la ciudad y por asesinato.

\bibleverse{20} Entonces Pilato les habló de nuevo, queriendo liberar a
Jesús, \bibleverse{21} pero ellos gritaron diciendo: ``¡Crucifícalo!
Crucifícalo!''

\bibleverse{22} La tercera vez les dijo: ``¿Por qué? ¿Qué mal ha hecho
este hombre? No he encontrado en él ningún delito capital. Por tanto, lo
castigaré y lo soltaré''. \bibleverse{23} Pero ellos urgían a grandes
voces, pidiendo que fuera crucificado. Sus voces y las de los jefes de
los sacerdotes prevalecieron. \bibleverse{24} Pilato decretó que se
hiciera lo que ellos pedían. \bibleverse{25} Liberó al que habían metido
en la cárcel por insurrección y asesinato, por el que pedían, pero
entregó a Jesús a la voluntad de ellos.

\hypertarget{el-camino-de-la-muerte-de-jesuxfas-al-guxf3lgota-y-sus-palabras-a-las-mujeres-de-luto-de-jerusaluxe9n-su-crucifixiuxf3n-y-su-muerte}{%
\subsection{El camino de la muerte de Jesús al Gólgota y sus palabras a
las mujeres de luto de Jerusalén; su crucifixión y su
muerte}\label{el-camino-de-la-muerte-de-jesuxfas-al-guxf3lgota-y-sus-palabras-a-las-mujeres-de-luto-de-jerusaluxe9n-su-crucifixiuxf3n-y-su-muerte}}

\bibleverse{26} Cuando se lo llevaron, agarraron a un tal Simón de
Cirene, que venía del campo, y le pusieron la cruz para que la llevara
tras Jesús. \bibleverse{27} Le seguía una gran multitud del pueblo,
incluidas las mujeres, que también le lloraban y se lamentaban.
\bibleverse{28} Pero Jesús, dirigiéndose a ellas, les dijo: ``Hijas de
Jerusalén, no lloréis por mí, sino llorad por vosotras y por vuestros
hijos. \bibleverse{29} Porque he aquí que vienen días en que dirán:
`Benditas sean las estériles, los vientres que nunca dieron a luz y los
pechos que nunca amamantaron'. \footnote{\textbf{23:29} Luc 21,23}
\bibleverse{30} Entonces comenzarán a decir a los montes: ``¡Caigan
sobre nosotros!'' y a las colinas: ``Cúbrannos. \footnote{\textbf{23:30}
  Oseas 10:8} \footnote{\textbf{23:30} Os 10,8; Apoc 6,16; Apoc 9,6}
\bibleverse{31} Porque si hacen estas cosas en el árbol verde, ¿qué se
hará en el seco?'' \footnote{\textbf{23:31} 1Pe 4,17}

\bibleverse{32} Había también otros, dos delincuentes, conducidos con él
para ser ejecutados. \bibleverse{33} Cuando llegaron al lugar que se
llama ``La Calavera'', lo crucificaron allí con los criminales, uno a la
derecha y el otro a la izquierda.

\bibleverse{34} Jesús dijo: ``Padre, perdónalos, porque no saben lo que
hacen''. Repartiendo sus vestidos entre ellos, echaron suertes.
\footnote{\textbf{23:34} Is 53,12; Sal 22,18; Hech 3,17; Hech 7,59}
\bibleverse{35} El pueblo se quedó mirando. Los jefes que estaban con
ellos también se burlaban de él, diciendo: ``Ha salvado a otros. Que se
salve a sí mismo, si éste es el Cristo de Dios, su elegido''.

\bibleverse{36} Los soldados también se burlaron de él, acercándose y
ofreciéndole vinagre, \bibleverse{37} y diciendo: ``Si eres el Rey de
los Judíos, sálvate''.

\bibleverse{38} También se escribió sobre él una inscripción en letras
de griego, latín y hebreo: ``ESTE ES EL REY DE LOS JUDÍOS''.

\hypertarget{jesuxfas-y-los-dos-ladrones}{%
\subsection{Jesús y los dos
ladrones}\label{jesuxfas-y-los-dos-ladrones}}

\bibleverse{39} Uno de los delincuentes ahorcados le insultó diciendo:
``¡Si eres el Cristo, sálvate a ti mismo y a nosotros!''.

\bibleverse{40} Pero el otro contestó, y reprendiéndole le dijo: ``¿Ni
siquiera temes a Dios, viendo que estás bajo la misma condena?
\bibleverse{41} Y nosotros, ciertamente, con justicia, pues recibimos la
debida recompensa por nuestras obras, pero este hombre no ha hecho nada
malo.'' \bibleverse{42} Le dijo a Jesús: ``Señor, acuérdate de mí cuando
vengas a tu Reino''. \footnote{\textbf{23:42} Mat 16,28}

\bibleverse{43} Jesús le dijo: ``Te aseguro que hoy estarás conmigo en
el Paraíso''. \footnote{\textbf{23:43} 2Cor 12,4; Apoc 14,13}

\hypertarget{la-muerte-de-jesuxfas-las-seuxf1ales-milagrosas-de-su-muerte}{%
\subsection{La muerte de Jesús; las señales milagrosas de su
muerte}\label{la-muerte-de-jesuxfas-las-seuxf1ales-milagrosas-de-su-muerte}}

\bibleverse{44} Era ya como la hora sexta, \footnote{\textbf{23:44} La
  ``Fiesta de la Dedicación'' es el nombre griego de ``Hanukkah'', una
  celebración de la rededicación del Templo.} y las tinieblas llegaron a
toda la tierra hasta la hora novena. \bibleverse{45} El sol se oscureció
y el velo del templo se rasgó en dos. \footnote{\textbf{23:45} Éxod
  36,35} \bibleverse{46} Jesús, gritando a gran voz, dijo: ``Padre, en
tus manos encomiendo mi espíritu''. Dicho esto, expiró. \footnote{\textbf{23:46}
  Sal 31,5; Hech 7,58}

\bibleverse{47} Cuando el centurión vio lo que se había hecho, glorificó
a Dios, diciendo: ``Ciertamente éste era un hombre justo.''
\bibleverse{48} Toda la multitud que se había reunido para ver esto, al
ver lo que se había hecho, volvió a su casa golpeándose el pecho.
\bibleverse{49} Todos sus conocidos y las mujeres que le seguían desde
Galilea se quedaron a distancia, viendo estas cosas. \footnote{\textbf{23:49}
  Luc 8,2-3}

\hypertarget{el-entierro-de-jesuxfas}{%
\subsection{El entierro de Jesús}\label{el-entierro-de-jesuxfas}}

\bibleverse{50} He aquí que había un hombre llamado José, que era
miembro del consejo, hombre bueno y justo \bibleverse{51} (no había
consentido su consejo y su obra), de Arimatea, ciudad de los judíos, que
también esperaba el Reino de Dios. \footnote{\textbf{23:51} Luc 2,25;
  Luc 2,38} \bibleverse{52} Este hombre fue a Pilato y pidió el cuerpo
de Jesús. \bibleverse{53} Lo bajó, lo envolvió en una tela de lino y lo
puso en un sepulcro tallado en piedra, donde nunca se había puesto a
nadie. \bibleverse{54} Era el día de la Preparación, y se acercaba el
sábado. \bibleverse{55} Las mujeres que habían venido con él desde
Galilea le siguieron, y vieron el sepulcro y cómo estaba colocado su
cuerpo. \bibleverse{56} Volvieron y prepararon especias y ungüentos. El
sábado descansaron según el mandamiento. \footnote{\textbf{23:56} Éxod
  20,10}

\hypertarget{descubrimiento-de-la-tumba-vacuxeda-en-la-mauxf1ana-de-pascua-la-revelaciuxf3n-a-las-mujeres}{%
\subsection{Descubrimiento de la tumba vacía en la mañana de Pascua; la
revelación a las
mujeres}\label{descubrimiento-de-la-tumba-vacuxeda-en-la-mauxf1ana-de-pascua-la-revelaciuxf3n-a-las-mujeres}}

\hypertarget{section-23}{%
\section{24}\label{section-23}}

\bibleverse{1} Pero el primer día de la semana, al amanecer, llegaron al
sepulcro con otras personas, trayendo las especias que habían preparado.
\bibleverse{2} Encontraron la piedra removida del sepulcro.
\bibleverse{3} Entraron y no encontraron el cuerpo del Señor Jesús.
\bibleverse{4} Mientras estaban muy desconcertadas por esto, he aquí que
se les presentaron dos hombres con ropas deslumbrantes. \bibleverse{5}
Aterrados, bajaron el rostro a la tierra. Los hombres les dijeron:
``¿Por qué buscáis al vivo entre los muertos? \bibleverse{6} No está
aquí, sino que ha resucitado. ¿Recordáis lo que os dijo cuando aún
estaba en Galilea, \bibleverse{7} diciendo que el Hijo del Hombre debía
ser entregado en manos de hombres pecadores y ser crucificado, y al
tercer día resucitar?'' \footnote{\textbf{24:7} Luc 9,22}

\bibleverse{8} Se acordaron de sus palabras, \bibleverse{9} volvieron
del sepulcro y contaron todas estas cosas a los once y a todos los
demás. \bibleverse{10} Eran María Magdalena, Juana y María la madre de
Santiago. Las otras mujeres que estaban con ellas contaron estas cosas a
los apóstoles. \footnote{\textbf{24:10} Luc 8,2-3} \bibleverse{11} Estas
palabras les parecieron una tontería, y no las creyeron. \bibleverse{12}
Pero Pedro se levantó y corrió al sepulcro. Al agacharse y mirar dentro,
vio las tiras de lino tendidas por sí solas, y se marchó a su casa,
preguntándose qué había pasado. \footnote{\textbf{24:12} Juan 20,6-10}

\hypertarget{los-discuxedpulos-de-emauxfas}{%
\subsection{Los discípulos de
Emaús}\label{los-discuxedpulos-de-emauxfas}}

\bibleverse{13} He aquí que dos de ellos iban aquel mismo día a una
aldea llamada Emaús, que estaba a sesenta estadios de Jerusalén.
\bibleverse{14} Hablaban entre sí de todas estas cosas que habían
sucedido. \bibleverse{15} Mientras hablaban y preguntaban juntos, el
mismo Jesús se acercó y fue con ellos. \bibleverse{16} Pero los ojos de
ellos no le reconocían. \bibleverse{17} Él les dijo: ``¿De qué habláis
mientras camináis y estáis tristes?''

\bibleverse{18} Uno de ellos, llamado Cleofás, le respondió: ``¿Eres tú
el único forastero en Jerusalén que no sabe las cosas que han sucedido
allí en estos días?''

\bibleverse{19} Les dijo: ``¿Qué cosas?'' Le dijeron: ``Lo que se
refiere a Jesús el Nazareno, que fue un profeta poderoso en obra y en
palabra ante Dios y ante todo el pueblo; \footnote{\textbf{24:19} Mat
  21,11} \bibleverse{20} y cómo los jefes de los sacerdotes y nuestros
gobernantes lo entregaron para que fuera condenado a muerte, y lo
crucificaron. \bibleverse{21} Pero nosotros esperábamos que fuera él
quien redimiera a Israel. Sí, y además de todo esto, ya es el tercer día
desde que sucedieron estas cosas. \footnote{\textbf{24:21} Hech 1,6}
\bibleverse{22} También nos sorprendieron algunas mujeres de nuestra
compañía, que llegaron temprano al sepulcro; \bibleverse{23} y al no
encontrar su cuerpo, vinieron diciendo que también habían visto una
visión de ángeles, que decían que estaba vivo. \bibleverse{24} Algunos
de nosotros fueron al sepulcro y lo encontraron tal como habían dicho
las mujeres, pero no lo vieron.''

\bibleverse{25} Les dijo: ``¡Pueblo necio y lento de corazón para creer
en todo lo que han dicho los profetas! \bibleverse{26} ¿No tenía el
Cristo que sufrir estas cosas y entrar en su gloria?'' \bibleverse{27}
Empezando por Moisés y por todos los profetas, les explicó en todas las
Escrituras lo que se refería a él. \footnote{\textbf{24:27} Deut 18,15;
  Sal 22,1; Is 53,1}

\bibleverse{28} Se acercaron a la aldea a la que se dirigían, y él actuó
como si fuera a ir más lejos.

\bibleverse{29} Le instaron, diciendo: ``Quédate con nosotros, porque ya
está anocheciendo y el día está por terminar''. Entró para quedarse con
ellos. \bibleverse{30} Cuando se sentó a la mesa con ellos, tomó el pan
y dio gracias. Lo partió y se lo dio. \footnote{\textbf{24:30} Luc 9,16;
  Luc 22,19} \bibleverse{31} Se les abrieron los ojos y le reconocieron;
luego desapareció de su vista. \bibleverse{32} Se decían unos a otros:
``¿No ardía nuestro corazón mientras nos hablaba por el camino y nos
abría las Escrituras?'' \bibleverse{33} Se levantaron en aquella misma
hora, volvieron a Jerusalén y encontraron reunidos a los once y a los
que estaban con ellos, \bibleverse{34} diciendo: ``¡El Señor ha
resucitado realmente y se ha aparecido a Simón!'' \footnote{\textbf{24:34}
  1Cor 15,4-5} \bibleverse{35} Contaron las cosas que habían sucedido en
el camino, y cómo fue reconocido por ellos al partir el pan.

\hypertarget{jesuxfas-se-apareciuxf3-al-cuxedrculo-de-los-discuxedpulos-la-noche-del-domingo-de-pascua-su-mandato-misionero-y-despedida-de-los-discuxedpulos}{%
\subsection{Jesús se apareció al círculo de los discípulos la noche del
domingo de Pascua; su mandato misionero y despedida de los
discípulos}\label{jesuxfas-se-apareciuxf3-al-cuxedrculo-de-los-discuxedpulos-la-noche-del-domingo-de-pascua-su-mandato-misionero-y-despedida-de-los-discuxedpulos}}

\bibleverse{36} Mientras decían estas cosas, Jesús mismo se puso en
medio de ellos y les dijo: ``La paz sea con vosotros''.

\bibleverse{37} Pero ellos se aterraron y se llenaron de miedo, y
supusieron que habían visto un espíritu. \footnote{\textbf{24:37} Mat
  14,26}

\bibleverse{38} Les dijo: ``¿Por qué estáis turbados? ¿Por qué surgen
dudas en vuestros corazones? \bibleverse{39} Ved mis manos y mis pies,
que en verdad soy yo. Tóquenme y vean, porque un espíritu no tiene carne
ni huesos, como ven que yo tengo''. \bibleverse{40} Cuando hubo dicho
esto, les mostró sus manos y sus pies. \footnote{\textbf{24:40} Juan
  20,20} \bibleverse{41} Mientras ellos todavía no creían de alegría y
se preguntaban, les dijo: ``¿Tenéis aquí algo de comer?''

\bibleverse{42} Le dieron un trozo de pescado asado y un panal de miel.
\footnote{\textbf{24:42} Juan 21,5; Juan 21,10; Hech 10,41}
\bibleverse{43} Él los tomó y comió delante de ellos. \bibleverse{44}
Les dijo: ``Esto es lo que os dije mientras estaba con vosotros, que era
necesario que se cumpliera todo lo que está escrito en la ley de Moisés,
en los profetas y en los salmos acerca de mí.'' \footnote{\textbf{24:44}
  Luc 9,22; Luc 18,31-33}

\bibleverse{45} Entonces les abrió el entendimiento para que
comprendieran las Escrituras. \footnote{\textbf{24:45} Luc 9,45}
\bibleverse{46} Les dijo: ``Así está escrito, y así fue necesario que el
Cristo padeciera y resucitara de entre los muertos al tercer día,
\footnote{\textbf{24:46} Os 6,2; Juan 12,16} \bibleverse{47} y que se
predicara en su nombre el arrepentimiento y la remisión de los pecados a
todas las naciones, empezando por Jerusalén. \footnote{\textbf{24:47}
  Hech 2,38; Hech 17,30} \bibleverse{48} Vosotros sois testigos de estas
cosas. \bibleverse{49} He aquí que yo envío sobre vosotros la promesa de
mi Padre. Pero esperad en la ciudad de Jerusalén hasta que seáis
revestidos del poder de lo alto''. \footnote{\textbf{24:49} Juan 15,26;
  Juan 16,7; Hech 2,1-4}

\hypertarget{ascensiuxf3n-de-jesuxfas}{%
\subsection{Ascensión de jesús}\label{ascensiuxf3n-de-jesuxfas}}

\bibleverse{50} Los condujo hasta Betania, y alzando las manos los
bendijo. \bibleverse{51} Mientras los bendecía, se apartó de ellos y fue
llevado al cielo. \bibleverse{52} Ellos le adoraron y volvieron a
Jerusalén con gran alegría, \bibleverse{53} y estaban continuamente en
el templo, alabando y bendiciendo a Dios. Amén.
