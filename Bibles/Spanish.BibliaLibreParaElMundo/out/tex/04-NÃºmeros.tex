\hypertarget{primer-recuento-de-los-hombres-de-guerra}{%
\subsection{Primer recuento de los hombres de
guerra}\label{primer-recuento-de-los-hombres-de-guerra}}

\hypertarget{section}{%
\section{1}\label{section}}

\bibleverse{1} Yahvé habló a Moisés en el desierto del Sinaí, en la
Tienda del Encuentro, el primer día del segundo mes, en el segundo año
después de haber salido de la tierra de Egipto, diciendo: \bibleverse{2}
``Haz un censo de toda la congregación de los hijos de Israel, por sus
familias, por las casas de sus padres, según el número de los nombres,
cada varón, uno por uno, \footnote{\textbf{1:2} Núm 26,2-51; Éxod 30,12}
\bibleverse{3} de veinte años en adelante, todos los que puedan salir a
la guerra en Israel. Tú y Aarón los contarán por sus divisiones.
\bibleverse{4} Con vosotros habrá un hombre de cada tribu, cada uno jefe
de la casa de sus padres. \bibleverse{5} Estos son los nombres de los
hombres que estarán con vosotros: De Reuben: Elizur el hijo de Shedeur.

\bibleverse{6} De Simeón: Shelumiel, hijo de Zurishaddai.

\bibleverse{7} De Judá: Nahsón, hijo de Aminadab. \footnote{\textbf{1:7}
  Éxod 6,23}

\bibleverse{8} De Isacar: Netanel, hijo de Zuar.

\bibleverse{9} De Zebulón: Eliab, hijo de Helón.

\bibleverse{10} De los hijos de José: de Efraín: Elishama hijo de
Ammihud; de Manasés Gamaliel, hijo de Pedahzur. \footnote{\textbf{1:10}
  1Cró 7,26}

\bibleverse{11} De Benjamín: Abidán, hijo de Gideoni.

\bibleverse{12} De Dan: Ahiezer, hijo de Ammishaddai.

\bibleverse{13} De Asher: Pagiel, hijo de Ochran.

\bibleverse{14} De Gad: Eliasaf, hijo de Deuel.

\bibleverse{15} De Neftalí: Ahira, hijo de Enán''.

\bibleverse{16} Estos son los que fueron llamados de la congregación,
los príncipes de las tribus de sus padres; eran los jefes de los
millares de Israel. \bibleverse{17} Moisés y Aarón tomaron a estos
hombres mencionados por su nombre. \bibleverse{18} Reunieron a toda la
congregación el primer día del segundo mes, y declararon su ascendencia
por sus familias, por las casas de sus padres, según el número de los
nombres, de veinte años en adelante, uno por uno. \bibleverse{19} Como
Yahvé le ordenó a Moisés, así los contó en el desierto de Sinaí.

\hypertarget{los-resultados-del-censo}{%
\subsection{Los resultados del censo}\label{los-resultados-del-censo}}

\bibleverse{20} Los hijos de Rubén, primogénitos de Israel, sus
generaciones, por sus familias, por las casas de sus padres, según el
número de los nombres, uno por uno, todo varón de veinte años para
arriba, todos los que podían salir a la guerra: \bibleverse{21} Los
contados de ellos, de la tribu de Rubén, fueron cuarenta y seis mil
quinientos.

\bibleverse{22} De los hijos de Simeón, sus generaciones, por sus
familias, por las casas de sus padres, los que fueron contados de ella,
según el número de los nombres, uno por uno, todo varón de veinte años
para arriba, todos los que podían salir a la guerra: \bibleverse{23} los
que fueron contados de ellos, de la tribu de Simeón, fueron cincuenta y
nueve mil trescientos.

\bibleverse{24} De los hijos de Gad, sus generaciones, por sus familias,
por las casas de sus padres, según el número de los nombres, de veinte
años para arriba, todos los que podían salir a la guerra:
\bibleverse{25} Los contados de ellos, de la tribu de Gad, fueron
cuarenta y cinco mil seiscientos cincuenta.

\bibleverse{26} De los hijos de Judá, sus generaciones, por sus
familias, por las casas de sus padres, según el número de los nombres,
de veinte años para arriba, todos los que podían salir a la guerra:
\bibleverse{27} Los contados de ellos, de la tribu de Judá, fueron
setenta y cuatro mil seiscientos.

\bibleverse{28} De los hijos de Isacar, sus generaciones, por sus
familias, por las casas de sus padres, según el número de los nombres,
de veinte años para arriba, todos los que podían salir a la guerra:
\bibleverse{29} Los contados de ellos, de la tribu de Isacar, fueron
cincuenta y cuatro mil cuatrocientos.

\bibleverse{30} De los hijos de Zabulón, sus generaciones, por sus
familias, por las casas de sus padres, según el número de los nombres,
de veinte años para arriba, todos los que podían salir a la guerra:
\bibleverse{31} Los contados de ellos, de la tribu de Zabulón, fueron
cincuenta y siete mil cuatrocientos.

\bibleverse{32} De los hijos de José: de los hijos de Efraín, sus
generaciones, por sus familias, por las casas de sus padres, según el
número de los nombres, de veinte años para arriba, todos los que podían
salir a la guerra: \bibleverse{33} los contados de ellos, de la tribu de
Efraín, fueron cuarenta mil quinientos.

\bibleverse{34} De los hijos de Manasés, sus generaciones, por sus
familias, por las casas de sus padres, según el número de los nombres,
de veinte años para arriba, todos los que podían salir a la guerra:
\bibleverse{35} Los contados de ellos, de la tribu de Manasés, fueron
treinta y dos mil doscientos.

\bibleverse{36} De los hijos de Benjamín, sus generaciones, por sus
familias, por las casas de sus padres, según el número de los nombres,
de veinte años para arriba, todos los que podían salir a la guerra:
\bibleverse{37} Los contados de ellos, de la tribu de Benjamín, fueron
treinta y cinco mil cuatrocientos.

\bibleverse{38} De los hijos de Dan, sus generaciones, por sus familias,
por las casas de sus padres, según el número de los nombres, de veinte
años para arriba, todos los que podían salir a la guerra:
\bibleverse{39} Los contados de ellos, de la tribu de Dan, fueron
sesenta y dos mil setecientos.

\bibleverse{40} De los hijos de Aser, sus generaciones, por sus
familias, por las casas de sus padres, según el número de los nombres,
de veinte años para arriba, todos los que podían salir a la guerra:
\bibleverse{41} Los contados de ellos, de la tribu de Aser, fueron
cuarenta y un mil quinientos.

\bibleverse{42} De los hijos de Neftalí, sus generaciones, por sus
familias, por las casas de sus padres, según el número de los nombres,
de veinte años para arriba, todos los que podían salir a la guerra:
\bibleverse{43} Los contados de ellos, de la tribu de Neftalí, fueron
cincuenta y tres mil cuatrocientos.

\bibleverse{44} Estos son los que fueron contados, los cuales contaron
Moisés y Aarón, y los doce hombres que eran príncipes de Israel, cada
uno por la casa de su padre. \bibleverse{45} Así que todos los que
fueron contados de los hijos de Israel por las casas de sus padres, de
veinte años para arriba, todos los que podían salir a la guerra en
Israel --- \bibleverse{46} todos los que fueron contados fueron
seiscientos tres mil quinientos cincuenta. \footnote{\textbf{1:46} Núm
  2,32; Éxod 12,37}

\hypertarget{la-posiciuxf3n-excepcional-de-los-levitas}{%
\subsection{La posición excepcional de los
levitas}\label{la-posiciuxf3n-excepcional-de-los-levitas}}

\bibleverse{47} Pero los levitas según la tribu de sus padres no fueron
contados entre ellos. \bibleverse{48} Porque Yahvé habló a Moisés,
diciendo: \bibleverse{49} ``Sólo la tribu de Leví no contarás, ni harás
censo de ellos entre los hijos de Israel; \footnote{\textbf{1:49} Núm
  2,33; Núm 3,15} \bibleverse{50} sino que designarás a los levitas
sobre el Tabernáculo del Testimonio, y sobre todos sus enseres, y sobre
todo lo que le pertenece. Ellos llevarán el tabernáculo y todos sus
enseres; lo cuidarán y acamparán alrededor de él. \footnote{\textbf{1:50}
  Núm 4,1; Núm 3,23-38} \bibleverse{51} Cuando el tabernáculo deba
trasladarse, los levitas lo desmontarán; y cuando el tabernáculo deba
levantarse, los levitas lo armarán. El extranjero que se acerque morirá.
\footnote{\textbf{1:51} Núm 3,10; Núm 3,38} \bibleverse{52} Los hijos de
Israel acamparán, cada uno en su campamento, y cada uno en su
estandarte, según sus divisiones. \bibleverse{53} Pero los levitas
acamparán alrededor del Tabernáculo del Testimonio, para que no haya ira
en la congregación de los hijos de Israel. Los levitas serán
responsables del Tabernáculo del Testimonio''.

\bibleverse{54} Así hicieron los hijos de Israel. Según todo lo que
Yahvé ordenó a Moisés, así lo hicieron.

\hypertarget{el-orden-de-acampamiento-de-las-tribus}{%
\subsection{El orden de acampamiento de las
tribus}\label{el-orden-de-acampamiento-de-las-tribus}}

\hypertarget{section-1}{%
\section{2}\label{section-1}}

\bibleverse{1} Yahvé habló a Moisés y a Aarón, diciendo: \bibleverse{2}
``Los hijos de Israel acamparán cada uno con su propio estandarte, con
los estandartes de las casas de sus padres. Acamparán alrededor de la
Tienda del Encuentro, a distancia de ella. \footnote{\textbf{2:2} Núm
  1,1}

\bibleverse{3} ``Los que acampen en el lado oriental hacia la salida del
sol serán del estandarte del campamento de Judá, según sus divisiones.
El príncipe de los hijos de Judá será Naasón, hijo de Aminadab.
\bibleverse{4} Su división, y los que fueron contados de ellos, fueron
setenta y cuatro mil seiscientos.

\bibleverse{5} ``Los que acampen junto a él serán de la tribu de Isacar.
El príncipe de los hijos de Isacar será Netanel, hijo de Zuar.
\bibleverse{6} Su división, y los que fueron contados de ella, fueron
cincuenta y cuatro mil cuatrocientos.

\bibleverse{7} ``La tribu de Zabulón: el príncipe de los hijos de
Zabulón será Eliab hijo de Helón. \bibleverse{8} Su división, y los
contados de ella, fueron cincuenta y siete mil cuatrocientos.

\bibleverse{9} ``Todos los contados del campamento de Judá fueron ciento
ochenta y seis mil cuatrocientos, según sus divisiones. Ellos partirán
primero.

\bibleverse{10} ``En el lado sur estará el estandarte del campamento de
Rubén según sus divisiones. El príncipe de los hijos de Rubén será
Elisur, hijo de Sedeur. \bibleverse{11} Su división, y los que se
contaron de ella, fueron cuarenta y seis mil quinientos.

\bibleverse{12} ``Los que acampen junto a él serán la tribu de Simeón.
El príncipe de los hijos de Simeón será Selumiel, hijo de Zurishaddai.
\bibleverse{13} Su división, y los que fueron contados de ellos, fueron
cincuenta y nueve mil trescientos.

\bibleverse{14} ``La tribu de Gad: el príncipe de los hijos de Gad será
Eliasaf, hijo de Reuel. \bibleverse{15} Su división, y los contados de
ellos, fueron cuarenta y cinco mil seiscientos cincuenta.

\bibleverse{16} ``Todos los contados del campamento de Rubén fueron
ciento cincuenta y un mil cuatrocientos cincuenta, según sus ejércitos.
Ellos partirán en segundo lugar.

\bibleverse{17} ``Entonces saldrá la Tienda de la Reunión, con el
campamento de los levitas en medio de los campamentos. Así como acampan,
así saldrán, cada uno en su lugar, por sus estandartes.

\bibleverse{18} ``En el lado occidental estará el estandarte del
campamento de Efraín según sus divisiones. El príncipe de los hijos de
Efraín será Elisama, hijo de Ammihud. \bibleverse{19} Su división, y los
que fueron contados de ellos, fueron cuarenta mil quinientos.

\bibleverse{20} ``Junto a él estará la tribu de Manasés. El príncipe de
los hijos de Manasés será Gamaliel, hijo de Pedahzur. \bibleverse{21} Su
división, y los que fueron contados de ellos, fueron treinta y dos mil
doscientos.

\bibleverse{22} ``La tribu de Benjamín: el príncipe de los hijos de
Benjamín será Abidán, hijo de Gedeón. \bibleverse{23} Su ejército, y los
contados de ellos, fueron treinta y cinco mil cuatrocientos.

\bibleverse{24} ``Todos los contados del campamento de Efraín fueron
ciento ocho mil cien, según sus divisiones. Ellos partirán en tercer
lugar.

\bibleverse{25} ``En el lado norte estará el estandarte del campamento
de Dan según sus divisiones. El jefe de los hijos de Dan será Ahiezer
hijo de Amisadái. \bibleverse{26} Su división, y los que fueron contados
de ellos, fueron sesenta y dos mil setecientos.

\bibleverse{27} ``Los que acampen junto a él serán la tribu de Aser. El
príncipe de los hijos de Aser será Pagiel, hijo de Ocrán.
\bibleverse{28} Su división, y los que fueron contados de ellos, fueron
cuarenta y un mil quinientos.

\bibleverse{29} ``La tribu de Neftalí: el príncipe de los hijos de
Neftalí será Ahira, hijo de Enán. \bibleverse{30} Su división, y los que
fueron contados de ellos, fueron cincuenta y tres mil cuatrocientos.

\bibleverse{31} ``Todos los contados del campamento de Dan fueron ciento
cincuenta y siete mil seiscientos. Saldrán los últimos por sus
estandartes''.

\bibleverse{32} Estos son los que fueron contados de los hijos de Israel
por sus casas paternas. Todos los que fueron contados de los campamentos
según sus ejércitos fueron seiscientos tres mil quinientos cincuenta.
\footnote{\textbf{2:32} Núm 1,46} \bibleverse{33} Pero los levitas no
fueron contados entre los hijos de Israel, tal como Yahvé ordenó a
Moisés. \footnote{\textbf{2:33} Núm 1,48-49}

\bibleverse{34} Así hicieron los hijos de Israel. Conforme a todo lo que
el Señor ordenó a Moisés, acamparon por sus banderas, y así se pusieron
en marcha, cada uno por su familia, según las casas de sus padres.
\footnote{\textbf{2:34} Núm 2,2}

\hypertarget{los-hijos-de-aaron}{%
\subsection{Los hijos de Aaron}\label{los-hijos-de-aaron}}

\hypertarget{section-2}{%
\section{3}\label{section-2}}

\bibleverse{1} Esta es la historia de las generaciones de Aarón y Moisés
en el día en que Yahvé habló con Moisés en el monte Sinaí. \footnote{\textbf{3:1}
  Éxod 6,23} \bibleverse{2} Estos son los nombres de los hijos de Aarón:
Nadab, el primogénito, y Abiú, Eleazar e Itamar.

\bibleverse{3} Estos son los nombres de los hijos de Aarón, los
sacerdotes que fueron ungidos, a quienes consagró para que ejercieran el
ministerio sacerdotal. \bibleverse{4} Nadab y Abiú murieron ante Yavé
cuando ofrecieron fuego extraño ante Yavé en el desierto de Sinaí, y no
tuvieron hijos. Eleazar e Itamar ejercieron el ministerio sacerdotal en
presencia de Aarón, su padre. \footnote{\textbf{3:4} Lev 10,1-2}

\hypertarget{los-levitas-fueron-designados-para-ayudar-a-los-sacerdotes-y-servir-en-el-santuario}{%
\subsection{Los levitas fueron designados para ayudar a los sacerdotes y
servir en el
santuario}\label{los-levitas-fueron-designados-para-ayudar-a-los-sacerdotes-y-servir-en-el-santuario}}

\bibleverse{5} Yahvé habló a Moisés, diciendo: \bibleverse{6} ``Haz que
se acerque la tribu de Leví, y ponlos delante del sacerdote Aarón, para
que le sirvan. \footnote{\textbf{3:6} Éxod 32,29} \bibleverse{7} Ellos
guardarán sus requerimientos, y los requerimientos de toda la
congregación ante la Tienda de Reunión, para hacer el servicio del
tabernáculo. \bibleverse{8} Guardarán todo el mobiliario de la Tienda de
reunión y las obligaciones de los hijos de Israel, para hacer el
servicio del tabernáculo. \footnote{\textbf{3:8} Núm 4,1} \bibleverse{9}
Darás los levitas a Aarón y a sus hijos. Le serán entregados
íntegramente en nombre de los hijos de Israel. \bibleverse{10}
Designarás a Aarón y a sus hijos, y ellos conservarán su sacerdocio,
pero el extranjero que se acerque será condenado a muerte.'' \footnote{\textbf{3:10}
  Núm 1,51}

\hypertarget{los-levitas-fueran-designados-para-redimir-al-primoguxe9nito-israelita}{%
\subsection{Los levitas fueran designados para redimir al primogénito
israelita}\label{los-levitas-fueran-designados-para-redimir-al-primoguxe9nito-israelita}}

\bibleverse{11} Yahvé habló a Moisés, diciendo: \bibleverse{12} ``He
aquí que he tomado a los levitas de entre los hijos de Israel en lugar
de todos los primogénitos que abren el vientre entre los hijos de
Israel; y los levitas serán míos, \footnote{\textbf{3:12} Núm 8,16; Éxod
  13,2} \bibleverse{13} porque todos los primogénitos son míos. El día
en que derribé a todos los primogénitos en la tierra de Egipto,
santifiqué para mí a todos los primogénitos de Israel, tanto hombres
como animales. Serán míos. Yo soy Yahvé''.

\hypertarget{conteo-lugar-de-almacenamiento-luxedder-y-reglamentos-de-los-levitas-masculinos}{%
\subsection{Conteo, lugar de almacenamiento, líder y reglamentos de los
levitas
masculinos}\label{conteo-lugar-de-almacenamiento-luxedder-y-reglamentos-de-los-levitas-masculinos}}

\bibleverse{14} Yahvé habló a Moisés en el desierto del Sinaí, diciendo:
\bibleverse{15} ``Cuenta a los hijos de Leví por las casas de sus
padres, por sus familias. Contarás a todos los varones de un mes en
adelante''.

\bibleverse{16} Moisés los contó según la palabra de Yahvé, como se le
había ordenado.

\bibleverse{17} Estos fueron los hijos de Leví por sus nombres: Gersón,
Coat y Merari. \footnote{\textbf{3:17} Éxod 6,16-19; Núm 26,57-64}

\bibleverse{18} Estos son los nombres de los hijos de Gersón por sus
familias Libni y Simei.

\bibleverse{19} Los hijos de Coat por sus familias: Amram, Izhar, Hebrón
y Uziel.

\bibleverse{20} Los hijos de Merari por sus familias: Mahli y Mushi.
Estas son las familias de los levitas según las casas de sus padres.

\bibleverse{21} De Gersón era la familia de los libnitas, y la familia
de los simeítas. Estas son las familias de los gersonitas.

\bibleverse{22} Los que fueron contados de ellos, según el número de
todos los varones de un mes para arriba, fueron siete mil quinientos.

\bibleverse{23} Las familias de los gersonitas acamparán detrás del
tabernáculo hacia el oeste.

\bibleverse{24} Eliasaf, hijo de Lael, será el príncipe de la casa
paterna de los gersonitas. \bibleverse{25} La tarea de los hijos de
Gersón en la Tienda de reunión será el tabernáculo, la tienda, su
cubierta, la cortina de la puerta de la Tienda de reunión,
\bibleverse{26} las cortinas del atrio, la cortina de la puerta del
atrio que está junto al tabernáculo y alrededor del altar, y sus cuerdas
para todo su servicio.

\bibleverse{27} De Coat era la familia de los amramitas, la familia de
los izharitas, la familia de los hebronitas y la familia de los
uzielitas. Estas son las familias de los coatitas. \bibleverse{28} Según
el número de todos los varones de un mes en adelante, había ocho mil
seiscientos que cumplían con los requisitos del santuario.

\bibleverse{29} Las familias de los hijos de Coat acamparán al sur de la
tienda. \bibleverse{30} El príncipe de la casa paterna de las familias
de Coat será Elizafán hijo de Uziel. \footnote{\textbf{3:30} Lev 10,4}
\bibleverse{31} Su tarea será el arca, la mesa, el candelabro, los
altares, los utensilios del santuario con los que se ministra, la
cortina y todo su servicio. \footnote{\textbf{3:31} Núm 7,9}
\bibleverse{32} Eleazar, hijo del sacerdote Aarón, será el príncipe de
los príncipes de los levitas, con la supervisión de los que cumplen los
requisitos del santuario.

\bibleverse{33} De Merari era la familia de los Mahlitas y la familia de
los Mushitas. Estas son las familias de Merari. \bibleverse{34} Los
contados de ellos, según el número de todos los varones de un mes para
arriba, fueron seis mil doscientos.

\bibleverse{35} El príncipe de la casa paterna de las familias de Merari
era Zuriel hijo de Abihail. Acamparán en el lado norte del tabernáculo.
\bibleverse{36} La tarea asignada a los hijos de Merari será las tablas
del tabernáculo, sus barras, sus pilares, sus bases, todos sus
instrumentos, todo su servicio, \bibleverse{37} los pilares del atrio
que lo rodea, sus bases, sus clavijas y sus cuerdas.

\bibleverse{38} Los que acampen delante del tabernáculo hacia el
oriente, frente a la Tienda de Reunión hacia la salida del sol, serán
Moisés, con Aarón y sus hijos, guardando los requisitos del santuario
para el deber de los hijos de Israel. El forastero que se acerque será
condenado a muerte. \footnote{\textbf{3:38} Núm 3,10} \bibleverse{39}
Todos los contados de los levitas, que Moisés y Aarón contaron por orden
de Yahvé, por sus familias, todos los varones de un mes en adelante,
fueron veintidós mil.

\hypertarget{examen-y-resoluciuxf3n-del-primoguxe9nito-masculino-en-israel}{%
\subsection{Examen y resolución del primogénito masculino en
Israel}\label{examen-y-resoluciuxf3n-del-primoguxe9nito-masculino-en-israel}}

\bibleverse{40} Yahvé dijo a Moisés: ``Cuenta todos los primogénitos
varones de los hijos de Israel de un mes en adelante, y toma la cuenta
de sus nombres. \bibleverse{41} Tomarás a los levitas para mí --- yo soy
Yahvé --- en lugar de todos los primogénitos entre los hijos de Israel;
y el ganado de los levitas en lugar de todos los primogénitos entre el
ganado de los hijos de Israel.''

\bibleverse{42} Moisés contó, como le había ordenado Yahvé, todos los
primogénitos de los hijos de Israel. \bibleverse{43} Todos los
primogénitos varones, según el número de nombres, de un mes para arriba,
de los que fueron contados, fueron veintidós mil doscientos setenta y
tres.

\bibleverse{44} Yahvé habló a Moisés diciendo: \bibleverse{45} ``Toma a
los levitas en lugar de todos los primogénitos de los hijos de Israel, y
el ganado de los levitas en lugar de su ganado; y los levitas serán
míos. Yo soy Yahvé. \footnote{\textbf{3:45} Núm 3,12} \bibleverse{46}
Para la redención de los doscientos setenta y tres primogénitos de los
hijos de Israel que excedan el número de los levitas, \footnote{\textbf{3:46}
  Núm 3,39; Núm 3,43} \bibleverse{47} tomarás cinco siclos por cada uno;
según el siclo del santuario los tomarás (el siclo es de veinte gerahs);
\bibleverse{48} y darás el dinero, con el que se redime su resto, a
Aarón y a sus hijos.''

\bibleverse{49} Moisés tomó el dinero de la redención de los que
excedían el número de los redimidos por los levitas; \bibleverse{50} de
los primogénitos de los hijos de Israel tomó el dinero, mil trescientos
sesenta y cinco siclos, según el siclo del santuario; \bibleverse{51} y
Moisés dio el dinero de la redención a Aarón y a sus hijos, según la
palabra de Yahvé, como Yahvé le ordenó a Moisés.

\hypertarget{examen-de-los-levitas-aptos-para-el-servicio-incluidas-las-normas-de-servicio}{%
\subsection{Examen de los levitas aptos para el servicio, incluidas las
normas de
servicio}\label{examen-de-los-levitas-aptos-para-el-servicio-incluidas-las-normas-de-servicio}}

\hypertarget{section-3}{%
\section{4}\label{section-3}}

\bibleverse{1} Yahvé habló a Moisés y a Aarón, diciendo: \bibleverse{2}
``Haz un censo de los hijos de Coat de entre los hijos de Leví, por sus
familias, por las casas de sus padres, \bibleverse{3} desde los treinta
años en adelante hasta los cincuenta años, todos los que entren en el
servicio para hacer el trabajo en la Tienda del Encuentro. \footnote{\textbf{4:3}
  Núm 8,24}

\bibleverse{4} ``Este es el servicio de los hijos de Coat en la Tienda
del Encuentro, en lo que respecta a las cosas más sagradas.
\bibleverse{5} Cuando el campamento avance, Aarón entrará con sus hijos;
y ellos quitarán el velo de la cortina, cubrirán con él el Arca del
Testimonio, \bibleverse{6} le pondrán una cubierta de piel de foca,
extenderán sobre ella un paño azul y colocarán sus varas.

\bibleverse{7} ``Sobre la mesa del pan de la función extenderán un paño
azul, y pondrán sobre él los platos, las cucharas, los tazones y las
copas con las que se sirve; y sobre él estará el pan continuo.
\bibleverse{8} Extenderán sobre ella un paño de color escarlata y la
cubrirán con un revestimiento de piel de foca, y pondrán sus varas.

\bibleverse{9} ``Tomarán un paño azul y cubrirán el candelabro de la
lámpara, sus lámparas, sus apagadores, sus tabaqueras y todos sus
recipientes de aceite, con los que la atienden. \footnote{\textbf{4:9}
  Éxod 25,31} \bibleverse{10} La pondrán, junto con todos sus
recipientes, dentro de una cubierta de piel de foca, y la pondrán sobre
el armazón.

\bibleverse{11} ``Sobre el altar de oro extenderán un paño azul y lo
cubrirán con una cubierta de piel de foca, y pondrán sus varas.

\bibleverse{12} ``Tomarán todos los utensilios del ministerio con los
que ministran en el santuario, los pondrán en un paño azul, los cubrirán
con una funda de piel de foca y los pondrán sobre el bastidor.

\bibleverse{13} ``Quitarán la ceniza del altar y extenderán sobre él un
paño de color púrpura. \bibleverse{14} Pondrán sobre él todos los
utensilios con los que ministran a su alrededor, las sartenes para el
fuego, los ganchos para la carne, las palas y las palanganas, todos los
utensilios del altar; y extenderán sobre él una cubierta de piel de
foca, y pondrán sus varas.

\bibleverse{15} ``Cuando Aarón y sus hijos hayan terminado de cubrir el
santuario y todos los muebles del santuario, mientras el campamento
avanza; después de eso, los hijos de Coat vendrán a llevarlo; pero no
tocarán el santuario, para no morir. Los hijos de Coat llevarán estas
cosas que pertenecen a la Tienda de reunión. \footnote{\textbf{4:15} Núm
  7,9; 2Sam 6,6-7}

\bibleverse{16} ``La tarea del sacerdote Eleazar, hijo de Aarón, será el
aceite para la luz, el incienso aromático, la ofrenda continua y el
aceite para la unción, los requisitos de todo el tabernáculo y de todo
lo que hay en él, el santuario y su mobiliario.''

\bibleverse{17} Yahvé habló a Moisés y a Aarón, diciendo:
\bibleverse{18} ``No eliminen a la tribu de las familias de los coatitas
de entre los levitas; \bibleverse{19} sino que hagan esto con ellos,
para que vivan y no mueran cuando se acerquen a las cosas más santas:
Aarón y sus hijos entrarán y asignarán a cada uno su servicio y su
carga; \bibleverse{20} pero no entrarán a ver el santuario ni siquiera
por un momento, para que no mueran.'' \footnote{\textbf{4:20} 1Sam 6,19}

\bibleverse{21} Yahvé habló a Moisés diciendo: \bibleverse{22} ``Haz un
censo también de los hijos de Gersón, por las casas de sus padres, por
sus familias; \bibleverse{23} los contarás desde los treinta años en
adelante hasta los cincuenta años: todos los que entran a servir, a
hacer el trabajo en la Tienda del Encuentro.

\bibleverse{24} ``Este es el servicio de las familias de los gersonitas,
para servir y llevar cargas: \bibleverse{25} llevarán las cortinas del
tabernáculo y de la Tienda de reunión, su cubierta, la cubierta de piel
de sello que está sobre ella, la cortina de la puerta de la Tienda de
reunión, \bibleverse{26} las cortinas del atrio, la cortina de la puerta
del atrio que está junto al tabernáculo y alrededor del altar, sus
cuerdas y todos los instrumentos de su servicio, y todo lo que se haga
con ellos. Ellos servirán allí. \bibleverse{27} A las órdenes de Aarón y
de sus hijos estará todo el servicio de los hijos de los gersonitas, en
toda su carga y en todo su servicio; y les asignarás su deber en todas
sus responsabilidades. \bibleverse{28} Este es el servicio de las
familias de los hijos de los gersonitas en la Tienda de Reunión. Su
deber estará bajo la mano de Itamar, hijo del sacerdote Aarón.

\bibleverse{29} ``En cuanto a los hijos de Merari, los contarás por sus
familias, por las casas de sus padres; \bibleverse{30} los contarás
desde los treinta años y hasta los cincuenta, todos los que entren en el
servicio, para hacer la obra de la Tienda de reunión. \bibleverse{31}
Esta es la tarea de su carga, según todo su servicio en la Tienda de
reunión: las tablas del tabernáculo, sus barras, sus pilares, sus bases,
\bibleverse{32} los pilares del atrio que lo rodea, sus bases, sus
clavijas, sus cuerdas, con todos sus instrumentos y con todo su
servicio. Les asignarás los instrumentos del servicio de su carga por su
nombre. \bibleverse{33} Este es el servicio de las familias de los hijos
de Merari, según todo su servicio en la Tienda de reunión, bajo la mano
de Itamar hijo del sacerdote Aarón.''

\hypertarget{resultados-de-la-inspecciuxf3n}{%
\subsection{Resultados de la
inspección}\label{resultados-de-la-inspecciuxf3n}}

\bibleverse{34} Moisés y Aarón y los príncipes de la congregación
contaron a los hijos de los coatitas por sus familias y por las casas de
sus padres, \bibleverse{35} desde los treinta años y hasta los
cincuenta, a todos los que entraban en el servicio para trabajar en la
Tienda del Encuentro. \bibleverse{36} Los contados de ellos por sus
familias fueron dos mil setecientos cincuenta. \bibleverse{37} Estos son
los que fueron contados de las familias de los coatitas, todos los que
servían en la Tienda del Encuentro, los cuales fueron contados por
Moisés y Aarón según el mandato de Yahvé por medio de Moisés.

\bibleverse{38} Los que fueron contados de los hijos de Gersón, por sus
familias y por las casas de sus padres, \bibleverse{39} desde los
treinta años de edad hasta los cincuenta, todos los que entraron en el
servicio para trabajar en la Tienda del Encuentro, \bibleverse{40} los
que fueron contados de ellos, por sus familias y por las casas de sus
padres, fueron dos mil seiscientos treinta. \bibleverse{41} Estos son
los que fueron contados de las familias de los hijos de Gersón, todos
los que servían en la Tienda del Encuentro, los cuales fueron contados
por Moisés y Aarón según el mandamiento de Yahvé.

\bibleverse{42} Los que fueron contados de las familias de los hijos de
Merari, por sus familias, por las casas de sus padres, \bibleverse{43}
de treinta años en adelante hasta los cincuenta años, todos los que
entraron en el servicio para trabajar en la Tienda de Reunión,
\bibleverse{44} hasta los que fueron contados de ellos por sus familias,
fueron tres mil doscientos. \bibleverse{45} Estos son los que fueron
contados de las familias de los hijos de Merari, que Moisés y Aarón
contaron según el mandato de Yahvé por medio de Moisés.

\bibleverse{46} Todos los contados de los levitas que Moisés y Aarón y
los príncipes de Israel contaron, por sus familias y por las casas de
sus padres, \bibleverse{47} de treinta años en adelante hasta los
cincuenta, todos los que entraron a hacer el trabajo de servicio y el
trabajo de llevar cargas en la Tienda del Encuentro, \bibleverse{48} los
contados de ellos, fueron ocho mil quinientos ochenta. \bibleverse{49}
Según el mandato de Yahvé fueron contados por Moisés, cada uno según su
servicio y según su carga. Así fueron contados por él, como Yahvé le
ordenó a Moisés.

\hypertarget{extracciuxf3n-de-los-inmundos-del-campamento}{%
\subsection{Extracción de los inmundos del
campamento}\label{extracciuxf3n-de-los-inmundos-del-campamento}}

\hypertarget{section-4}{%
\section{5}\label{section-4}}

\bibleverse{1} Yahvé habló a Moisés, diciendo: \bibleverse{2} ``Ordena a
los hijos de Israel que saquen del campamento a todo leproso, a todo el
que tenga flujo y a todo el que esté impuro por un cadáver. \footnote{\textbf{5:2}
  Lev 13,46; Lev 15,2} \bibleverse{3} Pondrán fuera del campamento tanto
a los hombres como a las mujeres, para que no contaminen su campamento,
en medio del cual yo habito''. \footnote{\textbf{5:3} Núm 12,14; Núm
  35,34}

\bibleverse{4} Los hijos de Israel lo hicieron, y los pusieron fuera del
campamento; como Yahvé habló a Moisés, así lo hicieron los hijos de
Israel.

\hypertarget{malversaciuxf3n-y-su-expiaciuxf3n}{%
\subsection{Malversación y su
expiación}\label{malversaciuxf3n-y-su-expiaciuxf3n}}

\bibleverse{5} Yahvé habló a Moisés, diciendo: \bibleverse{6} ``Habla a
los hijos de Israel: `Cuando un hombre o una mujer cometa cualquier
pecado de los que cometen los hombres, de modo que transgreda a Yahvé, y
esa alma sea culpable, \footnote{\textbf{5:6} Lev 6,2-7} \bibleverse{7}
entonces confesará su pecado que ha cometido; y hará la restitución de
su culpa en su totalidad, añadiendo a ella la quinta parte de la misma,
y se la dará a aquel respecto del cual ha sido culpable. \bibleverse{8}
Pero si el hombre no tiene pariente a quien hacer la restitución por la
culpa, la restitución por la culpa que se haga a Yahvé será del
sacerdote, además del carnero de la expiación, con el cual se hará la
expiación por él. \bibleverse{9} Toda ofrenda de todas las cosas santas
de los hijos de Israel, que presenten al sacerdote, será suya.
\footnote{\textbf{5:9} Núm 18,8} \bibleverse{10} Las cosas santas de
cada uno serán suyas; todo lo que cualquiera dé al sacerdote, será
suyo''.

\hypertarget{sacrificio-de-celo-y-agua-de-maldiciuxf3n-de-una-mujer-sospechosa-de-adulterio}{%
\subsection{Sacrificio de celo y agua de maldición de una mujer
sospechosa de
adulterio}\label{sacrificio-de-celo-y-agua-de-maldiciuxf3n-de-una-mujer-sospechosa-de-adulterio}}

\bibleverse{11} Yahvé habló a Moisés diciendo: \bibleverse{12} ``Habla a
los hijos de Israel y diles: `Si la mujer de un hombre se extravía y le
es infiel, \bibleverse{13} y un hombre se acuesta con ella carnalmente,
y se oculta a los ojos de su marido y esto se mantiene oculto, y ella se
mancha, no hay testigo contra ella, y no es tomada en el acto;
\bibleverse{14} y el espíritu de celos se apodera de él, y tiene celos
de su mujer y ella está contaminada; o si el espíritu de celos se
apodera de él, y tiene celos de su mujer y ella no está contaminada;
\bibleverse{15} entonces el hombre traerá a su mujer al sacerdote, y
traerá su ofrenda por ella: una décima parte de un efa de harina de
cebada. No derramará aceite ni pondrá incienso sobre ella, porque es una
ofrenda de celos, una ofrenda de memoria, que trae la iniquidad a la
memoria. \bibleverse{16} El sacerdote la acercará y la pondrá delante de
Yahvé. \bibleverse{17} El sacerdote tomará agua bendita en una vasija de
barro; y el sacerdote tomará un poco del polvo que está en el piso del
tabernáculo y lo pondrá en el agua. \footnote{\textbf{5:17} Éxod 30,18}
\bibleverse{18} El sacerdote pondrá a la mujer delante de Yavé, y dejará
suelto el cabello de la cabeza de la mujer, y pondrá en sus manos la
ofrenda de comida conmemorativa, que es la ofrenda de celos. El
sacerdote tendrá en su mano el agua de la amargura que trae la
maldición. \bibleverse{19} El sacerdote le hará prestar juramento y le
dirá a la mujer: ``Si ningún hombre se ha acostado contigo y si no te
has desviado a la impureza, estando bajo la autoridad de tu marido,
libérate de esta agua de amargura que trae maldición. \bibleverse{20}
Pero si te has desviado, estando bajo la autoridad de tu marido, y si
estás impura, y algún hombre se ha acostado contigo además de tu marido
---'' \bibleverse{21} entonces el sacerdote hará que la mujer jure con
el juramento de maldición, y el sacerdote le dirá a la mujer: ``Que
Yahvé te haga una maldición y un juramento entre tu pueblo, cuando Yahvé
permita que tu muslo se desprenda y tu cuerpo se hinche; \bibleverse{22}
y esta agua que trae una maldición entrará en tus entrañas y hará que tu
cuerpo se hinche y tu muslo se desprenda.'' La mujer dirá: ``Amén,
Amén''.

\bibleverse{23} ``\,`El sacerdote escribirá estas maldiciones en un
libro, y las limpiará en el agua de la amargura. \bibleverse{24} Hará
que la mujer beba el agua de la amargura que causa la maldición; y el
agua que causa la maldición entrará en ella y se volverá amarga.
\bibleverse{25} El sacerdote tomará la ofrenda de celos de la mano de la
mujer, agitará la ofrenda delante de Yavé y la llevará al altar.
\bibleverse{26} El sacerdote tomará un puñado de la ofrenda de comida,
como su porción conmemorativa, y la quemará en el altar, y después hará
que la mujer beba el agua. \bibleverse{27} Cuando le haya hecho beber el
agua, sucederá que si ella está contaminada y ha cometido una
transgresión contra su marido, el agua que causa la maldición entrará en
ella y se volverá amarga, y su cuerpo se hinchará, y su muslo se caerá;
y la mujer será una maldición entre su pueblo. \bibleverse{28} Si la
mujer no está contaminada, sino que está limpia, entonces será libre y
concebirá descendencia.

\bibleverse{29} ``\,`Esta es la ley de los celos, cuando la mujer,
estando bajo su marido, se extravía y se contamina, \bibleverse{30} o
cuando el espíritu de los celos se apodera de un hombre, y éste tiene
celos de su mujer; entonces pondrá a la mujer delante de Yahvé, y el
sacerdote ejecutará sobre ella toda esta ley. \bibleverse{31} El hombre
quedará libre de iniquidad, y la mujer cargará con su iniquidad.'\,''

\hypertarget{normas-relativas-a-los-nazareos}{%
\subsection{Normas relativas a los
nazareos}\label{normas-relativas-a-los-nazareos}}

\hypertarget{section-5}{%
\section{6}\label{section-5}}

\bibleverse{1} Yahvé habló a Moisés, diciendo: \bibleverse{2} ``Habla a
los hijos de Israel y diles: `Cuando un hombre o una mujer haga un voto
especial, el voto de nazareo, para separarse de Yahvé, \footnote{\textbf{6:2}
  1Sam 1,11} \bibleverse{3} se separará del vino y de la bebida fuerte.
No beberá vinagre de vino, ni vinagre de bebida fermentada, ni beberá
jugo de uva, ni comerá uvas frescas o secas. \footnote{\textbf{6:3} Luc
  1,15} \bibleverse{4} Durante todos los días de su separación, no
comerá nada que esté hecho de la vid, desde las semillas hasta los
hollejos.

\bibleverse{5} ``\,`Durante todos los días de su voto de separación,
ninguna navaja se acercará a su cabeza, hasta que se cumplan los días en
que se separa de Yahvé. Será santo. Dejará crecer los mechones de su
cabeza. \footnote{\textbf{6:5} Jue 13,5}

\bibleverse{6} ``\,`Todos los días que se separe de Yahvé no se acercará
a un cadáver. \bibleverse{7} No se ensuciará por su padre, ni por su
madre, ni por su hermano, ni por su hermana, cuando mueran, porque su
separación a Dios está sobre su cabeza. \footnote{\textbf{6:7} Lev 21,11}
\bibleverse{8} Todos los días de su separación es santo para Yahvé.

\hypertarget{regulaciones-relativas-a-la-contaminaciuxf3n-del-nazareo}{%
\subsection{Regulaciones relativas a la contaminación del
nazareo}\label{regulaciones-relativas-a-la-contaminaciuxf3n-del-nazareo}}

\bibleverse{9} ``\,`Si alguno muere muy repentinamente junto a él, y
contamina la cabeza de su separación, entonces se afeitará la cabeza en
el día de su purificación. Al séptimo día se la afeitará. \footnote{\textbf{6:9}
  Núm 19,11} \bibleverse{10} Al octavo día traerá dos tórtolas o dos
pichones al sacerdote, a la puerta de la Tienda de reunión. \footnote{\textbf{6:10}
  Lev 5,7} \bibleverse{11} El sacerdote ofrecerá uno como ofrenda por el
pecado y el otro como holocausto, y hará expiación por él, porque pecó
por causa de la muerte, y santificará su cabeza ese mismo día.
\bibleverse{12} Separará para Yahvé los días de su separación, y traerá
un cordero macho de un año como ofrenda por la culpa; pero los días
anteriores serán anulados, porque su separación fue contaminada.

\hypertarget{ordenanzas-sobre-la-ceremonia-del-sacrificio-al-final-del-nazareo}{%
\subsection{Ordenanzas sobre la ceremonia del sacrificio al final del
nazareo}\label{ordenanzas-sobre-la-ceremonia-del-sacrificio-al-final-del-nazareo}}

\bibleverse{13} ``\,`Esta es la ley del nazireo: cuando se cumplan los
días de su separación, será llevado a la puerta de la Tienda del
Encuentro, \bibleverse{14} y ofrecerá su ofrenda a Yahvé: un cordero
macho de un año sin defecto para el holocausto, una oveja de un año sin
defecto para la ofrenda por el pecado, un carnero sin defecto para las
ofrendas de paz, \bibleverse{15} una cesta de panes sin levadura, tortas
de harina fina mezcladas con aceite, y obleas sin levadura untadas con
aceite con su ofrenda y sus libaciones. \bibleverse{16} El sacerdote los
presentará ante Yahvé, y ofrecerá su ofrenda por el pecado y su
holocausto. \bibleverse{17} Ofrecerá el carnero como sacrificio de paz a
Yahvé, con el cesto de los panes sin levadura. El sacerdote ofrecerá
también su ofrenda y su libación. \bibleverse{18} El nazireo se afeitará
la cabeza de su separación a la puerta de la Tienda del Encuentro,
tomará el pelo de la cabeza de su separación y lo pondrá sobre el fuego
que está debajo del sacrificio de paz. \footnote{\textbf{6:18} Hech
  18,18} \bibleverse{19} El sacerdote tomará la espaldilla cocida del
carnero, una torta sin levadura del canasto y una oblea sin levadura, y
las pondrá sobre las manos del nazareo después de que haya afeitado la
cabeza de su separación; \bibleverse{20} y el sacerdote las agitará como
ofrenda mecida ante Yahvé. Son sagrados para el sacerdote, junto con el
pecho que se agita y el muslo que se ofrece. Después, el nazireo podrá
beber vino. \footnote{\textbf{6:20} Lev 7,29-34}

\bibleverse{21} ``\,`Esta es la ley del nazireo que hace voto y de su
ofrenda a Yahvé por su separación, además de lo que pueda pagar. Según
su voto que hace, así debe hacer según la ley de su separación'\,''.

\hypertarget{orden-de-la-bendiciuxf3n-sacerdotal}{%
\subsection{Orden de la bendición
sacerdotal}\label{orden-de-la-bendiciuxf3n-sacerdotal}}

\bibleverse{22} Yahvé habló a Moisés diciendo: \bibleverse{23} ``Habla a
Aarón y a sus hijos, diciendo: `Así bendecirás a los hijos de Israel'.
Les dirás, \footnote{\textbf{6:23} Lev 9,22-23} \bibleverse{24} `Que el
Señor te bendiga y te guarde. \footnote{\textbf{6:24} Sal 121,1}
\bibleverse{25} Yahvé hace brillar su rostro sobre ti, y ser amable
contigo. \footnote{\textbf{6:25} Sal 80,3} \bibleverse{26} Yahvé levanta
su rostro hacia ti, y te dará la paz''. \footnote{\textbf{6:26} Sal
  69,16-17}

\bibleverse{27} ``Así pondrán mi nombre sobre los hijos de Israel, y los
bendeciré''.

\hypertarget{los-dones-de-consagraciuxf3n-de-los-jefes-tribales-para-el-santuario}{%
\subsection{Los dones de consagración de los jefes tribales para el
santuario}\label{los-dones-de-consagraciuxf3n-de-los-jefes-tribales-para-el-santuario}}

\hypertarget{section-6}{%
\section{7}\label{section-6}}

\bibleverse{1} El día en que Moisés terminó de levantar el tabernáculo,
lo ungió y lo santificó con todo su mobiliario, y el altar con todos sus
utensilios, y los ungió y santificó; \footnote{\textbf{7:1} Éxod 40,9-10}
\bibleverse{2} los príncipes de Israel, los jefes de las casas de sus
padres, dieron ofrendas. Estos eran los príncipes de las tribus. Estos
son los que estaban sobre los contados; \bibleverse{3} y trajeron su
ofrenda ante Yahvé, seis carros cubiertos y doce bueyes; un carro por
cada dos de los príncipes, y por cada uno un buey. Los presentaron ante
el tabernáculo. \bibleverse{4} Yahvé habló a Moisés, diciendo:
\bibleverse{5} ``Acéptalos de ellos, para que sean utilizados en el
servicio de la Tienda de Reunión; y los darás a los levitas, a cada uno
según su servicio.''

\bibleverse{6} Moisés tomó los carros y los bueyes y los entregó a los
levitas. \bibleverse{7} Dio dos carros y cuatro bueyes a los hijos de
Gersón, según su servicio. \bibleverse{8} A los hijos de Merari les dio
cuatro carros y ocho bueyes, según su servicio, bajo la dirección de
Itamar, hijo del sacerdote Aarón. \footnote{\textbf{7:8} Éxod 38,21; Núm
  4,28; Núm 4,33} \bibleverse{9} Pero a los hijos de Coat no les dio
ninguno, porque el servicio del santuario les correspondía a ellos; lo
llevaban sobre sus hombros.

\bibleverse{10} Los príncipes dieron ofrendas para la dedicación del
altar el día en que fue ungido. Los príncipes dieron sus ofrendas ante
el altar. \footnote{\textbf{7:10} 2Cró 7,9}

\bibleverse{11} Yahvé dijo a Moisés: ``Ofrecerán su ofrenda, cada
príncipe en su día, para la dedicación del altar''. \footnote{\textbf{7:11}
  Núm 1,4-16; Núm 2,3-29}

\bibleverse{12} El que ofreció su ofrenda el primer día fue Naasón hijo
de Aminadab, de la tribu de Judá, \bibleverse{13} y su ofrenda fue: una
bandeja de plata, cuyo peso era de ciento treinta siclos, un cuenco de
plata de setenta siclos, según el siclo del santuario, ambos llenos de
harina fina mezclada con aceite para una ofrenda de harina;

\bibleverse{14} un cazo de oro de diez siclos, lleno de incienso;

\bibleverse{15} un toro joven, un carnero, un cordero macho de un año,
para el holocausto;

\bibleverse{16} un macho cabrío como ofrenda por el pecado;

\bibleverse{17} y para el sacrificio de las ofrendas de paz, dos cabezas
de ganado, cinco carneros, cinco machos cabríos y cinco corderos de un
año. Esta fue la ofrenda de Naasón, hijo de Aminadab.

\bibleverse{18} El segundo día, Netanel hijo de Zuar, príncipe de
Isacar, presentó su ofrenda. \bibleverse{19} Ofreció por su ofrenda: una
bandeja de plata, cuyo peso era de ciento treinta siclos, un cuenco de
plata de setenta siclos, según el siclo del santuario, ambos llenos de
harina fina mezclada con aceite para una ofrenda de harina;

\bibleverse{20} un cazo de oro de diez siclos, lleno de incienso;

\bibleverse{21} un toro joven, un carnero, un cordero macho de un año,
para el holocausto;

\bibleverse{22} un macho cabrío como ofrenda por el pecado;

\bibleverse{23} y para el sacrificio de las ofrendas de paz, dos cabezas
de ganado, cinco carneros, cinco machos cabríos, cinco corderos de un
año. Esta fue la ofrenda de Natanel, hijo de Zuar.

\bibleverse{24} Al tercer día Eliab hijo de Helón, príncipe de los hijos
de Zabulón, \bibleverse{25} dio su ofrenda: una bandeja de plata, cuyo
peso era de ciento treinta siclos, un cuenco de plata de setenta siclos,
según el siclo del santuario, ambos llenos de harina fina mezclada con
aceite para una ofrenda de harina;

\bibleverse{26} un cazo de oro de diez siclos, lleno de incienso;

\bibleverse{27} un toro joven, un carnero, un cordero macho de un año,
para el holocausto;

\bibleverse{28} un macho cabrío como ofrenda por el pecado;

\bibleverse{29} y para el sacrificio de las ofrendas de paz, dos cabezas
de ganado, cinco carneros, cinco machos cabríos y cinco corderos de un
año. Esta fue la ofrenda de Eliab, hijo de Helón.

\bibleverse{30} El cuarto día Elizur, hijo de Sedeur, príncipe de los
hijos de Rubén, \bibleverse{31} dio su ofrenda: una bandeja de plata,
cuyo peso era de ciento treinta siclos, un cuenco de plata de setenta
siclos, según el siclo del santuario, ambos llenos de harina fina
mezclada con aceite para una ofrenda de harina;

\bibleverse{32} un cazo de oro de diez siclos, lleno de incienso;

\bibleverse{33} un toro joven, un carnero, un cordero macho de un año,
para el holocausto;

\bibleverse{34} un macho cabrío como ofrenda por el pecado;

\bibleverse{35} y para el sacrificio de las ofrendas de paz, dos cabezas
de ganado, cinco carneros, cinco machos cabríos y cinco corderos de un
año. Esta fue la ofrenda de Elizur, hijo de Sedeur.

\bibleverse{36} El quinto día, Selumiel, hijo de Zurishaddai, príncipe
de los hijos de Simeón, \bibleverse{37} dio su ofrenda: una bandeja de
plata, cuyo peso era de ciento treinta siclos, un cuenco de plata de
setenta siclos, según el siclo del santuario, ambos llenos de harina
fina mezclada con aceite para una ofrenda de harina;

\bibleverse{38} un cazo de oro de diez siclos, lleno de incienso;

\bibleverse{39} un toro joven, un carnero, un cordero macho de un año,
para el holocausto;

\bibleverse{40} un macho cabrío como ofrenda por el pecado;

\bibleverse{41} y para el sacrificio de las ofrendas de paz, dos cabezas
de ganado, cinco carneros, cinco machos cabríos y cinco corderos de un
año: esta fue la ofrenda de Selumiel, hijo de Zurishaddai.

\bibleverse{42} Al sexto día, Eliasaf hijo de Deuel, príncipe de los
hijos de Gad, \bibleverse{43} dio su ofrenda: una bandeja de plata, cuyo
peso era de ciento treinta siclos, un cuenco de plata de setenta siclos,
según el siclo del santuario, ambos llenos de harina fina mezclada con
aceite para una ofrenda de harina;

\bibleverse{44} un cazo de oro de diez siclos, lleno de incienso;

\bibleverse{45} un toro joven, un carnero, un cordero macho de un año,
para el holocausto;

\bibleverse{46} un macho cabrío como ofrenda por el pecado;

\bibleverse{47} y para el sacrificio de las ofrendas de paz, dos cabezas
de ganado, cinco carneros, cinco machos cabríos y cinco corderos de un
año. Esta fue la ofrenda de Eliasaf, hijo de Deuel.

\bibleverse{48} El séptimo día Elishama, hijo de Ammihud, príncipe de
los hijos de Efraín, \bibleverse{49} dio su ofrenda: una bandeja de
plata, cuyo peso era de ciento treinta siclos, un cuenco de plata de
setenta siclos, según el siclo del santuario, ambos llenos de harina
fina mezclada con aceite para una ofrenda de harina;

\bibleverse{50} un cazo de oro de diez siclos, lleno de incienso;

\bibleverse{51} un toro joven, un carnero, un cordero macho de un año,
para el holocausto;

\bibleverse{52} un macho cabrío como ofrenda por el pecado;

\bibleverse{53} y para el sacrificio de las ofrendas de paz, dos cabezas
de ganado, cinco carneros, cinco machos cabríos y cinco corderos de un
año. Esta fue la ofrenda de Elishama, hijo de Ammihud.

\bibleverse{54} El octavo día Gamaliel, hijo de Pedahzur, príncipe de
los hijos de Manasés, \bibleverse{55} dio su ofrenda: una bandeja de
plata, cuyo peso era de ciento treinta siclos, un cuenco de plata de
setenta siclos, según el siclo del santuario, ambos llenos de harina
fina mezclada con aceite para una ofrenda de harina;

\bibleverse{56} un cazo de oro de diez siclos, lleno de incienso;

\bibleverse{57} un toro joven, un carnero, un cordero macho de un año,
para el holocausto;

\bibleverse{58} un macho cabrío como ofrenda por el pecado;

\bibleverse{59} y para el sacrificio de las ofrendas de paz, dos cabezas
de ganado, cinco carneros, cinco machos cabríos y cinco corderos de un
año. Esta fue la ofrenda de Gamaliel, hijo de Pedahzur.

\bibleverse{60} El noveno día Abidán hijo de Gedeón, príncipe de los
hijos de Benjamín, \bibleverse{61} dio su ofrenda: una bandeja de plata,
cuyo peso era de ciento treinta siclos, un cuenco de plata de setenta
siclos, según el siclo del santuario, ambos llenos de harina fina
mezclada con aceite para una ofrenda de harina;

\bibleverse{62} un cazo de oro de diez siclos, lleno de incienso;

\bibleverse{63} un toro joven, un carnero, un cordero macho de un año,
para el holocausto;

\bibleverse{64} un macho cabrío como ofrenda por el pecado;

\bibleverse{65} y para el sacrificio de las ofrendas de paz, dos cabezas
de ganado, cinco carneros, cinco machos cabríos y cinco corderos de un
año. Esta fue la ofrenda de Abidán, hijo de Gedeón.

\bibleverse{66} El décimo día Ahiezer hijo de Ammishaddai, príncipe de
los hijos de Dan, \bibleverse{67} dio su ofrenda: una bandeja de plata,
cuyo peso era de ciento treinta siclos, un cuenco de plata de setenta
siclos, según el siclo del santuario, ambos llenos de harina fina
mezclada con aceite para una ofrenda de harina;

\bibleverse{68} un cazo de oro de diez siclos, lleno de incienso;

\bibleverse{69} un toro joven, un carnero, un cordero macho de un año,
para el holocausto;

\bibleverse{70} un macho cabrío como ofrenda por el pecado;

\bibleverse{71} y para el sacrificio de las ofrendas de paz, dos cabezas
de ganado, cinco carneros, cinco machos cabríos y cinco corderos de un
año. Esta fue la ofrenda de Ahiezer hijo de Ammishaddai.

\bibleverse{72} El undécimo día Pagiel, hijo de Ocrán, príncipe de los
hijos de Aser, \bibleverse{73} dio su ofrenda: una bandeja de plata,
cuyo peso era de ciento treinta siclos, un cuenco de plata de setenta
siclos, según el siclo del santuario, ambos llenos de harina fina
mezclada con aceite para una ofrenda de harina;

\bibleverse{74} un cazo de oro de diez siclos, lleno de incienso;

\bibleverse{75} un toro joven, un carnero, un cordero macho de un año,
para el holocausto;

\bibleverse{76} un macho cabrío como ofrenda por el pecado;

\bibleverse{77} y para el sacrificio de las ofrendas de paz, dos cabezas
de ganado, cinco carneros, cinco machos cabríos y cinco corderos de un
año. Esta fue la ofrenda de Pagiel, hijo de Ocrán.

\bibleverse{78} El duodécimo día Ahira, hijo de Enán, príncipe de los
hijos de Neftalí, \bibleverse{79} dio su ofrenda: una bandeja de plata,
cuyo peso era de ciento treinta siclos, un cuenco de plata de setenta
siclos, según el siclo del santuario, ambos llenos de harina fina
mezclada con aceite para una ofrenda de harina;

\bibleverse{80} un cazo de oro de diez siclos, lleno de incienso;

\bibleverse{81} un toro joven, un carnero, un cordero macho de un año,
para el holocausto;

\bibleverse{82} un macho cabrío como ofrenda por el pecado;

\bibleverse{83} y para el sacrificio de las ofrendas de paz, dos cabezas
de ganado, cinco carneros, cinco machos cabríos y cinco corderos de un
año. Esta fue la ofrenda de Ahira, hijo de Enán.

\bibleverse{84} Esta fue la ofrenda de dedicación del altar, el día en
que fue ungido, por los príncipes de Israel: doce fuentes de plata, doce
tazones de plata, doce cucharones de oro; \bibleverse{85} cada fuente de
plata pesaba ciento treinta siclos, y cada tazón setenta; toda la plata
de los utensilios dos mil cuatrocientos siclos, según el siclo del
santuario; \bibleverse{86} los doce cucharones de oro, llenos de
incienso, pesaban diez siclos cada uno, según el siclo del santuario;
todo el oro de los cucharones pesaba ciento veinte siclos;
\bibleverse{87} todo el ganado para el holocausto, doce toros, los
carneros doce, los corderos machos de un año doce, y su ofrenda de
comida; y doce machos cabríos para la ofrenda por el pecado;
\bibleverse{88} y todo el ganado para el sacrificio de las ofrendas de
paz: veinticuatro toros, sesenta carneros, sesenta machos cabríos y
sesenta corderos de un año. Esta fue la ofrenda de dedicación del altar,
después de ser ungido.

\bibleverse{89} Cuando Moisés entró en la Tienda del Encuentro para
hablar con Yahvé, oyó su voz que le hablaba desde lo alto del
propiciatorio que estaba sobre el arca del Testimonio, desde entre los
dos querubines; y le habló. \footnote{\textbf{7:89} Éxod 25,21-22; 1Sam
  3,3-14}

\hypertarget{las-siete-luxe1mparas-del-candelero}{%
\subsection{Las siete lámparas del
candelero}\label{las-siete-luxe1mparas-del-candelero}}

\hypertarget{section-7}{%
\section{8}\label{section-7}}

\bibleverse{1} Yahvé habló a Moisés diciendo: \bibleverse{2} ``Habla a
Aarón y dile: ``Cuando enciendas las lámparas, las siete lámparas
alumbrarán delante del candelabro''\,''. \footnote{\textbf{8:2} Éxod
  25,31-40}

\bibleverse{3} Aarón lo hizo. Encendió sus lámparas para iluminar el
área frente al candelabro, como Yahvé le ordenó a Moisés. \bibleverse{4}
Esta era la hechura del candelabro, obra de oro batido. Desde su base
hasta sus flores, era de oro batido. Hizo el candelabro según el modelo
que el Señor le había mostrado a Moisés.

\hypertarget{la-consagraciuxf3n-de-los-levitas-como-un-regalo-santo-a-dios}{%
\subsection{La consagración de los levitas como un regalo santo a
Dios}\label{la-consagraciuxf3n-de-los-levitas-como-un-regalo-santo-a-dios}}

\bibleverse{5} Yahvé habló a Moisés diciendo: \bibleverse{6} ``Toma a
los levitas de entre los hijos de Israel y purifícalos. \footnote{\textbf{8:6}
  Mal 3,3} \bibleverse{7} Harás lo siguiente para purificarlos: rocía
sobre ellos el agua de la purificación, deja que se afeiten todo el
cuerpo con una navaja de afeitar, que laven sus ropas y se purifiquen.
\footnote{\textbf{8:7} Núm 5,17; Núm 19,9; Núm 19,17; Lev 14,8}
\bibleverse{8} Luego tomarán un novillo y su ofrenda de harina fina
mezclada con aceite; y otro novillo lo tomarás como ofrenda por el
pecado. \bibleverse{9} Presentarás a los levitas ante la Tienda de
reunión. Reunirás a toda la congregación de los hijos de Israel.
\bibleverse{10} Presentarás a los levitas ante el Señor. Los hijos de
Israel pondrán sus manos sobre los levitas, \bibleverse{11} y Aarón
ofrecerá a los levitas ante Yavé como ofrenda mecida en nombre de los
hijos de Israel, para que sea de ellos el servicio de Yavé. \footnote{\textbf{8:11}
  Núm 8,21}

\bibleverse{12} ``Los levitas pondrán sus manos sobre las cabezas de los
toros, y ofreceréis uno como ofrenda por el pecado y el otro como
holocausto a Yahvé, para hacer expiación por los levitas.
\bibleverse{13} Pondrás a los levitas delante de Aarón y de sus hijos, y
los ofrecerás como ofrenda mecida a Yavé. \bibleverse{14} Así separarás
a los levitas de entre los hijos de Israel, y los levitas serán míos.
\footnote{\textbf{8:14} Núm 3,45}

\bibleverse{15} ``Después, los levitas entrarán a hacer el servicio de
la Tienda de reunión. Los purificarás y los ofrecerás como ofrenda
mecida. \bibleverse{16} Porque me son enteramente dados de entre los
hijos de Israel; en lugar de todos los que abren el vientre, los
primogénitos de todos los hijos de Israel, los he tomado para mí.
\footnote{\textbf{8:16} Núm 3,12} \bibleverse{17} Porque todos los
primogénitos de los hijos de Israel son míos, tanto los hombres como los
animales. El día en que herí a todos los primogénitos en la tierra de
Egipto, los sancioné para mí. \footnote{\textbf{8:17} Éxod 13,2}
\bibleverse{18} He tomado a los levitas en lugar de todos los
primogénitos de los hijos de Israel. \bibleverse{19} He dado a los
levitas como regalo a Aarón y a sus hijos de entre los hijos de Israel,
para que hagan el servicio de los hijos de Israel en la Tienda del
Encuentro, y para que hagan la expiación por los hijos de Israel, a fin
de que no haya plaga entre los hijos de Israel cuando éstos se acerquen
al santuario.'' \footnote{\textbf{8:19} Núm 3,9}

\bibleverse{20} Moisés, y Aarón, y toda la congregación de los hijos de
Israel hicieron así con los levitas. Conforme a todo lo que Yahvé mandó
a Moisés acerca de los levitas, así lo hicieron los hijos de Israel con
ellos. \bibleverse{21} Los levitas se purificaron del pecado y lavaron
sus ropas; y Aarón los ofreció como ofrenda mecida ante Yavé, y Aarón
hizo expiación por ellos para purificarlos. \footnote{\textbf{8:21} Núm
  8,11} \bibleverse{22} Después de eso, los levitas entraron a hacer su
servicio en la Tienda del Encuentro, delante de Aarón y de sus hijos;
como Yahvé había ordenado a Moisés acerca de los levitas, así hicieron
con ellos.

\hypertarget{el-tiempo-del-deber-de-los-levitas}{%
\subsection{El tiempo del deber de los
levitas}\label{el-tiempo-del-deber-de-los-levitas}}

\bibleverse{23} Yahvé habló a Moisés, diciendo: \bibleverse{24} ``Esto
es lo que se asigna a los levitas: de veinticinco años en adelante
entrarán a atender el servicio en la obra de la Tienda de Reunión;
\footnote{\textbf{8:24} Núm 4,3; Núm 4,23; Núm 4,30; Núm 4,47}
\bibleverse{25} y a partir de los cincuenta años se retirarán de hacer
la obra, y no servirán más, \bibleverse{26} sino que asistirán a sus
hermanos en la Tienda de Reunión, para cumplir con el deber, y no harán
ningún servicio. Así harás que los levitas cumplan con sus deberes''.

\hypertarget{la-celebraciuxf3n-posterior-a-la-pascua-para-los-inmundos-y-los-viajeros-la-pascua-de-los-extrauxf1os}{%
\subsection{La celebración posterior a la Pascua para los inmundos y los
viajeros; la pascua de los
extraños}\label{la-celebraciuxf3n-posterior-a-la-pascua-para-los-inmundos-y-los-viajeros-la-pascua-de-los-extrauxf1os}}

\hypertarget{section-8}{%
\section{9}\label{section-8}}

\bibleverse{1} Yahvé habló a Moisés en el desierto del Sinaí, en el
primer mes del segundo año después de que salieron de la tierra de
Egipto, diciendo: \bibleverse{2} ``Que los hijos de Israel celebren la
Pascua en su tiempo señalado. \footnote{\textbf{9:2} Éxod 12,3; Lev 23,5}
\bibleverse{3} El día catorce de este mes, al atardecer, la celebraréis
a su tiempo. La celebraréis según todos sus estatutos y según todas sus
ordenanzas''.

\bibleverse{4} Moisés dijo a los hijos de Israel que debían celebrar la
Pascua. \bibleverse{5} Celebraron la Pascua en el primer mes, el día
catorce del mes por la tarde, en el desierto de Sinaí. Conforme a todo
lo que el Señor ordenó a Moisés, así lo hicieron los hijos de Israel.
\bibleverse{6} Había ciertos hombres que estaban impuros a causa del
cadáver de un hombre, de modo que no podían celebrar la Pascua en ese
día, y se presentaron ante Moisés y Aarón en ese día. \footnote{\textbf{9:6}
  Núm 19,11} \bibleverse{7} Aquellos hombres le dijeron: ``Somos impuros
a causa del cadáver de un hombre. ¿Por qué se nos retiene, para que no
ofrezcamos la ofrenda de Yahvé en su tiempo señalado entre los hijos de
Israel?''

\bibleverse{8} Moisés les respondió: ``Esperen, para que yo oiga lo que
Yahvé mande sobre ustedes''.

\bibleverse{9} Yahvé habló a Moisés, diciendo: \bibleverse{10} ``Di a
los hijos de Israel: ``Si alguno de vosotros o de vuestras generaciones
es impuro por causa de un cadáver, o está de viaje lejos, aún así
celebrará la Pascua a Yahvé. \bibleverse{11} En el segundo mes, el día
catorce por la tarde la celebrarán; la comerán con panes sin levadura y
hierbas amargas. \bibleverse{12} No dejarán nada de ella para la mañana,
ni romperán ningún hueso. Conforme a todo el estatuto de la Pascua la
celebrarán. \bibleverse{13} Pero el hombre que esté limpio y no esté de
viaje, y no guarde la Pascua, esa persona será cortada de su pueblo. Por
no haber ofrecido la ofrenda de Yahvé en su tiempo señalado, ese hombre
cargará con su pecado.

\bibleverse{14} ``\,`Si un extranjero vive entre ustedes y desea
celebrar la Pascua a Yahvé, entonces lo hará según el estatuto de la
Pascua, y según su ordenanza. Tendréis un solo estatuto, tanto para el
extranjero como para el nacido en la tierra.'\,''

\hypertarget{la-apariciuxf3n-de-la-columna-de-nubes-y-fuego-sobre-el-santuario}{%
\subsection{La aparición de la columna de nubes y fuego sobre el
santuario}\label{la-apariciuxf3n-de-la-columna-de-nubes-y-fuego-sobre-el-santuario}}

\bibleverse{15} El día en que se levantó el tabernáculo, la nube cubrió
el tabernáculo, la Tienda del Testimonio. Al atardecer estaba sobre el
tabernáculo, como una apariencia de fuego, hasta la mañana. \footnote{\textbf{9:15}
  Éxod 40,34-38} \bibleverse{16} Así era continuamente. La nube lo
cubría, y la apariencia de fuego por la noche. \bibleverse{17} Cada vez
que la nube se alzaba sobre la Tienda, los hijos de Israel se
desplazaban; y en el lugar donde la nube permanecía, allí acampaban los
hijos de Israel. \bibleverse{18} Por orden de Yahvé, los hijos de Israel
viajaban, y por orden de Yahvé acampaban. Mientras la nube permaneció
sobre el tabernáculo, permanecieron acampados. \bibleverse{19} Cuando la
nube permanecía sobre el tabernáculo muchos días, entonces los hijos de
Israel cumplían la orden de Yavé y no viajaban. \bibleverse{20} A veces
la nube estaba unos pocos días sobre el tabernáculo; entonces, según el
mandato de Yavé, permanecían acampados, y según el mandato de Yavé,
viajaban. \bibleverse{21} A veces la nube estaba desde la tarde hasta la
mañana, y cuando la nube se levantaba por la mañana, viajaban; o de día
y de noche, cuando la nube se levantaba, viajaban. \bibleverse{22} Ya
sea que la nube permaneciera dos días, un mes o un año sobre el
tabernáculo, los hijos de Israel permanecían acampados y no viajaban;
pero cuando se levantaba, viajaban. \bibleverse{23} Al mandato del Señor
acamparon, y al mandato del Señor viajaron. Cumplieron el mandato de
Yahvé, a la orden de Yahvé por medio de Moisés.

\hypertarget{ordenanza-sobre-dos-trompetas-de-plata}{%
\subsection{Ordenanza sobre dos trompetas de
plata}\label{ordenanza-sobre-dos-trompetas-de-plata}}

\hypertarget{section-9}{%
\section{10}\label{section-9}}

\bibleverse{1} Yahvé habló a Moisés diciendo: \bibleverse{2} ``Haz dos
trompetas de plata. Las harás de plata labrada. Las usarás para llamar a
la congregación y para el desplazamiento de los campamentos. \footnote{\textbf{10:2}
  Núm 31,6} \bibleverse{3} Cuando las toquen, toda la congregación se
reunirá contigo a la puerta de la Tienda de Reunión. \bibleverse{4} Si
tocan una sola, los príncipes, los jefes de los millares de Israel, se
reunirán contigo. \bibleverse{5} Cuando toques una alarma, los
campamentos que se encuentran en el lado oriental avanzarán.
\bibleverse{6} Cuando toques la alarma por segunda vez, los campamentos
que se encuentran en el lado sur se adelantarán. Tocarán la alarma para
sus desplazamientos. \bibleverse{7} Pero cuando se reúna la asamblea,
tocaréis, pero no haréis sonar la alarma.

\bibleverse{8} ``Los hijos de Aarón, los sacerdotes, tocarán las
trompetas. Esto os servirá de estatuto para siempre por vuestras
generaciones. \bibleverse{9} Cuando vayas a la guerra en tu tierra
contra el adversario que te oprime, entonces tocarás la alarma con las
trompetas. Entonces serás recordado ante el Señor, tu Dios, y te
salvarás de tus enemigos.

\bibleverse{10} ``También en el día de tu alegría, y en tus fiestas
establecidas, y en los comienzos de tus meses, tocarás las trompetas
sobre tus holocaustos y sobre los sacrificios de tus ofrendas de paz; y
te servirán de recuerdo ante tu Dios. Yo soy Yahvé, tu Dios''.
\footnote{\textbf{10:10} Lev 23,24; 2Re 11,14; 2Cró 7,6}

\hypertarget{salida-del-sinauxed-hacia-el-desierto-de-paran}{%
\subsection{Salida del Sinaí hacia el desierto de
Paran}\label{salida-del-sinauxed-hacia-el-desierto-de-paran}}

\bibleverse{11} En el segundo año, en el segundo mes, a los veinte días
del mes, la nube se levantó de encima del tabernáculo de la alianza.
\bibleverse{12} Los hijos de Israel salieron del desierto de Sinaí, y la
nube se quedó en el desierto de Parán.

\hypertarget{descripciuxf3n-del-pedido-de-tren}{%
\subsection{Descripción del pedido de
tren}\label{descripciuxf3n-del-pedido-de-tren}}

\bibleverse{13} Primero avanzaron según el mandato de Yahvé por medio de
Moisés. \footnote{\textbf{10:13} Núm 1,1-4}

\bibleverse{14} En primer lugar, el estandarte del campamento de los
hijos de Judá avanzó según sus ejércitos. Nahsón hijo de Aminadab estaba
al frente de su ejército. \bibleverse{15} Netanel hijo de Zuar estaba al
frente del ejército de la tribu de los hijos de Isacar. \bibleverse{16}
Eliab hijo de Helón estaba al frente del ejército de la tribu de los
hijos de Zabulón. \bibleverse{17} El tabernáculo fue desmontado, y los
hijos de Gersón y los hijos de Merari, que llevaban el tabernáculo,
avanzaron. \bibleverse{18} El estandarte del campamento de Rubén avanzó
según sus ejércitos. Elizur hijo de Sedeur estaba al frente de su
ejército. \bibleverse{19} Selumiel hijo de Zurishaddai estaba al frente
del ejército de la tribu de los hijos de Simeón. \bibleverse{20} Eliasaf
hijo de Deuel estaba al frente del ejército de la tribu de los hijos de
Gad.

\bibleverse{21} Los coatitas se adelantaron llevando el santuario. Los
demás montaron el tabernáculo antes de que ellos llegaran.

\bibleverse{22} El estandarte del campamento de los hijos de Efraín
avanzaba según sus ejércitos. Elishama hijo de Ammihud estaba al frente
de su ejército. \bibleverse{23} Gamaliel hijo de Pedahzur estaba al
frente del ejército de la tribu de los hijos de Manasés. \bibleverse{24}
Abidán hijo de Gedeón estaba al frente del ejército de la tribu de los
hijos de Benjamín.

\bibleverse{25} El estandarte del campamento de los hijos de Dan, que
era la retaguardia de todos los campamentos, avanzaba según sus
ejércitos. Ahiezer hijo de Ammishaddai estaba al frente de su ejército.
\bibleverse{26} Pagiel hijo de Ocrán estaba al frente del ejército de la
tribu de los hijos de Aser. \bibleverse{27} Ahira hijo de Enán estaba al
frente del ejército de la tribu de los hijos de Neftalí. \bibleverse{28}
Así fueron los viajes de los hijos de Israel según sus ejércitos; y
avanzaron.

\hypertarget{moisuxe9s-intenta-ganarse-a-su-cuuxf1ado-hobab-como-guuxeda-para-el-viaje-hacia-adelante}{%
\subsection{Moisés intenta ganarse a su cuñado Hobab como guía para el
viaje hacia
adelante}\label{moisuxe9s-intenta-ganarse-a-su-cuuxf1ado-hobab-como-guuxeda-para-el-viaje-hacia-adelante}}

\bibleverse{29} Moisés le dijo a Hobab, hijo de Reuel el madianita, el
suegro de Moisés: ``Nos dirigimos al lugar del que Yahvé dijo: `Te lo
daré'. Ven con nosotros y te trataremos bien, porque Yahvé ha hablado
bien de Israel''. \footnote{\textbf{10:29} Jue 1,16; Éxod 2,18}

\bibleverse{30} Le dijo: ``No iré, sino que me iré a mi tierra y a mis
parientes''.

\bibleverse{31} Moisés dijo: ``No nos dejes, por favor; porque tú sabes
cómo hemos de acampar en el desierto, y puedes ser nuestros ojos.
\bibleverse{32} Será, si vas con nosotros --- sí, será --- que todo lo
bueno que haga Yahvé con nosotros, nosotros haremos lo mismo con
ustedes.''

\hypertarget{la-partida-del-monte-de-dios-bajo-la-guuxeda-del-arca}{%
\subsection{La partida del monte de Dios bajo la guía del
arca}\label{la-partida-del-monte-de-dios-bajo-la-guuxeda-del-arca}}

\bibleverse{33} Partieron del monte de Yahvé a tres días de camino. El
arca de la alianza de Yavé iba delante de ellos a tres días de camino,
para buscarles un lugar de descanso. \bibleverse{34} La nube del Señor
estaba sobre ellos durante el día, cuando partían del campamento.
\footnote{\textbf{10:34} Éxod 13,21} \bibleverse{35} Cuando el arca se
adelantó, Moisés dijo: ``¡Levántate, Yahvé, y que se dispersen tus
enemigos! Que los que te odian huyan ante ti''. \footnote{\textbf{10:35}
  Sal 68,1; Sal 132,8} \bibleverse{36} Cuando descansó, dijo: ``Vuelve,
Yahvé, a los diez mil de los miles de Israel''.

\hypertarget{el-murmullo-de-la-gente-y-la-fogata-en-thabera}{%
\subsection{El murmullo de la gente y la fogata en
Thabera}\label{el-murmullo-de-la-gente-y-la-fogata-en-thabera}}

\hypertarget{section-10}{%
\section{11}\label{section-10}}

\bibleverse{1} El pueblo se quejaba a los oídos del Señor. Cuando Yahvé
lo oyó, se encendió su ira; y el fuego de Yahvé ardió entre ellos, y
consumió algunas de las afueras del campamento. \footnote{\textbf{11:1}
  Lev 10,2} \bibleverse{2} El pueblo clamó a Moisés; Moisés oró a Yavé,
y el fuego se calmó. \bibleverse{3} El nombre de ese lugar fue llamado
Taberah, porque el fuego de Yavé ardía entre ellos.

\hypertarget{la-gente-se-queja-de-la-comida}{%
\subsection{La gente se queja de la
comida}\label{la-gente-se-queja-de-la-comida}}

\bibleverse{4} La multitud mixta que estaba en medio de ellos tuvo mucha
lujuria; también los hijos de Israel volvieron a llorar y dijeron:
``¿Quién nos dará de comer? \footnote{\textbf{11:4} Éxod 16,3}
\bibleverse{5} Nos acordamos del pescado que comimos en Egipto de balde;
de los pepinos, los melones, los puerros, las cebollas y los ajos;
\bibleverse{6} pero ahora hemos perdido el apetito. No hay nada más que
este maná para mirar''. \bibleverse{7} El maná era como una semilla de
cilantro, y parecía bdellium. \footnote{\textbf{11:7} El bdellium es una
  resina que se extrae de ciertos árboles africanos.} \footnote{\textbf{11:7}
  Éxod 16,14-31} \bibleverse{8} La gente iba de un lado a otro, lo
recogía y lo molía en molinos o lo batía en morteros, lo hervía en ollas
y hacía tortas con él. Su sabor era como el del aceite fresco.
\bibleverse{9} Cuando el rocío caía sobre el campamento por la noche, el
maná caía sobre él.

\hypertarget{el-lamento-de-moisuxe9s-ante-dios}{%
\subsection{El lamento de Moisés ante
Dios}\label{el-lamento-de-moisuxe9s-ante-dios}}

\bibleverse{10} Moisés oyó que el pueblo lloraba en sus familias, cada
uno a la puerta de su tienda; y la ira de Yahvé ardía en gran medida, y
Moisés se disgustó. \bibleverse{11} Moisés dijo a Yavé: ``¿Por qué has
tratado tan mal a tu siervo? ¿Por qué no he hallado gracia ante tus
ojos, para que pongas sobre mí la carga de todo este pueblo?
\bibleverse{12} ¿He concebido yo a todo este pueblo? ¿Acaso los he
sacado, para que me digas: ``Llévalos en tu seno, como una nodriza lleva
a un niño de pecho, a la tierra que juraste a sus padres''?
\bibleverse{13} ¿De dónde podría sacar carne para dar a todo este
pueblo? Porque lloran ante mí, diciendo: `Danos carne para que comamos'.
\bibleverse{14} No puedo soportar solo a todo este pueblo, porque es
demasiado pesado para mí. \bibleverse{15} Si me tratas así, por favor,
mátame ahora mismo, si es que he hallado gracia ante tus ojos; y no me
dejes ver mi miseria.'' \footnote{\textbf{11:15} Éxod 32,32}

\hypertarget{ordenanza-de-dios-nombramiento-de-setenta-asistentes-de-moisuxe9s-la-promesa-de-dios-de-donaciuxf3n-de-carne-la-respuesta-incruxe9dula-de-moisuxe9s}{%
\subsection{Ordenanza de Dios (nombramiento de setenta asistentes de
Moisés) La promesa de Dios de donación de carne; la respuesta incrédula
de
Moisés}\label{ordenanza-de-dios-nombramiento-de-setenta-asistentes-de-moisuxe9s-la-promesa-de-dios-de-donaciuxf3n-de-carne-la-respuesta-incruxe9dula-de-moisuxe9s}}

\bibleverse{16} El Señor le dijo a Moisés: ``Reúne conmigo a setenta
hombres de los ancianos de Israel, que sepas que son los ancianos del
pueblo y los oficiales sobre ellos, y tráelos a la Tienda de Reunión,
para que estén allí contigo. \footnote{\textbf{11:16} Éxod 18,21; Éxod
  24,1} \bibleverse{17} Yo bajaré y hablaré con ustedes allí. Tomaré del
Espíritu que está sobre ti, y lo pondré sobre ellos; y ellos llevarán la
carga del pueblo contigo, para que no la lleves tú solo.

\bibleverse{18} ``Di al pueblo: ``Santificaos para prepararos para
mañana, y comeréis carne; porque habéis llorado a oídos de Yahvé,
diciendo: ``¿Quién nos dará de comer? Porque nos fue bien en Egipto''.
Por lo tanto, Yahvé les dará carne, y ustedes comerán. \footnote{\textbf{11:18}
  Éxod 19,10} \bibleverse{19} No comeréis un solo día, ni dos días, ni
cinco días, ni diez días, ni veinte días, \bibleverse{20} sino un mes
entero, hasta que os salga por las narices, y os resulte repugnante;
porque habéis rechazado a Yahvé, que está en medio de vosotros, y habéis
llorado ante él, diciendo: ``¿Por qué salimos de Egipto?''\,''

\bibleverse{21} Moisés dijo: ``El pueblo, en medio del cual me
encuentro, es de seiscientos mil hombres de a pie; y tú has dicho: `Les
daré comida para que coman todo un mes'. \bibleverse{22} ¿Se
sacrificarán para ellos rebaños y manadas, para que les baste? ¿Se
reunirán para ellos todos los peces del mar, para que les basten?''
\footnote{\textbf{11:22} Juan 6,7}

\bibleverse{23} Yahvé dijo a Moisés: ``¿Se ha acortado la mano de Yahvé?
Ahora verás si mi palabra se cumple o no''. \footnote{\textbf{11:23} Is
  50,2; Is 59,1}

\hypertarget{el-entusiasmo-profuxe9tico-de-los-setenta-ancianos}{%
\subsection{El entusiasmo profético de los setenta
ancianos}\label{el-entusiasmo-profuxe9tico-de-los-setenta-ancianos}}

\bibleverse{24} Moisés salió y le contó al pueblo las palabras de Yahvé,
y reunió a setenta hombres de los ancianos del pueblo, y los puso
alrededor de la Tienda. \bibleverse{25} El Señor descendió en la nube y
le habló, y tomó del Espíritu que estaba sobre él y lo puso sobre los
setenta ancianos. Cuando el Espíritu se posó sobre ellos, profetizaron,
pero no lo hicieron más. \bibleverse{26} Pero dos hombres se quedaron en
el campamento. El nombre de uno era Eldad, y el del otro Medad; y el
Espíritu reposó sobre ellos. Eran de los que estaban escritos, pero no
habían salido a la Tienda; y profetizaban en el campamento.
\bibleverse{27} Un joven corrió y se lo comunicó a Moisés, diciendo:
``¡Eldad y Medad están profetizando en el campamento!''

\bibleverse{28} Josué, hijo de Nun, siervo de Moisés, uno de sus
elegidos, respondió: ``¡Mi señor Moisés, prohíbelo!'' \footnote{\textbf{11:28}
  Núm 13,16; Éxod 24,13}

\bibleverse{29} Moisés le dijo: ``¿Estás celoso por mí? Ojalá todo el
pueblo de Yahvé fuera profeta, para que Yahvé pusiera su Espíritu en
ellos''. \footnote{\textbf{11:29} Mar 9,39; Jl 2,28}

\bibleverse{30} Moisés entró en el campamento, él y los ancianos de
Israel.

\hypertarget{alimentaciuxf3n-de-codornices-juicio-de-dios-las-tumbas-del-placer}{%
\subsection{Alimentación de codornices; Juicio de Dios; las tumbas del
placer}\label{alimentaciuxf3n-de-codornices-juicio-de-dios-las-tumbas-del-placer}}

\bibleverse{31} Salió un viento de Yahvé y trajo codornices del mar, y
las dejó caer junto al campamento, como a un día de camino de un lado, y
a un día de camino del otro, alrededor del campamento, y como a dos
codos\footnote{\textbf{11:31} Un codo es la longitud desde la punta del
  dedo corazón hasta el codo del brazo de un hombre, es decir, unas 18
  pulgadas o 46 centímetros.} sobre la superficie de la tierra.
\footnote{\textbf{11:31} Éxod 16,13} \bibleverse{32} El pueblo se
levantó todo ese día, toda esa noche y todo el día siguiente, y recogió
las codornices. El que menos juntó, juntó diez homers;\footnote{\textbf{11:32}
  1 homer es de unos 220 litros o 6 bushels} y se las repartieron todas
alrededor del campamento. \bibleverse{33} Mientras la carne estaba
todavía entre sus dientes, antes de ser masticada, la ira de Yavé ardió
contra el pueblo, y Yavé hirió al pueblo con una plaga muy grande.
\bibleverse{34} El nombre de ese lugar fue llamado Kibroth
Hattaavah,\footnote{\textbf{11:34} Kibroth Hattaavah significa ``tumbas
  de la lujuria''} , porque allí enterraron a la gente que codiciaba.
\footnote{\textbf{11:34} 1Cor 10,6}

\bibleverse{35} Desde Kibroth Hattaavah el pueblo viajó a Hazeroth; y se
quedaron en Hazeroth.

\hypertarget{la-rebeliuxf3n-de-miriam-y-aaruxf3n-contra-moisuxe9s}{%
\subsection{La rebelión de Miriam y Aarón contra
Moisés}\label{la-rebeliuxf3n-de-miriam-y-aaruxf3n-contra-moisuxe9s}}

\hypertarget{section-11}{%
\section{12}\label{section-11}}

\bibleverse{1} Miriam y Aarón hablaron contra Moisés a causa de la mujer
cusita con la que se había casado, pues se había casado con una mujer
cusita. \footnote{\textbf{12:1} Éxod 2,21} \bibleverse{2} Dijeron:
``¿Acaso Yahvé ha hablado sólo con Moisés? ¿No ha hablado también con
nosotros?'' Y Yahvé lo escuchó.

\bibleverse{3} El hombre Moisés era muy humilde, más que todos los
hombres que había sobre la superficie de la tierra.

\hypertarget{dios-estuxe1-defendiendo-a-moisuxe9s-el-castigo-de-miriam}{%
\subsection{Dios está defendiendo a Moisés; El castigo de
miriam}\label{dios-estuxe1-defendiendo-a-moisuxe9s-el-castigo-de-miriam}}

\bibleverse{4} Yahvé habló de repente a Moisés, a Aarón y a Miriam:
``¡Salgan ustedes tres a la Tienda del Encuentro!'' Los tres salieron.
\bibleverse{5} El Señor descendió en una columna de nube, se paró a la
puerta de la Tienda y llamó a Aarón y a Miriam, y ambos se acercaron.
\footnote{\textbf{12:5} Éxod 16,10} \bibleverse{6} El dijo: ``Escuchen
ahora mis palabras. Si hay un profeta entre ustedes, yo, Yahvé, me daré
a conocer a él en una visión. Hablaré con él en un sueño. \bibleverse{7}
Mi siervo Moisés no es así. Él es fiel en toda mi casa. \footnote{\textbf{12:7}
  Heb 3,2} \bibleverse{8} Con él hablaré de boca a boca, claramente y no
con enigmas, y verá la forma de Yavé. ¿Por qué, pues, no temiste hablar
contra mi siervo, contra Moisés?'' \footnote{\textbf{12:8} Éxod 33,11;
  Éxod 33,23} \bibleverse{9} La ira de Yahvé ardió contra ellos, y se
marchó.

\bibleverse{10} La nube se apartó de la Tienda, y he aquí que Miriam
estaba leprosa, blanca como la nieve. Aarón miró a Miriam, y he aquí que
estaba leprosa. \footnote{\textbf{12:10} Deut 24,9}

\hypertarget{la-intercesiuxf3n-de-aaruxf3n-y-moisuxe9s-la-respuesta-de-dios-la-curaciuxf3n-de-miriam-llegada-al-desierto-de-paran}{%
\subsection{La intercesión de Aarón y Moisés; La respuesta de Dios; La
curación de Miriam; Llegada al desierto de
Paran}\label{la-intercesiuxf3n-de-aaruxf3n-y-moisuxe9s-la-respuesta-de-dios-la-curaciuxf3n-de-miriam-llegada-al-desierto-de-paran}}

\bibleverse{11} Aarón dijo a Moisés: ``Oh, señor mío, por favor, no nos
tomes en cuenta este pecado, en el que hemos actuado neciamente y en el
que hemos pecado. \bibleverse{12} Te ruego que no sea como un muerto,
cuya carne está medio consumida cuando sale del vientre de su madre''.

\bibleverse{13} Moisés clamó a Yahvé diciendo: ``¡Sánala, Dios, te lo
ruego!'' \footnote{\textbf{12:13} Éxod 15,26}

\bibleverse{14} Yahvé dijo a Moisés: ``Si su padre no hubiera hecho más
que escupirle en la cara, ¿no debería estar avergonzada siete días? Que
la encierren fuera del campamento durante siete días, y después la
volverán a meter''. \footnote{\textbf{12:14} Lev 13,46}

\bibleverse{15} Miriam fue encerrada fuera del campamento durante siete
días, y el pueblo no viajó hasta que Miriam fue traída de nuevo.
\bibleverse{16} Después, el pueblo partió de Hazerot y acampó en el
desierto de Parán.

\hypertarget{envuxedo-de-los-doce-exploradores}{%
\subsection{Envío de los doce
exploradores}\label{envuxedo-de-los-doce-exploradores}}

\hypertarget{section-12}{%
\section{13}\label{section-12}}

\bibleverse{1} Yahvé habló a Moisés, diciendo: \bibleverse{2} ``Envía
hombres para que espíen la tierra de Canaán, que yo doy a los hijos de
Israel. De cada tribu de sus padres, enviarás un hombre, cada uno de
ellos un príncipe entre ellos''.

\bibleverse{3} Moisés los envió desde el desierto de Parán, según el
mandato de Yahvé. Todos ellos eran hombres que eran jefes de los hijos
de Israel. \bibleverse{4} Estos eran sus nombres: De la tribu de Rubén,
Shammua hijo de Zaccur. \bibleverse{5} De la tribu de Simeón, Safat hijo
de Hori. \bibleverse{6} De la tribu de Judá, Caleb hijo de Jefone.
\footnote{\textbf{13:6} Jos 14,7} \bibleverse{7} De la tribu de Isacar,
Igal hijo de José. \bibleverse{8} De la tribu de Efraín, Oseas hijo de
Nun. \footnote{\textbf{13:8} Núm 13,16; 1Cró 7,27} \bibleverse{9} De la
tribu de Benjamín, Palti, hijo de Raphu. \bibleverse{10} De la tribu de
Zabulón, Gaddiel hijo de Sodi. \bibleverse{11} De la tribu de José, de
la tribu de Manasés, Gaddi hijo de Susi. \bibleverse{12} De la tribu de
Dan, Ammiel hijo de Gemalli. \bibleverse{13} De la tribu de Aser, Setur,
hijo de Miguel. \bibleverse{14} De la tribu de Neftalí, Nahbi hijo de
Vofi. \bibleverse{15} De la tribu de Gad, Geuel hijo de Machi.
\bibleverse{16} Estos son los nombres de los hombres que Moisés envió a
espiar la tierra. Moisés llamó Josué a Oseas hijo de Nun. \footnote{\textbf{13:16}
  Núm 11,28} \bibleverse{17} Moisés los envió a espiar la tierra de
Canaán, y les dijo: ``Suban por este camino del sur y suban a la región
montañosa.

\hypertarget{la-instrucciuxf3n-de-moisuxe9s-a-los-espuxedas}{%
\subsection{La instrucción de Moisés a los
espías}\label{la-instrucciuxf3n-de-moisuxe9s-a-los-espuxedas}}

\bibleverse{18} Ved la tierra, cómo es; y el pueblo que la habita, si es
fuerte o débil, si es poco o mucho; \bibleverse{19} y cómo es la tierra
que habitan, si es buena o mala; y qué ciudades son las que habitan, si
en campamentos o en fortalezas; \bibleverse{20} y cómo es la tierra, si
es fértil o pobre, si hay madera en ella o no. Sé valiente y trae algo
del fruto de la tierra''. Era el tiempo de las primeras uvas maduras.

\bibleverse{21} Subieron, pues, y reconocieron la tierra desde el
desierto de Zin hasta Rehob, hasta la entrada de Hamat.

\hypertarget{explorando-la-tierra}{%
\subsection{Explorando la tierra}\label{explorando-la-tierra}}

\bibleverse{22} Subieron por el sur y llegaron a Hebrón, donde estaban
Ahimán, Sesai y Talmai, hijos de Anac. (Ahora bien, Hebrón fue
construida siete años antes que Zoán en Egipto.) \bibleverse{23}
Llegaron al valle de Escol, y cortaron de allí una rama con un racimo de
uvas, y la llevaron en un bastón entre dos. También llevaron algunas
granadas e higos. \bibleverse{24} Aquel lugar fue llamado valle de
Escol, por el racimo que los hijos de Israel cortaron de allí.
\bibleverse{25} Volvieron de espiar la tierra al cabo de cuarenta días.

\hypertarget{regreso-e-informe-de-los-emisarios}{%
\subsection{Regreso e informe de los
emisarios}\label{regreso-e-informe-de-los-emisarios}}

\bibleverse{26} Fueron y vinieron a Moisés, a Aarón y a toda la
congregación de los hijos de Israel, al desierto de Parán, a Cades, y
les trajeron la noticia a ellos y a toda la congregación. Les mostraron
el fruto de la tierra. \bibleverse{27} Ellos se lo contaron y dijeron:
``Hemos llegado a la tierra a la que nos enviaste. Ciertamente fluye
leche y miel, y éste es su fruto. \footnote{\textbf{13:27} Éxod 3,8;
  Éxod 3,17} \bibleverse{28} Sin embargo, el pueblo que habita la tierra
es fuerte, y las ciudades están fortificadas y son muy grandes. Además,
vimos allí a los hijos de Anac. \bibleverse{29} Amalek habita en la
tierra del Sur. El hitita, el jebuseo y el amorreo habitan en la región
montañosa. El cananeo habita junto al mar y al lado del Jordán''.

\bibleverse{30} Caleb calmó al pueblo ante Moisés y dijo: ``¡Subamos de
inmediato y poseámosla, pues somos capaces de vencerla!'' \footnote{\textbf{13:30}
  Núm 13,6}

\hypertarget{las-palabras-tranquilizadoras-de-caleb-y-las-palabras-desalentadoras-de-los-otros-exploradores}{%
\subsection{Las palabras tranquilizadoras de Caleb y las palabras
desalentadoras de los otros
exploradores}\label{las-palabras-tranquilizadoras-de-caleb-y-las-palabras-desalentadoras-de-los-otros-exploradores}}

\bibleverse{31} Pero los hombres que subieron con él dijeron: ``No somos
capaces de subir contra ese pueblo, porque es más fuerte que nosotros''.
\bibleverse{32} Presentaron a los hijos de Israel un mal informe de la
tierra que habían espiado, diciendo: ``La tierra por la que hemos pasado
para espiarla es una tierra que devora a sus habitantes, y todos los
pueblos que vimos en ella son hombres de gran estatura. \bibleverse{33}
Allí vimos a los Nefilim,\footnote{\textbf{13:33} o, gigantes} los hijos
de Anak, que provienen de los Nefilim.\footnote{\textbf{13:33} o,
  gigantes} Éramos a nuestra vista como saltamontes, y así éramos a su
vista''. \footnote{\textbf{13:33} Deut 9,2}

\hypertarget{el-efecto-del-informe-en-la-gente}{%
\subsection{El efecto del informe en la
gente}\label{el-efecto-del-informe-en-la-gente}}

\hypertarget{section-13}{%
\section{14}\label{section-13}}

\bibleverse{1} Toda la congregación alzó la voz y gritó, y el pueblo
lloró aquella noche. \bibleverse{2} Todos los hijos de Israel murmuraron
contra Moisés y contra Aarón. Toda la congregación les dijo: ``¡Ojalá
hubiéramos muerto en la tierra de Egipto, o hubiéramos muerto en este
desierto! \footnote{\textbf{14:2} Éxod 16,3} \bibleverse{3} ¿Por qué nos
trae el Señor a esta tierra para que caigamos a espada? ¡Nuestras
esposas y nuestros pequeños serán capturados o asesinados! ¿No sería
mejor que volviéramos a Egipto?'' \footnote{\textbf{14:3} Sal 106,24}
\bibleverse{4} Se dijeron unos a otros: ``Elijamos un líder y volvamos a
Egipto''.

\bibleverse{5} Entonces Moisés y Aarón se postraron ante toda la
asamblea de la congregación de los hijos de Israel. \footnote{\textbf{14:5}
  Núm 16,4}

\hypertarget{el-intento-fallido-de-apaciguamiento-de-joshua-y-caleb}{%
\subsection{El intento fallido de apaciguamiento de Joshua y
Caleb}\label{el-intento-fallido-de-apaciguamiento-de-joshua-y-caleb}}

\bibleverse{6} Josué, hijo de Nun, y Caleb, hijo de Jefone, que eran de
los que espiaban la tierra, se rasgaron las vestiduras. \footnote{\textbf{14:6}
  Núm 13,16; Núm 13,30} \bibleverse{7} Hablaron a toda la congregación
de los hijos de Israel, diciendo: ``La tierra que atravesamos para
espiarla es una tierra sumamente buena. \bibleverse{8} Si Yahvé se
complace en nosotros, nos introducirá en esta tierra y nos la dará: una
tierra que mana leche y miel. \footnote{\textbf{14:8} Núm 13,27}
\bibleverse{9} Sólo que no se rebelen contra Yavé, ni teman al pueblo de
la tierra, porque ellos son el pan para nosotros. Su defensa ha sido
retirada de encima de ellos, y Yahvé está con nosotros. No les temas''.

\hypertarget{ira-de-dios-la-exitosa-intercesiuxf3n-de-moisuxe9s-el-juicio-divino}{%
\subsection{Ira de Dios; la exitosa intercesión de Moisés; el juicio
divino}\label{ira-de-dios-la-exitosa-intercesiuxf3n-de-moisuxe9s-el-juicio-divino}}

\bibleverse{10} Pero toda la congregación amenazó con apedrearlos. La
gloria de Yahvé se presentó en la Tienda del Encuentro a todos los hijos
de Israel. \footnote{\textbf{14:10} Éxod 17,4; Éxod 16,10}
\bibleverse{11} Yahvé dijo a Moisés: ``¿Hasta cuándo me despreciará este
pueblo? ¿Hasta cuándo no creerán en mí, por todas las señales que he
realizado entre ellos? \bibleverse{12} Los heriré con la peste y los
desheredaré, y haré de ustedes una nación más grande y poderosa que
ellos.'' \footnote{\textbf{14:12} Éxod 32,10-14}

\bibleverse{13} Moisés dijo a Yahvé: ``Entonces los egipcios lo oirán,
porque tú hiciste surgir a este pueblo con tu fuerza de entre ellos.
\bibleverse{14} Lo contarán a los habitantes de esta tierra. Han oído
que tú, Yahvé, estás en medio de este pueblo; porque a ti, Yahvé, te ven
cara a cara, y tu nube está sobre ellos, y tú vas delante de ellos, en
una columna de nube de día, y en una columna de fuego de noche.
\bibleverse{15} Ahora bien, si matas a este pueblo como a un solo
hombre, las naciones que han oído tu fama hablarán diciendo:
\bibleverse{16} `Porque Yahvé no pudo llevar a este pueblo a la tierra
que le había jurado, por eso lo ha matado en el desierto.' \footnote{\textbf{14:16}
  Deut 9,28} \bibleverse{17} Ahora, por favor, haz que el poder de
Yahvé\footnote{\textbf{14:17} La palabra traducida ``Señor'' es
  ``Adonai''.} sea grande, según has hablado, diciendo: \bibleverse{18}
`Yahvé es lento para la ira, y abundante en misericordia, que perdona la
iniquidad y la desobediencia; y de ninguna manera exculpará al culpable,
visitando la iniquidad de los padres en los hijos, en la tercera y en la
cuarta generación.' \footnote{\textbf{14:18} Éxod 34,6-7}
\bibleverse{19} Por favor, perdona la iniquidad de este pueblo según la
grandeza de tu amorosa bondad, y tal como has perdonado a este pueblo,
desde Egipto hasta ahora.''

\bibleverse{20} Yahvé dijo: ``Yo he perdonado según tu palabra;
\bibleverse{21} pero en realidad --- mientras yo viva y toda la tierra
se llene de la gloria de Yahvé --- \footnote{\textbf{14:21} Éxod 9,16}
\bibleverse{22} porque todos esos hombres que han visto mi gloria y mis
señales, que hice en Egipto y en el desierto, me han tentado estas diez
veces, y no han escuchado mi voz; \bibleverse{23} ciertamente no verán
la tierra que juré a sus padres, ni la verá ninguno de los que me
despreciaron. \footnote{\textbf{14:23} Sal 95,11; Heb 3,17-19}
\bibleverse{24} Pero a mi siervo Caleb, por haber tenido otro espíritu
con él y haberme seguido plenamente, lo introduciré en la tierra a la
que entró. Su descendencia la poseerá. \footnote{\textbf{14:24} Jos
  14,6; Jos 14,9} \bibleverse{25} Puesto que el amalecita y el cananeo
habitan en el valle, mañana se volverán y entrarán en el desierto por el
camino del Mar Rojo.''

\hypertarget{el-castigo-de-dios-para-las-personas-y-los-espuxedas-se-especifica-con-muxe1s-detalle}{%
\subsection{El castigo de Dios para las personas y los espías se
especifica con más
detalle}\label{el-castigo-de-dios-para-las-personas-y-los-espuxedas-se-especifica-con-muxe1s-detalle}}

\bibleverse{26} Yahvé habló a Moisés y a Aarón, diciendo:
\bibleverse{27} ``¿Hasta cuándo tendré que soportar a esta congregación
malvada que se queja contra mí? He oído las quejas de los hijos de
Israel, que se quejan contra mí. \bibleverse{28} Diles: ``Vivo yo, dice
Yavé, que tal como habéis hablado en mis oídos, así haré con vosotros.
\bibleverse{29} Vuestros cadáveres caerán en este desierto; y todos los
que fueron contados de vosotros, según vuestro número total, de veinte
años para arriba, que se han quejado contra mí, \bibleverse{30}
ciertamente no entraréis en la tierra sobre la cual juré que os haría
habitar en ella, excepto Caleb hijo de Jefone y Josué hijo de Nun.
\bibleverse{31} Pero traeré a vuestros pequeños que dijisteis que debían
ser capturados o muertos, y ellos conocerán la tierra que habéis
rechazado. \bibleverse{32} Pero en cuanto a vosotros, vuestros cadáveres
caerán en este desierto. \bibleverse{33} Vuestros hijos serán errantes
en el desierto durante cuarenta años, y soportarán vuestra prostitución,
hasta que vuestros cadáveres se consuman en el desierto. \bibleverse{34}
Después del número de los días en que espiasteis la tierra, cuarenta
días, por cada día un año, llevaréis vuestras iniquidades, cuarenta
años, y conoceréis mi alienación.' \footnote{\textbf{14:34} Jer 2,19}
\bibleverse{35} Yo, Yahvé, he hablado. Ciertamente haré esto a toda esta
congregación malvada que se ha reunido contra mí. En este desierto serán
consumidos, y allí morirán''.

\hypertarget{muerte-repentina-de-los-espuxedas-excepto-josuuxe9-y-caleb}{%
\subsection{Muerte repentina de los espías excepto Josué y
Caleb}\label{muerte-repentina-de-los-espuxedas-excepto-josuuxe9-y-caleb}}

\bibleverse{36} Los hombres que Moisés envió a espiar la tierra, y que
regresaron e hicieron que toda la congregación murmurara contra él
presentando un mal informe contra la tierra, \footnote{\textbf{14:36}
  1Cor 10,5; 1Cor 10,10; Jds 1,5} \bibleverse{37} incluso aquellos
hombres que presentaron un mal informe de la tierra, murieron por la
plaga ante Yahvé. \bibleverse{38} Pero Josué, hijo de Nun, y Caleb, hijo
de Jefone, quedaron vivos de aquellos hombres que fueron a espiar la
tierra. \footnote{\textbf{14:38} Núm 14,30}

\hypertarget{arrepentimiento-del-pueblo-el-intento-fallido-de-penetrar-en-el-pauxeds-enemigo}{%
\subsection{Arrepentimiento del pueblo; el intento fallido de penetrar
en el país
enemigo}\label{arrepentimiento-del-pueblo-el-intento-fallido-de-penetrar-en-el-pauxeds-enemigo}}

\bibleverse{39} Moisés contó estas palabras a todos los hijos de Israel,
y el pueblo se lamentó mucho. \bibleverse{40} Se levantaron de madrugada
y subieron a la cima del monte, diciendo: ``Ya estamos aquí y subiremos
al lugar que Yahvé ha prometido, porque hemos pecado.'' \footnote{\textbf{14:40}
  Núm 13,17}

\bibleverse{41} Moisés dijo: ``¿Por qué desobedecen ahora el mandamiento
de Yavé, ya que no prosperará? \bibleverse{42} No suban, porque Yahvé no
está en medio de ustedes; así no serán derribados ante sus enemigos.
\bibleverse{43} Porque allí están el amalecita y el cananeo delante de
ustedes, y caerán a espada porque se apartaron de seguir a Yahvé; por
eso Yahvé no estará con ustedes.''

\bibleverse{44} Pero ellos se atrevieron a subir a la cima de la
montaña. Sin embargo, el arca de la alianza de Yahvé y Moisés no
salieron del campamento. \bibleverse{45} Entonces bajaron los amalecitas
y los cananeos que vivían en ese monte, y los golpearon y derribaron
hasta Horma. \footnote{\textbf{14:45} Núm 31,3}

\hypertarget{regulaciones-con-respecto-a-las-ofrendas-de-comida-y-bebida-como-adiciuxf3n-a-los-holocaustos-y-las-ofrendas-de-salvaciuxf3n}{%
\subsection{Regulaciones con respecto a las ofrendas de comida y bebida
como adición a los holocaustos y las ofrendas de
salvación}\label{regulaciones-con-respecto-a-las-ofrendas-de-comida-y-bebida-como-adiciuxf3n-a-los-holocaustos-y-las-ofrendas-de-salvaciuxf3n}}

\hypertarget{section-14}{%
\section{15}\label{section-14}}

\bibleverse{1} Yahvé habló a Moisés, diciendo: \bibleverse{2} ``Habla a
los hijos de Israel y diles: `Cuando hayáis entrado en la tierra de
vuestras moradas, que yo os doy, \bibleverse{3} y hagáis una ofrenda por
fuego a Yahvé --- un holocausto, o un sacrificio, para cumplir un voto o
como ofrenda voluntaria, o en vuestras fiestas establecidas, para hacer
un aroma agradable a Yahvé, de la manada o del rebaño --- \footnote{\textbf{15:3}
  Lev 7,16} \bibleverse{4} entonces el que ofrezca su ofrenda ofrecerá a
Yahvé una ofrenda de harina de una décima parte de un efa\footnote{\textbf{15:4}
  1 efa son unos 22 litros o unos 2/3 de una fanega} de harina fina
mezclada con una cuarta parte de un hin\footnote{\textbf{15:4} Una hin
  es aproximadamente 6,5 litros o 1,7 galones.} de aceite.
\bibleverse{5} Prepararás vino para la libación, la cuarta parte de un
hin, con el holocausto o para el sacrificio, por cada cordero.
\footnote{\textbf{15:5} Núm 28,7}

\bibleverse{6} ``\,`Por un carnero, prepararás para una ofrenda dos
décimas de efa\footnote{\textbf{15:6} 1 efa son unos 22 litros o unos
  2/3 de una fanega} de harina fina mezclada con la tercera parte de un
hin de aceite; \bibleverse{7} y para la libación ofrecerás la tercera
parte de un hin de vino, de aroma agradable para Yahvé. \bibleverse{8}
Cuando prepares un toro para un holocausto o para un sacrificio, para
cumplir un voto, o para ofrendas de paz a Yahvé, \bibleverse{9} entonces
ofrecerás con el toro una ofrenda de harina de tres décimas de
efa\footnote{\textbf{15:9} 1 efa son unos 22 litros o unos 2/3 de una
  fanega} de harina fina mezclada con medio hin de aceite;
\bibleverse{10} y ofrecerás para la libación medio hin de vino, como
ofrenda encendida, de aroma agradable a Yahvé. \bibleverse{11} Así se
hará por cada toro, por cada carnero, por cada uno de los corderos o de
los cabritos. \bibleverse{12} Según el número que prepares, así harás
con cada uno según su número.

\bibleverse{13} ``\,`Todos los nativos harán estas cosas de esta manera,
al ofrecer una ofrenda encendida, de aroma agradable a Yahvé.
\bibleverse{14} Si un extranjero vive como forastero con vosotros, o
quienquiera que esté entre vosotros a lo largo de vuestras generaciones,
y ofrece una ofrenda encendida de aroma agradable a Yahvé, como hacéis
vosotros, así lo hará. \bibleverse{15} Para la asamblea, habrá un
estatuto para vosotros y para el extranjero que vive como tal, un
estatuto para siempre a lo largo de vuestras generaciones. Como
vosotros, así será el extranjero ante Yahvé. \footnote{\textbf{15:15}
  Éxod 12,49} \bibleverse{16} Una sola ley y un solo estatuto habrá para
vosotros y para el extranjero que vive como forastero con vosotros.'\,''

\hypertarget{disposiciuxf3n-sobre-los-primeros-pasteles}{%
\subsection{Disposición sobre los primeros
pasteles}\label{disposiciuxf3n-sobre-los-primeros-pasteles}}

\bibleverse{17} Yahvé habló a Moisés, diciendo: \bibleverse{18} ``Habla
a los hijos de Israel y diles: `Cuando lleguéis a la tierra a la que os
traigo, \bibleverse{19} entonces será que cuando comáis del pan de la
tierra, ofreceréis una ofrenda mecida a Yahvé. \footnote{\textbf{15:19}
  Éxod 23,16; Éxod 23,19} \bibleverse{20} De lo primero de tu masa
ofrecerás una torta como ofrenda mecida. Como la ofrenda mecida de la
era, así la ofrecerás. \bibleverse{21} De las primicias de tu masa,
ofrecerás a Yavé una ofrenda mecida a lo largo de tus generaciones.

\hypertarget{reglas-con-respecto-a-las-ofrendas-por-el-pecado-por-obrar-mal-involuntariamente-impunidad-por-transgresiones-intencionales}{%
\subsection{Reglas con respecto a las ofrendas por el pecado por obrar
mal involuntariamente; Impunidad por transgresiones
intencionales}\label{reglas-con-respecto-a-las-ofrendas-por-el-pecado-por-obrar-mal-involuntariamente-impunidad-por-transgresiones-intencionales}}

\bibleverse{22} ``\,`Cuando erréis y no observéis todos estos
mandamientos que Yahvé ha dicho a Moisés --- \footnote{\textbf{15:22}
  Lev 4,2; Lev 4,13} \bibleverse{23} todo lo que Yahvé os ha mandado por
medio de Moisés, desde el día en que Yahvé dio el mandamiento y en
adelante por vuestras generaciones --- \bibleverse{24} entonces será si
se hizo sin querer, sin conocimiento de la congregación, toda la
congregación ofrecerá un novillo en holocausto, como aroma agradable a
Yahvé, con su ofrenda y su libación, según la ordenanza, y un macho
cabrío como ofrenda por el pecado. \bibleverse{25} El sacerdote hará la
expiación por toda la congregación de los hijos de Israel, y serán
perdonados; porque fue un error, y han traído su ofrenda, una ofrenda
encendida a Yahvé, y su ofrenda por el pecado ante Yahvé, por su error.
\bibleverse{26} Toda la congregación de los hijos de Israel será
perdonada, así como el extranjero que vive como forastero en medio de
ellos; porque con respecto a todo el pueblo, se hizo sin querer.

\bibleverse{27} ``\,`Si una persona peca involuntariamente, ofrecerá una
cabra hembra de un año como ofrenda por el pecado. \footnote{\textbf{15:27}
  Lev 4,27-28} \bibleverse{28} El sacerdote hará expiación por el alma
que se equivoca cuando peca involuntariamente ante Yahvé. Hará expiación
por él, y será perdonado. \bibleverse{29} Tendrás una sola ley para el
que hace algo involuntariamente, para el nativo entre los hijos de
Israel y para el extranjero que vive como forastero entre ellos.

\bibleverse{30} ``\,`Pero el alma que hace algo con mano alzada, sea
nativo o extranjero, blasfema a Yahvé. Esa alma será cortada de entre su
pueblo. \footnote{\textbf{15:30} Hech 13,38; Heb 10,26-27}
\bibleverse{31} Porque ha despreciado la palabra de Yahvé y ha violado
su mandamiento, esa alma será cortada por completo. Su iniquidad recaerá
sobre él''.

\hypertarget{informe-de-la-lapidaciuxf3n-de-un-abusador-del-suxe1bado}{%
\subsection{Informe de la lapidación de un abusador del
sábado}\label{informe-de-la-lapidaciuxf3n-de-un-abusador-del-suxe1bado}}

\bibleverse{32} Mientras los hijos de Israel estaban en el desierto,
encontraron a un hombre recogiendo palos en el día de reposo.
\bibleverse{33} Los que lo encontraron recogiendo palos lo llevaron a
Moisés y a Aarón, y a toda la congregación. \bibleverse{34} Lo pusieron
en custodia, porque no se había declarado lo que debía hacerse con él.
\footnote{\textbf{15:34} Lev 24,12; Éxod 31,14; Éxod 35,2}

\bibleverse{35} El Señor le dijo a Moisés: ``Ese hombre deberá morir.
Toda la congregación lo apedreará fuera del campamento''.
\bibleverse{36} Toda la congregación lo sacó fuera del campamento y lo
apedreó hasta la muerte, como Yahvé le ordenó a Moisés.

\hypertarget{ordenanza-sobre-las-borlas-para-adherir-a-las-puntas-de-la-ropa}{%
\subsection{Ordenanza sobre las borlas para adherir a las puntas de la
ropa}\label{ordenanza-sobre-las-borlas-para-adherir-a-las-puntas-de-la-ropa}}

\bibleverse{37} Yahvé habló a Moisés, diciendo: \bibleverse{38} ``Habla
a los hijos de Israel y diles que se hagan flecos\footnote{\textbf{15:38}
  o, borlas (hebreo \hebrew{צִיצִ֛ת})} en los bordes de sus vestidos por
sus generaciones, y que pongan en el fleco\footnote{\textbf{15:38} o,
  borla} de cada borde un cordón de color azul. \footnote{\textbf{15:38}
  Deut 22,12; Mat 23,5} \bibleverse{39} Será para ti como un
fleco,\footnote{\textbf{15:39} o, borla} para que lo veas y te acuerdes
de todos los mandamientos de Yahvé, y los pongas en práctica; y para que
no sigas tu propio corazón y tus propios ojos, según los cuales solías
jugar a la prostitución; \bibleverse{40} para que te acuerdes y pongas
en práctica todos mis mandamientos, y seas santo para tu Dios.
\bibleverse{41} Yo soy el Señor, tu Dios, que te sacó de la tierra de
Egipto para ser tu Dios: Yo soy Yahvé tu Dios''.

\hypertarget{el-ultraje-de-coruxe9-y-los-rubenitas}{%
\subsection{El ultraje de Coré y los
rubenitas}\label{el-ultraje-de-coruxe9-y-los-rubenitas}}

\hypertarget{section-15}{%
\section{16}\label{section-15}}

\bibleverse{1} Coré, hijo de Izhar, hijo de Coat, hijo de Leví, con
Datán y Abiram, hijos de Eliab, y On, hijo de Pelet, hijos de Rubén,
tomaron algunos hombres. \footnote{\textbf{16:1} Éxod 6,18; Éxod 6,21;
  Núm 26,9; Jds 1,11} \bibleverse{2} Se levantaron ante Moisés, con
algunos de los hijos de Israel, doscientos cincuenta príncipes de la
congregación, llamados a la asamblea, hombres de renombre. \footnote{\textbf{16:2}
  Núm 12,1-2} \bibleverse{3} Se juntaron contra Moisés y contra Aarón, y
les dijeron: ``¡Os arrogáis demasiado, ya que toda la congregación es
santa, todos ellos, y Yahvé está en medio de ellos! ¿Por qué os alzáis
por encima de la asamblea de Yahvé?''

\hypertarget{moisuxe9s-confronta-al-grupo-de-coruxe9-y-anuncia-un-juicio-divino-en-el-santuario}{%
\subsection{Moisés confronta al grupo de Coré y anuncia un juicio divino
en el
santuario}\label{moisuxe9s-confronta-al-grupo-de-coruxe9-y-anuncia-un-juicio-divino-en-el-santuario}}

\bibleverse{4} Cuando Moisés lo oyó, se postró sobre su rostro.
\footnote{\textbf{16:4} Núm 14,5} \bibleverse{5} Dijo a Coré y a toda su
compañía: ``Por la mañana, el Señor mostrará quiénes son suyos y quiénes
son santos, y los hará acercarse a él. También hará que se acerque a él
el que él elija. \footnote{\textbf{16:5} 2Tim 2,19} \bibleverse{6} Hagan
esto: que Coré y toda su compañía tomen incensarios, \bibleverse{7}
pongan fuego en ellos, y pongan incienso en ellos ante Yahvé mañana. El
hombre que elija Yavé será santo. Habéis ido demasiado lejos, hijos de
Leví''.

\bibleverse{8} Moisés dijo a Coré: ``¡Escuchen ahora, hijos de Leví!
\bibleverse{9} ¿Os parece poco que el Dios de Israel os haya separado de
la congregación de Israel para acercaros a él, para que hagáis el
servicio del tabernáculo de Yahvé, y para que estéis delante de la
congregación para servirles; \footnote{\textbf{16:9} Núm 3,6-13; Núm
  4,4-20} \bibleverse{10} y que os haya acercado a vosotros, y a todos
vuestros hermanos los hijos de Leví con vosotros? ¿Acaso buscáis también
el sacerdocio? \bibleverse{11} ¡Por eso tú y toda tu compañía os habéis
reunido contra Yahvé! ¿Qué es Aarón para que os quejéis contra él?''
\footnote{\textbf{16:11} Éxod 16,7}

\hypertarget{datuxe1n-y-abiram-se-burlan-de-la-invitaciuxf3n-de-moisuxe9s-moisuxe9s-oraciuxf3n-a-dios}{%
\subsection{Datán y Abiram se burlan de la invitación de Moisés; Moisés
oración a
Dios}\label{datuxe1n-y-abiram-se-burlan-de-la-invitaciuxf3n-de-moisuxe9s-moisuxe9s-oraciuxf3n-a-dios}}

\bibleverse{12} Moisés mandó llamar a Datán y Abiram, hijos de Eliab, y
ellos dijeron: ``¡No subiremos! \bibleverse{13} ¿Es poca cosa que nos
hayas hecho subir de una tierra que mana leche y miel, para matarnos en
el desierto, y que además te hagas príncipe sobre nosotros?
\bibleverse{14} Además, no nos has traído a una tierra que fluye leche y
miel, ni nos has dado herencia de campos y viñedos. ¿Vas a sacarle los
ojos a estos hombres? No subiremos''. \footnote{\textbf{16:14} Éxod 3,8;
  Éxod 3,17}

\bibleverse{15} Moisés se enojó mucho y le dijo a Yahvé: ``No respetes
su ofrenda. No les he quitado ni un solo asno, ni he hecho daño a
ninguno de ellos''. \footnote{\textbf{16:15} 1Sam 12,3; Hech 20,33}

\hypertarget{moisuxe9s-convoca-a-coruxe9-y-sus-compauxf1eros-para-realizar-el-sacrificio-la-apariciuxf3n-de-la-gloria-de-dios-intercesiuxf3n-de-moisuxe9s}{%
\subsection{Moisés convoca a Coré y sus compañeros para realizar el
sacrificio; La aparición de la gloria de Dios; Intercesión de
Moisés}\label{moisuxe9s-convoca-a-coruxe9-y-sus-compauxf1eros-para-realizar-el-sacrificio-la-apariciuxf3n-de-la-gloria-de-dios-intercesiuxf3n-de-moisuxe9s}}

\bibleverse{16} Moisés dijo a Coré: ``Tú y toda tu compañía vayan mañana
ante Yahvé, tú y ellos, y Aarón. \bibleverse{17} Cada uno tomará su
incensario y pondrá incienso en él, y cada uno llevará ante Yavé su
incensario, doscientos cincuenta incensarios; tú también y Aarón, cada
uno con su incensario.''

\bibleverse{18} Cada uno de ellos tomó su incensario, puso fuego en él y
puso incienso, y se puso a la puerta de la Tienda de Reunión con Moisés
y Aarón. \bibleverse{19} Coré reunió a toda la congregación frente a
ellos a la puerta de la Tienda del Encuentro. La gloria de Yahvé
apareció a toda la congregación. \footnote{\textbf{16:19} Núm 14,10}
\bibleverse{20} Yahvé habló a Moisés y a Aarón, diciendo:
\bibleverse{21} ``¡Sepárense de entre esta congregación, para que yo los
consuma en un momento!''

\bibleverse{22} Se postraron sobre sus rostros y dijeron: ``Dios, el
Dios de los espíritus de toda carne, ¿pecará un solo hombre y te
enojarás con toda la congregación?'' \footnote{\textbf{16:22} Job 12,10;
  2Sam 24,17}

\bibleverse{23} Yahvé habló a Moisés diciendo: \bibleverse{24} ``Habla a
la congregación diciendo: ``¡Aléjate de los alrededores de la tienda de
Coré, Datán y Abiram!''\,''.

\hypertarget{moisuxe9s-con-datuxe1n-y-abiram-el-juicio-divino-sobre-ellos-y-sobre-los-250-compauxf1eros-de-coruxe9}{%
\subsection{Moisés con Datán y Abiram; el juicio divino sobre ellos y
sobre los 250 compañeros de
Coré}\label{moisuxe9s-con-datuxe1n-y-abiram-el-juicio-divino-sobre-ellos-y-sobre-los-250-compauxf1eros-de-coruxe9}}

\bibleverse{25} Moisés se levantó y se dirigió a Datán y Abiram, y los
ancianos de Israel lo siguieron. \bibleverse{26} Habló a la congregación
diciendo: ``¡Apártense, por favor, de las tiendas de estos malvados, y
no toquen nada de ellos, no sea que se consuman en todos sus pecados!''

\bibleverse{27} Y se alejaron de la tienda de Coré, Datán y Abiram, por
todos lados. Datán y Abiram salieron, y se pusieron a la puerta de sus
tiendas con sus mujeres, sus hijos y sus pequeños.

\bibleverse{28} Moisés dijo: ``Así sabrán que Yahvé me ha enviado a
hacer todas estas obras, porque no son de mi propia cosecha.
\bibleverse{29} Si estos hombres mueren la muerte común de todos los
hombres, o si experimentan lo que todos los hombres experimentan,
entonces Yahvé no me ha enviado. \bibleverse{30} Pero si Yahvé hace una
cosa nueva, y la tierra abre su boca y los traga con todo lo que les
pertenece, y bajan vivos al Seol,\footnote{\textbf{16:30} El Seol es el
  lugar de los muertos.} entonces entenderás que estos hombres han
despreciado a Yahvé.''

\bibleverse{31} Cuando terminó de decir todas estas palabras, la tierra
que estaba debajo de ellos se partió. \footnote{\textbf{16:31} Deut 11,6}
\bibleverse{32} La tierra abrió su boca y se los tragó con sus familias,
todos los hombres de Coré y todos sus bienes. \bibleverse{33} Ellos y
todo lo que les pertenecía descendieron vivos al Seol.\footnote{\textbf{16:33}
  El Seol es el lugar de los muertos.} La tierra se cerró sobre ellos, y
perecieron de entre la asamblea. \bibleverse{34} Todo Israel que estaba
alrededor de ellos huyó ante su grito, pues dijeron: ``¡No sea que la
tierra nos trague!'' \bibleverse{35} Salió fuego del Señor y devoró a
los doscientos cincuenta hombres que ofrecían el incienso. \footnote{\textbf{16:35}
  Lev 10,1-2; Sal 106,18}

\hypertarget{el-uso-de-las-250-ollas-humeantes-por-parte-de-coruxe9-y-sus-compauxf1eros-como-cubierta-para-el-altar-de-sacrificios}{%
\subsection{El uso de las 250 ollas humeantes por parte de Coré y sus
compañeros como cubierta para el altar de
sacrificios}\label{el-uso-de-las-250-ollas-humeantes-por-parte-de-coruxe9-y-sus-compauxf1eros-como-cubierta-para-el-altar-de-sacrificios}}

\bibleverse{36} Yahvé habló a Moisés, diciendo: \bibleverse{37} ``Habla
a Eleazar, hijo de Aarón, el sacerdote, para que saque los incensarios
de la quema, y esparza el fuego lejos del campamento; porque son
sagrados, \bibleverse{38} incluso los incensarios de los que pecaron
contra su propia vida. Que los golpeen en placas para cubrir el altar,
pues los ofrecieron ante Yahvé. Por eso son santos. Serán una señal para
los hijos de Israel''.

\bibleverse{39} El sacerdote Eleazar tomó los incensarios de bronce que
habían ofrecido los quemados, y los batieron para cubrir el altar,
\bibleverse{40} para que sirvieran de recuerdo a los hijos de Israel, a
fin de que ningún extranjero que no sea de la descendencia de Aarón se
acerque a quemar incienso delante de Yavé, para que no sea como Coré y
su compañía, como le habló Yavé por medio de Moisés. \footnote{\textbf{16:40}
  Núm 1,51}

\hypertarget{castigo-a-la-comunidad-quejuxe1ndose-por-la-desapariciuxf3n-de-los-alborotadores-la-expiaciuxf3n-hecha-por-moisuxe9s-y-aaruxf3n}{%
\subsection{Castigo a la comunidad quejándose por la desaparición de los
alborotadores; la expiación hecha por Moisés y
Aarón}\label{castigo-a-la-comunidad-quejuxe1ndose-por-la-desapariciuxf3n-de-los-alborotadores-la-expiaciuxf3n-hecha-por-moisuxe9s-y-aaruxf3n}}

\bibleverse{41} Pero al día siguiente toda la congregación de los hijos
de Israel se quejó contra Moisés y contra Aarón, diciendo: ``¡Habéis
matado al pueblo de Yahvé!''

\bibleverse{42} Cuando la congregación se reunió contra Moisés y contra
Aarón, miraron hacia la Tienda del Encuentro. He aquí que la nube la
cubría, y la gloria de Yahvé aparecía. \footnote{\textbf{16:42} Núm
  14,10} \bibleverse{43} Moisés y Aarón llegaron al frente de la Tienda
de Reunión. \bibleverse{44} Yahvé habló a Moisés, diciendo:
\bibleverse{45} ``¡Aléjate de entre esta congregación, para que los
consuma en un momento!'' Ellos cayeron de bruces. \footnote{\textbf{16:45}
  Núm 16,4; Núm 16,22}

\bibleverse{46} Moisés dijo a Aarón: ``¡Toma tu incensario, pon en él
fuego del altar, ponle incienso, llévalo rápidamente a la congregación y
haz expiación por ellos, porque la ira ha salido de Yahvé! La plaga ha
comenzado''. \footnote{\textbf{16:46} Éxod 28,38; Lev 16,13}

\bibleverse{47} Aarón hizo lo que dijo Moisés y corrió al centro de la
asamblea. La plaga ya había comenzado entre el pueblo. Se puso el
incienso e hizo expiación por el pueblo. \bibleverse{48} Se puso entre
los muertos y los vivos, y la plaga se detuvo. \bibleverse{49} Los que
murieron por la plaga fueron catorce mil setecientos, además de los que
murieron por el asunto de Coré. \bibleverse{50} Aarón volvió con Moisés
a la puerta de la Tienda del Encuentro, y la plaga se detuvo.

\hypertarget{prueba-del-derecho-sacerdotal-de-aaruxf3n-por-los-maravillosos-peldauxf1os-de-su-cayado}{%
\subsection{Prueba del derecho sacerdotal de Aarón por los maravillosos
peldaños de su
cayado}\label{prueba-del-derecho-sacerdotal-de-aaruxf3n-por-los-maravillosos-peldauxf1os-de-su-cayado}}

\hypertarget{section-16}{%
\section{17}\label{section-16}}

\bibleverse{1} Yahvé habló a Moisés diciendo: \bibleverse{2} ``Habla a
los hijos de Israel y toma varas de ellos, una por cada casa paterna, de
todos sus príncipes según sus casas paternas, doce varas. Escribe el
nombre de cada uno en su vara. \bibleverse{3} Escribirás el nombre de
Aarón en la vara de Leví. Habrá una vara por cada jefe de sus casas
paternas. \bibleverse{4} Las pondrás en la Tienda de Reunión, delante
del pacto, donde me reúno con ustedes. \footnote{\textbf{17:4} Éxod
  25,22} \bibleverse{5} Sucederá que la vara del hombre que yo elija
brotará. Haré que cesen las murmuraciones de los hijos de Israel, que
murmuran contra ti, a partir de mí.''

\bibleverse{6} Moisés habló a los hijos de Israel, y todos sus príncipes
le dieron varas, para cada príncipe una, según las casas de sus padres,
un total de doce varas. La vara de Aarón estaba entre sus varas.
\bibleverse{7} Moisés depositó las varas ante el Señor en la Tienda del
Testimonio.

\bibleverse{8} Al día siguiente, Moisés entró en la Tienda del
Testimonio; y he aquí que la vara de Aarón para la casa de Leví había
brotado, había echado brotes, había producido flores y había dado
almendras maduras. \bibleverse{9} Moisés sacó todas las varas de delante
de Yavé para todos los hijos de Israel. Ellos miraron, y cada uno tomó
su vara.

\bibleverse{10} Yahvé dijo a Moisés: ``Vuelve a poner la vara de Aarón
delante del pacto, para que la guardes como señal contra los hijos de la
rebelión; así pondrás fin a sus quejas contra mí, para que no mueran.''
\footnote{\textbf{17:10} Heb 9,4} \bibleverse{11} Así lo hizo Moisés.
Como Yahvé le ordenó, así lo hizo.

\bibleverse{12} Los hijos de Israel hablaron con Moisés, diciendo: ``¡He
aquí que perecemos! ¡Estamos perdidos! ¡Todos estamos deshechos!
\bibleverse{13} ¡Todos los que se acercan al tabernáculo de Yahvé
mueren! ¿Pereceremos todos?'' \footnote{\textbf{17:13} Núm 16,40}

\hypertarget{ordenanzas-generales-sobre-los-deberes-de-los-sacerdotes-y-sus-ayudantes-los-levitas}{%
\subsection{Ordenanzas generales sobre los deberes de los sacerdotes y
sus ayudantes, los
levitas}\label{ordenanzas-generales-sobre-los-deberes-de-los-sacerdotes-y-sus-ayudantes-los-levitas}}

\hypertarget{section-17}{%
\section{18}\label{section-17}}

\bibleverse{1} Yahvé dijo a Aarón: ``Tú y tus hijos, y la casa de tu
padre contigo, llevarán la iniquidad del santuario; y tú y tus hijos
contigo llevarán la iniquidad de tu sacerdocio. \footnote{\textbf{18:1}
  Éxod 28,38; Lev 16,32-33} \bibleverse{2} Trae también contigo a tus
hermanos de la tribu de Leví, la tribu de tu padre, para que se unan a
ti y te sirvan; pero tú y tus hijos contigo estarán delante de la Tienda
del Testimonio. \footnote{\textbf{18:2} Núm 3,6-10} \bibleverse{3} Ellos
guardarán tus mandatos y el deber de toda la Tienda; sólo que no se
acercarán a los utensilios del santuario ni al altar, para que no
mueran, ni ellos ni tú. \bibleverse{4} Se unirán a ti y guardarán la
responsabilidad de la Tienda de reunión, para todo el servicio de la
Tienda. Un extranjero no se acercará a vosotros.

\bibleverse{5} ``Cumplirás el deber del santuario y el deber del altar,
para que no haya más ira sobre los hijos de Israel. \footnote{\textbf{18:5}
  Núm 16,46} \bibleverse{6} He aquí que yo mismo he tomado a tus
hermanos los levitas de entre los hijos de Israel. Ellos son un regalo
para ti, dedicados a Yahvé, para hacer el servicio de la Tienda del
Encuentro. \footnote{\textbf{18:6} Núm 3,12; Núm 3,45} \bibleverse{7} Tú
y tus hijos contigo mantendrán su sacerdocio para todo lo que se refiere
al altar, y para lo que está dentro del velo. Servirás. Te doy el
servicio del sacerdocio como un regalo. El extranjero que se acerque
será condenado a muerte''. \footnote{\textbf{18:7} Núm 1,51}

\hypertarget{los-ingresos-de-los-sacerdotes}{%
\subsection{Los ingresos de los
sacerdotes}\label{los-ingresos-de-los-sacerdotes}}

\bibleverse{8} Yahvé habló a Aarón: ``He aquí que yo mismo te he dado el
mando de mis ofrendas mecidas, todas las cosas sagradas de los hijos de
Israel. Te las he dado a ti por razón de la unción, y a tus hijos, como
porción para siempre. \footnote{\textbf{18:8} Lev 2,3; Lev 2,10; Lev
  6,16-18; Lev 6,26-29; Lev 7,6-10} \bibleverse{9} Esto será para ti de
las cosas más santas del fuego: toda ofrenda de ellos, toda ofrenda por
el pecado de ellos y toda ofrenda por la culpa de ellos, que me
presenten, será santísima para ti y para tus hijos. \bibleverse{10}
Comeréis de ella como de las cosas más santas. Todo varón comerá de
ella. Será sagrado para ti.

\bibleverse{11} ``Esto también es tuyo: la ofrenda mecida de su regalo,
todas las ofrendas mecidas de los hijos de Israel. Te las he dado a ti,
a tus hijos y a tus hijas contigo, como porción para siempre. Todo el
que esté limpio en tu casa comerá de ella. \footnote{\textbf{18:11} Lev
  10,14}

\bibleverse{12} ``Te he dado todo lo mejor del aceite, todo lo mejor de
la cosecha y del grano, las primicias de ellos que den a Yahvé.
\bibleverse{13} Los primeros frutos de todo lo que hay en su tierra, que
traen a Yahvé, serán tuyos. Todo el que esté limpio en tu casa comerá de
ello. \footnote{\textbf{18:13} Éxod 23,19; Deut 18,4}

\bibleverse{14} ``Todo lo consagrado en Israel será tuyo. \footnote{\textbf{18:14}
  Lev 27,28} \bibleverse{15} Todo lo que abra el vientre, de toda la
carne que ofrezcan a Yahvé, tanto de hombres como de animales, será
tuyo. Sin embargo, redimirás al primogénito del hombre, y redimirás al
primogénito de los animales inmundos. \footnote{\textbf{18:15} Éxod
  13,12-13; Éxod 34,19-20} \bibleverse{16} Redimirás a los que deban ser
redimidos de un mes de edad, según tu estimación, por cinco siclos de
dinero, según el siclo\footnote{\textbf{18:16} Un siclo equivale a unos
  10 gramos o a unas 0,35 onzas.} del santuario, que pesa veinte gerahs.
\footnote{\textbf{18:16} Una gerah equivale a unos 0,5 gramos o a unos
  7,7 granos.}

\bibleverse{17} ``Pero no redimirás el primogénito de una vaca, ni el
primogénito de una oveja, ni el primogénito de una cabra. Son sagrados.
Rociarás su sangre sobre el altar, y quemarás su grasa como ofrenda
encendida, como aroma agradable para Yahvé. \bibleverse{18} Su carne
será tuya, como el pecho de la ofrenda mecida y como el muslo derecho,
será tuyo. \bibleverse{19} Todas las ofrendas mecidas de las cosas
santas que los hijos de Israel ofrecen a Yahvé, te las he dado a ti y a
tus hijos e hijas contigo, como una porción para siempre. Es un pacto de
sal para siempre ante Yahvé para ti y para tu descendencia contigo''.

\hypertarget{asignaciuxf3n-del-diezmo-a-los-levitas-por-la-negaciuxf3n-de-la-tierra}{%
\subsection{Asignación del diezmo a los levitas por la negación de la
tierra}\label{asignaciuxf3n-del-diezmo-a-los-levitas-por-la-negaciuxf3n-de-la-tierra}}

\bibleverse{20} Yahvé dijo a Aarón: ``No tendrás herencia en su tierra,
ni tendrás parte entre ellos. Yo soy tu porción y tu herencia entre los
hijos de Israel. \footnote{\textbf{18:20} Núm 35,5; Deut 10,9; Deut
  12,12; Jos 13,14; Jos 13,33}

\bibleverse{21} ``A los hijos de Leví, he aquí que yo les he dado en
herencia todo el diezmo en Israel, a cambio del servicio que prestan, el
servicio de la Tienda del Encuentro. \footnote{\textbf{18:21} Lev 27,30}
\bibleverse{22} De ahora en adelante los hijos de Israel no se acercarán
a la Tienda del Encuentro, para que no lleven el pecado y mueran.
\bibleverse{23} Pero los levitas harán el servicio de la Carpa del
Encuentro, y cargarán con su iniquidad. Será un estatuto para siempre a
lo largo de vuestras generaciones. Entre los hijos de Israel no tendrán
herencia. \bibleverse{24} Porque el diezmo de los hijos de Israel, que
ofrecen como ofrenda mecida a Yavé, lo he dado a los levitas como
herencia. Por eso les he dicho: `Entre los hijos de Israel no tendrán
herencia'\,''.

\hypertarget{el-diezmo-de-los-ingresos-de-los-levitas-a-los-sacerdotes}{%
\subsection{El diezmo de los ingresos de los levitas a los
sacerdotes}\label{el-diezmo-de-los-ingresos-de-los-levitas-a-los-sacerdotes}}

\bibleverse{25} Yahvé habló a Moisés diciendo: \bibleverse{26} ``Además,
hablarás a los levitas y les dirás: ``Cuando toméis de los hijos de
Israel el diezmo que os he dado de ellos como herencia, ofreceréis de él
una ofrenda mecida para Yahvé, un diezmo del diezmo. \bibleverse{27} Tu
ofrenda mecida se te acreditará como el grano de la era y como la
plenitud del lagar. \bibleverse{28} Así también ofrecerás una ofrenda
mecida a Yahvé de todos tus diezmos que recibas de los hijos de Israel;
y de ella darás la ofrenda mecida de Yahvé al sacerdote Aarón.
\bibleverse{29} De todas vuestras ofrendas, ofreceréis a Yahvé cada
ofrenda mecida, de todas sus mejores partes, la parte sagrada.'

\bibleverse{30} ``Por lo tanto, les dirás: `Cuando saques lo mejor de
ella, se acreditará a los levitas como el producto de la era y como el
producto del lagar. \bibleverse{31} Podéis comerlo en cualquier parte,
vosotros y vuestras familias, porque es vuestra recompensa en recompensa
por vuestro servicio en la Tienda de Reunión. \footnote{\textbf{18:31}
  Mat 10,10} \bibleverse{32} No cargarás con ningún pecado por causa de
ella, cuando hayas sacado de ella lo mejor. No profanarás las cosas
santas de los hijos de Israel, para que no mueras''.

\hypertarget{preparaciuxf3n-y-uso-del-agua-de-limpieza}{%
\subsection{Preparación y uso del agua de
limpieza}\label{preparaciuxf3n-y-uso-del-agua-de-limpieza}}

\hypertarget{section-18}{%
\section{19}\label{section-18}}

\bibleverse{1} Yahvé habló a Moisés y a Aarón, diciendo: \bibleverse{2}
``Este es el estatuto de la ley que Yahvé ha ordenado. Di a los hijos de
Israel que te traigan una novilla roja sin mancha, en la que no haya
ningún defecto, y que nunca haya sido unida. \footnote{\textbf{19:2} Heb
  9,13; Lev 22,20} \bibleverse{3} Se la daréis al sacerdote Eleazar, y
él la llevará fuera del campamento, y la matará delante de él.
\bibleverse{4} El sacerdote Eleazar tomará un poco de su sangre con su
dedo, y rociará su sangre hacia el frente de la Tienda de Reunión siete
veces. \footnote{\textbf{19:4} Lev 4,6; Lev 4,17} \bibleverse{5} El
sacerdote quemará la novilla delante de él; quemará su piel, su carne y
su sangre, con su estiércol. \bibleverse{6} El sacerdote tomará madera
de cedro, hisopo y escarlata, y lo echará en medio del incendio de la
novilla. \footnote{\textbf{19:6} Lev 14,6} \bibleverse{7} Luego el
sacerdote lavará sus vestidos y bañará su carne con agua, y después
entrará en el campamento, y el sacerdote quedará impuro hasta la noche.
\footnote{\textbf{19:7} Lev 16,28} \bibleverse{8} El que la queme lavará
sus ropas con agua y bañará su carne con agua, y quedará impuro hasta la
noche.

\bibleverse{9} ``El hombre limpio recogerá las cenizas de la vaquilla y
las depositará fuera del campamento, en un lugar limpio, y las guardará
para la congregación de los hijos de Israel para usarlas en el agua para
limpiar la impureza. Es una ofrenda por el pecado. \bibleverse{10} El
que recoja las cenizas de la vaquilla se lavará la ropa y quedará impuro
hasta la noche. Será para los hijos de Israel, y para el extranjero que
viva como forastero entre ellos, como un estatuto para siempre.

\bibleverse{11} ``El que toque el cadáver de un hombre será impuro
durante siete días. \bibleverse{12} Al tercer día se purificará con
agua, y al séptimo día quedará limpio; pero si no se purifica al tercer
día, al séptimo no quedará limpio. \bibleverse{13} El que toque a un
muerto, el cuerpo de un hombre que ha muerto, y no se purifique,
contamina el tabernáculo de Yahvé; y esa alma será cortada de Israel;
porque el agua para la impureza no fue rociada sobre él, será impuro. Su
impureza aún está sobre él. \footnote{\textbf{19:13} Lev 15,31}

\hypertarget{instrucciones-sobre-casos-especuxedficos-de-contaminaciuxf3n-y-su-tratamiento}{%
\subsection{Instrucciones sobre casos específicos de contaminación y su
tratamiento}\label{instrucciones-sobre-casos-especuxedficos-de-contaminaciuxf3n-y-su-tratamiento}}

\bibleverse{14} ``Esta es la ley cuando un hombre muere en una tienda:
todo el que entre en la tienda, y todo el que esté en ella, será impuro
durante siete días. \bibleverse{15} Toda vasija abierta, que no tenga
cubierta atada, es inmunda.

\bibleverse{16} ``Cualquiera que en el campo abierto toque a un muerto
con espada, o un cadáver, o un hueso de hombre, o una tumba, quedará
impuro siete días.

\bibleverse{17} ``En cuanto a los impuros, tomarán de la ceniza de la
quema de la ofrenda por el pecado, y se derramará sobre ellos agua
corriente en una vasija. \bibleverse{18} La persona limpia tomará
hisopo, lo mojará en el agua y lo rociará sobre la tienda, sobre todos
los utensilios, sobre las personas que estaban allí y sobre el que haya
tocado el hueso, o el muerto, o la tumba. \bibleverse{19} El limpio
rociará sobre el impuro al tercer día y al séptimo. Al séptimo día lo
purificará. Lavará su ropa y se bañará con agua, y quedará limpio al
anochecer. \bibleverse{20} Pero el hombre que sea impuro y no se
purifique, esa persona será cortada de entre la asamblea, porque ha
profanado el santuario de Yahvé. El agua para la impureza no ha sido
rociada sobre él. Es impuro. \bibleverse{21} Será un estatuto perpetuo
para ellos. El que rocíe el agua de la impureza lavará sus ropas, y el
que toque el agua de la impureza quedará impuro hasta la noche.

\bibleverse{22} ``Todo lo que toque el impuro será impuro, y el alma que
lo toque será impura hasta la noche''.

\hypertarget{llegada-a-kade-y-muerte-de-miriam-reonovada-queja-del-pueblo-la-fatuxeddicia-doncaciuxf3n-de-agua-de-la-roca-para-moisuxe9s-y-aaruxf3n}{%
\subsection{Llegada a Kade y muerte de Miriam; reonovada queja del
pueblo; la fatídicia doncación de agua de la roca para Moisés y
Aarón}\label{llegada-a-kade-y-muerte-de-miriam-reonovada-queja-del-pueblo-la-fatuxeddicia-doncaciuxf3n-de-agua-de-la-roca-para-moisuxe9s-y-aaruxf3n}}

\hypertarget{section-19}{%
\section{20}\label{section-19}}

\bibleverse{1} Los hijos de Israel, toda la congregación, llegaron al
desierto de Zin en el primer mes. El pueblo se quedó en Cades. Allí
murió Miriam, y allí fue enterrada. \footnote{\textbf{20:1} Núm 13,21}
\bibleverse{2} No había agua para la congregación, y se juntaron contra
Moisés y contra Aarón. \footnote{\textbf{20:2} Éxod 17,1-7}
\bibleverse{3} El pueblo discutió con Moisés y habló diciendo: ``¡Ojalá
hubiéramos muerto cuando nuestros hermanos murieron ante Yavé!
\bibleverse{4} ¿Por qué has traído la asamblea de Yavé a este desierto,
para que muramos allí, nosotros y nuestros animales? \bibleverse{5} ¿Por
qué nos has hecho subir de Egipto para traernos a este mal lugar? No es
lugar de semillas, ni de higos, ni de vides, ni de granadas; tampoco hay
agua para beber.''

\bibleverse{6} Moisés y Aarón salieron de la presencia de la asamblea a
la puerta de la Tienda de Reunión y se postraron sobre sus rostros. La
gloria de Yahvé se les apareció. \footnote{\textbf{20:6} Núm 14,10}
\bibleverse{7} Yahvé habló a Moisés, diciendo: \bibleverse{8} ``Toma la
vara y reúne a la congregación, tú y Aarón, tu hermano, y habla a la
roca ante sus ojos, para que derrame su agua. Les traerás agua de la
roca; así darás de beber a la congregación y a sus ganados''.

\bibleverse{9} Moisés tomó la vara de delante de Yahvé, como él le había
ordenado. \bibleverse{10} Moisés y Aarón reunieron a la asamblea ante la
roca, y les dijo: ``¡Escuchen ahora, rebeldes! ¿Hemos de sacar agua de
esta roca para ustedes?'' \footnote{\textbf{20:10} Sal 106,33}
\bibleverse{11} Moisés levantó su mano y golpeó la roca con su vara dos
veces, y el agua salió en abundancia. La congregación y su ganado
bebieron.

\bibleverse{12} Yahvé dijo a Moisés y a Aarón: ``Como no habéis creído
en mí para santificarme a los ojos de los hijos de Israel, no
introduciréis esta asamblea en la tierra que les he dado.'' \footnote{\textbf{20:12}
  Núm 27,14; Deut 1,37; Deut 3,26; Deut 4,21; Deut 32,51}

\bibleverse{13} Estas son las aguas de Meribá;\footnote{\textbf{20:13}
  ``Meribah'' significa ``pelea''.} porque los hijos de Israel lucharon
con Yahvé, y él se santificó en ellas. \footnote{\textbf{20:13} Sal 81,7}

\hypertarget{los-edomitas-se-niegan-a-permitir-el-paso-la-muerte-de-aaron}{%
\subsection{Los edomitas se niegan a permitir el paso; La muerte de
Aaron}\label{los-edomitas-se-niegan-a-permitir-el-paso-la-muerte-de-aaron}}

\bibleverse{14} Moisés envió mensajeros desde Cades al rey de Edom,
diciendo: ``Tu hermano Israel dice: Tú sabes toda la aflicción que nos
ha sucedido; \footnote{\textbf{20:14} Gén 32,3; Jue 11,17; Deut 23,7}
\bibleverse{15} cómo nuestros padres bajaron a Egipto, y vivimos en
Egipto mucho tiempo. Los egipcios nos maltrataron a nosotros y a
nuestros padres. \bibleverse{16} Cuando clamamos al Señor, él escuchó
nuestra voz, envió un ángel y nos sacó de Egipto. He aquí que estamos en
Cades, una ciudad en el límite de su frontera. \footnote{\textbf{20:16}
  Éxod 23,20}

\bibleverse{17} ``Por favor, déjanos pasar por tu tierra. No pasaremos
por el campo ni por la viña, ni beberemos del agua de los pozos. Iremos
por el camino del rey. No nos desviaremos a la derecha ni a la
izquierda, hasta que hayamos pasado tu frontera''. \footnote{\textbf{20:17}
  Núm 21,22}

\bibleverse{18} Edom le dijo: ``No pasarás por mí, no sea que salga con
la espada contra ti''.

\bibleverse{19} Los hijos de Israel le dijeron: ``Subiremos por el
camino, y si bebemos tu agua, yo y mi ganado, te daré su precio. Sólo
déjame, sin hacer nada más, pasar de pie''.

\bibleverse{20} Dijo: ``No pasarás''. Edom salió contra él con mucha
gente y con mano dura. \bibleverse{21} Así, Edom se negó a dar paso a
Israel por su frontera, por lo que Israel se alejó de él.

\hypertarget{el-tren-de-kades-al-monte-hor-la-muerte-de-aaron}{%
\subsection{El tren de Kades al monte Hor; La muerte de
Aaron}\label{el-tren-de-kades-al-monte-hor-la-muerte-de-aaron}}

\bibleverse{22} Partieron de Cades, y los hijos de Israel, toda la
congregación, llegaron al monte Hor. \bibleverse{23} Yahvé habló a
Moisés y a Aarón en el monte Hor, junto a la frontera de la tierra de
Edom, diciendo: \bibleverse{24} ``Aarón será reunido con su pueblo, pues
no entrará en la tierra que he dado a los hijos de Israel, porque os
habéis rebelado contra mi palabra en las aguas de Meribá.
\bibleverse{25} Toma a Aarón y a su hijo Eleazar, y llévalos al monte
Hor; \bibleverse{26} y despoja a Aarón de sus vestiduras, y pónselas a
su hijo Eleazar. Aarón será recogido y morirá allí''. \footnote{\textbf{20:26}
  Lev 21,10}

\bibleverse{27} Moisés hizo lo que Yahvé le ordenó. Subieron al monte
Hor a la vista de toda la congregación. \bibleverse{28} Moisés despojó a
Aarón de sus vestiduras y se las puso a su hijo Eleazar. Aarón murió
allí, en la cima del monte, y Moisés y Eleazar bajaron del monte.
\footnote{\textbf{20:28} Núm 33,38; Deut 10,6} \bibleverse{29} Cuando
toda la congregación vio que Aarón había muerto, lloraron a Aarón
durante treinta días, toda la casa de Israel.

\hypertarget{batalla-victoriosa-con-el-rey-de-arad}{%
\subsection{Batalla victoriosa con el Rey de
Arad}\label{batalla-victoriosa-con-el-rey-de-arad}}

\hypertarget{section-20}{%
\section{21}\label{section-20}}

\bibleverse{1} El cananeo, rey de Arad, que vivía en el sur, se enteró
de que Israel venía por el camino de Atarim. Luchó contra Israel y tomó
cautivos a algunos de ellos. \bibleverse{2} Israel hizo un voto a Yavé y
le dijo: ``Si realmente entregas a este pueblo en mi mano, destruiré por
completo sus ciudades.'' \footnote{\textbf{21:2} Deut 13,15; Jos 6,17;
  Jue 1,17; 1Sam 15,3} \bibleverse{3} Yahvé escuchó la voz de Israel y
entregó a los cananeos; y los destruyeron por completo a ellos y a sus
ciudades. El nombre del lugar fue llamado Hormah. \footnote{\textbf{21:3}
  ``Hormah'' significa ``destrucción''.}

\hypertarget{murmullos-de-la-gente-las-serpientes-venenosas-y-la-serpiente-de-bronce}{%
\subsection{Murmullos de la gente; las serpientes venenosas y la
serpiente de
bronce}\label{murmullos-de-la-gente-las-serpientes-venenosas-y-la-serpiente-de-bronce}}

\bibleverse{4} Viajaron desde el monte Hor por el camino del Mar Rojo,
para rodear la tierra de Edom. El alma del pueblo estaba muy desanimada
a causa del viaje. \footnote{\textbf{21:4} Núm 11,1-6; Núm 14,2}
\bibleverse{5} El pueblo hablaba contra Dios y contra Moisés: ``¿Por qué
nos has sacado de Egipto para morir en el desierto? Porque no hay pan,
no hay agua, y nuestra alma aborrece esta comida repugnante''.

\bibleverse{6} El Señor envió serpientes venenosas entre el pueblo, y
éstas mordieron al pueblo. Murió mucha gente de Israel. \footnote{\textbf{21:6}
  1Cor 10,9} \bibleverse{7} El pueblo se acercó a Moisés y le dijo:
``Hemos pecado, porque hemos hablado contra Yavé y contra ti. Ruega a
Yahvé que nos quite las serpientes''. Moisés oró por el pueblo.

\bibleverse{8} Yahvé dijo a Moisés: ``Haz una serpiente venenosa y ponla
en un poste. Sucederá que todo el que sea mordido, cuando la vea,
vivirá''. \bibleverse{9} Moisés hizo una serpiente de bronce y la puso
en el asta. Si una serpiente había mordido a algún hombre, cuando miraba
la serpiente de bronce, vivía.

\hypertarget{el-tren-al-arnuxf3n-ya-las-estepas-de-los-moabitas-la-canciuxf3n-de-la-fuente}{%
\subsection{El tren al Arnón ya las estepas de los moabitas; la canción
de la
fuente}\label{el-tren-al-arnuxf3n-ya-las-estepas-de-los-moabitas-la-canciuxf3n-de-la-fuente}}

\bibleverse{10} Los hijos de Israel partieron y acamparon en Obot.
\bibleverse{11} Partieron de Obot y acamparon en Iyeabarim, en el
desierto que está frente a Moab, hacia el amanecer. \bibleverse{12} De
allí partieron y acamparon en el valle de Zered. \bibleverse{13} De allí
partieron y acamparon al otro lado del Arnón, que está en el desierto
que sale de la frontera del amorreo; porque el Arnón es la frontera de
Moab, entre Moab y el amorreo. \bibleverse{14} Por eso se dice en el
Libro de las Guerras de Yahvé: ``Vaheb en Suphah, los valles del Arnón,
\footnote{\textbf{21:14} Jos 10,13} \bibleverse{15} la pendiente de los
valles que se inclinan hacia la morada de Ar, se inclina sobre la
frontera de Moab.''

\bibleverse{16} Desde allí viajaron a Beer; ése es el pozo del que Yahvé
dijo a Moisés: ``Reúne al pueblo y les daré agua''.

\bibleverse{17} Entonces Israel cantó esta canción: ``¡Surge, pues!
Cántale, \bibleverse{18} el pozo que cavaron los príncipes, que los
nobles del pueblo cavaron, con el cetro, y con sus varas''. Desde el
desierto viajaron a Matana; \bibleverse{19} y de Matana a Nahaliel; y de
Nahaliel a Bamot; \bibleverse{20} y de Bamot al valle que está en el
campo de Moab, a la cima del Pisga, que mira hacia el desierto.

\hypertarget{derrota-del-rey-amorreo-sehuxf3n-y-conquista-de-su-pauxeds-canciuxf3n-de-triunfo-de-los-israelitas}{%
\subsection{Derrota del rey amorreo Sehón y conquista de su país;
Canción de triunfo de los
israelitas}\label{derrota-del-rey-amorreo-sehuxf3n-y-conquista-de-su-pauxeds-canciuxf3n-de-triunfo-de-los-israelitas}}

\bibleverse{21} Israel envió mensajeros a Sehón, rey de los amorreos,
diciendo: \footnote{\textbf{21:21} Deut 2,26-37} \bibleverse{22}
``Déjame pasar por tu tierra. No nos apartaremos del campo ni de la
viña. No beberemos del agua de los pozos. Iremos por el camino del rey,
hasta que hayamos pasado tu frontera''.

\bibleverse{23} Sijón no permitió que Israel pasara por su frontera,
pero Sijón reunió a todo su pueblo y salió contra Israel en el desierto,
y llegó a Jahaz. Luchó contra Israel. \bibleverse{24} Israel lo hirió a
filo de espada y se apoderó de su tierra desde Arnón hasta Jaboc, hasta
los hijos de Amón, pues la frontera de los hijos de Amón estaba
fortificada. \bibleverse{25} Israel tomó todas estas ciudades. Israel
vivió en todas las ciudades de los amorreos, en Hesbón y en todas sus
aldeas. \bibleverse{26} Porque Hesbón era la ciudad de Sehón, rey de los
amorreos, que había luchado contra el antiguo rey de Moab y le había
arrebatado toda su tierra hasta el Arnón. \bibleverse{27} Por eso dicen
los que hablan en proverbios, ``Ven a Heshbon. Que se construya y se
establezca la ciudad de Sehón; \bibleverse{28} porque el fuego ha salido
de Hesbón, una llama de la ciudad de Sihon. Ha devorado a Ar de Moab,
Los señores de los lugares altos del Arnón. \bibleverse{29} ¡Ay de ti,
Moab! ¡Están deshechos, gente de Chemosh! Ha entregado a sus hijos como
fugitivos, y sus hijas en cautiverio, a Sehón, rey de los amorreos.
\bibleverse{30} Les hemos disparado. Hesbón ha perecido hasta Dibón.
Hemos arrasado incluso con Nophah, Que llega hasta Medeba''.

\hypertarget{mayor-avance-de-los-israelitas-derrota-del-rey-og-de-basan}{%
\subsection{Mayor avance de los israelitas; Derrota del rey Og de
Basan}\label{mayor-avance-de-los-israelitas-derrota-del-rey-og-de-basan}}

\bibleverse{31} Así vivió Israel en la tierra de los amorreos.
\bibleverse{32} Moisés envió a espiar a Jazer. Tomaron sus aldeas y
expulsaron a los amorreos que estaban allí. \bibleverse{33} Se volvieron
y subieron por el camino de Basán. Og, el rey de Basán, salió contra
ellos, él y todo su pueblo, para combatir en Edrei. \footnote{\textbf{21:33}
  Deut 3,1-11}

\bibleverse{34} El Señor dijo a Moisés: ``No le temas, porque lo he
entregado en tu mano, con todo su pueblo y su tierra. Harás con él lo
mismo que hiciste con Sehón, rey de los amorreos, que vivía en Hesbón''.
\footnote{\textbf{21:34} Sal 136,17-22}

\bibleverse{35} Así que lo hirieron, con sus hijos y todo su pueblo,
hasta que no hubo sobrevivientes; y se apoderaron de su tierra.

\hypertarget{der-moabiterkuxf6nig-balak-beschlieuxdft-gesandte-an-bileam-zu-schicken}{%
\subsection{Der Moabiterkönig Balak beschließt, Gesandte an Bileam zu
schicken}\label{der-moabiterkuxf6nig-balak-beschlieuxdft-gesandte-an-bileam-zu-schicken}}

\hypertarget{section-21}{%
\section{22}\label{section-21}}

\bibleverse{1} Los hijos de Israel partieron y acamparon en las llanuras
de Moab, al otro lado del Jordán, en Jericó. \bibleverse{2} Balac, hijo
de Zipor, vio todo lo que Israel había hecho a los amorreos.
\bibleverse{3} Moab tuvo mucho miedo del pueblo, porque era numeroso.
Moab estaba angustiado a causa de los hijos de Israel. \bibleverse{4}
Moab dijo a los ancianos de Madián: ``Ahora esta multitud lamerá todo lo
que nos rodea, como el buey lame la hierba del campo.'' Balac hijo de
Zipor era entonces rey de Moab. \bibleverse{5} Envió mensajeros a
Balaam, hijo de Beor, a Pethor, que está junto al río, a la tierra de
los hijos de su pueblo, para llamarlo, diciendo: ``He aquí que hay un
pueblo que salió de Egipto. He aquí que cubren la superficie de la
tierra, y se alojan frente a mí. \footnote{\textbf{22:5} Jos 24,9; Miq
  6,5} \bibleverse{6} Por tanto, ven ahora y maldice a este pueblo por
mí, porque es demasiado poderoso para mí. Tal vez prevalezca, para que
los golpeemos y los expulse de la tierra; porque sé que el que bendices
es bendito, y el que maldices es maldito.''

\hypertarget{la-primera-embajada-de-balac-a-balaam-sin-uxe9xito-su-mensaje-repetido}{%
\subsection{La primera embajada de Balac a Balaam sin éxito; su mensaje
repetido}\label{la-primera-embajada-de-balac-a-balaam-sin-uxe9xito-su-mensaje-repetido}}

\bibleverse{7} Los ancianos de Moab y los ancianos de Madián partieron
con los premios de adivinación en la mano. Vinieron a Balaam y le
hablaron de las palabras de Balac. \footnote{\textbf{22:7} 2Pe 2,15}

\bibleverse{8} Les dijo: ``Quédense aquí esta noche, y les traeré de
nuevo la palabra, según me hable Yahvé''. Los príncipes de Moab se
quedaron con Balaam.

\bibleverse{9} Dios se acercó a Balaam y le dijo: ``¿Quiénes son estos
hombres que están contigo?''

\bibleverse{10} Balaam dijo a Dios: ``Balac, hijo de Zipor, rey de Moab,
me ha dicho: \bibleverse{11} `He aquí que el pueblo que ha salido de
Egipto cubre la superficie de la tierra. Ahora, ven a maldecirlos por
mí. Tal vez pueda luchar contra ellos y los expulse'\,''.

\bibleverse{12} Dios dijo a Balaam: ``No irás con ellos. No maldecirás
al pueblo, porque está bendecido''.

\bibleverse{13} Balaam se levantó por la mañana y dijo a los príncipes
de Balac: ``Vayan a su tierra, porque Yahvé no me permite ir con
ustedes''.

\bibleverse{14} Los príncipes de Moab se levantaron y fueron a ver a
Balac y le dijeron: ``Balaam se niega a venir con nosotros''.

\bibleverse{15} Balac volvió a enviar príncipes, más, y más honorables
que ellos. \bibleverse{16} Ellos vinieron a Balaam y le dijeron:
``Balac, hijo de Zipor, dice: `Por favor, no dejes que nada te impida
venir a mí, \bibleverse{17} porque te ascenderé a un honor muy grande, y
todo lo que me digas lo haré. Ven, pues, y maldice a este pueblo por
mí'\,''.

\bibleverse{18} Balaam respondió a los siervos de Balac: ``Si Balac me
diera su casa llena de plata y oro, no podría ir más allá de la palabra
de Yahvé, mi Dios, para hacer menos o más. \footnote{\textbf{22:18} 1Re
  13,8} \bibleverse{19} Ahora, pues, quédate aquí también esta noche,
para saber qué más me dirá Yahvé''.

\bibleverse{20} Dios vino a Balaam de noche y le dijo: ``Si los hombres
han venido a llamarte, levántate y ve con ellos; pero sólo harás la
palabra que yo te diga.''

\bibleverse{21} Balaam se levantó por la mañana, ensilló su asno y se
fue con los príncipes de Moab.

\hypertarget{el-viaje-de-balaam-a-moab-y-el-incidente-con-el-burro}{%
\subsection{El viaje de Balaam a Moab y el incidente con el
burro}\label{el-viaje-de-balaam-a-moab-y-el-incidente-con-el-burro}}

\bibleverse{22} La ira de Dios se encendió porque él iba, y el ángel de
Yavé se puso en el camino como adversario suyo. Iba montado en su asno,
y lo acompañaban sus dos siervos. \bibleverse{23} El asno vio al ángel
de Yavé parado en el camino, con su espada desenvainada en la mano; y el
asno se apartó del camino y se metió en el campo. Balaam golpeó a la
burra para hacerla volver al camino. \footnote{\textbf{22:23} Gén 3,24;
  Jos 5,13} \bibleverse{24} Entonces el ángel de Yavé se paró en un
sendero estrecho entre las viñas, con un muro a un lado y otro a otro.
\bibleverse{25} La burra vio al ángel de Yavé, y se arrimó a la pared, y
aplastó el pie de Balaam contra la pared. Él la golpeó de nuevo.

\bibleverse{26} El ángel de Yahvé fue más allá y se paró en un lugar
estrecho, donde no había forma de girar ni a la derecha ni a la
izquierda. \bibleverse{27} La burra vio al ángel de Yavé y se acostó
debajo de Balaam. La ira de Balaam ardió, y golpeó a la burra con su
bastón.

\bibleverse{28} El Señor abrió la boca de la burra, y ella le dijo a
Balaam: ``¿Qué te he hecho, para que me hayas golpeado estas tres
veces?'' \footnote{\textbf{22:28} 2Pe 2,16}

\bibleverse{29} Balaam dijo al asno: ``Porque te has burlado de mí,
ojalá tuviera una espada en la mano, porque ahora te habría matado''.

\bibleverse{30} El asno dijo a Balaam: ``¿No soy yo tu asno, en el que
has montado toda tu vida hasta hoy? ¿Acaso he tenido la costumbre de
hacerlo contigo?'' Dijo: ``No''.

\bibleverse{31} Entonces el Señor abrió los ojos de Balaam, y vio al
ángel del Señor parado en el camino, con su espada desenvainada en la
mano; e inclinó la cabeza y se postró sobre su rostro. \bibleverse{32}
El ángel de Yahvé le dijo: ``¿Por qué has golpeado a tu asno estas tres
veces? He aquí que he salido como adversario, porque tu camino es
perverso ante mí. \bibleverse{33} La burra me vio y se apartó ante mí
estas tres veces. Si no se hubiera apartado de mí, seguramente ahora te
habría matado a ti y la habría salvado con vida''.

\bibleverse{34} Balaam le dijo al ángel de Yavé: ``He pecado, pues no
sabía que estabas en el camino contra mí. Ahora, pues, si te desagrada,
volveré a regresar''.

\bibleverse{35} El ángel de Yahvé dijo a Balaam: ``Ve con los hombres,
pero sólo hablarás la palabra que yo te diga''. Entonces Balaam fue con
los príncipes de Balac.

\hypertarget{la-llegada-de-balaam-a-balac}{%
\subsection{La llegada de Balaam a
Balac}\label{la-llegada-de-balaam-a-balac}}

\bibleverse{36} Cuando Balac oyó que Balaam había venido, salió a
recibirlo a la ciudad de Moab, que está en la frontera de Arnón, que
está en el extremo de la frontera. \bibleverse{37} Balac le dijo a
Balaam: ``¿No envié a buscarte con insistencia para convocarte? ¿Por qué
no viniste a mí? ¿Acaso no puedo promoverte a la honra?''

\bibleverse{38} Balaam dijo a Balac: ``He aquí que he venido a ti.
¿Tengo ahora algún poder para hablar algo? Hablaré la palabra que Dios
ponga en mi boca''.

\bibleverse{39} Balaam fue con Balac, y llegaron a Quiriat Huzot.
\bibleverse{40} Balac sacrificó ganado y ovejas, y envió a Balaam y a
los príncipes que estaban con él. \bibleverse{41} Por la mañana, Balac
tomó a Balaam y lo hizo subir a los lugares altos de Baal, y vio desde
allí a parte del pueblo. \footnote{\textbf{22:41} Núm 23,28}

\hypertarget{los-preparativos-para-la-revelaciuxf3n-divina-el-primer-dicho-de-balaam}{%
\subsection{Los preparativos para la revelación divina; el primer dicho
de
Balaam}\label{los-preparativos-para-la-revelaciuxf3n-divina-el-primer-dicho-de-balaam}}

\hypertarget{section-22}{%
\section{23}\label{section-22}}

\bibleverse{1} Balaam dijo a Balac: ``Construye aquí siete altares para
mí, y prepara aquí siete toros y siete carneros para mí''.

\bibleverse{2} Balac hizo lo que Balaam había dicho; y Balac y Balaam
ofrecieron en cada altar un toro y un carnero. \bibleverse{3} Balaam le
dijo a Balac: ``Quédate junto a tu holocausto, y yo me iré. Tal vez el
Señor venga a mi encuentro. Lo que él me muestre te lo diré''. Se
dirigió a una altura despojada. \bibleverse{4} Dios salió al encuentro
de Balaam y le dijo: ``He preparado los siete altares y he ofrecido un
toro y un carnero en cada altar.''

\bibleverse{5} Yahvé puso una palabra en la boca de Balaam y le dijo:
``Vuelve a Balac y así hablarás''.

\bibleverse{6} Volvió a él, y he aquí que estaba junto a su holocausto,
él y todos los príncipes de Moab.

\hypertarget{balaam-bendice-a-israel-desde-bamot-baal}{%
\subsection{Balaam bendice a Israel desde
Bamot-Baal}\label{balaam-bendice-a-israel-desde-bamot-baal}}

\bibleverse{7} Tomó su parábola y dijo, ``De Aram me ha traído Balak, el
rey de Moab desde las montañas del Este. Ven, maldice a Jacob por mí.
Ven, desafía a Israel. \bibleverse{8} ¿Cómo voy a maldecir a quien Dios
no ha maldecido? ¿Cómo voy a desafiar a quien Yahvé no ha desafiado?
\bibleverse{9} Porque desde lo alto de las rocas lo veo. Desde las
colinas lo veo. He aquí que es un pueblo que habita solo, y no será
catalogado entre las naciones. \bibleverse{10} Quién puede contar el
polvo de Jacob, o contar la cuarta parte de Israel? ¡Dejadme morir como
los justos! ¡Que mi último final sea como el suyo!'' \footnote{\textbf{23:10}
  Gén 13,16; Núm 31,8}

\bibleverse{11} Balac dijo a Balaam: ``¿Qué me has hecho? Te tomé para
maldecir a mis enemigos, y he aquí que los has bendecido por completo''.

\bibleverse{12} Respondió y dijo: ``¿No debo tener cuidado de decir lo
que Yahvé pone en mi boca?'' \footnote{\textbf{23:12} Núm 22,38}

\hypertarget{los-preparativos-para-la-nueva-revelaciuxf3n-divina-el-segundo-dicho-de-balaam}{%
\subsection{Los preparativos para la nueva revelación divina; el segundo
dicho de
Balaam}\label{los-preparativos-para-la-nueva-revelaciuxf3n-divina-el-segundo-dicho-de-balaam}}

\bibleverse{13} Balac le dijo: ``Por favor, ven conmigo a otro lugar,
donde puedas verlos. Sólo verás una parte de ellos, y no los verás
todos. Maldícelos desde allí por mí''.

\bibleverse{14} Lo llevó al campo de Zofim, a la cima del Pisga, y
construyó siete altares, y ofreció un toro y un carnero en cada altar.
\bibleverse{15} Le dijo a Balac: ``Quédate aquí con tu holocausto,
mientras yo me encuentro con Dios allá''.

\bibleverse{16} Yahvé salió al encuentro de Balaam y puso una palabra en
su boca, diciendo: ``Vuelve a Balac y dile esto''.

\bibleverse{17} Se acercó a él, y he aquí que estaba de pie junto a su
holocausto, y los príncipes de Moab con él. Balac le dijo: ``¿Qué ha
dicho Yahvé?''

\hypertarget{balaam-bendice-a-israel-desde-el-monte-pisga}{%
\subsection{Balaam bendice a Israel desde el monte
Pisga}\label{balaam-bendice-a-israel-desde-el-monte-pisga}}

\bibleverse{18} Retomó su parábola y dijo, ``¡Levántate, Balak, y
escucha! Escúchame, hijo de Zippor. \bibleverse{19} Dios no es un
hombre, para que mienta, ni hijo de hombre, que se arrepienta. ¿Ha
dicho, y no lo hará? ¿O ha hablado y no lo hará bien? \footnote{\textbf{23:19}
  1Sam 15,29} \bibleverse{20} He aquí que he recibido la orden de
bendecir. Ha bendecido, y no puedo revertirlo. \bibleverse{21} No ha
visto iniquidad en Jacob. Tampoco ha visto perversidad en Israel. Yavé,
su Dios, está con él. El grito de un rey está entre ellos.
\bibleverse{22} Dios los saca de Egipto. Tiene como la fuerza del buey
salvaje. \bibleverse{23} Seguramente no hay ningún encantamiento con
Jacob; tampoco hay adivinación con Israel. Ahora se dirá de Jacob y de
Israel, ``¡Qué ha hecho Dios! \bibleverse{24} He aquí que un pueblo se
levanta como una leona. Como un león se levanta. No se acostará hasta
que coma de la presa, y bebe la sangre de los muertos''. \footnote{\textbf{23:24}
  Núm 24,9}

\bibleverse{25} Balac dijo a Balaam: ``Ni los maldigas, ni los
bendigas''.

\bibleverse{26} Pero Balaam respondió a Balac: ``¿No te dije que todo lo
que diga Yahvé lo tengo que hacer?'' \footnote{\textbf{23:26} Núm 23,12}

\hypertarget{los-preparativos-para-la-tercera-revelaciuxf3n-divina-el-tercer-dicho-de-balaam}{%
\subsection{Los preparativos para la tercera revelación divina; el
tercer dicho de
Balaam}\label{los-preparativos-para-la-tercera-revelaciuxf3n-divina-el-tercer-dicho-de-balaam}}

\bibleverse{27} Balac dijo a Balaam: ``Ven ahora, te llevaré a otro
lugar; tal vez le plazca a Dios que los maldigas por mí desde allí''.

\bibleverse{28} Balac llevó a Balaam a la cima de Peor, que da al
desierto. \footnote{\textbf{23:28} Núm 25,3} \bibleverse{29} Balaam le
dijo a Balac: ``Construye aquí siete altares para mí, y prepara aquí
siete toros y siete carneros''. \footnote{\textbf{23:29} Núm 23,1}

\bibleverse{30} Balac hizo lo que había dicho Balaam, y ofreció un toro
y un carnero en cada altar.

\hypertarget{balaam-bendice-a-israel-desde-el-monte-peor}{%
\subsection{Balaam bendice a Israel desde el monte
Peor}\label{balaam-bendice-a-israel-desde-el-monte-peor}}

\hypertarget{section-23}{%
\section{24}\label{section-23}}

\bibleverse{1} Cuando Balaam vio que a Yahvé le agradaba bendecir a
Israel, no fue, como las otras veces, a usar la adivinación, sino que
puso su rostro hacia el desierto. \bibleverse{2} Balaam levantó sus ojos
y vio a Israel habitando según sus tribus, y el Espíritu de Dios vino
sobre él. \bibleverse{3} Tomó su parábola y dijo``Balaam el hijo de Beor
dice, el hombre que tiene los ojos abiertos dice; \footnote{\textbf{24:3}
  1Sam 9,9} \bibleverse{4} dice, que escucha las palabras de Dios, que
ve la visión del Todopoderoso, cayendo, y teniendo los ojos abiertos:
\footnote{\textbf{24:4} Is 50,4} \bibleverse{5} Qué buenas son tus
tiendas, Jacob, ¡y tus moradas, Israel! \bibleverse{6} Como valles se
extienden, como jardines a la orilla del río, como áloes que Yahvé ha
plantado, como los cedros junto a las aguas. \bibleverse{7} El agua
fluirá de sus cubos. Su semilla estará en muchas aguas. Su rey será más
alto que Agag. Su reino será exaltado. \bibleverse{8} Dios lo saca de
Egipto. Tiene como la fuerza del buey salvaje. Consumirá a las naciones
sus adversarios, romperá sus huesos en pedazos, y los atravesará con sus
flechas. \bibleverse{9} Se acuesta, se acuesta como un león, como una
leona; ¿quién lo despertará? Todos los que te bendicen son bendecidos.
Todo el que te maldiga está maldito''. \footnote{\textbf{24:9} Núm
  23,24; Gén 49,9; Gén 12,3}

\hypertarget{la-ira-de-balac-y-la-disculpa-de-balaam}{%
\subsection{La ira de Balac y la disculpa de
Balaam}\label{la-ira-de-balac-y-la-disculpa-de-balaam}}

\bibleverse{10} La ira de Balac ardió contra Balaam, y éste se golpeó
las manos. Balac dijo a Balaam: ``Te llamé para que maldijeras a mis
enemigos, y he aquí que los has bendecido por completo estas tres veces.
\bibleverse{11} Por lo tanto, ¡huye ahora a tu lugar! Yo pensaba
promoverte a un gran honor; pero, he aquí, Yahvé te ha alejado del
honor''.

\bibleverse{12} Balaam le dijo a Balac: ``¿No les dije también a tus
mensajeros que me enviaste, diciéndoles: \bibleverse{13} `Si Balac me da
su casa llena de plata y oro, no puedo ir más allá de la palabra de
Yahvé, para hacer el bien o el mal de mi propia mente. Diré lo que dice
Yahvé'? \footnote{\textbf{24:13} Núm 22,18} \bibleverse{14} Ahora, he
aquí que voy a mi pueblo. Ven, te informaré de lo que este pueblo hará a
tu pueblo en los últimos días''.

\hypertarget{cuarto-dicho-de-balaam-la-estrella-de-jacob-cuya-victoria-sobre-moab-y-edom}{%
\subsection{Cuarto dicho de Balaam: la estrella de Jacob; cuya victoria
sobre Moab y
Edom}\label{cuarto-dicho-de-balaam-la-estrella-de-jacob-cuya-victoria-sobre-moab-y-edom}}

\bibleverse{15} Retomó su parábola y dijo, ``Balaam el hijo de Beor
dice, el hombre que tiene los ojos abiertos dice; \footnote{\textbf{24:15}
  Núm 24,3-4} \bibleverse{16} dice, que escucha las palabras de Dios,
conoce el conocimiento del Altísimo, y que ve la visión del
Todopoderoso, cayendo, y teniendo los ojos abiertos: \bibleverse{17} Lo
veo, pero no ahora. Lo veo, pero no cerca. Una estrella saldrá de Jacob.
Un cetro se levantará de Israel, y golpeará los rincones de Moab, y
aplastar a todos los hijos de Sheth. \footnote{\textbf{24:17} Mat 2,2;
  Luc 1,78; 2Sam 8,2; Am 2,2} \bibleverse{18} Edom será una posesión.
Seir, su enemigo, también será una posesión, mientras que Israel lo hace
valientemente. \footnote{\textbf{24:18} 2Sam 8,14; Am 9,11; Am 1,9-12}
\bibleverse{19} De Jacob uno tendrá el dominio, y destruirá el remanente
de la ciudad''. \footnote{\textbf{24:19} Miq 5,2; Miq 5,8-9}

\hypertarget{proverbios-sobre-los-amalecitas-ceneos-y-asirios-fin-de-la-historia-de-balaam}{%
\subsection{Proverbios sobre los amalecitas, ceneos y asirios; Fin de la
historia de
Balaam}\label{proverbios-sobre-los-amalecitas-ceneos-y-asirios-fin-de-la-historia-de-balaam}}

\bibleverse{20} Miró a Amalec, retomó su parábola y dijo, ``Amalek fue
la primera de las naciones, pero su último fin será la destrucción''.
\footnote{\textbf{24:20} Éxod 17,14}

\bibleverse{21} Miró al ceneo, retomó su parábola y dijo``Tu morada es
fuerte. Su nido está enclavado en la roca. \footnote{\textbf{24:21} 1Sam
  15,6; Abd 1,3} \bibleverse{22} Sin embargo, Caín será destruido, hasta
que Asur te lleve cautivo''.

\bibleverse{23} Retomó su parábola y dijo, ``Ay, ¿quién vivirá cuando
Dios haga esto? \bibleverse{24} Pero los barcos vendrán de la costa de
Kittim. Afligirán a Asur y afligirán a Éber. También él vendrá a la
destrucción''. \footnote{\textbf{24:24} 1Maca 1,1}

\bibleverse{25} Balaam se levantó, y se fue y volvió a su lugar; y Balac
también se fue.

\hypertarget{la-deuda-de-israel-a-travuxe9s-de-la-fornicaciuxf3n-y-la-idolatruxeda}{%
\subsection{La deuda de Israel a través de la fornicación y la
idolatría}\label{la-deuda-de-israel-a-travuxe9s-de-la-fornicaciuxf3n-y-la-idolatruxeda}}

\hypertarget{section-24}{%
\section{25}\label{section-24}}

\bibleverse{1} Israel se quedó en Sitim, y el pueblo comenzó a
prostituirse con las hijas de Moab; \bibleverse{2} pues llamaron al
pueblo a los sacrificios de sus dioses. El pueblo comía y se inclinaba
ante sus dioses. \footnote{\textbf{25:2} Núm 31,16} \bibleverse{3}
Israel se unió a Baal Peor, y la ira de Yavé ardió contra Israel.
\footnote{\textbf{25:3} Deut 4,3} \bibleverse{4} Yahvé dijo a Moisés:
``Toma a todos los jefes del pueblo y cuélgalos a Yahvé ante el sol,
para que el furor de Yahvé se aparte de Israel.'' \footnote{\textbf{25:4}
  2Sam 21,6; 2Sam 21,9; Deut 21,22-23}

\bibleverse{5} Moisés dijo a los jueces de Israel: ``Maten todos a sus
hombres que se han unido a Baal Peor''.

\hypertarget{la-intervenciuxf3n-de-phinehas-su-enajenaciuxf3n-de-dios-con-un-sacerdocio-eterno}{%
\subsection{La intervención de Phinehas; su enajenación de Dios con un
sacerdocio
eterno}\label{la-intervenciuxf3n-de-phinehas-su-enajenaciuxf3n-de-dios-con-un-sacerdocio-eterno}}

\bibleverse{6} He aquí que uno de los hijos de Israel vino y trajo a sus
hermanos una mujer madianita a la vista de Moisés y de toda la
congregación de los hijos de Israel, mientras lloraban a la puerta de la
Tienda de Reunión. \bibleverse{7} Cuando Finees, hijo de Eleazar, hijo
del sacerdote Aarón, lo vio, se levantó de en medio de la congregación y
tomó una lanza en su mano. \bibleverse{8} Fue tras el hombre de Israel
al pabellón, y los atravesó a ambos, al hombre de Israel y a la mujer
por el cuerpo. Así se detuvo la plaga entre los hijos de Israel.
\bibleverse{9} Los que murieron por la plaga fueron veinticuatro mil.
\footnote{\textbf{25:9} 1Cor 10,8}

\bibleverse{10} Yahvé habló a Moisés y le dijo: \bibleverse{11}
``Finees, hijo de Eleazar, hijo del sacerdote Aarón, ha alejado mi ira
de los hijos de Israel, ya que se puso celoso con mis celos entre ellos,
para que yo no consumiera a los hijos de Israel en mis celos.
\bibleverse{12} Por tanto, di: ``He aquí que yo le doy mi pacto de paz.
\footnote{\textbf{25:12} 1Cró 9,20} \bibleverse{13} Será para él, y para
su descendencia después de él, el pacto de un sacerdocio eterno, porque
fue celoso por su Dios e hizo expiación por los hijos de Israel.'\,''
\footnote{\textbf{25:13} Sal 106,30-31}

\bibleverse{14} El nombre del hombre de Israel que fue asesinado con la
mujer madianita era Zimri, hijo de Salu, príncipe de una casa paterna
entre los simeonitas. \bibleverse{15} El nombre de la mujer madianita
que fue asesinada era Cozbi, hija de Zur. Era jefe de la gente de una
casa paterna en Madián. \footnote{\textbf{25:15} Núm 31,8}

\hypertarget{gottes-gebot-an-den-midianitern-rache-zu-nehmen}{%
\subsection{Gottes Gebot, an den Midianitern Rache zu
nehmen}\label{gottes-gebot-an-den-midianitern-rache-zu-nehmen}}

\bibleverse{16} Yahvé habló a Moisés, diciendo: \bibleverse{17} ``Acosa
a los madianitas y golpéalos; \footnote{\textbf{25:17} Núm 31,2-10}
\bibleverse{18} porque te han acosado con sus artimañas, en las que te
han engañado en el asunto de Peor, y en el incidente relativo a Cozbi,
la hija del príncipe de Madián, su hermana, que fue asesinada el día de
la plaga en el asunto de Peor.''

\hypertarget{el-segundo-censo-de-la-gente-en-la-llanura-de-los-moabitas-con-el-propuxf3sito-de-distribuir-la-tierra}{%
\subsection{El segundo censo de la gente en la llanura de los moabitas
con el propósito de distribuir la
tierra}\label{el-segundo-censo-de-la-gente-en-la-llanura-de-los-moabitas-con-el-propuxf3sito-de-distribuir-la-tierra}}

\hypertarget{section-25}{%
\section{26}\label{section-25}}

\bibleverse{1} Después de la plaga, Yahvé habló a Moisés y al sacerdote
Eleazar hijo de Aarón, diciendo: \bibleverse{2} ``Hagan un censo de toda
la congregación de los hijos de Israel, de veinte años en adelante, por
las casas de sus padres, todos los que puedan salir a la guerra en
Israel.'' \footnote{\textbf{26:2} Núm 1,2-47} \bibleverse{3} Moisés y el
sacerdote Eleazar hablaron con ellos en las llanuras de Moab, junto al
Jordán, en Jericó, diciendo: \bibleverse{4} ``Hagan un censo, de veinte
años en adelante, como Yahvé les ordenó a Moisés y a los hijos de
Israel.'' Estos son los que salieron de la tierra de Egipto.

\hypertarget{los-resultados-del-censo-1}{%
\subsection{Los resultados del censo}\label{los-resultados-del-censo-1}}

\bibleverse{5} Rubén, el primogénito de Israel; los hijos de Rubén: de
Hanoc, la familia de los Hanocitas; de Palú, la familia de los Palúitas;
\footnote{\textbf{26:5} Gén 46,8-27; 1Cró 4,1-7} \bibleverse{6} de
Hezrón, la familia de los Hezronitas; de Carmi, la familia de los
Carmitas. \bibleverse{7} Estas son las familias de los rubenitas; y los
contados de ellas fueron cuarenta y tres mil setecientos treinta.
\bibleverse{8} El hijo de Pallu: Eliab. \bibleverse{9} Los hijos de
Eliab: Nemuel, Datán y Abiram. Estos son aquellos Datán y Abiram que
fueron llamados por la congregación, que se rebelaron contra Moisés y
contra Aarón en la compañía de Coré cuando se rebelaron contra Yahvé;
\footnote{\textbf{26:9} Núm 16,1} \bibleverse{10} y la tierra abrió su
boca y los tragó junto con Coré cuando esa compañía murió; en ese
momento el fuego devoró a doscientos cincuenta hombres, y se
convirtieron en una señal. \bibleverse{11} Sin embargo, los hijos de
Coré no murieron. \bibleverse{12} Los hijos de Simeón por sus familias:
de Nemuel, la familia de los nemuelitas; de Jamín, la familia de los
jaminitas; de Jacín, la familia de los jacinitas; \bibleverse{13} de
Zera, la familia de los zeraítas; de Shaúl, la familia de los shaúlitas.
\bibleverse{14} Estas son las familias de los simeonitas, veintidós mil
doscientos. \bibleverse{15} Los hijos de Gad por sus familias: de Zefón,
la familia de los zefonitas; de Haggi, la familia de los haggitas; de
Shuni, la familia de los shunitas; \bibleverse{16} de Ozni, la familia
de los oznitas; de Eri, la familia de los eritas; \bibleverse{17} de
Arod, la familia de los aroditas; de Areli, la familia de los arelitas.
\bibleverse{18} Estas son las familias de los hijos de Gad según los
contados de ellos, cuarenta mil quinientos. \bibleverse{19} Los hijos de
Judá: Er y Onán. Er y Onán murieron en la tierra de Canaán. \footnote{\textbf{26:19}
  Gén 38,7; Gén 38,10} \bibleverse{20} Los hijos de Judá por sus
familias fueron: de Selá, la familia de los selanitas; de Pérez, la
familia de los pérezicos; de Zera, la familia de los zeraitas.
\bibleverse{21} Los hijos de Pérez fueron: de Hezrón, la familia de los
hezronitas; de Hamul, la familia de los hamulitas. \footnote{\textbf{26:21}
  Rut 4,18} \bibleverse{22} Estas son las familias de Judá según los
contados de ellas, setenta y seis mil quinientos. \bibleverse{23} Los
hijos de Isacar por sus familias: de Tola, la familia de los tolaítas;
de Puva, la familia de los punítas; \bibleverse{24} de Jasub, la familia
de los jasubitas; de Simrón, la familia de los simronitas.
\bibleverse{25} Estas son las familias de Isacar según los contados de
ellas, sesenta y cuatro mil trescientos. \bibleverse{26} Los hijos de
Zabulón por sus familias: de Sered, la familia de los sereditas; de
Elón, la familia de los elonitas; de Jahleel, la familia de los
jahleelitas. \bibleverse{27} Estas son las familias de los zabulonitas
según los contados de ellos, sesenta mil quinientos. \bibleverse{28} Los
hijos de José según sus familias: Manasés y Efraín. \bibleverse{29} Los
hijos de Manasés: de Maquir, la familia de los maquiritas; y Maquir fue
el padre de Galaad; de Galaad, la familia de los galaaditas. \footnote{\textbf{26:29}
  Jos 17,1-3} \bibleverse{30} Estos son los hijos de Galaad: de Iezer,
la familia de los Iezeritas; de Helek, la familia de los Helekitas;
\bibleverse{31} y de Asriel, la familia de los Asrielitas; y de Siquem,
la familia de los Siquemitas; \bibleverse{32} y de Semida, la familia de
los Semidaitas; y de Hefer, la familia de los Heferitas. \bibleverse{33}
Zelofehad, hijo de Hefer, no tuvo hijos, sino hijas; y los nombres de
las hijas de Zelofehad fueron Mahá, Noé, Hogá, Milca y Tirsa.
\footnote{\textbf{26:33} Núm 27,1} \bibleverse{34} Estas son las
familias de Manasés. Los contados de ellos fueron cincuenta y dos mil
setecientos. \bibleverse{35} Estos son los hijos de Efraín por sus
familias: de Sutela, la familia de los Sutelitas; de Becher, la familia
de los Becheritas; de Tahan, la familia de los Tahanitas.
\bibleverse{36} Estos son los hijos de Sutela: de Erán, la familia de
los eranitas. \bibleverse{37} Estas son las familias de los hijos de
Efraín, según los contados de ellos, treinta y dos mil quinientos. Estos
son los hijos de José por sus familias. \bibleverse{38} Los hijos de
Benjamín por sus familias: de Bela, la familia de los belaítas; de
Asbel, la familia de los asbelitas; de Ahiram, la familia de los
ahiramitas; \bibleverse{39} de Sefufam, la familia de los shufamitas; de
Hufam, la familia de los hufamitas. \bibleverse{40} Los hijos de Bela
fueron Ard y Naamán: la familia de los arditas; y de Naamán, la familia
de los naamitas. \bibleverse{41} Estos son los hijos de Benjamín por sus
familias; y los contados de ellos fueron cuarenta y cinco mil
seiscientos. \bibleverse{42} Estos son los hijos de Dan por sus
familias: de Shuham, la familia de los shuhamitas. Estas son las
familias de Dan por sus familias. \bibleverse{43} Todas las familias de
los suhamitas, según los contados de ellos, fueron sesenta y cuatro mil
cuatrocientos. \bibleverse{44} Los hijos de Aser por sus familias: de
Imna, la familia de los imnitas; de Ishvi, la familia de los ishvitas;
de Beriá, la familia de los beritas. \bibleverse{45} De los hijos de
Beriá: de Heber, la familia de los heberitas; de Malquiel, la familia de
los malquielitas. \bibleverse{46} El nombre de la hija de Aser fue Sera.
\bibleverse{47} Estas son las familias de los hijos de Aser según los
contados de ellos, cincuenta y tres mil cuatrocientos. \bibleverse{48}
Los hijos de Neftalí por sus familias: de Jahzeel, la familia de los
jahzeelitas; de Guni, la familia de los gunitas; \bibleverse{49} de
Jezer, la familia de los jezeritas; de Silim, la familia de los
sillemitas. \bibleverse{50} Estas son las familias de Neftalí según sus
familias; y los contados de ellas fueron cuarenta y cinco mil
cuatrocientos. \bibleverse{51} Estos son los contados de los hijos de
Israel, seiscientos un mil setecientos treinta.

\hypertarget{instrucciuxf3n-sobre-distribuciuxf3n-de-tierras}{%
\subsection{Instrucción sobre distribución de
tierras}\label{instrucciuxf3n-sobre-distribuciuxf3n-de-tierras}}

\bibleverse{52} Yahvé habló a Moisés diciendo: \bibleverse{53} ``A éstos
se les repartirá la tierra en herencia según el número de nombres.
\bibleverse{54} A los más les darás más herencia, y a los menos les
darás menos herencia. A cada uno se le dará su herencia según los
contados de él. \bibleverse{55} Sin embargo, la tierra se dividirá por
sorteo. Según los nombres de las tribus de sus padres heredarán.
\footnote{\textbf{26:55} Núm 33,54; Jos 14,2} \bibleverse{56} Según la
suerte se repartirá su herencia entre los más y los menos''.

\hypertarget{el-conteo-de-los-levitas}{%
\subsection{El conteo de los levitas}\label{el-conteo-de-los-levitas}}

\bibleverse{57} Estos son los contados de los levitas según sus
familias: de Gersón, la familia de los gersonitas; de Coat, la familia
de los coatitas; de Merari, la familia de los meraritas. \footnote{\textbf{26:57}
  Éxod 6,16-25} \bibleverse{58} Estas son las familias de Leví: la
familia de los libnitas, la familia de los hebronitas, la familia de los
mahlitas, la familia de los musitas y la familia de los corasitas. Coat
fue el padre de Amram. \bibleverse{59} El nombre de la esposa de Amram
era Jocabed, hija de Leví, que había nacido de Leví en Egipto. Ella dio
a luz a Amram, a Aarón y a Moisés, y a su hermana Miriam.
\bibleverse{60} De Aarón nacieron Nadab y Abiú, Eleazar e Itamar.
\bibleverse{61} Nadab y Abiú murieron cuando ofrecieron fuego extraño
ante Yahvé. \footnote{\textbf{26:61} Lev 10,1-2} \bibleverse{62} Los que
fueron contados de ellos fueron veintitrés mil, todos los varones de un
mes para arriba; pues no fueron contados entre los hijos de Israel,
porque no se les dio herencia entre los hijos de Israel. \bibleverse{63}
Estos son los que fueron contados por Moisés y el sacerdote Eleazar,
quienes contaron a los hijos de Israel en los llanos de Moab, junto al
Jordán de Jericó. \bibleverse{64} Pero entre éstos no hubo ninguno de
los que fueron contados por Moisés y el sacerdote Aarón, que contaron a
los hijos de Israel en el desierto de Sinaí. \footnote{\textbf{26:64}
  Núm 3,1-39} \bibleverse{65} Porque Yahvé había dicho de ellos:
``Ciertamente morirán en el desierto''. No quedó ningún hombre de ellos,
excepto Caleb, hijo de Jefone, y Josué, hijo de Nun. \footnote{\textbf{26:65}
  Núm 14,22-38}

\hypertarget{disposiciones-relativas-a-la-propiedad-de-los-herederos}{%
\subsection{Disposiciones relativas a la propiedad de los
herederos}\label{disposiciones-relativas-a-la-propiedad-de-los-herederos}}

\hypertarget{section-26}{%
\section{27}\label{section-26}}

\bibleverse{1} Entonces se acercaron las hijas de Zelofehad, hijo de
Hefer, hijo de Galaad, hijo de Maquir, hijo de Manasés, de las familias
de Manasés hijo de José. Estos son los nombres de sus hijas: Mahá, Noé,
Hogá, Milca y Tirsa. \footnote{\textbf{27:1} Núm 26,33; Núm 36,2; Jos
  17,3-6} \bibleverse{2} Se presentaron ante Moisés, ante el sacerdote
Eleazar y ante los príncipes y toda la congregación, a la puerta de la
Tienda del Encuentro, diciendo: \bibleverse{3} ``Nuestro padre murió en
el desierto. No estaba entre la compañía de los que se agruparon contra
Yavé en compañía de Coré, sino que murió en su propio pecado. No tuvo
hijos. \footnote{\textbf{27:3} Núm 16,2; Núm 26,65} \bibleverse{4} ¿Por
qué se ha de quitar el nombre de nuestro padre de entre su familia,
porque no tuvo hijo? Danos una posesión entre los hermanos de nuestro
padre''.

\bibleverse{5} Moisés llevó su causa ante el Señor. \footnote{\textbf{27:5}
  Lev 24,12} \bibleverse{6} Yahvé habló a Moisés, diciendo:
\bibleverse{7} ``Las hijas de Zelofehad hablan con razón. Ciertamente
les darás posesión de una herencia entre los hermanos de su padre. Harás
que la herencia de su padre pase a ellas. \bibleverse{8} Hablarás a los
hijos de Israel diciendo: ``Si un hombre muere y no tiene hijo, harás
que su herencia pase a su hija. \bibleverse{9} Si no tiene hija, darás
su herencia a sus hermanos. \bibleverse{10} Si no tiene hermanos, darás
su herencia a los hermanos de su padre. \bibleverse{11} Si su padre no
tiene hermanos, entonces darás su herencia a su pariente más cercano de
su familia, y él la poseerá. Esto será un estatuto y una ordenanza para
los hijos de Israel, como Yahvé le ordenó a Moisés''.

\hypertarget{anuncio-de-muerte-inminente-a-moisuxe9s-instalaciuxf3n-de-joshua-como-su-sucesor}{%
\subsection{Anuncio de muerte inminente a Moisés; Instalación de Joshua
como su
sucesor}\label{anuncio-de-muerte-inminente-a-moisuxe9s-instalaciuxf3n-de-joshua-como-su-sucesor}}

\bibleverse{12} Yahvé dijo a Moisés: ``Sube a este monte de Abarim y ve
la tierra que he dado a los hijos de Israel. \footnote{\textbf{27:12}
  Deut 32,48-49} \bibleverse{13} Cuando la hayas visto, tú también serás
reunido con tu pueblo, como fue reunido tu hermano Aarón; \footnote{\textbf{27:13}
  Núm 20,24; Núm 20,28} \bibleverse{14} porque en la contienda de la
congregación, te rebelaste contra mi palabra en el desierto de Zin, para
honrarme como santo en las aguas ante sus ojos.'' (Estas son las aguas
de Meribah de Cades en el desierto de Zin). \footnote{\textbf{27:14} Núm
  20,12-13}

\bibleverse{15} Moisés habló a Yahvé, diciendo: \bibleverse{16} ``Que
Yahvé, el Dios de los espíritus de toda carne, designe a un hombre sobre
la congregación, \footnote{\textbf{27:16} Núm 16,22} \bibleverse{17} que
salga delante de ellos, y que entre delante de ellos, y que los conduzca
fuera, y que los haga entrar, para que la congregación de Yahvé no sea
como ovejas que no tienen pastor.'' \footnote{\textbf{27:17} Mat 9,36}

\bibleverse{18} El Señor dijo a Moisés: ``Toma a Josué, hijo de Nun, un
hombre en el que está el Espíritu, y pon tu mano sobre él. \footnote{\textbf{27:18}
  Deut 3,21; Deut 34,9} \bibleverse{19} Ponlo delante del sacerdote
Eleazar y de toda la congregación, y encárgalo ante ellos.
\bibleverse{20} Le darás autoridad, para que toda la congregación de los
hijos de Israel obedezca. \footnote{\textbf{27:20} 2Re 2,9; 2Re 2,15}
\bibleverse{21} Se presentará ante el sacerdote Eleazar, quien
preguntará por él con el juicio del Urim ante Yahvé. A su palabra
saldrán, y a su palabra entrarán, él y todos los hijos de Israel con él,
toda la congregación.'' \footnote{\textbf{27:21} Éxod 28,30}

\bibleverse{22} Moisés hizo lo que el Señor le había ordenado. Tomó a
Josué y lo presentó ante el sacerdote Eleazar y ante toda la
congregación. \bibleverse{23} Le impuso las manos y lo comisionó, tal
como Yahvé habló por medio de Moisés.

\hypertarget{normativa-sobre-los-sacrificios-comunitarios-diarios-y-diarios}{%
\subsection{Normativa sobre los sacrificios comunitarios diarios y
diarios}\label{normativa-sobre-los-sacrificios-comunitarios-diarios-y-diarios}}

\hypertarget{section-27}{%
\section{28}\label{section-27}}

\bibleverse{1} Yahvé habló a Moisés, diciendo: \bibleverse{2} ``Ordena a
los hijos de Israel y diles: `Procuren presentar mi ofrenda, mi alimento
para mis ofrendas encendidas, como aroma agradable para mí, a su debido
tiempo'. \footnote{\textbf{28:2} Lev 21,6}

\hypertarget{el-holocausto-diario-de-la-mauxf1ana-y-de-la-tarde}{%
\subsection{El holocausto diario de la mañana y de la
tarde}\label{el-holocausto-diario-de-la-mauxf1ana-y-de-la-tarde}}

\bibleverse{3} Les dirás: `Esta es la ofrenda encendida que ofrecerás a
Yavé: corderos machos de un año sin defecto, dos al día, para un
holocausto continuo. \footnote{\textbf{28:3} Éxod 29,38-42}
\bibleverse{4} Ofrecerás un cordero por la mañana, y ofrecerás el otro
cordero al atardecer, \bibleverse{5} con la décima parte de un
efa\footnote{\textbf{28:5} 1 efa son unos 22 litros o unos 2/3 de una
  fanega} de harina fina como ofrenda, mezclada con la cuarta parte de
un hin\footnote{\textbf{28:5} Una hin es de unos 6,5 litros, por lo que
  1/4 de hin es de unos 1,6 litros o 1,7 cuartos de galón.} de aceite
batido. \footnote{\textbf{28:5} Lev 2,1} \bibleverse{6} Es un holocausto
continuo que fue ordenado en el monte Sinaí como aroma agradable, una
ofrenda hecha por fuego a Yahvé. \bibleverse{7} Su libación será la
cuarta parte de un hin por cada cordero. Derramarás una libación de
bebida fuerte a Yahvé en el lugar santo. \bibleverse{8} El otro cordero
lo ofrecerás al atardecer. Como la ofrenda de la mañana, y como su
libación, lo ofrecerás, ofrenda encendida, como aroma agradable a Yahvé.

\hypertarget{la-ofrenda-adicional-del-suxe1bado}{%
\subsection{La ofrenda adicional del
sábado}\label{la-ofrenda-adicional-del-suxe1bado}}

\bibleverse{9} ``\,`En el día de reposo, ofrecerás dos corderos machos
de un año sin defecto, y dos décimas de efa\footnote{\textbf{28:9} 1 efa
  son unos 22 litros o unos 2/3 de una fanega} de harina fina como
ofrenda mezclada con aceite, y su libación: \footnote{\textbf{28:9} Mat
  12,5} \bibleverse{10} Este es el holocausto de cada sábado, además del
holocausto continuo y su libación.

\hypertarget{el-sacrificio-adicional-en-el-duxeda-de-luna-nueva}{%
\subsection{El sacrificio adicional en el día de luna
nueva}\label{el-sacrificio-adicional-en-el-duxeda-de-luna-nueva}}

\bibleverse{11} ``\,`En los comienzos de tus meses, ofrecerás un
holocausto a Yahvé dos novillos, un carnero, siete corderos machos de un
año sin defecto, \footnote{\textbf{28:11} Núm 10,10} \bibleverse{12} y
tres décimas de un efa\footnote{\textbf{28:12} 1 efa equivale a unos 22
  litros o a 2/3 de una fanega} de harina fina para un holocausto
mezclado con aceite, para cada toro; y dos décimas de harina fina para
un holocausto mezclado con aceite, para el único carnero; \footnote{\textbf{28:12}
  Núm 28,20; Núm 28,28; Núm 15,2-13} \bibleverse{13} y una décima de
harina fina mezclada con aceite para un holocausto a cada cordero, como
ofrenda quemada de aroma agradable, ofrenda hecha por fuego a Yahvé.
\bibleverse{14} Sus libaciones serán la mitad de un hin de vino para el
toro, la tercera parte de un hin para el carnero y la cuarta parte de un
hin para el cordero. Este es el holocausto de cada mes durante todos los
meses del año. \bibleverse{15} También se ofrecerá un macho cabrío como
ofrenda por el pecado a Yahvé, además del holocausto continuo y su
libación. \footnote{\textbf{28:15} Núm 28,22}

\hypertarget{las-ofrendas-adicionales-para-los-siete-duxedas-de-la-fiesta-de-los-panes-sin-levadura}{%
\subsection{Las ofrendas adicionales para los siete días de la Fiesta de
los Panes sin
Levadura}\label{las-ofrendas-adicionales-para-los-siete-duxedas-de-la-fiesta-de-los-panes-sin-levadura}}

\bibleverse{16} ``\,`En el primer mes, el día catorce del mes, es la
Pascua de Yahvé. \footnote{\textbf{28:16} Lev 23,5-14} \bibleverse{17}
El decimoquinto día de este mes habrá una fiesta. Se comerá pan sin
levadura durante siete días. \bibleverse{18} En el primer día habrá una
santa convocación. No harás ningún trabajo regular, \footnote{\textbf{28:18}
  Núm 28,25-26} \bibleverse{19} sino que ofrecerás una ofrenda
encendida, un holocausto a Yahvé: dos novillos, un carnero y siete
corderos de un año. Serán sin defecto, \bibleverse{20} con su ofrenda de
harina, harina fina mezclada con aceite. Ofrecerás tres décimas por el
toro, y dos décimas por el carnero. \footnote{\textbf{28:20} Núm 28,12}
\bibleverse{21} Ofrecerás una décima por cada cordero de los siete
corderos; \bibleverse{22} y un macho cabrío como ofrenda por el pecado,
para hacer expiación por ti. \footnote{\textbf{28:22} Núm 28,15}
\bibleverse{23} Los ofrecerás además del holocausto de la mañana, que es
un holocausto continuo. \bibleverse{24} Así ofrecerás cada día, durante
siete días, el alimento de la ofrenda encendida, de aroma agradable para
Yahvé. Se ofrecerá además del holocausto continuo y su libación.
\bibleverse{25} El séptimo día tendréis una santa convocación. No harás
ningún trabajo regular.

\hypertarget{los-sacrificios-adicionales-en-la-fiesta-de-las-primicias}{%
\subsection{Los sacrificios adicionales en la fiesta de las
primicias}\label{los-sacrificios-adicionales-en-la-fiesta-de-las-primicias}}

\bibleverse{26} ``\,`También en el día de las primicias, cuando ofrezcas
una nueva ofrenda a Yahvé en tu fiesta de las semanas, tendrás una santa
convocación. No harás ningún trabajo regular; \footnote{\textbf{28:26}
  Lev 23,15-21} \bibleverse{27} sino que ofrecerás un holocausto como
aroma agradable a Yahvé: dos novillos, un carnero, siete corderos macho
de un año; \bibleverse{28} y su ofrenda de harina fina mezclada con
aceite, tres décimas por cada toro, dos décimas por el único carnero,
\bibleverse{29} una décima por cada cordero de los siete corderos;
\bibleverse{30} y un macho cabrío, para hacer expiación por ti.
\footnote{\textbf{28:30} Núm 28,15} \bibleverse{31} Además del
holocausto continuo y su ofrenda de comida, los ofrecerás junto con sus
libaciones. Procura que sean sin defecto.

\hypertarget{los-sacrificios-adicionales-el-duxeda-de-auxf1o-nuevo}{%
\subsection{Los sacrificios adicionales el día de Año
Nuevo}\label{los-sacrificios-adicionales-el-duxeda-de-auxf1o-nuevo}}

\hypertarget{section-28}{%
\section{29}\label{section-28}}

\bibleverse{1} ``\,`En el séptimo mes, el primer día del mes, tendréis
una santa convocación; no haréis ningún trabajo regular. Es un día de
toque de trompetas para ti. \footnote{\textbf{29:1} Lev 23,24-25}
\bibleverse{2} Ofrecerás un holocausto como aroma agradable a Yahvé: un
novillo, un carnero, siete corderos machos de un año sin defecto;
\bibleverse{3} y su ofrenda de harina fina mezclada con aceite: tres
décimas por el toro, dos décimas por el carnero, \bibleverse{4} y una
décima por cada cordero de los siete corderos; \bibleverse{5} y un macho
cabrío como ofrenda por el pecado, para hacer expiación por vosotros;
\bibleverse{6} además del holocausto de la luna nueva con su ofrenda, y
el holocausto continuo con su ofrenda, y sus libaciones, según su
ordenanza, como aroma agradable, ofrenda encendida a Yahvé.

\hypertarget{los-sacrificios-adicionales-en-el-gran-duxeda-de-la-expiaciuxf3n}{%
\subsection{Los sacrificios adicionales en el gran día de la
expiación}\label{los-sacrificios-adicionales-en-el-gran-duxeda-de-la-expiaciuxf3n}}

\bibleverse{7} ``\,`En el décimo día de este séptimo mes tendréis una
santa convocación. Afligiréis vuestras almas. No haréis ninguna clase de
trabajo; \footnote{\textbf{29:7} Lev 23,27-32} \bibleverse{8} sino que
ofreceréis a Yahvé un holocausto como aroma agradable: un novillo, un
carnero, siete corderos macho de un año, todos sin defecto;
\bibleverse{9} y su ofrenda de harina fina mezclada con aceite: tres
décimas por el toro, dos décimas por el único carnero, \bibleverse{10}
una décima por cada cordero de los siete corderos; \bibleverse{11} un
macho cabrío como ofrenda por el pecado, además de la ofrenda por el
pecado de la expiación, y el holocausto continuo, y su ofrenda de
comida, y sus libaciones. \footnote{\textbf{29:11} Lev 16,11-22}

\hypertarget{las-ofrendas-adicionales-para-los-siete-duxedas-de-la-fiesta-de-los-tabernuxe1culos}{%
\subsection{Las ofrendas adicionales para los siete días de la Fiesta de
los
Tabernáculos}\label{las-ofrendas-adicionales-para-los-siete-duxedas-de-la-fiesta-de-los-tabernuxe1culos}}

\bibleverse{12} ``\,`El decimoquinto día del séptimo mes tendrás una
santa convocación. No harás ningún trabajo regular. Celebrarás una
fiesta a Yahvé durante siete días. \footnote{\textbf{29:12} Lev 23,34-43}
\bibleverse{13} Ofrecerás un holocausto, una ofrenda encendida, de aroma
agradable a Yahvé: trece novillos, dos carneros, catorce corderos machos
de un año, todos sin defecto; \bibleverse{14} y su ofrenda de harina
fina mezclada con aceite: tres décimas por cada toro de los trece toros,
dos décimas por cada carnero de los dos carneros, \bibleverse{15} y una
décima por cada cordero de los catorce corderos; \bibleverse{16} y un
macho cabrío para la ofrenda por el pecado, además del holocausto
continuo, su ofrenda y su libación.

\bibleverse{17} ``\,`El segundo día ofrecerás doce novillos, dos
carneros y catorce corderos machos de un año sin defecto;
\bibleverse{18} y su ofrenda y su libación por los toros, por los
carneros y por los corderos, según su número, conforme a la ordenanza;
\bibleverse{19} y un macho cabrío como ofrenda por el pecado, además del
holocausto continuo, con su ofrenda y su libación.

\bibleverse{20} ``\,`Al tercer día: once toros, dos carneros, catorce
corderos machos de un año sin defecto; \bibleverse{21} y su ofrenda y su
libación por los toros, por los carneros y por los corderos, según su
número, conforme a la ordenanza; \bibleverse{22} y un macho cabrío como
ofrenda por el pecado, además del holocausto continuo, y su ofrenda y su
libación.

\bibleverse{23} ``\,`Al cuarto día diez toros, dos carneros, catorce
corderos machos de un año sin defecto; \bibleverse{24} su ofrenda y su
libación por los toros, por los carneros y por los corderos, según su
número, conforme a la ordenanza; \bibleverse{25} y un macho cabrío como
ofrenda por el pecado; además del holocausto continuo, su ofrenda y su
libación.

\bibleverse{26} ``\,`Al quinto día: nueve toros, dos carneros, catorce
corderos machos de un año sin defecto; \bibleverse{27} y su ofrenda y su
libación por los toros, por los carneros y por los corderos, según su
número, conforme a la ordenanza, \bibleverse{28} y un macho cabrío como
ofrenda por el pecado, además del holocausto continuo, y su ofrenda y su
libación.

\bibleverse{29} ``\,`En el sexto día: ocho toros, dos carneros, catorce
corderos machos de un año sin defecto; \bibleverse{30} y su ofrenda y su
libación por los toros, por los carneros y por los corderos, según su
número, conforme a la ordenanza, \bibleverse{31} y un macho cabrío como
ofrenda por el pecado; además del holocausto continuo, su ofrenda y su
libación.

\bibleverse{32} ``\,`En el séptimo día: siete toros, dos carneros,
catorce corderos machos de un año sin defecto; \bibleverse{33} y su
ofrenda y su libación por los toros, por los carneros y por los
corderos, según su número, conforme a la ordenanza, \bibleverse{34} y un
macho cabrío como ofrenda por el pecado; además del holocausto continuo,
su ofrenda y su libación.

\bibleverse{35} ``\,`El octavo día tendrás una asamblea solemne. No
harás ningún trabajo regular; \bibleverse{36} sino que ofrecerás un
holocausto, una ofrenda encendida, un aroma agradable a Yahvé: un toro,
un carnero, siete corderos machos de un año sin defecto; \bibleverse{37}
su ofrenda y su libación por el toro, por el carnero y por los corderos,
serán según su número, conforme a la ordenanza, \bibleverse{38} y un
macho cabrío como ofrenda por el pecado, además del holocausto continuo,
con su ofrenda y su libación.

\hypertarget{sentencia-final-de-las-leyes-de-vuxedctimas}{%
\subsection{Sentencia final de las leyes de
víctimas}\label{sentencia-final-de-las-leyes-de-vuxedctimas}}

\bibleverse{39} ``\,`Ofrecerás esto a Yahvé en tus fiestas establecidas
--- además de tus votos y tus ofrendas voluntarias --- para tus
holocaustos, tus ofrendas de comida, tus ofrendas de bebida y tus
ofrendas de paz'\,''.

\bibleverse{40} Moisés dijo a los hijos de Israel todo lo que Yahvé le
había ordenado a Moisés.

\hypertarget{reglamento-sobre-la-vinculaciuxf3n-o-nulidad-de-los-votos}{%
\subsection{Reglamento sobre la vinculación o nulidad de los
votos}\label{reglamento-sobre-la-vinculaciuxf3n-o-nulidad-de-los-votos}}

\hypertarget{section-29}{%
\section{30}\label{section-29}}

\bibleverse{1} Moisés habló a los jefes de las tribus de los hijos de
Israel, diciendo: ``Esto es lo que Yahvé ha ordenado. \bibleverse{2}
Cuando un hombre haga un voto a Yavé, o haga un juramento para atar su
alma con un vínculo, no deberá faltar a su palabra. Hará conforme a todo
lo que salga de su boca. \footnote{\textbf{30:2} Lev 27,2-25; Deut
  23,21; Jue 11,35; Ecl 5,4-5}

\bibleverse{3} ``Además, cuando una mujer hace un voto a Yahvé y se
compromete con una promesa, estando en la casa de su padre, en su
juventud, \bibleverse{4} y su padre oye su voto y su promesa con la que
ha vinculado su alma, y su padre no le dice nada, entonces todos sus
votos serán válidos, y toda promesa con la que haya vinculado su alma
será válida. \bibleverse{5} Pero si su padre se lo prohíbe el día que se
entere, ninguno de sus votos ni de sus promesas con los que haya ligado
su alma subsistirán. El Señor la perdonará, porque su padre se lo ha
prohibido.

\bibleverse{6} ``Si tiene marido, mientras sus votos están sobre ella, o
la imprudente expresión de sus labios con la que ha ligado su alma,
\bibleverse{7} y su marido lo oye, y no le dice nada el día que lo oye,
entonces sus votos serán firmes, y sus promesas con las que ha ligado su
alma serán firmes. \bibleverse{8} Pero si su marido se lo prohíbe el día
que lo oiga, entonces anulará su voto que está sobre ella y las palabras
imprudentes de sus labios, con las que ha ligado su alma. El Señor la
perdonará.

\bibleverse{9} ``Pero el voto de la viuda o de la divorciada, todo
aquello con lo que haya ligado su alma quedará en su contra.

\bibleverse{10} ``Si ella hizo un voto en casa de su marido o vinculó su
alma con un juramento, \bibleverse{11} y su marido lo oyó, y calló ante
ella y no la desautorizó, entonces todos sus votos serán válidos, y toda
prenda con la que vinculó su alma será válida. \bibleverse{12} Pero si
su marido los anuló el día que los oyó, entonces todo lo que haya salido
de sus labios en cuanto a sus votos, o en cuanto al vínculo de su alma,
no subsistirá. Su marido los ha anulado. El Señor la perdonará.
\bibleverse{13} Todo voto, y todo juramento vinculante para afligir el
alma, su marido puede establecerlo, o su marido puede anularlo.

\hypertarget{promulgaciuxf3n-renovada-de-los-derechos-del-marido}{%
\subsection{Promulgación renovada de los derechos del
marido}\label{promulgaciuxf3n-renovada-de-los-derechos-del-marido}}

\bibleverse{14} Pero si su esposo no le dice nada de un día para otro,
entonces él establece todos sus votos o todos sus juramentos que están
sobre ella. Los ha establecido, porque no le dijo nada el día que los
escuchó. \bibleverse{15} Pero si los anula después de haberlos
escuchado, entonces él cargará con su iniquidad.''

\bibleverse{16} Estosson los estatutos que Yahvé ordenó a Moisés, entre
un hombre y su esposa, entre un padre y su hija, estando en su juventud,
en la casa de su padre.

\hypertarget{guerra-de-venganza-de-los-israelitas-contra-los-madianitas}{%
\subsection{Guerra de venganza de los israelitas contra los
madianitas}\label{guerra-de-venganza-de-los-israelitas-contra-los-madianitas}}

\hypertarget{section-30}{%
\section{31}\label{section-30}}

\bibleverse{1} Yahvé habló a Moisés, diciendo: \bibleverse{2} ``Véngate
de los hijos de Israel contra los madianitas. Después te reunirás con tu
pueblo''. \footnote{\textbf{31:2} Núm 25,17; Núm 27,13}

\bibleverse{3} Moisés habló al pueblo diciendo: ``Arma a los hombres de
entre ustedes para la guerra, para que vayan contra Madián, para
ejecutar la venganza de Yavé contra Madián. \bibleverse{4} Enviaréis mil
de cada tribu, de todas las tribus de Israel, a la guerra.''
\bibleverse{5} Así que fueron entregados, de los miles de Israel, mil de
cada tribu, doce mil armados para la guerra. \bibleverse{6} Moisés los
envió, mil de cada tribu, a la guerra con Finees, hijo del sacerdote
Eleazar, a la guerra, con los utensilios del santuario y las trompetas
de alarma en su mano. \footnote{\textbf{31:6} Núm 25,7; Núm 10,2}
\bibleverse{7} Lucharon contra Madián, como Yahvé le ordenó a Moisés.
Mataron a todos los varones. \footnote{\textbf{31:7} Éxod 20,13}
\bibleverse{8} Mataron a los reyes de Madián con el resto de sus
muertos: Evi, Rekem, Zur, Hur y Reba, los cinco reyes de Madián. También
mataron a espada a Balaam, hijo de Beor. \footnote{\textbf{31:8} Jos
  13,21-22; Núm 22,5} \bibleverse{9} Los hijos de Israel tomaron
cautivas a las mujeres de Madián con sus hijos, y tomaron como botín
todo su ganado, todos sus rebaños y todos sus bienes. \bibleverse{10}
Quemaron todas sus ciudades en los lugares donde vivían y todos sus
campamentos. \bibleverse{11} Tomaron todos los cautivos y todo el botín,
tanto de hombres como de animales. \bibleverse{12} Llevaron a los
cautivos, con la presa y el botín, a Moisés, al sacerdote Eleazar y a la
congregación de los hijos de Israel, al campamento de los llanos de
Moab, que están junto al Jordán, en Jericó.

\hypertarget{ordenanza-sobre-la-matanza-de-todos-los-niuxf1os-varones-sobre-el-trato-de-las-reclusas-y-los-niuxf1os-y-sobre-la-limpieza-que-se-debe-realizar-antes-del-regreso}{%
\subsection{Ordenanza sobre la matanza de todos los niños varones, sobre
el trato de las reclusas y los niños y sobre la limpieza que se debe
realizar antes del
regreso}\label{ordenanza-sobre-la-matanza-de-todos-los-niuxf1os-varones-sobre-el-trato-de-las-reclusas-y-los-niuxf1os-y-sobre-la-limpieza-que-se-debe-realizar-antes-del-regreso}}

\bibleverse{13} Moisés y el sacerdote Eleazar, con todos los jefes de la
congregación, salieron a recibirlos fuera del campamento.
\bibleverse{14} Moisés se enojó con los oficiales del ejército, los
capitanes de mil y los de cien, que venían del servicio de guerra.
\bibleverse{15} Moisés les dijo: ``¿Habéis salvado a todas las mujeres
con vida? \bibleverse{16} He aquí que éstas hicieron que los hijos de
Israel, por consejo de Balaam, cometieran transgresión contra Yavé en el
asunto de Peor, y así fue la plaga en la congregación de Yavé.
\footnote{\textbf{31:16} Núm 25,1; Apoc 2,14} \bibleverse{17} Maten,
pues, a todo varón entre los pequeños, y maten a toda mujer que haya
conocido al hombre acostándose con él. \footnote{\textbf{31:17} Jue
  21,11} \bibleverse{18} Pero todas las muchachas que no hayan conocido
al hombre acostándose con él, manténganse vivas.

\bibleverse{19} ``Acampad fuera del campamento durante siete días. El
que haya matado a alguna persona, y el que haya tocado a algún muerto,
purificaos al tercer día y al séptimo, vosotros y vuestros cautivos.
\footnote{\textbf{31:19} Núm 19,11} \bibleverse{20} Purificaréis todos
los vestidos, todo lo que sea de piel, todo lo que sea de pelo de cabra
y todo lo que sea de madera.''

\bibleverse{21} El sacerdote Eleazar dijo a los hombres de guerra que
iban a la batalla: ``Este es el estatuto de la ley que Yahvé ha ordenado
a Moisés. \bibleverse{22} Sin embargo, el oro, la plata, el bronce, el
hierro, el estaño y el plomo, \bibleverse{23} todo lo que pueda resistir
el fuego, lo haréis pasar por el fuego, y quedará limpio; no obstante,
será purificado con el agua para la impureza. Todo lo que no resista el
fuego lo harás pasar por el agua. \bibleverse{24} El séptimo día lavarás
tus ropas, y quedarás limpio. Después entrarás en el campamento''.

\hypertarget{distribuciuxf3n-de-presas-vivas-humanos-y-ganado-regalo-de-navidad-de-los-luxedderes}{%
\subsection{Distribución de presas vivas (humanos y ganado); Regalo de
Navidad de los
líderes}\label{distribuciuxf3n-de-presas-vivas-humanos-y-ganado-regalo-de-navidad-de-los-luxedderes}}

\bibleverse{25} Yahvé habló a Moisés, diciendo: \bibleverse{26} ``Cuenta
el botín que fue tomado, tanto de personas como de animales, tú, y el
sacerdote Eleazar, y los jefes de familia de la congregación;
\bibleverse{27} y divide el botín en dos partes: entre los hombres
hábiles en la guerra, que salieron a la batalla, y toda la congregación.
\footnote{\textbf{31:27} Jos 22,8; 1Sam 30,24} \bibleverse{28} Levanten
un tributo a Yahvé de los hombres de guerra que salieron a la batalla:
un alma de cada quinientos; de las personas, del ganado, de los asnos y
de los rebaños. \bibleverse{29} Tómalo de la mitad de ellos y dáselo al
sacerdote Eleazar, para la ofrenda mecida de Yavé. \bibleverse{30} De la
mitad de los hijos de Israel, tomarás un alma de cada cincuenta, de las
personas, del ganado, de los asnos y de los rebaños, de todo el ganado,
y se los darás a los levitas, que cumplen con el deber del tabernáculo
de Yavé.''

\bibleverse{31} Moisés y el sacerdote Eleazar hicieron lo que Yahvé les
ordenó.

\bibleverse{32} El botín, además del botín que tomaron los hombres de
guerra, fue de seiscientas setenta y cinco mil ovejas, \bibleverse{33}
setenta y dos mil cabezas de ganado, \bibleverse{34} sesenta y un mil
asnos, \bibleverse{35} y treinta y dos mil personas en total, de las
mujeres que no habían conocido al hombre acostándose con él.
\bibleverse{36} La mitad, que era la porción de los que salían a la
guerra, era en número de trescientas treinta y siete mil quinientas
ovejas; \bibleverse{37} y el tributo de las ovejas era de seiscientas
setenta y cinco. \bibleverse{38} El ganado vacuno era de treinta y seis
mil, de los cuales el tributo de Yahvé era de setenta y dos.
\bibleverse{39} Los asnos eran treinta mil quinientos, de los cuales el
tributo del Señor era sesenta y uno. \bibleverse{40} Las personas eran
dieciséis mil, de las cuales el tributo de Yahvé era de treinta y dos
personas. \bibleverse{41} Moisés entregó el tributo, que era la ofrenda
mecida de Yavé, al sacerdote Eleazar, como Yavé se lo había ordenado a
Moisés. \bibleverse{42} De la mitad de los hijos de Israel, que Moisés
repartió entre los hombres que combatieron \bibleverse{43} (la mitad de
la congregación era de trescientas treinta y siete mil quinientas
ovejas, \bibleverse{44} treinta y seis mil cabezas de ganado,
\bibleverse{45} treinta mil quinientos asnos \bibleverse{46} y dieciséis
mil personas), \bibleverse{47} de la mitad de los hijos de Israel,
Moisés tomó un animal de cada cincuenta, tanto de hombres como de
animales, y se los dio a los levitas, que cumplían con el deber del
tabernáculo de Yahvé, como Yahvé le ordenó a Moisés.

\bibleverse{48} Se acercaron a Moisés los oficiales que estaban al
frente de los miles del ejército, los capitanes de miles y los de
cientos. \bibleverse{49} Le dijeron a Moisés: ``Tus siervos han tomado
la suma de los hombres de guerra que están bajo nuestro mando, y no
falta ni uno de nosotros. \bibleverse{50} Hemos traído la ofrenda de
Yavé, lo que cada uno encontró: adornos de oro, brazaletes, pulseras,
anillos de sello, pendientes y collares, para expiar nuestras almas ante
Yavé.''

\bibleverse{51} Moisés y el sacerdote Eleazar tomaron su oro, todas las
joyas trabajadas. \bibleverse{52} Todo el oro de la ofrenda de ola que
ofrecieron a Yavé, de los capitanes de millares y de los capitanes de
centenas, fue de dieciséis mil setecientos cincuenta siclos. \footnote{\textbf{31:52}
  Un siclo equivale a unos 10 gramos o a unas 0,35 onzas, por lo que
  16.750 siclos equivalen a unos 167,5 kilogramos o a unas 368,5 libras.}
\bibleverse{53} Los hombres de guerra habían tomado un botín, cada uno
para sí mismo. \bibleverse{54} Moisés y el sacerdote Eleazar tomaron el
oro de los capitanes de millares y de centenas, y lo llevaron a la
Tienda del Encuentro como memorial para los hijos de Israel ante Yavé.

\hypertarget{la-peticiuxf3n-de-los-rubenitas-y-gaditas-fue-rechazada-por-moisuxe9s-en-un-discurso-punitivo}{%
\subsection{La petición de los rubenitas y gaditas fue rechazada por
Moisés en un discurso
punitivo}\label{la-peticiuxf3n-de-los-rubenitas-y-gaditas-fue-rechazada-por-moisuxe9s-en-un-discurso-punitivo}}

\hypertarget{section-31}{%
\section{32}\label{section-31}}

\bibleverse{1} Los hijos de Rubén y los hijos de Gad tenían una gran
cantidad de ganado. Vieron la tierra de Jazer y la tierra de Galaad. He
aquí que el lugar era un sitio para el ganado. \bibleverse{2} Entonces
los hijos de Gad y los hijos de Rubén vinieron y hablaron a Moisés, al
sacerdote Eleazar y a los jefes de la congregación, diciendo:
\bibleverse{3} ``Atarot, Dibón, Jazer, Nimra, Hesbón, Eleale, Sebam,
Nebo y Beón, \bibleverse{4} la tierra que Yahvé hirió ante la
congregación de Israel, es tierra de ganado; y vuestros siervos tienen
ganado.'' \bibleverse{5} Ellos dijeron: ``Si hemos hallado gracia ante
tus ojos, que esta tierra sea dada a tus siervos como posesión. No nos
hagas pasar el Jordán''.

\bibleverse{6} Moisés dijo a los hijos de Gad y a los hijos de Rubén:
``¿Van a ir sus hermanos a la guerra mientras ustedes están sentados
aquí? \bibleverse{7} ¿Por qué desaniman el corazón de los hijos de
Israel para que no vayan a la tierra que el Señor les ha dado?
\bibleverse{8} Así lo hicieron sus padres cuando los envié desde Cades
Barnea a ver la tierra. \footnote{\textbf{32:8} Núm 13,1} \bibleverse{9}
Pues cuando subieron al valle de Escol y vieron la tierra, desanimaron
el corazón de los hijos de Israel para que no entraran en la tierra que
Yahvé les había dado. \bibleverse{10} Aquel día ardió la ira de Yavé, y
juró diciendo: \bibleverse{11} `Ciertamente ninguno de los hombres que
subieron de Egipto, de veinte años en adelante, verá la tierra que juré
a Abraham, a Isaac y a Jacob; porque no me han seguido del todo,
\footnote{\textbf{32:11} Núm 14,22-38; Núm 26,65} \bibleverse{12}
excepto Caleb hijo de Jefone cenezeo, y Josué hijo de Nun, porque han
seguido a Yavé del todo.' \bibleverse{13} La ira del Señor se encendió
contra Israel, y lo hizo vagar de un lado a otro del desierto durante
cuarenta años, hasta que se consumió toda la generación que había hecho
el mal a los ojos del Señor.

\bibleverse{14} ``He aquí que ustedes se han levantado en el lugar de
sus padres, un aumento de hombres pecadores, para aumentar el furor de
Yahvé contra Israel. \bibleverse{15} Porque si os apartáis de él,
volverá a dejarlos en el desierto, y destruiréis a todo este pueblo.''

\hypertarget{la-respuesta-de-los-rubenitas-y-gaditas}{%
\subsection{La respuesta de los rubenitas y
gaditas}\label{la-respuesta-de-los-rubenitas-y-gaditas}}

\bibleverse{16} Se acercaron a él y le dijeron: ``Construiremos aquí
apriscos para nuestros ganados y ciudades para nuestros pequeños;
\bibleverse{17} pero nosotros mismos estaremos listos armados para ir
delante de los hijos de Israel, hasta que los hayamos llevado a su
lugar. Nuestros pequeños vivirán en las ciudades fortificadas a causa de
los habitantes de la tierra. \bibleverse{18} No volveremos a nuestras
casas hasta que todos los hijos de Israel hayan recibido su herencia.
\bibleverse{19} Porque no heredaremos con ellos al otro lado del Jordán
y más allá, porque nuestra herencia nos ha llegado a este lado del
Jordán hacia el este.''

\hypertarget{la-promesa-de-moisuxe9s-declarando-las-condiciones-otorgando-el-pauxeds-al-este-del-jorduxe1n-a-las-tribus-suplicantes}{%
\subsection{La promesa de Moisés, declarando las condiciones; Otorgando
el País al este del Jordán a las tribus
suplicantes}\label{la-promesa-de-moisuxe9s-declarando-las-condiciones-otorgando-el-pauxeds-al-este-del-jorduxe1n-a-las-tribus-suplicantes}}

\bibleverse{20} Moisés les dijo: ``Si hacéis esto, si os armáis para ir
delante de Yavé a la guerra, \footnote{\textbf{32:20} Jos 1,13-15}
\bibleverse{21} y cada uno de vuestros hombres armados pasará el Jordán
delante de Yavé hasta que haya expulsado a sus enemigos de delante de
él, \bibleverse{22} y la tierra esté sometida delante de Yavé; entonces
después volveréis, y estaréis libres de obligaciones para con Yavé y
para con Israel. Entonces esta tierra será tu posesión ante el Señor.

\bibleverse{23} ``Pero si no lo haces, he aquí que has pecado contra
Yahvé; y ten por seguro que tu pecado te descubrirá. \bibleverse{24}
Construye ciudades para tus pequeños, y rediles para tus ovejas; y haz
lo que ha salido de tu boca.''

\bibleverse{25} Los hijos de Gad y los hijos de Rubén hablaron con
Moisés y le dijeron: ``Tus siervos harán lo que ordena mi señor.
\bibleverse{26} Nuestros pequeños, nuestras mujeres, nuestros rebaños y
todo nuestro ganado quedarán allí en las ciudades de Galaad;
\bibleverse{27} pero tus siervos pasarán, cada uno de ellos armado para
la guerra, delante de Yavé para combatir, como dice mi señor.''

\bibleverse{28} Entonces Moisés ordenó acerca de ellos al sacerdote
Eleazar y a Josué hijo de Nun, y a los jefes de familia de las tribus de
los hijos de Israel. \bibleverse{29} Moisés les dijo: ``Si los hijos de
Gad y los hijos de Rubén pasan con vosotros el Jordán, cada uno armado
para la batalla delante de Yahvé, y la tierra es sometida delante de
vosotros, les daréis la tierra de Galaad en posesión; \footnote{\textbf{32:29}
  Jos 4,12} \bibleverse{30} pero si no pasan con vosotros armados,
tendrán posesión entre vosotros en la tierra de Canaán.''

\bibleverse{31} Los hijos de Gad y los hijos de Rubén respondieron
diciendo: ``Como Yahvé ha dicho a tus siervos, así haremos.
\bibleverse{32} Pasaremos armados delante de Yavé a la tierra de Canaán,
y la posesión de nuestra herencia quedará con nosotros al otro lado del
Jordán.''

\bibleverse{33} Moisés les dio a los hijos de Gad, a los hijos de Rubén
y a la media tribu de Manasés hijo de José, el reino de Sehón, rey de
los amorreos, y el reino de Og, rey de Basán; la tierra, según sus
ciudades y sus límites, las ciudades de los alrededores. \footnote{\textbf{32:33}
  Jos 13,8-31}

\hypertarget{resumen-de-las-ciudades-reconstruidas-por-los-gaditas-y-los-rubenitas}{%
\subsection{Resumen de las ciudades reconstruidas por los gaditas y los
rubenitas}\label{resumen-de-las-ciudades-reconstruidas-por-los-gaditas-y-los-rubenitas}}

\bibleverse{34} Los hijos de Gad edificaron Dibón, Atarot, Aroer,
\bibleverse{35} Atrot-sofán, Jazer, Jogbehá, \bibleverse{36} Bet Nimra y
Bet Harán: ciudades fortificadas y rediles para las ovejas.
\bibleverse{37} Los hijos de Rubén edificaron Hesbón, Elealeh, Quiriatá,
\bibleverse{38} Nebo y Baal Meón (sus nombres fueron cambiados) y Sibma.
Dieron otros nombres a las ciudades que construyeron.

\hypertarget{los-descendientes-de-manasuxe9s-se-establecieron-en-la-ribera-oriental}{%
\subsection{Los descendientes de Manasés se establecieron en la Ribera
Oriental}\label{los-descendientes-de-manasuxe9s-se-establecieron-en-la-ribera-oriental}}

\bibleverse{39} Los hijos de Maquir, hijo de Manasés, fueron a Galaad,
la tomaron y despojaron a los amorreos que estaban en ella.
\bibleverse{40} Moisés dio Galaad a Maquir hijo de Manasés, y éste vivió
en ella. \bibleverse{41} Jair hijo de Manasés fue y tomó sus aldeas, y
las llamó Havvoth Jair. \footnote{\textbf{32:41} Deut 3,14}
\bibleverse{42} Nobah fue y tomó Kenath y sus aldeas, y la llamó Nobah,
según su propio nombre.

\hypertarget{lista-de-los-campamentos-en-los-que-pasaron-los-israelitas-durante-los-cuarenta-auxf1os-del-desierto}{%
\subsection{Lista de los campamentos en los que pasaron los israelitas
durante los cuarenta años del
desierto}\label{lista-de-los-campamentos-en-los-que-pasaron-los-israelitas-durante-los-cuarenta-auxf1os-del-desierto}}

\hypertarget{section-32}{%
\section{33}\label{section-32}}

\bibleverse{1} Estos son los viajes de los hijos de Israel, cuando
salieron de la tierra de Egipto con sus ejércitos bajo la mano de Moisés
y Aarón. \bibleverse{2} Moisés escribió los puntos de partida de sus
viajes por mandato de Yahvé. Estos son sus viajes según sus puntos de
partida. \bibleverse{3} Partieron de Ramsés en el primer mes, el día
quince del primer mes; al día siguiente de la Pascua, los hijos de
Israel salieron con la mano en alto a la vista de todos los egipcios,
\footnote{\textbf{33:3} Éxod 1,11; Éxod 14,8} \bibleverse{4} mientras
los egipcios enterraban a todos sus primogénitos, a los que Yahvé había
herido entre ellos. Yahvé también ejecutó juicios sobre sus dioses.
\footnote{\textbf{33:4} Éxod 12,12} \bibleverse{5} Los hijos de Israel
partieron de Ramsés y acamparon en Sucot. \footnote{\textbf{33:5} Éxod
  12,37} \bibleverse{6} Partieron de Sucot y acamparon en Etam, que está
en el límite del desierto. \footnote{\textbf{33:6} Éxod 13,20}
\bibleverse{7} Partieron de Etham y volvieron a Pihahiroth, que está
frente a Baal Zephon, y acamparon frente a Migdol. \footnote{\textbf{33:7}
  Éxod 14,2} \bibleverse{8} Partieron de delante de Hahirot y cruzaron
por el medio del mar hacia el desierto. Recorrieron tres días de camino
en el desierto de Etham, y acamparon en Mara. \footnote{\textbf{33:8}
  Éxod 14,22; Éxod 15,23} \bibleverse{9} Partieron de Mara y llegaron a
Elim. En Elim había doce fuentes de agua y setenta palmeras, y acamparon
allí. \footnote{\textbf{33:9} Éxod 15,27} \bibleverse{10} Partieron de
Elim y acamparon junto al Mar Rojo. \bibleverse{11} Partieron del Mar
Rojo y acamparon en el desierto de Sin. \footnote{\textbf{33:11} Éxod
  16,1} \bibleverse{12} Partieron del desierto de Sin y acamparon en
Dofka. \bibleverse{13} Partieron de Dofka y acamparon en Alush.
\bibleverse{14} Partieron de Alush y acamparon en Refidim, donde no
había agua para que el pueblo bebiera. \footnote{\textbf{33:14} Éxod
  17,1} \bibleverse{15} Partieron de Refidim y acamparon en el desierto
de Sinaí. \footnote{\textbf{33:15} Éxod 19,1} \bibleverse{16} Partieron
del desierto de Sinaí y acamparon en Kibroth Hattaavah. \footnote{\textbf{33:16}
  Núm 11,34} \bibleverse{17} Partieron de Kibroth Hattaavah y acamparon
en Hazeroth. \footnote{\textbf{33:17} Núm 11,35} \bibleverse{18}
Partieron de Hazeroth y acamparon en Rithmah. \footnote{\textbf{33:18}
  Núm 12,16} \bibleverse{19} Partieron de Ritma y acamparon en Rimón
Pérez. \bibleverse{20} Partieron de Rimón Pérez y acamparon en Libná.
\bibleverse{21} Partieron de Libná y acamparon en Risá. \bibleverse{22}
Partieron de Risá y acamparon en Kehelatá. \bibleverse{23} Partieron de
Kehelatá y acamparon en el monte Ovejero. \bibleverse{24} Partieron del
monte Ovejero y acamparon en Harada. \bibleverse{25} Partieron de Harada
y acamparon en Majelot. \bibleverse{26} Partieron de Majelot y acamparon
en Tahat. \bibleverse{27} Partieron de Tahat y acamparon en Taré.
\bibleverse{28} Partieron de Taré y acamparon en Mitcá. \bibleverse{29}
Partieron de Mitca y acamparon en Hasmona. \bibleverse{30} Partieron de
Hasmona y acamparon en Moserot. \bibleverse{31} Partieron de Moserot y
acamparon en Bene Jaakán. \footnote{\textbf{33:31} Deut 10,6}
\bibleverse{32} Partieron de Bene Jaakan y acamparon en Hor Haggidgad.
\bibleverse{33} Partieron de Hor Haggidgad y acamparon en Jotbatha.
\footnote{\textbf{33:33} Deut 10,7} \bibleverse{34} Partieron de Jotbata
y acamparon en Abrona. \bibleverse{35} Partieron de Abrona y acamparon
en Ezión Geber. \bibleverse{36} Partieron de Ezión Geber y acamparon en
Cades, en el desierto de Zin. \footnote{\textbf{33:36} Núm 20,1}
\bibleverse{37} Partieron de Cades y acamparon en el monte Hor, en el
límite de la tierra de Edom. \footnote{\textbf{33:37} Núm 20,22-29}
\bibleverse{38} El sacerdote Aarón subió al monte Hor por orden de Yavé
y murió allí, en el cuadragésimo año después de que los hijos de Israel
salieron de la tierra de Egipto, en el quinto mes, el primer día del
mes. \bibleverse{39} Aarón tenía ciento veintitrés años cuando murió en
el monte Hor. \bibleverse{40} El rey cananeo de Arad, que vivía en el
sur, en la tierra de Canaán, se enteró de la llegada de los hijos de
Israel. \footnote{\textbf{33:40} Núm 21,1} \bibleverse{41} Partieron del
monte Hor y acamparon en Salmoná. \bibleverse{42} Partieron de Salmoná y
acamparon en Punón. \bibleverse{43} Partieron de Punón y acamparon en
Obot. \footnote{\textbf{33:43} Núm 21,10} \bibleverse{44} Partieron de
Oboth y acamparon en Iye Abarim, en la frontera de Moab. \footnote{\textbf{33:44}
  Núm 21,11} \bibleverse{45} Partieron de Iyim y acamparon en Dibón Gad.
\bibleverse{46} Partieron de Dibón Gad y acamparon en Almon Diblathaim.
\bibleverse{47} Partieron de Almon Diblathaim y acamparon en los montes
de Abarim, frente a Nebo. \footnote{\textbf{33:47} Núm 21,20}
\bibleverse{48} Partieron de los montes de Abarim y acamparon en las
llanuras de Moab, junto al Jordán, en Jericó. \footnote{\textbf{33:48}
  Núm 22,1; Deut 32,49} \bibleverse{49} Acamparon junto al Jordán, desde
Bet Jeshimot hasta Abel Shittim, en las llanuras de Moab. \footnote{\textbf{33:49}
  Núm 25,1}

\hypertarget{ordenanzas-provisionales-de-dios-con-respecto-a-la-conquista-y-distribuciuxf3n-de-cisjordania-de-canauxe1n}{%
\subsection{Ordenanzas provisionales de Dios con respecto a la conquista
y distribución de Cisjordania de
Canaán}\label{ordenanzas-provisionales-de-dios-con-respecto-a-la-conquista-y-distribuciuxf3n-de-cisjordania-de-canauxe1n}}

\bibleverse{50} Yahvé habló a Moisés en las llanuras de Moab, junto al
Jordán de Jericó, diciendo: \bibleverse{51} Habla a los hijos de Israel
y diles: ``Cuando paséis el Jordán a la tierra de Canaán,
\bibleverse{52} entonces expulsaréis a todos los habitantes de la tierra
de delante de vosotros, destruiréis todos sus ídolos de piedra,
destruiréis todas sus imágenes fundidas y derribaréis todos sus lugares
altos. \bibleverse{53} Tomarás posesión de la tierra y habitarás en
ella, porque yo te he dado la tierra para que la poseas. \bibleverse{54}
Heredaréis la tierra por sorteo según vuestras familias; a los grupos
más numerosos les darás una herencia mayor, y a los más pequeños les
darás una herencia menor. Dondequiera que le toque la suerte a un
hombre, eso será suyo. Heredaréis según las tribus de vuestros padres.
\footnote{\textbf{33:54} Núm 26,55}

\bibleverse{55} ``Pero si no expulsas a los habitantes de la tierra de
delante de ti, los que dejes que queden de ellos serán como aguijones en
tus ojos y espinas en tus costados. Te acosarán en la tierra en la que
habitas. \footnote{\textbf{33:55} Jos 23,13} \bibleverse{56} Sucederá
que, como pensaba hacerles a ellos, así os haré a vosotros''.

\hypertarget{establecer-los-luxedmites-de-la-tierra-de-canauxe1n-que-se-tomaruxe1n}{%
\subsection{Establecer los límites de la tierra de Canaán que se
tomarán}\label{establecer-los-luxedmites-de-la-tierra-de-canauxe1n-que-se-tomaruxe1n}}

\hypertarget{section-33}{%
\section{34}\label{section-33}}

\bibleverse{1} Yahvé habló a Moisés, diciendo: \bibleverse{2} ``Manda a
los hijos de Israel y diles: `Cuando lleguéis a la tierra de Canaán
(ésta es la tierra que os corresponderá en herencia, la tierra de Canaán
según sus límites), \footnote{\textbf{34:2} Éxod 23,31} \bibleverse{3}
vuestro límite sur será desde el desierto de Zin a lo largo del lado de
Edom, y vuestro límite sur será desde el extremo del Mar Salado hacia el
este. \footnote{\textbf{34:3} Jos 15,1} \bibleverse{4} Tu frontera
girará hacia el sur de la subida de Akrabbim, y pasará por Zin; y pasará
por el sur de Cades Barnea; y de allí irá a Hazar Addar, y pasará por
Azmón. \bibleverse{5} La frontera girará desde Azmón hasta el arroyo de
Egipto, y terminará en el mar.

\bibleverse{6} ``\,`Para la frontera occidental, tendrás el gran mar y
su frontera. Esta será tu frontera occidental.

\bibleverse{7} ``\,`Esta será vuestra frontera del norte: desde el gran
mar os marcaréis el monte Hor. \bibleverse{8} Desde el monte Hor
marcaréis hasta la entrada de Hamat, y la frontera pasará por Zedad.
\bibleverse{9} Luego la frontera irá hasta Zifrón, y terminará en Hazar
Enán. Esta será vuestra frontera norte.

\bibleverse{10} ``\,`Marcarás tu frontera oriental desde Hazar Enán
hasta Sefam. \bibleverse{11} La frontera bajará desde Sefam hasta Ribla,
al lado oriental de Ain. La frontera descenderá y llegará hasta el lado
del mar de Cinneret, hacia el este. \footnote{\textbf{34:11} Luc 5,1}
\bibleverse{12} La frontera bajará hasta el Jordán y terminará en el Mar
Salado. Esta será tu tierra según sus límites alrededor''.

\bibleverse{13} Moisés ordenó a los hijos de Israel diciendo: ``Esta es
la tierra que heredaréis por sorteo, que Yahvé ha ordenado dar a las
nueve tribus y a la media tribu; \bibleverse{14} porque la tribu de los
hijos de Rubén según las casas de sus padres, la tribu de los hijos de
Gad según las casas de sus padres y la media tribu de Manasés han
recibido su herencia. \footnote{\textbf{34:14} Núm 32,33}
\bibleverse{15} Las dos tribus y la media tribu han recibido su herencia
al otro lado del Jordán, en Jericó, hacia el este, hacia la salida del
sol.''

\hypertarget{lista-de-hombres-que-se-encargaruxe1n-de-la-distribuciuxf3n-de-la-tierra}{%
\subsection{Lista de hombres que se encargarán de la distribución de la
tierra}\label{lista-de-hombres-que-se-encargaruxe1n-de-la-distribuciuxf3n-de-la-tierra}}

\bibleverse{16} Yahvé habló a Moisés diciendo: \bibleverse{17} ``Estos
son los nombres de los hombres que te repartirán la tierra en herencia
El sacerdote Eleazar y Josué, hijo de Nun. \footnote{\textbf{34:17} Jos
  14,1; Jos 21,1; Deut 1,38} \bibleverse{18} Tomarás un príncipe de cada
tribu para repartir la tierra en herencia. \bibleverse{19} Estos son los
nombres de los hombres: De la tribu de Judá, Caleb hijo de Jefone.
\footnote{\textbf{34:19} Núm 13,6; Núm 13,30} \bibleverse{20} De la
tribu de los hijos de Simeón, Shemuel hijo de Ammihud. \bibleverse{21}
De la tribu de Benjamín, Elidad hijo de Chislón. \bibleverse{22} De la
tribu de los hijos de Dan, un príncipe, Bukki hijo de Jogli.
\bibleverse{23} De los hijos de José: de la tribu de los hijos de
Manasés, un príncipe, Hanniel hijo de Efod. \bibleverse{24} De la tribu
de los hijos de Efraín, un príncipe, Kemuel hijo de Siftán.
\bibleverse{25} De la tribu de los hijos de Zabulón, un príncipe:
Elizafán, hijo de Parnac. \bibleverse{26} De la tribu de los hijos de
Isacar, un príncipe: Paltiel, hijo de Azzán. \bibleverse{27} De la tribu
de los hijos de Aser, un príncipe: Ahihud, hijo de Selomi.
\bibleverse{28} De la tribu de los hijos de Neftalí un príncipe, Pedahel
hijo de Ammihud''. \bibleverse{29} Estos son los que Yahvé mandó a
repartir la herencia a los hijos de Israel en la tierra de Canaán.

\hypertarget{regulaciones-relativas-a-las-ciudades-levitas-y-las-seis-ciudades-libres-designadas-para-asesinos}{%
\subsection{Regulaciones relativas a las ciudades levitas y las seis
ciudades libres designadas para
asesinos}\label{regulaciones-relativas-a-las-ciudades-levitas-y-las-seis-ciudades-libres-designadas-para-asesinos}}

\hypertarget{section-34}{%
\section{35}\label{section-34}}

\bibleverse{1} Yahvé habló a Moisés en las llanuras de Moab, junto al
Jordán, en Jericó, diciendo: \bibleverse{2} ``Ordena a los hijos de
Israel que den a los levitas ciudades para que las habiten de su
herencia. Darán a los levitas tierras de pastoreo para las ciudades de
los alrededores. \footnote{\textbf{35:2} Núm 18,20; Jos 21,2}
\bibleverse{3} Ellos tendrán las ciudades para habitarlas. Sus tierras
de pastoreo serán para su ganado, para sus posesiones y para todos sus
animales.

\bibleverse{4} ``Las tierras de pastoreo de las ciudades, que darás a
los levitas, serán desde el muro de la ciudad y hacia afuera mil
codos\footnote{\textbf{35:4} Un codo es la longitud desde la punta del
  dedo corazón hasta el codo del brazo de un hombre, es decir, unas 18
  pulgadas o 46 centímetros.} alrededor de ella. \bibleverse{5} Medirás
fuera de la ciudad dos mil codos por el lado este, dos mil codos por el
lado sur, dos mil codos por el lado oeste y dos mil codos por el lado
norte, quedando la ciudad en medio. Estas serán las tierras de pastoreo
de sus ciudades.

\bibleverse{6} ``Las ciudades que darás a los levitas serán las seis
ciudades de refugio que darás para que huya el homicida. Además de
ellas, darás cuarenta y dos ciudades. \footnote{\textbf{35:6} Éxod
  21,13; Deut 4,41; Deut 19,2; Deut 19,9; Jos 20,1} \bibleverse{7} Todas
las ciudades que darás a los levitas serán cuarenta y ocho ciudades
junto con sus tierras de pastoreo. \bibleverse{8} En cuanto a las
ciudades que darás de la posesión de los hijos de Israel, de los muchos
tomarás muchos, y de los pocos tomarás pocos. Cada uno, según su
herencia, dará parte de sus ciudades a los levitas''. \footnote{\textbf{35:8}
  Núm 26,54} \bibleverse{9} Yahvé habló a Moisés, diciendo:
\bibleverse{10} ``Habla a los hijos de Israel y diles: `Cuando paséis el
Jordán a la tierra de Canaán, \bibleverse{11} entonces os designaréis
ciudades para que os sirvan de refugio, para que huya allí el homicida
que mate a cualquier persona sin saberlo. \bibleverse{12} Las ciudades
os servirán de refugio contra el vengador, para que el homicida no muera
hasta que se presente ante la congregación para ser juzgado.
\bibleverse{13} Las ciudades que darás serán para ti seis ciudades de
refugio. \bibleverse{14} Darás tres ciudades al otro lado del Jordán, y
darás tres ciudades en la tierra de Canaán. Serán ciudades de refugio.
\bibleverse{15} Estas seis ciudades serán refugio para los hijos de
Israel, para el extranjero y para el forastero que viva entre ellos,
para que todo el que mate a alguien sin querer huya allí.

\hypertarget{el-castigo-del-asesino}{%
\subsection{El castigo del asesino}\label{el-castigo-del-asesino}}

\bibleverse{16} ``\,`Pero si lo golpeó con un instrumento de hierro, de
modo que murió, es un asesino. El asesino será condenado a muerte.
\bibleverse{17} Si lo golpeó con una piedra en la mano, con la que un
hombre puede morir, y murió, es un asesino. El homicida será condenado a
muerte. \bibleverse{18} O si lo golpea con un arma de madera en la mano,
con la que pueda morir un hombre, y muere, es un asesino. El asesino
morirá. \bibleverse{19} El vengador de la sangre dará muerte al asesino.
Cuando lo encuentre, lo matará. \bibleverse{20} Si lo empujó por odio, o
le arrojó algo mientras estaba al acecho, de modo que murió,
\bibleverse{21} o en hostilidad lo golpeó con su mano, de modo que
murió, el que lo golpeó ciertamente morirá. Es un asesino. El vengador
de la sangre dará muerte al asesino cuando lo encuentre.

\bibleverse{22} ``\,`Pero si lo empuja repentinamente sin hostilidad, o
arroja sobre él cualquier cosa sin estar al acecho, \bibleverse{23} o
con cualquier piedra, con la que pueda morir un hombre, sin verlo, y la
arroja sobre él de modo que muera, y no era su enemigo ni buscaba su
daño, \bibleverse{24} entonces la congregación juzgará entre el agresor
y el vengador de la sangre de acuerdo con estas ordenanzas.
\bibleverse{25} La congregación librará al homicida de la mano del
vengador de la sangre, y la congregación lo devolverá a su ciudad de
refugio, donde había huido. Allí habitará hasta la muerte del sumo
sacerdote, que fue ungido con el óleo santo. \footnote{\textbf{35:25}
  Lev 21,10}

\bibleverse{26} ``\,`Pero si el homicida sale en algún momento de la
frontera de su ciudad de refugio donde huye, \bibleverse{27} y el
vengador de la sangre lo encuentra fuera de la frontera de su ciudad de
refugio, y el vengador de la sangre mata al homicida, éste no será
culpable de sangre, \bibleverse{28} porque debería haber permanecido en
su ciudad de refugio hasta la muerte del sumo sacerdote. Pero después de
la muerte del sumo sacerdote, el homicida regresará a la tierra de su
posesión.

\bibleverse{29} ``\,`Estas cosas serán para vosotros un estatuto y una
ordenanza a lo largo de vuestras generaciones en todas vuestras moradas.

\bibleverse{30} ``\,`Cualquiera que mate a una persona, el asesino será
asesinado con base en el testimonio de los testigos; pero un solo
testigo no declarará contra una persona para que muera. \footnote{\textbf{35:30}
  Deut 17,6; Deut 19,15}

\bibleverse{31} ``\,`Además, no aceptarás ningún rescate por la vida de
un asesino que sea culpable de muerte. Será condenado a muerte.

\bibleverse{32} ``\,`No tomarás rescate por el que haya huido a su
ciudad de refugio, para que vuelva a habitar en la tierra antes de la
muerte del sacerdote.

\bibleverse{33} ``\,`Así no contaminarás la tierra donde vives; porque
la sangre contamina la tierra. No se puede hacer expiación de la tierra
por la sangre que se derrama en ella, sino por la sangre del que la
derramó. \footnote{\textbf{35:33} Gén 9,6} \bibleverse{34} No
contaminarás la tierra que habitas, donde yo habito; porque yo, Yahvé,
habito en medio de los hijos de Israel.'\,'' \footnote{\textbf{35:34}
  Éxod 29,45}

\hypertarget{apuxe9ndice-a-la-ley-de-reliquias}{%
\subsection{Apéndice a la ley de
reliquias}\label{apuxe9ndice-a-la-ley-de-reliquias}}

\hypertarget{section-35}{%
\section{36}\label{section-35}}

\bibleverse{1} Los jefes de familia de los hijos de Galaad, hijo de
Maquir, hijo de Manasés, de las familias de los hijos de José, se
acercaron y hablaron ante Moisés y ante los príncipes, los jefes de
familia de los hijos de Israel. \bibleverse{2} Ellos dijeron: ``Yahvé
ordenó a mi señor que diera la tierra en herencia por sorteo a los hijos
de Israel. Mi señor recibió la orden de Yahvé de dar la herencia de
nuestro hermano Zelofehad a sus hijas. \footnote{\textbf{36:2} Núm
  26,55; Núm 27,6-7} \bibleverse{3} Si ellas se casan con alguno de los
hijos de las otras tribus de los hijos de Israel, su herencia se quitará
de la herencia de nuestros padres y se añadirá a la herencia de la tribu
a la que pertenezcan. Así será quitada de la suerte de nuestra herencia.
\bibleverse{4} Cuando llegue el jubileo de los hijos de Israel, su
herencia se añadirá a la de la tribu a la que pertenezcan. Así que su
herencia será quitada de la herencia de la tribu de nuestros padres''.
\footnote{\textbf{36:4} Lev 25,10-13}

\hypertarget{la-nueva-regulaciuxf3n-de-aplicaciuxf3n-general-sobre-el-matrimonio-de-reliquias}{%
\subsection{La nueva regulación de aplicación general sobre el
matrimonio de
reliquias}\label{la-nueva-regulaciuxf3n-de-aplicaciuxf3n-general-sobre-el-matrimonio-de-reliquias}}

\bibleverse{5} Moisés ordenó a los hijos de Israel según la palabra de
Yahvé, diciendo: ``La tribu de los hijos de José dice lo que es justo.
\bibleverse{6} Esto es lo que Yahvé manda con respecto a las hijas de
Zelofehad, diciendo: ``Que se casen con quien mejor les parezca, sólo
que se casarán en la familia de la tribu de su padre. \bibleverse{7}
Así, ninguna herencia de los hijos de Israel pasará de una tribu a otra,
pues todos los hijos de Israel conservarán la herencia de la tribu de
sus padres. \bibleverse{8} Toda hija que posea una herencia en cualquier
tribu de los hijos de Israel será esposa de uno de la familia de la
tribu de su padre, para que los hijos de Israel posean cada uno la
herencia de sus padres. \bibleverse{9} Así, ninguna herencia pasará de
una tribu a otra, pues las tribus de los hijos de Israel conservarán
cada una su propia herencia''.

\bibleverse{10} Las hijas de Zelofehad hicieron lo que Yahvé mandó a
Moisés: \bibleverse{11} porque Mahá, Tirsa, Hogá, Milca y Noé, las hijas
de Zelofehad, se casaron con los hijos de los hermanos de su padre.
\footnote{\textbf{36:11} Núm 26,33} \bibleverse{12} Se casaron con las
familias de los hijos de Manasés, hijo de José. Su herencia permaneció
en la tribu de la familia de su padre.

\bibleverse{13} Estos son los mandamientos y las ordenanzas que Yahvé
ordenó por medio de Moisés a los hijos de Israel en las llanuras de
Moab, junto al Jordán, en Jericó.
