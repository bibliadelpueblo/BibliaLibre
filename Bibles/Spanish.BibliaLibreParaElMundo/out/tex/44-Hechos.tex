\hypertarget{las-ordenanzas-finales-de-jesuxfas-y-su-promesa-a-los-discuxedpulos-ascensiuxf3n}{%
\subsection{Las ordenanzas finales de Jesús y su promesa a los
discípulos;
Ascensión}\label{las-ordenanzas-finales-de-jesuxfas-y-su-promesa-a-los-discuxedpulos-ascensiuxf3n}}

\hypertarget{section}{%
\section{1}\label{section}}

\bibleverse{1} El primer libro que escribí, Teófilo, trataba de todo lo
que Jesús empezó a hacer y a enseñar, \footnote{\textbf{1:1} Luc 1,3}
\bibleverse{2} hasta el día en que fue recibido arriba, después de haber
dado el mandato por medio del Espíritu Santo a los apóstoles que había
elegido. \footnote{\textbf{1:2} Mat 28,19-20} \bibleverse{3} A éstos
también se mostró vivo después de haber padecido, con muchas pruebas,
apareciéndose a ellos durante cuarenta días y hablando del Reino de
Dios. \bibleverse{4} Estando reunido con ellos, les ordenó: ``No os
vayáis de Jerusalén, sino esperad la promesa del Padre, que habéis oído
de mí. \footnote{\textbf{1:4} Juan 15,26; Luc 24,49} \bibleverse{5}
Porque Juan ciertamente bautizó en agua, pero vosotros seréis bautizados
en el Espíritu Santo dentro de no muchos días.'' \footnote{\textbf{1:5}
  Mat 3,11}

\bibleverse{6} Por eso, cuando se reunieron, le preguntaron: ``Señor,
¿restauras ahora el reino a Israel?''. \footnote{\textbf{1:6} Luc 19,11;
  Luc 24,21}

\bibleverse{7} Les dijo: ``No os corresponde a vosotros conocer los
tiempos o las épocas que el Padre ha fijado con su propia autoridad.
\footnote{\textbf{1:7} Mat 24,36} \bibleverse{8} Pero recibiréis poder
cuando el Espíritu Santo haya venido sobre vosotros. Seréis testigos de
mí en Jerusalén, en toda Judea y Samaria, y hasta los confines de la
tierra.'' \footnote{\textbf{1:8} Hech 8,11; Luc 24,48}

\bibleverse{9} Cuando dijo estas cosas, mientras ellos miraban, fue
alzado, y una nube lo recibió fuera de su vista. \footnote{\textbf{1:9}
  Mar 16,19; Luc 24,51} \bibleverse{10} Mientras ellos miraban fijamente
al cielo mientras él se iba, he aquí que\footnote{\textbf{1:10} ``He
  aquí'', de ``\greek{ἰδοὺ}'', significa mirar, fijarse, observar, ver o
  contemplar. Se utiliza a menudo como interjección.} se pusieron junto
a ellos dos hombres vestidos de blanco, \footnote{\textbf{1:10} Luc 24,4}
\bibleverse{11} que también dijeron: ``Hombres de Galilea, ¿por qué
estáis mirando al cielo? Este Jesús, que ha sido recibido por vosotros
en el cielo, volverá de la misma manera que le habéis visto subir al
cielo.'' \footnote{\textbf{1:11} Luc 21,27}

\bibleverse{12} Luego volvieron a Jerusalén desde el monte llamado del
Olivar, que está cerca de Jerusalén, a un día de camino. \footnote{\textbf{1:12}
  Luc 24,50; Luc 24,52-53} \bibleverse{13} Cuando llegaron, subieron al
aposento alto donde se alojaban, es decir, Pedro, Juan, Santiago,
Andrés, Felipe, Tomás, Bartolomé, Mateo, Santiago hijo de Alfeo, Simón
el Zelote y Judas hijo de Santiago. \footnote{\textbf{1:13} Luc 6,13-16}
\bibleverse{14} Todos ellos perseveraban unánimemente en la oración y la
súplica, junto con las mujeres y María, la madre de Jesús, y con sus
hermanos. \footnote{\textbf{1:14} Juan 7,3}

\hypertarget{reemplazo-de-un-apuxf3stol-matuxedas-en-lugar-del-traidor-judas-iscariote}{%
\subsection{Reemplazo de un apóstol (Matías) en lugar del traidor Judas
Iscariote}\label{reemplazo-de-un-apuxf3stol-matuxedas-en-lugar-del-traidor-judas-iscariote}}

\bibleverse{15} En estos días, Pedro se levantó en medio de los
discípulos (y el número de nombres era como de ciento veinte), y dijo:
\footnote{\textbf{1:15} Juan 21,15-19} \bibleverse{16} ``Hermanos, era
necesario que se cumpliera esta Escritura que el Espíritu Santo habló
antes por boca de David acerca de Judas, que era guía de los que
prendieron a Jesús. \footnote{\textbf{1:16} Sal 41,9} \bibleverse{17}
Porque fue contado con nosotros, y recibió su parte en este ministerio.
\bibleverse{18} Este hombre obtuvo un campo con la recompensa de su
maldad; y cayendo de cabeza, su cuerpo se reventó y todos sus intestinos
brotaron. \footnote{\textbf{1:18} Mat 27,3-10} \bibleverse{19} Todos los
que vivían en Jerusalén supieron que aquel campo se llamaba en su lengua
``Akeldama'', es decir, ``El campo de la sangre''. \bibleverse{20}
Porque está escrito en el libro de los Salmos`Que su morada sea
desolada'. Que nadie habite en ella''. \footnote{\textbf{1:20} Salmo
  69:25} y,`Que otro tome su cargo'. \footnote{\textbf{1:20} Salmo 109:8}

\bibleverse{21} ``De los hombres, pues, que nos han acompañado todo el
tiempo que el Señor Jesús entró y salió de entre nosotros, \footnote{\textbf{1:21}
  Juan 15,27} \bibleverse{22} comenzando por el bautismo de Juan hasta
el día en que fue recibido de entre nosotros, de éstos uno debe ser
testigo con nosotros de su resurrección.''

\bibleverse{23} Ellos propusieron a dos: José, llamado Barsabas, que
también se llamaba Justo, y Matías. \bibleverse{24} Ellos oraron y
dijeron: ``Tú, Señor, que conoces el corazón de todos los hombres,
muestra a cuál de estos dos has elegido \footnote{\textbf{1:24} Hech 6,6}
\bibleverse{25} para que tome parte en este ministerio y apostolado del
que Judas se apartó, para que vaya a su propio lugar.'' \bibleverse{26}
Lo echaron a suertes, y la suerte recayó en Matías; y fue contado con
los once apóstoles. \footnote{\textbf{1:26} Prov 16,33}

\hypertarget{el-milagro-de-pentecostuxe9s-el-derramamiento-del-espuxedritu-santo-y-su-tremendo-testimonio-de-las-grandes-obras-de-dios}{%
\subsection{El milagro de Pentecostés: el derramamiento del Espíritu
Santo y su tremendo testimonio de las grandes obras de
Dios}\label{el-milagro-de-pentecostuxe9s-el-derramamiento-del-espuxedritu-santo-y-su-tremendo-testimonio-de-las-grandes-obras-de-dios}}

\hypertarget{section-1}{%
\section{2}\label{section-1}}

\bibleverse{1} Llegado el día de Pentecostés, estaban todos reunidos en
un mismo lugar. \footnote{\textbf{2:1} Lev 23,15-21} \bibleverse{2} De
repente, vino del cielo un ruido como el de un viento impetuoso, que
llenó toda la casa donde estaban sentados. \bibleverse{3} Aparecieron
unas lenguas como de fuego que se repartieron entre ellos, y una se posó
sobre cada uno de ellos. \footnote{\textbf{2:3} Mat 3,11} \bibleverse{4}
Todos fueron llenos del Espíritu Santo y comenzaron a hablar con otras
lenguas, según el Espíritu les daba la capacidad de hablar. \footnote{\textbf{2:4}
  Hech 4,31; Hech 10,44-46}

\bibleverse{5} En Jerusalén vivían judíos, hombres devotos, de todas las
naciones bajo el cielo. \footnote{\textbf{2:5} Hech 13,26}
\bibleverse{6} Al oír este ruido, la multitud se reunió y quedó
desconcertada, porque cada uno les oía hablar en su propia lengua.
\bibleverse{7} Todos se asombraron y se maravillaron, diciéndose unos a
otros: ``Mirad, ¿no son galileos todos estos que hablan? \bibleverse{8}
¿Cómo oímos nosotros, cada uno en su propia lengua? \bibleverse{9}
Partos, medos, elamitas y gentes de Mesopotamia, de Judea, de Capadocia,
del Ponto, de Asia, \bibleverse{10} de Frigia, de Panfilia, de Egipto,
de las partes de Libia en torno a Cirene, de los visitantes de Roma,
tanto judíos como prosélitos, \bibleverse{11} cretenses y árabes: ¡les
oímos hablar en nuestras lenguas de las maravillas de Dios!''
\bibleverse{12} Todos estaban asombrados y perplejos, diciéndose unos a
otros: ``¿Qué significa esto?'' \bibleverse{13} Otros, burlándose,
decían: ``Están llenos de vino nuevo''.

\hypertarget{explicaciuxf3n-del-milagro-de-pentecostuxe9s-como-el-cumplimiento-de-la-antigua-palabra-profuxe9tica-de-joel}{%
\subsection{Explicación del milagro de Pentecostés como el cumplimiento
de la antigua palabra profética de
Joel}\label{explicaciuxf3n-del-milagro-de-pentecostuxe9s-como-el-cumplimiento-de-la-antigua-palabra-profuxe9tica-de-joel}}

\bibleverse{14} Pero Pedro, poniéndose en pie con los once, alzó la voz
y les dijo: ``Hombres de Judea y todos los que habitáis en Jerusalén,
sabed esto y escuchad mis palabras. \bibleverse{15} Porque éstos no
están borrachos, como suponéis, ya que sólo es la tercera hora del día.
\footnote{\textbf{2:15} alrededor de las 09:00 h.} \bibleverse{16} Pero
esto es lo que se ha dicho por medio del profeta Joel: \bibleverse{17}
`Será en los últimos días, dice Dios, que derramaré mi Espíritu sobre
toda la carne. Tus hijos y tus hijas profetizarán. Sus jóvenes verán
visiones. Tus viejos soñarán sueños. \bibleverse{18} Sí, y sobre mis
siervos y mis siervas en aquellos días, Derramaré mi Espíritu, y ellos
profetizarán. \bibleverse{19} Mostraré maravillas en el cielo, y señales
en la tierra de abajo: sangre, y fuego, y oleadas de humo.
\bibleverse{20} El sol se convertirá en oscuridad, y la luna en sangre,
antes de que llegue el gran y glorioso día del Señor. \bibleverse{21} El
que invoque el nombre del Señor se salvará''. \footnote{\textbf{2:21}
  Joel 2:28-32}

\hypertarget{jesuxfas-crucificado-resucitado-y-exaltado-por-dios-tiene-las-dos-palabras-de-david}{%
\subsection{Jesús, crucificado, resucitado y exaltado por Dios, tiene
las dos palabras de
David}\label{jesuxfas-crucificado-resucitado-y-exaltado-por-dios-tiene-las-dos-palabras-de-david}}

\bibleverse{22} ``¡Hombres de Israel, escuchad estas palabras! Jesús de
Nazaret, un hombre aprobado por Dios para vosotros por las obras
poderosas y los prodigios y señales que Dios hizo por él entre vosotros,
como vosotros mismos sabéis, \bibleverse{23} a quien, entregado por el
consejo determinado y la previsión de Dios, habéis tomado por la mano de
hombres sin ley, crucificado y matado; \footnote{\textbf{2:23} Hech 4,28}
\bibleverse{24} a quien Dios resucitó, habiéndolo librado de la agonía
de la muerte, porque no era posible que fuera retenido por ella.
\bibleverse{25} Porque David dice acerca de él,`Vi al Señor siempre
delante de mi cara, porque él está a mi derecha, para que no sea
conmovido. \bibleverse{26} Por eso mi corazón se alegró y mi lengua se
regocijó. Además, mi carne también habitará en la esperanza,
\bibleverse{27} porque no dejarás mi alma en el Hades, \footnote{\textbf{2:27}
  o, Infierno} ni permitirá que su Santo vea la decadencia.
\bibleverse{28} Me diste a conocer los caminos de la vida. Me llenarás
de alegría con tu presencia.'\footnote{\textbf{2:28} Salmo 16:8-11}

\bibleverse{29} ``Hermanos, puedo decirles libremente del patriarca
David, que murió y fue sepultado, y su tumba está con nosotros hasta el
día de hoy. \footnote{\textbf{2:29} 1Re 2,10} \bibleverse{30} Por eso,
siendo profeta y sabiendo que Dios le había jurado que del fruto de su
cuerpo, según la carne, resucitaría al Cristo \footnote{\textbf{2:30}
  ``Cristo'' significa ``Ungido''.} para que se sentara en su trono,
\footnote{\textbf{2:30} Sal 89,3-4; 2Sam 7,12-13} \bibleverse{31}
previendo esto, habló de la resurrección del Cristo, que su alma no
quedó en el Hades \footnote{\textbf{2:31} o, Infierno} y su carne no vio
la decadencia. \bibleverse{32} A este Jesús lo resucitó Dios, de lo cual
todos somos testigos. \bibleverse{33} Siendo, pues, exaltado por la
diestra de Dios, y habiendo recibido del Padre la promesa del Espíritu
Santo, ha derramado esto que ahora veis y oís. \footnote{\textbf{2:33}
  Juan 15,26} \bibleverse{34} Porque David no subió a los cielos, sino
que él mismo dice, El Señor dijo a mi Señor: ``Siéntate a mi derecha
\bibleverse{35} hasta que haga de tus enemigos un escabel para tus
pies''. \footnote{\textbf{2:35} Salmo 110:1}

\bibleverse{36} ``Sepa, pues, ciertamente toda la casa de Israel que
Dios le ha hecho Señor y Cristo, a este Jesús a quien vosotros
crucificasteis.'' \footnote{\textbf{2:36} Hech 5,31}

\hypertarget{efecto-del-habla-primer-ministerio-pastoral-de-pedro-fundaciuxf3n-de-la-primera-iglesia}{%
\subsection{Efecto del habla; primer ministerio pastoral de Pedro;
Fundación de la primera
iglesia}\label{efecto-del-habla-primer-ministerio-pastoral-de-pedro-fundaciuxf3n-de-la-primera-iglesia}}

\bibleverse{37} Al oír esto, se les heló el corazón y dijeron a Pedro y
a los demás apóstoles: ``Hermanos, ¿qué haremos?'' \footnote{\textbf{2:37}
  Hech 16,30; Luc 3,10}

\bibleverse{38} Pedro les dijo: ``Arrepentíos y bautícese cada uno de
vosotros en el nombre de Jesucristo para el perdón de los pecados, y
recibiréis el don del Espíritu Santo. \footnote{\textbf{2:38} Hech
  3,17-19; Luc 24,47} \bibleverse{39} Porque la promesa es para vosotros
y para vuestros hijos, y para todos los que están lejos, para todos los
que el Señor nuestro Dios llame a sí.'' \footnote{\textbf{2:39} Jl 2,32}
\bibleverse{40} Con muchas otras palabras les testificó y exhortó,
diciendo: ``¡Salvaos de esta generación torcida!'' \footnote{\textbf{2:40}
  Mat 17,17; Fil 2,15}

\bibleverse{41} Entonces los que recibieron con gusto su palabra se
bautizaron. Aquel día se añadieron unas tres mil almas.

\hypertarget{la-vida-de-los-creyentes-en-la-primera-iglesia}{%
\subsection{La vida de los creyentes en la primera
iglesia}\label{la-vida-de-los-creyentes-en-la-primera-iglesia}}

\bibleverse{42} Continuaban con la enseñanza de los apóstoles, la
comunión, el partimiento del pan y la oración. \footnote{\textbf{2:42}
  Hech 20,7} \bibleverse{43} El temor se apoderó de todas las personas,
y se hicieron muchos prodigios y señales por medio de los apóstoles.
\bibleverse{44} Todos los que creían estaban juntos y tenían todo en
común. \footnote{\textbf{2:44} Hech 4,32-35} \bibleverse{45} Vendían sus
posesiones y bienes, y los distribuían a todos, según la necesidad de
cada uno. \bibleverse{46} De día en día, permaneciendo unánimes en el
templo y partiendo el pan en casa, tomaban el alimento con alegría y
sencillez de corazón, \bibleverse{47} alabando a Dios y gozando del
favor de todo el pueblo. El Señor añadía cada día a la asamblea a los
que se salvaban. \footnote{\textbf{2:47} Hech 4,4; Hech 5,14; Hech 6,7;
  Hech 11,21; Hech 14,1}

\hypertarget{pedro-y-juan-curan-a-un-cojo-de-nacimiento}{%
\subsection{Pedro y Juan curan a un cojo de
nacimiento}\label{pedro-y-juan-curan-a-un-cojo-de-nacimiento}}

\hypertarget{section-2}{%
\section{3}\label{section-2}}

\bibleverse{1} Pedro y Juan subían al templo a la hora de la oración, la
hora novena. \footnote{\textbf{3:1} 15:00 h.} \bibleverse{2} Llevaban a
un hombre cojo desde el vientre de su madre, al que ponían cada día a la
puerta del templo que se llama Hermoso, para pedir limosna para los
necesitados de los que entraban en el templo. \bibleverse{3} Viendo a
Pedro y a Juan a punto de entrar en el templo, les pidió limosna.
\bibleverse{4} Pedro, fijando sus ojos en él, con Juan, le dijo:
``Míranos''. \bibleverse{5} Él les escuchó, esperando recibir algo de
ellos. \bibleverse{6} Pero Pedro dijo: ``No tengo plata ni oro, pero lo
que tengo, eso te doy. En nombre de Jesucristo de Nazaret, levántate y
anda''. \bibleverse{7} Lo tomó de la mano derecha y lo levantó. Al
instante, sus pies y los huesos de sus tobillos cobraron fuerza.
\bibleverse{8} Se levantó de un salto y comenzó a caminar. Entró con
ellos en el templo, caminando, saltando y alabando a Dios.
\bibleverse{9} Todo el pueblo lo vio caminar y alabar a Dios.
\bibleverse{10} \,Le reconocieron, que era él quien solía sentarse a
pedir limosna a la puerta del templo, la Hermosa.. Se llenaron de
asombro y admiración por lo que le había sucedido. \bibleverse{11}
Mientras el cojo que había sido curado se agarraba a Pedro y a Juan,
todo el pueblo corría junto a ellos en el pórtico que se llama de
Salomón, muy maravillado. \footnote{\textbf{3:11} Hech 5,12; Juan 10,23}

\hypertarget{discurso-en-el-templo-sermuxf3n-penitencial-de-pedro-despuuxe9s-de-sanar-al-cojo}{%
\subsection{Discurso en el templo, sermón penitencial de Pedro después
de sanar al
cojo}\label{discurso-en-el-templo-sermuxf3n-penitencial-de-pedro-despuuxe9s-de-sanar-al-cojo}}

\bibleverse{12} Al verlo, Pedro respondió al pueblo: ``Hombres de
Israel, ¿por qué os maravilláis de este hombre? ¿Por qué fijáis vuestros
ojos en nosotros, como si por nuestro propio poder o piedad le
hubiéramos hecho caminar? \bibleverse{13} El Dios de Abraham, de Isaac y
de Jacob, el Dios de nuestros padres, ha glorificado a su Siervo Jesús,
a quien vosotros entregasteis y negasteis en presencia de Pilato, cuando
éste había decidido liberarlo. \bibleverse{14} Pero vosotros negasteis
al Santo y Justo y pedisteis que se os concediera un homicida,
\footnote{\textbf{3:14} Mat 27,20-21} \bibleverse{15} y matasteis al
Príncipe de la vida, a quien Dios resucitó de entre los muertos, de lo
cual somos testigos. \bibleverse{16} Por la fe en su nombre, su nombre
ha hecho fuerte a este hombre, al que veis y conocéis. Sí, la fe que es
por él le ha dado esta perfecta sanidad en presencia de todos vosotros.

\bibleverse{17} ``Ahora bien, hermanos, sé que lo hicisteis por
ignorancia, como también lo hicieron vuestros gobernantes. \footnote{\textbf{3:17}
  Luc 23,34} \bibleverse{18} Pero las cosas que Dios anunció por boca de
todos sus profetas, que Cristo había de padecer, las cumplió así.
\footnote{\textbf{3:18} Luc 24,44}

\bibleverse{19} ``Arrepiéntanse, pues, y vuélvanse, para que sean
borrados sus pecados, a fin de que vengan tiempos de refrigerio de la
presencia del Señor, \footnote{\textbf{3:19} Hech 2,38} \bibleverse{20}
y para que él envíe a Cristo Jesús, que fue ordenado para ustedes antes,
\bibleverse{21} a quien el cielo debe recibir hasta los tiempos de la
restauración de todas las cosas, de la que Dios habló hace mucho tiempo
por boca de sus santos profetas. \bibleverse{22} En efecto, Moisés dijo
a los padres: ``El Señor Dios os suscitará un profeta de entre vuestros
hermanos, como yo. Le escucharéis en todo lo que os diga.
\bibleverse{23} Será que toda persona que no escuche a ese profeta será
totalmente destruida de entre el pueblo.' \footnote{\textbf{3:23}
  Deuteronomio 18:15,18-19} \bibleverse{24} Sí, y todos los profetas,
desde Samuel y los que le siguieron, todos los que han hablado, también
contaron estos días. \footnote{\textbf{3:24} 2Sam 7,12-16}
\bibleverse{25} Vosotros sois los hijos de los profetas y de la alianza
que Dios hizo con nuestros padres, diciendo a Abraham: `Todas las
familias de la tierra serán bendecidas por tu descendencia'. \footnote{\textbf{3:25}
  o, semilla} \footnote{\textbf{3:25} Génesis 22:18; 26:4}
\bibleverse{26} Dios, habiendo suscitado a su siervo Jesús, os lo envió
primero para bendeciros, apartando a cada uno de vosotros de vuestra
maldad.'' \footnote{\textbf{3:26} Hech 13,46}

\hypertarget{pedro-y-juan-en-la-cuxe1rcel-y-ante-el-concilio}{%
\subsection{Pedro y Juan en la cárcel y ante el
concilio}\label{pedro-y-juan-en-la-cuxe1rcel-y-ante-el-concilio}}

\hypertarget{section-3}{%
\section{4}\label{section-3}}

\bibleverse{1} Mientras hablaban al pueblo, los sacerdotes, el jefe del
templo y los saduceos se acercaron a ellos, \footnote{\textbf{4:1} Luc
  22,4; Luc 22,52} \bibleverse{2} molestos porque enseñaban al pueblo y
proclamaban en Jesús la resurrección de entre los muertos. \footnote{\textbf{4:2}
  Hech 23,8} \bibleverse{3} Les echaron mano y los pusieron en custodia
hasta el día siguiente, pues ya era de noche. \bibleverse{4} Pero muchos
de los que oyeron la palabra creyeron, y el número de los hombres llegó
a ser como cinco mil. \footnote{\textbf{4:4} Hech 2,47}

\bibleverse{5} Por la mañana, se reunieron en Jerusalén sus jefes, los
ancianos y los escribas. \bibleverse{6} El sumo sacerdote Anás estaba
allí, con Caifás, Juan, Alejandro y todos los parientes del sumo
sacerdote. \footnote{\textbf{4:6} Luc 3,1} \bibleverse{7} Cuando
pusieron a Pedro y a Juan en medio de ellos, preguntaron: ``¿Con qué
poder o en qué nombre habéis hecho esto?'' \footnote{\textbf{4:7} Mat
  21,33}

\bibleverse{8} Entonces Pedro, lleno del Espíritu Santo, les dijo:
``Señores del pueblo y ancianos de Israel, \footnote{\textbf{4:8} Mat
  10,19-20} \bibleverse{9} si hoy somos examinados acerca de una buena
obra hecha a un lisiado, por qué medio ha sido curado este hombre,
\bibleverse{10} que os conste a todos vosotros y a todo el pueblo de
Israel, que en el nombre de Jesucristo de Nazaret, a quien vosotros
crucificasteis, y a quien Dios resucitó de entre los muertos, este
hombre está aquí delante de vosotros sano. \footnote{\textbf{4:10} Hech
  3,6; Hech 3,13-16} \bibleverse{11} Él es ``la piedra que vosotros, los
constructores, teníais por inútil, pero que se ha convertido en la
cabeza del ángulo''. \footnote{\textbf{4:11} Salmo 118:22} \footnote{\textbf{4:11}
  Mat 21,42} \bibleverse{12} En ningún otro hay salvación, pues no hay
bajo el cielo otro nombre dado a los hombres por el que debamos
salvarnos.'' \footnote{\textbf{4:12} Hech 10,43; Mat 1,21}

\bibleverse{13} Al ver la audacia de Pedro y de Juan, y al darse cuenta
de que eran hombres indoctos e ignorantes, se maravillaron. Reconocieron
que habían estado con Jesús. \bibleverse{14} Al ver que el hombre que
había sido curado estaba con ellos, no pudieron decir nada en contra.
\footnote{\textbf{4:14} Hech 3,8-9} \bibleverse{15} Pero cuando les
ordenaron que se apartaran del consejo, consultaron entre sí,
\bibleverse{16} diciendo: ``¿Qué haremos con estos hombres? Porque
ciertamente se ha hecho un notable milagro por medio de ellos, como lo
pueden ver claramente todos los que habitan en Jerusalén, y no podemos
negarlo. \footnote{\textbf{4:16} Juan 11,47} \bibleverse{17} Pero para
que esto no se extienda más entre el pueblo, vamos a amenazarlos, para
que de ahora en adelante no hablen con nadie en este nombre.''
\bibleverse{18} Los llamaron y les ordenaron que no hablaran en absoluto
ni enseñaran en el nombre de Jesús.

\bibleverse{19} Pero Pedro y Juan les respondieron: ``Si es justo a los
ojos de Dios escucharos a vosotros antes que a Dios, juzgadlo vosotros
mismos, \footnote{\textbf{4:19} Hech 5,28-29} \bibleverse{20} porque no
podemos dejar de contar lo que hemos visto y oído.''

\bibleverse{21} Cuando los amenazaron más, los dejaron ir, sin encontrar
la manera de castigarlos, a causa del pueblo; porque todos glorificaban
a Dios por lo que se había hecho. \bibleverse{22} Pues el hombre en el
que se realizó este milagro de curación tenía más de cuarenta años.

\hypertarget{regreso-de-los-apuxf3stoles-acciuxf3n-de-gracias-y-suxfaplica-de-la-congregaciuxf3n}{%
\subsection{Regreso de los apóstoles; Acción de gracias y súplica de la
congregación}\label{regreso-de-los-apuxf3stoles-acciuxf3n-de-gracias-y-suxfaplica-de-la-congregaciuxf3n}}

\bibleverse{23} Al ser dejados en libertad, volvieron a su casa y
contaron todo lo que les habían dicho los jefes de los sacerdotes y los
ancianos. \bibleverse{24} Cuando lo oyeron, alzaron la voz a Dios de
común acuerdo y dijeron: ``Señor, tú eres Dios, que hiciste el cielo, la
tierra, el mar y todo lo que hay en ellos; \bibleverse{25} que por boca
de tu siervo David, dijiste,`Por qué se enfurecen las naciones, ¿y los
pueblos traman una cosa vana? \bibleverse{26} Los reyes de la tierra se
ponen en pie, y los gobernantes conspiran juntos, contra el Señor y
contra su Cristo\footnote{\textbf{4:26} Cristo (griego) y Mesías
  (hebreo) significan ambos el Ungido.} \footnote{\textbf{4:26} Salmo
  2:1-2} ''.

\bibleverse{27} ``Porque en verdad,\footnote{\textbf{4:27} nu añade ``en
  esta ciudad''.} tanto Herodes como Poncio Pilato, con los gentiles y
el pueblo de Israel, se han reunido contra tu santo siervo Jesús, a
quien tú ungiste, \footnote{\textbf{4:27} Luc 23,12} \bibleverse{28}
para hacer todo lo que tu mano y tu consejo predijeron que sucediera.
\footnote{\textbf{4:28} Hech 2,23} \bibleverse{29} Ahora, Señor, mira
sus amenazas, y concede a tus siervos que hablen tu palabra con toda
valentía, \footnote{\textbf{4:29} Efes 6,19} \bibleverse{30} mientras
extiendes tu mano para sanar, y que se hagan señales y prodigios por el
nombre de tu santo Siervo Jesús.''

\bibleverse{31} Cuando oraron, el lugar donde estaban reunidos se
estremeció. Todos estaban llenos del Espíritu Santo, y hablaban la
palabra de Dios con valentía.

\hypertarget{la-comunidad-de-bienes}{%
\subsection{La comunidad de bienes}\label{la-comunidad-de-bienes}}

\bibleverse{32} La multitud de los creyentes tenía un solo corazón y una
sola alma. Ninguno de ellos pretendía que algo de lo que poseía fuera
suyo, sino que tenían todo en común. \footnote{\textbf{4:32} Hech 2,44}
\bibleverse{33} Con gran poder, los apóstoles daban su testimonio de la
resurrección del Señor Jesús. Una gran gracia estaba sobre todos ellos.
\footnote{\textbf{4:33} Hech 2,47} \bibleverse{34} Porque no había entre
ellos ningún necesitado, ya que todos los que poseían tierras o casas
las vendían, y traían el producto de lo vendido, \footnote{\textbf{4:34}
  Hech 2,45} \bibleverse{35} y lo ponían a los pies de los apóstoles; y
se repartía a cada uno según su necesidad.

\bibleverse{36} Josés, a quien los apóstoles llamaban también Bernabé
(que es, interpretado, Hijo del Consolación), levita, hombre de raza
chipriota, \footnote{\textbf{4:36} Hech 11,22-26; Hech 12,25; Hech 15,2;
  Gal 2,1; Col 4,10} \bibleverse{37} que tenía un campo, lo vendió,
trajo el dinero y lo puso a los pies de los apóstoles.

\hypertarget{un-ejemplo-de-disciplina-eclesiuxe1stica-seria-ananuxedas-y-safira}{%
\subsection{Un ejemplo de disciplina eclesiástica seria: Ananías y
Safira}\label{un-ejemplo-de-disciplina-eclesiuxe1stica-seria-ananuxedas-y-safira}}

\hypertarget{section-4}{%
\section{5}\label{section-4}}

\bibleverse{1} Pero un hombre llamado Ananías, con su mujer Safira,
vendió una propiedad, \bibleverse{2} y se quedó con una parte del
precio, sabiéndolo también su mujer, y luego trajo una parte y la puso a
los pies de los apóstoles. \footnote{\textbf{5:2} Hech 4,34-37}
\bibleverse{3} Pero Pedro dijo: ``Ananías, ¿por qué Satanás ha llenado
tu corazón para mentir al Espíritu Santo y retener parte del precio de
la tierra? \bibleverse{4} Mientras te la quedaste, ¿no era tuya? Después
de venderla, ¿no estaba en tu poder? ¿Cómo es que has concebido esto en
tu corazón? No has mentido a los hombres, sino a Dios''.

\bibleverse{5} Ananías, al oír estas palabras, cayó y murió. Un gran
temor invadió a todos los que oyeron estas cosas. \bibleverse{6} Los
jóvenes se levantaron, lo envolvieron, lo sacaron y lo enterraron.
\bibleverse{7} Unas tres horas después, entró su mujer, sin saber lo que
había pasado. \bibleverse{8} Pedro le respondió: ``Dime si has vendido
la tierra por tanto''. Ella dijo: ``Sí, en tanto''.

\bibleverse{9} Pero Pedro le preguntó: ``¿Cómo es que os habéis puesto
de acuerdo para tentar al Espíritu del Señor? He aquí que los pies de
los que han enterrado a tu marido están a la puerta, y te sacarán''.

\bibleverse{10} Ella cayó inmediatamente a sus pies y murió. Los jóvenes
entraron y la encontraron muerta, la sacaron y la enterraron junto a su
marido. \bibleverse{11} Un gran temor se apoderó de toda la asamblea y
de todos los que oyeron estas cosas.

\hypertarget{milagros-especialmente-la-curaciuxf3n-de-los-enfermos-de-los-apuxf3stoles-mayor-crecimiento-de-la-comunidad}{%
\subsection{Milagros (especialmente la curación de los enfermos) de los
apóstoles; mayor crecimiento de la
comunidad}\label{milagros-especialmente-la-curaciuxf3n-de-los-enfermos-de-los-apuxf3stoles-mayor-crecimiento-de-la-comunidad}}

\bibleverse{12} Por las manos de los apóstoles se hacían muchas señales
y prodigios entre el pueblo. Todos estaban de acuerdo en el pórtico de
Salomón. \footnote{\textbf{5:12} Hech 3,11} \bibleverse{13} Ninguno de
los demás se atrevía a unirse a ellos; sin embargo, el pueblo los
honraba. \bibleverse{14} Se añadieron más creyentes al Señor, multitudes
de hombres y mujeres. \footnote{\textbf{5:14} Hech 2,47} \bibleverse{15}
Incluso sacaban a los enfermos a la calle y los ponían en catres y
colchones, para que al pasar Pedro, al menos su sombra hiciera sombra a
algunos de ellos. \footnote{\textbf{5:15} Hech 19,11-12} \bibleverse{16}
También se reunió una multitud de las ciudades de los alrededores de
Jerusalén, trayendo enfermos y atormentados por espíritus inmundos; y
todos quedaron sanados.

\hypertarget{el-arresto-liberaciuxf3n-a-travuxe9s-de-un-uxe1ngel}{%
\subsection{El arresto; Liberación a través de un
ángel}\label{el-arresto-liberaciuxf3n-a-travuxe9s-de-un-uxe1ngel}}

\bibleverse{17} Pero el sumo sacerdote se levantó, y todos los que
estaban con él (que es la secta de los saduceos), y se llenaron de celos
\footnote{\textbf{5:17} Hech 4,1; Hech 4,6} \bibleverse{18} y echaron
mano a los apóstoles, y los pusieron en custodia pública.
\bibleverse{19} Pero un ángel del Señor abrió de noche las puertas de la
cárcel, los sacó y les dijo: \footnote{\textbf{5:19} Hech 12,7}
\bibleverse{20} ``Vayan y hablen en el templo al pueblo todas las
palabras de esta vida.''

\bibleverse{21} Al oír esto, entraron en el templo hacia el amanecer y
enseñaron. Pero el sumo sacerdote y los que estaban con él vinieron y
convocaron al consejo, con todo el senado de los hijos de Israel, y
enviaron a la cárcel para que los trajeran. \bibleverse{22} Pero los
funcionarios que vinieron no los encontraron en la cárcel. Volvieron e
informaron: \bibleverse{23} ``Encontramos la cárcel cerrada y con llave,
y a los guardias de pie ante las puertas; pero cuando las abrimos, no
encontramos a nadie dentro.''

\bibleverse{24} Cuando el sumo sacerdote, el capitán del templo y los
jefes de los sacerdotes oyeron estas palabras, se quedaron muy perplejos
acerca de ellas y de lo que podría suceder. \bibleverse{25} Uno vino y
les dijo: ``He aquí, los hombres que pusisteis en la cárcel están en el
templo, de pie y enseñando al pueblo.'' \bibleverse{26} Entonces el
capitán fue con los oficiales y los trajo sin violencia, pues temían que
el pueblo los apedreara.

\hypertarget{el-valiente-testimonio-del-apuxf3stol-de-la-resurrecciuxf3n-de-cristo}{%
\subsection{El valiente testimonio del apóstol de la resurrección de
Cristo}\label{el-valiente-testimonio-del-apuxf3stol-de-la-resurrecciuxf3n-de-cristo}}

\bibleverse{27} Cuando los trajeron, los presentaron ante el consejo. El
sumo sacerdote los interrogó, \bibleverse{28} diciendo: ``¿No os hemos
ordenado estrictamente que no enseñéis en este nombre? He aquí que
habéis llenado Jerusalén con vuestras enseñanzas, y pretendéis hacer
caer la sangre de este hombre sobre nosotros.'' \footnote{\textbf{5:28}
  Hech 4,18; Mat 27,25}

\bibleverse{29} Pero Pedro y los apóstoles respondieron: ``Hay que
obedecer a Dios antes que a los hombres. \footnote{\textbf{5:29} Hech
  4,19; Dan 3,16-18} \bibleverse{30} El Dios de nuestros padres resucitó
a Jesús, a quien vosotros matasteis, colgándolo en un madero.
\footnote{\textbf{5:30} Hech 3,15} \bibleverse{31} Dios lo exaltó con su
diestra para ser Príncipe y Salvador, para dar el arrepentimiento a
Israel y la remisión de los pecados. \footnote{\textbf{5:31} Hech 2,33}
\bibleverse{32} Nosotros somos sus testigos de estas cosas; y también el
Espíritu Santo, que Dios ha dado a los que le obedecen.'' \footnote{\textbf{5:32}
  Luc 24,48; Juan 15,26-27}

\bibleverse{33} Pero ellos, al oír esto, se sintieron heridos en el
corazón, y estaban decididos a matarlos.

\hypertarget{defensa-y-asesoramiento-de-gamaliel}{%
\subsection{Defensa y asesoramiento de
Gamaliel}\label{defensa-y-asesoramiento-de-gamaliel}}

\bibleverse{34} Pero uno se levantó en el concilio, un fariseo llamado
Gamaliel, maestro de la ley, honrado por todo el pueblo, y mandó sacar a
los apóstoles por un tiempo. \footnote{\textbf{5:34} Hech 22,3}
\bibleverse{35} Les dijo: ``Hombres de Israel, tened cuidado con estos
hombres, por lo que vais a hacer. \bibleverse{36} Porque antes de estos
días se levantó Teudas, haciéndose pasar por alguien; al cual se unió un
número de hombres, como cuatrocientos. Lo mataron; y todos, los que le
obedecían, se dispersaron y quedaron en nada. \bibleverse{37} Después de
este hombre, se levantó Judas de Galilea en los días de la inscripción,
y arrastró tras sí a algunas personas. También él pereció, y todos los
que le obedecían fueron dispersados. \bibleverse{38} Ahora os digo que
os apartéis de estos hombres y los dejéis en paz. Porque si este consejo
o esta obra son de los hombres, serán derribados. \footnote{\textbf{5:38}
  Mat 15,13} \bibleverse{39} Pero si es de Dios, no podréis derribarlo,
y se os encontraría incluso luchando contra Dios.''

\bibleverse{40} Estuvieron de acuerdo con él. Llamando a los apóstoles,
los golpearon y les ordenaron que no hablaran en nombre de Jesús, y los
dejaron ir. \footnote{\textbf{5:40} Mat 10,17} \bibleverse{41} Así pues,
salieron de la presencia del consejo, alegrándose de haber sido
considerados dignos de sufrir la deshonra por el nombre de Jesús.
\footnote{\textbf{5:41} Mat 5,10-12; 1Pe 4,13}

\bibleverse{42} Cada día, en el templo y en casa, no dejaban de enseñar
y predicar a Jesús, el Cristo.

\hypertarget{separaciuxf3n-de-la-oficina-de-predicaciuxf3n-y-ayuda-a-los-pobres-elecciuxf3n-y-nombramiento-de-los-siete-cuidadores-pobres}{%
\subsection{Separación de la oficina de predicación y ayuda a los
pobres; Elección y nombramiento de los siete cuidadores
pobres}\label{separaciuxf3n-de-la-oficina-de-predicaciuxf3n-y-ayuda-a-los-pobres-elecciuxf3n-y-nombramiento-de-los-siete-cuidadores-pobres}}

\hypertarget{section-5}{%
\section{6}\label{section-5}}

\bibleverse{1} En aquellos días, cuando el número de los discípulos se
multiplicaba, surgió una queja de los helenistas\footnote{\textbf{6:1}
  Los helenistas utilizaban la lengua y la cultura griega, aunque
  también eran de origen hebreo.} contra los hebreos, porque sus viudas
eran descuidadas en el servicio diario. \footnote{\textbf{6:1} Hech 4,35}
\bibleverse{2} Los doce convocaron a la multitud de los discípulos y
dijeron: ``No conviene que dejemos la palabra de Dios y sirvamos a las
mesas. \bibleverse{3} Por eso, hermanos, elegid de entre vosotros a
siete hombres de buena reputación, llenos de Espíritu Santo y de
sabiduría, a los que podamos nombrar para que se encarguen de este
asunto. \footnote{\textbf{6:3} 1Tim 3,8-10} \bibleverse{4} Pero nosotros
continuaremos firmemente en la oración y en el ministerio de la
palabra.''

\bibleverse{5} Estas palabras agradaron a toda la multitud. Escogieron a
Esteban, hombre lleno de fe y del Espíritu Santo, a Felipe, a Prócoro, a
Nicanor, a Timón, a Parmenas y a Nicolás, prosélito de Antioquía,
\footnote{\textbf{6:5} Hech 8,5} \bibleverse{6} a quienes pusieron
delante de los apóstoles. Después de orar, les impusieron las manos.
\footnote{\textbf{6:6} Hech 1,24; Hech 13,3; Hech 14,23}

\bibleverse{7} La palabra de Dios crecía y el número de los discípulos
se multiplicaba enormemente en Jerusalén. Un gran número de sacerdotes
obedecía a la fe. \footnote{\textbf{6:7} Hech 2,47; Hech 19,20}

\hypertarget{acusaciuxf3n-y-muerte-de-esteban-el-primer-muxe1rtir}{%
\subsection{Acusación y muerte de Esteban, el primer
mártir}\label{acusaciuxf3n-y-muerte-de-esteban-el-primer-muxe1rtir}}

\bibleverse{8} Esteban, lleno de fe y poder, realizaba grandes prodigios
y señales entre el pueblo. \bibleverse{9} Pero algunos de los que eran
de la sinagoga llamada ``Los Libertinos'', y de los Cireneos, de los
Alejandrinos, y de los de Cilicia y Asia se levantaron, disputando con
Esteban. \bibleverse{10} No pudieron resistir la sabiduría y el Espíritu
con que hablaba. \footnote{\textbf{6:10} Luc 21,15} \bibleverse{11}
Entonces indujeron secretamente a los hombres a decir: ``Le hemos oído
hablar palabras blasfemas contra Moisés y Dios.'' \footnote{\textbf{6:11}
  Mat 26,60-66} \bibleverse{12} Entonces incitaron al pueblo, a los
ancianos y a los escribas, y vinieron contra él y lo apresaron, y lo
llevaron al concilio, \bibleverse{13} y presentaron testigos falsos que
decían: ``Este hombre no deja de decir palabras blasfemas contra este
lugar santo y contra la ley. \footnote{\textbf{6:13} Jer 26,11}
\bibleverse{14} Porque le hemos oído decir que este Jesús de Nazaret
destruirá este lugar y cambiará las costumbres que nos entregó Moisés.''
\footnote{\textbf{6:14} Juan 2,19} \bibleverse{15} Todos los que estaban
sentados en el consejo, fijando sus ojos en él, vieron su rostro como si
fuera el de un ángel.

\hypertarget{discurso-de-defensa-de-esteban-la-uxe9poca-de-los-patriarcas}{%
\subsection{Discurso de defensa de Esteban: la época de los
patriarcas}\label{discurso-de-defensa-de-esteban-la-uxe9poca-de-los-patriarcas}}

\hypertarget{section-6}{%
\section{7}\label{section-6}}

\bibleverse{1} El sumo sacerdote dijo: Entonces ``¿Es esto así?''

\bibleverse{2} Dijo: ``Hermanos y padres, escuchad. El Dios de la gloria
se le apareció a nuestro padre Abraham cuando estaba en Mesopotamia,
antes de que viviera en Harán, \footnote{\textbf{7:2} Gén 11,1-50; Jos
  24,32} \bibleverse{3} y le dijo: `Sal de tu tierra y aléjate de tus
parientes, y ven a una tierra que yo te mostraré'. \footnote{\textbf{7:3}
  Génesis 12:1} \bibleverse{4} Entonces salió de la tierra de los
caldeos y vivió en Harán. Desde allí, cuando su padre murió, Dios lo
trasladó a esta tierra en la que tú vives ahora. \bibleverse{5} No le
dio ninguna herencia en ella, ni siquiera para poner el pie. Le prometió
que se la daría en posesión, y a su descendencia después de él, cuando
aún no tuviera hijos. \bibleverse{6} Dios habló así: que su descendencia
viviría como extranjera en una tierra extraña, y que sería esclavizada y
maltratada durante cuatrocientos años. \footnote{\textbf{7:6} Éxod 12,40}
\bibleverse{7} `Yo juzgaré a la nación a la que estarán esclavizados',
dijo Dios, `y después saldrán y me servirán en este lugar'. \footnote{\textbf{7:7}
  Génesis 15:13-14} \bibleverse{8} Le dio el pacto de la circuncisión. Y
Abraham fue padre de Isaac, y lo circuncidó al octavo día. Isaac fue el
padre de Jacob, y Jacob fue el padre de los doce patriarcas.

\bibleverse{9} ``Los patriarcas, movidos por los celos contra José, lo
vendieron a Egipto. Dios estuvo con él \bibleverse{10} y lo libró de
todas sus aflicciones, y le dio favor y sabiduría ante el Faraón, rey de
Egipto. Lo hizo gobernador de Egipto y de toda su casa. \bibleverse{11}
Pero vino un hambre sobre toda la tierra de Egipto y de Canaán, y una
gran aflicción. Nuestros padres no encontraron comida. \bibleverse{12}
Pero cuando Jacob oyó que había grano en Egipto, envió a nuestros padres
la primera vez. \bibleverse{13} La segunda vez José se dio a conocer a
sus hermanos, y la familia de José fue revelada al Faraón.
\bibleverse{14} José envió y convocó a su padre Jacob y a todos sus
parientes, setenta y cinco almas. \bibleverse{15} Jacob bajó a Egipto y
murió, él y nuestros padres; \bibleverse{16} y fueron llevados de vuelta
a Siquem y puestos en la tumba que Abraham compró por un precio en plata
a los hijos de Hamor de Siquem.

\hypertarget{el-tiempo-del-mosaico}{%
\subsection{El tiempo del mosaico}\label{el-tiempo-del-mosaico}}

\bibleverse{17} ``Pero al acercarse el tiempo de la promesa que Dios
había jurado a Abraham, el pueblo creció y se multiplicó en Egipto,
\footnote{\textbf{7:17} Éxod 1,1-3} \bibleverse{18} hasta que se levantó
otro rey que no conocía a José. \bibleverse{19} Este se aprovechó de
nuestra raza y maltrató a nuestros padres, y los obligó a abandonar a
sus bebés para que no quedaran vivos. \bibleverse{20} En aquel tiempo
nació Moisés, y era sumamente apuesto para Dios. Fue alimentado durante
tres meses en la casa de su padre. \bibleverse{21} Cuando fue
abandonado, la hija del faraón lo recogió y lo crió como si fuera su
propio hijo. \bibleverse{22} Moisés fue instruido en toda la sabiduría
de los egipcios. Era poderoso en sus palabras y en sus obras.
\bibleverse{23} Pero cuando tenía cuarenta años, se le ocurrió visitar a
sus hermanos, los hijos de Israel. \bibleverse{24} Al ver que uno de
ellos sufría un agravio, lo defendió y vengó al oprimido, golpeando al
egipcio. \bibleverse{25} Suponía que sus hermanos entendían que Dios,
por su mano, les daba la liberación; pero ellos no lo entendían.

\bibleverse{26} ``Al día siguiente, se les apareció mientras peleaban, y
les instó a que volvieran a estar en paz, diciendo: `Señores, sois
hermanos. ¿Por qué os hacéis daño los unos a los otros? \bibleverse{27}
Pero el que hacía mal a su prójimo lo apartó, diciendo: `¿Quién te ha
hecho gobernante y juez sobre nosotros? \bibleverse{28} ¿Quieres matarme
como mataste ayer al egipcio?' \footnote{\textbf{7:28} Éxodo 2:14}
\bibleverse{29} Al oír estas palabras, Moisés huyó y se convirtió en
forastero en la tierra de Madián, donde fue padre de dos hijos.
\footnote{\textbf{7:29} Éxod 18,3-4}

\bibleverse{30} ``Cuando se cumplieron los cuarenta años, un ángel del
Señor se le apareció en el desierto del monte Sinaí, en una llama de
fuego en una zarza. \bibleverse{31} Cuando Moisés lo vio, se asombró de
la visión. Al acercarse para ver, se le acercó la voz del Señor:
\bibleverse{32} `Yo soy el Dios de tus padres: el Dios de Abraham, el
Dios de Isaac y el Dios de Jacob'.\footnote{\textbf{7:32} Éxodo 3:6}
Moisés tembló y no se atrevió a mirar. \bibleverse{33} El Señor le dijo:
`Quítate las sandalias, porque el lugar donde estás es tierra santa.
\bibleverse{34} Ciertamente he visto la aflicción de mi pueblo que está
en Egipto, y he oído sus gemidos. He bajado para liberarlos. Ahora ven,
te enviaré a Egipto.'\footnote{\textbf{7:34} Éxodo 3:5,7-8,10}

\bibleverse{35} ``A este Moisés, al que rechazaron diciendo: ``¿Quién te
ha hecho gobernante y juez?'', Dios lo ha enviado como gobernante y
libertador por la mano del ángel que se le apareció en la zarza.
\bibleverse{36} Este hombre los sacó de allí, después de haber hecho
maravillas y señales en Egipto, en el Mar Rojo y en el desierto durante
cuarenta años. \footnote{\textbf{7:36} Éxod 7,10; Éxod 14,21}
\bibleverse{37} Este es el Moisés que dijo a los hijos de Israel: ``El
Señor, nuestro Dios, os levantará un profeta de entre vuestros hermanos,
como yo''. \footnote{\textbf{7:37} El TR agrega ``Deberán escucharlo''.}
\footnote{\textbf{7:37} Deuteronomio 18:15} \bibleverse{38} Este es el
que estuvo en la asamblea en el desierto con el ángel que le habló en el
monte Sinaí, y con nuestros padres, que recibió revelaciones vivas para
dárnoslas, \footnote{\textbf{7:38} Éxod 19,1; Deut 9,10} \bibleverse{39}
a quien nuestros padres no quisieron obedecer, sino que lo rechazaron y
se volvieron con el corazón a Egipto, \bibleverse{40} diciendo a Aarón:
``Haznos dioses que vayan delante de nosotros, porque en cuanto a este
Moisés que nos sacó de la tierra de Egipto, no sabemos qué ha sido de
él.' \footnote{\textbf{7:40} Éxodo 32:1} \bibleverse{41} En aquellos
días hicieron un becerro y llevaron un sacrificio al ídolo, y se
alegraron de las obras de sus manos. \bibleverse{42} Pero Dios se apartó
y los entregó para servir al ejército del cielo, \footnote{\textbf{7:42}
  Este modismo también podría traducirse como ``ejército del cielo'', o
  ``seres angélicos'', o ``cuerpos celestes''.} como está escrito en el
libro de los profetas,`¿Me ofrecisteis animales sacrificados y
sacrificios¿cuarenta años en el desierto, casa de Israel?
\bibleverse{43} Tú tomaste el tabernáculo de Moloch, la estrella de tu
dios Rephan, las figuras que has hecho para adorar, así que te llevaré
\footnote{\textbf{7:43} Amós 5:25-27} más allá de Babilonia'.

\hypertarget{el-tiempo-del-tabernuxe1culo-y-la-construcciuxf3n-del-templo}{%
\subsection{El tiempo del tabernáculo y la construcción del
templo}\label{el-tiempo-del-tabernuxe1culo-y-la-construcciuxf3n-del-templo}}

\bibleverse{44} ``Nuestros padres tuvieron el tabernáculo del testimonio
en el desierto, tal como el que habló con Moisés le ordenó que lo
hiciera según el modelo que había visto; \footnote{\textbf{7:44} Éxod
  25,1} \bibleverse{45} el cual también nuestros padres, a su vez,
introdujeron con Josué cuando entraron en posesión de las naciones que
Dios expulsó delante de nuestros padres hasta los días de David,
\footnote{\textbf{7:45} Jos 3,14; Jos 18,1} \bibleverse{46} que hallaron
gracia ante los ojos de Dios, y pidieron encontrar una morada para el
Dios de Jacob. \footnote{\textbf{7:46} 2Sam 7,1; Sal 132,1-5}
\bibleverse{47} Pero Salomón le construyó una casa. \footnote{\textbf{7:47}
  1Re 6,1} \bibleverse{48} Sin embargo, el Altísimo no habita en templos
hechos por las manos, como dice el profeta, \bibleverse{49} ``El cielo
es mi trono, y la tierra un escabel para mis pies. ¿Qué clase de casa me
vas a construir?' dice el Señor.`¿O cuál es el lugar de mi descanso?
\bibleverse{50} ¿No fue mi mano la que hizo todas estas cosas?'
\footnote{\textbf{7:50} Isaías 66:1-2}

\hypertarget{fin-del-discurso-acusaciuxf3n-del-pueblo}{%
\subsection{Fin del discurso; Acusación del
pueblo}\label{fin-del-discurso-acusaciuxf3n-del-pueblo}}

\bibleverse{51} ``¡De cuello duro e incircuncisos de corazón y de oídos,
siempre os resistís al Espíritu Santo! Como hicieron vuestros padres,
así hacéis vosotros. \footnote{\textbf{7:51} Éxod 32,9; Lev 26,41; Rom
  2,28-29} \bibleverse{52} ¿A cuál de los profetas no persiguieron
vuestros padres? Mataron a los que predijeron la venida del Justo, del
que ahora os habéis convertido en traidores y asesinos. \footnote{\textbf{7:52}
  2Cró 36,16; Mat 23,31} \bibleverse{53} ¡Recibisteis la ley como fue
ordenada por los ángeles, y no la guardasteis!'' \footnote{\textbf{7:53}
  Éxod 20,1; Gal 3,19; Heb 2,2}

\hypertarget{el-martirio-de-esteban}{%
\subsection{El martirio de Esteban}\label{el-martirio-de-esteban}}

\bibleverse{54} Al oír estas cosas, se sintieron heridos en el corazón y
rechinaron los dientes contra él. \bibleverse{55} Pero él, lleno del
Espíritu Santo, miró fijamente al cielo y vio la gloria de Dios, y a
Jesús de pie a la derecha de Dios, \bibleverse{56} y dijo: ``¡Mira, veo
los cielos abiertos y al Hijo del Hombre de pie a la derecha de Dios!''
\footnote{\textbf{7:56} Luc 22,69}

\bibleverse{57} Pero ellos gritaron con fuerza y se taparon los oídos, y
luego se abalanzaron sobre él al unísono. \bibleverse{58} Lo echaron de
la ciudad y lo apedrearon. Los testigos pusieron sus vestidos a los pies
de un joven llamado Saulo. \footnote{\textbf{7:58} Hech 22,20; Lev 24,16}
\bibleverse{59} Apedrearon a Esteban mientras gritaba diciendo: ``Señor
Jesús, recibe mi espíritu''. \footnote{\textbf{7:59} Luc 23,46}
\bibleverse{60} Se arrodilló y gritó con fuerza: ``¡Señor, no les eches
en cara este pecado!'' Cuando hubo dicho esto, se quedó dormido.

\hypertarget{la-primera-persecuciuxf3n-de-la-comunidad-cristiana-en-jerusaluxe9n}{%
\subsection{La primera persecución de la comunidad cristiana en
Jerusalén}\label{la-primera-persecuciuxf3n-de-la-comunidad-cristiana-en-jerusaluxe9n}}

\hypertarget{section-7}{%
\section{8}\label{section-7}}

\bibleverse{1} Saulo consintió en su muerte. Se levantó una gran
persecución contra la asamblea que estaba en Jerusalén en aquel día.
Todos estaban dispersos por las regiones de Judea y Samaria, excepto los
apóstoles. \footnote{\textbf{8:1} Hech 18,1; Hech 11,19} \bibleverse{2}
Los hombres devotos enterraron a Esteban y se lamentaron mucho por él.
\bibleverse{3} Pero Saulo asoló la asamblea, entró en todas las casas y
arrastró a la cárcel a hombres y mujeres. \footnote{\textbf{8:3} Hech
  9,1; Hech 22,4; 1Cor 15,9}

\hypertarget{felipe-predica-y-sana}{%
\subsection{Felipe predica y sana}\label{felipe-predica-y-sana}}

\bibleverse{4} Por eso, los que estaban dispersos iban por ahí
predicando la palabra. \bibleverse{5} Felipe bajó a la ciudad de Samaria
y les anunció al Cristo. \footnote{\textbf{8:5} Hech 6,5} \bibleverse{6}
Las multitudes escuchaban unánimemente lo que decía Felipe, al oír y ver
las señales que hacía. \bibleverse{7} Porque salieron espíritus inmundos
de muchos de los que los tenían. Salían gritando a gran voz. Muchos
paralíticos y cojos quedaron curados. \footnote{\textbf{8:7} Mar 16,17}
\bibleverse{8} Hubo gran alegría en aquella ciudad.

\hypertarget{el-mago-simuxf3n-en-samaria}{%
\subsection{El mago Simón en
Samaria}\label{el-mago-simuxf3n-en-samaria}}

\bibleverse{9} Pero había un hombre, de nombre Simón, que practicaba la
hechicería en la ciudad y asombraba a la gente de Samaria, haciéndose
pasar por alguien grande, \bibleverse{10} a quien todos escuchaban,
desde el más pequeño hasta el más grande, diciendo: ``Este hombre es ese
gran poder de Dios.'' \bibleverse{11} Le escuchaban porque durante mucho
tiempo les había asombrado con sus hechicerías. \bibleverse{12} Pero
cuando creyeron que Felipe predicaba la buena noticia del Reino de Dios
y el nombre de Jesucristo, se bautizaron, tanto hombres como mujeres.
\bibleverse{13} También Simón creyó. Al ser bautizado, siguió con
Felipe. Al ver que se producían señales y grandes milagros, quedó
maravillado.

\hypertarget{obra-de-pedro-y-juan-en-samaria}{%
\subsection{Obra de Pedro y Juan en
Samaria}\label{obra-de-pedro-y-juan-en-samaria}}

\bibleverse{14} Cuando los apóstoles que estaban en Jerusalén se
enteraron de que Samaria había recibido la palabra de Dios, enviaron a
Pedro y a Juan a ellos, \bibleverse{15} quienes, al bajar, oraron por
ellos para que recibieran el Espíritu Santo; \bibleverse{16} porque
todavía no había caído sobre ninguno de ellos. Sólo habían sido
bautizados en el nombre de Cristo Jesús. \bibleverse{17} Entonces les
impusieron las manos, y recibieron el Espíritu Santo. \bibleverse{18} Al
ver Simón que el Espíritu Santo se daba por la imposición de las manos
de los apóstoles, les ofreció dinero, \bibleverse{19} diciendo: ``Dadme
también a mí este poder, para que todo aquel a quien imponga las manos
reciba el Espíritu Santo.'' \bibleverse{20} Pero Pedro le dijo: ``¡Que
tu plata perezca contigo, porque pensaste que podías obtener el don de
Dios con dinero! \bibleverse{21} No tienes parte ni suerte en este
asunto, porque tu corazón no es recto ante Dios. \bibleverse{22}
Arrepiéntete, pues, de esta tu maldad, y pide a Dios si acaso te perdona
el pensamiento de tu corazón. \bibleverse{23} Porque veo que estás en el
veneno de la amargura y en la esclavitud de la iniquidad.''

\bibleverse{24} Simón respondió: ``Ruega por mí al Señor, para que no me
suceda nada de lo que has dicho''.

\bibleverse{25} Ellos, pues, después de haber dado testimonio y
pronunciado la palabra del Señor, volvieron a Jerusalén y predicaron la
Buena Nueva en muchas aldeas de los samaritanos.

\hypertarget{conversiuxf3n-y-bautismo-del-funcionario-de-la-corte-etuxedope-por-felipe}{%
\subsection{Conversión y bautismo del funcionario de la corte etíope por
Felipe}\label{conversiuxf3n-y-bautismo-del-funcionario-de-la-corte-etuxedope-por-felipe}}

\bibleverse{26} Entonces un ángel del Señor habló a Felipe, diciendo:
``Levántate y ve hacia el sur por el camino que baja de Jerusalén a
Gaza. Este es un desierto''.

\bibleverse{27} Se levantó y fue; y he aquí que había un hombre de
Etiopía, un eunuco de gran autoridad bajo Candace, reina de los etíopes,
que estaba sobre todo su tesoro, que había venido a Jerusalén para
adorar. \bibleverse{28} Volvía y estaba sentado en su carro, y leía el
profeta Isaías.

\bibleverse{29} El Espíritu dijo a Felipe: ``Acércate y únete a este
carro''.

\bibleverse{30} Felipe corrió hacia él y le oyó leer al profeta Isaías,
y le dijo: ``¿Entiendes lo que estás leyendo?''

\bibleverse{31} Dijo: ``¿Cómo voy a hacerlo si no me lo explican?''. Le
rogó a Felipe que subiera y se sentara con él. \bibleverse{32} El pasaje
de la Escritura que estaba leyendo era éste, ``Fue llevado como una
oveja al matadero. Como un cordero mudo ante su esquilador, Así que no
abrió su boca. \bibleverse{33} En su humillación, su juicio fue quitado.
¿Quién declarará su generación? Porque su vida es quitada de la
tierra''.\footnote{\textbf{8:33} Isaías 53:7,8}

\bibleverse{34} El eunuco respondió a Felipe: ``¿De quién habla el
profeta? ¿De sí mismo, o de otro?''

\bibleverse{35} Felipe abrió la boca y, partiendo de esta Escritura, le
predicó acerca de Jesús. \bibleverse{36} Mientras iban por el camino,
llegaron a un poco de agua; y el eunuco dijo: ``Mira, aquí hay agua.
¿Qué me impide ser bautizado?''

\bibleverse{37} \footnote{\textbf{8:37} TR añade que Felipe le dijo:
  ``Si crees con todo tu corazón, puedes''. Él respondió: ``Creo que
  Jesucristo es el Hijo de Dios''.} \footnote{\textbf{8:37} Mat 16,16}
\bibleverse{38} Mandó que se detuviera el carro, y ambos bajaron al
agua, tanto Felipe como el eunuco, y lo bautizó.

\bibleverse{39} Cuando salieron del agua, el Espíritu del Señor arrebató
a Felipe, y el eunuco no lo vio más, pues siguió su camino alegremente.
\footnote{\textbf{8:39} 1Re 18,12} \bibleverse{40} Pero Felipe se
encontró en Azoto. De paso, predicó la Buena Nueva a todas las ciudades
hasta llegar a Cesarea. \footnote{\textbf{8:40} Hech 21,8-9}

\hypertarget{la-experiencia-de-saulo-camino-a-damasco}{%
\subsection{La experiencia de Saulo camino a
Damasco}\label{la-experiencia-de-saulo-camino-a-damasco}}

\hypertarget{section-8}{%
\section{9}\label{section-8}}

\bibleverse{1} Pero Saulo, que seguía respirando amenazas y matanzas
contra los discípulos del Señor, se dirigió al sumo sacerdote
\footnote{\textbf{9:1} Hech 8,3; Hech 22,3-16; Hech 26,9-18}
\bibleverse{2} y le pidió que le enviara cartas a las sinagogas de
Damasco, para que, si encontraba a alguien del Camino, ya fuera hombre o
mujer, lo llevara atado a Jerusalén. \bibleverse{3} Mientras viajaba, se
acercó a Damasco, y de repente una luz del cielo brilló a su alrededor.
\footnote{\textbf{9:3} 1Cor 15,8} \bibleverse{4} Cayó en tierra y oyó
una voz que le decía: ``Saulo, Saulo, ¿por qué me persigues?''

\bibleverse{5} Él dijo: ``¿Quién eres, Señor?'' El Señor dijo: ``Yo soy
Jesús, a quien tú persigues. \footnote{\textbf{9:5} TR añade: ``Es
  difícil para ti dar una patada contra las picanas''.} \bibleverse{6}
Pero\footnote{\textbf{9:6} TR omite ``Pero''} levántate y entra en la
ciudad, entonces se te dirá lo que debes hacer''.

\bibleverse{7} Los hombres que viajaban con él se quedaron mudos, oyendo
el ruido, pero sin ver a nadie. \bibleverse{8} Saúl se levantó del
suelo, y cuando se le abrieron los ojos, no vio a nadie. Lo llevaron de
la mano y lo introdujeron en Damasco. \bibleverse{9} Estuvo tres días
sin ver, y no comió ni bebió.

\hypertarget{sanidad-y-bautismo-de-saulo-por-ananuxedas}{%
\subsection{Sanidad y bautismo de Saulo por
Ananías}\label{sanidad-y-bautismo-de-saulo-por-ananuxedas}}

\bibleverse{10} Había en Damasco un discípulo llamado Ananías. El Señor
le dijo en una visión: ``¡Ananías!'' Dijo: ``Mira, soy yo, Señor''.

\bibleverse{11} El Señor le dijo: ``Levántate y ve a la calle que se
llama Derecha, y pregunta en la casa de Judá\footnote{\textbf{9:11} o,
  Judas} por uno llamado Saulo, hombre de Tarso. Porque he aquí que está
orando, \bibleverse{12} y en una visión ha visto a un hombre llamado
Ananías que entra y le impone las manos para que reciba la vista.''

\bibleverse{13} Pero Ananías respondió: ``Señor, he oído de muchos
acerca de este hombre, cuánto mal hizo a tus santos en Jerusalén.
\bibleverse{14} Aquí tiene autoridad de los sumos sacerdotes para atar a
todos los que invocan tu nombre.''

\bibleverse{15} Pero el Señor le dijo: ``Vete, porque él es mi
instrumento elegido para llevar mi nombre ante las naciones, los reyes y
los hijos de Israel. \footnote{\textbf{9:15} Hech 13,46; Hech 26,2; Hech
  27,24} \bibleverse{16} Porque le mostraré cuántas cosas debe sufrir
por causa de mi nombre''. \footnote{\textbf{9:16} 2Cor 11,23-28}

\bibleverse{17} Ananías salió y entró en la casa. Imponiéndole las
manos, le dijo: ``Hermano Saulo, el Señor, que se te apareció en el
camino por el que venías, me ha enviado para que recibas la vista y seas
lleno del Espíritu Santo.'' \bibleverse{18} Al instante, algo parecido a
escamas cayó de sus ojos y recibió la vista. Se levantó y fue bautizado.
\bibleverse{19} Tomó alimento y se fortaleció. Saulo permaneció varios
días con los discípulos que estaban en Damasco.

\hypertarget{la-eficacia-de-pablo-en-damasco-y-su-huida}{%
\subsection{La eficacia de Pablo en Damasco y su
huida}\label{la-eficacia-de-pablo-en-damasco-y-su-huida}}

\bibleverse{20} Inmediatamente en las sinagogas proclamó al Cristo, que
es el Hijo de Dios. \bibleverse{21} Todos los que le oían se asombraban
y decían: ``¿No es éste el que en Jerusalén hacía estragos con los que
invocaban este nombre? Y había venido aquí con la intención de llevarlos
atados ante los sumos sacerdotes''. \footnote{\textbf{9:21} Hech 8,1;
  Hech 26,10}

\bibleverse{22} Pero Saulo aumentó su fuerza y confundió a los judíos
que vivían en Damasco, demostrando que éste era el Cristo. \footnote{\textbf{9:22}
  Hech 18,28} \bibleverse{23} Cuando se cumplieron muchos días, los
judíos conspiraron juntos para matarlo, \bibleverse{24} pero su plan fue
conocido por Saulo. Vigilaban las puertas de día y de noche para
matarlo, \bibleverse{25} pero sus discípulos lo tomaron de noche y lo
bajaron por el muro, bajándolo en una canasta. \footnote{\textbf{9:25}
  2Cor 11,32-33}

\hypertarget{pablo-por-primera-vez-como-cristiano-en-jerusaluxe9n}{%
\subsection{Pablo por primera vez como cristiano en
Jerusalén}\label{pablo-por-primera-vez-como-cristiano-en-jerusaluxe9n}}

\bibleverse{26} Cuando Saulo llegó a Jerusalén, trató de unirse a los
discípulos, pero todos le tenían miedo, pues no creían que fuera un
discípulo. \footnote{\textbf{9:26} Gal 1,17-19} \bibleverse{27} Pero
Bernabé lo tomó y lo llevó a los apóstoles, y les contó cómo había visto
al Señor en el camino y cómo le había hablado, y cómo en Damasco había
predicado con valentía en el nombre de Jesús. \bibleverse{28} Estaba con
ellos entrando en \footnote{\textbf{9:28} TR y NU añaden ``y saliendo''}
Jerusalén, \bibleverse{29} predicando con denuedo en el nombre del Señor
Jesús.\footnote{\textbf{9:29} TR y NU omiten ``Jesús'' e invierten el
  orden de los versículos 28 y 29.} Hablaba y discutía contra los
helenistas, \footnote{\textbf{9:29} Los helenistas eran hebreos que
  utilizaban la lengua y la cultura griega.} pero éstos buscaban
matarlo. \bibleverse{30} Cuando los hermanos lo supieron, lo bajaron a
Cesarea y lo enviaron a Tarso. \footnote{\textbf{9:30} Gal 1,21}

\hypertarget{milagros-de-pedro-en-lydda-y-jope}{%
\subsection{Milagros de Pedro en Lydda y
Jope}\label{milagros-de-pedro-en-lydda-y-jope}}

\bibleverse{31} Así, las asambleas de toda Judea, Galilea y Samaria
tenían paz y eran edificadas. Se multiplicaron, caminando en el temor
del Señor y en el consuelo del Espíritu Santo.

\hypertarget{sanaciuxf3n-del-paralizado-eneas-en-lydda}{%
\subsection{Sanación del paralizado Eneas en
Lydda}\label{sanaciuxf3n-del-paralizado-eneas-en-lydda}}

\bibleverse{32} Mientras Pedro recorría todas aquellas partes, bajó
también a los santos que vivían en Lida. \bibleverse{33} Allí encontró a
un hombre llamado Eneas, que llevaba ocho años postrado en la cama
porque estaba paralítico. \bibleverse{34} Pedro le dijo: ``Eneas,
Jesucristo te cura. Levántate y haz tu cama''. Inmediatamente se
levantó. \bibleverse{35} Todos los que vivían en Lida y en Sarón lo
vieron, y se volvieron al Señor.

\hypertarget{criar-a-tabitha-en-joppe}{%
\subsection{Criar a Tabitha en Joppe}\label{criar-a-tabitha-en-joppe}}

\bibleverse{36} Había en Jope una discípula llamada Tabita, que
traducida significa Dorcas.\footnote{\textbf{9:36} ``Dorcas'' significa
  en griego ``Gacela''.} Esta mujer estaba llena de buenas obras y actos
de misericordia que hacía. \bibleverse{37} En aquellos días, enfermó y
murió. Cuando la lavaron, la pusieron en un cuarto alto. \bibleverse{38}
Como Lida estaba cerca de Jope, los discípulos, al enterarse de que
Pedro estaba allí, le enviaron dos hombres\footnote{\textbf{9:38}
  Lectura de NU, TR; MT omite ``dos hombres''} , rogándole que no
tardara en ir a verlos. \bibleverse{39} Pedro se levantó y fue con
ellos. Cuando llegó, lo llevaron al aposento alto. Todas las viudas
estaban junto a él llorando y mostrando las túnicas y otros vestidos que
Dorcas había hecho mientras estaba con ellas. \bibleverse{40} Pedro las
despidió a todas, y se arrodilló a orar. Volviéndose hacia el cuerpo,
dijo: ``¡Tabita, levántate!''. Ella abrió los ojos y, al ver a Pedro, se
incorporó. \footnote{\textbf{9:40} Mar 5,41} \bibleverse{41} Él le dio
la mano y la levantó. Llamando a los santos y a las viudas, la presentó
viva. \bibleverse{42} Esto se dio a conocer en toda Jope, y muchos
creyeron en el Señor. \bibleverse{43} Se quedó muchos días en Jope con
un curtidor llamado Simón.

\hypertarget{la-visiuxf3n-de-cornelio-en-cesarea}{%
\subsection{La visión de Cornelio en
Cesarea}\label{la-visiuxf3n-de-cornelio-en-cesarea}}

\hypertarget{section-9}{%
\section{10}\label{section-9}}

\bibleverse{1} Había en Cesárea un hombre llamado Cornelio, centurión
del llamado Regimiento de Italia, \bibleverse{2} hombre piadoso y
temeroso de Dios con toda su casa, que daba generosamente al pueblo
donativos para los necesitados y oraba siempre a Dios. \bibleverse{3}
Hacia la hora novena del día,\footnote{\textbf{10:3} 15:00 h.} vio
claramente en una visión a un ángel de Dios que se le acercaba y le
decía: ``¡Cornelio!''

\bibleverse{4} Él, fijando sus ojos en él y asustado, dijo: ``¿Qué es,
Señor?'' Le dijo: ``Tus oraciones y tus ofrendas a los necesitados han
subido a la memoria ante Dios. \bibleverse{5} Ahora envía hombres a Jope
y busca a Simón, que también se llama Pedro. \bibleverse{6} Se aloja en
casa de un curtidor llamado Simón, cuya casa está a la orilla del mar.
\footnote{\textbf{10:6} El TR añade ``Éste les dirá lo que es necesario
  que hagan''.} \footnote{\textbf{10:6} Hech 9,43}

\bibleverse{7} Cuando el ángel que le hablaba se marchó, Cornelio llamó
a dos de los criados de su casa y a un soldado devoto de los que le
atendían continuamente. \bibleverse{8} Después de explicarles todo, los
envió a Jope. \footnote{\textbf{10:8} Hech 11,5-17}

\hypertarget{visiuxf3n-de-pedro-en-joppe-llegada-de-los-mensajeros-de-cornelio-a-pedro}{%
\subsection{Visión de Pedro en Joppe; Llegada de los mensajeros de
Cornelio a
Pedro}\label{visiuxf3n-de-pedro-en-joppe-llegada-de-los-mensajeros-de-cornelio-a-pedro}}

\bibleverse{9} Al día siguiente, cuando iban de camino y se acercaban a
la ciudad, Pedro subió a la azotea a orar, hacia el mediodía.
\bibleverse{10} Le entró hambre y quiso comer, pero mientras se
preparaba, cayó en trance. \bibleverse{11} Vio el cielo abierto y un
recipiente que descendía hacia él, como una gran sábana bajada por
cuatro esquinas sobre la tierra, \bibleverse{12} en la que había toda
clase de cuadrúpedos de la tierra, animales salvajes, reptiles y aves
del cielo. \bibleverse{13} Una voz se dirigió a él: ``¡Levántate, Pedro,
mata y come!''

\bibleverse{14} Pero Pedro dijo: ``No es así, Señor, porque nunca he
comido nada que sea común o impuro''. \footnote{\textbf{10:14} Ezeq
  4,14; Lev 11,1}

\bibleverse{15} La segunda vez le llegó una voz: ``Lo que Dios ha
limpiado, no lo llames impuro''. \footnote{\textbf{10:15} Rom 14,14}
\bibleverse{16} Esto lo hizo tres veces, e inmediatamente el objeto fue
recibido en el cielo.

\bibleverse{17} Mientras Pedro estaba muy perplejo sobre el significado
de la visión que había visto, he aquí que los hombres enviados por
Cornelio, habiendo preguntado por la casa de Simón, se presentaron ante
la puerta, \bibleverse{18} y llamaron preguntando si Simón, que también
se llamaba Pedro, se alojaba allí. \bibleverse{19} Mientras Pedro
reflexionaba sobre la visión, el Espíritu le dijo: ``Mira,
tres\footnote{\textbf{10:19} Lectura de TR y NU. MT omite ``tres''}
hombres te buscan. \bibleverse{20} Levántate, baja y ve con ellos, sin
dudar, porque yo los he enviado''.

\bibleverse{21} Pedro bajó a los hombres y les dijo: ``Mirad, yo soy el
que buscáis. ¿Por qué habéis venido?''

\bibleverse{22} Dijeron: ``Cornelio, centurión, hombre justo y temeroso
de Dios, y bien hablado por toda la nación de los judíos, fue dirigido
por un ángel santo para que os invitara a su casa y escuchara lo que
dijerais.''

\hypertarget{pedro-en-la-casa-de-cornelio}{%
\subsection{Pedro en la casa de
Cornelio}\label{pedro-en-la-casa-de-cornelio}}

\bibleverse{23} Así que los hizo pasar y les proporcionó un lugar donde
alojarse. Al día siguiente, Pedro se levantó y salió con ellos, y le
acompañaron algunos de los hermanos de Jope. \bibleverse{24} Al día
siguiente entraron en Cesarea. Cornelio los esperaba, habiendo reunido a
sus parientes y amigos cercanos. \bibleverse{25} Cuando Pedro entró,
Cornelio le salió al encuentro, se postró a sus pies y le adoró.
\bibleverse{26} Pero Pedro lo levantó, diciendo: ``¡Levántate! Yo
también soy un hombre''. \footnote{\textbf{10:26} Hech 14,15; Apoc 19,10}
\bibleverse{27} Mientras hablaba con él, entró y encontró a muchos
reunidos. \bibleverse{28} Les dijo: ``Vosotros mismos sabéis que es cosa
ilícita que un hombre que es judío se junte o se acerque a uno de otra
nación, pero Dios me ha mostrado que no debo llamar impuro o inmundo a
ningún hombre. \bibleverse{29} Por lo tanto, también yo vine sin
quejarme cuando se me mandó llamar. Pregunto, pues, por qué mandasteis a
buscarme''.

\bibleverse{30} Cornelio dijo: ``Hace cuatro días estuve ayunando hasta
esta hora; y a la hora novena,\footnote{\textbf{10:30} 15:00 h.} oré en
mi casa, y he aquí que se presentó ante mí un hombre con ropasbrillantes
\bibleverse{31} y dijo: ``Cornelio, tu oración ha sido escuchada, y tus
donativos a los necesitados son recordados a los ojos de Dios.
\bibleverse{32} Envía, pues, a Jope y convoca a Simón, que también se
llama Pedro. Está en casa de un curtidor llamado Simón, a la orilla del
mar. Cuando venga, te hablará''. \bibleverse{33} Por eso le envié
enseguida, y fue bueno que viniera. Ahora, pues, estamos todos aquí
presentes a los ojos de Dios para oír todo lo que os ha sido ordenado
por Dios.''

\bibleverse{34} Pedro abrió la boca y dijo: ``En verdad percibo que Dios
no muestra favoritismo; \footnote{\textbf{10:34} 1Sam 16,7; Rom 2,11}
\bibleverse{35} sino que en toda nación el que le teme y obra la
justicia es aceptable para él. \footnote{\textbf{10:35} Juan 10,16}
\bibleverse{36} La palabra que envió a los hijos de Israel, anunciando
la buena noticia de la paz por medio de Jesucristo --- que es el Señor
de todo --- \footnote{\textbf{10:36} Efes 2,17} \bibleverse{37} vosotros
mismos sabéis lo que sucedió, que se proclamó por toda Judea, empezando
por Galilea, después del bautismo que predicó Juan; \footnote{\textbf{10:37}
  Mat 4,12-17} \bibleverse{38} cómo Dios ungió con el Espíritu Santo y
con poder a Jesús de Nazaret, que anduvo haciendo el bien y curando a
todos los oprimidos por el diablo, porque Dios estaba con él.
\footnote{\textbf{10:38} Mat 3,16} \bibleverse{39} Nosotros somos
testigos de todo lo que hizo tanto en el país de los judíos como en
Jerusalén; a quien también\footnote{\textbf{10:39} TR omite ``también''}
mataron, colgándolo en un madero. \bibleverse{40} Dios le resucitó al
tercer día y le dio a conocer, \footnote{\textbf{10:40} 1Cor 15,4-7}
\bibleverse{41} no a todo el pueblo, sino a los testigos elegidos de
antemano por Dios, a nosotros, que comimos y bebimos con él después de
que resucitó de entre los muertos. \footnote{\textbf{10:41} Juan 14,19;
  Juan 14,22; Luc 24,30; Luc 24,43} \bibleverse{42} Nos ordenó que
predicáramos al pueblo y diéramos testimonio de que éste es el que ha
sido designado por Dios como Juez de los vivos y de los muertos.
\footnote{\textbf{10:42} Juan 5,22} \bibleverse{43} Todos los profetas
dan testimonio de él, de que por su nombre todo el que crea en él
recibirá la remisión de los pecados.'' \footnote{\textbf{10:43} Is
  53,5-6; Jer 31,34}

\bibleverse{44} Mientras Pedro seguía diciendo estas palabras, el
Espíritu Santo cayó sobre todos los que escuchaban la palabra.
\bibleverse{45} Los de la circuncisión que habían creído estaban
asombrados, y todos los que venían con Pedro, porque el don del Espíritu
Santo se derramaba también sobre los gentiles. \bibleverse{46} Porque
les oían hablar en otras lenguas y magnificar a Dios. Entonces Pedro
contestó: \footnote{\textbf{10:46} Hech 2,4} \bibleverse{47} ``¿Puede
alguien prohibir a esta gente que se bautice con agua? Han recibido el
Espíritu Santo igual que nosotros''. \bibleverse{48} Les ordenó que se
bautizaran en el nombre de Jesucristo. Luego le pidieron que se quedara
unos días.

\hypertarget{pedro-justifica-el-bautismo-pagano-en-jerusaluxe9n}{%
\subsection{Pedro justifica el bautismo pagano en
Jerusalén}\label{pedro-justifica-el-bautismo-pagano-en-jerusaluxe9n}}

\hypertarget{section-10}{%
\section{11}\label{section-10}}

\bibleverse{1} Los apóstoles y los hermanos que estaban en Judea oyeron
que también los gentiles habían recibido la palabra de Dios.
\bibleverse{2} Cuando Pedro subió a Jerusalén, los que eran de la
circuncisión discutieron con él, \bibleverse{3} diciendo: ``¡Te
acercaste a los incircuncisos y comiste con ellos!'' \footnote{\textbf{11:3}
  Gal 2,12}

\bibleverse{4} Pero Pedro comenzó, y les explicó por orden, diciendo:
\bibleverse{5} ``Yo estaba en la ciudad de Jope orando, y en trance vi
una visión: un cierto recipiente que descendía, como si fuera una gran
sábana bajada del cielo por cuatro esquinas. Llegó hasta mí. \footnote{\textbf{11:5}
  Hech 10,9-48} \bibleverse{6} Cuando lo miré atentamente, consideré y
vi los cuadrúpedos de la tierra, los animales salvajes, los reptiles y
las aves del cielo. \bibleverse{7} También oí una voz que me decía:
``¡Levántate, Pedro, mata y come!'' \bibleverse{8} Pero yo dije: ``No,
Señor, porque en mi boca nunca ha entrado nada impuro o inmundo.
\bibleverse{9} Pero una voz me respondió por segunda vez desde el cielo:
`Lo que Dios ha limpiado, no lo llames impuro'. \bibleverse{10} Esto se
hizo tres veces, y todos fueron llevados de nuevo al cielo.
\bibleverse{11} He aquí que en seguida se presentaron tres hombres ante
la casa donde yo estaba, enviados desde Cesarea a mí. \bibleverse{12} El
Espíritu me dijo que fuera con ellos sin discriminar. Me acompañaron
también estos seis hermanos, y entramos en la casa de aquel hombre.
\bibleverse{13} Nos contó cómo había visto al ángel de pie en su casa,
diciéndole: ``Envía a Jope y trae a Simón, que se llama Pedro,
\bibleverse{14} que te hablará palabras por las que te salvarás tú y
toda tu casa''. \bibleverse{15} Cuando comencé a hablar, el Espíritu
Santo cayó sobre ellos, como sobre nosotros al principio. \footnote{\textbf{11:15}
  Hech 2,1-4} \bibleverse{16} Me acordé de la palabra del Señor, que
había dicho: ``Juan bautizó en agua, pero vosotros seréis bautizados en
el Espíritu Santo. \footnote{\textbf{11:16} Hech 1,5} \bibleverse{17}
Si, pues, Dios les concedió el mismo don que a nosotros, cuando creímos
en el Señor Jesucristo, ¿quién era yo para resistir a Dios?''

\bibleverse{18} Al oír estas cosas, callaron y glorificaron a Dios,
diciendo: ``¡Entonces Dios también ha concedido a los gentiles el
arrepentimiento para la vida!''

\hypertarget{fundaciuxf3n-de-la-primera-comunidad-cristiana-gentil-en-antioquuxeda-en-siria-su-ayuda-para-los-cristianos-necesitados-en-judea}{%
\subsection{Fundación de la primera comunidad cristiana gentil en
Antioquía en Siria; su ayuda para los cristianos necesitados en
Judea}\label{fundaciuxf3n-de-la-primera-comunidad-cristiana-gentil-en-antioquuxeda-en-siria-su-ayuda-para-los-cristianos-necesitados-en-judea}}

\bibleverse{19} Así pues, los que estaban dispersos por la opresión que
surgió en torno a Esteban viajaron hasta Fenicia, Chipre y Antioquía,
sin hablar a nadie más que a los judíos. \footnote{\textbf{11:19} Hech
  8,1-4} \bibleverse{20} Pero hubo algunos de ellos, hombres de Chipre y
de Cirene, que, cuando llegaron a Antioquía, hablaron a los helenistas,
\footnote{\textbf{11:20} Un helenista es alguien que mantiene las
  costumbres y la cultura griegas.} predicando al Señor Jesús.
\bibleverse{21} La mano del Señor estaba con ellos, y un gran número
creyó y se convirtió al Señor. \footnote{\textbf{11:21} Hech 2,47}
\bibleverse{22} La noticia sobre ellos llegó a oídos de la asamblea que
estaba en Jerusalén. Enviaron a Bernabé para que fuera hasta Antioquía,
\footnote{\textbf{11:22} Hech 4,36} \bibleverse{23} el cual, cuando
llegó y vio la gracia de Dios, se alegró. Los exhortó a todos, para que
con propósito de corazón permanecieran cerca del Señor. \bibleverse{24}
Porque era un hombre bueno, lleno del Espíritu Santo y de fe, y se
añadía mucha gente al Señor. \footnote{\textbf{11:24} Hech 5,14}

\bibleverse{25} Bernabé salió a buscar a Saulo a Tarso. \footnote{\textbf{11:25}
  Hech 9,30} \bibleverse{26} Cuando lo encontró, lo llevó a Antioquía.
Durante todo un año estuvieron reunidos con la asamblea, y enseñaron a
mucha gente. Los discípulos fueron llamados por primera vez cristianos
en Antioquía. \footnote{\textbf{11:26} Gal 2,11}

\bibleverse{27} En aquellos días, los profetas bajaron de Jerusalén a
Antioquía. \footnote{\textbf{11:27} Hech 13,1; Hech 15,32}
\bibleverse{28} Uno de ellos, llamado Agabo, se levantó e indicó por el
Espíritu que habría una gran hambruna en todo el mundo, como también
ocurrió en los días de Claudio. \footnote{\textbf{11:28} Hech 21,10}
\bibleverse{29} Como alguno de los discípulos tenía abundancia, cada uno
determinó enviar socorro a los hermanos que vivían en Judea;
\bibleverse{30} lo que también hicieron, enviándolo a los ancianos por
medio de Bernabé y Saulo. \footnote{\textbf{11:30} Hech 12,25; 1Cor
  16,1-4}

\hypertarget{muerte-de-santiago-arresto-de-pedro}{%
\subsection{Muerte de Santiago, arresto de
Pedro}\label{muerte-de-santiago-arresto-de-pedro}}

\hypertarget{section-11}{%
\section{12}\label{section-11}}

\bibleverse{1} Por aquel tiempo, el rey Herodes extendió sus manos para
oprimir a algunos de la asamblea. \bibleverse{2} Mató a Santiago, el
hermano de Juan, con la espada. \footnote{\textbf{12:2} Mat 20,20-43}
\bibleverse{3} Al ver que esto agradaba a los judíos, procedió a apresar
también a Pedro. Esto ocurrió durante los días de los panes sin
levadura. \bibleverse{4} Cuando lo detuvo, lo metió en la cárcel y lo
entregó a cuatro escuadrones de cuatro soldados cada uno para que lo
custodiaran, con la intención de sacarlo al pueblo después de la Pascua.
\bibleverse{5} Así pues, Pedro fue retenido en la cárcel, pero la
asamblea oraba constantemente por él a Dios.

\hypertarget{maravillosa-salvaciuxf3n-de-pedro}{%
\subsection{Maravillosa salvación de
Pedro}\label{maravillosa-salvaciuxf3n-de-pedro}}

\bibleverse{6} La misma noche en que Herodes iba a sacarlo, Pedro dormía
entre dos soldados, atado con dos cadenas. Los guardias delante de la
puerta custodiaban la prisión.

\bibleverse{7} Y he aquí que un ángel del Señor se puso junto a él, y
una luz brilló en la celda. Golpeó a Pedro en el costado y lo despertó,
diciendo: ``¡Levántate rápido!''. Las cadenas se le cayeron de las
manos. \footnote{\textbf{12:7} Hech 5,19} \bibleverse{8} El ángel le
dijo: ``Vístete y ponte las sandalias''. Así lo hizo. Le dijo: ``Ponte
la capa y sígueme''. \bibleverse{9} Y salió y le siguió. No sabía que lo
que hacía el ángel era real, sino que creía ver una visión.
\bibleverse{10} Cuando pasaron la primera y la segunda guardia, llegaron
a la puerta de hierro que da acceso a la ciudad, que se les abrió sola.
Salieron y bajaron por una calle, e inmediatamente el ángel se alejó de
él.

\bibleverse{11} Cuando Pedro volvió en sí, dijo: ``Ahora sé
verdaderamente que el Señor ha enviado a su ángel y me ha librado de la
mano de Herodes y de todo lo que el pueblo judío esperaba.''
\bibleverse{12} Pensando en esto, llegó a la casa de María, la madre de
Juan, que se llamaba Marcos, donde había muchos reunidos y orando.
\footnote{\textbf{12:12} Hech 12,25; Hech 13,5; Hech 13,13; Hech 15,37}
\bibleverse{13} Cuando Pedro llamó a la puerta del portal, una sirvienta
llamada Roda vino a responder. \bibleverse{14} Al reconocer la voz de
Pedro, no abrió la puerta de alegría, sino que entró corriendo e informó
de que Pedro estaba delante de la puerta.

\bibleverse{15} Le dijeron: ``¡Estás loca!'' Pero ella insistió en que
era así. Le dijeron: ``Es su ángel''. \bibleverse{16} Pero Pedro siguió
llamando. Cuando abrieron, lo vieron y se asombraron. \bibleverse{17}
Pero él, haciéndoles una señal con la mano para que se callaran, les
contó cómo el Señor le había sacado de la cárcel. Dijo: ``Contad estas
cosas a Santiago y a los hermanos''. Luego partió y se fue a otro lugar.

\hypertarget{ira-de-herodes-su-cauxedda-en-cesarea-por-un-juicio-divino}{%
\subsection{Ira de Herodes; su caída en Cesarea por un juicio
divino}\label{ira-de-herodes-su-cauxedda-en-cesarea-por-un-juicio-divino}}

\bibleverse{18} Tan pronto como se hizo de día, hubo no poco revuelo
entre los soldados acerca de lo que había sucedido con Pedro.
\footnote{\textbf{12:18} Hech 5,21-22} \bibleverse{19} Cuando Herodes lo
buscó y no lo encontró, examinó a los guardias y ordenó que los mataran.
Bajó de Judea a Cesarea y se quedó allí.

\bibleverse{20} Herodes estaba muy enojado con los habitantes de Tiro y
Sidón. Ellos acudieron de común acuerdo a él y, habiendo hecho amigo a
Blasto, el ayudante personal del rey, le pidieron la paz, porque su país
dependía del país del rey para alimentarse. \footnote{\textbf{12:20} 1Re
  5,11; Ezeq 27,17} \bibleverse{21} El día señalado, Herodes se vistió
con ropas reales, se sentó en el trono y les dirigió un discurso.
\bibleverse{22} El pueblo gritó: ``¡La voz de un dios y no de un
hombre!'' \footnote{\textbf{12:22} Ezeq 28,2} \bibleverse{23}
Inmediatamente un ángel del Señor lo golpeó, porque no le dio la gloria
a Dios. Entonces fue devorado por los gusanos y murió. \footnote{\textbf{12:23}
  Dan 5,20}

\bibleverse{24} Pero la palabra de Dios crecía y se multiplicaba.
\footnote{\textbf{12:24} Hech 6,7; Is 55,11} \bibleverse{25} Bernabé y
Saulo volvieron a \footnote{\textbf{12:25} TR dice ``de'' en lugar de
  ``a''} Jerusalén cuando cumplieron su servicio, llevando también
consigo a Juan, que se llamaba Marcos. \footnote{\textbf{12:25} Hech
  11,29-30; Hech 13,5}

\hypertarget{consagraciuxf3n-envuxedo-y-partida-de-pablo-y-bernabuxe9-su-eficacia-en-chipre}{%
\subsection{Consagración, envío y partida de Pablo y Bernabé; su
eficacia en
Chipre}\label{consagraciuxf3n-envuxedo-y-partida-de-pablo-y-bernabuxe9-su-eficacia-en-chipre}}

\hypertarget{section-12}{%
\section{13}\label{section-12}}

\bibleverse{1} En la asamblea que estaba en Antioquía había algunos
profetas y maestros: Bernabé, Simeón que se llamaba Níger, Lucio de
Cirene, Manaén el hermano adoptivo de Herodes el tetrarca, y Saulo.
\footnote{\textbf{13:1} Hech 11,27; 1Cor 12,28} \bibleverse{2} Mientras
servían al Señor y ayunaban, el Espíritu Santo dijo: ``Separadme a
Bernabé y a Saulo para la obra a la que los he llamado.'' \footnote{\textbf{13:2}
  Hech 9,15}

\bibleverse{3} Entonces, después de ayunar y orar, y de imponerles las
manos, los despidieron. \footnote{\textbf{13:3} Hech 6,6} \bibleverse{4}
Así que, enviados por el Espíritu Santo, bajaron a Seleucia. Desde allí
navegaron hasta Chipre. \bibleverse{5} Cuando estuvieron en Salamina,
proclamaron la palabra de Dios en las sinagogas judías. También tenían a
Juan como ayudante. \footnote{\textbf{13:5} Hech 12,12; Hech 12,25}
\bibleverse{6} Cuando atravesaron la isla hasta llegar a Pafos,
encontraron a un hechicero, falso profeta, un judío que se llamaba Bar
Jesús, \bibleverse{7} que estaba con el procónsul, Sergio Paulo, hombre
de entendimiento. Este hombre convocó a Bernabé y a Saulo, y buscó
escuchar la palabra de Dios. \bibleverse{8} Pero el hechicero Elimas
(pues así se llama según la interpretación) se les opuso, tratando de
apartar al procónsul de la fe. \bibleverse{9} Pero Saulo, que también se
llama Pablo, lleno del Espíritu Santo, fijó sus ojos en él
\bibleverse{10} y dijo: ``Hijo del diablo, lleno de todo engaño y de
toda astucia, enemigo de toda justicia, ¿no dejarás de pervertir los
caminos rectos del Señor? \bibleverse{11} Ahora, he aquí que la mano del
Señor está sobre ti, y quedarás ciego, sin ver el sol por un tiempo.''
Inmediatamente una niebla y la oscuridad cayeron sobre él. Anduvo
buscando a alguien que lo llevara de la mano. \bibleverse{12} Entonces
el procónsul, al ver lo que se hacía, creyó, asombrado por la enseñanza
del Señor.

\hypertarget{continuaciuxf3n-del-viaje-a-asia-menor-y-estancia-en-antioquuxeda-de-pisidia}{%
\subsection{Continuación del viaje a Asia Menor y estancia en Antioquía
de
Pisidia}\label{continuaciuxf3n-del-viaje-a-asia-menor-y-estancia-en-antioquuxeda-de-pisidia}}

\bibleverse{13} Pablo y su compañía zarparon de Pafos y llegaron a
Perga, en Panfilia. Juan se separó de ellos y volvió a Jerusalén.
\footnote{\textbf{13:13} Hech 15,38} \bibleverse{14} Pero ellos, pasando
de Perga, llegaron a Antioquía de Pisidia. Entraron en la sinagoga el
día de reposo y se sentaron. \bibleverse{15} Después de la lectura de la
ley y de los profetas, los jefes de la sinagoga les enviaron a decir:
``Hermanos, si tenéis alguna palabra de exhortación para el pueblo,
hablad.'' \footnote{\textbf{13:15} Hech 15,21}

\bibleverse{16} Pablo se puso en pie y, haciendo un gesto con la mano,
dijo: ``Hombres de Israel, y vosotros que teméis a Dios, escuchad.
\bibleverse{17} El Dios de este pueblo \footnote{\textbf{13:17} TR, NU
  añaden ``Israel''} eligió a nuestros padres, y exaltó al pueblo cuando
permanecía como extranjero en la tierra de Egipto, y con el brazo
levantado lo sacó de ella. \footnote{\textbf{13:17} Éxod 12,37; Éxod
  12,41; Éxod 14,8} \bibleverse{18} Durante un período de unos cuarenta
años los soportó en el desierto. \footnote{\textbf{13:18} Éxod 16,35}
\bibleverse{19} Después de haber destruido siete naciones en la tierra
de Canaán, les dio su tierra en herencia durante unos cuatrocientos
cincuenta años. \footnote{\textbf{13:19} Deut 7,1; Jos 14,2}
\bibleverse{20} Después de esto, les dio jueces hasta el profeta Samuel.
\footnote{\textbf{13:20} Jue 2,16; 1Sam 3,20} \bibleverse{21} Después
pidieron un rey, y Dios les dio a Saúl, hijo de Cis, un hombre de la
tribu de Benjamín, durante cuarenta años. \footnote{\textbf{13:21} 1Sam
  8,5; 1Sam 10,21; 1Sam 10,24} \bibleverse{22} Cuando lo destituyó,
levantó a David para que fuera su rey, a quien también le dijo: ``He
encontrado a David, hijo de Jesé, un hombre según mi corazón, que hará
toda mi voluntad''. \bibleverse{23} De la descendencia de este hombre,
Dios ha traído la salvación\footnote{\textbf{13:23} TR, NU léase ``un
  Salvador, Jesús'' en lugar de ``salvación''} a Israel según su
promesa, \footnote{\textbf{13:23} Is 11,1} \bibleverse{24} antes de su
venida, cuando Juan había predicado por primera vez el bautismo de
arrepentimiento a Israel. \footnote{\textbf{13:24} TR, NU léase ``a todo
  el pueblo de Israel'' en lugar de ``a Israel''} \footnote{\textbf{13:24}
  Luc 3,3} \bibleverse{25} Mientras Juan cumplía su curso, dijo:
``¿Quién suponéis que soy yo? Yo no soy. Pero he aquí que viene uno
detrás de mí, cuyas sandalias no soy digno de desatar''. \footnote{\textbf{13:25}
  Juan 1,20; Juan 1,27; Luc 3,16; Mar 1,7}

\bibleverse{26} ``Hermanos, hijos del linaje de Abraham, y los que entre
vosotros temen a Dios, se os envía la palabra de esta salvación.
\bibleverse{27} Porque los que habitan en Jerusalén y sus gobernantes,
por no conocerle, ni las voces de los profetas que se leen cada sábado,
las cumplieron condenándole. \footnote{\textbf{13:27} Juan 16,3}
\bibleverse{28} Aunque no encontraron ninguna causa de muerte, aun así
pidieron a Pilato que lo mandara matar. \footnote{\textbf{13:28} Mat
  27,22-23} \bibleverse{29} Cuando se cumplieron todas las cosas que
estaban escritas sobre él, lo bajaron del madero y lo pusieron en un
sepulcro. \footnote{\textbf{13:29} Mat 27,59-60} \bibleverse{30} Pero
Dios lo resucitó de entre los muertos, \footnote{\textbf{13:30} Hech
  3,15} \bibleverse{31} y lo vieron durante muchos días los que subieron
con él de Galilea a Jerusalén, que son sus testigos ante el pueblo.
\footnote{\textbf{13:31} Hech 1,3} \bibleverse{32} Os anunciamos la
buena noticia de la promesa hecha a los padres, \bibleverse{33} que Dios
ha cumplido con nosotros, sus hijos, al resucitar a Jesús. Como también
está escrito en el segundo salmo,`Tú eres mi Hijo. Hoy me he convertido
en tu padre\footnote{\textbf{13:33} Salmo 2:7} ''.

\bibleverse{34} ``En cuanto a que lo resucitó de entre los muertos, para
que ya no vuelva a la corrupción, ha hablado así: `Te daré las
bendiciones santas y seguras de David'. \footnote{\textbf{13:34} Isaías
  55:3} \bibleverse{35} Por eso dice también en otro salmo: `No
permitirás que tu Santo vea la decadencia.' \footnote{\textbf{13:35}
  Salmo 16:10} \bibleverse{36} Porque David, después de haber servido en
su propia generación al consejo de Dios, se durmió, fue acostado con sus
padres y vio la decadencia. \bibleverse{37} Pero el que Dios resucitó no
vio la decadencia. \bibleverse{38} Sabed, pues, hermanos, que por medio
de este hombre se os anuncia la remisión de los pecados; \bibleverse{39}
y que por él todo el que cree es justificado de todo, de lo cual no
podíais ser justificados por la ley de Moisés. \footnote{\textbf{13:39}
  Rom 8,3-4; Rom 10,4} \bibleverse{40} Tened, pues, cuidado, no sea que
venga sobre vosotros lo que se dice en los profetas: \bibleverse{41}
``¡Mirad, burlones! Maravíllate y perece, porque yo trabajo una obra en
tus días, una obra que no creerás de ninguna manera, si alguien te la
declara'\,''. \footnote{\textbf{13:41} Habacuc 1:5}

\hypertarget{varios-uxe9xitos-del-discurso}{%
\subsection{Varios éxitos del
discurso}\label{varios-uxe9xitos-del-discurso}}

\bibleverse{42} Cuando los judíos salieron de la sinagoga, los gentiles
pidieron que se les predicaran estas palabras el sábado siguiente.
\bibleverse{43} Cuando la sinagoga se disolvió, muchos de los judíos y
de los prosélitos devotos siguieron a Pablo y a Bernabé, quienes,
hablándoles, les exhortaron a continuar en la gracia de Dios.

\bibleverse{44} El sábado siguiente se reunió casi toda la ciudad para
oír la palabra de Dios. \bibleverse{45} Pero los judíos, al ver las
multitudes, se llenaron de celos, contradijeron lo dicho por Pablo y
blasfemaron.

\bibleverse{46} Pablo y Bernabé hablaron con valentía y dijeron: ``Era
necesario que la palabra de Dios se os dijera primero. Puesto que, en
efecto, la rechazáis y os juzgáis indignos de la vida eterna, he aquí
que nos dirigimos a los gentiles. \footnote{\textbf{13:46} Hech 3,25-26;
  Mat 10,5-6} \bibleverse{47} Porque así nos lo ha ordenado el Señor,
diciendo, Te he puesto como luz para los gentiles, para que lleves la
salvación hasta los confines de la tierra''. \footnote{\textbf{13:47}
  Isaías 49:6}

\bibleverse{48} Al oír esto, los gentiles se alegraron y glorificaron la
palabra de Dios. Todos los que estaban destinados a la vida eterna
creyeron. \footnote{\textbf{13:48} Rom 8,29-30} \bibleverse{49} La
palabra del Señor se difundió por toda la región. \bibleverse{50} Pero
los judíos incitaron a las mujeres devotas y prominentes y a los
principales hombres de la ciudad, y suscitaron una persecución contra
Pablo y Bernabé, y los expulsaron de sus fronteras. \bibleverse{51} Pero
ellos se sacudieron el polvo de sus pies contra ellos y llegaron a
Iconio. \footnote{\textbf{13:51} Hech 18,6; Mat 10,14} \bibleverse{52}
Los discípulos se llenaron de alegría y del Espíritu Santo.

\hypertarget{efectividad-de-los-apuxf3stoles-en-iconio}{%
\subsection{Efectividad de los Apóstoles en
Iconio}\label{efectividad-de-los-apuxf3stoles-en-iconio}}

\hypertarget{section-13}{%
\section{14}\label{section-13}}

\bibleverse{1} En Iconio, entraron juntos en la sinagoga de los judíos,
y hablaron de tal manera que una gran multitud, tanto de judíos como de
griegos, creyó. \bibleverse{2} Pero los judíos incrédulos\footnote{\textbf{14:2}
  o, desobediente} agitaron y amargaron las almas de los gentiles contra
los hermanos. \bibleverse{3} Por tanto, permanecieron allí mucho tiempo,
hablando con denuedo en el Señor, que daba testimonio de la palabra de
su gracia, concediendo que se hicieran señales y prodigios por sus
manos. \footnote{\textbf{14:3} Hech 19,11; Heb 2,4} \bibleverse{4} Pero
la multitud de la ciudad estaba dividida. Una parte se puso del lado de
los judíos y otra de los apóstoles. \bibleverse{5} Cuando algunos de los
gentiles y de los judíos, con sus jefes, intentaron violentamente
maltratarlos y apedrearlos, \footnote{\textbf{14:5} 2Tim 3,11}
\bibleverse{6} ellos se dieron cuenta y huyeron a las ciudades de
Licaonia, Listra, Derbe y la región circundante. \bibleverse{7} Allí
predicaron la Buena Nueva.

\hypertarget{curaciuxf3n-de-un-cojo-y-lapidaciuxf3n-de-pablo-en-listra-los-dos-apuxf3stoles-escapan-a-derbe}{%
\subsection{Curación de un cojo y lapidación de Pablo en Listra; los dos
apóstoles escapan a
Derbe}\label{curaciuxf3n-de-un-cojo-y-lapidaciuxf3n-de-pablo-en-listra-los-dos-apuxf3stoles-escapan-a-derbe}}

\bibleverse{8} En Listra estaba sentado un hombre impotente de los pies,
tullido desde el vientre de su madre, que nunca había caminado.
\bibleverse{9} Estaba oyendo hablar a Pablo, el cual, fijando los ojos
en él y viendo que tenía fe para quedar sano, \footnote{\textbf{14:9}
  Mat 9,28} \bibleverse{10} le dijo con voz potente: ``¡Ponte de pie!''
Se levantó de un salto y caminó. \bibleverse{11} Al ver la multitud lo
que Pablo había hecho, alzaron la voz diciendo en la lengua de Licaonia:
``¡Los dioses han bajado a nosotros en forma de hombres!'' \footnote{\textbf{14:11}
  Hech 28,6} \bibleverse{12} Llamaban a Bernabé ``Júpiter'', y a Pablo
``Mercurio'', porque era el orador principal. \bibleverse{13} El
sacerdote de Júpiter, cuyo templo estaba frente a su ciudad, traía
bueyes y guirnaldas a las puertas, y quería hacer un sacrificio junto
con las multitudes.

\bibleverse{14} Pero cuando los apóstoles Bernabé y Pablo lo oyeron, se
rasgaron las vestiduras y se lanzaron a la multitud, gritando:
\bibleverse{15} ``Hombres, ¿por qué hacéis estas cosas? Nosotros también
somos hombres de la misma naturaleza que vosotros, y os traemos la buena
noticia, para que os convirtáis de estas cosas vanas al Dios vivo, que
hizo el cielo, la tierra, el mar y todo lo que hay en ellos; \footnote{\textbf{14:15}
  Hech 10,26} \bibleverse{16} que en las generaciones pasadas permitió
que todas las naciones anduvieran por sus propios caminos. \footnote{\textbf{14:16}
  Hech 17,30} \bibleverse{17} Sin embargo, no se dejó sin testimonio, ya
que hizo el bien y os dio \footnote{\textbf{14:17} TR dice ``nosotros''
  en lugar de ``ustedes''} lluvias del cielo y estaciones fructíferas,
llenando nuestros corazones de alimento y alegría.''

\bibleverse{18} Aun diciendo estas cosas, apenas impidieron que las
multitudes les hicieran un sacrificio. \bibleverse{19} Pero algunos
judíos de Antioquía e Iconio llegaron allí, y habiendo persuadido a las
multitudes, apedrearon a Pablo y lo arrastraron fuera de la ciudad,
suponiendo que estaba muerto. \footnote{\textbf{14:19} 2Cor 11,25; 2Tim
  3,11}

\hypertarget{los-apuxf3stoles-en-derbe-fortalecimiento-de-las-comunidades-fundadas-regreso-a-antioquuxeda-en-siria}{%
\subsection{Los apóstoles en Derbe; Fortalecimiento de las comunidades
fundadas; Regreso a Antioquía en
Siria}\label{los-apuxf3stoles-en-derbe-fortalecimiento-de-las-comunidades-fundadas-regreso-a-antioquuxeda-en-siria}}

\bibleverse{20} Pero como los discípulos estaban a su alrededor, se
levantó y entró en la ciudad. Al día siguiente salió con Bernabé hacia
Derbe.

\bibleverse{21} Después de haber predicado la Buena Nueva en aquella
ciudad y de haber hecho muchos discípulos, volvieron a Listra, Iconio y
Antioquía, \bibleverse{22} fortaleciendo las almas de los discípulos,
exhortándoles a que permanecieran en la fe, y que a través de muchas
aflicciones hay que entrar en el Reino de Dios. \footnote{\textbf{14:22}
  Rom 5,3-5; 1Tes 3,3} \bibleverse{23} Cuando les nombraron ancianos en
cada asamblea, y oraron con ayuno, los encomendaron al Señor en quien
habían creído. \footnote{\textbf{14:23} Hech 6,6}

\bibleverse{24} Pasaron por Pisidia y llegaron a Panfilia.
\bibleverse{25} Después de pronunciar la palabra en Perga, bajaron a
Attalia. \bibleverse{26} De allí navegaron a Antioquía, desde donde se
encomendaron a la gracia de Dios por la obra que habían realizado.
\footnote{\textbf{14:26} Hech 13,1-2} \bibleverse{27} Cuando llegaron y
reunieron a la asamblea, informaron de todo lo que Dios había hecho con
ellos y de que había abierto una puerta de fe a las naciones.
\footnote{\textbf{14:27} 1Cor 16,9} \bibleverse{28} Se quedaron allí con
los discípulos durante mucho tiempo.

\hypertarget{la-causa-de-la-convenciuxf3n-envuxedo-de-pablo-y-bernabuxe9-a-jerusaluxe9n}{%
\subsection{La causa de la Convención; Envío de Pablo y Bernabé a
Jerusalén}\label{la-causa-de-la-convenciuxf3n-envuxedo-de-pablo-y-bernabuxe9-a-jerusaluxe9n}}

\hypertarget{section-14}{%
\section{15}\label{section-14}}

\bibleverse{1} Algunos hombres bajaron de Judea y enseñaron a los
hermanos: ``Si no os circuncidáis según la costumbre de Moisés, no
podéis salvaros.'' \footnote{\textbf{15:1} Gal 5,2} \bibleverse{2} Por
lo tanto, como Pablo y Bernabé tuvieron no poca discordia y discusión
con ellos, designaron a Pablo, a Bernabé y a algunos otros de ellos para
que subieran a Jerusalén a ver a los apóstoles y a los ancianos sobre
esta cuestión. \footnote{\textbf{15:2} Gal 2,1} \bibleverse{3} Ellos,
enviados por la asamblea, pasaron por Fenicia y Samaria, anunciando la
conversión de los gentiles. Causaron gran alegría a todos los hermanos.
\bibleverse{4} Cuando llegaron a Jerusalén, fueron recibidos por la
asamblea, los apóstoles y los ancianos, y les contaron todo lo que Dios
había hecho con ellos.

\bibleverse{5} Pero algunos de la secta de los fariseos que creían se
levantaron diciendo: ``Es necesario circuncidarlos y mandarles guardar
la ley de Moisés.''

\hypertarget{las-negociaciones-discursos-de-pedro-y-santiago}{%
\subsection{Las negociaciones; Discursos de Pedro y
Santiago}\label{las-negociaciones-discursos-de-pedro-y-santiago}}

\bibleverse{6} Los apóstoles y los ancianos estaban reunidos para ver
este asunto. \bibleverse{7} Cuando se discutió mucho, Pedro se levantó y
les dijo: ``Hermanos, sabéis que hace tiempo que Dios eligió entre
vosotros que por mi boca las naciones oyeran la palabra de la Buena
Nueva y creyeran. \footnote{\textbf{15:7} Hech 10,44; Hech 11,15}
\bibleverse{8} Dios, que conoce el corazón, dio testimonio de ellos,
otorgándoles el Espíritu Santo, como lo hizo con nosotros.
\bibleverse{9} No hizo distinción entre nosotros y ellos, limpiando sus
corazones por la fe. \bibleverse{10} Ahora bien, ¿por qué tentáis a
Dios, poniendo sobre el cuello de los discípulos un yugo que ni nuestros
padres ni nosotros pudimos soportar? \footnote{\textbf{15:10} Mat 23,4;
  Gal 5,1} \bibleverse{11} Pero nosotros creemos que estamos salvados
por la gracia del Señor Jesús,\footnote{\textbf{15:11} TR añade
  ``Cristo''} al igual que ellos.'' \footnote{\textbf{15:11} Gal 2,16;
  Efes 2,4-10}

\bibleverse{12} Toda la multitud guardaba silencio, y escuchaba a
Bernabé y a Pablo informar de las señales y prodigios que Dios había
hecho entre las naciones por medio de ellos. \bibleverse{13} Después de
que guardaron silencio, Santiago respondió: ``Hermanos, escuchadme.
\footnote{\textbf{15:13} Hech 21,18; Gal 2,9} \bibleverse{14} Simeón ha
informado de cómo Dios visitó primero a las naciones para sacar de ellas
un pueblo para su nombre. \bibleverse{15} Esto concuerda con las
palabras de los profetas. Como está escrito, \bibleverse{16} ``Después
de esto volveré. Volveré a construir el tabernáculo de David, que ha
caído. Volveré a construir sus ruinas. Lo pondré \bibleverse{17} para
que el resto de los hombres busquen al Señor: todos los gentiles que son
llamados por mi nombre, dice el Señor, que hace todas estas cosas''.
\footnote{\textbf{15:17} Amós 9:11-12}

\bibleverse{18} ``Todas las obras de Dios son conocidas por él desde la
eternidad. \bibleverse{19} Por lo tanto, mi juicio es que no molestemos
a los de entre los gentiles que se convierten a Dios, \bibleverse{20}
sino que les escribamos que se abstengan de la contaminación de los
ídolos, de la inmoralidad sexual, de lo estrangulado y de la sangre.
\footnote{\textbf{15:20} Gén 9,4; Lev 17,10-14; Lev 19,4; Lev 19,29}
\bibleverse{21} Porque Moisés, desde generaciones, tiene en cada ciudad
quienes lo predican, siendo leído en las sinagogas todos los sábados.''
\footnote{\textbf{15:21} Hech 13,15}

\hypertarget{la-resoluciuxf3n-y-su-implementaciuxf3n}{%
\subsection{La resolución y su
implementación}\label{la-resoluciuxf3n-y-su-implementaciuxf3n}}

\bibleverse{22} Entonces les pareció bien a los apóstoles y a los
ancianos, con toda la asamblea, elegir hombres de su compañía y
enviarlos a Antioquía con Pablo y Bernabé: Judas, llamado Barrabás, y
Silas, hombres principales entre los hermanos. \bibleverse{23} Ellos
escribieron estas cosas de su mano: ``Los apóstoles, los ancianos y los
hermanos, a los hermanos que son de los gentiles en Antioquía, Siria y
Cilicia: saludos. \bibleverse{24} Como hemos oído que algunos de los que
salieron de nosotros os han perturbado con palabras, inquietando
vuestras almas, diciendo: ``Tenéis que circuncidaros y guardar la ley'',
a quienes no dimos ningún mandamiento, \bibleverse{25} nos ha parecido
bien, habiendo llegado a un acuerdo, elegir a unos hombres y enviarlos a
vosotros con nuestros amados Bernabé y Pablo, \bibleverse{26} hombres
que han arriesgado su vida por el nombre de nuestro Señor Jesucristo.
\bibleverse{27} Hemos enviado, pues, a Judas y a Silas, que también os
dirán lo mismo de palabra. \bibleverse{28} Porque al Espíritu Santo y a
nosotros nos ha parecido bien no imponeros mayor carga que estas cosas
necesarias: \bibleverse{29} que os abstengáis de lo sacrificado a los
ídolos, de la sangre, de lo estrangulado y de la inmoralidad sexual, de
lo cual, si os guardáis, os irá bien. Adiós''.

\hypertarget{el-resultado-judas-y-silas-en-antioquuxeda}{%
\subsection{El resultado: Judas y Silas en
Antioquía}\label{el-resultado-judas-y-silas-en-antioquuxeda}}

\bibleverse{30} Así que, cuando fueron enviados, llegaron a Antioquía.
Tras reunir a la multitud, les entregaron la carta. \bibleverse{31}
Cuando la leyeron, se alegraron de los ánimos. \bibleverse{32} Judas y
Silas, siendo también profetas, animaron a los hermanos con muchas
palabras y los fortalecieron. \footnote{\textbf{15:32} Hech 11,27; Hech
  13,1} \bibleverse{33} Después de haber pasado algún tiempo allí, los
hermanos los despidieron en paz con los apóstoles. \bibleverse{34}
\footnote{\textbf{15:34} TR dice ``Y los judíos que no estaban
  persuadidos, se volvieron envidiosos y se llevaron'' en lugar de
  ``Pero los judíos no persuadidos se llevaron''}

\hypertarget{la-pelea-de-pablo-con-bernabuxe9-salida-de-pablo-y-silas-de-antioquuxeda}{%
\subsection{La pelea de Pablo con Bernabé; Salida de Pablo y Silas de
Antioquía}\label{la-pelea-de-pablo-con-bernabuxe9-salida-de-pablo-y-silas-de-antioquuxeda}}

\bibleverse{35} Pero Pablo y Bernabé se quedaron en Antioquía, enseñando
y predicando la palabra del Señor, con muchos otros también.

\bibleverse{36} Al cabo de unos días, Pablo dijo a Bernabé: ``Volvamos
ahora a visitar a nuestros hermanos en todas las ciudades en las que
hemos proclamado la palabra del Señor, para ver cómo les va.''
\bibleverse{37} Bernabé pensaba llevar también a Juan, que se llamaba
Marcos, con ellos. \footnote{\textbf{15:37} Hech 1,12; Hech 1,25}
\bibleverse{38} Pero a Pablo no le pareció buena idea llevar con ellos a
alguien que se había alejado de ellos en Panfilia, y no fue con ellos a
hacer la obra. \footnote{\textbf{15:38} Hech 13,13} \bibleverse{39}
Entonces la disputa se agudizó tanto que se separaron unos de otros.
Bernabé se llevó a Marcos y se embarcó hacia Chipre, \bibleverse{40}
pero Pablo eligió a Silas y salió, encomendado por los hermanos a la
gracia de Dios. \bibleverse{41} Recorrió Siria y Cilicia, fortaleciendo
las asambleas.

\hypertarget{el-viaje-por-tierra-a-travuxe9s-de-asia-menor-hasta-troas}{%
\subsection{El viaje por tierra a través de Asia Menor hasta
Troas}\label{el-viaje-por-tierra-a-travuxe9s-de-asia-menor-hasta-troas}}

\hypertarget{section-15}{%
\section{16}\label{section-15}}

\bibleverse{1} Llegó a Derbe y Listra; y he aquí que había allí un
discípulo llamado Timoteo, hijo de una judía creyente, pero su padre era
griego. \footnote{\textbf{16:1} Hech 17,14; Hech 19,22; Hech 20,4; Fil
  2,19-22; 1Tes 3,2; 1Tes 3,6; 2Tim 1,5} \bibleverse{2} Los hermanos que
estaban en Listra e Iconio dieron buen testimonio de él. \bibleverse{3}
Pablo quiso que saliera con él, y lo tomó y lo circuncidó a causa de los
judíos que había en aquellos lugares, pues todos sabían que su padre era
griego. \footnote{\textbf{16:3} Gal 2,3} \bibleverse{4} Mientras iban
por las ciudades, les entregaban los decretos que habían sido ordenados
por los apóstoles y los ancianos que estaban en Jerusalén. \footnote{\textbf{16:4}
  Hech 15,23-29} \bibleverse{5} Así las asambleas se fortalecían en la
fe y aumentaban en número cada día.

\bibleverse{6} Cuando pasaron por la región de Frigia y Galacia, el
Espíritu Santo les prohibió hablar la palabra en Asia. \footnote{\textbf{16:6}
  Hech 18,23} \bibleverse{7} Cuando llegaron frente a Misia, intentaron
entrar en Bitinia, pero el Espíritu no se lo permitió. \bibleverse{8}
Pasando por Misia, bajaron a Troas. \bibleverse{9} Una visión se le
apareció a Pablo durante la noche. Había un hombre de Macedonia que le
rogaba y le decía: ``Pasa a Macedonia y ayúdanos''. \bibleverse{10} Al
ver la visión, inmediatamente tratamos de ir a Macedonia, concluyendo
que el Señor nos había llamado para predicarles la Buena Nueva.

\hypertarget{el-viaje-por-mar-a-macedonia-pablo-en-filipos}{%
\subsection{El viaje por mar a Macedonia; Pablo en
Filipos}\label{el-viaje-por-mar-a-macedonia-pablo-en-filipos}}

\bibleverse{11} Zarpando, pues, de Troas, pusimos rumbo directo a
Samotracia, y al día siguiente a Neápolis; \bibleverse{12} y de allí a
Filipos, que es una ciudad de Macedonia, la más importante de la
comarca, una colonia romana. Estuvimos algunos días en esta ciudad.

\hypertarget{conversiuxf3n-de-la-trader-morada-lydia}{%
\subsection{Conversión de la trader morada
Lydia}\label{conversiuxf3n-de-la-trader-morada-lydia}}

\bibleverse{13} El sábado fuimos fuera de la ciudad, a la orilla de un
río, donde suponíamos que había un lugar de oración, y nos sentamos a
hablar con las mujeres que se habían reunido. \bibleverse{14} Una mujer
llamada Lidia, vendedora de púrpura, de la ciudad de Tiatira, que
adoraba a Dios, nos escuchó. El Señor le abrió el corazón para que
escuchara lo que decía Pablo. \bibleverse{15} Cuando ella y su familia
se bautizaron, nos rogó diciendo: ``Si habéis juzgado que soy fiel al
Señor, entrad en mi casa y quedaos''. Y nos convenció.

\hypertarget{la-doncella-adivina-pablo-y-silas-en-la-corte-y-en-la-cuxe1rcel}{%
\subsection{La doncella adivina; Pablo y Silas en la corte y en la
cárcel}\label{la-doncella-adivina-pablo-y-silas-en-la-corte-y-en-la-cuxe1rcel}}

\bibleverse{16} Mientras íbamos a la oración, nos salió al encuentro una
muchacha con espíritu de adivinación, que hacía ganar mucho a sus amos
con la adivinación. \bibleverse{17} Siguiendo a Pablo y a nosotros,
gritó: ``¡Estos hombres son servidores del Dios Altísimo, que nos
anuncian un camino de salvación!'' \footnote{\textbf{16:17} Mar 1,24;
  Mar 1,34} \bibleverse{18} Estuvo haciendo esto durante muchos días.
Pero Pablo, molestándose mucho, se volvió y le dijo al espíritu: ``¡Te
ordeno en nombre de Jesucristo que salgas de ella!''. Salió en esa misma
hora. \footnote{\textbf{16:18} Mar 16,17} \bibleverse{19} Pero cuando
sus amos vieron que la esperanza de su ganancia se había esfumado,
agarraron a Pablo y a Silas y los arrastraron a la plaza ante los
magistrados. \bibleverse{20} Cuando los llevaron ante los magistrados,
éstos dijeron: ``Estos hombres, siendo judíos, agitan nuestra ciudad
\footnote{\textbf{16:20} Hech 17,6} \bibleverse{21} y defienden
costumbres que no nos es lícito aceptar ni observar, siendo romanos.''

\bibleverse{22} La multitud se alzó contra ellos y los magistrados les
arrancaron las ropas, y luego ordenaron que los golpearan con varas.
\footnote{\textbf{16:22} 2Cor 11,25; Fil 1,30; 1Tes 2,2} \bibleverse{23}
Después de haberles dado muchos azotes, los metieron en la cárcel,
encargando al carcelero que los guardara con seguridad. \bibleverse{24}
Recibida tal orden, los metió en la cárcel interior y les aseguró los
pies en el cepo.

\hypertarget{la-conversiuxf3n-del-carcelero}{%
\subsection{La conversión del
carcelero}\label{la-conversiuxf3n-del-carcelero}}

\bibleverse{25} Pero hacia la medianoche Pablo y Silas estaban orando y
cantando himnos a Dios, y los presos los escuchaban. \bibleverse{26} De
repente se produjo un gran terremoto, que hizo temblar los cimientos de
la cárcel, y al instante se abrieron todas las puertas y se soltaron las
cadenas de todos. \bibleverse{27} El carcelero, despertando del sueño y
viendo las puertas de la cárcel abiertas, sacó su espada y se iba a
matar, suponiendo que los presos se habían escapado. \bibleverse{28}
Pero Pablo gritó a gran voz, diciendo: ``¡No te hagas daño, pues estamos
todos aquí!''

\bibleverse{29} Llamó a las luces, entró de un salto, se postró
tembloroso ante Pablo y Silas, \bibleverse{30} los sacó y dijo:
``Señores, ¿qué debo hacer para salvarme?'' \footnote{\textbf{16:30}
  Hech 2,37}

\bibleverse{31} Le dijeron: ``Cree en el Señor Jesucristo y te salvarás,
tú y tu familia''. \bibleverse{32} Le hablaron de la palabra del Señor a
él y a todos los que estaban en su casa.

\bibleverse{33} Los tomó a la misma hora de la noche y les lavó las
vestiduras, e inmediatamente se bautizó, él y toda su familia.
\bibleverse{34} Los hizo subir a su casa y les puso la comida delante, y
se alegró mucho con toda su familia, por haber creído en Dios.

\hypertarget{la-liberaciuxf3n-de-pablo-y-silas-de-la-cuxe1rcel}{%
\subsection{La liberación de Pablo y Silas de la
cárcel}\label{la-liberaciuxf3n-de-pablo-y-silas-de-la-cuxe1rcel}}

\bibleverse{35} Pero cuando se hizo de día, los magistrados enviaron a
los sargentos, diciendo: ``Dejen ir a esos hombres''.

\bibleverse{36} El carcelero comunicó estas palabras a Pablo, diciendo:
``Los magistrados han enviado a dejarte ir; ahora, pues, sal y vete en
paz.''

\bibleverse{37} Pero Pablo les dijo: ``¡Nos han golpeado públicamente
sin juicio, hombres que son romanos, y nos han echado en la cárcel! ¿Nos
liberan ahora en secreto? No, ciertamente, sino que vengan ellos mismos
y nos saquen''. \footnote{\textbf{16:37} Hech 22,25}

\bibleverse{38} Los sargentos comunicaron estas palabras a los
magistrados, y éstos, al oír que eran romanos, se asustaron,
\bibleverse{39} y vinieron a rogarles. Cuando los sacaron, les pidieron
que se fueran de la ciudad. \bibleverse{40} Salieron de la cárcel y
entraron en casa de Lidia. Cuando vieron a los hermanos, los animaron y
se marcharon.

\hypertarget{pablo-en-tesaluxf3nica}{%
\subsection{Pablo en Tesalónica}\label{pablo-en-tesaluxf3nica}}

\hypertarget{section-16}{%
\section{17}\label{section-16}}

\bibleverse{1} Cuando pasaron por Anfípolis y Apolonia, llegaron a
Tesalónica, donde había una sinagoga judía. \footnote{\textbf{17:1} 1Tes
  2,2} \bibleverse{2} Pablo, como era su costumbre, entró en ella, y
durante tres sábados razonó con ellos a partir de las Escrituras,
\bibleverse{3} explicando y demostrando que el Cristo tenía que padecer
y resucitar de entre los muertos, y diciendo: ``Este Jesús, que yo os
anuncio, es el Cristo.'' \footnote{\textbf{17:3} Luc 24,26-27; Luc
  24,45-46}

\bibleverse{4} Algunos de ellos fueron persuadidos y se unieron a Pablo
y a Silas: de los griegos devotos, una gran multitud, y no pocas de las
mujeres principales. \footnote{\textbf{17:4} 1Tes 1,1; 2Tes 1,1}
\bibleverse{5} Pero los judíos no persuadidos tomaron a algunos malvados
de la plaza y, reuniendo una multitud, alborotaron la ciudad. Asaltando
la casa de Jasón, trataron de sacarlos al pueblo. \bibleverse{6} Al no
encontrarlos, arrastraron a Jasón y a algunos hermanos ante los
gobernantes de la ciudad, gritando: ``También han venido aquí estos que
han puesto el mundo patas arriba, \footnote{\textbf{17:6} Hech 16,20}
\bibleverse{7} a los que Jasón ha recibido. Todos estos actúan en contra
de los decretos del César, diciendo que hay otro rey, Jesús!''
\footnote{\textbf{17:7} Luc 23,2} \bibleverse{8} La multitud y los
gobernantes de la ciudad se turbaron al oír estas cosas. \bibleverse{9}
Cuando tomaron fianza de Jasón y de los demás, los dejaron ir.

\hypertarget{las-experiencias-de-pablo-en-berea-y-su-viaje-a-atenas}{%
\subsection{Las experiencias de Pablo en Berea y su viaje a
Atenas}\label{las-experiencias-de-pablo-en-berea-y-su-viaje-a-atenas}}

\bibleverse{10} Los hermanos enviaron inmediatamente a Pablo y a Silas
de noche a Berea. Cuando llegaron, entraron en la sinagoga judía.

\bibleverse{11} Estos eran más nobles que los de Tesalónica, pues
recibieron la palabra con toda prontitud, examinando cada día las
Escrituras para ver si estas cosas eran así. \footnote{\textbf{17:11}
  Juan 5,39} \bibleverse{12} Por lo tanto, muchos de ellos creyeron;
también de las mujeres griegas prominentes, y no pocos hombres.
\bibleverse{13} Pero cuando los judíos de Tesalónica tuvieron
conocimiento de que la palabra de Dios era proclamada por Pablo también
en Berea, acudieron también allí, agitando a las multitudes.
\bibleverse{14} Entonces los hermanos enviaron inmediatamente a Pablo
para que fuera hasta el mar, y Silas y Timoteo se quedaron allí.
\footnote{\textbf{17:14} Hech 16,1} \bibleverse{15} Pero los que
acompañaban a Pablo lo llevaron hasta Atenas. Recibiendo la orden de
Silas y Timoteo de que fueran a verle muy pronto, partieron.

\hypertarget{pablo-en-atenas}{%
\subsection{Pablo en Atenas}\label{pablo-en-atenas}}

\bibleverse{16} Mientras Pablo los esperaba en Atenas, su espíritu se
encendió en su interior al ver la ciudad llena de ídolos.
\bibleverse{17} Así que discutía en la sinagoga con los judíos y los
devotos, y en la plaza todos los días con los que se encontraban con él.
\bibleverse{18} También\footnote{\textbf{17:18} TR omite ``también''}
conversaban con él algunos filósofos epicúreos y estoicos. Algunos
decían: ``¿Qué quiere decir este charlatán?''. Otros dijeron: ``Parece
que aboga por deidades extranjeras'', porque predicaba a Jesús y la
resurrección. \footnote{\textbf{17:18} 1Cor 4,12}

\bibleverse{19} Se apoderaron de él y lo llevaron al Areópago, diciendo:
``¿Podemos saber qué es esta nueva enseñanza de la que hablas?
\bibleverse{20} Porque traes a nuestros oídos ciertas cosas extrañas.
Queremos, pues, saber qué significan estas cosas''. \bibleverse{21}
Ahora bien, todos los atenienses y los forasteros que vivían allí no
dedicaban su tiempo a otra cosa que a contar u oír alguna cosa nueva.

\hypertarget{discurso-de-pablo-en-el-cerro-del-areuxf3pago}{%
\subsection{Discurso de Pablo en el cerro del
Areópago}\label{discurso-de-pablo-en-el-cerro-del-areuxf3pago}}

\bibleverse{22} Pablo, de pie en medio del Areópago, dijo: ``Hombres de
Atenas, veo que sois muy religiosos en todo. \bibleverse{23} Pues al
pasar y observar los objetos de vuestro culto, encontré también un altar
con esta inscripción: ``A UN DIOS DESCONOCIDO''; por lo tanto, lo que
adoráis en la ignorancia, os lo anuncio. \bibleverse{24} El Dios que
hizo el mundo y todas las cosas que hay en él, siendo Señor del cielo y
de la tierra, no habita en templos hechos por manos. \footnote{\textbf{17:24}
  1Re 8,27} \bibleverse{25} No es servido por manos de hombres, como si
necesitara algo, ya que él mismo da a todos la vida y el aliento y todas
las cosas. \footnote{\textbf{17:25} Sal 50,9-12} \bibleverse{26} Hizo de
una sola sangre a todas las naciones de los hombres para que habitasen
en toda la superficie de la tierra, habiendo determinado las estaciones
y los límites de sus moradas, \footnote{\textbf{17:26} Deut 32,8}
\bibleverse{27} para que buscasen al Señor, por si acaso lo buscaban y
lo encontraban, aunque no está lejos de cada uno de nosotros.
\footnote{\textbf{17:27} Is 55,6} \bibleverse{28} ``Porque en él
vivimos, nos movemos y somos''. Como han dicho algunos de tus propios
poetas: ``Porque también somos su descendencia''. \bibleverse{29}
Siendo, pues, la descendencia de Dios, no debemos pensar que la
naturaleza divina es como el oro, o la plata, o la piedra, grabada por
arte y diseño del hombre. \footnote{\textbf{17:29} Gén 1,27; Is 40,18}
\bibleverse{30} Por eso, Dios pasó por alto los tiempos de la
ignorancia. Pero ahora manda que todos los hombres se arrepientan en
todas partes, \footnote{\textbf{17:30} Hech 14,16; Luc 24,47}
\bibleverse{31} porque ha fijado un día en el que juzgará al mundo con
justicia por medio del hombre que él ha ordenado; de lo cual ha dado
seguridad a todos los hombres, en que lo ha resucitado de entre los
muertos.'' \footnote{\textbf{17:31} Hech 10,42; Mat 25,31-33}

\bibleverse{32} Al oír hablar de la resurrección de los muertos, algunos
se burlaban; pero otros decían: ``Queremos oírte otra vez sobre esto''.

\bibleverse{33} Así, Pablo salió de entre ellos. \bibleverse{34} Pero
algunos hombres se unieron a él y creyeron, entre ellos Dionisio el
Areopagita, y una mujer llamada Damaris, y otros con ellos.

\hypertarget{pablo-en-corinto}{%
\subsection{Pablo en Corinto}\label{pablo-en-corinto}}

\hypertarget{section-17}{%
\section{18}\label{section-17}}

\bibleverse{1} Después de estas cosas, Pablo partió de Atenas y llegó a
Corinto. \bibleverse{2} Encontró a un judío llamado Aquila, de raza del
Ponto, que había llegado recientemente de Italia con su mujer Priscila,
porque Claudio había ordenado a todos los judíos que salieran de Roma.
Llegó a ellos, \footnote{\textbf{18:2} Rom 16,3} \bibleverse{3} y como
ejercía el mismo oficio, vivió con ellos y trabajó, pues de oficio eran
fabricantes de tiendas. \footnote{\textbf{18:3} Hech 20,34; 1Cor 4,12}
\bibleverse{4} Todos los sábados razonaba en la sinagoga y persuadía a
judíos y griegos.

\bibleverse{5} Cuando Silas y Timoteo bajaron de Macedonia, Pablo fue
impulsado por el Espíritu, testificando a los judíos que Jesús era el
Cristo. \footnote{\textbf{18:5} Hech 17,14-15; 2Cor 1,19} \bibleverse{6}
Cuando se opusieron a él y blasfemaron, sacudió su ropa y les dijo:
``¡Su sangre caiga sobre sus propias cabezas! Yo estoy limpio. A partir
de ahora, iré a los gentiles''. \footnote{\textbf{18:6} Hech 13,51; Hech
  20,26}

\bibleverse{7} Salió de allí y entró en casa de un hombre llamado Justo,
que adoraba a Dios, cuya casa estaba al lado de la sinagoga.
\bibleverse{8} Crispo, el jefe de la sinagoga, creyó en el Señor con
toda su casa. Muchos de los corintios, al oírlo, creyeron y se
bautizaron. \footnote{\textbf{18:8} 1Cor 1,14; 1Cor 1,19} \bibleverse{9}
El Señor le dijo a Pablo en una visión nocturna: ``No tengas miedo,
habla y no te calles; \footnote{\textbf{18:9} 1Cor 2,3} \bibleverse{10}
porque yo estoy contigo y nadie te atacará para hacerte daño, pues tengo
mucha gente en esta ciudad.'' \footnote{\textbf{18:10} Jer 1,8; Juan
  10,16}

\bibleverse{11} Vivió allí un año y seis meses, enseñando la palabra de
Dios entre ellos.

\hypertarget{la-acusaciuxf3n-contra-los-juduxedos-fue-rechazada-por-el-gobernador-galiuxf3n}{%
\subsection{La acusación contra los judíos fue rechazada por el
gobernador
Galión}\label{la-acusaciuxf3n-contra-los-juduxedos-fue-rechazada-por-el-gobernador-galiuxf3n}}

\bibleverse{12} Pero cuando Galión era procónsul de Acaya, los judíos,
de común acuerdo, se levantaron contra Pablo y lo llevaron ante el
tribunal, \bibleverse{13} diciendo: ``Este hombre persuade a los hombres
a adorar a Dios en contra de la ley.''

\bibleverse{14} Pero cuando Pablo estaba a punto de abrir la boca,
Galión dijo a los judíos: ``Si en verdad se tratara de un asunto
incorrecto o de un delito inicuo, vosotros los judíos, sería razonable
que yo os soportara; \footnote{\textbf{18:14} Hech 25,18-20}
\bibleverse{15} pero si se trata de cuestiones de palabras y nombres y
de vuestra propia ley, miradlo vosotros mismos. Porque no quiero ser
juez de estos asuntos''. \footnote{\textbf{18:15} Juan 18,31}
\bibleverse{16} Así que los expulsó del tribunal.

\bibleverse{17} Entonces todos los griegos agarraron a Sóstenes, el jefe
de la sinagoga, y lo golpearon ante el tribunal. A Galio no le importó
nada de esto.

\hypertarget{regreso-de-pablo-vuxeda-uxe9feso-y-judea-a-antioquuxeda-en-siria}{%
\subsection{Regreso de Pablo vía Éfeso y Judea a Antioquía en
Siria}\label{regreso-de-pablo-vuxeda-uxe9feso-y-judea-a-antioquuxeda-en-siria}}

\bibleverse{18} Después de esto, Pablo se despidió de los hermanos y se
embarcó de allí hacia Siria, junto con Priscila y Aquila. En Cencreas se
afeitó la cabeza, pues tenía un voto. \footnote{\textbf{18:18} Hech
  21,24; Núm 6,2; Núm 6,5; Núm 6,13; Núm 6,18} \bibleverse{19} Llegó a
Éfeso y los dejó allí; pero él mismo entró en la sinagoga y discutió con
los judíos. \bibleverse{20} Cuando le pidieron que se quedara con ellos
más tiempo, lo rechazó; \bibleverse{21} pero despidiéndose de ellos, les
dijo: ``Tengo que celebrar esta próxima fiesta en Jerusalén, pero
volveré de nuevo a vosotros si Dios quiere.'' Entonces partió de Éfeso.
\footnote{\textbf{18:21} Sant 4,15}

\bibleverse{22} Cuando desembarcó en Cesarea, subió a saludar a la
asamblea y bajó a Antioquía. \footnote{\textbf{18:22} Hech 21,15}

\hypertarget{inicio-del-viaje-apolos-en-uxe9feso-y-corinto}{%
\subsection{Inicio del viaje; Apolos en Éfeso y
Corinto}\label{inicio-del-viaje-apolos-en-uxe9feso-y-corinto}}

\bibleverse{23} Después de pasar algún tiempo allí, partió y recorrió
por orden la región de Galacia y Frigia, estableciendo a todos los
discípulos. \bibleverse{24} Llegó a Éfeso un judío llamado Apolos, de
raza alejandrina, hombre elocuente. Era poderoso en las Escrituras.
\footnote{\textbf{18:24} 1Cor 3,5-6} \bibleverse{25} Este hombre había
sido instruido en el camino del Señor; y siendo ferviente de espíritu,
hablaba y enseñaba con exactitud las cosas relativas a Jesús, aunque
sólo conocía el bautismo de Juan. \footnote{\textbf{18:25} Hech 19,3}
\bibleverse{26} Comenzó a hablar con valentía en la sinagoga. Pero
cuando Priscila y Aquila le oyeron, le llevaron aparte y le explicaron
con más precisión el camino de Dios.

\bibleverse{27} Cuando decidió pasar a Acaya, los hermanos le animaron y
escribieron a los discípulos para que le recibieran. Cuando llegó, ayudó
mucho a los que habían creído por medio de la gracia; \bibleverse{28}
pues refutó poderosamente a los judíos, mostrando públicamente con las
Escrituras que Jesús era el Cristo. \footnote{\textbf{18:28} Hech 9,22;
  Hech 17,3}

\hypertarget{conversiuxf3n-y-bautismo-de-los-discuxedpulos-de-juan}{%
\subsection{Conversión y bautismo de los discípulos de
Juan}\label{conversiuxf3n-y-bautismo-de-los-discuxedpulos-de-juan}}

\hypertarget{section-18}{%
\section{19}\label{section-18}}

\bibleverse{1} Mientras Apolos estaba en Corinto, Pablo, habiendo pasado
por la zona alta, llegó a Éfeso y encontró a algunos discípulos.
\bibleverse{2} Les dijo: ``¿Recibisteis el Espíritu Santo cuando
creísteis?'' Le dijeron: ``No, ni siquiera hemos oído que exista el
Espíritu Santo''. \footnote{\textbf{19:2} Hech 2,38}

\bibleverse{3} Él dijo: ``¿En qué fuisteis bautizados?'' Dijeron: ``En
el bautismo de Juan''.

\bibleverse{4} Pablo dijo: ``Juan, en efecto, bautizó con el bautismo
del arrepentimiento, diciendo a la gente que debía creer en el que
vendría después de él, es decir, en Cristo Jesús.'' \footnote{\textbf{19:4}
  NU omite a Cristo.} \footnote{\textbf{19:4} Mat 3,11}

\bibleverse{5} Al oír esto, se bautizaron en el nombre del Señor Jesús.
\bibleverse{6} Cuando Pablo les impuso las manos, el Espíritu Santo vino
sobre ellos y hablaron en otras lenguas y profetizaron. \footnote{\textbf{19:6}
  Hech 8,17; Hech 10,44; Hech 10,46} \bibleverse{7} Eran unos doce
hombres en total.

\hypertarget{la-actividad-de-dos-auxf1os-de-enseuxf1anza-y-milagros-de-pablo-en-uxe9feso}{%
\subsection{La actividad de dos años de enseñanza y milagros de Pablo en
Éfeso}\label{la-actividad-de-dos-auxf1os-de-enseuxf1anza-y-milagros-de-pablo-en-uxe9feso}}

\bibleverse{8} Entró en la sinagoga y habló con valentía durante tres
meses, razonando y persuadiendo sobre las cosas relativas al Reino de
Dios.

\bibleverse{9} Pero como algunos estaban endurecidos y desobedientes,
hablando mal del Camino ante la multitud, se apartó de ellos y separó a
los discípulos, razonando cada día en la escuela de Tirano.
\bibleverse{10} Esto continuó durante dos años, de modo que todos los
que vivían en Asia oyeron la palabra del Señor Jesús, tanto judíos como
griegos.

\bibleverse{11} Dios obró milagros especiales por las manos de Pablo,
\footnote{\textbf{19:11} Hech 14,3; 2Cor 12,12} \bibleverse{12} de modo
que hasta los pañuelos o delantales se llevaban de su cuerpo a los
enfermos, y las enfermedades se iban de ellos, y los espíritus malignos
salían. \footnote{\textbf{19:12} Hech 5,15}

\hypertarget{superar-la-supersticiuxf3n-los-invocadores-y-los-libros-de-hechizos}{%
\subsection{Superar la superstición (Los invocadores y los libros de
hechizos)}\label{superar-la-supersticiuxf3n-los-invocadores-y-los-libros-de-hechizos}}

\bibleverse{13} Pero algunos de los judíos itinerantes, exorcistas, se
encargaron de invocar sobre los que tenían los espíritus malignos el
nombre del Señor Jesús, diciendo: ``Os conjuramos por Jesús que Pablo
predica.'' \footnote{\textbf{19:13} Luc 9,49} \bibleverse{14} Había
siete hijos de un tal Esceva, jefe de los sacerdotes judíos, que hacían
esto.

\bibleverse{15} El espíritu maligno respondió: ``A Jesús lo conozco, y a
Pablo lo conozco, pero vosotros ¿quiénes sois?'' \bibleverse{16} El
hombre en el que estaba el espíritu maligno saltó sobre ellos, los
dominó y los venció, de modo que huyeron de aquella casa desnudos y
heridos. \bibleverse{17} Esto fue conocido por todos, tanto judíos como
griegos, que vivían en Éfeso. El temor cayó sobre todos ellos, y el
nombre del Señor Jesús fue magnificado. \bibleverse{18} También vinieron
muchos de los que habían creído, confesando y declarando sus hechos.
\bibleverse{19} Muchos de los que practicaban artes mágicas reunieron
sus libros y los quemaron a la vista de todos. Contaron su precio, y
encontraron que era de cincuenta mil piezas de plata. \footnote{\textbf{19:19}
  Las 50.000 piezas de plata aquí probablemente se referían a 50.000
  dracmas. Si es así, el valor de los libros quemados equivalía a unos
  160 años-hombre de salario para los trabajadores agrícolas}
\bibleverse{20} Así, la palabra del Señor crecía y se hacía poderosa.
\footnote{\textbf{19:20} Hech 12,24}

\hypertarget{planes-de-viaje-de-pablo}{%
\subsection{Planes de viaje de Pablo}\label{planes-de-viaje-de-pablo}}

\bibleverse{21} Una vez terminadas estas cosas, Pablo determinó en el
Espíritu, cuando pasó por Macedonia y Acaya, ir a Jerusalén, diciendo:
``Después de haber estado allí, debo ver también Roma.'' \footnote{\textbf{19:21}
  Hech 23,11}

\bibleverse{22} Habiendo enviado a Macedonia a dos de los que le
servían, Timoteo y Erasto, él mismo se quedó en Asia por un tiempo.
\footnote{\textbf{19:22} 2Tim 4,20}

\hypertarget{el-motuxedn-de-los-plateros-de-demetrio}{%
\subsection{El motín de los plateros de
Demetrio}\label{el-motuxedn-de-los-plateros-de-demetrio}}

\bibleverse{23} Por aquel tiempo se produjo un disturbio no pequeño en
relación con el Camino. \footnote{\textbf{19:23} 2Cor 1,8-9}
\bibleverse{24} Porque cierto hombre llamado Demetrio, platero que hacía
santuarios de plata de Artemisa, llevó un negocio no pequeño a los
artesanos, \bibleverse{25} a los que reunió con los obreros de ocupación
similar, y les dijo: ``Señores, sabéis que con este negocio tenemos
nuestra riqueza. \bibleverse{26} Vosotros veis y oís que no sólo en
Éfeso, sino casi en toda Asia, este Pablo ha persuadido y alejado a
mucha gente, diciendo que no son dioses los que se hacen con las manos.
\bibleverse{27} No sólo existe el peligro de que este nuestro comercio
caiga en descrédito, sino también de que el templo de la gran diosa
Artemisa sea contado como nada y su majestad destruida, a la que toda
Asia y el mundo adoran.''

\bibleverse{28} Al oír esto, se llenaron de ira y gritaron diciendo:
``¡Grande es Artemisa de los efesios!'' \bibleverse{29} Toda la ciudad
se llenó de confusión y se precipitaron al teatro al unísono, habiendo
apresado a Gayo y Aristarco, hombres de Macedonia, compañeros de viaje
de Pablo. \footnote{\textbf{19:29} Hech 20,4} \bibleverse{30} Cuando
Pablo quiso entrar al pueblo, los discípulos no se lo permitieron.
\bibleverse{31} También algunos de los asiarcas, siendo amigos suyos, le
enviaron a rogarle que no se aventurara en el teatro. \bibleverse{32}
Así pues, unos gritaban una cosa y otros otra, pues la asamblea estaba
confundida. La mayoría no sabía por qué se habían reunido.
\bibleverse{33} Hicieron salir a Alejandro de entre la multitud,
adelantándose los judíos. Alejandro hizo una seña con la mano, y hubiera
querido hacer una defensa ante el pueblo. \bibleverse{34} Pero cuando se
dieron cuenta de que era judío, todos a una voz, durante un tiempo de
unas dos horas, gritaron: ``¡Grande es Artemisa de los efesios!''

\bibleverse{35} Cuando el secretario municipal hubo calmado a la
multitud, dijo: ``Hombres de Éfeso, ¿qué hombre hay que no sepa que la
ciudad de los efesios es templo de la gran diosa Artemisa y de la imagen
que cayó de Zeus? \bibleverse{36} Viendo, pues, que estas cosas no se
pueden negar, debéis estar tranquilos y no hacer nada precipitado.
\bibleverse{37} Pues habéis traído aquí a estos hombres, que no son ni
ladrones de templos ni blasfemos de vuestra diosa. \bibleverse{38} Por
tanto, si Demetrio y los artesanos que están con él tienen algún asunto
contra alguien, los tribunales están abiertos y hay procónsules. Que se
acusen unos a otros. \bibleverse{39} Pero si buscan algo sobre otros
asuntos, se resolverá en la asamblea ordinaria. \bibleverse{40} Porque,
en efecto, corremos el peligro de ser acusados en relación con el motín
de hoy, sin que haya ninguna causa. Con respecto a ella, no podríamos
dar cuenta de este alboroto''. \bibleverse{41} Cuando hubo hablado así,
despidió a la asamblea.

\hypertarget{viaje-a-grecia-y-regresa-a-troas}{%
\subsection{Viaje a Grecia y regresa a
Troas}\label{viaje-a-grecia-y-regresa-a-troas}}

\hypertarget{section-19}{%
\section{20}\label{section-19}}

\bibleverse{1} Cuando cesó el alboroto, Pablo mandó llamar a los
discípulos, se despidió de ellos y partió para ir a Macedonia.
\footnote{\textbf{20:1} 2Cor 2,13} \bibleverse{2} Después de recorrer
aquellas tierras y de animarles con muchas palabras, llegó a Grecia.
\bibleverse{3} Después de haber pasado tres meses allí, y cuando estaba
a punto de embarcarse para Siria, los judíos tramaron un complot contra
él, por lo que decidió volver por Macedonia. \bibleverse{4} Estos le
acompañaron hasta Asia: Sópater de Berea, Aristarco y Segundo de los
tesalonicenses, Gayo de Derbe, Timoteo, y Tíquico y Trófimo de Asia.
\footnote{\textbf{20:4} Hech 17,10; Hech 19,29; Hech 16,1; Hech 21,29;
  Efes 6,21} \bibleverse{5} Pero éstos se habían adelantado y nos
esperaban en Troas. \bibleverse{6} Zarpamos de Filipos después de los
días de los Panes sin Levadura, y llegamos a ellos en Troas en cinco
días, donde permanecimos siete días.

\hypertarget{celebraciuxf3n-de-despedida-de-pablo-en-troas-reanimaciuxf3n-del-fallido-eutico}{%
\subsection{Celebración de despedida de Pablo en Troas; Reanimación del
fallido
Eutico}\label{celebraciuxf3n-de-despedida-de-pablo-en-troas-reanimaciuxf3n-del-fallido-eutico}}

\bibleverse{7} El primer día de la semana, cuando los discípulos estaban
reunidos para partir el pan, Pablo habló con ellos, con la intención de
partir al día siguiente; y continuó su discurso hasta la medianoche.
\footnote{\textbf{20:7} Hech 2,42; Hech 2,46; Mat 28,1} \bibleverse{8}
Había muchas luces en la sala superior donde \footnote{\textbf{20:8} TR
  lee ``ellos'' en lugar de ``nosotros''} estábamos reunidos.
\bibleverse{9} Un joven llamado Eutico estaba sentado en la ventana,
agobiado por un profundo sueño. Como Pablo seguía hablando, agobiado por
el sueño, se cayó del tercer piso y lo subieron muerto. \bibleverse{10}
Pablo bajó, se echó sobre él y, abrazándolo, le dijo: ``No te preocupes,
porque su vida está en él.'' \footnote{\textbf{20:10} 1Re 17,21}

\bibleverse{11} Cuando subió, partió el pan y comió, y habló con ellos
un largo rato, hasta el amanecer, se fue. \bibleverse{12} Trajeron al
muchacho vivo, y se consolaron mucho.

\hypertarget{el-viaje-de-pablo-de-troas-a-mileto}{%
\subsection{El viaje de Pablo de Troas a
Mileto}\label{el-viaje-de-pablo-de-troas-a-mileto}}

\bibleverse{13} Pero nosotros, adelantándonos a la nave, zarpamos hacia
Assos, con la intención de embarcar allí a Pablo, pues él así lo había
dispuesto, con la intención de ir por tierra. \bibleverse{14} Cuando se
encontró con nosotros en Assos, lo subimos a bordo y llegamos a
Mitilene. \bibleverse{15} Partiendo de allí, llegamos al día siguiente
frente a Quíos. Al día siguiente tocamos en Samos y nos quedamos en
Trogilio, y al día siguiente llegamos a Mileto. \bibleverse{16} Porque
Pablo había decidido navegar más allá de Éfeso, para no tener que pasar
tiempo en Asia, pues se apresuraba, si le era posible, a estar en
Jerusalén el día de Pentecostés. \footnote{\textbf{20:16} Hech 18,21}

\hypertarget{encuentro-de-pablo-con-los-ancianos-de-uxe9feso-en-mileto-su-discurso-de-despedida-y-su-despedida}{%
\subsection{Encuentro de Pablo con los ancianos de Éfeso en Mileto; su
discurso de despedida y su
despedida}\label{encuentro-de-pablo-con-los-ancianos-de-uxe9feso-en-mileto-su-discurso-de-despedida-y-su-despedida}}

\bibleverse{17} Desde Mileto envió a Éfeso y llamó a los ancianos de la
asamblea. \bibleverse{18} Cuando vinieron a él, les dijo: ``Vosotros
mismos sabéis, desde el primer día que puse el pie en Asia, cómo estuve
con vosotros todo el tiempo, \footnote{\textbf{20:18} Hech 18,19; Hech
  19,10} \bibleverse{19} sirviendo al Señor con toda humildad, con
muchas lágrimas y con pruebas que me sucedieron por las conspiraciones
de los judíos; \bibleverse{20} cómo no rehusé declararos todo lo que era
provechoso, enseñándoos públicamente y de casa en casa, \bibleverse{21}
testificando tanto a judíos como a griegos el arrepentimiento para con
Dios y la fe en nuestro Señor Jesús. \footnote{\textbf{20:21} TR añade
  ``Cristo''} \bibleverse{22} Ahora bien, he aquí que voy atado por el
Espíritu a Jerusalén, sin saber lo que me sucederá allí; \footnote{\textbf{20:22}
  Hech 19,21} \bibleverse{23} salvo que el Espíritu Santo da testimonio
en cada ciudad, diciendo que me esperan prisiones y aflicciones.
\footnote{\textbf{20:23} Hech 9,16; Hech 21,4; Hech 21,11}
\bibleverse{24} Pero estas cosas no cuentan, ni estimo mi vida, para
terminar mi carrera con alegría, y el ministerio que recibí del Señor
Jesús, para dar pleno testimonio de la Buena Nueva de la gracia de Dios.
\footnote{\textbf{20:24} Hech 21,13; 2Tim 4,7}

\bibleverse{25} ``Ahora, he aquí, sé que todos vosotros, entre los que
anduve predicando el Reino de Dios, no veréis más mi rostro.
\bibleverse{26} Por tanto, hoy os testifico que estoy limpio de la
sangre de todos los hombres, \footnote{\textbf{20:26} Hech 18,6; Ezeq
  3,17-19} \bibleverse{27} pues no he rehuido declararos todo el consejo
de Dios. \bibleverse{28} Velad, pues, por vosotros mismos y por todo el
rebaño, en el que el Espíritu Santo os ha puesto como pastores de la
asamblea del Señor y\footnote{\textbf{20:28} TR, NU omiten ``el Señor
  y''} Dios, que él adquirió con su propia sangre. \footnote{\textbf{20:28}
  1Tim 4,16; 1Pe 5,2-4} \bibleverse{29} Porque sé que, después de mi
partida, entrarán entre vosotros lobos rapaces que no perdonarán al
rebaño. \footnote{\textbf{20:29} Mat 7,15} \bibleverse{30} Se levantarán
hombres de entre vosotros, hablando cosas perversas, para arrastrar a
los discípulos tras ellos. \footnote{\textbf{20:30} 1Jn 2,18; 1Jn 1,2-19}
\bibleverse{31} Velad, pues, recordando que durante tres años no dejé de
amonestar a todos noche y día con lágrimas. \bibleverse{32} Ahora,
hermanos, os encomiendo a Dios y a la palabra de su gracia, que es capaz
de edificar y daros la herencia entre todos los santificados.
\bibleverse{33} No he codiciado la plata, el oro ni la ropa de nadie.
\bibleverse{34} Vosotros mismos sabéis que estas manos sirvieron a mis
necesidades, y a las de los que estaban conmigo. \footnote{\textbf{20:34}
  Hech 18,3; 1Cor 4,12; 1Tes 2,9} \bibleverse{35} En todo os he dado
ejemplo de que, trabajando así, debéis ayudar a los débiles, y recordar
las palabras del Señor Jesús, que él mismo dijo: ``Más bienaventurado es
dar que recibir''.''

\bibleverse{36} Después de decir estas cosas, se arrodilló y oró con
todos ellos. \footnote{\textbf{20:36} Hech 21,5} \bibleverse{37} Todos
lloraban a lágrima viva, se echaban al cuello de Pablo y lo besaban,
\bibleverse{38} apenados sobre todo por la palabra que había dicho de no
ver más su rostro. Luego lo acompañaron a la nave.

\hypertarget{continuaciuxf3n-del-viaje-de-mileto-a-tiro-y-cesarea}{%
\subsection{Continuación del viaje de Mileto a Tiro y
Cesarea}\label{continuaciuxf3n-del-viaje-de-mileto-a-tiro-y-cesarea}}

\hypertarget{section-20}{%
\section{21}\label{section-20}}

\bibleverse{1} Cuando nos alejamos de ellos, zarpamos y navegamos, con
rumbo, directo a Cos, y al día siguiente a Rodas, y de allí a Patara.
\bibleverse{2} Habiendo encontrado un barco que cruzaba a Fenicia,
subimos a bordo y nos hicimos a la mar. \bibleverse{3} Cuando llegamos a
la vista de Chipre, dejándola a la izquierda, navegamos hacia Siria y
desembarcamos en Tiro, pues la nave estaba allí para descargar su carga.
\bibleverse{4} Habiendo encontrado discípulos, nos quedamos allí siete
días. Estos dijeron a Pablo por el Espíritu que no subiera a Jerusalén.
\footnote{\textbf{21:4} Hech 20,23} \bibleverse{5} Pasados esos días,
partimos y nos pusimos en camino. Todos ellos, con esposas e hijos, nos
acompañaron en nuestro camino hasta que salimos de la ciudad.
Arrodillados en la playa, oramos. \footnote{\textbf{21:5} Hech 20,36}
\bibleverse{6} Después de despedirnos unos de otros, subimos a bordo del
barco, y ellos volvieron a casa.

\bibleverse{7} Cuando terminamos el viaje desde Tiro, llegamos a
Tolemaida. Saludamos a los hermanos y nos quedamos con ellos un día.
\bibleverse{8} Al día siguiente, los que éramos compañeros de Pablo
partimos y llegamos a Cesarea. Entramos en casa de Felipe el
evangelista, que era uno de los siete, y nos quedamos con él.
\footnote{\textbf{21:8} Hech 6,5; Hech 8,40} \bibleverse{9} Este hombre
tenía cuatro hijas vírgenes que profetizaban. \bibleverse{10} Mientras
permanecíamos allí algunos días, bajó de Judea un profeta llamado Agabo.
\footnote{\textbf{21:10} Hech 11,28} \bibleverse{11} Viniendo a nosotros
y tomando el cinturón de Pablo, se ató los pies y las manos, y dijo:
``El Espíritu Santo dice: `Así los judíos de Jerusalén atarán al hombre
que tiene este cinturón y lo entregarán en manos de los gentiles'.''
\footnote{\textbf{21:11} Hech 20,23}

\bibleverse{12} Al oír estas cosas, tanto nosotros como la gente de
aquel lugar le rogamos que no subiera a Jerusalén. \footnote{\textbf{21:12}
  Mat 16,22} \bibleverse{13} Entonces Pablo respondió: ``¿Qué hacéis
llorando y rompiendo mi corazón? Porque estoy dispuesto no sólo a ser
atado, sino también a morir en Jerusalén por el nombre del Señor
Jesús.'' \footnote{\textbf{21:13} Hech 20,24}

\bibleverse{14} Como no se dejaba persuadir, cesamos, diciendo: ``Hágase
la voluntad del Señor''. \footnote{\textbf{21:14} Luc 22,42}

\hypertarget{pablo-en-jerusaluxe9n-y-preso-en-cesarea}{%
\subsection{Pablo en Jerusalén y preso en
Cesarea}\label{pablo-en-jerusaluxe9n-y-preso-en-cesarea}}

\bibleverse{15} Pasados estos días, tomamos nuestro equipaje y subimos a
Jerusalén. \bibleverse{16} Algunos de los discípulos de Cesárea también
fueron con nosotros, trayendo a un tal Mnasón de Chipre, discípulo de
los primeros tiempos, con quien nos quedaríamos.

\bibleverse{17} Cuando llegamos a Jerusalén, los hermanos nos recibieron
de buen grado. \bibleverse{18} Al día siguiente, Pablo entró con
nosotros en casa de Santiago, y estaban presentes todos los ancianos.
\footnote{\textbf{21:18} Hech 15,13} \bibleverse{19} Después de
saludarlos, les contó una por una las cosas que Dios había obrado entre
los gentiles por medio de su ministerio. \bibleverse{20} Ellos, al
oírlo, glorificaron a Dios. Le dijeron: ``Ya ves, hermano, cuántos miles
hay entre los judíos de los que han creído, y todos son celosos de la
ley. \footnote{\textbf{21:20} Hech 15,1} \bibleverse{21} Se han
informado acerca de ti, que enseñas a todos los judíos que están entre
los gentiles a abandonar a Moisés, diciéndoles que no circunciden a sus
hijos y que no sigan las costumbres. \footnote{\textbf{21:21} Hech 16,3}
\bibleverse{22} ¿Qué, pues? La asamblea debe reunirse ciertamente,
porque oirán que has venido. \bibleverse{23} Haced, pues, lo que os
decimos. Tenemos cuatro hombres que han hecho un voto. \bibleverse{24}
Tómalos y purifícate con ellos, y paga sus gastos por ellos, para que se
afeiten la cabeza. Entonces todos sabrán que no hay verdad en las cosas
que se les ha informado acerca de ti, sino que tú también andas
cumpliendo la ley. \footnote{\textbf{21:24} Hech 18,18} \bibleverse{25}
Pero en cuanto a los gentiles que creen, hemos escrito nuestra decisión
de que no observen tal cosa, sino que se guarden de la comida ofrecida a
los ídolos, de la sangre, de las cosas estranguladas y de la inmoralidad
sexual.'' \footnote{\textbf{21:25} Hech 15,20; Hech 15,29}

\bibleverse{26} Entonces Pablo tomó a los hombres, y al día siguiente se
purificó y entró con ellos en el templo, declarando el cumplimiento de
los días de purificación, hasta que se ofreció la ofrenda por cada uno
de ellos. \footnote{\textbf{21:26} Núm 6,1-20; 1Cor 9,20}

\hypertarget{pablo-arrestado-por-los-juduxedos-en-el-templo-el-levantamiento-en-jerusaluxe9n}{%
\subsection{Pablo arrestado por los judíos en el templo; el
levantamiento en
Jerusalén}\label{pablo-arrestado-por-los-juduxedos-en-el-templo-el-levantamiento-en-jerusaluxe9n}}

\bibleverse{27} Cuando casi se habían cumplido los siete días, los
judíos de Asia, al verle en el templo, agitaron a toda la multitud y le
echaron mano, \bibleverse{28} gritando: ``¡Hombres de Israel, ayudad!
Este es el hombre que enseña a todos los hombres en todas partes contra
el pueblo, y la ley, y este lugar. Además, ¡también ha metido griegos en
el templo y ha profanado este lugar santo!'' \footnote{\textbf{21:28}
  Hech 6,13; Ezeq 44,7} \bibleverse{29} Porque habían visto a Trófimo,
el efesio, con él en la ciudad, y suponían que Pablo lo había
introducido en el templo. \footnote{\textbf{21:29} Hech 20,4; 2Tim 4,20}

\bibleverse{30} Toda la ciudad se conmovió y el pueblo corrió en masa.
Agarraron a Pablo y lo arrastraron fuera del templo. Inmediatamente se
cerraron las puertas.

\hypertarget{captura-de-pablo-por-el-coronel-romano-lisias}{%
\subsection{Captura de Pablo por el coronel romano
Lisias}\label{captura-de-pablo-por-el-coronel-romano-lisias}}

\bibleverse{31} Mientras intentaban matarlo, llegó la noticia al
comandante del regimiento de que toda Jerusalén estaba alborotada.
\bibleverse{32} Inmediatamente tomó soldados y centuriones y corrió
hacia ellos. Ellos, al ver al jefe del regimiento y a los soldados,
dejaron de golpear a Pablo. \bibleverse{33} Entonces el comandante se
acercó, lo arrestó, mandó que lo ataran con dos cadenas y preguntó quién
era y qué había hecho. \footnote{\textbf{21:33} Hech 20,23}
\bibleverse{34} Unos gritaban una cosa y otros otra, entre la multitud.
Como no pudo averiguar la verdad a causa del ruido, mandó que lo
llevaran al cuartel.

\bibleverse{35} Cuando llegó a la escalera, fue llevado por los soldados
a causa de la violencia de la muchedumbre; \bibleverse{36} pues la
multitud del pueblo lo seguía, gritando: ``¡Fuera!'' \footnote{\textbf{21:36}
  Hech 22,22; Luc 23,18} \bibleverse{37} Cuando Pablo estaba a punto de
ser llevado al cuartel, preguntó al oficial al mando: ``¿Puedo hablar
con usted?'' Dijo: ``¿Sabes griego? \bibleverse{38} ¿No eres tú entonces
el egipcio que antes de estos días incitó a la sedición y condujo al
desierto a los cuatro mil hombres de los Asesinos?''

\bibleverse{39} Pero Pablo dijo: ``Soy un judío de Tarso de Cilicia,
ciudadano de una ciudad nada insignificante. Te ruego que me permitas
hablar al pueblo''.

\bibleverse{40} Cuando le dio permiso, Pablo, de pie en la escalera,
hizo una señal con la mano a la gente. Cuando se hizo un gran silencio,
les habló en lengua hebrea, diciendo

\hypertarget{el-discurso-de-pablo-al-pueblo}{%
\subsection{El discurso de Pablo al
pueblo}\label{el-discurso-de-pablo-al-pueblo}}

\hypertarget{section-21}{%
\section{22}\label{section-21}}

\bibleverse{1} ``Hermanos y padres, escuchad la defensa que ahora os
hago''.

\bibleverse{2} Cuando oyeron que les hablaba en lengua hebrea, se
callaron aún más. Dijo: \footnote{\textbf{22:2} Hech 21,40}
\bibleverse{3} ``En verdad soy judío, nacido en Tarso de Cilicia, pero
criado en esta ciudad a los pies de Gamaliel, instruido según la
estricta tradición de la ley de nuestros padres, siendo celoso de Dios,
como lo sois todos vosotros hoy. \footnote{\textbf{22:3} Hech 5,34; Hech
  9,1-29; Hech 26,9-20} \bibleverse{4} Perseguí a este Camino hasta la
muerte, atando y entregando en las cárceles tanto a hombres como a
mujeres, \footnote{\textbf{22:4} Hech 8,3} \bibleverse{5} como también
lo atestiguan el sumo sacerdote y todo el consejo de ancianos, de
quienes también recibí cartas para los hermanos, y viajé a Damasco para
llevar también a Jerusalén a los que estaban allí atados para ser
castigados.

\bibleverse{6} ``Mientras hacía mi viaje y me acercaba a Damasco, hacia
el mediodía, una gran luz brilló a mi alrededor desde el cielo.
\bibleverse{7} Caí al suelo y oí una voz que me decía: ``Saulo, Saulo,
¿por qué me persigues?'' \bibleverse{8} Respondí: ``¿Quién eres,
Señor?'' Me dijo: ``Yo soy Jesús de Nazaret, a quien tú persigues''.

\bibleverse{9} ``Los que estaban conmigo, en efecto, vieron la luz y
tuvieron miedo, pero no entendieron la voz del que me hablaba.
\bibleverse{10} Yo dije: ``¿Qué debo hacer, Señor? El Señor me dijo:
`Levántate y ve a Damasco. Allí se te informará de todo lo que está
previsto que hagas'. \bibleverse{11} Cuando no podía ver por la gloria
de aquella luz, siendo conducido de la mano de los que estaban conmigo,
entré en Damasco.

\bibleverse{12} ``Un tal Ananías, hombre piadoso según la ley, del que
tenían buena fama todos los judíos que vivían en Damasco,
\bibleverse{13} se acercó a mí y, de pie, me dijo: ``Hermano Saulo,
recibe la vista''. En aquella misma hora le miré. \bibleverse{14} Me
dijo: ``El Dios de nuestros padres te ha designado para que conozcas su
voluntad, para que veas al Justo y oigas la voz de su boca.
\bibleverse{15} Porque serás testigo de él ante todos los hombres de lo
que has visto y oído. \bibleverse{16} Ahora, ¿por qué esperáis?
Levántate, bautízate y lava tus pecados, invocando el nombre del Señor'.

\bibleverse{17} ``Cuando volví a Jerusalén y mientras oraba en el
templo, caí en un trance \bibleverse{18} y vi que me decía: `Date prisa
y sal de Jerusalén rápidamente, porque no recibirán el testimonio de ti
sobre mí'. \bibleverse{19} Le dije: `Señor, ellos mismos saben que yo
encarcelé y golpeé en todas las sinagogas a los que creían en ti.
\bibleverse{20} Cuando se derramó la sangre de Esteban, tu testigo, yo
también estaba de pie, consintiendo su muerte y guardando los mantos de
los que lo mataron.' \footnote{\textbf{22:20} Hech 7,57; Hech 8,1}

\bibleverse{21} ``Me dijo: `Vete, porque te enviaré lejos de aquí a los
gentiles'\,''. \footnote{\textbf{22:21} Hech 13,2}

\hypertarget{el-efecto-del-habla-pablo-bajo-custodia-con-el-coronel-romano}{%
\subsection{El efecto del habla; Pablo bajo custodia con el coronel
romano}\label{el-efecto-del-habla-pablo-bajo-custodia-con-el-coronel-romano}}

\bibleverse{22} Le escucharon hasta que dijo eso; entonces levantaron la
voz y dijeron: ``¡Limpia la tierra de este tipo, porque no es apto para
vivir!'' \footnote{\textbf{22:22} Hech 21,36}

\bibleverse{23} Mientras gritaban, se quitaban los mantos y arrojaban
polvo al aire, \bibleverse{24} el comandante mandó que lo llevaran al
cuartel, ordenando que lo examinaran mediante la flagelación, para saber
por qué delito gritaban así contra él. \bibleverse{25} Cuando le ataron
con correas, Pablo preguntó al centurión que estaba allí: ``¿Os es
lícito azotar a un hombre que es romano y no ha sido declarado
culpable?'' \footnote{\textbf{22:25} Hech 16,37; Hech 23,27}

\bibleverse{26} Cuando el centurión lo oyó, se dirigió al oficial al
mando y le dijo: ``¡Cuidado con lo que vas a hacer, porque este hombre
es un romano!''

\bibleverse{27} El oficial al mando se acercó y le preguntó: ``Dime,
¿eres romano?'' Dijo: ``Sí''.

\bibleverse{28} El comandante respondió: ``Compré mi ciudadanía a un
gran precio''. Pablo dijo: ``Pero yo nací romano''.

\bibleverse{29} Inmediatamente se apartaron de él los que iban a
interrogarle, y también el comandante tuvo miedo al ver que era romano,
porque le había atado.

\hypertarget{pablo-ante-el-sumo-consejo-juduxedo}{%
\subsection{Pablo ante el sumo consejo
judío}\label{pablo-ante-el-sumo-consejo-juduxedo}}

\bibleverse{30} Pero al día siguiente, deseando saber la verdad sobre el
motivo por el que era acusado por los judíos, le liberó de las ataduras
y mandó reunir a los jefes de los sacerdotes y a todo el consejo, e hizo
bajar a Pablo y lo presentó ante ellos.

\hypertarget{section-22}{%
\section{23}\label{section-22}}

\bibleverse{1} Pablo, mirando fijamente al consejo, dijo: ``Hermanos,
hasta hoy he vivido ante Dios con toda la buena conciencia.''
\footnote{\textbf{23:1} Hech 24,16}

\bibleverse{2} El sumo sacerdote, Ananías, ordenó a los que estaban
junto a él que le golpearan en la boca.

\bibleverse{3} Entonces Pablo le dijo: ``¡Dios te va a golpear a ti,
muro blanqueado! ¿Te sientas a juzgarme según la ley, y mandas que me
golpeen en contra de la ley?'' \footnote{\textbf{23:3} Mat 23,37}

\bibleverse{4} Los que estaban de pie dijeron: ``¿Maltratas al sumo
sacerdote de Dios?''

\bibleverse{5} Pablo dijo: ``No sabía, hermanos, que era sumo sacerdote.
Porque está escrito: `No hablarás mal de un gobernante de tu
pueblo\footnote{\textbf{23:5} Éxodo 22:28} '\,''.

\bibleverse{6} Pero cuando Pablo se dio cuenta de que una parte eran
saduceos y la otra fariseos, gritó en el concilio: ``Hombres y hermanos,
yo soy fariseo, hijo de fariseos. En cuanto a la esperanza y la
resurrección de los muertos estoy siendo juzgado''. \footnote{\textbf{23:6}
  Hech 22,3; Hech 26,5; Gal 1,14}

\bibleverse{7} Al decir esto, surgió una discusión entre fariseos y
saduceos, y la multitud se dividió. \bibleverse{8} Porque los saduceos
dicen que no hay resurrección, ni ángel, ni espíritu; pero los fariseos
confiesan todo esto. \footnote{\textbf{23:8} Mat 22,23} \bibleverse{9}
Se armó un gran alboroto, y algunos de los escribas de la parte de los
fariseos se levantaron y discutieron diciendo: ``No encontramos ningún
mal en este hombre. Pero si un espíritu o un ángel le ha hablado, ¡no
luchemos contra Dios!'' \footnote{\textbf{23:9} Hech 25,25; Hech 5,39}

\bibleverse{10} Cuando se produjo una gran discusión, el oficial al
mando, temiendo que Pablo fuera despedazado por ellos, ordenó a los
soldados que bajaran y lo sacaran por la fuerza de entre ellos y lo
llevaran al cuartel.

\bibleverse{11} La noche siguiente, el Señor se puso a su lado y le
dijo: ``Anímate, Pablo, porque así como has dado testimonio de mí en
Jerusalén, también debes darlo en Roma.'' \footnote{\textbf{23:11} Hech
  25,11-12; Hech 27,23-24}

\hypertarget{intento-de-asesinato-de-los-juduxedos-contra-pablo}{%
\subsection{Intento de asesinato de los judíos contra
Pablo}\label{intento-de-asesinato-de-los-juduxedos-contra-pablo}}

\bibleverse{12} Cuando se hizo de día, algunos de los judíos se
agruparon y se obligaron bajo una maldición, diciendo que no comerían ni
beberían hasta que hubieran matado a Pablo. \bibleverse{13} Eran más de
cuarenta los que habían hecho esta conspiración. \bibleverse{14} Se
presentaron ante los jefes de los sacerdotes y los ancianos y dijeron:
``Nos hemos obligado bajo una gran maldición a no probar nada hasta que
hayamos matado a Pablo. \bibleverse{15} Ahora, pues, vosotros, con el
consejo, informad al comandante para que lo haga bajar a vosotros
mañana, como si fuerais a juzgar su caso con más exactitud. Estamos
dispuestos a matarlo antes de que se acerque''.

\bibleverse{16} Pero el hijo de la hermana de Pablo se enteró de que
estaban al acecho y, entrando en el cuartel, se lo comunicó a Pablo.
\bibleverse{17} Pablo llamó a uno de los centuriones y le dijo: ``Lleva
a este joven ante el oficial al mando, porque tiene algo que decirle.''

\bibleverse{18} Así que lo tomó y lo llevó al oficial al mando y le
dijo: ``Pablo, el prisionero, me convocó y me pidió que le trajera a
este joven. Tiene algo que decirle''.

\bibleverse{19} El comandante le tomó de la mano y, apartándose, le
preguntó en privado: ``¿Qué es lo que tienes que decirme?''.

\bibleverse{20} Dijo: ``Los judíos han acordado pedirte que mañana
lleves a Pablo al consejo, como si tuvieran la intención de indagar algo
más sobre él. \bibleverse{21} Por tanto, no cedas ante ellos, pues le
acechan más de cuarenta hombres que se han obligado bajo maldición a no
comer ni beber hasta que le hayan matado. Ahora están preparados,
esperando la promesa de tu parte''.

\bibleverse{22} Entonces el comandante dejó ir al joven, encargándole:
``No digas a nadie que me has revelado estas cosas''.

\hypertarget{carta-del-coronel-lysias-al-gobernador-fuxe9lix-traslado-de-pablo-de-jerusaluxe9n-a-cesarea}{%
\subsection{Carta del coronel Lysias al gobernador Félix; Traslado de
Pablo de Jerusalén a
Cesarea}\label{carta-del-coronel-lysias-al-gobernador-fuxe9lix-traslado-de-pablo-de-jerusaluxe9n-a-cesarea}}

\bibleverse{23} Llamó a dos de los centuriones y les dijo: ``Preparad
doscientos soldados para ir hasta Cesarea, con setenta jinetes y
doscientos hombres armados con lanzas, a la tercera hora de la noche.''
\footnote{\textbf{23:23} alrededor de las 21:00 h.} \bibleverse{24} Les
pidió que le proporcionaran monturas, para que montaran a Pablo en una
de ellas y lo llevaran sano y salvo a Félix, el gobernador.
\bibleverse{25} Escribió una carta como ésta:

\bibleverse{26} ``Claudio Lisias al excelentísimo gobernador Félix:
Saludos.

\bibleverse{27} ``Este hombre fue apresado por los judíos y estaba a
punto de ser asesinado por ellos, cuando llegué con los soldados y lo
rescaté, tras saber que era romano. \footnote{\textbf{23:27} Hech 21,33;
  Hech 22,25} \bibleverse{28} Deseando saber la causa por la que lo
acusaban, lo llevé a su consejo. \footnote{\textbf{23:28} Hech 22,30}
\bibleverse{29} Encontré que lo acusaban de cuestiones de su ley, pero
no de nada digno de muerte o de prisión. \footnote{\textbf{23:29} Hech
  18,14-15} \bibleverse{30} Cuando me dijeron que los judíos estaban al
acecho del hombre, lo envié inmediatamente a ti, encargando también a
sus acusadores que presentaran sus acusaciones contra él ante ti.
Adiós''. \footnote{\textbf{23:30} Hech 24,8}

\bibleverse{31} Así que los soldados, cumpliendo sus órdenes, tomaron a
Pablo y lo llevaron de noche a Antipatris. \bibleverse{32} Pero al día
siguiente dejaron a los jinetes que lo acompañaban y volvieron al
cuartel. \bibleverse{33} Cuando llegaron a Cesarea y entregaron la carta
al gobernador, también le presentaron a Pablo. \bibleverse{34} Cuando el
gobernador la leyó, le preguntó de qué provincia era. Al comprender que
era de Cilicia, dijo: \footnote{\textbf{23:34} Hech 22,3}
\bibleverse{35} ``Te escucharé plenamente cuando lleguen también tus
acusadores''. Y ordenó que lo recluyeran en el palacio de Herodes.

\hypertarget{juicio-ante-el-gobernador-fuxe9lix}{%
\subsection{Juicio ante el gobernador
Félix}\label{juicio-ante-el-gobernador-fuxe9lix}}

\hypertarget{section-23}{%
\section{24}\label{section-23}}

\bibleverse{1} Al cabo de cinco días, el sumo sacerdote Ananías bajó con
algunos ancianos y un orador, un tal Tértulo. Informaron al gobernador
contra Pablo. \bibleverse{2} Cuando lo llamaron, Tertulio comenzó a
acusarlo, diciendo: ``Viendo que por ti gozamos de mucha paz y que la
prosperidad llega a esta nación por tu previsión, \bibleverse{3} lo
aceptamos de todas las maneras y en todos los lugares, excelentísimo
Félix, con todo agradecimiento. \bibleverse{4} Pero para no retrasaros,
os ruego que tengáis paciencia con nosotros y escuchéis unas palabras.
\bibleverse{5} Porque hemos descubierto que este hombre es una plaga,
instigador de insurrecciones entre todos los judíos del mundo, y
cabecilla de la secta de los nazarenos. \footnote{\textbf{24:5} Hech
  17,6} \bibleverse{6} Incluso intentó profanar el templo, y lo
arrestamos. \footnote{\textbf{24:6} TR añade ``Queríamos juzgarlo según
  nuestra ley''.} \footnote{\textbf{24:6} Hech 21,28-29} \bibleverse{7}
\footnote{\textbf{24:7} El TR añade ``pero el oficial al mando, Lisias,
  se acercó y con gran violencia lo arrebató de nuestras manos''.}
\bibleverse{8} \footnote{\textbf{24:8} TR añade ``ordenando a sus
  acusadores que vengan a ti''.} Examinándolo tú mismo podrás comprobar
todas estas cosas de las que lo acusamos.'' \footnote{\textbf{24:8} Hech
  21,17}

\bibleverse{9} Los judíos también se unieron al ataque, afirmando que
estas cosas eran así.

\bibleverse{10} Cuando el gobernador le hizo una señal para que hablara,
Pablo respondió: ``Como sé que tú eres juez de esta nación desde hace
muchos años, hago alegremente mi defensa, \bibleverse{11} ya que puedes
comprobar que no hace más de doce días que subí a adorar a Jerusalén.
\bibleverse{12} En el templo no me encontraron disputando con nadie ni
agitando a la multitud, ni en las sinagogas ni en la ciudad.
\bibleverse{13} Tampoco pueden probaros las cosas de las que ahora me
acusan. \bibleverse{14} Pero esto os confieso: que según el Camino, al
que llaman secta, así sirvo al Dios de nuestros padres, creyendo en todo
lo que es conforme a la ley y lo que está escrito en los profetas;
\bibleverse{15} teniendo esperanza en Dios, que también éstos esperan,
de que habrá una resurrección de los muertos, tanto de los justos como
de los injustos. \footnote{\textbf{24:15} Dan 12,2; Juan 5,28-29}
\bibleverse{16} En esto también practico teniendo siempre una conciencia
libre de ofensas para con Dios y los hombres. \footnote{\textbf{24:16}
  Hech 23,1} \bibleverse{17} Después de algunos años, vine a traer dones
para los necesitados de mi nación, y ofrendas; \footnote{\textbf{24:17}
  Rom 15,25-26; Gal 2,10} \bibleverse{18} en medio de lo cual algunos
judíos de Asia me encontraron purificado en el templo, no con una turba,
ni con alboroto. \footnote{\textbf{24:18} Hech 21,27} \bibleverse{19}
Deberían haber estado aquí antes que tú y haber hecho la acusación si
tenían algo contra mí. \bibleverse{20} O bien, que sean estos mismos los
que digan qué injusticia encontraron en mí cuando me presenté ante el
concilio, \bibleverse{21} a no ser que sea por esta única cosa por la
que grité de pie en medio de ellos: ``¡Acerca de la resurrección de los
muertos estoy siendo juzgado hoy ante vosotros!'' \footnote{\textbf{24:21}
  Hech 23,6}

\bibleverse{22} Pero Félix, que tenía un conocimiento más exacto del
Camino, los aplazó diciendo: ``Cuando baje Lisias, el oficial al mando,
decidiré tu caso.'' \footnote{\textbf{24:22} Hech 23,26} \bibleverse{23}
Ordenó al centurión que Pablo fuera custodiado y tuviera algunos
privilegios, y que no prohibiera a ninguno de sus amigos servirle o
visitarle. \footnote{\textbf{24:23} Hech 27,3}

\hypertarget{pablo-ante-felix-y-drusilla-felix-retrasuxf3-el-juicio}{%
\subsection{Pablo ante Felix y Drusilla; Felix retrasó el
juicio}\label{pablo-ante-felix-y-drusilla-felix-retrasuxf3-el-juicio}}

\bibleverse{24} Al cabo de algunos días, Félix vino con su esposa
Drusila, que era judía, y mandó llamar a Pablo para oírle acerca de la
fe en Cristo Jesús. \bibleverse{25} Al razonar sobre la justicia, el
dominio propio y el juicio que ha de venir, Félix se aterrorizó y
respondió: ``Vete por ahora, y cuando me convenga, te convocaré.''
\bibleverse{26} Mientras tanto, también esperaba que Pablo le diera
dinero para poder liberarlo. Por eso también le mandó llamar más a
menudo y habló con él.

\bibleverse{27} Pero cuando se cumplieron dos años, Félix fue sucedido
por Porcio Festo, y deseando ganarse el favor de los judíos, Félix dejó
a Pablo en prisión.

\hypertarget{reanudaciuxf3n-del-proceso-festo-en-jerusaluxe9n-y-cesarea-pablo-apela-al-emperador}{%
\subsection{Reanudación del proceso; Festo en Jerusalén y Cesarea; Pablo
apela al
emperador}\label{reanudaciuxf3n-del-proceso-festo-en-jerusaluxe9n-y-cesarea-pablo-apela-al-emperador}}

\hypertarget{section-24}{%
\section{25}\label{section-24}}

\bibleverse{1} Festo, pues, habiendo llegado a la provincia, después de
tres días subió a Jerusalén desde Cesarea. \bibleverse{2} Entonces el
sumo sacerdote y los principales hombres de los judíos le informaron
contra Pablo, y le rogaron, \footnote{\textbf{25:2} Hech 24,1}
\bibleverse{3} pidiendo un favor contra él, que lo convocara a
Jerusalén, tramando matarlo en el camino. \footnote{\textbf{25:3} Hech
  23,15} \bibleverse{4} Sin embargo, Festo respondió que Pablo debía ser
custodiado en Cesarea, y que él mismo iba a partir en breve.
\bibleverse{5} ``Dejad, pues, que bajen conmigo los que están en el
poder entre vosotros, y si hay algo malo en el hombre, que lo acusen.''

\bibleverse{6} Después de haber permanecido entre ellos más de diez
días, bajó a Cesarea, y al día siguiente se sentó en el tribunal y mandó
traer a Pablo. \bibleverse{7} Cuando llegó, los judíos que habían bajado
de Jerusalén se pusieron a su alrededor, presentando contra él muchas y
graves acusaciones que no podían probar, \bibleverse{8} mientras él
decía en su defensa: ``Ni contra la ley de los judíos, ni contra el
templo, ni contra el César, he pecado en absoluto.''

\bibleverse{9} Pero Festo, deseando ganarse el favor de los judíos,
respondió a Pablo y le dijo: ``¿Estás dispuesto a subir a Jerusalén y
ser juzgado por mí allí respecto a estas cosas?''

\bibleverse{10} Pero Pablo dijo: ``Estoy ante el tribunal del César,
donde debo ser juzgado. No he hecho ningún mal a los judíos, como tú
también sabes muy bien. \bibleverse{11} Pues si he obrado mal y he
cometido algo digno de muerte, no me niego a morir; pero si no es cierto
nada de lo que me acusan, nadie puede entregarme a ellos. Apelo al
César''. \footnote{\textbf{25:11} Hech 23,11; Hech 28,19}

\bibleverse{12} Entonces Festo, tras consultar con el consejo,
respondió: ``Habéis apelado al César. Al César irás''.

\hypertarget{herodes-agripa-ii-y-berenice-como-invitados-en-festo-en-cesarea-festo-informa-a-agripa-de-la-causa-de-pablo}{%
\subsection{Herodes Agripa II y Berenice como invitados en Festo en
Cesarea; Festo informa a Agripa de la causa de
Pablo}\label{herodes-agripa-ii-y-berenice-como-invitados-en-festo-en-cesarea-festo-informa-a-agripa-de-la-causa-de-pablo}}

\bibleverse{13} Transcurridos algunos días, el rey Agripa y Berenice
llegaron a Cesarea y saludaron a Festo. \bibleverse{14} Como permaneció
allí muchos días, Festo expuso el caso de Pablo ante el rey, diciendo:
``Hay un hombre que Félix dejó preso; \footnote{\textbf{25:14} Hech
  24,27} \bibleverse{15} sobre el cual, estando yo en Jerusalén, me
informaron los jefes de los sacerdotes y los ancianos de los judíos,
pidiendo que se le condenara. \bibleverse{16} Les respondí que los
romanos no acostumbran a entregar a ningún hombre a la muerte antes de
que el acusado se haya encontrado cara a cara con los acusadores y haya
tenido la oportunidad de defenderse del asunto que se le imputa.
\footnote{\textbf{25:16} Hech 22,25} \bibleverse{17} Así pues, cuando se
reunieron aquí, no me demoré, sino que al día siguiente me senté en el
tribunal y ordené que se trajera al hombre. \bibleverse{18} Cuando los
acusadores se pusieron en pie, no presentaron contra él ninguna
acusación de las que yo suponía; \bibleverse{19} sino que tenían ciertas
preguntas contra él sobre su propia religión y sobre un tal Jesús, que
estaba muerto, del que Pablo afirmaba que estaba vivo. \footnote{\textbf{25:19}
  Hech 18,15} \bibleverse{20} Estando perplejo sobre cómo preguntar
sobre estas cosas, le pregunté si estaba dispuesto a ir a Jerusalén y
ser juzgado allí sobre estos asuntos. \bibleverse{21} Pero cuando Pablo
apeló a que se le retuviera para la decisión del emperador, ordené que
se le retuviera hasta que pudiera enviarlo al César.''

\bibleverse{22} Agripa le dijo a Festo: ``Yo también quisiera escuchar a
ese hombre''. ``Mañana'', dijo, ``lo escucharás''. \footnote{\textbf{25:22}
  Luc 23,8}

\hypertarget{discurso-de-manifestaciuxf3n-y-defensa-de-pablo-frente-a-agripa-y-festo}{%
\subsection{Discurso de manifestación y defensa de Pablo frente a Agripa
y
Festo}\label{discurso-de-manifestaciuxf3n-y-defensa-de-pablo-frente-a-agripa-y-festo}}

\bibleverse{23} Al día siguiente, cuando Agripa y Berenice vinieron con
gran pompa, y entraron en el lugar de la audiencia con los comandantes y
los principales hombres de la ciudad, por orden de Festo, Pablo fue
introducido. \bibleverse{24} Festo dijo: ``Rey Agripa, y todos los
hombres que están aquí presentes con nosotros, veis a este hombre sobre
el cual toda la multitud de los judíos me ha hecho peticiones, tanto en
Jerusalén como aquí, clamando que no debe vivir más. \footnote{\textbf{25:24}
  Hech 22,22} \bibleverse{25} Pero cuando comprobé que no había cometido
nada digno de muerte, y como él mismo apeló al emperador, decidí
enviarlo, \bibleverse{26} de quien no tengo nada seguro que escribir a
mi señor. Por eso lo he traído ante ti, y especialmente ante ti, rey
Agripa, para que, después de examinarlo, tenga algo que escribir.
\bibleverse{27} Porque me parece poco razonable, al enviar a un
prisionero, no especificar también los cargos que se le imputan.''

\hypertarget{discurso-defensivo-de-pablo-ante-agripa}{%
\subsection{Discurso defensivo de Pablo ante
Agripa}\label{discurso-defensivo-de-pablo-ante-agripa}}

\hypertarget{section-25}{%
\section{26}\label{section-25}}

\bibleverse{1} Agripa dijo a Pablo: ``Puedes hablar por ti mismo''.
Entonces Pablo extendió la mano e hizo su defensa. \bibleverse{2} ``Me
considero feliz, rey Agripa, de poder hacer hoy mi defensa ante ti de
todo lo que me acusan los judíos, \bibleverse{3} especialmente porque
eres experto en todas las costumbres y cuestiones que hay entre los
judíos. Por eso te ruego que me escuches con paciencia.

\bibleverse{4} ``En efecto, todos los judíos conocen mi modo de vida
desde mi juventud, que fue desde el principio entre mi propia nación y
en Jerusalén; \bibleverse{5} habiéndome conocido desde el principio, si
están dispuestos a testificar, que según la secta más estricta de
nuestra religión viví como fariseo. \footnote{\textbf{26:5} Hech 23,6;
  Fil 3,5} \bibleverse{6} Ahora estoy aquí para ser juzgado por la
esperanza de la promesa hecha por Dios a nuestros padres, \footnote{\textbf{26:6}
  Hech 28,20} \bibleverse{7} que nuestras doce tribus, sirviendo
fervientemente noche y día, esperan alcanzar. Sobre esta esperanza me
acusan los judíos, rey Agripa. \footnote{\textbf{26:7} Hech 24,15}
\bibleverse{8} ¿Por qué se juzga increíble para ti que Dios resucite a
los muertos? \footnote{\textbf{26:8} Hech 23,8}

\bibleverse{9} ``Yo mismo pensé que debía hacer muchas cosas contrarias
al nombre de Jesús de Nazaret. \footnote{\textbf{26:9} Hech 9,1-29; Hech
  22,3-21} \bibleverse{10} También hice esto en Jerusalén. Encerré a
muchos de los santos en las cárceles, habiendo recibido autoridad de los
sumos sacerdotes; y cuando fueron condenados a muerte, di mi voto contra
ellos. \bibleverse{11} Castigándolos a menudo en todas las sinagogas,
traté de hacerlos blasfemar. Enfurecido en extremo contra ellos, los
perseguí hasta en ciudades extranjeras.

\bibleverse{12} ``Entonces, mientras viajaba a Damasco con la autoridad
y la comisión de los jefes de los sacerdotes, \bibleverse{13} al
mediodía, oh rey, vi en el camino una luz del cielo, más brillante que
el sol, que me rodeaba a mí y a los que viajaban conmigo.
\bibleverse{14} Cuando todos caímos a tierra, oí una voz que me decía en
lengua hebrea: ``Saúl, Saúl, ¿por qué me persigues? Es difícil para ti
dar coces contra los aguijones'.

\bibleverse{15} ``Dije: `¿Quién eres, Señor? ``Dijo: `Yo soy Jesús, a
quien tú persigues. \bibleverse{16} Pero levántate y ponte en pie,
porque para esto me he aparecido a ti: para ponerte como servidor y
testigo tanto de las cosas que has visto como de las que te voy a
revelar; \bibleverse{17} para librarte del pueblo y de los gentiles, a
quienes te envío, \bibleverse{18} para abrirles los ojos, a fin de que
se conviertan de las tinieblas a la luz y del poder de Satanás a Dios,
para que reciban la remisión de los pecados y la herencia entre los
santificados por la fe en mí.' \footnote{\textbf{26:18} Hech 20,32}

\bibleverse{19} ``Por eso, rey Agripa, no fui desobediente a la visión
celestial, \footnote{\textbf{26:19} Gal 1,16} \bibleverse{20} sino que
declaré primero a los de Damasco, a los de Jerusalén y a los de todo el
país de Judea, y también a los gentiles, que se arrepintieran y se
convirtieran a Dios, haciendo obras dignas de arrepentimiento.
\bibleverse{21} Por eso los judíos me apresaron en el templo y trataron
de matarme. \footnote{\textbf{26:21} Hech 21,30-31} \bibleverse{22} Por
lo tanto, habiendo obtenido la ayuda que viene de Dios, estoy hasta el
día de hoy dando testimonio tanto a pequeños como a grandes, no diciendo
más que lo que los profetas y Moisés dijeron que sucedería, \footnote{\textbf{26:22}
  Luc 24,44-47} \bibleverse{23} cómo el Cristo debía sufrir y cómo, por
la resurrección de los muertos, sería el primero en anunciar la luz
tanto a este pueblo como a los gentiles.'' \footnote{\textbf{26:23} 1Cor
  15,20}

\hypertarget{impresiuxf3n-del-discurso}{%
\subsection{Impresión del discurso}\label{impresiuxf3n-del-discurso}}

\bibleverse{24} Mientras hacía su defensa, Festo dijo en voz alta:
``¡Pablo, estás loco! Tu gran aprendizaje te está volviendo loco''.

\bibleverse{25} Pero él dijo: ``No estoy loco, excelentísimo Festo, sino
que declaro audazmente palabras de verdad y razonables. \bibleverse{26}
Porque el rey sabe de estas cosas, a quien también hablo libremente.
Porque estoy persuadido de que nada de esto se le oculta, pues esto no
se ha hecho en un rincón. \footnote{\textbf{26:26} Juan 18,20}
\bibleverse{27} Rey Agripa, ¿crees en los profetas? Yo sé que tú
crees''.

\bibleverse{28} Agripa dijo a Pablo: ``¿Con un poco de persuasión
pretendes hacerme cristiano?''

\bibleverse{29} Pablo dijo: ``Ruego a Dios que, ya sea con poco o con
mucho, no sólo ustedes, sino también todos los que me escuchan hoy,
lleguen a ser como yo, excepto estas ataduras.''

\bibleverse{30} El rey se levantó con el gobernador y Berenice, y los
que estaban sentados con ellos. \bibleverse{31} Cuando se retiraron,
hablaron entre sí, diciendo: ``Este hombre no hace nada digno de muerte
ni de prisión.'' \bibleverse{32} Agripa dijo a Festo: ``Este hombre
podría haber sido liberado si no hubiera apelado al César.'' \footnote{\textbf{26:32}
  Hech 25,11}

\hypertarget{el-viaje-de-pablo-de-cesarea-a-roma}{%
\subsection{El viaje de Pablo de Cesarea a
Roma}\label{el-viaje-de-pablo-de-cesarea-a-roma}}

\hypertarget{section-26}{%
\section{27}\label{section-26}}

\bibleverse{1} Cuando se decidió que nos embarcáramos para Italia,
entregaron a Pablo y a algunos otros prisioneros a un centurión llamado
Julio, de la banda de Augusto. \footnote{\textbf{27:1} Hech 25,12}
\bibleverse{2} Embarcándonos en una nave de Adramitrio, que se disponía
a navegar hacia lugares de la costa de Asia, nos hicimos a la mar,
estando con nosotros Aristarco, macedonio de Tesalónica. \footnote{\textbf{27:2}
  Hech 20,4} \bibleverse{3} Al día siguiente llegamos en Sidón. Julio
trató a Pablo con amabilidad y le dio permiso para ir a ver a sus amigos
y refrescarse. \footnote{\textbf{27:3} Hech 24,23; Hech 28,16}
\bibleverse{4} Haciéndonos a la mar desde allí, navegamos a sotavento de
Chipre, porque los vientos eran contrarios. \bibleverse{5} Después de
navegar por el mar que da a Cilicia y Panfilia, llegamos a Myra, ciudad
de Licia. \bibleverse{6} Allí el centurión encontró una nave de
Alejandría que navegaba hacia Italia, y nos subió a bordo.
\bibleverse{7} Después de haber navegado lentamente durante muchos días
y de haber llegado con dificultad frente a Cnidus, ya que el viento no
nos permitía avanzar, navegamos a sotavento de Creta, frente a Salmone.
\bibleverse{8} Navegando con dificultad a lo largo de ella, llegamos a
un lugar llamado ``Fair Havens'', cerca de la ciudad de Lasea.

\bibleverse{9} Cuando pasó mucho tiempo y el viaje era ya peligroso,
porque ya había pasado el Rápido, Pablo los amonestó \footnote{\textbf{27:9}
  2Cor 11,25-26; Lev 16,29} \bibleverse{10} y les dijo: ``Señores, veo
que el viaje será con perjuicio y mucha pérdida, no sólo de la carga y
de la nave, sino también de nuestras vidas.'' \bibleverse{11} Pero el
centurión hizo más caso al patrón y al dueño de la nave que a lo dicho
por Pablo. \bibleverse{12} Como el puerto no era apto para invernar, la
mayoría aconsejó hacerse a la mar desde allí, si por algún medio podían
llegar a Fénix e invernar allí, que es un puerto de Creta, mirando al
suroeste y al noroeste.

\hypertarget{tormenta-marina-y-naufragio-rescate-en-malta}{%
\subsection{Tormenta marina y naufragio; Rescate en
Malta}\label{tormenta-marina-y-naufragio-rescate-en-malta}}

\bibleverse{13} Cuando el viento del sur sopló suavemente, suponiendo
que habían conseguido su propósito, levaron anclas y navegaron a lo
largo de Creta, cerca de la costa. \bibleverse{14} Pero al poco tiempo,
un viento tempestuoso se abatió desde la orilla, lo que se llama
Euroclydon. \footnote{\textbf{27:14} O bien, ``un noreste''.}
\bibleverse{15} Cuando la nave quedó atrapada y no pudo hacer frente al
viento, cedimos a éste y fuimos conducidos. \bibleverse{16} Corriendo a
sotavento de una pequeña isla llamada Clauda, pudimos, con dificultad,
asegurar el barco. \bibleverse{17} Después de izarlo, utilizaron cables
para ayudar a reforzar el barco. Temiendo encallar en los bancos de
arena de Syrtis, bajaron el ancla de mar, y así fueron conducidos.
\bibleverse{18} Al día siguiente, mientras trabajábamos intensamente con
la tormenta, empezaron a tirar cosas por la borda. \bibleverse{19} Al
tercer día, echaron los aparejos de la nave con sus propias manos.
\bibleverse{20} Cuando ni el sol ni las estrellas brillaron sobre
nosotros durante muchos días, y no había una pequeña tormenta que nos
presionara, se desvaneció toda esperanza de que nos salváramos.

\hypertarget{pablo-como-consejero-consolador-y-salvador-en-angustia}{%
\subsection{Pablo como consejero, consolador y salvador en
angustia}\label{pablo-como-consejero-consolador-y-salvador-en-angustia}}

\bibleverse{21} Cuando llevaban mucho tiempo sin comer, Pablo se levantó
en medio de ellos y les dijo: ``Señores, deberíais haberme escuchado y
no haber zarpado de Creta y haber tenido este perjuicio y pérdida.
\bibleverse{22} Ahora os exhorto a que os animéis, pues no habrá pérdida
de vidas entre vosotros, sino sólo de la nave. \bibleverse{23} Porque
esta noche ha estado junto a mí un ángel, que pertenece al Dios del que
soy y al que sirvo, \bibleverse{24} diciendo: ``No temas, Pablo. Debes
presentarte ante el César. He aquí que Dios te ha concedido a todos los
que navegan contigo.' \footnote{\textbf{27:24} Hech 23,11}
\bibleverse{25} Por tanto, señores, ¡anímense! Porque yo creo en Dios,
que será tal como se me ha dicho. \bibleverse{26} Pero debemos encallar
en cierta isla''. \footnote{\textbf{27:26} Hech 28,1}

\bibleverse{27} Pero cuando llegó la decimocuarta noche, mientras íbamos
de un lado a otro del mar Adriático, hacia la medianoche los marineros
supusieron que se acercaban a alguna tierra. \bibleverse{28} Tomaron
sondeos y encontraron veinte brazas.\footnote{\textbf{27:28} 20 brazas =
  120 pies = 36,6 metros} Al cabo de un rato, volvieron a sondear y
encontraron quince brazas. \footnote{\textbf{27:28} 15 brazas = 90 pies
  = 27,4 metros} \bibleverse{29} Temiendo encallar en terreno rocoso,
soltaron cuatro anclas de la popa y desearon que se hiciera de día.
\bibleverse{30} Mientras los marineros intentaban huir de la nave y
habían echado la barca al mar, fingiendo que iban a echar las anclas por
la proa, \bibleverse{31} Pablo dijo al centurión y a los soldados: ``Si
éstos no se quedan en la nave, no podréis salvaros.'' \bibleverse{32}
Entonces los soldados cortaron las cuerdas de la barca y la dejaron
caer.

\bibleverse{33} Mientras se acercaba el día, Pablo les rogó a todos que
tomaran algo de comida, diciendo: ``Hoy es el decimocuarto día que
esperáis y seguís ayunando, sin haber tomado nada. \bibleverse{34} Por
lo tanto, os ruego que toméis algo de comida, porque esto es para
vuestra seguridad, ya que no perecerá ni un pelo de la cabeza de ninguno
de vosotros.'' \footnote{\textbf{27:34} Mat 10,30} \bibleverse{35}
Cuando dijo esto y tomó el pan, dio gracias a Dios en presencia de
todos; luego lo partió y comenzó a comer. \footnote{\textbf{27:35} Juan
  6,11} \bibleverse{36} Entonces todos se animaron y también tomaron
comida. \bibleverse{37} En total éramos doscientas setenta y seis
personas en la nave. \bibleverse{38} Cuando hubieron comido bastante,
aligeraron la nave, arrojando el trigo al mar.

\hypertarget{naufragio-en-la-faz-de-la-isla-de-malta-rescata-a-los-nuxe1ufragos}{%
\subsection{Naufragio en la faz de la isla de Malta; Rescata a los
náufragos}\label{naufragio-en-la-faz-de-la-isla-de-malta-rescata-a-los-nuxe1ufragos}}

\bibleverse{39} Cuando se hizo de día, no reconocieron la tierra, pero
se fijaron en cierta bahía con una playa, y decidieron intentar conducir
la nave hasta ella. \bibleverse{40} Echando las anclas, las dejaron en
el mar, desatando al mismo tiempo los cabos del timón. Levantando el
trinquete al viento, se dirigieron a la playa. \bibleverse{41} Pero al
llegar a un lugar donde confluían dos mares, encallaron la nave. La proa
golpeó y permaneció inmóvil, pero la popa comenzó a romperse por la
violencia de las olas.

\bibleverse{42} El consejo de los soldados era matar a los prisioneros,
para que ninguno de ellos saliera nadando y escapara. \bibleverse{43}
Pero el centurión, deseando salvar a Pablo, les impidió su propósito, y
ordenó que los que supieran nadar se arrojaran primero por la borda para
ir a tierra; \bibleverse{44} y que los demás los siguieran, unos en
tablas y otros en otras cosas de la nave. Así todos escaparon sanos y
salvos a tierra. \footnote{\textbf{27:44} Hech 27,22-25}

\hypertarget{invernada-en-la-isla-de-malta-continuaciuxf3n-del-viaje-a-roma}{%
\subsection{Invernada en la isla de Malta; Continuación del viaje a
Roma}\label{invernada-en-la-isla-de-malta-continuaciuxf3n-del-viaje-a-roma}}

\hypertarget{section-27}{%
\section{28}\label{section-27}}

\bibleverse{1} Cuando hubimos escapado, se \footnote{\textbf{28:1} NU se
  lee ``nosotros''} enteraron de que la isla se llamaba Malta.
\bibleverse{2} Los nativos nos mostraron una amabilidad poco común, pues
encendieron un fuego y nos recibieron a todos, a causa de la lluvia
presente y del frío. \footnote{\textbf{28:2} 2Cor 11,27}

\hypertarget{salvaciuxf3n-de-pablo-del-peligro-de-la-vida}{%
\subsection{Salvación de Pablo del peligro de la
vida}\label{salvaciuxf3n-de-pablo-del-peligro-de-la-vida}}

\bibleverse{3} Pero cuando Pablo reunió un manojo de palos y los puso
sobre el fuego, una víbora salió a causa del calor y se le prendió en la
mano. \bibleverse{4} Cuando los nativos vieron la criatura colgando de
su mano, se dijeron unos a otros: ``Sin duda este hombre es un asesino,
al que, aunque ha escapado del mar, la Justicia no ha dejado vivir.''
\bibleverse{5} Sin embargo, él se sacudió la criatura en el fuego, y no
sufrió ningún daño. \footnote{\textbf{28:5} Mar 16,18} \bibleverse{6}
Pero ellos esperaban que se hubiera hinchado o que hubiera caído muerto
de repente, pero cuando observaron durante mucho tiempo y vieron que no
le ocurría nada malo, cambiaron de opinión y dijeron que era un dios.
\footnote{\textbf{28:6} Hech 14,11}

\hypertarget{pablo-sana-al-padre-de-publio-y-a-otras-personas-enfermas}{%
\subsection{Pablo sana al padre de Publio y a otras personas
enfermas}\label{pablo-sana-al-padre-de-publio-y-a-otras-personas-enfermas}}

\bibleverse{7} En la vecindad de aquel lugar había tierras que
pertenecían al jefe de la isla, llamado Publio, quien nos recibió y nos
agasajó cortésmente durante tres días. \bibleverse{8} El padre de Publio
estaba enfermo de fiebre y disentería. Pablo entró en él, oró y,
imponiéndole las manos, le sanó. \bibleverse{9} Hecho esto, vinieron
también los demás enfermos de la isla y se curaron. \bibleverse{10}
También nos honraron con muchos honores; y cuando zarpamos, pusieron a
bordo las cosas que necesitábamos.

\hypertarget{continuaciuxf3n-del-viaje-vuxeda-siracusa-y-puteoli-hasta-roma}{%
\subsection{Continuación del viaje vía Siracusa y Puteoli hasta
Roma}\label{continuaciuxf3n-del-viaje-vuxeda-siracusa-y-puteoli-hasta-roma}}

\bibleverse{11} Al cabo de tres meses, zarpamos en una nave de
Alejandría que había invernado en la isla, cuyo mascarón de proa era
``Los hermanos gemelos''. \bibleverse{12} Al llegar a Siracusa,
permanecimos allí tres días. \bibleverse{13} Desde allí dimos la vuelta
y llegamos a Rhegium. Al cabo de un día, se levantó un viento del sur, y
al segundo día llegamos a Puteoli, \bibleverse{14} donde encontramos
hermanos, \footnote{\textbf{28:14} o, terminación, o fin} y nos rogaron
que nos quedáramos con ellos siete días. Así llegamos a Roma.
\bibleverse{15} Desde allí, los hermanos, al saber de nosotros, salieron
a nuestro encuentro hasta el Mercado de Apio y las Tres Tabernas. Al
verlos, Pablo dio gracias a Dios y se animó.

\hypertarget{pablo-en-roma}{%
\subsection{Pablo en Roma}\label{pablo-en-roma}}

\bibleverse{16} Cuando entramos en Roma, el centurión entregó los
prisioneros al capitán de la guardia, pero a Pablo se le permitió
quedarse solo con el soldado que lo custodiaba. \footnote{\textbf{28:16}
  Hech 27,3}

\hypertarget{negociaciones-de-pablo-con-los-jefes-de-los-juduxedos-romanos}{%
\subsection{Negociaciones de Pablo con los jefes de los judíos
romanos}\label{negociaciones-de-pablo-con-los-jefes-de-los-juduxedos-romanos}}

\bibleverse{17} Al cabo de tres días, Pablo convocó a los jefes de los
judíos. Cuando se reunieron, les dijo: ``Yo, hermanos, aunque no había
hecho nada contra el pueblo ni contra las costumbres de nuestros padres,
fui entregado prisionero desde Jerusalén en manos de los romanos,
\footnote{\textbf{28:17} Hech 23,1} \bibleverse{18} los cuales, después
de examinarme, quisieron ponerme en libertad, porque no había en mí
ninguna causa de muerte. \bibleverse{19} Pero cuando los judíos se
pronunciaron en contra, me vi obligado a apelar al César, sin tener nada
por lo que acusar a mi nación. \footnote{\textbf{28:19} Hech 25,11}
\bibleverse{20} Por eso pedí verte y hablar contigo. Porque a causa de
la esperanza de Israel estoy atado con esta cadena''. \footnote{\textbf{28:20}
  Hech 26,6-7}

\bibleverse{21} Le dijeron: ``No hemos recibido cartas de Judea acerca
de ti, ni ninguno de los hermanos ha venido a informar o a hablar mal de
ti. \bibleverse{22} Pero deseamos oír de ti lo que piensas. Porque, en
cuanto a esta secta, nos consta que en todas partes se habla mal de
ella.'' \footnote{\textbf{28:22} Hech 24,14; Luc 2,34}

\bibleverse{23} Cuando le señalaron un día, acudió mucha gente a su
alojamiento. Él les explicaba, testificando acerca del Reino de Dios, y
persuadiéndolos acerca de Jesús, tanto de la ley de Moisés como de los
profetas, desde la mañana hasta la noche. \bibleverse{24} Algunos
creyeron lo que se decía, y otros no creyeron. \bibleverse{25} Como no
se ponían de acuerdo entre sí, se marchaban después de que Pablo había
pronunciado un solo mensaje: ``El Espíritu Santo habló correctamente por
medio del profeta Isaías a nuestros padres, \bibleverse{26} diciendo,`Ve
a este pueblo y dile, en la audición, oirás, pero no lo entenderá de
ninguna manera. Al ver, verás, pero no percibirá de ninguna manera.
\bibleverse{27} Porque el corazón de este pueblo se ha vuelto
insensible. Sus oídos oyen con dificultad. Sus ojos se han cerrado. No
sea que vean con sus ojos, oigan con sus oídos, entiendan con el
corazón, y volvería a girar, entonces yo los sanaría'.

\bibleverse{28} ``Sabed, pues, que la salvación de Dios es enviada a las
naciones, y ellas escucharán''. \footnote{\textbf{28:28} Hech 13,46}

\bibleverse{29} Cuando dijo estas palabras, los judíos se marcharon,
teniendo una gran disputa entre ellos.

\hypertarget{el-ministerio-de-dos-auxf1os-de-pablo-en-cautiverio-en-roma}{%
\subsection{El ministerio de dos años de Pablo en cautiverio en
Roma}\label{el-ministerio-de-dos-auxf1os-de-pablo-en-cautiverio-en-roma}}

\bibleverse{30} Pablo permaneció dos años enteros en su propia casa
alquilada y recibía a todos los que venían a él, \bibleverse{31}
predicando el Reino de Dios y enseñando las cosas relativas al Señor
Jesucristo con toda valentía, sin obstáculos. \footnote{\textbf{28:31}
  Efes 6,20}
