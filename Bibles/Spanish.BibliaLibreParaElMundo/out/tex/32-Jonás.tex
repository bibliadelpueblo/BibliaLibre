\hypertarget{el-llamado-la-desobediencia-y-el-castigo-de-jonuxe1s}{%
\subsection{El llamado, la desobediencia y el castigo de
Jonás}\label{el-llamado-la-desobediencia-y-el-castigo-de-jonuxe1s}}

\hypertarget{section}{%
\section{1}\label{section}}

\bibleverse{1} La palabra de Yahvé llegó a Jonás, hijo de Amittai,
diciendo: \footnote{\textbf{1:1} 2Re 14,25} \bibleverse{2} ``Levántate,
ve a Nínive, esa gran ciudad, y predica contra ella, porque su maldad ha
subido ante mí.''

\bibleverse{3} Pero Jonás se levantó para huir a Tarsis de la presencia
de Yahvé. Bajó a Jope y encontró un barco que iba a Tarsis; pagó el
pasaje y bajó en él para ir con ellos a Tarsis de la presencia de Yahvé.
\footnote{\textbf{1:3} Sal 139,7; Sal 139,9-10}

\bibleverse{4} Pero Yahvé envió un gran viento sobre el mar, y se desató
una poderosa tormenta sobre el mar, de modo que la nave corría el riesgo
de romperse. \bibleverse{5} Entonces los marineros tuvieron miedo y cada
uno clamó a su dios. Arrojaron al mar la carga que había en la nave para
aligerarla. Pero Jonás había bajado al interior de la nave y se había
acostado, y estaba profundamente dormido. \bibleverse{6} El patrón de la
nave se acercó a él y le dijo: ``¿Qué quieres decir, dormilón?
¡Levántate, invoca a tu Dios! Tal vez tu Dios se fije en nosotros, para
que no perezcamos''.

\bibleverse{7} Todos se dijeron: ``¡Vengan! Echemos suertes, para saber
quién es el responsable de este mal que nos aqueja''. Así que echaron
suertes, y la suerte cayó sobre Jonás. \footnote{\textbf{1:7} Prov 16,33}
\bibleverse{8} Entonces le preguntaron: ``Dinos, por favor, por causa de
quién es este mal que nos aqueja. ¿Cuál es tu ocupación? ¿De dónde
vienes? ¿Cuál es tu país? ¿De qué pueblo eres?''

\bibleverse{9} Les dijo: ``Soy hebreo y temo a Yahvé, el Dios del cielo,
que ha hecho el mar y la tierra seca.'' \footnote{\textbf{1:9} Gén 1,9;
  Gén 1,19}

\bibleverse{10} Entonces los hombres tuvieron mucho miedo y le dijeron:
``¿Qué has hecho?''. Porque los hombres sabían que él huía de la
presencia de Yahvé, porque se lo había dicho. \bibleverse{11} Entonces
le dijeron: ``¿Qué te haremos para que el mar se calme para nosotros?''
Pues el mar se ponía cada vez más tempestuoso.

\bibleverse{12} Les dijo: ``Levántenme y arrójenme al mar. Entonces el
mar se calmará para vosotros; porque sé que por mi culpa os ha caído
esta gran tormenta''.

\bibleverse{13} Sin embargo, los hombres remaron con ahínco para
devolverlos a tierra, pero no pudieron, porque el mar se puso cada vez
más tempestuoso contra ellos. \bibleverse{14} Entonces clamaron a Yavé y
dijeron: ``Te rogamos, Yavé, te rogamos que no nos dejes morir por la
vida de este hombre, y que no hagas recaer sobre nosotros la sangre
inocente; porque tú, Yavé, has hecho lo que te ha parecido.''
\bibleverse{15} Entonces tomaron a Jonás y lo arrojaron al mar; y el mar
cesó su furia. \bibleverse{16} Entonces los hombres temieron mucho a
Yavé, ofrecieron un sacrificio a Yavé e hicieron votos.

\bibleverse{17} Yahvé preparó un enorme pez para que se tragara a Jonás,
y éste estuvo en el vientre del pez tres días y tres noches. \footnote{\textbf{1:17}
  Mat 12,40; Mat 16,4}

\hypertarget{jonuxe1s-oraciuxf3n-y-salvaciuxf3n}{%
\subsection{Jonás oración y
salvación}\label{jonuxe1s-oraciuxf3n-y-salvaciuxf3n}}

\hypertarget{section-1}{%
\section{2}\label{section-1}}

\bibleverse{1} Entonces Jonás oró a Yavé, su Dios, desde el vientre del
pez. \bibleverse{2} Dijo, ``Llamé a causa de mi aflicción a Yahvé. Me
respondió. Desde el vientre del Seol lloré. Has oído mi voz. \footnote{\textbf{2:2}
  Sal 120,1} \bibleverse{3} Porque me arrojaste a las profundidades, en
el corazón de los mares. El diluvio estaba a mi alrededor. Todas tus
olas y tus olas pasaron sobre mí. \footnote{\textbf{2:3} Sal 42,7}
\bibleverse{4} Dije: `He sido desterrado de tu vista; pero volveré a
mirar hacia tu santo templo''. \footnote{\textbf{2:4} Sal 31,22}
\bibleverse{5} Las aguas me rodearon, hasta el alma. Lo profundo me
rodeaba. La maleza se enredó en mi cabeza. \footnote{\textbf{2:5} Sal
  18,4; Sal 69,1} \bibleverse{6} Bajé a los fondos de las montañas. La
tierra me impidió entrar para siempre; pero tú has sacado mi vida del
pozo, Yahvé, mi Dios. \footnote{\textbf{2:6} Sal 103,4} \bibleverse{7}
``Cuando mi alma se desmayó dentro de mí, me acordé de Yahvé. Mi oración
llegó a ti, a tu santo templo. \footnote{\textbf{2:7} Sal 142,3}
\bibleverse{8} Los que consideran a los ídolos vanos abandonan su propia
misericordia. \footnote{\textbf{2:8} Sal 31,6} \bibleverse{9} Pero yo te
sacrificaré con voz de agradecimiento. Pagaré lo que he prometido. La
salvación pertenece a Yahvé''. \footnote{\textbf{2:9} Sal 50,14; Sal
  116,17-18}

\bibleverse{10} Entonces Yahvé habló al pez, y éste vomitó a Jonás en
tierra firme.

\hypertarget{jonuxe1s-exitoso-sermuxf3n-penitencial-en-nuxednive}{%
\subsection{Jonás exitoso sermón penitencial en
Nínive}\label{jonuxe1s-exitoso-sermuxf3n-penitencial-en-nuxednive}}

\hypertarget{section-2}{%
\section{3}\label{section-2}}

\bibleverse{1} La palabra de Yahvé vino a Jonás por segunda vez,
diciendo: \bibleverse{2} ``Levántate, ve a Nínive, esa gran ciudad, y
predícale el mensaje que te doy.'' \footnote{\textbf{3:2} Jon 1,2}

\bibleverse{3} Jonás se levantó y se dirigió a Nínive, según la palabra
de Yahvé. Nínive era una ciudad muy grande, que estaba a tres días de
camino. \footnote{\textbf{3:3} Jon 4,11} \bibleverse{4} Jonás comenzó a
entrar en la ciudad a un día de camino, y gritó diciendo: ``¡Dentro de
cuarenta días, Nínive será destruida!''

\bibleverse{5} El pueblo de Nínive creyó a Dios, y proclamó un ayuno y
se vistió de cilicio, desde el más grande hasta el más pequeño.
\footnote{\textbf{3:5} Mat 12,41} \bibleverse{6} La noticia llegó a
oídos del rey de Nínive, quien se levantó de su trono, se quitó el manto
real, se cubrió de cilicio y se sentó en cenizas. \bibleverse{7} Hizo
una proclama y la publicó por Nínive por decreto del rey y de sus
nobles, diciendo: ``Que ni el hombre ni el animal, ni la manada ni el
rebaño, prueben nada; que no se alimenten ni beban agua; \bibleverse{8}
sino que se cubran de cilicio, tanto el hombre como el animal, y que
clamen poderosamente a Dios. Sí, que se conviertan todos de su mal
camino y de la violencia que hay en sus manos. \bibleverse{9} ¿Quién
sabe si Dios no se convertirá y se arrepentirá, y se apartará de su
feroz ira, para que no perezcamos?'' \footnote{\textbf{3:9} Jl 2,14}

\bibleverse{10} Dios vio sus obras, que se convirtieron de su mal
camino. Dios cedió del desastre que dijo que les haría, y no lo hizo.
\footnote{\textbf{3:10} Jer 18,7-8}

\hypertarget{jonuxe1s-disgusto-y-reprensiuxf3n}{%
\subsection{Jonás disgusto y
reprensión}\label{jonuxe1s-disgusto-y-reprensiuxf3n}}

\hypertarget{section-3}{%
\section{4}\label{section-3}}

\bibleverse{1} Pero esto disgustó mucho a Jonás, y se enojó.
\bibleverse{2} Oró a Yavé y dijo: ``Por favor, Yavé, ¿no fue esto lo que
dije cuando todavía estaba en mi país? Por eso me apresuré a huir a
Tarsis, porque sabía que eres un Dios clemente y misericordioso, lento
para la ira y abundante en bondades amorosas, y que renuncias a hacer
daño. \footnote{\textbf{4:2} Éxod 34,6} \bibleverse{3} Por eso ahora,
Yahvé, te ruego que me quites la vida, pues es mejor para mí morir que
vivir.'' \footnote{\textbf{4:3} 1Re 19,4}

\bibleverse{4} Yahvé dijo: ``¿Está bien que te enojes?'' \footnote{\textbf{4:4}
  Jon 4,9}

\bibleverse{5} Entonces Jonás salió de la ciudad y se sentó en el lado
oriental de la ciudad, y allí se hizo una caseta y se sentó bajo ella a
la sombra, hasta que viera lo que iba a ser de la ciudad. \bibleverse{6}
El Señor Dios preparó una parra y la hizo subir sobre Jonás, para que
fuera una sombra sobre su cabeza y lo librara de su malestar. Y Jonás se
alegró mucho por la vid. \bibleverse{7} Pero Dios preparó un gusano al
amanecer del día siguiente, y éste masticó la vid de modo que se
marchitó. \bibleverse{8} Cuando salió el sol, Dios preparó un viento
bochornoso del este; y el sol golpeó la cabeza de Jonás, de modo que se
desmayó y pidió para sí la muerte. Dijo: ``Es mejor para mí morir que
vivir''.

\bibleverse{9} Dios le dijo a Jonás: ``¿Está bien que te enojes por la
vid?'' Dijo: ``Tengo derecho a enfadarme, incluso hasta la muerte''.
\footnote{\textbf{4:9} Jon 4,4}

\bibleverse{10} El Señor dijo: ``Te has preocupado por la vid, por la
que no has trabajado ni la has hecho crecer, que surgió en una noche y
pereció en una noche. \bibleverse{11} ¿No deberíapreocuparme por Nínive,
esa gran ciudad en la que hay más de ciento veinte mil personas que no
saben discernir entre su mano derecha y su mano izquierda, y también
muchos animales?'' \footnote{\textbf{4:11} Jon 3,3}
