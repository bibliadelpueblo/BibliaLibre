\hypertarget{acceso-de-salomuxf3n-al-gobierno-su-ejuxe9rcito-y-su-riqueza}{%
\subsection{Acceso de Salomón al gobierno; su ejército y su
riqueza}\label{acceso-de-salomuxf3n-al-gobierno-su-ejuxe9rcito-y-su-riqueza}}

\hypertarget{section}{%
\section{1}\label{section}}

\bibleverse{1} Salomón, hijo de David, estaba firmemente establecido en
su reino, y Yahvé\footnote{\textbf{1:1} ``Yahvé'' es el nombre propio de
  Dios, a veces traducido como ``\textsc{Señor}'' (en mayúsculas) en
  otras traducciones.} su Dios\footnote{\textbf{1:1} La palabra hebrea
  traducida como ``Dios'' es ``\hebrew{אֱלֹהִ֑ים}'' (Elohim).} estaba con
él, y lo hizo sumamente grande. \footnote{\textbf{1:1} 1Re 2,12; 1Re
  2,46}

\bibleverse{2} Salomón habló a todo Israel, a los capitanes de millares
y de centenas, a los jueces y a todos los príncipes de todo Israel, a
los jefes de familia. \bibleverse{3} Entonces Salomón, y toda la
asamblea con él, se dirigió al lugar alto que estaba en Gabaón, porque
allí estaba la Tienda de Reunión de Dios, que Moisés, siervo de Yavé,
había hecho en el desierto. \footnote{\textbf{1:3} 1Cró 16,39; 1Cró
  21,29} \bibleverse{4} Pero David había hecho subir el Arca de Dios
desde Quiriat Jearim al lugar que David había preparado para ella, pues
le había montado una tienda en Jerusalén. \footnote{\textbf{1:4} 1Cró
  13,6; 1Cró 15,3; 1Cró 15,28; 1Cró 16,1} \bibleverse{5} Además, el
altar de bronce que había hecho Bezalel, hijo de Uri, hijo de Hur,
estaba allí delante del tabernáculo de Yahvé, y Salomón y la asamblea
estaban buscando consejo allí. \footnote{\textbf{1:5} Éxod 38,1-8; 2Cró
  1,3} \bibleverse{6} Salomón subió allí al altar de bronce que estaba
delante de Yavé, en la Tienda del Encuentro, y ofreció sobre él mil
holocaustos.

\hypertarget{la-apariciuxf3n-de-dios-o-sueuxf1o-despuuxe9s-del-sacrificio}{%
\subsection{La aparición de Dios (o sueño) después del
sacrificio}\label{la-apariciuxf3n-de-dios-o-sueuxf1o-despuuxe9s-del-sacrificio}}

\bibleverse{7} Esa noche, Dios se le apareció a Salomón y le dijo:
``Pide lo que quieras que te dé''. \footnote{\textbf{1:7} 1Re 3,5-15}

\bibleverse{8} Salomón dijo a Dios: ``Has mostrado una gran bondad
amorosa con David mi padre, y me has hecho rey en su lugar.
\bibleverse{9} Ahora, Yahvé Dios, haz que se cumpla tu promesa a David
mi padre, pues me has hecho rey de un pueblo como el polvo de la tierra
en multitud. \bibleverse{10} Ahora dame sabiduría y conocimiento, para
que pueda salir y entrar ante este pueblo; porque ¿quién podrá juzgar a
este gran pueblo tuyo?''

\bibleverse{11} Dios dijo a Salomón: ``Porque esto estaba en tu corazón,
y no has pedido riquezas, ni riquezas, ni honores, ni la vida de los que
te odian, ni tampoco has pedido larga vida; sino que has pedido
sabiduría y conocimiento para ti, para juzgar a mi pueblo, sobre el cual
te he hecho rey, \bibleverse{12} por lo tanto, la sabiduría y el
conocimiento te son concedidos. Te daré riquezas, riqueza y honor, como
no lo ha tenido ninguno de los reyes que han sido antes de ti, ni lo
tendrá ninguno después de ti.''

\bibleverse{13} Salomón vino desde el lugar alto que estaba en Gabaón,
delante de la Tienda de las Reuniones, a Jerusalén, y reinó sobre
Israel.

\hypertarget{la-riqueza-y-el-comercio-de-salomuxf3n-en-carros-y-caballos}{%
\subsection{La riqueza y el comercio de Salomón en carros y
caballos}\label{la-riqueza-y-el-comercio-de-salomuxf3n-en-carros-y-caballos}}

\bibleverse{14} Salomón reunió carros y jinetes. Tenía mil cuatrocientos
carros y doce mil jinetes que colocó en las ciudades de los carros, y
con el rey en Jerusalén. \footnote{\textbf{1:14} 1Re 10,26-29}
\bibleverse{15} El rey hizo que la plata y el oro fueran tan comunes
como las piedras en Jerusalén, e hizo que los cedros fueran tan comunes
como los sicómoros que hay en la llanura. \footnote{\textbf{1:15} 2Cró
  9,27} \bibleverse{16} Los caballos que tenía Salomón fueron traídos de
Egipto y de Kue. Los mercaderes del rey los compraron de Kue.
\bibleverse{17} Importaron de Egipto y luego exportaron un carro por
seiscientas piezas de plata y un caballo por ciento
cincuenta.\footnote{\textbf{1:17} Las piezas de plata eran probablemente
  siclos, por lo que 600 piezas serían unas 13,2 libras o 6 kilogramos
  de plata, y 150 serían unas 3,3 libras o 1,5 kilogramos de plata.}
También los exportaron a los reyes hititas y a los reyes sirios.
\footnote{\textbf{1:17} ``Beracah'' significa ``bendición''.}

\hypertarget{el-tratado-de-salomuxf3n-con-hiram-de-tiro-preparativos-para-la-construcciuxf3n-del-templo}{%
\subsection{El tratado de Salomón con Hiram de Tiro; Preparativos para
la construcción del
templo}\label{el-tratado-de-salomuxf3n-con-hiram-de-tiro-preparativos-para-la-construcciuxf3n-del-templo}}

\hypertarget{section-1}{%
\section{2}\label{section-1}}

\bibleverse{1} Salomón decidió construir una casa para el nombre de
Yahvé y una casa para su reino. \bibleverse{2} Salomón contó con setenta
mil hombres para llevar cargas, ochenta mil hombres que cortaban piedras
en las montañas, y tres mil seiscientos para supervisarlos. \footnote{\textbf{2:2}
  1Cró 14,1}

\hypertarget{mensaje-de-salomuxf3n-y-peticiuxf3n-a-hiram}{%
\subsection{Mensaje de Salomón y petición a
Hiram}\label{mensaje-de-salomuxf3n-y-peticiuxf3n-a-hiram}}

\bibleverse{3} Salomón envió a decir a Hiram, rey de Tiro: ``De la misma
manera que trataste con David mi padre y le enviaste cedros para que le
construyera una casa en la que habitar, así trata conmigo.
\bibleverse{4} He aquí que voy a construir una casa al nombre de Yavé,
mi Dios, para dedicársela, para quemar ante él incienso de especias
dulces, para el pan de la muestra continua y para los holocaustos de la
mañana y de la tarde, en los sábados, en las lunas nuevas y en las
fiestas señaladas de Yavé, nuestro Dios. Esta es una ordenanza para
siempre para Israel.

\bibleverse{5} ``La casa que estoy construyendo será grande, porque
nuestro Dios es más grande que todos los dioses. \footnote{\textbf{2:5}
  Sal 86,8} \bibleverse{6} Pero ¿quién puede construirle una casa, ya
que el cielo y el cielo de los cielos no pueden contenerlo? ¿Quién soy
yo, pues, para construirle una casa, sino para quemar incienso ante él?
\footnote{\textbf{2:6} 2Cró 6,18; 1Re 8,27}

\bibleverse{7} ``Ahora, pues, envíame un hombre hábil para trabajar el
oro, la plata, el bronce, el hierro, la púrpura, el carmesí y el azul, y
que sepa grabar, para que esté con los hombres hábiles que están conmigo
en Judá y en Jerusalén, que me proporcionó mi padre.

\bibleverse{8} ``Envíame también cedros, cipreses y álgumes del Líbano,
porque sé que tus siervos saben cortar madera en el Líbano. He aquí que
mis siervos estarán con tus siervos, \bibleverse{9} para prepararme
madera en abundancia, porque la casa que voy a construir será grande y
maravillosa. \bibleverse{10} He aquí que yo daré a tus siervos, los
cortadores que cortan la madera, veinte mil cors de trigo batido, veinte
mil baños de cebada, veinte mil baños de vino y veinte mil baños de
aceite.''

\hypertarget{respuesta-y-aceptaciuxf3n-de-hiram}{%
\subsection{Respuesta y aceptación de
Hiram}\label{respuesta-y-aceptaciuxf3n-de-hiram}}

\bibleverse{11} Entonces Huram, rey de Tiro, respondió por escrito, que
envió a Salomón: ``Porque Yahvé ama a su pueblo, te ha hecho rey sobre
él.'' \bibleverse{12} Huram continuó: ``Bendito sea Yavé, el Dios de
Israel, que hizo el cielo y la tierra, que ha dado al rey David un hijo
sabio, dotado de discreción y entendimiento, que construya una casa para
Yavé y una casa para su reino.

\bibleverse{13} He enviado a un hombre hábil, dotado de entendimiento,
Huram-abi, \bibleverse{14} hijo de una mujer de las hijas de Dan; y su
padre era un hombre de Tiro. Él es hábil para trabajar en oro, en plata,
en bronce, en hierro, en piedra, en madera, en púrpura, en azul, en lino
fino y en carmesí, también para grabar cualquier clase de grabado e
idear cualquier dispositivo, para que se le asigne un lugar con tus
hombres hábiles, y con los hombres hábiles de mi señor David, tu padre.
\footnote{\textbf{2:14} Éxod 31,2-6}

\bibleverse{15} ``Ahora, pues, el trigo, la cebada, el aceite y el vino
de que ha hablado mi señor, envíenlo a sus siervos; \bibleverse{16} y
cortaremos madera del Líbano, toda la que necesites. Te la llevaremos en
balsas por mar hasta Jope; luego la subirás a Jerusalén''.

\hypertarget{salomuxf3n-eleva-a-los-no-israelitas-al-trabajo-esclavo}{%
\subsection{Salomón eleva a los no israelitas al trabajo
esclavo}\label{salomuxf3n-eleva-a-los-no-israelitas-al-trabajo-esclavo}}

\bibleverse{17} Salomón contó a todos los extranjeros que estaban en la
tierra de Israel, según el censo con que los había contado su padre
David, y hallaron ciento cincuenta y tres mil seiscientos.
\bibleverse{18} Puso a setenta mil de ellos para que llevaran cargas, a
ochenta mil que eran cortadores de piedra en las montañas, y a tres mil
seiscientos capataces para que asignaran al pueblo su trabajo.
\footnote{\textbf{2:18} Jos 9,27}

\hypertarget{inicio-de-la-construcciuxf3n-del-templo-los-muebles-del-templo}{%
\subsection{Inicio de la construcción del templo; los muebles del
templo}\label{inicio-de-la-construcciuxf3n-del-templo-los-muebles-del-templo}}

\hypertarget{section-2}{%
\section{3}\label{section-2}}

\bibleverse{1} Salomón comenzó a edificar la Casa de Yahvé en Jerusalén,
en el monte Moriah, donde Yahvé se había aparecido a David su padre, la
cual preparó en el lugar que David había designado, en la era de Ornán
el jebuseo. \footnote{\textbf{3:1} Gén 22,2; 1Cró 21,18-26}
\bibleverse{2} Comenzó a construir en el segundo día del segundo mes, en
el cuarto año de su reinado.

\hypertarget{dimensiones-y-decoraciones-de-la-casa-del-templo}{%
\subsection{Dimensiones y decoraciones de la casa del
templo}\label{dimensiones-y-decoraciones-de-la-casa-del-templo}}

\bibleverse{3} Estos son los cimientos que Salomón puso para el edificio
de la casa de Dios: la longitud por codos después de la primera medida
era de sesenta codos, y la anchura de veinte codos. \bibleverse{4} El
pórtico que estaba delante, su longitud, a lo ancho de la casa, era de
veinte codos, y la altura de ciento veinte; y lo recubrió por dentro con
oro puro. \bibleverse{5} Hizo la sala mayor con un techo de madera de
ciprés, que recubrió de oro fino, y la adornó con palmeras y cadenas.
\bibleverse{6} Decoró la casa con piedras preciosas para embellecerla.
El oro era de Parvaim. \bibleverse{7} También recubrió de oro la casa,
las vigas, los umbrales, las paredes y las puertas, y grabó querubines
en las paredes.

\hypertarget{equipo-del-lugar-santuxedsimo}{%
\subsection{Equipo del lugar
santísimo}\label{equipo-del-lugar-santuxedsimo}}

\bibleverse{8} Hizo el lugar santísimo. Su longitud, según la anchura de
la casa, era de veinte codos, y su anchura de veinte codos; y lo
recubrió de oro fino, que ascendía a seiscientos talentos.
\bibleverse{9} El peso de los clavos era de cincuenta siclos de oro.
Recubrió de oro las habitaciones superiores.

\bibleverse{10} En el lugar santísimo hizo dos querubines tallados, y
los recubrió de oro. \bibleverse{11} Las alas de los querubines medían
veinte codos; el ala de uno de ellos medía cinco codos y llegaba hasta
la pared de la Casa; la otra ala medía cinco codos y llegaba hasta el
ala del otro querubín. \bibleverse{12} El ala del otro querubín medía
cinco codos y llegaba hasta la pared de la casa; la otra ala medía cinco
codos y se unía al ala del otro querubín. \bibleverse{13} Las alas de
estos querubines se extendían veinte codos. Estaban de pie, y sus
rostros se dirigían hacia la casa. \bibleverse{14} Hizo el velo de azul,
púrpura, carmesí y lino fino, y lo adornó con querubines. \footnote{\textbf{3:14}
  Éxod 26,31}

\hypertarget{los-dos-pilares-de-bronce-frente-a-la-casa-del-templo}{%
\subsection{Los dos pilares de bronce frente a la casa del
templo}\label{los-dos-pilares-de-bronce-frente-a-la-casa-del-templo}}

\bibleverse{15} También hizo delante de la casa dos columnas de treinta
y cinco codos de altura, y el capitel que estaba en la parte superior de
cada una de ellas era de cinco codos. \bibleverse{16} Hizo cadenas en el
santuario interior y las puso en la parte superior de las columnas; hizo
cien granadas y las puso en las cadenas. \bibleverse{17} Colocó las
columnas delante del templo, una a la derecha y otra a la izquierda, y
llamó al de la derecha Jaquín, y al de la izquierda Boaz.

\hypertarget{fabricaciuxf3n-de-implementos-para-el-templo}{%
\subsection{Fabricación de implementos para el
templo}\label{fabricaciuxf3n-de-implementos-para-el-templo}}

\hypertarget{section-3}{%
\section{4}\label{section-3}}

\bibleverse{1} Luego hizo un altar de bronce de veinte codos de largo,
veinte codos de ancho y diez codos de alto. \footnote{\textbf{4:1} 2Cró
  7,7} \bibleverse{2} También hizo el mar fundido de diez codos de borde
a borde. Era redondo, de cinco codos de altura y de treinta codos de
circunferencia. \bibleverse{3} Debajo de él había figuras de bueyes que
lo rodeaban por diez codos, rodeando el mar. Los bueyes estaban en dos
hileras, fundidos cuando fue fundido. \bibleverse{4} Estaba sobre doce
bueyes, tres que miraban hacia el norte, tres que miraban hacia el
occidente, tres que miraban hacia el sur y tres que miraban hacia el
oriente; y el mar estaba puesto sobre ellos por encima, y todos sus
cuartos traseros estaban hacia adentro. \bibleverse{5} Su grosor era de
un palmo. Su borde estaba hecho como el borde de una copa, como la flor
de un lirio. Recibía y contenía tres mil baños. \bibleverse{6} También
hizo diez pilas, y puso cinco a la derecha y cinco a la izquierda, para
lavar en ellas. En ellas se lavaban las cosas que pertenecían al
holocausto, pero el mar era para que se lavaran los sacerdotes.

\bibleverse{7} Hizo los diez candelabros de oro, según la ordenanza
relativa a ellos, y los puso en el templo, cinco a la derecha y cinco a
la izquierda. \bibleverse{8} Hizo también diez mesas y las colocó en el
templo, cinco a la derecha y cinco a la izquierda. Hizo cien pilas de
oro. \bibleverse{9} Además, hizo el atrio de los sacerdotes, el gran
atrio, y las puertas para el atrio, y recubrió sus puertas con bronce.
\bibleverse{10} Colocó el mar en el lado derecho de la casa, al este,
hacia el sur.

\bibleverse{11} Huram hizo las ollas, las palas y las cuencas. Entonces
Huram terminó de hacer el trabajo que hizo para el rey Salomón en la
casa de Dios: \bibleverse{12} las dos columnas, las copas, los dos
capiteles que estaban en la parte superior de las columnas, las dos
redes para cubrir las dos copas de los capiteles que estaban en la parte
superior de las columnas, \bibleverse{13} y las cuatrocientas granadas
para las dos redes --- dos filas de granadas para cada red, para cubrir
las dos copas de los capiteles que estaban en las columnas.
\bibleverse{14} También hizo las bases, y sobre ellas hizo las cuencas
--- \bibleverse{15} un mar, y los doce bueyes debajo de él.
\bibleverse{16} Huram-abi hizo también las ollas, las palas, los
tenedores y todos sus recipientes para el rey Salomón, para la casa de
Yahvé, de bronce brillante. \bibleverse{17} El rey los fundió en la
llanura del Jordán, en la tierra arcillosa entre Sucot y Zereda.
\bibleverse{18} Así Salomón hizo todos estos recipientes en gran
cantidad, de tal manera que no se podía determinar el peso del bronce.

\bibleverse{19} Salomón hizo todos los utensilios que había en la casa
de Dios: el altar de oro, las mesas con el pan de la función sobre
ellas, \bibleverse{20} y los candelabros con sus lámparas para arder
según la ordenanza ante el santuario interior, de oro puro;
\bibleverse{21} y las flores, las lámparas y las tenazas de oro
purísimo; \bibleverse{22} y los apagavelas, las palanganas, las cucharas
y los recipientes para el fuego, de oro puro. En cuanto a la entrada de
la casa, sus puertas interiores para el lugar santísimo y las puertas de
la sala principal del templo eran de oro.

\hypertarget{los-objetos-de-valor-almacenados-en-las-cuxe1maras-del-tesoro}{%
\subsection{Los objetos de valor almacenados en las cámaras del
tesoro}\label{los-objetos-de-valor-almacenados-en-las-cuxe1maras-del-tesoro}}

\hypertarget{section-4}{%
\section{5}\label{section-4}}

\bibleverse{1} Así quedó terminada toda la obra que Salomón hizo para la
casa de Yavé. Salomón trajo las cosas que su padre David había dedicado,
la plata, el oro y todos los utensilios, y los puso en los tesoros de la
casa de Dios. \footnote{\textbf{5:1} 1Cró 28,14-18}

\hypertarget{la-transferencia-del-arca-al-lugar-santuxedsimo}{%
\subsection{La transferencia del arca al lugar
santísimo}\label{la-transferencia-del-arca-al-lugar-santuxedsimo}}

\bibleverse{2} Entonces Salomón reunió en Jerusalén a los ancianos de
Israel y a todos los jefes de las tribus, a los jefes de familia de los
hijos de Israel, para hacer subir el arca de la alianza de Yahvé desde
la ciudad de David, que es Sión. \bibleverse{3} Así que todos los
hombres de Israel se reunieron con el rey en la fiesta, que era en el
mes séptimo. \footnote{\textbf{5:3} Lev 23,34} \bibleverse{4} Vinieron
todos los ancianos de Israel. Los levitas subieron el arca.
\bibleverse{5} Subieron el arca, la Tienda del Encuentro y todos los
utensilios sagrados que estaban en la Tienda. Los sacerdotes levitas los
subieron. \bibleverse{6} El rey Salomón y toda la congregación de Israel
que se había reunido con él estaban ante el arca, sacrificando ovejas y
ganado que no se podía contar ni numerar por la multitud. \bibleverse{7}
Los sacerdotes introdujeron el arca de la alianza de Yavé en su lugar,
en el santuario interior de la casa, en el lugar santísimo, bajo las
alas de los querubines. \bibleverse{8} Porque los querubines extendían
sus alas sobre el lugar del arca, y los querubines cubrían el arca y sus
varas por encima. \bibleverse{9} Los postes eran tan largos que los
extremos de los postes se veían desde el arca frente al santuario
interior, pero no se veían afuera; y así es hasta el día de hoy.
\bibleverse{10} En el arca no había nada más que las dos tablas que
Moisés puso allí en Horeb, cuando Yahvé hizo la alianza con los hijos de
Israel, al salir de Egipto. \footnote{\textbf{5:10} Heb 9,4}

\hypertarget{la-apariciuxf3n-de-la-gloria-de-dios}{%
\subsection{La aparición de la gloria de
Dios}\label{la-apariciuxf3n-de-la-gloria-de-dios}}

\bibleverse{11} Cuando los sacerdotes salieron del lugar santo (porque
todos los sacerdotes que estaban presentes se habían santificado y no
guardaban sus divisiones; \bibleverse{12} también los levitas cantores,
todos ellos, Asaf, Hemán, Jedutún, sus hijos y sus hermanos, vestidos de
lino fino, con címbalos e instrumentos de cuerda y arpas, estaban de pie
al extremo oriental del altar, y con ellos ciento veinte sacerdotes que
tocaban las trompetas); \footnote{\textbf{5:12} 1Cró 15,19; 1Cró 16,37;
  1Cró 16,41-42; 1Cró 25,1-7} \bibleverse{13} cuando los trompetistas y
los cantores eran como uno solo, para hacer oír un solo sonido en la
alabanza y la acción de gracias a Yahvé; y cuando alzaban la voz con las
trompetas y los címbalos y los instrumentos de música, y alababan a
Yahvé, diciendo``Porque él es bueno, porque su bondad es eterna''.
Entonces la casa se llenó de una nube, la casa de Yahvé, \footnote{\textbf{5:13}
  1Cró 16,34} \bibleverse{14} de modo que los sacerdotes no podían estar
de pie para ministrar a causa de la nube, porque la gloria de Yahvé
llenaba la casa de Dios. \footnote{\textbf{5:14} 2Cró 7,1; 2Cró 7,3}

\hypertarget{el-discurso-de-ordenaciuxf3n-y-consagraciuxf3n-del-rey-al-pueblo}{%
\subsection{El discurso de ordenación y consagración del rey al
pueblo}\label{el-discurso-de-ordenaciuxf3n-y-consagraciuxf3n-del-rey-al-pueblo}}

\hypertarget{section-5}{%
\section{6}\label{section-5}}

\bibleverse{1} Entonces Salomón dijo: ``Yahvé ha dicho que moraría en la
espesa oscuridad. \bibleverse{2} Pero yo te he construido una casa y un
hogar, un lugar para que habites para siempre''.

\bibleverse{3} El rey volvió su rostro y bendijo a toda la asamblea de
Israel; y toda la asamblea de Israel se puso en pie.

\bibleverse{4} Dijo: ``Bendito sea Yavé, el Dios de Israel, que habló
con su boca a David, mi padre, y con sus manos lo ha cumplido, diciendo:
\bibleverse{5} `Desde el día en que saqué a mi pueblo de la tierra de
Egipto, no elegí ninguna ciudad de todas las tribus de Israel para
edificar una casa en la que estuviera mi nombre, y no elegí a ningún
hombre para que fuera príncipe de mi pueblo Israel; \bibleverse{6} pero
ahora he elegido Jerusalén, para que mi nombre esté allí; y he elegido a
David para que esté sobre mi pueblo Israel.' \bibleverse{7} El corazón
de mi padre era construir una casa para el nombre de Yavé, el Dios de
Israel. \footnote{\textbf{6:7} 2Sam 7,2-13} \bibleverse{8} Pero Yahvé
dijo a David mi padre: `Si bien estaba en tu corazón construir una casa
para mi nombre, hiciste bien en que estuviera en tu corazón;
\bibleverse{9} sin embargo, no construirás la casa, sino tu hijo que
saldrá de tu cuerpo, él construirá la casa para mi nombre.'

\bibleverse{10} ``Yavé ha cumplido su palabra que había pronunciado,
pues me he levantado en lugar de David, mi padre, y me he sentado en el
trono de Israel, como lo había prometido Yavé, y he edificado la Casa
para el nombre de Yavé, el Dios de Israel. \bibleverse{11} Allí he
colocado el arca, en la que está la alianza de Yavé que hizo con los
hijos de Israel.''

\hypertarget{oraciuxf3n-de-consagraciuxf3n-de-salomuxf3n}{%
\subsection{Oración de consagración de
Salomón}\label{oraciuxf3n-de-consagraciuxf3n-de-salomuxf3n}}

\bibleverse{12} Se puso de pie ante el altar de Yavé, en presencia de
toda la asamblea de Israel, y extendió sus manos \bibleverse{13} (porque
Salomón había hecho una plataforma de bronce de cinco codos de largo,
cinco codos de ancho y tres codos de alto, y la había colocado en medio
del atrio; y se puso de pie sobre ella, y se arrodilló ante toda la
asamblea de Israel, y extendió sus manos hacia el cielo).
\bibleverse{14} Entonces dijo: ``Yavé, Dios de Israel, no hay Dios como
tú en el cielo ni en la tierra; tú que guardas el pacto y la bondad
amorosa con tus siervos que caminan ante ti de todo corazón;
\bibleverse{15} que has cumplido con tu siervo David, mi padre, lo que
le prometiste. Sí, tú hablaste con tu boca, y lo has cumplido con tu
mano, como sucede hoy.

\bibleverse{16} ``Ahora, pues, Yahvé, Dios de Israel, mantén con tu
siervo David, mi padre, lo que le prometiste, diciendo: `No te faltará
un hombre ante mis ojos para sentarse en el trono de Israel, con tal que
tus hijos cuiden su camino, para andar en mi ley como tú has andado
delante de mí'. \footnote{\textbf{6:16} 2Sam 7,16} \bibleverse{17}
Ahora, pues, Yahvé, Dios de Israel, haz que se cumpla tu palabra, que
has dicho a tu siervo David.

\bibleverse{18} ``Pero, ¿acaso habitará Dios con los hombres en la
tierra? He aquí que el cielo y el cielo de los cielos no pueden
contenerte; ¡cuánto menos esta casa que he construido! \footnote{\textbf{6:18}
  2Cró 2,6} \bibleverse{19} Sin embargo, respeta la oración de tu siervo
y su súplica, Yahvé, mi Dios, para escuchar el clamor y la oración que
tu siervo hace ante ti; \bibleverse{20} para que tus ojos estén abiertos
hacia esta casa de día y de noche, hacia el lugar donde has dicho que
pondrías tu nombre, para escuchar la oración que tu siervo hará hacia
este lugar. \footnote{\textbf{6:20} Éxod 20,24} \bibleverse{21} Escucha
las peticiones de tu siervo y de tu pueblo Israel, cuando oren hacia
este lugar. Sí, escucha desde tu morada, desde el cielo; y cuando oigas,
perdona.

\bibleverse{22} ``Si un hombre peca contra su prójimo, y se le impone un
juramento para hacerle jurar, y viene y jura ante tu altar en esta casa,
\footnote{\textbf{6:22} Éxod 22,11} \bibleverse{23} entonces escucha
desde el cielo, actúa, y juzga a tus siervos, trayendo el castigo al
impío, para hacer recaer su camino sobre su propia cabeza; y
justificando al justo, para darle según su justicia.

\bibleverse{24} ``Si tu pueblo Israel es abatido ante el enemigo por
haber pecado contra ti, y se vuelve y confiesa tu nombre, y ora y
suplica ante ti en esta casa, \footnote{\textbf{6:24} Deut 28,25}
\bibleverse{25} entonces escucha desde el cielo, y perdona el pecado de
tu pueblo Israel, y hazlo volver a la tierra que les diste a ellos y a
sus padres.

\bibleverse{26} ``Cuando el cielo se cierra y no hay lluvia porque han
pecado contra ti, si oran hacia este lugar y confiesan tu nombre, y se
convierten de su pecado cuando los afliges, \footnote{\textbf{6:26} Deut
  28,23-24} \bibleverse{27} entonces escucha en el cielo, y perdona el
pecado de tus siervos, tu pueblo Israel, cuando les enseñas el buen
camino por el que deben andar, y envía la lluvia sobre tu tierra, que
has dado a tu pueblo como herencia.

\bibleverse{28} ``Si hay hambre en la tierra, si hay peste, si hay tizón
o moho, langosta u oruga; si sus enemigos los asedian en la tierra de
sus ciudades cualquier plaga o cualquier enfermedad que haya ---
\bibleverse{29} cualquier oración y súplica que haga cualquier hombre, o
todo tu pueblo Israel, que conozca cada uno su propia plaga y su propio
dolor, y extienda sus manos hacia esta casa, \bibleverse{30} entonces
escucha desde el cielo tu morada y perdona, y rinde a cada uno según
todos sus caminos, cuyo corazón conoces (porque tú, sólo tú, conoces los
corazones de los hijos de los hombres), \footnote{\textbf{6:30} 1Cró
  29,17; Sal 7,9} \bibleverse{31} para que te teman, para que anden en
tus caminos mientras vivan en la tierra que diste a nuestros padres.

\bibleverse{32} ``Además, en cuanto al extranjero, que no es de tu
pueblo Israel, cuando venga de un país lejano por causa de tu gran
nombre y de tu mano poderosa y de tu brazo extendido, cuando vengan y
oren hacia esta casa, \bibleverse{33} entonces escucha desde el cielo,
desde tu morada, y haz conforme a todo lo que el extranjero te pida;
para que todos los pueblos de la tierra conozcan tu nombre y te teman,
como tu pueblo Israel, y para que sepan que esta casa que he edificado
se llama con tu nombre.

\bibleverse{34} ``Si tu pueblo sale a combatir contra sus enemigos, por
cualquier camino que lo envíes, y te ruega hacia esta ciudad que tú has
elegido, y hacia la casa que he edificado a tu nombre; \footnote{\textbf{6:34}
  Dan 6,10} \bibleverse{35} entonces escucha desde el cielo su oración y
su súplica, y defiende su causa.

\bibleverse{36} ``Si pecan contra ti (pues no hay hombre que no peque),
y te enojas con ellos y los entregas al enemigo, de modo que los llevan
cautivos a una tierra lejana o cercana; \bibleverse{37} pero si vuelven
a entrar en razón en la tierra donde son llevados cautivos, y se
vuelven, y te suplican en la tierra de su cautiverio, diciendo: `Hemos
pecado, hemos actuado perversamente, y hemos hecho mal;' \footnote{\textbf{6:37}
  Dan 9,5} \bibleverse{38} si se vuelven a ti con todo su corazón y con
toda su alma en la tierra de su cautiverio, donde han sido llevados
cautivos, y oran hacia su tierra que diste a sus padres, y hacia la
ciudad que has elegido, y hacia la casa que he edificado a tu nombre;
\bibleverse{39} entonces escucha desde el cielo, desde tu morada, su
oración y sus peticiones, y defiende su causa, y perdona a tu pueblo que
ha pecado contra ti.

\bibleverse{40} ``Ahora, Dios mío, permite, te lo ruego, que tus ojos
estén abiertos y que tus oídos estén atentos a la oración que se hace en
este lugar.

\bibleverse{41} ``Ahora, pues, levántate, Yahvé Dios, a tu lugar de
descanso, tú y el arca de tu fuerza. Que tus sacerdotes, Yahvé Dios, se
revistan de salvación, y que tus santos se regocijen en la bondad.
\footnote{\textbf{6:41} Sal 132,8-9}

\bibleverse{42} ``Yahvé Dios, no rechaces el rostro de tu ungido.
Acuérdate de tus bondades para con David, tu siervo''. \footnote{\textbf{6:42}
  2Sam 7,13}

\hypertarget{apariciuxf3n-de-la-gloria-de-dios-salomuxf3n-y-el-pueblo-fiesta-solemne-de-sacrificios-y-asamblea-de-celebraciuxf3n}{%
\subsection{Aparición de la gloria de Dios; Salomón y el pueblo fiesta
solemne de sacrificios y asamblea de
celebración}\label{apariciuxf3n-de-la-gloria-de-dios-salomuxf3n-y-el-pueblo-fiesta-solemne-de-sacrificios-y-asamblea-de-celebraciuxf3n}}

\hypertarget{section-6}{%
\section{7}\label{section-6}}

\bibleverse{1} Cuando Salomón terminó de orar, bajó fuego del cielo y
consumió el holocausto y los sacrificios, y la gloria de Yavé llenó la
casa. \footnote{\textbf{7:1} Lev 9,24; 1Re 18,38; Éxod 40,34}
\bibleverse{2} Los sacerdotes no podían entrar en la casa de Yavé,
porque la gloria de Yavé llenaba la casa de Yavé. \bibleverse{3} Todos
los hijos de Israel miraban, cuando el fuego descendía y la gloria de
Yavé estaba sobre la casa. Se inclinaron con el rostro hacia el suelo
sobre el pavimento, adoraron y dieron gracias a Yavé, diciendo``Porque
él es bueno, porque su bondad es eterna''. \footnote{\textbf{7:3} 2Cró
  5,13; Sal 136,1}

\bibleverse{4} Entonces el rey y todo el pueblo ofrecieron sacrificios
ante el Señor. \bibleverse{5} El rey Salomón ofreció un sacrificio de
veintidós mil cabezas de ganado y ciento veinte mil ovejas. Así el rey y
todo el pueblo dedicaron la casa de Dios. \bibleverse{6} Los sacerdotes
estaban de pie, según sus cargos; los levitas también con instrumentos
de música de Yavé, que el rey David había hecho para dar gracias a Yavé,
cuando David alababa por su ministerio, diciendo: ``Porque su bondad es
eterna.'' Los sacerdotes tocaron las trompetas delante de ellos, y todo
Israel se puso en pie.

\bibleverse{7} Además, Salomón santificó el centro del atrio que estaba
delante de la casa de Yahvé, porque allí ofrecía los holocaustos y la
grasa de los sacrificios de paz, porque el altar de bronce que Salomón
había hecho no podía recibir el holocausto, el presente y la grasa.
\footnote{\textbf{7:7} 1Re 8,62-66}

\bibleverse{8} Salomón celebró entonces la fiesta durante siete días, y
todo Israel con él, una asamblea muy grande, desde la entrada de Hamat
hasta el arroyo de Egipto.

\bibleverse{9} El octavo día celebraron una asamblea solemne, pues
celebraron la dedicación del altar durante siete días, y la fiesta
durante siete días. \footnote{\textbf{7:9} Núm 7,10} \bibleverse{10} El
día veintitrés del mes séptimo, despidió al pueblo para que se fuera a
sus tiendas, alegres y contentos de corazón por la bondad que Yahvé
había mostrado a David, a Salomón y a su pueblo Israel.

\hypertarget{la-repetida-apariciuxf3n-de-dios-y-su-respuesta-promesa-y-amenaza-a-la-oraciuxf3n-de-salomuxf3n}{%
\subsection{La repetida aparición de Dios y su respuesta (promesa y
amenaza) a la oración de
Salomón}\label{la-repetida-apariciuxf3n-de-dios-y-su-respuesta-promesa-y-amenaza-a-la-oraciuxf3n-de-salomuxf3n}}

\bibleverse{11} Así terminó Salomón la casa de Yahvé y la casa del rey,
y completó con éxito todo lo que le vino al corazón de Salomón para
hacer en la casa de Yahvé y en su propia casa.

\bibleverse{12} Entonces Yahvé se le apareció a Salomón de noche y le
dijo: ``He escuchado tu oración y he elegido este lugar para mí como
casa de sacrificio. \footnote{\textbf{7:12} Deut 12,5}

\bibleverse{13} ``Si cierro el cielo para que no llueva, o si ordeno a
la langosta que devore la tierra, o si envío la peste entre mi pueblo,
\bibleverse{14} si mi pueblo, llamado por mi nombre, se humilla, ora,
busca mi rostro y se convierte de sus malos caminos, entonces yo
escucharé desde el cielo, perdonaré su pecado y sanaré su tierra.
\bibleverse{15} Ahora mis ojos estarán abiertos y mis oídos atentos a la
oración que se haga en este lugar. \footnote{\textbf{7:15} 2Cró 6,40}
\bibleverse{16} Porque ahora he elegido y santificado esta casa, para
que mi nombre esté allí para siempre; y mis ojos y mi corazón estarán
allí perpetuamente.

\bibleverse{17} ``En cuanto a ti, si andas delante de mí como anduvo
David, tu padre, y haces todo lo que te he mandado, y guardas mis
estatutos y mis ordenanzas, \bibleverse{18} entonces estableceré el
trono de tu reino, según el pacto que hice con David, tu padre,
diciendo: `No te faltará un hombre que sea gobernante en Israel'.
\footnote{\textbf{7:18} 2Sam 7,12; 2Sam 7,16}

\bibleverse{19} Pero si se apartan y abandonan mis estatutos y mis
mandamientos que he puesto delante de ustedes, y van a servir a otros
dioses y los adoran, \bibleverse{20} entonces los arrancaré de raíz de
mi tierra que les he dado; y esta casa, que he santificado para mi
nombre, la echaré de mi vista, y la convertiré en proverbio y en palabra
de guerra entre todos los pueblos. \footnote{\textbf{7:20} Deut 28,37}
\bibleverse{21} Esta casa, que es tan alta, todo el que pase por ella se
asombrará y dirá: ``¿Por qué ha hecho esto Yahvé a esta tierra y a esta
casa?'' \footnote{\textbf{7:21} Deut 29,24-27; Jer 22,8-9}
\bibleverse{22} Responderán: ``Porque abandonaron a Yahvé, el Dios de
sus padres, que los sacó de la tierra de Egipto, y tomaron otros dioses,
los adoraron y los sirvieron. Por eso ha traído sobre ellos todo este
mal'\,''.

\hypertarget{informaciuxf3n-sobre-las-ciudades-y-fortalezas-de-salomuxf3n}{%
\subsection{Información sobre las ciudades y fortalezas de
Salomón}\label{informaciuxf3n-sobre-las-ciudades-y-fortalezas-de-salomuxf3n}}

\hypertarget{section-7}{%
\section{8}\label{section-7}}

\bibleverse{1} Al cabo de veinte años, en los que Salomón había
edificado la casa de Yahvé y su propia casa, \bibleverse{2} Salomón
edificó las ciudades que Huram le había dado a Salomón, e hizo habitar a
los hijos de Israel.

\bibleverse{3} Salomón se dirigió a Hamat-Zobá y la venció.
\bibleverse{4} Edificó Tadmor en el desierto, y todas las ciudades de
almacenamiento que edificó en Hamat. \bibleverse{5} También edificó Bet
Horón, la de arriba, y Bet Horón, la de abajo, ciudades fortificadas con
murallas, puertas y rejas; \bibleverse{6} y Baalat, y todas las ciudades
de almacenamiento que tenía Salomón, y todas las ciudades para sus
carros, las ciudades para su gente de a caballo, y todo lo que Salomón
quiso edificar a su gusto en Jerusalén, en el Líbano y en toda la tierra
de su dominio.

\hypertarget{los-obreros-de-salomuxf3n-y-sus-capataces-su-esposa-la-princesa-egipcia-se-traslada-al-palacio-construido-para-ella}{%
\subsection{Los obreros de Salomón y sus capataces; Su esposa, la
princesa egipcia, se traslada al palacio construido para
ella}\label{los-obreros-de-salomuxf3n-y-sus-capataces-su-esposa-la-princesa-egipcia-se-traslada-al-palacio-construido-para-ella}}

\bibleverse{7} En cuanto a todos los pueblos que quedaron de los
hititas, los amorreos, los ferezeos, los heveos y los jebuseos, que no
eran de Israel --- \bibleverse{8} de sus hijos que quedaron después de
ellos en la tierra, que los hijos de Israel no consumieron --- de ellos
Salomón reclutó mano de obra forzada hasta el día de hoy. \footnote{\textbf{8:8}
  Jos 16,10} \bibleverse{9} Pero de los hijos de Israel, Salomón no hizo
siervos para su trabajo, sino que fueron hombres de guerra, jefes de sus
capitanes y gobernantes de sus carros y de su caballería.
\bibleverse{10} Estos eran los principales oficiales del rey Salomón,
doscientos cincuenta, que gobernaban al pueblo.

\bibleverse{11} Salomón sacó a la hija del faraón de la ciudad de David
a la casa que le había construido, porque dijo: ``Mi mujer no habitará
en la casa de David, rey de Israel, porque los lugares donde ha llegado
el arca de Yahvé son sagrados.''

\hypertarget{orden-de-sacrificio-y-servicio-en-el-templo-de-salomuxf3n}{%
\subsection{Orden de sacrificio y servicio en el templo de
Salomón}\label{orden-de-sacrificio-y-servicio-en-el-templo-de-salomuxf3n}}

\bibleverse{12} Entonces Salomón ofreció holocaustos a Yahvé en el altar
de Yahvé que había construido ante el pórtico, \footnote{\textbf{8:12}
  2Cró 1,3-6} \bibleverse{13} tal como lo exigía el deber de cada día,
ofreciendo según el mandamiento de Moisés en los sábados, en las lunas
nuevas y en las fiestas establecidas, tres veces al año, durante la
fiesta de los panes sin levadura, durante la fiesta de las semanas y
durante la fiesta de las cabañas. \footnote{\textbf{8:13} Núm 28,2; Núm
  28,9; Núm 28,11; Núm 28,17; Núm 28,26; Núm 29,12}

\bibleverse{14} Designó, según la ordenanza de su padre David, a las
divisiones de los sacerdotes para su servicio, y a los levitas para sus
oficios, para que alabaran y ministraran delante de los sacerdotes,
según el deber de cada día, a los porteros también por sus divisiones en
cada puerta, porque así lo había ordenado David, el hombre de Dios.
\footnote{\textbf{8:14} 1Cró 23,1-26} \bibleverse{15} No se apartaron
del mandato del rey a los sacerdotes y a los levitas en cuanto a
cualquier asunto o en cuanto a los tesoros.

\bibleverse{16} Toda la obra de Salomón se llevó a cabo desde el día de
la fundación de la casa de Yahvé hasta su finalización. Así que la casa
de Yahvé fue completada.

\hypertarget{paseos-de-ofir-de-salomuxf3n}{%
\subsection{Paseos de Ofir de
Salomón}\label{paseos-de-ofir-de-salomuxf3n}}

\bibleverse{17} Entonces Salomón fue a Ezión Geber y a Elot, a la orilla
del mar en la tierra de Edom. \bibleverse{18} Huram le envió barcos y
siervos que conocían el mar por mano de sus siervos; y vinieron con los
siervos de Salomón a Ofir, y trajeron de allí cuatrocientos cincuenta
talentos de oro, y los llevaron al rey Salomón.

\hypertarget{visita-de-la-reina-de-saba}{%
\subsection{Visita de la Reina de
Saba}\label{visita-de-la-reina-de-saba}}

\hypertarget{section-8}{%
\section{9}\label{section-8}}

\bibleverse{1} Cuando la reina de Sabá se enteró de la fama de Salomón,
vino a poner a prueba a Salomón con preguntas difíciles en Jerusalén,
con una caravana muy grande, con camellos que llevaban especias, oro en
abundancia y piedras preciosas. Cuando llegó a Salomón, le habló de todo
lo que tenía en su corazón. \bibleverse{2} Salomón respondió a todas sus
preguntas. No hubo nada que se le ocultara a Salomón que no le dijera.
\bibleverse{3} Cuando la reina de Sabá vio la sabiduría de Salomón, la
casa que había construido, \bibleverse{4} la comida de su mesa, los
asientos de sus sirvientes, la asistencia de sus ministros, su ropa, sus
coperos y su vestimenta, y su ascenso por el que subía a la casa de
Yahvé, no hubo más espíritu en ella.

\bibleverse{5} Ella dijo al rey: ``Fue un informe verdadero el que oí en
mi tierra sobre tus actos y tu sabiduría. \bibleverse{6} Sin embargo, no
creí sus palabras hasta que llegué y mis ojos lo vieron; y he aquí que
la mitad de la grandeza de tu sabiduría no me fue contada. Superas la
fama que he oído. \bibleverse{7} Dichosos tus hombres, y dichosos estos
tus siervos, que están continuamente ante ti y oyen tu sabiduría.
\footnote{\textbf{9:7} Luc 10,23} \bibleverse{8} Bendito sea Yahvé, tu
Dios, que se deleitó en ti y te puso en su trono para que fueras rey de
Yahvé, tu Dios, porque tu Dios amó a Israel, para establecerlo para
siempre. Por eso te hizo rey sobre ellos, para que hicieras justicia y
rectitud''.

\bibleverse{9} Ella le dio al rey ciento veinte talentos de oro,
especias en gran abundancia y piedras preciosas. Nunca antes hubo tantas
especias como las que la reina de Saba dio al rey Salomón.

\bibleverse{10} Los siervos de Hiram y los siervos de Salomón, que
trajeron oro de Ofir, también trajeron árboles de algum y piedras
preciosas. \bibleverse{11} El rey utilizó la madera de algum para hacer
terrazas para la casa de Yavé y para la casa del rey, y arpas e
instrumentos de cuerda para los cantantes. No se había visto nada igual
en el país de Judá. \bibleverse{12} El rey Salomón le dio a la reina de
Sabá todo lo que pidió, más de lo que había traído al rey. Entonces se
volvió y se fue a su tierra, ella y sus sirvientes.

\hypertarget{riqueza-obras-de-arte-y-esplendor-de-salomuxf3n-y-artuxedculos-de-comercio-exterior}{%
\subsection{Riqueza, obras de arte y esplendor de Salomón y artículos de
comercio
exterior}\label{riqueza-obras-de-arte-y-esplendor-de-salomuxf3n-y-artuxedculos-de-comercio-exterior}}

\bibleverse{13} El peso del oro que llegó a Salomón en un año fue de
seiscientos sesenta y seis talentosde oro, \bibleverse{14} además de lo
que trajeron los comerciantes y mercaderes. Todos los reyes de Arabia y
los gobernadores del país trajeron oro y plata a Salomón.
\bibleverse{15} El rey Salomón hizo doscientos escudos grandes de oro
batido. Seiscientos siclos de oro batido fueron para un escudo grande.
\bibleverse{16} Hizo trescientos escudos de oro batido. Trescientos
siclos de oro fueron para un escudo. El rey los puso en la Casa del
Bosque del Líbano. \bibleverse{17} Además, el rey hizo un gran trono de
marfil y lo recubrió de oro puro. \bibleverse{18} Había seis escalones
para el trono, con un escabel de oro, que estaban fijados al trono, y
reposabrazos a cada lado junto al lugar del asiento, y dos leones de pie
junto a los reposabrazos. \bibleverse{19} Doce leones se encontraban
allí a un lado y al otro en los seis escalones. No se hizo nada parecido
en ningún otro reino. \bibleverse{20} Todos los vasos del rey Salomón
eran de oro, y todos los vasos de la Casa del Bosque del Líbano eran de
oro puro. La plata no se consideraba valiosa en los días de Salomón.
\bibleverse{21} Porque el rey tenía barcos que iban a Tarsis con los
servidores de Hiram. Una vez cada tres años, los barcos de Tarsis
llegaban trayendo oro, plata, marfil, monos y pavos reales.

\hypertarget{la-posiciuxf3n-de-poder-de-salomuxf3n-y-la-riqueza-que-promueve}{%
\subsection{La posición de poder de Salomón y la riqueza que
promueve}\label{la-posiciuxf3n-de-poder-de-salomuxf3n-y-la-riqueza-que-promueve}}

\bibleverse{22} Así, el rey Salomón superó a todos los reyes de la
tierra en riqueza y sabiduría. \bibleverse{23} Todos los reyes de la
tierra buscaban la presencia de Salomón para escuchar su sabiduría, que
Dios había puesto en su corazón. \bibleverse{24} Cada uno de ellos traía
un tributo: vasos de plata, vasos de oro, ropa, armaduras, especias,
caballos y mulas cada año. \bibleverse{25} Salomón tenía cuatro mil
establos para caballos y carros, y doce mil jinetes que destinaba a las
ciudades de los carros y al rey en Jerusalén. \footnote{\textbf{9:25}
  2Cró 1,14-17; 1Re 4,26} \bibleverse{26} Dominaba a todos los reyes
desde el río hasta el país de los filisteos y hasta la frontera de
Egipto. \bibleverse{27} El rey hizo que la plata fuera tan común en
Jerusalén como las piedras, e hizo que los cedros fueran tan abundantes
como los sicómoros que hay en las tierras bajas. \bibleverse{28}
Trajeron caballos para Salomón de Egipto y de todas las tierras.

\hypertarget{las-fuentes-de-la-historia-de-salomuxf3n-su-muerte}{%
\subsection{Las fuentes de la historia de Salomón; su
muerte}\label{las-fuentes-de-la-historia-de-salomuxf3n-su-muerte}}

\bibleverse{29} Los demás hechos de Salomón, los primeros y los últimos,
¿no están escritos en la historia del profeta Natán, en la profecía de
Ahías el silonita y en las visiones del vidente Iddo sobre Jeroboam hijo
de Nabat? \footnote{\textbf{9:29} 1Re 11,29} \bibleverse{30} Salomón
reinó en Jerusalén sobre todo Israel durante cuarenta años.
\bibleverse{31} Salomón durmió con sus padres, y fue enterrado en la
ciudad de su padre David; y reinó en su lugar Roboam, su hijo.

\hypertarget{roboam-y-jeroboam-en-siquem-la-divisiuxf3n-del-imperio}{%
\subsection{Roboam y Jeroboam en Siquem; la división del
imperio}\label{roboam-y-jeroboam-en-siquem-la-divisiuxf3n-del-imperio}}

\hypertarget{section-9}{%
\section{10}\label{section-9}}

\bibleverse{1} Roboam fue a Siquem, porque todo Israel había acudido a
Siquem para hacerle rey. \bibleverse{2} Cuando Jeroboam hijo de Nabat se
enteró de ello (pues estaba en Egipto, donde había huido de la presencia
del rey Salomón), Jeroboam volvió de Egipto. \footnote{\textbf{10:2} 1Re
  11,40} \bibleverse{3} Enviaron y lo llamaron; y vino Jeroboam y todo
Israel, y hablaron a Roboam, diciendo: \bibleverse{4} ``Tu padre hizo
gravoso nuestro yugo. Ahora, pues, aligera el penoso servicio de tu
padre y el pesado yugo que puso sobre nosotros, y te serviremos''.

\bibleverse{5} Les dijo: ``Volved a mí después de tres días''. Así que
la gente se fue.

\hypertarget{consejeruxeda-de-rehoboams}{%
\subsection{Consejería de Rehoboams}\label{consejeruxeda-de-rehoboams}}

\bibleverse{6} El rey Roboam consultó a los ancianos que habían estado
delante de Salomón, su padre, cuando aún vivía, diciendo: ``¿Qué consejo
me dais sobre cómo responder a esta gente?''

\bibleverse{7} Le hablaron diciendo: ``Si eres amable con esta gente, la
complaces y les hablas con buenas palabras, entonces serán tus siervos
para siempre.''

\bibleverse{8} Pero abandonó el consejo de los ancianos que le habían
dado, y tomó consejo con los jóvenes que habían crecido con él, que
estaban delante de él. \bibleverse{9} Les dijo: ``¿Qué consejo les dais
para que respondamos a esta gente, que me ha hablado diciendo: ``Aligera
el yugo que tu padre puso sobre nosotros''?''

\bibleverse{10} Los jóvenes que se habían criado con él le hablaron
diciendo: ``Así dirás al pueblo que te habló diciendo: ``Tu padre hizo
pesado nuestro yugo, pero aligéralo sobre nosotros''; así les dirás:
``Mi dedo meñique es más grueso que la cintura de mi padre''.
\bibleverse{11} Ahora bien, mientras mi padre os cargó con un yugo
pesado, yo añadiré a vuestro yugo. Mi padre os castigó con látigos, pero
yo os castigaré con escorpiones'\,''.

\hypertarget{descenso-de-las-diez-tribus-elecciuxf3n-de-jeroboam-como-rey-de-israel}{%
\subsection{Descenso de las diez tribus; Elección de Jeroboam como rey
de
Israel}\label{descenso-de-las-diez-tribus-elecciuxf3n-de-jeroboam-como-rey-de-israel}}

\bibleverse{12} Entonces Jeroboam y todo el pueblo vinieron a Roboam al
tercer día, tal como el rey lo había pedido, diciendo: ``Volved a mí al
tercer día''. \bibleverse{13} El rey les respondió con aspereza; y el
rey Roboam abandonó el consejo de los ancianos, \bibleverse{14} y les
habló según el consejo de los jóvenes, diciendo: ``Mi padre hizo pesado
vuestro yugo, pero yo lo aumentaré. Mi padre os castigó con látigos,
pero yo os castigaré con escorpiones''.

\bibleverse{15} Así que el rey no escuchó al pueblo, pues esto fue
provocado por Dios, para que Yahvé confirmara su palabra, que habló por
medio de Ahías el silonita a Jeroboam hijo de Nabat. \footnote{\textbf{10:15}
  1Re 11,29; 1Re 11,31}

\bibleverse{16} Cuando todo Israel vio que el rey no los escuchaba, el
pueblo respondió al rey diciendo: ``¿Qué parte tenemos en David? ¡No
tenemos herencia en el hijo de Isaí! ¡Cada uno a sus tiendas, Israel!
Ahora ocúpate de tu propia casa, David''. Y todo Israel se fue a sus
tiendas.

\bibleverse{17} Pero en cuanto a los hijos de Israel que vivían en las
ciudades de Judá, Roboam reinó sobre ellos. \bibleverse{18} Entonces el
rey Roboam envió a Hadoram, que estaba a cargo de los hombres sometidos
a trabajos forzados, y los hijos de Israel lo mataron a pedradas. El rey
Roboam se apresuró a subir a su carro, para huir a Jerusalén.
\bibleverse{19} Así se rebeló Israel contra la casa de David hasta el
día de hoy.

\hypertarget{roboam-se-abstiene-de-la-guerra-contra-israel-bajo-la-direcciuxf3n-de-dios}{%
\subsection{Roboam se abstiene de la guerra contra Israel bajo la
dirección de
Dios}\label{roboam-se-abstiene-de-la-guerra-contra-israel-bajo-la-direcciuxf3n-de-dios}}

\hypertarget{section-10}{%
\section{11}\label{section-10}}

\bibleverse{1} Cuando Roboam llegó a Jerusalén, reunió a la casa de Judá
y de Benjamín, ciento ochenta mil hombres escogidos que eran guerreros,
para luchar contra Israel, para devolver el reino a Roboam.
\bibleverse{2} Pero la palabra de Yavé llegó a Semaías, hombre de Dios,
diciendo: \bibleverse{3} ``Habla a Roboam hijo de Salomón, rey de Judá,
y a todo Israel en Judá y Benjamín, diciendo: \bibleverse{4} `Dice Yavé:
``¡No subiréis ni lucharéis contra vuestros hermanos! Volved cada uno a
su casa, porque esto es cosa mía''\,''. Así que escucharon las palabras
de Yahvé, y volvieron de ir contra Jeroboam.

\hypertarget{fortalezas-de-roboam}{%
\subsection{Fortalezas de Roboam}\label{fortalezas-de-roboam}}

\bibleverse{5} Roboam vivió en Jerusalén y construyó ciudades de defensa
en Judá. \bibleverse{6} Edificó Belén, Etam, Tecoa, \bibleverse{7} Bet
Zur, Soco, Adulam, \bibleverse{8} Gat, Mareshah, Zif, \bibleverse{9}
Adoraim, Laquis, Azeca, \bibleverse{10} Zora, Ajalón y Hebrón, que son
ciudades fortificadas en Judá y en Benjamín. \bibleverse{11} Fortificó
las fortalezas y puso en ellas capitanes con provisiones de comida,
aceite y vino. \bibleverse{12} Puso escudos y lanzas en todas las
ciudades y las hizo muy fuertes. Judá y Benjamín le pertenecían.

\hypertarget{entrada-de-sacerdotes-levitas-y-personas-piadosas-del-reino-de-diez-tribus}{%
\subsection{Entrada de sacerdotes, levitas y personas piadosas del reino
de diez
tribus}\label{entrada-de-sacerdotes-levitas-y-personas-piadosas-del-reino-de-diez-tribus}}

\bibleverse{13} Los sacerdotes y los levitas que había en todo Israel se
presentaron con él desde todo su territorio. \bibleverse{14} Porque los
levitas dejaron sus tierras de pastoreo y sus posesiones y vinieron a
Judá y a Jerusalén, pues Jeroboam y sus hijos los desecharon para que no
ejercieran el oficio de sacerdote a Yahvé. \footnote{\textbf{11:14} 2Cró
  13,9} \bibleverse{15} Él mismo nombró sacerdotes para los lugares
altos, para los ídolos machos cabríos y becerros que había hecho.
\footnote{\textbf{11:15} 1Re 12,31} \bibleverse{16} Después de ellos, de
todas las tribus de Israel, los que se propusieron buscar a Yavé, el
Dios de Israel, vinieron a Jerusalén a sacrificar a Yavé, el Dios de sus
padres. \bibleverse{17} Así fortalecieron el reino de Judá e hicieron
fuerte a Roboam, hijo de Salomón, durante tres años, pues caminaron tres
años por el camino de David y Salomón.

\hypertarget{historia-familiar-de-rehaboam}{%
\subsection{Historia familiar de
rehaboam}\label{historia-familiar-de-rehaboam}}

\bibleverse{18} Roboam tomó como esposa a Mahalat, hija de Jerimot, hijo
de David, y de Abihail, hija de Eliab, hijo de Jesé. \footnote{\textbf{11:18}
  1Sam 16,6} \bibleverse{19} Ella le dio hijos: Jeús, Semarías y Zaham.
\bibleverse{20} Después de ella, tomó a Maaca, nieta de Absalón, y ella
le dio a luz a Abías, Atai, Ziza y Selomit. \bibleverse{21} Roboam amaba
a Maaca, nieta de Absalón, por encima de todas sus esposas y concubinas,
pues tomó dieciocho esposas y sesenta concubinas, y fue padre de
veintiocho hijos y sesenta hijas. \bibleverse{22} Roboam designó a
Abías, hijo de Maaca, como jefe, como príncipe entre sus hermanos, pues
pensaba hacerlo rey. \bibleverse{23} Hizo un trato sabio, y dispersó a
algunos de sus hijos por todas las tierras de Judá y Benjamín, en todas
las ciudades fortificadas. Les dio comida en abundancia, y les buscó
muchas esposas. \footnote{\textbf{11:23} 2Cró 21,3}

\hypertarget{incursiuxf3n-y-saqueo-del-rey-egipcio-sisak-apariciuxf3n-del-profeta-semeuxedas}{%
\subsection{Incursión y saqueo del rey egipcio Sisak; Aparición del
profeta
Semeías}\label{incursiuxf3n-y-saqueo-del-rey-egipcio-sisak-apariciuxf3n-del-profeta-semeuxedas}}

\hypertarget{section-11}{%
\section{12}\label{section-11}}

\bibleverse{1} Cuando el reino de Roboam se estableció y se hizo fuerte,
abandonó la ley de Yahvé, y todo Israel con él. \bibleverse{2} En el
quinto año del rey Roboam, Sisac, rey de Egipto, subió contra Jerusalén,
porque habían prevaricado contra Yavé, \bibleverse{3} con mil doscientos
carros y sesenta mil jinetes. Los pueblos que vinieron con él desde
Egipto eran innumerables: los lubines, los suquines y los etíopes.
\bibleverse{4} Tomó las ciudades fortificadas que pertenecían a Judá y
llegó a Jerusalén. \footnote{\textbf{12:4} 2Cró 11,4-10} \bibleverse{5}
El profeta Semaías vino a Roboam y a los príncipes de Judá que estaban
reunidos en Jerusalén a causa de Sisac, y les dijo: ``Yahvé dice:
`Ustedes me han abandonado, por eso yo también los he dejado en manos de
Sisac'.''

\bibleverse{6} Entonces los príncipes de Israel y el rey se humillaron y
dijeron: ``Yahvé es justo''.

\bibleverse{7} Cuando Yahvé vio que se humillaban, llegó la palabra de
Yahvé a Semaías, diciendo: ``Se han humillado. No los destruiré, sino
que les concederé alguna liberación, y mi ira no se derramará sobre
Jerusalén por la mano de Sisac. \bibleverse{8} Sin embargo, serán sus
servidores, para que conozcan mi servicio y el de los reinos de los
países.''

\bibleverse{9} Entonces Sisac, rey de Egipto, subió contra Jerusalén y
se llevó los tesoros de la casa de Yahvé y los tesoros de la casa del
rey. Se lo llevó todo. También se llevó los escudos de oro que había
hecho Salomón. \bibleverse{10} El rey Roboam hizo escudos de bronce en
su lugar, y los encomendó a los capitanes de la guardia que custodiaban
la puerta de la casa del rey. \bibleverse{11} Cada vez que el rey
entraba en la casa de Yavé, la guardia venía y los llevaba, y luego los
devolvía a la sala de guardia. \bibleverse{12} Cuando se humilló, la ira
de Yavé se apartó de él, para no destruirlo del todo. Además, se
encontraron cosas buenas en Judá.

\hypertarget{conclusiuxf3n-del-gobierno-de-roboam-y-las-fuentes-de-su-historia}{%
\subsection{Conclusión del gobierno de Roboam y las fuentes de su
historia}\label{conclusiuxf3n-del-gobierno-de-roboam-y-las-fuentes-de-su-historia}}

\bibleverse{13} El rey Roboam se afianzó en Jerusalén y reinó; pues
Roboam tenía cuarenta y un años cuando comenzó a reinar, y reinó
diecisiete años en Jerusalén, la ciudad que Yahvé había elegido de entre
todas las tribus de Israel para poner su nombre en ella. Su madre se
llamaba Naamah la amonita. \footnote{\textbf{12:13} 2Cró 6,20}
\bibleverse{14} Hizo lo que era malo, porque no puso su corazón a buscar
a Yavé.

\bibleverse{15} Los hechos de Roboam, primero y último, ¿no están
escritos en las historias del profeta Semaías y del vidente Iddo, en las
genealogías? Hubo guerras entre Roboam y Jeroboam continuamente.
\footnote{\textbf{12:15} 2Cró 13,22} \bibleverse{16} Roboam durmió con
sus padres y fue sepultado en la ciudad de David; y su hijo Abías reinó
en su lugar.

\hypertarget{la-guerra-de-abias-con-jeroboam-su-discurso-al-ejuxe9rcito-de-jeroboam}{%
\subsection{La guerra de Abias con Jeroboam; su discurso al ejército de
Jeroboam}\label{la-guerra-de-abias-con-jeroboam-su-discurso-al-ejuxe9rcito-de-jeroboam}}

\hypertarget{section-12}{%
\section{13}\label{section-12}}

\bibleverse{1} En el año dieciocho del rey Jeroboam, Abías comenzó a
reinar sobre Judá. \bibleverse{2} Reinó tres años en Jerusalén. Su madre
se llamaba Micaías, hija de Uriel de Guibeá. Hubo guerra entre Abías y
Jeroboam. \bibleverse{3} Abías se alistó en la batalla con un ejército
de valientes hombres de guerra, cuatrocientos mil hombres escogidos, y
Jeroboam preparó la batalla contra él con ochocientos mil hombres
escogidos, que eran hombres valientes. \bibleverse{4} Abías se levantó
en el monte Zemaraim, que está en la región montañosa de Efraín, y dijo:
``Oídme, Jeroboam y todo Israel: \bibleverse{5} ¿No sabéis que Yahvé, el
Dios de Israel, dio el reino sobre Israel a David para siempre, a él y a
sus hijos por un pacto de sal? \footnote{\textbf{13:5} Lev 2,13; Núm
  18,19} \bibleverse{6} Pero Jeroboam, hijo de Nabat, siervo de Salomón,
hijo de David, se levantó y se rebeló contra su señor. \bibleverse{7} Se
juntaron con él hombres inútiles, compañeros perversos que se
fortalecieron contra Roboam hijo de Salomón, cuando Roboam era joven y
de corazón tierno, y no pudo resistirlos.

\bibleverse{8} ``Ahora pretendéis resistir el reino de Yahvé en manos de
los hijos de David. Sois una gran multitud, y los becerros de oro que
Jeroboam os hizo como dioses están con vosotros. \footnote{\textbf{13:8}
  1Re 12,28} \bibleverse{9} ¿No habéis expulsado a los sacerdotes de
Yavé, a los hijos de Aarón y a los levitas, y os habéis hecho sacerdotes
según las costumbres de los pueblos de otras tierras? El que viene a
consagrarse con un novillo y siete carneros puede ser sacerdote de los
que no son dioses. \footnote{\textbf{13:9} 2Cró 11,15}

\bibleverse{10} ``Pero en cuanto a nosotros, Yahvé es nuestro Dios, y no
lo hemos abandonado. Tenemos sacerdotes que sirven a Yavé, los hijos de
Aarón y los levitas en su trabajo. \bibleverse{11} Ellos queman a Yavé
todas las mañanas y todas las tardes holocaustos e incienso aromático.
También ponen en orden el pan de la feria en la mesa pura, y cuidan el
candelabro de oro con sus lámparas, para que ardan todas las tardes;
porque nosotros guardamos la instrucción de Yavé, nuestro Dios, pero
ustedes lo han abandonado. \footnote{\textbf{13:11} Núm 28,3-8}
\bibleverse{12} He aquí que Dios está con nosotros a la cabeza, y sus
sacerdotes con las trompetas de alarma para dar la alarma contra
vosotros. Hijos de Israel, no luchéis contra Yahvé, el Dios de vuestros
padres, porque no prosperaréis.'' \footnote{\textbf{13:12} Núm 10,9}

\hypertarget{victoria-de-abias-sobre-jeroboam}{%
\subsection{Victoria de Abias sobre
Jeroboam}\label{victoria-de-abias-sobre-jeroboam}}

\bibleverse{13} Pero Jeroboam hizo que se formara una emboscada detrás
de ellos; así que estaban delante de Judá, y la emboscada estaba detrás
de ellos. \bibleverse{14} Cuando Judá miró hacia atrás, he aquí que la
batalla estaba delante y detrás de ellos; y clamaron a Yahvé, y los
sacerdotes tocaron las trompetas. \bibleverse{15} Entonces los hombres
de Judá dieron un grito. Mientras los hombres de Judá gritaban, Dios
hirió a Jeroboam y a todo Israel ante Abías y Judá. \bibleverse{16} Los
hijos de Israel huyeron ante Judá, y Dios los entregó en su mano.
\bibleverse{17} Abías y su gente los mataron con gran mortandad, de modo
que quinientos mil hombres escogidos de Israel cayeron muertos.
\bibleverse{18} Así fueron doblegados los hijos de Israel en aquel
tiempo, y los hijos de Judá prevalecieron, porque se apoyaron en Yavé,
el Dios de sus padres. \bibleverse{19} Abías persiguió a Jeroboam y le
arrebató ciudades: Betel con sus aldeas, Jeshana con sus aldeas y Efrón
con sus aldeas.

\bibleverse{20} Jeroboam no volvió a recuperar fuerzas en los días de
Abías. El Señor lo hirió y murió.

\hypertarget{conclusiuxf3n-y-fuentes-de-la-historia-de-abias}{%
\subsection{Conclusión y fuentes de la historia de
Abias}\label{conclusiuxf3n-y-fuentes-de-la-historia-de-abias}}

\bibleverse{21} Pero Abías se hizo poderoso y tomó para sí catorce
esposas, y fue padre de veintidós hijos y dieciséis hijas.
\bibleverse{22} El resto de los hechos de Abías, sus caminos y sus
dichos están escritos en el comentario del profeta Iddo. \footnote{\textbf{13:22}
  2Cró 12,15}

\hypertarget{la-intervenciuxf3n-de-asa-contra-la-idolatruxeda}{%
\subsection{La intervención de Asa contra la
idolatría}\label{la-intervenciuxf3n-de-asa-contra-la-idolatruxeda}}

\hypertarget{section-13}{%
\section{14}\label{section-13}}

\bibleverse{1} Así que Abías durmió con sus padres, y lo enterraron en
la ciudad de David; y su hijo Asa reinó en su lugar. En sus días, la
tierra estuvo tranquila diez años. \bibleverse{2} Asá hizo lo que era
bueno y correcto a los ojos de Yavé, su Dios, \footnote{\textbf{14:2}
  1Re 15,11-12} \bibleverse{3} pues quitó los altares extranjeros y los
lugares altos, derribó las columnas, cortó los postes de Asera,
\bibleverse{4} y ordenó a Judá que buscara a Yavé, el Dios de sus
padres, y que obedeciera su ley y su mandato. \bibleverse{5} También
quitó de todas las ciudades de Judá los lugares altos y las imágenes del
sol, y el reino quedó tranquilo ante él.

\hypertarget{eleva-la-fuerza-defensiva-del-imperio}{%
\subsection{Eleva la fuerza defensiva del
imperio}\label{eleva-la-fuerza-defensiva-del-imperio}}

\bibleverse{6} Edificó ciudades fortificadas en Judá, pues la tierra
estaba tranquila, y no tuvo guerras en esos años, porque el Señor le
había dado descanso. \footnote{\textbf{14:6} 2Cró 15,15} \bibleverse{7}
Pues dijo a Judá: ``Construyamos estas ciudades y hagamos muros
alrededor de ellas, con torres, puertas y rejas. La tierra está aún ante
nosotros, porque hemos buscado a Yavé, nuestro Dios. Lo hemos buscado, y
él nos ha dado descanso por todos lados''. Así construyeron y
prosperaron.

\hypertarget{la-victoria-de-asa-sobre-los-cusitas-serah}{%
\subsection{La victoria de Asa sobre los cusitas
Serah}\label{la-victoria-de-asa-sobre-los-cusitas-serah}}

\bibleverse{8} Asá tenía un ejército de trescientos mil de Judá, que
llevaban escudos y lanzas, y doscientos ochenta mil de Benjamín, que
llevaban escudos y tensaban arcos. Todos ellos eran hombres de gran
valor.

\bibleverse{9} Zéraj el etíope salió contra ellos con un ejército de un
millón de soldados y trescientos carros, y llegó a Maresá.
\bibleverse{10} Asá salió a su encuentro, y prepararon la batalla en el
valle de Cefatá, en Maresá. \bibleverse{11} Asá clamó a su Dios, y dijo:
``Señor, no hay nadie más que tú para ayudar, entre el poderoso y el que
no tiene fuerza. Ayúdanos, Yahvé, nuestro Dios, porque en ti confiamos,
y en tu nombre venimos contra esta multitud. Yahvé, tú eres nuestro
Dios. No dejes que el hombre prevalezca contra ti''. \footnote{\textbf{14:11}
  1Sam 14,6}

\bibleverse{12} El Señor hirió a los etíopes ante Asa y ante Judá, y los
etíopes huyeron. \bibleverse{13} Asá y el pueblo que estaba con él los
persiguieron hasta Gerar. Fueron tantos los etíopes que cayeron que no
pudieron recuperarse, pues fueron destruidos ante el Señor y ante su
ejército. El ejército de Judá se llevó mucho botín. \bibleverse{14}
Atacaron todas las ciudades alrededor de Gerar, porque el temor de Yavé
se apoderó de ellas. Saquearon todas las ciudades, pues había mucho
botín en ellas. \bibleverse{15} También atacaron las tiendas de los que
tenían ganado, y se llevaron ovejas y camellos en abundancia, y luego
regresaron a Jerusalén.

\hypertarget{la-amonestaciuxf3n-del-profeta-azaruxedas}{%
\subsection{La amonestación del profeta
Azarías}\label{la-amonestaciuxf3n-del-profeta-azaruxedas}}

\hypertarget{section-14}{%
\section{15}\label{section-14}}

\bibleverse{1} El Espíritu de Dios vino sobre Azarías, hijo de Oded.
\bibleverse{2} Salió al encuentro de Asá y le dijo: ``¡Escúchame, Asá, y
todo Judá y Benjamín! Yahvé está con vosotros mientras estéis con él; y
si lo buscáis, será encontrado por vosotros; pero si lo abandonáis, él
os abandonará. \bibleverse{3} Durante mucho tiempo Israel estuvo sin el
Dios verdadero, sin sacerdote que enseñara y sin ley. \footnote{\textbf{15:3}
  Os 3,4} \bibleverse{4} Pero cuando en su angustia se volvieron a
Yahvé, el Dios de Israel, y lo buscaron, fue encontrado por ellos.
\footnote{\textbf{15:4} Jer 29,13-14} \bibleverse{5} En aquellos tiempos
no había paz para el que salía ni para el que entraba, sino que había
grandes problemas para todos los habitantes de las tierras.
\bibleverse{6} Fueron despedazados, nación contra nación, y ciudad
contra ciudad; porque Dios los turbó con toda adversidad. \bibleverse{7}
¡Pero tú sé fuerte! No dejes que tus manos se aflojen, porque tu trabajo
será recompensado''. \footnote{\textbf{15:7} 1Cor 15,58}

\hypertarget{renovaciuxf3n-de-asa-del-pacto-con-dios}{%
\subsection{Renovación de Asa del pacto con
Dios}\label{renovaciuxf3n-de-asa-del-pacto-con-dios}}

\bibleverse{8} Cuando Asa oyó estas palabras y la profecía del profeta
Oded, se animó y quitó las abominaciones de toda la tierra de Judá y de
Benjamín, y de las ciudades que había tomado de la región montañosa de
Efraín; y renovó el altar de Yahvé que estaba delante del pórtico de
Yahvé. \bibleverse{9} Reunió a todo Judá y Benjamín, y a los que vivían
con ellos, de Efraín, Manasés y Simeón; porque vinieron a él desde
Israel en abundancia, al ver que Yahvé, su Dios, estaba con él.
\bibleverse{10} Y se reunieron en Jerusalén en el mes tercero, en el año
quince del reinado de Asá. \bibleverse{11} Aquel día sacrificaron a
Yavé, del botín que habían traído, setecientas cabezas de ganado y siete
mil ovejas. \bibleverse{12} Hicieron el pacto de buscar a Yavé, el Dios
de sus padres, con todo su corazón y con toda su alma; \footnote{\textbf{15:12}
  Jos 24,25} \bibleverse{13} y de que todo aquel que no buscara a Yavé,
el Dios de Israel, debía morir, ya fuera pequeño o grande, ya fuera
hombre o mujer. \bibleverse{14} Juraron a Yavé a gran voz, con gritos,
con trompetas y con cornetas. \bibleverse{15} Todo Judá se alegró del
juramento, porque lo habían jurado de todo corazón y lo buscaban con
todo su deseo, y lo encontraron. Entonces Yahvé les dio descanso a
todos. \footnote{\textbf{15:15} 2Cró 14,6-7; 2Cró 20,30}

\bibleverse{16} También a Maaca, la madre del rey Asa, la destituyó de
su condición de reina madre, porque había hecho una imagen abominable
como Asera; así que Asa cortó su imagen, la redujo a polvo y la quemó en
el arroyo Cedrón. \footnote{\textbf{15:16} 1Re 15,13-15} \bibleverse{17}
Pero los lugares altos no fueron quitados de Israel; sin embargo, el
corazón de Asa fue perfecto durante todos sus días. \bibleverse{18}
Llevó a la casa de Dios las cosas que su padre había dedicado y que él
mismo había dedicado, plata, oro y utensilios.

\hypertarget{la-guerra-de-asa-con-baesa-de-israel-su-refugio-en-ben-adad-de-siria}{%
\subsection{La guerra de Asa con Baesa de Israel; su refugio en Ben-adad
de
Siria}\label{la-guerra-de-asa-con-baesa-de-israel-su-refugio-en-ben-adad-de-siria}}

\bibleverse{19} No hubo más guerra hasta el año treinta y cinco del
reinado de Asa.

\hypertarget{section-15}{%
\section{16}\label{section-15}}

\bibleverse{1} En el año treinta y seis del reinado de Asa, Baasa, rey
de Israel, subió contra Judá y edificó Ramá, para no dejar salir ni
entrar a nadie a Asa, rey de Judá. \footnote{\textbf{16:1} 1Re 15,16-22}
\bibleverse{2} Entonces Asa sacó plata y oro de los tesoros de la casa
de Yavé y de la casa real, y envió a Ben Hadad, rey de Siria, que vivía
en Damasco, diciendo: \bibleverse{3} ``Que haya un tratado entre tú y
yo, como lo hubo entre mi padre y tu padre. He aquí que te he enviado
plata y oro. Ve, rompe tu tratado con Baasa, rey de Israel, para que se
aparte de mí''.

\bibleverse{4} Ben Hadad escuchó al rey Asá y envió a los capitanes de
sus ejércitos contra las ciudades de Israel, y atacaron Ijón, Dan, Abel
Maim y todas las ciudades de almacenamiento de Neftalí. \bibleverse{5}
Cuando Baasa se enteró de esto, dejó de construir Ramá y dejó de
trabajar. \bibleverse{6} Entonces el rey Asá tomó a todo Judá, y se
llevaron las piedras y la madera de Rama, con las que Baasa había
construido; y con ellas edificó Geba y Mizpa.

\hypertarget{el-discurso-de-castigo-de-hanani-a-asa-tiene-un-efecto-negativo}{%
\subsection{El discurso de castigo de Hanani a Asa tiene un efecto
negativo}\label{el-discurso-de-castigo-de-hanani-a-asa-tiene-un-efecto-negativo}}

\bibleverse{7} En aquel tiempo el vidente Hanani vino a Asa, rey de
Judá, y le dijo: ``Como te has apoyado en el rey de Aram, y no te has
apoyado en Yavé, tu Dios, el ejército del rey de Aram se ha escapado de
tu mano. \footnote{\textbf{16:7} Jer 17,5} \bibleverse{8} ¿No eran los
etíopes y los lubinos un ejército enorme, con carros y muchísima gente
de a caballo? Sin embargo, como te apoyaste en el Señor, él los entregó
en tu mano. \footnote{\textbf{16:8} 2Cró 14,9-13} \bibleverse{9} Porque
los ojos de Yavé recorren toda la tierra, para mostrarse fuerte en favor
de aquellos cuyo corazón es perfecto para con él. Has hecho una tontería
en esto; porque a partir de ahora tendrás guerras''.

\bibleverse{10} Entonces Asa se enojó con el vidente y lo metió en la
cárcel, pues estaba furioso con él por este asunto. Asa oprimió al mismo
tiempo a algunos del pueblo. \footnote{\textbf{16:10} 2Cró 18,26; Mat
  14,3}

\hypertarget{el-fin-de-asa-y-un-entierro-honorable}{%
\subsection{El fin de Asa y un entierro
honorable}\label{el-fin-de-asa-y-un-entierro-honorable}}

\bibleverse{11} He aquí que los hechos de Asa, primeros y últimos, están
escritos en el libro de los reyes de Judá e Israel. \bibleverse{12} En
el año treinta y nueve de su reinado, Asa enfermó de los pies. Su
enfermedad era muy grave; sin embargo, en su enfermedad no buscó a Yavé,
sino sólo a los médicos. \bibleverse{13} Asá durmió con sus padres y
murió en el año cuarenta y uno de su reinado. \bibleverse{14} Lo
enterraron en su propia tumba, que él mismo había cavado en la ciudad de
David, y lo pusieron en el lecho que estaba lleno de olores dulces y de
diversas clases de especias preparadas por el arte de los perfumistas; y
le hicieron un fuego muy grande. \footnote{\textbf{16:14} 2Cró 21,19;
  Jer 34,5}

\hypertarget{el-gobierno-piadoso-y-feliz-de-josafat}{%
\subsection{El gobierno piadoso y feliz de
Josafat}\label{el-gobierno-piadoso-y-feliz-de-josafat}}

\hypertarget{section-16}{%
\section{17}\label{section-16}}

\bibleverse{1} Su hijo Josafat reinó en su lugar y se fortaleció contra
Israel. \footnote{\textbf{17:1} 1Re 15,24} \bibleverse{2} Colocó fuerzas
en todas las ciudades fortificadas de Judá, y puso guarniciones en la
tierra de Judá y en las ciudades de Efraín, que su padre había tomado.
\bibleverse{3} Yahvé estaba con Josafat, porque anduvo en los primeros
caminos de su padre David, y no buscó a los baales, \bibleverse{4} sino
que buscó al Dios de su padre, y anduvo en sus mandamientos, y no en los
caminos de Israel. \bibleverse{5} Por eso el Señor estableció el reino
en su mano. Todo Judá trajo tributo a Josafat, y él tuvo riquezas y
honores en abundancia. \footnote{\textbf{17:5} 2Cró 18,1} \bibleverse{6}
Su corazón se enalteció en los caminos de Yavé. Además, quitó de Judá
los lugares altos y los postes de Asera.

\hypertarget{josafat-instruye-al-pueblo-en-la-ley-del-seuxf1or}{%
\subsection{Josafat instruye al pueblo en la ley del
Señor}\label{josafat-instruye-al-pueblo-en-la-ley-del-seuxf1or}}

\bibleverse{7} También en el tercer año de su reinado envió a sus
príncipes: Ben Hail, Abdías, Zacarías, Netanel y Micaías, para que
enseñaran en las ciudades de Judá; \bibleverse{8} y con ellos a los
levitas: Semaías, Netanías, Zebadías, Asael, Semiramot, Jonatán,
Adonías, Tobías y Tobadonías, los levitas; y con ellos a Elisama y
Joram, los sacerdotes. \bibleverse{9} Ellos enseñaban en Judá, llevando
consigo el libro de la ley de Yahvé. Recorrieron todas las ciudades de
Judá y enseñaron entre el pueblo.

\hypertarget{la-reputaciuxf3n-de-josafat-entre-los-pueblos-vecinos-y-su-importante-poder-militar}{%
\subsection{La reputación de Josafat entre los pueblos vecinos y su
importante poder
militar}\label{la-reputaciuxf3n-de-josafat-entre-los-pueblos-vecinos-y-su-importante-poder-militar}}

\bibleverse{10} El temor del Señor cayó sobre todos los reinos de las
tierras que rodeaban a Judá, de modo que no hicieron guerra contra
Josafat. \bibleverse{11} Algunos de los filisteos le trajeron a Josafat
regalos y plata como tributo. Los árabes también le trajeron rebaños:
siete mil setecientos carneros y siete mil setecientos machos cabríos.
\footnote{\textbf{17:11} 1Re 4,21} \bibleverse{12} Josafat se
engrandeció mucho, y construyó fortalezas y ciudades de almacenamiento
en Judá. \bibleverse{13} Tuvo muchas obras en las ciudades de Judá; y
hombres de guerra, valientes, en Jerusalén. \bibleverse{14} Esta fue la
enumeración de ellos según las casas de sus padres: De Judá, los
capitanes de millares: Adná, el capitán, y con él trescientos mil
hombres valientes; \bibleverse{15} y junto a él Johanán, el capitán, y
con él doscientos ochenta mil; \bibleverse{16} y junto a él Amasías,
hijo de Zicri, que se ofreció voluntariamente a Yahvé, y con él
doscientos mil hombres valientes. \bibleverse{17} De Benjamín: Eliada,
hombre valiente, y con él doscientos mil armados con arco y escudo;
\bibleverse{18} y junto a él Jozabad, y con él ciento ochenta mil listos
y preparados para la guerra. \bibleverse{19} Estos eran los que
esperaban al rey, además de los que el rey puso en las ciudades
fortificadas de todo Judá.

\hypertarget{josafat-y-acab-unen-fuerzas-en-una-guerra-contra-los-sirios}{%
\subsection{Josafat y Acab unen fuerzas en una guerra contra los
sirios}\label{josafat-y-acab-unen-fuerzas-en-una-guerra-contra-los-sirios}}

\hypertarget{section-17}{%
\section{18}\label{section-17}}

\bibleverse{1} Josafat tenía riquezas y honores en abundancia, y se alió
con Acab. \footnote{\textbf{18:1} 2Cró 17,5} \bibleverse{2} Después de
algunos años, descendió con Acab a Samaria. Ajab mató para él ovejas y
ganado en abundancia, y para la gente que estaba con él, y lo movió a
subir con él a Ramot de Galaad. \bibleverse{3} Ajab, rey de Israel, dijo
a Josafat, rey de Judá: ``¿Quieres ir conmigo a Ramot de Galaad?'' Él le
respondió: ``Yo soy como tú, y mi pueblo como tu pueblo. Estaremos
contigo en la guerra''.

\hypertarget{el-mensaje-favorable-de-los-400-profetas-micha-deberuxeda-ser-entrevistado}{%
\subsection{El mensaje favorable de los 400 profetas; Micha debería ser
entrevistado}\label{el-mensaje-favorable-de-los-400-profetas-micha-deberuxeda-ser-entrevistado}}

\bibleverse{4} Josafat dijo al rey de Israel: ``Por favor, consulta
primero la palabra de Yavé''. \footnote{\textbf{18:4} 2Re 3,11}

\bibleverse{5} Entonces el rey de Israel reunió a los profetas,
cuatrocientos hombres, y les dijo: ``¿Vamos a Ramot de Galaad a
combatir, o me abstengo?'' Dijeron: ``Sube, porque Dios lo entregará en
mano del rey''.

\bibleverse{6} Pero Josafat dijo: ``¿No hay aquí otro profeta de Yahvé
para que podamos consultar con él?''

\bibleverse{7} El rey de Israel dijo a Josafat: ``Todavía hay un hombre
por el que podemos consultar a Yavé; pero lo odio, porque nunca
profetiza el bien respecto a mí, sino siempre el mal. Es Micaías, hijo
de Imla''. Josafat dijo: ``Que no lo diga el rey''.

\bibleverse{8} Entonces el rey de Israel llamó a un oficial y le dijo:
``Trae rápido a Micaías, hijo de Imla''.

\bibleverse{9} El rey de Israel y Josafat, rey de Judá, estaban sentados
cada uno en su trono, vestidos con sus ropas, y estaban sentados en un
lugar abierto a la entrada de la puerta de Samaria; y todos los profetas
estaban profetizando delante de ellos. \bibleverse{10} Sedequías, hijo
de Quená, se hizo unos cuernos de hierro y dijo: ``Yahvé dice: `Con
estos empujarás a los sirios hasta consumirlos'\,''.

\bibleverse{11} Todos los profetas lo profetizaron, diciendo: ``Sube a
Ramot de Galaad y prospera, porque Yahvé la entregará en manos del
rey.''

\hypertarget{la-buena-fortuna-inicial-de-micha-luego-su-anuncio-de-la-perdiciuxf3n}{%
\subsection{La buena fortuna inicial de Micha, luego su anuncio de la
perdición}\label{la-buena-fortuna-inicial-de-micha-luego-su-anuncio-de-la-perdiciuxf3n}}

\bibleverse{12} El mensajero que fue a llamar a Micaías le habló
diciendo: ``He aquí que las palabras de los profetas declaran el bien al
rey con una sola boca. Por lo tanto, haz que tu palabra sea como una de
las suyas, y habla bien''.

\bibleverse{13} Micaías dijo: ``Vive Yahvé, diré lo que dice mi Dios''.

\bibleverse{14} Cuando se presentó ante el rey, éste le dijo: ``Micaías,
¿vamos a Ramot de Galaad a combatir o me abstengo?'' Dijo: ``Sube y
prospera. Serán entregados en tu mano''.

\bibleverse{15} El rey le dijo: ``¿Cuántas veces he de conjurarte para
que no me digas más que la verdad en nombre de Yahvé?''

\bibleverse{16} Dijo: ``Vi a todo Israel disperso por los montes, como
ovejas que no tienen pastor. El Señor dijo: `Estas no tienen dueño. Que
cada uno vuelva a su casa en paz'\,''.

\bibleverse{17} El rey de Israel dijo a Josafat: ``¿No te dije que no
profetizaría el bien sobre mí, sino el mal?''

\bibleverse{18} Micaías dijo: ``Escuchen, pues, la palabra de Yahvé: Vi
a Yahvé sentado en su trono, y a todo el ejército del cielo de pie a su
derecha y a su izquierda. \bibleverse{19} Yahvé dijo: ``¿Quién atraerá a
Ajab, rey de Israel, para que suba y caiga en Ramot de Galaad? Uno habló
diciendo de esta manera, y otro diciendo de la otra. \bibleverse{20}
Salió un espíritu, se puso delante de Yavé y dijo: ``Yo lo atraeré.
``Yahvé le dijo: `¿Cómo?

\bibleverse{21} ``Dijo: `Iré y seré un espíritu mentiroso en la boca de
todos sus profetas'. ``Él dijo: `Tú lo atraerás, y también prevalecerás.
Ve y hazlo'.

\bibleverse{22} ``Ahora, pues, he aquí que Yahvé ha puesto un espíritu
mentiroso en la boca de estos tus profetas, y Yahvé ha hablado mal de
ti.''

\hypertarget{el-maltrato-de-miqueas-por-sedequuxedas-y-su-captura-por-acab}{%
\subsection{El maltrato de Miqueas por Sedequías y su captura por
Acab}\label{el-maltrato-de-miqueas-por-sedequuxedas-y-su-captura-por-acab}}

\bibleverse{23} Entonces se acercó Sedequías, hijo de Quená, y golpeó a
Micaías en la mejilla, y le dijo: ``¿Por dónde se fue de mí el Espíritu
de Yahvé para hablarte?'' \footnote{\textbf{18:23} 2Cró 18,10}

\bibleverse{24} Micaías dijo: ``He aquí, verás en ese día, cuando entres
en una habitación interior para esconderte''.

\bibleverse{25} El rey de Israel dijo: ``Tomen a Micaías y llévenlo a
Amón, el gobernador de la ciudad, y a Joás, el hijo del rey;
\bibleverse{26} y digan: ``El rey dice: ``Pongan a este hombre en la
cárcel, y aliméntenlo con pan de aflicción y con agua de aflicción,
hasta que yo regrese en paz''\,''. \footnote{\textbf{18:26} 2Cró 16,10}

\bibleverse{27} Micaías dijo: ``Si regresan en paz, Yahvé no ha hablado
por mí''. Dijo: ``¡Escuchen, pueblo, todos ustedes!''

\hypertarget{derrota-de-los-aliados-en-ramoth-muerte-de-acab}{%
\subsection{Derrota de los aliados en Ramoth; Muerte de
Acab}\label{derrota-de-los-aliados-en-ramoth-muerte-de-acab}}

\bibleverse{28} El rey de Israel y Josafat, rey de Judá, subieron a
Ramot de Galaad. \bibleverse{29} El rey de Israel dijo a Josafat: ``Yo
me disfrazaré y entraré en la batalla; pero tú ponte tus ropas''. Y el
rey de Israel se disfrazó, y entraron en la batalla. \bibleverse{30} El
rey de Siria había ordenado a los capitanes de sus carros que dijeran:
``No peleen con los pequeños ni con los grandes, sino sólo con el rey de
Israel.''

\bibleverse{31} Cuando los capitanes de los carros vieron a Josafat,
dijeron: ``¡Es el rey de Israel!'' Por eso se volvieron para luchar
contra él. Pero Josafat gritó, y el Señor lo ayudó; y Dios los hizo
alejarse de él. \bibleverse{32} Cuando los capitanes de los carros
vieron que no era el rey de Israel, dejaron de perseguirlo.
\bibleverse{33} Un hombre sacó su arco al azar e hirió al rey de Israel
entre las junturas de la armadura. Entonces dijo al conductor del carro:
``Da la vuelta y sácame de la batalla, porque estoy gravemente herido''.
\bibleverse{34} La batalla se intensificó aquel día. Sin embargo, el rey
de Israel se apuntaló en su carro contra los sirios hasta el atardecer;
y a eso de la puesta del sol, murió.

\hypertarget{discurso-de-castigo-del-profeta-jehuxfa-a-josafat}{%
\subsection{Discurso de castigo del profeta Jehú a
Josafat}\label{discurso-de-castigo-del-profeta-jehuxfa-a-josafat}}

\hypertarget{section-18}{%
\section{19}\label{section-18}}

\bibleverse{1} Josafat, rey de Judá, regresó a su casa en paz a
Jerusalén. \bibleverse{2} Jehú, hijo del vidente Hanani, salió a su
encuentro y le dijo al rey Josafat: ``¿Debes ayudar a los impíos y amar
a los que odian a Yavé? A causa de esto, la ira está sobre ti de parte
de Yahvé. \bibleverse{3} Sin embargo, se han encontrado cosas buenas en
ti, ya que has desechado a los asheróticos de la tierra y has puesto tu
corazón para buscar a Dios.'' \footnote{\textbf{19:3} 2Cró 17,3-6}

\hypertarget{la-reorganizaciuxf3n-de-la-administraciuxf3n-de-justicia-por-parte-de-josafat}{%
\subsection{La reorganización de la administración de justicia por parte
de
Josafat}\label{la-reorganizaciuxf3n-de-la-administraciuxf3n-de-justicia-por-parte-de-josafat}}

\bibleverse{4} Josafat vivía en Jerusalén, y volvió a salir entre el
pueblo desde Beerseba hasta la región montañosa de Efraín, y los hizo
volver a Yahvé, el Dios de sus padres. \bibleverse{5} Puso jueces en el
país por todas las ciudades fortificadas de Judá, ciudad por ciudad,
\bibleverse{6} y dijo a los jueces: ``Consideren lo que hacen, porque no
juzgan por el hombre, sino por Yavé; y él está con ustedes en el juicio.
\bibleverse{7} Ahora, pues, que el temor de Yahvé esté sobre vosotros.
Tened cuidado y hacedlo; porque no hay iniquidad con Yahvé nuestro Dios,
ni acepción de personas, ni aceptación de sobornos.'' \footnote{\textbf{19:7}
  Éxod 18,21; Deut 10,17}

\bibleverse{8} Además, Josafat nombró en Jerusalén a algunos levitas,
sacerdotes y jefes de familia de Israel para que juzgaran por Yahvé y
por las controversias. Ellos regresaron a Jerusalén. \footnote{\textbf{19:8}
  Deut 17,8-9; Deut 19,17} \bibleverse{9} Él les ordenó diciendo:
``Haréis esto en el temor de Yavé, con fidelidad y con un corazón
perfecto. \bibleverse{10} Siempre que os llegue alguna controversia de
vuestros hermanos que habitan en sus ciudades, entre sangre y sangre,
entre ley y mandamiento, estatutos y ordenanzas, debéis amonestarlos,
para que no sean culpables ante Yavé, y venga así la ira sobre vosotros
y sobre vuestros hermanos. Haz esto, y no serás culpable.
\bibleverse{11} He aquí que el sumo sacerdote Amarías está sobre
vosotros en todos los asuntos de Yavé; y Zebadías, hijo de Ismael, jefe
de la casa de Judá, en todos los asuntos del rey. También los levitas
serán oficiales ante ti. Trata con valentía, y que Yahvé esté con el
bien''.

\hypertarget{la-oraciuxf3n-de-josafat-despuuxe9s-de-que-el-enemigo-invadiuxf3}{%
\subsection{La oración de Josafat después de que el enemigo
invadió}\label{la-oraciuxf3n-de-josafat-despuuxe9s-de-que-el-enemigo-invadiuxf3}}

\hypertarget{section-19}{%
\section{20}\label{section-19}}

\bibleverse{1} Después de esto, los hijos de Moab, los hijos de Amón, y
con ellos algunos de los amonitas, vinieron contra Josafat para
combatir. \bibleverse{2} Entonces vinieron algunos que le dijeron a
Josafat: ``Una gran multitud viene contra ti desde el otro lado del mar,
desde Siria. He aquí que están en Hazazón Tamar'' (es decir, En Gedi).
\bibleverse{3} Josafat se alarmó y se puso a buscar a Yahvé. Proclamó un
ayuno en todo Judá. \bibleverse{4} Judá se reunió para pedir ayuda a
Yavé. Salieron de todas las ciudades de Judá para buscar a Yavé.
\footnote{\textbf{20:4} 2Cró 15,9-15}

\bibleverse{5} Josafat se puso de pie en la asamblea de Judá y
Jerusalén, en la casa de Yahvé, ante el nuevo tribunal; \bibleverse{6} y
dijo: ``Yahvé, el Dios de nuestros padres, ¿no eres tú el Dios del
cielo? ¿No eres tú el que gobierna todos los reinos de las naciones? El
poder y la fuerza están en tu mano, de modo que nadie puede resistirte.
\footnote{\textbf{20:6} 1Cró 29,12; 2Cró 14,11} \bibleverse{7} ¿No
expulsaste tú, Dios nuestro, a los habitantes de esta tierra antes que
tu pueblo Israel, y se la diste a la descendencia de Abraham, tu amigo,
para siempre? \bibleverse{8} Ellos vivieron en ella y te construyeron un
santuario en tu nombre, diciendo: \bibleverse{9} `Si nos sobreviene el
mal --- la espada, el juicio, la peste o el hambre --- nos presentaremos
ante esta casa y ante ti (pues tu nombre está en esta casa), y
clamaremos a ti en nuestra aflicción, y tú nos escucharás y salvarás.'
\footnote{\textbf{20:9} 2Cró 6,28-30} \bibleverse{10} Ahora bien, he
aquí que los hijos de Amón, de Moab y del monte Seír, a quienes no
dejaste invadir a Israel cuando salieron de la tierra de Egipto, pero se
apartaron de ellos y no los destruyeron; \footnote{\textbf{20:10} Deut
  2,4-5; Deut 2,9; Deut 2,19} \bibleverse{11} he aquí que nos
recompensan, para venir a echarnos de tu posesión, que nos has dado en
herencia. \bibleverse{12} Dios nuestro, ¿no los juzgarás? Porque no
tenemos fuerza contra esta gran compañía que viene contra nosotros. No
sabemos qué hacer, pero nuestros ojos están puestos en ti''. \footnote{\textbf{20:12}
  Éxod 14,14}

\bibleverse{13} Todo Judá se presentó ante el Señor, con sus pequeños,
sus mujeres y sus hijos.

\hypertarget{la-respuesta-de-dios-promesa-de-victoria-del-levita-jahaziel}{%
\subsection{La respuesta de Dios: promesa de victoria del levita
Jahaziel}\label{la-respuesta-de-dios-promesa-de-victoria-del-levita-jahaziel}}

\bibleverse{14} Entonces el Espíritu de Yahvé vino sobre Jahaziel hijo
de Zacarías, hijo de Benaía, hijo de Jeiel, hijo de Mattanías, levita,
de los hijos de Asaf, en medio de la asamblea; \bibleverse{15} y dijo:
``Escuchad, todo Judá, y vosotros, habitantes de Jerusalén, y tú, rey
Josafat. El Señor les dice: `No teman, ni se amedrenten a causa de esta
gran multitud; porque la batalla no es de ustedes, sino de Dios.
\bibleverse{16} Mañana, baja contra ellos. He aquí que suben por la
subida de Ziz. Los encontrarás al final del valle, antes del desierto de
Jeruel. \bibleverse{17} No será necesario que luchéis en esta batalla.
Pónganse firmes, quédense quietos y vean la salvación de Yavé con
ustedes, oh Judá y Jerusalén. No tengan miedo, ni se amedrenten. Salid
mañana contra ellos, porque Yahvé está con vosotros''.

\bibleverse{18} Josafat inclinó la cabeza con el rostro hacia el suelo,
y todo Judá y los habitantes de Jerusalén se postraron ante Yavé,
adorando a Yavé. \bibleverse{19} Los levitas, de los hijos de los
coatitas y de los hijos de los corasitas, se levantaron para alabar a
Yavé, el Dios de Israel, con una voz muy fuerte.

\hypertarget{la-salida-contra-el-enemigo-autodestrucciuxf3n-del-enemigo-enorme-botuxedn-de-los-juduxedos}{%
\subsection{La salida contra el enemigo; Autodestrucción del enemigo;
enorme botín de los
judíos}\label{la-salida-contra-el-enemigo-autodestrucciuxf3n-del-enemigo-enorme-botuxedn-de-los-juduxedos}}

\bibleverse{20} Se levantaron de madrugada y salieron al desierto de
Tecoa. Mientras salían, Josafat se puso de pie y dijo: ``¡Escúchenme,
Judá y ustedes, habitantes de Jerusalén! Creed en Yahvé, vuestro Dios,
para que seáis firmes. Creed a sus profetas, así prosperaréis''.
\footnote{\textbf{20:20} Is 28,16}

\bibleverse{21} Después de consultar con el pueblo, designó a los que
debían cantar a Yavé y alabar en formación sagrada al salir delante del
ejército, y decir: ``Dad gracias a Yavé, porque su bondad es eterna.''
\footnote{\textbf{20:21} Sal 106,1} \bibleverse{22} Cuando comenzaron a
cantar y a alabar, Yahvé puso emboscadas contra los hijos de Amón, de
Moab y del monte Seír, que habían venido contra Judá, y fueron
derrotados. \bibleverse{23} Porque los hijos de Amón y de Moab se
levantaron contra los habitantes del monte de Seír para matarlos y
destruirlos por completo. Cuando acabaron con los habitantes de Seir,
todos se ayudaron a destruirse mutuamente. \footnote{\textbf{20:23} 1Sam
  14,20}

\bibleverse{24} Cuando Judá llegó al lugar que daba al desierto, miraron
a la multitud; y he aquí que eran cadáveres caídos en tierra, y no había
quien escapara. \bibleverse{25} Cuando Josafat y su gente vinieron a
tomar su botín, encontraron entre ellos en abundancia tanto riquezas
como cadáveres con joyas preciosas, que despojaron para sí, más de lo
que podían llevar. Tomaron el botín durante tres días, pues era mucho.
\bibleverse{26} Al cuarto día se reunieron en el valle de Beracah,
porque allí bendijeron a Yavé. Por eso el nombre de ese lugar se llamó
``Valle de Beracá'' hasta el día de hoy. \bibleverse{27} Luego
regresaron, todos los hombres de Judá y de Jerusalén, con Josafat al
frente, para volver a Jerusalén con alegría, porque Yavé les había hecho
alegrarse de sus enemigos. \bibleverse{28} Llegaron a Jerusalén con
instrumentos de cuerda, arpas y trompetas a la casa de Yavé.
\bibleverse{29} El temor de Dios se apoderó de todos los reinos de los
países cuando oyeron que Yavé luchaba contra los enemigos de Israel.
\bibleverse{30} Así, el reino de Josafat estaba tranquilo, pues su Dios
le daba descanso en todo el territorio. \footnote{\textbf{20:30} 2Cró
  15,15}

\hypertarget{la-conclusiuxf3n-del-gobierno-de-josafat-y-las-fuentes-de-su-historia}{%
\subsection{La conclusión del gobierno de Josafat y las fuentes de su
historia}\label{la-conclusiuxf3n-del-gobierno-de-josafat-y-las-fuentes-de-su-historia}}

\bibleverse{31} Así reinó Josafat sobre Judá. Tenía treinta y cinco años
cuando comenzó a reinar. Reinó veinticinco años en Jerusalén. Su madre
se llamaba Azubá, hija de Silí. \footnote{\textbf{20:31} 1Re 22,41-50}
\bibleverse{32} Siguió el camino de su padre Asá y no se apartó de él,
haciendo lo que era justo a los ojos del Señor. \bibleverse{33} Sin
embargo, los lugares altos no fueron quitados, y el pueblo aún no había
puesto su corazón en el Dios de sus padres.

\bibleverse{34} El resto de los hechos de Josafat, los primeros y los
últimos, están escritos en la historia de Jehú, hijo de Hanani, que está
incluida en el libro de los reyes de Israel.

\hypertarget{la-alianza-de-josafat-con-ocozuxedas-de-israel-y-su-castigo-su-muerte}{%
\subsection{La alianza de Josafat con Ocozías de Israel y su castigo; su
muerte}\label{la-alianza-de-josafat-con-ocozuxedas-de-israel-y-su-castigo-su-muerte}}

\bibleverse{35} Después de esto, Josafat, rey de Judá, se unió a
Ocozías, rey de Israel. Éste hizo muy mal. \footnote{\textbf{20:35} 1Re
  22,51-53} \bibleverse{36} Se unió a él para hacer barcos para ir a
Tarsis. Hicieron las naves en Ezión Geber. \bibleverse{37} Entonces
Eliezer, hijo de Dodavahu, de Mareshah, profetizó contra Josafat,
diciendo: ``Por haberte unido a Ocozías, el Señor ha destruido tus
obras.'' Los barcos naufragaron, de modo que no pudieron ir a Tarsis.

\hypertarget{el-gobierno-del-rey-joram}{%
\subsection{El gobierno del rey Joram}\label{el-gobierno-del-rey-joram}}

\hypertarget{section-20}{%
\section{21}\label{section-20}}

\bibleverse{1} Josafat durmió con sus padres, y fue enterrado con ellos
en la ciudad de David; y su hijo Joram reinó en su lugar.

\hypertarget{asesinato-de-sus-hermanos}{%
\subsection{Asesinato de sus hermanos}\label{asesinato-de-sus-hermanos}}

\bibleverse{2} Tuvo hermanos, los hijos de Josafat: Azarías, Jehiel,
Zacarías, Azarías, Miguel y Sefatías. Todos ellos eran hijos de Josafat,
rey de Israel. \bibleverse{3} Su padre les dio grandes regalos de plata,
de oro y de cosas preciosas, con ciudades fortificadas en Judá; pero le
dio el reino a Joram, porque era el primogénito. \bibleverse{4} Cuando
Joram se alzó sobre el reino de su padre, y se fortaleció, mató a espada
a todos sus hermanos, y también a algunos de los príncipes de Israel.

\hypertarget{la-posiciuxf3n-de-dios-sobre-la-apostasuxeda-de-joram}{%
\subsection{La posición de Dios sobre la apostasía de
Joram}\label{la-posiciuxf3n-de-dios-sobre-la-apostasuxeda-de-joram}}

\bibleverse{5} Joram tenía treinta y dos años cuando comenzó a reinar, y
reinó ocho años en Jerusalén. \bibleverse{6} Siguió el camino de los
reyes de Israel, al igual que la casa de Ajab, pues tuvo como esposa a
la hija de Ajab. Hizo lo que era malo a los ojos del Señor.
\bibleverse{7} Sin embargo, Yavé no quiso destruir la casa de David, a
causa de la alianza que había hecho con él, y porque había prometido
darle siempre una lámpara a él y a sus hijos. \footnote{\textbf{21:7}
  2Sam 7,12; 1Re 11,36; Sal 132,17}

\hypertarget{apostasuxeda-de-los-edomitas-y-la-ciudad-de-libna}{%
\subsection{Apostasía de los edomitas y la ciudad de
Libna}\label{apostasuxeda-de-los-edomitas-y-la-ciudad-de-libna}}

\bibleverse{8} En sus días Edom se rebeló de la mano de Judá y se hizo
un rey sobre ellos. \bibleverse{9} Entonces Joram fue allí con sus
capitanes y todos sus carros con él. Se levantó de noche e hirió a los
edomitas que lo rodeaban, junto con los capitanes de los carros.
\bibleverse{10} Así se rebeló Edom bajo la mano de Judá hasta el día de
hoy. También Libna se rebeló al mismo tiempo de debajo de su mano,
porque había abandonado a Yavé, el Dios de sus padres.

\hypertarget{la-carta-amenazante-del-profeta-eluxedas-a-joram}{%
\subsection{La carta amenazante del profeta Elías a
Joram}\label{la-carta-amenazante-del-profeta-eluxedas-a-joram}}

\bibleverse{11} Además, hizo lugares altos en los montes de Judá, e hizo
que los habitantes de Jerusalén se prostituyeran, y desvió a Judá.
\bibleverse{12} Le llegó una carta del profeta Elías que decía: ``Yahvé,
el Dios de David, tu padre, dice: `Porque no has seguido los caminos de
Josafat, tu padre, ni los de Asá, rey de Judá, \bibleverse{13} sino que
has seguido el camino de los reyes de Israel y has hecho que Judá y los
habitantes de Jerusalén se prostituyan como lo hizo la casa de Ajab, y
también has matado a tus hermanos de la casa de tu padre, que eran
mejores que tú, \bibleverse{14} he aquí que el Señor golpeará a tu
pueblo con una gran plaga, incluyendo a tus hijos, a tus esposas y a
todas tus posesiones; \bibleverse{15} y tendréis una gran enfermedad de
las entrañas, hasta que se os caigan las entrañas a causa de la
enfermedad, día tras día.'\,''

\hypertarget{incursiones-filisteas-y-uxe1rabes}{%
\subsection{Incursiones filisteas y
árabes}\label{incursiones-filisteas-y-uxe1rabes}}

\bibleverse{16} Yahvé despertó contra Joram el espíritu de los filisteos
y de los árabes que están junto a los etíopes; \bibleverse{17} y
subieron contra Judá, la asaltaron y se llevaron todos los bienes que se
encontraban en la casa del rey, incluidos sus hijos y sus mujeres, de
modo que no le quedó ningún hijo, excepto Joacaz, el menor de sus hijos.

\hypertarget{el-agonizante-final-y-el-entierro-deshonroso-de-joram}{%
\subsection{El agonizante final y el entierro deshonroso de
Joram}\label{el-agonizante-final-y-el-entierro-deshonroso-de-joram}}

\bibleverse{18} Después de todo esto, Yahvé lo hirió en sus entrañas con
una enfermedad incurable. \bibleverse{19} Con el tiempo, al cabo de dos
años, se le cayeron los intestinos a causa de su enfermedad, y murió de
graves enfermedades. Su pueblo no le hizo ninguna quema, como la de sus
padres. \footnote{\textbf{21:19} 2Cró 16,14} \bibleverse{20} Tenía
treinta y dos años cuando comenzó a reinar, y reinó en Jerusalén ocho
años. Partió sin que nadie lo lamentara. Lo enterraron en la ciudad de
David, pero no en las tumbas de los reyes. \footnote{\textbf{21:20} 2Cró
  21,5; 2Cró 24,25}

\hypertarget{el-gobierno-del-rey-ochuxf4zuxedas-su-gobierno-desaprobando-a-dios}{%
\subsection{El gobierno del rey Ochôzías; Su gobierno desaprobando a
Dios}\label{el-gobierno-del-rey-ochuxf4zuxedas-su-gobierno-desaprobando-a-dios}}

\hypertarget{section-21}{%
\section{22}\label{section-21}}

\bibleverse{1} Los habitantes de Jerusalén nombraron rey a Ocozías, su
hijo menor, en su lugar, porque la banda de hombres que vino con los
árabes al campamento había matado a todos los mayores. Así reinó
Ocozías, hijo de Joram, rey de Judá. \footnote{\textbf{22:1} 2Re 8,25-29}
\bibleverse{2} Ocozías tenía cuarenta y dos años cuando comenzó a
reinar, y reinó un año en Jerusalén. Su madre se llamaba Atalía, hija de
Omri. \bibleverse{3} También él anduvo en los caminos de la casa de
Acab, porque su madre fue su consejera para actuar con maldad.
\bibleverse{4} Hizo lo que era malo a los ojos de Yavé, al igual que la
casa de Acab, pues ellos fueron sus consejeros después de la muerte de
su padre, para su destrucción.

\hypertarget{su-pacto-con-joram-de-israel-y-su-muerte-por-jehuxfa}{%
\subsection{Su pacto con Joram de Israel y su muerte por
Jehú}\label{su-pacto-con-joram-de-israel-y-su-muerte-por-jehuxfa}}

\bibleverse{5} También siguió su consejo y fue con Joram, hijo de Ajab,
rey de Israel, a la guerra contra Hazael, rey de Siria, en Ramot de
Galaad; y los sirios hirieron a Joram. \bibleverse{6} Volvió para
curarse en Jezreel de las heridas que le habían hecho en Ramá, cuando
luchó contra Hazael, rey de Siria. Azarías hijo de Joram, rey de Judá,
bajó a ver a Joram hijo de Ajab en Jezreel, porque estaba enfermo.

\bibleverse{7} La destrucción de Ocozías fue obra de Dios, ya que se
dirigió a Joram; pues cuando éste llegó, salió con Joram contra Jehú,
hijo de Nimsí, a quien Yahvé había ungido para que destruyera la casa de
Acab. \footnote{\textbf{22:7} 1Re 19,16; 2Re 9,6} \bibleverse{8} Cuando
Jehú ejecutaba el juicio sobre la casa de Ajab, encontró a los príncipes
de Judá y a los hijos de los hermanos de Ocozías sirviendo a Ocozías, y
los mató. \footnote{\textbf{22:8} 2Re 10,12-14} \bibleverse{9} Buscó a
Ocozías, y lo capturaron (ahora estaba escondido en Samaria), lo
llevaron a Jehú y lo mataron; y lo enterraron, porque dijeron: ``Es el
hijo de Josafat, que buscó a Yavé con todo su corazón.'' La casa de
Ocozías no tenía poder para mantener el reino. \footnote{\textbf{22:9}
  2Re 9,27-29}

\hypertarget{el-robo-y-el-asesinato-de-ataluxeda-rescate-de-jouxe1s}{%
\subsection{El robo y el asesinato de Atalía; Rescate de
Joás}\label{el-robo-y-el-asesinato-de-ataluxeda-rescate-de-jouxe1s}}

\bibleverse{10} Cuando Atalía, madre de Ocozías, vio que su hijo había
muerto, se levantó y destruyó toda la descendencia real de la casa de
Judá. \bibleverse{11} Pero Josabet, hija del rey, tomó a Joás, hijo de
Ocozías, y lo rescató sigilosamente de entre los hijos del rey que
habían sido asesinados, y lo puso a él y a su nodriza en la alcoba.
Entonces Josabet, hija del rey Joram, esposa del sacerdote Joiada (pues
era hermana de Ocozías), lo escondió de Atalía, para que no lo matara.
\bibleverse{12} Estuvo con ellos escondido en la casa de Dios seis años,
mientras Atalía reinaba sobre el país.

\hypertarget{la-conspiraciuxf3n-de-joiada}{%
\subsection{La conspiración de
Joiada}\label{la-conspiraciuxf3n-de-joiada}}

\hypertarget{section-22}{%
\section{23}\label{section-22}}

\bibleverse{1} En el séptimo año, Joiada se fortaleció y tomó en alianza
con él a los jefes de centenas: Azarías hijo de Jeroham, Ismael hijo de
Johanán, Azarías hijo de Obed, Maasías hijo de Adaías y Elisafat hijo de
Zicri. \bibleverse{2} Ellos recorrieron Judá y reunieron a los levitas
de todas las ciudades de Judá y a los jefes de familia de Israel, y
llegaron a Jerusalén. \bibleverse{3} Toda la asamblea hizo un pacto con
el rey en la casa de Dios. Joiada\footnote{\textbf{23:3} hebreo He} les
dijo: ``He aquí que el hijo del rey debe reinar, como Yahvé ha dicho
respecto a los hijos de David. \bibleverse{4} Esto es lo que debéis
hacer: una tercera parte de vosotros, los que entran en sábado, de los
sacerdotes y de los levitas, serán guardianes de los umbrales.
\bibleverse{5} Una tercera parte estará en la casa del rey, y otra
tercera parte en la puerta de la fundación. Todo el pueblo estará en los
atrios de la casa de Yavé. \bibleverse{6} Pero que nadie entre en la
casa de Yavé, sino los sacerdotes y los que ejercen el ministerio de los
levitas. Ellos entrarán, porque son santos, pero todo el pueblo seguirá
las instrucciones de Yavé. \bibleverse{7} Los levitas rodearán al rey,
cada uno con sus armas en la mano. El que entre en la casa, que lo
maten. Acompañen al rey cuando entre y cuando salga''.

\hypertarget{captura-y-asesinato-de-athalja-elevaciuxf3n-de-jouxe1s-a-rey}{%
\subsection{Captura y asesinato de Athalja; Elevación de Joás a
rey}\label{captura-y-asesinato-de-athalja-elevaciuxf3n-de-jouxe1s-a-rey}}

\bibleverse{8} Los levitas y todo Judá hicieron, pues, todo lo que mandó
el sacerdote Joiada. Cada uno tomó a sus hombres, los que debían entrar
en sábado, con los que debían salir en sábado, pues el sacerdote Joiada
no despidió el turno. \bibleverse{9} El sacerdote Joiada entregó a los
capitanes de centenas las lanzas, las rodelas y los escudos que habían
sido del rey David y que estaban en la casa de Dios. \bibleverse{10}
Puso a todo el pueblo, cada uno con su arma en la mano, desde el lado
derecho de la casa hasta el lado izquierdo, cerca del altar y de la
casa, alrededor del rey. \bibleverse{11} Entonces sacaron al hijo del
rey, le pusieron la corona, le dieron la alianza y lo hicieron rey.
Joiada y sus hijos lo ungieron, y dijeron: ``¡Viva el rey!''.

\bibleverse{12} Cuando Atalía oyó el ruido del pueblo que corría y
alababa al rey, entró con la gente en la casa de Yahvé. \bibleverse{13}
Entonces ella miró, y he aquí que el rey estaba de pie junto a su
columna a la entrada, con los capitanes y los trompetistas junto al rey.
Todo el pueblo del país se alegró y tocó las trompetas. Los cantores
también tocaban instrumentos musicales y dirigían los cantos de
alabanza. Entonces Atalía se rasgó las vestiduras y dijo: ``¡Traición!
¡Traición!''

\bibleverse{14} El sacerdote Joiada sacó a los capitanes de centenas que
estaban al frente del ejército y les dijo: ``Sacadla entre las filas, y
el que la siga, que lo maten a espada.'' Porque el sacerdote dijo: ``No
la maten en la casa de Yavé''. \bibleverse{15} Así que le abrieron paso.
Ella se dirigió a la entrada de la puerta de los caballos a la casa del
rey; y allí la mataron.

\hypertarget{medidas-de-joiada-para-la-gloria-de-dios-coronaciuxf3n-de-jouxe1s}{%
\subsection{Medidas de Joiada para la gloria de Dios; Coronación de
Joás}\label{medidas-de-joiada-para-la-gloria-de-dios-coronaciuxf3n-de-jouxe1s}}

\bibleverse{16} Joiada hizo un pacto entre él, todo el pueblo y el rey,
para que fueran pueblo de Yavé. \footnote{\textbf{23:16} 2Cró 15,12}
\bibleverse{17} Todo el pueblo fue a la casa de Baal, la derribó, rompió
sus altares y sus imágenes en pedazos, y mató a Matán, el sacerdote de
Baal, ante los altares. \bibleverse{18} Joiada designó a los
funcionarios de la casa de Yavé bajo la mano de los sacerdotes levitas,
que David había distribuido en la casa de Yavé, para que ofrecieran los
holocaustos de Yavé, como está escrito en la ley de Moisés, con alegría
y con cantos, tal como lo había ordenado David. \footnote{\textbf{23:18}
  2Cró 29,30} \bibleverse{19} Puso a los porteros en las puertas de la
casa de Yavé, para que no entrara nadie impuro en nada. \bibleverse{20}
Tomó a los jefes de centenas, a los nobles, a los gobernantes del pueblo
y a toda la gente del país, e hizo bajar al rey de la casa de Yavé.
Entraron por la puerta superior a la casa del rey, y pusieron al rey en
el trono del reino. \bibleverse{21} Entonces todo el pueblo del país se
alegró, y la ciudad se tranquilizó. Habían matado a Atalía con la
espada.

\hypertarget{el-gobierno-del-rey-jouxe1s}{%
\subsection{El gobierno del rey
Joás}\label{el-gobierno-del-rey-jouxe1s}}

\hypertarget{section-23}{%
\section{24}\label{section-23}}

\bibleverse{1} Joás tenía siete años cuando comenzó a reinar, y reinó
cuarenta años en Jerusalén. Su madre se llamaba Sibías, de Beerseba.
\footnote{\textbf{24:1} 2Re 11,21} \bibleverse{2} Joás hizo lo que era
justo a los ojos de Yavé durante todos los días del sacerdote Joiada.
\bibleverse{3} Joiada tomó para él dos esposas, y fue padre de hijos e
hijas.

\hypertarget{reparando-el-templo-ordenanza-sobre-la-administraciuxf3n-y-el-uso-del-dinero-entrante-para-el-templo}{%
\subsection{Reparando el templo; Ordenanza sobre la administración y el
uso del dinero entrante para el
templo}\label{reparando-el-templo-ordenanza-sobre-la-administraciuxf3n-y-el-uso-del-dinero-entrante-para-el-templo}}

\bibleverse{4} Después de esto, Joás se propuso restaurar la casa de
Yavé. \bibleverse{5} Reunió a los sacerdotes y a los levitas y les dijo:
``Salgan a las ciudades de Judá y reúnan dinero para reparar la casa de
su Dios de todo Israel de año en año. Procurad agilizar este asunto''.
Sin embargo, los levitas no lo hicieron de inmediato. \bibleverse{6} El
rey llamó al jefe Joiada y le dijo: ``¿Por qué no has exigido a los
levitas que traigan el impuesto de Moisés, siervo de Yavé, y de la
asamblea de Israel, de Judá y de Jerusalén, para la Tienda del
Testimonio?'' \footnote{\textbf{24:6} Éxod 30,12-13} \bibleverse{7}
Porque los hijos de Atalía, esa mujer impía, habían destrozado la Casa
de Dios, y también entregaron a los baales todas las cosas consagradas
de la Casa de Yavé. \footnote{\textbf{24:7} 2Cró 22,3-4}

\bibleverse{8} El rey ordenó, pues, que hicieran un cofre y lo pusieran
fuera, a la puerta de la casa de Yavé. \bibleverse{9} Hicieron un pregón
por Judá y Jerusalén, para que trajeran para Yavé el impuesto que
Moisés, siervo de Dios, impuso a Israel en el desierto. \footnote{\textbf{24:9}
  2Cró 24,6} \bibleverse{10} Todos los príncipes y todo el pueblo se
alegraron, y trajeron y echaron en el cofre hasta llenarlo.
\bibleverse{11} Cuando el cofre era llevado a los oficiales del rey por
mano de los levitas, y al ver que había mucho dinero, el escriba del rey
y el oficial del sumo sacerdote venían y vaciaban el cofre, lo tomaban y
lo llevaban de nuevo a su lugar. Así hacían día tras día, y recogían
dinero en abundancia. \bibleverse{12} El rey y Joiada lo dieron a los
que hacían el trabajo del servicio de la casa de Yavé. Contrataron
albañiles y carpinteros para restaurar la casa de Yavé, y también a los
que trabajaban el hierro y el bronce para reparar la casa de Yavé.
\bibleverse{13} Así trabajaron los obreros, y la obra de reparación
avanzó en sus manos. Arreglaron la casa de Dios tal como estaba
diseñada, y la reforzaron. \bibleverse{14} Cuando terminaron, trajeron
el resto del dinero ante el rey y Joiada, con el cual se hicieron los
utensilios para la casa de Yavé, los utensilios con los que se
ministraba y se ofrecía, incluyendo cucharas y recipientes de oro y
plata. Ofrecieron holocaustos en la casa de Yahvé continuamente durante
todos los días de Joiada.

\hypertarget{el-alejamiento-de-jouxe1s-de-dios-despuuxe9s-de-la-muerte-de-joiada-el-discurso-de-zachuxe2ruxedas-y-su-lapidaciuxf3n}{%
\subsection{El alejamiento de Joás de Dios después de la muerte de
Joiada; El discurso de Zachârías y su
lapidación}\label{el-alejamiento-de-jouxe1s-de-dios-despuuxe9s-de-la-muerte-de-joiada-el-discurso-de-zachuxe2ruxedas-y-su-lapidaciuxf3n}}

\bibleverse{15} Pero Joiada envejeció y se llenó de días, y murió. Tenía
ciento treinta años cuando murió. \bibleverse{16} Lo enterraron en la
ciudad de David, entre los reyes, porque había hecho el bien en Israel,
y hacia Dios y su casa.

\bibleverse{17} Después de la muerte de Joiada, los príncipes de Judá
vinieron y se inclinaron ante el rey. Entonces el rey los escuchó.
\bibleverse{18} Abandonaron la casa de Yavé, el Dios de sus padres, y
sirvieron a los postes de Asera y a los ídolos, por lo que la ira cayó
sobre Judá y Jerusalén por esta su culpabilidad. \bibleverse{19} Sin
embargo, les envió profetas para que volvieran a Yahvé, y ellos dieron
testimonio contra ellos; pero no quisieron escuchar.

\bibleverse{20} El Espíritu de Dios vino sobre Zacarías, hijo del
sacerdote Joiada, y se puso de pie sobre el pueblo y les dijo: ``Dios
dice: `¿Por qué desobedecéis los mandamientos de Yahvé, para que no
podáis prosperar? Porque habéis abandonado a Yahvé, él también os ha
abandonado'\,''.

\bibleverse{21} Conspiraron contra él y lo apedrearon por orden del rey
en el patio de la casa de Yahvé. \footnote{\textbf{24:21} Mat 23,35; Heb
  11,37} \bibleverse{22} Así el rey Joás no se acordó de la bondad que
le había hecho su padre Joiada, sino que mató a su hijo. Cuando murió,
dijo: ``Que Yahvé lo mire y lo pague''.

\hypertarget{guerra-desafortunada-con-los-sirios-asesinato-del-rey-por-conspiradores-palabra-final}{%
\subsection{Guerra desafortunada con los sirios; Asesinato del rey por
conspiradores; Palabra
final}\label{guerra-desafortunada-con-los-sirios-asesinato-del-rey-por-conspiradores-palabra-final}}

\bibleverse{23} Al final del año, el ejército de los sirios subió contra
él. Llegaron a Judá y a Jerusalén, y destruyeron a todos los príncipes
del pueblo de entre el pueblo, y enviaron todo su botín al rey de
Damasco. \bibleverse{24} Porque el ejército de los sirios vino con una
pequeña compañía de hombres, y Yahvé entregó en sus manos un ejército
muy grande, porque habían abandonado a Yahvé, el Dios de sus padres. Así
ejecutaron el juicio contra Joás.

\bibleverse{25} Cuando se alejaron de él (pues lo dejaron gravemente
herido), sus propios servidores conspiraron contra él por la sangre de
los hijos del sacerdote Joiada, y lo mataron en su lecho, y murió. Lo
enterraron en la ciudad de David, pero no lo enterraron en las tumbas de
los reyes. \footnote{\textbf{24:25} 2Cró 21,20} \bibleverse{26} Estos
son los que conspiraron contra él Zabad, hijo de Simeat, la amonita, y
Jozabad, hijo de Simrit, la moabita. \bibleverse{27} En cuanto a sus
hijos, la grandeza de las cargas que le fueron impuestas y la
reconstrucción de la casa de Dios, he aquí que están escritas en el
comentario del libro de los reyes. Su hijo Amasías reinó en su lugar.

\hypertarget{el-gobierno-del-rey-amasuxedas-buen-comienzo-para-el-gobierno}{%
\subsection{El gobierno del rey Amasías; Buen comienzo para el
gobierno}\label{el-gobierno-del-rey-amasuxedas-buen-comienzo-para-el-gobierno}}

\hypertarget{section-24}{%
\section{25}\label{section-24}}

\bibleverse{1} Amasías tenía veinticinco años cuando comenzó a reinar, y
reinó veintinueve años en Jerusalén. Su madre se llamaba Joadán, de
Jerusalén. \bibleverse{2} Hizo lo que era justo a los ojos de Yavé, pero
no con un corazón perfecto. \bibleverse{3} Cuando se le estableció el
reino, mató a sus siervos que habían matado a su padre el rey.
\footnote{\textbf{25:3} 2Cró 24,25} \bibleverse{4} Pero no dio muerte a
sus hijos, sino que hizo lo que está escrito en la ley en el libro de
Moisés, como lo ordenó Yavé, diciendo: ``Los padres no morirán por los
hijos, ni los hijos morirán por los padres, sino que cada uno morirá por
su propio pecado.'' \footnote{\textbf{25:4} Deut 24,16}

\hypertarget{la-victoria-de-amasuxedas-sobre-los-edomitas-despuuxe9s-de-que-los-mercenarios-israelitas-fueran-devueltos-la-venganza-de-estas-tropas}{%
\subsection{La victoria de Amasías sobre los edomitas después de que los
mercenarios israelitas fueran devueltos; la venganza de estas
tropas}\label{la-victoria-de-amasuxedas-sobre-los-edomitas-despuuxe9s-de-que-los-mercenarios-israelitas-fueran-devueltos-la-venganza-de-estas-tropas}}

\bibleverse{5} Además, Amasías reunió a Judá y los ordenó según las
casas de sus padres, bajo capitanes de millares y de centenas, todo Judá
y Benjamín. Los contó de veinte años en adelante, y encontró que había
trescientos mil hombres escogidos, capaces de salir a la guerra, que
podían manejar la lanza y el escudo. \bibleverse{6} También contrató a
cien mil hombres valientes de Israel por cien talentos\footnote{\textbf{25:6}
  1 cor es lo mismo que un homer, es decir, unos 55,9 galones americanos
  (líquidos) o 211 litros o 6 bushels. 10.000 cors de trigo pesan unas
  1.640 toneladas métricas.} de plata. \bibleverse{7} Un hombre de Dios
se acercó a él y le dijo: ``Oh rey, no dejes que el ejército de Israel
vaya contigo, porque Yahvé no está con Israel, con todos los hijos de
Efraín. \bibleverse{8} Pero si vas a ir, ponte en acción y sé fuerte
para la batalla. Dios te derribará ante el enemigo; porque Dios tiene
poder para ayudar y para derribar.''

\bibleverse{9} Amasías dijo al hombre de Dios: ``¿Pero qué haremos con
los cien talentos que he dado al ejército de Israel?'' El hombre de Dios
respondió: ``Yahvé es capaz de darte mucho más que esto''.

\bibleverse{10} Entonces Amasías los separó, al ejército que había
venido a él desde Efraín, para que volvieran a casa. Por lo tanto, su
ira se encendió en gran medida contra Judá, y volvieron a su casa con
una ira feroz.

\bibleverse{11} Amasías se armó de valor y condujo a su pueblo hasta el
Valle de la Sal, e hirió a diez mil de los hijos de Seír.
\bibleverse{12} Los hijos de Judá se llevaron vivos a diez mil, los
llevaron a la cima de la roca y los arrojaron desde la cima de la roca,
de modo que todos quedaron destrozados. \bibleverse{13} Pero los hombres
del ejército que Amasías envió de regreso, para que no fueran con él a
la batalla, cayeron sobre las ciudades de Judá desde Samaria hasta Bet
Horón, e hirieron a tres mil de ellas, y tomaron mucho botín.

\hypertarget{amasuxedas-se-aparta-de-dios-advertencia-de-un-profeta}{%
\subsection{Amasías se aparta de Dios; Advertencia de un
profeta}\label{amasuxedas-se-aparta-de-dios-advertencia-de-un-profeta}}

\bibleverse{14} Cuando Amasías regresó de la matanza de los edomitas,
trajo a los dioses de los hijos de Seír y los erigió en sus dioses, y se
inclinó ante ellos y les quemó incienso. \bibleverse{15} Por eso la ira
de Yavé ardió contra Amasías, y le envió un profeta que le dijo: ``¿Por
qué has buscado los dioses de los pueblos, que no han librado a su
propio pueblo de tus manos?''

\bibleverse{16} Mientras hablaba con él, el rey le dijo: ``¿Te hemos
hecho uno de los consejeros del rey? Detente. ¿Por qué has de ser
abatido?'' Entonces el profeta se detuvo y dijo: ``Sé que Dios ha
determinado destruirte, porque has hecho esto y no has escuchado mi
consejo''.

\hypertarget{la-desafortunada-guerra-de-amasuxedas-con-jouxe1s-de-israel}{%
\subsection{La desafortunada guerra de Amasías con Joás de
Israel}\label{la-desafortunada-guerra-de-amasuxedas-con-jouxe1s-de-israel}}

\bibleverse{17} Entonces Amasías, rey de Judá, consultó a sus consejeros
y envió a Joás, hijo de Joacaz, hijo de Jehú, rey de Israel, diciendo:
``¡Ven! Vamos a mirarnos a la cara''.

\bibleverse{18} Joás, rey de Israel, envió a decir a Amasías, rey de
Judá: ``El cardo que estaba en el Líbano envió a decir al cedro que
estaba en el Líbano: `Dale tu hija a mi hijo como esposa'. Entonces pasó
un animal salvaje que estaba en el Líbano y pisoteó el cardo.
\footnote{\textbf{25:18} Jue 9,14} \bibleverse{19} Te dices a ti mismo
que has golpeado a Edom, y tu corazón te eleva para presumir. Ahora
quédate en casa. ¿Por qué te has de meter en líos, para que caigas, tú y
Judá contigo?''

\bibleverse{20} Pero Amasías no quiso escuchar, porque era de Dios, para
entregarlos en manos de sus enemigos, porque habían buscado los dioses
de Edom. \bibleverse{21} Entonces subió Joás, rey de Israel, y él y
Amasías, rey de Judá, se miraron a la cara en Bet Semes, que pertenece a
Judá. \bibleverse{22} Judá fue derrotado por Israel, y cada uno huyó a
su tienda.

\bibleverse{23} Joás, rey de Israel, apresó a Amasías, rey de Judá, hijo
de Joás, hijo de Joacaz, en Bet Semes y lo llevó a Jerusalén, y derribó
el muro de Jerusalén desde la puerta de Efraín hasta la puerta de la
esquina, cuatrocientos codos. \bibleverse{24} Tomó todo el oro y la
plata, y todos los utensilios que se encontraban en la casa de Dios con
Obed-Edom, y los tesoros de la casa del rey, y los rehenes, y regresó a
Samaria.

\hypertarget{palabra-final-asesinato-del-rey-por-conspiradores}{%
\subsection{Palabra final; Asesinato del rey por
conspiradores}\label{palabra-final-asesinato-del-rey-por-conspiradores}}

\bibleverse{25} Amasías hijo de Joás, rey de Judá, vivió quince años
después de la muerte de Joás, hijo de Joacaz, rey de Israel.
\bibleverse{26} Los demás hechos de Amasías, los primeros y los últimos,
¿no están escritos en el libro de los reyes de Judá e Israel?
\bibleverse{27} Desde el momento en que Amasías se apartó de seguir a
Yavé, hicieron una conspiración contra él en Jerusalén. Él huyó a
Laquis, pero enviaron tras él a Laquis y lo mataron allí. \footnote{\textbf{25:27}
  2Cró 24,25} \bibleverse{28} Lo llevaron a caballo y lo enterraron con
sus padres en la Ciudad de Judá.

\hypertarget{el-gobierno-del-rey-ussia-buen-comienzo-para-el-gobierno-la-felicidad-de-ussia-en-la-guerra-y-la-paz}{%
\subsection{El gobierno del rey Ussia; Buen comienzo para el gobierno;
La felicidad de Ussia en la guerra y la
paz}\label{el-gobierno-del-rey-ussia-buen-comienzo-para-el-gobierno-la-felicidad-de-ussia-en-la-guerra-y-la-paz}}

\hypertarget{section-25}{%
\section{26}\label{section-25}}

\bibleverse{1} Todo el pueblo de Judá tomó a Uzías, que tenía dieciséis
años, y lo nombró rey en lugar de su padre Amasías. \footnote{\textbf{26:1}
  2Re 14,21-22; 2Re 15,1-3} \bibleverse{2} Él edificó Elot y la restauró
para Judá. Después el rey durmió con sus padres. \bibleverse{3} Uzías
tenía dieciséis años cuando comenzó a reinar, y reinó cincuenta y dos
años en Jerusalén. Su madre se llamaba Jecilia, de Jerusalén.
\bibleverse{4} Hizo lo que era justo a los ojos de Yavé, conforme a todo
lo que había hecho su padre Amasías. \footnote{\textbf{26:4} 2Cró 25,2}
\bibleverse{5} Se puso a buscar a Dios en los días de Zacarías, quien
tenía entendimiento en la visión de Dios; y mientras buscó a Yavé, Dios
lo hizo prosperar.

\bibleverse{6} Salió y luchó contra los filisteos, y derribó el muro de
Gat, el muro de Jabne y el muro de Asdod; y edificó ciudades en el país
de Asdod y entre los filisteos. \bibleverse{7} Dios lo ayudó contra los
filisteos y contra los árabes que vivían en Gur Baal y los meunitas.
\bibleverse{8} Los amonitas dieron tributo a Uzías. Su nombre se
extendió hasta la entrada de Egipto, pues se hizo muy fuerte.
\bibleverse{9} Además, Uzías construyó torres en Jerusalén, en la puerta
de la esquina, en la puerta del valle y en la curva de la muralla, y las
fortificó. \bibleverse{10} Construyó torres en el desierto y cavó muchas
cisternas, porque tenía mucho ganado, tanto en las tierras bajas como en
las llanuras. Tuvo labradores y viñadores en las montañas y en los
campos fructíferos, pues amaba la agricultura.

\hypertarget{la-preocupaciuxf3n-de-ussia-por-un-ejuxe9rcito-capaz-y-por-la-seguridad-del-pauxeds}{%
\subsection{La preocupación de Ussia por un ejército capaz y por la
seguridad del
país}\label{la-preocupaciuxf3n-de-ussia-por-un-ejuxe9rcito-capaz-y-por-la-seguridad-del-pauxeds}}

\bibleverse{11} Además, Uzías tenía un ejército de combatientes que
salían a la guerra por bandas, según el número de su cuenta hecho por el
escriba Jeiel y el oficial Maasías, bajo la mano de Hananías, uno de los
capitanes del rey. \bibleverse{12} El número total de los jefes de
familia, de los hombres valientes, era de dos mil seiscientos.
\bibleverse{13} Bajo su mano había un ejército de trescientos siete mil
quinientos, que hacían la guerra con gran poder, para ayudar al rey
contra el enemigo. \bibleverse{14} Uzías preparó para ellos, para todo
el ejército, escudos, lanzas, cascos, cotas de malla, arcos y piedras
para la honda. \bibleverse{15} En Jerusalén hizo dispositivos,
inventados por hombres hábiles, para que estuvieran en las torres y en
las almenas, con los que se pudieran lanzar flechas y grandes piedras.
Su nombre se extendió por todo el mundo, porque fue ayudado
maravillosamente hasta que se hizo fuerte.

\hypertarget{la-invasiuxf3n-de-ussia-al-sacerdocio-es-castigada-por-dios-con-lepra}{%
\subsection{La invasión de Ussia al sacerdocio es castigada por Dios con
lepra}\label{la-invasiuxf3n-de-ussia-al-sacerdocio-es-castigada-por-dios-con-lepra}}

\bibleverse{16} Pero cuando se fortaleció, su corazón se enalteció, de
modo que actuó de manera corrupta y cometió una infracción contra Yavé,
su Dios, pues entró en el templo de Yavé para quemar incienso en el
altar del incienso. \footnote{\textbf{26:16} 2Cró 25,19} \bibleverse{17}
El sacerdote Azarías entró tras él, y con él ochenta sacerdotes de Yavé,
que eran hombres valientes. \bibleverse{18} Ellos se resistieron al rey
Uzías y le dijeron: ``No te corresponde a ti, Uzías, quemar incienso a
Yavé, sino a los sacerdotes hijos de Aarón, que están consagrados a
quemar incienso. Sal del santuario, porque has cometido una infracción.
No será para tu honor de parte de Yahvé Dios''. \footnote{\textbf{26:18}
  Núm 18,7}

\bibleverse{19} Entonces Uzías se enojó. Tenía un incensario en la mano
para quemar incienso, y mientras estaba enojado con los sacerdotes, le
brotó la lepra en la frente ante los sacerdotes en la casa de Yahvé,
junto al altar del incienso. \bibleverse{20} El sumo sacerdote Azarías y
todos los sacerdotes lo miraron, y he aquí que tenía lepra en la frente;
y lo echaron rápidamente de allí. De hecho, él mismo se apresuró a
salir, porque el Señor lo había golpeado. \bibleverse{21} El rey Uzías
fue leproso hasta el día de su muerte, y vivía en una casa separada,
siendo leproso, pues fue cortado de la casa de Yavé. Su hijo Jotam
estaba al frente de la casa real, juzgando al pueblo del país.
\footnote{\textbf{26:21} 2Re 15,5-7; Núm 5,2}

\hypertarget{muerte-y-entierro-de-ussia}{%
\subsection{Muerte y entierro de
Ussia}\label{muerte-y-entierro-de-ussia}}

\bibleverse{22} El resto de los hechos de Uzías, los primeros y los
últimos, los escribió el profeta Isaías, hijo de Amoz. \footnote{\textbf{26:22}
  2Re 15,5-7; Is 1,1; Is 6,1} \bibleverse{23} Así que Uzías durmió con
sus padres; y lo enterraron con sus padres en el campo de enterramiento
que pertenecía a los reyes, porque decían: ``Es un leproso''. Su hijo
Jotam reinó en su lugar.

\hypertarget{el-gobierno-del-rey-jotam-gobierno-bueno-y-feliz-edificios-y-guerras-exitosas}{%
\subsection{El gobierno del rey Jotam; Gobierno bueno y feliz; Edificios
y guerras
exitosas}\label{el-gobierno-del-rey-jotam-gobierno-bueno-y-feliz-edificios-y-guerras-exitosas}}

\hypertarget{section-26}{%
\section{27}\label{section-26}}

\bibleverse{1} Jotam tenía veinticinco años cuando comenzó a reinar, y
reinó dieciséis años en Jerusalén. Su madre se llamaba Jerusá, hija de
Sadoc. \bibleverse{2} Hizo lo que era justo a los ojos de Yavé, según
todo lo que había hecho su padre Uzías. Sin embargo, no entró en el
templo de Yavé. El pueblo seguía actuando de manera corrupta.
\footnote{\textbf{27:2} 2Cró 26,16} \bibleverse{3} Construyó la puerta
superior de la casa de Yavé, y edificó mucho en el muro de Ofel.
\bibleverse{4} Además, construyó ciudades en la región montañosa de
Judá, y en los bosques edificó fortalezas y torres. \footnote{\textbf{27:4}
  2Cró 26,10} \bibleverse{5} También luchó con el rey de los hijos de
Amón y los venció. Los hijos de Amón le dieron ese mismo año cien
talentos de plata, diez mil cors de trigo y diez mil cors de
cebada.\footnote{\textbf{27:5} 10.000 cors de cebada pesan alrededor de
  1.310 toneladas métricas.} Los hijos de Amón también le dieron esa
cantidad el segundo año y el tercero. \bibleverse{6} Así, Jotam se hizo
poderoso, porque ordenó sus caminos ante Yahvé, su Dios. \bibleverse{7}
El resto de los hechos de Jotam, todas sus guerras y sus caminos, están
escritos en el libro de los reyes de Israel y de Judá. \bibleverse{8}
Tenía veinticinco años cuando comenzó a reinar, y reinó dieciséis años
en Jerusalén. \bibleverse{9} Jotam durmió con sus padres, y lo
enterraron en la ciudad de David; y su hijo Acaz reinó en su lugar.

\hypertarget{el-reinado-del-rey-acaz-las-abominaciones-paganas-de-acaz}{%
\subsection{El reinado del rey Acaz; Las abominaciones paganas de
Acaz}\label{el-reinado-del-rey-acaz-las-abominaciones-paganas-de-acaz}}

\hypertarget{section-27}{%
\section{28}\label{section-27}}

\bibleverse{1} Acaz tenía veinte años cuando comenzó a reinar, y reinó
dieciséis años en Jerusalén. No hizo lo que era justo a los ojos de
Yavé, como su padre David, \footnote{\textbf{28:1} 2Re 16,2-5}
\bibleverse{2} sino que siguió los caminos de los reyes de Israel, y
también hizo imágenes fundidas para los baales. \bibleverse{3} Además,
quemó incienso en el valle del hijo de Hinom, y quemó a sus hijos en el
fuego, según las abominaciones de las naciones que Yahvé expulsó delante
de los hijos de Israel. \footnote{\textbf{28:3} Deut 18,9-10; Deut 18,12}
\bibleverse{4} Sacrificó y quemó incienso en los lugares altos, en las
colinas y debajo de todo árbol verde. \footnote{\textbf{28:4} 1Re 14,23}

\hypertarget{visitaciones-severas-de-sirios-e-israelitas}{%
\subsection{Visitaciones severas de sirios e
israelitas}\label{visitaciones-severas-de-sirios-e-israelitas}}

\bibleverse{5} Por eso el Señor, su Dios, lo entregó en manos del rey de
Siria. Lo hirieron y le arrebataron una gran cantidad de cautivos, y los
llevaron a Damasco. También fue entregado en manos del rey de Israel,
quien lo golpeó con una gran matanza. \bibleverse{6} Porque Peka, hijo
de Remalías, mató en Judá a ciento veinte mil personas en un solo día,
todos ellos hombres valientes, porque habían abandonado a Yavé, el Dios
de sus padres. \bibleverse{7} Zicri, hombre poderoso de Efraín, mató a
Maasías, hijo del rey, a Azricam, jefe de la casa, y a Elcana, que
estaba junto al rey. \bibleverse{8} Los hijos de Israel llevaron
cautivos de sus hermanos a doscientos mil mujeres, hijos e hijas, y
también les quitaron mucho botín, y llevaron el botín a Samaria.

\hypertarget{liberaciuxf3n-de-los-prisioneros-de-guerra-de-judea-en-samaria-siguiendo-la-amonestaciuxf3n-del-profeta-oded}{%
\subsection{Liberación de los prisioneros de guerra de Judea en Samaria
siguiendo la amonestación del profeta
Oded}\label{liberaciuxf3n-de-los-prisioneros-de-guerra-de-judea-en-samaria-siguiendo-la-amonestaciuxf3n-del-profeta-oded}}

\bibleverse{9} Pero estaba allí un profeta de Yavé, que se llamaba Oded,
y salió al encuentro del ejército que había llegado a Samaria, y les
dijo: ``Miren, porque Yavé, el Dios de sus padres, se enojó con Judá,
los ha entregado en sus manos, y ustedes los han matado con una furia
que ha llegado hasta el cielo. \footnote{\textbf{28:9} Gén 18,21; Esd
  9,6} \bibleverse{10} Ahora pretendéis degradar a los hijos de Judá y
de Jerusalén como esclavos y esclavas para vosotros. ¿Acaso no hay en
vosotros delitos propios contra el Señor, vuestro Dios? \bibleverse{11}
Ahora, pues, escúchenme y devuelvan a los cautivos que han tomado de sus
hermanos, porque la feroz ira de Yavé está sobre ustedes.''
\bibleverse{12} Entonces algunos de los jefes de los hijos de Efraín,
Azarías hijo de Johanán, Berequías hijo de Meshillemot, Jehizquías hijo
de Salum y Amasa hijo de Hadlai, se levantaron contra los que venían de
la guerra, \bibleverse{13} y les dijeron: ``No traigan aquí a los
cautivos, porque ustedes pretenden lo que traerá sobre nosotros una
transgresión contra Yavé, para añadir a nuestros pecados y a nuestra
culpa; pues nuestra culpa es grande, y hay una ira feroz contra
Israel.''

\bibleverse{14} Entonces los hombres armados dejaron a los cautivos y el
botín ante los príncipes y toda la asamblea. \bibleverse{15} Los hombres
mencionados por su nombre se levantaron y tomaron a los cautivos, y con
el botín vistieron a todos los que estaban desnudos entre ellos, los
vistieron, les dieron sandalias, les dieron de comer y de beber, los
ungieron, cargaron a todos los débiles en asnos y los llevaron a Jericó,
la ciudad de las palmeras, con sus hermanos. Luego volvieron a Samaria.
\footnote{\textbf{28:15} Prov 25,21-22}

\hypertarget{fuertes-visitaciones-de-los-edomitas-filisteos-y-asirios}{%
\subsection{Fuertes visitaciones de los edomitas, filisteos y
asirios}\label{fuertes-visitaciones-de-los-edomitas-filisteos-y-asirios}}

\bibleverse{16} En aquel tiempo el rey Acaz envió a los reyes de Asiria
para que lo ayudaran. \bibleverse{17} Porque de nuevo los edomitas
habían llegado y atacado a Judá, y se habían llevado cautivos.
\bibleverse{18} Los filisteos también habían invadido las ciudades de la
llanura y del sur de Judá, y habían tomado Bet Semes, Ajalón, Gederot,
Soco con sus aldeas, Timná con sus aldeas, y también Gimzo y sus aldeas;
y vivían allí. \bibleverse{19} Porque Yahvé rebajó a Judá por culpa de
Ajaz, rey de Israel, porque actuó sin freno en Judá y cometió graves
infracciones contra Yahvé. \bibleverse{20} Tilgat-pilneser, rey de
Asiria, vino a él y le dio problemas, pero no lo fortaleció.
\bibleverse{21} Porque Acaz tomó una parte de la casa de Yavé, de la
casa del rey y de los príncipes, y se la dio al rey de Asiria, pero no
lo ayudó.

\hypertarget{la-creciente-maldad-de-acaz-palabra-final}{%
\subsection{La creciente maldad de Acaz; Palabra
final}\label{la-creciente-maldad-de-acaz-palabra-final}}

\bibleverse{22} En el tiempo de su angustia, este mismo rey Acaz se
rebeló aún más contra el Señor. \bibleverse{23} Porque sacrificó a los
dioses de Damasco que lo habían derrotado. Dijo: ``Porque los dioses de
los reyes de Siria los ayudaron, les sacrificaré para que me ayuden''.
Pero fueron la ruina de él y de todo Israel. \bibleverse{24} Acaz reunió
los utensilios de la casa de Dios, cortó en pedazos los utensilios de la
casa de Dios y cerró las puertas de la casa de Yavé, y se hizo altares
en todos los rincones de Jerusalén. \bibleverse{25} En todas las
ciudades de Judá hizo lugares altos para quemar incienso a otros dioses,
y provocó la ira de Yavé, el Dios de sus padres.

\bibleverse{26} El resto de sus actos y todos sus caminos, primeros y
últimos, están escritos en el libro de los reyes de Judá e Israel.
\bibleverse{27} Acaz durmió con sus padres, y lo enterraron en la
ciudad, en Jerusalén, porque no lo llevaron a los sepulcros de los reyes
de Israel; y su hijo Ezequías reinó en su lugar. \footnote{\textbf{28:27}
  2Cró 21,20}

\hypertarget{el-gobierno-del-rey-ezechuxeeas-restauraciuxf3n-del-templo-y-adoraciuxf3n-pura}{%
\subsection{El gobierno del rey Ezechîas; Restauración del templo y
adoración
pura}\label{el-gobierno-del-rey-ezechuxeeas-restauraciuxf3n-del-templo-y-adoraciuxf3n-pura}}

\hypertarget{section-28}{%
\section{29}\label{section-28}}

\bibleverse{1} Ezequías comenzó a reinar cuando tenía veinticinco años,
y reinó veintinueve años en Jerusalén. Su madre se llamaba Abías, hija
de Zacarías. \footnote{\textbf{29:1} 2Re 18,1-3} \bibleverse{2} Hizo lo
que era justo a los ojos de Yavé, conforme a todo lo que había hecho su
padre David.

\hypertarget{la-exhortaciuxf3n-de-ezechuxeeas-a-los-sacerdotes-y-levitas}{%
\subsection{La exhortación de Ezechîas a los sacerdotes y
levitas}\label{la-exhortaciuxf3n-de-ezechuxeeas-a-los-sacerdotes-y-levitas}}

\bibleverse{3} En el primer año de su reinado, en el primer mes, abrió
las puertas de la casa de Yavé y las reparó. \bibleverse{4} Hizo venir a
los sacerdotes y a los levitas y los reunió en el amplio lugar del este,
\bibleverse{5} y les dijo: ``¡Escúchenme, levitas! Ahora santifíquense y
santifiquen la casa de Yavé, el Dios de sus padres, y saquen la
inmundicia del lugar santo. \bibleverse{6} Porque nuestros padres fueron
infieles, e hicieron lo que era malo a los ojos de Yavé, nuestro Dios, y
lo abandonaron, y apartaron sus rostros de la morada de Yavé, y le
dieron la espalda. \bibleverse{7} También han cerrado las puertas del
pórtico y han apagado las lámparas, y no han quemado incienso ni
ofrecido holocaustos en el lugar santo al Dios de Israel. \footnote{\textbf{29:7}
  2Cró 28,24} \bibleverse{8} Por eso la ira de Yahvé ha caído sobre Judá
y Jerusalén, y los ha entregado para que sean zarandeados de un lado a
otro, para que sean un asombro y un silbido, como lo ves con tus ojos.
\bibleverse{9} Porque he aquí que nuestros padres han caído a espada, y
nuestros hijos, nuestras hijas y nuestras esposas están en cautiverio
por esto. \footnote{\textbf{29:9} 2Cró 28,5-8} \bibleverse{10} Ahora
está en mi corazón hacer un pacto con Yavé, el Dios de Israel, para que
su feroz ira se aparte de nosotros. \bibleverse{11} Hijos míos, no os
descuidéis ahora, porque Yahvé os ha elegido para que estéis delante de
él, para que le sirváis, y para que seáis sus ministros y queméis
incienso.''

\hypertarget{purificaciuxf3n-del-templo-por-los-levitas}{%
\subsection{Purificación del templo por los
levitas}\label{purificaciuxf3n-del-templo-por-los-levitas}}

\bibleverse{12} Entonces se levantaron los levitas: Mahat, hijo de
Amasai, y Joel, hijo de Azarías, de los hijos de los coatitas; y de los
hijos de Merari, Cis, hijo de Abdi, y Azarías, hijo de Jehallelel; y de
los gersonitas, Joah, hijo de Zimma, y Edén, hijo de Joah
\bibleverse{13} y de los hijos de Elizafán, Simri y Jeuel; y de los
hijos de Asaf, Zacarías y Matanías; \bibleverse{14} y de los hijos de
Hemán, Jehuel y Simei; y de los hijos de Jedutún, Semaías y Uziel.
\bibleverse{15} Reunieron a sus hermanos, se santificaron y entraron,
según el mandato del rey por palabras de Yavé, a limpiar la casa de
Yavé. \bibleverse{16} Los sacerdotes entraron en el interior de la casa
de Yavé para limpiarla, y sacaron toda la impureza que encontraron en el
templo de Yavé al atrio de la casa de Yavé. Los levitas la sacaron de
allí para llevarla al arroyo Cedrón. \bibleverse{17} El primer día del
primer mes comenzaron a santificar, y el octavo día del mes llegaron al
pórtico de Yavé. Santificaron la casa de Yavé en ocho días, y el día
dieciséis del primer mes terminaron. \bibleverse{18} Luego entraron al
rey Ezequías dentro del palacio y le dijeron: ``Hemos limpiado toda la
casa de Yavé, incluyendo el altar del holocausto con todos sus
utensilios, y la mesa del pan de la feria con todos sus utensilios.
\bibleverse{19} Además, hemos preparado y santificado todos los
utensilios que el rey Ajaz tiró en su reinado cuando fue infiel. He aquí
que están ante el altar de Yahvé''.

\hypertarget{la-nueva-consagraciuxf3n-del-templo-con-sacrificios-oraciuxf3n-y-cuxe1nticos}{%
\subsection{La nueva consagración del templo con sacrificios, oración y
cánticos}\label{la-nueva-consagraciuxf3n-del-templo-con-sacrificios-oraciuxf3n-y-cuxe1nticos}}

\bibleverse{20} Entonces el rey Ezequías se levantó temprano, reunió a
los príncipes de la ciudad y subió a la casa de Yahvé. \bibleverse{21}
Trajeron siete toros, siete carneros, siete corderos y siete machos
cabríos, como ofrenda por el pecado por el reino, por el santuario y por
Judá. Ordenó a los sacerdotes hijos de Aarón que los ofrecieran sobre el
altar de Yavé. \bibleverse{22} Mataron los toros, los sacerdotes
recibieron la sangre y la rociaron sobre el altar. Mataron los carneros
y rociaron la sangre sobre el altar. También mataron los corderos y
rociaron la sangre sobre el altar. \bibleverse{23} Acercaban los machos
cabríos para la ofrenda por el pecado ante el rey y la asamblea, y les
imponían las manos. \bibleverse{24} Luego los sacerdotes los mataron, e
hicieron una ofrenda por el pecado con su sangre sobre el altar, para
hacer expiación por todo Israel, pues el rey ordenó que el holocausto y
la ofrenda por el pecado se hicieran por todo Israel.

\bibleverse{25} Puso a los levitas en la casa de Yavé con címbalos, con
instrumentos de cuerda y con arpas, según el mandato de David, de Gad,
el vidente del rey, y del profeta Natán; porque el mandato era de Yavé
por medio de sus profetas. \footnote{\textbf{29:25} 1Cró 25,1}
\bibleverse{26} Los levitas estaban con los instrumentos de David, y los
sacerdotes con las trompetas. \bibleverse{27} Ezequías les ordenó que
ofrecieran el holocausto sobre el altar. Cuando comenzó el holocausto,
también comenzó el canto de Yahvé, junto con las trompetas y los
instrumentos de David, rey de Israel. \bibleverse{28} Toda la asamblea
adoraba, los cantores cantaban y los trompeteros tocaban. Todo esto
continuó hasta que se terminó el holocausto.

\bibleverse{29} Cuando terminaron de ofrecer, el rey y todos los que
estaban presentes con él se inclinaron y adoraron. \bibleverse{30}
Además, el rey Ezequías y los príncipes ordenaron a los levitas que
cantaran alabanzas a Yavé con las palabras de David y del vidente Asaf.
Cantaron alabanzas con alegría, e inclinaron la cabeza y adoraron.
\footnote{\textbf{29:30} 2Cró 23,18}

\bibleverse{31} Entonces Ezequías respondió: ``Ahora ustedes se han
consagrado a Yavé. Acérquense y traigan sacrificios y ofrendas de
agradecimiento a la casa de Yavé''. La asamblea trajo sacrificios y
ofrendas de agradecimiento, y todos los que tenían un corazón dispuesto
trajeron holocaustos. \bibleverse{32} El número de los holocaustos que
trajo la asamblea fue de setenta toros, cien carneros y doscientos
corderos. Todo esto era para el holocausto a Yahvé. \bibleverse{33} Las
cosas consagradas eran seiscientas cabezas de ganado y tres mil ovejas.
\bibleverse{34} Pero los sacerdotes eran muy pocos, de modo que no
podían desollar todos los holocaustos. Por lo tanto, sus hermanos los
levitas les ayudaron hasta que se terminó la obra y hasta que los
sacerdotes se santificaron, pues los levitas eran más rectos de corazón
para santificarse que los sacerdotes. \footnote{\textbf{29:34} 2Cró
  30,3; 2Cró 30,16-17} \bibleverse{35} Además, los holocaustos eran
abundantes, con la grasa de las ofrendas de paz y con las libaciones de
cada holocausto. Así se puso en orden el servicio de la casa de Yahvé.
\footnote{\textbf{29:35} Lev 3,16-17; Núm 15,5; Núm 15,7; Núm 15,10}
\bibleverse{36} Ezequías y todo el pueblo se alegraron por lo que Dios
había preparado para el pueblo, pues la cosa se hizo de repente.

\hypertarget{celebraciuxf3n-de-la-pascua-de-ezechuxeeas}{%
\subsection{Celebración de la Pascua de
Ezechîas}\label{celebraciuxf3n-de-la-pascua-de-ezechuxeeas}}

\hypertarget{section-29}{%
\section{30}\label{section-29}}

\bibleverse{1} Ezequías envió a todo Israel y a Judá, y escribió también
cartas a Efraín y a Manasés, para que vinieran a la casa de Yavé en
Jerusalén, a celebrar la Pascua a Yavé, el Dios de Israel. \footnote{\textbf{30:1}
  2Cró 35,1} \bibleverse{2} Porque el rey había aconsejado a sus
príncipes y a toda la asamblea de Jerusalén que celebraran la Pascua en
el segundo mes. \footnote{\textbf{30:2} 2Cró 30,15} \bibleverse{3} Pues
no podían celebrarla en ese momento, porque los sacerdotes no se habían
santificado en número suficiente, y el pueblo no se había reunido en
Jerusalén. \bibleverse{4} La cosa era justa a los ojos del rey y de toda
la asamblea. \bibleverse{5} Así que establecieron un decreto para hacer
la proclamación en todo Israel, desde Beerseba hasta Dan, de que debían
venir a celebrar la Pascua a Yavé, el Dios de Israel, en Jerusalén,
porque no la habían celebrado en gran número en la forma en que está
escrito.

\bibleverse{6} Así que los mensajeros fueron con las cartas del rey y de
sus príncipes por todo Israel y Judá, según el mandato del rey,
diciendo: ``Ustedes, hijos de Israel, vuélvanse a Yavé, el Dios de
Abraham, de Isaac y de Israel, para que él regrese al remanente de
ustedes que ha escapado de la mano de los reyes de Asiria.
\bibleverse{7} No seáis como vuestros padres y como vuestros hermanos,
que prevaricaron contra Yavé, el Dios de sus padres, de modo que él los
entregó a la desolación, como veis. \bibleverse{8} Ahora no seáis de
cuello duro, como vuestros padres, sino someteos a Yavé, y entrad en su
santuario, que él ha santificado para siempre, y servid a Yavé vuestro
Dios, para que se aparte de vosotros su furia. \bibleverse{9} Porque si
os volvéis a Yahvé, vuestros hermanos y vuestros hijos encontrarán
compasión con los que los llevaron cautivos, y volverán a esta tierra,
porque Yahvé, vuestro Dios, es clemente y misericordioso, y no apartará
su rostro de vosotros si os volvéis a él.''

\bibleverse{10} Así que los mensajeros pasaron de ciudad en ciudad por
el país de Efraín y Manasés, hasta llegar a Zabulón, pero la gente los
ridiculizaba y se burlaba de ellos. \bibleverse{11} Sin embargo, algunos
hombres de Aser, Manasés y Zabulón se humillaron y llegaron a Jerusalén.
\bibleverse{12} También la mano de Dios vino sobre Judá para darles un
solo corazón, para cumplir el mandato del rey y de los príncipes por
palabra de Yavé.

\hypertarget{curso-de-pascua-en-la-primera-semana}{%
\subsection{Curso de Pascua en la primera
semana}\label{curso-de-pascua-en-la-primera-semana}}

\bibleverse{13} Mucha gente se reunió en Jerusalén para celebrar la
fiesta de los panes sin levadura en el segundo mes, una asamblea muy
grande. \bibleverse{14} Se levantaron y quitaron los altares que había
en Jerusalén, y se llevaron todos los altares para el incienso y los
arrojaron al arroyo Cedrón. \bibleverse{15} Luego sacrificaron la Pascua
el día catorce del segundo mes. Los sacerdotes y los levitas se
avergonzaron, se santificaron y trajeron holocaustos a la casa de Yahvé.
\footnote{\textbf{30:15} Núm 9,11} \bibleverse{16} Se colocaron en su
lugar, según su orden, de acuerdo con la ley de Moisés, el hombre de
Dios. Los sacerdotes rociaban la sangre que recibían de la mano de los
levitas. \footnote{\textbf{30:16} 2Cró 29,34} \bibleverse{17} Porque
había muchos en la asamblea que no se habían santificado; por eso los
levitas estaban encargados de matar las pascuas de todos los que no
estaban limpios, para santificarlos a Yahvé. \bibleverse{18} Porque una
multitud del pueblo, incluso muchos de Efraín, Manasés, Isacar y
Zabulón, no se habían purificado, y sin embargo comían la Pascua de
manera distinta a como está escrito. Pues Ezequías había orado por
ellos, diciendo: ``Que el buen Yahvé perdone a todos \footnote{\textbf{30:18}
  Éxod 12,1} \bibleverse{19} que pongan su corazón a buscar a Dios,
Yahvé, el Dios de sus padres, aunque no estén limpios según la
purificación del santuario.''

\bibleverse{20} El Señor escuchó a Ezequías y sanó al pueblo.
\bibleverse{21} Los hijos de Israel que estaban en Jerusalén celebraron
la fiesta de los panes sin levadura durante siete días con gran alegría.
Los levitas y los sacerdotes alababan a Yavé todos los días, cantando
con instrumentos fuertes a Yavé. \bibleverse{22} Ezequías hablaba con
ánimo a todos los levitas que tenían buen entendimiento en el servicio
de Yavé. Así comieron durante los siete días de la fiesta, ofreciendo
sacrificios de ofrendas de paz y confesando a Yavé, el Dios de sus
padres. \footnote{\textbf{30:22} 2Cró 32,6}

\hypertarget{continuaciuxf3n-de-la-celebraciuxf3n-en-la-segunda-semana}{%
\subsection{Continuación de la celebración en la segunda
semana}\label{continuaciuxf3n-de-la-celebraciuxf3n-en-la-segunda-semana}}

\bibleverse{23} Toda la asamblea tomó el consejo de celebrar otros siete
días, y celebraron otros siete días con alegría. \bibleverse{24} Porque
Ezequías, rey de Judá, dio a la asamblea como ofrendas mil toros y siete
mil ovejas, y los príncipes dieron a la asamblea mil toros y diez mil
ovejas; y un gran número de sacerdotes se santificó. \footnote{\textbf{30:24}
  2Cró 35,7} \bibleverse{25} Toda la asamblea de Judá, con los
sacerdotes y los levitas, y toda la asamblea que salió de Israel, y los
extranjeros que salieron del país de Israel y que vivían en Judá, se
alegraron. \bibleverse{26} Hubo, pues, gran alegría en Jerusalén; porque
desde los tiempos de Salomón hijo de David, rey de Israel, no había
habido nada semejante en Jerusalén. \bibleverse{27} Entonces los
sacerdotes levitas se levantaron y bendijeron al pueblo. Su voz fue
escuchada, y su oración subió hasta su santa morada, hasta el cielo.

\hypertarget{limpiando-la-tierra-de-la-idolatruxeda}{%
\subsection{Limpiando la tierra de la
idolatría}\label{limpiando-la-tierra-de-la-idolatruxeda}}

\hypertarget{section-30}{%
\section{31}\label{section-30}}

\bibleverse{1} Cuando todo esto terminó, todo Israel que estaba presente
salió a las ciudades de Judá y rompió las columnas, cortó los postes de
Asera y derribó los lugares altos y los altares de todo Judá y Benjamín,
también en Efraín y Manasés, hasta destruirlos todos. Entonces todos los
hijos de Israel volvieron, cada uno a su posesión, a sus propias
ciudades. \footnote{\textbf{31:1} Deut 7,5; 2Re 18,4}

\hypertarget{cuidado-exitoso-de-los-ingresos-de-los-sacerdotes-y-levitas}{%
\subsection{Cuidado exitoso de los ingresos de los sacerdotes y
levitas}\label{cuidado-exitoso-de-los-ingresos-de-los-sacerdotes-y-levitas}}

\bibleverse{2} Ezequías designó las divisiones de los sacerdotes y de
los levitas según sus divisiones, cada uno según su servicio, tanto los
sacerdotes como los levitas, para los holocaustos y las ofrendas de paz,
para ministrar, dar gracias y alabar en las puertas del campamento de
Yahvé. \bibleverse{3} También destinó la parte de los bienes del rey
para los holocaustos: para los holocaustos matutinos y vespertinos, y
para los holocaustos de los sábados, de las lunas nuevas y de las
fiestas señaladas, como está escrito en la ley de Yahvé. \footnote{\textbf{31:3}
  Núm 28,1; Núm 29,1-29} \bibleverse{4} Además, ordenó al pueblo que
vivía en Jerusalén que diera la parte de los sacerdotes y de los
levitas, para que se entregaran a la ley de Yavé. \bibleverse{5} Tan
pronto como salió el mandamiento, los hijos de Israel dieron en
abundancia las primicias del grano, del vino nuevo, del aceite, de la
miel y de todo el producto del campo; y trajeron el diezmo de todas las
cosas en abundancia. \footnote{\textbf{31:5} Éxod 23,19; Deut 14,22-23}
\bibleverse{6} Los hijos de Israel y de Judá, que vivían en las ciudades
de Judá, trajeron también el diezmo del ganado y de las ovejas, y el
diezmo de las cosas consagradas a Yahvé su Dios, y lo pusieron en
montones.

\bibleverse{7} En el tercer mes comenzaron a poner los cimientos de los
montones, y los terminaron en el séptimo mes. \bibleverse{8} Cuando
Ezequías y los príncipes llegaron y vieron los montones, bendijeron a
Yavé y a su pueblo Israel. \bibleverse{9} Luego Ezequías interrogó a los
sacerdotes y a los levitas acerca de los montones. \bibleverse{10}
Azarías, el jefe de los sacerdotes, de la casa de Sadoc, le respondió y
dijo: ``Desde que el pueblo comenzó a traer las ofrendas a la casa de
Yavé, hemos comido y nos hemos saciado, y nos ha sobrado, porque Yavé ha
bendecido a su pueblo; y lo que ha quedado es este gran montón.''

\bibleverse{11} Entonces Ezequías les ordenó que prepararan habitaciones
en la casa de Yahvé, y las prepararon. \bibleverse{12} Trajeron
fielmente las ofrendas, los diezmos y las cosas dedicadas. El levita
Conanías era el jefe de ellos, y su hermano Simei era el segundo.
\bibleverse{13} Jehiel, Azazías, Nahat, Asael, Jerimot, Jozabad, Eliel,
Ismaquías, Mahat y Benaía eran supervisores bajo la mano de Conanías y
de Simei, su hermano, por designación del rey Ezequías y de Azarías,
jefe de la casa de Dios. \bibleverse{14} Coré, hijo del levita Imna,
guardián de la puerta oriental, estaba a cargo de las ofrendas
voluntarias de Dios, para distribuir las ofrendas de Yahvé y las cosas
más sagradas. \bibleverse{15} Debajo de él estaban Edén, Miniamín,
Jesúa, Semaías, Amarías y Secanías, en las ciudades de los sacerdotes,
en su oficio de confianza, para dar a sus hermanos por divisiones, tanto
a los grandes como a los pequeños; \bibleverse{16} además de los que
estaban listados por genealogía de varones, de tres años en adelante,
todos los que entraban en la casa de Yavé, según el deber de cada día,
para su servicio en sus oficios según sus divisiones;

\hypertarget{elaboraciuxf3n-de-listas-de-sacerdotes-y-levitas-palabra-final}{%
\subsection{Elaboración de listas de sacerdotes y levitas; Palabra
final}\label{elaboraciuxf3n-de-listas-de-sacerdotes-y-levitas-palabra-final}}

\bibleverse{17} y los que estaban en la lista por genealogía de los
sacerdotes por sus casas paternas, y los levitas de veinte años en
adelante, en sus oficios por sus divisiones; \bibleverse{18} y los que
estaban en la lista por genealogía de todos sus pequeños, sus esposas,
sus hijos y sus hijas, por toda la congregación; porque en su oficio de
confianza se santificaban en santidad. \bibleverse{19} También para los
hijos de Aarón, los sacerdotes, que estaban en los campos de las tierras
de pastoreo de sus ciudades, en cada ciudad, había hombres mencionados
por su nombre para dar porciones a todos los varones entre los
sacerdotes y a todos los que estaban listados por genealogía entre los
levitas.

\bibleverse{20} Así lo hizo Ezequías en todo Judá; e hizo lo bueno, lo
justo y lo fiel ante Yahvé su Dios. \bibleverse{21} En toda obra que
comenzó en el servicio de la casa de Dios, en la ley y en los
mandamientos, para buscar a su Dios, lo hizo de todo corazón y prosperó.
\footnote{\textbf{31:21} Sal 1,3}

\hypertarget{la-incursiuxf3n-de-senaquerib-y-el-resto-de-ezechuxeeas}{%
\subsection{La incursión de Senaquerib y el resto de
Ezechîas}\label{la-incursiuxf3n-de-senaquerib-y-el-resto-de-ezechuxeeas}}

\hypertarget{section-31}{%
\section{32}\label{section-31}}

\bibleverse{1} Después de estas cosas y de esta fidelidad, vino
Senaquerib, rey de Asiria, entró en Judá, acampó contra las ciudades
fortificadas y pretendió ganarlas para sí. \footnote{\textbf{32:1} 2Cró
  31,20} \bibleverse{2} Cuando Ezequías vio que Senaquerib había llegado
y que planeaba luchar contra Jerusalén, \bibleverse{3} aconsejó a sus
príncipes y a sus valientes que detuvieran las aguas de los manantiales
que estaban fuera de la ciudad, y le ayudaron. \bibleverse{4} Entonces
se reunió mucha gente y detuvieron todos los manantiales y el arroyo que
fluía por el centro de la tierra, diciendo: ``¿Por qué han de venir los
reyes de Asiria y encontrar agua abundante?''

\bibleverse{5} Se armó de valor, reconstruyó toda la muralla derribada y
la levantó hasta las torres, con la otra muralla por fuera, y fortaleció
a Millo en la ciudad de David, e hizo armas y escudos en abundancia.
\footnote{\textbf{32:5} 2Cró 25,23} \bibleverse{6} Puso capitanes de
guerra al frente del pueblo, los reunió junto a él en el lugar amplio de
la puerta de la ciudad y les habló con ánimo, diciendo: \footnote{\textbf{32:6}
  2Cró 30,22} \bibleverse{7} ``Sed fuertes y valientes. No tengáis miedo
ni os acobardéis por el rey de Asiria, ni por toda la multitud que está
con él; porque hay uno mayor con nosotros que con él. \footnote{\textbf{32:7}
  2Re 6,16} \bibleverse{8} Un brazo de carne está con él, pero el Señor,
nuestro Dios, está con nosotros para ayudarnos y librar nuestras
batallas.'' El pueblo se apoyó en las palabras de Ezequías, rey de Judá.
\footnote{\textbf{32:8} Jer 17,5; Jer 17,7}

\hypertarget{la-solicitud-de-senaquerib-de-entregar-la-ciudad-a-lachis}{%
\subsection{La solicitud de Senaquerib de entregar la ciudad a
Lachis}\label{la-solicitud-de-senaquerib-de-entregar-la-ciudad-a-lachis}}

\bibleverse{9} Después de esto, Senaquerib, rey de Asiria, envió a sus
siervos a Jerusalén (ahora estaba atacando Laquis, y todas sus fuerzas
estaban con él), a Ezequías, rey de Judá, y a todo Judá que estaba en
Jerusalén, diciendo: \bibleverse{10} Senaquerib, rey de Asiria, dice:
``¿En quién confían ustedes, que permanecen sitiados en Jerusalén?
\bibleverse{11} ¿No os persuade Ezequías para entregaros a la muerte por
hambre y por sed, diciendo: `El Señor, nuestro Dios, nos librará de la
mano del rey de Asiria'? \bibleverse{12} ¿No ha quitado el mismo
Ezequías sus lugares altos y sus altares, y ha ordenado a Judá y a
Jerusalén, diciendo: `Adoraréis ante un solo altar, y quemaréis incienso
en él'? \bibleverse{13} ¿No sabéis lo que yo y mis padres hemos hecho a
todos los pueblos de las tierras? ¿Acaso los dioses de las naciones de
esas tierras fueron capaces de librar su tierra de mi mano?
\bibleverse{14} ¿Quién había entre todos los dioses de las naciones que
mis padres destruyeron que pudiera librar a su pueblo de mi mano, para
que vuestro Dios pudiera libraros de mi mano? \bibleverse{15} Ahora
bien, no dejes que Ezequías te engañe ni te persuada de esta manera. No
le creas, porque ningún dios de ninguna nación o reino ha podido librar
a su pueblo de mi mano, ni de la mano de mis padres. ¿Cuánto menos te
librará tu Dios de mi mano?''

\hypertarget{senaquerib-y-la-arrogancia-de-sus-embajadores}{%
\subsection{Senaquerib y la arrogancia de sus
embajadores}\label{senaquerib-y-la-arrogancia-de-sus-embajadores}}

\bibleverse{16} Sus servidores hablaron aún más contra el Dios Yahvé y
contra su siervo Ezequías. \bibleverse{17} También escribió cartas
insultando a Yavé, el Dios de Israel, y hablando contra él, diciendo:
``Como los dioses de las naciones de las tierras, que no han librado a
su pueblo de mi mano, así el Dios de Ezequías no librará a su pueblo de
mi mano.'' \bibleverse{18} Llamaron a viva voz, en lengua judía, a los
habitantes de Jerusalén que estaban en la muralla, para atemorizarlos y
molestarlos, a fin de tomar la ciudad. \bibleverse{19} Hablaron del Dios
de Jerusalén como de los dioses de los pueblos de la tierra, que son
obra de manos de hombres.

\hypertarget{oraciuxf3n-de-ezequuxedas-la-ayuda-de-dios-la-destrucciuxf3n-el-retiro-y-la-muerte-de-senaquerib}{%
\subsection{Oración de Ezequías; La ayuda de Dios: la destrucción, el
retiro y la muerte de
Senaquerib}\label{oraciuxf3n-de-ezequuxedas-la-ayuda-de-dios-la-destrucciuxf3n-el-retiro-y-la-muerte-de-senaquerib}}

\bibleverse{20} El rey Ezequías y el profeta Isaías, hijo de Amoz,
oraron a causa de esto y clamaron al cielo.

\bibleverse{21} El Señor envió a un ángel que eliminó a todos los
hombres valientes, a los jefes y a los capitanes del campamento del rey
de Asiria. Así que regresó con el rostro avergonzado a su propia tierra.
Cuando entró en la casa de su dios, los que salieron de su propio
cuerpo\footnote{\textbf{32:21} decir, sus propios hijos} lo mataron allí
a espada. \bibleverse{22} Así salvó Yahvé a Ezequías y a los habitantes
de Jerusalén de la mano de Senaquerib, rey de Asiria, y de la mano de
todos los demás, y los guió por todos lados. \bibleverse{23} Muchos
llevaron regalos a Yahvé en Jerusalén, y cosas preciosas a Ezequías, rey
de Judá, de modo que desde entonces fue exaltado a la vista de todas las
naciones.

\hypertarget{la-enfermedad-la-arrogancia-y-la-penitencia-de-ezechuxeeas}{%
\subsection{La enfermedad, la arrogancia y la penitencia de
Ezechîas}\label{la-enfermedad-la-arrogancia-y-la-penitencia-de-ezechuxeeas}}

\bibleverse{24} En aquellos días Ezequías tenía una enfermedad terminal,
y oró a Yavé; y éste le habló y le dio una señal. \bibleverse{25} Pero
Ezequías no correspondió adecuadamente al beneficio que se le hacía,
porque su corazón estaba enardecido. Por eso hubo ira sobre él, sobre
Judá y sobre Jerusalén. \footnote{\textbf{32:25} 2Cró 26,16}
\bibleverse{26} Sin embargo, Ezequías se humilló por la soberbia de su
corazón, tanto él como los habitantes de Jerusalén, de modo que la ira
del Señor no cayó sobre ellos en los días de Ezequías.

\hypertarget{la-riqueza-de-ezechuxeeas-abastecimiento-de-agua-a-jerusaluxe9n-y-tentaciuxf3n-de-la-embajada-de-babilonia}{%
\subsection{La riqueza de Ezechîas; Abastecimiento de agua a Jerusalén y
tentación de la embajada de
Babilonia}\label{la-riqueza-de-ezechuxeeas-abastecimiento-de-agua-a-jerusaluxe9n-y-tentaciuxf3n-de-la-embajada-de-babilonia}}

\bibleverse{27} Ezequías tenía grandes riquezas y honores. Se proveyó de
tesoros de plata, de oro, de piedras preciosas, de especias, de escudos
y de toda clase de objetos de valor; \bibleverse{28} también de
almacenes para el aumento del grano, del vino nuevo y del aceite; y de
establos para toda clase de animales, y de rebaños en rediles.
\bibleverse{29} Además, se proveyó de ciudades y de posesiones de
rebaños y manadas en abundancia, porque Dios le había dado abundantes
posesiones. \bibleverse{30} Este mismo Ezequías también detuvo el
manantial superior de las aguas de Gihón, y las hizo descender
directamente al lado occidental de la ciudad de David. Ezequías prosperó
en todas sus obras.

\bibleverse{31} Sin embargo, en cuanto a los embajadores de los
príncipes de Babilonia, que le enviaron a preguntar por la maravilla que
se hacía en el país, Dios lo dejó para que lo probara, a fin de conocer
todo lo que había en su corazón.

\hypertarget{termina-la-historia-de-ezechuxeeas}{%
\subsection{Termina la historia de
Ezechîas}\label{termina-la-historia-de-ezechuxeeas}}

\bibleverse{32} El resto de los hechos de Ezequías y sus buenas
acciones, he aquí que están escritos en la visión del profeta Isaías,
hijo de Amoz, en el libro de los reyes de Judá e Israel. \bibleverse{33}
Ezequías durmió con sus padres, y lo enterraron en la subida a las
tumbas de los hijos de David. Todo Judá y los habitantes de Jerusalén lo
honraron a su muerte. Su hijo Manasés reinó en su lugar. \footnote{\textbf{32:33}
  2Cró 16,14}

\hypertarget{manasuxe9s-rey-de-juduxe1-idolatruxeda-manasuxe9s}{%
\subsection{Manasés rey de Judá; Idolatría
manasés}\label{manasuxe9s-rey-de-juduxe1-idolatruxeda-manasuxe9s}}

\hypertarget{section-32}{%
\section{33}\label{section-32}}

\bibleverse{1} Manasés tenía doce años cuando comenzó a reinar, y reinó
cincuenta y cinco años en Jerusalén. \bibleverse{2} Hizo lo que era malo
a los ojos de Yavé, según las abominaciones de las naciones que Yavé
arrojó delante de los hijos de Israel. \footnote{\textbf{33:2} Deut 18,9}
\bibleverse{3} Porque volvió a edificar los lugares altos que Ezequías,
su padre, había derribado, y levantó altares para los baales, hizo a
Asherot, y adoró a todo el ejército del cielo, y les sirvió. \footnote{\textbf{33:3}
  2Re 18,4} \bibleverse{4} Edificó altares en la casa de Yavé, de la
cual dijo Yavé: ``Mi nombre estará en Jerusalén para siempre.''
\footnote{\textbf{33:4} Deut 12,5; Deut 12,11; 1Re 9,3} \bibleverse{5}
Construyó altares para todo el ejército del cielo en los dos atrios de
la casa de Yavé. \bibleverse{6} También hizo pasar a sus hijos por el
fuego en el valle del hijo de Hinom. Practicó la hechicería, la
adivinación y la brujería, y trató con los que tenían espíritus
familiares y con los magos. Hizo mucho mal a los ojos de Yahvé, para
provocarlo a la ira. \bibleverse{7} Puso la imagen grabada del ídolo que
había hecho en la casa de Dios, de la cual Dios dijo a David y a Salomón
su hijo: ``En esta casa y en Jerusalén, que he elegido de entre todas
las tribus de Israel, pondré mi nombre para siempre. \bibleverse{8} No
volveré a apartar el pie de Israel de la tierra que he destinado a
vuestros padres, con tal de que observen todas las cosas que les he
mandado, es decir, toda la ley, los estatutos y los reglamentos dados
por Moisés.'' \bibleverse{9} Manasés sedujo a Judá y a los habitantes de
Jerusalén, de modo que hicieron más mal que las naciones que Yahvé
destruyó antes de los hijos de Israel.

\bibleverse{10} El Señor habló a Manasés y a su pueblo, pero ellos no
escucharon.

\hypertarget{la-gira-del-prisionero-de-manasuxe9s-a-babilonia-su-arrepentimiento-y-regreso-a-casa}{%
\subsection{La gira del prisionero de Manasés a Babilonia, su
arrepentimiento y regreso a
casa}\label{la-gira-del-prisionero-de-manasuxe9s-a-babilonia-su-arrepentimiento-y-regreso-a-casa}}

\bibleverse{11} Por eso el Señor hizo venir a los capitanes del ejército
del rey de Asiria, quienes tomaron a Manasés encadenado, lo ataron con
grilletes y lo llevaron a Babilonia.

\bibleverse{12} Cuando se vio en apuros, suplicó a Yahvé, su Dios, y se
humilló mucho ante el Dios de sus padres. \bibleverse{13} Le oró, y él
se dejó rogar, escuchó su súplica y lo hizo volver a Jerusalén, a su
reino. Entonces Manasés supo que Yahvé era Dios. \footnote{\textbf{33:13}
  1Re 18,39}

\hypertarget{manasuxe9s-construyendo-muros-y-esfuerzos-para-eliminar-la-idolatruxeda}{%
\subsection{Manasés construyendo muros y esfuerzos para eliminar la
idolatría}\label{manasuxe9s-construyendo-muros-y-esfuerzos-para-eliminar-la-idolatruxeda}}

\bibleverse{14} Después de esto, edificó un muro exterior a la ciudad de
David en el lado occidental de Gihón, en el valle, hasta la entrada en
la puerta del pescado. Rodeó con ella a Ofel, y la levantó a gran
altura; y puso capitanes valientes en todas las ciudades fortificadas de
Judá. \bibleverse{15} Quitó los dioses extranjeros y el ídolo de la casa
de Yavé, y todos los altares que había construido en el monte de la casa
de Yavé y en Jerusalén, y los echó de la ciudad. \bibleverse{16} Edificó
el altar de Yavé y ofreció en él sacrificios de paz y de acción de
gracias, y ordenó a Judá que sirviera a Yavé, el Dios de Israel.
\bibleverse{17} Sin embargo, el pueblo seguía sacrificando en los
lugares altos, pero sólo a Yavé, su Dios.

\bibleverse{18} El resto de los hechos de Manasés, y su oración a su
Dios, y las palabras de los videntes que le hablaron en nombre de Yahvé,
el Dios de Israel, he aquí que están escritos entre los hechos de los
reyes de Israel. \footnote{\textbf{33:18} 2Re 21,17-18} \bibleverse{19}
Su oración también, y cómo Dios escuchó su petición, y todo su pecado y
su transgresión, y los lugares en los que construyó lugares altos y
levantó los postes de Asera y las imágenes grabadas antes de humillarse:
he aquí, están escritos en la historia de Hozai. \footnote{\textbf{33:19}
  o, los videntes}

\hypertarget{amuxf3n-rey-de-juduxe1}{%
\subsection{Amón Rey de Judá}\label{amuxf3n-rey-de-juduxe1}}

\bibleverse{20} Manasés, pues, durmió con sus padres y lo enterraron en
su propia casa; y su hijo Amón reinó en su lugar. \footnote{\textbf{33:20}
  2Re 21,17-18}

\bibleverse{21} Amón tenía veintidós años cuando comenzó a reinar, y
reinó dos años en Jerusalén. \bibleverse{22} Hizo lo que era malo a los
ojos de Yavé, como lo hizo su padre Manasés; y Amón sacrificó a todas
las imágenes grabadas que había hecho su padre Manasés, y las sirvió.
\bibleverse{23} No se humilló ante Yavé, como se había humillado su
padre Manasés, sino que este mismo Amón prevaricó más y más. \footnote{\textbf{33:23}
  2Cró 33,12} \bibleverse{24} Sus servidores conspiraron contra él y lo
mataron en su propia casa. \bibleverse{25} Pero el pueblo del país mató
a todos los que habían conspirado contra el rey Amón, y el pueblo del
país hizo rey a su hijo Josías en su lugar.

\hypertarget{el-gobierno-del-rey-josuxedas}{%
\subsection{El gobierno del rey
Josías}\label{el-gobierno-del-rey-josuxedas}}

\hypertarget{section-33}{%
\section{34}\label{section-33}}

\bibleverse{1} Josías tenía ocho años cuando comenzó a reinar, y reinó
treinta y un años en Jerusalén. \footnote{\textbf{34:1} 2Re 22,1-2}
\bibleverse{2} Hizo lo que era justo a los ojos de Yavé, y anduvo en los
caminos de su padre David, y no se apartó ni a la derecha ni a la
izquierda. \footnote{\textbf{34:2} 2Re 22,1-2; 2Cró 29,2}

\hypertarget{restauraciuxf3n-del-culto-puro}{%
\subsection{Restauración del culto
puro}\label{restauraciuxf3n-del-culto-puro}}

\bibleverse{3} Porque en el octavo año de su reinado, siendo aún joven,
comenzó a buscar al Dios de David su padre; y en el duodécimo año
comenzó a purificar a Judá y a Jerusalén de los lugares altos, de los
postes de Asera, de las imágenes grabadas y de las imágenes fundidas.
\footnote{\textbf{34:3} 2Re 23,4-20} \bibleverse{4} Derribaron los
altares de los baales en su presencia, y cortó los altares de incienso
que estaban en lo alto. Rompió los postes de Asera, las imágenes
grabadas y las imágenes fundidas en pedazos, hizo polvo con ellas y lo
esparció sobre las tumbas de los que les habían sacrificado. \footnote{\textbf{34:4}
  2Cró 14,5; Lev 26,30} \bibleverse{5} Quemó los huesos de los
sacerdotes en sus altares y purificó a Judá y Jerusalén. \footnote{\textbf{34:5}
  1Re 13,2} \bibleverse{6} Hizo esto en las ciudades de Manasés, Efraín
y Simeón, hasta Neftalí, alrededor de sus ruinas. \bibleverse{7} Derribó
los altares, redujo a polvo los postes de Asera y las imágenes grabadas,
y cortó todos los altares de incienso en toda la tierra de Israel, y
luego regresó a Jerusalén.

\hypertarget{explicar-los-procedimientos-que-se-siguen-para-restaurar-y-mantener-el-templo}{%
\subsection{Explicar los procedimientos que se siguen para restaurar y
mantener el
templo}\label{explicar-los-procedimientos-que-se-siguen-para-restaurar-y-mantener-el-templo}}

\bibleverse{8} En el año dieciocho de su reinado, después de haber
purificado la tierra y la casa, envió a Safán, hijo de Azalías, a
Maasías, gobernador de la ciudad, y a Joá, hijo de Joacaz, registrador,
a reparar la casa de Yavé, su Dios. \footnote{\textbf{34:8} 2Re 22,3-6}
\bibleverse{9} Vinieron a ver al sumo sacerdote Hilcías y le entregaron
el dinero que se había introducido en la casa de Dios, que los levitas,
guardianes del umbral, habían recogido de manos de Manasés, Efraín, de
todo el resto de Israel, de todo Judá y Benjamín, y de los habitantes de
Jerusalén. \bibleverse{10} Lo entregaron en manos de los obreros que
tenían a su cargo la casa de Yavé; y los obreros que trabajaban en la
casa de Yavé lo dieron para reparar y arreglar la casa. \bibleverse{11}
Se lo daban a los carpinteros y a los constructores para que compraran
piedra cortada y madera para los empalmes, y para que hicieran vigas
para las casas que los reyes de Judá habían destruido. \bibleverse{12}
Los hombres hicieron el trabajo fielmente. Sus capataces eran los
levitas Jahat y Abdías, de los hijos de Merari; y Zacarías y Mesulam, de
los hijos de Coat, para que dieran la dirección; y otros de los levitas,
que eran todos hábiles con los instrumentos musicales. \bibleverse{13}
También estaban a cargo de los portadores de cargas, y dirigían a todos
los que hacían el trabajo en toda clase de servicio. De los levitas,
había escribas, funcionarios y porteros.

\hypertarget{informe-sobre-el-descubrimiento-del-cuxf3digo-y-su-primer-efecto}{%
\subsection{Informe sobre el descubrimiento del código y su primer
efecto}\label{informe-sobre-el-descubrimiento-del-cuxf3digo-y-su-primer-efecto}}

\bibleverse{14} Cuando sacaron el dinero que se había llevado a la casa
de Yavé, el sacerdote Hilcías encontró el libro de la ley de Yavé dado
por Moisés. \bibleverse{15} Hilcías respondió al escriba Safán: ``He
encontrado el libro de la ley en la casa de Yahvé.'' Entonces Hilcías
entregó el libro a Safán.

\bibleverse{16} Safán llevó el libro al rey, y además trajo de vuelta la
noticia al rey, diciendo: ``Todo lo que fue encomendado a tus siervos,
ellos lo están haciendo. \bibleverse{17} Han vaciado el dinero que se
encontraba en la casa de Yavé, y lo han entregado en manos de los
capataces y en manos de los obreros.'' \bibleverse{18} El escriba Safán
informó al rey diciendo: ``El sacerdote Hilcías me ha entregado un
libro.'' Safán leyó de él al rey.

\bibleverse{19} Cuando el rey escuchó las palabras de la ley, se rasgó
las vestiduras. \bibleverse{20} El rey mandó a Hilcías, a Ajicam hijo de
Safán, a Abdón hijo de Micá, al escriba Safán y a Asaías, siervo del
rey, diciendo: \bibleverse{21} ``Vayan a consultar a Yavé por mí y por
los que quedan en Israel y en Judá, acerca de las palabras del libro que
se ha encontrado; porque es grande la ira de Yavé que se ha derramado
sobre nosotros, porque nuestros padres no han guardado la palabra de
Yavé, para hacer conforme a todo lo que está escrito en este libro.''

\hypertarget{interrogatorio-y-respuesta-de-la-profetisa-hulda}{%
\subsection{Interrogatorio y respuesta de la profetisa
Hulda}\label{interrogatorio-y-respuesta-de-la-profetisa-hulda}}

\bibleverse{22} Entonces Jilquías y los que el rey había mandado fueron
a ver a la profetisa Hulda, esposa de Salum, hijo de Tójat, hijo de
Hasra, guardián del armario (ella vivía en Jerusalén, en el segundo
barrio), y le hablaron en ese sentido.

\bibleverse{23} Ella les dijo: ``Yahvé, el Dios de Israel, dice:
`Díganle al hombre que los envió a mí: \bibleverse{24} ``Yahvé dice: `He
aquí que yo traigo el mal sobre este lugar y sobre sus habitantes, hasta
todas las maldiciones que están escritas en el libro que han leído ante
el rey de Judá. \footnote{\textbf{34:24} Lev 26,14-39; Deut 28,15-68}
\bibleverse{25} Porque me han abandonado y han quemado incienso a otros
dioses, para provocarme a la ira con todas las obras de sus manos, por
eso mi ira se ha derramado sobre este lugar, y no se apagará.'\,''
\bibleverse{26} Pero al rey de Judá, que te envió a consultar a Yavé, le
dirás lo siguiente: ``Yavé, el Dios de Israel, dice: ``Acerca de las
palabras que has oído, \bibleverse{27} porque tu corazón se enterneció y
te humillaste ante Dios cuando oíste sus palabras contra este lugar y
contra sus habitantes, y te humillaste ante mí, y te rasgaste las
vestiduras y lloraste ante mí, yo también te he oído'', dice Yavé.
\footnote{\textbf{34:27} 2Cró 33,12} \bibleverse{28} ``He aquí que yo te
reuniré con tus padres, y serás reunido a tu tumba en paz. Tus ojos no
verán todo el mal que traeré sobre este lugar y sobre sus
habitantes''\,''. Llevaron este mensaje al rey.

\hypertarget{josuxedas-concluye-el-nuevo-pacto-de-dios-en-asociaciuxf3n-con-los-ancianos-del-pueblo}{%
\subsection{Josías concluye el nuevo pacto de Dios en asociación con los
ancianos del
pueblo}\label{josuxedas-concluye-el-nuevo-pacto-de-dios-en-asociaciuxf3n-con-los-ancianos-del-pueblo}}

\bibleverse{29} Entonces el rey envió a reunir a todos los ancianos de
Judá y de Jerusalén. \footnote{\textbf{34:29} 2Re 23,1-3}
\bibleverse{30} El rey subió a la casa de Yavé con todos los hombres de
Judá y los habitantes de Jerusalén --- los sacerdotes, los levitas y
todo el pueblo, tanto los grandes como los pequeños --- y leyó en su
presencia todas las palabras del libro de la alianza que se encontraba
en la casa de Yavé. \bibleverse{31} El rey se puso de pie en su lugar e
hizo un pacto delante de Yavé, de caminar en pos de Yavé y de guardar
sus mandamientos, sus testimonios y sus estatutos con todo su corazón y
con toda su alma, para cumplir las palabras del pacto que estaban
escritas en este libro. \footnote{\textbf{34:31} 2Cró 15,12; Jos 24,25}
\bibleverse{32} Hizo que todos los que se encontraban en Jerusalén y en
Benjamín se pusieran de pie. Los habitantes de Jerusalén hicieron
conforme al pacto de Dios, el Dios de sus padres. \footnote{\textbf{34:32}
  2Re 23,3} \bibleverse{33} Josías quitó todas las abominaciones de
todos los países que pertenecían a los hijos de Israel, e hizo que todos
los que se encontraban en Israel sirvieran a su Dios. En todos sus días
no se apartaron de seguir a Yavé, el Dios de sus padres.

\hypertarget{la-estricta-celebraciuxf3n-de-la-pascua-de-josuxedas}{%
\subsection{La estricta celebración de la Pascua de
Josías}\label{la-estricta-celebraciuxf3n-de-la-pascua-de-josuxedas}}

\hypertarget{section-34}{%
\section{35}\label{section-34}}

\bibleverse{1} Josías celebró la Pascua a Yahvé en Jerusalén. Mataron la
Pascua el día catorce del primer mes. \bibleverse{2} Puso a los
sacerdotes en sus cargos y los animó en el servicio de la casa de Yavé.
\bibleverse{3} Dijo a los levitas que enseñaban a todo Israel, que eran
santos para Yavé: ``Poned el arca sagrada en la casa que construyó
Salomón, hijo de David, rey de Israel. Ya no será una carga para tus
hombros. Ahora sirvan a Yavé su Dios y a su pueblo Israel. \footnote{\textbf{35:3}
  1Re 6,1} \bibleverse{4} Preparaos según las casas de vuestros padres
por vuestras divisiones, según la escritura de David, rey de Israel, y
según la escritura de Salomón, su hijo. \bibleverse{5} Poneos en el
lugar santo según las divisiones de las casas paternas de vuestros
hermanos los hijos del pueblo, y que haya para cada uno una porción de
una casa paterna de los levitas. \bibleverse{6} Maten el cordero de la
Pascua, santifíquense y preparen a sus hermanos, para hacer lo que Yahvé
dijo por medio de Moisés.''

\bibleverse{7} Josías dio a los hijos del pueblo, del rebaño, corderos y
cabritos, todos ellos para las ofrendas de la Pascua, a todos los
presentes, en número de treinta mil, y tres mil toros. Estos eran de la
hacienda del rey. \footnote{\textbf{35:7} 2Cró 30,24} \bibleverse{8} Sus
príncipes dieron una ofrenda voluntaria al pueblo, a los sacerdotes y a
los levitas. Hilcías, Zacarías y Jehiel, jefes de la casa de Dios,
dieron a los sacerdotes para las ofrendas de la Pascua dos mil
seiscientos animales pequeños y trescientas cabezas de ganado.
\bibleverse{9} También Conanías, Semaías y Netanel, sus hermanos, y
Hasabías, Jeiel y Jozabad, jefes de los levitas, dieron a los levitas
para las ofrendas de la Pascua cinco mil animales pequeños y quinientas
cabezas de ganado.

\bibleverse{10} Se preparó, pues, el servicio, y los sacerdotes se
colocaron en su lugar, y los levitas en sus divisiones, según el mandato
del rey. \bibleverse{11} Mataron los corderos de la Pascua, y los
sacerdotes rociaron la sangre que recibieron de sus manos, y los levitas
los desollaron. \bibleverse{12} Sacaron los holocaustos para darlos
según las divisiones de las casas paternas de los hijos del pueblo, para
ofrecerlos a Yavé, como está escrito en el libro de Moisés. Lo mismo
hicieron con el ganado. \bibleverse{13} Asaron la Pascua al fuego según
la ordenanza. Hirvieron las ofrendas sagradas en ollas, en calderos y en
sartenes, y las llevaron rápidamente a todos los hijos del pueblo.
\bibleverse{14} Después prepararon para ellos y para los sacerdotes,
porque los sacerdotes hijos de Aarón estaban ocupados en ofrecer los
holocaustos y la grasa hasta la noche. Por lo tanto, los levitas se
prepararon para sí mismos y para los sacerdotes hijos de Aarón.
\bibleverse{15} Los cantores, hijos de Asaf, estaban en su lugar, según
el mandato de David, Asaf, Hemán y Jedutún, vidente del rey; y los
porteros estaban en cada puerta. No necesitaban apartarse de su
servicio, porque sus hermanos los levitas se preparaban para ellos.
\footnote{\textbf{35:15} 1Cró 25,1; 1Cró 26,1}

\bibleverse{16} Así que todo el servicio de Yahvé se preparó el mismo
día, para celebrar la Pascua y ofrecer holocaustos en el altar de Yahvé,
según el mandato del rey Josías. \bibleverse{17} Los hijos de Israel que
estaban presentes celebraron la Pascua en ese momento, y la fiesta de
los panes sin levadura durante siete días. \bibleverse{18} No hubo una
Pascua como la que se celebró en Israel desde los días del profeta
Samuel, ni ninguno de los reyes de Israel celebró una Pascua como la que
celebró Josías, con los sacerdotes, los levitas y todo Judá e Israel que
estaban presentes, y los habitantes de Jerusalén. \footnote{\textbf{35:18}
  2Cró 30,26} \bibleverse{19} Esta Pascua se celebró en el año dieciocho
del reinado de Josías.

\hypertarget{necao-de-egipto-y-la-muerte-de-josuxedas-dolor-por-el}{%
\subsection{Necao de Egipto y la muerte de Josías; Dolor por
el}\label{necao-de-egipto-y-la-muerte-de-josuxedas-dolor-por-el}}

\bibleverse{20} Después de todo esto, cuando Josías había preparado el
templo, Neco, rey de Egipto, subió a luchar contra Carquemis junto al
Éufrates, y Josías salió contra él. \bibleverse{21} Pero él le envió
embajadores, diciendo: ``¿Qué tengo yo que ver contigo, rey de Judá? No
vengo hoy contra ti, sino contra la casa con la que tengo guerra. Dios
me ha ordenado que me apresure. Ten cuidado de que sea Dios quien esté
conmigo, para que no te destruya''.

\bibleverse{22} Sin embargo, Josías no quiso apartar su rostro de él,
sino que se disfrazó para luchar con él, y no escuchó las palabras de
Neco de boca de Dios, y vino a luchar en el valle de Meguido.
\bibleverse{23} Los arqueros dispararon contra el rey Josías, y el rey
dijo a sus servidores: ``¡Llévenme, porque estoy gravemente herido!''

\bibleverse{24} Sus servidores lo sacaron del carro, lo metieron en el
segundo carro que tenía y lo llevaron a Jerusalén; murió y fue enterrado
en los sepulcros de sus padres. Todo Judá y Jerusalén hicieron duelo por
Josías. \bibleverse{25} Jeremías se lamentó por Josías, y todos los
cantores y cantoras hablaron de Josías en sus lamentaciones hasta el día
de hoy; y las convirtieron en una ordenanza en Israel. He aquí que están
escritas en las lamentaciones. \footnote{\textbf{35:25} Jer 22,10-11}
\bibleverse{26} El resto de los hechos de Josías y sus buenas obras,
según lo que está escrito en la ley de Yahvé, \bibleverse{27} y sus
hechos, primeros y últimos, he aquí que están escritos en el libro de
los reyes de Israel y de Judá.

\hypertarget{joachuxe2z-rey-de-juduxe1}{%
\subsection{Joachâz rey de Judá}\label{joachuxe2z-rey-de-juduxe1}}

\hypertarget{section-35}{%
\section{36}\label{section-35}}

\bibleverse{1} Entonces el pueblo del país tomó a Joacaz, hijo de
Josías, y lo hizo rey en lugar de su padre en Jerusalén. \bibleverse{2}
Joacaz tenía veintitrés años cuando comenzó a reinar, y reinó tres meses
en Jerusalén. \bibleverse{3} El rey de Egipto lo destituyó de su cargo
en Jerusalén y le impuso una multa de cien talentos de plata y un
talento\footnote{\textbf{36:3} Un talento son unos 30 kilogramos o 66
  libras o 965 onzas troy} de oro.

\hypertarget{joacim-kuxf6nig-von-juda}{%
\subsection{Joacim König von Juda}\label{joacim-kuxf6nig-von-juda}}

\bibleverse{4} El rey de Egipto hizo a Eliaquim, su hermano, rey de Judá
y de Jerusalén, y le cambió el nombre por el de Joaquín. Neco tomó a
Joacaz, su hermano, y lo llevó a Egipto.

\bibleverse{5} Joacim tenía veinticinco años cuando comenzó a reinar, y
reinó once años en Jerusalén. Hizo lo que era malo a los ojos de Yavé,
su Dios. \bibleverse{6} Nabucodonosor, rey de Babilonia, subió contra él
y lo ató con grilletes para llevarlo a Babilonia. \footnote{\textbf{36:6}
  Jer 22,18} \bibleverse{7} Nabucodonosor también llevó a Babilonia
algunos de los utensilios de la casa de Yavé, y los puso en su templo en
Babilonia. \footnote{\textbf{36:7} Esd 1,7} \bibleverse{8} Los demás
hechos de Joacim, y las abominaciones que hizo, y lo que se halló en él,
he aquí que están escritos en el libro de los reyes de Israel y de Judá;
y reinó en su lugar Joaquín, su hijo.

\hypertarget{joachuxeen-rey-de-juduxe1}{%
\subsection{Joachîn rey de Judá}\label{joachuxeen-rey-de-juduxe1}}

\bibleverse{9} Joaquín tenía ocho años cuando comenzó a reinar, y reinó
tres meses y diez días en Jerusalén. Hizo lo que era malo a los ojos de
Yahvé. \bibleverse{10} A la vuelta del año, el rey Nabucodonosor lo
envió y lo llevó a Babilonia, junto con los objetos de valor de la casa
de Yavé, y puso a su hermano Sedequías como rey de Judá y de Jerusalén.
\footnote{\textbf{36:10} Jer 22,24-30}

\hypertarget{sedecuxedas-rey-de-juduxe1-la-ruina-de-uxe9l-y-de-su-gente}{%
\subsection{Sedecías, rey de Judá; la ruina de él y de su
gente}\label{sedecuxedas-rey-de-juduxe1-la-ruina-de-uxe9l-y-de-su-gente}}

\bibleverse{11} Sedequías tenía veintiún años cuando comenzó a reinar, y
reinó once años en Jerusalén. \footnote{\textbf{36:11} Jer 52,1-27}
\bibleverse{12} Hizo lo que era malo a los ojos de Yavé, su Dios. No se
humilló ante el profeta Jeremías que hablaba por boca de Yavé.
\footnote{\textbf{36:12} Jer 37,1; Jer 38,1-38} \bibleverse{13} También
se rebeló contra el rey Nabucodonosor, que lo había hecho jurar por
Dios; pero endureció su cerviz y su corazón para no volverse a Yahvé, el
Dios de Israel. \bibleverse{14} Además, todos los jefes de los
sacerdotes y del pueblo cometieron una gran prevaricación, siguiendo
todas las abominaciones de las naciones, y contaminaron la casa de Yavé
que él había santificado en Jerusalén. \footnote{\textbf{36:14} Deut
  18,9}

\bibleverse{15} Yahvé, el Dios de sus padres, les envió por medio de sus
mensajeros, madrugando y enviando, porque se compadecía de su pueblo y
de su morada; \bibleverse{16} pero se burlaron de los mensajeros de
Dios, despreciaron sus palabras y se mofaron de sus profetas, hasta que
la ira de Yahvé se levantó contra su pueblo, hasta que no hubo remedio.
\footnote{\textbf{36:16} Luc 20,10-12; Hech 7,52}

\hypertarget{destrucciuxf3n-del-imperio-por-nabucodonosor-el-cautiverio-babiluxf3nico}{%
\subsection{Destrucción del imperio por Nabucodonosor; el cautiverio
babilónico}\label{destrucciuxf3n-del-imperio-por-nabucodonosor-el-cautiverio-babiluxf3nico}}

\bibleverse{17} Por eso trajo sobre ellos al rey de los caldeos, que
mató a sus jóvenes a espada en la casa de su santuario, y no tuvo
compasión ni de jóvenes ni de vírgenes, ni de ancianos ni de enfermos.
Los entregó a todos en su mano. \bibleverse{18} Todos los utensilios de
la casa de Dios, grandes y pequeños, y los tesoros de la casa de Yahvé,
y los tesoros del rey y de sus príncipes, todo eso lo llevó a Babilonia.
\bibleverse{19} Incendiaron la casa de Dios, derribaron la muralla de
Jerusalén, quemaron todos sus palacios con fuego y destruyeron todos sus
objetos de valor. \bibleverse{20} Llevó a Babilonia a los que habían
escapado de la espada, y fueron siervos de él y de sus hijos hasta el
reinado del reino de Persia, \bibleverse{21} para que se cumpliera la
palabra de Yavé por boca de Jeremías, hasta que la tierra disfrutara de
sus sábados. Mientras estuvo desolada, guardó el sábado, para cumplir
setenta años. \footnote{\textbf{36:21} Lev 26,34; Jer 25,8-11}

\hypertarget{el-permiso-para-regresar-a-casa-del-rey-persa-ciro}{%
\subsection{El permiso para regresar a casa del rey persa
Ciro}\label{el-permiso-para-regresar-a-casa-del-rey-persa-ciro}}

\bibleverse{22} En el primer año de Ciro, rey de Persia, para que se
cumpliera la palabra de Yahvé por boca de Jeremías, Yahvé despertó el
espíritu de Ciro, rey de Persia, de modo que hizo una proclama por todo
su reino y la puso también por escrito, diciendo: \footnote{\textbf{36:22}
  Jer 29,10; Is 44,28} \bibleverse{23} ``Ciro, rey de Persia, dice:
`Yahvé, el Dios del cielo, me ha dado todos los reinos de la tierra y me
ha ordenado que le construya una casa en Jerusalén, que está en Judá. El
que esté entre vosotros de todo su pueblo, que Yahvé su Dios esté con
él, y que suba'\,''.
