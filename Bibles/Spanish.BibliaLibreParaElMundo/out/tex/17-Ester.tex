\hypertarget{la-fiesta-del-rey-persa-assuero-en-susa-para-los-dignatarios-y-altos-funcionarios-de-su-imperio}{%
\subsection{La fiesta del rey persa Assuero en Susa para los dignatarios
y altos funcionarios de su
imperio}\label{la-fiesta-del-rey-persa-assuero-en-susa-para-los-dignatarios-y-altos-funcionarios-de-su-imperio}}

\hypertarget{section}{%
\section{1}\label{section}}

\bibleverse{1} En los días de Asuero (éste es Asuero, que reinó desde la
India hasta Etiopía, sobre ciento veintisiete provincias),
\bibleverse{2} en aquellos días, cuando el rey Asuero se sentó en el
trono de su reino, que estaba en el palacio de Susa, \bibleverse{3} en
el tercer año de su reinado, hizo una fiesta para todos sus príncipes y
sus servidores; el ejército de Persia y de Media, los nobles y los
príncipes de las provincias estaban delante de él. \bibleverse{4}
Exhibió las riquezas de su glorioso reino y el honor de su excelente
majestad muchos días, hasta ciento ochenta días.

\hypertarget{la-comida-de-los-habitantes-de-la-ciudad-real-susa-la-fiesta-de-la-reina-wasthi}{%
\subsection{La comida de los habitantes de la ciudad real Susa; la
fiesta de la reina
Wasthi}\label{la-comida-de-los-habitantes-de-la-ciudad-real-susa-la-fiesta-de-la-reina-wasthi}}

\bibleverse{5} Cuando se cumplieron estos días, el rey hizo una fiesta
de siete días para todo el pueblo presente en el palacio de Susa, tanto
el grande como el pequeño, en el patio del jardín del palacio real.
\bibleverse{6} Había colgaduras de material blanco y azul, sujetas con
cordones de lino fino y púrpura a anillos de plata y a columnas de
mármol. Los divanes eran de oro y plata, sobre un pavimento de mármol
rojo, blanco, amarillo y negro. \bibleverse{7} Les daban de beber en
vasos de oro de diversas clases, incluso vino real en abundancia, según
la generosidad del rey. \bibleverse{8} De acuerdo con la ley, la bebida
no era obligatoria, pues así lo había ordenado el rey a todos los
funcionarios de su casa, para que hicieran lo que cada uno quisiera.

\bibleverse{9} También la reina Vasti hizo un banquete para las mujeres
de la casa real que pertenecía al rey Asuero.

\hypertarget{wasthi-se-niega-a-aparecer-en-el-saluxf3n-de-baile}{%
\subsection{Wasthi se niega a aparecer en el salón de
baile}\label{wasthi-se-niega-a-aparecer-en-el-saluxf3n-de-baile}}

\bibleverse{10} El séptimo día, cuando el corazón del rey estaba alegre
por el vino, ordenó a Mehumán, Biztha, Harbona, Bigtha y Abagtha, Zethar
y Carcass, los siete eunucos que servían en presencia del rey Asuero,
\bibleverse{11} que trajeran a la reina Vasti ante el rey con la corona
real, para mostrar al pueblo y a los príncipes su belleza, pues era
hermosa. \bibleverse{12} Pero la reina Vasti se negó a presentarse a la
orden del rey por medio de los eunucos. Por eso el rey se enojó mucho, y
su ira ardía en él.

\hypertarget{asesoramiento-y-toma-de-decisiones-sobre-quuxe9-castigo-anuncio-del-repudio-en-todo-el-imperio}{%
\subsection{Asesoramiento y toma de decisiones sobre qué castigo;
Anuncio del repudio en todo el
imperio}\label{asesoramiento-y-toma-de-decisiones-sobre-quuxe9-castigo-anuncio-del-repudio-en-todo-el-imperio}}

\bibleverse{13} Entonces el rey dijo a los sabios, que conocían los
tiempos (pues era costumbre del rey consultar a los que conocían la ley
y el juicio; \footnote{\textbf{1:13} 1Cró 12,31} \bibleverse{14} y junto
a él estaban Carshena, Shethar, Admatha, Tarshish, Meres, Marsena y
Memucan, los siete príncipes de Persia y de Media, que veían la cara del
rey y se sentaban los primeros en el reino), \bibleverse{15} ``¿Qué
haremos con la reina Vasti según la ley, porque no ha cumplido la orden
del rey Asuero por los eunucos?''

\bibleverse{16} Memucán respondió ante el rey y los príncipes: ``La
reina Vasti no ha hecho mal sólo al rey, sino también a todos los
príncipes y a todo el pueblo que está en todas las provincias del rey
Asuero. \bibleverse{17} Porque esta acción de la reina será conocida por
todas las mujeres, haciéndolas despreciar a sus maridos cuando se diga:
`El rey Asuero mandó traer a la reina Vasti ante él, pero ella no vino'.
\bibleverse{18} Hoy, las princesas de Persia y de Media que se han
enterado de la acción de la reina lo contarán a todos los príncipes del
rey. Esto causará mucho desprecio e ira.

\bibleverse{19} ``Si al rey le parece bien, que salga de él un
mandamiento real y que se escriba entre las leyes de los persas y de los
medos, para que no pueda ser alterado, a fin de que Vasti no vuelva a
presentarse ante el rey Asuero; y que el rey dé su patrimonio real a
otra que sea mejor que ella. \footnote{\textbf{1:19} Dan 6,8}
\bibleverse{20} Cuando se publique en todo el reino el decreto del rey
que él hará (porque es grande), todas las mujeres darán honor a sus
maridos, tanto a los grandes como a los pequeños.''

\bibleverse{21} Este consejo agradó al rey y a los príncipes, y el rey
hizo conforme a la palabra de Memucán: \bibleverse{22} pues envió cartas
a todas las provincias del rey, a cada provincia según su escritura, y a
cada pueblo en su lengua, para que cada uno gobernara su casa, hablando
en la lengua de su pueblo. \footnote{\textbf{1:22} Est 3,12; Est 8,9;
  Gén 3,16}

\hypertarget{organizaciuxf3n-de-un-gran-espectuxe1culo-nupcial-para-el-rey}{%
\subsection{Organización de un gran espectáculo nupcial para el
rey}\label{organizaciuxf3n-de-un-gran-espectuxe1culo-nupcial-para-el-rey}}

\hypertarget{section-1}{%
\section{2}\label{section-1}}

\bibleverse{1} Después de estas cosas, cuando la ira del rey Asuero se
apaciguó, se acordó de Vasti, de lo que había hecho y de lo que se había
decretado contra ella. \bibleverse{2} Entonces los servidores del rey
que le servían dijeron: ``Que se busquen jóvenes vírgenes hermosas para
el rey. \bibleverse{3} Que el rey designe oficiales en todas las
provincias de su reino, para que reúnan a todas las jóvenes vírgenes
hermosas en la ciudadela de Susa, en la casa de las mujeres, bajo la
custodia de Hegai, el eunuco del rey, guardián de las mujeres. Que se
les den cosméticos; \bibleverse{4} y que la doncella que agrade al rey
sea reina en lugar de Vasti''. El asunto agradó al rey, y así lo hizo.

\hypertarget{informaciuxf3n-sobre-la-prehistoria-de-esther}{%
\subsection{Información sobre la prehistoria de
Esther}\label{informaciuxf3n-sobre-la-prehistoria-de-esther}}

\bibleverse{5} Había en la ciudadela de Susa un judío que se llamaba
Mardoqueo, hijo de Jair, hijo de Simei, hijo de Cis, benjamita,
\footnote{\textbf{2:5} 1Sam 14,51} \bibleverse{6} que había sido llevado
de Jerusalén con los cautivos que habían sido llevados con Jeconías, rey
de Judá, a quienes Nabucodonosor, rey de Babilonia, había llevado.
\footnote{\textbf{2:6} 2Re 24,15-16} \bibleverse{7} Crió a Hadasa, es
decir, a Ester, la hija de su tío, pues no tenía padre ni madre. La
doncella era hermosa y bella; y cuando su padre y su madre murieron,
Mardoqueo la tomó como hija propia. \footnote{\textbf{2:7} Est 2,15}

\hypertarget{el-auxf1o-de-preparaciuxf3n-de-ester-en-el-palacio-real-y-su-elevaciuxf3n-a-reina}{%
\subsection{El año de preparación de Ester en el palacio real y su
elevación a
reina}\label{el-auxf1o-de-preparaciuxf3n-de-ester-en-el-palacio-real-y-su-elevaciuxf3n-a-reina}}

\bibleverse{8} Así que, cuando se oyó la orden del rey y su decreto, y
cuando se reunieron muchas doncellas en la ciudadela de Susa, para la
custodia de Hegai, Ester fue llevada a la casa del rey, a la custodia de
Hegai, guardián de las mujeres. \bibleverse{9} La doncella le agradó y
obtuvo de él su benevolencia. Se apresuró a darle cosméticos y sus
raciones de comida, así como las siete doncellas selectas que debían
entregársele fuera de la casa del rey. La trasladó a ella y a sus
doncellas al mejor lugar de la casa de las mujeres. \bibleverse{10}
Ester no había dado a conocer su pueblo ni sus parientes, porque
Mardoqueo le había ordenado que no lo hiciera. \bibleverse{11} Mardoqueo
se paseaba todos los días frente al patio de la casa de las mujeres,
para saber cómo estaba Ester y qué sería de ella.

\bibleverse{12} A cada joven le llegaba el turno de entrar ante el rey
Asuero después de su purificación durante doce meses (pues así se
cumplían los días de su purificación, seis meses con aceite de mirra y
seis meses con fragancias dulces y con preparaciones para embellecer a
las mujeres). \bibleverse{13} La joven se presentó entonces ante el rey
de esta manera: se le dio todo lo que deseaba para que fuera con ella de
la casa de las mujeres a la casa del rey. \bibleverse{14} Al anochecer
se fue, y al día siguiente volvió a la segunda casa de las mujeres, a la
custodia de Shaashgaz, el eunuco del rey, que guardaba las concubinas.
No volvió a entrar en la casa del rey, a no ser que el rey se
complaciera en ella, y la llamara por su nombre.

\bibleverse{15} Cuando llegó el turno de Ester, hija de Abihail, tío de
Mardoqueo, que la había tomado por hija, para entrar al rey, no exigió
nada más que lo que le aconsejaba Hegai, el eunuco del rey, guardián de
las mujeres. Ester obtuvo el favor de todos los que la miraban.

\bibleverse{16} Ester fue llevada a la casa real del rey Asuero en el
mes décimo, que es el mes de Tebet, en el séptimo año de su reinado.
\bibleverse{17} El rey amó a Ester más que a todas las mujeres, y ella
obtuvo el favor y la bondad ante sus ojos más que todas las vírgenes, de
modo que puso la corona real sobre su cabeza y la nombró reina en lugar
de Vasti.

\bibleverse{18} Entonces el rey hizo una gran fiesta para todos sus
príncipes y sus siervos, la fiesta de Ester; y proclamó un día de fiesta
en las provincias, y dio regalos según la generosidad del rey.

\hypertarget{mardochai-descubre-una-conspiraciuxf3n-contra-el-rey-su-muxe9rito-estuxe1-registrado-en-las-cruxf3nicas-del-reino}{%
\subsection{Mardochai descubre una conspiración contra el rey; su mérito
está registrado en las crónicas del
reino}\label{mardochai-descubre-una-conspiraciuxf3n-contra-el-rey-su-muxe9rito-estuxe1-registrado-en-las-cruxf3nicas-del-reino}}

\bibleverse{19} Cuando las vírgenes se reunieron por segunda vez,
Mardoqueo estaba sentado en la puerta del rey. \bibleverse{20} Ester aún
no había dado a conocer a sus parientes ni a su pueblo, como le había
ordenado Mardoqueo; porque Ester obedecía a Mardoqueo, como lo hacía
cuando era educada por él. \footnote{\textbf{2:20} Est 2,10}
\bibleverse{21} En aquellos días, mientras Mardoqueo estaba sentado en
la puerta del rey, dos de los eunucos del rey, Bigtán y Teres, que eran
porteros, se enojaron y trataron de poner las manos sobre el rey Asuero.
\bibleverse{22} Este asunto llegó a conocimiento de Mardoqueo, quien
informó a la reina Ester; y Ester informó al rey en nombre de Mardoqueo.
\bibleverse{23} Cuando se investigó este asunto, y se comprobó que era
así, ambos fueron colgados en una horca; y se escribió en el libro de
las crónicas en presencia del rey. \footnote{\textbf{2:23} Est 6,1-2}

\hypertarget{promociuxf3n-de-amuxe1n-al-muxe1s-alto-honor-mardochai-se-niega-a-doblar-las-rodillas-amuxe1n-decide-exterminar-a-todos-los-juduxedos}{%
\subsection{Promoción de Amán al más alto honor; Mardochai se niega a
doblar las rodillas; Amán decide exterminar a todos los
judíos}\label{promociuxf3n-de-amuxe1n-al-muxe1s-alto-honor-mardochai-se-niega-a-doblar-las-rodillas-amuxe1n-decide-exterminar-a-todos-los-juduxedos}}

\hypertarget{section-2}{%
\section{3}\label{section-2}}

\bibleverse{1} Después de estas cosas, el rey Asuero ascendió a Amán,
hijo de Hamedata el Agagita, y lo hizo avanzar, y puso su asiento por
encima de todos los príncipes que estaban con él. \bibleverse{2} Todos
los siervos del rey que estaban en la puerta del rey se inclinaron y
rindieron homenaje a Amán, porque el rey así lo había ordenado con
respecto a él. Pero Mardoqueo no se inclinó ni le rindió homenaje.
\bibleverse{3} Entonces los servidores del rey que estaban en la puerta
del rey dijeron a Mardoqueo: ``¿Por qué desobedeces el mandato del
rey?'' \bibleverse{4} Como le hablaban todos los días y él no los
escuchaba, se lo comunicaron a Amán para ver si la razón de Mardoqueo se
mantenía, pues él les había dicho que era judío. \bibleverse{5} Cuando
Amán vio que Mardoqueo no se inclinaba ni le rendía homenaje, se llenó
de ira. \bibleverse{6} Pero despreció la idea de poner las manos sobre
Mardoqueo solo, porque le habían dado a conocer el pueblo de Mardoqueo.
Por lo tanto, Amán trató de destruir a todos los judíos que había en
todo el reino de Asuero, incluso al pueblo de Mardoqueo.

\hypertarget{amuxe1n-hace-cumplir-su-resoluciuxf3n-con-el-rey}{%
\subsection{Amán hace cumplir su resolución con el
rey}\label{amuxe1n-hace-cumplir-su-resoluciuxf3n-con-el-rey}}

\bibleverse{7} En el mes primero, que es el mes de Nisán, en el año
duodécimo del rey Asuero, echaron Pur, es decir, la suerte, delante de
Amán de día en día y de mes en mes, y eligieron el mes duodécimo, que es
el mes de Adar. \footnote{\textbf{3:7} Est 9,24} \bibleverse{8} Amán
dijo al rey Asuero: ``Hay un pueblo disperso y diseminado entre los
pueblos de todas las provincias de tu reino, y sus leyes son diferentes
a las de los demás pueblos. No cumplen las leyes del rey. Por lo tanto,
al rey no le conviene permitir que se queden. \bibleverse{9} Si al rey
le parece bien, que se escriba que sean destruidos; y yo pagaré diez mil
talentos de plata en manos de los encargados de los negocios del rey,
para que los ingresen en las arcas del rey.''

\bibleverse{10} El rey se quitó el anillo de la mano y se lo dio a Amán,
hijo de Hamedata el Agagita, enemigo de los judíos. \footnote{\textbf{3:10}
  Est 8,2} \bibleverse{11} El rey dijo a Amán: ``La plata se te da a ti,
también el pueblo, para que hagas con él lo que te parezca''.

\hypertarget{el-exterminio-de-los-juduxedos-en-todo-el-imperio-ordenado-por-el-rey}{%
\subsection{El exterminio de los judíos en todo el imperio ordenado por
el
rey}\label{el-exterminio-de-los-juduxedos-en-todo-el-imperio-ordenado-por-el-rey}}

\bibleverse{12} Entonces los escribas del rey fueron convocados el
primer mes, a los trece días del mes; y todo lo que Amán mandó fue
escrito a los gobernadores locales del rey, y a los gobernadores que
estaban sobre cada provincia, y a los príncipes de cada pueblo, a cada
provincia según su escritura, y a cada pueblo en su idioma. Estaba
escrito en nombre del rey Asuero, y estaba sellado con el anillo del
rey. \footnote{\textbf{3:12} Est 1,22} \bibleverse{13} Se enviaron
cartas por medio de correos a todas las provincias del rey, para
destruir, matar y hacer perecer a todos los judíos, jóvenes y ancianos,
niños y mujeres, en un solo día, el día trece del mes duodécimo, que es
el mes de Adar, y para saquear sus bienes. \bibleverse{14} Se publicó
una copia de la carta para que el decreto se distribuyera en todas las
provincias, a fin de que todos los pueblos estuvieran preparados para
ese día. \bibleverse{15} Los mensajeros salieron a toda prisa por orden
del rey, y el decreto se repartió en la ciudadela de Susa. El rey y Amán
se sentaron a beber; pero la ciudad de Susa estaba perpleja.

\hypertarget{el-dolor-de-mardochai-sus-esfuerzos-para-mover-a-ester-a-salvar-a-los-juduxedos}{%
\subsection{El dolor de Mardochai; sus esfuerzos para mover a Ester a
salvar a los
judíos}\label{el-dolor-de-mardochai-sus-esfuerzos-para-mover-a-ester-a-salvar-a-los-juduxedos}}

\hypertarget{section-3}{%
\section{4}\label{section-3}}

\bibleverse{1} Cuando Mardoqueo se enteró de todo lo que se había hecho,
rasgó sus ropas y se vistió de saco con cenizas, y salió al centro de la
ciudad, y se lamentó fuerte y amargamente. \bibleverse{2} Llegó hasta la
puerta del rey, pues a nadie se le permite entrar en la puerta del rey
vestido de cilicio. \bibleverse{3} En todas las provincias, dondequiera
que llegaba la orden del rey y su decreto, había gran luto entre los
judíos, y ayuno, llanto y lamentos; y muchos se acostaban en cilicio y
ceniza.

\hypertarget{ester-es-informada-por-mardochai-sobre-el-desastre-inminente-y-le-pide-que-ruegue-al-rey-por-misericordia}{%
\subsection{Ester es informada por Mardochai sobre el desastre inminente
y le pide que ruegue al rey por
misericordia}\label{ester-es-informada-por-mardochai-sobre-el-desastre-inminente-y-le-pide-que-ruegue-al-rey-por-misericordia}}

\bibleverse{4} Vinieron las doncellas de Ester y sus eunucos y le
contaron esto, y la reina se entristeció mucho. Envió ropa a Mardoqueo,
para reemplazar su cilicio, pero él no la recibió. \bibleverse{5}
Entonces Ester llamó a Hatac, uno de los eunucos del rey, a quien había
designado para que la atendiera, y le ordenó que fuera a ver a
Mardoqueo, para averiguar qué era esto y por qué era. \bibleverse{6}
Salió, pues, Hatac a ver a Mardoqueo, a la plaza de la ciudad que estaba
delante de la puerta del rey. \bibleverse{7} Mardoqueo le contó todo lo
que le había sucedido y la suma exacta del dinero que Amán había
prometido pagar a las arcas del rey por la destrucción de los judíos.
\footnote{\textbf{4:7} Est 3,9} \bibleverse{8} También le dio la copia
del escrito del decreto que se había dado en Susa para destruirlos, para
que se lo mostrara a Ester y se lo declarara, y para que la instara a
entrar al rey para suplicarle y pedirle por su pueblo.

\hypertarget{la-negativa-de-esther-es-derrotada-por-mardochai-sin-embargo-requiere-que-los-juduxedos-mantengan-un-ayuno-estricto-a-su-favor}{%
\subsection{La negativa de Esther es derrotada por Mardochai; Sin
embargo, requiere que los judíos mantengan un ayuno estricto a su
favor}\label{la-negativa-de-esther-es-derrotada-por-mardochai-sin-embargo-requiere-que-los-juduxedos-mantengan-un-ayuno-estricto-a-su-favor}}

\bibleverse{9} Hathach vino y le contó a Ester las palabras de
Mardoqueo. \bibleverse{10} Entonces Ester habló con Hatac y le dio un
mensaje para Mardoqueo \bibleverse{11} ``Todos los siervos del rey y el
pueblo de las provincias del rey saben que cualquiera, sea hombre o
mujer, que se presente al rey en el patio interior sin ser llamado, hay
una ley para él: que sea condenado a muerte, excepto aquellos a quienes
el rey les extienda el cetro de oro, para que vivan. No he sido llamado
a entrar al rey en estos treinta días''. \footnote{\textbf{4:11} Est
  5,2; Est 8,4}

\bibleverse{12} Le contaron a Mardoqueo las palabras de Ester.
\bibleverse{13} Entonces Mardoqueo les pidió que le devolvieran a Ester
esta respuesta ``No pienses para ti que vas a escapar en la casa del rey
más que todos los judíos. \bibleverse{14} Porque si ahora callas, el
alivio y la liberación vendrán a los judíos desde otro lugar, pero tú y
la casa de tu padre pereceréis. ¿Quién sabe si no has venido al reino
para un momento como éste?'' \footnote{\textbf{4:14} Gén 45,7}

\bibleverse{15} Entonces Ester les pidió que respondieran a Mardoqueo,
\bibleverse{16} ``Ve, reúne a todos los judíos que están presentes en
Susa, y ayunen por mí, y no coman ni beban durante tres días, ni de
noche ni de día. Yo y mis doncellas también ayunaremos de la misma
manera. Entonces entraré a ver al rey, lo cual es contrario a la ley; y
si perezco, perezco''. \footnote{\textbf{4:16} 2Re 7,4} \bibleverse{17}
Así pues, Mardoqueo se puso en camino e hizo todo lo que Ester le había
ordenado.

\hypertarget{la-recepciuxf3n-amistosa-de-ester-por-parte-del-rey-y-el-engauxf1o-de-amuxe1n}{%
\subsection{La recepción amistosa de Ester por parte del rey y el engaño
de
Amán}\label{la-recepciuxf3n-amistosa-de-ester-por-parte-del-rey-y-el-engauxf1o-de-amuxe1n}}

\hypertarget{section-4}{%
\section{5}\label{section-4}}

\bibleverse{1} Al tercer día, Ester se vistió con sus ropas reales y se
puso en el patio interior de la casa real, junto a la casa del rey. El
rey estaba sentado en su trono real en la casa real, junto a la entrada
de la casa. \bibleverse{2} Cuando el rey vio a la reina Ester de pie en
el patio, ella obtuvo el favor de sus ojos; y el rey le tendió a Ester
el cetro de oro que tenía en la mano. Entonces Ester se acercó y tocó la
punta del cetro. \footnote{\textbf{5:2} Est 4,11; Est 8,4}

\bibleverse{3} Entonces el rey le preguntó: ``¿Qué quieres, reina Ester?
¿Cuál es tu petición? Se te dará hasta la mitad del reino''.

\bibleverse{4} Ester dijo: ``Si al rey le parece bien, que el rey y Amán
vengan hoy al banquete que le he preparado''. \footnote{\textbf{5:4} Est
  1,19}

\hypertarget{el-rey-invitado-por-ester-a-cenar-con-amuxe1n-acepta-otra-invitaciuxf3n-a-cenar}{%
\subsection{El rey, invitado por Ester a cenar con Amán, acepta otra
invitación a
cenar}\label{el-rey-invitado-por-ester-a-cenar-con-amuxe1n-acepta-otra-invitaciuxf3n-a-cenar}}

\bibleverse{5} Entonces el rey dijo: ``Trae pronto a Amán, para que se
haga lo que ha dicho Ester''. Así que el rey y Amán llegaron al banquete
que Ester había preparado.

\bibleverse{6} El rey dijo a Ester en el banquete del vino: ``¿Cuál es
tu petición? Se te concederá. ¿Cuál es tu petición? Hasta la mitad del
reino se cumplirá''. \footnote{\textbf{5:6} Est 9,12}

\bibleverse{7} Entonces Ester respondió y dijo: ``Mi petición y mi
solicitud es ésta. \bibleverse{8} Si he hallado gracia ante los ojos del
rey, y si al rey le agrada conceder mi petición y cumplir mi solicitud,
que el rey y Amán vengan al banquete que les prepararé, y yo haré mañana
lo que el rey ha dicho.''

\hypertarget{el-altivo-engauxf1o-de-amuxe1n-su-intenciuxf3n-de-deshacerse-de-mardochai}{%
\subsection{El altivo engaño de Amán; su intención de deshacerse de
Mardochai}\label{el-altivo-engauxf1o-de-amuxe1n-su-intenciuxf3n-de-deshacerse-de-mardochai}}

\bibleverse{9} Entonces Amán salió aquel día alegre y contento de
corazón, pero cuando vio a Mardoqueo en la puerta del rey, que no se
levantaba ni se movía por él, se llenó de ira contra Mardoqueo.
\bibleverse{10} Sin embargo, Amán se contuvo y se fue a su casa. Allí
mandó llamar a sus amigos y a Zeresh, su esposa. \bibleverse{11} Amán
les contó la gloria de sus riquezas, la multitud de sus hijos, todas las
cosas en que el rey lo había promovido, y cómo lo había aventajado por
encima de los príncipes y servidores del rey.

\bibleverse{12} También dijo Amán: ``Sí, la reina Ester no dejó entrar a
nadie con el rey al banquete que ella había preparado, sino a mí; y
mañana también estoy invitado por ella junto con el rey. \bibleverse{13}
Pero todo esto no me sirve de nada, mientras vea al judío Mardoqueo
sentado a la puerta del rey.''

\bibleverse{14} Entonces Zeresh, su mujer, y todos sus amigos le
dijeron: ``Que se haga una horca de cincuenta codos de altura, y por la
mañana habla con el rey de colgar a Mardoqueo en ella. Entonces entra
alegremente con el rey al banquete''. Esto le gustó a Amán, así que
mandó hacer la horca.

\hypertarget{mardochai-criado-en-alto-honor-por-amuxe1n}{%
\subsection{Mardochai criado en alto honor por
Amán}\label{mardochai-criado-en-alto-honor-por-amuxe1n}}

\hypertarget{section-5}{%
\section{6}\label{section-5}}

\bibleverse{1} Aquella noche, el rey no podía dormir. Mandó traer el
libro de los registros de las crónicas, y se los leyeron al rey.
\bibleverse{2} Se encontró escrito que Mardoqueo había hablado de
Bigtana y Teresh, dos eunucos del rey, que eran porteros, y que habían
tratado de ponerle las manos encima al rey Asuero. \footnote{\textbf{6:2}
  Est 2,21-23} \bibleverse{3} El rey dijo: ``¿Qué honor y dignidad se le
ha dado a Mardoqueo por esto?'' Entonces los sirvientes del rey que lo
atendían dijeron: ``No se ha hecho nada por él''.

\bibleverse{4} El rey dijo: ``¿Quién está en el patio?'' Ahora bien,
Amán había entrado en el atrio exterior de la casa real, para hablar con
el rey acerca de colgar a Mardoqueo en la horca que había preparado para
él. \footnote{\textbf{6:4} Est 5,14}

\hypertarget{amuxe1n-involuntariamente-hace-que-el-rey-decida-sobre-un-honor-extraordinario-para-mardochai-y-que-lo-lleve-a-cabo-personalmente}{%
\subsection{Amán involuntariamente hace que el rey decida sobre un honor
extraordinario para Mardochai y que lo lleve a cabo
personalmente}\label{amuxe1n-involuntariamente-hace-que-el-rey-decida-sobre-un-honor-extraordinario-para-mardochai-y-que-lo-lleve-a-cabo-personalmente}}

\bibleverse{5} Los servidores del rey le dijeron: ``Mira, Amán está en
el patio''. El rey dijo: ``Que entre''. \bibleverse{6} Así que Amán
entró. El rey le dijo: ``¿Qué se hará con el hombre a quien el rey se
complace en honrar?'' Y dijo Amán en su corazón: ``¿A quién se deleita
el rey en honrar más que a mí mismo?'' \bibleverse{7} Amán dijo al rey:
``Para el hombre a quien el rey se deleita en honrar, \bibleverse{8} que
se traigan las ropas reales que el rey acostumbra a usar, y el caballo
en que el rey monta, y en cuya cabeza está puesta una corona real.
\bibleverse{9} Que la ropa y el caballo sean entregados a la mano de uno
de los príncipes más nobles del rey, para que se vista con ellos al
hombre a quien el rey se complace en honrar, y se le haga cabalgar por
la plaza de la ciudad, y se proclame ante él: ``¡Así se hará con el
hombre a quien el rey se complace en honrar!''

\bibleverse{10} Entonces el rey dijo a Amán: ``Apresúrate a tomar la
ropa y el caballo, como has dicho, y hazlo por el judío Mardoqueo, que
se sienta a la puerta del rey. Que no falte nada de todo lo que has
dicho''.

\bibleverse{11} Entonces Amán tomó la ropa y el caballo y vistió a
Mardoqueo, lo hizo cabalgar por la plaza de la ciudad y proclamó ante
él: ``¡Así se hará con el hombre al que el rey se complace en honrar!''

\hypertarget{el-dolor-de-amuxe1n-lleno-de-presentimientos-fue-al-banquete-de-la-reina}{%
\subsection{El dolor de Amán; Lleno de presentimientos, fue al banquete
de la
reina}\label{el-dolor-de-amuxe1n-lleno-de-presentimientos-fue-al-banquete-de-la-reina}}

\bibleverse{12} Mardoqueo regresó a la puerta del rey, pero Amán se
apresuró a ir a su casa, de luto y con la cabeza cubierta.
\bibleverse{13} Amán contó a Zeres su mujer y a todos sus amigos todo lo
que le había sucedido. Entonces sus sabios y Zeresh su mujer le dijeron:
``Si Mardoqueo, ante quien has empezado a caer, es de ascendencia judía,
no prevalecerás contra él, sino que seguramente caerás ante él.''
\bibleverse{14} Mientras aún hablaban con él, llegaron los eunucos del
rey y se apresuraron a llevar a Amán al banquete que había preparado
Ester. \footnote{\textbf{6:14} Est 5,8}

\hypertarget{durante-la-cena-ester-revela-los-planes-de-amuxe1n-de-matar-al-rey-el-rey-se-levanta-enojado-de-la-cena}{%
\subsection{Durante la cena, Ester revela los planes de Amán de matar al
rey; el rey se levanta enojado de la
cena}\label{durante-la-cena-ester-revela-los-planes-de-amuxe1n-de-matar-al-rey-el-rey-se-levanta-enojado-de-la-cena}}

\hypertarget{section-6}{%
\section{7}\label{section-6}}

\bibleverse{1} El rey y Amán vinieron a banquete con la reina Ester.
\footnote{\textbf{7:1} Est 5,8; Est 6,14} \bibleverse{2} El rey volvió a
decir a Ester el segundo día en el banquete de vino: ``¿Cuál es tu
petición, reina Ester? Se te concederá. ¿Cuál es tu petición? Hasta la
mitad del reino se cumplirá''.

\bibleverse{3} La reina Ester respondió: ``Si he hallado gracia ante tus
ojos, oh rey, y si al rey le parece bien, que se me dé la vida a
petición mía, y a mi pueblo a petición mía. \bibleverse{4} Porque hemos
sido vendidos, yo y mi pueblo, para ser destruidos, para ser asesinados
y para perecer. Pero si hubiéramos sido vendidos por esclavos y
esclavas, habría callado, aunque el adversario no hubiera podido
compensar la pérdida del rey.''

\bibleverse{5} Entonces el rey Asuero dijo a la reina Ester: ``¿Quién es
y dónde está el que se atrevió a hacer eso en su corazón?''

\bibleverse{6} Ester dijo: ``¡Un adversario y un enemigo, este malvado
Amán!'' Entonces Amán tuvo miedo ante el rey y la reina. \bibleverse{7}
El rey se levantó furioso del banquete de vino y se dirigió al jardín
del palacio. Amán se levantó para pedir su vida a la reina Ester, pues
veía que había un mal determinado contra él por parte del rey.

\hypertarget{a-su-regreso-el-rey-condenuxf3-a-muerte-a-amuxe1n-e-inmediatamente-lo-hizo-colgar-en-la-estaca-erigida-para-mardochai}{%
\subsection{A su regreso, el rey condenó a muerte a Amán e
inmediatamente lo hizo colgar en la estaca erigida para
Mardochai}\label{a-su-regreso-el-rey-condenuxf3-a-muerte-a-amuxe1n-e-inmediatamente-lo-hizo-colgar-en-la-estaca-erigida-para-mardochai}}

\bibleverse{8} Entonces el rey volvió a salir del jardín del palacio al
lugar del banquete del vino, y Amán había caído en el diván donde estaba
Ester. Entonces el rey dijo: ``¿Acaso va a agredir a la reina delante de
mí en la casa?'' Al salir la palabra de la boca del rey, cubrieron el
rostro de Amán.

\bibleverse{9} Entonces Harbonah, uno de los eunucos que estaban con el
rey, dijo: ``He aquí, la horca de cincuenta codos de altura, que Amán ha
hecho para Mardoqueo, que habló bien para el rey, está de pie en la casa
de Amán.'' El rey dijo: ``¡Cuélguenlo!''

\bibleverse{10} Así que colgaron a Amán en la horca que había preparado
para Mardoqueo. Entonces se aplacó la ira del rey.

\hypertarget{el-regalo-de-ester-y-la-exaltaciuxf3n-de-mardochai-por-parte-del-rey}{%
\subsection{El regalo de Ester y la exaltación de Mardochai por parte
del
rey}\label{el-regalo-de-ester-y-la-exaltaciuxf3n-de-mardochai-por-parte-del-rey}}

\hypertarget{section-7}{%
\section{8}\label{section-7}}

\bibleverse{1} Aquel día, el rey Asuero entregó la casa de Amán, el
enemigo de los judíos, a la reina Ester. Mardoqueo se presentó ante el
rey, pues Ester le había contado lo que era. \bibleverse{2} El rey se
quitó el anillo que le había quitado a Amán y se lo dio a Mardoqueo.
Ester puso a Mardoqueo al frente de la casa de Amán. \footnote{\textbf{8:2}
  Est 3,10}

\hypertarget{establecer-y-promulgar-medidas-de-protecciuxf3n-para-los-juduxedos-contra-sus-enemigos}{%
\subsection{Establecer y promulgar medidas de protección para los judíos
contra sus
enemigos}\label{establecer-y-promulgar-medidas-de-protecciuxf3n-para-los-juduxedos-contra-sus-enemigos}}

\bibleverse{3} Ester volvió a hablar ante el rey, se postró a sus pies y
le rogó con lágrimas que acabara con la maldad de Amán el agagita y con
el plan que había planeado contra los judíos. \bibleverse{4} Entonces el
rey le tendió a Ester el cetro de oro. Entonces Ester se levantó y se
presentó ante el rey. \footnote{\textbf{8:4} Est 5,2} \bibleverse{5}
Ella dijo: ``Si al rey le agrada, y si he hallado gracia ante sus ojos,
y la cosa le parece bien al rey, y soy agradable a sus ojos, que se
escriba para anular las cartas ideadas por Amán, hijo de Hamedata el
agagueo, que él escribió para destruir a los judíos que están en todas
las provincias del rey. \bibleverse{6} Porque, ¿cómo podré soportar ver
el mal que le espera a mi pueblo? ¿Cómo podré soportar ver la
destrucción de mis parientes?''

\bibleverse{7} Entonces el rey Asuero dijo a la reina Ester y al judío
Mardoqueo: ``Mira, he entregado a Ester a la casa de Amán, y lo han
colgado en la horca porque puso su mano sobre los judíos. \bibleverse{8}
Escribe también a los judíos lo que te plazca, en nombre del rey, y
séllalo con el anillo del rey; porque lo que se escribe en nombre del
rey, y se sella con el anillo del rey, no puede ser revocado por
nadie.''

\bibleverse{9} Entonces fueron llamados los escribas del rey en aquel
tiempo, en el mes tercero, que es el mes de Siván, a los veintitrés días
del mes; y se escribió según todo lo que Mardoqueo mandó a los judíos, y
a los gobernadores locales, y a los gobernadores y príncipes de las
provincias que están desde la India hasta Etiopía, ciento veintisiete
provincias, a cada provincia según su escritura, y a cada pueblo en su
lengua, y a los judíos en su escritura y en su lengua. \bibleverse{10}
Escribió en nombre del rey Asuero, y lo selló con el anillo del rey, y
envió las cartas por correo a caballo, montando en caballos reales
criados de corceles veloces. \bibleverse{11} En esas cartas, el rey
concedía a los judíos que estaban en cada ciudad que se reunieran y
defendieran sus vidas, para destruir, matar y hacer perecer a todo el
poder del pueblo y de la provincia que los asaltara, a sus pequeños y a
sus mujeres, y para saquear sus posesiones, \bibleverse{12} en un solo
día en todas las provincias del rey Asuero, el día trece del duodécimo
mes, que es el mes de Adar. \bibleverse{13} Se publicó en todos los
pueblos una copia de la carta para que el decreto se repartiera en todas
las provincias, a fin de que los judíos estuvieran preparados para ese
día para vengarse de sus enemigos. \bibleverse{14} Salieron, pues, los
mensajeros que montaban en caballos reales, se apresuraron y se pusieron
en marcha por orden del rey. El decreto se dio en la ciudadela de Susa.

\hypertarget{mardochai-aparece-en-susa-con-un-traje-principesco-alegruxeda-de-los-juduxedos-en-todo-el-imperio}{%
\subsection{Mardochai aparece en Susa con un traje principesco; Alegría
de los judíos en todo el
imperio}\label{mardochai-aparece-en-susa-con-un-traje-principesco-alegruxeda-de-los-juduxedos-en-todo-el-imperio}}

\bibleverse{15} Mardoqueo salió de la presencia del rey con ropas reales
de azul y blanco, y con una gran corona de oro, y con un manto de lino
fino y púrpura; y la ciudad de Susa gritó y se alegró. \bibleverse{16}
Los judíos tuvieron luz, alegría, gozo y honor. \bibleverse{17} En todas
las provincias y en todas las ciudades, dondequiera que llegaba el
mandamiento del rey y su decreto, los judíos tenían alegría, gozo,
fiesta y festividad. Muchos de entre los pueblos de la tierra se
hicieron judíos, porque el temor de los judíos había caído sobre ellos.
\footnote{\textbf{8:17} Éxod 15,14-16}

\hypertarget{exterminio-de-enemigos-de-los-juduxedos-en-todo-el-imperio-el-duxeda-13-del-mes-de-adar}{%
\subsection{Exterminio de enemigos de los judíos en todo el imperio el
día 13 del mes de
Adar}\label{exterminio-de-enemigos-de-los-juduxedos-en-todo-el-imperio-el-duxeda-13-del-mes-de-adar}}

\hypertarget{section-8}{%
\section{9}\label{section-8}}

\bibleverse{1} En el mes duodécimo, que es el mes de Adar, a los trece
días del mes, cuando el mandamiento del rey y su decreto estaban a punto
de ser ejecutados, el día en que los enemigos de los judíos esperaban
conquistarlos, (pero resultó lo contrario, que los judíos conquistaron a
los que los odiaban), \bibleverse{2} los judíos se reunieron en sus
ciudades por todas las provincias del rey Asuero, para echar mano a los
que querían hacerles daño. Nadie pudo resistirlos, porque el temor a
ellos había caído sobre todo el pueblo. \footnote{\textbf{9:2} Est 8,17}
\bibleverse{3} Todos los príncipes de las provincias, los gobernadores
locales, los intendentes y los que se ocupaban de los asuntos del rey
ayudaban a los judíos, porque el temor a Mardoqueo había caído sobre
ellos. \bibleverse{4} Porque Mardoqueo era grande en la casa del rey, y
su fama se extendía por todas las provincias, pues el hombre Mardoqueo
se hacía cada vez más grande. \bibleverse{5} Los judíos golpeaban a
todos sus enemigos a golpe de espada, con matanza y destrucción, y
hacían lo que querían con los que los odiaban. \bibleverse{6} En la
ciudadela de Susa, los judíos mataron y destruyeron a quinientos
hombres. \bibleverse{7} Mataron a Parshandatha, Dalphon, Aspatha,
\bibleverse{8} Poratha, Adalia, Aridatha, \bibleverse{9} Parmashta,
Arisai, Aridai, y Vaizatha, \bibleverse{10} los diez hijos de Haman hijo
de Hammedatha, el enemigo de los judíos, pero no pusieron su mano en el
botín.

\hypertarget{continuaciuxf3n-de-la-matanza-el-duxeda-14-del-mes-regocijo-de-los-juduxedos-para-celebrar-su-salvaciuxf3n}{%
\subsection{Continuación de la matanza el día 14 del mes; Regocijo de
los judíos para celebrar su
salvación}\label{continuaciuxf3n-de-la-matanza-el-duxeda-14-del-mes-regocijo-de-los-juduxedos-para-celebrar-su-salvaciuxf3n}}

\bibleverse{11} Aquel día se presentó ante el rey el número de los
muertos en la ciudadela de Susa. \bibleverse{12} El rey dijo a la reina
Ester: ``Los judíos han matado y destruido a quinientos hombres en la
ciudadela de Susa, incluidos los diez hijos de Amán; ¡qué han hecho
entonces en el resto de las provincias del rey! ¿Cuál es tu petición? Se
te concederá. ¿Cuál es tu otra petición? Se hará''. \footnote{\textbf{9:12}
  Est 5,6; Est 7,2}

\bibleverse{13} Entonces Ester dijo: ``Si al rey le parece bien, que se
conceda a los judíos que están en Susa que hagan también mañana lo que
se ha decretado hoy, y que los diez hijos de Amán sean colgados en la
horca.''

\bibleverse{14} El rey ordenó que se hiciera esto. Se dio un decreto en
Susa, y colgaron a los diez hijos de Amán. \bibleverse{15} Los judíos
que estaban en Susa se reunieron también el día catorce del mes de Adar
y mataron a trescientos hombres en Susa, pero no pusieron la mano en el
botín.

\bibleverse{16} Los demás judíos que estaban en las provincias del rey
se reunieron, defendieron sus vidas, descansaron de sus enemigos y
mataron a setenta y cinco mil de los que los odiaban; pero no pusieron
su mano en el botín. \bibleverse{17} Esto lo hicieron el día trece del
mes de Adar, y el día catorce de ese mes descansaron y lo convirtieron
en un día de fiesta y alegría.

\bibleverse{18} Pero los judíos que estaban en Susa se reunieron los
días trece y catorce del mes; y el día quince de ese mes descansaron, y
lo convirtieron en un día de fiesta y alegría. \bibleverse{19} Por lo
tanto, los judíos de las aldeas que viven en las ciudades no
amuralladas, hacen del decimocuarto día del mes de Adar un día de
alegría y de fiesta, un día festivo, y un día para enviarse regalos de
comida unos a otros.

\hypertarget{mardochai-ordena-la-celebraciuxf3n-de-la-fiesta-de-purim-para-todos-los-futuros}{%
\subsection{Mardochai ordena la celebración de la fiesta de Purim para
todos los
futuros}\label{mardochai-ordena-la-celebraciuxf3n-de-la-fiesta-de-purim-para-todos-los-futuros}}

\bibleverse{20} Mardoqueo escribió estas cosas y envió cartas a todos
los judíos que se encontraban en todas las provincias del rey Asuero,
tanto de cerca como de lejos, \bibleverse{21} para ordenarles que
celebraran anualmente los días catorce y quince del mes de Adar,
\bibleverse{22} como los días en que los judíos descansaban de sus
enemigos, y el mes que se convertía para ellos de tristeza en alegría, y
de luto en fiesta; para que los hicieran días de fiesta y de alegría, y
para que se enviaran regalos de comida unos a otros, y regalos a los
necesitados. \bibleverse{23} Los judíos aceptaron la costumbre que
habían iniciado, como les había escrito Mardoqueo, \bibleverse{24}
porque Amán, hijo de Hamedata, el agagita, enemigo de todos los judíos,
había tramado contra los judíos para destruirlos, y había echado a
``Pur'', es decir, a la suerte, para consumirlos y destruirlos;
\footnote{\textbf{9:24} Est 3,7} \bibleverse{25} pero cuando esto fue
conocido por el rey, éste ordenó por cartas que su malvado plan, que
había planeado contra los judíos, volviera sobre su propia cabeza, y que
él y sus hijos fueran colgados en la horca. \footnote{\textbf{9:25} Est
  9,14; Est 7,10}

\bibleverse{26} Por eso llamaron a estos días ``Purim'', de la palabra
``Pur''. Por lo tanto, a causa de todas las palabras de esta carta, y de
lo que habían visto sobre este asunto, y de lo que les había llegado,
\bibleverse{27} los judíos establecieron y se impusieron a sí mismos, a
sus descendientes y a todos los que se unieron a ellos, para que no
dejasen de guardar estos dos días según lo que estaba escrito y según su
tiempo señalado cada año; \bibleverse{28} y que estos días fueran
recordados y guardados a través de cada generación, cada familia, cada
provincia y cada ciudad; y que estos días de Purim no desaparecieran de
entre los judíos, ni su memoria pereciera de su descendencia.

\bibleverse{29} Entonces la reina Ester, hija de Abihail, y el judío
Mardoqueo escribieron con toda autoridad para confirmar esta segunda
carta de Purim. \bibleverse{30} Envió cartas a todos los judíos de las
ciento veintisiete provincias del reino de Asuero con palabras de paz y
de verdad, \bibleverse{31} para confirmar estos días de Purim en sus
tiempos señalados, como habían decretado Mardoqueo el judío y la reina
Ester, y como se habían impuesto a sí mismos y a sus descendientes en
materia de ayunos y de luto. \bibleverse{32} El mandamiento de Ester
confirmó estos asuntos de Purim; y fue escrito en el libro.

\hypertarget{posiciuxf3n-de-poder-y-servicios-de-mardochai-para-el-bienestar-de-los-juduxedos}{%
\subsection{Posición de poder y servicios de Mardochai para el bienestar
de los
judíos}\label{posiciuxf3n-de-poder-y-servicios-de-mardochai-para-el-bienestar-de-los-juduxedos}}

\hypertarget{section-9}{%
\section{10}\label{section-9}}

\bibleverse{1} El rey Asuero impuso un tributo en la tierra y en las
islas del mar. \bibleverse{2} ¿No están escritos en el libro de las
crónicas de los reyes de Media y de Persia todos los actos de su poder y
de su fuerza, y la relación completa de la grandeza de Mardoqueo, a la
que el rey le hizo avanzar? \footnote{\textbf{10:2} Est 8,2; Est 8,15}
\bibleverse{3} Porque Mardoqueo, el judío, estaba junto al rey Asuero, y
era grande entre los judíos y aceptado por la multitud de sus hermanos,
buscando el bien de su pueblo y hablando de paz a todos sus
descendientes.
