\hypertarget{section}{%
\section{1}\label{section}}

\bibleverse{1} Como voy a demostrar una proposición de lo más
filosófica, a saber, que el razonamiento religioso es dueño absoluto de
las emociones. De buen grado te aconsejo que prestes la máxima atención
a la filosofía. \bibleverse{2} Pues la razón es necesaria para todos
como paso a la ciencia. Además, abarca el elogio del autocontrol, la más
alta virtud. \bibleverse{3} Así pues, si la razón parece dominar los
afectos que se oponen a la templanza, como la gula y la lujuria,
\bibleverse{4} seguramente también domina de manera manifiesta los
afectos contrarios a la justicia, como la malicia, y los que impiden el
valor, como la ira, el dolor y el miedo. \bibleverse{5} Tal vez algunos
se pregunten: ``¿Cómo es, entonces, que la razón, si gobierna los
afectos, no es también dueña del olvido y la ignorancia?'' Intentan un
argumento ridículo. \bibleverse{6} Pues el razonamiento no gobierna sus
propias emociones, sino las que son contrarias a la justicia, el valor,
la templanza y el autocontrol; y, sin embargo, sobre éstas, de modo que
las resiste, sin destruirlas.

\bibleverse{7} Podría demostrarte por muchas otras consideraciones, que
el razonamiento religioso es el único dueño de las emociones;
\bibleverse{8} pero lo demostraré con la mayor fuerza por la fortaleza
de Eleazar, y siete parientes, y su madre, que sufrieron la muerte en
defensa de la virtud. \bibleverse{9} Pues todos ellos, tratando los
dolores con desprecio hasta la muerte, con este desprecio, demostraron
que el razonamiento tiene dominio sobre las emociones. \bibleverse{10}
Por sus virtudes, pues, es justo que elogie a los hombres que murieron
con su madre en este momento en nombre de la nobleza y la bondad; y por
sus honores, que los tenga por bienaventurados. \bibleverse{11} Porque
ellos, ganando la admiración no sólo de los hombres en general, sino
incluso de los perseguidores, por su valor y resistencia, se
convirtieron en el medio de la destrucción de la tiranía contra su
nación, habiendo vencido al tirano por su resistencia, de modo que por
ellos su país fue purificado. \bibleverse{12} Pero ahora podemos entrar
de inmediato en la cuestión, habiendo comenzado, como es nuestra
costumbre, con la exposición de la doctrina, y así proceder a la cuenta
de estas personas, dando gloria al Dios omnisciente.

\bibleverse{13} Por lo tanto, la pregunta es si el razonamiento es dueño
absoluto de las emociones. \bibleverse{14} Determinemos, pues, qué es el
razonamiento y qué es la emoción, y cuántas formas de emoción hay, y si
el razonamiento domina a todas ellas. \bibleverse{15} El razonamiento es
el intelecto acompañado de una vida de rectitud, poniendo por delante la
consideración de la sabiduría. \bibleverse{16} La sabiduría es el
conocimiento de las cosas divinas y humanas y de sus causas.
\bibleverse{17} Está contenida en la educación de la ley, por medio de
la cual aprendemos reverentemente las cosas divinas y provechosamente
las humanas. \bibleverse{18} Las formas de la sabiduría son el
autocontrol, la justicia, el valor y la templanza. \bibleverse{19} La
principal de ellas es el autodominio, por cuyo medio, en efecto, es que
el razonamiento gobierna sobre las emociones. \bibleverse{20} De las
emociones, el placer y el dolor son las dos más amplias; y también por
naturaleza se refieren al alma. \bibleverse{21} Al placer y al dolor les
acompañan muchos afectos. \bibleverse{22} Antes del placer está la
lujuria; y después del placer, la alegría. \bibleverse{23} Antes del
dolor está el miedo; y después del dolor, la tristeza. \bibleverse{24}
La ira es un afecto, común al placer y al dolor, si alguien presta
atención cuando le sobreviene. \bibleverse{25} En el placer existe una
disposición maliciosa, que es el más complejo de todos los afectos.
\bibleverse{26} En el alma, es la arrogancia, el amor al dinero, la sed
de honores, la contienda, la falta de fe y el mal de ojo.
\bibleverse{27} En el cuerpo, es la codicia, la alimentación
indiscriminada y la gula solitaria.

\bibleverse{28} Así como el placer y el dolor son, por lo tanto, dos
crecimientos del cuerpo y del alma, hay muchos retoños de estas
emociones. \bibleverse{29} La razón, el agricultor universal, purgando y
podando cada uno de ellos, atando, regando y trasplantando, mejora en
todo sentido los materiales de la moral y los afectos. \bibleverse{30}
Porque el razonamiento es el líder de las virtudes, pero es el único
gobernante de las emociones.

Obsérvese, pues, en primer lugar, a través de las mismas cosas que se
interponen en el camino de la templanza, que el raciocinio es el
dominador absoluto de las emociones. \bibleverse{31} Ahora bien, la
templanza consiste en el dominio de los deseos. \bibleverse{32} Pero de
las concupiscencias, unas pertenecen al alma y otras al cuerpo. El
razonamiento parece gobernar a ambos. \bibleverse{33} De lo contrario,
¿cómo es que cuando se nos incita a las carnes prohibidas, rechazamos la
gratificación que de ellas se derivaría? ¿No es porque el razonamiento
es capaz de ordenar los apetitos? Yo creo que sí. \bibleverse{34} De
ahí, pues, que al apetecer mariscos, aves, cuadrúpedos y toda clase de
alimentos que nos están prohibidos por la ley, nos retenemos mediante el
dominio del razonamiento. \bibleverse{35} Porque los afectos de nuestros
apetitos son resistidos por el entendimiento templado, y se retraen, y
todos los impulsos del cuerpo son refrenados por el razonamiento.

\hypertarget{section-1}{%
\section{2}\label{section-1}}

\bibleverse{1} ¿Es de extrañar? Si las lujurias del alma, después de la
participación con lo que es bello, se frustran, \bibleverse{2} por este
motivo, el templado José es alabado en que por el razonamiento, sometió,
al reflexionar, la indulgencia de los sentidos. \bibleverse{3} Pues,
aunque era joven y estaba maduro para las relaciones sexuales, anuló
mediante el razonamiento el estímulo de sus emociones. \bibleverse{4} No
es sólo el estímulo de la indulgencia sensual, sino el de todo deseo, lo
que el razonamiento es capaz de dominar. \bibleverse{5} Por ejemplo, la
ley dice: ``No codiciarás la mujer de tu prójimo, ni nada que sea de tu
prójimo''. \bibleverse{6} Ahora bien, puesto que es la ley la que nos ha
prohibido desear, con mucha mayor facilidad te persuadiré de que el
razonamiento es capaz de gobernar nuestras concupiscencias, al igual que
lo hace con los afectos que son impedimentos para la justicia.
\bibleverse{7} Ya que, ¿de qué manera se puede reclamar a un comedor
solitario, a un glotón y a un borracho, si no es evidente que el
razonamiento es el señor de las emociones? \bibleverse{8} Por lo tanto,
el hombre que regula su conducta por la ley, aunque sea amante del
dinero, presiona inmediatamente su propia disposición prestando al
necesitado sin intereses y cancelando la deuda al séptimo año.
\bibleverse{9} Si un hombre es avaro, se rige por la ley actuando por
medio del razonamiento, para que no espigue sus cosechas ni su vendimia.
En referencia a otros puntos podemos percibir que es el razonamiento el
que vence sus emociones. \bibleverse{10} Pues la ley vence incluso el
afecto hacia los padres, no renunciando a la virtud por ellos.
\bibleverse{11} Se impone sobre el amor a la esposa, reprendiéndola
cuando infringe la ley. \bibleverse{12} Se enseñorea del amor de los
padres hacia sus hijos, pues los castiga por el vicio. Se enseñorea de
la intimidad de los amigos, reprendiéndolos cuando son malvados.
\bibleverse{13} No creas que es una afirmación extraña que el
razonamiento pueda, en nombre de la ley, vencer incluso la enemistad.
\bibleverse{14} No permite cortar los árboles frutales de un enemigo,
sino que los preserva de los destructores y recoge sus ruinas caídas.
\bibleverse{15} La razón parece ser dueña de las emociones más
violentas, como el amor al imperio, la jactancia vacía y la calumnia.
\bibleverse{16} Pues el entendimiento templado repele todas estas
emociones malignas, como lo hace con la ira; pues domina incluso ésta.
\bibleverse{17} Así Moisés, cuando se enojó contra Datán y Abiram, no
les hizo nada con ira, sino que reguló su ira con el razonamiento.
\bibleverse{18} Porque la mente templada es capaz, como he dicho, de ser
superior a las emociones, y de corregir unas y destruir otras.
\bibleverse{19} Pues, ¿por qué, si no, nuestro sapientísimo padre Jacob
culpó a Simeón y a Leví de haber matado irracionalmente a toda la raza
de los siquemitas, diciendo: ``¡Maldita sea su ira!''? \bibleverse{20}
Porque si el razonamiento no poseyera el poder de dominar los afectos
coléricos, no habría dicho esto. \bibleverse{21} Porque en el momento en
que Dios creó al hombre, implantó en él sus emociones y su naturaleza
moral. \bibleverse{22} En ese momento entronizó la mente por encima de
todo como el líder sagrado, a través del medio de los sentidos.
\bibleverse{23} Le dio una ley a esta mente, viviendo según la cual
mantendrá un reino templado, justo, bueno y valiente. \bibleverse{24}
¿Cómo, entonces, puede decir un hombre, si el razonamiento es dueño de
las emociones, no tiene control sobre el olvido y la ignorancia?

\hypertarget{section-2}{%
\section{3}\label{section-2}}

\bibleverse{1} El argumento es sumamente ridículo, pues el razonamiento
no parece gobernar sobre sus propios afectos, sino sobre los del cuerpo,
\bibleverse{2} de tal manera que cualquiera de vosotros puede no ser
capaz de desarraigar el deseo, pero el razonamiento os permitirá evitar
ser esclavos de él. \bibleverse{3} Uno puede no ser capaz de desarraigar
la ira del alma, pero es posible soportar la ira. \bibleverse{4} Puede
que uno no sea capaz de erradicar la malicia, pero el razonamiento tiene
fuerza para trabajar con vosotros y evitar que cedáis a la malicia.
\bibleverse{5} Porque el razonamiento no es un erradicador, sino un
antagonista de las emociones.

\bibleverse{6} Esto puede comprenderse más claramente por la sed del rey
David. \bibleverse{7} Pues después de que David estuvo atacando a los
filisteos durante todo el día, él y los soldados de su nación mataron a
muchos de ellos; \bibleverse{8} luego, cuando llegó la noche, sudando y
muy cansado, llegó a la tienda real, alrededor de la cual estaba
acampado todo el ejército de nuestros antepasados. \bibleverse{9} Ahora
bien, todos los demás estaban cenando; \bibleverse{10} pero el rey,
teniendo mucha sed, aunque tenía numerosos manantiales, no pudo por sus
medios saciar su sed; \bibleverse{11} sino que un cierto anhelo
irracional por el agua en el campamento del enemigo se hizo más fuerte y
más feroz sobre él, lo deshizo y lo consumió. \bibleverse{12} Por lo
tanto, sus guardaespaldas se inquietaron ante este anhelo del rey, y dos
jóvenes y valientes soldados, respetando el deseo del rey, se armaron
completamente y, tomando un cántaro, saltaron las murallas de los
enemigos. \bibleverse{13} Sin ser percibidos por los guardianes de la
puerta, recorrieron todo el campamento de los enemigos en su búsqueda.
\bibleverse{14} Habiendo descubierto audazmente la fuente, llenaron de
ella la bebida para el rey. \bibleverse{15} Pero éste, aunque muerto de
sed, razonó que una bebida considerada de igual valor que la sangre
sería terriblemente peligrosa para su alma. \bibleverse{16} Por eso,
oponiendo el razonamiento a su deseo, derramó la bebida para Dios.
\bibleverse{17} Porque la mente templada tiene poder para vencer la
presión de las emociones, para apagar los fuegos de la excitación,
\bibleverse{18} y para luchar contra los dolores del cuerpo, por
excesivos que sean, y por la excelencia del razonamiento, para rechazar
todos los asaltos de las emociones. \bibleverse{19} Pero la ocasión nos
invita ahora a dar un ejemplo de razonamiento templado a partir de la
historia. \bibleverse{20} Porque en un tiempo en que nuestros padres
gozaban de una paz imperturbable por la obediencia a la ley y eran
prósperos, de modo que Seleuco Nicanor, el rey de Asia, les asignaba
dinero para el servicio divino y aceptaba su forma de gobierno,
\bibleverse{21} entonces ciertas personas, introduciendo cosas nuevas
contrarias a la armonía pública, cayeron de diversas maneras en
calamidades.

\hypertarget{section-3}{%
\section{4}\label{section-3}}

\bibleverse{1} Por cierto hombre llamado Simón, que se oponía a un
hombre honorable y bueno que en otro tiempo ostentaba el sumo sacerdocio
vitalicio, llamado Onías. Después de calumniar a Onías en todos los
sentidos, Simón no pudo perjudicarle con el pueblo, así que se marchó
como exiliado, con la intención de traicionar a su país. \bibleverse{2}
Al llegar a Apolonio, el gobernador militar de Siria, Fenicia y Cilicia,
le dijo: \bibleverse{3} ``Teniendo buena voluntad para los asuntos del
rey, he venido a informarte de que en los tesoros de Jerusalén están
depositadas decenas de miles de riquezas privadas que no pertenecen al
templo, sino al rey Seleuco.'' \bibleverse{4} Apolonio, enterado de los
detalles de esto, alabó a Simón por su cuidado de los intereses del rey,
y subiendo a Seleuco le informó del tesoro. \bibleverse{5} Obteniendo
autoridad al respecto, y avanzando rápidamente hacia nuestro país con el
maldito Simón y una fuerza muy pesada, \bibleverse{6} dijo que venía con
las órdenes del rey de que tomara el dinero privado del tesoro.
\bibleverse{7} La nación, indignada por esta proclama, y respondiendo
que era sumamente injusto que se privara de los depósitos del tesoro
sagrado a quienes los habían comprometido, se resistió como pudo.
\bibleverse{8} Pero Apolonio se fue con amenazas al templo.
\bibleverse{9} Los sacerdotes, con las mujeres y los niños, pidieron a
Dios que arrojara su escudo sobre el lugar sagrado y despreciado,
\bibleverse{10} y Apolonio subía con su fuerza armada para apoderarse
del tesoro, cuando aparecieron ángeles del cielo montados a caballo,
todos radiantes de armadura, que los llenaron de mucho temor y temblor.
\bibleverse{11} Apolonio cayó medio muerto en el patio abierto a todas
las naciones, extendió las manos al cielo e imploró con lágrimas a los
hebreos que rezaran por él y le quitaran la ira del ejército celestial.
\bibleverse{12} Porque decía que había pecado, para ser consecuentemente
digno de la muerte, y que si se salvaba, anunciaría a todos los pueblos
la bendición del lugar santo. \bibleverse{13} El sumo sacerdote Onías,
inducido por estas palabras, aunque por otras razones ansioso de que el
rey Seleuco no supusiera que Apolonio había sido asesinado por un
artificio humano y no por un castigo divino, rogó por él;
\bibleverse{14} y siendo así salvado inesperadamente, partió para
informar al rey de lo que le había sucedido. \bibleverse{15} Pero a la
muerte del rey Seleuco, su hijo Antíoco Epífanes le sucedió en el reino,
un hombre terrible de orgullo arrogante.

\bibleverse{16} El, habiendo depuesto a Onías del sumo sacerdocio,
nombró a su hermano Jasón como sumo sacerdote, \bibleverse{17} quien
había hecho un pacto, si le daba esta autoridad, de pagar anualmente
tres mil seiscientos sesenta talentos. \bibleverse{18} Le encomendó el
sumo sacerdocio y el gobierno de la nación. \bibleverse{19} Cambió la
manera de vivir del pueblo y pervirtió sus costumbres civiles hasta
convertirlas en toda una anarquía. \bibleverse{20} De modo que no sólo
erigió un gimnasio en la misma ciudadela de nuestro país, sino que
descuidó la custodia del templo. \bibleverse{21} A causa de ello, la
venganza divina se enfureció e instigó al propio Antíoco contra ellos.
\bibleverse{22} Pues estando en guerra con Ptolomeo en Egipto, oyó que
al difundirse la noticia de su muerte, los habitantes de Jerusalén se
habían alegrado mucho, y rápidamente marchó contra ellos.
\bibleverse{23} Después de someterlos, estableció un decreto según el
cual si alguno de ellos vivía según las leyes ancestrales, debía morir.
\bibleverse{24} Cuando no pudo destruir con sus decretos la obediencia a
la ley de la nación, sino que vio que todas sus amenazas y castigos no
surtían efecto, \bibleverse{25} pues incluso las mujeres, por seguir
circuncidando a sus hijos, fueron arrojadas a un precipicio junto con
ellos, sabiendo de antemano el castigo. \bibleverse{26} Por lo tanto,
cuando sus decretos fueron desatendidos por el pueblo, él mismo obligó
por medio de torturas a cada uno de esta raza, probando carnes
prohibidas, a renunciar a la religión judía.

\hypertarget{section-4}{%
\section{5}\label{section-4}}

\bibleverse{1} El tirano Antíoco, por lo tanto, sentado en público con
sus asesores en un lugar elevado, con sus tropas armadas de pie en un
círculo alrededor de él, \bibleverse{2} ordenó a sus lanzas que
agarraran a cada uno de los hebreos, y que los obligaran a probar carne
de cerdo y cosas ofrecidas a los ídolos. \bibleverse{3} Si alguno de
ellos no estaba dispuesto a comer el alimento maldito, debía ser
torturado en la rueda y así ser asesinado. \bibleverse{4} Cuando muchos
habían sido apresados, se acercó a él un hombre principal de la
asamblea, un hebreo, de nombre Eleazar, sacerdote de familia, de
profesión abogado y de edad avanzada, por lo que era conocido por muchos
de los seguidores del rey.

\bibleverse{5} Antíoco, al verlo, dijo: \bibleverse{6} ``Quisiera
aconsejarte, anciano, antes de que comiencen tus torturas, que pruebes
la carne de cerdo y salves tu vida; pues siento respeto por tu edad y tu
cabeza canosa, que desde hace tanto tiempo me parece que no eres
filósofo al conservar la superstición de los judíos. \bibleverse{7} Por
tanto, ya que la naturaleza te ha conferido la carne más excelente de
este animal, ¿la aborreces? \bibleverse{8} Parece insensato no disfrutar
de lo que es placentero, pero no vergonzoso; y por nociones de
pecaminosidad, rechazar los dones de la naturaleza. \bibleverse{9} Creo
que actuarás de manera aún más insensata, si sigues vanas concepciones
sobre la verdad. \bibleverse{10} Además, me estarás despreciando para tu
propio castigo. \bibleverse{11} ¿No despertarás de tu insignificante
filosofía, abandonarás la locura de tus ideas y, recuperando un
entendimiento digno de tu edad, investigarás la verdad de un curso
conveniente? \bibleverse{12} ¿No respetarás mi amable advertencia y te
apiadarás de tus propios años? \bibleverse{13} Porque tened en cuenta
que si hay algún poder que vigila esta religión vuestra, os perdonará
todas las transgresiones de la ley que cometáis por obligación.''

\bibleverse{14} Mientras el tirano lo incitaba de esta manera a comer
carne ilegalmente, Eleazar pidió permiso para hablar. \bibleverse{15}
Una vez obtenido el permiso para hablar, comenzó a dirigirse al pueblo
de la siguiente manera \bibleverse{16} ``Nosotros, oh Antíoco, que
estamos persuadidos de que vivimos bajo una ley divina, consideramos que
ninguna coacción es tan forzosa como la obediencia a esa ley.
\bibleverse{17} Por lo tanto, consideramos que no debemos transgredir la
ley de ninguna manera. \bibleverse{18} En efecto, si nuestra ley (como
supones) no fuera verdaderamente divina, y si la consideramos
erróneamente como divina, no tendríamos derecho ni siquiera en ese caso
a destruir nuestro sentido de la religión. \bibleverse{19} No pienses
que comer carne impura es una ofensa insignificante. \bibleverse{20}
Porque la transgresión de la ley, ya sea en lo pequeño o en lo grande,
es de igual importancia; \bibleverse{21} pues en cualquiera de los dos
casos la ley es igualmente menospreciada. \bibleverse{22} Pero vosotros
os burláis de nuestra filosofía, como si viviéramos en ella
irracionalmente. \bibleverse{23} Sin embargo, nos instruye en el
autocontrol, para que seamos superiores a todos los placeres y lujurias;
y nos entrena en el valor, para que suframos alegremente todo agravio.
\bibleverse{24} Nos instruye en la justicia, para que en todos nuestros
tratos demos lo que es debido. Nos enseña la piedad, para que adoremos
debidamente al único Dios. \bibleverse{25} Por eso no comemos lo
inmundo; porque creyendo que la ley fue establecida por Dios, estamos
convencidos de que el Creador del mundo, al dar sus leyes, se compadece
de nuestra naturaleza. \bibleverse{26} Nos ha ordenado comer lo que
conviene a nuestra alma, pero nos ha prohibido lo que no conviene.
\bibleverse{27} Pero, como un tirano, no sólo nos obligas a infringir la
ley, sino también a comer, para ridiculizarnos mientras comemos
profanamente. \bibleverse{28} Pero tú no tendrás este motivo de risa
contra mí, \bibleverse{29} ni transgrediré los sagrados juramentos de
mis antepasados de cumplir la ley. \bibleverse{30} No, aunque me saques
los ojos y consumas mis entrañas. \bibleverse{31} No soy tan viejo y
vacío de valor como para no ser joven en la razón y en la defensa de mi
religión. \bibleverse{32} Ahora, pues, preparad vuestras ruedas, y
encended una llama más feroz. \bibleverse{33} No me compadeceré tanto de
mi vejez como para quebrantar por mi causa la ley de mi patria.
\bibleverse{34} No te engañaré, oh ley, mi instructor, ni te abandonaré,
oh amado autocontrol. \bibleverse{35} No te avergonzaré, oh filósofo
Razón, ni te negaré, oh honrado sacerdocio y conocimiento de la ley.
\bibleverse{36} ¡Boca! No contaminarás mi vejez, ni la plena estatura de
una vida perfecta. \bibleverse{37} Mis antepasados me recibirán como
puro, sin haber temido tu coacción, incluso hasta la muerte.
\bibleverse{38} Porque gobernarás como un tirano a los impíos, pero no
te enseñorearás de mis pensamientos sobre la religión, ni con tus
argumentos ni con los hechos.''

\hypertarget{section-5}{%
\section{6}\label{section-5}}

\bibleverse{1} Cuando Eleazar hubo respondido de esta manera a las
exhortaciones del tirano, los lancero se acercaron y arrastraron
rudamente a Eleazar hacia los instrumentos de tortura. \bibleverse{2}
Primero desnudaron al anciano, adornado como estaba con la belleza de la
piedad. \bibleverse{3} Luego, atándole los brazos y las manos, lo
azotaron con desprecio. \bibleverse{4} Un heraldo de enfrente gritó:
``¡Obedezcan las órdenes del rey!''

\bibleverse{5} Pero el altivo y verdaderamente noble Eleazar, como quien
es torturado en un sueño, lo ignoró. \bibleverse{6} Pero alzando los
ojos a lo alto, hacia el cielo, la carne del anciano fue despojada por
los azotes, y su sangre corrió, y sus costados fueron atravesados.
\bibleverse{7} Cayendo al suelo por no tener su cuerpo fuerzas para
soportar los dolores, seguía manteniendo su razón erguida y sin
doblarse. \bibleverse{8} Entonces uno de los duros lancero se abalanzó
sobre él y comenzó a darle patadas en el costado para obligarle a
levantarse de nuevo después de haber caído. \bibleverse{9} Pero él
soportó los dolores, despreció la crueldad y perseveró en las
indignidades. \bibleverse{10} Como un noble atleta, el anciano, al ser
golpeado, venció a sus torturadores. \bibleverse{11} Con el rostro
sudoroso y jadeando, era admirado incluso por los torturadores por su
valor.

\bibleverse{12} Por eso, en parte por compasión de su vejez,
\bibleverse{13} en parte por la simpatía de los conocidos, y en parte
por admiración de su resistencia, algunos de los asistentes del rey
dijeron: \bibleverse{14} ``¿Por qué te destruyes sin razón, oh Eleazar,
con estas miserias? \bibleverse{15} Te traeremos algo de carne cocinada
por ti mismo, y podrás salvarte fingiendo que has comido carne de
cerdo.''

\bibleverse{16} Eleazar, como si el consejo lo torturara más
dolorosamente, gritó: \bibleverse{17} ``Que nosotros, que somos hijos de
Abraham, no seamos tan mal aconsejados como para ceder a hacer uso de
una pretensión impropia. \bibleverse{18} Porque sería irracional que,
habiendo vivido hasta la vejez con toda verdad, y habiendo guardado
escrupulosamente nuestro carácter para ello, nos volviéramos ahora
\bibleverse{19} y nos convirtiéramos nosotros mismos en un modelo de
impiedad para los jóvenes, como ejemplo de contaminación alimenticia.
\bibleverse{20} Sería vergonzoso que viviéramos poco tiempo, y que
fuéramos despreciados por todos los hombres por cobardía,
\bibleverse{21} y que fuéramos condenados por el tirano por cobardía al
no contender hasta la muerte por nuestra ley divina. \bibleverse{22} Por
tanto, vosotros, oh hijos de Abraham, morid noblemente por vuestra
religión. \bibleverse{23} Vosotros, lanzas del tirano, ¿por qué os
quedáis?''

\bibleverse{24} Viéndolo tan altivo contra la miseria, y sin cambiar por
su piedad, lo llevaron al fuego. \bibleverse{25} Luego, con sus malvados
instrumentos, lo quemaron en el fuego y le echaron fluidos hediondos en
las narices.

\bibleverse{26} Al final, calcinado hasta los huesos y a punto de
expirar, levantó los ojos hacia Dios y dijo: \bibleverse{27} ``Tú sabes,
oh Dios, que cuando podía haberme salvado, he sido asesinado por causa
de la ley con torturas de fuego. \bibleverse{28} Sé misericordioso con
tu pueblo, y satisfazte del castigo que recibo por su causa.
\bibleverse{29} Que mi sangre sea una purificación para ellos, y toma mi
vida a cambio de la suya''. \bibleverse{30} Hablando así, el santo varón
partió, noble en sus tormentos, y hasta las agonías de la muerte
resistió en sus razonamientos por causa de la ley. \bibleverse{31} Así
pues, el razonamiento religioso es dueño de las emociones.
\bibleverse{32} Pues si las emociones hubieran sido superiores al
razonamiento, les habría dado el testimonio de este dominio.
\bibleverse{33} Pero ahora, puesto que el razonamiento ha vencido a las
emociones, le concedemos con toda justicia la autoridad del primer
lugar. \bibleverse{34} Es justo que permitamos que el poder pertenezca
al razonamiento, ya que domina las miserias externas. \bibleverse{35}
Sería ridículo si no fuera así. Demuestro que el razonamiento no sólo
domina los dolores, sino que también es superior a los placeres y los
soporta.

\hypertarget{section-6}{%
\section{7}\label{section-6}}

\bibleverse{1} El raciocinio de nuestro padre Eleazar, como un piloto de
primer orden, dirigiendo la nave de la piedad en el mar de las
emociones, \bibleverse{2} y burlado por las amenazas del tirano, y
abrumado con las rompientes de la tortura, \bibleverse{3} no movió en
absoluto el timón de la piedad hasta que navegó hacia el puerto de la
victoria sobre la muerte. \bibleverse{4} Ninguna ciudad asediada ha
resistido jamás a muchas y diversas máquinas de guerra como lo hizo
aquel santo varón cuando su alma piadosa fue probada con la ardiente
prueba de las torturas y de los desgarros y conmovió a sus asediadores
por el razonamiento religioso que lo amparaba. \bibleverse{5} Pues el
padre Eleazar, proyectando su disposición, rompió las furiosas olas de
las emociones como con un acantilado saliente. \bibleverse{6} ¡Oh
sacerdote digno del sacerdocio! No contaminaste tus dientes sagrados, ni
hiciste partícipe de la profanidad a tu apetito, que siempre había
abrazado lo limpio y lícito. \bibleverse{7} ¡Oh, armonizador de la ley y
sabio consagrado a la vida divina! \bibleverse{8} De tal carácter deben
ser los que cumplen los deberes de la ley con riesgo de su propia
sangre, y la defienden con sudor generoso mediante sufrimientos hasta la
muerte. \bibleverse{9} Tú, padre, has establecido gloriosamente nuestro
recto gobierno con tu resistencia; y haciendo mucho caso de nuestro
pasado servicio, has impedido su destrucción, y con tus actos has hecho
creíbles las palabras de la filosofía. \bibleverse{10} ¡Oh anciano de
más poder que los suplicios, anciano más vigoroso que el fuego, mayor
rey sobre las emociones, Eleazar! \bibleverse{11} Porque así como el
padre Aarón, armado con un incensario, apresurándose a través del fuego
consumidor, venció al ángel portador de llamas, \bibleverse{12} así,
Eleazar, el descendiente de Aarón, consumido por el fuego, no abandonó
su razonamiento. \bibleverse{13} Lo más maravilloso es que, aunque era
un anciano, aunque los trabajos de su cuerpo estaban ya agotados, sus
músculos estaban relajados y sus tendones desgastados, recuperó la
juventud. \bibleverse{14} Con el espíritu de la razón, y el razonamiento
de Isaac, dejó sin poder al potro de tortura de muchas cabezas.
\bibleverse{15} Oh, bendita vejez, y reverente cabeza ronca, y vida
obediente a la ley, que el fiel sello de la muerte perfeccionó.
\bibleverse{16} Si, pues, un anciano, por medio de la religión,
despreció las torturas hasta la muerte, ciertamente el razonamiento
religioso es rector de las emociones. \bibleverse{17} Pero tal vez
algunos digan: ``No todos conquistan las emociones, como no todos poseen
un razonamiento sabio''. \bibleverse{18} Pero los que han meditado la
religión con todo su corazón, éstos son los únicos que pueden dominar
las emociones de la carne: \bibleverse{19} los que creen que para Dios
no mueren; pues, como nuestros antepasados, Abraham, Isaac y Jacob,
viven para Dios. \bibleverse{20} Esta circunstancia, pues, no es en
absoluto una objeción, que algunos que tienen un razonamiento débil se
rigen por sus emociones, \bibleverse{21} ya que ¿qué persona, caminando
religiosamente por toda la regla de la filosofía, y creyendo en Dios,
\bibleverse{22} y sabiendo que es una cosa bendita soportar todo tipo de
dificultades por la virtud, no dominaría, por el bien de la religión, su
emoción? \bibleverse{23} Pues sólo el hombre sabio y valiente es señor
de sus emociones. \bibleverse{24} Por eso, incluso los muchachos,
entrenados con la filosofía del razonamiento religioso, han vencido
torturas aún más amargas; \bibleverse{25} pues cuando el tirano fue
manifiestamente vencido en su primer intento, al no poder obligar al
anciano a comer la cosa impura,

\hypertarget{section-7}{%
\section{8}\label{section-7}}

\bibleverse{1} entonces, en efecto, vehementemente sacudido por la
emoción, ordenó traer a otros de los hebreos adultos, y si querían comer
de la cosa impura, dejarlos ir cuando hubiesen comido; pero si se
oponían, atormentarlos más gravemente. \bibleverse{2} Una vez que el
tirano dio esta orden, fueron llevados a su presencia siete parientes,
junto con su anciana madre. Eran guapos, modestos, bien nacidos, y en
general muy bien parecidos. \bibleverse{3} Cuando el tirano los vio
rodear a su madre como en una danza, se sintió complacido con ellos.
Impresionado por sus modales inocentes, les sonrió y, llamándoles, les
dijo: \bibleverse{4} ``Oh jóvenes, con sentimientos favorables, admiro
la belleza de cada uno de vosotros. Al honrar a un grupo tan numeroso de
parientes, no sólo os aconsejo que no compartáis la locura del anciano
que ha sido torturado antes, \bibleverse{5} sino que os ruego que os
rindáis y disfrutéis de mi amistad, pues poseo el poder, no sólo de
castigar a los que desobedecen mis órdenes, sino de hacer el bien a los
que las obedecen. \bibleverse{6} Confiad, pues, en mí y recibiréis
puestos de autoridad en mi gobierno, si abandonáis vuestro modo de vida
nacional, \bibleverse{7} y, ajustándoos al modo de vida griego, alteráis
vuestro gobierno y os deleitáis con las delicias de la juventud.
\bibleverse{8} Porque si me provocáis con vuestra desobediencia, me
obligaréis a destruir a cada uno de vosotros con terribles castigos
mediante torturas. \bibleverse{9} Tened, pues, piedad de vosotros
mismos, de los que yo, aunque soy enemigo, tengo compasión por vuestra
edad y vuestra atractiva apariencia. \bibleverse{10} ¿No considerarán
esto: que si desobedecen, no les quedará más que morir en la tortura?''

\bibleverse{11} Una vez dicho esto, ordenó que se presentaran los
instrumentos de tortura, para que el miedo les hiciera comer carne
impura. \bibleverse{12} Cuando el lancero hizo pasar las ruedas, los
bastidores, los ganchos, las rejillas, las calderas, las sartenes, los
dedos, las manos y las cuñas de hierro y los fuelles, el tirano continuó
\bibleverse{13} ``Temed, jóvenes, y la justicia a la que adoráis se
apiadará de vosotros, si os desviáis a causa de la coacción.''

\bibleverse{14} Ahora bien, habiendo escuchado estas palabras de
persuasión, y viendo los temibles instrumentos, no sólo no tuvieron
miedo, sino que incluso respondieron a los argumentos del tirano, y
mediante su buen razonamiento destruyeron su poder. \bibleverse{15}
Ahora consideremos el asunto. Si entre ellos hubiera habido alguno de
espíritu débil y cobarde, ¿qué razonamientos habrían empleado sino
estos? \bibleverse{16} ``¡Oh, miserables que somos, y sumamente
insensatos! Cuando el rey nos exhorta y nos llama a su bondad, ¿no
debemos obedecerle? \bibleverse{17} ¿Por qué nos animamos con vanos
consejos y nos aventuramos a una desobediencia que trae la muerte?
\bibleverse{18} ¿No hemos de temer, oh parientes, los instrumentos de
tortura y sopesar las amenazas de tormento y rehuir esta vana gloria y
este orgullo destructor? \bibleverse{19} Tengamos compasión de nuestra
edad y cedamos ante los años de nuestra madre. \bibleverse{20} Tengamos
en cuenta que moriremos como rebeldes. \bibleverse{21} La justicia
divina nos perdonará si tememos al rey por necesidad. \bibleverse{22}
¿Por qué retirarnos de una vida dulcísima y privarnos de este mundo
agradable? \bibleverse{23} No nos opongamos a la necesidad, ni busquemos
la vana gloria con nuestra propia tortura. \bibleverse{24} La misma ley
no nos condenaría a muerte arbitrariamente porque tememos la tortura.
\bibleverse{25} ¿Por qué ha arraigado en nosotros un celo tan furioso y
se ha aprobado una obstinación tan fatal, cuando podríamos vivir sin ser
molestados por el rey?''

\bibleverse{26} Pero los jóvenes no decían ni pensaban nada de esto
cuando iban a ser torturados. \bibleverse{27} Porque estaban bien
enterados de los sufrimientos y eran dueños de los dolores.
\bibleverse{28}-29 De modo que, en cuanto el tirano dejó de aconsejarles
que comieran lo impuro, todos a una voz, como de un mismo corazón,
dijeron

\hypertarget{section-8}{%
\section{9}\label{section-8}}

\bibleverse{1} ``¿Por qué te demoras, oh tirano? Porque estamos más
dispuestos a morir que a transgredir los mandatos de nuestros padres.
\bibleverse{2} Deshonraríamos a nuestros padres si no obedeciéramos la
ley y tomáramos el conocimiento como guía. \bibleverse{3} Oh tirano,
consejero de la transgresión de la ley, no te compadezcas de nosotros
como lo haces, más de lo que nosotros mismos nos compadecemos.
\bibleverse{4} Pues consideramos que tu huida es peor que la muerte.
\bibleverse{5} Tratas de asustarnos amenazándonos con la muerte por
torturas, como si no hubieras aprendido nada con la muerte de Eleazar.
\bibleverse{6} Pero si ancianos de los hebreos han muerto por la causa
de la religión después de soportar la tortura, con más razón deberíamos
morir nosotros, los más jóvenes, despreciando vuestras crueles torturas,
que nuestro anciano instructor superó. \bibleverse{7} Haz, pues, el
intento, oh tirano. Si nos condenas a muerte por nuestra religión, no
pienses que nos perjudicas al torturarnos. \bibleverse{8} Pues nosotros,
mediante estos malos tratos y esta resistencia, obtendremos las
recompensas de la virtud. \bibleverse{9} Pero tú, por la matanza inicua
y despótica de nosotros, soportarás, por la venganza divina, la tortura
eterna por el fuego''.

\bibleverse{10} Cuando dijeron esto, el tirano no sólo se exasperó
contra ellos por ser desobedientes, sino que se enfureció con ellos por
ser ingratos. \bibleverse{11} Así que, por orden suya, los torturadores
trajeron al más viejo de ellos y, rasgando su túnica, le ataron las
manos y los brazos a cada lado con correas. \bibleverse{12} Cuando se
esforzaron sin efecto en azotarlo, lo arrojaron sobre la rueda.
\bibleverse{13} El noble joven, extendido sobre ésta, se dislocó.
\bibleverse{14} Con todos los miembros desarticulados, denunció al
tirano, diciendo: \bibleverse{15} ``Oh tirano maldito, y enemigo de la
justicia celestial, y cruel de corazón, no soy un asesino, ni un
sacrílego, a quien torturas, sino un defensor de la ley divina.''

\bibleverse{16} Y cuando los lanceros le dijeron: ``Consiente en comer,
para que te liberes de tus torturas'', \bibleverse{17} él respondió:
``No es tan poderosa, oh lacayos malditos, vuestra rueda, como para
ahogar mi razonamiento. Cortad mis miembros, quemad mi carne y retorced
mis articulaciones. \bibleverse{18} Pues a través de todos mis tormentos
os convenceré de que los hijos de los hebreos son los únicos invictos en
nombre de la virtud.''

\bibleverse{19} Mientras decía esto, amontonaron combustible y,
prendiéndole fuego, lo tensaron aún más sobre la rueda. \bibleverse{20}
La rueda quedó manchada de sangre por todas partes. Las cenizas
calientes se apagaron con los excrementos de las vísceras, y los trozos
de carne quedaron esparcidos por los ejes de la máquina. \bibleverse{21}
Aunque el armazón de sus huesos estaba ahora destruido, el joven altivo
y abrahámico no gimió. \bibleverse{22} Sino que, como si fuera
transformado por el fuego en inmortalidad, soportó noblemente los
azotes, diciendo: \bibleverse{23} ``Imitadme, oh, raza. Nunca abandonéis
vuestro puesto, ni renunciéis a mi hermandad con valor. Combatid la
santa y honorable lucha de la religión, \bibleverse{24} por cuyo medio
nuestra justa y paternal Providencia, haciéndose misericordiosa con la
nación, castigará al pestilente tirano.'' \bibleverse{25} Diciendo esto,
el venerado joven cerró abruptamente su vida.

\bibleverse{26} Cuando todos admiraron su alma valerosa, los lanceros
sacaron al segundo más viejo, y habiéndole puesto guantes de hierro con
ganchos afilados, lo ataron al potro. \bibleverse{27} Al preguntarle si
quería comer antes de ser torturado, escucharon su noble sentir.
\bibleverse{28} Después de que con los guanteletes de hierro le
arrastraran violentamente toda la carne desde el cuello hasta la
barbilla, los animales, que parecían panteras, le arrancaron la misma
piel de la cabeza, pero él, soportando con firmeza esta miseria, dijo:
\bibleverse{29} ``¡Qué dulce es toda forma de muerte por la religión de
nuestros padres!'' Luego le dijo al tirano: \bibleverse{30} ``¿No crees,
el más cruel de todos los tiranos, que ahora te torturan más que a mí,
al ver que tu arrogante concepción de la tiranía ha sido vencida por
nuestra perseverancia en nombre de nuestra religión? \bibleverse{31}
Porque yo aligero mi sufrimiento con los placeres que están relacionados
con la virtud. \bibleverse{32} Pero a ti te torturan con amenazas por
impiedad. No escaparás, tirano corrupto, de la venganza de la ira
divina''.

\hypertarget{section-9}{%
\section{10}\label{section-9}}

\bibleverse{1} Ahora bien, éste soportó esta loable muerte. El tercero
fue traído y exhortado por muchos a probar y salvar su vida.
\bibleverse{2} Pero él gritó y dijo: ``¿No sabéis que el padre de los
que han muerto es también mi padre, y que la misma madre me dio a luz, y
que fui criado de la misma manera? \bibleverse{3} No renuncio a la noble
relación de mi parentela. \bibleverse{4} Ahora bien, cualquier
instrumento de venganza que tengáis, aplicadlo a mi cuerpo, pues no
podéis tocar mi alma, aunque queráis.'' \bibleverse{5} Pero ellos, muy
indignados por su audacia al hablar, le dislocaron las manos y los pies
con máquinas de estruendo y, arrancándolos de sus órbitas, lo
desmembraron. \bibleverse{6} Arrastraron sus dedos, sus brazos, sus
piernas y sus tobillos. \bibleverse{7} No pudiendo estrangularlo de
ninguna manera, le arrancaron la piel, junto con las puntas extremas de
los dedos, y luego lo arrastraron hasta la rueda, \bibleverse{8}
alrededor de la cual se soltaron sus articulaciones vertebrales, y vio
su propia carne hecha jirones, y chorros de sangre fluyendo de sus
entrañas. \bibleverse{9} Cuando estaba a punto de morir, dijo:
\bibleverse{10} ``Nosotros, oh tirano maldito, sufrimos esto por la
educación y la virtud divinas. \bibleverse{11} Pero tú, por tu impiedad
y derramamiento de sangre, sufrirás tormentos incesantes''.

\bibleverse{12} Así, habiendo muerto dignamente su parentela,
arrastraron al cuarto, diciendo: \bibleverse{13} ``No compartas la
locura de tu parentela, sino respeta al rey y sálvate.''

\bibleverse{14} Pero él les dijo: ``No tenéis un fuego tan abrasador
como para hacerme el cobarde. \bibleverse{15} Por la bendita muerte de
mi parentela, y el eterno castigo del tirano, y la gloriosa vida de los
piadosos, no repudiaré la noble hermandad. \bibleverse{16} Inventa, oh
tirano, torturas, para que aprendas, incluso a través de ellas, que soy
hermano de los atormentados antes.''

\bibleverse{17} Cuando hubo dicho esto, el sanguinario, asesino e impío
Antíoco ordenó que le cortaran la lengua. \bibleverse{18} Pero él dijo:
``Aunque me quiten el órgano de la palabra, Dios sigue escuchando a los
silenciosos. \bibleverse{19} He aquí que mi lengua está extendida,
córtala, pues a pesar de ello no silenciarás nuestros razonamientos.
\bibleverse{20} De buena gana perdemos nuestros miembros en nombre de
Dios. \bibleverse{21} Pero Dios te encontrará pronto, ya que cortaste la
lengua, instrumento de la melodía divina.''

\hypertarget{section-10}{%
\section{11}\label{section-10}}

\bibleverse{1} Cuando murió, desfigurado en sus tormentos, el quinto se
adelantó de un salto y dijo: \bibleverse{2} ``No pretendo, oh tirano,
excusarme del tormento que es en nombre de la virtud. \bibleverse{3}
Pero he venido por mi propia voluntad, para que con mi muerte debas la
venganza celestial y el castigo por más crímenes. \bibleverse{4} Oh, tú
que odias la virtud y a los hombres, ¿qué hemos hecho para que te
deleites así con nuestra sangre? \bibleverse{5} ¿Te parece mal que
adoremos al Fundador de todas las cosas y vivamos según su ley
superadora? \bibleverse{6} Pero esto es digno de honores, no de
tormentos, \bibleverse{7} si hubieras sido capaz de los sentimientos más
elevados de los hombres, y poseído la esperanza de la salvación de Dios.
\bibleverse{8} He aquí que ahora, siendo ajenos a Dios, hacéis la guerra
a los que son religiosos para con Dios.''

\bibleverse{9} Mientras decía esto, los lancero lo ataron y lo llevaron
al potro, \bibleverse{10} al cual atándolo por las rodillas, y
sujetándolas con grilletes de hierro, le doblaron los lomos sobre la
cuña de la rueda; y entonces su cuerpo fue desmembrado, a la manera de
un escorpión. \bibleverse{11} Con su aliento así confinado y su cuerpo
estrangulado, dijo: \bibleverse{12} ``Un gran favor nos concedes, oh
tirano, al permitirnos manifestar nuestra adhesión a la ley por medio de
sufrimientos más nobles.''

\bibleverse{13} Muerto también él, fue sacado el sexto, muy joven. Al
preguntarle el tirano si quería comer y ser liberado, dijo:
\bibleverse{14} ``Ciertamente soy más joven que mis hermanos, pero en
cuanto a entendimiento soy tan viejo; \bibleverse{15} por haber nacido y
crecido con el mismo fin. Estamos obligados a morir también por la misma
causa. \bibleverse{16} De modo que si os parece oportuno atormentarnos
por no comer lo impuro, ¡atormentadnos!''

\bibleverse{17} Mientras decía esto, lo llevaron a la rueda.
\bibleverse{18} Extendido sobre ésta, con los miembros desgarrados y
dislocados, lo asaron gradualmente desde abajo. \bibleverse{19} Habiendo
calentado afilados escupitajos, los acercaron a su espalda; y habiendo
traspasado sus costados, quemaron sus entrañas. \bibleverse{20} Él,
mientras estaba atormentado, dijo: ``Oh, buena y santa contienda, en la
que, por causa de la religión, hemos sido llamados a la arena del dolor,
y no hemos sido vencidos. \bibleverse{21} Pues el entendimiento
religioso, oh tirano, es invicto. \bibleverse{22} Armado con virtudes
rectas, yo también partiré con mi parentela. \bibleverse{23} Yo también,
llevando conmigo un gran vengador, oh inventor de torturas, y enemigo de
los verdaderamente piadosos. \bibleverse{24} Nosotros, seis jóvenes,
hemos destruido tu tiranía. \bibleverse{25} Pues tu incapacidad para
anular nuestro razonamiento y obligarnos a comer lo impuro, ¿no es tu
destrucción? \bibleverse{26} Tu fuego es frío para nosotros. Tus
bastidores son indoloros, y tu violencia inofensiva. \bibleverse{27}
Porque los guardianes, no de un tirano sino de una ley divina, son
nuestros defensores. Por esto mantenemos nuestra razón inconquistable''.

\hypertarget{section-11}{%
\section{12}\label{section-11}}

\bibleverse{1} Cuando también él sufrió el bendito martirio y murió en
la caldera en la que había sido arrojado, se presentó el séptimo, el más
joven de todos, \bibleverse{2} del que el tirano, compadecido, aunque
había sido terriblemente reprochado por su parentela, \bibleverse{3}
viéndolo ya rodeado de cadenas, lo hizo acercar y se esforzó en
aconsejarle, diciendo: \bibleverse{4} ``Ya ves el fin de la locura de tu
parentela, pues han muerto torturados por la desobediencia. Tú, si eres
desobediente, habiendo sido miserablemente atormentado, perecerás tú
mismo prematuramente. \bibleverse{5} Pero si obedeces, serás mi amigo y
tendrás a tu cargo los asuntos del reino''. \bibleverse{6} Después de
haberle exhortado así, mandó llamar a la madre del muchacho, para que,
mostrándole compasión por la pérdida de tantos hijos, la inclinara, por
la esperanza de seguridad, a hacer obedecer al superviviente.

\bibleverse{7} Él, después de que su madre le instara en lengua hebrea,
(como pronto relataremos) dijo: \bibleverse{8} ``Libérame para que pueda
hablar al rey y a todos sus amigos.'' \bibleverse{9} Ellos, regocijados
en extremo por la promesa del joven, lo soltaron rápidamente.
\bibleverse{10} Él, corriendo hacia las cacerolas, dijo: \bibleverse{11}
``Tirano impío, y hombre muy blasfemo, ¿no te has avergonzado, habiendo
recibido prosperidad y un reino de Dios, de matar a sus siervos y de
atormentar a los hacedores de la piedad? \bibleverse{12} Por eso la
venganza divina te reserva para el fuego y los tormentos eternos, que se
aferrarán a ti para siempre. \bibleverse{13} ¿No te avergüenza, hombre
como eres, pero muy salvaje, cortar la lengua a hombres de sentimientos
y origen semejantes, y habiendo abusado así de ellos torturarlos?
\bibleverse{14} Pero ellos, muriendo valientemente, cumplieron con su
religión hacia Dios. \bibleverse{15} Pero vosotros gemiréis como
merecéis por haber matado sin causa a los campeones de la virtud.
\bibleverse{16} Por eso --- continuó --- yo mismo, estando a punto de
morir, \bibleverse{17} no abandonaré a mi parentela. \bibleverse{18}
Invoco al Dios de mis padres para que sea misericordioso con mi raza.
\bibleverse{19} Pero a vosotros, vivos y muertos, os castigará''.
\bibleverse{20} Habiendo orado así, se arrojó a las ollas, y así expiró.

\hypertarget{section-12}{%
\section{13}\label{section-12}}

\bibleverse{1} Si, pues, los siete afines despreciaron los problemas
hasta la muerte, se admite por todas partes que la recta razón es dueña
absoluta de las emociones. \bibleverse{2} Pues igual que si hubieran
comido de lo impío como esclavos de las emociones, habríamos dicho que
habían sido vencidos por ellas. \bibleverse{3} Ahora no es así. Pero por
medio del razonamiento que es alabado por Dios, ellos dominaron sus
emociones. \bibleverse{4} Es imposible pasar por alto el liderazgo de la
reflexión, pues obtuvo la victoria tanto sobre las emociones como sobre
los problemas. \bibleverse{5} ¿Cómo, entonces, podemos evitar según
estos hombres el dominio de las emociones por medio del razonamiento
correcto, ya que no se apartaron de las penas del fuego? \bibleverse{6}
Porque así como por medio de las torres que se proyectan frente a los
puertos los hombres rompen las olas amenazantes, y así aseguran un curso
tranquilo a las naves que entran en el puerto, \bibleverse{7} así el
recto razonamiento de siete torres de los jóvenes, asegurando el puerto
de la religión, conquistó la tempestad de las emociones. \bibleverse{8}
Pues habiendo dispuesto un santo coro de piedad, se animaban unos a
otros, diciendo: \bibleverse{9} ``Hermanos, muramos fraternalmente por
la ley. Imitemos a los tres jóvenes de Asiria que despreciaron el horno
igualmente afligido. \bibleverse{10} No seamos cobardes en la
manifestación de la piedad''. \bibleverse{11} Uno dijo: ``¡Ánimo,
hermano!'' y otro: ``¡Resiste noblemente!'' \bibleverse{12} Otro dijo:
``Acuérdate de la estirpe que tienes'', y por la mano de nuestro padre
Isaac soportó ser muerto por causa de la piedad. \bibleverse{13} Unos y
otros, mirándose serenos y confiados, dijeron: ``Sacrifiquemos de todo
corazón nuestras almas a Dios, que las dio, y empleemos nuestros cuerpos
en el cumplimiento de la ley. \bibleverse{14} No temamos al que piensa
que mata; \bibleverse{15} porque grande es la prueba del alma y el
peligro del tormento eterno que les espera a los que transgreden el
mandamiento de Dios. \bibleverse{16} Armémonos, pues, en el dominio
propio, que es el razonamiento divino. \bibleverse{17} Si sufrimos así,
Abraham, Isaac y Jacob nos recibirán, y todos los padres nos elogiarán.
\bibleverse{18} Mientras cada uno de los parientes era arrastrado, el
resto exclamó: ``¡No nos deshonres, oh hermano, ni falsifiques a los que
murieron antes que tú!''

\bibleverse{19} Ahora bien, no ignoras el encanto de la hermandad, que
la divina y sapientísima Providencia ha impartido a través de los padres
a los hijos, y ha engendrado a través del vientre de la madre.
\bibleverse{20} En el que estos hermanos, habiendo permanecido un tiempo
igual, y habiendo sido formados durante el mismo período, y habiendo
sido aumentados por la misma sangre, y habiendo sido perfeccionados a
través del mismo principio de vida, \bibleverse{21} y habiendo sido
criados a intervalos iguales, y habiendo mamado leche de los mismos
manantiales, por lo que sus almas fraternales son criadas amorosamente
juntas, \bibleverse{22} y aumentan más poderosamente a causa de esta
crianza simultánea, y por la compañía diaria, y por otra educación, y el
ejercicio en la ley de Dios.

\bibleverse{23} Constituido así el amor fraternal, los siete parientes
tenían una armonía mutua más simpática. \bibleverse{24} Porque al ser
educados en la misma ley, y al practicar las mismas virtudes, y al ser
criados en un curso de vida justo, aumentaron esta armonía entre ellos.
\bibleverse{25} Porque el mismo ardor por lo que es justo y honorable
aumentó su buena voluntad y armonía entre ellos. \bibleverse{26} Pues
actuando junto con la religión, les hacía más deseable el sentimiento
fraternal. \bibleverse{27} Y, sin embargo, aunque la naturaleza, el
compañerismo y la moral virtuosa aumentaban su amor fraternal, los que
quedaban soportaban ver a sus parientes, que eran maltratados por su
religión, torturados incluso hasta la muerte.

\hypertarget{section-13}{%
\section{14}\label{section-13}}

\bibleverse{1} Más que esto, incluso los instaron a este maltrato; de
modo que no sólo despreciaron los dolores en sí mismos, sino que incluso
sacaron lo mejor de sus afectos de amor fraternal. \bibleverse{2} ¡El
razonamiento es más real que un rey, y más libre que los hombres libres!
\bibleverse{3} ¡Qué sagrado y armonioso concierto de los siete parientes
en cuanto a la piedad! \bibleverse{4} Ninguno de los siete jóvenes se
acobardó ni rehuyó la muerte. \bibleverse{5} Pero todos ellos, como si
corrieran el camino de la inmortalidad, se apresuraron a la muerte a
través de las torturas. \bibleverse{6} Pues así como las manos y los
pies se mueven con simpatía a las direcciones del alma, así esos santos
jóvenes aceptaron la muerte por la religión, como por el alma inmortal
de la religión. \bibleverse{7} ¡Oh, santos siete de armoniosa parentela!
Pues como los siete días de la creación, en torno a la religión,
\bibleverse{8} así los jóvenes, girando en torno al número siete,
anularon el miedo a los tormentos. \bibleverse{9} Ahora nos estremecemos
ante el relato de la aflicción de aquellos jóvenes; pero ellos no sólo
vieron y no sólo oyeron la ejecución inmediata de la amenaza, sino que,
sometiéndose a ella, perseveraron; y eso a través de las penas del
fuego. \bibleverse{10} ¿Qué puede ser más doloroso? Porque el poder del
fuego, siendo agudo y rápido, disolvió rápidamente sus cuerpos.
\bibleverse{11} No te parezca maravilloso que el raciocinio dominara a
aquellos hombres en sus tormentos, cuando incluso la mente de una mujer
desprecia dolores más múltiples. \bibleverse{12} Pues la madre de
aquellos siete jóvenes soportó los tormentos de cada uno de sus hijos.

\bibleverse{13} Considera cuán amplio es el amor a la prole, que atrae a
todos a la simpatía del afecto, \bibleverse{14} donde los animales
irracionales poseen una simpatía y amor por sus crías similar a la de
los hombres. \bibleverse{15} Los pájaros mansos que frecuentan los
tejados de nuestras casas defienden a sus polluelos. \bibleverse{16}
Otros construyen sus nidos, y empollan sus crías, en las cimas de las
montañas y en los precipicios de los valles, y en los huecos y las copas
de los árboles, y alejan al intruso. \bibleverse{17} Si no pueden hacer
esto, vuelan en círculos alrededor de ellos en agonía de afecto,
llamando en su propia nota, y salvan a sus crías de cualquier manera que
puedan. \bibleverse{18} Pero, ¿por qué hemos de llamar la atención sobre
la simpatía hacia los niños que muestran los animales irracionales?
\bibleverse{19} Incluso las abejas, en la época de la producción de
miel, atacan a todos los que se acercan, y atraviesan con su aguijón,
como con una espada, a los que se acercan a su colmena, y los repelen
hasta la muerte. \bibleverse{20} Pero la simpatía por sus hijos no
apartó a la madre de los jóvenes, que tenía un espíritu afín al de
Abraham.

\hypertarget{section-14}{%
\section{15}\label{section-14}}

\bibleverse{1} ¡Oh, razonamiento de los hijos, señor de las emociones, y
religión más deseable para una madre que los hijos! \bibleverse{2} La
madre, cuando se le presentaron dos cosas, la religión y la seguridad de
sus siete hijos por un tiempo, sobre la promesa condicional de un
tirano, \bibleverse{3} eligió más bien la religión que según Dios
preserva a la vida eterna. \bibleverse{4} ¡De qué manera puedo describir
éticamente el afecto de los padres hacia sus hijos, la semejanza de alma
y de forma impresa en el pequeño tipo de un niño de manera maravillosa,
especialmente por la mayor simpatía de las madres con los sentimientos
de los nacidos de ellas! \bibleverse{5} Pues por lo mucho que las madres
son por naturaleza débiles en disposición y prolíficas en descendencia,
por lo mucho que son más afectuosas con los hijos. \bibleverse{6} De
todas las madres, la más cariñosa con los hijos fue la madre de los
siete, que en siete partos había engendrado profundamente el amor hacia
ellos. \bibleverse{7} A causa de los muchos dolores sufridos en relación
con cada uno de ellos, se vio obligada a sentir simpatía por ellos;
\bibleverse{8} sin embargo, por temor a Dios, descuidó la salvación
temporal de sus hijos. \bibleverse{9} No sólo eso, sino que, debido a la
excelente disposición a la ley, su afecto maternal hacia ellos se
incrementó. \bibleverse{10} Porque ambos eran justos y templados, y
valientes, de gran altura de miras, y querían tanto a sus parientes, que
hasta la muerte la obedecían observando la ley.

\bibleverse{11} Sin embargo, aunque había tantas circunstancias
relacionadas con el amor a los hijos para atraer a una madre a la
simpatía, en el caso de ninguno de ellos las diversas torturas fueron
capaces de pervertir su principio. \bibleverse{12} Pero ella inclinó a
cada uno por separado y a todos juntos a la muerte por la religión.
\bibleverse{13} ¡Oh, naturaleza santa y sentimiento paternal, y
recompensa de educar a los hijos, y afecto maternal inconquistable!
\bibleverse{14} A la hora de atormentar y asar a cada uno de ellos, la
madre observadora se vio impedida por la religión a cambiar.
\bibleverse{15} Ella vio cómo la carne de sus hijos se disolvía
alrededor del fuego, y sus extremidades se estremecían en el suelo, y la
carne de sus cabezas caía hacia adelante hasta sus barbas, como si
fueran máscaras.

\bibleverse{16} ¡Oh tú, madre, que fuiste probada en este momento con
dolores más amargos que los del nacimiento! \bibleverse{17} ¡Oh tú,
única mujer que has producido la santidad perfecta! \bibleverse{18} Tu
primogénito, expirando, no te convirtió, ni el segundo, mirando
miserablemente en sus tormentos, ni el tercero, exhalando su alma.
\bibleverse{19} ¡No lloraste cuando viste los ojos de cada uno de ellos
mirando con severidad sus torturas, y sus fosas nasales presagiando la
muerte! \bibleverse{20} Cuando viste la carne de los niños amontonada
sobre la carne de los niños arrancada, las cabezas decapitadas sobre las
cabezas, los muertos cayendo sobre los muertos, y un coro de niños
convertido por la tortura en un cementerio, no te lamentaste.
\bibleverse{21} ¡No así las melodías de las sirenas o los cantos de los
cisnes atraen a los oyentes a escuchar, oh voces de niños que llaman a
su madre en medio de los tormentos! \bibleverse{22} ¡Con qué y qué clase
de tormentos fue torturada la propia madre, mientras sus hijos se
sometían a la rueda y a los fuegos! \bibleverse{23} Pero el razonamiento
religioso, habiendo fortalecido su valor en medio de los sufrimientos,
le permitió renunciar, por el momento, al amor paterno.

\bibleverse{24} Aunque viendo la destrucción de siete hijos, la noble
madre, después de un abrazo, se despojó de sus sentimientos por la fe en
Dios. \bibleverse{25} Pues como en una sala de consejo, viendo en su
propia alma a los consejeros vehementes, a la naturaleza y a la
filiación y al amor de sus hijos, y al atropello de sus hijos,
\bibleverse{26} teniendo dos votos, uno para la muerte, el otro para la
preservación de sus hijos, \bibleverse{27} no se inclinó por el que
hubiera salvado a sus hijos por la seguridad de un breve espacio.
\bibleverse{28} Pero esta hija de Abraham se acordó de su santa
fortaleza.

\bibleverse{29} ¡Oh, santa madre de una nación, vengadora de la ley,
defensora de la religión y primera portadora en la batalla de los
afectos! \bibleverse{30} ¡Oh tú, más noble en la resistencia que los
varones, y más valiente que los hombres en la perseverancia!
\bibleverse{31} Pues como la nave de Noé, que llevaba el mundo en el
diluvio que lo llenaba, aguantó contra las olas, \bibleverse{32} así tú,
guardiana de la ley, cuando estabas rodeada por todas partes por el
diluvio de las emociones, y asaltada por violentas tormentas que eran
los tormentos de tus hijos, aguantaste noblemente contra las tormentas
contra la religión.

\hypertarget{section-15}{%
\section{16}\label{section-15}}

\bibleverse{1} Si, pues, incluso una mujer, y eso que era anciana y
madre de siete hijos, soportó ver los tormentos de sus hijos hasta la
muerte, hay que admitir que la razón religiosa es dueña incluso de las
emociones. \bibleverse{2} He demostrado, pues, que no sólo los hombres
han obtenido el dominio de sus emociones, sino también que una mujer
despreció los mayores tormentos. \bibleverse{3} Los leones que rodeaban
a Daniel no eran tan feroces, ni el horno de Misael ardía con los fuegos
más vehementes, como el amor natural a los hijos que ardía en ella,
cuando vio torturar a sus siete hijos. \bibleverse{4} Pero con el
razonamiento de la religión la madre apagó emociones tan grandes y
poderosas. \bibleverse{5} Porque debemos considerar también esto: que,
si la mujer hubiera tenido el corazón débil, por ser su madre, se habría
lamentado por ellos, y tal vez habría hablado así: \bibleverse{6} ``¡Ah!
soy desgraciada y muchas veces miserable, que habiendo nacido siete
hijos, no he llegado a ser madre de ninguno. \bibleverse{7} Oh, siete
partos inútiles, y siete períodos de parto sin provecho, y mamadas
infructuosas, y amamantamientos miserables. \bibleverse{8} En vano, por
vosotros, oh hijos, he soportado muchos dolores y las más difíciles
angustias de la crianza. \bibleverse{9} Ay, de mis hijos, algunos de
vosotros solteros, y otros que se han casado sin provecho, no veré a
vuestros hijos, ni tendré la alegría de ser abuela. \bibleverse{10} ¡Ah,
que yo, que tuve muchos y hermosos hijos, sea una viuda solitaria y
llena de penas! \bibleverse{11} Ni, si muero, tendré un hijo que me
entierre''. Pero con semejante lamento, la santa y temerosa madre no
lloró por ninguno de ellos. \bibleverse{12} Ni apartó a ninguno de ellos
de la muerte, ni se afligió por ellos como por los muertos.
\bibleverse{13} Sino que, como poseída de una mente firme, y como quien
vuelve a dar a luz a todos sus hijos a la inmortalidad, más bien los
instó a la muerte en nombre de la religión. \bibleverse{14} Oh, mujer,
soldado de Dios por la religión, tú, anciana y mujer, has vencido por
medio de la resistencia incluso a un tirano; y aunque débil, has sido
encontrada más poderosa en los hechos y en las palabras. \bibleverse{15}
Pues cuando fuiste apresada junto con tus hijos, te quedaste mirando a
Eleazar en el suplicio, y dijiste a tus hijos en lengua hebrea:
\bibleverse{16} ``Oh hijos, es noble la contienda a la que habéis sido
llamados como testigos de la nación, luchad celosamente por las leyes de
vuestra patria. \bibleverse{17} Porque sería vergonzoso que este anciano
soportara dolores por causa de la justicia, y que vosotros, que sois más
jóvenes, tuvierais miedo de los suplicios. \bibleverse{18} Recordad que,
por medio de Dios, obtuvisteis la existencia y la habéis disfrutado.
\bibleverse{19} Por tanto, debéis soportar toda aflicción por causa de
Dios. \bibleverse{20} Porque también nuestro padre Abraham se empeñó en
sacrificar a Isaac, nuestro progenitor, y no se estremeció al ver que su
propia mano paterna descendía con la espada sobre él. \bibleverse{21} El
justo Daniel fue arrojado a los leones; y Ananías, Azarías y Misael
fueron arrojados al horno de fuego, pero resistieron por Dios.
\bibleverse{22} Vosotros, pues, teniendo la misma fe hacia Dios, no os
turbéis. \bibleverse{23} Porque no es razonable que los que conocen la
religión no se pongan de pie ante los problemas. \bibleverse{24} Con
estos argumentos, la madre de los siete, exhortando a cada uno de sus
hijos, los animaba y persuadía a no transgredir el mandamiento de Dios.
\bibleverse{25} También vieron esto: que los que mueren por Dios, viven
para Dios, como Abraham, Isaac, Jacob y todos los patriarcas.

\hypertarget{section-16}{%
\section{17}\label{section-16}}

\bibleverse{1} Algunos de los lancero dicen que cuando ella misma estaba
a punto de ser apresada para ser ejecutada, se arrojó sobre el montón,
antes de dejar que tocaran su cuerpo. \bibleverse{2} ¡Oh tú, madre, que
junto con siete hijos destruiste la violencia del tirano, y anulaste sus
malvadas intenciones, y exhibiste la nobleza de la fe! \bibleverse{3}
Porque tú, como una casa valientemente construida sobre la columna de
tus hijos, soportaste el choque de las torturas sin tambalearte.
\bibleverse{4} ¡Anímate, pues, oh madre de espíritu santo! Mantén la
firme esperanza de tu firmeza ante Dios. \bibleverse{5} No tan graciosa
aparece la luna con las estrellas en el cielo, como tú eres establecida
como honorable ante Dios, y fijada en el cielo con tus hijos a quienes
iluminaste con la religión a las estrellas. \bibleverse{6} Porque tu
procreación fue a la manera de un hijo de Abraham.

\bibleverse{7} Si nos fuera lícito pintar como en una tabla la religión
de tu historia, los espectadores no se estremecerían al ver a la madre
de siete hijos soportando por la religión diversas torturas hasta la
muerte. \bibleverse{8} Hubiera sido digno de inscribirse en la propia
tumba estas palabras como recuerdo para los de la nación, \bibleverse{9}
``Aquí están enterrados un anciano sacerdote, y una anciana, y siete
hijos, por la violencia de un tirano, que quiso destruir la sociedad de
los hebreos. \bibleverse{10} Estos también vengaron a su nación, mirando
a Dios y soportando tormentos hasta la muerte.'' \bibleverse{11} Pues
fue verdaderamente una contienda divina la que llevaron a cabo.
\bibleverse{12} Porque en ese momento la virtud presidió la contienda,
aprobando la victoria por medio de la resistencia, es decir, la
inmortalidad, la vida eterna. \bibleverse{13} Eleazar fue el primero en
contender. La madre de los siete hijos entró en la contienda, y la
parentela contendió. \bibleverse{14} El tirano fue el antagonista, y el
mundo y los hombres vivos fueron los espectadores. \bibleverse{15} La
reverencia a Dios venció y coronó a sus propios atletas. \bibleverse{16}
¿Quién no admiró a esos campeones de la verdadera legislación? ¿Quién no
se asombró? \bibleverse{17} El propio tirano, y todo su consejo,
admiraron su resistencia, \bibleverse{18} por lo cual, ellos también
están ahora junto al trono divino y viven una vida bendita.
\bibleverse{19} Porque Moisés dice: ``Todos los santos están bajo tus
manos''. \bibleverse{20} Estos, por lo tanto, habiendo sido santificados
por medio de Dios, han sido honrados no sólo con este honor, sino
también por el hecho de que, gracias a ellos, el enemigo no venció a
nuestra nación. \bibleverse{21} Ese tirano fue castigado y su país
purificado. \bibleverse{22} Porque ellos se convirtieron en el rescate
del pecado de la nación. La Divina Providencia salvó a Israel, que antes
estaba afligido, por la sangre de aquellos piadosos y la muerte que
aplacó la ira. \bibleverse{23} Pues el tirano Antíoco, fijándose en su
valerosa virtud y en su resistencia a la tortura, proclamó esa
resistencia como ejemplo para sus soldados. \bibleverse{24} Le
resultaron nobles y valientes para las batallas terrestres y para los
asedios; y conquistó y asaltó las ciudades de todos sus enemigos.

\hypertarget{section-17}{%
\section{18}\label{section-17}}

\bibleverse{1} Oh hijos de Israel, descendientes de la semilla de
Abraham, obedeced esta ley y sed religiosos en todos los sentidos,
\bibleverse{2} sabiendo que el razonamiento religioso es señor de las
emociones, y éstas no sólo hacia adentro sino hacia afuera.

\bibleverse{3} Por lo tanto, aquellas personas que entregaron sus
cuerpos a los dolores por causa de la religión no sólo fueron admiradas
por los hombres, sino que fueron consideradas dignas de una porción
divina. \bibleverse{4} Por medio de ellos, la nación obtuvo la paz, y
habiendo renovado la observancia de la ley en su país, expulsó al
enemigo del país. \bibleverse{5} El tirano Antíoco fue castigado en la
tierra, y es castigado ahora que está muerto; pues cuando fue totalmente
incapaz de obligar a los israelitas a adoptar costumbres extranjeras y a
abandonar el modo de vida de sus padres, \bibleverse{6} entonces,
partiendo de Jerusalén, hizo la guerra contra los persas. \bibleverse{7}
La justa madre de los siete hijos habló también de la siguiente manera a
su descendencia: ``Yo era una virgen pura, y no salí de la casa de mi
padre, sino que cuidé la costilla de la que fue hecha la mujer.
\bibleverse{8} Ningún destructor del desierto o asaltante de la llanura
me hirió, ni la serpiente destructiva y engañosa hizo botín de mi casta
virginidad. Permanecí con mi esposo durante el tiempo de mi madurez.
\bibleverse{9} Cuando estos, mis hijos, llegaron a la madurez, su padre
murió. Él fue bendecido. Porque habiendo buscado una vida de fecundidad
en los hijos, no se afligió con un período de pérdida de hijos.
\bibleverse{10} Solía enseñarles, cuando aún estaban con ustedes, la ley
y los profetas. \bibleverse{11} Te leía sobre el asesinato de Abel por
Caín, la ofrenda de Isaac y el encarcelamiento de José. \bibleverse{12}
Solía hablarte del celoso Finehas, y te informaba sobre Ananías, Azarías
y Misael en el fuego. \bibleverse{13} Solía glorificar a Daniel, que
estaba en el foso de los leones, y lo declaraba bendito. \bibleverse{14}
Solía recordarte la escritura de Esaías, que dice: ``Aunque pases por el
fuego, no te quemará''. \bibleverse{15} Te cantó a David, el escritor de
himnos, que dice: ``Muchas son las aflicciones del justo.''
\bibleverse{16} Declaró los proverbios de Salomón, que dice: ``Es un
árbol de vida para todos los que hacen su voluntad.'' \bibleverse{17}
Confirmó lo que dijo Ezequiel: ``¿Vivirán estos huesos secos?''
\bibleverse{18} Porque no olvidó el cántico que enseñó Moisés,
proclamando: ``Yo mataré y haré vivir.'' \bibleverse{19} Esta es nuestra
vida y la duración de nuestros días.

\bibleverse{20} ¡Oh, aquel día amargo, y sin embargo no amargo, en que
el amargo tirano de los griegos, apagando fuego con fuego en sus crueles
calderas, llevó con hirviente rabia a los siete hijos de la hija de
Abraham al potro de tortura y a todos sus tormentos! \bibleverse{21} Les
perforó las bolas de los ojos, les cortó la lengua y los condenó a
muerte con diversos suplicios. \bibleverse{22} Por eso la retribución
divina persiguió y perseguirá al infeliz. \bibleverse{23} Pero los hijos
de Abraham, con su madre victoriosa, están reunidos en el coro de su
padre, habiendo recibido de Dios almas puras e inmortales.
\bibleverse{24} A él sea la gloria por los siglos de los siglos. Amén.
