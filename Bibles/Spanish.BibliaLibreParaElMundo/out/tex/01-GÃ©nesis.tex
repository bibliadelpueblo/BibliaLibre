\hypertarget{la-creaciuxf3n-del-mundo}{%
\subsection{La creación del mundo}\label{la-creaciuxf3n-del-mundo}}

\hypertarget{section}{%
\section{1}\label{section}}

\bibleverse{1} En el principio, Dios\footnote{\textbf{1:1} La palabra
  hebrea traducida como ``Dios'' es ``\hebrew{אֱלֹהִ֑ים}'' (Elohim).} creó
los cielos y la tierra. \footnote{\textbf{1:1} Hech 17,24; Apoc 4,11;
  Heb 11,3; Juan 1,1-3} \bibleverse{2} La tierra estaba sin forma y
vacía. Las tinieblas estaban en la superficie de las profundidades y el
Espíritu de Dios se cernía sobre la superficie de las aguas.

\hypertarget{la-creaciuxf3n-de-la-luz}{%
\subsection{La creación de la luz}\label{la-creaciuxf3n-de-la-luz}}

\bibleverse{3} Dios dijo: ``Que se haga la luz'', y se hizo la luz.
\footnote{\textbf{1:3} Sal 33,9; 2Cor 4,6} \bibleverse{4} Dios vio la
luz y vio que era buena. Dios separó la luz de las tinieblas.
\bibleverse{5} Dios llamó a la luz ``día'', y a las tinieblas las llamó
``noche''. Hubo tarde y hubo mañana, el primer día.

\hypertarget{la-creaciuxf3n-de-la-expansion-de-los-cielos}{%
\subsection{La creación de la expansion de los
Cielos}\label{la-creaciuxf3n-de-la-expansion-de-los-cielos}}

\bibleverse{6} Dios dijo: ``Que haya una extensión en medio de las
aguas, y que divida las aguas de las aguas''. \bibleverse{7} Dios hizo
la expansión y dividió las aguas que estaban debajo de la expansión de
las aguas que estaban encima de la expansión; y así fue. \footnote{\textbf{1:7}
  Sal 19,1} \bibleverse{8} Dios llamó a la expansión ``cielo''. Hubo
tarde y mañana, un segundo día.

\hypertarget{la-separacion-de-la-tierra-y-el-mar-y-la-creaciuxf3n-de-las-plantas}{%
\subsection{La separacion de la tierra y el mar y la creación de las
plantas}\label{la-separacion-de-la-tierra-y-el-mar-y-la-creaciuxf3n-de-las-plantas}}

\bibleverse{9} Dios dijo: ``Que las aguas bajo el cielo se reúnan en un
solo lugar, y que aparezca la tierra seca''; y así fue. \footnote{\textbf{1:9}
  2Pe 3,5; Job 38,8-11} \bibleverse{10} Dios llamó a la tierra seca
``tierra'', y a la reunión de las aguas la llamó ``mares''. Dios vio que
era bueno. \bibleverse{11} Dijo Dios: ``Produzca la tierra hierba,
hierbas que den semillas y árboles frutales que den fruto según su
especie, con sus semillas, sobre la tierra''; y así fue. \bibleverse{12}
La tierra dio hierba, hierbas que producen semillas según su género, y
árboles que dan fruto, con sus semillas, según su género; y vio Dios que
era bueno. \bibleverse{13} Se hizo la tarde y la mañana, un tercer día.

\hypertarget{la-creaciuxf3n-de-las-estrellas}{%
\subsection{La creación de las
estrellas}\label{la-creaciuxf3n-de-las-estrellas}}

\bibleverse{14} Dios dijo: ``Que haya luces en la extensión del cielo
para separar el día de la noche; y que sean para señales que marquen las
estaciones, los días y los años; \footnote{\textbf{1:14} Sal 74,16}
\bibleverse{15} y que sean para luces en la extensión del cielo para
alumbrar la tierra''; y así fue. \bibleverse{16} Dios hizo las dos
grandes luces: la luz mayor para gobernar el día, y la luz menor para
gobernar la noche. También hizo las estrellas. \footnote{\textbf{1:16}
  Sal 136,7-9} \bibleverse{17} Dios las puso en la extensión del cielo
para que alumbraran la tierra, \bibleverse{18} y para que dominaran el
día y la noche, y para que separaran la luz de las tinieblas. Dios vio
que era bueno. \bibleverse{19} Se hizo la tarde y se hizo la mañana, un
cuarto día.

\hypertarget{la-creaciuxf3n-de-los-animales-acuuxe1ticos-y-de-los-aves}{%
\subsection{La creación de los animales acuáticos y de los
aves}\label{la-creaciuxf3n-de-los-animales-acuuxe1ticos-y-de-los-aves}}

\bibleverse{20} Dios dijo: ``Que las aguas abunden en seres vivos, y que
las aves vuelen sobre la tierra en la abierta extensión del cielo''.
\bibleverse{21} Dios creó las grandes criaturas marinas y toda criatura
viviente que se mueve, con las que pululan las aguas, según su especie,
y toda ave alada según su especie. Dios vio que era bueno.
\bibleverse{22} Dios los bendijo diciendo: ``Sean fecundos y
multiplíquense, llenen las aguas de los mares y multiplíquense las aves
en la tierra.'' \bibleverse{23} Se hizo la tarde y la mañana, un quinto
día.

\hypertarget{la-creaciuxf3n-de-los-animales-terrestres-y-del-hombre}{%
\subsection{La creación de los animales terrestres y del
hombre}\label{la-creaciuxf3n-de-los-animales-terrestres-y-del-hombre}}

\bibleverse{24} Dios dijo: ``Que la tierra produzca seres vivos según su
especie, ganado, reptiles y animales de la tierra según su especie''; y
así fue. \bibleverse{25} Dios hizo a los animales de la tierra según su
especie, a los animales según su especie y a todo lo que se arrastra por
el suelo según su especie. Dios vio que era bueno.

\bibleverse{26} Dios dijo: ``Hagamos al hombre a nuestra imagen y
semejanza. Que tenga dominio sobre los peces del mar, sobre las aves del
cielo, sobre el ganado, sobre toda la tierra y sobre todo lo que se
arrastra sobre ella''. \footnote{\textbf{1:26} Sal 8,5-8}
\bibleverse{27} Dios creó al hombre a su imagen y semejanza. A imagen y
semejanza de Dios lo creó; hombre y mujer los creó. \footnote{\textbf{1:27}
  Efes 4,24; Gén 2,7; Gén 2,22; Mat 19,4} \bibleverse{28} Dios los
bendijo. Dios les dijo: ``Sed fecundos, multiplicaos, llenad la tierra y
sometedla. Dominen a los peces del mar, a las aves del cielo y a todo
ser viviente que se mueve sobre la tierra''. \footnote{\textbf{1:28}
  Hech 17,26} \bibleverse{29} Dios dijo: ``Mira,\footnote{\textbf{1:29}
  ``He aquí'', de ``\hebrew{הִנֵּה}'', significa mirar, fijarse, observar,
  ver o contemplar. Se utiliza a menudo como interjección.} te he dado
toda hierba que da semilla, que está en la superficie de toda la tierra,
y todo árbol que da fruto que da semilla. Serán su alimento.
\bibleverse{30} A todo animal de la tierra, y a toda ave del cielo, y a
todo lo que se arrastra sobre la tierra, en el que hay vida, les he dado
toda hierba verde como alimento''; y así fue.

\bibleverse{31} Dios vio todo lo que había hecho, y he aquí que era muy
bueno. Hubo tarde y mañana, un sexto día.

\hypertarget{el-dia-de-reposo}{%
\subsection{El dia de reposo}\label{el-dia-de-reposo}}

\hypertarget{section-1}{%
\section{2}\label{section-1}}

\bibleverse{1} Los cielos, la tierra y todo su vasto conjunto fueron
terminados. \bibleverse{2} En el séptimo día Dios terminó su obra que
había hecho; y descansó en el séptimo día de toda su obra que había
hecho. \footnote{\textbf{2:2} Juan 5,17; Heb 4,4; Heb 4,10}
\bibleverse{3} Dios bendijo el séptimo día y lo santificó, porque en él
descansó de toda su obra de creación que había hecho. \footnote{\textbf{2:3}
  Éxod 20,8-11}

\hypertarget{la-creacion-del-hombre-y-de-la-mujer-en-el-parauxedso}{%
\subsection{La creacion del hombre y de la mujer en el
paraíso}\label{la-creacion-del-hombre-y-de-la-mujer-en-el-parauxedso}}

\bibleverse{4} Esta es la historia de las generaciones de los cielos y
de la tierra cuando fueron creados, el día en que Yahvé\footnote{\textbf{2:4}
  ``Yahvé'' es el nombre propio de Dios, a veces traducido como
  ``\textsc{Señor}'' (en mayúsculas) en otras traducciones.} Dios hizo
la tierra y los cielos. \bibleverse{5} Todavía no había en la tierra
ninguna planta del campo, ni había brotado ninguna hierba del campo,
porque Yahvé Dios no había hecho llover sobre la tierra. No había ningún
hombre que labrara la tierra, \bibleverse{6} sino que una niebla subía
de la tierra y regaba toda la superficie del suelo. \bibleverse{7} Yahvé
Dios formó al hombre del polvo de la tierra y sopló en su nariz aliento
de vida, y el hombre se convirtió en un alma viviente. \footnote{\textbf{2:7}
  1Cor 15,45} \bibleverse{8} Yahvé Dios plantó un jardín hacia el este,
en el Edén, y allí puso al hombre que había formado. \bibleverse{9} De
la tierra Yahvé Dios hizo crecer todo árbol agradable a la vista y bueno
para comer, incluyendo el árbol de la vida en medio del jardín y el
árbol del conocimiento del bien y del mal. \footnote{\textbf{2:9} Gén
  3,22; Gén 3,24; Apoc 2,7; Apoc 22,2}

\hypertarget{el-ruxedo-en-el-parauxedso-y-sus-ramales}{%
\subsection{El río en el paraíso y sus
ramales}\label{el-ruxedo-en-el-parauxedso-y-sus-ramales}}

\bibleverse{10} Un río salía del Edén para regar el jardín, y desde allí
se dividía y se convertía en la fuente de cuatro ríos. \bibleverse{11}
El nombre del primero es Pishón; fluye por toda la tierra de Havilah,
donde hay oro; \bibleverse{12} y el oro de esa tierra es bueno. También
hay allí bdellium\footnote{\textbf{2:12} o, resina aromática} y piedra
de ónice. \bibleverse{13} El nombre del segundo río es Gihón. Es el
mismo río que atraviesa toda la tierra de Cus. \bibleverse{14} El nombre
del tercer río es Hiddekel. Es el que fluye frente a Asiria. El cuarto
río es el Éufrates.

\hypertarget{el-mandamiento-de-dios-por-adam}{%
\subsection{El mandamiento de Dios por
Adam}\label{el-mandamiento-de-dios-por-adam}}

\bibleverse{15} Yahvé Dios tomó al hombre y lo puso en el jardín del
Edén para que lo cultivara y lo cuidara. \bibleverse{16} Yahvé Dios
ordenó al hombre diciendo: ``Puedes comer libremente de todos los
árboles del jardín; \bibleverse{17} pero no comerás del árbol del
conocimiento del bien y del mal, porque el día que comas de él,
morirás.'' \footnote{\textbf{2:17} Rom 5,12; 1Cor 15,21}

\hypertarget{la-creaciuxf3n-de-la-mujer-y-la-fundaciuxf3n-del-matrimonio}{%
\subsection{La creación de la mujer y la fundación del
matrimonio}\label{la-creaciuxf3n-de-la-mujer-y-la-fundaciuxf3n-del-matrimonio}}

\bibleverse{18} Yahvé Dios dijo: ``No es bueno que el hombre esté solo.
Le haré un ayudante comparable a\footnote{\textbf{2:18} o, adecuado
  para, o apropiado para.} el''. \footnote{\textbf{2:18} Prov 31,10-31}
\bibleverse{19} Yahvé Dios formó de la tierra todo animal del campo y
toda ave del cielo, y se los llevó al hombre para ver cómo los llamaba.
Lo que el hombre llamó a cada criatura viviente se convirtió en su
nombre. \bibleverse{20} El hombre dio nombres a todo el ganado, a las
aves del cielo y a todo animal del campo; pero para el hombre no se
encontró un ayudante comparable a él. \bibleverse{21} El Señor Dios hizo
que el hombre cayera en un profundo sueño. Mientras el hombre dormía,
tomó una de sus costillas y cerró la carne en su lugar. \bibleverse{22}
Yahvé Dios hizo una mujer a partir de la costilla que había tomado del
hombre, y se la llevó al hombre. \footnote{\textbf{2:22} 1Cor 11,7-9;
  1Cor 11,12; 1Tim 2,13} \bibleverse{23} El hombre dijo: ``Esto es ahora
hueso de mis huesos y carne de mi carne. Se llamará `mujer', porque fue
tomada del Hombre''. \bibleverse{24} Por tanto, el hombre dejará a su
padre y a su madre y se unirá a su mujer, y serán una sola carne.
\footnote{\textbf{2:24} Mat 19,5-6; Efes 5,28-31} \bibleverse{25} El
hombre y su mujer estaban desnudos, y no se avergonzaban.

\hypertarget{la-tentacion-y-la-cauxedda-del-hombre}{%
\subsection{La tentacion y la caída del
hombre}\label{la-tentacion-y-la-cauxedda-del-hombre}}

\hypertarget{section-2}{%
\section{3}\label{section-2}}

\bibleverse{1} La serpiente era más astuta que cualquier otro animal del
campo que había hecho Yahvé Dios. Le dijo a la mujer: ``¿De verdad ha
dicho Dios: ``No comerás de ningún árbol del jardín''?'' \footnote{\textbf{3:1}
  Apoc 12,9; Apoc 20,2}

\bibleverse{2} La mujer dijo a la serpiente: ``Podemos comer del fruto
de los árboles del jardín, \footnote{\textbf{3:2} Gén 2,16}
\bibleverse{3} pero no del fruto del árbol que está en medio del jardín.
Dios ha dicho: `No comerás de él. No lo tocarás, para que no mueras''.
\footnote{\textbf{3:3} Gén 2,17}

\bibleverse{4} La serpiente dijo a la mujer: ``No morirás realmente,
\footnote{\textbf{3:4} Juan 8,44} \bibleverse{5} porque Dios sabe que el
día que lo comas se te abrirán los ojos y serás como Dios, conociendo el
bien y el mal.''

\bibleverse{6} Cuando la mujer vio que el árbol era bueno para comer y
que era un deleite para los ojos, y que el árbol era deseable para
hacerse sabio, tomó parte de su fruto y comió. Luego le dio un poco a su
marido, que también comió. \footnote{\textbf{3:6} Sant 1,14; 1Tim 2,14}
\bibleverse{7} Se les abrieron los ojos y ambos se dieron cuenta de que
estaban desnudos. Cosieron hojas de higuera y se cubrieron. \footnote{\textbf{3:7}
  Gén 2,25}

\hypertarget{el-interrogatorio-y-el-castigo}{%
\subsection{El interrogatorio y el
castigo}\label{el-interrogatorio-y-el-castigo}}

\bibleverse{8} Oyeron la voz de Yahvé Dios que se paseaba por el jardín
en el fresco del día, y el hombre y su mujer se escondieron de la
presencia de Yahvé Dios entre los árboles del jardín. \footnote{\textbf{3:8}
  Jer 23,24}

\bibleverse{9} Yahvé Dios llamó al hombre y le dijo: ``¿Dónde estás?''.

\bibleverse{10} El hombre dijo: ``Oí tu voz en el jardín, y tuve miedo,
porque estaba desnudo; así que me escondí''.

\bibleverse{11} Dios dijo: ``¿Quién te dijo que estabas desnudo? ¿Has
comido del árbol del que te ordené no comer?''

\bibleverse{12} El hombre dijo: ``La mujer que me diste para estar
conmigo, me dio fruto del árbol y lo comí''.

\bibleverse{13} Yahvé Dios dijo a la mujer: ``¿Qué has hecho?'' La mujer
dijo: ``La serpiente me engañó y comí''. \footnote{\textbf{3:13} 2Cor
  11,3}

\bibleverse{14} Yahvé Dios dijo a la serpiente, ``Porque has hecho esto,
estás maldito por encima de todo el ganado, y por encima de todo animal
del campo. Irás sobre tu vientrey comerás polvo todos los días de tu
vida. \footnote{\textbf{3:14} Is 65,25} \bibleverse{15} Pondré
hostilidad entre tú y la mujer, y entre tu descendencia y la de ella. Te
va a magullar la cabeza, y le magullarás el talón''. \footnote{\textbf{3:15}
  Gal 4,4; 1Jn 3,8; Heb 2,14; Rom 16,20; Juan 14,30; Apoc 12,17}

\bibleverse{16} A la mujer le dijo, ``Multiplicaré en gran medida tus
dolores de parto. Tendrás hijos con dolor. Tu deseo será para tu marido,
y te gobernará''. \footnote{\textbf{3:16} Efes 5,22; 1Tim 2,11-12}

\bibleverse{17} A Adán le dijo, ``Porque has escuchado la voz de tu
mujer, y han comido del árbol, sobre lo que te ordené, diciendo: `No
comerás de él'. la tierra está maldita por tu causa. Comerás de él con
mucho trabajo todos los días de tu vida. \bibleverse{18} Te dará espinas
y cardos; y comerás la hierba del campo. \bibleverse{19} Comerás el pan
con el sudor de tu rostro hasta que vuelvas a la tierra, ya que fuiste
sacado de ella. Porque tú eres polvo, y volverás al polvo''. \footnote{\textbf{3:19}
  2Tes 3,10; Ecl 12,7}

\hypertarget{la-expulsiuxf3n-del-parauxedso}{%
\subsection{La expulsión del
paraíso}\label{la-expulsiuxf3n-del-parauxedso}}

\bibleverse{20} El hombre llamó a su mujer Eva, porque ella sería la
madre de todos los vivientes. \bibleverse{21} Yahvé Dios hizo vestidos
de pieles de animales para Adán y para su mujer, y los vistió.

\bibleverse{22} Yahvé Dios dijo: ``He aquí que el hombre ha llegado a
ser como uno de nosotros, conociendo el bien y el mal. Ahora bien, para
que no extienda su mano y tome también del árbol de la vida, y coma, y
viva para siempre ---'' \footnote{\textbf{3:22} Gén 3,5} \bibleverse{23}
Por eso Yahvé Dios lo envió fuera del jardín de Edén, para que labrara
la tierra de la que fue tomado. \bibleverse{24} Y expulsó al hombre; y
puso querubines\footnote{\textbf{3:24} Los querubines son poderosas
  criaturas angélicas, mensajeros de Dios con alas. Véase Ezequiel 10.}
al oriente del jardín del Edén, y una espada flamígera que se volvía
hacia todos lados, para guardar el camino del árbol de la vida.
\footnote{\textbf{3:24} Ezeq 10,1}

\hypertarget{cauxedn-y-abel}{%
\subsection{Caín y Abel}\label{cauxedn-y-abel}}

\hypertarget{section-3}{%
\section{4}\label{section-3}}

\bibleverse{1} El hombre conoció\footnote{\textbf{4:1} o, yacer con, o,
  tener relaciones con} Eva, su mujer. Ella concibió,\footnote{\textbf{4:1}
  o, se quedó embarazada} y dio a luz a Caín, y dijo: ``He conseguido un
hombre con la ayuda de Yahvé''. \bibleverse{2} De nuevo dio a luz a
Abel, el hermano de Caín. Abel era cuidador de ovejas, pero Caín era
labrador de la tierra. \bibleverse{3} Con el tiempo, Caín trajo una
ofrenda a Yahvé del fruto de la tierra. \bibleverse{4} Abel también
trajo parte de los primogénitos de su rebaño y de su grasa. Yahvé
respetó a Abel y su ofrenda, \footnote{\textbf{4:4} Heb 11,4}
\bibleverse{5} pero no respetó a Caín y su ofrenda. Caín se enfadó
mucho, y la expresión de su rostro decayó. \bibleverse{6} Yahvé dijo a
Caín: ``¿Por qué estás enojado? ¿Por qué ha decaído la expresión de tu
rostro? \bibleverse{7} Si haces bien, ¿no se levantará? Si no haces
bien, el pecado se agazapa a la puerta. Su deseo es para ti, pero tú
debes dominarlo''. \footnote{\textbf{4:7} Gal 5,17; Rom 6,12}
\bibleverse{8} Caín dijo a Abel, su hermano: ``Vamos al campo''.
Mientras estaban en el campo, Caín se levantó contra Abel, su hermano, y
lo mató. \footnote{\textbf{4:8} 1Jn 3,12; 1Jn 1,3-15}

\hypertarget{el-castigo-de-cauxedn}{%
\subsection{El castigo de Caín}\label{el-castigo-de-cauxedn}}

\bibleverse{9} Yahvé dijo a Caín: ``¿Dónde está Abel, tu hermano?''
Dijo: ``No lo sé. ¿Soy el guardián de mi hermano?''

\bibleverse{10} El Señor dijo: ``¿Qué has hecho? La voz de la sangre de
tu hermano clama a mí desde la tierra. \footnote{\textbf{4:10} Sal 9,12;
  Mat 23,35; Heb 12,24} \bibleverse{11} Ahora estás maldito por culpa de
la tierra, que ha abierto su boca para recibir la sangre de tu hermano
de tu mano. \bibleverse{12} De ahora en adelante, cuando labres la
tierra, no te cederá su fuerza. Serás un fugitivo y un vagabundo en la
tierra''.

\bibleverse{13} Caín dijo a Yahvé: ``Mi castigo es mayor de lo que puedo
soportar. \bibleverse{14} He aquí que hoy me has expulsado de la
superficie de la tierra. Quedaré oculto de tu rostro, y seré un fugitivo
y un vagabundo en la tierra. Quien me encuentre me matará''. \footnote{\textbf{4:14}
  Job 15,20-24}

\bibleverse{15} Yahvé le dijo: ``Por lo tanto, quien mate a Caín, se
vengará de él siete veces''. Yahvé designó una señal para Caín, para que
quien lo encontrara no lo golpeara.

\bibleverse{16} Caín dejó la presencia de Yahvé y vivió en la tierra de
Nod, al este de Edén.

\hypertarget{los-hijos-de-cauxedn}{%
\subsection{Los hijos de Caín}\label{los-hijos-de-cauxedn}}

\bibleverse{17} Caín conoció a su esposa. Ella concibió y dio a luz a
Enoc. Él construyó una ciudad, y llamó a la ciudad con el nombre de su
hijo, Enoc. \bibleverse{18} De Enoc nació Irad. Irad se convirtió en el
padre de Mehujael. Mehujael fue el padre de Matusalén. Matusalén fue el
padre de Lamec. \bibleverse{19} Lamec tomó dos esposas: el nombre de la
primera fue Ada, y el nombre de la segunda fue Zillah. \bibleverse{20}
Ada dio a luz a Jabal, que fue el padre de los que habitan en tiendas y
tienen ganado. \bibleverse{21} Su hermano se llamaba Jubal, que fue el
padre de todos los que manejan el arpa y la flauta. \bibleverse{22} Zila
también dio a luz a Tubal Caín, el forjador de todo instrumento cortante
de bronce y hierro. La hermana de Tubal Caín fue Naamah. \bibleverse{23}
Lamec dijo a sus esposas, ``Adah y Zillah, escuchen mi voz. Esposas de
Lamec, escuchad mi discurso, porque he matado a un hombre por herirme,
un joven por haberme golpeado. \bibleverse{24} Si Caín será vengado
siete veces, verdaderamente Lamec setenta y siete veces''. \footnote{\textbf{4:24}
  Gén 4,15; Mat 18,21-22}

\hypertarget{el-nacimiento-de-seth}{%
\subsection{El nacimiento de Seth}\label{el-nacimiento-de-seth}}

\bibleverse{25} Adán volvió a conocer a su mujer. Ella dio a luz un
hijo, y le puso el nombre de Set, diciendo: ``Porque Dios me ha dado
otro hijo en lugar de Abel, ya que Caín lo mató''. \bibleverse{26}
También le nació un hijo a Set, y lo llamó Enosh. En aquel tiempo los
hombres comenzaron a invocar el nombre de Yahvé. \footnote{\textbf{4:26}
  Gén 12,8}

\hypertarget{la-descendiencia-de-seth-hasta-nouxe9}{%
\subsection{La descendiencia de Seth hasta
Noé}\label{la-descendiencia-de-seth-hasta-nouxe9}}

\hypertarget{section-4}{%
\section{5}\label{section-4}}

\bibleverse{1} Este es el libro de las generaciones de Adán. El día que
Dios creó al hombre, lo hizo a su semejanza. \footnote{\textbf{5:1} Gén
  1,27; Luc 3,38} \bibleverse{2} Los creó varón y mujer, y los bendijo.
El día en que fueron creados, les puso el nombre de Adán. \footnote{\textbf{5:2}
  ``Adán'' y ``Hombre'' se escriben exactamente con las mismas
  consonantes en hebreo, por lo que se puede traducir correctamente de
  cualquier manera.} \bibleverse{3} Adán vivió ciento treinta años y fue
padre de un hijo a su imagen y semejanza, al que llamó Set. \footnote{\textbf{5:3}
  Sal 51,5; 1Cor 15,49} \bibleverse{4} Los días de Adán después de ser
padre de Set fueron ochocientos años, y llegó a ser padre de otros hijos
e hijas. \bibleverse{5} Todos los días que vivió Adán fueron novecientos
treinta años, y luego murió.

\bibleverse{6} Set vivió ciento cinco años y luego fue padre de Enós.
\bibleverse{7} Set vivió después de ser padre de Enós ochocientos siete
años, y fue padre de otros hijos e hijas. \bibleverse{8} Todos los días
de Set fueron novecientos doce años, y luego murió.

\bibleverse{9} Enosh vivió noventa años y fue padre de Kenan.
\bibleverse{10} Enosh vivió, después de ser padre de Kenán, ochocientos
quince años, y fue padre de otros hijos e hijas. \bibleverse{11} Todos
los días de Enosh fueron novecientos cinco años, y luego murió.

\bibleverse{12} Kenan vivió setenta años, y luego fue padre de
Mahalalel. \bibleverse{13} Kenan vivió después de ser padre de Mahalalel
ochocientos cuarenta años, y fue padre de otros hijos e hijas
\bibleverse{14} y todos los días de Kenan fueron novecientos diez años,
luego murió.

\bibleverse{15} Mahalalel vivió sesenta y cinco años, y luego fue padre
de Jared. \bibleverse{16} Mahalalel vivió, después de ser padre de
Jared, ochocientos treinta años, y fue padre de otros hijos e hijas.
\bibleverse{17} Todos los días de Mahalalel fueron ochocientos noventa y
cinco años, y luego murió.

\bibleverse{18} Jared vivió ciento sesenta y dos años, y luego fue padre
de Enoc. \bibleverse{19} Jared vivió después de ser padre de Enoc
ochocientos años, y fue padre de otros hijos e hijas. \bibleverse{20}
Todos los días de Jared fueron novecientos sesenta y dos años, y luego
murió.

\bibleverse{21} Enoc vivió sesenta y cinco años, y luego fue padre de
Matusalén. \bibleverse{22} Después del nacimiento de Matusalén, Enoc
caminó con Dios durante trescientos años y fue padre de más hijos e
hijas. \footnote{\textbf{5:22} Gén 6,9; Jds 1,14} \bibleverse{23} Todos
los días de Enoc fueron trescientos sesenta y cinco años.
\bibleverse{24} Enoc caminó con Dios, y no fue hallado, pues Dios se lo
llevó. \footnote{\textbf{5:24} Heb 11,5; 2Re 2,11; Is 57,1-2}

\bibleverse{25} Matusalén vivió ciento ochenta y siete años, y luego fue
padre de Lamec. \bibleverse{26} Matusalén vivió, después de ser padre de
Lamec, setecientos ochenta y dos años, y fue padre de otros hijos e
hijas. \bibleverse{27} Todos los días de Matusalén fueron novecientos
sesenta y nueve años, y luego murió.

\bibleverse{28} Lamec vivió ciento ochenta y dos años, y luego fue padre
de un hijo. \bibleverse{29} Le puso el nombre de Noé, diciendo: ``Éste
nos consolará en nuestro trabajo y en el trabajo de nuestras manos,
causado por la tierra que Yahvé ha maldecido.'' \footnote{\textbf{5:29}
  Gén 3,17-19} \bibleverse{30} Lamec vivió, después de ser padre de Noé,
quinientos noventa y cinco años, y fue padre de otros hijos e hijas.
\bibleverse{31} Todos los días de Lamec fueron setecientos setenta y
siete años, y luego murió.

\bibleverse{32} Noé tenía quinientos años, entonces Noé fue padre de
Sem, Cam y Jafet.

\hypertarget{los-matrimonios-de-los-hijos-de-dios-con-las-hijas-de-los-hombres}{%
\subsection{Los matrimonios de los hijos de Dios con las hijas de los
hombres}\label{los-matrimonios-de-los-hijos-de-dios-con-las-hijas-de-los-hombres}}

\hypertarget{section-5}{%
\section{6}\label{section-5}}

\bibleverse{1} Cuando los hombres comenzaron a multiplicarse sobre la
superficie de la tierra, y les nacieron hijas, \bibleverse{2} Los hijos
de Dios vieron que las hijas de los hombres eran hermosas, y tomaron
para sí las que quisieron como esposas. \footnote{\textbf{6:2} Mat 24,38}
\bibleverse{3} Yahvé dijo: ``Mi Espíritu no luchará con el hombre para
siempre, porque él también es carne; así que sus días serán de ciento
veinte años.'' \footnote{\textbf{6:3} 1Pe 3,20} \bibleverse{4} Los
Nefilim\footnote{\textbf{6:4} o, gigantes} estaban en la tierra en esos
días, y también después de eso, cuando los hijos de Dios entraron a las
hijas de los hombres y tuvieron hijos con ellas. Esos eran los hombres
poderosos que había en la antigüedad, hombres de renombre.

\hypertarget{la-maldad-de-los-hombres.-anuncio-del-diluvio}{%
\subsection{La maldad de los hombres. Anuncio del
diluvio}\label{la-maldad-de-los-hombres.-anuncio-del-diluvio}}

\bibleverse{5} Yahvé vio que la maldad del hombre era grande en la
tierra, y que todo designio de los pensamientos del corazón del hombre
era de continuo sólo el mal. \footnote{\textbf{6:5} Gén 8,21}
\bibleverse{6} Yahvé se arrepintió de haber hecho al hombre en la
tierra, y le dolió en su corazón. \footnote{\textbf{6:6} Jer 18,10; Núm
  23,19; Sal 18,26} \bibleverse{7} Yahvé dijo: ``Destruiré al hombre que
he creado de la superficie de la tierra, junto con los animales, los
reptiles y las aves del cielo, pues me arrepiento de haberlos hecho.''
\bibleverse{8} Pero Noé encontró el favor de los ojos de Yahvé.

\hypertarget{llamado-de-nouxe9-y-la-construcciuxf3n-del-arca}{%
\subsection{Llamado de Noé y la construcción del
arca}\label{llamado-de-nouxe9-y-la-construcciuxf3n-del-arca}}

\bibleverse{9} Esta es la historia de las generaciones de Noé: Noé era
un hombre justo, irreprochable entre la gente de su tiempo. Noé caminó
con Dios. \footnote{\textbf{6:9} Heb 11,7; Gén 5,22; Gén 5,24}
\bibleverse{10} Noé fue padre de tres hijos: Sem, Cam y Jafet.
\bibleverse{11} La tierra estaba corrompida ante Dios, y la tierra
estaba llena de violencia. \bibleverse{12} Dios vio la tierra y vio que
estaba corrompida, porque toda la carne había corrompido su camino en la
tierra. \footnote{\textbf{6:12} Sal 14,2-3}

\bibleverse{13} Dios dijo a Noé: ``Voy a acabar con toda la carne,
porque la tierra está llena de violencia por culpa de ellos. He aquí que
los destruiré a ellos y a la tierra. \footnote{\textbf{6:13} Am 8,2}
\bibleverse{14} Haz un barco de madera de topo. Harás habitaciones en la
nave, y la sellarás por dentro y por fuera con brea. \bibleverse{15} Así
lo harás. La longitud de la nave será de trescientos codos,\footnote{\textbf{6:15}
  Un codo es la longitud desde la punta del dedo corazón hasta el codo
  del brazo de un hombre, es decir, unas 18 pulgadas o 46 centímetros.}
su anchura de cincuenta codos, y su altura de treinta codos.
\bibleverse{16} Harás un techo en la nave, y lo terminarás a un codo
hacia arriba. Pondrás la puerta de la nave en su costado. La harás con
niveles inferior, segundo y tercero. \bibleverse{17} Yo, yo mismo,
traeré el diluvio de aguas sobre esta tierra, para destruir toda carne
que tenga aliento de vida de debajo del cielo. Todo lo que hay en la
tierra morirá. \bibleverse{18} Pero yo estableceré mi pacto con ustedes.
Entrarás en la nave, tú, tus hijos, tu mujer y las mujeres de tus hijos
contigo. \bibleverse{19} De todo ser viviente de toda carne, traerás dos
de cada especie a la nave, para mantenerlos vivos contigo. Serán macho y
hembra. \bibleverse{20} De las aves según su especie, del ganado según
su especie, de todo reptil del suelo según su especie, dos de cada
especie irán con vosotros, para mantenerlos con vida. \bibleverse{21}
Toma contigo algo de todo lo que se come, y recógelo para ti, y te
servirá de alimento a ti y a ellos.'' \bibleverse{22} Así hizo Noé. Hizo
todo lo que Dios le ordenó.

\hypertarget{el-diluvio.-nouxe9-entra-la-arca}{%
\subsection{El diluvio. Noé entra la
arca}\label{el-diluvio.-nouxe9-entra-la-arca}}

\hypertarget{section-6}{%
\section{7}\label{section-6}}

\bibleverse{1} Yahvé dijo a Noé: ``Sube con toda tu familia a la nave,
porque he visto tu justicia ante mí en esta generación. \bibleverse{2}
Llevarás contigo siete parejas de cada animal limpio, el macho y su
hembra. De los animales que no están limpios, toma dos, el macho y su
hembra. \footnote{\textbf{7:2} Gén 8,20; Lev 11,1} \bibleverse{3}
También de las aves del cielo, siete y siete, macho y hembra, para
mantener viva la semilla en la superficie de toda la tierra.
\bibleverse{4} En siete días haré llover sobre la tierra durante
cuarenta días y cuarenta noches. Destruiré todo ser viviente que he
hecho de la superficie de la tierra''.

\bibleverse{5} Noé hizo todo lo que Yahvé le ordenó. \footnote{\textbf{7:5}
  Gén 6,22}

\bibleverse{6} Noé tenía seiscientos años cuando el diluvio de aguas
llegó a la tierra. \bibleverse{7} Noé subió a la nave con sus hijos, su
mujer y las mujeres de sus hijos, a causa de las aguas del diluvio.
\footnote{\textbf{7:7} 1Pe 3,20} \bibleverse{8} Los animales limpios,
los inmundos, las aves y todo lo que se arrastra por el suelo
\bibleverse{9} entraron por parejas con Noé en la nave, machos y
hembras, como Dios le había ordenado a Noé. \footnote{\textbf{7:9} Gén
  6,19} \bibleverse{10} Después de los siete días, las aguas de la
inundación llegaron a la tierra.

\hypertarget{el-aumento-del-diluvio}{%
\subsection{El aumento del diluvio}\label{el-aumento-del-diluvio}}

\bibleverse{11} En el año seiscientos de la vida de Noé, en el segundo
mes, a los diecisiete días del mes, ese día estallaron todas las fuentes
del gran abismo y se abrieron las ventanas del cielo. \bibleverse{12}
Llovió sobre la tierra durante cuarenta días y cuarenta noches.

\bibleverse{13} En el mismo día Noé, y Sem, Cam y Jafet --- los hijos de
Noé --- y la esposa de Noé y las tres esposas de sus hijos con ellos,
entraron en la nave --- \bibleverse{14} ellos, y todo animal según su
especie, todo el ganado según su especie, todo reptil que se arrastra
sobre la tierra según su especie, y toda ave según su especie, toda ave
de toda clase. \bibleverse{15} Las parejas de toda carne con aliento de
vida entraron en la nave hacia Noé. \bibleverse{16} Los que entraron,
entraron macho y hembra de toda carne, como Dios le ordenó; entonces
Yahvé lo encerró. \footnote{\textbf{7:16} Gén 6,19} \bibleverse{17} El
diluvio duró cuarenta días sobre la tierra. Las aguas aumentaron, y
levantaron la nave, y ésta se elevó sobre la tierra. \bibleverse{18} Las
aguas crecieron y aumentaron mucho sobre la tierra, y el barco flotaba
sobre la superficie de las aguas. \bibleverse{19} Las aguas se elevaron
mucho sobre la tierra. Todos los montes altos que había bajo todo el
cielo quedaron cubiertos. \bibleverse{20} Las aguas subieron quince
codos\footnote{\textbf{7:20} Un codo es la longitud desde la punta del
  dedo corazón hasta el codo del brazo de un hombre, es decir, unas 18
  pulgadas o 46 centímetros.} más, y las montañas quedaron cubiertas.
\bibleverse{21} Murió toda la carne que se movía sobre la tierra,
incluyendo las aves, el ganado, los animales, todo lo que se arrastra
sobre la tierra y todo hombre. \footnote{\textbf{7:21} 2Pe 3,6; Job
  22,15-16} \bibleverse{22} Murió todo lo que estaba en la tierra firme,
en cuyas narices había aliento de espíritu de vida. \bibleverse{23} Fue
destruido todo ser viviente que estaba sobre la superficie de la tierra,
incluidos el hombre, el ganado, los reptiles y las aves del cielo.
Fueron destruidos de la tierra. Sólo quedó Noé y los que estaban con él
en la nave. \bibleverse{24} Las aguas inundaron la tierra durante ciento
cincuenta días.

\hypertarget{el-fin-del-diluvio}{%
\subsection{El fin del diluvio}\label{el-fin-del-diluvio}}

\hypertarget{section-7}{%
\section{8}\label{section-7}}

\bibleverse{1} Dios se acordó de Noé, de todos los animales y de todo el
ganado que estaba con él en el barco; y Dios hizo pasar un viento sobre
la tierra. Las aguas se calmaron. \bibleverse{2} También se detuvieron
las fuentes de las profundidades y las ventanas del cielo, y se frenó la
lluvia del cielo. \footnote{\textbf{8:2} Gén 7,11-12} \bibleverse{3} Las
aguas se retiraron continuamente de la tierra. Al cabo de ciento
cincuenta días las aguas se retiraron. \bibleverse{4} La nave se detuvo
en el séptimo mes, el día diecisiete del mes, sobre las montañas de
Ararat. \bibleverse{5} Las aguas retrocedieron continuamente hasta el
décimo mes. En el décimo mes, el primer día del mes, las cimas de las
montañas fueron visibles.

\bibleverse{6} Al cabo de cuarenta días, Noé abrió la ventana de la nave
que había hecho, \bibleverse{7} y envió un cuervo. Fue de un lado a
otro, hasta que las aguas se secaron de la tierra. \bibleverse{8} Él
mismo envió una paloma para ver si las aguas se habían retirado de la
superficie de la tierra, \bibleverse{9} pero la paloma no encontró lugar
para posar su pie, y volvió a la nave hacia él, porque las aguas estaban
en la superficie de toda la tierra. Él extendió la mano, la tomó y la
introdujo en la nave. \bibleverse{10} Esperó aún otros siete días, y
volvió a enviar a la paloma fuera de la nave. \bibleverse{11} Al
anochecer, la paloma regresó a él, y he aquí que en su boca había una
hoja de olivo recién arrancada. Así Noé supo que las aguas habían
desaparecido de la tierra. \bibleverse{12} Esperó aún otros siete días y
envió a la paloma, y ésta ya no volvió a él.

\bibleverse{13} En el año seiscientos uno, en el primer mes, el primer
día del mes, las aguas se secaron de la tierra. Noé quitó la cubierta de
la nave y miró. Vio que la superficie de la tierra estaba seca.
\bibleverse{14} En el segundo mes, a los veintisiete días del mes, la
tierra estaba seca.

\bibleverse{15} Dios habló a Noé, diciendo: \bibleverse{16} ``Sal de la
nave, tú, tu mujer, tus hijos y las mujeres de tus hijos contigo.
\bibleverse{17} Saca contigo todo ser viviente de toda carne, incluyendo
las aves, el ganado y todo animal que se arrastra sobre la tierra, para
que se reproduzcan abundantemente en la tierra, y sean fructíferos y se
multipliquen sobre la tierra.'' \footnote{\textbf{8:17} Gén 1,22; Gén
  1,28}

\bibleverse{18} Noé salió con sus hijos, su mujer y las mujeres de sus
hijos. \footnote{\textbf{8:18} 2Pe 2,5} \bibleverse{19} Todo animal,
todo reptil y toda ave, todo lo que se mueve en la tierra, según sus
familias, salió de la nave.

\hypertarget{el-holocausto-de-nouxe9-y-la-promesa-de-dios}{%
\subsection{El holocausto de Noé y la promesa de
Dios}\label{el-holocausto-de-nouxe9-y-la-promesa-de-dios}}

\bibleverse{20} Noé construyó un altar a Yahvé, y tomó de todo animal
limpio y de toda ave limpia, y ofreció holocaustos sobre el altar.
\footnote{\textbf{8:20} Gén 7,2} \bibleverse{21} Yahvé olió el agradable
aroma. Yavé dijo en su corazón: ``No volveré a maldecir la tierra por
causa del hombre, porque la imaginación del corazón del hombre es mala
desde su juventud. No volveré a golpear a todo ser viviente, como lo he
hecho. \footnote{\textbf{8:21} Gén 6,5; Sal 14,3; Job 14,4; Mat 15,19;
  Rom 3,23; Is 54,9} \bibleverse{22} Mientras la tierra permanezca, no
cesarán el tiempo de la siembra y la cosecha, el frío y el calor, el
verano y el invierno, el día y la noche.'' \footnote{\textbf{8:22} Jer
  33,20; Jer 33,25}

\hypertarget{renovaciuxf3n-de-la-bendiciuxf3n-de-la-creaciuxf3n}{%
\subsection{Renovación de la bendición de la
creación}\label{renovaciuxf3n-de-la-bendiciuxf3n-de-la-creaciuxf3n}}

\hypertarget{section-8}{%
\section{9}\label{section-8}}

\bibleverse{1} Dios bendijo a Noé y a sus hijos y les dijo: ``Sed
fecundos, multiplicaos y llenad la tierra. \footnote{\textbf{9:1} Gén
  1,28} \bibleverse{2} El temor y el miedo a vosotros recaerán sobre
todos los animales de la tierra y sobre todas las aves del cielo. Todo
lo que se mueve por la tierra, y todos los peces del mar, serán
entregados en tu mano. \bibleverse{3} Todo lo que se mueve y vive será
alimento para ti. Así como te di la hierba verde, te he dado todo.
\footnote{\textbf{9:3} Gén 1,29; Col 2,16} \bibleverse{4} Pero la carne
con vida, es decir, su sangre, no la comeréis. \footnote{\textbf{9:4}
  Lev 3,17} \bibleverse{5} Ciertamente pediré cuentas por la sangre de
tu vida. A la mano de todo animal se la exigiré. A la mano del hombre,
incluso a la mano del hermano de todo hombre, le exigiré la vida del
hombre. \footnote{\textbf{9:5} Éxod 21,28-29; Gén 4,11} \bibleverse{6}
El que derrame sangre de hombre, su sangre será derramada por el hombre,
porque Dios hizo al hombre a su imagen y semejanza. \footnote{\textbf{9:6}
  Éxod 21,12; Lev 24,17; Mat 26,52; Apoc 13,10; Gén 1,27} \bibleverse{7}
Sed fecundos y multiplicaos. Creced en abundancia en la tierra y
multiplicaos en ella''.

\hypertarget{el-pacto-entro-dios-y-nouxe9-y-la-creaciuxf3n}{%
\subsection{El pacto entro Dios y Noé y la
creación}\label{el-pacto-entro-dios-y-nouxe9-y-la-creaciuxf3n}}

\bibleverse{8} Dios habló a Noé y a sus hijos con él, diciendo:
\bibleverse{9} ``En cuanto a mí, he aquí que yo establezco mi pacto con
vosotros, y con vuestra descendencia después de vosotros, \footnote{\textbf{9:9}
  Gén 6,18} \bibleverse{10} y con toda criatura viviente que está con
vosotros: las aves, el ganado y todo animal de la tierra con vosotros,
de todos los que salen de la nave, todo animal de la tierra. \footnote{\textbf{9:10}
  Os 2,8} \bibleverse{11} Estableceré mi pacto con vosotros: Toda la
carne no volverá a ser eliminada por las aguas del diluvio. Nunca más
habrá un diluvio que destruya la tierra''. \footnote{\textbf{9:11} Gén
  8,21-22} \bibleverse{12} Dios dijo: ``Esta es la señal de la alianza
que hago entre ustedes y yo, y toda criatura viviente que está con
ustedes, por generaciones perpetuas: \bibleverse{13} Yo pongo mi arco
iris en la nube, y será una señal de alianza entre la tierra y yo.
\bibleverse{14} Cuando traiga una nube sobre la tierra, para que el arco
iris se vea en la nube, \bibleverse{15} me acordaré de mi pacto, que es
entre yo y vosotros y toda criatura viviente de toda carne, y las aguas
no se convertirán más en un diluvio para destruir toda carne.
\bibleverse{16} El arco iris estará en la nube. Lo miraré para acordarme
del pacto eterno entre Dios y toda criatura viviente de toda carne que
está en la tierra.'' \bibleverse{17} Dios dijo a Noé: ``Esta es la señal
de la alianza que he establecido entre yo y toda la carne que está sobre
la tierra.''

\hypertarget{la-embriaguez-de-nouxe9}{%
\subsection{La embriaguez de Noé}\label{la-embriaguez-de-nouxe9}}

\bibleverse{18} Los hijos de Noé que salieron de la nave fueron Sem, Cam
y Jafet. Cam es el padre de Canaán. \bibleverse{19} Estos tres fueron
los hijos de Noé, y de ellos se pobló toda la tierra.

\bibleverse{20} Noé comenzó a ser agricultor y plantó una viña.
\bibleverse{21} Bebió del vino y se emborrachó. Se descubrió dentro de
su tienda. \bibleverse{22} Cam, el padre de Canaán, vio la desnudez de
su padre y se lo dijo a sus dos hermanos que estaban fuera. \footnote{\textbf{9:22}
  Prov 30,17; Sir 3,12} \bibleverse{23} Sem y Jafet tomaron una prenda
de vestir, se la pusieron sobre los hombros de ambos, entraron de
espaldas y cubrieron la desnudez de su padre. Sus rostros estaban al
revés, y no vieron la desnudez de su padre. \bibleverse{24} Noé despertó
de su vino y supo lo que su hijo menor le había hecho. \bibleverse{25}
Dijo, ``Canaán está maldito. Será siervo de los siervos de sus
hermanos''.

\bibleverse{26} Él dijo, ``Bendito sea Yahvé, el Dios de Sem. Que Canaán
sea su siervo. \footnote{\textbf{9:26} Rom 9,16} \bibleverse{27} Que
Dios engrandezca a Jafet. Que habite en las tiendas de Sem. Que Canaán
sea su siervo''. \footnote{\textbf{9:27} Efes 3,6}

\bibleverse{28} Noé vivió trescientos cincuenta años después del
diluvio. \bibleverse{29} Todos los días de Noé fueron novecientos
cincuenta años, y luego murió.

\hypertarget{los-descendientes-de-los-hijos-de-nouxe9}{%
\subsection{Los descendientes de los hijos de
Noé}\label{los-descendientes-de-los-hijos-de-nouxe9}}

\hypertarget{section-9}{%
\section{10}\label{section-9}}

\bibleverse{1} Esta es la historia de las generaciones de los hijos de
Noé y de Sem, Cam y Jafet. Les nacieron hijos después del diluvio.

\bibleverse{2} Los hijos de Jafet fueron: Gomer, Magog, Madai, Javan,
Tubal, Meshech y Tiras. \bibleverse{3} Los hijos de Gomer fueron:
Ashkenaz, Riphath y Togarmah. \bibleverse{4} Los hijos de Javán fueron:
Elishah, Tarsis, Kittim y Dodanim. \bibleverse{5} De éstos se dividieron
las islas de las naciones en sus tierras, cada uno según su lengua,
según sus familias, en sus naciones. \footnote{\textbf{10:5} Zac 2,11}

\bibleverse{6} Los hijos de Cam fueron: Cus, Mizraim, Put y Canaán.
\bibleverse{7} Los hijos de Cus fueron: Seba, Havilah, Sabtah, Raamah y
Sabteca. Los hijos de Raamah fueron: Sabá y Dedán. \bibleverse{8} Cus
fue el padre de Nimrod. Él comenzó a ser un poderoso en la tierra.
\bibleverse{9} Fue un poderoso cazador ante Yahvé. Por eso se dice:
``como Nimrod, un poderoso cazador ante Yahvé''. \bibleverse{10} El
principio de su reino fue Babel, Erec, Acad y Calneh, en la tierra de
Sinar. \bibleverse{11} De esa tierra pasó a Asiria y construyó Nínive,
Rehobot Ir, Calah, \footnote{\textbf{10:11} Jon 1,2} \bibleverse{12} y
Resen entre Nínive y la gran ciudad Calah. \bibleverse{13} Mizraim fue
el padre de Ludim, Anamim, Lehabim, Naphtuhim, \bibleverse{14}
Pathrusim, Casluhim (del que descienden los filisteos) y Caphtorim.

\bibleverse{15} Canaán fue padre de Sidón (su primogénito), de Het, de
\bibleverse{16} de los jebuseos, de los amorreos, de los gergeseos, de
\bibleverse{17} de los heveos, de los arquitas, de los sinitas, de
\bibleverse{18} de los arvaditas, de los zemaritas y de los hamateos.
Después, las familias de los cananeos se extendieron por el mundo.
\bibleverse{19} La frontera de los cananeos iba desde Sidón --- en
dirección a Gerar --- hasta Gaza --- en dirección a Sodoma, Gomorra,
Adma y Zeboiim --- hasta Lasa. \bibleverse{20} Estos son los hijos de
Cam, por sus familias, según sus lenguas, en sus tierras y sus naciones.

\bibleverse{21} También le nacieron hijos a Sem (el hermano mayor de
Jafet), el padre de todos los hijos de Éber. \footnote{\textbf{10:21}
  Gén 11,10} \bibleverse{22} Los hijos de Sem fueron: Elam, Asur,
Arpachshad, Lud y Aram. \bibleverse{23} Los hijos de Aram fueron: Uz,
Hul, Gether y Mash. \bibleverse{24} Arpachshad fue el padre de Shelah.
Sela fue el padre de Éber. \bibleverse{25} A Eber le nacieron dos hijos.
El nombre de uno fue Peleg, porque en sus días la tierra fue dividida.
El nombre de su hermano fue Joktán. \footnote{\textbf{10:25} Gén 11,8}
\bibleverse{26} Joktán fue padre de Almodad, Shelef, Hazarmaveth, Jerah,
\bibleverse{27} Hadoram, Uzal, Diklah, \bibleverse{28} Obal, Abimael,
Sheba, \bibleverse{29} Ophir, Havilah y Jobab. Todos ellos eran hijos de
Joktán. \bibleverse{30} Su morada se extendía desde Meshá, a medida que
se avanza hacia Sefar, la montaña del oriente. \bibleverse{31} Estos son
los hijos de Sem, por sus familias, según sus lenguas, tierras y
naciones.

\bibleverse{32} Estas son las familias de los hijos de Noé, por sus
generaciones, según sus naciones. Las naciones se dividieron de éstas en
la tierra después del diluvio. \footnote{\textbf{10:32} Gén 9,1; Gén
  9,19}

\hypertarget{la-torre-de-babel}{%
\subsection{La torre de Babel}\label{la-torre-de-babel}}

\hypertarget{section-10}{%
\section{11}\label{section-10}}

\bibleverse{1} Toda la tierra tenía una misma lengua y un mismo
lenguaje. \bibleverse{2} Mientras viajaban hacia el este,\footnote{\textbf{11:2}
  La LXX dice ``desde el este''.} encontraron una llanura en la tierra
de Sinar, y allí vivieron. \bibleverse{3} Se dijeron unos a otros:
``Venid, hagamos ladrillos y quemémoslos bien''. Tenían ladrillos por
piedra, y usaban alquitrán como mortero. \bibleverse{4} Dijeron:
``Vengan, construyamos una ciudad y una torre cuya cima llegue al cielo,
y hagamos un nombre para nosotros, no sea que nos dispersemos por la
superficie de toda la tierra.''

\bibleverse{5} Yahvé bajó a ver la ciudad y la torre que los hijos de
los hombres construyeron. \footnote{\textbf{11:5} Gén 18,21; Sal 18,9;
  Sal 14,2} \bibleverse{6} Yahvé dijo: ``He aquí que son un solo pueblo,
y todos tienen una sola lengua, y esto es lo que comienzan a hacer.
Ahora no se les impedirá nada de lo que pretenden hacer. \bibleverse{7}
Vamos, bajemos y confundamos allí su lengua, para que no entiendan el
habla de los demás.'' \bibleverse{8} Así que el Señor los dispersó desde
allí por la superficie de toda la tierra. Dejaron de construir la
ciudad. \footnote{\textbf{11:8} Luc 1,51} \bibleverse{9} Por eso su
nombre fue llamado Babel, porque allí Yahvé confundió el lenguaje de
toda la tierra. Desde allí, Yahvé los dispersó por la superficie de toda
la tierra.

\hypertarget{los-descendientes-de-sem}{%
\subsection{Los descendientes de Sem}\label{los-descendientes-de-sem}}

\bibleverse{10} Esta es la historia de las generaciones de Sem: Sem
tenía cien años cuando fue padre de Arpachshad dos años después del
diluvio. \footnote{\textbf{11:10} Gén 10,22; Luc 3,36} \bibleverse{11}
Sem vivió quinientos años después de ser padre de Arpachshad, y fue
padre de más hijos e hijas.

\bibleverse{12} Arpachshad vivió treinta y cinco años y llegó a ser el
padre de Shelah. \bibleverse{13} Arpachshad vivió cuatrocientos tres
años después de ser el padre de Shelah, y llegó a ser el padre de más
hijos e hijas.

\bibleverse{14} Selá vivió treinta años y fue padre de Éber.
\bibleverse{15} Selá vivió cuatrocientos tres años después de ser padre
de Éber, y fue padre de más hijos e hijas.

\bibleverse{16} Eber vivió treinta y cuatro años y fue padre de Peleg.
\bibleverse{17} Eber vivió cuatrocientos treinta años después de ser
padre de Peleg, y fue padre de más hijos e hijas.

\bibleverse{18} Peleg vivió treinta años y fue padre de Reu.
\bibleverse{19} Peleg vivió doscientos nueve años después de ser padre
de Reu, y fue padre de más hijos e hijas.

\bibleverse{20} Reu vivió treinta y dos años, y llegó a ser el padre de
Serug. \bibleverse{21} Reu vivió doscientos siete años después de ser
padre de Serug, y fue padre de más hijos e hijas.

\bibleverse{22} Serug vivió treinta años y llegó a ser padre de Nacor.
\bibleverse{23} Serug vivió doscientos años después de ser padre de
Nacor, y llegó a ser padre de más hijos e hijas.

\bibleverse{24} Nacor vivió veintinueve años, y llegó a ser padre de
Taré. \bibleverse{25} Nacor vivió ciento diecinueve años después de ser
padre de Taré, y llegó a ser padre de más hijos e hijas.

\bibleverse{26} Taré vivió setenta años y fue padre de Abram, Nacor y
Harán.

\hypertarget{los-descendientes-de-thare}{%
\subsection{Los descendientes de
Thare}\label{los-descendientes-de-thare}}

\bibleverse{27} Esta es la historia de las generaciones de Taré. Taré
fue el padre de Abram, Nacor y Harán. Harán fue el padre de Lot.
\bibleverse{28} Harán murió en su tierra natal, en Ur de los Caldeos,
mientras su padre Taré aún vivía. \bibleverse{29} Abram y Nacor se
casaron con esposas. El nombre de la esposa de Abram era Sarai, y el
nombre de la esposa de Nacor era Milca, hija de Harán, quien también era
el padre de Isca. \footnote{\textbf{11:29} Gén 22,20} \bibleverse{30}
Sarai era estéril. No tuvo ningún hijo. \footnote{\textbf{11:30} Jos
  24,2; Neh 9,7} \bibleverse{31} Taré tomó a Abram, su hijo, a Lot, hijo
de Harán, y a Sarai, su nuera, esposa de su hijo Abram. Salieron de Ur
de los Caldeos para ir a la tierra de Canaán. Llegaron a Harán y
vivieron allí. \bibleverse{32} Los días de Taré fueron doscientos cinco
años. Taré murió en Harán.

\hypertarget{el-llamado-de-abram}{%
\subsection{El llamado de Abram}\label{el-llamado-de-abram}}

\hypertarget{section-11}{%
\section{12}\label{section-11}}

\bibleverse{1} El Señor dijo a Abram: ``Deja tu país, tus parientes y la
casa de tu padre, y vete a la tierra que te mostraré. \footnote{\textbf{12:1}
  Hech 7,3; Heb 11,8} \bibleverse{2} Haré de ti una gran nación. Te
bendeciré y engrandeceré tu nombre. Serás una bendición. \footnote{\textbf{12:2}
  Gén 24,1; Gén 24,35; Sal 72,17} \bibleverse{3} Bendeciré a los que te
bendigan y maldeciré a los que te traten con desprecio. Todas las
familias de la tierra serán bendecidas por ti''. \footnote{\textbf{12:3}
  Éxod 23,22; Gén 18,18; Gén 22,18; Gén 26,4; Gén 28,14; Hech 3,25; Gal
  3,8}

\hypertarget{la-inmigraciuxf3n-de-abram-a-canauxe1n}{%
\subsection{La Inmigración de Abram a
Canaán}\label{la-inmigraciuxf3n-de-abram-a-canauxe1n}}

\bibleverse{4} Así que Abram se fue, como Yahvé le había dicho. Lot lo
acompañó. Abram tenía setenta y cinco años cuando partió de Harán.
\bibleverse{5} Abram tomó a Sarai, su esposa, a Lot, el hijo de su
hermano, todas las posesiones que habían reunido y el pueblo que habían
adquirido en Harán, y se fueron a la tierra de Canaán. Entraron en la
tierra de Canaán. \bibleverse{6} Abram pasó por la tierra hasta el lugar
de Siquem, hasta la encina de Moreh. En ese momento, los cananeos
estaban en la tierra.

\bibleverse{7} Yahvé se le apareció a Abram y le dijo: ``Le daré esta
tierra a tu descendencia''. \footnote{\textbf{12:7} o, semilla} Allí
construyó un altar a Yahvé, que se le había aparecido. \footnote{\textbf{12:7}
  Gén 13,15; Gén 15,18; Gén 17,8; Gén 24,7; Gén 26,3-4; Gén 28,13; Gén
  35,12; Éxod 6,4; Éxod 6,8; Éxod 32,13; Jos 21,43; Hech 7,5}
\bibleverse{8} Salió de allí para ir a la montaña que está al este de
Betel y acampó, teniendo Betel al oeste y Hai al este. Allí construyó un
altar a Yavé e invocó el nombre de Yavé. \footnote{\textbf{12:8} Gén
  4,26} \bibleverse{9} Abram siguió viajando, todavía en dirección al
sur.

\hypertarget{abram-y-sarai-en-egipto}{%
\subsection{Abram y Sarai en Egipto}\label{abram-y-sarai-en-egipto}}

\bibleverse{10} Hubo hambre en la tierra. Abram bajó a Egipto para vivir
como extranjero allí, porque el hambre era grave en la tierra.
\footnote{\textbf{12:10} Gén 20,1; Gén 26,1-11} \bibleverse{11} Cuando
estuvo a punto de entrar en Egipto, le dijo a Sarai, su esposa: ``Mira
ahora, sé que eres una mujer hermosa de ver. \bibleverse{12} Sucederá
que cuando los egipcios te vean, dirán: `Esta es su mujer'. A mí me
matarán, pero a ti te salvarán viva. \bibleverse{13} Por favor, di que
eres mi hermana, para que me vaya bien por ti y para que mi alma viva
gracias a ti.''

\bibleverse{14} Cuando Abram llegó a Egipto, los egipcios vieron que la
mujer era muy hermosa. \bibleverse{15} Los príncipes del faraón la
vieron y la alabaron ante el faraón; y la mujer fue llevada a la casa
del faraón. \bibleverse{16} Este trató bien a Abram por causa de ella.
Tuvo ovejas, ganado, asnos machos, siervos machos, siervas hembras,
asnos hembras y camellos. \bibleverse{17} El Señor afligió al faraón y a
su casa con grandes plagas a causa de Sarai, la esposa de Abram.
\footnote{\textbf{12:17} Sal 105,14} \bibleverse{18} El faraón llamó a
Abram y le dijo: ``¿Qué es esto que me has hecho? ¿Por qué no me dijiste
que era tu mujer? \bibleverse{19} ¿Por qué dijiste: `Es mi hermana',
para que la tomara por esposa? Ahora, pues, ve a tu mujer, tómala y
vete''.

\bibleverse{20} El faraón ordenó a los hombres que se ocuparan de él, y
lo escoltaron con su mujer y todo lo que tenía.

\hypertarget{el-regreso-de-abram}{%
\subsection{El regreso de Abram}\label{el-regreso-de-abram}}

\hypertarget{section-12}{%
\section{13}\label{section-12}}

\bibleverse{1} Abram subió de Egipto --- él, su mujer, todo lo que
tenía, y Lot con él --- al Sur. \bibleverse{2} Abram era muy rico en
ganado, en plata y en oro. \footnote{\textbf{13:2} Prov 10,22}
\bibleverse{3} Siguió su camino desde el sur hasta Betel, hasta el lugar
donde había estado su tienda al principio, entre Betel y Hai,
\bibleverse{4} hasta el lugar del altar que había hecho allí al
principio. Allí Abram invocó el nombre de Yahvé. \footnote{\textbf{13:4}
  Gén 12,8}

\hypertarget{abram-se-separa-de-lot}{%
\subsection{Abram se separa de Lot}\label{abram-se-separa-de-lot}}

\bibleverse{5} También Lot, que iba con Abram, tenía rebaños, vacas y
tiendas. \bibleverse{6} La tierra no podía soportarlos para que vivieran
juntos, pues sus posesiones eran tan grandes que no podían vivir juntos.
\bibleverse{7} Hubo disputas entre los pastores del ganado de Abram y
los pastores del ganado de Lot. Los cananeos y los ferezeos vivían
entonces en la tierra. \bibleverse{8} Abram le dijo a Lot: ``Por favor,
que no haya disputas entre tú y yo, y entre tus pastores y los míos,
porque somos parientes. \footnote{\textbf{13:8} Sal 133,1}
\bibleverse{9} ¿No está toda la tierra ante ti? Por favor, sepárense de
mí. Si te vas a la izquierda, yo me iré a la derecha. O si te vas a la
derecha, entonces yo me iré a la izquierda''.

\hypertarget{salida-de-lot-por-el-valle-del-jorduxe1n}{%
\subsection{Salida de Lot por el valle del
Jordán}\label{salida-de-lot-por-el-valle-del-jorduxe1n}}

\bibleverse{10} Lot alzó los ojos y vio toda la llanura del Jordán, que
estaba bien regada por todas partes, antes de que Yahvé destruyera a
Sodoma y Gomorra, como el jardín de Yahvé, como la tierra de Egipto, al
ir a Zoar. \bibleverse{11} Así que Lot eligió para sí la llanura del
Jordán. Lot viajó hacia el este, y se separaron el uno del otro.
\bibleverse{12} Abram vivió en la tierra de Canaán, y Lot vivió en las
ciudades de la llanura, y trasladó su tienda hasta Sodoma.
\bibleverse{13} Los hombres de Sodoma eran sumamente malvados y
pecadores contra el Señor. \footnote{\textbf{13:13} Gén 18,20; Gén
  19,4-9}

\hypertarget{dios-promete-a-abram-el-pauxeds-de-canuxe1n}{%
\subsection{Dios promete a Abram el país de
Canán}\label{dios-promete-a-abram-el-pauxeds-de-canuxe1n}}

\bibleverse{14} Yahvé dijo a Abram, después de que Lot se separó de él:
``Ahora, levanta tus ojos y mira desde el lugar donde estás, hacia el
norte y el sur y hacia el este y el oeste, \bibleverse{15} porque daré
toda la tierra que ves a ti y a tu descendencia para siempre.
\footnote{\textbf{13:15} Gén 12,7} \bibleverse{16} Haré que tu
descendencia sea como el polvo de la tierra, de modo que si un hombre
puede contar el polvo de la tierra, también tu descendencia podrá ser
contada. \footnote{\textbf{13:16} Gén 28,14; Núm 23,10} \bibleverse{17}
Levántate, recorre la tierra a lo largo y a lo ancho, porque yo te la
daré''.

\bibleverse{18} Abram trasladó su tienda y vino a vivir junto a las
encinas de Mambré, que están en Hebrón, y construyó allí un altar a
Yahvé. \footnote{\textbf{13:18} Gén 14,24}

\hypertarget{guerra-del-rey-kedorlaomer-en-el-valle-del-jordan}{%
\subsection{Guerra del rey Kedorlaomer en el valle del
Jordan}\label{guerra-del-rey-kedorlaomer-en-el-valle-del-jordan}}

\hypertarget{section-13}{%
\section{14}\label{section-13}}

\bibleverse{1} En los días de Amrafel, rey de Sinar; Arioch, rey de
Ellasar; Chedorlaomer, rey de Elam; y Tidal, rey de Goiim,
\bibleverse{2} hicieron la guerra a Bera, rey de Sodoma; a Birsha, rey
de Gomorra; a Shinab, rey de Adma; a Shemeber, rey de Zeboiim; y al rey
de Bela (también llamado Zoar). \footnote{\textbf{14:2} Deut 29,23}
\bibleverse{3} Todos ellos se unieron en el valle de Siddim (también
llamado Mar Salado). \bibleverse{4} Sirvieron a Quedorlaomer durante
doce años, y en el año trece se rebelaron. \bibleverse{5} En el
decimocuarto año Chedorlaomer y los reyes que estaban con él vinieron y
golpearon a los refaítas en Ashteroth Karnaim, a los zuzim en Ham, a los
emim en Shaveh Kiriathaim, \bibleverse{6} y a los horeos en su monte
Seir, hasta El Paran, que está junto al desierto. \bibleverse{7}
Volvieron y llegaron a En Mishpat (también llamado Cades), y atacaron
todo el país de los amalecitas, y también a los amorreos que vivían en
Hazazón Tamar. \bibleverse{8} Salieron el rey de Sodoma y el rey de
Gomorra, el rey de Adma, el rey de Zeboiim y el rey de Bela (también
llamada Zoar), y prepararon la batalla contra ellos en el valle de
Siddim \bibleverse{9} contra Chedorlaomer, rey de Elam, Tidal, rey de
Goiim, Amrafel, rey de Shinar, y Arioc, rey de Ellasar; cuatro reyes
contra los cinco. \bibleverse{10} El valle de Siddim estaba lleno de
pozos de brea, y los reyes de Sodoma y Gomorra huyeron, y algunos
cayeron allí. Los que quedaron huyeron a las colinas. \bibleverse{11}
Tomaron todos los bienes de Sodoma y Gomorra, y toda su comida, y se
fueron. \bibleverse{12} Tomaron a Lot, el hijo del hermano de Abram, que
vivía en Sodoma, y sus bienes, y se fueron. \footnote{\textbf{14:12} Gén
  13,10-12}

\hypertarget{ayuda-de-abram-por-lot}{%
\subsection{Ayuda de Abram por Lot}\label{ayuda-de-abram-por-lot}}

\bibleverse{13} Uno que había escapado vino y se lo contó a Abram, el
hebreo. En aquel tiempo, vivía junto a los robles de Mamre, el amorreo,
hermano de Escol y hermano de Aner. Eran aliados de Abram.
\bibleverse{14} Cuando Abram se enteró de que su pariente estaba
cautivo, sacó a sus trescientos dieciocho hombres entrenados, nacidos en
su casa, y los persiguió hasta Dan. \bibleverse{15} Se dividió contra
ellos de noche, él y sus siervos, y los atacó, y los persiguió hasta
Hoba, que está a la izquierda de Damasco. \bibleverse{16} Hizo volver
todos los bienes, y también hizo volver a su pariente Lot y sus bienes,
y también a las mujeres y a las demás personas.

\hypertarget{abram-encuentra-melchuxeesedec-rey-de-salem}{%
\subsection{Abram encuentra Melchîsedec, rey de
Salem}\label{abram-encuentra-melchuxeesedec-rey-de-salem}}

\bibleverse{17} El rey de Sodoma salió a recibirlo después de su regreso
de la matanza de Quedorlaomer y de los reyes que estaban con él, en el
valle de Shaveh (es decir, el Valle del Rey). \bibleverse{18}
Melquisedec, rey de Salem, sacó pan y vino. Era sacerdote del Dios
Altísimo. \footnote{\textbf{14:18} Sal 110,4; Heb 7,1-4; Sal 76,2}
\bibleverse{19} Lo bendijo y dijo: ``Bendito sea Abram del Dios
Altísimo, poseedor del cielo y de la tierra. \bibleverse{20} Bendito sea
el Dios Altísimo, que ha entregado a tus enemigos en tu mano''. Abram le
dio la décima parte de todo.

\hypertarget{la-humildad-de-abram-con-el-rey-de-sodoma}{%
\subsection{La humildad de Abram con el rey de
Sodoma}\label{la-humildad-de-abram-con-el-rey-de-sodoma}}

\bibleverse{21} El rey de Sodoma dijo a Abram: ``Dame la gente y toma
los bienes para ti''.

\bibleverse{22} Abram dijo al rey de Sodoma: ``He levantado mi mano a
Yahvé, Dios Altísimo, poseedor del cielo y de la tierra, \bibleverse{23}
que no tomaré ni un hilo ni una correa de sandalia ni nada que sea tuyo,
para que no digas: `Yo he enriquecido a Abram'. \bibleverse{24} No
aceptaré nada de ti, excepto lo que hayan comido los jóvenes y la
porción de los hombres que fueron conmigo: Aner, Escol y Mamre. Que
tomen su porción''.

\hypertarget{dios-promete-a-abram-un-hijo}{%
\subsection{Dios promete a Abram un
hijo}\label{dios-promete-a-abram-un-hijo}}

\hypertarget{section-14}{%
\section{15}\label{section-14}}

\bibleverse{1} Después de estas cosas, la palabra de Yahvé vino a Abram
en una visión, diciendo: ``No temas, Abram. Yo soy tu escudo, tu gran
recompensa''. \footnote{\textbf{15:1} Sal 3,3; Sal 84,11; Sal 119,114}

\bibleverse{2} Abram dijo: ``Señor\footnote{\textbf{15:2} La palabra
  traducida ``Señor'' es ``Adonai''.} Yahvé, ¿qué me darás, ya que me
voy sin hijos, y el que heredará mis bienes es Eliezer de Damasco?''
\bibleverse{3} Abram respondió: ``He aquí que no me has dado hijos, y he
aquí que uno nacido en mi casa es mi heredero.''

\bibleverse{4} He aquí que la palabra de Yahvé vino a él, diciendo:
``Este hombre no será tu heredero, pero el que saldrá de tu propio
cuerpo será tu heredero.'' \bibleverse{5} Yahvé lo sacó fuera y le dijo:
``Mira ahora hacia el cielo y cuenta las estrellas, si eres capaz de
contarlas''. Le dijo a Abram: ``Así será tu descendencia''. \footnote{\textbf{15:5}
  Gén 22,17; Éxod 32,13; Deut 1,10} \bibleverse{6} Él creyó en Yahvé,
que se lo acreditó por justicia. \footnote{\textbf{15:6} Rom 4,3-5; Rom
  4,18-22; Sant 2,23}

\hypertarget{dios-confirme-su-promesa}{%
\subsection{Dios confirme su promesa}\label{dios-confirme-su-promesa}}

\bibleverse{7} Le dijo a Abram: ``Yo soy Yahvé, que te saqué de Ur de
los caldeos, para darte esta tierra en herencia.'' \footnote{\textbf{15:7}
  Gén 11,31}

\bibleverse{8} Dijo: ``Señor Yahvé, ¿cómo sabré que lo heredaré?''
\footnote{\textbf{15:8} 2Re 20,8; Luc 1,18}

\bibleverse{9} Le dijo: ``Tráeme una ternera de tres años, una cabra de
tres años, un carnero de tres años, una tórtola y un pichón''.
\bibleverse{10} Él le trajo todo esto, lo dividió en el medio y puso
cada mitad frente a la otra; pero no dividió las aves. \footnote{\textbf{15:10}
  Jer 34,18-19} \bibleverse{11} Las aves de rapiña descendieron sobre
los cadáveres, y Abram las ahuyentó.

\bibleverse{12} Cuando el sol se ponía, un profundo sueño cayó sobre
Abram. El terror y la gran oscuridad cayeron sobre él. \footnote{\textbf{15:12}
  Job 4,13-14} \bibleverse{13} Le dijo a Abram: ``Ten por seguro que tu
descendencia vivirá como extranjera en una tierra que no es la suya, y
les servirá. Los afligirán durante cuatrocientos años. \footnote{\textbf{15:13}
  Éxod 12,40; Hech 7,6} \bibleverse{14} Yo también juzgaré a esa nación,
a la que servirán. Después saldrán con grandes riquezas; \footnote{\textbf{15:14}
  Éxod 3,21-22} \bibleverse{15} pero tú irás con tus padres en paz.
Serás enterrado a una buena edad. \bibleverse{16} En la cuarta
generación volverán a venir aquí, porque la iniquidad del amorreo aún no
está completa.'' \bibleverse{17} Sucedió que, cuando se puso el sol y
estuvo oscuro, he aquí que un horno humeante y una antorcha encendida
pasaron entre estas piezas. \bibleverse{18} Aquel día Yahvé hizo un
pacto con Abram, diciendo: ``He dado esta tierra a tu descendencia,
desde el río de Egipto hasta el gran río, el río Éufrates: \footnote{\textbf{15:18}
  Gén 12,7} \bibleverse{19} la tierra de los ceneos, de los cenecitas,
de los cadmonitas, \footnote{\textbf{15:19} Gén 10,15-18}
\bibleverse{20} de los hititas, de los ferezeos, de los refaítas,
\footnote{\textbf{15:20} Núm 13,33} \bibleverse{21} de los amorreos, de
los cananeos, de los gergeseos y de los jebuseos.''

\hypertarget{sarai-da-a-su-sierva-agar-como-mujer-uxe1-abram}{%
\subsection{Sarai da a su sierva Agar como mujer á
Abram}\label{sarai-da-a-su-sierva-agar-como-mujer-uxe1-abram}}

\hypertarget{section-15}{%
\section{16}\label{section-15}}

\bibleverse{1} Sarai, la esposa de Abram, no le dio hijos. Tenía una
sierva, una egipcia, que se llamaba Agar. \bibleverse{2} Sarai le dijo a
Abram: ``Mira ahora, Yahvé me ha impedido parir. Por favor, acude a mi
sierva. Puede ser que obtenga hijos de ella''. Abram escuchó la voz de
Sarai. \footnote{\textbf{16:2} Gén 30,3; Gén 30,9; 1Cor 7,2}
\bibleverse{3} Sarai, la esposa de Abram, tomó a Agar la egipcia, su
sierva, después de que Abram había vivido diez años en la tierra de
Canaán, y se la dio a Abram su esposo para que fuera su esposa.
\bibleverse{4} Él se acercó a Agar, y ella concibió. Al ver que había
concebido, su ama se despreció ante sus ojos. \bibleverse{5} Sarai dijo
a Abram: ``Este mal es culpa tuya. Entregué a mi sierva en tu seno, y
cuando vio que había concebido, me despreció. Que el Señor juzgue entre
tú y yo''.

\bibleverse{6} Pero Abram dijo a Sarai: ``He aquí que tu doncella está
en tu mano. Haz con ella lo que te parezca bien''. Sarai la trató con
dureza, y ella huyó de su rostro.

\hypertarget{dios-se-aparece-a-agar-uxe1-una-fuente-de-agua-en-el-desierto}{%
\subsection{Dios se aparece a Agar á una fuente de agua en el
desierto}\label{dios-se-aparece-a-agar-uxe1-una-fuente-de-agua-en-el-desierto}}

\bibleverse{7} El ángel de Yahvé la encontró junto a una fuente de agua
en el desierto, junto a la fuente del camino de Shur. \bibleverse{8} Le
dijo: ``Agar, sierva de Sarai, ¿de dónde vienes? ¿A dónde vas?'' Ella
dijo: ``Huyo de la cara de mi señora Sarai''.

\bibleverse{9} El ángel de Yahvé le dijo: ``Vuelve a tu señora y
sométete a sus manos''. \bibleverse{10} El ángel de Yahvé le dijo:
``Multiplicaré en gran medida tu descendencia, que no será contada como
multitud''. \footnote{\textbf{16:10} Gén 17,20} \bibleverse{11} El ángel
de Yahvé le dijo: ``He aquí que estás encinta y darás a luz un hijo. Lo
llamarás Ismael, porque Yahvé ha escuchado tu aflicción. \bibleverse{12}
Será como un asno salvaje entre los hombres. Su mano estará en contra de
todo hombre, y la mano de todo hombre en contra de él. Vivirá opuesto a
todos sus hermanos''. \footnote{\textbf{16:12} Gén 25,18}

\bibleverse{13} Ella llamó al nombre de Yahvé que le habló: ``Tú eres un
Dios que ve'', pues dijo: ``¿Acaso he quedado viva después de verlo?''
\bibleverse{14} Por eso el pozo se llamó Beer Lahai Roi.\footnote{\textbf{16:14}
  Beer Lahai Roi significa ``pozo del que vive y me ve''.} He aquí que
está entre Cades y Bered. \footnote{\textbf{16:14} Gén 24,62; Gén 25,11}

\bibleverse{15} Agar dio a luz un hijo para Abram. Abram llamó el nombre
de su hijo, que Agar dio a luz, Ismael. \bibleverse{16} Abram tenía
ochenta y seis años cuando Agar dio a luz a Ismael.

\hypertarget{dios-confirme-su-pacto-con-abram}{%
\subsection{Dios confirme su pacto con
Abram}\label{dios-confirme-su-pacto-con-abram}}

\hypertarget{section-16}{%
\section{17}\label{section-16}}

\bibleverse{1} Cuando Abram tenía noventa y nueve años, Yahvé se le
apareció y le dijo: ``Yo soy el Dios Todopoderoso. Camina delante de mí
y sé irreprochable. \footnote{\textbf{17:1} Gén 35,11; Éxod 6,3; Gén
  48,15} \bibleverse{2} Haré mi pacto entre yo y tú, y te multiplicaré
en gran manera''.

\bibleverse{3} Abram se postró sobre su rostro. Dios habló con él,
diciendo: \bibleverse{4} ``En cuanto a mí, he aquí que mi pacto es
contigo. Serás el padre de una multitud de naciones. \bibleverse{5} Ya
no te llamarás Abram, sino que tu nombre será Abraham, porque te he
hecho padre de una multitud de naciones. \footnote{\textbf{17:5} Rom
  4,11; Rom 4,17} \bibleverse{6} Te haré fructificar en gran medida, y
haré naciones de ti. De ti saldrán reyes. \bibleverse{7} Estableceré mi
pacto entre mí y tú y tu descendencia después de ti por sus
generaciones, como un pacto eterno, para ser un Dios para ti y para tu
descendencia después de ti. \bibleverse{8} Te daré a ti, y a tu
descendencia después de ti, la tierra por la que viajas, toda la tierra
de Canaán, como posesión eterna. Yo seré su Dios''. \footnote{\textbf{17:8}
  Gén 23,4; Gén 35,27; Heb 11,9-16}

\hypertarget{la-circuncision}{%
\subsection{La circuncision}\label{la-circuncision}}

\bibleverse{9} Dios dijo a Abraham: ``En cuanto a ti, guardarás mi
pacto, tú y tu descendencia después de ti por sus generaciones.
\bibleverse{10} Este es mi pacto, que guardarás, entre tú y yo y tu
descendencia después de ti. Todo varón de entre vosotros será
circuncidado. \footnote{\textbf{17:10} Lev 12,3; Hech 7,8}
\bibleverse{11} Será circuncidado en la carne de su prepucio. Será una
señal de la alianza entre mí y vosotros. \bibleverse{12} Será
circuncidado entre vosotros el que tenga ocho días de edad, todo varón a
lo largo de vuestras generaciones, el que haya nacido en la casa o haya
sido comprado con dinero a cualquier extranjero que no sea de vuestra
descendencia. \bibleverse{13} El que nazca en tu casa, y el que sea
comprado con tu dinero, debe ser circuncidado. Mi pacto estará en tu
carne como pacto eterno. \bibleverse{14} El varón incircunciso que no
esté circuncidado en la carne de su prepucio, esa alma será cortada de
su pueblo. Ha roto mi pacto''.

\hypertarget{dios-promete-abraham-un-hijo-de-sara}{%
\subsection{Dios promete Abraham un hijo de
Sara}\label{dios-promete-abraham-un-hijo-de-sara}}

\bibleverse{15} Dios dijo a Abraham: ``En cuanto a Sarai, tu mujer, no
la llamarás Sarai, sino que su nombre será Sara. \bibleverse{16} Yo la
bendeciré, y además te daré un hijo de ella. Sí, la bendeciré, y será
madre de naciones. De ella saldrán reyes de pueblos''.

\bibleverse{17} Entonces Abraham se postró sobre su rostro y se rió, y
dijo en su corazón: ``¿Le nacerá un hijo a quien tiene cien años? ¿Dará
a luz Sara, que tiene noventa años?'' \footnote{\textbf{17:17} Gén
  18,12; Gén 21,6; Luc 1,18} \bibleverse{18} Abraham dijo a Dios: ``¡Oh,
que Ismael viva ante ti!''

\bibleverse{19} Dios dijo: ``No, pero Sara, tu mujer, te dará un hijo.
Lo llamarás Isaac.\footnote{\textbf{17:19} Isaac significa ``se ríe''.}
Estableceré mi pacto con él como un pacto eterno para su descendencia
después de él. \footnote{\textbf{17:19} Gén 26,3} \bibleverse{20} En
cuanto a Ismael, te he escuchado. He aquí que lo he bendecido, lo haré
fructificar y lo multiplicaré en gran manera. Llegará a ser padre de
doce príncipes, y haré de él una gran nación. \footnote{\textbf{17:20}
  Gén 16,10; Gén 21,13; Gén 21,18; Gén 25,16} \bibleverse{21} Pero yo
estableceré mi alianza con Isaac, a quien Sara dará a luz en esta época
del año próximo.''

\bibleverse{22} Cuando terminó de hablar con él, Dios se alejó de
Abraham. \footnote{\textbf{17:22} Gén 35,13}

\hypertarget{abraham-realiza-la-circuncisiuxf3n}{%
\subsection{Abraham realiza la
circuncisión}\label{abraham-realiza-la-circuncisiuxf3n}}

\bibleverse{23} Abraham tomó a su hijo Ismael, a todos los nacidos en su
casa y a todos los comprados con su dinero; a todo varón de la casa de
Abraham, y circuncidó la carne de su prepucio en el mismo día, como Dios
le había dicho. \bibleverse{24} Abraham tenía noventa y nueve años
cuando fue circuncidado en la carne de su prepucio. \bibleverse{25}
Ismael, su hijo, tenía trece años cuando fue circuncidado en la carne de
su prepucio. \bibleverse{26} El mismo día fueron circuncidados Abraham e
Ismael, su hijo. \bibleverse{27} Todos los hombres de su casa, los
nacidos en ella y los comprados con dinero a un extranjero, fueron
circuncidados con él.

\hypertarget{dios-visita-a-abraham}{%
\subsection{Dios visita a Abraham}\label{dios-visita-a-abraham}}

\hypertarget{section-17}{%
\section{18}\label{section-17}}

\bibleverse{1} Yahvé se le apareció junto a los robles de Mambré,
mientras estaba sentado en la puerta de la tienda en el calor del día.
\bibleverse{2} Levantó los ojos y miró, y vio que tres hombres estaban
cerca de él. Al verlos, corrió a su encuentro desde la puerta de la
tienda, se inclinó hacia la tierra, \footnote{\textbf{18:2} Heb 13,2}
\bibleverse{3} y dijo: ``Señor mío, si ahora he encontrado gracia ante
tus ojos, por favor no te alejes de tu siervo. \bibleverse{4} Traigan
ahora un poco de agua, lávense los pies y descansen bajo el árbol.
\bibleverse{5} Yo traeré un trozo de pan para que refresquéis vuestro
corazón. Después podéis seguir vuestro camino, ya que habéis acudido a
vuestro siervo''. Dijeron: ``Muy bien, haz lo que has dicho''.

\bibleverse{6} Abraham se apresuró a entrar en la tienda con Sara y le
dijo: ``Prepara rápidamente tres seahs\footnote{\textbf{18:6} Zoar
  significa ``pequeño''.} de harina fina, amásala y haz tortas.''
\bibleverse{7} Abraham corrió hacia el rebaño, cogió un ternero tierno y
bueno y se lo dio al criado. Éste se apresuró a aderezarlo.
\bibleverse{8} Tomó mantequilla, leche y el ternero que había aderezado,
y lo puso delante de ellos. Se puso junto a ellos, bajo el árbol, y
comieron.

\bibleverse{9} Le preguntaron: ``¿Dónde está Sara, tu mujer?'' Dijo:
``Allí, en la tienda''.

\bibleverse{10} Dijo: ``Ciertamente volveré a ti por esta época el año
que viene, y he aquí que Sara, tu mujer, tendrá un hijo''. Sara oyó en
la puerta de la tienda, que estaba detrás de él. \footnote{\textbf{18:10}
  Gén 17,19; Rom 9,9} \bibleverse{11} Abraham y Sara eran viejos, de
edad avanzada. Sara había pasado la edad de tener hijos. \bibleverse{12}
Sara se reía en su interior, diciendo: ``¿Después de envejecer tendré
placer, siendo mi señor también viejo?'' \footnote{\textbf{18:12} Gén
  17,17; 1Pe 3,6}

\bibleverse{13} Yahvé dijo a Abraham: ``¿Por qué se rió Sara, diciendo:
``¿De verdad voy a dar a luz cuando sea vieja?'' \bibleverse{14} ¿Hay
algo demasiado difícil para Yahvé? A la hora fijada volveré a ti, cuando
llegue la estación, y Sara tendrá un hijo''. \footnote{\textbf{18:14}
  Luc 1,37}

\bibleverse{15} Entonces Sara lo negó, diciendo: ``No me he reído'',
pues tenía miedo. Me dijo: ``No, pero te reíste''.

\hypertarget{la-intercesiuxf3n-de-abraham-por-sodoma}{%
\subsection{La intercesión de Abraham por
Sodoma}\label{la-intercesiuxf3n-de-abraham-por-sodoma}}

\bibleverse{16} Los hombres se levantaron de allí y miraron hacia
Sodoma. Abraham fue con ellos para verlos en su camino. \bibleverse{17}
Yahvé dijo: ``¿Voy a ocultar a Abraham lo que hago, \bibleverse{18} ya
que Abraham llegará a ser una nación grande y poderosa, y todas las
naciones de la tierra serán bendecidas en él? \footnote{\textbf{18:18}
  Gén 12,3} \bibleverse{19} Porque lo he conocido, a fin de que mande a
sus hijos y a su casa después de él, para que guarden el camino de
Yahvé, haciendo justicia y rectitud; a fin de que Yahvé haga cumplir a
Abraham lo que ha dicho de él.'' \footnote{\textbf{18:19} Deut 6,7; Deut
  32,46} \bibleverse{20} Yahvé dijo: ``Porque el clamor de Sodoma y
Gomorra es grande, y porque su pecado es muy grave, \footnote{\textbf{18:20}
  Gén 19,13} \bibleverse{21} bajaré ahora y veré si sus obras son tan
malas como los informes que me han llegado. Si no es así, lo sabré''.
\footnote{\textbf{18:21} Gén 11,5; Sal 34,15-16}

\bibleverse{22} Los hombres se apartaron de allí y se dirigieron a
Sodoma, pero Abraham estaba todavía delante de Yahvé. \footnote{\textbf{18:22}
  Gén 19,1} \bibleverse{23} Abraham se acercó y dijo: ``¿Consumirás al
justo con el impío? \footnote{\textbf{18:23} Núm 16,22; 2Sam 24,17}
\bibleverse{24} ¿Y si hay cincuenta justos en la ciudad? ¿Consumirás y
no perdonarás el lugar por los cincuenta justos que están en ella?
\bibleverse{25} Que esté lejos de ti hacer cosas así, matar al justo con
el impío, para que el justo sea como el impío. Que eso esté lejos de ti.
¿No debería el Juez de toda la tierra hacer lo correcto?''

\bibleverse{26} Yahvé dijo: ``Si encuentro en Sodoma cincuenta justos
dentro de la ciudad, entonces perdonaré a todo el lugar por ellos''.
\footnote{\textbf{18:26} Is 65,8; Mat 24,22; Ezeq 22,30} \bibleverse{27}
Abraham respondió: ``Mira ahora, me he encargado de hablar con el Señor,
aunque soy polvo y ceniza. \bibleverse{28} ¿Y si faltan cinco de los
cincuenta justos? ¿Destruirás toda la ciudad por falta de cinco?'' Dijo:
``No lo destruiré si encuentro cuarenta y cinco allí''.

\bibleverse{29} Volvió a hablarle y le dijo: ``¿Y si se encuentran allí
cuarenta?''. Dijo: ``No lo haré por los cuarenta''.

\bibleverse{30} Él dijo: ``Oh, no dejes que el Señor se enoje, y yo
hablaré. ¿Y si se encuentran treinta allí?'' Dijo: ``No lo haré si
encuentro treinta allí''. \footnote{\textbf{18:30} Jue 6,39}

\bibleverse{31} Dijo: ``Mira ahora, me he encargado de hablar con el
Señor. ¿Y si se encuentran veinte allí?'' Dijo: ``No lo destruiré por el
bien de los veinte''.

\bibleverse{32} Él dijo: ``Oh, no dejes que el Señor se enoje, y hablaré
sólo una vez más. ¿Y si se encuentran diez allí?'' Dijo: ``No lo
destruiré por el bien de los diez''.

\bibleverse{33} El Señor se fue en cuanto terminó de hablar con Abraham,
y éste volvió a su lugar.

\hypertarget{la-cauxedda-de-sodoma-y-gomorrha}{%
\subsection{La caída de Sodoma y
Gomorrha}\label{la-cauxedda-de-sodoma-y-gomorrha}}

\hypertarget{section-18}{%
\section{19}\label{section-18}}

\bibleverse{1} Los dos ángeles llegaron a Sodoma al anochecer. Lot
estaba sentado en la puerta de Sodoma. Lot los vio y se levantó a
recibirlos. Se inclinó con el rostro hacia la tierra, \footnote{\textbf{19:1}
  Gén 18,22} \bibleverse{2} y les dijo: ``Vean ahora, señores míos, por
favor, entren en la casa de su siervo, quédense toda la noche, lávense
los pies, y podrán levantarse temprano y seguir su camino.'' Dijeron:
``No, pero nos quedaremos en la calle toda la noche''.

\bibleverse{3} Los exhortó mucho, y entraron con él en su casa. Les hizo
un banquete, y horneó panes sin levadura, y comieron. \bibleverse{4}
Pero antes de que se acostaran, los hombres de la ciudad, los hombres de
Sodoma, rodearon la casa, tanto los jóvenes como los ancianos, toda la
gente de todas partes. \bibleverse{5} Llamaron a Lot y le dijeron:
``¿Dónde están los hombres que entraron en tu casa esta noche?
Tráenoslos, para que nos acostemos con ellos''.

\bibleverse{6} Lot salió hacia ellos por la puerta, y cerró la puerta
tras de sí. \bibleverse{7} Dijo: ``Por favor, hermanos míos, no actuéis
con tanta maldad. \bibleverse{8} Mirad, tengo dos hijas vírgenes. Por
favor, dejad que os las traiga, y podéis hacer con ellas lo que os
parezca bien. Sólo que no les hagáis nada a estos hombres, porque han
venido bajo la sombra de mi techo''.

\bibleverse{9} Dijeron: ``¡Atrás!'' Entonces dijeron: ``Este tipo entró
a vivir como extranjero, y se nombra a sí mismo juez. Ahora te
trataremos peor que a ellos''. Presionaron con fuerza al hombre Lot, y
se acercaron para romper la puerta. \footnote{\textbf{19:9} 2Pe 2,7-8}
\bibleverse{10} Pero los hombres extendieron la mano y metieron a Lot en
la casa, y cerraron la puerta. \bibleverse{11} A los hombres que estaban
a la puerta de la casa los hirieron con ceguera, tanto a los pequeños
como a los grandes, de modo que se cansaron de encontrar la puerta.
\footnote{\textbf{19:11} 2Re 6,18}

\hypertarget{la-salvacion-de-lot}{%
\subsection{La salvacion de Lot}\label{la-salvacion-de-lot}}

\bibleverse{12} Los hombres dijeron a Lot: ``¿Tienes a alguien más aquí?
Yernos, hijos, hijas y todos los que tengas en la ciudad, sácalos del
lugar: \bibleverse{13} porque vamos a destruir este lugar, porque el
clamor contra ellos ha crecido tanto ante Yahvé que Yahvé nos ha enviado
a destruirlo.'' \footnote{\textbf{19:13} Gén 18,20}

\bibleverse{14} Lot salió y habló a sus yernos, que estaban
comprometidos a casarse con sus hijas, y les dijo: ``¡Levántense! Salid
de este lugar, porque Yahvé va a destruir la ciudad''. Pero a sus yernos
les pareció que estaba bromeando. \footnote{\textbf{19:14} Núm 16,21}
\bibleverse{15} Cuando llegó la mañana, los ángeles apresuraron a Lot,
diciendo: ``¡Levántate! Toma a tu mujer y a tus dos hijas que están
aquí, para que no te consumas en la iniquidad de la ciudad''.
\bibleverse{16} Pero él se demoró, y los hombres lo tomaron de la mano,
de la mano de su esposa y de la mano de sus dos hijas, siendo Yahvé
misericordioso con él, y lo sacaron y lo pusieron fuera de la ciudad.
\bibleverse{17} Cuando los sacaron, les dijo: ``¡Escapen por su vida! No
mires detrás de ti, y no te quedes en ningún lugar de la llanura.
Escapad a las montañas, no sea que os consuman''. \footnote{\textbf{19:17}
  Mat 24,16}

\bibleverse{18} Lot les dijo: ``Oh, no es así, mi señor. \bibleverse{19}
Mira ahora, tu siervo ha hallado gracia ante tus ojos, y has
engrandecido tu amorosa bondad, que has mostrado conmigo al salvar mi
vida. No puedo escapar al monte, no sea que el mal me alcance y muera.
\bibleverse{20} Mira ahora, esta ciudad está cerca para huir, y es
pequeña. Oh, déjame escapar allí (¿no es una pequeña?), y mi alma
vivirá''.

\bibleverse{21} Le dijo: ``He aquí que he concedido tu petición en
cuanto a esto también, que no derribaré la ciudad de la que has hablado.
\bibleverse{22} Date prisa, escapa allí, porque no puedo hacer nada
hasta que llegues''. Por eso el nombre de la ciudad se llamó Zoar.

\bibleverse{23} El sol había salido sobre la tierra cuando Lot llegó a
Zoar. \bibleverse{24} Entonces Yahvé hizo llover sobre Sodoma y sobre
Gomorra azufre y fuego de Yahvé desde el cielo. \footnote{\textbf{19:24}
  Deut 29,23; Sal 11,6; Am 4,11; Luc 17,29; 2Pe 2,6; Is 1,9-10; Is 13,19}
\bibleverse{25} Derribó aquellas ciudades, toda la llanura, todos los
habitantes de las ciudades y lo que crecía en el suelo. \bibleverse{26}
Pero la esposa de Lot miró hacia atrás desde su espalda, y se convirtió
en una columna de sal. \footnote{\textbf{19:26} Luc 17,32}

\bibleverse{27} Abraham subió de madrugada al lugar donde había estado
frente a Yavé. \bibleverse{28} Miró hacia Sodoma y Gomorra, y hacia toda
la tierra de la llanura, y vio que el humo de la tierra subía como el
humo de un horno.

\bibleverse{29} Cuando Dios destruyó las ciudades de la llanura, se
acordó de Abraham y envió a Lot en medio de la destrucción, cuando
derribó las ciudades en las que vivía Lot.

\hypertarget{el-pecado-de-las-hijas-de-lot-el-nacimiento-de-los-padres-de-los-moabitas-y-ammonitas}{%
\subsection{El pecado de las hijas de Lot; El nacimiento de los padres
de los Moabitas y
Ammonitas}\label{el-pecado-de-las-hijas-de-lot-el-nacimiento-de-los-padres-de-los-moabitas-y-ammonitas}}

\bibleverse{30} Lot subió de Zoar y vivió en el monte, con sus dos
hijas, porque tenía miedo de vivir en Zoar. Vivió en una cueva con sus
dos hijas. \bibleverse{31} La primogénita dijo a la menor: ``Nuestro
padre es viejo, y no hay hombre en la tierra que pueda entrar con
nosotras en el camino de toda la tierra. \bibleverse{32} Vengan, hagamos
que nuestro padre beba vino y nos acostaremos con él, para conservar el
linaje de nuestro padre''. \footnote{\textbf{19:32} Lev 18,7}
\bibleverse{33} Hicieron beber vino a su padre aquella noche, y la
primogénita entró y se acostó con su padre. Él no supo cuándo se acostó,
ni cuándo se levantó. \bibleverse{34} Al día siguiente, la primogénita
dijo a la menor: ``Mira, anoche me acosté con mi padre. Hagamos que esta
noche vuelva a beber vino. Entra tú y acuéstate con él, para que
conservemos el linaje de nuestro padre''. \bibleverse{35} También esa
noche hicieron beber vino a su padre. La más joven fue y se acostó con
él. Él no supo cuándo se acostó, ni cuándo se levantó. \bibleverse{36}
Así, las dos hijas de Lot quedaron embarazadas de su padre.
\bibleverse{37} La primogénita dio a luz un hijo y lo llamó Moab. Él es
el padre de los moabitas hasta el día de hoy. \footnote{\textbf{19:37}
  Deut 2,9} \bibleverse{38} La menor también dio a luz un hijo y lo
llamó Ben Ammi. Él es el padre de los hijos de Amón hasta el día de hoy.
\footnote{\textbf{19:38} Deut 2,19}

\hypertarget{abraham-donde-abimelech-en-gerar}{%
\subsection{Abraham donde Abimelech en
Gerar}\label{abraham-donde-abimelech-en-gerar}}

\hypertarget{section-19}{%
\section{20}\label{section-19}}

\bibleverse{1} Abraham viajó desde allí hacia la tierra del Sur, y vivió
entre Cades y Shur. Vivió como extranjero en Gerar. \footnote{\textbf{20:1}
  Gén 12,9-10; Gén 26,1} \bibleverse{2} Abraham dijo de su esposa Sara:
``Es mi hermana''. Abimelec, rey de Gerar, envió y tomó a Sara.
\bibleverse{3} Pero Dios vino a Abimelec en un sueño nocturno y le dijo:
``He aquí que eres un hombre muerto a causa de la mujer que has tomado,
porque es mujer de hombre.''

\bibleverse{4} Ahora bien, Abimelec no se había acercado a ella. Dijo:
``Señor, ¿vas a matar incluso a una nación justa? \bibleverse{5} ¿No me
dijo: `Es mi hermana'? Ella, incluso ella misma, dijo: `Es mi hermano'.
He hecho esto con la integridad de mi corazón y la inocencia de mis
manos''.

\bibleverse{6} Dios le dijo en el sueño: ``Sí, sé que en la integridad
de tu corazón has hecho esto, y también te he impedido pecar contra mí.
Por eso no te permití tocarla. \bibleverse{7} Ahora, pues, restituye a
ese hombre su mujer. Porque él es un profeta, y orará por ti, y vivirás.
Si no la restituyes, ten por seguro que morirás, tú y todos los tuyos''.
\footnote{\textbf{20:7} Sal 105,15}

\bibleverse{8} Abimelec se levantó de madrugada, llamó a todos sus
siervos y les dijo todo esto al oído. Los hombres estaban muy asustados.
\bibleverse{9} Entonces Abimelec llamó a Abraham y le dijo: ``¿Qué nos
has hecho? ¿Cómo he pecado contra ti, que has traído sobre mí y sobre mi
reino un gran pecado? Me has hecho obras que no debían hacerse''.
\bibleverse{10} Abimelec le dijo a Abraham: ``¿Qué has visto para que
hayas hecho esto?''

\bibleverse{11} Abraham dijo: ``Porque pensé: `Seguramente el temor de
Dios no está en este lugar. Me matarán por causa de mi mujer'.
\bibleverse{12} Además, ella es en verdad mi hermana, la hija de mi
padre, pero no la hija de mi madre; y se convirtió en mi esposa.
\bibleverse{13} Cuando Dios hizo que me alejara de la casa de mi padre,
le dije a ella: `Esta es la bondad que mostrarás conmigo. Dondequiera
que vayamos, di de mí: ``Es mi hermano''\,'\,''.

\bibleverse{14} Abimelec tomó ovejas y ganado, siervos y siervas, y se
los dio a Abraham, y le devolvió a Sara, su esposa. \bibleverse{15}
Abimelec dijo: ``Mira, mi tierra está delante de ti. Habita donde te
plazca''. \bibleverse{16} A Sara le dijo: ``He aquí que le he dado a tu
hermano mil monedas de plata. He aquí que es para ti una cubierta de los
ojos para todos los que están contigo. Delante de todos estás
reivindicada''.

\bibleverse{17} Abraham oró a Dios. Entonces Dios sanó a Abimelec, a su
esposa y a sus siervas, y éstas dieron a luz. \bibleverse{18} Porque
Yahvé había cerrado bien todos los vientres de la casa de Abimelec, a
causa de Sara, la mujer de Abraham.

\hypertarget{nacimiento-de-isaac}{%
\subsection{Nacimiento de Isaac}\label{nacimiento-de-isaac}}

\hypertarget{section-20}{%
\section{21}\label{section-20}}

\bibleverse{1} Yahvé visitó a Sara como había dicho, y Yahvé hizo con
Sara lo que había dicho. \footnote{\textbf{21:1} Gén 18,10}
\bibleverse{2} Sara concibió y dio a luz un hijo a Abraham en su vejez,
en el tiempo establecido del que Dios le había hablado. \footnote{\textbf{21:2}
  Heb 11,11} \bibleverse{3} Abraham llamó a su hijo que le había nacido,
y que Sara le dio a luz, Isaac. \footnote{\textbf{21:3} Isaac significa
  ``Él se ríe''.} \footnote{\textbf{21:3} Gén 17,19} \bibleverse{4}
Abraham circuncidó a su hijo Isaac a los ocho días de nacido, como Dios
le había ordenado. \footnote{\textbf{21:4} Gén 17,11-12} \bibleverse{5}
Abraham tenía cien años cuando le nació su hijo Isaac. \footnote{\textbf{21:5}
  Gén 17,17; Rom 4,19} \bibleverse{6} Sara dijo: ``Dios me ha hecho
reír. Todo el que oiga se reirá conmigo''. \footnote{\textbf{21:6} Gén
  18,12} \bibleverse{7} Ella dijo: ``¿Quién le habría dicho a Abraham
que Sara amamantaría a sus hijos? Pues le he dado a luz un hijo en su
vejez''.

\bibleverse{8} El niño creció y fue destetado. Abraham hizo una gran
fiesta el día en que Isaac fue destetado.

\hypertarget{el-repudio-y-la-salvaciuxf3n-de-ismael}{%
\subsection{El repudio y la salvación de
Ismael}\label{el-repudio-y-la-salvaciuxf3n-de-ismael}}

\bibleverse{9} Sara vio que el hijo de Agar la egipcia, que había dado a
luz a Abraham, se burlaba. \bibleverse{10} Entonces dijo a Abraham:
``¡Echa a esta sierva y a su hijo! Porque el hijo de esta sierva no será
heredero de mi hijo Isaac''. \footnote{\textbf{21:10} Gal 4,30}

\bibleverse{11} La cosa fue muy penosa a los ojos de Abraham a causa de
su hijo. \bibleverse{12} Dios le dijo a Abraham: ``No te aflijas por el
niño y por tu sierva. En todo lo que te diga Sara, escucha su voz.
Porque tu descendencia llevará el nombre de Isaac. \footnote{\textbf{21:12}
  Rom 9,7-8; Heb 11,18} \bibleverse{13} También haré una nación del hijo
de la sierva, porque es tu hijo.'' \footnote{\textbf{21:13} Gén 17,20}
\bibleverse{14} Abraham se levantó de madrugada, tomó pan y un
recipiente de agua y se lo dio a Agar, poniéndoselo al hombro; le dio el
niño y la despidió. Ella partió y anduvo errante por el desierto de
Beersheba. \bibleverse{15} El agua de la vasija se agotó, y ella puso al
niño debajo de uno de los arbustos. \bibleverse{16} Fue y se sentó
frente a él, a una buena distancia, como a un tiro de arco. Porque dijo:
``No me dejes ver la muerte del niño''. Se sentó frente a él, alzó la
voz y lloró. \bibleverse{17} Dios escuchó la voz del niño. El ángel de
Dios llamó a Agar desde el cielo y le dijo: ``¿Qué te preocupa, Agar? No
tengas miedo. Porque Dios ha escuchado la voz del niño donde está.
\bibleverse{18} Levántate, levanta al niño y sostenlo con tu mano.
Porque yo haré de él una gran nación''.

\bibleverse{19} Dios le abrió los ojos y vio un pozo de agua. Fue, llenó
el recipiente de agua y le dio de beber al niño.

\bibleverse{20} Dios estuvo con el niño, y éste creció. Vivió en el
desierto, y al crecer se convirtió en arquero. \bibleverse{21} Vivió en
el desierto de Parán. Su madre le consiguió una esposa de la tierra de
Egipto. \footnote{\textbf{21:21} Gén 16,3}

\hypertarget{el-pacto-entre-abraham-y-abimelech}{%
\subsection{El pacto entre Abraham y
Abimelech}\label{el-pacto-entre-abraham-y-abimelech}}

\bibleverse{22} En aquel tiempo, Abimelec y Ficol, el capitán de su
ejército, hablaron con Abraham, diciendo: ``Dios está contigo en todo lo
que haces. \footnote{\textbf{21:22} Gén 26,26} \bibleverse{23} Ahora,
pues, júrame aquí por Dios que no harás un trato falso conmigo, ni con
mi hijo, ni con el hijo de mi hijo. Sino que según la bondad que yo he
hecho contigo, tú harás conmigo y con la tierra en la que has vivido
como extranjero.'' \footnote{\textbf{21:23} Gén 20,15}

\bibleverse{24} Abraham dijo: ``Lo juraré''. \bibleverse{25} Abraham se
quejó a Abimelec a causa de un pozo de agua, que los siervos de Abimelec
habían quitado con violencia. \footnote{\textbf{21:25} Gén 26,15; Gén
  26,18} \bibleverse{26} Abimelec dijo: ``No sé quién ha hecho esto. No
me lo has dicho, y no me he enterado hasta hoy''.

\bibleverse{27} Abraham tomó ovejas y ganado y se los dio a Abimelec.
Aquellos dos hicieron un pacto. \bibleverse{28} Abraham puso siete
corderos del rebaño por separado. \bibleverse{29} Abimelec le dijo a
Abraham: ``¿Qué significan estas siete ovejas que has puesto solas?''

\bibleverse{30} Dijo: ``Tomarás estas siete ovejas de mi mano, para que
me sirvan de testimonio de que he cavado este pozo''. \bibleverse{31}
Por eso llamó a ese lugar Beersheba,\footnote{\textbf{21:31} Beersheba
  puede significar ``pozo del juramento'' o ``pozo de los siete''.}
porque ambos hicieron allí un juramento. \footnote{\textbf{21:31} Gén
  26,33} \bibleverse{32} Así que hicieron un pacto en Beerseba. Abimelec
se levantó con Ficol, el capitán de su ejército, y volvieron a la tierra
de los filisteos. \bibleverse{33} Abraham plantó un tamarisco en
Beerseba, y allí invocó el nombre de Yavé, el Dios eterno. \footnote{\textbf{21:33}
  Gén 12,8; Is 40,28; Rom 14,25} \bibleverse{34} Abraham vivió muchos
días como extranjero en la tierra de los filisteos.

\hypertarget{el-orden-de-dios-para-sacrificar-isaac}{%
\subsection{El orden de Dios para sacrificar
Isaac}\label{el-orden-de-dios-para-sacrificar-isaac}}

\hypertarget{section-21}{%
\section{22}\label{section-21}}

\bibleverse{1} Después de estas cosas, Dios probó a Abraham y le dijo:
``¡Abraham!'' Dijo: ``Aquí estoy''. \footnote{\textbf{22:1} Heb 11,17;
  Sant 1,12}

\bibleverse{2} Dijo: ``Ahora toma a tu hijo, tu único hijo, Isaac, a
quien amas, y vete a la tierra de Moriah. Ofrécelo allí como holocausto
en uno de los montes que te diré''. \footnote{\textbf{22:2} 2Cró 3,1}

\hypertarget{la-obedencia-de-abraham}{%
\subsection{La obedencia de Abraham}\label{la-obedencia-de-abraham}}

\bibleverse{3} Abraham se levantó de madrugada, ensilló su asno y tomó
consigo a dos de sus jóvenes y a su hijo Isaac. Partió la leña para el
holocausto, se levantó y se dirigió al lugar que Dios le había indicado.
\bibleverse{4} Al tercer día, Abraham alzó los ojos y vio el lugar a lo
lejos. \bibleverse{5} Abraham dijo a sus jóvenes: ``Quedaos aquí con el
burro. El muchacho y yo iremos allí. Adoraremos, y volveremos a ti''.
\bibleverse{6} Abraham tomó la madera del holocausto y la puso sobre
Isaac, su hijo. Tomó en su mano el fuego y el cuchillo. Ambos fueron
juntos. \bibleverse{7} Isaac se dirigió a su padre Abraham y le dijo:
``¿Padre mío?'' Dijo: ``Aquí estoy, hijo mío''. Dijo: ``Aquí está el
fuego y la leña, pero ¿dónde está el cordero para el holocausto?''.

\bibleverse{8} Abraham dijo: ``Dios se proveerá del cordero para el
holocausto, hijo mío''. Así que se fueron los dos juntos.

\hypertarget{la-preparacion-del-holocausto-y-la-intervenciuxf3n-de-dios}{%
\subsection{La preparacion del holocausto y la intervención de
Dios}\label{la-preparacion-del-holocausto-y-la-intervenciuxf3n-de-dios}}

\bibleverse{9} Llegaron al lugar que Dios le había indicado. Abraham
construyó allí el altar, y puso la madera en orden, ató a su hijo Isaac
y lo puso sobre el altar, sobre la madera. \bibleverse{10} Abraham
extendió su mano y tomó el cuchillo para matar a su hijo. \footnote{\textbf{22:10}
  Sant 2,21}

\bibleverse{11} El ángel de Yahvé le llamó desde el cielo y le dijo:
``¡Abraham, Abraham!'' Dijo: ``Aquí estoy''.

\bibleverse{12} Él dijo: ``No pongas tu mano sobre el niño ni le hagas
nada. Porque ahora sé que temes a Dios, ya que no me has ocultado a tu
hijo, tu único hijo''. \footnote{\textbf{22:12} Jer 7,31; Rom 8,32}

\bibleverse{13} Abraham alzó los ojos y miró, y vio que detrás de él
había un carnero atrapado en la espesura por sus cuernos. Abraham fue y
tomó el carnero, y lo ofreció en holocausto en lugar de su hijo.
\bibleverse{14} Abraham llamó el nombre de aquel lugar ``Yahvé
proveerá''.\footnote{\textbf{22:14} o, Yahvé-Jireh, o, Yahvé-Vista} Como
se dice hasta hoy: ``En el monte de Yahvé se proveerá''.

\hypertarget{la-aprobaciuxf3n-de-dios-y-promesas-por-abraham}{%
\subsection{La aprobación de Dios y promesas por
Abraham}\label{la-aprobaciuxf3n-de-dios-y-promesas-por-abraham}}

\bibleverse{15} El ángel de Yahvé llamó a Abraham por segunda vez desde
el cielo, \bibleverse{16} y le dijo: ``\,`He jurado por mí mismo', dice
Yahvé, `porque has hecho esto y no has retenido a tu hijo, tu único
hijo, \footnote{\textbf{22:16} Heb 6,13} \bibleverse{17} que te
bendeciré en gran manera, y multiplicaré tu descendencia en gran manera
como las estrellas del cielo y como la arena que está a la orilla del
mar. Tu descendencia poseerá la puerta de sus enemigos. \footnote{\textbf{22:17}
  Gén 13,16; Gén 15,5; Heb 11,12; Gén 24,60} \bibleverse{18} Todas las
naciones de la tierra serán bendecidas por tu descendencia, porque has
obedecido mi voz.'\,'' \footnote{\textbf{22:18} Gén 12,3; Gal 3,16}

\bibleverse{19} Entonces Abraham volvió con sus jóvenes, y se levantaron
y se fueron juntos a Beerseba. Abraham vivía en Beerseba.

\hypertarget{los-descendientes-de-nahor-el-hermano-de-abraham}{%
\subsection{Los descendientes de Nahor, el hermano de
Abraham}\label{los-descendientes-de-nahor-el-hermano-de-abraham}}

\bibleverse{20} Después de estas cosas, se le dijo a Abraham: ``He aquí
que Milca también ha dado a luz hijos a tu hermano Nacor: \footnote{\textbf{22:20}
  Gén 11,29} \bibleverse{21} Uz su primogénito, Buz su hermano, Kemuel
el padre de Aram, \bibleverse{22} Chesed, Hazo, Pildash, Jidlaph y
Betuel.'' \bibleverse{23} Betuel fue el padre de Rebeca. Estos ocho
Milcah dio a luz a Nahor, hermano de Abraham. \footnote{\textbf{22:23}
  Gén 24,15} \bibleverse{24} Su concubina, que se llamaba Reumah,
también dio a luz a Teba, Gaham, Tahash y Maacah.

\hypertarget{muerte-y-sepultura-de-sara}{%
\subsection{Muerte y sepultura de
Sara}\label{muerte-y-sepultura-de-sara}}

\hypertarget{section-22}{%
\section{23}\label{section-22}}

\bibleverse{1} Sara vivió ciento veintisiete años. Esta fue la duración
de la vida de Sara. \bibleverse{2} Sara murió en Quiriat Arba (también
llamada Hebrón), en la tierra de Canaán. Abraham vino a llorar a Sara y
a llorarla. \bibleverse{3} Abraham se levantó de entre sus muertos y
habló a los hijos de Het, diciendo: \bibleverse{4} ``Soy extranjero y
forastero y vivo con vosotros. Dadme posesión de un lugar de
enterramiento con vosotros, para que pueda enterrar a mis muertos fuera
de mi vista''. \footnote{\textbf{23:4} Gén 17,8}

\bibleverse{5} Los hijos de Het respondieron a Abraham, diciéndole:
\bibleverse{6} ``Escúchanos, mi señor. Tú eres un príncipe de Dios entre
nosotros. Entierra a tus muertos en la mejor de nuestras tumbas. Ninguno
de nosotros te negará su tumba. Entierra a tus muertos''.

\bibleverse{7} Abraham se levantó y se inclinó ante el pueblo de la
tierra, ante los hijos de Het. \bibleverse{8} Habló con ellos diciendo:
``Si estáis de acuerdo en que entierre a mis muertos fuera de mi vista,
escuchadme y rogad por mí a Efrón, hijo de Zohar, \bibleverse{9} para
que me venda la cueva de Macpela que tiene, que está en el extremo de su
campo. Por el precio completo que me la venda entre ustedes como
posesión para un lugar de entierro''.

\bibleverse{10} Efrón estaba sentado en medio de los hijos de Het. Efrón
el hitita respondió a Abraham a la vista de los hijos de Het, de todos
los que entraban por la puerta de su ciudad, diciendo: \bibleverse{11}
``No, señor mío, escúchame. Yo te doy el campo, y te doy la cueva que
hay en él. En presencia de los hijos de mi pueblo te lo doy. Entierra a
tus muertos''.

\bibleverse{12} Abraham se inclinó ante el pueblo de la tierra.
\bibleverse{13} Habló a Efrón en la audiencia del pueblo de la tierra,
diciendo Daré el precio del campo. Tómalo de mi parte, y enterraré allí
a mis muertos''.

\bibleverse{14} Efrón respondió a Abraham, diciéndole: \bibleverse{15}
``Señor mío, escúchame. ¿Qué es un pedazo de tierra que vale
cuatrocientos siclos de plata\footnote{\textbf{23:15} Un siclo equivale
  a unos 10 gramos, por lo que 400 siclos serían unos 4 kg. u 8,8
  libras.} entre tú y yo? Entierra, pues, a tus muertos''.

\bibleverse{16} Abraham escuchó a Efrón. Abraham pesó a Efrón la plata
que había nombrado al oír a los hijos de Het, cuatrocientos siclos de
plata, según el patrón de los mercaderes corrientes.

\bibleverse{17} Así que el campo de Efrón, que estaba en Macpela, que
estaba delante de Mamre, el campo, la cueva que había en él, y todos los
árboles que había en el campo, que estaban en todos sus límites, fueron
escriturados \bibleverse{18} a Abraham como posesión en presencia de los
hijos de Het, ante todos los que entraban a la puerta de su ciudad.
\bibleverse{19} Después de esto, Abraham enterró a Sara, su esposa, en
la cueva del campo de Macpela, frente a Mamre (es decir, Hebrón), en la
tierra de Canaán. \bibleverse{20} El campo y la cueva que hay en él
fueron legados a Abraham por los hijos de Het como posesión para un
lugar de enterramiento. \footnote{\textbf{23:20} Gén 25,9-10; Gén 47,30;
  Gén 49,29-30; Gén 50,13}

\hypertarget{abraham-envuxeda-a-su-criado-para-buscar-una-esposa-por-isaac}{%
\subsection{Abraham envía a su criado para buscar una esposa por
Isaac}\label{abraham-envuxeda-a-su-criado-para-buscar-una-esposa-por-isaac}}

\hypertarget{section-23}{%
\section{24}\label{section-23}}

\bibleverse{1} Abraham era viejo y de edad avanzada. Yahvé había
bendecido a Abraham en todo. \footnote{\textbf{24:1} Gén 12,2; Sal
  112,2-3} \bibleverse{2} Abraham dijo a su siervo, el mayor de su casa,
que gobernaba todo lo que tenía: ``Por favor, pon tu mano debajo de mi
muslo. \bibleverse{3} Te haré jurar por Yavé, el Dios de los cielos y el
Dios de la tierra, que no tomarás para mi hijo una esposa de las hijas
de los cananeos, entre los que vivo. \footnote{\textbf{24:3} Gén 28,1;
  Éxod 34,16} \bibleverse{4} Sino que irás a mi país y a mis parientes y
tomarás una esposa para mi hijo Isaac''.

\bibleverse{5} El criado le dijo: ``¿Y si la mujer no está dispuesta a
seguirme a esta tierra? ¿Debo traer a su hijo de nuevo a la tierra de la
que vino?''

\bibleverse{6} Abraham le dijo: ``Cuídate de no volver a llevar a mi
hijo allí. \bibleverse{7} Yahvé, el Dios del cielo, que me sacó de la
casa de mi padre y de la tierra donde nací, que me habló y me juró
diciendo: `Daré esta tierra a tu descendencia', enviará a su ángel
delante de ti, y tomarás de allí una mujer para mi hijo. \footnote{\textbf{24:7}
  Gén 12,1; Gén 12,7} \bibleverse{8} Si la mujer no está dispuesta a
seguirte, entonces quedarás libre de este juramento a mí. Sólo que no
volverás a llevar a mi hijo allí''.

\bibleverse{9} El siervo puso su mano bajo el muslo de Abraham, su amo,
y le juró sobre este asunto.

\hypertarget{la-viaje-del-criado-por-haran}{%
\subsection{La viaje del criado por
Haran}\label{la-viaje-del-criado-por-haran}}

\bibleverse{10} El siervo tomó diez de los camellos de su amo y partió,
llevando consigo una variedad de cosas buenas de su amo. Se levantó y
fue a Mesopotamia, a la ciudad de Najor. \footnote{\textbf{24:10} Gén
  11,31; Gén 27,43} \bibleverse{11} Hizo que los camellos se
arrodillaran fuera de la ciudad, junto al pozo de agua, a la hora del
atardecer, la hora en que las mujeres salen a sacar agua.
\bibleverse{12} Dijo: ``Yahvé, el Dios de mi amo Abraham, por favor dame
éxito hoy, y muestra bondad a mi amo Abraham. \bibleverse{13} He aquí
que estoy junto al manantial de agua. Las hijas de los hombres de la
ciudad están saliendo a sacar agua. \bibleverse{14} Que la joven a la
que le diga: ``Por favor, baja tu cántaro para que pueda beber'', y que
diga: ``Bebe, y yo también daré de beber a tus camellos'', sea la que
has designado para tu siervo Isaac. Así sabré que has sido amable con mi
señor''.

\bibleverse{15} Antes de que terminara de hablar, he aquí que salía
Rebeca, nacida de Betuel, hijo de Milca, mujer de Nacor, hermano de
Abraham, con su cántaro al hombro. \footnote{\textbf{24:15} Gén 22,23}
\bibleverse{16} La joven era muy hermosa de ver, virgen. Ningún hombre
la había conocido. Bajó a la fuente, llenó su cántaro y subió.
\bibleverse{17} El criado corrió a su encuentro y le dijo: ``Por favor,
dame de beber, un poco de agua de tu cántaro''.

\bibleverse{18} Ella dijo: ``Bebe, mi señor''. Ella se apresuró a bajar
el cántaro de su mano y le dio de beber. \bibleverse{19} Cuando terminó
de darle de beber, dijo: ``Yo también sacaré para tus camellos, hasta
que terminen de beber.'' \bibleverse{20} Ella se apresuró a vaciar su
cántaro en el abrevadero, y corrió de nuevo al pozo para sacar, y sacó
para todos sus camellos.

\bibleverse{21} El hombre la miró fijamente, permaneciendo en silencio,
para saber si Yahvé había hecho próspero su viaje o no.

\hypertarget{el-criado-llega-a-la-casa-de-nachuxf4r}{%
\subsection{El criado llega a la casa de
Nachôr}\label{el-criado-llega-a-la-casa-de-nachuxf4r}}

\bibleverse{22} Cuando los camellos terminaron de beber, el hombre tomó
un anillo de oro de medio siclo de peso,\footnote{\textbf{24:22} Un
  siclo equivale a unos 10 gramos o a unas 0,35 onzas.} y dos brazaletes
para sus manos de diez siclos de peso de oro, \bibleverse{23} y dijo:
``¿De quién eres hija? Por favor, dime. ¿Hay sitio en la casa de tu
padre para que nos quedemos?''

\bibleverse{24} Ella le dijo: ``Soy hija de Betuel, hijo de Milca, que
dio a luz a Nacor''. \bibleverse{25} Además, le dijo: ``Tenemos paja y
alimento suficientes, y espacio para alojarnos''.

\bibleverse{26} El hombre inclinó la cabeza y adoró a Yavé.
\bibleverse{27} Dijo: ``Bendito sea Yavé, el Dios de mi amo Abraham, que
no ha abandonado su bondad y su verdad para con mi amo. En cuanto a mí,
Yahvé me ha conducido por el camino a la casa de los parientes de mi
amo''.

\bibleverse{28} La joven corrió y contó estas palabras a la casa de su
madre. \bibleverse{29} Rebeca tenía un hermano que se llamaba Labán.
Labán salió corriendo hacia el hombre, hacia la fuente. \bibleverse{30}
Cuando vio el anillo y los brazaletes en las manos de su hermana, y
cuando oyó las palabras de su hermana Rebeca, diciendo: ``Esto es lo que
me ha dicho el hombre'', se acercó al hombre. He aquí que él estaba
junto a los camellos en el manantial. \bibleverse{31} Le dijo: ``Entra,
bendito de Yahvé. ¿Por qué te quedas fuera? Porque he preparado la casa
y el espacio para los camellos''.

\bibleverse{32} El hombre entró en la casa y descargó los camellos. Dio
paja y pienso para los camellos, y agua para lavar sus pies y los de los
hombres que le acompañaban. \bibleverse{33} Se le puso comida para que
comiera, pero él dijo: ``No comeré hasta que haya dicho mi mensaje''.
Labán dijo: ``Habla''.

\hypertarget{el-cortejo-por-la-novia}{%
\subsection{El cortejo por la novia}\label{el-cortejo-por-la-novia}}

\bibleverse{34} Él dijo: ``Yo soy el siervo de Abraham. \bibleverse{35}
El Señor ha bendecido mucho a mi amo. Se ha hecho grande. El Señor le ha
dado rebaños y manadas, plata y oro, siervos y siervas, camellos y
asnos. \bibleverse{36} Sara, la esposa de mi amo, le dio un hijo a mi
amo cuando ya era viejo. Le ha dado todo lo que tiene. \bibleverse{37}
Mi amo me hizo jurar, diciendo: `No tomarás mujer para mi hijo de entre
las hijas de los cananeos, en cuya tierra vivo, \bibleverse{38} sino que
irás a la casa de mi padre y de mis parientes y tomarás mujer para mi
hijo.' \bibleverse{39} Yo le pregunté a mi amo: `¿Y si la mujer no me
sigue?' \bibleverse{40} Él me dijo: `El Señor, ante quien yo ando,
enviará su ángel contigo y prosperará tu camino. Tomarás una mujer para
mi hijo de entre mis parientes y de la casa de mi padre. \footnote{\textbf{24:40}
  Gén 17,1} \bibleverse{41} Así quedarás libre de mi juramento, cuando
llegues a mis parientes. Si no te la dan, quedarás libre de mi
juramento'. \bibleverse{42} Vine hoy al manantial y dije: `Yahvé, el
Dios de mi amo Abraham, si ahora haces prosperar mi camino que voy ---
\bibleverse{43} he aquí que estoy junto a este manantial de agua. Que la
doncella que salga a sacar, a la que yo le diga: ``Por favor, dame un
poco de agua de tu cántaro para que beba'', \bibleverse{44} entonces me
diga: ``Bebe, y yo también sacaré para tus camellos'', sea la mujer que
Yahvé ha designado para el hijo de mi amo.' \bibleverse{45} Antes de que
terminara de hablar en mi corazón, he aquí que Rebeca salió con su
cántaro al hombro. Bajó al manantial y sacó. Le dije: `Por favor, déjame
beber'. \bibleverse{46} Ella se apresuró a bajar el cántaro de su hombro
y dijo: `Bebe, y yo también daré de beber a tus camellos'. Así que bebí,
y ella también dio de beber a los camellos. \bibleverse{47} Le pregunté:
``¿De quién eres hija? Ella respondió: `La hija de Betuel, hijo de
Nacor, que le dio Milca'. Le puse el anillo en la nariz y los brazaletes
en las manos. \bibleverse{48} Incliné la cabeza, adoré a Yavé y bendije
a Yavé, el Dios de mi amo Abraham, que me había guiado por el camino
correcto para tomar a la hija del hermano de mi amo para su hijo.
\bibleverse{49} Ahora bien, si tú tratas con bondad y verdad a mi amo,
dímelo. Si no, dímelo, para que me vuelva a la derecha o a la
izquierda''.

\hypertarget{la-despedida-de-rebeca}{%
\subsection{La despedida de Rebeca}\label{la-despedida-de-rebeca}}

\bibleverse{50} Entonces Labán y Betuel respondieron: ``La cosa procede
de Yahvé. No podemos hablarte ni mal ni bien. \bibleverse{51} He aquí
que Rebeca está delante de ti. Tómenla y váyanse, y que sea la esposa
del hijo de su amo, como ha dicho Yahvé''.

\bibleverse{52} Cuando el siervo de Abraham escuchó sus palabras, se
postró en tierra ante Yahvé. \bibleverse{53} El siervo sacó joyas de
plata, joyas de oro y ropa, y se las dio a Rebeca. También dio cosas
preciosas a su hermano y a su madre. \bibleverse{54} Comieron y
bebieron, él y los hombres que estaban con él, y se quedaron toda la
noche. Se levantaron por la mañana, y él dijo: ``Envíenme a mi amo''.

\bibleverse{55} Su hermano y su madre dijeron: ``Que la joven se quede
con nosotros unos días, al menos diez. Después se irá''.

\bibleverse{56} Él les dijo: ``No me estorben, pues Yahvé ha prosperado
mi camino. Despídanme para que vaya con mi amo''.

\bibleverse{57} Dijeron: ``Llamaremos a la joven y le preguntaremos''.
\bibleverse{58} Llamaron a Rebeca y le dijeron: ``¿Quieres ir con este
hombre?'' Ella dijo: ``Iré''.

\bibleverse{59} Despidieron a Rebeca, su hermana, con su nodriza, el
siervo de Abraham, y sus hombres. \bibleverse{60} Bendijeron a Rebeca y
le dijeron: ``Hermana nuestra, que seas madre de miles de diez mil, y
que tu descendencia posea la puerta de los que la odian.'' \footnote{\textbf{24:60}
  Gén 22,17}

\bibleverse{61} Rebeca se levantó con sus damas. Montaron en los
camellos y siguieron al hombre. El siervo tomó a Rebeca y siguió su
camino.

\hypertarget{la-llegada-de-la-novia-al-novio}{%
\subsection{La llegada de la novia al
novio}\label{la-llegada-de-la-novia-al-novio}}

\bibleverse{62} Isaac venía del camino de Beer Lahai Roi, pues vivía en
la tierra del Sur. \footnote{\textbf{24:62} Gén 16,14; Gén 25,11}
\bibleverse{63} Isaac salió a meditar en el campo al atardecer. Levantó
sus ojos y miró. He aquí que venían camellos. \bibleverse{64} Rebeca
levantó los ojos y, al ver a Isaac, se bajó del camello. \bibleverse{65}
Dijo al criado: ``¿Quién es el hombre que viene al campo a recibirnos?''
El criado dijo: ``Es mi amo''. Tomó su velo y se cubrió. \bibleverse{66}
El siervo le contó a Isaac todo lo que había hecho. \bibleverse{67}
Isaac la llevó a la tienda de su madre Sara, y tomó a Rebeca, y ella se
convirtió en su esposa. Él la amaba. Así que Isaac se sintió
reconfortado después de la muerte de su madre. \footnote{\textbf{24:67}
  Gén 23,2}

\hypertarget{segundo-matrimonio-de-abraham-su-muerte-y-entierro}{%
\subsection{Segundo matrimonio de Abraham; su muerte y
entierro}\label{segundo-matrimonio-de-abraham-su-muerte-y-entierro}}

\hypertarget{section-24}{%
\section{25}\label{section-24}}

\bibleverse{1} Abraham tomó otra esposa, que se llamaba Cetura.
\bibleverse{2} Ella le dio a luz a Zimran, Jokshan, Medan, Midian,
Ishbak y Shuah. \bibleverse{3} Jokshan fue el padre de Sheba y de Dedan.
Los hijos de Dedán fueron Assurim, Letushim y Leummim. \bibleverse{4}
Los hijos de Madián fueron Efá, Efer, Hanoc, Abida y Eldaá. Todos ellos
eran hijos de Cetura. \bibleverse{5} Abraham dio todo lo que tenía a
Isaac, \bibleverse{6} pero Abraham dio regalos a los hijos de las
concubinas de Abraham. Mientras él vivía, los envió lejos de su hijo
Isaac, hacia el este, al país oriental. \bibleverse{7} Estos son los
días de los años que vivió Abraham: ciento setenta y cinco años.
\bibleverse{8} Abraham renunció a su espíritu y murió en buena edad,
anciano y lleno de años, y fue reunido con su pueblo. \footnote{\textbf{25:8}
  Gén 15,15; Job 5,26} \bibleverse{9} Isaac e Ismael, sus hijos, lo
enterraron en la cueva de Macpela, en el campo de Efrón, hijo de Zohar
el hitita, que está cerca de Mamre, \bibleverse{10} el campo que Abraham
compró a los hijos de Het. Abraham fue enterrado allí con Sara, su
esposa. \footnote{\textbf{25:10} Gén 23,16-17} \bibleverse{11} Después
de la muerte de Abraham, Dios bendijo a Isaac, su hijo. Isaac vivía en
Beer Lahai Roi. \footnote{\textbf{25:11} Gén 24,62}

\hypertarget{los-descendientes-de-ismael}{%
\subsection{Los descendientes de
Ismael}\label{los-descendientes-de-ismael}}

\bibleverse{12} Esta es la historia de las generaciones de Ismael, hijo
de Abraham, que Agar la egipcia, sierva de Sara, dio a luz a Abraham.
\footnote{\textbf{25:12} Gén 21,13} \bibleverse{13} Estos son los
nombres de los hijos de Ismael, por sus nombres, según el orden de su
nacimiento: el primogénito de Ismael, Nebaiot, luego Cedar, Adbeel,
Mibsam, \bibleverse{14} Mishma, Dumah, Massa, \bibleverse{15} Hadad,
Tema, Jetur, Nafis y Cedemah. \bibleverse{16} Estos son los hijos de
Ismael, y estos son sus nombres, por sus pueblos y por sus campamentos:
doce príncipes, según sus naciones. \footnote{\textbf{25:16} Gén 17,20}
\bibleverse{17} Estos son los años de la vida de Ismael: ciento treinta
y siete años. Entregó su espíritu y murió, y fue reunido con su pueblo.
\bibleverse{18} Vivió desde Havila hasta Shur, que está delante de
Egipto, en dirección a Asiria. Vivió frente a todos sus parientes.
\footnote{\textbf{25:18} Gén 16,12}

\hypertarget{el-nacimiento-de-esau-y-jacob}{%
\subsection{El nacimiento de Esau y
Jacob}\label{el-nacimiento-de-esau-y-jacob}}

\bibleverse{19} Esta es la historia de las generaciones de Isaac, hijo
de Abraham. Abraham fue el padre de Isaac. \bibleverse{20} Isaac tenía
cuarenta años cuando tomó por esposa a Rebeca, hija de Betuel el sirio
de Paddán Aram, hermana de Labán el sirio. \bibleverse{21} Isaac suplicó
a Yahvé por su esposa, porque era estéril. Yahvé fue suplicado por él, y
Rebeca, su esposa, concibió. \bibleverse{22} Los hijos lucharon juntos
dentro de ella. Ella dijo: ``Si es así, ¿para qué vivo?''. Fue a
consultar a Yahvé. \bibleverse{23} Yahvé le dijo, ``Dos naciones están
en tu vientre. Dos personas serán separadas de su cuerpo. Un pueblo será
más fuerte que el otro. El mayor servirá al menor''. \footnote{\textbf{25:23}
  Gén 27,29; Mal 1,2; Rom 9,10-12}

\bibleverse{24} Cuando se cumplieron sus días de parto, he aquí que
había gemelos en su vientre. \bibleverse{25} El primero salió rojo por
todas partes, como una prenda velluda. Le pusieron el nombre de Esaú.
\bibleverse{26} Después salió su hermano, y su mano se aferró al talón
de Esaú. Le pusieron el nombre de Jacob. Isaac tenía sesenta años cuando
los dio a luz.

\bibleverse{27} Los muchachos crecieron. Esaú era un hábil cazador, un
hombre de campo. Jacob era un hombre tranquilo, que vivía en tiendas.
\bibleverse{28} Isaac amaba a Esaú, porque comía su carne de venado.
Rebeca amaba a Jacob.

\hypertarget{jacob-compra-la-primogenitura-de-esauxfa}{%
\subsection{Jacob compra la primogenitura de
Esaú}\label{jacob-compra-la-primogenitura-de-esauxfa}}

\bibleverse{29} Jacob hervía un guiso. Esaú llegó del campo, y estaba
hambriento. \bibleverse{30} Esaú le dijo a Jacob: ``Por favor,
aliméntame con un poco de ese guiso rojo, porque estoy hambriento''. Por
eso se llamó Edom. \footnote{\textbf{25:30} ``Edom'' significa ``rojo''.}

\bibleverse{31} Jacob dijo: ``Primero, véndeme tu primogenitura''.

\bibleverse{32} Esaú dijo: ``He aquí que estoy a punto de morir. ¿De qué
me sirve la primogenitura?''

\bibleverse{33} Jacob dijo: ``Júrame primero''. Se lo juró. Vendió su
primogenitura a Jacob. \footnote{\textbf{25:33} Gén 27,36; Heb 12,16}
\bibleverse{34} Jacob dio a Esaú pan y guiso de lentejas. Comió y bebió,
se levantó y siguió su camino. Entonces Esaú despreció su primogenitura.

\hypertarget{isaac-se-muda-a-gerar-cuando-hay-hambre}{%
\subsection{Isaac se muda a Gerar cuando hay
hambre}\label{isaac-se-muda-a-gerar-cuando-hay-hambre}}

\hypertarget{section-25}{%
\section{26}\label{section-25}}

\bibleverse{1} Hubo una hambruna en la tierra, además de la primera
hambruna que hubo en los días de Abraham. Isaac fue a Abimelec, rey de
los filisteos, a Gerar. \footnote{\textbf{26:1} Gén 12,10; Gén 20,2}
\bibleverse{2} Yahvé se le apareció y le dijo: ``No bajes a Egipto. Vive
en la tierra de la que te hablaré. \bibleverse{3} Vive en esta tierra, y
yo estaré contigo y te bendeciré. Porque te daré a ti y a tu
descendencia todas estas tierras, y confirmaré el juramento que le hice
a Abraham, tu padre. \footnote{\textbf{26:3} Gén 12,7; Gén 22,16}
\bibleverse{4} Multiplicaré tu descendencia como las estrellas del
cielo, y daré todas estas tierras a tu descendencia. En tu descendencia
serán bendecidas todas las naciones de la tierra, \footnote{\textbf{26:4}
  Gén 15,5; Gén 12,3} \bibleverse{5} porque Abraham obedeció mi voz y
guardó mis requerimientos, mis mandamientos, mis estatutos y mis
leyes.''

\bibleverse{6} Isaac vivía en Gerar. \bibleverse{7} Los hombres del
lugar le preguntaron por su esposa. Él respondió: ``Es mi hermana'',
pues temía decir: ``Mi esposa'', no sea que, pensó, ``los hombres del
lugar me maten por Rebeca, porque es hermosa de ver''. \bibleverse{8}
Cuando ya llevaba mucho tiempo allí, Abimelec, rey de los filisteos, se
asomó a una ventana y vio que Isaac estaba acariciando a Rebeca, su
esposa. \footnote{\textbf{26:8} Prov 5,18} \bibleverse{9} Abimelec llamó
a Isaac y le dijo: ``He aquí que ella es tu mujer. ¿Por qué has dicho:
`Es mi hermana'?'' Isaac le respondió: ``Porque dije: ``No sea que muera
por su culpa''\,''.

\bibleverse{10} Abimelec dijo: ``¿Qué es lo que nos has hecho? Uno del
pueblo podría haberse acostado fácilmente con tu mujer, ¡y nos habrías
hecho caer la culpa!''

\bibleverse{11} Abimelec ordenó a todo el pueblo que dijera: ``El que
toque a este hombre o a su mujer, morirá''.

\hypertarget{la-creciente-riqueza-de-isaac-disputas-de-fuentes}{%
\subsection{La creciente riqueza de Isaac; Disputas de
fuentes;}\label{la-creciente-riqueza-de-isaac-disputas-de-fuentes}}

\bibleverse{12} Isaac sembró en esa tierra y cosechó en el mismo año
cien veces lo que había plantado. El Señor lo bendijo. \footnote{\textbf{26:12}
  Prov 10,22} \bibleverse{13} El hombre se hizo grande, y creció más y
más hasta llegar a ser muy grande. \bibleverse{14} Tenía posesiones de
rebaños, posesiones de manadas y una gran casa. Los filisteos lo
envidiaban. \bibleverse{15} Ahora bien, todos los pozos que los siervos
de su padre habían cavado en los días de Abraham, su padre, los
filisteos los habían cerrado y llenado de tierra. \footnote{\textbf{26:15}
  Gén 21,25} \bibleverse{16} Abimelec dijo a Isaac: ``Vete de nosotros,
porque eres mucho más poderoso que nosotros''.

\bibleverse{17} Isaac partió de allí, acampó en el valle de Gerar y
vivió allí.

\bibleverse{18} Isaac volvió a cavar los pozos de agua que habían cavado
en los días de Abraham, su padre, pues los filisteos los habían detenido
después de la muerte de Abraham. Les puso los nombres que su padre les
había puesto. \bibleverse{19} Los siervos de Isaac cavaron en el valle y
encontraron allí un pozo de agua que fluía.\footnote{\textbf{26:19} O,
  fresco.} \bibleverse{20} Los pastores de Gerar discutieron con los
pastores de Isaac, diciendo: ``El agua es nuestra''. Así que llamó el
nombre del pozo Esek,\footnote{\textbf{26:20} ``Esek'' significa
  ``contención''.} porque discutían con él. \bibleverse{21} Ellos
cavaron otro pozo, y también discutieron por él. Así que lo llamó
Sitnah. \footnote{\textbf{26:21} ``Sitnah'' significa ``hostilidad''.}
\bibleverse{22} Dejó ese lugar y cavó otro pozo. No discutieron por ese.
Así que lo llamó Rehobot.\footnote{\textbf{26:22} ``Rehoboth'' significa
  ``lugares amplios''.} Dijo: ``Porque ahora el Señor nos ha hecho un
lugar, y seremos fructíferos en la tierra''.

\bibleverse{23} De allí subió a Beerseba. \bibleverse{24} Esa misma
noche se le apareció el Señor y le dijo: ``Yo soy el Dios de Abraham, tu
padre. No temas, porque yo estoy contigo y te bendeciré y multiplicaré
tu descendencia por amor a mi siervo Abraham.''

\bibleverse{25} Allí construyó un altar, invocó el nombre de Yavé y
acampó. Allí los siervos de Isaac cavaron un pozo. \footnote{\textbf{26:25}
  Gén 12,8}

\hypertarget{el-pacto-entre-isaac-y-abimelech-en-beerseba}{%
\subsection{El pacto entre Isaac y Abimelech en
Beerseba}\label{el-pacto-entre-isaac-y-abimelech-en-beerseba}}

\bibleverse{26} Entonces Abimelec fue a él desde Gerar con Ahuzzat, su
amigo, y Ficol, el capitán de su ejército. \footnote{\textbf{26:26} Gén
  21,22} \bibleverse{27} Isaac les dijo: ``¿Por qué habéis venido a mí,
ya que me odiáis y me habéis enviado lejos de vosotros?''

\bibleverse{28} Dijeron: ``Vimos claramente que el Señor estaba con
ustedes. Dijimos: `Que haya ahora un juramento entre nosotros, incluso
entre nosotros y vosotros, y hagamos un pacto con vosotros,
\bibleverse{29} de que no nos haréis ningún daño, como no os hemos
tocado, y como no os hemos hecho más que el bien, y os hemos despedido
en paz.' Ahora sois los benditos de Yahvé''.

\bibleverse{30} Les hizo un banquete, y comieron y bebieron.
\bibleverse{31} Se levantaron por la mañana y se juraron mutuamente.
Isaac los despidió, y ellos se alejaron de él en paz. \bibleverse{32} El
mismo día, los siervos de Isaac vinieron y le contaron sobre el pozo que
habían cavado, y le dijeron: ``Hemos encontrado agua''. \bibleverse{33}
Lo llamó ``Shibah''.\footnote{\textbf{26:33} Shibah significa
  ``juramento'' o ``siete''.} Por eso el nombre de la ciudad es
``Beersheba''\footnote{\textbf{26:33} Beersheba significa ``pozo del
  juramento'' o ``pozo de los siete''} hasta el día de hoy. \footnote{\textbf{26:33}
  Gén 21,31}

\hypertarget{esauxfa-se-casa-con-dos-mujeres-hititas-en-contra-de-la-voluntad-de-sus-padres}{%
\subsection{Esaú se casa con dos mujeres hititas en contra de la
voluntad de sus
padres}\label{esauxfa-se-casa-con-dos-mujeres-hititas-en-contra-de-la-voluntad-de-sus-padres}}

\bibleverse{34} Cuando Esaú tenía cuarenta años, tomó por esposa a
Judit, hija de Beeri el hitita, y a Basemat, hija de Elón el hitita.
\footnote{\textbf{26:34} Gén 36,2-3} \bibleverse{35} Ellas afligieronlos
espíritus de Isaac y Rebeca.

\hypertarget{isaac-se-prepara-para-bendecir-esau}{%
\subsection{Isaac se prepara para bendecir
Esau}\label{isaac-se-prepara-para-bendecir-esau}}

\hypertarget{section-26}{%
\section{27}\label{section-26}}

\bibleverse{1} Cuando Isaac envejeció, y sus ojos se oscurecieron, de
modo que no podía ver, llamó a Esaú, su hijo mayor, y le dijo: ``¿Hijo
mío?'' Le dijo: ``Aquí estoy''.

\bibleverse{2} Él dijo: ``Mira ahora, soy viejo. No sé el día de mi
muerte. \bibleverse{3} Ahora, pues, por favor, toma tus armas, tu carcaj
y tu arco, y sal al campo, y tráeme venado. \bibleverse{4} Prepárame una
comida sabrosa, como las que me gustan, y tráemela, para que coma y mi
alma te bendiga antes de morir.'' \footnote{\textbf{27:4} Heb 11,20}

\hypertarget{la-intervenciuxf3n-engauxf1osa-de-rebeca}{%
\subsection{La intervención engañosa de
Rebeca}\label{la-intervenciuxf3n-engauxf1osa-de-rebeca}}

\bibleverse{5} Rebeca escuchó cuando Isaac habló con su hijo Esaú. Esaú
fue al campo a cazar venado y a traerlo. \bibleverse{6} Rebeca habló a
su hijo Jacob, diciendo: ``He aquí que he oído a tu padre hablar a Esaú,
tu hermano, diciendo: \bibleverse{7} `Tráeme venado, y hazme comida
sabrosa, para que yo coma, y te bendiga delante de Yahvé antes de mi
muerte'. \bibleverse{8} Ahora, pues, hijo mío, obedece mi voz según lo
que te mando. \bibleverse{9} Ve ahora al rebaño y tráeme dos buenos
cabritos de allí. Yo los haré comida sabrosa para tu padre, como a él le
gusta. \bibleverse{10} Se lo llevarás a tu padre para que coma y te
bendiga antes de su muerte.''

\bibleverse{11} Jacob dijo a su madre Rebeca: ``Mira, mi hermano Esaú es
un hombre velludo, y yo soy un hombre liso. \footnote{\textbf{27:11} Gén
  25,25} \bibleverse{12} ¿Y si mi padre me toca? Le pareceré un
engañador, y traería una maldición sobre mí, y no una bendición''.

\bibleverse{13} Su madre le dijo: ``Que tu maldición caiga sobre mí,
hijo mío. Sólo obedece mi voz, y ve a buscarlos por mí''.

\bibleverse{14} Fue a buscarlos y se los llevó a su madre. Su madre
preparó una comida sabrosa, como la que le gustaba a su padre.
\bibleverse{15} Rebeca tomó los buenos vestidos de Esaú, su hijo mayor,
que estaban con ella en la casa, y se los puso a Jacob, su hijo menor.
\bibleverse{16} Puso las pieles de los cabritos en sus manos y en la
parte lisa de su cuello. \bibleverse{17} Dio la comida sabrosa y el pan
que había preparado en manos de su hijo Jacob.

\hypertarget{jacob-recibe-la-bendiciuxf3n-del-primoguxe9nito}{%
\subsection{Jacob recibe la bendición del
primogénito}\label{jacob-recibe-la-bendiciuxf3n-del-primoguxe9nito}}

\bibleverse{18} Se acercó a su padre y le dijo: ``¿Padre mío?'' Dijo:
``Aquí estoy. ¿Quién eres tú, hijo mío?''

\bibleverse{19} Jacob dijo a su padre: ``Yo soy Esaú, tu primogénito. He
hecho lo que me pediste. Por favor, levántate, siéntate y come de mi
venado, para que tu alma me bendiga''.

\bibleverse{20} Isaac dijo a su hijo: ``¿Cómo es que lo has encontrado
tan rápido, hijo mío?'' Dijo: ``Porque Yahvé, tu Dios, me dio el
éxito''.

\bibleverse{21} Isaac dijo a Jacob: ``Por favor, acércate para que pueda
sentirte, hijo mío, si realmente eres mi hijo Esaú o no''.

\bibleverse{22} Jacob se acercó a su padre Isaac. Lo palpó y dijo: ``La
voz es de Jacob, pero las manos son de Esaú''. \bibleverse{23} No lo
reconoció, porque sus manos eran peludas, como las de su hermano Esaú.
Así que lo bendijo. \bibleverse{24} Le dijo: ``¿Eres realmente mi hijo
Esaú?'' Él dijo: ``Yo soy''.

\bibleverse{25} Dijo: ``Acércamelo, y comeré del venado de mi hijo, para
que mi alma te bendiga''. Se lo acercó, y comió. Le trajo vino, y bebió.
\bibleverse{26} Su padre Isaac le dijo: ``Acércate ahora y bésame, hijo
mío''. \bibleverse{27} Se acercó y lo besó. Olió el olor de su ropa, lo
bendijo y dijo, ``He aquí el olor de mi hijoes como el olor de un campo
que Yahvé ha bendecido. \bibleverse{28} Dios te dé del rocío del cielo,
de la grasa de la tierra, y mucho grano y vino nuevo. \bibleverse{29}
Que los pueblos te sirvan, y las naciones se inclinan ante ti. Sé el
señor de tus hermanos. Que los hijos de tu madre se inclinen ante ti.
Maldito sea todo aquel que te maldiga. Bendito sea todo aquel que te
bendiga''. \footnote{\textbf{27:29} Gén 25,23; Gén 12,3}

\hypertarget{el-regreso-de-esauxfa-su-lamento-y-la-bendiciuxf3n-que-le-dio-su-padre}{%
\subsection{El regreso de Esaú, su lamento y la bendición que le dio su
padre}\label{el-regreso-de-esauxfa-su-lamento-y-la-bendiciuxf3n-que-le-dio-su-padre}}

\bibleverse{30} Cuando Isaac terminó de bendecir a Jacob, y éste acababa
de salir de la presencia de su padre Isaac, su hermano Esaú llegó de su
cacería. \bibleverse{31} Él también preparó comida sabrosa y se la llevó
a su padre. Dijo a su padre: ``Que mi padre se levante y coma de la
carne de caza de su hijo, para que tu alma me bendiga''.

\bibleverse{32} Su padre Isaac le dijo: ``¿Quién eres tú?'' Dijo: ``Soy
tu hijo, tu primogénito, Esaú''.

\bibleverse{33} Isaac se estremeció violentamente y dijo: ``¿Quién es,
pues, el que ha tomado carne de venado y me la ha traído, y yo he comido
de todo antes de que vinieras, y lo he bendecido? Sí, será bendecido''.

\bibleverse{34} Cuando Esaú escuchó las palabras de su padre, lloró con
un grito muy grande y amargo, y dijo a su padre: ``Bendíceme, a mí
también, padre mío''. \footnote{\textbf{27:34} Heb 12,17}

\bibleverse{35} Dijo: ``Tu hermano vino con engaño y te ha quitado la
bendición''.

\bibleverse{36} Dijo: ``¿No se llama Jacob con razón? Porque me ha
suplantado estas dos veces. Me quitó la primogenitura. Mira, ahora me ha
quitado la bendición''. Dijo: ``¿No me has reservado una bendición?''.
\footnote{\textbf{27:36} Gén 25,26; Gén 25,33}

\bibleverse{37} Isaac respondió a Esaú: ``He aquí que lo he hecho tu
señor, y a todos sus hermanos se los he dado por servidores. Lo he
mantenido con grano y vino nuevo. ¿Qué haré entonces por ti, hijo mío?''

\bibleverse{38} Esaú dijo a su padre: ``¿Tienes una sola bendición,
padre mío? Bendíceme a mí también, padre mío''. Esaú alzó la voz y
lloró.

\bibleverse{39} Isaac, su padre, le respondió, ``He aquí que tu morada
será de la grosura de la tierra, y del rocío del cielo desde arriba.
\bibleverse{40} Vivirás con tu espada y servirás a tu hermano. Ocurrirá,
cuando te liberes, que sacudirás su yugo de tu cuello''. \footnote{\textbf{27:40}
  2Re 8,20}

\hypertarget{esauxfa-busca-matar-a-su-hermano}{%
\subsection{Esaú busca matar a su
hermano}\label{esauxfa-busca-matar-a-su-hermano}}

\bibleverse{41} Esaú odiaba a Jacob a causa de la bendición con que su
padre lo había bendecido. Esaú dijo en su corazón: ``Se acercan los días
de luto por mi padre. Entonces mataré a mi hermano Jacob''.

\bibleverse{42} Las palabras de Esaú, su hijo mayor, fueron contadas a
Rebeca. Ella envió y llamó a Jacob, su hijo menor, y le dijo: ``Mira, tu
hermano Esaú se consuela de ti planeando matarte. \bibleverse{43} Ahora,
pues, hijo mío, obedece mi voz. Levántate y huye a Labán, mi hermano, en
Harán. \footnote{\textbf{27:43} Gén 24,10} \bibleverse{44} Quédate con
él unos días, hasta que la furia de tu hermano se aleje...
\bibleverse{45} hasta que la ira de tu hermano se aleje de ti, y se
olvide de lo que le has hecho. Entonces enviaré y te sacaré de allí.
¿Por qué he de perderos a los dos en un solo día?''

\bibleverse{46} Rebeca dijo a Isaac: ``Estoy cansada de mi vida a causa
de las hijas de Het. Si Jacob toma una esposa de las hijas de Het, como
éstas, de las hijas de la tierra, ¿de qué me servirá mi vida?''
\footnote{\textbf{27:46} Gén 26,34-35}

\hypertarget{jacob-huye-a-padan-aram}{%
\subsection{Jacob huye a Padan-aram}\label{jacob-huye-a-padan-aram}}

\hypertarget{section-27}{%
\section{28}\label{section-27}}

\bibleverse{1} Isaac llamó a Jacob, lo bendijo y le ordenó: ``No tomarás
mujer de las hijas de Canaán. \footnote{\textbf{28:1} Gén 24,3}
\bibleverse{2} Levántate, ve a Paddán Aram, a la casa de Betuel, el
padre de tu madre. Toma de allí una esposa de entre las hijas de Labán,
el hermano de tu madre. \footnote{\textbf{28:2} Gén 22,23; Gén 24,29}
\bibleverse{3} Que Dios Todopoderoso te bendiga, te haga fructificar y
te multiplique, para que seas una compañía de pueblos, \bibleverse{4} y
te dé la bendición de Abraham, a ti y a tu descendencia contigo, para
que heredes la tierra por la que transitas, que Dios le dio a Abraham.''
\footnote{\textbf{28:4} Gén 12,2}

\bibleverse{5} Isaac despidió a Jacob. Fue a Paddán Aram, a Labán, hijo
de Betuel el sirio, hermano de Rebeca, madre de Jacob y Esaú.

\hypertarget{el-nuevo-matrimonio-de-esauxfa-con-una-hija-de-ismael}{%
\subsection{El nuevo matrimonio de Esaú con una hija de
Ismael}\label{el-nuevo-matrimonio-de-esauxfa-con-una-hija-de-ismael}}

\bibleverse{6} Esaú vio que Isaac había bendecido a Jacob y lo había
enviado a Paddán Aram para que tomara una esposa de allí, y que al
bendecirlo le dio una orden, diciendo: ``No tomarás esposa de las hijas
de Canaán''; \bibleverse{7} y que Jacob obedeció a su padre y a su
madre, y se fue a Paddán Aram. \bibleverse{8} Esaú vio que las hijas de
Canaán no agradaban a Isaac, su padre. \bibleverse{9} Entonces Esaú se
fue a Ismael y tomó, además de las esposas que tenía, a Mahalat, hija de
Ismael, hijo de Abraham, hermana de Nebaiot, para que fuera su esposa.
\footnote{\textbf{28:9} Gén 26,34; Gén 25,13}

\hypertarget{el-sueuxf1o-de-jacob-en-betel-de-la-escalera-al-cielo}{%
\subsection{El sueño de Jacob en Betel de la escalera al
cielo}\label{el-sueuxf1o-de-jacob-en-betel-de-la-escalera-al-cielo}}

\bibleverse{10} Jacob salió de Beerseba y se dirigió a Harán.
\bibleverse{11} Llegó a un lugar y se quedó allí toda la noche, porque
el sol se había puesto. Tomó una de las piedras del lugar, la puso
debajo de su cabeza y se acostó en ese lugar para dormir.
\bibleverse{12} Soñó y vio una escalera colocada sobre la tierra, cuya
cima llegaba hasta el cielo. Los ángeles de Dios subían y bajaban por
ella. \footnote{\textbf{28:12} Juan 1,51} \bibleverse{13} He aquí que
Yahvé estaba de pie sobre ella y decía: ``Yo soy Yahvé, el Dios de
Abraham, tu padre, y el Dios de Isaac. Daré la tierra sobre la que te
acuestas a ti y a tu descendencia. \footnote{\textbf{28:13} Gén 12,7}
\bibleverse{14} Tu descendencia será como el polvo de la tierra, y te
extenderás al oeste, al este, al norte y al sur. En ti y en tu
descendencia serán bendecidas todas las familias de la tierra.
\footnote{\textbf{28:14} Gén 13,16; Gén 12,3} \bibleverse{15} He aquí
que yo estoy con vosotros y os guardaré dondequiera que vayáis, y os
haré volver a esta tierra. Porque no te dejaré hasta que haya hecho lo
que te he dicho''.

\hypertarget{jacob-consagra-una-piedra-conmemorativa-como-el-comienzo-de-una-casa-de-dios-en-betel}{%
\subsection{Jacob consagra una piedra conmemorativa como el comienzo de
una casa de Dios en
Betel}\label{jacob-consagra-una-piedra-conmemorativa-como-el-comienzo-de-una-casa-de-dios-en-betel}}

\bibleverse{16} Jacob despertó de su sueño y dijo: ``Ciertamente Yahvé
está en este lugar, y yo no lo sabía''. \bibleverse{17} Tuvo miedo y
dijo: ``¡Qué impresionante es este lugar! Esto no es más que la casa de
Dios, y ésta es la puerta del cielo''. \footnote{\textbf{28:17} Éxod 3,5}

\bibleverse{18} Jacob se levantó de madrugada, tomó la piedra que había
puesto debajo de su cabeza, la puso como pilar y derramó aceite en su
parte superior. \bibleverse{19} Llamó el nombre de aquel lugar Betel,
pero el nombre de la ciudad era Luz al principio. \footnote{\textbf{28:19}
  Gén 35,14-15} \bibleverse{20} Jacob hizo un voto, diciendo: ``Si Dios
está conmigo y me guarda en este camino que recorro, y me da pan para
comer y ropa para vestir, \bibleverse{21} de modo que vuelva a la casa
de mi padre en paz, y Yahvé sea mi Dios, \bibleverse{22} entonces esta
piedra, que he levantado como columna, será la casa de Dios. De todo lo
que me des te daré seguramente la décima parte''. \footnote{\textbf{28:22}
  Gén 35,1; Gén 35,7}

\hypertarget{jacob-al-pozo-de-haran}{%
\subsection{Jacob al pozo de Haran}\label{jacob-al-pozo-de-haran}}

\hypertarget{section-28}{%
\section{29}\label{section-28}}

\bibleverse{1} Entonces Jacob siguió su camino y llegó a la tierra de
los hijos de Oriente. \bibleverse{2} Miró, y vio un pozo en el campo, y
vio tres rebaños de ovejas acostados junto a él. Porque de ese pozo
abrevaban los rebaños. La piedra de la boca del pozo era grande.
\bibleverse{3} Allí estaban reunidos todos los rebaños. Rodaron la
piedra de la boca del pozo, dieron de beber a las ovejas y volvieron a
poner la piedra en la boca del pozo en su lugar. \bibleverse{4} Jacob
les dijo: ``Parientes míos, ¿de dónde sois?'' Dijeron: ``Somos de
Harán''.

\bibleverse{5} Les dijo: ``¿Conocéis a Labán, hijo de Najor?'' Dijeron:
``Lo conocemos''.

\bibleverse{6} Les dijo: ``¿Le va bien?''. Dijeron: ``Está bien. Mira,
Raquel, su hija, viene con las ovejas''.

\bibleverse{7} Dijo: ``Mira, todavía es mediodía, no es hora de reunir
el ganado. Da de beber a las ovejas y ve a darles de comer''.

\bibleverse{8} Dijeron: ``No podemos, hasta que se reúnan todos los
rebaños y se quite la piedra de la boca del pozo. Entonces abrevaremos
las ovejas''.

\hypertarget{el-saludo-de-jacob-con-rachuxeal-y-su-admisiuxf3n-a-labuxe1n}{%
\subsection{El saludo de Jacob con Rachêl y su admisión a
Labán}\label{el-saludo-de-jacob-con-rachuxeal-y-su-admisiuxf3n-a-labuxe1n}}

\bibleverse{9} Mientras aún hablaba con ellos, llegó Raquel con las
ovejas de su padre, pues las guardaba. \bibleverse{10} Cuando Jacob vio
a Raquel, la hija de Labán, hermano de su madre, y las ovejas de Labán,
hermano de su madre, se acercó, hizo rodar la piedra de la boca del pozo
y dio de beber al rebaño de Labán, hermano de su madre. \bibleverse{11}
Jacob besó a Raquel, alzó la voz y lloró. \bibleverse{12} Jacob le dijo
a Raquel que era pariente de su padre y que era hijo de Rebeca. Ella
corrió y se lo contó a su padre.

\bibleverse{13} Cuando Labán oyó la noticia de Jacob, el hijo de su
hermana, corrió a recibir a Jacob, lo abrazó y lo besó, y lo llevó a su
casa. Jacob le contó a Labán todas estas cosas. \bibleverse{14} Labán le
dijo: ``Ciertamente tú eres mi hueso y mi carne''. Jacob se quedó con él
durante un mes.

\hypertarget{jacob-entra-en-servicio-con-labuxe1n}{%
\subsection{Jacob entra en servicio con
Labán}\label{jacob-entra-en-servicio-con-labuxe1n}}

\bibleverse{15} Labán le dijo a Jacob: ``Porque eres mi pariente, ¿debes
servirme por nada? Dime, ¿cuál será tu salario?''

\bibleverse{16} Labán tenía dos hijas. El nombre de la mayor era Lía, y
el de la menor, Raquel. \bibleverse{17} Los ojos de Lea eran débiles,
pero Raquel era hermosa en forma y atractiva. \bibleverse{18} Jacob
amaba a Raquel. Dijo: ``Te serviré siete años por Raquel, tu hija
menor''.

\bibleverse{19} Labán dijo: ``Es mejor que te la entregue a ti que a
otro hombre. Quédate conmigo''.

\bibleverse{20} Jacob sirvió siete años por Raquel. Le parecieron pocos
días, para el amor que sentía por ella.

\bibleverse{21} Jacob dijo a Labán: ``Dame a mi mujer, pues mis días
están cumplidos, para que entre con ella.''

\bibleverse{22} Labán reunió a todos los hombres del lugar e hizo un
banquete. \bibleverse{23} Al anochecer, tomó a su hija Lea y la llevó a
Jacob. Él entró con ella. \bibleverse{24} Labán le dio a su hija Lea a
Zilpá como sirvienta. \bibleverse{25} Por la mañana, he aquí que era
Lía. Le dijo a Labán: ``¿Qué es esto que has hecho conmigo? ¿No he
servido contigo por Raquel? ¿Por qué entonces me has engañado?''

\bibleverse{26} Labán dijo: ``No se hace así en nuestro lugar, dar al
menor antes que al primogénito. \bibleverse{27} Cumple la semana de
éste, y te daremos también el otro para el servicio que prestarás
conmigo durante siete años más.''

\bibleverse{28} Jacob lo hizo y cumplió su semana. Le dio a su hija
Raquel como esposa. \bibleverse{29} Labán dio a Bilhá, su sierva, a su
hija Raquel para que fuera su sirvienta. \bibleverse{30} Entró también a
Raquel, y amó también a Raquel más que a Lea, y sirvió con él siete años
más. \footnote{\textbf{29:30} Lev 18,18}

\hypertarget{los-primer-cuatro-hijos-de-lea}{%
\subsection{Los primer cuatro hijos de
Lea}\label{los-primer-cuatro-hijos-de-lea}}

\bibleverse{31} Yahvé vio que Lea era odiosa, y abrió su vientre, pero
Raquel era estéril. \bibleverse{32} Lea concibió y dio a luz un hijo, al
que llamó Rubén. Porque dijo: ``Porque Yahvé ha mirado mi aflicción,
pues ahora mi esposo me amará''. \bibleverse{33} Concibió de nuevo y dio
a luz un hijo, y dijo: ``Porque Yahvé ha oído que soy odiada, por eso me
ha dado también este hijo.'' Le puso el nombre de Simeón.
\bibleverse{34} Concibió de nuevo y dio a luz un hijo. Dijo: ``Esta vez
mi esposo se unirá a mí, porque le he dado tres hijos''. Por eso se
llamó Leví. \bibleverse{35} Concibió de nuevo y dio a luz un hijo. Dijo:
``Esta vez alabaré a Yahvé''. Por eso lo llamó Judá. Luego dejó de dar a
luz.

\hypertarget{los-dos-hijos-de-bilha-la-sierva-de-rachuxeal}{%
\subsection{Los dos hijos de Bilha, la sierva de
Rachêl}\label{los-dos-hijos-de-bilha-la-sierva-de-rachuxeal}}

\hypertarget{section-29}{%
\section{30}\label{section-29}}

\bibleverse{1} Cuando Raquel vio que no daba hijos a Jacob, envidió a su
hermana. Le dijo a Jacob: ``Dame hijos o moriré''.

\bibleverse{2} La ira de Jacob ardió contra Raquel y dijo: ``¿Estoy yo
en lugar de Dios, que te ha negado el fruto del vientre?'' \footnote{\textbf{30:2}
  Sal 127,3}

\bibleverse{3} Ella dijo: ``He aquí mi doncella Bilhá. Entra con ella,
para que dé a luz sobre mis rodillas, y yo también pueda obtener hijos
de ella''. \footnote{\textbf{30:3} Gén 16,2} \bibleverse{4} Ella le dio
como esposa a su sierva Bilhá, y Jacob se acercó a ella. \bibleverse{5}
Bilhá concibió y dio a luz un hijo a Jacob. \bibleverse{6} Raquel dijo:
``Dios me ha juzgado, y también ha escuchado mi voz, y me ha dado un
hijo''. Por eso lo llamó Dan. \bibleverse{7} Bilhah, la sierva de
Raquel, concibió de nuevo y dio a Jacob un segundo hijo. \bibleverse{8}
Raquel dijo: ``He luchado con mi hermana con poderosos combates, y he
vencido.'' Lo llamó Neftalí.

\hypertarget{los-dos-hijos-de-silpa-la-sierva-de-lea}{%
\subsection{Los dos hijos de Silpa, la sierva de
Lea}\label{los-dos-hijos-de-silpa-la-sierva-de-lea}}

\bibleverse{9} Cuando Lía vio que había terminado de parir, tomó a
Zilpá, su sierva, y se la dio a Jacob como esposa. \footnote{\textbf{30:9}
  Gén 29,35} \bibleverse{10} Zilpa, la sierva de Lea, dio a luz un hijo
a Jacob. \bibleverse{11} Lea dijo: ``¡Qué suerte!''. Le puso el nombre
de Gad. \bibleverse{12} Zilpa, la sierva de Lía, dio a luz un segundo
hijo a Jacob. \bibleverse{13} Lea dijo: ``Feliz soy, porque las hijas me
llamarán feliz''. Lo llamó Aser.

\hypertarget{los-ultimos-niuxf1os-de-lea}{%
\subsection{Los ultimos niños de
Lea}\label{los-ultimos-niuxf1os-de-lea}}

\bibleverse{14} Rubén fue en los días de la cosecha del trigo y encontró
mandrágoras en el campo, y se las llevó a su madre, Lea. Entonces Raquel
le dijo a Lea: ``Por favor, dame algunas de las mandrágoras de tu
hijo''.

\bibleverse{15} Lea le dijo: ``¿Es poca cosa que me hayas quitado a mi
marido? ¿Quieres quitarle también las mandrágoras a mi hijo?'' Raquel
dijo: ``Por eso se acostará contigo esta noche por las mandrágoras de tu
hijo''.

\bibleverse{16} Al anochecer, Jacob volvió del campo, y Lea salió a su
encuentro y le dijo: ``Tienes que entrar en mi casa, porque te he
contratado con las mandrágoras de mi hijo.'' Aquella noche se acostó con
ella. \bibleverse{17} Dios escuchó a Lea, y ella concibió y dio a luz a
Jacob un quinto hijo. \bibleverse{18} Lea dijo: ``Dios me ha dado mi
salario, porque le di mi siervo a mi marido''. Lo llamó Isacar.
\bibleverse{19} Lea concibió de nuevo y dio a luz un sexto hijo a Jacob.
\bibleverse{20} Lea dijo: ``Dios me ha dotado de una buena dote. Ahora
mi marido vivirá conmigo, porque le he dado seis hijos''. Le puso el
nombre de Zabulón. \bibleverse{21} Después dio a luz a una hija y la
llamó Dina.

\hypertarget{rachuxeal-cresce-madre-de-josuxe9}{%
\subsection{Rachêl cresce madre de
José}\label{rachuxeal-cresce-madre-de-josuxe9}}

\bibleverse{22} Dios se acordó de Raquel, la escuchó y le abrió el
vientre. \footnote{\textbf{30:22} 1Sam 1,19} \bibleverse{23} Concibió,
dio a luz un hijo y dijo: ``Dios ha quitado mi afrenta''. \footnote{\textbf{30:23}
  Is 4,1; Luc 1,25} \bibleverse{24} Le puso el nombre de
José,\footnote{\textbf{30:24} José significa ``puede añadir''.}
diciendo: ``Que Yahvé me añada otro hijo''.

\hypertarget{el-nuevo-pacto-de-servicio-de-jacob-con-labuxe1n}{%
\subsection{El nuevo pacto de servicio de Jacob con
Labán}\label{el-nuevo-pacto-de-servicio-de-jacob-con-labuxe1n}}

\bibleverse{25} Cuando Raquel dio a luz a José, Jacob dijo a Labán:
``Despídeme para que me vaya a mi lugar y a mi país. \bibleverse{26}
Dame mis esposas y mis hijos por los que te he servido, y déjame ir;
porque tú conoces mi servicio con el que te he servido.'' \footnote{\textbf{30:26}
  Gén 29,20; Gén 29,30}

\bibleverse{27} Labán le dijo: ``Si ahora he hallado gracia ante tus
ojos, quédate aquí, pues he adivinado que Yahvé me ha bendecido por tu
causa.'' \footnote{\textbf{30:27} Gén 39,5} \bibleverse{28} Él le dijo:
``Ponme tu salario, y te lo daré''.

\bibleverse{29} Jacob le dijo: ``Tú sabes cómo te he servido y cómo me
ha ido con tu ganado. \bibleverse{30} Porque era poco lo que tenías
antes de que yo llegara, y ha aumentado hasta convertirse en una
multitud. El Señor te ha bendecido dondequiera que me he vuelto. Ahora,
¿cuándo proveeré también para mi propia casa?''

\bibleverse{31} Labán dijo: ``¿Qué te doy?'' Jacob dijo: ``No me darás
nada. Si haces esto por mí, volveré a apacentar tu rebaño y lo
mantendré. \bibleverse{32} Hoy pasaré por todo tu rebaño, eliminando de
él a toda oveja manchada y a toda oveja negra, y a la manchada y a la
manchada entre las cabras. Este será mi salario. \bibleverse{33} Así mi
justicia responderá por mí en adelante, cuando vengas a hablar de mi
salario que está delante de ti. Todo el que no esté moteado y manchado
entre las cabras, y negro entre las ovejas, que pueda estar conmigo, se
considerará robado.''

\bibleverse{34} Labán dijo: ``He aquí, que sea según tu palabra''.

\hypertarget{jacob-obtuvo-una-gran-propiedad-de-ganado-a-travuxe9s-de-la-astucia}{%
\subsection{Jacob obtuvo una gran propiedad de ganado a través de la
astucia}\label{jacob-obtuvo-una-gran-propiedad-de-ganado-a-travuxe9s-de-la-astucia}}

\bibleverse{35} Aquel día quitó los machos cabríos rayados y manchados,
y todas las cabras moteadas y manchadas, todas las que tenían blanco, y
todas las negras entre las ovejas, y las entregó en manos de sus hijos.
\bibleverse{36} Puso tres días de camino entre él y Jacob, y éste
apacentó el resto de los rebaños de Labán.

\bibleverse{37} Jacob tomó para sí varas de álamo, almendro y plátano
frescos, peló en ellas vetas blancas e hizo aparecer el blanco que había
en las varas. \bibleverse{38} Puso las varas que había pelado frente a
los rebaños en los abrevaderos donde éstos venían a beber. Ellas
concebían cuando venían a beber. \bibleverse{39} Los rebaños concibieron
delante de las varas, y los rebaños produjeron rayados, moteados y
manchados. \bibleverse{40} Jacob separó los corderos, y puso las caras
de los rebaños hacia los rayados y todos los negros del rebaño de Labán.
Apartó sus propios rebaños y no los puso en el rebaño de Labán.
\bibleverse{41} Cuando las más fuertes del rebaño concebían, Jacob ponía
las varas delante de los ojos del rebaño en los abrevaderos, para que
concibieran entre las varas; \bibleverse{42} pero cuando el rebaño era
débil, no las metía. Así que las más débiles eran de Labán, y las más
fuertes de Jacob. \bibleverse{43} El hombre crecía mucho, y tenía
grandes rebaños, siervas y siervos, y camellos y asnos. \footnote{\textbf{30:43}
  Gén 12,16}

\hypertarget{las-razones-de-la-fuga-de-jacob}{%
\subsection{Las razones de la fuga de
Jacob}\label{las-razones-de-la-fuga-de-jacob}}

\hypertarget{section-30}{%
\section{31}\label{section-30}}

\bibleverse{1} Jacob escuchó las palabras de los hijos de Labán, que
decían: ``Jacob se ha llevado todo lo que era de nuestro padre. Ha
obtenido toda esta riqueza de lo que era de nuestro padre''. \footnote{\textbf{31:1}
  Gén 30,35} \bibleverse{2} Jacob vio la expresión del rostro de Labán,
y he aquí que no era hacia él como antes. \bibleverse{3} Yahvé dijo a
Jacob: ``Vuelve a la tierra de tus padres y a tus parientes, y yo estaré
contigo.'' \footnote{\textbf{31:3} Gén 28,15}

\hypertarget{la-consulta-de-jacob-con-sus-esposas}{%
\subsection{La consulta de Jacob con sus
esposas}\label{la-consulta-de-jacob-con-sus-esposas}}

\bibleverse{4} Jacob envió a llamar a Raquel y a Lía al campo, a su
rebaño, \bibleverse{5} y les dijo: ``Veo la expresión del rostro de
vuestro padre, que no es hacia mí como antes; pero el Dios de mi padre
ha estado conmigo. \footnote{\textbf{31:5} Gén 26,24} \bibleverse{6}
Sabéis que he servido a vuestro padre con todas mis fuerzas.
\bibleverse{7} Tu padre me ha engañado y ha cambiado mi salario diez
veces, pero Dios no le ha permitido hacerme daño. \bibleverse{8} Si
dijo: `El moteado será tu salario', entonces todo el rebaño llevó
moteado. Si dijo: ``El salario será el moteado'', entonces todo el
rebaño dio un moteado. \footnote{\textbf{31:8} Gén 30,32; Gén 30,39}
\bibleverse{9} Así, Dios se llevó el ganado de tu padre y me lo dio a
mí. \bibleverse{10} Durante la época de apareamiento, levanté los ojos y
vi en sueños que los machos cabríos que saltaban en el rebaño estaban
rayados, moteados y canosos. \bibleverse{11} El ángel de Dios me dijo en
el sueño: ``Jacob'', y yo dije: ``Aquí estoy''. \bibleverse{12} Me dijo:
`Ahora levanta tus ojos y mira que todos los machos cabríos que saltan
en el rebaño están rayados, moteados y canosos, porque he visto todo lo
que Labán te hace. \bibleverse{13} Yo soy el Dios de Betel, donde
ungiste una columna, donde me hiciste un voto. Ahora levántate, sal de
esta tierra y vuelve a la tierra donde naciste'\,''. \footnote{\textbf{31:13}
  Gén 28,18-22}

\bibleverse{14} Raquel y Lea le respondieron: ``¿Hay todavía alguna
porción o herencia para nosotras en la casa de nuestro padre?
\bibleverse{15} ¿No somos consideradas por él como extranjeras? Porque
nos ha vendido, y también ha agotado nuestro dinero. \footnote{\textbf{31:15}
  Gén 29,18; Gén 29,27} \bibleverse{16} Pues todas las riquezas que Dios
ha quitado a nuestro padre son nuestras y de nuestros hijos. Ahora bien,
todo lo que Dios te ha dicho, hazlo''.

\hypertarget{la-fuga-de-jacob-y-la-persecuciuxf3n-de-labuxe1n}{%
\subsection{La fuga de Jacob y la persecución de
Labán}\label{la-fuga-de-jacob-y-la-persecuciuxf3n-de-labuxe1n}}

\bibleverse{17} Entonces Jacob se levantó y puso a sus hijos y a sus
mujeres sobre los camellos, \bibleverse{18} y se llevó todo su ganado y
todas sus posesiones que había reunido, incluyendo el ganado que había
ganado en Paddán Aram, para ir a Isaac, su padre, a la tierra de Canaán.
\bibleverse{19} Labán había ido a esquilar sus ovejas, y Raquel robó los
terafines\footnote{\textbf{31:19} Los terafines eran ídolos domésticos
  que podían estar asociados a los derechos de herencia de los bienes
  del hogar.} que eran de su padre.

\bibleverse{20} Jacob engañó a Labán el sirio, pues no le dijo que
estaba huyendo. \bibleverse{21} Así que huyó con todo lo que tenía. Se
levantó, pasó el río y puso su rostro en dirección al monte de Galaad.

\bibleverse{22} Al tercer día le avisaron a Labán que Jacob había huido.
\bibleverse{23} Tomó consigo a sus parientes y lo persiguió durante
siete días de viaje. Lo alcanzó en la montaña de Galaad. \footnote{\textbf{31:23}
  Gén 31,47} \bibleverse{24} Dios vino a Labán el sirio en un sueño
nocturno y le dijo: ``Ten cuidado de no hablarle a Jacob ni bien ni
mal''. \footnote{\textbf{31:24} Gén 20,3; Prov 16,7}

\hypertarget{discurso-de-castigo-de-labuxe1n-y-registro-de-la-casa}{%
\subsection{Discurso de castigo de Labán y registro de la
casa}\label{discurso-de-castigo-de-labuxe1n-y-registro-de-la-casa}}

\bibleverse{25} Labán alcanzó a Jacob. Jacob había acampado en la
montaña, y Labán con sus parientes acampó en la montaña de Galaad.
\bibleverse{26} Labán dijo a Jacob: ``¿Qué has hecho, que me has
engañado y te has llevado a mis hijas como cautivas de la espada?
\bibleverse{27} ¿Por qué huiste en secreto y me engañaste, y no me lo
dijiste, para que te despidiera con alegría y con cantos, con pandereta
y con arpa; \bibleverse{28} y no me dejaste besar a mis hijos y a mis
hijas? Ahora has hecho una tontería. \bibleverse{29} Está en poder de mi
mano hacerte daño, pero el Dios de tu padre me habló anoche, diciendo:
`Ten cuidado de no hablarle a Jacob ni bien ni mal'. \bibleverse{30}
Ahora bien, tú quieres irte, porque anhelabas mucho la casa de tu padre,
pero ¿por qué has robado mis dioses?''

\bibleverse{31} Jacob respondió a Labán: ``Porque tuve miedo, pues dije:
`No sea que me quites a tus hijas por la fuerza'. \bibleverse{32}
Cualquiera con quien encuentres a tus dioses no vivirá. Ante nuestros
parientes, discierne lo que es tuyo conmigo, y tómalo''. Pues Jacob no
sabía que Raquel las había robado. \footnote{\textbf{31:32} Gén 31,19}

\bibleverse{33} Labán entró en la tienda de Jacob, en la tienda de Lea y
en la tienda de las dos siervas, pero no las encontró. Salió de la
tienda de Lea y entró en la tienda de Raquel. \bibleverse{34} Raquel
había tomado los terafines, los había puesto en la silla del camello y
se había sentado sobre ellos. Labán tanteó toda la tienda, pero no los
encontró. \bibleverse{35} Ella le dijo a su padre: ``Que mi señor no se
enoje porque no puedo levantarme delante de ti, porque tengo la regla''.
Buscó, pero no encontró los terafines.

\hypertarget{discurso-de-acusaciuxf3n-de-jacob}{%
\subsection{Discurso de acusación de
Jacob}\label{discurso-de-acusaciuxf3n-de-jacob}}

\bibleverse{36} Jacob se enojó y discutió con Labán. Jacob respondió a
Labán: ``¿Cuál es mi infracción? ¿Cuál es mi pecado, para que me hayas
perseguido acaloradamente? \bibleverse{37} Ahora que has hurgado en
todas mis cosas, ¿qué has encontrado de todas las cosas de tu casa?
Ponlo aquí delante de mis parientes y de los tuyos, para que juzguen
entre nosotros dos.

\bibleverse{38} ``Estos veinte años he estado con ustedes. Tus ovejas y
tus cabras no han echado sus crías, y no me he comido los carneros de
tus rebaños. \bibleverse{39} Lo que fue arrancado de los animales, no te
lo traje. Yo soporté su pérdida. De mi mano lo exigiste, ya fuera robado
de día o de noche. \footnote{\textbf{31:39} Éxod 22,12-13}
\bibleverse{40} Esta era mi situación: de día me consumía la sequía, y
de noche la helada; y mi sueño huía de mis ojos. \bibleverse{41} Estos
veinte años he estado en tu casa. Te he servido catorce años por tus dos
hijas, y seis años por tu rebaño, y has cambiado mi salario diez veces.
\footnote{\textbf{31:41} Gén 29,20; Gén 29,30; Gén 30,21; Gén 30,32}
\bibleverse{42} Si el Dios de mi padre, el Dios de Abraham, y el temor
de Isaac, no hubieran estado conmigo, seguramente ahora me habrías
despedido con las manos vacías. Dios ha visto mi aflicción y el trabajo
de mis manos, y te reprendió anoche''. \footnote{\textbf{31:42} Gén
  31,54; Gén 31,24}

\hypertarget{la-respuesta-de-labuxe1n-el-tratado-de-paz-entre-uxe9l-y-jacob}{%
\subsection{La respuesta de Labán; el tratado de paz entre él y
Jacob}\label{la-respuesta-de-labuxe1n-el-tratado-de-paz-entre-uxe9l-y-jacob}}

\bibleverse{43} Labán respondió a Jacob: ``¡Las hijas son mis hijas, los
hijos son mis hijos, los rebaños son mis rebaños, y todo lo que ves es
mío! ¿Qué puedo hacer hoy a estas mis hijas, o a sus hijos que han dado
a luz? \bibleverse{44} Ahora ven, hagamos un pacto, tú y yo. Que sea
para que haya un testimonio entre tú y yo''.

\bibleverse{45} Jacob tomó una piedra y la puso como pilar. \footnote{\textbf{31:45}
  Gén 28,22} \bibleverse{46} Jacob dijo a sus parientes: ``Recojan
piedras''. Tomaron piedras e hicieron un montón. Comieron allí junto al
montón. \bibleverse{47} Labán lo llamó Jegar Sahadutha,\footnote{\textbf{31:47}
  ``Jegar Sahadutha'' significa ``Montón de Testigos'' en arameo.} pero
Jacob lo llamó Galeed. \footnote{\textbf{31:47} ``Galeed'' significa
  ``Montón de Testigos'' en hebreo.} \bibleverse{48} Labán dijo: ``Este
montón es testigo entre tú y yo hoy''. Por eso se llamó Galeed
\footnote{\textbf{31:48} Jos 22,27; Jos 24,27} \bibleverse{49} y Mizpa,
porque dijo: ``Yahvé vela entre mí y tú, cuando estamos ausentes el uno
del otro. \bibleverse{50} Si afliges a mis hijas, o si tomas esposas
además de mis hijas, ningún hombre está con nosotros; he aquí que Dios
es testigo entre mí y tú.'' \bibleverse{51} Labán dijo a Jacob: ``Mira
este montón y mira la columna que he puesto entre mí y tú.
\bibleverse{52} Que este montón sea testigo, y la columna sea testigo,
de que yo no pasaré por encima de este montón para ti, y de que tú no
pasarás por encima de este montón y de esta columna para mí, para hacer
daño. \bibleverse{53} El Dios de Abraham y el Dios de Nacor, el Dios de
su padre, juzguen entre nosotros''. Entonces Jacob juró por el temor de
su padre, Isaac. \footnote{\textbf{31:53} Gén 16,5} \bibleverse{54}
Jacob ofreció un sacrificio en el monte, y llamó a sus parientes para
que comieran pan. Comieron pan y se quedaron toda la noche en el monte.
\footnote{\textbf{31:54} Gén 31,42} \bibleverse{55} Al amanecer, Labán
se levantó, besó a sus hijos y a sus hijas y los bendijo. Labán partió y
regresó a su lugar.

\hypertarget{jacob-envuxeda-mensajeros-a-esauxfa}{%
\subsection{Jacob envía mensajeros a
Esaú}\label{jacob-envuxeda-mensajeros-a-esauxfa}}

\hypertarget{section-31}{%
\section{32}\label{section-31}}

\bibleverse{1} Jacob siguió su camino, y los ángeles de Dios salieron a
su encuentro. \footnote{\textbf{32:1} Gén 28,12; Sal 34,7}
\bibleverse{2} Al verlos, Jacob dijo: ``Este es el ejército de Dios''.
Llamó el nombre de aquel lugar Mahanaim. \footnote{\textbf{32:2}
  ``Mahanaim'' significa ``dos campamentos''.}

\bibleverse{3} Jacob envió mensajeros delante de él a Esaú, su hermano,
a la tierra de Seir, el campo de Edom. \footnote{\textbf{32:3} Gén 36,8}
\bibleverse{4} Les ordenó diciendo: ``Esto es lo que le diréis a mi
señor Esaú: `Esto es lo que dice tu siervo, Jacob. He vivido como
extranjero con Labán, y me he quedado hasta ahora. \bibleverse{5} Tengo
ganado, asnos, rebaños, siervos y siervas. He enviado a decírselo a mi
señor, para que encuentre gracia ante tus ojos''. \bibleverse{6} Los
mensajeros regresaron a Jacob diciendo: ``Hemos venido a ver a tu
hermano Esaú. Viene a tu encuentro, y cuatrocientos hombres están con
él''. \bibleverse{7} Entonces Jacob tuvo mucho miedo y se angustió.
Dividió a la gente que estaba con él, junto con los rebaños, las manadas
y los camellos, en dos grupos. \bibleverse{8} Dijo: ``Si Esaú llega a
una de las compañías y la golpea, la compañía que queda escapará''.

\hypertarget{la-oracion-de-jacob-por-la-ayuda-de-dios}{%
\subsection{La oracion de Jacob por la ayuda de
Dios}\label{la-oracion-de-jacob-por-la-ayuda-de-dios}}

\bibleverse{9} Jacob dijo: ``Dios de mi padre Abraham, y Dios de mi
padre Isaac, Yahvé, que me dijo: `Vuelve a tu país y a tus parientes, y
yo te haré el bien', \footnote{\textbf{32:9} Gén 31,3; Gén 31,13}
\bibleverse{10} No soy digno de la menor de todas las bondades y de toda
la verdad que has mostrado a tu siervo, pues sólo con mi bastón crucé
este Jordán, y ahora me he convertido en dos compañías. \footnote{\textbf{32:10}
  2Sam 7,18} \bibleverse{11} Por favor, líbrame de la mano de mi
hermano, de la mano de Esaú; porque le temo, no sea que venga y me hiera
a mí y a las madres con los hijos. \bibleverse{12} Dijiste: `Ciertamente
te haré un bien y haré que tu descendencia sea como la arena del mar,
que no se puede contar porque es muy numerosa'\,''. \footnote{\textbf{32:12}
  Gén 28,13-14}

\hypertarget{jacob-enviuxe1-regalos-a-esau}{%
\subsection{Jacob enviá regalos a
Esau}\label{jacob-enviuxe1-regalos-a-esau}}

\bibleverse{13} Aquella noche se quedó allí y tomó de lo que llevaba
consigo un regalo para Esaú, su hermano \bibleverse{14} doscientas
cabras hembras y veinte machos cabríos, doscientas ovejas y veinte
carneros, \bibleverse{15} treinta camellos de leche y sus potros,
cuarenta vacas, diez toros, veinte asnos hembras y diez potros.
\bibleverse{16} Los entregó en manos de sus siervos, cada rebaño por
separado, y dijo a sus siervos: ``Pasad delante de mí y poned un espacio
entre rebaño y rebaño.'' \bibleverse{17} Y ordenó a los primeros que
dijeran: ``Cuando Esaú, mi hermano, se encuentre con vosotros y os
pregunte diciendo: ``¿De quién sois? ¿Adónde vas? ¿De quién son estos
que tienes delante?' \bibleverse{18} Entonces dirás: `Son de tu siervo,
de Jacob. Es un regalo enviado a mi señor, Esaú. He aquí que él también
está detrás de nosotros'\,''. \bibleverse{19} Mandó también al segundo,
al tercero y a todos los que seguían a los rebaños, diciendo: ``Así
hablaréis a Esaú cuando lo encontréis. \bibleverse{20} Diréis: ``No sólo
eso, sino que he aquí que tu siervo Jacob está detrás de nosotros''.
Porque, dijo, ``Lo apaciguaré con el presente que va delante de mí, y
después veré su rostro. Tal vez me acepte''.

\bibleverse{21} Así que el presente pasó ante él, y él mismo se quedó
aquella noche en el campamento.

\hypertarget{jacob-luchando-con-dios-por-la-noche}{%
\subsection{Jacob luchando con Dios por la
noche}\label{jacob-luchando-con-dios-por-la-noche}}

\bibleverse{22} Aquella noche se levantó y tomó a sus dos mujeres, a sus
dos siervos y a sus once hijos, y cruzó el vado del Jaboc.
\bibleverse{23} Los tomó y los hizo pasar por el arroyo, y envió lo que
tenía. \bibleverse{24} Jacob se quedó solo, y luchó allí con un hombre
hasta el amanecer. \footnote{\textbf{32:24} Os 12,3-4} \bibleverse{25}
Al ver que no prevalecía contra él, el hombre le tocó el hueco del
muslo, y el hueco del muslo de Jacob se tensó mientras luchaba.
\bibleverse{26} El hombre dijo: ``Déjame ir, porque amanece''. Jacob
dijo: ``No te dejaré ir si no me bendices''. \footnote{\textbf{32:26}
  Mat 15,22-28}

\bibleverse{27} Le dijo: ``¿Cuál es tu nombre?'' Dijo: ``Jacob''.

\bibleverse{28} Dijo: ``Tu nombre ya no se llamará Jacob, sino Israel,
porque has luchado con Dios y con los hombres, y has vencido.''
\footnote{\textbf{32:28} Gén 35,10}

\bibleverse{29} Jacob le preguntó: ``Por favor, dime tu nombre''. Le
dijo: ``¿Por qué preguntas cuál es mi nombre?''. Y allí lo bendijo.
\footnote{\textbf{32:29} Jue 13,17-18}

\bibleverse{30} Jacob llamó el nombre del lugar Peniel;\footnote{\textbf{32:30}
  Peniel significa ``rostro de Dios''.} porque dijo: ``He visto a Dios
cara a cara, y mi vida se ha conservado.'' \footnote{\textbf{32:30} Éxod
  33,20} \bibleverse{31} Al pasar por Peniel, el sol se puso sobre él, y
cojeó a causa de su muslo. \bibleverse{32} Por eso los hijos de Israel
no comen el tendón de la cadera, que está en el hueco del muslo, hasta
el día de hoy, porque tocó el hueco del muslo de Jacob en el tendón de
la cadera.

\hypertarget{la-reconciliaciuxf3n-de-jacob-con-esauxfa}{%
\subsection{La reconciliación de Jacob con
Esaú}\label{la-reconciliaciuxf3n-de-jacob-con-esauxfa}}

\hypertarget{section-32}{%
\section{33}\label{section-32}}

\bibleverse{1} Jacob alzó los ojos y miró, y he aquí que Esaú venía, y
con él cuatrocientos hombres. Repartió los niños entre Lea, Raquel y los
dos criados. \footnote{\textbf{33:1} Gén 32,6} \bibleverse{2} Puso a los
siervos y a sus hijos al frente, a Lea y a sus hijos después, y a Raquel
y a José en la retaguardia. \bibleverse{3} Él mismo pasó delante de
ellos y se inclinó hasta el suelo siete veces, hasta llegar cerca de su
hermano.

\bibleverse{4} Esaú corrió a su encuentro, lo abrazó, se echó a su
cuello y lo besó, y lloraron. \bibleverse{5} Levantó los ojos y vio a
las mujeres y a los niños, y dijo: ``¿Quiénes son estos que están
contigo?'' Dijo: ``Los hijos que Dios ha dado a tu siervo''. \footnote{\textbf{33:5}
  Sal 127,3} \bibleverse{6} Entonces las siervas se acercaron con sus
hijos y se inclinaron. \bibleverse{7} También Lía y sus hijos se
acercaron y se inclinaron. Después de ellos, José se acercó con Raquel,
y se inclinaron.

\bibleverse{8} Esaú dijo: ``¿Qué quieres decir con toda esta compañía
que he conocido?'' Jacob dijo: ``Para encontrar el favor a los ojos de
mi señor''. \footnote{\textbf{33:8} Gén 32,13-20}

\bibleverse{9} Esaú dijo: ``Tengo suficiente, hermano mío; que lo que
tienes sea tuyo''.

\bibleverse{10} Jacob dijo: ``Por favor, no, si ahora he encontrado
gracia ante tus ojos, recibe mi regalo de mi mano, porque he visto tu
rostro, como se ve el rostro de Dios, y te has complacido en mí.
\footnote{\textbf{33:10} 2Sam 14,17} \bibleverse{11} Toma, por favor, el
regalo que te he traído, porque Dios ha sido benévolo conmigo, y porque
tengo bastante''. Le instó, y lo tomó. \footnote{\textbf{33:11} 1Sam
  25,27; 1Sam 30,26}

\hypertarget{jacob-se-niega-a-escoltar-a-esauxfa-esto-vuelve-a-seir}{%
\subsection{Jacob se niega a escoltar a Esaú; esto vuelve a
Seir}\label{jacob-se-niega-a-escoltar-a-esauxfa-esto-vuelve-a-seir}}

\bibleverse{12} Esaú dijo: ``Emprendamos nuestro viaje y vayamos, y yo
iré delante de ti''.

\bibleverse{13} Jacob le dijo: ``Mi señor sabe que los niños son
tiernos, y que los rebaños y las manadas que están conmigo tienen sus
crías, y si un día se exceden, todos los rebaños morirán.
\bibleverse{14} Por favor, deja que mi señor pase delante de su siervo,
y yo seguiré con suavidad, según el paso del ganado que va delante de mí
y según el paso de los niños, hasta que llegue a mi señor a Seir.''

\bibleverse{15} Esaú dijo: ``Déjame ahora dejar contigo a algunos de los
que están conmigo''. Dijo: ``¿Por qué? Déjeme encontrar el favor a los
ojos de mi señor''.

\bibleverse{16} Así que Esaú regresó aquel día de camino a Seir.

\hypertarget{jacob-se-traslada-a-succoth-y-se-instala-con-sichuxeam}{%
\subsection{Jacob se traslada a Succoth y se instala con
Sichêm}\label{jacob-se-traslada-a-succoth-y-se-instala-con-sichuxeam}}

\bibleverse{17} Jacob viajó a Succoth, se construyó una casa e hizo
refugios para su ganado. Por eso el nombre del lugar se llama Sucot.
\footnote{\textbf{33:17} succoth significa refugios o cabinas.}

\bibleverse{18} Jacob llegó en paz a la ciudad de Siquem, que está en la
tierra de Canaán, cuando venía de Paddán Aram; y acampó ante la ciudad.
\bibleverse{19} Compró la parcela donde había tendido su tienda, de mano
de los hijos de Hamor, padre de Siquem, por cien monedas. \footnote{\textbf{33:19}
  Jos 24,32} \bibleverse{20} Levantó allí un altar y lo llamó El Elohe
Israel. \footnote{\textbf{33:20} El Elohe Israel significa ``Dios, el
  Dios de Israel'' o ``El Dios de Israel es poderoso''.} \footnote{\textbf{33:20}
  Gén 12,7-8}

\hypertarget{la-ofensa-de-sichuxeam-contra-dina}{%
\subsection{La ofensa de Sichêm contra
Dina}\label{la-ofensa-de-sichuxeam-contra-dina}}

\hypertarget{section-33}{%
\section{34}\label{section-33}}

\bibleverse{1} Dina, la hija de Lea, que dio a luz a Jacob, salió a ver
a las hijas de la tierra. \footnote{\textbf{34:1} Gén 30,21}
\bibleverse{2} La vio Siquem, hijo de Hamor el heveo, príncipe de la
tierra. La tomó, se acostó con ella y la humilló. \bibleverse{3} Su alma
se unió a Dina, la hija de Jacob, y amó a la joven, y le habló con
cariño. \bibleverse{4} Siquem habló con su padre, Hamor, diciendo:
``Consígueme a esta joven como esposa''.

\bibleverse{5} Jacob se enteró de que había mancillado a su hija Dina, y
sus hijos estaban con su ganado en el campo. Jacob calló hasta que
llegaron.

\hypertarget{hemor-corteja-a-dina-de-los-hijos-de-jacob}{%
\subsection{Hemor corteja a Dina de los hijos de
Jacob}\label{hemor-corteja-a-dina-de-los-hijos-de-jacob}}

\bibleverse{6} Hamor, el padre de Siquem, salió a buscar a Jacob para
hablar con él. \bibleverse{7} Los hijos de Jacob vinieron del campo
cuando lo oyeron. Los hombres se entristecieron y se enojaron mucho,
porque él había hecho una locura en Israel al acostarse con la hija de
Jacob, cosa que no debía hacerse. \footnote{\textbf{34:7} Deut 22,21}
\bibleverse{8} Hamor habló con ellos, diciendo: ``El alma de mi hijo
Siquem anhela a su hija. Por favor, dénsela como esposa. \bibleverse{9}
Hagan matrimonios con nosotros. Dennos sus hijas y tomen las nuestras
para ustedes. \bibleverse{10} Viviréis con nosotros, y la tierra estará
ante vosotros. Vivan y comercien en ella, y obtengan posesiones en
ella''.

\bibleverse{11} Siquem dijo a su padre y a sus hermanos: ``Dejadme
encontrar el favor de vuestros ojos, y todo lo que me digáis os lo daré.
\bibleverse{12} Pedidme una gran cantidad como dote, y os daré lo que me
pidáis, pero dadme a la joven como esposa''. \footnote{\textbf{34:12}
  Éxod 22,16}

\hypertarget{la-demanda-de-los-hijos-de-jacob-es-aceptada-por-los-sichuxeamitas}{%
\subsection{La demanda de los hijos de Jacob es aceptada por los
sichêmitas}\label{la-demanda-de-los-hijos-de-jacob-es-aceptada-por-los-sichuxeamitas}}

\bibleverse{13} Los hijos de Jacob respondieron a Siquem y a su padre
Hamor con engaño cuando hablaron, porque había mancillado a Dina, su
hermana, \bibleverse{14} y les dijeron: ``No podemos hacer esto,
entregar a nuestra hermana a un incircunciso; porque eso es un reproche
para nosotros. \bibleverse{15} Sólo con esta condición os consentiremos.
Si sois como nosotros, que todo varón de vosotros sea circuncidado,
\bibleverse{16} entonces os daremos nuestras hijas; y tomaremos vuestras
hijas para nosotros, y moraremos con vosotros, y seremos un solo pueblo.
\bibleverse{17} Pero si no nos escucháis y os circuncidáis, entonces
tomaremos a nuestra hermana,\footnote{\textbf{34:17} El hebreo tiene,
  literalmente, ``hija''} y nos iremos''.

\bibleverse{18} Sus palabras agradaron a Hamor y a Siquem, hijo de
Hamor. \bibleverse{19} El joven no esperó para hacer esto, porque se
había deleitado en la hija de Jacob, y fue honrado sobre toda la casa de
su padre. \bibleverse{20} Hamor y Siquem, su hijo, llegaron a la puerta
de su ciudad, y hablaron con los hombres de su ciudad, diciendo:
\bibleverse{21} ``Estos hombres son pacíficos con nosotros. Por lo
tanto, déjenlos vivir en la tierra y comerciar en ella. Porque he aquí
que la tierra es suficientemente grande para ellos. Tomemos a sus hijas
para nosotros como esposas, y démosles nuestras hijas. \bibleverse{22}
Sólo con esta condición los hombres consentirán en vivir con nosotros,
para ser un solo pueblo, si todo varón de entre nosotros se circuncida,
como ellos se circuncidan. \bibleverse{23} ¿No será nuestro su ganado y
sus posesiones y todos sus animales? Sólo démosles nuestro
consentimiento, y ellos habitarán con nosotros''.

\bibleverse{24} Todos los que salían de la puerta de su ciudad
escuchaban a Jamor y a su hijo Siquem, y todo varón era circuncidado,
todos los que salían de la puerta de su ciudad.

\hypertarget{la-venganza-engauxf1osa-de-los-hijos-de-jacob}{%
\subsection{La venganza engañosa de los hijos de
Jacob}\label{la-venganza-engauxf1osa-de-los-hijos-de-jacob}}

\bibleverse{25} Al tercer día, cuando ya estaban adoloridos, dos de los
hijos de Jacob, Simeón y Leví, hermanos de Dina, tomaron cada uno su
espada, vinieron a la ciudad desprevenida y mataron a todos los varones.
\footnote{\textbf{34:25} Gén 49,5-7} \bibleverse{26} Mataron a Hamor y a
Siquem, su hijo, a filo de espada, y sacaron a Dina de la casa de Siquem
y se fueron. \bibleverse{27} Los hijos de Jacob vinieron sobre los
muertos y saquearon la ciudad, porque habían profanado a su hermana.
\bibleverse{28} Tomaron sus rebaños, sus vacas, sus asnos, lo que había
en la ciudad, lo que había en el campo, \bibleverse{29} y toda su
riqueza. Llevaron cautivos a todos sus pequeños y a sus mujeres, y
tomaron como botín todo lo que había en la casa.

\hypertarget{el-disgusto-de-jacob-por-el-acto-reprensible-de-sus-hijos}{%
\subsection{El disgusto de Jacob por el acto reprensible de sus
hijos}\label{el-disgusto-de-jacob-por-el-acto-reprensible-de-sus-hijos}}

\bibleverse{30} Jacob dijo a Simeón y a Leví: ``Me habéis turbado para
hacerme odioso a los habitantes del país, entre los cananeos y los
ferezeos. Soy poco numeroso. Se reunirán contra mí y me golpearán, y
seré destruido, yo y mi casa''. \footnote{\textbf{34:30} Éxod 5,21}

\bibleverse{31} Dijeron: ``¿Debe tratar a nuestra hermana como a una
prostituta?''

\hypertarget{por-amonestaciuxf3n-de-dios-jacob-parte-de-siquem}{%
\subsection{Por amonestación de Dios, Jacob parte de
Siquem}\label{por-amonestaciuxf3n-de-dios-jacob-parte-de-siquem}}

\hypertarget{section-34}{%
\section{35}\label{section-34}}

\bibleverse{1} Dios dijo a Jacob: ``Levántate, sube a Betel y vive allí.
Haz allí un altar a Dios, que se te apareció cuando huías de la cara de
tu hermano Esaú''. \footnote{\textbf{35:1} Gén 28,12-19; Gén 31,13}

\bibleverse{2} Entonces Jacob dijo a su familia y a todos los que
estaban con él: ``Quitad los dioses extranjeros que hay entre vosotros,
purificaos y cambiad vuestros vestidos. \footnote{\textbf{35:2} Gén
  31,19; Jos 24,23; 1Sam 7,3} \bibleverse{3} Levantémonos y subamos a
Betel. Haré allí un altar a Dios, que me respondió en el día de mi
angustia y estuvo conmigo en el camino que recorrí.'' \footnote{\textbf{35:3}
  Gén 28,15; Gén 28,20-22}

\bibleverse{4} Entregaron a Jacob todos los dioses extranjeros que
tenían en sus manos, y los anillos que tenían en sus orejas; y Jacob los
escondió bajo la encina que estaba junto a Siquem. \footnote{\textbf{35:4}
  Jos 24,26; Jue 9,6} \bibleverse{5} Viajaron, y un terror de Dios
estaba sobre las ciudades que estaban alrededor, y no persiguieron a los
hijos de Jacob.

\hypertarget{llegada-de-jacob-y-construcciuxf3n-del-altar-en-betel}{%
\subsection{Llegada de Jacob y construcción del altar en
Betel}\label{llegada-de-jacob-y-construcciuxf3n-del-altar-en-betel}}

\bibleverse{6} Entonces Jacob llegó a Luz (es decir, Betel), que está en
la tierra de Canaán, él y todo el pueblo que estaba con él.
\bibleverse{7} Edificó allí un altar y llamó al lugar El Betel, porque
allí se le reveló Dios, cuando huía de la cara de su hermano.
\footnote{\textbf{35:7} Gén 12,8} \bibleverse{8} Murió Débora, la
nodriza de Rebeca, y fue enterrada debajo de Betel, bajo la encina; y su
nombre fue llamado Allon Bacuth.

\hypertarget{jacob-bendecido-por-dios}{%
\subsection{Jacob bendecido por Dios}\label{jacob-bendecido-por-dios}}

\bibleverse{9} Dios se le apareció de nuevo a Jacob, cuando venía de
Paddán Aram, y lo bendijo. \bibleverse{10} Dios le dijo: ``Tu nombre es
Jacob. Ya no te llamarás Jacob, sino que te llamarás Israel''. Le puso
el nombre de Israel. \footnote{\textbf{35:10} Gén 32,28} \bibleverse{11}
Dios le dijo: ``Yo soy el Dios Todopoderoso. Sé fecundo y multiplícate.
De ti saldrá una nación y una compañía de naciones, y de tu cuerpo
saldrán reyes. \footnote{\textbf{35:11} Gén 17,1; Gén 28,3-4; Gén 17,6}
\bibleverse{12} La tierra que di a Abraham y a Isaac, te la daré a ti, y
a tu descendencia después de ti le daré la tierra''.

\bibleverse{13} Dios se alejó de él en el lugar donde habló con él.
\footnote{\textbf{35:13} Gén 17,22} \bibleverse{14} Jacob levantó una
columna en el lugar donde habló con él, una columna de piedra. Derramó
sobre ella una libación y derramó sobre ella aceite. \footnote{\textbf{35:14}
  Gén 28,18-19} \bibleverse{15} Jacob llamó ``Betel'' al lugar donde
Dios habló con él.

\hypertarget{salida-de-betel-rahel-muere-al-dar-a-luz-a-benjamuxedn}{%
\subsection{Salida de Betel; Rahel muere al dar a luz a
Benjamín}\label{salida-de-betel-rahel-muere-al-dar-a-luz-a-benjamuxedn}}

\bibleverse{16} Viajaron desde Betel. Todavía faltaba una distancia para
llegar a Efraín, y Raquel estaba de parto. Tuvo un duro parto.
\bibleverse{17} Cuando estaba de parto, la partera le dijo: ``No temas,
porque ahora tendrás otro hijo.''

\bibleverse{18} Cuando su alma partió (pues murió), le puso el nombre de
Benoni,\footnote{\textbf{35:18} ``Benoni'' significa ``hijo de mi
  problema''.} pero su padre le puso el nombre de Benjamín. \footnote{\textbf{35:18}
  ``Benjamín'' significa ``hijo de mi mano derecha''.} \bibleverse{19}
Raquel murió y fue enterrada en el camino de Efrata (también llamada
Belén). \footnote{\textbf{35:19} Miq 5,2} \bibleverse{20} Jacob levantó
una columna sobre su tumba. El mismo es el pilar de la tumba de Raquel
hasta el día de hoy.

\hypertarget{la-indignaciuxf3n-de-rubens-los-doce-hijos-de-jacob-su-regreso-a-hebruxf3n-muerte-y-entierro-de-isaac}{%
\subsection{La indignación de Rubens; Los doce hijos de Jacob; su
regreso a Hebrón; Muerte y entierro de
Isaac}\label{la-indignaciuxf3n-de-rubens-los-doce-hijos-de-jacob-su-regreso-a-hebruxf3n-muerte-y-entierro-de-isaac}}

\bibleverse{21} Israel viajó y extendió su tienda más allá de la torre
de Eder. \footnote{\textbf{35:21} Miq 4,8} \bibleverse{22} Mientras
Israel vivía en esa tierra, Rubén fue y se acostó con Bilhá, la
concubina de su padre, e Israel se enteró. Los hijos de Jacob eran doce.
\footnote{\textbf{35:22} Gén 49,4} \bibleverse{23} Los hijos de Lea:
Rubén (primogénito de Jacob), Simeón, Leví, Judá, Isacar y Zabulón.
\bibleverse{24} Los hijos de Raquel: José y Benjamín. \bibleverse{25}
Los hijos de Bilhah (sierva de Raquel): Dan y Neftalí. \bibleverse{26}
Los hijos de Zilpa (sierva de Lea): Gad y Aser. Estos son los hijos de
Jacob, que le nacieron en Padan Aram. \bibleverse{27} Jacob vino a
Isaac, su padre, a Mamre, a Quiriat Arba (que es Hebrón), donde Abraham
e Isaac vivían como extranjeros.

\bibleverse{28} Los días de Isaac fueron ciento ochenta años.
\bibleverse{29} Isaac entregó el espíritu y murió, y fue reunido con su
pueblo, viejo y lleno de días. Esaú y Jacob, sus hijos, lo enterraron.
\footnote{\textbf{35:29} Gén 25,8}

\hypertarget{la-familia-y-la-residencia-de-esauxfa}{%
\subsection{La familia y la residencia de
Esaú}\label{la-familia-y-la-residencia-de-esauxfa}}

\hypertarget{section-35}{%
\section{36}\label{section-35}}

\bibleverse{1} Esta es la historia de las generaciones de Esaú (es
decir, Edom). \footnote{\textbf{36:1} Gén 25,30} \bibleverse{2} Esaú
tomó sus esposas de las hijas de Canaán: Ada, hija de Elón, el hitita; y
Oholibama, hija de Aná, hija de Zibeón, el heveo; \footnote{\textbf{36:2}
  Gén 26,34} \bibleverse{3} y Basemat, hija de Ismael, hermana de
Nebaiot. \footnote{\textbf{36:3} Gén 28,9} \bibleverse{4} Ada dio a luz
a Esaú, Elifaz. Basemat dio a luz a Reuel. \bibleverse{5} Oholibama dio
a luz a Jeús, Jalam y Coré. Estos son los hijos de Esaú, que le nacieron
en la tierra de Canaán. \bibleverse{6} Esaú tomó a sus esposas, a sus
hijos, a sus hijas y a todos los miembros de su familia, con su ganado,
todos sus animales y todas sus posesiones, que había reunido en la
tierra de Canaán, y se fue a una tierra alejada de su hermano Jacob.
\bibleverse{7} Porque su riqueza era demasiado grande para que pudieran
habitar juntos, y la tierra de sus viajes no podía soportarlos a causa
de su ganado. \footnote{\textbf{36:7} Gén 13,6} \bibleverse{8} Esaú
vivió en la región montañosa de Seir. Esaú es Edom.

\hypertarget{los-hijos-y-nietos-de-esauxfa-como-progenitores}{%
\subsection{Los hijos y nietos de Esaú como
progenitores}\label{los-hijos-y-nietos-de-esauxfa-como-progenitores}}

\bibleverse{9} Esta es la historia de las generaciones de Esaú, padre de
los edomitas, en la región montañosa de Seír: \bibleverse{10} Estos son
los nombres de los hijos de Esaú Elifaz, hijo de Ada, esposa de Esaú; y
Reuel, hijo de Basemat, esposa de Esaú. \bibleverse{11} Los hijos de
Elifaz fueron Temán, Omar, Zefo, Gatam y Cenaz. \bibleverse{12} Timna
fue concubina de Elifaz, hijo de Esaú, y dio a luz a Amalec. Estos son
los descendientes de Ada, esposa de Esaú. \bibleverse{13} Estos son los
hijos de Reuel: Nahath, Zerah, Shammah y Mizzah. Estos fueron los
descendientes de Basemat, esposa de Esaú. \bibleverse{14} Estos fueron
los hijos de Oholibama, hija de Aná, hija de Zibeón, mujer de Esaú; ella
dio a luz a Esaú: Jeús, Jalam y Coré.

\hypertarget{los-duques-descendieron-de-esauxfa}{%
\subsection{Los duques descendieron de
Esaú}\label{los-duques-descendieron-de-esauxfa}}

\bibleverse{15} Estos son los jefes de los hijos de Esaú: los hijos de
Elifaz, primogénito de Esaú: el jefe Temán, el jefe Omar, el jefe Zefo,
el jefe Cenaz, \bibleverse{16} el jefe Coré, el jefe Gatam, el jefe
Amalec. Estos son los jefes que vinieron de Elifaz en la tierra de Edom.
Estos son los hijos de Ada. \bibleverse{17} Estos son los hijos de
Reuel, hijo de Esaú: el jefe Nahat, el jefe Zerah, el jefe Shammah, el
jefe Mizzah. Estos son los jefes que vinieron de Reuel en la tierra de
Edom. Estos son los hijos de Basemat, mujer de Esaú. \bibleverse{18}
Estos son los hijos de Oholibama, mujer de Esaú: el jefe Jeús, el jefe
Jalam y el jefe Coré. Estos son los jefes que vinieron de Oholibama,
hija de Aná, mujer de Esaú. \bibleverse{19} Estos son los hijos de Esaú
(es decir, Edom), y estos son sus jefes.

\hypertarget{los-horeos-que-eran-independientes-de-esauxfa}{%
\subsection{Los horeos que eran independientes de
Esaú}\label{los-horeos-que-eran-independientes-de-esauxfa}}

\bibleverse{20} Estos son los hijos de Seir el horeo, los habitantes de
la tierra: Lotán, Sobal, Zibeón, Aná, \footnote{\textbf{36:20} Gén 14,6;
  Deut 2,12} \bibleverse{21} Disón, Ezer y Disán. Estos son los jefes
que vinieron de los horeos, los hijos de Seír en la tierra de Edom.
\bibleverse{22} Los hijos de Lotán fueron Hori y Hemán. La hermana de
Lotán fue Timna. \bibleverse{23} Estos son los hijos de Sobal: Alván,
Manahat, Ebal, Safo y Onam. \bibleverse{24} Estos son los hijos de
Zibeón Aiah y Anah. Este es Aná, que encontró las aguas termales en el
desierto, mientras alimentaba a los asnos de Zibeón, su padre.
\bibleverse{25} Estos son los hijos de Aná: Disón y Oholibama, hija de
Aná. \bibleverse{26} Estos son los hijos de Disón: Hemdán, Eshbán, Itrán
y Querán. \bibleverse{27} Estos son los hijos de Ezer Bilhán, Zaaván y
Acán. \bibleverse{28} Estos son los hijos de Disán: Uz y Arán.
\bibleverse{29} Estos son los jefes que vinieron de los horeos: el jefe
Lotán, el jefe Sobal, el jefe Zibeón, el jefe Aná, \bibleverse{30} el
jefe Disón, el jefe Ezer y el jefe Disán. Estos son los jefes que
vinieron de los horeos, según sus jefes en la tierra de Seir.

\hypertarget{los-reyes-de-la-tierra-de-edom-hasta-david}{%
\subsection{Los reyes de la tierra de Edom hasta
David}\label{los-reyes-de-la-tierra-de-edom-hasta-david}}

\bibleverse{31} Estos son los reyes que reinaron en la tierra de Edom,
antes de que ningún rey reinara sobre los hijos de Israel.
\bibleverse{32} Bela, hijo de Beor, reinó en Edom. El nombre de su
ciudad fue Dinhabah. \bibleverse{33} Bela murió, y en su lugar reinó
Jobab, hijo de Zera de Bosra. \bibleverse{34} Murió Jobab, y en su lugar
reinó Husam, del país de los temanitas. \bibleverse{35} Murió Husam, y
reinó en su lugar Hadad, hijo de Bedad, que hirió a Madián en el campo
de Moab. El nombre de su ciudad fue Avit. \bibleverse{36} Murió Hadad, y
en su lugar reinó Samá de Masreca. \bibleverse{37} Murió Samá, y en su
lugar reinó Saúl, de Rehobot, junto al río. \bibleverse{38} Murió Saúl,
y en su lugar reinó Baal Hanán, hijo de Acbor. \bibleverse{39} Murió
Baal Hanán, hijo de Acbor, y en su lugar reinó Hadar. El nombre de su
ciudad fue Pau. Su esposa se llamaba Mehetabel, hija de Matred, hija de
Mezahab.

\hypertarget{los-duques-de-edom-por-sus-lugares}{%
\subsection{Los duques de Edom por sus
lugares}\label{los-duques-de-edom-por-sus-lugares}}

\bibleverse{40} Estos son los nombres de los jefes que vinieron de Esaú,
según sus familias, por sus lugares y por sus nombres: el jefe Timna, el
jefe Alvah, el jefe Jetheth, \bibleverse{41} el jefe Oholibamah, el jefe
Elah, el jefe Pinon, \bibleverse{42} el jefe Kenaz, el jefe Teman, el
jefe Mibzar, \bibleverse{43} el jefe Magdiel y el jefe Iram. Estos son
los jefes de Edom, según sus domicilios en la tierra de su posesión.
Este es Esaú, el padre de los edomitas.

\hypertarget{los-inicios-de-la-enemistad-de-los-hermanos-contra-josuxe9}{%
\subsection{Los inicios de la enemistad de los hermanos contra
José}\label{los-inicios-de-la-enemistad-de-los-hermanos-contra-josuxe9}}

\hypertarget{section-36}{%
\section{37}\label{section-36}}

\bibleverse{1} Jacob vivía en la tierra de los viajes de su padre, en la
tierra de Canaán. \bibleverse{2} Esta es la historia de las generaciones
de Jacob. José, teniendo diecisiete años, apacentaba el rebaño con sus
hermanos. Era un muchacho con los hijos de Bilha y Zilpa, las esposas de
su padre. José informó a su padre sobre su maldad. \bibleverse{3} Israel
amaba a José más que a todos sus hijos, porque era el hijo de su vejez,
y le hizo una túnica de muchos colores. \bibleverse{4} Sus hermanos
vieron que su padre lo amaba más que a todos sus hermanos, y lo odiaron
y no pudieron hablarle en paz.

\hypertarget{los-sueuxf1os-de-josuxe9}{%
\subsection{Los sueños de José}\label{los-sueuxf1os-de-josuxe9}}

\bibleverse{5} José soñó un sueño, y se lo contó a sus hermanos, y éstos
lo odiaron aún más. \bibleverse{6} Les dijo: ``Escuchad este sueño que
he soñado: \bibleverse{7} porque he aquí que estábamos atando gavillas
en el campo, y he aquí que mi gavilla se levantó y también se puso de
pie; y he aquí que vuestras gavillas se acercaron y se inclinaron hacia
mi gavilla.''

\bibleverse{8} Sus hermanos le preguntaron: ``¿De verdad vas a reinar
sobre nosotros? ¿Realmente tendrás dominio sobre nosotros?'' Lo odiaban
aún más por sus sueños y por sus palabras. \bibleverse{9} Soñó aún otro
sueño y lo contó a sus hermanos, diciendo: ``He aquí que he soñado otro
sueño, y he aquí que el sol, la luna y once estrellas se inclinaban ante
mí.'' \bibleverse{10} Se lo contó a su padre y a sus hermanos. Su padre
lo reprendió y le dijo: ``¿Qué es este sueño que has soñado? ¿Acaso yo y
tu madre y tus hermanos vendremos a postrarnos en la tierra ante ti?''
\bibleverse{11} Sus hermanos le envidiaban, pero su padre tenía presente
esta frase.

\hypertarget{la-oportunidad-de-deshacerse-de-joseph}{%
\subsection{La oportunidad de deshacerse de
Joseph}\label{la-oportunidad-de-deshacerse-de-joseph}}

\bibleverse{12} Sus hermanos fueron a apacentar el rebaño de su padre en
Siquem. \footnote{\textbf{37:12} Gén 33,18-19} \bibleverse{13} Israel
dijo a José: ``¿No están tus hermanos apacentando el rebaño en Siquem?
Ven, y te enviaré con ellos''. Él le respondió: ``Aquí estoy''.

\bibleverse{14} Le dijo: ``Ve ahora a ver si les va bien a tus hermanos
y al rebaño, y tráeme otra vez la noticia''. Y lo envió fuera del valle
de Hebrón, y llegó a Siquem. \footnote{\textbf{37:14} Gén 35,27}
\bibleverse{15} Cierto hombre lo encontró, y he aquí que estaba vagando
por el campo. El hombre le preguntó: ``¿Qué buscas?''

\bibleverse{16} Dijo: ``Busco a mis hermanos. Dime, por favor, dónde
están apacentando el rebaño''.

\bibleverse{17} El hombre dijo: ``Se han ido de aquí, porque les he oído
decir: ``Vamos a Dotán''\,''. José fue tras sus hermanos y los encontró
en Dotán. \bibleverse{18} Lo vieron de lejos, y antes de que se acercara
a ellos, conspiraron contra él para matarlo. \bibleverse{19} Se decían
unos a otros: ``He aquí que viene este soñador. \bibleverse{20} Venid,
pues, y matémosle, y echémosle en uno de los pozos, y diremos: `Un
animal malvado le ha devorado'. Veremos qué será de sus sueños''.

\hypertarget{rubuxe9n-y-juduxe1-intentan-salvar-a-josuxe9}{%
\subsection{Rubén y Judá intentan salvar a
José}\label{rubuxe9n-y-juduxe1-intentan-salvar-a-josuxe9}}

\bibleverse{21} Rubén lo oyó y lo libró de sus manos y dijo: ``No le
quitemos la vida''. \footnote{\textbf{37:21} Gén 42,22} \bibleverse{22}
Rubén les dijo: ``No derramen sangre. Arrojadlo a este pozo que está en
el desierto, pero no le pongáis la mano encima'', para librarlo de sus
manos y devolverlo a su padre. \bibleverse{23} Cuando José llegó a manos
de sus hermanos, éstos le quitaron la túnica de muchos colores que
llevaba puesta; \footnote{\textbf{37:23} Gén 37,3} \bibleverse{24} lo
tomaron y lo arrojaron a la fosa. La fosa estaba vacía. No había agua en
ella. \footnote{\textbf{37:24} Jer 38,6}

\bibleverse{25} Se sentaron a comer el pan, y levantaron los ojos y
miraron, y vieron que una caravana de ismaelitas venía de Galaad, con
sus camellos cargados de especias, bálsamo y mirra, que iban a llevar a
Egipto. \bibleverse{26} Judá dijo a sus hermanos: ``¿De qué nos sirve
matar a nuestro hermano y ocultar su sangre? \bibleverse{27} Vengan y
vendámoslo a los ismaelitas, y que nuestra mano no lo toque, porque es
nuestro hermano, nuestra carne.'' Sus hermanos le hicieron caso.

\hypertarget{josuxe9-es-vendido-a-egipto}{%
\subsection{José es vendido a
Egipto}\label{josuxe9-es-vendido-a-egipto}}

\bibleverse{28} Pasaron unos madianitas que eran mercaderes, y sacaron y
levantaron a José de la fosa, y vendieron a José a los ismaelitas por
veinte monedas de plata. Los mercaderes llevaron a José a Egipto.
\footnote{\textbf{37:28} Gén 25,2}

\bibleverse{29} Rubén volvió a la fosa y vio que José no estaba en ella,
y se rasgó las vestiduras. \footnote{\textbf{37:29} Gén 44,13; 2Sam 1,11}
\bibleverse{30} Volvió a sus hermanos y dijo: ``El niño ya no está; y
yo, ¿a dónde iré?'' \bibleverse{31} Tomaron la túnica de José, mataron
un macho cabrío y mojaron la túnica en la sangre. \bibleverse{32}
Tomaron la túnica de muchos colores, la llevaron a su padre y le
dijeron: ``Hemos encontrado esto. Examínala ahora y comprueba si es la
túnica de tu hijo o no''.

\hypertarget{el-dolor-de-jacob-josuxe9-vendido-a-potifar-en-egipto}{%
\subsection{El dolor de Jacob; José vendido a Potifar en
Egipto}\label{el-dolor-de-jacob-josuxe9-vendido-a-potifar-en-egipto}}

\bibleverse{33} Lo reconoció y dijo: ``Es la túnica de mi hijo. Un
animal malvado lo ha devorado. Sin duda, José está despedazado''.
\footnote{\textbf{37:33} Gén 37,20} \bibleverse{34} Jacob se rasgó las
vestiduras, se puso tela de saco en la cintura y lloró a su hijo durante
muchos días. \footnote{\textbf{37:34} Gén 37,29} \bibleverse{35} Todos
sus hijos y todas sus hijas se levantaron para consolarlo, pero él se
negó a ser consolado. Dijo: ``Porque bajaré al Seol\footnote{\textbf{37:35}
  El Seol es el lugar de los muertos.} a mi hijo, de luto''. Su padre
lloró por él. \bibleverse{36} Los madianitas lo vendieron a Egipto a
Potifar, un oficial del Faraón, el capitán de la guardia.

\hypertarget{los-hijos-de-juda-y-thamar}{%
\subsection{Los hijos de Juda y
Thamar}\label{los-hijos-de-juda-y-thamar}}

\hypertarget{section-37}{%
\section{38}\label{section-37}}

\bibleverse{1} En aquel tiempo, Judá bajó de entre sus hermanos y visitó
a un adulamita que se llamaba Hira. \bibleverse{2} Allí, Judá vio a la
hija de un cananeo llamado Súa. La tomó y se acercó a ella.
\bibleverse{3} Ella concibió y dio a luz un hijo, al que llamó Er.
\bibleverse{4} Concibió de nuevo y dio a luz un hijo, al que llamó Onán.
\bibleverse{5} Concibió de nuevo y dio a luz un hijo, al que llamó Sela.
Estaba en Chezib cuando lo dio a luz. \bibleverse{6} Judá tomó una
esposa para Er, su primogénito, y su nombre fue Tamar. \bibleverse{7}
Er, el primogénito de Judá, era malvado a los ojos de Yavé. Así que
Yahvé lo mató. \bibleverse{8} Judá le dijo a Onán: ``Acércate a la mujer
de tu hermano y cumple con ella el deber de un marido hermano, y cría
descendencia para tu hermano.'' \footnote{\textbf{38:8} Deut 25,5}
\bibleverse{9} Onán sabía que la descendencia no sería suya; y cuando
entró a la mujer de su hermano, derramó su semen en el suelo, para no
dar descendencia a su hermano. \bibleverse{10} Lo que hizo fue malo a
los ojos de Yavé, y también lo mató. \bibleverse{11} Entonces Judá le
dijo a Tamar, su nuera: ``Quédate viuda en la casa de tu padre hasta que
crezca Sela, mi hijo'', pues dijo: ``No sea que él también muera como
sus hermanos''. Tamar se fue a vivir a la casa de su padre.

\hypertarget{thamar-usa-astucia-para-obtener-descendencia-de-su-suegro-judah}{%
\subsection{Thamar usa astucia para obtener descendencia de su suegro
Judah}\label{thamar-usa-astucia-para-obtener-descendencia-de-su-suegro-judah}}

\bibleverse{12} Después de muchos días, murió la hija de Súa, esposa de
Judá. Judá se consoló y subió con sus esquiladores de ovejas a Timná, él
y su amigo Hira, el adulamita. \bibleverse{13} Le dijeron a Tamar:
``Mira, tu suegro sube a Timná a esquilar sus ovejas''. \bibleverse{14}
Ella se quitó las prendas de su viudez, se cubrió con su velo y se
envolvió, y se sentó en la puerta de Enaim, que está en el camino de
Timná, porque vio que Selá era mayor, y que ella no le había sido dada
como esposa. \bibleverse{15} Cuando Judá la vio, pensó que era una
prostituta, pues se había cubierto el rostro. \bibleverse{16} Se dirigió
a ella por el camino y le dijo: ``Por favor, ven, déjame entrar
contigo'', pues no sabía que era su nuera. Ella dijo: ``¿Qué me darás,
para que puedas entrar en mí?'' \footnote{\textbf{38:16} Lev 18,15}

\bibleverse{17} Dijo: ``Te enviaré un cabrito del rebaño''. Ella dijo:
``¿Me darás una prenda, hasta que la envíes?''

\bibleverse{18} Él dijo: ``¿Qué prenda te daré?'' Ella dijo: ``Tu sello
y tu cordón, y tu bastón que está en tu mano''. Se los dio, y entró en
ella, y ella concibió por él. \bibleverse{19} Ella se levantó y se fue,
y se quitó el velo de encima y se puso las ropas de su viudez.
\bibleverse{20} Judá envió al cabrito de la mano de su amigo, el
adulamita, a recibir la prenda de la mano de la mujer, pero no la
encontró. \bibleverse{21} Entonces preguntó a los hombres de su lugar,
diciendo: ``¿Dónde está la prostituta que estaba en Enaim, junto al
camino?'' Dijeron: ``Aquí no ha habido ninguna prostituta''.

\bibleverse{22} Volvió a Judá y le dijo: ``No la he encontrado; y
también los hombres del lugar dijeron: ``Aquí no ha habido ninguna
prostituta''\,''. \bibleverse{23} Judá dijo: ``Que se quede con ella, no
sea que nos avergoncemos. He aquí que he enviado esta cabrita, y no la
has encontrado''.

\hypertarget{judas-juicio-justo-sobre-suxed-mismo-y-thamar}{%
\subsection{Judas juicio justo sobre sí mismo y
Thamar}\label{judas-juicio-justo-sobre-suxed-mismo-y-thamar}}

\bibleverse{24} Unos tres meses después, se le dijo a Judá: ``Tamar, tu
nuera, se ha prostituido. Además, he aquí que está embarazada por
prostitución''. Judá dijo: ``Sácala y que la quemen''. \bibleverse{25}
Cuando la sacaron, envió a decir a su suegro: ``Estoy embarazada del
hombre que tiene esto''. También le dijo: ``Por favor, discierne de
quién son estos: el sello, los cordones y el bastón''.

\bibleverse{26} Judá los reconoció y dijo: ``Ella es más justa que yo,
porque no se la di a Sela, mi hijo''. No volvió a conocerla.

\hypertarget{thamar-da-a-luz-a-los-gemelos-puxe9rez-y-serah}{%
\subsection{Thamar da a luz a los gemelos Pérez y
Serah}\label{thamar-da-a-luz-a-los-gemelos-puxe9rez-y-serah}}

\bibleverse{27} En el tiempo de su parto, he aquí que había gemelos en
su seno. \bibleverse{28} Cuando dio a luz, uno de ellos sacó una mano, y
la partera tomó y ató un hilo de grana en su mano, diciendo: ``Este
salió primero.'' \bibleverse{29} Al retirar la mano, he aquí que su
hermano salió, y ella le dijo: ``¿Por qué te has hecho una brecha?'' Por
eso se llamó Pérez. \footnote{\textbf{38:29} Pérez significa ``romper''.}
\footnote{\textbf{38:29} Mat 1,3} \bibleverse{30} Después salió su
hermano, que tenía el hilo escarlata en la mano, y se llamó Zerah.
\footnote{\textbf{38:30} Zerah significa ``escarlata'' o ``brillo''.}

\hypertarget{josuxe9-en-la-casa-de-potifar}{%
\subsection{José en la casa de
Potifar}\label{josuxe9-en-la-casa-de-potifar}}

\hypertarget{section-38}{%
\section{39}\label{section-38}}

\bibleverse{1} José fue llevado a Egipto. Potifar, un oficial del
Faraón, el capitán de la guardia, un egipcio, lo compró de la mano de
los ismaelitas que lo habían hecho descender. \footnote{\textbf{39:1}
  Gén 37,28} \bibleverse{2} El Señor estaba con José, y éste era un
hombre próspero. Estaba en la casa de su amo el egipcio. \bibleverse{3}
Su amo vio que Yavé estaba con él, y que Yavé hacía prosperar en su mano
todo lo que hacía. \bibleverse{4} José halló gracia ante sus ojos. Le
sirvió, y Potifar lo nombró supervisor de su casa, y todo lo que tenía
lo puso en sus manos. \bibleverse{5} Desde el momento en que lo nombró
supervisor de su casa y de todo lo que tenía, Yavé bendijo la casa del
egipcio por causa de José. La bendición del Señor recayó sobre todo lo
que tenía, en la casa y en el campo. \footnote{\textbf{39:5} Gén 30,27}
\bibleverse{6} Dejó todo lo que tenía en manos de José. No se preocupó
por nada, excepto por la comida que comía. José era bien parecido y
guapo.

\hypertarget{la-seducciuxf3n-de-la-esposa-de-potifar}{%
\subsection{La seducción de la esposa de
Potifar}\label{la-seducciuxf3n-de-la-esposa-de-potifar}}

\bibleverse{7} Después de esto, la mujer de su amo puso sus ojos en José
y le dijo: ``Acuéstate conmigo''. \footnote{\textbf{39:7} Prov 5,3}

\bibleverse{8} Pero él se negó y dijo a la mujer de su amo: ``He aquí
que mi amo no sabe lo que hay conmigo en la casa, y ha puesto en mi mano
todo lo que tiene. \bibleverse{9} Nadie es mayor que yo en esta casa, y
no me ha ocultado nada más que a ti, porque eres su mujer. ¿Cómo, pues,
puedo hacer esta gran maldad, y pecar contra Dios?'' \footnote{\textbf{39:9}
  Éxod 20,14}

\bibleverse{10} Mientras ella le hablaba a José cada día, él no la
escuchaba, ni se acostaba junto a ella, ni estaba con ella.
\bibleverse{11} Por aquel entonces, él entró en la casa para hacer su
trabajo, y no había ninguno de los hombres de la casa dentro.
\bibleverse{12} Ella lo agarró por el manto, diciendo: ``Acuéstate
conmigo''. Él dejó su manto en la mano de ella y salió corriendo.
\bibleverse{13} Cuando ella vio que él había dejado su manto en la mano
de ella, y había corrido afuera, \bibleverse{14} llamó a los hombres de
su casa, y les habló diciendo: ``He aquí, él ha traído a un hebreo para
burlarse de nosotros. Entró en mi casa para acostarse conmigo, y yo
grité con fuerza. \bibleverse{15} Cuando oyó que yo levantaba la voz y
gritaba, dejó su manto junto a mí y salió corriendo.'' \bibleverse{16}
Ella dejó su ropa junto a ella, hasta que su amo volvió a casa.
\bibleverse{17} Ella le habló según estas palabras, diciendo: ``El
siervo hebreo que nos has traído, entró a burlarse de mí,
\bibleverse{18} y al levantar mi voz y gritar, dejó su ropa junto a mí y
salió corriendo.''

\hypertarget{josuxe9-en-el-carcel}{%
\subsection{José en el Carcel}\label{josuxe9-en-el-carcel}}

\bibleverse{19} Cuando su amo oyó las palabras de su mujer, que le dijo:
``Esto es lo que me hizo tu siervo'', se encendió su ira.
\bibleverse{20} El amo de José lo apresó y lo metió en la cárcel, el
lugar donde estaban atados los prisioneros del rey, y allí estuvo
detenido. \bibleverse{21} Pero el Señor estaba con José, y se mostró
bondadoso con él, y le dio favor a los ojos del guardián de la prisión.
\bibleverse{22} El guardián de la cárcel puso en manos de José a todos
los presos que estaban en la cárcel. Todo lo que hicieran allí, él era
responsable de ello. \bibleverse{23} El guardián de la cárcel no se
ocupaba de nada de lo que estaba bajo su mano, porque el Señor estaba
con él; y lo que él hacía, el Señor lo hacía prosperar.

\hypertarget{encarcelamiento-del-copero-y-panadero-del-farauxf3n}{%
\subsection{Encarcelamiento del copero y panadero del
faraón}\label{encarcelamiento-del-copero-y-panadero-del-farauxf3n}}

\hypertarget{section-39}{%
\section{40}\label{section-39}}

\bibleverse{1} Después de estas cosas, el copero del rey de Egipto y su
panadero ofendieron a su señor, el rey de Egipto. \bibleverse{2} El
faraón se enojó con sus dos oficiales, el jefe de los coperos y el jefe
de los panaderos. \bibleverse{3} Los puso en custodia en la casa del
capitán de la guardia, en la cárcel, el lugar donde estaba atado José.
\footnote{\textbf{40:3} Gén 39,20} \bibleverse{4} El capitán de la
guardia se los asignó a José, y él se ocupó de ellos. Permanecieron en
la cárcel muchos días.

\hypertarget{josuxe9-consuela-a-los-dos-oficiales-de-la-corte}{%
\subsection{José consuela a los dos oficiales de la
corte}\label{josuxe9-consuela-a-los-dos-oficiales-de-la-corte}}

\bibleverse{5} Ambos soñaron un sueño, cada uno su sueño, en una noche,
cada uno según la interpretación de su sueño, el copero y el panadero
del rey de Egipto, que estaban atados en la cárcel. \bibleverse{6} José
entró a ellos por la mañana, los vio y vio que estaban tristes.
\bibleverse{7} Preguntó a los oficiales del faraón que estaban con él
detenidos en la casa de su amo, diciendo: ``¿Por qué parecen tan tristes
hoy?''

\bibleverse{8} Le dijeron: ``Hemos soñado un sueño y no hay nadie que
pueda interpretarlo''. José les dijo: ``¿Las interpretaciones no son de
Dios? Por favor, díganmelo a mí''. \footnote{\textbf{40:8} Gén 41,16;
  Dan 2,27-28}

\hypertarget{el-sueuxf1o-del-copero-y-su-interpretaciuxf3n}{%
\subsection{El sueño del copero y su
interpretación}\label{el-sueuxf1o-del-copero-y-su-interpretaciuxf3n}}

\bibleverse{9} El jefe de los coperos contó su sueño a José y le dijo:
``En mi sueño, he aquí que una vid estaba delante de mí, \bibleverse{10}
y en la vid había tres sarmientos. Era como si hubiera brotado,
florecido, y sus racimos producían uvas maduras. \bibleverse{11} La copa
del faraón estaba en mi mano; tomé las uvas, las exprimí en la copa del
faraón y entregué la copa en la mano del faraón.''

\bibleverse{12} José le dijo: ``Esta es su interpretación: las tres
ramas son tres días. \bibleverse{13} Dentro de tres días más, el faraón
levantará tu cabeza y te devolverá tu cargo. Entregarás la copa del
Faraón en su mano, como lo hacías cuando eras su copero. \bibleverse{14}
Pero acuérdate de mí cuando te vaya bien. Por favor, muéstrate bondadoso
conmigo, y mencióname ante el Faraón, y sácame de esta casa.
\bibleverse{15} Porque ciertamente fui robado de la tierra de los
hebreos, y aquí tampoco he hecho nada para que me metan en el
calabozo.'' \footnote{\textbf{40:15} Gén 37,28}

\hypertarget{el-sueuxf1o-del-panadero-y-su-interpretaciuxf3n}{%
\subsection{El sueño del panadero y su
interpretación}\label{el-sueuxf1o-del-panadero-y-su-interpretaciuxf3n}}

\bibleverse{16} Cuando el jefe de los panaderos vio que la
interpretación era buena, dijo a José: ``Yo también estaba en mi sueño,
y he aquí que tres cestas de pan blanco estaban sobre mi cabeza.
\bibleverse{17} En el cesto de arriba había toda clase de alimentos
horneados para el Faraón, y las aves los comían del cesto sobre mi
cabeza.''

\bibleverse{18} José respondió: ``Esta es su interpretación. Los tres
cestos son tres días. \bibleverse{19} Dentro de tres días más, el Faraón
levantará tu cabeza de encima, te colgará en un árbol y las aves comerán
tu carne de encima.''

\hypertarget{el-cumplimiento-de-ambos-sueuxf1os}{%
\subsection{El cumplimiento de ambos
sueños}\label{el-cumplimiento-de-ambos-sueuxf1os}}

\bibleverse{20} Al tercer día, que era el cumpleaños del Faraón, éste
hizo un banquete para todos sus servidores, y levantó la cabeza del jefe
de los coperos y la del jefe de los panaderos entre sus servidores.
\bibleverse{21} Restituyó al jefe de los coperos a su puesto, y entregó
la copa a la mano del Faraón; \bibleverse{22} pero colgó al jefe de los
panaderos, como José les había interpretado. \bibleverse{23} Sin
embargo, el jefe de los coperos no se acordó de José, sino que lo
olvidó. \footnote{\textbf{40:23} Gén 40,14}

\hypertarget{los-dos-sueuxf1os-del-farauxf3n-son-insolubles-para-los-intuxe9rpretes-egipcios}{%
\subsection{Los dos sueños del faraón son insolubles para los
intérpretes
egipcios}\label{los-dos-sueuxf1os-del-farauxf3n-son-insolubles-para-los-intuxe9rpretes-egipcios}}

\hypertarget{section-40}{%
\section{41}\label{section-40}}

\bibleverse{1} Al cabo de dos años completos, Faraón soñó, y he aquí que
estaba junto al río. \bibleverse{2} He aquí que siete reses subían del
río. Estaban lisos y gordos, y se alimentaban en la hierba del pantano.
\bibleverse{3} He aquí que otras siete reses subían tras ellas del río,
feas y flacas, y se quedaban junto a las otras reses al borde del río.
\bibleverse{4} Las reses feas y flacas se comieron a las siete reses
lisas y gordas. Entonces el faraón se despertó. \bibleverse{5} Se durmió
y soñó por segunda vez; y he aquí que siete cabezas de grano surgían en
un solo tallo, sanas y buenas. \bibleverse{6} He aquí que siete cabezas
de grano, delgadas y arruinadas por el viento del este, brotaban tras
ellas. \bibleverse{7} Las cabezas de grano delgadas se tragaron las
siete espigas sanas y llenas. El faraón se despertó, y he aquí que era
un sueño. \bibleverse{8} Por la mañana, su espíritu se turbó y mandó
llamar a todos los magos y sabios de Egipto. El Faraón les contó sus
sueños, pero no había nadie que pudiera interpretárselos. \footnote{\textbf{41:8}
  Dan 2,2}

\hypertarget{el-copero-hace-los-arreglos-para-que-vayan-a-buscar-a-josuxe9}{%
\subsection{El copero hace los arreglos para que vayan a buscar a
José}\label{el-copero-hace-los-arreglos-para-que-vayan-a-buscar-a-josuxe9}}

\bibleverse{9} Entonces el jefe de los coperos habló al Faraón,
diciendo: ``Hoy me acuerdo de mis faltas. \bibleverse{10} El faraón se
enojó con sus servidores y me puso en custodia en la casa del capitán de
la guardia, con el jefe de los panaderos. \bibleverse{11} En una noche
soñamos un sueño, él y yo. Cada uno soñó según la interpretación de su
sueño. \bibleverse{12} Estaba allí con nosotros un joven hebreo,
sirviente del capitán de la guardia, y se lo contamos, y él nos
interpretó nuestros sueños. Él interpretó a cada uno según su sueño.
\bibleverse{13} Tal como nos lo interpretó, así fue. Me restituyó en mi
cargo, y lo colgó''.

\bibleverse{14} Entonces el Faraón envió a llamar a José, y lo sacaron
apresuradamente del calabozo. Se afeitó, se cambió de ropa y entró a ver
al Faraón.

\hypertarget{josuxe9-interpreta-los-sueuxf1os-del-farauxf3n}{%
\subsection{José interpreta los sueños del
faraón}\label{josuxe9-interpreta-los-sueuxf1os-del-farauxf3n}}

\bibleverse{15} El faraón dijo a José: ``He soñado un sueño, y no hay
nadie que pueda interpretarlo. He oído decir de ti que cuando oyes un
sueño puedes interpretarlo''.

\bibleverse{16} José respondió al Faraón diciendo: ``No está en mí. Dios
le dará al Faraón una respuesta de paz''. \footnote{\textbf{41:16} Gén
  40,8}

\bibleverse{17} El faraón habló a José: ``En mi sueño, he aquí que yo
estaba al borde del río; \bibleverse{18} y he aquí que siete reses
gordas y lisas subían del río. Se alimentaban en la hierba del pantano;
\bibleverse{19} y he aquí que otras siete reses subían tras ellas,
pobres y muy feas y flacas, como nunca vi en toda la tierra de Egipto
por su fealdad. \bibleverse{20} Las reses flacas y feas se comieron a
las primeras siete reses gordas; \bibleverse{21} y cuando se las
comieron, no se podía saber que se las habían comido, sino que seguían
siendo feas, como al principio. Entonces me desperté. \bibleverse{22} Vi
en mi sueño, y he aquí que siete cabezas de grano salían de un solo
tallo, llenas y buenas; \bibleverse{23} y he aquí que siete cabezas de
grano, marchitas, flacas y arrasadas por el viento del este, salían
detrás de ellas. \bibleverse{24} Las cabezas de grano flacas se tragaron
a las siete cabezas de grano buenas. Se lo conté a los magos, pero no
hubo nadie que pudiera explicármelo''.

\bibleverse{25} José dijo al Faraón: ``El sueño del Faraón es uno. Lo
que Dios va a hacer se lo ha declarado al Faraón. \bibleverse{26} Las
siete reses buenas son siete años, y las siete cabezas de grano buenas
son siete años. El sueño es uno. \bibleverse{27} Las siete reses flacas
y feas que subieron después de ellas son siete años, y también las siete
cabezas de grano vacías que fueron destruidas por el viento del este;
serán siete años de hambre. \bibleverse{28} Esto es lo que le he dicho
al Faraón. Dios ha mostrado al Faraón lo que va a hacer. \bibleverse{29}
He aquí que vienen siete años de gran abundancia en toda la tierra de
Egipto. \bibleverse{30} Después de ellos vendrán siete años de hambre, y
toda la abundancia será olvidada en la tierra de Egipto. El hambre
consumirá la tierra, \bibleverse{31} y la abundancia no se conocerá en
la tierra a causa de esa hambruna que sigue, pues será muy grave.
\bibleverse{32} El sueño se duplicó para el Faraón, porque la cosa está
establecida por Dios, y Dios la llevará a cabo en breve.

\hypertarget{el-consejo-de-josuxe9-por-el-farauxf3n}{%
\subsection{El consejo de José por el
faraón}\label{el-consejo-de-josuxe9-por-el-farauxf3n}}

\bibleverse{33} ``Ahora, pues, busque el Faraón un hombre discreto y
sabio, y póngalo sobre la tierra de Egipto. \bibleverse{34} Que el
Faraón haga esto, y que designe capataces sobre la tierra, y que recoja
la quinta parte de los productos de la tierra de Egipto en los siete
años de abundancia. \bibleverse{35} Que recojan todo el alimento de
estos años buenos que vienen, y que almacenen el grano bajo la mano del
Faraón para la alimentación en las ciudades, y que lo guarden.
\bibleverse{36} El alimento será para abastecer la tierra contra los
siete años de hambre que habrá en la tierra de Egipto, para que la
tierra no perezca por el hambre.''

\hypertarget{joseph-fue-ascendido-a-funcionario-muxe1s-alto-del-estado}{%
\subsection{Joseph fue ascendido a funcionario más alto del
estado}\label{joseph-fue-ascendido-a-funcionario-muxe1s-alto-del-estado}}

\bibleverse{37} La cosa fue buena a los ojos del Faraón y de todos sus
siervos. \bibleverse{38} El faraón dijo a sus siervos: ``¿Acaso podemos
encontrar a alguien como éste, un hombre en quien esté el Espíritu de
Dios?'' \footnote{\textbf{41:38} Prov 14,35} \bibleverse{39} El faraón
dijo a José: ``Porque Dios te ha mostrado todo esto, no hay nadie tan
discreto y sabio como tú. \bibleverse{40} Tú estarás al frente de mi
casa. Todo mi pueblo será gobernado según tu palabra. Sólo en el trono
seré más grande que tú''. \footnote{\textbf{41:40} Ecl 4,14; Sal 113,7;
  Sal 37,37} \bibleverse{41} El faraón dijo a José: ``He aquí que te he
puesto sobre toda la tierra de Egipto''. \footnote{\textbf{41:41} Hech
  7,10} \bibleverse{42} El faraón se quitó el anillo de sello de su mano
y lo puso en la de José; lo vistió con ropas de lino fino y le puso una
cadena de oro al cuello. \footnote{\textbf{41:42} Est 3,10; Est 8,2; Dan
  5,29} \bibleverse{43} Lo hizo montar en el segundo carro que tenía.
Gritaron ante él: ``¡Inclina la rodilla!'' Lo puso sobre toda la tierra
de Egipto. \bibleverse{44} El faraón dijo a José: ``Yo soy el faraón.
Sin ti, nadie levantará la mano ni el pie en toda la tierra de Egipto''.
\bibleverse{45} El faraón llamó a José Zafnat-Panea. Le dio por esposa a
Asenat, hija de Potifera, sacerdote de On. José salió a recorrer la
tierra de Egipto.

\hypertarget{medidas-de-josuxe9-durante-los-siete-auxf1os-fuxe9rtiles-el-nacimiento-de-sus-dos-hijos}{%
\subsection{Medidas de José durante los siete años fértiles; el
nacimiento de sus dos
hijos}\label{medidas-de-josuxe9-durante-los-siete-auxf1os-fuxe9rtiles-el-nacimiento-de-sus-dos-hijos}}

\bibleverse{46} José tenía treinta años cuando se presentó ante el
Faraón, rey de Egipto. José salió de la presencia del Faraón y recorrió
toda la tierra de Egipto. \bibleverse{47} En los siete años de
abundancia, la tierra produjo en abundancia. \bibleverse{48} Recogió
toda la comida de los siete años que había en la tierra de Egipto, y
guardó la comida en las ciudades. En cada ciudad almacenó alimentos de
los campos que rodeaban la ciudad. \bibleverse{49} José acumuló grano
como la arena del mar, mucho, hasta que dejó de contarlo, porque no
tenía número. \bibleverse{50} A José le nacieron dos hijos antes de que
llegara el año del hambre, que le dio a luz Asenat, hija de Potifera,
sacerdote de On. \bibleverse{51} José llamó al primogénito
Manasés,\footnote{\textbf{41:51} ``Manasés'' suena como el hebreo para
  ``olvidar''.} ``Porque'', dijo, ``Dios me ha hecho olvidar todo mi
trabajo y toda la casa de mi padre.'' \bibleverse{52} El nombre del
segundo, lo llamó Efraín:\footnote{\textbf{41:52} ``Efraín'' suena como
  el hebreo para ``dos veces fructífero''.} ``Porque Dios me ha hecho
fructificar en la tierra de mi aflicción.''

\hypertarget{los-siete-auxf1os-estuxe9riles-y-las-ventas-de-cereales-de-josuxe9-durante-la-hambruna}{%
\subsection{Los siete años estériles y las ventas de cereales de José
durante la
hambruna}\label{los-siete-auxf1os-estuxe9riles-y-las-ventas-de-cereales-de-josuxe9-durante-la-hambruna}}

\bibleverse{53} Los siete años de abundancia que hubo en la tierra de
Egipto llegaron a su fin. \bibleverse{54} Comenzaron a llegar los siete
años de hambre, tal como había dicho José. Hubo hambre en todas las
tierras, pero en toda la tierra de Egipto hubo pan. \bibleverse{55}
Cuando toda la tierra de Egipto estaba hambrienta, el pueblo clamó al
Faraón por pan, y el Faraón dijo a todos los egipcios: ``Vayan a José.
Haced lo que él os diga''. \bibleverse{56} El hambre se extendía por
toda la superficie de la tierra. José abrió todos los almacenes y vendió
a los egipcios. El hambre era grave en la tierra de Egipto.
\bibleverse{57} Todos los países vinieron a Egipto, a José, para comprar
grano, porque el hambre era grave en toda la tierra. \footnote{\textbf{41:57}
  Gén 12,10}

\hypertarget{los-diez-hijos-mayores-de-jacob-se-mudan-a-egipto-para-comprar-grano}{%
\subsection{Los diez hijos mayores de Jacob se mudan a Egipto para
comprar
grano}\label{los-diez-hijos-mayores-de-jacob-se-mudan-a-egipto-para-comprar-grano}}

\hypertarget{section-41}{%
\section{42}\label{section-41}}

\bibleverse{1} Vio Jacob que había grano en Egipto, y dijo a sus hijos:
``¿Por qué os miráis unos a otros?''. \bibleverse{2} Dijo: ``He aquí, he
oído que hay grano en Egipto. Bajad allí y comprad para nosotros de
allí, para que vivamos y no muramos''. \bibleverse{3} Los diez hermanos
de José bajaron a comprar grano a Egipto. \bibleverse{4} Pero Jacob no
envió a Benjamín, el hermano de José, con sus hermanos, porque dijo:
``No sea que le pase algo malo.'' \bibleverse{5} Los hijos de Israel
fueron a comprar entre los que venían, porque había hambre en la tierra
de Canaán.

\hypertarget{la-primera-conversaciuxf3n-dura-de-josuxe9-con-sus-hermanos}{%
\subsection{La primera conversación dura de José con sus
hermanos}\label{la-primera-conversaciuxf3n-dura-de-josuxe9-con-sus-hermanos}}

\bibleverse{6} José era el gobernador de la tierra. Era él quien vendía
a todo el pueblo de la tierra. Vinieron los hermanos de José y se
inclinaron ante él con el rostro hacia la tierra. \bibleverse{7} José
vio a sus hermanos y los reconoció, pero se comportó con ellos como un
extraño y les habló con rudeza. Les dijo: ``¿De dónde venís?''. Dijeron:
``De la tierra de Canaán, para comprar comida''.

\bibleverse{8} José reconoció a sus hermanos, pero ellos no lo
reconocieron a él. \bibleverse{9} José se acordó de los sueños que había
soñado con ellos y les dijo: ``¡Sois espías! Habéis venido a ver la
desnudez de la tierra''. \footnote{\textbf{42:9} Gén 37,5-9}

\bibleverse{10} Le dijeron: ``No, mi señor, pero tus siervos han venido
a comprar comida. \bibleverse{11} Todos somos hijos de un hombre; somos
hombres honrados. Tus siervos no son espías''.

\bibleverse{12} Les dijo: ``¡No, pero habéis venido a ver la desnudez de
la tierra!''

\bibleverse{13} Ellos dijeron: ``Nosotros, tus siervos, somos doce
hermanos, hijos de un solo hombre en la tierra de Canaán; y he aquí que
el menor está hoy con nuestro padre, y uno ya no está.''

\bibleverse{14} José les dijo: ``Es como les dije, diciendo: `Ustedes
son espías'. \bibleverse{15} Por esto seréis probados. Por la vida del
Faraón, no saldréis de aquí, a menos que venga vuestro hermano menor.
\bibleverse{16} Enviad a uno de vosotros y que traiga a vuestro hermano,
y seréis atados, para que se compruebe si vuestras palabras son
verdaderas, o si por la vida del faraón ciertamente sois espías.''
\bibleverse{17} Los puso a todos juntos en custodia durante tres días.

\hypertarget{la-segunda-conversaciuxf3n-simeuxf3n-como-rehuxe9n}{%
\subsection{La segunda conversación: Simeón como
rehén}\label{la-segunda-conversaciuxf3n-simeuxf3n-como-rehuxe9n}}

\bibleverse{18} Al tercer día, José les dijo: ``Haced esto y vivid,
porque temo a Dios. \bibleverse{19} Si sois hombres honrados, dejad que
uno de vuestros hermanos sea atado en vuestra prisión; pero vosotros id,
llevad grano para el hambre de vuestras casas. \bibleverse{20} Traedme a
vuestro hermano menor; así se verificarán vuestras palabras y no
moriréis.'' Así lo hicieron. \bibleverse{21} Se dijeron unos a otros:
``Ciertamente somos culpables respecto a nuestro hermano, ya que vimos
la angustia de su alma, cuando nos suplicó, y no quisimos escucharlo.
Por eso nos ha sobrevenido esta angustia''. \footnote{\textbf{42:21} Sal
  50,21} \bibleverse{22} Rubén les respondió: ``¿No os dije que no
pecarais contra el niño, y no quisisteis escuchar? Por eso también, he
aquí que se requiere su sangre''. \footnote{\textbf{42:22} Gén 37,21-22}
\bibleverse{23} Ellos no sabían que José los entendía, pues había un
intérprete entre ellos. \bibleverse{24} Se apartó de ellos y lloró.
Luego volvió a ellos y les habló, y tomó a Simeón de entre ellos y lo
ató ante sus ojos.

\hypertarget{regreso-de-los-hermanos-a-canauxe1n}{%
\subsection{Regreso de los hermanos a
Canaán}\label{regreso-de-los-hermanos-a-canauxe1n}}

\bibleverse{25} Entonces José dio la orden de llenar sus sacos de grano,
y de devolver a cada uno su dinero en su saco, y de darles comida para
el camino. Así se hizo con ellos.

\bibleverse{26} Cargaron sus asnos con el grano y se fueron de allí.
\bibleverse{27} Cuando uno de ellos abrió su saco para dar de comer a su
asno en el lugar de alojamiento, vio su dinero. Estaba en la boca de su
saco. \bibleverse{28} Dijo a sus hermanos: ``¡Mi dinero se ha
recuperado! He aquí que está en mi saco''. Les falló el corazón, y se
volvieron temblando unos a otros, diciendo: ``¿Qué es esto que Dios ha
hecho con nosotros?'' \bibleverse{29} Vinieron a Jacob, su padre, a la
tierra de Canaán, y le contaron todo lo que les había sucedido,
diciendo: \bibleverse{30} ``El hombre, el señor de la tierra, habló con
nosotros con aspereza y nos tomó por espías del país. \bibleverse{31}
Nosotros le dijimos: `Somos hombres honestos. No somos espías.
\bibleverse{32} Somos doce hermanos, hijos de nuestro padre; uno ya no
existe, y el más joven está hoy con nuestro padre en la tierra de
Canaán.' \bibleverse{33} El hombre, el señor de la tierra, nos dijo:
`Por esto sabré que sois hombres honrados: dejad conmigo a uno de
vuestros hermanos, tomad grano para el hambre de vuestras casas y seguid
vuestro camino. \bibleverse{34} Traedme a vuestro hermano menor.
Entonces sabré que no sois espías, sino que sois hombres honrados. Así
te entregaré a tu hermano, y comerciarás en la tierra''.

\bibleverse{35} Cuando vaciaron sus sacos, he aquí que el fajo de dinero
de cada uno estaba en su saco. Cuando ellos y su padre vieron sus fajos
de dinero, se asustaron. \bibleverse{36} Jacob, su padre, les dijo:
``¡Me habéis privado de mis hijos! Ya no está José, ya no está Simeón, y
queréis llevaros a Benjamín. Todo esto es contra mí''.

\bibleverse{37} Rubén habló a su padre diciendo: ``Mata a mis dos hijos
si no te lo traigo. Confíalo a mi cuidado, y te lo traeré de nuevo''.

\bibleverse{38} Dijo: ``Mi hijo no bajará con vosotros, pues su hermano
ha muerto y sólo queda él. Si le ocurre algún daño en el camino por el
que vas, entonces harás descender mis canas con dolor al Seol''.
\footnote{\textbf{42:38} El Seol es el lugar de los muertos.}

\hypertarget{segundo-viaje-de-los-hermanos-de-josuxe9-a-egipto-con-benjamuxedn}{%
\subsection{Segundo viaje de los hermanos de José a Egipto con
Benjamín}\label{segundo-viaje-de-los-hermanos-de-josuxe9-a-egipto-con-benjamuxedn}}

\hypertarget{section-42}{%
\section{43}\label{section-42}}

\bibleverse{1} El hambre era severa en la tierra. \bibleverse{2} Cuando
se consumió el grano que habían sacado de Egipto, su padre les dijo:
``Volved a comprarnos un poco más de comida''.

\bibleverse{3} Judá le habló diciendo: ``El hombre nos advirtió
solemnemente, diciendo: `No veréis mi rostro, a menos que vuestro
hermano esté con vosotros'. \footnote{\textbf{43:3} Gén 42,15}
\bibleverse{4} Si envías a nuestro hermano con nosotros, bajaremos a
comprarte comida; \bibleverse{5} pero si no lo envías, no bajaremos,
porque el hombre nos dijo: `No veréis mi rostro, a menos que vuestro
hermano esté con vosotros'\,''.

\bibleverse{6} Israel dijo: ``¿Por qué me trataste tan mal, diciéndole
al hombre que tenías otro hermano?''

\bibleverse{7} Dijeron: ``El hombre preguntó directamente por nosotros y
por nuestros parientes, diciendo: `¿Vive aún tu padre? ¿Tenéis otro
hermano? Nos limitamos a responder a sus preguntas. ¿Acaso podíamos
saber que iba a decir: `Bajad a vuestro hermano'?'' \footnote{\textbf{43:7}
  Gén 42,7-13}

\bibleverse{8} Judá dijo a Israel, su padre: ``Envía al muchacho
conmigo, y nos levantaremos y nos iremos, para que vivamos y no muramos,
tanto nosotros como tú, y también nuestros pequeños. \bibleverse{9} Yo
seré la garantía para él. De mi mano lo requerirás. Si no te lo traigo,
y lo pongo delante de ti, entonces déjame cargar con la culpa para
siempre; \bibleverse{10} porque si no nos hubiéramos demorado,
seguramente ya habríamos regresado por segunda vez.''

\bibleverse{11} Su padre, Israel, les dijo: ``Si ha de ser así, haced
esto: Tomad de los frutos selectos de la tierra en vuestros sacos, y
bajad un regalo para el hombre, un poco de bálsamo, un poco de miel,
especias y mirra, nueces y almendras; \footnote{\textbf{43:11} Prov
  18,16} \bibleverse{12} y tomad el doble de dinero en vuestra mano, y
llevad el dinero que se os devolvió en la boca de vuestros sacos. Tal
vez fue un descuido. \footnote{\textbf{43:12} Gén 42,27; Gén 42,35}
\bibleverse{13} Toma también a tu hermano, levántate y vuelve con él.
\bibleverse{14} Que el Dios Todopoderoso te dé misericordia ante el
hombre, para que te libere a tu otro hermano y a Benjamín. Si estoy
desprovisto de mis hijos, estoy desprovisto''. \footnote{\textbf{43:14}
  Gén 42,36}

\bibleverse{15} Los hombres aceptaron ese regalo, y tomaron el doble de
dinero en su mano, y a Benjamín; se levantaron, bajaron a Egipto y se
presentaron ante José.

\hypertarget{acogida-amistosa-por-parte-de-josuxe9-de-sus-hermanos}{%
\subsection{Acogida amistosa por parte de José de sus
hermanos}\label{acogida-amistosa-por-parte-de-josuxe9-de-sus-hermanos}}

\bibleverse{16} Cuando José vio a Benjamín con ellos, le dijo al
mayordomo de su casa: ``Lleva a los hombres a la casa, descuartiza un
animal y prepárate, porque los hombres cenarán conmigo al mediodía.''

\bibleverse{17} El hombre hizo lo que José le ordenó, y llevó a los
hombres a la casa de José. \bibleverse{18} Los hombres tuvieron miedo de
que los llevaran a la casa de José, y dijeron: ``Por el dinero que se
devolvió en nuestros sacos la primera vez, nos han traído, para que
busque ocasión contra nosotros, nos ataque y nos tome como esclavos,
junto con nuestros asnos.'' \footnote{\textbf{43:18} Gén 42,28}
\bibleverse{19} Se acercaron al mayordomo de la casa de José, y le
hablaron a la puerta de la casa, \bibleverse{20} y le dijeron: ``Oh,
señor mío, la primera vez bajamos a comprar comida. \bibleverse{21}
Cuando llegamos al lugar de alojamiento, abrimos nuestros sacos, y he
aquí que el dinero de cada uno estaba en la boca de su saco, nuestro
dinero en su totalidad. Lo hemos traído en nuestra mano. \bibleverse{22}
Hemos bajado otro dinero en nuestra mano para comprar comida. No sabemos
quién puso nuestro dinero en nuestros sacos''.

\bibleverse{23} Dijo: ``La paz sea con vosotros. No tengáis miedo. Tu
Dios, y el Dios de tu padre, te ha dado un tesoro en tus sacos. He
recibido vuestro dinero''. Hizo salir a Simeón hacia ellos. \footnote{\textbf{43:23}
  Gén 42,24} \bibleverse{24} El hombre llevó a los hombres a la casa de
José, les dio agua y les lavó los pies. Les dio forraje a sus burros.
\footnote{\textbf{43:24} Gén 18,4} \bibleverse{25} Prepararon el regalo
para la llegada de José al mediodía, pues se enteraron de que debían
comer pan allí.

\hypertarget{josuxe9-recibe-y-entretiene-a-sus-hermanos-de-la-manera-muxe1s-amistosa}{%
\subsection{José recibe y entretiene a sus hermanos de la manera más
amistosa}\label{josuxe9-recibe-y-entretiene-a-sus-hermanos-de-la-manera-muxe1s-amistosa}}

\bibleverse{26} Cuando José volvió a casa, le llevaron el regalo que
tenían en la mano a la casa, y se postraron en tierra ante él.
\bibleverse{27} Él les preguntó por su bienestar y les dijo: ``¿Está
bien vuestro padre, el anciano del que habéis hablado? ¿Aún vive?''
\footnote{\textbf{43:27} Gén 42,13}

\bibleverse{28} Ellos dijeron: ``Tu siervo, nuestro padre, está bien.
Todavía está vivo''. Se inclinaron humildemente. \footnote{\textbf{43:28}
  Gén 37,7; Gén 37,9} \bibleverse{29} Él levantó los ojos y vio a
Benjamín, su hermano, hijo de su madre, y dijo: ``¿Es éste tu hermano
menor, del que me hablaste?'' Y él respondió: ``Que Dios se apiade de
ti, hijo mío''. \bibleverse{30} José se apresuró, pues su corazón
anhelaba a su hermano, y buscó un lugar para llorar. Entró en su
habitación y lloró allí. \bibleverse{31} Se lavó la cara y salió. Se
controló y dijo: ``Sirve la comida''.

\bibleverse{32} Le servían a él solo, y a ellos solos, y a los egipcios
que comían con él solos, porque los egipcios no comen con los hebreos,
pues eso es una abominación para los egipcios. \footnote{\textbf{43:32}
  Gén 46,34; Éxod 8,26} \bibleverse{33} Se sentaron delante de él, el
primogénito según su primogenitura y el menor según su juventud, y los
hombres se maravillaron entre sí. \bibleverse{34} Les mandó porciones de
delante, pero la porción de Benjamín fue cinco veces mayor que la de
cualquiera de ellos. Bebieron y se alegraron con él.

\hypertarget{josuxe9-estuxe1-probando-a-sus-hermanos-por-uxfaltima-vez}{%
\subsection{José está probando a sus hermanos por última
vez}\label{josuxe9-estuxe1-probando-a-sus-hermanos-por-uxfaltima-vez}}

\hypertarget{section-43}{%
\section{44}\label{section-43}}

\bibleverse{1} Mandó al administrador de su casa, diciendo: ``Llena los
sacos de los hombres con comida, toda la que puedan llevar, y pon el
dinero de cada uno en la boca de su saco. \bibleverse{2} Pon mi copa, la
copa de plata, en la boca del saco del más joven, con su dinero del
grano''. Él hizo conforme a la palabra que José había dicho.
\bibleverse{3} Tan pronto como amaneció, los hombres fueron despedidos,
ellos y sus asnos. \bibleverse{4} Cuando salieron de la ciudad y aún no
estaban lejos, José dijo a su mayordomo: ``Arriba, sigue a los hombres.
Cuando los alcances, pregúntales: `¿Por qué habéis premiado el mal con
el bien? \bibleverse{5} ¿No es esto de lo que bebe mi señor, y por lo
que en verdad adivina? Habéis hecho el mal al hacerlo'\,''.
\bibleverse{6} Los alcanzó y les dijo estas palabras.

\bibleverse{7} Le dijeron: ``¿Por qué habla mi señor tales palabras?
¡Lejos están tus siervos de hacer tal cosa! \bibleverse{8} He aquí, el
dinero que encontramos en la boca de nuestros sacos, te lo trajimos de
la tierra de Canaán. ¿Cómo, pues, habríamos de robar plata u oro de la
casa de tu señor? \footnote{\textbf{44:8} Gén 43,22} \bibleverse{9} Con
cualquiera de tus siervos que se encuentre, que muera, y nosotros
también seremos esclavos de mi señor''.

\bibleverse{10} Dijo: ``Ahora también sea según tus palabras. Aquel con
quien se encuentre será mi esclavo; y tú serás irreprochable''.

\bibleverse{11} Entonces se apresuraron, y cada uno bajó su saco al
suelo, y cada uno abrió su saco. \bibleverse{12} Buscó, comenzando por
el más viejo y terminando por el más joven. La copa se encontró en el
saco de Benjamín. \bibleverse{13} Luego se rasgaron las vestiduras, y
cada uno cargó su asno, y regresaron a la ciudad. \footnote{\textbf{44:13}
  Gén 37,29}

\hypertarget{los-hermanos-regresan-a-la-ciudad-y-se-humillan-ante-josuxe9}{%
\subsection{Los hermanos regresan a la ciudad y se humillan ante
José}\label{los-hermanos-regresan-a-la-ciudad-y-se-humillan-ante-josuxe9}}

\bibleverse{14} Judá y sus hermanos llegaron a la casa de José, y éste
todavía estaba allí. Se postraron en el suelo ante él. \bibleverse{15}
José les dijo: ``¿Qué obra es ésta que habéis hecho? ¿No sabéis que un
hombre como yo sí puede hacer adivinación?''

\bibleverse{16} Judá dijo: ``¿Qué le diremos a mi señor? ¿Qué vamos a
decir? ¿Cómo nos exculparemos? Dios ha descubierto la iniquidad de tus
siervos. He aquí que somos esclavos de mi señor, tanto nosotros como
aquel en cuya mano se encuentra la copa''. \footnote{\textbf{44:16} Gén
  42,21-22; Lam 1,14}

\bibleverse{17} Él dijo: ``Lejos de mí el hacerlo. El hombre en cuya
mano se encuentre la copa, será mi esclavo; pero en cuanto a ti, sube en
paz a tu padre.''

\bibleverse{18} Entonces Judá se acercó a él y le dijo: ``Oh, señor mío,
por favor, deja que tu siervo hable una palabra en los oídos de mi
señor, y no dejes que tu ira arda contra tu siervo, porque eres como
Faraón. \bibleverse{19} Mi señor preguntó a sus siervos, diciendo:
``¿Tenéis padre o hermano?'' \footnote{\textbf{44:19} Gén 42,7; Gén
  42,13; Gén 43,7} \bibleverse{20} Dijimos a mi señor: ``Tenemos un
padre, un anciano, y un hijo de su edad, un pequeño; y su hermano ha
muerto, y sólo queda él de su madre; y su padre lo quiere.
\bibleverse{21} Dijiste a tus siervos: `Tráiganlo a mí, para que ponga
mis ojos en él'. \bibleverse{22} Dijimos a mi señor: `El muchacho no
puede dejar a su padre, pues si lo dejara, su padre moriría'.
\bibleverse{23} Dijiste a tus siervos: Si tu hermano menor no baja
contigo, no verás más mi rostro'. \footnote{\textbf{44:23} Gén 42,15;
  Gén 43,3-5} \bibleverse{24} Cuando subimos donde tu siervo mi padre,
le contamos las palabras de mi señor. \bibleverse{25} Nuestro padre
dijo: `Vuelve a comprarnos un poco de comida'. \bibleverse{26} Nosotros
dijimos: `No podemos bajar. Si nuestro hermano menor está con nosotros,
entonces bajaremos; porque no podremos ver el rostro del hombre, a menos
que nuestro hermano menor esté con nosotros.' \bibleverse{27} Tu siervo,
mi padre, nos dijo: `Ustedes saben que mi esposa me dio dos hijos.
\bibleverse{28} Uno salió de mí, y dije: ``Seguramente está
despedazado''; y no lo he vuelto a ver. \footnote{\textbf{44:28} Gén
  37,32-33} \bibleverse{29} Si me quitan a éste también, y le sucede
algún daño, harán descender mis canas con dolor al Seol.' \footnote{\textbf{44:29}
  El Seol es el lugar de los muertos.} \footnote{\textbf{44:29} Gén
  42,38} \bibleverse{30} Ahora, pues, cuando vaya a ver a tu siervo mi
padre, y el muchacho no esté con nosotros, ya que su vida está ligada a
la del muchacho, \bibleverse{31} sucederá que cuando vea que el muchacho
ya no está, morirá. Tus siervos harán descender las canas de tu siervo,
nuestro padre, con dolor al Seol. \footnote{\textbf{44:31} El Seol es el
  lugar de los muertos.} \bibleverse{32} Porque tu siervo se hizo
garante del muchacho ante mi padre, diciendo: `Si no te lo traigo,
entonces cargaré con la culpa ante mi padre para siempre'. \footnote{\textbf{44:32}
  Gén 43,9} \bibleverse{33} Ahora, pues, por favor, deja que tu siervo
se quede en lugar del muchacho, el esclavo de mi señor; y deja que el
muchacho suba con sus hermanos. \bibleverse{34} Porque ¿cómo voy a subir
a mi padre si el muchacho no está conmigo? para que no vea el mal que le
sobrevendrá a mi padre''.

\hypertarget{josuxe9-se-revela-a-sus-hermanos}{%
\subsection{José se revela a sus
hermanos}\label{josuxe9-se-revela-a-sus-hermanos}}

\hypertarget{section-44}{%
\section{45}\label{section-44}}

\bibleverse{1} Entonces José no pudo controlarse ante todos los que
estaban frente a él, y gritó: ``¡Que todos salgan de mí!''. Nadie más
estaba con él, mientras José se daba a conocer a sus hermanos.
\bibleverse{2} Lloró en voz alta. Los egipcios lo oyeron, y la casa del
faraón también. \bibleverse{3} José dijo a sus hermanos: ``¡Yo soy José!
¿Vive aún mi padre?'' Sus hermanos no pudieron responderle, pues estaban
aterrados ante su presencia. \bibleverse{4} José dijo a sus hermanos:
``Acérquense a mí, por favor''. Se acercaron. Él les dijo: ``Yo soy
José, vuestro hermano, a quien vendisteis a Egipto. \footnote{\textbf{45:4}
  Gén 37,28} \bibleverse{5} No os entristezcáis ni os enfadéis por
haberme vendido aquí, pues Dios me ha enviado delante de vosotros para
preservar la vida. \footnote{\textbf{45:5} Gén 50,20}

\hypertarget{josuxe9-fue-enviado-por-dios-para-la-salvaciuxf3n-de-israel}{%
\subsection{José fue enviado por Dios para la salvación de
Israel}\label{josuxe9-fue-enviado-por-dios-para-la-salvaciuxf3n-de-israel}}

\bibleverse{6} Durante estos dos años el hambre ha estado en la tierra,
y aún quedan cinco años, en los que no habrá arado ni cosecha.
\bibleverse{7} Dios me envió delante de ti para preservar para ti un
remanente en la tierra, y para salvarte con vida mediante una gran
liberación. \bibleverse{8} Así que ahora no fuiste tú quien me envió
aquí, sino Dios, y él me ha hecho padre del Faraón, señor de toda su
casa y gobernante de toda la tierra de Egipto. \footnote{\textbf{45:8}
  Gén 41,40-43} \bibleverse{9} Date prisa, sube a ver a mi padre y dile:
``Esto es lo que dice tu hijo José: ``Dios me ha hecho señor de todo
Egipto. Baja a verme. No esperes. \bibleverse{10} Habitarás en la tierra
de Gosén, y estarás cerca de mí, tú, tus hijos, los hijos de tus hijos,
tus rebaños, tus manadas y todo lo que tengas. \bibleverse{11} Allí te
proveeré, porque todavía hay cinco años de hambre; para que no llegues a
la pobreza, tú y tu familia, y todo lo que tienes''\,'. \bibleverse{12}
He aquí que tus ojos ven, y los ojos de mi hermano Benjamín, que es mi
boca la que te habla. \bibleverse{13} Contarás a mi padre toda mi gloria
en Egipto y todo lo que has visto. Te apresurarás a traer a mi padre
aquí''. \bibleverse{14} Se echó al cuello de su hermano Benjamín y
lloró, y Benjamín lloró sobre su cuello. \bibleverse{15} Besó a todos
sus hermanos y lloró sobre ellos. Después sus hermanos hablaron con él.

\hypertarget{la-amable-invitaciuxf3n-del-farauxf3n-a-jacob-para-que-se-mudara-a-egipto}{%
\subsection{La amable invitación del faraón a Jacob para que se mudara a
Egipto}\label{la-amable-invitaciuxf3n-del-farauxf3n-a-jacob-para-que-se-mudara-a-egipto}}

\bibleverse{16} La noticia se oyó en la casa del faraón, diciendo: ``Han
venido los hermanos de José''. Esto agradó al Faraón y a sus siervos.
\bibleverse{17} El faraón le dijo a José: ``Dile a tus hermanos que
hagan esto: Carguen sus animales y vayan, viajen a la tierra de Canaán.
\bibleverse{18} Tomad a vuestro padre y a vuestras familias, y venid a
mí, y os daré el bien de la tierra de Egipto, y comeréis la grasa de la
tierra.' \bibleverse{19} Ahora se os ordena hacer esto: Tomad carros de
la tierra de Egipto para vuestros pequeños y para vuestras mujeres, y
traed a vuestro padre, y venid. \bibleverse{20} Además, no os preocupéis
por vuestras pertenencias, porque el bien de toda la tierra de Egipto es
vuestro.''

\hypertarget{josuxe9-da-obsequios-generosos-a-sus-hermanos-que-regresan-a-casa-y-los-amonesta-con-amor}{%
\subsection{José da obsequios generosos a sus hermanos que regresan a
casa y los amonesta con
amor}\label{josuxe9-da-obsequios-generosos-a-sus-hermanos-que-regresan-a-casa-y-los-amonesta-con-amor}}

\bibleverse{21} Así lo hicieron los hijos de Israel. José les dio
carros, según el mandato del faraón, y les dio provisiones para el
camino. \bibleverse{22} A cada uno de ellos le dio mudas de ropa, pero a
Benjamín le dio trescientas piezas de plata y cinco mudas de ropa.
\bibleverse{23} Envió a su padre lo siguiente: diez burros cargados con
los bienes de Egipto, y diez burras cargadas de grano y pan y
provisiones para su padre en el camino. \bibleverse{24} Entonces
despidió a sus hermanos y se fueron. Les dijo: ``Mirad que no os peleéis
por el camino''. \footnote{\textbf{45:24} Gén 42,22}

\hypertarget{jacob-se-muda-a-su-hijo-en-egipto}{%
\subsection{Jacob se muda a su hijo en
Egipto}\label{jacob-se-muda-a-su-hijo-en-egipto}}

\bibleverse{25} Subieron de Egipto y llegaron a la tierra de Canaán,
donde su padre Jacob. \bibleverse{26} Le contaron, diciendo: ``José aún
vive, y es soberano de toda la tierra de Egipto''. Su corazón se
desmayó, pues no les creyó. \bibleverse{27} Le contaron todas las
palabras de José que él les había dicho. Cuando vio los carros que José
había enviado para llevarlo, el espíritu de Jacob, su padre, revivió.
\bibleverse{28} Israel dijo: ``Es suficiente. José, mi hijo, sigue vivo.
Iré a verlo antes de morir''. \footnote{\textbf{45:28} Gén 46,30}

\hypertarget{dios-aprueba-el-traslado-de-jacob-a-beerseba-en-una-revelaciuxf3n}{%
\subsection{Dios aprueba el traslado de Jacob a Beerseba en una
revelación}\label{dios-aprueba-el-traslado-de-jacob-a-beerseba-en-una-revelaciuxf3n}}

\hypertarget{section-45}{%
\section{46}\label{section-45}}

\bibleverse{1} Israel viajó con todo lo que tenía, llegó a Beerseba y
ofreció sacrificios al Dios de su padre, Isaac. \footnote{\textbf{46:1}
  Gén 26,23-25} \bibleverse{2} Dios habló a Israel en las visiones de la
noche, y dijo: ``¡Jacob, Jacob!'' Dijo: ``Aquí estoy''.

\bibleverse{3} Dijo: ``Yo soy Dios, el Dios de tu padre. No tengas miedo
de bajar a Egipto, porque allí haré de ti una gran nación.
\bibleverse{4} Yo bajaré contigo a Egipto. También te haré subir con
toda seguridad. La mano de José cerrará tus ojos''.

\bibleverse{5} Jacob se levantó de Beerseba, y los hijos de Israel
llevaron a Jacob, a su padre, a sus hijos y a sus mujeres, en los carros
que el faraón había enviado para transportarlo. \bibleverse{6} Tomaron
su ganado y sus bienes, que habían adquirido en la tierra de Canaán, y
entraron en Egipto: Jacob, y toda su descendencia con él, \bibleverse{7}
sus hijos, y los hijos de sus hijos con él, sus hijas y las hijas de sus
hijos, y llevó toda su descendencia con él a Egipto.

\hypertarget{el-linaje-de-toda-la-familia-de-jacob}{%
\subsection{El linaje de toda la familia de
Jacob}\label{el-linaje-de-toda-la-familia-de-jacob}}

\bibleverse{8} Estos son los nombres de los hijos de Israel que entraron
en Egipto, Jacob y sus hijos: Rubén, primogénito de Jacob. \footnote{\textbf{46:8}
  Éxod 6,14-16} \bibleverse{9} Los hijos de Rubén: Hanoc, Palú, Hezrón y
Carmi. \bibleverse{10} Los hijos de Simeón: Jemuel, Jamín, Ohad, Jacín,
Zohar y Shaúl, hijo de una cananea. \bibleverse{11} Los hijos de Leví:
Gersón, Coat y Merari. \bibleverse{12} Los hijos de Judá: Er, Onán,
Sela, Pérez y Zéraj; pero Er y Onán murieron en la tierra de Canaán. Los
hijos de Pérez fueron Hezrón y Hamul. \footnote{\textbf{46:12} Gén
  38,3-4; Gén 38,29-30} \bibleverse{13} Los hijos de Isacar: Tola,
Puvah, Iob y Shimron. \bibleverse{14} Los hijos de Zabulón: Sered, Elón
y Jahleel. \bibleverse{15} Estos son los hijos de Lea, que dio a luz a
Jacob en Padan Aram, con su hija Dina. Todas las almas de sus hijos e
hijas fueron treinta y tres. \bibleverse{16} Los hijos de Gad: Zifón,
Haggi, Shuni, Ezbón, Eri, Arodi y Areli. \bibleverse{17} Los hijos de
Aser: Imna, Ishva, Ishvi, Beriá y su hermana Sera. Los hijos de Beriá:
Heber y Malquiel. \bibleverse{18} Estos son los hijos de Zilpá, que
Labán dio a su hija Lea, y que ella dio a luz a Jacob, dieciséis almas.
\bibleverse{19} Los hijos de Raquel, mujer de Jacob José y Benjamín.
\bibleverse{20} A José le nacieron en la tierra de Egipto Manasés y
Efraín, que le dio a luz Asenat, hija de Potifera, sacerdote de On.
\footnote{\textbf{46:20} Gén 41,50-52} \bibleverse{21} Los hijos de
Benjamín: Bela, Becher, Ashbel, Gera, Naamán, Ehi, Rosh, Muppim, Huppim
y Ard. \bibleverse{22} Estos son los hijos de Raquel que le nacieron a
Jacob: todos fueron catorce. \bibleverse{23} El hijo de Dan: Hushim.
\bibleverse{24} Los hijos de Neftalí: Jahzeel, Guni, Jezer y Silim.
\bibleverse{25} Estos son los hijos de Bilhá, que Labán dio a su hija
Raquel, y éstos fueron los que ella dio a luz a Jacob: todas las almas
fueron siete. \bibleverse{26} Todas las almas que vinieron con Jacob a
Egipto, que fueron su descendencia directa, además de las mujeres de los
hijos de Jacob, todas las almas fueron sesenta y seis. \bibleverse{27}
Los hijos de José, que le nacieron en Egipto, fueron dos almas. Todas
las almas de la casa de Jacob, que llegaron a Egipto, fueron setenta.
\footnote{\textbf{46:27} Éxod 1,5}

\hypertarget{josuxe9-saluda-a-su-padre-en-gosen}{%
\subsection{José saluda a su padre en
Gosen}\label{josuxe9-saluda-a-su-padre-en-gosen}}

\bibleverse{28} Jacob envió a Judá delante de José para que le mostrara
el camino a Gosén, y llegaron a la tierra de Gosén. \footnote{\textbf{46:28}
  Gén 45,10} \bibleverse{29} José preparó su carro y subió a recibir a
Israel, su padre, en Gosén. Se presentó ante él, y se echó sobre su
cuello, y lloró sobre su cuello un buen rato. \bibleverse{30} Israel
dijo a José: ``Ahora déjame morir, ya que he visto tu rostro, que aún
estás vivo''. \footnote{\textbf{46:30} Gén 45,28}

\bibleverse{31} José dijo a sus hermanos y a la casa de su padre:
``Subiré y hablaré con el Faraón y le diré: `Mis hermanos y la casa de
mi padre, que estaban en la tierra de Canaán, han venido a mí.
\bibleverse{32} Estos hombres son pastores, pues han sido cuidadores de
ganado, y han traído sus rebaños, sus manadas y todo lo que tienen.'
\bibleverse{33} Sucederá que cuando el Faraón os llame y os diga: ``¿A
qué os dedicáis? \bibleverse{34} que diréis: `Tus siervos han sido
cuidadores de ganado desde nuestra juventud hasta ahora, tanto nosotros
como nuestros padres', para que podáis habitar en la tierra de Gosén;
porque todo pastor es una abominación para los egipcios.'' \footnote{\textbf{46:34}
  Gén 43,32}

\hypertarget{el-farauxf3n-promete-a-los-hijos-de-jacob-establecerse-en-gosen}{%
\subsection{El faraón promete a los hijos de Jacob establecerse en
Gosen}\label{el-farauxf3n-promete-a-los-hijos-de-jacob-establecerse-en-gosen}}

\hypertarget{section-46}{%
\section{47}\label{section-46}}

\bibleverse{1} Entonces José entró y dio cuenta al Faraón, diciendo:
``Mi padre y mis hermanos, con sus rebaños, sus manadas y todo lo que
poseen, han salido de la tierra de Canaán; y he aquí que están en la
tierra de Gosén.'' \bibleverse{2} De entre sus hermanos tomó cinco
hombres y los presentó al Faraón. \bibleverse{3} El faraón dijo a sus
hermanos: ``¿A qué os dedicáis?'' Dijeron al Faraón: ``Tus siervos son
pastores, tanto nosotros como nuestros padres''. \footnote{\textbf{47:3}
  Gén 46,33-34} \bibleverse{4} También dijeron al Faraón: ``Hemos venido
a vivir como extranjeros en la tierra, porque no hay pastos para los
rebaños de tus siervos. Porque el hambre es grave en la tierra de
Canaán. Ahora, pues, por favor, deja que tus siervos habiten en la
tierra de Gosén''.

\bibleverse{5} El faraón habló a José, diciendo: ``Tu padre y tus
hermanos han venido a ti. \bibleverse{6} La tierra de Egipto está ante
ti. Haz que tu padre y tus hermanos habiten en lo mejor de la tierra.
Que habiten en la tierra de Gosén. Si conoces a algún hombre capaz entre
ellos, ponlo a cargo de mi ganado''.

\hypertarget{jacob-se-presentuxf3-al-farauxf3n-y-luego-se-instaluxf3-en-gosen}{%
\subsection{Jacob se presentó al faraón y luego se instaló en
Gosen}\label{jacob-se-presentuxf3-al-farauxf3n-y-luego-se-instaluxf3-en-gosen}}

\bibleverse{7} José hizo entrar a Jacob, su padre, y lo presentó ante el
Faraón; y Jacob bendijo al Faraón. \bibleverse{8} El faraón dijo a
Jacob: ``¿Cuántos años tienes?''

\bibleverse{9} Jacob dijo al Faraón: ``Los años de mi peregrinación son
ciento treinta años. Los días de los años de mi vida han sido pocos y
malos. No han llegado a los días de los años de la vida de mis padres en
los días de su peregrinación''. \footnote{\textbf{47:9} Sal 90,10; Sal
  39,12} \bibleverse{10} Jacob bendijo al faraón y salió de la presencia
del faraón.

\bibleverse{11} José colocó a su padre y a sus hermanos, y les dio una
posesión en la tierra de Egipto, en lo mejor de la tierra, en la tierra
de Ramsés, como lo había ordenado el Faraón. \bibleverse{12} José
proveyó de pan a su padre, a sus hermanos y a toda la familia de su
padre, según el tamaño de sus familias. \footnote{\textbf{47:12} Gén
  45,11}

\hypertarget{josuxe9-compra-la-tierra-para-el-farauxf3n}{%
\subsection{José compra la tierra para el
faraón}\label{josuxe9-compra-la-tierra-para-el-farauxf3n}}

\bibleverse{13} No había pan en toda la tierra, pues el hambre era muy
severa, de modo que la tierra de Egipto y la tierra de Canaán
desfallecían a causa del hambre. \bibleverse{14} José reunió todo el
dinero que se encontró en la tierra de Egipto y en la tierra de Canaán,
por el grano que compraron; y José llevó el dinero a la casa del Faraón.
\bibleverse{15} Cuando se gastó todo el dinero en la tierra de Egipto y
en la tierra de Canaán, todos los egipcios se acercaron a José y le
dijeron: ``Danos pan, pues ¿para qué vamos a morir en tu presencia?
Porque nuestro dinero se agota''.

\bibleverse{16} José dijo: ``Dame tu ganado, y yo te daré comida para tu
ganado, si se acaba tu dinero''.

\bibleverse{17} Trajeron sus ganados a José, y éste les dio pan a cambio
de los caballos, de los rebaños y de los asnos, y los alimentó con pan a
cambio de todos sus ganados de aquel año. \bibleverse{18} Terminado
aquel año, vinieron a él el segundo año y le dijeron: ``No vamos a
ocultar a mi señor que todo nuestro dinero se ha gastado, y que los
rebaños son de mi señor. No queda nada a la vista de mi señor, sino
nuestros cuerpos y nuestras tierras. \bibleverse{19} ¿Por qué hemos de
morir ante sus ojos, nosotros y nuestras tierras? Cómpranos a nosotros y
a nuestra tierra a cambio de pan, y nosotros y nuestra tierra seremos
siervos del faraón. Danos semilla, para que vivamos y no muramos, y para
que la tierra no quede desolada''.

\bibleverse{20} Así que José compró toda la tierra de Egipto para el
Faraón, pues cada hombre de los egipcios vendió su campo, porque la
hambruna se cebó con ellos, y la tierra pasó a ser del Faraón.
\bibleverse{21} En cuanto al pueblo, lo trasladó a las ciudades desde un
extremo de la frontera de Egipto hasta el otro. \bibleverse{22} Sólo que
no compró la tierra de los sacerdotes, porque los sacerdotes tenían una
porción del Faraón y comían su porción que el Faraón les daba. Por eso
no vendieron sus tierras. \bibleverse{23} Entonces José dijo al pueblo:
``Miren, hoy les he comprado a ustedes y a sus tierras para el Faraón.
He aquí que hay semilla para ustedes, y ustedes sembrarán la tierra.
\bibleverse{24} Sucederá que en las cosechas daréis una quinta parte al
Faraón, y cuatro partes serán vuestras, para semilla del campo, para
vuestro alimento, para los de vuestras casas y para el alimento de
vuestros hijos.''

\bibleverse{25} Dijeron: ``¡Nos has salvado la vida! Hallemos el favor a
los ojos de mi señor, y seremos siervos del Faraón''.

\bibleverse{26} José hizo un estatuto sobre la tierra de Egipto hasta el
día de hoy, para que el Faraón tuviera el quinto. Sólo la tierra de los
sacerdotes no pasó a ser del Faraón.

\hypertarget{feliz-situaciuxf3n-para-los-israelitas-en-egipto-el-uxfaltimo-deseo-de-jacob-con-respecto-a-su-funeral}{%
\subsection{Feliz situación para los israelitas en Egipto; El último
deseo de Jacob con respecto a su
funeral}\label{feliz-situaciuxf3n-para-los-israelitas-en-egipto-el-uxfaltimo-deseo-de-jacob-con-respecto-a-su-funeral}}

\bibleverse{27} Israel vivió en la tierra de Egipto, en la tierra de
Gosén, y se apropió de ella, y fructificó y se multiplicó en gran
manera. \footnote{\textbf{47:27} Gén 46,3; Éxod 1,7; Éxod 1,12}
\bibleverse{28} Jacob vivió en la tierra de Egipto diecisiete años. Así
que los días de Jacob, los años de su vida, fueron ciento cuarenta y
siete años. \bibleverse{29} Se acercó el momento en que Israel debía
morir, y llamó a su hijo José, y le dijo: ``Si ahora he hallado gracia
ante tus ojos, por favor, pon tu mano debajo de mi muslo y trátame con
bondad y sinceridad. Por favor, no me entierres en Egipto, \footnote{\textbf{47:29}
  Gén 24,2} \bibleverse{30} sino que cuando duerma con mis padres, me
sacarás de Egipto y me enterrarás en su sepultura.'' José dijo: ``Haré
lo que has dicho''. \footnote{\textbf{47:30} Gén 25,9-10; Gén 49,29-32}

\bibleverse{31} Israel dijo: ``Júrame'', y él le juró. Entonces Israel
se inclinó sobre la cabecera de la cama.

\hypertarget{jacob-toma-a-los-dos-hijos-de-josuxe9-en-lugar-de-niuxf1os}{%
\subsection{Jacob toma a los dos hijos de José en lugar de
niños}\label{jacob-toma-a-los-dos-hijos-de-josuxe9-en-lugar-de-niuxf1os}}

\hypertarget{section-47}{%
\section{48}\label{section-47}}

\bibleverse{1} Después de estas cosas, alguien dijo a José: ``He aquí
que tu padre está enfermo''. Tomó consigo a sus dos hijos, Manasés y
Efraín. \bibleverse{2} Alguien avisó a Jacob y le dijo: ``He aquí que tu
hijo José viene a ti'', e Israel se fortaleció y se sentó en la cama.
\bibleverse{3} Jacob dijo a José: ``El Dios Todopoderoso se me apareció
en Luz, en la tierra de Canaán, y me bendijo, \footnote{\textbf{48:3}
  Gén 28,19} \bibleverse{4} y me dijo: `He aquí que te haré fructificar
y te multiplicaré, y haré de ti una compañía de pueblos, y daré esta
tierra a tu descendencia después de ti como posesión eterna'.
\footnote{\textbf{48:4} Gén 35,11-12} \bibleverse{5} Tus dos hijos, que
te nacieron en la tierra de Egipto antes de que yo viniera a ti a
Egipto, son míos; Efraín y Manasés, como Rubén y Simeón, serán míos.
\footnote{\textbf{48:5} Gén 41,50-52} \bibleverse{6} Tu descendencia, de
la que seas padre después de ellos, será tuya. Se llamarán con el nombre
de sus hermanos en su herencia. \bibleverse{7} En cuanto a mí, cuando
vine de Paddán, Raquel murió a mi lado en la tierra de Canaán, en el
camino, cuando aún faltaba para llegar a Efrat, y la enterré allí en el
camino a Efrat (también llamada Belén).'' \footnote{\textbf{48:7} Gén
  35,19}

\hypertarget{jacob-bendice-a-los-dos-hijos-de-josuxe9}{%
\subsection{Jacob bendice a los dos hijos de
José}\label{jacob-bendice-a-los-dos-hijos-de-josuxe9}}

\bibleverse{8} Israel vio a los hijos de José y dijo: ``¿Quiénes son
estos?''

\bibleverse{9} José dijo a su padre: ``Son mis hijos, que Dios me ha
dado aquí''. Dijo: ``Por favor, tráemelos, y los bendeciré''.
\footnote{\textbf{48:9} Gén 33,5} \bibleverse{10} Ahora bien, los ojos
de Israel estaban apagados por la edad, de modo que no podía ver bien.
José los acercó, los besó y los abrazó. \bibleverse{11} Israel dijo a
José: ``No creía que fuera a ver tu rostro, y he aquí que Dios me ha
permitido ver también tu descendencia.'' \footnote{\textbf{48:11} Gén
  37,33; Gén 37,35; Gén 45,26; Sal 128,6} \bibleverse{12} José los sacó
de entre sus rodillas y se inclinó con el rostro hacia la tierra.
\bibleverse{13} José tomó a ambos, a Efraín con su mano derecha hacia la
mano izquierda de Israel, y a Manasés con su mano izquierda hacia la
mano derecha de Israel, y los acercó a él. \bibleverse{14} Israel
extendió su mano derecha y la puso sobre la cabeza de Efraín, que era el
menor, y su mano izquierda sobre la cabeza de Manasés, guiando sus manos
a sabiendas, pues Manasés era el primogénito. \bibleverse{15} Bendijo a
José y dijo, ``El Dios ante el que caminaron mis padres Abraham e Isaac,
el Dios que me ha alimentado durante toda mi vida hasta el día de hoy,
\footnote{\textbf{48:15} Gén 32,9; Sal 23,1} \bibleverse{16} el ángel
que me ha redimido de todo mal, bendice a los muchachos, y que mi nombre
sea nombrado en ellos, y el nombre de mis padres Abraham e Isaac. Que
crezcan en multitud sobre la tierra''. \footnote{\textbf{48:16} Gén
  31,11-13}

\bibleverse{17} Cuando José vio que su padre ponía su mano derecha sobre
la cabeza de Efraín, le disgustó. Levantó la mano de su padre para
quitarla de la cabeza de Efraín a la de Manasés. \bibleverse{18} José
dijo a su padre: ``No es así, padre mío, porque éste es el primogénito.
Pon tu mano derecha sobre su cabeza''.

\bibleverse{19} Su padre se negó y dijo: ``Lo sé, hijo mío, lo sé. Él
también llegará a ser un pueblo, y también será grande. Sin embargo, su
hermano menor será más grande que él, y su descendencia llegará a ser
una multitud de naciones''. \footnote{\textbf{48:19} Núm 1,33; Núm 1,35;
  Deut 33,17} \bibleverse{20} Aquel día los bendijo diciendo: ``Israel
bendecirá en vosotros, diciendo: ``Dios os haga como Efraín y como
Manasés''\,'' Puso a Efraín por delante de Manasés. \footnote{\textbf{48:20}
  Heb 11,21} \bibleverse{21} Israel dijo a José: ``He aquí que yo muero,
pero Dios estará contigo y te hará volver a la tierra de tus padres.
\bibleverse{22} Además, te he dado una porción por encima de tus
hermanos, que tomé de la mano del amorreo con mi espada y con mi arco.''
\footnote{\textbf{48:22} Juan 4,5}

\hypertarget{las-profecuxedas-de-jacob-sobre-sus-hijos}{%
\subsection{Las profecías de Jacob sobre sus
hijos}\label{las-profecuxedas-de-jacob-sobre-sus-hijos}}

\hypertarget{section-48}{%
\section{49}\label{section-48}}

\bibleverse{1} Jacob llamó a sus hijos y les dijo ``Reúnanse, para que
les diga lo que les sucederá en los días venideros. \bibleverse{2}
Reúnanse y escuchen, hijos de Jacob. Escucha a Israel, tu padre.
\bibleverse{3} ``Rubén, tú eres mi primogénito, mi fuerza y el principio
de mi fortaleza, sobresaliendo en dignidad, y sobresaliendo en poder.
\footnote{\textbf{49:3} Gén 29,32; Deut 21,17} \bibleverse{4} Hirviendo
como el agua, no sobresaldrás, porque subiste a la cama de tu padre, y
luego lo profanó. Subió a mi sofá. \footnote{\textbf{49:4} Gén 35,22}
\bibleverse{5} ``Simeón y Leví son hermanos. Sus espadas son armas de
violencia. \bibleverse{6} Alma mía, no entres en su consejo. Gloria mía,
no te unas a su asamblea; porque en su ira mataron a los hombres. En su
voluntad propia, han maniatado al ganado. \footnote{\textbf{49:6} Sal
  16,9; Sal 30,12; Gén 34,25} \bibleverse{7} Maldita sea su cólera,
porque era feroz; y su ira, pues era cruel. Los dividiré en Jacob, y
dispersarlos en Israel. \footnote{\textbf{49:7} Jos 19,1-9; Jos 21,1-42}
\bibleverse{8} ``Judá, tus hermanos te alabarán. Tu mano estará en el
cuello de tus enemigos. Los hijos de tu padre se inclinarán ante ti.
\footnote{\textbf{49:8} Núm 10,14; Jue 1,1-2} \bibleverse{9} Judá es un
cachorro de león. De la presa, hijo mío, has subido. Se agachó, se
agazapó como un león, como una leona. ¿Quién lo despertará? \footnote{\textbf{49:9}
  Núm 23,24; Apoc 5,5} \bibleverse{10} El cetro no se apartará de Judá,
ni el bastón de mando de entre sus pies, hasta que llegue a quien le
corresponde. La obediencia de los pueblos será a él. \footnote{\textbf{49:10}
  Núm 24,17; 1Cró 5,2; Heb 7,14} \bibleverse{11} Atando su potro a la
vid, el potro de su asno a la cepa elegida, ha lavado su ropa en vino,
sus ropas en la sangre de las uvas. \footnote{\textbf{49:11} Jl 3,18}
\bibleverse{12} Sus ojos estarán rojos de vino, sus dientes blancos de
leche. \bibleverse{13} ``Zabulón habitará en el puerto del mar. Será
para un puerto de barcos. Su frontera estará en Sidón. \footnote{\textbf{49:13}
  Jos 19,10-16} \bibleverse{14} ``Isacar es un asno fuerte, tumbado
entre las alforjas. \bibleverse{15} Vio un lugar de descanso, que era
bueno, la tierra, que era agradable. Inclina su hombro ante la carga, y
se convierte en un siervo haciendo trabajos forzados. \bibleverse{16}
``Dan juzgará a su pueblo, como una de las tribus de Israel. \footnote{\textbf{49:16}
  Jue 13,25} \bibleverse{17} Dan será una serpiente en el camino, un
sumador en el camino, que muerde los talones del caballo, para que su
jinete caiga hacia atrás. \bibleverse{18} He esperado tu salvación,
Yahvé. \footnote{\textbf{49:18} Sal 119,166; Hab 2,3} \bibleverse{19}
``Una tropa presionará a Gad, pero les presionará el talón.
\bibleverse{20} ``La comida de Asher será rica. Producirá manjares
reales. \footnote{\textbf{49:20} Jos 19,24-31} \bibleverse{21} ``Neftalí
es una cierva liberada, que lleva hermosos cervatillos. \footnote{\textbf{49:21}
  Jue 4,6-10} \bibleverse{22} ``José es una vid fructífera, una vid
fructífera junto a un manantial. Sus ramas pasan por encima de la pared.
\footnote{\textbf{49:22} Os 13,15} \bibleverse{23} Los arqueros lo han
afligido gravemente, le dispararon y le persiguieron: \bibleverse{24}
Pero su arco siguió siendo fuerte. Los brazos de sus manos se hicieron
fuertes, por las manos del Poderoso de Jacob,(de allí es el pastor, la
piedra de Israel), \bibleverse{25} por el Dios de tu padre, que te
ayudará, por el Todopoderoso, que te bendecirá, con las bendiciones del
cielo, las bendiciones de las profundidades que se encuentran debajo,
bendiciones de los pechos, y del vientre. \bibleverse{26} Las
bendiciones de tu padre han prevalecido sobre las de mis antepasados,
por encima de los límites de las antiguas colinas. Estarán en la cabeza
de José, en la coronilla del que se separa de sus hermanos. \footnote{\textbf{49:26}
  Gén 45,8} \bibleverse{27} ``Benjamín es un lobo voraz. Por la mañana
devorará la presa. Al anochecer repartirá el botín''. \footnote{\textbf{49:27}
  Jue 20,25; 1Sam 9,1-2}

\hypertarget{la-solicitud-de-jacob-para-su-entierro-en-hebruxf3n}{%
\subsection{La solicitud de Jacob para su entierro en
Hebrón}\label{la-solicitud-de-jacob-para-su-entierro-en-hebruxf3n}}

\bibleverse{28} Todas estas son las doce tribus de Israel, y esto es lo
que su padre les habló y los bendijo. Bendijo a cada uno según su propia
bendición. \bibleverse{29} Los instruyó y les dijo: ``Voy a ser reunido
con mi pueblo. Entiérrenme con mis padres en la cueva que está en el
campo de Efrón el hitita, \footnote{\textbf{49:29} Gén 23,16-20; Gén
  47,30} \bibleverse{30} en la cueva que está en el campo de Macpela,
que está delante de Mamre, en la tierra de Canaán, que Abraham compró
con el campo de Efrón el hitita como lugar de sepultura. \bibleverse{31}
Allí enterraron a Abraham y a Sara, su esposa. Allí enterraron a Isaac y
a Rebeca, su mujer, y allí enterré a Lea: \footnote{\textbf{49:31} Gén
  25,9; Gén 35,29} \bibleverse{32} el campo y la cueva que hay en él,
que fue comprada a los hijos de Het.'' \bibleverse{33} Cuando Jacob
terminó de encargar a sus hijos, recogió sus pies en el lecho, exhaló su
último aliento y se reunió con su pueblo.

\hypertarget{embalsamamiento-y-traslado-solemne-de-jacob-despuuxe9s-del-entierro-hereditario-en-hebruxf3n}{%
\subsection{Embalsamamiento y traslado solemne de Jacob después del
entierro hereditario en
Hebrón}\label{embalsamamiento-y-traslado-solemne-de-jacob-despuuxe9s-del-entierro-hereditario-en-hebruxf3n}}

\hypertarget{section-49}{%
\section{50}\label{section-49}}

\bibleverse{1} José se postró sobre el rostro de su padre, lloró sobre
él y lo besó. \footnote{\textbf{50:1} Gén 46,4} \bibleverse{2} José
ordenó a sus servidores, los médicos, que embalsamaran a su padre; y los
médicos embalsamaron a Israel. \bibleverse{3} Le dedicaron cuarenta
días, pues son los que se necesitan para embalsamar. Los egipcios
lloraron a Israel durante setenta días.

\bibleverse{4} Cuando pasaron los días de llanto por él, José habló al
bastón del faraón, diciendo: ``Si ahora he encontrado gracia ante tus
ojos, por favor, habla en los oídos del faraón, diciendo: \bibleverse{5}
`Mi padre me hizo jurar, diciendo: ``He aquí que me estoy muriendo.
Entiérrame en mi tumba que me he cavado en la tierra de Canaán''. Ahora,
pues, te ruego que me dejes subir a enterrar a mi padre, y volveré''.
\footnote{\textbf{50:5} Gén 47,29-30}

\bibleverse{6} El faraón dijo: ``Sube y entierra a tu padre, como te
hizo jurar''.

\bibleverse{7} José subió a enterrar a su padre, y con él subieron todos
los servidores del faraón, los ancianos de su casa, todos los ancianos
del país de Egipto, \bibleverse{8} toda la casa de José, sus hermanos y
la casa de su padre. Sólo dejaron en la tierra de Gosén a sus pequeños,
sus rebaños y sus manadas. \bibleverse{9} Tanto los carros como los
jinetes subieron con él. Era una compañía muy grande. \bibleverse{10}
Llegaron a la era de Atad, que está al otro lado del Jordán, y allí se
lamentaron con un lamento muy grande y severo. Hicieron duelo por su
padre durante siete días. \bibleverse{11} Cuando los habitantes de la
tierra, los cananeos, vieron el luto en la era de Atad, dijeron: ``Este
es un luto grave de los egipcios''. Por eso su nombre fue llamado Abel
Mizraim, que está al otro lado del Jordán. \bibleverse{12} Sus hijos le
hicieron tal como él les había ordenado, \footnote{\textbf{50:12} Gén
  49,29} \bibleverse{13} pues sus hijos lo llevaron a la tierra de
Canaán y lo enterraron en la cueva del campo de Macpela, que Abraham
compró con el campo, como posesión para un lugar de entierro, a Efrón el
hitita, cerca de Mamre. \footnote{\textbf{50:13} Gén 23,16}
\bibleverse{14} José regresó a Egipto, junto con sus hermanos y todos
los que subieron con él para enterrar a su padre, después de haber
enterrado a su padre.

\hypertarget{la-generosidad-de-josuxe9-hacia-sus-hermanos}{%
\subsection{La generosidad de José hacia sus
hermanos}\label{la-generosidad-de-josuxe9-hacia-sus-hermanos}}

\bibleverse{15} Cuando los hermanos de José vieron que su padre había
muerto, dijeron: ``Puede ser que José nos odie y nos pague plenamente
todo el mal que le hicimos.'' \bibleverse{16} Enviaron un mensaje a
José, diciendo: ``Tu padre ordenó antes de morir, diciendo:
\bibleverse{17} ``Ahora dirás a José: ``Por favor, perdona la
desobediencia de tus hermanos y su pecado, porque te hicieron mal''.
Ahora, por favor, perdona la desobediencia de los siervos del Dios de tu
padre''. José lloró cuando le hablaron. \bibleverse{18} Sus hermanos
también fueron y se postraron ante su rostro, y dijeron: ``He aquí que
somos tus siervos''. \bibleverse{19} José les dijo: ``No tengáis miedo,
porque ¿estoy en el lugar de Dios? \bibleverse{20} En cuanto a ustedes,
quisieron hacer el mal contra mí, pero Dios lo quiso para el bien, para
salvar a mucha gente con vida, como sucede hoy. \footnote{\textbf{50:20}
  Gén 45,5; Is 28,29} \bibleverse{21} Ahora, pues, no tengas miedo. Yo
los mantendré a ustedes y a sus hijos''. Los consoló y les habló con
amabilidad.

\hypertarget{la-vejez-y-la-muerte-de-josuxe9-su-ultimo-deseo}{%
\subsection{La vejez y la muerte de José; su ultimo
deseo}\label{la-vejez-y-la-muerte-de-josuxe9-su-ultimo-deseo}}

\bibleverse{22} José vivió en Egipto, él y la casa de su padre. José
vivió ciento diez años. \bibleverse{23} José vio a los hijos de Efraín
hasta la tercera generación. También los hijos de Maquir, hijo de
Manasés, nacieron sobre las rodillas de José. \footnote{\textbf{50:23}
  Gén 30,3} \bibleverse{24} José dijo a sus hermanos: ``Yo me estoy
muriendo, pero seguramente Dios los visitará y los hará subir de esta
tierra a la tierra que juró a Abraham, a Isaac y a Jacob.'' \footnote{\textbf{50:24}
  Heb 11,22} \bibleverse{25} José hizo un juramento a los hijos de
Israel, diciendo: ``Ciertamente Dios os visitará y haréis subir mis
huesos de aquí.'' \footnote{\textbf{50:25} Éxod 13,19; Jos 24,32}
\bibleverse{26} Murió, pues, José, de ciento diez años de edad; lo
embalsamaron y lo pusieron en un ataúd en Egipto.
