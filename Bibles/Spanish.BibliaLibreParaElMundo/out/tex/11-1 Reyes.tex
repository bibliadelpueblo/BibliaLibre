\hypertarget{la-vejez-de-david-abisag-ordenuxf3-por-sunem-cuidar-al-rey}{%
\subsection{La vejez de David; Abisag ordenó por Sunem cuidar al
rey}\label{la-vejez-de-david-abisag-ordenuxf3-por-sunem-cuidar-al-rey}}

\hypertarget{section}{%
\section{1}\label{section}}

\bibleverse{1} El rey David era ya viejo y entrado en años, y lo cubrían
con ropas, pero no podía calentarse. \bibleverse{2} Por lo tanto, sus
siervos le dijeron: ``Que se busque una joven virgen para mi señor el
rey. Que se ponga delante del rey y lo abrigue, y que se acueste en su
seno, para que mi señor el rey se abrigue''. \bibleverse{3} Así que
buscaron a una joven hermosa por todos los confines de Israel, y
encontraron a Abisag la sunamita, y la llevaron ante el rey.
\bibleverse{4} La joven era muy hermosa, y se puso al servicio del rey,
pero el rey no la conocía íntimamente.

\hypertarget{las-aspiraciones-de-adonias-a-gobernar-su-organizaciuxf3n-de-una-comida-de-sacrificio}{%
\subsection{Las aspiraciones de Adonias a gobernar; su organización de
una comida de
sacrificio}\label{las-aspiraciones-de-adonias-a-gobernar-su-organizaciuxf3n-de-una-comida-de-sacrificio}}

\bibleverse{5} Entonces Adonías, hijo de Haggit, se exaltó diciendo:
``Seré rey''. Entonces le preparó carros y jinetes, y cincuenta hombres
para que corrieran delante de él. \footnote{\textbf{1:5} 2Sam 3,4; 2Sam
  15,1} \bibleverse{6} Su padre no le había disgustado en ningún momento
diciéndole: ``¿Por qué lo has hecho?'', y además era un hombre muy
apuesto; y había nacido después de Absalón. \bibleverse{7} Se puso de
acuerdo con Joab, hijo de Sarvia, y con el sacerdote Abiatar, y ellos
siguieron a Adonías y lo ayudaron. \footnote{\textbf{1:7} 1Re 2,22}
\bibleverse{8} Pero el sacerdote Sadoc, Benaía hijo de Joiada, el
profeta Natán, Simei, Rei y los valientes que pertenecían a David, no
estaban con Adonías.

\bibleverse{9} Adonías mató ovejas, vacas y animales cebados junto a la
piedra de Zohelet, que está al lado de En Rogel; y llamó a todos sus
hermanos, los hijos del rey, y a todos los hombres de Judá, los
servidores del rey; \footnote{\textbf{1:9} Jos 15,7} \bibleverse{10}
pero no llamó al profeta Natán, ni a Benaía, ni a los valientes, ni a su
hermano Salomón.

\hypertarget{la-cita-de-nathan-con-betsabuxe9}{%
\subsection{La cita de Nathan con
Betsabé}\label{la-cita-de-nathan-con-betsabuxe9}}

\bibleverse{11} Entonces Natán habló a Betsabé, madre de Salomón,
diciendo: ``¿No has oído que Adonías, hijo de Haggit, reina, y que
nuestro señor David no lo sabe? \bibleverse{12} Ahora, pues, ven, déjame
que te aconseje, para que salves tu vida y la de tu hijo Salomón.
\bibleverse{13} Entra al rey David y dile: ``¿No juraste tú, mi señor el
rey, a tu siervo, diciendo: ``Ciertamente tu hijo Salomón reinará
después de mí y se sentará en mi trono''? ¿Por qué, pues, reina
Adonías?' \bibleverse{14} Mira,\footnote{\textbf{1:14} ``He aquí'', de
  ``\hebrew{הִנֵּה}'', significa mirar, fijarse, observar, ver o
  contemplar. Se utiliza a menudo como interjección.} mientras aún estás
hablando allí con el rey, yo también entraré después de ti y confirmaré
tus palabras.''

\hypertarget{betsabuxe9-le-recuerda-al-rey-su-promesa-la-intervenciuxf3n-de-nathan}{%
\subsection{Betsabé le recuerda al rey su promesa; La intervención de
Nathan}\label{betsabuxe9-le-recuerda-al-rey-su-promesa-la-intervenciuxf3n-de-nathan}}

\bibleverse{15} Betsabé entró en la habitación del rey. El rey era muy
anciano, y Abisag la sunamita servía al rey. \bibleverse{16} Betsabé se
inclinó y mostró respeto al rey. El rey le dijo: ``¿Qué quieres?''.

\bibleverse{17} Ella le dijo: ``Señor mío, tú juraste por
Yahvé\footnote{\textbf{1:17} ``Yahvé'' es el nombre propio de Dios, a
  veces traducido como ``\textsc{Señor}'' (en mayúsculas) en otras
  traducciones.} tu Dios\footnote{\textbf{1:17} Un talento equivale a
  unos 30 kilogramos o 66 libras o 965 onzas troy, por lo que 666
  talentos son unas 20 toneladas métricas} a tu siervo: `Ciertamente tu
hijo Salomón reinará después de mí y se sentará en mi trono'.
\bibleverse{18} Ahora, he aquí que Adonías reina, y tú, mi señor el rey,
no lo sabes. \bibleverse{19} Ha matado ganado, animales gordos y ovejas
en abundancia, y ha llamado a todos los hijos del rey, al sacerdote
Abiatar y a Joab, capitán del ejército; pero no ha llamado a Salomón, tu
siervo. \footnote{\textbf{1:19} 1Re 1,9-10} \bibleverse{20} Tú, mi señor
el rey, los ojos de todo Israel están puestos en ti, para que les digas
quién se sentará en el trono de mi señor el rey después de él.
\bibleverse{21} De lo contrario, cuando mi señor el rey duerma con sus
padres, yo y mi hijo Salomón seremos considerados criminales.''
\footnote{\textbf{1:21} Éxod 5,16}

\bibleverse{22} Mientras ella seguía hablando con el rey, entró el
profeta Natán. \bibleverse{23} Se lo comunicaron al rey, diciendo: ``He
aquí el profeta Natán''. Cuando entró ante el rey, se inclinó ante él
con el rostro en tierra. \bibleverse{24} Natán le dijo: ``Rey, señor
mío, ¿has dicho que Adonías reinará después de mí y que se sentará en mi
trono?'' \bibleverse{25} Porque hoy ha bajado y ha matado ganado,
animales gordos y ovejas en abundancia, y ha llamado a todos los hijos
del rey, a los capitanes del ejército y al sacerdote Abiatar. He aquí
que están comiendo y bebiendo delante de él, y diciendo: ``¡Viva el rey
Adonías!'' \footnote{\textbf{1:25} 2Sam 16,16} \bibleverse{26} Pero no
me ha llamado a mí, ni a tu siervo el sacerdote Sadoc, ni a Benaía hijo
de Joiada, ni a tu siervo Salomón. \footnote{\textbf{1:26} 1Re 1,10}
\bibleverse{27} ¿Acaso ha hecho esto mi señor el rey, y no has mostrado
a tus siervos quién debe sentarse en el trono de mi señor el rey después
de él?''

\hypertarget{david-confirma-su-promesa-anterior-con-un-juramento-nombra-a-salomuxf3n-como-su-sucesor-y-determina-su-unciuxf3n-inmediata}{%
\subsection{David confirma su promesa anterior con un juramento, nombra
a Salomón como su sucesor y determina su unción
inmediata}\label{david-confirma-su-promesa-anterior-con-un-juramento-nombra-a-salomuxf3n-como-su-sucesor-y-determina-su-unciuxf3n-inmediata}}

\bibleverse{28} El rey David respondió: ``Llama a Betsabé''. Ella vino a
la presencia del rey y se puso de pie ante el rey. \bibleverse{29} El
rey hizo un voto y dijo: ``Vive Yahvé, que ha redimido mi alma de toda
adversidad, \bibleverse{30} ciertamente, como te juré por Yahvé, el Dios
de Israel, diciendo: `Ciertamente tu hijo Salomón reinará después de mí,
y se sentará en mi trono en mi lugar'; ciertamente lo haré hoy.''

\bibleverse{31} Entonces Betsabé se inclinó con el rostro hacia la
tierra y mostró respeto al rey, y dijo: ``¡Viva mi señor el rey David
para siempre!''

\bibleverse{32} El rey David dijo: ``Llama a mí al sacerdote Sadoc, al
profeta Natán y a Benaía, hijo de Joiada''. Ellos se presentaron ante el
rey. \bibleverse{33} El rey les dijo: ``Llevad con vosotros a los
siervos de vuestro señor, y haced que mi hijo Salomón monte en mi propia
mula, y llevadlo a Gihón. \bibleverse{34} Que el sacerdote Sadoc y el
profeta Natán lo unjan allí como rey de Israel. Tocad la trompeta y
decid: ``¡Viva el rey Salomón! \bibleverse{35} Sube después de él, y
vendrá y se sentará en mi trono; porque él será rey en mi lugar. Yo lo
he designado para que sea príncipe sobre Israel y sobre Judá''.

\bibleverse{36} Benaía, hijo de Joiada, respondió al rey y dijo: ``Amén.
Que Yahvé, el Dios de mi señor el rey, lo diga. \bibleverse{37} Como
Yahvé ha estado con mi señor el rey, así esté con Salomón, y haga su
trono más grande que el trono de mi señor el rey David.''

\hypertarget{la-unciuxf3n-solemne-de-salomuxf3n-efecto-del-mensaje-en-cuestiuxf3n-en-los-reunidos-en-la-comida-del-sacrificio}{%
\subsection{La unción solemne de Salomón; Efecto del mensaje en cuestión
en los reunidos en la comida del
sacrificio}\label{la-unciuxf3n-solemne-de-salomuxf3n-efecto-del-mensaje-en-cuestiuxf3n-en-los-reunidos-en-la-comida-del-sacrificio}}

\bibleverse{38} Entonces el sacerdote Sadoc, el profeta Natán, Benaía
hijo de Joiada, y los queretanos y los peletanos, bajaron e hicieron
montar a Salomón en la mula del rey David, y lo llevaron a Gihón.
\bibleverse{39} El sacerdote Sadoc tomó el cuerno de aceite de la Tienda
y ungió a Salomón. Tocaron la trompeta, y todo el pueblo dijo: ``¡Viva
el rey Salomón!''. \footnote{\textbf{1:39} 1Cró 23,1; 1Cró 29,22}

\bibleverse{40} Todo el pueblo subió detrás de él, y el pueblo tocó la
flauta y se alegró mucho, de modo que la tierra tembló con su sonido.
\bibleverse{41} Adonías y todos los invitados que estaban con él lo
oyeron al terminar de comer. Cuando Joab oyó el sonido de la trompeta,
dijo: ``¿Por qué este ruido de la ciudad alborotada?''

\bibleverse{42} Mientras él aún hablaba, he aquí que llegó Jonatán, hijo
del sacerdote Abiatar, y Adonías le dijo: ``Entra, porque eres un hombre
digno y traes buenas noticias.'' \footnote{\textbf{1:42} 2Sam 15,27;
  2Sam 15,36}

\bibleverse{43} Jonatán respondió a Adonías: ``Ciertamente nuestro señor
el rey David ha hecho rey a Salomón. \bibleverse{44} El rey ha enviado
con él al sacerdote Sadoc, al profeta Natán, a Benaía hijo de Joiada, a
los cereteos y a los peleteos, y lo han hecho montar en la mula del rey.
\bibleverse{45} El sacerdote Sadoc y el profeta Natán lo han ungido como
rey en Gihón. Han subido de allí regocijados, de modo que la ciudad
volvió a resonar. Este es el ruido que han escuchado. \bibleverse{46}
Además, Salomón está sentado en el trono del reino. \footnote{\textbf{1:46}
  1Cró 28,5} \bibleverse{47} Además, los servidores del rey vinieron a
bendecir a nuestro señor, el rey David, diciendo: ``Que tu Dios haga que
el nombre de Salomón sea mejor que tu nombre, y que su trono sea más
grande que el tuyo''; y el rey se inclinó sobre el lecho.
\bibleverse{48} También dijo así el rey: `Bendito sea Yahvé, el Dios de
Israel, que ha dado uno para sentarse hoy en mi trono, viéndolo mis
ojos'\,''. \footnote{\textbf{1:48} 1Re 3,6}

\bibleverse{49} Todos los invitados de Adonías tuvieron miedo, se
levantaron y cada uno se fue por su lado.

\hypertarget{perdon-adonias}{%
\subsection{Perdon adonias}\label{perdon-adonias}}

\bibleverse{50} Adonías tuvo miedo a causa de Salomón, y se levantó y
fue a colgarse de los cuernos del altar. \bibleverse{51} Se le dijo a
Salomón: ``He aquí que Adonías teme al rey Salomón, pues está colgado de
los cuernos del altar, diciendo: `Que el rey Salomón me jure primero que
no matará a su siervo a espada'.'' \footnote{\textbf{1:51} 1Re 3,6}

\bibleverse{52} Salomón dijo: ``Si se muestra como un hombre digno, ni
un pelo suyo caerá a la tierra; pero si se encuentra maldad en él,
morirá.'' \footnote{\textbf{1:52} 2Sam 14,11}

\bibleverse{53} Entonces el rey Salomón envió, y lo hicieron bajar del
altar. Vino y se inclinó ante el rey Salomón; y éste le dijo: ``Vete a
tu casa''.

\hypertarget{las-instrucciones-de-david-a-salomuxf3n-y-su-muerte}{%
\subsection{Las instrucciones de David a Salomón y su
muerte}\label{las-instrucciones-de-david-a-salomuxf3n-y-su-muerte}}

\hypertarget{section-1}{%
\section{2}\label{section-1}}

\bibleverse{1} Se acercaban los días de David para que muriera; y mandó
a Salomón, su hijo, diciendo: \bibleverse{2} ``Me voy por el camino de
toda la tierra. Esfuérzate, pues, y muéstrate como un hombre;
\bibleverse{3} y guarda la instrucción de Yahvé, tu Dios, para andar por
sus caminos, para guardar sus estatutos, sus mandamientos, sus
ordenanzas y sus testimonios, según lo que está escrito en la ley de
Moisés, para que seas prosperado en todo lo que hagas y dondequiera que
te dirijas. \footnote{\textbf{2:3} Deut 17,14-20; Jos 1,7; Jos 23,6}
\bibleverse{4} Así Yahvé podrá confirmar su palabra que pronunció sobre
mí, diciendo: ``Si tus hijos son cuidadosos de su camino, para andar
delante de mí en la verdad con todo su corazón y con toda su alma, no te
faltará --- dijo --- un hombre en el trono de Israel.

\bibleverse{5} ``Además, tú sabes también lo que me hizo Joab, hijo de
Sarvia, lo que hizo a los dos capitanes de los ejércitos de Israel, a
Abner hijo de Ner y a Amasa hijo de Jeter, a quienes mató, y derramó la
sangre de la guerra en paz, y puso la sangre de la guerra en su faja que
tenía alrededor de la cintura y en sus sandalias que tenía en los pies.
\footnote{\textbf{2:5} 2Sam 3,27; 2Sam 20,10} \bibleverse{6} Haz, pues,
según tu sabiduría, y no dejes que su cabeza gris descienda al Seol en
paz. \footnote{\textbf{2:6} Gén 42,38} \bibleverse{7} Pero muestra
bondad a los hijos de Barzilai, el Galaadita, y que estén entre los que
comen en tu mesa; porque así vinieron a mí cuando huí de Absalón, tu
hermano. \footnote{\textbf{2:7} 2Sam 17,27; 2Sam 19,31-40}

\bibleverse{8} ``He aquí que está contigo Simei hijo de Gera, el
benjamita de Bahurim, que me maldijo con una grave maldición el día en
que fui a Mahanaim; pero él bajó a recibirme al Jordán, y yo le juré por
Yahvé, diciendo: `No te mataré a espada'. \footnote{\textbf{2:8} 2Sam
  16,5; 2Sam 19,16-23} \bibleverse{9} Ahora, pues, no lo tengas por
inocente, porque eres un hombre sabio; y sabrás lo que debes hacer con
él, y harás descender su cabeza gris al Seol con sangre.'' \footnote{\textbf{2:9}
  1Re 2,6; Sal 101,4; Sal 101,8}

\hypertarget{la-muerte-de-david-adquisiciuxf3n-de-salomuxf3n}{%
\subsection{La muerte de David; Adquisición de
Salomón}\label{la-muerte-de-david-adquisiciuxf3n-de-salomuxf3n}}

\bibleverse{10} David durmió con sus padres y fue enterrado en la ciudad
de David. \footnote{\textbf{2:10} Hech 13,36} \bibleverse{11} Los días
que David reinó sobre Israel fueron cuarenta años; reinó siete años en
Hebrón, y treinta y tres años en Jerusalén. \footnote{\textbf{2:11} 2Sam
  5,4-5; 1Cró 29,27} \bibleverse{12} Salomón se sentó en el trono de su
padre, y su reino se consolidó.

\hypertarget{adonia-asesinada-por-su-deseo-imprudente}{%
\subsection{Adonia asesinada por su deseo
imprudente}\label{adonia-asesinada-por-su-deseo-imprudente}}

\bibleverse{13} Entonces Adonías, hijo de Haggit, se acercó a Betsabé,
madre de Salomón. Ella le dijo: ``¿Vienes en paz?'' Dijo: ``En paz''.
\bibleverse{14} Dijo además: ``Tengo algo que decirte''. Ella dijo:
``Diga''.

\bibleverse{15} Él dijo: ``Ustedes saben que el reino era mío, y que
todo Israel se fijó en mí para que yo reinara. Sin embargo, el reino se
ha invertido y ha pasado a ser de mi hermano, pues era suyo de parte de
Yahvé. \footnote{\textbf{2:15} 1Re 1,5-40} \bibleverse{16} Ahora te pido
una petición. No me niegues''. Ella le dijo: ``Diga''.

\bibleverse{17} Le dijo: ``Por favor, habla con el rey Salomón (porque
no te dirá que no), para que me dé por esposa a Abisag la sunamita''.
\footnote{\textbf{2:17} 1Re 1,3; 2Sam 3,7}

\bibleverse{18} Betsabé dijo: ``Está bien. Hablaré por ti con el rey''.

\bibleverse{19} Betsabé fue, pues, a ver al rey Salomón para hablarle en
nombre de Adonías. El rey se levantó para recibirla y se inclinó ante
ella, se sentó en su trono e hizo que se pusiera un trono para la madre
del rey, y ella se sentó a su derecha. \bibleverse{20} Entonces ella le
dijo: ``Te pido una pequeña petición; no me la niegues''. El rey le
dijo: ``Pide, madre mía, porque no te lo negaré''.

\bibleverse{21} Ella dijo: ``Que Abisag la sunamita sea dada por esposa
a Adonías, tu hermano''.

\bibleverse{22} El rey Salomón respondió a su madre: ``¿Por qué pides a
Abisag la sunamita para Adonías? Pide también para él el reino, pues es
mi hermano mayor; también para él, y para el sacerdote Abiatar, y para
Joab hijo de Sarvia''. \footnote{\textbf{2:22} 1Re 1,6-7}
\bibleverse{23} Entonces el rey Salomón juró por Yahvé, diciendo: ``Que
Dios me haga así, y más aún, si Adonías no ha dicho esta palabra contra
su propia vida. \bibleverse{24} Ahora, pues, vive Yahvé, que me ha
afirmado y me ha puesto en el trono de mi padre David, y que me ha hecho
una casa como lo había prometido, ciertamente Adonías morirá hoy.''

\bibleverse{25} El rey Salomón envió a Benaía, hijo de Joiada, y cayó
sobre él, de modo que murió.

\hypertarget{el-sacerdote-abiatar-depuesto-y-desterrado}{%
\subsection{El sacerdote Abiatar depuesto y
desterrado}\label{el-sacerdote-abiatar-depuesto-y-desterrado}}

\bibleverse{26} El rey dijo al sacerdote Abiatar: ``Vete a Anatot, a tus
campos, porque eres digno de muerte. Pero no te daré muerte en este
momento, porque llevaste el arca del Señor delante de David mi padre, y
porque fuiste afligido en todo lo que mi padre fue afligido.''
\footnote{\textbf{2:26} Jer 1,1; 1Re 1,7; 1Sam 22,20; 1Sam 30,7; 2Sam
  15,24} \bibleverse{27} Así que Salomón echó a Abiatar de ser sacerdote
de Yavé, para que se cumpliera la palabra de Yavé que había dicho sobre
la casa de Elí en Silo.

\hypertarget{joab-ejecutado}{%
\subsection{Joab ejecutado}\label{joab-ejecutado}}

\bibleverse{28} Estas noticias llegaron a Joab, pues éste había seguido
a Adonías, aunque no había seguido a Absalón. Joab huyó a la Tienda de
Yahvé y se aferró a los cuernos del altar. \footnote{\textbf{2:28} 1Re
  1,51} \bibleverse{29} Le dijeron al rey Salomón: ``Joab ha huido a la
Tienda de Yahvé; y he aquí que está junto al altar''. Entonces Salomón
envió a Benaía, hijo de Joiada, diciendo: ``Ve y cae sobre él''.
\footnote{\textbf{2:29} Éxod 21,14}

\bibleverse{30} Benaía se acercó a la Tienda de Yahvé y le dijo: ``El
rey dice que salgas''. Dijo: ``No; pero moriré aquí''. Benaía volvió a
traer la palabra al rey, diciendo: ``Esto es lo que dijo Joab, y así me
respondió''.

\bibleverse{31} El rey le dijo: ``Haz lo que ha dicho, cae sobre él y
entiérralo, para que quites de mí y de la casa de mi padre la sangre que
Joab derramó sin causa. \bibleverse{32} El Señor le devolverá su sangre
sobre su propia cabeza, porque cayó sobre dos hombres más justos y
mejores que él y los mató a espada, sin que mi padre David lo supiera:
Abner hijo de Ner, capitán del ejército de Israel, y Amasa hijo de
Jeter, capitán del ejército de Judá. \footnote{\textbf{2:32} 1Re 2,5}
\bibleverse{33} Así que su sangre volverá sobre la cabeza de Joab y
sobre la cabeza de su descendencia para siempre. Pero para David, para
su descendencia, para su casa y para su trono, habrá paz para siempre de
parte de Yahvé''.

\bibleverse{34} Entonces Benaía hijo de Joiada subió y cayó sobre él, y
lo mató; y fue sepultado en su propia casa en el desierto.
\bibleverse{35} El rey puso a Benaía hijo de Joiada en su lugar al
frente del ejército, y el rey puso al sacerdote Sadoc en lugar de
Abiatar. \footnote{\textbf{2:35} 1Re 4,4}

\hypertarget{simei-fue-detenido-primero-en-jerusaluxe9n-y-luego-asesinado}{%
\subsection{Simei fue detenido primero en Jerusalén y luego
asesinado}\label{simei-fue-detenido-primero-en-jerusaluxe9n-y-luego-asesinado}}

\bibleverse{36} El rey envió a llamar a Simei y le dijo: ``Constrúyete
una casa en Jerusalén y vive allí, y no vayas a ninguna otra parte.
\bibleverse{37} Porque el día que salgas y pases por el arroyo Cedrón,
ten por seguro que morirás. Tu sangre estará sobre tu propia cabeza''.

\bibleverse{38} Simei dijo al rey: ``Lo que dices es bueno. Como mi
señor el rey ha dicho, así lo hará tu siervo''. Simei vivió muchos días
en Jerusalén.

\bibleverse{39} Al cabo de tres años, dos de los esclavos de Simei
huyeron a Aquis, hijo de Maaca, rey de Gat. Se lo contaron a Simei,
diciendo: ``He aquí que tus esclavos están en Gat''.

\bibleverse{40} Simei se levantó, ensilló su asno y fue a Gat, a Aquis,
para buscar sus esclavos; y Simei fue y trajo sus esclavos de Gat.
\bibleverse{41} Se le dijo a Salomón que Simei había ido de Jerusalén a
Gat, y que había vuelto.

\bibleverse{42} El rey mandó a llamar a Simei y le dijo: ``¿No te
advertí por Yahvé y te advertí diciendo: `Ten por seguro que el día que
salgas y andes por otro lado, ciertamente morirás'? \footnote{\textbf{2:42}
  1Re 2,38} \bibleverse{43} ¿Por qué, pues, no has cumplido el juramento
de Yavé y el mandamiento que te he instruido?'' \bibleverse{44} El rey
dijo además a Simei: ``Tú sabes en tu corazón toda la maldad que le
hiciste a mi padre David. Por lo tanto, Yahvé devolverá tu maldad sobre
tu propia cabeza. \footnote{\textbf{2:44} 1Re 2,8} \bibleverse{45} Pero
el rey Salomón será bendecido, y el trono de David será establecido ante
Yavé para siempre.'' \bibleverse{46} El rey ordenó entonces a Benaía,
hijo de Joiada, que saliera y cayera sobre él, para que muriera. El
reino quedó establecido en manos de Salomón. \footnote{\textbf{2:46}
  2Cró 1,1}

\hypertarget{el-matrimonio-de-salomuxf3n-con-una-princesa-egipcia-su-ofrenda-inaugural-y-su-sueuxf1o-en-gabauxf3n}{%
\subsection{El matrimonio de Salomón con una princesa egipcia; su
ofrenda inaugural y su sueño en
Gabaón}\label{el-matrimonio-de-salomuxf3n-con-una-princesa-egipcia-su-ofrenda-inaugural-y-su-sueuxf1o-en-gabauxf3n}}

\hypertarget{section-2}{%
\section{3}\label{section-2}}

\bibleverse{1} Salomón hizo una alianza matrimonial con el faraón, rey
de Egipto. Tomó a la hija del faraón y la llevó a la ciudad de David
hasta que terminó de construir su propia casa, la casa de Yahvé, y la
muralla alrededor de Jerusalén. \footnote{\textbf{3:1} Deut 23,7}
\bibleverse{2} Sin embargo, el pueblo sacrificaba en los lugares altos,
porque aún no había una casa construida para el nombre de Yavé.
\bibleverse{3} Salomón amaba a Yavé, caminando en los estatutos de David
su padre, excepto que sacrificaba y quemaba incienso en los lugares
altos.

\hypertarget{el-sacrificio-de-salomuxf3n-y-la-apariciuxf3n-de-dios-en-gabauxf3n}{%
\subsection{El sacrificio de Salomón y la aparición de Dios en
Gabaón}\label{el-sacrificio-de-salomuxf3n-y-la-apariciuxf3n-de-dios-en-gabauxf3n}}

\bibleverse{4} El rey fue a Gabaón para sacrificar allí, pues ese era el
gran lugar alto. Salomón ofreció mil holocaustos en ese altar.
\footnote{\textbf{3:4} 1Cró 21,29} \bibleverse{5} En Gabaón, Yavé se le
apareció a Salomón en un sueño de noche, y Dios le dijo: ``Pide lo que
debo darte''. \footnote{\textbf{3:5} 1Re 9,2}

\bibleverse{6} Salomón dijo: ``Has mostrado a tu siervo David, mi padre,
una gran bondad amorosa, porque caminó ante ti con verdad, con justicia
y con rectitud de corazón. Has guardado para él esta gran bondad
amorosa, que le has dado un hijo para que se siente en su trono, como
sucede hoy. \footnote{\textbf{3:6} 1Re 1,48} \bibleverse{7} Ahora,
Yahvé, mi Dios, has hecho rey a tu siervo en lugar de David, mi padre.
Yo sólo soy un niño pequeño. No sé salir ni entrar. \bibleverse{8} Tu
siervo está entre tu pueblo que has elegido, un pueblo grande, que no se
puede contar ni numerar por la multitud. \footnote{\textbf{3:8} 1Re 4,20}
\bibleverse{9} Da, pues, a tu siervo un corazón comprensivo para juzgar
a tu pueblo, para que pueda discernir entre el bien y el mal; porque
¿quién es capaz de juzgar a este gran pueblo tuyo?'' \footnote{\textbf{3:9}
  Sal 143,10}

\bibleverse{10} Esta petición agradó al Señor, pues Salomón había pedido
esto. \bibleverse{11} Dios le dijo: ``Porque has pedido esto, y no has
pedido para ti larga vida, ni has pedido riquezas para ti, ni has pedido
la vida de tus enemigos, sino que has pedido para ti entendimiento para
discernir la justicia, \bibleverse{12} he aquí que he hecho conforme a
tu palabra. He aquí que te he dado un corazón sabio y entendido, de modo
que no ha habido nadie como tú antes de ti, y después de ti no se
levantará ninguno como tú. \footnote{\textbf{3:12} Prov 2,3-6}
\bibleverse{13} También te he dado lo que no has pedido, riquezas y
honores, de modo que no habrá entre los reyes ninguno como tú en todos
tus días. \footnote{\textbf{3:13} Prov 3,13-16; Mat 6,33}
\bibleverse{14} Si andas en mis caminos, guardando mis estatutos y mis
mandamientos, como anduvo tu padre David, yo alargaré tus días.''

\bibleverse{15} Salomón se despertó, y he aquí que era un sueño.
Entonces vino a Jerusalén y se puso delante del arca de la alianza de
Yahvé, y ofreció holocaustos, ofreció ofrendas de paz e hizo un banquete
para todos sus servidores.

\hypertarget{el-sabio-juicio-de-salomuxf3n}{%
\subsection{El sabio juicio de
Salomón}\label{el-sabio-juicio-de-salomuxf3n}}

\bibleverse{16} Entonces vinieron al rey dos mujeres que eran
prostitutas y se presentaron ante él. \bibleverse{17} La primera dijo:
``Señor mío, yo y esta mujer vivimos en una misma casa. Yo di a luz con
ella en la casa. \bibleverse{18} Al tercer día de mi parto, esta mujer
también dio a luz. Estábamos juntas. No había ningún extraño con
nosotras en la casa, sólo nosotras dos en la casa. \bibleverse{19} El
hijo de esta mujer murió durante la noche, porque se acostó sobre él.
\bibleverse{20} Se levantó a medianoche y tomó a mi hijo de mi lado
mientras tu sierva dormía, y lo puso en su seno, y puso su hijo muerto
en mi seno. \bibleverse{21} Cuando me levanté por la mañana para
amamantar a mi hijo, he aquí que estaba muerto; pero cuando lo miré por
la mañana, he aquí que no era mi hijo que yo había dado a luz.''

\bibleverse{22} La otra mujer dijo: ``No, pero el vivo es mi hijo y el
muerto es tu hijo''. El primero dijo: ``¡No! Pero el muerto es tu hijo,
y el vivo es mi hijo''. Discutieron así ante el rey.

\bibleverse{23} Entonces el rey dijo: ``Uno dice: `Este es mi hijo que
vive, y tu hijo es el muerto'; y el otro dice: `No, pero tu hijo es el
muerto, y mi hijo es el vivo'\,''.

\bibleverse{24} El rey dijo: ``Tráiganme una espada''. Así que trajeron
una espada ante el rey.

\bibleverse{25} El rey dijo: ``Divide al niño vivo en dos y dale la
mitad a uno y la otra''.

\bibleverse{26} Entonces la mujer de quien era el niño vivo habló con el
rey, pues su corazón anhelaba a su hijo, y dijo: ``¡Oh, señor mío, dale
el niño vivo y no lo mates de ninguna manera!'' Pero el otro dijo: ``No
será ni mío ni tuyo. Divídelo''. \footnote{\textbf{3:26} Is 49,15}

\bibleverse{27} Entonces el rey respondió: ``Dale a la primera mujer el
niño vivo, y definitivamente no lo mates. Ella es su madre''.

\bibleverse{28} Todo Israel oyó el juicio que el rey había dictado; y
temieron al rey, porque vieron que la sabiduría de Dios estaba en él
para hacer justicia.

\hypertarget{los-principales-funcionarios-y-gobernadores-de-salomuxf3n-su-cortejo-poder-y-sabiduruxeda}{%
\subsection{Los principales funcionarios y gobernadores de Salomón; su
cortejo, poder y
sabiduría}\label{los-principales-funcionarios-y-gobernadores-de-salomuxf3n-su-cortejo-poder-y-sabiduruxeda}}

\hypertarget{section-3}{%
\section{4}\label{section-3}}

\bibleverse{1} El rey Salomón era rey de todo Israel. \bibleverse{2}
Estos fueron los príncipes que tuvo: Azarías hijo de Sadoc, sacerdote;
\footnote{\textbf{4:2} 1Re 2,35} \bibleverse{3} Elihoref y Ahías, hijos
de Sisá, escribas; Josafat hijo de Ahilud, registrador; \bibleverse{4}
Benaía hijo de Joiada estaba al frente del ejército; Sadoc y Abiatar
eran sacerdotes; \footnote{\textbf{4:4} 1Re 2,35; 2Sam 23,20}
\bibleverse{5} Azarías, hijo de Natán, estaba al frente de los
oficiales; Zabud, hijo de Natán, era ministro principal, amigo del rey;
\bibleverse{6} Ahishar estaba al frente de la casa; y Adoniram, hijo de
Abda, estaba al frente de los hombres sometidos a trabajos forzados.
\footnote{\textbf{4:6} 1Re 5,14}

\bibleverse{7} Salomón tenía doce oficiales sobre todo Israel, que
proveían de alimentos al rey y a su casa. Cada uno tenía que hacer
provisión para un mes del año. \bibleverse{8} Estos son sus nombres: Ben
Hur, en la región montañosa de Efraín; \bibleverse{9} Ben Deker, en
Makaz, en Shaalbim, Bet Shemesh y Elón Bet Hanan; \bibleverse{10} Ben
Hesed, en Arubboth (Socoh y toda la tierra de Hefer le pertenecían);
\bibleverse{11} Ben Abinadab, en toda la altura de Dor (tenía como
esposa a Tafat, la hija de Salomón); \footnote{\textbf{4:11} 1Sam 16,8}
\bibleverse{12} Baana hijo de Ahilud, en Taanac y Meguido, y en toda Bet
Shean que está junto a Zaretán, debajo de Jezreel, desde Bet Shean hasta
Abel Meholá, hasta más allá de Jokmeam; \bibleverse{13} Ben Geber, en
Ramot Galaad (las ciudades de Jair hijo de Manasés, que están en Galaad,
le pertenecían; y la región de Argob, que está en Basán, sesenta grandes
ciudades con murallas y barras de bronce, le pertenecían); \footnote{\textbf{4:13}
  Núm 32,41} \bibleverse{14} Ahinadab hijo de Iddo, en Mahanaim;
\bibleverse{15} Ahimaas, en Neftalí (también tomó por esposa a Basemat,
hija de Salomón); \bibleverse{16} Baana hijo de Husai, en Aser y Bealot;
\bibleverse{17} Josafat hijo de Parúa, en Isacar; \bibleverse{18} Simei
hijo de Ela, en Benjamín; \bibleverse{19} Geber hijo de Uri, en la
tierra de Galaad, país de Sehón, rey de los amorreos, y de Og, rey de
Basán.

\bibleverse{20} Judá e Israel eran numerosos como la arena que está
junto al mar, en multitud, comiendo y bebiendo y alegrándose.
\footnote{\textbf{4:20} 1Re 3,8; Gén 13,16; Gén 22,17} \bibleverse{21}
Salomón dominaba todos los reinos desde el río hasta el país de los
filisteos y hasta la frontera de Egipto. Traían tributo y servían a
Salomón todos los días de su vida. \bibleverse{22} La provisión de
Salomón para un día era de treinta cors de harina fina, sesenta medidas
de harina, \bibleverse{23} diez cabezas de ganado gordo, veinte cabezas
de ganado de los pastos y cien ovejas, además de ciervos, gacelas,
corzos y aves de corral cebadas.

\hypertarget{el-poder-y-la-gloria-la-sabiduruxeda-y-la-poesuxeda-de-salomuxf3n}{%
\subsection{El poder y la gloria, la sabiduría y la poesía de
Salomón}\label{el-poder-y-la-gloria-la-sabiduruxeda-y-la-poesuxeda-de-salomuxf3n}}

\bibleverse{24} Porque tenía dominio sobre todo lo que había a este lado
del río, desde Tiphsah hasta Gaza, sobre todos los reyes de este lado
del río; y tenía paz por todos lados alrededor de él. \bibleverse{25}
Judá e Israel vivían seguros, cada uno bajo su vid y bajo su higuera,
desde Dan hasta Beerseba, todos los días de Salomón. \footnote{\textbf{4:25}
  Lev 25,18; 2Re 18,31} \bibleverse{26} Salomón tenía cuarenta mil
puestos de caballos para sus carros, y doce mil jinetes. \bibleverse{27}
Esos oficiales proveían de comida al rey Salomón y a todos los que
venían a la mesa del rey Salomón, cada uno en su mes. No dejaron que
faltara nada. \bibleverse{28} También llevaron cebada y paja para los
caballos y corceles veloces al lugar donde estaban los oficiales, cada
uno según su deber. \bibleverse{29} Dios le dio a Salomón abundante
sabiduría, entendimiento y amplitud de mente como la arena que está a la
orilla del mar. \footnote{\textbf{4:29} 1Re 3,12} \bibleverse{30} La
sabiduría de Salomón superó a la de todos los hijos de Oriente y a toda
la sabiduría de Egipto. \bibleverse{31} Porque era más sabio que todos
los hombres: más sabio que Etán el ezraíta, Hemán, Calcol y Darda, hijos
de Mahol; y su fama se extendía por todas las naciones de alrededor.
\bibleverse{32} Habló tres mil proverbios, y sus canciones fueron mil
cinco. \footnote{\textbf{4:32} Ecl 12,9} \bibleverse{33} Hablaba de los
árboles, desde el cedro que está en el Líbano hasta el hisopo que crece
en el muro; también hablaba de los animales, de las aves, de los
reptiles y de los peces. \bibleverse{34} A la sabiduría de Salomón
acudían gentes de todas las naciones, enviadas por todos los reyes de la
tierra que habían oído hablar de su sabiduría. \footnote{\textbf{4:34}
  1Re 10,1; 1Re 10,6}

\hypertarget{el-contrato-de-salomuxf3n-con-hiram-de-tiro-y-los-preparativos-para-la-construcciuxf3n-de-un-templo}{%
\subsection{El contrato de Salomón con Hiram de Tiro y los preparativos
para la construcción de un
templo}\label{el-contrato-de-salomuxf3n-con-hiram-de-tiro-y-los-preparativos-para-la-construcciuxf3n-de-un-templo}}

\hypertarget{section-4}{%
\section{5}\label{section-4}}

\bibleverse{1} Hiram, rey de Tiro, envió a sus servidores a Salomón,
pues había oído que lo habían ungido rey en lugar de su padre, y Hiram
siempre había amado a David. \footnote{\textbf{5:1} 2Sam 5,11}

\hypertarget{mensaje-de-salomuxf3n-y-peticiuxf3n-a-hiram}{%
\subsection{Mensaje de Salomón y petición a
Hiram}\label{mensaje-de-salomuxf3n-y-peticiuxf3n-a-hiram}}

\bibleverse{2} Salomón envió a decir a Hiram: \bibleverse{3} ``Sabes que
mi padre David no pudo construir una casa para el nombre de Yavé, su
Dios, a causa de las guerras que lo rodeaban por todas partes, hasta que
Yavé puso a sus enemigos bajo la planta de sus pies. \bibleverse{4} Pero
ahora el Señor, mi Dios, me ha dado descanso por todos lados. No hay
enemigo ni ocurrencia del mal. \bibleverse{5} He aquí que me propongo
edificar una casa para el nombre de Yavé, mi Dios, tal como Yavé habló a
David, mi padre, diciendo: ``Tu hijo, a quien pondré en tu trono en tu
lugar, edificará la casa para mi nombre.' \footnote{\textbf{5:5} 2Sam
  7,13} \bibleverse{6} Ordena, pues, que se corten para mí cedros del
Líbano. Mis siervos estarán con los tuyos, y yo te daré un salario para
tus siervos, según todo lo que digas. Porque sabes que no hay nadie
entre nosotros que sepa cortar madera como los sidonios''.

\hypertarget{la-respuesta-y-la-promesa-de-hiram}{%
\subsection{La respuesta y la promesa de
Hiram}\label{la-respuesta-y-la-promesa-de-hiram}}

\bibleverse{7} Cuando Hiram oyó las palabras de Salomón, se alegró mucho
y dijo: ``Bendito sea hoy Yahvé, que ha dado a David un hijo sabio para
gobernar este gran pueblo.'' \footnote{\textbf{5:7} 1Re 10,9}
\bibleverse{8} Hiram envió a decir a Salomón: ``He oído el mensaje que
me has enviado. Haré todo lo que desees en cuanto a la madera de cedro y
de ciprés. \bibleverse{9} Mis servidores las bajarán del Líbano al mar.
Yo las convertiré en balsas para que vayan por mar al lugar que me
indiques, y allí las haré partir, y tú las recibirás. Cumplirás mi
deseo, al dar alimento a mi casa''.

\bibleverse{10} Entonces Hiram le dio a Salomón madera de cedro y de
ciprés según su deseo. \bibleverse{11} Salomón le dio a Hiram veinte mil
cors de trigo para la comida de su casa, y veinte cors de aceite puro.
Salomón le daba esto a Hiram año tras año. \bibleverse{12} El Señor le
dio a Salomón la sabiduría que le había prometido. Hubo paz entre Hiram
y Salomón, y ambos hicieron un tratado juntos. \footnote{\textbf{5:12}
  1Re 4,29; 1Re 3,12}

\hypertarget{los-obreros-de-salomuxf3n-y-la-obra-preparatoria-final-para-la-construcciuxf3n-del-templo}{%
\subsection{Los obreros de Salomón y la obra preparatoria final para la
construcción del
templo}\label{los-obreros-de-salomuxf3n-y-la-obra-preparatoria-final-para-la-construcciuxf3n-del-templo}}

\bibleverse{13} El rey Salomón levantó una leva de todo Israel; la leva
fue de treinta mil hombres. \bibleverse{14} Los envió al Líbano, diez
mil al mes por cursos; durante un mes estaban en el Líbano, y dos meses
en casa; y Adoniram estaba al frente de los hombres sometidos a trabajos
forzados. \footnote{\textbf{5:14} 1Re 4,6} \bibleverse{15} Salomón tenía
setenta mil que llevaban cargas, y ochenta mil que eran cortadores de
piedra en las montañas, \bibleverse{16} además de los oficiales
principales de Salomón que estaban al frente de la obra: tres mil
trescientos que mandaban sobre el pueblo que trabajaba en la obra.
\bibleverse{17} El rey ordenó, y ellos cortaron piedras grandes, piedras
costosas, para poner los cimientos de la casa con piedra labrada.
\bibleverse{18} Los constructores de Salomóny los constructores de Hiram
y los gebalitas las cortaron y prepararon la madera y las piedras para
construir la casa. \footnote{\textbf{5:18} Jos 13,5; Ezeq 27,9}

\hypertarget{construcciuxf3n-del-templo-y-palacios-reales}{%
\subsection{Construcción del templo y palacios
reales}\label{construcciuxf3n-del-templo-y-palacios-reales}}

\hypertarget{section-5}{%
\section{6}\label{section-5}}

\bibleverse{1} En el año cuatrocientos ochenta después de que los hijos
de Israel salieron de la tierra de Egipto, en el cuarto año del reinado
de Salomón sobre Israel, en el mes de Ziv, que es el segundo mes,
comenzó a edificar la casa de Yavé. \bibleverse{2} La casa que el rey
Salomón construyó para Yavé tenía una longitud de sesenta codos, y su
anchura veinte, y su altura treinta codos. \bibleverse{3} El pórtico
frente al templo de la casa tenía una longitud de veinte codos, que se
extendía a lo ancho de la casa. Su anchura era de diez codos frente a la
casa. \footnote{\textbf{6:3} 1Re 7,15-22; Juan 10,23} \bibleverse{4}
Hizo ventanas de celosía fija para la casa. \bibleverse{5} Contra la
pared de la casa, construyó pisos alrededor, contra las paredes de la
casa alrededor, tanto del templo como del santuario interior; e hizo
habitaciones laterales alrededor. \bibleverse{6} El piso más bajo tenía
cinco codos de ancho, el del medio seis codos de ancho y el tercero
siete codos de ancho, pues por fuera hizo desviaciones en la pared de la
casa en todo su perímetro, para que las vigas no se introdujeran en las
paredes de la casa. \bibleverse{7} La casa, cuando se estaba
construyendo, era de piedra preparada en la cantera; y no se oyó en la
casa ningún martillo ni hacha ni ninguna herramienta de hierro mientras
se estaba construyendo. \bibleverse{8} La puerta de las habitaciones del
medio estaba en el lado derecho de la casa. Se subía por una escalera de
caracol al piso del medio, y del medio se subía al tercero.
\bibleverse{9} Así que construyó la casa y la terminó; y la cubrió con
vigas y tablas de cedro. \bibleverse{10} Construyó los pisos a lo largo
de la casa, cada uno de cinco codos de altura, y los apoyó en la casa
con maderas de cedro.

\hypertarget{una-promesa-de-dios-a-salomuxf3n}{%
\subsection{Una promesa de Dios a
Salomón}\label{una-promesa-de-dios-a-salomuxf3n}}

\bibleverse{11} La palabra de Yahvé vino a Salomón, diciendo:
\bibleverse{12} ``En cuanto a esta casa que estás construyendo, si andas
en mis estatutos, y ejecutas mis ordenanzas, y guardas todos mis
mandamientos para andar en ellos, entonces confirmaré contigo mi
palabra, que hablé a David tu padre. \footnote{\textbf{6:12} 2Sam 7,13}
\bibleverse{13} Habitaré en medio de los hijos de Israel y no abandonaré
a mi pueblo Israel''. \footnote{\textbf{6:13} Éxod 29,45}

\hypertarget{el-interior-del-templo}{%
\subsection{El interior del templo}\label{el-interior-del-templo}}

\bibleverse{14} Salomón construyó la casa y la terminó. \bibleverse{15}
Construyó las paredes de la casa por dentro con tablas de cedro; desde
el piso de la casa hasta las paredes del techo, las cubrió por dentro
con madera. Cubrió el piso de la casa con tablas de ciprés.
\bibleverse{16} Construyó veinte codos de la parte trasera de la casa
con tablas de cedro desde el suelo hasta el techo. Construyó esto por
dentro, para un santuario interior, para el lugar santísimo.
\bibleverse{17} La parte delantera del santuario tenía cuarenta codos de
largo. \bibleverse{18} En el interior de la casa había cedro tallado con
capullos y flores abiertas. Todo era de cedro. No se veía ninguna
piedra. \bibleverse{19} Preparó un santuario interior en medio de la
casa de adentro, para colocar allí el arca de la alianza de Yavé.
\bibleverse{20} El santuario interior tenía veinte metros de largo,
veinte metros de ancho y veinte metros de alto. La recubrió de oro puro.
Cubrió el altar con cedro. \bibleverse{21} Salomón recubrió de oro puro
la casa por dentro. Trazó cadenas de oro delante del santuario interior
y lo recubrió de oro. \bibleverse{22} Recubrió de oro toda la casa,
hasta que estuvo terminada. También recubrió de oro todo el altar que
pertenecía al santuario interior.

\hypertarget{la-decoraciuxf3n-del-santuario}{%
\subsection{La decoración del
santuario}\label{la-decoraciuxf3n-del-santuario}}

\bibleverse{23} En el santuario interior hizo dos querubines de madera
de olivo, de diez codos de altura cada uno. \footnote{\textbf{6:23} Éxod
  37,7-9} \bibleverse{24} La longitud de un ala del querubín era de
cinco codos, y la del otro, de cinco codos. Desde la punta de un ala
hasta la punta de la otra había diez codos. \bibleverse{25} El otro
querubín medía diez codos. Ambos querubines tenían una misma medida y
una misma forma. \bibleverse{26} Un querubín tenía diez codos de altura,
y el otro querubín también. \bibleverse{27} Puso los querubines dentro
de la casa interior. Las alas de los querubines estaban extendidas, de
tal manera que el ala de uno tocaba una pared y el ala del otro querubín
tocaba la otra pared; y sus alas se tocaban entre sí en medio de la
casa. \bibleverse{28} Recubrió los querubines con oro.

\bibleverse{29} Talló todas las paredes de la casa alrededor con figuras
talladas de querubines, palmeras y flores abiertas, por dentro y por
fuera. \bibleverse{30} Recubrió de oro el suelo de la casa, por dentro y
por fuera.

\hypertarget{las-puertas-y-el-patio-interior}{%
\subsection{Las puertas y el patio
interior}\label{las-puertas-y-el-patio-interior}}

\bibleverse{31} Para la entrada del santuario interior, hizo puertas de
madera de olivo. El dintel y los postes de la puerta eran la quinta
parte de la pared. \bibleverse{32} Hizo, pues, dos puertas de madera de
olivo, y talló en ellas tallas de querubines, palmeras y flores
abiertas, y las recubrió de oro. Extendió el oro sobre los querubines y
las palmeras. \bibleverse{33} También hizo los postes de la entrada del
templo de madera de olivo, de una cuarta parte de la pared,
\bibleverse{34} y dos puertas de madera de ciprés. Las dos hojas de una
puerta eran plegables, y las dos hojas de la otra puerta eran plegables.
\bibleverse{35} Esculpió querubines, palmeras y flores abiertas, y los
recubrió con oro ajustado a la obra grabada. \bibleverse{36} Construyó
el atrio interior con tres hileras de piedra tallada y una hilera de
vigas de cedro.

\hypertarget{tiempo-de-construcciuxf3n}{%
\subsection{Tiempo de construcción}\label{tiempo-de-construcciuxf3n}}

\bibleverse{37} Los cimientos de la Casa de Yahvé fueron puestos en el
cuarto año, en el mes de Ziv. \footnote{\textbf{6:37} 1Re 6,1}
\bibleverse{38} En el undécimo año, en el mes de Bul, que es el octavo
mes, la casa fue terminada en todas sus partes y según todas sus
especificaciones. Así pasó siete años construyéndola.

\hypertarget{descripciuxf3n-de-los-otros-edificios-seculares-de-salomuxf3n}{%
\subsection{Descripción de los otros edificios (seculares) de
Salomón}\label{descripciuxf3n-de-los-otros-edificios-seculares-de-salomuxf3n}}

\hypertarget{section-6}{%
\section{7}\label{section-6}}

\bibleverse{1} Salomón estuvo construyendo su propia casa durante trece
años, y terminó toda su casa. \footnote{\textbf{7:1} 1Re 9,10}

\hypertarget{la-casa-del-bosque-del-luxedbano}{%
\subsection{La casa del bosque del
Líbano}\label{la-casa-del-bosque-del-luxedbano}}

\bibleverse{2} Pues construyó la Casa del Bosque del Líbano. Su longitud
era de cien codos, su anchura de cincuenta codos, y su altura de treinta
codos, sobre cuatro hileras de pilares de cedro, con vigas de cedro
sobre los pilares. \footnote{\textbf{7:2} Is 22,8} \bibleverse{3} Estaba
cubierto de cedro por encima de las cuarenta y cinco vigas que había
sobre las columnas, quince en cada fila. \bibleverse{4} Había vigas en
tres filas, y la ventana estaba frente a la ventana en tres filas.
\bibleverse{5} Todas las puertas y los postes estaban hechos a escuadra
con vigas, y la ventana estaba frente a la ventana en tres filas.

\hypertarget{los-edificios-restantes-del-palacio-de-salomuxf3n}{%
\subsection{Los edificios restantes del palacio de
Salomón}\label{los-edificios-restantes-del-palacio-de-salomuxf3n}}

\bibleverse{6} Hizo la sala de pilares. Su longitud era de cincuenta
codos y su anchura de treinta codos, con un pórtico delante de ellos, y
pilares y un umbral delante de ellos. \bibleverse{7} Hizo el pórtico del
trono donde iba a juzgar, el pórtico del juicio; y estaba cubierto de
cedro de piso a piso. \bibleverse{8} La casa donde iba a habitar, el
otro atrio dentro del pórtico, era de la misma construcción. Hizo
también una casa para la hija de Faraón (a quien Salomón había tomado
por esposa), como este pórtico. \footnote{\textbf{7:8} 1Re 3,1}
\bibleverse{9} Todo esto era de piedras costosas, de piedra cortada a
medida, aserrada con sierras, por dentro y por fuera, desde los
cimientos hasta la albardilla, y así por fuera hasta el gran patio.
\bibleverse{10} Los cimientos eran de piedras costosas, piedras grandes,
piedras de diez codos y piedras de ocho codos. \bibleverse{11} Encima
había piedras costosas, piedras cortadas a medida, y madera de cedro.
\bibleverse{12} El gran atrio que lo rodeaba tenía tres hileras de
piedra tallada con una hilera de vigas de cedro, como el atrio interior
de la casa de Yahvé y el pórtico de la casa.

\hypertarget{informaciuxf3n-sobre-el-artista-hiram-de-tiro}{%
\subsection{Información sobre el artista Hiram de
Tiro}\label{informaciuxf3n-sobre-el-artista-hiram-de-tiro}}

\bibleverse{13} El rey Salomón envió y trajo a Hiram de Tiro.
\footnote{\textbf{7:13} 2Cró 2,13-14} \bibleverse{14} Era hijo de una
viuda de la tribu de Neftalí, y su padre era un hombre de Tiro, obrero
del bronce; y estaba lleno de sabiduría y entendimiento y habilidad para
trabajar todas las obras en bronce. Vino al rey Salomón y realizó todo
su trabajo. \footnote{\textbf{7:14} Gén 4,22; Éxod 31,3-4}

\hypertarget{los-dos-pilares-de-bronce-jachin-y-boaz}{%
\subsection{Los dos pilares de bronce (Jachin y
Boaz)}\label{los-dos-pilares-de-bronce-jachin-y-boaz}}

\bibleverse{15} Hizo las dos columnas de bronce, de dieciocho codos de
altura cada una, y una línea de doce codos rodeaba cada una de ellas.
\footnote{\textbf{7:15} 2Re 25,17; 2Cró 3,15-17} \bibleverse{16} Hizo
dos capiteles de bronce fundido para colocarlos en la parte superior de
las columnas. La altura de un capitel era de cinco codos, y la del otro,
de cinco codos. \bibleverse{17} Para los capiteles que estaban en la
parte superior de las columnas, había redes de damas y coronas de
cadenas: siete para un capitel y siete para el otro. \bibleverse{18}
Hizo, pues, las columnas, y alrededor de la primera red hubo dos hileras
de granadas para cubrir los capiteles que estaban en la parte superior
de las columnas; y lo mismo hizo para el otro capitel. \bibleverse{19}
Los capiteles que estaban en la parte superior de las columnas del
pórtico eran de obra de lirio, de cuatro codos. \bibleverse{20} También
había capiteles encima de las dos columnas, cerca del vientre que estaba
junto a la red. Había doscientas granadas en hileras alrededor del otro
capitel. \bibleverse{21} Colocó las columnas en el pórtico del templo.
Levantó la columna de la derecha y llamó su nombre Jachín; y levantó la
columna de la izquierda y llamó su nombre Boaz. \bibleverse{22} En la
parte superior de las columnas había un trabajo de lirios. Así quedó
terminada la obra de las columnas.

\hypertarget{el-mar-descarado-o-grandes-charcos-de-agua}{%
\subsection{El mar descarado (o grandes charcos de
agua)}\label{el-mar-descarado-o-grandes-charcos-de-agua}}

\bibleverse{23} Hizo el mar fundido de diez codos de borde a borde, de
forma redonda. Su altura era de cinco codos, y lo rodeaba una línea de
treinta codos. \footnote{\textbf{7:23} 2Cró 4,2-5} \bibleverse{24}
Debajo de su borde había brotes que lo rodeaban por diez codos, rodeando
el mar. Los brotes estaban en dos hileras, fundidos cuando fue fundido.
\bibleverse{25} Estaba sobre doce bueyes, tres que miraban hacia el
norte, tres que miraban hacia el oeste, tres que miraban hacia el sur y
tres que miraban hacia el este. \bibleverse{26} Su grosor era de un
palmo. Su borde estaba labrado como el borde de una copa, como la flor
de un lirio. Tenía capacidad para dos mil baños.

\hypertarget{las-diez-sillas-y-los-diez-calderos-de-sacrificio}{%
\subsection{Las diez sillas y los diez calderos de
sacrificio}\label{las-diez-sillas-y-los-diez-calderos-de-sacrificio}}

\bibleverse{27} Hizo las diez bases de bronce. La longitud de una base
era de cuatro codos, su anchura de cuatro codos y su altura de tres
codos. \footnote{\textbf{7:27} 2Cró 4,6; 2Cró 4,10} \bibleverse{28} El
trabajo de las bases era así: tenían paneles, y había paneles entre las
repisas; \bibleverse{29} y en los paneles que estaban entre las repisas
había leones, bueyes y querubines; y en las repisas había un pedestal
arriba; y debajo de los leones y los bueyes había coronas de obra
colgante. \bibleverse{30} Cada base tenía cuatro ruedas y ejes de
bronce, y sus cuatro pies tenían soportes. Los soportes estaban fundidos
debajo de la cuenca, con coronas a los lados de cada uno.
\bibleverse{31} Su abertura dentro del capitel y por encima era de un
codo. Su abertura era redonda como la obra de un pedestal, de un codo y
medio; y también en su abertura había grabados, y sus paneles eran
cuadrados, no redondos. \bibleverse{32} Las cuatro ruedas estaban debajo
de los paneles, y los ejes de las ruedas estaban en la base. La altura
de una rueda era de un codo y medio codo. \bibleverse{33} El trabajo de
las ruedas era como el de una rueda de carro. Sus ejes, sus llantas, sus
radios y sus cubos eran todos de metal fundido. \bibleverse{34} Había
cuatro soportes en las cuatro esquinas de cada base. Sus soportes eran
de la propia base. \bibleverse{35} En la parte superior de la base había
una banda redonda de medio codo de altura; y en la parte superior de la
base sus soportes y sus paneles eran iguales. \bibleverse{36} En las
placas de sus soportes y en sus paneles, grabó querubines, leones y
palmeras, cada uno en su espacio, con coronas alrededor. \bibleverse{37}
Hizo las diez basas de esta manera: todas tenían una misma fundición,
una misma medida y una misma forma. \bibleverse{38} Hizo diez pilas de
bronce. Una cuenca contenía cuarenta baños. Cada cuenca medía cuatro
codos. Una palangana estaba en cada una de las diez bases.
\bibleverse{39} Colocó las bases, cinco a la derecha y cinco a la
izquierda de la casa. Puso el mar en el lado derecho de la casa hacia el
este y hacia el sur.

\hypertarget{otras-herramientas-de-sacrificio-de-minerales-resumen-general}{%
\subsection{Otras herramientas de sacrificio de minerales; resumen
general}\label{otras-herramientas-de-sacrificio-de-minerales-resumen-general}}

\bibleverse{40} Hiram hizo las ollas, las palas y las pilas. Así terminó
Hiram de hacer toda la obra que trabajó para el rey Salomón en la casa
de Yahvé: \footnote{\textbf{7:40} 2Cró 4,11-18} \bibleverse{41} las dos
columnas; las dos copas de los capiteles que estaban en la parte
superior de las columnas; las dos redes para cubrir las dos copas de los
capiteles que estaban en la parte superior de las columnas;
\bibleverse{42} las cuatrocientas granadas para las dos redes; dos
hileras de granadas para cada red, para cubrir las dos copas de los
capiteles que estaban sobre las columnas; \bibleverse{43} las diez
bases; las diez cuencas sobre las bases; \bibleverse{44} el único mar;
los doce bueyes bajo el mar; \bibleverse{45} las ollas; las palas; y las
cuencas. Todos estos recipientes, que Hiram hizo para el rey Salomón en
la casa de Yahvé, eran de bronce bruñido. \bibleverse{46} El rey los
fundió en la llanura del Jordán, en la tierra arcillosa entre Sucot y
Zaretán. \bibleverse{47} Salomón dejó todos los recipientes sin pesar,
porque eran muchos. No se pudo determinar el peso del bronce.

\hypertarget{los-instrumentos-de-oro-del-templo-graduaciuxf3n}{%
\subsection{Los instrumentos de oro del templo;
Graduación}\label{los-instrumentos-de-oro-del-templo-graduaciuxf3n}}

\bibleverse{48} Salomón hizo todos los utensilios que había en la casa
de Yahvé el altar de oro y la mesa sobre la que estaba el pan de la
función, de oro; \footnote{\textbf{7:48} 2Cró 4,19-999} \bibleverse{49}
y los candelabros, cinco a la derecha y cinco a la izquierda, delante
del santuario interior, de oro puro; y las flores, las lámparas y las
pinzas, de oro; \bibleverse{50} las copas, los apagadores, las jofainas,
las cucharas y las sartenes para el fuego, de oro puro; y los goznes,
tanto para las puertas de la casa interior, el lugar santísimo, como
para las puertas de la casa, del templo, de oro.

\bibleverse{51} Asítoda la obra que el rey Salomón hizo en la casa de
Yavé fue terminada. Salomón trajo las cosas que su padre David había
dedicado --- la plata, el oro y los utensilios --- y las puso en los
tesoros de la casa de Yavé.

\hypertarget{traslado-del-arca-al-templo}{%
\subsection{Traslado del arca al
templo}\label{traslado-del-arca-al-templo}}

\hypertarget{section-7}{%
\section{8}\label{section-7}}

\bibleverse{1} Entonces Salomón reunió a los ancianos de Israel con
todos los jefes de las tribus, los jefes de las casas paternas de los
hijos de Israel, ante el rey Salomón en Jerusalén, para hacer subir el
arca de la alianza de Yahvé desde la ciudad de David, que es Sión.
\footnote{\textbf{8:1} 2Cró 5,1} \bibleverse{2} Todos los hombres de
Israel se reunieron con el rey Salomón en la fiesta del mes de Etanim,
que es el séptimo mes. \bibleverse{3} Vinieron todos los ancianos de
Israel, y los sacerdotes recogieron el arca. \bibleverse{4} Trajeron el
arca de Yavé, la Carpa del Encuentro y todos los utensilios sagrados que
estaban en la Carpa. Los sacerdotes y los levitas los subieron.
\bibleverse{5} El rey Salomón y toda la congregación de Israel, que se
había reunido con él, estaban con él ante el arca, sacrificando ovejas y
ganado que no se podía contar ni numerar por la multitud. \footnote{\textbf{8:5}
  2Sam 6,13} \bibleverse{6} Los sacerdotes introdujeron el arca de la
alianza de Yahvé en su lugar, en el santuario interior de la casa, en el
lugar santísimo, bajo las alas de los querubines. \bibleverse{7} Los
querubines extendían sus alas sobre el lugar del arca, y los querubines
cubrían el arca y sus varas por encima. \bibleverse{8} Los postes eran
tan largos que los extremos de los postes se veían desde el lugar santo,
delante del santuario interior, pero no se veían afuera. Allí están
hasta el día de hoy. \footnote{\textbf{8:8} Éxod 25,13-15}
\bibleverse{9} En el arca no había nada más que las dos tablas de piedra
que Moisés puso allí en Horeb, cuando Yahvé hizo la alianza con los
hijos de Israel, al salir de la tierra de Egipto. \footnote{\textbf{8:9}
  Heb 9,4}

\hypertarget{la-apariciuxf3n-de-la-gloria-de-dios}{%
\subsection{La aparición de la gloria de
Dios}\label{la-apariciuxf3n-de-la-gloria-de-dios}}

\bibleverse{10} Cuando los sacerdotes salieron del lugar santo, la nube
llenó la casa de Yavé, \bibleverse{11} de modo que los sacerdotes no
podían estar de pie para ejercer su ministerio a causa de la nube,
porque la gloria de Yavé llenaba la casa de Yavé. \footnote{\textbf{8:11}
  Éxod 40,34-35}

\bibleverse{12} Entonces Salomón dijo: ``Yahvé ha dicho que habitará en
la espesa oscuridad. \footnote{\textbf{8:12} Deut 4,11; Éxod 20,21; 2Cró
  6,1-40} \bibleverse{13} Ciertamente te he construido una casa de
habitación, un lugar para que habites para siempre''.

\hypertarget{discurso-de-ordenaciuxf3n-y-consagraciuxf3n-de-salomuxf3n-al-pueblo}{%
\subsection{Discurso de ordenación y consagración de Salomón al
pueblo}\label{discurso-de-ordenaciuxf3n-y-consagraciuxf3n-de-salomuxf3n-al-pueblo}}

\bibleverse{14} El rey volvió su rostro y bendijo a toda la asamblea de
Israel; y toda la asamblea de Israel se puso de pie. \bibleverse{15}
Dijo: ``Bendito sea Yavé, el Dios de Israel, que habló con su boca a
David, tu padre, y con su mano lo ha cumplido, diciendo: \bibleverse{16}
`Desde el día en que saqué a mi pueblo Israel de Egipto, no elegí
ninguna ciudad de todas las tribus de Israel para edificar una casa,
para que mi nombre estuviera allí; pero elegí a David para que estuviera
sobre mi pueblo Israel.'

\bibleverse{17} ``El corazón de mi padre era construir una casa para el
nombre de Yahvé, el Dios de Israel. \footnote{\textbf{8:17} 2Sam 7,1}
\bibleverse{18} Pero Yahvé dijo a David, mi padre: ``Ya que tenías en tu
corazón construir una casa a mi nombre, hiciste bien en tenerlo.
\bibleverse{19} Sin embargo, no construirás la casa, sino que tu hijo,
que saldrá de tu cuerpo, construirá la casa a mi nombre'.
\bibleverse{20} Yahvé ha cumplido su palabra que había pronunciado;
porque yo me he levantado en lugar de David mi padre, y me he sentado en
el trono de Israel, como Yahvé había prometido, y he edificado la casa
para el nombre de Yahvé, el Dios de Israel. \bibleverse{21} Allí he
puesto un lugar para el arca, en la que está la alianza de Yavé, que
hizo con nuestros padres cuando los sacó de la tierra de Egipto.''

\hypertarget{oraciuxf3n-de-consagraciuxf3n-de-salomuxf3n}{%
\subsection{Oración de consagración de
Salomón}\label{oraciuxf3n-de-consagraciuxf3n-de-salomuxf3n}}

\bibleverse{22} Salomón se puso de pie ante el altar de Yavé, en
presencia de toda la asamblea de Israel, y extendió sus manos hacia el
cielo; \bibleverse{23} y dijo: ``Yavé, Dios de Israel, no hay Dios como
tú, ni en los cielos de arriba ni en la tierra de abajo; que guardas el
pacto y la bondad amorosa con tus siervos que caminan ante ti de todo
corazón; \bibleverse{24} que has cumplido con tu siervo David, mi padre,
lo que le prometiste. Sí, tú hablaste con tu boca, y lo has cumplido con
tu mano, como sucede hoy. \bibleverse{25} Ahora, pues, que Yahvé, el
Dios de Israel, guarde con tu siervo David, mi padre, lo que le
prometiste, diciendo: ``No faltará de ti un hombre que se siente en el
trono de Israel, con tal que tus hijos cuiden su camino, para andar
delante de mí como tú has andado delante de mí''.

\bibleverse{26} ``Ahora, pues, Dios de Israel, te ruego que se cumpla tu
palabra, que hablaste a tu siervo David, mi padre. \bibleverse{27} Pero,
¿acaso Dios va a habitar en la tierra? He aquí que el cielo y el cielo
de los cielos no pueden contenerte; ¡cuánto menos esta casa que he
construido! \footnote{\textbf{8:27} Is 66,1; Hech 7,49; Hech 17,24}
\bibleverse{28} Sin embargo, respeta la oración de tu siervo y su
súplica, Yahvé, mi Dios, para escuchar el clamor y la oración que tu
siervo hace hoy ante ti; \bibleverse{29} para que tus ojos estén
abiertos hacia esta casa de noche y de día, hacia el lugar del que has
dicho: `Mi nombre estará allí'; para escuchar la oración que tu siervo
hace hacia este lugar. \footnote{\textbf{8:29} Zac 12,4; Éxod 20,24;
  Deut 12,5; Deut 12,11} \bibleverse{30} Escucha la súplica de tu siervo
y de tu pueblo Israel, cuando oren hacia este lugar. Sí, escucha en el
cielo, tu morada; y cuando oigas, perdona.

\bibleverse{31} ``Si un hombre peca contra su prójimo, y se le impone un
juramento para que jure, y viene y jura ante tu altar en esta casa,
\bibleverse{32} entonces escucha en el cielo, y actúa, y juzga a tus
siervos, condenando al impío, para hacer recaer su camino sobre su
propia cabeza, y justificando al justo, para darle según su justicia.

\bibleverse{33} ``Cuando tu pueblo Israel sea abatido ante el enemigo
por haber pecado contra ti, si se vuelve a ti y confiesa tu nombre, y
ora y te suplica en esta casa, \bibleverse{34} entonces escucha en el
cielo, y perdona el pecado de tu pueblo Israel, y hazlo volver a la
tierra que diste a sus padres.

\bibleverse{35} ``Cuando el cielo se cierra y no hay lluvia porque han
pecado contra ti, si oran hacia este lugar y confiesan tu nombre, y se
convierten de su pecado cuando los afliges, \footnote{\textbf{8:35} 1Re
  17,1} \bibleverse{36} entonces escucha en el cielo, y perdona el
pecado de tus siervos, y de tu pueblo Israel, cuando les enseñas el buen
camino por el que deben andar; y envía la lluvia sobre tu tierra que has
dado a tu pueblo como herencia.

\bibleverse{37} ``Si hay hambre en la tierra, si hay peste, si hay
tizón, moho, langosta u oruga si su enemigo los asedia en la tierra de
sus ciudades, cualquier plaga, cualquier enfermedad que haya,
\bibleverse{38} cualquier oración y súplica que haga cualquier hombre, o
todo tu pueblo Israel, que conozca cada uno la plaga de su propio
corazón, y extienda sus manos hacia esta casa, \bibleverse{39} entonces
escucha en el cielo, tu morada, y perdona, y actúa, y da a cada uno
según todos sus caminos, cuyo corazón conoces (porque tú, sólo tú,
conoces los corazones de todos los hijos de los hombres); \footnote{\textbf{8:39}
  Sal 7,9; Sal 139,1-2} \bibleverse{40} para que te teman todos los días
que vivan en la tierra que diste a nuestros padres.

\bibleverse{41} ``Además, en cuanto al extranjero, que no es de tu
pueblo Israel, cuando venga de un país lejano por causa de tu nombre
\footnote{\textbf{8:41} Núm 15,14-16} \bibleverse{42} (porque oirán
hablar de tu gran nombre y de tu mano poderosa y de tu brazo extendido),
cuando venga y ore hacia esta casa, \bibleverse{43} escucha en el cielo,
tu morada, y haz conforme a todo lo que el extranjero te pida; para que
todos los pueblos de la tierra conozcan tu nombre, para que te teman,
como tu pueblo Israel, y para que sepan que esta casa que he edificado
se llama con tu nombre.

\bibleverse{44} ``Si tu pueblo sale a combatir contra su enemigo, por
cualquier camino que lo envíes, y ora a Yahvé hacia la ciudad que tú has
elegido y hacia la casa que he edificado a tu nombre, \bibleverse{45}
entonces escucha en el cielo su oración y su súplica, y mantén su causa.
\bibleverse{46} Si pecan contra ti (pues no hay hombre que no peque), y
te enojas con ellos y los entregas al enemigo, de modo que los llevan
cautivos a la tierra del enemigo, lejos o cerca; \footnote{\textbf{8:46}
  Rom 3,23} \bibleverse{47} pero si se arrepienten en la tierra donde
son llevados cautivos, y se vuelven, y te suplican en la tierra de los
que los llevaron cautivos, diciendo: `Hemos pecado y hemos hecho
perversamente hemos actuado con maldad,' \footnote{\textbf{8:47} Dan 9,5}
\bibleverse{48} si se vuelven a ti con todo su corazón y con toda su
alma en la tierra de sus enemigos que los llevaron cautivos, y te ruegan
hacia su tierra que diste a sus padres, la ciudad que tú has elegido y
la casa que yo he edificado a tu nombre, \bibleverse{49} entonces
escucha su oración y su súplica en el cielo, tu morada, y defiende su
causa; \bibleverse{50} y perdona a tu pueblo que ha pecado contra ti, y
todas sus transgresiones en las que se ha rebelado contra ti; y dales
compasión ante los que los llevaron cautivos, para que se compadezcan de
ellos \bibleverse{51} (porque son tu pueblo y tu herencia, que sacaste
de Egipto, de en medio del horno de hierro); \bibleverse{52} para que
tus ojos estén abiertos a la súplica de tu siervo y a la súplica de tu
pueblo Israel, para escucharlos siempre que clamen a ti. \bibleverse{53}
Porque tú los separaste de entre todos los pueblos de la tierra para que
fueran tu herencia, como hablaste por medio de Moisés, tu siervo, cuando
sacaste a nuestros padres de Egipto, Señor Yahvé''.

\hypertarget{palabras-finales-de-advertencia-y-bendiciuxf3n-de-salomuxf3n}{%
\subsection{Palabras finales de advertencia y bendición de
Salomón}\label{palabras-finales-de-advertencia-y-bendiciuxf3n-de-salomuxf3n}}

\bibleverse{54} Fue así que, cuando Salomón terminó de rezar toda esta
oración y súplica a Yavé, se levantó de delante del altar de Yavé, de
rodillas y con las manos extendidas hacia el cielo. \bibleverse{55} Se
puso de pie y bendijo a toda la asamblea de Israel en voz alta,
diciendo: \footnote{\textbf{8:55} 2Sam 6,18} \bibleverse{56} ``Bendito
sea Yavé, que ha dado descanso a su pueblo Israel, según todo lo que
había prometido. No ha faltado ni una palabra de toda su buena promesa,
que prometió por medio de Moisés, su siervo. \footnote{\textbf{8:56} Jos
  21,45} \bibleverse{57} Que el Señor, nuestro Dios, esté con nosotros
como estuvo con nuestros padres. Que no nos deje ni nos abandone,
\bibleverse{58} que incline nuestros corazones hacia él, para que
andemos en todos sus caminos y guardemos sus mandamientos, sus estatutos
y sus ordenanzas, que mandó a nuestros padres. \bibleverse{59} Que estas
palabras mías, con las que he suplicado ante Yavé, estén cerca de Yavé,
nuestro Dios, de día y de noche, para que mantenga la causa de su siervo
y la de su pueblo Israel, como cada día lo requiere; \bibleverse{60}
para que todos los pueblos de la tierra sepan que Yavé mismo es Dios. No
hay otro.

\bibleverse{61} ``Sea, pues, perfecto vuestro corazón con Yahvé nuestro
Dios, para andar en sus estatutos y guardar sus mandamientos, como
hoy.''

\hypertarget{las-fiestas-se-concluyen-con-una-gran-fiesta-de-sacrificios}{%
\subsection{Las fiestas se concluyen con una gran fiesta de
sacrificios}\label{las-fiestas-se-concluyen-con-una-gran-fiesta-de-sacrificios}}

\bibleverse{62} El rey, y todo Israel con él, ofrecieron sacrificios
ante Yavé. \footnote{\textbf{8:62} 2Cró 7,4-10} \bibleverse{63} Salomón
ofreció para el sacrificio de las ofrendas de paz, que ofreció a Yavé,
veintidós mil cabezas de ganado y ciento veinte mil ovejas. Así el rey y
todos los hijos de Israel dedicaron la casa de Yavé. \bibleverse{64} Ese
mismo día el rey santificó el centro del atrio que estaba frente a la
casa de Yavé, pues allí ofreció el holocausto, el presente y la grasa de
los sacrificios de paz, porque el altar de bronce que estaba frente a
Yavé era demasiado pequeño para recibir el holocausto, el presente y la
grasa de los sacrificios de paz.

\bibleverse{65} Salomón celebró entonces la fiesta, y todo Israel con
él, una gran asamblea, desde la entrada de Hamat hasta el arroyo de
Egipto, en presencia de Yahvé nuestro Dios, durante siete días y siete
días más, es decir, catorce días. \bibleverse{66} Al octavo día despidió
al pueblo, que bendijo al rey y se fue a sus tiendas alegres y contentos
de corazón por toda la bondad que Yahvé había mostrado a su siervo David
y a su pueblo Israel.

\hypertarget{la-reapariciuxf3n-de-dios-y-la-respuesta-a-la-oraciuxf3n-de-salomuxf3n}{%
\subsection{La reaparición de Dios y la respuesta a la oración de
Salomón}\label{la-reapariciuxf3n-de-dios-y-la-respuesta-a-la-oraciuxf3n-de-salomuxf3n}}

\hypertarget{section-8}{%
\section{9}\label{section-8}}

\bibleverse{1} Cuando Salomón terminó de construir la casa de Yavé, la
casa del rey y todo lo que Salomón deseaba hacer, \bibleverse{2} Yavé se
le apareció a Salomón por segunda vez, como se le había aparecido en
Gabaón. \footnote{\textbf{9:2} 1Re 3,5} \bibleverse{3} Yahvé le dijo:
``He escuchado tu oración y tu súplica que has hecho ante mí. He
santificado esta casa que has edificado, para poner allí mi nombre para
siempre; y mis ojos y mi corazón estarán allí perpetuamente. \footnote{\textbf{9:3}
  1Re 8,29} \bibleverse{4} En cuanto a ti, si andas delante de mí como
anduvo David tu padre, con integridad de corazón y rectitud, para hacer
conforme a todo lo que te he mandado, y guardas mis estatutos y mis
ordenanzas, \bibleverse{5} entonces yo estableceré el trono de tu reino
sobre Israel para siempre, como se lo prometí a David tu padre,
diciendo: `No faltará de ti un hombre en el trono de Israel'.
\footnote{\textbf{9:5} 2Sam 7,12} \bibleverse{6} Pero si te apartas de
seguirme, tú o tus hijos, y no guardas mis mandamientos y mis estatutos
que he puesto delante de ti, sino que vas y sirves a otros dioses y los
adoras, \bibleverse{7} entonces cortaré a Israel de la tierra que les he
dado; y echaré de mi vista esta casa que he santificado para mi nombre,
e Israel será un proverbio y una palabra de orden entre todos los
pueblos. \footnote{\textbf{9:7} Deut 4,26; Deut 8,19-20; Mat 23,38}
\bibleverse{8} Aunque esta casa es tan alta, todos los que pasen por
ella se asombrarán y silbarán; y dirán: ``¿Por qué ha hecho esto Yahvé a
esta tierra y a esta casa?'' \bibleverse{9} Y responderán: ``Porque
abandonaron a Yahvé, su Dios, que sacó a sus padres de la tierra de
Egipto, y abrazaron a otros dioses, los adoraron y les sirvieron. Por
eso Yahvé ha traído sobre ellos todo este mal'\,''.

\hypertarget{asignaciuxf3n-de-tierras-a-hiram-a-cambio}{%
\subsection{Asignación de tierras a Hiram a
cambio}\label{asignaciuxf3n-de-tierras-a-hiram-a-cambio}}

\bibleverse{10} Al cabo de veinte años, en los que Salomón había
construido las dos casas, la de Yahvé y la del rey \footnote{\textbf{9:10}
  1Re 6,38; 1Re 7,1} \bibleverse{11} (ya que Hiram, rey de Tiro, había
provisto a Salomón de cedros y cipreses, y de oro, según su deseo), el
rey Salomón le dio a Hiram veinte ciudades en la tierra de Galilea.
\bibleverse{12} Hiram salió de Tiro para ver las ciudades que Salomón le
había dado, y no le gustaron. \bibleverse{13} Dijo: ``¿Qué ciudades son
éstas que me has dado, hermano mío?'' Las llamó la tierra de Cabul hasta
el día de hoy. \bibleverse{14} Hiram envió al rey ciento veinte talentos
de oro.

\hypertarget{de-los-trabajadores-de-salomuxf3n-fortificaciones-ciudades-de-almacenamiento-sacrificios-regulares-etc.}{%
\subsection{De los trabajadores de Salomón, fortificaciones, ciudades de
almacenamiento, sacrificios regulares,
etc.}\label{de-los-trabajadores-de-salomuxf3n-fortificaciones-ciudades-de-almacenamiento-sacrificios-regulares-etc.}}

\bibleverse{15} Esta es la razón de los trabajos forzados que el rey
Salomón reclutó: para construir la casa de Yavé, su propia casa, Milo,
la muralla de Jerusalén, Hazor, Meguido y Gezer. \bibleverse{16} El
faraón, rey de Egipto, había subido, tomado Gezer, la había quemado con
fuego, había matado a los cananeos que vivían en la ciudad y se la había
dado como regalo de bodas a su hija, la esposa de Salomón. \footnote{\textbf{9:16}
  Jos 16,10; 1Re 3,1} \bibleverse{17} Salomón edificó en la tierra
Gezer, Bet Horón el inferior, \bibleverse{18} Baalat, Tamar en el
desierto, \bibleverse{19} todas las ciudades de almacenamiento que tenía
Salomón, las ciudades para sus carros, las ciudades para su caballería,
y lo que Salomón deseaba edificar para su placer en Jerusalén, en el
Líbano y en toda la tierra de su dominio. \footnote{\textbf{9:19} 1Re
  10,26} \bibleverse{20} En cuanto a todos los pueblos que quedaron de
los amorreos, los hititas, los ferezeos, los heveos y los jebuseos, que
no eran de los hijos de Israel --- \bibleverse{21} sus hijos que
quedaron después de ellos en la tierra, a quienes los hijos de Israel no
pudieron destruir del todo --- de ellos Salomón levantó una leva de
siervos hasta el día de hoy. \footnote{\textbf{9:21} Jos 16,10}
\bibleverse{22} Pero de los hijos de Israel Salomón no hizo siervos,
sino que fueron los hombres de guerra, sus siervos, sus príncipes, sus
capitanes y los jefes de sus carros y de su caballería. \bibleverse{23}
Estos eran los quinientos cincuenta oficiales principales que estaban al
frente de la obra de Salomón, que gobernaban al pueblo que trabajaba en
la obra.

\bibleverse{24} Pero la hija del faraón subió de la ciudad de David a su
casa que Salomón había construido para ella. Entonces construyó Millo.

\bibleverse{25} Salomón ofrecía holocaustos y ofrendas de paz en el
altar que construyó a Yavé tres veces al año, quemando con ellos
incienso en el altar que estaba delante de Yavé. Así terminó la casa.

\hypertarget{construcciuxf3n-y-envuxedo-de-la-flota-de-salomuxf3n}{%
\subsection{Construcción y envío de la flota de
Salomón}\label{construcciuxf3n-y-envuxedo-de-la-flota-de-salomuxf3n}}

\bibleverse{26} El rey Salomón hizo una flota de barcos en Ezión Geber,
que está junto a Elot, a orillas del Mar Rojo, en la tierra de Edom.
\bibleverse{27} Hiram envió en la flota a sus siervos, marineros
conocedores del mar, con los siervos de Salomón. \footnote{\textbf{9:27}
  1Re 10,11} \bibleverse{28} Llegaron a Ofir y sacaron de allí oro,
cuatrocientos veinte talentos, y se lo llevaron al rey Salomón.
\footnote{\textbf{9:28} Gén 10,29}

\hypertarget{visita-de-la-reina-de-saba}{%
\subsection{Visita de la Reina de
Saba}\label{visita-de-la-reina-de-saba}}

\hypertarget{section-9}{%
\section{10}\label{section-9}}

\bibleverse{1} Cuando la reina de Sabá se enteró de la fama de Salomón
en cuanto al nombre de Yavé, vino a probarlo con preguntas difíciles.
\footnote{\textbf{10:1} Mat 12,42} \bibleverse{2} Llegó a Jerusalén con
una caravana muy grande, con camellos que llevaban especias, mucho oro y
piedras preciosas; y cuando llegó a Salomón, habló con él de todo lo que
tenía en su corazón. \bibleverse{3} Salomón respondió a todas sus
preguntas. No hubo nada que se le ocultara al rey que no le dijera.
\bibleverse{4} Cuando la reina de Sabá vio toda la sabiduría de Salomón,
la casa que había construido, \bibleverse{5} la comida de su mesa, la
asistencia de sus sirvientes, la asistencia de sus funcionarios, sus
ropas, sus coperos, y su ascenso por el cual subía a la casa de Yavé, no
hubo más espíritu en ella. \bibleverse{6} Ella le dijo al rey: ``Fue una
noticia verdadera la que oí en mi tierra acerca de tus actos y de tu
sabiduría. \bibleverse{7} Sin embargo, no creí las palabras hasta que
llegué y mis ojos lo vieron. He aquí que no se me dijo ni la mitad. Tu
sabiduría y tu prosperidad superan la fama que he oído. \bibleverse{8}
Felices son tus hombres, felices son estos tus siervos que están
continuamente ante ti, que escuchan tu sabiduría. \footnote{\textbf{10:8}
  Luc 10,23} \bibleverse{9} Bendito sea el Señor, tu Dios, que se
complació en ti para ponerte en el trono de Israel. Porque Yahvé amó a
Israel para siempre, por eso te hizo rey, para que hicieras justicia y
rectitud.'' \footnote{\textbf{10:9} 1Re 5,7} \bibleverse{10} Ella le dio
al rey ciento veinte talentos de oro, una gran cantidad de especias y
piedras preciosas. Nunca más hubo tanta abundancia de especias como las
que la reina de Sabá dio al rey Salomón.

\bibleverse{11} La flota de Hiram que traía oro de Ofir también trajo de
Ofir grandes cantidades de almugas y piedras preciosas. \footnote{\textbf{10:11}
  1Re 9,27-28} \bibleverse{12} El rey hizo de los almugares pilares para
la casa de Yahvé y para la casa del rey, también arpas e instrumentos de
cuerda para los cantores; no vinieron ni se vieron hasta hoy tales
almugares.

\bibleverse{13} El rey Salomón dio a la reina de Sabá todo lo que
deseaba, todo lo que pedía, además de lo que Salomón le daba de su
generosidad real. Entonces se volvió y se fue a su tierra, ella y sus
sirvientes.

\hypertarget{informaciuxf3n-diversa-sobre-los-ingresos-los-objetos-de-valor-la-reputaciuxf3n-y-el-poder-de-salomuxf3n-el-comercio-de-caballos}{%
\subsection{Información diversa sobre los ingresos, los objetos de
valor, la reputación y el poder de Salomón, el comercio de
caballos}\label{informaciuxf3n-diversa-sobre-los-ingresos-los-objetos-de-valor-la-reputaciuxf3n-y-el-poder-de-salomuxf3n-el-comercio-de-caballos}}

\bibleverse{14} El peso del oro que llegó a Salomón en un año fue de
seiscientos sesenta y seis talentosde oro, \bibleverse{15} además de lo
que traían los comerciantes, y el tráfico de los mercaderes, y de todos
los reyes de los pueblos mixtos, y de los gobernadores del país.
\bibleverse{16} El rey Salomón hizo doscientos escudos de oro batido;
seiscientos siclos\footnote{\textbf{10:16} Un siclo equivale a unos 10
  gramos o a unas 0,32 onzas troy, por lo que 600 siclos son unos 6
  kilogramos o 13,2 libras o 192 onzas troy.} de oro fueron para un
escudo. \footnote{\textbf{10:16} 1Re 14,26} \bibleverse{17} Hizo
trescientos escudos de oro batido; tres minas\footnote{\textbf{10:17}
  Una mina equivale a unos 600 gramos o 1,3 libras estadounidenses.} de
oro fueron para un escudo; y el rey los puso en la Casa del Bosque del
Líbano. \bibleverse{18} Además, el rey hizo un gran trono de marfil y lo
cubrió con el mejor oro. \bibleverse{19} El trono tenía seis peldaños, y
la parte superior del trono era redonda por detrás; y había apoyos para
los brazos a ambos lados del lugar del asiento, y dos leones de pie
junto a los apoyos para los brazos. \bibleverse{20} Doce leones estaban
de pie a un lado y al otro en los seis escalones. No se hizo nada
parecido en ningún reino. \bibleverse{21} Todos los vasos del rey
Salomón eran de oro, y todos los vasos de la Casa del Bosque del Líbano
eran de oro puro. Ninguno era de plata, porque se consideraba de poco
valor en los días de Salomón. \bibleverse{22} Porque el rey tenía una
flota de barcos de Tarsis en el mar con la flota de Hiram. Una vez cada
tres años la flota de Tarsis venía trayendo oro, plata, marfil, monos y
pavos reales.

\bibleverse{23} Así, el rey Salomón superó a todos los reyes de la
tierra en riquezas y en sabiduría. \bibleverse{24} Toda la tierra
buscaba la presencia de Salomón para escuchar su sabiduría que Dios
había puesto en su corazón. \bibleverse{25} Año tras año, cada hombre
traía su tributo, vasos de plata, vasos de oro, ropa, armaduras,
especias, caballos y mulas.

\bibleverse{26} Salomón reunió carros y jinetes. Tenía mil cuatrocientos
carros y doce mil jinetes. Los mantuvo en las ciudades de los carros y
con el rey en Jerusalén. \footnote{\textbf{10:26} 1Re 4,26; 2Cró 1,14-17}
\bibleverse{27} El rey hizo que la plata fuera tan común como las
piedras en Jerusalén, y los cedros tan comunes como los sicómoros que
hay en la llanura. \bibleverse{28} Los caballos que tenía Salomón fueron
traídos de Egipto. Los mercaderes del rey los recibieron en tropel, cada
uno de ellos conducido a un precio. \bibleverse{29} Un carro fue
importado de Egipto por seiscientos siclos\footnote{\textbf{10:29} Un
  siclo equivale a unos 10 gramos o a unas 0,35 onzas.} de plata, y un
caballo por ciento cincuenta siclos; y así los exportaron a todos los
reyes de los hititas y a los reyes de Siria.

\hypertarget{la-poligamia-y-la-idolatruxeda-de-salomuxf3n-la-amenaza-de-dios}{%
\subsection{La poligamia y la idolatría de Salomón; La amenaza de
dios}\label{la-poligamia-y-la-idolatruxeda-de-salomuxf3n-la-amenaza-de-dios}}

\hypertarget{section-10}{%
\section{11}\label{section-10}}

\bibleverse{1} El rey Salomón amó a muchas mujeres extranjeras, junto
con la hija del faraón: mujeres moabitas, amonitas, edomitas, sidonianas
e hititas, \footnote{\textbf{11:1} Deut 17,17} \bibleverse{2} de las
naciones sobre las que Yahvé dijo a los hijos de Israel: ``No iréis
entre ellas, ni ellas vendrán entre vosotros, porque ciertamente
desviarán vuestro corazón tras sus dioses.'' Salomón se unió a ellas por
amor. \footnote{\textbf{11:2} Éxod 34,16} \bibleverse{3} Tuvo
setecientas esposas, princesas y trescientas concubinas. Sus esposas
desviaron su corazón. \bibleverse{4} Cuando Salomón envejeció, sus
esposas desviaron su corazón en pos de otros dioses, y su corazón no era
perfecto con Yavé, su Dios, como lo era el corazón de David, su padre.
\bibleverse{5} Porque Salomón siguió a Astoret, diosa de los sidonios, y
a Milcom, la abominación de los amonitas. \bibleverse{6} Salomón hizo lo
que era malo a los ojos de Yavé, y no siguió plenamente a Yavé, como lo
hizo su padre David. \bibleverse{7} Entonces Salomón edificó un lugar
alto para Quemos, la abominación de Moab, en el monte que está frente a
Jerusalén, y para Moloc, la abominación de los hijos de Amón.
\footnote{\textbf{11:7} Núm 21,29; 2Re 23,13} \bibleverse{8} Así hizo
para todas sus mujeres extranjeras, que quemaban incienso y sacrificaban
a sus dioses. \bibleverse{9} Yahvé se enojó con Salomón, porque su
corazón se apartó de Yahvé, el Dios de Israel, que se le había aparecido
dos veces, \footnote{\textbf{11:9} 1Re 3,5; 1Re 9,2} \bibleverse{10} y
le había ordenado respecto a esto, que no fuera en pos de otros dioses;
pero él no cumplió lo que Yahvé le había ordenado. \bibleverse{11} Por
lo tanto, Yahvé dijo a Salomón: ``Por cuanto has hecho esto, y no has
guardado mi pacto y mis estatutos que te he ordenado, ciertamente te
arrancaré el reino y se lo daré a tu siervo. \footnote{\textbf{11:11}
  1Sam 15,28} \bibleverse{12} Sin embargo, no lo haré en tus días, por
amor a David tu padre, sino que lo arrancaré de la mano de tu hijo.
\footnote{\textbf{11:12} 1Re 12,19} \bibleverse{13} Sin embargo, no
arrancaré todo el reino, sino que daré una tribu a tu hijo, por amor a
David, mi siervo, y por amor a Jerusalén, que yo he elegido.''

\hypertarget{los-enemigos-externos-de-salomuxf3n-el-edomita-hadad-y-el-sirio-reson}{%
\subsection{Los enemigos externos de Salomón (el edomita Hadad y el
sirio
Reson)}\label{los-enemigos-externos-de-salomuxf3n-el-edomita-hadad-y-el-sirio-reson}}

\bibleverse{14} El Señor le levantó un adversario a Salomón: Hadad el
edomita. Era uno de los descendientes del rey en Edom. \bibleverse{15}
Porque cuando David estaba en Edom, y Joab, el capitán del ejército,
había subido a enterrar a los muertos y había matado a todos los varones
de Edom \footnote{\textbf{11:15} 2Sam 8,14} \bibleverse{16} (pues Joab y
todo Israel permanecieron allí seis meses, hasta que hubo matado a todos
los varones de Edom), \bibleverse{17} Hadad huyó, él y algunos edomitas
de los siervos de su padre con él, para ir a Egipto, cuando Hadad era
todavía un niño. \bibleverse{18} Se levantaron de Madián y llegaron a
Parán; tomaron hombres con ellos de Parán y llegaron a Egipto, al
Faraón, rey de Egipto, quien le dio una casa, le asignó alimentos y le
dio tierras. \bibleverse{19} Hadad halló gran favor a los ojos del
faraón, de modo que le dio como esposa a la hermana de su propia esposa,
la hermana de la reina Tahpenes. \bibleverse{20} La hermana de Tahpenes
le dio a luz a su hijo Genubat, a quien Tahpenes destetó en la casa del
faraón; y Genubat estaba en la casa del faraón entre los hijos del
faraón. \bibleverse{21} Cuando Hadad oyó en Egipto que David dormía con
sus padres y que Joab, el capitán del ejército, había muerto, dijo a
Faraón: ``Déjame ir a mi país.''

\bibleverse{22} Entonces el Faraón le dijo: ``¿Pero qué te ha faltado a
ti conmigo, que buscas ir a tu país?'' Él respondió: ``Nada, sin
embargo, sólo déjame partir''.

\bibleverse{23} Dios le levantó un adversario, Rezón hijo de Eliada, que
había huido de su señor, Hadadézer, rey de Soba. \bibleverse{24} Él
reunió hombres para sí, y llegó a ser capitán de una tropa, cuando David
los mató de Soba. Se fue a Damasco y vivió allí, y reinó en Damasco.
\footnote{\textbf{11:24} 2Sam 8,3; 2Sam 10,18} \bibleverse{25} Fue un
adversario de Israel durante todos los días de Salomón, además de la
maldad de Hadad. Aborreció a Israel y reinó sobre Siria.

\hypertarget{la-indignaciuxf3n-del-efraimita-jeroboam}{%
\subsection{La indignación del efraimita
Jeroboam}\label{la-indignaciuxf3n-del-efraimita-jeroboam}}

\bibleverse{26} Jeroboam hijo de Nabat, efraimita de Zereda, siervo de
Salomón, cuya madre se llamaba Zerúa, viuda, también levantó su mano
contra el rey. \bibleverse{27} Esta fue la razón por la que levantó su
mano contra el rey: Salomón edificó Millo y reparó la brecha de la
ciudad de su padre David. \footnote{\textbf{11:27} 1Re 9,15; 1Re 9,24}
\bibleverse{28} Aquel hombre, Jeroboam, era un hombre de gran valor, y
Salomón vio que el joven era laborioso, y lo puso a cargo de todo el
trabajo de la casa de José. \bibleverse{29} En aquel tiempo, cuando
Jeroboam salió de Jerusalén, el profeta Ahías, el silonita, lo encontró
en el camino. Ajías se había vestido con un traje nuevo, y los dos
estaban solos en el campo. \bibleverse{30} Ajías tomó el vestido nuevo
que llevaba puesto y lo rompió en doce pedazos. \bibleverse{31} Le dijo
a Jeroboam: ``Toma diez pedazos porque Yahvé, el Dios de Israel, dice:
`He aquí que yo arranco el reino de la mano de Salomón y te daré diez
tribus \footnote{\textbf{11:31} 1Re 12,15; 1Re 14,2} \bibleverse{32}
(pero él tendrá una tribu, por amor a mi siervo David y por amor a
Jerusalén, la ciudad que he elegido de entre todas las tribus de
Israel), \bibleverse{33} porque me han abandonado y han adorado a
Astoret, diosa de los sidonios, a Quemos, dios de Moab, y a Milcom, dios
de los hijos de Amón. No han andado en mis caminos, para hacer lo que es
recto a mis ojos, y para guardar mis estatutos y mis ordenanzas, como
hizo David su padre.

\bibleverse{34} ``\,`Sin embargo, no quitaré todo el reino de su mano,
sino que lo haré príncipe todos los días de su vida por amor a David, mi
siervo, a quien elegí, quien guardó mis mandamientos y mis estatutos,
\footnote{\textbf{11:34} 2Sam 7,12} \bibleverse{35} pero quitaré el
reino de la mano de su hijo y te lo daré a ti, diez tribus. \footnote{\textbf{11:35}
  1Re 12,16} \bibleverse{36} Le daré una tribu a su hijo, para que mi
siervo David tenga siempre una lámpara delante de mí en Jerusalén, la
ciudad que he elegido para poner mi nombre en ella. \bibleverse{37} Yo
te tomaré a ti, y tú reinarás según todo lo que tu alma desee, y serás
rey sobre Israel. \bibleverse{38} Si escuchas todo lo que te mando, y
andas en mis caminos, y haces lo que es recto a mis ojos, guardando mis
estatutos y mis mandamientos, como lo hizo mi siervo David, yo estaré
contigo, y te edificaré una casa segura, como la que edifiqué para
David, y te entregaré a Israel. \footnote{\textbf{11:38} 1Re 9,4}
\bibleverse{39} Por esto afligiré a la descendencia de David, pero no
para siempre'\,''.

\bibleverse{40} Por eso Salomón trató de matar a Jeroboam, pero éste se
levantó y huyó a Egipto, a Sisac, rey de Egipto, y estuvo en Egipto
hasta la muerte de Salomón. \footnote{\textbf{11:40} 1Re 14,25}

\hypertarget{las-fuentes-de-la-historia-de-salomuxf3n-su-muerte}{%
\subsection{Las fuentes de la historia de Salomón; su
muerte}\label{las-fuentes-de-la-historia-de-salomuxf3n-su-muerte}}

\bibleverse{41} Los demás hechos de Salomón, y todo lo que hizo, y su
sabiduría, ¿no están escritos en el libro de los hechos de Salomón?
\bibleverse{42} El tiempo que Salomón reinó en Jerusalén sobre todo
Israel fue de cuarenta años. \bibleverse{43} Salomón durmió con sus
padres, y fue enterrado en la ciudad de su padre David; y reinó en su
lugar Roboam, su hijo.

\hypertarget{la-divisiuxf3n-del-imperio-la-dieta-de-siquem-pide-ayuda-a-los-israelitas}{%
\subsection{La división del imperio; La dieta de Siquem; Pide ayuda a
los
israelitas}\label{la-divisiuxf3n-del-imperio-la-dieta-de-siquem-pide-ayuda-a-los-israelitas}}

\hypertarget{section-11}{%
\section{12}\label{section-11}}

\bibleverse{1} Roboam fue a Siquem, porque todo Israel había acudido a
Siquem para hacerle rey. \footnote{\textbf{12:1} 2Cró 10,1}
\bibleverse{2} Cuando Jeroboam hijo de Nabat se enteró de ello (pues aún
estaba en Egipto, donde había huido de la presencia del rey Salomón, y
Jeroboam vivía en Egipto; \footnote{\textbf{12:2} 1Re 11,40}
\bibleverse{3} y enviaron a llamarlo), Jeroboam y toda la asamblea de
Israel vinieron y hablaron con Roboam, diciendo: \bibleverse{4} ``Tu
padre hizo difícil nuestro yugo. Ahora, pues, aligera el duro servicio
de tu padre y el pesado yugo que nos impuso, y te serviremos.''

\bibleverse{5} Les dijo: ``Vayan por tres días y luego vuelvan a mí''.
Así que la gente se marchó.

\hypertarget{consejeruxeda-de-rehoboams}{%
\subsection{Consejería de Rehoboams}\label{consejeruxeda-de-rehoboams}}

\bibleverse{6} El rey Roboam se asesoró con los ancianos que habían
estado delante de Salomón, su padre, cuando aún vivía, diciendo: ``¿Qué
consejo me dais para responder a esta gente?'' \footnote{\textbf{12:6}
  Prov 12,5}

\bibleverse{7} Ellos respondieron: ``Si hoy eres un siervo para este
pueblo y le sirves y le respondes con buenas palabras, entonces ellos
serán tus siervos para siempre.''

\bibleverse{8} Pero abandonó el consejo de los ancianos que le habían
dado, y tomó consejo con los jóvenes que habían crecido con él, que
estaban delante de él. \bibleverse{9} Les dijo: ``¿Qué consejo les dais
para que respondamos a esta gente que me ha hablado diciendo: ``Aligerad
el yugo que vuestro padre nos puso''?''

\bibleverse{10} Los jóvenes que se habían criado con él le dijeron:
``Dile a esa gente que te ha hablado diciendo: ``Tu padre ha hecho
pesado nuestro yugo, pero aligéralo para nosotros''; diles: ``Mi dedo
meñique es más grueso que la cintura de mi padre. \bibleverse{11} Mi
padre os cargó con un yugo pesado, pero yo añadiré a vuestro yugo. Mi
padre os castigó con látigos, pero yo os castigaré con escorpiones'\,''.

\hypertarget{descenso-de-las-diez-tribus-elecciuxf3n-de-jeroboam-como-rey-de-israel}{%
\subsection{Descenso de las diez tribus; Elección de Jeroboam como rey
de
Israel}\label{descenso-de-las-diez-tribus-elecciuxf3n-de-jeroboam-como-rey-de-israel}}

\bibleverse{12} Entonces Jeroboam y todo el pueblo vinieron a Roboam al
tercer día, tal como el rey lo había pedido, diciendo: ``Volved a mí al
tercer día''. \footnote{\textbf{12:12} 1Re 12,5} \bibleverse{13} El rey
respondió al pueblo con aspereza, y abandonó el consejo de los ancianos
que le habían dado, \bibleverse{14} y les habló según el consejo de los
jóvenes, diciendo: ``Mi padre hizo pesado vuestro yugo, pero yo añadiré
a vuestro yugo. Mi padre os castigó con látigos, pero yo os castigaré
con escorpiones''.

\bibleverse{15} El rey, pues, no escuchó al pueblo, porque era una cosa
traída de Yahvé, para confirmar su palabra, que Yahvé habló por medio de
Ahías el silonita a Jeroboam hijo de Nabat. \footnote{\textbf{12:15} 1Re
  11,31} \bibleverse{16} Cuando todo Israel vio que el rey no los
escuchaba, el pueblo respondió al rey diciendo: ``¿Qué parte tenemos en
David? No tenemos herencia en el hijo de Isaí. ¡A tus tiendas, Israel!
Ahora ocúpate de tu propia casa, David''. Así que Israel se fue a sus
tiendas. \footnote{\textbf{12:16} Prov 15,1; 2Sam 20,1}

\bibleverse{17} Pero en cuanto a los hijos de Israel que vivían en las
ciudades de Judá, Roboam reinó sobre ellos. \bibleverse{18} Entonces el
rey Roboam envió a Adoram, que estaba a cargo de los hombres sometidos a
trabajos forzados, y todo Israel lo mató a pedradas. El rey Roboam se
apresuró a subir a su carro, para huir a Jerusalén. \bibleverse{19} Así
se rebeló Israel contra la casa de David hasta el día de hoy.

\bibleverse{20} Cuando todo Israel se enteró de que Jeroboam había
regresado, enviaron a llamarlo a la congregación y lo hicieron rey de
todo Israel. No hubo nadie que siguiera a la casa de David, sino sólo la
tribu de Judá.

\hypertarget{roboam-se-abstiene-de-la-guerra-contra-israel-bajo-la-direcciuxf3n-de-dios}{%
\subsection{Roboam se abstiene de la guerra contra Israel bajo la
dirección de
Dios}\label{roboam-se-abstiene-de-la-guerra-contra-israel-bajo-la-direcciuxf3n-de-dios}}

\bibleverse{21} Cuando Roboam llegó a Jerusalén, reunió a toda la casa
de Judá y a la tribu de Benjamín, ciento ochenta mil hombres escogidos
que eran guerreros, para luchar contra la casa de Israel, a fin de
devolver el reino a Roboam hijo de Salomón. \footnote{\textbf{12:21}
  2Cró 11,1-4} \bibleverse{22} Pero vino la palabra de Dios a Semaías,
hombre de Dios, diciendo: \bibleverse{23} ``Habla a Roboam hijo de
Salomón, rey de Judá, y a toda la casa de Judá y de Benjamín, y al resto
del pueblo, diciendo: \bibleverse{24} `Dice el Señor: ``No subiréis ni
lucharéis contra vuestros hermanos, los hijos de Israel. Volved cada uno
a su casa, porque esto viene de mí''\,''. Así que escucharon la palabra
de Yahvé, y volvieron y se fueron, según la palabra de Yahvé.

\hypertarget{jeroboam-i.-gobierno-e-idolatruxeda}{%
\subsection{Jeroboam I. Gobierno e
idolatría}\label{jeroboam-i.-gobierno-e-idolatruxeda}}

\bibleverse{25} Entonces Jeroboam edificó Siquem en la región montañosa
de Efraín, y vivió en ella; luego salió de allí y edificó Penuel.
\footnote{\textbf{12:25} Gén 32,30} \bibleverse{26} Jeroboam decía en su
corazón: ``Ahora el reino volverá a la casa de David. \bibleverse{27} Si
este pueblo sube a ofrecer sacrificios en la casa de Yavé en Jerusalén,
entonces el corazón de este pueblo se volverá a su señor, a Roboam, rey
de Judá, y me matarán y volverán con Roboam, rey de Judá.''

\hypertarget{la-introducciuxf3n-del-servicio-de-toros-en-betel-y-dan}{%
\subsection{La introducción del servicio de toros en Betel y
Dan}\label{la-introducciuxf3n-del-servicio-de-toros-en-betel-y-dan}}

\bibleverse{28} Entonces el rey tomó consejo e hizo dos becerros de oro,
y les dijo: ``Es demasiado para ustedes subir a Jerusalén. Mirad y ved
vuestros dioses, Israel, que os hicieron subir de la tierra de Egipto''.
\footnote{\textbf{12:28} Éxod 32,4; Éxod 32,8} \bibleverse{29} Puso el
uno en Betel, y el otro lo puso en Dan. \bibleverse{30} Esto se
convirtió en un pecado, pues el pueblo llegó hasta Dan para adorar ante
el de allí. \footnote{\textbf{12:30} 1Re 14,16} \bibleverse{31} Hizo
casas de altos, e hizo sacerdotes de entre todo el pueblo, que no eran
de los hijos de Leví. \bibleverse{32} Jeroboam ordenó una fiesta en el
octavo mes, a los quince días del mes, como la fiesta que hay en Judá, y
subió al altar. Lo hizo en Betel, sacrificando a los becerros que había
hecho, y colocó en Betel a los sacerdotes de los lugares altos que había
hecho.

\hypertarget{amenaza-de-un-profeta-contra-el-altar-de-betel-desobediencia-y-muerte-de-este-profeta}{%
\subsection{Amenaza de un profeta contra el altar de Betel;
Desobediencia y muerte de este
profeta}\label{amenaza-de-un-profeta-contra-el-altar-de-betel-desobediencia-y-muerte-de-este-profeta}}

\bibleverse{33} Subió al altar que había hecho en Betel el día quince
del mes octavo, el mes que había ideado de su propio corazón, e
instituyó una fiesta para los hijos de Israel, y subió al altar a quemar
incienso.

\hypertarget{section-12}{%
\section{13}\label{section-12}}

\bibleverse{1} He aquí que un hombre de Dios vino de Judá, por palabra
de Yavé, a Betel; y Jeroboam estaba junto al altar para quemar incienso.
\bibleverse{2} El gritó contra el altar por palabra de Yavé, y dijo:
``¡Altar! ¡Altar! Yahvé dice: `He aquí que va a nacer un hijo en la casa
de David, cuyo nombre es Josías. Sobre ti sacrificará a los sacerdotes
de los lugares altos que queman incienso, y sobre ti quemarán huesos de
hombres'\,''. \footnote{\textbf{13:2} 2Re 23,16} \bibleverse{3} Ese
mismo día dio una señal, diciendo: ``Esta es la señal que ha dicho el
Señor: He aquí que el altar se partirá, y las cenizas que están sobre él
se derramarán.''

\bibleverse{4} Cuando el rey oyó la palabra del hombre de Dios, que
clamaba contra el altar de Betel, Jeroboam extendió su mano desde el
altar, diciendo: ``¡Agárrenlo!'' La mano que extendió contra él se secó,
de modo que no pudo volver a atraerla hacia sí. \bibleverse{5} El altar
también se partió, y las cenizas se derramaron del altar, según la señal
que el hombre de Dios había dado por palabra de Yavé. \bibleverse{6} El
rey respondió al hombre de Dios: ``Intercede ahora por el favor de Yavé,
tu Dios, y ruega por mí, para que mi mano me sea devuelta.'' El hombre
de Dios intercedió ante Yahvé, y la mano del rey le fue devuelta de
nuevo, y volvió a ser como antes. \footnote{\textbf{13:6} Éxod 8,8; Éxod
  8,12}

\bibleverse{7} El rey dijo al hombre de Dios: ``Ven conmigo a casa y
refréscate, y te daré una recompensa''.

\bibleverse{8} El hombre de Dios dijo al rey: ``Aunque me dieras la
mitad de tu casa, no entraría contigo, ni comería pan ni bebería agua en
este lugar; \footnote{\textbf{13:8} Núm 22,18} \bibleverse{9} porque así
me lo ha ordenado la palabra de Yahvé, diciendo: `No comerás pan, ni
beberás agua, y no volverás por el camino que viniste'\,''.
\bibleverse{10} Así que se fue por otro camino, y no volvió por el
camino por el que había venido a Betel.

\hypertarget{el-proceso-en-la-casa-del-anciano-profeta-en-betel}{%
\subsection{El proceso en la casa del anciano profeta en
Betel}\label{el-proceso-en-la-casa-del-anciano-profeta-en-betel}}

\bibleverse{11} Un viejo profeta vivía en Betel, y uno de sus hijos vino
a contarle todas las obras que el hombre de Dios había hecho aquel día
en Betel. También le contaron a su padre las palabras que había dicho al
rey.

\bibleverse{12} Su padre les dijo: ``¿Por qué camino se fue?'' Sus hijos
habían visto por dónde iba el hombre de Dios, que venía de Judá.
\bibleverse{13} Dijo a sus hijos: ``Ensilladme el asno''. Así que le
ensillaron el asno, y se montó en él. \bibleverse{14} Fue tras el hombre
de Dios y lo encontró sentado bajo una encina. Le dijo: ``¿Eres tú el
hombre de Dios que vino de Judá?''. Él dijo: ``Yo soy''.

\bibleverse{15} Entonces le dijo: ``Ven conmigo a casa y come pan''.

\bibleverse{16} Dijo: ``No puedo volver con vosotros ni entrar con
vosotros. No comeré pan ni beberé agua contigo en este lugar.
\bibleverse{17} Porque se me ha dicho por palabra de Yahvé: `No comerás
pan ni beberás agua allí, y no te vuelvas a ir por el camino por el que
viniste'.'' \footnote{\textbf{13:17} 1Re 13,9}

\bibleverse{18} Él le dijo: ``Yo también soy profeta como tú, y un ángel
me habló por palabra de Yahvé, diciendo: ``Tráelo contigo a tu casa,
para que coma pan y beba agua''\,''. Le mintió.

\bibleverse{19} Así que volvió con él, comió pan en su casa y bebió
agua.

\hypertarget{la-muerte-y-el-entierro-del-profeta-desobediente}{%
\subsection{La muerte y el entierro del profeta
desobediente}\label{la-muerte-y-el-entierro-del-profeta-desobediente}}

\bibleverse{20} Mientras estaban sentados a la mesa, llegó la palabra de
Yavé al profeta que lo trajo de vuelta \bibleverse{21} y gritó al hombre
de Dios que venía de Judá, diciendo: ``Dice Yahvé: `Por haber sido
desobediente a la palabra de Yahvé, y no haber cumplido el mandamiento
que Yahvé tu Dios te había ordenado, \bibleverse{22} sino que volviste,
y has comido pan y bebido agua en el lugar del que te dijo: ``No comas
pan ni bebas agua'', tu cuerpo no llegará a la tumba de tus padres.'\,''

\bibleverse{23} Después de comer el pan y de beber, ensilló el asno para
el profeta que había traído. \bibleverse{24} Cuando se fue, un león lo
encontró en el camino y lo mató. Su cuerpo fue arrojado al camino, y el
asno se quedó junto a él. El león también se quedó junto al cuerpo.
\footnote{\textbf{13:24} 1Re 20,36} \bibleverse{25} Pasaron unos hombres
que vieron el cuerpo tirado en el camino y al león junto al cadáver, y
vinieron a contarlo en la ciudad donde vivía el viejo profeta.
\bibleverse{26} Cuando el profeta que lo trajo de vuelta del camino se
enteró, dijo: ``Es el hombre de Dios que fue desobediente a la palabra
de Yavé. Por eso Yahvé lo ha entregado al león, que lo ha mutilado y lo
ha matado, según la palabra de Yahvé que le había dicho.''
\bibleverse{27} Dijo a sus hijos: ``Ensilladme el asno'', y lo
ensillaron. \bibleverse{28} Fue y encontró su cuerpo tirado en el
camino, y al burro y al león de pie junto al cuerpo. El león no se había
comido el cuerpo ni había mutilado al asno. \bibleverse{29} El profeta
tomó el cuerpo del hombre de Dios, lo puso sobre el asno y lo trajo de
vuelta. Llegó a la ciudad del viejo profeta para hacer el duelo y
enterrarlo. \bibleverse{30} Puso su cuerpo en su propia tumba, y lo
lloraron diciendo: ``¡Ay, hermano mío!'' \footnote{\textbf{13:30} Jer
  22,18}

\bibleverse{31} Después de enterrarlo, habló a sus hijos diciendo:
``Cuando haya muerto, enterradme en la tumba en la que está enterrado el
hombre de Dios. Pongan mis huesos junto a los suyos. \bibleverse{32}
Porque ciertamente se cumplirá lo que gritó por palabra de Yavé contra
el altar de Betel y contra todas las casas de los lugares altos que hay
en las ciudades de Samaria.''

\hypertarget{jeroboam-continuxfaa-su-actividad-pecaminosa}{%
\subsection{Jeroboam continúa su actividad
pecaminosa}\label{jeroboam-continuxfaa-su-actividad-pecaminosa}}

\bibleverse{33} Después de esto, Jeroboam no se apartó de su mal camino,
sino que volvió a hacer sacerdotes de los lugares altos de entre todo el
pueblo. Al que quería, lo consagraba, para que hubiera sacerdotes de los
lugares altos. \footnote{\textbf{13:33} 1Re 12,31; Éxod 28,41}
\bibleverse{34} Esto se convirtió en pecado para la casa de Jeroboam,
hasta cortarla y destruirla de la superficie de la tierra. \footnote{\textbf{13:34}
  1Re 12,30}

\hypertarget{amenaza-de-castigo-del-profeta-ahuxedas-muerte-de-jeroboam}{%
\subsection{Amenaza de castigo del profeta Ahías; Muerte de
Jeroboam}\label{amenaza-de-castigo-del-profeta-ahuxedas-muerte-de-jeroboam}}

\hypertarget{section-13}{%
\section{14}\label{section-13}}

\bibleverse{1} En aquel tiempo Abías, hijo de Jeroboam, enfermó.
\bibleverse{2} Jeroboam le dijo a su esposa: ``Por favor, levántate y
disfrázate para que no te reconozcan como la esposa de Jeroboam. Ve a
Silo. Allí está el profeta Ahías, que dijo que yo sería rey de este
pueblo. \footnote{\textbf{14:2} 1Re 11,31} \bibleverse{3} Toma contigo
diez panes, algunas tortas y un tarro de miel, y ve a él. Él te dirá qué
será del niño''.

\bibleverse{4} La mujer de Jeroboam lo hizo, y se levantó y fue a Silo,
y llegó a la casa de Ahías. Ajías no podía ver, pues tenía los ojos
entornados a causa de su edad. \bibleverse{5} El Señor le dijo a Ajías:
``Mira, la mujer de Jeroboam viene a preguntarte por su hijo, porque
está enfermo. Dile que tal y tal; porque será, cuando venga, que se hará
pasar por otra mujer.''

\hypertarget{discurso-de-castigo-y-amenaza-de-ahuxedas-contra-jeroboam}{%
\subsection{Discurso de castigo y amenaza de Ahías contra
Jeroboam}\label{discurso-de-castigo-y-amenaza-de-ahuxedas-contra-jeroboam}}

\bibleverse{6} Cuando Ahías oyó el ruido de sus pies al entrar por la
puerta, dijo: ``¡Entra, mujer de Jeroboam! ¿Por qué te haces pasar por
otra? Porque he sido enviado a ti con noticias pesadas. \bibleverse{7}
Ve y dile a Jeroboam: ``Yahvé, el Dios de Israel, dice ``Porque te
exalté de entre el pueblo y te hice príncipe de mi pueblo Israel,
\footnote{\textbf{14:7} 1Re 11,37; 1Re 16,2} \bibleverse{8} y arranqué
el reino de la casa de David y te lo di a ti y sin embargo no has sido
como mi siervo David, que guardó mis mandamientos y me siguió con todo
su corazón, para hacer sólo lo que era justo a mis ojos, \bibleverse{9}
sino que has hecho lo malo por encima de todos los que fueron antes de
ti, y has ido a hacerte otros dioses, imágenes de fundición, para
provocarme a la ira y me has echado a tus espaldas, \bibleverse{10} por
tanto, he aquí que yo traeré el mal sobre la casa de Jeroboam, y cortaré
de Jeroboam a todo el que orine en una pared,\footnote{\textbf{14:10} o,
  masculino} al que esté encerrado y al que quede suelto en Israel, y
barreré totalmente la casa de Jeroboam, como se barre el estiércol hasta
que desaparezca todo. \footnote{\textbf{14:10} 1Re 15,29; 1Re 21,21}
\bibleverse{11} Los perros se comerán al de Jeroboam que muera en la
ciudad, y las aves del cielo se comerán al que muera en el campo, porque
Yahvé lo ha dicho''. \footnote{\textbf{14:11} 1Re 16,4; 1Re 21,24}
\bibleverse{12} Levántate, pues, y vete a tu casa. Cuando tus pies
entren en la ciudad, el niño morirá. \bibleverse{13} Todo Israel lo
llorará y lo enterrará, porque sólo el de Jeroboam llegará a la tumba,
porque en él se ha encontrado algo bueno para Yahvé, el Dios de Israel,
en la casa de Jeroboam. \bibleverse{14} Además, Yahvé suscitará para sí
un rey sobre Israel, que eliminará la casa de Jeroboam. ¡Este es el día!
¿Qué? Ahora mismo. \footnote{\textbf{14:14} 1Re 15,29} \bibleverse{15}
Porque Yahvé golpeará a Israel, como se agita una caña en el agua; y
desarraigará a Israel de esta buena tierra que dio a sus padres, y los
dispersará más allá del río,\footnote{\textbf{14:15} Es decir, el
  Éufrates.} porque han hecho sus postes de Asera, provocando la ira de
Yahvé. \footnote{\textbf{14:15} 2Re 17,23} \bibleverse{16} Entregará a
Israel a causa de los pecados de Jeroboam, que ha cometido y con los que
ha hecho pecar a Israel.'' \footnote{\textbf{14:16} 1Re 12,30; 1Re 13,34}

\hypertarget{cumplimiento-de-la-profecuxeda-palabra-final}{%
\subsection{Cumplimiento de la profecía; Palabra
final}\label{cumplimiento-de-la-profecuxeda-palabra-final}}

\bibleverse{17} La mujer de Jeroboam se levantó y partió, y llegó a
Tirsa. Al llegar al umbral de la casa, el niño murió. \bibleverse{18}
Todo Israel lo enterró y lo lloró, según la palabra de Yavé, que habló
por medio de su siervo el profeta Ahías.

\bibleverse{19} Los demás hechos de Jeroboam, cómo luchó y cómo reinó,
he aquí que están escritos en el libro de las crónicas de los reyes de
Israel. \bibleverse{20} Los días que reinó Jeroboam fueron veintidós
años; luego durmió con sus padres, y reinó en su lugar Nadab, su hijo.
\footnote{\textbf{14:20} 1Re 15,25}

\hypertarget{la-idolatruxeda-de-juduxe1-bajo-roboam}{%
\subsection{La idolatría de Judá bajo
Roboam}\label{la-idolatruxeda-de-juduxe1-bajo-roboam}}

\bibleverse{21} Roboam, hijo de Salomón, reinó en Judá. Roboam tenía
cuarenta y un años cuando comenzó a reinar, y reinó diecisiete años en
Jerusalén, la ciudad que el Señor había elegido de entre todas las
tribus de Israel para poner su nombre en ella. Su madre se llamaba
Naamah la amonita. \footnote{\textbf{14:21} 1Re 12,17} \bibleverse{22}
Judá hizo lo que era malo a los ojos de Yavé, y lo provocaron a celos
con los pecados que cometieron, más allá de todo lo que habían hecho sus
padres. \bibleverse{23} Porque también se construyeron lugares altos,
pilares sagrados y postes de Asera en todo cerro alto y debajo de todo
árbol verde. \footnote{\textbf{14:23} 2Re 16,4} \bibleverse{24} También
había sodomitas en la tierra. Hicieron según todas las abominaciones de
las naciones que Yahvé expulsó ante los hijos de Israel. \footnote{\textbf{14:24}
  Deut 23,17}

\hypertarget{incursiuxf3n-y-saqueo-del-rey-egipcio-sisak-palabra-final}{%
\subsection{Incursión y saqueo del rey egipcio Sisak; Palabra
final}\label{incursiuxf3n-y-saqueo-del-rey-egipcio-sisak-palabra-final}}

\bibleverse{25} En el quinto año del rey Roboam, Sisac, rey de Egipto,
subió contra Jerusalén; \footnote{\textbf{14:25} 1Re 11,40}
\bibleverse{26} y se llevó los tesoros de la casa de Yahvé y los tesoros
de la casa real. Incluso se llevó todo, incluyendo todos los escudos de
oro que Salomón había hecho. \footnote{\textbf{14:26} 1Re 10,16}
\bibleverse{27} El rey Roboam hizo escudos de bronce en su lugar, y los
encomendó a los capitanes de la guardia que guardaban la puerta de la
casa del rey. \bibleverse{28} Y cada vez que el rey entraba en la casa
del rey, la guardia los llevaba y los traía a la sala de guardia.

\bibleverse{29} Los demás hechos de Roboam, y todo lo que hizo, ¿no
están escritos en el libro de las crónicas de los reyes de Judá?
\bibleverse{30} Hubo guerra entre Roboam y Jeroboam continuamente.
\footnote{\textbf{14:30} 1Re 15,6} \bibleverse{31} Roboam durmió con sus
padres y fue enterrado con ellos en la ciudad de David. Su madre se
llamaba Naamah la amonita. Su hijo Abijam reinó en su lugar. \footnote{\textbf{14:31}
  1Re 14,21}

\hypertarget{gobierno-del-rey-abia-de-juduxe1}{%
\subsection{Gobierno del rey Abia de
Judá}\label{gobierno-del-rey-abia-de-juduxe1}}

\hypertarget{section-14}{%
\section{15}\label{section-14}}

\bibleverse{1} En el año dieciocho del rey Jeroboam hijo de Nabat,
Abiyam comenzó a reinar sobre Judá. \bibleverse{2} Reinó tres años en
Jerusalén. Su madre se llamaba Maaca, hija de Abisalón. \bibleverse{3}
Anduvo en todos los pecados de su padre, que había hecho antes de él, y
su corazón no era perfecto con Yahvé su Dios, como el corazón de David
su padre. \bibleverse{4} Sin embargo, por causa de David, Yahvé su Dios
le dio una lámpara en Jerusalén, para que pusiera a su hijo después de
él y para que estableciera a Jerusalén; \footnote{\textbf{15:4} 1Re
  11,36} \bibleverse{5} porque David hizo lo que era justo a los ojos de
Yahvé, y no se apartó de nada de lo que le mandó en todos los días de su
vida, excepto solamente en el asunto de Urías el hitita. \footnote{\textbf{15:5}
  2Sam 11,27; 2Sam 12,9} \bibleverse{6} Hubo guerra entre Roboam y
Jeroboam todos los días de su vida. \footnote{\textbf{15:6} 1Re 14,30}
\bibleverse{7} Los demás hechos de Abijam, y todo lo que hizo, ¿no están
escritos en el libro de las crónicas de los reyes de Judá? Hubo guerra
entre Abijam y Jeroboam. \bibleverse{8} Abijam durmió con sus padres, y
lo enterraron en la ciudad de David; y su hijo Asa reinó en su lugar.

\hypertarget{la-intervenciuxf3n-de-asa-contra-la-idolatruxeda}{%
\subsection{La intervención de Asa contra la
idolatría}\label{la-intervenciuxf3n-de-asa-contra-la-idolatruxeda}}

\bibleverse{9} En el vigésimo año de Jeroboam, rey de Israel, Asa
comenzó a reinar sobre Judá. \bibleverse{10} Reinó cuarenta y un años en
Jerusalén. Su madre se llamaba Maaca, hija de Abisalón. \footnote{\textbf{15:10}
  1Re 15,2} \bibleverse{11} Asá hizo lo que era justo a los ojos de
Yavé, como lo hizo su padre David. \footnote{\textbf{15:11} 2Cró 14,2-5;
  2Cró 15,16-18} \bibleverse{12} Expulsó a los sodomitas del país y
eliminó todos los ídolos que habían hecho sus padres. \footnote{\textbf{15:12}
  1Re 14,24; 1Re 22,46} \bibleverse{13} También destituyó a su madre
Maacá como reina, porque había hecho una imagen abominable como Asera.
Asa cortó su imagen y la quemó en el arroyo Cedrón. \bibleverse{14} Pero
los lugares altos no fueron quitados. Sin embargo, el corazón de Asa fue
perfecto con Yavé todos sus días. \footnote{\textbf{15:14} 1Re 22,43}
\bibleverse{15} Llevó a la casa de Yavé las cosas que su padre había
dedicado, y las que él mismo había dedicado: plata, oro y utensilios.

\hypertarget{la-guerra-de-asa-con-el-rey-baesa-de-israel-palabra-final}{%
\subsection{La guerra de Asa con el rey Baesa de Israel; Palabra
final}\label{la-guerra-de-asa-con-el-rey-baesa-de-israel-palabra-final}}

\bibleverse{16} Hubo guerra entre Asa y Baasa, rey de Israel, durante
todos sus días. \footnote{\textbf{15:16} 2Cró 16,1-8; 2Cró 15,11-14}
\bibleverse{17} Baasa, rey de Israel, subió contra Judá y edificó Ramá,
para no permitir que nadie saliera ni entrara a Asa, rey de Judá.
\bibleverse{18} Entonces Asa tomó toda la plata y el oro que quedaba en
los tesoros de la casa de Yavé y en los tesoros de la casa del rey, y lo
entregó en manos de sus servidores. Entonces el rey Asá los envió a Ben
Hadad, hijo de Tabrimón, hijo de Hezión, rey de Siria, que vivía en
Damasco, diciendo: \footnote{\textbf{15:18} 2Re 12,18; 2Re 16,8}
\bibleverse{19} ``Que haya un tratado entre tú y yo, como el que hubo
entre mi padre y tu padre. He aquí que te he enviado un presente de
plata y oro. Ve, rompe tu tratado con Baasa, rey de Israel, para que se
aparte de mí''.

\bibleverse{20} Ben Hadad escuchó al rey Asá y envió a los capitanes de
sus ejércitos contra las ciudades de Israel, y atacó a Ijón, a Dan, a
Abel Bet Maaca y a toda Cinerot, con toda la tierra de Neftalí.
\footnote{\textbf{15:20} 2Re 15,29} \bibleverse{21} Cuando Baasa se
enteró de esto, dejó de construir Rama y vivió en Tirsa. \bibleverse{22}
Entonces el rey Asá hizo una proclama a todo Judá. Nadie quedó exento.
Se llevaron las piedras de Rama y su madera, con las que Baasa había
construido; y el rey Asa las utilizó para construir Geba de Benjamín y
Mizpa. \bibleverse{23} El resto de todos los hechos de Asa, y todo su
poderío, y todo lo que hizo, y las ciudades que edificó, ¿no están
escritos en el libro de las crónicas de los reyes de Judá? Pero en el
tiempo de su vejez enfermó de los pies. \footnote{\textbf{15:23} 2Cró
  14,6} \bibleverse{24} Asa durmió con sus padres y fue sepultado con
ellos en la ciudad de su padre David; y su hijo Josafat reinó en su
lugar. \footnote{\textbf{15:24} 1Re 22,41}

\hypertarget{gobierno-del-rey-nadab-de-israel-su-cauxedda-por-baesa}{%
\subsection{Gobierno del rey Nadab de Israel; su caída por
Baesa}\label{gobierno-del-rey-nadab-de-israel-su-cauxedda-por-baesa}}

\bibleverse{25} Nadab hijo de Jeroboam comenzó a reinar sobre Israel en
el segundo año de Asa, rey de Judá; y reinó sobre Israel dos años.
\footnote{\textbf{15:25} 1Re 14,20} \bibleverse{26} Hizo lo que era malo
a los ojos de Yahvé, y anduvo en el camino de su padre, y en su pecado
con que hizo pecar a Israel. \footnote{\textbf{15:26} 1Re 12,30}
\bibleverse{27} Baasa, hijo de Ajías, de la casa de Isacar, conspiró
contra él; y Baasa lo hirió en Gibbetón, que era de los filisteos, pues
Nadab y todo Israel estaban sitiando Gibbetón. \footnote{\textbf{15:27}
  1Re 16,9} \bibleverse{28} En el tercer año de Asa, rey de Judá, Baasa
lo mató y reinó en su lugar. \bibleverse{29} Tan pronto como fue rey,
golpeó a toda la casa de Jeroboam. No dejó a Jeroboam ni un solo
aliento, hasta que lo destruyó, según la palabra de Yavé, que habló por
medio de su siervo Ahías, el silonita; \footnote{\textbf{15:29} 1Re
  14,10-11} \bibleverse{30} por los pecados de Jeroboam que cometió y
con los que hizo pecar a Israel, a causa de su provocación con la que
hizo enojar a Yavé, el Dios de Israel.

\bibleverse{31} El resto de los hechos de Nadab y todo lo que hizo, ¿no
está escrito en el libro de las crónicas de los reyes de Israel?
\bibleverse{32} Hubo guerra entre Asa y Baasa, rey de Israel, durante
todos sus días. \footnote{\textbf{15:32} 1Re 15,16}

\hypertarget{gobierno-del-rey-baesa-de-israel}{%
\subsection{Gobierno del Rey Baesa de
Israel}\label{gobierno-del-rey-baesa-de-israel}}

\bibleverse{33} En el tercer año de Asá, rey de Judá, Baasa hijo de
Ahías comenzó a reinar sobre todo Israel en Tirsa durante veinticuatro
años. \footnote{\textbf{15:33} 1Re 15,28} \bibleverse{34} Hizo lo que
era malo a los ojos de Yavé, y anduvo en el camino de Jeroboam, y en su
pecado con que hizo pecar a Israel. \footnote{\textbf{15:34} 1Re 15,26}

\hypertarget{la-amenaza-de-juicio-del-profeta-jehuxfa-contra-baesa}{%
\subsection{La amenaza de juicio del profeta Jehú contra
Baesa}\label{la-amenaza-de-juicio-del-profeta-jehuxfa-contra-baesa}}

\hypertarget{section-15}{%
\section{16}\label{section-15}}

\bibleverse{1} La palabra de Yavé vino a Jehú hijo de Hanani contra
Baasa, diciendo: \footnote{\textbf{16:1} 1Re 16,7} \bibleverse{2} ``Por
cuanto te exalté del polvo y te hice príncipe de mi pueblo Israel, y has
andado en el camino de Jeroboam y has hecho pecar a mi pueblo Israel,
para provocarme a la ira con sus pecados, \footnote{\textbf{16:2} 1Re
  14,7} \bibleverse{3} he aquí que yo barreré por completo a Baasa y a
su casa; y pondré tu casa como la casa de Jeroboam hijo de Nabat.
\footnote{\textbf{16:3} 1Re 15,29} \bibleverse{4} Los perros se comerán
a los descendientes de Baasa que mueran en la ciudad; y al que muera de
los suyos en el campo, se lo comerán las aves del cielo.'' \footnote{\textbf{16:4}
  1Re 14,11}

\bibleverse{5} El resto de los hechos de Baasa, lo que hizo y su
poderío, ¿no están escritos en el libro de las crónicas de los reyes de
Israel? \bibleverse{6} Baasa durmió con sus padres y fue enterrado en
Tirsa, y su hijo Ela reinó en su lugar.

\bibleverse{7} Además, la palabra de Yahvé vino por medio del profeta
Jehú, hijo de Hanani, contra Baasa y contra su casa, tanto por todo el
mal que hizo ante los ojos de Yahvé, para provocarlo a la ira con la
obra de sus manos, al ser como la casa de Jeroboam, como porque lo
golpeó. \footnote{\textbf{16:7} 1Re 16,1}

\hypertarget{gobierno-del-rey-ela-de-israel}{%
\subsection{Gobierno del rey Ela de
Israel}\label{gobierno-del-rey-ela-de-israel}}

\bibleverse{8} En el año veintiséis de Asa, rey de Judá, Ela hijo de
Baasa comenzó a reinar sobre Israel en Tirsa durante dos años.
\footnote{\textbf{16:8} 1Re 16,6} \bibleverse{9} Su siervo Zimri,
capitán de la mitad de sus carros, conspiró contra él. Él estaba en
Tirsa, emborrachándose en la casa de Arza, que estaba al frente de la
casa en Tirsa; \footnote{\textbf{16:9} 1Re 15,27} \bibleverse{10} y
Zimri entró, lo golpeó y lo mató en el año veintisiete de Asá, rey de
Judá, y reinó en su lugar. \footnote{\textbf{16:10} 2Re 9,31; 2Re 15,10;
  2Re 15,14; 2Re 15,25; 2Re 15,30}

\bibleverse{11} Cuando comenzó a reinar, apenas se sentó en su trono,
atacó a toda la casa de Baasa. No le dejó ni un solo que orinara en una
pared\footnote{\textbf{16:11} o, masculino} entre sus parientes o sus
amigos. \bibleverse{12} Así destruyó Zimri toda la casa de Baasa, según
la palabra de Yavé que habló contra Baasa por medio del profeta Jehú,
\footnote{\textbf{16:12} 1Re 16,1-4} \bibleverse{13} por todos los
pecados de Baasa y los pecados de Elá, su hijo, que cometieron y con los
que hicieron pecar a Israel, para provocar la ira de Yavé, el Dios de
Israel, con sus vanidades. \bibleverse{14} El resto de los hechos de Elá
y todo lo que hizo, ¿no están escritos en el libro de las crónicas de
los reyes de Israel?

\hypertarget{gobierno-del-rey-zimri-de-israel}{%
\subsection{Gobierno del rey Zimri de
Israel}\label{gobierno-del-rey-zimri-de-israel}}

\bibleverse{15} En el año veintisiete de Asa, rey de Judá, Zimri reinó
siete días en Tirsa. El pueblo estaba acampado frente a Gibbetón, que
pertenecía a los filisteos. \footnote{\textbf{16:15} 1Re 15,27}
\bibleverse{16} El pueblo que estaba acampado oyó que Zimri había
conspirado y que también había matado al rey. Por eso todo Israel nombró
aquel día en el campamento a Omri, capitán del ejército, como rey de
Israel. \footnote{\textbf{16:16} 1Re 16,9-10} \bibleverse{17} Omri subió
desde Gibbetón, y todo Israel con él, y sitiaron Tirsa. \bibleverse{18}
Cuando Zimri vio que la ciudad estaba tomada, entró en la parte
fortificada de la casa del rey y quemó la casa del rey sobre él con
fuego, y murió, \bibleverse{19} por sus pecados que cometió al hacer lo
que era malo a los ojos de Yavé, al andar en el camino de Jeroboam, y
por su pecado que hizo para hacer pecar a Israel. \bibleverse{20} El
resto de los hechos de Zimri y la traición que cometió, ¿no están
escritos en el libro de las crónicas de los reyes de Israel?

\hypertarget{divisiuxf3n-del-reino-de-israel-regla-uxfanica-omri}{%
\subsection{División del reino de Israel; Regla única
Omri}\label{divisiuxf3n-del-reino-de-israel-regla-uxfanica-omri}}

\bibleverse{21} Entonces el pueblo de Israel se dividió en dos partes:
la mitad del pueblo seguía a Tibni hijo de Ginat, para hacerlo rey, y la
otra mitad seguía a Omri. \bibleverse{22} Pero el pueblo que seguía a
Omri se impuso al pueblo que seguía a Tibni hijo de Ginat; así que Tibni
murió, y Omri reinó.

\hypertarget{gobierno-del-rey-omri-de-israel-fundaciuxf3n-de-la-capital-samaria}{%
\subsection{Gobierno del rey Omri de Israel; Fundación de la capital
Samaria}\label{gobierno-del-rey-omri-de-israel-fundaciuxf3n-de-la-capital-samaria}}

\bibleverse{23} En el año treinta y uno de Asá, rey de Judá, Omri
comenzó a reinar sobre Israel durante doce años. Reinó seis años en
Tirsa. \bibleverse{24} Compró la colina de Samaria a Semer por dos
talentos\footnote{\textbf{16:24} Un talento son unos 30 kilos o 66
  libras.} de plata; y edificó en la colina, y llamó el nombre de la
ciudad que edificó, Samaria, por el nombre de Semer, el dueño de la
colina. \bibleverse{25} Omri hizo lo que era malo a los ojos de Yahvé, y
actuó con maldad por encima de todos los que fueron antes de él.
\footnote{\textbf{16:25} Miq 6,16} \bibleverse{26} Porque anduvo en todo
el camino de Jeroboam hijo de Nabat, y en sus pecados con que hizo pecar
a Israel, para provocar la ira de Yavé, el Dios de Israel, con sus
vanidades. \footnote{\textbf{16:26} 1Re 12,30} \bibleverse{27} El resto
de los hechos de Omri que hizo, y su poderío que mostró, ¿no están
escritos en el libro de las crónicas de los reyes de Israel?
\bibleverse{28} Omri durmió con sus padres y fue enterrado en Samaria, y
su hijo Acab reinó en su lugar.

\hypertarget{los-pecados-del-rey-acab-de-israel-y-su-esposa-jezabel}{%
\subsection{Los pecados del rey Acab de Israel y su esposa
Jezabel}\label{los-pecados-del-rey-acab-de-israel-y-su-esposa-jezabel}}

\bibleverse{29} En el año treinta y ocho de Asá, rey de Judá, Ajab, hijo
de Omri, comenzó a reinar sobre Israel. Ajab hijo de Omri reinó sobre
Israel en Samaria veintidós años. \bibleverse{30} Ajab hijo de Omri hizo
lo que era malo a los ojos de Yavé por encima de todos los que lo
precedieron. \bibleverse{31} Como si le pareciera poco andar en los
pecados de Jeroboam hijo de Nabat, tomó por esposa a Jezabel, hija de
Etbaal, rey de los sidonios, y fue a servir a Baal y a adorarlo.
\footnote{\textbf{16:31} 1Re 16,26} \bibleverse{32} Levantó un altar
para Baal en la casa de Baal que había construido en Samaria.
\footnote{\textbf{16:32} 2Re 3,2; 2Re 10,27-28} \bibleverse{33} Ajab
hizo la Asera; y aún hizo Ajab más para provocar la ira de Yahvé, el
Dios de Israel, que todos los reyes de Israel que fueron antes de él.

\hypertarget{reconstrucciuxf3n-de-la-ciudad-de-jericuxf3}{%
\subsection{Reconstrucción de la ciudad de
Jericó}\label{reconstrucciuxf3n-de-la-ciudad-de-jericuxf3}}

\bibleverse{34} En sus días, Hiel el betelita construyó Jericó. Puso sus
cimientos con la pérdida de Abiram, su primogénito, y levantó sus
puertas con la pérdida de su hijo menor, Segub, según la palabra de
Yahvé, que habló por medio de Josué, hijo de Nun. \footnote{\textbf{16:34}
  Jos 6,26}

\hypertarget{eluxedas-delante-del-rey-acab-y-en-el-arroyo-krith}{%
\subsection{Elías delante del rey Acab y en el arroyo
Krith}\label{eluxedas-delante-del-rey-acab-y-en-el-arroyo-krith}}

\hypertarget{section-16}{%
\section{17}\label{section-16}}

\bibleverse{1} Elías tisbita, que era uno de los colonos de Galaad, dijo
a Ajab: ``Vive Yahvé, el Dios de Israel, ante quien estoy, que no habrá
rocío ni lluvia estos años, sino según mi palabra.'' \footnote{\textbf{17:1}
  Sant 5,17; Apoc 11,6}

\bibleverse{2} Entonces le llegó la palabra de Yahvé, diciendo:
\bibleverse{3} ``Vete de aquí, vuélvete hacia el este, y escóndete junto
al arroyo de Querit, que está frente al Jordán. \bibleverse{4} Beberás
del arroyo. He ordenado a los cuervos que te den de comer allí''.
\bibleverse{5} Fue, pues, y cumplió la palabra de Yavé, pues se fue a
vivir junto al arroyo de Querit que está frente al Jordán.
\bibleverse{6} Los cuervos le llevaban pan y carne por la mañana, y pan
y carne por la tarde; y él bebía del arroyo.

\hypertarget{el-milagro-de-eluxedas-con-la-viuda-en-sarepta-en-fenicia}{%
\subsection{El milagro de Elías con la viuda en Sarepta en
Fenicia}\label{el-milagro-de-eluxedas-con-la-viuda-en-sarepta-en-fenicia}}

\bibleverse{7} Al cabo de un tiempo, el arroyo se secó, porque no llovía
en la tierra.

\bibleverse{8} La palabra de Yahvé vino a él, diciendo: \bibleverse{9}
``Levántate, ve a Sarepta, que pertenece a Sidón, y quédate allí. He
aquí que he ordenado a una viuda de allí que te sostenga''. \footnote{\textbf{17:9}
  Luc 4,25-26}

\bibleverse{10} Se levantó, pues, y fue a Sarepta; y cuando llegó a la
puerta de la ciudad, he aquí que una viuda estaba allí recogiendo palos.
La llamó y le dijo: ``Por favor, tráeme un poco de agua en una jarra,
para que pueda beber''.

\bibleverse{11} Cuando iba a cogerlo, la llamó y le dijo: ``Por favor,
tráeme un bocado de pan en la mano''.

\bibleverse{12} Ella dijo: ``Vive Yahvé, tu Dios, que no tengo nada
cocido, sino sólo un puñado de harina en una vasija y un poco de aceite
en una jarra. Estoy juntando dos palos, para entrar a hornearlo para mí
y para mi hijo, para que lo comamos y muramos.'' \footnote{\textbf{17:12}
  1Re 18,10}

\bibleverse{13} Elías le dijo: ``No tengas miedo. Ve y haz lo que has
dicho; pero hazme primero una pequeña torta de ella, y tráemela, y
después hazla para ti y para tu hijo. \bibleverse{14} Porque Yahvé, el
Dios de Israel, dice: `La vasija de harina no se agotará, y la vasija de
aceite no se agotará, hasta el día en que Yahvé envíe la lluvia a la
tierra'.'' \footnote{\textbf{17:14} 2Re 4,2-4}

\bibleverse{15} Ella fue e hizo lo que dijo Elías; y ella, él y su
familia comieron durante muchos días. \bibleverse{16} La vasija de
harina no se agotó y la vasija de aceite no falló, según la palabra de
Yahvé, que habló por medio de Elías.

\hypertarget{la-reanimaciuxf3n-del-hijo-de-la-viuda}{%
\subsection{La reanimación del hijo de la
viuda}\label{la-reanimaciuxf3n-del-hijo-de-la-viuda}}

\bibleverse{17} Después de estas cosas, el hijo de la mujer, la dueña de
la casa, enfermó; y su enfermedad era tan grave que no le quedaba
aliento. \bibleverse{18} Ella le dijo a Elías: ``¿Qué tengo que hacer
contigo, hombre de Dios? Has venido a mí para traer a la memoria mi
pecado, y para matar a mi hijo''. \footnote{\textbf{17:18} Luc 5,8}

\bibleverse{19} Le dijo: ``Dame a tu hijo''. Lo sacó de su seno, lo
subió a la habitación donde se hospedaba y lo puso en su propia cama.
\bibleverse{20} Clamó a Yahvé y dijo: ``Yahvé, mi Dios, ¿también has
traído el mal a la viuda con la que me hospedo, matando a su hijo?''

\bibleverse{21} Se tendió sobre el niño tres veces y clamó a Yahvé
diciendo: ``Yahvé, Dios mío, por favor, haz que el alma de este niño
vuelva a entrar en él.'' \footnote{\textbf{17:21} 2Re 4,34; Hech 20,10}

\bibleverse{22} El Señor escuchó la voz de Elías, y el alma del niño
volvió a entrar en él, y revivió. \bibleverse{23} Elías tomó al niño y
lo bajó de la habitación a la casa, y lo entregó a su madre; y Elías
dijo: ``He aquí que tu hijo vive.'' \footnote{\textbf{17:23} Luc 7,15;
  Heb 11,35}

\bibleverse{24} La mujer dijo a Elías: ``Ahora sé que eres un hombre de
Dios y que la palabra de Yahvé en tu boca es verdad.''

\hypertarget{el-mandato-de-dios-a-eluxedas-el-encuentro-de-obadja-con-elijah}{%
\subsection{El mandato de Dios a Elías; El encuentro de Obadja con
Elijah}\label{el-mandato-de-dios-a-eluxedas-el-encuentro-de-obadja-con-elijah}}

\hypertarget{section-17}{%
\section{18}\label{section-17}}

\bibleverse{1} Después de muchos días, llegó la palabra de Yahvé a
Elías, en el tercer año, diciendo: ``Ve, muéstrate a Ajab, y yo enviaré
lluvia sobre la tierra.''

\bibleverse{2} Elías fue a mostrarse a Acab. La hambruna era grave en
Samaria. \bibleverse{3} Ajab llamó a Abdías, que estaba a cargo de la
casa. (Ahora bien, Abdías temía mucho a Yavé; \footnote{\textbf{18:3}
  1Re 18,12} \bibleverse{4} porque cuando Jezabel eliminó a los profetas
de Yavé, Abdías tomó a cien profetas y los escondió a cincuenta en una
cueva, y los alimentó con pan y agua). \bibleverse{5} Ajab le dijo a
Abdías: ``Recorre la tierra, ve a todas las fuentes de agua y a todos
los arroyos. Tal vez encontremos hierba y salvemos vivos a los caballos
y a las mulas, para que no perdamos todos los animales''.

\bibleverse{6} Así que se repartieron la tierra para pasar por ella.
Ajab se fue por un camino, y Abdías por otro. \bibleverse{7} Cuando
Abdías iba por el camino, he aquí que Elías le salió al encuentro. Lo
reconoció, se postró sobre su rostro y dijo: ``¿Eres tú, mi señor
Elías?''.

\bibleverse{8} Él le respondió: ``Soy yo. Ve y dile a tu señor: ``¡Hay
que ver que Elías está aquí!''.

\bibleverse{9} Él dijo: ``¿En qué he pecado, para que entregues a tu
siervo en manos de Ajab, para que me mate? \bibleverse{10} Vive Yahvé,
tu Dios, que no hay nación ni reino donde mi señor no haya enviado a
buscarte. Cuando le dijeron: `No está aquí', juró al reino y a la nación
que no te encontrarían. \footnote{\textbf{18:10} 1Re 17,12}
\bibleverse{11} Ahora dices: ``Ve y dile a tu señor: ``Aquí está
Elías''\,''. \bibleverse{12} Ocurrirá que, en cuanto te deje, el
Espíritu de Yavé te llevará no sé a dónde; y así, cuando venga y se lo
diga a Ajab, y no te encuentre, me matará. Pero yo, tu siervo, he temido
al Señor desde mi juventud. \footnote{\textbf{18:12} 1Re 18,3}
\bibleverse{13} ¿No se le dijo a mi señor lo que hice cuando Jezabel
mató a los profetas de Yavé, cómo escondí a cien hombres de los profetas
de Yavé con cincuenta a una cueva, y los alimenté con pan y agua?
\bibleverse{14} Ahora dices: ``Ve y dile a tu señor: ``Aquí está
Elías''. Me matará''.

\bibleverse{15} Elías dijo: ``Vive el Señor de los Ejércitos, ante quien
estoy, que hoy me mostraré ante él''. \footnote{\textbf{18:15} 1Re 17,1;
  2Re 3,14}

\hypertarget{eluxedas-antes-que-acab-llamando-a-los-profetas-de-los-uxeddolos-al-monte-carmelo}{%
\subsection{Elías antes que Acab; Llamando a los profetas de los ídolos
al monte
Carmelo}\label{eluxedas-antes-que-acab-llamando-a-los-profetas-de-los-uxeddolos-al-monte-carmelo}}

\bibleverse{16} Entonces Abdías fue a reunirse con Ajab y se lo
comunicó, y Ajab fue a reunirse con Elías.

\bibleverse{17} Cuando Ajab vio a Elías, le dijo: ``¿Eres tú,
perturbador de Israel?'' \footnote{\textbf{18:17} Am 7,10; Hech 16,20}

\bibleverse{18} El respondió: ``No he molestado a Israel, sino a ti y a
la casa de tu padre, porque habéis abandonado los mandamientos de Yahvé
y habéis seguido a los baales. \footnote{\textbf{18:18} 1Re 16,31-32}
\bibleverse{19} Ahora, pues, envía y reúne conmigo a todo Israel en el
monte Carmelo, y a cuatrocientos cincuenta de los profetas de Baal, y a
cuatrocientos de los profetas de Asera, que comen en la mesa de
Jezabel.'' \footnote{\textbf{18:19} 1Re 16,33}

\bibleverse{20} Entonces Acab envió a todos los hijos de Israel y reunió
a los profetas en el monte Carmelo.

\hypertarget{el-juicio-de-dios-en-el-monte-carmelo-la-matanza-de-los-profetas-de-baal}{%
\subsection{El juicio de Dios en el monte Carmelo; la matanza de los
profetas de
Baal}\label{el-juicio-de-dios-en-el-monte-carmelo-la-matanza-de-los-profetas-de-baal}}

\bibleverse{21} Elías se acercó a todo el pueblo y dijo: ``¿Hasta cuándo
vacilaréis entre los dos bandos? Si Yahvé es Dios, seguidlo; pero si es
Baal, seguidlo''. La gente no dijo nada. \footnote{\textbf{18:21} Jos
  24,15; Mat 6,24}

\bibleverse{22} Entonces Elías dijo al pueblo: ``Yo, sólo yo, he quedado
como profeta de Yahvé; pero los profetas de Baal son cuatrocientos
cincuenta hombres. \bibleverse{23} Que nos den, pues, dos toros, y que
escojan un toro para ellos, lo corten en pedazos, lo pongan sobre la
leña y no pongan fuego debajo; y yo aderezaré el otro toro, lo pondré
sobre la leña y no pondré fuego debajo. \bibleverse{24} Tú invocas el
nombre de tu dios, y yo invocaré el nombre de Yahvé. El Dios que
responde con fuego, que sea Dios''. Toda la gente respondió: ``Lo que
dices es bueno''.

\bibleverse{25} Elías dijo a los profetas de Baal: ``Escoged un solo
toro para vosotros y aderezadlo primero, porque sois muchos; e invocad
el nombre de vuestro dios, pero no pongáis fuego debajo.''

\bibleverse{26} Tomaron el toro que les habían dado, lo aderezaron e
invocaron el nombre de Baal desde la mañana hasta el mediodía, diciendo:
``¡Baal, escúchanos!''. Pero no hubo voz, ni nadie respondió. Saltaron
alrededor del altar que se había hecho.

\bibleverse{27} Al mediodía, Elías se burló de ellos y dijo: ``Griten,
porque es un dios. O está sumido en sus pensamientos, o se ha ido a
alguna parte, o está de viaje, o tal vez duerme y hay que despertarlo''.

\bibleverse{28} Gritaron en voz alta y se cortaron en su camino con
cuchillos y lanzas hasta que la sangre brotó sobre ellos.
\bibleverse{29} Cuando pasó el mediodía, profetizaron hasta la hora de
la ofrenda de la tarde; pero no hubo voz ni respuesta, y nadie les
prestó atención. \footnote{\textbf{18:29} 1Sam 18,10; Núm 28,4-5}

\bibleverse{30} Elías dijo a todo el pueblo: ``¡Acérquense a mí!''; y
todo el pueblo se acercó a él. Él reparó el altar de Yavé que había sido
derribado. \bibleverse{31} Elías tomó doce piedras, según el número de
las tribus de los hijos de Jacob, a quienes llegó la palabra de Yavé
diciendo: ``Israel será tu nombre.'' \footnote{\textbf{18:31} Éxod 24,4;
  Gén 32,28} \bibleverse{32} Con las piedras construyó un altar en
nombre de Yavé. Hizo una zanja alrededor del altar lo suficientemente
grande como para contener dos seahs\footnote{\textbf{18:32} 1 marino
  equivale a unos 7 litros o 1,9 galones o 0,8 picotazos} de semillas.
\bibleverse{33} Puso la madera en orden, cortó el toro en pedazos y lo
puso sobre la madera. Dijo: ``Llena cuatro tinajas con agua, y viértela
sobre el holocausto y sobre la madera''. \bibleverse{34} Dijo: ``Háganlo
por segunda vez;'' y lo hicieron por segunda vez. Dijo: ``Háganlo por
tercera vez'', y lo hicieron por tercera vez. \bibleverse{35} El agua
corrió alrededor del altar, y también llenó de agua la zanja.

\hypertarget{la-oraciuxf3n-de-eluxedas-escuchada-de-dios-matanza-de-los-pastores-de-baal}{%
\subsection{La oración de Elías escuchada de Dios; Matanza de los
pastores de
Baal}\label{la-oraciuxf3n-de-eluxedas-escuchada-de-dios-matanza-de-los-pastores-de-baal}}

\bibleverse{36} A la hora de la ofrenda de la tarde, el profeta Elías se
acercó y dijo: ``Yahvé, Dios de Abraham, de Isaac y de Israel, haz que
se sepa hoy que tú eres Dios en Israel y que yo soy tu siervo, y que he
hecho todo esto por tu palabra. \bibleverse{37} Escúchame, Yahvé,
escúchame, para que este pueblo sepa que tú, Yahvé, eres Dios, y que has
hecho volver su corazón''.

\bibleverse{38} Entonces el fuego del Señor cayó y consumió el
holocausto, la madera, las piedras y el polvo; y lamió el agua que
estaba en la zanja. \footnote{\textbf{18:38} Lev 9,24} \bibleverse{39}
Cuando todo el pueblo lo vio, se postró sobre sus rostros. Decían:
``¡Yahvé, él es Dios! Yahvé, él es Dios!''

\bibleverse{40} Elías les dijo: ``¡Atrapen a los profetas de Baal! No
dejéis que se escape ni uno de ellos''. Los apresaron, y Elías los hizo
descender al arroyo Cisón, y allí los mató. \footnote{\textbf{18:40}
  Deut 13,5; 2Re 10,25}

\hypertarget{descripciuxf3n-de-la-tormenta-eluxe9ctrica-ascendente-el-impulso-de-acab-y-la-resistencia-de-eluxedas-corren-a-jezreel}{%
\subsection{Descripción de la tormenta eléctrica ascendente; El impulso
de Acab y la resistencia de Elías corren a
Jezreel}\label{descripciuxf3n-de-la-tormenta-eluxe9ctrica-ascendente-el-impulso-de-acab-y-la-resistencia-de-eluxedas-corren-a-jezreel}}

\bibleverse{41} Elías dijo a Ajab: ``Levántate, come y bebe, porque se
oye el ruido de la lluvia abundante''.

\bibleverse{42} Entonces Acab subió a comer y a beber. Elías subió a la
cima del Carmelo, se postró en tierra y puso su rostro entre sus
rodillas. \footnote{\textbf{18:42} Sant 5,18} \bibleverse{43} Dijo a su
siervo: ``Sube ahora y mira hacia el mar''. Subió, miró y dijo: ``No hay
nada''. Dijo: ``Vuelve a ir'' siete veces.

\bibleverse{44} A la séptima vez, dijo: ``He aquí que una pequeña nube,
como la mano de un hombre, se levanta del mar''. Dijo: ``Sube y dile a
Ajab: ``Prepárate y baja, para que la lluvia no te detenga''\,''.

\bibleverse{45} Al poco tiempo, el cielo se oscureció con nubes y
viento, y hubo una gran lluvia. Acab cabalgó y se dirigió a Jezreel.
\bibleverse{46} La mano de Yahvé estaba sobre Elías; éste se metió el
manto en el cinturón y corrió delante de Ajab hasta la entrada de
Jezreel.

\hypertarget{la-amenaza-de-jezabel-el-desaliento-de-eluxedas-su-fortalecimiento-por-un-uxe1ngel-y-su-viaje-a-horeb}{%
\subsection{La amenaza de Jezabel; El desaliento de Elías; su
fortalecimiento por un ángel y su viaje a
Horeb}\label{la-amenaza-de-jezabel-el-desaliento-de-eluxedas-su-fortalecimiento-por-un-uxe1ngel-y-su-viaje-a-horeb}}

\hypertarget{section-18}{%
\section{19}\label{section-18}}

\bibleverse{1} Ajab le contó a Jezabel todo lo que había hecho Elías y
cómo había matado a todos los profetas a espada. \footnote{\textbf{19:1}
  1Re 18,40} \bibleverse{2} Entonces Jezabel envió un mensajero a Elías,
diciéndole: ``¡Así me hagan los dioses, y más también, si no hago tu
vida como la de uno de ellos para mañana a esta hora!''

\bibleverse{3} Al ver esto, se levantó y corrió por su vida, y llegó a
Beerseba, que pertenece a Judá, y dejó allí a su siervo. \bibleverse{4}
Pero él mismo se fue un día de camino al desierto, y llegó y se sentó
bajo un enebro. Entonces pidió para sí mismo la muerte, y dijo: ``Ya es
suficiente. Ahora, oh Yahvé, quita mi vida, pues no soy mejor que mis
padres''. \footnote{\textbf{19:4} Job 7,16; Jon 4,3; Fil 1,23}

\bibleverse{5} Se acostó y durmió bajo un enebro; y he aquí que un ángel
le tocó y le dijo: ``¡Levántate y come!''

\bibleverse{6} Miró, y he aquí que había junto a su cabeza una torta
cocida sobre las brasas y una jarra de agua. Comió y bebió, y volvió a
acostarse. \bibleverse{7} El ángel de Yavé volvió a venir por segunda
vez, lo tocó y le dijo: ``Levántate y come, porque el viaje es demasiado
grande para ti.''

\bibleverse{8} Se levantó, comió y bebió, y con la fuerza de ese
alimento se dirigió durante cuarenta días y cuarenta noches a Horeb, la
Montaña de Dios. \footnote{\textbf{19:8} Éxod 24,18} \bibleverse{9}
Llegó a una cueva de allí, y acampó allí; y he aquí que la palabra de
Yahvé vino a él, y le dijo: ``¿Qué haces aquí, Elías?''

\hypertarget{la-revelaciuxf3n-divina-en-horeb}{%
\subsection{La revelación divina en
Horeb}\label{la-revelaciuxf3n-divina-en-horeb}}

\bibleverse{10} Dijo: ``He sentido muchos celos por Yahvé, el Dios de
los Ejércitos, porque los hijos de Israel han abandonado tu alianza, han
derribado tus altares y han matado a tus profetas a espada. Yo, sólo yo,
he quedado; y buscan mi vida para quitármela''. \footnote{\textbf{19:10}
  Is 49,4; Rom 11,3; 1Re 18,22}

\bibleverse{11} Dijo: ``Sal y ponte en el monte delante de Yahvé''. Pasó
el Señor, y un viento grande y fuerte desgarró los montes y desmenuzó
las rocas ante el Señor; pero el Señor no estaba en el viento. Después
del viento hubo un terremoto, pero el Señor no estaba en el terremoto.
\footnote{\textbf{19:11} Éxod 33,22} \bibleverse{12} Después del
terremoto pasó un fuego, pero el Señor no estaba en el fuego. Después
del fuego, se oyó una voz tranquila y pequeña. \footnote{\textbf{19:12}
  Éxod 34,6} \bibleverse{13} Al oírla, Elías se envolvió en su manto,
salió y se puso a la entrada de la cueva. Se le acercó una voz y le
dijo: ``¿Qué haces aquí, Elías?''.

\bibleverse{14} Dijo: ``He sentido muchos celos por Yahvé, el Dios de
los Ejércitos; porque los hijos de Israel han abandonado tu alianza, han
derribado tus altares y han matado a tus profetas a espada. Yo, sólo yo,
he quedado; y buscan mi vida para quitármela''. \footnote{\textbf{19:14}
  1Re 19,10; Sal 69,9}

\hypertarget{eluxedas-recibe-la-orden-de-preparar-las-herramientas-de-la-venganza-divina-hazael-jehuxfa-eliseo}{%
\subsection{Elías recibe la orden de preparar las herramientas de la
venganza divina (Hazael, Jehú,
Eliseo)}\label{eluxedas-recibe-la-orden-de-preparar-las-herramientas-de-la-venganza-divina-hazael-jehuxfa-eliseo}}

\bibleverse{15} El Señor le dijo: ``Ve, regresa por tu camino al
desierto de Damasco. Cuando llegues, unge a Hazael como rey de Siria.
\footnote{\textbf{19:15} 2Re 8,13; 2Re 8,15} \bibleverse{16} Unge a
Jehú, hijo de Nimsí, para que sea rey de Israel; y unge a Eliseo, hijo
de Safat, de Abel Meholá, para que sea profeta en tu lugar. \footnote{\textbf{19:16}
  2Re 9,2-3; 1Re 19,19} \bibleverse{17} Al que escape de la espada de
Hazael, lo matará Jehú; y al que escape de la espada de Jehú, lo matará
Eliseo. \bibleverse{18} Pero he reservado siete mil en Israel, todas las
rodillas que no se han inclinado ante Baal, y toda boca que no lo ha
besado.'' \footnote{\textbf{19:18} Rom 11,4}

\hypertarget{el-llamado-de-eliseo}{%
\subsection{El llamado de Eliseo}\label{el-llamado-de-eliseo}}

\bibleverse{19} Se fue de allí y encontró a Eliseo, hijo de Safat, que
estaba arando con doce yuntas de bueyes delante de él, y él con la
duodécima. Elías se acercó a él y le puso su manto. \bibleverse{20}
Eliseo dejó los bueyes y corrió detrás de Elías, diciendo: ``Déjame por
favor besar a mi padre y a mi madre, y luego te seguiré''. Le dijo:
``Vuelve a la carga, porque ¿qué te he hecho?''. \footnote{\textbf{19:20}
  Luc 9,61}

\bibleverse{21} Volvió de seguirlo, y tomó la yunta de bueyes, los mató
y coció su carne con el equipo de los bueyes, y dio al pueblo; y ellos
comieron. Luego se levantó, y fue en pos de Elías, y le sirvió.

\hypertarget{ben-adad-asedia-samaria-la-debilidad-inicial-de-acab-luego-un-comportamiento-decidido}{%
\subsection{Ben-adad asedia Samaria; La debilidad inicial de Acab, luego
un comportamiento
decidido}\label{ben-adad-asedia-samaria-la-debilidad-inicial-de-acab-luego-un-comportamiento-decidido}}

\hypertarget{section-19}{%
\section{20}\label{section-19}}

\bibleverse{1} Ben Hadad, rey de Siria, reunió a todo su ejército, y
había con él treinta y dos reyes, con caballos y carros. Subió y sitió a
Samaria, y luchó contra ella. \bibleverse{2} Envió mensajeros a la
ciudad a Ajab, rey de Israel, y le dijo: ``Ben Hadad dice:
\bibleverse{3} `Tu plata y tu oro son míos. Tus esposas también y tus
hijos, incluso los mejores, son míos'\,''.

\bibleverse{4} El rey de Israel respondió: ``Así es, mi señor, oh rey.
Soy tuyo, y todo lo que tengo''.

\bibleverse{5} Los mensajeros volvieron a decir: ``Ben Hadad dice: `En
efecto, te envié a decir: ``Me entregarás tu plata, tu oro, tus esposas
y tus hijos; \bibleverse{6} pero mañana, a esta hora, te enviaré a mis
siervos y registrarán tu casa y las casas de tus siervos. Todo lo que
sea agradable a tus ojos, lo pondrán en su mano y se lo llevarán''\,''.

\bibleverse{7} Entonces el rey de Israel llamó a todos los ancianos del
país y les dijo: ``Fíjense en que este hombre busca el mal, porque me
mandó a pedir mis mujeres, mis hijos, mi plata y mi oro, y no se lo
negué.'' \footnote{\textbf{20:7} 2Re 5,7}

\bibleverse{8} Todos los ancianos y todo el pueblo le dijeron: ``No
escuches y no consientas''.

\bibleverse{9} Por eso dijo a los mensajeros de Ben Hadad: ``Decid a mi
señor el rey: ``Todo lo que mandasteis a vuestro siervo al principio lo
haré, pero esto no puedo hacerlo''\,''. Los mensajeros partieron y le
trajeron el mensaje. \bibleverse{10} Ben Hadad le mandó decir: ``Los
dioses me lo hacen, y más aún, si el polvo de Samaria alcanza para
puñados para todo el pueblo que me sigue''.

\bibleverse{11} El rey de Israel contestó: ``Dile que no se jacte el que
se pone la armadura como el que se la quita''.

\bibleverse{12} Cuando Ben Hadad escuchó este mensaje mientras bebía, él
y los reyes en los pabellones, dijo a sus sirvientes: ``¡Prepárense para
atacar!'' Así que se prepararon para atacar la ciudad.

\hypertarget{las-instrucciones-de-un-profeta-a-acab-gran-victoria-de-los-israelitas-sobre-ben-adad}{%
\subsection{Las instrucciones de un profeta a Acab; gran victoria de los
israelitas sobre
Ben-adad}\label{las-instrucciones-de-un-profeta-a-acab-gran-victoria-de-los-israelitas-sobre-ben-adad}}

\bibleverse{13} He aquí que un profeta se acercó a Ajab, rey de Israel,
y le dijo: ``El Señor dice: `¿Has visto toda esta gran multitud? He aquí
que hoy la entregaré en tu mano. Entonces sabrás que yo soy Yahvé'\,''.

\bibleverse{14} Ahab dijo: ``¿Por quién?'' Dijo: ``Yahvé dice: `Por los
jóvenes de los príncipes de las provincias'\,''. Entonces dijo: ``¿Quién
comenzará la batalla?'' Él respondió: ``Tú''.

\bibleverse{15} Luego reunió a los jóvenes de los príncipes de las
provincias, que eran doscientos treinta y dos. Después de ellos, reunió
a todo el pueblo, a todos los hijos de Israel, que eran siete mil.
\bibleverse{16} Salieron al mediodía. Pero Ben Hadad se emborrachaba en
los pabellones, él y los reyes, los treinta y dos reyes que le ayudaban.
\bibleverse{17} Los jóvenes de los príncipes de las provincias salieron
primero, y Ben Hadad mandó a decir: ``Salen hombres de Samaria.''

\bibleverse{18} Dijo: ``Si han salido por la paz, tómenlos vivos; o si
han salido por la guerra, tómenlos vivos''.

\bibleverse{19} Estos salieron de la ciudad, los jóvenes de los
príncipes de las provincias y el ejército que los seguía.
\bibleverse{20} Cada uno mató a su hombre. Los sirios huyeron, e Israel
los persiguió. Ben Hadad, rey de Siria, escapó en un caballo con gente
de a caballo. \bibleverse{21} El rey de Israel salió y golpeó a los
caballos y a los carros, y mató a los sirios con una gran matanza.

\hypertarget{segunda-campauxf1a-de-benhadad-nueva-instrucciuxf3n-del-profeta-a-acab-asesorar-a-los-sirios-victoria-de-los-israelitas}{%
\subsection{Segunda campaña de Benhadad; nueva instrucción del profeta a
Acab; Asesorar a los sirios; Victoria de los
israelitas}\label{segunda-campauxf1a-de-benhadad-nueva-instrucciuxf3n-del-profeta-a-acab-asesorar-a-los-sirios-victoria-de-los-israelitas}}

\bibleverse{22} El profeta se acercó al rey de Israel y le dijo: ``Ve,
fortalécete y planifica lo que debes hacer, porque a la vuelta del año,
el rey de Siria subirá contra ti.'' \footnote{\textbf{20:22} 1Re 20,13}

\bibleverse{23} Los servidores del rey de Siria le dijeron: ``Su dios es
un dios de las colinas; por eso fueron más fuertes que nosotros. Pero
luchemos contra ellos en la llanura, y seguramente seremos más fuertes
que ellos. \footnote{\textbf{20:23} 1Re 20,25} \bibleverse{24} Haz esto:
quita a los reyes, cada uno de su lugar, y pon capitanes en su lugar.
\bibleverse{25} Reúne un ejército como el que has perdido, caballo por
caballo y carro por carro. Lucharemos contra ellos en la llanura, y
seguramente seremos más fuertes que ellos''. Él escuchó su voz y así lo
hizo. \bibleverse{26} A la vuelta del año, Ben Hadad reunió a los sirios
y subió a Afec para luchar contra Israel. \bibleverse{27} Los hijos de
Israel se reunieron y recibieron provisiones, y fueron contra ellos. Los
hijos de Israel acamparon frente a ellos como dos rebaños pequeños de
cabras jóvenes, pero los sirios llenaron el país. \bibleverse{28} Un
hombre de Dios se acercó y habló al rey de Israel y le dijo: ``Yahvé
dice: `Como los sirios han dicho: ``Yahvé es un dios de las colinas,
pero no es un dios de los valles'', por eso entregaré a toda esta gran
multitud en tu mano, y sabrás que yo soy Yahvé'\,''. \footnote{\textbf{20:28}
  1Re 20,22}

\bibleverse{29} Acamparon uno frente al otro durante siete días. Al
séptimo día se entabló la batalla, y los hijos de Israel mataron a cien
mil hombres de a pie de los sirios en un solo día.

\hypertarget{ben-adad-sitiada-en-aphek-y-obligada-a-rendirse-la-descuidada-dulzura-de-acab-hacia-uxe9l}{%
\subsection{Ben-adad sitiada en Aphek y obligada a rendirse; La
descuidada dulzura de Acab hacia
él}\label{ben-adad-sitiada-en-aphek-y-obligada-a-rendirse-la-descuidada-dulzura-de-acab-hacia-uxe9l}}

\bibleverse{30} Pero los demás huyeron a Afec, a la ciudad, y el muro
cayó sobre veintisiete mil hombres que quedaban. Ben Hadad huyó y entró
en la ciudad, en una habitación interior. \bibleverse{31} Sus siervos le
dijeron: ``Mira, hemos oído que los reyes de la casa de Israel son reyes
misericordiosos. Por favor, pongamos sacos en nuestros cuerpos y cuerdas
en nuestras cabezas, y salgamos a ver al rey de Israel. Tal vez él te
salve la vida''.

\bibleverse{32} Así que se pusieron tela de saco en el cuerpo y cuerdas
en la cabeza, y vinieron al rey de Israel y le dijeron: ``Tu siervo Ben
Hadad dice: ``Por favor, déjame vivir''\,''. Dijo: ``¿Aún está vivo? Es
mi hermano''.

\bibleverse{33} Los hombres observaron con diligencia y se apresuraron a
tomar esta frase, y dijeron: ``Tu hermano Ben Hadad''. Entonces dijo:
``Ve, tráelo''. Entonces Ben Hadad salió hacia él, y lo hizo subir al
carro. \bibleverse{34} Ben Hadad le dijo: ``Las ciudades que mi padre
tomó de tu padre, yo las restauraré. Te harás calles en Damasco, como
las que mi padre hizo en Samaria''. ``Yo'', dijo Ajab, ``te dejaré ir
con este pacto''. Así que hizo un pacto con él y lo dejó ir.

\hypertarget{un-discipulo-de-los-profetas-confronta-a-acab-con-su-error}{%
\subsection{Un discipulo de los profetas confronta a Acab con su
error}\label{un-discipulo-de-los-profetas-confronta-a-acab-con-su-error}}

\bibleverse{35} Un hombre de los hijos de los profetas dijo a su
compañero por palabra de Yahvé: ``¡Por favor, golpéame!'' El hombre se
negó a golpearlo. \bibleverse{36} Entonces le dijo: ``Como no has
obedecido la voz de Yahvé, he aquí que en cuanto te apartes de mí, te
matará un león''. En cuanto se apartó de él, un león lo encontró y lo
mató. \footnote{\textbf{20:36} 1Re 13,24}

\bibleverse{37} Entonces encontró a otro hombre y le dijo: ``Por favor,
golpéame''. El hombre lo golpeó y lo hirió. \bibleverse{38} Entonces el
profeta partió y esperó al rey en el camino, y se disfrazó con su
cintillo sobre los ojos. \bibleverse{39} Al pasar el rey, gritó al rey y
le dijo: ``Tu siervo salió en medio de la batalla; y he aquí que un
hombre se acercó y me trajo a un hombre, y me dijo: ``¡Guarda a este
hombre! Si por cualquier medio se pierde, entonces tu vida será por la
suya, o si no pagarás un talento\footnote{\textbf{20:39} Un talento son
  unos 30 kilos o 66 libras} de plata.' \footnote{\textbf{20:39} 2Re
  10,24} \bibleverse{40} Como tu siervo estaba ocupado aquí y allá,
desapareció''. El rey de Israel le dijo: ``Así será tu juicio. Tú mismo
lo has decidido''.

\bibleverse{41} Se apresuró a quitarse la cinta de los ojos, y el rey de
Israel reconoció que era uno de los profetas. \bibleverse{42} Le dijo:
``Yahvé dice: `Como has dejado ir de tu mano al hombre que yo había
consagrado a la destrucción, por eso tu vida tomará el lugar de su vida,
y tu pueblo el lugar de su pueblo'.''

\bibleverse{43} El rey de Israel se fue a su casa hosco y enojado, y
llegó a Samaria.

\hypertarget{el-vergonzoso-acto-de-violencia-de-acab-contra-nabot}{%
\subsection{El vergonzoso acto de violencia de Acab contra
Nabot}\label{el-vergonzoso-acto-de-violencia-de-acab-contra-nabot}}

\hypertarget{section-20}{%
\section{21}\label{section-20}}

\bibleverse{1} Después de estas cosas, Nabot de Jezreel tenía una viña
que estaba en Jezreel, junto al palacio de Acab, rey de Samaria.
\bibleverse{2} Acab habló a Nabot, diciendo: ``Dame tu viña, para que la
tenga como jardín de hierbas, porque está cerca de mi casa; y yo te daré
por ella una viña mejor que ésta. O, si te parece bien, te daré su valor
en dinero''.

\bibleverse{3} Nabot dijo a Ajab: ``¡Que Yahvé me prohíba dar la
herencia de mis padres a ti!''

\bibleverse{4} Acab entró en su casa hosco y enojado por la palabra que
le había dicho Nabot de Jezreel, pues había dicho: ``No te daré la
herencia de mis padres''. Se acostó en su cama, apartó su rostro y no
quiso comer pan.

\hypertarget{la-ominosa-intervenciuxf3n-de-jezabel-su-carta-sin-valor-a-los-ancianos-de-la-ciudad-de-jezreel}{%
\subsection{La ominosa intervención de Jezabel; su carta sin valor a los
ancianos de la ciudad de
Jezreel}\label{la-ominosa-intervenciuxf3n-de-jezabel-su-carta-sin-valor-a-los-ancianos-de-la-ciudad-de-jezreel}}

\bibleverse{5} Pero Jezabel, su mujer, se acercó a él y le dijo: ``¿Por
qué está tu espíritu tan triste que no comes pan?''

\bibleverse{6} Le dijo: ``Porque hablé con Nabot de Jezreel y le dije:
`Dame tu viña por dinero, o si te place, te daré otra viña por ella'. Él
respondió: `No te daré mi viña'\,''.

\bibleverse{7} Jezabel, su mujer, le dijo: ``¿Ahora gobiernas el reino
de Israel? Levántate, come pan y alegra tu corazón. Te daré la viña de
Nabot el jezreelita''. \bibleverse{8} Así que ella escribió cartas en
nombre de Acab y las selló con su sello, y envió las cartas a los
ancianos y a los nobles que estaban en su ciudad, que vivían con Nabot.
\bibleverse{9} Ella escribió en las cartas, diciendo: ``Proclamen un
ayuno y pongan a Nabot en alto entre el pueblo. \bibleverse{10} Poned
delante de él a dos hombres malvados, y que testifiquen contra él
diciendo: `¡Has maldecido a Dios y al rey! Luego llévenlo y mátenlo a
pedradas''. \footnote{\textbf{21:10} Job 1,5; Éxod 22,28}

\hypertarget{nefasto-asesinato-de-naboth-la-violenta-posesiuxf3n-de-la-viuxf1a-por-acab}{%
\subsection{Nefasto asesinato de Naboth; La violenta posesión de la viña
por
Acab}\label{nefasto-asesinato-de-naboth-la-violenta-posesiuxf3n-de-la-viuxf1a-por-acab}}

\bibleverse{11} Los hombres de su ciudad, los ancianos y los nobles que
vivían en ella, hicieron lo que Jezabel les había ordenado en las cartas
que les había escrito y enviado. \bibleverse{12} Proclamaron un ayuno y
pusieron a Nabot en lo alto del pueblo. \bibleverse{13} Los dos hombres,
los malvados, entraron y se sentaron ante él. Los malvados testificaron
contra él, contra Nabot, en presencia del pueblo, diciendo: ``¡Nabot
maldijo a Dios y al rey!'' Entonces lo sacaron de la ciudad y lo mataron
a pedradas. \bibleverse{14} Luego enviaron a Jezabel diciendo: ``Nabot
ha sido apedreado y ha muerto''.

\bibleverse{15} Cuando Jezabel se enteró de que Nabot había sido
apedreado y estaba muerto, le dijo a Ajab: ``Levántate y toma posesión
de la viña de Nabot de Jezreel, que él se negó a darte por dinero;
porque Nabot no está vivo, sino muerto.''

\bibleverse{16} Cuando Ajab se enteró de que Nabot había muerto, se
levantó para bajar a la viña de Nabot de Jezreel, para tomar posesión de
ella.

\hypertarget{eluxedas-anuncia-el-juicio-divino-a-la-pareja-real}{%
\subsection{Elías anuncia el juicio divino a la pareja
real}\label{eluxedas-anuncia-el-juicio-divino-a-la-pareja-real}}

\bibleverse{17} La palabra de Yahvé llegó a Elías el tisbita, diciendo:
\bibleverse{18} ``Levántate y baja a recibir a Ajab, rey de Israel, que
vive en Samaria. He aquí que está en la viña de Nabot, a la que ha
bajado para tomar posesión de ella. \bibleverse{19} Le hablarás
diciendo: ``El Señor dice: ``¿Has matado y también tomado posesión? Le
hablarás diciendo: `Dice el Señor: ``En el lugar donde los perros
lamieron la sangre de Nabot, los perros lamerán tu sangre, la
tuya''\,'\,''. \footnote{\textbf{21:19} 1Re 22,38}

\bibleverse{20} Acab dijo a Elías: ``¿Me has encontrado, mi enemigo?''
Él respondió: ``Te he encontrado, porque te has vendido a hacer lo que
es malo a los ojos de Yahvé. \bibleverse{21} He aquí que yo traigo el
mal sobre ti, y te barreré por completo, y cortaré de Acab a todo el que
orine contra una pared,\footnote{\textbf{21:21} o, hombre} y al que esté
encerrado y al que quede suelto en Israel. \footnote{\textbf{21:21} 2Re
  9,7-8} \bibleverse{22} Haré que tu casa sea como la casa de Jeroboam,
hijo de Nabat, y como la casa de Baasa, hijo de Ahías, por la
provocación con que me has hecho enojar y has hecho pecar a Israel.''
\footnote{\textbf{21:22} 1Re 15,29; 1Re 16,11-12} \bibleverse{23} Yahvé
también habló de Jezabel, diciendo: ``Los perros se comerán a Jezabel
junto a la muralla de Jezreel. \footnote{\textbf{21:23} 2Re 9,33-36}
\bibleverse{24} Los perros se comerán al que muera de Ajab en la ciudad,
y las aves del cielo se comerán al que muera en el campo.'' \footnote{\textbf{21:24}
  1Re 14,11}

\bibleverse{25} Pero no hubo nadie como Ajab, que se vendió para hacer
lo que era malo a los ojos de Yavé, a quien Jezabel, su esposa, incitó.
\bibleverse{26} Hizo de manera muy abominable al seguir a los ídolos,
conforme a todo lo que hicieron los amorreos, a quienes Yahvé expulsó
ante los hijos de Israel.

\hypertarget{el-arrepentimiento-de-acab-la-mitigaciuxf3n-de-dios-de-la-amenaza-de-dauxf1o}{%
\subsection{El arrepentimiento de Acab; La mitigación de Dios de la
amenaza de
daño}\label{el-arrepentimiento-de-acab-la-mitigaciuxf3n-de-dios-de-la-amenaza-de-dauxf1o}}

\bibleverse{27} Al oír estas palabras, Ajab se rasgó las vestiduras, se
puso un saco en el cuerpo, ayunó, se acostó en un saco y anduvo abatido.

\bibleverse{28} La palabra de Yahvé vino a Elías el tisbita, diciendo:
\bibleverse{29} ``¿Ves cómo se humilla Ajab ante mí? Porque se humilla
ante mí, no traeré el mal en sus días; pero traeré el mal sobre su casa
en los días de su hijo.'' \footnote{\textbf{21:29} 2Re 9,22; 2Re 9,26}

\hypertarget{acab-y-josafat-unen-fuerzas-en-la-guerra-contra-siria}{%
\subsection{Acab y Josafat unen fuerzas en la guerra contra
Siria}\label{acab-y-josafat-unen-fuerzas-en-la-guerra-contra-siria}}

\hypertarget{section-21}{%
\section{22}\label{section-21}}

\bibleverse{1} Continuaron tres años sin guerra entre Siria e Israel.
\bibleverse{2} Al tercer año, Josafat, rey de Judá, bajó a ver al rey de
Israel. \footnote{\textbf{22:2} 1Re 22,41; 2Cró 18,2-13} \bibleverse{3}
El rey de Israel dijo a sus siervos: ``¿Sabéis que Ramot de Galaad es
nuestra, y no hacemos nada, y no la quitamos de la mano del rey de
Siria?'' \footnote{\textbf{22:3} Jos 21,38} \bibleverse{4} Dijo a
Josafat: ``¿Quieres ir conmigo a la batalla de Ramot de Galaad?''
Josafat dijo al rey de Israel: ``Yo soy como tú, mi pueblo como tu
pueblo, mis caballos como tus caballos''. \footnote{\textbf{22:4} 2Re
  3,7; 2Cró 19,2}

\hypertarget{el-mensaje-favorable-de-los-400-profetas-micha-deberuxeda-ser-entrevistado}{%
\subsection{El mensaje favorable de los 400 profetas; Micha debería ser
entrevistado}\label{el-mensaje-favorable-de-los-400-profetas-micha-deberuxeda-ser-entrevistado}}

\bibleverse{5} Josafat dijo al rey de Israel: ``Por favor, consulta
primero la palabra de Yahvé''.

\bibleverse{6} Entonces el rey de Israel reunió a los profetas, unos
cuatrocientos hombres, y les dijo: ``¿Debo ir contra Ramot de Galaad a
combatir, o me abstengo?'' Dijeron: ``Sube, porque el Señor la entregará
en mano del rey''.

\bibleverse{7} Pero Josafat dijo: ``¿No hay aquí un profeta de Yahvé,
para que le preguntemos?'' \footnote{\textbf{22:7} 2Re 3,11}

\bibleverse{8} El rey de Israel dijo a Josafat: ``Todavía hay un hombre
por el que podemos consultar a Yavé, Micaías hijo de Imá; pero lo odio,
porque no profetiza el bien sobre mí, sino el mal.'' Josafat dijo: ``Que
no lo diga el rey''.

\bibleverse{9} Entonces el rey de Israel llamó a un oficial y le dijo:
``Trae rápidamente a Micaías, hijo de Imlah''.

\bibleverse{10} El rey de Israel y Josafat, rey de Judá, estaban
sentados cada uno en su trono, vestidos con sus ropas, en un lugar
abierto a la entrada de la puerta de Samaria, y todos los profetas
profetizaban delante de ellos. \bibleverse{11} Sedequías, hijo de Quená,
se hizo unos cuernos de hierro y dijo: ``Yahvé dice: `Con estos
empujarás a los sirios hasta consumirlos'\,''. \bibleverse{12} Así lo
profetizaron todos los profetas, diciendo: ``Sube a Ramot de Galaad y
prospera, porque Yavé la entregará en manos del rey.''

\hypertarget{la-buena-fortuna-inicial-de-micha-luego-su-anuncio-de-la-perdiciuxf3n}{%
\subsection{La buena fortuna inicial de Micha, luego su anuncio de la
perdición}\label{la-buena-fortuna-inicial-de-micha-luego-su-anuncio-de-la-perdiciuxf3n}}

\bibleverse{13} El mensajero que fue a llamar a Micaías le habló
diciendo: ``Mira ahora, los profetas declaran el bien al rey con una
sola boca. Por favor, que tu palabra sea como la de uno de ellos, y
habla bien''.

\bibleverse{14} Micaías dijo: ``Vive Yahvé, lo que Yahvé me diga, eso
hablaré''.

\bibleverse{15} Cuando llegó al rey, éste le dijo: ``Micaías, ¿vamos a
Ramot de Galaad a combatir o nos abstenemos?'' Él le respondió: ``Sube y
prospera, y Yahvé la entregará en mano del rey''.

\bibleverse{16} El rey le dijo: ``¿Cuántas veces tengo que conjurarte
para que no me digas más que la verdad en nombre de Yahvé?''

\bibleverse{17} Dijo: ``Vi a todo Israel disperso por los montes, como
ovejas que no tienen pastor. El Señor dijo: `Estas no tienen dueño. Que
cada uno vuelva a su casa en paz'\,''.

\bibleverse{18} El rey de Israel dijo a Josafat: ``¿No te dije que no
profetizaría el bien sobre mí, sino el mal?'' \footnote{\textbf{22:18}
  1Re 22,8}

\bibleverse{19} Micaías dijo: ``Escuchen, pues, la palabra de Yahvé. Vi
al Señor sentado en su trono, y a todo el ejército del cielo junto a él,
a su derecha y a su izquierda. \footnote{\textbf{22:19} Apoc 5,11}
\bibleverse{20} Yavé dijo: ``¿Quién va a tentar a Ajab para que suba y
caiga en Ramot de Galaad? Uno dijo una cosa, y otro dijo otra.

\bibleverse{21} Un espíritu salió y se puso delante de Yahvé, y dijo:
``Lo voy a seducir''. \footnote{\textbf{22:21} Is 19,14}

\bibleverse{22} El Señor le dijo: ``¿Cómo? Dijo: `Saldré y seré un
espíritu mentiroso en la boca de todos sus profetas'. Dijo: `Lo
atraerás, y además vencerás. Salid y hacedlo'. \footnote{\textbf{22:22}
  Juan 8,44; Apoc 16,14} \bibleverse{23} Ahora, pues, he aquí que Yahvé
ha puesto un espíritu mentiroso en la boca de todos estos tus profetas,
y Yahvé ha hablado mal de ti.''

\hypertarget{el-maltrato-de-miqueas-por-sedechuxeeas-y-su-captura-por-acab}{%
\subsection{El maltrato de Miqueas por Sedechîas y su captura por
Acab}\label{el-maltrato-de-miqueas-por-sedechuxeeas-y-su-captura-por-acab}}

\bibleverse{24} Entonces Sedequías, hijo de Quenaana, se acercó y golpeó
a Micaías en la mejilla, y dijo: ``¿Por dónde se fue el Espíritu de
Yahvé de mí para hablarte?''

\bibleverse{25} Micaías dijo: ``He aquí que verás aquel día cuando
entres en una habitación interior para esconderte''.

\bibleverse{26} El rey de Israel dijo: ``Toma a Micaías y llévalo a
Amón, el gobernador de la ciudad, y a Joás, el hijo del rey.
\bibleverse{27} Di: ``El rey dice: ``Pon a este hombre en la cárcel y
aliméntalo con pan de aflicción y con agua de aflicción, hasta que yo
venga en paz''\,''.

\bibleverse{28} Micaías dijo: ``Si regresan en paz, Yahvé no ha hablado
por mí''. Dijo: ``¡Escuchen, todos ustedes!'' \footnote{\textbf{22:28}
  2Cró 18,28-34}

\hypertarget{derrota-de-los-aliados-en-ramoth-la-muerte-de-acab-en-batalla-su-entierro-en-samaria-palabra-final}{%
\subsection{Derrota de los aliados en Ramoth; La muerte de Acab en
batalla; su entierro en Samaria; Palabra
final}\label{derrota-de-los-aliados-en-ramoth-la-muerte-de-acab-en-batalla-su-entierro-en-samaria-palabra-final}}

\bibleverse{29} El rey de Israel y Josafat, rey de Judá, subieron a
Ramot de Galaad. \bibleverse{30} El rey de Israel dijo a Josafat: ``Yo
me disfrazaré y entraré en la batalla, pero tú ponte tus ropas''. El rey
de Israel se disfrazó y entró en la batalla.

\bibleverse{31} El rey de Siria había ordenado a los treinta y dos
capitanes de sus carros que dijeran: ``No peleen con los pequeños ni con
los grandes, sino sólo con el rey de Israel.''

\bibleverse{32} Cuando los capitanes de los carros vieron a Josafat,
dijeron: ``¡Seguramente ése es el rey de Israel!'' Y se acercaron a
pelear contra él. Josafat gritó. \bibleverse{33} Cuando los capitanes de
los carros vieron que no era el rey de Israel, dejaron de perseguirlo.
\bibleverse{34} Un hombre sacó su arco al azar e hirió al rey de Israel
entre las junturas de la armadura. Entonces dijo al conductor de su
carro: ``Da la vuelta y sácame de la batalla, porque estoy gravemente
herido''. \footnote{\textbf{22:34} 2Cró 35,23} \bibleverse{35} La
batalla se intensificó aquel día. El rey fue apuntalado en su carro de
cara a los sirios, y murió al atardecer. La sangre corría por la herida
hasta el fondo del carro. \bibleverse{36} Un grito recorrió el ejército
al ponerse el sol, diciendo: ``¡Cada uno a su ciudad y cada uno a su
país!''

\bibleverse{37} El rey murió y fue llevado a Samaria; y enterraron al
rey en Samaria. \bibleverse{38} Lavaron el carro junto al estanque de
Samaria, y los perros lamieron su sangre donde se lavaban las
prostitutas, según la palabra de Yahvé que él había dicho. \footnote{\textbf{22:38}
  1Re 21,19; 2Re 9,25}

\bibleverse{39} Los demás hechos de Acab, y todo lo que hizo, y la casa
de marfil que construyó, y todas las ciudades que edificó, ¿no están
escritos en el libro de las crónicas de los reyes de Israel?
\bibleverse{40} Ajab, pues, durmió con sus padres, y su hijo Ocozías
reinó en su lugar. \footnote{\textbf{22:40} 1Re 22,51}

\hypertarget{josafat-rey-de-juduxe1}{%
\subsection{Josafat, rey de Judá}\label{josafat-rey-de-juduxe1}}

\bibleverse{41} Josafat hijo de Asa comenzó a reinar sobre Judá en el
cuarto año de Acab, rey de Israel. \footnote{\textbf{22:41} 1Re 15,24}
\bibleverse{42} Josafat tenía treinta y cinco años cuando comenzó a
reinar, y reinó veinticinco años en Jerusalén. Su madre se llamaba
Azubá, hija de Silí. \bibleverse{43} Siguió todo el camino de su padre
Asa. No se apartó de él, haciendo lo que era correcto a los ojos de
Yahvé. Sin embargo, los lugares altos no fueron quitados. El pueblo
seguía sacrificando y quemando incienso en los lugares altos.
\footnote{\textbf{22:43} 1Re 15,14; 2Re 12,3} \bibleverse{44} Josafat
hizo la paz con el rey de Israel.

\bibleverse{45} El resto de los hechos de Josafat, y el poderío que
mostró, y cómo luchó, ¿no están escritos en el libro de las crónicas de
los reyes de Judá? \footnote{\textbf{22:45} 2Cró 17,1-20}
\bibleverse{46} El resto de los sodomitas que quedaron en los días de su
padre Asa, él los expulsó del país. \footnote{\textbf{22:46} 1Re 15,12}
\bibleverse{47} No hubo rey en Edom. Gobernaba un suplente.
\bibleverse{48} Josafat hizo que los barcos de Tarsis fueran a buscar
oro a Ofir, pero no fueron, pues los barcos naufragaron en Ezión Geber.
\footnote{\textbf{22:48} 1Re 9,28} \bibleverse{49} Entonces Ocozías,
hijo de Ajab, dijo a Josafat: ``Deja que mis siervos vayan con los tuyos
en los barcos''. Pero Josafat no quiso. \bibleverse{50} Josafat durmió
con sus padres, y fue enterrado con sus padres en la ciudad de su padre
David. Su hijo Joram reinó en su lugar. \footnote{\textbf{22:50} 2Re
  8,16}

\bibleverse{51} Ocozías hijo de Acab comenzó a reinar sobre Israel en
Samaria en el año diecisiete de Josafat, rey de Judá, y reinó dos años
sobre Israel. \footnote{\textbf{22:51} 1Re 22,40}

\hypertarget{ochuxf4zuxedas-rey-de-israel}{%
\subsection{Ochôzías, rey de
Israel}\label{ochuxf4zuxedas-rey-de-israel}}

\bibleverse{52} Hizo lo malo ante los ojos de Yahvé, y anduvo en el
camino de su padre, y en el camino de su madre, y en el camino de
Jeroboam hijo de Nabat, en el cual hizo pecar a Israel. \footnote{\textbf{22:52}
  1Re 12,30} \bibleverse{53} Sirvió a Baal y lo adoró, y provocó la ira
de Yavé, el Dios de Israel, en todas las formas en que su padre lo había
hecho. \footnote{\textbf{22:53} 1Re 16,31-33}
