\hypertarget{el-nacimiento-y-ordenaciuxf3n-de-samuel-como-siervo-del-seuxf1or-en-silo-canciuxf3n-de-alabanza-de-hanna}{%
\subsection{El nacimiento y ordenación de Samuel como siervo del Señor
en Silo; Canción de alabanza de
Hanna}\label{el-nacimiento-y-ordenaciuxf3n-de-samuel-como-siervo-del-seuxf1or-en-silo-canciuxf3n-de-alabanza-de-hanna}}

\hypertarget{section}{%
\section{1}\label{section}}

\bibleverse{1} Había una vez un hombre de Ramataim de Zofim, en la
región montañosa de Efraín. Su nombre era Elcana, hijo de Jeroham, hijo
de Eliú, hijo de Tohu, hijo de Zuf, de la tribu de Efraín. \footnote{\textbf{1:1}
  1Cró 6,11-12; 1Cró 6,19-20} \bibleverse{2} Elcana tenía dos esposas.
El nombre de la primera esposa era Ana, y el de la segunda, Penina.
Penina tenía hijos, pero Ana no tenía ninguno. \footnote{\textbf{1:2}
  Gén 29,31} \bibleverse{3} Todos los años Elcana salía de su ciudad y
se iba a adorar y sacrificar al Señor Todopoderoso en Silo, donde los
dos hijos de Elí, Ofni y Finees, eran los sacerdotes del Señor.
\footnote{\textbf{1:3} Jos 18,1} \bibleverse{4} Cada vez que Elcana
ofrecía un sacrificio, daba porciones del mismo a Penina, su esposa, y a
todos sus hijos e hijas. \bibleverse{5} Y le daba una porción\footnote{\textbf{1:5}
  Al dar una porción extra, Elcanah estaba mostrando a todos que trataba
  a Ana como si tuviera un hijo.} extra a Ana, para mostrar su amor por
ella aunque el Señor no le había dado ningún hijo. \bibleverse{6} Su
rival -- la otra esposa -- se burlaba de ella para entristecerla porque
el Señor no le había dado hijos. \bibleverse{7} Esta situación duró
años, y cada vez que Ana iba al Templo del Señor, Penina se burlaba de
ella hasta que Ana lloraba y no podía comer. \bibleverse{8} Su esposo le
preguntaba: ``Ana, ¿por qué lloras? ¿Por qué no comes? ¿Por qué estás
tan alterada? ¿No soy mejor para ti que diez hijos?''

\hypertarget{los-votos-de-hanna-en-silo-y-su-conversaciuxf3n-con-eli}{%
\subsection{Los votos de Hanna en Silo y su conversación con
Eli}\label{los-votos-de-hanna-en-silo-y-su-conversaciuxf3n-con-eli}}

\bibleverse{9} En cierta ocasión, después haber comido y bebido en Silo,
Ana se levantó y se dirigió al Templo.\footnote{\textbf{1:9} ``Y se
  dirigió al Templo'': Añadido para mayor claridad.} El sacerdote Elí
estaba sentado en su silla junto a la entrada del Templo del Señor.
\bibleverse{10} Ana estaba terriblemente disgustada, y le oraba al Señor
mientras lloraba inconsolablemente. \bibleverse{11} Allí hizo un voto,
pidiendo: ``Señor Todopoderoso, si tan sólo te fijas en el sufrimiento
de tu sierva y te acuerdas de mí, y no me olvidas, sino que me das un
hijo, lo dedicaré al Señor durante toda su vida, y ninguna navaja de
afeitar tocará su cabeza''.

\bibleverse{12} Mientras Ana seguía orando ante el Señor, Elí observaba
su boca. \bibleverse{13} Ana oraba mentalmente, y aunque sus labios se
movían, su voz no producía ningún sonido, y por eso Elí pensó que debía
estar ebria. \bibleverse{14} ``¿Tienes que venir aquí estando ebria?'' ,
le preguntó. ``¡Ya deja el vino!''

\bibleverse{15} ``No es eso, mi señor'', le respondió Ana. ``Soy una
mujer muy desdichada. No he estado bebiendo vino ni cerveza; sólo estoy
derramando mi corazón ante el Señor. \footnote{\textbf{1:15} Sal 62,9}
\bibleverse{16} ¡Por favor, no pienses que soy una mala mujer! He estado
orando a causa de todos mis problemas y penas''.

\bibleverse{17} ``Ve en paz, y que el Dios de Israel te conceda lo que
le has pedido'', respondió Elí.

\bibleverse{18} ``Gracias por tu bondad con tu sierva'', dijo ella.
Luego siguió su camino, comió algo y ya no se veía triste.

\hypertarget{nacimiento-de-samuel-primera-infancia-y-consagraciuxf3n-en-silo}{%
\subsection{Nacimiento de Samuel, primera infancia y consagración en
Silo}\label{nacimiento-de-samuel-primera-infancia-y-consagraciuxf3n-en-silo}}

\bibleverse{19} A la mañana siguiente, Elcana y Ana se levantaron
temprano para adorar al Señor y luego se fueron a su casa en Ramá.
Elcana hizo el amor con su esposa Ana, y el Señor accedió a su petición.

\bibleverse{20} A su debido tiempo, Ana quedó embarazada y dio a luz un
hijo. Le puso el nombre de Samuel, diciendo: ``Porque se lo pedí al
Señor''.

\bibleverse{21} Elcana y toda su familia fueron a hacer el sacrificio
anual al Señor y a cumplir sus votos. \bibleverse{22} Pero Ana no fue.
Le dijo a su marido: ``Una vez destetado el niño, lo llevaré para
presentarlo al Señor y que se quede allí para siempre''.

\bibleverse{23} ``Haz lo que creas conveniente'', le respondió su marido
Elcana. ``Quédate aquí hasta que lo hayas destetado, y que el Señor
cumpla lo que ha dicho''.\footnote{\textbf{1:23} ``Lo que ha dicho'':
  refiriéndose al Señor. En la Septuaginta y en un rollo de Qumrán se
  lee ``lo que has dicho'', refiriéndose a Ana.} Así que Ana se quedó y
amamantó a su hijo hasta que lo destetó.

\bibleverse{24} Cuando hubo destetado al niño, Ana se lo llevó junto con
un novillo de tres años, un efa de harina y un odre con vino. Aunque el
niño era pequeño, lo llevó al Templo del Señor en Silo. \bibleverse{25}
Después de sacrificar el novillo, presentaron el niño a Elí.
\bibleverse{26} ``Por favor, mi señor'', dijo Ana, ``con toda seguridad,
mi señor, yo soy la mujer que estuvo aquí con usted orando al Señor.
\bibleverse{27} Yo oré por este niño, y como el Señor me ha dado lo que
le pedí, \footnote{\textbf{1:27} 1Sam 1,17} \bibleverse{28} ahora se lo
entrego al Señor. Mientras viva estará dedicado al Señor''.
Entoncesadoró\footnote{\textbf{1:28} ``Adoró'': Se presume que se
  refiere a Elcana. Algunas versiones cambian el pronombre y dice
  ``Adoraron''.} al Señor en ese lugar.\footnote{\textbf{1:28} 1Sam 1,11}

\hypertarget{himno-de-alabanza-a-hanna-inicio-de-servicio-de-samuel-en-silo}{%
\subsection{Himno de alabanza a Hanna; Inicio de servicio de Samuel en
Silo}\label{himno-de-alabanza-a-hanna-inicio-de-servicio-de-samuel-en-silo}}

\hypertarget{section-1}{%
\section{2}\label{section-1}}

\bibleverse{1} Ana oró: ``¡Estoy tan feliz en el Señor! ¡Él me ha dado
poder! Ahora tengo mucho que decir en respuesta a los que me odian.
¡Celebro su salvación! \footnote{\textbf{2:1} Luc 1,46-55}
\bibleverse{2} ¡No hay nadie santo como el Señor, nadie aparte de ti,
ninguna Roca como nuestro Dios! \bibleverse{3} ``¡No hables con tanta
arrogancia! ¡No hablen con tanta arrogancia! Porque el Señor es un Dios
que lo sabe todo: ¿acaso no juzga lo que hacen? \bibleverse{4} ``Las
armas\footnote{\textbf{2:4} ``Armas'': literalmente, ``arco''.} de los
poderosos son destrozadas, mientras que los que tropiezan se vuelven
fuertes. \bibleverse{5} Los que tenían mucha comida ahora tienen que
trabajar para ganarse un mendrugo, mientras que los que tenían hambre
ahora han engordado. La mujer que no tenía hijos ahora tiene siete,
mientras que la que tenía muchos se desvanece. \bibleverse{6} ``El Señor
mata y otorga vida; a unos los manda a la tumba, pero a otros los
resucita. \bibleverse{7} El Señor empobrece a unos, pero enriquece a
otros; abate a unos, pero levanta a otros. \footnote{\textbf{2:7} Sal
  75,8} \bibleverse{8} Ayuda a los pobres a levantarse del polvo; saca a
los humildes del muladar y los sienta con la clase alta en lugares de
gran honor. Porque los cimientos de la tierra son del Señor, y sobre
ellos ha colocado el mundo. \footnote{\textbf{2:8} Sal 113,7-8}
\bibleverse{9} ``Él cuidará de los que confían en él, pero los malvados
se desvanecen en las tinieblas, pues el hombre no puede triunfar por sus
propias fuerzas. \footnote{\textbf{2:9} Sal 33,16} \bibleverse{10} El
Señor aplasta a sus enemigos, truena desde el cielo contra ellos. Él
gobierna\footnote{\textbf{2:10} ``Gobierna'': o ``juzga''.} toda la
tierra; fortalece a su rey y otorga poder al que ha ungido''.
\footnote{\textbf{2:10} Sal 132,17}

\bibleverse{11} Entonces Elcana se fue a su casa en Ramá, mientras el
niño se quedó con el sacerdote Elí sirviendo al Señor.

\hypertarget{la-maldad-de-los-hijos-de-eluxed-anuncio-del-juicio-divino}{%
\subsection{La maldad de los hijos de Elí; Anuncio del juicio
divino}\label{la-maldad-de-los-hijos-de-eluxed-anuncio-del-juicio-divino}}

\bibleverse{12} Los hijos de Elí eran hombres inútiles que no tenían
tiempo para el Señor \bibleverse{13} ni para su función como sacerdotes
del pueblo. Enviaban a uno de sus siervos con un tenedor cuando alguien
venía a ofrecer un sacrificio. \footnote{\textbf{2:13} Éxod 27,3}
\bibleverse{14} El siervo metía el tenedor en la olla mientras se hervía
la carne del sacrificio, y les llevaba a los hijos de Elí la carne que
salía en el tenedor. Así trataban a todos los israelitas que llegaban a
Silo. \bibleverse{15} De hecho, incluso antes de que se quemara la grasa
del sacrificio, el sirviente venía y exigía al hombre que sacrificaba:
``Deme la carne para asarla para el sacerdote. Él no quiere la carne
hervida sino cruda''.

\bibleverse{16} El hombre podía responder: ``Déjame, primero quemar toda
la grasa, y luego puedes tener toda la que quieras''. Pero el criado del
sacerdote le contestaba: ``No, debes dármela ahora. Si no lo haces, la
tomaré por la fuerza''. \bibleverse{17} Los pecados de estos jóvenes
eran extremadamente graves ante los ojos del Señor, porque estaban
tratando las ofrendas del Señor con desprecio.

\hypertarget{hanna-y-el-niuxf1o-del-coro-samuel}{%
\subsection{Hanna y el niño del coro
Samuel}\label{hanna-y-el-niuxf1o-del-coro-samuel}}

\bibleverse{18} Pero Samuel servía ante el Señor: era un muchacho
vestido de sacerdote, con un efod de lino. \bibleverse{19} Cada año, su
madre le hacía un pequeño manto y se lo llevaba cuando iba con su marido
a ofrecer el sacrificio anual. \bibleverse{20} Elí bendecía a Elcana y a
su esposa, diciendo: ``Que el Señor le dé hijos de esta mujer para
reemplazar al que ella dedicó al Señor''. Luego regresaban a casa.
\bibleverse{21} Y el Señor bendijo\footnote{\textbf{2:21} ``Bendijo'':
  literalmente, ``Le presto atención''.} a Ana con tres hijos y dos
hijas. El niño Samuel creció en la presencia del Señor. \footnote{\textbf{2:21}
  Luc 1,80}

\hypertarget{las-suaves-amonestaciones-de-eluxed-a-sus-hijos-degenerados}{%
\subsection{Las suaves amonestaciones de Elí a sus hijos
degenerados}\label{las-suaves-amonestaciones-de-eluxed-a-sus-hijos-degenerados}}

\bibleverse{22} Elí era muy anciano, pero se había enterado de todas las
cosas que sus hijos hacían con el pueblo de Israel, y de cómo seducían a
las mujeres que servían a la entrada del Tabernáculo de Reunión.
\footnote{\textbf{2:22} Éxod 38,8} \bibleverse{23} Entonces les
preguntó: ``¿Por qué se comportan de esta manera? Sigo oyendo las quejas
de todo el mundo por sus malas acciones. \bibleverse{24} No, hijos míos,
lo que escucho sobre ustedes de parte del pueblo del Señor no es bueno.
\bibleverse{25} Si un hombre peca contra alguien, Dios puede interceder
por él; pero si un hombre peca contra el Señor, ¿quién podrá interceder
por él?'' Pero no prestaron atención a lo que les dijo su padre, pues el
Señor planeaba darles muerte.

\bibleverse{26} El niño Samuel crecía en estatura, y también crecía en
cuanto a la aprobación del Señor y del pueblo. \footnote{\textbf{2:26}
  Luc 2,52}

\hypertarget{refruxe1n-del-profeta-anuncio-de-la-cauxedda-de-eli-y-su-casa}{%
\subsection{Refrán del Profeta: Anuncio de la caída de Eli y su
casa}\label{refruxe1n-del-profeta-anuncio-de-la-cauxedda-de-eli-y-su-casa}}

\bibleverse{27} Un hombre de Dios se acercó a Elí y le dijo: ``Esto es
lo que dice el Señor: ¿Acaso no me revelé claramente a la familia de tu
antepasado cuando era gobernado por el faraón en Egipto? \bibleverse{28}
Yo lo elegí\footnote{\textbf{2:28} Refiiriéndose a Aarón.} de todas las
tribus de Israel como mi sacerdote, para ofrecer sacrificios en mi
altar, para quemar incienso y llevar un efod en mi presencia. También le
di a la familia de tu antepasado todos los holocaustos de los
israelitas. \bibleverse{29} ¿Por qué, entonces, has tratado con
desprecio mis sacrificios y las ofrendas que he ordenado para mi lugar
de culto? Ustedes honran más a sus hijos que a mí, se engordan ustedes
con las mejores partes de todas las ofrendas de mi pueblo Israel.
\bibleverse{30} ``En consecuencia, esta es la declaración del Señor:
Hice la promesa definitiva de que tu familia y la de tu padre me
servirían siempre como sacerdotes. Pero ahora el Señor declara: ¡Ya no
más! En cambio, honraré a los que me honran, pero trataré con desprecio
a los que me desprecian. \footnote{\textbf{2:30} Éxod 28,1}
\bibleverse{31} Se acerca el momento en que pondré fin a tu familia y a
la de tu padre.\footnote{\textbf{2:31} ``Pondré fin a tu familia'':
  literalmente, ``cortaré tu fuerza''.} Nadie vivirá hasta la vejez.
\footnote{\textbf{2:31} 1Re 2,27} \bibleverse{32} Verás tragedia en el
lugar de adoración.\footnote{\textbf{2:32} Quizás refiriéndose a la
  pérdida del Arca a manos de los filisteos.} Mientras Israel prospere,
ninguno en tu familia volverá a vivir hasta la vejez. \bibleverse{33}
Cualquiera de tu familia que no haya sido apartado para servir en mi
altar, te hará llorar y te causará dolor. Todos tus descendientes
morirán aún estando llenos de vida. \footnote{\textbf{2:33} 1Sam 22,20}
\bibleverse{34} He aquí una señal para ti de que esto sucederá con
respecto a tus dos hijos Ofni y Finees: ambos morirán el mismo día.
\footnote{\textbf{2:34} 1Sam 4,11}

\bibleverse{35} Yo elegiré para mí a un sacerdote digno de confianza que
hará lo que realmente quiero, lo que tengo en mente. Me aseguraré de que
él y sus descendientes sean dignos de confianza y que siempre sirvan a
mi ungido. \bibleverse{36} Cada uno de tus descendientes que quede
vendrá y se inclinará ante él, pidiendo dinero y comida, diciendo: `Por
favor, dame trabajo como sacerdote para que pueda tener comida'''.

\hypertarget{dios-se-revela-a-samuel-y-anuncia-la-cauxedda-de-la-casa-de-eluxed}{%
\subsection{Dios se revela a Samuel y anuncia la caída de la casa de
Elí}\label{dios-se-revela-a-samuel-y-anuncia-la-cauxedda-de-la-casa-de-eluxed}}

\hypertarget{section-2}{%
\section{3}\label{section-2}}

\bibleverse{1} El niño Samuel servía ante el Señor bajo la supervisión
de Elí. En aquella época no se escuchaba un mensaje del Señor con
frecuencia, y las visiones no eran comunes. \footnote{\textbf{3:1} Am
  8,11} \bibleverse{2} Una noche, Elí se había ido a acostar en su
habitación. Sus ojos estaban tan débiles que no podía ver.
\bibleverse{3} La lámpara de Dios aún no se había apagado, y Samuel
estaba durmiendo en el Templo del Señor, donde estaba el Arca de Dios.
\bibleverse{4} Entonces el Señor lo llamó: ``¡Samuel!'' Samuel entonces
respondió: ``Aquí estoy''.

\bibleverse{5} Entonces corrió hacia Elí y le dijo: ``Aquí estoy ¿Me
llamabas?'' . ``Yo no te he llamado'', respondió Elí. ``Vuelve a la
cama''. Así que Samuel volvió a la cama.

\bibleverse{6} Entonces el Señor volvió a llamar: ``¡Samuel!''. Y Samuel
se levantó, fue a ver a Elí y le dijo: ``Aquí estoy ¿Me llamabas?'' .
``Yo no te he llamado, hijo mío'', respondió Elí. ``Vuelve a la cama''.

\bibleverse{7} (Samuel aún no había llegado a conocer al Señor y no
había recibido ningún mensaje de él). \bibleverse{8} El Señor volvió a
llamar por tercera vez: ``¡Samuel!''. Éste se levantó, fue a ver a Elí y
le dijo: ``Aquí estoy ¿Me llamabas?'' . Entonces Elí se dio cuenta de
que era el Señor quien llamaba al muchacho.

\bibleverse{9} Así que Elí le dijo a Samuel: ``Vuelve a la cama, y si
escuchas el llamado, dile: `Habla, Señor, porque tu siervo te
escucha'\,''. Así que Samuel volvió a su cama. \bibleverse{10} El Señor
llegó y se quedó allí, llamando igual que antes: ``¡Samuel! Samuel!''
Entonces Samuel respondió: ``Habla, porque tu siervo te escucha''.

\bibleverse{11} El Señor le dijo entonces a Samuel: ``Presta atención,
porque voy a hacer algo en Israel que sorprenderá a todos los que lo
escuchen.\footnote{\textbf{3:11} ``Sorprenderá a todos los que lo
  escuchen'': literalmente, ``estremecerá los oídos de todos los que lo
  escuchen''.} \bibleverse{12} Es entonces cuando cumpliré todo lo que
he dicho, de principio a fin, contra Elí y su familia. \bibleverse{13}
Le dije que juzgaré a su familia para siempre por los pecados que él
conoce, porque sus hijos blasfemaron contra Dios y él no trató de
detenerlos. \bibleverse{14} Por eso le juré a Elí y a su familia: `La
culpa de Elí y de sus descendientes no se quitará nunca con sacrificios
ni con ofrendas'\,''.

\hypertarget{samuel-comparte-la-revelaciuxf3n-con-eluxed-y-comienza-su-trabajo-como-profeta-para-todo-israel}{%
\subsection{Samuel comparte la revelación con Elí y comienza su trabajo
como profeta para todo
Israel}\label{samuel-comparte-la-revelaciuxf3n-con-eluxed-y-comienza-su-trabajo-como-profeta-para-todo-israel}}

\bibleverse{15} Samuel permaneció en la cama hasta la mañana. Luego se
levantó y abrió las puertas del Templo del Señor como de costumbre.
Tenía miedo de contarle a Elí la visión. \bibleverse{16} Pero Elí lo
llamó y le dijo: ``Samuel, hijo mío''. ``Aquí estoy'', respondió Samuel.

\bibleverse{17} ``¿Qué te ha dicho?'' preguntó Elí. ``No me lo ocultes.
Que Dios te castigue muy severamente si me ocultas algo de lo que te
dijo''.

\bibleverse{18} Así que Samuel le contó todo y no le ocultó nada. ``Es
el Señor'', respondió Elí. ``Que haga lo que le parezca bien''.
\footnote{\textbf{3:18} 2Sam 15,26}

\bibleverse{19} Samuel siguió creciendo. El Señor estaba con él y se
aseguraba de que todo lo que decía era fiel. \bibleverse{20} Todos en
todo Israel, desde Dan hasta Beerseba, reconocían que Samuel era un
profeta del Señor digno de confianza. \bibleverse{21} El Señor siguió
apareciendo en Silo, porque allí se revelaba a Samuel y le entregaba sus
mensajes,

\hypertarget{die-bundeslade-ins-lager-der-israeliten-geholt}{%
\subsection{Die Bundeslade ins Lager der Israeliten
geholt}\label{die-bundeslade-ins-lager-der-israeliten-geholt}}

\hypertarget{section-3}{%
\section{4}\label{section-3}}

\bibleverse{1} y las palabras de Samuel se comunicaron a todos los
israelitas. Los israelitas marcharon para enfrentarse a los filisteos en
la batalla. Acamparon en Ebenezer, mientras los filisteos lo hacían en
Afec.

\bibleverse{2} Los filisteos atacaron a los israelitas en formación, y
cuando la batalla se extendió, los filisteos derrotaron a los
israelitas, matando a 4. 000 de ellos en el campo de batalla.
\bibleverse{3} Cuando el ejército israelita regresó al campamento, los
ancianos de Israel preguntaron: ``¿Por qué el Señor nos ha derrotado hoy
ante los filisteos? Vayamos a buscar el Arca del Pacto del Señor a Silo,
para que nos acompañe y nos salve de nuestros enemigos''. \footnote{\textbf{4:3}
  1Sam 14,18}

\bibleverse{4} Así que el ejército envió hombres a Silo, y trajeron de
vuelta el Arca del Pacto del Señor Todopoderoso, el que está sentado en
su trono entre los querubines. Ofni y Finees, los dos hijos de Elí,
estaban allí con el Arca del Pacto del Señor. \footnote{\textbf{4:4}
  2Sam 6,2}

\hypertarget{el-efecto-de-este-evento-en-las-partes-en-conflicto-derrota-de-los-israelitas-y-puxe9rdida-del-arca}{%
\subsection{El efecto de este evento en las partes en conflicto; Derrota
de los israelitas y pérdida del
arca}\label{el-efecto-de-este-evento-en-las-partes-en-conflicto-derrota-de-los-israelitas-y-puxe9rdida-del-arca}}

\bibleverse{5} Cuando el Arca del Pacto del Señor llegó al campamento,
todos los israelitas dieron un grito tan fuerte que hizo temblar el
suelo. \bibleverse{6} Cuando los filisteos oyeron todo el griterío,
preguntaron: ``¿Qué significa este griterío en el campamento
israelita?'' Cuando se enteraron de que el Arca del Señor había llegado
al campamento, \bibleverse{7} los filisteos se asustaron. ``Un dios ha
llegado al campamento'', dijeron. ``Estamos en problemas, pues nunca
antes había sucedido algo así. \bibleverse{8} ¡Esto es un desastre para
nosotros! ¿Quién nos salvará del poder de estos poderosos dioses? Estos
son los dioses que atacaron a los egipcios con toda clase de plagas en
el desierto. \bibleverse{9} ¡Sean valientes y luchen como verdaderos
hombres, filisteos! De lo contrario, terminarán como esclavos de los
israelitas, tal como ellos fueron sus esclavos. Sean hombres de verdad y
luchen''. \footnote{\textbf{4:9} Jue 13,1} \bibleverse{10} Así que los
filisteos lucharon, y los israelitas fueron derrotados: cada uno huyó a
su casa. El número de muertos fue muy grande: treinta mil de la
infantería israelita murieron. \bibleverse{11} El Arca de Dios fue
capturada y murieron Ofni y Finees, los dos hijos de Elí.

\hypertarget{los-tristes-efectos-del-mensaje-en-shiloh-la-muerte-de-eli-y-su-nuera}{%
\subsection{Los tristes efectos del mensaje en Shiloh; la muerte de Eli
y su
nuera}\label{los-tristes-efectos-del-mensaje-en-shiloh-la-muerte-de-eli-y-su-nuera}}

\bibleverse{12} Un hombre de la tribu de Benjamín huyó aquel día de la
batalla hasta Silo. Su ropa estaba rota y tenía tierra en la
cabeza.\footnote{\textbf{4:12} ``Su ropa estaba rota y tenía tierra en
  la cabeza''. Esto simbolizaba una gran angustia.} \bibleverse{13}
Cuando llegó, Elí estaba sentado en su silla junto al camino, atento a
las noticias porque estaba preocupado por el Arca de Dios. Cuando el
hombre llegó a la ciudad y dio su informe, todo el pueblo lloró a
gritos. \bibleverse{14} Elí oyó el llanto y preguntó: ``¿Qué es todo
este ruido?'' . El hombre corrió hacia Elí y le contó lo que había
sucedido.

\bibleverse{15} Elí tenía noventa y ocho años, y sus ojos estaban fijos
porque no podía ver. \bibleverse{16} ``Acabo de llegar de la batalla'',
dijo el hombre. ``Hoy he huido de ella''. ``¿Qué pasó, hijo mío?''
preguntó Elí.

\bibleverse{17} ``Israel huyó de los filisteos; fuimos derrotados'',
respondió el mensajero. ``También tus dos hijos, Ofni y Finees, fueron
asesinados, y el Arca de Dios ha sido capturada''.

\bibleverse{18} En cuanto se mencionó el Arca de Dios, Elí cayó de
espaldas de su silla junto a la puerta de la ciudad. Como era viejo y
pesado, se rompió la nuca y murió. Elí había sido el líder de Israel
durante cuarenta años.

\bibleverse{19} Su nuera, la esposa de Finees, estaba embarazada y a
punto de dar a luz. Cuando escuchó la noticia de que el Arca de Dios
había sido capturada, y que su suegro y su marido habían muerto, se puso
de parto y dio a luz, pero sus dolores de parto fueron demasiado
fuertes. \bibleverse{20} Y justo antes de morir, las mujeres que la
atendían le dijeron: ``No te rindas, has dado a luz un hijo''. Pero ella
no contestó ni dio ninguna respuesta. \footnote{\textbf{4:20} Gén 35,17}
\bibleverse{21} Entonces llamó al niño Icabod, diciendo: ``La gloria se
ha ido de Israel'', porque el Arca de Dios había sido capturada, y su
suegro y su marido habían muerto. \footnote{\textbf{4:21} Sal 78,61}

\bibleverse{22} Ella dijo: ``La gloria ha dejado a Israel, porque el
Arca de Dios ha sido capturada''.

\hypertarget{en-la-tierra-de-los-filisteos-el-arca-estuxe1-causando-estragos-en-varias-ciudades}{%
\subsection{En la tierra de los filisteos, el arca está causando
estragos en varias
ciudades}\label{en-la-tierra-de-los-filisteos-el-arca-estuxe1-causando-estragos-en-varias-ciudades}}

\hypertarget{section-4}{%
\section{5}\label{section-4}}

\bibleverse{1} Después de que los filisteos capturaron el Arca de Dios,
la llevaron de Ebenezer a Asdod. \bibleverse{2} Llevaron el Arca de Dios
al Templo de Dagón y la colocaron junto a Dagón. \footnote{\textbf{5:2}
  Jue 16,23} \bibleverse{3} Cuando el pueblo de Asdod se levantó
temprano al día siguiente, vio que Dagón había caído de bruces frente al
Arca del Señor. Así que tomaron a Dagón y lo volvieron a colocar.
\bibleverse{4} Cuando se levantaron temprano a la mañana siguiente,
vieron que Dagón había caído de bruces frente al Arca del Señor, con la
cabeza y las manos rotas, tirado en el umbral. Sólo su cuerpo permanecía
intacto. \bibleverse{5} (Por eso los sacerdotes de Dagón, y todos los
que entran en el templo de Dagón en Asdod, no pisan el umbral, ni
siquiera hasta ahora). \bibleverse{6} El Señor castigó\footnote{\textbf{5:6}
  ``El Señor castigó'': literalmente, ``La mano del Señor fue pesada''.}
a los habitantes de Asdod y sus alrededores, devastándolos y plagándolos
de hinchazones.\footnote{\textbf{5:6} Algunos piensan que estas
  ``hinchazones'' o ``tumores'' estaban relacionados con la peste
  bubónica. La Septuaginta añade al final de este versículo: ``y las
  ratas pululaban por toda la tierra, y había muerte y destrucción en la
  ciudad''.}

\bibleverse{7} Cuando los habitantes de Asdod vieron lo que sucedía,
dijeron: ``No podemos dejar que el Arca del Dios de Israel se quede aquí
con nosotros, porque nos está castigando a nosotros y a Dagón, nuestro
dios''. \bibleverse{8} Así que mandaron llamar a todos los gobernantes
filisteos y les preguntaron: ``¿Qué debemos hacer con el Arca del Dios
de Israel?'' ``Lleven el Arca del Dios de Israel a Gat'', respondieron.
Así que la trasladaron a Gat.

\bibleverse{9} Pero una vez que trasladaron el Arca a Gat, el Señor
también actuó contra esa ciudad, sumiéndola en una gran confusión y
atacando a la gente de la ciudad, jóvenes y ancianos, con una plaga de
hinchazones. \bibleverse{10} Entonces enviaron el Arca de Dios a Ecrón,
pero en cuanto llegó, los dirigentes de Ecrón gritaron: ``¡Han
trasladado aquí el Arca del Dios de Israel para matarnos a nosotros y a
nuestro pueblo!''

\bibleverse{11} Así que mandaron llamar a todos los gobernantes
filisteos y les dijeron: ``Que el Arca del Dios de Israel se vaya,
vuelva al lugar de donde vino, porque si no nos va a matar a nosotros y
a nuestro pueblo''. La gente moría en toda la ciudad, creando un pánico
terrible, pues el castigo de Dios era muy duro. \bibleverse{12} Los que
no morían estaban plagados de hinchazones, y el grito de auxilio del
pueblo llegaba hasta el cielo.

\hypertarget{resoluciuxf3n-de-los-filisteos-sobre-el-regreso-del-arca}{%
\subsection{Resolución de los filisteos sobre el regreso del
arca}\label{resoluciuxf3n-de-los-filisteos-sobre-el-regreso-del-arca}}

\hypertarget{section-5}{%
\section{6}\label{section-5}}

\bibleverse{1} Después de que el Arca del Señor estuvo en el país de los
filisteos durante siete meses, \bibleverse{2} los filisteos convocaron a
los sacerdotes y adivinos y les preguntaron: ``¿Qué debemos hacer con el
Arca del Señor? Explíquennos cómo devolverla al lugar de donde vino''.

\bibleverse{3} ``Si van a enviar de vuelta el Arca del Dios de Israel,
no la envíen con las manos vacías, sino asegúrense de enviar junto con
ella un regalo de ofrenda por la culpa para él'', respondieron.
``Entonces serán sanados y entenderán por qué los ha tratado así''.

\bibleverse{4} ``¿Qué clase de ofrenda por la culpa debemos enviarle?''
, preguntaron los filisteos. ``Cinco objetos de oro en forma de tumor y
cinco ratas de oro que representen el número de gobernantes de los
filisteos'', respondieron. ``La misma plaga los atacó a ustedes y a sus
gobernantes. \footnote{\textbf{6:4} Jos 13,3}

\bibleverse{5} Haz modelos que representen tus hinchazones y las ratas
que destruyen el país, y honra al Dios de Israel. Tal vez deje de
castigarte a ti, a tus dioses y a tu tierra. \bibleverse{6} ¿Por qué ser
tercos como los egipcios y el faraón? ¿Acaso cuando Dios los castigó no
dejaron ir a los israelitas para seguir su camino?

\bibleverse{7} ``Así que preparen un nuevo carro, tirado por dos vacas
con crías y que nunca hayan sido uncidas. Aten las vacas al carro, pero
quiten sus terneros y pónganlos en un establo.\footnote{\textbf{6:7} El
  propósito de esto era forzar a las vacas a hacer algo inusual dejando
  voluntariamente sus terneros. De este modo, el pueblo estaría seguro
  de que esta acción contaba con la aprobación de Dios si la hacía
  realidad.} \bibleverse{8} Recojan el Arca del Señor, pónganla en el
carro y coloquen los objetos de oro que envían como ofrenda por la culpa
en un cofre junto a ella. Luego envíen el Arca. Dejen que se vaya por
donde quiera, \bibleverse{9} pero no dejen de vigilarla. Si sube por el
camino hacia su patria, hacia Bet-Semes, entonces es el Señor quien nos
ha causado todo este terrible problema. Pero si no lo hace, entonces
sabremos que no fue él quien nos castigó, sino que nos ocurrió por
casualidad''.

\hypertarget{ejecuciuxf3n-de-la-resoluciuxf3n-llegada-y-recepciuxf3n-del-arca-en-bet-semes}{%
\subsection{Ejecución de la resolución; Llegada y recepción del arca en
Bet-semes}\label{ejecuciuxf3n-de-la-resoluciuxf3n-llegada-y-recepciuxf3n-del-arca-en-bet-semes}}

\bibleverse{10} Entonces el pueblo lo hizo así. Tomaron dos vacas con
crías y las ataron al carro, y guardaron sus terneros en un establo.
\bibleverse{11} Pusieron el Arca del Señor en el carro, junto con el
cofre que contenía las ratas de oro y los modelos de sus hinchazones.
\bibleverse{12} Las vacas subieron en línea recta por el camino de
Bet-Semes, mugiendo mientras avanzaban, yendo directamente por el camino
principal y sin girar ni a la izquierda ni a la derecha. Los jefes
filisteos las siguieron hasta la frontera de Bet-Semes. \bibleverse{13}
Los habitantes de Bet-semes estaban cosechando trigo en el valle. Cuando
levantaron la vista y vieron el Arca, se alegraron mucho de verla.
\bibleverse{14} El carro entró en el campo de Josué de Bet-semes y se
detuvo allí junto a una gran roca. El pueblo cortó la madera del carro y
sacrificaron las vacas como holocausto al Señor. \bibleverse{15} Los
levitas bajaron el Arca del Señor y el cofre que contenía los objetos de
oro, y los pusieron sobre la gran roca. El pueblo de Bet-semes presentó
holocaustos e hizo sacrificios al Señor ese día. \bibleverse{16} Los
cinco jefes filisteos vieron todo lo que sucedió y regresaron a Ecrón
ese mismo día. \bibleverse{17} Los cinco modelos de oro de las hinchadas
enviados por los filisteos como ofrenda de culpa al Señor eran de los
gobernantes de Asdod, Gaza, Ascalón, Gat y Ecrón. \bibleverse{18} Las
ratas de oro representaban el número de ciudades filisteas de los cinco
gobernantes: las ciudades fortificadas y sus aldeas circundantes. La
gran roca sobre la que colocaron el Arca del Señor sigue en pie hasta el
día de hoy en el campo de Josué de Bet-semes como testigo de lo que allí
ocurrió.

\hypertarget{se-instala-el-arca-en-quiriat-jearim}{%
\subsection{Se instala el arca en
Quiriat-Jearim}\label{se-instala-el-arca-en-quiriat-jearim}}

\bibleverse{19} Pero Dios mató a algunos de los habitantes de Bet-semes
porque revisaron el interior del Arca del Señor. Mató a
setenta,\footnote{\textbf{6:19} Algunos manuscritos parecen decir 50.
  070, pero esta es una cifra improbable para un pequeño asentamiento.}
y el pueblo se lamentó profundamente porque el Señor había matado a
tantos. \footnote{\textbf{6:19} Núm 4,20; 2Sam 6,6-7} \bibleverse{20} El
pueblo de Bet-semes preguntó: ``¿Quién puede estar frente al Señor, este
Dios santo? ¿Adónde debe ir el Arca de aquí en adelante?''

\bibleverse{21} Entonces enviaron mensajeros al pueblo de Quiriat-jearim
para decirles: ``Los filisteos han devuelto el Arca del Señor.
Desciendan y llévensela a casa''.

\hypertarget{section-6}{%
\section{7}\label{section-6}}

\bibleverse{1} Entonces el pueblo de Quiriat-jearim vino y se apropió
del Arca del Señor. La pusieron en la casa de Abinadab, en la colina.
Designaron a su hijo Eleazar para cuidar el Arca del Señor.

\hypertarget{los-israelitas-se-vuelven-arrepentidos-a-dios}{%
\subsection{Los israelitas se vuelven arrepentidos a
Dios}\label{los-israelitas-se-vuelven-arrepentidos-a-dios}}

\bibleverse{2} El Arca permaneció allí, en Quiriat-jearim, desde aquel
día, durante mucho tiempo, hasta veinte años. Todos en Israel se
lamentaron y, arrepentidos, volvieron al Señor. \bibleverse{3} Entonces
Samuel le dijo a todo Israel: ``Si desean sinceramente volver al Señor,
desháganse de los dioses extranjeros y de las imágenes de Astoret,
entréguense al Señor y adórenlo sólo a él, y él los salvará de los
filisteos''. \footnote{\textbf{7:3} Gén 35,2; Jos 24,23} \bibleverse{4}
El pueblo de Israel se deshizo de sus baales e imágenes de Astoret y
sólo adoró al Señor. \footnote{\textbf{7:4} Jue 10,6; Jue 10,16}

\hypertarget{la-intercesiuxf3n-y-el-sacrificio-de-samuel-por-israel-en-mizpa-derrota-de-los-filisteos-la-piedra-eben-eser}{%
\subsection{La intercesión y el sacrificio de Samuel por Israel en
Mizpa; Derrota de los filisteos; la piedra
Eben-Eser}\label{la-intercesiuxf3n-y-el-sacrificio-de-samuel-por-israel-en-mizpa-derrota-de-los-filisteos-la-piedra-eben-eser}}

\bibleverse{5} Entonces Samuel dijo: ``Que todo el pueblo de Israel se
reúna en Mizpa, y yo oraré al Señor por ustedes''. \footnote{\textbf{7:5}
  1Sam 10,17; Jue 11,11; Jue 20,1} \bibleverse{6} Una vez reunidos en
Mizpa, sacaron agua y la derramaron ante el Señor. Ese día ayunaron y
reconocieron: ``Hemos pecado contra el Señor''. Y Samuel se convirtió en
el líder\footnote{\textbf{7:6} Literalmente ``juez'', que era el
  equivalente a ``líder''. Ver también, vers. 15.} de los israelitas en
Mizpa.

\bibleverse{7} Cuando los filisteos se enteraron de que los israelitas
se habían reunido en Mizpa, sus gobernantes dirigieron un ataque contra
Israel. Cuando los israelitas se enteraron de esto, se aterraron por lo
que los filisteos podrían hacer. \bibleverse{8} Le dijeron a Samuel:
``No dejes de rogarle al Señor nuestro Dios por nosotros, para que nos
salve de los filisteos''. \bibleverse{9} Samuel tomó un cordero joven y
lo presentó como holocausto completo al Señor. Clamó al Señor por ayuda
para Israel, y el Señor le respondió. \bibleverse{10} Mientras Samuel
presentaba el holocausto, los filisteos se acercaron para atacar a
Israel. Pero aquel día el Señor tronó muy fuerte contra los filisteos,
lo que los confundió totalmente, y fueron derrotados ante la mirada de
Israel. \bibleverse{11} Entonces los hombres de Israel salieron
corriendo de Mizpa y los persiguieron, matándolos hasta llegar a un
lugar cercano a Bet-car.

\bibleverse{12} Después de esto, Samuel tomó una piedra y la colocó
entre Mizpa y Sen.~La llamó Ebenezer, diciendo: ``¡El Señor nos ayudó
hasta aquí!''.

\hypertarget{estado-de-paz-en-el-pauxeds-la-eficacia-de-samuel-como-juez}{%
\subsection{Estado de paz en el país; La eficacia de Samuel como
juez}\label{estado-de-paz-en-el-pauxeds-la-eficacia-de-samuel-como-juez}}

\bibleverse{13} Fue así como los filisteos se mantuvieron bajo control y
no volvieron a invadir Israel. A lo largo de la vida de Samuel, el Señor
usó su poder contra los filisteos.

\bibleverse{14} Los filisteos le devolvieron a Israel las ciudades que
les habían arrebatado, desde Ecrón hasta Gat, e Israel también liberó el
territorio vecino de manos de los filisteos. También hubo paz entre
Israel y los amorreos.

\bibleverse{15} Y Samuel fue el líder de Israel por el resto de su vida.
\bibleverse{16} Todos los años recorría el país, yendo a Betel, Gilgal y
Mizpa. En todos estos lugares atendía los asuntos de Israel.
\bibleverse{17} Luego regresaba a Ramá, porque allí vivía. Desde allí
gobernaba a Israel, y también construyó un altar para el Señor.

\hypertarget{el-deseo-de-israel-por-un-rey-la-demanda-del-pueblo-despierta-el-disgusto-de-samuel-pero-encuentra-la-aprobaciuxf3n-de-dios}{%
\subsection{El deseo de Israel por un rey; La demanda del pueblo
despierta el disgusto de Samuel, pero encuentra la aprobación de
Dios}\label{el-deseo-de-israel-por-un-rey-la-demanda-del-pueblo-despierta-el-disgusto-de-samuel-pero-encuentra-la-aprobaciuxf3n-de-dios}}

\hypertarget{section-7}{%
\section{8}\label{section-7}}

\bibleverse{1} Cuando Samuel envejeció, nombró a sus hijos como
jefes\footnote{\textbf{8:1} Nuevamente la palabra utilizada es
  ``jueces'', pero en este período de la historia de Israel, antes de
  que tuvieran reyes, los jueces no sólo resolvían casos legales, sino
  que actuaban como gobernantes.} de Israel. \footnote{\textbf{8:1} 1Cró
  6,13} \bibleverse{2} Su primer hijo se llamaba Joel, y su segundo hijo
se llamaba Abías. Ambos fueron gobernantes en Beerseba. \bibleverse{3}
Sin embargo, sus hijos no siguieron su camino. Eran corruptos, ganaban
dinero aceptando sobornos y pervertían la justicia.

\bibleverse{4} Así que los ancianos de Israel se reunieron y fueron a
buscar a Samuel a Ramá. \footnote{\textbf{8:4} 1Sam 7,17} \bibleverse{5}
``Mira'' -- le dijeron -- ``tú ya eres viejo y tus hijos no siguen tus
caminos. Elige un rey que nos gobierne como a todas las demás
naciones''. \footnote{\textbf{8:5} Deut 17,14; Os 13,10; Hech 13,21}
\bibleverse{6} A Samuel le pareció que era una mala idea cuando le
dijeron: ``Danos un rey que nos gobierne'', así que oró al Señor al
respecto.

\bibleverse{7} ``Haz lo que el pueblo te diga'', le dijo el Señor a
Samuel, ``porque no es a ti a quien rechazan, sino a mí como su rey.
\bibleverse{8} Están haciendo lo mismo que siempre han hecho desde que
los saqué de Egipto hasta ahora. Me han abandonado y han adorado a otros
dioses, y lo mismo están haciendo contigo. \bibleverse{9} Así que haz lo
que quieran, pero dales una advertencia solemne: explícales lo que hará
un rey cuando los gobierne''.

\hypertarget{samuel-le-dice-a-la-gente-los-derechos-de-un-rey}{%
\subsection{Samuel le dice a la gente los derechos de un
rey}\label{samuel-le-dice-a-la-gente-los-derechos-de-un-rey}}

\bibleverse{10} Samuel repitió delante de todo el pueblo todo lo que el
Señor le había dicho en cuanto al pueblo pidiéndole que les diera un
rey. \bibleverse{11} Entonces les dijo: ``Esto es lo que hará un rey
cuando gobierne sobre Israel: Tomará a sus hijos y los hará servir como
soldados y jinetes, y para que corran como guardia delante de sus
propios carruajes. \bibleverse{12} A algunos de los asignará como
comandantes de millares y comandantes de cincuentenas, y otros tendrán
que arar sus campos y segar su cosecha. A algunos los destinará a
fabricar armas y equipos para los carros de guerra. \bibleverse{13}
Tomará a las hijas de ustedes y las hará trabajar como perfumistas,
cocineras y panaderas. \bibleverse{14} Tomará de entre ustedes los
mejores campos, viñedos y olivares y se los dará a sus funcionarios.
\bibleverse{15} Tomará la décima parte de las cosechas de grano de
ustedes, así como el producto de sus viñedos y la asignará a sus jefes y
funcionarios. \bibleverse{16} Tomará a los siervos y siervas de ustedes,
así como a sus mejores jóvenes y asnos, y los pondrá a trabajar para él.
\bibleverse{17} Tomará la décima parte de los rebaños de ustedes, y
ustedes mismos serán ahora sus esclavos. \bibleverse{18} Ese día ustedes
suplicarán ser rescatados del rey que han elegido, pero el Señor no les
responderá''.

\hypertarget{la-gente-persiste-en-su-demanda-la-aprobaciuxf3n-de-dios}{%
\subsection{La gente persiste en su demanda; La aprobación de
dios}\label{la-gente-persiste-en-su-demanda-la-aprobaciuxf3n-de-dios}}

\bibleverse{19} Pero el pueblo se negó a escuchar lo que Samuel decía.
``¡No!'', insistieron. ``¡Queremos nuestro propio rey! \bibleverse{20}
Así podremos ser como las demás naciones. Nuestro rey nos gobernará y
nos guiará cuando salgamos a pelear nuestras batallas''.

\bibleverse{21} Samuel escuchó todo lo que el pueblo decía y se lo
repitió al Señor. \bibleverse{22} Entonces el Señor le dijo a Samuel:
``Haz lo que ellos te piden y dales un rey''. Entonces Samuel les dijo a
los israelitas: ``Vuelvan a sus casas''.\footnote{\textbf{8:22} 1Sam
  8,7; 1Sam 8,9}

\hypertarget{sauxfal-llega-a-la-casa-de-samuel-en-busca-de-los-asnos-de-su-padre}{%
\subsection{Saúl llega a la casa de Samuel en busca de los asnos de su
padre}\label{sauxfal-llega-a-la-casa-de-samuel-en-busca-de-los-asnos-de-su-padre}}

\hypertarget{section-8}{%
\section{9}\label{section-8}}

\bibleverse{1} Había un hombre rico e influyente de la tribu de
Benjamín, que se llamaba Cis, hijo de Abiel, hijo de Zeror, hijo de
Becorat, hijo de Afía, descendiente de la tribu de Benjamín.
\bibleverse{2} Cis tenía un hijo llamado Saúl. Este era el joven más
guapo de todo Israel. Era más alto que cualquier otro.

\bibleverse{3} En cierta ocasión, los burros del padre de Saúl, Cis, se
extraviaron. Cis le dijo a su hijo Saúl: ``Por favor, ve a buscar los
burros. Puedes llevar a uno de los siervos contigo''. \bibleverse{4}
Saúl buscó en la región montañosa de Efraín y luego en la tierra de
Salisa, pero no encontró los burros. Entonces buscaron en la región de
Saalim, pero tampoco estaban allí. Luego buscaron en la tierra de
Benjamín, pero tampoco pudieron encontrarlos allí.

\bibleverse{5} Cuando llegaron a la tierra de Zuf, Saúl le dijo a su
criado: ``Vamos, volvamos, porque si no mi padre no se preocupará
solamente por los burros, sino también por nosotros''. \footnote{\textbf{9:5}
  1Sam 10,2}

\bibleverse{6} Pero el criado le respondió: ``¡Espera! Hay un hombre de
Dios en esta ciudad. Tiene muy buena fama, y todo lo que dice se cumple.
Vamos a verle. Tal vez él pueda decirnos qué camino debemos tomar''.

\bibleverse{7} ``Pero si vamos, ¿qué podemos darle?'' respondió Saúl.
``Todo el pan de nuestras bolsas se ha acabado. No tenemos nada que
llevarle al hombre de Dios. ¿Qué tenemos con nosotros?''

\bibleverse{8} ``Mira, tengo un cuarto de siclo de plata conmigo. Se lo
daré al hombre de Dios para que nos indique el camino que debemos
tomar'', le dijo el criado a Saúl. \bibleverse{9} (Antiguamente, en
Israel, alguien que iba a consultar a Dios decía: ``Ven, vamos a ver al
vidente'', porque a los profetas se les solía llamar videntes).

\hypertarget{la-cuxe1lida-bienvenida-de-sauxfal-y-el-trato-honorable-de-parte-de-samuel}{%
\subsection{La cálida bienvenida de Saúl y el trato honorable de parte
de
Samuel}\label{la-cuxe1lida-bienvenida-de-sauxfal-y-el-trato-honorable-de-parte-de-samuel}}

\bibleverse{10} ``Me parece bien'', le dijo Saúl a su criado. ``Vamos
entonces''. Y se fueron al pueblo donde estaba el hombre de Dios.
\bibleverse{11} Mientras subían la colina hacia el pueblo, se
encontraron con unas jóvenes que salían a sacar agua y les preguntaron:
``¿Está el vidente aquí?''

\bibleverse{12} Ellas les respondieron: ``Está más adelante. Pero
tendrán que apresurarse. Hoy ha venido a la ciudad porque el pueblo está
celebrando un sacrificio en el lugar de adoración. \bibleverse{13}
Cuando entren a la ciudad podrán encontrarlo antes de que suba a comer
en lugar de adoración. El pueblo no comerá antes de que él haya llegado,
porque él tiene que bendecir el sacrificio. Después comerán los que han
sido invitados. Si se van ahora, lo alcanzarán''.

\bibleverse{14} Así que siguieron su camino hasta la ciudad. Cuando
llegaron allí estaba Samuel yendo en dirección contraria. Se encontraron
con él cuando subía al lugar de adoración.

\bibleverse{15} El día anterior a la llegada de Saúl, el Señor le había
dicho a Samuel: \bibleverse{16} ``Mañana a esta hora te voy a enviar un
hombre de la tierra de Benjamín. Nómbralo como gobernante de mi pueblo
Israel, y él los rescatará de los filisteos. He visto lo que le pasa a
mi pueblo y he escuchado su ruego de ayuda''.

\bibleverse{17} Cuando Samuel vio a Saúl, el Señor le dijo: ``Este es el
hombre del que te hablé. Es el que va a gobernar a mi pueblo''.

\bibleverse{18} Saúl se acercó a Samuel en la puerta y le preguntó:
``¿Podrías decirme dónde está la casa del vidente?''

\bibleverse{19} ``Yo soy el vidente'', le dijo Samuel a Saúl. ``Sube
delante de mí y comeremos juntos. Luego, por la mañana, responderé a
todas tus preguntas y te enviaré por el camino. \bibleverse{20} En
cuanto a los burros que perdiste hace tres días, no te preocupes por
ellos porque los han encontrado. Pero ahora, la esperanza de todo Israel
descansa en ti y en tu linaje''

\bibleverse{21} ``¡Pero yo soy de la tribu de Benjamín, la más pequeña
de Israel, y mi familia es la menos importante de todas las familias de
la tribu de Benjamín!'' respondió Saúl. ``¿Por qué me dices esto?''
\footnote{\textbf{9:21} 1Sam 15,17}

\bibleverse{22} Entonces Samuel llevó a Saúl y a su criado al salón, y
los sentó a la cabeza de las treinta personas que habían sido invitadas.
\bibleverse{23} Entonces Samuel le dijo al cocinero: ``Trae el trozo de
carne especial que te di y que te dije que reservaras''. \bibleverse{24}
Así que el cocinero tomó el muslo superior\footnote{\textbf{9:24} A Saúl
  se le dio la carne que sólo debían comer los sacerdotes. Ver Levítico
  10:14-15.} de la carne y lo puso delante de Saúl. Entonces Samuel le
dijo: ``Mira, esto es lo que estaba reservado para ti. Cómelo, pues
estaba apartado para ti, para este momento en particular, desde que
dije: `He invitado al pueblo'\,''. Así que Saúl comió con Samuel aquel
día.

\bibleverse{25} Cuando descendieron del lugar de adoración en lo alto a
la ciudad, Samuel habló con Saúl en el techo de su casa.\footnote{\textbf{9:25}
  A falta de otras habitaciones, la azotea de la casa se utilizaba como
  alojamiento temporal.}

\hypertarget{sauxfal-ungido-rey-por-samuel-su-regreso-a-guibeuxe1}{%
\subsection{Saúl ungido rey por Samuel; su regreso a
Guibeá}\label{sauxfal-ungido-rey-por-samuel-su-regreso-a-guibeuxe1}}

\bibleverse{26} Al amanecer del día siguiente, Samuel llamó a Saúl desde
el tejado: ``¡Levántate! Tengo que enviarte de regreso''. Así que Saúl
se levantó y salió con Samuel. \bibleverse{27} Cuando se acercaban a las
afueras de la ciudad, Samuel le dijo a Saúl: ``Dile a tu siervo que se
vaya adelante, antes que nosotros. Cuando se haya ido, quédate aquí un
rato, porque tengo un mensaje de Dios para ti''. Así que el criado se
adelantó.

\hypertarget{section-9}{%
\section{10}\label{section-9}}

\bibleverse{1} Entonces Samuel tomó un frasco de aceite de oliva y lo
derramó sobre la cabeza de Saúl, y lo besó diciendo: ``El Señor te ha
ungido como gobernante de su pueblo elegido.\footnote{\textbf{10:1} Esta
  línea se da en forma de pregunta, pero es mejor traducirla como una
  declaración, ya que una pregunta puede implicar incertidumbre.}

\hypertarget{samuel-profetiza-tres-seuxf1ales-que-sauxfal-recibiruxe1-de-camino-a-casa-y-lo-envuxeda-a-gilgal}{%
\subsection{Samuel profetiza tres señales que Saúl recibirá de camino a
casa y lo envía a
Gilgal}\label{samuel-profetiza-tres-seuxf1ales-que-sauxfal-recibiruxe1-de-camino-a-casa-y-lo-envuxeda-a-gilgal}}

\bibleverse{2} Cuando me dejes hoy, te encontrarás con dos hombres cerca
de la tumba de Raquel en Selsa, en la frontera del territorio de
Benjamín. Te dirán que han encontrado los burros que fuiste a buscar.
``Ahora tu padre no está preocupado por ellos, sino por ti, y se
pregunta: `¿Qué pasará con mi hijo?' .

\bibleverse{3} ``Saldrás de allí y seguirás hasta la encina de Tabor,
donde te encontrarás con tres hombres que van a adorar a Dios en Betel.
Uno llevará tres cabritos, otro llevará tres panes y otro llevará un
odre de vino. \bibleverse{4} Te saludarán\footnote{\textbf{10:4}
  Literalmente, ``shalom'', el saludo usual de la época.} y te darán dos
panes que deberás tomar.

\bibleverse{5} ``A continuación llegarás a Guibeá de Dios, donde los
filisteos tienen una guarnición. Al entrar en la ciudad, te encontrarás
con una procesión de profetas que desciende del lugar alto, tocando
arpas, panderetas, flautas y liras, y estarán profetizando.
\bibleverse{6} Entonces el Espíritu del Señor vendrá sobre ti con poder.
Profetizarás con ellos, y te convertirás en un hombre diferente.
\footnote{\textbf{10:6} 1Sam 10,10} \bibleverse{7} Después de que hayan
ocurrido estas señales, haz lo que tengas que hacer, porque Dios está
contigo.

\bibleverse{8} Luego ve delante de mí a Gilgal. Te aseguro que iré y me
reuniré contigo para presentar holocaustos y ofrendas de paz. Espera
allí siete días hasta que yo vaya a verte y te haga saber lo que debes
hacer''.

\hypertarget{la-llegada-de-los-carteles-anunciados-saulo-entre-los-profetas}{%
\subsection{La llegada de los carteles anunciados; Saulo entre los
profetas}\label{la-llegada-de-los-carteles-anunciados-saulo-entre-los-profetas}}

\bibleverse{9} En el momento mismo que Saúl se volvió y dejó a Samuel,
Dios le dio a Saúl una forma de pensar diferente,\footnote{\textbf{10:9}
  ``Una manera de pensardiferente'': literalmente ``hizo que su corazón
  fuera otro''. Dado que en hebreo el corazón era donde se pensaba, esto
  se relaciona con la mente. En muchos sentidos, esto se corresponde con
  el concepto griego de un ``cambio de mente'', que es el verdadero
  significado de la conversión. Así que en cierto sentido se podría
  decir que Saúl se ``convirtió'' en ese momento.} y todas las señales
se cumplieron aquel día. \bibleverse{10} Cuando Saúl y su criado
llegaron a Guibeá, había una procesión de profetas que salía a su
encuentro. Y el Espíritu de Dios vino sobre Saúl con poder, y él también
comenzó a profetizar con ellos. \footnote{\textbf{10:10} 1Sam 19,20-24}
\bibleverse{11} Todos los que conocían a Saúl y lo veían profetizar con
los profetas se decían: ``¿Qué pasa con el hijo de Cis? ¿Acaso Saúl es
también uno de los profetas?''

\bibleverse{12} Un hombre que vivía allí respondió: ``¿Pero quién es su
padre?''\footnote{\textbf{10:12} En otras palabras, el don profético no
  depende de la genealogía.} Así que se convirtió en un dicho: ``¿Es
Saúl también uno de los profetas?''

\hypertarget{saul-de-regreso-a-casa-su-conversaciuxf3n-reservada-con-su-prima}{%
\subsection{Saul de regreso a casa; su conversación reservada con su
prima}\label{saul-de-regreso-a-casa-su-conversaciuxf3n-reservada-con-su-prima}}

\bibleverse{13} Cuando Saúl terminó de profetizar, fue al lugar alto de
adoración. \bibleverse{14} El tío de Saúl le preguntó a éste y a su
criado: ``¿Dónde estaban?'' ``Estábamos buscando los burros'', respondió
Saúl. ``Como no los encontramos, fuimos a ver a Samuel''.

\bibleverse{15} ``Por favor, díganme qué les dijo'', preguntó el tío de
Saúl.

\bibleverse{16} ``Nos aseguró que los burros habían sido encontrados'',
respondió Saúl. Pero Saúl no le dijo a su tío lo que Samuel le había
dicho que sería rey.

\hypertarget{sauxfal-estuxe1-decidido-a-ser-rey-en-mizpa-por-la-santa-suerte}{%
\subsection{Saúl está decidido a ser rey en Mizpa por la santa
suerte}\label{sauxfal-estuxe1-decidido-a-ser-rey-en-mizpa-por-la-santa-suerte}}

\bibleverse{17} Entonces Samuel convocó al pueblo de Israel a
presentarse ante el Señor en Mizpa. \bibleverse{18} Y les dijo a los
israelitas: ``Esto es lo que dice el Señor, el Dios de Israel: Yo saqué
a Israel de Egipto y los salvé de los egipcios y de todos los reinos que
los oprimían. \bibleverse{19} Pero ahora ustedes han rechazado a su
Dios, el que los salva de todos sus problemas y aflicciones. Y le han
dicho: `Tienes que nombrar un rey que nos gobierne'. Así que ahora
preséntense ante el Señor por tribus y grupos familiares''.

\bibleverse{20} Entonces Samuel hizo que todo Israel se presentara por
tribus, y la tribu de Benjamín fue elegida por sorteo. \footnote{\textbf{10:20}
  1Sam 14,41-42; Jos 7,16} \bibleverse{21} Luego hizo que la tribu de
Benjamín se presentara por sus grupos familiares, y fue elegido el grupo
familiar de Matri. Por último, se eligió a Saúl, hijo de Cis. Pero
cuando lo buscaron, no lo encontraron. \bibleverse{22} Y le preguntaron
al Señor: ``¿Ya está aquí?'' . Y el Señor respondió: ``Vayan a buscarlo;
está escondido entre el equipaje''.

\bibleverse{23} Así que corrieron y trajeron a Saúl. Cuando se puso de
pie entre la gente, era más alto que los demás. \bibleverse{24} Samuel
les dijo a todos: ``¿Ven al que el Señor ha elegido? No hay nadie como
él en ninguna parte''. Y todo el pueblo gritó: ``¡Viva el rey!''.

\bibleverse{25} Entonces Samuel le explicó al pueblo todo lo que haría
un rey. Lo escribió en un pergamino y lo puso ante el Señor. Luego
Samuel los envió a todos a casa. \footnote{\textbf{10:25} 1Sam 8,11;
  Deut 17,14-20}

\bibleverse{26} Saúl también regresó a su casa en Guibeá, acompañado de
los guerreros a quienes Dios había convencido para que lo ayudaran.
\bibleverse{27} Pero algunos hombres odiosos preguntaron: ``¿Cómo podría
salvarnos este hombre?'' . Lo odiaron y no le trajeron ningún regalo;
pero Saúl no tomó represalias.\footnote{\textbf{10:27} En el texto
  hebreo tradicional el capítulo termina aquí. Sin embargo, en un rollo
  encontrado en Qumrán hay la siguiente información adicional que se
  relaciona con el siguiente capítulo y se incluye aquí por su interés.
  ``Nahas, rey de los amonitas, había estado oprimiendo severamente al
  pueblo de Gad y Rubén. Les sacaba el ojo derecho y no dejaba que nadie
  los ayudara. No quedó nadie de los israelitas al otro lado del Jordán
  a quien Nahas, rey de los amonitas, no le hubiera sacado el ojo
  derecho. Sin embargo, había siete mil hombres que habían escapado de
  los amonitas y se habían ido a vivir a Jabes de Galaad''.}

\hypertarget{la-ciudad-de-jabuxe9s-que-estuxe1-en-apuros-por-los-amonitas-nahas-pide-la-ayuda-de-los-israelitas.}{%
\subsection{La ciudad de Jabés, que está en apuros por los amonitas
Nahas, pide la ayuda de los
israelitas.}\label{la-ciudad-de-jabuxe9s-que-estuxe1-en-apuros-por-los-amonitas-nahas-pide-la-ayuda-de-los-israelitas.}}

\hypertarget{section-10}{%
\section{11}\label{section-10}}

\bibleverse{1} Nahas el amonita llegó con su ejército\footnote{\textbf{11:1}
  ``Con su ejército'': añadido para mayor claridad.} y sitió Jabes de
Galaad. Todo el pueblo de Jabes le dijo: ``Haz un acuerdo de paz con
nosotros, y seremos tus súbditos''. \footnote{\textbf{11:1} 1Sam 31,11}

\bibleverse{2} Pero Nahas el amonita respondió: ``Haré un tratado de paz
con ustedes con una condición: que les saque a todos el ojo derecho para
avergonzar a todos los israelitas''. \footnote{\textbf{11:2} Jer 39,7}

\bibleverse{3} ``Déjanos siete días para que podamos enviar mensajeros
por todo Israel'', respondieron los ancianos del pueblo de Jabes. ``Si
nadie viene a ayudarnos, nos rendiremos ante ustedes''.

\hypertarget{la-conducta-decidida-de-sauxfal-y-su-espluxe9ndida-victoria}{%
\subsection{La conducta decidida de Saúl y su espléndida
victoria}\label{la-conducta-decidida-de-sauxfal-y-su-espluxe9ndida-victoria}}

\bibleverse{4} Cuando los mensajeros llegaron a Guibeá de Saúl y dieron
el mensaje mientras el pueblo escuchaba, todos lloraron a gritos.

\bibleverse{5} Justo en ese momento Saúl volvía de arar un campo con sus
bueyes. ``¿Por qué están todos tan alterados?'' , preguntó. Entonces le
contaron lo que habían dicho los hombres de Jabes. \bibleverse{6} Cuando
se enteró de esto, el Espíritu de Dios se apoderó de Saúl, y se enojó
mucho. \footnote{\textbf{11:6} Jue 14,6} \bibleverse{7} Entonces tomó un
par de bueyes y los cortó en pedazos. Luego los envió con los mensajeros
por todo Israel con el mensaje: ``Esto es lo que pasará con los bueyes
de cualquiera que no siga a Saúl y a Samuel''. Y el Señor hizo que el
pueblo se pusiera ansioso\footnote{\textbf{11:7} ``El Señor hizo que el
  pueblo se pudiera ansioso'': literalmente ``El temor del Señor cayó
  sobre el pueblo''. Esto podría interpretarse como que el Señor es la
  fuente del temor, o el objeto de temor. En cualquier caso, el
  resultado es que el pueblo apoya a Saúl.} por hacerlo, y el pueblo
salió como si fueran uno solo. \footnote{\textbf{11:7} Jue 19,29}
\bibleverse{8} Cuando Saúl los contó en Bezek, había 300. 000 hombres de
Israel y 30. 000 de Judá. \bibleverse{9} A los mensajeros que llegaron
les dijeron: ``Diganles a los hombres de Jabes de Galaad: `Mañana serán
rescatados, para cuando el sol está caliente'.''. El pueblo de Jabes se
puso muy contento cuando los mensajeros les dieron este mensaje.
\bibleverse{10} Entonces les dijeron a los amonitas: ``Nos rendiremos a
ustedes mañana, y entonces podrán hacer con nosotros lo que quieran''.
\bibleverse{11} Al día siguiente, Saúl organizó al ejército en tres
divisiones. Atacaron el campamento amonita antes del amanecer y
siguieron matándolos hasta que llegó el medio día. Los sobrevivientes
estaban tan dispersos que ni siquiera quedaban dos de ellos juntos.

\hypertarget{la-generosidad-de-sauxfal-hacia-sus-despreciadores-celebraciuxf3n-de-la-alegruxeda-en-gilgal}{%
\subsection{La generosidad de Saúl hacia sus despreciadores; Celebración
de la alegría en
Gilgal}\label{la-generosidad-de-sauxfal-hacia-sus-despreciadores-celebraciuxf3n-de-la-alegruxeda-en-gilgal}}

\bibleverse{12} Entonces el pueblo le preguntó a Samuel: ``¿Dónde están
los que dijeron `¿Por qué debemos tener a Saúl como rey?' Entreguen a
estos hombres para ejecutarlos''. \footnote{\textbf{11:12} 1Sam 10,27}

\bibleverse{13} Pero Saúl respondió: ``Nadie va a ser ejecutado hoy,
porque éste es el día en que el Señor ha salvado a Israel''. \footnote{\textbf{11:13}
  1Sam 14,45}

\bibleverse{14} Entonces Samuel le dijo al pueblo: ``Vengan conmigo,
vayamos a Gilgal y renovemos el reino''. \footnote{\textbf{11:14} 1Sam
  10,8}

\bibleverse{15} Y todos fueron a Gilgal, y confirmaron a Saúl como rey
ante el Señor. Sacaron ofrendas de paz para el Señor, y Saúl, junto con
todos los israelitas, hizo una gran celebración.

\hypertarget{la-renuncia-voluntaria-de-samuel-y-la-solemne-despedida-del-pueblo}{%
\subsection{La renuncia voluntaria de Samuel y la solemne despedida del
pueblo}\label{la-renuncia-voluntaria-de-samuel-y-la-solemne-despedida-del-pueblo}}

\hypertarget{section-11}{%
\section{12}\label{section-11}}

\bibleverse{1} Entonces Samuel le dijo a todo Israel: ``He prestado
atención a todo lo que me han pedido, y les he dado un rey para que los
gobierne. \bibleverse{2} Ahora su rey es su líder. Yo soy viejo y
canoso, y mis hijos están aquí con ustedes. Los he guiado desde que era
un niño hasta hoy. \bibleverse{3} Aquí estoy ante ustedes. Traigan
cualquier acusación que tengan contra mí en presencia del Señor y de su
ungido.\footnote{\textbf{12:3} ``Su ungido'': Refiriéndose al rey.} ¿Me
he apropiadodel buey o del burro de alguien? ¿He perjudicado a alguien?
¿He oprimido a alguien? ¿He aceptado un soborno de alguien para hacerme
el de la vista gorda? Díganmelo y les pagaré por ello''. \footnote{\textbf{12:3}
  Núm 16,15}

\bibleverse{4} ``No, nunca nos has engañado ni nos has oprimido'',
respondieron, ``y nunca has tomado nada de nadie''.

\bibleverse{5} Samuel les dijo: ``El Señor es testigo, y su ungido es
testigo hoy, en este caso que les concierne, de que no soy culpable de
nada''.\footnote{\textbf{12:5} ``No soy culpable de nada'':
  literalmente, ``no han encontrado nada en mi mano''.} ``El Señor es
testigo'', respondieron.

\hypertarget{samuel-le-recuerda-al-pueblo-los-muchos-beneficios-de-dios}{%
\subsection{Samuel le recuerda al pueblo los muchos beneficios de
Dios}\label{samuel-le-recuerda-al-pueblo-los-muchos-beneficios-de-dios}}

\bibleverse{6} ``El Señor es testigo,\footnote{\textbf{12:6} Tomado de
  la Septuaginta.} el que designó a Moisés y a Aarón'', continuó Samuel.
``Él sacó a sus antepasados de la tierra de Egipto. \bibleverse{7} Así
pues, permanezcan aquí mientras les presento, en presencia del Señor, la
prueba de todas las cosas buenas que el Señor ha hecho por ustedes y por
sus antepasados.

\bibleverse{8} ``Después de que Jacob fue a Egipto, sus padres clamaron
al Señor por ayuda, y él envió a Moisés y a Aarón para que ayudaran a
sus antepasados a salir de Egipto y para establecerse aquí.
\bibleverse{9} Pero se olvidaron del Señor, su Dios, y éste los abandonó
en manos de Sísara, comandante del ejército de Hazor, de los filisteos y
del rey de Moab, que los atacó. \footnote{\textbf{12:9} Jue 4,2; Jue
  10,7; Jue 13,1; Jue 3,12} \bibleverse{10} ``Ellos clamaron al Señor
por ayuda y dijeron: `Hemos pecado, pues hemos rechazado al Señor y
hemos adorado a los baales y a Astoret. Por favor, sálvanos de las manos
de nuestros enemigos, y te adoraremos'. \bibleverse{11} Entonces el
Señor envió a Gedeón,\footnote{\textbf{12:11} ``Gedeón'': Llamado aquí
  ``Jerub-Baal''.} Barak,\footnote{\textbf{12:11} Tomado de la
  Septuaginta y version siríaca. El hebreo era ``Bedán''.} Jefté y
Samuel, y los salvó de los enemigos que los rodeaban para que pudieran
vivir con seguridad.

\hypertarget{samuel-demuestra-al-pueblo-a-travuxe9s-de-una-maravillosa-seuxf1al-divina-que-han-pecado-al-elegir-un-rey}{%
\subsection{Samuel demuestra al pueblo a través de una maravillosa señal
divina que han pecado al elegir un
rey}\label{samuel-demuestra-al-pueblo-a-travuxe9s-de-una-maravillosa-seuxf1al-divina-que-han-pecado-al-elegir-un-rey}}

\bibleverse{12} ``Pero cuando vieron que Nahas, rey de los amonitas,
venía a atacarlos, me dijeron: `No, queremos nuestro propio rey', aunque
el Señor, su Dios, era su rey. \footnote{\textbf{12:12} 1Sam 8,19}
\bibleverse{13} Así que aquí está el rey que ustedes han elegido, el que
pidieron. Miren: ¡el Señor se los entrega ahora como su rey!
\bibleverse{14} ``Si honran al Señor, lo adoran, hacen lo que les dice y
no se rebelan contra las instrucciones del Señor, y si tanto ustedes
como su rey siguen al Señor su Dios, ¡entonces todo estará bien!
\bibleverse{15} Sin embargo, si se niegan a hacer su voluntad y se
rebelan contra las instrucciones del Señor, entonces el Señor estará
contra ustedes como lo estuvo contra sus antepasados.

\bibleverse{16} ``Ahora quédense quietos y observen lo que el Señor va a
hacer, ante sus propios ojos. \bibleverse{17} ¿No es el tiempo de la
cosecha de trigo?\footnote{\textbf{12:17} En esta época no solía
  producirse la lluvia.} Pues bien, le pediré al Señor que envíe truenos
y lluvia. Entonces se darán cuenta del mal que han hecho ante los ojos
del Señor cuando exigieron su propio rey''.

\bibleverse{18} Entonces Samuel oró al Señor, y ese mismo día el Señor
envió truenos y lluvia. Todos estaban totalmente asombrados del Señor y
de Samuel.

\hypertarget{samuel-anima-al-pueblo-les-exhorta-a-temer-a-dios-y-les-manda-recibir-bendiciones-divinas}{%
\subsection{Samuel anima al pueblo, les exhorta a temer a Dios y les
manda recibir bendiciones
divinas}\label{samuel-anima-al-pueblo-les-exhorta-a-temer-a-dios-y-les-manda-recibir-bendiciones-divinas}}

\bibleverse{19} ``¡Por favor, ruega al Señor tu Dios por nosotros, tus
siervos, para que no muramos!'', le rogaron a Samuel. ``Porque hemos
añadido a todos nuestros pecados la maldad de pedir nuestro propio
rey''.

\bibleverse{20} ``No tengan miedo'', respondió Samuel. ``Aunque en
verdad hayan hecho todas estas maldades, no dejen de seguir al Señor,
sino dedíquense por completo a adorarlo. \bibleverse{21} No adoren a
ídolos sin valor que los tales no pueden ayudarlos ni salvarlos, porque
no son nada. \bibleverse{22} Lo cierto es que, gracias a la clase de
persona que es el Señor, no abandonará a su pueblo, porque se alegró de
reclamarlos a ustedes como suyos. \footnote{\textbf{12:22} Éxod 19,6}
\bibleverse{23} ``En cuanto a mí, ¿cómo podría pecar contra el Señor
dejando de orar por ustedes? También seguiré enseñándoles el camino del
bien y la rectitud. \footnote{\textbf{12:23} 1Sam 7,8} \bibleverse{24}
Asegúrense de honrar a Dios y adorarlo fielmente, con total dedicación.
Piensen en las maravillas que ha hecho por ustedes. \footnote{\textbf{12:24}
  2Re 17,39}

\bibleverse{25} Pero si siguen haciendo lo malo, ustedes y su rey serán
eliminados''.

\hypertarget{estallido-de-la-guerra-filistea-la-primera-desobediencia-de-sauxfal-mediante-un-sacrificio-apresurado}{%
\subsection{Estallido de la guerra filistea; La primera desobediencia de
Saúl mediante un sacrificio
apresurado}\label{estallido-de-la-guerra-filistea-la-primera-desobediencia-de-sauxfal-mediante-un-sacrificio-apresurado}}

\hypertarget{section-12}{%
\section{13}\label{section-12}}

\bibleverse{1} Saúl tenía treinta años cuando llegó a ser rey, y reinó
sobre Israel durante cuarenta y dos años.

\bibleverse{2} Saúl había elegido a tres mil hombres de Israel. Dos mil
de ellos estaban con Saúl en Micmas y en la región montañosa de Betel, y
otros mil estaban con Jonatán en Guibeá de Benjamín. Y envió al resto
del ejército a casa. \bibleverse{3} Tiempo después, Jonatán atacó la
guarnición de los filisteos en Geba. Los filisteos no tardaron en
enterarse, así que Saúl hizo sonar la trompeta de llamada a las armas
por todo el país, diciendo: ``Hebreos,\footnote{\textbf{13:3}
  ``Hebreos'': el término es el nombre dado por otros a los israelitas,
  y así utilizado aquí recuerda a los israelitas que son dominados por
  otras naciones. Algunos han sugerido incluso que el término se
  utilizaba para los israelitas que eran esclavos de los extranjeros.}
presten atención!'' \bibleverse{4} Entonces todo Israel escuchó la
noticia: ``¡Saúl ha atacado la guarnición filistea, y ahora los
filisteos odian a Israel!'' Así que todo el ejército fue convocado para
unirse a Saúl en Gilgal. \bibleverse{5} Los filisteos se reunieron para
pelear contra Israel. Tenían tres mil\footnote{\textbf{13:5} El texto
  hebreo dice ``30. 000'', lo que parece excesivo. La versión luciana de
  la Septuaginta y la versión siríaca dicen 3. 000.} carros, seis mil
jinetes y soldados tan numerosos como la arena en la orilla del mar.
Avanzaron y acamparon en Micmas, al este de Bet-aven. \bibleverse{6}
Cuando los hombres israelitas se dieron cuenta de la difícil situación
en la que se encontraban y de que el ejército estaba recibiendo una
paliza, se escondieron en cuevas, agujeros, rocas, pozos y cisternas.
\bibleverse{7} Algunos de los hebreos incluso cruzaron el Jordán hacia
el territorio de Gad y Galaad, pero Saúl se quedó en Gilgal, y todos los
hombres que estaban con él temblaban de miedo.

\hypertarget{el-sacrificio-apresurado-y-arbitrario-de-sauxfal-en-gilgal-rompe-entre-samuel-y-el-rey-el-rechazo-de-sauxfal}{%
\subsection{El sacrificio apresurado y arbitrario de Saúl en Gilgal;
Rompe entre Samuel y el rey; El rechazo de
Saúl}\label{el-sacrificio-apresurado-y-arbitrario-de-sauxfal-en-gilgal-rompe-entre-samuel-y-el-rey-el-rechazo-de-sauxfal}}

\bibleverse{8} Saúl esperó allí siete días el tiempo que Samuel había
dicho, pero Samuel no llegó a Gilgal, y el ejército comenzó a
abandonarlo. \footnote{\textbf{13:8} 1Sam 10,8} \bibleverse{9} Entonces
Saúl ordenó: ``Tráiganme el holocausto y las ofrendas de paz'', y
presentó el holocausto.

\bibleverse{10} Justo cuando terminó de presentar el holocausto, vio
llegar a Samuel. Saúl fue a recibirlo y a saludarlo. \bibleverse{11}
``¿Qué has hecho?'' le preguntó Samuel. Saúl respondió: ``Bueno, vi que
mis hombres me abandonaban, y que tú no habías llegado cuando dijiste
que lo harías, y que los filisteos se estaban reuniendo en Micmas para
atacar.

\bibleverse{12} Así que pensé: `Los filisteos están a punto de atacarme
en Gilgal, y no he pedido la ayuda del Señor'. Así que sentí que debía
presentar yo mismo el holocausto''.

\bibleverse{13} ``Has sido muy estúpido'', le dijo Samuel. ``No has
cumplido los mandatos del Señor, tu Dios. Si lo hubieras hecho, el Señor
habría asegurado tu reino sobre Israel para siempre. \bibleverse{14}
Pero ahora tu reino no durará. El Señor ha encontrado para sí un hombre
que piensa como él, y lo ha elegido para que sea el gobernante de su
pueblo, porque tú no has cumplido los mandatos del Señor''.

\hypertarget{el-pequeuxf1o-ejuxe9rcito-de-sauxfal-el-pillaje-de-los-filisteos-indefensiuxf3n-de-los-israelitas}{%
\subsection{El pequeño ejército de Saúl; el pillaje de los filisteos;
Indefensión de los
israelitas}\label{el-pequeuxf1o-ejuxe9rcito-de-sauxfal-el-pillaje-de-los-filisteos-indefensiuxf3n-de-los-israelitas}}

\bibleverse{15} Entonces Samuel se fue de Gilgal. El resto de los
soldados siguió a Saúl para reunirse con el ejército, yendo de Gilgal a
Geba, en Benjamín.\footnote{\textbf{13:15} En el texto hebreo falta una
  parte de este versículo, probablemente debido a un error de los
  copistas. Aquí se sigue la Septuaginta.} Saúl contó el número de
soldados que estaban con él y eran unos seiscientos. \bibleverse{16}
Saúl, su hijo Jonatán y los soldados que estaban con ellos se alojaban
en Geba de Benjamín, mientras los filisteos estaban acampados en Micmas.
\bibleverse{17} Tres grupos de asaltantes salieron del campamento
filisteo para ir a atacar. Un grupo se dirigió hacia Ofra en la tierra
de Shual, \bibleverse{18} otro hacia Bet-horón, y otro hacia la frontera
que da al Valle de Seboim por el desierto. \bibleverse{19} En esos días
no había un herrero en ninguna parte de Israel. Los filisteos lo
impedían para que los hebreos no hicieran espadas y lanzas.
\bibleverse{20} Todos los israelitas tenían que acudir a los filisteos
para afilar sus rejas de hierro, picos, hachas y hoces. \bibleverse{21}
La tarifa era de dos tercios de siclo\footnote{\textbf{13:21} ``Dos
  tercios de siclo'': literalmente ``un pim''.} por rejas de arado y
picos, y un tercio de siclo para afilar las hachas y las picas de
ganado. \bibleverse{22} Así que cuando llegó el día de la batalla
ninguno de los soldados que acompañaban a Saúl y a Jonatán tenía espadas
ni lanzas; sólo Saúl y su hijo Jonatán tenían esas armas.

\bibleverse{23} Una guarnición filistea había tomado el control del paso
de Micmas.\footnote{\textbf{13:23} Este versículo es mejor tomarlo como
  parte del siguiente capítulo.}

\hypertarget{el-herouxedsmo-de-jonathan-la-victoria-de-sauxfal-sobre-los-filisteos}{%
\subsection{El heroísmo de Jonathan; La victoria de Saúl sobre los
filisteos}\label{el-herouxedsmo-de-jonathan-la-victoria-de-sauxfal-sobre-los-filisteos}}

\hypertarget{section-13}{%
\section{14}\label{section-13}}

\bibleverse{1} Un día Jonatán, hijo de Saúl, le dijo al joven escudero:
``Vamos, crucemos a la guarnición filistea del otro lado''. Pero no le
hizo saber a su padre acerca de sus planes. \bibleverse{2} Saúl se
encontraba cerca de Guibeá, bajo un granado\footnote{\textbf{14:2} ``Un
  granado'' {[}árbol{]}: o ``la roca de Rimón''.} en Migrón. Tenía unos
seiscientos hombres con él, \bibleverse{3} incluyendo a Ahija, que
llevaba un efod.\footnote{\textbf{14:3} ``Efod'': Un accesorio
  sacerdotal.} Era hijo del hermano de Icabod, Ahitob, hijo de Finees,
hijo de Elí, sacerdote del Señor en Silo. Nadie se dio cuenta de que
Jonatán se había ido. \footnote{\textbf{14:3} 1Sam 4,19; 1Sam 4,21}

\bibleverse{4} A ambos lados del paso que Jonatán planeaba cruzar para
llegar a la guarnición filistea se erigían dos acantilados, uno llamado
Boses y el otro Sene. \bibleverse{5} El acantilado del norte estaba en
el lado de Michmash, el del sur en el lado de Geba. \bibleverse{6}
Jonatán le dijo al joven que llevaba la armadura: ``Vamos, crucemos a la
guarnición de estos hombres paganos.\footnote{\textbf{14:6} ``Paganos'':
  literalmente, ``incircuncisos''.} Tal vez el Señor nos ayude. Al Señor
no le cuesta ganar, sea por muchos o por pocos''.

\bibleverse{7} ``Tú decides qué hacer'', respondió el escudero. ``¡Estoy
contigo sin importar lo que decidas!''

\bibleverse{8} ``¡Vamos entonces!'' dijo Jonathan. ``Cruzaremos en su
dirección para que nos vean. \bibleverse{9} Si nos dicen: `Esperen allí
hasta que bajemos a ustedes', esperaremos donde estamos y no subiremos a
ellos. \bibleverse{10} Pero si nos dicen: `Suban hacia nosotros',
subiremos, porque eso será la señal de que el Señor nos los ha
entregado''.

\bibleverse{11} Así que ambos se dejaron ver por la guarnición filistea.
``¡Mira!'', gritaron los filisteos. ``Los hebreos están saliendo de los
huecos\footnote{\textbf{14:11} ``Huecos'': la palabra se utiliza a
  menudo para describir las madrigueras donde viven los animales.} donde
se escondían''. \bibleverse{12} Los hombres de la guarnición llamaron a
Jonatán y a su escudero: ``¡Suban aquí y les mostraremos un par de
cosas!''. ``Sígueme arriba'', dijo Jonatán a su escudero, ``porque el
Señor los ha entregado a Israel''.

\bibleverse{13} Así que Jonatán subió de manos y pies, con su escudero
que iba justo detrás de él. Jonatán los atacó y los mató,\footnote{\textbf{14:13}
  ``Jonatán los atacó y los mató'': literalmente, ``cayeron ante
  Jonatán''.} y su escudero le siguió haciendo lo mismo. \footnote{\textbf{14:13}
  Lev 26,7-8} \bibleverse{14} En este primer ataque, Jonatán y su
escudero mataron a unos veinte hombres en un área de media hectárea.

\bibleverse{15} Entonces los filisteos entraron en pánico, en el
campamento, en el campo y en todo su ejército. Incluso los que estaban
en los puestos de avanzada y los grupos de asaltantes se aterrorizaron.
La tierra se estremeció. Era terror proveniente de Dios.

\hypertarget{sauxfal-interviene-y-obtiene-una-brillante-victoria}{%
\subsection{Saúl interviene y obtiene una brillante
victoria}\label{sauxfal-interviene-y-obtiene-una-brillante-victoria}}

\bibleverse{16} Los vigías de Saúl en Guibeá, en Benjamín, vieron cómo
el ejército filisteo se desvanecía y se dispersaba en todas direcciones.
\bibleverse{17} Saúl les dijo a los soldados que estaban con él: ``Pasen
lista y averigüen quiénes no están con nosotros''. Cuando pasaron lista,
descubrieron que Jonatán y su escudero no estaban allí.

\bibleverse{18} Saúl le dijo a Ajías: ``Trae el Arca de Dios aquí''. (En
esa época el Arca de Dios viajaba con los israelitas). \bibleverse{19}
Pero mientras Saúl hablaba con el sacerdote, el alboroto que venía del
campamento filisteo era cada vez más fuerte. Así que Saúl le dijo al
sacerdote: ``¡Olvídalo!''\footnote{\textbf{14:19} ``¡Olvídalo!'':
  literalmente, ``Quita tu mano''. El sacerdote estaba a punto de
  intentar determinar la voluntad del Señor con respecto a un ataque
  contra los filisteos, tal vez consultando el Urim y el Tumin en el
  efod o mediante el uso del Arca de Dios de alguna manera. Cualquiera
  que sea el caso, Saúl revocó su orden anterior de guía divina
  diciéndole al sacerdote que detuviera lo que estaba a punto de hacer.}

\bibleverse{20} Entonces Saúl y todo su ejército se reunieron y entraron
en batalla. Descubrieron que los filisteos estaban en total desorden,
atacándose unos a otros con las espadas. \footnote{\textbf{14:20} Jue
  7,22; 2Cró 20,23} \bibleverse{21} Los hebreos que antes se habían
puesto del lado de los filisteos, y que estaban con ellos en su
campamento, cambiaron de bando y se unieron a los israelitas que estaban
con Saúl y Jonatán. \bibleverse{22} Cuando todos los israelitas que se
habían escondido en la región montañosa de Efraín se enteraron de que
los filisteos estaban huyendo, también se unieron para perseguir a los
filisteos y atacarlos. \bibleverse{23} Ese día el Señor salvó a Israel,
y la batalla se extendió más allá de Bet-aven.\footnote{\textbf{14:23}
  La Septuaginta añade lo siguiente en este punto: ``y el ejército que
  acompañaba a Saúl contaba con unos diez mil hombres. La batalla se
  extendió por la región montañosa de Efraín''.}

\hypertarget{el-celo-intempestivo-de-sauxfal-jonathan-estuxe1-amenazado-de-muerte-las-guerras-de-sauxfal-y-su-familia}{%
\subsection{El celo intempestivo de Saúl; Jonathan está amenazado de
muerte; Las guerras de Saúl y su
familia}\label{el-celo-intempestivo-de-sauxfal-jonathan-estuxe1-amenazado-de-muerte-las-guerras-de-sauxfal-y-su-familia}}

\bibleverse{24} Aquel día fue difícil para los hombres de Israel porque
Saúl había ordenado al ejército hacer un juramento, diciendo: ``Maldito
el que coma algo antes de la noche, antes de que me haya vengado de mis
enemigos''. Así que nadie del ejército había comido nada.

\bibleverse{25} Cuando todos entraron en el bosque, encontraron panales
de miel en el suelo. \bibleverse{26} Mientras estaban en el bosque,
vieron que la miel se acababa, pero nadie la recogió para comerla porque
todos tenían miedo del juramento que habían hecho. \bibleverse{27} Pero
Jonatán no se había enterado de que su padre había ordenado al ejército
hacer ese juramento. Así que metió la punta de su bastón en el panal,
cogió un trozo para comer y se sintió mucho mejor.\footnote{\textbf{14:27}
  ``Se sintió mucho mejor'': literalmente, ``sus ojos brillaron''. Igual
  que en el versículo 29.} \bibleverse{28} Pero uno de los soldados le
dijo: ``Tu padre hizo que el ejército hiciera un juramento solemne,
diciendo: `¡Maldito el que coma algo hoy!' Por eso los hombres están
agotados''.

\bibleverse{29} ``Mi padre nos ha causado un montón de problemas a
todos'',\footnote{\textbf{14:29} ``A todos'': literalmente, ``la
  tierra''.} respondió Jonatán. ``Mira qué bien estoy porque he comido
un poco de esta miel. \bibleverse{30} ¡Habría sido mucho mejor si el
ejército hubiera comido hoy en abundancia del botín tomado a sus
enemigos! ¿Cuántos filisteos más habrían matado?'' \bibleverse{31}
Después de derrotar a los filisteos ese día, matándolos desde Micmas
hasta Ajalón, los israelitas estaban totalmente agotados.
\bibleverse{32} Se apoderaron del botín, tomando ovejas, vacas y
terneros, y los sacrificaron allí mismo en el suelo. Pero se los
comieron con la sangre. \bibleverse{33} Entonces le dijeron a Saúl:
``Mira, los hombres están pecando contra el Señor al comer carne con la
sangre''. ``¡Infractores de la ley!'', les dijo Saúl. ``¡Tira una piedra
grande aquí ahora mismo!''

\bibleverse{34} Luego les dijo: ``Recorran todo el lugar donde están los
soldados y díganles, `Cada uno debe traerme su ganado o sus ovejas y
sacrificarlos aquí, y luego comer. No pequen contra el Señor comiendo
carne con sangre'\,''. Cada uno del ejército trajo lo que
tenía\footnote{\textbf{14:34} ``Lo que tenía'': Tomado de la
  Septuaginta.} y lo sacrificó allí aquella noche.

\bibleverse{35} Entonces Saúl construyó un altar al Señor. Este fue el
primer altar que construyó al Señor.

\hypertarget{jonatuxe1n-amenazado-de-muerte-por-el-celo-ciego-de-sauxfal-es-salvado-por-el-ejuxe9rcito}{%
\subsection{Jonatán, amenazado de muerte por el celo ciego de Saúl, es
salvado por el
ejército}\label{jonatuxe1n-amenazado-de-muerte-por-el-celo-ciego-de-sauxfal-es-salvado-por-el-ejuxe9rcito}}

\bibleverse{36} Saúl dijo: ``Vamos a perseguir a los filisteos durante
la noche y a saquearlos hasta el amanecer, sin dejar sobrevivientes''.
``Haz lo que creas conveniente'', respondieron. Pero el sacerdote dijo:
``Preguntémosle primero a Dios''.

\bibleverse{37} Saúl preguntó a Dios: ``¿Debo bajar y perseguir a los
filisteos? ¿Los entregarás a Israel?'' Pero ese día Dios no le
respondió. \footnote{\textbf{14:37} 1Sam 14,18; 1Sam 23,9}
\bibleverse{38} Entonces Saúl dio la orden: ``Todos los comandantes del
ejército, vengan aquí para que podamos investigar qué pecado ha ocurrido
hoy. \bibleverse{39} ¡Juro por la vida del Señor que salva a Israel que,
aunque sea mi hijo Jonatán, tendrá que morir!'' Pero nadie en todo el
ejército dijo nada. \bibleverse{40} Saúl les dijo a todos: ``Ustedes
pónganse a un lado, y yo y mi hijo Jonatán nos pondremos en el lado
opuesto''. ``Hagan lo que les parezca mejor'', respondió el ejército.

\bibleverse{41} Saúl oró al Señor, el Dios de Israel: ``Que el Tumím nos
muestre''.\footnote{\textbf{14:41} En otras palabras, que el Tumin
  muestre quién es el culpable.} Jonatán y Saúl fueron identificados,
mientras que todos los demás fueron absueltos.

\bibleverse{42} Entonces Saúl dijo: ``Echen suertes entre mi hijo
Jonatán y yo''. Jonatán fue seleccionado.

\bibleverse{43} ``Dime qué has hecho'', le preguntó Saúl a Jonatán.
``Sólo probé un poco de miel con la punta de mi bastón'', le dijo
Jonatán. ``Aquí estoy, y tengo que morir''. \footnote{\textbf{14:43} Jos
  7,19}

\bibleverse{44} Saúl dijo: ``¡Que Dios me castigue muy severamente si no
mueres, Jonatán!''

\bibleverse{45} Pero el pueblo le dijo a Saúl: ``¿Tiene que morir
Jonatán, el que logró esta gran victoria en Israel? ¡De ninguna manera!
Juramos por la vida del Señor que ni un solo cabello de su cabeza caerá
al suelo, pues fue con la ayuda de Dios que logró esto hoy''. El pueblo
salvó a Jonatán, y éste no murió. \bibleverse{46} Entonces Saúl dejó de
perseguir a los filisteos, y los filisteos se fueron a su propio país.

\hypertarget{los-otros-actos-de-guerra-de-sauxfal-y-su-familia}{%
\subsection{Los otros actos de guerra de Saúl y su
familia}\label{los-otros-actos-de-guerra-de-sauxfal-y-su-familia}}

\bibleverse{47} Después de que Saúl aseguró su dominio sobre Israel,
luchó contra todos sus enemigos de alrededor: Moabitas, amonitas,
edomitas, los reyes de Soba y los filisteos. En cualquier dirección que
tomara, los derrotaba a todos. \bibleverse{48} Luchó con valentía,
conquistando a los amalecitas y salvando a Israel de los que los
saqueaban. \bibleverse{49} Los hijos de Saúl fueron Jonatán,
Isvi,\footnote{\textbf{14:49} Llamado también Isboset.} y Malquisúa. Los
nombres de sus dos hijas eran Merab, (la primogénita), y Mical, (la
menor). \bibleverse{50} El nombre de su esposa era Ahinoam, hija de
Ahimaas. El nombre del comandante del ejército de Saúl era Abner, hijo
de Ner, y Ner era tío de Saúl. \footnote{\textbf{14:50} 1Sam 17,55}
\bibleverse{51} Cis, padre de Saúl, y Ner, padre de Abner, eran hijos de
Abiel.

\bibleverse{52} Durante toda su vida Saúl estuvo en guerra constante con
los filisteos. Saúl reclutó para su ejército a todo guerrero fuerte y a
todo luchador valiente que encontró.

\hypertarget{la-campauxf1a-de-sauxfal-contra-los-amalecitas-su-desobediencia-a-dios-y-su-rechazo}{%
\subsection{La campaña de Saúl contra los amalecitas; su desobediencia a
Dios y su
rechazo}\label{la-campauxf1a-de-sauxfal-contra-los-amalecitas-su-desobediencia-a-dios-y-su-rechazo}}

\hypertarget{section-14}{%
\section{15}\label{section-14}}

\bibleverse{1} Entonces Samuel le dijo a Saúl: ``El Señor me ha enviado
para ungirte como rey de su pueblo Israel. Así que presta atención a lo
que el Señor dice. \bibleverse{2} Y esto es lo que dice el Señor
Todopoderoso: He observado lo que los amalecitas le hicieron a Israel
cuando los emboscaron en su camino desde Egipto. \footnote{\textbf{15:2}
  Éxod 17,8-16; Deut 25,17-19} \bibleverse{3} Ve y ataca a los
amalecitas y extermínalos a todos. No perdones a nadie, sino que mata a
todo hombre, mujer, niño y bebé; a todo buey, oveja, camello y asno''.
\footnote{\textbf{15:3} Núm 21,2}

\bibleverse{4} Saúl convocó a su ejército en Telem.\footnote{\textbf{15:4}
  Aquí se escribe Telaim, pero se cree que es la misma ciudad llamada
  Telem en Josué 15:24.} Había 200. 000 infantes israelitas y 10. 000
hombres de Judá. \bibleverse{5} Saúl avanzó hacia el pueblo de Amalec y
preparó una emboscada en el valle. \bibleverse{6} Saúl envió un mensaje
para advertirles a los ceneos: ``Salgan de la zona y dejen a los
amalecitas para que no los destruya a ustedes con ellos, porque ustedes
mostraron bondad con todo el pueblo de Israel en su camino desde
Egipto''. Así que los ceneos se alejaron y dejaron abandonaron a los
amalecitas. \footnote{\textbf{15:6} Jue 1,16}

\bibleverse{7} Saúl derrotó a los amalecitas desde Havila hasta Shur, al
oriente de Egipto. \bibleverse{8} Capturó vivo a Agag, rey de Amalec,
pero exterminó a todo el pueblo a espada. \bibleverse{9} Saúl y su
ejército perdonaron a Agag, junto con las mejores ovejas y ganado, los
terneros y corderos gordos, y todo lo que era bueno. No quisieron
destruir eso, sino que destruyeron por completo todo lo despreciable y
que no tenía valor.

\hypertarget{saulo-rechazado-por-dios-a-causa-de-su-desobediencia-el-discurso-de-samuel-y-la-admisiuxf3n-de-culpabilidad-de-sauxfal}{%
\subsection{Saulo rechazado por Dios a causa de su desobediencia; El
discurso de Samuel y la admisión de culpabilidad de
Saúl}\label{saulo-rechazado-por-dios-a-causa-de-su-desobediencia-el-discurso-de-samuel-y-la-admisiuxf3n-de-culpabilidad-de-sauxfal}}

\bibleverse{10} El Señor envió un mensaje a Samuel, diciendo:
\bibleverse{11} ``Lamento haber hecho rey a Saúl, porque ha dejado de
seguirme y no ha hecho lo que le ordené''. Samuel se molestó y clamó al
Señor durante toda la noche.

\bibleverse{12} Entonces Samuel se levantó de madrugada y fue a buscar a
Saúl, pero le dijeron: ``Saúl se ha ido al Carmelo. Allí incluso ha
erigido un monumento para honrarse a sí mismo, y ahora se ha marchado y
ha bajado a Gilgal''.

\bibleverse{13} Cuando Samuel lo alcanzó, Saúl dijo: ``¡El Señor te
bendiga! He hecho lo que el Señor me ha ordenado''.

\bibleverse{14} ``¿Qué es ese balido de las ovejas que escuchan mis
oídos? ¿Qué es ese mugido del ganado que estoy oyendo?'' preguntó
Samuel.

\bibleverse{15} ``El ejército las trajo de los amalecitas'', respondió
Saúl. ``Les perdonaron las mejores ovejas y reses para sacrificarlas al
Señor, tu Dios, pero nosotros destruimos por completo el resto''.

\bibleverse{16} ``¡Cállate!'' le dijo Samuel a Saúl. ``Déjame contarte
lo que el Señor me dijo anoche''. ``Dime lo que dijo'', respondió Saúl.

\bibleverse{17} ``Antes no solías pensar mucho en ti mismo, ¿pero no
eres ahora el líder de las tribus de Israel?'' preguntó Samuel. ``El
Señor te ungió como rey de Israel. \bibleverse{18} Luego te dio una
orden, diciéndote: `Ve y extermina a esos pecadores, los amalecitas.
Atácalos hasta destruirlos a todos'. \bibleverse{19} ¿Por qué no hiciste
lo que el Señor te ordenó? ¿Por qué te abalanzaste sobre el despojo e
hiciste lo malo ante los ojos del Señor?''

\bibleverse{20} ``¡Pero si hice lo que el Señor me ordenó!'' respondió
Saúl. ``Fui e hice lo que el Señor me mandó hacer. Hice regresar a Agag,
rey de Amalec, y destruí por completo a los amalecitas. \bibleverse{21}
El ejército tomó ovejas y ganado del botín, lo mejor de lo que estaba
apartado para Dios, para sacrificarlo al Señor, tu Dios, en Gilgal''.

\bibleverse{22} ``¿Qué crees que prefiere el Señor? ¿Los holocaustos y
los sacrificios? ¿O que seas obediente a su palabra?'' le preguntó
Samuel. ``¡Escucha! ¡La obediencia es mejor que los sacrificios! Prestar
atención es más importante que ofrecer la grasa de los carneros.
\footnote{\textbf{15:22} Os 6,6; Is 1,11; Mat 9,13; Mat 12,7}
\bibleverse{23} La rebelión es tan mala como la brujería, y la
arrogancia es tan mala como el pecado de la idolatría. Porque has
rechazado los mandatos del Señor, él te ha rechazado como rey''.
\footnote{\textbf{15:23} 1Sam 16,1}

\bibleverse{24} ``He pecado'', confesó Saúl a Samuel. ``Desobedecí las
órdenes del Señor y tus instrucciones, porque tuve miedo del pueblo y
seguí lo que ellos decían. \bibleverse{25} Así que, por favor, perdona
mi pecado y vuelve conmigo, para que pueda adorar al Señor''.

\bibleverse{26} Pero Samuel le dijo: ``No voy a volver contigo. Has
rechazado las órdenes del Señor, y el Señor te ha rechazado como rey de
Israel''. \bibleverse{27} Cuando Samuel se dio la vuelta para marcharse,
Saúl se agarró del dobladillo de su túnica, y ésta se rasgó.
\bibleverse{28} Samuel le dijo: ``¡El Señor te ha arrancado hoy el reino
de Israel y se lo ha dado a tu prójimo, a uno que es mejor que tú!
\footnote{\textbf{15:28} 1Sam 28,17} \bibleverse{29} ¡Además la Gloria
de Israel no miente ni cambia de opinión, porque él no es un ser
humano!'' \footnote{\textbf{15:29} Núm 23,19}

\bibleverse{30} ``Sí, he pecado'', respondió Saúl. ``Por favor, hónrame
ahora ante los ancianos de mi pueblo y ante Israel; vuelve conmigo, para
que pueda adorar al Señor, tu Dios''.

\bibleverse{31} Así que Samuel regresó con Saúl después de todo, y Saúl
adoró al Señor.

\hypertarget{samuel-realiza-la-proscripciuxf3n-del-rey-agag-y-se-separa-de-sauxfal-para-no-volver-a-ser-visto}{%
\subsection{Samuel realiza la proscripción del rey Agag y se separa de
Saúl para no volver a ser
visto}\label{samuel-realiza-la-proscripciuxf3n-del-rey-agag-y-se-separa-de-sauxfal-para-no-volver-a-ser-visto}}

\bibleverse{32} Entonces Samuel dijo: ``Tráeme a Agag, rey de los
amalecitas''. Agag se acercó a él confiado, pues pensó: ``La amenaza de
muerte debe haber pasado ya''.

\bibleverse{33} Pero Samuel le dijo: ``De la misma manera que tu espada
ha dejado sin hijos a las mujeres, también tu madre quedará sin hijos
entre las mujeres''. Entonces Samuel descuartizó a Agag ante el Señor en
Gilgal.

\bibleverse{34} Samuel se fue a Ramá, y Saúl se fue a su casa en Guibeá
de Saúl. \bibleverse{35} Hasta el día de su muerte, Samuel no volvió a
visitar a Saúl. Samuel se lamentó por Saúl, y el Señor se arrepintió de
haber hecho a Saúl rey de Israel.

\hypertarget{el-llamado-y-unciuxf3n-de-david-por-samuel}{%
\subsection{El llamado y unción de David por
Samuel}\label{el-llamado-y-unciuxf3n-de-david-por-samuel}}

\hypertarget{section-15}{%
\section{16}\label{section-15}}

\bibleverse{1} El Señor un día le preguntó a Samuel: ``¿Hasta cuándo vas
a seguir llorando a Saúl porque lo he rechazado como rey de Israel?
Llena tu frasco\footnote{\textbf{16:1} ``Frasco'': literalmente,
  ``cuerno''.} con aceite de oliva y vete. Ve donde Isaí de Belén,
porque he elegido un rey para mí de entre sus hijos''. \footnote{\textbf{16:1}
  1Sam 15,23; 1Sam 15,35}

\bibleverse{2} ``¿Cómo puedo ir a hacer eso?'' preguntó Samuel. ``¡Saúl
se enterará y me matará!''. El Señor respondió: ``Lleva contigo una
novilla y di: `He venido a sacrificar al Señor'.

\bibleverse{3} Invita a Isaí al sacrificio, y yo te enseñaré lo que
tienes que hacer. Unge para mí al que yo te diga''.

\bibleverse{4} Samuel hizo lo que el Señor le había dicho y fue a Belén.
Cuando los ancianos de la ciudad le salieron al encuentro, se asustaron
y le preguntaron: ``¿Vienes en son de paz?''

\bibleverse{5} ``Sí, vengo en son de paz'', respondió. ``He venido a
presentar sacrificio al Señor. Purifíquense y vengan conmigo a hacer el
sacrificio''. Entonces purificó a Isaí y a sus hijos y los invitó al
sacrificio.

\hypertarget{samuel-unge-como-rey-al-hijo-menor-de-isauxed-david}{%
\subsection{Samuel unge como rey al hijo menor de Isaí,
David}\label{samuel-unge-como-rey-al-hijo-menor-de-isauxed-david}}

\bibleverse{6} Cuando llegaron y Samuel vio a Eliab, pensó para sí:
``¡Este tiene que ser el ungido del Señor!''.

\bibleverse{7} Pero el Señor le dijo a Samuel: ``No te fijes en su
aspecto exterior ni en su altura porque lo he rechazado. Porque el Señor
no mira como los seres humanos. Los seres humanos sólo ven con sus ojos
lo que está en el exterior, pero el Señor mira la forma de pensar de las
personas en su interior''. \footnote{\textbf{16:7} Hech 10,34; Sal 7,10}

\bibleverse{8} Entonces Isaí llamó a Abinadab y lo hizo venir ante
Samuel, quien dijo: ``El Señor tampoco ha elegido a éste''.
\bibleverse{9} Entonces Isaí hizo que Simea se presentara. Pero Samuel
dijo: ``El Señor tampoco ha elegido a éste''. \bibleverse{10} Isaí hizo
que siete de sus hijos se presentaran ante Samuel, pero éste le dijo:
``El Señor no ha elegido a ninguno de éstos''. \bibleverse{11} Entonces
le preguntó a Isaí: ``¿No tienes más hijos?'' . ``Bueno, aún queda el
más joven'', respondió Isaí, ``pero está fuera cuidando las ovejas''.
``Manda a buscarlo y tráelo aquí, porque no nos vamos a sentar a
comer\footnote{\textbf{16:11} ``Sentar a comer'': literalmente,
  ``rodear''. Normalmente se piensa que significa rodear una mesa antes
  de sentarse, pero también podría significar ``rodear'' un altar, es
  decir, el comienzo de los rituales de sacrificio.} hasta que llegue
aquí'', le dijo Samuel a Isaí. \footnote{\textbf{16:11} 1Sam 17,14}

\bibleverse{12} Así que Isaí mandó a buscarlo y lo trajo delante de
Samuel. Tenía una tez roja y unos ojos hermosos, y tenía buen parecer.
El Señor dijo: ``Ve a ungirlo, porque es él''.

\bibleverse{13} Samuel tomó el frasco de aceite de oliva y lo ungió en
presencia de sus hermanos, y el Espíritu del Señor vino sobre David con
poder desde aquel día. Luego Samuel se fue y regresó a Ramá.

\hypertarget{david-es-llamado-a-tocar-el-arpa-en-la-corte-de-sauxfal-y-entra-al-servicio-real}{%
\subsection{David es llamado a tocar el arpa en la corte de Saúl y entra
al servicio
real}\label{david-es-llamado-a-tocar-el-arpa-en-la-corte-de-sauxfal-y-entra-al-servicio-real}}

\bibleverse{14} El Espíritu del Señor había abandonado a Saúl, y un
espíritu maligno del Señor lo atormentaba.\footnote{\textbf{16:14} Como
  en otras partes de la Escritura, a veces se presenta a Dios como si
  hiciera algo que en realidad no impide. La eliminación del Espíritu
  del Señor dejó a Saúl abierto al control de otro espíritu. La forma en
  que los siervos reaccionan muestra que esta era una visión común de la
  época: se responsabiliza a Dios de los problemas de Saúl.} \footnote{\textbf{16:14}
  1Sam 18,10} \bibleverse{15} Los siervos de Saúl le dijeron: ``Sin duda
es un espíritu maligno de Dios el que te atormenta. \bibleverse{16}
Danos aquí la orden de encontrar a alguien que sea bueno tocando el
arpa, para que cuando el espíritu maligno de Dios venga sobre ti, pueda
tocar y te sientas mucho mejor''.

\bibleverse{17} Saúl dio la orden a sus siervos: ``Busquen a alguien que
sea bueno tocando el arpa y tráiganlo aquí''.

\bibleverse{18} Uno de los criados respondió: ``Conozco a un hijo de
Isaí, de Belén, que es bueno tocando el arpa. Es un hombre valiente,
buen luchador, de buen hablar y guapo, y el Señor está con él''.

\bibleverse{19} Saúl envió mensajeros a Isaí, diciéndole: ``Envíame a tu
hijo David, el que cuida las ovejas''.

\bibleverse{20} Así que Isaí cargó un burro con pan, un odre de vino y
un cabrito y los envió con su hijo David a Saúl. \bibleverse{21} David
llegó a Saúl y comenzó a trabajar para él. Saúl lo apreciaba mucho, y
David se convirtió en su escudero. \bibleverse{22} Saúl envió un mensaje
a Isaí, diciendo: ``Por favor, permite que David siga trabajando para
mí, porque estoy complacido con él''. \bibleverse{23} Así, cada vez que
el espíritu de Dios se apoderaba de Saúl, David tomaba su arpa y tocaba,
y Saúl se aliviaba y se sentía mejor, y el espíritu maligno lo
dejaba.\footnote{\textbf{16:23} 1Sam 16,14}

\hypertarget{david-y-el-campeuxf3n-enemigo-goliat}{%
\subsection{David y el campeón enemigo
Goliat}\label{david-y-el-campeuxf3n-enemigo-goliat}}

\hypertarget{section-16}{%
\section{17}\label{section-16}}

\bibleverse{1} Los ejércitos filisteos se reunieron para la batalla en
Soco, en Judá. Acamparon en Efes-damim, entre Socoh y Azeca.
\bibleverse{2} Saúl y los israelitas se reunieron y acamparon en el
Valle de Ela y tomaron sus posiciones para comenzar la batalla contra
los filisteos. \bibleverse{3} Los filisteos estaban en una colina y los
israelitas en otra, con el valle entre ellos. \bibleverse{4} Entonces
salió del campamento filisteo un campeón.\footnote{\textbf{17:4}
  ``Campeón'': literalmente ``un hombre del espacio intermedio''. Suele
  entenderse como un campeón que luchará contra otro en una especie de
  batalla por delegación, pero su significado preciso es incierto, ya
  que sólo aparece aquí y en el versículo 23 en todo el Antiguo
  Testamento.} Se llamaba Goliat, de Gat, y medía seis codos y un
palmo.\footnote{\textbf{17:4} ``Seis codos y un palmo de altura''. Esto
  equivale a unos nueve pies y medio. La Septuaginta y un manuscrito de
  Qumrán tienen cuatro codos y un palmo, lo que equivale a seis pies y
  medio.} \bibleverse{5} Tenía en la cabeza un casco de bronce y llevaba
una cota de malla de bronce que pesaba cinco mil siclos. \bibleverse{6}
En las piernas llevaba una armadura de bronce y una jabalina\footnote{\textbf{17:6}
  ``Jabalina'': algunos creen que se trata más bien de una espada curva
  o una cimitarra, y ciertamente se hace referencia a una espada en el
  versículo 51.} colgada entre sus hombros. \bibleverse{7} El asta de su
lanza era tan gruesa como una viga de tejedor, con una punta de hierro
que pesaba seiscientos siclos. Su escudero caminaba delante de él
llevando su escudo.\footnote{\textbf{17:7} ``Llevando su escudo'':
  añadido para mayor claridad.} \bibleverse{8} Goliat se puso de pie y
gritó a las filas de soldados israelitas: ``¿Por qué han venido y se han
puesto en fila para la batalla? Yo soy el filisteo, y ustedes son los
siervos de Saúl. Elijan a uno de sus hombres y hagan que descienda a
pelear conmigo. \bibleverse{9} Si él puede pelear conmigo y logra
matarme, entonces los filisteos seremos sus esclavos. Pero si lo venzo y
lo mato, entonces ustedes serán nuestros esclavos y trabajarán para
nosotros''. \bibleverse{10} Entonces el filisteo dijo: ``¡Me burlo de
las líneas de batalla de Israel hoy! Dénme un hombre para que podamos
luchar los dos''. \footnote{\textbf{17:10} 2Re 19,4; 2Re 19,16}

\bibleverse{11} Saúl y todos los soldados israelitas quedaron
destrozados y absolutamente aterrados cuando oyeron lo que dijo el
filisteo.

\hypertarget{david-enviado-por-su-padre-a-sus-hermanos-en-el-campamento-estuxe1-indignado-por-la-arrogancia-de-goliat-y-se-siente-llamado-a-pelear-con-uxe9l}{%
\subsection{David, enviado por su padre a sus hermanos en el campamento,
está indignado por la arrogancia de Goliat y se siente llamado a pelear
con
él}\label{david-enviado-por-su-padre-a-sus-hermanos-en-el-campamento-estuxe1-indignado-por-la-arrogancia-de-goliat-y-se-siente-llamado-a-pelear-con-uxe9l}}

\bibleverse{12} David era uno de los hijos de Isaí, un efrateo de Belén
de Judá que tenía ocho hijos. En la época en que Saúl era rey, Isaí era
muy viejo. \bibleverse{13} Los tres hijos mayores de Isaí se habían
unido a la guerra como parte del ejército de Saúl. Ellos eran Eliab (el
primogénito), Abinadab (el segundo) y Simea (el tercero). \footnote{\textbf{17:13}
  1Sam 16,6; 1Sam 16,8-9} \bibleverse{14} David era el más joven. Los
tres hijos mayores estaban con Saúl, \bibleverse{15} mientras que David
iba con Saúl y luego volvía para cuidar las ovejas de su padre.

\bibleverse{16} Todas las mañanas y las tardes, durante cuarenta días,
el filisteo salió y se puso en pie en el mismo lugar.

\bibleverse{17} Isaí le dijo a su hijo David: ``Por favor, lleva a tus
hermanos este efa de grano tostado y estos diez panes para tus hermanos.
Llévalos rápidamente al campamento de tus hermanos. \bibleverse{18}
Además, lleva estos diez trozos de queso a su comandante. Comprueba con
cuidado cómo están tus hermanos y tráeme noticias de ellos''.
\bibleverse{19} Sus hermanos estaban con Saúl y todo el ejército
israelita en el Valle de Ela, luchando contra los filisteos.

\bibleverse{20} David se levantó de madrugada y dejó el rebaño con un
pastor. Tomó las provisiones y se puso en marcha como se lo había dicho
Isaí. Llegó al campamento justo cuando el ejército marchaba hacia su
línea de batalla, gritando el grito de guerra. \bibleverse{21} Los
israelitas se colocaron en su línea de batalla y los filisteos en la del
lado opuesto. \bibleverse{22} David dejó sus provisiones con el
responsable y corrió a la línea de batalla. Cuando llegó allí, preguntó
a sus hermanos cómo estaban. \bibleverse{23} Mientras hablaba con ellos,
Goliat, el campeón filisteo de Gat, salió de sus filas y gritó su
desafío como antes, y esta vez David escuchó lo que decía.
\bibleverse{24} Todos los soldados israelitas huyeron al verlo, porque
tenían un miedo terrible. \bibleverse{25} ``¿Han visto a ese hombre que
no deja de salir para burlarse de Israel?'' , preguntaron. ``El rey hará
muy rico al hombre que lo mate. También le dará a su hija en matrimonio,
y su familia vivirá libre de impuestos en Israel''.

\bibleverse{26} Entonces David les preguntó a los hombres que estaban a
su lado: ``¿Qué recibirá el hombre que mate a este filisteo y elimine
esta vergüenza de Israel? ¿Quién se cree que es este Filisteo
pagano\footnote{\textbf{17:26} ``Pagano'': literalmente,
  ``incircunciso''. Del mismo modo ocurre en el versículo 36.} para
burlarse del Dios vivo de los ejércitos?'' .

\bibleverse{27} Los soldados repitieron lo que habían dicho, diciéndole:
``Esto es lo que recibirá el que lo mate''.

\bibleverse{28} Cuando Eliab, el hermano mayor de David, lo oyó hablar
con los hombres, se enojó con él. ``¿Qué haces aquí?'' , le preguntó.
``¿Con quién has dejado esas pocas ovejas en el desierto? ¡Sé lo
orgulloso y malvado que eres! Sólo has venido a ver la batalla''.

\bibleverse{29} ``¿Qué he hecho ahora?'' preguntó David. ``¿No puedo ni
siquiera hacer una pregunta?'' . \bibleverse{30} Se acercó a otros y les
hizo la misma pregunta, y ellos le dieron la misma respuesta que antes.

\hypertarget{david-se-ofrece-a-duelo-pero-rechaza-la-armadura-de-sauxfal-y-solo-usa-su-honda-como-arma}{%
\subsection{David se ofrece a duelo, pero rechaza la armadura de Saúl y
solo usa su honda como
arma}\label{david-se-ofrece-a-duelo-pero-rechaza-la-armadura-de-sauxfal-y-solo-usa-su-honda-como-arma}}

\bibleverse{31} Alguien escuchó lo que dijo David y se lo comunicó a
Saúl, que mandó a buscarlo. \bibleverse{32} David le dijo a Saúl: ``Que
nadie se desanime por culpa de este filisteo. Yo, tu siervo, iré a
luchar contra él''.

\bibleverse{33} ``No puedes ir a luchar contra ese filisteo'', respondió
Saúl. ``Tú eres sólo un muchacho, y él es un guerrero entrenado desde su
juventud''.

\bibleverse{34} David respondió: ``Tu siervo ha estado cuidando las
ovejas de su padre. Cuando venía un león o un oso y se llevaba un
cordero del rebaño, \bibleverse{35} yo lo perseguía, lo derribaba y
salvaba el cordero de su boca. Si se volvía para atacarme, le agarraba
el pelo, lo golpeaba y lo mataba. \bibleverse{36} He matado leones y
osos, y este pagano filisteo será como uno de ellos, pues se ha burlado
de los ejércitos del Dios vivo''. \bibleverse{37} David concluyó: ``El
Señor, que me salvó de las garras del león y del oso, y del mismo modo
me salvará de este filisteo''. ``Ve, y que el Señor esté contigo'',
respondió Saúl.

\bibleverse{38} Saúl le dio a David su propia ropa de combate para que
se la pusiera, le colocó un casco de bronce en la cabeza y le puso una
armadura. \bibleverse{39} David se puso la espada sobre la armadura,
pero no podía caminar porque no estaba acostumbrado. ``No puedo caminar
con todo esto'', le dijo David a Saúl. ``No estoy acostumbrado''. Así
que David se quitó toda la armadura.

\bibleverse{40} Tomó su bastón, escogió cinco piedras lisas del arroyo y
las puso en su bolsa de pastor. Llevando su honda en la mano, se acercó
al filisteo.

\hypertarget{la-lucha-victoriosa-de-david-con-goliat}{%
\subsection{La lucha victoriosa de David con
Goliat}\label{la-lucha-victoriosa-de-david-con-goliat}}

\bibleverse{41} El filisteo se acercó a David, cada vez más cerca, con
su escudero al frente. \bibleverse{42} Cuando el filisteo miró de cerca,
pudo ver que David era sólo un joven apuesto de cara roja, y entonces
trató a David con desprecio. \footnote{\textbf{17:42} 1Sam 16,12}
\bibleverse{43} ``¿Piensas que soy un perro para venir a pelear conmigo
con un palo?'' , le preguntó el filisteo a David, y lo maldijo por sus
dioses. \bibleverse{44} Entonces el filisteo le gritó a David: ``Ven
aquí, y daré de comer tu carne a las aves y a los animales salvajes''.

\bibleverse{45} David le respondió al filisteo: ``Tú vienes a atacarme
con espada, lanza y jabalina. Pero yo vengo a atacarte en nombre del
Señor Todopoderoso, el Dios de los ejércitos de Israel, del que te has
burlado. \bibleverse{46} Hoy el Señor te entregará en mis manos, y yo te
derribaré; te cortaré la cabeza y entregaré los cadáveres de los
soldados filisteos a las aves y a los animales salvajes. Entonces todo
el mundo sabrá que hay un Dios que actúa por Israel. \bibleverse{47}
Todos los aquí reunidos se darán cuenta de que el Señor salva, pero no
con espada y lanza. Porque la batalla es del Señor, y él nos entregará a
todos los filisteos''.

\bibleverse{48} Cuando el filisteo avanzó para atacarlo, David corrió
hacia la línea de batalla para enfrentarlo. \bibleverse{49} David metió
la mano en su bolsa, sacó una piedra y la disparó con su honda,
golpeando al filisteo en la frente. La piedra se le clavó en la frente,
y Goliat se desplomó boca abajo en el suelo. \bibleverse{50} Así fue
como David derrotó al filisteo con sólo una honda y una piedra; sin
espada en la mano, David derribó al filisteo y lo mató. \bibleverse{51}
Entonces David corrió y se paró sobre el filisteo. Tomó la espada del
filisteo y la sacó de su vaina. Lo mató y luego le cortó la cabeza con
la espada. Cuando los filisteos vieron que su campeón estaba muerto,
dieron la vuelta y huyeron.

\bibleverse{52} Entonces los hombres de Israel y de Judá se lanzaron al
grito de guerra y persiguieron a los filisteos hasta Gat y hasta las
puertas de Ecrón. Sus cuerpos fueron esparcidos a lo largo del camino de
Saaraim hacia Gat y Ecrón. \bibleverse{53} Cuando los israelitas
regresaron de su acalorada persecución a los filisteos, saquearon sus
campamentos. \bibleverse{54} David tomó la cabeza del filisteo y la
llevó a Jerusalén, pero puso las armas del filisteo en su propia tienda.

\hypertarget{sauxfal-pregunta-por-david}{%
\subsection{Saúl pregunta por David}\label{sauxfal-pregunta-por-david}}

\bibleverse{55} Cuando Saúl vio que David salía a luchar contra el
filisteo, le preguntó a Abner, el comandante del ejército: ``Abner, ¿de
quién es hijo ese joven?'' ``Por su vida, Su Majestad, no lo sé'',
respondió Abner. \footnote{\textbf{17:55} 1Sam 14,50}

\bibleverse{56} ``Averigua de quién es hijo este joven'', ordenó el rey.

\bibleverse{57} En cuanto David regresó de matar al filisteo, Abner lo
tomó y lo llevó ante Saúl. David todavía tenía la cabeza del filisteo en
la mano. \bibleverse{58} ``¿De quién eres hijo, joven?'' preguntó Saúl.
``Soy hijo de tu siervo Isaí de Belén'', respondió David.

\hypertarget{david-llega-a-la-corte-de-sauxfal-su-relaciuxf3n-con-sauxfal-y-jonatuxe1n}{%
\subsection{David llega a la corte de Saúl; su relación con Saúl y
Jonatán}\label{david-llega-a-la-corte-de-sauxfal-su-relaciuxf3n-con-sauxfal-y-jonatuxe1n}}

\hypertarget{section-17}{%
\section{18}\label{section-17}}

\bibleverse{1} Después de que David terminó de hablar con Saúl, Jonatán
se hizo gran amigo de David. Amaba a David como a sí mismo.
\bibleverse{2} Desde entonces, Jonatán hizo que David trabajara para él
y no lo dejó volver a su casa. \bibleverse{3} Jonatán hizo un acuerdo
solemne con David porque lo amaba como a sí mismo. \footnote{\textbf{18:3}
  1Sam 19,1; 1Sam 20,17; 1Sam 23,18; 2Sam 1,26; 2Sam 21,7}
\bibleverse{4} Jonatán se quitó la túnica que llevaba puesta y se la dio
a David, junto con su túnica, su espada, su arco y su
cinturón.\footnote{\textbf{18:4} Estas acciones fueron una forma de
  confirmar el acuerdo.}

\bibleverse{5} David tenían éxito al hacer todo lo que Saúl le pedía,
así que Saúl lo nombró oficial del ejército. Esto complació a todos,
incluso a los demás oficiales de Saúl.

\hypertarget{regreso-festivo-de-los-guerreros-david-fue-celebrado-como-el-vencedor-por-la-gente}{%
\subsection{Regreso festivo de los guerreros; David fue celebrado como
el vencedor por la
gente}\label{regreso-festivo-de-los-guerreros-david-fue-celebrado-como-el-vencedor-por-la-gente}}

\bibleverse{6} Cuando los soldados regresaron a casa después de que
David había matado al filisteo, las mujeres de todos los pueblos de
Israel salieron cantando y bailando al encuentro del rey Saúl,
celebrando alegremente con panderetas e instrumentos musicales.
\footnote{\textbf{18:6} Jue 11,34} \bibleverse{7} Mientras bailaban, las
mujeres cantaban: ``Saúl ha matado a sus miles, y David a sus decenas de
miles''. \footnote{\textbf{18:7} 1Sam 21,12; 1Sam 29,5}

\bibleverse{8} Lo que cantaban enojó mucho a Saúl, pues no le pareció
bien. Se dijo a sí mismo: ``A David le han dado el crédito de haber
matado a decenas de miles, pero a mí sólo a miles. Lo único que falta es
darle el reino''. \bibleverse{9} Desde entonces Saúl miró a David con
recelo.

\hypertarget{david-odiado-mortalmente-por-sauxfal-demuestra-ser-un-huxe9roe-de-guerra}{%
\subsection{David, odiado mortalmente por Saúl, demuestra ser un héroe
de
guerra}\label{david-odiado-mortalmente-por-sauxfal-demuestra-ser-un-huxe9roe-de-guerra}}

\bibleverse{10} Al día siguiente, un espíritu maligno de Dios se apoderó
de Saúl con fuerza, y despotricó\footnote{\textbf{18:10} ``Despotricó'':
  la palabra se traduce normalmente como ``profetizar'' (véase, por
  ejemplo, 10:10 cuando se aplica a Saúl), pero la función principal de
  un verdadero profeta de Dios era entregar mensajes de Dios. Que la
  fuente fuera ``un espíritu maligno'' no encaja en tal imagen, incluso
  si el espíritu maligno ``viniera de Dios''.} dentro de la casa
mientras David tocaba el arpa como lo hacía habitualmente. Resulta que
Saúl tenía una lanza en la mano, \footnote{\textbf{18:10} 1Sam 16,14}
\bibleverse{11} y se la lanzó a David, mientras pensaba: ``Clavaré a
David en la pared''. Pero David logró escapar de él dos veces.
\footnote{\textbf{18:11} 1Sam 19,10; 1Sam 20,33} \bibleverse{12} Saúl
tenía miedo de David, porque el Señor estaba con él, pero se había
rendido ante Saúl. \bibleverse{13} Así que Saúl despidió a David y lo
nombró comandante de mil soldados, dirigiéndolos de ida y vuelta como
parte del ejército.

\bibleverse{14} David siguió teniendo mucho éxito en todo lo que hacía,
porque el Señor estaba con él. \footnote{\textbf{18:14} 1Sam 18,5}
\bibleverse{15} Cuando Saúl vio el éxito de David, le tuvo aún más
miedo. \bibleverse{16} Pero todos en Israel y en Judá amaban a David,
por su liderazgo en el ejército.

\hypertarget{david-engauxf1ado-para-casarse-con-la-hija-mayor-de-sauxfal-se-casuxf3-con-su-hermana-menor-michal}{%
\subsection{David, engañado para casarse con la hija mayor de Saúl, se
casó con su hermana menor,
Michal}\label{david-engauxf1ado-para-casarse-con-la-hija-mayor-de-sauxfal-se-casuxf3-con-su-hermana-menor-michal}}

\bibleverse{17} Un día Saúl le dijo a David: ``Aquí está mi hija mayor,
Merab. Te la daré en matrimonio, pero sólo si me demuestras que eres un
guerrero valiente y luchas en las batallas del Señor''. Porque Saúl
pensaba: ``No hace falta que sea yo quien lo mate; que lo hagan los
filisteos''.

\bibleverse{18} ``Pero, ¿quién soy yo, y qué categoría tiene mi familia
en Israel, para que me convierta en yerno del rey?'' respondió
David.\footnote{\textbf{18:18} David may have been concerned at the cost
  of providing a dowry, especially as this is a condition of marriage
  mentioned later in verse 25.}

\bibleverse{19} Sin embargo, cuando llegó el momento de entregar a
Merab, la hija de Saúl, a David, ésta fue dada en matrimonio a Adriel de
Meholá en su lugar.

\hypertarget{el-servicio-militar-de-david-para-la-novia}{%
\subsection{El servicio militar de David para la
novia}\label{el-servicio-militar-de-david-para-la-novia}}

\bibleverse{20} Mientras tanto, la hija de Saúl, Mical, se había
enamorado de David, y cuando se lo dijeron a Saúl, se alegró de ello.
\bibleverse{21} ``Se la daré a David'', pensó Saúl. ``Ella puede ser la
carnada para que los filisteos lo atrapen''. Entonces Saúl le dijo a
David: ``Esta es la segunda vez que puedes ser mi yerno''.

\bibleverse{22} Saúl les dio estas instrucciones a sus siervos: ``Hablen
con David en privado y díganle: `Mira, el rey está muy contento contigo
y todos te queremos. ¿Por qué no te conviertes en el yerno del rey?'\,''
\footnote{\textbf{18:22} 1Sam 22,14}

\bibleverse{23} Los sirvientes de Saúl hablaron en privado con David,
pero él respondió: ``¿Creen que no es nada hacerse yerno del rey? Soy un
hombre pobre y no soy importante''.

\bibleverse{24} Cuando los sirvientes de Saúl le explicaron lo que David
había dicho,

\bibleverse{25} Saúl les dijo: ``Díganle a David que la única dote que
el rey quiere para la novia son cien prepucios de filisteos muertos,
como forma de vengarse de sus enemigos''. El plan de Saúl era hacer que
los filisteos mataran a David. \bibleverse{26} Cuando los sirvientes le
informaron a David de lo que el rey había dicho, éste se alegró de ser
el yerno del rey. Mientras había tiempo, \bibleverse{27} David partió
con sus hombres y mató a doscientos filisteos, y trajo sus prepucios.
Los contaron todos ante el rey para que David se convirtiera en yerno
del rey. Entonces Saúl le dio a su hija Mical en matrimonio.
\bibleverse{28} Saúl se dio cuenta de que el Señor estaba con David y de
que su hija Mical estaba enamorada de David, \bibleverse{29} por lo que
se volvió aún más temeroso de David, y fue enemigo de éste por el resto
de su vida.

\bibleverse{30} Cada vez que los comandantes filisteos atacaban, David
tenía más éxito en la batalla que todos los oficiales de Saúl, por lo
que su fama se extendió rápidamente.

\hypertarget{la-reconciliaciuxf3n-de-sauxfal-con-david-como-resultado-de-la-intercesiuxf3n-de-jonatuxe1n-despuuxe9s-de-los-repetidos-asesinatos-de-sauxfal-david-huye-a-samuel}{%
\subsection{La reconciliación de Saúl con David como resultado de la
intercesión de Jonatán; Después de los repetidos asesinatos de Saúl,
David huye a
Samuel}\label{la-reconciliaciuxf3n-de-sauxfal-con-david-como-resultado-de-la-intercesiuxf3n-de-jonatuxe1n-despuuxe9s-de-los-repetidos-asesinatos-de-sauxfal-david-huye-a-samuel}}

\hypertarget{section-18}{%
\section{19}\label{section-18}}

\bibleverse{1} Entonces Saúl ordenó a su hijo Jonatán y a todos sus
funcionarios que mataran a David. Pero Jonatán apreciaba mucho David,
\footnote{\textbf{19:1} 1Sam 18,3} \bibleverse{2} así que le advirtió:
``Mi padre Saúl está tratando de matarte. Así que ten cuidado mañana por
la mañana: busca un lugar donde esconderte y permanece oculto.
\bibleverse{3} Yo saldré con mi padre y me pondré en el campo cerca de
donde te escondes. Hablaré con él sobre ti y veré lo que puedo
averiguar, y luego te avisaré''.

\bibleverse{4} Entonces Jonatán habló positivamente de David a su padre
Saúl, y le dijo: ``El rey no debe hacer nada malo a su siervo David,
porque él no le ha hecho nada malo; siempre le ha servido bien.
\bibleverse{5} Se tomó la vida en sus manos cuando mató al filisteo, y
el Señor logró una gran salvación para todo Israel. Tú lo viste y te
alegraste, así que ¿por qué pecar y derramar sangre inocente matando a
David sin tener ninguna razón?''

\bibleverse{6} Saúl aceptó lo que Jonatán tenía que decir y prometió con
un juramento ``Juro por la vida del Señor que no lo matarán''.

\bibleverse{7} Más tarde Jonatán llamó a David y le contó todo lo que se
había dicho. Luego lo llevó ante Saúl, y David trabajó para Saúl como lo
había hecho antes.

\hypertarget{la-nueva-fortuna-de-david-en-la-guerra-el-repetido-intento-de-asesinato-de-sauxfal}{%
\subsection{La nueva fortuna de David en la guerra; El repetido intento
de asesinato de
Saúl}\label{la-nueva-fortuna-de-david-en-la-guerra-el-repetido-intento-de-asesinato-de-sauxfal}}

\bibleverse{8} La guerra estalló de nuevo, y David fue a luchar contra
los filisteos. Los atacó con tanta fuerza que huyeron derrotados.

\bibleverse{9} Algún tiempo después, un espíritu maligno del Señor se
apoderó de Saúl mientras estaba sentado en su casa con su lanza en la
mano. Mientras David tocaba la lira, \footnote{\textbf{19:9} 1Sam
  18,10-11} \bibleverse{10} Saúl intentó clavar a David en la pared con
la lanza. Pero David logró esquivar la lanza que se incrustó en la
pared. Entonces David escapó y huyó en la noche.

\hypertarget{el-escape-de-david-a-su-hogar-y-su-salvaciuxf3n-a-travuxe9s-de-la-astucia-de-michal}{%
\subsection{El escape de David a su hogar y su salvación a través de la
astucia de
Michal}\label{el-escape-de-david-a-su-hogar-y-su-salvaciuxf3n-a-travuxe9s-de-la-astucia-de-michal}}

\bibleverse{11} Saúl envió algunos mensajeros a la casa de David para
que vigilaran y lo mataran por la mañana. Pero Mical, la mujer de David,
le advirtió: ``Si no te escapas esta noche, mañana te matarán''.
\bibleverse{12} Mical bajó a David desde una ventana, y él salió
corriendo, logrando escapar. \bibleverse{13} Luego tomó un ídolo de
casa\footnote{\textbf{19:13} ``Ídolo de casa'': la palabra hebrea
  utilizada aquí es teraphim y se mencionan por primera vez en Génesis
  31. Eran objetos de culto que se utilizaban para determinar la
  voluntad del ``dios'', véase Ezequiel 21:21; Zacarías 10:2 . El hecho
  de que tales ídolos estuvieran allí en la casa de David muestra el
  grado en que la ``religión pura'' se había corrompido con el tiempo.}
y lo acostó en la cama, le puso una peluca de pelo de cabra en la cabeza
y lo cubrió con la ropa de cama. \bibleverse{14} Cuando Saúl envió a los
mensajeros a detener a David, Mical les dijo: ``Está enfermo''.

\bibleverse{15} Saúl envió a los mensajeros a ver a David, diciendo:
``Tráiganmelo en la cama para que lo mate''. \bibleverse{16} Pero cuando
los mensajeros entraron en el dormitorio, allí estaba el ídolo en la
cama con la peluca de pelo de cabra en la cabeza.

\bibleverse{17} ``¿Por qué me has engañado así, ayudando a mi enemigo a
escaparse para que pueda huir?'' preguntó Saúl a Mical. ``Me dijo:
`¡Apártate de mi camino! No quiero tener que matarte'\,'', respondió
Mical.

\hypertarget{david-con-samuel-en-rama-el-rapto-profuxe9tico-de-sauxfal-en-la-casa-profuxe9tica-alluxed}{%
\subsection{David con Samuel en Rama; El rapto profético de Saúl en la
casa profética
allí}\label{david-con-samuel-en-rama-el-rapto-profuxe9tico-de-sauxfal-en-la-casa-profuxe9tica-alluxed}}

\bibleverse{18} Así fue como David se alejó y escapó. Fue a ver a Samuel
en Ramá y le explicó todo lo que Saúl le había hecho. Luego, él y Samuel
se fueron a hospedar en Naiot. \bibleverse{19} Cuando Saúl se enteró de
que David estaba en Naiot, en Ramá,

\bibleverse{20} envió mensajeros para arrestarlo. Pero cuando vieron a
un grupo de profetas que profetizaban con Samuel al frente, el Espíritu
de Dios vino sobre los mensajeros de Saúl y ellos también comenzaron a
profetizar. \footnote{\textbf{19:20} 1Sam 10,10-12} \bibleverse{21} Saúl
fue informado de lo que había sucedido, así que envió más mensajeros, y
ellos también comenzaron a profetizar. \bibleverse{22} Por tercera vez
Saúl envió mensajeros, y ellos también comenzaron a profetizar.

\bibleverse{23} Al final, Saúl fue él mismo a Ramá y llegó a la gran
cisterna de Secu. ``¿Dónde están Samuel y David?'' , preguntó. ``En
Naiot, en Ramá'', le dijeron. Así que Saúl se dirigió a Naiot en Ramá,
pero el Espíritu de Dios incluso vino sobre él, y estuvo profetizando
mientras caminaba hasta que llegó a Naiot. \bibleverse{24} Entonces Saúl
también se quitó la ropa y también profetizó en presencia de Samuel.
Luego se postró y estuvo desnudo todo ese día y toda esa noche. Por eso
se dice: ``¿Es Saúl también uno de los profetas?''

\hypertarget{la-reuniuxf3n-de-david-y-la-discusiuxf3n-con-jonatuxe1n-renovaciuxf3n-de-su-alianza-de-amistad}{%
\subsection{La reunión de David y la discusión con Jonatán; Renovación
de su alianza de
amistad}\label{la-reuniuxf3n-de-david-y-la-discusiuxf3n-con-jonatuxe1n-renovaciuxf3n-de-su-alianza-de-amistad}}

\hypertarget{section-19}{%
\section{20}\label{section-19}}

\bibleverse{1} David corrió desde Naiot en Ramá hasta donde estaba
Jonatán y le preguntó: ``¿Qué he hecho? ¿Qué mal he hecho? ¿Qué cosa
terrible le he hecho a tu padre para que quiera matarme?''

\bibleverse{2} ``¡Nada!'' Respondió Jonatán. ``¡No vas a morir!
¡Escucha! Mi padre me cuenta todo lo que planea, sea lo que sea. ¿Por
qué iba mi padre a ocultarme algo así? No es cierto''.

\bibleverse{3} Pero David volvió a jurar: ``Tu padre sabe muy bien que
soy tu amigo, y por eso seguro ha pensado: `Jonatán no puede enterarse
de esto, porque si no se enfadará mucho'. Te juro por la vida del Señor,
y por tu propia vida, que mi vida pende de un hilo''.\footnote{\textbf{20:3}
  ``Mi vida pende de un hilo'': literalmente, ``sólo hay un paso entre
  mí y la muerte''.}

\bibleverse{4} ``Dime qué quieres que haga por ti y lo haré'', le dijo
Jonatán a David.

\hypertarget{la-sugerencia-de-david}{%
\subsection{La sugerencia de David}\label{la-sugerencia-de-david}}

\bibleverse{5} ``Bueno, la fiesta de la Luna Nueva es mañana, y tengo
que sentarme a comer con el rey. Pero si te parece bien, pienso ir a
esconderme en el campo hasta la noche de dentro de tres días.
\bibleverse{6} Si tu padre me echa de menos, dile: `David ha tenido que
pedirme urgentemente permiso para bajar a Belén, su ciudad natal, a
causa de un sacrificio anual que se celebra allí para todo su grupo
familiar'. \bibleverse{7} Si dice: `Está bien', entonces no hay problema
para mí, tu siervo, pero si se enfada, sabrás que pretende hacerme daño.
\bibleverse{8} Así que, por favor, trátame bien, como prometiste cuando
hiciste un acuerdo conmigo ante el Señor. Si he hecho mal, ¡mátame tú
mismo! ¿Por qué me llevas a tu padre para que lo haga?''

\bibleverse{9} ``¡De ninguna manera!'' respondió Jonatán. ``Si supiera
con certeza que mi padre tiene planes para hacerte daño, ¿no crees que
te lo diría?''

\bibleverse{10} ``Entonces, ¿quién me va a avisar si tu padre te da una
respuesta desagradable?'' preguntó David.

\bibleverse{11} ``Vamos, salgamos al campo'', dijo Jonatán. Así que
ambos salieron al campo.

\hypertarget{el-juramento-mutuo}{%
\subsection{El juramento mutuo}\label{el-juramento-mutuo}}

\bibleverse{12} Jonatán le dijo a David: ``Te prometo por el Señor, el
Dios de Israel, que mañana a esta hora o pasado mañana interrogaré a mi
padre. Si las cosas se ven bien para ti, te enviaré un mensaje y te lo
haré saber. \bibleverse{13} Pero si mi padre planea hacerte daño, que el
Señor me castigue muy severamente, si no te lo hago saber enviándote un
mensaje para que puedas salir a salvo. Que el Señor esté contigo, como
lo estuvo con mi padre. \bibleverse{14} Mientras viva, por favor,
demuéstrame un amor digno de confianza como el del Señor para que no
muera, \bibleverse{15} y por favor, no retires tu amor fiel a mi
familia, aunque el Señor haya eliminado a todos tus enemigos de la
tierra''. \bibleverse{16} Jonatán hizo un acuerdo solemne con la familia
de David, diciendo: ``Que el Señor imponga su castigo a los enemigos de
David''.\footnote{\textbf{20:16} Éste y los versos anteriores tienen una
  serie de problemas de traducción.}

\bibleverse{17} Jonatán se lo hizo jurar a David una vez más, basándose
en el amor que le profesaba, pues Jonatán ya amaba a David como a sí
mismo. \footnote{\textbf{20:17} 1Sam 18,3}

\hypertarget{acordar-el-procedimiento-a-seguir-para-la-comunicaciuxf3n-de-la-informaciuxf3n}{%
\subsection{Acordar el procedimiento a seguir para la comunicación de la
información}\label{acordar-el-procedimiento-a-seguir-para-la-comunicaciuxf3n-de-la-informaciuxf3n}}

\bibleverse{18} Entonces Jonatán le dijo a David: ``La fiesta de la Luna
Nueva es mañana. Se te echará de menos, porque tu lugar estará vacío.
\bibleverse{19} Dentro de tres días, ve rápidamente al lugar donde te
escondiste cuando todo esto empezó, y quédate allí junto al montón de
piedras. \bibleverse{20} Yo lanzaré tres flechas a su lado, como si
estuviera disparando a un blanco. \bibleverse{21} Luego enviaré a un
muchacho y le diré: `¡Ve a buscar las flechas!' Si le digo
concretamente: `Mira, las flechas están a este lado; tráelas aquí',
entonces te juro por la vida del Señor que puedes salir sin peligro.
\bibleverse{22} Pero si le digo al muchacho: `Mira, las flechas están
más allá de ti', entonces tendrás que salir, porque el Señor quiere que
te vayas. \bibleverse{23} En cuanto a lo que tú y yo hablamos, recuerda
que el Señor es testigo entre tú y yo para siempre''.

\hypertarget{curso-de-las-dos-comidas-del-medioduxeda-en-casa-de-sauxfal-en-la-luna-nueva-y-al-duxeda-siguiente}{%
\subsection{Curso de las dos comidas del mediodía en casa de Saúl en la
luna nueva y al día
siguiente}\label{curso-de-las-dos-comidas-del-medioduxeda-en-casa-de-sauxfal-en-la-luna-nueva-y-al-duxeda-siguiente}}

\bibleverse{24} Así que David se escondió en el campo. Cuando llegó la
fiesta de la Luna Nueva, el rey se sentó a comer. \bibleverse{25} Se
sentó en su lugar habitual, junto al muro, frente a Jonatán. Abner se
sentó junto a Saúl, pero el lugar de David estaba vacío. \bibleverse{26}
Saúl no dijo nada ese día porque pensó: ``Seguramente le ha pasado algo
a David que lo hace ceremonialmente impuro; sí, seguro está impuro''.

\bibleverse{27} Pero el segundo día, el día después de la Luna Nueva, el
lugar de David seguía vacío. Saúl le preguntó a su hijo Jonatán: ``¿Por
qué el hijo de Isaí no ha venido a cenar ni ayer ni hoy?'' .

\bibleverse{28} Jonatán respondió: ``David tuvo que pedirme urgentemente
permiso para ir a Belén. \bibleverse{29} Me dijo: `Por favor, déjame ir,
porque nuestra familia va a celebrar un sacrificio en la ciudad y mi
hermano me dijo que tenía que estar allí. Si piensas bien de mí, por
favor, déjame ir a ver a mis hermanos'. Por eso se ausentó de la mesa
del rey''.

\bibleverse{30} Saúl se enojó mucho con Jonatán y le dijo: ``¡Rebelde
hijo de puta! ¿Crees que no sé que prefieres al hijo de Isaí? ¡Qué
vergüenza! ¡Eres una vergüenza para la madre que te dio a luz!
\bibleverse{31} Mientras el hijo de Isaí siga vivo, tú y tu reinado no
estarán seguros. Ahora ve y tráemelo, porque tiene que morir''.

\bibleverse{32} ``¿Por qué tiene que morir?'' preguntó Jonatán. ``¿Qué
ha hecho?''

\bibleverse{33} Entonces Saúl lanzó su lanza contra Jonatán, tratando de
matarlo, por lo que supo que su padre definitivamente quería a David
muerto. \footnote{\textbf{20:33} 1Sam 18,11} \bibleverse{34} Jonatán
abandonó la mesa, y estaba absolutamente furioso. No quiso comer nada en
el segundo día de la fiesta, pues estaba muy molesto por la forma
vergonzosa en que su padre había tratado a David.

\hypertarget{jonatuxe1n-informa-a-david-de-la-situaciuxf3n-desfavorable-y-se-despide-de-uxe9l}{%
\subsection{Jonatán informa a David de la situación desfavorable y se
despide de
él}\label{jonatuxe1n-informa-a-david-de-la-situaciuxf3n-desfavorable-y-se-despide-de-uxe9l}}

\bibleverse{35} Por la mañana, Jonatán fue al campo, al lugar que había
acordado con David, y un muchacho iba con él. \bibleverse{36} Entonces
le dijo al muchacho: ``Corre y encuentra las flechas que yo tire''. De
modo que el muchacho comenzó a correr y Jonatán le disparó una flecha.
\bibleverse{37} Cuando el muchacho llegó al lugar donde había caído la
flecha de Jonatán, éste le gritó: ``¿No ves que la flecha está más
adelante? \bibleverse{38} ¡Apúrate! ¡Hazlo rápido! ¡No esperes!'' El
muchacho recogió las flechas y se las llevó a su amo. \bibleverse{39} El
muchacho no sospechaba nada; sólo Jonatán y David sabían lo que
significaba. \bibleverse{40} Jonatán le dio el arco y las flechas al
muchacho y le dijo: ``Llévatelas a la ciudad''.

\bibleverse{41} Después de que el muchacho se había ido, David se
levantó de donde estaba, junto al montón de piedras, se tiró al suelo
boca abajo y se inclinó tres veces. Entonces él y Jonatán se besaron y
lloraron juntos como amigos, aunque David fue el que más lloró.
\bibleverse{42} Jonatán le dijo a David: ``Vete en paz, porque los dos
hemos hecho un juramento solemne en nombre del Señor. Dijimos: `El Señor
será testigo entre tú y yo, y entre mis descendientes y los tuyos para
siempre'\,''. Entonces David se marchó, y Jonatán volvió a la ciudad.

\hypertarget{david-como-refugiado-en-nob-y-gat-el-asesinato-del-sacerdote-por-parte-de-sauxfal}{%
\subsection{David como refugiado en Nob y Gat; El asesinato del
sacerdote por parte de
Saúl}\label{david-como-refugiado-en-nob-y-gat-el-asesinato-del-sacerdote-por-parte-de-sauxfal}}

\hypertarget{section-20}{%
\section{21}\label{section-20}}

\bibleverse{1} David fue a la ciudad de Nob para ver al sacerdote
Ahimelec. Cuando se encontró con David, Ahimelec temblaba de miedo, y le
preguntó: ``¿Por qué estás aquí solo? ¿Por qué no hay nadie contigo?'' .
\bibleverse{2} ``El rey me ha dado un encargo'', respondió David. ``Me
dijo: `Nadie debe saber nada de la misión que te he enviado a cumplir'.
En cuanto a mis hombres, les he dicho dónde encontrarme. \bibleverse{3}
¿Qué tienes a la mano para comer? Dame cinco panes, o lo que puedas
encontrar''.

\bibleverse{4} ``No hay pan ordinario'', le dijo el sacerdote a David,
``pero hay pan sagrado, siempre que tus hombres no se hayan acostado con
ninguna mujer últimamente''.

\bibleverse{5} ``No nos hemos acostado con ninguna mujer'', respondió
David. ``De hecho, esa es la norma cuando dirijo las tropas en misión.
Se mantienen puros incluso durante las misiones ordinarias, y con mayor
razón en este momento''. \footnote{\textbf{21:5} Lev 24,5-9; Lev 22,3-7;
  Éxod 19,15} \bibleverse{6} Entonces el sacerdote le dio el pan
sagrado, ya que allí no tenían otro pan que el ``Pan de la Presencia'',
que había sido retirado de la presencia del Señor\footnote{\textbf{21:6}
  En otras palabras, colocado en la Tienda de la Reunión.} ese día y lo
sustituyeron por pan fresco.

\bibleverse{7} Uno de los siervos de Saúl estaba allí ese día, tratando
de enmendarse\footnote{\textbf{21:7} ``Enmendarse'': literalmente
  ``detenerse''. Parece que Doeg estaba ofreciendo un sacrificio por
  algún pecado que había cometido y que el sacerdote Ahimelec conocía.
  Esta parece ser una de las razones por las que Doeg delató a David
  (22:9) y ejecutó la orden de Saúl de matar a Ahimelec y a los demás
  sacerdotes.} con el Señor. Era Doeg el edomita, el pastor principal de
Saúl.

\bibleverse{8} ``¿Tienes aquí una lanza o una espada?'' le preguntó
David a Ahimelec. ``No traje mi espada ni ninguna de mis armas, porque
lo que el rey necesitaba que hiciera era urgente''. \footnote{\textbf{21:8}
  1Sam 22,9; 1Sam 22,18}

\bibleverse{9} Entonces el sacerdote respondió: ``Tengo aquí la espada
de Goliat, el filisteo que mataste en el Valle de Ela. Está envuelta en
un paño detrás del efod. Puedes cogerla si quieres. Es el único que hay
aquí''. ``¡Es mejor que cualquier otra espada! Por favor, dámela'',
respondió David.

\hypertarget{david-se-vuelve-loco-con-el-rey-achis-en-gat}{%
\subsection{David se vuelve loco con el rey Achis en
Gat}\label{david-se-vuelve-loco-con-el-rey-achis-en-gat}}

\bibleverse{10} Ese día David huyó de Saúl y se dirigió a Aquis, rey de
Gat.\footnote{\textbf{21:10} Gat era una ciudad filistea.}
\bibleverse{11} Pero los oficiales de Aquis preguntaron al rey: ``¿No es
éste David, el rey de ese país? ¿No cantaban sobre él en sus danzas:
`Saúl ha matado a sus miles, y David a sus decenas de miles'\,''?
\footnote{\textbf{21:11} Sal 56,1}

\bibleverse{12} David escuchó atentamente lo que decían y esto le hizo
temer mucho a Aquis, el rey de Gat. \footnote{\textbf{21:12} 1Sam 18,7;
  1Sam 29,5} \bibleverse{13} Así que cambió su forma de actuar con ellos
y se hizo el loco. Hizo marcas en las puertas de la ciudad y dejó que su
saliva corriera por su barba. \bibleverse{14} Aquis les dijo a sus
oficiales: ``¡Como ven, ese hombre está completamente loco! ¿Por qué me
lo han traído? \footnote{\textbf{21:14} Sal 34,1}

\bibleverse{15} ¿Acaso necesito más locos para que me traigan a este
hombre y que se vuelva loco delante de mí? ¿Creen que voy a dejar que
entre en mi casa?''

\hypertarget{la-posterior-huida-de-david-a-adullam-mizpe-en-moab-y-jaar-hereth-en-juduxe1-su-cuidado-por-sus-padres}{%
\subsection{La posterior huida de David a Adullam, Mizpe en Moab y
Jaar-Hereth en Judá; su cuidado por sus
padres}\label{la-posterior-huida-de-david-a-adullam-mizpe-en-moab-y-jaar-hereth-en-juduxe1-su-cuidado-por-sus-padres}}

\hypertarget{section-21}{%
\section{22}\label{section-21}}

\bibleverse{1} Después David escapó y se fue a la cueva de Adulam.
Cuando se enteraron de dónde estaba, sus hermanos y todo el resto de su
familia fueron y se reunieron con él allí. \bibleverse{2} Todos los que
tenían problemas o deudas o estaban resentidos también acudieron a él y
se convirtió en su líder. Ahora tenía unos cuatrocientos hombres con él.
\footnote{\textbf{22:2} Jue 11,3} \bibleverse{3} Luego David se fue a
Mizpa, en el país de Moab. Le pidió al rey de Moab: ``Por favor, deja
que mi padre y mi madre vengan y se queden contigo hasta que averigüe lo
que Dios planea para mí''. \bibleverse{4} Así que los dejó con el rey de
Moab, y se quedaron con el rey todo el tiempo que David vivió en la
fortaleza.\footnote{\textbf{22:4} ``Fortaleza'': probablemente
  refiriéndose a su campamento en la cueva de Adulam.} \bibleverse{5}
Pero entonces el profeta Gad le dijo a David: ``No te quedes en la
fortaleza. Vuelve a la tierra de Judá''. Así que David se marchó y se
dirigió al bosque de Haret.

\hypertarget{la-queja-de-sauxfal-a-los-que-lo-rodeaban-en-guibeuxe1-traiciuxf3n-del-edomita-doeg-la-sangrienta-venganza-de-sauxfal-contra-los-sacerdotes-de-nob}{%
\subsection{La queja de Saúl a los que lo rodeaban en Guibeá; Traición
del edomita Doeg; La sangrienta venganza de Saúl contra los sacerdotes
de
Nob}\label{la-queja-de-sauxfal-a-los-que-lo-rodeaban-en-guibeuxe1-traiciuxf3n-del-edomita-doeg-la-sangrienta-venganza-de-sauxfal-contra-los-sacerdotes-de-nob}}

\bibleverse{6} Saúl se enteró de que David había regresado y de dónde
estaba. Saúl estaba sentado bajo el tamarisco en la colina de Guibeá.
Tenía su lanza en la mano, con todos sus oficiales rodeándolo.
\bibleverse{7} Entonces Saúl les dijo: ``¡Escúchenme, hombres de
Benjamín! ¿Acaso el hijo de Isaí les va a dar a todos ustedes campos y
viñedos y los va a hacer comandantes y oficiales del ejército?
\bibleverse{8} ¿Es por eso que todos ustedes han conspirado contra mí?
Ni uno solo de ustedes me dijo que mi propio hijo había hecho un acuerdo
con el hijo de Isaí. Ni uno solo de ustedes ha demostrado que se
preocupa por mí, ni me ha explicado que mi hijo lo ha animado para que
intente matarme. ¡Eso es lo que está haciendo ahora!'' \footnote{\textbf{22:8}
  1Sam 18,3}

\bibleverse{9} Doeg el edomita, que estaba allí con los oficiales de
Saúl, habló diciendo: ``Vi al hijo de Isaí visitar a Ahimelec, hijo de
Ahitob, en Nob. \footnote{\textbf{22:9} 1Sam 22,22; Sal 52,2}
\bibleverse{10} Ahimelec pidió consejo al Señor para él y le dio comida.
También le dio la espada de Goliat el filisteo''. \footnote{\textbf{22:10}
  1Sam 21,7-10}

\hypertarget{el-plato-de-sangre-en-guibeuxe1}{%
\subsection{El plato de sangre en
Guibeá}\label{el-plato-de-sangre-en-guibeuxe1}}

\bibleverse{11} El rey envió un mensaje para convocar al sacerdote
Ahimelec, hijo de Ahitob, y a toda su familia, que eran sacerdotes en
Nob. Todos ellos acudieron al rey. \bibleverse{12} ``Ahora escucha, hijo
de Ahitob'', le gritó el rey. ``¿Qué pasa, mi señor?'' preguntó
Ahimelec.

\bibleverse{13} ``¿Por qué tú y el hijo de Isaí han conspirado contra
mí? Le diste pan y una espada, y le pediste consejo a Dios para que se
rebelara contra mí y tratara de matarme, ¡que es lo que está haciendo
ahora!''

\bibleverse{14} ``¿Quién de todos tus oficiales es tan confiable como
David, el yerno del rey? ¡Él está a cargo de su escolta, y es muy
respetado en su familia!'' respondió Ahimelec. \bibleverse{15} ``¿Y fue
ese día la primera vez que pidió consejo a Dios en su favor? ¡Por
supuesto que no! El rey no debe acusarme a mí, tu siervo, ni a nadie de
mi familia, pues yo no sabía nada de todo esto''.

\bibleverse{16} ``¡Vas a morir por esto!'', declaró el rey. ``¡Tú y toda
tu familia!'' \bibleverse{17} Entonces el rey se dirigió a sus
guardaespaldas que estaban allí y les ordenó: ``¡Maten a estos
sacerdotes del Señor, porque están del lado de David! Sabían que era un
fugitivo y, sin embargo, no me lo dijeron''. Pero los guardias del rey
se negaron a atacar a los sacerdotes del Señor.

\bibleverse{18} Entonces el rey le ordenó a Doeg: ``¡Mata tú a los
sacerdotes!'' Doeg el edomita atacó y mató a los sacerdotes, matando a
ochenta y cinco hombres que llevaban puesta su ropa sacerdotal.

\bibleverse{19} Luego se dirigió a Nob, la ciudad de los sacerdotes, y
mató a sus hombres y mujeres, niños y bebés, ganado, asnos y ovejas.
\footnote{\textbf{22:19} 1Sam 21,2}

\hypertarget{el-sacerdote-fugitivo-abjathar-encuentra-una-recepciuxf3n-amistosa-con-david}{%
\subsection{El sacerdote fugitivo Abjathar encuentra una recepción
amistosa con
David}\label{el-sacerdote-fugitivo-abjathar-encuentra-una-recepciuxf3n-amistosa-con-david}}

\bibleverse{20} Pero uno de los hijos de Ahimelec, hijo de Ahitob, logró
escapar. Se llamaba Abiatar, y huyó y se unió a David. \bibleverse{21}
Le dijo a David que Saúl había matado a los sacerdotes del Señor.

\bibleverse{22} Entonces David le dijo a Abiatar: ``Yo sabía que ese
día, cuando Doeg el edomita estaba allí, iba a contárselo a Saúl. Es mi
culpa que toda tu familia haya muerto. \bibleverse{23} Pero puedes
quedarte conmigo y no debes tener miedo, porque el hombre que quiere
matarte también quiere matarme a mí. Yo cuidaré bien de ti''.

\hypertarget{david-en-el-desierto-de-juduxe1-en-kegila-y-maon-su-uxfaltimo-encuentro-con-jonathan-traiciuxf3n-de-los-sifitas}{%
\subsection{David en el desierto de Judá (en Kegila y Maon); su último
encuentro con Jonathan; Traición de los
sifitas}\label{david-en-el-desierto-de-juduxe1-en-kegila-y-maon-su-uxfaltimo-encuentro-con-jonathan-traiciuxf3n-de-los-sifitas}}

\hypertarget{section-22}{%
\section{23}\label{section-22}}

\bibleverse{1} Un día David escuchó la noticia: ``Los filisteos están
atacando Keila y están robando el grano de las eras''. \footnote{\textbf{23:1}
  Jos 15,44}

\bibleverse{2} Entonces David le pidió consejo al Señor: ``¿Debo ir a
atacar a esos filisteos?'' . Y el Señor le dijo a David: ``Ve y ataca a
los filisteos y salva a Keila''.

\bibleverse{3} Pero los hombres de David le dijeron: ``Incluso aquí en
Judá sentimos miedo. Si fuéramos a Keila a luchar contra los ejércitos
filisteos, ¡estaríamos absolutamente aterrorizados!''

\bibleverse{4} Entonces David volvió a pedir consejo al Señor, y éste le
dijo: ``Ve inmediatamente a Keila, porque te daré la victoria sobre los
filisteos''.

\bibleverse{5} Entonces David y sus hombres fueron a Keila y lucharon
contra los filisteos. Los mataron y expulsaron su ganado. De esta manera
David salvó al pueblo de Keila.

\bibleverse{6} (Cuando Abiatar, hijo de Ahimelec, huyó hacia David en
Keila, llevó consigo el efod). \footnote{\textbf{23:6} 1Sam 22,20}

\bibleverse{7} Cuando Saúl se enteró de que David había ido a Keila,
dijo: ``Dios me lo ha entregado, porque se ha encerrado en una ciudad
con puertas que se pueden cerrar con barrotes''. \bibleverse{8} Entonces
Saúl convocó a todo su ejército para ir a atacar a Keila y sitiar a
David y a sus hombres. \bibleverse{9} Cuando David se enteró de que Saúl
estaba tramando atacarlo, le pidió al sacerdote Abiatar: ``Por favor,
trae el efod''. \bibleverse{10} David oró: ``Señor, Dios de Israel, a
mí, tu siervo, me han dicho claramente que Saúl planea venir a Keila y
destruir la ciudad por mi culpa. \bibleverse{11} ¿Van a entregarme los
jefes de la ciudad de Keila? ¿Va a venir Saúl, como he oído? Señor, Dios
de Israel, por favor, dímelo''. ``Sí, vendrá'', respondió el Señor.

\bibleverse{12} ``¿Y los jefes de la ciudad de Keila me entregarán a mí
y a mis hombres a Saúl?'' preguntó David. ``Sí, lo harán'', respondió el
Señor.

\bibleverse{13} Así que David y sus hombres, que eran unos seiscientos,
salieron de Keila y se desplazaron de un lugar a otro. Cuando Saúl
descubrió que David había escapado de Keila, no se molestó en ir allí.

\hypertarget{david-perseguido-por-sauxfal-en-el-desierto-de-siph-su-entrevista-con-jonathan-en-horesa}{%
\subsection{David perseguido por Saúl en el desierto de Siph; su
entrevista con Jonathan en
Horesa}\label{david-perseguido-por-sauxfal-en-el-desierto-de-siph-su-entrevista-con-jonathan-en-horesa}}

\bibleverse{14} David acampó en las fortalezas del desierto, quedándose
en las montañas del desierto de Zif. Saúl lo buscó continuamente, pero
Dios no permitió que David fuera capturado. \footnote{\textbf{23:14}
  1Sam 23,19; 1Sam 24,1} \bibleverse{15} Mientras David se alojaba en
Horesh, en el desierto de Zif, descubrió\footnote{\textbf{23:15}
  ``Descubrió'': o ``temió''.} que Saúl iba a matarlo.

\bibleverse{16} El hijo de Saúl, Jonatán, fue a ver a David a Horesh y
lo animó a seguir confiando en Dios, diciéndole: \bibleverse{17} ``No te
preocupes, porque mi padre Saúl nunca te va a encontrar. Vas a ser rey
de Israel y yo seré tu sustituto. Hasta mi padre Saúl lo sabe''.
\bibleverse{18} Los dos hicieron un acuerdo ante el Señor. David se
quedó en Horesh y Jonatán se fue a su casa.

\hypertarget{david-traicionado-por-los-sifitas-y-maravillosamente-salvado-de-sauxfal-en-el-desierto-de-mauxf3n}{%
\subsection{David traicionado por los sifitas y maravillosamente salvado
de Saúl en el desierto de
Maón}\label{david-traicionado-por-los-sifitas-y-maravillosamente-salvado-de-sauxfal-en-el-desierto-de-mauxf3n}}

\bibleverse{19} Entonces los hombres de Zif fueron a ver a Saúl a Guibeá
y le dijeron: ``David se esconde en nuestra zona, en las fortalezas de
Hores, en la colina de Haquila, en los páramos del sur. \footnote{\textbf{23:19}
  1Sam 26,1; Sal 54,2} \bibleverse{20} Así que, Su Majestad, venga
cuando quiera, y nos aseguraremos de entregárselo''.

\bibleverse{21} Y Saúl le respondió: ``Que el Señor te bendiga por
pensar en mí. \bibleverse{22} Por favor, ve y asegúrate de saber
exactamente dónde está -- dónde se hospeda y quién lo ha visitado --
porque la gente me dice que es muy taimado. \bibleverse{23} Busca y
anota todos sus escondites. Luego vuelve a mí cuando estés seguro, y yo
volveré contigo. Si está aquí en el campo, lo cazaré entre todo el
pueblo de Judá''.

\bibleverse{24} Así que los hombres de Zif se pusieron en marcha,
regresando a Zif por delante de Saúl. David y sus hombres estaban en el
desierto de Maón, en el valle de Araba\footnote{\textbf{23:24} ``El
  valle de Araba'': otro nombre para el Valle del Jordán.} en los
páramos del sur. \bibleverse{25} Saúl y sus hombres comenzaron a
buscarlo. Cuando David se enteró, bajó a la roca y se quedó en el
desierto de Maón. Y cuando Saúl se enteró, persiguió a David en el
desierto de Maón. \bibleverse{26} Saúl iba por un lado de la montaña,
mientras que David y sus hombres iban por el otro lado, apurando la
marcha. Pero justo cuando Saúl y sus hombres se acercaban a David y a
los suyos, a punto de capturarlos, \bibleverse{27} llegó un mensajero
para decirle a Saúl: ``¡Ven de inmediato! Los filisteos han invadido el
país''. \bibleverse{28} Así que Saúl tuvo que dejar de perseguir a David
y fue a enfrentarse a los filisteos. Por eso el lugar se llama ``Roca de
la Fuga''. \bibleverse{29} Entonces David partió y se fue a vivir a las
fortalezas de En-gadi.

\hypertarget{la-generosidad-de-david-hacia-sauxfal-en-la-cueva-cerca-de-engedi}{%
\subsection{La generosidad de David hacia Saúl en la cueva cerca de
Engedi}\label{la-generosidad-de-david-hacia-sauxfal-en-la-cueva-cerca-de-engedi}}

\hypertarget{section-23}{%
\section{24}\label{section-23}}

\bibleverse{1} Cuando Saúl volvió de perseguir a los filisteos, le
informaron: ``David está en el desierto de En-gadi''. \bibleverse{2} Así
que Saúl tomó tres mil hombres especialmente escogidos de todo Israel y
fue a buscar a David y a sus hombres en los alrededores de las Rocas de
las Cabras Salvajes. \bibleverse{3} Cuando Saúl pasó por los corrales de
las ovejas en el camino, había una cueva, y entró a hacer sus
necesidades. David y sus hombres estaban escondidos en lo profundo de la
cueva. \bibleverse{4} Los hombres de David le dijeron: ``Hoy es el día
que el Señor te prometió al decirte: `Escucha, voy a entregarte a tu
enemigo, para que hagas con él lo que quieras'\,''. Entonces David se
acercó sigilosamente y cortó un trozo del borde del manto de Saúl.
\bibleverse{5} Pero después David se sintió muy mal porque había cortado
un trozo del manto de Saúl. \bibleverse{6} Y les dijo a sus hombres:
``Que el Señor me impida hacer algo así\footnote{\textbf{24:6} ``Algo
  así'': probablemente refiriéndose al deseo de sus hombres de atacar al
  rey.} a mi amo, el ungido del Señor. Nunca lo atacaré, porque es el
ungido del Señor''. \bibleverse{7} Y reprendió a sus hombres, y no les
permitió atacar a Saúl. Saúl se levantó y siguió su camino. \footnote{\textbf{24:7}
  2Sam 1,14; Sal 105,15} \bibleverse{8} Un poco más tarde, David salió
de la cueva y gritó: ``¡Mi amo el rey!''. Cuando Saúl miró a su
alrededor, David se inclinó con el rostro hacia el suelo.

\hypertarget{los-discursos-intercambiados-entre-sauxfal-y-david-su-despedida}{%
\subsection{Los discursos intercambiados entre Saúl y David; su
despedida}\label{los-discursos-intercambiados-entre-sauxfal-y-david-su-despedida}}

\bibleverse{9} ``¿Por qué haces caso a la gente que dice que yo quiero
hacerte daño''? preguntó David. \bibleverse{10} ``¡Sólo mira! Hoy has
visto con tus propios ojos que el Señor te entregó a mí en la cueva.
Algunos me instaron a matarte, pero yo te mostré compasivo y dije: `Me
niego a atacar a mi amo, porque es el ungido del Señor'. \bibleverse{11}
¡Mira, padre mío! ¿Ves este pedazo de tu túnica que estoy sosteniendo?
Sí, te lo he cortado, pero no te he matado. Ahora puedes verlo por ti
mismo y puedes estar seguro de que no he hecho nada malo ni rebelde. No
he pecado contra ti, pero tú me persigues, tratando de matarme.
\bibleverse{12} ``Que el Señor decida entre tú y yo quién de los dos
tiene razón, y que el Señor te castigue, pero yo nunca intentaré hacerte
daño. \bibleverse{13} Como dice el viejo refrán: `Del malvado salen
actos malvados', pero yo nunca trataré de hacerte daño. \bibleverse{14}
¿A quién persigue el rey de Israel? ¿A quién persigue? ¡A un perro
muerto! ¡Sólo una pulga! \bibleverse{15} Que el Señor decida y elija
entre tú y yo. Que preste atención a mi caso y lo apoye; que me salve de
ti''.

\bibleverse{16} Cuando David terminó de decir esto, Saúl preguntó:
``¿Eres tú el que habla, David, hijo mío?'' , y lloró en voz alta.
\bibleverse{17} Entonces le dijo a David: ``Tú eres mejor persona que
yo, porque me has pagado con el bien, pero yo te he pagado con el mal.
\bibleverse{18} Hoy has demostrado lo bien que me has tratado, pues
cuando el Señor me entregó a ti, no me mataste. \bibleverse{19} Porque
si un hombre agarrara a su enemigo, ¿lo dejaría escapar ileso? ¡Que el
Señor te recompense bien por cómo me has tratado hoy! \bibleverse{20}
Escucha, sé que definitivamente serás rey, y tu gobierno sobre el reino
de Israel será seguro. \bibleverse{21} Ahora júrame por el Señor que no
destruirás a mis descendientes que me siguen y que no borrarás mi nombre
de mi linaje''. \footnote{\textbf{24:21} 1Sam 23,17}

\bibleverse{22} Así que David le prometió esto a Saúl con un juramento.
Entonces Saúl regresó a su casa, pero David y sus hombres volvieron a la
fortaleza.

\hypertarget{la-muerte-de-samuel-la-locura-de-nabal-david-y-abigail}{%
\subsection{La muerte de Samuel; La locura de Nabal; David y
Abigail}\label{la-muerte-de-samuel-la-locura-de-nabal-david-y-abigail}}

\hypertarget{section-24}{%
\section{25}\label{section-24}}

\bibleverse{1} Samuel murió. Todos en Israel se reunieron para llorar
por él, y lo enterraron en su casa de Ramá. David partió y se fue al
desierto de Parán.

\hypertarget{el-comportamiento-necio-de-nabal-hacia-la-peticiuxf3n-de-david}{%
\subsection{El comportamiento necio de Nabal hacia la petición de
David}\label{el-comportamiento-necio-de-nabal-hacia-la-peticiuxf3n-de-david}}

\bibleverse{2} Había un hombre de Maón que era muy rico. Tenía
propiedades en el Carmelo y poseía mil cabras y tres mil ovejas. Estaba
en el Carmelo esquilando las ovejas. \bibleverse{3} El hombre se llamaba
Nabal,\footnote{\textbf{25:3} ``Nabal'' significa ``tonto''.} y su
esposa se llamaba Abigail. Era una mujer sabia y hermosa, pero su marido
era cruel y trataba mal a la gente. Era descendiente de Caleb.
\bibleverse{4} David estaba en el desierto y se enteró de que Nabal
estaba esquilando ovejas. \bibleverse{5} Entonces David envió a diez de
sus jóvenes y les dijo: ``Vayan a ver a Nabal al Carmelo. Salúdenlo en
mi nombre y salúdenlo de mi parte. \bibleverse{6} Díganle: `¡Te deseo
una larga vida! Paz a ti y a tu familia, y que todo lo que hagas
prospere. \bibleverse{7} Me he enterado de que estás ocupado esquilando.
Cuando tus pastores estuvieron con nosotros, no los maltratamos, y nada
de lo que les pertenecía fue robado en todo el tiempo que estuvieron en
el Carmelo. \bibleverse{8} Pregúntales a tus hombres y ellos te lo
confirmarán. Por favor, sean amables con mis hombres, sobre todo porque
hemos venido en este día de fiesta. Por favor, danos la comida que
puedas a nosotros y a tu buen amigo David'\,''.

\bibleverse{9} Los jóvenes de David llegaron, le dieron a Nabal este
mensaje de parte de David y esperaron su respuesta.

\bibleverse{10} ``¿Quién se cree ese `David, hijo de Isaí'\,''?
respondió Nabal. ``¡Hoy en día hay muchos siervos que huyen de sus amos!
\bibleverse{11} ¿Por qué habría de tomar el pan y el agua que he
suministrado, y la carne que he sacrificado para mis esquiladores, y
entregárselos a estos extraños? ¡Ni siquiera sé de dónde son!''.

\bibleverse{12} Así que los hombres de David se dieron la vuelta y
regresaron por donde habían venido. Cuando regresaron, le contaron a
David todo lo que Nabal había dicho.

\hypertarget{david-se-lanza-a-la-venganza-abigail-se-entera-de-la-erupciuxf3n-de-su-marido}{%
\subsection{David se lanza a la venganza; Abigail se entera de la
erupción de su
marido}\label{david-se-lanza-a-la-venganza-abigail-se-entera-de-la-erupciuxf3n-de-su-marido}}

\bibleverse{13} ``¡Todos, tomen las espadas!'' ordenó David. Y todos se
pusieron las espadas, y David también lo hizo. Unos cuatrocientos
hombres siguieron a David, mientras que doscientos se quedaron atrás
para custodiar sus pertrechos.

\bibleverse{14} Mientras tanto, uno de los hombres de Nabal le dijo a
Abigail, la esposa de Nabal: ``David envió a unos mensajeros del
desierto para que le trajeran saludos a nuestro amo, pero él sólo los
insultó. \bibleverse{15} Los hombres de David siempre fueron muy buenos
con nosotros y nunca nos maltrataron. Todo el tiempo que estuvimos en el
campo con ellos no nos robaron nada. \bibleverse{16} Fueron como un muro
protector para nosotros, tanto de día como de noche, durante todo el
tiempo que estuvimos con ellos cuidando las ovejas. \bibleverse{17}
Debes saber lo que ha pasado y pensar en lo que debes hacer al respecto.
El desastre está a punto de golpear a nuestro amo y a toda su familia,
¡pero es tan odioso que nadie puede hacerlo entrar en razón!''

\hypertarget{abigail-usa-muxe9todos-inteligentes-para-evitar-que-david-tome-su-venganza}{%
\subsection{Abigail usa métodos inteligentes para evitar que David tome
su
venganza}\label{abigail-usa-muxe9todos-inteligentes-para-evitar-que-david-tome-su-venganza}}

\bibleverse{18} Abigail recolectó rápidamente doscientos panes, dos
cueros de vino, cinco ovejas ya sacrificadas, cinco seahs de grano
tostado, cien tortas de pasas y doscientas tortas de higos, y luego
cargó todo en los asnos. \bibleverse{19} Entonces les dijo a sus
hombres: ``Vayan ustedes adelante. Yo los seguiré''. Pero no le dijo
nada a su marido Nabal. \bibleverse{20} Mientras Abigail montaba en su
burro por un valle de la montaña, vio que David y sus hombres bajaban
hacia ella, y les salió al encuentro.

\bibleverse{21} David acababa de quejarse: ``¡De nada sirvió proteger
las pertenencias de este hombre en el desierto! No le han robado nada en
absoluto y, sin embargo, ¿qué hace? ¡Me devuelve mal por bien!
\bibleverse{22} ¡Que Dios me castigue muy severamente si dejo vivo a uno
solo de sus hombres para la mañana!'' \footnote{\textbf{25:22} 1Re 14,10}

\bibleverse{23} Cuando Abigail vio a David, se bajó rápidamente del asno
y se inclinó ante él, con el rostro en el suelo. \bibleverse{24} Cayendo
a sus pies en señal de respeto, le dijo: ``Señor, acepto toda la
responsabilidad por lo que ha sucedido. Por favor, escuche lo que yo, su
sierva, tengo que decir. \bibleverse{25} Por favor, no te inquietes por
ese despreciable de Nabal. Su nombre significa `tonto', y él es
realmente tonto. En cuanto a mí, tu siervo, ni siquiera vi a los hombres
que enviaste. \bibleverse{26} ``Ahora, señor, vive el Señor y vives tú,
el Señor te ha impedido derramar sangre y tomar tu propia venganza.
Señor, que tus enemigos y los que quieren hacerte daño sean como Nabal.
\bibleverse{27} Te ruego que aceptes este presente que yo, tu sierva, te
he traído, señor, y se lo des a tus hombres. \bibleverse{28} Por favor,
perdona cualquier ofensa que yo, tu sierva, haya cometido, porque el
Señor está seguro de establecer una dinastía para ti que durará mucho
tiempo, porque tú, señor, peleas las batallas del Señor. La maldad no
debe encontrarse en ti mientras vivas.\footnote{\textbf{25:28} Tal vez
  Abigail está sugiriendo que la misión actual de David no está
  sancionada por Dios y que seguir con ella sería comprometer su
  reputación, especialmente como futuro rey de Israel.} \bibleverse{29}
Si alguien te persigue y trata de matarte, tu vida quedará ligada a los
que el Señor, tu Dios, cuida, a salvo en su cuidado. Pero él tirará las
vidas de tus enemigos como piedras de una honda. \bibleverse{30} Así que
cuando el Señor haya hecho por ti, señor, todo el bien que te prometió,
y te haya hecho gobernar sobre Israel, \bibleverse{31} no tendrás
sentimientos de remordimiento ni conciencia culpable por el
derramamiento innecesario de sangre ni por tomar tu propia venganza. Y
cuando el Señor haya hecho estas cosas buenas por ti, señor, por favor
acuérdate de mí, tu sierva''.

\bibleverse{32} Entonces David le dijo a Abigail: ``¡Alabado sea el
Señor, el Dios de Israel, que te ha enviado hoy a mi encuentro!
\bibleverse{33} Que seas recompensada por tus sabias decisiones, por
haber evitado que hoy derramara sangre y me vengara. \bibleverse{34} Por
el contrario, vive el Señor, el Dios de Israel, que me ha impedido
hacerte daño, si no hubieras salido corriendo a mi encuentro,
definitivamente no habría quedado vivo ni uno solo de los hombres de
Nabal al amanecer''.

\bibleverse{35} David aceptó de Abigail lo que le había traído y le
dijo: ``Puedes irte a casa en paz, porque estoy de acuerdo con tu
consejo y te concedo tu petición''.

\hypertarget{la-muerte-repentina-de-nabal-el-matrimonio-de-david-con-abigail-y-ahinoam}{%
\subsection{La muerte repentina de Nabal; El matrimonio de David con
Abigail (y
Ahinoam)}\label{la-muerte-repentina-de-nabal-el-matrimonio-de-david-con-abigail-y-ahinoam}}

\bibleverse{36} Cuando Abigail volvió a casa de Nabal, éste estaba en la
casa, de fiesta como un rey. Se sentía muy alegre y estaba muy borracho.
Así que ella no le dijo nada hasta la mañana. \bibleverse{37} A la
mañana siguiente, cuando Nabal estaba sobrio, su mujer le contó lo que
había sucedido. Cuando él la escuchó, le dio un ataque al corazón y se
quedó paralizado.\footnote{\textbf{25:37} ``Quedó paralizado'':
  literalmente, ``estaba como una piedra''.} \bibleverse{38} Unos diez
días después, el Señor abatió a Nabal y éste murió. \bibleverse{39}
Cuando David se enteró de que Nabal había muerto, dijo: ``Alabado sea el
Señor, que me ha apoyado contra la injuria de Nabal y me ha impedido
hacer el mal. Porque el Señor hizo que la maldad de Nabal recayera sobre
él''. Entonces David envió un mensaje a Abigail, pidiéndole que se
casara con él.

\bibleverse{40} Cuando los hombres de David llegaron al Carmelo, le
dijeron a Abigail: ``David nos ha enviado a traerte para que seas su
esposa''.

\bibleverse{41} Ella se levantó, se inclinó y dijo: ``Soy la sierva de
David. Estoy dispuesta a servir y a lavar los pies de los siervos de mi
señor''. \bibleverse{42} Abigail subió rápidamente a un burro y, con sus
cinco sirvientas, regresó con los hombres de David y se convirtió en su
esposa. \footnote{\textbf{25:42} 1Sam 27,3; 1Sam 30,5} \bibleverse{43}
David también se había casado con Ahinoam de Jezreel. Así que ambas
fueron sus esposas.

\bibleverse{44} Sin embargo, Saúl había dado a su hija Mical, esposa de
David, a Paltiel, hijo de Laish. Él era de Galim.

\hypertarget{la-renovada-generosidad-de-david-hacia-sauxfal-en-el-desierto-de-siph}{%
\subsection{La renovada generosidad de David hacia Saúl en el desierto
de
Siph}\label{la-renovada-generosidad-de-david-hacia-sauxfal-en-el-desierto-de-siph}}

\hypertarget{section-25}{%
\section{26}\label{section-25}}

\bibleverse{1} El pueblo de Zif fue a ver a Saúl a Guibeá y le dijeron:
``David se esconde en la colina de Haquilá, frente a los páramos''.
\footnote{\textbf{26:1} 1Sam 23,19; Sal 54,2} \bibleverse{2} Así que
Saúl se dirigió al desierto de Zif junto con tres mil hombres de Israel
especialmente escogidos para buscar a David allí. \bibleverse{3} Saúl
acampó junto al camino en la colina de Haquilá, frente a los páramos,
cerca de donde David vivía en el desierto. Cuando se dio cuenta de que
Saúl había ido a buscarlo allí, \bibleverse{4} envió espías y descubrió
que Saúl había llegado definitivamente. \bibleverse{5} Una
noche,\footnote{\textbf{26:5} ``Una noche'': implícito.} David se
levantó y fue al campamento de Saúl y vio dónde dormía éste, junto con
Abner, hijo de Ner, el comandante del ejército. Saúl estaba acostado en
medio del campamento, con sus hombres a su alrededor.

\bibleverse{6} David les preguntó a Ahimelec el hitita y a Abisai, hijo
de Sarvia,\footnote{\textbf{26:6} Servia era hermana de David y madre de
  Joab, Abisai y Asahel.} hermano de Joab: ``¿Quién quiere acompañarme
al campamento a ver a Saúl?'' ``Iré contigo'', respondió Abisai.

\bibleverse{7} Así que David y Abisai fueron al campamento del ejército
por la noche. Saúl estaba durmiendo en el campamento con su lanza
clavada en el suelo junto a su cabeza, y Abner y sus hombres dormían a
su alrededor. \bibleverse{8} Abisai le dijo a David: ``Dios te ha
entregado hoy a tu enemigo. Así que, por favor, déjame clavarle la lanza
en el suelo de una sola vez. No necesitaré hacerlo dos veces''.
\footnote{\textbf{26:8} 2Sam 16,9}

\bibleverse{9} Pero David le dijo a Abisai: ``¡No, no lo mates! ¿Quién
puede atacar al ungido del Señor y no ser culpable de un crimen?
\bibleverse{10} Vive el Señor, el Señor mismo lo matará. O le llegará su
hora y morirá, o irá a la batalla y lo matarán. \bibleverse{11} Que el
Señor me impida atacar al ungido del Señor. Recoge la lanza y el cántaro
de agua junto a su cabeza, y vámonos''.

\bibleverse{12} David tomó la lanza y la jarra de agua junto a la cabeza
de Saúl, y se fueron. Nadie vio nada; nadie supo lo que había pasado;
nadie se despertó. Todos se quedaron dormidos, porque el Señor los había
hecho caer en un profundo sueño. \footnote{\textbf{26:12} Gén 2,21; Gén
  15,12}

\hypertarget{la-aclamaciuxf3n-burlona-de-david-a-abner}{%
\subsection{La aclamación burlona de David a
Abner}\label{la-aclamaciuxf3n-burlona-de-david-a-abner}}

\bibleverse{13} Entonces David volvió al otro lado, y se situó en la
cima de la colina, lo suficientemente lejos -había una distancia
considerable entre ellos. \bibleverse{14} Gritó al ejército y a Abner,
hijo de Ner: ``¿No vas a responderme, Abner?'' . ``¿Quién es el que
grita, molestando al rey?'' respondió Abner.

\bibleverse{15} David llamó a Abner: ``¿No estás destinado a ser ese
gran hombre? ¿Hay alguien en Israel que sea mejor que tú? ¿Por qué no
protegiste a tu amo el rey cuando alguien vino a matarlo?
\bibleverse{16} No has hecho nada bien. Vive el Señor, que todos ustedes
merecen morir, porque no protegieron a su amo, el ungido del Señor.
Miren a su alrededor. ¿Dónde están la lanza y el cántaro del rey que
estaban junto a su cabeza?''

\hypertarget{los-discursos-intercambiados-entre-sauxfal-y-david-la-divergencia-de-ambos}{%
\subsection{Los discursos intercambiados entre Saúl y David; la
divergencia de
ambos}\label{los-discursos-intercambiados-entre-sauxfal-y-david-la-divergencia-de-ambos}}

\bibleverse{17} Saúl reconoció la voz de David y preguntó: ``¿Eres tú
quien habla, David, hijo mío?'' ``Sí, soy yo, mi señor y rey'',
respondió David.

\bibleverse{18} ``¿Por qué me persigue mi señor, su siervo? ¿Qué es lo
que he hecho? ¿De qué crimen soy culpable?'' , continuó. \bibleverse{19}
``Por favor, escúchame, mi señor y rey. Si el Señor se ha enfadado
conmigo, que se alegre de aceptar una ofrenda. Pero si son los hombres
los que lo han hecho, ¡que sean malditos ante el Señor! Durante todo
este tiempo me han expulsado de vivir entre el pueblo elegido por Dios,
diciéndome: `Vete y adora a otros dioses'. \bibleverse{20} Por favor, no
me dejes morir tan lejos de la presencia del Señor. El rey de Israel ha
venido a perseguir una pequeña pulga, cazándome como quien caza una
perdiz en el monte''.

\bibleverse{21} ``He hecho mal'', respondió Saúl, ``vuelve, David, hijo
mío. No volveré a intentar hacerte daño, porque hoy me has valorado y me
has perdonado la vida. ¡He sido tan estúpido! He cometido un gran
error''.

\bibleverse{22} ``Tengo aquí la lanza del rey'', dijo David. ``Envía a
uno de tus hombres a recogerla. \bibleverse{23} El Señor recompensa a
todos los que hacen lo correcto y son fieles. El Señor me ha entregado
hoy a ti, pero me he negado a dañar al ungido del Señor. \bibleverse{24}
De la misma manera que hoy he valorado tu vida, que el Señor valore la
mía y me rescate de todos mis problemas''.

\bibleverse{25} Saúl entonces le dijo a David: ``Que seas bendecido,
David, hijo mío. Lograrás muchas cosas y siempre tendrás éxito''. Y
David se fue, y Saúl volvió a su casa.

\hypertarget{la-conversiuxf3n-de-david-a-los-filisteos-su-estancia-con-el-pruxedncipe-filisteo-achis-en-gat-y-en-siclag}{%
\subsection{La conversión de David a los filisteos; su estancia con el
príncipe filisteo Achis en Gat y en
Siclag}\label{la-conversiuxf3n-de-david-a-los-filisteos-su-estancia-con-el-pruxedncipe-filisteo-achis-en-gat-y-en-siclag}}

\hypertarget{section-26}{%
\section{27}\label{section-26}}

\bibleverse{1} Pero David pensó para sí mismo: ``Un día de estos Saúl va
a atraparme. Creo que será mejor que huya a la tierra de los filisteos.
Así Saúl dejará de buscarme por todo Israel y no me atrapará''.
\bibleverse{2} Así que David y los seiscientos hombres que lo
acompañaban se pusieron en marcha, cruzaron la frontera y se dirigieron
a Aquis, hijo de Maoc, el rey de Gat. \bibleverse{3} David y sus hombres
se instalaron con Aquis en Gat. Todos los hombres tenían a sus familias
con ellos, y David tenía a sus dos esposas, Ahinoam de Jezreel y Abigail
del Carmelo, la viuda de Nabal. \footnote{\textbf{27:3} 1Sam 25,40-43}
\bibleverse{4} Cuando Saúl se enteró de que David había huido a Gat, no
siguió buscándolo.

\bibleverse{5} David le dijo a Aquis: ``Por favor, hazme un favor:
asígname un lugar en una de las ciudades del campo para que pueda vivir
allí. Yo, tu siervo, no merezco vivir en la ciudad real contigo''.
\bibleverse{6} Aquis le dio de inmediato Siclag, y la ciudad sigue
perteneciendo a los reyes de Judá hasta el día de hoy. \bibleverse{7} Y
David vivió en el país de los filisteos durante un año y cuatro meses.

\hypertarget{la-vida-privada-de-david-su-engauxf1o-a-los-filisteos}{%
\subsection{La vida privada de David; su engaño a los
filisteos}\label{la-vida-privada-de-david-su-engauxf1o-a-los-filisteos}}

\bibleverse{8} Durante ese tiempo, David y sus hombres hicieron
incursiones contra los guesuritas, los girzitas y los amalecitas. Estos
pueblos habían vivido en la tierra hasta Sur y Egipto desde tiempos
antiguos. \bibleverse{9} Cuando David atacaba un lugar, no dejaba a
nadie con vida. Tomaba los rebaños y las manadas, los asnos, los
camellos y la ropa. Luego regresaba a Aquis.

\bibleverse{10} Cuando Aquis le preguntaba: ``¿Dónde has estado
asaltando hoy?'' David respondía: ``En el desierto\footnote{\textbf{27:10}
  ``Desierto'', literalmente ``el Negev'', la región árida del sur.} de
Judá'', o ``el desierto de Jerameel'', o ``el desierto de los ceneos''.

\bibleverse{11} David no dejó a nadie con vida que pudiera ir a Gat
porque pensó: ``Podrían delatarnos y decir: `David hizo esto'\,''. Así
hizo todo el tiempo que vivió en el país de los filisteos.

\bibleverse{12} Aquis confió en David y pensaba: ``Se ha hecho tan
ofensivo para su pueblo, los israelitas, que tendrá que servirme para
siempre''.\footnote{\textbf{27:12} Gén 34,30; Éxod 5,21}

\hypertarget{la-guerra-con-los-filisteos-saul-con-el-nigromante-en-endor}{%
\subsection{La guerra con los filisteos; Saul con el nigromante en
Endor}\label{la-guerra-con-los-filisteos-saul-con-el-nigromante-en-endor}}

\hypertarget{section-27}{%
\section{28}\label{section-27}}

\bibleverse{1} Por aquel entonces, los filisteos convocaron a sus
ejércitos para ir a la guerra contra Israel. Entonces Aquis le dijo a
David: ``Esperamos que tú y tus hombres me acompañen como parte del
ejército''.

\bibleverse{2} ``¡Está bien!'' respondió David. ``Entonces tú mismo
descubrirás lo que yo, tu siervo, puedo hacer''. ``Eso también está
bien'', respondió Aquis. ``Te haré mi guardaespaldas de por vida''.

\hypertarget{comienzo-de-la-guerra-en-su-perplejidad-sauxfal-decide-cuestionar-un-oruxe1culo-de-los-muertos}{%
\subsection{Comienzo de la guerra; En su perplejidad, Saúl decide
cuestionar un oráculo de los
muertos}\label{comienzo-de-la-guerra-en-su-perplejidad-sauxfal-decide-cuestionar-un-oruxe1culo-de-los-muertos}}

\bibleverse{3} Para entonces Samuel había muerto, y todo Israel lo había
llorado y enterrado en Ramá, su ciudad natal. Saúl se había deshecho de
los médiums y espiritistas del país.

\bibleverse{4} Los ejércitos filisteos se reunieron y acamparon en
Sunem. Saúl convocó a todo el ejército israelita y acampó en Gilboa.
\bibleverse{5} Cuando Saúl vio al ejército filisteo, se aterrorizó y
tembló de miedo. \bibleverse{6} Pidió consejo al Señor, pero éste no le
respondió ni por sueños, ni por Urim, ni por profetas. \footnote{\textbf{28:6}
  Éxod 28,30; 1Sam 14,37; 1Sam 23,9} \bibleverse{7} Entonces Saúl les
dijo a sus oficiales: ``Búsquenme una mujer que sea médium para que
pueda ir a pedirle consejo''. ``Hay una mujer que es médium en Endor'',
respondieron sus oficiales. \footnote{\textbf{28:7} Hech 16,16}

\hypertarget{saul-con-el-nigromante-en-endor-la-apariciuxf3n-y-profecuxeda-de-la-desgracia-del-espuxedritu-de-samuel}{%
\subsection{Saul con el nigromante en Endor; la aparición y profecía de
la desgracia del espíritu de
Samuel}\label{saul-con-el-nigromante-en-endor-la-apariciuxf3n-y-profecuxeda-de-la-desgracia-del-espuxedritu-de-samuel}}

\bibleverse{8} Saúl se disfrazó vistiendo ropas diferentes. Fue con dos
de sus hombres a ver a la mujer por la noche. Saúl le dijo: ``Tráeme un
espíritu para que pueda hacer algunas preguntas. Te daré el nombre''.

\bibleverse{9} ``¿No sabes lo que ha hecho Saúl?'' , respondió ella.
``Se ha deshecho de los médiums y espiritistas del país. ¿Intenta
tenderme una trampa y hacer que me maten?''

\bibleverse{10} Saúl le hizo un juramento por el Señor: ``Vive el Señor,
no serás considerada culpable por hacer esto''.

\bibleverse{11} ``¿A quién quieres que traiga para ti?'' , preguntó la
mujer. ``Trae a Samuel'', respondió él.

\bibleverse{12} Pero cuando la mujer vio a Samuel, gritó con fuerza y le
dijo a Saúl: ``¿Por qué me has engañado? ¡Tú eres Saúl!''

\bibleverse{13} ``No te asustes'', le dijo el rey. ``¿Qué ves?'' ``Veo
un dios que sale de la tierra'', respondió la mujer.

\bibleverse{14} ``¿Qué aspecto tiene?'' preguntó Saúl. ``Un anciano está
subiendo'', respondió ella. ``Tiene una capa envuelta alrededor de él''.
Saúl pensó que debía ser Samuel y se inclinó hacia abajo en señal de
respeto.

\bibleverse{15} Entonces Samuel le dijo a Saúl: ``¿Por qué me molestas
haciéndome subir?'' . ``Estoy en un problema terrible'', respondió Saúl.
``Los filisteos me atacan, y Dios me ha abandonado. Ya no me responde,
ni con profetas ni con sueños. Por eso te he llamado para que me digas
qué hacer''.

\bibleverse{16} ``¿Por qué vienes a preguntarme si el Señor te ha
abandonado y se ha convertido en tu enemigo?'' preguntó Samuel.
\bibleverse{17} ``El Señor ha hecho contigo exactamente lo que te dijo a
través de mí, pues el Señor te ha arrancado el reino y se lo ha dado a
tu vecino, David. \bibleverse{18} El Señor te ha hecho esto hoy porque
no hiciste lo que el Señor te mandó y no ejecutaste su furia sobre los
amalecitas. \footnote{\textbf{28:18} 1Sam 15,18-19} \bibleverse{19} El
Señor te entregará a ti y a Israel a los filisteos. Mañana tú y tus
hijos morirán y estarán conmigo. El Señor también entregará el ejército
israelita de Israel a los filisteos''. \footnote{\textbf{28:19} 1Sam
  31,6}

\hypertarget{efecto-de-la-profecuxeda-sobre-saulo}{%
\subsection{Efecto de la profecía sobre
Saulo}\label{efecto-de-la-profecuxeda-sobre-saulo}}

\bibleverse{20} Saúl se derrumbó boca abajo en el suelo, aterrorizado
por lo que Samuel había dicho. No tenía fuerzas, porque no había comido
nada en todo ese día y esa noche.

\bibleverse{21} La mujer se acercó a Saúl y vio que estaba absolutamente
aterrado. Ella le dijo: ``Mire, señor, yo hice lo que usted me pidió.
Arriesgué mi vida e hice lo que usted me dijo. \bibleverse{22} Ahora,
por favor, haga lo que le digo. Deje que le traiga un poco de comida.
Cómasela y tendrá fuerzas para seguir su camino''.

\bibleverse{23} Pero él se negó, diciendo: ``No puedo comer nada''. Pero
sus hombres y la mujer le animaron a comer, y él hizo lo que le dijeron.
Se levantó del suelo y se sentó en la cama. \bibleverse{24} La mujer
tenía un ternero cebado en la casa, y rápidamente fue a sacrificarlo.
También cogió harina, la amasó y coció panes sin levadura.
\bibleverse{25} Luego ella sirvió la comida a Saúl y a sus hombres, y
ellos la comieron. Luego se levantaron y se fueron, esa misma noche.

\hypertarget{el-envuxedo-de-david-a-casa-a-instancias-de-los-sospechosos-pruxedncipes-filisteos}{%
\subsection{El envío de David a casa a instancias de los sospechosos
príncipes
filisteos}\label{el-envuxedo-de-david-a-casa-a-instancias-de-los-sospechosos-pruxedncipes-filisteos}}

\hypertarget{section-28}{%
\section{29}\label{section-28}}

\bibleverse{1} Los filisteos reunieron todos sus ejércitos en Afec, y
los israelitas acamparon junto al manantial de Jezreel. \footnote{\textbf{29:1}
  1Sam 4,1} \bibleverse{2} Los jefes filisteos marchaban en sus
divisiones de cientos y miles de personas, con David y sus hombres en la
retaguardia con el rey Aquis.

\bibleverse{3} Pero los jefes filisteos preguntaron: ``¿Qué hacen aquí
estos hebreos?''\footnote{\textbf{29:3} Esto también podría traducirse
  como ``¿Quiénes son estos hebreos?'' , ya que el texto simplemente
  dice ``Qué estos hebreos?''} Entonces Aquis les respondió a los
comandantes filisteos: ``Ese es David, un oficial del rey Saúl de
Israel. Lleva mucho tiempo conmigo, incluso años, y no he encontrado
ninguna falta en él desde el día en que se pasó a nuestro lado hasta
ahora''.

\bibleverse{4} Pero los comandantes filisteos se enojaron con Aquis y le
dijeron: ``Envíalo de vuelta al lugar de donde vino, a la ciudad que le
asignaste. No puede ir con nosotros a la batalla. ¿Y si se vuelve contra
nosotros durante la lucha? ¡Qué buena manera de complacer a su amo,
entregando las cabezas de nuestros hombres! \bibleverse{5} ¿No es éste
el David que cantan en sus danzas? `Saúl ha matado a sus miles, y David
a sus decenas de miles'\,''?

\bibleverse{6} Entonces Aquis llamó a David y le dijo: ``Vive el Señor,
tú eres honesto y has hecho lo correcto por lo que veo. Por lo que a mí
respecta, debes marchar conmigo a la batalla porque no he encontrado
ningún fallo en ti desde el día en que llegaste hasta ahora. Pero los
otros líderes no te aprueban. \bibleverse{7} Así que vuelve a tu casa en
paz, y así no harás nada que moleste a los líderes filisteos''.

\bibleverse{8} ``¿Pero, qué he hecho?'' preguntó David. ``¿Qué falta has
encontrado en mí, tu siervo, desde el día en que vine a ti hasta ahora,
que me impida ir a luchar contra los enemigos de mi señor el rey?''

\bibleverse{9} ``Por lo que a mí respecta, eres tan bueno como un ángel
de Dios'', respondió Aquis. ``Pero los comandantes filisteos han
declarado: `No puede entrar en batalla con nosotros'. \footnote{\textbf{29:9}
  2Sam 19,28} \bibleverse{10} Así que levántate temprano mañana y sal
con tus hombres en cuanto amanezca''.

\bibleverse{11} David y sus hombres se levantaron de madrugada y
volvieron al país de los filisteos. Pero los filisteos avanzaron hacia
Jezreel.

\hypertarget{david-encuentra-siclag-devastada-por-los-amalecitas-su-consternaciuxf3n-y-aliento}{%
\subsection{David encuentra Siclag devastada por los amalecitas; su
consternación y
aliento}\label{david-encuentra-siclag-devastada-por-los-amalecitas-su-consternaciuxf3n-y-aliento}}

\hypertarget{section-29}{%
\section{30}\label{section-29}}

\bibleverse{1} Tres días después, David y sus hombres llegaron de nuevo
a Siclag. Unos amalecitas habían hecho una incursión en el Néguev y en
Siclag. Habían atacado Siclag y la habían incendiado. \bibleverse{2}
Habían capturado a las mujeres y a todos los demás allí, jóvenes y
ancianos. No habían matado a nadie, pero se llevaron a todos con ellos
al marcharse. \bibleverse{3} Cuando David y sus hombres volvieron a la
ciudad, la encontraron quemada hasta los cimientos, y a sus mujeres e
hijos capturados. \bibleverse{4} David y sus hombres lloraron a gritos
hasta no poder más. \bibleverse{5} Las dos esposas de David también
habían sido tomadas como prisioneras: Ahinoam, de Jezreel, y Abigail, la
viuda de Nabal, de Carmel. \bibleverse{6} David estaba en un gran apuro,
porque los hombres estaban tan molestos por la pérdida de sus hijos que
empezaron a hablar de apedrearlo. Pero confiando en el Señor, su Dios,
\bibleverse{7} David fue a ver al sacerdote Abiatar, hijo de Ahimelec, y
le dijo: ``Tráeme el efod''. Y Abiatar se lo trajo. \footnote{\textbf{30:7}
  1Sam 23,9}

\bibleverse{8} Entonces David le preguntó al Señor: ``¿Debo perseguir a
estos asaltantes? ¿Los alcanzaré?'' ``Sí, persíguelos'', contestó el
Señor, ``porque definitivamente los alcanzarás y rescatarás a los
prisioneros''.

\bibleverse{9} David y seiscientos de sus hombres partieron hacia el
valle de Besor. \bibleverse{10} Doscientos de ellos se quedaron allí
porque estaban demasiado cansados para cruzar el valle, mientras que
David siguió adelante con cuatrocientos hombres.

\hypertarget{la-persecuciuxf3n-y-destrucciuxf3n-de-david-de-la-banda-de-ladrones-de-amalecita}{%
\subsection{La persecución y destrucción de David de la banda de
ladrones de
Amalecita}\label{la-persecuciuxf3n-y-destrucciuxf3n-de-david-de-la-banda-de-ladrones-de-amalecita}}

\bibleverse{11} Se encontraron con un egipcio en el campo y se lo
llevaron a David. Le dieron de comer y de beber. \bibleverse{12} También
le dieron un trozo de una torta de higos y dos tortas de pasas. Se los
comió y se recuperó, porque llevaba tres días y tres noches sin comer ni
beber. \bibleverse{13} ``¿De quién eres esclavo y de dónde vienes?'' le
preguntó David. ``Soy egipcio -- respondió --, esclavo de un amalecita.
Mi amo me abandonó hace tres días cuando me enfermé.

\bibleverse{14} Asaltamos a los queretanos en el Neguev, así como la
parte que pertenece a Judá y el Neguev de Caleb. También quemamos
Siclag''. \footnote{\textbf{30:14} 2Sam 8,18; Jos 14,13}

\bibleverse{15} ``¿Puedes guiarme hasta esos asaltantes?'' preguntó
David. ``Si me juras por Dios que no me matarás ni me entregarás a mi
amo, entonces te llevaré hasta ellos'', respondió el hombre.

\bibleverse{16} Entonces llevó a David hasta donde los amalecitas,
quienes estaban esparcidos por todo el lugar, comiendo, bebiendo y
bailando debido al gran botín que habían tomado de las tierras de los
filisteos y de Judá. \bibleverse{17} David los atacó desde el atardecer
hasta la noche siguiente. Nadie escapó, excepto cuatrocientos hombres
que lograron huir, montados en camellos. \bibleverse{18} David recuperó
todo lo que los amalecitas habían tomado, incluidas sus dos esposas.
\bibleverse{19} Todo fue contabilizado: todos los adultos y niños, así
como todo el botín que los amalecitas habían tomado. David recuperó
todo. \bibleverse{20} También recuperó todos los rebaños y manadas. Sus
hombres los llevaron por delante del resto del ganado, gritando: ``¡Este
es el botín de David!''.

\hypertarget{david-hace-que-su-pueblo-lleve-ante-la-justicia-a-sus-camaradas}{%
\subsection{David hace que su pueblo lleve ante la justicia a sus
camaradas}\label{david-hace-que-su-pueblo-lleve-ante-la-justicia-a-sus-camaradas}}

\bibleverse{21} Cuando David recuperó a los doscientos hombres que
habían estado demasiado cansados para seguir con él desde el valle de
Besor, salieron a recibirlo a él y a los hombres que lo acompañaban.
Cuando David se acercó a los hombres para saludarlos, \bibleverse{22}
todos los hombres desagradables y buenos para nada de los que habían ido
con David dijeron: ``Ellos no estaban con nosotros, así que no
compartiremos el botín que tomamos, excepto para devolverles a sus
esposas e hijos. Que los tomen y se vayan''.

\bibleverse{23} Pero David intervino diciendo: ``No, hermanos míos, no
deben hacer esto con lo que el Señor nos ha dado. Él nos ha protegido y
nos ha entregado a los asaltantes que nos habían atacado.
\bibleverse{24} ¿Quién los va a escuchar cuando digan tales cosas? La
parte que reciban los que fueron a la batalla será la misma que la de
los que se quedaron para guardar las provisiones''. \bibleverse{25}
David hizo que esta fuera la regla y norma para Israel desde ese día
hasta ahora.

\hypertarget{david-envuxeda-regalos-a-los-ancianos-en-numerosas-ciudades-de-juduxe1}{%
\subsection{David envía regalos a los ancianos en numerosas ciudades de
Judá}\label{david-envuxeda-regalos-a-los-ancianos-en-numerosas-ciudades-de-juduxe1}}

\bibleverse{26} Cuando David regresó a Siclag, envió parte del botín a
cada uno de sus amigos entre los ancianos de Judá, diciendo: ``Aquí
tienen un regalo para ustedes del botín de los enemigos del Señor''.
\bibleverse{27} David lo envió a los que vivían en Betuel,\footnote{\textbf{30:27}
  ``Betuel'': mucho más probable que ``Betel'' como aparece en el texto
  hebreo.} Ramot Néguev, Jattir, \bibleverse{28} Aroer, Sifmot,
Eshtemoa, \bibleverse{29} Racal, y las ciudades de los jeraelitas y
ceneos, \bibleverse{30} Hormah, Bor-ashan, Athach, \bibleverse{31}
Hebrón: todos los lugares a los que David y sus hombres habían ido.

\hypertarget{la-derrota-de-israel-y-el-desastre-de-sauxfal-y-su-casa}{%
\subsection{La derrota de Israel y el desastre de Saúl y su
casa}\label{la-derrota-de-israel-y-el-desastre-de-sauxfal-y-su-casa}}

\hypertarget{section-30}{%
\section{31}\label{section-30}}

\bibleverse{1} Mientras tanto, los filisteos habían atacado a Israel, y
el ejército israelita huyó de ellos, y muchos murieron en el monte
Gilboa. \bibleverse{2} Los filisteos persiguieron a Saúl y a sus hijos,
y mataron a los hijos de Saúl: Jonatán, Abinadab y Malquisúa.
\bibleverse{3} La lucha se hizo muy intensa en torno a Saúl, y las
flechas de los arqueros filisteos encontraron su objetivo, hiriendo
gravemente a Saúl. \bibleverse{4} Entonces Saúl le dijo a su escudero:
``Toma tu espada y mátame, o estos hombres paganos\footnote{\textbf{31:4}
  ``Paganos'': literalmente, ``incircuncisos''.} vendrán a matarme y a
torturarme''. Pero el escudero no quiso hacerlo porque tenía demasiado
miedo. Entonces Saúl tomó su propia espada y cayó sobre ella.
\footnote{\textbf{31:4} Jue 9,54} \bibleverse{5} Cuando su escudero vio
que Saúl estaba muerto, también cayó sobre su propia espada y murió con
él. \bibleverse{6} Saúl, sus tres hijos, su escudero y todos los hombres
que estaban con él murieron el mismo día.

\bibleverse{7} Cuando los israelitas que vivían a lo largo del valle y
los del otro lado del Jordán se dieron cuenta de que el ejército
israelita había huido y de que Saúl y sus hijos habían muerto,
abandonaron sus ciudades y también huyeron. Entonces llegaron los
filisteos y se apoderaron de ellas.

\hypertarget{el-destino-de-los-caduxe1veres-de-sauxfal-y-sus-hijos}{%
\subsection{El destino de los cadáveres de Saúl y sus
hijos}\label{el-destino-de-los-caduxe1veres-de-sauxfal-y-sus-hijos}}

\bibleverse{8} Al día siguiente, cuando los filisteos fueron a despojar
a los muertos, encontraron a Saúl y a sus tres hijos tendidos en el
monte Gilboa. \bibleverse{9} Le cortaron la cabeza a Saúl, lo despojaron
de su armadura y enviaron mensajeros por todo el país de los filisteos
para que anunciaran la noticia en los templos de sus ídolos y a su
pueblo. \bibleverse{10} Entonces colocaron su armadura en el templo de
Astoret y clavaron su cuerpo en el muro de la ciudad de Bet-San.
\bibleverse{11} Sin embargo, cuando el pueblo de Jabes de Galaad se
enteró de lo que los filisteos le habían hecho a Saúl, \bibleverse{12}
todos sus fuertes guerreros se pusieron en marcha, viajaron toda la
noche y descolgaron los cuerpos de Saúl y de sus hijos de la muralla de
Bet-sán. Cuando volvieron a Jabes, quemaron allí los cuerpos.
\bibleverse{13} Luego tomaron sus huesos y los enterraron bajo el
tamarisco en Jabes, y ayunaron durante siete días.
