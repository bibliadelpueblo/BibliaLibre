\hypertarget{el-destino-de-noemuxed-en-la-tierra-de-los-moabitas}{%
\subsection{El destino de Noemí en la tierra de los
moabitas}\label{el-destino-de-noemuxed-en-la-tierra-de-los-moabitas}}

\hypertarget{section}{%
\section{1}\label{section}}

\bibleverse{1} Hubo una hambruna durante la época en la que los jueces
gobernaban\footnote{\textbf{1:1} Literalmente ``cuando los jueces
  juzgaban'', pero esto era de un modo ejecutivo, más que simplemente
  judicial.} Israel. Un hombre dejó Belén de Judá y se fue a vivir como
exiliado en el país de Moab, junto con su esposa y sus dos hijos.
\bibleverse{2} Se llamaba Elimelec y su mujer Noemí. Sus hijos se
llamaban Mahlón y Quelión. Eran efrateos\footnote{\textbf{1:2} Se cree
  que Efrata es un nombre más antiguo para este Belén en particular, o
  una forma de identificarlo específicamente. Los dos nombres aparecen
  juntos en Miqueas 5:2.} de Belén de Judá. Se fueron al país de Moab y
vivían allí. \bibleverse{3} Sin embargo, Elimelec, el esposo de Noemí,
murió, y ella se quedó con sus dos hijos. \bibleverse{4} Los hijos se
casaron con mujeres moabitas. Una se llamaba Orfa y la otra Rut. Después
de unos diez años, \bibleverse{5} tanto Mahlón como Quelión murieron.
Noemí se quedó sola, sin sus dos hijos ni su marido. \bibleverse{6} Así
que ella y sus nueras se prepararon para abandonar el país de Moab y
volver a casa, porque habían oído que el Señor había bendecido a su
pueblo allí con alimentos.

\hypertarget{partida-de-noemuxed-y-sus-dos-nueras-para-regresar-a-beluxe9n-la-despedida-de-orpa-la-lealtad-de-ruth}{%
\subsection{Partida de Noemí y sus dos nueras para regresar a Belén; La
despedida de Orpa, la lealtad de
Ruth}\label{partida-de-noemuxed-y-sus-dos-nueras-para-regresar-a-beluxe9n-la-despedida-de-orpa-la-lealtad-de-ruth}}

\bibleverse{7} Así que Nohemí se fue del lugar donde vivía y, con sus
dos nueras, emprendió el camino de regreso a la tierra de Judá.
\bibleverse{8} Sin embargo, al partir, Noemí le dijo a sus dos nueras:
``Vuelvan cada una a la casa de sus madres, y que el Señor sea tan bueno
con ustedes como lo ha sido conmigo y con los que han muerto.
\bibleverse{9} Que el Señor les de un buen hogar con otro marido''.
Entonces las besó, y todas se pusieron a llorar a gritos.

\bibleverse{10} ``¡No! Queremos volver contigo a tu pueblo'',
respondieron.

\bibleverse{11} ``¿Por qué quieren volver conmigo?'' preguntó Noemí.
``No puedo tener más hijos para que se casen con ellos. \bibleverse{12}
Regresena casa, hijas mías, porque soy demasiado vieja para volver a
casarme. Aunque esta noche me acostara con un nuevo marido y tuviera
hijos, \bibleverse{13} ¿esperarían a que crecieran? ¿Decidirían que no
van a casarse con nadie más? No.~Toda esta situación es más amarga para
mí que para ustedes, ¡pues el Señor se ha vuelto contra mí!''\footnote{\textbf{1:13}
  ``El Señor se ha vuelto contra mi'': Literalmente, ``la mano del Señor
  se ha puesto contra mi''.} \footnote{\textbf{1:13} Job 19,21}

\bibleverse{14} Y volvieron a llorar a gritos. Entonces Orfa se despidió
de su suegra con un beso. Pero Rut se aferró con fuerza a Noemí.
\bibleverse{15} ``Mira, tu cuñada vuelve con su pueblo y sus dioses.
Vuelve a casa con ella'', dijo Noemí.

\bibleverse{16} Pero Rut contestó: ``Por favor, no sigas pidiéndome que
te deje y vuelva. Donde tú vayas, yo iré. Donde tú vivas, viviré yo. Tu
pueblo será mi pueblo. Tu Dios será mi Dios. \bibleverse{17} Donde tú
mueras, moriré yo, y allí seré enterrada. Que el Señor me castigue
duramente si dejo que algo que no sea la muerte nos separe''.

\bibleverse{18} Cuando Noemí vio que Rut estaba decidida a irse con
ella, dejó de decirle que se fuera a casa.

\hypertarget{llegada-y-recepciuxf3n-de-las-dos-mujeres-en-beluxe9n}{%
\subsection{Llegada y recepción de las dos mujeres en
Belén}\label{llegada-y-recepciuxf3n-de-las-dos-mujeres-en-beluxe9n}}

\bibleverse{19} Así que las dos siguieron caminando hasta llegar a
Belén. Cuando llegaron allí, todo el pueblo se alborotó. ``¿Es ésta
Noemí?''\footnote{\textbf{1:19} No es que no la reconocieran, sino que
  volvía como una viuda en malas circunstancias.} le preguntaron las
mujeres.

\bibleverse{20} Ella les dijo: ``¡No me llamen Noemí! Llámenme
Mara,\footnote{\textbf{1:20} Naomi significa ``feliz'', mientras que
  Mara significa ``amarga''.} porque el Todopoderoso me ha tratado muy
amargamente. \footnote{\textbf{1:20} Éxod 15,23}

\bibleverse{21} Salí de aquí llena, pero el Señor me ha traído a casa
vacía. ¿Por qué me llaman Noemí cuando el Señor me ha condenado, cuando
el Todopoderoso ha traído el desastre sobre mí?'' \bibleverse{22} Así
regresó Noemí de Moab con Rut, la moabita, su nuera. Llegaron a Belén al
comienzo de la cosecha de cebada.

\hypertarget{rut-viene-a-recoger-espigas-en-el-campo-del-booz-quien-pregunta-por-ella-y-la-recibe-amablemente}{%
\subsection{Rut viene a recoger espigas en el campo del booz, quien
pregunta por ella y la recibe
amablemente}\label{rut-viene-a-recoger-espigas-en-el-campo-del-booz-quien-pregunta-por-ella-y-la-recibe-amablemente}}

\hypertarget{section-1}{%
\section{2}\label{section-1}}

\bibleverse{1} Noemí tenía un pariente por parte de su marido que se
llamaba Booz. Era un hombre rico e influyente de la familia de Elimelec.
\bibleverse{2} Poco después, Rut la moabita le dijo a Noemí: ``Por
favor, déjame ir a los campos a recoger el grano que ha quedado, si
encuentro a alguien que me dé permiso''. ``Sí, adelante, hija mía'',
respondió Noemí.

\bibleverse{3} Así que fue a recoger el grano que habían dejado los
segadores. Resulta que estaba trabajando en un campo que pertenecía a
Booz, un pariente de Elimelec.

\bibleverse{4} Más tarde, Booz llegó de Belén y le dijo a los segadores:
``¡El Señor esté con ustedes!''. Y ellos respondieron: ``¡El Señor lo
bendiga!''.

\bibleverse{5} Entonces Boozle preguntó a su criado, que estaba a cargo
de los segadores: ``¿Con quién está emparentada esta joven?''\footnote{\textbf{2:5}
  Literalmente, ``¿De quién es esa joven?''}

\bibleverse{6} ``La joven es una moabita que volvió con Noemí de Moab'',
respondió el criado. \bibleverse{7} ``Me pidió permiso para recoger el
grano detrás de los segadores.\footnote{\textbf{2:7} El hebreo añade
  ``entre las gavillas'', pero es probable que sea una transposición del
  versículo 15. Booz no le dio este inusual permiso hasta más tarde.}
Así que vino, y ha estado trabajando aquí desde la mañana hasta ahora,
salvo un breve descanso en el refugio''.

\bibleverse{8} Booz fue a hablar con Rut. ``Escúchame, hija mía'', le
dijo. ``No te vayas a recoger el grano en el campo de otro. Quédate
cerca de mis mujeres. \bibleverse{9} Presta atención a la parte del
campo que cosechan los hombres y sigue a las mujeres.\footnote{\textbf{2:9}
  Se cree que los hombres realizaban el trabajo de cortar los tallos del
  grano, mientras que las mujeres iban detrás atándolos en gavillas.}
Les he dicho a los hombres que no te molesten. Cuando tengas sed, ve a
beber de las jarras de agua que han llenado los criados''.

\bibleverse{10} Ella se inclinó con el rostro hacia el suelo. ``¿Por qué
eres tan amable conmigo o te fijas en mí, viendo que soy extranjera?'' ,
le preguntó.

\bibleverse{11} ``Me he enterado de todo lo que has hecho por tu suegra
desde que murió tu marido'', respondió Booz. ``Y también cómo dejaste a
tu padre y a tu madre, y la tierra donde naciste, para venir a vivir
entre gente que no conocías. \bibleverse{12} Que el Señor te recompense
plenamente por todo lo que has hecho: el Señor, el Dios de Israel, a
quien has acudido en busca de protección''.\footnote{\textbf{2:12}
  Literalmente, ``bajo cuyas alas se han refugiado''.}

\bibleverse{13} ``Gracias por ser tan bueno conmigo, señor'' --
respondió ella -- ``Me has tranquilizado al hablarme con amabilidad. Ni
siquiera soy uno de tus siervos''.

\hypertarget{rut-sigue-siendo-tratada-amablemente-por-booz-llega-a-casa-con-una-rica-cosecha-y-recibe-informaciuxf3n-sobre-booz-de-su-suegra}{%
\subsection{Rut sigue siendo tratada amablemente por Booz, llega a casa
con una rica cosecha y recibe información sobre Booz de su
suegra}\label{rut-sigue-siendo-tratada-amablemente-por-booz-llega-a-casa-con-una-rica-cosecha-y-recibe-informaciuxf3n-sobre-booz-de-su-suegra}}

\bibleverse{14} Cuando llegó la hora de comer, Booz la llamó. ``Ven
aquí'', le dijo. ``Toma un poco de pan y mójalo en vinagre de vino''.
Así que ella se sentó con los trabajadores y Booz le pasó un poco de
grano tostado para que comiera. Ella comió hasta saciarse y le sobró
algo.

\bibleverse{15} Cuando Rut volvió a trabajar, Booz dijo a sus hombres:
``Dejen que recoja el grano incluso entre las gavillas. No le digan nada
que la avergüence. \bibleverse{16} De hecho, saquen algunos tallos de
los manojos que estén cortando y déjenlos para que los recoja. No la
regañen''. \footnote{\textbf{2:16} Lev 19,9}

\bibleverse{17} Rut trabajó en el campo hasta la noche. Cuando sacó el
grano que había recogido era una gran cantidad.\footnote{\textbf{2:17}
  ``Una gran cantidad'',: Literalmente, ``Un efa'', unidad de medida de
  cantidad incierta, estimada entre 22 y 45 litros.} \bibleverse{18} Lo
recogió y lo llevó a la ciudad para mostrarle a su suegra la cantidad
que había recogido. Rut también le dio lo que le había sobrado de la
comida.

\bibleverse{19} Noemí le preguntó: ``¿Dónde has recogido hoy el grano?
¿Dónde has trabajado exactamente? Bendice a quien se haya preocupado lo
suficiente por ti como para darte atención''. Entonces ella le contó a
su suegra con quién había trabajado. ``El hombre con el que he trabajado
hoy se llama Booz''.

\bibleverse{20} ``¡Que el Señor lo bendiga!'' exclamó Noemí a su nuera.
``Sigue mostrando su bondad con los vivos y con los muertos. Ese hombre
es un pariente cercano a nosotros, es un `redentor de la
familia'\,''.\footnote{\textbf{2:20} ``Redentor de la familia'': término
  para designar a alguien que tenía la responsabilidad de proteger los
  intereses de la familia, especialmente en el caso de alguien que
  moría.}

\bibleverse{21} Rut añadió: ``También me dijo: `Quédate cerca de mis
trabajadores hasta que terminen de recoger toda mi cosecha'\,''.

\bibleverse{22} ``Eso está bien, hija mía'', le dijo Noemí a Rut.
``Quédate con sus trabajadoras. No vayas a otros campos donde podrían
molestarte''. \bibleverse{23} Así que Rut se quedó con las trabajadoras
de Booz recogiendo el grano hasta el final de la cosecha de cebada, y
luego hasta el final de la cosecha de trigo. Vivió con su suegra todo el
tiempo.

\hypertarget{siguiendo-el-consejo-de-noemuxed-rut-va-a-la-era-de-booz-y-se-acuesta-a-sus-pies}{%
\subsection{Siguiendo el consejo de Noemí, Rut va a la era de Booz y se
acuesta a sus
pies}\label{siguiendo-el-consejo-de-noemuxed-rut-va-a-la-era-de-booz-y-se-acuesta-a-sus-pies}}

\hypertarget{section-2}{%
\section{3}\label{section-2}}

\bibleverse{1} Un tiempo más tarde, Noemí le dijo a Rut: ``Hija mía, ¿no
crees que debería encontrarte un marido y un buen hogar?\footnote{\textbf{3:1}
  ``Un marido y un buen hogar'': La palabra utilizada aquí se refiere al
  descanso y la seguridad que proporciona el estar casado.}
\bibleverse{2} No ignores que Booz, con cuyas mujeres trabajaste, está
muy emparentado con nosotros. Esta noche estará ocupado aventando el
grano en la era.\footnote{\textbf{3:2} El grano se procesaba primero
  mediante la trilla, procedimiento por el cual se separaba el grano de
  los tallos. Luego se aventaba lanzándolo al aire para que el viento se
  llevara la cáscara exterior del grano, llamada paja, y el grano
  volviera a caer para ser recogido.} \bibleverse{3} Báñate, ponte
perfume, ponte tu mejor\footnote{\textbf{3:3} El hebreo no dice
  específicamente ``lo mejor'', pero seguramente está implícito.} ropa y
baja a la era, pero que no te reconozca. Cuando haya terminado de comer
y beber, \bibleverse{4} observa dónde se acuesta. Entonces ve y descubre
sus pies y acuéstate. Entonces él te dirá lo que tienes que
hacer''.\footnote{\textbf{3:4} La acción de Rut era un símbolo
  reconocido de pedir protección e iniciar la obligación de ``redentor
  de la familia'' (véase 2:20). Por eso dice que Booz ``te dirá lo que
  debes hacer'', refiriéndose a los requisitos necesarios para cumplir
  con esta obligación.}

\bibleverse{5} ``Haré todo lo que me has dicho'', dijo Rut.
\bibleverse{6} Bajó a la era e hizo lo que su suegra le había dicho.
\bibleverse{7} Cuando Booz terminó de comer y beber, y se sintió
satisfecho, fue a acostarse junto al montón de grano. Rut se acercó
tranquilamente a él, le descubrió los pies y se acostó.

\hypertarget{rut-habla-con-booz-recibe-la-confirmaciuxf3n-solicitada-y-regresa-a-noemuxed-con-un-regalo}{%
\subsection{Rut habla con Booz, recibe la confirmación solicitada y
regresa a Noemí con un
regalo}\label{rut-habla-con-booz-recibe-la-confirmaciuxf3n-solicitada-y-regresa-a-noemuxed-con-un-regalo}}

\bibleverse{8} Hacia la medianoche, Booz se despertó de repente. Al
inclinarse hacia delante, se sorprendió al ver a una mujer tendida a sus
pies. \bibleverse{9} ``¿Quién eres?'' , preguntó. ``Soy Rut, tu
sierva'', respondió ella. ``Por favor, extiende la esquina de tu manto
sobre mí, porque eres el redentor de mi familia''.\footnote{\textbf{3:9}
  De nuevo, este acto simbólico era una petición para cumplir con la
  obligación de redimir a la familia, lo cual incluía el matrimonio.}
\footnote{\textbf{3:9} Deut 25,5; Ezeq 16,8}

\bibleverse{10} ``Que el Señor te bendiga, hija mía'', dijo él. ``Estás
mostrando aún más lealtad y amor a la familia que antes. No has ido a
buscar a un hombre más joven, sea cual sea su condición
social.\footnote{\textbf{3:10} ``Condición social'': Literalmente,
  ``pobre o rico''.} \footnote{\textbf{3:10} Rut 2,11} \bibleverse{11}
Así que no te preocupes, hija mía. Haré todo lo que me pidas; todo el
pueblo sabe que eres una mujer de buen carácter. \bibleverse{12} Sin
embargo, aunque soy uno de los redentores de tu familia, hay uno que
está más emparentado que yo. \bibleverse{13} Quédate aquí esta noche, y
por la mañana si él quiere redimirte, pues bien, que lo haga. Pero si no
lo hace, te prometo, en nombre del Señor vivo, que te redimiré.
Acuéstate aquí hasta la mañana''.

\bibleverse{14} Así que Rut se acostó a sus pies hasta la mañana. Luego
se levantó antes de que hubiera luz suficiente para reconocer a alguien,
porque Booz le había dicho: ``Nadie debe saber que una mujer vino aquí a
la era''.\footnote{\textbf{3:14} Es evidente que Booz estaba preocupado
  por proteger la reputación de Rut.} \bibleverse{15} También le dijo:
``Tráeme el manto que llevas puesto y extiéndelo''. Ella se lo tendió y
él echó en él seis medidas\footnote{\textbf{3:15} Estimado en 24 litros
  o 50 libras.} de cebada en él. La ayudó a ponérselo a la espalda y
ella\footnote{\textbf{3:15} La mayoría de los manuscritos hebreos leen
  ``él''. Aquí se siguen los manuscritos minoritarios.} regresó a la
ciudad.

\bibleverse{16} Rut fue a ver a su suegra, que le preguntó: ``¿Cómo te
ha ido, hija mía?''\footnote{\textbf{3:16} Aqui ``¿Cómo te ha ido?'' es
  literalmente, ``¿Qué noticias traes, hija mía?''} Entonces Rut le
contó todo lo que Booz había hecho por ella.

\bibleverse{17} ``Y también me dio estas seis medidas de cebada'',
añadió. ``Me dijo: `No debes ir a casa de tu suegra con las manos
vacías'\,''.

\bibleverse{18} Noemí dijo a Rut: ``Espera con paciencia, hija mía,
hasta que sepas cómo se resuelve todo. Booz no descansará hasta tenerlo
resuelto hoy''.

\hypertarget{la-negociaciuxf3n-puxfablica-entre-booz-y-el-solver}{%
\subsection{La negociación pública entre Booz y el
Solver}\label{la-negociaciuxf3n-puxfablica-entre-booz-y-el-solver}}

\hypertarget{section-3}{%
\section{4}\label{section-3}}

\bibleverse{1} Booz fue a la puerta de la ciudad,\footnote{\textbf{4:1}
  Los asuntos civiles, incluidos los jurídicos, se llevaban a cabo en
  los alrededores de la puerta de la ciudad.} y se sentó allí. El
redentor de la familia que Booz había mencionado pasó por allí, así que
Booz le dijo: ``Ven aquí, amigo, y siéntate''. El hombre se acercó y se
sentó. \bibleverse{2} Entonces Booz seleccionó a diez de los ancianos
del pueblo y les pidió que se sentaran allí con ellos. \bibleverse{3}
Booz le dijo al redentor de la familia: ``Noemí, que ha regresado del
país de Moab, está vendiendo el terreno que pertenecía a Elimelec,
nuestro pariente. \bibleverse{4} Hedecidido decírtelo por si quieres
comprarlo aquí, en presencia de estos ancianos del pueblo. Si quieres
redimirla, adelante. Pero si no quieres, dímelo para que lo sepa, porque
tú eres el primero en la fila para canjearlo, y yo soy el siguiente''.
``Quiero redimirla'',\footnote{\textbf{4:4} La respuesta no es muy
  positiva.} dijo el redentor de la familia. \footnote{\textbf{4:4} Lev
  25,25}

\bibleverse{5} ``Cuando compras la tierra a Noemí, también adquieres a
Rut la moabita, la viuda de Mahlón, para poder casarte con ella y tener
hijos con ella para asegurar la continuidad del linaje del
hombre'',\footnote{\textbf{4:5} La disposición sobre el matrimonio se
  encuentra en Deuteronomio 25:5-10 y siguientes, mientras que las leyes
  de transferencia de tierras están en Levítico 25:23-28.} explicó Booz.
\footnote{\textbf{4:5} Deut 25,5-6}

\bibleverse{6} ``Pues entonces no puedo hacerlo'', respondió el redentor
de la familia. ``Si la redimiera, eso podría poner en peligro lo que ya
poseo.\footnote{\textbf{4:6} Al hombre le preocupaba que cualquier
  propiedad que ya tuviera se incluyera también en el legado a cualquier
  hijo que tuviera Rut, y que se acreditara a la línea de su marido
  muerto.} Redímela tú, porque yo no puedo''.

\bibleverse{7} (Ahora bien, en aquellos tiempos era costumbre en Israel
confirmar la acción del redentor familiar, el traspaso de la propiedad o
cualquier asunto legal similar, quitándose una sandalia y entregándola.
Esta era la forma de validar una transacción en Israel). \footnote{\textbf{4:7}
  Deut 25,7-10} \bibleverse{8} Así que el redentor familiar se quitó la
sandalia y le dijo a Booz: ``Cómprala tú''.

\bibleverse{9} Entonces Booz dijo a los ancianos y a todo el pueblo
presente: ``Ustedes son testigos de que hoy he comprado a Noemí todo lo
que pertenecía a Elimelec, Mahlón y Quelión. \bibleverse{10} También he
adquirido como esposa a Rut la moabita, viuda de Mahlón. Al tener hijos
que puedan heredar sus bienes, su nombre se mantendrá vivo en su familia
y en su ciudad natal. Ustedes son testigos de esto hoy''.

\bibleverse{11} Los ancianos y todo el pueblo presente en la puerta de
la ciudad dijeron: ``Sí, somos testigos. Que el Señor haga que la mujer
que viene a tu casa sea como Raquel y Lea, que entre ambas dieron a luz
al pueblo de Israel. Que seas próspera en Efrata y famosa en Belén.
\bibleverse{12} Que tu descendencia que el Señor te da a través de esta
joven llegue a ser como la descendencia de Fares, el hijo que Tamar dio
a Judá''.

\hypertarget{el-matrimonio-de-des-booz-con-rut-se-completuxf3-y-fue-bendecido-con-el-nacimiento-de-obed-uxedndice-de-guxe9nero-de-puxe9rez-a-david}{%
\subsection{El matrimonio de Des Booz con Rut se completó y fue
bendecido con el nacimiento de Obed; Índice de género de Pérez a
David}\label{el-matrimonio-de-des-booz-con-rut-se-completuxf3-y-fue-bendecido-con-el-nacimiento-de-obed-uxedndice-de-guxe9nero-de-puxe9rez-a-david}}

\bibleverse{13} Booz se llevó a Rut a su casa y ella se convirtió en su
esposa. Se acostó con ella, y el Señor dispuso que quedara embarazada, y
dio a luz un hijo. \footnote{\textbf{4:13} Sal 127,3} \bibleverse{14}
Las mujeres de la ciudad\footnote{\textbf{4:14} Véase 1:19.} se
acercaron a Noemí y le dijeron: ``Alaba al Señor, porque hoy no te ha
dejado sin redentor de familia al darte este nieto\footnote{\textbf{4:14}
  ``Al darte este nieto'': implícito.} que tendrá gran nombre en todo
Israel. \bibleverse{15} Él te dará una nueva vida y te mantendrá en tu
vejez, porque tu nuera, que te ama y que es mejor que siete hijos para
ti, lo ha dado a luz''. \bibleverse{16} Noemí cogió al niño y lo abrazó.
Lo cuidó como a su propio hijo.\footnote{\textbf{4:16} Literalmente,
  ``se convirtió en su niñera''.} \bibleverse{17} Las vecinas le
pusieron el nombre de Obed\footnote{\textbf{4:17} ``Obed'', que
  significa ``siervo'' como en ``siervo de Dios''.} diciendo: ``¡Noemí
tiene ahora un hijo!'' Era el padre de Jesé, que fue el padre de David.

\bibleverse{18} Este es el linaje de Fares: Fares fue el padre de
Jezrón. \bibleverse{19} Jezron fue el padre de Ram. Ram fue el padre de
Aminadab. \bibleverse{20} Aminadab fue el padre de Naasón. Naasón fue el
padre de Salmón. \bibleverse{21} Salmón fue el padre de Booz. Booz fue
el padre de Obed. \bibleverse{22} Obed fue el padre de Isaí. Isaí fue el
padre de David.
