\hypertarget{jesuxfas-como-el-verbo-hecho-hombre}{%
\subsection{Jesús como el ``Verbo'' hecho
hombre}\label{jesuxfas-como-el-verbo-hecho-hombre}}

\hypertarget{section}{%
\section{1}\label{section}}

\bibleverse{1} En el principio, la Palabra ya existía.\footnote{\textbf{1:1}
  En otras palabras, la Palabra existía desde la eternidad pasada. El
  concepto de la Palabra significa más que letras que conforman una
  palabra: es la mente divina, la expresión de Dios, es el aspecto
  activo de la divinidad que habla y da vida, como se expresa en Génesis
  1:1.} La Palabra estaba con Dios, y la Palabra era Dios.
\bibleverse{2} En el principio, Jesús ---quien era la palabra--- estaba
con Dios. \bibleverse{3} Todo llegó a existir por medio de él; y sin él
nada llegó a existir. \footnote{\textbf{1:3} 1Cor 8,6; Col 1,16-17; Heb
  1,2} \bibleverse{4} En él estaba la vida, la vida que era la luz de
todos. \footnote{\textbf{1:4} Juan 8,12} \bibleverse{5} La luz brilla en
la oscuridad, y la oscuridad no la ha apagado.\footnote{\textbf{1:5}
  Esta palabra, en el original, también puede significar ``subyugada'' o
  ``entendida''.} \footnote{\textbf{1:5} Juan 3,19}

\bibleverse{6} Dios envió a un hombre llamado Juan. \footnote{\textbf{1:6}
  Mat 3,1; Mar 1,4} \bibleverse{7} Él vino como testigo para hablar
acerca de la luz, a fin de que todos pudieran creer por medio de él.
\footnote{\textbf{1:7} Hech 19,4} \bibleverse{8} Él mismo no era la luz,
sino que vino a testificar de la luz. \bibleverse{9} La luz verdadera
estaba por venir al mundo para dar luz a todos.

\bibleverse{10} Él estuvo en el mundo, y aunque el mundo fue hecho por
medio de él, el mundo no supo quién era él.\footnote{\textbf{1:10} O
  ``no lo identificaron''.} \bibleverse{11} Él vino a su pueblo, pero
ellos no lo aceptaron. \bibleverse{12} Pero a aquellos que lo aceptaron,
a quienes creyeron en él, les dio el derecho de convertirse en hijos de
Dios. \bibleverse{13} Estos son los hijos que no nacieron de forma
habitual, o como resultado de los deseos o de la voluntad humana, sino
nacidos de Dios. \footnote{\textbf{1:13} Juan 3,5-6}

\bibleverse{14} La Palabra se volvió humana y vivió entre nosotros, y
nosotros vimos su gloria, la gloria del único\footnote{\textbf{1:14}
  Literalmente, ``unigénito''. Esto hace referencia a posición y
  singularidad más que al nacimiento.} hijo del Padre, lleno de gracia y
verdad. \footnote{\textbf{1:14} Is 7,14; Is 60,1; 2Pe 1,16-17}
\bibleverse{15} Juan dio su testimonio acerca de él, exclamando al
pueblo: ``Este es del cual yo les hablaba cuando les dije: `El que viene
después de mi es más importante que yo, porque antes de que yo viviera,
ya él existía'\,''. \bibleverse{16} Nosotros todos hemos sido receptores
de su generosidad, de un don gratuito tras otro. \footnote{\textbf{1:16}
  Juan 3,34; Col 1,19} \bibleverse{17} La ley fue dada por medio de
Moisés; pero la gracia y la verdad vinieron por medio de Jesucristo.
\footnote{\textbf{1:17} Rom 10,4} \bibleverse{18} Aunque ninguno ha
visto a Dios, Jesucristo, el Único e Incomparable, quien está cerca del
Padre, nos ha mostrado cómo es Dios.\footnote{\textbf{1:18} O ``lo ha
  dado a conocer''.} \footnote{\textbf{1:18} Juan 6,46; Mat 11,27}

\hypertarget{el-testimonio-de-suxed-mismo-del-bautista}{%
\subsection{El testimonio de sí mismo del
Bautista}\label{el-testimonio-de-suxed-mismo-del-bautista}}

\bibleverse{19} Esto es lo que Juan afirmó públicamente cuando los
líderes judíos enviaron sacerdotes y Levitas desde Jerusalén para
preguntarle: ``¿Quién eres tú?''

\bibleverse{20} Juan declaró claramente y sin dudar: ``Yo no soy el
Mesías''.

\bibleverse{21} ``Entonces, ¿quién eres?'' preguntaron ellos.
``¿Elías?'' ``No, no lo soy'', respondió él. ``¿Eres tú el
Profeta?''\footnote{\textbf{1:21} En el pensamiento judío se esperaba un
  profeta especial antes del fin.} ``No'', respondió él.

\bibleverse{22} ``¿Quién eres tú, entonces?'' preguntaron ellos.
``Tenemos que dar una respuesta a quienes nos enviaron. ¿Qué dices de ti
mismo?''

\bibleverse{23} ``Yo soy `una voz que clama en el desierto: ``¡Enderecen
el camino del Señor!''\,'\,'' dijo él, usando las palabras del profeta
Isaías\footnote{\textbf{1:23} Citando Isaías 40:3.} .

\bibleverse{24} Los sacerdotes y los Levitas\footnote{\textbf{1:24}
  ``Sacerdotes y Levitas'': Esto está implícito en el versículo 19.}
enviados por los Fariseos \bibleverse{25} le preguntaron: ``¿Por qué,
entonces, estás bautizando, si no eres el Mesías, ni Elías, ni el
Profeta?''

\bibleverse{26} Juan respondió: ``Yo bautizo con agua, pero entre
ustedes está alguien a quien ustedes no conocen. \footnote{\textbf{1:26}
  Luc 17,21} \bibleverse{27} Él viene después de mí, pero yo ni siquiera
soy digno de desabrochar sus sandalias''. \bibleverse{28} Todo esto
ocurrió en Betania, al otro lado del Jordán, donde Juan estaba
bautizando.

\hypertarget{el-testimonio-del-bautista-acerca-de-jesuxfas}{%
\subsection{El testimonio del Bautista acerca de
Jesús}\label{el-testimonio-del-bautista-acerca-de-jesuxfas}}

\bibleverse{29} Al día siguiente, Juan vio que Jesús se acercaba a él, y
dijo: ``¡Miren, el Cordero de Dios que quita el pecado del mundo!
\bibleverse{30} Este es del cual yo les hablaba cuando dije: `El hombre
que viene después de mí es más importante que yo, porque antes de que yo
existiera él ya existía'. \bibleverse{31} Yo mismo no sabía quién era
él, pero vine a bautizar con agua a fin de que él pudiera ser revelado a
Israel''. \bibleverse{32} Juan dio su testimonio acerca de él, diciendo:
``Vi al Espíritu descender del cielo como una paloma y se posó sobre él.
\bibleverse{33} Yo no lo habría conocido si no fuera porque el que me
envió a bautizar con agua me había dicho: `Aquél sobre el cual veas
descender el Espíritu y posarse sobre él, ese es quien bautiza con el
Espíritu Santo'. \bibleverse{34} Yo lo vi, y declaro que este es el Hijo
de Dios''.

\bibleverse{35} El día siguiente Juan estaba allí con dos de sus
discípulos. \bibleverse{36} Él vio a Jesús que pasaba y dijo: ``¡Miren!
¡Este es el Cordero de Dios!'' \bibleverse{37} Cuando los dos discípulos
escucharon lo que él dijo, fueron y siguieron a Jesús. \bibleverse{38}
Jesús volteó y vio que estos le seguían. ``¿Qué están buscando?'' les
preguntó, ``Rabí (que significa `Maestro'), ¿dónde vives?'' le
preguntaron ellos, como respuesta.

\bibleverse{39} ``Vengan y vean'', les dijo. Así que ellos se fueron con
él y vieron donde vivía. Eran cerca de las cuatro de la tarde, y pasaron
el resto del día con él.

\bibleverse{40} Andrés, el hermano de Simón Pedro, era uno de estos
discípulos que habían escuchado lo que Juan dijo y que habían seguido a
Jesús. \footnote{\textbf{1:40} Mat 4,18-20} \bibleverse{41} Él se fue de
inmediato a buscar a su hermano Simón y le dijo: ``¡Hemos encontrado al
Mesías!'' (Que significa ``Cristo'').\footnote{\textbf{1:41} Cristo
  significa ``el Ungido''.} \bibleverse{42} Él lo llevó donde estaba
Jesús. Mirándolo fijamente, Jesús le dijo: ``Tú eres Simón, hijo de
Juan. Pero ahora te llamarás Cefas'', (que significa
``Pedro'').\footnote{\textbf{1:42} Tanto Cefas como Pedro significan
  ``roca'' o ``piedra''.}

\bibleverse{43} El siguiente día, Jesús decidió ir a Galilea. Allí
encontró a Felipe, y le dijo: ``Sígueme''. \bibleverse{44} Felipe era de
Betsaida, la misma ciudad de donde venían Andrés y Pedro.
\bibleverse{45} Felipe encontró a Natanael y le dijo: ``Hemos encontrado
a aquél de quien Moisés hablaba en la ley y de quien hablaban los
profetas también: Jesús de Nazaret, el hijo de José''. \footnote{\textbf{1:45}
  Deut 18,18; Is 53,2; Jer 23,5; Ezeq 34,23}

\bibleverse{46} ``¿De Nazaret? ¿Puede salir algo bueno de allí?''
preguntó Natanael. ``Solo ven y mira'', respondió Felipe. \footnote{\textbf{1:46}
  Juan 7,41}

\bibleverse{47} Cuando Jesús vio que Natanael se acercaba, dijo de él:
``¡Miren, aquí hay un verdadero israelita! No hay ninguna falsedad en
él''.

\bibleverse{48} ``¿Cómo sabes quien soy yo?'' preguntó Natanael. ``Te vi
bajo aquella higuera, antes que Felipe te llamara'', respondió Jesús.

\bibleverse{49} ``¡Rabí, tu eres el Hijo de Dios, el rey de Israel!''
exclamó Natanael. \footnote{\textbf{1:49} Sal 2,7; Jer 23,5; Juan 6,69;
  Mat 14,33; Mat 16,16}

\bibleverse{50} ``¿Crees esto solo porque te dije que te vi bajo aquella
higuera?'' respondió Jesús. ``¡Verás mucho más que eso!''
\bibleverse{51} Luego Jesús dijo: ``Les digo la verdad: verán el cielo
abierto, y los ángeles de Dios subiendo y bajando sobre el Hijo del
hombre''.\footnote{\textbf{1:51} Refiriéndose a la experiencia de Jacob
  en Génesis 28:12 con el término ``Hijo de Dios'' reemplazando la
  palabra ``escalera''.}

\hypertarget{la-primera-seuxf1al-milagrosa-de-jesuxfas-en-las-bodas-de-canuxe1}{%
\subsection{La primera señal milagrosa de Jesús en las bodas de
Caná}\label{la-primera-seuxf1al-milagrosa-de-jesuxfas-en-las-bodas-de-canuxe1}}

\hypertarget{section-1}{%
\section{2}\label{section-1}}

\bibleverse{1} Dos días\footnote{\textbf{2:1} Literalmente ``el tercer
  día'' (por cálculos inclusivos).} después, se estaba celebrando una
boda en Caná de Galilea y la madre de Jesús estaba allí. \bibleverse{2}
Jesús y sus discípulos también habían sido invitados a la boda.
\bibleverse{3} El vino se acabó, así que la madre de Jesús le dijo: ``No
tienen más vino''.

\bibleverse{4} ``Madre, ¿por qué deberías involucrarme?\footnote{\textbf{2:4}
  Literalmente, ``¿Qué tiene que ver contigo y conmigo?''} Mi tiempo no
ha llegado aún'', respondió él. \footnote{\textbf{2:4} Juan 19,26}

\bibleverse{5} Su madre dijo a los sirvientes: ``Hagan todo lo que él
les diga''.

\bibleverse{6} Cerca de allí había seis tinajas que usaban los judíos
para la purificación ceremonial, en cada una cabían veinte o treinta
galones.\footnote{\textbf{2:6} Literalmente ``dos o tres medidas''.}
\bibleverse{7} ``Llenen las tinajas con agua'', les dijo Jesús. Así que
ellos las llenaron por completo. \bibleverse{8} Luego les dijo: ``Sirvan
un poco y llévenlo al maestro de ceremonias''. Entonces ellos sirvieron
un poco. \bibleverse{9} El maestro de ceremonias no sabía de dónde había
venido, solamente los sirvientes lo sabían. Pero cuando probó el agua
que había sido convertida en vino, llamó al esposo. \bibleverse{10}
``Todo el mundo sirve primero el mejor vino'', le dijo, ``y cuando las
personas ya han bebido suficiente, entonces sirven el vino más barato.
¡Pero tú has servido el mejor vino hasta el final!'' \bibleverse{11}
Esta fue la primera de las señales milagrosas de Jesús, y fue realizada
en Caná de Galilea. Aquí él dio a conocer su gloria, y sus discípulos
pusieron su confianza en él. \footnote{\textbf{2:11} Juan 1,14}

\hypertarget{jesuxfas-por-primera-vez-en-jerusaluxe9n-en-la-pascua}{%
\subsection{Jesús por primera vez en Jerusalén en la
Pascua}\label{jesuxfas-por-primera-vez-en-jerusaluxe9n-en-la-pascua}}

\bibleverse{12} Después de esto, Jesús partió hacia Capernaúm con su
madre, sus hermanos y sus discípulos, y se quedaron allí unos pocos
días. \footnote{\textbf{2:12} Juan 7,3; Mat 13,55}

\bibleverse{13} Como ya casi era la fecha de la Pascua de los Judíos,
Jesús se fue a Jerusalén. \footnote{\textbf{2:13} Mat 20,18; Mar 11,1;
  Luc 19,28; Juan 5,1} \bibleverse{14} En el Templo, encontró personas
vendiendo ganado, ovejas y palomas; y los cambistas de monedas estaban
sentados en sus mesas. \bibleverse{15} Él elaboró un látigo con cuerdas
y los hizo salir a todos del Templo, junto con las ovejas y el ganado,
esparciendo las monedas de los cambistas y volteando sus mesas.
\bibleverse{16} Ordenó a los vendedores de palomas: ``¡Saquen todas
estas cosas de aquí! ¡No conviertan la casa de mi Padre en un mercado!''
\bibleverse{17} Sus discípulos recordaron la Escritura que dice: ``¡Mi
devoción por tu casa es como un fuego que arde dentro de
mí!''\footnote{\textbf{2:17} Citando Salmos 69:9.}

\bibleverse{18} Los líderes judíos reaccionaron, preguntándole: ``¿Qué
derecho tienes para hacer esto? ¡Muéstranos una señal milagrosa para
probarlo!''

\bibleverse{19} Jesús respondió: ``¡Destruyan este Templo, y en tres
días lo levantaré!'' \footnote{\textbf{2:19} Mat 26,61; Mat 27,40}

\bibleverse{20} ``Tomó cuarenta y seis años construir este Templo, ¿y tú
vas a levantarlo en tres días?'' respondieron los líderes judíos.
\bibleverse{21} Pero el Templo del cual hablaba Jesús era su cuerpo.
\bibleverse{22} Después que Jesús se levantó de entre los muertos, sus
discípulos recordaron lo que él dijo, y por esto creyeron en la
Escritura y en las propias palabras de Jesús. \footnote{\textbf{2:22} Os
  6,2}

\bibleverse{23} Como resultado de los milagros que Jesús hizo mientras
estuvo en Jerusalén durante la Pascua, muchos creyeron en él.
\bibleverse{24} Pero Jesús mismo no se confiaba de ellos, porque él
conocía a todas las personas. \bibleverse{25} Él no necesitaba que nadie
le hablara acerca de la naturaleza humana porque él conocía cómo
pensaban las personas.

\hypertarget{jesuxfas-y-nicodemo}{%
\subsection{Jesús y Nicodemo}\label{jesuxfas-y-nicodemo}}

\hypertarget{section-2}{%
\section{3}\label{section-2}}

\bibleverse{1} Había allí un hombre llamado Nicodemo, quien era un
Fariseo y miembro del Concilio Supremo. \footnote{\textbf{3:1} Juan
  7,50; Juan 19,39} \bibleverse{2} Él vino por la noche donde Jesús
estaba y le dijo: ``Rabí, sabemos que eres un maestro que ha venido de
parte Dios, porque nadie podría hacer las señales milagrosas que tú
estás haciendo a menos que Dios esté con él''.

\bibleverse{3} ``Te digo la verdad'' respondió Jesús, ``A menos que
vuelvas a nacer,\footnote{\textbf{3:3} O ``nacido desde arriba''.} no
puedes experimentar el reino de Dios''.

\bibleverse{4} ``¿Cómo puede alguien volver a nacer, cuando ya es
viejo?'' preguntó Nicodemo. ``¡Nadie puede regresar al vientre de su
madre y nacer por segunda vez!''

\bibleverse{5} ``Te digo la verdad, no puedes entrar al reino de Dios a
menos que hayas nacido de agua y del Espíritu'', le dijo Jesús.
\footnote{\textbf{3:5} Ezeq 36,25-27; Mat 3,11; Tit 3,5} \bibleverse{6}
``Lo que nace de la carne, es carne, y lo que nace del Espíritu, es
Espíritu. \footnote{\textbf{3:6} Juan 1,13; Rom 8,5-9} \bibleverse{7} No
te sorprendas de que te dije: `Debes volver a nacer'.\footnote{\textbf{3:7}
  La frase ``no te sorprendas'' se refiere a Nicodemo, en singular. La
  frase ``debes volver a nacer'' es plural, se refiere a una audiencia
  más amplia.} \bibleverse{8} El viento sopla hacia donde quiere y
apenas se alcanza a escuchar el sonido que hace, pero no sabes de dónde
viene ni hacia dónde va; así ocurre con todo aquél que nace del
Espíritu''.

\bibleverse{9} ``¿Cómo es esto posible?'' preguntó Nicodemo.

\bibleverse{10} ``Tu eres un maestro famoso en Israel,\footnote{\textbf{3:10}
  Literalmente, ``tú eres el maestro de Israel''.} ¿y aún así no
entiendes tales cosas?'' respondió Jesús. \bibleverse{11} ``Te digo la
verdad: nosotros hablamos de lo que sabemos y damos testimonio de lo que
hemos visto, pero ustedes se niegan a aceptar nuestro testimonio.
\bibleverse{12} Si ustedes no creen lo que yo digo cuando les hablo de
cosas terrenales, ¿cómo podrán creer si les hablara de cosas
celestiales? \bibleverse{13} Nadie ha subido al cielo, sino que el Hijo
del hombre descendió del cielo. \bibleverse{14} Del mismo modo que
Moisés levantó la serpiente en el desierto,\footnote{\textbf{3:14} Ver
  Números 21:9.} así debe ser levantado el Hijo del hombre, \footnote{\textbf{3:14}
  Núm 21,8-9} \bibleverse{15} de modo que todos los que confíen en él,
tendrán vida eterna. \bibleverse{16} ``Porque Dios amó al mundo, y lo
hizo de esta manera:\footnote{\textbf{3:16} La palabra a menudo
  traducida como ``tal'' (como se lee en ``amó de tal manera'') describe
  ante todo la forma o la manera en que Dios ama, más que la medida o la
  intensidad de su amor.} entregó a su único Hijo, a fin de que todos
los que crean en él no mueran, sino que tengan vida eterna.
\bibleverse{17} Dios no envió al Hijo al mundo para condenarlo, sino
para salvar al mundo por medio de él. \footnote{\textbf{3:17} Luc 19,10}
\bibleverse{18} Aquellos que creen en él no están condenados, mientras
que aquellos que no creen en él ya están condenados porque no creyeron
en el único Hijo de Dios. \footnote{\textbf{3:18} Juan 5,24}
\bibleverse{19} Así es como se decide\footnote{\textbf{3:19} O
  ``juicio''.} esto: la luz vino al mundo, pero las personas amaban las
tinieblas más que a la luz, porque sus acciones eran malvadas.
\footnote{\textbf{3:19} Juan 1,5; Juan 1,9-11} \bibleverse{20} Todos los
que hacen el mal odian la luz y no vienen a la luz, porque no quieren
que sus acciones sean expuestas. \footnote{\textbf{3:20} Efes 5,13}
\bibleverse{21} Pero aquellos que hacen el bien\footnote{\textbf{3:21}
  Literalmente, ``hacen la verdad''.} vienen a la luz, para que se dé a
conocer lo que Dios ha hecho en ellos''. \footnote{\textbf{3:21} 1Jn 1,6}

\hypertarget{jesuxfas-en-judea-y-el-testimonio-final-del-bautista}{%
\subsection{Jesús en Judea y el testimonio final del
Bautista}\label{jesuxfas-en-judea-y-el-testimonio-final-del-bautista}}

\bibleverse{22} Después de esto, Jesús y sus discípulos fueron a Judea y
pasaron allí un tiempo con la gente, bautizándoles. \footnote{\textbf{3:22}
  Juan 4,1-2} \bibleverse{23} Juan también estaba bautizando en Enón,
cerca de Salim, porque allí había mucha agua y las personas seguían
viniendo para ser bautizadas. \bibleverse{24} (Esto ocurrió antes de que
metieran a Juan en la cárcel). \footnote{\textbf{3:24} Mar 1,14}
\bibleverse{25} Surgió un debate entre los discípulos de Juan y los
judíos respecto a la purificación ceremonial. \bibleverse{26} Ellos
fueron donde Juan y le dijeron: ``Rabí, el hombre con el que estabas al
otro lado del Jordán, del cual diste un testimonio favorable, ¡mira,
ahora está bautizando y todos están acudiendo a él!''

\bibleverse{27} ``Nadie recibe nada a menos que le sea dado del cielo'',
respondió Juan. \footnote{\textbf{3:27} Heb 5,4} \bibleverse{28}
``Ustedes mismos pueden testificar de que yo he declarado: `Yo no soy el
Mesías. He sido enviado para preparar su camino'. \footnote{\textbf{3:28}
  Juan 1,20; Juan 1,23; Juan 1,27} \bibleverse{29} ¡El novio es quien se
casa con la novia! El padrino espera y escucha al novio, y se alegra
cuando escucha la voz de alegría del novio, así que ahora mi felicidad
está completa. \footnote{\textbf{3:29} Mat 9,15} \bibleverse{30} Él debe
volverse más importante, y yo debo volverme menos importante''.

\bibleverse{31} El que viene de arriba es más grande\footnote{\textbf{3:31}
  O ``está encima'' en el sentido de autoridad.} que todos; el que viene
de la tierra pertenece a la tierra y habla cosas terrenales. El que
viene del cielo es más grande que todos. \bibleverse{32} El da
testimonio acerca de lo que ha visto y escuchado, pero nadie acepta lo
que él viene a decir. \bibleverse{33} Sin embargo, todo aquél que acepta
lo que el dice, confirma\footnote{\textbf{3:33} Literalmente ``sello de
  aprobación''.} que Dios habla la verdad. \bibleverse{34} Porque el que
Dios envió habla las palabras de Dios, porque Dios no limita al
Espíritu. \footnote{\textbf{3:34} Juan 1,16} \bibleverse{35} El Padre
ama al Hijo y ha puesto todo en sus manos. \footnote{\textbf{3:35} Juan
  5,20; Mat 11,27}

\bibleverse{36} Cualquiera que confía en el Hijo tiene vida eterna, pero
cualquiera que se niega a creer en el Hijo, no experimentará vida
eterna, sino que sigue bajo la condenación de Dios.

\hypertarget{jesuxfas-habla-con-la-mujer-samaritana-junto-al-pozo-de-jacob}{%
\subsection{Jesús habla con la mujer samaritana junto al pozo de
Jacob}\label{jesuxfas-habla-con-la-mujer-samaritana-junto-al-pozo-de-jacob}}

\hypertarget{section-3}{%
\section{4}\label{section-3}}

\bibleverse{1} Cuando Jesús se dio cuenta que los Fariseos habían
descubierto que él estaba ganando y bautizando más discípulos que Juan,
\footnote{\textbf{4:1} Juan 3,22; Juan 3,26} \bibleverse{2} (aunque no
era Jesús quien estaba bautizando, sino sus discípulos), \bibleverse{3}
se fue de Judea y regresó a Galilea. \bibleverse{4} En su camino, tenía
que pasar por Samaria. \bibleverse{5} Así que llegó a la ciudad de
Sicar, cerca del campo que Jacob había entregado a su hijo José.
\bibleverse{6} Allí estaba el pozo de Jacob, y Jesús, estando cansado
del viaje, se sentó junto al pozo. Era medio día.

\bibleverse{7} Una mujer samaritana vino a buscar agua. Y Jesús le dijo:
``¿Podrías darme de beber, por favor?'' \bibleverse{8} pues sus
discípulos habían ido a comprar comida a la ciudad.

\bibleverse{9} ``Tú eres un judío, y yo soy una mujer samaritana. ¿Cómo
puedes pedirme que te dé de beber?'' respondió la mujer, pues los judíos
no se asocian con los samaritanos.\footnote{\textbf{4:9} O ``los judíos
  no comparten comidas con los samaritanos''.} \footnote{\textbf{4:9}
  Luc 9,52-53}

\bibleverse{10} Jesús le respondió: ``Si tan solo reconocieras el don de
Dios y a quien te está pidiendo `dame de beber,' tú le habrías pedido a
él y él te habría dado el agua de vida''. \footnote{\textbf{4:10} Juan
  7,38-39}

\bibleverse{11} ``Señor, tú no tienes un cántaro, y el pozo es profundo.
¿De dónde vas a sacar el agua de vida?'' respondió ella. \bibleverse{12}
``Nuestro Padre Jacob nos dio el pozo. Él mismo bebió de él, así como
sus hijos y sus animales. ¿Eres tu más grande que él?''

\bibleverse{13} Jesús respondió: ``Todo el que bebe agua de este pozo,
volverá a tener sed. \footnote{\textbf{4:13} Juan 6,58} \bibleverse{14}
Pero los que beban del agua que yo doy, no volverán a tener sed de
nuevo. El agua que yo doy se convierte en una fuente de agua rebosante
dentro de ellos, dándoles vida eterna''. \footnote{\textbf{4:14} Juan
  6,35; Juan 7,38-39}

\bibleverse{15} ``Señor'', respondió la mujer, ``¡Por favor, dame de esa
agua para que yo no tenga más sed y no tenga que venir aquí a buscar
agua!''

\bibleverse{16} ``Ve y llama a tu esposo, y regresa aquí'', le dijo
Jesús.

\bibleverse{17} ``No tengo un esposo'', respondió la mujer. ``Estás en
lo correcto al decir que no tienes un esposo'', le dijo Jesús.

\bibleverse{18} ``Has tenido cinco esposos, y el hombre con el que estás
viviendo ahora no es tu esposo. ¡Así que lo que dices es cierto!''

\bibleverse{19} ``Puedo ver que eres un profeta, señor'', respondió la
mujer. \bibleverse{20} ``Dime esto: nuestros ancestros adoraron aquí en
este monte, pero tú\footnote{\textbf{4:20} Como judío.} dices que en
Jerusalén es donde debemos adorar''. \footnote{\textbf{4:20} Deut 12,5;
  Sal 122,-1}

\bibleverse{21} Jesús respondió:\footnote{\textbf{4:21} Jesús se dirige
  a ella como ``mujer'', el cual es el término común utilizado, pero en
  español suena descortés.} ``Créeme que viene el tiempo en que ustedes
no adorarán al Padre ni en este monte, ni en Jerusalén. \bibleverse{22}
Ustedes no conocen realmente al Dios\footnote{\textbf{4:22}
  Literalmente, ``lo que''} que están adorando, mientras que nosotros
adoramos al Dios que conocemos, porque la salvación viene de los judíos.
\bibleverse{23} Pero viene el tiempo---y de hecho, ya llegó---cuando los
adoradores adorarán al Padre en Espíritu y en verdad, porque este es el
tipo de adoradores que el Padre quiere. \bibleverse{24} Dios es
Espíritu, así que los adoradores deben adorar en Espíritu y en verdad''.
\footnote{\textbf{4:24} Rom 12,1; 2Cor 3,17}

\bibleverse{25} La mujer dijo: ``Bueno, yo sé que el Mesías vendrá'',
(al que llaman Cristo). ``Cuando él venga, él nos lo explicará a todos
nosotros''. \footnote{\textbf{4:25} Juan 1,41}

\bibleverse{26} Jesús respondió: ``YO SOY---el que habla
contigo''.\footnote{\textbf{4:26} ``YO SOY'' es usado en el Antiguo
  Testamento como un nombre para referirse a Dios. Jesús está diciéndole
  que él les el Mesías y a la vez está identificando su divinidad.}

\hypertarget{jesuxfas-y-los-discuxedpulos}{%
\subsection{Jesús y los discípulos}\label{jesuxfas-y-los-discuxedpulos}}

\bibleverse{27} Justo en ese momento, regresaron los discípulos. Ellos
estaban sorprendidos de que él estuviera hablando con una mujer, pero
ninguno de ellos le preguntó ``¿qué haces?'' o ``¿por qué estás hablando
con ella?'' \bibleverse{28} La mujer dejó su tinaja de agua y corrió de
regreso a la ciudad, diciendo a la gente: \bibleverse{29} ``¡Vengan y
conozcan a un hombre que me dijo todo lo que he hecho! ¿Podría ser este
el Mesías?'' \bibleverse{30} Entonces la gente se fue de la ciudad para
verlo.

\bibleverse{31} Mientras tanto, los discípulos de Jesús estaban
insistiéndole: ``¡Maestro, come algo, por favor!''

\bibleverse{32} Pero Jesús respondió: ``La comida que yo tengo para
comer es una de la que ustedes no saben''.

\bibleverse{33} ``¿Le trajo comida alguien?'' se preguntaban los
discípulos unos a otros.

\bibleverse{34} Jesús les explicó: ``Mi comida es hacer la voluntad de
Aquél que me envió y completar su obra. \footnote{\textbf{4:34} Juan
  6,38; Juan 17,4} \bibleverse{35} ¿No tienen ustedes el dicho: `hay
cuatro meses entre la siembra y la cosecha?'\footnote{\textbf{4:35}
  Usualmente había cuatro meses entre la siembra y la cosecha.} ¡Abran
sus ojos y miren a su alrededor! Los cultivos están maduros, listos para
la siega. \footnote{\textbf{4:35} Mat 9,37} \bibleverse{36} Al segador
se le paga bien y la cosecha es para vida eterna, a fin de que tanto el
sembrador como el segador puedan celebrar juntos. \bibleverse{37} Así
que el proverbio que dice `uno es el que siembra y otro es el que
cosecha,' es verdadero. \bibleverse{38} Yo los envío a ustedes a
cosechar aquello que no sembraron. Otros hicieron la obra, y ustedes han
segado ahora los beneficios de lo que ellos hicieron''.

\bibleverse{39} Muchos samaritanos de aquella ciudad creyeron en él
porque la mujer dijo ``Él me dijo todo lo que yo he hecho''.
\bibleverse{40} Así que cuando vinieron a verlo, le suplicaron que se
quedara con ellos. Él permaneció allí por dos días, \bibleverse{41} y
por lo que él les dijo, muchos creyeron en él. \bibleverse{42} Ellos le
dijeron a la mujer: ``Ahora nuestra confianza en él no es por lo que tú
nos dijiste sino porque nosotros mismos lo hemos oído. Estamos
convencidos de que él es realmente el Salvador del mundo''. \footnote{\textbf{4:42}
  Hech 8,5-8}

\hypertarget{curaciuxf3n-del-hijo-de-un-funcionario-real-en-cafarnauxfam}{%
\subsection{Curación del hijo de un funcionario real en
Cafarnaúm}\label{curaciuxf3n-del-hijo-de-un-funcionario-real-en-cafarnauxfam}}

\bibleverse{43} Después de dos días, siguió camino a Galilea.
\footnote{\textbf{4:43} Mat 4,12} \bibleverse{44} Jesús mismo había
hecho el comentario de que un profeta no es respetado en su propia
tierra. \footnote{\textbf{4:44} Mat 13,57} \bibleverse{45} Pero cuando
llegó a Galilea, el pueblo lo recibió porque ellos también habían estado
en la fiesta de la Pascua y habían visto todo lo que él había hecho en
Jerusalén. \footnote{\textbf{4:45} Juan 2,23} \bibleverse{46} Él visitó
nuevamente Caná de Galilea, donde había convertido el agua en vino.
Cerca, en la ciudad de Capernaúm, vivía un oficial del rey cuyo hijo
estaba muy enfermo. \footnote{\textbf{4:46} Juan 2,1; Juan 2,9}
\bibleverse{47} Cuando él escuchó que Jesús había regresado de Judea a
Galilea, fue a Jesús y le rogó que viniese y sanase a su hijo que estaba
a punto de morir. \bibleverse{48} ``A menos que vean señales y milagros,
ustedes no creerán realmente en mi'', dijo Jesús.

\bibleverse{49} ``Señor, solo ven antes de que mi hijo muera'', suplicó
el oficial.

\bibleverse{50} ``Ve a casa'', le dijo Jesús. ``¡Tu hijo vivirá!'' El
hombre creyó lo que Jesús le dijo y se fue a casa. \bibleverse{51}
Mientras aún iba de camino, sus siervos salieron a su encuentro, y al
verlo, le dijeron la noticia de que su hijo estaba vivo y recuperándose.
\bibleverse{52} Él les preguntó a qué hora había comenzado a mejorar su
hijo. ``Ayer a la una de la tarde dejó de tener fiebre'', le dijeron.
\bibleverse{53} Entonces el padre se dio cuenta de que esa era la hora
precisa en la que Jesús le había dicho ``¡Tu hijo vivirá!'' Entonces él
y todos en su casa creyeron en Jesús. \bibleverse{54} Este fue el
segundo milagro que Jesús hizo después de regresar de Judea a
Galilea.\footnote{\textbf{4:54} Juan 2,11}

\hypertarget{sanaciuxf3n-de-los-enfermos-en-el-estanque-de-betesda-cerca-de-jerusaluxe9n-y-concurso-del-suxe1bado}{%
\subsection{Sanación de los enfermos en el estanque de Betesda cerca de
Jerusalén y concurso del
sábado}\label{sanaciuxf3n-de-los-enfermos-en-el-estanque-de-betesda-cerca-de-jerusaluxe9n-y-concurso-del-suxe1bado}}

\hypertarget{section-4}{%
\section{5}\label{section-4}}

\bibleverse{1} Después de esto, hubo una celebración judía, así que
Jesús fue a Jerusalén. \footnote{\textbf{5:1} Juan 2,13} \bibleverse{2}
Resulta que junto a la Puerta de las Ovejas, en Jerusalén, hay un
estanque llamado ``Betesda'' en hebreo, con cinco pórticos a los lados.
\footnote{\textbf{5:2} Neh 3,1} \bibleverse{3} Multitudes de personas
enfermas yacían en estos pórticos: ---ciegos, cojos, y paralíticos.
\bibleverse{4} \footnote{\textbf{5:4} 5:3b, 4. Estos versículos no están
  en los primeros manuscritos y parecen haber sido añadidos para
  explicar el versículo 7. Fueron añadidos con fines informativos:
  ``Allí ellos esperaban que el agua se moviera, porque un ángel del
  Señor venía de vez en cuando al estanque y agitaba el agua. Aquél que
  primero entrara al agua, después de haber sido agitada, era sanado de
  cualquier enfermedad que tuviera''. Parece que esto era lo que algunos
  creían en ese tiempo.} \bibleverse{5} Un hombre que estaba allí, había
estado enfermo durante treinta y ocho años. Jesús lo miró, sabiendo que
había estado allí por mucho tiempo, y le preguntó: \bibleverse{6}
``¿Quieres ser sanado?''

\bibleverse{7} ``Señor'', respondió el hombre enfermo, ``No tengo a
nadie que me ayude a entrar al estanque cuando el agua es agitada.
Mientras trato de llegar allí, alguien llega primero que yo.''

\bibleverse{8} ``¡Levántate, toma tu camilla y comienza a caminar!'' le
dijo Jesús.

\bibleverse{9} De inmediato el hombre fue sanado. Recogió su camilla y
comenzó a caminar. Aconteció que el día que ocurrió esto era sábado.

\bibleverse{10} Así que los judíos le dijeron al hombre que había sido
sanado: ``¡Es Sábado! ¡Es contra la ley cargar una camilla!''

\bibleverse{11} Pero él respondió: ``El hombre que me sanó me dijo que
recogiera mi camilla y comenzara a caminar''.

\bibleverse{12} ``¿Quién es esta persona que te dijo que cargaras tu
camilla y caminaras?'' preguntaron ellos.

\bibleverse{13} Sin embargo, el hombre que había sido sanado no sabía
quién era, pues Jesús había desaparecido entre la multitud que le
rodeaba.

\bibleverse{14} Después de esto, Jesús encontró al hombre en el Templo,
y le dijo: ``Mira, ahora has sido sanado. Deja de pecar o podría
ocurrirte algo peor''. \footnote{\textbf{5:14} Juan 8,11}

\bibleverse{15} Entonces el hombre fue donde los judíos y les dijo que
había sido Jesús quien lo había sanado. \bibleverse{16} Entonces los
judíos comenzaron a perseguir a Jesús porque él estaba haciendo estas
cosas el día sábado. \bibleverse{17} Pero Jesús les dijo: ``Mi Padre aún
trabaja, y yo también''.\footnote{\textbf{5:17} O, ``Mi Padre siempre
  está trabajando, y yo estoy trabajando también''.} \footnote{\textbf{5:17}
  Juan 9,4}

\bibleverse{18} Fue por esto que los judíos se esforzaron más aún en
matarlo, porque no solamente quebrantaba el Sábado sino que también
llamaba a Dios su Padre, haciéndose así semejante a Dios. \footnote{\textbf{5:18}
  Juan 7,30; Juan 10,33}

\hypertarget{el-testimonio-de-jesuxfas-de-su-obra-divina-y-de-su-filiaciuxf3n-divina-jesuxfas-como-juez-y-dador-de-vida}{%
\subsection{El testimonio de Jesús de su obra divina y de su filiación
divina; Jesús como juez y dador de
vida}\label{el-testimonio-de-jesuxfas-de-su-obra-divina-y-de-su-filiaciuxf3n-divina-jesuxfas-como-juez-y-dador-de-vida}}

\bibleverse{19} Jesús les explicó: ``Les digo la verdad, el Hijo no
puede hacer nada por su propia cuenta; él solo puede hacer lo que ve
hacer al Padre. Todo lo que el Padre hace, lo hace también el Hijo.
\footnote{\textbf{5:19} Juan 3,11; Juan 3,32} \bibleverse{20} Porque el
Padre ama al Hijo y le revela todo lo que hace; y el Padre le mostrará
incluso cosas más increíbles que van a dejarlos asombrados a ustedes por
completo. \footnote{\textbf{5:20} Juan 3,35} \bibleverse{21} Porque así
como el Padre da vida a los que resucita de la muerte, del mismo modo el
Hijo también da vida a los que Él quiere. \bibleverse{22} El padre no
juzga a nadie. Él le ha dado toda la autoridad al Hijo para juzgar,
\footnote{\textbf{5:22} Dan 7,12; Dan 7,14; Hech 10,42} \bibleverse{23}
a fin de que todos puedan honrar al Hijo así como honran al Padre.
Cualquiera que no honra al Hijo, no honra al Padre que lo envió.
\footnote{\textbf{5:23} Fil 2,10-11; 1Jn 2,23}

\bibleverse{24} Les digo la verdad: aquellos que siguen\footnote{\textbf{5:24}
  Literalmente, ``escuchan''.} lo que yo digo y creen en Aquél que me
envió, tienen vida eterna. Ellos no serán condenados, sino que habrán
pasado de la muerte a la vida. \footnote{\textbf{5:24} Juan 3,16; Juan
  3,18} \bibleverse{25} ``Les digo la verdad: Se acerca el tempo---de
hecho, ya está aquí---cuando los muertos escucharán la voz del Hijo de
Dios; y los que le escuchen, vivirán. \footnote{\textbf{5:25} Efes 2,5-6}
\bibleverse{26} Así como el Padre tiene en sí mismo el poder de dar
vida, así también le ha dado al Hijo el poder de dar vida. \footnote{\textbf{5:26}
  Juan 1,1-4} \bibleverse{27} El Padre también le otorgó al Hijo la
autoridad de juzgar, porque él es el Hijo del hombre. \footnote{\textbf{5:27}
  Dan 7,13-14} \bibleverse{28} No se sorprendan de esto, porque viene el
tiempo en que todos los que estén en el sepulcro escucharán su voz
\bibleverse{29} y se levantarán de nuevo. Aquellos que han hecho bien,
resucitarán para vida; y los que han hecho mal, resucitarán para
condenación.\footnote{\textbf{5:29} Ver Daniel 12:2.} \footnote{\textbf{5:29}
  Dan 12,2; Mat 25,46; 2Cor 5,10} \bibleverse{30} Yo no puedo hacer nada
por mi propia cuenta. Juzgo basándome en lo que se me dice,\footnote{\textbf{5:30}
  De manera implícita: ``lo que me dice Dios el Padre''.} y mi decisión
es justa, porque no estoy haciendo mi propia voluntad sino la voluntad
de Aquél que me envió. \footnote{\textbf{5:30} Juan 6,38}

\hypertarget{el-testimonio-de-juan}{%
\subsection{El testimonio de Juan}\label{el-testimonio-de-juan}}

\bibleverse{31} Si yo quisiera atribuirme alguna gloria para mí mismo,
esas atribuciones no serían válidas; \bibleverse{32} pero hay alguien
más que da evidencia acerca de mí, y yo sé que lo que él dice de mí es
verdad. \bibleverse{33} Ustedes le preguntaron a Juan sobre mí y él dijo
la verdad, \footnote{\textbf{5:33} Juan 1,19-34} \bibleverse{34} pero yo
no necesito ninguna aprobación humana. Estoy explicándoles esto para que
sean salvos. \bibleverse{35} Juan fue como una lámpara resplandeciente,
y ustedes estuvieron dispuestos a disfrutar de su luz por un tiempo.

\hypertarget{el-testimonio-del-padre}{%
\subsection{El testimonio del padre}\label{el-testimonio-del-padre}}

\bibleverse{36} Pero la evidencia que les estoy dando es más grande que
la de Juan. Porque yo estoy haciendo el trabajo que mi Padre me dio para
que hiciera, \bibleverse{37} y esta es la evidencia de que el Padre me
envió. El Padre que me envió, Él mismo habla en mi favor. Ustedes nunca
han escuchado su voz y nunca han visto cómo es Él, \footnote{\textbf{5:37}
  Mat 3,17} \bibleverse{38} y no aceptan lo que Él dice, porque no
confían en el que envió.

\bibleverse{39} ``Ustedes examinan las Escrituras porque piensan que a
través de ellas obtendrán la vida eterna. ¡Pero la evidencia que ellas
dan está a mi favor! \bibleverse{40} Y sin embargo, ustedes no quieren
venir a mí para que tengan vida.

\hypertarget{ataque-a-la-incredulidad-y-ambiciuxf3n-de-los-juduxedos-testimonio-de-moisuxe9s}{%
\subsection{Ataque a la incredulidad y ambición de los judíos;
Testimonio de
moisés}\label{ataque-a-la-incredulidad-y-ambiciuxf3n-de-los-juduxedos-testimonio-de-moisuxe9s}}

\bibleverse{41} Yo no estoy buscando aprobación humana \bibleverse{42}
---Yo los conozco, y sé que no tienen el amor de Dios en ustedes.
\bibleverse{43} Pues yo he venido a representar\footnote{\textbf{5:43}
  Literalmente, ``en nombre de''} a mi Padre, y ustedes no me aceptarán;
¡pero si alguno viene representándose a sí mismo, entonces ustedes lo
aceptan! \footnote{\textbf{5:43} Mat 24,5} \bibleverse{44} ¿Cómo pueden
creer en mí si buscan alabanza entre los unos y los otros y no la
alabanza del único Dios verdadero? \footnote{\textbf{5:44} Juan
  12,42-43; 1Tes 2,6}

\bibleverse{45} Pero no crean que yo estaré haciendo acusaciones sobre
ustedes ante el Padre. Es Moisés quien los acusa, el mismo en quien
ustedes han puesto tal confianza. \footnote{\textbf{5:45} Deut 31,26-27}
\bibleverse{46} Pues si ustedes realmente creyeran en Moisés, creerían
en mí, porque él escribió acerca de mí. \footnote{\textbf{5:46} Gén
  3,15; Gén 49,10; Deut 18,15} \bibleverse{47} Pero como ustedes no
creen en lo que él dijo, ¿porqué confiarían en lo que yo
digo?''\footnote{\textbf{5:47} Luc 16,31}

\hypertarget{jesuxfas-alimenta-a-los-cinco-mil}{%
\subsection{Jesús alimenta a los cinco
mil}\label{jesuxfas-alimenta-a-los-cinco-mil}}

\hypertarget{section-5}{%
\section{6}\label{section-5}}

\bibleverse{1} Después de esto, Jesús se marchó al otro lado del Mar de
Galilea (conocido también como el Mar de Tiberias). \bibleverse{2} Una
gran multitud le seguía, porque habían visto sus milagros de sanación.
\bibleverse{3} Jesús subió a una colina y se sentó allí con sus
discípulos. \bibleverse{4} Se acercaba la fecha de la fiesta judía de la
Pascua. \bibleverse{5} Cuando Jesús levantó la vista y vio una gran
multitud que venía hacia él, le preguntó a Felipe: ``¿Dónde podremos
conseguir suficiente pan para alimentar a todas estas personas?''
\bibleverse{6} Pero Jesús preguntaba solamente para ver cómo respondía
Felipe, porque él ya sabía lo que iba a hacer.

\bibleverse{7} ``Doscientas monedas de plata\footnote{\textbf{6:7}
  Literalmente, denario. Un denario equivalía al salario de un día.} no
alcanzarían para comprar suficiente pan y darle a todos aunque fuera un
poco'', respondió Felipe.

\bibleverse{8} Uno de sus discípulos, Andrés, hermano de Simón Pedro,
dijo en voz alta: \bibleverse{9} ``Hay un niño aquí que tiene cinco
panes de cebada y un par de peces, pero ¿de qué sirve eso si hay tantas
personas?''

\bibleverse{10} ``Pidan a todos que se sienten'', dijo Jesús. Allí había
mucha hierba, así que todos se sentaron, y los hombres que estaban allí
sumaban como cinco mil. \bibleverse{11} Jesús tomó el pan, dio gracias,
y lo repartió entre las personas que estaban ahí sentadas. Luego hizo lo
mismo con los peces, asegurándose de que todos recibieran tanto como
querían. \bibleverse{12} Cuando todos estuvieron saciados, dijo a sus
discípulos: ``Recojan lo que sobró para que nada se desperdicie''.
\bibleverse{13} Entonces ellos recogieron todo y llenaron doce canastas
con los trozos de los cinco panes que las personas habían comido.
\bibleverse{14} Cuando la gente vio este milagro, dijeron: ``De verdad
este es el profeta que iba a venir al mundo''. \footnote{\textbf{6:14}
  Deut 18,15} \bibleverse{15} Jesús se dio cuenta de que ellos estaban a
punto de obligarlo a convertirse en su rey, así que se fue de allí y
subió a la montaña para estar solo. \footnote{\textbf{6:15} Juan 18,36}

\hypertarget{jesuxfas-camina-sobre-el-lago}{%
\subsection{Jesús camina sobre el
lago}\label{jesuxfas-camina-sobre-el-lago}}

\bibleverse{16} Cuando llegó la tarde, sus discípulos descendieron al
mar, \bibleverse{17} se subieron a una barca, y comenzaron a cruzar
rumbo a Capernaúm. Para ese momento, ya era de noche y Jesús no los
había alcanzado. \bibleverse{18} Comenzó a soplar un fuerte viento y el
mar se enfureció. \bibleverse{19} Cuando habían remado tres o cuatro
millas, vieron a Jesús caminando sobre el mar, dirigiéndose hacia la
barca. Estaban muy asustados. \bibleverse{20} ``¡No tengan miedo!'' les
dijo. ``Soy yo''. \bibleverse{21} Entonces ellos se alegraron en
recibirlo en la barca e inmediatamente llegaron a la orilla hacia la
cual se dirigían.

\hypertarget{el-reencuentro-con-el-pueblo-y-la-demanda-de-seuxf1al-del-pueblo}{%
\subsection{El reencuentro con el pueblo y la demanda de señal del
pueblo}\label{el-reencuentro-con-el-pueblo-y-la-demanda-de-seuxf1al-del-pueblo}}

\bibleverse{22} Al día siguiente, la multitud que se había quedado al
otro lado del mar se dio cuenta de que quedaba solamente una barca allí
y que Jesús no había subido a la barca con sus discípulos, sino que
ellos se habían marchado sin él. \bibleverse{23} Luego llegaron desde
Tiberias otras barcas, cerca del lugar donde ellos habían comido el pan
después de que el Señor lo bendijo. \bibleverse{24} Cuando la multitud
se dio cuenta que ni Jesús ni sus discípulos estaban ahí, se subieron a
las barcas y se fueron a Capernaúm en busca de Jesús. \bibleverse{25}
Cuando lo encontraron al otro lado del mar, le preguntaron, ``Maestro,
¿cuándo llegaste acá?''\footnote{\textbf{6:25} Una pregunta indirecta
  pues ellos en realidad se preguntaban era cómo había llegado
  allí\ldots{}}

\bibleverse{26} ``Les digo la verdad'', respondió Jesús, ``ustedes me
buscan porque comieron todo el pan que quisieron, no porque hayan
entendido los milagros. \bibleverse{27} No se preocupen por la comida
que perece, sino concéntrense en la comida que permanece, la de la vida
eterna, la cual les dará el Hijo del hombre, porque Dios el Padre ha
colocado su sello de aprobación en él''. \footnote{\textbf{6:27} Juan
  5,36}

\bibleverse{28} Entonces ellos le preguntaron: ``¿Qué tenemos que hacer
para hacer la voluntad de Dios?''

\bibleverse{29} Jesús respondió: ``Lo que Dios quiere que hagan es que
crean en aquél a quien Él envió''.

\bibleverse{30} ``¿Qué milagro harás para que lo veamos y podamos
creerte? ¿Qué puedes hacer?'' le preguntaron. \bibleverse{31} ``Nuestros
padres comieron maná en el desierto en cumplimiento de la Escritura que
dice: `Él les dio a comer pan del cielo'\,''\footnote{\textbf{6:31}
  Citando Salmos 78:24 refiriéndose a Éxodo 16:4.} .

\hypertarget{el-discurso-de-jesuxfas-sobre-el-pan-de-vida}{%
\subsection{El discurso de Jesús sobre el pan de
vida}\label{el-discurso-de-jesuxfas-sobre-el-pan-de-vida}}

\bibleverse{32} ``Les diré la verdad: No fue Moisés quien les dio pan
del cielo'', respondió Jesús. ``Es mi Padre quien les da el verdadero
pan del cielo. \bibleverse{33} Porque el pan de Dios es el que viene del
cielo y el que da vida al mundo''.

\bibleverse{34} ``¡Señor, por favor danos de ese pan todo el tiempo!''
dijeron.

\bibleverse{35} ``Yo soy el pan de vida'', respondió Jesús. ``Cualquiera
que viene a mí nunca más tendrá hambre, y cualquiera que cree en mí
nunca más tendrá sed. \footnote{\textbf{6:35} Juan 4,14; Juan 7,37}
\bibleverse{36} Pero como ya les expliqué antes, ustedes me han
visto,\footnote{\textbf{6:36} Refiriéndose a todo lo que Jesús había
  hecho, no solo verlo en persona. De hecho, la palabra ``a mí'' no se
  encuentra en los manuscritos antiguos.} pero aún no creen en mí.
\bibleverse{37} Todos los que el Padre me entrega, vendrán a mí, y yo no
rechazaré a ninguno de ellos. \bibleverse{38} Porque yo no descendí del
cielo para hacer mi voluntad sino la voluntad del que me envió.
\footnote{\textbf{6:38} Juan 4,34} \bibleverse{39} Lo que Él quiere es
que yo no deje perder a ninguno de los que me ha dado, sino que los
levante en el día final.\footnote{\textbf{6:39} ``Último día'',
  refiriéndose al día del juicio. También aparece en los versículos 40,
  44, y 54.} \footnote{\textbf{6:39} Juan 10,28-29; Juan 17,12}
\bibleverse{40} Lo que mi Padre quiere es que cualquiera que vea al Hijo
y crea en Él tenga vida eterna, y yo lo levantaré en el día final''.
\footnote{\textbf{6:40} Juan 5,29; Juan 11,24}

\bibleverse{41} Entonces los judíos comenzaron a murmurar acerca de él
porque había dicho ``yo soy el pan que descendió del cielo''.
\bibleverse{42} Ellos dijeron: ``¿No es este Jesús, el hijo de José?
Nosotros conocemos a su padre y a su madre. ¿Cómo es que ahora puede
decirnos `yo descendí del cielo'?''

\bibleverse{43} ``Dejen de murmurar unos con otros'', dijo Jesús.
\bibleverse{44} ``Ninguno viene a mí a menos que lo atraiga el Padre que
me envió, y yo lo levantaré en el día final. \bibleverse{45} Tal como
está escrito por los profetas en las Escrituras: `Todos serán instruidos
por Dios'.\footnote{\textbf{6:45} Citando Isaías 54:13.} Todo aquél que
escucha y aprende del Padre, viene a mí. \bibleverse{46} Ninguno ha
visto a Dios, excepto el que es de Dios. Ese ha visto al Padre.
\footnote{\textbf{6:46} Juan 1,18} \bibleverse{47} Les diré la verdad:
Cualquiera que cree en Él tiene vida eterna. \footnote{\textbf{6:47}
  Juan 3,16} \bibleverse{48} Yo soy el pan de vida. \footnote{\textbf{6:48}
  Juan 6,35} \bibleverse{49} Sus padres comieron maná en el desierto y
aun así murieron. \footnote{\textbf{6:49} 1Cor 10,3-5} \bibleverse{50}
Pero este es el pan que viene del cielo, y cualquiera que lo coma no
morirá jamás. \bibleverse{51} Yo soy el pan vivo que bajó del cielo, y
cualquiera que coma de este pan, vivirá para siempre. Este pan es mi
carne, la cual daré para que el mundo viva''.

\bibleverse{52} Entonces los judíos comenzaron a pelear acaloradamente
entre ellos. ``¿Cómo puede este hombre darnos a comer su carne?''
preguntaban.

\bibleverse{53} Jesús les dijo: ``Les diré la verdad, a menos que coman
la carne del Hijo del hombre y beban su sangre, no podrán vivir
realmente. \bibleverse{54} Aquellos que comen mi carne y beben mi
sangre, tienen vida eterna y yo los levantaré en el día final.
\footnote{\textbf{6:54} Mat 26,26-28} \bibleverse{55} Porque mi carne es
verdadera comida, y mi sangre es verdadera bebida. \bibleverse{56}
Aquellos que comen mi carne y beben mi sangre permanecen en mí y yo en
ellos. \bibleverse{57} Tal como me envió el Padre viviente y yo vivo por
el Padre, de igual modo, todo aquel que se alimenta de mi vivirá por mí.
\bibleverse{58} Este es el pan que descendió del cielo, no el que
comieron sus padres y murieron. Cualquiera que come de este pan vivirá
para siempre''. \bibleverse{59} Jesús explicó esto mientras enseñaba en
una sinagoga en Capernaúm.

\hypertarget{el-divorcio-de-los-discuxedpulos-de-jesuxfas-como-efecto-del-habla}{%
\subsection{El divorcio de los discípulos de Jesús como efecto del
habla}\label{el-divorcio-de-los-discuxedpulos-de-jesuxfas-como-efecto-del-habla}}

\bibleverse{60} Muchos de sus discípulos cuando lo escucharon dijeron:
``¡Esto es algo difícil de aceptar! ¿Quién puede consentir\footnote{\textbf{6:60}
  ``consentir'' no solo en el sentido de ``entender'', sino también de
  ``observar'' o ``estar de acuerdo''.} con esto?''

\bibleverse{61} Jesús vio que sus discípulos estaban murmurando sobre
esto, así que les preguntó: ``¿Están ofendidos por esto? \bibleverse{62}
¿Qué tal si tuvieran que ver al Hijo del hombre ascender a donde estaba
antes? \footnote{\textbf{6:62} Luc 24,50-51} \bibleverse{63} El Espíritu
da vida; el cuerpo físico no sirve para nada.\footnote{\textbf{6:63} O
  ``no vale nada''.} ¡Las palabras que les he dicho son Espíritu y son
vida! \footnote{\textbf{6:63} 2Cor 3,6} \bibleverse{64} Sin embargo, hay
algunos entre ustedes que no creen en mí''. (Jesús sabía, desde el mismo
comienzo, quién creía en él y quién lo traicionaría). \bibleverse{65}
Jesús añadió: ``Esta es la razón por la que les dije que nadie puede
venir a mí a menos que le sea posible\footnote{\textbf{6:65} O
  ``concedido''.} por parte del Padre''.

\bibleverse{66} A partir de ese momento, muchos de los discípulos de
Jesús le dieron la espalda y ya no le seguían. \bibleverse{67} Entonces
Jesús le preguntó a los doce discípulos: ``¿Y ustedes? ¿Se irán
también?''

\bibleverse{68} Simón Pedro respondió, ``Señor, ¿a quién seguiremos? Tú
eres el único que tiene palabras de vida eterna. \bibleverse{69}
Nosotros creemos en ti y estamos convencidos de que eres el Santo de
Dios''. \footnote{\textbf{6:69} Mat 16,16}

\bibleverse{70} Jesús respondió: ``¿Acaso no los escogí yo a ustedes,
los doce discípulos? Sin embargo, uno de ustedes es un demonio'',
\bibleverse{71} (Jesús se estaba refiriendo a Judas, hijo de Simón
Iscariote. Él era el discípulo que traicionaría a Jesús).

\hypertarget{jesuxfas-viaja-a-jerusaluxe9n-para-la-fiesta-de-los-tabernuxe1culos}{%
\subsection{Jesús viaja a Jerusalén para la Fiesta de los
Tabernáculos}\label{jesuxfas-viaja-a-jerusaluxe9n-para-la-fiesta-de-los-tabernuxe1culos}}

\hypertarget{section-6}{%
\section{7}\label{section-6}}

\bibleverse{1} Después de esto, Jesús se dedicó a ir de un lugar a otro,
por toda Galilea. Él no quería hacer lo mismo en Judea porque los judíos
intentaban matarlo. \bibleverse{2} Pero como ya casi era la fecha de la
fiesta judía de los Tabernáculos, \footnote{\textbf{7:2} Lev 23,34-36}
\bibleverse{3} sus hermanos le dijeron: ``Debes marcharte a Judea para
que tus seguidores puedan ver los milagros que puedes hacer. \footnote{\textbf{7:3}
  Juan 2,12; Mat 12,46; Hech 1,14} \bibleverse{4} Ninguno que quiera ser
famoso mantiene ocultas las cosas que hace. Si puedes hacer tales
milagros, ¡entonces muéstrate al mundo!'' \bibleverse{5} Porque incluso
sus propios hermanos no creían realmente en él.

\bibleverse{6} Jesús les dijo: ``Este no es mi momento de irme. No
todavía. Pero ustedes pueden irse cuando quieran, porque para ustedes
cualquier momento es correcto. \footnote{\textbf{7:6} Juan 2,4}
\bibleverse{7} El mundo no tiene razones para odiarlos a ustedes, pero
me odia a mí porque yo dejo claro que sus caminos son malvados.
\footnote{\textbf{7:7} Juan 15,18} \bibleverse{8} Váyanse ustedes a la
fiesta. Yo no iré a esta fiesta porque no es mi momento de ir, no aún''.

\bibleverse{9} Después de decir esto, se quedó en Galilea.
\bibleverse{10} Después que sus hermanos se marcharon para ir a la
fiesta, Jesús también fue, pero no abiertamente, sino que se mantuvo
oculto. \footnote{\textbf{7:10} Juan 2,13} \bibleverse{11} Ahora, los
líderes judíos en la fiesta estaban buscándolo y no dejaban de preguntar
``¿Dónde está Jesús?'' \bibleverse{12} Muchas personas entre la multitud
se quejaban de él. Algunos decían: ``Él es un buen hombre'', mientras
que otros discutían: ``¡No, Él engaña a la gente!'' \bibleverse{13} Pero
ninguno se atrevía a hablar abiertamente acerca de él porque tenían
miedo de lo que los líderes judíos pudieran hacerles.

\hypertarget{la-apariciuxf3n-y-el-testimonio-de-suxed-mismo-de-jesuxfas-en-la-fiesta-de-los-tabernuxe1culos}{%
\subsection{La aparición y el testimonio de sí mismo de Jesús en la
Fiesta de los
Tabernáculos}\label{la-apariciuxf3n-y-el-testimonio-de-suxed-mismo-de-jesuxfas-en-la-fiesta-de-los-tabernuxe1culos}}

\bibleverse{14} Durante la mitad de la fiesta, Jesús fue al Templo y
comenzó a enseñar. \bibleverse{15} Los líderes judíos estaban muy
sorprendidos y preguntaban: ``¿Cómo es que este hombre tiene tanto
conocimiento\footnote{\textbf{7:15} En el sentido de una educación
  religiosa.} si él no ha sido educado?'' \footnote{\textbf{7:15} Mat
  13,56}

\bibleverse{16} Jesús respondió: ``Mi enseñanza no viene de mí, sino de
Aquél que me envió. \bibleverse{17} Cualquiera que escoge seguir la
voluntad de Dios, sabrá si mi enseñanza viene de Dios o si solamente
hablo por mí mismo. \bibleverse{18} Aquellos que hablan por sí mismos
quieren glorificarse a sí mismos, pero aquél que glorifica al que lo
envió es veraz y no engañoso. \bibleverse{19} Moisés les dio a ustedes
la ley, ¿no es así? Sin embargo, ¡ninguno de ustedes guarda la ley! ¿Por
qué están tratando de matarme?'' \footnote{\textbf{7:19} Juan 5,16; Juan
  5,18; Rom 2,17-24}

\bibleverse{20} ``¡Estás poseído por el demonio!'' respondió la
multitud. ``¡Ninguno está tratando de matarte!'' \footnote{\textbf{7:20}
  Juan 10,20}

\bibleverse{21} ``Hice un milagro\footnote{\textbf{7:21} En Sábado,
  refiriéndose a lo que había ocurrido según el texto 5:1-9.} y todos
ustedes están escandalizados por ello'', respondió Jesús. \footnote{\textbf{7:21}
  Juan 5,16} \bibleverse{22} ``Sin embargo, como Moisés les dijo que se
circuncidaran---no porque esta enseñanza viniera realmente de Moisés,
sino de sus padres que estuvieron mucho antes que él---por eso ustedes
hacen la circuncisión en Sábado. \footnote{\textbf{7:22} Gén 17,10-12;
  Lev 12,3} \bibleverse{23} Si ustedes se circuncidan en sábado para
asegurarse de que la ley de Moisés se guarda, ¿por qué están enojados
conmigo por sanar a alguien en sábado? \bibleverse{24} ¡No juzguen por
las apariencias! ¡Decidan lo que es justo!''

\hypertarget{jesuxfas-viene-de-dios}{%
\subsection{Jesús viene de Dios}\label{jesuxfas-viene-de-dios}}

\bibleverse{25} Entonces algunos de los que venían desde Jerusalén
comenzaron a preguntarse: ``¿No es este al que estamos intentando matar?
\bibleverse{26} Pero miren cómo habla abiertamente y no le dicen nada.
¿Creen ustedes que las autoridades creen que él es el Mesías?
\bibleverse{27} Pero eso no es posible porque nosotros sabemos de dónde
viene. Cuando el Mesías venga, nadie sabrá de dónde viene''. \footnote{\textbf{7:27}
  Heb 7,3}

\bibleverse{28} Mientras enseñaba en el Templo, Jesús dijo en voz alta:
``¿Entonces ustedes piensan que me conocen y que saben de dónde vengo?
Sin embargo, yo no vine por mi propio beneficio. El que me envió es
verdadero. Ustedes no lo conocen, \bibleverse{29} pero yo lo conozco,
porque yo vengo de él, y él me ha enviado''.

\bibleverse{30} Entonces ellos trataron de arrestarlo, pero ninguno puso
una sola mano sobre él porque su tiempo aún no había llegado.
\footnote{\textbf{7:30} Juan 8,20; Luc 22,53} \bibleverse{31} Sin
embargo, muchos de la multitud creyeron en él. ``Cuando el Mesías
aparezca, ¿hará acaso más milagros que los que este hombre ha hecho?''
decían. \bibleverse{32} Cuando los Fariseos escucharon a la multitud
murmurar esto acerca de él, ellos y los jefes de los sacerdotes enviaron
guardias para arrestarle.

\hypertarget{jesuxfas-anuncia-su-regressa-a-dios}{%
\subsection{Jesús anuncia su regressa a
Dios}\label{jesuxfas-anuncia-su-regressa-a-dios}}

\bibleverse{33} Entonces Jesús le dijo a la gente: ``Estaré con ustedes
solo un poco más, pero luego regresaré a Aquél que me envió.
\bibleverse{34} Ustedes me buscarán pero no me encontrarán; y adonde yo
voy, ustedes no pueden ir''. \footnote{\textbf{7:34} Juan 8,21}

\bibleverse{35} Los judíos se decían unos a otros: ``¿A dónde irá que no
podremos encontrarlo? ¿Acaso está planeando irse donde están las
personas dispersas entre los extranjeros\footnote{\textbf{7:35}
  Literalmente, ``Los griegos''.} , y les enseñará a ellos?
\bibleverse{36} ¿Qué quiere decir con `me buscarán pero no me
encontrarán', y `adonde yo voy ustedes no pueden ir'?''

\hypertarget{jesuxfas-en-el-apogeo-de-la-fiesta-como-dador-del-agua-de-vida}{%
\subsection{Jesús en el apogeo de la fiesta como dador del agua de
vida}\label{jesuxfas-en-el-apogeo-de-la-fiesta-como-dador-del-agua-de-vida}}

\bibleverse{37} El último día y el más importante de la fiesta, Jesús se
puso en pie y dijo a gran voz: ``Si están sedientos, vengan a mí y
beban. \bibleverse{38} Si creen en mí, de ustedes fluirán ríos de agua
viva, como dice la Escritura''.\footnote{\textbf{7:38} La referencia más
  cercana parece ser Cantar de los Cantares 4:15.} \footnote{\textbf{7:38}
  Is 58,11} \bibleverse{39} Él se refería al Espíritu que recibirían
aquellos que creyeran en él. El Espíritu aún no se había enviado porque
todavía Jesús no había sido glorificado. \footnote{\textbf{7:39} Juan
  16,7}

\bibleverse{40} Cuando ellos escucharon estas palabras, algunas personas
dijeron: ``¡Este hombre es definitivamente el Profeta\footnote{\textbf{7:40}
  Ver 6:14.} !'' \footnote{\textbf{7:40} Juan 6,14} \bibleverse{41}
Otros decían: ``¡Él es el Mesías!'' Y otros también decían: ``¿Cómo
puede el Mesías venir de Galilea? \footnote{\textbf{7:41} Juan 1,46}
\bibleverse{42} ¿Acaso no dice la Escritura que el Mesías viene del
linaje de David y de la casa de David en Belén?''\footnote{\textbf{7:42}
  Refiriéndose a Miqueas 5:2.} \footnote{\textbf{7:42} Miq 5,1; Mat
  2,5-6; Mat 22,42} \bibleverse{43} Entonces había entre la multitud
grandes diferencias de opiniones acerca de él. \footnote{\textbf{7:43}
  Juan 9,16} \bibleverse{44} Algunos querían arrestarlo, pero nadie puso
una sola mano sobre él.

\hypertarget{fracaso-del-plan-de-arresto-de-los-luxedderes-divisiuxf3n-entre-los-miembros-del-sumo-consejo-amonestaciuxf3n-de-nicodemo}{%
\subsection{Fracaso del plan de arresto de los líderes; División entre
los miembros del sumo consejo; Amonestación de
Nicodemo}\label{fracaso-del-plan-de-arresto-de-los-luxedderes-divisiuxf3n-entre-los-miembros-del-sumo-consejo-amonestaciuxf3n-de-nicodemo}}

\bibleverse{45} Entonces los guardias regresaron a los jefes de los
sacerdotes y a los Fariseos, quienes les preguntaron: ``¿Por qué no lo
trajeron?''

\bibleverse{46} ``Nadie nunca habló como habla este hombre'',
respondieron los guardias. \footnote{\textbf{7:46} Mat 7,28-29}

\bibleverse{47} ``¿Acaso los ha engañado a ustedes también?'' les
preguntaron los Fariseos. \bibleverse{48} ``¿Acaso alguno de los
gobernantes o Fariseos ha creído en él? ¡No! \bibleverse{49} Pero ésta
multitud de gente que no conoce nada acerca de las enseñanzas de la
ley--- ¡están todos condenados de cualquier modo!''

\bibleverse{50} Nicodemo, quien había ido a encontrarse con Jesús
anteriormente, era uno de ellos y les preguntó: \bibleverse{51} ``¿Acaso
nuestra ley condena a un hombre sin escucharlo y sin saber lo que
realmente ha hecho?'' \footnote{\textbf{7:51} Deut 1,16-17}

\bibleverse{52} ``¿De modo que eres un galileo también?'' respondieron
ellos. ``¡Revisa las Escrituras y descubrirás que ningún profeta viene
de Galilea!''

\bibleverse{53} Entonces se fueron todos a sus casas,\footnote{\textbf{7:53}
  Los versículos 7:53-8:11 no aparecen en este lugar en los manuscritos.
  Sin embargo, representan con certeza un relato auténtico.}

\hypertarget{jesuxfas-y-la-aduxfaltera}{%
\subsection{Jesús y la adúltera}\label{jesuxfas-y-la-aduxfaltera}}

\hypertarget{section-7}{%
\section{8}\label{section-7}}

\bibleverse{1} pero Jesús fue al Monte de los Olivos.

\bibleverse{2} Temprano por la mañana, Jesús regresó al Templo donde
muchas personas se reunieron alrededor de él, y él se sentó y les
enseñaba. \bibleverse{3} Los maestros y los Fariseos le trajeron una
mujer que fue atrapada mientras cometía adulterio y la hicieron
permanecer ahí en pie, delante de todos. \bibleverse{4} Ellos le dijeron
a Jesús: ``Maestro, esta mujer fue atrapada en el acto del adulterio.
\bibleverse{5} Ahora, en la Ley, Moisés ordenó que debemos apedrear a
estas mujeres. ¿Qué dices tú?'' \bibleverse{6} Ellos decían esto para
ponerle una trampa a Jesús, a fin de condenarlo. Pero Jesús se arrodilló
y escribía en la tierra con su dedo.

\bibleverse{7} Ellos seguían exigiendo una respuesta, así que él se
levantó y les dijo: ``Cualquiera de ustedes que nunca haya pecado puede
lanzar la primera piedra sobre ella''. \footnote{\textbf{8:7} Rom 2,1}
\bibleverse{8} Entonces se arrodilló otra vez y siguió escribiendo en la
tierra.

\bibleverse{9} Cuando ellos escucharon esto, comenzaron a marcharse, uno
a uno, comenzado desde el más anciano hasta que Jesús quedó en medio de
la multitud con la mujer que aún estaba allí. \bibleverse{10} Jesús se
levantó y le preguntó: ``¿Dónde están ellos? ¿No quedó ninguno para
condenarte?''

\bibleverse{11} ``Ninguno, Señor'', respondió ella. ``Yo tampoco te
condeno'', le dijo Jesús. ``Vete y no peques más''.

\hypertarget{el-testimonio-de-suxed-mismo-de-jesuxfas-como-la-luz-del-mundo-y-el-hijo-de-dios}{%
\subsection{El testimonio de sí mismo de Jesús como la luz del mundo y
el Hijo de
Dios}\label{el-testimonio-de-suxed-mismo-de-jesuxfas-como-la-luz-del-mundo-y-el-hijo-de-dios}}

\bibleverse{12} Jesús habló una vez más al pueblo, diciéndoles: ``Yo soy
la luz del mundo. Si me siguen, no caminarán en la oscuridad, porque
tendrán la luz de la vida''. \footnote{\textbf{8:12} Is 49,6; Juan 1,5;
  Juan 1,9; Mat 5,14-16}

\bibleverse{13} Los Fariseos respondieron: ``¡Tú no puedes ser tu propio
testigo!\footnote{\textbf{8:13} O, ``¡tu solo estás haciendo alardes de
  ti mismo!''} ¡Lo que dices no prueba nada!''

\bibleverse{14} ``Incluso si yo soy mi propio testigo, mi testimonio es
verdadero'', les dijo Jesús, ``porque sé de dónde vengo y hacia dónde
voy. Pero ustedes no saben de dónde vengo ni hacia dónde voy.
\bibleverse{15} Ustedes juzgan humanamente, pero yo no juzgo a nadie.
\footnote{\textbf{8:15} Juan 3,17} \bibleverse{16} Incluso si yo
juzgara, mi juicio sería justo porque no estoy haciendo esto por mi
cuenta. El Padre que me envió está conmigo. \bibleverse{17} La misma ley
de ustedes dice\footnote{\textbf{8:17} Ver Deuteronomio 17:6 y
  Deuteronomio 19:15.} que el testimonio de dos testigos es válido.
\bibleverse{18} Yo soy mi propio testigo, y mi otro testigo es mi Padre
que me envió.

\bibleverse{19} ``¿Dónde está tu padre?'' le preguntaron. ``Ustedes no
me conocen a mí ni a mi Padre'', respondió Jesús. ``Si ustedes me
conocieran, entonces conocerían a mi Padre también''. \footnote{\textbf{8:19}
  Juan 14,7}

\bibleverse{20} Jesús explicaba esto mientras enseñaba cerca de la
tesorería del Templo. Sin embargo, nadie lo arrestó porque aún no había
llegado su tiempo. \footnote{\textbf{8:20} Juan 7,30}

\hypertarget{jesuxfas-da-testimonio-del-profundo-abismo-que-lo-separa-de-los-juduxedos-seguxfan-sus-oruxedgenes}{%
\subsection{Jesús da testimonio del profundo abismo que lo separa de los
judíos según sus
orígenes}\label{jesuxfas-da-testimonio-del-profundo-abismo-que-lo-separa-de-los-juduxedos-seguxfan-sus-oruxedgenes}}

\bibleverse{21} Jesús les dijo de nuevo: ``Yo me voy y ustedes me
buscarán, pero morirán en su pecado. Adonde yo voy, ustedes no pueden
ir''. \footnote{\textbf{8:21} Juan 7,34-35; Juan 13,33}

\bibleverse{22} Los judíos preguntaban en voz alta: ``¿Acaso va a
matarse a sí mismo? ¿Es eso a lo que se refiere cuando dice `adonde yo
voy ustedes no pueden ir'?''

\bibleverse{23} Jesús les dijo: ``Ustedes son de abajo, yo soy de
arriba. Ustedes son de este mundo; yo no soy de este mundo.
\bibleverse{24} Es por eso que les dije que ustedes morirán en sus
pecados. Porque si no creen en mí, en el `Yo soy,' morirán en sus
pecados''.

\bibleverse{25} Entonces ellos le preguntaron, ``¿Quién eres tú?'' ``Soy
exactamente quien les dije que era desde el principio'', respondió
Jesús.

\bibleverse{26} ``Hay muchas cosas que yo podría decir de ustedes, y
muchas cosas que podría condenar. Pero el que me envió dice la verdad, y
lo que yo les digo aquí en este mundo es lo que escuché de Él''.

\bibleverse{27} Ellos no entendían que él estaba hablando del Padre. Así
que Jesús les explicó: \bibleverse{28} ``Cuando ustedes hayan levantado
al Hijo del hombre sabrán entonces que yo soy el `Yo soy,' y que no hago
nada por mí mismo, sino que digo lo que el Padre me enseñó. \footnote{\textbf{8:28}
  Juan 3,14; Juan 12,32} \bibleverse{29} Aquél que me envió está
conmigo; Él no me ha abandonado, porque yo siempre hago lo que a Él le
agrada''.

\hypertarget{el-testimonio-de-jesuxfas-de-su-filiaciuxf3n-de-dios-y-de-la-esclavitud-del-pecado-de-los-juduxedos-a-pesar-de-su-descendencia-de-abraham}{%
\subsection{El testimonio de Jesús de su filiación de Dios y de la
esclavitud del pecado de los judíos a pesar de su descendencia de
Abraham}\label{el-testimonio-de-jesuxfas-de-su-filiaciuxf3n-de-dios-y-de-la-esclavitud-del-pecado-de-los-juduxedos-a-pesar-de-su-descendencia-de-abraham}}

\bibleverse{30} Muchos de los que escucharon a Jesús decir estas cosas,
creyeron en Él. \bibleverse{31} Entonces Jesús le dijo a los judíos que
creyeron en él: ``Si siguen mi enseñanza, entonces ustedes son realmente
mis discípulos. \bibleverse{32} Conocerán la verdad y la verdad los hará
libres''.

\bibleverse{33} ``¡Nosotros somos descendientes de Abraham! Nosotros
nunca hemos sido esclavos de nadie'', respondieron ellos. ``¿Cómo puedes
decir que seremos libres?'' \footnote{\textbf{8:33} Mat 3,9}

\bibleverse{34} Jesús respondió: ``Les digo la verdad, todo el que peca
es un esclavo del pecado. \bibleverse{35} Un esclavo no tiene un lugar
permanente en la familia, pero el hijo siempre es parte de la familia.
\bibleverse{36} Si el Hijo los libera, entonces ustedes son
verdaderamente libres.

\hypertarget{los-juduxedos-incruxe9dulos-no-son-hijos-de-abraham-ni-de-dios-sino-hijos-del-diablo}{%
\subsection{Los judíos incrédulos no son hijos de Abraham ni de Dios,
sino hijos del
diablo}\label{los-juduxedos-incruxe9dulos-no-son-hijos-de-abraham-ni-de-dios-sino-hijos-del-diablo}}

\bibleverse{37} Yo sé que ustedes son descendientes de Abraham. Sin
embargo, ustedes están tratando de matarme porque se niegan a aceptar
mis palabras. \bibleverse{38} Yo les estoy diciendo lo que el Padre me
ha revelado,\footnote{\textbf{8:38} O ``lo que yo he visto con el
  Padre''.} mientras que ustedes hacen lo que su padre les ha
enseñado''.

\bibleverse{39} ``Nuestro padre es Abraham'', respondieron ellos. ``Si
ustedes realmente fueran hijos de Abraham, harían lo que Abraham hizo'',
les dijo Jesús.

\bibleverse{40} ``Pero ustedes están tratando de matarme ahora, porque
les dije la verdad que yo escuché de Dios. Abraham nunca habría hecho
eso. \bibleverse{41} Ustedes están haciendo lo que hace el padre de
ustedes''. ``Pues nosotros\footnote{\textbf{8:41} En el original, esta
  palabra está enfatizada. Ellos están sugiriendo que aunque ellos no
  eran ilegítimos, Jesús sí lo era.} no somos hijos ilegítimos'',
respondieron ellos. ``¡Solo Dios es nuestro padre!''

\bibleverse{42} Jesús respondió: ``Si Dios fuese realmente el padre de
ustedes, ustedes me amarían. Yo vine de Dios y estoy aquí. No fue mi
propia decisión venir, sino la de Uno que me envió. \bibleverse{43} ¿Por
qué no pueden entender lo que estoy diciendo? ¡Es porque ustedes se
niegan a escuchar mi mensaje! \footnote{\textbf{8:43} 1Cor 2,14}
\bibleverse{44} El padre de ustedes es el Diablo, y ustedes aman seguir
los deseos malos de él. Él fue un asesino desde el principio. Nunca
estuvo de parte de la verdad, porque no hay verdad en él. Cuando él
miente, revela su verdadero carácter, porque él es un mentiroso y padre
de mentiras. \footnote{\textbf{8:44} 1Jn 3,8-10; Gén 3,4; Gén 3,19}
\bibleverse{45} ¡Entonces, como yo les digo la verdad, ustedes no me
creen! \bibleverse{46} ¿Acaso puede alguno de ustedes demostrarme que
soy culpable de pecado? Si les estoy diciendo la verdad, ¿por qué no me
creen? \footnote{\textbf{8:46} 2Cor 5,21; 1Pe 2,22; 1Jn 3,5; Heb 4,15}
\bibleverse{47} Todo el que pertenece a Dios, escucha lo que Dios dice.
La razón por la que ustedes no escuchan es porque ustedes no pertenecen
a Dios''. \footnote{\textbf{8:47} Juan 18,37}

\hypertarget{el-testimonio-de-jesuxfas-de-la-majestad-de-suxed-mismo-y-de-su-superioridad-sobre-abraham}{%
\subsection{El testimonio de Jesús de la majestad de sí mismo y de su
superioridad sobre
Abraham}\label{el-testimonio-de-jesuxfas-de-la-majestad-de-suxed-mismo-y-de-su-superioridad-sobre-abraham}}

\bibleverse{48} ``¿Acaso no tenemos razón en decir que eres un
samaritano poseído por el demonio?'' dijeron los judíos. \footnote{\textbf{8:48}
  Juan 7,20}

\bibleverse{49} ``No, yo no tengo demonio alguno'', respondió Jesús.
``Yo glorifico a mi padre, pero ustedes me deshonran. \bibleverse{50} Yo
no vine aquí buscando honra para mí mismo. Pero hay Uno que lo hace por
mí y quien juzga a mi favor. \bibleverse{51} Les digo la verdad,
cualquiera que sigue mi enseñanza, no morirá jamás''.

\bibleverse{52} ``Ahora sabemos que estás poseído por el demonio'',
dijeron los judíos. ``Abraham murió, y los profetas también, ¡y tú estás
diciéndonos `cualquiera que sigue mi enseñanza, no morirá jamás!'
\bibleverse{53} ¿Acaso eres tú más grande que nuestro padre Abraham? Él
murió, y los profetas también murieron. ¿Quién crees que eres?''

\bibleverse{54} Jesús respondió: ``Si yo me glorifico a mí mismo, mi
Gloria no significa nada. Pero es Dios mismo quien me glorifica, el
mismo del cual ustedes dicen `Él es nuestro Dios'. \bibleverse{55}
Ustedes no lo conocen, pero yo sí lo conozco. Si yo dijera `No lo
conozco,' sería un mentiroso, tal como ustedes. Pero yo sí lo conozco, y
hago lo que Él dice. \footnote{\textbf{8:55} Juan 7,28-29}
\bibleverse{56} Abrahám se deleitó en esperar mi venida, y se alegró
cuando la vio''.

\bibleverse{57} Los judíos respondieron: ``Aún no tienes ni cincuenta
años de edad, ¿y dices que has visto a Abraham?''

\bibleverse{58} ``Les digo la verdad: antes de que Abraham naciera, Yo
soy'',\footnote{\textbf{8:58} Literalmente, ``Antes de que Abraham
  fuera, Yo soy''. Una vez más, Jesús usa el mismo nombre de Dios que se
  presenta en Éxodo 3:14. Tal significado es entendido por los oyentes y
  esto se evidencia en su reacción al querer apedrearlo por blasfemia.}
dijo Jesús.

\bibleverse{59} Ante esto, ellos tomaron piedras para arrojárselas, pero
Jesús se ocultó de ellos y se fue del Templo.\footnote{\textbf{8:59}
  Juan 10,31}

\hypertarget{la-curaciuxf3n-del-ciego-de-nacimiento-en-suxe1bado}{%
\subsection{La curación del ciego de nacimiento en
sábado}\label{la-curaciuxf3n-del-ciego-de-nacimiento-en-suxe1bado}}

\hypertarget{section-8}{%
\section{9}\label{section-8}}

\bibleverse{1} Mientras Jesús caminaba, vio a un hombre que era ciego
desde su nacimiento. \bibleverse{2} Sus discípulos le preguntaron:
``Maestro, ¿porqué nació ciego este hombre? ¿Fue él quien pecó, o fueron
sus padres?''

\bibleverse{3} Jesús respondió: ``Ni él, ni sus padres pecaron. Pero
para que el poder de Dios pueda manifestarse en su vida, \footnote{\textbf{9:3}
  Juan 11,4} \bibleverse{4} tenemos que seguir haciendo la obra de Aquél
que me envió mientras aún es de día. Cuando la noche venga, nadie podrá
trabajar. \footnote{\textbf{9:4} Juan 5,17; Jer 13,16} \bibleverse{5}
Mientras estoy aquí en el mundo, yo soy la luz del mundo''. \footnote{\textbf{9:5}
  Juan 12,35; Juan 8,12} \bibleverse{6} Después que dijo esto, Jesús
escupió en el suelo e hizo barro con su saliva, el cual puso después
sobre los ojos del hombre ciego. \footnote{\textbf{9:6} Mar 8,23}
\bibleverse{7} Entonces Jesús le dijo: ``Ve y lávate tú mismo en el
estanque de Siloé'' (que significa ``enviado''). Así que el hombre fue y
se lavó a sí mismo, y cuando se dirigía hacia su casa, ya podía ver.

\bibleverse{8} Sus vecinos y aquellos que lo habían conocido como un
mendigo, preguntaban: ``¿No es este el hombre que solía sentarse y
mendigar?'' \bibleverse{9} Algunos decían que él era, mientras que otros
decían: ``no, es alguien que se parece a él''. Pero el hombre seguía
diciendo ``¡Soy yo!''

\bibleverse{10} ``¿Cómo es posible que puedas ver?'' le preguntaron.

\bibleverse{11} Él respondió: ``Un hombre llamado Jesús hizo barro y lo
puso sobre mis ojos y me dijo `ve y lávate tú mismo en el estanque de
Siloé'. Entonces yo fui, y me lavé, y ahora puedo ver''.

\bibleverse{12} ``¿Dónde está?'' le preguntaron. ``No lo sé'', respondió
él.

\hypertarget{el-primer-interrogatorio-de-los-fariseos}{%
\subsection{El primer interrogatorio de los
fariseos}\label{el-primer-interrogatorio-de-los-fariseos}}

\bibleverse{13} Ellos llevaron al hombre que había estado ciego ante los
Fariseos. \bibleverse{14} Y era el día sábado cuando Jesús había
preparado el barro y había abierto los ojos de aquél hombre.
\bibleverse{15} Así que los Fariseos también le preguntaron cómo pudo
ver. Él les dijo: ``Él puso barro sobre mis ojos, y yo me lavé, y ahora
puedo ver''.

\bibleverse{16} Algunos de los Fariseos dijeron: ``El hombre que hizo
esto no puede venir de Dios porque no guarda el Sábado''. Pero otros se
preguntaban: ``¿Cómo puede un pecador hacer tales milagros?'' De modo
que tenían opiniones divididas.

\bibleverse{17} Entonces siguieron interrogando al hombre: ``Ya que
fueron tus ojos los que él abrió, ¿cuál es tu opinión acerca de él?''
preguntaron ellos. ``Sin duda, él es un profeta'', respondió el hombre.

\hypertarget{el-interrogatorio-de-los-padres}{%
\subsection{El interrogatorio de los
padres}\label{el-interrogatorio-de-los-padres}}

\bibleverse{18} Los líderes judíos aún se negaban a creer que el hombre
que había sido ciego ahora pudiera ver, hasta que llamaron a sus padres.
\bibleverse{19} Ellos les preguntaron: ``¿Es este su hijo, que estaba
ciego desde el nacimiento? ¿Cómo, entonces, es posible que ahora pueda
ver?''

\bibleverse{20} Sus padres respondieron: ``Sabemos que este es nuestro
hijo que nació siendo ciego. \bibleverse{21} Pero no tenemos idea de
cómo es posible que ahora vea, o de quién lo sanó. ¿Por qué no le
preguntan a él? pues ya está suficientemente grande. Él puede hablar por
sí mismo''. \bibleverse{22} La razón por la que sus padres dijeron esto,
es porque tenían miedo de lo que pudieran hacer los líderes judíos.
Éstos ya habían anunciado que cualquiera que declarara que Jesús era el
Mesías, sería expulsado de la sinagoga. \footnote{\textbf{9:22} Juan
  7,13; Juan 12,42} \bibleverse{23} Esa fue la razón por la que sus
padres dijeron ``pregúntenle a él, pues ya está suficientemente
grande''.

\hypertarget{el-segundo-interrogatorio-del-curado}{%
\subsection{El segundo interrogatorio del
curado}\label{el-segundo-interrogatorio-del-curado}}

\bibleverse{24} Por segunda vez, llamaron al hombre que había estado
ciego y le dijeron: ``¡Dale la gloria a Dios! Sabemos que este hombre es
un pecador''.

\bibleverse{25} El hombre respondió: ``Yo no sé si él es o no un
pecador. Todo lo que sé es que yo estaba ciego y ahora puedo ver''.

\bibleverse{26} Entonces ellos le preguntaron: ``¿Qué te hizo? ¿Cómo fue
que abrió tus ojos?''

\bibleverse{27} El hombre respondió: ``Ya les dije. ¿Acaso no estaban
escuchando? ¿Por qué quieren escucharlo de nuevo? ¿Acaso quieren
convertirse en sus discípulos también?''

\bibleverse{28} Entonces ellos lo insultaron y le dijeron: ``Tú eres
discípulo de ese hombre. \bibleverse{29} Nosotros somos discípulos de
Moisés. Sabemos que Dios le habló a Moisés, pero en lo que respecta a
esta persona, ni siquiera sabemos de dónde viene''.

\bibleverse{30} El hombre respondió: ``¡Es algo increíble! Ustedes no
saben de dónde viene pero él abrió mis ojos. \bibleverse{31} Nosotros
sabemos que Dios no escucha a los pecadores, pero sí escucha a todo el
que lo adora y hace su voluntad. \bibleverse{32} Nunca antes en toda la
historia se ha escuchado de un hombre que haya nacido ciego y haya sido
sanado. \bibleverse{33} Si este hombre no viniera de Dios, no podría
hacer nada''.

\bibleverse{34} ``Tú naciste siendo completamente pecador, y sin embargo
estás tratando de enseñarnos'', respondieron ellos. Y lo expulsaron de
lo sinagoga.

\hypertarget{la-fe-del-sanado-en-jesuxfas-jesuxfas-como-la-luz-de-los-que-no-ven-y-como-la-ceguera-de-los-que-ven}{%
\subsection{La fe del sanado en Jesús; Jesús como la luz de los que no
ven y como la ceguera de los que
ven}\label{la-fe-del-sanado-en-jesuxfas-jesuxfas-como-la-luz-de-los-que-no-ven-y-como-la-ceguera-de-los-que-ven}}

\bibleverse{35} Cuando Jesús escuchó que lo habían expulsado, encontró
al hombre y le preguntó: ``¿Crees en el Hijo del hombre?''

\bibleverse{36} El hombre respondió: ``Dime quién es, para creer en
él''.

\bibleverse{37} ``Ya lo has visto. ¡Es el que habla contigo ahora!'' le
dijo Jesús. \footnote{\textbf{9:37} Juan 4,26}

\bibleverse{38} ``¡Creo en ti, Señor!'' dijo él, y se arrodilló para
adorar a Jesús.

\bibleverse{39} Entonces Jesús le dijo: ``He venido al mundo para traer
juicio,\footnote{\textbf{9:39} ``Juicio'' en términos de tomar una
  decisión, no condenación.} a fin de que aquellos que son ciegos puedan
ver, y aquellos que ven se vuelvan ciegos''.

\bibleverse{40} Algunos Fariseos que estaban allí con Jesús le
preguntaron: ``Nosotros no somos ciegos también, ¿o sí?''

\bibleverse{41} Jesús respondió: ``Si ustedes estuvieran ciegos, no
serían culpables. Pero ahora que dicen que ven, mantienen su
culpa''.\footnote{\textbf{9:41} Prov 26,12; Juan 15,22}

\hypertarget{el-lenguaje-figurado-del-pastor-y-ladruxf3n-y-del-buen-pastor-y-asalariado}{%
\subsection{El lenguaje figurado del pastor y ladrón y del buen pastor y
asalariado}\label{el-lenguaje-figurado-del-pastor-y-ladruxf3n-y-del-buen-pastor-y-asalariado}}

\hypertarget{section-9}{%
\section{10}\label{section-9}}

\bibleverse{1} ``Les digo la verdad, cualquiera que no entra por la
puerta del redil, sino que trepa de alguna otra manera, es un ladrón.
\bibleverse{2} El que entra por la puerta es el pastor de las ovejas.
\bibleverse{3} El portero le abre la puerta y las ovejas responden a su
voz. Él llama a sus ovejas por nombre, y las saca del redil.
\bibleverse{4} Después, camina delante de ellas y las ovejas lo siguen
porque reconocen su voz. \bibleverse{5} Ellas no siguen a ningún
extraño. De hecho, ellas huyen de cualquier extraño porque no reconocen
su voz''. \bibleverse{6} Cuando Jesús hizo esta ilustración, los que le
escuchaban no entendieron lo que él quiso decir.

\hypertarget{yo-soy-la-puerta-para-las-ovejas}{%
\subsection{¡Yo soy la puerta para las
ovejas!}\label{yo-soy-la-puerta-para-las-ovejas}}

\bibleverse{7} Entonces Jesús les explicó nuevamente. ``Les digo la
verdad: Yo soy la puerta del redil. \bibleverse{8} Todos los que
vinieron antes de mi eran ladrones, pero las ovejas no los escucharon.
\bibleverse{9} Yo soy la puerta. Todo el que entra a través de mi, será
sanado\footnote{\textbf{10:9} O ``salvo''.} . Podrá ir y venir, y
encontrará la comida que necesite. \bibleverse{10} El ladrón solo viene
a robar, matar y destruir. Yo he venido para traerles vida, una vida
abundante.

\hypertarget{jesuxfas-como-el-buen-pastor}{%
\subsection{Jesús como el buen
pastor}\label{jesuxfas-como-el-buen-pastor}}

\bibleverse{11} Yo soy el buen pastor. El buen pastor entrega su vida
por sus ovejas. \bibleverse{12} El hombre a quien se le paga para cuidar
de las ovejas no es el pastor, y huye apenas ve que se acerca el lobo.
Él abandona a las ovejas porque no son suyas, y entonces el lobo ataca y
dispersa a las ovejas \footnote{\textbf{10:12} Sal 23,-1; Is 40,11; Ezeq
  34,11-23; Juan 15,13; Heb 13,20} \bibleverse{13} pues este hombre solo
trabaja para recibir su pago y no le importan las ovejas.
\bibleverse{14} Yo soy el buen pastor. Yo sé cuáles son mías, y ellas me
conocen \bibleverse{15} así como el Padre me conoce y yo lo conozco a
él. Yo entrego mi vida por las ovejas. \bibleverse{16} Tengo otras
ovejas que no están en este redil. Debo traerlas también. Ellas
escucharán mi voz, y entonces habrá un solo rebaño con un solo pastor.
\footnote{\textbf{10:16} Juan 11,52; Hech 10,34-35} \bibleverse{17} ``Es
por esto que el Padre me ama, porque yo doy mi vida para tomarla de
nuevo. \bibleverse{18} Ninguno puede quitarme la vida; Yo elijo
entregarla. Tengo el derecho de entregar mi vida y tengo el derecho de
volverla a tomar. Este es el mandato que me dio mi Padre''.

\bibleverse{19} Otra vez los judíos estaban dando opiniones sobre estas
palabras que dijo Jesús. \footnote{\textbf{10:19} Juan 7,43; Juan 9,16}
\bibleverse{20} Muchos de ellos decían: ``¡Está poseído por un demonio!
¡Está loco! ¿Por qué lo escuchan?'' \footnote{\textbf{10:20} Juan 7,20;
  Mar 3,21} \bibleverse{21} Otros decían: ``Estas no son las palabras de
alguien que está endemoniado. Además, un demonio no puede devolver la
vista a un ciego''.

\hypertarget{la-uxfaltima-justificaciuxf3n-de-jesuxfas-a-los-juduxedos-en-la-fiesta-de-la-dedicaciuxf3n-del-templo}{%
\subsection{La última justificación de Jesús a los judíos en la fiesta
de la dedicación del
templo}\label{la-uxfaltima-justificaciuxf3n-de-jesuxfas-a-los-juduxedos-en-la-fiesta-de-la-dedicaciuxf3n-del-templo}}

\bibleverse{22} Era invierno y era la fecha de la Fiesta de la
Dedicación en Jerusalén. \bibleverse{23} Jesús estaba caminando en el
Templo por el pórtico de Salomón. Los judíos lo rodearon y le
preguntaron: \footnote{\textbf{10:23} Hech 3,11} \bibleverse{24} ``¿Por
cuánto tiempo nos tendrás en suspenso\footnote{\textbf{10:24} Expresión
  coloquial que literalmente quiere decir ``levanta nuestras almas'', y
  se refiere a que estaba creando un estado de incertidumbre.} ? ¡Si
eres el Mesías, entonces dínoslo claramente!''

\bibleverse{25} Jesús respondió: ``Ya les dije, pero ustedes se negaron
a creerlo. Los milagros que yo hago en nombre de mi Padre son prueba de
quien yo soy. \bibleverse{26} Ustedes no creen en mí porque no son mis
ovejas. \footnote{\textbf{10:26} Juan 8,45; Juan 8,47} \bibleverse{27}
Mis ovejas reconocen mi voz; yo las conozco, y ellas me siguen.
\bibleverse{28} Yo les doy vida eterna; ellas nunca estarán perdidas, y
nadie me las puede arrebatar.\footnote{\textbf{10:28} Literalmente,
  ``quitar de las manos''. Similar al texto del versículo 29.}
\bibleverse{29} Mi Padre, quien me las entregó, es más grande que
cualquier otra persona; y a Él nadie se las puede arrebatar.
\bibleverse{30} Yo y el Padre somos uno''.

\bibleverse{31} Una vez más los judíos tomaron piedras para lanzárselas.
\bibleverse{32} Jesús les dijo: ``Ustedes han visto muchas cosas buenas
que he hecho, gracias al Padre. ¿Por cuál de todas ellas me van a
apedrear?''

\bibleverse{33} Loa judíos respondieron: ``No vamos a apedrearte por
hacer cosas buenas, sino por blasfemia, porque tú eres solamente un
hombre y estás afirmando que eres Dios''. \footnote{\textbf{10:33} Juan
  5,18; Mat 9,3; Mat 26,65}

\bibleverse{34} Jesús les respondió: ``¿Acaso no está escrito en la ley
de ustedes: `Yo dije, ustedes son dioses'? \footnote{\textbf{10:34}
  Citando Salmos 82:6.} \bibleverse{35} Él llamó `dioses' a estas
personas, a aquellos a quienes entregó la palabra de Dios---y la
Escritura no se puede modificar. \bibleverse{36} Entonces, ¿por qué
están diciendo ustedes que aquél a quien Dios apartó y envió al mundo
está blasfemando, porque dije `yo soy el Hijo de Dios'? \bibleverse{37}
Si no estoy haciendo lo que hace mi Padre, entonces no me crean.
\bibleverse{38} Pero si lo hago, deberían creerme por la evidencia de lo
que he hecho. Así podrán ustedes entender que el Padre está en mí, y que
yo estoy en el Padre''.

\bibleverse{39} Nuevamente ellos trataron de arrestarlo, pero él escapó
de ellos. \footnote{\textbf{10:39} Juan 8,59; Luc 4,30}

\hypertarget{jesuxfas-y-luxe1zaro-jesuxfas-como-la-resurrecciuxf3n-y-la-vida}{%
\subsection{Jesús y Lázaro; Jesús como la resurrección y la
vida}\label{jesuxfas-y-luxe1zaro-jesuxfas-como-la-resurrecciuxf3n-y-la-vida}}

\bibleverse{40} Se fue al otro lado del río Jordán, al lugar donde Juan
había comenzado a bautizar, y se quedó allí. \footnote{\textbf{10:40}
  Juan 1,28}

\bibleverse{41} Muchas personas llegaron donde él estaba, y decían:
``Juan no hizo milagros, pero todo lo que él dijo acerca de este hombre
se ha hecho realidad''. \bibleverse{42} Muchos de los que estaban allí
pusieron su confianza en Jesús.

\hypertarget{section-10}{%
\section{11}\label{section-10}}

\bibleverse{1} Un hombre llamado Lázaro estaba enfermo. Él vivía en
Betania con sus hermanas\footnote{\textbf{11:1} En el original se dice
  que Lázaro vivía en Betania con María y su hermana Marta. Sin embargo,
  en el versículo 2 se menciona que Lázaro es el hermano de María, de
  modo que su relación se identifica muy bien desde el comienzo.} María
y Marta. \footnote{\textbf{11:1} Luc 10,38-39} \bibleverse{2} María fue
la que ungió al Señor con perfume y secó sus pies con su cabello. El que
estaba enfermo era su hermano Lázaro. \footnote{\textbf{11:2} Juan 12,3}
\bibleverse{3} Entonces las hermanas enviaron un mensaje a Jesús:
``Señor, tu amigo está enfermo''.

\bibleverse{4} Cuando Jesús escuchó la noticia dijo: ``El resultado
final de esta enfermedad no será la muerte. A través de esto, será
revelada la gloria de Dios, a fin de que el Hijo de Dios sea
glorificado''. \footnote{\textbf{11:4} Juan 9,3} \bibleverse{5} Aunque
Jesús amaba a Marta, María y Lázaro, \bibleverse{6} y aunque había
escuchado que Lázaro estaba enfermo, se quedó en el lugar donde estaba
durante dos días más. \bibleverse{7} Entonces le dijo a los discípulos:
``Regresemos a Judea''.

\bibleverse{8} Los discípulos respondieron: ``Maestro, hace apenas unos
días los judíos estaban tratando de apedrearte. ¿Realmente quieres
regresar allá ahora?''

\bibleverse{9} ``¿Acaso no tiene doce horas el día?'' respondió Jesús.
\footnote{\textbf{11:9} Juan 9,4-5} \bibleverse{10} ``Si la gente camina
durante el día, no se tropieza porque puede ver hacia dónde va, gracias
a la luz de este mundo. Pero si camina por la noche, se tropieza porque
no hay luz''. \footnote{\textbf{11:10} Juan 12,35} \bibleverse{11}
Después de decirles esto, les dijo: ``Nuestro amigo Lázaro se ha
dormido, ¡pero yo voy para despertarlo!'' \footnote{\textbf{11:11} Mat
  9,24}

\bibleverse{12} Los discípulos dijeron: ``Señor, si está durmiendo, se
pondrá mejor''.

\bibleverse{13} Jesús se había estado refiriendo a la muerte de Lázaro,
pero los discípulos pensaban que él se refería realmente al acto de
dormir.\footnote{\textbf{11:13} En el Nuevo Testamento, dormir a menudo
  hace referencia a la muerte.} \bibleverse{14} Así que Jesús les dijo
claramente: ``Lázaro está muerto. \bibleverse{15} Me alegro por ustedes
de que yo no estaba allí, porque ahora ustedes podrán creer en mí.
Vayamos y veámoslo''.

\bibleverse{16} Tomás, el gemelo, dijo a sus condiscípulos: ``Vayamos
también para que muramos con él''.\footnote{\textbf{11:16} Refiriéndose
  a Jesús.}

\hypertarget{el-regreso-de-jesuxfas-a-betania-su-encuentro-con-martha-y-maria}{%
\subsection{El regreso de Jesús a Betania; su encuentro con Martha y
Maria}\label{el-regreso-de-jesuxfas-a-betania-su-encuentro-con-martha-y-maria}}

\bibleverse{17} Cuando Jesús llegó, se enteró de que Lázaro había estado
en la tumba por cuatro días. \bibleverse{18} Betania estaba apenas a dos
millas de Jerusalén, \bibleverse{19} y muchos judíos habían venido a
consolar a María y Marta ante la pérdida de su hermano. \bibleverse{20}
Cuando Marta supo que Jesús venía, salió a su encuentro, pero María se
quedó en casa. \bibleverse{21} Marta le dijo a Jesús: ``Señor, si
hubieras estado aquí, mi hermano no habría muerto. \bibleverse{22} Pero
estoy segura de que incluso ahora Dios te concederá cualquier cosa que
le pidas''.

\bibleverse{23} Jesús le dijo: ``Tu hermano se levantará de nuevo''.

\bibleverse{24} ``Sé que se levantará en la resurrección, en el día
final'', respondió Marta. \footnote{\textbf{11:24} Juan 5,28-29; Juan
  6,40; Mat 22,23-33}

\bibleverse{25} Jesús dijo: ``Yo soy la resurrección y la vida. Aquellos
que creen en mí, vivirán aunque mueran. \bibleverse{26} Todos los que
viven en mí y creen en mí, no morirán jamás. ¿Crees esto?''

\bibleverse{27} ``Sí, Señor'', respondió ella, ``Yo creo que eres el
Mesías, el Hijo de Dios, el que esperábamos que viniera al mundo''.
\footnote{\textbf{11:27} Mat 16,16}

\bibleverse{28} Cuando ella terminó de decir esto, fue y le dijo a su
hermana María, en privado: ``El Maestro está aquí y ha dicho que quiere
verte''.

\bibleverse{29} Tan pronto escuchó esto, María se levantó y fue a verlo.
\bibleverse{30} Jesús todavía no había llegado a la aldea. Aún estaba en
el lugar donde Marta lo había ido a recibir. \bibleverse{31} Los judíos
que habían estado consolando a María en la casa vieron cómo ella se
levantó rápidamente y salió. Entonces la siguieron, pensado que se
dirigía a la tumba a llorar.

\bibleverse{32} Cuando María llegó al lugar donde estaba Jesús y lo vio,
se postró a sus pies y dijo: ``Señor, si hubieras estado aquí, mi
hermano no habría muerto''.

\bibleverse{33} Cuando la vio llorando a ella y a los judíos que habían
venido con ella, Jesús se sintió atribulado\footnote{\textbf{11:33} La
  palabra que se usa aquí expresa una intensa emoción, incluso rabia.
  También se usa en el versículo 38.} y triste.

\hypertarget{jesuxfas-en-la-tumba-y-su-oraciuxf3n-la-resurrecciuxf3n-de-luxe1zaro-de-entre-los-muertos}{%
\subsection{Jesús en la tumba y su oración; la resurrección de Lázaro de
entre los
muertos}\label{jesuxfas-en-la-tumba-y-su-oraciuxf3n-la-resurrecciuxf3n-de-luxe1zaro-de-entre-los-muertos}}

\bibleverse{34} ``¿Dónde lo han puesto?'' preguntó él. Ellos
respondieron: ``Señor, ven y ve''.

\bibleverse{35} Entonces Jesús también lloró.

\bibleverse{36} ``Miren cuánto lo amaba'', dijeron los judíos.
\bibleverse{37} Pero algunos de ellos decían: ``Si pudo abrir los ojos
de un hombre ciego, ¿no podía haber impedido la muerte de Lázaro?''
\footnote{\textbf{11:37} Juan 9,7}

\bibleverse{38} Muy atribulado, Jesús se dirigió a la tumba. Era una
cueva con una gran piedra que tapaba la entrada. \footnote{\textbf{11:38}
  Mat 27,60} \bibleverse{39} ``Quiten la piedra'', les dijo Jesús. Pero
Marta, la hermana del difunto, dijo: ``Señor, en este momento ya debe
haber mal olor porque él ha estado muerto por cuatro días''.

\bibleverse{40} ``¿No te dije que si crees en mi verás la Gloria de
Dios?'' respondió Jesús.

\bibleverse{41} Entonces quitaron la piedra. Jesús levantó su mirada
hacia el cielo y dijo: ``Padre, gracias por escucharme. \bibleverse{42}
Yo sé que siempre me escuchas. Dije esto por causa de la multitud que
está aquí, a fin de que crean que tú me enviaste''. \footnote{\textbf{11:42}
  Juan 12,30} \bibleverse{43} Después de decir esto, Jesús dijo en voz
alta: ``¡Lázaro, sal de ahí!''

\bibleverse{44} El difunto salió. Sus manos y sus pies estaban envueltos
con tiras de lino, y su cabeza estaba envuelta con un paño. ``Quítenle
las vendas y déjenlo ir'', les dijo Jesús.

\hypertarget{los-efectos-del-milagro-resoluciuxf3n-de-muerte-del-sumo-consejo-jesuxfas-escapa-a-efrauxedn}{%
\subsection{Los efectos del milagro; Resolución de muerte del sumo
consejo; Jesús escapa a
Efraín}\label{los-efectos-del-milagro-resoluciuxf3n-de-muerte-del-sumo-consejo-jesuxfas-escapa-a-efrauxedn}}

\bibleverse{45} Como consecuencia de esto, muchos de los judíos que
habían venido a consolar a María y que vieron lo que Jesús hizo,
creyeron en él. \bibleverse{46} Pero otros fueron donde los Fariseos y
les contaron lo que Jesús había hecho. \bibleverse{47} Entonces el jefe
de los sacerdotes y los Fariseos convocaron una reunión del Concilio
Supremo. ``¿Qué haremos?'' preguntaban. ``Este hombre está haciendo
muchos milagros. \bibleverse{48} Si dejamos que siga, todos creerán en
él, y entonces los romanos destruirán tanto el Templo como nuestra
nación''.\footnote{\textbf{11:48} Literalmente, ``el lugar y la
  nación''.}

\bibleverse{49} ``¡Ustedes no entienden nada!'' dijo Caifás, quien era
el Sumo sacerdote en ese año. \bibleverse{50} ``¿Acaso no se dan cuenta
de que es mejor para ustedes que un solo hombre muera por el pueblo y no
que toda la nación sea destruida?'' \footnote{\textbf{11:50} Juan 18,14}
\bibleverse{51} Él no decía esto por su propia cuenta, sino que como
Sumo sacerdote en ese año, él estaba profetizando que Jesús moriría por
la nación. \footnote{\textbf{11:51} Éxod 28,30; Núm 27,21}
\bibleverse{52} Y no solo por la nación judía, sino por todos los hijos
de Dios que estaban esparcidos, a fin de que volvieran a reunirse y ser
un solo pueblo. \footnote{\textbf{11:52} Juan 7,35; Juan 10,16; 1Jn 2,2}
\bibleverse{53} A partir de ese momento, ellos conspiraban sobre cómo
podían matar a Jesús. \bibleverse{54} De modo que Jesús no viajaba de
manera pública entre los judíos sino que se fue a una ciudad llamada
Efraín, en la región cercana al desierto, y permaneció allí con sus
discípulos.

\bibleverse{55} Ya casi era la fecha de la celebración de la Pascua
judía, y mucha gente se fue del campo hasta Jerusalén para
purificarse\footnote{\textbf{11:55} Mediante una serie de rituales
  religiosos.} para la Pascua. \bibleverse{56} La gente buscaba a Jesús
y hablaban de él mientras estaban en el Templo. ``¿Qué piensan de
esto?'' se preguntaban unos a otros. ``¿Será que no vendrá a la
fiesta?'' \bibleverse{57} Los jefes de los sacerdotes y los Fariseos
habían dado la orden de que cualquiera que supiera dónde estaba Jesús
debía informarles para así poder arrestarlo.

\hypertarget{la-unciuxf3n-de-jesuxfas-consagraciuxf3n-de-la-muerte-en-betania}{%
\subsection{La unción de Jesús (consagración de la muerte) en
Betania}\label{la-unciuxf3n-de-jesuxfas-consagraciuxf3n-de-la-muerte-en-betania}}

\hypertarget{section-11}{%
\section{12}\label{section-11}}

\bibleverse{1} Seis días después de la Pascua, Jesús fue a Betania, al
hogar de Lázaro, quien había sido levantado de los muertos. \footnote{\textbf{12:1}
  Juan 11,1; Juan 11,43} \bibleverse{2} Había allí una cena preparada en
su honor. Marta ayudaba a servir la comida mientras que Lázaro estaba
sentado en la mesa con Jesús y con los demás invitados. \bibleverse{3}
María trajo medio litro de perfume de nardo puro y ungió los pies de
Jesús, secándolos con su cabello. El aroma del perfume se esparció por
toda la casa.

\bibleverse{4} Pero uno de los discípulos, Judas Iscariote, quien
después traicionaría a Jesús, preguntó: \bibleverse{5} ``¿No era mejor
vender este perfume y regalar el dinero a los pobres? El perfume costaba
trescientos denarios''.\footnote{\textbf{12:5} Aproximadamente un año de
  salarios de un denario por día.} \bibleverse{6} Él no decía esto
porque le interesaran los pobres, sino porque era un ladrón. Él era
quien administraba el dinero de los discípulos y a menudo tomaba de ese
dinero para sí mismo. \footnote{\textbf{12:6} Luc 8,3}

\bibleverse{7} ``No la critiquen'',\footnote{\textbf{12:7} O, ``déjenla
  en paz''.} respondió Jesús. ``Ella hizo esto como una preparación para
el día de mi entierro. \bibleverse{8} Ustedes siempre tendrán a los
pobres aquí con ustedes,\footnote{\textbf{12:8} Ver Deuteronomio 15:11.}
pero no siempre me tendrán a mí aquí''.

\bibleverse{9} Una gran multitud había descubierto que él estaba allí.
Llegaron al lugar no solo para ver a Jesús sino porque querían ver a
Lázaro, el hombre a quien Jesús había levantado de los muertos.
\bibleverse{10} Entonces los jefes de los sacerdotes planeaban matar a
Lázaro también, \bibleverse{11} pues era por él que muchos judíos ya no
los seguían a ellos sino que estaban creyendo en Jesús.

\hypertarget{la-entrada-de-jesuxfas-a-jerusaluxe9n-el-domingo-de-ramos}{%
\subsection{La entrada de Jesús a Jerusalén el Domingo de
Ramos}\label{la-entrada-de-jesuxfas-a-jerusaluxe9n-el-domingo-de-ramos}}

\bibleverse{12} Al día siguiente, las multitudes de personas que habían
venido a la fiesta de la Pascua escucharon que Jesús iba de camino hacia
Jerusalén. \bibleverse{13} Entonces cortaron ramas de palmeras y
salieron a darle la bienvenida, gritando: ``¡Hosanna! Bendito es el que
viene en el nombre del Señor. Bendito es el rey de Israel''.\footnote{\textbf{12:13}
  Citando Salmos 118:26.} \footnote{\textbf{12:13} Sal 118,25-26}

\bibleverse{14} Jesús encontró un potrillo y se montó sobre él, tal como
dice la Escritura: \bibleverse{15} ``No temas, hija de Sión. Mira, tu
rey viene, montado en un potrillo''.\footnote{\textbf{12:15} Citando
  Zacarías 9:9.} \bibleverse{16} En ese momento, los discípulos de Jesús
no entendían lo que significaban estas cosas. Fue después, cuando Jesús
fue glorificado,\footnote{\textbf{12:16} Glorificado: en su resurrección
  y ascensión.} que ellos entendieron que lo que había ocurrido ya había
sido profetizado y se había aplicado a él. \bibleverse{17} Muchos en la
multitud habían visto a Jesús llamar a Lázaro de la tumba y levantarlo
de los muertos, y estaban contando el hecho. \bibleverse{18} Esa fue la
razón por la que tantas personas fueron a conocer a Jesús---porque
habían escuchado acerca de este milagro. \bibleverse{19} Los Fariseos se
decían unos a otros: ``Miren, no estamos logrando nada. Todos corren
detrás de él''.

\hypertarget{jesuxfas-anuncia-su-sufrimiento-mortal-y-su-subsiguiente-glorificaciuxf3n-como-salvador-del-mundo}{%
\subsection{Jesús anuncia su sufrimiento mortal y su subsiguiente
glorificación como salvador del
mundo}\label{jesuxfas-anuncia-su-sufrimiento-mortal-y-su-subsiguiente-glorificaciuxf3n-como-salvador-del-mundo}}

\bibleverse{20} Sucedió que unos griegos habían venido a adorar durante
la fiesta. \bibleverse{21} Ellos se acercaron a Felipe de Betsaida, de
Galilea, y le dijeron: ``Señor, quisiéramos ver a Jesús''. \footnote{\textbf{12:21}
  Juan 1,44} \bibleverse{22} Felipe fue y le dijo a Andrés. Entonces
ambos se acercaron a Jesús y le dijeron esto.

\bibleverse{23} Jesús respondió: ``Ha llegado el momento para que el
Hijo del hombre sea glorificado. \bibleverse{24} Les digo la verdad:
hasta que un grano de trigo no se plante en la tierra y
muera,\footnote{\textbf{12:24} Queriendo decir con claridad que el grano
  muere aparentemente.} sigue siendo un grano. Pero si muere, produce
muchos más granos de trigo. \bibleverse{25} Si ustedes aman su propia
vida, la perderán; pero si no aman su propia vida en este mundo,
salvarán sus vidas para siempre. \footnote{\textbf{12:25} Mat 10,39; Mat
  16,25; Luc 17,33} \bibleverse{26} Si ustedes quieren servirme, tienen
que seguirme. Mis siervos estarán donde yo esté, y mi Padre honrará a
todo el que me sirva. \footnote{\textbf{12:26} Juan 17,24}

\bibleverse{27} ``Ahora estoy atribulado. ¿Qué debo decir, `Padre,
guárdame de este momento de sufrimiento que está por venir'?\footnote{\textbf{12:27}
  Literalmente, ``esta hora''.} No, porque esta es la razón por la cual
vine---para vivir este momento de sufrimiento. \footnote{\textbf{12:27}
  Mat 26,38} \bibleverse{28} Padre, muéstrame la gloria de tu
carácter''.\footnote{\textbf{12:28} O ``nombre''. Nombre es sinónimo de
  carácter.} Vino una voz del cielo que decía: ``He mostrado la gloria
de mi carácter, y la volveré a mostrar''. \footnote{\textbf{12:28} Mat
  3,17; Mat 17,5; Juan 13,31}

\bibleverse{29} La multitud que estaba allí en pie escuchó la voz.
Algunos decían que era un trueno; otros decían que un ángel le había
hablado.

\bibleverse{30} Jesús les dijo: ``Esta voz no habló por mí, sino por
causa de ustedes. \footnote{\textbf{12:30} Juan 11,42} \bibleverse{31}
Ahora es el juicio de este mundo; ahora el príncipe de este mundo será
lanzado fuera. \footnote{\textbf{12:31} Juan 14,30; Juan 16,11; Luc
  10,18} \bibleverse{32} Pero cuando yo sea levantado, a todos atraeré
hacia mí''. \footnote{\textbf{12:32} Juan 8,28} \bibleverse{33} (Él dijo
esto para señalar el tipo de muerte que iba a sufrir).

\bibleverse{34} La multitud respondió: ``la Ley\footnote{\textbf{12:34}
  Refiriéndose a lo que nosotros llamamos como El Antiguo Testamento.}
nos dice que el Mesías vivirá para siempre, ¿cómo puedes decir tú que el
Hijo del hombre debe ser `levantado'? ¿Quién es este `Hijo del
hombre'?''

\bibleverse{35} Jesús respondió: ``La luz está aquí con ustedes un poco
más. Caminen mientras tienen la luz para que la oscuridad no los
sorprenda. Los que caminan en la oscuridad no saben hacia dónde van.
\footnote{\textbf{12:35} Juan 11,10} \bibleverse{36} Confíen en la luz
mientras la tienen para que sean hijos de la luz''. Cuando Jesús terminó
de decirles esto, se fue y se ocultó de ellos. \footnote{\textbf{12:36}
  Efes 5,9}

\hypertarget{la-revisiuxf3n-del-evangelista-de-la-actividad-puxfablica-de-jesuxfas}{%
\subsection{La revisión del evangelista de la actividad pública de
Jesús}\label{la-revisiuxf3n-del-evangelista-de-la-actividad-puxfablica-de-jesuxfas}}

\bibleverse{37} Pero a pesar de todos los milagros que él había hecho en
presencia de ellos, aún no creían en Jesús. \bibleverse{38} Esto era en
cumplimiento del mensaje del profeta Isaías, quien dijo: ``Señor, ¿quién
ha creído en lo que hemos dicho? ¿A quién le ha sido revelado el poder
del Señor?''\footnote{\textbf{12:38} Citando Isaías 53:1.}

\bibleverse{39} Ellos no podían creer en él, y como consecuencia,
cumplieron lo que Isaías dijo: \bibleverse{40} ``Él cegó sus ojos, y
oscureció sus mentes a fin de que sus ojos no vieran, y sus mentes no
pensaran, y no se volvieran a mí---porque si lo hacían, yo los
sanaría''.\footnote{\textbf{12:40} Citando Isaías 6:10.} \footnote{\textbf{12:40}
  Mat 13,14-15}

\bibleverse{41} Isaías vio la gloria de Jesús y dijo esto en referencia
a él. \footnote{\textbf{12:41} Is 6,1} \bibleverse{42} Incluso muchos de
los líderes creían en él. Sin embargo, no lo admitían abiertamente
porque no querían que los Fariseos los expulsaran de la sinagoga,
\footnote{\textbf{12:42} Juan 9,22} \bibleverse{43} demostrando que
amaban la admiración humana más que la aprobación de Dios. \footnote{\textbf{12:43}
  Juan 5,44}

\hypertarget{el-testimonio-de-jesuxfas-sobre-suxed-mismo-y-sobre-su-relaciuxf3n-con-dios}{%
\subsection{El testimonio de Jesús sobre sí mismo y sobre su relación
con
Dios}\label{el-testimonio-de-jesuxfas-sobre-suxed-mismo-y-sobre-su-relaciuxf3n-con-dios}}

\bibleverse{44} Jesús dijo a gran voz: ``Si creen en mí, no solamente
están creyendo en mí sino también en Aquél que me envió. \bibleverse{45}
Cuando me ven a mí, están viendo al que me envió. \footnote{\textbf{12:45}
  Juan 14,9}

\bibleverse{46} He venido como una luz que ilumina al mundo, así que si
creen en mí no permanecerán en la oscuridad. \bibleverse{47} Yo no juzgo
a ninguno que escucha mis palabras y no hace lo que yo digo. Yo vine a
salvar al mundo, no a juzgarlo. \bibleverse{48} Cualquiera que me
rechaza y no acepta mis palabras, será juzgado en el juicio final,
conforme a lo que he dicho. \bibleverse{49} Porque no estoy hablando por
mí mismo sino por mi Padre que me envió. Él fue quien me instruyó en
cuanto a lo que debo decir y cómo lo debo decir. \bibleverse{50} Yo sé
que lo que Él me ordenó que les dijera, trae vida eterna---Así que todo
lo que yo digo es lo que el Padre me dijo a mí''.

\hypertarget{el-lavado-de-pies}{%
\subsection{El lavado de pies}\label{el-lavado-de-pies}}

\hypertarget{section-12}{%
\section{13}\label{section-12}}

\bibleverse{1} Era el día antes de la fiesta de la Pascua, y Jesús sabía
que había llegado la hora de abandonar este mundo y volver a su Padre.
Había amado a quienes estaban en el mundo y que eran suyos, y ahora les
había demostrado por completo su amor hacia ellos. \footnote{\textbf{13:1}
  Juan 7,30; Juan 17,1} \bibleverse{2} Era el momento de la cena, y el
Diablo ya había inculcado la idea de traicionar a Jesús en la mente de
Judas, el hijo de Simón Iscariote. \footnote{\textbf{13:2} Luc 22,3}
\bibleverse{3} Jesús sabía que el Padre había puesto todas las cosas
bajo su autoridad,\footnote{\textbf{13:3} Literalmente, ``en sus
  manos''.} y él había venido de Dios y ahora iba a regresar a Dios.
\footnote{\textbf{13:3} Juan 3,35; Juan 16,28} \bibleverse{4} Entonces
Jesús se levantó en medio de la cena, quitó su bata y se ciñó con una
toalla. \bibleverse{5} Echó agua en un tazón y comenzó a lavar los pies
de sus discípulos, secándolos con la toalla con la que se había ceñido.
\bibleverse{6} Se acercó a Simón Pedro, quien le preguntó: ``Señor, ¿vas
a lavar mis pies?''

\bibleverse{7} Jesús respondió: ``Ahora no entenderás lo que estoy
haciendo por ti. Pero un día entenderás''.

\bibleverse{8} ``¡No!'' protestó Pedro. ``¡Nunca lavarás mis pies!''
Jesús respondió, ``Si no te lavo, no tendrás parte conmigo'',

\bibleverse{9} ``¡Entonces, Señor, no laves solamente mis pies, sino
también mis manos y mi cabeza!'' exclamó Simón Pedro.

\bibleverse{10} Jesús respondió, ``Cualquiera que ya se ha bañado, solo
necesita lavar sus pies y entonces estará completamente limpio. Ustedes
están limpios---pero no todos''. \bibleverse{11} Pues él sabía quién era
el que iba a traicionarlo. Por eso dijo ``No todos están limpios''.

\hypertarget{la-interpretaciuxf3n-de-jesuxfas-de-su-humilde-servicio-de-amor}{%
\subsection{La interpretación de Jesús de su humilde servicio de
amor}\label{la-interpretaciuxf3n-de-jesuxfas-de-su-humilde-servicio-de-amor}}

\bibleverse{12} Después que Jesús hubo lavado los pies de los
discípulos, volvió a ponerse su bata y se sentó. ``¿Entienden ustedes lo
que les he hecho?'' les preguntó. \bibleverse{13} ``Ustedes me llaman
`Maestro' y `Señor,' y está bien que lo hagan, pues eso es lo que soy.
\footnote{\textbf{13:13} Mat 23,8; Mat 23,10} \bibleverse{14} Así que si
yo, que soy su Maestro y su Señor, he lavado sus pies, ustedes deben
lavarse los pies unos a otros. \footnote{\textbf{13:14} Luc 22,27}
\bibleverse{15} Yo les he dejado un ejemplo, para que ustedes hagan como
yo hice. \footnote{\textbf{13:15} Fil 2,5; 1Pe 2,21} \bibleverse{16} Les
digo la verdad, los siervos no son más importantes que su amo, y el que
es enviado no es mayor que quien lo envía. \footnote{\textbf{13:16} Mat
  10,24} \bibleverse{17} Ahora que ustedes entienden estas cosas, serán
benditos si las hacen. \footnote{\textbf{13:17} Mat 7,24}
\bibleverse{18} No estoy hablando de todos ustedes---Yo conozco a los
que he escogido. Pero para cumplir la Escritura: `El que comparte mi
comida se ha vuelto contra mí'.\footnote{\textbf{13:18} Citando Salmos
  41:9.} \bibleverse{19} Les digo ahora, antes de que ocurra, para que
cuando ocurra, estén convencidos de que yo soy quien soy.
\bibleverse{20} Les digo la verdad, cualquiera que recibe a quien yo
envío, me recibe a mí, y recibe a Aquél que me envió''.

\hypertarget{identificaciuxf3n-y-remociuxf3n-del-traidor}{%
\subsection{Identificación y remoción del
traidor}\label{identificaciuxf3n-y-remociuxf3n-del-traidor}}

\bibleverse{21} Después que dijo esto, Jesús estuvo profundamente
atribulado, y declaró: ``Les digo la verdad, uno de ustedes va a
traicionarme''. \footnote{\textbf{13:21} Juan 12,27}

\bibleverse{22} Los discípulos se miraron unos a otros, preguntándose de
cuál de ellos hablaba Jesús. \bibleverse{23} El discípulo a quien Jesús
amaba\footnote{\textbf{13:23} A menudo se entiende como Juan
  refiriéndose a sí mismo. (Ver también 20:2, 21:7, 21:20).} estaba
sentado junto a él en la mesa, apoyado cerca de él. \bibleverse{24}
Simón Pedro le hizo señas para que le preguntara a Jesús de cuál de
todos ellos hablaba.

\bibleverse{25} Entonces él se inclinó hacia Jesús y le preguntó,
``Señor, ¿quién es?''

\bibleverse{26} Jesús respondió: ``Es aquél a quien yo le entregue un
trozo de pan después de haberlo mojado''. \bibleverse{27} Después de
haber mojado el trozo de pan, lo dio a Judas, hijo de Simón Iscariote.
Cuando Judas tomó el pan, Satanás entró en él. ``Lo que vas a hacer,
hazlo rápido'', le dijo Jesús.

\bibleverse{28} Ninguno en la mesa entendió lo que Jesús quiso decir con
esto. \bibleverse{29} Como Judas estaba a cargo del dinero, algunos de
ellos pensaron que Jesús le estaba diciendo que se fuera y comprara lo
necesario para la fiesta de la Pascua, o que fuera a donar algo a los
pobres. \bibleverse{30} Judas se fue inmediatamente después que hubo
tomado el trozo de pan y se marchó. Y era de noche.

\hypertarget{el-anuncio-de-jesuxfas-de-su-glorificaciuxf3n}{%
\subsection{El anuncio de Jesús de su
glorificación}\label{el-anuncio-de-jesuxfas-de-su-glorificaciuxf3n}}

\bibleverse{31} Después que Judas se fue, Jesús dijo: ``Ahora el Hijo
del hombre es glorificado, y por medio de él, Dios es glorificado.
\footnote{\textbf{13:31} Juan 12,23; Juan 12,28} \bibleverse{32} Si Dios
es glorificado por medio de él, entonces Dios mismo glorificará al hijo,
y lo glorificará inmediatamente. \footnote{\textbf{13:32} Juan 17,1-5}
\bibleverse{33} Mis hijos, yo estaré con ustedes solo un poco más. Me
buscarán, pero les digo lo mismo que le dije a los judíos: adonde yo
voy, ustedes no pueden ir. \footnote{\textbf{13:33} Juan 9,21}

\hypertarget{el-nuevo-mandamiento-de-amar}{%
\subsection{El nuevo mandamiento de
amar}\label{el-nuevo-mandamiento-de-amar}}

\bibleverse{34} ``Les estoy dando un nuevo mandato: ámense los unos a
los otros. Ámense los unos a los otros de la misma manera que yo los he
amado a ustedes. \footnote{\textbf{13:34} Juan 15,12-13; Juan 15,17}
\bibleverse{35} Si ustedes se aman los unos a los otros, demostrarán a
todos que son mis discípulos''.

\hypertarget{anuncio-de-la-negaciuxf3n-de-pedro}{%
\subsection{Anuncio de la negación de
Pedro}\label{anuncio-de-la-negaciuxf3n-de-pedro}}

\bibleverse{36} Simón Pedro le preguntó: ``¿Adónde vas, Señor?'' Jesús
respondió: ``Adonde yo voy, ustedes no pueden seguirme. Ustedes me
seguirán después''. \footnote{\textbf{13:36} Juan 21,18-19}

\bibleverse{37} ``Señor, ¿por qué no puedo seguirte ahora?'' preguntó
Pedro. ``Entregaré mi vida por ti''.

\bibleverse{38} ``¿Realmente estás preparado para morir por mí? Te digo
la verdad: antes de que el gallo cante tú me negarás tres veces'', le
respondió Jesús.

\hypertarget{jesuxfas-el-camino-a-dios-su-uniuxf3n-con-dios}{%
\subsection{Jesús el camino a Dios, su unión con
Dios}\label{jesuxfas-el-camino-a-dios-su-uniuxf3n-con-dios}}

\hypertarget{section-13}{%
\section{14}\label{section-13}}

\bibleverse{1} ``No dejen que sus mentes estén ansiosas. Crean en Dios,
crean en mí también.\footnote{\textbf{14:1} O ``Ustedes creen en Dios,
  crean en mí también''.} \bibleverse{2} En la casa de mi Padre hay
espacio suficiente. Si no fuese así yo se los hubiera dicho. Yo voy a
preparar un lugar para ustedes. \bibleverse{3} Cuando me haya ido y haya
preparado lugar para ustedes, regresaré nuevamente y los llevaré
conmigo, para que puedan estar allí conmigo también. \footnote{\textbf{14:3}
  Juan 12,26; Juan 17,24} \bibleverse{4} Ustedes conocen el camino hacia
donde yo voy''.

\bibleverse{5} Tomás le dijo: ``Señor, no sabemos a dónde vas. ¿Cómo
podemos conocer el camino?'' .

\bibleverse{6} Jesús respondió: ``Yo soy el camino, la verdad y la vida.
Nadie viene al Padre si no es a través de mí. \bibleverse{7} Si ustedes
me han conocido, conocerán también a mi Padre. A partir de ahora,
ustedes lo conocen y lo han visto''.

\bibleverse{8} Felipe dijo: ``Señor, muéstranos al Padre, y estaremos
convencidos''.

\bibleverse{9} Jesús respondió: ``He estado con ustedes por tanto
tiempo, Felipe, ¿y sin embargo aún no me conoces? Todo el que me ha
visto a mí ha visto al Padre. ¿Cómo puedes decir `muéstranos al Padre'?
\footnote{\textbf{14:9} Juan 12,45; Heb 1,3} \bibleverse{10} ¿No crees
que yo vivo en el Padre y que el Padre vive en mí? Las palabras que yo
hablo no son mías; es el Padre que vive en mí quien está haciendo su
obra. \footnote{\textbf{14:10} Juan 12,49} \bibleverse{11} Créanme
cuando les digo que yo vivo en el Padre y el Padre en mí, o al menos
créanlo por la evidencia de todo lo que he hecho. \footnote{\textbf{14:11}
  Juan 10,25; Juan 10,38}

\hypertarget{promesa-del-espuxedritu-santo}{%
\subsection{Promesa del Espíritu
Santo}\label{promesa-del-espuxedritu-santo}}

\bibleverse{12} ``Les digo la verdad, todo el que cree en mí hará las
mismas cosas que yo estoy haciendo. De hecho, hará cosas incluso más
grandes\footnote{\textbf{14:12} Más grandes en cuanto a su alcance.}
porque yo voy ahora al Padre. \footnote{\textbf{14:12} Mat 28,19}
\bibleverse{13} Yo haré cualquier cosa que ustedes pidan en mi nombre,
para que mi Padre sea glorificado a través del Hijo. \footnote{\textbf{14:13}
  Juan 15,7; Juan 16,24; Mar 11,24; 1Jn 5,14; 1Jn 1,5-15}
\bibleverse{14} Cualquier cosa que ustedes pidan en mi nombre, yo la
haré. \bibleverse{15} ``Si ustedes me aman, guardarán mis mandamientos.
\bibleverse{16} Yo le pediré al padre, y él les enviará a ustedes otro
Consolador,\footnote{\textbf{14:16} Consolador. La palabra en el
  original (transliterada en español como ``Parakletos'') se refiere a
  alguien que está llamado a ``acompañar'' y ayudar. Ver también 14:26,
  15:26, 16:7, y 1 Juan 2:1.} \footnote{\textbf{14:16} Juan 15,26; Juan
  16,7} \bibleverse{17} el Espíritu de verdad, que siempre estará con
ustedes. El mundo no puede aceptarlo porque ellos no lo buscan y no lo
conocen. Pero ustedes lo conocen porque él vive con ustedes y estará en
ustedes. \footnote{\textbf{14:17} Juan 16,13} \bibleverse{18} ``Yo no
los abandonaré como huérfanos: regresaré a ustedes. \bibleverse{19} No
pasará mucho tiempo antes de que el mundo ya no me vea más, pero ustedes
me verán. Porque yo vivo, y ustedes vivirán también. \footnote{\textbf{14:19}
  Juan 20,20} \bibleverse{20} Ese día\footnote{\textbf{14:20}
  Refiriéndose al versículo 18, haciendo referencia principalmente a su
  venida después de su resurrección.} ustedes sabrán que yo vivo en el
Padre, que ustedes viven en mí, y que yo vivo en ustedes.

\hypertarget{promesa-de-la-muxe1s-uxedntima-comunidad-de-espuxedritu-y-amor-con-dios-y-jesuxfas}{%
\subsection{Promesa de la más íntima comunidad de espíritu y amor con
Dios y
Jesús}\label{promesa-de-la-muxe1s-uxedntima-comunidad-de-espuxedritu-y-amor-con-dios-y-jesuxfas}}

\bibleverse{21} Aquellos que guardan mis mandamientos son los que me
aman; aquellos que me aman, serán amados por mi Padre. Yo también los
amaré, y me revelaré en ellos''.

\bibleverse{22} Judas (no Iscariote) respondió: ``Señor, ¿por qué te
revelarás a nosotros y no al mundo?'' \footnote{\textbf{14:22} Hech
  10,40-41}

\bibleverse{23} Jesús respondió: ``Aquellos que me aman harán lo que yo
digo. Mi Padre los amará, y vendremos a crear un hogar con ellos.
\footnote{\textbf{14:23} Prov 8,17; Efes 3,17} \bibleverse{24} Los que
no me aman, no hacen lo que yo digo. Estas palabras no vienen de mí,
vienen del Padre que me envió. \footnote{\textbf{14:24} Juan 7,16-17}

\hypertarget{promesa-de-enseuxf1ar-del-espuxedritu-santo}{%
\subsection{Promesa de enseñar del Espíritu
Santo}\label{promesa-de-enseuxf1ar-del-espuxedritu-santo}}

\bibleverse{25} ``Les estoy explicando esto ahora, mientras aún estoy
con ustedes. \bibleverse{26} Pero cuando el Padre envíe al Consolador,
el Espíritu Santo, en mi lugar,\footnote{\textbf{14:26} Literalmente,
  ``en mi nombre''. Esta frase puede significar ``con mi autoridad'',
  ``a través de mí'', ``por mí'', ``perteneciéndome a mí'' etc. En
  realidad es una forma de referirse a la persona y su carácter.} él les
enseñará todas las cosas y les recordará todo lo que yo les dije.
\bibleverse{27} ``Yo les dejo paz; les estoy dando mi paz. La paz que yo
les doy no se asemeja a ninguna cosa que ofrezca el mundo. No dejen que
sus mentes estén ansiosas, y no tengan miedo. \bibleverse{28} ``Ustedes
me han escuchado decirles `Me voy pero regresaré a ustedes'. Si ustedes
realmente me amaran, estarían felices porque voy al Padre, pues el Padre
es más grande que yo. \bibleverse{29} Yo les he explicado esto ahora,
antes de que ocurra, para que cuando ocurra estén convencidos.
\bibleverse{30} Ahora no puedo hablarles por más tiempo, porque el
príncipe de este mundo se acerca. Él no tiene poder para controlarme,
\footnote{\textbf{14:30} Juan 12,31; Efes 2,2} \bibleverse{31} pero yo
estoy haciendo lo que mi Padre me dijo que hiciera, a fin de que el
mundo sepa que yo amo al Padre. Ahora levántense. Vámonos''.\footnote{\textbf{14:31}
  Juan 10,18}

\hypertarget{paruxe1bola-de-la-vid-y-las-ramas}{%
\subsection{Parábola de la vid y las
ramas}\label{paruxe1bola-de-la-vid-y-las-ramas}}

\hypertarget{section-14}{%
\section{15}\label{section-14}}

\bibleverse{1} ``Yo soy la vid verdadera y mi padre es el jardinero.
\bibleverse{2} Él corta de mí cada una de las ramas que no llevan fruto.
Él poda las ramas que llevan fruto a fin de que lleven mucho más fruto.
\bibleverse{3} Ustedes ya están podados y limpios\footnote{\textbf{15:3}
  La palabra que se usa como ``podar'' en este contexto significa
  Literalmente, ``limpiar''.} por lo que les he dicho. \footnote{\textbf{15:3}
  Juan 13,10; 1Pe 1,23} \bibleverse{4} Permanezcan en mí, y yo
permaneceré en ustedes.\footnote{\textbf{15:4} Obviamente, la palabra
  ``en'' debe tomarse como ``en conexión con'' tal como lo deja claro el
  resto del versículo.} Así como una rama no puede producir fruto a
menos que permanezca siendo parte de la vid, así ocurre con ustedes: no
pueden llevar fruto a menos que permanezcan en mí. \bibleverse{5} Yo soy
la vid y ustedes las ramas. Los que permanezcan en mí, y yo en ellos,
producirán mucho fruto---porque lejos de mí, ustedes no pueden hacer
nada. \bibleverse{6} Todo aquél que no permanece en mí es como una rama
que es cortada y se seca. Tales ramas se juntan, son lanzadas al fuego y
quemadas. \bibleverse{7} Si ustedes permanecen en mí, y mis palabras en
ustedes, entonces pueden pedir cualquier cosa que quieran, y les será
dada. \footnote{\textbf{15:7} Mar 11,24}

\bibleverse{8} Mi Padre es glorificado cuando ustedes producen mucho
fruto, demostrando que son mis discípulos. \footnote{\textbf{15:8} Mat
  5,16}

\hypertarget{el-mandamiento-del-amor-permanezcan-en-la-comunidad-de-amor-conmigo-y-entre-nosotros}{%
\subsection{El mandamiento del amor: ¡Permanezcan en la comunidad de
amor conmigo y entre
nosotros!}\label{el-mandamiento-del-amor-permanezcan-en-la-comunidad-de-amor-conmigo-y-entre-nosotros}}

\bibleverse{9} ``Así como me amó el Padre, yo los he amado a ustedes.
\bibleverse{10} Si ustedes hacen lo que yo digo, permanecerán en mi
amor, así como yo hago lo que mi Padre dice y permanezco en su amor.
\bibleverse{11} Les he explicado esto para que mi alegría esté en
ustedes y así su alegría esté completa. \footnote{\textbf{15:11} Juan
  17,13}

\bibleverse{12} ``Este es mi mandato: ámense unos a otros como yo los he
amado a ustedes. \footnote{\textbf{15:12} Juan 13,34} \bibleverse{13} No
hay amor más grande que dar la vida por los amigos. \footnote{\textbf{15:13}
  Juan 10,12; 1Jn 3,16} \bibleverse{14} Ustedes son mis amigos si hacen
lo que yo les digo. \footnote{\textbf{15:14} Juan 8,31; Mat 12,50}
\bibleverse{15} Yo no los llamaré más siervos, porque los siervos no son
considerados como de confianza por su amo.\footnote{\textbf{15:15}
  Literalmente, ``Un siervo no sabe lo que hace su señor''.} Yo los
llamo amigos, porque todo lo que mi Padre me dijo yo se los he dicho a
ustedes. \bibleverse{16} Ustedes no me eligieron a mí, yo los elegí a
ustedes. Yo les he dado a ustedes la responsabilidad de ir y producir
fruto duradero. Entonces el Padre les dará todo lo que pidan en mi
nombre.

\bibleverse{17} Este es mi mandato para ustedes: ámense los unos a los
otros.

\hypertarget{profecuxeda-del-destino-de-los-discuxedpulos-sufriendo-el-odio-del-mundo}{%
\subsection{Profecía del destino de los discípulos, sufriendo el odio
del
mundo}\label{profecuxeda-del-destino-de-los-discuxedpulos-sufriendo-el-odio-del-mundo}}

\bibleverse{18} ``Si el mundo los odia, recuerden que ya me odió a mi
antes que a ustedes. \footnote{\textbf{15:18} Juan 7,7} \bibleverse{19}
Si ustedes fueran parte de este mundo, el mundo los amaría como parte
suya. Pero ustedes no son parte del mundo, y yo los separé del
mundo---por eso el mundo los odia. \footnote{\textbf{15:19} 1Jn 4,4; 1Jn
  1,4-5; Juan 17,14} \bibleverse{20} ``Recuerden lo que les dije: los
siervos no son más importantes que su amo. Si ellos me persiguen a mí,
los perseguirán a ustedes también. Si hicieron lo que yo les dije, harán
lo que ustedes les digan también. \footnote{\textbf{15:20} Juan 13,16;
  Mat 10,24-25} \bibleverse{21} Pero todo lo que les hagan a ustedes
será por mi causa, porque ellos no conocen a Aquél que me envió.
\footnote{\textbf{15:21} Juan 16,3} \bibleverse{22} Si yo no hubiera
venido a hablarles, ellos no serían culpables de pecado---pero ahora
ellos no tienen excusa para su pecado. \footnote{\textbf{15:22} Juan
  9,41} \bibleverse{23} Cualquiera que me odia, odia también a mi Padre.
\footnote{\textbf{15:23} Luc 10,16} \bibleverse{24} Si yo no les hubiera
dado una demostración mediante cosas que nadie ha hecho antes, ellos no
serían culpables de pecado; pero a pesar de haber visto todo esto, me
odiaron a mí y también a mi Padre. \bibleverse{25} Pero esto solo es
cumplimiento de lo que dice la Escritura: `Ellos me odiaron sin ninguna
razón'.\footnote{\textbf{15:25} Citando Salmos 35:19 o Salmos 69:5.}

\bibleverse{26} ``Pero yo les enviaré al Consolador de parte del Padre.
Cuando él venga, les dará testimonio de mí. Él es el Espíritu de verdad
que viene del Padre. \footnote{\textbf{15:26} Juan 14,16; Juan 14,26;
  Luc 24,49} \bibleverse{27} Ustedes también darán testimonio de mí
porque ustedes estuvieron conmigo desde el principio''.\footnote{\textbf{15:27}
  Hech 1,8; Hech 1,21-22; Hech 5,32}

\hypertarget{section-15}{%
\section{16}\label{section-15}}

\bibleverse{1} ``Yo les he dicho esto para que no abandonen su confianza
en mí. \bibleverse{2} Ellos los expulsarán de las sinagogas---de hecho,
viene el tiempo en que las personas que los maten, pensarán que están
sirviendo a Dios. \footnote{\textbf{16:2} Mat 10,17; Mat 10,22; Mat 24,9}
\bibleverse{3} Y harán esto porque nunca han conocido al Padre ni a mí.
Les he dicho esto para que cuando estas cosas ocurran, recuerden lo que
les dije. \footnote{\textbf{16:3} Juan 15,21} \bibleverse{4} Yo no
necesitaba decirles esto al comienzo porque yo iba a estar con ustedes.

\hypertarget{promesa-del-espuxedritu-santo-y-su-obra-benuxe9fica-en-el-mundo-y-en-los-discuxedpulos}{%
\subsection{Promesa del Espíritu Santo y su obra benéfica en el mundo y
en los
discípulos}\label{promesa-del-espuxedritu-santo-y-su-obra-benuxe9fica-en-el-mundo-y-en-los-discuxedpulos}}

\bibleverse{5} Pero ahora voy al que me envió, aunque ninguno de ustedes
me está preguntando a dónde voy. \bibleverse{6} Por supuesto, ahora que
les he dicho, están acongojados. \bibleverse{7} ``Pero les digo la
verdad: es mejor para ustedes que yo me vaya, porque si no me voy, el
Consolador no vendría a ustedes. Si yo me voy, lo enviaré a ustedes.
\footnote{\textbf{16:7} Juan 14,16; Juan 14,26} \bibleverse{8} Y cuando
él venga, convencerá a los que están en el mundo de que tienen ideas
equivocadas sobre el pecado, sobre lo que es correcto y sobre el juicio.
\bibleverse{9} Sobre el pecado, porque no creen en mí. \bibleverse{10}
Sobre lo que es correcto, porque yo voy al Padre y ustedes no me verán
por más tiempo. \footnote{\textbf{16:10} Hech 5,31; Rom 4,25}
\bibleverse{11} Sobre el juicio, porque el gobernante de este mundo ha
sido condenado.\footnote{\textbf{16:11} O ``juzgado''.} \footnote{\textbf{16:11}
  Juan 12,31}

\bibleverse{12} ``Hay muchas cosas más que quiero explicarles, pero no
podrían entenderlas ahora. \footnote{\textbf{16:12} 1Cor 3,1}
\bibleverse{13} Sin embargo, cuando el Espíritu de verdad venga, él les
enseñará toda la verdad. Él no habla por su propia cuenta, sino que solo
dice lo que escucha, y les dirá lo que va a suceder. \footnote{\textbf{16:13}
  Juan 14,26; 1Jn 2,27} \bibleverse{14} Él me trae gloria porque él les
enseña todo lo que recibe de mí. \bibleverse{15} Todo lo que pertenece
al Padre es mío. Es por esto que les dije que el Espíritu les enseñará a
ustedes lo que reciba de mí. \footnote{\textbf{16:15} Juan 3,35; Juan
  17,10}

\hypertarget{promesa-de-una-reuniuxf3n-temprana-y-amonestaciuxf3n-de-orar-en-el-nombre-de-jesuxfas}{%
\subsection{Promesa de una reunión temprana y amonestación de orar en el
nombre de
Jesús}\label{promesa-de-una-reuniuxf3n-temprana-y-amonestaciuxf3n-de-orar-en-el-nombre-de-jesuxfas}}

\bibleverse{16} Dentro de poco ustedes no me verán más, pero dentro de
poco me verán otra vez''. \footnote{\textbf{16:16} Juan 14,19}

\bibleverse{17} Algunos de sus discípulos se decían unos a otros: ``¿Qué
quiere decir cuando dice `dentro de poco no me verán más, pero dentro de
poco me verán otra vez'? ¿Y cuando dice `porque voy al Padre'?''
\bibleverse{18} Ellos se preguntaban ``¿Qué quiere decir cuando dice
`dentro de poco'? No sabemos de qué está hablando''.

\bibleverse{19} Jesús se dio cuenta de que ellos querían preguntarle
acerca de esto. Así que les preguntó: ``¿Están inquietos por que dije
`dentro de poco no me verán más, pero dentro de poco otra vez me verán'?
\bibleverse{20} Les digo la verdad, y es que ustedes van a llorar y
lamentarse, pero el mundo se alegrará. Ustedes estarán afligidos, pero
su aflicción se convertirá en alegría. \footnote{\textbf{16:20} Mar
  16,10} \bibleverse{21} Una mujer que está en proceso de parto sufre de
dolores porque ha llegado el momento, pero cuando el bebé nace, ella
olvida la agonía por la alegría de que ha llegado un niño al mundo.
\footnote{\textbf{16:21} Is 26,17} \bibleverse{22} Sí, ustedes se
lamentan ahora, pero yo los veré otra vez; y ustedes se alegrarán y
nadie les podrá arrebatar su alegría.

\bibleverse{23} ``Cuando llegue el momento, no tendrán necesidad de
preguntarme nada. Les digo la verdad, el Padre les dará cualquier cosa
que pidan en mi nombre. \footnote{\textbf{16:23} Juan 14,13-14}
\bibleverse{24} Hasta ahora ustedes no han pedido nada en mi nombre, así
que pidan y recibirán, y su alegría estará completa. \footnote{\textbf{16:24}
  Juan 15,11}

\hypertarget{promesa-de-completar-la-comuniuxf3n-con-dios-para-los-discuxedpulos-conclusiuxf3n-de-los-discursos-de-despedida}{%
\subsection{Promesa de completar la comunión con Dios para los
discípulos; Conclusión de los discursos de
despedida}\label{promesa-de-completar-la-comuniuxf3n-con-dios-para-los-discuxedpulos-conclusiuxf3n-de-los-discursos-de-despedida}}

\bibleverse{25} He estado hablándoles mediante un lenguaje figurado.
Pero dentro de poco dejaré de usar el lenguaje figurado cuando hable con
ustedes. En lugar de ello, voy a mostrarles al Padre claramente.
\bibleverse{26} ``En ese momento, pedirán en mi nombre. No les estoy
diciendo que yo rogaré al Padre en favor de ustedes, \bibleverse{27}
porque el Padre mismo los ama---porque ustedes me aman y creen que vine
de parte de Dios. \footnote{\textbf{16:27} Juan 14,21} \bibleverse{28}
Yo dejé al Padre y vine al mundo; ahora dejo el mundo y regreso a mi
Padre''.

\bibleverse{29} Entonces los discípulos dijeron: ``Ahora estás
hablándonos claramente y no estás usando lenguaje figurado.
\bibleverse{30} Ahora estamos seguros de que lo sabes todo, y que para
conocer las preguntas que tiene la gente, no necesitas
preguntarles.\footnote{\textbf{16:30} Refiriéndose a lo que había
  ocurrido en el versículo 16:19.} Esto nos convence de que viniste de
Dios''.

\bibleverse{31} ``¿Están realmente convencidos ahora?'' preguntó Jesús.
\bibleverse{32} ``Se acerca el momento---de hecho está a punto de
ocurrir---cuando ustedes se separarán; cada uno de ustedes irá a su
propia casa, dejándome solo. Pero yo no estoy realmente solo, porque el
Padre está conmigo. \bibleverse{33} Les he dicho todo esto a fin de que
tengan paz porque ustedes son uno conmigo.\footnote{\textbf{16:33}
  Literalmente, ``Paz en mí''.} Ustedes sufrirán en este mundo, pero
sean valientes--- ¡Yo he derrotado al mundo!''\footnote{\textbf{16:33}
  Juan 14,27; Rom 5,1; 1Jn 5,4}

\hypertarget{oraciuxf3n-de-despedida-de-jesuxfas-con-los-suyos-y-para-los-suyos}{%
\subsection{Oración de despedida de Jesús con los suyos y para los
suyos}\label{oraciuxf3n-de-despedida-de-jesuxfas-con-los-suyos-y-para-los-suyos}}

\hypertarget{section-16}{%
\section{17}\label{section-16}}

\bibleverse{1} Cuando Jesús terminó de decir esto, levantó su Mirada al
cielo y dijo: ``Padre, ha llegado el momento. Glorifica a tu Hijo para
que el Hijo pueda glorificarte. \bibleverse{2} Porque tú le has dado
autoridad sobre todas las personas para que él pueda darle vida eterna a
todos los que tú le has entregado. \bibleverse{3} La vida eterna es
esta: conocerte, a ti que eres el único Dios verdadero, y a Jesucristo a
quien enviaste. \footnote{\textbf{17:3} 1Jn 5,20} \bibleverse{4} Yo te
he dado gloria aquí en la tierra al terminar la obra que me mandaste a
hacer. \bibleverse{5} Ahora, Padre, glorifícame ante ti con la gloria
que tuve contigo antes de la creación del mundo.

\hypertarget{la-intercesiuxf3n-de-jesuxfas-por-el-mantenimiento-de-los-discuxedpulos-en-el-conocimiento-correcto-de-dios}{%
\subsection{La intercesión de Jesús por el mantenimiento de los
discípulos en el conocimiento correcto de
Dios}\label{la-intercesiuxf3n-de-jesuxfas-por-el-mantenimiento-de-los-discuxedpulos-en-el-conocimiento-correcto-de-dios}}

\bibleverse{6} ``Yo he revelado tu carácter\footnote{\textbf{17:6} O
  ``nombre''.} a aquellos que me diste del mundo. Ellos te pertenecían;
me los diste a mí, y he hecho lo que tú dijiste. \bibleverse{7} Ahora
ellos saben que todo lo que me has dado viene de ti. \bibleverse{8} Yo
les he dado el mensaje que tú me diste a mí. Ellos lo aceptaron, estando
completamente convencidos de que vine de ti, y ellos creyeron que tú me
enviaste. \footnote{\textbf{17:8} Juan 16,30} \bibleverse{9} Estoy
orando por ello---no por el mundo, sino por los que me diste, porque
ellos te pertenecen. \footnote{\textbf{17:9} Juan 6,37; Juan 6,44}
\bibleverse{10} Todos los que me pertenecen son tuyos, y los que te
pertenecen a ti son míos, y yo he sido glorificado por medio de ellos.
\footnote{\textbf{17:10} Juan 16,15} \bibleverse{11} ``Dejo el mundo,
pero ellos seguirán en el mundo mientras yo regreso a ti. Padre Santo,
protégelos en tu nombre, el nombre que me diste a mí, para que ellos
sean uno, así como nosotros somos uno. \bibleverse{12} Mientras estuve
con ellos, los protegí en tu nombre, el nombre que me diste. Cuidé de
ellos para que ninguno se perdiera, excepto el `hijo de perdición,' para
que se cumpliera la Escritura. \bibleverse{13} ``Ahora vuelvo a ti y
digo estas cosas mientras estoy aún en el mundo para que ellos puedan
compartir completamente mi alegría. \footnote{\textbf{17:13} Juan 15,11}
\bibleverse{14} Les di tu mensaje, y el mundo los odió porque ellos no
son del mundo, así como yo no soy del mundo. \footnote{\textbf{17:14}
  Juan 15,19} \bibleverse{15} No te estoy pidiendo que los quites del
mundo, sino que los protejas del maligno. \footnote{\textbf{17:15} Mat
  6,13; 2Tes 3,3} \bibleverse{16} Ellos no son del mundo, así como yo no
soy del mundo. \bibleverse{17} Santifícalos por la verdad; tu palabra es
verdad. \bibleverse{18} Así como tú me enviaste al mundo, yo los he
enviado al mundo. \footnote{\textbf{17:18} Juan 20,21} \bibleverse{19}
Yo me consagro\footnote{\textbf{17:19} ``Consagrar'': esta es la misma
  palabra que se traduce como ``santificar'' en el versículo 17.} a mí
mismo por ellos para que ellos también sean verdaderamente santos.
\footnote{\textbf{17:19} Heb 10,10}

\hypertarget{intercesiuxf3n-por-todos-los-creyentes}{%
\subsection{Intercesión por todos los
creyentes}\label{intercesiuxf3n-por-todos-los-creyentes}}

\bibleverse{20} ``No solo estoy orando por ellos, también oro por los
que crean en mí por el mensaje de ellos. \footnote{\textbf{17:20} Rom
  10,17} \bibleverse{21} Oro para que todos puedan ser uno, así como tú,
Padre, vives en mí y yo vivo en ti, para que ellos también puedan vivir
en nosotros a fin de que el mundo crea que tú me enviaste. \footnote{\textbf{17:21}
  Gal 3,28} \bibleverse{22} Yo les he dado la gloria que tú me diste,
para que puedan ser uno, así como nosotros somos uno. \footnote{\textbf{17:22}
  Hech 4,32} \bibleverse{23} Yo vivo en ellos, y tú vives en mí. Que
ellos puedan ser uno completamente, para que el mundo entero sepa que tú
me enviaste, y que tú los amas, así como me amas a mí. \footnote{\textbf{17:23}
  1Cor 6,17} \bibleverse{24} ``Padre, quiero que los que me has dado
estén conmigo donde yo esté, para que puedan ver la gloria que me
diste---porque tú me amaste antes de que el mundo fuera creado.
\footnote{\textbf{17:24} Juan 12,26}

\bibleverse{25} Padre bueno,\footnote{\textbf{17:25} Literalmente,
  ``Padre Justo''.} el mundo no te conoce, pero yo te conozco, y estos
que están aquí ahora conmigo saben que tú me enviaste. \bibleverse{26}
Yo les he mostrado tu carácter y seguiré dándolo a conocer, para que el
amor que tienes por mí esté en ellos, y yo viviré en ellos''.

\hypertarget{jesuxfas-en-getsemanuxed-judas-malco-arresto-de-jesuxfas}{%
\subsection{Jesús en Getsemaní: Judas, Malco, arresto de
Jesús}\label{jesuxfas-en-getsemanuxed-judas-malco-arresto-de-jesuxfas}}

\hypertarget{section-17}{%
\section{18}\label{section-17}}

\bibleverse{1} Después que Jesús hubo terminado de hablar, él y sus
discípulos cruzaron el arroyo de Cedrón y entraron a un olivar.
\bibleverse{2} Judas, el traidor, conocía el lugar porque Jesús había
ido allí a menudo con sus discípulos. \bibleverse{3} Entonces Judas
llevó consigo una tropa de soldados y guardias enviados de parte de los
jefes de los sacerdotes y los Fariseos. Llegaron al lugar con antorchas,
lámparas y armas. \bibleverse{4} Jesús sabía todo lo que le iba a pasar.
Así que fue a recibirlos y preguntó: ``¿A quién buscan ustedes?''

\bibleverse{5} ``¿Eres tú Jesús de Nazaret?'' dijeron ellos. ``Yo soy'',
les dijo Jesús.\footnote{\textbf{18:5} Las palabras de Jesús no son
  solamente una afirmación de su identidad sino también un eco del
  nombre de Dios que aparece desde el Éxodo.} Judas, el traidor, estaba
con ellos.

\bibleverse{6} Cuando Jesús dijo ``Yo soy'', ellos retrocedieron y
cayeron al suelo.

\bibleverse{7} Entonces él les preguntó nuevamente: ``¿A quién buscan?''
``¿Eres tú Jesús de Nazaret?'' le preguntaron una vez más.

\bibleverse{8} ``Ya les dije que yo soy'', respondió Jesús. ``Así que si
es a mí a quien buscan, dejen ir a estos que están aquí''.
\bibleverse{9} Estas palabras cumplieron lo que él había dicho
anteriormente: ``No he dejado perder a ninguno de los que me diste''.
\footnote{\textbf{18:9} Juan 17,12}

\bibleverse{10} Entonces Simón Pedro sacó una espada e hirió a Malco, el
siervo del Sumo sacerdote, cortándole la oreja derecha. \bibleverse{11}
Jesús le dijo a Pedro: ``¡Guarda esa espada! ¿Crees\footnote{\textbf{18:11}
  ``Piensas''---implícito.} que no debo beber la copa que mi Padre me ha
dado?''

\bibleverse{12} Entonces los soldados, su comandante y los guardias
judíos arrestaron a Jesús y ataron sus manos. \bibleverse{13} Primero lo
llevaron ante Anás, quien era el suegro de Caifás, el actual Sumo
sacerdote. \bibleverse{14} Caifás fue el que dijo a los judíos: ``Es
mejor que muera un solo hombre por el pueblo''.\footnote{\textbf{18:14}
  Ver 11:50.}

\hypertarget{primera-negaciuxf3n-de-pedro}{%
\subsection{Primera negación de
Pedro}\label{primera-negaciuxf3n-de-pedro}}

\bibleverse{15} Simón Pedro siguió a Jesús, y otro discípulo también lo
hizo. Este discípulo era muy conocido por el Sumo sacerdote, y por eso
entró al patio del Sumo sacerdote con Jesús. \bibleverse{16} Pedro tuvo
que permanecer fuera, cerca de la puerta. Entonces el otro discípulo,
que era conocido del Sumo sacerdote, fue y habló con la criada que
cuidaba de la puerta, e hizo entrar a Pedro. \bibleverse{17} La criada
le preguntó a Pedro: ``¿No eres tú uno de los discípulos de ese
hombre?'' ``¿Yo? No, no lo soy'', respondió.

\bibleverse{18} Hacía frío y los siervos y guardias estaban junto a una
fogata que habían hecho para calentarse. Pedro se les acercó y se quedó
allí con ellos, calentándose también.

\hypertarget{jesuxfas-ante-los-sumos-sacerdotes-anuxe1s-y-caifuxe1s}{%
\subsection{Jesús ante los sumos sacerdotes Anás y
Caifás}\label{jesuxfas-ante-los-sumos-sacerdotes-anuxe1s-y-caifuxe1s}}

\bibleverse{19} Entonces el jefe de los sacerdotes interrogó a Jesús
sobre sus discípulos y lo que él había estado enseñando.

\bibleverse{20} ``Yo le he hablado abiertamente a todos'',\footnote{\textbf{18:20}
  Literalmente, ``al mundo''.} respondió Jesús. ``Siempre enseñé en las
sinagogas y en el Templo, donde se reunían todos los judíos. No he dicho
nada en secreto. \footnote{\textbf{18:20} Juan 7,14; Juan 7,26}
\bibleverse{21} Entonces ¿por qué me interrogan? Pregúntenles a las
personas que me escucharon lo que les dije. Ellos saben lo que dije''.

\bibleverse{22} Cuando él dijo esto, uno de los guardias que estaba
cerca le dio una bofetada a Jesús, diciendo: ``¿Es esa la manera de
hablarle al Sumo sacerdote?''

\bibleverse{23} Jesús respondió: ``Si he dicho algo malo, díganle a
todos qué fue lo que dije. Pero si lo que dije estuvo bien, ¿por qué me
golpeaste?''

\bibleverse{24} Anás lo envió, con las manos atadas, ante Caifás, el
Sumo sacerdote.

\hypertarget{segunda-y-tercera-negaciuxf3n-de-pedro}{%
\subsection{Segunda y tercera negación de
Pedro}\label{segunda-y-tercera-negaciuxf3n-de-pedro}}

\bibleverse{25} Mientras Simón Pedro estaba calentándose cerca a la
fogata, las personas que estaban allí le preguntaron: ``¿No eres tú uno
de sus discípulos?'' Pedro lo negó y dijo: ``No, no lo soy''.

\bibleverse{26} Uno de los siervos del sumo sacerdote, que era familiar
del hombre a quien Pedro le había cortado la oreja, le preguntó a Pedro:
``¿Acaso no te vi en el olivar con él?''

\bibleverse{27} Pedro lo negó una vez más, e inmediatamente un galló
cantó.

\hypertarget{el-interrogatorio-y-la-confesiuxf3n-de-jesuxfas-ante-el-gobernador-romano-pilato-su-flagelaciuxf3n-burla-y-condena}{%
\subsection{El interrogatorio y la confesión de Jesús ante el gobernador
romano Pilato; su flagelación, burla y
condena}\label{el-interrogatorio-y-la-confesiuxf3n-de-jesuxfas-ante-el-gobernador-romano-pilato-su-flagelaciuxf3n-burla-y-condena}}

\bibleverse{28} Temprano en la mañana, llevaron a Jesús de donde Caifás
hasta el palacio del gobernador romano. Los líderes judíos\footnote{\textbf{18:28}
  Implícito.} no entraron al palacio, porque si lo hacían se
contaminarían ceremonialmente, y ellos querían estar aptos para comer la
Pascua. \bibleverse{29} Entonces Pilato salió a recibirlos. ``¿Qué
cargos traen en contra de este hombre?'' preguntó él.

\bibleverse{30} ``Si no fuera un criminal, no lo habríamos traído ante
ti'', respondieron ellos.

\bibleverse{31} ``Entonces llévenselo y júzguenlo conforme a la ley de
ustedes'', les dijo Pilato. ``No se nos permite ejecutar a nadie'',
respondieron los judíos.

\bibleverse{32} Esto cumplía lo que Jesús había dicho acerca de la
manera en que iba a morir. \footnote{\textbf{18:32} Juan 12,32-33; Mat
  20,19}

\bibleverse{33} Pilato regresó al palacio del gobernador. Llamó a Jesús
y le preguntó: ``¿Eres tú el rey de los judíos?''

\bibleverse{34} ``¿Se te ocurrió a ti mismo esta pregunta, o ya otros te
han hablado de mí?'' respondió Jesús.

\bibleverse{35} ``¿Soy yo un judío acaso?'' argumentó Pilato. ``Fue tu
propio pueblo y también los sumos sacerdotes quienes te trajeron aquí
ante mí. ¿Qué es lo que has hecho?''

\bibleverse{36} Jesús respondió: ``Mi reino no es de este mundo. Si mi
reino fuera de este mundo, mis súbditos pelearían para protegerme de los
judíos. Pero mi reino no es de aquí''.

\bibleverse{37} Entonces Pilato preguntó: ``¿Entonces eres un rey?''
``Tú dices que yo soy un rey'', respondió Jesús. ``La razón por la que
nací y vine al mundo fue para dar evidencia en favor de la verdad. Todos
los que aceptan la verdad, atienden lo que yo digo''.

\bibleverse{38} ``¿Qué es verdad?'' preguntó Pilato. Habiendo dicho
esto, Pilato regresó afuera, donde estaban los judíos, y les dijo: ``Yo
no lo encuentro culpable de ningún crimen.

\bibleverse{39} Sin embargo, como es costumbre liberar a un prisionero
para la fiesta de la Pascua, ¿quieren que libere al rey de los judíos?''

\bibleverse{40} ``¡No, no lo sueltes a él! ¡Preferimos que sueltes a
Barrabás!'' volvieron a gritar. Barrabás era un rebelde.\footnote{\textbf{18:40}
  A menudo se traduce como ``ladrón''. Es posible que Barrabás hubiera
  sido parte de algún amotinamiento.}

\hypertarget{section-18}{%
\section{19}\label{section-18}}

\bibleverse{1} Entonces Pilato llevó a Jesús y mandó que lo azotaran.
\bibleverse{2} Los soldados hicieron una corona de espinas y la pusieron
sobre su cabeza, y lo vistieron con una túnica de color púrpura.
\bibleverse{3} Una y otra vez iban a él y le decían: ``¡Oh, Rey de los
Judíos!'' y lo abofeteaban.

\bibleverse{4} Pilato salió una vez más y les dijo: ``Lo traeré aquí
para que sepan que no lo encuentro culpable de ningún crimen''.

\bibleverse{5} Entonces Jesús salió usando la corona de espinas y la
túnica de color púrpura. ``Miren, aquí está el hombre'', dijo Pilato.

\bibleverse{6} Cuando el jefe de los sacerdotes y los guardias vieron a
Jesús, gritaron: ``¡Crucifícale! ¡Crucifícale!'' ``Llévenselo ustedes y
crucifíquenlo'', respondió Pilato. ``Yo no le hallo culpable''.

\bibleverse{7} Los líderes judíos respondieron: ``Tenemos una ley, y de
acuerdo a esa ley, él debe morir porque se proclamó a sí mismo como el
Hijo de Dios''. \footnote{\textbf{19:7} Juan 10,33; Lev 24,16}

\bibleverse{8} Cuando Pilato escuchó esto, tuvo más temor que nunca
antes \bibleverse{9} y regresó al palacio del gobernador. Pilato le
preguntó a Jesús, ``¿De dónde vienes?'' Pero Jesús no respondió.
\bibleverse{10} ``¿Estás negándote a hablarme?'' le dijo Pilato. ``¿No
te das cuenta de que tengo el poder para liberarte o crucificarte?''

\bibleverse{11} ``Tú no tendrías ningún poder a menos que se te conceda
desde arriba'', le respondió Jesús. ``Así que el que me entregó en tus
manos es culpable de mayor pecado''.

\bibleverse{12} Cuando Pilato escuchó esto, trató de liberar a Jesús,
pero los líderes judíos gritaban: ``Si liberas a este hombre, no eres
amigo del César. Cualquiera que se proclama a sí mismo como rey, se
rebela contra el César''.

\bibleverse{13} Cuando Pilato escuchó esto, trajo a Jesús afuera y se
sentó en el tribunal, en un lugar que se llamaba El Enlosado (``Gabata''
en Hebreo). \bibleverse{14} Era casi la tarde del día de preparación
para la Pascua. ``Miren, aquí tienen a su rey'', le dijo a los judíos.

\bibleverse{15} ``¡Mátalo! ¡Mátalo! ¡Crucifícalo!'' gritaban ellos.
``¿Quieren que crucifique a su rey?'' preguntó Pilato. ``El único rey
que tenemos es el César'', respondieron los jefes de los sacerdotes.
\footnote{\textbf{19:15} Juan 18,37}

\hypertarget{la-crucifixiuxf3n-y-muerte-de-jesuxfas}{%
\subsection{La crucifixión y muerte de
Jesús}\label{la-crucifixiuxf3n-y-muerte-de-jesuxfas}}

\bibleverse{16} Entonces Pilato les entregó a Jesús para que lo
crucificaran. \bibleverse{17} Ellos condujeron a Jesús fuera de allí,
cargando él su propia cruz, y se dirigió al lugar llamado ``La
Calavera'', (Gólgota en hebreo). \bibleverse{18} Lo crucificaron allí, y
a otros dos con él: uno a cada lado, poniendo a Jesús en medio de ellos.
\bibleverse{19} Pilato mandó a poner un letrero en la cruz que decía:
``Jesús de Nazaret, el Rey de los Judíos''. \bibleverse{20} Muchas
personas leyeron el letrero porque el lugar donde Jesús fue crucificado
estaba cerca de la ciudad, y estaba escrito en hebreo, latín y griego.
\bibleverse{21} Entonces los jefes de los sacerdotes se acercaron a
Pilato y le dijeron ``No escribas `el Rey de los Judíos,' sino `Este
hombre decía: Yo soy el Rey de los Judíos'\,''.

\bibleverse{22} Pilato respondió: ``Lo que escribí, ya está escrito''.

\bibleverse{23} Cuando los soldados hubieron crucificado a Jesús,
tomaron sus ropas y las dividieron en cuatro partes a fin de que cada
soldado tuviera una. También estaba allí su túnica hecha sin costuras,
tejida en una sola pieza. \bibleverse{24} Entonces ellos se dijeron unos
a otros: ``No la botemos, sino decidamos quién se quedará con ella
lanzando un dado''. Esto cumplía la Escritura que dice: ``Dividieron mis
vestidos entre ellos y lanzaron un dado por mis vestiduras''.\footnote{\textbf{19:24}
  Citando Salmos 22:18.}

\bibleverse{25} Y así lo hicieron. Junto a la cruz estaba la madre de
Jesús, la hermana de su madre, María la esposa de Cleofás y María
Magdalena.\footnote{\textbf{19:25} No está claro si había tres mujeres
  presentes o cuatro. Algunos creen que la hermana de María es la misma
  persona que María, esposa de Clopas.} \bibleverse{26} Cuando Jesús vio
a su madre, y al discípulo que él amaba junto a ella, le dijo a su
madre: ``Madre,\footnote{\textbf{19:26} Literalmente, ``mujer'', pero
  este término no tiene la misma función en español.} este es tu hijo''.
\bibleverse{27} Luego le dijo al discípulo: ``Esta es tu madre''. Desde
ese momento el discípulo se la llevó a su casa.

\bibleverse{28} Jesús se dio cuenta entonces que había completado todo
lo que había venido a hacer. En cumplimiento de la Escritura, dijo:
``Tengo sed''.\footnote{\textbf{19:28} Citando Salmos 69:21.}
\footnote{\textbf{19:28} Sal 22,16} \bibleverse{29} Y allí había una
tinaja llena de vinagre de vino; así que ellos mojaron una esponja en el
vinagre, la pusieron en una vara de hisopo, y la acercaron a sus labios.
\footnote{\textbf{19:29} Sal 69,22} \bibleverse{30} Después que bebió el
vinagre, Jesús dijo: ``¡Está terminado!'' Entonces inclinó su cabeza y
dio su último respiro.

\bibleverse{31} Era el día de la preparación, y los líderes judíos no
querían dejar los cuerpos en la cruz durante el día sábado (de hecho,
este era un sábado especial), así que le pidieron a Pilato que mandara a
partirles las piernas para poder quitar los cuerpos. \footnote{\textbf{19:31}
  Lev 23,7; Deut 21,23} \bibleverse{32} Entonces los soldados vinieron y
partieron las piernas del primero y luego del otro, de los dos hombres
crucificados con Jesús, \bibleverse{33} pero cuando se acercaron a
Jesús, vieron que ya estaba muerto, así que no le partieron sus piernas.
\bibleverse{34} Sin embargo, uno de los soldados clavó una lanza en su
costado, y salió sangre mezclada con agua. \bibleverse{35} El que vio
esto dio testimonio de ello, y su testimonio es verdadero. Él está
seguro de que lo que dice es verdadero a fin de que ustedes crean
también. \bibleverse{36} Ocurrió así para que se cumpliera la Escritura:
``Ninguno de sus huesos será partido'', \bibleverse{37} y como dice otra
Escritura: ``Ellos mirarán al que traspasaron''.\footnote{\textbf{19:37}
  Refiriéndose a Éxodo 12:46, Números 9:12, o Salmos 34:20.}

\hypertarget{descenso-de-la-cruz-y-sepultura-de-jesuxfas}{%
\subsection{Descenso de la cruz y sepultura de
Jesús}\label{descenso-de-la-cruz-y-sepultura-de-jesuxfas}}

\bibleverse{38} Después de esto, José de Arimatea le preguntó a Pilato
si podría bajar el cuerpo de Jesús, y Pilato le dio su permiso. José era
un discípulo de Jesús, pero en secreto porque tenía miedo de los judíos.
Así que José fue y se llevó el cuerpo. \footnote{\textbf{19:38} Juan
  7,13} \bibleverse{39} Con él estaba Nicodemo, el hombre que había
visitado de noche a Jesús anteriormente. Él trajo consigo una mezcla de
mirra y aloes que pesaba aproximadamente setenta y cinco libras.
\footnote{\textbf{19:39} Juan 3,2}

\bibleverse{40} Ellos se llevaron el cuerpo de Jesús y lo envolvieron en
un paño de lino junto con la mezcla de especias, conforme a la costumbre
judía de sepultura. Cerca del lugar donde Jesús había sido crucificado,
había un jardín; \bibleverse{41} y en ese jardín había una tumba nueva,
sin usar. \bibleverse{42} Como era el día de la preparación y la tumba
estaba cerca, ellos pusieron allí a Jesús.

\hypertarget{maruxeda-magdalena-y-el-sepulcro-vacuxedo-pedro-y-juan-en-la-tumba}{%
\subsection{María Magdalena y el sepulcro vacío; Pedro y Juan en la
tumba}\label{maruxeda-magdalena-y-el-sepulcro-vacuxedo-pedro-y-juan-en-la-tumba}}

\hypertarget{section-19}{%
\section{20}\label{section-19}}

\bibleverse{1} Temprano, el primer día de la semana,\footnote{\textbf{20:1}
  Es decir, domingo.} mientras aún estaba oscuro, María Magdalena fue a
la tumba y vio que habían movido la piedra que estaba a la entrada.
\bibleverse{2} Entonces ella salió corriendo para decirle a Simón Pedro
y al otro discípulo, al que Jesús amaba: ``Se han llevado al Señor de la
tumba, y no sabemos dónde lo han puesto''. \footnote{\textbf{20:2} Juan
  13,23}

\bibleverse{3} Entonces Pedro y el otro discípulo fueron a la tumba.
\bibleverse{4} Ambos iban corriendo, pero el otro discípulo corrió más
rápido y llegó primero. \bibleverse{5} Se agachó, y al mirar hacia
adentro, vio que los paños fúnebres estaban allí, pero no entró.
\bibleverse{6} Entonces Simón Pedro llegó después de él y entró a la
tumba. Vio los paños fúnebres de lino que estaban allí, \bibleverse{7} y
que el paño con que habían cubierto la cabeza de Jesús no estaba con los
demás paños fúnebres sino que lo habían doblado y lo habían colocado
solo aparte. \bibleverse{8} Entonces el otro discípulo que había llegado
primero a la tumba, entró también. \bibleverse{9} Miró alrededor y creyó
entonces que era verdad\footnote{\textbf{20:9} Que Jesús se había
  levantado de los muertos.} ---porque hasta ese momento ellos no habían
entendido la Escritura de que Jesús tenía que levantase de los muertos.
\footnote{\textbf{20:9} Luc 24,25-27; Hech 2,24-32; 1Cor 15,4}
\bibleverse{10} Entonces los discípulos regresaron al lugar donde se
estaban quedando.

\hypertarget{apariciuxf3n-de-jesuxfas-a-maruxeda-magdalena}{%
\subsection{Aparición de Jesús a María
Magdalena}\label{apariciuxf3n-de-jesuxfas-a-maruxeda-magdalena}}

\bibleverse{11} Pero María permaneció fuera de la tumba llorando, y
mientras lloraba, se agachó y miró hacia adentro de la tumba.
\bibleverse{12} Vio allí a dos ángeles vestidos de blanco, uno sentado a
la cabeza y el otro sentado a los pies del lugar donde había estado el
cuerpo de Jesús. \bibleverse{13} ``¿Por qué estás llorando?'' le
preguntaron. Ella respondió: ``Porque se han llevado a mi Señor, y no sé
dónde lo han puesto''.

\bibleverse{14} Después que dijo esto, volvió a mirar y vio a Jesús que
estaba allí, pero ella no se dio cuenta de que era Jesús.

\bibleverse{15} ``¿Por qué estás llorando?'' le preguntó él. ``¿A quién
estás buscando?'' Creyendo que era el jardinero, ella le dijo: ``Señor,
si te lo has llevado, dime dónde lo has puesto para yo ir a buscarlo''.

\bibleverse{16} Jesús le dijo: ``María''. Ella se dirigió hacia él y
dijo: ``Rabboni'', que significa ``Maestro'' en hebreo.

\bibleverse{17} ``Suéltame'',\footnote{\textbf{20:17} Queriendo decir:
  no me detengas sujetándome.} le dijo Jesús, ``porque aún no he
ascendido a mi Padre; más bien ve donde mis hermanos y diles que voy a
ascender a mi Padre, y Padre de ustedes, mi Dios y el Dios de ustedes''.

\bibleverse{18} Entonces María Magdalena fue y le dijo a los discípulos:
``He visto al Señor'', y les explicó lo que él le había dicho.

\hypertarget{jesuxfas-y-los-discuxedpulos-en-la-noche-del-domingo-de-pascua}{%
\subsection{Jesús y los discípulos en la noche del domingo de
Pascua}\label{jesuxfas-y-los-discuxedpulos-en-la-noche-del-domingo-de-pascua}}

\bibleverse{19} Esa noche, siendo el primer día de la semana, cuando los
discípulos se reunieron a puerta cerrada porque tenían mucho temor de
los judíos, Jesús llegó y se puso en medio de ellos y dijo: ``Tengan
paz''.

\bibleverse{20} Después de este saludo, les mostró sus manos y su
costado. Los discípulos estaban llenos de alegría por ver al Señor.
\footnote{\textbf{20:20} 1Jn 1,1} \bibleverse{21} ``¡Tengan paz!'' les
dijo Jesús otra vez. ``De la misma manera que el Padre me envió, así yo
los estoy enviando a ustedes''. \footnote{\textbf{20:21} Juan 17,18}
\bibleverse{22} Mientras decía esto, sopló sobre ellos y les dijo:
``Reciban el Espíritu Santo. \bibleverse{23} Si ustedes perdonan los
pecados a alguien, le serán perdonados; pero si ustedes no lo perdonan,
quedarán sin ser perdonados''. \footnote{\textbf{20:23} Mat 18,16}

\hypertarget{los-discuxedpulos-con-tomuxe1s}{%
\subsection{Los discípulos con
Tomás}\label{los-discuxedpulos-con-tomuxe1s}}

\bibleverse{24} Uno de los doce discípulos, Tomás, a quien le decían el
gemelo, no estaba allí cuando Jesús llegó. \footnote{\textbf{20:24} Juan
  11,16; Juan 14,5; Juan 21,2} \bibleverse{25} Así que los otros
discípulos le dijeron: ``Hemos visto al Señor''. Pero él respondió: ``No
lo creeré hasta que vea las marcas de los clavos en sus manos y ponga mi
dedo en ellas, y ponga mi mano en su costado''. \footnote{\textbf{20:25}
  Juan 19,34}

\bibleverse{26} Una semana después, los discípulos estaban reunidos
dentro de la casa y Tomás estaba con ellos. Las puertas estaban
cerradas, y Jesús llegó y se puso en medio de ellos. ``¡Tengan paz!''
dijo. \bibleverse{27} Entonces le dijo a Tomás: ``Coloca aquí tu dedo, y
mira mis manos. Coloca tu mano en la herida que tengo en mi costado.
¡Deja de dudar y cree en mí!

\bibleverse{28} ``¡Mi señor y mi Dios!'' respondió Tomás.

\bibleverse{29} ``Crees en mí porque me has visto'', le dijo Jesús.
``Felices aquellos que no han visto, y sin embargo aún creen en mí''.
\footnote{\textbf{20:29} 1Pe 1,8; Heb 11,1}

\bibleverse{30} Jesús hizo muchas otras señales milagrosas mientras
estuvo con los discípulos, y que no se registran en este libro.
\footnote{\textbf{20:30} Juan 21,24-25} \bibleverse{31} Pero estas cosas
están escritas aquí para que ustedes puedan creer que Jesús es el
Mesías, el Hijo de Dios, y que al creer en quien él es,\footnote{\textbf{20:31}
  Literalmente, ``en su nombre''.} ustedes tengan vida.\footnote{\textbf{20:31}
  1Jn 5,13}

\hypertarget{jesuxfas-se-revela-a-sus-discuxedpulos-en-el-lago-de-tiberuxedades}{%
\subsection{Jesús se revela a sus discípulos en el lago de
Tiberíades}\label{jesuxfas-se-revela-a-sus-discuxedpulos-en-el-lago-de-tiberuxedades}}

\hypertarget{section-20}{%
\section{21}\label{section-20}}

\bibleverse{1} Después Jesús se les apareció de nuevo a los discípulos
junto al Mar de Galilea.\footnote{\textbf{21:1} Literalmente, ``Mar de
  Tiberias''.} Así es como ocurrió: \bibleverse{2} Estaban juntos Simón
Pedro, Tomás el gemelo, Natanael de Caná de Galilea, los hijos de
Zebedeo y otros dos discípulos. \bibleverse{3} ``Voy a pescar'', dijo
Simón Pedro. ``Iremos contigo'', respondieron ellos. Entonces fueron y
se montaron en una barca, pero en toda la noche no atraparon nada.

\bibleverse{4} Cuando llegó el alba, Jesús estaba en la orilla, pero los
discípulos no sabían que era él. \footnote{\textbf{21:4} Juan 20,14; Luc
  24,16} \bibleverse{5} Jesús los llamó: ``Amigos, ¿no han atrapado
nada?'' ``No'', respondieron ellos. \footnote{\textbf{21:5} Luc 24,41}

\bibleverse{6} ``Lancen la red del lado derecho de la barca, y atraparán
algunos'', les dijo. Entonces ellos lanzaron la red, y no podían subirla
porque tenía muchos peces en ella. \footnote{\textbf{21:6} Luc 5,4-7}

\bibleverse{7} El discípulo a quien Jesús amaba le dijo a Pedro: ``Es el
Señor''. Cuando Pedro escuchó que era el Señor, se puso ropa, pues hasta
ese momento estaba desnudo, y se lanzó al mar. \footnote{\textbf{21:7}
  Juan 13,23}

\bibleverse{8} Los demás discípulos siguieron en la barca jalando la red
llena de peces, pues no estaban muy lejos de la orilla, apenas a unas
cien yardas. \bibleverse{9} Cuando llegaron a la orilla, vieron una
fogata con algunos peces cocinándose y además había panes.
\bibleverse{10} Jesús les dijo: ``Traigan algunos de los peces de los
que acaban de atrapar''.

\bibleverse{11} Simón Pedro subió a la barca y jaló la red llena de
peces hacia la orilla. Había 153 peces grandes, y sin embargo la red no
se había roto.

\bibleverse{12} ``Vengan y desayunen'', les dijo Jesús. Ninguno de los
discípulos fue capaz de preguntarle ``¿Quién eres?'' Ellos sabían que
era el Señor.

\bibleverse{13} Jesús tomó el pan y se los dio así como el pescado
también. \footnote{\textbf{21:13} Juan 6,11} \bibleverse{14} Esta fue la
tercera vez que Jesús se le apareció a los discípulos después de haberse
levantado de entre los muertos.

\hypertarget{trus-reinstalado-en-su-cargo-pastoral-profecuxeda-sobre-el-fin-de-la-vida-de-pedro-y-el-discuxedpulo-amado}{%
\subsection{Trus reinstalado en su cargo pastoral; Profecía sobre el fin
de la vida de Pedro y el discípulo
amado}\label{trus-reinstalado-en-su-cargo-pastoral-profecuxeda-sobre-el-fin-de-la-vida-de-pedro-y-el-discuxedpulo-amado}}

\bibleverse{15} Después del desayuno, Jesús le preguntó a Simón Pedro:
``Simón, hijo de Juan, ¿me amas más que estos?''\footnote{\textbf{21:15}
  ``Estos''. Esto podía referirse a los objetos que estaban a su
  alrededor, es decir, propios del negocio de pescador, pero es más
  probable que se refiera a los otros discípulos. Lo que estaba en
  cuestión era el amor de Pedro por Jesús, no el amor por los
  discípulos.} ``Sí, Señor'', respondió él, ``tú sabes que te amo'',

\bibleverse{16} ``Cuida de mi corderos'', le dijo Jesús. ``Simón, hijo
de Juan, ¿me amas?'' le preguntó por segunda vez. ``Sí, Señor'', le
respondió, ``tú sabes que te amo'', \footnote{\textbf{21:16} 1Pe 5,2;
  1Pe 5,4}

\bibleverse{17} ``Cuida de mis ovejas'', le dijo Jesús. ``Simón, hijo de
Juan, ¿me amas?'' le preguntó por tercera vez. Pedro estaba triste de
que Jesús le hubiera preguntado por tercera vez si él lo amaba. ``Señor,
tú lo sabes todo. Tú sabes que te amo'', le dijo Pedro. ``Cuida de mis
ovejas'', dijo Jesús. \footnote{\textbf{21:17} Juan 13,38; Juan 16,30}

\bibleverse{18} ``Te digo la verdad'', dijo Jesús, ``cuando estabas
joven, te vestías solo e ibas donde querías. Pero cuando estás viejo,
extiendes tus manos y otra persona te viste y vas donde no quieres ir''.

\bibleverse{19} Jesús decía esto para explicar la forma en que Pedro
glorificaría a Dios al morir. Luego le dijo a Pedro: ``Sígueme''.
\footnote{\textbf{21:19} Juan 13,36}

\bibleverse{20} Cuando Pedro se dio la vuelta, vio que el discípulo a
quien Jesús amaba los seguía, el que estaba junto a Jesús durante la
cena y que le preguntó, ``Señor, ¿quién va a traicionarte?'' \footnote{\textbf{21:20}
  Juan 13,23; Juan 13,25} \bibleverse{21} Pedro le preguntó a Jesús:
``¿Qué de él, Señor?''

\bibleverse{22} Jesús le dijo: ``Si yo quiero que él siga vivo hasta que
yo regrese, ¿por qué te preocupa eso a ti? ¡Tú sígueme!''
\bibleverse{23} Esta es la razón por la que se difundió el rumor entre
los creyentes de que este discípulo no moriría. Pero Jesús no dijo que
él no moriría, solo dijo ``si yo quiero que él siga vivo hasta que yo
regrese, ¿por qué te preocupa a ti?''

\bibleverse{24} Este es el discípulo que confirma lo que ocurrió y quien
escribió todas estas cosas. Sabemos que lo que él dice es verdad.
\bibleverse{25} Jesús hizo muchas otras cosas también, y si se
escribieran, dudo que el mundo entero pueda contener todos los libros
que se escribirían.
