\hypertarget{el-permiso-de-ciro-para-el-regreso-de-los-juduxedos-restantes-y-para-la-construcciuxf3n-del-templo}{%
\subsection{El permiso de Ciro para el regreso de los judíos restantes y
para la construcción del
templo}\label{el-permiso-de-ciro-para-el-regreso-de-los-juduxedos-restantes-y-para-la-construcciuxf3n-del-templo}}

\hypertarget{section}{%
\section{1}\label{section}}

\bibleverse{1} Para que se cumpliera la profecía del Señor dada a través
de Jeremías, el Señor animó a Ciro, rey de Persia, a emitir una proclama
en todo su reino y también a ponerla por escrito, diciendo: \footnote{\textbf{1:1}
  2Cró 36,22-23; Jer 25,11; Jer 29,10} \bibleverse{2} ``Esto es lo que
dice Ciro, rey de Persia: `El Señor, el Dios de los cielos, que me ha
dado todos los reinos de la tierra, me ha dado la responsabilidad de
construirle un Templo en Jerusalén, en Judá. \footnote{\textbf{1:2} Is
  44,28; Is 45,1} \bibleverse{3} Cualquiera de ustedes que pertenezca a
su pueblo puede ir a Jerusalén, en Judá, para reconstruir este Templo
del Señor, el Dios de Israel, que vive en Jerusalén. Que su Dios esté
con ustedes. \bibleverse{4} Donde quiera que los sobrevivientes vivan
actualmente, que sean ayudados por la gente de esa región con plata,
oro, bienes y ganado, junto con una donación voluntaria para el Templo
de Dios en Jerusalén'\,''.\footnote{\textbf{1:4} Esta proclamación de
  Ciro se encuentra también al final de 2 Crónicas.}

\hypertarget{el-efecto-y-ejecuciuxf3n-de-la-disposiciuxf3n}{%
\subsection{El efecto y ejecución de la
disposición}\label{el-efecto-y-ejecuciuxf3n-de-la-disposiciuxf3n}}

\bibleverse{5} Entonces Dios animó a los jefes de familia de Judá y
Benjamín, así como a los sacerdotes y a los levitas, a ir y reconstruir
el Templo del Señor en Jerusalén. \bibleverse{6} Todos sus vecinos les
apoyaron con regalos de plata y oro, con bienes y ganado, y con otros
objetos de valor, además de todas sus donaciones voluntarias.

\hypertarget{publicaciuxf3n-y-listado-de-los-implementos-del-templo-entregados-a-sesbassar-zorobabel}{%
\subsection{Publicación y listado de los implementos del templo
entregados a Sesbassar
(Zorobabel)}\label{publicaciuxf3n-y-listado-de-los-implementos-del-templo-entregados-a-sesbassar-zorobabel}}

\bibleverse{7} El rey Ciro también recuperó los objetos pertenecientes
al Templo del Señor que Nabucodonosor había tomado de Jerusalén y
colocado en el templo de su dios. \bibleverse{8} Ciro hizo que
Mitrídates, el tesorero, los recuperara, los contara y se los diera a
Sesbasar,\footnote{\textbf{1:8} ``Sesbasar'': algunos han identificado a
  Sesbasar (nombre babilónico) con Zorobabel (nombre hebreo).} que era
el líder de Judá. \footnote{\textbf{1:8} Esd 2,2; Esd 2,63; Esd 5,14}

\bibleverse{9} Esta era la lista: 30 cuencas de oro, 1. 000 cuencas de
plata, 29 cubiertos de plata, \bibleverse{10} 30 cuencos de oro, 410
cuencos de plata a juego y otros 1. 000 artículos. \bibleverse{11} En
total había 5. 400 objetos de oro y plata. Cuando los exiliados salieron
de Babilonia para ir a Jerusalén Sesbasar se llevó todo esto con ellos.

\hypertarget{directorio-de-juduxedos-que-regresan}{%
\subsection{Directorio de judíos que
regresan}\label{directorio-de-juduxedos-que-regresan}}

\hypertarget{section-1}{%
\section{2}\label{section-1}}

\bibleverse{1} Esta es una lista de los exiliados judíos de la
provincia\footnote{\textbf{2:1} ``Provincia'': bajo el dominio persa,
  Judá era simplemente una provincia del imperio.} que regresaron del
cautiverio después de que el rey Nabucodonosor se los llevara a
Babilonia. Volvieron a Jerusalén y a sus propias ciudades en Judá.

\hypertarget{lista-de-repatriados}{%
\subsection{Lista de repatriados}\label{lista-de-repatriados}}

\bibleverse{2} Sus líderes eran Zorobabel, Jesúa, Nehemías, Seraías,
Reelaías, Mardoqueo, Bilsán, Mispar, Bigvai, Rehum y Baana. Este es el
número de los hombres del pueblo de Israel:

\bibleverse{3} los hijos de Paros, 2. 172; \bibleverse{4} los hijos de
Sefatías, 372; \bibleverse{5} los hijos de Ara, 775; \bibleverse{6} los
hijos de Pahat-moab (hijos de Jesúa y Joab), 2. 812; \bibleverse{7} los
hijos de Elam, 1. 254; \bibleverse{8} los hijos de Zatu, 945;
\bibleverse{9} los hijos de Zacai, 760; \bibleverse{10} los hijos de
Bani, 642; \bibleverse{11} los hijos de Bebai, 623 \bibleverse{12} los
hijos de Azgad, 1. 222; \bibleverse{13} los hijos de Adonicam, 666;
\bibleverse{14} los hijos de Bigvai, 2. 056; \bibleverse{15} los hijos
de Adin, 454; \bibleverse{16} los hijos de Ater, (hijos de Ezequías),
98; \bibleverse{17} los hijos de Bezai, 323; \bibleverse{18} los hijos
de Jora, 112; \bibleverse{19} los hijos de Hasum, 223; \bibleverse{20}
los hijos de Gibar, 95; \bibleverse{21} el pueblo de Belén, 123;
\bibleverse{22} el pueblo de Netofa, 56; \bibleverse{23} el pueblo de
Anatot, 128; \bibleverse{24} el pueblo de Bet-azmavet, 42;
\bibleverse{25} el pueblo de Quiriat-jearim, Cafira y Beerot, 743;
\bibleverse{26} el pueblo de Ramá y Geba, 621; \bibleverse{27} el pueblo
de Micmas, 122 \bibleverse{28} el pueblo de Betel y de Hai, 223;
\bibleverse{29} los hijos de Nebo, 52; \bibleverse{30} los hijos de
Magbis, 156; \bibleverse{31} los hijos de Elam, 1. 254; \bibleverse{32}
los hijos de Harim, 320; \bibleverse{33} los hijos de Lod, Hadid y Ono,
725; \bibleverse{34} los hijos de Jericó, 345; \bibleverse{35} los hijos
de Senaa, 3. 630.

\bibleverse{36} Este es el número de los sacerdotes: los hijos de
Jedaías (por la familia de Jesúa), 973; \bibleverse{37} los hijos de
Imer, 1. 052; \bibleverse{38} los hijos de Pasur, 1. 247;
\bibleverse{39} los hijos de Harim, 1. 017.

\bibleverse{40} Este es el número de los levitas: los hijos de Jesúa y
Cadmiel (hijos de Hodavías), 74; \footnote{\textbf{2:40} Neh 12,8}
\bibleverse{41} los cantores de los hijos de Asaf, 128; \bibleverse{42}
los porteros de las familias de Salum, Ater, Talmón, Acub, Hatita y
Sobai, 139.

\bibleverse{43} Los descendientes de estos servidores del Templo: Ziha,
Hasufa, Tabaot, \bibleverse{44} Queros, Siaha, Padón, \bibleverse{45}
Lebana, Hagaba, Acub, \bibleverse{46} Hagab, Salmai, Hanán,
\bibleverse{47} Gidel, Gahar, Reaía, \bibleverse{48} Rezín, Necoda,
Gazam, \bibleverse{49} Uza, Paseah, Besai, \bibleverse{50} Asena,
Mehunim, Nefusim, \bibleverse{51} Bacbuc, Hacufa, Harhur,
\bibleverse{52} Bazlut, Mehída, Harsa, \bibleverse{53} Barcos, Sísara,
Tema, \bibleverse{54} Nezía, y Hatifa.

\bibleverse{55} Los descendientes de los siervos del rey Salomón: Sotai,
Hasoferet, Peruda, \footnote{\textbf{2:55} 1Re 9,21} \bibleverse{56}
Jaala, Darcón, Gidel, \bibleverse{57} Sefatías, Hatil,
Poqueret-hazebaim, y Ami. \bibleverse{58} El total de los siervos del
Templo y de los descendientes de los siervos de Salomón era de 392.

\bibleverse{59} Los que procedían de las ciudades de Tel-mela,
Tel-Harsa, Querub, Addán e Imer no podían demostrar su genealogía
familiar, ni siquiera que eran descendientes de Israel. \bibleverse{60}
Entre ellos estaban las familias de Delaía, Tobías y Necoda, 652 en
total. \bibleverse{61} Además había tres familias sacerdotales, hijos de
Habaía, Cos y Barzilai. (Barzilai se había casado con una mujer que
descendía de Barzilai de Galaad, y se llamaba así). \footnote{\textbf{2:61}
  2Sam 19,32} \bibleverse{62} Se buscó un registro de ellos en las
genealogías, pero no se encontraron sus nombres, por lo que se les
prohibió servir como sacerdotes. \bibleverse{63} El
gobernador\footnote{\textbf{2:63} ``Gobernador'': una palabra persa,
  probablemente referida a Sésbazar.} les ordenó que no comieran nada de
los sacrificios del santuario hasta que un sacerdote pudiera consular
con el Señor sobre el asunto a través del Urim y el Tumim.\footnote{\textbf{2:63}
  ``Urim and Tumim'': una metodología usada para determinar la voluntad
  de Dios respecto a un asunto. Véase Éxodo 28:30.}

\hypertarget{nuxfamero-total-de-personas-y-animales-de-carga-en-el-municipio}{%
\subsection{Número total de personas y animales de carga en el
municipio}\label{nuxfamero-total-de-personas-y-animales-de-carga-en-el-municipio}}

\bibleverse{64} El total de personas que regresaron fue de 42. 360.
\bibleverse{65} Además había 7. 337 sirvientes y 200 cantores y
cantoras. \bibleverse{66} Tenían 736 caballos, 245 mulas,
\bibleverse{67} 435 camellos y 6. 720 asnos.

\hypertarget{contribuciones-a-la-construcciuxf3n-del-templo-en-jerusaluxe9n-palabra-final}{%
\subsection{Contribuciones a la construcción del templo en Jerusalén;
Palabra
final}\label{contribuciones-a-la-construcciuxf3n-del-templo-en-jerusaluxe9n-palabra-final}}

\bibleverse{68} Cuando llegaron al Templo del Señor en Jerusalén,
algunos de los jefes de familia hicieron contribuciones voluntarias para
reconstruir el Templo de Dios en el lugar donde antes estaba.
\bibleverse{69} Dieron según lo que tenían, poniendo su donativo en el
tesoro. El total ascendió a 61. 000 dáricos de oro, 5. 000 minas de
plata y 100 túnicas para los sacerdotes.

\bibleverse{70} Los sacerdotes, los levitas, los cantores, los porteros
y los servidores del Templo, así como parte del pueblo, volvieron a
vivir en sus pueblos específicos. Los demás regresaron a sus propias
ciudades en todo Israel.

\hypertarget{construcciuxf3n-del-altar-de-las-ofrendas-quemadas-y-establecimiento-del-servicio-de-sacrificios-regular-celebraciuxf3n-de-la-fiesta-de-los-tabernuxe1culos}{%
\subsection{Construcción del altar de las ofrendas quemadas y
establecimiento del servicio de sacrificios regular; Celebración de la
Fiesta de los
Tabernáculos}\label{construcciuxf3n-del-altar-de-las-ofrendas-quemadas-y-establecimiento-del-servicio-de-sacrificios-regular-celebraciuxf3n-de-la-fiesta-de-los-tabernuxe1culos}}

\hypertarget{section-2}{%
\section{3}\label{section-2}}

\bibleverse{1} Al llegar el séptimo mes, los israelitas se habían
instalado en sus ciudades, y el pueblo se reunió como uno solo en
Jerusalén. \footnote{\textbf{3:1} Esd 2,64} \bibleverse{2} Entonces
Jesúa, hijo de Josadac, y los sacerdotes que estaban con él, junto con
Zorobabel, hijo de Salatiel, y sus parientes, empezaron a construir el
altar del Dios de Israel para sacrificar en él holocaustos, según las
instrucciones de la Ley de Moisés, el hombre de Dios. \footnote{\textbf{3:2}
  Esd 2,2; 1Cró 3,17-19; Éxod 27,1; Lev 6,2} \bibleverse{3} Aunque
tenían miedo de los habitantes del lugar, levantaron el altar sobre sus
cimientos originales y sacrificaron en él holocaustos al Señor, tanto en
la mañana como en la tarde. \bibleverse{4} Y observaban la Fiesta de los
Tabernáculos tal y como exigía la Ley, sacrificando el número
especificado de holocaustos cada día. \footnote{\textbf{3:4} Lev 23,34;
  Núm 29,12-38} \bibleverse{5} Después presentaron también los
holocaustos diarios y las ofrendas de la luna nueva, así como los de
todas las fiestas anuales del señor y de los que traían ofrendas
voluntarias al señor. \bibleverse{6} Así que, desde el primer día del
séptimo mes, los israelitas comenzaron a presentar holocaustos al Señor,
aunque los cimientos del Templo del Señor no habían sido puestos
todavía.

\hypertarget{preparaciones-para-la-construcciuxf3n-de-templos-colocaciuxf3n-ceremonial-de-la-primera-piedra}{%
\subsection{Preparaciones para la construcción de templos; Colocación
ceremonial de la primera
piedra}\label{preparaciones-para-la-construcciuxf3n-de-templos-colocaciuxf3n-ceremonial-de-la-primera-piedra}}

\bibleverse{7} Pagaron a albañiles y carpinteros, y proporcionaron
comida y bebida y aceite de oliva a los habitantes de Sidón y Tiro para
que trajeran troncos de cedro del Líbano a Jope por mar, tal como había
autorizado el rey Ciro de Persia.

\bibleverse{8} En el segundo mes del segundo año después de llegar al
Templo de Dios en Jerusalén, Zorobabel, hijo de Sealtiel, Jesúa, hijo de
Josadac, y los que estaban con ellos -- los sacerdotes, los levitas y
todos los que habían regresado a Jerusalén del cautiverio -- comenzaron
la obra. Pusieron a los levitas de veinte años o más a cargo de la
construcción del Templo del Señor. \bibleverse{9} Jesúa y sus hijos y
parientes, Cadmiel y sus hijos, los descendientes de Judá, los hijos de
Henadad y sus hijos y parientes, todos ellos levitas, supervisaban a los
que trabajaban en el Templo de Dios.

\bibleverse{10} Cuando los constructores pusieron los cimientos del
Templo del Señor, los sacerdotes vestidos con sus ropas especiales y
portando trompetas, y los levitas (los hijos de Asaf) portando címbalos,
todos ocuparon sus lugares para alabar al Señor, siguiendo las
instrucciones dadas por el rey David de Israel. \bibleverse{11} Cantaron
con alabanza y agradecimiento al Señor: ``Dios es bueno, porque su amor
fiel a Israel es eterno''. Entonces todos los presentes dieron un
tremendo grito de alabanza al Señor, porque se habían puesto los
cimientos del Templo del Señor. \footnote{\textbf{3:11} 2Cró 5,13; 2Cró
  7,3; Sal 118,1}

\bibleverse{12} Pero muchos de los sacerdotes, levitas y jefes de
familia más antiguos, que recordaban el primer Templo, lloraron
fuertemente cuando vieron los cimientos de este Templo, aunque muchos
otros gritaron de alegría.\footnote{\textbf{3:12} Se suele pensar que la
  razón de la tristeza de los mayores es que este Templo sustitutivo era
  muy inferior al primero.} \footnote{\textbf{3:12} Ag 2,3}

\bibleverse{13} Sin embargo, nadie podía distinguir los gritos de
alegría de los gritos de llanto, porque todos hacían mucho ruido, tanto
que se oía a gran distancia.

\hypertarget{rechazo-de-los-samaritanos-de-participar-en-la-construcciuxf3n-del-templo}{%
\subsection{Rechazo de los samaritanos de participar en la construcción
del
templo}\label{rechazo-de-los-samaritanos-de-participar-en-la-construcciuxf3n-del-templo}}

\hypertarget{section-3}{%
\section{4}\label{section-3}}

\bibleverse{1} Los enemigos de Judá y Benjamín se enteraron de que los
exiliados estaban construyendo un Templo para el Señor, el Dios de
Israel. \bibleverse{2} Se acercaron a Zorobabel y a los jefes de familia
y les dijeron: ``Por favor, dejad que os ayudemos en la construcción,
porque adoramos a vuestro Dios como vosotros. De hecho, le hemos estado
sacrificando desde la época de Esar-hadón, rey de Asiria, quien nos
trajo aquí''. \footnote{\textbf{4:2} 2Re 17,24; 2Re 17,33; 2Re 19,37}

\bibleverse{3} Pero Zorobabel, Jesúa y los líderes de la familia de
Israel respondieron: ``Ustedes no pueden compartir con nosotros la
construcción de un Templo para nuestro Dios. Sólo nosotros podemos
construirlo para el Señor, el Dios de Israel. Esto es lo que Ciro, el
rey de Persia, nos ha ordenado hacer''. \footnote{\textbf{4:3} Esd 1,3}

\bibleverse{4} Entonces, los lugareños se dispusieron a intimidar a los
habitantes de Judá y hacer que tuvieran demasiado miedo para seguir
construyendo. \bibleverse{5} Entonces sobornaron a
funcionarios\footnote{\textbf{4:5} Serían funcionarios locales cuya
  cadena de mando se remontaba al rey persa.} para oponerse a ellos y
obstruir sus planes. Esto continuó durante todo el reinado de Ciro, rey
de Persia, hasta el reinado de Darío, rey de Persia. \footnote{\textbf{4:5}
  Esd 4,24}

\hypertarget{varias-acusaciones-contra-los-juduxedos-y-su-templo-y-la-construcciuxf3n-de-muros-bajo-el-gobierno-de-jerjes-y-artajerjes.}{%
\subsection{Varias acusaciones contra los judíos y su templo y la
construcción de muros bajo el gobierno de Jerjes y
Artajerjes.}\label{varias-acusaciones-contra-los-juduxedos-y-su-templo-y-la-construcciuxf3n-de-muros-bajo-el-gobierno-de-jerjes-y-artajerjes.}}

\bibleverse{6} Cuando Asuero se convirtió en rey, los lugareños le
enviaron una acusación escrita contra el pueblo de Judá y Jerusalén.

\bibleverse{7} En tiempos de Artajerjes, rey de Persia, Bislam,
Mitrídates, Tabeel y sus compañeros escribieron una carta a Artajerjes.
La carta fue escrita en arameo y fue traducida.\footnote{\textbf{4:7} El
  pasaje de 4:8 a 6:18 está en arameo.} \bibleverse{8} Rehum, el oficial
al mando, y Simsai, el escriba, escribieron una carta al rey Artajerjes
en la que condenaban a Jerusalén. \bibleverse{9} Esto proviene de Rehum,
el oficial al mando, Simsai, el escriba, y los compañeros oficiales: los
jueces y funcionarios y los responsables de Persia, Erec y Babilonia,
los elamitas de Susa, \bibleverse{10} y el resto del pueblo que el gran
y noble Asurbanipal deportó y reasentó en las ciudades de Samaria y
otros lugares al oeste del Éufrates.

\bibleverse{11} La siguiente es una copia de la carta que le enviaron:
``Al rey Artajerjes, de parte de tus siervos, hombres de más allá del
río Éufrates:

\bibleverse{12} ``Su Majestad debe ser informado de que los judíos que
vinieron de usted a nosotros han regresado a Jerusalén. Están
reconstruyendo esa ciudad rebelde y malvada, completando las
reparaciones de las murallas y arreglando sus cimientos. \bibleverse{13}
Su Majestad debería darse cuenta de que si esta ciudad es reconstruida y
sus murallas reparadas, no pagarán impuestos, tributos o tasas, y los
ingresos del rey se verán afectados. \bibleverse{14} Ahora bien, como
estamos al servicio del rey\footnote{\textbf{4:14} ``Servicio del Rey'':
  literalmente, ``comer la sal del palacio''.} y no nos parece bien que
se le falte al respeto a Su Majestad, le enviamos esta carta para que
esté informado, \bibleverse{15} y ordenar una búsqueda en los archivos
reales. Descubrirá en estos registros que se trata de una ciudad
rebelde, que perjudica a los reyes y a los países,\footnote{\textbf{4:15}
  ``Países'': literalmente, ``provincias''.} habiéndose levantado a
menudo en rebelión en el pasado. Este es motivo por el cual esta ciudad
había sido destruída. \bibleverse{16} Queremos informar a Su Majestad de
que si se reconstruye esta ciudad y se completan las murallas, perderá
esta provincia al oeste del Éufrates''.

\hypertarget{la-construcciuxf3n-del-templo-se-paralizuxf3-como-consecuencia-de-un-real-decreto}{%
\subsection{La construcción del templo se paralizó como consecuencia de
un real
decreto}\label{la-construcciuxf3n-del-templo-se-paralizuxf3-como-consecuencia-de-un-real-decreto}}

\bibleverse{17} El rey respondió lo siguiente ``Al comandante Rehum, al
escriba Simsai y a los compañeros que viven en Samaria y en otras zonas
al oeste del Éufrates: Saludos. \bibleverse{18} La carta que ustedes nos
enviaron ha sido traducida y la han leído ante mí. \bibleverse{19} He
ordenado que se realice una investigación. Se ha descubierto que esta
ciudad se ha levantado a menudo en rebelión contra los reyes en el
pasado, promoviendo frecuentemente la insurrección y la rebelión.
\bibleverse{20} Poderosos reyes han gobernado en Jerusalén y en toda la
zona al oeste del Éufrates, y han recibido impuestos, tributos y tasas.
\bibleverse{21} Emitan una orden inmediata para que estos hombres dejen
de trabajar. Esta ciudad no debe ser reconstruida hasta que yo lo
autorice. \bibleverse{22} Procura no descuidar este asunto. ¿Por qué
habríamos de dejar que este problema crezca y perjudique los intereses
reales?'' \bibleverse{23} Tan pronto como esta carta del rey Artajerjes
fue leída a Rehum, al escriba Simsai y a sus compañeros, se precipitaron
hacia los judíos de Jerusalén y utilizaron su poder para obligarlos a
detener los trabajos. \bibleverse{24} En consecuencia, las obras del
Templo de Dios en Jerusalén se detuvieron. La paralización continuó
hasta el segundo año del reinado del rey Darío de Persia.\footnote{\textbf{4:24}
  Esd 4,5; Esd 6,15}

\hypertarget{profecuxedas-favorables-de-dos-profetas-permiso-del-gobernador-para-reanudar-la-construcciuxf3n}{%
\subsection{Profecías favorables de dos profetas; Permiso del gobernador
para reanudar la
construcción}\label{profecuxedas-favorables-de-dos-profetas-permiso-del-gobernador-para-reanudar-la-construcciuxf3n}}

\hypertarget{section-4}{%
\section{5}\label{section-4}}

\bibleverse{1} Los profetas Hageo y Zacarías, hijo de Iddo, le enviaron
mensajes\footnote{\textbf{5:1} ``Le enviaron mensajes'': literalmente,
  ``profetizó''. A partir de la reacción de Zorobabel los mensajes eran
  para reiniciar la construcción del Templo.} a los judíos de Judá y
Jerusalén de parte del Dios de Israel, su gobernante. \footnote{\textbf{5:1}
  Ag 1,1; Zac 1,1} \bibleverse{2} Entonces Zorobabel, hijo de Sealtiel,
y Jesúa, hijo de Josadac, decidieron empezar a trabajar en la
reconstrucción del Templo de Dios en Jerusalén. Los profetas de Dios los
animaron y los ayudaron.

\bibleverse{3} Casi inmediatamente llegaron Tatnai, el gobernador de la
provincia al oeste del Éufrates, Setar-boznai, y sus colegas
funcionarios y preguntaron: ``¿Quién les dio permiso para reconstruir
este Templo y terminarlo?'' \bibleverse{4} Luego preguntaron: ``¿Cuáles
son los nombres de los hombres que están trabajando en este edificio?''
\bibleverse{5} Pero su Dios velaba por los dirigentes judíos, de modo
que no se les impidió trabajar hasta que se pudo enviar un informe a
Darío y se recibió una respuesta escrita con instrucciones. \footnote{\textbf{5:5}
  Deut 11,12; 1Re 8,29}

\hypertarget{informe-e-investigaciuxf3n-del-gobernador-al-rey-daruxedo-sobre-la-construcciuxf3n-del-templo}{%
\subsection{Informe e investigación del gobernador al rey Darío sobre la
construcción del
templo}\label{informe-e-investigaciuxf3n-del-gobernador-al-rey-daruxedo-sobre-la-construcciuxf3n-del-templo}}

\bibleverse{6} La siguiente es una copia de la carta que Tatnai, el
gobernador de la provincia al oeste del Éufrates, Setar-boznai, y sus
compañeros, funcionarios de la provincia, enviaron al rey Darío.
\bibleverse{7} El informe que le enviaron decía lo siguiente ``Al rey
Darío: Saludos.

\bibleverse{8} Deseamos informar a Su Majestad que fuimos a la provincia
de Judá, al Templo del gran Dios. Se está construyendo con grandes
piedras, y se están colocando vigas de madera en las paredes. Esta obra
se está realizando correctamente y avanza a buen ritmo. \bibleverse{9}
``Interrogamos a los dirigentes, preguntándoles: `¿Quién les dio permiso
para reconstruir este Templo y terminarlo?' \bibleverse{10} También les
pedimos sus nombres, para anotarlos y hacerles saber los nombres de sus
dirigentes. \bibleverse{11} ``Esta es la respuesta que nos dieron:
`Somos servidores del Dios del cielo y de la tierra. Estamos
reconstruyendo el Templo construido y terminado hace muchos años por un
gran rey de Israel. \bibleverse{12} Pero nuestros antepasados hicieron
enojar al Dios del cielo, por lo cual los entregó a Nabucodonosor, rey
de Babilonia, el caldeo, quien destruyó este Templo y deportó al pueblo
a Babilonia. \bibleverse{13} Sin embargo, Ciro, rey de Babilonia, en el
primer año de su reinado, emitió un decreto para reconstruir este Templo
de Dios. \footnote{\textbf{5:13} Esd 1,1} \bibleverse{14} Incluso
devolvió los objetos de oro y plata pertenecientes al Templo de Dios,
que Nabucodonosor había tomado del Templo de Jerusalén y colocado en su
templo de Babilonia. El rey Ciro se los entregó a un hombre llamado
Sesbasar, a quien había nombrado gobernador, \footnote{\textbf{5:14} Esd
  1,8} \bibleverse{15} diciéndole: Toma estos artículos y colócalos en
el Templo de Jerusalén. Reconstruye el Templo de Dios en su sitio
original. \bibleverse{16} Así que Sesbasar vino y puso los cimientos del
Templo de Dios en Jerusalén. Desde entonces está en construcción, pero
aún no se ha completado'.

\bibleverse{17} ``Por lo tanto, si Su Majestad lo desea, autorice que se
haga una búsqueda en los archivos reales de Babilonia para descubrir si
hay un registro de que el rey Ciro emitió un decreto para reconstruir el
Templo de Dios en Jerusalén. Entonces, por favor, háganos saber la
decisión de Su Majestad en este asunto''.

\hypertarget{encontrar-el-decreto-de-cyrus-en-ekbatana-e-informaciuxf3n-de-uxe9l}{%
\subsection{Encontrar el decreto de Cyrus en Ekbatana e información de
él}\label{encontrar-el-decreto-de-cyrus-en-ekbatana-e-informaciuxf3n-de-uxe9l}}

\hypertarget{section-5}{%
\section{6}\label{section-5}}

\bibleverse{1} Así, el rey Darío ordenó que se buscara en los archivos
que se encontraban en el tesoro de Babilonia. \bibleverse{2} Pero en
realidad fue en la fortaleza de Ecbatana, en la provincia de Media,
donde se encontró un pergamino que registraba lo siguiente
\bibleverse{3} En el primer año del rey Ciro, éste emitió un decreto
relativo al Templo de Dios en Jerusalén: ``Que se reconstruya el Templo
como lugar donde se ofrezcan sacrificios, y que tenga unos cimientos
fuertes y firmes. Hazlo de sesenta codos de alto y sesenta codos de
ancho, \footnote{\textbf{6:3} Esd 1,1} \bibleverse{4} con tres capas de
bloques de piedra y una de madera. Los gastos se pagarán con el tesoro
real. \bibleverse{5} Además, los objetos de oro y plata del Templo de
Dios, que Nabucodonosor tomó del Templo de Jerusalén y llevó a
Babilonia, también deben ser devueltos al Templo de Jerusalén y
colocados allí.

\hypertarget{decreto-de-daruxedo-para-continuar-sin-trabas-y-promover-la-construcciuxf3n-del-templo}{%
\subsection{Decreto de Darío para continuar sin trabas y promover la
construcción del
templo}\label{decreto-de-daruxedo-para-continuar-sin-trabas-y-promover-la-construcciuxf3n-del-templo}}

\bibleverse{6} ``Estas son mis instrucciones para ti, Tatnai, gobernador
de la provincia al oeste del Éufrates, Setar-boznai, y para tus
compañeros y funcionarios de la provincia: ¡Aléjate de allí!
\bibleverse{7} ¡Deja en paz esta obra en el Templo de Dios! Dejen que el
gobernador y los líderes de los judíos continúen con la reconstrucción
de este Templo de Dios en su sitio original. \bibleverse{8} Además, este
es mi decreto en cuanto a lo que debes hacer por estos líderes judíos en
cuanto a la reconstrucción de este Templo de Dios. El gasto total de la
obra se pagará con los ingresos reales, el tributo de la provincia al
oeste del Éufrates, para que la obra no se retrase. \bibleverse{9}
Proporciona todo lo que necesiten los sacerdotes de Jerusalén: novillos,
carneros y corderos para los holocaustos al Dios del cielo, y trigo,
sal, vino y aceite de oliva. Asegúrate de darles esto cada día sin
falta. \bibleverse{10} De esta manera podrán ofrecer sacrificios
aceptables al Dios del cielo, y pedir por la vida del rey y de sus
hijos. \bibleverse{11} Además, declaro que si alguien interfiere con
este decreto, se arrancará una viga de su casa y se clavará en el suelo,
y él será empalado en ella. Su propia casa se convertirá en un montón de
escombros por desobedecer este decreto. \bibleverse{12} Que Dios, que
eligió la ciudad de Jerusalén como el lugar donde sería honrado,
destruya a cualquier rey o pueblo que intente alterar lo que he dicho o
que destruya este Templo. Yo, Darío, emito este decreto. Que se cumpla
fielmente''.

\hypertarget{terminaciuxf3n-y-dedicaciuxf3n-solemne-del-templo}{%
\subsection{Terminación y dedicación solemne del
templo}\label{terminaciuxf3n-y-dedicaciuxf3n-solemne-del-templo}}

\bibleverse{13} Tatnai, el gobernador de la provincia al oeste del
Éufrates, Setar-boznai, y sus compañeros oficiales cumplieron fielmente
lo que el rey Darío había decretado.

\bibleverse{14} Como resultado, los líderes judíos siguieron
construyendo, y se sintieron alentados por los mensajes del profeta Ageo
y de Zacarías, hijo de Iddo. Terminaron de construir el Templo siguiendo
el mandato del Dios de Israel y los decretos de Ciro, Darío y
Artajerjes, reyes de Persia. \bibleverse{15} El Templo fue terminado el
tercer día del mes de Adar, en el sexto año del reinado del rey Darío.

\bibleverse{16} Entonces el pueblo de Israel, los sacerdotes, los
levitas y el resto de los que habían regresado del exilio, celebraron
con alegría la dedicación del Templo de Dios. \footnote{\textbf{6:16}
  Núm 7,10; 1Re 8,62-66} \bibleverse{17} Para dedicar el Templo de Dios
sacrificaron cien toros, doscientos carneros, cuatrocientos corderos y
una ofrenda por el pecado para todo Israel compuesta por doce machos
cabríos, uno por cada tribu israelita. \footnote{\textbf{6:17} Esd 8,35}
\bibleverse{18} Organizaron a los sacerdotes y a los levitas por sus
divisiones para servir a Dios en el Templo de Jerusalén, de acuerdo con
el Libro de Moisés. \footnote{\textbf{6:18} Núm 3,6; Núm 8,24}

\hypertarget{celebraciuxf3n-de-la-pascua}{%
\subsection{Celebración de la
Pascua}\label{celebraciuxf3n-de-la-pascua}}

\bibleverse{19} Los exiliados que habían regresado celebraban la Pascua
el día catorce del primer mes. \footnote{\textbf{6:19} Éxod 12,6}

\bibleverse{20} Los sacerdotes y los levitas se habían purificado para
estar limpios según la ley ceremonial.\footnote{\textbf{6:20} ``Según la
  ley ceremonial'': implícito.} Así que mataron el cordero de la Pascua
para todos los exiliados que habían regresado, para sus compañeros
sacerdotes y para ellos mismos. \bibleverse{21} La Pascua la comían el
pueblo de Israel que había regresado del exilio y los que se habían
unido a ellos y habían rechazado las prácticas paganas de los pueblos de
la tierra para adorar al Señor, el Dios de Israel. \bibleverse{22}
Entonces celebraron la Fiesta de los Panes sin Levadura durante siete
días. Todos los habitantes del país estaban muy contentos porque el
Señor había hecho que el rey de Asiria les fuera favorable, ayudándoles
a reconstruir el Templo de Dios, el Dios de Israel.

\hypertarget{el-regreso-de-esdras-y-su-banda-de-babilonia-a-jerusaluxe9n}{%
\subsection{El regreso de Esdras y su banda de Babilonia a
Jerusalén}\label{el-regreso-de-esdras-y-su-banda-de-babilonia-a-jerusaluxe9n}}

\hypertarget{section-6}{%
\section{7}\label{section-6}}

\bibleverse{1} Después de todo esto, durante el reinado de Artajerjes,
rey de Persia, llegó Esdras desde Babilonia. Era hijo de Seraías, hijo
de Azarías, hijo de Hilcías, \footnote{\textbf{7:1} 1Cró 5,40}
\bibleverse{2} hijo de Salum, hijo de Sadoc, hijo de Ajitub,
\bibleverse{3} hijo de Amarías, hijo de Azarías, hijo de Meraiot,
\bibleverse{4} hijo de Zeraías, hijo de Uzí, hijo de Bucí,
\bibleverse{5} hijo de Abisúa, hijo de Finees, hijo de Eleazar, hijo del
sumo sacerdote Aarón. \bibleverse{6} Este Esdras llegó de Babilonia y
era un escriba experto en la Ley de Moisés, que el Señor, el Dios de
Israel, había dado a Israel. El rey había concedido a Esdras todo lo que
había pedido, porque el Señor, su Dios, estaba con él. \bibleverse{7} En
el séptimo año del rey Artajerjes partió hacia Jerusalén, acompañado de
parte del pueblo de Israel y de algunos de los sacerdotes y levitas,
cantores y porteros, y servidores del Templo. \footnote{\textbf{7:7} Esd
  2,43} \bibleverse{8} Esdras llegó a Jerusalén en el quinto mes del
séptimo año del reinado de Artajerjes. \bibleverse{9} Había emprendido
el viaje desde Babilonia el primer día del primer mes, y llegó a
Jerusalén el primer día del quinto mes, yendo con él su Dios bondadoso.
\bibleverse{10} Porque Esdras se había comprometido a adquirir
conocimientos de la Ley del Señor, queriendo practicarla y enseñar en
Israel sus reglas y cómo vivir.

\hypertarget{redacciuxf3n-de-la-carta-real-carta-de-salvoconducto-con-detalles-de-los-poderes-otorgados-a-ezra}{%
\subsection{Redacción de la carta real (= carta de salvoconducto) con
detalles de los poderes otorgados a
Ezra}\label{redacciuxf3n-de-la-carta-real-carta-de-salvoconducto-con-detalles-de-los-poderes-otorgados-a-ezra}}

\bibleverse{11} Esta es una copia de la carta que el rey Artajerjes
entregó al sacerdote y escriba Esdras, que había estudiado los
mandamientos y reglamentos del Señor dados a Israel:\footnote{\textbf{7:11}
  El texto de 7:12-26 está en arameo.} \bibleverse{12} ``Artajerjes, rey
de reyes, al sacerdote Esdras,\footnote{\textbf{7:12} ``Sacerdote'':
  curiosamente, no se utiliza la palabra habitual para sacerdote en
  arameo. En cambio, es una palabra derivada del hebreo, lo que sugiere
  que el decreto fue redactado por primera vez por un judío,
  posiblemente el propio Esdras.} el escriba de la Ley del Dios del
cielo: Saludos. \footnote{\textbf{7:12} Ezeq 26,7}

\bibleverse{13} Por la presente emito este decreto: Cualquiera del
pueblo de Israel o de sus sacerdotes o levitas en mi reino que
voluntariamente decida ir a Jerusalén con ustedes puede hacerlo.
\bibleverse{14} Ustedes son enviados por el rey y sus siete consejeros
para investigar la situación en Judá y Jerusalén en lo que se refiere a
la Ley de su Dios, que ustedes llevan consigo.\footnote{\textbf{7:14}
  ``Que lleves contigo'': literalmente, ``que está en tu mano''.}
\bibleverse{15} También te ordenamos que lleves contigo la plata y el
oro que el rey y sus consejeros han donado voluntariamente al Dios de
Israel, que vive en Jerusalén, \bibleverse{16} junto con toda la plata y
el oro que recibas de la provincia de Babilonia, así como las donaciones
voluntarias del pueblo y los sacerdotes al Templo de su Dios en
Jerusalén. \bibleverse{17} Con este dinero comprarás primero todos los
toros, carneros y corderos que sean necesarios, junto con sus ofrendas
de grano y de bebida, y los presentarás en el altar del Templo de tu
Dios en Jerusalén. \bibleverse{18} Luego, tú y los que están contigo
pueden decidir usar el resto de la plata y el oro de la manera que mejor
les parezca, de acuerdo con la voluntad de tu Dios. \bibleverse{19}
``Pero los objetos que te han dado para el servicio del Templo de tu
Dios deben ser entregados todos al Dios de Jerusalén. \bibleverse{20} Si
hay alguna otra cosa necesaria para el Templo de tu Dios que tengas que
proveer, puedes cargarla al tesoro real.

\bibleverse{21} ``Yo, el rey Artajerjes, decreto que todos los tesoreros
al oeste del Éufrates deben proveer todo lo que el sacerdote Esdras, el
escriba de la Ley del Dios del cielo, les pida, y debe ser provisto en
su totalidad, \bibleverse{22} hasta cien talentos de plata, cien corsos
de trigo, cien baños de vino, cien baños de aceite de oliva y cantidades
ilimitadas de sal. \bibleverse{23} Asegúrate de proveer en su totalidad
todo lo que el Dios del cielo requiera para su Templo, pues ¿por qué
habría de caer su ira sobre el rey y sus hijos?

\bibleverse{24} Ten en cuenta también que todos los sacerdotes, levitas,
cantores, porteros, sirvientes del Templo u otros trabajadores de este
Templo están exentos de pagar cualquier impuesto, tributo o tasa, y no
estás autorizado a cobrarles.

\bibleverse{25} ``Tú, Esdras, debes seguir la sabiduría de tu Dios que
posees, debes nombrar magistrados y jueces para impartir justicia a todo
el pueblo al oeste del Éufrates, a todos los que siguen las leyes de tu
Dios. Tú deberás enseñar estas leyes a los que no las cumplen.
\bibleverse{26} Cualquiera que no cumpla la ley de tu Dios y la ley del
rey, será castigado severamente, ya sea con la muerte, el destierro, la
confiscación de bienes o la prisión''.

\hypertarget{oraciuxf3n-de-acciuxf3n-de-gracias-de-esdras-e-inicio-de-su-actividad}{%
\subsection{Oración de acción de gracias de Esdras e inicio de su
actividad}\label{oraciuxf3n-de-acciuxf3n-de-gracias-de-esdras-e-inicio-de-su-actividad}}

\bibleverse{27} Alabado sea el Señor, el Dios de nuestros antepasados,
que puso en la mente del rey honrar así el Templo del Señor en
Jerusalén, \bibleverse{28} y que me ha mostrado tanta bondad al honrarme
ante el rey, sus consejeros y todos sus altos funcionarios. Como el
Señor, mi Dios, estaba conmigo, me animé y convoqué a los jefes de
Israel para que regresaran conmigo a Jerusalén.

\hypertarget{directorio-de-los-jefes-de-las-familias-de-judea-que-regresan-con-esdras}{%
\subsection{Directorio de los jefes de las familias de Judea que
regresan con
Esdras}\label{directorio-de-los-jefes-de-las-familias-de-judea-que-regresan-con-esdras}}

\hypertarget{section-7}{%
\section{8}\label{section-7}}

\bibleverse{1} Esta es una lista de los jefes de familia y los registros
genealógicos de los que volvieron conmigo de Babilonia durante el
reinado del rey Artajerjes: \footnote{\textbf{8:1} Esd 7,1; Esd 7,7}
\bibleverse{2} De los hijos de Finees, Gersón. De los hijos de Itamar,
Daniel. De los hijos de David, Hattush, \bibleverse{3} hijo de Secanías.
De los hijos de Paros, Zacarías, y con él se registraron 150 hombres.
\bibleverse{4} De los hijos de Pahat-moab, Elioenai, hijo de Zeraías, y
con él 200 hombres. \footnote{\textbf{8:4} Esd 2,6} \bibleverse{5} De
los hijos de Zatu,\footnote{\textbf{8:5} ``Zatu'': Tomado de la
  Septuaginta.} Secanías, hijo de Jahaziel, y con él 300 hombres.
\footnote{\textbf{8:5} Esd 2,8} \bibleverse{6} De los hijos de Adín,
Ebed, hijo de Jonatán, y con él 50 hombres. \bibleverse{7} De los hijos
de Elam, Jesaías, hijo de Atalía, y con él 70 hombres. \bibleverse{8} De
los hijos de Sefatías, Zebadías, hijo de Micael, y con él 80 hombres.
\bibleverse{9} De los hijos de Joab, Abdías, hijo de Jehiel, y con él
218 hombres. \bibleverse{10} De los hijos de Bani,\footnote{\textbf{8:10}
  ``Bani'': Tomado de la Septuaginta.} Selomit, hijo de Josifías, y con
él 160 hombres. \footnote{\textbf{8:10} Esd 2,10} \bibleverse{11} De los
hijos de Bebai, Zacarías, hijo de Bebai, y con él 28 hombres.
\bibleverse{12} De los hijos de Azgad, Johanán, hijo de Hacatán, y con
él 110 hombres. \bibleverse{13} De los hijos de Adonicam, los
últimos,\footnote{\textbf{8:13} ``Los últimos'': lo más probable es que
  se trate de los hijos menores de Adonicam, por lo que fueron los
  últimos de su familia en regresar a Jerusalén.} sus nombres eran
Elifelet, Jeuel y Semaías, y con ellos 60 hombres. \bibleverse{14} De
los hijos de Bigvai, Utai y Zacur, y con ellos 70 hombres.

\hypertarget{los-preparativos-finales-para-la-salida}{%
\subsection{Los preparativos finales para la
salida}\label{los-preparativos-finales-para-la-salida}}

\bibleverse{15} Reuní a los exiliados que regresaban en el canal de
Ahava. Acampamos allí durante tres días mientras revisaba quiénes habían
venido: la gente común, los sacerdotes y los levitas. Descubrí que no
había ni un solo levita \bibleverse{16} Así que mandé llamar a Eliezer,
Ariel, Semaías, Elnatán, Jarib, Elnatán, Natán, Zacarías y Mesulam, que
eran líderes, y a Joiarib y Elnatán, que eran hombres con buena visión.
\bibleverse{17} Les dije que fueran a ver a Iddo, el jefe de los
servidores del Templo en Casifia, pidiéndole a él y a sus parientes que
nos enviaran ministros para el Templo de nuestro Dios. \bibleverse{18}
Como nuestro Dios bondadoso estaba con nosotros, nos trajeron a
Serebías, un hombre con buena visión de los hijos de Mahli, hijo de
Leví, hijo de Israel, junto con sus hijos y hermanos, un total de
dieciocho hombres; \footnote{\textbf{8:18} Esd 7,6} \bibleverse{19} y
Hasabías, junto con Jesaías, de los hijos de Merari, y sus hermanos y
sus hijos, un total de veinte hombres. \bibleverse{20} Además, trajeron
a 220 de los sirvientes del Templo, un grupo designado por David y sus
funcionarios para ayudar a los levitas. Todos ellos estaban registrados
por su nombre.

\hypertarget{ayuno-y-oraciuxf3n-de-los-que-regresan-a-casa-entrega-de-los-dones-del-templo-a-hombres-confiables}{%
\subsection{Ayuno y oración de los que regresan a casa; Entrega de los
dones del templo a hombres
confiables}\label{ayuno-y-oraciuxf3n-de-los-que-regresan-a-casa-entrega-de-los-dones-del-templo-a-hombres-confiables}}

\bibleverse{21} En el canal de Ahava convoqué un ayuno para confesar
nuestros pecados ante Dios y pedirle un viaje seguro para nosotros y
nuestros hijos, junto con todas nuestras posesiones. \bibleverse{22} Me
había resistido a pedir al rey que nos diera una escolta militar para
protegernos de los enemigos en el camino. Le habíamos dicho al rey:
``Nuestro Dios bondadoso cuida de todo el que lo sigue, pero muestra su
ira contra el que lo abandona''. \footnote{\textbf{8:22} Esd 7,6}
\bibleverse{23} Así que ayunamos y pedimos a Dios que nos protegiera, y
él respondió a nuestras oraciones.

\bibleverse{24} Entonces designé a doce de los principales sacerdotes,
y\footnote{\textbf{8:24} ``Y'': Tomado de la Septuaginta, haciendo
  distinción de un grupo de doce sacerdotes, y otro grupo de doce
  levitas. Serebías y Hasabíasacaban de ser identificados como levitas,
  no como sacerdotes (ver versos 18 y 19).} Serebías, Hasabías y diez de
sus hermanos, \bibleverse{25} y los pesé y los entregué\footnote{\textbf{8:25}
  Confiándoles la responsabilidad de salvaguardar estos valiosos
  artículos.} las donaciones de plata y oro, y los artículos que el rey,
sus consejeros, sus dirigentes y todo el pueblo de Israel habían dado
allí para el Templo de nuestro Dios. \bibleverse{26} Pesé y puse en sus
manos 650 talentos de plata, artículos de plata del Templo que pesaban
100 talentos, 100 talentos de oro, \bibleverse{27} 20 cuencos de oro que
valían 1. 000 dáricos, y dos artículos de bronce muy pulido, tan
valiosos como el oro. \bibleverse{28} Les dije: ``Ustedes están
apartados para el Señor, y estos objetos del Templo también lo están. La
plata y el oro son una ofrenda voluntaria al Señor, el Dios de vuestros
antepasados. \bibleverse{29} Ustedes deben custodiarlos y guardarlos
hasta que los entreguen, pesándolos ante los sumos sacerdotes, los
levitas y los jefes de familia de Israel en Jerusalén, en las salas del
tesoro dentro del Templo del Señor''.

\bibleverse{30} Los sacerdotes y los levitas se hicieron cargo de la
plata y el oro y de los objetos del Templo que habían sido pesados para
ser llevados al Templo de nuestro Dios en Jerusalén.

\hypertarget{llegada-a-jerusaluxe9n-entrega-de-los-obsequios-votivos-hacer-ofrendas-apoyo-de-funcionarios-reales}{%
\subsection{Llegada a Jerusalén; Entrega de los obsequios votivos; Hacer
ofrendas; Apoyo de funcionarios
reales}\label{llegada-a-jerusaluxe9n-entrega-de-los-obsequios-votivos-hacer-ofrendas-apoyo-de-funcionarios-reales}}

\bibleverse{31} El duodécimo día del primer mes, salimos del Canal de
Ahava para ir a Jerusalén, y nuestro Dios estaba con nosotros para
protegernos de las emboscadas enemigas en el camino. \bibleverse{32}
Finalmente llegamos a Jerusalén y descansamos allí durante tres días.
\bibleverse{33} Al cuarto día, la plata y el oro y los objetos del
Templo fueron pesados en el Templo de nuestro Dios y entregados a
Meremot, hijo del sacerdote Urías, acompañado por Eleazar, hijo de
Finees. También estaban presentes los levitas Jozabad, hijo de Jesúa, y
Noadías, hijo de Binui. \bibleverse{34} Todo fue revisado, tanto en
número como en peso, y el peso total fue anotado en ese momento.

\bibleverse{35} Entonces los exiliados que habían regresado del
cautiverio sacrificaron holocaustos al Dios de Israel: doce toros por
todo Israel, noventa y seis carneros, setenta y siete corderos y una
ofrenda por el pecado de doce cabras. Todo fue sacrificado como
holocausto al Señor. \bibleverse{36} También entregaron los decretos del
rey a los oficiales principales\footnote{\textbf{8:36} ``Oficiales
  principales'': literalmente, ``Sátrapas''.} del rey y de los
gobernadores de la provincia al oeste del Éufrates, que entonces
prestaron asistencia al pueblo y al Templo de Dios.

\hypertarget{esra-se-da-cuenta-de-los-matrimonios-mixtos-su-consternaciuxf3n-por-estos-funcionarios}{%
\subsection{Esra se da cuenta de los matrimonios mixtos; su
consternación por estos
funcionarios}\label{esra-se-da-cuenta-de-los-matrimonios-mixtos-su-consternaciuxf3n-por-estos-funcionarios}}

\hypertarget{section-8}{%
\section{9}\label{section-8}}

\bibleverse{1} Algún tiempo después de todo esto, los líderes\footnote{\textbf{9:1}
  Estos eran líderes civiles, no líderes religiosos.} vino y me dijo:
``El pueblo de Israel, incluidos los sacerdotes y los levitas, no se ha
mantenido separado de los pueblos que nos rodean, cuyas repugnantes
prácticas religiosas son similares a las de los cananeos, hititas,
ferezeos, jebuseos, amonitas, moabitas, egipcios y amorreos.
\bibleverse{2} Algunos israelitas incluso se han casado con mujeres de
estos pueblos, tanto ellos como sus hijos, mezclando la raza santa con
estos pueblos de la tierra. Nuestros líderes y funcionarios están al
frente de este comportamiento pecaminoso''. \footnote{\textbf{9:2} Esd
  9,11-12; Neh 13,23}

\bibleverse{3} Cuando me enteré de esto, me rasgué las vestiduras, me
arranqué un poco de pelo de la cabeza y de la barba y me senté,
absolutamente horrorizado. \footnote{\textbf{9:3} Gén 37,34}
\bibleverse{4} Todos los que respetaban las instrucciones del Dios de
Israel\footnote{\textbf{9:4} ``Todos los que respetaban las
  instrucciones del Dios de Israel'': literalmente, ``Todos los que
  temían las palabras del Dios de Israel''. El énfasis aquí no es tanto
  el miedo como la disposición a seguir la instrucción y la obediencia a
  lo que Dios había dicho.} se reunieron a mi alrededor por este pecado
de los exiliados. Me senté allí, conmocionado y horrorizado, hasta el
sacrificio de la tarde.

\hypertarget{la-oraciuxf3n-penitencial-de-esdras}{%
\subsection{La oración penitencial de
Esdras}\label{la-oraciuxf3n-penitencial-de-esdras}}

\bibleverse{5} A la hora del sacrificio vespertino, me levanté de donde
había estado sentado, apesadumbrado, con mis ropas rasgadas, y me
arrodillé y extendí mis manos al Señor, mi Dios. \bibleverse{6} Oré:
``Dios mío, me siento tan avergonzado y abochornado de venir a orar a
ti,\footnote{\textbf{9:6} ``Venir a orar por ti'': literalmente, ``Alzar
  mi rostro hacia ti''.} Dios mío, porque estamos sobrepasados por el
pecado, y nuestra culpa ha subido a los cielos. \footnote{\textbf{9:6}
  Dan 9,7-8; Sal 38,5} \bibleverse{7} Desde el tiempo de nuestros
antepasados hasta ahora, hemos sido profundamente culpables. A causa de
nuestros pecados, nosotros, nuestros reyes y nuestros sacerdotes hemos
sido entregados a los reyes de la tierra, asesinados y hechos
prisioneros, robados y humillados, como lo somos hoy. \bibleverse{8}
``Ahora, por un corto tiempo, el Señor, nuestro Dios, nos ha dado
gracia, preservando a algunos de nosotros como un remanente, y dándonos
seguridad\footnote{\textbf{9:8} ``Seguridad'': literalmente,
  ``picaporte''.} en su lugar santo. Nuestro Dios ha iluminado nuestras
vidas\footnote{\textbf{9:8} ``Vidas'': literalmente, ``ojos''.} dándonos
un alivio de nuestra esclavitud. \bibleverse{9} Aunque somos esclavos,
nuestro Dios no nos ha abandonado en nuestra esclavitud, sino que nos ha
mostrado su amor confiable al hacer que los reyes de Persia sean
bondadosos con nosotros, al revivirnos para que podamos reconstruir el
Templo de nuestro Dios y reparar su estado ruinoso, y al darnos un muro
de protección alrededor de Judá y Jerusalén. \footnote{\textbf{9:9} Is
  5,5}

\bibleverse{10} ``Pero ahora, Dios nuestro, ¿qué tenemos que decir en
nuestro favor después de todo esto? Porque hemos renunciado a seguir tus
mandatos \bibleverse{11} que diste por medio de tus siervos los
profetas, diciéndonos: `La tierra en la que vais a entrar para
convertiros en sus dueños está contaminada por los pecados de sus
pueblos, por las repugnantes prácticas religiosas de las que la han
llenado, de un lado a otro. \bibleverse{12} Por tanto, no permitan que
sus hijas se casen con sus hijos, ni que sus hijas se casen con vuestros
hijos. No hagan nunca un tratado de paz o de amistad con ellos, para que
puedan vivir bien y comer los buenos alimentos que produce la tierra, y
dar la tierra como herencia a vuestros hijos para siempre'. \footnote{\textbf{9:12}
  Deut 7,2-3}

\bibleverse{13} ``Ahora que estamos recibiendo todo este castigo a causa
de nuestras acciones pecaminosas y nuestra terrible culpa -- aunquetú,
nuestro Dios, no nos has castigado tanto como merecen nuestros pecados,
y aún nos has dado este remanente\footnote{\textbf{9:13} ``Remanente'':
  refiriéndose a los que habían regresado del exilio.} ---
\bibleverse{14} ¿acaso que brantaremos otra vez tus mandamientos para
casarnos con los pueblos que cometen estas prácticas religiosas
abominables? ¿Acaso no te enfadarías tanto con nosotros y hasta nos
destrurías? No quedaría ningún remanente, ni un solo superviviente.
\bibleverse{15} Señor, Dios de Israel, tú procedes con justicia. Hoy
somos todo lo que queda, un remanente. Estamos ante ti con nuestra
culpa, y por su causa nadie puede permanecer ante ti''.

\hypertarget{la-acciuxf3n-contra-los-matrimonios-mixtos}{%
\subsection{La acción contra los matrimonios
mixtos}\label{la-acciuxf3n-contra-los-matrimonios-mixtos}}

\hypertarget{section-9}{%
\section{10}\label{section-9}}

\bibleverse{1} Mientras Esdras oraba y confesaba sus pecados, llorando y
cayendo de bruces ante el Templo de Dios, una gran multitud de
israelitas, hombres, mujeres y niños, se reunió a su alrededor. El
pueblo también lloraba amargamente. \bibleverse{2} Secanías, hijo de
Jehiel, un elamita, dijo a Esdras: ``Sí, hemos sido infieles a nuestro
Dios porque nos hemos casado con mujeres extranjeras de los pueblos de
la tierra. Pero aun así, todavía hay esperanza para Israel en cuanto a
esto. \bibleverse{3} Hagamos un acuerdo solemne ahora mismo ante nuestro
Dios de que despediremos a todas las esposas extranjeras y a sus hijos.
Seguiremos las instrucciones dadas por ti y por los que respetan las
instrucciones de nuestro Dios, llevadas a cabo de acuerdo con la Ley.
\bibleverse{4} ¡Haz algo! Porque es tu responsabilidad. Estamos contigo.
Sé valiente y hazlo''.

\bibleverse{5} Entonces Esdras se puso de pie e hizo que los principales
sacerdotes, los levitas y todos los israelitas presentes prestaran
juramento de actuar conforme a lo que se acababa de decir. Todos
hicieron el juramento. \bibleverse{6} Entonces Esdras los dejó frente al
Templo de Dios y se dirigió a la habitación de Johanán, hijo de Eliasib.
Durante el tiempo que permaneció allí, no comió ni bebió nada, porque
seguía lamentando la infidelidad de los exiliados. \bibleverse{7}
Entonces se emitió una proclama en todo Judá y Jerusalén para que todos
los exiliados se reunieran en Jerusalén. \bibleverse{8} Al que no
viniera en el plazo de tres días se le confiscarían todos sus bienes y
se le prohibiría participar en la asamblea de los exiliados. Esta fue la
decisión de los líderes y de los ancianos.

\bibleverse{9} A los tres días, todos los de Judá y Benjamín se
reunieron en Jerusalén. El vigésimo día del noveno mes, todo el pueblo
se sentó en la plaza junto al Templo de Dios, temblando por este asunto
y también por la fuerte lluvia.

\bibleverse{10} El sacerdote Esdras se levantó y les dijo: ``Ustedes han
cometido un pecado al casarse con mujeres extranjeras, agravando aún más
la culpa de Israel. \bibleverse{11} Ahora deben confesar su pecado al
Señor, el Dios de sus antepasados, y hacer lo que él les pide. Corta tus
vínculos con la gente de la tierra y con tus esposas extranjeras''.

\bibleverse{12} Toda la asamblea respondió en voz alta: ``¡Estamos de
acuerdo y prometemos hacer lo que dices! \bibleverse{13} Pero hay mucha
gente aquí, y está lloviendo a cántaros. No podemos quedarnos fuera. Más
aún, esto no es algo que se pueda arreglar en uno o dos días, pues hemos
pecado gravemente en esto. \bibleverse{14} Que nuestros líderes actúen
en nombre de toda la asamblea. Entonces, que cada hombre de cada una de
nuestras ciudades que se haya casado con una mujer extranjera reciba una
cita para venir a reunirse, junto con los ancianos y los jueces de esa
ciudad, hasta que nuestro Dios deje de estar enojado con nosotros por
esto''.

\bibleverse{15} Los únicos que se opusieron a esto fueron Jonatán, hijo
de Asahel, y Jahzeías, hijo de Ticva, apoyados por Mesulam y el levita
Sabetai.

\bibleverse{16} Esto fue lo que hicieron los exiliados, seleccionando al
sacerdote Esdras y a los jefes de familia, según sus divisiones
familiares, todos ellos nombrados específicamente. El primer día del
décimo mes se sentaron para comenzar la investigación, \bibleverse{17} y
para el primer día del primer mes habían terminado de tratar todos los
casos de hombres que se habían casado con mujeres extranjeras.

\hypertarget{lista-de-sacerdotes-levitas-y-laicos-que-se-casaron-con-mujeres-extrauxf1as}{%
\subsection{Lista de sacerdotes, levitas y laicos que se casaron con
mujeres
extrañas}\label{lista-de-sacerdotes-levitas-y-laicos-que-se-casaron-con-mujeres-extrauxf1as}}

\bibleverse{18} Entre los descendientes de los sacerdotes, los
siguientes se habían casado con mujeres extranjeras: de los hijos de
Jesúa hijo de Josadac, y de sus hermanos Maasías, Eliezer, Jarib y
Gedalías. \footnote{\textbf{10:18} Esd 3,2; Esd 9,2} \bibleverse{19}
Hicieron voto de despedir a sus mujeres, y presentaron un carnero del
rebaño como ofrenda por su culpa. \bibleverse{20} De los hijos de Imer:
Hanani y Zebadías. \bibleverse{21} De los hijos de Harim Maasías, Elías,
Semaías, Jehiel y Uzías. \bibleverse{22} De los hijos de Pasur Elioenai,
Maasías, Ismael, Natanael, Jozabad y Elasa. \bibleverse{23} De los
levitas: Jozabad, Simei, Kelaía (o Kelita), Petaías, Judá y Eliezer.
\bibleverse{24} Entre los cantores: Eliasib. Entre los porteros: Salum,
Telem y Uri. \bibleverse{25} Entre los israelitas: De los hijos de
Paros: Ramía, Jezías, Malquías, Mijamín, Eleazar, Hasabías,\footnote{\textbf{10:25}
  Siguiendo la lista paralela de 1 Esdras 9:26. El texto hebreo es
  Malquías, pero ya ha sido enumerado en el mismo versículo.} y Benaía.
\bibleverse{26} De los hijos de Elam: Matanías, Zacarías, Jehiel, Abdi,
Jeremot y Elías. \bibleverse{27} De los hijos de Zatu: Elioenai,
Eliasib, Matanías, Jeremot, Zabad y Aziza. \bibleverse{28} De los hijos
de Bebai Johanán, Ananías, Zabai y Atlai. \bibleverse{29} De los hijos
de Bani Mesulam, Maluc, Adaía, Jasub, Seal y Jeremot. \bibleverse{30} De
los hijos de Pahat-moab Adna, Quelal, Benaías, Maasías, Matanías,
Bezaleel, Binui y Manasés. \bibleverse{31} De los hijos de Harim
Eliezer, Isías, Malquías, Semaías, Simeón, \bibleverse{32} Benjamín,
Maluc y Semarías. \bibleverse{33} De los hijos de Hasum Matenai, Matata,
Zabad, Elifelet, Jeremai, Manasés y Simei. \bibleverse{34} De los hijos
de Baní Madai, Amram, Uel, \bibleverse{35} Benaías, Bedías, Quelúhi,
\bibleverse{36} Vanías, Meremot, Eliasib, \bibleverse{37} Matanías,
Matenai y Jaasai. \bibleverse{38} De los hijos de Binui:\footnote{\textbf{10:38}
  Tomado de la Septuaginta.} Simei, \bibleverse{39} Selemías, Natán,
Adaía, \bibleverse{40} Macnadebai, Shashai, Sharai, \bibleverse{41}
Azarel, Selemías, Semarías, \bibleverse{42} Salum, Amarías y José.
\bibleverse{43} De los hijos de Nebo: Jeiel, Matatías, Zabad, Zebina,
Jadau, Joel y Benaías.

\bibleverse{44} Todos estos hombres mencionados se habían casado con
mujeres extranjeras. Se divorciaron de ellas\footnote{\textbf{10:44}
  ``Se divorciaron de ellas'': implícito.} y las despidieron con sus
hijos.\footnote{\textbf{10:44} ``Se divorciaron de ellas y la
  despidieron con sus hijos'', o ``y algunos de ellos tenían esposas con
  las que tenían hijos''. El hebreo no está claro. La Septuaginta de 1
  Esdras 9:36 da la primera traducción.}
