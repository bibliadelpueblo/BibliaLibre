\hypertarget{bendiciones}{%
\subsection{Bendiciones}\label{bendiciones}}

\hypertarget{section}{%
\section{1}\label{section}}

\bibleverse{1} Esta carta viene de parte de Pablo, apóstol de Cristo
Jesús y escogido por Dios, y es enviada con el fin de contar sobre la
promesa de una vida real,\footnote{\textbf{1:1} El griego solo usa la
  palabra ``vida'', pero Pablo aquí está haciendo referencia a la vida
  abundante que se refiere a su vez a la vida eterna (ver 1 Timoteo
  1:16).} es decir, en Cristo Jesús. \bibleverse{2} Te la envío a ti,
Timoteo, mi querido Hijo. Ten gracia, misericordia, y paz de parte de
Dios el Padre y de Cristo Jesús, nuestro Señor.

\hypertarget{acciuxf3n-de-gracias-del-apuxf3stol-por-la-firmeza-de-la-fe-de-timoteo}{%
\subsection{Acción de gracias del apóstol por la firmeza de la fe de
Timoteo}\label{acciuxf3n-de-gracias-del-apuxf3stol-por-la-firmeza-de-la-fe-de-timoteo}}

\bibleverse{3} Siempre pienso en ti y estoy muy agradecido con Dios, a
quien sirvo así como lo hicieron mis ancestros, con una clara
conciencia. Nunca te olvido en mis oraciones. \bibleverse{4} ¡Recuerdo
cuánto llorabas y deseo tanto verte! Eso me haría realmente feliz.
\footnote{\textbf{1:4} 2Tim 4,9} \bibleverse{5} En mi mente siempre está
el recuerdo de tu fe sincera en Dios, la misma fe que tenían tu abuela
Loida y tu madre Eunice, y sé que esa misma fe sigue viva en ti.
\footnote{\textbf{1:5} Hech 16,1-3}

\hypertarget{el-regalo-de-dios-mantiene-a-timoteo-y-pablo-conectados}{%
\subsection{El regalo de Dios mantiene a Timoteo y Pablo
conectados}\label{el-regalo-de-dios-mantiene-a-timoteo-y-pablo-conectados}}

\bibleverse{6} Por eso quiero recordarte que debes revitalizar el don de
la gracia de Dios que recibiste cuando puse mis manos sobre
ti.\footnote{\textbf{1:6} Sin duda Pablo ``impuso sus manos'' sobre
  Timoteo como una forma de denominar una bendición especial.}
\footnote{\textbf{1:6} 1Tim 4,14} \bibleverse{7} Dios no nos dio un
espíritu de temor, sino un espíritu de poder, de amor y de cordura.
\footnote{\textbf{1:7} Rom 8,15} \bibleverse{8} Así mismo no se
avergüencen de contar a otros sobre nuestro Señor, ni se avergüencen de
mí. En lugar de ello, estén listos para participar del sufrimiento por
causa de la buena noticia a medida que Dios los fortalece. \footnote{\textbf{1:8}
  Rom 1,16} \bibleverse{9} Él es el que nos ha salvado y nos ha llamado
para vivir una vida santa, no por medio de lo que hacemos, sino por
medio del propio plan de Dios y por medio de su gracia. \footnote{\textbf{1:9}
  Tit 3,5} \bibleverse{10} Él nos dio esta gracia en Cristo Jesús antes
del principio de los tiempos, y ahora está revelada en la aparición de
nuestro Salvador Cristo Jesús. Él destruyó la muerte, dejando en
evidencia la vida y la inmortalidad por medio de la buena noticia.
\footnote{\textbf{1:10} 1Cor 15,55; 1Cor 15,57; Heb 2,14}
\bibleverse{11} Fui designado como predicador, apóstol y maestro de esta
buena noticia. \footnote{\textbf{1:11} 1Tim 2,7}

\hypertarget{referencia-al-ejemplo-del-apuxf3stol-la-infidelidad-de-algunos-hermanos-y-el-comportamiento-glorioso-de-onesuxedforo}{%
\subsection{Referencia al ejemplo del apóstol, la infidelidad de algunos
hermanos y el comportamiento glorioso de
Onesíforo}\label{referencia-al-ejemplo-del-apuxf3stol-la-infidelidad-de-algunos-hermanos-y-el-comportamiento-glorioso-de-onesuxedforo}}

\bibleverse{12} Esa también es una razón por la cual sufro todas estas
cosas, pero no me avergüenzo, porque sé en quién he confiado. Estoy
seguro de que él puede cuidar de lo que le he confiado hasta el
Día\footnote{\textbf{1:12} ``Día'', haciendo referencia al Día del
  Juicio del fin de los tiempos.} de su regreso.

\bibleverse{13} Deberían seguir el modelo del buen consejo que
aprendieron de mí, con una actitud de fe y amor en Cristo Jesús.
\footnote{\textbf{1:13} 1Tim 6,3; Tit 2,1} \bibleverse{14} Guarden la
verdad que les fue confiada por medio del Espíritu Santo que vive en
nosotros. \footnote{\textbf{1:14} 1Tim 6,20}

\bibleverse{15} Ustedes ya saben que todos los de Asia\footnote{\textbf{1:15}
  La provincia romana de Asia Menor (En la actualidad es Turquía).} me
abandonaron, incluso Figelo y Hermógenes. \footnote{\textbf{1:15} 2Tim
  4,16} \bibleverse{16} Que el señor sea bondadoso con la familia de
Onesíforo, porque a menudo me cuidó y no se avergonzaba de que yo
estuviera en la cárcel. \footnote{\textbf{1:16} 2Tim 4,20}

\bibleverse{17} Cuando estuve en Roma, se tomó la molestia de buscarme y
me encontró. \bibleverse{18} Que el Señor le otorgue su bendición en el
Día del Juicio. (Timoteo, tu eres muy consciente de cuántas cosas
Onesíforo hizo por mi cuando estuve en Éfeso).

\hypertarget{exhortaciuxf3n-a-timoteo-para-que-se-preocupe-por-la-predicaciuxf3n-de-la-doctrina-de-la-salvaciuxf3n-y-se-fortalezca-en-la-batalla-y-el-sufrimiento}{%
\subsection{Exhortación a Timoteo para que se preocupe por la
predicación de la doctrina de la salvación y se fortalezca en la batalla
y el
sufrimiento}\label{exhortaciuxf3n-a-timoteo-para-que-se-preocupe-por-la-predicaciuxf3n-de-la-doctrina-de-la-salvaciuxf3n-y-se-fortalezca-en-la-batalla-y-el-sufrimiento}}

\hypertarget{section-1}{%
\section{2}\label{section-1}}

\bibleverse{1} Así que, hijo mío, sé fuerte en la gracia de Cristo
Jesús. \footnote{\textbf{2:1} Efes 6,10} \bibleverse{2} Toma todo lo que
me escuchaste decir delante de muchos testigos y compártelo con personas
fieles, que luego también las enseñen a otros. \bibleverse{3} Sufre
conmigo como un buen soldado de Cristo Jesús. \bibleverse{4} Un soldado
activo que no se enreda con los asuntos de la vida diaria. Uno que
quiere agradar a quien lo reclutó. \bibleverse{5} Del mismo modo, los
atletas que compiten en los juegos no ganan un premio si no siguen las
normas. \footnote{\textbf{2:5} 1Cor 9,24-27; 2Tim 4,8} \bibleverse{6} El
granjero que hace todo el trabajo duro debe ser el primero en
beneficiarse de la cosecha. \bibleverse{7} Considera todo lo que te
digo. Y el Señor te ayudará a comprender todas estas cosas.

\hypertarget{fuerza-y-consuelo-del-guerrero-de-cristo-en-la-batalla-del-sufrimiento}{%
\subsection{Fuerza y \hspace{0pt}\hspace{0pt}consuelo del guerrero de
Cristo en la batalla del
sufrimiento}\label{fuerza-y-consuelo-del-guerrero-de-cristo-en-la-batalla-del-sufrimiento}}

\bibleverse{8} Fija tu mente en Jesucristo, descendiente de David, que
fue levantado de los muertos. Esta es mi buena noticia \bibleverse{9} y
estoy sufriendo en la cárcel como si fuese un criminal, pero la palabra
de Dios no está en una cárcel. \footnote{\textbf{2:9} Fil 1,12-14}
\bibleverse{10} A pesar de todo esto, estoy dispuesto a continuar por la
causa del pueblo de Dios\footnote{\textbf{2:10} Literalmente, ``el
  elegido''.} para que puedan recibir la salvación de Cristo Jesús, que
es su gloria eterna. \footnote{\textbf{2:10} Col 1,24} \bibleverse{11}
Este decir es sabio: ``Si morimos con él, también viviremos con él;
\footnote{\textbf{2:11} 2Cor 4,11} \bibleverse{12} si persistimos,
también reinaremos con él; si lo negamos, él también nos negará.
\footnote{\textbf{2:12} Mat 10,33} \bibleverse{13} Si somos infieles, él
sigue siendo fiel, porque él no puede ser infiel consigo mismo''.
\footnote{\textbf{2:13} Núm 23,19; Sal 89,31-34; Rom 3,2-3; Tit 1,2}

\hypertarget{advertencia-de-verborrea-inuxfatil-de-chuxe1chara-vacuxeda-y-de-falsedades-de-falsos-maestros}{%
\subsection{Advertencia de verborrea inútil, de cháchara vacía y de
falsedades de falsos
maestros}\label{advertencia-de-verborrea-inuxfatil-de-chuxe1chara-vacuxeda-y-de-falsedades-de-falsos-maestros}}

\bibleverse{14} Esas son las cosas que debes recordarle a la gente,
diciéndoles ante Dios que no tengan discusiones vanas en cuanto a las
palabras. Porque hacer esto solo hace daño a quien escucha. \footnote{\textbf{2:14}
  1Tim 6,4; Tit 3,9}

\bibleverse{15} Esfuérzate arduamente en poder presentarte ante Dios y
ser aprobado por él. Sé un obrero que no tenga nada de qué avergonzarse,
usando correctamente la palabra de verdad. \footnote{\textbf{2:15} 1Tim
  4,6; Tit 2,7; Tit 1,2-8} \bibleverse{16} Evita las conversaciones
inútiles, pues los que hacen esto están lejos de Dios en su caminar.
\footnote{\textbf{2:16} 1Tim 4,7} \bibleverse{17} Sus enseñanzas son
destructivas como la gangrena que destruye la carne que está sana. Así
son Himeneo y Fileto. \footnote{\textbf{2:17} 1Tim 1,20} \bibleverse{18}
Ellos se han desviado de la verdad al decir que la resurrección ya
ocurrió, y esto ha destruido la fe de algunos.

\hypertarget{la-desesperanza-de-los-falsos-maestros-debido-al-suxf3lido-fundamento-de-la-iglesia-puesto-por-dios}{%
\subsection{La desesperanza de los falsos maestros debido al sólido
fundamento de la iglesia puesto por
Dios}\label{la-desesperanza-de-los-falsos-maestros-debido-al-suxf3lido-fundamento-de-la-iglesia-puesto-por-dios}}

\bibleverse{19} Pero el fundamento sólido y fiel de Dios se mantiene
firme, con esta inscripción: ``El Señor conoce a los que son suyos'', y
``Todo el que invoque el nombre del Señor está apartado de todo
mal''.\footnote{\textbf{2:19} Citando Números 16:5.}

\bibleverse{20} Una casa majestuosa no solo tiene copas y
tazas\footnote{\textbf{2:20} Literalmente ``vasijas'' o ``utensilios''.
  Parece que no existe en nuestro idioma un buen equivalente a
  ``recipientes de casa''.} de oro y plata. También tiene algunas de
madera y barro. Algunas son para uso especial; otras para funciones más
comunes. \bibleverse{21} Así que si te despojas de lo malo, serás una
vasija especial, que es santa y única, útil para el Señor, lista para
hacer lo bueno.

\bibleverse{22} Huye de todo lo que incite tus deseos juveniles. Busca
las cosas justas y rectas, busca el amor y la paz así como a los que son
cristianos y verdaderos. \footnote{\textbf{2:22} 1Tim 4,12; 1Tim 6,11;
  Heb 12,14} \bibleverse{23} Evita las discusiones inmaduras y necias,
pues tú sabes que esto solo conduce a contiendas. \footnote{\textbf{2:23}
  1Tim 4,7} \bibleverse{24} Porque el ministro del Señor no debe entrar
en contiendas, sino ser amable con todos, capaz de enseñar, paciente,
\footnote{\textbf{2:24} Tit 1,7}

\bibleverse{25} mansos para corregir a los que se oponen. Porque puede
ser que a esos Dios les ayude a arrepentirse y entender la verdad.
\bibleverse{26} Así podrán entrar en razón y escapar de la trampa del
diablo. Porque él los ha capturado para que hagan su voluntad.

\hypertarget{descripciuxf3n-de-los-futuros-falsos-maestros-y-la-corrupciuxf3n-moral-del-uxfaltimo-tiempo}{%
\subsection{Descripción de los futuros falsos maestros y la corrupción
moral del último
tiempo}\label{descripciuxf3n-de-los-futuros-falsos-maestros-y-la-corrupciuxf3n-moral-del-uxfaltimo-tiempo}}

\hypertarget{section-2}{%
\section{3}\label{section-2}}

\bibleverse{1} Debes saber que habrá momentos difíciles en los últimos
días. \bibleverse{2} Habrá personas amadoras de sí mismas y del dinero.
Serán jactanciosas, arrogantes, abusivas, desobedientes a sus padres,
ingratas, y con ausencia de Dios en sus vidas. \bibleverse{3} Con
crueldad y sin perdón calumniarán y carecerán de dominio propio. Serán
personas despiadadas que odian el bien, \bibleverse{4} y engañarán a
otros, con total desconsideración. Son personas absurdamente vanidosas,
que viven con tanto interés por el placer que no se preocuparán por amar
a Dios. \bibleverse{5} Estas personas podrían dar una impresión externa
de piedad, pero realmente no creen en su eficacia. ¡Aléjate de tales
personas! \footnote{\textbf{3:5} Mat 7,15; Mat 7,21; Tit 1,16}
\bibleverse{6} Esa es la clase de personas que sutilmente entra a los
hogares y toman el control de esas mujeres vulnerables que cargan con la
culpa del pecado y se distraen con todo tipo de deseos. \bibleverse{7}
Estas mujeres siempre están intentando aprender pero nunca pueden
entender la verdad. \bibleverse{8} Así como Janes y Jambres se opusieron
a Moisés, estos maestros también se oponen a la verdad. Son personas con
mentes corruptas cuya supuesta fe en Dios es una mentira.\footnote{\textbf{3:8}
  O ``cuya fe en Dios es falsa''.} \bibleverse{9} Pero estas personas no
llegan muy lejos. Su estupidez será evidente para todos, así como la de
Janes y Jambres.

\hypertarget{referencia-al-ejemplo-de-pablo-y-un-recordatorio-de-perseverar-y-aferrarse-a-las-enseuxf1anzas-tradicionales-y-las-sagradas-escrituras-a-pesar-de-todo-el-sufrimiento}{%
\subsection{Referencia al ejemplo de Pablo y un recordatorio de
perseverar y aferrarse a las enseñanzas tradicionales y las Sagradas
Escrituras a pesar de todo el
sufrimiento}\label{referencia-al-ejemplo-de-pablo-y-un-recordatorio-de-perseverar-y-aferrarse-a-las-enseuxf1anzas-tradicionales-y-las-sagradas-escrituras-a-pesar-de-todo-el-sufrimiento}}

\bibleverse{10} Pero tú conoces mi enseñanza y mi conducta, así como mi
objetivo en la vida. Conoces mi fe en Dios y mi amor. Sabes lo que he
tenido que soportar, \bibleverse{11} y cuánto he sufrido y he sido
perseguido. Ya sabes lo que me sucedió en Antioquía, en Iconio y Listra,
y las persecuciones que tuve y cómo el Señor me rescató de todas esas
cosas. \footnote{\textbf{3:11} Hech 13,13-999; Sal 34,20}
\bibleverse{12} Sin duda, todos los que quieren vivir una vida de
devoción a Dios en Cristo Jesús experimentarán persecución, \footnote{\textbf{3:12}
  Mat 16,24; Hech 14,22; 1Tes 3,3} \bibleverse{13} mientras que las
personas malas y los fraudulentos prosperen, siendo malos y después
peores, engañando a los demás y engañándose ellos mismos también.
\footnote{\textbf{3:13} 1Tim 4,1} \bibleverse{14} Pero tú mantente fiel
a las cosas que has aprendido y que sabes que son verdaderas. Porque
sabes quién te las enseñó. \bibleverse{15} Desde tu niñez has conocido
las Escrituras que pueden darte entendimiento para la salvación por la
fe en Cristo Jesús. \bibleverse{16} Toda la Escritura inspirada por Dios
es útil para enseñar, para confrontar lo que está mal, para enderezar
nuestro camino, y para enseñarnos lo recto. \footnote{\textbf{3:16} 2Pe
  1,19-21} \bibleverse{17} Así es como Dios provee una preparación
completa para aquellos que trabajan para él, para lograr todo lo que es
bueno.\footnote{\textbf{3:17} 1Tim 6,11}

\hypertarget{otro-llamado-a-timoteo-para-que-sea-fiel-a-su-oficio}{%
\subsection{Otro llamado a Timoteo para que sea fiel a su
oficio}\label{otro-llamado-a-timoteo-para-que-sea-fiel-a-su-oficio}}

\hypertarget{section-3}{%
\section{4}\label{section-3}}

\bibleverse{1} Te pido, ante Dios y ante Cristo Jesús, que juzgará a los
vivos y a los muertos cuando venga a establecer su reino: \footnote{\textbf{4:1}
  1Pe 4,5} \bibleverse{2} Que prediques la palabra de Dios, sea
conveniente o no, y dile a las personas lo que están haciendo mal; dales
consejo y ánimo. Y enséñales esto con mucha paciencia. \footnote{\textbf{4:2}
  Hech 20,20; Hech 20,31} \bibleverse{3} Pues viene el tiempo cuando las
personas no se interesarán en escuchar la verdadera enseñanza. Sino que
tendrán curiosidad por oír algo diferente,\footnote{\textbf{4:3}
  Literalmente ``tendrán picor en los oídos''.} y se rodearán de
maestros que les enseñen lo que quieren oír. \footnote{\textbf{4:3} 1Tim
  4,1} \bibleverse{4} Dejarán de escuchar la verdad y andarán errantes,
siguiendo mitos. \footnote{\textbf{4:4} 1Tim 4,7; 2Tes 2,11}
\bibleverse{5} Debes mantenerte alerta todo el tiempo. Haz frente a las
dificultades, trabaja en la predicación de la buena noticia, y cumple tu
ministerio. \footnote{\textbf{4:5} 2Tim 2,3}

\hypertarget{solemne-referencia-del-apuxf3stol-al-pruxf3ximo-final-de-su-vida-su-auto-testimonio-y-su-esperanza}{%
\subsection{Solemne referencia del apóstol al próximo final de su vida;
su auto-testimonio y su
esperanza}\label{solemne-referencia-del-apuxf3stol-al-pruxf3ximo-final-de-su-vida-su-auto-testimonio-y-su-esperanza}}

\bibleverse{6} Pues estoy a punto de ser sacrificado, y se aproxima la
hora de mi muerte. \footnote{\textbf{4:6} Fil 2,17} \bibleverse{7} He
peleado la buena batalla, he terminado la carrera, y he mantenido mi fe
en Dios. \footnote{\textbf{4:7} Hech 20,24; 1Cor 9,25; Fil 3,14; 1Tim
  6,12} \bibleverse{8} Ahora tengo un premio reservado, la corona de la
vida, conforme a lo que es justo. El Señor, (que es el juez que siempre
hace justicia), me dará ese premio ese Día.\footnote{\textbf{4:8} Ver
  nota sobre el versículo 1:12.} Y no solo a mí, sino a todos los que
anhelan su venida. \footnote{\textbf{4:8} 2Tim 2,5; Mat 25,21; 1Pe 5,4;
  Sant 1,12; Apoc 2,10}

\hypertarget{situaciuxf3n-personal-del-apuxf3stol-uxfaltimas-uxf3rdenes-peticiones-mensajes-saludos-y-bendiciones}{%
\subsection{Situación personal del apóstol, últimas órdenes, peticiones,
mensajes, saludos y
bendiciones}\label{situaciuxf3n-personal-del-apuxf3stol-uxfaltimas-uxf3rdenes-peticiones-mensajes-saludos-y-bendiciones}}

\bibleverse{9} Por favor, procura venir a visitarme tan pronto como
puedas. \footnote{\textbf{4:9} 2Tim 1,4} \bibleverse{10} Demas me ha
abandonado porque tiene más amor por las cosas de este mundo, y se fue a
Tesalónica. Crescente se fue a Galacia, y Tito a Dalmacia. \footnote{\textbf{4:10}
  Col 4,7; Col 4,10; Col 4,14} \bibleverse{11} Solamente Lucas está aquí
conmigo. Trae contigo a Marcos, porque él puede ayudarme en mi obra.
\footnote{\textbf{4:11} Hech 15,37; Col 4,10} \bibleverse{12} Envié a
Tíquico a Éfeso. \footnote{\textbf{4:12} Efes 6,21} \bibleverse{13} Por
favor, cuando vengas, trae el abrigo que dejé donde Carpo en Troas, y
los libros, especialmente los pergaminos. \bibleverse{14} Alexander, el
herrero, me causó muchos problemas. Que Dios lo juzgue por lo que hizo.
\footnote{\textbf{4:14} 1Tim 1,20} \bibleverse{15} Cuídate tú también de
él, porque ejerció gran oposición a lo que decíamos.

\bibleverse{16} La primera vez que tuve que defenderme,\footnote{\textbf{4:16}
  Refiriéndose a un juicio en la corte.} nadie estuvo allí
acompañándome, sino que todos me abandonaron. Ojalá no se les tenga en
cuenta. \bibleverse{17} Pero el Señor estuvo conmigo y me dio fuerzas
para declarar todo el mensaje,\footnote{\textbf{4:17} Literalmente,
  ``Gentiles''.} de modo que los extranjeros pudieron oírlo. ¡Fui
rescatado de la boca del león! \footnote{\textbf{4:17} Hech 23,11; Hech
  27,23} \bibleverse{18} El Señor me rescatará de todas las cosas malas
que me han hecho, y me llevará salvo a su reino. Porque suya es la
gloria por siempre y para siempre. Amén.

\bibleverse{19} Mis saludos a Prisca\footnote{\textbf{4:19} Llamada
  Priscila en Hechos 18:2.} y a Aquiles, y a la familia de Onesíforo.
\bibleverse{20} Erasto se quedó en Corinto, y dejé a Trófimo en Mileto
porque se enfermó. \footnote{\textbf{4:20} Hech 19,22; Hech 20,4; 2Tim
  1,16} \bibleverse{21} Por favor, procura venir antes del invierno.
Eubulo te envía sus saludos, así como Pudente, Lino, Claudia y todos los
hermanos y hermanas también.

\bibleverse{22} Que el Señor esté contigo.\footnote{\textbf{4:22}
  Literalmente, ``sea con tu espíritu''.} Que su gracia esté con todos
ustedes.
