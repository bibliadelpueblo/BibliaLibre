\hypertarget{bendiciones}{%
\subsection{Bendiciones}\label{bendiciones}}

\hypertarget{section}{%
\section{1}\label{section}}

\bibleverse{1} Esta carta viene de Pedro, apóstol de Jesucristo, y es
enviada al pueblo escogido de Dios: a los exiliados que están dispersos
por todas las provincias de Ponto, Galacia, Capadocia, Asia, y Bitinia.
\bibleverse{2} Ustedes fueron elegidos por Dios, el Padre, en su
sabiduría, y son un pueblo santo por el Espíritu, que obedece a
Jesucristo y que está rociado con su sangre. Tengan gracia y paz cada
vez más. \footnote{\textbf{1:2} Rom 8,29}

\hypertarget{la-salvaciuxf3n-que-se-nos-da-da-gozo-jubiloso-incluso-si-la-fe-tiene-que-demostrar-su-valuxeda-en-la-tribulaciuxf3n}{%
\subsection{La salvación que se nos da da gozo jubiloso, incluso si la
fe tiene que demostrar su valía en la
tribulación}\label{la-salvaciuxf3n-que-se-nos-da-da-gozo-jubiloso-incluso-si-la-fe-tiene-que-demostrar-su-valuxeda-en-la-tribulaciuxf3n}}

\bibleverse{3} ¡Alabado sea Dios, el Padre de nuestro Señor Jesucristo!
Por su gran misericordia hemos nacido de nuevo y se nos ha dado una
esperanza viva\footnote{\textbf{1:3} O ``una esperanza que nos trae
  vida''.} por la resurrección de Jesucristo de entre los muertos.
\footnote{\textbf{1:3} Col 1,5} \bibleverse{4} Esta herencia es eterna,
y nunca se daña ni se desvanece, y está ahí segura para ustedes.
\footnote{\textbf{1:4} Col 1,12} \bibleverse{5} Por la fe de ustedes en
él, Dios los protegerá con su poder hasta que venga la salvación. La
salvación que está lista para ser revelada en el último día. \footnote{\textbf{1:5}
  Juan 10,28} \bibleverse{6} Así que estén felices por esto, aunque
estén tristes por un poco de tiempo, mientras soportan distintas
pruebas. \footnote{\textbf{1:6} 1Pe 5,10; 2Cor 4,17} \bibleverse{7}
Estas demuestran que su fe en Dios es genuina---aunque también puede ser
destruida---y esa fe es más valiosa que el oro. De este modo, su fe en
Dios será reconocida y ustedes recibirán alabanza, gloria y honra cuando
Cristo aparezca. \footnote{\textbf{1:7} Prov 17,3; Mal 3,3}
\bibleverse{8} Ustedes lo aman aunque nunca lo han visto. Aunque no
pueden verlo ahora, creen en él y están llenos de una felicidad
maravillosa e indescriptible. \footnote{\textbf{1:8} Juan 20,29; 2Cor
  5,7} \bibleverse{9} ¡Y por creer en él, su recompensa será la
salvación!

\hypertarget{la-salvaciuxf3n-prometida-hace-muchas-veces-por-los-profetas-ahora-finalmente-se-ha-realizado}{%
\subsection{La salvación prometida hace muchas veces por los profetas
ahora finalmente se ha
realizado}\label{la-salvaciuxf3n-prometida-hace-muchas-veces-por-los-profetas-ahora-finalmente-se-ha-realizado}}

\bibleverse{10} La salvación que buscaban y de la cual investigaban los
profetas cuando hablaban de la gracia que estaba preparada para ustedes.
\bibleverse{11} Trataton de descubrir cuándo y cómo esto sucedería,
porque el Espíritu de Cristo dentro de ellos hablaba de manera clara
sobre los sufrimientos de Cristo y la gloria que vendría. \footnote{\textbf{1:11}
  Is 53,-1; Sal 22,-1} \bibleverse{12} A ellos se les explicó que lo que
hacían no era para ellos mismos, sino para ustedes, pues aquello de lo
que ellos hablaron, ustedes lo han aprendido de aquellos que
compartieron la buena noticia con ustedes por el Espíritu Santo que el
cielo envió. ¡Hasta los ángeles quieren saber sobre esto! \footnote{\textbf{1:12}
  Efes 3,10}

\hypertarget{camine-en-santo-temor-de-dios-con-gozosa-confianza-en-la-salvaciuxf3n-que-se-logra-mediante-la-redenciuxf3n-y-con-la-esperanza-de-gloria}{%
\subsection{Camine en santo temor de Dios con gozosa confianza en la
salvación que se logra mediante la redención y con la esperanza de
gloria}\label{camine-en-santo-temor-de-dios-con-gozosa-confianza-en-la-salvaciuxf3n-que-se-logra-mediante-la-redenciuxf3n-y-con-la-esperanza-de-gloria}}

\bibleverse{13} Asegúrense de que sus mentes estén alerta. Tengan un
pensamiento claro. Fijen su esperanza exclusivamente en la gracia que
les será dada cuando Jesús sea revelado. \footnote{\textbf{1:13} Luc
  12,35-36} \bibleverse{14} Vivan como hijos obedientes. No se permitan
a ustedes mismos ser moldeados por sus antiguos deseos pecaminosos,
cuando no conocían algo mejor. \footnote{\textbf{1:14} Rom 12,2}
\bibleverse{15} Ahora necesitan ser santos en todo lo que hagan, así
como Aquél que los llamó es santo. \bibleverse{16} Tal como dice la
Escritura: ``Sean santos, porque yo soy santo''.\footnote{\textbf{1:16}
  Citando Levítico 11:44-45 o Levítico 19:2.}

\bibleverse{17} Puesto que ustedes le llaman Padre, y reconocen que él
juzga a todos de manera imparcial, basado en sus obras, tomen en serio
su vida aquí en la tierra, guardando reverencia hacia él. \footnote{\textbf{1:17}
  Rom 2,11; Fil 2,12} \bibleverse{18} Ya saben que no fueron liberados
por su vana forma de vivir que heredaron de sus antepasados, por cosas
que no tenían valor duradero, como el oro o la plata. \footnote{\textbf{1:18}
  1Cor 6,20; 1Cor 7,23; 1Pe 4,3} \bibleverse{19} Sino que fueron
liberados con la preciosa sangre de Cristo, que fue como un cordero sin
mancha ni defecto. \footnote{\textbf{1:19} Juan 1,29; Is 53,7; Heb 9,14}
\bibleverse{20} Él fue elegido antes de la creación del mundo, pero fue
revelado en estos últimos días\footnote{\textbf{1:20} O ``al final del
  tiempo''.} para beneficio de ustedes. \footnote{\textbf{1:20} Rom
  16,25-26} \bibleverse{21} Por medio de él, ustedes creen en Dios,
quien lo levantó de los muertos, y lo glorificó, para que ustedes puedan
confiar y tener esperanza en Dios.

\hypertarget{camine-dentro-de-la-iglesia-en-puro-amor-fraternal-y-digno-de-la-nueva-vida-que-se-le-ha-dado}{%
\subsection{Camine dentro de la iglesia en puro amor fraternal y digno
de la nueva vida que se le ha
dado}\label{camine-dentro-de-la-iglesia-en-puro-amor-fraternal-y-digno-de-la-nueva-vida-que-se-le-ha-dado}}

\bibleverse{22} Ahora que se han consagrado a seguir la verdad, ámense
unos a otros con sinceridad, como una verdadera familia.\footnote{\textbf{1:22}
  O ``con amor fraternal''.} \bibleverse{23} Ustedes han nacido de
nuevo, no son el producto de una ``semilla'' mortal,\footnote{\textbf{1:23}
  Aquí en énfasis está en el hecho de que distintas ``semillas''
  producen distintas clases de ``vida''.} sino inmortal, por la palabra
viva y eternal de Dios. \footnote{\textbf{1:23} Juan 1,13; Juan 3,5;
  Sant 1,18} \bibleverse{24} Porque: ``Todas las personas son como la
hierba, y su gloria es como flores del campo. La hierba se seca y las
flores se marchitan. \footnote{\textbf{1:24} Sant 1,10; Sant 1,1-11}

\bibleverse{25} Pero la palabra de Dios permanece para
siempre''.\footnote{\textbf{1:25} Citando Isaías 40:6-8.} Esta palabra
es la buena noticia de la que les hablaron antes.

\hypertarget{procedan-en-la-santificaciuxf3n-y-edifuxedquense-como-piedras-vivas-sobre-cristo-la-piedra-angular-para-un-pueblo-espiritual-de-sacerdotes}{%
\subsection{Procedan en la santificación y edifíquense como piedras
vivas sobre Cristo, la piedra angular, para un pueblo espiritual de
sacerdotes}\label{procedan-en-la-santificaciuxf3n-y-edifuxedquense-como-piedras-vivas-sobre-cristo-la-piedra-angular-para-un-pueblo-espiritual-de-sacerdotes}}

\hypertarget{section-1}{%
\section{2}\label{section-1}}

\bibleverse{1} Así que renuncien a las malas obras que hacen: la
deshonestidad, la hipocresía, el hablar mal de los demás. \footnote{\textbf{2:1}
  Sant 1,21} \bibleverse{2} Deben volverse como bebés recién nacidos que
solo quieren leche espiritual pura, para que puedan crecer en la
salvación \footnote{\textbf{2:2} Mat 18,3; Heb 5,12-13} \bibleverse{3}
ahora que han probado cuán bueno es el Señor. \footnote{\textbf{2:3} Sal
  34,9} \bibleverse{4} Cuando se acerquen a él, la piedra viva que la
gente rechazó como si fuera inútil, - pero que es elegida por Dios y
preciada para él -- \footnote{\textbf{2:4} Sal 118,22; Mat 21,42}
\bibleverse{5} ustedes también se convierten en piedras vivas,
edificadas en una casa espiritual. Ustedes son sacerdocio santo que
ofrece sacrificios espirituales y que Dios recibe con agrado por medio
de Jesucristo. \footnote{\textbf{2:5} Efes 2,21-22; Heb 3,6; Rom 12,1}
\bibleverse{6} Como dice la Escritura:\footnote{\textbf{2:6} Citando
  Isaías 28:16.} ``¡Miren! Yo establezco en Sión su piedra angular, una
piedra escogida de manera especial y valiosa. Todo el que crea en él no
será defraudado''.\footnote{\textbf{2:6} O ``avergonzado''.}

\bibleverse{7} Él es muy valioso para todos ustedes los que creen. Pero
para los que no creen, ``La piedra que los constructores rechazaron, y
que llegó a ser la piedra angular del fundamento''.\footnote{\textbf{2:7}
  Citando Salmos 118:22.}

\bibleverse{8} es ``La piedra que hace tropezar y los hace
caer''.\footnote{\textbf{2:8} Citando Isaías 8:14.} La gente tropieza
con este mensaje porque se niegan a aceptarlo, lo cual es completamente
predecible en cuanto a ellos.

\bibleverse{9} En cambio, ustedes son una familia elegida de manera
especial, un sacerdocio real, una nación santa, un pueblo que pertenece
a Dios. Por eso, pueden revelar las cosas maravillosas que él ha hecho,
al sacarlos de la oscuridad a su luz admirable. \footnote{\textbf{2:9}
  Éxod 19,6; Apoc 1,6; Efes 5,8} \bibleverse{10} En el pasado, ustedes
no eran nadie, pero ahora son el pueblo de Dios. En el pasado carecieron
de misericordia, pero ahora la han recibido. \footnote{\textbf{2:10} Rom
  9,24-26}

\hypertarget{invitaciuxf3n-general-a-un-caminar-puro-ante-los-gentiles}{%
\subsection{Invitación general a un caminar puro ante los
gentiles}\label{invitaciuxf3n-general-a-un-caminar-puro-ante-los-gentiles}}

\bibleverse{11} Amigos míos, les ruego como si fueran
extranjeros\footnote{\textbf{2:11} ``Peregrinos y extranjeros'' que no
  ven este mundo como su hogar.} en este mundo, que no se rindan ante
los deseos físicos que están en oposición a lo espiritual. \footnote{\textbf{2:11}
  Sal 39,13} \bibleverse{12} Asegúrense de actuar apropiadamente cuando
estén en compañía de quienes no son cristianos, para que incluso si los
acusaran de hacer lo malo, ellos puedan ver sus buenas obras y
glorifiquen a Dios cuando venga.\footnote{\textbf{2:12} Literalmente,
  ``día de visitación''.} \footnote{\textbf{2:12} Mat 5,16}

\hypertarget{exhortaciuxf3n-a-obedecer-a-las-autoridades-paganas}{%
\subsection{Exhortación a obedecer a las autoridades
paganas}\label{exhortaciuxf3n-a-obedecer-a-las-autoridades-paganas}}

\bibleverse{13} Obedezcan a la autoridad humana, por causa del Señor, ya
sea al rey, como autoridad suprema, \footnote{\textbf{2:13} Rom 13,1-7;
  Tit 3,1} \bibleverse{14} o a los gobernantes que Dios designa para
castigar a los que hacen el mal y dar reconocimiento a los que hacen el
bien. \bibleverse{15} Dios quiere que al hacer el bien ustedes hagan
callar las acusaciones ignorantes de los necios. \bibleverse{16} ¡Sí!
¡Ustedes son un pueblo libre! Así que no usen la libertad para disimular
la maldad, sino vivan como siervos de Dios. \footnote{\textbf{2:16} Gal
  5,13; 2Pe 2,19}

\bibleverse{17} Respeten a todos. Muestren su amor por la comunidad de
creyentes. Reverencien a Dios. Respeten al rey. \footnote{\textbf{2:17}
  Rom 12,10; Prov 24,21}

\hypertarget{admoniciones-a-los-esclavos-para-que-toleren-seguxfan-el-ejemplo-de-cristo}{%
\subsection{Admoniciones a los esclavos para que toleren según el
ejemplo de
Cristo}\label{admoniciones-a-los-esclavos-para-que-toleren-seguxfan-el-ejemplo-de-cristo}}

\bibleverse{18} Si eres un siervo, entonces mantente sujeto a tu amo, no
solo a los que son buenos y nobles, sino también a los que son duros.
\footnote{\textbf{2:18} Efes 6,5; Tit 2,9} \bibleverse{19} Porque en
esto consiste la gracia: soportar el dolor de la vida y el sufrimiento
injusto, pero manteniendo la mente enfocada en Dios. \bibleverse{20} Sin
embargo, no hay crédito si eres castigado por hacer el mal. Pero si
sufres por hacer lo recto, y lo soportas, entonces la gracia de Dios
está contigo. \bibleverse{21} En efecto, a esto han sido llamados,
porque Cristo sufrió por ustedes y les dio un ejemplo, para que
siguieran sus pasos. \footnote{\textbf{2:21} 1Pe 3,18; Mat 16,24}
\bibleverse{22} Él nunca pecó, ni mintió;\footnote{\textbf{2:22} Citando
  Isaías 53:9.} \footnote{\textbf{2:22} Is 53,9; Juan 8,46}
\bibleverse{23} y cuando fue maltratado, no replicó. Cuando sufrió, no
amenazó con venganza. Simplemente se puso en manos de Aquél que juzga
siempre con justicia. \bibleverse{24} Tomó las consecuencias de nuestros
pecados\footnote{\textbf{2:24} ``Las consecuencias de nuestros
  pecados'': son los resultados del pecado los que se manifiestan en la
  muerte de Jesús. Los pecados son intransferibles por naturaleza: son
  cometidos por el pecador y no se pueden pasar a nadie ni a nada más,
  ya que los pecados son las acciones específicas del pecador
  individual.} sobre sí mismo en su cuerpo en la cruz para que nosotros
pudiéramos morir al pecado y vivir en justicia. ``Por sus heridas, somos
sanados''.\footnote{\textbf{2:24} Citando Isaías 53:5, explicando que la
  salvación tiene que ver con la curación de nuestra enfermedad fatal
  del pecado, no con algún reajuste legal con Dios, o pago a él.}
\footnote{\textbf{2:24} Rom 6,8; Rom 6,11; Gal 3,13; 1Jn 3,5}
\bibleverse{25} En un tiempo ustedes eran como ovejas que habían perdido
su camino, pero ahora han regresado al pastor, al que cuida de
ustedes.\footnote{\textbf{2:25} Is 53,6; Juan 10,12}

\hypertarget{advertencias-para-los-cuxf3nyuges}{%
\subsection{Advertencias para los
cónyuges}\label{advertencias-para-los-cuxf3nyuges}}

\hypertarget{section-2}{%
\section{3}\label{section-2}}

\bibleverse{1} Esposas, acepten la autoridad de sus esposos de la misma
manera, para que si ellos se niegan a aceptar la palabra, puedan ser
ganados sin palabras, por la conducta de ustedes, \footnote{\textbf{3:1}
  Efes 5,22; 1Cor 7,16} \bibleverse{2} reconociendo que su conducta es
pura y reverente. \bibleverse{3} No se concentren en el atractivo
físico, ni en el corte de cabello, ni en las joyas de oro, o en las
ropas elegantes; \bibleverse{4} sino por el contrario, que el atractivo
sea interior, que sea el de un espíritu manso y pacífico que nace desde
el interior. Porque eso es lo que Dios estima. \bibleverse{5} Así es
como en el pasado, las mujeres santas que ponían su fe en Dios, se
embellecían, con la ternura que brindaban a sus esposos, \bibleverse{6}
como Sara, que obedecía a Abrahán, y lo llamaba ``señor''.\footnote{\textbf{3:6}
  Or ``maestro''. Hoy esta formalidad en el matrimonio es inusual.}
Ustedes son sus hijas si hacen lo recto y sin temor. \footnote{\textbf{3:6}
  Gén 18,12}

\bibleverse{7} Esposos, del mismo modo, sean considerados con sus
esposas en su vida diaria juntos. Aunque tu esposa no sea tan fuerte
como tú, debes honrarla, porque ella heredará en igual proporción junto
a ti el don de la vida de Dios. Asegúrense de hacer estas cosas para que
nada estorbe sus oraciones. \footnote{\textbf{3:7} Efes 5,25; 1Cor 7,5}

\hypertarget{advertencias-generales-para-los-miembros-de-la-iglesia}{%
\subsection{Advertencias generales para los miembros de la
iglesia}\label{advertencias-generales-para-los-miembros-de-la-iglesia}}

\bibleverse{8} Finalmente, tengan todos un mismo propósito. Sean amables
y amorosos unos con otros. Sean compasivos y humildes. \bibleverse{9} No
paguen mal por mal, ni reclamen cuando otros sean abusivos, sino
bendíganlos, porque a eso fueron llamados, para que puedan recibir
bendiciones ustedes mismos también. \footnote{\textbf{3:9} 1Tes 5,15}
\bibleverse{10} Recuerden: ``Los que quieren amar sus vidas y ver días
felices, deben abstenerse de hablar el mal, y no decir mentiras.
\footnote{\textbf{3:10} Sant 1,26} \bibleverse{11} Aléjense del mal y
hagan el bien; ¡busquen la paz y síganla! \bibleverse{12} Porque Dios
está atento a los justos y escucha sus oraciones, pero aborrece a los
que hacen el mal''.\footnote{\textbf{3:12} Citando Salmos 34:12-16.}

\hypertarget{en-el-sufrimiento-eres-testigo-de-los-que-te-rodean}{%
\subsection{En el sufrimiento eres testigo de los que te
rodean}\label{en-el-sufrimiento-eres-testigo-de-los-que-te-rodean}}

\bibleverse{13} ¿Quién les hará daño si la intención de ustedes es hacer
el bien? \bibleverse{14} Porque incluso si experimentan sufrimiento por
hacer lo recto, ustedes están mucho mejor. No teman las amenazas de la
gente, no se preocupen por esas cosas, \footnote{\textbf{3:14} 1Pe 2,20;
  Mat 5,10} \bibleverse{15} solo tengan en su mente a Cristo como Señor.
Estén siempre listos para dar explicaciones a todo el que pregunte la
razón de su esperanza. Y háganlo con mansedumbre y respeto.
\bibleverse{16} Asegúrense de tener una conciencia limpia, para que si
alguno los acusa, sean avergonzados por hablar mal sobre la buena manera
de vivir de ustedes, en Cristo. \bibleverse{17} Sin duda alguna, es
mejor sufrir haciendo el bien, (si eso es lo que Dios quiere), que
sufrir haciendo el mal.

\hypertarget{las-benditas-consecuencias-del-sufrimiento-involuntario-de-cristo}{%
\subsection{Las benditas consecuencias del sufrimiento involuntario de
Cristo}\label{las-benditas-consecuencias-del-sufrimiento-involuntario-de-cristo}}

\bibleverse{18} Y Jesús murió por culpa de los pecados, una vez y para
siempre, el Único que es completamente verdadero y justo, por aquellos
que somos malos,\footnote{\textbf{3:18} Literalmente, ``el justo por los
  injustos''. En este versículo se aclaran los resultados inevitables
  del pecado a través de la muerte de Jesús. Experimentó las
  consecuencias del pecado de la manera más dramática y contundente
  posible, y también demostró que no es Dios quien mata, sino que el
  pecado mismo trae su inevitable resultado fatal (ver Romanos 6:23).}
para poder llevarnos a Dios. Fue llevado a muerte en su cuerpo, pero
vino a la vida en el espíritu. \bibleverse{19} Él fue a hablar a los que
estaban ``presos''\footnote{\textbf{3:19} O ``almas prisioneras''. Ha
  existido mucho debate sobre esta frase. Debemos notar que la misma
  palabra que se usa para ``almas'' aquí, se usa también en el versículo
  10. Algunos entienden que ``prisioneras'' se refiere a las personas
  que vivían en la época del diluvio y que estaban ``cautivas'' por su
  pecaminosidad (ver Génesis 6:5).} \footnote{\textbf{3:19} 1Pe 4,6}
\bibleverse{20} y que se negaban a creer, siendo que Dios con paciencia
esperó, durante los días de Noé, cuando estaban construyendo el arca.
Apenas unos cuantos---de hecho, ocho personas---se salvaron ``por el
agua''. \footnote{\textbf{3:20} Gén 7,7; Gén 7,17; 2Pe 2,5}
\bibleverse{21} Esta agua simboliza el bautismo que los salva ahora, no
limpiando la suciedad de sus cuerpos, sino como una respuesta positiva a
Dios, que surge de una conciencia limpia. La resurrección de Jesús es la
que posibilita la salvación. \footnote{\textbf{3:21} Efes 5,26; Heb
  10,22} \bibleverse{22} Después de haber ascendido al cielo, él está en
pie a la diestra de Dios, con ángeles, autoridades, y poderes puestos
bajo su control.\footnote{\textbf{3:22} Efes 1,20-21}

\hypertarget{la-voluntad-de-sufrir-resiste-la-sensaciuxf3n-de-pecado-amortigua-la-lujuria-y-ayuda-a-las-personas-a-caminar-con-devociuxf3n}{%
\subsection{La voluntad de sufrir resiste la sensación de pecado,
amortigua la lujuria y ayuda a las personas a caminar con
devoción}\label{la-voluntad-de-sufrir-resiste-la-sensaciuxf3n-de-pecado-amortigua-la-lujuria-y-ayuda-a-las-personas-a-caminar-con-devociuxf3n}}

\hypertarget{section-3}{%
\section{4}\label{section-3}}

\bibleverse{1} Y como Cristo padeció sufrimiento físico, ustedes deben
prepararse con la misma actitud que él tuvo, porque los que sufren
físicamente, han abandonado el pecado.\footnote{\textbf{4:1} Este es un
  versículo difícil, pues sin duda el sufrimiento no implica que no haya
  pecado. Queda implícito que así como Jesús sufrió injustamente, cuando
  los cristianos sufren, participan de la experiencia de Cristo.}
\bibleverse{2} Ustedes no vivirán el resto de sus vidas siguiendo los
deseos humanos, sino haciendo la voluntad de Dios. \bibleverse{3} En el
pasado vivieron mucho tiempo siguiendo los caminos del mundo:
inmoralidad, complacencia sexual, orgías, fiestas, borracheras, e
idolatría abominable. \footnote{\textbf{4:3} Efes 2,2-3; Tit 3,3}
\bibleverse{4} La gente piensa que es extraño que ustedes ya no
participen con ellos de este estilo de vida lleno de excesos, y por eso
los maldicen. Pero ellos tendrán que dar cuentas de lo que han hecho
contra Aquél que está listo para juzgar a los vivos y a los muertos.
\bibleverse{5} Por eso, la buena noticia fue compartida con los que ya
murieron, \bibleverse{6} para que aunque hayan sido juzgados
correctamente según la justicia humana y pecaminosa, ellos puedan vivir
en el espíritu según la justicia de Dios. \footnote{\textbf{4:6} 1Pe
  3,19}

\hypertarget{advertencia-a-la-preservaciuxf3n-de-la-vida-comunitaria-cristiana-con-miras-al-fin-del-mundo-pruxf3ximo}{%
\subsection{Advertencia a la preservación de la vida comunitaria
cristiana con miras al fin del mundo
próximo}\label{advertencia-a-la-preservaciuxf3n-de-la-vida-comunitaria-cristiana-con-miras-al-fin-del-mundo-pruxf3ximo}}

\bibleverse{7} ¡Todo llegará a su fin! Así que piensen con claridad y
manténganse vigilantes cuando oren. \footnote{\textbf{4:7} 1Cor 10,11;
  1Jn 2,18} \bibleverse{8} Por encima de todo, ámense unos a otros con
amor profundo, porque el amor cubre muchas de las faltas que la gente
comete. \footnote{\textbf{4:8} Sant 5,20} \bibleverse{9} Muestren
hospitalidad unos con otros y no se quejen. \footnote{\textbf{4:9} Heb
  13,2} \bibleverse{10} Cualquiera sea el don que hayan recibido,
compártanlo con otros entre ustedes, como un pueblo que demuestra
sabiamente la gracia de Dios, en todas sus formas. \bibleverse{11} Todo
el que hable, hágalo como si Dios hablara a través de él. Todo aquél que
quiera ayudar a otros, hágalo por medio de la fuerza que Dios le da,
para que en todo Dios sea glorificado por medio de Jesucristo. Que la
gloria y el poder sean suyos por siempre y para siempre. Amén.
\footnote{\textbf{4:11} Rom 12,7}

\hypertarget{exhortaciuxf3n-a-probar-el-verdadero-espuxedritu-cristiano-en-el-fuego-de-la-purificaciuxf3n-del-sufrimiento-en-vista-de-la-gloria-que-se-alcanzaruxe1}{%
\subsection{Exhortación a probar el verdadero espíritu cristiano en el
fuego de la purificación del sufrimiento en vista de la gloria que se
alcanzará}\label{exhortaciuxf3n-a-probar-el-verdadero-espuxedritu-cristiano-en-el-fuego-de-la-purificaciuxf3n-del-sufrimiento-en-vista-de-la-gloria-que-se-alcanzaruxe1}}

\bibleverse{12} Amigos míos, no se sorprendan ante las ``pruebas de
fuego''\footnote{\textbf{4:12} Literalmente ``una prueba de fuego para
  probarlos''.} que están experimentando, como si estas fueran algo
inesperado. \footnote{\textbf{4:12} 1Pe 1,6-7} \bibleverse{13} Estén
contentos en la medida que participan del sufrimiento de Cristo, porque
cuando aparezca en su gloria, ustedes serán muy felices. \footnote{\textbf{4:13}
  Hech 5,41; Rom 8,17; Sant 1,2} \bibleverse{14} Si alguien los maldice
en el nombre de Cristo, en realidad son bendecidos, porque el espíritu
glorioso de Dios reposa sobre ustedes. \footnote{\textbf{4:14} Mat 5,11;
  Efes 1,13} \bibleverse{15} Y si sufren, no será como asesinos, como
ladrones, como criminales o como chismosos, \bibleverse{16} sino que si
es como un cristiano, entonces no tendrán de qué avergonzarse. Más bien,
oren para que sean llamados cristianos. \footnote{\textbf{4:16} Fil 1,20}
\bibleverse{17} Porque el tiempo del juicio ha llegado, y comienza por
la casa de Dios. Y si comienza por nosotros, entonces ¿cuál será el fin
de los que rechazan la buena noticia de Dios? \footnote{\textbf{4:17}
  Jer 25,29; Ezeq 9,6} \bibleverse{18} ``Si ya es difícil salvarse para
los que son justos, ¿qué será de los pecadores que aborrecen a
Dios?''\footnote{\textbf{4:18} Citando Proverbios 11:31.} \footnote{\textbf{4:18}
  Prov 11,31} \bibleverse{19} De modo que los que sufren conforme a la
voluntad de Dios, del Creador fiel, deben asegurarse de que están
haciendo el bien.\footnote{\textbf{4:19} Sal 31,6}

\hypertarget{advertencia-a-los-ancianos-y-a-los-muxe1s-juxf3venes-de-la-iglesia}{%
\subsection{Advertencia a los ancianos y a los más jóvenes de la
iglesia}\label{advertencia-a-los-ancianos-y-a-los-muxe1s-juxf3venes-de-la-iglesia}}

\hypertarget{section-4}{%
\section{5}\label{section-4}}

\bibleverse{1} Quiero animar a los ancianos que están entre ustedes.
Pues yo también soy un anciano, un testigo de los sufrimientos de
Cristo, y participaré de la gloria que está por venir. \footnote{\textbf{5:1}
  Rom 8,17; 2Jn 1,-1} \bibleverse{2} Cuiden del rebaño que se les ha
encomendado, no porque estén obligados a vigilarlos, sino con agrado,
como Dios quiere que sea. Háganlo de buena gana, sin buscar beneficio de
ello. \footnote{\textbf{5:2} Juan 21,16; Hech 20,28; 1Tim 3,2-7}
\bibleverse{3} No sean arrogantes, enseñoreándose de aquellos que están
bajo su cuidado, sino sean un ejemplo para el rebaño. \footnote{\textbf{5:3}
  Ezeq 34,2-4; 2Cor 1,24; Tit 2,7} \bibleverse{4} Cuando aparezca el
Pastor supremo, ustedes recibirán una corona de gloria, que nunca se
dañará. \footnote{\textbf{5:4} 1Cor 9,25; 2Tim 4,8; Heb 13,20}

\bibleverse{5} Jóvenes, hagan lo que los ancianos les dicen. Sin duda
deberían todos servirse unos a otros con humildad, porque ``Dios
aborrece a los orgullosos, pero obra en favor de los
humildes''.\footnote{\textbf{5:5} Citando Proverbios 3:34.} \footnote{\textbf{5:5}
  Prov 3,34; Mat 23,12; Efes 5,21; Sant 4,6} \bibleverse{6} Humíllense
ante la mano poderosa de Dios, para que los exalte en su debido tiempo.
\footnote{\textbf{5:6} Job 22,29; Sant 4,10} \bibleverse{7} Entreguen
todas sus preocupaciones a él, porque él tiene cuidado de ustedes.
\footnote{\textbf{5:7} Sal 55,23; Mat 6,25; Fil 4,6}

\hypertarget{estuxe9-atento-a-las-tentaciones-del-diablo}{%
\subsection{Esté atento a las tentaciones del
diablo}\label{estuxe9-atento-a-las-tentaciones-del-diablo}}

\bibleverse{8} Sean responsables, y estén vigilantes. El diablo, su
enemigo, anda por ahí, como león rugiente, buscando a quién devorar.
\footnote{\textbf{5:8} 2Cor 2,11; 1Tes 5,6; Luc 22,31} \bibleverse{9}
Manténganse firmes contra él, confiando en Dios. Recuerden que sus
hermanos creyentes en todo el mundo están viviendo dificultades
similares. \bibleverse{10} Pero después de que hayan sufrido un poco, el
Dios de toda gracia, que los llamó a su gloria eterna en Cristo, él
mismo los restaurará, los sostendrá, los fortalecerá y les dará un
fundamento sólido. \footnote{\textbf{5:10} 1Pe 1,6} \bibleverse{11} El
poder sea suyo, por siempre y para siempre. Amén.

\hypertarget{fin-de-la-carta-saludos-y-bendiciones}{%
\subsection{Fin de la carta; Saludos y
bendiciones}\label{fin-de-la-carta-saludos-y-bendiciones}}

\bibleverse{12} Esta carta se las envío con ayuda de Silvano, a quien
considero como un hermano fiel. En estas pocas palabras que les he
escrito, quiero animarlos y testificar que esta es la verdadera gracia
de Dios. ¡Manténganse firmes en ella! \bibleverse{13} Los creyentes de
aquí, de ``Babilonia'',\footnote{\textbf{5:13} Literalmente, ``los que
  están en Babilonia''. En el Nuevo Testamento Babilonia es un símbolo
  de Roma.} escogidos junto a ustedes, les envían su saludo, así como
Marcos, mi hijo. \bibleverse{14} Salúdense unos a otros con un beso de
amor. Paz a todos ustedes que están en Cristo.
