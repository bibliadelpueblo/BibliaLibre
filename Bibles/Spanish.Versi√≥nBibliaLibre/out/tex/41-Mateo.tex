\hypertarget{uxe1rbol-genealuxf3gico-de-jesuxfas-como-descendiente-de-abraham-y-david}{%
\subsection{Árbol genealógico de Jesús como descendiente de Abraham y
David}\label{uxe1rbol-genealuxf3gico-de-jesuxfas-como-descendiente-de-abraham-y-david}}

\hypertarget{section}{%
\section{1}\label{section}}

\bibleverse{1} Este libro es el registro de Jesús el Mesías,\footnote{\textbf{1:1}
  O ``Cristo''. Cristo es el término griego para decir ``Mesías'' en
  hebreo.} Hijo de David, Hijo de Abraham, comenzando con el linaje de
su familia:

\bibleverse{2} Abraham fue el padre\footnote{\textbf{1:2} O
  ``engendró''.} de Isaac; e Isaac el padre de Jacob; y Jacob el padre
de Judá y de sus hermanos; \footnote{\textbf{1:2} Gén 21,3; Gén 21,12;
  Gén 25,26; Gén 29,35; Gén 49,10} \bibleverse{3} y Judá fue el Padre de
Fares y Zarah (su madre fue Tamar); y Fares fue el padre de Esrom; y
Esrom el padre de Ram; \footnote{\textbf{1:3} Gén 38,29-30; Rut 4,18-22}
\bibleverse{4} y Ram fue el padre de Aminadab; y Aminadab el padre de
Nasón; y Nasón el padre de Salmón; \bibleverse{5} y Salmón el padre de
Booz (su madre fue Rahab); y Booz el padre de Obed (su madre fue Rut); y
Obed el padre de Isaí; \footnote{\textbf{1:5} Jos 2,1; Rut 4,13-17}
\bibleverse{6} e Isaí el padre del Rey David. David fue el padre de
Salomón (su madre había sido la esposa de Urías); \footnote{\textbf{1:6}
  2Sam 12,24} \bibleverse{7} y Salomón el padre de Roboam; y Roboam el
padre de Abías; y Abías el padre de Asa; \footnote{\textbf{1:7} 1Cró
  3,10-16} \bibleverse{8} y Asa fue el padre de Josafat; y Josafat el
padre de Joram; y Joram el padre de Uzías; \bibleverse{9} y Uzías fue el
padre de Jotam; y Jotam el padre de Acaz; y Acaz el padre de Ezequías;
\bibleverse{10} y Ezequías el padre de Manasés; y Manasés el padre de
Amón; y Amón el padre de Josías; \bibleverse{11} y Josías el padre de
Joaquín y de sus hermanos, durante el tiempo del exilio a Babilonia.

\bibleverse{12} Después del exilio a Babilonia, Joacím fue el padre de
Salatiel; y Salatiel el padre de Zorobabel; \footnote{\textbf{1:12} 1Cró
  3,17; Esd 3,2} \bibleverse{13} y Zorobabel el padre de Abiud; y Abiud
fue el padre de Eliaquim; y Eliaquim el padre de Azor; \bibleverse{14} y
Azor el padre de Sadoc; y Sadoc el padre de Aquim; y Aquim el padre de
Eliud; \bibleverse{15} y Eliud fue el padre de Eleazar; y Eleazar el
padre de Matán; y Matán el padre de Jacob; \bibleverse{16} y Jacob fue
el padre de José, quien fue el esposo de María, de quien nació Jesus, el
que es llamado el Mesías.

\bibleverse{17} Así que todas las generaciones desde Abraham hasta David
suman catorce; desde David hasta el exilio de babilonia, catorce; y
desde el exilio de Babilonia hasta el Mesías, catorce.

\hypertarget{nacimiento-y-nombre-de-jesuxfas}{%
\subsection{Nacimiento y nombre de
Jesús}\label{nacimiento-y-nombre-de-jesuxfas}}

\bibleverse{18} Así fue como ocurrió el nacimiento de Jesús el Mesías:
su madre, María, estaba comprometida con José, pero antes de que
durmieran juntos ella quedó embarazada por obra del Espíritu Santo.
\footnote{\textbf{1:18} Luc 1,35} \bibleverse{19} José, su prometido,
era un buen hombre y no quería avergonzarla públicamente, de modo que
decidió romper el compromiso de manera discreta. \bibleverse{20}
Mientras José pensaba en todo esto, un ángel del Señor se le apareció en
un sueño y le dijo: ``José, hijo de David, no temas casarte con María
porque ella está embarazada por obra del Espíritu Santo. \bibleverse{21}
Ella tendrá un hijo y tú le llamarás Jesús, porque él salvará a su
pueblo de sus pecados''.

\bibleverse{22} Y todo esto ocurrió para cumplir lo que el Señor dijo a
través del profeta: \bibleverse{23} ``Una virgen quedará embarazada y
tendrá un hijo. Y le llamarán Emanuel'', que significa ``Dios con
nosotros''.\footnote{\textbf{1:23} Ver Isaías 7:14.}

\bibleverse{24} José se despertó e hizo lo que el ángel del Señor le
dijo que hiciera. \bibleverse{25} José se casó con María, pero no durmió
con ella hasta después que tuvo un hijo, a quien llamó Jesús.\footnote{\textbf{1:25}
  Luc 2,7}

\hypertarget{los-magos-del-oriente-vienen-al-niuxf1o-jesuxfas-y-le-rinden-homenaje.}{%
\subsection{Los magos del oriente vienen al niño Jesús y le rinden
homenaje.}\label{los-magos-del-oriente-vienen-al-niuxf1o-jesuxfas-y-le-rinden-homenaje.}}

\hypertarget{section-1}{%
\section{2}\label{section-1}}

\bibleverse{1} Después de que Jesús nació en Belén de Judea, durante el
reinado del rey Herodes, unos hombres sabios\footnote{\textbf{2:1} O
  ``Magos''. Se creía que estos eran sacerdotes gobernantes de Persia,
  quienes estudiaban las estrellas.} vinieron desde el oriente hasta
Jerusalén. \footnote{\textbf{2:1} Luc 2,1-7} \bibleverse{2} ``¿Dónde
está el Rey de los judíos que ha nacido?'' preguntaron. ``Vimos su
estrella en el oriente y hemos venido a adorarlo''. \footnote{\textbf{2:2}
  Núm 24,17} \bibleverse{3} Cuando el rey Herodes escuchó esto, se
preocupó mucho, y toda Jerusalén con él. \bibleverse{4} Entonces Herodes
llamó a todos los jefes de los sacerdotes y a los maestros religiosos
del pueblo, y les preguntó dónde se suponía que nacería el Mesías.
\bibleverse{5} ``En Belén de Judea'', le dijeron ellos, ``pues eso fue
lo que escribió el profeta: \bibleverse{6} `y tu, Belén, en la tierra de
Judea, no eres la menor entre las ciudades reinantes de
Judea,\footnote{\textbf{2:6} ``Ciudades'' está implícito.} porque de ti
saldrá un gobernante que será el pastor de mi pueblo
Israel'\,''.\footnote{\textbf{2:6} Haciendo referencia a Miqueas 5:2 y2
  Samuel 5:2.}

\bibleverse{7} Entonces Herodes llamó a los sabios y se reunió con ellos
en privado, y logró saber por medio de ellos el momento exacto en que
había aparecido la estrella. \bibleverse{8} Los envió a Belén,
diciéndoles: ``cuando lleguen allí, busquen al niño, y cuando lo
encuentren, háganmelo saber para yo ir a adorarlo también''.

\bibleverse{9} Después que los sabios escucharon lo que el rey iba a
decirles, siguieron su camino, y la estrella que habían visto en el
oriente los guió hasta que se detuvo justo sobre el lugar donde estaba
el niño. \bibleverse{10} Cuando los sabios vieron la
estrella,\footnote{\textbf{2:10} Claramente indica que fue cuando vieron
  que la estrella se detuvo, puesto que ellos ya habían visto la
  estrella y la habían seguido durante todo el camino desde su hogar en
  el oriente.} no pudieron contener la felicidad. \bibleverse{11}
Entraron a la casa y vieron al niño con María, su madre. Se arrodillaron
y lo adoraron. Entonces abrieron sus bolsas de tesoros y le obsequiaron
regalos de oro, incienso y mirra. \footnote{\textbf{2:11} Sal 72,10; Sal
  72,15; Is 60,6} \bibleverse{12} Advertidos por un sueño de no regresar
ante Herodes, se marcharon a su país tomando otro camino.

\hypertarget{la-huida-de-josuxe9-a-egipto}{%
\subsection{La huida de José a
Egipto}\label{la-huida-de-josuxe9-a-egipto}}

\bibleverse{13} Después que se fueron los sabios, un ángel del Señor se
le apareció a José en un sueño, y le dijo: ``Levántate, toma al niño y a
su madre, y huyan a Egipto. Quédense allí hasta que yo se los diga,
porque Herodes va a buscar al niño para matarlo''.

\bibleverse{14} Entonces José se levantó y tomó al niño y a su madre, y
partió hacia Egipto en medio de la noche. \bibleverse{15} Permanecieron
allí hasta que Herodes murió. Esto cumplió lo que el Señor dijo a través
del profeta: ``De Egipto llamé a mi hijo''.\footnote{\textbf{2:15}
  Citando Oseas 11:1.}

\hypertarget{el-asesinato-del-niuxf1o-de-herodes-en-beluxe9n}{%
\subsection{El asesinato del niño de Herodes en
Belén}\label{el-asesinato-del-niuxf1o-de-herodes-en-beluxe9n}}

\bibleverse{16} Cuando Herodes se dio cuenta que había sido engañado por
los sabios, se enojó mucho. Entonces envió hombres para que matasen a
todos los niños de Belén y de las regiones cercanas que tuvieran menos
de dos años de edad. Esto se basaba en el marco de tiempo que escuchó de
los sabios.\footnote{\textbf{2:16} En otras palabras, hacía dos años que
  la estrella ya se les había aparecido previamente a los sabios.}
\bibleverse{17} Así se cumplió la profecía del profeta Jeremías:
\bibleverse{18} ``En Ramá se oyó una voz, llanto y gran lamento. Raquel
llora por sus hijos, se niega a que la consuelen, porque están
muertos''.\footnote{\textbf{2:18} Citando Jeremías 31:15.}

\hypertarget{el-regreso-de-josuxe9-de-egipto-y-su-asentamiento-en-nazaret}{%
\subsection{El regreso de José de Egipto y su asentamiento en
Nazaret}\label{el-regreso-de-josuxe9-de-egipto-y-su-asentamiento-en-nazaret}}

\bibleverse{19} Después que Herodes murió, el ángel del Señor se le
apareció en un sueño a José en Egipto, y le dijo: \bibleverse{20}
``¡Levántate! Toma al niño y a su madre y regresa a la tierra de Israel,
porque los que trataban de matar al niño están muertos''. \footnote{\textbf{2:20}
  Éxod 4,19}

\bibleverse{21} Entonces José se levantó y tomó al niño y a su madre, y
regresó a la tierra de Israel. \bibleverse{22} Pero José tenía miedo de
ir allá después que supo que Arquelao había sucedido a su padre, el rey
Herodes, como rey de Judá. Habiendo sido advertido por medio de un
sueño, José se fue a Galilea, \bibleverse{23} y se estableció en
Nazaret. Esto cumplió lo que los profetas habían dicho: ``Él será
llamado Nazareno''.\footnote{\textbf{2:23} Refiriéndose a Jesús.}

\hypertarget{apariciuxf3n-y-sermuxf3n-penitencial-de-juan-el-bautista}{%
\subsection{Aparición y sermón penitencial de Juan el
Bautista}\label{apariciuxf3n-y-sermuxf3n-penitencial-de-juan-el-bautista}}

\hypertarget{section-2}{%
\section{3}\label{section-2}}

\bibleverse{1} Tiempo después, apareció en escena Juan el Bautista,
predicando en el desierto de Judea: \footnote{\textbf{3:1} Luc 1,13}
\bibleverse{2} ``Arrepiéntanse, porque el reino de los cielos está
cerca''. \footnote{\textbf{3:2} Mat 4,17; Rom 12,2} \bibleverse{3} Él
era de quien hablaba el profeta Isaías cuando dijo: ``Se oye una voz que
clama en el desierto: `preparen el camino del Señor. Enderecen la senda
para él'\,''.\footnote{\textbf{3:3} Ver Isaías 40:3.} \footnote{\textbf{3:3}
  Juan 1,23}

\bibleverse{4} Juan tenía ropas hechas con pelo de camello, con un
cinturón de cuero puesto en su cintura. Su alimento era
langostas\footnote{\textbf{3:4} Probablemente, algarrobas.} y miel
silvestre. \footnote{\textbf{3:4} 2Re 1,8} \bibleverse{5} La gente venía
a él desde Jerusalén, de toda Judea y de toda la región del Jordán,
\bibleverse{6} y eran bautizados en el río Jordán, reconociendo
públicamente sus pecados.

\bibleverse{7} Pero cuando Juan vio que muchos de los fariseos y
saduceos venían a ser bautizados, les dijo: ``¡Camada de víboras! ¿Quién
les advirtió que huyeran del juicio que vendrá?\footnote{\textbf{3:7}
  Literalmente, ``ira''.} \footnote{\textbf{3:7} Mat 23,33}
\bibleverse{8} Muestren a través de sus actos que están verdaderamente
arrepentidos,\footnote{\textbf{3:8} Literalmente, ``Produzcan fruto
  equivalente al arrepentimiento''.} \bibleverse{9} y no se jacten de
decirse a ustedes mismos: `Abrahán es nuestro padre'. Les digo que Dios
podría convertir estas piedras en hijos de Abrahán. \bibleverse{10} De
hecho, el hacha está lista para derribar los árboles. Todo árbol que no
produce buen fruto, será derribado y lanzado al fuego. \footnote{\textbf{3:10}
  Luc 13,6-9}

\bibleverse{11} ``Sí, yo los bautizo en agua para mostrar
arrepentimiento, pero después de mi viene uno que es más grande que yo.
Yo no soy siquiera digno de quitar sus sandalias. Él los bautizará con
el Espíritu Santo y con fuego. \footnote{\textbf{3:11} Juan 1,26-27;
  Juan 1,33; Hech 1,5; Hech 2,3-4} \bibleverse{12} Él tiene el
aventador\footnote{\textbf{3:12} Usada después de la cosecha para
  separar el trigo de la paja.} listo en su mano. Limpiará la era y
almacenará el trigo en el granero, pero quemará la paja en el fuego que
no se puede apagar''. \footnote{\textbf{3:12} Mat 13,30}

\hypertarget{el-bautismo-y-la-consagraciuxf3n-del-mesuxedas-de-jesuxfas}{%
\subsection{El bautismo y la consagración del Mesías de
Jesús}\label{el-bautismo-y-la-consagraciuxf3n-del-mesuxedas-de-jesuxfas}}

\bibleverse{13} Luego Jesús vino desde Galilea hasta el Río Jordán para
ser bautizado por Juan. \bibleverse{14} Pero Juan trató de hacerlo
cambiar de opinión, diciendo, ``Yo soy quien necesito ser bautizado por
ti, ¿y tu vienes a mí para que yo te bautice?''

\bibleverse{15} ``Por favor, hazlo, porque es bueno que hagamos lo que
Dios dice que es correcto'', le dijo Jesús. Entonces Juan estuvo de
acuerdo en hacerlo.

\bibleverse{16} Justo después de haber sido bautizado, Jesús salió del
agua. Los cielos se abrieron y él vio al Espíritu de Dios como una
paloma que descendía, posándose sobre él. \footnote{\textbf{3:16} Is
  11,2} \bibleverse{17} Entonces una voz desde el cielo dijo: ``este es
mi hijo a quien amo, el cual me complace''.\footnote{\textbf{3:17} Mat
  17,5; Is 42,1}

\hypertarget{la-tentaciuxf3n-de-jesuxfas-como-prueba-de-mesuxedas}{%
\subsection{La tentación de Jesús como prueba de
Mesías}\label{la-tentaciuxf3n-de-jesuxfas-como-prueba-de-mesuxedas}}

\hypertarget{section-3}{%
\section{4}\label{section-3}}

\bibleverse{1} Entonces Jesús fue guiado por el Espíritu hasta el
desierto para ser tentado por el diablo. \footnote{\textbf{4:1} Heb 4,15}
\bibleverse{2} Después de haber ayunado por cuarenta días y cuarenta
noches, tenía hambre. \footnote{\textbf{4:2} Éxod 34,28; 1Re 19,8}
\bibleverse{3} El tentador vino y le dijo: ``Si realmente eres el hijo
de Dios, ordena a estas piedras que se conviertan en pan''. \footnote{\textbf{4:3}
  Gén 3,1-7}

\bibleverse{4} Jesús respondió: ``Como dicen las Escrituras, `los seres
humanos no viven solo de comer pan, sino de cada palabra que sale de la
boca de Dios'\,''.\footnote{\textbf{4:4} Citando Deuteronomio 8:3.}

\bibleverse{5} Entonces el diablo lo llevó hasta la ciudad
santa\footnote{\textbf{4:5} Refiriéndose a Jerusalén.} y lo puso en la
parte más alta del Templo. \bibleverse{6} ``Si realmente eres el hijo de
Dios, tírate'', le dijo a Jesús. ``Tal como dicen las Escrituras: `Él
mandará a sus ángeles para que te guarden del peligro. Te atraparán para
que no caigas al tropezarte con una roca'\,''.\footnote{\textbf{4:6}
  Citando Salmos 91:11-12.}

\bibleverse{7} Jesús respondió: ``Tal como dicen también las Escrituras,
`No tentarás al Señor tu Dios'\,''.\footnote{\textbf{4:7} Citando
  Deuteronomio 6:16.}

\bibleverse{8} Entonces el diablo llevó a Jesús a una montaña muy alta y
le mostró todos los reinos del mundo en toda su gloria. \bibleverse{9}
Le dijo a Jesús: ``Te daré todos estos reinos si te arrodillas y me
adoras''.

\bibleverse{10} ``¡Vete de aquí Satanás!'' dijo Jesús. ``Tal como dicen
las Escrituras: `Adorarás al Señor tu Dios y le servirás solo a
Él'\,''.\footnote{\textbf{4:10} Citando Deuteronomio 6:13.}

\bibleverse{11} Entonces el diablo lo dejó, y los ángeles vinieron a
cuidar de él. \footnote{\textbf{4:11} Juan 1,51; Heb 1,6; Heb 1,14}

\hypertarget{jesuxfas-asume-enseuxf1ar-en-capernaum}{%
\subsection{Jesús asume enseñar en
Capernaum}\label{jesuxfas-asume-enseuxf1ar-en-capernaum}}

\bibleverse{12} Cuando Jesús escuchó que Juan había sido arrestado,
regresó a Galilea. \footnote{\textbf{4:12} Mat 14,3} \bibleverse{13}
Después de salir de Nazaret, se quedó en Capernaúm, a orillas del mar,
en las regiones de Zabulón y Neftalí. \bibleverse{14} Esto cumplió lo
que el profeta Isaías dijo: \bibleverse{15} ``En la tierra de Zabulón y
en la tierra de Neftalí, camino al mar, más allá del Jordán, en Galilea,
donde viven los gentiles: \bibleverse{16} La gente que vive en la
oscuridad vio una gran luz; la luz de la mañana ha brillado sobre
aquellos que viven en la tierra de la oscuridad y la
muerte''.\footnote{\textbf{4:16} Citando Isaías 9:1-2.} \footnote{\textbf{4:16}
  Juan 8,12}

\bibleverse{17} Desde ese momento, Jesús comenzó a declarar su mensaje,
diciendo: ``Arrepiéntanse, porque el reino de los cielos está cerca''.
\footnote{\textbf{4:17} Mat 3,2}

\hypertarget{jesuxfas-llama-a-los-dos-primeros-pares-de-discuxedpulos}{%
\subsection{Jesús llama a los dos primeros pares de
discípulos}\label{jesuxfas-llama-a-los-dos-primeros-pares-de-discuxedpulos}}

\bibleverse{18} Mientras caminaba por el mar de Galilea, Jesús vio a dos
hermanos: Simón, también llamado Pedro, y su hermano Andrés, que estaban
lanzando una red en el mar. Ellos vivían de la pesca. \bibleverse{19}
``Vengan y síganme, y yo les enseñaré cómo pescar personas'', les dijo.
\footnote{\textbf{4:19} Mat 28,19-20}

\bibleverse{20} Ellos dejaron sus redes de inmediato y lo siguieron.
\footnote{\textbf{4:20} Mat 19,27} \bibleverse{21} De camino, vio
nuevamente a otros dos hermanos: Santiago y Juan. Ellos estaban en un
bote con su padre Zebedeo, reparando sus redes. Él los llamó para que lo
siguieran.\footnote{\textbf{4:21} ``Para que lo siguieran'', está
  implícito.} \bibleverse{22} Ellos inmediatamente dejaron el bote y a
su padre, y lo siguieron.

\hypertarget{descripciuxf3n-de-los-efectos-de-enseuxf1anza-y-curaciuxf3n-de-jesuxfas-y-su-uxe9xito}{%
\subsection{Descripción de los efectos de enseñanza y curación de Jesús
y su
éxito}\label{descripciuxf3n-de-los-efectos-de-enseuxf1anza-y-curaciuxf3n-de-jesuxfas-y-su-uxe9xito}}

\bibleverse{23} Jesús viajó por toda Galilea, enseñando en las
sinagogas, contando las buena nueva del reino, y sanando todas las
enfermedades que tenían las personas. \footnote{\textbf{4:23} Mar 1,39;
  Luc 4,44} \bibleverse{24} Entonces comenzó a difundirse la noticia
acerca de él por toda la provincia de Siria.\footnote{\textbf{4:24} El
  área del norte de Galilea.} La gente traía delante de él a todos los
que estaban enfermos: personas afligidas por todo tipo de enfermedades,
personas poseídas por demonios, enfermos mentales, paralíticos, y él los
sanaba a todos. \footnote{\textbf{4:24} Mar 6,55} \bibleverse{25}
Grandes multitudes le siguieron desde Galilea, Decápolis, Jerusalén,
Judea y la región que estaba al otro lado del Jordán.\footnote{\textbf{4:25}
  Mar 3,7-8; Luc 6,17-19}

\hypertarget{el-sermuxf3n-del-monte}{%
\subsection{El sermón del monte}\label{el-sermuxf3n-del-monte}}

\hypertarget{section-4}{%
\section{5}\label{section-4}}

\bibleverse{1} Cuando Jesús vio que las multitudes le seguían, subió a
una montaña. Allí se sentó junto con sus discípulos. \bibleverse{2} Y
comenzó a enseñarles, diciendo:

\hypertarget{las-bienaventuranzas}{%
\subsection{Las Bienaventuranzas}\label{las-bienaventuranzas}}

\bibleverse{3} ``Benditos son los que reconocen que son pobres
espiritualmente, porque de ellos es el reino de los cielos.
\bibleverse{4} Benditos son los que lloran, porque ellos serán
consolados. \footnote{\textbf{5:4} Sal 126,5; Apoc 7,17} \bibleverse{5}
Benditos son los que son bondadosos,\footnote{\textbf{5:5} Queriendo
  decir mansos, de temperamento afable.} porque ellos poseerán el mundo
entero. \footnote{\textbf{5:5} Mat 11,29; Sal 37,11} \bibleverse{6}
Benditos son aquellos cuyo mayor deseo\footnote{\textbf{5:6}
  Literalmente, ``aquellos que están hambrientos y sedientos''.} es
hacer lo justo, porque su deseo será saciado. \footnote{\textbf{5:6} Luc
  18,9-14; Juan 6,35} \bibleverse{7} Benditos aquellos que son
misericordiosos, porque a ellos se les mostrará misericordia.
\footnote{\textbf{5:7} Mat 25,35-46; Sant 2,13} \bibleverse{8} Benditos
son; los corazón puro, porque ellos verán a Dios. \footnote{\textbf{5:8}
  Sal 24,3-5; Sal 51,12; 1Jn 3,2; 1Jn 1,3} \bibleverse{9} Benditos
aquellos que trabajan por traer la paz, porque ellos serán llamados
hijos de Dios. \footnote{\textbf{5:9} Heb 12,14} \bibleverse{10}
Benditos aquellos que son perseguidos por lo que es justo, porque de
ellos es el reino de los cielos. \footnote{\textbf{5:10} 1Pe 3,14}

\bibleverse{11} Benditos ustedes cuando las personas los insulten y los
persigan, y los acusen de todo tipo de males por mi causa. \footnote{\textbf{5:11}
  Mat 10,22; Hech 5,41; 1Pe 4,14} \bibleverse{12} Estén felices, muy
felices, porque recibirán una gran recompensa en el cielo---pues ellos
persiguieron de esa misma manera a los profetas que vinieron antes de
ustedes. \footnote{\textbf{5:12} Sant 5,10; Heb 11,33-38}

\hypertarget{sal-de-la-tierra-luz-del-mundo}{%
\subsection{Sal de la tierra, luz del
mundo}\label{sal-de-la-tierra-luz-del-mundo}}

\bibleverse{13} ``Ustedes son la sal de la tierra, pero si la sal pierde
su sabor,\footnote{\textbf{5:13} O ``inútil''.} ¿cómo podrán hacer que
sea salada nuevamente? No sirve para nada, sino que se bota y es
pisoteada. \footnote{\textbf{5:13} Mar 9,50; Luc 14,34-35}

\bibleverse{14} Ustedes son la luz del mundo. Una ciudad que está
construida sobre lo alto de una montaña no puede ocultarse. \footnote{\textbf{5:14}
  Juan 8,12} \bibleverse{15} Nadie enciende una lámpara para luego
ocultarla bajo una cesta. No, se le coloca sobre un candelero y así da
luz a todos los que están en la casa. \footnote{\textbf{5:15} Mar 4,21;
  Luc 8,16} \bibleverse{16} De la misma manera, ustedes deben dejar que
su luz brille delante de todos a fin de que ellos puedan ver las cosas
buenas que ustedes hacen y alaben a su Padre celestial. \footnote{\textbf{5:16}
  Juan 15,8; Efes 5,8-9; Fil 2,14-15}

\hypertarget{perfecciuxf3n-comparada-con-las-exigencias-del-antiguo-pacto}{%
\subsection{Perfección comparada con las exigencias del antiguo
pacto}\label{perfecciuxf3n-comparada-con-las-exigencias-del-antiguo-pacto}}

\bibleverse{17} ``No piensen que vine a abolir la ley o los escritos de
los profetas. No vine a abolirlos, sino a cumplirlos. \footnote{\textbf{5:17}
  Mat 3,15; Rom 3,31; 1Jn 2,7} \bibleverse{18} Les aseguro que hasta que
el cielo y la tierra lleguen a su fin, ni una sola letra, ni un solo
punto que está en la ley quedarán descontinuados antes de que todo se
haya cumplido. \footnote{\textbf{5:18} Luc 16,17} \bibleverse{19} De
manera que cualquiera que desprecia\footnote{\textbf{5:19} O
  ``invalida''.} el mandamiento menos importante, y enseña a las
personas a hacer lo mismo, será considerado como el menos importante en
el reino de los cielos; pero cualquiera que practica y enseña los
mandamientos será considerado grande en el reino de los cielos.
\footnote{\textbf{5:19} Sant 2,10} \bibleverse{20} Les digo que a menos
que la justicia de ustedes no sea mayor que la justicia de los maestros
religiosos y de los Fariseos, no podrán entrar nunca al reino de los
cielos. \footnote{\textbf{5:20} Mat 23,2-33}

\hypertarget{acerca-de-matar-y-juzgar}{%
\subsection{Acerca de matar y juzgar}\label{acerca-de-matar-y-juzgar}}

\bibleverse{21} ``Ustedes han escuchado que la ley dijo\footnote{\textbf{5:21}
  Literalmente, ``Ustedes han escuchado que fue dicho''. Esta frase se
  usa a menudo en este pasaje del texto por parte de Jesús para
  referirse a las leyes que se encuentran en el Antiguo Testamento.} al
pueblo de hace mucho tiempo: `No matarás, y cualquiera que cometa
asesinato será condenado como culpable'.\footnote{\textbf{5:21} O,
  ``responsable de juicio''. Éxodo 20:13 o Deuteronomio 5:17.}
\bibleverse{22} Pero yo les digo: cualquiera que está enojado con su
hermano será condenado como culpable. Cualquiera que llama a su hermano
`idiota' tiene que dar cuenta ante el concilio,\footnote{\textbf{5:22}
  Probablemente, el concilio del Sanedrín.} y cualquiera que insulta a
la gente, de seguro irá al fuego de Gehena''.\footnote{\textbf{5:22} La
  palabra aquí, literalmente, es ``Gehena'', que a menudo se traduce
  como ``infierno'' o ``fuego infernal''. Gehena era el lugar situado a
  las afueras de Jerusalén donde se encendían fogatas para quemar la
  basura. ``Infierno'' es un concepto derivado de la mitología nórdica y
  anglosajona y no tiene paralelo con la idea de la cual se habla aquí.}

\bibleverse{23} ``Si estás delante del altar presentando una ofrenda, y
recuerdas que tu hermano tiene algo contra ti, \bibleverse{24} deja tu
ofrenda sobre el altar y ve y haz las paces con él primero, y luego
regresa y presenta tu ofrenda. \footnote{\textbf{5:24} Mar 11,25}
\bibleverse{25} Cuando vayas camino a la corte con tu adversario,
asegúrate de arreglar las cosas rápidamente. De lo contrario, tu
acusador podría entregarte ante el juez, y el juez te entregará a la
corte oficial, y serás llevado a la cárcel. \footnote{\textbf{5:25} Mat
  18,23-35; Luc 12,58-59} \bibleverse{26} En verdad te digo: no saldrás
de allí hasta que hayas pagado hasta el último centavo.

\hypertarget{sobre-adulteria}{%
\subsection{Sobre adulteria}\label{sobre-adulteria}}

\bibleverse{27} ``Ustedes han escuchado que la ley dijo: `No cometerás
adulterio'.\footnote{\textbf{5:27} Citando Éxodo 20:14 o Deuteronomio
  5:18.} \bibleverse{28} Pero yo les digo que todo el que mira con
lujuria a una mujer ya ha cometido adulterio en su corazón. \footnote{\textbf{5:28}
  2Sam 11,2; Job 31,1; 2Pe 2,14} \bibleverse{29} Si tu ojo derecho te
lleva a pecar, entonces sácalo y bótalo, porque es mejor perder una
parte de tu cuerpo y no que todo tu cuerpo sea lanzado en el fuego de
Gehena. \footnote{\textbf{5:29} Mat 18,8-9; Mar 9,43; Mar 9,47; Col 3,5}
\bibleverse{30} Si tu mano derecha te lleva a pecar, entonces córtala y
bótala, porque es mejor que pierdas uno de tus miembros y no que todo tu
cuerpo vaya al fuego de Gehena.

\bibleverse{31} ``La ley también dijo: `Si un hombre se divorcia de su
esposa, debe darle un certificado de divorcio'.\footnote{\textbf{5:31}
  Citando Deuteronomio 24:1.} \footnote{\textbf{5:31} Mat 19,3-9; Mar
  10,4-12} \bibleverse{32} Pero yo les digo que cualquier hombre que se
divorcia de su esposa, a menos que sea por inmoralidad sexual, la hace
cometer adulterio, y cualquiera que se case con una mujer divorciada,
comete adulterio. \footnote{\textbf{5:32} Luc 16,18; 1Cor 7,10-11}

\hypertarget{sobre-jurar}{%
\subsection{Sobre jurar}\label{sobre-jurar}}

\bibleverse{33} ``Y una vez más, ustedes han escuchado que la ley dijo
al pueblo de hace mucho tiempo: `No jurarás en falso. En lugar de ello,
asegúrese de cumplir sus juramentos al Señor'.\footnote{\textbf{5:33}
  Citando Números 30:2.} \bibleverse{34} Pero yo les digo: no juren
nada. No juren por el cielo, porque ese es el trono de Dios. \footnote{\textbf{5:34}
  Mat 2,16-22; Is 66,1} \bibleverse{35} No juren por la tierra, porque
es allí donde descansan sus pies. No juren por Jerusalén, por que es la
ciudad del gran Rey. \footnote{\textbf{5:35} Sal 48,3} \bibleverse{36}
Ni siquiera juren por su cabeza, porque ustedes no tienen el poder de
hacer que uno solo de sus cabellos sea blanco o negro. \bibleverse{37}
Solamente digan sí o no; cualquier cosa aparte de esto viene del
Maligno. \footnote{\textbf{5:37} Sant 5,12}

\hypertarget{sobre-lamor-al-projimo-y-al-enemigo}{%
\subsection{Sobre l'amor al projimo y al
enemigo}\label{sobre-lamor-al-projimo-y-al-enemigo}}

\bibleverse{38} ``Ustedes han escuchado que la ley dijo: `Ojo por ojo,
diente por diente'.\footnote{\textbf{5:38} Citando Éxodo 21:24; Levítico
  24:20; Deuteronomio 19:21.} \bibleverse{39} Pero yo les digo, no
pongan resistencia a alguien que es malvado. Si alguien les da una
bofetada, pongan la otra mejilla también. \bibleverse{40} Si alguien
quiere demandarte en una corte y toma tu camisa, dale tu abrigo
también.\footnote{\textbf{5:40} El abrigo era una prenda de vestir mucho
  más valiosa.} \footnote{\textbf{5:40} 1Cor 6,7; Heb 10,34}
\bibleverse{41} Si alguien te pide que le acompañes una milla,
acompáñale dos millas.\footnote{\textbf{5:41} Probablemente refiriéndose
  a un soldado romano que pedía que otra persona le llevara sus
  pertenencias.} \bibleverse{42} Da a quienes te pidan, y no rechaces a
quienes vengan a pedirte algo prestado.

\bibleverse{43} ``Ustedes han escuchado que la ley dijo: `Ama a tu
prójimo y odia a tu enemigo'.\footnote{\textbf{5:43} Citando Levítico
  19:18.} \bibleverse{44} Pero yo les digo: amen a sus enemigos y oren
por los que los persiguen, \bibleverse{45} a fin de que ustedes lleguen
a ser hijos del Padre celestial. Porque su sol sale sobre buenos y
malos; y él hace que la lluvia caiga sobre aquellos que hacen el bien y
también sobre los que hacen el mal. \footnote{\textbf{5:45} Efes 5,1}

\bibleverse{46} Porque si ustedes solamente aman a quienes los aman,
¿qué recompensa tienen por eso? ¿No hacen eso incluso los recaudadores
de impuestos? \bibleverse{47} Si ustedes solo hablan de manera amable
con su familia, ¿qué estarán haciendo que no hagan todos los demás?
¡Incluso los paganos\footnote{\textbf{5:47} Literalmente, ``naciones'',
  o ``gentiles''. Es un término comúnmente utilizado en al Nuevo
  Testamento para identificar a quienes no eran judíos, a aquellos
  quienes se consideraba que no seguían al verdadero Dios.} hacen eso!
\bibleverse{48} Crezcan y sean completamente fieles,\footnote{\textbf{5:48}
  Literalmente, ``perfectos, completos, sin división, integrales,
  maduros''. El concepto aquí se refiere a un estilo de vida totalmente
  dedicado a Dios más que a un concepto abstracto de perfección. El
  enfoque está en la madurez spiritual que se traduce en el hecho de que
  se pueda depender de esa persona, alguien en quien se puede confiar.}
así como su Padre que está en el cielo es fiel''.

\hypertarget{ten-cuidado-al-dar-limosna}{%
\subsection{Ten cuidado al dar
limosna}\label{ten-cuidado-al-dar-limosna}}

\hypertarget{section-5}{%
\section{6}\label{section-5}}

\bibleverse{1} ``Asegúrense de que sus buenas obras no sean delante de
la gente, solo para que los vean. De lo contrario, no tendrán ninguna
recompensa de su Padre que está en el cielo. \bibleverse{2} Cuando den a
los pobres, no sean como los hipócritas\footnote{\textbf{6:2} Esta es
  una palabra tomada del griego que literalmente significa
  ``actuación''.} que se jactan anunciando en las sinagogas y en las
calles lo que hacen para que la gente los alabe. Yo les digo la verdad:
ellos ya tienen su recompensa. \footnote{\textbf{6:2} 1Cor 13,3}
\bibleverse{3} Cuando den a los pobres, que su mano izquierda no sepa lo
que está haciendo su mano derecha. \footnote{\textbf{6:3} Mat 25,37-40;
  Rom 12,8} \bibleverse{4} De esta manera, lo que den será secreto, y su
Padre que ve lo que ocurre en secreto, los recompensará.

\hypertarget{ten-cuidado-cuando-oras}{%
\subsection{Ten cuidado cuando oras}\label{ten-cuidado-cuando-oras}}

\bibleverse{5} ``Cuando oren, no sean como los hipócritas, porque a
ellos les encanta ponerse en pie y orar en las sinagogas y en las
esquinas de las calles para que la gente los vea. Yo les prometo que
ellos ya tienen su recompensa. \bibleverse{6} Pero ustedes, cuando oren,
entren a su casa y cierren la puerta, y oren a su Padre en privado, y su
Padre que ve lo que ocurre en privado, los recompensará. \bibleverse{7}
Cuando oren, no usen palabrerías incoherentes como hacen los gentiles,
que piensan que serán escuchados por todas las palabras que repiten.
\footnote{\textbf{6:7} Is 1,15} \bibleverse{8} No sean como ellos,
porque su Padre sabe lo que ustedes necesitan incluso antes de que
ustedes se lo pidan. \bibleverse{9} Así que oren de esta manera:
``Nuestro Padre celestial, que tu nombre sean honrado. \bibleverse{10}
Venga tu reino. Que tu voluntad sea hecha en la tierra como se hace en
el cielo. \footnote{\textbf{6:10} Luc 22,42} \bibleverse{11} Por favor,
danos hoy el alimento que necesitamos. \bibleverse{12} Perdona nuestros
pecados, así como nosotros hemos perdonado a quienes han pecado contra
nosotros. \bibleverse{13} No dejes que seamos tentados a hacer el
mal,\footnote{\textbf{6:13} O, ``Por favor, ayúdanos a no rendirnos ante
  la tentación''.} y sálvanos del Maligno. \footnote{\textbf{6:13} 1Cró
  29,11-13; Juan 17,15}

\bibleverse{14} ``Porque si perdonan a quienes pecan contra ustedes, su
Padre celestial también los perdonará. \bibleverse{15} Pero si no
perdonan a quienes pecan contra ustedes, entonces su Padre celestial no
les perdonará sus pecados.

\hypertarget{ten-cuidado-cuando-ayunas}{%
\subsection{Ten cuidado cuando ayunas}\label{ten-cuidado-cuando-ayunas}}

\bibleverse{16} ``Cuando ayunen, no sean como los hipócritas que ponen
caras tristes y un semblante espantoso para que todos vean que están
ayunando. \footnote{\textbf{6:16} Is 58,5-9} \bibleverse{17} En lugar de
eso, cuando ayunen, laven sus rostros y luzcan elegantes,
\bibleverse{18} a fin de que las personas no vean que ustedes están
ayunando, y su Padre que es invisible y que ve lo que ocurre en privado,
los recompensará.

\hypertarget{recoge-tesoros-en-el-cielo}{%
\subsection{Recoge tesoros en el
cielo}\label{recoge-tesoros-en-el-cielo}}

\bibleverse{19} ``No acumulen riquezas aquí en la tierra donde la
polilla y el óxido las dañan, y donde los ladrones entran y las roban.
\bibleverse{20} En lugar de ello, ustedes deben acumular sus riquezas en
el cielo, donde la polilla y el óxido no las dañan, y donde los ladrones
no entran ni las roban. \bibleverse{21} Porque donde acumulen su
riqueza, allí es donde estará su corazón también.

\bibleverse{22} ``El ojo es como una lámpara que ilumina el cuerpo. De
manera que si tu ojo es sano,\footnote{\textbf{6:22} O, ``bueno,
  inocente''.} entonces todo tu cuerpo tendrá luz. \footnote{\textbf{6:22}
  Luc 11,34-36} \bibleverse{23} Pero si tu ojo es perverso, entonces
todo tu cuerpo estará en tinieblas. Si la luz dentro de ustedes está en
tinieblas, ¡cuán oscuro es eso! \footnote{\textbf{6:23} Juan 11,10}

\bibleverse{24} Nadie puede servir a dos amos. Odiarán a uno y amarán al
otro, o serán devotos a uno y despreciarán al otro. Ustedes no pueden
servir a Dios y al dinero a la vez.\footnote{\textbf{6:24} Literalmente,
  ``Mammón'', una transliteración de la palabra aramea que se usa para
  referirse al dios sirio del dinero y la riqueza.} \footnote{\textbf{6:24}
  Luc 16,9; Luc 16,13; Sant 4,4}

\hypertarget{busque-el-reino-de-dios-primero}{%
\subsection{Busque el reino de Dios
primero}\label{busque-el-reino-de-dios-primero}}

\bibleverse{25} ``Por eso les digo que no se preocupen por sus vidas. No
se preocupen por lo que van a comer, o por lo que van a beber, o por la
ropa con la que van a vestir. ¿Acaso no es la vida más importante que la
comida, y el cuerpo más que la ropa? \footnote{\textbf{6:25} Fil 4,6;
  1Pe 5,7; Luc 12,22-31} \bibleverse{26} Miren las aves\footnote{\textbf{6:26}
  Literalmente, ``aves del cielo'', refiriéndose a las aves silvestres
  más que a las aves domésticas.} ---ellas no siembran ni cosechan, ni
guardan alimento en los graneros, porque el Padre celestial las
alimenta. ¿No son ustedes más que las aves? \footnote{\textbf{6:26} Mat
  10,29-31; Luc 12,6-7}

\bibleverse{27} ¿Quién de ustedes puede, por mucho que se afane, añadir
un minuto a su vida? \bibleverse{28} ¿Y por qué se preocupan por la
ropa? Miren las hermosas flores del campo. Miren cómo crecen: No
trabajan ni hilan. \bibleverse{29} Pero les digo que ni siquiera Salomón
en todo su esplendor se vistió como una de esas flores. \bibleverse{30}
De modo que si Dios decora los campos así, la hierba que está hoy aquí y
que mañana es lanzada al fuego, ¿no hará mucho más por ustedes que son
personas que creen tan poco?

\bibleverse{31} Así que no se preocupen diciendo `¿Qué comeremos?' o
`¿Qué beberemos?' o `¿Qué vestiremos?' \bibleverse{32} Todas estas son
las cosas que por las que los paganos se afanan, pero el Padre celestial
ya sabe todo lo que ustedes necesitan. \bibleverse{33} Busquen su reino
en primer lugar, y su senda de justicia, y todo se les dará. \footnote{\textbf{6:33}
  Rom 14,17; 1Re 3,13-14; Sal 37,4; Sal 37,25} \bibleverse{34} Así que
no se preocupen por el día de mañana, porque el mañana puede preocuparse
por sí mismo. Cada día trae su propio mal.\footnote{\textbf{6:34} Éxod
  16,19}

\hypertarget{no-juzguuxe9is}{%
\subsection{No juzguéis}\label{no-juzguuxe9is}}

\hypertarget{section-6}{%
\section{7}\label{section-6}}

\bibleverse{1} ``No juzguen a otros, para que ustedes no sean juzgados.
\footnote{\textbf{7:1} Rom 2,1; 1Cor 4,5} \bibleverse{2} Porque
cualquiera que sea el criterio que usen para juzgar a otros, será usado
para juzgarlos a ustedes, y cualquiera que sea la medida que ustedes
usen para medir a otros, será usada para medirlos a ustedes. \footnote{\textbf{7:2}
  Mar 4,24} \bibleverse{3} ¿Por qué miras la astilla que está en el ojo
de tu hermano? ¿No te das cuenta de la viga que está en tu propio ojo?
\bibleverse{4} ¿Cómo puedes decirle a tu hermano: `Déjame sacarte esa
astilla de tu ojo' mientras tu tienes una viga en tu propio ojo?
\bibleverse{5} ¡Estás siendo un hipócrita! Primero saca la viga que
tienes en tu propio ojo. Entonces podrás ver con claridad y sacar la
astilla del ojo de tu hermano.

\bibleverse{6} ``No den a los perros lo que es santo. No tiren sus
perlas a los cerdos. Así los cerdos no las pisotearán, y los perros no
vendrán a atacarlos a ustedes. \footnote{\textbf{7:6} Mat 10,11; Luc
  23,9}

\hypertarget{pedid-buscad-llaman}{%
\subsection{Pedid, buscad, llaman}\label{pedid-buscad-llaman}}

\bibleverse{7} ``Pidan y se les dará, busquen y encontrarán, toquen a la
puerta y la puerta se abrirá para ustedes.\footnote{\textbf{7:7} En el
  texto original, estos son presentes imperativos, y podría traducirse
  como ``sigan pidiendo'' etc.} \footnote{\textbf{7:7} Jer 29,13-14; Mar
  11,24; Luc 11,5-13; Juan 14,13} \bibleverse{8} Todo el que pide,
recibe; todo el que busca, encuentra; a todo el que toca, se le abre la
puerta. \bibleverse{9} ¿Acaso alguno de ustedes le daría una piedra a su
hijo si este le pide un pan? \bibleverse{10} ¿O si le pidiera un pez, le
daría una serpiente? \bibleverse{11} De modo que si incluso ustedes que
son malos saben dar cosas buenas a sus hijos, cuánto más el Padre
celestial dará cosas buenas a quienes le piden. \footnote{\textbf{7:11}
  Sant 1,17}

\hypertarget{la-regla-de-oro-de-la-caridad}{%
\subsection{La regla de oro de la
caridad}\label{la-regla-de-oro-de-la-caridad}}

\bibleverse{12} ``Traten a los demás como quieren que los traten a
ustedes. Esto resume la ley y los profetas. \footnote{\textbf{7:12} Mat
  22,36-40; Rom 13,8-10; Gal 5,14}

\bibleverse{13} Entren por la puerta estrecha. Porque es amplia la
puerta y espacioso el camino que lleva a la destrucción, y muchos andan
por él. \footnote{\textbf{7:13} Luc 13,24} \bibleverse{14} Pero estrecha
es la puerta y angosto el camino que llevan a la vida, y solo unos pocos
lo encuentran. \footnote{\textbf{7:14} Mat 19,24; Hech 14,22}

\hypertarget{guardaos-de-los-falsos-profetas}{%
\subsection{Guardaos de los falsos
profetas}\label{guardaos-de-los-falsos-profetas}}

\bibleverse{15} ``Tengan cuidado con los falsos profetas que vienen
vestidos de ovejas, pero por dentro son lobos feroces. \footnote{\textbf{7:15}
  Mat 24,4-5; Mat 24,24; 2Cor 11,13-15} \bibleverse{16} Pueden
reconocerlos por sus frutos.\footnote{\textbf{7:16} En otras palabras,
  ustedes pueden reconocerlos por los resultados de lo que hacen.}
¿Acaso las personas cosechan uvas de los matorrales de espinos, o higos
de los cardos? \footnote{\textbf{7:16} Gal 5,19-22; Sant 3,12}
\bibleverse{17} De modo que todo árbol bueno produce frutos buenos,
mientras que un árbol malo produce frutos malos. \footnote{\textbf{7:17}
  Mat 12,33} \bibleverse{18} Un buen árbol no puede producir frutos
malos, y un árbol malo no puede producir frutos buenos. \bibleverse{19}
Todo árbol que no produce frutos buenos, se corta y se lanza al fuego.
\bibleverse{20} Así que por sus frutos los conocerán.

\hypertarget{sea-el-hacedor-de-la-palabra-no-solo-un-oyente}{%
\subsection{Sea el hacedor de la palabra, no solo un
oyente}\label{sea-el-hacedor-de-la-palabra-no-solo-un-oyente}}

\bibleverse{21} ``No todo el que me dice `Señor, Señor' entrará al reino
de los cielos, sino solo aquellos que hacen la voluntad de mi Padre que
está en el cielo. \footnote{\textbf{7:21} Rom 2,13; Sant 1,22}
\bibleverse{22} Muchos me dirán el día del juicio: `Señor, Señor, ¿acaso
no profetizamos, nos sacamos demonios e hicimos muchos milagros en tu
nombre?' \footnote{\textbf{7:22} Jer 27,13; Luc 13,25-27}
\bibleverse{23} Entonces yo les diré: `Yo nunca los conocí a ustedes.
¡Apártense de mi, practicantes de la maldad!'\footnote{\textbf{7:23} Ver
  Salmos 6:8.} \footnote{\textbf{7:23} Mat 25,12; 2Tim 2,19}

\bibleverse{24} Todo aquél que escucha las palabras que yo digo, y las
sigue, es como el hombre sabio que construyó su casa sobre la roca
sólida. \bibleverse{25} La lluvia cayó, hubo inundación y los vientos
soplaron fuertemente contra aquella casa, pero no se cayó porque su
fundamento estaba sobre la roca sólida. \bibleverse{26} Pero todo aquél
que escucha las palabras que yo digo y no las sigue, es como el hombre
necio que construyó su casa sobre la arena. \bibleverse{27} La lluvia
cayó, hubo inundación y los vientos soplaron fuertemente contra aquella
casa, y se cayó. Colapsó por completo''.

\bibleverse{28} Cuando Jesús terminó de explicar estas cosas, las
multitudes se maravillaban de su enseñanza, \footnote{\textbf{7:28} Hech
  2,12} \bibleverse{29} porque él enseñaba como alguien que tenía
autoridad, y no como sus maestros religiosos.\footnote{\textbf{7:29}
  Juan 7,16; Juan 7,46}

\hypertarget{sanando-a-un-leproso}{%
\subsection{Sanando a un leproso}\label{sanando-a-un-leproso}}

\hypertarget{section-7}{%
\section{8}\label{section-7}}

\bibleverse{1} Grandes multitudes siguieron a Jesús cuando bajó de la
montaña. \bibleverse{2} Un leproso se acercó a él, y se arrodilló,
adorándolo, y le dijo: ``Señor, si quieres, por favor sáname''.

\bibleverse{3} Jesús se extendió hacia él y lo tocó con su mano.
``Quiero'', le dijo. ``Queda sano''. Inmediatamente este hombre fue
sanado de su lepra. \bibleverse{4} ``Asegúrate de no contárselo a
nadie'', le dijo Jesús. ``Ve y preséntate ante el sacerdote y da la
ofrenda que Moisés ordenó, como evidencia pública''.\footnote{\textbf{8:4}
  Como prueba de que había sido sanado y de que estaba ceremonialmente
  limpio. Ver Levítico 14} \footnote{\textbf{8:4} Mar 8,30; Lev 14,2-32}

\hypertarget{sanaciuxf3n-del-siervo-del-centuriuxf3n-de-capernaum}{%
\subsection{Sanación del siervo del centurión de
Capernaum}\label{sanaciuxf3n-del-siervo-del-centuriuxf3n-de-capernaum}}

\bibleverse{5} Cuando Jesús entró a Capernaúm, un centurión se le
acercó, suplicándole su ayuda, \bibleverse{6} ``Señor, mi siervo está en
casa, acostado y sin poder moverse. Está sufriendo una terrible
agonía''.

\bibleverse{7} ``Iré y lo sanaré'', respondió Jesús.

\bibleverse{8} El centurión respondió: ``Señor, no merezco una visita a
mi casa. Solo di la palabra y mi siervo quedará sano. \bibleverse{9}
Porque yo mismo estoy bajo la autoridad de mis superiores, y a la vez yo
también tengo soldados bajo mi mando. Yo le ordeno a uno: `¡Ve!' y él
va. Mando a otro: `¡Ven!' y él viene. Digo a mi siervo: `¡Haz esto!' y
él lo hace''.

\bibleverse{10} Cuando Jesús escuchó lo que este hombre dijo, se quedó
asombrado. Entonces le dijo a los que le seguían: ``En verdad les digo
que no he encontrado este tipo de confianza en ninguna parte de Israel.
\bibleverse{11} Les digo que muchos vendrán del este y del oeste, y se
sentarán con Abraham e Isaac en el reino de los cielos. \footnote{\textbf{8:11}
  Luc 13,28-29} \bibleverse{12} Pero los herederos\footnote{\textbf{8:12}
  Refiriéndose a los descendientes de Abraham e Isaac que confiaron en
  su ascendencia para la salvación.} del reino serán lanzados a la
oscuridad absoluta, donde habrá lamento y crujir de dientes''.
\bibleverse{13} Entonces Jesús le dijo al centurión, ``Ve a casa. Lo que
pediste ya fue hecho, como creíste que pasaría''. Y el siervo fue sanado
inmediatamente.

\hypertarget{sanaciuxf3n-de-la-suegra-de-pedro-y-de-muchos-otros-enfermos-en-cafarnauxfam}{%
\subsection{Sanación de la suegra de Pedro y de muchos otros enfermos en
Cafarnaúm}\label{sanaciuxf3n-de-la-suegra-de-pedro-y-de-muchos-otros-enfermos-en-cafarnauxfam}}

\bibleverse{14} Cuando Jesús llegó a la casa de Pedro, vio que la suegra
de Pedro estaba enferma en cama y tenía una fiebre muy alta. \footnote{\textbf{8:14}
  1Cor 9,5} \bibleverse{15} Entonces Jesús tocó su mano y se le quitó la
fiebre. Ella se levantó y comenzó a prepararle comida a Jesús.
\bibleverse{16} Cuando llegó la noche, trajeron ante Jesús a un hombre
endemoniado. Con solo una orden, Jesús hizo que los espíritus salieran
de él, y sanó a todos los que estaban enfermos. \bibleverse{17} Esto
cumplió lo que el profeta Isaías dijo: ``Él sanó nuestras enfermedades y
nos libertó de nuestras dolencias''.\footnote{\textbf{8:17} Citando
  Isaías 53:4.}

\hypertarget{jesuxfas-escapa-a-la-otra-orilla-del-lago-proverbios-sobre-seguir-a-jesuxfas}{%
\subsection{Jesús escapa a la otra orilla del lago; Proverbios sobre
seguir a
Jesús}\label{jesuxfas-escapa-a-la-otra-orilla-del-lago-proverbios-sobre-seguir-a-jesuxfas}}

\bibleverse{18} Cuando Jesús vio las multitudes que lo rodeaban, dio
instrucciones de que debían\footnote{\textbf{8:18} ``debían'' se refiere
  a Jesús y los discípulos.} ir al otro lado del lago.

\bibleverse{19} Entonces uno de los maestros religiosos se acercó a él y
le dijo: ``Maestro, te seguiré adonde vayas''.

\bibleverse{20} ``Los zorros tienen guaridas y las aves silvestres
tienen nidos, pero el Hijo del hombre no tiene dónde recostarse y
descansar'',\footnote{\textbf{8:20} Literalmente, ``recostar su
  cabeza''.} le dijo Jesús.

\bibleverse{21} Otro discípulo le dijo a Jesús: ``Señor, primero déjame
ir y sepultar a mi padre''. \footnote{\textbf{8:21} Mat 10,37}

\bibleverse{22} ``Sígueme. Deja que los muertos sepulten a sus propios
muertos'', le respondió Jesús.

\hypertarget{jesuxfas-apacigua-la-tormenta-del-mar}{%
\subsection{Jesús apacigua la tormenta del
mar}\label{jesuxfas-apacigua-la-tormenta-del-mar}}

\bibleverse{23} Entonces Jesús subió a una barca y sus discípulos se
fueron con él. \bibleverse{24} De repente, sopló una fuerte tormenta, y
las olas golpeaban fuertemente contra la barca, pero Jesús seguía
durmiendo. \bibleverse{25} Los discípulos se acercaron a él y lo
despertaron gritándole: ``¡Sálvanos, Señor! ¡Vamos a hundirnos!''

\bibleverse{26} ``¿Por qué tienen tanto miedo? ¿Por qué tienen tan poca
confianza?'' les preguntó Jesús. Entonces se levantó y ordenó a los
vientos y las olas que se detuvieran. Y todo quedó completamente en
calma.

\bibleverse{27} Los discípulos estaban asombrados y decían: ``¿Quién es
este? ¿Incluso los vientos y las olas le obedecen?''

\hypertarget{curaciuxf3n-de-dos-poseuxeddos-en-la-tierra-de-los-gadarenos}{%
\subsection{Curación de dos poseídos en la tierra de los
gadarenos}\label{curaciuxf3n-de-dos-poseuxeddos-en-la-tierra-de-los-gadarenos}}

\bibleverse{28} Cuando llegaron al otro lado, a la región de los
gadarenos, dos hombres endemoniados salieron del cementerio para
encontrarse con Jesús. Estos hombres eran tan peligrosos que nadie se
atrevía a pasar por ese camino. \footnote{\textbf{8:28} Luc 4,41; 2Pe
  2,4; Sant 2,19} \bibleverse{29} Y ellos gritaban: ``¿Qué tienes que
ver con nosotros, Hijo de Dios? ¿Has venido a torturarnos antes de
tiempo?'' \bibleverse{30} A lo lejos, había un gran hato de cerdos
comiendo. \bibleverse{31} Los demonios le suplicaron a Jesús: ``Si vas a
sacarnos de aquí, envíanos a ese hato de cerdos''.

\bibleverse{32} ``¡Vayan!'' les dijo Jesús. Los demonios salieron de los
dos hombres y huyeron hacia el hato de cerdos. Todo el hato de cerdos
corrió, descendiendo por una pendiente, hasta que cayeron al mar y se
ahogaron.

\bibleverse{33} Los que cuidaban el rebaño de cerdos, salieron
corriendo. Entonces se fueron hacia la ciudad y le contaron a la gente
que estaba allí todo lo que había sucedido y lo que había ocurrido con
los dos hombres endemoniados. \bibleverse{34} Y toda la ciudad salió
para encontrarse con Jesús. Cuando lo encontraron, le suplicaron que
abandonara su ciudad.

\hypertarget{curaciuxf3n-de-un-paraluxedtico-en-capernaum-jesuxfas-perdona-los-pecados}{%
\subsection{Curación de un paralítico en Capernaum; Jesús perdona los
pecados}\label{curaciuxf3n-de-un-paraluxedtico-en-capernaum-jesuxfas-perdona-los-pecados}}

\hypertarget{section-8}{%
\section{9}\label{section-8}}

\bibleverse{1} Entonces Jesús tomó una barca para cruzar nuevamente el
lago hacia la ciudad donde él vivía. \bibleverse{2} Allí le trajeron a
un hombre paralítico acostado en una estera. Cuando Jesús vio cuánto
confiaban en él, le dijo al paralítico: ``¡Anímate, amigo
mío!\footnote{\textbf{9:2} Literalmente, ``hijo''.} Tus pecados están
perdonados''. \footnote{\textbf{9:2} Éxod 34,6-7; Sal 103,3}

\bibleverse{3} En respuesta a esto, algunos de los maestros religiosos
decían para sí mismos: ``¡Está blasfemando!'' \footnote{\textbf{9:3} Mat
  26,65}

\bibleverse{4} Pero Jesús sabía lo que ellos estaban pensando. Entonces
les preguntó: ``¿Por qué tienen pensamientos malvados en sus corazones?
\footnote{\textbf{9:4} Juan 2,25} \bibleverse{5} ¿Qué es más fácil
decir, `tus pecados están perdonados', o `levántate y camina'?
\bibleverse{6} Pero ahora, para convencerlos de que el Hijo del hombre
tiene autoridad para perdonar pecados\ldots{}'' Dirigiéndose al hombre
paralítico, le dijo: ``¡Levántate, toma tu estera y vete a casa!''

\bibleverse{7} El hombre se levantó y se fue a su casa. \bibleverse{8}
Cuando las multitudes vieron lo que había sucedido, estaban
atemorizados. Entonces alabaron a Dios por haber dado a los seres
humanos semejante poder.

\hypertarget{llamada-del-recaudador-de-impuestos-mateo-jesuxfas-como-compauxf1ero-de-mesa-para-recaudadores-de-impuestos-y-pecadores}{%
\subsection{Llamada del recaudador de impuestos Mateo; Jesús como
compañero de mesa para recaudadores de impuestos y
pecadores}\label{llamada-del-recaudador-de-impuestos-mateo-jesuxfas-como-compauxf1ero-de-mesa-para-recaudadores-de-impuestos-y-pecadores}}

\bibleverse{9} Cuando Jesús se fue de allí, vio a un hombre llamado
Mateo que estaba sentado en su cabina de cobro de impuestos. Jesús lo
llamó diciéndole ``Sígueme''. Entonces él se levantó y siguió a Jesús.
\footnote{\textbf{9:9} Mat 10,3} \bibleverse{10} Mientras Jesús comía en
la casa de Mateo, muchos recaudadores de impuestos vinieron y se
sentaron en la mesa con él y sus discípulos. \bibleverse{11} Y cuando
los Fariseos vieron esto, le preguntaron a los discípulos de Jesús:
``¿Por qué el Maestro de ustedes come con los recaudadores de impuestos
y pecadores?''

\bibleverse{12} Cuando Jesús escuchó la pregunta, respondió: ``Los que
están sanos no necesitan de un médico, pero los que están enfermos, sí.
\bibleverse{13} Vayan y descubran lo que esto significa: `quiero
misericordia, no sacrificio. Porque no vine a llamar a los que hacen el
bien---Vine a llamar a los pecadores'\,''.\footnote{\textbf{9:13}
  Citando Oseas 6:6.} \footnote{\textbf{9:13} 1Sam 15,22; Mat 18,11}

\hypertarget{la-pregunta-del-ayuno-de-los-discuxedpulos-de-juan}{%
\subsection{La pregunta del ayuno de los discípulos de
Juan}\label{la-pregunta-del-ayuno-de-los-discuxedpulos-de-juan}}

\bibleverse{14} Entonces los discípulos de Juan vinieron y le
preguntaron: ``¿Por qué nosotros y los Fariseos ayunamos a menudo y tus
discípulos no lo hacen?'' \footnote{\textbf{9:14} Luc 18,12}

\bibleverse{15} ``¿Acaso los invitados a la boda lloran cuando el novio
está con ellos?'' respondió Jesús. ``Pero viene el tiempo cuando el
novio ya no estará y entonces ayunarán. \footnote{\textbf{9:15} Juan
  3,29} \bibleverse{16} Nadie pone un parche nuevo en ropas viejas, de
lo contrario, se encogerá y hará que la rotura luzca peor. \footnote{\textbf{9:16}
  Rom 7,6} \bibleverse{17} Nadie echa tampoco el vino nuevo en odres
viejos, de lo contrario los odres podrían romperse, derramando así el
vino y dañando los odres. No, el vino nuevo se coloca en odres nuevos, y
así ambos perduran''.

\hypertarget{resucitar-a-la-hija-de-jairo-y-curar-a-la-mujer-asolada-por-la-sangre}{%
\subsection{Resucitar a la hija de Jairo y curar a la mujer asolada por
la
sangre}\label{resucitar-a-la-hija-de-jairo-y-curar-a-la-mujer-asolada-por-la-sangre}}

\bibleverse{18} Mientras él les decía esto, uno de los oficiales
principales llegó y se postró delante de él. ``Mi hija acaba de morir'',
le dijo el hombre a Jesús. ``Pero sé que si tú vas y colocas tu mano
sobre ella, volverá a vivir''.

\bibleverse{19} Jesús y sus discípulos se levantaron y lo siguieron.
\bibleverse{20} En ese momento, una mujer que había estado enferma con
sangrado durante doce años, venía detrás de él y tocó el dobladillo de
su manto. \bibleverse{21} Ella había pensado para sí: ``Si tan solo
puedo llegar a tocar su manto, seré sanada''. \footnote{\textbf{9:21}
  Mat 14,36}

\bibleverse{22} Jesús se dio vuelta y la vio. ``Alégrate hija, pues tu
confianza en mi te ha sanado'', le dijo. Y la mujer fue sanada de
inmediato.

\bibleverse{23} Jesús llegó a la casa del oficial. Vio a los que tocaban
las flautas y escuchó a la multitud que lloraba a gritos.
\bibleverse{24} ``Por favor, salgan'' -- les dijo -- ``porque esta niña
no está muerta, sino que simplemente está dormida''. Pero ellos se
rieron y se burlaron de él.

\bibleverse{25} Sin embargo, cuando la multitud había sido despedida,
Jesús entró y tomó a la niña por la mano y esta se levantó.
\bibleverse{26} Y la noticia sobre lo que había ocurrido se esparció por
toda esa región.

\hypertarget{curaciuxf3n-de-dos-ciegos-y-un-mudo-endemoniado-graduaciuxf3n}{%
\subsection{Curación de dos ciegos y un mudo endemoniado;
Graduación}\label{curaciuxf3n-de-dos-ciegos-y-un-mudo-endemoniado-graduaciuxf3n}}

\bibleverse{27} Al seguir Jesús su camino, dos hombres ciegos lo seguían
y le gritaban: ``¡Hijo de David, ten misericordia de nosotros!''
\footnote{\textbf{9:27} Mat 20,29-34} \bibleverse{28} Y cuando Jesús
entró a la casa donde se alojaba, los hombres ciegos entraron también.
``¿Están convencidos de que yo puedo hacer esto?'' les preguntó. ``Sí,
Señor'', respondieron ellos. \footnote{\textbf{9:28} Hech 14,9}

\bibleverse{29} Entonces Jesús tocó los ojos de ellos, y dijo: ``¡Por la
confianza que tienen en mí, así será!'' \footnote{\textbf{9:29} Mat 8,13}
\bibleverse{30} Y ellos pudieron ver. Jesús les advirtió: ``Asegúrense
de que nadie sepa esto''. \footnote{\textbf{9:30} Mat 8,4}
\bibleverse{31} Pero ellos se fueron y dieron a conocer acerca de Jesús
por todas partes.

\bibleverse{32} Cuando Jesús y sus discípulos ya se marchaban, trajeron
ante Jesús a un hombre que estaba mudo y endemoniado. \bibleverse{33}
Cuando el demonio fue expulsado de él, el hombre habló, y las multitudes
estaban maravilladas. ``Nunca antes había ocurrido algo como esto en
Israel'', decían.

\bibleverse{34} Pero los Fariseos comentaban diciendo: ``el echa fuera
los demonios con el poder del jefe de los demonios''. \footnote{\textbf{9:34}
  Mat 12,24-32}

\bibleverse{35} Jesús iba a todas partes, visitando ciudades y aldeas.
Enseñaba en sus sinagogas, les enseñaba acerca de la buena noticia del
reino, y sanaba todo tipo de enfermedades.

\hypertarget{la-compasiuxf3n-de-jesuxfas-a-la-vista-de-la-gente-la-palabra-de-la-cosecha}{%
\subsection{La compasión de Jesús a la vista de la gente; la palabra de
la
cosecha}\label{la-compasiuxf3n-de-jesuxfas-a-la-vista-de-la-gente-la-palabra-de-la-cosecha}}

\bibleverse{36} Cuando veía las multitudes, Jesús sentía gran compasión
por ellos, porque estaban atribulados y desamparados, como ovejas sin
pastor. \bibleverse{37} Entonces le dijo a sus discípulos, ``la cosecha
es grande, pero hay apenas unos pocos trabajadores. \footnote{\textbf{9:37}
  Luc 10,2}

\bibleverse{38} Oren al Señor de la cosecha, y pídanle que envíe más
trabajadores para su cosecha''.

\hypertarget{llamadas-y-nombres-de-los-doce-discuxedpulos}{%
\subsection{Llamadas y nombres de los doce
discípulos}\label{llamadas-y-nombres-de-los-doce-discuxedpulos}}

\hypertarget{section-9}{%
\section{10}\label{section-9}}

\bibleverse{1} Jesús llamó y reunió a sus doce discípulos y les dio
poder para echar fuera espíritus malos y para sanar todo tipo de
enfermedades. \bibleverse{2} Estos son los nombres de los doce
apóstoles: primero, Simón (también llamado Pedro), su hermano Andrés,
Santiago el hijo de Zebedeo, su hermano Juan, \bibleverse{3} Felipe,
Bartolomé, Tomás, Mateo el recaudador de impuestos, Santiago el hijo de
Alfeo, Tadeo, \bibleverse{4} Simón el revolucionario y Judas Iscariote,
quien entregó a Jesús.

\hypertarget{el-mensaje-enviado-a-los-doce-discuxedpulos}{%
\subsection{El mensaje enviado a los doce
discípulos}\label{el-mensaje-enviado-a-los-doce-discuxedpulos}}

\bibleverse{5} A estos doce envió Jesús, diciéndoles: ``no vayan a los
gentiles, ni a ninguna ciudad samaritana. \bibleverse{6} Ustedes deben
ir a las ovejas perdidas de la casa de Israel. \footnote{\textbf{10:6}
  Mat 15,24; Hech 13,46} \bibleverse{7} Donde vayan, díganle a la gente:
`el reino de los cielos está cerca'. \footnote{\textbf{10:7} Mat 4,17;
  Luc 10,9} \bibleverse{8} Sanen a los que estén enfermos. Resuciten a
los muertos. Sanen a los leprosos. Echen fuera demonios. ¡Ustedes
recibieron gratuitamente, entonces den gratuitamente! \footnote{\textbf{10:8}
  Mar 16,17; Hech 20,33} \bibleverse{9} No lleven oro, plata, ni monedas
de cobre en sus bolsillos, \bibleverse{10} ni lleven una bolsa de
provisiones para el camino, ni dos abrigos, o sandalias, ni un bastón
para caminar, porque todo trabajador merece su sustento.\footnote{\textbf{10:10}
  O ``alimento''.} \bibleverse{11} Donde vayan, cualquiera sea la ciudad
o aldea, pregunten por alguien que viva conforme a buenos principios, y
quédense allí hasta que se marchen. \bibleverse{12} Cuando lleguen a una
casa, dejen bendición en ella. \footnote{\textbf{10:12} Luc 10,5-6}
\bibleverse{13} Si esa casa la merece, dejen su paz\footnote{\textbf{10:13}
  ``Paz'', refiriéndose a bendición.} en ella, pero si no la merece, la
paz regresará a ustedes. \bibleverse{14} ``Si alguien no los recibe
bien, y se niega a escuchar el mensaje que ustedes tienen que decir,
entonces váyanse de esa casa o de esa ciudad, sacudiendo el polvo de sus
pies mientras se marchan. \bibleverse{15} Les digo la verdad: ¡Mejor
será el Día del Juicio para Sodoma y Gomorra que para esa ciudad!
\footnote{\textbf{10:15} Gén 19,1-29}

\bibleverse{16} ``Miren que los estoy enviando como ovejas entre lobos.
Así que sean astutos como serpientes y mansos como palomas. \footnote{\textbf{10:16}
  Rom 16,19; Efes 5,15}

\hypertarget{anuncio-de-los-sufrimientos-que-vendruxe1n-a-los-discuxedpulos}{%
\subsection{Anuncio de los sufrimientos que vendrán a los
discípulos}\label{anuncio-de-los-sufrimientos-que-vendruxe1n-a-los-discuxedpulos}}

\bibleverse{17} Cuídense de aquellos que los entregarán para ser
juzgados en los concilios de las ciudades\footnote{\textbf{10:17}
  Literalmente, ``sanedrines'', que eran cortes religiosas locales.} y
que los azotarán en sus sinagogas. \footnote{\textbf{10:17} Hech 5,40;
  2Cor 11,24} \bibleverse{18} Ustedes serán arrastrados ante gobernantes
y reyes por mi causa, para dar testimonio a ellos y a los gentiles.
\footnote{\textbf{10:18} Hech 25,23; Hech 27,24} \bibleverse{19} Pero
cuando ellos los lleven a juicio, no se preocupen por la manera como
deben hablar o por lo que deben decir, porque a ustedes se les dirá lo
que deben decir en el momento correcto. \footnote{\textbf{10:19} Luc
  12,11-12; Hech 4,8} \bibleverse{20} Porque no serán ustedes los que
hablarán, sino el espíritu del Padre quien hablará por medio de ustedes.
\footnote{\textbf{10:20} 1Cor 2,4}

\bibleverse{21} El hermano entregará a su hermano y lo mandará a matar,
y el padre hará lo mismo con su hijo. Los hijos se rebelarán contra sus
padres, y los entregarán a la muerte. \footnote{\textbf{10:21} Miq 7,6}
\bibleverse{22} Todo el mundo los odiará a ustedes porque ustedes me
siguen a mi, pero todo aquél que persevere hasta el fin, será salvo.
\footnote{\textbf{10:22} Mat 24,9-13; 2Tim 2,12} \bibleverse{23}
``Cuando ustedes sean perseguidos en una ciudad, huyan a otra. Les digo
la verdad: no terminarán de ir a las ciudades de Israel antes de que
venga el Hijo del hombre. \footnote{\textbf{10:23} Mat 16,28; Hech 8,1}

\bibleverse{24} Los discípulos no son más importantes que su maestro;
\footnote{\textbf{10:24} Luc 6,40; Juan 13,16; Juan 15,20}
\bibleverse{25} ellos deben estar satisfechos con llegar a ser como su
maestro, y los siervos como su amo. Si a quien es la cabeza del hogar le
han llamado demonio Belcebú,\footnote{\textbf{10:25} Belcebú,
  refiriéndose a Satanás.} ¡aún más llamarán demonios a los demás
miembros de esta casa! \footnote{\textbf{10:25} Mat 12,24}

\hypertarget{aliento-para-perseverar-fielmente-y-consuelo-en-tiempos-de-tribulaciuxf3n}{%
\subsection{Aliento para perseverar fielmente y consuelo en tiempos de
tribulación}\label{aliento-para-perseverar-fielmente-y-consuelo-en-tiempos-de-tribulaciuxf3n}}

\bibleverse{26} Así que no les tengan miedo, porque no hay nada
encubierto que no salga a la luz, ni hay nada oculto que no se llegue a
saber. \footnote{\textbf{10:26} Mar 4,22; Luc 8,17} \bibleverse{27} Lo
que yo les digo aquí en la oscuridad, díganlo a la luz del día, y lo que
han oído como un susurro en sus oídos, grítenlo desde las azoteas.
\bibleverse{28} No tengan miedo de aquellos que pueden matarlos
físicamente, pero que no pueden matarlos espiritualmente. En lugar de
ello, tengan miedo de Aquel que puede destruirlos física y
espiritualmente en Gehena.\footnote{\textbf{10:28} ``Gehena''. Ver la
  nota del versículo 5:22.} \footnote{\textbf{10:28} Heb 10,31; Sant
  4,12}

\bibleverse{29} ¿No se venden dos gorriones por el precio de un solo
centavo? Pero ninguno de ellos cae al suelo sin que el Padre lo sepa.
\bibleverse{30} Incluso los cabellos que ustedes tienen en sus cabezas
han sido contados. \bibleverse{31} Así que no se preocupen. ¡Ustedes
valen más que muchos gorriones! \footnote{\textbf{10:31} Mat 6,26}
\bibleverse{32} ``Si alguno declara públicamente su
compromiso\footnote{\textbf{10:32} Literalmente, ``confiesa''.} conmigo,
yo también declararé mi compromiso con él ante mi Padre que está en el
cielo. \footnote{\textbf{10:32} Apoc 3,5} \bibleverse{33} Pero si alguno
me niega públicamente, yo también lo negaré ante mi Padre en el cielo.
\footnote{\textbf{10:33} Mar 8,38; Luc 9,26; 2Tim 2,12}

\hypertarget{paz-y-espada-puxe9rdida-y-ganancia}{%
\subsection{Paz y espada, pérdida y
ganancia}\label{paz-y-espada-puxe9rdida-y-ganancia}}

\bibleverse{34} No piensen que he venido a traer paz a la tierra. No he
venido a traer paz sino espada. \footnote{\textbf{10:34} Luc 12,51-53}
\bibleverse{35} He venido `a poner al hombre contra su padre, a la hija
contra su madre, y a la nuera contra su suegra. \bibleverse{36} ¡Sus
enemigos serán los de su propia familia!'\footnote{\textbf{10:36}
  Haciendo referencia a Miqueas 7:6.} \footnote{\textbf{10:36} Miq 7,6}
\bibleverse{37} Si ustedes aman a su padre o su madre más que a mi, no
merecen ser míos; y si aman a su hijo o hija más que a mi, no merecen
ser míos. \footnote{\textbf{10:37} Deut 13,7-12; Deut 33,9; Luc 14,26-27}
\bibleverse{38} Si no cargan su cruz y me siguen, no merecen ser míos.
\footnote{\textbf{10:38} Mat 16,24-25} \bibleverse{39} Si tratan de
salvar su vida, la perderán,\footnote{\textbf{10:39} En otras palabras,
  si tratas de aferrarte a la vida por medio de tus propios esfuerzos
  humanos, no lo lograrás.} pero si pierden su vida por causa de mí, la
salvarán. \footnote{\textbf{10:39} Luc 9,24; Juan 12,25}

\hypertarget{promesas-de-ayuda-fraternal}{%
\subsection{Promesas de ayuda
fraternal}\label{promesas-de-ayuda-fraternal}}

\bibleverse{40} Aquellos que los reciban a ustedes me reciben a mi, y
aquellos que me reciben a mi, reciben al que me envió. \footnote{\textbf{10:40}
  Luc 9,48; Juan 13,20; Gal 4,14} \bibleverse{41} Aquellos que reciben
al profeta por ser profeta, recibirán recompensa de un profeta. Los que
reciben a quien hace el bien, recibirán la misma recompensa como quien
hace el bien. \footnote{\textbf{10:41} 1Re 17,9-24} \bibleverse{42} Les
digo la verdad: los que den una bebida de agua fresca al menos
importante de mis discípulos, no perderán su recompensa''.\footnote{\textbf{10:42}
  Mat 25,40; Mar 9,41}

\hypertarget{section-10}{%
\section{11}\label{section-10}}

\bibleverse{1} Cuando Jesús hubo terminado de darles instrucciones a sus
doce discípulos, se fue de allí para ir a enseñar y predicar
públicamente en las ciudades cercanas.

\hypertarget{embajada-de-juan-el-bautista-desde-la-prisiuxf3n-la-respuesta-y-el-testimonio-de-jesuxfas-sobre-juan}{%
\subsection{Embajada de Juan el Bautista desde la prisión; La respuesta
y el testimonio de Jesús sobre
Juan}\label{embajada-de-juan-el-bautista-desde-la-prisiuxf3n-la-respuesta-y-el-testimonio-de-jesuxfas-sobre-juan}}

\bibleverse{2} Estando Juan en prisión, escuchó sobre lo que el Mesías
estaba haciendo, así que envió a sus discípulos \bibleverse{3} para que
preguntaran en su nombre, ``¿Eres tú el que estábamos esperando, o
debemos seguir esperando a alguien más?'' \footnote{\textbf{11:3} Mat
  3,11; Mal 3,1}

\bibleverse{4} Jesús les respondió: ``Regresen y díganle a Juan lo que
ustedes oyen y lo que ven. \bibleverse{5} Los ciegos pueden ver, los
paralíticos pueden caminar, los leprosos son sanados, los sordos pueden
oír, los muertos han vuelto a vivir y los pobres escuchan la buena
noticia. \bibleverse{6} ¡Benditos son los que no me desprecian!''
\footnote{\textbf{11:6} Mat 13,57; Mat 26,31}

\bibleverse{7} Cuando los discípulos de Juan se fueron, Jesús comenzó a
hablarles a las multitudes sobre Juan. ``¿Qué esperaban ver cuando
salieron al desierto? ¿Una caña zarandeada por el viento? \footnote{\textbf{11:7}
  Mat 3,1; Mat 3,5} \bibleverse{8} ¿Entonces qué salieron a ver? ¿A un
hombre vestido con ropas finas? Las personas que visten así viven en los
palacios de los reyes. \bibleverse{9} ¿Qué salieron a ver, entonces? ¿A
un profeta? Sí, ¡Y les digo que él es mucho más que un profeta!
\footnote{\textbf{11:9} Luc 1,76; Luc 20,6} \bibleverse{10} Él es de
quien habló la Escritura: `Yo envío a mi mensajero por anticipado. Él
preparará el camino para ti'.\footnote{\textbf{11:10} Citando Malaquías
  3:1.} \bibleverse{11} Les digo la verdad, y es que entre la
humanidad,\footnote{\textbf{11:11} Literalmente, ``entre aquellos que
  son nacidos de mujer''.} no hay ninguno más grande que Juan el
Bautista, pero incluso el menos importante en el reino de los cielos es
más grande que él. \bibleverse{12} Desde los tiempos de Juan el Bautista
hasta ahora el reino de los cielos sigue estando bajo ataque y personas
violentas están tratando de apoderarse de él a la fuerza. \footnote{\textbf{11:12}
  Luc 16,16} \bibleverse{13} Pues todos los profetas y la ley\footnote{\textbf{11:13}
  Refiriéndose al mensaje del Antiguo Testamento.} hablaron por Dios
hasta que vino Juan. \bibleverse{14} Si ustedes están listos para
creerlo, él es Elías, el que debía venir.\footnote{\textbf{11:14} Véase
  Malaquías 4:5.} \bibleverse{15} ¡Todo el que tenga oídos, oiga!

\bibleverse{16} ``¿Con qué compararé esta generación? Son como unos
niños que están en la plaza del mercado y se gritan unos a otros
diciendo: \bibleverse{17} `tocamos la flauta para ustedes y no danzaron;
cantamos canciones tristes y no lloraron'. \footnote{\textbf{11:17} Prov
  29,9; Juan 5,35} \bibleverse{18} Juan no vino para festejar o beber,
entonces la gente dice: `él está endemoniado' \footnote{\textbf{11:18}
  Mat 3,4; Juan 10,20} \bibleverse{19} Pero el Hijo del hombre, por el
contrario, vino y festejó y bebió, y la gente dice: `¡Miren, es un
glotón y bebe mucho; es amigo de los recaudadores de impuestos y de los
pecadores!' Pero la sabiduría demuestra ser correcta por los resultados
de lo que hace''\ldots{} \footnote{\textbf{11:19} Mat 9,10-15; Juan 2,2;
  1Cor 1,24-30}

\hypertarget{lamento-de-jesuxfas-por-las-ciudades-galileas-impenitentes}{%
\subsection{Lamento de Jesús por las ciudades galileas
impenitentes}\label{lamento-de-jesuxfas-por-las-ciudades-galileas-impenitentes}}

\bibleverse{20} Entonces Jesús comenzó a reprender a las ciudades donde
había hecho muchos de sus milagros porque no se habían arrepentido.
\bibleverse{21} ``¡Qué vergüenza tienes, Korazin! ¡Qué vergüenza tienes,
Betsaida! Si los milagros que hice entre ustedes se hubieran hecho en
Tiro y Sidón, hace mucho tiempo ellos se habrían arrepentido en silicio
y cenizas. \bibleverse{22} ¡Pero les digo que el Día del Juicio será
mejor para Tiro y Sidón que para ustedes! \bibleverse{23} Y ¿qué decir
de ti, Capernaúm? ¿Serás exaltada hasta el cielo? No, ¡Tú irás al Hades!
Si los milagros que hice entre ustedes hubieran sido hechos en Sodoma,
aún hoy existiría Sodoma. \footnote{\textbf{11:23} Mat 4,13; Mat 8,5;
  Mat 9,1; Is 14,13; Is 14,15} \bibleverse{24} ¡Pero te digo que a
Sodoma le irá mejor en el Día del Juicio que a ti!'' \footnote{\textbf{11:24}
  Mat 10,15}

\hypertarget{el-juxfabilo-y-la-alabanza-de-jesuxfas-por-el-padre}{%
\subsection{El júbilo y la alabanza de Jesús por el
Padre}\label{el-juxfabilo-y-la-alabanza-de-jesuxfas-por-el-padre}}

\bibleverse{25} Entonces Jesús oró: ``Te alabo, Padre, Señor del cielo y
de la tierra, porque has ocultado estas cosas de las mentes de los
inteligentes y sabios. Por el contrario, las has revelado a personas
comunes.\footnote{\textbf{11:25} Literalmente, a ``infantes''.}
\footnote{\textbf{11:25} 1Cor 1,19-29; Is 29,14; Luc 10,21-22; Juan
  17,25} \bibleverse{26} ¡Sí, Padre, te complaciste en hacerlo así!
\bibleverse{27} El Padre lo ha confiado todo en mis manos, y ninguno
entiende verdaderamente al Hijo, excepto el Padre, y nadie entiende
verdaderamente al Padre, excepto el Hijo, y aquellos a quienes el Hijo
elige para mostrarles al Padre.

\hypertarget{el-llamado-del-salvador-a-los-cansados-y-agobiados}{%
\subsection{El llamado del Salvador a los cansados
\hspace{0pt}\hspace{0pt}y
agobiados}\label{el-llamado-del-salvador-a-los-cansados-y-agobiados}}

\bibleverse{28} Vengan a mí todos ustedes que luchan y están cargados.
Yo les daré descanso. \footnote{\textbf{11:28} Mat 12,20; Mat 23,4; Jer
  31,25} \bibleverse{29} Acepten mi yugo, y aprendan de mí. Porque yo
soy bondadoso y tengo un corazón humilde, y en mí encontrarán el
descanso que necesitan. \footnote{\textbf{11:29} Is 28,12; Jer 6,16; Zac
  9,9} \bibleverse{30} Pues mi yugo es suave, y mi carga es
ligera''.\footnote{\textbf{11:30} Luc 11,46; 1Jn 5,3}

\hypertarget{los-ouxeddos-de-los-discuxedpulos-en-suxe1bado-la-primera-disputa-de-jesuxfas-con-los-fariseos-sobre-la-santificaciuxf3n-del-duxeda-de-reposo}{%
\subsection{Los oídos de los discípulos en sábado; La primera disputa de
Jesús con los fariseos sobre la santificación del día de
reposo}\label{los-ouxeddos-de-los-discuxedpulos-en-suxe1bado-la-primera-disputa-de-jesuxfas-con-los-fariseos-sobre-la-santificaciuxf3n-del-duxeda-de-reposo}}

\hypertarget{section-11}{%
\section{12}\label{section-11}}

\bibleverse{1} En esos días, Jesús caminaba por los campos de grano en
el día Sábado. Sus discípulos tenían hambre, así que comenzaron a
recoger espigas y a comérselas. \footnote{\textbf{12:1} Deut 23,26}
\bibleverse{2} Cuando los Fariseos vieron esto, le dijeron a Jesús:
``¡Mira a tus discípulos! ¡Están haciendo lo que no se debe hacer en
Sábado!'' \footnote{\textbf{12:2} Éxod 20,10}

\bibleverse{3} Pero Jesús les dijo: ``¿No han leído lo que hizo David
cuando él y sus hombres tuvieron hambre? \footnote{\textbf{12:3} 1Sam
  21,7} \bibleverse{4} Él entró a la casa de Dios, y allí él y sus
hombres comieron del pan sagrado que no debían comer pues este pan
estaba reservado solo para los sacerdotes. \footnote{\textbf{12:4} Lev
  24,9} \bibleverse{5} ¿No han leído en la ley que los sacerdotes que
están en el Templo quebrantan el sábado pero no son considerados como
culpables? \footnote{\textbf{12:5} Núm 28,9} \bibleverse{6} Sin embargo
yo les digo a ustedes: ¡Aquí hay alguien que es aún más grande que el
Templo! \bibleverse{7} Si ustedes conocieran el significado de lo que
dice la Escritura: `misericordia quiero y no sacrificio',\footnote{\textbf{12:7}
  Citando Oseas 6:6.} no habrían condenado a un hombre inocente.
\footnote{\textbf{12:7} Mat 9,13} \bibleverse{8} Porque el Hijo del
hombre es Señor del Sábado''.

\hypertarget{sanaciuxf3n-del-hombre-con-el-brazo-paralizado-en-suxe1bado-el-segundo-argumento-sobre-la-observancia-del-suxe1bado}{%
\subsection{Sanación del hombre con el brazo paralizado en sábado; el
segundo argumento sobre la observancia del
sábado}\label{sanaciuxf3n-del-hombre-con-el-brazo-paralizado-en-suxe1bado-el-segundo-argumento-sobre-la-observancia-del-suxe1bado}}

\bibleverse{9} Entonces Jesús se fue de allí y entró a la sinagoga de
ellos. \bibleverse{10} Allí había un hombre que tenía la mano tullida.
``¿Acaso permite la ley sanar en Sábado?'' le preguntaron ellos,
buscando así un motivo para acusarlo.

\bibleverse{11} ``Supongan que tienen una oveja y ésta se cae en un
hueco, en Sábado. ¿Acaso no la agarran y tratan de sacarla?'' les
preguntó Jesús. \bibleverse{12} ``¿No creen ustedes que un ser humano
vale mucho más que una oveja? De modo que sí, es permitido hacer el bien
en Sábado''. \bibleverse{13} Entonces le dijo al hombre: ``Extiende tu
mano''. El hombre entonces extendió su mano y fue sanada, quedando como
la otra mano que estaba sana. \bibleverse{14} Pero los Fariseos salieron
y conspiraban sobre cómo matar a Jesús. \footnote{\textbf{12:14} Juan
  5,16}

\hypertarget{jesuxfas-evade-la-persecuciuxf3n-su-actividad-sanadora-piadosa}{%
\subsection{Jesús evade la persecución; su actividad sanadora
piadosa}\label{jesuxfas-evade-la-persecuciuxf3n-su-actividad-sanadora-piadosa}}

\bibleverse{15} Sabiendo esto, Jesús salió de allí, con una multitud que
le seguía. Y Jesús los sanaba a todos, \bibleverse{16} pero les decía
que no dijeran quién era él. \bibleverse{17} Esto cumplió lo que dijo el
profeta Isaías: \bibleverse{18} ``Este es mi siervo a quien Yo he
escogido, Mi siervo a quien amo, el cual me complace. Yo pondré mi
Espíritu sobre él, Y él le dirá a los extranjeros lo que es correcto.
\footnote{\textbf{12:18} Mat 3,17; Hech 3,13; Hech 3,26} \bibleverse{19}
Él no peleará, no gritará, Y ninguno oirá su voz por las calles.
\bibleverse{20} Él no quebrará ni una caña dañada, Y no apagará una
mecha que titila, Hasta que haya demostrado que su juicio es
justo,\footnote{\textbf{12:20} O ``haya dado la victoria a la
  justicia''.} \bibleverse{21} Y los gentiles pondrán su confianza en
él''.\footnote{\textbf{12:21} Literalmente, ``esperanza en su nombre''.}

\hypertarget{jesuxfas-se-defiende-de-la-blasfemia-de-los-fariseos-contra-beelzebul}{%
\subsection{Jesús se defiende de la blasfemia de los fariseos contra
Beelzebul}\label{jesuxfas-se-defiende-de-la-blasfemia-de-los-fariseos-contra-beelzebul}}

\bibleverse{22} Entonces trajeron delante de Jesús a un hombre que
estaba endemoniado, ciego y mudo. Jesús lo sanó, y el hombre mudo pudo
hablar y ver. \bibleverse{23} Todas las multitudes estaban asombradas, y
preguntaban, ``¿Será que este es el hijo de David?''\footnote{\textbf{12:23}
  Queriendo decir, el Mesías que vendría.} \bibleverse{24} Pero cuando
los Fariseos escucharon esto, respondieron: ``¡Este hombre solo puede
echar fuera demonios mediante el poder de Belcebú, el jefe de los
demonios!''

\bibleverse{25} Pero sabiendo lo que ellos estaban pensando, Jesús les
dijo: ``Cualquier reino que está dividido contra sí mismo, será
destruido. Ninguna ciudad que está dividida contra sí misma puede
permanecer. \bibleverse{26} Si Satanás echa fuera a Satanás, entonces
está dividido contra sí mismo, ¿cómo podría entonces permanecer su
reino? \bibleverse{27} Si yo estoy echando fuera los demonios en el
nombre de Belcebú, entonces, ¿en nombre de quién echan fuera demonios
los exorcistas de ustedes? ¡Ellos mismos son prueba de que ustedes están
equivocados! \bibleverse{28} ¡Pero si yo echo fuera demonios mediante el
poder del Espíritu de Dios, entonces el reino de Dios ha venido a
ustedes! \footnote{\textbf{12:28} 1Jn 3,8} \bibleverse{29} ``¿Puede
alguien entrar a la casa de un hombre fuerte y robar sus pertenencias
sin atarlo primero? Si haces esto, entonces puedes robar todo lo que hay
en su casa. \footnote{\textbf{12:29} Mat 4,1-11; Is 49,24}

\bibleverse{30} Los que no están conmigo, están contra mí, y los que no
se reúnen conmigo hacen lo contrario: están dispersos. \footnote{\textbf{12:30}
  Mar 9,40}

\hypertarget{advertencia-de-la-blasfemia-del-espuxedritu-del-uxe1rbol-y-los-frutos}{%
\subsection{Advertencia de la blasfemia del espíritu; del árbol y los
frutos}\label{advertencia-de-la-blasfemia-del-espuxedritu-del-uxe1rbol-y-los-frutos}}

\bibleverse{31} Esa es la razón por la que les digo que cada pecado y
blasfemia que ustedes cometan será perdonada, excepto la blasfemia
contra el Espíritu Santo, la cual no será perdonada. \footnote{\textbf{12:31}
  Heb 6,4-6; Heb 10,26; 1Jn 5,16} \bibleverse{32} Aquellos que digan
algo en contra del Hijo del hombre serán perdonados, pero aquellos que
digan algo contra el Espíritu Santo no serán perdonados, ni en esta vida
ni en la siguiente. \footnote{\textbf{12:32} 1Tim 1,13}

\bibleverse{33} Un árbol bueno se conoce porque su fruto es bueno, y un
árbol malo se conoce porque su fruto es malo, pues un árbol se conoce
por su fruto. \footnote{\textbf{12:33} Mat 7,17} \bibleverse{34} ¡Cría
de víboras! ¿Cómo pueden ustedes decir algo bueno siendo malos? Pues la
boca de ustedes solo dice lo que pasa por sus mentes. \footnote{\textbf{12:34}
  Mat 3,7} \bibleverse{35} Una buena persona saca cosas buenas de las
cosas buenas que tiene guardadas, y una persona mala saca cosas malas de
las cosas malas que tiene guardadas. \bibleverse{36} Yo les digo,
ustedes tendrán que dar cuenta en el Día del Juicio de cada cosa que
hayan dicho de manera descuidada. \bibleverse{37} Porque lo que ustedes
digan los vindicará o los condenará''.

\hypertarget{el-rechazo-de-jesuxfas-a-la-demanda-de-seuxf1ales-la-seuxf1al-de-jonuxe1s-la-paruxe1bola-de-la-recauxedda}{%
\subsection{El rechazo de Jesús a la demanda de señales; la señal de
Jonás; la parábola de la
recaída}\label{el-rechazo-de-jesuxfas-a-la-demanda-de-seuxf1ales-la-seuxf1al-de-jonuxe1s-la-paruxe1bola-de-la-recauxedda}}

\bibleverse{38} Entonces algunos de los maestros religiosos y Fariseos
que estaban allí le dijeron: ``Maestro, queremos que nos muestres una
señal milagrosa''.

\bibleverse{39} ``Las personas malvadas que no creen en Dios son las que
buscan una señal milagrosa. A esas personas no se les dará ninguna señal
sino la señal del profeta Jonás'', les dijo Jesús. \bibleverse{40} ``De
la misma manera que Jonás estuvo en el vientre de un gran pez durante
tres días y tres noches, el Hijo del hombre estará en el corazón de la
tierra por tres días y tres noches. \footnote{\textbf{12:40} Jon 2,1-2;
  Efes 4,9; 1Pe 3,19} \bibleverse{41} El pueblo de Nínive se levantará
en el juicio junto con esta generación y la condenarán, porque ellos se
arrepintieron como respuesta al mensaje de Jonás--- ¡Y como pueden ver,
aquí hay alguien más grande que Jonás! \footnote{\textbf{12:41} Jon 3,5}
\bibleverse{42} La reina del Sur se levantará en el juicio junto con
esta generación y la condenará, porque ella vino desde los fines de la
tierra para escuchar la sabiduría de Salomón--- ¡Y como pueden ver, aquí
hay alguien más grande que Salomón! \footnote{\textbf{12:42} 1Re 10,1-10}

\bibleverse{43} Cuando un espíritu maligno sale de una persona, anda por
lugares desiertos buscando descanso, y no encuentra dónde quedarse.
\bibleverse{44} Entonces dice: `regresaré al lugar de donde salí,' y
cuando regresa, encuentra el lugar vacío, limpio y organizado.
\bibleverse{45} Entonces va y trae consigo otros siete espíritus mucho
peores que él, y entra y vive allí. De modo que entonces la persona
termina siendo peor de lo que era al comienzo. Así sucederá con esta
generación malvada''.

\hypertarget{los-verdaderos-parientes-de-jesuxfas}{%
\subsection{Los verdaderos parientes de
Jesús}\label{los-verdaderos-parientes-de-jesuxfas}}

\bibleverse{46} Mientras Jesús hablaba a las multitudes, su madre y sus
hermanos llegaron y lo esperaban fuera, y querían hablar con él.
\footnote{\textbf{12:46} Mat 13,55} \bibleverse{47} Entonces alguien
vino y le dijo: ``mira, tu madre y tus hermanos están afuera y quieren
hablar contigo''.

\bibleverse{48} ``¿Quién es mi madre? ¿Quiénes son mis hermanos?''
preguntó Jesús. \bibleverse{49} Entonces Jesús señaló a sus discípulos y
dijo: ``¡Miren, ellos son mi madre y mis hermanos! \footnote{\textbf{12:49}
  Heb 2,11} \bibleverse{50} Porque los que hacen la voluntad de mi Padre
celestial, ¡ellos son mi hermano, mi hermana y mi madre!''\footnote{\textbf{12:50}
  Rom 8,29}

\hypertarget{la-paruxe1bola-del-sembrador-y-el-campo-cuuxe1druple}{%
\subsection{la parábola del sembrador y el campo
cuádruple}\label{la-paruxe1bola-del-sembrador-y-el-campo-cuuxe1druple}}

\hypertarget{section-12}{%
\section{13}\label{section-12}}

\bibleverse{1} Más tarde, ese día, Jesús se fue de la casa y se sentó a
enseñar\footnote{\textbf{13:1} Está implícito. Los maestros religiosos
  se sentaban cuando querían instruir a sus discípulos.} junto al lago.
\bibleverse{2} Pero se reunieron a su alrededor tantas personas, que
tuvo que subirse a una barca y allí se sentó a enseñar, mientras que
todas las multitudes se quedaron de pie en la playa. \bibleverse{3} Él
les enseñaba muchas cosas, usando relatos para ilustrarlas.\footnote{\textbf{13:3}
  ``Relatos en forma de ilustraciones'', literalmente, ``parábolas''.}
``El sembrador salió a sembrar'', comenzó a decir. \bibleverse{4}
``Mientras sembraba, algunas de las semillas cayeron por el camino.
Entonces las aves vinieron y se las comieron. \bibleverse{5} Otras
semillas cayeron en suelo rocoso y porque no habia mucha tierra,
germinaron pronto''. \bibleverse{6} El sol salió y las chamuscó y se
murieron porque no tenían raíces. \bibleverse{7} Otras semillas cayeron
entre espinos que crecieron y las sofocaron. \bibleverse{8} No obstante,
otras semillas cayeron en buen suelo. Esas semillas produjeron una
cosecha---algunas cien, otras sesenta, y otras treinta veces lo que se
había plantado. \bibleverse{9} ¡Todo el que tenga oídos, escuche!

\hypertarget{explicaciuxf3n-de-jesuxfas-de-la-razuxf3n-y-el-propuxf3sito-de-sus-paruxe1bolas}{%
\subsection{Explicación de Jesús de la razón y el propósito de sus
parábolas}\label{explicaciuxf3n-de-jesuxfas-de-la-razuxf3n-y-el-propuxf3sito-de-sus-paruxe1bolas}}

\bibleverse{10} Los discípulos vinieron a Jesús y le preguntaron, ``¿Por
qué usas ilustraciones cuando hablas a la gente?''

\bibleverse{11} ``Ustedes son privilegiados porque a ustedes se les han
revelado los misterios del reino de los cielos, pero ellos no tienen ese
conocimiento'', respondió Jesús. \footnote{\textbf{13:11} 1Cor 2,10}
\bibleverse{12} ``Aquellos que ya tienen\footnote{\textbf{13:12}
  Probablemente queriendo decir que ``tienen entendimiento''.} recibirán
más, más que suficiente. Pero aquellos que no tienen, lo que lleguen a
tener se les quitará. \footnote{\textbf{13:12} Mat 25,28-29; Mar 4,25;
  Luc 8,18} \bibleverse{13} Esa es la razón por la que les hablo a ellos
a través de ilustraciones. Porque aunque ellos pueden ver, no ven; y
aunque pueden oír, no oyen; ni entienden tampoco. \footnote{\textbf{13:13}
  Deut 29,3; Juan 16,25} \bibleverse{14} ``La profecía de Isaías se
cumple en ellos: `aunque ustedes oigan, no entenderán, y aunque vean, no
percibirán. \bibleverse{15} Ellos tienen un corazón duro, no quieren
escuchar y han cerrado sus ojos. Si no fuera así, entonces podrían ver
con sus ojos, oír con sus oídos y entender con sus mentes. Entonces
podrían regresar a mí y yo los sanaría'.\footnote{\textbf{13:15} Citando
  Isaías 6:9-10.}

\bibleverse{16} ``Benditos los ojos de ustedes, porque pueden ver.
También sus oídos, porque pueden oír. \footnote{\textbf{13:16} Luc
  10,23-24} \bibleverse{17} Les digo que muchos profetas y personas
buenas anhelaron ver lo que ustedes están viendo ahora, pero no lo
vieron. Ellos anhelaban escuchar lo que ustedes están escuchando, pero
no lo escucharon. \footnote{\textbf{13:17} 1Pe 1,10}

\hypertarget{interpretaciuxf3n-de-la-paruxe1bola-del-sembrador}{%
\subsection{Interpretación de la parábola del
sembrador}\label{interpretaciuxf3n-de-la-paruxe1bola-del-sembrador}}

\bibleverse{18} ``Así que escuchen el relato del sembrador:
\bibleverse{19} Cuando las personas oyen el mensaje del reino, y no lo
entienden, el maligno viene y arranca lo que fue sembrado en sus
corazones. Esto es lo que ocurre con las semillas que cayeron en el
camino. \bibleverse{20} Las semillas sembradas en el suelo rocoso son
las personas que escuchan el mensaje e inmediatamente lo aceptan con
alegría. \bibleverse{21} De esta manera permanecen por un tiempo, pero
como no tienen raíces, cuando los problemas llegan, se apartan
rápidamente. \bibleverse{22} Las semillas que fueron sembradas entre los
espinos son las personas que escuchan el mensaje, pero luego las
preocupaciones de la vida y la tentación por el dinero ahogan el mensaje
y éste no produce fruto. \footnote{\textbf{13:22} Mat 6,19-34; 1Tim 6,9}
\bibleverse{23} Las semillas sembradas en buen suelo son las personas
que escuchan el mensaje, lo entienden, y producen buena
cosecha---algunos cien, otros sesenta, y otros treinta veces lo que fue
sembrado''.

\hypertarget{la-paruxe1bola-de-la-cizauxf1a-entre-el-trigo}{%
\subsection{La parábola de la cizaña entre el
trigo}\label{la-paruxe1bola-de-la-cizauxf1a-entre-el-trigo}}

\bibleverse{24} Entonces les contó otro relato ilustrado: ``El reino de
los cielos es como un granjero que sembró buena semilla en su campo.
\bibleverse{25} Pero mientras sus trabajadores dormían, llegó un enemigo
y sembró maleza\footnote{\textbf{13:25} De hecho, se refiere a
  ``cizaña'', o ``trigo falso'', una maleza que se parecía mucho al
  trigo.} encima del trigo. Y se fueron. \bibleverse{26} Cuando el trigo
creció y produjo espigas, la maleza también creció. \bibleverse{27} Los
trabajadores del granjero vinieron a preguntarle: `Señor, ¿acaso no
sembraste buena semilla en tu campo? ¿De dónde salió esta maleza?'

\bibleverse{28} ```Algún enemigo hizo esto', respondió él. `¿Quieres que
vayamos y arranquemos la maleza?' le preguntaron.

\bibleverse{29} `No,' respondió él, `al arrancar la maleza podrían
arrancar de raíz el trigo también. \bibleverse{30} Dejen que crezcan
juntos hasta la cosecha, y entonces le diré a los segadores: reúnan
primero la maleza, átenla en bultos y quémenlos. Luego reúnan el trigo y
almacénenlo en mi granero'''.

\hypertarget{las-dos-paruxe1bolas-de-la-semilla-de-mostaza-y-la-levadura}{%
\subsection{Las dos parábolas de la semilla de mostaza y la
levadura}\label{las-dos-paruxe1bolas-de-la-semilla-de-mostaza-y-la-levadura}}

\bibleverse{31} Les dio otra ilustración: ``El reino de los cielos es
como una semilla de mostaza que sembró un granjero en su campo.
\bibleverse{32} Aunque es la semilla más pequeña de todas, ésta crece y
llega a ser mucho más grande que las otras plantas. De hecho, se
convierte en un árbol tan grande, que las aves pueden posarse en sus
ramas''. \footnote{\textbf{13:32} Ezeq 17,23}

\bibleverse{33} Y les contó otro relato ilustrado: ``El reino de los
cielos es como la levadura que una mujer mezcló con una gran cantidad
de\footnote{\textbf{13:33} Aproximadamente, 50 libras, o 23 kilogramos.}
harina, hasta que toda la masa creció''. \footnote{\textbf{13:33} Luc
  13,20-21}

\hypertarget{interpretaciuxf3n-de-la-paruxe1bola-de-la-cizauxf1a-del-trigo}{%
\subsection{Interpretación de la parábola de la cizaña del
trigo}\label{interpretaciuxf3n-de-la-paruxe1bola-de-la-cizauxf1a-del-trigo}}

\bibleverse{34} Y Jesús le enseñaba todas estas cosas a las multitudes
por medio de relatos ilustrados---de hecho, él no les hablaba sin usar
relatos. \footnote{\textbf{13:34} Mar 4,33-34} \bibleverse{35} Esto
cumplía las palabras del profeta: ``Hablaré por medio de relatos, y
enseñaré cosas ocultas desde la creación del mundo''.\footnote{\textbf{13:35}
  Citando Salmos 78:2.}

\bibleverse{36} Jesús se fue de donde estaba la multitude a una casa.
Sus discípulos vinieron donde él estaba y le dijeron: ``Por favor,
explícanos el relato de la maleza en el campo''.

\bibleverse{37} ``El que siembra la buena semilla es el Hijo del
hombre'', les explicó Jesús. \bibleverse{38} ``El campo es el mundo. Las
semillas buenas son los hijos del reino. Las semillas de maleza son los
hijos del maligno. \bibleverse{39} El enemigo que sembró las semillas de
maleza es el diablo. La cosecha es el fin del mundo. Los segadores son
ángeles. \bibleverse{40} Así como la maleza se recoge y se quema, así
será en el fin del mundo. \bibleverse{41} El Hijo del hombre enviará
ángeles, y ellos recogerán todo lo que es pecaminoso y a todos los que
hacen el mal, \footnote{\textbf{13:41} Mat 25,31-46} \bibleverse{42} y
los lanzarán en el horno abrasador, donde habrá llanto y crujir de
dientes. \bibleverse{43} Entonces aquellos que viven justamente
brillarán como el sol en el reino de su padre.\footnote{\textbf{13:43}
  Ver Daniel 12:3.} ¡Todo el que tiene oídos, oiga!

\hypertarget{las-uxfaltimas-tres-paruxe1bolas-tesoro-en-el-campo-perla-preciosa-red-de-pesca-conclusiuxf3n-de-la-paruxe1bola}{%
\subsection{Las últimas tres parábolas (tesoro en el campo; perla
preciosa; red de pesca); Conclusión de la
parábola}\label{las-uxfaltimas-tres-paruxe1bolas-tesoro-en-el-campo-perla-preciosa-red-de-pesca-conclusiuxf3n-de-la-paruxe1bola}}

\bibleverse{44} ``El reino de los cielos es como un tesoro escondido en
un campo. Un hombre lo encontró, lo volvió a enterrar, y lleno de
alegría se fue y vendió todo lo que tenía y entonces compró ese campo.
\footnote{\textbf{13:44} Mat 19,29; Luc 14,33; Fil 3,7}

\bibleverse{45} El reino de los cielos es también como un mercader que
busca perlas preciosas. \bibleverse{46} Cuando encontró la perla más
costosa que alguna vez conociera, se fue y vendió todo lo que tenía y la
compró.

\bibleverse{47} Una vez más, el reino de los cielos es como una red de
pescar que fue lanzada al mar y atrapó todo tipo de peces.
\bibleverse{48} Cuando estaba llena, fue sacada a la orilla. Los buenos
peces fueron colocados en las canastas, mientras que los malos peces
fueron echados a la basura. \bibleverse{49} ``Así serán las cosas cuando
llegue el fin del mundo. Los ángeles saldrán y separarán a las personas
malas de las personas buenas, \footnote{\textbf{13:49} Mat 25,32}
\bibleverse{50} y las lanzarán en el horno abrasador, donde habrá llanto
y crujir de dientes. \bibleverse{51} ``¿Ahora lo entienden todo?''
``Sí'', respondieron ellos.

\bibleverse{52} ``Todo maestro religioso que haya aprendido acerca del
reino de los cielos es como el propietario de una casa que saca de su
despensa tesoros nuevos y viejos'', respondió Jesús.

\hypertarget{rechazo-y-fracaso-de-jesuxfas-en-su-natal-nazaret}{%
\subsection{Rechazo y fracaso de Jesús en su natal
Nazaret}\label{rechazo-y-fracaso-de-jesuxfas-en-su-natal-nazaret}}

\bibleverse{53} Después que Jesús terminó de contar estos relatos, se
fue de allí. \bibleverse{54} Entonces regresó a la ciudad donde se había
criado\footnote{\textbf{13:54} Nazaret.} y allí enseñaba en la sinagoga.
Las personas estaban asombradas, y preguntaban: ``¿De dónde obtiene su
sabiduría y sus milagros? \bibleverse{55} ¿No es este el hijo del
carpintero? ¿No es este el hijo de María, y hermano de Santiago, José,
Simón y Judas? \bibleverse{56} ¿No viven sus hermanas entre nosotros?
¿De dónde, entonces recibe todo esto?'' \bibleverse{57} Y por esta razón
se negaban a creer en él. ``Un profeta es honrado en todas partes,
excepto en su propia tierra y entre su familia'', les dijo Jesús.
\footnote{\textbf{13:57} Juan 4,44}

\bibleverse{58} Como ellos no lograron creer en él, Jesús no hizo muchos
milagros allí.

\hypertarget{jesuxfas-y-herodes-el-fin-de-juan-el-bautista}{%
\subsection{Jesús y Herodes; el fin de Juan el
Bautista}\label{jesuxfas-y-herodes-el-fin-de-juan-el-bautista}}

\hypertarget{section-13}{%
\section{14}\label{section-13}}

\bibleverse{1} En ese tiempo, Herodes el tetrarca\footnote{\textbf{14:1}
  ``Tetrarca'' quiere decir que era gobernante de una cuarta parte. En
  este caso, de la región de Galilea.} escuchó lo que Jesús hacía
\bibleverse{2} y le dijo a sus siervos: ``¡Él debe ser Juan el Bautista
que resucitó de entre los muertos, y por eso tiene tales poderes!''
\bibleverse{3} Herodes había arrestado a Juan, lo había encadenado y lo
había puesto en prisión por petición de Herodías, la esposa de Felipe,
su hermano. \bibleverse{4} Esto lo hicieron porque Juan le había dicho:
``No es legal que te cases con ella''. \footnote{\textbf{14:4} Mat 19,9;
  Lev 18,16} \bibleverse{5} Herodes quería matar a Juan pero tenía miedo
de la reacción del pueblo, pues ellos consideraban que él era un
profeta. \footnote{\textbf{14:5} Mat 21,26} \bibleverse{6} Sin embargo,
el día del cumpleaños de Herodes, la hija de Herodías\footnote{\textbf{14:6}
  Comúnmente se le identifica como Salomé.} danzó en la fiesta, y
Herodes estaba contento con ella. \bibleverse{7} Así que le prometió con
juramento darle cualquier cosa que ella deseara. \bibleverse{8}
Impulsada por su madre, Herodías dijo: ``Dame aquí en un plato la cabeza
de Juan el Bautista''.

\bibleverse{9} Entonces el rey se arrepintió de la promesa que había
hecho, pero por los juramentos que había hecho frente a todos los
invitados a su cena, dio la orden de hacerlo. \bibleverse{10} La orden
fue enviada y Juan fue decapitado en la cárcel. \bibleverse{11} Trajeron
la cabeza de Juan en un plato y le fue entregado a la joven, quien lo
entregó a su madre. \bibleverse{12} Entonces los discípulos de Juan
vinieron y se llevaron el cuerpo y lo sepultaron. Luego fueron a
decírselo a Jesús.

\hypertarget{alimentando-a-los-cinco-mil}{%
\subsection{Alimentando a los cinco
mil}\label{alimentando-a-los-cinco-mil}}

\bibleverse{13} Cuando Jesús escuchó la noticia, se fue lejos en una
barca a un lugar tranquilo para estar solo, pero cuando la multitud supo
dónde estaba, lo siguieron a pie desde las ciudades.

\bibleverse{14} Cuando Jesús salió de la barca y vio a la gran multitud,
se llenó de simpatía por ellos, y sanó a los enfermos que había entre
ellos. \bibleverse{15} Al llegar la noche, los discípulos se le
acercaron y le dijeron, ``Este lugar está a millas de distancia de
cualquier parte y se está haciendo tarde. Despide la multitud para que
puedan irse a las aldeas y comprar comida para ellos''.

\bibleverse{16} Pero Jesús les dijo: ``Ellos no necesitan irse. ¡Denles
ustedes de comer!''

\bibleverse{17} ``Lo único que tenemos son cinco panes y un par de
peces'', respondieron ellos.

\bibleverse{18} ``Tráiganmelos'', dijo Jesús. \bibleverse{19} Entonces
les dijo a las multitudes que se sentaran en la hierba. Luego tomó los
cinco panes y los dos peces, miró al cielo y los bendijo. Después de
esto, partió los panes en pedazos y dio el pan a los discípulos, y los
discípulos lo daban a las multitudes. \bibleverse{20} Todos comieron
hasta que quedaron saciados. Entonces los discípulos recogieron las
sobras y llenaron doce canastas. \footnote{\textbf{14:20} 2Re 4,44}
\bibleverse{21} Aproximadamente cinco mil hombres comieron de aquella
comida, sin contar las mujeres y los niños.

\hypertarget{regreso-de-los-discuxedpulos-al-otro-lado-del-lago-por-la-noche-el-caminar-de-jesuxfas-sobre-el-lago-el-desembarco-en-gennesaret}{%
\subsection{Regreso de los discípulos al otro lado del lago por la
noche; el caminar de Jesús sobre el lago; el desembarco en
Gennesaret}\label{regreso-de-los-discuxedpulos-al-otro-lado-del-lago-por-la-noche-el-caminar-de-jesuxfas-sobre-el-lago-el-desembarco-en-gennesaret}}

\bibleverse{22} Justo después de esto, Jesús llamó a los discípulos a
que subieran a la barca para cruzar al otro lado del lago, mientras
despedía a la multitud. \bibleverse{23} Después que los despidió a
todos, subió a la montaña para orar. Llegó la noche y él estaba allí
solo. \bibleverse{24} En ese momento, ya la barca estaba lejos del suelo
firme, las olas la arrastraban porque el viento soplaba contra ella.
\bibleverse{25} Cerca de las 3 a. m.\footnote{\textbf{14:25}
  Literalmente, ``la cuarta vigilia de la noche''.} Jesús los alcanzó,
caminando sobre el mar. \bibleverse{26} Cuando los discípulos lo vieron
caminando sobre el mar, se asustaron. Entonces gritaron con terror:
``¡Es un fantasma!'' \footnote{\textbf{14:26} Luc 24,37} \bibleverse{27}
Pero inmediatamente Jesús les dijo: ``¡No se preocupen, soy yo! ¡No
tengan miedo!''

\bibleverse{28} ``Señor, si eres tú realmente, haz que yo llegue donde
tu estás, caminando también sobre el agua'', respondió Pedro.

\bibleverse{29} ``Ven'', le dijo Jesús. Entonces Pedro salió de la barca
y caminó sobre el agua hacia Jesús.

\bibleverse{30} Pero cuando vio cuán fuerte soplaba el viento, se asustó
y comenzó a hundirse. ``¡Señor! ¡Sálvame!'', gritaba.

\bibleverse{31} De inmediato Jesús se extendió y lo tomó, y le dijo:
``Tienes tan poca confianza en mi. ¿Por qué dudaste?'' \bibleverse{32} Y
cuando entraron a la barca, el viento dejó de soplar. \bibleverse{33} Y
los que estaban en la barca lo adoraban, diciendo: ``¡Realmente eres el
Hijo de Dios!''

\hypertarget{la-reuniuxf3n-de-personas-y-la-curaciuxf3n-de-los-enfermos-en-gennesaret}{%
\subsection{La reunión de personas y la curación de los enfermos en
Gennesaret}\label{la-reuniuxf3n-de-personas-y-la-curaciuxf3n-de-los-enfermos-en-gennesaret}}

\bibleverse{34} Después de cruzar el lago, llegaron a Genesaret.
\bibleverse{35} Cuando la gente de allí se dio cuenta de que era Jesús,
lo hicieron saber a todos en la región. Entonces trajeron ante Jesús a
todos los que estaban enfermos, \bibleverse{36} y le imploraban que
dejara que los enfermos tan solo tocasen su manto. Todos los que lo
tocaban eran sanados.\footnote{\textbf{14:36} Mat 9,21; Luc 6,19}

\hypertarget{la-disputa-de-jesuxfas-con-sus-oponentes-por-lavarse-las-manos-su-advertencia-de-los-estatutos-humanos-y-la-marca-de-la-verdadera-impureza}{%
\subsection{La disputa de Jesús con sus oponentes por lavarse las manos;
su advertencia de los estatutos humanos y la marca de la verdadera
impureza}\label{la-disputa-de-jesuxfas-con-sus-oponentes-por-lavarse-las-manos-su-advertencia-de-los-estatutos-humanos-y-la-marca-de-la-verdadera-impureza}}

\hypertarget{section-14}{%
\section{15}\label{section-14}}

\bibleverse{1} Entonces algunos Fariseos y maestros religiosos de
Jerusalén vinieron a Jesús y le preguntaron: \bibleverse{2} ``¿Por qué
tus discípulos quebrantan la tradición de nuestros antepasados al no
lavar sus manos antes de comer?''

\bibleverse{3} ``¿Por qué ustedes quebrantan el mandamiento por causa de
su tradición?'' respondió Jesús. \bibleverse{4} ``Pues Dios dijo: `Honra
a tu padre y a tu madre',\footnote{\textbf{15:4} Citando Éxodo 20:12 o
  Deuteronomio 5:16.} y `Cualquiera que maldice a su padre o a su madre
debe ser condenado a muerte'.\footnote{\textbf{15:4} Citando Éxodo 21:17
  o Levítico 20:9.} \bibleverse{5} Pero ustedes dicen que si alguno le
dice su padre o a su madre `todo lo que yo deba darles a ustedes ahora
lo doy como ofrenda a Dios,' entonces \footnote{\textbf{15:5} Prov 28,24}
\bibleverse{6} no tiene que honrar a su padre. De esta manera ustedes
han anulado la palabra de Dios por causa de sus tradiciones. \footnote{\textbf{15:6}
  1Tim 5,8} \bibleverse{7} ¡Ustedes son unos hipócritas! Bien los
describió Isaías cuando profetizó: \bibleverse{8} `Este pueblo dice que
me honra pero en sus mentes no hay interés hacia mí.\footnote{\textbf{15:8}
  O, ``Esas personas me honran con sus labios, pero sus corazones están
  lejos de mi''.} \bibleverse{9} Su adoración hacia mi es inútil. Lo que
enseñan son solo exigencias humanas'\,''.\footnote{\textbf{15:9} Citando
  Isaías 29:13.}

\bibleverse{10} Entonces Jesús llamó a la multitud y les dijo:
``Escuchen y entiendan esto: \bibleverse{11} No es lo que entra por la
boca lo que los contamina, sino lo que sale de ella''. \footnote{\textbf{15:11}
  Hech 10,15; 1Tim 4,4; Tit 1,15}

\bibleverse{12} Entonces los discípulos de Jesús vinieron a él y le
dijeron: ``Ciertamente te das cuenta de que los Fariseos se ofendieron
por lo que dijiste''.

\bibleverse{13} ``Toda planta que no haya sembrado mi Padre será
arrancada'', respondió Jesús. \bibleverse{14} ``Olvídense de
ellos---ellos son guías ciegos.\footnote{\textbf{15:14} Refiriéndose a
  los Fariseos.} Si un hombre ciego guía a otro hombre ciego, los dos
caerán en una zanja''. \footnote{\textbf{15:14} Mat 23,24; Luc 6,39; Rom
  2,19}

\bibleverse{15} Entonces Pedro dijo: ``Por favor, dinos lo que quieres
decir con esta ilustración''.

\bibleverse{16} ``¿Aún no lo han entendido?'' respondió Jesús.
\bibleverse{17} ``¿No ven que todo lo que entra a la boca pasa por el
estómago y luego sale del cuerpo como un desperdicio?\footnote{\textbf{15:17}
  Literalmente, ``botadas en el alcantarillado''.} \bibleverse{18} Pero
lo que sale de la boca viene de la mente, y eso es lo que los contamina.
\bibleverse{19} Porque lo que sale de la mente son pensamientos malos,
asesinatos, adulterio, inmoralidad sexual, hurto, falso testimonio, y
blasfemia, \bibleverse{20} y esas son las cosas que los contaminan a
ustedes. Comer sin lavarse las manos no los contamina''.

\hypertarget{jesuxfas-y-la-mujer-cananea-en-el-uxe1rea-de-tiro-y-siduxf3n}{%
\subsection{Jesús y la mujer cananea en el área de Tiro y
Sidón}\label{jesuxfas-y-la-mujer-cananea-en-el-uxe1rea-de-tiro-y-siduxf3n}}

\bibleverse{21} Jesús se fue de allí y se dirigió hacia la región de
Tiro y Sidón. \bibleverse{22} Una mujer cananea de ese lugar vino
gritando: ``¡Señor, Hijo de David! ¡Por favor, ten misericordia de mi,
pues mi hija sufre grandemente porque está poseída por un demonio!''

\bibleverse{23} Pero Jesús no respondió en absoluto. Sus discípulos
vinieron y le dijeron: ``Dile que deje de seguirnos. ¡Sus gritos son muy
molestos!''

\bibleverse{24} ``Yo fui enviado únicamente a las ovejas perdidas de la
casa de Israel'', le dijo Jesús a la mujer. \footnote{\textbf{15:24} Mat
  10,5-6; Rom 15,8}

\bibleverse{25} Pero la mujer vino y se arrodilló delante de él, y le
dijo: ``¡Señor, por favor, ayúdame!''

\bibleverse{26} ``No es correcto tomar el alimento de los hijos para
dárselo a los perros'',\footnote{\textbf{15:26} La palabra usada para
  ``perros'' aquí se refiere a perros domésticos, o cachorros.} le dijo
Jesús.

\bibleverse{27} ``Sí, Señor, pero aun así, a los perros se les deja
comer las migajas que caen de la mesa de su amo'', respondió ella.

\bibleverse{28} ``Tu confías en mí grandemente'', le respondió Jesús.
``¡Tu deseo está concedido!'' Y su hija fue sanada de inmediato.

\hypertarget{actividad-curativa-de-jesuxfas-en-galilea-en-la-orilla-oriental-del-lago-alimentando-a-los-cuatro-mil}{%
\subsection{Actividad curativa de Jesús en Galilea en la orilla oriental
del lago; Alimentando a los cuatro
mil}\label{actividad-curativa-de-jesuxfas-en-galilea-en-la-orilla-oriental-del-lago-alimentando-a-los-cuatro-mil}}

\bibleverse{29} Entonces Jesús regresó, pasando por el mar de Galilea.
Se fue hacia las montañas cercanas y allí se sentó. \bibleverse{30}
Grandes multitudes vinieron a él, trayéndole a aquellos que estaban
cojos, ciegos, paralíticos, mudos y también muchos otros que estaban
enfermos. Los ponían en el piso, a sus pies, y él los sanaba.
\bibleverse{31} La multitud estaba asombrada ante lo que ocurría: los
sordos podían hablar, los paralíticos eran sanados, los cojos podían
caminar, y los ciegos podían ver. Y alababan al Dios de Israel.
\footnote{\textbf{15:31} Mar 7,37}

\bibleverse{32} Entonces Jesús llamó a sus discípulos y les dijo:
``Siento pesar por estas personas, porque han estado conmigo por tres
días y no tienen nada que comer. No quiero que se vayan con hambre, no
sea que se desmayen por el camino''. \footnote{\textbf{15:32} Mat
  14,13-21}

\bibleverse{33} ``¿Dónde podríamos encontrar suficiente pan en este
desierto para alimentar a semejante multitud tan grande?'' respondieron
los discípulos.

\bibleverse{34} ``¿Cuántos panes tienen ustedes allí?'' preguntó Jesús.
``Siete, y unos cuantos peces pequeños'', respondieron ellos.

\bibleverse{35} Jesús dijo a la multitud que se sentara en el suelo.
\bibleverse{36} Entonces tomó los siete panes y los peces, y después de
bendecir la comida, la partió en trozos y la dio a los discípulos, y los
discípulos la daban a la multitud. \bibleverse{37} Todos comieron hasta
que estuvieron saciados, y entonces recogieron las sobras, llenando así
siete canastas. \bibleverse{38} Cuatro mil hombres comieron de esta
comida, sin contar a las mujeres y a los niños. \bibleverse{39} Entonces
Jesús despidió a la multitud, subió a la barca, y se fue a la región de
Magadán.

\hypertarget{rechazo-de-la-demanda-de-los-oponentes-de-seuxf1ales-y-advertencia-contra-la-levadura-de-los-fariseos}{%
\subsection{Rechazo de la demanda de los oponentes de señales y
advertencia contra la levadura de los
fariseos}\label{rechazo-de-la-demanda-de-los-oponentes-de-seuxf1ales-y-advertencia-contra-la-levadura-de-los-fariseos}}

\hypertarget{section-15}{%
\section{16}\label{section-15}}

\bibleverse{1} Los Fariseos y los Saduceos vinieron para
probar\footnote{\textbf{16:1} Puesto que la prueba era una tentativa
  para desacreditar a Jesús, esto también podría traducirse como
  ``vinieron a ponerle una trampa a Jesús''.} a Jesús, exigiéndole que
les mostrara una señal del cielo. \footnote{\textbf{16:1} Mat 12,38}
\bibleverse{2} Jesús les dijo: ``Por la noche, ustedes dicen, `mañana
habrá buen tiempo, porque el cielo se ve rojo,' \bibleverse{3} pero por
la mañana dicen: `habrá mal tiempo hoy, porque el cielo está rojo y
nublado'. ¡Ustedes saben predecir el clima por cómo se ve el cielo, pero
no son capaces de reconocer las señales de los tiempos! \bibleverse{4}
La gente mala que no confía en Dios es la que espera una señal
milagrosa, y a esas personas no se les dará ninguna señal excepto la
señal de Jonás''. Y entonces se fue de allí. \footnote{\textbf{16:4} Mat
  12,39-40}

\bibleverse{5} Cuando iban hacia el otro lado del lago, los discípulos
olvidaron llevar pan. \bibleverse{6} ``Cuídense de la levadura de los
Fariseos y los Saduceos'', les dijo Jesús.

\bibleverse{7} Los discípulos comenzaron a discutir entre ellos. ``Está
diciendo eso\footnote{\textbf{16:7} Está implícito en el texto.} porque
no trajimos pan'', concluyeron.

\bibleverse{8} Pero Jesús sabía lo que ellos estaban diciendo y les
dijo: ``¡Ustedes confían muy poco en mi! ¿Por qué están discutiendo
entre ustedes por no tener pan? \bibleverse{9} ¿Acaso aún no lo han
entendido? ¿No recuerdan los cinco panes que alimentaron cinco mil
personas? ¿Cuántas canastas sobraron? \footnote{\textbf{16:9} Mat
  14,17-21} \bibleverse{10} ¿Y qué hay de los siete panes que
alimentaron a los cuatro mil? ¿Cuántas canastas sobraron? \footnote{\textbf{16:10}
  Mat 15,34-38} \bibleverse{11} ¿No se han dado cuenta aún de que yo no
hablaba sobre el pan? ¡Cuídense de la levadura de los Fariseos y los
Saduceos!''

\bibleverse{12} Entonces se dieron cuenta de que él no les estaba
advirtiendo sobre levadura de pan, sino sobre las enseñanzas de los
Fariseos y los Saduceos.

\hypertarget{la-confesiuxf3n-del-mesuxedas-de-pedro-en-cesarea-de-filipo-llamando-a-pedro-a-ser-el-fundador-y-luxedder-de-la-iglesia}{%
\subsection{La Confesión del Mesías de Pedro en Cesarea de Filipo;
Llamando a Pedro a ser el fundador y líder de la
iglesia}\label{la-confesiuxf3n-del-mesuxedas-de-pedro-en-cesarea-de-filipo-llamando-a-pedro-a-ser-el-fundador-y-luxedder-de-la-iglesia}}

\bibleverse{13} Cuando llegó a la región de Cesarea de Filipo, Jesús le
preguntó a sus discípulos: ``¿Quién dice la gente que es el Hijo del
hombre?''

\bibleverse{14} ``Algunos dicen que Juan el Bautista, otros dicen que
Elías, y otros dicen que Jeremías o uno de los otros profetas'',
respondieron ellos. \footnote{\textbf{16:14} Mat 14,2; Mat 17,10; Luc
  7,16}

\bibleverse{15} ``¿Y ustedes?'' preguntó él. ``¿Quién dicen ustedes que
soy yo?''

\bibleverse{16} ``Tú eres el Mesías, el Hijo del Dios viviente'',
respondió Simón Pedro.

\bibleverse{17} ``Verdaderamente eres bendito, Simón hijo de Juan'', le
dijo Jesús. ``Porque esto no te fue revelado por carne ni sangre humana,
sino por mi Padre que está en el cielo. \footnote{\textbf{16:17} Mat
  11,27; Gal 1,15-16} \bibleverse{18} También te digo que tú eres
Pedro,\footnote{\textbf{16:18} Pedro significa ``piedra'', en contraste
  con la palabra que se usa para roca sólida en este versículo.} y sobre
esta roca edificaré mi iglesia, y los poderes de la muerte\footnote{\textbf{16:18}
  Literalmente, ``las puertas del Hades''.} no la destruirán.
\footnote{\textbf{16:18} Juan 1,42; Efes 2,20} \bibleverse{19} Te daré
las llaves del reino de los cielos, y todo lo que prohíbas en la tierra,
será prohibido en los cielos; y todo lo que permitas en la tierra, será
permitido en los cielos''. \footnote{\textbf{16:19} Mat 18,18}
\bibleverse{20} Entonces le advirtió a sus discípulos de no decirle a
nadie que él era el Mesías. \footnote{\textbf{16:20} Mat 17,9}

\hypertarget{primer-anuncio-de-sufrimiento}{%
\subsection{Primer anuncio de
sufrimiento}\label{primer-anuncio-de-sufrimiento}}

\bibleverse{21} A partir de entonces Jesús comenzó a explicarle a sus
discípulos que él tendría que ir a Jerusalén, y que sufriría
terriblemente en manos de los ancianos, de los jefes de los sacerdotes y
de los maestros religiosos, y que lo matarían, pero que él se levantaría
otra vez al tercer día. \footnote{\textbf{16:21} Mat 12,40; Juan 2,19}

\bibleverse{22} Pedro levó a Jesús con él aparte y comenzó a decirle que
no era bueno que hablara así. ``¡Dios no permita, Señor, que algo así
llegue a ocurrirte!'' le dijo.

\bibleverse{23} Jesús se volvió hacia Pedro y le dijo: ``¡Aléjate de mi,
Satanás! ¡Eres una trampa para hacerme tropezar,\footnote{\textbf{16:23}
  Literalmente, ``una piedra de tropiezo'' o una ``trampa''.} porque
estás pensando humanamente, y no como Dios piensa!''

\hypertarget{proverbios-sobre-el-seguimiento-de-los-discuxedpulos-en-el-sufrimiento}{%
\subsection{Proverbios sobre el seguimiento de los discípulos en el
sufrimiento}\label{proverbios-sobre-el-seguimiento-de-los-discuxedpulos-en-el-sufrimiento}}

\bibleverse{24} Entonces Jesús le dijo a sus discípulos: ``El que quiera
seguirme, debe negarse a sí mismo, tomar su cruz y seguirme.
\bibleverse{25} Porque el que quiera salvar su vida, la perderá, y el
que pierda la vida por mi causa, la ganará. \footnote{\textbf{16:25}
  Apoc 12,11} \bibleverse{26} ¿Qué beneficio tiene ganar el mundo entero
y perder la vida? ¿Qué darán ustedes a cambio de su vida? \footnote{\textbf{16:26}
  Luc 12,20} \bibleverse{27} Porque el Hijo del hombre vendrá en la
gloria de su Padre, junto con sus ángeles. Entonces le dará a cada uno
lo que merece conforme a lo que haya hecho. \footnote{\textbf{16:27} Rom
  2,6} \bibleverse{28} Les digo la verdad: hay algunos aquí que no
morirán\footnote{\textbf{16:28} Literalmente, ``probarán la muerte''.}
antes de que vean al Hijo del hombre venir en su reino''.\footnote{\textbf{16:28}
  Mat 10,23}

\hypertarget{la-transfiguraciuxf3n-de-jesuxfas-en-la-montauxf1a}{%
\subsection{La transfiguración de Jesús en la
montaña}\label{la-transfiguraciuxf3n-de-jesuxfas-en-la-montauxf1a}}

\hypertarget{section-16}{%
\section{17}\label{section-16}}

\bibleverse{1} Seis días después Jesús llevó consigo a Pedro, a
Santiago, y a su hermano Juan hacia una montaña alta para estar solos
allí. \footnote{\textbf{17:1} Mat 26,37; Mar 5,37; Mar 13,3; Mar 14,33;
  Luc 8,51} \bibleverse{2} Entonces Jesús se transformó frente a ellos.
Su rostro brillaba como el sol. Sus vestiduras se volvieron blancas como
la luz. \footnote{\textbf{17:2} 2Pe 1,16-18; Apoc 1,16} \bibleverse{3}
De repente, aparecieron Moisés y Elías delante de ellos, y estos dos
estaban hablando con Jesús.

\bibleverse{4} Pedro los interrumpió\footnote{\textbf{17:4} Está
  implícito. En el original dice: ``Pero respondiendo, Pedro dijo''.}
diciéndole a Jesús: ``Señor, qué bien se siente estar aquí. Si tú
quieres haré tres enramadas---una para ti, una para Moisés, y una para
Elías''.

\bibleverse{5} Mientras Pedro aún hablaba, una nube brillante los
cubrió. Entonces se escuchó una voz que salía desde la nube, que decía:
``Este es mi hijo a quien amo, el cual me complace. Escúchenlo''.
\footnote{\textbf{17:5} Mat 3,17}

\bibleverse{6} Cuando oyeron esto, los discípulos cayeron sobre sus
rostros, completamente aterrorizados. \bibleverse{7} Jesús se acercó a
ellos y los tocó. ``Levántense'', les dijo. ``No tengan miedo''.
\bibleverse{8} Cuando levantaron la vista, no vieron a nadie más allí,
excepto a Jesús.

\bibleverse{9} Cuando descendieron de la montaña, Jesús les dio
instrucciones precisas: ``No le digan a nadie lo que vieron hasta que el
Hijo del hombre se haya levantado de entre los muertos'', les dijo.

\bibleverse{10} ``¿Por qué, entonces, los maestros religiosos dicen que
Elías debe venir primero?'' preguntaron sus discípulos. \footnote{\textbf{17:10}
  Mat 11,14}

\bibleverse{11} ``Es cierto que Elías viene a poner cada cosa en su
lugar, \bibleverse{12} pero déjenme decirles que Elías ya vino y sin
embargo la gente no reconoció quién era él. Hicieron con él todo lo que
quisieron. De la misma manera, el Hijo del hombre también sufrirá en
manos de ellos''. \bibleverse{13} Entonces los discípulos se dieron
cuenta de que Jesús se estaba refiriendo a Juan el Bautista. \footnote{\textbf{17:13}
  Luc 1,17}

\hypertarget{curaciuxf3n-de-un-niuxf1o-epiluxe9ptico-enseuxf1ando-sobre-el-fracaso-de-los-discuxedpulos}{%
\subsection{Curación de un niño epiléptico; Enseñando sobre el fracaso
de los
discípulos}\label{curaciuxf3n-de-un-niuxf1o-epiluxe9ptico-enseuxf1ando-sobre-el-fracaso-de-los-discuxedpulos}}

\bibleverse{14} Cuando se aproximaban a la multitud, un hombre llegó y
se arrodilló delante de Jesús. \bibleverse{15} ``Señor, por favor, ten
misericordia de mi hijo'', le dijo. ``Él se vuelve loco\footnote{\textbf{17:15}
  Literalmente, ``que está loco''. Este término es paralelo al término
  ``lunático'', del latín ``lunaticus''.} y sufre ataques tan terribles
que a veces hasta se lanza al fuego o al agua. \bibleverse{16} Lo traje
ante tus discípulos pero ellos no pudieron sanarlo''.

\bibleverse{17} ``¡Este pueblo\footnote{\textbf{17:17} Literalmente,
  ``generación''.} se niega a confiar en mi, y todos están corruptos!''
respondió Jesús. ``¿Cuánto tiempo más tengo que permanecer aquí con
ustedes? ¿Cuánto tiempo más tendré que aguantarlos? ¡Tráiganmelo aquí!''
\bibleverse{18} Jesús confrontó al demonio y éste salió del joven, y
quedó sano de inmediato.

\bibleverse{19} Después de esto, los discípulos vinieron a Jesús en
privado y le preguntaron: ``¿Por qué nosotros no pudimos sacarlo?''

\bibleverse{20} hol``Porque ustedes no creen lo suficiente'', les dijo
Jesús. ``Les digo que aún si la confianza de ustedes fuera tan pequeña
como una semilla de mostaza, ustedes podrían decir a esta montaña
`muévete de aquí para allá,' y esta se movería. Nada sería imposible
para ustedes''. \footnote{\textbf{17:20} Mat 21,21; Luc 17,6; 1Cor 13,2}
\bibleverse{21} \footnote{\textbf{17:21} El versículo 21 no está en los
  primeros manuscritos.} \footnote{\textbf{17:21} Mar 9,29}

\hypertarget{segundo-anuncio-del-sufrimiento-en-galilea}{%
\subsection{Segundo anuncio del sufrimiento en
Galilea}\label{segundo-anuncio-del-sufrimiento-en-galilea}}

\bibleverse{22} Mientras caminaban por Galilea, Jesús les dijo: ``El
Hijo del hombre será traicionado y la gente tendrá poder\footnote{\textbf{17:22}
  Literalmente, ``entregado en manos de hombres''.} sobre él.
\footnote{\textbf{17:22} Mat 16,21; Mat 20,18-19} \bibleverse{23} Lo
matarán, pero el tercer día, él se levantará de nuevo''. Los discípulos
se entristecieron.

\hypertarget{el-impuesto-del-templo-y-su-maravilloso-pago-en-capernaum}{%
\subsection{El impuesto del templo y su maravilloso pago en
Capernaum}\label{el-impuesto-del-templo-y-su-maravilloso-pago-en-capernaum}}

\bibleverse{24} Cuando llegaron a Capernaúm, los que estaban encargados
de recolectar el impuesto de medio siclo en el Templo, vinieron donde
estaba Pedro y le preguntaron: ``Tu maestro paga el medio siclo, ¿no es
así?'' \bibleverse{25} ``Si, por supuesto'', respondió Pedro. Cuando
regresó donde estaban todos, Jesús se anticipó al hecho. ``¿Qué piensas
tu, Simón?'' le preguntó Jesús. ``¿Acaso los reyes de este mundo le
cobran los impuestos a sus propios hijos o a los otros?''

\bibleverse{26} ``A los otros'', respondió Pedro. Entonces Jesús le
dijo: ``En ese caso, los hijos están excentos.

\bibleverse{27} Pero para no ofender a nadie, ve al lago y saca un pez
con un anzuelo. Saca el primer pez que atrapes, y cuando abras su boca
encontrarás una moneda estatera.\footnote{\textbf{17:27} Equivalente a
  un siclo. El impuesto del Templo en esa época era medio siclo por
  persona.} Toma la moneda y paga por ti y por mi''.

\hypertarget{controversia-entre-discuxedpulos-la-exhortaciuxf3n-de-jesuxfas-a-la-humildad}{%
\subsection{Controversia entre discípulos; La exhortación de Jesús a la
humildad}\label{controversia-entre-discuxedpulos-la-exhortaciuxf3n-de-jesuxfas-a-la-humildad}}

\hypertarget{section-17}{%
\section{18}\label{section-17}}

\bibleverse{1} En esos días los discípulos vinieron a Jesús, y le
preguntaron: ``¿Quién es el más grande en el reino de los cielos?''

\bibleverse{2} Jesús llamó a un niño pequeño. Puso al niño de pie frente
a ellos. \bibleverse{3} ``Les digo la verdad: a menos que cambien su
manera de pensar y se vuelvan como niños pequeños, nunca entrarán en el
reino de los cielos. \footnote{\textbf{18:3} Mat 19,14} \bibleverse{4}
Pero cualquiera que se humilla y se vuelve como este niño, ese es el más
grande en el reino de los cielos. \bibleverse{5} Cualquiera que acepta a
un niño como este en mi nombre, me acepta a mí.

\hypertarget{la-preocupaciuxf3n-de-jesuxfas-por-los-pequeuxf1os-y-los-duxe9biles-advertencia-contra-los-seductores-del-mal}{%
\subsection{La preocupación de Jesús por los pequeños y los débiles;
Advertencia contra los seductores del
mal}\label{la-preocupaciuxf3n-de-jesuxfas-por-los-pequeuxf1os-y-los-duxe9biles-advertencia-contra-los-seductores-del-mal}}

\bibleverse{6} Pero cualquiera que hace pecar a uno de estos niños que
cree en mí, sería mejor que atase a su cuello una piedra de
moler\footnote{\textbf{18:6} Literalmente, ``un molino de asno'',
  refiriéndose a los molinos que eran girados por un asno, y no a los
  molinos que se manejaban manualmente.} y se lance a las profundidades
del mar. \footnote{\textbf{18:6} Luc 17,1-2}

\bibleverse{7} ``¡Cuán grande es el desastre que sobrevendrá en el mundo
por todas sus tentaciones a pecar! ¡Las tentaciones ciertamente vendrán,
pero será un desastre para la persona por quien viene la tentación!
\bibleverse{8} Si tu mano o tu pie te hacen pecar, córtalo y bótalo. Es
mejor que entres a la vida eterna siendo paralítico o cojo, que tener
dos manos o dos pies y ser lanzado al fuego eterno. \bibleverse{9} Si tu
ojo te hace pecar, sácalo y bótalo. Es mejor que entres a la vida eterna
con un solo ojo que tener dos ojos y ser lanzado al fuego de Gehena.
\bibleverse{10} Asegúrense de no menospreciar a estos pequeños. Yo les
digo que en el cielo sus ángeles siempre están con\footnote{\textbf{18:10}
  Literalmente, ``ven el rostro de''.} mi Padre celestial. \footnote{\textbf{18:10}
  Heb 1,14} \bibleverse{11} \footnote{\textbf{18:11} El versículo 11 no
  está en los primeros manuscritos.} \footnote{\textbf{18:11} Mat 9,13;
  Luc 19,10}

\hypertarget{la-paruxe1bola-de-la-oveja-perdida}{%
\subsection{La parábola de la oveja
perdida}\label{la-paruxe1bola-de-la-oveja-perdida}}

\bibleverse{12} ¿Qué piensan ustedes? Si un hombre tiene cien ovejas y
una de ellas se pierde, ¿acaso no dejará él las noventa y nueve en la
colina e irá en búsqueda de la que está perdida? \bibleverse{13} Y si la
encuentra, yo les digo que ese hombre se regocija más por esa oveja que
por las noventa y nueve que no se perdieron. \bibleverse{14} De la misma
manera, mi Padre celestial no quiere que ninguno de estos pequeños se
pierda.

\hypertarget{de-comportamiento-hacia-el-hermano-pecador-sobre-el-efecto-del-juicio-y-la-oraciuxf3n-de-la-iglesia}{%
\subsection{De comportamiento hacia el hermano pecador; sobre el efecto
del juicio y la oración de la
iglesia}\label{de-comportamiento-hacia-el-hermano-pecador-sobre-el-efecto-del-juicio-y-la-oraciuxf3n-de-la-iglesia}}

\bibleverse{15} ``Si un hermano\footnote{\textbf{18:15} O ``hermano en
  la fe''.} peca contra ti, ve y muéstrale el error a él, solo entre
ustedes dos. Si te escucha, habrás convencido a tu hermano. \footnote{\textbf{18:15}
  Lev 19,17; Luc 17,3; Gal 6,1} \bibleverse{16} Pero si no escucha,
entonces lleva contigo a una o dos personas, para que con dos o tres
testigos pueda confirmarse la verdad.\footnote{\textbf{18:16} Ver
  Deuteronomio 19:15.} \footnote{\textbf{18:16} Deut 19,15}
\bibleverse{17} Si aun así él se niega a escucharte, entonces dilo a la
iglesia. Si se niega a escuchar a la iglesia, entonces trátalo como a un
extranjero\footnote{\textbf{18:17} Literalmente, un ``gentil'', un
  incrédulo.} y recaudador de impuestos. \footnote{\textbf{18:17} 1Cor
  5,13; 2Tes 3,6; Tit 3,10} \bibleverse{18} Les digo la verdad: todo lo
que prohíban en la tierra será prohibido en el cielo, y todo lo que
permitan en la tierra, será permitido en el cielo. \footnote{\textbf{18:18}
  Mat 16,19; Juan 20,23} \bibleverse{19} ``También les digo que si dos
de ustedes se ponen de acuerdo aquí en la tierra acerca de algo por lo
que están orando, entonces mi Padre celestial lo hará por ustedes.
\footnote{\textbf{18:19} Mar 11,24} \bibleverse{20} Porque donde dos o
tres se reúnen en mi nombre, allí estoy con ellos''. \footnote{\textbf{18:20}
  Mat 28,20}

\hypertarget{del-perduxf3n-la-paruxe1bola-del-sinverguxfcenza}{%
\subsection{Del perdón; la parábola del
sinvergüenza}\label{del-perduxf3n-la-paruxe1bola-del-sinverguxfcenza}}

\bibleverse{21} Entonces Pedro vino donde estaba Jesús y le preguntó:
``¿Cuántas veces debo perdonar a mi hermano por pecar contra mi? ¿Siete
veces?''

\bibleverse{22} ``No, siete veces no. ¡Yo diría hasta setenta veces
siete!'' le dijo Jesús. \footnote{\textbf{18:22} Gén 4,24; Luc 17,4;
  Efes 4,32} \bibleverse{23} ``Por eso el reino de los cielos es como un
rey que quería saldar cuentas con los siervos que le debían dinero.
\bibleverse{24} Cuando comenzó a saldar cuentas, fue presentado delante
de él un siervo que le debía diez mil talentos.\footnote{\textbf{18:24}
  Una cantidad astronómica.} \bibleverse{25} Como este hombre no tenía
dinero para pagar, su amo dio la orden de venderlo, junto con su esposa,
sus hijos y todas sus posesiones para poder pagar la deuda.
\bibleverse{26} El siervo se arrodilló y le dijo a su amo: `¡Por favor,
ten paciencia conmigo! ¡Yo lo pagaré todo!' \bibleverse{27} El amo tuvo
misericordia del siervo, lo liberó y canceló la deuda.

\bibleverse{28} Pero cuando ese mismo siervo salió de allí, se encontró
con uno de sus consiervos que le debía apenas cien denarios.\footnote{\textbf{18:28}
  Un denario era una moneda pequeña. Se hace contraste entre la gran
  cantidad que se le perdonó al primer siervo y la pequeña cantidad que
  le debía a éste el segundo siervo.} Lo tomó por el cuello y
ahorcándolo, le decía: `¡Págame lo que me debes!'

\bibleverse{29} Su consiervo se lanzó a los pies de este hombre y le
rogó: `¡Por favor, sé paciente conmigo! ¡Yo te pagaré!' \bibleverse{30}
Pero el hombre se negó, y fue y puso a su consiervo en prisión hasta que
le pagara lo que le debía. \bibleverse{31} ``Cuando los otros siervos
vieron lo que había pasado, se aturdieron y estaban molestos. Fueron a
decirle a su amo todo lo que había ocurrido. \bibleverse{32} Entonces el
amo volvió a llamar a aquél hombre y le dijo: `¡Siervo malo! Te perdoné
toda la deuda porque me rogaste que te perdonara. \bibleverse{33} ¿Acaso
no deberías haber sido misericordioso con tu consiervo también, así como
yo lo fui contigo?' \footnote{\textbf{18:33} 1Jn 4,11} \bibleverse{34}
Su amo se enojó y lo entregó a los carceleros hasta que pagase toda la
deuda. \footnote{\textbf{18:34} Mat 5,26} \bibleverse{35} Esto es lo que
mi Padre celestial hará con cada uno de ustedes a menos que con
sinceridad\footnote{\textbf{18:35} Literalmente, ``de corazón''.}
ustedes perdonen a sus hermanos''.\footnote{\textbf{18:35} Mat 6,14-15;
  Sant 2,13}

\hypertarget{salida-hacia-jerusaluxe9n-y-caminata-por-la-ribera-oriental-conversaciones-sobre-el-matrimonio-sobre-el-divorcio-y-la-renuncia-al-matrimonio}{%
\subsection{Salida hacia Jerusalén y caminata por la Ribera Oriental;
Conversaciones sobre el matrimonio, sobre el divorcio y la renuncia al
matrimonio}\label{salida-hacia-jerusaluxe9n-y-caminata-por-la-ribera-oriental-conversaciones-sobre-el-matrimonio-sobre-el-divorcio-y-la-renuncia-al-matrimonio}}

\hypertarget{section-18}{%
\section{19}\label{section-18}}

\bibleverse{1} Cuando Jesús terminó de hablar se fue de Galilea y se
dirigió a la región de Judea, al otro lado del Jordán. \bibleverse{2}
Grandes multitudes le seguían, y él sanaba a los que allí estaban
enfermos.

\bibleverse{3} Entonces ciertos Fariseos vinieron para probarlo. ``¿Se
le permite a un hombre divorciarse de su esposa por cualquier razón?''
le preguntaron.

\bibleverse{4} Jesús respondió: ``¿No han leído que Dios, quien creó a
las personas en el principio, los creó hombre y mujer?\footnote{\textbf{19:4}
  Ver Génesis 1:27 y Génesis 5:2.} \footnote{\textbf{19:4} Gén 1,27}
\bibleverse{5} Entonces dijo: `Esta es la razón por la cual el hombre se
irá de donde su padre y su madre y se unirá a su esposa, y entonces los
dos se convertirán en uno'.\footnote{\textbf{19:5} Literalmente, ``una
  carne''. Citando Génesis 2:24} \bibleverse{6} Ahora no son dos, sino
uno. Lo que Dios ha unido, nadie debe separarlo''.

\bibleverse{7} ``¿Entonces por qué Moisés entregó una ley que dice que
un hombre puede divorciarse de su esposa entregándole un certificado de
divorcio escrito y despidiéndola?''\footnote{\textbf{19:7} Ver
  Deuteronomio 24:1.} le preguntaron. \footnote{\textbf{19:7} Deut 24,1}

\bibleverse{8} ``Por la actitud insensible de ustedes, Moisés les
permitió divorciarse de sus esposas, pero no era así al comienzo'',
respondió Jesús. \bibleverse{9} ``Les digo que cualquiera que se
divorcia de su esposa -- a menos que sea por inmoralidad sexual --, y
luego se casa con otra mujer, comete adulterio''.

\bibleverse{10} ``¡Si esa es la situación entre el esposo y la esposa,
es mejor no casarse!'' respondieron sus discípulos.

\bibleverse{11} ``No cualquiera puede aceptar esta
instrucción,\footnote{\textbf{19:11} Literalmente, ``palabra''.} solo
aquellos a quienes se les da'', les dijo Jesús. \footnote{\textbf{19:11}
  1Cor 7,7; 1Cor 7,17} \bibleverse{12} ``Algunos nacen siendo eunucos,
algunos se vuelven eunucos por causa de otros hombres, y otros deciden
ser eunucos por causa del reino de los cielos. Los que aceptan hacerlo,
deben aceptar tal enseñanza''.

\hypertarget{jesuxfas-bendice-a-los-niuxf1os}{%
\subsection{Jesús bendice a los
niños}\label{jesuxfas-bendice-a-los-niuxf1os}}

\bibleverse{13} Entonces la gente traía niños pequeños delante de él
para que los bendijera y orara por ellos. Pero los discípulos les decían
que no lo hicieran. \bibleverse{14} Pero Jesús dijo: ``Dejen que los
niños vengan a mi. No se lo impidan. ¡El reino de los cielos pertenece a
quienes son como ellos!'' \bibleverse{15} Entonces él puso sus manos
sobre ellos para bendecirlos y luego se fue.

\hypertarget{la-conversaciuxf3n-de-jesuxfas-con-el-joven-rico-el-peligro-de-la-riqueza}{%
\subsection{La conversación de Jesús con el joven rico; el peligro de la
riqueza}\label{la-conversaciuxf3n-de-jesuxfas-con-el-joven-rico-el-peligro-de-la-riqueza}}

\bibleverse{16} Un hombre vino a Jesús y le dijo: ``Maestro, ¿qué cosas
buenas debo hacer para recibir vida eterna?''

\bibleverse{17} ``¿Por qué me preguntas a mi lo que es bueno?''
respondió Jesús. ``Solo hay uno que es bueno. Pero si quieres tener vida
eterna,\footnote{\textbf{19:17} Literalmente, ``entrar a la vida''.}
entonces guarda los mandamientos''.

\bibleverse{18} ``¿Cuáles?'' preguntó el hombre. ``No mates, no cometas
adulterio, no robes, no des falso testimonio, \footnote{\textbf{19:18}
  Éxod 20,12-16}

\bibleverse{19} honra a tu padre y a tu madre, y ama a tu prójimo como a
ti mismo'',\footnote{\textbf{19:19} Citando Éxodo 20:12-16; Levítico
  19:18; Deuteronomio 5:16-20 .} respondió Jesús. \footnote{\textbf{19:19}
  Lev 19,18}

\bibleverse{20} ``Yo he guardado todos estos mandamientos'', dijo el
joven. ``¿Qué más debo hacer?''

\bibleverse{21} Jesús le dijo: ``Si quieres ser perfecto,\footnote{\textbf{19:21}
  ``Perfecto'' aquí conlleva la idea de algo realizado o completo.}
entonces ve y vende todas tus posesiones, da el dinero a los pobres, y
tendrás tesoro en el cielo. Entonces ven y sígueme''. \footnote{\textbf{19:21}
  Mat 6,20; Luc 12,33} \bibleverse{22} Cuando el joven escuchó la
respuesta de Jesús, se fue muy triste, porque tenía muchas posesiones.
\footnote{\textbf{19:22} Sal 62,11}

\bibleverse{23} ``Les digo la verdad'', dijo Jesús a sus discípulos, ``a
la gente rica se le hace difícil entrar al reino de los cielos.
\bibleverse{24} También les digo esto: es más fácil que un camello pase
a través del ojo de una aguja que un rico entre al reino de los
cielos''. \footnote{\textbf{19:24} Mat 7,14}

\bibleverse{25} Cuando los discípulos oyeron esto, se sorprendieron, y
preguntaron: ``¿Quién puede salvarse entonces?''

\bibleverse{26} Jesús los miró y dijo: ``Desde un punto de vista humano,
es imposible, pero con Dios todas las cosas son posibles''.

\hypertarget{la-recompensa-de-seguir-a-jesuxfas-y-la-renuncia}{%
\subsection{La recompensa de seguir a Jesús y la
renuncia}\label{la-recompensa-de-seguir-a-jesuxfas-y-la-renuncia}}

\bibleverse{27} Pedro le respondió: ``Mira, hemos dejado todo y te hemos
seguido. ¿Qué recompensa tendremos?'' \footnote{\textbf{19:27} Mat 4,20;
  Mat 4,22}

\bibleverse{28} Jesús respondió: ``Les digo la verdad: cuando todo sea
hecho de nuevo y el Hijo del hombre se siente en su trono glorioso,
ustedes que me han seguido también se sentarán en tronos, y serán jueces
de las doce tribus de Israel. \footnote{\textbf{19:28} Luc 22,30; 1Cor
  6,2; Apoc 3,21}

\bibleverse{29} Todos los que dejan su hogar, a sus hermanos, a sus
hermanas, a sus padres, a sus madres, a sus hijos y sus campos por mi
causa, recibirán cien veces más y recibirán la vida eterna.
\bibleverse{30} Porque muchos que son los primeros serán dejados de
ultimo, y muchos que son los últimos, serán los primeros.

\hypertarget{paruxe1bola-de-los-trabajadores-de-la-viuxf1a}{%
\subsection{Parábola de los trabajadores de la
viña}\label{paruxe1bola-de-los-trabajadores-de-la-viuxf1a}}

\hypertarget{section-19}{%
\section{20}\label{section-19}}

\bibleverse{1} ``Porque el reino de los cielos es como un terrateniente
que salió temprano por la mañana para contratar trabajadores para su
viña. \bibleverse{2} Él decidió pagar un denario por día a los
trabajadores, y los envió a trabajar en ella. \bibleverse{3} Cerca de
las 9 a. m. salió y vio a otros que estaban sin trabajar en la plaza del
mercado. \bibleverse{4} ```Vayan y trabajen en la viña también, y yo les
pagaré lo justo', les dijo. Entonces ellos se fueron a trabajar.
\bibleverse{5} Entre el medio día y las 3 p.~m. salió e hizo lo mismo.
\bibleverse{6} A las 5 p.~m. salió y encontró a otros que estaban allí.
`¿Por qué están por ahí todo el día sin hacer nada?' les preguntó.

\bibleverse{7} `Porque nadie nos ha contratado', respondieron ellos.
`Vayan y trabajen en la viña también', les dijo.

\bibleverse{8} ``Cuando llegó la noche, el propietario de la viña le
dijo a su administrador: `Llama a los trabajadores y págales sus
salarios. Comienza con los trabajadores que fueron contratados al final
y luego continúa con los que fueron contratados al principio'.
\bibleverse{9} Cuando entraron los que fueron contratados a las 5 p.~m.,
cada uno recibió un denario. \bibleverse{10} Así que cuando entraron los
que fueron contratados al principio, ellos pensaron que recibirían más,
pero también recibieron un denario. \bibleverse{11} Cuando recibieron su
pago, se quejaron del propietario. \bibleverse{12} `Los que fueron
contratados al final solo trabajaron una hora, y les pagaste lo mismo
que a nosotros que trabajamos todo el día en medio del calor abrasante,'
refunfuñaban.

\bibleverse{13} ``El propietario le respondió a uno de ellos: `Amigo, no
he sido injusto contigo. ¿No estuviste de acuerdo conmigo en trabajar
por un denario? \bibleverse{14} Toma tu pago y vete. Lo mismo que te
pagué a ti, lo quiero pagar a los que fueron contratados al final.
\bibleverse{15} ¿Acaso no puedo decidir qué hacer con mi propio dinero?
¿Por qué deberías mirarme con desprecio por querer hacer un bien?'
\footnote{\textbf{20:15} Rom 9,16; Rom 9,21} \bibleverse{16} De esta
manera, los últimos serán los primeros, y los primeros serán los
últimos''.

\hypertarget{salida-hacia-jerusaluxe9n-tercer-anuncio-del-sufrimiento-de-jesuxfas}{%
\subsection{Salida hacia Jerusalén; tercer anuncio del sufrimiento de
Jesús}\label{salida-hacia-jerusaluxe9n-tercer-anuncio-del-sufrimiento-de-jesuxfas}}

\bibleverse{17} Cuando iba de camino hacia Jerusalén, Jesús llevó
consigo a los doce discípulos aparte mientras caminaban y les dijo:
\bibleverse{18} ``Miren, vamos hacia Jerusalén, y el Hijo del hombre
será entregado a los jefes de los sacerdotes y los maestros religiosos.
Ellos lo condenarán a muerte \bibleverse{19} y lo entregarán a los
gentiles\footnote{\textbf{20:19} Aquí se está refiriendo a los romanos.}
para que se burlen de él, lo azoten y lo crucifiquen. Pero el tercer día
será levantado de entre los muertos''.

\hypertarget{solicitud-ambiciosa-de-salomuxe9-para-sus-hijos-santiago-y-juan}{%
\subsection{Solicitud ambiciosa de Salomé para sus hijos Santiago y
Juan}\label{solicitud-ambiciosa-de-salomuxe9-para-sus-hijos-santiago-y-juan}}

\bibleverse{20} Entonces la madre de los hijos de Zebedeo vino a Jesús
con sus dos hijos. Se arrodilló delante de él para hacerle una petición.
\footnote{\textbf{20:20} Mat 10,2} \bibleverse{21} ``¿Qué es lo que me
pides?'' le dijo Jesús. ``Por favor, aparta a mis hijos para que se
sienten a tu lado en tu reino, uno a tu derecha y el otro a tu
izquierda'', le pidió ella. \footnote{\textbf{20:21} Mat 19,28}

\bibleverse{22} ``No sabes lo que estás pidiendo'', le dijo Jesús.
``¿Pueden ustedes beber la copa\footnote{\textbf{20:22} Refiriéndose a
  la copa de sufrimiento.} que yo estoy a punto de beber?'' . ``Sí
podemos'', le dijeron. \footnote{\textbf{20:22} Mat 26,39; Luc 12,50}

\bibleverse{23} ``Sin duda alguna ustedes beberán de mi copa'', les
dijo, ``pero el privilegio de sentarse a mi derecha y a mi izquierda no
me corresponde darlo a mi. Mi Padre es el que decide quién
será''.\footnote{\textbf{20:23} O, ``es para aquellos para quienes ha
  sido preparado por mi Padre''.} \footnote{\textbf{20:23} Hech 12,2;
  Apoc 1,9}

\hypertarget{del-deber-de-servir-por-el-reino-de-los-cielos}{%
\subsection{Del deber de servir por el reino de los
cielos}\label{del-deber-de-servir-por-el-reino-de-los-cielos}}

\bibleverse{24} Cuando los otros diez discípulos escucharon lo que ellos
habían pedido, se molestaron con los dos hermanos. \footnote{\textbf{20:24}
  Luc 22,24-26}

\bibleverse{25} Jesús los llamó y les dijo: ``Ustedes saben que los
gobernantes extranjeros se enseñorean sobre sus pueblos, y los líderes
poderosos los oprimen. \bibleverse{26} No será así para ustedes.
Cualquiera entre ustedes que quiera ser el más importante, será siervo
de todos. \bibleverse{27} Cualquiera entre ustedes que quiera ser el
primero, será como un esclavo. \footnote{\textbf{20:27} Mar 9,35}
\bibleverse{28} De la misma manera, el Hijo del hombre no vino a que le
sirvan, sino a servir, y a dar su vida como rescate para muchos''.
\footnote{\textbf{20:28} Luc 22,27; Fil 2,7; 1Pe 1,18-19}

\hypertarget{curaciuxf3n-de-dos-ciegos-cerca-de-jericuxf3}{%
\subsection{Curación de dos ciegos cerca de
Jericó}\label{curaciuxf3n-de-dos-ciegos-cerca-de-jericuxf3}}

\bibleverse{29} Cuando se fueron de Jericó, una gran multitud siguió a
Jesús. \bibleverse{30} Dos hombres ciegos estaban sentados junto al
camino. Y cuando escucharon que Jesús iba pasando por allí, clamaron:
``¡Ten misericordia de nosotros, Señor, hijo de David!'' \bibleverse{31}
Y la multitud les decía que se callaran, pero ellos gritaban aún más
fuerte: ``¡Ten misericordia de nosotros, Señor, hijo de David!''

\bibleverse{32} Entonces Jesús se detuvo. Los llamó, preguntándoles:
``¿Qué quieren que haga por ustedes?''

\bibleverse{33} ``Señor, por favor, haz que podamos ver'', respondieron
ellos.

\bibleverse{34} Jesús tuvo compasión de ellos y tocó sus ojos. Ellos
pudieron ver de inmediato, y le siguieron.

\hypertarget{entrada-de-jesuxfas-a-jerusaluxe9n}{%
\subsection{Entrada de Jesús a
Jerusalén}\label{entrada-de-jesuxfas-a-jerusaluxe9n}}

\hypertarget{section-20}{%
\section{21}\label{section-20}}

\bibleverse{1} Entonces Jesús y sus discípulos fueron a Jerusalén.
Cuando se acercaban, llegaron a la aldea de Betfagé sobre el Monte de
los Olivos. Entonces Jesús envió a dos discípulos para que se
adelantaran, \bibleverse{2} y les dijo: ``Vayan a la aldea. Apenas
lleguen, encontrarán allí un asno amarrado junto a un pollino
Desamárrenlos y tráiganmelos. \bibleverse{3} Si alguien les pregunta qué
hacen, solo díganle: `El Señor los necesita', y ellos los enviarán de
inmediato''. \footnote{\textbf{21:3} Mat 26,18}

\bibleverse{4} Esto cumplía lo que el profeta dijo: \bibleverse{5} ``Di
a la hija de Sión: `Mira, tu rey viene hacia ti. Es humilde, y monta un
asno y un pollino la cría de un asno'\,''.\footnote{\textbf{21:5}
  Citando Isaías 62:11, Zacarías 9:9.}

\bibleverse{6} Los discípulos fueron e hicieron lo que Jesús les había
dicho. \bibleverse{7} Trajeron el asno y el pollino. Colocaron sus
mantos sobre ellos y Jesús se sentó encima. \bibleverse{8} Muchas
personas que estaban entre la multitud extendían sus mantos en el
camino, mientras que otros cortaban ramas de los árboles y las colocaban
en el camino. \bibleverse{9} Las multitudes que iban delante de él y las
que lo seguían gritaban: ``¡Hosanna\footnote{\textbf{21:9} Una palabra
  aramea que significa ``por favor, sálvanos'', y era usada como una
  exclamación de alabanza.} al hijo de David! ¡Bendito es el que viene
en el nombre del Señor! ¡Hosanna en las alturas!'' \footnote{\textbf{21:9}
  Sal 118,25-26}

\bibleverse{10} Cuando Jesús llegó a Jerusalén, toda la ciudad estaba
alborotada. ``¿Quién es este?'' preguntaban.

\bibleverse{11} ``Este es Jesús, el profeta de Nazaret, en Galilea'',
respondieron las multitudes.

\hypertarget{la-limpieza-del-templo}{%
\subsection{La limpieza del templo}\label{la-limpieza-del-templo}}

\bibleverse{12} Jesús entró al Templo, y sacó de allí a todas las
personas que estaban comprando y vendiendo. Volteó las mesas de los
cambistas y las sillas de los vendedores de palomas. \bibleverse{13}
Entonces les dijo: ``La Escritura dice: `Mi casa será llamada casa de
oración',\footnote{\textbf{21:13} Citando Isaías 56:7.} pero ustedes la
han convertido en una guarida de ladrones''.

\hypertarget{sanaciones-en-el-templo-y-homenaje-a-los-niuxf1os}{%
\subsection{Sanaciones en el templo y homenaje a los
niños}\label{sanaciones-en-el-templo-y-homenaje-a-los-niuxf1os}}

\bibleverse{14} Los ciegos y los paralíticos venían a Jesús al Templo, y
él los sanaba. \bibleverse{15} Pero cuando el jefe de los sacerdotes y
los maestros religiosos vieron los milagros asombrosos que él hacía, y a
los niños que gritaban en el Templo, ``Hosanna al hijo de David'', se
sintieron ofendidos. ``¿Escuchas lo que dicen estos niños?'' le
preguntaron. \bibleverse{16} ``Sí'', respondió Jesús. ``¿Acaso no han
leído que la Escritura dice `Preparaste a los niños y a los bebés para
ofrecerte alabanza perfecta'\,''?\footnote{\textbf{21:16} Citando Salmos
  8:2.}

\bibleverse{17} Y dejándolos allí, se fue entonces a las afueras de la
ciudad para quedarse en Betania.

\hypertarget{la-maldiciuxf3n-de-la-higuera-estuxe9ril}{%
\subsection{La maldición de la higuera
estéril}\label{la-maldiciuxf3n-de-la-higuera-estuxe9ril}}

\bibleverse{18} A la mañana siguiente, mientras caminaba de regreso a la
ciudad, Jesús sintió hambre. \bibleverse{19} Entonces vio una higuera
junto al camino, y se dirigió hacia ella pero no encontró ningún fruto,
sino solamente hojas. Entonces le dijo a la higuera: ``¡Ojalá que nunca
más puedas producir fruto!'' E inmediatamente la higuera se marchitó.
\footnote{\textbf{21:19} Luc 13,6}

\bibleverse{20} Los discípulos se asombraron al ver esto. ``¿Cómo pudo
marchitarse la higuera así de repente?'' preguntaban.

\bibleverse{21} ``Les digo la verdad'', respondió Jesús, ``Si ustedes
realmente creen en Dios, y no dudan de él, no solo podrían hacer lo que
acaba de suceder con la higuera, sino mucho más. Si ustedes dijeran a
esta montaña, `levántate y lánzate al mar', ¡así sucedería!
\bibleverse{22} Ustedes recibirán todo lo que pidan en oración, siempre
que crean en Dios''.

\hypertarget{la-pregunta-del-sumo-consejo-sobre-la-autoridad-de-jesuxfas}{%
\subsection{La pregunta del sumo consejo sobre la autoridad de
Jesús}\label{la-pregunta-del-sumo-consejo-sobre-la-autoridad-de-jesuxfas}}

\bibleverse{23} Entonces Jesús entró al Templo. Los jefes de los
sacerdotes y los ancianos del pueblo vinieron a él mientras enseñaba y
le preguntaron, ``¿Con qué autoridad haces estas cosas? ¿Quién te dio
esta autoridad?'' \footnote{\textbf{21:23} Juan 2,18; Hech 4,7}

\bibleverse{24} ``Yo también les haré una pregunta'', respondió Jesús.
``Si me responden, yo les diré con qué autoridad hago estas cosas.
\bibleverse{25} ¿Con qué autoridad bautizaba Juan? ¿Acaso su autoridad
venía del cielo, o de los hombres?'' Entonces ellos discutían unos con
otros: ``Si decimos que venía del cielo, entonces nos preguntará por qué
no creímos en él.

\bibleverse{26} Pero si decimos que venía de los hombres, entonces la
multitud se podrá en contra de nosotros,\footnote{\textbf{21:26}
  Literalmente, ``tenemos miedo de la multitud''.} porque todos ellos
consideran a Juan como un profeta''. \bibleverse{27} Entonces le
respondieron a Jesús: ``No sabemos''. ``Entonces yo no les diré con qué
autoridad hago estas cosas'', respondió Jesús.

\hypertarget{la-paruxe1bola-de-los-dos-hijos-desiguales}{%
\subsection{La parábola de los dos hijos
desiguales}\label{la-paruxe1bola-de-los-dos-hijos-desiguales}}

\bibleverse{28} ``Pero ¿qué piensan de esta ilustración? Había una vez
un hombre que tenía dos hijos. Entonces fue donde el primer hijo y le
dijo: `Hijo, ve y trabaja en la viña hoy', \bibleverse{29} Y el hijo le
respondió, `No iré', pero después se arrepintió de lo que dijo y fue.
\bibleverse{30} Luego el hombre fue donde el segundo hijo y le dijo lo
mismo. Y él le dijo: `Iré', pero no lo hizo. \footnote{\textbf{21:30}
  Mat 7,21} \bibleverse{31} ¿Cuál de los dos hijos hizo lo que su padre
quería?'' ``El primero'', respondieron ellos. ``Les digo la verdad: los
recaudadores de impuestos y las prostitutas están entrando al reino de
los cielos antes que ustedes'', les dijo Jesús. \footnote{\textbf{21:31}
  Luc 18,14}

\bibleverse{32} ``Juan vino para mostrarles a ustedes la manera correcta
de vivir con Dios, y ustedes no creyeron en él, pero los recaudadores de
impuestos y las prostitutas creyeron en él. Después, cuando vieron lo
que sucedió, ustedes tampoco se arrepintieron ni creyeron en él.
\footnote{\textbf{21:32} Luc 7,29}

\hypertarget{la-paruxe1bola-de-los-viticultores-infieles}{%
\subsection{La parábola de los viticultores
infieles}\label{la-paruxe1bola-de-los-viticultores-infieles}}

\bibleverse{33} ``Esta es otra ilustración: había una vez un hombre, un
terrateniente, que plantó una viña. Puso una cerca alrededor de ella,
hizo un lagar y construyó una torre de vigilancia. La alquiló a unos
granjeros, y luego se fue a otro país. \footnote{\textbf{21:33} Is 5,1-2}
\bibleverse{34} Cuando llegó el tiempo de la cosecha, el hombre envió a
sus siervos donde los granjeros para recoger el fruto que le pertenecía.
\bibleverse{35} Pero los granjeros atacaron a sus siervos. Golpearon a
uno, mataron a otro y a otro también lo apedrearon. \bibleverse{36}
Entonces el terrateniente envió más siervos, pero los granjeros hicieron
lo mismo con ellos. \bibleverse{37} Entonces el terrateniente envió a su
hijo. `A mi hijo lo respetarán', pensó para sí. \bibleverse{38} Pero los
granjeros, cuando vieron al hijo, se dijeron unos a otros, `¡Aquí viene
el heredero! ¡Vamos! ¡Matémoslo para quedarnos con su herencia!'
\footnote{\textbf{21:38} Mat 26,3-5; Juan 1,11} \bibleverse{39} Lo
agarraron, lo sacaron de la viña y lo mataron. \bibleverse{40} Entonces,
cuando el dueño de la viña regrese, ¿qué hará con esos granjeros?''

\bibleverse{41} Entonces los jefes de los sacerdotes le dijeron a Jesús:
``Mandará a matar a esos hombres malvados de la manera más atroz, y
alquilará la viña a otros granjeros que de seguro sí le darán su fruto
en tiempo de la cosecha''.

\bibleverse{42} ``¿Acaso no han leído las Escrituras?'' les preguntó
Jesús. ```La piedra que rechazaron los constructores se ha convertido en
la piedra angular. El Señor ha hecho esto, y es maravilloso ante
nuestros ojos'.\footnote{\textbf{21:42} Citando Salmos 118:22-23}

\bibleverse{43} Por eso les digo que a ustedes se les quitará el reino
de Dios. Será entregado a un pueblo que producirá el fruto apropiado.
\bibleverse{44} Cualquiera que tropiece con esta piedra, será destruido,
pero esta aplastará por completo a quien le caiga encima''. \footnote{\textbf{21:44}
  Dan 2,34-35; Dan 2,44-45}

\bibleverse{45} Cuando los jefes de los sacerdotes y los Fariseos
escucharon sus ilustraciones, se dieron cuenta de que Jesús estaba
hablando de ellos. \bibleverse{46} Querían arrestarlo, pero tenían miedo
de lo que el pueblo pudiera hacer porque la gente creía que él era un
profeta.

\hypertarget{la-paruxe1bola-de-la-cena-de-bodas-real}{%
\subsection{La parábola de la cena de bodas
real}\label{la-paruxe1bola-de-la-cena-de-bodas-real}}

\hypertarget{section-21}{%
\section{22}\label{section-21}}

\bibleverse{1} Jesús les habló usando más relatos ilustrados.
\bibleverse{2} ``El reino de los cielos es como un rey que organizó una
celebración de boda para su hijo'', explicó Jesús. \bibleverse{3}
``Envió a sus siervos donde todos los que estaban invitados a la boda
para decirles que vinieran, pero ellos se negaron a ir. \bibleverse{4}
Entonces envió más siervos con las siguientes instrucciones: `Díganles a
los que están invitados que he preparado un banquete de bodas. Se han
matado toros y becerros. ¡Todo está listo, así que vengan a la
boda!'\,'' \bibleverse{5} ``Pero ellos ignoraron la invitación y se
fueron. Uno se fue a sus campos; otro fue a ocuparse de su negocio.
\bibleverse{6} El resto tomó a los siervos del rey, los maltrataron, y
los mataron. \footnote{\textbf{22:6} Mat 21,35} \bibleverse{7} El rey se
puso furioso. Entonces envió a sus soldados para destruir a esos
asesinos y quemar su ciudad. \footnote{\textbf{22:7} Mat 24,2}

\bibleverse{8} ``Entonces el rey le dijo a sus siervos, `el banquete de
la boda está listo, pero los que estaban invitados no merecían asistir.
\bibleverse{9} Vayan a las calles e inviten a todos los que encuentren
para que vengan a la boda'. \footnote{\textbf{22:9} Mat 13,47}
\bibleverse{10} Así que los siervos salieron a las calles y trajeron a
todos los que pudieron encontrar, tanto buenos como malos. El salón de
la boda estaba lleno.

\bibleverse{11} ``Pero cuando el rey llegó a ver a los invitados, se dio
cuenta de que había un hombre que no tenía puesto el vestido adecuado
para la boda. \bibleverse{12} Entonces le preguntó: `amigo, ¿cómo
entraste aquí sin vestido de bodas?' El hombre no sabía qué decir.
\bibleverse{13} Entonces el rey dijo a sus siervos: `Aten sus manos y
pies y láncenlo a la oscuridad, donde habrá llanto y crujir de dientes'.
\bibleverse{14} Porque muchos son invitados, pero pocos son escogidos''.

\hypertarget{la-cuestiuxf3n-fiscal-de-los-fariseos}{%
\subsection{La cuestión fiscal de los
fariseos}\label{la-cuestiuxf3n-fiscal-de-los-fariseos}}

\bibleverse{15} Entonces los Fariseos se fueron de allí y se reunieron
para conspirar la manera en que podrían atraparlo por las cosas que
decía. \bibleverse{16} Y enviaron a algunos de sus propios discípulos
donde él junto con algunos de los seguidores de Herodes. ``Maestro,
sabemos que eres un hombre veraz, y que el camino de Dios que enseñas es
el verdadero'', comenzaron ellos. ``Tú no te dejas influir por ningún
otro, y no te preocupa el rango o la posición social. \footnote{\textbf{22:16}
  Juan 3,2} \bibleverse{17} Así que déjanos saber lo que opinas. ¿Es
correcto pagar los impuestos del César, o no?''

\bibleverse{18} Jesús sabía que sus intenciones eran malvadas. Entonces
les preguntó: ``¿Por qué están tratando de ponerme una trampa,
hipócritas? \bibleverse{19} Muéstrenme la moneda que se usa para pagar
el impuesto''. Entonces le trajeron una moneda de denario.\footnote{\textbf{22:19}
  Una moneda romana de plata que se usaba para pagar el impuesto exigido
  por los romanos.}

\bibleverse{20} ``¿De quién es la imagen y el título que está inscrito
en ella?'' les preguntó.

\bibleverse{21} ``Es del césar'', respondieron ellos. ``Ustedes deben
dar al César lo es del César, y a Dios lo que es de Dios'', les dijo.

\bibleverse{22} Cuando escucharon la respuesta de Jesús, se quedaron
asombrados. Entonces se marcharon y lo dejaron allí.

\hypertarget{sobre-la-resurrecciuxf3n-de-los-muertos}{%
\subsection{Sobre la resurrección de los
muertos}\label{sobre-la-resurrecciuxf3n-de-los-muertos}}

\bibleverse{23} Más tarde, ese mismo día, vinieron unos Saduceos a
verlo. (Los saduceos son los que dicen que no hay resurrección).
\footnote{\textbf{22:23} Hech 4,2; Hech 23,6; Hech 23,8} \bibleverse{24}
Entonces le preguntaron: ``Maestro, Moisés dijo que si un hombre
casado\footnote{\textbf{22:24} Implícito.} muere sin haber tenido hijos,
su hermano debe casarse con la viuda y así tener hijos en representación
de su hermano\footnote{\textbf{22:24} Ver Deuteronomio 25:5-6.} .
\bibleverse{25} Pues bien, supongamos que había siete hermanos. El
primero se casó y murió, y como no había tenido hijos, dejó la viuda a
su hermano. \bibleverse{26} Lo mismo ocurrió con el segundo y el tercer
esposo, hasta que llegaron al séptimo. \bibleverse{27} Al final, la
mujer también murió. \bibleverse{28} Así que cuando ocurra la
resurrección, ¿cuál de todos ellos será su esposo si ella se casó con
todos?''

\bibleverse{29} Jesús respondió: ``El error de ustedes es que no conocen
la Escritura ni lo que Dios puede hacer. \bibleverse{30} Porque en la
resurrección las personas no se casarán ni serán entregados en
matrimonio tampoco, pues en el cielo son como ángeles. \bibleverse{31}
En cuanto a la resurrección de los muertos, ¿no han leído lo que Dios
les dijo a ustedes: \bibleverse{32} `Yo soy el Dios de Abraham, el Dios
de Isaac, y el Dios de Jacob?' Él no es Dios de los muertos, sino de los
que viven''.

\bibleverse{33} Cuando las multitudes oyeron lo que dijo, se quedaron
asombrados de su enseñanza.

\hypertarget{la-pregunta-de-un-intuxe9rprete-de-la-ley-sobre-el-mandamiento-muxe1s-noble}{%
\subsection{La pregunta de un intérprete de la ley sobre el mandamiento
más
noble}\label{la-pregunta-de-un-intuxe9rprete-de-la-ley-sobre-el-mandamiento-muxe1s-noble}}

\bibleverse{34} Cuando los Fariseos oyeron que Jesús había dejado sin
palabras a los Saduceos, se reunieron y fueron a hacerle más preguntas.
\bibleverse{35} Uno de ellos, quien era un experto en la ley, le hizo
una pregunta para probarlo: \bibleverse{36} ``Maestro, ¿cuál es el
mandamiento más importante de la ley?''

\bibleverse{37} Jesús les dijo: ``\,`Ama al Señor tu Dios en todo lo que
piensas, en todo lo que sientes, y en todo lo que haces'.\footnote{\textbf{22:37}
  Citando Deuteronomio 6:5.} \bibleverse{38} Este es el mandamiento más
importante, el primer mandamiento. \bibleverse{39} El segundo es
similar: `Ama a tu prójimo como a ti mismo'.\footnote{\textbf{22:39}
  Citando Levítico 19:18.} \bibleverse{40} Toda la ley bíblica y los
escritos de los profetas dependen de estos dos mandamientos''.

\hypertarget{la-contrapregunta-de-jesuxfas-sobre-el-mesuxedas-como-hijo-de-david}{%
\subsection{La contrapregunta de Jesús sobre el Mesías como hijo de
David}\label{la-contrapregunta-de-jesuxfas-sobre-el-mesuxedas-como-hijo-de-david}}

\bibleverse{41} Mientras los fariseos estaban allí reunidos, Jesús les
hizo una pregunta: \bibleverse{42} ``¿Qué piensan ustedes del Mesías?''
les preguntó. ``¿De quién es hijo?'' ``El hijo de David'', respondieron
ellos. \footnote{\textbf{22:42} Is 11,1; Juan 7,42}

\bibleverse{43} ``¿Cómo pudo David, bajo inspiración, llamarlo
`Señor?'\,'' les preguntó Jesús. ``Él dice: \bibleverse{44} `El Señor le
dijo a mi Señor: siéntate a mi diestra hasta que derrote a todos tus
enemigos\footnote{\textbf{22:44} Literalmente, ``coloque a todos tus
  enemigos debajo de tus pies''. Citando Salmos 110:1.}'.

\bibleverse{45} Si David lo llamó Señor, ¿cómo puede el Mesías ser su
hijo?''

\bibleverse{46} Ninguno pudo responderle, y desde entonces ninguno se
atrevió a hacerle más preguntas.

\hypertarget{el-gran-discurso-de-castigo-de-jesuxfas-contra-los-escribas-y-fariseos}{%
\subsection{El gran discurso de castigo de Jesús contra los escribas y
fariseos}\label{el-gran-discurso-de-castigo-de-jesuxfas-contra-los-escribas-y-fariseos}}

\hypertarget{section-22}{%
\section{23}\label{section-22}}

\bibleverse{1} Entonces Jesús le habló a la multitud y a sus discípulos:

\hypertarget{reprimenda-por-el-comportamiento-reprobable-de-los-luxedderes-espirituales-del-pueblo-en-su-alto-cargo}{%
\subsection{Reprimenda por el comportamiento reprobable de los líderes
espirituales del pueblo en su alto
cargo}\label{reprimenda-por-el-comportamiento-reprobable-de-los-luxedderes-espirituales-del-pueblo-en-su-alto-cargo}}

\bibleverse{2} ``Los maestros religiosos y los Fariseos tienen la
responsabilidad de ser intérpretes de la ley de Moisés,\footnote{\textbf{23:2}
  Literalmente, ``se sientan en la silla de Moisés''.} \bibleverse{3}
así que obedezcan y hagan lo que ellos les digan. Pero no imiten lo que
ellos hacen, porque ellos no practican lo que predican. \footnote{\textbf{23:3}
  Mal 2,7-8; Rom 2,21-23} \bibleverse{4} Ellos colocan cargas pesadas en
los hombros del pueblo, pero ellos mismos no mueven ni un dedo para
ayudarles. \footnote{\textbf{23:4} Mat 11,28-30; Hech 15,10; Hech 15,28}
\bibleverse{5} Todo lo que hacen es con el fin de hacerse notar. Ellos
se alistan grandes cajas de oraciones\footnote{\textbf{23:5} O
  ``filacterias''. Estas eran cajas hechas con cuero que se ataban en la
  frente y los brazos y contenían textos escritos: Éxodo 13:1-6 y
  Deuteronomio 6:4-9; 11:13-21.} para usarlas y colocan largas borlas en
sus vestidos.\footnote{\textbf{23:5} Estas borlas servían para mostrar
  su devoción a Dios. Ver Números 15:37-41.} \footnote{\textbf{23:5} Mat
  6,1; Éxod 13,9; Núm 15,38-39} \bibleverse{6} Les gusta tener lugares
de honor en los banquetes y tener los mejores asientos en las sinagogas.
\footnote{\textbf{23:6} Luc 14,7} \bibleverse{7} A ellos les gusta que
los saluden con respeto en las plazas del mercado, y que la gente les
llame `Rabí'.\footnote{\textbf{23:7} 23:7 Esta es una palabra Hebrea que
  significa ``mi gran {[}señor{]}'', y se usaba como un término que
  denotaba respeto hacia los maestros religiosos.} \bibleverse{8} ``No
dejen que la gente los llame `Rabí'. El Gran Maestro de ustedes es solo
uno, y ustedes son todos hermanos. \bibleverse{9} No llamen a nadie con
el título de `Padre' aquí en la tierra. El Padre de ustedes es solo uno,
y está en el cielo. \bibleverse{10} No dejen que la gente los llame
`Maestro'. El Maestro de ustedes es solo uno, el Mesías. \bibleverse{11}
El más importante entre ustedes tendrá que ser siervo entre ustedes.
\footnote{\textbf{23:11} Mat 20,26-27} \bibleverse{12} Cualquiera que se
enaltezca a sí mismo, será humillado, y cualquiera que se humille, será
enaltecido. \footnote{\textbf{23:12} Prov 29,23; Job 22,29; Ezeq 21,31;
  Luc 18,14; 1Pe 5,5}

\hypertarget{los-siete-ayes-de-los-escribas-y-fariseos}{%
\subsection{Los siete ayes de los escribas y
fariseos}\label{los-siete-ayes-de-los-escribas-y-fariseos}}

\bibleverse{13} ``¡Pero qué desastre viene sobre ustedes, maestros
religiosos y Fariseos hipócritas! Ustedes cierran de golpe las puertas
del reino de los cielos en el rostro de la gente. No entran ustedes
mismos, ni dejan entrar a quien está tratando de hacerlo.

\bibleverse{14} \footnote{\textbf{23:14} El versículo 14 no aparece en
  los primeros manuscritos más auténticos.} \footnote{\textbf{23:14}
  Ezeq 22,25} \bibleverse{15} ¡Qué desastre viene sobre ustedes,
maestros religiosos y Fariseos hipócritas! Porque ustedes viajan por
tierra y mar para convertir a un solo individuo, y cuando lo convierten,
lo convierten dos veces más en un hijo de Gehena\footnote{\textbf{23:15}
  ``Gehenna'' (ver 5:22). El énfasis aquí está en el destino de los
  malvados.} como lo son ustedes.

\bibleverse{16} ¡Qué desastre viene sobre ustedes los que dicen: `si
juras por el Templo, no tiene importancia, pero si juras por el oro del
Templo, entonces debes cumplir tu juramento!' ¡Cuán necios y ciegos
están ustedes! \bibleverse{17} ¿Qué es más importante: el oro o el
Templo que santifica el oro? \bibleverse{18} Ustedes dicen: `si juras
sobre el altar, no tiene importancia, pero si juras sobre el sacrificio
que está sobre el altar, entonces debes cumplir tu juramento'.
\bibleverse{19} ¡Cuán ciegos están ustedes! ¿Qué es más importante: el
sacrificio, o el altar que santifica el sacrificio? \footnote{\textbf{23:19}
  Éxod 29,37} \bibleverse{20} Si ustedes juran por el altar, están
jurando por el altar y por todo lo que está sobre él. \bibleverse{21} Si
juran por el Templo, están jurando por el Templo y por Aquél que vive
allí. \bibleverse{22} Si juran por el cielo, están jurando por el trono
de Dios y por Aquél que se sienta en él.

\bibleverse{23} ``¡Qué desastre viene sobre ustedes, maestros religiosos
y Fariseos hipócritas! Pagan el diezmo de la menta, de la semilla de
anís y del comino, pero son negligentes en los aspectos vitales de la
ley: hacer lo correcto, mostrar misericordia, ejercer la fe. Sí, es
cierto que deben pagar sus diezmos, pero no olviden estas otras cosas.
\bibleverse{24} ¡Ustedes son guías ciegos que cuelan la bebida para no
dejar pasar una mosca, pero se tragan un camello!

\bibleverse{25} ``¡Qué desastre viene sobre ustedes, maestros religiosos
y fariseos hipócritas! Limpian el exterior de la taza y del plato, pero
por dentro ustedes están llenos de glotonería y autocomplacencia.
\footnote{\textbf{23:25} Mar 7,4; Mar 7,8} \bibleverse{26} ¡Fariseos
ciegos! Limpien primero el interior de la taza y del plato, para que
entonces el exterior esté limpio también. \footnote{\textbf{23:26} Juan
  9,40; Tit 1,15}

\bibleverse{27} ``¡Qué desastre viene sobre ustedes, maestros religiosos
y Fariseos hipócritas! Son como sepulcros blanqueados, que se ven bien
por fuera, pero por dentro están llenos de esqueletos y todo tipo de
putrefacción.\footnote{\textbf{23:27} Literalmente, ``inmundicia''.}
\bibleverse{28} Ustedes son simplemente una vergüenza. Por fuera parecen
buenas personas, pero por dentro están llenos de hipocresía y maldad.

\bibleverse{29} ``¡Qué desastre viene sobre ustedes, maestros religiosos
y fariseos hipócritas! Construyen sepulcros en memoria de los profetas,
y decoran las tumbas de los buenos, \bibleverse{30} y dicen: `si
hubiéramos vivido en los tiempos de nuestros ancestros, no habríamos
participado en el derramamiento de la sangre de los profetas'.
\bibleverse{31} ¡Pero al decir esto testifican contra ustedes mismos,
demostrando que hacen parte de esos que mataron a los profetas!
\footnote{\textbf{23:31} Mat 5,12; Hech 7,52} \bibleverse{32} ¡Entonces
sigan y acaben la obra de una vez por todas usando los métodos de sus
antepasados! \bibleverse{33} Serpientes, camada de víboras, ¿cómo
escaparán del juicio de Gehena?\footnote{\textbf{23:33} ``Gehenna'' (ver
  la nota del versículo 5:22). Hace referencia al juicio del fin de los
  tiempos.}

\hypertarget{amenaza-contra-las-personas-manchadas-de-sangre-que-se-resisten-a-su-salvaciuxf3n}{%
\subsection{Amenaza contra las personas manchadas de sangre que se
resisten a su
salvación}\label{amenaza-contra-las-personas-manchadas-de-sangre-que-se-resisten-a-su-salvaciuxf3n}}

\bibleverse{34} ``Por eso yo les envío profetas, hombres sabios y
maestros. A algunos los matarán, a otros los crucificarán, y a otros los
azotarán en las sinagogas, y los perseguirán de ciudad en ciudad.
\bibleverse{35} Como consecuencia de ello, ustedes tendrán que dar
cuenta de la sangre de todas las personas buenas que se ha derramado
sobre la tierra: desde la sangre de Abel, que hizo lo correcto, hasta la
sangre de Zacarías, el hijo de Berequías, a quien ustedes mataron entre
el Templo y el altar. \bibleverse{36} ``Yo les digo que las
consecuencias de todo esto caerán sobre esta generación.

\hypertarget{salida-de-jesuxfas-de-la-ciudad-de-jerusaluxe9n-y-anuncio-de-su-regreso}{%
\subsection{Salida de Jesús de la ciudad de Jerusalén y anuncio de su
regreso}\label{salida-de-jesuxfas-de-la-ciudad-de-jerusaluxe9n-y-anuncio-de-su-regreso}}

\bibleverse{37} ¡Oh Jerusalén, Jerusalén, tu matas a los profetas y
apedreas a los que se te envían! Tantas veces he querido reunir a tus
hijos así como una gallina reúne a sus polluelos bajo sus alas, pero no
me dejaste. \bibleverse{38} Ahora mira, tu casa\footnote{\textbf{23:38}
  La palabra ``Casa'' puede referirse al Templo.} ha sido abandonada, y
está completamente vacía. \footnote{\textbf{23:38} 1Re 9,7-8}
\bibleverse{39} Yo te digo esto: no me volverás a ver hasta que digas:
`Bendito es el que viene en el nombre del
Señor'''.\textsuperscript{{[}\textbf{23:39} Citando Salmos
118:26.{]}}{[}\textbf{23:39} Mat 21,9; Mat 26,64{]}

\hypertarget{el-monte-de-los-olivos-de-jesuxfas-a-sus-discuxedpulos-desde-la-destrucciuxf3n-del-templo-desde-el-fin-de-este-mundo-desde-su-segunda-venida-y-desde-el-juicio-sobre-los-pueblos}{%
\subsection{El monte de los Olivos de Jesús a sus discípulos desde la
destrucción del templo, desde el fin de este mundo, desde su segunda
venida y desde el juicio sobre los
pueblos}\label{el-monte-de-los-olivos-de-jesuxfas-a-sus-discuxedpulos-desde-la-destrucciuxf3n-del-templo-desde-el-fin-de-este-mundo-desde-su-segunda-venida-y-desde-el-juicio-sobre-los-pueblos}}

\hypertarget{section-23}{%
\section{24}\label{section-23}}

\bibleverse{1} Cuando Jesús iba saliendo del Templo, sus discípulos
venían hacia donde él estaba y mostraban con orgullo los edificios del
Templo. \bibleverse{2} Pero Jesús respondió: ``¿Ven todos estos
edificios? Les digo la verdad: no quedará piedra sobre piedra. ¡Cada una
de las piedras que queden serán derribadas!'' \footnote{\textbf{24:2}
  Luc 19,44}

\bibleverse{3} Cuando Jesús se sentó en el Monte de los Olivos, los
discípulos vinieron donde él estaba y en privado le preguntaron: ``Por
favor, dinos cuándo ocurrirá esto. ¿Cuál será la señal de tu venida y
del fin del mundo?'' \footnote{\textbf{24:3} Hech 1,6-8}

\hypertarget{el-fin-de-este-tiempo-mundial-las-primeras-seuxf1ales}{%
\subsection{El fin de este tiempo mundial; Las primeras
señales}\label{el-fin-de-este-tiempo-mundial-las-primeras-seuxf1ales}}

\bibleverse{4} ``Asegúrense de que nadie los confunda'', respondió
Jesús. \bibleverse{5} ``Muchos vendrán diciendo que soy yo, y dirán `yo
soy el Mesías', y engañarán a muchas personas. \footnote{\textbf{24:5}
  Juan 5,43; 1Jn 2,18} \bibleverse{6} Ustedes oirán de guerras de y
rumores de guerras, pero no estén ansiosos. Estas cosas tienen que
pasar, pero este no es el fin. \bibleverse{7} Habrá naciones que
atacarán a otras naciones, y reinos que pelearán contra otros reinos.
Habrá hambrunas y terremotos en diferentes lugares, \bibleverse{8} pero
todas estas cosas son solo el principio de los dolores del parto.

\hypertarget{las-persecuciones-de-los-discuxedpulos}{%
\subsection{Las persecuciones de los
discípulos}\label{las-persecuciones-de-los-discuxedpulos}}

\bibleverse{9} ``Entonces a ustedes los arrestarán, los perseguirán y
los matarán. Todas las personas los odiarán por mi causa.
\bibleverse{10} En ese tiempo muchos que eran creyentes dejarán de
creer. Se entregarán unos a otros con traición y se odiarán unos a
otros. \bibleverse{11} Muchos falsos profetas vendrán y engañarán a
muchas personas. \footnote{\textbf{24:11} 2Pe 2,1; 1Jn 4,1}
\bibleverse{12} El aumento del mal hará que el amor de muchos se enfríe,
\footnote{\textbf{24:12} 2Tim 3,1-5} \bibleverse{13} pero aquellos que
se mantengan firmes hasta el fin serán salvos. \footnote{\textbf{24:13}
  Apoc 13,10} \bibleverse{14} La buena noticia del reino será proclamada
en todo el mundo de tal modo que todos la escucharán, y entonces vendrá
el fin. \footnote{\textbf{24:14} Mat 28,19}

\hypertarget{el-cluxedmax-de-la-tribulaciuxf3n-en-judea}{%
\subsection{El clímax de la tribulación en
Judea}\label{el-cluxedmax-de-la-tribulaciuxf3n-en-judea}}

\bibleverse{15} Así que cuando vean el `mal que profana'\footnote{\textbf{24:15}
  O, ``la abominación desoladora'', refiriéndose a Daniel 9:27, Daniel
  11:31, Daniel 12:11.} en el lugar santo del cual habló el profeta
Daniel (los que leen esto, por favor, examínenlo cuidadosamente),
\bibleverse{16} entonces las personas que viven en Judea, deben huir a
las montañas. \bibleverse{17} Todo el que esté en el tejado de la casa
no debe descender para buscar lo que hay en ella. \footnote{\textbf{24:17}
  Luc 17,31} \bibleverse{18} El que esté en los campos, no debe regresar
a buscar el abrigo. \bibleverse{19} ¡Cuán terrible será para aquellas
que estén embarazadas y para las que estén amamantando a sus bebés en
esos días! \bibleverse{20} Oren para que no tengan que huir en invierno,
o en día Sábado. \bibleverse{21} Porque en ese tiempo, habrá una
persecución terrible, más terrible que cualquier cosa que haya ocurrido
desde el principio del mundo hasta ahora, ni ocurrirá jamás. \footnote{\textbf{24:21}
  Dan 12,1} \bibleverse{22} A menos que esos días sean acortados, nadie
será salvo, pero por el bien de los elegidos, esos días serán acortados.

\hypertarget{profecuxeda-de-los-falsos-profetas}{%
\subsection{Profecía de los falsos
profetas}\label{profecuxeda-de-los-falsos-profetas}}

\bibleverse{23} ``Así que si alguien les dice: `miren, este es el
Mesías,' o, `allá está,' no lo crean. \bibleverse{24} Porque aparecerán
falsos mesías y falsos profetas también, y harán señales y milagros
increíbles para engañar a los escogidos, si fuera posible.

\bibleverse{25} Noten que les he dicho esto antes de que siquiera
ocurra.

\bibleverse{26} De modo que si les dicen: `miren, está en el desierto,'
no vayan a verlo allá; o si dicen: `miren, está oculto aquí,' no lo
crean. \bibleverse{27} Porque la venida del Hijo del hombre será como el
relámpago que ilumina desde el oriente hasta el occidente. \footnote{\textbf{24:27}
  Luc 17,23-24} \bibleverse{28} `Los buitres se amontonan donde está el
cadáver'. \footnote{\textbf{24:28} Job 39,30; Luc 17,37; Apoc 19,17-18}

\hypertarget{los-uxfaltimos-augurios-y-la-segunda-venida-del-hijo-del-hombre-con-los-fenuxf3menos-que-los-acompauxf1an}{%
\subsection{Los últimos augurios y la segunda venida del Hijo del Hombre
con los fenómenos que los
acompañan}\label{los-uxfaltimos-augurios-y-la-segunda-venida-del-hijo-del-hombre-con-los-fenuxf3menos-que-los-acompauxf1an}}

\bibleverse{29} ``Pero justo después de estos días de persecución, el
sol se oscurecerá, la luna no brillará, las estrellas caerán del cielo,
y las potencias del cielo se conmoverán. \footnote{\textbf{24:29} Is
  13,10; 2Pe 3,10; Apoc 6,12-13} \bibleverse{30} Entonces aparecerá en
el cielo la señal del Hijo del hombre, y todos los pueblos de la tierra
se lamentarán. Verán al Hijo del hombre viniendo sobre las nubes del
cielo con poder y gran gloria.\footnote{\textbf{24:30} Véase Daniel
  7:13-14.} \footnote{\textbf{24:30} Mat 26,64; Dan 7,13-14; Apoc 1,7;
  Apoc 19,11-13} \bibleverse{31} Con el toque de una trompeta él enviará
a sus ángeles para reunir a sus escogidos de todas partes, desde un
confín del cielo y de la tierra hasta el otro\footnote{\textbf{24:31}
  Literalmente, ``de los cuatro vientos, desde un extremo del cielo
  hasta el otro''.} . \footnote{\textbf{24:31} 1Cor 15,52; Apoc 8,1-2}

\bibleverse{32} ``Aprendan una ilustración de la higuera. Cuando sus
brotes se vuelven más blandos y comienzan a salir las hojas, ya ustedes
saben que se acerca el verano. \bibleverse{33} De la misma manera,
cuando vean que están ocurriendo todas estas cosas, ya sabrán que su
venida está cerca, ¡de hecho, está justo en la puerta! \bibleverse{34}
Les digo la verdad: esta generación no morirá hasta que todas estas
cosas hayan pasado. \bibleverse{35} El cielo y la tierra podrán perecer,
pero mis palabras no morirán.

\bibleverse{36} ``Pero nadie sabe el día ni la hora en que esto
ocurrirá, ni siquiera los ángeles en el cielo, ni el Hijo. Solo el Padre
lo sabe. \footnote{\textbf{24:36} Hech 1,7} \bibleverse{37} Cuando el
Hijo del hombre venga, será como en los días de Noé. \footnote{\textbf{24:37}
  Gén 6,11-13; Luc 17,26-27} \bibleverse{38} Será como en los días antes
del diluvio, donde todos comían y bebían y se casaban y se entregaban en
matrimonio, hasta el día que Noé entró al arca. \bibleverse{39} Ellos no
se dieron cuenta de lo que estaba ocurriendo hasta que el diluvio vino y
se los llevó a todos. Así será la venida del Hijo del hombre.
\bibleverse{40} ``Dos hombres estarán trabajando en los campos. Se
tomará a uno y se dejará al otro. \footnote{\textbf{24:40} Luc 17,35-36}
\bibleverse{41} Dos mujeres estarán moliendo grano en un molino. Se
tomará a una y se dejará a la otra.

\hypertarget{advertencia-de-estar-alerta-en-general}{%
\subsection{Advertencia de estar alerta en
general}\label{advertencia-de-estar-alerta-en-general}}

\bibleverse{42} Así que estén prevenidos, porque ustedes no saben qué
día viene el Señor. \bibleverse{43} Pero consideren esto: si el dueño de
la casa supiera a qué hora vendrá el ladrón, permanecería vigilando. No
dejaría que entre y robe en su casa. \bibleverse{44} Ustedes también
necesitan estar listos, porque el Hijo del hombre viene en un momento en
que ustedes no lo esperan. \footnote{\textbf{24:44} 1Tes 5,2}

\hypertarget{paruxe1bola-del-siervo-fiel-y-del-infiel}{%
\subsection{Parábola del siervo fiel y del
infiel}\label{paruxe1bola-del-siervo-fiel-y-del-infiel}}

\bibleverse{45} ``Pues ¿quién es el siervo fiel y considerado? Es el que
su amo pone a cargo de la familia para que provea el alimento en el
momento adecuado. \bibleverse{46} ¡Cuán bueno es que el siervo se
encuentre haciendo esto cuando su amo regrese! \bibleverse{47} Les digo
la verdad: el amo pondrá a ese siervo a cargo de todas sus posesiones.
\bibleverse{48} Pero si fuese un siervo malo, diría para sí mismo: `mi
señor se está demorando', \footnote{\textbf{24:48} 2Pe 3,4}

\bibleverse{49} y comenzaría a golpear a los otros siervos, a festejar y
a beber con los borrachos. \bibleverse{50} Entonces el amo de ese siervo
regresará cuando este no lo espera, en un momento que no sabe.
\bibleverse{51} Entonces el amo lo hará pedazos, y lo tratará como a los
hipócritas\footnote{\textbf{24:51} Los que dicen que siguen a su Señor
  pero en realidad no lo hacen.} , enviándolo a un lugar donde hay
lamento y crujir de dientes''.

\hypertarget{la-paruxe1bola-de-las-diez-vuxedrgenes-prudentes-y-necias}{%
\subsection{La parábola de las diez vírgenes prudentes y
necias}\label{la-paruxe1bola-de-las-diez-vuxedrgenes-prudentes-y-necias}}

\hypertarget{section-24}{%
\section{25}\label{section-24}}

\bibleverse{1} ``El reino de los cielos es como diez jovencitas, que
llevaron sus lámparas para ir al encuentro del novio. \bibleverse{2}
Cinco de ellas eran necias, y cinco eran sabias. \bibleverse{3} Las
jóvenes necias llevaron sus lámparas pero no llevaron aceite,
\bibleverse{4} mientras que las sabias llevaron frascos de aceite junto
con sus lámparas. \bibleverse{5} El novio se demoró mucho y todas las
jóvenes comenzaron a sentirse somnolientas y se durmieron.
\bibleverse{6} A la media noche se escuchó el grito: `¡Miren aquí está
el novio! ¡Vengan a su encuentro!' \bibleverse{7} Todas las jovencitas
se levantaron y cortaron la mecha de sus lámparas. Las jóvenes necias le
dijeron a las jóvenes sabias: \bibleverse{8} `Dénnos un poco de su
aceite porque nuestras lámparas se están apagando'. Pero las jovencitas
sabias respondieron: \bibleverse{9} `No, porque así no habrá suficiente
aceite para ustedes ni para nosotras. Vayan a las tiendas y compren
aceite para ustedes'. \bibleverse{10} Mientras fueron a comprar el
aceite, llegó el novio y los que estaban listos entraron con él a la
boda, y la puerta se cerró con llave. \bibleverse{11} Las otras jóvenes
llegaron más tarde. `Señor, Señor', llamaron, `¡ábrenos la puerta!'
\footnote{\textbf{25:11} Luc 13,25; Luc 13,27} \bibleverse{12} Pero él
respondió: `En verdad les digo que no las conozco'. \footnote{\textbf{25:12}
  Mat 7,23} \bibleverse{13} Así que estén alerta, porque ustedes no
saben el día ni la hora. \footnote{\textbf{25:13} Mat 24,42}

\hypertarget{paruxe1bola-de-los-talentos-confiados}{%
\subsection{Parábola de los talentos
confiados}\label{paruxe1bola-de-los-talentos-confiados}}

\bibleverse{14} ``Es como un hombre que se fue de viaje. Llamó a sus
siervos y los dejó a cargo de sus posesiones. \bibleverse{15} A uno de
ellos le entregó cinco talentos,\footnote{\textbf{25:15} Refiriéndose
  probablemente a talentos de plata, una gran cantidad de dinero.} a
otro le dio dos, y a otro le dio uno, conforme a sus capacidades. Luego
se fue. \bibleverse{16} De inmediato, el que tenía cinco talentos fue y
los invirtió en un negocio, y obtuvo otros cinco talentos.
\bibleverse{17} De la misma manera, el que tenía dos talentos obtuvo
otros dos. \bibleverse{18} Pero el que recibió un talento se fue y cavó
un hoyo y escondió allí el dinero de su amo.

\bibleverse{19} Mucho tiempo después, el amo de estos siervos regresó y
se dispuso a ajustar cuentas con ellos. \bibleverse{20} El que recibió
cinco talentos vino y presentó otros cinco talentos. `Mi señor', le
dijo, `me diste cinco talentos. Mira, obtuve una ganancia de cinco
talentos más'.

\bibleverse{21} Su amo le dijo: `Has hecho bien, eres un siervo bueno y
fiel. Has demostrado que eres fiel en cosas pequeñas, así que ahora te
colocaré a cargo de muchas cosas. ¡Alégrate porque estoy muy complacido
de ti!' \footnote{\textbf{25:21} Mat 24,45-47}

\bibleverse{22} El siervo que recibió dos talentos también vino. `Mi
señor', le dijo, `me entregaste dos talentos. Mira, he obtenido una
ganancia de dos talentos más'.

\bibleverse{23} Su amo le dijo: `Has hecho bien, eres un siervo bueno y
fiel. Has demostrado que eres fiel en cosas pequeñas, así que ahora te
pondré a cargo de muchas cosas. ¡Alégrate porque estoy muy complacido de
ti!'

\bibleverse{24} ``Entonces vino el hombre que recibió un talento. `Mi
señor', le dijo, `sé que eres un hombre duro. Siegas donde no sembraste
y recoges cosechas que no plantaste. \bibleverse{25} Así que como tuve
miedo de ti fui y enterré tu talento. Mira, aquí tienes lo que te
pertenece'.

\bibleverse{26} Pero su amo le respondió: `¡Eres un siervo malo y
perezoso! Si crees que siego donde no sembré, y que recojo cosechas que
no planté, \bibleverse{27} entonces debiste depositar en el banco la
plata que me pertenece y así yo habría recibido mi dinero con intereses
al regresar. \bibleverse{28} Quítenle el talento que tiene y dénselo al
que tiene diez talentos. \bibleverse{29} Porque al que tiene se le dará
aún más; y al que no tiene nada, incluso lo que tenga se le quitará.
\bibleverse{30} Ahora lancen a este siervo inútil en la oscuridad donde
habrá llanto y crujir de dientes'.

\hypertarget{el-juicio-de-jesuxfas-sobre-los-pueblos-y-las-personas-la-separaciuxf3n-de-las-ovejas-de-las-cabras}{%
\subsection{El juicio de Jesús sobre los pueblos y las personas; la
separación de las ovejas de las
cabras}\label{el-juicio-de-jesuxfas-sobre-los-pueblos-y-las-personas-la-separaciuxf3n-de-las-ovejas-de-las-cabras}}

\bibleverse{31} ``Pero cuando el Hijo del hombre venga en su gloria, y
todos los ángeles con él, se sentará en su trono majestuoso. \footnote{\textbf{25:31}
  Mat 16,27; Apoc 20,11-13} \bibleverse{32} Traerán a todos delante de
él. Entonces él separará a los unos de los otros, así como un pastor
separa a las ovejas de los cabritos. \footnote{\textbf{25:32} Mat 13,49;
  Rom 14,10} \bibleverse{33} Entonces colocará a las ovejas a su
derecha, y a los cabritos en su mano izquierda. \footnote{\textbf{25:33}
  Ezeq 34,17} \bibleverse{34} Entonces el rey dirá a los de su derecha:
`vengan ustedes, benditos de mi Padre, hereden el reino que ha sido
preparado para ustedes desde el principio del mundo. \bibleverse{35}
Porque tuve hambre y me dieron alimento para comer. Tuve sed, y me
dieron de beber. Fui forastero y me hospedaron. \bibleverse{36} Estuve
desnudo y me vistieron. Estuve enfermo y cuidaron de mí. Estuve en la
cárcel y me visitaron'.

\bibleverse{37} Entonces los de la derecha responderán: `Señor, ¿cuándo
te vimos con hambre y te alimentamos, o sediento y te dimos de beber?
\footnote{\textbf{25:37} Mat 6,3} \bibleverse{38} ¿Cuándo te vimos como
forastero y te hospedamos, o desnudo y te vestimos? \bibleverse{39}
¿Cuándo te vimos enfermo o en la cárcel y te visitamos?'

\bibleverse{40} El rey les dirá: `en verdad les digo que todo lo que
hicieron por uno de estos de menor importancia, lo hicieron por mi'.
\bibleverse{41} ``También dirá a los de su izquierda: `¡apártense de mi,
ustedes malditos, vayan al fuego eterno\footnote{\textbf{25:41} Eterno
  en consecuencia, no en duración.} preparado para el diablo y sus
ángeles! \footnote{\textbf{25:41} Apoc 20,10; Apoc 20,15}
\bibleverse{42} Porque tuve hambre y no me dieron nada de comer. Tuve
sed y no me dieron de beber. \bibleverse{43} Fui forastero y no me
hospedaron. Estuve desnudo y no me vistieron. Estuve enfermo y en la
cárcel y no me visitaron'.

\bibleverse{44} Entonces ellos también responderán: `Señor, ¿cuándo te
vimos con hambre, con sed, o como forastero, o desnudo, o enfermo, o en
la cárcel y no cuidamos de ti?'

\bibleverse{45} Entonces él les dirá: `en verdad les digo que todo lo
que no hicieron por uno de estos de menor importancia, no lo hicieron
por mi'. \bibleverse{46} Ellos se irán a la condenación eterna, pero los
justos entrarán a la vida eterna''.

\hypertarget{uxfaltimo-anuncio-del-sufrimiento-de-jesuxfas-intento-de-asesinato-por-parte-de-los-luxedderes-del-pueblo}{%
\subsection{Último anuncio del sufrimiento de Jesús; Intento de
asesinato por parte de los líderes del
pueblo}\label{uxfaltimo-anuncio-del-sufrimiento-de-jesuxfas-intento-de-asesinato-por-parte-de-los-luxedderes-del-pueblo}}

\hypertarget{section-25}{%
\section{26}\label{section-25}}

\bibleverse{1} Después que hubo dicho todo esto, Jesús le dijo a los
discípulos: \bibleverse{2} ``Ustedes saben que en dos días es la Pascua,
y el Hijo del hombre será entregado y crucificado''. \footnote{\textbf{26:2}
  Mat 20,18; Éxod 12,1-20}

\bibleverse{3} Entonces los jefes de los sacerdotes y los ancianos del
pueblo se reunieron en el patio de Caifás, el sumo sacerdote.
\footnote{\textbf{26:3} Luc 3,1-2} \bibleverse{4} Allí conspiraron para
arrestar a Jesús bajo algún pretexto engañoso\footnote{\textbf{26:4}
  Literalmente, ``con una artimaña''.} y matarlo. \bibleverse{5} Pero
dijeron: ``no hagamos esto durante el festival para que no haya
disturbios en el pueblo''.

\hypertarget{unciuxf3n-de-jesuxfas-en-betania}{%
\subsection{Unción de Jesús en
Betania}\label{unciuxf3n-de-jesuxfas-en-betania}}

\bibleverse{6} Mientras Jesús estaba en la casa de Simón el leproso, en
Betania, \bibleverse{7} vino una mujer que traía un frasco de alabastro
que contenía un perfume muy costoso. Ella lo derramó en la cabeza de
Jesús mientras él estaba sentado y comía. Pero cuando los discípulos
vieron lo que ella hizo, se incomodaron por ello. \bibleverse{8} ``¡Qué
gran desperdicio!'' objetaron. \bibleverse{9} ``¡Este perfume pudo
haberse vendido por mucho dinero y lo habríamos regalado a los pobres!''

\bibleverse{10} Jesús sabía lo que estaba pasando y les dijo: ``¿Por qué
están enojados con esta mujer? ¡Ella ha hecho algo maravilloso por mí!
\bibleverse{11} Los pobres siempre estarán entre ustedes,\footnote{\textbf{26:11}
  Ver Deuteronomio 15:11.} pero no siempre me tendrán a mí. \footnote{\textbf{26:11}
  Deut 15,11} \bibleverse{12} Al derramar este perfume en mi cuerpo,
ella me ha preparado para mi sepultura. \bibleverse{13} Les digo la
verdad: dondequiera que se difunda esta buena noticia, se contará lo que
esta mujer ha hecho, en memoria de ella''.

\hypertarget{traiciuxf3n-de-judas}{%
\subsection{Traición de Judas}\label{traiciuxf3n-de-judas}}

\bibleverse{14} Entonces Judas Iscariote, uno de los doce discípulos,
fue donde estaban los jefes de los sacerdotes \bibleverse{15} y les
preguntó: ``¿Cuánto me pagarán por entregarles a Jesús?'' Y ellos le
pagaron treinta monedas de plata. \bibleverse{16} A partir de ese
momento, Judas buscaba una oportunidad para entregar a Jesús.

\hypertarget{preparaciuxf3n-de-la-comida-pascual}{%
\subsection{Preparación de la comida
pascual}\label{preparaciuxf3n-de-la-comida-pascual}}

\bibleverse{17} El primer día del festival del pan sin levadura, los
discípulos vinieron donde Jesús y le preguntaron: ``¿Dónde quieres que
preparemos la cena de la Pascua para ti?'' \footnote{\textbf{26:17} Éxod
  12,18-20}

\bibleverse{18} Jesús les dijo: ``vayan a la ciudad y busquen a cierto
hombre que está ahí y díganle que el Maestro dice: `Se acerca mi hora.
Voy a celebrar la Pascua con mis discípulos en tu casa'\,''. \footnote{\textbf{26:18}
  Mat 21,3}

\bibleverse{19} Entonces los discípulos hicieron lo que Jesús les dijo,
y prepararon allí la cena de la Pascua.

\hypertarget{la-uxfaltima-cena-de-jesuxfas-en-el-cuxedrculo-de-los-discuxedpulos-exposiciuxf3n-de-la-traiciuxf3n-de-judas-instituciuxf3n-de-la-santa-comuniuxf3n}{%
\subsection{La última cena de Jesús en el círculo de los discípulos;
Exposición de la traición de Judas; Institución de la santa
comunión}\label{la-uxfaltima-cena-de-jesuxfas-en-el-cuxedrculo-de-los-discuxedpulos-exposiciuxf3n-de-la-traiciuxf3n-de-judas-instituciuxf3n-de-la-santa-comuniuxf3n}}

\bibleverse{20} Cuando llegó la noche, Jesús se sentó allí a comer con
los doce. \bibleverse{21} Mientras comían, les dijo: ``En verdad les
digo que uno de ustedes va a entregarme''.

\bibleverse{22} Ellos estaban extremadamente incómodos. Uno por uno le
preguntaban: ``Señor, no soy yo, ¿cierto?''

\bibleverse{23} ``El que ha metido su mano conmigo en el plato, me
entregará'', respondió Jesús. \bibleverse{24} ``El Hijo del hombre
morirá tal como fue profetizado acerca de él, pero ¡qué desgracia vendrá
sobre el hombre que entregue al Hijo del hombre! ¡Habría sido mejor que
nunca hubiera nacido!'' \footnote{\textbf{26:24} Luc 17,1}

\bibleverse{25} Judas, el que lo iba a entregar, preguntó ``¿Seré yo,
Rabí?'' ``Tu lo has dicho'', respondió Jesús.

\bibleverse{26} Mientras comían, Jesús tomó del pan y lo bendijo.
Entonces lo partió y lo repartió entre los discípulos. ``Tomen este pan
y cómanlo porque este es mi cuerpo'', dijo Jesús. \bibleverse{27}
Entonces cogió la copa, la bendijo y se la entregó a ellos. ``Tomen
todos de esta copa'', les dijo. \bibleverse{28} ``Porque esta es mi
sangre del pacto, derramada por muchos para el perdón de pecados.
\footnote{\textbf{26:28} Éxod 24,8; Jer 31,31; Heb 9,15-16}
\bibleverse{29} Sin embargo, les digo, yo no beberé más de este fruto de
la vid hasta el día en que vuelva a beberlo nuevamente con ustedes en el
reino de mi Padre''.

\hypertarget{camina-a-getsemanuxed}{%
\subsection{Camina a Getsemaní}\label{camina-a-getsemanuxed}}

\bibleverse{30} Después que terminaron de cantar, se fueron al Monte de
los Olivos.

\bibleverse{31} ``Todos ustedes me abandonarán esta noche'', les dijo
Jesús. ``Como dice la Escritura: `Yo golpearé al pastor, y el rebaño
estará completamente disperso'.\footnote{\textbf{26:31} Citando Zacarías
  13:7.} \footnote{\textbf{26:31} Juan 16,32} \bibleverse{32} Pero
después que me haya levantado, yo iré delante de ustedes a Galilea''.
\footnote{\textbf{26:32} Mat 28,7}

\bibleverse{33} Pero Pedro objetó: ``incluso si todos los demás te
abandonan, yo nunca te abandonaré''.

\bibleverse{34} ``Te digo la verdad'', le dijo Jesús, ``esta misma
noche, antes de que el gallo cante, me negarás tres veces''. \footnote{\textbf{26:34}
  Juan 13,18}

\bibleverse{35} ``¡Aun si tengo que morir contigo, nunca te negaré!''
insistió Pedro. Y todos los discípulos dijeron lo mismo.

\hypertarget{el-conflicto-y-la-oraciuxf3n-de-jesuxfas-en-getsemanuxed-debilidad-de-los-discuxedpulos}{%
\subsection{El conflicto y la oración de Jesús en Getsemaní; Debilidad
de los
discípulos}\label{el-conflicto-y-la-oraciuxf3n-de-jesuxfas-en-getsemanuxed-debilidad-de-los-discuxedpulos}}

\bibleverse{36} Entonces Jesús se fue con sus discípulos a un lugar
llamado Getsemaní. Les dijo: ``Siéntense aquí mientras yo voy allá a
orar''. \bibleverse{37} Entonces llevó consigo a Pedro y a los dos hijos
de Zebedeo, y comenzó a sufrir tristeza y aflicción agonizantes.
\bibleverse{38} Entonces les dijo: ``Estoy tan inundado de tristeza, que
siento morir. Esperen aquí y estén en vigilia conmigo''. \footnote{\textbf{26:38}
  Juan 12,27}

\bibleverse{39} Entonces se fue un poco más lejos, se postró sobre su
rostro y oró. ``Padre mío, por favor, si es posible, quítame esta copa
de sufrimiento'', pidió Jesús. ``Aun así, que no sea lo que yo quiero
sino lo que tu quieres''. \footnote{\textbf{26:39} Juan 6,38; Juan
  18,11; Heb 5,8}

\bibleverse{40} Entonces regresó donde estaban los discípulos y los
encontró dormidos. Le dijo entonces a Pedro: ``¿Cómo es que no pudieron
estar despiertos conmigo apenas una hora? \bibleverse{41} Estén
despiertos y oren, para que no caigan en tentación. Sí, el espíritu está
dispuesto, pero el cuerpo es débil''. \footnote{\textbf{26:41} Efes
  6,18; Heb 2,18}

\bibleverse{42} Entonces se fue por segunda vez y oró. ``Padre mío, si
no puedes quitarme esta copa sin que yo la beba, entonces se hará tu
voluntad'', dijo.

\bibleverse{43} Regresó entonces y encontró a los discípulos durmiendo,
porque no pudieron mantenerse despiertos.\footnote{\textbf{26:43}
  Literalmente, ``sus ojos estaban pesados''.} \bibleverse{44} Entonces
los dejó allí una vez más y se fue y oró por tercera vez, repitiendo las
mismas cosas. \bibleverse{45} Entonces regresó donde estaban sus
discípulos, y les dijo: ``¿Cómo es posible que aún estén durmiendo y
descansando? Miren, el momento ha llegado. ¡El Hijo del hombre está a
punto de ser entregado en manos de pecadores! \bibleverse{46}
¡Levántense, vámonos! Miren, acaba de llegar el que me entrega''.

\hypertarget{encarcelamiento-de-jesuxfas-escape-de-los-discuxedpulos}{%
\subsection{Encarcelamiento de Jesús; Escape de los
discípulos}\label{encarcelamiento-de-jesuxfas-escape-de-los-discuxedpulos}}

\bibleverse{47} Cuando dijo esto, Judas, uno de los doce, llegó con una
gran turba que estaba armada con espadas y palos, y habían sido enviados
por los jefes de los sacerdotes y por los ancianos del pueblo.
\bibleverse{48} El traidor había acordado que les daría una señal: ``Al
que yo bese, ese es\ldots{} ¡arréstenlo'', les dijo. \bibleverse{49}
Judas llegó inmediatamente donde estaba Jesús y dijo: ``Hola, Rabí'', y
lo besó.

\bibleverse{50} ``Amigo mío, haz lo que viniste a hacer'', le dijo Jesús
a Judas. Entonces vinieron y tomaron a Jesús y lo arrestaron.

\bibleverse{51} Uno de los que estaban con Jesús alcanzó su espada y la
sacó. Atacó con ella al siervo del sumo sacerdote, cortándole la oreja.

\bibleverse{52} Pero Jesús le dijo: ``Guarda tu espada. Todo el que
pelea con una espada, morirá a espada. \footnote{\textbf{26:52} Gén 9,6}
\bibleverse{53} ¿Acaso no crees que yo podría rogar a mi Padre, y él
enviaría más de doce legiones de ángeles de inmediato? \footnote{\textbf{26:53}
  Mat 4,11} \bibleverse{54} Pero entonces ¿cómo podría cumplirse la
Escritura que dice que esto debe ocurrir?''

\bibleverse{55} Entonces Jesús le dijo a la turba: ``¿Han venido con
espadas y palos para arrestarme como si yo fuese algún criminal? Todos
los días me sentaba en el Templo a enseñarles y en ese momento no me
arrestaron. \bibleverse{56} Pero todo esto está ocurriendo para que se
cumpla lo que escribieron los profetas''. Entonces todos los discípulos
lo abandonaron y huyeron.

\hypertarget{el-interrogatorio-y-la-condena-de-jesuxfas-ante-el-sumo-sacerdote-y-el-concilio}{%
\subsection{El interrogatorio y la condena de Jesús ante el sumo
sacerdote y el
concilio}\label{el-interrogatorio-y-la-condena-de-jesuxfas-ante-el-sumo-sacerdote-y-el-concilio}}

\bibleverse{57} Los que habían arrestado a Jesús lo llevaron a la casa
de Caifás, el sumo sacerdote, donde se habían reunido los maestros
religiosos y los ancianos. \bibleverse{58} Pedro los seguía a la
distancia, y entró al patio de los sumos sacerdotes. Se sentó allí con
los guardias para ver cómo terminaban las cosas.

\bibleverse{59} Los jefes de los sacerdotes y todo el concilio estaban
tratando de encontrar alguna prueba falsa contra Jesús para mandarlo a
matar. \bibleverse{60} Pero no podían encontrar nada, aun cuando habían
venido muchos testigos falsos. Finalmente, llegaron dos \bibleverse{61}
e informaron: ``Este hombre dijo: `yo puedo destruir el Templo de Dios,
y volver a construirlo en tres días'\,''. \footnote{\textbf{26:61} Juan
  2,19-21; Hech 6,14}

\bibleverse{62} El sumo sacerdote se levantó y le preguntó a Jesús:
``¿No tienes nada que responder? ¿Qué tienes para decir en tu defensa?''
\bibleverse{63} Pero Jesús se quedó en silencio. El sumo sacerdote le
dijo a Jesús: ``En nombre del Dios vivo, te coloco bajo juramento. Dinos
si eres el Mesías, el Hijo de Dios''.

\bibleverse{64} ``Tu lo has dicho'', respondió Jesús. ``Y también te
digo que en el futuro verás al Hijo de Dios sentado a la diestra del
Todopoderoso, y viniendo en las nubes de los cielos''.\footnote{\textbf{26:64}
  Ver Salmos 110:1 y Daniel 7:13.} \footnote{\textbf{26:64} Sal 110,1;
  Mat 16,27; Mat 24,30; 2Cor 13,4}

\bibleverse{65} Entonces el sumo sacerdote rasgó su ropa, y dijo:
``¡Está diciendo blasfemia! ¿Para qué necesitamos testigos? ¡Miren,
ustedes mismos han escuchado su blasfemia! \footnote{\textbf{26:65} Juan
  10,33} \bibleverse{66} ¿Qué veredicto dan ustedes?'' ``¡Culpable!
¡Merece morir!'' respondieron ellos. \footnote{\textbf{26:66} Juan 19,7;
  Lev 24,16}

\bibleverse{67} Entonces escupieron su rostro y lo golpearon. Algunos de
ellos lo abofetearon con sus manos, \footnote{\textbf{26:67} Is 50,6}
\bibleverse{68} y dijeron: ``¡Profetízanos, `Mesías'! ¿Quién es el que
te acaba de golpear?''

\hypertarget{negaciuxf3n-y-arrepentimiento-de-pedro}{%
\subsection{Negación y arrepentimiento de
Pedro}\label{negaciuxf3n-y-arrepentimiento-de-pedro}}

\bibleverse{69} Mientras tanto, Pedro estaba sentado afuera en el patio.
Una joven criada vino donde él estaba y dijo: ``¡Tu también estabas con
Jesús el galileo!''

\bibleverse{70} Pero él lo negó delante de todos. ``No sé de qué
hablas'', dijo él.

\bibleverse{71} Entonces regresó a la entrada de la casa, donde otra
persona lo vio y le dijo a las personas que estaban allí: ``Este hombre
estaba con Jesús de Nazaret''.

\bibleverse{72} Una vez más, Pedro lo negó, diciendo con juramento: ``Yo
no lo conozco''.

\bibleverse{73} Un poco más tarde, las personas que estaban allí
vinieron donde estaba Pedro y dijeron: ``Definitivamente tu eres uno de
ellos. Tu acento te delata''.

\bibleverse{74} Entonces comenzó a jurar: ``¡Que me caiga una maldición
si estoy mintiendo!\footnote{\textbf{26:74} O, ``invocó maldiciones
  sobre sí mismo''.} ¡No conozco al hombre!'' E inmediatamente el gallo
cantó.

\bibleverse{75} Entonces Pedro recordó lo que Jesús le había dicho:
``Antes de que el gallo cante, negarás tres veces que me conoces''.
Entonces salió y lloró amargamente.

\hypertarget{uxfaltima-deliberaciuxf3n-del-sumo-consejo-extradiciuxf3n-de-los-condenados-al-gobernador-romano-pilato}{%
\subsection{Última deliberación del sumo consejo; Extradición de los
condenados al gobernador romano
Pilato}\label{uxfaltima-deliberaciuxf3n-del-sumo-consejo-extradiciuxf3n-de-los-condenados-al-gobernador-romano-pilato}}

\hypertarget{section-26}{%
\section{27}\label{section-26}}

\bibleverse{1} Temprano en la mañana, todos los jefes de los sacerdotes
y los ancianos del pueblo se reunieron a consultar y decidieron mandar a
matar a Jesús. \bibleverse{2} Lo ataron, se lo llevaron y se lo enviaron
a Pilato, el gobernador.

\bibleverse{3} Cuando Judas, el que había entregado a Jesús, vio que
Jesús había sido condenado a muerte, se arrepintió de lo que había hecho
y devolvió las treinta monedas de plata a los jefes de los sacerdotes y
a los ancianos. \footnote{\textbf{27:3} Mat 26,15} \bibleverse{4} ``¡He
pecado! ¡He entregado sangre inocente!'' les dijo. ``¿A nosotros qué nos
importa eso?'' respondieron ellos. ``¡Ese es tu problema!''

\bibleverse{5} Judas lanzó las monedas de plata en el santuario y se
fue. Huyó y se ahorcó.

\bibleverse{6} Los jefes de los sacerdotes tomaron las monedas de plata
y dijeron: ``Este es dinero de sangre, es contra la ley poner este
dinero en la tesorería del Templo''. \footnote{\textbf{27:6} Deut 23,19}
\bibleverse{7} Entonces se pusieron de acuerdo para comprar el campo del
alfarero para usarlo como el lugar donde sepultarían a los extranjeros.
\bibleverse{8} Por eso hasta hoy a ese campo se le llama el ``Campo de
Sangre''. \bibleverse{9} Esto cumplió la profecía dicha por el profeta
Jeremías: ``Tomaron treinta monedas de plata --- el `valor' de aquel que
fue comprado por el precio que le pusieron unos hijos de Israel---
\bibleverse{10} y las usaron para pagar el campo del alfarero, como el
Señor me mandó a hacerlo''\footnote{\textbf{27:10} Ver Zacarías
  11:12-13, haciendo referencia a Jeremías 32:6-15.}

\hypertarget{interrogatorio-de-jesuxfas-ante-pilato-jesuxfas-rechazado-por-la-gente-su-condenaciuxf3n-y-flagelaciuxf3n}{%
\subsection{Interrogatorio de Jesús ante Pilato; Jesús rechazado por la
gente; su condenación y
flagelación}\label{interrogatorio-de-jesuxfas-ante-pilato-jesuxfas-rechazado-por-la-gente-su-condenaciuxf3n-y-flagelaciuxf3n}}

\bibleverse{11} Jesús fue llevado delante de Pilato el gobernador, quien
le preguntó: ``¿Eres tu el Rey de los Judíos?'' ``Tú lo has dicho'',
respondió Jesús.

\bibleverse{12} Pero cuando el jefe de los sacerdotes y los ancianos
presentaron cargos contra él, Jesús no respondió. \bibleverse{13} ``¿No
escuchas todos los cargos que ellos están presentando contra ti?'' le
preguntó Pilato.

\bibleverse{14} Pero Jesús no dijo nada, ni una sola palabra. Esto
sorprendió en gran manera al gobernador. \footnote{\textbf{27:14} Juan
  19,9}

\hypertarget{jesuxfas-y-barrabuxe1s}{%
\subsection{Jesús y Barrabás}\label{jesuxfas-y-barrabuxe1s}}

\bibleverse{15} Y era costumbre del gobernador, durante la fiesta,
liberar delante de la multitud a cualquier prisionero que ellos
quisieran. \bibleverse{16} En esa época, estaba preso un hombre llamado
Barrabás. \bibleverse{17} Así que Pilato le preguntó a las multitudes
que se habían reunido: ``¿A quién quieren que libere: a Barrabás, o a
Jesús, llamado el Mesías?'' \bibleverse{18} (Él se había dado cuenta que
ellos habían arrestado a Jesús por celos para juzgarlo).

\bibleverse{19} Mientras estaba sentado en la silla de juez, su esposa
le envió un mensaje que decía: ``No le hagas nada a este hombre
inocente, porque he sufrido terriblemente en el día de hoy por un sueño
que tuve sobre él''.

\bibleverse{20} Pero los jefes de los sacerdotes y los ancianos
convencieron a las multitudes de pedir a Barrabás, y mandar a matar a
Jesús. \bibleverse{21} Cuando el gobernador les preguntó: ``¿A cuál de
los dos quieren que les libere entonces?'' ellos respondieron:
``Barrabás''.

\bibleverse{22} ``¿Entonces qué hare con Jesús, el Mesías?'' les
preguntó. Todos gritaron: ``¡Que lo crucifiquen!''

\bibleverse{23} ``¿Por qué? ¿Qué crimen ha cometido él?'' preguntó
Pilato. Pero ellos gritaban aún más fuerte: ``¡Crucifícalo!''

\bibleverse{24} Cuando Pilato vio que la causa estaba perdida, y que se
estaba formando un motín, trajo agua y lavó sus manos frente a la
multitud. ``Soy inocente de la sangre de este hombre. ¡Su sangre estará
sobre sus cabezas!''\footnote{\textbf{27:24} Literalmente, ``ustedes
  mismos sean responsables de ello''.} \footnote{\textbf{27:24} Deut
  21,6}

\bibleverse{25} Todo el pueblo respondió: ``¡Que su sangre sea sobre
nuestras cabezas y las de nuestros hijos!'' \footnote{\textbf{27:25}
  Hech 5,28}

\bibleverse{26} Entonces Pilato liberó a Barrabás, pero mandó a azotar a
Jesús y a crucificarlo.

\hypertarget{la-burla-y-el-maltrato-de-jesuxfas-por-parte-de-los-soldados-romanos}{%
\subsection{La burla y el maltrato de Jesús por parte de los soldados
romanos}\label{la-burla-y-el-maltrato-de-jesuxfas-por-parte-de-los-soldados-romanos}}

\bibleverse{27} Los soldados del gobernador llevaron a Jesús hasta el
Pretorio\footnote{\textbf{27:27} El cuartel militar.} y toda la tropa de
soldados lo rodeaba. \bibleverse{28} Entonces lo desnudaron y pusieron
un manto de color escarlata sobre él. \bibleverse{29} Hicieron una
corona de espinas y la colocaron sobre su cabeza, y le pusieron un palo
en su mano derecha. Y se arrodillaban frente a él y se burlaban
diciendo: ``¡Salve, Rey de los judíos!'' \bibleverse{30} Luego lo
escupieron, y tomando el palo que tenía, le golpeaban la cabeza con él.
\footnote{\textbf{27:30} Is 50,6} \bibleverse{31} Cuando terminaron de
burlarse de él, le quitaron el manto y volvieron a ponerle su ropa.
Entonces se lo llevaron para crucificarlo.

\hypertarget{el-curso-de-la-muerte-de-jesuxfas-despuuxe9s-del-guxf3lgota-su-crucifixiuxf3n-y-su-muerte}{%
\subsection{El curso de la muerte de Jesús después del Gólgota, su
crucifixión y su
muerte}\label{el-curso-de-la-muerte-de-jesuxfas-despuuxe9s-del-guxf3lgota-su-crucifixiuxf3n-y-su-muerte}}

\bibleverse{32} En el camino, se encontraron a un hombre llamado Simón,
de Cirene, y lo obligaron a llevar la cruz de Jesús. \bibleverse{33}
Cuando llegaron a Gólgota, que significa ``Lugar de la Calavera'',
\bibleverse{34} le dieron vino mezclado con hiel. Pero después de
probarlo, se negó a beberlo. \bibleverse{35} Después de haberlo
crucificado, lanzaron unos dados para dividir su ropa entre
ellos.\footnote{\textbf{27:35} Ver Salmos 22:18.} \footnote{\textbf{27:35}
  Juan 19,24} \bibleverse{36} Entonces se sentaron y se quedaron allí
vigilándolo. \bibleverse{37} Colocaron una señal sobre su cabeza con el
cargo que fue presentado contra él. Decía: ``Este es Jesús, el Rey de
los judíos''.

\bibleverse{38} Entonces crucificaron a dos criminales con él, uno a su
derecha, y el otro a su izquierda.

\bibleverse{39} Los que pasaban por ahí le gritaban insultos, sacudiendo
sus cabezas, \footnote{\textbf{27:39} Sal 22,8} \bibleverse{40} y
decían: ``¡Tú que prometiste destruir el Templo y reconstruirlo en tres
días, por qué no te salvas a ti mismo! Si realmente eres el Hijo de
Dios, entonces bájate de la cruz''. \footnote{\textbf{27:40} Mat 26,61;
  Juan 2,19}

\bibleverse{41} Los jefes de los sacerdotes se burlaban de él de la
misma manera, igual que los maestros religiosos y los ancianos.
\bibleverse{42} ``¡Salvó a otros pero no puede salvarse a sí mismo!''
decían. ``¡Si realmente él es el rey de Israel, que se baje de la cruz y
le creeremos! \bibleverse{43} Él cree en Dios con tanta seguridad,
---pues entonces que Dios lo rescate si lo quiere, pues él decía `yo soy
el Hijo de Dios'\,''. \footnote{\textbf{27:43} Sal 22,9} \bibleverse{44}
Y los criminales que estaban crucificados con él también lo insultaban
de la misma manera.

\hypertarget{la-muerte-de-jesuxfas-las-seuxf1ales-milagrosas-de-su-muerte}{%
\subsection{La muerte de Jesús; las señales milagrosas de su
muerte}\label{la-muerte-de-jesuxfas-las-seuxf1ales-milagrosas-de-su-muerte}}

\bibleverse{45} Desde el medio día hasta las tres de la tarde hubo
tinieblas en todo el país. \bibleverse{46} Aproximadamente a las tres de
la tarde, Jesús gritó fuertemente diciendo: ``Eli, Eli, lama
sabachthani?'' que significa: ``Dios mío, Dios mío, ¿por qué me has
abandonado?''\footnote{\textbf{27:46} Citando Salmos 22:1.}

\bibleverse{47} Cuando algunos de los que estaban allí lo escucharon,
dijeron: ``¡Está llamando a Elías!''

\bibleverse{48} E inmediatamente uno de ellos tomó una esponja, la
sumergió en vinagre y se lo dio a beber a Jesús. \footnote{\textbf{27:48}
  Sal 69,22} \bibleverse{49} Pero los otros decían: ``Déjalo solo.
Veamos si Elías viene y lo salva''.

\bibleverse{50} Jesus gritó otra vez a gran voz, y dio su último
respiro.\footnote{\textbf{27:50} Esta expresión es hebrea y quiere decir
  que murió.}

\bibleverse{51} Justo en ese momento, el velo del Templo se rasgó de
arriba a abajo. La tierra tembló, las rocas se partieron,
\bibleverse{52} y las tumbas se abrieron. Muchos de los que habían
vivido de manera justa y habían muerto, fueron levantados a la vida.
\bibleverse{53} Y después de la resurrección de Jesús, estos salieron de
los cementerios y entraron a la ciudad santa\footnote{\textbf{27:53}
  Refiriéndose a Jerusalén.} donde muchos los vieron.

\bibleverse{54} Cuando el centurión y los que estaban con él vigilando a
Jesús vieron el terremoto y lo que había ocurrido, se atemorizaron y
dijeron: ``¡Este era realmente el Hijo de Dios!''

\bibleverse{55} Muchas mujeres también miraban a la distancia, las que
habían seguido a Jesús desde Galilea y lo habían apoyado. \footnote{\textbf{27:55}
  Luc 8,2-3} \bibleverse{56} Entre estas estaba María Magdalena, María
la madre de Jesús, María la madre de Santiago y José, y la madre de los
hijos de Zebedeo.

\hypertarget{entierro-de-jesuxfas-orden-de-los-guardias-de-la-tumba}{%
\subsection{Entierro de Jesús; Orden de los guardias de la
tumba}\label{entierro-de-jesuxfas-orden-de-los-guardias-de-la-tumba}}

\bibleverse{57} Cuando llegó la noche, un hombre rico llamado José, de
Arimatea, (quien también era discípulo de Jesús), \bibleverse{58} fue
donde Pilato y pidió que le entregaran el cuerpo de Jesús. Entonces
Pilato ordenó que se le entregara. \bibleverse{59} José tomó el cuerpo y
lo envolvió en un paño nuevo de lino, \bibleverse{60} y lo puso en su
propia tumba que estaba nueva, hecha de roca sólida. Entonces rodó una
gran piedra que estaba puesta a la entrada de la tumba, y se fue.
\footnote{\textbf{27:60} Is 53,9} \bibleverse{61} María Magdalena y la
otra mujer llamada María, estaban allí sentadas al otro lado de la
tumba.

\bibleverse{62} Al día siguiente,\footnote{\textbf{27:62} Refiriéndose
  al Sábado.} después del día de la Preparación, los jefes de los
sacerdotes fueron juntos a ver a Pilato. \bibleverse{63} Y le dijeron:
``Señor, recordamos que el impostor cuando estaba vivo dijo: `Después de
tres días me levantaré de nuevo'. \bibleverse{64} Da la orden para
vigilar la tumba hasta el tercer día. Así sus discípulos no pueden
llegar y robar el cuerpo y decir al pueblo que él se levantó de entre
los muertos, y que la decepción al final llegue a ser peor que lo que
era al principio''.

\bibleverse{65} ``Les daré una guardia de soldados'', les dijo Pilato.
``Ahora vayan y aseguren la tumba tanto como puedan''. \bibleverse{66}
Entonces ellos fueron y aseguraron la tumba, sellando la entrada con una
piedra y colocando soldados como guardas de ella.

\hypertarget{las-dos-mujeres-junto-a-la-tumba-vacuxeda-en-la-mauxf1ana-de-pascua-la-primera-apariciuxf3n-de-jesuxfas-engauxf1ar-al-luxedder-del-pueblo}{%
\subsection{Las dos mujeres junto a la tumba vacía en la mañana de
Pascua; La primera aparición de Jesús; Engañar al líder del
pueblo}\label{las-dos-mujeres-junto-a-la-tumba-vacuxeda-en-la-mauxf1ana-de-pascua-la-primera-apariciuxf3n-de-jesuxfas-engauxf1ar-al-luxedder-del-pueblo}}

\hypertarget{section-27}{%
\section{28}\label{section-27}}

\bibleverse{1} Después del Sábado, al amanecer del primer día de la
semana,\footnote{\textbf{28:1} Esto correspondería al día que
  identificamos como domingo. El texto claramente identifica esto como
  ``día uno'', el día después del Sábado ``el séptimo día''.} María
Magdalena y la otra mujer llamada María, fueron a ver la tumba.
\footnote{\textbf{28:1} Hech 20,7; 1Cor 16,2; Apoc 1,10} \bibleverse{2}
De repente, hubo un gran terremoto, pues un ángel del Señor bajó del
cielo, rodó la piedra, y se sentó sobre ella. \bibleverse{3} Su rostro
resplandecía como un relámpago, y sus ropas eran blancas como la nieve.
\bibleverse{4} Los guardias temblaban de miedo, y cayeron como si
estuvieran muertos. \bibleverse{5} El ángel dijo a las mujeres: ``¡No
tengan miedo! Yo sé que ustedes buscan a Jesús, el que fue crucificado.
\bibleverse{6} Él no está aquí. Se ha levantado de entre los muertos,
tal como dijo que lo haría. Vengan y vean donde estuvo puesto el Señor.
\footnote{\textbf{28:6} Mat 12,40; Mat 16,21; Mat 17,23; Mat 20,19}
\bibleverse{7} Ahora vayan rápidamente y digan a sus discípulos que
Jesús se ha levantado de entre los muertos y que va delante de ustedes
hacia Galilea. ¡Les prometo que allí lo verán!'' \footnote{\textbf{28:7}
  Mat 26,32}

\bibleverse{8} Con miedo y a la vez muy felices, las mujeres se fueron
rápidamente de la tumba, e iban corriendo para decírselo a los
discípulos. \bibleverse{9} De repente, Jesús llegó a su encuentro, y las
saludó. Ellas se lanzaron hacia él, se aferraron a sus pies y lo
adoraron.

\bibleverse{10} Entonces Jesús les dijo: ``¡No tengan miedo! Vayan y
díganle a mis hermanos que vayan a Galilea, y allí me verán''.
\footnote{\textbf{28:10} Heb 2,11}

\hypertarget{la-falsa-afirmaciuxf3n-de-los-luxedderes-del-pueblo-del-cuerpo-robado-de-jesuxfas}{%
\subsection{La falsa afirmación de los líderes del pueblo del cuerpo
robado de
Jesús}\label{la-falsa-afirmaciuxf3n-de-los-luxedderes-del-pueblo-del-cuerpo-robado-de-jesuxfas}}

\bibleverse{11} Cuando se fueron, algunos de los guardias fueron a la
ciudad y le contaron a los jefes de los sacerdotes todo lo que había
ocurrido. \bibleverse{12} Después que los jefes de los sacerdotes se
hubieron reunido con los ancianos y hubieron elaborado un plan,
sobornaron a los soldados con una gran cantidad de dinero.
\bibleverse{13} ``Digan así: `Sus discípulos vinieron por la noche y
robaron el cuerpo mientras dormíamos,'\,'' dijeron a los soldados.
\bibleverse{14} ``Y si el gobernador llega a saber de esto, nosotros
hablaremos con él y ustedes no tendrán que preocuparse''.
\bibleverse{15} Así que los soldados tomaron el dinero e hicieron lo que
les habían dicho. Esta historia se ha difundido entre el pueblo judío
hasta el día de hoy.

\hypertarget{jesuxfas-apareciuxf3-en-la-montauxf1a-de-galilea-su-uxfaltimo-mandato-a-los-once-discuxedpulos}{%
\subsection{Jesús apareció en la montaña de Galilea; su último mandato a
los once
discípulos}\label{jesuxfas-apareciuxf3-en-la-montauxf1a-de-galilea-su-uxfaltimo-mandato-a-los-once-discuxedpulos}}

\bibleverse{16} Pero los once discípulos fueron a Galilea, a la montaña
donde Jesús les había dicho que fueran. \bibleverse{17} Cuando lo
vieron, lo adoraron, aunque algunos dudaban. \bibleverse{18} Jesús vino
donde ellos estaban y les dijo: ``Se me ha entregado todo el poder del
cielo y de la tierra. \bibleverse{19} Así que vayan y hagan discípulos
entre la gente de todas las naciones, bautizándolos en el nombre del
Padre, del Hijo y del Espíritu Santo. \bibleverse{20} Enséñenles a
seguir todos los mandamientos que yo les he dado a ustedes. Recuerden,
yo estoy siempre con ustedes hasta el fin del mundo''.
