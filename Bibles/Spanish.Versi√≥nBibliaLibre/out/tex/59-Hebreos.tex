\hypertarget{la-soberanuxeda-uxfanica-del-hijo-de-dios-sobre-los-mensajeros-de-dios-del-antiguo-testamento}{%
\subsection{La soberanía única del Hijo de Dios sobre los mensajeros de
Dios del Antiguo
Testamento}\label{la-soberanuxeda-uxfanica-del-hijo-de-dios-sobre-los-mensajeros-de-dios-del-antiguo-testamento}}

\hypertarget{section}{%
\section{1}\label{section}}

\bibleverse{1} Dios, que en el pasado habló a nuestros padres por medio
de los profetas en distintas épocas y de muchas maneras, \bibleverse{2}
en estos días nos ha hablado por medio de su Hijo. Dios designó al Hijo
como heredero de todo, e hizo el universo por medio de él. \footnote{\textbf{1:2}
  Sal 2,8; Juan 1,3; Col 1,16} \bibleverse{3} El Hijo es la gloria
radiante de Dios, y la expresión visible de su verdadero carácter. Él
sostiene todas las cosas con su poderoso mandato. Cuando hizo provisión
para limpiar el pecado, se sentó a la diestra de la Majestad del cielo.
\footnote{\textbf{1:3} 2Cor 4,4; Col 1,15; Heb 9,14; Heb 9,26; Mar 16,19}
\bibleverse{4} Y fue puesto en un lugar más elevado que los ángeles
porque recibió un nombre más grande que ellos. \footnote{\textbf{1:4}
  Fil 2,9; 1Pe 3,22}

\hypertarget{evidencia-del-antiguo-testamento-de-la-exaltaciuxf3n-del-hijo-de-dios-sobre-los-uxe1ngeles}{%
\subsection{Evidencia del Antiguo Testamento de la exaltación del Hijo
de Dios sobre los
ángeles}\label{evidencia-del-antiguo-testamento-de-la-exaltaciuxf3n-del-hijo-de-dios-sobre-los-uxe1ngeles}}

\bibleverse{5} Dios nunca le dijo a ningún ángel: ``Tú eres mi hijo; hoy
me he convertido en tu Padre'', o ``Seré su Padre, y él será mi
Hijo''.\footnote{\textbf{1:5} Hebreos contiene muchas citas y alusiones
  al Antiguo Testamento, algunas de las cuales no están citadas de
  manera exacta o son presentadas de manera resumida. Por eso, en
  ocasiones es difícil identificar la fuente exacta y con el fin de no
  sobrecargar el texto con tantos pie de página, las citas del Antiguo
  Testamento a menudo no aparecerán aquí. Tenga en cuenta que están
  tomados de la Septuaginta, la traducción griega de las Escrituras
  Hebreas. Las citas a las que se hace referencia en este versículo
  parecen ser: Salmos 2:7, 2 Samuel 7:14 y 1 Crónicas 17:13.}

\bibleverse{6} Además, cuando trajo a su Hijo primogénito\footnote{\textbf{1:6}
  ``Primogénito'': Este término no debe usarse como si hubiera algún
  tiempo en que Jesús no existió; más bien se usa para señalar un rango,
  mas no una cronología.} al mundo, dijo: ``Adórenlo todos los ángeles
de Dios''.\footnote{\textbf{1:6} Citando Deuteronomio 32:43 de la
  Septuaginta.} \bibleverse{7} En cuanto a los ángeles, él dijo: ``Él
transforma a sus ángeles en vientos, y a sus siervos en llamas de
fuego'',\footnote{\textbf{1:7} Citando Salmos 45:6-7.}

\bibleverse{8} pero respecto al Hijo, dice: ``Tu trono, oh Dios, perdura
por siempre y para siempre, y la justicia es el cetro de tu reino.
\bibleverse{9} Tú amas lo recto, y aborreces el desorden. Es por eso que
Dios, tu Dios, te ha puesto por encima de todos los demás,
ungiéndote\footnote{\textbf{1:9} La Antigua práctica de poner aceite
  sobre la cabeza de una persona tenía como fin indicar que la persona
  era escogida para una posición específica, un alto honor.} con el
aceite del gozo''.\footnote{\textbf{1:9} Citando Salmos 45:6-7.}

\bibleverse{10} ``Tú, Señor, pusiste los fundamentos de la tierra en el
principio. Los cielos son producto de tus manos. \bibleverse{11} Un día
se acabarán, pero tú seguirás. Se desgastarán como se desgasta la ropa,
\bibleverse{12} y los enrollarás como un manto. Los cambiarás como
cambiar la ropa, y tu vida no cesa jamás''.\footnote{\textbf{1:12}
  Literalmente, ``tus años nunca terminan''.}

\bibleverse{13} Pero nunca le dijo a ningún ángel: ``Siéntate a mi
diestra hasta que sujete a tus enemigos debajo de tus pies''.\footnote{\textbf{1:13}
  Citando Salmos 110:1.}

\bibleverse{14} ¿Qué son los ángeles? Son seres que sirven, que han sido
enviados para ayudar a los que recibirán la salvación.\footnote{\textbf{1:14}
  Sal 34,8; Sal 91,11-12}

\hypertarget{de-ahuxed-surge-la-obligaciuxf3n-de-que-obedezcamos-voluntariamente-las-palabras-que-nos-ha-dicho-este-hijo}{%
\subsection{De ahí surge la obligación de que obedezcamos
voluntariamente las palabras que nos ha dicho este
Hijo}\label{de-ahuxed-surge-la-obligaciuxf3n-de-que-obedezcamos-voluntariamente-las-palabras-que-nos-ha-dicho-este-hijo}}

\hypertarget{section-1}{%
\section{2}\label{section-1}}

\bibleverse{1} Por lo tanto deberíamos estar aún más atentos a lo que
hemos aprendido para no descarriarnos. \bibleverse{2} Si el mensaje que
los ángeles trajeron es fiel, y si cada pecado y acto de desobediencia
trae su propia consecuencia,\footnote{\textbf{2:2} Literalmente,
  ``recibe su recompensa''.} \bibleverse{3} ¿cómo escaparemos si no
atendemos esta gran salvación que el Señor anunció desde el principio, y
que después nos confirmó por medio de quienes lo oyeron? \footnote{\textbf{2:3}
  Heb 10,29; Heb 12,25} \bibleverse{4} Dios también dio testimonio por
medio de señales y milagros, por actos que demuestran su poder, y por
medio de los dones del Espíritu Santo, que repartió como quiso.
\footnote{\textbf{2:4} Mar 16,20; 1Cor 12,4-11; 2Cor 12,12; Hech 1,2-13;
  Hech 10,44-45}

\hypertarget{su-humillaciuxf3n-encarnaciuxf3n-y-sufrimiento-de-muerte-no-limita-su-sublimidad}{%
\subsection{Su humillación, encarnación y sufrimiento de muerte, no
limita su
sublimidad}\label{su-humillaciuxf3n-encarnaciuxf3n-y-sufrimiento-de-muerte-no-limita-su-sublimidad}}

\bibleverse{5} No serán los ángeles los encargados del mundo venidero
del cual hablamos. \bibleverse{6} Sino que, como se ha dicho: ``¿Qué son
los seres humanos para que te preocupes por ellos? ¿Quién es el hijo de
hombre\footnote{\textbf{2:6} ``Hijo de hombre'': En su uso normal se
  refiere solo a un ser humano; sin embargo, Jesús aplicó este término
  genérico a sí mismo.} para que cuides de él? \bibleverse{7} Lo hiciste
un poco inferior a los ángeles; lo coronaste con gloria y honra, y lo
pusiste por encima de toda tu creación.\footnote{\textbf{2:7} En lugar
  de referirse solo a la humanidad, también puede referirse a Jesús:
  ``Lo hiciste un poco menor que los ángeles, y luego lo coronaste de
  gloria y honra''. Todo el texto puede verse de manera dual,
  refiriéndose a Jesús como el hijo de hombre, siendo tanto
  representante como Salvador de la humanidad.} \bibleverse{8} Le diste
autoridad sobre todas las cosas''.\footnote{\textbf{2:8} Una vez más,
  esto puede aplicarse a la humanidad, a Dios dando autoridad sobre las
  criaturas como se menciona en Génesis 1, o puede aplicarse a la
  autoridad de Jesús como Señor.} No quedó nada por fuera cuando Dios le
dio autoridad sobre todas las cosas. Sin embargo, vemos que no todo está
sujeto a su autoridad todavía.

\bibleverse{9} Pero vemos a Jesús, puesto en un lugar un poco inferior
al de los ángeles, coronado de gloria y honra por el sufrimiento de la
muerte. Por medio de la gracia de Dios, Jesús experimentó la muerte por
todos. \footnote{\textbf{2:9} Fil 2,8-9}

\hypertarget{la-necesidad-de-la-humillaciuxf3n-especialmente-el-sufrimiento-de-la-muerte}{%
\subsection{La necesidad de la humillación, especialmente el sufrimiento
de la
muerte}\label{la-necesidad-de-la-humillaciuxf3n-especialmente-el-sufrimiento-de-la-muerte}}

\bibleverse{10} Era conveniente que Dios, quien crea y sostiene todas
las cosas, preparara por medio del sufrimiento a Aquél que los lleva a
la salvación, para llevar a muchos de sus hijos a la gloria. \footnote{\textbf{2:10}
  Heb 12,2} \bibleverse{11} Pues tanto el que santifica como los que son
santificados pertenecen a la misma familia.\footnote{\textbf{2:11}
  Literalmente, ``todos de una''.} Por eso no vacila en llamarlos
``hermanos'' \footnote{\textbf{2:11} Mar 3,34-35; Juan 17,19; Juan 20,17}
\bibleverse{12} al decir: ``Anunciaré tu nombre a mis hermanos; te
alabaré entre tu pueblo cuando se reúna''.\footnote{\textbf{2:12} ``Se
  reúna'': la palabra griega es ``ecclesia'' que más adelante llegó a
  significar ``iglesia''. La cita es de Salmos 22:22}

\bibleverse{13} Y también dice: ``Pondré mi confianza en él'', y ``Aquí
estoy, junto a los hijos que Dios me ha dado''.\footnote{\textbf{2:13}
  Citando Isaías 8:17-18.}

\hypertarget{las-beneficiosas-consecuencias-de-la-humillaciuxf3n}{%
\subsection{Las beneficiosas consecuencias de la
humillación}\label{las-beneficiosas-consecuencias-de-la-humillaciuxf3n}}

\bibleverse{14} Y como los hijos tienen en común carne y sangre, él
participó de su carne y sangre del mismo modo, para así destruir por
medio de la muerte a aquél que tiene el poder de la muerte---el
diablo--- \bibleverse{15} y liberar a todos los que habían estado
esclavizados toda la vida por miedo a la muerte. \bibleverse{16} Sin
duda alguna, los ángeles no son su preocupación; él se preocupa por
ayudar a los hijos de Abrahán. \bibleverse{17} Por ello le fue necesario
volverse como sus hermanos en todo, para poder llegar a ser un sumo
sacerdote, misericordioso y fiel, en las cosas de Dios, para perdonar
los pecados de su pueblo. \footnote{\textbf{2:17} Fil 2,7}
\bibleverse{18} Y como él mismo sufrió la tentación, puede ayudar a los
que son tentados.\footnote{\textbf{2:18} Heb 4,15}

\hypertarget{el-hijo-de-dios-jesuxfas-en-su-majestad-sobre-el-ministro-de-dios-moisuxe9s}{%
\subsection{El Hijo de Dios Jesús en su majestad sobre el ministro de
Dios
Moisés}\label{el-hijo-de-dios-jesuxfas-en-su-majestad-sobre-el-ministro-de-dios-moisuxe9s}}

\hypertarget{section-2}{%
\section{3}\label{section-2}}

\bibleverse{1} Así que, mis hermanos y hermanas que viven para Dios y
participan de este celestial llamado: necesitamos pensar con cuidado
acerca de Jesús, el que decimos que fue enviado por Dios,\footnote{\textbf{3:1}
  Literalmente, ``apóstol''.} y quien es el Sumo Sacerdote. \footnote{\textbf{3:1}
  Heb 4,14} \bibleverse{2} Él fue fiel a Dios en la obra para la cual
fue elegido, así como Moisés fue fiel a Dios en la casa de
Dios.\footnote{\textbf{3:2} La palabra ``casa'' aquí significa más que
  el edificio: se refiere a los miembros de una casa, la familia, el
  hogar. Aquí y en el versículo 5, la fidelidad de Moisés como siervo en
  la casa de Dios hace referencia a Números 12:7.} \footnote{\textbf{3:2}
  Núm 12,7} \bibleverse{3} Pero Jesús es merecedor de mayor gloria que
Moisés, del mismo modo que el constructor de una casa merece más crédito
que la misma casa. \bibleverse{4} Cada casa tiene su constructor; Dios
es el constructor de todo. \bibleverse{5} Y como siervo, Moisés fue fiel
en la casa de Dios. Él nos dio evidencia de lo que sería anunciado
después. \bibleverse{6} Pero Cristo es un hijo, a cargo de la casa de
Dios. Y nosotros somos la casa de Dios siempre y cuando nos aferremos
con confianza a la esperanza en la cual decimos que creemos con orgullo.
\footnote{\textbf{3:6} 1Pe 2,5; Efes 2,19}

\hypertarget{la-advertencia-del-salmista-contra-la-incredulidad-y-la-apostasuxeda}{%
\subsection{La advertencia del salmista contra la incredulidad y la
apostasía}\label{la-advertencia-del-salmista-contra-la-incredulidad-y-la-apostasuxeda}}

\bibleverse{7} Por eso el Espíritu Santo dice: ``Si oyen lo que Dios les
está diciendo hoy, \footnote{\textbf{3:7} Heb 4,7} \bibleverse{8} no
endurezcan sus corazones\footnote{\textbf{3:8} ``Endurezcan sus
  corazones'', queriendo decir, volverse tercos u obstinados.} como en
aquél tiempo en que se rebelaron contra él, cuando lo pusieron a prueba
en el desierto. \footnote{\textbf{3:8} Éxod 17,7; Núm 20,2-5}
\bibleverse{9} Los padres de ustedes me pusieron a prueba, y probaron mi
paciencia, y vieron la evidencia que les mostré durante cuarenta años.
\bibleverse{10} ``Tal generación despertó mi enojo\footnote{\textbf{3:10}
  Como siempre, Dios aquí usa términos humanos. No debemos entender que
  Dios se enoja de la manera que nosotros lo hacemos, especialmente
  cuando se trata de ``perder la paciencia'' y actuar sin amor o
  irracionalmente. Lo mismo aplica para el versículo 3:11.} y por ello
dije: `Siempre se equivocan en su manera de pensar. No me conocen ni
saben lo que estoy haciendo'. \bibleverse{11} Por ello, en mi
frustración hice un juramento: `No entrarán a mi reposo'\,''.\footnote{\textbf{3:11}
  ``Reposo''. Este concepto se desarrolla más en el capítulo 4 y se
  relaciona con el Sábado, la Tierra Prometida, y la invitación de Dios
  de venir a él. Aunque no es la más fácil de las frases, ``entrar en su
  reposo'' quizás es la mejor traducción pues mantiene la base de lo que
  más adelante se desarrolla en el texto, e incluye todas las alusiones.
  La cita es deSalmos 95:7-11.}

\bibleverse{12} Hermanos y hermanas, asegúrense de que ninguno de
ustedes tenga un pensamiento malvado y alejado de la fe en el Dios de la
vida. \bibleverse{13} Anímense unos a otros cada día mientras dure el
``hoy'', para que ninguno de ustedes pueda ser engañado por el pecado ni
se endurezcan sus corazones. \footnote{\textbf{3:13} 1Tes 5,11}

\hypertarget{el-ejemplo-de-advertencia-de-los-israelitas-en-el-desierto}{%
\subsection{El ejemplo de advertencia de los israelitas en el
desierto}\label{el-ejemplo-de-advertencia-de-los-israelitas-en-el-desierto}}

\bibleverse{14} Porque somos socios con Cristo siempre y cuando
mantengamos nuestra confianza en Dios de principio a fin. \footnote{\textbf{3:14}
  Heb 6,11} \bibleverse{15} Como dice la Escritura: ``Si oyen lo que
Dios les dice hoy, no endurezcan sus corazones como aquél tiempo en que
se rebelaron contra él''.\footnote{\textbf{3:15} Citando Salmos 95:7-8.}

\bibleverse{16} ¿Quién se rebeló contra Dios aun habiendo oído lo que él
dijo? ¿No fueron acaso los que fueron sacados de Egipto por Moisés?
\bibleverse{17} ¿Contra quienes estuvo enojado Dios durante cuarenta
años? ¿No fue contra aquellos que fueron sepultados en el desierto?
\footnote{\textbf{3:17} Núm 14,29; 1Cor 10,10}

\bibleverse{18} ¿De quién hablaba Dios cuando hizo juramento de que no
entrarían en su reposo? ¿No fue de los que lo desobedecieron?
\bibleverse{19} Así vemos que ellos no pudieron entrar, porque no
confiaron en él.

\hypertarget{interpretaciuxf3n-de-la-promesa-del-salmo-sobre-el-resto-del-pueblo-de-dios}{%
\subsection{Interpretación de la promesa del salmo sobre el resto del
pueblo de
Dios}\label{interpretaciuxf3n-de-la-promesa-del-salmo-sobre-el-resto-del-pueblo-de-dios}}

\hypertarget{section-3}{%
\section{4}\label{section-3}}

\bibleverse{1} Por lo tanto seamos cuidadosos y asegurémonos de no
perdernos la oportunidad de entrar a su reposo, aunque Dios ya nos dio
la promesa. \bibleverse{2} Porque hemos oído buenas noticias tal como
ellos lo hicieron, pero eso no fue suficiente porque ellos no aceptaron
ni creyeron lo que oyeron. \bibleverse{3} Sin embargo, los que creen en
Dios ya han entrado al reposo mencionado por Dios cuando dijo: ``En mi
frustración hice un juramento: `No entrarán a mi reposo'\,''.\footnote{\textbf{4:3}
  Citando Salmos 95:11.} (Esto es así aunque los planes de Dios ya
estaban completos cuando creó el mundo). \bibleverse{4} En cuanto al
séptimo día, hay un lugar en la Escritura que dice: ``Dios reposó el
séptimo día de toda su obra''.\footnote{\textbf{4:4} Citando Génesis
  2:2.} \bibleverse{5} Y como lo afirmaba el pasaje anterior: ``Ellos no
entrarán a mi reposo''.

\bibleverse{6} El reposo de Dios aún está disponible para que entremos
en él, aunque aquellos que habían oído antes la buena noticia no
lograron entrar por su desobediencia. \bibleverse{7} Así que Dios una
vez más coloca un día---hoy---diciéndonos mucho tiempo después por medio
de David,\footnote{\textbf{4:7} Refiriéndose a Salmos 95:7.} como lo
hizo antes: ``Si oyen lo que Dios les dice hoy, no endurezcan sus
corazones''.\footnote{\textbf{4:7} Citando Salmos 95:7.}

\bibleverse{8} Porque si Josué hubiera podido darles reposo, Dios no
habría dicho nada después sobre otro día. \footnote{\textbf{4:8} Deut
  31,7; Jos 22,4} \bibleverse{9} De modo que el reposo del Sábado
todavía permanece para el pueblo de Dios. \bibleverse{10} Porque todo el
que entra al reposo de Dios también descansa de su labor, así como Dios
lo hizo.

\hypertarget{exhortaciuxf3n-final-en-referencia-a-la-seriedad-y-el-poder-de-la-palabra-de-dios}{%
\subsection{Exhortación final en referencia a la seriedad y el poder de
la palabra de
Dios}\label{exhortaciuxf3n-final-en-referencia-a-la-seriedad-y-el-poder-de-la-palabra-de-dios}}

\bibleverse{11} En consecuencia, debemos esforzarnos por entrar al
reposo de Dios para que nadie caiga al seguir el mismo ejemplo de
desobediencia. \footnote{\textbf{4:11} Heb 3,16-19} \bibleverse{12} Pues
la palabra de Dios es viva y eficaz, y más afilada que espada de dos
filos, que penetra hasta separar la vida y el aliento,\footnote{\textbf{4:12}
  Las palabras griegas ``psuche'' y ``pneuma'', en ocasiones traducidas
  como ``alma'' y ``espíritu'', aunque es difícil entender el
  significado ya que no hay diferencia entre ``alma'' y ``espíritu''. Se
  emplea la traducción de ``vida'' y ``aliento'' porque se considera que
  expresa mejor el pensamiento original.} así como los tendones y los
tuétanos, juzgando los pensamientos y las intenciones de la mente.
\footnote{\textbf{4:12} Jer 23,29; Apoc 2,12} \bibleverse{13} No hay ser
vivo que esté oculto de su vista; todo está expuesto y es visible ante
aquél a quien hemos de rendirle cuentas.

\hypertarget{jesuxfas-conoce-las-debilidades-humanas-por-experiencia-personal}{%
\subsection{Jesús conoce las debilidades humanas por experiencia
personal}\label{jesuxfas-conoce-las-debilidades-humanas-por-experiencia-personal}}

\bibleverse{14} Y como tenemos tal sumo sacerdote que ha ascendido al
cielo, Jesús, el Hijo de Dios, asegurémonos de mantenernos en lo que
decimos creer. \footnote{\textbf{4:14} Heb 3,1; Heb 9,11-12; Heb 10,23}
\bibleverse{15} Pues el sumo sacerdote que tenemos no es uno que no
pueda entender nuestras debilidades, sino uno que fue tentado de la
misma forma que nosotros, pero no pecó. \footnote{\textbf{4:15} Heb
  2,18; Juan 8,46} \bibleverse{16} Así que deberíamos acercarnos
confiados a Dios, en su trono de gracia, para recibir misericordia, y
descubrir la gracia que nos ayuda cuando realmente la
necesitamos.\footnote{\textbf{4:16} Rom 3,25; Rom 5,2}

\hypertarget{con-cristo-se-encuentran-los-requisitos-necesarios-del-sumo-sacerdote-sugeridos-en-melquisedec}{%
\subsection{Con Cristo se encuentran los requisitos necesarios del sumo
sacerdote sugeridos en
Melquisedec}\label{con-cristo-se-encuentran-los-requisitos-necesarios-del-sumo-sacerdote-sugeridos-en-melquisedec}}

\hypertarget{section-4}{%
\section{5}\label{section-4}}

\bibleverse{1} Todo sumo sacerdote es elegido dentro del mismo pueblo y
está designado para trabajar por el pueblo en cuanto a su relación con
Dios. Él presenta a Dios tanto sus dones como sus sacrificios por sus
pecados. \bibleverse{2} El sumo sacerdote comprende cuán ignorantes y
engañadas se sienten las personas porque él también experimenta las
mismas debilidades humanas que ellos. \bibleverse{3} En consecuencia, él
tiene que ofrecer sacrificios por sus pecados así como por los del
pueblo. \bibleverse{4} Nadie puede tomar la posición de sumo sacerdote
por sí mismo, sino que debe ser elegido por Dios, como lo fue Aarón.
\footnote{\textbf{5:4} Éxod 28,1} \bibleverse{5} Del mismo modo en que
Cristo no se atribuyó honra a sí mismo convirtiéndose en sumo sacerdote.
Sino que fue Dios quien le dijo: ``Tú eres mi hijo. Hoy yo me convierto
en tu Padre''.\footnote{\textbf{5:5} Citando Salmos 2:7.}

\bibleverse{6} Y en otro versículo, Dios dice: ``Eres un sacerdote por
siempre, siguiendo el orden de Melquisedec''.\footnote{\textbf{5:6}
  Citando Salmos 110:4.}

\bibleverse{7} Jesús, mientras estuvo aquí, en forma humana, oró y clamó
a Dios con grandes gemidos y lágrimas, al único que tenía el poder de
salvarlo de la muerte. Y Jesús fue escuchado por su respeto hacia Dios.
\footnote{\textbf{5:7} Mat 26,39-46} \bibleverse{8} Aunque era el Hijo
de Dios, Jesús aprendió de manera práctica el significado de la
obediencia a través del sufrimiento.\footnote{\textbf{5:8} La traducción
  común de que Jesús ``aprendió obediencia por medio del sufrimiento''
  podría sugerir que originalmente Jesús no era obediente, o que le era
  necesario sufrir para aprender, las cuales son ideas extrañas en lo
  que se refiere a Jesús, el hijo pre-existente de Dios. De alguna
  manera, esto es paralelo a la petición de Jesús de que le quiten la
  copa del sufrimiento, pero luego entrega su voluntad en obediencia a
  su Padre. Ver Mateo 26:39.} \footnote{\textbf{5:8} Fil 2,8}
\bibleverse{9} Y cuando su experiencia culminó,\footnote{\textbf{5:9}
  Evitar el término ``habiendo sido perfeccionado'', que en la mente
  podría sugerir que no era perfecto desde el principio.} se convirtió
en la fuente de salvación eterna para todos los que hacen su voluntad,
\bibleverse{10} habiendo sido designado por Dios como sumo sacerdote,
conforme al orden de Melquisedec. \footnote{\textbf{5:10} Heb 7,-1}

\hypertarget{quejarse-de-la-inmadurez-la-indolencia-intelectual-y-el-atraso-de-los-lectores}{%
\subsection{Quejarse de la inmadurez, la indolencia intelectual y el
atraso de los
lectores}\label{quejarse-de-la-inmadurez-la-indolencia-intelectual-y-el-atraso-de-los-lectores}}

\bibleverse{11} Hay mucho que decir acerca de Jesús, y no es fácil
explicarlo porque ustedes parecen no entender. \bibleverse{12} Para esta
hora, ustedes ya han tenido suficiente tiempo para ser maestros, pero
todavía necesitan de alguien que les enseñe los fundamentos, los
principios de la palabra de Dios. ¡Es como si necesitaran volver a beber
leche en lugar de comida sólida! \bibleverse{13} Los que beben leche no
tienen la experiencia para vivir de manera correcta, pues apenas son
bebés. \footnote{\textbf{5:13} Efes 4,14}

\bibleverse{14} La comida sólida es para los adultos, para los que han
aprendido siempre a usar su cerebro para poder decir la diferencia entre
el bien y el mal.

\hypertarget{es-una-cuestiuxf3n-de-progreso-la-recauxedda-es-peligrosa-y-puede-provocar-dauxf1os-incurables}{%
\subsection{Es una cuestión de progreso; La recaída es peligrosa y puede
provocar daños
incurables}\label{es-una-cuestiuxf3n-de-progreso-la-recauxedda-es-peligrosa-y-puede-provocar-dauxf1os-incurables}}

\hypertarget{section-5}{%
\section{6}\label{section-5}}

\bibleverse{1} Así que no nos estanquemos en las enseñanzas básicas
acerca de Cristo, sino progresemos a un entendimiento más maduro. No
necesitamos volver una y otra vez a los conceptos sobre el
arrepentimiento de lo que solíamos hacer, o sobre la fe en Dios,
\bibleverse{2} o enseñanzas acerca del bautismo, la imposición de manos,
la resurrección de los muertos, y el juicio eterno. \bibleverse{3}
Avancemos en la medida que Dios nos lo permite. \bibleverse{4} Es
imposible que los que una vez comprendieron y experimentaron el don
celestial de Dios---que participaron del recibimiento del Espíritu
Santo, \bibleverse{5} que habían conocido la palabra de Dios y el poder
de la era que está por venir--- \bibleverse{6} y luego abandonaron por
completo a Dios, vuelvan al arrepentimiento una vez más. Ellos mismos
han crucificado al Hijo de Dios una y otra vez, y lo han humillado
públicamente.\footnote{\textbf{6:6} ``Abandonaron por completo''. La
  palabra en el texto griego solo se usa una vez en el Nuevo Testamento
  y significa renunciar y repudiar totalmente una creencia. No es la
  palabra usual para apostasía.} \bibleverse{7} La tierra que ha sido
regada por la lluvia, y produce cosecha para quienes la trabajan, tiene
la bendición de Dios. \bibleverse{8} Pero la tierra que solo produce
monte y espinas no sirve para nada, y está condenada. Y al final lo
único que puede hacerse es quemarla.

\hypertarget{confiada-esperanza-de-superar-este-angustioso-estado-de-los-lectores-y-el-peligro-que-los-amenaza}{%
\subsection{Confiada esperanza de superar este angustioso estado de los
lectores y el peligro que los
amenaza}\label{confiada-esperanza-de-superar-este-angustioso-estado-de-los-lectores-y-el-peligro-que-los-amenaza}}

\bibleverse{9} Pero queridos amigos, nosotros deseamos cosas mejores
para ustedes, y también su salvación, aunque les hablemos así.
\bibleverse{10} Dios no hubiera sido injusto como para olvidarse de lo
que ustedes han hecho y del amor que le han demostrado mediante el
cuidado que han brindado a los hermanos creyentes, lo cual es algo que
todavía siguen haciendo. \footnote{\textbf{6:10} Heb 10,32-34}
\bibleverse{11} Queremos que cada uno de ustedes demuestre el mismo
compromiso y confianza en la esperanza de Dios, hasta que sea cumplida.
\footnote{\textbf{6:11} Heb 3,14; Fil 1,6} \bibleverse{12} No sean
espiritualmente perezosos, sino sigan el ejemplo de los que por medio de
su fe en Dios y paciencia son herederos de lo que Dios ha prometido.

\hypertarget{el-fundamento-firme-de-la-esperanza-en-la-gloria-que-seguramente-se-espera-radica-en-las-confiables-promesas-de-dios}{%
\subsection{El fundamento firme de la esperanza en la gloria que
seguramente se espera radica en las confiables promesas de
Dios}\label{el-fundamento-firme-de-la-esperanza-en-la-gloria-que-seguramente-se-espera-radica-en-las-confiables-promesas-de-dios}}

\bibleverse{13} Cuando Dios le dio su promesa a Abrahán, no pudo jurar
por alguien superior, así que hizo un juramento consigo mismo,
\bibleverse{14} diciendo: ``Sin duda alguna te bendeciré, y multiplicaré
tus descendientes''.\footnote{\textbf{6:14} Citando Génesis 22:17.}
\bibleverse{15} Y así, después de esperar pacientemente, Abrahán recibió
la promesa. \bibleverse{16} Las personas juran por cosas que son
superiores a ellas, y cuando tienen alguna discusión, hacen un juramento
como la última palabra sobre tal asunto. \footnote{\textbf{6:16} Éxod
  22,10} \bibleverse{17} Es por ello que Dios quería demostrar más
claramente a los que heredarían la promesa, que él nunca cambiaría su
decisión. \bibleverse{18} De modo que por estas dos acciones\footnote{\textbf{6:18}
  Es decir, la promesa y el juramento.} que no pueden cambiarse, y, como
Dios no puede mentir, podemos tener plena confianza en que al huir
buscando seguridad, podemos aferrarnos de la esperanza que Dios nos
presentó. \bibleverse{19} Esta esperanza es nuestra ancla espiritual, es
segura y confiable, y nos lleva más allá de la cortina, a la presencia
de Dios. \bibleverse{20} Allí entró Jesús en nuestro favor, porque tenía
que convertirse en un sumo sacerdote conforme al orden de
Melquisedec.\footnote{\textbf{6:20} Heb 5,6}

\hypertarget{jesuxfas-el-sumo-sacerdote-perfecto-para-siempre-seguxfan-el-orden-de-melquisedec}{%
\subsection{Jesús, el sumo sacerdote perfecto para siempre según el
orden de
Melquisedec}\label{jesuxfas-el-sumo-sacerdote-perfecto-para-siempre-seguxfan-el-orden-de-melquisedec}}

\hypertarget{section-6}{%
\section{7}\label{section-6}}

\bibleverse{1} Melquisedec fue rey de Salem y sacerdote del Dios
Supremo. Conoció a Abrahán, quien venía de regreso después de haber
derrotado a los reyes, y lo bendijo.\footnote{\textbf{7:1} Ver Génesis
  14:18.} \footnote{\textbf{7:1} Gén 14,18-20} \bibleverse{2} Y Abrahán
le dio diezmo de todo lo que había ganado. El nombre Melquisedec
significa ``rey de justicia'' mientras que el rey de Salem significa
``rey de paz''. \bibleverse{3} No tenemos información sobre su padre o
su madre, o sobre su genealogía. No sabemos cuándo nació ni cuándo
murió. Así como el Hijo de Dios, sigue siendo sacerdote para siempre.
\footnote{\textbf{7:3} Juan 7,27}

\hypertarget{melquisedec-es-muxe1s-digno-que-los-sacerdotes-levitas}{%
\subsection{Melquisedec es más digno que los sacerdotes
levitas}\label{melquisedec-es-muxe1s-digno-que-los-sacerdotes-levitas}}

\bibleverse{4} Consideremos la grandeza de este hombre ante los ojos de
Abrahán, el patriarca, que incluso le entregó diezmo de lo que había
ganado en la batalla. \bibleverse{5} Sí, pues los hijos de Leví, que son
sacerdotes, tienen mandato por la ley de recibir diezmo del pueblo, que
son sus hermanos y hermanas, y que son descendientes de Abrahán.
\bibleverse{6} Pero Melquisedec, sin pertenecer a esta descendencia,
recibió diezmos de Abrahán, y bendijo al que tenía las promesas de Dios.
\bibleverse{7} No existe duda de que quien recibe bendición es inferior
a quien bendice. \bibleverse{8} En el primer caso, los que reciben el
diezmo son hombres mortales, pero en el otro caso, se dice que los
recibió uno que sigue viviendo. \bibleverse{9} Entonces podríamos decir
que Leví, el que recibe los diezmos, ha pagado diezmos por ser
descendiente de Abrahán, \bibleverse{10} pues aún no había nacido de su
padre\footnote{\textbf{7:10} Literalmente ``en hombros de su padre''.}
cuando Melquisedec conoció a Abrahán.

\hypertarget{el-cambio-y-aboliciuxf3n-del-sacerdocio-provocado-por-el-sacerdocio-peculiar-de-jesuxfas}{%
\subsection{El cambio y abolición del sacerdocio provocado por el
sacerdocio peculiar de
Jesús}\label{el-cambio-y-aboliciuxf3n-del-sacerdocio-provocado-por-el-sacerdocio-peculiar-de-jesuxfas}}

\bibleverse{11} Ahora, si hubiera sido posible lograr la perfección por
el sacerdocio de Leví (pues así fue como se recibió la ley), ¿Por qué
había necesidad de otro sacerdote que siguiera el orden de Melquisedec,
y no del orden de Aarón? \bibleverse{12} Si se cambia el sacerdocio, la
ley necesitaría cambiarse también. \bibleverse{13} Pero este de quien
hablamos viene de otra tribu, una tribu que nunca ha provisto sacerdotes
que sirvan en el altar. \bibleverse{14} Está claro que nuestro Señor es
descendiente de Judá, y Moisés nunca hizo mención sobre sacerdotes que
provinieran de esta tribu. \footnote{\textbf{7:14} Gén 49,10; Is 11,1;
  Mat 1,1-3} \bibleverse{15} Y esto queda aún más claro cuando vemos que
aparece otro sacerdote similar a Melquisedec, \bibleverse{16} que no
llegó al sacerdocio por virtud de su ascendencia, sino por el poder de
una vida que no puede ser destruida. \bibleverse{17} Por eso dice: ``Tú
eres sacerdote para siempre, conforme al orden de
Melquisedec''.\footnote{\textbf{7:17} Citando Salmos 110:4.}

\hypertarget{la-razuxf3n-del-cambio-en-el-orden-de-los-sacerdotes-es-que-jesuxfas-deberuxeda-ser-el-garante-de-un-pacto-superior}{%
\subsection{La razón del cambio en el orden de los sacerdotes es que
Jesús debería ser el garante de un pacto
superior}\label{la-razuxf3n-del-cambio-en-el-orden-de-los-sacerdotes-es-que-jesuxfas-deberuxeda-ser-el-garante-de-un-pacto-superior}}

\bibleverse{18} De modo que la norma anterior ha sido anulada porque era
débil e inútil, \bibleverse{19} (porque la ley nunca perfeccionó nada).
Pero ahora ha sido reemplazada por una esperanza mejor, por la cual
podemos acercarnos a Dios. \bibleverse{20} Esto\footnote{\textbf{7:20}
  Refiriéndose a una nueva forma de acercarse a Dios.} no se hizo sin un
juramento, aunque los que se convierten en sacerdotes lo hacen con un
juramento. \bibleverse{21} Pero él se convirtió en sacerdote con un
juramento porque Dios le dijo: ``El Señor ha hecho un juramento solemne
y no cambiará de opinión: Tú eres sacerdote para siempre''.\footnote{\textbf{7:21}
  Citando Salmos 110:4.}

\bibleverse{22} Es así como Jesús se convirtió en la garantía de un
acuerdo de una relación con Dios\footnote{\textbf{7:22} ``Un acuerdo de
  relación con Dios''. Esto traduce una sola palabra que en griego se
  traduce tradicionalmente como ``pacto''. Sin embargo, la palabra
  ``pacto'' normalmente no se usa en nuestro lenguaje coloquial y por
  ello se ha convertido en una palabra ``teológica''. Se ha escrito
  mucho sobre este concepto y los términos usaos, y ``pacto'' a menudo
  se ha preservado porque parece no haber una manera eficaz de explicar
  lo que se quiere decir aquí. El concepto de pacto se desarrolla más
  ampliamente en los capítulos 8 y 9. Y existen problemas con palabras
  alternativas. La palabra ``contrato'' puede significar el resultado de
  una negociación, que no es el caso. Del mismo modo, ``tratado'' o
  ``acuerdo'', desde el punto de vista humano, puede referirse a
  negociaciones mutuas. Pero aquí la palabra hace referencia a la
  iniciativa de Dios, y sin duda no se lleva a cabo entre dos
  semejantes. Quizás un mejor concepto sería ``una promesa que se pacta
  con obligaciones correspondientes'', pero tal palabrería sería más
  engorrosa.} que es mucho mejor. \footnote{\textbf{7:22} Heb 8,6; Heb
  12,24}

\bibleverse{23} Ha habido muchos sacerdotes porque la muerte les impidió
continuar su sacerdocio; \bibleverse{24} pero como Jesús vive para
siempre, su sacerdocio es permanente. \bibleverse{25} En consecuencia,
tiene el poder para salvar por completo a los que se acercan a Dios por
medio de él, viviendo siempre para rogar su caso a favor de ellos.

\hypertarget{jesuxfas-como-el-sumo-sacerdote-perfecto-y-eterno}{%
\subsection{Jesús como el sumo sacerdote perfecto y
eterno}\label{jesuxfas-como-el-sumo-sacerdote-perfecto-y-eterno}}

\bibleverse{26} Él es justamente el sumo sacerdote que necesitamos:
santo y sin falta, puro y apartado de los pecadores, y con un lugar en
lo más alto de los cielos. \bibleverse{27} A diferencia de los sumos
sacerdotes humanos, él no necesita ofrecer sacrificios diarios por sus
pecados y los de las personas. Él lo hizo una vez, y por todos, cuando
se dio a sí mismo como ofrenda. \footnote{\textbf{7:27} Lev 16,6; Lev
  16,15}

\bibleverse{28} La ley designa hombres imperfectos como sumos
sacerdotes, pero después de la ley, Dios hizo un juramento solemne, y
designó a su hijo, que es perfecto para siempre.

\hypertarget{la-superioridad-del-ministerio-sumo-sacerdotal-celestial-de-jesuxfas-y-el-nuevo-pacto-del-que-uxe9l-es-mediador}{%
\subsection{La superioridad del ministerio sumo sacerdotal celestial de
Jesús y el nuevo pacto del que él es
mediador}\label{la-superioridad-del-ministerio-sumo-sacerdotal-celestial-de-jesuxfas-y-el-nuevo-pacto-del-que-uxe9l-es-mediador}}

\hypertarget{section-7}{%
\section{8}\label{section-7}}

\bibleverse{1} El punto principal de lo que estamos diciendo es este:
tenemos tal sumo sacerdote que está sentado a la diestra de Dios, que
está sentado en majestad sobre su trono en el cielo. \bibleverse{2} Él
sirve en el santuario, el verdadero tabernáculo que fue establecido por
el Señor y no por seres humanos. \bibleverse{3} Como es responsabilidad
de todo sumo sacerdote ofrecer dones y sacrificios, este sumo sacerdote
también tiene algo que ofrecer. \bibleverse{4} Ahora bien, si él
estuviera aquí en la tierra, no sería un sacerdote en absoluto, porque
ya hay sacerdotes para presentar las ofrendas que exige la ley.
\bibleverse{5} Pero el lugar donde ellos sirven es una copia, una mera
sombra de lo que hay en el cielo. Y eso fue lo que Dios le dijo a Moisés
cuando iba a construir el tabernáculo: ``Ten cuidado de hacer todo
conforme al modelo que se te mostró en la montaña''.\footnote{\textbf{8:5}
  Citando Éxodo 25:40.} \footnote{\textbf{8:5} Col 2,17} \bibleverse{6}
Pero a Jesús se le ha dado un ministerio mucho mejor, pues él es el
único mediador de una relación mejor entre nosotros y Dios. Una relación
basada en mejores promesas. \footnote{\textbf{8:6} Heb 7,22}

\bibleverse{7} Si el primer pacto hubiera sido perfecto, no se habría
necesitado un segundo pacto. \bibleverse{8} Y hablando sus
fallos,\footnote{\textbf{8:8} Aclarando que el problema con el ``primer
  pacto'' no se debió a un acuerdo defectuoso sino a que el pueblo de
  Dios no cumplió con las responsabilidades del acuerdo.} Dios le dijo a
su pueblo: ``Estén atentos, dice el Señor, porque vienen días en que
haré un nuevo pacto en relación a la casa de Israel y la casa de Judá.
\footnote{\textbf{8:8} Heb 10,16-17} \bibleverse{9} No será como el
pacto prometido que hice con los ancestros cuando los llevé de la mano
fuera de la tierra de Egipto. Porque ellos no cumplieron con su parte en
la relación que habíamos acordado, y por eso los abandoné, dice el
Señor''. \footnote{\textbf{8:9} Éxod 19,5-6} \bibleverse{10} ``Esta es
la relación que le prometo a la casa de Israel: Después de ese tiempo,
dice el Señor, yo pondré mis leyes en sus mentes, y las escribiré en sus
corazones. Yo seré su Dios, y ellos serán mi pueblo. \bibleverse{11}
Nadie tendrá que enseñarle a su prójimo, y nadie necesitará enseñar en
su familia, diciendo: `Debes conocer al Señor'. Porque todos me
conocerán, desde el más pequeño hasta el más grande. \bibleverse{12} Y
yo seré misericordioso cuando se equivoquen, y me olvidaré de sus
pecados''.\footnote{\textbf{8:12} Citando Jeremías 31:31-34.}

\bibleverse{13} Al decir ``pacto de una nueva relación'', Dios abandona
el primer pacto. Ese pacto que ya está obsoleto y desgastado, y que casi
ha desaparecido.\footnote{\textbf{8:13} Rom 10,4}

\hypertarget{la-imperfecciuxf3n-del-ministerio-sacerdotal-levuxedtico-y-la-perfecciuxf3n-o-superioridad-del-ministerio-sumo-sacerdotal-de-cristo}{%
\subsection{La imperfección del ministerio sacerdotal levítico y la
perfección (o superioridad) del ministerio sumo sacerdotal de
Cristo}\label{la-imperfecciuxf3n-del-ministerio-sacerdotal-levuxedtico-y-la-perfecciuxf3n-o-superioridad-del-ministerio-sumo-sacerdotal-de-cristo}}

\hypertarget{section-8}{%
\section{9}\label{section-8}}

\bibleverse{1} El antiguo sistema tenía instrucciones sobre cómo adorar,
y un santuario terrenal. \bibleverse{2} En la primera sala del
tabernáculo estaba el candelabro, la mesa, y el pan sagrado. A este
lugar se le llamaba el Lugar Santo. \bibleverse{3} Al pasar el segundo
velo, se encontraba la sala que se llamaba el Lugar Santísimo.
\footnote{\textbf{9:3} Éxod 26,33} \bibleverse{4} Dentro de este lugar
estaba el altar de oro del incienso, y el ``arca del pacto'', cubierta
de oro. \footnote{\textbf{9:4} 9:4a. Traducida comúnmente como ``arca
  del pacto'', era una caja de madera que simbolizaba un sitio de
  reunión, de reconciliación, y acuerdo entre Dios y su pueblo.} Dentro
del arca se encontraba una taza de oro que contenía maná, la vara de
Aarón que reverdeció, y las inscripciones del pacto sobre
piedras.\footnote{\textbf{9:4} 9:4b. Se creía que era la piedra con las
  inscripciones de los 10 mandamientos.} \footnote{\textbf{9:4} Éxod
  16,33; Éxod 25,10-22; Núm 17,23-25} \bibleverse{5} Y encima del arca
estaba el ángel querubín protegiendo el lugar de la reconciliación. Pero
ahora no podemos hablar de esto en detalle.

\bibleverse{6} Cuando todo esto estuvo establecido, los sacerdotes ya
podrían entrar con regularidad a la primera sala para llevar a cabo sus
labores. \footnote{\textbf{9:6} Núm 18,3-4; Éxod 30,10; Lev 16,2; Lev
  16,14-15} \bibleverse{7} Pero solo el sumo sacerdote entraba a la
segunda sala, y solo una vez al año. Incluso en ese momento tenía que
hacer un sacrificio que incluyera sangre,\footnote{\textbf{9:7} La
  sangre es un tema muy frecuente en la última parte del libro de
  Hebreos. Es un símbolo abreviado de la vida, y la sangre derramada
  significa muerte, y aunque el contexto original del Sistema de
  sacrificios es literal, sin duda alguna, su uso en el libro de
  Hebreos, al aplicarlo a Cristo, es principalmente como símbolo de lo
  que él logró con su vida, muerte y resurrección.} el cual era ofrecido
por sí mismo y por los pecados que el pueblo hubiera cometido por
ignorancia. \bibleverse{8} Con esto, el Espíritu Santo indicaba que el
camino al verdadero Lugar Santísimo no se había revelado mientras aún
existía el primer tabernáculo.\footnote{\textbf{9:8} El significado de
  esta afirmación es tema de debate. En general, podríamos concluir que
  a la luz de la nueva revelación de Dios por medio de Jesús, que es el
  centro del nuevo testamento, y particularmente del libro de Hebreos,
  este pasaje se refiere a Jesús como la plena revelación de Dios,
  proporcionando un ``acceso'' hacia él, lo cual no había sucedido bajo
  el antiguo sistema (ver como referencia la afirmación de Jesús en Juan
  14:6).} \bibleverse{9} Esta es una ilustración para nosotros en el
presente, demostrándonos que los dones y sacrificios que se ofrecen no
pueden limpiar la conciencia del adorador. \footnote{\textbf{9:9} Heb
  7,19; Heb 10,1-2} \bibleverse{10} Pues esos son solamente requisitos
religiosos, que tienen que ver con la comida y la bebida, y diversas
ceremonias que implican el lavamiento, las cuales fueron impuestas hasta
que llegó el tiempo en que Dios estableció una nueva forma de
relacionarnos con él. \footnote{\textbf{9:10} Lev 11,-1; Núm 19,-1}

\bibleverse{11} Cristo ha venido como sumo sacerdote de todas las buenas
experiencias que ahora tenemos. Entró a un tabernáculo más grande y
completo que no fue hecho por manos humanas, ni es parte de este mundo
creado. \bibleverse{12} Él no entró por medio de la sangre de cabras y
becerros, sino por medio de su propia sangre. Entró una sola vez y por
todas, en el Lugar Santísimo, liberándonos para siempre. \bibleverse{13}
Pues si la sangre de cabras y toros, y las cenizas de vaca rociadas
sobre lo que está ritualmente impuro pueden hacer que el cuerpo esté
ceremonialmente puro, \footnote{\textbf{9:13} Núm 19,2; Núm 19,9; Núm
  19,17} \bibleverse{14} ¿cuánto más la sangre de Cristo, quien se
ofreció a Dios teniendo una vida sin pecado por medio del Espíritu
eterno, puede limpiar sus conciencias de sus antiguas vidas de pecado,
para que puedan servir al Dios vivo? \footnote{\textbf{9:14} Heb 1,3;
  1Pe 1,18-19; 1Jn 1,7; Apoc 1,5}

\hypertarget{cristo-como-mediador-de-un-nuevo-pacto-y-su-muerte-sacrificial-uxfanica-como-medio-eterno-de-su-servicio-celestial-como-sumo-sacerdote}{%
\subsection{Cristo como mediador de un nuevo pacto y su muerte
sacrificial única como medio eterno de su servicio celestial como sumo
sacerdote}\label{cristo-como-mediador-de-un-nuevo-pacto-y-su-muerte-sacrificial-uxfanica-como-medio-eterno-de-su-servicio-celestial-como-sumo-sacerdote}}

\bibleverse{15} Por eso él es el mediador de una nueva relación de
pacto. Puesto que la muerte ha ocurrido para liberarlos de los pecados
cometidos bajo la relación del primer pacto, ahora los que son llamados
pueden recibir la promesa de una herencia eterna. \footnote{\textbf{9:15}
  Heb 12,24; 1Tim 2,5} \bibleverse{16} Pues para que se cumpla un
testamento, quien lo hace debe morir primero. \bibleverse{17} El
testamento solo es válido cuando hay muerte, y nunca se cumple mientras
la persona aun esté viva. \bibleverse{18} Por eso el primer pacto fue
establecido con sangre. \bibleverse{19} Después que Moisés presentó
todos los mandamientos de la ley a todo el pueblo, tomó la sangre de
cabras y becerros junto con agua y roció el libro\footnote{\textbf{9:19}
  El libro de la ley.} y también a todo el pueblo, usando lana escarlata
e hisopo. \bibleverse{20} Y les dijo: ``Esta es la sangre de la relación
de pacto que Dios les ha dicho que quiere tener con
ustedes''.\footnote{\textbf{9:20} Citando Éxodo 24:8.}

\bibleverse{21} Del mismo modo, Moisés roció la sangre en el tabernáculo
y en todo lo que se usaba para el culto. \footnote{\textbf{9:21} Lev
  8,15; Lev 8,19} \bibleverse{22} Conforme a la ley ceremonial, casi
todo se purificaba con sangre, y sin derramamiento de sangre, nada
quedaría ritualmente limpio de la mancha del pecado. \footnote{\textbf{9:22}
  Lev 17,11}

\hypertarget{el-autosacrificio-uxfanico-y-sangriento-de-cristo-y-su-tremendo-significado-de-salvaciuxf3n-para-los-creyentes}{%
\subsection{El autosacrificio único y sangriento de Cristo y su tremendo
significado de salvación para los
creyentes}\label{el-autosacrificio-uxfanico-y-sangriento-de-cristo-y-su-tremendo-significado-de-salvaciuxf3n-para-los-creyentes}}

\bibleverse{23} De modo que si las copias de lo que hay en el cielo
necesitaban limpiarse de esta manera, las cosas que están en el cielo
necesitaban limpiarse con mejores sacrificios. \bibleverse{24} Porque
Cristo no ha entrado al Lugar Santísimo construido por seres humanos y
que es apenas un modelo del original. Él entró al cielo mismo, y ahora
aparece en representación de nosotros, hablando a nuestro favor en
presencia de Dios. \footnote{\textbf{9:24} Heb 7,25; 1Jn 2,1}
\bibleverse{25} Esto no tiene como fin ofrecerse repetidas veces, como
un sumo sacerdote que tiene que entrar al Lugar Santísimo después de un
año, ofreciendo sangre que no es suya. \bibleverse{26} De otro modo,
Cristo habría tenido que sufrir muchas veces desde la creación del
mundo. Pero no fue así: fue solo una vez al final de la era presente que
él vino a eliminar el pecado al sacrificarse a sí mismo. \bibleverse{27}
Y así como los seres humanos mueren una sola vez, y luego son juzgados,
\footnote{\textbf{9:27} Gén 3,19} \bibleverse{28} del mismo modo ocurre
con Cristo. Pues al haber sido sacrificado una sola vez para quitar los
pecados de muchos, vendrá otra vez, no para hacerse cargo del pecado,
sino para salvar a quienes lo esperan.\footnote{\textbf{9:28} Heb 10,10;
  Heb 10,12; Heb 10,14}

\hypertarget{el-ejemplo-sombruxedo-y-la-insuficiencia-del-sacrificio-anual-de-reconciliaciuxf3n-del-sumo-sacerdote-levuxedtico-la-perfecciuxf3n-del-sacrificio-de-jesuxfas}{%
\subsection{El ejemplo sombrío y la insuficiencia del sacrificio anual
de reconciliación del sumo sacerdote levítico; la perfección del
sacrificio de
Jesús}\label{el-ejemplo-sombruxedo-y-la-insuficiencia-del-sacrificio-anual-de-reconciliaciuxf3n-del-sumo-sacerdote-levuxedtico-la-perfecciuxf3n-del-sacrificio-de-jesuxfas}}

\hypertarget{section-9}{%
\section{10}\label{section-9}}

\bibleverse{1} La ley es apenas una sombra de las cosas buenas que
vendrían, y no de la realidad como tal. De modo que no podía justificar
a los que venían a adorar a Dios por medio de sacrificios repetitivos
que se ofrecían cada año. \footnote{\textbf{10:1} Heb 8,5}
\bibleverse{2} De otro modo ¿no se habrían detenido los sacrificios? Si
los adoradores hubieran sido limpiados una vez y para siempre, nunca más
habrían tenido conciencias culpables. \bibleverse{3} Pero tales
sacrificios, en efecto, le recuerdan a la gente los pecados año tras
año, \bibleverse{4} porque es imposible que la sangre de toros y cabras
quite los pecados. \bibleverse{5} Por eso, cuando Cristo\footnote{\textbf{10:5}
  El original dice simplemente ``él;'' Se infiere que es Cristo por los
  versículos 9:24, 9:28.} vino al mundo dijo: ``Tú no querías
sacrificios ni ofrendas, sino que preparaste un cuerpo para mí.
\bibleverse{6} Las ofrendas quemadas y los sacrificios por el pecado no
te agradaron. \bibleverse{7} Entonces dije: `Dios, considera que he
venido a hacer tu voluntad, tal como se dice de mí en el
libro'\,''.\footnote{\textbf{10:7} En realidad dice ``el encabezamiento
  de un rollo'', queriendo decir, las Escrituras. La cita es de Salmos
  40:6-8.}

\bibleverse{8} Como se menciona arriba: ``No quisiste sacrificios ni
ofrendas. Las ofrendas quemadas y los sacrificios por el pecado no te
agradaron'', (aunque eran ofrecidos conforme a los requisitos de la
ley). \bibleverse{9} Entonces él dijo: ``Mira, he venido a hacer tu
voluntad''. Entonces él abandona el primer pacto para establecer el
segundo, \bibleverse{10} por medio del cual todos somos santificados a
través de Jesucristo, quien ofrece su cuerpo una vez y para siempre.
\footnote{\textbf{10:10} Juan 17,19}

\hypertarget{el-autosacrificio-uxfanico-y-perfectamente-vuxe1lido-de-jesuxfas-hace-que-todos-los-demuxe1s-sacrificios-por-el-pecado-sean-innecesarios-porque-hizo-que-los-creyentes-fueran-completamente-perfectos-ante-dios}{%
\subsection{El autosacrificio único y perfectamente válido de Jesús hace
que todos los demás sacrificios por el pecado sean innecesarios porque
hizo que los creyentes fueran completamente perfectos ante
Dios}\label{el-autosacrificio-uxfanico-y-perfectamente-vuxe1lido-de-jesuxfas-hace-que-todos-los-demuxe1s-sacrificios-por-el-pecado-sean-innecesarios-porque-hizo-que-los-creyentes-fueran-completamente-perfectos-ante-dios}}

\bibleverse{11} Todos los sumos sacerdotes ofician en los servicios cada
día, una y otra vez, ofreciendo los mismos sacrificios que no pueden
quitar los pecados. \footnote{\textbf{10:11} Éxod 29,38} \bibleverse{12}
Pero este Sacerdote, después de ofrecer un solo sacrificio por los
pecados, que dura para siempre, se sentó a la diestra de Dios.
\bibleverse{13} Y ahora espera hasta que todos sus enemigos sean
vencidos, y vengan a ser como banquillo para sus pies. \footnote{\textbf{10:13}
  Sal 110,1} \bibleverse{14} Porque con un solo sacrificio él justificó
para siempre a los que están siendo santificados.

\bibleverse{15} Tal como nos dice el Espíritu Santo, por haber dicho:
\bibleverse{16} ``Este es el pacto que haré con ellos más adelante, dice
el Señor. Pondré mis leyes en sus corazones, y las escribiré en sus
mentes''. Entonces añade:

\bibleverse{17} ``Nunca más me acordaré de sus pecados e
iniquidades''.\footnote{\textbf{10:17} Citando Jeremías 31:33-34.}
\footnote{\textbf{10:17} Heb 8,12}

\bibleverse{18} Después de estar libres de tales cosas, las ofrendas por
el pecado ya no son necesarias.

\hypertarget{amonestaciuxf3n-general-para-perseverar-en-la-fe-la-esperanza-y-el-amor-en-comunidad-con-toda-la-comunidad}{%
\subsection{Amonestación general para perseverar en la fe, la esperanza
y el amor, en comunidad con toda la
comunidad}\label{amonestaciuxf3n-general-para-perseverar-en-la-fe-la-esperanza-y-el-amor-en-comunidad-con-toda-la-comunidad}}

\bibleverse{19} Ahora tenemos esta seguridad, hermanos y hermanas, de
poder entrar al Lugar Santísimo por la sangre de Jesús. \bibleverse{20}
Por medio de su vida y muerte,\footnote{\textbf{10:20} ``Su vida y
  muerte'': literalmente ``su cuerpo''.} él abrió a través del velo que
nos lleva hacia Dios, una nueva forma de vivir. \footnote{\textbf{10:20}
  Heb 9,8} \bibleverse{21} Siendo que tenemos este gran sacerdote que
está a cargo de la casa de Dios, \bibleverse{22} acerquémonos a Dios,
con mentes sinceras y plena confianza. Nuestras mentes han sido rociadas
para purificarlas de nuestros malos pensamientos, y nuestros cuerpos han
sido lavados y limpiados con agua pura.

\bibleverse{23} Así que aferrémonos a la esperanza de la cual les
hablamos a otros, y sin dudar, porque el Dios que prometió es fiel.
\footnote{\textbf{10:23} Heb 4,14}

\bibleverse{24} Pensemos en cómo podemos animarnos unos a otros a amar y
hacer el bien. \bibleverse{25} No deberíamos desistir en cuanto a
reunirnos, como algunos lo han hecho. De hecho, deberíamos animarnos
unos a otros, especialmente cuando vemos que el Fin\footnote{\textbf{10:25}
  Literalmente ``el Día''.} se acerca.

\hypertarget{advertencia-de-apostasuxeda-y-del-juicio-divino-que-golpearuxe1-a-los-que-se-burlan-de-la-gracia}{%
\subsection{Advertencia de apostasía y del juicio divino que golpeará a
los que se burlan de la
gracia}\label{advertencia-de-apostasuxeda-y-del-juicio-divino-que-golpearuxe1-a-los-que-se-burlan-de-la-gracia}}

\bibleverse{26} Porque si seguimos pecando deliberadamente después de
haber entendido la verdad, ya no hay sacrificio para los pecados.
\footnote{\textbf{10:26} Heb 6,4-8} \bibleverse{27} Lo único que queda
es el temor, la espera de un juicio inminente y el fuego terrible que
destruye a los que son rebeldes con Dios. \bibleverse{28} Quien rechaza
la ley de Moisés es llevado a muerte sin misericordia, ante la evidencia
de dos o tres testigos. \bibleverse{29} ¿Cuánto más merecedores de
castigo creen que serán quienes hayan pisoteado al Hijo de Dios, siendo
que han menospreciado la sangre que selló el pacto que nos santificaba,
considerándolo como ordinario y trivial, y que han abusado del Espíritu
de gracia? \footnote{\textbf{10:29} Heb 2,3; Heb 12,25} \bibleverse{30}
Conocemos a Dios, y él dijo: ``Me aseguraré de hacer justicia; le daré a
la gente lo que merece''. También dijo: ``El Señor juzgará a su
pueblo''.\footnote{\textbf{10:30} Citando Deuteronomio 32:35-36; Salmos
  135:14.} \bibleverse{31} ¡Cosa terrible es caer en manos del Dios
vivo!

\hypertarget{recordatorio-para-ser-fiel-y-tener-confianza-en-la-esperanza-frente-al-sufrimiento-creciente-en-vista-de-la-recompensa-prometida}{%
\subsection{Recordatorio para ser fiel y tener confianza en la esperanza
frente al sufrimiento creciente en vista de la recompensa
prometida}\label{recordatorio-para-ser-fiel-y-tener-confianza-en-la-esperanza-frente-al-sufrimiento-creciente-en-vista-de-la-recompensa-prometida}}

\bibleverse{32} Recuerden el pasado, cuando después de entender la
verdad,\footnote{\textbf{10:32} Literalmente ``fueron iluminaos''.}
experimentaron gran sufrimiento. \footnote{\textbf{10:32} Heb 6,4}
\bibleverse{33} En ocasiones fueron mostrados como espectáculos, siendo
insultados y atacados. En otros tiempos ustedes se mantuvieron siendo
solidarios con los que sufrían. \footnote{\textbf{10:33} 1Cor 4,9}
\bibleverse{34} Mostraron compasión con los que estaban en la cárcel, y
entregaron con alegría sus posesiones cuando les fueron confiscadas,
sabiendo que cosas mejores vendrán para ustedes, cosas que realmente
perdurarán. \footnote{\textbf{10:34} Mat 6,20; Mat 19,21; Mat 19,29}
\bibleverse{35} Así que no pierdan su fe en Dios, porque será
recompensada con abundancia. \bibleverse{36} Es necesario que sean
pacientes, para que habiendo hecho la voluntad de Dios, reciban lo que
él ha prometido. \bibleverse{37} ``En poco tiempo vendrá, tal como lo
dijo, y no tardará. \bibleverse{38} Los que hacen lo recto vivirán por
su fe en Dios, y si se retractan de su compromiso, no me agradaré de
ellos''.\footnote{\textbf{10:38} 10:37-38. Esta es más bien una
  referencia libre a Isaías 26:20 y a Habacuc 2:3-4. Sin duda el que
  prometió regresar, en este contexto, es visto como Jesús.} \footnote{\textbf{10:38}
  Rom 1,17}

\bibleverse{39} Pero nosotros no somos la clase de personas que se
retracta y termina en la perdición. Nosotros somos los que creemos en
Dios y su salvación.\footnote{\textbf{10:39} 1Tes 3,3}

\hypertarget{section-10}{%
\section{11}\label{section-10}}

\bibleverse{1} Ahora bien, nuestra fe en Dios es la seguridad de lo que
esperamos, la evidencia de lo que no podemos ver. \footnote{\textbf{11:1}
  2Cor 5,7}

\hypertarget{modelos-del-antiguo-testamento-de-tal-fe}{%
\subsection{Modelos del Antiguo Testamento de tal
fe}\label{modelos-del-antiguo-testamento-de-tal-fe}}

\bibleverse{2} Los que vivieron hace mucho tiempo, creyeron en Dios y
eso fue lo que les hizo obtener la aprobación de Dios. \bibleverse{3}
Mediante nuestra fe en Dios comprendemos que todo el universo fue creado
por su mandato, y que lo que se ve fue hecho a partir de lo que no se
puede ver.

\hypertarget{tres-ejemplos-de-huxe9roes-de-la-fe-de-la-uxe9poca-de-los-antepasados-de-abel-a-nouxe9}{%
\subsection{Tres ejemplos de héroes de la fe de la época de los
antepasados \hspace{0pt}\hspace{0pt}de Abel a
Noé}\label{tres-ejemplos-de-huxe9roes-de-la-fe-de-la-uxe9poca-de-los-antepasados-de-abel-a-nouxe9}}

\bibleverse{4} Por la fe en Dios Abel ofreció a Dios mejor sacrificio
que Caín, y por eso Dios lo señaló como alguien que vivía rectamente.
Dios lo demostró al aceptar su ofrenda. Aunque Abel ha estado muerto por
mucho tiempo, todavía Dios nos habla por medio de lo que él hizo.
\footnote{\textbf{11:4} Gén 4,4}

\bibleverse{5} Por fe en Dios Enoc fue llevado al cielo para que no
experimentara la muerte. Y no pudieron encontrarlo en la tierra porque
fue llevado al cielo. Y antes de esto, a Enoc se le conocía como alguien
que agradaba a Dios.\footnote{\textbf{11:5} Véase Génesis 5:24.}
\footnote{\textbf{11:5} Gén 5,24} \bibleverse{6} ¡No podemos esperar que
Dios se agrade de nosotros si no confiamos en él! Todo el que se acerca
a Dios debe creer que él existe, y que recompensa a quienes lo buscan.

\bibleverse{7} Noé creyó en Dios, y él mismo le advirtió sobre cosas que
nunca antes habían sucedido. Y como Noé atendió lo que Dios le dijo,
construyó un arca para salvar a su familia. Y por fe en Dios, Noé mostró
que el mundo estaba equivocado, y recibió la recompensa de ser
justificado por Dios. \footnote{\textbf{11:7} Gén 6,8-9; Gén 6,13-22}

\hypertarget{ejemplos-de-la-uxe9poca-de-abraham-y-su-familia}{%
\subsection{Ejemplos de la época de Abraham y su
familia}\label{ejemplos-de-la-uxe9poca-de-abraham-y-su-familia}}

\bibleverse{8} Por la fe en Dios Abrahán obedeció cuando Dios lo llamó
para ir a la tierra que él le daría. Y partió sin saber hacia dónde iba.
\footnote{\textbf{11:8} Gén 12,1-21} \bibleverse{9} Por fe en Dios vivió
en la tierra prometida, pero como extranjero, viviendo en tiendas junto
a Isaac y Jacob, quienes participaron con él al ser herederos de la
misma promesa. \bibleverse{10} Porque Abrahán buscaba una ciudad
construida sobre fundamentos duraderos, siendo Dios el constructor y
hacedor de ella.

\bibleverse{11} Por su fe en Dios, incluso la misma Sara\footnote{\textbf{11:11}
  Algunas versiones dicen Abrahán.} pudo concebir un hijo aunque fuera
muy vieja para hacerlo, pues creyó en Dios, que había hecho la promesa.
\bibleverse{12} Por eso, los descendientes de Abrahán, (¡que ya estaba a
punto de morir!), se volvieron numerosos como las estrellas del cielo e
innumerables como la arena del mar.

\bibleverse{13} Y todos ellos murieron creyendo aún en Dios. Aunque no
recibieron las cosas que Dios prometió, todavía las esperaban, como
desde la distancia y lo aceptaron gustosos, sabiendo que eran
extranjeros en esta tierra, pasajeros solamente. \footnote{\textbf{11:13}
  Gén 23,4; Gén 47,9} \bibleverse{14} Quienes hablan de esta manera
dejan ver que esperan un país que es de ellos. \bibleverse{15} Porque si
les importara el país que habían dejado atrás, habrían regresado.
\bibleverse{16} Pero ellos esperan un mejor país, un país celestial. Por
eso Dios no se defrauda de ellos, y se alegra de llamarse su Dios,
porque él ha construido una ciudad para ellos.

\bibleverse{17} Abrahán creyó en Dios cuando fue puesto a prueba y
ofreció a Isaac como ofrenda a Dios. Abrahán, quien había aceptado las
promesas de Dios, incluso estuvo listo para dar a su único
hijo,\footnote{\textbf{11:17} Por supuesto que Isaac no era literalmente
  el único hijo de Abrahán; el término griego indica primacía.} como
ofrenda \footnote{\textbf{11:17} Gén 22,-1; Sant 2,21} \bibleverse{18}
aun cuando se le había dicho: ``Por medio de Isaac se contará tu
descendencia''.\footnote{\textbf{11:18} Véase Génesis 21:12.}
\bibleverse{19} Abrahán consideró las cosas y concluyó que Dios podía
resucitar a Isaac de los muertos. Y en cierto modo eso fue lo que
sucedió: Abrahán recibió de vuelta a Isaac de entre los muertos.

\bibleverse{20} Por la fe en Dios, Isaac bendijo a Jacob y a Esaú,
considerando lo que el futuro traería.

\bibleverse{21} Confiando en Dios, Jacob, casi a punto de morir, bendijo
a los hijos de José, y adoró a Dios apoyado en su bastón.

\bibleverse{22} Por fe en Dios, José, cuando se acercaba su hora de
muerte también, habló sobre el éxodo de los israelitas, e instruyó sobre
lo que debían hacer con sus huesos.

\hypertarget{ejemplos-de-la-uxe9poca-de-moisuxe9s-y-josuuxe9}{%
\subsection{Ejemplos de la época de Moisés y
Josué}\label{ejemplos-de-la-uxe9poca-de-moisuxe9s-y-josuuxe9}}

\bibleverse{23} Por fe en Dios, los padres de Moisés lo ocultaron
durante tres meses después de nacer. Reconocieron que era un niño
especial. Y no temieron ir en contra de la orden que se había dado.
\footnote{\textbf{11:23} Éxod 2,-1; Éxod 12,1-12; Éxod 14,1-14}

\bibleverse{24} Por fe en Dios, Moisés, siendo ya adulto, se rehusó a
ser conocido como el hijo adoptivo de la hija del Faraón.
\bibleverse{25} Sino que prefirió participar de los sufrimientos del
pueblo de Dios antes que disfrutar los placeres pasajeros del pecado.
\bibleverse{26} Y consideró que el rechazo que experimentaría por seguir
a Cristo sería de mayor valor que la riqueza de Egipto, porque estaba
concentrado en la recompensa que vendría. \bibleverse{27} Por fe en
Dios, salió de Egipto y no tuvo temor de la ira del Faraón, sino que
siguió adelante con sus ojos fijos en el Dios invisible. \bibleverse{28}
Por fe en Dios, Moisés observó la Pascua y la aspersión de la sangre en
los dinteles, para que el ángel destructor no tocara a los
israelitas.\footnote{\textbf{11:28} ``Ángel'' e ``Israelitas'' por
  contexto.}

\bibleverse{29} Por fe en Dios, los israelitas cruzaron en Mar Rojo como
si caminaran por tierra seca. Y cuando los egipcios quisieron hacer lo
mismo, murieron ahogados.

\bibleverse{30} Por la fe en Dios, los israelitas marcharon alrededor de
los muros de Jericó durante siete días, y los muros cayeron.

\bibleverse{31} Por fe en Dios, Rahab, la prostituta, no murió junto a
los que rechazaban a Dios, porque había recibido a los espías israelitas
en paz.

\hypertarget{ejemplos-de-huxe9roes-de-la-fe-de-la-historia-posterior-de-israel}{%
\subsection{Ejemplos de héroes de la fe de la historia posterior de
Israel}\label{ejemplos-de-huxe9roes-de-la-fe-de-la-historia-posterior-de-israel}}

\bibleverse{32} ¿Qué otro ejemplo podría mostrarles? El tiempo no me
alcanza para hablar de Gedeón, Barac, Sansón, Jefté; o sobre David,
Samuel y los profetas. \bibleverse{33} Ellos, por su fe en Dios
conquistaron reinos, hicieron lo recto, recibieron las promesas de Dios,
cerraron la boca de leones, \bibleverse{34} apagaron incendios,
escaparon de la muerte por espada, eran débiles pero se volvieron
fuertes, lograron grandes cosas en guerras, y dirigieron ejércitos.
\bibleverse{35} Muchas mujeres recibieron a sus familiares con vida por
medio de la resurrección. Otros fueron torturados, al negarse a rechazar
a Dios para ser perdonados, porque querían ser parte de una mejor
resurrección. \bibleverse{36} E incluso otros recibieron insultos y
latigazos; y fueron encadenados y encarcelados. \bibleverse{37} Algunos
fueron apedreados, tentados, muertos a espada. Algunos fueron vestidos
con pieles de corderos y cabras: destituidos, oprimidos y maltratados.
\bibleverse{38} Les digo que el mundo no era digno de tener a tales
personas errantes en los desiertos y montañas, viviendo en cuevas y en
huecos debajo de la tierra.

\bibleverse{39} Todas estas personas, aunque tenían la aprobación de
Dios, no recibieron lo que Dios había prometido. \bibleverse{40} Él nos
ha dado algo aún mejor, para que ellos no llegaran a la plenitud sin
nosotros.

\hypertarget{exhortaciuxf3n-a-mantener-la-fidelidad-especialmente-con-respecto-al-ejemplo-de-jesuxfas}{%
\subsection{Exhortación a mantener la fidelidad, especialmente con
respecto al ejemplo de
Jesús}\label{exhortaciuxf3n-a-mantener-la-fidelidad-especialmente-con-respecto-al-ejemplo-de-jesuxfas}}

\hypertarget{section-11}{%
\section{12}\label{section-11}}

\bibleverse{1} Por eso, siendo que estamos rodeados de tal multitud de
personas que demostraron su fe en Dios, despojémonos de todo lo que nos
detiene, del pecado seductor que nos hace tropezar, y sigamos corriendo
la carrea que tenemos por delante. \bibleverse{2} Debemos seguir con la
mirada puesta en Jesús, el autor y perfeccionador de nuestra fe en Dios.
Pues por el gozo que tenía delante, Jesús soportó la cruz, sin
importarle su vergüenza, y se sentó a la diestra del trono de Dios.
\footnote{\textbf{12:2} Heb 5,8-9; Fil 2,8; Fil 2,10}

\bibleverse{3} Piensen en Jesús, quien soportó tal hostilidad de un
pueblo pecador, y así no se cansarán ni se desanimarán. \footnote{\textbf{12:3}
  Luc 2,34; Mat 26,67}

\hypertarget{recordatorio-para-permitir-que-los-desafuxedos-del-sufrimiento-sirvan-como-medio-para-promover-la-vida-de-fe}{%
\subsection{Recordatorio para permitir que los desafíos del sufrimiento
sirvan como medio para promover la vida de
fe}\label{recordatorio-para-permitir-que-los-desafuxedos-del-sufrimiento-sirvan-como-medio-para-promover-la-vida-de-fe}}

\bibleverse{4} Hasta ahora, la lucha contra el pecado no les ha costado
su sangre. \bibleverse{5} ¿Acaso han olvidado\footnote{\textbf{12:5} O
  ``Ustedes han olvidado''.} el llamado de Dios cuando les habla como a
hijos suyos? Él dice: ``Hijo mío, no tomes con ligereza la disciplina de
Dios, ni te des por vencido cuando te corrige. \bibleverse{6} Porque el
Señor disciplina a los que ama, y castiga a todos los que recibe como
sus hijos''. \footnote{\textbf{12:6} Apoc 3,19}

\bibleverse{7} Así que sean pacientes cuando experimenten la disciplina
de Dios, porque quiere decir que los está tratando como a sus hijos.
¿Qué hijo no experimenta la disciplina de su padre? \bibleverse{8} Si no
reciben disciplina, (la cual todos hemos recibido), entonces son
ilegítimos, y no son hijos de verdad. \bibleverse{9} Porque si
respetábamos a nuestros padres terrenales que nos disciplinan, ¿cuánto
más deberíamos estar sujetos a la disciplina de nuestro Padre
espiritual, que nos conduce a la vida? \bibleverse{10} Ellos nos
disciplinaron por un tiempo, en lo que ellos consideraban inapropiado,
pero Dios lo hace por nuestro bien, a fin de que podamos participar de
su carácter santo. \bibleverse{11} Cuando la recibimos, la disciplina
nos parece dolorosa, y no sentimos que traiga felicidad. Pero después
produce paz en los que han sido entrenados de esta forma para hacer lo
recto.

\hypertarget{una-advertencia-a-la-comunidad-para-que-se-levante-y-cuide-a-los-miembros-duxe9biles-y-vulnerables}{%
\subsection{Una advertencia a la comunidad para que se levante y cuide a
los miembros débiles y
vulnerables}\label{una-advertencia-a-la-comunidad-para-que-se-levante-y-cuide-a-los-miembros-duxe9biles-y-vulnerables}}

\bibleverse{12} Así que fortalezcan sus manos cansadas, y sus rodillas
débiles.\footnote{\textbf{12:12} Citando Isaías 35:3.} \footnote{\textbf{12:12}
  Is 35,3} \bibleverse{13} Tracen un camino recto sobre el cual
caminar,\footnote{\textbf{12:13} Citando Proverbios 4:26.} para que los
que son inválidos no se descarríen, sino que sean sanados. \footnote{\textbf{12:13}
  Prov 4,26-27}

\bibleverse{14} Esfuércense por estar en paz con todos y buscar la
santidad, pues de lo contrario no verán al Señor. \footnote{\textbf{12:14}
  Rom 12,18; 2Tim 2,22} \bibleverse{15} Asegúrense de que no les falte
la gracia de Dios, en caso de que surja alguna causa de
amargura\footnote{\textbf{12:15} Ver Deuteronomio 29:18.} y tribulación
y termine corrompiendo a muchos entre ustedes. \footnote{\textbf{12:15}
  Deut 29,17} \bibleverse{16} Asegúrense de que ninguno sea sexualmente
inmoral o profano como Esaú. Él vendió su primogenitura por una sola
comida. \footnote{\textbf{12:16} Gén 25,33-34} \bibleverse{17} Recuerden
que incluso quiso recibir la bendición después que le fue negada. Y
aunque lo intentó, y lloró amargamente, no pudo cambiar lo que había
hecho. \footnote{\textbf{12:17} Gén 27,30-40}

\hypertarget{otra-referencia-a-la-soberanuxeda-del-nuevo-pacto-y-la-inminente-decisiuxf3n-final}{%
\subsection{Otra referencia a la soberanía del nuevo pacto y la
inminente decisión
final}\label{otra-referencia-a-la-soberanuxeda-del-nuevo-pacto-y-la-inminente-decisiuxf3n-final}}

\bibleverse{18} Ustedes no han llegado a una montaña de
verdad\footnote{\textbf{12:18} Sin duda en este contexto se hace
  referencia al Monte Sinaí.} que pueda tocarse, ni a un lugar que arda
con fuego, ni tampoco a un lugar de tormenta u oscuridad, \footnote{\textbf{12:18}
  Éxod 19,12; Éxod 19,16; Éxod 19,18; Deut 4,11} \bibleverse{19} donde
se haya escuchado una trompeta o voz que habla, y quienes oyeron esa voz
rogaron no volver a oírla nunca más. \footnote{\textbf{12:19} Éxod 20,19}
\bibleverse{20} Porque no pudieron obedecer lo que se les dijo, como por
ejemplo: ``Incluso si un animal toca la montaña, será apedreado hasta la
muerte''.\footnote{\textbf{12:20} Citando Éxodo 19:12-13}
\bibleverse{21} Semejante panorama era tan aterrador, que el mismo
Moisés dijo: ``¡Tengo tanto miedo que estoy temblando!''\footnote{\textbf{12:21}
  Citando Deuteronomio 9:19.}

\bibleverse{22} Pero ustedes han llegado al Monte de Sión, la ciudad del
Dios viviente, la Jerusalén celestial, con sus miles y miles de ángeles.
\footnote{\textbf{12:22} Gal 4,26; Efes 2,6; Fil 3,20; Apoc 5,11; Apoc
  21,2} \bibleverse{23} Han venido a la iglesia de los primogénitos
cuyos nombres están escritos en el cielo; a Dios, el juez de todos, y
donde están las personas buenas, cuyas vidas están completas.
\footnote{\textbf{12:23} Luc 10,20} \bibleverse{24} Han venido a Jesús,
quien participa con nosotros de esta nueva relación de pacto; han venido
a la sangre esparcida que tiene más valor que la de Abel.\footnote{\textbf{12:24}
  Probablemente quiere decir que Jesús derramó su sangre en un espíritu
  de perdón, mientras que en el contexto de la primera muerte Dios hace
  referencia a la sangre de Abel, como pidiendo venganza.} \footnote{\textbf{12:24}
  Heb 9,15; Gén 4,10}

\hypertarget{la-gloria-del-fin-de-los-tiempos-aterradora-para-los-reacios-y-dichosa-para-los-obedientes}{%
\subsection{La gloria del fin de los tiempos, aterradora para los
reacios y dichosa para los
obedientes}\label{la-gloria-del-fin-de-los-tiempos-aterradora-para-los-reacios-y-dichosa-para-los-obedientes}}

\bibleverse{25} ¡Asegúrense de no rechazar al que les está hablando! Si
ellos no pudieron escapar cuando rechazaron a Dios en la tierra, sin
duda alguna nosotros tampoco podremos escapar si volvemos nuestra
espalda a Dios, quien nos advierte desde el cielo. \footnote{\textbf{12:25}
  Heb 2,2; Heb 10,28-29} \bibleverse{26} En ese tiempo la voz de Dios
agitó la tierra, pero ahora su promesa es: ``Una vez más voy a agitar no
solo la tierra sino también el cielo''.\footnote{\textbf{12:26} Citando
  Ageo 2:6.} \bibleverse{27} La expresión ``una vez más'', indica que
toda la creación será agitada y removida para que solo permanezca lo
inconmovible. \bibleverse{28} Siendo que estamos recibiendo un reino
inconmovible, tengamos una actitud llena de gracia, para que sirvamos a
Dios de una manera que le agrade, con reverencia y respeto.
\bibleverse{29} Porque ``nuestro Dios es fuego
consumidor''.\textsuperscript{{[}\textbf{12:29} Citando Deuteronomio
4:24.{]}}{[}\textbf{12:29} Heb 10,31; Deut 4,24{]}

\hypertarget{advertencias-individuales-por-el-amor-fraterno-la-pureza-moral-y-la-promociuxf3n-de-la-vida-comunitaria}{%
\subsection{Advertencias individuales por el amor fraterno, la pureza
moral y la promoción de la vida
comunitaria}\label{advertencias-individuales-por-el-amor-fraterno-la-pureza-moral-y-la-promociuxf3n-de-la-vida-comunitaria}}

\hypertarget{section-12}{%
\section{13}\label{section-12}}

\bibleverse{1} ¡Que siempre permanezca el amor que tienen unos por otros
como hermanos y hermanas! \footnote{\textbf{13:1} Juan 13,34; 2Pe 1,7}
\bibleverse{2} No olviden mostrar amor por los extranjeros también,
porque al hacerlo muchos han recibido ángeles sin saberlo. \footnote{\textbf{13:2}
  Gén 18,3; Gén 19,2-3; Rom 12,13; 1Pe 4,9; 3Jn 1,5-8} \bibleverse{3}
Acuérdense de los que están en la cárcel, como si ustedes estuvieran
presos con ellos. Acuérdense de aquellos que son maltratados, como si
ustedes sufrieran físicamente con ellos. \footnote{\textbf{13:3} Mat
  25,36} \bibleverse{4} Todos deben honrar el matrimonio. Los esposos y
esposas deben ser fieles unos a otros.\footnote{\textbf{13:4}
  Literalmente, ``la cama no contaminada''.} Pues Dios juzgará a los
adúlteros.

\bibleverse{5} No amen el dinero. Estén contentos con lo que tienen.
Dios mismo dijo: ``Nunca te defraudaré; nunca te
abandonaré''.\footnote{\textbf{13:5} Citando Deuteronomio 31:6-8; Josué
  1:5.} \footnote{\textbf{13:5} 1Tim 6,6} \bibleverse{6} Por eso podemos
decir con toda confianza: ``Es Señor es mi ayudador, por lo tanto no
temeré. ¿Qué puede hacerme cualquier persona?''\footnote{\textbf{13:6}
  Citando Salmos 118:6.}

\hypertarget{amonestaciuxf3n-principal-de-ser-fieles-a-los-gobernantes-y-a-jesuxfas-el-que-permanece-en-la-eternidad-y-el-fin-del-servicio-del-sacrificio-por-el-pecado-juduxedo}{%
\subsection{Amonestación principal de ser fieles a los gobernantes y a
Jesús, el que permanece en la eternidad y el fin del servicio del
sacrificio por el pecado
judío}\label{amonestaciuxf3n-principal-de-ser-fieles-a-los-gobernantes-y-a-jesuxfas-el-que-permanece-en-la-eternidad-y-el-fin-del-servicio-del-sacrificio-por-el-pecado-juduxedo}}

\bibleverse{7} Recuerden a los líderes que les enseñaron la palabra de
Dios. Miren nuevamente los frutos de sus vidas, e imiten su fe en Dios.
\bibleverse{8} Jesucristo es el mismo ayer, hoy y para siempre.
\bibleverse{9} No se distraigan con distintas clases de enseñanzas
extrañas. Es mejor que la mente esté convencida por gracia y no por
leyes en lo que concierne a los alimentos.\footnote{\textbf{13:9} Aquí,
  la palabra simplemente es ``comida'', pero el contexto que sigue se
  refiere a la ley ceremonial y a los tipos de comida que se permitían.}
Los que seguían tales leyes no lograron nada. \footnote{\textbf{13:9}
  2Cor 1,21; 1Tim 4,8; Rom 14,17; Efes 4,14}

\bibleverse{10} Tenemos un altar del cual no pueden comer los sacerdotes
del tabernáculo. \bibleverse{11} Los cuerpos muertos de animales, cuya
sangre es llevada por el sumo sacerdote al lugar santísimo como ofrenda
para el pecado, son quemados a las afueras del campamento.
\bibleverse{12} Del mismo modo, Jesús, murió también fuera de las
puertas de la ciudad para santificar al pueblo de Dios por medio de su
propia sangre. \footnote{\textbf{13:12} Juan 19,17; Mat 21,39}
\bibleverse{13} Así que vayamos a él, fuera del campamento, y
experimentemos su vergüenza. \footnote{\textbf{13:13} Heb 11,26; Heb
  12,2} \bibleverse{14} Pues no tenemos una ciudad permanente en la cual
vivir aquí, sino que esperamos un hogar que está por venir. \footnote{\textbf{13:14}
  Heb 11,10; Heb 12,22} \bibleverse{15} Ofrezcamos, pues, por medio de
Jesús, un sacrificio continuo de alabanza a Dios, es decir, hablando
bien de Dios, y declarando su carácter.\footnote{\textbf{13:15}
  Literalmente, ``nombre'', que a menudo se refiere a la naturaleza y
  carácter de la persona que se describe. Esto se logra ver en algunas
  expresiones como ``ser de buen nombre'', para indicar el carácter.}
\footnote{\textbf{13:15} Os 14,3; Sal 50,14; Sal 50,23}

\hypertarget{advertencias-individuales-repetidas-especialmente-con-respecto-al-comportamiento-contra-los-luxedderes-comunitarios}{%
\subsection{Advertencias individuales repetidas, especialmente con
respecto al comportamiento contra los líderes
comunitarios}\label{advertencias-individuales-repetidas-especialmente-con-respecto-al-comportamiento-contra-los-luxedderes-comunitarios}}

\bibleverse{16} Y no olviden hacer lo bueno, y compartir lo que tienen
con otros, porque Dios se agrada cuando hacen tales sacrificios.

\bibleverse{17} Sigan a sus líderes, y hagan lo que ellos les piden,
porque ellos cuidan de ustedes y darán cuenta. Actúen de tal manera que
ellos puedan hacerlo con alegría, y no con tristeza, pues eso no sería
bueno para ustedes. \footnote{\textbf{13:17} 1Tes 5,12; Ezeq 3,17-19}

\bibleverse{18} Por favor, oren por nosotros. Pues estamos seguros de
que hemos actuado bien y con buena conciencia, procurando siempre hacer
lo correcto en cada situación. \footnote{\textbf{13:18} Rom 15,30; 2Cor
  1,11-12} \bibleverse{19} De verdad quiero que oren mucho para que
pueda ir pronto a verlos.

\hypertarget{clausura-de-la-carta-bendiciuxf3n-mensajes-personales-saludos}{%
\subsection{Clausura de la carta, bendición, mensajes personales,
saludos}\label{clausura-de-la-carta-bendiciuxf3n-mensajes-personales-saludos}}

\bibleverse{20} Ahora pues, que el Dios de paz que resucitó de los
muertos a nuestro Señor Jesús, el gran pastor de las ovejas, y lo hizo
con la sangre de un pacto eterno, \footnote{\textbf{13:20} Juan 10,12;
  1Pe 2,25} \bibleverse{21} provea todo lo bueno para ustedes a fin de
que puedan cumplir su voluntad. Que obre en nosotros, haciendo su
voluntad, por medio de Jesucristo, a él sea la gloria por siempre y para
siempre. Amén.

\bibleverse{22} Quiero animarlos, hermanos y hermanas, a que pongan
cuidado a lo que les he dicho en esta pequeña carta. \bibleverse{23}
Sepan que Timoteo ha sido liberado. Si llega pronto aquí, iré con él a
verlos.

\bibleverse{24} Envíen mi saludo a todos sus líderes, y a todos los
creyentes que hay allá. Los creyentes que están aquí en Italia envían
sus saludos.

\bibleverse{25} Que el Dios de gracia esté con todos ustedes. Amén.
