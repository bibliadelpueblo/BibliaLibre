\hypertarget{la-queja-de-david-por-sauxfal-y-jonatuxe1n-ante-la-noticia-de-sus-muertes}{%
\subsection{La queja de David por Saúl y Jonatán ante la noticia de sus
muertes}\label{la-queja-de-david-por-sauxfal-y-jonatuxe1n-ante-la-noticia-de-sus-muertes}}

\hypertarget{section}{%
\section{1}\label{section}}

\bibleverse{1} Después de la muerte de Saúl, David volvió de atacar a
los amalecitas, y se quedó en Siclag durante dos días. \bibleverse{2} Al
tercer día llegó un hombre del campamento de Saúl. Sus ropas estaban
rasgadas y traía polvo sobre la cabeza. Y cuando se acercó a David, se
inclinó ante él y se postró en el suelo en señal de respeto.

\hypertarget{el-informe-del-mensajero-sobre-los-momentos-finales-de-sauxfal}{%
\subsection{El informe del mensajero sobre los momentos finales de
Saúl}\label{el-informe-del-mensajero-sobre-los-momentos-finales-de-sauxfal}}

\bibleverse{3} ``¿De dónde vienes?'' le preguntó David. ``Me alejé del
campamento israelita'', respondió.

\bibleverse{4} ``Cuéntame qué pasó'', le preguntó David. ``El ejército
huyó de la batalla'', respondió el hombre. ``Muchos de ellos murieron, y
también murieron Saúl y su hijo Jonatán''.

\bibleverse{5} ``¿Cómo sabes que murieron Saúl y Jonatán?'' le preguntó
David al hombre que daba el informe.

\bibleverse{6} ``Casualmente estaba allí, en el monte Gilboa'',
respondió. ``Vi a Saúl, apoyado en su lanza, con los carros enemigos y
los auriculares avanzando hacia él. \bibleverse{7} Se volvió y me vio.
Me llamó y le respondí: `Estoy aquí para ayudar'. \bibleverse{8} ``Me
preguntó: `¿Quién eres tú?' ``Le dije: `Soy amalecita'. \bibleverse{9}
``Entonces me dijo: `¡Por favor, ven aquí y mátame! Estoy sufriendo una
terrible agonía, pero la vida aún resiste'. \bibleverse{10} ``Así que me
acerqué a él y lo maté, porque sabía que, herido como estaba, no
aguantaría mucho tiempo. Le quité la corona de la cabeza y el brazalete
del brazo, y te los he traído aquí, mi señor''.

\hypertarget{el-dolor-de-david-matando-al-mensajero}{%
\subsection{El dolor de David; Matando al
mensajero}\label{el-dolor-de-david-matando-al-mensajero}}

\bibleverse{11} Entonces David se agarró su ropa y la rasgó,\footnote{\textbf{1:11}
  Una señal de emoción extrema, generalmente de dolor.} así como lo
habían hecho sus hombres. \footnote{\textbf{1:11} Gén 37,29}
\bibleverse{12} Se lamentaron, lloraron y ayunaron hasta la noche por
Saúl y su hijo Jonatán, y por el ejército del Señor, los israelitas, que
habían muerto a espada. \footnote{\textbf{1:12} 1Sam 31,13}

\bibleverse{13} David preguntó al hombre que le trajo el informe: ``¿De
dónde eres?'' ``Soy hijo de un extranjero'', respondió, ``soy
amalecita''.

\bibleverse{14} ``¿Por qué no te preocupaste por matar al ungido del
Señor?'' preguntó David. \footnote{\textbf{1:14} 1Sam 24,7}
\bibleverse{15} David llamó a uno de sus hombres y le dijo: ``¡Adelante,
mátalo!''. Así que el hombre cortó al amalecita y lo mató. \footnote{\textbf{1:15}
  2Sam 4,10; 2Sam 4,12} \bibleverse{16} David le dijo al amalecita: ``Tu
muerte es culpa tuya, porque has testificado contra ti mismo al decir:
`Yo maté al ungido del Señor'\,''. \footnote{\textbf{1:16} 1Re 2,23; 1Re
  2,33}

\hypertarget{lamentaciuxf3n-de-david-por-sauxfal-y-jonatuxe1n}{%
\subsection{Lamentación de David por Saúl y
Jonatán}\label{lamentaciuxf3n-de-david-por-sauxfal-y-jonatuxe1n}}

\bibleverse{17} Entonces David cantó este lamento por Saúl y su hijo
Jonatán. \bibleverse{18} Ordenó que se enseñara al pueblo de Judá. Se
llama ``el Arco'' y está registrado en el Libro de los Justos:
\bibleverse{19} ``Israel, el glorioso yace muerto en tus montañas. ¡Cómo
han caído los poderosos! \bibleverse{20} No lo anuncies en la ciudad de
Gat, no lo proclames en las calles de Ascalón, para que las mujeres
filisteas no se alegren, para que las mujeres paganas no lo celebren.
\footnote{\textbf{1:20} Miq 1,10; 1Sam 18,6} \bibleverse{21} ¡Montes de
Gilboa, que no caiga rocío ni lluvia sobre ustedes! Que no tengas campos
que produzcan ofrendas de grano. Porque allí fue profanado el escudo de
los poderosos; el escudo de Saúl, ya no se cuida con aceite de
oliva.\footnote{\textbf{1:21} El escudo de Saúl sería ritualmente
  profanado por la sangre, y ya no se cuidaría con aplicaciones
  regulares de aceite de oliva.} \footnote{\textbf{1:21} Núm 15,18-21}

\bibleverse{22} Jonatán con su arco no se retiró de atacar al enemigo;
Saúl con su espada no regresó con las manos vacías de derramar sangre.
\bibleverse{23} Durante su vida, Saúl y Jonatán fueron muy queridos y
agradables, y la muerte no los dividió. Eran más rápidos que las
águilas, más fuertes que los leones. \bibleverse{24} Mujeres de Israel,
lloren por Saúl, que les ha dado ropas finas de color escarlata
adornadas con adornos de oro. \bibleverse{25} ¡Cómo han caído los
poderosos en la batalla! Jonatán yace muerto en vuestros montes.
\bibleverse{26} ¡Lloro tanto por ti, hermano mío Jonatán! ¡Eras tan
querido para mí! Tu amor por mí era tan maravilloso, más grande que el
amor de las mujeres. \bibleverse{27} ¡Cómo han caído los poderosos! ¡Las
armas de la guerra han desaparecido!''

\hypertarget{david-llega-a-ser-rey-sobre-la-tribu-de-juduxe1-isboseth-sobre-israel}{%
\subsection{David llega a ser rey sobre la tribu de Judá, Isboseth sobre
Israel}\label{david-llega-a-ser-rey-sobre-la-tribu-de-juduxe1-isboseth-sobre-israel}}

\hypertarget{section-1}{%
\section{2}\label{section-1}}

\bibleverse{1} Algún tiempo después de esto, David le preguntó al Señor:
``¿Debo ir a una de las ciudades de Judá?'' . ``Sí, hazlo'', respondió
el Señor. ``¿A cuál debo ir?'' preguntó David. ``Ve a Hebrón'', dijo el
Señor. \footnote{\textbf{2:1} 1Sam 30,8}

\bibleverse{2} Así que David se trasladó allí con sus dos esposas,
Ahinoam, de Jezreel, y Abigail, la viuda de Nabal, de Carmel.
\footnote{\textbf{2:2} 1Sam 25,42-43} \bibleverse{3} También trajo a los
hombres que estaban con él, junto con sus familias, y se instalaron en
las aldeas cercanas a Hebrón.

\hypertarget{el-mensaje-de-david-al-pueblo-de-jabes}{%
\subsection{El mensaje de David al pueblo de
Jabes}\label{el-mensaje-de-david-al-pueblo-de-jabes}}

\bibleverse{4} Entonces los hombres de Judá llegaron a Hebrón, y allí
ungieron a David como rey del pueblo de Judá. Cuando David se enteró de
que eran los hombres de Jabes de Galaad los que habían enterrado a Saúl,
\footnote{\textbf{2:4} 2Sam 5,3; 1Sam 16,13; 1Sam 31,12} \bibleverse{5}
les envió mensajeros, diciendo: ``Que el Señor los bendiga, porque
demostraron su amor leal a Saúl, su amo, y lo enterraron debidamente.
\bibleverse{6} Que el Señor les demuestre amor leal y confianza, y yo
también seré bueno con ustedes por lo que hicieron por Saúl.
\bibleverse{7} Así que sé fuerte y valiente, porque aunque Saúl, tu amo,
ha muerto, el pueblo de Judá me ha ungido como su rey''.

\hypertarget{isboseth-hijo-de-sauxfal-se-convierte-en-rey-de-israel}{%
\subsection{Isboseth, hijo de Saúl, se convierte en rey de
Israel}\label{isboseth-hijo-de-sauxfal-se-convierte-en-rey-de-israel}}

\bibleverse{8} Sin embargo, Abner, hijo de Ner, comandante del ejército
de Saúl, había tomado a Isboset,\footnote{\textbf{2:8} Isboset. Es muy
  improbable que se le llamara así en su cara. Se le identifica como
  ``Eshbaal'' en 1 Crónicas 8:33 y 1 Crónicas 9:39, que significa
  ``hombre de Baal''. Sin embargo, al escritor de este relato le pareció
  intolerable que el nombre del rey incluyera una referencia al dios
  pagano ``Baal'', por lo que modificó el nombre a Isboset, que
  significa ``hombre de la vergüenza''.} hijo de Saúl, a Mahanaim.
\bibleverse{9} Allí puso a Isboset como rey sobre Galaad, Aser, Jezreel,
Efraín y Benjamín, de hecho sobre todo Israel. \bibleverse{10} Isboset,
hijo de Saúl, tenía cuarenta años cuando se convirtió en rey de Israel,
y reinó durante dos años. Sin embargo, el pueblo de Judá estaba del lado
de David. \bibleverse{11} David gobernó en Hebrón como rey del pueblo de
Judá durante siete años y seis meses.

\hypertarget{juego-de-lucha-y-batalla-en-gabauxf3n-la-victoria-de-joab}{%
\subsection{Juego de lucha y batalla en Gabaón; La victoria de
Joab}\label{juego-de-lucha-y-batalla-en-gabauxf3n-la-victoria-de-joab}}

\bibleverse{12} Un día, Abner y los hombres de Isboset salieron de
Mahanaim y fueron a la ciudad de Gabaón. \bibleverse{13} Joab, hijo de
Sarvia, y los hombres de David partieron y se encontraron con ellos en
el estanque de Gabaón, donde todos se sentaron, uno frente al otro, al
otro lado del estanque. \bibleverse{14} Abner le dijo a Joab: ``¿Por qué
no hacemos que algunos de los hombres luchen en combate cuerpo a cuerpo
delante de nosotros?'' ``Bien'', aceptó Joab.

\bibleverse{15} Así que se presentaron doce hombres de cada bando: doce
por Benjamín e Isboset, y doce por David. \bibleverse{16} Cada uno
agarró la cabeza de su adversario y le clavó la espada en el costado, de
modo que todos cayeron muertos juntos. Por eso este lugar de Gabaón se
llama el Campo de las Espadas. \bibleverse{17} La batalla que siguió fue
muy reñida, pero finalmente Abner y sus hombres fueron derrotados por
los de David.

\hypertarget{asael-el-hermano-menor-de-joab-muerto-en-persecuciuxf3n-de-abner}{%
\subsection{Asael, el hermano menor de Joab, muerto en persecución de
Abner}\label{asael-el-hermano-menor-de-joab-muerto-en-persecuciuxf3n-de-abner}}

\bibleverse{18} Los tres hijos de Sarvia estaban allí: Joab, Abisai y
Asael. Asael era un corredor rápido, como una gacela que corre por el
campo. \footnote{\textbf{2:18} 1Cró 2,16} \bibleverse{19} Persiguió a
Abner con una determinación absoluta.\footnote{\textbf{2:19}
  ``Determinación absoluta'': literalmente, ``no girar a la derecha ni a
  la izquierda''.}

\bibleverse{20} Abner miró hacia atrás y preguntó: ``¿Eres tú, Asahel?''
``Sí, soy yo'', respondió Asahel.

\bibleverse{21} Abner le dijo: ``¡Déjame en paz! Ve a pelear con otro y
toma sus armas para ti''. Pero Asahel se negó a dejar de perseguirlo.
\bibleverse{22} Abner volvió a advertir a Asahel. ``¡Deja de
perseguirme!'', le gritó. ``¿Por qué quieres que te mate? ¿Cómo podría
enfrentarme a tu hermano Joab?'' \bibleverse{23} Pero Asahel no dejaba
de perseguirlo, así que Abner le clavó el mango\footnote{\textbf{2:23}
  El mango solía estar afilado en punta para poder clavarlo en el suelo.}
de su lanza en el vientre. Salió por la espalda, y cayó muerto allí
mismo. Todos los que pasaban se detenían en el lugar donde Asahel había
caído muerto.

\hypertarget{fin-de-la-persecuciuxf3n-continuaciuxf3n-de-la-guerra}{%
\subsection{Fin de la persecución; Continuación de la
guerra}\label{fin-de-la-persecuciuxf3n-continuaciuxf3n-de-la-guerra}}

\bibleverse{24} Pero Joab y Abisá\footnote{\textbf{2:24} Joab y Abisai
  eran hermanos de Asahel.} se pusieron a perseguir a Abner. Cuando se
puso el sol, llegaron hasta la colina de Amma, cerca de Giah, en el
camino hacia el desierto de Gabaón. \bibleverse{25} Los hombres de
Abner, de la tribu de Benjamín, se unieron a él y formaron un grupo
compacto a su alrededor, de pie en la cima de la colina. \bibleverse{26}
Abner le gritó a Joab ``¿Tenemos que seguir matándonos para siempre? ¿No
te das cuenta de que si seguimos así sólo será peor? ¿Cuánto tiempo vas
a esperar antes de ordenar a tus hombres que dejen de perseguir a sus
hermanos?''

\bibleverse{27} ``Vive Dios'', respondió Joab, ``si no hubieras dicho
nada, mis hombres habrían seguido persiguiendo a sus hermanos hasta la
mañana''. \bibleverse{28} Joab tocó el cuerno y todos los hombres se
detuvieron; no siguieron persiguiendo ni luchando contra los israelitas.
\bibleverse{29} Durante toda la noche Abner y sus hombres marcharon por
el valle del Jordán. Cruzaron el río Jordán y continuaron toda la mañana
hasta llegar de nuevo a Mahanaim.

\bibleverse{30} Cuando Joab regresó de perseguir a Abner, reunió a todos
los hombres. Faltaban diecinueve de los hombres de David, además de
Asahel. \bibleverse{31} Sin embargo, habían matado a trescientos sesenta
hombres de Abner de la tribu de Benjamín. \bibleverse{32} Tomaron el
cuerpo de Asael y lo enterraron en la tumba de su padre en Belén. Luego
marcharon durante toda la noche y llegaron a Hebrón al amanecer.

\hypertarget{section-2}{%
\section{3}\label{section-2}}

\bibleverse{1} Hubo una larga guerra entre los del bando de Saúl y los
del bando de David. El bando de David se fortalecía, mientras que el de
Saúl se debilitaba. \footnote{\textbf{3:1} 2Sam 5,10}

\hypertarget{la-familia-de-david-en-hebruxf3n}{%
\subsection{La familia de David en
Hebrón}\label{la-familia-de-david-en-hebruxf3n}}

\bibleverse{2} Los hijos de David nacidos en Hebrón fueron: Amnón
(primogénito), cuya madre fue Ahinoam de Jezreel; \footnote{\textbf{3:2}
  1Cró 3,1-4; 2Sam 13,1} \bibleverse{3} Queliab (segundo), cuya madre
fue Abigail, viuda de Nabal, de Carmel; Absalón (tercero), cuya madre
fue Maaca, hija del rey Talmai, de Gesur; \bibleverse{4} Adonías
(cuarto), cuya madre fue Haguit; Sefatías (quinto), cuya madre fue
Abital; \footnote{\textbf{3:4} 1Re 1,5} \bibleverse{5} Itream (sexto),
cuya madre fue Egla, esposa de David. Esos fueron los hijos que le
nacieron a David en Hebrón.

\hypertarget{abner-se-estuxe1-peleando-con-isboseth}{%
\subsection{Abner se está peleando con
Isboseth}\label{abner-se-estuxe1-peleando-con-isboseth}}

\bibleverse{6} Abner había estado fortaleciendo su posición entre los
partidarios de la dinastía de Saúl durante la guerra entre los del bando
de Saúl y los del bando de David. \bibleverse{7} Saúl tenía una
concubina llamada Rizpa, hija de Aia. Un día Isboset acusó a Abner,
diciendo: ``¿Por qué te has acostado con la concubina de mi padre?'' .

\bibleverse{8} Abner se enfadó mucho ante la acusación de Isboset.
``¿Acaso soy un cabeza de perro que se pone del lado de Judá?'' ,
respondió. ``Hasta el día de hoy he sido leal a tu dinastía, a tu padre
Saúl y a sus hermanos y amigos. No te he traicionado con David. ¡Pero
ahora te atreves a acusarme de pecar con esta mujer! \bibleverse{9} ¡Que
Dios me castigue severamente si no ayudo a David a cumplir lo que el
Señor le ha prometido! \bibleverse{10} Entregaré el reino de la dinastía
de Saúl y ayudaré a establecer el gobierno de David sobre Israel y Judá,
desde Dan hasta Beerseba''.

\bibleverse{11} Isboset no se atrevió a decirle nada más a Abner porque
le tenía miedo.

\hypertarget{las-negociaciones-de-abner-con-david-y-los-jefes-de-israel}{%
\subsection{Las negociaciones de Abner con David y los jefes de
Israel}\label{las-negociaciones-de-abner-con-david-y-los-jefes-de-israel}}

\bibleverse{12} Entonces Abner envió mensajeros para que hablaran en su
nombre con David, diciéndole: ``Después de todo ¿a quién pertenece el
país? Haz un acuerdo conmigo, y puedes estar seguro de que estaré de tu
lado para que todo Israel te siga''.

\bibleverse{13} ``Bien'', respondió David, ``haré un acuerdo contigo.
Pero tengo una condición: No te veré a menos que traigas a la hija de
Saúl, Mical, cuando vengas''. \bibleverse{14} Entonces David envió
mensajeros para decirle a Isboset, hijo de Saúl: ``Devuélveme a mi mujer
Mical; pagué por ella una dote de cien prepucios filisteos''.
\footnote{\textbf{3:14} 1Sam 18,25-27}

\bibleverse{15} Isboset envió a buscarla y se la quitó a su marido
Paltiel, hijo de Laish. \bibleverse{16} Su marido la siguió hasta la
ciudad de Bahurim, llorando mientras iba. Entonces Abner le ordenó:
``¡Vuelve a casa!''. Así que se fue a su casa.

\bibleverse{17} Abner habló con los ancianos de Israel y les dijo:
``Hace tiempo que quieren tener a David como rey. \bibleverse{18} Ahora
es el momento de hacerlo, porque el Señor le prometió a David: `Por
medio de mi siervo David voy a salvar a mi pueblo Israel de los
filisteos y de todos sus enemigos'\,''.

\bibleverse{19} Abner también habló con la gente de Benjamín y fue a
Hebrón para comunicarle a David todo lo que los israelitas y toda la
tribu de Benjamín habían decidido hacer.

\hypertarget{el-encuentro-de-abner-con-david-en-hebruxf3n-su-asesinato-por-joab}{%
\subsection{El encuentro de Abner con David en Hebrón; su asesinato por
Joab}\label{el-encuentro-de-abner-con-david-en-hebruxf3n-su-asesinato-por-joab}}

\bibleverse{20} Abner fue con veinte de sus hombres a ver a David a
Hebrón, y David les preparó un banquete. \bibleverse{21} Entonces Abner
le dijo a David: ``Déjame ir inmediatamente y convocar a todo Israel
para una reunión con mi señor el rey, para que se pongan de acuerdo
contigo y puedas gobernar todo lo que quieras''. Entonces David envió a
Abner sano y salvo.

\bibleverse{22} Poco después, Joab y los hombres de David regresaron de
una incursión, trayendo consigo una gran cantidad de botín. Sin embargo,
Abner no estaba con David en Hebrón porque éste ya lo había enviado sano
y salvo en paz. \bibleverse{23} Cuando Joab y todo el ejército que lo
acompañaba llegaron, le dijeron: ``Abner, hijo de Ner, vino a ver al
rey, quien lo envió sano y salvo''.

\bibleverse{24} Joab fue a ver al rey y le preguntó: ``¿Qué crees que
estás haciendo? Aquí está Abner, que ha venido a verte. ¿Por qué motivo
lo enviaste por el camino? ¡Ahora se ha escapado limpiamente!
\bibleverse{25} ¿Te da cuenta de que Abner, hijo de Ner, vino a
engañarte, a espiar los movimientos de tu ejército y a averiguar todo lo
que haces?''

\bibleverse{26} Cuando Joab salió de la presencia de David, envió
mensajeros tras Abner. Lo encontraron en el pozo de Sira y lo trajeron
de vuelta, pero David no sabía nada al respecto. \bibleverse{27} Cuando
Abner regresó a Hebrón, Joab lo llevó aparte a la puerta de la ciudad,
como si fuera a hablar con él en privado. Pero Joab lo apuñaló en el
vientre, matándolo en venganza por haber matado a Asael, el hermano de
Joab.

\hypertarget{abner-hizo-duelo-por-david-y-fue-sepultado-con-honor-la-declaraciuxf3n-de-inocencia-de-david-graduaciuxf3n}{%
\subsection{Abner hizo duelo por David y fue sepultado con honor; La
declaración de inocencia de David;
Graduación}\label{abner-hizo-duelo-por-david-y-fue-sepultado-con-honor-la-declaraciuxf3n-de-inocencia-de-david-graduaciuxf3n}}

\bibleverse{28} Y cuando David se enteró de esto, dijo: ``¡Yo y mi reino
somos totalmente inocentes ante el Señor en lo que respecta a la muerte
de Abner, hijo de Ner! \bibleverse{29} Que la culpa de su muerte caiga
sobre Joab y su familia. Que los descendientes de Joab siempre tengan a
alguien que tenga llagas, o lepra, o esté lisiado\footnote{\textbf{3:29}
  ``Lisiado'': siguiendo la lectura de la Septuaginta que sugiere que
  una persona así siempre tiene que apoyarse en una muleta.} o que lo
maten a espada, o que se muera de hambre''. \bibleverse{30} (Por eso
Joab y su hermano Abisai mataron a Abner, porque éste había matado a su
hermano Asael durante la batalla de Gabaón).

\bibleverse{31} Entonces David les ordenó a Joab y a todos los que
estaban allí: ``Rasguen sus ropas, pónganse silicio y hagan duelo por
Abner''. El mismo rey David siguió el cuerpo mientras lo llevaban a la
tumba. \bibleverse{32} Enterraron a Abner en Hebrón, y el rey lloró a
gritos ante la tumba, junto con todo el pueblo. \footnote{\textbf{3:32}
  1Sam 30,4} \bibleverse{33} El rey cantó este lamento por Abner:
``¿Merecía Abner morir como un criminal? \bibleverse{34} Sus manos no
estaban atadas, sus pies no tenían grilletes. Pero al igual que la
víctima de un asesino, tú también fuiste asesinado''. Todo el pueblo
lloró aún más por él.

\bibleverse{35} Entonces la gente se acercó a David y trataron de
persuadirlo para que comiera algo durante el día. Pero David hizo un
juramento, diciendo: ``¡Que Dios me castigue severamente si como pan o
cualquier otra cosa antes de la puesta del sol!''

\bibleverse{36} Todos vieron esto y pensaron que era lo correcto, de la
misma manera que pensaban que todo lo que hacía el rey era lo correcto.
\bibleverse{37} Ese día todos en Judá y en todo Israel se dieron cuenta
de que David no había ordenado el asesinato de Abner. \bibleverse{38}
Entonces el rey dijo a sus oficiales: ``¿No reconocen que hoy ha caído
en Israel un comandante y un hombre verdaderamente grande?
\bibleverse{39} En este momento soy débil, a pesar de haber sido ungido
como rey, y estos hombres, los hijos de Sarvia, son demasiado poderosos
para mí. Pero que el Señor pague al hombre malo según el mal que haya
hecho''.

\hypertarget{asesinato-de-isboseth-coronaciuxf3n-de-david-como-rey-de-todo-israel}{%
\subsection{Asesinato de Isboseth; Coronación de David como Rey de todo
Israel}\label{asesinato-de-isboseth-coronaciuxf3n-de-david-como-rey-de-todo-israel}}

\hypertarget{section-3}{%
\section{4}\label{section-3}}

\bibleverse{1} Cuando Isboset,\footnote{\textbf{4:1} Tanto aquí como en
  el versículo 2, se hace referencia a Isboset simplemente como ``hijo
  de Saúl''. No se da su nombre.} hijo de Saúl, supo que Abner había
muerto en Hebrón, sintió gran desánimo,\footnote{\textbf{4:1} ``Sintió
  gran desánimo'': literalmente, ``sus manos colgaban sin fuerza''.} y
todos en Israel estaban consternados. \bibleverse{2} Isboset tenía dos
comandantes de sus bandas de asalto. Eran hermanos y sus nombres eran
Baná y Recab. Eran hijos de Rimón, de la tribu de Benjamín, de la ciudad
de Berot. Berot se considera parte del territorio de Benjamín,
\bibleverse{3} porque el pueblo que había vivido en Berot antes huyó a
Guitayin y han vivido allí como extranjeros hasta el presente.

\bibleverse{4} Jonatán, hijo de Saúl, tenía un hijo que era cojo de
ambos pies. Cuando el niño tenía cinco años, llegó de Jezreel la noticia
de la muerte de Saúl y Jonatán. Su nodriza lo había recogido y había
salido corriendo con él para huir. Pero mientras corría, el niño se cayó
y quedó cojo. Su nombre era Mefi-boset.\footnote{\textbf{4:4}
  Mefi-boset. Su nombre se da como ``Meribaal'' en 1 Crónicas 8:34 y 1
  Crónicas 9:40. El nombre aquí refleja la reticencia del escritor a
  utilizar el nombre de un dios pagano en el nombre de uno de los reyes
  de Israel. Véase la nota a pie de página de 2:8.} \footnote{\textbf{4:4}
  2Sam 9,3}

\bibleverse{5} Recab y Baná, hijos de Rimón de Berot, se dirigieron a la
casa de Isboset, llegando con el calor del día, cuando el rey estaba
tomando su descanso de mediodía. \bibleverse{6} La portera había estado
limpiando el trigo, pero se había cansado y se había quedado dormida,
así que Recab y Baná pudieron entrar sin que se dieran
cuenta.\footnote{\textbf{4:6} Este versículo plantea una serie de
  problemas. En este caso, la traducción sigue a la Septuaginta. El
  texto hebreo dice: ``Entraron en la casa como para coger trigo, y le
  apuñalaron en el vientre. Entonces Recab y su hermano Baná se
  escabulleron''. Lamentablemente este verso no existe en ninguno de los
  Rollos del Mar Muerto.}

\hypertarget{david-castiga-a-los-asesinos-y-honra-al-muerto-isboseth}{%
\subsection{David castiga a los asesinos y honra al muerto
Isboseth}\label{david-castiga-a-los-asesinos-y-honra-al-muerto-isboseth}}

\bibleverse{7} Entraron en la casa mientras Isboset dormía en su
habitación. Después de apuñalarlo y matarlo, le cortaron la cabeza, la
cual se llevaron, y viajaron toda la noche por el camino del valle del
Jordán. \bibleverse{8} Luego le llevaron la cabeza de Isboset a David en
Hebrón. Le dijeron al rey: ``Aquí está la cabeza de Isboset, hijo de
Saúl, tu enemigo que intentó matarte. Hoy el Señor se ha vengado de Saúl
y su familia por mi señor el rey''.

\bibleverse{9} Pero David respondió a Recab y a su hermano Baná, hijos
de Rimón de Berot: ``Vive el Señor, que me ha salvado de todas mis
angustias, \bibleverse{10} cuando alguien me dijo: `Mira, Saúl ha
muerto' y creyó que me traía buenas noticias, lo agarré y lo hice matar
en Siclag. ¡Eso fue lo que recibió por traerme sus noticias!
\bibleverse{11} ¡Con más razón, cuando hombres malos matan a un hombre
bueno en su propia casa y en su propia cama, ¿no debería exigirles que
paguen por su vida con sus propias vidas, y exterminarlos?!''
\bibleverse{12} Entonces David dio la orden a sus hombres, y mataron a
Recab y a Baná. Les cortaron las manos y los pies, y colgaron sus
cuerpos junto al estanque de Hebrón. Luego tomaron la cabeza de Isboset
y la enterraron en la tumba de Abner en Hebrón.

\hypertarget{david-ungido-rey-por-todos-los-israelitas-en-hebruxf3n}{%
\subsection{David ungido rey por todos los israelitas en
Hebrón}\label{david-ungido-rey-por-todos-los-israelitas-en-hebruxf3n}}

\hypertarget{section-4}{%
\section{5}\label{section-4}}

\bibleverse{1} Todas las tribus de Israel se acercaron a David en Hebrón
y le dijeron: ``Somos tu carne y tu sangre. \footnote{\textbf{5:1} 2Sam
  19,13} \bibleverse{2} Antes, cuando Saúl era nuestro rey, tú eras el
que dirigía el ejército israelita en la batalla. El Señor te dijo: `Tú
serás el pastor de mi pueblo Israel y serás su gobernante'\,''.
\footnote{\textbf{5:2} 1Sam 13,14; 1Sam 25,30} \bibleverse{3} Todos los
ancianos de Israel acudieron al rey en Hebrón, donde el rey David llegó
a un acuerdo con ellos en presencia del Señor. Entonces lo ungieron como
rey de Israel. \footnote{\textbf{5:3} 2Sam 2,4; 1Sam 16,13}

\bibleverse{4} David tenía treinta años cuando llegó a ser rey, y reinó
durante cuarenta años. \footnote{\textbf{5:4} 1Re 2,11; 1Cró 29,27}
\bibleverse{5} Reinó sobre Judá siete años y seis meses desde Hebrón, y
reinó sobre todo Israel y Judá durante treinta y tres años desde
Jerusalén.

\hypertarget{david-conquista-jerusaluxe9n-y-la-convierte-en-su-capital-y-su-residencia}{%
\subsection{David conquista Jerusalén y la convierte en su capital y su
residencia}\label{david-conquista-jerusaluxe9n-y-la-convierte-en-su-capital-y-su-residencia}}

\bibleverse{6} El rey David y sus hombres fueron a Jerusalén para atacar
a los jebuseos que vivían allí. Los jebuseos le dijeron a David: ``Nunca
entrarás aquí. Hasta los ciegos y los cojos podrían impedírtelo''.
Estaban convencidos de que David no podría entrar. \bibleverse{7} Pero
David sí capturó la fortaleza de Sión, ahora conocida como la Ciudad de
David. \bibleverse{8} En ese momento dijo: ``Si queremos conquistar a
los jebuseos, tendremos que subir por el pozo de agua para atacar a esos
`cojos y ciegos', a esa gente que odia a David. Por eso se dice: `Los
ciegos y los cojos nunca entrarán en la casa'\,''.\footnote{\textbf{5:8}
  Existe un debate sobre el significado de ``casa'' aquí. Podría
  referirse a las casas ordinarias, o a la casa del rey (palacio). Sin
  embargo, la Septuaginta dice ``casa del Señor'', lo que probablemente
  se refiera a lo estipulado en Levítico 21:17-23.}

\bibleverse{9} David se fue a vivir a la fortaleza y la llamó Ciudad de
David. La extendió en todas las direcciones, empezando por las terrazas
de apoyo exteriores y avanzando hacia el interior. \bibleverse{10} David
se volvía cada vez más poderoso, porque el Señor Dios Todopoderoso
estaba con él. \footnote{\textbf{5:10} 2Sam 3,1}

\hypertarget{sus-construcciones-con-la-ayuda-de-hiram-de-tiro-aumentando-el-nuxfamero-de-sus-esposas-sus-hijos-nacidos-en-jerusaluxe9n}{%
\subsection{Sus construcciones (con la ayuda de Hiram de Tiro);
Aumentando el número de sus esposas; sus hijos nacidos en
Jerusalén}\label{sus-construcciones-con-la-ayuda-de-hiram-de-tiro-aumentando-el-nuxfamero-de-sus-esposas-sus-hijos-nacidos-en-jerusaluxe9n}}

\bibleverse{11} Tiempo después, el rey Hiram de Tiro envió
representantes a David, junto con madera de cedro, carpinteros y
canteros, construyeron un palacio para David. \bibleverse{12} David se
dio cuenta de que el Señor lo había instalado como rey de Israel y había
engrandecido su reino por el bien de su pueblo Israel.

\bibleverse{13} Después de mudarse de Hebrón, David tomó más concubinas
y esposas de Jerusalén, y tuvo más hijos e hijas. \bibleverse{14} Estos
son los nombres de sus hijos nacidos en Jerusalén Samúa, Sobab, Natán,
Salomón, \bibleverse{15} Ibhar, Elisúa, Nefeg, Jafía, \bibleverse{16}
Elisama, Eliada y Elifelet.

\hypertarget{sus-dos-batallas-victoriosas-con-los-filisteos}{%
\subsection{Sus dos batallas victoriosas con los
filisteos}\label{sus-dos-batallas-victoriosas-con-los-filisteos}}

\bibleverse{17} Cuando los filisteos se enteraron de que David había
sido ungido rey de Israel, todo el ejército filisteo salió a capturarlo,
pero David se enteró y entró en la fortaleza. \bibleverse{18} Entonces
los filisteos llegaron y se extendieron por el valle de Refaim.
\bibleverse{19} Y David preguntó al Señor: ``¿Debo ir a atacar a los
filisteos? ¿Me los entregarás?'' ``Sí, ve'', respondió el Señor,
``porque sin duda alguna te los entregaré''. \footnote{\textbf{5:19}
  1Sam 30,8}

\bibleverse{20} David fue a Baal-perazim y allí derrotó a los filisteos.
``Como un torrente que se desborda, así ha estallado el Señor contra mis
enemigos delante de mí'', declaró David. Y llamó a ese lugar
Baal-perazim. \bibleverse{21} Los filisteos dejaron sus ídolos, y David
y sus hombres los quitaron.

\bibleverse{22} Un tiempo después, los filisteos volvieron a llegar y se
extendieron por el valle de Refaim. \bibleverse{23} David le preguntó al
Señor qué hacer. El Señor le respondió: ``No los ataques directamente,
sino que rodea por detrás de ellos y atácalos frente a los árboles de
bálsamo. \bibleverse{24} En cuanto oigas el ruido de la marcha en las
copas de los bálsamos prepárate, porque eso significa que el Señor ha
salido delante de ti para atacar el campamento filisteo''.

\bibleverse{25} David cumplió las órdenes del Señor, y mató a los
filisteos desde Geba hasta Gezer.

\hypertarget{traslado-del-arca-a-sion-en-jerusaluxe9n-fracaso-del-primer-intento}{%
\subsection{Traslado del arca a Sion en Jerusalén; Fracaso del primer
intento}\label{traslado-del-arca-a-sion-en-jerusaluxe9n-fracaso-del-primer-intento}}

\hypertarget{section-5}{%
\section{6}\label{section-5}}

\bibleverse{1} Una vez más, David convocó a todos los hombres
especialmente elegidos de Israel, y eran treinta mil en total.
\bibleverse{2} Entonces fue con todos sus hombres a Baalá, en Judá, para
traer de vuelta el Arca de Dios, que pertenece al Señor Todopoderoso,
que está sentado entre los querubines que están sobre el Arca.
\footnote{\textbf{6:2} Jos 15,9; Éxod 25,22} \bibleverse{3} Colocaron el
Arca de Dios en un carro nuevo y la trajeron desde la casa de Abinadab,
que estaba en una colina. Uza y Ahío, hijos de Abinadab, dirigían el
carruaje \footnote{\textbf{6:3} 1Sam 7,1} \bibleverse{4} que
transportaba el Arca de Dios, y Ahío caminaba delante de él.\footnote{\textbf{6:4}
  Lectura de la Septuaginta, apoyada por uno de los Rollos del Mar
  Muerto.} \bibleverse{5} David y todos los israelitas estaban
celebrando en presencia del Señor, cantando canciones acompañadas de
cítaras, arpas, panderetas, sonajas y címbalos.\footnote{\textbf{6:5} La
  referencia en el hebreo a los abetos es improbable en este caso, y el
  pasaje paralelo de 1 Crónicas 13:8 ayuda a clarificar.}

\bibleverse{6} Pero cuando llegaron a la era de Nachón, los bueyes
tropezaron, por lo que Uza extendió la mano para evitar que el Arca de
Dios cayera. \bibleverse{7} El Señor se enojó con Uza, y Dios lo hirió
allí mismo por su desobediencia,\footnote{\textbf{6:7} El significado de
  la palabra utilizada aquí es incierto. Puede indicar una acción
  precipitada o irreverente. Aquí parece reflejar una actitud
  presuntuosa que trataba el Arca como un simple objeto ordinario.} y
murió junto al Arca de Dios. \footnote{\textbf{6:7} Núm 4,15; 1Sam 6,19}
\bibleverse{8} David se enfadó por el violento arrebato del Señor contra
Uza. Llamó al lugar Perez-uza,\footnote{\textbf{6:8} Significa
  ``arrebato contra Uza''.} que sigue siendo su nombre hasta hoy.
\bibleverse{9} Ese día, David tuvo miedo del Señor. ``¿Cómo podré traer
el Arca de Dios a mi casa?'' , se preguntó. \bibleverse{10} Al no querer
traer el Arca del Señor para que estuviera con él en la Ciudad de David,
la hizo llevar a la casa de Obed-edom el gitano. \bibleverse{11}
Entonces el Arca del Señor permaneció en la casa de Obed-edom durante
tres meses, y el Señor bendijo a toda la familia de Obed-edom.

\hypertarget{traslado-solemne-del-arca-a-jerusaluxe9n-fiesta-del-sacrificio-y-acciuxf3n-de-gracias-del-pueblo}{%
\subsection{Traslado solemne del arca a Jerusalén; Fiesta del sacrificio
y acción de gracias del
pueblo}\label{traslado-solemne-del-arca-a-jerusaluxe9n-fiesta-del-sacrificio-y-acciuxf3n-de-gracias-del-pueblo}}

\bibleverse{12} Y le dijeron al rey David: ``El Señor ha bendecido la
casa de Obed-edom y todo lo que tiene a causa del Arca de Dios''. Así
que David fue y mandó traer el Arca de Dios de la casa de Obed-edom a la
Ciudad de David. Entonces hubo gran celebración.

\bibleverse{13} Cuando los hombres que llevaban el Arca del Señor dieron
seis pasos, David sacrificó un toro y un ternero cebado. \footnote{\textbf{6:13}
  1Re 8,5} \bibleverse{14} Luego, con un efod sacerdotal, David bailó
con todas sus fuerzas ante el Señor \bibleverse{15} mientras él y todos
los israelitas llevaban el Arca del Señor, con muchos gritos y sonido de
cuernos.

\bibleverse{16} Mientras el Arca del Señor era transportada a la ciudad
de David, la hija de Saúl, Mical, miraba desde una ventana. Vio al rey
David saltando y danzando ante el Señor, y realmente lo aborreció.
\bibleverse{17} Entonces trajeron el Arca del Señor y la colocaron en su
lugar dentro de la tienda que David había montado para ella. Luego David
ofreció holocaustos y ofrendas de paz ante el Señor. \bibleverse{18}
Cuando terminó de ofrecer los sacrificios, David bendijo al pueblo en
nombre del Señor Todopoderoso. \footnote{\textbf{6:18} 1Re 8,55}
\bibleverse{19} Entonces les dio a todos los israelitas, tanto a hombres
como a mujeres, una hogaza de pan, una torta de dátiles y una torta de
pasas. Luego los envió a todos a sus casas.

\hypertarget{la-noble-conducta-de-david-y-su-humilde-declaraciuxf3n-contra-mical}{%
\subsection{La noble conducta de David y su humilde declaración contra
Mical}\label{la-noble-conducta-de-david-y-su-humilde-declaraciuxf3n-contra-mical}}

\bibleverse{20} Cuando David llegó a su casa para bendecir a su familia,
Mical, la hija de Saúl, salió a su encuentro y le dijo: ``¡Qué
distinguido se ha puesto hoy el rey de Israel, quitándose la túnica para
que lo vieran todas las sirvientas, como se expondría cualquier personal
vulgar!''

\bibleverse{21} David le dijo a Mical: ``Estuve bailando ante el Señor,
que me eligió a mí en lugar de a tu padre y a toda su familia cuando me
nombró gobernante del pueblo del Señor, Israel. Seguiré celebrando ante
el Señor, \bibleverse{22} de hecho voy a hacerme aún menos distinguido,
me volveré aún más humilde. Sin embargo, seré respetado por esas siervas
de las que hablaste''.

\bibleverse{23} Y Mical, la hija de Saúl, nunca tuvo hijos.

\hypertarget{natuxe1n-aprueba-el-plan-de-david-para-construir-el-templo}{%
\subsection{Natán aprueba el plan de David para construir el
templo}\label{natuxe1n-aprueba-el-plan-de-david-para-construir-el-templo}}

\hypertarget{section-6}{%
\section{7}\label{section-6}}

\bibleverse{1} A estas alturas el rey estaba cómodo en su palacio y el
Señor le había dado la paz de todas las naciones enemigas que lo
rodeaban. \bibleverse{2} Así que le dijo al profeta Natán: ``Mírame:
vivo en un palacio hecho de cedro, pero el Arca de Dios sigue en una
tienda de campaña''. \footnote{\textbf{7:2} Sal 132,-1}

\bibleverse{3} ``Adelante, haz lo que quieras, porque el Señor está
contigo'', le dijo Natán al rey.

\hypertarget{rechazo-de-dios-del-plan-el-discurso-profuxe9tico-de-nathan-el-templo-seruxe1-construido-por-el-hijo-de-david}{%
\subsection{Rechazo de Dios del plan; El discurso profético de Nathan;
el templo será construido por el hijo de
David}\label{rechazo-de-dios-del-plan-el-discurso-profuxe9tico-de-nathan-el-templo-seruxe1-construido-por-el-hijo-de-david}}

\bibleverse{4} Pero esa noche el Señor habló a Natán y le dijo:
\bibleverse{5} ``Ve y dile a mi siervo David: Esto es lo que dice el
Señor: ¿Debes ser tú quien construya una casa para que yo viva en ella?
\bibleverse{6} Porque nunca he vivido en una casa, desde que saqué a los
israelitas de Egipto hasta ahora. Siempre me he trasladado de un lugar a
otro, viviendo en una tienda y en un tabernáculo. \footnote{\textbf{7:6}
  1Re 8,16; 1Re 8,27; Is 66,1} \bibleverse{7} Pero en todos esos viajes
con todo Israel, ¿le reclamé alguna vez a algún líder israelita al que
puse a cargo de mi pueblo: `Por qué no has construido una casa de cedro
para mí'? \bibleverse{8} ``Entonces, dile a mi siervo David que esto es
lo que dice el Señor Todopoderoso: Fui yo quien te sacó del campo, de
cuidar ovejas, para convertirte en jefe de mi pueblo Israel.
\bibleverse{9} He estado contigo dondequiera que has ido. He destruido a
todos tus enemigos delante de ti, y haré que tu reputación sea tan
grande como la de las personas más famosas de la tierra. \bibleverse{10}
Elegiré un lugar para mi pueblo Israel. Allí los asentaré y ya no serán
molestados. Los malvados no los perseguirán como antes, \bibleverse{11}
desde que puse jueces a cargo de mi pueblo. Derrotaré a todos sus
enemigos. ``También quiero dejar claro que yo, el Señor, les construiré
una casa.\footnote{\textbf{7:11} En otras palabras, el Señor construiría
  una ``casa'' para David en el sentido de establecer una dinastía real.}
\bibleverse{12} Porque cuando llegues al final de tu vida y te unas a
tus antepasados en la muerte, llevaré al poder a uno de tus
descendientes, a uno de tus hijos, y me aseguraré de que su reino tenga
éxito. \footnote{\textbf{7:12} 1Re 8,20; Is 9,6} \bibleverse{13} Él será
quien me construya una casa, y me aseguraré de que su reino dure para
siempre. \footnote{\textbf{7:13} 1Re 5,19; 1Re 6,12; Sal 89,4-5}

\hypertarget{la-gran-proclamaciuxf3n-de-salvaciuxf3n-de-dios-a-david-con-respecto-a-la-eternidad-de-su-casa}{%
\subsection{La gran proclamación de salvación de Dios a David con
respecto a la eternidad de su
casa}\label{la-gran-proclamaciuxf3n-de-salvaciuxf3n-de-dios-a-david-con-respecto-a-la-eternidad-de-su-casa}}

\bibleverse{14} Yo seré un padre para él, y él será un hijo para mí. Si
hace el mal, lo disciplinaré con la vara como se hace con la gente, como
un padre que castiga a un hijo. \footnote{\textbf{7:14} Sal 89,27; Heb
  1,5; Luc 1,32} \bibleverse{15} Pero nunca le quitaré mi bondad y mi
amor, como hice en el caso de Saúl, a quien quité de en medio.
\footnote{\textbf{7:15} 1Sam 15,23; 1Sam 15,26} \bibleverse{16} Tu casa
y tu reino serán eternos; tu dinastía estará segura para siempre''.
\footnote{\textbf{7:16} Sal 72,17; Is 55,3}

\hypertarget{acciuxf3n-de-gracias-y-suxfaplica-de-david}{%
\subsection{Acción de gracias y súplica de
David}\label{acciuxf3n-de-gracias-y-suxfaplica-de-david}}

\bibleverse{17} Así que esto fue lo que Natán le explicó a David, y fue
todo lo que se le dijo en esta revelación divina.

\bibleverse{18} Entonces el rey David fue y se sentó en presencia del
Señor. Oró: ``¿Quién soy yo, Señor Dios? ¿Qué importancia tiene mi
familia para que me hayas traído hasta este lugar? \bibleverse{19} Dios,
hablas como si esto fuera una cosa pequeña para ti, y también has
hablado del futuro de mi casa, de la dinastía de mi familia.\footnote{\textbf{7:19}
  ``La dinastía demi familia'': explicando el significado de ``casa'' en
  este contexto.} ¿Es esta tu forma habitual de tratar con los seres
humanos, Señor Dios? \bibleverse{20} ``¿Qué más puedo decirte? Tú sabes
perfectamente cómo es tu siervo David, Señor Dios. \bibleverse{21} Todo
esto lo haces por mí y me lo has explicado, a mí, tu siervo, por tu
promesa y porque así lo deseas tú. \bibleverse{22} ``¡Qué grande eres,
Señor Dios! Realmente no hay nadie como tú; no hay otro Dios, sólo tú.
Nunca hemos oído hablar de otro. \bibleverse{23} ¿Quién más es tan
afortunado como tu pueblo Israel? ¿A quién más en la tierra fue Dios a
redimir para hacer su propio pueblo? Te ganaste una reputación
maravillosa por todas las cosas grandes y asombrosas que hiciste al
expulsar a otras naciones y a sus dioses delante de tu pueblo cuando lo
redimiste de Egipto. \footnote{\textbf{7:23} Deut 4,7} \bibleverse{24}
Hiciste tuyo a tu pueblo Israel para siempre, y tú, Señor, te has
convertido en su Dios.

\bibleverse{25} ``Así que ahora, Señor Dios, haz que lo que has dicho de
mí y de mi casa se cumpla y se confirme para siempre. Por favor, haz lo
que has prometido, \bibleverse{26} y que tu verdadera naturaleza sea
honrada para siempre, y que el pueblo declare: `¡El Señor Todopoderoso
es el Dios de Israel!' Que la casa de tu siervo David siga estando en tu
presencia. \bibleverse{27} Señor Todopoderoso, Dios de Israel, tú me lo
has revelado a mí, tu siervo, diciéndome: `Voy a construir una casa para
ti'. Por eso tu siervo ha tenido el valor de hacerte esta oración.

\bibleverse{28} ¡Señor Todopoderoso, tú eres Dios! Tus palabras son
verdaderas, y tú eres quien ha prometido estas cosas buenas a tu siervo.
\footnote{\textbf{7:28} 1Re 8,26}

\bibleverse{29} Así que ahora, por favor, bendice la casa de tu siervo
para que continúe en tu presencia para siempre. Porque tú has hablado,
Señor Dios, y con tu bendición la casa de tu siervo será bendecida para
siempre''.

\hypertarget{las-victorias-de-david-sobre-los-filisteos-moabitas-y-sirios}{%
\subsection{Las victorias de David sobre los filisteos, moabitas y
sirios}\label{las-victorias-de-david-sobre-los-filisteos-moabitas-y-sirios}}

\hypertarget{section-7}{%
\section{8}\label{section-7}}

\bibleverse{1} Poco después de esto, David atacó y subyugó a los
filisteos, quitándoles Metheg-ammah.\footnote{\textbf{8:1} Se desconoce
  el significado de este término. Puede ser un nombre de lugar. El
  pasaje paralelo en Crónicas identifica ``Gat y sus ciudades
  cercanas''. 1 Crónicas 18:1.} \bibleverse{2} David también derrotó a
los moabitas. Los hizo tumbarse en el suelo y los midió con un tramo de
cuerda. Midió dos tramos para los que debían morir, y un tramo de cuerda
para los que debían vivir. Entonces los sometió bajo su gobierno, y les
exigió que pagaran impuestos.

\bibleverse{3} David también derrotó a Hadad-ezer, hijo de Rehob, rey de
Soba, cuando intentaba imponer su control a lo largo del río Éufrates.
\bibleverse{4} David capturó 1. 000 de sus carros, 7. 000 jinetes y 20.
000 soldados de a pie. Ató por las patas a todos los caballos que
llevaban carros, pero dejó suficientes caballos para 100 carros de
guerra. \bibleverse{5} Cuando los arameos de Damasco vinieron a ayudar
al rey Hadad-ezer de Soba, David mató a veintidós mil de ellos.
\bibleverse{6} Colocó guarniciones en el reino arameo con capital en
Damasco, e hizo que los arameos se sometieran a él y les exigió el pago
de impuestos. El Señor le daba victorias a David por donde quiera que
iba.

\hypertarget{el-botuxedn-y-sus-usos-felicitaciones-del-rey-thoi}{%
\subsection{El botín y sus usos; Felicitaciones del Rey
Thoi}\label{el-botuxedn-y-sus-usos-felicitaciones-del-rey-thoi}}

\bibleverse{7} David tomó los escudos de oro que le pertenecían a los
oficiales de Hadad-ezer y los llevó a Jerusalén. \bibleverse{8} El rey
David también tomó una gran cantidad de bronce de Beta y Berotai,
ciudades que habían pertenecido a Hadad-ezer.

\bibleverse{9} Cuando Toi, rey de Hamat, se enteró de que David había
destruido todo el ejército de Hadadzer, rey de Soba, \bibleverse{10}
envió a su hijo Joram a David para que se hiciera amigo de él y lo
felicitara por su victoria en la batalla contra Hadad-ezer. Toi y
Hadadezer habían estado en guerra con frecuencia. Joram trajo toda clase
de regalos de oro, plata y bronce. \bibleverse{11} El rey David le
dedicó estos regalos al Señor, junto con la plata y el oro que había
tomado de todas las naciones que había sometido: \bibleverse{12} Edom,
Moab, los amonitas, los filisteos y los amalecitas; así como el botín
tomado a Hadad-ezer, hijo de Rehob, rey de Soba.

\hypertarget{derrota-y-subyugaciuxf3n-de-los-edomitas}{%
\subsection{Derrota y subyugación de los
edomitas}\label{derrota-y-subyugaciuxf3n-de-los-edomitas}}

\bibleverse{13} David también se dio a conocer cuando regresó tras
derrotar a dieciocho mil edomitas\footnote{\textbf{8:13} El texto hebreo
  se refiere en realidad a los arameos, pero en el contexto debe ser un
  error de los escribas.} en el Valle de la Sal. \footnote{\textbf{8:13}
  Sal 60,2} \bibleverse{14} Colocó guarniciones por todo Edom, y todos
los edomitas se sometieron a David. El Señor le dio a David victorias
dondequiera que fuera. \footnote{\textbf{8:14} Gén 27,40}

\hypertarget{directorio-de-los-principales-oficiales-de-david}{%
\subsection{Directorio de los principales oficiales de
David}\label{directorio-de-los-principales-oficiales-de-david}}

\bibleverse{15} David gobernó sobre todo Israel. Hizo lo que era justo y
correcto para todo su pueblo. \bibleverse{16} Joab, hijo de Sarvia, era
el comandante del ejército, y Josafat, hijo de Ahilud, llevaba los
registros oficiales. \footnote{\textbf{8:16} 2Sam 20,23-26}

\bibleverse{17} Sadoc, hijo de Ahitob, y Ahimelec, hijo de Abiatar, eran
sacerdotes, mientras que Seraías era el secretario. \bibleverse{18}
Benaía, hijo de Joiada, estaba a cargo de los queretanos y
peletanos;\footnote{\textbf{8:18} ``Los queretanos y peletanos'': la
  guardia personal del rey.} y los hijos de David eran
sacerdotes.\footnote{\textbf{8:18} ``Los hijos de David eran
  sacerdotes'': evidentemente, al no ser levitas, los hijos de David no
  serían sacerdotes en el sentido de oficiar en ceremonias religiosas.
  Algunos han sugerido que, tal como se usa aquí, la palabra significa
  ``administradores''. Véase el pasaje paralelo en 1 Crónicas 18:17.}

\hypertarget{la-generosidad-de-david-hacia-el-hijo-de-jonatuxe1n-mefiboset}{%
\subsection{La generosidad de David hacia el hijo de Jonatán,
Mefiboset}\label{la-generosidad-de-david-hacia-el-hijo-de-jonatuxe1n-mefiboset}}

\hypertarget{section-8}{%
\section{9}\label{section-8}}

\bibleverse{1} Entonces David dijo: ``¿Queda alguien de la familia de
Saúl a quien yo pueda mostrarle mi bondad por amor a Jonatán?'' .
\bibleverse{2} Y había un hombre llamado Siba que era siervo de la
familia de Saúl. Lo llamaron para que se acercara a David, y el rey le
preguntó: ``¿Eres Siba?'' . ``Sí, soy tu siervo'', respondió.
\footnote{\textbf{9:2} 2Sam 16,1}

\bibleverse{3} El rey le preguntó: ``¿Queda alguien de la familia de
Saúl a quien pueda mostrar mi bondad como se lo prometí ante Dios?''
\footnote{\textbf{9:3} ``Se lo prometí ante Dios'': Probablemente David
  está recordando la promesa mutua compartida con Jonatán. 1 Samuel
  20:42.} ``Todavía queda uno de los hijos de Jonatán, que es cojo de
ambos pies'', respondió Siba. \footnote{\textbf{9:3} 2Sam 4,4}

\bibleverse{4} ``¿Dónde está?'' , preguntó el rey. ``Está en la ciudad
de Lo-debar, viviendo en la casa de Maquir, hijo de Amiel'', respondió
Siba. \footnote{\textbf{9:4} 2Sam 17,27}

\hypertarget{las-magnuxe1nimas-disposiciones-de-david-con-respecto-a-mefiboset}{%
\subsection{Las magnánimas disposiciones de David con respecto a
Mefiboset}\label{las-magnuxe1nimas-disposiciones-de-david-con-respecto-a-mefiboset}}

\bibleverse{5} Así que el rey David hizo que lo trajeran de la casa de
Maquir. \bibleverse{6} Cuando Mefi-boset,\footnote{\textbf{9:6} Se le
  llama Meribbaal en 1 Crónicas 8:34 y 1 Crónicas 9:40. En este caso se
  plantea la misma cuestión que en la nota a pie de página de 2:8.} hijo
de Jonatán, hijo de Saúl, se acercó a David, se inclinó hacia el suelo
en señal de respeto. Entonces David dijo: ``Bienvenido Mefi-boset''.
``Soy tu siervo'', respondió él.

\bibleverse{7} ``No temas -- le dijo David -- porque de verdad seré
bondadoso contigo por amor a tu padre Jonatán. Te devolveré toda la
tierra que poseía tu abuelo Saúl, y siempre comerás en mi mesa''.

\bibleverse{8} Mefi-boset se inclinó y dijo: ``¿Quién soy yo, tu siervo,
para que te preocupes de un perro muerto como yo?''

\bibleverse{9} Entonces el rey llamó a Siba, el siervo de Saúl, y le
dijo: ``Le he dado al nieto de tu amo todo lo que pertenecía a Saúl y a
su familia. \bibleverse{10} Tú y tus hijos y los trabajadores deben
cultivar la tierra para él y traer el producto, para que el nieto de tu
amo tenga comida. Pero Mefi-boset, el nieto de tu amo, comerá siempre en
mi mesa''. Siba tenía quince hijos y veinte trabajadores.

\bibleverse{11} Entonces Siba le respondió al rey: ``Mi señor el rey, tu
siervo hará todo lo que le has ordenado''. Así que Mefi-boset comía en
la mesa de David como uno de los hijos del rey. \footnote{\textbf{9:11}
  2Sam 19,29}

\bibleverse{12} Mefi-boset tenía un hijo pequeño llamado Mica. Todos los
que vivían en la casa de Siba se convirtieron en siervos de Mefi-boset.
\bibleverse{13} Pero Mefi-boset vivía en Jerusalén, porque siempre comía
en la mesa del rey, y era cojo de ambos pies.

\hypertarget{el-vergonzoso-crimen-de-los-amonitas-contra-el-mensajero-de-david}{%
\subsection{El vergonzoso crimen de los amonitas contra el mensajero de
David}\label{el-vergonzoso-crimen-de-los-amonitas-contra-el-mensajero-de-david}}

\hypertarget{section-9}{%
\section{10}\label{section-9}}

\bibleverse{1} Algún tiempo después de esto, Nahas, el rey amonita murió
y su hijo Hanún lo sucedió. \bibleverse{2} David dijo: ``Seré bondadoso
con Hanún, hijo de Nahas, como su padre lo fue conmigo''. Así que David
envió representantes para llevar sus condolencias a Hanún por la muerte
de su padre. Pero cuando llegaron al país de los amonitas,

\bibleverse{3} los jefes militares amonitas le dijeron a Hanún, su rey:
``¿Realmente crees que David te envió sus condolencias por respeto a tu
padre? ¿No es más probable que David enviara a sus representantes para
explorar la ciudad, espiarla y luego conquistarla?''

\bibleverse{4} Entonces Hanún hizo detener a los representantes de
David, les afeitó la mitad de la barba a cada uno, les cortó la ropa a
la altura de las nalgas y los envió de vuelta a casa. \bibleverse{5}
Cuando David se enteró de esto, envió mensajeros a recibirlos, porque
estaban muy avergonzados. El rey les dijo: ``Quédense en Jericó hasta
que les vuelva a crecer la barba, y entonces podrán regresar''.

\hypertarget{comienzo-de-la-guerra-primeros-trabajos-ganados}{%
\subsection{Comienzo de la guerra; primeros trabajos
ganados}\label{comienzo-de-la-guerra-primeros-trabajos-ganados}}

\bibleverse{6} Cuando los amonitas se dieron cuenta de que se habían
vuelto como un mal olor para David, enviaron una solicitud a los arameos
y contrataron a veinte mil de sus soldados de a pie de Bet Rehob y Zoba,
así como a mil hombres del rey de Maaca, y también a doce mil hombres de
Tob. \bibleverse{7} Y cuando David se enteró de esto, envió a Joab y a
todo el ejército a enfrentarlos. \bibleverse{8} Los amonitas
establecieron sus líneas de batalla cerca de la entrada de la puerta de
su ciudad, mientras que los arameos de Soba y Rehob y los hombres de Tob
y Maaca tomaron posiciones por su cuenta en los campos abiertos.
\bibleverse{9} Joab se dio cuenta de que tendría que luchar tanto
delante como detrás de él, escogió algunas de las mejores tropas de
Israel y se puso al frente de ellas para dirigir el ataque a los
arameos. \bibleverse{10} Al resto del ejército lo puso bajo el mando de
Abisai, su hermano, pues también debían atacar a los amonitas.
\bibleverse{11} Entonces Joab le dijo: ``Si los arameos son más fuertes
que yo, ven a ayudarme. Si los amonitas son más fuertes que tú, yo
vendré a ayudarte. \bibleverse{12} Sé valiente y lucha lo mejor que
puedas por nuestro pueblo y las ciudades de nuestro Dios. Que el Señor
haga lo que considere bueno''. \bibleverse{13} Joab atacó con sus
fuerzas a los arameos y éstos huyeron de él. \bibleverse{14} Cuando los
amonitas vieron que los arameos habían huido, también huyeron de Abisai
y se retiraron a la ciudad. Entonces Joab regresó a Jerusalén después de
combatir a los amonitas.

\hypertarget{el-mismo-david-en-el-campo-su-victoria-sobre-los-sirios-aliados-con-los-amonitas}{%
\subsection{El mismo David en el campo; su victoria sobre los sirios
aliados con los
amonitas}\label{el-mismo-david-en-el-campo-su-victoria-sobre-los-sirios-aliados-con-los-amonitas}}

\bibleverse{15} En cuanto los arameos vieron que habían sido derrotados
por los israelitas, volvieron a reunir sus fuerzas. \bibleverse{16}
Hadad-ezer mandó traer más arameos de más allá del río Éufrates.
Llegaron a Helam bajo el mando de Sobac, comandante del ejército de
Hadad-ezer. \bibleverse{17} Cuando se le informó de esto a David, este
reunió a todo Israel. Cruzó el Jordán y avanzó sobre Helam. Los arameos
se colocaron en línea de batalla contra David y lo combatieron.
\bibleverse{18} Pero el ejército arameo huyó de los israelitas, y David
mató a 700 aurigas y a 40. 000 soldados de infantería. También atacó a
Sobac, el comandante de su ejército, y allí murió. \bibleverse{19}
Cuando todos los reyes aliados de Hadad-ezer se dieron cuenta de que
habían sido derrotados por Israel, hicieron la paz con David y se
sometieron a él. Como resultado, los arameos tuvieron miedo de seguir
ayudando a los amonitas.

\hypertarget{el-adulterio-de-david-con-betsabuxe9}{%
\subsection{El adulterio de David con
Betsabé}\label{el-adulterio-de-david-con-betsabuxe9}}

\hypertarget{section-10}{%
\section{11}\label{section-10}}

\bibleverse{1} En la primavera, en la época del año en que los reyes
salen a la guerra, David envió a Joab y a sus oficiales y a todo el
ejército israelí al ataque. Masacraron a los amonitas y sitiaron Rabá.
Sin embargo, David se quedó en Jerusalén. \footnote{\textbf{11:1} 1Cró
  20,1} \bibleverse{2} Una tarde, David se levantó de dormir la siesta y
se paseó por el tejado del palacio. Desde el tejado vio a una mujer
bañándose, una mujer muy hermosa. \footnote{\textbf{11:2} Mat 5,28-29}
\bibleverse{3} David envió a alguien a averiguar sobre la mujer. Le
dijeron: ``Es Betsabé, hija de Eliam y esposa de Urías el hitita''.
\footnote{\textbf{11:3} 2Sam 23,29}

\bibleverse{4} David envió mensajeros a buscarla. Cuando ella llegó a
él, David tuvo relaciones sexuales con ella. (Ella acababa de
purificarse al tener la regla).\footnote{\textbf{11:4} El hebreo se
  refiere a la ``impureza''.} Después volvió a casa. \footnote{\textbf{11:4}
  Lev 15,18} \bibleverse{5} Betsabé quedó embarazada y le envió un
mensaje a David para decirle: ``Estoy embarazada''.

\hypertarget{el-comportamiento-ejemplar-de-uria-durante-su-estancia-en-el-palacio-de-david}{%
\subsection{El comportamiento ejemplar de Uria durante su estancia en el
Palacio de
David}\label{el-comportamiento-ejemplar-de-uria-durante-su-estancia-en-el-palacio-de-david}}

\bibleverse{6} Entonces David envió un mensaje a Joab, diciéndole:
``Envíame a Urías el hitita''. Y Joab lo envió a David. \bibleverse{7}
Cuando Urías fue a verlo, David le preguntó cómo estaba Joab, cómo
estaba el ejército y cómo iba la guerra. \bibleverse{8} Entonces David
le dijo a Urías: ``Vete a casa y descansa''.\footnote{\textbf{11:8}
  ``Descansa'': literalmente, ``lava tus pies''.} Urías abandonó el
palacio, y el rey le envió un regalo después de su partida.
\bibleverse{9} Pero Urías no se fue a su casa. Durmió en la sala de
guardia a la entrada del palacio con todos los guardias del rey.
\bibleverse{10} A David le dijeron: ``Urías no fue a casa'', así que le
preguntó a Urías: ``¿No acabas de regresar de estar fuera? ¿Por qué no
has ido a casa?''

\bibleverse{11} Urías respondió: ``El Arca y los ejércitos de Israel y
de Judá están viviendo en tiendas, y mi amo Joab y sus hombres están
acampados al aire libre. ¿Cómo voy a ir a casa a comer y beber y a
dormir con mi mujer? Por mi vida no haré tal cosa''. \footnote{\textbf{11:11}
  1Sam 4,4}

\bibleverse{12} Pero David le dijo: ``Quédate aquí hoy, y mañana te
enviaré de vuelta''. Así que Urías se quedó en Jerusalén ese día y el
siguiente. \bibleverse{13} David invitó a Urías a cenar. Urías comió y
bebió con él, y David emborrachó a Urías. Pero por la noche se fue a
dormir en su estera con los guardias del rey, y no volvió a casa.

\hypertarget{la-carta-de-urias-la-muerte-de-uruxedas-el-mensaje-de-joab-a-david-aviso-del-rey}{%
\subsection{La carta de Urias; La muerte de Urías; El mensaje de Joab a
David; Aviso del
rey}\label{la-carta-de-urias-la-muerte-de-uruxedas-el-mensaje-de-joab-a-david-aviso-del-rey}}

\bibleverse{14} Por la mañana, David le escribió una carta a Joab y se
la dio a Urías para que se la llevara. \bibleverse{15} En la carta,
David le decía a Joab: ``Pon a Urías justo al frente, donde la lucha es
peor, y luego retrocede detrás de él para que lo ataquen y lo maten''.

\bibleverse{16} Mientras Joab asediaba la ciudad, hizo que Urías ocupara
un lugar donde sabía que lucharían los hombres más fuertes del enemigo.
\bibleverse{17} Cuando los defensores de la ciudad salieron y atacaron a
Joab, algunos de los hombres de David murieron, incluido Urías el
hitita. \bibleverse{18} Joab envió a David un informe completo sobre la
batalla. \bibleverse{19} Le ordenó al mensajero que dijera: ``Cuando
termines de contarle al rey todo sobre la batalla, \bibleverse{20} si el
rey se enoja y te pregunta: `¿Por qué te acercaste tanto al pueblo en el
ataque? ¿Acaso no sabías que iban a lanzar flechas desde la muralla?
\bibleverse{21} ¿Quién mató a Abimelec, hijo de Jerub-Beshet? ¿No fue
una mujer la que dejó caer una piedra de molino sobre él desde el muro,
matándolo allí en Tebez? ¿Por qué se acercó tanto a la muralla?' Tú
dile: `Además, tu oficial Urías el hitita fue asesinado'\,''.

\bibleverse{22} El mensajero se fue, y cuando llegó le dijo a David todo
lo que Joab le había indicado. \bibleverse{23} El mensajero le explicó a
David: ``Los defensores eran más fuertes que nosotros, y salieron a
atacarnos en campo abierto, pero los obligamos a retroceder hasta la
entrada de la puerta de la ciudad. \bibleverse{24} Sus arqueros nos
dispararon desde la muralla y mataron a algunos de los hombres del rey.
También mataron a su oficial Urías el hitita''.

\bibleverse{25} Entonces David le dijo al mensajero: ``Dile esto a Joab:
`No te alteres por esto, pues la espada destruye a la gente al azar.
Prosigue tu ataque contra la ciudad y conquístala'. Anímalo diciéndole
esto''.

\hypertarget{el-duelo-de-betsabuxe9-por-su-marido-su-matrimonio-con-david}{%
\subsection{El duelo de Betsabé por su marido; su matrimonio con
David}\label{el-duelo-de-betsabuxe9-por-su-marido-su-matrimonio-con-david}}

\bibleverse{26} Cuando la mujer de Urías se enteró de que su marido
había muerto, se puso de luto por él. \bibleverse{27} Una vez terminado
el período de luto, David mandó traerla a su palacio, y ella se
convirtió en su esposa y le dio un hijo. Pero lo que David había hecho
estaba mal ante los ojos del Señor.\footnote{\textbf{11:27} Éxod
  20,13-14}

\hypertarget{el-discurso-de-nathan-y-el-anuncio-de-la-perdiciuxf3n-la-confesiuxf3n-de-culpa-y-arrepentimiento-de-david}{%
\subsection{El discurso de Nathan y el anuncio de la perdición; La
confesión de culpa y arrepentimiento de
David}\label{el-discurso-de-nathan-y-el-anuncio-de-la-perdiciuxf3n-la-confesiuxf3n-de-culpa-y-arrepentimiento-de-david}}

\hypertarget{section-11}{%
\section{12}\label{section-11}}

\bibleverse{1} El Señor envió a Natán a ver a David. Cuando llegó allí,
le dijo: ``Había una vez dos hombres que vivían en la misma ciudad. Uno
era rico y el otro pobre. \bibleverse{2} El rico tenía muchos miles de
ovejas y ganado, \bibleverse{3} pero el pobre no tenía más que una
pequeña oveja que había comprado. La cuidó y creció con él y con sus
hijos. Comía de su plato y bebía de su copa. Dormía en su regazo y era
como una hija para él. \bibleverse{4} Un día, el hombre rico tuvo una
visita. No quiso tomar una de sus ovejas o ganado para alimentar a su
visitante. En cambio, tomó el cordero del pobre para preparar una comida
para su visitante''.

\bibleverse{5} David se puso absolutamente furioso con lo que hizo aquel
hombre, y le dijo airadamente a Natán ``¡Vive el Señor, que el hombre
que hizo esto debe ser condenado a muerte! \bibleverse{6} Debe pagar a
ese cordero con cuatro\footnote{\textbf{12:6} Ver Éxodo 22:1.} de los
suyos por hacer esto y por ser tan despiadado''.

\bibleverse{7} ``Tú eres ese hombre''. Le dijo Natán a David. ``Esto es
lo que dice el Señor, el Dios de Israel: `Yo te ungí rey de Israel y te
salvé de Saúl. \footnote{\textbf{12:7} 1Re 20,40} \bibleverse{8} Te di
la casa de tu amo y puse en tu regazo a las mujeres de tu amo. Te di el
reino de Israel y de Judá, y si eso no hubiera sido suficiente, te
habría dado mucho más. \bibleverse{9} Entonces, ¿por qué has tratado con
desprecio lo que dijo el Señor, haciendo el mal ante sus ojos? Mataste a
Urías el hitita con la espada y le robaste su esposa; lo mataste usando
la espada de los amonitas. \bibleverse{10} Por ello tus descendientes
siempre se enfrentarán a la espada\footnote{\textbf{12:10} ``Espada''
  utilizada en estos versos se refiere a cualquier tipo de muerte
  violenta.} porque me has despreciado y robaste la mujer de Urías'.
\footnote{\textbf{12:10} 2Sam 13,28-29; 2Sam 18,14; 2Re 25,7}

\bibleverse{11} ``Esto es lo que dice el Señor ahora: `Por lo que
hiciste, voy a traer desastre sobre tu propia familia. Arrebataré a tus
esposas ante tus propios ojos y se las daré a otro, y él se acostará con
tus esposas a la vista de todos. \bibleverse{12} Tú lo hiciste todo en
secreto, pero yo lo haré abiertamente donde todo Israel lo pueda
ver'\,''.

\bibleverse{13} Entonces David le dijo a Natán: ``He pecado contra el
Señor''. ``El Señor ha perdonado tus pecados. No vas a morir'',
respondió Natán.

\bibleverse{14} ``Pero como al hacer esto has tratado al Señor con total
desprecio, el hijo que tienes morirá''. \footnote{\textbf{12:14} 2Sam
  11,27}

\hypertarget{enfermedad-y-muerte-del-niuxf1o-betsabuxe9-el-dolor-y-la-renovada-valentuxeda-de-david-nacimiento-y-educaciuxf3n-de-salomuxf3n}{%
\subsection{Enfermedad y muerte del niño Betsabé; El dolor y la renovada
valentía de David; Nacimiento y educación de
Salomón}\label{enfermedad-y-muerte-del-niuxf1o-betsabuxe9-el-dolor-y-la-renovada-valentuxeda-de-david-nacimiento-y-educaciuxf3n-de-salomuxf3n}}

\bibleverse{15} Entonces Natán se fue a su casa. El Señor hizo que el
hijo que la mujer de Urías había dado a luz a David se pusiera muy
enfermo.

\bibleverse{16} David suplicó a Dios en favor del niño. Ayunó, se
dirigió a su habitación y pasó la noche tumbado en el suelo con ropa de
silicio.\footnote{\textbf{12:16} ``En tela de silicio'': Lectura de la
  Septuaginta y de los Rollos del Mar Muerto.} \bibleverse{17} Sus
superiores se acercaron a él y trataron de ayudarle a levantarse del
suelo, pero no quiso y rechazó sus llamados para ir comer.
\bibleverse{18} Al séptimo día el niño murió. Pero los funcionarios de
David tenían miedo de decirle que el niño había muerto, pues se decían
unos a otros: ``Mira, mientras el niño estaba vivo, hablamos con él, y
se negó a escucharnos. ¿Cómo vamos a decirle que el niño ha muerto?
Puede hacer algo muy malo''.

\bibleverse{19} Pero David vio que sus funcionarios cuchicheaban entre
ellos y se dio cuenta de que el niño había muerto. Así que preguntó a
sus funcionarios: ``¿Ha muerto el niño?'' . ``Sí, ha muerto'', le
respondieron.

\bibleverse{20} David se levantó del suelo, se lavó, se puso aceites
perfumados y se cambió de ropa. Luego fue a la casa del Señor y adoró.
Después volvió a su casa y pidió algo de comer. Entonces le sirvieron
una comida que él comió. \bibleverse{21} ``¿Por qué te comportas así?''
, le preguntaron sus funcionarios. ``Mientras el niño estaba vivo,
ayunabas y llorabas en voz alta, pero ahora que ha muerto, te levantas y
comes''.

\bibleverse{22} David respondió: ``Mientras el niño vivía, ayunaba y
lloraba en voz alta, porque pensaba: `Tal vez el Señor se apiade de mí y
lo deje vivir'. \bibleverse{23} Pero ahora que ha muerto, ¿qué sentido
tiene que siga ayunando? ¿Podré traerlo de vuelta? Un día moriré y me
iré con él, pero él nunca volverá a mí''.

\bibleverse{24} David consoló a su esposa Betsabé y le hizo el amor.
Ella dio a luz a un hijo y lo llamó Salomón. El Señor amaba al niño,
\bibleverse{25} por lo que envió un mensaje a través del profeta Natán
para que le pusiera el nombre de Jedidías,\footnote{\textbf{12:25} Que
  significa ``amado por el Lord''.} porque el Señor lo amaba.

\hypertarget{joab-conquista-rabuxe1-castigo-de-los-amonitas}{%
\subsection{Joab conquista Rabá; Castigo de los
amonitas}\label{joab-conquista-rabuxe1-castigo-de-los-amonitas}}

\bibleverse{26} En ese momento, Joab había estado luchando contra la
ciudad amonita de Rabá, y había capturado la fortaleza real.
\bibleverse{27} Joab envió mensajeros a David para decirle: ``He atacado
Rabá y también he capturado su suministro de agua. \bibleverse{28} Así
que llama al resto del ejército, asedia la ciudad y captúrala. De lo
contrario, capturaré la ciudad y me llevaré el crédito''.

\bibleverse{29} Así que David convocó al resto del ejército y marchó
hacia Rabá. La atacó y la capturó. \bibleverse{30} Tomó la corona de la
cabeza de su rey y la colocó en la cabeza de David. Pesaba un talento de
oro y estaba decorada con piedras preciosas. David tomó una gran
cantidad de botín de la ciudad. \bibleverse{31} David tomó a los
habitantes y los obligó a trabajar con sierras, picos de hierro y
hachas, y también los hizo trabajar haciendo ladrillos.\footnote{\textbf{12:31}
  El hebreo aquí no está claro.} Lo mismo hizo en todas las ciudades
amonitas. Luego David y todo el ejército israelita regresaron a
Jerusalén.

\hypertarget{el-amor-apasionado-de-amnuxf3n-su-indignaciuxf3n-hacia-su-media-hermana-thamar}{%
\subsection{El amor apasionado de Amnón; su indignación hacia su media
hermana
Thamar}\label{el-amor-apasionado-de-amnuxf3n-su-indignaciuxf3n-hacia-su-media-hermana-thamar}}

\hypertarget{section-12}{%
\section{13}\label{section-12}}

\bibleverse{1} El hijo de David, Absalón, tenía una hermosa hermana
llamada Tamar, y otro de los hijos de David, Amnón, se enamoró de ella.
\footnote{\textbf{13:1} 2Sam 3,2-3} \bibleverse{2} Amnón se encaprichó
tanto de su hermana Tamar que se sintió mal. Ella era virgen, y Amnón
vio que era imposible tenerla. \bibleverse{3} Sin embargo, Amnón tenía
un amigo llamado Jonadab, que era hijo de Simea, el hermano de David.
Jonadab era un hombre muy astuto. \bibleverse{4} Le preguntó a Amnón:
``¿Por qué tú, hijo del rey, estás tan decaído cada mañana? ¿Por qué no
me dices qué te pasa?'' ``Estoy enamorado de Tamar, la hermana de mi
hermano Absalón'', respondió Amnón.

\bibleverse{5} ``Acuéstate en tu cama y finge que estás enfermo'', le
dijo Jonadab. ``Cuando tu padre venga a verte, dile: `Por favor, haz que
mi hermana Tamar venga a darme algo de comer. Ella puede prepararla
mientras yo miro y puede entregármela'\,''.

\hypertarget{ejecuciuxf3n-del-infame-ataque}{%
\subsection{Ejecución del infame
ataque}\label{ejecuciuxf3n-del-infame-ataque}}

\bibleverse{6} Entonces Amnón se acostó y fingió estar enfermo. Cuando
el rey fue a verlo, Amnón le pidió: ``Por favor, haz que mi hermana
Tamar venga a hacer un par de tortas mientras yo miro, y que me las
entregue para comer''.

\bibleverse{7} Entonces David envió un mensaje a Tamar al palacio: ``Por
favor, ten la amabilidad de ir a la casa de tu hermano Amnón y
prepararle algo de comida''. \bibleverse{8} Así que Tamar fue a la casa
de su hermano Amnón, donde estaba acostado. Tomó un poco de masa, la
amasó y cocinó las tortas mientras él miraba. \bibleverse{9} Luego tomó
la sartén y la vació ante él, pero éste se negó a comer. ``¡Déjenme
todos!'' gritó Amnón. Y todos se fueron. \bibleverse{10} Entonces Amnón
le dijo a Tamar: ``Trae la comida aquí a mi habitación para que me la
des para comer''. Así que Tamar llevó al dormitorio de Amnón las tortas
que le había preparado. \bibleverse{11} Pero al entregarle la comida, él
la agarró y le dijo: ``¡Ven a la cama conmigo, hermana mía!''.
\footnote{\textbf{13:11} Lev 18,9}

\bibleverse{12} ``¡No, tú eres mi hermano!'', exclamó ella. ``¡No me
violes! Eso no es lo que hacemos en Israel. ¡No hagas algo tan
vergonzoso! \footnote{\textbf{13:12} Deut 22,21} \bibleverse{13} ¡Para y
piensa en mí! ¿Cómo podría soportar una desgracia semejante? Piensa
también en ti. ¡En Israel te tratarían con desprecio como a un completo
tonto! Por favor, habla con el rey, pues él no te impedirá casarte
conmigo''.

\bibleverse{14} Pero Amnón no quiso escucharla, y como era más fuerte
que ella, la violó.

\hypertarget{otro-pecado-vergonzoso-de-amnuxf3n-a-thamar}{%
\subsection{Otro pecado vergonzoso de Amnón a
Thamar}\label{otro-pecado-vergonzoso-de-amnuxf3n-a-thamar}}

\bibleverse{15} Entonces Amnón sintió rechazo por Tamar con un odio
inmenso. Su odio era tan fuerte que era mayor que el amor que le había
tenido antes. ``¡Levántate! ¡veye de aquí!'', le dijo.

\bibleverse{16} ``¡No! ¡No lo hagas!'', respondió ella. ``Despedirme en
desgracia sería un mal aún mayor que el que ya me has hecho''. Pero él
no la escuchó.

\bibleverse{17} Llamó a su criado y le dijo: ``¡Deshazte de esta mujer y
cierra la puerta tras ella!''

\bibleverse{18} Así que su criado la echó y cerró la puerta tras ella.
Tamar llevaba la larga túnica de una princesa, que es lo que llevaban
las hijas vírgenes del rey. \bibleverse{19} Entonces se puso ceniza en
la cabeza y se rasgó su larga túnica. Se puso las manos en la cabeza y
se fue llorando a gritos. \footnote{\textbf{13:19} Job 2,12}

\hypertarget{el-comportamiento-de-absaluxf3n-y-el-rey-despuuxe9s-del-ultraje}{%
\subsection{El comportamiento de Absalón y el rey después del
ultraje}\label{el-comportamiento-de-absaluxf3n-y-el-rey-despuuxe9s-del-ultraje}}

\bibleverse{20} Su hermano Absalón la encontró y le preguntó: ``¿Ha
estado tu hermano Amnón contigo? Cállate por el momento, hermana mía. Es
tu hermano. No te alteres tanto por ello''. Así que Tamar vivió como una
mujer arruinada y abandonada en la casa de su hermano Absalón.

\bibleverse{21} Cuando el rey David se enteró, se enojó mucho.
\bibleverse{22} Absalón no hablaba con Amnón lo odiaba por haber violado
a su hermana Tamar.

\hypertarget{la-venganza-de-absaluxf3n-contra-amnuxf3n}{%
\subsection{La venganza de Absalón contra
Amnón}\label{la-venganza-de-absaluxf3n-contra-amnuxf3n}}

\bibleverse{23} Unos dos años después, cuando sus pastores estaban en
Baal-hazor, cerca de Efraín, Absalón invitó a todos los hijos del rey a
unirse a las celebraciones.\footnote{\textbf{13:23} ``A unirse a las
  celebraciones''. Para mayor claridad, la esquila anual de ovejas era
  también un momento de fiesta.} \bibleverse{24} Se dirigió al rey y le
dijo: ``Yo, tu siervo, he contratado esquiladores. ¿Podrían acompañarme
el rey y sus siervos?''

\bibleverse{25} ``No, hijo mío'', respondió el rey, ``no podemos ir
todos. Seríamos una carga para ti''. Aunque Absalón insistió, David no
estuvo dispuesto a ir, pero le dio a Absalón su bendición.

\bibleverse{26} ``Pues entonces, al menos deja que mi hermano Amnón nos
acompañe'', respondió Absalón. ``¿Por qué quieres que vaya?'' , preguntó
el rey.

\bibleverse{27} Pero Absalón insistió, así que el rey envió a Amnón y a
sus otros hijos. \bibleverse{28} Absalón les dio órdenes a sus hombres,
diciendo: ``¡Atención! Cuando Amnón se sienta contento por haber bebido
vino y yo les diga: `¡Ataquen a Amnón!', entonces mátenlo. No tengan
miedo. Yo mismo se los orderno. Sean fuertes y valientes''.

\bibleverse{29} Así que los hombres de Absalón hicieron lo que éste les
había ordenado y mataron a Amnón. Entonces todo el resto de los hijos
del rey se levantó, subió a sus mulas y huyó.

\hypertarget{los-eventos-en-el-palacio-de-david-cuando-llegaron-las-terribles-noticias}{%
\subsection{Los eventos en el palacio de David cuando llegaron las
terribles
noticias}\label{los-eventos-en-el-palacio-de-david-cuando-llegaron-las-terribles-noticias}}

\bibleverse{30} Mientras regresaban, David recibió un mensaje: ``Absalón
ha matado a todos los hijos del rey; no queda ni uno solo''.

\bibleverse{31} El rey se levantó, se rasgó las vestiduras y se acostó
en el suelo. Y todos sus funcionarios estaban a su lado con las ropas
rasgadas. \bibleverse{32} Pero Jonadab, hijo de Simea, hermano de David,
le dijo: ``Su Majestad no debe pensar que han matado a todos los hijos
del rey; sólo ha muerto Amnón. Absalón lo ha estado planificando desde
el día en que Amnón violó a su hermana Tamar. \bibleverse{33} Así que,
Su Majestad, no crea el informe de que todos los hijos del rey han
muerto. Sólo Amnón está muerto''. \bibleverse{34} Mientras tanto,
Absalón había huido. Cuando el vigilante de Jerusalén\footnote{\textbf{13:34}
  ``En Jerusalén'': Añadido para mayor a claridad.} se asomó, vio que
una gran multitud se acercaba por el camino al oeste de él, bajando por
la ladera de la colina.\footnote{\textbf{13:34} La Septuaginta añade
  aquí: ``El vigilante fue y le dijo al rey: `Veo hombres que vienen de
  la dirección de Beth-horon, por la ladera de la colina'.''}
\bibleverse{35} Jonadab le dijo al rey: ``¿Lo ves? ¡Los hijos del rey
están llegando! Es exactamente como lo dijo tu siervo''. \bibleverse{36}
Cuando terminó de hablar, los hijos del rey entraron llorando y
lamentándose. Entonces el rey y todos sus funcionarios también lloraron.

\hypertarget{la-fuga-de-absaluxf3n-a-gesur-a-su-abuelo}{%
\subsection{La fuga de Absalón a Gesur a su
abuelo}\label{la-fuga-de-absaluxf3n-a-gesur-a-su-abuelo}}

\bibleverse{37} Absalón huyó a Talmai, hijo de Amihud, el rey de Gesur.
Todos los días David se lamentaba por su hijo Amnón.\footnote{\textbf{13:37}
  ``Amnón'': el nombre no se da explícitamente en el texto hebreo.}
\footnote{\textbf{13:37} 2Sam 3,3; 2Sam 14,23} \bibleverse{38} Después
de que Absalón huyó a Gesur, permaneció allí durante tres años.

\hypertarget{la-intervenciuxf3n-de-joab-la-conversaciuxf3n-de-la-sabia-esposa-de-thecoa-con-david}{%
\subsection{La intervención de Joab; la conversación de la sabia esposa
de Thecoa con
David}\label{la-intervenciuxf3n-de-joab-la-conversaciuxf3n-de-la-sabia-esposa-de-thecoa-con-david}}

\bibleverse{39} El rey David anhelaba ir a ver a Absalón, pues había
terminado de llorar la muerte de Amnón.

\hypertarget{section-13}{%
\section{14}\label{section-13}}

\bibleverse{1} Joab, hijo de Sarvia, sabía que el rey seguía pensando en
Absalón.\footnote{\textbf{14:1} El texto no dice si eran pensamientos
  positivos o negativos. Tal vez lo mejor sea mantenerlo neutral, ya que
  David ciertamente habría tenido sentimientos encontrados hacia
  Absalón.} \bibleverse{2} Entonces Joab envió un mensajero a Tecoa para
que trajera a una mujer sabia que vivía allí. Y le dijo: ``Finge estar
de luto. Ponte ropa de luto y no uses aceites perfumados. Ponte como una
mujer que lleva mucho tiempo de luto por los muertos. \bibleverse{3}
Luego ve al rey y dile esto''. Entonces Joab le indicó lo que debía
decir.

\hypertarget{el-primer-discurso-de-la-sabia}{%
\subsection{El primer discurso de la
sabia}\label{el-primer-discurso-de-la-sabia}}

\bibleverse{4} Cuando la mujer de Tecoa fue a ver al rey, se inclinó
hacia el suelo en señal de respeto y dijo: ``¡Por favor, ayúdeme, Su
Majestad!''

\bibleverse{5} ``¿Qué pasa?'' , le preguntó el rey. ``Lamentablemente
soy viuda. Mi marido ha muerto'', respondió ella.

\bibleverse{6} ``Su Majestad, yo tenía dos hijos. Se pelearon fuera, y
no había nadie para detenerlos. Uno de ellos golpeó al otro y lo mató.
\bibleverse{7} Ahora toda la familia está en mi contra, y me dicen:
`Entrega a tu hijo, que ha matado a su hermano, para que lo condenemos a
muerte por haber asesinado a su hermano. Así tampoco heredará nada'. Con
esto apagarían el último tizón de esperanza que tengo para continuar con
el nombre de mi marido y su familia en el mundo''.

\bibleverse{8} ``Vete a casa'', le dijo el rey a la mujer, ``y yo mismo
me encargaré de que se resuelva tu caso''.

\bibleverse{9} ``Gracias, Su Majestad'', dijo la mujer. ``Yo y mi
familia asumiremos la culpa,\footnote{\textbf{14:9} La mujer está
  sugiriendo que porque no está siguiendo la Ley de Moisés en la
  ejecución del asesino entonces ella y su familia deben ser culpados.}
y que Su Majestad y su familia sean considerados inocentes''.

\bibleverse{10} ``Si alguien se queja de ello, tráemelo aquí y no
volverá a molestarte'', le dijo el rey.

\bibleverse{11} ``Por favor, Majestad'', continuó la mujer, ``¡jura por
el Señor, tu Dios, que impedirás que la persona que quiere vengar el
asesinato lo empeore matando a mi hijo!'' ``Vive el Señor'', prometió,
``ni un solo pelo de la cabeza de tu hijo caerá al suelo''. \footnote{\textbf{14:11}
  1Sam 14,45; 1Re 1,52}

\hypertarget{nuevo-discurso-de-mujer-sabia}{%
\subsection{Nuevo discurso de mujer
sabia}\label{nuevo-discurso-de-mujer-sabia}}

\bibleverse{12} ``¿Podría pedir otra cosa, Su Majestad?'' , preguntó la
mujer. ``Adelante'', respondió él.

\bibleverse{13} ``¿Por qué has tramado algo similar contra el pueblo de
Dios?'' , preguntó la mujer. ``Ya que Su Majestad acaba de decidir mi
caso por lo que dijo, ¿no se ha condenado a sí mismo porque se niega a
traer de vuelta al hijo que desterró? \bibleverse{14} Sí, todos tenemos
que morir. Somos como el agua derramada en el suelo que no se puede
volver a recoger. Pero eso no es lo que hace Dios. Por el contrario, él
obra para que todo aquel que es desterrado pueda volver a casa con él.
\bibleverse{15} Por eso he venido a explicarle esto a Su Majestad,
porque alguien me ha asustado. Así que he pensado que es mejor hablar
con el rey, y que tal vez me conceda mi petición. \bibleverse{16} Tal
vez el rey me escuche y me salve del hombre que quiere separarnos a mí y
a mi hijo del pueblo elegido por Dios. \bibleverse{17} Y pensé: `Que lo
que diga Su Majestad me traiga la paz, pues Su Majestad es capaz de
distinguir entre el bien y el mal, como un ángel de Dios. Que el Señor,
tu Dios, esté contigo'\,''. \footnote{\textbf{14:17} 2Sam 19,28}

\hypertarget{el-rey-ve-a-travuxe9s-del-astuto-plan}{%
\subsection{El rey ve a través del astuto
plan}\label{el-rey-ve-a-travuxe9s-del-astuto-plan}}

\bibleverse{18} ``Por favor, no te niegues a responder a la pregunta que
voy a hacer'', le dijo el rey a la mujer. ``Por favor, haga su pregunta,
Su Majestad'', respondió ella.

\bibleverse{19} ``¿Todo esto es obra de Joab?'' , preguntó el rey. La
mujer respondió: ``Como usted vive, Su Majestad, nadie puede ocultarle
nada. Sí, fue Joab, tu oficial, quien me ordenó hacer esto; me dijo
exactamente lo que tenía que decir.

\bibleverse{20} Lo hizo para mostrar el otro lado de la situación, pero
Su Majestad es tan sabio como un ángel de Dios, y usted sabe todo lo que
sucede en este país''.

\hypertarget{la-promesa-de-david-joab-agradece-al-rey-por-haber-cumplido-su-pedido-y-trae-a-absaluxf3n}{%
\subsection{La promesa de David; Joab agradece al rey por haber cumplido
su pedido y trae a
Absalón}\label{la-promesa-de-david-joab-agradece-al-rey-por-haber-cumplido-su-pedido-y-trae-a-absaluxf3n}}

\bibleverse{21} El rey le dijo a Joab: ``Bien, lo haré. Ve y trae de
vuelta al joven Absalón''.

\bibleverse{22} Joab se inclinó con el rostro hacia el suelo en señal de
respeto y bendijo al rey. ``Hoy'', dijo Joab, ``yo, tu siervo, sé que me
apruebas, Su Majestad, porque has concedido mi petición''.

\bibleverse{23} Joab fue a Gesur y trajo a Absalón de vuelta a
Jerusalén. \footnote{\textbf{14:23} 2Sam 13,37} \bibleverse{24} Pero el
rey dio esta orden: ``Puede volver a su casa, pero no debe venir a
verme''. Así que Absalón volvió a su casa, pero no fue a ver al rey.

\hypertarget{la-belleza-de-absaluxf3n-sus-hijos}{%
\subsection{La belleza de Absalón; sus
hijos}\label{la-belleza-de-absaluxf3n-sus-hijos}}

\bibleverse{25} Absalón era admirado como el hombre más apuesto de todo
Israel. No tenía ni un solo defecto de la cabeza a los pies.
\bibleverse{26} Se cortaba el pelo todos los años porque se le ponía muy
pesado: pesaba doscientos siclos reales. \bibleverse{27} Tenía tres
hijos y una hija llamada Tamar, una mujer muy hermosa.

\hypertarget{absaluxf3n-hace-que-joab-lo-reconcilie-formalmente-con-su-padre}{%
\subsection{Absalón hace que Joab lo reconcilie formalmente con su
padre}\label{absaluxf3n-hace-que-joab-lo-reconcilie-formalmente-con-su-padre}}

\bibleverse{28} Absalón vivió en Jerusalén durante dos años, pero no se
le permitió ver al rey. \bibleverse{29} Absalón llamó a Joab para que le
permitiera ver al rey, para que Joab lo enviara al rey, pero Joab se
negó a ir. Absalón volvió a llamar a Joab, pero éste siguió sin venir.
\bibleverse{30} Entonces Absalón les dijo a sus siervos: ``Miren, el
campo de Joab está al lado del mío, y tiene cebada creciendo allí. Vayan
y préndanle fuego''. Los siervos de Absalón fueron y prendieron fuego al
campo.

\bibleverse{31} Joab fue a la casa de Absalón y preguntó: ``¿Por qué tus
siervos incendiaron mi campo?'' .

\bibleverse{32} ``Mira'' -- dijo Absalón, -- ``te he mandado llamar
diciendo: `Ven aquí. Quiero que vayas a ver al rey y le preguntes: ¿Por
qué me he molestado en volver de Gesur? Hubiera sido mejor que me
quedara allí'. Así que ve y haz que me vea el rey, y si soy culpable de
algo, que me mate''.

\bibleverse{33} Así que Joab fue y le contó al rey lo que Absalón había
dicho. Entonces David llamó a Absalón, quien vino y se inclinó con el
rostro en el suelo ante él en señal de respeto. Entonces el rey besó a
Absalón.

\hypertarget{actividades-ambiciosas-y-favorables-de-absalom}{%
\subsection{Actividades ambiciosas y favorables de
Absalom}\label{actividades-ambiciosas-y-favorables-de-absalom}}

\hypertarget{section-14}{%
\section{15}\label{section-14}}

\bibleverse{1} Algún tiempo después, Absalón se consiguió un carro de
guerra con caballos y cincuenta hombres como guardaespaldas para que
corrieran delante de él. \footnote{\textbf{15:1} 1Re 1,5} \bibleverse{2}
Solía levantarse temprano y se colocaba junto al camino principal que
conducía a la puerta de la ciudad. Cuando la gente llevaba un caso al
rey para que decidiera, Absalón los llamaba y les preguntaba: ``¿De qué
ciudad son ustedes?'' . Si respondían: ``Tu siervo es de tal tribu de
Israel'',

\bibleverse{3} Absalón les decía: ``Tienes razón y tienes un buen caso.
Es una pena que no haya nadie de parte del rey que los escuche''.
\bibleverse{4} Entonces les decía: ``Ojalá hubiera alguien que me
nombrara juez del país. Entonces todos podrían venir a mí con su caso o
su queja, y yo les haría justicia''. \bibleverse{5} Así, Cuando alguien
venía a inclinarse ante él, Absalón lo detenía extendiendo la mano,
tomándolo y besándolo. \bibleverse{6} Absalón trataba así a todos los
israelitas que acudían al rey para que les hiciera justicia. De esta
manera captó la lealtad de los hombres de Israel.

\hypertarget{la-conspiraciuxf3n-y-la-indignaciuxf3n-de-absaluxf3n-en-hebruxf3n}{%
\subsection{La conspiración y la indignación de Absalón en
Hebrón}\label{la-conspiraciuxf3n-y-la-indignaciuxf3n-de-absaluxf3n-en-hebruxf3n}}

\bibleverse{7} Cuatro\footnote{\textbf{15:7} Lectura de la Septuaginta y
  de la versión siríaca. El hebreo dice ``cuarenta''.} años después,
Absalón le pidió al rey: ``Por favor, déjame ir a Hebrón para cumplir
una promesa que le hice al Señor. \bibleverse{8} Porque yo, tu siervo,
hice esta promesa mientras vivía en Guesur, en Aram, diciendo: `Si el
Señor me hace volver a Jerusalén, adoraré al Señor en Hebrón'\,''.

\bibleverse{9} ``Ve en paz'', dijo el rey. Así que Absalón se fue a
Hebrón.

\bibleverse{10} Entonces Absalón envió a sus cómplices de entre todas
las tribus de Israel, diciendo: ``Cuando oigan el sonido del cuerno de
carnero, griten: `¡Absalón es rey en Hebrón!'\,''

\bibleverse{11} Doscientos hombres de Jerusalén se fueron con Absalón.
Habían sido invitados y fueron con toda inocencia, porque no sabían nada
de lo que se había planeado. \bibleverse{12} Mientras Absalón ofrecía
sacrificios, mandó llamar a Ahitofel el gilonita, consejero de David,
pidiéndole que viniera desde Gilo, la ciudad donde vivía. La
conspiración se hizo más fuerte, y los seguidores de Absalón seguían
aumentando. \footnote{\textbf{15:12} 2Sam 23,34}

\hypertarget{david-huye-apresuradamente-de-jerusaluxe9n-despuuxe9s-de-dejar-atruxe1s-algunas-concubinas}{%
\subsection{David huye apresuradamente de Jerusalén después de dejar
atrás algunas
concubinas}\label{david-huye-apresuradamente-de-jerusaluxe9n-despuuxe9s-de-dejar-atruxe1s-algunas-concubinas}}

\bibleverse{13} Un mensajero vino a decirle a David: ``Absalón tiene la
lealtad de los hombres de Israel''.

\bibleverse{14} David dijo a todos los funcionarios que estaban con él
en Jerusalén: ``¡Rápido! ¡Vayamos! De lo contrario, ¡no podremos
alejarnos de Absalón! Debemos partir de inmediato, o pronto nos
alcanzará, nos atacará y matará a la gente de la ciudad''.

\bibleverse{15} ``Sea cual sea la decisión de Su Majestad, haremos lo
que usted quiera'', respondieron los servidores del rey.

\bibleverse{16} El rey partió con toda su casa siguiéndolo, pero dejó
diez concubinas para que cuidaran el palacio.

\hypertarget{la-gente-de-guerra-marchando-frente-al-rey-la-lealtad-de-itthai}{%
\subsection{La gente de guerra marchando frente al rey; la lealtad de
Itthai}\label{la-gente-de-guerra-marchando-frente-al-rey-la-lealtad-de-itthai}}

\bibleverse{17} El rey partió con todos sus soldados siguiéndolo. Se
detuvo en la última casa, \bibleverse{18} y pasaron por delante de él
todos sus hombres, incluidos todos los cereteos y peletitas, y
seiscientos gitanos que habían venido con él desde Gat.

\bibleverse{19} Entonces el rey le dijo a Itai, de Gat: ``¿Por qué
vienes tú también con nosotros? Regresa y quédate con el nuevo rey,
porque eres un extranjero y un exiliado que está muy lejos de su casa.
\footnote{\textbf{15:19} 2Sam 18,2} \bibleverse{20} Acabas de llegar
aquí, así que ¿por qué voy a hacerte vagar con nosotros ahora, cuando ni
siquiera yo sé a dónde voy? Vuelve y llévate a tus hombres contigo. Que
el Señor te muestre bondad y fidelidad''.

\bibleverse{21} Pero Itai le respondió al rey: ``¡Vive el Señor y vive
su majestad, dondequiera que esté su majestad, viva o muerta, allí
estará su servidor!''

\bibleverse{22} ``¡Adelante, marchen!'' respondió David. Itai el gitano
pasó marchando con todos sus hombres y todas las familias que estaban
con él. \bibleverse{23} Toda la gente del campo gritaba al ver pasar a
todos los que estaban con David. Atravesaron el valle del Cedrón con el
rey en dirección al desierto.

\hypertarget{la-comisiuxf3n-de-david-a-sadoc-y-abiatar}{%
\subsection{La comisión de David a Sadoc y
Abiatar}\label{la-comisiuxf3n-de-david-a-sadoc-y-abiatar}}

\bibleverse{24} Sadoc también estaba allí, y todos los levitas estaban
con él, llevando el Arca del Pacto de Dios. Depositaron el Arca de Dios,
y Abiatar ofreció sacrificios hasta que todos salieron de la ciudad.
\bibleverse{25} Entonces el rey le dijo a Sadoc: ``Lleva el Arca de Dios
de vuelta a la ciudad. Si el Señor me aprueba, me hará volver y me
dejará ver de nuevo el Arca y su Tienda. \bibleverse{26} Pero si dice:
`No estoy conforme contigo', aquí estoy. Que me haga lo que mejor le
parezca''. \footnote{\textbf{15:26} 2Sam 10,12; 1Sam 3,18}
\bibleverse{27} El rey también le dijo al sacerdote Sadoc: ``Entiendes
la situación, ¿cierto?\footnote{\textbf{15:27} ``Entiendes la situación,
  ¿cierto?'' Esto podría traducirse como ``¿Sí ves?'' o ``¿No eres
  vidente?'' . La implicación es que David está confiando en Sadoc para
  que le haga saber lo que está sucediendo en Jerusalén.} Regresa a la
ciudad sano y salvo con tu hijo Ajimaz, y también con Jonatán, hijo de
Abiatar. Tú y Abiatar llevad a vuestros dos hijos de vuelta con ustedes.
\footnote{\textbf{15:27} 1Re 1,42} \bibleverse{28} Esperaré en los vados
del desierto hasta que tenga noticias tuyas''. \bibleverse{29} Sadoc y
Abiatar llevaron el Arca de Dios de vuelta a Jerusalén y se quedaron
allí.

\hypertarget{marcha-de-david-en-el-monte-de-los-olivos-su-orden-a-husai}{%
\subsection{Marcha de David en el monte de los Olivos; su orden a
Husai}\label{marcha-de-david-en-el-monte-de-los-olivos-su-orden-a-husai}}

\bibleverse{30} David siguió su camino hacia el Monte de los Olivos,
llorando mientras lo hacía. Llevaba la cabeza cubierta y caminaba
descalzo. Toda la gente que lo acompañaba se cubría la cabeza, llorando
a su paso.

\bibleverse{31} Y a David le dijeron: ``Ahitofel\footnote{\textbf{15:31}
  Ahitofel, consejero de David, era el padre de Eliam, según 23:34,
  quien a su vez era el padre de Betsabé (11:3). Esto seguramente habría
  sido un factor para que Ahitofel se uniera a la rebelión de Absalón.}
es uno de los que conspiran con Absalón''. Así que David oró: ``Señor,
por favor, haz que el consejo de Ahitofel no prospere''.

\bibleverse{32} Cuando David llegó a la cima del monte de los Olivos,
donde la gente adoraba a Dios, le salió al encuentro Husai, el arquita,
con el manto roto y con polvo en la cabeza. \bibleverse{33} David le
dijo: ``Si vienes conmigo, sólo serás una carga para mí, \bibleverse{34}
pero si regresas a la ciudad y le dices a Absalón: `¡Seré tu siervo, Su
Majestad! Antes trabajaba para tu padre, pero ahora trabajaré para ti',
entonces podrás bloquear el consejo de Ahitofel para mí. \footnote{\textbf{15:34}
  2Sam 17,7} \bibleverse{35} Sadoc y Abiatar, los sacerdotes, también
estarán allí. Cuéntales todo lo que oigas en el palacio del rey.
\bibleverse{36} Sus dos hijos, Ajimaz y Jonatán, están allí con ellos.
Envíamelos para que me cuenten todo lo que oyes''.

\bibleverse{37} El amigo de David, Husai, llegó a Jerusalén al mismo
tiempo que Absalón entraba en la ciudad.\footnote{\textbf{15:37} 1Cró
  27,33}

\hypertarget{siba-el-siervo-de-mefiboseth-da-presentes-al-rey-su-informe-de-mentiras-sobre-mefiboset}{%
\subsection{Siba, el siervo de Mefiboseth, da presentes al rey; su
informe de mentiras sobre
Mefiboset}\label{siba-el-siervo-de-mefiboseth-da-presentes-al-rey-su-informe-de-mentiras-sobre-mefiboset}}

\hypertarget{section-15}{%
\section{16}\label{section-15}}

\bibleverse{1} Cuando David pasó un poco más allá de la cima de la
montaña, allí estaba Ziba, el siervo de Mefi-boset, esperándole. Llevaba
ya ensillados dos asnos con doscientos panes, cien tortas de pasas, cien
frutas de verano,\footnote{\textbf{16:1} ``Frutas de verano'':
  Probablementehigos.} y un odre de vino. \footnote{\textbf{16:1} 2Sam
  9,2} \bibleverse{2} ``¿Para qué has traído esto?'' le preguntó David a
Siba. Siba respondió: ``Los burros son para que los monte la familia del
rey, el pan y la fruta de verano son para que coman los hombres, y el
vino es para que lo beban los que se desgastan en el desierto''.

\bibleverse{3} ``¿Dónde está el nieto de tu amo?'' ,\footnote{\textbf{16:3}
  Refiriéndose a Mefi-boset.} preguntó el rey. Siba respondió: ``Ha
decidido quedarse en Jerusalén. Dice: `Hoy el pueblo de Israel me
devolverá el reino de mi abuelo'\,''. \footnote{\textbf{16:3} 2Sam 19,27}

\bibleverse{4} El rey le dijo a Siba: ``¡Te doy todo lo que pertenece a
Mefi-boset!'' ``Me inclino ante ti'', respondió Siba. ``Que me apruebe,
Su Majestad''.

\hypertarget{comportamiento-indigno-de-simei-hacia-el-rey}{%
\subsection{Comportamiento indigno de Simei hacia el
rey}\label{comportamiento-indigno-de-simei-hacia-el-rey}}

\bibleverse{5} Cuando el rey David llegó a la ciudad de Bahurim, un
hombre de la familia de Saúl estaba saliendo. Se llamaba Simei, hijo de
Gera, y gritaba maldiciones al llegar. \bibleverse{6} Arrojó piedras a
David y a todos los oficiales del rey, a pesar de que los hombres del
rey y todos sus guardaespaldas rodeaban a David. \bibleverse{7} ``¡Sal
de aquí, vete, asesino, malvado!'' dijo Simei mientras maldecía.
\bibleverse{8} ``El Señor te ha pagado por toda la familia de Saúl que
mataste y por robarle el trono. El Señor le ha dado el reino a tu hijo
Absalón. Mira cómo has acabado en el desastre por ser un asesino''.

\bibleverse{9} Abisai, hijo de Sarvia, preguntó al rey: ``¿Por qué este
perro muerto debe maldecir a Su Majestad? Deja que vaya y le corte la
cabeza''. \footnote{\textbf{16:9} 1Sam 26,8} \bibleverse{10} ``¿Qué
tiene eso que ver con ustedes, hijos de Sarvia?'' , respondió el rey.
``Si me está maldiciendo porque el Señor se lo ha dicho, ¿quién puede
cuestionar lo que hace?'' .

\bibleverse{11} David dijo a Abisai y a todos sus oficiales: ``Miren, si
mi propio hijo está tratando de matarme, ¿por qué no va a quererlo aún
más este Benjamíta?\footnote{\textbf{16:11} La gente de la tribu de
  Benjamín era generalmente partidaria de Saúl, y en 1 Samuel 9:21 se
  describe a Saúl como benjamita.} Déjenlo en paz; que me maldiga,
porque el Señor se lo ha dicho. \bibleverse{12} Tal vez el Señor vea
cómo estoy sufriendo y me pague con bien sus maldiciones de hoy''.
\bibleverse{13} David y sus hombres continuaron por el camino, y Simei
los seguía por la ladera de enfrente. Siguió maldiciendo mientras
avanzaba, arrojándole piedras y tierra a David. \bibleverse{14} El rey y
todos los que estaban con él estaban cansados cuando llegaron al
Jordán.\footnote{\textbf{16:14} ``Jordán''. No aparece en el hebreo,
  pero sí en algunos manuscritos de la Septuaginta. Dado que el destino
  fue dado en 15:28 como ``los vados del desierto'' esto parece
  razonable.} Y David descansó allí.

\hypertarget{absaluxf3n-engauxf1ado-por-husai}{%
\subsection{Absalón engañado por
Husai}\label{absaluxf3n-engauxf1ado-por-husai}}

\bibleverse{15} Mientras tanto, Absalón y todos los israelitas que lo
acompañaban llegaron a Jerusalén, junto con Ahitofel. \bibleverse{16}
Husai el arquita, amigo de David, fue a ver a Absalón y declaró: ``¡Viva
el rey! ¡Viva el rey!''

\bibleverse{17} ``¿Así es como demuestras lealtad a tu amigo?'' preguntó
Absalón. ``¿Por qué no te fuiste con tu amigo?''

\bibleverse{18} ``¡Claro que no!'' respondió Husai. ``Estoy del lado del
elegido por el Señor, por el ejército y por todo el pueblo de Israel. Me
mantendré leal a él. \bibleverse{19} En todo caso, ¿por qué no habría de
servir a su hijo? De la misma manera que serví a tu padre, te serviré a
ti''.

\hypertarget{se-siguiuxf3-el-primer-consejo-de-ahitofel-de-absaluxf3n}{%
\subsection{Se siguió el primer consejo de Ahitofel de
Absalón}\label{se-siguiuxf3-el-primer-consejo-de-ahitofel-de-absaluxf3n}}

\bibleverse{20} Entonces Absalón le preguntó a Ahitofel: ``Dame tu
consejo. ¿Qué debemos hacer?''

\bibleverse{21} Ahitofel le dijo: ``Ve a dormir con las concubinas de tu
padre, las que él dejó aquí para cuidar el palacio. Así todos en Israel
se darán cuenta de que has ofendido tanto a tu padre que no hay vuelta
atrás, lo que animará a todos tus partidarios''. \footnote{\textbf{16:21}
  2Sam 15,16}

\bibleverse{22} Así que montaron una tienda en el techo del palacio y
Absalón entró y tuvo relaciones sexuales con las concubinas de su padre
a la vista de todos. \footnote{\textbf{16:22} 2Sam 12,11; Lev 18,8}

\bibleverse{23} En ese momento los consejos de Ahitofel eran como si
recibieran mensajes del propio Dios. Así consideraban tanto David como
Absalón los consejos de Ahitofel.

\hypertarget{el-segundo-excelente-consejo-de-ahitofel-fue-rechazado-por-husai-y-rechazado-por-absaluxf3n}{%
\subsection{El segundo excelente consejo de Ahitofel fue rechazado por
Husai y rechazado por
Absalón}\label{el-segundo-excelente-consejo-de-ahitofel-fue-rechazado-por-husai-y-rechazado-por-absaluxf3n}}

\hypertarget{section-16}{%
\section{17}\label{section-16}}

\bibleverse{1} Ahitofel le dijo a Absalón: ``Permíteme elegir doce mil
hombres y salir a perseguir a David esta noche. \footnote{\textbf{17:1}
  Sal 71,11} \bibleverse{2} Lo atacaré cuando esté cansado y débil. Lo
atraparé por sorpresa y todos sus hombres hayan huido. Sólo mataré al
rey \bibleverse{3} y traeré a todos los demás de vuelta. Cuando todos
regresen, todo el país estará en paz, salvo el único hombre que
persigues''.

\bibleverse{4} Este plan les pareció bien a Absalón y a todos los
ancianos de Israel. \bibleverse{5} Pero entonces Absalón dijo: ``Llama
también a Husai, el arquita, y oigamos también su consejo''.

\bibleverse{6} Cuando Husai entró, Absalón le preguntó: ``Ahitofel ha
recomendado este plan. ¿Debemos seguir adelante con él? Si no, ¿qué
sugieres?''

\bibleverse{7} ``Por primera vez el consejo de Ahitofel no es bueno'',
respondió Husai. \bibleverse{8} ``Tú sabes cómo son tu padre y sus
hombres. Son grandes luchadores, y ahora están tan furiosos como una osa
a la que le han robado sus cachorros. En todo caso, tu padre tiene
experiencia en tácticas militares, y no pasará la noche con sus hombres.
\bibleverse{9} Ahora mismo está escondido en una cueva o en un lugar
así. Si ataca primero y mueren algunos de tus hombres, la gente que se
entere dirá: `Los hombres de Absalón están siendo masacrados'.
\bibleverse{10} Entonces hasta el soldado más valiente que tenga el
corazón de un león se morirá de miedo, porque todos en Israel saben que
tu padre es un hombre poderoso que tiene hombres valientes con él.
\bibleverse{11} ``Mi recomendación es que convoques a todo el ejército
israelita desde Dan hasta Beerseba, un ejército tan numeroso como la
arena de la orilla del mar. Una vez que se hayan reunido, ¡tú mismo los
guiarás a la batalla! \bibleverse{12} Entonces atacaremos a David
dondequiera que esté, y caeremos sobre él como el rocío cae sobre la
tierra. Ni él ni uno solo de los hombres que lo acompañan quedarán con
vida. \bibleverse{13} Si trata de buscar protección en una ciudad, todo
Israel llevará cuerdas a esa ciudad, y la derribaremos en el valle para
que no quede ni una piedra''.

\bibleverse{14} Absalón y todos los jefes israelitas dijeron: ``El
consejo de Husai el arquita es mejor que el de Ahitofel''. Pero el Señor
había decidido frustrar el buen consejo de Ahitofel, que era mejor, para
traer el desastre a Absalón. \footnote{\textbf{17:14} 2Sam 15,31; 2Sam
  15,34}

\hypertarget{husai-y-los-sacerdotes-envuxedan-mensajes-en-secreto-al-rey-david-pone-sobre-el-jorduxe1n}{%
\subsection{Husai y los sacerdotes envían mensajes en secreto al rey;
David pone sobre el
Jordán}\label{husai-y-los-sacerdotes-envuxedan-mensajes-en-secreto-al-rey-david-pone-sobre-el-jorduxe1n}}

\bibleverse{15} Husai habló con Sadoc y Abiatar, los sacerdotes, y les
dijo: ``Ahitofel ha aconsejado a Absalón y a los dirigentes israelitas
que actúen de una manera, pero yo les he aconsejado que actúen de esta
otra manera. \bibleverse{16} Así que envíen rápidamente un mensaje a
David y díganle: `No esperes ni pases la noche en los vados del
desierto, sino cruza inmediatamente o el rey y todos los que están con
él serán destruidos'\,''.\footnote{\textbf{17:16} ``Destruidos'':
  literalmente, ``engullidos''.}

\bibleverse{17} Jonatán y Ahimaas se alojaban en En-rogel porque no
podían ser vistos al entrar en la ciudad. Una sirvienta vendría a
decirles lo que estaba sucediendo. Luego irían a avisarle al rey David.
\bibleverse{18} Pero un muchacho los vio y se lo dijo a Absalón. Así que
los dos salieron inmediatamente y fueron a la casa de un hombre en la
ciudad de Bahurim. Él tenía un pozo en su patio, y ellos se metieron en
él. \bibleverse{19} Su mujer tomó una tela para cubrir el pozo y la
extendió sobre la abertura, y luego esparció grano sobre ella. Nadie
sabía que los hombres estaban allí. \bibleverse{20} Cuando llegaron los
oficiales de Absalón, le preguntaron a la mujer: ``¿Dónde están Ahimaas
y Jonatán?'' . ``Cruzaron el arroyo'', respondió ella. Los hombres los
buscaron pero no los encontraron, así que volvieron a Jerusalén.

\bibleverse{21} Cuando los oficiales de Absalón se fueron, los dos
hombres salieron del pozo y se apresuraron a dar su mensaje al rey.
``Que todos se levanten y crucen el río de inmediato, porque el consejo
de Ahitofel es que te ataquen de inmediato''.

\bibleverse{22} Entonces David y todos los que estaban con él se
levantaron y cruzaron el Jordán. Cuando amaneció, no faltaba ya ninguno
por cruzar.

\hypertarget{el-suicidio-de-ahitofel}{%
\subsection{El suicidio de Ahitofel}\label{el-suicidio-de-ahitofel}}

\bibleverse{23} Cuando Ahitofel se dio cuenta de que su consejo había
sido ignorado, ensilló su burro y se fue a su casa en la ciudad donde
vivía. Puso en orden sus asuntos y luego se ahorcó. Murió y fue
enterrado en la tumba de su padre. \footnote{\textbf{17:23} Mat 27,5}

\hypertarget{absaluxf3n-comienza-la-persecuciuxf3n-de-david-y-le-da-a-amasa-el-mando-supremo-david-en-mahanaim}{%
\subsection{Absalón comienza la persecución de David y le da a Amasa el
mando supremo; David en
Mahanaim}\label{absaluxf3n-comienza-la-persecuciuxf3n-de-david-y-le-da-a-amasa-el-mando-supremo-david-en-mahanaim}}

\bibleverse{24} David siguió hasta Mahanaim, y Absalón cruzó el Jordán
con todo el ejército israelita. \bibleverse{25} Absalón había puesto a
Amasa al frente del ejército para sustituir a Joab. Amasa era hijo de un
hombre llamado Itra, el ismaelita\footnote{\textbf{17:25} ``Ismaelita'':
  Según 1 Crónicas 2:17. El hebreo aquí dice ``israelita''.} que vivía
con Abigail, la hija de Nahas y hermana de Servia, la madre de Joab.
\bibleverse{26} Los israelitas al mando de Absalón acamparon en la
tierra de Galaad.

\bibleverse{27} Cuando David llegó a Mahanaim, lo recibieron Sobi, hijo
de Nahas, de Rabá de los amonitas, Maquir, hijo de Ammiel, de Lo-debar,
y Barzilai el galaadita de Rogelim. \footnote{\textbf{17:27} 2Sam 9,4;
  1Re 2,7}

\bibleverse{28} Trajeron lechos, cuencos y jarras de barro, así como
trigo, cebada, harina, grano tostado, frijoles, lentejas,
\bibleverse{29} miel, cuajada, ovejas y queso de leche de vaca para que
David y el pueblo que lo acompañaba comieran. Porque decían: ``El pueblo
está hambriento, cansado y sediento por su travesía por el desierto''.

\hypertarget{las-uxf3rdenes-militares-de-david-salida-de-su-ejuxe9rcito}{%
\subsection{Las órdenes militares de David; Salida de su
ejército}\label{las-uxf3rdenes-militares-de-david-salida-de-su-ejuxe9rcito}}

\hypertarget{section-17}{%
\section{18}\label{section-17}}

\bibleverse{1} Entonces David organizó a los hombres que estaban con él
y puso al frente de ellos a comandantes de millares y comandantes de
centenas. \bibleverse{2} David envió el ejército dividido en tres
secciones. Un tercio estaba al mando de Joab, otro tercio estaba al
mando de Abisai, hijo de Sarvia, hermano de Joab, y otro tercio estaba
al mando de Ittai el geteo. El rey dijo a los hombres: ``Yo mismo saldré
a la batalla con ustedes''. \footnote{\textbf{18:2} 2Sam 15,19}

\bibleverse{3} Pero los hombres respondieron: ``¡No, no debes salir a la
batalla! Porque si tenemos que huir, no se preocuparán por nosotros.
Incluso si la mitad de nosotros muere, tampoco les importará. Pero tú
vales por diez mil de nosotros, así que es mejor que te quedes aquí y
nos envíes ayuda desde el pueblo''.

\bibleverse{4} ``Haré lo que te parezca mejor'', respondió el rey. El
rey se quedó junto a la puerta mientras todos sus hombres salían por
cientos y por miles.

\bibleverse{5} El rey ordenó a Joab, Abisai e Ittai: ``Traten al joven
Absalón con delicadeza por mí''. Todos los hombres oyeron que el rey
daba órdenes a cada uno de sus comandantes sobre Absalón.

\hypertarget{absaluxf3n-es-derrotado-y-asesinado-por-el-mismo-joab-su-tumba}{%
\subsection{Absalón es derrotado y asesinado por el mismo Joab; su
tumba}\label{absaluxf3n-es-derrotado-y-asesinado-por-el-mismo-joab-su-tumba}}

\bibleverse{6} El ejército de David salió a enfrentar a los israelitas
en una batalla, que se libró en el bosque de Efraín. \bibleverse{7} Los
israelitas fueron derrotados por los hombres de David y ese día murieron
muchos, unos veinte mil. \bibleverse{8} La batalla abarcó toda la
campiña, y ese día murieron más por culpa del bosque que por la espada.

\bibleverse{9} Absalón se topó con algunos de los hombres de David
cuando iba montado en su mula. Cuando la mula pasó por debajo de las
ramas retorcidas de un gran roble, los cabellos de Absalón se enredaron
en el árbol. La mula que montaba siguió avanzando, dejándolo colgado
entre la tierra y el cielo. \bibleverse{10} Uno de los hombres de David
vio lo sucedido y le dijo a Joab: ``¡Acabo de ver a Absalón colgado de
un roble!''

\bibleverse{11} ``¿Qué? ¿Lo viste así?'' le dijo Joab al hombre. ``¿Por
qué no lo mataste allí mismo? ¡Te habría dado diez siclos de plata y un
cinturón de soldado como recompensa!''

\bibleverse{12} Pero el hombre respondió: ``Aunque me dieras mil siclos
de plata, no le haría daño al hijo del rey. Todos oímos que el rey les
dio la orden a ti, a Abisai y a Itai: `Cuiden al joven Absalón por
mí'.\footnote{\textbf{18:12} El hebreo aquí es difícil, y no es el mismo
  que el del verso 5.} \footnote{\textbf{18:12} 2Sam 18,5}
\bibleverse{13} Si hubiera desobedecido y matado a Absalón\footnote{\textbf{18:13}
  Alternativamente, ``Si hubiera puesto mi propia vida en peligro
  matando a Absalón''.} --- y el rey se entera de todo, tú mismo no me
habrías defendido''.

\bibleverse{14} ``No voy a perder el tiempo esperando así contigo'', le
dijo Joab. Agarró tres lanzas y se las clavó en el corazón a Absalón
cuando aún estaba vivo, colgado de la encina. \bibleverse{15} Diez de
los guardias de Joab rodearon a Absalón y lo mataron a hachazos.
\bibleverse{16} Entonces Joab tocó el cuerno de carnero, y sus hombres
dejaron de perseguir a los israelitas porque Joab les había indicado que
se detuvieran. \bibleverse{17} Tomaron a Absalón y lo arrojaron a un
pozo profundo en el bosque, y amontonaron un gran montón de piedras
sobre él. Y todos los israelitas huyeron a sus casas.

\bibleverse{18} Absalón, en vida, había hecho una columna de piedra y la
había erigido en el Valle del Rey como monumento a sí mismo, pues
pensaba: ``No tengo un hijo\footnote{\textbf{18:18} En 14:27 consta que
  Absalón tenía tres hijos, así que o bien habían muerto o bien Absalón
  los había repudiado.} que mantenga vivo el recuerdo de mi nombre''. Le
puso su nombre al pilar, y aún hoy se llama Monumento a Absalón.

\hypertarget{david-recibe-la-noticia-de-la-muerte-de-absaluxf3n-su-dolor}{%
\subsection{David recibe la noticia de la muerte de Absalón; su
dolor}\label{david-recibe-la-noticia-de-la-muerte-de-absaluxf3n-su-dolor}}

\bibleverse{19} Entonces Ahimaas, hijo de Sadoc, dijo: ``Por favor,
déjame correr y llevar la buena noticia al rey de que el Señor lo ha
vindicado sobre sus enemigos''.

\bibleverse{20} ``No eres el hombre adecuado para llevar la buena
noticia hoy'', respondió Joab. ``Puedes hacerlo en otro momento, pero no
lo hagas hoy, porque el hijo del rey ha muerto''.

\bibleverse{21} Entonces Joab le dijo a un hombre de Etiopía: ``Ve y
dile al rey lo que has visto''. Este se inclinó ante Joab y se fue
corriendo.

\bibleverse{22} Ahimaas volvió a pedirle a Joab: ``¡No importa lo que
pase, por favor déjame correr también tras el etíope!'' ``Hijo, ¿por qué
quieres correr? no vas a conseguir nada por ello'', respondió Joab.

\bibleverse{23} ``No importa, quiero correr de todos modos'', dijo.
``¡Bien, empieza a correr!'' le dijo Joab. Ahimaas tomó la ruta por un
terreno más llano y alcanzó al etíope.

\hypertarget{david-en-la-puerta-de-mahanaim-su-dolor-por-la-muerte-de-absaluxf3n}{%
\subsection{David en la puerta de Mahanaim; su dolor por la muerte de
Absalón}\label{david-en-la-puerta-de-mahanaim-su-dolor-por-la-muerte-de-absaluxf3n}}

\bibleverse{24} David estaba sentado entre las puertas interiores y
exteriores. El vigilante subió al techo de la puerta junto a la muralla.
Se asomó y vio a un hombre que corría solo. \bibleverse{25} Así que bajó
gritando para avisar al rey. ``Si está solo, es que trae buenas
noticias'', respondió el rey. Cuando el primer corredor se acercó,

\bibleverse{26} el vigilante vio a otro que corría, y gritó al portero:
``¡Mira! Hay otro hombre que corre solo''. ``También él traerá buenas
noticias'', dijo el rey.

\bibleverse{27} ``El primer hombre me parece que corre como Ahimaas,
hijo de Sadoc'', dijo el vigilante. ``Es un buen hombre'', respondió el
rey. ``Traerá buenas noticias''.

\bibleverse{28} Ahimaas saludó a gritos al rey. Luego se acercó y se
inclinó boca abajo ante el rey. ``¡Bendito sea el Señor, tu Dios!'',
dijo. ``¡Ha derrotado a los hombres que se rebelaron contra Su
Majestad!''

\bibleverse{29} ``¿Cómo está el joven Absalón? ¿Está bien?'' , preguntó
el rey. Ahimaas respondió: ``Era muy caótico cuando me envió su oficial
Joab, su servidor. Realmente no sé qué estaba pasando''.

\bibleverse{30} ``Ponte a un lado y espera'', le dijo el rey. Así que
Ahimaas se puso a un lado y esperó.

\bibleverse{31} En ese momento llegó el etíope y dijo: ``¡Su Majestad,
escuche la buena noticia! Hoy el Señor ha derrotado a todos los que se
rebelaron contra ti''.

\bibleverse{32} ``¿Cómo está el joven Absalón? ¿Está bien?'' , preguntó
el rey. El etíope respondió: ``¡Que lo que le ha sucedido al joven les
suceda a los enemigos de Su Majestad y a todos los que se rebelan contra
usted!''

\bibleverse{33} El rey se derrumbó. Subió a la sala sobre la puerta y
lloró. Mientras caminaba, sollozaba: ``¡Hijo mío Absalón! ¡Hijo mío,
hijo mío Absalón! ¡Ojalá hubiera muerto yo en tu lugar, Absalón, hijo
mío, hijo mío!''

\hypertarget{section-18}{%
\section{19}\label{section-18}}

\bibleverse{1} Pronto le dijeron a Joab: ``El rey está llorando y
haciendo duelo por Absalón''.

\hypertarget{efecto-maligno-del-dolor-de-david-en-el-ejuxe9rcito-la-reprensiuxf3n-de-joab-david-se-levanta}{%
\subsection{Efecto maligno del dolor de David en el ejército; La
reprensión de Joab; David se
levanta}\label{efecto-maligno-del-dolor-de-david-en-el-ejuxe9rcito-la-reprensiuxf3n-de-joab-david-se-levanta}}

\bibleverse{2} La victoria de ese día se convirtió en luto para todo el
ejército, porque les dijeron: ``El rey está de luto por su hijo''.

\bibleverse{3} Aquel día volvieron a la ciudad como lo hacen los
derrotados, avergonzados por haber huido de la batalla. \bibleverse{4}
El rey se tomó el rostro entre las manos y sollozó en voz alta: ``¡Hijo
mío Absalón! Absalón, hijo mío, hijo mío!''

\bibleverse{5} Entonces Joab entró y le dijo al rey: ``Hoy has humillado
a todos tus hombres que han salvado tu vida y la de tus hijos, tus
hijas, tus esposas y tus concubinas. \bibleverse{6} Lo has hecho amando
a los que te odian y odiando a los que te aman. Hoy has dejado claro que
los comandantes y los hombres no significan nada para ti. Hoy estoy
seguro de que serías muy feliz si Absalón estuviera vivo y todos
nosotros estuviéramos muertos. \bibleverse{7} Así que levántate, sal y
da las gracias a tus hombres. Te juro por el Señor que si no lo haces,
no te quedará ni un hombre para esta noche. Eso será mucho peor para ti
que todos los desastres que has tenido desde tu juventud hasta ahora''.

\bibleverse{8} Entonces el rey se levantó y fue a sentarse a la puerta
de la ciudad.\footnote{\textbf{19:8} En otras palabras, David se hizo
  accesible a ellos, en lugar de quedarse encerrado en su habitación.} A
todos se les dijo: ``Mira, el rey está sentado en la puerta de la
ciudad''. Todos vinieron a ver al rey. Mientras tanto, los israelitas
habían huido y se habían ido a sus casas.

\hypertarget{sobre-de-sentimiento-popular-por-david-las-negociaciones-de-david-con-los-ancianos-de-juduxe1-y-con-amasa}{%
\subsection{Sobre de sentimiento popular por David; Las negociaciones de
David con los ancianos de Judá y con
Amasa}\label{sobre-de-sentimiento-popular-por-david-las-negociaciones-de-david-con-los-ancianos-de-juduxe1-y-con-amasa}}

\bibleverse{9} Todos entre las tribus de Israel discutían entre sí,
diciendo: ``El rey nos rescató de la persecución de nuestros enemigos,
nos salvó de los filisteos, pero ahora ha tenido que huir del país por
culpa de Absalón. \bibleverse{10} Ahora Absalón, el hombre que elegimos
para ser nuestro rey al ungirlo, ha muerto en la batalla. ¿Por qué no
hacemos algo e invitamos al rey David\footnote{\textbf{19:10} ``David'':
  nombre suministrado para mayor claridad.} a volver?''

\bibleverse{11} El rey David envió este mensaje a Sadoc y Abiatar, los
sacerdotes: ``Díganles a los ancianos de Judá: `¿Van a ser ustedes los
últimos en llevar al rey a su palacio, ya que el rey ha oído que todo
Israel lo quiere? \bibleverse{12} Ustedes son mis hermanos, mi propia
carne y sangre. ¿Por qué tendrían que ser los últimos en querer el
regreso del rey?' \bibleverse{13} Díganle a Amasa: `¿No eres tú también
mi carne y mi sangre? Que Dios me castigue muy severamente si a partir
de ahora no eres tú el comandante de mi ejército en lugar de Joab'\,''.
\bibleverse{14} Amasa convenció a todo el pueblo de Judá para que
apoyara unánimemente a David,\footnote{\textbf{19:14} ``David'': nombre
  suministrado para mayor claridad.} así que enviaron un mensaje al rey:
``Por favor, regresa, tú y todos los que están contigo''. \footnote{\textbf{19:14}
  2Sam 17,25; 1Cró 2,16-17}

\bibleverse{15} El rey emprendió su viaje de regreso, y cuando llegó al
Jordán, los hombres de Judá se reunieron con él en Gilgal para ayudarle
a cruzar el río.

\hypertarget{david-regresa-y-es-alcanzado-por-los-juduxedos-su-dulzura-hacia-simei}{%
\subsection{David regresa y es alcanzado por los judíos; su dulzura
hacia
Simei}\label{david-regresa-y-es-alcanzado-por-los-juduxedos-su-dulzura-hacia-simei}}

\bibleverse{16} Simeí,\footnote{\textbf{19:16} Ver 16:5.} hijo de Gera,
el benjamita de Bahurim, se apresuró a bajar con los hombres de Judá a
recibir al rey David. \bibleverse{17} Con él iban mil hombres de la
tribu de Benjamín, incluyendo a Siba, siervo de la familia de Saúl, así
como los quince hijos de Siba y veinte siervos. Se apresuraron a bajar
al Jordán para recibir al rey. \bibleverse{18} Cruzaron por el vado para
llevar la casa del rey y todo lo que éste quisiera. Simei cruzó el
Jordán y cayó de bruces ante el rey. \footnote{\textbf{19:18} 2Sam
  16,1-4; 2Sam 9,2; 2Sam 9,10}

\bibleverse{19} ``Su Majestad, por favor, perdóneme y no tenga en cuenta
el mal que yo, su siervo, hice cuando Su Majestad salió de Jerusalén.
Por favor, olvídelo todo. \bibleverse{20} Yo, tu siervo, reconozco que
he pecado. Pero ¡mira! Hoy soy el primero de las tribus de José que baja
a recibir a Su Majestad''.

\bibleverse{21} Abisai, hijo de Sarvia, dijo: ``¿No debería ser
ejecutado Simei por esto, por haber maldecido al ungido del Señor?''

\bibleverse{22} Pero David respondió: ``¿Qué tiene que ver eso con
ustedes, hijos de Sarvia?\footnote{\textbf{19:22} David no sólo responde
  a Abisai, sino también a Joab, el hermano de Abisai.} ¿Quieren ser mis
enemigos hoy? ¿Es este un día para ejecutar a alguien en Israel? ¿No
estoy seguro de que hoy vuelvo a ser el rey de Israel?'' \bibleverse{23}
David se volvió hacia Simei y le juró: ``No vas a morir''. \footnote{\textbf{19:23}
  2Sam 16,10}

\bibleverse{24} Entonces Mefi-boset, nieto de Saúl, fue a recibir al
rey. Se había negado a cuidarse los pies, a recortarse el bigote y a
lavarse la ropa desde el día en que el rey se fue hasta el día de su
regreso pacífico.

\hypertarget{mefiboset-se-justifica-a-suxed-mismo-contra-david}{%
\subsection{Mefiboset se justifica a sí mismo contra
David}\label{mefiboset-se-justifica-a-suxed-mismo-contra-david}}

\bibleverse{25} Cuando llegó de Jerusalén al encuentro del rey, éste le
preguntó: ``¿Por qué no has venido conmigo, Mefi-boset?'' .

\bibleverse{26} Mefi-boset respondió: ``Su Majestad, mi siervo Ziba me
engañó. Le dije: `Ensilla mi asno\footnote{\textbf{19:26} Lectura de la
  Septuaginta. Hebreo: ``Déjame ensillar mi asno''.} para que pueda
montarlo y partir con el rey', porque sabes que soy cojo.
\bibleverse{27} Siba me ha representado mal a mí, tu siervo, ante Su
Majestad. Sin embargo, Su Majestad es como un ángel de Dios, así que
haga lo que crea mejor. \bibleverse{28} Toda la familia de mi abuelo
sólo podía esperar la muerte de Su Majestad, pero usted me incluyó a mí,
su siervo, entre los que comen en su mesa. Entonces, ¿qué derecho tengo
a pedirle al rey algo más?'' . \footnote{\textbf{19:28} 2Sam 16,3; 2Sam
  14,17}

\bibleverse{29} ``¿Para qué hablar más de estos asuntos tuyos?''
respondió David. ``He decidido que tú y Siba se repartan la tierra''.
\footnote{\textbf{19:29} 2Sam 9,11}

\bibleverse{30} Entonces Mefi-boset le contestó al rey: ``¡Que se lo
quede todo! Me alegro de que Su Majestad haya vuelto a casa en paz''.
\footnote{\textbf{19:30} 2Sam 9,9-10; 2Sam 16,4}

\bibleverse{31} Barzilai, el Galaadita, también había bajado de Rogelim
para ayudar al rey a cruzar el Jordán y seguir su camino desde allí.

\hypertarget{cordial-conversaciuxf3n-de-barsillai-con-david-cruzando-el-jorduxe1n}{%
\subsection{Cordial conversación de Barsillai con David; Cruzando el
Jordán}\label{cordial-conversaciuxf3n-de-barsillai-con-david-cruzando-el-jorduxe1n}}

\bibleverse{32} Barzilai era muy anciano, de ochenta años de edad, y
como era un hombre muy rico, le había proporcionado alimentos al rey
mientras se encontraba en Mahanaim. \bibleverse{33} El rey le dijo a
Barzilai: ``Cruza el Jordán conmigo, y yo te mantendré mientras te
quedes conmigo en Jerusalén''. \footnote{\textbf{19:33} 2Sam 17,27}

\bibleverse{34} ``¿Cuánto tiempo crees que tengo que vivir para poder ir
a Jerusalén y quedarme allí con el rey?'' Barzilai respondió.
\bibleverse{35} ``Ya tengo ochenta años. Ya no disfruto de nada. No
puedo saborear lo que como o bebo. No puedo oír cuando la gente canta.
No tiene sentido que yo, tu siervo, sea otra carga para tu majestad.
\bibleverse{36} ¡Que tu siervo pueza cruzar el río Jordán con el rey es
suficiente recompensa para mí!\footnote{\textbf{19:36} Estos dos versos
  se presentan como preguntas en hebreo, pero funcionan mejor como
  afirmaciones en inglés y español.} \bibleverse{37} Entonces, que tu
siervo vuelva a su casa, para que yo muera en mi ciudad natal, cerca de
la tumba de mi padre y de mi madre. Pero aquí está tu siervo, hijo
mío\footnote{\textbf{19:37} El texto no dice explícitamente que Quimán
  es hijo de Barzillai, pero algunos manuscritos de la Septuaginta lo
  hacen y es una conclusión probable.} Quimán. Deja que cruce con Tu
Majestad, y trátalo como mejor te parezca''.

\bibleverse{38} El rey respondió: ``Quimán cruzará conmigo, y yo lo
trataré como mejor te parezca, y haré por ti lo que quieras''.

\bibleverse{39} Así que todos cruzaron el Jordán primero, y luego cruzó
el rey. El rey besó a Barzilai y lo bendijo, y luego Barzilai regresó a
su casa. \bibleverse{40} Luego el rey siguió hasta Gilgal, y Quimán fue
con él. Todo el ejército de Judá y la mitad del ejército de Israel
acompañaron al rey. \bibleverse{41} Pero pronto los hombres de Israel
que estaban allí se acercaron al rey y le preguntaron: ``¿Por qué
nuestros hermanos, los hombres de Judá, se llevaron en secreto a Su
Majestad y lo llevaron a usted y a su casa al otro lado del Jordán,
junto con todos sus hombres?''

\hypertarget{celos-y-amarga-disputa-entre-israel-y-juduxe1-para-alcanzar-a-david}{%
\subsection{Celos y amarga disputa entre Israel y Judá para alcanzar a
David}\label{celos-y-amarga-disputa-entre-israel-y-juduxe1-para-alcanzar-a-david}}

\bibleverse{42} Los hombres de Judá explicaron a los hombres de Israel:
``Lo hicimos porque el rey es uno de nuestros parientes. ¿Por qué se
molestan por esto? ¿Cuándo hemos comido la comida del rey? ¿Cuándo hemos
recibido algo para ustedes?''

\bibleverse{43} ``Tenemos diez acciones en el rey'', respondieron los
hombres de Israel, ``así que tenemos más derecho a David que ustedes.
Entonces, ¿por qué nos desprecian? ¿No fuimos nosotros los primeros en
hablar de recuperar a nuestro rey?'' Pero los hombres de Judá
argumentaron con más fuerza que los de Israel.

\hypertarget{uxf3rdenes-de-david-en-jerusaluxe9n}{%
\subsection{Órdenes de David en
Jerusalén}\label{uxf3rdenes-de-david-en-jerusaluxe9n}}

\hypertarget{section-19}{%
\section{20}\label{section-19}}

\bibleverse{1} Un agitador llamado Seba, hijo de Bicri, de la tribu de
Benjamín, se encontraba allí. Hizo sonar el cuerno de carnero y gritó:
``No tenemos ningún interés en David, ningún compromiso con el hijo de
Isaí. Israelitas, vámonos todos a casa''.

\bibleverse{2} Así que todos los hombres de Israel abandonaron a David
para seguir a Seba, hijo de Bicri. Pero los hombres de Judá acompañaron
a su rey todo el camino desde el Jordán hasta Jerusalén.

\bibleverse{3} Cuando David regresó a su palacio en Jerusalén, tomó a
las diez concubinas que había dejado para que cuidaran el
palacio\footnote{\textbf{20:3} See 15:16.} y los puso en una casa bajo
vigilancia. Se ocupó de sus necesidades, pero no se acostó con ellas.
Estuvieron presas hasta que murieron, viviendo como viudas. \footnote{\textbf{20:3}
  2Sam 16,21}

\bibleverse{4} Entonces el rey ordenó a Amasa: ``Convoca al ejército de
Judá. Haz que vengan a mí dentro de tres días, y ven tú también''.

\bibleverse{5} Amasa convocó al ejército de Judá, pero tardó más del
tiempo que le habían dado. \bibleverse{6} David habló entonces con
Abisai y le dijo: ``Ahora Seba, hijo de Bichri, nos va a causar más
problemas que Absalón. Lleva a los hombres del rey y persíguelo, o se
apoderará de las ciudades fortificadas y se alejará de nosotros''.

\bibleverse{7} Así que los hombres de Joab, junto con los queretanos,
los peletanos,\footnote{\textbf{20:7} ``Los queretanos, los peletanos'':
  La guardia personal de David.} y todos los combatientes
experimentados, salieron de Jerusalén para perseguir a Seba, hijo de
Bichri.

\hypertarget{el-asesinato-de-amasa-por-joab}{%
\subsection{El asesinato de Amasa por
Joab}\label{el-asesinato-de-amasa-por-joab}}

\bibleverse{8} Mientras estaban en la gran roca de Gabaón, Amasa los
alcanzó. Joab estaba vestido para la batalla. Sobre su ropa llevaba un
cinturón alrededor de la cintura con una daga en su vaina. Al avanzar,
se le cayó.\footnote{\textbf{20:8} Los detalles de lo que ocurre aquí no
  están claros. Algunos piensan que Joab tenía una daga oculta que se le
  cayó, tal vez dentro de su túnica. Otros piensan que dejó caer
  intencionadamente su espada para que pareciera que estaba desarmado,
  pero que tenía otra arma, una daga, todavía en su cinturón.}
\bibleverse{9} ``¿Cómo estás, hermano mío?'' preguntó Joab a Amasa. Joab
tomó a Amasa por la barba con su mano derecha para besarlo.
\bibleverse{10} Amasa no estaba preparado para el puñal que Joab tenía
en la mano izquierda. Joab lo apuñaló en el vientre y sus intestinos se
derramaron en el suelo. Joab no necesitó apuñalarlo dos veces, porque
Amasa ya estaba muerto. Entonces Joab y su hermano Abisai salieron en
persecución de Sabá. \footnote{\textbf{20:10} 1Re 2,5} \bibleverse{11}
Uno de los hombres de Joab se puso al lado de Amasa y le gritó: ``Si
están del lado de Joab y de David, ¡vengan y siguan a Joab!''.

\bibleverse{12} Pero Amasa estaba allí, tendido en su sangre en medio
del camino principal. Cuando el hombre vio que todo el mundo se detenía
a mirar, sacó el cuerpo del camino a un campo y arrojó un paño sobre él.
\bibleverse{13} Una vez que el cuerpo de Amasa estuvo fuera del camino,
todos los hombres siguieron a Joab en busca de Sabá.

\hypertarget{seba-de-joab-guerrea-y-asesina-a-instigaciuxf3n-de-una-mujer-inteligente-el-regreso-de-joab-a-jerusaluxe9n}{%
\subsection{Seba de Joab guerrea y asesina a instigación de una mujer
inteligente; El regreso de Joab a
Jerusalén}\label{seba-de-joab-guerrea-y-asesina-a-instigaciuxf3n-de-una-mujer-inteligente-el-regreso-de-joab-a-jerusaluxe9n}}

\bibleverse{14} Mientras tanto, Sabá había recorrido todas las tribus de
Israel y finalmente llegó a la ciudad de Abel-bet-maaca. Todos los
bicritas\footnote{\textbf{20:14} Miembros de su propio grupo familiar.}
se reunieron para la batalla y lo siguieron hasta la ciudad.
\bibleverse{15} El ejército de Joab llegó y sitió a Sabá en
Abel-Bet-Maacá. Construyeron una rampa de asedio contra la muralla
exterior de la ciudad. Mientras todo el ejército de Joab golpeaba la
muralla para derribarla,

\bibleverse{16} una mujer sabia de la ciudad gritó: ``¡Escuchen! ¡Por
favor, escuchen! Díganle a Joab que venga aquí para hablar con él''.
\bibleverse{17} Él se acercó a ella, y la mujer le preguntó: ``¿Eres
Joab?'' ``Sí, soy yo'', respondió él. ``Por favor, escucha lo que yo, tu
sierva, tengo que decirte'', le dijo ella. ``Te escucho'', respondió él.

\bibleverse{18} Entonces la mujer dijo: ``En tiempos pasados se decía:
`Si quieres un consejo, acude a Abel', y así se resolvían las
discusiones. \bibleverse{19} Yo soy del pueblo pacífico y fiel de
Israel. Tú tratas de destruir un pueblo que es como una madre en Israel.
¿Por qué quieres derribar la posesión del Señor?'' .

\bibleverse{20} ``¡Claro que no!'' respondió Joab. ``¡No es eso lo que
quiero! ¡No deseo destruir ni derribar esta ciudad! \bibleverse{21} Esa
no es la intención. Pero un hombre llamado Seba, hijo de Bicri, de la
región montañosa de Efraín, se ha rebelado contra el rey, contra David.
Entrega a este hombre y me retiraré de la ciudad''. ``Bien'', respondió
la mujer, ``su cabeza será arrojada por encima del muro para ti''.

\bibleverse{22} La mujer fue y habló con todos sobre su sabio plan. Así
que cortaron la cabeza de Sabá y se la arrojaron a Joab. Entonces Joab
hizo sonar el cuerno de carnero para dar la retirada, y todos sus
hombres abandonaron la ciudad y se fueron a casa. Y Joab regresó con el
rey a Jerusalén.

\hypertarget{los-altos-funcionarios-de-david}{%
\subsection{Los altos funcionarios de
David}\label{los-altos-funcionarios-de-david}}

\bibleverse{23} Joab comandaba todo el ejército de Israel. Benaía, hijo
de Joiada, estaba a cargo de los cereteos y los peleteos.
\bibleverse{24} Adoniram estaba a cargo de la fuerza de trabajo.
Josafat, hijo de Ahilud, llevaba los registros oficiales.
\bibleverse{25} Seva era el secretario. Sadoc y Abiatar eran los
sacerdotes, \bibleverse{26} e Ira el jairita era el sacerdote de David.

\hypertarget{declaraciuxf3n-de-la-deuda-de-sauxfal-el-requisito-de-los-gabaonitas}{%
\subsection{Declaración de la deuda de Saúl; el requisito de los
gabaonitas}\label{declaraciuxf3n-de-la-deuda-de-sauxfal-el-requisito-de-los-gabaonitas}}

\hypertarget{section-20}{%
\section{21}\label{section-20}}

\bibleverse{1} Una vez, durante el reinado de David, hubo una hambruna
durante tres años seguidos, y David le preguntó al Señor por ello. El
Señor le respondió: ``Es porque Saúl y su familia son culpables de
asesinar a los gabaonitas''.

\bibleverse{2} Entonces David convocó a los gabaonitas y habló con
ellos. Los gabaonitas no eran israelitas, sino que eran lo que quedaba
del pueblo de los amorreos. Los israelitas les habían hecho un
juramento,\footnote{\textbf{21:2} Ver Josué 3.} pero en su fervor
nacionalista por los israelitas y Judá, Saúl había tratado de
eliminarlos. \footnote{\textbf{21:2} Jos 9,15; Jos 9,19} \bibleverse{3}
``¿Qué puedo hacer por ustedes?'' les preguntó David a los gabaonitas.
``¿Cómo puedo compensaros para que puedan bendecir al pueblo del
Señor?''

\bibleverse{4} ``No se trata de que recibamos un pago en plata u oro de
Saúl o de su familia'', respondieron los gabaonitas. ``Además, no
tenemos derecho a que nadie en Israel muera por nosotros''. ``Haré lo
que me pidan'', respondió David.

\bibleverse{5} Ellos replicaron: ``En cuanto al hombre que nos destruyó,
que planeó impedir que tuviéramos un lugar donde vivir en todo el país
de Israel, \bibleverse{6} haz que nos entreguen a siete de los
descendientes varones de Saúl, y los colgaremos en presencia del Señor
en Gabaón de Saúl, el elegido del Señor''. ``Se los entregaré'', dijo el
rey.

\hypertarget{la-promesa-de-david-y-su-ejecuciuxf3n-a-la-familia-de-sauxfal}{%
\subsection{La promesa de David y su ejecución a la familia de
Saúl}\label{la-promesa-de-david-y-su-ejecuciuxf3n-a-la-familia-de-sauxfal}}

\bibleverse{7} Sin embargo, el rey perdonó a Mefi-boset, hijo de
Jonatán, hijo de Saúl, a causa del juramento hecho ante el Señor entre
David y Jonatán, hijo de Saúl. \footnote{\textbf{21:7} 1Sam 20,15-17}
\bibleverse{8} El rey tomó a Armoni y a Mefi-boset, los dos hijos de
Rizpa, hija de Aia, que ella había dado a luz a Saúl, y los cinco hijos
de Merab,\footnote{\textbf{21:8} En el texto hebreo se lee Mical, pero
  se la identifica como sin hijos en 6:23, y Merab se da como esposa de
  Adriel en 1 Samuel 18:19.} la hija de Saúl, que había dado a luz a
Adriel, hijo de Barzillai de Meola. \footnote{\textbf{21:8} 2Sam 3,7;
  1Sam 18,19} \bibleverse{9} Los entregó a los gabaonitas, y ellos los
colgaron en la colina en presencia del Señor. Los siete murieron al
mismo tiempo, ejecutados al comienzo de la cosecha de cebada.

\hypertarget{la-maravillosa-muestra-de-amor-de-rizpa-sepultura-de-los-huesos-de-sauxfal-y-sus-descendientes}{%
\subsection{La maravillosa muestra de amor de Rizpa; Sepultura de los
huesos de Saúl y sus
descendientes}\label{la-maravillosa-muestra-de-amor-de-rizpa-sepultura-de-los-huesos-de-sauxfal-y-sus-descendientes}}

\bibleverse{10} Rizpa, hija de Aja, tomó un poco de tela de silicio y la
extendió para sí misma sobre una roca.\footnote{\textbf{21:10}
  Probablemente para cubrir el suelo y como una sábana sobre ella para
  protegerla del sol y la lluvia.} Desde el comienzo de la cosecha hasta
el momento en que llegaron las lluvias y se derramaron sobre los
cuerpos, ella mantenía alejados a los pájaros durante el día y a los
animales salvajes durante la noche. \bibleverse{11} Cuando David se
enteró de lo que había hecho Rizpa, hija de Aja, concubina de Saúl,
\bibleverse{12} recuperó los huesos de Saúl y de su hijo Jonatán de
manos de los hombres de Jabes de Galaad, que los habían sacado de la
plaza pública de Bet-sán, donde los filisteos habían colgado los cuerpos
después de matar a Saúl en Gilboa. \footnote{\textbf{21:12} 1Sam 31,12}
\bibleverse{13} David hizo traer los huesos de Saúl y de su hijo
Jonatán, y también hizo recoger los huesos de los ahorcados.
\bibleverse{14} Entonces enterraron los huesos de Saúl y de su hijo
Jonatán en Zela, en la tierra de Benjamín, en la tumba de Cis, el padre
de Saúl. Una vez que terminaron de hacer todo lo que el rey había
ordenado, Dios respondió a sus oraciones para poner fin al hambre en la
tierra.

\hypertarget{algunas-hazauxf1as-de-los-guerreros-de-david-en-las-guerras-filisteas}{%
\subsection{Algunas hazañas de los guerreros de David en las guerras
filisteas}\label{algunas-hazauxf1as-de-los-guerreros-de-david-en-las-guerras-filisteas}}

\bibleverse{15} Después volvió a haber guerra entre los filisteos e
Israel. David bajó con sus hombres a luchar contra los filisteos, y se
quedó sin fuerzas. \bibleverse{16} Isbi-benob, uno de los descendientes
de Refa, cuya lanza de bronce pesaba trescientos siclos, y que llevaba
una espada nueva, dijo que iba a matar a David. \bibleverse{17} Pero
Abisai, hijo de Servia, acudió en su ayuda, atacó al filisteo y lo mató.
Entonces los hombres de David le juraron: ``¡No vuelvas a salir con
nosotros a combatir, para que no se apague la luz de Israel!''
\footnote{\textbf{21:17} 2Sam 23,18}

\bibleverse{18} Algún tiempo después hubo otra batalla con los filisteos
en Gob. Pero entonces Sibecai el husatita mató a Saf, uno de los
descendientes de Refa. \footnote{\textbf{21:18} 1Cró 20,4-8}
\bibleverse{19} En otra batalla con los filisteos en Gob, Elhanán, hijo
de Jair de Belén, mató al hermano de Goliat de Gat. El asta de su lanza
era tan gruesa como una vara de tejer. \footnote{\textbf{21:19} 1Sam
  17,7}

\bibleverse{20} En otra batalla en Gat, había un hombre gigantesco, que
tenía seis dedos en cada mano y seis dedos en cada pie, haciendo un
total de veinticuatro. También él descendía de los gigantes.
\bibleverse{21} Pero cuando insultó a Israel, Jonatán, hijo de Simea,
hermano de David, lo mató. \bibleverse{22} Estos cuatro eran los
descendientes de los gigantes de Gat, pero todos murieron en manos de
David y de sus hombres.

\hypertarget{el-cuxe1ntico-de-acciuxf3n-de-gracias-y-victoria-de-david-despuuxe9s-de-derrotar-a-sus-enemigos}{%
\subsection{El cántico de acción de gracias y victoria de David después
de derrotar a sus
enemigos}\label{el-cuxe1ntico-de-acciuxf3n-de-gracias-y-victoria-de-david-despuuxe9s-de-derrotar-a-sus-enemigos}}

\hypertarget{section-21}{%
\section{22}\label{section-21}}

\bibleverse{1} David cantó las palabras de este cántico al Señor el día
en que el Señor lo salvó de todos sus enemigos y de Saúl.\footnote{\textbf{22:1}
  Este pasaje es paralelo a Salmos 18.} \bibleverse{2} Entonces cantó:
El Señor es mi roca, mi fortaleza y mi salvador. \bibleverse{3} Él es mi
Dios, mi roca que me protege. Él me protege del mal, su poder me
salva,\footnote{\textbf{22:3} Literalmente, ``cuerno de mi salvación''.}
me mantiene seguro. Él es mi protector; es mi salvador; me libra de la
violencia. \bibleverse{4} Pido ayuda al Señor, merecedor de alabanza, y
me salva de los que me odian. \bibleverse{5} Las olas de la muerte me
arrastran, las aguas de la destrucción me inundan; \bibleverse{6} El
sepulcro enrolló sus cuerdas en torno a mí; la muerte me tendió trampas.
\bibleverse{7} En mi desesperación invoqué al Señor; clamé a mi Dios. Él
escuchó mi voz desde su Templo; mi grito de auxilio llegó a sus oídos.
\bibleverse{8} La tierra se estremeció, los cimientos de los cielos
temblaron por su cólera \bibleverse{9} Humo salía de sus narices, y
fuego de su boca, carbones ardientes que ardían ante él. \bibleverse{10}
Apartó los cielos y descendió, con nubes oscuras bajo sus pies.
\bibleverse{11} Montado en un ser celestial\footnote{\textbf{22:11}
  Literalmente, ``querubín'', pero en inglés se ha asociado con un bebé
  angelical.} voló, abalanzándose sobre las alas del viento.
\bibleverse{12} Se escondió en las tinieblas, cubriéndose con negras
nubes de lluvia. \bibleverse{13} De su resplandor brotaron carbones
ardientes. \bibleverse{14} El Señor tronó desde el cielo; resonó la voz
del Altísimo. \bibleverse{15} Disparó sus flechas, dispersando a sus
enemigos,\footnote{\textbf{22:15} Implícito.} los derrotó con sus rayos.
\bibleverse{16} El Señor rugió, y con el viento del aliento de su nariz
se vieron los valles del mar y se descubrieron los cimientos de la
tierra. \bibleverse{17} Bajó su mano desde arriba y me agarró. Me sacó
de las aguas profundas. \bibleverse{18} Me rescató de mis poderosos
enemigos, de los que me odiaban y eran mucho más fuertes que yo.
\bibleverse{19} Se abalanzaron sobre mí en mi peor momento,\footnote{\textbf{22:19}
  Literalmente, ``el día de mi destraste''.} pero el Señor me sostuvo.
\bibleverse{20} Me liberó,\footnote{\textbf{22:20} Literalmente, ``me
  llevó a un lugar espacioso''.} me rescató porque es feliz
conmigo.\footnote{\textbf{22:20} O ``Se deleira en mi''.}
\bibleverse{21} El Señor me recompensó por hacer lo correcto; me pagó
porque soy inocente.\footnote{\textbf{22:21} Literalmente, ``por la
  limpieza de mis manos''.} \bibleverse{22} Porque he seguido los
caminos del Señor; no he pecado apartándome de mi Dios. \bibleverse{23}
He tenido presente todas sus leyes; no he ignorado sus mandamientos.
\bibleverse{24} Soy irreprochable a sus ojos; me guardo de pecar.
\bibleverse{25} El Señor me ha recompensado por hacer lo justo. Soy
inocente ante sus ojos. \bibleverse{26} Demuestras tu fidelidad a los
que son fieles; demuestras integridad a los que son íntegros,\footnote{\textbf{22:26}
  La palabra utilizada aquí significa ``complete'' o ``bueno''.}
\bibleverse{27} Te muestras puro a los que son puros, pero te muestras
astuto con los astutos. \bibleverse{28} Tú salvas a los humildes, pero
tus ojos vigilan a los soberbios para abatirlos. \bibleverse{29} Tú,
Señor, eres mi lámpara. El Señor ilumina mis tinieblas. \bibleverse{30}
Contigo puedo abatir una tropa de soldados; contigo, Dios mío, puedo
escalar un muro de la fortaleza. \bibleverse{31} El camino de Dios es
absolutamente correcto.\footnote{\textbf{22:31} La palabra utilizada
  aquí, a menudo traducida como ``perfecto'', es la misma que en 18:25.}
La palabra del Señor es digna de confianza. Es un escudo para todos los
que acuden a él en busca de protección. \bibleverse{32} Porque ¿quién es
Dios sino el Señor? ¿Quién es la Roca, sino nuestro Dios?
\bibleverse{33} Dios me hace fuerte y me mantiene seguro.
\bibleverse{34} Me hace seguro como el ciervo, capaz de caminar por las
alturas con seguridad. \bibleverse{35} Me enseña a luchar en la batalla;
me da la fuerza para tensar un arco de bronce. \bibleverse{36} Me
protege con el escudo de su salvación; su ayuda me ha engrandecido.
\bibleverse{37} Me diste espacio para caminar y evitaste que mis pies
resbalaran. \bibleverse{38} Perseguí a mis enemigos y los alcancé. No me
devolví hasta haberlos destruido. \bibleverse{39} Los derribé y no
pudieron levantarse. Cayeron a mis pies. \bibleverse{40} Me hiciste
fuerte para la batalla; hiciste que los que se levantaron contra mí se
arrodillaran ante mí. \bibleverse{41} Hiciste que mis enemigos huyeran;
destruí a todos mis enemigos. \bibleverse{42} Ellos clamaron por ayuda,
pero nadie vino a rescatarlos. Incluso clamaron al Señor, pero él no les
respondió. \bibleverse{43} Los convertí en polvo, como el polvo de la
tierra. Los aplasté y los arrojé como lodo en la calle. \bibleverse{44}
Me rescataste de los pueblos rebeldes; me mantuviste como gobernante de
las naciones: gente que no conocía ahora me sirve. \bibleverse{45} Los
extranjeros se acobardan ante mí; en cuanto oyen hablar de mí, me
obedecen. \bibleverse{46} Se desalientan y salen temblando de sus
fortalezas. \bibleverse{47} ¡El Señor vive! ¡Bendita sea mi Roca!
¡Alabado sea el Dios que me salva! \bibleverse{48} Dios me vindica, pone
a los pueblos bajo mis pies, \bibleverse{49} Y me libera de los que me
odian. Me mantiene a salvo de los que se rebelan contra mí, me salva de
los hombres violentos. \bibleverse{50} Por eso te alabaré entre las
naciones, Señor; cantaré alabanzas sobre lo que tú eres.\footnote{\textbf{22:50}
  ``Sobrelo quetú eres'': literalmente, ``a tu nombre'': el concepto de
  nombre en hebreo es mucho más que una simple designación; se refiere
  al carácter de la persona.} \bibleverse{51} Has salvado al rey tantas
veces,\footnote{\textbf{22:51} O ``Has dado muchas victorias al rey''.}
mostrando tu amor fiel a David, tu ungido, y a sus descendientes por
siempre.

\hypertarget{las-uxfaltimas-palabras-de-david}{%
\subsection{Las últimas palabras de
David}\label{las-uxfaltimas-palabras-de-david}}

\hypertarget{section-22}{%
\section{23}\label{section-22}}

\bibleverse{1} Estas son las últimas palabras de David. El mensaje
divino de David hijo de Isaí, el mensaje divino del hombre engrandecido
por Dios, el ungido por el Dios de Jacob, el maravilloso salmista de
Israel: \bibleverse{2} ``El Espíritu del Señor habló a través de mí; mi
lengua dio su mensaje. \bibleverse{3} ``El Dios de Israel habló; la Roca
de Israel me dijo: `El que gobierna al pueblo con justicia, el que
gobierna respetando a Dios, \bibleverse{4} es como la luz del sol de la
mañana que sale en un amanecer sin nubes; como el brillo de las gotas de
lluvia sobre la hierba nueva que crece de la tierra'. \bibleverse{5}
``¿No es así mi familia con Dios? Porque él ha hecho un acuerdo eterno
conmigo, detallado y con todas las partes garantizadas. Se asegurará de
salvarme y de darme todo lo que quiero. \bibleverse{6} ``Pero las
personas malas son como espinas que hay que arrojar a un lado, pues no
se las puede sostener con la mano. \bibleverse{7} La única manera de
tratar con ellos es usar una herramienta de hierro o el mango de una
lanza. Se queman por completo allí donde están''.

\hypertarget{directorio-y-hazauxf1as-de-los-guerreros-de-david}{%
\subsection{Directorio y hazañas de los guerreros de
David}\label{directorio-y-hazauxf1as-de-los-guerreros-de-david}}

\bibleverse{8} Estos son los nombres de los principales guerreros que
apoyaron a David: Joseb-basebet, un tacmonita, líder de los Tres. Usando
su lanza, una vez mató a ochocientos hombres en una sola batalla.
\bibleverse{9} Después de él vino Eleazar, hijo de Dodai, el ahohita,
uno de los Tres guerreros principales. Estaba con David cuando
desafiaron a los filisteos reunidos para la batalla en Pas-damin. Los
israelitas se retiraron, \bibleverse{10} pero Eleazar se mantuvo firme y
siguió matando filisteos hasta que se le quedó la mano en la espada. El
Señor los salvó concediéndoles una gran victoria. El ejército israelita
regresó, pero sólo para despojar a los muertos. \bibleverse{11} Después
de él vino Sama, hijo de Agee, el harareo. Cuando los filisteos se
reunieron en Lehi, en un campo lleno de lentejas, el ejército israelita
huyó de ellos, \bibleverse{12} pero Sama se apostó en medio del campo,
lo defendió y mató a los filisteos. El Señor les dio una gran victoria.

\hypertarget{riesgo-de-tres-huxe9roes}{%
\subsection{Riesgo de tres héroes}\label{riesgo-de-tres-huxe9roes}}

\bibleverse{13} En la época de la cosecha, los Tres, que formaban parte
de los Treinta guerreros principales, bajaron a recibir a David cuando
estaba en la cueva de Adulam. El ejército filisteo estaba acampado en el
valle de Refaim. \bibleverse{14} En ese momento David estaba en la
fortaleza, y la guarnición filistea estaba en Belén. \bibleverse{15}
David tenía mucha sed y dijo: ``¡Si tan solo alguien pudiera traerme un
trago de agua del pozo que está junto a la puerta de entrada de Belén!''

\bibleverse{16} Los tres guerreros principales rompieron las defensas
filisteas, tomaron un poco de agua del pozo de la puerta de Belén y se
la llevaron a David. Pero David se negó a beberla y la derramó como
ofrenda al Señor. \bibleverse{17} ``¡Señor, no me dejes hacer esto!'',
dijo. ``¿No es como beber la sangre de esos hombres que arriesgaron sus
vidas?'' Así que no la bebió. Estas son algunas de las cosas que
hicieron los tres guerreros principales.

\hypertarget{abisai-y-benaja}{%
\subsection{Abisai y Benaja}\label{abisai-y-benaja}}

\bibleverse{18} Abisai, hermano de Joab, era el líder del segundo
Tres.\footnote{\textbf{23:18} Sin embargo, ya se ha mencionado a
  Jasobeam como líder de los Tres (11:11), y también se ha mencionado la
  muerte de 300 con su lanza. Algunos sugieren una confusión de nombres
  o una ortografía alternativa, o que esto se refiere a otra persona en
  conjunto como líder no de los Tres sino de los Treinta, o que había
  otro ``Tres''.} Usando su lanza, una vez mató a 300 hombres, y se hizo
famoso entre los Tres. \footnote{\textbf{23:18} 2Sam 21,17}
\bibleverse{19} Era el más apreciado de los Tres y era su comandante,
aunque no fue uno del grupo del primer Tres.\footnote{\textbf{23:19}
  Identificar un primer y un segundo Tres parece ser la solución más
  sencilla a estos versos confusos.}

\bibleverse{20} Benaía, hijo de Joiada, un fuerte guerrero de Cabseel,
hizo muchas cosas sorprendentes. Mató a dos hijos de Ariel de
Moab.\footnote{\textbf{23:20} Entendido en la Septuaginta; puede
  referirse a dos campeones de lucha de Moab.} También fue tras un león
a un pozo en la nieve y lo mató. \bibleverse{21} En otra ocasión mató a
un enorme egipcio. El egipcio tenía una lanza en la mano, pero Benaía lo
atacó sólo con un garrote. Agarró la lanza de la mano del egipcio y lo
mató con su propia lanza. \bibleverse{22} Este fue el tipo de cosas que
hizo Benaía y que lo hicieron tan famoso como los tres principales
guerreros. \bibleverse{23} Era el más apreciado de los Treinta, aunque
no era uno de los Tres. David lo puso a cargo de su guardia personal.

\hypertarget{una-lista-de-otros-huxe9roes-de-david}{%
\subsection{Una lista de otros héroes de
David}\label{una-lista-de-otros-huxe9roes-de-david}}

\bibleverse{24} Entre los Treinta estaban: Asael, hermano de Joab;
Elhanan, hijo de Dodo, de Belén; \footnote{\textbf{23:24} 2Sam 2,18}
\bibleverse{25} Sama el harorita; Elica el harodita, \bibleverse{26}
Heles el paltita; Ira, hijo de Iques, de Tecoa; \bibleverse{27} Abiezer
de Anatot; Mebunai el husatita; \bibleverse{28} Salmón el ahohita;
Maharai el netofita; \bibleverse{29} Heleb, hijo de Baná el netofita;
Itai, hijo de Ribai, de Guibeá de los benjamitas; \bibleverse{30} Benaía
el piratonita; Hidai, de los arroyos de Gaas; \bibleverse{31} Abi-albón
el arbateo; Azmavet el baharumita; \bibleverse{32} Eliahba el
saalbonita; los hijos de Jasem; Jonatán, \bibleverse{33} hijo
de\footnote{\textbf{23:33} El hebreo no tiene ``hijo de''.} Sage el
hararita; Ahiam, hijo de Sarar el ararita; \bibleverse{34} Elifelet,
hijo de Ahasbai, hijo del maacateo; Eliam, hijo de Ahitofel, de Gilo,
\footnote{\textbf{23:34} 2Sam 15,12}

\bibleverse{35} Hezro el carmelita; Paarai el arbita; \bibleverse{36}
Igal, hijo de Natán de Soba; Bani el gadita, \bibleverse{37} Selec el
amonita; Naharai, el berotita, que era el escudero de Joab, hijo de
Servia, \bibleverse{38} Ira de Jatir; Gareb de Jatir, \bibleverse{39} y
Urías el hitita, para un total de treinta y siete.

\hypertarget{david-decide-el-censo-a-pesar-de-la-advertencia-de-joab}{%
\subsection{David decide el censo a pesar de la advertencia de
Joab}\label{david-decide-el-censo-a-pesar-de-la-advertencia-de-joab}}

\hypertarget{section-23}{%
\section{24}\label{section-23}}

\bibleverse{1} El Señor\footnote{\textbf{24:1} En 1 Crónicas 21:1 se
  identifica a Satanás como el que provocó a David a realizar el censo.
  Aquí, como en otras partes de la Escritura, puede ser que, puesto que
  Dios es todopoderoso, se le atribuya la responsabilidad incluso de las
  acciones que no comete específicamente.} estaba enojado con Israel, y
provocó a David contra ellos, diciendo: ``Ve y haz un censo de Israel y
de Judá''. \footnote{\textbf{24:1} 2Sam 21,1} \bibleverse{2} Entonces
David le dijo a Joab, el comandante del ejército: ``Ve y cuenta a los
israelitas desde Dan hasta Beerseba, para que yo tenga un número
total''.

\bibleverse{3} Pero Joab le respondió al rey: ``¡Que el Señor
multiplique su pueblo cien veces, Su Majestad, y que usted viva para
verlo! Pero, ¿por qué quiere hacer esto Su Majestad?''

\bibleverse{4} Pero el rey se mostró inflexible, así que Joab y los
comandantes del ejército dejaron al rey y fueron a censar al
pueblo\footnote{\textbf{24:4} Por supuesto, David está interesado
  principalmente en el número de hombres que puede llamar para servir en
  su ejército.} de Israel.

\hypertarget{ejecuciuxf3n-del-censo-y-su-resultado}{%
\subsection{Ejecución del censo y su
resultado}\label{ejecuciuxf3n-del-censo-y-su-resultado}}

\bibleverse{5} Cruzaron el Jordán y acamparon al sur de la ciudad de
Aroer, en medio del valle, y luego siguieron hacia Gad y Jazer.
\bibleverse{6} Luego pasaron a Galaad y a la tierra de Tahtim-hodshi, y
después siguieron hacia Dan, y de Dan a Sidón. \bibleverse{7} Después
fueron a la fortaleza de Tiro y a todas las ciudades de los heveos y
cananeos. Terminaron en el Néguev de Judá, en Beerseba. \bibleverse{8}
Después de recorrer todo el país durante nueve meses y veinte días,
regresaron a Jerusalén. \bibleverse{9} Joab informó al rey del número de
personas que habían sido contadas. En Israel había 800. 000 hombres
combatientes que podían usar la espada, y en Judá había 500. 000.

\hypertarget{el-arrepentimiento-de-david-intervenciuxf3n-del-profeta-gad-david-elige-una-muerte-popular-para-expiar-su-culpa-la-penitencia-y-la-suxfaplica-de-david}{%
\subsection{El arrepentimiento de David; Intervención del profeta Gad;
David elige una muerte popular para expiar su culpa; La penitencia y la
súplica de
David}\label{el-arrepentimiento-de-david-intervenciuxf3n-del-profeta-gad-david-elige-una-muerte-popular-para-expiar-su-culpa-la-penitencia-y-la-suxfaplica-de-david}}

\bibleverse{10} Después, David se sintió muy mal por haber ordenado el
censo. Le dijo a Dios: ``He cometido un terrible pecado al hacer esto.
Por favor, quita la culpa de tu siervo, porque he sido muy estúpido''.

\bibleverse{11} Cuando David se levantó por la mañana, el Señor había
enviado un mensaje al profeta Gad, vidente de David, diciendo:
\bibleverse{12} ``Ve y dile a David que esto es lo que dice el Señor:
`Te doy tres opciones. Elige una de ellas, y eso es lo que haré
contigo'\,''.

\bibleverse{13} Entonces Gad fue y le dijo a David: ``Puedes elegir
tres\footnote{\textbf{24:13} Lectura de la Septuaginta. En hebreo se lee
  ``siete años'', como en 1 Crónicas 21:12.} años de hambre en tu
tierra; o tres meses huyendo de tus enemigos mientras te persiguen; o
tres días de peste en tu tierra. Así que piénsalo y decide cómo debo
responder a quien me ha enviado''. \footnote{\textbf{24:13} Jer 24,10;
  Jer 29,17; Ezeq 6,12}

\bibleverse{14} Entonces David le respondió a Gad: ``¡Esta es una
situación terrible para mí! Por favor, deja que el Señor decida mi
castigo,\footnote{\textbf{24:14} ``Deja que el Señor decida mi
  castigo'': literalmente, ``déjame caer en manos del Señor''. También
  al final del verso, ``No me dejes caer en manos humanas''.} porque es
misericordioso. No permitas que la gente me castigue''.

\bibleverse{15} Así que el Señor envió una plaga sobre Israel desde
aquella mañana hasta la hora señalada, y murieron setenta mil personas
desde Dan hasta Beerseba. \bibleverse{16} Pero justo cuando el ángel se
disponía a destruir Jerusalén, el Señor cedió de causar semejante
desastre y le dijo al ángel destructor: ``Es suficiente. Ya puedes
parar''. Justo en ese momento el ángel del Señor estaba junto a la era
de Arauna el jebuseo.

\bibleverse{17} Cuando David vio que el ángel destruía al pueblo, le
dijo al Señor: ``Yo soy el que ha pecado; yo soy el que ha hecho el mal.
Esta gente es sólo una oveja. ¿Qué han hecho? Castígame a mí y a mi
familia''.

\hypertarget{montaje-de-un-altar-en-la-era-de-arawnas-fin-de-la-plaga}{%
\subsection{Montaje de un altar en la era de Arawnas; Fin de la
plaga}\label{montaje-de-un-altar-en-la-era-de-arawnas-fin-de-la-plaga}}

\bibleverse{18} Ese día Gad fue a ver a David y le dijo: ``Ve y
construye un altar al Señor en la era de Arauna el jebuseo''.

\bibleverse{19} Así que David fue e hizo lo que el Señor le había
ordenado, tal como Gad le había dicho. \bibleverse{20} Cuando Arauna
levantó la vista, vio que el rey y sus funcionarios se acercaban. Así
que salió y se inclinó ante el rey con el rostro en el suelo.
\bibleverse{21} ``¿Por qué ha venido Su Majestad a verme a mí, su
siervo?'' preguntó Arauna. ``Para comprar tu era y así poder construir
un altar al Señor para que se detenga la plaga del pueblo'', respondió
David.

\bibleverse{22} ``Tómala, y tu majestad podrá usarla para hacer ofrendas
como mejor te parezca'', le dijo Arauná a David. ``Aquí están los bueyes
para los holocaustos, y aquí están las tablas de trillar y los yugos
para los bueyes para la leña. \bibleverse{23} Su Majestad, yo, Arauna,
se lo doy todo al rey''. Arauna concluyó diciendo: ``Que el Señor, tu
Dios, te responda positivamente''.\footnote{\textbf{24:23} ``Te responda
  positivamente'': o, ``acepte''.}

\bibleverse{24} ``No, insisto en pagarte por ello'', respondió el rey.
``No presentaré al Señor, mi Dios, holocaustos que no me han costado
nada''. David compró la era y los bueyes por cincuenta siclos de plata.
\bibleverse{25} David construyó allí un altar al Señor y presentó
holocaustos y ofrendas de paz. El Señor respondió a su oración por el
país, y la plaga de Israel se detuvo.
