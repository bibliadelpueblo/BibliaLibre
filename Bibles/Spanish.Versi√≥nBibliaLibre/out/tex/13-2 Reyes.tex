\hypertarget{eluxedas-anuncia-la-muerte-del-rey-enfermo-e-iduxf3latra-ochuxf4zuxedas}{%
\subsection{Elías anuncia la muerte del rey enfermo e idólatra
Ochôzías}\label{eluxedas-anuncia-la-muerte-del-rey-enfermo-e-iduxf3latra-ochuxf4zuxedas}}

\hypertarget{section}{%
\section{1}\label{section}}

\bibleverse{1} Tras la muerte de Acab, Moab se rebeló contra Israel.
\footnote{\textbf{1:1} 2Re 3,5}

\bibleverse{2} Ocozías\footnote{\textbf{1:2} El hijo de Acab que le
  había sucedido como rey de Israel.} había caído por la
celosía\footnote{\textbf{1:2} Celosía: tal vez la persiana utilizada
  para cubrir una ventana.} de su habitación superior en Samaria y se
había herido gravemente. Así que envió mensajeros, diciéndoles: ``Vayan
y pregúntenle a Baal-zebub, el dios de Ecrón, si me curaré de esta
herida''. \footnote{\textbf{1:2} 1Re 22,52; Is 19,3}

\bibleverse{3} Pero el ángel del Señor le dijo a Elías tisbita: ``Ve a
encontrarte con los mensajeros del rey de Samaria y pregúntales: `¿Es
porque no hay Dios en Israel que vas a pedir consejo a Baal-zebub, el
dios de Ecrón?' \footnote{\textbf{1:3} Is 8,19} \bibleverse{4} Pues esta
es la respuesta del Señor: `No te levantarás del lecho en el que estás
acostado. Definitivamente vas a morir'\,''. Y Elías se fue.

\bibleverse{5} Los mensajeros volvieron al rey, y éste les preguntó:
``¿Por qué han vuelto?''

\bibleverse{6} ``Un hombre vino a nuestro encuentro'', respondieron.
``Nos dijo: `Vuelvan ante el rey que los ha enviado y díganle: Esto es
lo que dice el Señor: ¿Es porque no hay Dios en Israel por lo que mandas
a pedir consejo a Baal-zebub, el dios de Ecrón? Como resultado, no te
levantarásdel lecho en el que estás acostado. Definitivamente vas a
morir'\,''.

\bibleverse{7} ``¿Cómo era ese hombre que se reunió contigo y te contó
todo esto?'' , preguntó el rey.

\bibleverse{8} ``Era un hombre velludo que llevaba un cinturón de cuero
en la cintura'', respondieron. ``Es Elías el tisbita'', dijo el rey.
\footnote{\textbf{1:8} Zac 13,4; Mat 3,4}

\hypertarget{elijah-y-los-tres-capitanes}{%
\subsection{Elijah y los tres
capitanes}\label{elijah-y-los-tres-capitanes}}

\bibleverse{9} Entonces el rey envió a un capitán del ejército con
cincuenta hombres a Elías. El capitán se acercó a Elías, que estaba
sentado en la cima de un monte, y le dijo: ``Hombre de Dios, el rey te
ordena que bajes''.

\bibleverse{10} Elías respondió al capitán: ``Si soy un hombre de Dios,
que caiga fuego del cielo y te queme a ti y a tus cincuenta hombres''.
El fuego cayó del cielo y quemó al capitán y a sus hombres.

\bibleverse{11} Entonces el rey envió a otro capitán con sus cincuenta
hombres a Elías. El capitán le dijo a Elías: ``Hombre de Dios, el rey te
ordena que bajes inmediatamente''.

\bibleverse{12} Elías entonces le respondió al capitán: ``Si soy un
hombre de Dios, que caiga fuego del cielo y te queme a ti y a tus
cincuenta hombres''. El fuego cayó del cielo y quemó al capitán y a sus
hombres.

\bibleverse{13} Luego el rey envió a un tercer capitán con sus cincuenta
hombres. El tercer capitán subió, se arrodilló ante Elías y le suplicó:
``Hombre de Dios, por favor, valora mi vida y la de estos cincuenta
hombres. \bibleverse{14} Sí, ha caído fuego del cielo y ha quemado a los
dos primeros capitanes de cincuenta, junto con todos sus hombres. Pero
ahora, por favor, considera darme la vida''.

\hypertarget{eluxedas-con-ochuxf4zuxedas-muerte-del-rey}{%
\subsection{Elías con Ochôzías; Muerte del
rey}\label{eluxedas-con-ochuxf4zuxedas-muerte-del-rey}}

\bibleverse{15} Entonces el ángel del Señor le dijo a Elías: ``Baja con
él. No tienes que tener miedo de él''. Así que Elías se levantó y bajó
con él hasta el rey.

\bibleverse{16} Elías le dijo al rey: ``Esto es lo que dice el Señor:
`¿Es porque no hay Dios en Israel para que lo consultes que has enviado
mensajeros a pedir consejo a Baal-zebú, el dios de Ecrón? Como
resultado, no dejarás el lecho en el que estás acostado. Definitivamente
vas a morir'\,''. \footnote{\textbf{1:16} 2Re 1,3-4}

\bibleverse{17} Ocozías murió tal como el Señor había dicho por medio de
Elías. Como no tenía hijos, Joram\footnote{\textbf{1:17} ``Joram'',
  escrito originalmente Jehoram. Para mayor claridad, en esta traducción
  el rey de Israel se llamará Joram, mientras que el rey de Judá se
  llamará Jehoram, aunque los nombres son básicamente los mismos y se
  utilizan indistintamente en el texto hebreo.} le sucedió como rey en
el segundo año del reinado de Jehoram, hijo de Josafat, rey de Judá.
\footnote{\textbf{1:17} 2Re 3,1}

\bibleverse{18} El resto de lo que sucedió en el reinado de Ocozías y de
lo que éste hizo está registrado en el Libro de las Crónicas de los
Reyes de Israel.

\hypertarget{eluxedas-en-la-caminata-con-su-fiel-sirvienta-elisa}{%
\subsection{Elías en la caminata con su fiel sirvienta
Elisa}\label{eluxedas-en-la-caminata-con-su-fiel-sirvienta-elisa}}

\hypertarget{section-1}{%
\section{2}\label{section-1}}

\bibleverse{1} Justo antes de que el Señor se llevara a Elías al cielo
en un torbellino, Elías y Eliseo caminaban juntos en su camino desde
Gilgal. \bibleverse{2} Y Elías le dijo a Eliseo: ``Por favor, quédate
aquí, porque el Señor me ha enviado a Betel''. Pero Eliseo respondió:
``Vive el Señor y vives tú, no te dejaré''. Así que se fueron a Betel.

\bibleverse{3} Los hijos de los profetas que vivían en Betel se
acercaron a Eliseo y le dijeron: ``Sabes que el Señor te va a quitar hoy
a tu señor, ¿no?'' . ``Sí, lo sé'', respondió. ``No hables de ello''.

\bibleverse{4} Entonces Elías le dijo: ``Quédate aquí, Eliseo, porque el
Señor me ha enviado a Jericó''. Él respondió: ``Vive el Señor y vives
tú, no te dejaré''. Así que fueron a Jericó.

\bibleverse{5} Los hijos de los profetas que vivían en Jericó se
acercaron a Eliseo y le dijeron: ``Sabes que el Señor te va a quitar hoy
a tu señor, ¿no?'' . ``Sí, lo sé'', respondió. ``No hables de ello''.

\bibleverse{6} Entonces Elías le dijo: ``Quédate aquí, Eliseo, porque el
Señor me ha enviado al Jordán''. Él respondió: ``Vive el Señor y vives
tú, no te dejaré''. Así que siguieron viajando juntos.

\bibleverse{7} Entonces un grupo de cincuenta de los hijos de los
profetas fue y se puso frente a Elías y Eliseo a cierta distancia,
mientras los dos estaban junto al Jordán. \bibleverse{8} Elías tomó su
manto, lo enrolló y golpeó el agua. Se dividió a un lado y al otro y
ambos cruzaron en seco. \footnote{\textbf{2:8} Éxod 14,21-22; Jos 3,16}

\hypertarget{eluxedas-se-despide-de-eliseo-su-ascensiuxf3n}{%
\subsection{Elías se despide de Eliseo; su
ascensión}\label{eluxedas-se-despide-de-eliseo-su-ascensiuxf3n}}

\bibleverse{9} Cuando llegaron al otro lado, Elías le preguntó a Eliseo:
``¿Qué puedo hacer por ti antes de ser llevado?'' . ``Por favor, dame
una cantidad doble de tu espíritu'', respondió Eliseo. \footnote{\textbf{2:9}
  Deut 21,17}

\bibleverse{10} ``Lo que has pedido es difícil'', respondió Elías.
``Pero si me ves cuando me quiten de ti, lo tendrás, si no, no''.

\bibleverse{11} Mientras caminaban hablando, un carro de fuego y
caballos de fuego se interpuso entre ellos, y Elías fue llevado en el
torbellino al cielo. \footnote{\textbf{2:11} Gén 5,24} \bibleverse{12}
Eliseo vio lo ocurrido y gritó: ``¡Padre mío! ¡Padre mío! ¡Mira! ¡Los
carros y los jinetes de Israel!'' Entonces Eliseo ya no pudo verlo. Tomó
sus ropas y las hizo pedazos.\footnote{\textbf{2:12} Un acto simbólico
  de gran aflicción.} \footnote{\textbf{2:12} 2Re 13,14}

\hypertarget{el-regreso-de-eliseo-a-travuxe9s-del-jorduxe1n-a-jericuxf3-elijah-se-ha-ido}{%
\subsection{El regreso de Eliseo a través del Jordán a Jericó; Elijah se
ha
ido}\label{el-regreso-de-eliseo-a-travuxe9s-del-jorduxe1n-a-jericuxf3-elijah-se-ha-ido}}

\bibleverse{13} Entonces Eliseo recogió el manto de Elías que se le
había caído, y regresó y se puso a la orilla del Jordán. \footnote{\textbf{2:13}
  2Re 2,8} \bibleverse{14} Tomó el manto de Elías que se le había caído,
golpeó el agua y gritó: ``¿Dónde está el Señor, el Dios de Elías?'' .
Cuando golpeó el agua, ésta se dividió hacia un lado y hacia el otro y
Eliseo cruzó.

\bibleverse{15} Los hijos de los profetas que vivían en Jericó lo vieron
desde el lado opuesto y gritaron: ``¡El espíritu de Elías descansa ahora
sobre Eliseo!'' Fueron a su encuentro y se postraron en el suelo ante
él. \bibleverse{16} ``Mira'', le dijeron a Eliseo, ``nosotros, tus
siervos, tenemos aquí cincuenta hombres buenos. Por favor, permíteles ir
a buscar a tu amo. Tal vez el Espíritu del Señor se lo ha llevado y lo
ha puesto en una montaña o en un valle en alguna parte''. ``No te
molestes en enviarlos'', respondió Eliseo.

\bibleverse{17} Pero ellos siguieron tratando de persuadirlo hasta que
se sintió demasiado avergonzado para decir que no. ``Adelante,
envíenlos'', les dijo. Así que enviaron a cincuenta hombres, que
buscaron a Elías durante tres días, pero no pudieron encontrarlo.

\bibleverse{18} Cuando regresaron a Eliseo, que se encontraba en Jericó,
éste les dijo: ``¿No les dije que no se molestaran en ir?''

\hypertarget{primera-apariciuxf3n-de-eliseo-el-milagro-del-agua-malsana-en-jericuxf3}{%
\subsection{Primera aparición de Eliseo: El milagro del agua malsana en
Jericó}\label{primera-apariciuxf3n-de-eliseo-el-milagro-del-agua-malsana-en-jericuxf3}}

\bibleverse{19} La gente del pueblo le dijo a Eliseo: ``Mira, señor,
aunque nuestro pueblo tiene una buena ubicación, como puedes ver, el
agua es mala y la tierra es pobre''.

\bibleverse{20} ``Tráiganme un cuenco nuevo y pónganle sal'', respondió
él. Así que se lo trajeron. \bibleverse{21} Entonces Eliseo fue al
manantial, echó la sal en él y dijo: ``Esto es lo que dice el Señor: `He
purificado esta agua. Ya no causará muertes ni abortos'\,''.
\bibleverse{22} El agua de allí sigue siendo pura hasta el día de hoy,
tal como dijo Eliseo que sería.

\hypertarget{eliseo-y-los-chicos-malos-de-betel}{%
\subsection{Eliseo y los chicos malos de
Betel}\label{eliseo-y-los-chicos-malos-de-betel}}

\bibleverse{23} Eliseo siguió desde allí hasta Betel. Cuando iba por el
camino, llegó un grupo de jóvenes del pueblo. Se burlaron de él,
gritando: ``¡Sube, calvo! ¡Sube, calvo!''\footnote{\textbf{2:23} Parece
  que se burlaban de Eliseo diciéndole que debía irse de la misma manera
  que Elías.} \bibleverse{24} Volviéndose, los miró y lanzó una
maldición sobre ellos en nombre del Señor. De repente, dos osos hembras
salieron del bosque y mutilaron a cuarenta y dos de ellos.
\bibleverse{25} Eliseo continuó hasta el monte Carmelo, y desde allí
volvió a Samaria.\footnote{\textbf{2:25} 2Re 4,25}

\hypertarget{rey-joram-de-israel}{%
\subsection{Rey Joram de Israel}\label{rey-joram-de-israel}}

\hypertarget{section-2}{%
\section{3}\label{section-2}}

\bibleverse{1} Joram, hijo de Acab, se convirtió en rey de Israel en el
año dieciocho del reinado de Josafat de Judá. Reinó en Samaria durante
doce años. \footnote{\textbf{3:1} 2Re 1,17} \bibleverse{2} Sus hechos
fueron malos a los ojos del Señor, pero no como los de su padre y su
madre, pues se deshizo de la imagen de piedra de Baal que había hecho su
padre. \footnote{\textbf{3:2} 1Re 16,32} \bibleverse{3} Sin embargo, aún
se aferró a los pecados que Jeroboam, hijo de Nabat, había hecho cometer
a Israel; no los abandonó. \footnote{\textbf{3:3} 1Re 12,30}

\hypertarget{estallido-de-la-guerra-con-los-moabitas-el-pacto-de-joram-con-josafat-marcha-hacia-la-estepa-de-edom}{%
\subsection{Estallido de la guerra con los moabitas; El pacto de Joram
con Josafat; Marcha hacia la estepa de
Edom}\label{estallido-de-la-guerra-con-los-moabitas-el-pacto-de-joram-con-josafat-marcha-hacia-la-estepa-de-edom}}

\bibleverse{4} Mesha, rey de Moab, era criador de ovejas. Solía dar un
tributo al rey de Israel de cien mil corderos y la lana de cien mil
carneros. \bibleverse{5} Pero después de la muerte de Acab, el rey de
Moab se rebeló contra el rey de Israel. \bibleverse{6} Inmediatamente el
rey Joram convocó a todo el ejército israelita y salió de Samaria.
\bibleverse{7} En su camino envió un mensaje a Josafat, rey de Judá,
diciendo: ``El rey de Moab se ha rebelado contra mí. ¿Te unirás a mí
para atacar a Moab?'' Josafat respondió: ``Sí, me uniré a ti. Tú y yo
somos como uno, mis hombres y tus hombres son como uno, y mis caballos y
tus caballos son como uno''. \footnote{\textbf{3:7} 1Re 22,4}

\bibleverse{8} Entonces preguntó: ``¿Por qué camino iremos?''
``Tomaremos el camino que atraviesa el desierto de Edom'', respondió.

\hypertarget{mala-situaciuxf3n-del-ejuxe9rcito-por-falta-de-agua-la-auspiciosa-profecuxeda-de-eliseo}{%
\subsection{Mala situación del ejército por falta de agua; La auspiciosa
profecía de
Eliseo}\label{mala-situaciuxf3n-del-ejuxe9rcito-por-falta-de-agua-la-auspiciosa-profecuxeda-de-eliseo}}

\bibleverse{9} Así que el rey de Israel, el rey de Judá y el rey de Edom
se pusieron en marcha. Después de seguir una ruta indirecta durante
siete días, se quedaron sin agua para su ejército y para sus animales.
\bibleverse{10} ``¿Qué estamos haciendo?'' , se quejó el rey de Israel.
``¡El Señor nos ha traído aquí a tres reyes para entregarnos a los
moabitas!''.

\bibleverse{11} Pero Josafat preguntó: ``¿No hay aquí con nosotros un
profeta del Señor? Consultemos al Señor por medio de él''. Uno de los
oficiales del rey de Israel respondió: ``Eliseo, hijo de Safat, está
aquí. Era el ayudante de Elías''.\footnote{\textbf{3:11} ``Era el
  ayudante de Elías'': Literalmente, ``Solía echar agua en las manos de
  Elías''.}

\bibleverse{12} Josafat aceptó: ``El Señor se comunica por medio de
él''. Así que el rey de Israel, Josafat y el rey de Edom fueron a verlo.

\bibleverse{13} Eliseo le dijo al rey de Israel: ``¿Qué tengo que ver
contigo? Ve con tus propios profetas, los de tu padre y tu madre''. Pero
el rey de Israel le dijo: ``¡No, porque es el Señor quien ha traído aquí
a estos tres reyes para entregarlos a los moabitas!''

\bibleverse{14} Eliseo respondió: ``Vive el Señor Todopoderoso, a quien
sirvo, si no respetara el hecho de que Josafat, rey de Judá, está aquí,
ni siquiera miraría en tu dirección ni te reconocería. \footnote{\textbf{3:14}
  1Re 18,15; Sal 15,4} \bibleverse{15} Ahora tráeme un
músico''.\footnote{\textbf{3:15} ``Músico'': normalmente se refiere a
  alguien que puede tocar un instrumento de cuerda. A menudo se sugiere
  un arpa.} Mientras el músico tocaba, el poder del Señor cayó sobre
Eliseo, \bibleverse{16} y anunció: ``Esto es lo que dice el Señor: Este
valle se llenará de estanques de agua. Porque el Señor dice:
\bibleverse{17} No verás viento, no verás lluvia, pero aun así este
valle se llenará de agua. Beberás tú, y tu ganado, y tus animales.
\bibleverse{18} El Señor considera que esto es algo trivial, y también
te hará victorioso sobre los moabitas. \bibleverse{19} Conquistarás toda
ciudad fortificada y toda ciudad importante. Cortarás todos los árboles
buenos, bloquearás todos los manantiales y arruinarás todos los campos
buenos arrojando piedras sobre ellos''.

\bibleverse{20} Al día siguiente, alrededor de la hora del sacrificio
matutino, el agua fluyó repentinamente desde la dirección de Edom,
llenando de agua todo el campo.

\hypertarget{victoria-de-los-israelitas-mesa-sacrifica-a-su-hijo-primoguxe9nito-lo-que-hace-que-los-israelitas-se-vayan}{%
\subsection{Victoria de los israelitas; Mesa sacrifica a su hijo
primogénito, lo que hace que los israelitas se
vayan}\label{victoria-de-los-israelitas-mesa-sacrifica-a-su-hijo-primoguxe9nito-lo-que-hace-que-los-israelitas-se-vayan}}

\bibleverse{21} Todos los moabitas habían oído que los reyes habían
venido a atacarlos. Así que todos los que podían llevar una espada,
jóvenes y viejos, fueron llamados y fueron a vigilar la frontera.
\bibleverse{22} Pero a la mañana siguiente, cuando se levantaron, el sol
brillaba sobre el agua, y a los moabitas del otro lado les pareció roja
como la sangre. \bibleverse{23} ``¡Esto es sangre!'', dijeron. ``¡Los
reyes y sus ejércitos deben haberse atacado y matado! Moabitas, vamos a
coger el botín''.

\bibleverse{24} Pero cuando los moabitas llegaron al campamento
israelita, los israelitas salieron corriendo y los atacaron, y ellos
huyeron de ellos. Entonces los israelitas invadieron su país y mataron a
los moabitas. \bibleverse{25} Destruyeron las ciudades, y cada soldado
arrojó piedras sobre todo campo bueno hasta cubrirlo. Bloquearon todos
los manantiales y cortaron todos los árboles buenos. Sólo Quir Jaréset
conservaba sus murallas, pero los soldados, usando hondas, la rodearon y
la atacaron también. \bibleverse{26} Cuando el rey de Moab se dio cuenta
de que había perdido la batalla, dirigió a setecientos espadachines para
intentar abrirse paso y atacar al rey de Edom, pero no pudieron hacerlo.
\bibleverse{27} Entonces el rey de Moab tomó a su hijo primogénito, que
estaba destinado a sucederlo, y lo sacrificó como holocausto en el muro
de la ciudad. Un gran enojo se apoderó de los israelitas, así que se
fueron y regresaron a su país.\footnote{\textbf{3:27} Se discute si esto
  fue una gran ira mostrada por los moabitas contra los israelitas, o si
  un acto tan horrible hizo que los israelitas se enfadaran mucho.
  Parece más probable que un sacrificio humano tan espantoso fuera tan
  ofensivo, incluso para los israelitas en su laxo estado espiritual,
  que simplemente dejaron.}

\hypertarget{la-historia-del-cuxe1ntaro-de-aceite-de-la-viuda}{%
\subsection{La historia del cántaro de aceite de la
viuda}\label{la-historia-del-cuxe1ntaro-de-aceite-de-la-viuda}}

\hypertarget{section-3}{%
\section{4}\label{section-3}}

\bibleverse{1} La mujer de uno de los hijos de los profetas se dirigió a
Eliseo: ``Mi marido, tu siervo, ha muerto, y tú sabes que honraba al
Señor. Pero ahora, para pagar sus deudas, su acreedor viene a llevarse a
mis dos hijos como esclavos''.

\bibleverse{2} ``¿Qué puedo hacer para ayudarte?'' , preguntó Eliseo.
``Dime, ¿qué tienes en tu casa?'' ``Yo, tu sierva, no tengo nada en mi
casa, excepto una jarra de aceite de oliva'', respondió ella.
\footnote{\textbf{4:2} 1Re 17,12}

\bibleverse{3} ``Ve y pide prestadas jarras vacías a tus vecinos, todas
las que puedas, no sólo unas pocas'', le dijo Eliseo. \bibleverse{4}
``Luego entra, cierra la puerta detrás de ti y de tus hijos, y comienza
a verter aceite de oliva en todas estas tinajas, colocando las tinajas
llenas a un lado''.

\bibleverse{5} Ella dejó a Eliseo, se fue a su casa y cerró la puerta
detrás de ella y de sus hijos. Le trajeron las tinajas y ella siguió
vertiendo. \bibleverse{6} Cuando todas las tinajas estaban llenas, le
dijo a su hijo: ``Tráeme otra''. Pero él le contestó: ``No queda ninguna
tinaja''. Entonces el aceite de oliva dejó de fluir.

\bibleverse{7} Ella fue a contarle al hombre de Dios lo que había
sucedido, y él le dijo: ``Ve y vende el aceite de oliva y paga tus
deudas, y tú y tus hijos podrán vivir con lo que quede''.

\hypertarget{eliseo-y-sunamitin-eliseo-le-promete-un-hijo-al-sunamitin}{%
\subsection{Eliseo y Sunamitin; Eliseo le promete un hijo al
Sunamitin}\label{eliseo-y-sunamitin-eliseo-le-promete-un-hijo-al-sunamitin}}

\bibleverse{8} Un día, mientras Eliseo iba de paso por Sunem, una mujer
rica que vivía allí lo convenció para que comiera. Después de esa
ocasión, cada vez que pasaba por allí, llegaba allí a comer.
\bibleverse{9} Un día, ella le dijo a su marido: ``Estoy segura de que
este hombre que nos visita regularmente es un santo varón de Dios.
\bibleverse{10} Por favor, hagamos una pequeña habitación en el tejado.
Podemos ponerle una cama, una mesa, una silla y una lámpara. Así podrá
quedarse allí cada vez que nos visite''.

\bibleverse{11} Un día llegó Eliseo, subió a su habitación y se acostó.
\bibleverse{12} Y le dijo a su criado Giezi: ``Pídele a la
sunamita\footnote{\textbf{4:12} Refiriéndose a la mujer que había
  preparado su alojamiento.} que venga aquí''. Entonces Giezi la llamó y
ella vino a ver a Eliseo. \bibleverse{13} Entonces Eliseo le dijo a
Giezi: ``Por favor, dile: `te has tomado muchas molestias por nosotros.
¿Qué podemos hacer ahora por ti? ¿Quieres que hablemos por ti al rey o
al comandante del ejército?'\,'' ``Vivo con mi propia
gente'',\footnote{\textbf{4:13} En otras palabras, tenía todo lo que
  necesitaba.} respondió ella.

\bibleverse{14} Después de que ella se fue,\footnote{\textbf{4:14}
  ``Después de que ella se fue'': implícito.} Eliseo preguntó: ``¿Qué
podemos hacer por ella?''. ``No tiene hijo, y su marido es viejo'',
respondió Giezi.

\bibleverse{15} Eliseo dijo: ``Pídele que vuelva''. Así que Giezi la
llamó, y ella vino de pie junto a la puerta. \bibleverse{16} Entonces
Eliseo le dijo: ``El año que viene, por estas fechas, tendrás un hijo en
brazos''. ``¡No, mi señor!'', respondió ella. ``¡Hombre de Dios, no le
mientas a tu sierva!'' \footnote{\textbf{4:16} Gén 18,10; Gén 18,14}

\bibleverse{17} Pero la mujer quedó efectivamente embarazada, y al año
siguiente, por esas mismas fechas, dio a luz a un hijo, tal como Eliseo
se lo había prometido.

\hypertarget{la-muerte-del-niuxf1o-caminata-de-la-madre-a-elisa}{%
\subsection{La muerte del niño; Caminata de la madre a
Elisa}\label{la-muerte-del-niuxf1o-caminata-de-la-madre-a-elisa}}

\bibleverse{18} El niño creció, pero un día, cuando salió a ver a su
padre, que estaba con los segadores, \bibleverse{19} se quejó a su
padre: ``¡Me duele la cabeza! ¡Me duele la cabeza!'' Su padre dijo a uno
de sus criados: ``Llévenlo de vuelta con su madre''.

\bibleverse{20} El criado lo levantó y lo llevó donde su madre. El niño
se sentó en su regazo hasta el mediodía, y entonces murió.
\bibleverse{21} Entonces ella subió y lo puso en la cama del hombre de
Dios. Luego cerró la puerta y se fue. \bibleverse{22} Llamó a su marido
y le dijo: ``Por favor, envíame uno de los criados y un asno para que
pueda ir corriendo a ver al hombre de Dios y volver''.

\bibleverse{23} ``¿Por qué necesitas ir a verlo hoy?'' , le preguntó él.
``No es luna nueva ni sábado''. ``No te preocupes por eso'', respondió
ella.

\bibleverse{24} Puso la silla de montar en el asno y le dijo a su
criado: ``¡Vamos rápido! No te detengas por mí si no te lo digo yo''.

\hypertarget{elisa-va-a-la-casa-de-la-madre}{%
\subsection{Elisa va a la casa de la
madre}\label{elisa-va-a-la-casa-de-la-madre}}

\bibleverse{25} Así que se puso en marcha y se dirigió al hombre de Dios
que estaba en el monte Carmelo. Cuando vio su camino a lo lejos, el
hombre de Dios le dijo a su siervo Giezi: ``¡Mira! ¡Ahí está la
sunamita! \bibleverse{26} Por favor, corre a su encuentro y pregúntale:
`¿Va todo bien contigo, con tu marido y con tu hijo?'\,'' . ``Todo está
bien'', respondió ella.

\bibleverse{27} Pero cuando llegó hasta el hombre de Dios en la montaña,
se agarró a sus pies. Giezi se acercó para apartarla, pero el hombre de
Dios le dijo: ``Déjala en paz, porque tiene una miseria terrible, pero
el Señor me lo ha ocultado y no me lo ha explicado''.

\bibleverse{28} ``¿Te pedí un hijo, mi señor?'' , preguntó ella. ``¿No
te dije: `No me digas mentiras'?'' \footnote{\textbf{4:28} 2Re 4,16}

\bibleverse{29} Eliseo le dijo a Giezi: ``¡Guarda tu capa en tu
cinturón, toma mi bastón y vete! No saludes a nadie que te encuentres, y
si alguien te saluda, no respondas. Coloca mi bastón en la cara del
muchacho''. \footnote{\textbf{4:29} Luc 10,4}

\bibleverse{30} Pero la madre del muchacho le dijo: ``¡Vive el Señor y
vives tú, no me iré sin ti!''. Así que él se levantó y se fue con ella.

\bibleverse{31} Giezi siguió corriendo y puso el bastón en la cara del
muchacho, pero no hubo sonido ni señal de vida. Entonces Giezi volvió a
reunirse con Eliseo y le dijo: ``El muchacho no ha despertado''.

\bibleverse{32} Cuando Eliseo llegó a la casa, allí estaba el muchacho,
muerto en su cama.

\hypertarget{reanimaciuxf3n-del-niuxf1o}{%
\subsection{Reanimación del niño}\label{reanimaciuxf3n-del-niuxf1o}}

\bibleverse{33} Eliseo entró, cerró la puerta detrás de ambos y oró al
Señor. \footnote{\textbf{4:33} Hech 9,40} \bibleverse{34} Luego se subió
a la cama y se puso encima del muchacho, y puso su boca sobre la boca
del muchacho, sus ojos sobre los ojos del muchacho, sus manos sobre las
manos del muchacho. Mientras se extendía sobre él, el cuerpo del
muchacho se calentaba. \footnote{\textbf{4:34} 1Re 17,21}
\bibleverse{35} Eliseo se levantó, caminó de un lado a otro de la
habitación y luego volvió a la cama y se tendió sobre él de nuevo. El
muchacho estornudó siete veces y luego abrió los ojos. \bibleverse{36}
Eliseo llamó a Giezi y le dijo: ``Pide a la sunamita que venga''. Así lo
hizo. Cuando llegó, Eliseo le dijo: ``Aquí está tu hijo. Puedes
recogerlo''. \footnote{\textbf{4:36} Luc 7,15; Heb 11,35}

\bibleverse{37} Ella entró, se postró a sus pies y se inclinó hasta el
suelo. Luego recogió a su hijo y se fue.

\hypertarget{muerte-comida-venenosa-en-la-olla-y-la-maravillosa-alimentaciuxf3n-de-los-cien}{%
\subsection{Muerte (comida venenosa) en la olla y la maravillosa
alimentación de los
cien}\label{muerte-comida-venenosa-en-la-olla-y-la-maravillosa-alimentaciuxf3n-de-los-cien}}

\bibleverse{38} Cuando Eliseo regresó a Gilgal, había hambre en esa
zona. Los hijos de los profetas estaban sentados a sus pies, y él dijo a
su criado: ``Usa la olla grande y hierve un poco de guiso para los hijos
de los profetas''.

\bibleverse{39} Uno de ellos salió al campo a recoger hierbas. Encontró
una viña silvestre y recogió tantas calabazas silvestres como le cabía
en su manto. Luego regresó y las picó en la olla del guiso. Pero nadie
sabía que era peligroso comerlas.\footnote{\textbf{4:39} ``Era peligroso
  comerlas'': implícito.} \bibleverse{40} Se lo sirvieron a los hombres
para que comieran, pero cuando probaron el guiso gritaron: ``¡Hay muerte
en la olla, hombre de Dios!''. No pudieron comerlo.

\bibleverse{41} Eliseo dijo: ``Trae harina''. La echó en la olla y dijo:
``Sírvesela a la gente para que coma''. No había nada malo para comer en
la olla.

\bibleverse{42} Un hombre de Baal-Salisa se acercó al hombre de Dios con
un saco de primicias, el primer grano del año, junto con veinte panes de
cebada. ``Dáselo a la gente para que coma'', dijo Eliseo.

\bibleverse{43} ``¿Cómo voy a servir sólo veinte panes a cien hombres?''
, preguntó su criado. ``Dáselo a la gente para que coma'', dijo Eliseo,
``porque esto es lo que dice el Señor: `Comerán y aún sobrará'\,''.

\bibleverse{44} Así que les sirvió el pan. Comieron y les sobró, tal
como había dicho el Señor.\footnote{\textbf{4:44} Mat 16,9-10}

\hypertarget{naeman-el-leproso-busca-sanidad-en-samaria}{%
\subsection{Naeman el leproso busca sanidad en
Samaria}\label{naeman-el-leproso-busca-sanidad-en-samaria}}

\hypertarget{section-4}{%
\section{5}\label{section-4}}

\bibleverse{1} Naamán, el comandante del ejército del rey de Aram, era
considerado un gran hombre por su amo y muy respetado, pues a través de
él el Señor había hecho victoriosos a los arameos. Era un poderoso
guerrero, pero tenía lepra. \bibleverse{2} Unos arameos habían hecho una
incursión y habían capturado a una joven de la tierra de Israel. La
habían hecho sierva de la esposa de Naamán. \bibleverse{3} Ella le dijo
a su ama: ``Si mi amo fuera a ver al profeta que vive en Samaria. Estoy
segura de que él podría curarlo de su lepra''.

\bibleverse{4} Entonces Naamán fue a ver a su amo y le explicó lo que
había dicho la muchacha israelita.

\bibleverse{5} ``Puedes ir'', dijo el rey de Aram, ``y enviaré una carta
contigo al rey de Israel''. Así que Naamán partió. Llevó consigo diez
talentos de plata, seis mil siclos de oro y diez conjuntos de ropa.

\bibleverse{6} La carta que llevó al rey de Israel decía: ``Esta carta
acompaña a mi siervo Naamán, enviada a ti para que lo cures de su
lepra''.

\bibleverse{7} Cuando el rey de Israel leyó la carta, se rasgó las
vestiduras presa del pánico y dijo: ``¿Acaso este hombre se cree Dios,
que tiene poder sobre la vida y la muerte, y me envía a curar a un
leproso? Evidentemente, sólo está tratando de inventar una excusa para
atacarme, como cualquiera puede ver''.

\hypertarget{la-curaciuxf3n-de-naeman-a-travuxe9s-de-eliseo}{%
\subsection{La curación de Naeman a través de
Eliseo}\label{la-curaciuxf3n-de-naeman-a-travuxe9s-de-eliseo}}

\bibleverse{8} Pero cuando Eliseo, el hombre de Dios, se enteró de que
el rey de Israel se había rasgado las vestiduras presa del pánico, envió
un mensaje al rey, diciendo ``¿Por qué te has rasgado las vestiduras?
Por favor, envíame a ese hombre, para que se convenza de que hay un
profeta en Israel''.

\bibleverse{9} Así que Naamán llegó con sus caballos y carros y se quedó
esperando a la puerta de la casa de Eliseo. \bibleverse{10} Eliseo le
envió un mensajero diciendo: ``Ve y lávate siete veces en el Jordán.
Entonces tu cuerpo se curará y quedarás limpio''.\footnote{\textbf{5:10}
  ``Limpio'': desde el punto de vista israelita, cualquier persona con
  lepra era impura.}

\bibleverse{11} Pero Naamán se enfadó y se marchó, diciendo: ``Esperaba
que al menos saliera, se quedara allí e invocara el nombre del Señor, su
Dios, y agitara su mano sobre donde está mi lepra y la sanara.
\bibleverse{12} ¿No son los ríos de Damasco, de Abana y de Farfar
mejores que cualquiera de estos arroyos de Israel? ¿No podría haberme
lavado en ellos y haberme curado?'' Así que se dio la vuelta y se marchó
furioso.

\bibleverse{13} Pero los funcionarios de Naamán se acercaron a él y le
dijeron: ``Señor, si el profeta te hubiera dicho que tenías que hacer
algo extraordinario, ¿no lo habrías hecho? ¿Cuánto más fácil es hacer lo
que él dice: `Lávate y quedarás curado'?''

\bibleverse{14} Así que Naamán bajó y se sumergió siete veces en el
Jordán, como le había dicho el hombre de Dios. Su cuerpo quedó curado,
su piel se volvió como la de un bebé, y quedó limpio. \footnote{\textbf{5:14}
  Luc 4,27}

\hypertarget{acciuxf3n-de-gracias-y-alabanza-de-naeman-a-dios}{%
\subsection{Acción de gracias y alabanza de Naeman a
Dios}\label{acciuxf3n-de-gracias-y-alabanza-de-naeman-a-dios}}

\bibleverse{15} Entonces Naamán y todo su séquito volvieron al hombre de
Dios, se presentaron ante él y Naamán anunció: ``Ahora estoy convencido
de que no hay Dios en todo el mundo, excepto en Israel. Por favor,
acepta un regalo de mí, tu siervo''. \footnote{\textbf{5:15} 2Re 5,5}

\bibleverse{16} Pero Eliseo respondió: ``Vive el Señor, al que sirvo,
que no aceptaré nada''. Aunque Naamán trató de persuadirlo para que
aceptara el regalo, éste se negó.

\bibleverse{17} Entonces Naamán dijo: ``Si no lo haces, por favor,
permíteme a mi, tu siervo, llevarme dos cargas de tierra, porque nunca
más traeré un holocausto ni haré un sacrificio a ningún otro dios que no
sea el Señor. \bibleverse{18} Además, que el Señor me perdone por hacer
esto: Cuando mi amo entre en el templo de Rimón para adorar allí, y yo
lo asista, y me incline en el templo de Rimón, que el Señor me perdone
por hacerlo''. \footnote{\textbf{5:18} 2Re 7,2}

\bibleverse{19} ``Ve en paz'', dijo Eliseo, y Naamán se fue. Pero sólo
había recorrido un corto trecho \bibleverse{20} cuando Giezi, el siervo
de Eliseo, el hombre de Dios, pensó para sí: ``¡Mira cómo mi amo ha
dejado ir a Naamán el sirio sin aceptar los regalos que trajo! Vive el
Señor, que correré tras él y le sacaré algo''.

\bibleverse{21} Así que Giezi persiguió a Naamán. Cuando Naamán lo vio
correr tras él, bajó del carro para salir a su encuentro y le preguntó:
``¿Está todo bien?''

\bibleverse{22} ``Todo está bien'', respondió Giezi. ``Mi amo me envió a
decirte: `Acabo de enterarme de que han llegado a verme dos jóvenes de
los hijos de los profetas que viven la región montañosa de Efraín. Por
favor, dales un talento de plata y dos conjuntos de ropa'\,''.

\bibleverse{23} Pero Naamán respondió: ``Por favor, toma dos talentos''.
Insistió en que Giezi los aceptara. Entonces ató dos talentos de plata
en dos bolsas, así como dos juegos de ropa. Se los dio a dos de sus
siervos, que los llevaron para Giezi. \bibleverse{24} Cuando Giezi llegó
a la fortaleza de la colina, tomó los regalos de los sirvientes y los
puso en la casa. Les dijo a los hombres que podían irse, y se fueron.
\bibleverse{25} Cuando Giezi regresó y atendió a su amo, Eliseo le
preguntó: ``¿Dónde has estado, Giezi?'' ``Tu siervo no ha estado en
ninguna parte'', respondió.

\bibleverse{26} Pero Eliseo le dijo: ``¿No te vi en mi mente cuando el
hombre bajó de su carro para recibirte? ¿Es éste el momento de tomar
dinero, ropa, olivares, viñedos, ovejas, bueyes, siervos y siervas?
\bibleverse{27} ``¡Ahora por causa de esto, la lepra de Naamán se te
pegará a ti y a tus descendientes para siempre!'' Y cuando Giezise
marchó, tenía la lepra: se veía blanco como la nieve.

\hypertarget{el-hierro-flotante}{%
\subsection{El hierro flotante}\label{el-hierro-flotante}}

\hypertarget{section-5}{%
\section{6}\label{section-5}}

\bibleverse{1} Entonces los hijos de los profetas le dijeron a Eliseo:
``Mira, el lugar donde nos reunimos contigo es demasiado pequeño para
nosotros. \bibleverse{2} Vayamos mejor al Jordán. Cada uno de nosotros
puede llevar un tronco de vuelta, y podemos construir allí un nuevo
lugar para reunirnos''. ``Adelante'', dijo Eliseo.

\bibleverse{3} Uno de ellos pidió: ``Por favor, ven con tus
sirvientes''. ``Iré'', respondió él.

\bibleverse{4} Así que fue con ellos. Cuando llegaron al Jordán,
comenzaron a cortar árboles. \bibleverse{5} Pero cuando uno de ellos
estaba cortando un árbol, la cabeza del hacha de hierro cayó en el agua.
``¡Oh, no! ¡Mi amo, esta era un hacha que me habían prestado!'', gritó.

\bibleverse{6} ``¿Dónde se ha caído?'' , preguntó el hombre de Dios. Y
cuando le mostró el lugar, el hombre de Dios cortó un palo, lo arrojó
allí y la cabeza del hacha, que era de hierro, apareció flotando.
\bibleverse{7} ``Ve y recógela'', le dijo Eliseo al hombre. Entonces
éste extendió la mano y la recogió.

\hypertarget{la-emboscada-traicionada-varias-veces}{%
\subsection{La emboscada traicionada varias
veces}\label{la-emboscada-traicionada-varias-veces}}

\bibleverse{8} También aconteció que el rey arameo estaba en guerra con
Israel. Después de consultar con sus oficiales, dijo: ``Estableceré mi
campamento en este lugar''.

\bibleverse{9} Entonces el hombre de Dios envió una advertencia al rey
de Israel: ``Ten cuidado si te acercas a este lugar, porque los arameos
van a estar allí''. \bibleverse{10} Así que el rey de Israel envió una
advertencia al lugar que el hombre de Dios había indicado. Eliseo
advirtió repetidamente al rey, para que estuviera alerta en esos
lugares. \bibleverse{11} Esto hizo enojar mucho al rey arameo. Convocó a
sus oficiales, exigiendo una respuesta: ``Díganme, ¿quién de nosotros
está del lado del rey de Israel?''

\bibleverse{12} ``No es ninguno de nosotros, mi señor el rey'',
respondió uno de sus oficiales. ``Es Eliseo, el profeta que vive en
Israel; él le dice al rey de Israel hasta lo que tú dices en tu
habitación''.

\hypertarget{el-cegamiento-de-los-sirios}{%
\subsection{El cegamiento de los
sirios}\label{el-cegamiento-de-los-sirios}}

\bibleverse{13} Así que el rey dio la orden: ``Ve y averigua dónde está
para que pueda enviar soldados a capturarlo''. Ellos le dijeron:
``Eliseo está en Dotán''.

\bibleverse{14} Así que envió caballos, carros y un gran ejército.
Llegaron de noche y rodearon la ciudad. \bibleverse{15} Por la mañana,
cuando el siervo del hombre de Dios se levantó, salió y vio que un
ejército con caballos y carros había rodeado la ciudad. ``Señor mío,
¿qué vamos a hacer?'' , le preguntó a Eliseo.

\bibleverse{16} Eliseo le contestó: ``¡No tengas miedo, porque son
muchos más los que están con nosotros que los que están con ellos!''
\bibleverse{17} Eliseo oró diciendo: ``Señor, por favor abre sus ojos
para que pueda ver''. El Señor abrió los ojos del siervo, y cuando miró
vio las colinas llenas de caballos y carros de fuego alrededor de
Eliseo. \bibleverse{18} Mientras el ejército\footnote{\textbf{6:18}
  Refiriéndose a los arameos.} descendió sobre él, Eliseo rogó al Señor:
``Por favor, hiere a esta gente con ceguera''. Así que los golpeó con
ceguera, como Eliseo había pedido. \footnote{\textbf{6:18} Gén 19,11}

\bibleverse{19} Entonces Eliseo fue y les dijo: ``Este no es el camino
correcto, y este no es el pueblo correcto. Síganme, y los llevaré hasta
el hombre que buscan''. Los condujo a Samaria. \bibleverse{20} Cuando
entraron en Samaria, Eliseo oró: ``Señor, abre los ojos de estos hombres
para que puedan ver''. El Señor les abrió los ojos, y ellos miraron a su
alrededor y vieron que estaban en Samaria.

\bibleverse{21} Cuando el rey de Israel los vio, le preguntó a Eliseo:
``Padre mío, ¿los mato? ¿Debo matarlos?''

\bibleverse{22} ``¡No, no los mates!'', respondió. ``¿Matarías a los
prisioneros que capturaras con tu propia espada o arco? Dales comida y
agua para que coman y beban, y luego deja que vuelvan con su amo''.

\bibleverse{23} Así que el rey mandó preparar un gran banquete para
ellos, y una vez que terminaron de comer y beber, los envió de vuelta
con su amo. Los invasores arameos no volvieron a entrar en la tierra de
Israel.

\hypertarget{asedio-de-samaria-y-hambre}{%
\subsection{Asedio de Samaria y
hambre}\label{asedio-de-samaria-y-hambre}}

\bibleverse{24} Algún tiempo después de esto, Ben Adad, rey de Aram,
convocó a todo su ejército y fue a sitiar Samaria. \bibleverse{25} Así
que hubo una gran hambruna en Samaria. De hecho, el asedio duró tanto
que una cabeza de burro costaba ochenta siclos de plata, y un cuarto de
litro de cab de estiércol de paloma\footnote{\textbf{6:25} ``Estiércol
  de paloma'': Algunos creen que se refería a una especie de verdura
  silvestre. Un cab equivale a unos 1,2 litros.} costó cinco siclos de
plata. \bibleverse{26} Cuando el rey de Israel pasaba por la muralla de
la ciudad, una mujer le gritó: ``¡Ayúdame, mi señor el rey!''

\bibleverse{27} ``Si el Señor no te ayuda, ¿por qué crees que yo puedo
ayudarte?'' , respondió el rey. ``No tengo grano de la era, ni vino del
lagar''. \bibleverse{28} Pero entonces le preguntó: ``¿Cuál es el
problema?'' ``Cierta mujer me dijo: `Entrega a tu hijo y lo comeremos
hoy, y mañana nos comeremos a mi hijo'\,'', respondió ella.

\bibleverse{29} ``Así que cocinamos a mi hijo y nos lo comimos. Al día
siguiente le dije: `Entrega a tu hijo para que nos lo comamos', pero
ella escondió a su hijo''. \footnote{\textbf{6:29} Deut 28,53}

\bibleverse{30} Cuando el rey oyó lo que decía la mujer, se rasgó las
vestiduras. Al pasar por la muralla, la gente vio que llevaba un sayo
debajo de la ropa junto a su piel. \bibleverse{31} ``¡Que Dios me
castigue muy severamente si la cabeza de Eliseo, hijo de Safat, queda
hoy sobre sus hombros!'', declaró.

\hypertarget{la-promesa-de-suerte-de-eliseo-para-la-ciudad}{%
\subsection{La promesa de suerte de Eliseo para la
ciudad}\label{la-promesa-de-suerte-de-eliseo-para-la-ciudad}}

\bibleverse{32} Eliseo estaba sentado en su casa con los ancianos. El
rey había enviado un mensajero por delante, pero antes de que llegara,
Eliseo dijo a los ancianos: ``¿Ven cómo este asesino envía a alguien a
cortarme la cabeza? Así que, en cuanto llegue el mensajero, cierren la
puerta y no lo dejen entrar. ¿No es el sonido de los pasos de su amo
siguiéndolo?'' .

\bibleverse{33} Mientras Eliseo seguía hablando con ellos, llegó el
mensajero. El rey dijo: ``Este desastre viene del Señor. ¿Por qué debo
esperar más al Señor?'' \footnote{\textbf{6:33} El rey creía que el
  Señor había causado los problemas y como no parecía haber ninguna
  acción del Señor para resolverlos, el rey estaba tomando el asunto en
  sus propias manos. Estaba rechazando a Dios, y pretendía vengarse del
  profeta de Dios, Eliseo.}

\hypertarget{section-6}{%
\section{7}\label{section-6}}

\bibleverse{1} Entonces Eliseo respondió: ``Escucha el mensaje del
Señor. Esto es lo que dice el Señor: Mañana a esta hora, un seah de la
mejor harina se venderá por un siclo, y dos seahs de cebada se venderán
por un siclo en la puerta de Samaria''.\footnote{\textbf{7:1} En otras
  palabras, los productos alimentarios básicos se venderían a precios
  bajos.} \footnote{\textbf{7:1} 2Re 7,16}

\bibleverse{2} El oficial que era ayudante del rey le dijo al hombre de
Dios: ``¡Aunque el Señor abriera ventanas en el cielo no podría suceder
lo que tú dices!''. Eliseo respondió: ``Lo verás con tus propios ojos,
pero no podrás comer nada de eso''. \footnote{\textbf{7:2} 2Re 7,17; 2Re
  5,18}

\hypertarget{experiencias-de-los-cuatro-leprosos-en-el-campamento-sirio}{%
\subsection{Experiencias de los cuatro leprosos en el campamento
sirio}\label{experiencias-de-los-cuatro-leprosos-en-el-campamento-sirio}}

\bibleverse{3} Acontecióque había cuatro hombres con lepra a la entrada
de la puerta de la ciudad. Y se decían unos a otros: ``¿Qué ganaremos
con quedarnos aquí sentados hasta morir? \footnote{\textbf{7:3} Lev
  13,46} \bibleverse{4} Si decimos: `Vamos a la ciudad', moriremos a
causa del hambre que hay allí; pero si seguimos sentados aquí, también
moriremos. Así que vamos, vayamos al campamento de los arameos y
entreguémonos a ellos. Si nos dejan vivir, viviremos; si nos matan,
moriremos''. \footnote{\textbf{7:4} Est 4,16}

\bibleverse{5} Así que cuando estaba oscureciendo se pusieron en marcha
y se dirigieron al campamento de los arameos. Pero cuando llegaron a la
entradadel campamento, no había nadie. \bibleverse{6} Esto era porque el
Señor había hecho que los arameos oyeran el ruido de carros, caballos,
como si fuera un gran ejército que se acercaba. Así que los arameos
dijeron: ``Seguro que el rey de Israel ha contratado a los reyes de los
hititas y de los egipcios para que vengan a atacarnos''. \footnote{\textbf{7:6}
  2Re 19,7} \bibleverse{7} Entonces saltaron y huyeron por la noche,
dejando atrás sus tiendas, sus caballos y sus asnos. De hecho, el
campamento quedó tal como estaba cuando huyeron para salvar sus vidas.
\bibleverse{8} Cuando los leprosos llegaron a las afueras del
campamento, entraron en una tienda, comieron y bebieron. Luego tomaron
la plata, el oro y la ropa, y lo escondieron todo. Después volvieron a
otra tienda, tomaron algunas cosas de allí y las escondieron.
\bibleverse{9} Entonces se dijeron unos a otros: ``No está bien lo que
estamos haciendo. Este es un día de buenas noticias, y si nos callamos y
esperamos hasta que amanezca, seguro nos castigarán. Así que vayamos
enseguida a dar aviso en el palacio del rey''.

\hypertarget{reporta-los-leprosos-en-la-ciudad-y-sus-efectos}{%
\subsection{Reporta los leprosos en la ciudad y sus
efectos}\label{reporta-los-leprosos-en-la-ciudad-y-sus-efectos}}

\bibleverse{10} Entonces fueron y llamaron a los guardianes de la
ciudad: ``Pasamos por el campamento arameo y no había nadie, ¡ni un
ruido de nadie! Sólo había caballos y asnos atados, y las tiendas las
dejaron tal como estaban''.

\bibleverse{11} Los porteros dieron la noticia a gritos, y los informes
llegaron al palacio real.

\bibleverse{12} Entonces el rey se levantó por la noche y dijo a sus
oficiales: ``Déjenme decirles el truco que los arameos están tratando de
hacernos. Saben que nos estamos muriendo de hambre, así que han
abandonado el campamento y se han escondido en el campo, pensando que
cuando salgan de la ciudad, los agarraremos vivos y podremos entrar en
ella''.

\bibleverse{13} Uno de sus oficiales sugirió: ``Que algunos hombres
tomen cinco de los caballos que quedan en la ciudad. Lo que les ocurra a
ellos será lo mismo que a todos los israelitas que quedan aquí. De
cierto, todos los israelitas de aquí están condenados. Enviémoslos a
averiguar qué sucede''.

\bibleverse{14} Así que prepararon dos carros con sus caballos, y el rey
los envió al campamento arameo, diciéndoles: ``Vayan y echen un
vistazo''.

\hypertarget{la-profecuxeda-de-eliseo-se-hace-realidad}{%
\subsection{La profecía de Eliseo se hace
realidad}\label{la-profecuxeda-de-eliseo-se-hace-realidad}}

\bibleverse{15} Fueron tras ellos hasta el Jordán, y todo el camino
estaba lleno de ropa y objetos que los arameos habían tirado al huir.
Los mensajeros regresaron e informaron al rey. \bibleverse{16} Entonces
el pueblo salió y saqueó el campamento de los arameos. Así, un seah de
la mejor harina se vendió por un siclo, y dos seahs de cebada se
vendieron por un siclo, tal como el Señor lo había predicho.
\bibleverse{17} El rey había puesto al oficial que era su asistente a
cargo de la puerta. En su afán, el pueblo lo pisoteó en la puerta y
murió, tal como había dicho el hombre de Dios cuando el rey lo visitó.
\footnote{\textbf{7:17} 2Re 7,2}

\bibleverse{18} También se cumplió lo que el hombre de Dios le había
dicho al rey cuando le dijo: ``Mañana a esta hora un seah de la mejor
harina se venderá por un siclo, y dos seahs de cebada se venderán por un
siclo en la puerta de Samaria''. \bibleverse{19} También el oficial que
era ayudante del rey le había dicho al hombre de Dios: ``¡Mira, aunque
el Señor abriera ventanas en el cielo no podría suceder lo que tú
dices!'' Y Eliseo había respondido: ``Lo verás con tus propios ojos,
pero no podrás comer nada de eso''. \bibleverse{20} Esto es lo que le
sucedió. La gente lo pisoteó en la puerta y murió.

\hypertarget{elisa-y-la-sunamita}{%
\subsection{Elisa y la Sunamita}\label{elisa-y-la-sunamita}}

\hypertarget{section-7}{%
\section{8}\label{section-7}}

\bibleverse{1} Eliseo le dijo a la mujer cuyo hijo había resucitado:
``Tú y tu familia tienen que prepararsu equipajee irse, y tendrán que
vivir donde puedan, en otro lugar como extranjeros. Porque el Señor ha
anunciado que vendrá una hambruna a la tierra y que durará siete años''.

\bibleverse{2} La mujer preparó su equipaje e hizo lo que el hombre de
Dios le había dicho. Ella y su familia se fueron a vivir como
extranjeros durante siete años en el país de los filisteos.
\bibleverse{3} Cuando pasaron los siete años, ella regresó del país de
los filisteos y fue a ver al rey para pedirle que le devolviera su casa
y sus tierras. \bibleverse{4} El rey estaba hablando con Giezi, el
siervo del hombre de Dios, y le pidió: ``Por favor, cuéntame todas las
cosas maravillosas que hizo Eliseo''. \bibleverse{5} Sucedió que justo
en ese momento Giezi le estaba contando al rey la historia de cómo
Eliseo había hecho revivir al niño muerto cuando su madre llegó para
hacer su petición al rey de que le devolviera su casa y sus tierras.
``Mi señor el rey'', dijo Giezi, ``esta es la mujer, y este es su hijo,
el que Eliseo hizo vivir nuevamente''.

\bibleverse{6} El rey le preguntó a la mujer y ella le explicó toda la
historia. Entonces el rey le dio órdenes a un funcionario, diciendo:
``Asegúrate de que se le devuelva todo lo que le pertenecía, junto con
todas las ganancias de sus tierras desde el día en que salió del país
hasta ahora''.

\hypertarget{eliseo-en-damasco-preguntado-por-hazael-sobre-el-rey-ben-adad-enfermo}{%
\subsection{Eliseo en Damasco preguntado por Hazael sobre el rey
Ben-adad
enfermo}\label{eliseo-en-damasco-preguntado-por-hazael-sobre-el-rey-ben-adad-enfermo}}

\bibleverse{7} Eliseo fue a Damasco cuando Ben Adad, rey de Aram, estaba
enfermo. Y le informaron al rey: ``El hombre de Dios ha llegado a la
ciudad''.

\bibleverse{8} Entonces el rey ordenó a Jazael: ``Lleva un regalo y ve
al encuentro del hombre de Dios. Pídele que le pregunte al Señor: `¿Me
recuperaré de esta enfermedad?'\,'' .

\bibleverse{9} Así que Jazael fue al encuentro de Eliseo. Llevó consigo
un regalo de todas las mejores cosas de Damasco: cuarenta camellos
cargados de mercancías. Entró, se puso delante de él y le dijo: ``Tu
hijo Ben Adad, rey de Aram, me ha enviado a preguntarte: `¿Me recuperaré
de esta enfermedad?'\,''

\hypertarget{la-apertura-de-eliseo-a-hazael-el-asesinato-de-benhadad-hasael-asumiuxf3-el-cargo}{%
\subsection{La apertura de Eliseo a Hazael; El asesinato de Benhadad;
Hasael asumió el
cargo}\label{la-apertura-de-eliseo-a-hazael-el-asesinato-de-benhadad-hasael-asumiuxf3-el-cargo}}

\bibleverse{10} ``Ve y dile: `De seguro te recuperarás'. Pero el Señor
me ha mostrado que definitivamente vas a morir'', respondió
Eliseo.\footnote{\textbf{8:10} El profeta no le dice a Jazael que
  mienta. La pregunta del rey se refería a su enfermedad. La muerte del
  rey no fue causada por su enfermedad, sino por Jazael que lo asesinó.}
\bibleverse{11} Eliseo lo miró fijamente durante mucho tiempo hasta que
Jazael se sintió incómodo. Entonces el hombre de Dios comenzó a llorar.
\footnote{\textbf{8:11} Luc 19,41}

\bibleverse{12} ``¿Por qué lloras, mi señor?'' , preguntó Jazael.
``Porque sé las cosas malas que les vas a hacer a los israelitas'', le
respondió Eliseo. ``Prenderás fuego a sus fortalezas, matarás a sus
jóvenes con la espada, harás pedazos a sus pequeños y desgarrarás a sus
mujeres embarazadas''. \footnote{\textbf{8:12} 2Re 10,32}

\bibleverse{13} ``Pero, ¿cómo podría lograr algo así alguien como yo,
que no soy más que un `perro'?'' le preguntó Jazael. ``El Señor me ha
mostrado que vas a ser el rey de Aram'', respondió Eliseo. \footnote{\textbf{8:13}
  1Sam 24,15; 1Re 19,15}

\bibleverse{14} Entonces Jazael dejó a Eliseo y fue a ver a su amo,
quien le preguntó: ``¿Qué te dijo Eliseo?'' . Jazaelle respondió: ``Me
dijo que de seguro te recuperarías''.

\bibleverse{15} Pero al día siguiente Jazael tomó la cubierta de la
cama, la empapó en agua y la puso sobre el rostro del rey hasta que éste
murió. Entonces Jazael lo relevó como rey.

\hypertarget{joram-y-ocozuxedas-su-hijo-reyes-de-juduxe1}{%
\subsection{Joram y Ocozías su hijo, reyes de
Judá}\label{joram-y-ocozuxedas-su-hijo-reyes-de-juduxe1}}

\bibleverse{16} Jehoram, hijo de Josafat, comenzó su reinado como rey de
Judá en el quinto año del reinado de Joram, hijo de Acab, rey de Israel,
mientras Josafat aún era rey de Judá.\footnote{\textbf{8:16} Esto es
  claramente una corregencia.} \bibleverse{17} Tenía treinta y dos años
cuando llegó a ser rey, y reinó en Jerusalén durante ocho años.
\bibleverse{18} Jehoram siguió los caminos de los reyes de Israel, tal
como lo había hecho la casa de Acab, pues se casó con una hija de Acaby
sus hechos fueron malos a los ojos del Señor. \footnote{\textbf{8:18}
  2Re 8,26} \bibleverse{19} Pero por amor a David, su siervo, el Señor
no quiso destruir a Judá, ya que le había prometido que siempre habría
un gobernante de su descendencia, como una lámpara para
siempre.\footnote{\textbf{8:19} Véase, por ejemplo, 1 Reyes 11:36.}
\footnote{\textbf{8:19} 2Sam 7,11-16; 1Re 11,36}

\hypertarget{la-cauxedda-de-los-edomitas-y-la-muerte-de-joram}{%
\subsection{La caída de los edomitas y la muerte de
Joram}\label{la-cauxedda-de-los-edomitas-y-la-muerte-de-joram}}

\bibleverse{20} Durante el tiempo en que Jehoram fue rey, Edom se rebeló
contra el gobierno de Judá y eligió a su propio rey. \bibleverse{21} Así
que Jehoram se dirigió a Zair con todos sus carros. Los edomitas lo
rodearon a él y a sus comandantes de carros, pero él actuó y atacó de
noche. Pero su ejército huyó de vuelta a sus casas. \bibleverse{22} Como
resultado, Edom se rebeló contra el gobierno de Judá hasta el día de
hoy. Al mismo tiempo, Libna también decidió rebelarse. \bibleverse{23}
El resto de lo que sucedió en el reinado de Jehoram y todo lo que hizo
está registrado en el Libro de las Crónicas de los Reyes de Judá.
\bibleverse{24} Jehoram murió y fue enterrado con sus antepasados en la
Ciudad de David. Su hijo Ocozías lo sucedió como rey.

\hypertarget{ochuxf4zuxedas-de-juduxe1-guerra-con-hazael}{%
\subsection{Ochôzías de Judá; Guerra con
Hazael}\label{ochuxf4zuxedas-de-juduxe1-guerra-con-hazael}}

\bibleverse{25} Ocozías, hijo de Jehoram, llegó a ser rey de Judá en el
duodécimo año del reinado de Joram, hijo de Acab, rey de Israel.
\bibleverse{26} Ocozías tenía veintidós años cuando llegó a ser rey, y
reinó en Jerusalén durante un año. Su madre se llamaba Atalía, nieta de
Omri, rey de Israel. \footnote{\textbf{8:26} 2Re 8,18; 2Re 11,1}
\bibleverse{27} Ocozías también siguió los malos caminos de la familia
de Acab, e hizo el mal a los ojos del Señor, como lo había hecho la
familia de Acab, pues estaba emparentado con ellos por matrimonio.

\bibleverse{28} Ocozías fue con Joram, hijo de Acab, a luchar contra
Jazael, rey de Aram, en Ramot de Galaad. Los arameos hirieron a Joram,
\bibleverse{29} y éste regresó a Jezrel para recuperarse de las heridas
que había recibido en Ramá luchando contra Jazael, rey de Aram. Ocozías,
hijo de Jehoram, rey de Judá, fue a Jezrel a visitar a Joram, hijo de
Acab, porque Joram estaba herido.

\hypertarget{jehuxfa-ungiuxf3-rey-por-instigaciuxf3n-de-eliseo}{%
\subsection{Jehú ungió rey por instigación de
Eliseo}\label{jehuxfa-ungiuxf3-rey-por-instigaciuxf3n-de-eliseo}}

\hypertarget{section-8}{%
\section{9}\label{section-8}}

\bibleverse{1} El profeta Eliseo llamó a uno de los hijos de los
profetas y le dijo: ``Pon tu capa en tu cinturón, toma este frasco de
aceite de oliva y ve a Ramot de Galaad. \bibleverse{2} Cuando llegues
allí, busca a Jehú, hijo de Josafat, hijo de Nimsí. Entra, aléjalo de
sus compañeros y llévalo a una habitación interior. Haz que deje a sus
amigos, llévalo a una habitación privada, \bibleverse{3} toma el frasco
de aceite de oliva y viértelo sobre su cabeza. Dile: `Esto es lo que
dice el Señor: te unjo como rey de Israel'. Luego abre la puerta y sal
corriendo. No te quedes esperando''. \footnote{\textbf{9:3} 1Re 19,16}

\bibleverse{4} Así que el joven profeta fue a Ramot de Galaad.
\bibleverse{5} Cuando llegó, vio a los comandantes del ejército
sentados. ``Tengo un mensaje para usted, comandante'', dijo. ``¿Para
cuál de nosotros?'' preguntó Jehú. ``Para usted, comandante'',
respondió.

\bibleverse{6} Jehú se levantó y entró, donde el joven profeta le echó
el aceite de oliva en la cabeza y le anunció: ``Esto es lo que dice el
Señor, el Dios de Israel: `Te unjo como rey del pueblo del Señor,
Israel. \bibleverse{7} Vas a destruir a la familia de Acab, tu señor.
Vengarás la sangre de mis profetas y la de todos los siervos del Señor
asesinados por Jezabel. \bibleverse{8} Toda la casa de Acab será
erradicada: destruiré a todos los varones de la familia de Acab en
Israel, tanto esclavos como libres. \footnote{\textbf{9:8} 1Re 14,10}
\bibleverse{9} Haré que la casa de Acab sea como la casa de Jeroboam,
hijo de Nabat, y como la casa de Basá, hijo de Ahías. \footnote{\textbf{9:9}
  1Re 15,29; 1Re 16,3; 1Re 16,11} \bibleverse{10} Los perros se comerán
a Jezabel, la mujer de Acab, en el solar de Jezrel, y nadie la
enterrará'\,''. Entonces el joven profeta abrió la puerta y salió
corriendo.

\hypertarget{jehuxfa-reconocido-como-rey-por-los-luxedderes-militares}{%
\subsection{Jehú reconocido como rey por los líderes
militares}\label{jehuxfa-reconocido-como-rey-por-los-luxedderes-militares}}

\bibleverse{11} Cuando Jehú volvió a salir con los otros oficiales de su
amo, éstos le preguntaron: ``¿Está todo bien? ¿Por qué ha venido a verte
este loco?'' ``Ya lo conocen. Saben que habla mucho'', respondió.

\bibleverse{12} ``¡Mentiroso!'', le dijeron. ``Por favor, dinos qué
pasa''. ``Bueno, me habló de esto y de lo otro, y me dijo: `Esto es lo
que dice el Señor: te unjo como rey sobre Israel'\,''.

\bibleverse{13} Rápidamente tomaron sus mantos y los extendieron sobre
los escalones desnudos. Tocaron la trompeta y gritaron: ``¡Jehú es
rey!''. \footnote{\textbf{9:13} Mat 21,7}

\bibleverse{14} Entonces Jehú, hijo de Josafat, hijo de Nimsí, tramó una
rebelión contra Joram. Joram y todo el ejército israelita habían estado
defendiendo Ramot-Galad contra Jazael, rey de Aram. \bibleverse{15} Pero
Joram había regresado a Jezrel para recuperarse de las heridas que había
recibido luchando contra Jazael, rey de Aram. Entonces Jehú dijo: ``Si
ustedes, los comandantes, quieren hacerme rey, no dejen que nadie salga
de la ciudad y vayan a anunciarlo en Jezrel''.

\hypertarget{jehuxfa-mata-a-joram-y-ochuxf4zuxedas}{%
\subsection{Jehú mata a Joram y
Ochôzías}\label{jehuxfa-mata-a-joram-y-ochuxf4zuxedas}}

\bibleverse{16} Jehú subió a su carro y se dirigió a Jezrel, pues Joram
se estaba recuperando allí. Ocozías, rey de Judá, también estaba allí
porque había venido a visitar a Joram. \footnote{\textbf{9:16} 2Re 8,29}
\bibleverse{17} El vigilante de la torre de Jezrel vio que se acercaban
los soldados de Jehú, y gritó: ``¡Veo que se acerca un soldado!''
``Elige un jinete'', ordenó Joram. ``Envíalo a recibirlos y pregúntales:
`¿Vienen en son de paz?'\,'' .

\bibleverse{18} Así que un jinete salió al encuentro de Jehú y le dijo:
``El rey te manda a preguntar: `¿Vienes en paz?'\,'' ``¿Qué les importa
la paz a ustedes?'' Respondió Jehú. ``Date la vuelta y sígueme''. El
vigilante informó: ``El mensajero ha llegado hasta ellos, pero no
regresa''.

\bibleverse{19} El rey envió a un segundo jinete. Se acercó a ellos y
les dijo: ``El rey te manda a preguntar: `¿Vienes en son de paz?'\,'' .
``¿Qué les importa la paz a ustedes?'' Respondió Jehú. ``Date la vuelta
y sígueme''.

\bibleverse{20} El vigilante informó: ``El mensajero ha llegado hasta
ellos, pero no regresa. ¡Su forma de conducir\footnote{\textbf{9:20} Se
  refiere a la conducción del carro.} hace que parezca que es Jehú, hijo
de Nimsi, pues conduce como un loco!''

\bibleverse{21} ``¡Preparen mi carro!'' gritó Joram, y ya tenían su
carro preparado. Entonces Joram, rey de Israel, y Ocozías, rey de Judá,
partieron en sus carros por separado, y se encontraron con Jehú en el
terreno que antes era propiedad de Nabot de Jezrel.

\bibleverse{22} Cuando Joram vio a Jehú, le preguntó: ``¿Vienes en son
de paz, Jehú?'' ``¿Qué paz puede haber con tanta prostitución\footnote{\textbf{9:22}
  ``Prostitución'': en el sentido espiritual de ir tras los dioses
  paganos, y también en el sentido literal, ya que el culto pagano a
  menudo implicaba sexo con prostitutas del templo, tanto masculinas
  como femeninas.} y brujería causada por tu madre Jezabel?'' respondió
Jehú.

\bibleverse{23} Entonces Joram se dio la vuelta y se alejó corriendo,
gritándole a Ocozías: ``¡Es una traición, Ocozías!''

\bibleverse{24} Jehú tomó su arco y le disparó a Joram entre los
hombros. La flecha le atravesó el corazón y cayó muerto en su carro.
\bibleverse{25} Jehú dijo a Bidcar, su oficial: ``Recógelo y arrójalo al
campo de Nabot de Jezrel. Recuerda que cuando tú y yo cabalgábamos
juntos detrás de su padre Acab, el Señor hizo esta profecía contra él:
\footnote{\textbf{9:25} 1Re 21,19} \bibleverse{26} `Así como ayer vi la
sangre de Nabot y la sangre de sus hijos, dice el Señor, así también te
pagaré en este mismo terreno, dice el Señor'. Ahora, pues, siguiendo lo
que ha dicho el Señor, recógelo y arrójalo sobre la parcela''.

\bibleverse{27} Cuando Ocozías, rey de Judá, vio lo sucedido, corrió por
el camino hacia Bet-hagán. Pero Jehú lo persiguió, gritando:
``¡Dispárenle también a él!''. Así que fusilaron a Ocozías en su carro
en el camino hacia Gur, cerca de Ibleam. Logró escapar a Meguido, pero
allí murió. \bibleverse{28} Sus servidores lo llevaron en carro a
Jerusalén y lo enterraron con sus antepasados en su tumba en la Ciudad
de David. \footnote{\textbf{9:28} 2Re 14,2; 2Re 23,30} \bibleverse{29}
Ocozías llegó a ser rey de Judá en el undécimo año del reinado de Joram,
hijo de Acab.

\hypertarget{el-espantoso-final-de-jezabel}{%
\subsection{El espantoso final de
Jezabel}\label{el-espantoso-final-de-jezabel}}

\bibleverse{30} Cuando Jezabel se enteró de que Jehú había llegado a
Jezrel, se puso su sombra de ojos negra, se puso joyas en el pelo y
observó desde una ventana. \bibleverse{31} Cuando Jehú entró por la
puerta, ella gritó: ``¿Vienes en son de paz? ¿O eres como Zimri, un
asesino de tu señor?''

\bibleverse{32} Jehú miró hacia la ventana y gritó: ``¿Quién está de mi
lado? ¿Alguien?'' Dos o tres eunucos lo miraron.

\bibleverse{33} ``¡Tírenla al suelo!'', gritó. Y ellos la arrojaron al
suelo. Su sangre salpicó la pared y los caballos, que la pisotearon.

\bibleverse{34} Jehú entró, comió y bebió. Luego dijo: ``Por favor,
traten a esa mujer maldita y entiérrenla, porque era la hija de un
rey''.

\bibleverse{35} Salieron a enterrarla pero sólo encontraron su cráneo,
sus pies y sus manos. \bibleverse{36} Volvieron y se lo contaron a Jehú,
quien dijo: ``Esto es lo que ha dicho el Señor por medio de su siervo
Elías tisbita: 'Los perros comerán la carne de Jezabel en la parcela de
Jezrel. \footnote{\textbf{9:36} 2Re 9,10; 1Re 21,23}

\bibleverse{37} Su cuerpo yacerá como estiércol en el campo, en la
parcela de Jezrel, para que nadie pueda siquiera decir: Aquí es donde
está enterrada Jezabel'\,''.

\hypertarget{jehuxfa-asesinuxf3-a-los-setenta-pruxedncipes-reales-y-exterminuxf3-a-toda-la-casa-de-acab}{%
\subsection{Jehú asesinó a los setenta príncipes reales y exterminó a
toda la casa de
Acab}\label{jehuxfa-asesinuxf3-a-los-setenta-pruxedncipes-reales-y-exterminuxf3-a-toda-la-casa-de-acab}}

\hypertarget{section-9}{%
\section{10}\label{section-9}}

\bibleverse{1} Había setenta hijos de la casa de Acab viviendo en
Samaria. Entonces Jehú escribió cartas y las envió a los funcionarios de
Samaria,\footnote{\textbf{10:1} ``Samaria''. El texto hebreo identifica
  ``Jezrel'', pero Jehú ya estaba allí.} a los ancianos y a los
guardianes de los hijos de Acab, diciendo: \bibleverse{2} ``Puesto que
los hijos de tu amo están contigo, y tienes a tu disposición carros,
caballos, una ciudad fortificada y armas, cuando recibas esta carta,
\bibleverse{3} elige al mejor y más apropiado hijo de tu amo, colócalo
en el trono de su padre y lucha por la casa de tu amo''.

\bibleverse{4} Pero ellos se asustaron mucho y se dijeron: ``Si dos
reyes no pudieron derrotarlo, ¿cómo podríamos nosotros?'' \bibleverse{5}
Así que los jefes del palacio y de la ciudad, los ancianos y los
guardianes enviaron un mensaje a Jehú: ``Somos tus siervos y haremos
todo lo que nos digas. No vamos a hacer rey a nadie. Haz lo que te
parezca mejor''.

\bibleverse{6} Entonces Jehú les escribió una segunda carta en la que
les decía: ``Si están de mi parte, y si van a obedecer lo que yo diga,
tráiganme mañana a esta hora a Jezrel las cabezas de los hijos de su
señor''. Los setenta hijos del rey estaban siendo criados por los
principales hombres de la ciudad.

\bibleverse{7} Cuando llegó la carta, agarraron a los hijos del rey y
mataron a los setenta, pusieron sus cabezas en canastas y las enviaron a
Jehú en Jezrel. \bibleverse{8} Un mensajero llegó y le dijo a Jehú:
``Han traído las cabezas de los hijos del rey''. Jehú dio la orden:
``Ponlas en dos montones a la entrada de la puerta de la ciudad hasta la
mañana''.

\bibleverse{9} Por la mañana Jehú salió a hablar con el pueblo que se
había reunido. ``Ustedes no han hecho nada malo'',\footnote{\textbf{10:9}
  ``Ustedes no ha hecho nada malo'': Literalmente, ``Ustedes son
  justos''.} les dijo. ``Yo fui el que conspiró contra mi maestro y lo
mató. Pero ¿quién mató a todos estos? \bibleverse{10} Tengan la
seguridad de que nada de lo que el Señor ha profetizado contra la casa
de Acab fallará, porque el Señor ha hecho lo que prometió por medio de
su siervo Elías''.

\bibleverse{11} Así que Jehú mató a todos los que quedaban en Jezrel de
la casa de Acab, así como a todos sus altos funcionarios, amigos
cercanos y sacerdotes. Esto dejó a Acab sin un solo sobreviviente.

\hypertarget{jehuxfa-mata-a-los-pruxedncipes-de-judea}{%
\subsection{Jehú mata a los príncipes de
Judea}\label{jehuxfa-mata-a-los-pruxedncipes-de-judea}}

\bibleverse{12} Entonces Jehú partió y se dirigió a Samaria. En
Bed-Équed de los Pastores, \bibleverse{13} se encontró con algunos
parientes de Ocozías, rey de Judá. ``¿Quiénes son ustedes?'' , les
preguntó. ``Somos parientes de Ocozías'', le respondieron. ``Hemos
venido a visitar a los hijos del rey y de la reina madre''. \footnote{\textbf{10:13}
  2Cró 22,8}

\bibleverse{14} ``¡Atrápenlos vivos!'' ordenó Jehú. Así que los tomaron
vivos y los mataron en el pozo de Bed-Equed. Eran cuarenta y dos
hombres. No permitió que ninguno de ellos viviera.

\hypertarget{jehuxfa-lleva-al-recabita-jonadab-a-su-amistad}{%
\subsection{Jehú lleva al recabita Jonadab a su
amistad}\label{jehuxfa-lleva-al-recabita-jonadab-a-su-amistad}}

\bibleverse{15} Salió de allí y se encontró con Jonadab, hijo de Recab,
que venía a su encuentro. Jehú lo saludó y le preguntó: ``¿Estás tan
comprometido conmigo como yo contigo?'' . ``Sí, lo estoy'', respondió
Jonadab. ``En ese caso, dame tu mano'', dijo Jehú. Así que él extendió
su mano, y Jehú lo ayudó a subir al carro.

\bibleverse{16} ``Acompáñame y verás lo dedicado que estoy al Señor'',
dijo Jehú, y lo hizo subir a su carro. \bibleverse{17} Cuando Jehú llegó
a Samaria, fue matando a todos los que quedaban de la familia de Acab
hasta que los mató a todos, tal como el Señor había dicho por medio de
Elías. \footnote{\textbf{10:17} 1Re 21,21-22}

\hypertarget{jehuxfa-extermina-a-los-adoradores-de-baals-en-samaria}{%
\subsection{Jehú extermina a los adoradores de Baals en
Samaria}\label{jehuxfa-extermina-a-los-adoradores-de-baals-en-samaria}}

\bibleverse{18} Jehú hizo reunir a todo el pueblo y les dijo: ``Acab
adoraba un poco a Baal, pero Jehú lo adorará mucho. \footnote{\textbf{10:18}
  1Re 16,31-33} \bibleverse{19} Así que convoca a todos los profetas de
Baal, a todos sus servidores y a todos sus sacerdotes. Asegúrate de que
no falte nadie, porque estoy organizando un gran sacrificio para Baal.
El que no asista será ejecutado''. Pero el plan de Jehú era un truco
para destruir a los seguidores de Baal.

\bibleverse{20} Jehú dio la orden: ``¡Convoca una asamblea religiosa
para honrar a Baal!'' Así lo hicieron.

\bibleverse{21} Jehú envió el anuncio por todo Israel. Todos los
seguidores de Baal acudieron; no faltó ni un solo hombre. Entraron en el
templo de Baal, llenándolo de punta a punta. \bibleverse{22} Jehú dijo
al guardián del guardarropa: ``Distribuye la ropa para todos los siervos
de Baal''. Así que sacó ropa para ellos.

\bibleverse{23} Luego Jehú y Jonadab, hijo de Recab, entraron en el
templo de Baal. Jehú dijo a los seguidores de Baal: ``Miren a su
alrededor y asegúrense de que nadie que siga al Señor esté aquí con
ustedes, sólo los adoradores de Baal''. \footnote{\textbf{10:23} 2Re
  10,15}

\bibleverse{24} Estaban dentro presentando sacrificios y holocaustos.
Ahora bien, Jehú había colocado a ochenta hombres afuera y les advirtió:
``Les estoy entregando a estos hombres. Si dejan escapar a alguno de
ellos, ustedes pagarán sus vidascon las vidas de ustedes''. \footnote{\textbf{10:24}
  1Re 20,39}

\bibleverse{25} En cuanto Jehú terminó de presentar el holocausto,
ordenó a sus guardias y oficiales: ``¡Entren y mátenlos a todos! No
dejen que se escape ni uno solo''. Así que los mataron con sus espadas.
Los guardias y los oficiales arrojaron sus cuerpos fuera, y luego
entraron en el santuario interior del templo de Baal. \footnote{\textbf{10:25}
  1Re 18,40} \bibleverse{26} Sacaron los pilares de los ídolos y los
quemaron. \footnote{\textbf{10:26} 2Re 11,18} \bibleverse{27}
Destrozaron el pilar sagrado de Baal, derribaron el templo de Baal y lo
convirtieron en un retrete, lo que sigue siendo hasta hoy. \footnote{\textbf{10:27}
  2Re 3,2}

\hypertarget{la-predicaciuxf3n-de-dios-a-jehuxfa-fracasos-de-jehuxfa-conclusiuxf3n-de-la-historia-de-jehuxfa}{%
\subsection{La predicación de Dios a Jehú; Fracasos de Jehú; Conclusión
de la historia de
Jehú}\label{la-predicaciuxf3n-de-dios-a-jehuxfa-fracasos-de-jehuxfa-conclusiuxf3n-de-la-historia-de-jehuxfa}}

\bibleverse{28} Así fue como Jehú destruyó el culto a Baal en Israel,
\bibleverse{29} pero no puso fin a los pecados que Jeroboam, hijo de
Nabat, había hecho cometer a Israel: la adoración de los becerros de oro
en Betel y Dan. \bibleverse{30} El Señor le dijo a Jehú: ``Puesto que
has hecho bien y has llevado a cabo lo que es justo a mis ojos, y has
cumplido todo lo que planeé para la casa de Acab, tus descendientes se
sentarán en el trono de Israel hasta la cuarta generación''. \footnote{\textbf{10:30}
  2Re 15,12}

\bibleverse{31} Pero Jehú no se comprometió del todo a seguir la ley del
Señor, el Dios de Israel. No puso fin a los pecados que Jeroboam había
hecho cometer a Israel.

\bibleverse{32} En ese momento el Señor comenzó a reducir la extensión
de Israel. Jazael derrotó a los israelitas en todo su territorio
\bibleverse{33} al este del Jordán, en toda la tierra de Galaad (la
región ocupada por Gad, Rubén y Manasés), y desde Aroer por el valle de
Arnón hasta Galaad y Basán. \bibleverse{34} El resto de lo que sucedió
en el reinado de Jehú, todo lo que hizo y lo que logró, está registrado
en el Libro de las Crónicas de los Reyes de Israel. \bibleverse{35} Jehú
murió y fue enterrado en Samaria. Su hijo Joacaz lo sucedió como rey.
\footnote{\textbf{10:35} 2Re 13,1}

\bibleverse{36} Jehú reinó sobre Israel en Samaria durante veintiocho
años.

\hypertarget{el-robo-y-el-asesinato-de-ataluxeda-rescate-de-jouxe1s}{%
\subsection{El robo y el asesinato de Atalía; Rescate de
Joás}\label{el-robo-y-el-asesinato-de-ataluxeda-rescate-de-jouxe1s}}

\hypertarget{section-10}{%
\section{11}\label{section-10}}

\bibleverse{1} Cuando Atalía, la madre de Ocozías,\footnote{\textbf{11:1}
  Ocozías era el rey de Judá.} vio que su hijo había muerto, dio la
orden de asesinar a todo el resto de la familia real. \bibleverse{2}
Pero Josaba, hija del rey Jehoram, hermana de Ocozías, tomó a Joás, hijo
de Ocozías, apartándolo del resto de los hijos del rey que estaban
siendo asesinados. Lo puso a él y a su nodriza en un dormitorio para
ocultarlo de Atalía, y no fue asesinado. \bibleverse{3} Joás permaneció
escondido en el Templo del Señor durante seis años, mientras Atalía
gobernaba el país.

\hypertarget{la-conspiraciuxf3n-de-joiada}{%
\subsection{La conspiración de
Joiada}\label{la-conspiraciuxf3n-de-joiada}}

\bibleverse{4} En el séptimo año, Joyadá\footnote{\textbf{11:4} Joyadá
  era el sumo sacerdote.} mandó llamar a los comandantes de centenares,
a los cereteos,\footnote{\textbf{11:4} ``Cereteos'': probablemente
  mercenarios extranjeros utilizados como guardia real.} y los guardias,
y los llevó al Templo del Señor. Hizo un acuerdo con ellos y les hizo
prestar un juramento. Allí, en el Templo del Señor, les mostró al hijo
del rey \bibleverse{5} y les ordenó: ``Esto es lo que vais a hacer: Un
tercio de vosotros, que viene de servicio el sábado, vigilará el palacio
real. \bibleverse{6} Un tercio estará en la Puerta Sur, y otro tercio en
la puerta detrás de los guardias. Se alternarán en la vigilancia del
palacio. \bibleverse{7} Las dos divisiones que normalmente salen de
servicio en el día de reposo vigilarán el Templo del Señor para el rey.
\bibleverse{8} Rodeen al rey con las armas desenfundadas, y cualquiera
que se acerque a esta línea debe ser asesinado. Permanezcan cerca del
rey dondequiera que vaya''.

\bibleverse{9} Los comandantes de centenares siguieron todas las
instrucciones que había dado el sacerdote Joyadá. Cada uno tomó a sus
propios hombres, los que venían de servicio el sábado y los que salían
de servicio, y se presentaron ante el sacerdote Joyadá. \bibleverse{10}
Entonces el sacerdote entregó a los comandantes de centenas las lanzas y
los escudos que habían pertenecido al rey David y que estaban guardados
en el Templo del Señor. \footnote{\textbf{11:10} 2Sam 8,7}
\bibleverse{11} Los guardias se pusieron de pie con las armas
desenfundadas rodeando al rey junto al altar, y en una línea alrededor
del Templo, desde el lado sur hasta el lado norte del Templo.
\bibleverse{12} Entonces Joyadá sacó al hijo del rey, le puso la corona
y le entregó un ejemplar de la Ley de Dios. Lo proclamaron rey y lo
ungieron. El pueblo aplaudió y gritó: ``¡Viva el rey!''.

\hypertarget{captura-y-asesinato-de-athaluxeda}{%
\subsection{Captura y asesinato de
Athalía}\label{captura-y-asesinato-de-athaluxeda}}

\bibleverse{13} Cuando Atalía oyó el ruido de los guardias y del pueblo,
corrió hacia la multitud en el Templo del Señor. \bibleverse{14} Vio al
rey de pie junto a su columna, como era costumbre. Los comandantes y los
trompetistas estaban con el rey, y todos celebraban y tocaban las
trompetas. Atalía se rasgó las vestiduras y gritó: ``¡Traición!
Traición!''

\bibleverse{15} Joyadá ordenó a los comandantes del ejército: ``Llévenla
ante los hombres que están frente al Templo y maten a cualquiera que la
siga''. Antes, el sacerdote había dejado claro: ``No deben matarla en el
Templo del Señor''. \bibleverse{16} Entonces la agarraron, la llevaron
hasta donde los caballos entran en el recinto del palacio y allí la
mataron. \footnote{\textbf{11:16} Neh 3,28}

\hypertarget{medidas-de-joiada-para-la-gloria-de-dios-coronaciuxf3n-de-jouxe1s}{%
\subsection{Medidas de Joiada para la gloria de Dios; Coronación de
Joás}\label{medidas-de-joiada-para-la-gloria-de-dios-coronaciuxf3n-de-jouxe1s}}

\bibleverse{17} Entonces Joyadá hizo un acuerdo solemne entre el Señor,
el rey y el pueblo de que serían el pueblo del Señor. También hizo un
acuerdo entre el rey y el pueblo. \bibleverse{18} Todos fueron al Templo
de Baal y derribaron sus altares y destrozaron los ídolos. Mataron a
Matán, el sacerdote de Baal, frente al altar. Entonces el sacerdote
Joyadá mandó poner guardias en el Templo del Señor. \bibleverse{19}
Junto con los comandantes, los nobles, los gobernantes del pueblo y todo
el pueblo, condujo al rey en una procesión desde el Templo del Señor,
entrando por la puerta superior al palacio real. Allí sentaron al rey en
el trono real. \bibleverse{20} En todo el país la gente celebró, y
Jerusalén estaba en paz, porque Atalía había sido muerta a espada en el
palacio.

\bibleverse{21} Joás tenía siete años cuando se convirtió en rey.

\hypertarget{jouxe1s-rey-de-juduxe1}{%
\subsection{Joás rey de Judá}\label{jouxe1s-rey-de-juduxe1}}

\hypertarget{section-11}{%
\section{12}\label{section-11}}

\bibleverse{1} Joás\footnote{\textbf{12:1} Aquí y en otros lugares se
  escribe ``Joás''.} llegó a ser rey en el séptimo año del reinado de
Jehú, y reinó en Jerusalén durante cuarenta años. Su madre se llamaba
Sibia de Beerseba. \bibleverse{2} Joás hizo lo que era correcto a los
ojos del Señor durante los años en que el sacerdote Joyadá le aconsejó.
\bibleverse{3} Aun así, los altares paganos no se quitaron: el pueblo
siguió sacrificando y presentando holocaustos en esos lugares.

\bibleverse{4} Entonces Joásles dijo a los sacerdotes: ``Reúnan todo el
dinero que se trae como ofrendas sagradas al Templo del Señor, ya sea el
dinero del censo, el dinero de los votos individuales y el dinero que se
trae como donación voluntaria al Templo del Señor. \footnote{\textbf{12:4}
  2Re 14,4; 1Re 22,44}

\hypertarget{ordenanza-del-rey-sobre-la-reparaciuxf3n-del-templo-y-sobre-la-administraciuxf3n-y-uso-del-dinero-del-templo}{%
\subsection{Ordenanza del rey sobre la reparación del templo y sobre la
administración y uso del dinero del
templo}\label{ordenanza-del-rey-sobre-la-reparaciuxf3n-del-templo-y-sobre-la-administraciuxf3n-y-uso-del-dinero-del-templo}}

\bibleverse{5} Que cada sacerdote reciba el dinero de los que dan, y lo
use para reparar cualquier daño que se descubra en el Templo''.

\bibleverse{6} Pero en el año veintitrés del reinado de Joás, los
sacerdotes aún no habían reparado los daños del Templo. \bibleverse{7}
Entonces el rey Joás convocó a Joyadá y a los demás sacerdotes y les
preguntó: ``¿Por qué no han reparado los daños del Templo? No usen más
dinero que se les ha dado para ustedes, en cambio entréguenlo a otros
para que reparen el Templo''.

\bibleverse{8} Entonces los sacerdotes acordaron no recibir más dinero
del pueblo, y que no realizarían ellos mismos las reparaciones del
Templo. \bibleverse{9} El sacerdote Joyadá tomó una gran caja de madera,
hizo un agujero en su tapa y la colocó a la derecha del altar, junto a
la entrada del Templo del Señor. Allí, los sacerdotes que custodiaban la
entrada ponían en la caja todo el dinero que se traía al Templo del
Señor. \bibleverse{10} Cuando veían que había mucho dinero en la caja,
el secretario del rey y el sumo sacerdote se acercaban, contaban el
dinero que entraba en el Templo del Señor y lo ponían en bolsas.
\bibleverse{11} Luego pesaban el dinero y lo entregaban a los
supervisores de la obra del Templo del Señor. Ellos pagaban a los que
hacían la obra: los carpinteros, los constructores, los \bibleverse{12}
albañiles y los canteros. También compraron la madera y los bloques de
piedra cortada que se necesitaban para la reparación del Templo del
Señor, y pagaron todos los demás gastos de la restauración del Templo.
\bibleverse{13} Sin embargo, el dinero recaudado para el Templo del
Señor no se utilizaba para fabricar jofainas de plata, adornos para
lámparas, cuencos, trompetas o cualquier otro artículo de oro o plata
para el Templo del Señor. \bibleverse{14} Se utilizaba para pagar a los
obreros que hacían las reparaciones en el Templo del Señor.
\bibleverse{15} No se pedían cuentas a los hombres que recibían el
dinero para pagar a los trabajadores, porque lo hacían todo
honestamente. \bibleverse{16} El dinero de las ofrendas por la culpa y
por el pecado no se recogía para el Templo del Señor, porque pertenecía
a los sacerdotes.

\hypertarget{jouxe1s-salvuxf3-a-jerusaluxe9n-del-ataque-de-hazael-pagando-dinero-su-asesinato}{%
\subsection{Joás salvó a Jerusalén del ataque de Hazael pagando dinero;
su
asesinato}\label{jouxe1s-salvuxf3-a-jerusaluxe9n-del-ataque-de-hazael-pagando-dinero-su-asesinato}}

\bibleverse{17} Por ese tiempo, Jazael, rey de Aram, fue a atacar Gat y
la capturó. Luego marchó para atacar a Jerusalén. \bibleverse{18}
Entonces el rey Joás de Judá tomó todos los objetos sagrados dedicados
por sus antepasados Josafat, Jehoram y Ocozías, los reyes de Judá, junto
con todos los objetos que él mismo había dedicado, y todo el oro que se
encontraba en los tesoros del Templo del Señor y del palacio real, y
envió todo a Jazael, rey de Aram. Entonces Jazael se retiró de
Jerusalén. \footnote{\textbf{12:18} 2Re 10,32}

\bibleverse{19} El resto de lo que sucedió en el reinado de Joás y todo
lo que hizo está registrado en el Libro de las Crónicas de los Reyes de
Judá. \footnote{\textbf{12:19} 1Re 15,18} \bibleverse{20} Sus
funcionarios conspiraron contra él y lo asesinaron en Bet Miló, en el
camino que baja a Silla. \bibleverse{21} Los funcionarios que lo
atacaron y mataron fueron Jozacar, hijo de Simat, y Jozabad, hijo de
Semer. Lo enterraron con sus antepasados en la Ciudad de David. Su hijo
Amasías le sucedió como rey.\footnote{\textbf{12:21} 2Re 14,5}

\hypertarget{joachuxe2z-rey-de-israel}{%
\subsection{Joachâz rey de Israel}\label{joachuxe2z-rey-de-israel}}

\hypertarget{section-12}{%
\section{13}\label{section-12}}

\bibleverse{1} Joacaz, hijo de Jehú, llegó a ser rey de Israel en el año
veintitrés del reinado de Joás, hijo de Ocozías, rey de Judá. Reinó en
Samaria durante diecisiete años. \footnote{\textbf{13:1} 2Re 10,35}
\bibleverse{2} E hizo lo malo a los ojos del Señor y siguió los pecados
que Jeroboam, hijo de Nabat, había hecho cometer a Israel; no les puso
fin. \footnote{\textbf{13:2} 1Re 12,26-33} \bibleverse{3} Así que el
Señor se enojó con Israel, y permitió repetidamente que fuera derrotado
por Jazael, rey de Aram, y su hijo Ben-hadad. \footnote{\textbf{13:3}
  2Re 10,32} \bibleverse{4} Joacaz pidió ayuda al Señor, y el Señor
respondió a su petición porque vio lo mal que el rey de Aram trataba a
Israel. \bibleverse{5} El Señor le dio a Israel alguien que los salvara
para que dejaran de estar bajo el dominio arameo. Entonces los
israelitas pudieron volver a vivir con seguridad como antes. \footnote{\textbf{13:5}
  2Re 14,27} \bibleverse{6} Aun así, no pusieron fin a los pecados que
la casa de Jeroboam había hecho cometer a Israel: continuaron
siguiéndolos. El ídolo de Asera seguía en pie en Samaria. \footnote{\textbf{13:6}
  1Re 16,33} \bibleverse{7} Todo lo que quedó del ejército de Joacaz
fueron cincuenta jinetes, diez carros y diez mil soldados, pues el rey
de Aram había destruido al resto, convirtiéndolos en polvo como cuando
se trilla el grano. \bibleverse{8} El resto de lo que sucedió en el
reinado de Joacaz, todo lo que hizo y sus grandes logros están
registrados en el Libro de las Crónicas de los Reyes de Israel.
\bibleverse{9} Joacaz murió y fue enterrado en Samaria. Su hijo
Joás\footnote{\textbf{13:9} ``Joás'': El mismo nombre del rey de Judá.
  Nótese que este es el hijo de Joacaz, y no debe confundirse con Joás,
  el rey de Judá.} le sucedió como rey.

\hypertarget{joas-kuxf6nig-von-israel}{%
\subsection{Joas König von Israel}\label{joas-kuxf6nig-von-israel}}

\bibleverse{10} Joás, hijo de Joacaz, llegó a ser rey de Israel en
Samaria en el año treinta y siete del reinado del rey Joásde Judá, y
reinó durante dieciséis años. \bibleverse{11} Hizolo malo a los ojos del
Señor y no puso fin a todos los pecados que Jeroboam, hijo de Nabat,
había hecho cometer a Israel: continuó siguiéndolos. \footnote{\textbf{13:11}
  2Re 13,2} \bibleverse{12} El resto de lo que sucedió en el reinado de
Joás, todo lo que hizo y sus grandes logros, como su guerra contra
Amasías, rey de Judá, están registrados en el Libro de las Crónicas de
los Reyes de Israel. \footnote{\textbf{13:12} 2Re 14,8-16}
\bibleverse{13} Joás murió, y Jeroboam se sentó en su trono.\footnote{\textbf{13:13}
  Algunos creen que este cambio en la fórmula que describe la sucesión
  real sugiere que Jeroboam ya reinaba como corregente con su padre.}
Fue enterrado en Samaria con los reyes de Israel. \footnote{\textbf{13:13}
  2Re 14,23}

\hypertarget{jouxe1s-con-el-enfermo-eliseo-la-muerte-de-eliseo}{%
\subsection{Joás con el enfermo Eliseo; La muerte de
Eliseo}\label{jouxe1s-con-el-enfermo-eliseo-la-muerte-de-eliseo}}

\bibleverse{14} Eliseo había enfermado de una enfermedad que acabaría
matándolo. Joás, rey de Israel, fue a visitarlo y lloró por él,
diciendo: ``¡Padre mío, padre mío, los carros y los jinetes de Israel!''
\footnote{\textbf{13:14} 2Re 2,12}

\bibleverse{15} Eliseo le dijo: ``Busca un arco y unas flechas''. Así
que Joás encontró un arco y algunas flechas. \bibleverse{16} Entonces
Eliseo le dijo al rey de Israel: ``Recoge el arco''. Así que el rey
recogió el arco. Eliseo puso sus manos sobre las del rey.
\bibleverse{17} ``Abre la ventana del este'', le dijo. Así que el rey la
abrió y Eliseo le dijo: ``¡Dispara!''. Y disparó una flecha. Entonces
Eliseo le explicó: ``Esta es la flecha de la victoria del Señor, que
representa la flecha de la victoria sobre los arameos. Atacarás a los
arameos en Afec y acabarás con ellos''.

\bibleverse{18} Entonces Eliseo dijo: ``¡Recoge las flechas!'' Así que
las recogió. Eliseo le dijo al rey de Israel: ``¡Golpea el suelo con
ellas!'' Golpeó el suelo tres veces, y luego se detuvo. \bibleverse{19}
El hombre de Dios se enfadó con él y le dijo: ``Deberías haber golpeado
el suelo cinco o seis veces. Entonces habrías atacado a los arameos
hasta destruirlos por completo. Pero ahora sólo atacarás a los arameos
tres veces''.

\hypertarget{elisa-sigue-trabajando-milagrosamente-en-su-tumba}{%
\subsection{Elisa sigue trabajando milagrosamente en su
tumba}\label{elisa-sigue-trabajando-milagrosamente-en-su-tumba}}

\bibleverse{20} Eliseo murió y fue enterrado. Los asaltantes del país de
Moab solían invadir Israel cada primavera.

\bibleverse{21} En cierta ocasión, unos israelitas estaban enterrando a
un hombre cuando de pronto vieron que se acercaban unos asaltantes, así
que rápidamente arrojaron al hombre a la tumba de Eliseo. En cuanto tocó
los huesos de Eliseo, el hombre volvió a la vida y se levantó.

\hypertarget{las-tres-victorias-de-jouxe1s-sobre-los-sirios}{%
\subsection{Las tres victorias de Joás sobre los
sirios}\label{las-tres-victorias-de-jouxe1s-sobre-los-sirios}}

\bibleverse{22} Jazael, rey de Aram, causó problemas a Israel durante
todo el reinado de Joacaz. \bibleverse{23} Pero el Señor los ayudó con
gracia y fue bondadoso con ellos. Los cuidó por su acuerdo con Abraham,
Isaac y Jacob. Hasta el día de hoy no ha querido destruirlos ni echarlos
de su presencia. \footnote{\textbf{13:23} Lev 26,42}

\bibleverse{24} Cuando murió Jazael, rey de Aram, su hijo Ben-hadad lo
sucedió como rey. \bibleverse{25} Entonces Joás, hijo de Joacaz,
recuperó de Ben-hadad, hijo de Jazael, las ciudades que Jazael había
capturado de su padre Joacaz. Joás derrotó a Ben-hadad tres veces, y así
recuperó las ciudades israelitas.

\hypertarget{amasuxedas-rey-de-juduxe1-buen-comienzo-para-el-gobierno}{%
\subsection{Amasías rey de Judá; Buen comienzo para el
gobierno}\label{amasuxedas-rey-de-juduxe1-buen-comienzo-para-el-gobierno}}

\hypertarget{section-13}{%
\section{14}\label{section-13}}

\bibleverse{1} Amasías, hijo de Joás, llegó a ser rey de Judá en el
segundo año del reinado de Joás, hijo de Joacaz, rey de Israel.
\footnote{\textbf{14:1} 2Re 12,22} \bibleverse{2} Tenía veinticinco años
cuando llegó a ser rey, y reinó en Jerusalén durante veintinueve años.
Su madre se llamaba Joadán de Jerusalén. \bibleverse{3} Hizo lo que era
correcto a los ojos del Señor, pero no de la misma manera que su
antepasado David. Hizo todo como lo había hecho su padre Joás.
\bibleverse{4} Pero los altares paganos no fueron destruidos. El pueblo
seguía sacrificando y presentando holocaustos en esos lugares.
\footnote{\textbf{14:4} 2Re 15,4} \bibleverse{5} Una vez que estuvo
seguro en el trono, ejecutó a los funcionarios que habían asesinado a su
padre el rey. \footnote{\textbf{14:5} 2Re 12,21-22} \bibleverse{6} Pero
no ejecutó a los hijos de los asesinos, siguiendo el mandato del Señor
en la ley de Moisés: ``Los padres no deben morir por los pecados de sus
hijos, ni los hijos deben morir por los pecados de sus padres. Cada uno
debe morir por su propio pecado''. \footnote{\textbf{14:6} Deut 24,16}

\bibleverse{7} Amasías mató a diez mil edomitas en el Valle de la Sal.
Atacó y capturó Sela y la rebautizó como Joktheel, que es como se llama
hasta hoy.

\hypertarget{la-desafortunada-guerra-de-amasuxedas-con-jouxe1s-de-israel}{%
\subsection{La desafortunada guerra de Amasías con Joás de
Israel}\label{la-desafortunada-guerra-de-amasuxedas-con-jouxe1s-de-israel}}

\bibleverse{8} Amasías envió mensajeros al rey de Israel, Joás, hijo de
Joacaz, hijo de Jehú, diciéndole: ``Luchemos,\footnote{\textbf{14:8}
  ``Luchemos'': Literalmente ``encontrémonos'', pero el contexto deja
  claro que Amasías intentaba provocar un conflicto armado.} cara a
cara!''

\bibleverse{9} Joás, rey de Israel, respondió a Amasías, rey de Judá:
``En el Líbano, un cardo envió un mensaje a un cedro, diciendo: `Dale tu
hija como esposa a mi hijo'. Pero pasó un animal salvaje del Líbano y
pisoteó el cardo. \bibleverse{10} Puede que hayas derrotado a Edom.
Ahora te has vuelto arrogante. Quédate en casa y disfruta de tu
victoria. ¿Por qué provocar problemas que te harán caer a ti, y a Judá
contigo?''

\bibleverse{11} Pero Amasías se negó a escuchar, así que Joás, rey de
Israel, vino a atacarlo. Él y Amasías, rey de Judá, se encontraron cara
a cara en Bet Semes, en Judá. \bibleverse{12} El ejército de Judá fue
derrotado por Israel y huyó a su casa. \bibleverse{13} Joás, rey de
Israel, capturó a Amasías, rey de Judá, hijo de Joás, hijo de Ocozías,
en Bet Semes. Entonces Joás atacó Jerusalén y derribó la muralla de la
ciudad desde la puerta de Efraín hasta la puerta de la Esquina, de unos
cuatrocientos codos de longitud. \bibleverse{14} Se llevó todo el oro y
la plata, y todos los objetos que se encontraban en el Templo del Señor
y en los tesoros del palacio real, y también algunos rehenes. Luego
regresó a Samaria.

\hypertarget{palabras-de-clausura-sobre-jouxe1s-de-israel}{%
\subsection{Palabras de clausura sobre Joás de
Israel}\label{palabras-de-clausura-sobre-jouxe1s-de-israel}}

\bibleverse{15} El resto de lo que sucedió en el reinado de Joás, todo
lo que hizo y sus grandes logros y su guerra con Amasías, rey de Judá,
están registrados en el Libro de las Crónicas de los Reyes de Israel.
\bibleverse{16} Joás murió y fue enterrado en Samaria con los reyes de
Israel. Su hijo Jeroboam le sucedió como rey. \footnote{\textbf{14:16}
  2Re 13,13}

\hypertarget{palabras-finales-sobre-amasuxedas-de-juduxe1-su-asesinato}{%
\subsection{Palabras finales sobre Amasías de Judá; su
asesinato}\label{palabras-finales-sobre-amasuxedas-de-juduxe1-su-asesinato}}

\bibleverse{17} Amasías, hijo de Joás, rey de Judá, vivió quince años
más después de la muerte de Joás, hijo de Joacaz, rey de Israel.
\bibleverse{18} El resto de los acontecimientos que ocurrieron en el
reinado de Amasías están registrados en el Libro de las Crónicas de los
Reyes de Judá. \bibleverse{19} Una conspiración contra Amasías tuvo
lugar en Jerusalén, y él huyó a Laquis. Pero se enviaron hombres tras él
y lo asesinaron allí. \bibleverse{20} Lo trajeron de vuelta a caballo y
lo enterraron en Jerusalén con sus antepasados en la Ciudad de David.
\footnote{\textbf{14:20} 2Re 9,28}

\hypertarget{azaruxedas-asume-el-cargo}{%
\subsection{Azarías asume el cargo}\label{azaruxedas-asume-el-cargo}}

\bibleverse{21} Entonces todo el pueblo de Judá nombró rey al hijo de
Amasías, Azarías, para que reemplazara a su padre. Azarías tenía
dieciséis años. \footnote{\textbf{14:21} 2Re 15,1-2} \bibleverse{22}
Azarías reconquistó Elat para Judá y la reconstruyó después de la muerte
de su padre. \footnote{\textbf{14:22} 2Re 16,6}

\hypertarget{jeroboam-ii-rey-de-israel}{%
\subsection{Jeroboam II Rey de Israel}\label{jeroboam-ii-rey-de-israel}}

\bibleverse{23} Jeroboam, hijo de Joás, llegó a ser rey de Israel en el
año quince del reinado de Amasías, hijo de Joás, rey de Judá. Reinó en
Samaria durante cuarenta y un años. \footnote{\textbf{14:23} 2Re 14,16;
  Os 1,1; Am 1,1} \bibleverse{24} Hizo lo malo a los ojos del Señor y no
puso fin a todos los pecados que Jeroboam, hijo de Nabat, había hecho
cometer a Israel. \footnote{\textbf{14:24} 1Re 12,26-33} \bibleverse{25}
Restituyó la frontera de Israel a donde estaba, desde Lebó-Jamat hasta
el Mar de la Arabá,\footnote{\textbf{14:25} ``Mar de la Arabá'': el
  Arabah es el Valle del Jordán, por lo que se referiría al Mar Muerto.}
como el Señor, el Dios de Israel, había dicho por medio de su siervo
Jonás, hijo de Amitai, el profeta, que venía de Gat-Jefer. \footnote{\textbf{14:25}
  Jon 1,1} \bibleverse{26} El Señor había visto lo mucho que estaban
sufriendo los israelitas, tanto los esclavos como los libres. Nadie
estaba allí para ayudar a Israel. \footnote{\textbf{14:26} Deut 32,36}
\bibleverse{27} Sin embargo, como el Señor había dicho que no eliminaría
a Israel, lo salvó por medio de Jeroboam, hijo de Joás. \footnote{\textbf{14:27}
  2Re 13,5} \bibleverse{28} El resto de lo que sucedió en el reinado de
Jeroboam, todo lo que hizo, sus grandes logros y las batallas que libró,
y cómo recuperó para Israel tanto Damasco como Jamat, están registrados
en el Libro de las Crónicas de los Reyes de Israel. \bibleverse{29}
Jeroboam murió y fue enterrado con los reyes de Israel. Su hijo Zacarías
le sucedió como rey.\footnote{\textbf{14:29} 2Re 15,8}

\hypertarget{azaruxedas-rey-de-juduxe1}{%
\subsection{Azarías rey de Judá}\label{azaruxedas-rey-de-juduxe1}}

\hypertarget{section-14}{%
\section{15}\label{section-14}}

\bibleverse{1} Azarías, hijo de Amasías, llegó a ser rey de Judá en el
año veintisiete del reinado de Jeroboam, rey de Israel. \footnote{\textbf{15:1}
  2Re 14,21} \bibleverse{2} Tenía dieciséis años cuando llegó a ser rey,
y reinó en Jerusalén durante cincuenta y dos años. Su madre se llamaba
Jecolías, de Jerusalén. \bibleverse{3} Hizo lo que era correcto a los
ojos del Señor, tal como lo había hecho su padre Amasías. \bibleverse{4}
Pero los altares paganos no fueron destruidos. El pueblo seguía
sacrificando y presentando holocaustos en esos lugares. \footnote{\textbf{15:4}
  2Re 14,3-4} \bibleverse{5} El Señor tocó al rey y tuvo lepra hasta el
día de su muerte. Vivió aislado en una casa aparte. Su hijo Jotam estaba
a cargo del palacio y era el verdadero gobernante del país. \footnote{\textbf{15:5}
  Lev 13,46} \bibleverse{6} El resto de lo que sucedió en el reinado de
Azarías y todo lo que hizo está registrado en el Libro de las Crónicas
de los Reyes de Judá. \bibleverse{7} Azarías murió y fue enterrado con
sus antepasados en la Ciudad de David. Su hijo Jotam le sucedió como
rey. \footnote{\textbf{15:7} 2Re 15,32}

\hypertarget{zacaruxedas-rey-de-israel}{%
\subsection{Zacarías rey de Israel}\label{zacaruxedas-rey-de-israel}}

\bibleverse{8} Zacarías, hijo de Jeroboam, llegó a ser rey de Israel en
el año treinta y ocho del reinado de Azarías, rey de Judá. Reinó en
Samaria durante seis meses. \footnote{\textbf{15:8} 2Re 14,29}
\bibleverse{9} Sus hechos fueron malos a los ojos del Señor, como los de
sus antepasados también. No puso fin a los pecados que Jeroboam, hijo de
Nabat, había hecho cometer a Israel. \footnote{\textbf{15:9} 1Re
  12,26-33} \bibleverse{10} Entonces Salum, hijo de Jabes, conspiró
contra Zacarías. Lo atacó, lo asesinó delante del pueblo y se apoderó de
él como rey. \footnote{\textbf{15:10} 2Re 15,14; Am 7,9} \bibleverse{11}
El resto de los acontecimientos del reinado de Zacarías están
registrados en el Libro de las Crónicas de los Reyes de Israel.
\bibleverse{12} Así se cumplió lo que el Señor le dijo a Jehú: ``Tus
descendientes se sentarán en el trono de Israel hasta la cuarta
generación''. \footnote{\textbf{15:12} 2Re 10,30}

\hypertarget{sallum-kuxf6nig-von-israel}{%
\subsection{Sallum König von Israel}\label{sallum-kuxf6nig-von-israel}}

\bibleverse{13} Salum, hijo de Jabes, llegó a ser rey en el año treinta
y nueve del reinado de Uzías de Judá. Reinó en Samaria durante un mes.
\bibleverse{14} Entonces Menajem, hijo de Gadi, fue de Tirsa a Samaria,
atacó y asesinó a Salum, hijo de Jabes, y asumió como rey.
\bibleverse{15} El resto de los sucesos del reinado de Salum y la
rebelión que tramó están registrados en el Libro de las Crónicas de los
Reyes de Israel.

\bibleverse{16} En aquel tiempo, Menajem, partiendo de Tirsa, atacó
Tifsa y la región cercana, porque no le entregaban la ciudad. Así que
destruyó a Tifsa y desgarró a todas las mujeres embarazadas.

\hypertarget{menahem-rey-de-israel}{%
\subsection{Menahem Rey de Israel}\label{menahem-rey-de-israel}}

\bibleverse{17} Menajem, hijo de Gadi, se convirtió en rey de Israel en
el año treinta y nueve del reinado de Azarías de Judá. Reinó en Samaria
durante diez años. \bibleverse{18} Durante todo su reinado hizo lo malo
a los ojos del Señor. No puso fin a los pecados que Jeroboam, hijo de
Nabat, había hecho cometer a Israel. \footnote{\textbf{15:18} 2Re 15,9}
\bibleverse{19} Pul,\footnote{\textbf{15:19} A menudo se le asocia con
  Tiglat-Piléser.} rey de Asiria, invadió el país. Menajem pagó a Pul
mil talentos de plata para que apoyara a Menajem en la consolidación de
su poder sobre el reino. \bibleverse{20} Menajem exigió el pago de todos
los hombres ricos de Israel, cincuenta siclos de plata cada uno, para
dárselos al rey de Asiria. Entonces el rey de Asiria se retiró y no se
quedó en el país. \bibleverse{21} El resto de lo que sucedió en el
reinado de Menajem y todo lo que hizo está registrado en el Libro de las
Crónicas de los Reyes de Israel. \bibleverse{22} Menajem murió y su hijo
Pecajías lo sucedió como rey.

\hypertarget{pekaja-rey-de-israel}{%
\subsection{Pekaja, rey de Israel}\label{pekaja-rey-de-israel}}

\bibleverse{23} Pecajías, hijo de Menajem, llegó a ser rey de Israel en
Samaria en el año cincuenta del reinado de Azarías de Judá, y reinó dos
años. \bibleverse{24} Hizo lo malo a los ojos del Señor. No puso fin a
los pecados que Jeroboam, hijo de Nabat, había hecho cometer a Israel.
\footnote{\textbf{15:24} 2Re 15,9} \bibleverse{25} Peca, hijo de
Remalías, uno de sus oficiales, conspiró contra él junto con Argob,
Arieh y cincuenta hombres de Galaad. Peca atacó y mató a Pecajías en la
fortaleza del palacio del rey en Samaria, y asumió como rey. \footnote{\textbf{15:25}
  2Re 15,10; 2Re 15,14; 2Re 15,30} \bibleverse{26} El resto de lo que
sucedió en el reinado de Pecajías y todo lo que hizo está registrado en
el Libro de las Crónicas de los Reyes de Israel.

\hypertarget{peka-rey-de-israel}{%
\subsection{Peka, rey de Israel}\label{peka-rey-de-israel}}

\bibleverse{27} Pecajá, hijo de Remalías, llegó a ser rey de Israel en
el año cincuenta y dos del reinado de Azarías de Judá. Reinó en Samaria
durante veinte años. \bibleverse{28} Sus hechos fueron malos a los ojos
del Señor. No puso fin a los pecados que Jeroboam, hijo de Nabat, había
hecho cometer a Israel. \footnote{\textbf{15:28} 2Re 15,9}
\bibleverse{29} Durante el reinado de Peca, rey de Israel,
Tiglat-Pileser, rey de Asiria, invadió y capturó Iyón, Abel-Bet-Macá,
Janoa, Cedes, Jazor, Galaad, Galilea y toda la tierra de Neftalí, y
llevó al pueblo a Asiria como prisioneros. \footnote{\textbf{15:29} 1Cró
  5,26} \bibleverse{30} Entonces Oseas, hijo de Ela, conspiró contra
Peca, hijo de Remalías. En el vigésimo año del reinado de Jotam, hijo de
Uzías, Oseas atacó a Peca, lo mató y asumió como rey. \footnote{\textbf{15:30}
  2Re 17,1; 2Re 15,25} \bibleverse{31} El resto de lo que sucedió en el
reinado de Peca y todo lo que hizo está registrado en el Libro de las
Crónicas de los Reyes de Israel.

\hypertarget{jotam-rey-de-juduxe1}{%
\subsection{Jotam rey de Judá}\label{jotam-rey-de-juduxe1}}

\bibleverse{32} Jotam, hijo de Uzías, llegó a ser rey de Judá en el
segundo año del reinado de Peca hijo de Remalías, rey de Israel.
\bibleverse{33} Tenía veinticinco años cuando llegó a ser rey, y reinó
en Jerusalén durante dieciséis años. Su madre se llamaba Jerusa, hija de
Sadoc. \bibleverse{34} Hizo lo correcto a los ojos del Señor, tal como
lo había hecho su padre Uzías. \footnote{\textbf{15:34} 2Re 15,3-4}

\bibleverse{35} Pero los altares paganos no fueron destruidos. El pueblo
seguía sacrificando y presentando holocaustos en esos lugares.
Reconstruyó la puerta superior del Templo del Señor. \bibleverse{36} El
resto de los acontecimientos del reinado de Jotam están registrados en
el Libro de las Crónicas de los Reyes de Judá. \bibleverse{37} Durante
ese tiempo el Señor comenzó a enviar a Rezín, rey de Aram, y a Peca,
hijo de Remalías, para que atacaran a Judá. \bibleverse{38} Jotam murió
y fue enterrado con sus antepasados en la Ciudad de David, su
antepasado. Su hijo Acaz lo sucedió como rey.

\hypertarget{las-abominaciones-paganas-de-acaz}{%
\subsection{Las abominaciones paganas de
Acaz}\label{las-abominaciones-paganas-de-acaz}}

\hypertarget{section-15}{%
\section{16}\label{section-15}}

\bibleverse{1} Acaz, hijo de Jotam, llegó a ser rey de Judá en el año
diecisiete del reinado de Peca, hijo de Remalías. \footnote{\textbf{16:1}
  2Re 15,38} \bibleverse{2} Acaz tenía veinte años cuando llegó a ser
rey, y reinó en Jerusalén durante dieciséis años. Pero a diferencia de
David, su antepasado, no hizo bien las cosas a los ojos del Señor, su
Dios. \bibleverse{3} Siguió los caminos de los reyes de Israel, e
incluso sacrificó a su hijo en el fuego, participando en las prácticas
repugnantes de las naciones que el Señor había expulsado ante los
israelitas. \bibleverse{4} Sacrificó y presentó holocaustos en los
lugares altos y en las colinas, y bajo todo árbol verde.

\hypertarget{su-guerra-con-siria-e-israel-acaz-se-convierte-en-tributo-a-los-asirios}{%
\subsection{Su guerra con Siria e Israel; Acaz se convierte en tributo a
los
asirios}\label{su-guerra-con-siria-e-israel-acaz-se-convierte-en-tributo-a-los-asirios}}

\bibleverse{5} Rezín, rey de Aram, y Peca, hijo de Remalías, rey de
Israel, llegaron y atacaron Jerusalén. Asediaron a Acaz, pero no
pudieron derrotarlo. \footnote{\textbf{16:5} Is 7,1-9} \bibleverse{6}
Fue entonces cuando Rezín, rey de Aram, recuperó Elat para
Edom.\footnote{\textbf{16:6} Aram/Edom. Ambas palabras son similares en
  hebreo. Parece poco probable que los arameos quisieran mantener una
  ciudad tan lejos de su propio territorio en la tierra de los edomitas.
  Algunas versiones sustituyen ``Rezin, rey de Aram'' por ``el rey de
  Edom'', pero no se le menciona previamente. La conclusión es que no se
  sabe con certeza si Elat fue conquistada por arameos o por edomitas;
  sin embargo, es seguro que la ciudad se perdió para el pueblo de Judá
  y fue ocupada por edomitas.} Expulsó al pueblo de Judá y envió a los
edomitas a Elat, donde todavía viven. \footnote{\textbf{16:6} 2Re 14,22}
\bibleverse{7} Ajaz envió mensajeros a Tiglat-pileser, rey de Asiria,
diciendo: ``Soy tu siervo y tu hijo. Por favor, ven a rescatarme de los
reyes de Aram e Israel que me están atacando''. \footnote{\textbf{16:7}
  2Re 15,29} \bibleverse{8} Acaz tomó la plata y el oro del Templo del
Señor y de los tesoros del palacio real, y se lo envió al rey de Asiria
como regalo. \footnote{\textbf{16:8} 1Re 15,18} \bibleverse{9} El rey de
Asiria le respondió positivamente. Fue y atacó Damasco, y la capturó.
Deportó a sus habitantes a Quir y ejecutó a Rezín.

\hypertarget{acaz-tiene-un-nuevo-altar-para-los-holocaustos-construido-emite-una-nueva-orden-de-sacrificio-e-interviene-en-la-propiedad-del-templo}{%
\subsection{Acaz tiene un nuevo altar para los holocaustos construido,
emite una nueva orden de sacrificio e interviene en la propiedad del
templo}\label{acaz-tiene-un-nuevo-altar-para-los-holocaustos-construido-emite-una-nueva-orden-de-sacrificio-e-interviene-en-la-propiedad-del-templo}}

\bibleverse{10} El rey Acaz fue a Damasco para reunirse con
Tiglat-Pileser, rey de Asiria. Durante su visita vio un altar\footnote{\textbf{16:10}
  Claramente un altar pagano, probablemente asirio. Es probable que
  Tiglat-pileser requiriera que los reyes subordinados le demostraran su
  lealtad, y esta acción de Acaz lo habría demostrado.} en Damasco, y
envió al sacerdote Urías un dibujo del altar, junto con las
instrucciones de cómo construirlo. \bibleverse{11} El sacerdote Urías
construyó el altar siguiendo todas las instrucciones que el rey Acaz
había enviado desde Damasco, y lo terminó antes de que el rey Acaz
regresara. \bibleverse{12} Cuando el rey regresó de Damasco vio el
altar. Se acercó a él y ofreció ofrendas en él. \bibleverse{13} Presentó
su holocausto y su ofrenda de grano, derramó su libación y roció sobre
él la sangre de sus ofrendas de paz. \bibleverse{14} También trasladó el
altar de bronce que estaba ante el Señor desde el frente del Templo,
entre el nuevo altar y el Templo del Señor, y lo colocó al norte del
nuevo altar. \bibleverse{15} Entonces el rey Acaz ordenó al sacerdote
Urías ``Usa este nuevo e importante altar para ofrecer el holocausto de
la mañana, la ofrenda de grano de la tarde, el holocausto y la ofrenda
de grano del rey, y el holocausto de todo el pueblo, y sus ofrendas de
grano y sus libaciones. Rocía sobre este altar la sangre de todos los
holocaustos y sacrificios. El viejo altar de bronce lo usaré para la
adivinación''. \bibleverse{16} El sacerdote Urías siguió las órdenes del
rey Acaz.

\bibleverse{17} El rey Acaz también quitó los armazones de los carros
móviles, y también sacó la pila de bronce de cada uno de ellos. Quitó el
Mar de los toros de bronce sobre los que descansaba y lo colocó sobre un
pedestal de piedra. \footnote{\textbf{16:17} 1Re 7,23-39}

\bibleverse{18} Derribó el dosel del sábado que habían construido en el
Templo, así como la entrada exterior del rey al Templo del Señor. Hizo
esto para complacer al rey de Asiria. \bibleverse{19} El resto de lo que
sucedió en el reinado de Acaz y todo lo que hizo está registrado en el
Libro de las Crónicas de los Reyes de Judá. \bibleverse{20} Acaz murió y
fue enterrado con sus antepasados en la Ciudad de David. Su hijo
Ezequías le sucedió como rey.

\hypertarget{oseas-rey-de-israel-cauxedda-del-imperio-cautiverio-asirio}{%
\subsection{Oseas rey de Israel; Caída del imperio; Cautiverio
asirio}\label{oseas-rey-de-israel-cauxedda-del-imperio-cautiverio-asirio}}

\hypertarget{section-16}{%
\section{17}\label{section-16}}

\bibleverse{1} Oseas, hijo de Ela, se convirtió en rey de Israel, en el
duodécimo año del reinado de Acaz de Judá. Reinó en Samaria durante
nueve años. \footnote{\textbf{17:1} 2Re 15,30} \bibleverse{2} Y sus
hechos fueron malos alos ojos del Señor, pero no de la misma manera que
los reyes de Israel que lo precedieron. \bibleverse{3} Salmanasar, rey
de Asiria, vino y lo atacó, y Oseas se sometió a él y le pagó tributo.
\bibleverse{4} Pero entonces el rey de Asiria descubrió que Oseas estaba
siendo desleal. Oseas había enviado mensajeros a So, rey de Egipto,
pidiendo ayuda, y también había dejado de enviar el tributo anual al rey
de Asiria como lo había hecho anteriormente. Entonces el rey de Asiria
arrestó a Oseas y lo puso en prisión. \footnote{\textbf{17:4} Os 12,2}
\bibleverse{5} Entonces el rey de Asiria invadió todo el país y atacó
Samaria, sitiándola durante tres años. \bibleverse{6} En el noveno año
del reinado de Oseas, el rey de Asiria capturó Samaria y deportó a los
israelitas a Asiria. Los asentó en Jalaj, en Gozán, sobre el río Jabor,
y en las ciudades de los medos.

\bibleverse{7} Todo esto sucedió porque el pueblo de Israel había pecado
contra el Señor, su Dios, el que los había sacado de Egipto, salvándolos
del poder del faraón, rey de Egipto. Habían adorado a otros dioses,
\bibleverse{8} adoptando los caminos de las naciones que el Señor había
expulsado antes de los israelitas, y las prácticas paganas introducidas
por los reyes de Israel. \bibleverse{9} En secreto, los israelitas
hicieron cosas que no eran correctas contra el Señor, su Dios.
Construyeron lugares altos en todas sus ciudades, desde torres de
vigilancia hasta ciudades fortificadas. \bibleverse{10} Levantaron
pilares de piedra paganos y postes de Asera en todas las colinas altas y
bajo todos los árboles verdes. \footnote{\textbf{17:10} 2Re 16,4; 1Re
  14,23} \bibleverse{11} Ofrecieron sacrificios en todos los lugares
altos, como las naciones que el Señor expulsó antes de ellos. Hicieron
cosas malas, enojando al Señor. \footnote{\textbf{17:11} 2Re 17,8}
\bibleverse{12} Adoraban a los ídolos, a pesar de que el Señor les había
dicho: ``No deben hacer eso''. \footnote{\textbf{17:12} Éxod 20,2-3;
  Éxod 23,13} \bibleverse{13} Sin embargo, el Señor les había advertido
repetidamente a Israel y a Judá, por medio de todos sus profetas y
videntes, diciendo: ``Dejen sus malos caminos y guarden mis mandamientos
e instrucciones. Sigan toda la ley que ordené a sus antepasados que
obedecieran, y que les di por medio de mis siervos los profetas''.
\bibleverse{14} Pero ellos se negaron a escuchar, y fueron tan tercos
como sus antepasados, que no confiaron en el Señor, su Dios.
\bibleverse{15} Abandonaron sus reglamentos y el pacto que había hecho
con sus antepasados, así como los decretos que les había dado. Siguieron
ídolos inútiles y ellos mismos se volvieron inútiles, imitando a las
naciones vecinas que el Señor les ordenó no imitar. \bibleverse{16}
Ignoraron todos los mandamientos del Señor, su Dios, y se hicieron dos
ídolos de metal, un becerro y un poste de Asera. Se inclinaron en
adoración al sol, la luna y las estrellas y sirvieron a Baal.
\footnote{\textbf{17:16} 1Re 12,28; 1Re 16,33} \bibleverse{17}
Sacrificaban a sus hijos e hijas como holocaustos paganos, y practicaban
la adivinación y la brujería. Se dedicaron a hacer el mal a los ojos del
Señor, haciéndolo enojar. \footnote{\textbf{17:17} 2Re 16,3}
\bibleverse{18} Así que el Señor se enfadó mucho con Israel, y los
desterró de su presencia. Sólo quedó la tribu de Judá,

\bibleverse{19} pero ni siquiera Judá guardó los mandamientos del Señor,
su Dios, sino que siguió la idolatría que Israel había introducido.
\bibleverse{20} El Señor se desentendió de todos los descendientes de
Israel. Los castigó y los entregó a sus enemigos,\footnote{\textbf{17:20}
  ``Enemigos'': Literalmente, ``saqueadores''.} hasta que los desterró
de su presencia.

\hypertarget{las-causas-que-provocaron-el-rechazo-y-la-cauxedda-del-reino-del-norte}{%
\subsection{Las causas que provocaron el rechazo y la caída del reino
del
norte}\label{las-causas-que-provocaron-el-rechazo-y-la-cauxedda-del-reino-del-norte}}

\bibleverse{21} Cuando el Señor arrancó a Israel de la casa de David,
hicieron rey a Jeroboam, hijo de Nabat. Jeroboam alejó a Israel del
Señor y les hizo cometer pecados terribles. \footnote{\textbf{17:21} 1Re
  12,20} \bibleverse{22} Los israelitas siguieron practicando todos los
pecados que cometió Jeroboam. No dejaron de cometerlos, \bibleverse{23}
así que el Señor terminó por expulsarlos de su presencia, tal como había
dicho que lo haría a través de todos sus siervos, los profetas. Así que
los israelitas fueron deportados de su tierra y llevados a Asiria, donde
se encuentran hasta hoy.

\hypertarget{repoblaciuxf3n-del-pauxeds-origen-de-los-samaritanos-y-su-religiuxf3n}{%
\subsection{Repoblación del país; Origen de los samaritanos y su
religión}\label{repoblaciuxf3n-del-pauxeds-origen-de-los-samaritanos-y-su-religiuxf3n}}

\bibleverse{24} El rey de Asiria trajo gente de Babilonia, de Cuta, de
Avá, de Jamat y de Sefarvaim y los estableció en las ciudades de Samaria
en lugar de los israelitas. Ellos se apoderaron de la propiedad de
Samaria y vivieron en sus ciudades. \bibleverse{25} Cuando empezaron a
vivir allí no adoraron al Señor, por lo que éste envió leones entre
ellos, matando a algunos de ellos. \bibleverse{26} Entonces fueron a
decirle al rey de Asiria: ``Los pueblos que tú trajiste y estableciste
en las ciudades de Samaria no conocen las reglas del Dios de la tierra.
En consecuencia, él ha enviado entre ellos leones que los están matando
porque no conocen lo que el Dios de la tierra exige''.

\bibleverse{27} El rey de Asiria dio la orden: ``Envía de vuelta a uno
de los sacerdotes que deportaste de Samaria, y que vuelva a vivir allí y
enseñe las reglas del Dios de la tierra''.

\bibleverse{28} Así que uno de los sacerdotes que había sido deportado
de Samaria regresó a vivir en Betel y les enseñó cómo adorar al Señor.

\bibleverse{29} Pero los pueblos de las distintas naciones siguieron
haciendo sus propios dioses en las ciudades donde se habían establecido,
y los colocaron en los santuarios de los lugares altos que había hecho
el pueblo de Samaria. \bibleverse{30} Los de Babilonia hicieron a Sucot
Benot, los de Cuta hicieron a Nergal y los de Jamat hicieron a Asimá.
\bibleverse{31} Losavitas hicieron un Nibhaz y un Tartac, y los
sefarvitas sacrificaron a sus hijos como holocaustos a sus dioses
Adramelec y Anamelec. \footnote{\textbf{17:31} 2Re 17,17}
\bibleverse{32} Mientras adoraban al Señor, también designaban
sacerdotes de toda clase de su propio pueblo para que ofrecieran
sacrificios por ellos en los santuarios de los lugares altos.
\bibleverse{33} Así que, aunque adoraban al Señor, también adoraban a
sus propios dioses siguiendo las prácticas de sus naciones de origen.
\bibleverse{34} Hasta el día de hoy siguen sus prácticas antiguas.
Ninguno de ellos adora verdaderamente al Señor ni observa los
reglamentos, requisitos, leyes y mandamientos que el Señor dio a los
descendientes de Jacob, al que llamó Israel. \bibleverse{35} Porque el
Señor había hecho un acuerdo con los israelitas, ordenándoles: ``No
adoren a otros dioses ni se inclinen ante ellos; no les sirvan ni les
ofrezcan sacrificios. \bibleverse{36} Sólo deben adorar al Señor, que
los sacó de Egipto, ayudándolos con su gran poder y su fuerte brazo.
Inclínense solo ante él y ofrézcanle sacrificios solo a él.
\bibleverse{37} Tengan cuidado siempre de observar los reglamentos, las
normas, las leyes y los mandamientos que él te dio por escrito, y no
adoren a otros dioses. \bibleverse{38} No olviden el acuerdo que he
hecho con ustedes, y no adoren a otros dioses. \bibleverse{39} Solo
deben adorar al Señor, su Dios, y él los salvará de todos sus
enemigos''. \footnote{\textbf{17:39} Deut 6,12-19}

\bibleverse{40} Pero ellos se negaron a escuchar, y continuaron con sus
antiguas prácticas idólatras. \bibleverse{41} Incluso cuando estas
personas de diferentes naciones adoraban al Señor, en realidad estaban
adorando a sus ídolos. Sus hijos y nietos siguen haciendo lo mismo que
sus antepasados hasta el día de hoy.

\hypertarget{ezequuxedas-asumiuxf3-el-cargo-su-piedad-y-sus-servicios-al-culto-y-al-bien-puxfablico}{%
\subsection{Ezequías asumió el cargo, su piedad y sus servicios al culto
y al bien
público}\label{ezequuxedas-asumiuxf3-el-cargo-su-piedad-y-sus-servicios-al-culto-y-al-bien-puxfablico}}

\hypertarget{section-17}{%
\section{18}\label{section-17}}

\bibleverse{1} Ezequías, hijo de Acaz, llegó a ser rey de Judá en el
tercer año del reinado de Oseas, hijo de Ela, rey de Israel.
\bibleverse{2} Tenía veinticinco años cuando llegó a ser rey, y reinó en
Jerusalén durante veintinueve años. Su madre se llamaba Abi, hija de
Zacarías. \footnote{\textbf{18:2} 2Cró 29,1-2} \bibleverse{3} E hizo lo
justo alos ojos del Señor, siguiendo todo lo que había hecho su
antepasado David. \footnote{\textbf{18:3} 2Re 20,3} \bibleverse{4} Quitó
los lugares altos, destrozó los ídolos de piedra y cortó los postes de
Asera. Hizo pedazos la serpiente de bronce que había hecho Moisés,
porque hasta entonces los israelitas le habían sacrificado ofrendas. Se
llamaba Nehustán. \footnote{\textbf{18:4} 2Cró 31,1; 2Re 15,35; Núm
  21,8-9} \bibleverse{5} Ezequías puso su confianza en el Señor, el Dios
de Israel. Entre los reyes de Judá no hubo nadie como él, ni antes ni
después. \footnote{\textbf{18:5} 2Re 23,25} \bibleverse{6} Se mantuvo
fiel al Señor y no dejó de seguirlo. Guardó los mandamientos que el
Señor había dado a Moisés. \bibleverse{7} El Señor estaba con él; tuvo
éxito en todo lo que hizo. Desafió al rey de Asiria y se negó a
someterse a él. \bibleverse{8} Derrotó a los filisteos hasta Gaza y sus
alrededores, desde la torre de vigilancia hasta la ciudad fortificada.

\hypertarget{la-cauxedda-de-samaria}{%
\subsection{La caída de Samaria}\label{la-cauxedda-de-samaria}}

\bibleverse{9} En el cuarto año del reinado de Ezequías, equivalente al
séptimo año del reinado de Oseas, hijo de Ela, rey de Israel,
Salmanasar, rey de Asiria, atacó Samaria, sitiándola. \footnote{\textbf{18:9}
  2Re 17,3-6} \bibleverse{10} Los asirios la conquistaron después de
tres años. Esto ocurrió durante el sexto año de Ezequías, equivalente al
noveno año de Oseas, rey de Israel. \bibleverse{11} El rey de Asiria
deportó a los israelitas a Asiria. Los asentó en Halah, en Gozán, sobre
el río Jabor, y en las ciudades de los medos. \bibleverse{12} Esto
sucedió porque se negaron a escuchar al Señor, su Dios, y rompieron su
acuerdo: todo lo que Moisés, el siervo del Señor, había ordenado. Se
negaron a escuchar y no obedecieron.

\hypertarget{ezequuxedas-envuxeda-sin-uxe9xito-el-tributo-exigido-por-senaquerib}{%
\subsection{Ezequías envía sin éxito el tributo exigido por
Senaquerib}\label{ezequuxedas-envuxeda-sin-uxe9xito-el-tributo-exigido-por-senaquerib}}

\bibleverse{13} Senaquerib, rey de Asiria, atacó y conquistó todas las
ciudades fortificadas de Judá en el año catorce del reinado de Ezequías.
\bibleverse{14} Entonces Ezequías, rey de Judá, envió un mensaje al rey
de Asiria que estaba en Laquis, diciendo: ``¡He cometido un terrible
error! Por favor, retírate y déjame en paz, ¡y te pagaré lo que
quieras!'' El rey de Asiria exigió a Ezequías, rey de Judá, el pago de
trescientos talentos de plata y treinta talentos de oro. \bibleverse{15}
Ezequías le pagó usando toda la plata del Templo del Señor y de los
tesoros del palacio real. \footnote{\textbf{18:15} 2Re 16,8}
\bibleverse{16} Incluso se despojó del oro que había utilizado para
recubrir las puertas y los postes del Templo del Señor y se lo dio todo
al rey de Asiria.

\hypertarget{desde-laquis-senaquerib-hace-que-la-ciudad-de-jerusaluxe9n-sea-convocada-desdeuxf1osamente-a-rendirse-por-su-gran-visir}{%
\subsection{Desde Laquis, Senaquerib hace que la ciudad de Jerusalén sea
convocada desdeñosamente a rendirse por su gran
visir}\label{desde-laquis-senaquerib-hace-que-la-ciudad-de-jerusaluxe9n-sea-convocada-desdeuxf1osamente-a-rendirse-por-su-gran-visir}}

\bibleverse{17} Aun así, el rey de Asiria envió a su comandante en jefe,
a su oficial principal y a su general del ejército,\footnote{\textbf{18:17}
  Literalmente, ``Tartan, Rab-saris, y Rabsakeh''. Sin embargo, son
  títulos asirios, no nombres personales.} junto con un gran ejército,
desde Laquis hasta el rey Ezequías en Jerusalén. Se acercaron a
Jerusalén y acamparon junto al acueducto del estanque superior, en el
camino hacia donde se lava la ropa. \bibleverse{18} Entonces llamaron al
rey. Salieron a hablar con ellos Eliaquim, hijo de Jilquías, el
administrador del palacio, Sebná, el escriba, y Joa, hijo de Asaf, el
secretario que llevaba el archivo. \bibleverse{19} El general del
ejército asirio les dijo: ``Dile a Ezequías que esto es lo que dice el
gran rey, el rey de Asiria: ¿En qué confías que tesientes con tanta
seguridad? \bibleverse{20} Dicestener una estrategia y que estás listo
para la guerra, pero esas son palabras vacías. ¿En quién confías, ahora
que te has rebelado contra mí? \bibleverse{21} ¡Cuidado! Estás confiando
en Egipto, un bastón que es como una caña rota que atravesará la mano de
quien se apoye en ella. Así es el Faraón, rey de Egipto, para todos los
que confían en él. \bibleverse{22} ``Y si me dicen: `Confiamos en el
Señor nuestro Dios', ¿acaso no quitó Ezequías sus lugares altos y sus
altares, diciéndole a Judá y a Jerusalén: `Tienen que adorar en este
altar de Jerusalén'? \bibleverse{23} ``¿Por qué no aceptan el desafío de
mi amo, el rey de Asiria? Él dice: ¡Te daré dos mil caballos, si puedes
encontrar suficientes jinetes para ellos! \bibleverse{24} ¿Cómo podrías
derrotar siquiera a un solo oficial a cargo de los hombres más débiles
de mi amo, cuando confías en Egipto para obtener carros y jinetes?
\bibleverse{25} Más aún: ¿habría venido a atacar a este paso sin el
aliento del Señor? Fue el Señor mismo quien me dijo: `Ve y ataca esta
tierra y destrúyela'\,''.

\hypertarget{senaquerib-y-la-arrogancia-de-sus-embajadores}{%
\subsection{Senaquerib y la arrogancia de sus
embajadores}\label{senaquerib-y-la-arrogancia-de-sus-embajadores}}

\bibleverse{26} Entonces Eliaquim, hijo de Jilquías, junto con Sebná y
Joa, le dijeron al general del ejército: ``Por favor, háblanos a
nosotros, tus siervos, en arameo, para que podamos entender. No nos
hables en hebreo mientras la gente de la muralla esté escuchando''.

\bibleverse{27} Pero el general del ejército respondió: ``¿Acaso mi amo
me envió a decirles estas cosas a tu amo y a ti, y no a la gente que
está sentada en el muro? También ellos, al igual que ustedes, van a
tener que comer sus propios excrementos y beber su propia orina''.

\bibleverse{28} Entonces el general del ejército gritó en hebreo:
``¡Escuchen esto de parte del gran rey, el rey de Asiria!
\bibleverse{29} Esto es lo que dice el rey: ¡No se dejen engañar por
Ezequías! ¡No puede salvarlos de mí! \bibleverse{30} Nole crean a
Ezequías cuando les diga que confíen en el Señor, diciendo: `Estoy
seguro de que el Señor nos salvará. Esta ciudad nunca caerá en manos del
rey de Asiria'. \bibleverse{31} No escuchen a Ezequías. Esto es lo que
dice el rey: Haz un tratado de paz conmigo y ríndete a mí. Así cada uno
comerá de su propia vid y de su propia higuera, y beberá agua de su
propio pozo. \footnote{\textbf{18:31} 1Re 5,5} \bibleverse{32} Vendré y
los llevaré a una tierra como la suya, una tierra de grano y vino nuevo,
una tierra de pan y viñedos, una tierra de olivos y miel. Entonces
vivirán y no morirán. ``Pero no escuchen a Ezequías, pues los está
engañando cuando dice: `El Señor nos librará'. \bibleverse{33} ¿Acaso
alguno de los dioses de alguna nación ha salvado su tierra del poder del
rey de Asiria? \bibleverse{34} ¿Dónde estaban los dioses de Jamat y
Arpad? ¿Dónde estaban los dioses de Sefarvaim, Hená e Ivá? ¿Pudieron
ellos salvar a Samaria de mí? \bibleverse{35} ¿Cuál de todos los dioses
de estos países ha salvado su nación de mí? ¿Cómo podría entonces el
Señor salvar a Jerusalén de mí?''

\bibleverse{36} Pero el pueblo permaneció en silencio y no dijo nada,
pues Ezequías había dado la orden: ``No le respondan''. \bibleverse{37}
Entonces Eliaquim, hijo de Jilquías, el administrador del palacio,
Sebná, el escriba, y Joa, hijo de Asaf, el secretario, fueron a Ezequías
con las ropas rasgadas, y le contaron lo que había dicho el general del
ejército asirio.

\hypertarget{el-estuxedmulo-de-ezequuxedas-de-isauxedas}{%
\subsection{El estímulo de Ezequías de
Isaías}\label{el-estuxedmulo-de-ezequuxedas-de-isauxedas}}

\hypertarget{section-18}{%
\section{19}\label{section-18}}

\bibleverse{1} Cuando Ezequías lo oyó, se rasgó las vestiduras, se
vistió de cilicio y entró en el Templo del Señor. \bibleverse{2}
Entonces envió a Eliaquim, el administrador del palacio, a Sebná, el
escriba, y a los principales sacerdotes, todos vestidos de saco, a ver
al profeta Isaías, hijo de Amoz. \bibleverse{3} Ellos le dijeron: ``Esto
es lo que dice Ezequías: Hoy es un día de angustia, de castigo. Es como
cuando los bebés llegan a la entrada del canal de parto, pero no hay
fuerzas para darlos a luz. \bibleverse{4} Tal vez el Señor, tu Dios, al
oír el mensaje que el comandante del ejército entregó en nombre de su
amo, el rey de Asiria -- unmensaje enviado para insultar al Dios vivo --
locastigue por sus palabras. Por favor, haz una oración por el remanente
de nosotros que aún sobrevive''. \footnote{\textbf{19:4} 2Re 18,35}

\bibleverse{5} Después de que los funcionarios de Ezequías le entregaron
su mensaje a Isaías, \bibleverse{6} éste les respondió: ``Díganle a su
amo: Esto es lo que dice el Señor: `No te asustes por las palabras que
has oído, las que usan los servidores del rey de Asiria para blasfemar
contra mí. \bibleverse{7} Nota cómo voy a asustarlo: oirá un rumor y
tendrá que volver a su país. Cuando esté allí lo haré morir a
espada'\,''.

\hypertarget{la-segunda-solicitud-de-sennacherib-a-travuxe9s-de-una-carta-amenazante-de-libna}{%
\subsection{La segunda solicitud de Sennacherib a través de una carta
amenazante de
Libna}\label{la-segunda-solicitud-de-sennacherib-a-travuxe9s-de-una-carta-amenazante-de-libna}}

\bibleverse{8} El comandante del ejército asirio se marchó y regresó
para reunirse con el rey de Asiria, tras oír que el rey había salido de
Laquis y estaba atacando Libna. \bibleverse{9} Senaquerib había recibido
un mensaje sobre Tirhaca, rey de Etiopía, que decía: ``¡Cuidado! Se ha
propuesto atacarte''. Entonces Senaquerib volvió a enviar mensajeros a
Ezequías, diciendo: \bibleverse{10} ``Dile a Ezequías, rey de Judá: `No
dejes que tu Dios, en el que confías, te engañe diciendo que Jerusalén
no caerá en manos del rey de Asiria. \footnote{\textbf{19:10} 2Re 18,30}
\bibleverse{11} ¡Cuidado! Has oído lo que los reyes de Asiria han hecho
a todos los países que han invadido\footnote{\textbf{19:11} ``Han
  invadido'': implícito.} --- ¡los destruyeron por completo! ¿Realmente
creen que se salvarán? \bibleverse{12} ¿Acaso los salvaron los dioses de
las naciones que mis antepasados destruyeron, los dioses de Gozán,
Harán, Rezef y el pueblo de Edén, que vivía en Telasar? \bibleverse{13}
¿Dónde está hoy el rey de Jamat, el rey de Arpad, el rey de la ciudad de
Sefarvaim, el rey de Hená o el rey de Iáa?'\,''

\hypertarget{la-suxfaplica-de-ezequuxedas-en-el-templo}{%
\subsection{La súplica de Ezequías en el
templo}\label{la-suxfaplica-de-ezequuxedas-en-el-templo}}

\bibleverse{14} Ezequías recibió la carta de los mensajeros y la leyó.
Luego subió al Templo del Señor y la abrió ante el Señor.
\bibleverse{15} Entonces Ezequías oró al Señor diciendo: ``Señor, Dios
de Israel, tú que vives encima de los querubines, sólo tú eres Dios
sobre todos los reinos de la tierra, tú eres el Creador del cielo y de
la tierra. \footnote{\textbf{19:15} Éxod 25,22; Sal 80,2}
\bibleverse{16} Por favor, escucha con tus oídos, Señor, y oye; abre tus
ojos, Señor, y mira. Escucha el mensaje que Senaquerib ha enviado para
insultar al Dios vivo. \footnote{\textbf{19:16} 2Re 19,4; 1Sam 17,10}
\bibleverse{17} ``Sí, es cierto, Señor, que los reyes asirios han
destruido estas naciones y sus tierras. \bibleverse{18} Han arrojado sus
dioses al fuego porque no son realmente dioses; son sólo obra de manos
humanas, hechos de madera y piedra para poder destruirlos.
\bibleverse{19} Ahora, Señor, Dios nuestro, sálvanos de él, para que
todos los reinos de la tierra sepan que sólo tú, Señor, eres Dios''.

\hypertarget{isauxedas-envuxeda-notificaciuxf3n-de-su-oraciuxf3n-al-rey-ezequuxedas-en-el-nombre-de-dios}{%
\subsection{Isaías envía notificación de su oración al rey Ezequías en
el nombre de
Dios}\label{isauxedas-envuxeda-notificaciuxf3n-de-su-oraciuxf3n-al-rey-ezequuxedas-en-el-nombre-de-dios}}

\bibleverse{20} Entonces Isaías, hijo de Amoz, envió un mensaje a
Ezequías, diciendo: ``Esto es lo que dice el Señor, el Dios de Israel:
He escuchado tu oración sobre Senaquerib, rey de Asiria. \bibleverse{21}
Esta es la palabra con la que el Señor lo condena: La virgen hija de
Sión te desprecia y se burla de ti; la hija de Jerusalén mueve la cabeza
cuando huyes. \bibleverse{22} ¿A quién has insultado y ridiculizado?
¿Contra quién has levantado la voz? ¿A quién miraste con ojos tan
orgullosos? ¡Fue contra el Santo de Israel! \bibleverse{23} Por medio de
tus siervos te has burlado del Señor. Dijiste: `Con mis muchos carros he
subido a las altas montañas, a las más lejanas cumbres del Líbano. He
cortado sus cedros más altos, los mejores cipreses. He llegado a sus
puestos más lejanos, a sus bosques más profundos. \bibleverse{24} He
cavado pozos y bebido agua en tierras extranjeras. Con las plantas de
mis pies he secado todos los ríos de Egipto'\,''. \bibleverse{25} El
Señor responde:\footnote{\textbf{19:25} ``El Señor responde'': Añadido
  para mayor claridad.} ``¿No te has enterado? Lo decidí hace mucho
tiempo; lo planeé en los viejos tiempos. Ahora me estoy asegurando de
que ocurra, de que derribes las ciudades fortificadas hasta convertirlas
en montones de escombros. \bibleverse{26} Supueblo, impotente, está
aterrorizado y humillado. Son como plantas en un campo, como brotes
verdes y blandos, como hierba que brota en un tejado: están quemados
antes de que puedan crecer. \bibleverse{27} ``Pero yo te conozco muy
bien: dónde vives, cuándo entras, cuándo sales, y tu furia contra mí.
\bibleverse{28} A causa de tu fiera ira contra mí, y porque sé cómo me
faltas al respeto, voy a poner mi garfio en tu nariz y mi bocado en tu
boca, y te obligaré a regresar por donde viniste''.

\bibleverse{29} ``Ezequías, esta será una señal para demostrar que esto
es cierto:\footnote{\textbf{19:29} ``Para demostrar que esto es
  cierto'': implícito.} Este año comerás lo que crezca solo. El segundo
año comerás lo que crezca por sí mismo. Pero el tercer año sembrarás y
cosecharás, plantarás viñas y comerás su fruto. \bibleverse{30} El
remanente que quede de Judá revivirá de nuevo, echando raíces abajo y
dando frutos arriba. \bibleverse{31} Porque de Jerusalén saldrá un
remanente, y del monte Sión vendrán supervivientes. La intensa
determinación del Señor se encargará de que esto ocurra. \footnote{\textbf{19:31}
  Is 9,6}

\bibleverse{32} Esto es lo que dice el Señor sobre el rey de Asiria: No
entrará en esta ciudad ni lanzará una flecha contra ella. No avanzará
hacia ella con un escudo, ni construirá una rampa de asedio contra ella.
\bibleverse{33} Volverá por donde vino y no entrará en esta ciudad, dice
el Señor. \bibleverse{34} Yo defenderé esta ciudad y la salvaré, por mí
y por mi siervo David''.

\hypertarget{el-cumplimiento-de-la-promesa-la-partida-y-el-asesinato-de-senaquerib}{%
\subsection{El cumplimiento de la promesa: la partida y el asesinato de
Senaquerib}\label{el-cumplimiento-de-la-promesa-la-partida-y-el-asesinato-de-senaquerib}}

\bibleverse{35} Aquella noche el ángel del Señor fue al campamento
asirio y mató a 185. 000 personas. Cuando los supervivientes se
despertaron por la mañana, estaban rodeados de cadáveres.
\bibleverse{36} Senaquerib, rey de Asiria, se rindió y se fue. Regresó a
su casa en Nínive y se quedó allí. \bibleverse{37} Mientras adoraba en
el templo de su dios Nisroc, sus hijos Adramelec y Sarézer lo mataron
con la espada y luego huyeron a la tierra de Ararat. Su hijo Esar-hadón
le sucedió como rey.\footnote{\textbf{19:37} 2Re 19,7}

\hypertarget{la-enfermedad-y-la-recuperaciuxf3n-de-ezequuxedas-la-embajada-de-babilonia}{%
\subsection{La enfermedad y la recuperación de Ezequías; la embajada de
babilonia}\label{la-enfermedad-y-la-recuperaciuxf3n-de-ezequuxedas-la-embajada-de-babilonia}}

\hypertarget{section-19}{%
\section{20}\label{section-19}}

\bibleverse{1} Por aquel entonces Ezequías cayó muy enfermo y estaba a
punto de morir. El profeta Isaías, hijo de Amoz, fue a verlo y le dijo:
``Esto es lo que dice el Señor: Pon en orden tus asuntos, porque vas a
morir. No te recuperarás''.

\bibleverse{2} Cuando Ezequías escuchó esto, fue a orar en
privado\footnote{\textbf{20:2} ``En privado'': Literalmente, ``se puso
  de cara a la pared''.} al Señor, diciendo \bibleverse{3} ``Por favor,
recuerda Señor cómo te he seguido fielmente con todo mi corazón. He
hecho lo que es bueno a tus ojos''. Entonces Ezequías gritó y lloró.

\bibleverse{4} Antes de que Isaías saliera del patio central, el Señor
le habló diciendo: \bibleverse{5} ``Vuelve a entrar y dile a Ezequías,
el gobernante de mi pueblo: Esto es lo que dice el Señor, el Dios de tu
antepasado David: He oído tu oración, he visto tus lágrimas. Te prometo
que te voy a curar. Por eso, dentro de tres días irás al Templo del
Señor. \bibleverse{6} Añadiré quince años a tu vida. Te salvaré a ti y a
esta ciudad del rey de Asiria. Defenderé esta ciudad por mí y por mi
siervo David''.

\bibleverse{7} Entonces Isaías dijo: ``Preparen un aderezo de higos''.
Los siervos de Ezequías así lo hicieron y lo pusieron sobre las llagas
de la piel, y Ezequías mejoró.

\hypertarget{el-signo-del-milagro-divino-en-el-reloj-de-sol}{%
\subsection{El signo del milagro divino en el reloj de
sol}\label{el-signo-del-milagro-divino-en-el-reloj-de-sol}}

\bibleverse{8} Ezequías había preguntado antes a Isaías: ``¿Cuál es la
señal que confirma que el Señor me va a curar y que voy a ir al Templo
del Señor dentro de tres días?''

\bibleverse{9} Isaías respondió: ``Esta es la señal del Señor para ti de
que el Señor hará lo que prometió: ¿Quieres que la sombra avance diez
pasos, o que retroceda diez pasos?''

\bibleverse{10} ``Es bastante fácil que la sombra avance diez pasos,
pero no que retroceda diez pasos'', respondió Ezequías.

\bibleverse{11} Entonces el profeta Isaías le pidió esto al Señor, y él
hizo retroceder la sombra los diez pasos que había bajado en la escalera
de Acaz.

\hypertarget{embajada-de-merodac-baladan-de-babilonia}{%
\subsection{Embajada de Merodac-Baladan de
Babilonia}\label{embajada-de-merodac-baladan-de-babilonia}}

\bibleverse{12} Al mismo tiempo, Merodac-Baladán, hijo de Baladán, rey
de Babilonia, envió cartas y un regalo a Ezequías, porque había oído que
éste estaba enfermo. \bibleverse{13} Ezequías recibió a los visitantes y
les mostró todo lo que había en su tesoro: toda la plata, el oro, las
especias y los aceites costosos. También les mostró su arsenal y todo lo
que tenía en sus almacenes. De hecho, no había nada en su palacio ni en
todo su reino que Ezequías no les mostrara.

\hypertarget{el-discurso-de-castigo-de-isauxedas-sobre-la-pompa-descuidada-del-rey-y-su-profecuxeda-sobre-el-cautiverio-en-babilonia}{%
\subsection{El discurso de castigo de Isaías sobre la pompa descuidada
del rey y su profecía sobre el cautiverio en
Babilonia}\label{el-discurso-de-castigo-de-isauxedas-sobre-la-pompa-descuidada-del-rey-y-su-profecuxeda-sobre-el-cautiverio-en-babilonia}}

\bibleverse{14} Entonces el profeta Isaías fue a ver al rey Ezequías y
le preguntó: ``¿De dónde vinieron esos hombres y qué te dijeron?'' .
``Vinieron de muy lejos, de Babilonia'', respondió Ezequías.

\bibleverse{15} ``¿Qué vieron en tu palacio?'' preguntó Isaías. ``Vieron
todo en mi palacio'', respondió Ezequías. ``No hubo nada en todos mis
almacenes que no les mostrara''.

\bibleverse{16} Isaías le dijo a Ezequías: ``Escucha lo que dice el
Señor: \bibleverse{17} Puedes estar seguro de que se acerca el momento
en que todo lo que hay en tu palacio, y todo lo que tus antepasados han
guardado hasta ahora, será llevado a Babilonia. No quedará nada, dice el
Señor. \footnote{\textbf{20:17} 2Re 24,13-14} \bibleverse{18} Algunos de
tus hijos, tus propios descendientes, serán llevados para servir como
eunucos en el palacio del rey de Babilonia''. \footnote{\textbf{20:18}
  Dan 1,3-4}

\hypertarget{la-respuesta-devota-pero-impenitente-de-ezequuxedas}{%
\subsection{La respuesta devota pero impenitente de
Ezequías}\label{la-respuesta-devota-pero-impenitente-de-ezequuxedas}}

\bibleverse{19} Entonces Ezequías le dijo a Isaías: ``El mensaje del
Señor que me has contado está bien''. Pues pensó: ``Después de todo,
habrá paz y seguridad durante mis años de vida''. \footnote{\textbf{20:19}
  1Sam 3,18}

\bibleverse{20} El resto de lo que sucedió en el reinado de Ezequías,
todo lo que hizo, y cómo hizo el estanque y el túnel para llevar agua a
la ciudad, están registrados en el Libro de las Crónicas de los Reyes de
Judá. \bibleverse{21} Ezequías murió y su hijo Manasés lo sucedió como
rey.

\hypertarget{idolatruxeda-manasuxe9s}{%
\subsection{Idolatría manasés}\label{idolatruxeda-manasuxe9s}}

\hypertarget{section-20}{%
\section{21}\label{section-20}}

\bibleverse{1} Manasés tenía doce años cuando se convirtió en rey, y
reinó en Jerusalén durante cincuenta y cinco años. El nombre de su madre
era Hefzibá. \bibleverse{2} Sus hechos fueron malos a los ojos del
Señor, al seguir las repugnantes prácticas paganas de las naciones que
el Señor había expulsado ante los israelitas. \bibleverse{3} Reconstruyó
los lugares altos que su padre Ezequías había destruido, y levantó
altares para Baal. Hizo un poste de ídolos de Asera, tal como lo había
hecho Acab, rey de Israel, y adoró y sirvió al sol, la luna y las
estrellas. \bibleverse{4} Levantó altares paganos en el Templo del
Señor, justo donde el Señor había dicho: ``Pondré mi nombre en Jerusalén
para siempre''. \footnote{\textbf{21:4} 2Re 21,7} \bibleverse{5} Erigió
altares para adorar al sol, la luna y las estrellas en los dos patios
del Templo del Señor. \footnote{\textbf{21:5} 2Re 23,12} \bibleverse{6}
Incluso sacrificó a su propio hijo como holocausto, y utilizó la
adivinación y la brujería, y trató con médiums y espiritistas. Hizo
mucho mal a los ojos del Señor, haciendo que éste se enojara.
\footnote{\textbf{21:6} 2Re 16,3} \bibleverse{7} Tomó el poste del ídolo
de Asera que había hecho y lo colocó en el Templo. Este era el lugar al
que se refería el Señor cuando les dijo a David y a Salomón, su hijo:
``En este Templo y en Jerusalén, que he elegido entre todas las tribus
de Israel, pondré mi nombre para siempre. \footnote{\textbf{21:7} 1Re
  8,29; 1Re 9,3} \bibleverse{8} Nunca más haré que los israelitas se
alejen de la tierra que les di a sus antepasados si tienen cuidado de
seguir todo lo que les he ordenado: toda la ley que les dio mi siervo
Moisés''. \bibleverse{9} El pueblo se negó a escuchar y Manasés los
llevó a pecar, de modo que el mal que hicieron fue aún peor que el de
las naciones que el Señor había destruido antes de los israelitas.

\hypertarget{la-amenaza-de-dios-a-manasuxe9s-la-crueldad-de-manasuxe9s-y-las-palabras-finales-sobre-uxe9l}{%
\subsection{La amenaza de Dios a Manasés; La crueldad de Manasés y las
palabras finales sobre
él}\label{la-amenaza-de-dios-a-manasuxe9s-la-crueldad-de-manasuxe9s-y-las-palabras-finales-sobre-uxe9l}}

\bibleverse{10} El Señor dijo por medio de sus siervos los profetas
\bibleverse{11} ``Puesto que Manasés, rey de Judá, ha cometido todos
estos repugnantes pecados, haciendo cosas aún más malas que los amorreos
que vivieron antes que él, y con su fomento del culto a los ídolos ha
hecho pecar a Judá, \bibleverse{12} esto es lo que dice el Señor, el
Dios de Israel: ¡Cuidado! Voy a hacer caer sobre Jerusalén y Judá un
desastre tal que hará zumbar los oídos de todo el que lo oiga.
\footnote{\textbf{21:12} 1Sam 3,11; Jer 19,3} \bibleverse{13} Extenderé
sobre Jerusalén el cordel de medir usado contra Samaria y la plomada
usada contra la casa de Acab,\footnote{\textbf{21:13} En otras palabras,
  Dios dice que aplicará el mismo criterio para juzgar a Judá que el que
  aplicó a Israel.} y limpiaré a Jerusalén como se limpia un cuenco,
limpiándolo y dándole la vuelta. \bibleverse{14} Abandonaré al resto de
mi pueblo especial y lo entregaré a sus enemigos. Serán despojo y botín
para todos sus enemigos, \bibleverse{15} porque han hecho lo que es malo
a mis ojos y me han hecho enojar desde el día en que sus padres salieron
de Egipto hasta hoy''.

\bibleverse{16} Además, Manasés asesinó a tantos inocentes que Jerusalén
se llenó de un lado a otro con su sangre. Esto se sumaba al pecado que
había hecho cometer a Judá, haciendo el mal a los ojos del Señor.

\bibleverse{17} El resto de lo que sucedió en el reinado de Manasés,
todo lo que hizo, así como los pecados que cometió, están registrados en
el Libro de las Crónicas de los Reyes de Judá. \bibleverse{18} Manasés
murió y fue enterrado en el jardín de su palacio, el jardín de Uzza. Su
hijo Amón le sucedió como rey.

\hypertarget{amon-von-juda}{%
\subsection{Amon von Juda}\label{amon-von-juda}}

\bibleverse{19} Amón tenía veintidós años cuando llegó a ser rey, y
reinó en Jerusalén durante dos años. Su madre se llamaba Mesulémet, hija
de Jaruz. Ella era de Jotba. \footnote{\textbf{21:19} 2Cró 33,21-22;
  2Cró 33,24-25} \bibleverse{20} Sus hechos fueron malos a los ojos del
Señor, tal como los de su padre Manasés. \bibleverse{21} Siguió todos
los caminos de su padre, y sirvió a los ídolos que su padre había
servido, inclinándose en adoración a ellos. \bibleverse{22} Rechazó al
Señor, el Dios de sus antepasados, y no siguió el camino del Señor.
\bibleverse{23} Los funcionarios de Amón conspiraron contra él y lo
asesinaron en su palacio real. \bibleverse{24} Pero entonces el pueblo
del país mató a todos los que habían conspirado contra el rey Amón, y
eligieron a su hijo Josías como rey para sucederlo. \bibleverse{25} El
resto de lo que sucedió en el reinado de Amón, y todo lo que hizo, están
registrados en el Libro de las Crónicas de los Reyes de Judá.
\bibleverse{26} Fue enterrado en su tumba en el jardín de Uza, y su hijo
Josías lo sucedió como rey.\footnote{\textbf{21:26} 2Re 21,18}

\hypertarget{el-rey-josuxedas-encontrar-el-cuxf3digo-legal-y-limpiar-la-adoraciuxf3n}{%
\subsection{El rey Josías; Encontrar el código legal y limpiar la
adoración}\label{el-rey-josuxedas-encontrar-el-cuxf3digo-legal-y-limpiar-la-adoraciuxf3n}}

\hypertarget{section-21}{%
\section{22}\label{section-21}}

\bibleverse{1} Josías tenía ocho años cuando se convirtió en rey, y
reinó en Jerusalén durante treinta y un años. Su madre se llamaba
Jedidá, hija de Adaías. Ella era de Bozkat. \footnote{\textbf{22:1} 2Cró
  34,1-2; 2Cró 34,8-11} \bibleverse{2} E hizo lo recto a los ojos del
Señor, y siguió todos los caminos de David, su antepasado; no se desvió
ni a la derecha ni a la izquierda. \footnote{\textbf{22:2} 2Re 18,3;
  Deut 5,29}

\hypertarget{josuxedas-se-encarga-de-la-reparaciuxf3n-del-templo-informe-sobre-el-descubrimiento-del-cuxf3digo-y-su-primer-efecto}{%
\subsection{Josías se encarga de la reparación del templo; Informe sobre
el descubrimiento del código y su primer
efecto}\label{josuxedas-se-encarga-de-la-reparaciuxf3n-del-templo-informe-sobre-el-descubrimiento-del-cuxf3digo-y-su-primer-efecto}}

\bibleverse{3} En el año dieciocho de su reinado, Josías envió a Safán,
hijo de Asalías, hijo de Mesulam, al Templo del Señor. Le dijo:
\bibleverse{4} ``Ve al sumo sacerdote Jilquías y dile que cuente el
dinero que los porteros han recogido de la gente que viene al Templo del
Señor. \bibleverse{5} Luego entrégalo a los que supervisan las obras del
Templo del Señor, y haz que les paguen a los obreros que reparan el
Templo del Señor, \bibleverse{6} a los carpinteros, a los constructores
y a los albañiles. Además, haz que compren madera y corten piedra para
reparar el Templo. \bibleverse{7} No les pidas cuentas a los hombres que
han recibido el dinero, porque ellos tratan con honestidad''.

\bibleverse{8} El sumo sacerdote Jilquías le dijo a Safán, el escriba:
``He encontrado el Libro de la Ley en el Templo del Señor''. Se lo dio a
Safán, quien lo leyó. \bibleverse{9} El escriba Safán fue a ver al rey y
a darle un informe, diciendo: ``Tus funcionarios han pagado el dinero
que estaba en el Templo del Señor y lo han entregado a los designados
para supervisar el trabajo en el Templo del Señor''. \bibleverse{10}
Entonces el escriba Safán le dijo al rey: ``El sacerdote Jilquías me ha
dado un libro''. Safán se lo leyó al rey.

\bibleverse{11} Cuando el rey oyó lo que había en el libro de la Ley, se
rasgó las vestiduras. \bibleverse{12} Luego dio órdenes al sacerdote
Jilquías, a Ahicam, hijo de Safán, a Acbor, hijo de Micaías, a Safán, el
escriba, y a Asaías, el ayudante del rey, diciendo: \bibleverse{13}
``Vayan y hablen con el Señor por mí, por el pueblo y por todo Judá,
sobre lo que dice el libro que se ha encontrado. Porque el Señor debe
estar realmente enojado con nosotros, porque nuestros antepasados no han
obedecido las instrucciones del Señor en este libro; no han hecho lo que
está escrito allí para que lo hagamos''.

\hypertarget{interrogatorio-y-respuesta-de-la-profetisa-hulda}{%
\subsection{Interrogatorio y respuesta de la profetisa
Hulda}\label{interrogatorio-y-respuesta-de-la-profetisa-hulda}}

\bibleverse{14} El sacerdote Jilquías, Ahicam, Acbor, Safán y Asaías
fueron y hablaron con la profetisa Huldá, esposa de Salum, hijo de
Ticvá, hijo de Jarjás, guardián del guardarropa.\footnote{\textbf{22:14}
  ``Guardarropa'': Puede ser de las vestiduras del rey o delos
  sacerdotes.} Vivía en Jerusalén, en el segundo barrio de la ciudad.
\bibleverse{15} Ella les dijo: ``Esto es lo que dice el Señor, el Dios
de Israel: Dile al hombre que te ha enviado a mí: \bibleverse{16} Esto
es lo que dice el Señor: Estoy a punto de hacer caer el desastre sobre
este lugar y sobre su pueblo, de acuerdo con todo lo que está escrito en
el libro que se ha leído al rey de Judá. \bibleverse{17} Me han
abandonado y han ofrecido sacrificios a otros dioses, haciéndome enojar
por todo lo que han hecho. Mi ira se derramará sobre este lugar y no se
detendrá. \footnote{\textbf{22:17} Deut 31,29; Deut 32,21-23}

\bibleverse{18} ``Pero dile al rey de Judá que te ha enviado a preguntar
al Señor, que le diga que esto es lo que dice el Señor, el Dios de
Israel: En cuanto a lo que oíste que te leyeron, \bibleverse{19} como te
conmoviste y te arrepentiste ante Dios cuando oíste sus advertencias
contra este lugar y contra su pueblo -- quese convertiría en desolación
y en maldición -- yporque te rasgaste las vestiduras y lloraste ante mí,
yo también te he oído,\footnote{\textbf{22:19} ``Oído'': en el sentido
  de una respuesta positiva.} declara el Señor. \bibleverse{20} Todo
esto no sucederá hasta después de tu muerte, y morirás en
paz.\footnote{\textbf{22:20} ``Morirás en paz'': Por supuesto, esto no
  ocurrió, porque Josías decidió enfrentarse al faraón egipcio en la
  batalla y fue asesinado. Ver 23:29.} No verás todo el desastre que voy
a hacer caer sobre este lugar''. Volvieron al rey y le dieron su
respuesta.

\hypertarget{josuxedas-concluye-el-nuevo-pacto-de-dios-en-asociaciuxf3n-con-los-ancianos-del-pueblo}{%
\subsection{Josías concluye el nuevo pacto de Dios en asociación con los
ancianos del
pueblo}\label{josuxedas-concluye-el-nuevo-pacto-de-dios-en-asociaciuxf3n-con-los-ancianos-del-pueblo}}

\hypertarget{section-22}{%
\section{23}\label{section-22}}

\bibleverse{1} Entonces el rey convocó a todos los ancianos de Judá y
Jerusalén. \bibleverse{2} Fue al Templo del Señor con todo el pueblo de
Judá y de Jerusalén, junto con los sacerdotes y los levitas, todo el
pueblo, desde el más pequeño hasta el más grande, y les leyó todo el
Libro del Acuerdo que había sido descubierto en el Templo del Señor.
\bibleverse{3} El rey se puso de pie junto a la columna e hizo un
acuerdo solemne ante el Señor de seguirlo y de cumplir sus mandamientos,
leyes y reglamentos con total dedicación, y de observar los requisitos
del acuerdo tal como estaban escritos en el libro. Todo el pueblo aceptó
el acuerdo. \footnote{\textbf{23:3} 2Re 11,14; Jos 24,25}

\hypertarget{josuxedas-limpia-el-templo-y-todo-el-culto-puxfablico}{%
\subsection{Josías limpia el templo y todo el culto
público}\label{josuxedas-limpia-el-templo-y-todo-el-culto-puxfablico}}

\bibleverse{4} Entonces el rey ordenó al sumo sacerdote Jilquías, a los
sacerdotes de segundo rango y a los porteros que sacaran del Templo del
Señor todo lo que se había hecho para Baal, Asera y la adoración del
sol, la luna y las estrellas. Los quemó fuera de Jerusalén, en los
campos de Cedrón, y llevó sus cenizas a Betel. \footnote{\textbf{23:4}
  2Re 21,3} \bibleverse{5} También despidió a los sacerdotes designados
por los reyes de Judá para presentar holocaustos en los lugares altos de
las ciudades de Judá y en los lugares de los alrededores de Jerusalén, a
los que habían sacrificado a Baal, al sol y a la luna, a las
constelaciones y a todos los poderes del cielo. \bibleverse{6} Quitó el
poste de Asera del Templo del Señor y lo llevó al Valle del Cedrón, en
las afueras de Jerusalén. Allí lo quemó, lo redujo a polvo y arrojó su
polvo sobre las tumbas de la gente común. \bibleverse{7} También demolió
las habitaciones de las prostitutas del culto\footnote{\textbf{23:7}
  Refiriéndose tanto a los hombres como a las mujeres.} que estaban en
el Templo del Señor, donde las mujeres solían tejer tapices para la
Asera. \footnote{\textbf{23:7} 1Re 14,24} \bibleverse{8} Josías llevó a
Jerusalén\footnote{\textbf{23:8} ``A Jerusalén'': implícito.} a todos
los sacerdotes de las ciudades de Judá y profanó los lugares altos,
desde Gueba hasta Beerseba, donde los sacerdotes habían sacrificado
holocaustos. Derribó los lugares altos de las puertas, cerca de la
entrada de la puerta de Josué, el gobernador de la ciudad, que quedaba
de la puerta del pueblo. \bibleverse{9} Aunque los sacerdotes de los
lugares altos no servían en el altar del Señor en Jerusalén, comían
panes sin levadura con sus hermanos sacerdotes. \bibleverse{10} Profanó
el altar de Tofet, en el valle de Ben-Hinom, para que nadie pudiera
sacrificar a su hijo o hija en el fuego a Moloc. \bibleverse{11} Quitó
de la entrada del Templo del Señor los caballos que los reyes de Judá
habían dedicado al sol. Estaban en el patio, cerca de la habitación de
un eunuco llamado Natán-melec. Josías también quemó los carros dedicados
al sol. \bibleverse{12} Derribó los altares que los reyes de Judá habían
colocado en el techo, cerca de la cámara alta de Acaz, y los altares que
Manasés había colocado en los dos patios del Templo del Señor. El rey
los hizo pedazos y los esparció en el valle del Cedrón. \footnote{\textbf{23:12}
  2Re 16,10-11; 2Re 21,4-5; 2Cró 28,24} \bibleverse{13} El rey también
profanó los lugares altos al este de Jerusalén, al sur del Monte de la
Corrupción, los lugares que el rey Salomón de Israel había construido
para Astoret, la vil diosa de los sidonios, para Quemos, el vil dios de
los moabitas, y para Moloc, el vil dios de los amonitas. \footnote{\textbf{23:13}
  1Re 11,7} \bibleverse{14} Hizo pedazos los pilares de piedra sagrados,
derribó los postes de Asera y cubrió los lugares con huesos humanos.

\hypertarget{el-juicio-de-josuxedas-en-betel-y-contra-el-servicio-en-las-alturas-en-samaria}{%
\subsection{El juicio de Josías en Betel y contra el servicio en las
alturas en
Samaria}\label{el-juicio-de-josuxedas-en-betel-y-contra-el-servicio-en-las-alturas-en-samaria}}

\bibleverse{15} También demolió el altar de Betel, el lugar alto erigido
por Jeroboam, hijo de Nabat, que había hecho pecar a Israel. Luego quemó
el lugar alto, lo redujo a polvo y quemó el poste de Asera. \footnote{\textbf{23:15}
  1Re 12,32} \bibleverse{16} Cuando Josías miró a su alrededor, vio unas
tumbas en la colina. Hizo sacar los huesos de las tumbas y los quemó en
el altar para profanarlo, tal como el Señor había dicho por medio del
hombre de Dios que había profetizado estas cosas. \footnote{\textbf{23:16}
  1Re 13,2} \bibleverse{17} Entonces preguntó: ``¿De quién es la lápida
que veo?'' ``Es la tumba del hombre de Dios que vino de Judá y proclamó
exactamente lo que tú has hecho con el altar de Betel'',\footnote{\textbf{23:17}
  Véase 1 Reyes 13:2.} respondió la gente del pueblo. \footnote{\textbf{23:17}
  1Re 13,30}

\bibleverse{18} ``Déjenlo descansar en paz'', dijo Josías. ``Que nadie
toque sus huesos''. Así que dejaron sus huesos sin tocar, junto con los
del profeta que vino de Samaria. \bibleverse{19} Josías destruyó, como
lo hizo en Betel, todos los santuarios de los lugares altos de las
ciudades de Samaria que habían construido los reyes de Israel que habían
enojado al Señor. \bibleverse{20} Josías sacrificó a todos los
sacerdotes que estaban allí en los lugares altos, en los altares, y
quemó huesos humanos sobre ellos. Luego regresó a Jerusalén.

\hypertarget{celebraciuxf3n-estricta-de-la-pascua}{%
\subsection{Celebración estricta de la
Pascua}\label{celebraciuxf3n-estricta-de-la-pascua}}

\bibleverse{21} El rey envió una orden a todo el pueblo: ``Celebren la
Pascua del Señor, su Dios, como está escrito en este Libro del
Acuerdo''. \bibleverse{22} Una Pascua como ésta no se había observado
desde los días de los jueces que gobernaban Israel hasta todos los días
de los reyes de Israel y de Judá. \bibleverse{23} Pero en el año
dieciocho del rey Josías, se observó esta Pascua para honrar al Señor en
Jerusalén.

\hypertarget{actuar-contra-la-idolatruxeda-en-la-vida-privada-persistencia-de-la-ira-divina-contra-juduxe1}{%
\subsection{Actuar contra la idolatría en la vida privada; Persistencia
de la ira divina contra
Judá}\label{actuar-contra-la-idolatruxeda-en-la-vida-privada-persistencia-de-la-ira-divina-contra-juduxe1}}

\bibleverse{24} Además, Josías se deshizo de los médiums y de los
espiritistas, de los dioses domésticos y de los ídolos, y de todas las
prácticas repugnantes que había en la tierra de Judá y en Jerusalén. Lo
hizo para cumplir las palabras de la ley escritas en el libro que el
sacerdote Jilquías había encontrado en el Templo del Señor. \footnote{\textbf{23:24}
  Lev 20,27; Deut 29,16-17} \bibleverse{25} Nunca antes hubo un rey como
él que se comprometiera con el Señor en todos sus pensamientos y
actitudes, y con todas sus fuerzas, guardando toda la Ley de Moisés.
Tampoco hubo después un rey como él. \footnote{\textbf{23:25} 2Re 18,5}
\bibleverse{26} Sin embargo, el Señor no había abandonado su furiosa
hostilidad, que ardía contra Judá por todo lo que Manasés había hecho
para enfurecerlo. \footnote{\textbf{23:26} 2Re 21,11-16} \bibleverse{27}
Así que el Señor anunció: ``También voy a desterrar a Judá de mi
presencia, así como desterré a Israel. Abandonaré esta ciudad que he
escogido, Jerusalén, y el Templo respecto al cual dije: Mi nombre estará
allí''. \footnote{\textbf{23:27} 2Re 17,18; 1Re 8,29}

\hypertarget{palabra-final-necao-de-egipto-y-la-muerte-de-josuxedas}{%
\subsection{Palabra final; Necao de Egipto y la muerte de
Josías}\label{palabra-final-necao-de-egipto-y-la-muerte-de-josuxedas}}

\bibleverse{28} El resto de lo que sucedió en el reinado de Josías, y
todo lo que hizo, están registrados en el Libro de las Crónicas de los
Reyes de Judá. \bibleverse{29} Mientras Josías aún era rey, el faraón
Neco, rey de Egipto, dirigió su ejército para ayudar al rey de Asiria en
el río Éufrates. El rey Josías llevó a su ejército a luchar contra él en
Meguido, pero cuando Neco vio a Josías lo mató. \bibleverse{30} Sus
servidores pusieron su cuerpo en un carro, lo trajeron de Meguido a
Jerusalén y lo enterraron en su propia tumba. Entonces el pueblo del
país eligió a Joacaz, hijo de Josías, lo ungió y lo hizo rey en sucesión
de su padre. \footnote{\textbf{23:30} 2Re 9,28}

\hypertarget{los-hijos-de-josuxedas-y-su-nieto-reyes-de-juduxe1-joachuxe2z}{%
\subsection{Los hijos de Josías y su nieto reyes de Judá;
Joachâz}\label{los-hijos-de-josuxedas-y-su-nieto-reyes-de-juduxe1-joachuxe2z}}

\bibleverse{31} Joacaz tenía veintitrés años cuando llegó a ser rey, y
reinó en Jerusalén durante tres meses. Su madre se llamaba Jamutal, hija
de Jeremías. Ella era de Libna. \bibleverse{32} Sus hechos fueron malos
a los ojos del Señor, como los de todos sus antepasados. \bibleverse{33}
El faraón Neco encarceló a Joacaz en Riblá, en la tierra de Jamat, para
impedir que gobernara en Jerusalén. También impuso a Judá un tributo de
cien talentos de plata y un talento de oro. \bibleverse{34} El faraón
Neco nombró a Eliaquim, hijo de Josías, rey en sucesión de su padre
Josías, y cambió el nombre de Eliaquim por el de Joaquim. Neco llevó a
Joacaz a Egipto, donde murió. \bibleverse{35} Joaquim pagó la plata y el
oro al faraón Neco, pero para satisfacer la demanda del faraón, éste
gravó la tierra y exigió el pago de la plata y el oro al pueblo, cada
uno en proporción a su riqueza. \footnote{\textbf{23:35} 2Re 15,20}

\hypertarget{joacim-de-juduxe1}{%
\subsection{Joacim de Judá}\label{joacim-de-juduxe1}}

\bibleverse{36} Joaquim tenía veinticinco años cuando llegó a ser rey, y
reinó en Jerusalén durante once años. Su madre se llamaba Zebida, hija
de Pedaías; era de Rumá. \bibleverse{37} Sus hechos fueron malos a los
ojos del Señor, como los de sus antepasados.

\hypertarget{section-23}{%
\section{24}\label{section-23}}

\bibleverse{1} Durante el reinado de Joaquim, Nabucodonosor, tipo de
Babilonia, invadió el país y Joaquim se sometió a él. Pero después de
tres años Joaquim se rebeló contra Nabucodonosor. \bibleverse{2}
Entonces el Señor envió bandas de asaltantes contra Judá para
destruirlos. Vinieron de Babilonia, Aram, Moab y Amón, tal como el Señor
había dicho por medio de sus siervos los profetas. \bibleverse{3} El
Señor habló contra Judá para desterrarlos de su presencia a causa de
todos los pecados que Manasés había cometido y de la gente inocente que
había matado, \bibleverse{4} llenando Jerusalén con su sangre. El Señor
no estaba dispuesto a perdonar esto. \bibleverse{5} El resto de lo que
sucedió en el reinado de Joaquim, y todo lo que hizo, está registrado en
el Libro de las Crónicas de los Reyes de Judá. \bibleverse{6} Joaquim
murió, y su hijo Joaquim lo sucedió como rey.

\bibleverse{7} El rey de Egipto no volvió a salir de su país, pues el
rey de Babilonia se había apoderado de todo el territorio que le
pertenecía, desde el Wadi de Egipto hasta el río Éufrates.

\hypertarget{joaquuxedn-de-juduxe1-la-primera-conquista-de-jerusaluxe9n-y-la-primera-ruta-a-babilonia}{%
\subsection{Joaquín de Judá; la primera conquista de Jerusalén y la
primera ruta a
Babilonia}\label{joaquuxedn-de-juduxe1-la-primera-conquista-de-jerusaluxe9n-y-la-primera-ruta-a-babilonia}}

\bibleverse{8} Joaquim tenía dieciocho años cuando llegó a ser rey, y
reinó en Jerusalén durante tres meses. Su madre era Nejustá, hija de
Elnatán. Ella era de Jerusalén. \bibleverse{9} Joaquim hizo lo malo a
los ojos del Señor, tal como lo había hecho su padre. \footnote{\textbf{24:9}
  2Re 23,37} \bibleverse{10} En aquel tiempo los oficiales de
Nabucodonosor, rey de Babilonia, atacaron Jerusalén y la sitiaron.
\bibleverse{11} Entonces Nabucodonosor, rey de Babilonia, vino en
persona mientras sus oficiales estaban sitiando la ciudad.
\bibleverse{12} Joaquim, rey de Israel, se rindió al rey de Babilonia,
junto con su madre, sus oficiales, sus comandantes y sus funcionarios.
Fue en el octavo año de su reinado cuando Nabucodonosor capturó a
Joaquim. \bibleverse{13} Nabucodonosor tomó todos los tesoros del Templo
del Señor y del palacio real, y cortó todos los objetos de oro que
Salomón, rey de Israel, había hecho para el Templo del Señor, como el
Señor había dicho que sucedería. \bibleverse{14} Deportó a toda
Jerusalén, a todos los comandantes y a los soldados experimentados, a
todos los artesanos y a los trabajadores del metal, un total de diez mil
prisioneros. Sólo quedó la gente muy pobre del país. \bibleverse{15} Se
llevó a Joaquim al exilio en Babilonia, así como a la madre del rey y a
las esposas del rey y a sus funcionarios y a los principales hombres del
país, a todos los deportó de Jerusalén a Babilonia. \footnote{\textbf{24:15}
  Jer 22,26; Jer 24,1; 2Re 25,27} \bibleverse{16} El rey de Babilonia
también deportó a Babilonia a los siete mil hombres de combate y a los
mil artesanos y metalúrgicos, todos ellos fuertes y preparados para la
batalla. \bibleverse{17} El rey de Babilonia nombró rey a Matanías, tío
de Joaquim, en su lugar, y le cambió el nombre por el de Sedequías.

\hypertarget{sedequuxedas-rey-de-juduxe1-fin-del-reino-de-juduxe1}{%
\subsection{Sedequías, rey de Judá; Fin del reino de
Judá}\label{sedequuxedas-rey-de-juduxe1-fin-del-reino-de-juduxe1}}

\bibleverse{18} Sedequías tenía veintiún años cuando llegó a ser rey, y
reinó en Jerusalén durante once años. Su madre se llamaba Jamutal, hija
de Jeremías. \bibleverse{19} Sus hechos fueron malos a los ojos del
Señor, tal como los de Joaquim. \bibleverse{20} Todo esto sucedió en
Jerusalén y en Judá, a causa de la ira del Señor, hasta que finalmente
los desterró de su presencia. Sedequías se rebeló contra el rey de
Babilonia.\footnote{\textbf{24:20} 2Re 23,27}

\hypertarget{los-desperdicios-de-sedequuxedas-asedio-de-jerusaluxe9n-escape-y-captura-del-rey-juzgado-penal-de-ribla}{%
\subsection{Los desperdicios de Sedequías; Asedio de Jerusalén; Escape y
captura del rey; Juzgado penal de
Ribla}\label{los-desperdicios-de-sedequuxedas-asedio-de-jerusaluxe9n-escape-y-captura-del-rey-juzgado-penal-de-ribla}}

\hypertarget{section-24}{%
\section{25}\label{section-24}}

\bibleverse{1} En el noveno año del reinado de Sedequías, el décimo día
del décimo mes, Nabucodonosor, rey de Babilonia, atacó Jerusalén con
todo su ejército. Acampó alrededor de la ciudad y construyó rampas de
asedio contra las murallas. \bibleverse{2} La ciudad permaneció sitiada
hasta el undécimo año del rey Sedequías. \bibleverse{3} Para el noveno
día del cuarto mes, la hambruna en la ciudad era tan grave que la gente
no tenía nada que comer. \bibleverse{4} Entonces se rompió la muralla de
la ciudad, y todos los soldados escaparon de noche por la puerta entre
las dos murallas junto al jardín del rey, aunque los babilonios tenían
la ciudad rodeada. Huyeron en dirección al Arabá,\footnote{\textbf{25:4}
  ``Arabá'': El valle del Jordán.} \bibleverse{5} pero el ejército
babilónico persiguió al rey y lo alcanzó en las llanuras de Jericó. Todo
su ejército se había dispersado y lo había abandonado. \bibleverse{6}
Capturaron al rey y lo llevaron ante el rey de Babilonia en Riblá, donde
fue condenado. \bibleverse{7} Mataron a los hijos de Sedequías mientras
él miraba, y luego le sacaron los ojos, lo ataron con grilletes de
bronce y lo llevaron a Babilonia.

\hypertarget{conquista-y-destrucciuxf3n-de-jerusaluxe9n-saqueo-e-incendio-del-templo-traslado-de-habitantes-a-babilonia-ejecuciones-en-ribla}{%
\subsection{Conquista y destrucción de Jerusalén; Saqueo e incendio del
templo; Traslado de habitantes a Babilonia; Ejecuciones en
Ribla}\label{conquista-y-destrucciuxf3n-de-jerusaluxe9n-saqueo-e-incendio-del-templo-traslado-de-habitantes-a-babilonia-ejecuciones-en-ribla}}

\bibleverse{8} El séptimo día del quinto mes, en el año decimonoveno de
Nabucodonosor, rey de Babilonia, entró en Jerusalén Nabuzaradán,
comandante de la guardia, un oficial del rey de Babilonia.
\bibleverse{9} Quemó el Templo del Señor, el palacio real y todos los
grandes edificios de Jerusalén. \bibleverse{10} Todo el ejército
babilónico, bajo el mando del comandante de la guardia, derribó las
murallas alrededor de Jerusalén. \bibleverse{11} Nabuzaradán, el
comandante de la guardia, deportó a los que quedaban en la ciudad,
incluso a los que se habían pasado al lado del rey de Babilonia, así
como al resto de la población. \bibleverse{12} Pero el comandante de la
guardia permitió que los pobres que habían quedado en el campo se
quedaran cuidando las viñas y los campos.

\bibleverse{13} Los babilonios rompieron en pedazos las columnas de
bronce, los carros móviles y el mar de bronce que pertenecían al Templo
del Señor, y se llevaron todo el bronce a Babilonia. \bibleverse{14}
También se llevaron todas las ollas, las palas, los apagadores de
lámparas, los platos y todos los demás objetos de bronce que se
utilizaban en el servicio del Templo. \bibleverse{15} El comandante de
la guardia se llevó los incensarios y las copas, todo lo que era de oro
puro o de plata. \bibleverse{16} La cantidad de bronce que provenía de
las dos columnas, del mar y de los carros móviles, que Salomón había
hecho para el Templo del Señor, todo esto pesaba más de lo que se podía
medir. \footnote{\textbf{25:16} 1Re 7,15; 1Re 7,23; 1Re 7,27}
\bibleverse{17} Cada columna tenía dieciocho codos de altura. El capitel
de bronce de una de las columnas tenía tres codos de altura, con una red
de granadas de bronce a su alrededor. La segunda columna era igual, y
también tenía una red decorativa.

\bibleverse{18} El comandante de la guardia tomó como prisioneros a
Seraías, el jefe de los sacerdotes, al sacerdote Sofonías, segundo en
rango, y a los tres porteros del Templo. \bibleverse{19} De los que
quedaron en la ciudad tomó al oficial a cargo de los soldados y a cinco
de los consejeros del rey. También se llevó al secretario del comandante
del ejército, encargado de convocar al pueblo para el servicio militar,
y a otros sesenta hombres que estaban presentes en la ciudad.
\bibleverse{20} Nabuzaradán, el comandante de la guardia, los tomó y los
llevó ante el rey de Babilonia en Riblá. \bibleverse{21} El rey de
Babilonia los hizo ejecutar en Riblá, en la tierra de Jamat. Entonces el
pueblo de Judá tuvo que abandonar su tierra.

\hypertarget{gedalja-gobernador-designado-reuxfane-a-los-juduxedos-en-una-colonia-en-mizpa.-despuuxe9s-de-su-asesinato-los-juduxedos-emigran-a-egipto}{%
\subsection{Gedalja, gobernador designado, reúne a los judíos en una
colonia en Mizpa. Después de su asesinato, los judíos emigran a
Egipto}\label{gedalja-gobernador-designado-reuxfane-a-los-juduxedos-en-una-colonia-en-mizpa.-despuuxe9s-de-su-asesinato-los-juduxedos-emigran-a-egipto}}

\bibleverse{22} Nabucodonosor, rey de Babilonia, nombró a Guedalías,
hijo de Ahicam, hijo de Safán, como gobernador sobre el pueblo que había
dejado en la tierra de Judá. \bibleverse{23} Cuando todos los oficiales
del ejército de Judá\footnote{\textbf{25:23} ``De Judá'': Añadido para
  mayor claridad.} y sus hombres se enteraron de que el rey de Babilonia
había nombrado a Guedalías como gobernador, ellos y sus hombres se
reunieron con Guedalías en Mizpa. Entre ellos estaban: Ismael hijo de
Netanías, Johanán, hijo de Carea, Seraías, hijo de Tanjumet el netofita,
Jazanías, hijo del maacateo. \bibleverse{24} Guedalías les hizo un
juramento a ellos y a sus hombres, diciéndoles: ``No tengan miedo de los
funcionarios babilónicos. Quédense aquí en la tierra y sirvan al rey de
Babilonia, y estarán bien''.

\bibleverse{25} Pero en el séptimo mes, Ismael, hijo de Netanías, hijo
de Elisama, de sangre real, vino con diez hombres. Atacaron y mataron a
Guedalías, junto con los hombres de Judea y de Babilonia que estaban con
él en Mizpa. \bibleverse{26} Como resultado, todo el pueblo, desde el
más pequeño hasta el más grande, junto con los comandantes del ejército,
huyeron a Egipto, aterrorizados por lo que harían los babilonios.

\hypertarget{jojachuxedn-indultado-tras-treinta-y-siete-auxf1os-de-prisiuxf3n}{%
\subsection{Jojachín indultado tras treinta y siete años de
prisión}\label{jojachuxedn-indultado-tras-treinta-y-siete-auxf1os-de-prisiuxf3n}}

\bibleverse{27} En el año en que Evil-Merodac se convirtió en rey de
Babilonia, liberó a Joaquim, rey de Judá, de la prisión. Esto sucedió el
día veintisiete del duodécimo mes del trigésimo séptimo año del
destierro de Joaquim, rey de Judá. \bibleverse{28} El rey de Babilonia
lo trató bien y le dio una posición de honor superior a la de los otros
reyes que estaban con él en Babilonia. \bibleverse{29} Así que Joaquim
pudo quitarse la ropa de la cárcel, y comió con frecuencia en la mesa
del rey durante el resto de su vida. \bibleverse{30} El rey le dio a
Joaquim una pensión diaria por el resto de su vida.
