\hypertarget{nombre-del-remitente-y-destinatario-de-la-carta-y-bendiciuxf3n-apostuxf3lica-a-la-congregaciuxf3n}{%
\subsection{Nombre del remitente y destinatario de la carta y bendición
apostólica a la
congregación}\label{nombre-del-remitente-y-destinatario-de-la-carta-y-bendiciuxf3n-apostuxf3lica-a-la-congregaciuxf3n}}

\hypertarget{section}{%
\section{1}\label{section}}

\bibleverse{1} Esta carta viene de Pablo, siervo de Jesucristo. Fui
llamado por Dios para ser apóstol. Él me designó para anunciar la buena
noticia \footnote{\textbf{1:1} Hech 9,15; Hech 13,2; Gal 1,15}
\bibleverse{2} que anteriormente había prometido a través de sus
profetas en las Sagradas Escrituras. \footnote{\textbf{1:2} Rom
  16,25-26; Tit 1,2; Luc 1,70} \bibleverse{3} La buena noticia es sobre
su Hijo, cuyo antepasado fue David, \footnote{\textbf{1:3} 2Sam 7,12;
  Mat 22,42; Rom 9,5} \bibleverse{4} pero que fue revelado como Hijo de
Dios por medio de su resurrección de los muertos por el poder del
Espíritu Santo. Él es Jesucristo, nuestro Señor. \footnote{\textbf{1:4}
  Hech 13,33; Mat 28,18} \bibleverse{5} Fue a través de él que recibí el
privilegio de convertirme en apóstol para llamar a todas las naciones a
creer en él y obedecerle. \footnote{\textbf{1:5} Rom 15,18; Gal 2,7; Gal
  2,9; Hech 26,16-18} \bibleverse{6} Ustedes también hacen parte de los
que fueron llamados a pertenecer a Jesucristo. \bibleverse{7} Les
escribo a todos ustedes que están en Roma, que son amados de Dios y
están llamados para ser su pueblo especial. ¡Gracia y paz a ustedes de
parte de Dios nuestro Padre y del Señor Jesucristo! \footnote{\textbf{1:7}
  1Cor 1,2; 2Cor 1,1-2; Efes 1,1; Núm 6,24-26}

\hypertarget{acciuxf3n-de-gracias-del-apuxf3stol-a-dios-por-el-estado-de-fe-de-la-comunidad-y-expresiuxf3n-del-deseo-de-poder-predicar-el-mensaje-de-salvaciuxf3n-tambiuxe9n-en-roma}{%
\subsection{Acción de gracias del Apóstol a Dios por el estado de fe de
la comunidad y expresión del deseo de poder predicar el mensaje de
salvación también en
Roma}\label{acciuxf3n-de-gracias-del-apuxf3stol-a-dios-por-el-estado-de-fe-de-la-comunidad-y-expresiuxf3n-del-deseo-de-poder-predicar-el-mensaje-de-salvaciuxf3n-tambiuxe9n-en-roma}}

\bibleverse{8} Permítanme comenzar diciendo que agradezco a mi Dios por
medio de Jesucristo por todos ustedes, porque en todo el mundo se habla
acerca de la forma en que ustedes creen en Dios. \footnote{\textbf{1:8}
  Rom 16,19} \bibleverse{9} Siempre estoy orando por ustedes, tal como
Dios mismo puede confirmarlo, el Dios al cual sirvo con todo mi corazón
al compartir la buena noticia de su Hijo. \footnote{\textbf{1:9} Efes
  1,16} \bibleverse{10} En mis oraciones siempre le pido que pronto
pueda ir a verlos, si es su voluntad. \footnote{\textbf{1:10} Rom 15,23;
  Rom 15,32; Hech 19,21} \bibleverse{11} Realmente deseo visitarlos y
compartir con ustedes una bendición espiritual para fortalecerlos.
\footnote{\textbf{1:11} Rom 15,29} \bibleverse{12} Así podemos animarnos
unos a otros por medio de la fe que cada uno tiene en Dios, tanto la fe
de ustedes como la mía. \footnote{\textbf{1:12} 2Pe 1,1}

\bibleverse{13} Quiero que sepan, mis hermanos y hermanas, que a menudo
he hecho planes para visitarlos, pero me fue imposible hacerlo hasta
hora. Quiero ver buenos frutos espirituales entre ustedes así como los
he visto entre otros pueblos\footnote{\textbf{1:13} Literalmente, ``los
  gentiles''.} . \bibleverse{14} Porque tengo la obligación de trabajar
tanto para los civilizados como los incivilizados, tanto para los
educados como los no educados. \bibleverse{15} Es por eso que en verdad
tengo un gran deseo de ir a Roma y compartir la buena noticia con
ustedes.

\hypertarget{indicaciuxf3n-del-tema-la-justificaciuxf3n}{%
\subsection{Indicación del tema: La
justificación}\label{indicaciuxf3n-del-tema-la-justificaciuxf3n}}

\bibleverse{16} Sin lugar a dudas, no me avergüenzo de la buena noticia,
porque es poder de Dios para salvar a todos los que creen en él, primero
a los judíos, y luego a todos los demás también. \footnote{\textbf{1:16}
  Sal 119,46; 1Cor 1,18; 1Cor 1,24; 2Tim 1,8} \bibleverse{17} Porque en
la buena noticia Dios se revela como bueno y justo\footnote{\textbf{1:17}
  Literalmente, ``la justicia de Dios''.} , fiel desde el principio
hasta el fin. Tal como lo dice la Escritura: ``Los que son justos viven
por la fe en él\footnote{\textbf{1:17} Las palabras reales en el texto
  original son: ``el que es recto vivirá por fe''. La cita es de Habacuc
  2:4}''. \footnote{\textbf{1:17} Rom 3,21-22}

\hypertarget{la-culpa-del-pecado-de-todo-paganismo}{%
\subsection{La culpa del pecado de todo
paganismo}\label{la-culpa-del-pecado-de-todo-paganismo}}

\bibleverse{18} La hostilidad\footnote{\textbf{1:18} Literalmente,
  ``ira''. Existen debates en cuanto a la atribución de emociones
  humanas negativas a Dios.} de Dios se revela dese el cielo contra
aquellos que son impíos e injustos, contra aquellos que sofocan la
verdad con sus malas obras. \bibleverse{19} Lo que puede llegar a
saberse de Dios es obvio, porque él se los ha mostrado claramente.
\footnote{\textbf{1:19} Hech 14,15-17; Hech 17,24-28} \bibleverse{20}
Desde la creación del mundo, los aspectos invisibles de Dios---su poder
y divinidad eternos---son claramente visibles en lo que él hizo. Tales
personas no tienen excusa, \footnote{\textbf{1:20} Sal 19,2; Heb 11,3}
\bibleverse{21} porque aunque conocieron a Dios, no lo alabaron ni le
agradecieron, sino que su pensamiento respecto a Dios se convirtió en
necedad, y la oscuridad llenó sus mentes vacías. \footnote{\textbf{1:21}
  Efes 4,18}

\bibleverse{22} Y aunque aseguraban ser sabios, se volvieron necios.
\footnote{\textbf{1:22} Jer 10,14; 1Cor 1,20} \bibleverse{23} Cambiaron
la gloria del Dios inmortal por ídolos, imágenes de seres, aves,
animales y reptiles. \footnote{\textbf{1:23} Deut 4,15-19}

\hypertarget{el-juicio-divino-sobre-el-mundo-pagano-debido-a-su-ruina}{%
\subsection{El juicio divino sobre el mundo pagano debido a su
ruina}\label{el-juicio-divino-sobre-el-mundo-pagano-debido-a-su-ruina}}

\bibleverse{24} Así que Dios los dejó a merced de los malos deseos de
sus mentes depravadas, y ellos se hicieron, unos a otros, cosas
vergonzosas y degradantes. \footnote{\textbf{1:24} Hech 14,16}
\bibleverse{25} Cambiaron la verdad de Dios por una mentira, adorando y
sirviendo criaturas en lugar del Creador, quien es digno de alabanza por
siempre. Amén.

\bibleverse{26} Por eso Dios los dejó a merced de sus malos deseos. Sus
mujeres cambiaron el sexo natural por lo que no es natural,
\bibleverse{27} y del mismo modo los hombres renunciaron al sexo con
mujeres y ardieron en lujuria unos con otros. Los hombres hicieron cosas
indecentes unos con otros, y como resultado de ello sufrieron las
consecuencias inevitables de sus perversiones. \footnote{\textbf{1:27}
  Lev 18,22; Lev 20,13; 1Cor 6,9}

\bibleverse{28} Como no consideraron la importancia de conocer a Dios,
él los dejó a merced de su forma de pensar inútil e infiel, y dejó que
hicieran lo que nunca debe hacerse. \bibleverse{29} Se llenaron de toda
clase de perversiones: maldad, avaricia, odio, envidia, asesinatos,
peleas, engaño, malicia, y chisme. \bibleverse{30} Son traidores y odian
a Dios. Son arrogantes, orgullosos y jactanciosos. Idean nuevas formas
de pecar. Se rebelan contra sus padres. \bibleverse{31} No quieren
entender, no cumplen sus promesas, no muestran ningún tipo de bondad o
compasión. \bibleverse{32} Aunque conocen claramente la voluntad de
Dios, hacen cosas que merecen la muerte. Y no solo hacen estas cosas
sino que apoyan a otros para que las hagan.

\hypertarget{el-juicio-de-la-ira-tambiuxe9n-estuxe1-ante-los-juduxedos-juzgar-a-los-demuxe1s-no-los-libera-del-juicio-de-dios}{%
\subsection{El juicio de la ira también está ante los judíos; juzgar a
los demás no los libera del juicio de
Dios}\label{el-juicio-de-la-ira-tambiuxe9n-estuxe1-ante-los-juduxedos-juzgar-a-los-demuxe1s-no-los-libera-del-juicio-de-dios}}

\hypertarget{section-1}{%
\section{2}\label{section-1}}

\bibleverse{1} Así que si juzgas a otros, no tienes excusa, quienquiera
que seas. Pues en todo lo que condenas a otros, te estás juzgando a ti
mismo, porque tú haces las mismas cosas. \bibleverse{2} Sabemos que el
juicio de Dios sobre aquellos que hacen tales cosas está basado en la
verdad. \bibleverse{3} Pero cuando tú los juzgas, ¿realmente crees que
de alguna manera podrás escapar del juicio de Dios? \bibleverse{4} ¿O es
que menosprecias su maravillosa bondad y tolerancia, sin darte cuenta de
que Dios, en su bondad, está tratando de conducirte al arrepentimiento?
\footnote{\textbf{2:4} 2Pe 3,9; 2Pe 3,15} \bibleverse{5} Ahora por tu
corazón endurecido y tu rechazo al arrepentimiento, estás empeorando tu
situación para el día de la recompensa, cuando se demuestre la rectitud
del juicio de Dios. \bibleverse{6} Dios se encargará de que todos
reciban lo que merecen, conforme a lo que han hecho.\footnote{\textbf{2:6}
  Citando Salmos 62:12.} \bibleverse{7} Así que los que han seguido
haciendo lo correcto, recibirán gloria, honor, inmortalidad y vida
eterna. \bibleverse{8} Pero los que solo piensan en sí mismos,
rechazando la verdad y eligiendo deliberadamente hacer el mal, recibirán
castigo con furia y hostilidad. \footnote{\textbf{2:8} 2Tes 1,8}
\bibleverse{9} Todos los que hacen el mal tendrán pena y sufrimiento.
Primero los del pueblo judío, y luego los extranjeros también.

\bibleverse{10} Pero todos los que hacen lo bueno tendrán gloria, honor
y paz. Primero los del pueblo judío, y luego los extranjeros también.
\bibleverse{11} Pues Dios no tiene favoritos.

\hypertarget{el-juicio-de-dios-es-el-mismo-para-los-juduxedos-que-para-los-gentiles-determinado-uxfanicamente-por-el-cumplimiento-de-la-ley}{%
\subsection{El juicio de Dios es el mismo para los judíos que para los
gentiles, determinado únicamente por el cumplimiento de la
ley}\label{el-juicio-de-dios-es-el-mismo-para-los-juduxedos-que-para-los-gentiles-determinado-uxfanicamente-por-el-cumplimiento-de-la-ley}}

\bibleverse{12} Aquellos que pecan aunque no tienen la ley
escrita\footnote{\textbf{2:12} Refiriéndose a la ley escrita por Moisés.
  Los que no tiene la ley escrita son los ``extranjeros'', y los que sí
  tienen la ley escrita son los judíos.} están perdidos, pero aquellos
que pecan y sí tienen la ley escrita, serán condenados por esa misma
ley. \bibleverse{13} Porque el solo hecho de oír lo que dice la ley no
nos hace justos ante los ojos de Dios. Los que hacen lo que dice la ley
son los que reciben justificación. \footnote{\textbf{2:13} Mat 7,21;
  Sant 1,22} \bibleverse{14} Los extranjeros no tienen la ley escrita,
pero cuando hacen por instinto lo que la ley dice, están siguiendo la
ley aunque no la tengan. \footnote{\textbf{2:14} Hech 10,35}
\bibleverse{15} De esta manera, ellos demuestran cómo obra la ley que
está escrita en sus corazones. Pues cuando piensan en lo que están
haciendo, su conciencia los acusa por hacer el mal o los defiende por
hacer el bien. \footnote{\textbf{2:15} Rom 1,32} \bibleverse{16} La
buena noticia que yo les comparto es que viene un día cuando Dios
juzgará, por medio de Jesucristo, los pensamientos secretos de todos.
\footnote{\textbf{2:16} Luc 8,17}

\hypertarget{un-mejor-conocimiento-moral-y-la-capacidad-de-enseuxf1ar-no-hacen-que-los-juduxedos-sean-justos-ante-dios-su-fama-por-la-ley-es-nula-porque-la-transgrede}{%
\subsection{Un mejor conocimiento moral y la capacidad de enseñar no
hacen que los judíos sean justos ante Dios; su fama por la ley es nula
porque la
transgrede}\label{un-mejor-conocimiento-moral-y-la-capacidad-de-enseuxf1ar-no-hacen-que-los-juduxedos-sean-justos-ante-dios-su-fama-por-la-ley-es-nula-porque-la-transgrede}}

\bibleverse{17} ¿Qué hay de ti, que te llamas judío? Confías en la ley
escrita y te jactas de tener una relación especial con Dios.
\bibleverse{18} Conoces su voluntad. Haces lo recto porque has aprendido
de la ley. \bibleverse{19} Estás completamente seguro de que puedes
guiar a los ciegos y que eres luz para los que están en oscuridad.
\footnote{\textbf{2:19} Mat 15,14} \bibleverse{20} Crees que puedes
corregir a los ignorantes y que eres un maestro de ``niños'', porque
conoces por la ley toda la verdad que existe. \bibleverse{21} Y si estás
tan afanado en enseñar a otros, ¿por qué no te enseñas a ti mismo?
Puedes decirle a la gente que no robe, pero ¿estás tú robando?
\bibleverse{22} Puedes decirle a la gente que no cometa adulterio, pero
¿estás tú adulterando? Puedes decirle a la gente que no adore ídolos,
pero ¿profanas tú los templos?\footnote{\textbf{2:22} O, ``robar
  templos''.} \bibleverse{23} Te jactas de tener la ley, pero ¿acaso no
das una imagen distorsionada de Dios al quebrantarla? \bibleverse{24}
Como dice la Escritura, ``Por tu causa es difamado el carácter de Dios
entre los extranjeros''.\footnote{\textbf{2:24} Citando Isaías 52:5.
  Literalmente, ``el nombre de Dios'', que fundamentalmente tiene que
  ver con su carácter.}

\hypertarget{la-circuncisiuxf3n-no-tiene-valor-para-el-juduxedo-si-infringe-la-ley-la-circuncisiuxf3n-del-corazuxf3n-es-necesaria}{%
\subsection{La circuncisión no tiene valor para el judío si infringe la
ley; La circuncisión del ``corazón'' es
necesaria}\label{la-circuncisiuxf3n-no-tiene-valor-para-el-juduxedo-si-infringe-la-ley-la-circuncisiuxf3n-del-corazuxf3n-es-necesaria}}

\bibleverse{25} Estar circuncidado\footnote{\textbf{2:25} La
  circuncisión, dada por Dios a Israel en el Antiguo Testamento, era una
  señal de que ellos eran su pueblo especial.} solo tiene valor si haces
lo que dice la ley. Pero si quebrantas la ley, tu circuncisión es tan
inútil como la de aquellos que no están circuncidados. \footnote{\textbf{2:25}
  Jer 4,4} \bibleverse{26} Si un hombre que no está
circuncidado\footnote{\textbf{2:26} No circuncidado, queriendo decir que
  no era judío, o que era un ``extranjero''.} guarda la ley, debe
considerársele como si lo estuviera aunque no lo esté. \footnote{\textbf{2:26}
  Gal 5,6} \bibleverse{27} Los extranjeros incircuncisos que guardan la
ley te condenarán si tú la quebrantas, aunque tengas la ley y estés
circuncidado. \bibleverse{28} No es lo externo lo que te convierte en
judío; no es la señal física de la circuncisión. \bibleverse{29} Lo que
te hace judío es lo que llevas por dentro, una ``circuncisión del
corazón'' que no sigue la letra de la ley sino la del Espíritu. Alguien
así busca alabanza de Dios y no de la gente.\footnote{\textbf{2:29} Deut
  30,6; Fil 3,3; Col 2,11}

\hypertarget{sin-embargo-la-posiciuxf3n-privilegiada-de-los-juduxedos-permanece-su-infidelidad-pone-la-fidelidad-de-dios-en-una-luz-muxe1s-brillante}{%
\subsection{Sin embargo, la posición privilegiada de los judíos
permanece; su infidelidad pone la fidelidad de Dios en una luz más
brillante}\label{sin-embargo-la-posiciuxf3n-privilegiada-de-los-juduxedos-permanece-su-infidelidad-pone-la-fidelidad-de-dios-en-una-luz-muxe1s-brillante}}

\hypertarget{section-2}{%
\section{3}\label{section-2}}

\bibleverse{1} ¿Tienen entonces los judíos alguna ventaja? ¿Tiene algún
beneficio la circuncisión? \bibleverse{2} Sí. ¡Hay muchos beneficios! En
primer lugar, el mensaje de Dios fue confiado a los judíos.
\bibleverse{3} ¿Qué pasaría si alguno de ellos no creyera en Dios?
¿Acaso su falta de fe en Dios anula la fidelidad de Dios? \footnote{\textbf{3:3}
  Rom 9,6; Rom 11,29; 2Tim 2,13} \bibleverse{4} ¡Claro que no! Incluso
si todos los demás mienten, Dios siempre dice la verdad. Como dice la
Escritura: ``Quedará demostrado que tienes la razón en lo que dices, y
ganarás tu caso\footnote{\textbf{3:4} O, ``serás vindicado''.} cuando
seas juzgado''\footnote{\textbf{3:4} Citando Salmos 51:4.} \footnote{\textbf{3:4}
  Sal 116,11}

\bibleverse{5} Pero si el hecho de que estamos equivocados ayuda a
demostrar que Dios está en lo correcto, ¿qué debemos concluir? ¿Que Dios
se equivoca al pronunciar juicio sobre nosotros? (Aquí estoy hablando
desde una perspectiva humana). \bibleverse{6} ¡Por supuesto que no! ¿De
qué otra manera podría Dios juzgar al mundo? \bibleverse{7} Alguno
podría decir: ``¿Por qué sigo siendo condenado como pecador si mis
mentiras hacen que la verdad de Dios y su gloria sean más obvias al
contrastarlas?'' \bibleverse{8} ¿Acaso se trata de ``Vamos a pecar para
dar lugar al bien''? Algunos con calumnia nos acusan de decir eso.
¡Tales personas deberían ser condenadas! \footnote{\textbf{3:8} Rom 6,1}

\hypertarget{resultado-la-corrupciuxf3n-del-pecado-se-extiende-a-gentiles-y-juduxedos-y-es-confirmada-por-numerosas-escrituras}{%
\subsection{Resultado: la corrupción del pecado se extiende a gentiles y
judíos y es confirmada por numerosas
escrituras}\label{resultado-la-corrupciuxf3n-del-pecado-se-extiende-a-gentiles-y-juduxedos-y-es-confirmada-por-numerosas-escrituras}}

\bibleverse{9} Entonces, ¿son los judíos mejores que los demás?
¡Ciertamente no! Recordemos que ya hemos demostrado que tanto judíos
como extranjeros estamos bajo el control del pecado. \footnote{\textbf{3:9}
  Rom 1,18-999} \bibleverse{10} Como dice la Escritura: ``Nadie hace lo
recto, ni siquiera uno. \footnote{\textbf{3:10} Sal 14,1-3; Sal 53,2-4}
\bibleverse{11} Nadie entiende, nadie busca a Dios. \bibleverse{12}
Todos le han dado la espalda, todos hacen lo que es malo. Nadie hace lo
que es bueno, ni siquiera uno. \bibleverse{13} Sus gargantas son como
una tumba abierta; sus lenguas esparcen engaño; sus labios rebosan
veneno de serpientes. \bibleverse{14} Sus bocas están llenas de amargura
y maldiciones, \footnote{\textbf{3:14} Sal 10,7} \bibleverse{15} y están
prestos para causar dolor y muerte. \footnote{\textbf{3:15} Is 59,7-8}
\bibleverse{16} Su camino los lleva al desastre y la miseria;
\bibleverse{17} no saben cómo vivir en paz. \footnote{\textbf{3:17} Luc
  1,79} \bibleverse{18} No les importa en absoluto respetar a
Dios''.\footnote{\textbf{3:18} Este texto incluye referencias a Salmos
  14:1-3, Salmos 5:9, Salmos 140:3, Salmos 10:7, Isaías 59:7-8 ,
  Proverbios 1:16, Salmos 36:1.} \footnote{\textbf{3:18} Sal 36,2}

\bibleverse{19} Está claro que todo lo que dice la ley se aplica a
aquellos que viven bajo la ley, para que nadie pueda tener excusa
alguna, y para asegurar que todos en el mundo sean responsables ante
Dios. \footnote{\textbf{3:19} Rom 2,12; Gal 3,22} \bibleverse{20} Porque
nadie es justificado ante Dios por hacer lo que la ley exige. La ley
solo nos ayuda a reconocer lo que es realmente el pecado. \footnote{\textbf{3:20}
  Sal 143,2; Gal 2,16; Rom 7,7}

\hypertarget{la-justicia-de-dios-se-otorga-a-los-que-creen-en-jesuxfas}{%
\subsection{La justicia de Dios se otorga a los que creen en
Jesús}\label{la-justicia-de-dios-se-otorga-a-los-que-creen-en-jesuxfas}}

\bibleverse{21} Pero ahora se ha demostrado el carácter bondadoso y
recto\footnote{\textbf{3:21} Ver el versículo 1:17. También 3:22.} de
Dios. Y no tiene nada que ver con el cumplimiento de la ley, aunque ya
se habló de él por medio de la ley y los profetas. \footnote{\textbf{3:21}
  Rom 1,17; Hech 10,43} \bibleverse{22} Este carácter recto de Dios
viene a todo aquél que cree en Jesucristo, aquellos que ponen su
confianza en él. No importa quienes seamos: \footnote{\textbf{3:22} Fil
  3,9} \bibleverse{23} Todos hemos pecado y hemos fallado en alcanzar el
ideal glorioso de Dios. \footnote{\textbf{3:23} Rom 5,2; Juan 5,44; Sal
  84,12} \bibleverse{24} Sin embargo, por medio del regalo de su gracia,
Dios nos hace justos, a través de Jesucristo, quien nos hace libres.
\footnote{\textbf{3:24} Rom 5,1; 2Cor 5,19; Efes 2,8} \bibleverse{25}
Dios presentó abiertamente a Jesús como el don que trae paz\footnote{\textbf{3:25}
  O, ``lugar de expiación''.} a aquellos que creen en él, quien derramó
su sangre. Hizo esto con el fin de demostrar que él es verdaderamente
recto, porque anteriormente se contuvo y pasó por alto los pecados,
\footnote{\textbf{3:25} Lev 16,12-15; Heb 4,16} \bibleverse{26} pero
ahora, en el presente, Dios demuestra que es justo y hace lo recto, y
que hace justos a los que creen en Jesús.

\hypertarget{la-justicia-de-dios-por-la-fe-excluye-toda-fama-propia-y-se-aplica-tanto-a-los-gentiles-como-a-los-juduxedos}{%
\subsection{La justicia de Dios por la fe excluye toda fama propia y se
aplica tanto a los gentiles como a los
judíos}\label{la-justicia-de-dios-por-la-fe-excluye-toda-fama-propia-y-se-aplica-tanto-a-los-gentiles-como-a-los-juduxedos}}

\bibleverse{27} ¿Acaso tenemos algo de qué jactarnos? Por supuesto que
no, ¡no hay lugar para ello! ¿Por qué? ¿Acaso es porque seguimos la ley
de guardar los requisitos? No, nosotros seguimos la ley de la fe en
Dios. \bibleverse{28} Entonces concluimos que somos hechos justos por
Dios por medio de nuestra fe en él, y no por la observancia de la ley.
\footnote{\textbf{3:28} Gal 2,16} \bibleverse{29} ¿Acaso Dios es
solamente Dios de los judíos? ¿Acaso él no es el Dios de los demás
pueblos también? ¡Por supuesto que sí! \footnote{\textbf{3:29} Rom 10,12}
\bibleverse{30} Solo hay un Dios, y él nos justifica por nuestra fe en
él, quienesquiera que seamos, judíos o extranjeros. \footnote{\textbf{3:30}
  Rom 4,11-12}

\bibleverse{31} ¿Significa eso que por creer en Dios desechamos de la
ley? ¡Por supuesto que no! De hecho, afirmamos la importancia de la
ley.\footnote{\textbf{3:31} Mat 5,17}

\hypertarget{evidencia-de-la-justicia-de-la-fe-en-abraham-y-mediante-un-testimonio-de-david}{%
\subsection{Evidencia de la justicia de la fe en Abraham y mediante un
testimonio de
David}\label{evidencia-de-la-justicia-de-la-fe-en-abraham-y-mediante-un-testimonio-de-david}}

\hypertarget{section-3}{%
\section{4}\label{section-3}}

\bibleverse{1} Miremos el ejemplo de Abraham. Desde la perspectiva
humana, él es el padre de nuestra nación. Preguntemos: ``¿Cuál fue su
experiencia?'' \bibleverse{2} Porque si Abraham hubiera sido justificado
por lo que hizo, habría tenido algo de lo cual jactarse, pero no ante
los ojos de Dios. \bibleverse{3} Sin embargo, ¿qué dice la Escritura?
``Abraham creyó en Dios, y por ello fue considerado justo''.\footnote{\textbf{4:3}
  Citando Génesis 15:6.} \footnote{\textbf{4:3} Gal 3,6} \bibleverse{4}
Todo el que trabaja recibe su pago, no como un regalo, sino porque se ha
ganado su salario. \footnote{\textbf{4:4} Rom 11,6} \bibleverse{5} Pero
Dios, quien hace justos a los pecadores, los considera justos no porque
hayan trabajado por ello, sino porque confían en él. \footnote{\textbf{4:5}
  Rom 3,26} \bibleverse{6} Es por ello que David habla de la felicidad
de aquellos a quienes Dios acepta como justos, y no porque ellos
trabajen por ello: \bibleverse{7} ``Cuán felices son los que reciben
perdón por sus errores y cuyos pecados son cubiertos. \bibleverse{8}
Cuán felices son aquellos a quienes el Señor no considera
pecadores''.\footnote{\textbf{4:8} Citando Salmos 32:1-2.}

\hypertarget{abraham-como-el-padre-de-todos-los-creyentes-incluidos-los-gentiles}{%
\subsection{Abraham como el padre de todos los creyentes, incluidos los
gentiles}\label{abraham-como-el-padre-de-todos-los-creyentes-incluidos-los-gentiles}}

\bibleverse{9} Ahora, ¿es acaso esta bendición solo para los judíos, o
es para los demás también? Acabamos de afirmar que Abraham fue aceptado
como justo porque confió en Dios. \bibleverse{10} Pero ¿cuándo sucedió
esto? ¿Acaso fue cuando Abraham era judío o antes? \bibleverse{11} De
hecho, fue antes de que Abraham fuera judío por ser circuncidado, lo
cual era una confirmación de su confianza en Dios para hacerlo justo.
Esto ocurrió antes de ser circuncidado, de modo que él es el padre de
todos los que confían en Dios y son considerados justos por él, aunque
no sean judíos circuncidados. \bibleverse{12} También es el padre de los
judíos circuncidados, no solo porque estén circuncidados, sino porque
siguen el ejemplo de la confianza en Dios que nuestro padre Abraham tuvo
antes de ser circuncidado. \footnote{\textbf{4:12} Mat 3,9}

\hypertarget{la-promesa-de-salvaciuxf3n-no-le-lleguxf3-a-abraham-por-la-ley-sino-por-la-fe}{%
\subsection{La promesa de salvación no le llegó a Abraham por la ley,
sino por la
fe}\label{la-promesa-de-salvaciuxf3n-no-le-lleguxf3-a-abraham-por-la-ley-sino-por-la-fe}}

\bibleverse{13} La promesa que Dios le hizo a Abraham y a sus
descendientes de que el mundo les pertenecería no estaba basada en su
cumplimiento de la ley, sino en que él fue justificado por su confianza
en Dios. \footnote{\textbf{4:13} Gén 22,17-18} \bibleverse{14} Porque si
la herencia prometida estuviera basada en el cumplimiento de la ley,
entonces confiar en Dios no sería necesario, y la promesa sería inútil.
\bibleverse{15} Porque la ley resulta en castigo,\footnote{\textbf{4:15}
  Castigo por el incumplimiento de la ley, que por supuesto incluye a
  todos.} pero si no hay ley, entonces no puede ser quebrantada.
\footnote{\textbf{4:15} Rom 3,20; Rom 5,13; Rom 7,8; Rom 7,10}

\bibleverse{16} De modo que la promesa está basada en la confianza en
Dios. Es dada como un don, garantizada a todos los hijos de Abraham, y
no solo a los que siguen la ley,\footnote{\textbf{4:16} Pablo no está
  diciendo que los que obedecen la ley de Moisés son justificados ante
  Dios. Ya había tratado ese tema. Sencillamente está señalando que los
  que no siguen la ley de moisés no son excluidos por Dios.} sino
también a todos los que creen como Abraham, el padre de todos nosotros.
\bibleverse{17} Como dice la Escritura: ``Yo te he hecho el padre de
muchas naciones''.\footnote{\textbf{4:17} Citando Génesis 17:5.} Porque
en presencia de Dios, Abraham creyó en el Dios que hace resucitar a los
muertos y trajo a la existencia lo que no existía antes.

\hypertarget{la-fe-ejemplar-de-abraham}{%
\subsection{La fe ejemplar de Abraham}\label{la-fe-ejemplar-de-abraham}}

\bibleverse{18} Contra toda esperanza, Abrahán tuvo esperanza y confió
en Dios, y de este modo pudo llegar a ser el padre de muchos pueblos,
tal como Dios se lo prometió: ``¡Tendrás muchos descendientes!''
\bibleverse{19} Su confianza en Dios no se debilitó aun cuando creía que
su cuerpo ya estaba prácticamente muerto (tenía casi cien años de edad),
y sabía que Sara estaba muy vieja para tener hijos. \footnote{\textbf{4:19}
  Gén 17,17} \bibleverse{20} Sino que se aferró a la promesa de Dios y
no dudó. Por el contrario, su confianza en Dios se fortalecía y daba
gloria a Dios. \footnote{\textbf{4:20} Heb 11,11} \bibleverse{21} Él
estaba completamente convencido que Dios tenía el poder para cumplir la
promesa. \bibleverse{22} Por eso Dios consideró justo a Abraham.

\hypertarget{tal-fe-tambiuxe9n-nos-trae-justicia-y-felicidad}{%
\subsection{Tal fe también nos trae justicia y
felicidad}\label{tal-fe-tambiuxe9n-nos-trae-justicia-y-felicidad}}

\bibleverse{23} Las palabras ``Abraham fue considerado
justo''\footnote{\textbf{4:23} CitandoGénesis 15:6.} no fueron escritas
solo para su beneficio. \bibleverse{24} También fueron escritas para
beneficio de nosotros, para los que seremos considerados justos porque
confiamos en Dios, quien levantó a nuestro Señor Jesús de los muertos.
\bibleverse{25} Jesús fue entregado a la muerte por causa de nuestros
pecados,\footnote{\textbf{4:25} Ver Isaías 53:4-5.} y fue levantado a la
vida para justificarnos.\footnote{\textbf{4:25} Is 53,4-5; Rom 8,32; Rom
  8,34}

\hypertarget{la-salvaciuxf3n-futura-estuxe1-garantizada-para-los-justificados-a-pesar-de-todas-las-tribulaciones-debido-al-amor-de-dios-demostrado-por-la-muerte-sacrificial-de-cristo}{%
\subsection{La salvación futura está garantizada para los justificados a
pesar de todas las tribulaciones debido al amor de Dios demostrado por
la muerte sacrificial de
Cristo}\label{la-salvaciuxf3n-futura-estuxe1-garantizada-para-los-justificados-a-pesar-de-todas-las-tribulaciones-debido-al-amor-de-dios-demostrado-por-la-muerte-sacrificial-de-cristo}}

\hypertarget{section-4}{%
\section{5}\label{section-4}}

\bibleverse{1} Ahora que hemos sido justificados por Dios, por nuestra
confianza en él, tenemos paz con él a través de nuestro Señor
Jesucristo. \footnote{\textbf{5:1} Rom 3,24; Rom 3,28; Is 53,5}
\bibleverse{2} Porque es por medio de Jesús, y por nuestra fe en él, que
hemos recibido acceso a esta posición de gracia en la que estamos,
esperando con alegría y confianza que podamos participar de la gloria de
Dios. \footnote{\textbf{5:2} Juan 14,6; Efes 3,12} \bibleverse{3} No
solo esto, sino que mantenemos la confianza cuando vienen los problemas,
porque sabemos que experimentar dificultades desarrolla nuestra
fortaleza espiritual.\footnote{\textbf{5:3} O ``perseverancia''.}
\footnote{\textbf{5:3} Sant 1,2; Sant 1,1-3} \bibleverse{4} La fortaleza
espiritual, a su vez, desarrolla un carácter maduro, y este carácter
maduro trae como resultado una esperanza que cree. \footnote{\textbf{5:4}
  Sant 1,12} \bibleverse{5} Ya que tenemos esta esperanza, nunca seremos
defraudados, porque el amor de Dios ha sido derramado en nuestros
corazones a través del Espíritu Santo que él nos ha dado. \footnote{\textbf{5:5}
  Heb 6,18-19; Sal 22,6; Sal 25,3; Sal 25,20}

\bibleverse{6} Cuando estábamos completamente indefensos, en ese momento
justo, Cristo murió por nosotros los impíos. \bibleverse{7} ¿Quién
moriría por otra persona, incluso si se tratara de alguien que hace lo
recto? (Aunque quizás alguno sería suficientemente valiente para morir
por alguien que es realmente bueno). \bibleverse{8} Pero Dios demuestra
su amor en que Cristo murió por nosotros aunque todavía éramos
pecadores. \footnote{\textbf{5:8} Juan 3,16; 1Jn 4,10}

\bibleverse{9} Ahora que somos justificados por su muerte,\footnote{\textbf{5:9}
  Literalmente, ``sangre''.} podemos estar totalmente seguros de que él
nos salvará del juicio que viene. \footnote{\textbf{5:9} Rom 1,18; Rom
  2,5; Rom 2,8} \bibleverse{10} Aunque éramos sus enemigos, Dios nos
convirtió en sus amigos por medio de la muerte de su Hijo, y así podemos
estar totalmente seguros de que él nos salvará por la vida de su Hijo.
\footnote{\textbf{5:10} Rom 8,7; Col 1,21; 2Cor 5,18}

\bibleverse{11} Además de esto celebramos ahora lo que Dios ha hecho por
medio de nuestro Señor Jesucristo para reconciliarnos y convertirnos en
sus amigos.

\hypertarget{cristo-como-lo-opuesto-a-aduxe1n-la-gracia-que-trae-vida-inmortal-es-muxe1s-poderosa-que-el-pecado-mortal}{%
\subsection{Cristo como lo opuesto a Adán; la gracia que trae vida
inmortal es más poderosa que el pecado
mortal}\label{cristo-como-lo-opuesto-a-aduxe1n-la-gracia-que-trae-vida-inmortal-es-muxe1s-poderosa-que-el-pecado-mortal}}

\bibleverse{12} Porque a través de un hombre el pecado entró al mundo, y
el pecado condujo a la muerte. Y de esta manera la muerte llegó a todos,
porque todos eran pecadores. \bibleverse{13} Incluso antes de que se
diera la ley, el pecado ya estaba en el mundo, pero no era considerado
pecado porque no había ley. \footnote{\textbf{5:13} Rom 4,15}
\bibleverse{14} Pero la muerte gobernaba desde Adán hasta Moisés,
incluso sobre aquellos que no pecaron de la misma manera que lo hizo
Adán. Pues Adán prefiguraba a Aquél que vendría.\footnote{\textbf{5:14}
  En otras palabras, Adán era un símbolo o tipo de Jesús, quien vendría.}

\bibleverse{15} Pero el don de Jesús no es como el pecado de
Adán.\footnote{\textbf{5:15} Haciendo explícito lo que quiere decir con
  ``don'' y ``pecado''.} Aunque mucha gente murió por culpa del pecado
de un hombre, la gracia de Dios es mucho más grande y ha sido dada a
muchos a través de su don gratuito en la persona de Jesucristo.
\bibleverse{16} El resultado de este don no es como el resultado del
pecado. El resultado del pecado de Adán fue juicio y condenación, pero
este don nos justifica con Dios, a pesar de nuestros muchos pecados.
\bibleverse{17} Como resultado del pecado de un hombre, la muerte
gobernó por su culpa. Pero la gracia de Dios es mucho más grande y su
don nos justifica, porque todo el que lo recibe gobernará en vida a
través de la persona de Jesucristo.

\bibleverse{18} Del mismo modo que un pecado trajo condenación a todos,
un acto de justicia nos dio a todos la oportunidad de vivir en justicia.
\bibleverse{19} Así como por la desobediencia de un hombre muchos se
convirtieron en pecadores, de la misma manera, a través de la obediencia
de un hombre, muchos son justificados delante de Dios. \footnote{\textbf{5:19}
  Rom 3,26; Is 53,11} \bibleverse{20} Pues cuando se introdujo la ley,
el pecado se hizo más evidente. ¡Pero aunque el pecado se volvió más
evidente, la gracia se volvió más evidente aun! \footnote{\textbf{5:20}
  Rom 7,8; Rom 7,13; Gal 3,19} \bibleverse{21} Así como el pecado
gobernó sobre nosotros y nos llevó a la muerte, ahora la gracia es la
que gobierna al justificarnos delante de Dios, trayéndonos vida eterna
por medio de Jesucristo, nuestro Señor.\footnote{\textbf{5:21} Rom 6,23}

\hypertarget{fuimos-crucificados-con-ellos-morimos-con-ellos-sepultados-con-ellos-y-resucitamos-con-cristo-jesuxfas}{%
\subsection{Fuimos crucificados con ellos, morimos con ellos, sepultados
con ellos y resucitamos con Cristo
Jesús}\label{fuimos-crucificados-con-ellos-morimos-con-ellos-sepultados-con-ellos-y-resucitamos-con-cristo-jesuxfas}}

\hypertarget{section-5}{%
\section{6}\label{section-5}}

\bibleverse{1} ¿Cuál es nuestra respuesta, entonces? ¿Debemos seguir
pecando para tener aún más gracia? \footnote{\textbf{6:1} Rom 3,5-8}
\bibleverse{2} ¡Por supuesto que no!\footnote{\textbf{6:2} Literalmente,
  ``¡que no ocurra así!'' Esta reacción enérgica es traducida en
  diversas maneras así: ¡Por supuesto que no! ¡De ninguna manera! ¡Que
  Dios no lo quiera! También en el versículo 6:15 etc.} Pues si estamos
muertos al pecado, ¿cómo podríamos seguir viviendo en pecado?
\bibleverse{3} ¿No saben que todos los que fuimos bautizados en
Jesucristo, fuimos bautizados en su muerte? \footnote{\textbf{6:3} Gal
  3,27} \bibleverse{4} A través del bautismo fuimos sepultados con él en
la muerte, para que así como Cristo fue levantado de los muertos por
medio de la gloria del Padre, nosotros también podamos vivir una vida
nueva. \footnote{\textbf{6:4} Col 2,12; 1Pe 3,21}

\bibleverse{5} Si hemos sido hechos uno con él, al morir como él murió,
entonces seremos levantados como él también. \bibleverse{6} Sabemos que
nuestro antiguo ser fue crucificado con él para deshacernos del cuerpo
muerto del pecado, a fin de que ya no pudiéramos ser más esclavos del
pecado. \footnote{\textbf{6:6} Gal 5,24} \bibleverse{7} Todo el que ha
muerto, ha sido liberado del pecado.

\hypertarget{viviendo-con-cristo-resucitado}{%
\subsection{Viviendo con Cristo
resucitado}\label{viviendo-con-cristo-resucitado}}

\bibleverse{8} Y como morimos con Cristo, tenemos la confianza de que
también viviremos con él, \bibleverse{9} porque sabemos que si Cristo ha
sido levantado de los muertos, no morirá más, porque la muerte ya no
tiene ningún poder sobre él. \bibleverse{10} Al morir, él murió al
pecado una vez y por todos, pero ahora vive, y vive para Dios.
\bibleverse{11} De esta misma manera, ustedes deben considerarse muertos
al pecado, pero vivos para Dios, por medio de Jesucristo. \footnote{\textbf{6:11}
  2Cor 5,15; 1Pe 2,24}

\hypertarget{la-amonestaciuxf3n-del-apuxf3stol-a-los-fieles-de-permanecer-en-este-conocimiento-de-la-salvaciuxf3n-y-no-seguir-sirviendo-al-pecado}{%
\subsection{La amonestación del apóstol a los fieles de permanecer en
este conocimiento de la salvación y no seguir sirviendo al
pecado}\label{la-amonestaciuxf3n-del-apuxf3stol-a-los-fieles-de-permanecer-en-este-conocimiento-de-la-salvaciuxf3n-y-no-seguir-sirviendo-al-pecado}}

\bibleverse{12} No permitan que el pecado controle sus cuerpos mortales,
no se rindan ante sus tentaciones, \footnote{\textbf{6:12} Gén 4,7}
\bibleverse{13} y no usen ninguna parte de su cuerpo como herramientas
de pecado para el mal. Por el contrario, conságrense a Dios como quienes
han sido traídos de vuelta a la vida, y usen todas las partes de su
cuerpo como herramientas para hacer el bien para Dios. \footnote{\textbf{6:13}
  Rom 12,1} \bibleverse{14} El pecado no gobernará sobre ustedes, porque
ustedes no están bajo la ley sino bajo la gracia. \footnote{\textbf{6:14}
  Rom 7,4-6; 1Jn 3,6}

\hypertarget{el-servicio-del-pecado-ha-dado-paso-a-la-justicia}{%
\subsection{El servicio del pecado ha dado paso a la
justicia}\label{el-servicio-del-pecado-ha-dado-paso-a-la-justicia}}

\bibleverse{15} ¿Acaso vamos a pecar porque no estamos bajo la ley sino
bajo la gracia? ¡Por supuesto que no! \footnote{\textbf{6:15} Rom 5,17;
  Rom 5,21} \bibleverse{16} ¿No se dan cuenta de que si ustedes se
someten a alguien, y obedecen sus órdenes, entonces son esclavos de
aquél a quien obedecen? Si ustedes son esclavos del pecado, el resultado
es muerte; si obedecen a Dios, el resultado es que serán justificados
delante de él. \footnote{\textbf{6:16} Juan 8,34} \bibleverse{17}
Gracias a Dios porque aunque una vez ustedes eran esclavos del pecado,
escogieron de todo corazón seguir la verdad que aprendieron acerca de
Dios. \bibleverse{18} Ahora que han sido liberados del pecado, se han
convertido en esclavos de hacer lo recto. \footnote{\textbf{6:18} Juan
  8,32}

\bibleverse{19} Hago uso de este ejemplo cotidiano porque su forma
humana de pensar es limitada. Así como una vez ustedes mismos se
hicieron esclavos de la inmoralidad, ahora deben volverse esclavos de lo
que es puro y recto. \bibleverse{20} Cuando eran esclavos del pecado, no
se les exigía que hicieran lo recto. \bibleverse{21} Pero ¿cuáles eran
los resultados en ese entonces? ¿No se avergüenzan de las cosas que
hicieron? ¡Eran cosas que conducen a la muerte! \bibleverse{22} Pero
ahora que han sido liberados del pecado y se han convertido en esclavos
de Dios, los resultados serán una vida pura, y al final, vida eterna.
\bibleverse{23} La paga del pecado es muerte, pero el regalo de Dios es
vida eterna por medio de Jesucristo, nuestro Señor.\footnote{\textbf{6:23}
  Rom 5,12; Sant 1,15}

\hypertarget{cuando-hemos-muerto-y-resucitado-con-cristo-estamos-leguxedtimamente-libres-de-la-ley-y-estamos-obligados-a-servir-al-cristo-resucitado-creyuxe9ndonos-muertos-al-pecado}{%
\subsection{Cuando hemos muerto y resucitado con Cristo, estamos
legítimamente libres de la ley y estamos obligados a servir al Cristo
resucitado creyéndonos muertos al
pecado}\label{cuando-hemos-muerto-y-resucitado-con-cristo-estamos-leguxedtimamente-libres-de-la-ley-y-estamos-obligados-a-servir-al-cristo-resucitado-creyuxe9ndonos-muertos-al-pecado}}

\hypertarget{section-6}{%
\section{7}\label{section-6}}

\bibleverse{1} Hermanos y hermanas, (hablo para personas que conocen la
ley\footnote{\textbf{7:1} El uso que Pablo hace de la palabra ley puede
  tener varios significados, pero a menudo se refiere al sistema de
  creencias judías. Parte de esto tiene que ver con el cumplimiento de
  las reglas.} ), ¿no ven que la ley tiene autoridad sobre alguien solo
mientras esta persona esté viva? \bibleverse{2} Por ejemplo, una mujer
casada está sujeta por ley a su esposo mientras él esté vivo, pero si
muere, ella queda libre de esta obligación legal con él. \bibleverse{3}
De modo que si ella vive con otro hombre mientras su esposo está vivo,
ella estaría cometiendo adulterio. Sin embargo, si su esposo muere y
ella se casa con otro hombre, entonces ella no sería culpable de
adulterio. \bibleverse{4} Del mismo modo, mis amigos, ustedes han muerto
para la ley mediante el cuerpo de Cristo, y ahora ustedes le pertenecen
a otro, a Cristo, quien ha resucitado de los muertos para que nosotros
pudiéramos vivir una vida productiva\footnote{\textbf{7:4} Literalmente,
  ``que lleve fruto para Dios''.} para Dios. \bibleverse{5} Cuando
éramos controlados por la vieja naturaleza, nuestros deseos pecaminosos
(tal como los revela la ley) obraban dentro de nosotros y traían como
resultado la muerte. \footnote{\textbf{7:5} Rom 6,21} \bibleverse{6}
Pero ahora hemos sido libertados de la ley, y hemos muerto a lo que nos
encadenaba, a fin de que podamos servir de un nuevo modo, en el
Espíritu, y no a la manera de la antigua letra de la ley. \footnote{\textbf{7:6}
  Rom 8,1-2; Rom 6,2; Rom 6,4}

\hypertarget{el-efecto-calamitoso-de-la-ley-que-familiariza-al-hombre-con-el-pecado-y-le-da-vida-al-pecado-en-la-carne}{%
\subsection{El efecto calamitoso de la ley, que familiariza al hombre
con el pecado y le da vida al pecado en la
carne}\label{el-efecto-calamitoso-de-la-ley-que-familiariza-al-hombre-con-el-pecado-y-le-da-vida-al-pecado-en-la-carne}}

\bibleverse{7} ¿Qué concluimos entonces? ¿Que la ley es pecado? ¡Por
supuesto que no! Pues yo no habría conocido lo que era el pecado si no
fuera porque la ley lo define. Yo no me habría dado cuenta de que el
deseo de tener las cosas de otras personas estaba mal si no fuera porque
la ley dice: ``No desees para ti lo que le pertenece a
otro''.\footnote{\textbf{7:7} Citando Éxodo 20:17 o Deuteronomio 5:21.}
\bibleverse{8} Pero a través de este mandamiento el pecado encontró la
manera de despertar en mí todo tipo de deseos egoístas. Porque sin la
ley, el pecado está muerto. \footnote{\textbf{7:8} Rom 5,13; 1Cor 15,56}
\bibleverse{9} Yo solía vivir sin darme cuenta de lo que la ley
realmente significaba, pero cuando comprendí las implicaciones de ese
mandamiento, entonces el pecado volvió a la vida y morí. \bibleverse{10}
Descubrí que el mismo mandamiento que tenía como propósito traerme vida,
me trajo muerte en lugar de ello, \bibleverse{11} porque el pecado
encontró su camino a través del mandamiento para engañarme, y lo usó
para matarme. \footnote{\textbf{7:11} Heb 3,13} \bibleverse{12} Sin
embargo, la ley es santa, y el mandamiento es santo, justo y recto.
\footnote{\textbf{7:12} 1Tim 1,8}

\bibleverse{13} Ahora, ¿acaso podría matarme algo que es bueno? ¡Por
supuesto que no! Pero el pecado se muestra como pecado usando lo bueno
para causar mi muerte. Así que por medio del mandamiento se revela cuán
malo es el pecado realmente. \footnote{\textbf{7:13} Rom 5,20}

\hypertarget{la-impotencia-de-la-ley-y-de-la-buena-voluntad-ante-el-pecado-como-poder-en-la-carne}{%
\subsection{La impotencia de la ley y de la buena voluntad ante el
pecado como poder en la
carne}\label{la-impotencia-de-la-ley-y-de-la-buena-voluntad-ante-el-pecado-como-poder-en-la-carne}}

\bibleverse{14} Comprendemos que la ley es espiritual, pero yo soy
totalmente humano,\footnote{\textbf{7:14} Literalmente, ``carne''.} un
esclavo del pecado. \footnote{\textbf{7:14} Juan 3,6} \bibleverse{15}
Realmente no entiendo lo que hago. ¡Hago las cosas que no quiero hacer,
y lo que odio hacer es precisamente lo que hago! \bibleverse{16} Pero si
digo que hago lo que no quiero hacer, esto demuestra que yo admito que
la ley es buena. \bibleverse{17} De modo que ya no soy yo quien hace
esto, sino el pecado que vive en mí \bibleverse{18} porque yo sé que no
hay nada bueno en mí en lo que tiene que ver con mi naturaleza humana
pecaminosa. Aunque quiero hacer el bien, simplemente no puedo hacerlo.
\footnote{\textbf{7:18} Gén 6,5; Gén 8,21} \bibleverse{19} ¡El bien que
quiero hacer no lo hago; mientras que el mal que no quiero hacer es lo
que termino haciendo! \bibleverse{20} Sin embargo, si hago lo que no
quiero, entonces ya no soy yo quien lo hace, sino el pecado que vive en
mí. \bibleverse{21} Este es el principio que he descubierto: si quiero
hacer lo bueno, el mal también está siempre ahí. \bibleverse{22} Mi ser
interior se deleita en la ley de Dios, \bibleverse{23} pero veo que hay
una ley distinta que obra dentro de mí y que está en guerra con la ley
que mi mente ha decidido seguir, convirtiéndome en un prisionero de la
ley de pecado que está dentro de mí. \bibleverse{24} ¡Cuán miserable
soy! ¿Quién me rescatará de este cuerpo que causa mi muerte?\footnote{\textbf{7:24}
  Literalmente, ``cuerpo de muerte''.} ¡Gracias a Dios, porque él me
salva a través de Jesucristo, nuestro Señor! \bibleverse{25} La
situación es esta: Aunque yo mismo elijo en mi mente obedecer la ley de
Dios, mi naturaleza humana obedece la ley del pecado.\footnote{\textbf{7:25}
  1Cor 15,57}

\hypertarget{el-cristiano-estuxe1-bajo-la-ley-del-espuxedritu}{%
\subsection{El cristiano está bajo la ley del
Espíritu}\label{el-cristiano-estuxe1-bajo-la-ley-del-espuxedritu}}

\hypertarget{section-7}{%
\section{8}\label{section-7}}

\bibleverse{1} Así que ahora no hay condenación para los que están en
Cristo Jesús. \footnote{\textbf{8:1} Rom 8,33-34} \bibleverse{2} La ley
del Espíritu de vida en Jesucristo me ha libertado de la ley del pecado
y muerte. \bibleverse{3} Lo que la ley no pudo hacer porque no tenía el
poder para hacerlo debido a nuestra naturaleza pecaminosa,\footnote{\textbf{8:3}
  ``Naturaleza pecaminosa'', literalmente ``carne'', refiriéndose a la
  naturaleza física pecaminosa y caída de la humanidad. A menudo se usa
  esta palabra en los versículos que siguen para hacer un contraste con
  la naturaleza espiritual.} Dios pudo hacerlo. Al enviar a su propio
Hijo en forma humana, Dios se hizo cargo del problema del
pecado\footnote{\textbf{8:3} O ``hacienda un sacrificio de sí mismo por
  el pecado''.} y destruyó el poder del pecado en nuestra naturaleza
humana pecaminosa. \footnote{\textbf{8:3} Hech 13,38; Hech 15,10; Heb
  2,17} \bibleverse{4} De este modo, pudimos cumplir los buenos
requisitos de la ley, siguiendo al Espíritu y no a nuestra naturaleza
pecaminosa. \footnote{\textbf{8:4} Gal 5,16; Gal 5,25}

\hypertarget{el-contraste-entre-los-que-sirven-a-dios-en-el-espuxedritu-y-los-que-viven-por-los-instintos-de-la-carne}{%
\subsection{El contraste entre los que sirven a Dios en el Espíritu y
los que viven por los instintos de la
carne}\label{el-contraste-entre-los-que-sirven-a-dios-en-el-espuxedritu-y-los-que-viven-por-los-instintos-de-la-carne}}

\bibleverse{5} Aquellos que siguen su naturaleza pecaminosa están
preocupados por cosas pecaminosas, pero los que siguen al Espíritu, se
concentran en cosas espirituales. \bibleverse{6} La mente humana y
pecaminosa lleva a la muerte, pero cuando la mente es guiada por el
Espíritu, trae vida y paz. \footnote{\textbf{8:6} Rom 6,21; Gal 6,8}
\bibleverse{7} La mente humana y pecaminosa es reacia a Dios porque se
niega a obedecer la ley de Dios. Y de hecho, no puede hacerlo;
\footnote{\textbf{8:7} Sant 4,4} \bibleverse{8} y aquellos que siguen su
naturaleza pecaminosa no pueden agradar a Dios.

\hypertarget{el-cristiano-como-morada-del-espuxedritu}{%
\subsection{El cristiano como morada del
Espíritu}\label{el-cristiano-como-morada-del-espuxedritu}}

\bibleverse{9} Pero ustedes no siguen su naturaleza pecaminosa sino al
Espíritu, si es que el Espíritu de Dios vive en ustedes. Porque aquellos
que no tienen el Espíritu de Cristo dentro de ellos, no le pertenecen a
él. \bibleverse{10} Sin embargo, si Cristo está en ustedes, aunque su
cuerpo vaya a morir por causa del pecado, el Espíritu les da vida porque
ahora ustedes están justificados delante de Dios. \footnote{\textbf{8:10}
  Gal 2,20} \bibleverse{11} El Espíritu de Dios que levantó a Jesús de
los muertos, vive en ustedes. Él, que levantó a Jesús de los muertos,
dará vida a sus cuerpos muertos a través de su Espíritu que vive en
ustedes.

\hypertarget{la-posesiuxf3n-del-espuxedritu-garantiza-la-redenciuxf3n-fuxedsica-de-los-hijos-de-dios-si-soportan-los-sufrimientos-de-este-tiempo}{%
\subsection{La posesión del espíritu garantiza la redención física de
los hijos de Dios si soportan los sufrimientos de este
tiempo}\label{la-posesiuxf3n-del-espuxedritu-garantiza-la-redenciuxf3n-fuxedsica-de-los-hijos-de-dios-si-soportan-los-sufrimientos-de-este-tiempo}}

\bibleverse{12} Así que, hermanos y hermanas, no tenemos que
seguir\footnote{\textbf{8:12} O ``no tenemos obligación''.} nuestra
naturaleza pecaminosa que obra conforme a nuestros deseos humanos.
\bibleverse{13} Porque si ustedes viven bajo el control de su naturaleza
pecaminosa, van a morir. Pero si siguen el camino del Espíritu, dando
muerte a las cosas malas que hacen, entonces vivirán. \footnote{\textbf{8:13}
  Rom 7,24; Gal 6,8; Efes 4,22-24} \bibleverse{14} Todos los que son
guiados por el Espíritu de Dios son hijos de Dios. \bibleverse{15} No se
les ha dado un espíritu de esclavitud ni de temor una vez más. No, lo
que recibieron fue el espíritu que los convierte en hijos, para que
estén dentro de la familia de Dios. Ahora podemos decir a viva voz:
``¡Dios es nuestro Padre!''

\bibleverse{16} El Espíritu mismo está de acuerdo con
nosotros\footnote{\textbf{8:16} Literalmente, ``nuestro espíritu''.} en
que somos hijos de Dios. \footnote{\textbf{8:16} 2Cor 1,22}
\bibleverse{17} Y si somos sus hijos, entonces somos sus herederos.
Somos herederos de Dios, y herederos junto con Cristo. Pero si queremos
participar de su gloria, debemos participar de sus sufrimientos.
\footnote{\textbf{8:17} Gal 4,7; Apoc 21,7}

\bibleverse{18} Sin embargo, estoy convencido de que lo que sufrimos en
el presente no es nada si lo comparamos con la gloria futura que se nos
revelará. \footnote{\textbf{8:18} 2Cor 4,17} \bibleverse{19} Toda la
creación espera con paciencia, anhelando que Dios se revele a sus hijos.
\footnote{\textbf{8:19} Col 3,4; 1Jn 3,2} \bibleverse{20} Porque Dios
permitió que fuera frustrado el propósito de la creación. \footnote{\textbf{8:20}
  Gén 3,17; Ecl 1,2} \bibleverse{21} Pero la creación misma mantiene la
esperanza puesta en ese momento en que será liberada de la esclavitud de
la degradación y participará de la gloriosa libertad de los hijos de
Dios. \footnote{\textbf{8:21} 2Pe 3,13} \bibleverse{22} Sabemos que toda
la creación clama con anhelo, sufriendo dolores de parto hasta hoy.
\bibleverse{23} Y no solo la creación, sino que nosotros también,
quienes tenemos un anticipo del Espíritu, y clamamos por dentro mientras
esperamos que Dios nos ``adopte'', que realice la redención de nuestros
cuerpos. \footnote{\textbf{8:23} 2Cor 5,2} \bibleverse{24} Sin embargo,
la esperanza que ya ha sido vista no es esperanza en absoluto. ¿Acaso
quién espera lo que ya puede ver? \footnote{\textbf{8:24} 2Cor 5,7}
\bibleverse{25} Como nosotros esperamos lo que no hemos visto todavía,
esperamos pacientemente por ello. \footnote{\textbf{8:25} Gal 5,5}

\bibleverse{26} De la misma manera, el Espíritu nos ayuda en nuestra
debilidad. Nosotros no sabemos cómo hablar con Dios, pero el Espíritu
mismo intercede con nosotros y por nosotros mediante gemidos que las
palabras no pueden expresar. \bibleverse{27} Aquél que examina las
mentes de todos conoce las motivaciones del Espíritu,\footnote{\textbf{8:27}
  O, ``la mente del Espíritu''.} porque el Espíritu aboga la causa de
Dios en favor de los creyentes.

\hypertarget{el-comienzo-de-nuestra-comuniuxf3n-con-dios-obra-de-dios-garantiza-su-finalizaciuxf3n-final}{%
\subsection{El comienzo de nuestra comunión con Dios, obra de Dios,
garantiza su finalización
final}\label{el-comienzo-de-nuestra-comuniuxf3n-con-dios-obra-de-dios-garantiza-su-finalizaciuxf3n-final}}

\bibleverse{28} Sabemos que en todas las cosas Dios obra para el bien de
los que le aman, aquellos a quienes él ha llamado para formar parte de
su plan. \bibleverse{29} Porque Dios, escogiéndolos de antemano, los
separó para ser como su Hijo, a fin de que el Hijo pudiera ser el
primero de muchos hermanos y hermanas. \footnote{\textbf{8:29} Col 1,18;
  Heb 1,6} \bibleverse{30} A los que escogió también llamó, y a aquellos
a quienes llamó también justificó, y a quienes justificó también
glorificó. \footnote{\textbf{8:30} Rom 3,26; 2Tes 2,13-14}

\hypertarget{por-tanto-nuestro-estado-de-salvaciuxf3n-estuxe1-divinamente-asegurado-contra-todos-los-poderes-y-nuestra-certeza-de-fe-y-seguridad-de-la-salvaciuxf3n-estuxe1-justificada}{%
\subsection{Por tanto, nuestro estado de salvación está divinamente
asegurado contra todos los poderes y nuestra certeza de fe y seguridad
de la salvación está
justificada}\label{por-tanto-nuestro-estado-de-salvaciuxf3n-estuxe1-divinamente-asegurado-contra-todos-los-poderes-y-nuestra-certeza-de-fe-y-seguridad-de-la-salvaciuxf3n-estuxe1-justificada}}

\bibleverse{31} ¿Cuál es, entonces, nuestra respuesta a todo esto? Si
Dios está a nuestro favor, ¿quién puede estar en contra de nosotros?
\footnote{\textbf{8:31} Sal 118,6} \bibleverse{32} Dios, quien no retuvo
a su propio Hijo, sino que lo entregó por todos nosotros, ¿no nos dará
gratuitamente todas las cosas? \footnote{\textbf{8:32} Juan 3,16}
\bibleverse{33} ¿Quién puede acusar de alguna cosa al pueblo de Dios? Es
Dios quien nos justifica, \bibleverse{34} así que ¿quién puede
condenarnos? Fue Cristo quien murió---y más importante aún, quien se
levantó de los muertos---el que se sienta a la diestra de Dios,
presentando nuestro caso. \footnote{\textbf{8:34} 1Jn 2,1; Heb 7,25}

\bibleverse{35} ¿Quién puede separarnos del amor de Cristo? ¿Acaso la
opresión, la angustia, o la persecución? ¿O acaso el hambre, la pobreza,
el peligro, o la violencia? \bibleverse{36} Tal como dice la Escritura:
``Por tu causa estamos todo el tiempo en peligro de morir. Somos
tratados como ovejas que serán llevadas al sacrificio''.\footnote{\textbf{8:36}
  CitandoSalmos 44:22.} \bibleverse{37} No.~En todas las cosas que nos
suceden somos más que vencedores por medio de Aquél que nos amó.
\footnote{\textbf{8:37} 1Jn 5,4} \bibleverse{38} Por eso estoy
plenamente convencido de que ni la muerte, ni la vida, ni los ángeles,
ni los demonios, ni el presente, ni el futuro, ni las potencias,
\footnote{\textbf{8:38} Efes 6,12}

\bibleverse{39} ni lo alto, ni lo profundo, y, de hecho, ninguna cosa en
toda la creación puede separarnos del amor de Dios en Jesucristo,
nuestro Señor.

\hypertarget{introducciuxf3n-el-profundo-dolor-del-apuxf3stol-por-la-exclusiuxf3n-temporal-de-su-pueblo-de-la-salvaciuxf3n}{%
\subsection{Introducción: El profundo dolor del apóstol por la exclusión
temporal de su pueblo de la
salvación}\label{introducciuxf3n-el-profundo-dolor-del-apuxf3stol-por-la-exclusiuxf3n-temporal-de-su-pueblo-de-la-salvaciuxf3n}}

\hypertarget{section-8}{%
\section{9}\label{section-8}}

\bibleverse{1} Yo estoy en Cristo, y lo que digo es verdad. ¡No les
miento! Mi conciencia y el Espíritu Santo confirman \bibleverse{2} cuán
triste estoy, y el dolor infinito que tengo en mi corazón \bibleverse{3}
por mi propio pueblo, por mis hermanos y hermanas. Preferiría yo mismo
ser maldecido, estar separado de Cristo, si eso pudiera ayudarlos.
\footnote{\textbf{9:3} Éxod 32,32} \bibleverse{4} Ellos son mis hermanos
de raza, los israelitas, el pueblo escogido de Dios. Dios les reveló su
gloria e hizo tratados\footnote{\textbf{9:4} Literalmente, ``pactos''.}
con ellos, dándoles la ley, el verdadero culto, y sus promesas.
\footnote{\textbf{9:4} Éxod 4,22; Deut 7,6; Gén 17,7; Éxod 20,-1; Éxod
  40,34} \bibleverse{5} Ellos son nuestros antepasados, ancestros de
Cristo, humanamente hablando, de Aquél que gobierna sobre todo, el Dios
bendito por la eternidad. Amén. \footnote{\textbf{9:5} Mat 1,-1; Luc
  3,23-34; Juan 1,1; Rom 1,3}

\hypertarget{las-promesas-de-dios-a-israel-son-inquebrantables-pero-no-se-aplican-a-todo-el-cuerpo-sino-solo-a-los-descendientes-espirituales-de-abraham}{%
\subsection{Las promesas de Dios a Israel son inquebrantables, pero no
se aplican a todo el cuerpo, sino solo a los descendientes espirituales
de
Abraham}\label{las-promesas-de-dios-a-israel-son-inquebrantables-pero-no-se-aplican-a-todo-el-cuerpo-sino-solo-a-los-descendientes-espirituales-de-abraham}}

\bibleverse{6} No es que la promesa de Dios haya fallado. Porque no todo
israelita es un verdadero israelita, \footnote{\textbf{9:6} Núm 23,19;
  Rom 2,28} \bibleverse{7} y no todos los que son descendientes de
Abraham son sus verdaderos hijos. Pues la Escritura dice: ``Tus
descendientes serán contados por medio de Isaac'',\footnote{\textbf{9:7}
  Citando Génesis 21:12.} \bibleverse{8} de modo que no son los hijos
reales de Abrahán los que se cuentan como hijos de Dios, sino que son
considerados como sus verdaderos descendientes solo los hijos de la
promesa. \footnote{\textbf{9:8} Gal 4,23} \bibleverse{9} Y esta fue la
promesa: ``Regresaré el próximo año y Sara tendrá un hijo''.\footnote{\textbf{9:9}
  Citando Génesis 18:10-14.} \bibleverse{10} Además, los hijos gemelos
de Rebeca tenían el mismo padre, nuestro antepasado Isaac.
\bibleverse{11} Pero incluso antes de que los niños nacieran, y antes de
que hubieran hecho algo bueno o malo, (a fin de que pudiera continuar el
propósito de Dios, demostrando que el llamado de Dios a las personas no
está basado en la conducta humana), \bibleverse{12} a ella se le dijo:
``El hermano mayor servirá al hermano menor''.\footnote{\textbf{9:12}
  Citando Génesis 25:23.} \bibleverse{13} Como dice la Escritura: ``Yo
escogí a Jacob, pero rechacé a Esaú''.\footnote{\textbf{9:13} Citando
  Malaquías 1:2-3.}

\hypertarget{la-elecciuxf3n-para-la-salvaciuxf3n-es-obra-gratuita-de-la-gracia-de-dios-la-negaciuxf3n-de-la-salvaciuxf3n-y-la-gracia-no-permite-al-hombre-pelear-con-dios}{%
\subsection{La elección para la salvación es obra gratuita de la gracia
de Dios; la negación de la salvación y la gracia no permite al hombre
pelear con
Dios}\label{la-elecciuxf3n-para-la-salvaciuxf3n-es-obra-gratuita-de-la-gracia-de-dios-la-negaciuxf3n-de-la-salvaciuxf3n-y-la-gracia-no-permite-al-hombre-pelear-con-dios}}

\bibleverse{14} Entonces, ¿qué debemos concluir? ¿Diremos que Dios es
injusto? ¡Por supuesto que no! \bibleverse{15} Como dijo a Moisés:
``Tendré misericordia de quien deba tener misericordia, y tendré
compasión de quien deba tener compasión''.\footnote{\textbf{9:15}
  Citando Éxodo 33:19.} \bibleverse{16} De modo que no depende de lo que
nosotros queremos o de nuestros propios esfuerzos, sino del carácter
misericordioso de Dios. \bibleverse{17} La Escritura registra que Dios
le dijo al Faraón: ``Te puse aquí por una razón: para que por ti yo
pudiera demostrar mi poder, y para que mi nombre sea conocido por toda
la tierra''.\footnote{\textbf{9:17} Citando Éxodo 9:16.} \bibleverse{18}
De modo que Dios es misericordioso con quienes él desea serlo, y
endurece el corazón de quienes él desea\footnote{\textbf{9:18} En el
  Antiguo Testamento esta expresión se usa para describir un rechazo
  obstinado por Dios, tal como la experiencia del Faraón de Éxodo. En
  Citando Éxodo 9 Faraón es presentado en varias ocasiones con corazón
  endurecido, o menciona que Dios endurecía su corazón, o en voz pasiva,
  diciendo que su corazón era endurecido. De manera que este versículo
  en el libro de Romanos no debe usarse para decir que Dios
  deliberadamente endurece el corazón de las personas y luego los
  castiga por ello. El endurecimiento del corazón es un rechazo a la
  gracia divina.} . \footnote{\textbf{9:18} Éxod 4,21; 1Pe 2,8}

\bibleverse{19} Ahora bien, ustedes discutirán conmigo y preguntarán:
``Entonces, ¿por qué sigue culpándonos? ¿Quién puede oponerse a la
voluntad de Dios?'' \bibleverse{20} Y esa no es manera de hablar, porque
¿quién eres tú, ---un simple mortal---, para contradecir a Dios? ¿Puede
alguna cosa creada decirle a su creador: ``por qué me hiciste así?''
\bibleverse{21} ¿Acaso el alfarero no tiene el derecho de usar la misma
arcilla ya sea para hacer una vasija decorativa o una vasija
común?\footnote{\textbf{9:21} Literalmente, ``vasijas de valor y
  deshonra''.} \bibleverse{22} Es como si Dios, queriendo demostrar su
oposición al pecado\footnote{\textbf{9:22} Literalmente ``mostrar su
  ira''.} y para revelar su poder, soportara con paciencia estas
``vasijas destinadas a la destrucción'', \footnote{\textbf{9:22} Rom
  2,4; Prov 16,4} \bibleverse{23} a fin de revelar la grandeza de su
gloria mediante estas ``vasijas de misericordia'', las cuales él ha
preparado de antemano para la gloria. \footnote{\textbf{9:23} Rom 8,29;
  Efes 1,3-12} \bibleverse{24} Esto es lo que somos, personas que él ha
llamado, no solo de entre los judíos, sino de entre los extranjeros
también\ldots{} \bibleverse{25} Como dijo Dios en el libro de Oseas:
``Llamaré mi pueblo a los que no son mi pueblo, y a los que no son
amados llamaré mis amados'',\footnote{\textbf{9:25} Citando Oseas 2:23.}
\bibleverse{26} y ``sucederá que en el lugar donde les dijeron `tú no
eres mi pueblo' serán llamados hijos del Dios viviente''.\footnote{\textbf{9:26}
  Citando Oseas 1:10.}

\bibleverse{27} Isaías clama, respecto a Israel: ``Aun cuando los hijos
de Israel han llegado a ser tantos como la arena del mar, solo unos
cuantos\footnote{\textbf{9:27} Literalmente, ``remanente''.} se
salvarán. \footnote{\textbf{9:27} Rom 11,5} \bibleverse{28} Porque el
Señor terminará rápida y completamente su obra de juicio sobre la
tierra''.

\bibleverse{29} Como había dicho antes Isaías: ``Si el Señor
Todopoderoso no nos hubiera dejado algunos descendientes, nos habríamos
convertido en algo semejante a Sodoma y Gomorra''.\footnote{\textbf{9:29}
  Citando Isaías 1:9.}

\hypertarget{la-culpa-de-los-juduxedos-consistiuxf3-en-el-rechazo-de-la-justicia-de-la-fe-y-en-la-persecuciuxf3n-excesiva-de-la-justicia-de-las-obras}{%
\subsection{La culpa de los judíos consistió en el rechazo de la
justicia de la fe y en la persecución excesiva de la justicia de las
obras}\label{la-culpa-de-los-juduxedos-consistiuxf3-en-el-rechazo-de-la-justicia-de-la-fe-y-en-la-persecuciuxf3n-excesiva-de-la-justicia-de-las-obras}}

\bibleverse{30} ¿Qué concluiremos, entonces? Que aunque los extranjeros
ni siquiera procuraban hacer lo recto, comprendieron lo recto, y por
medio de su fe en Dios hicieron lo recto.

\bibleverse{31} Pero el pueblo de Israel, que seguía la ley, para que
ella los justificara con Dios, nunca lo logró. \footnote{\textbf{9:31}
  Rom 10,2-3}

\bibleverse{32} ¿Por qué no? Porque dependían de lo que hacían y no de
su confianza en Dios. Tropezaron con la piedra de tropiezo,
\bibleverse{33} tal como lo predijo la Escritura: ``Miren, en Sión pongo
una piedra de tropiezo, una roca que ofenderá a la gente. Pero los que
confían en él, no serán frustrados''.\footnote{\textbf{9:33} Citando
  Isaías 28:16, y Isaías 8:14.}

\hypertarget{section-9}{%
\section{10}\label{section-9}}

\bibleverse{1} Mis hermanos y hermanas, el deseo de mi corazón---mi
oración a Dios---es la salvación del pueblo de Israel. \bibleverse{2}
Puedo dar testimonio de su ferviente dedicación a Dios, pero esta
dedicación no está basada en conocerlo como él realmente es.
\bibleverse{3} Ellos no comprenden cómo Dios nos hace justos, y tratan
de justificarse a sí mismos. Se niegan a aceptar la manera en que Dios
justifica a las personas.

\hypertarget{la-falta-de-israel-es-auxfan-muxe1s-grave-ya-que-dios-no-ha-descuidado-nada-para-llevar-a-israel-a-la-justicia-de-la-fe-desde-la-uxe9poca-de-moisuxe9s}{%
\subsection{La falta de Israel es aún más grave ya que Dios no ha
descuidado nada para llevar a Israel a la justicia de la fe desde la
época de
Moisés}\label{la-falta-de-israel-es-auxfan-muxe1s-grave-ya-que-dios-no-ha-descuidado-nada-para-llevar-a-israel-a-la-justicia-de-la-fe-desde-la-uxe9poca-de-moisuxe9s}}

\bibleverse{4} Porque Cristo es el cumplimiento de la ley. Todos los que
confían en él son justificados. \footnote{\textbf{10:4} Mat 5,17; Heb
  8,13; Juan 3,18; Gal 3,24-25}

\bibleverse{5} Moisés escribió: ``Todo el que hace lo recto mediante la
obediencia de la ley, vivirá''.\footnote{\textbf{10:5} Citando Levítico
  18:5.} \bibleverse{6} Pero la disposición de hacer lo recto que
proviene de la fe, dice esto: ``No preguntes `¿quién subirá al cielo?'
(Pidiendo que Cristo descienda a nosotros), \bibleverse{7} o `¿quién irá
al lugar de los muertos?'\footnote{\textbf{10:7} Literalmente, ``el
  abismo'', pozo sin fondo.} (Pidiendo que Cristo regrese de entre los
muertos)''.\footnote{\textbf{10:7} Ver Deuteronomio 30:12.}
\bibleverse{8} Lo que la Escritura realmente dice es: ``Este mensaje
está muy cerca de ti, es lo que hablas y lo que está en tu
mente''.\footnote{\textbf{10:8} Estas son alusiones a Deuteronomio
  30:11-14. Originalmente se aplicaban a la ley, y servían para indicar
  que la ley no era distante e inalcanzable, negando claramente que
  fuera difícil su observancia. Ahora Pablo lo aplica a la persona de
  Cristo, aclarando que este ``mensaje de la ley'' se cumplió en él.} De
hecho, lo que estamos mostrando es este mensaje, basado en la fe.
\bibleverse{9} Porque si declaras que aceptas a Jesús como Señor, y
estás convencido en tu mente de que Dios lo levantó de los muertos,
entonces serás salvo. \bibleverse{10} Tu fe en Dios te hace justo, y tu
declaración de aceptación a Dios te salva. \bibleverse{11} Como dice la
Escritura: ``Los que creen en él no serán frustrados''.\footnote{\textbf{10:11}
  Citando Isaías 28:16. Frustrados: o ``avergonzados''.}

\bibleverse{12} No hay diferencia entre judío y griego, porque el mismo
Señor es Señor de todos, y da generosamente a todos los que le piden.
\footnote{\textbf{10:12} Hech 10,34-35; Hech 15,9} \bibleverse{13}
Porque ``todo el que invoque el nombre del Señor será
salvo''.\footnote{\textbf{10:13} Citando Joel 2:32.} \bibleverse{14}
Pero ¿cómo podrá la gente invocar a alguien en quien no creen? ¿Cómo
podrían creer en alguien de quien no han escuchado hablar? ¿Y cómo
podrían escuchar si no se les habla? \bibleverse{15} ¿Cómo podrán ir a
hablarles si no se les envía? Tal como dice la Escritura: ``Bienvenidos
son los que traen la buena noticia!''\footnote{\textbf{10:15} Citando
  Isaías 52:7.}

\hypertarget{la-inexcusabilidad-de-la-parte-incruxe9dula-de-israel-que-ha-rechazado-la-salvaciuxf3n-que-tambiuxe9n-le-fue-ofrecida}{%
\subsection{La inexcusabilidad de la parte incrédula de Israel, que ha
rechazado la salvación que también le fue
ofrecida}\label{la-inexcusabilidad-de-la-parte-incruxe9dula-de-israel-que-ha-rechazado-la-salvaciuxf3n-que-tambiuxe9n-le-fue-ofrecida}}

\bibleverse{16} Pero no todos han aceptado la buena noticia. Como
pregunta Isaías: ``Señor, ¿quién creyó en la noticia de la que nos
oyeron hablar?''\footnote{\textbf{10:16} Citando Isaías 53:1.}
\bibleverse{17} Creer en Dios viene de oír, de oír el mensaje de Cristo.
\bibleverse{18} Y no es que no hayan oído. Muy por el contrario: ``Las
voces de los que hablan de Dios\footnote{\textbf{10:18} Implícito.} se
han oído por toda la tierra. Su mensaje se extendió por todo el
mundo''.\footnote{\textbf{10:18} Citando Salmos 19:4.} \footnote{\textbf{10:18}
  Rom 15,19}

\bibleverse{19} Así que mi pregunta es: ``¿No sabía Israel?'' Primero
que nada, Moisés dice: ``Les haré sentir celos usando un pueblo que ni
siquiera es una nación; los haré enojarse usando extranjeros
ignorantes''.\footnote{\textbf{10:19} Citando Deuteronomio 32:21.}

\bibleverse{20} Luego Isaías lo dijo con mayor vehemencia: ``Fui
encontrado por personas que ni siquiera me estaban buscando; me presenté
a personas que ni siquiera estaban preguntando por mí''.\footnote{\textbf{10:20}
  Citando Isaías 65:1.}

\bibleverse{21} Como dice Dios a Israel: ``Todo el día extendí mis manos
a un pueblo desobediente y terco''.\footnote{\textbf{10:21} Citando
  Isaías 65:2.}

\hypertarget{la-mayor-parte-de-los-juduxedos-es-terca-y-rechazada-por-dios-pero-incluso-ahora-una-pequeuxf1a-parte-estuxe1-destinada-a-la-salvaciuxf3n-a-travuxe9s-de-la-gracia-de-dios}{%
\subsection{La mayor parte de los judíos es terca y rechazada por Dios,
pero incluso ahora una pequeña parte está destinada a la salvación a
través de la gracia de
Dios}\label{la-mayor-parte-de-los-juduxedos-es-terca-y-rechazada-por-dios-pero-incluso-ahora-una-pequeuxf1a-parte-estuxe1-destinada-a-la-salvaciuxf3n-a-travuxe9s-de-la-gracia-de-dios}}

\hypertarget{section-10}{%
\section{11}\label{section-10}}

\bibleverse{1} Pero entonces pregunto: ``¿Acaso Dios ha rechazado a su
pueblo?'' ¡Por supuesto que no! Yo mismo soy israelita, de la tribu de
Benjamín. \footnote{\textbf{11:1} Sal 94,14; Jer 31,37; Fil 3,5}
\bibleverse{2} Dios no ha rechazado a su pueblo escogido. ¿Acaso no
recuerdan lo que dice la Escritura acerca de Elías? Cómo se quejó de
Israel ante Dios, diciendo: \bibleverse{3} ``Señor, han matado a tus
profetas y han destruido tus altares. ¡Soy el único que queda y también
están tratando de matarme!'' \bibleverse{4} ¿Cómo le respondió Dios?
``Aun me quedan siete mil personas que no han adorado a
Baal''.\footnote{\textbf{11:4} Citando 1 Reyes 19:10-14.} \bibleverse{5}
Hoy sucede exactamente lo mismo: aún quedan algunas personas fieles,
escogidas por la gracia de Dios. \bibleverse{6} Y como es por medio de
la gracia, entonces claramente no se basa en lo que la gente hace, ¡de
otro modo no sería gracia!

\bibleverse{7} ¿Qué concluiremos, entonces? Que el pueblo de Israel no
logró aquello por lo que estaba luchando. Solo los escogidos, mientras
que el resto endureció su corazón. \footnote{\textbf{11:7} Rom 9,31}
\bibleverse{8} Como dice la Escritura: ``Dios opacó sus mentes para que
sus ojos no pudieran ver y sus oídos no pudieran oír, hasta el día de
hoy''.\footnote{\textbf{11:8} Citando Deuteronomio 29:4; Isaías 6:9-10;
  Isaías 29:10.} \footnote{\textbf{11:8} Deut 29,3}

\bibleverse{9} David agrega: ``Que sus fiestas se conviertan en una
trampa para ellos, una red que los atrape, una tentación que traiga
castigo. \bibleverse{10} Que sus ojos se vuelvan ciegos para que no
puedan ver, y que sus espaldas siempre estén dobladas de
abatimiento''.\footnote{\textbf{11:10} Citando Salmos 69:22-23.}

\hypertarget{el-propuxf3sito-divino-de-la-salvaciuxf3n-en-el-llamado-de-los-gentiles-era-vencer-la-incredulidad-de-los-juduxedos-estimuluxe1ndolos-a-emularlos-su-rechazo-no-es-definitivo}{%
\subsection{El propósito divino de la salvación en el llamado de los
gentiles era vencer la incredulidad de los judíos estimulándolos a
emularlos; su rechazo no es
definitivo}\label{el-propuxf3sito-divino-de-la-salvaciuxf3n-en-el-llamado-de-los-gentiles-era-vencer-la-incredulidad-de-los-juduxedos-estimuluxe1ndolos-a-emularlos-su-rechazo-no-es-definitivo}}

\bibleverse{11} Ahora, ¿estoy diciendo que ellos tropezaron y fracasaron
por completo? ¡Por supuesto que no! Pero como resultado de sus errores,
la salvación llegó a otras naciones, para ``hacerlos sentir celos''.
\footnote{\textbf{11:11} Hech 13,46; Rom 10,19; Deut 32,21}
\bibleverse{12} Ahora pues, si su fracaso beneficia al mundo, y su
pérdida es de beneficio para los extranjeros, ¡cuánto más benéfico sería
si ellos lograran lo que debían llegar a ser!\footnote{\textbf{11:12}
  Implícito.}

\bibleverse{13} Ahora déjenme hablarles a ustedes, extranjeros. En tanto
que soy un misionero para los extranjeros, promuevo lo que hago
\bibleverse{14} para que de alguna manera pueda despertar celo en mi
pueblo y salvar a algunos de ellos. \bibleverse{15} Si el resultado del
rechazo de Dios hacia ellos es la reconciliación del mundo con Dios,
¡entonces el resultado de la aceptación de Dios hacia ellos sería como
si los muertos volvieran a vivir!

\bibleverse{16} Si la primera parte de la masa del pan es santa, también
lo es todo el resto; si las raíces de un árbol son santas, entonces
también lo son las ramas. \bibleverse{17} Ahora, si algunas de las ramas
han sido arrancadas, y tú---un brote silvestre de olivo---has sido
injertado, y has compartido con las demás ramas el beneficio de las
raíces del árbol de olivo, \footnote{\textbf{11:17} Efes 2,11-14}
\bibleverse{18} entonces no debes menospreciar a las demás ramas. Si te
sientes tentado a jactarte, recuerda que no eres tu quien sustenta a las
raíces, sino que las raíces te sustentan a ti. \footnote{\textbf{11:18}
  Juan 4,22} \bibleverse{19} Podrías presumir, diciendo: ``Las ramas
fueron cortadas, por ello pueden injertarme a mí''. \bibleverse{20} Todo
eso estaría bien, pero estas ramas fueron cortadas por su falta de fe en
Dios, y tú sigues allí por tu fe en él. De modo que no te tengas en un
alto concepto, sino sé respetuoso, \footnote{\textbf{11:20} 1Cor 10,12}
\bibleverse{21} porque si Dios no perdonó a las ramas que originalmente
estaban allí, a ti tampoco te perdonará. \bibleverse{22} De modo que
debes reconocer la bondad y también la dureza de Dios, pues fue duro con
los caídos, pero es bondadoso contigo siempre que confíes en su bondad,
de lo contrario también serías cortado. \bibleverse{23} Si estas ramas
no se niegan más a confiar en Dios, podrán ser injertadas también,
porque Dios puede injertarlas nuevamente. \bibleverse{24} Si tú pudiste
ser cortado de un árbol de olivo, y luego injertado de manera artificial
en un árbol de olivo cultivado, cuánto más fácilmente podrán ser
injertadas nuevamente, de manera natural, las ramas de su propio árbol.

\hypertarget{todo-el-resto-del-pueblo-de-israel-eventualmente-llegaruxe1-a-la-fe-despuuxe9s-de-que-las-elecciones-gentiles-se-conviertan-y-todo-seruxe1-usado-para-la-justificaciuxf3n-y-glorificaciuxf3n-de-dios}{%
\subsection{Todo el resto del pueblo de Israel eventualmente llegará a
la fe después de que las elecciones gentiles se conviertan, y todo será
usado para la justificación y glorificación de
Dios}\label{todo-el-resto-del-pueblo-de-israel-eventualmente-llegaruxe1-a-la-fe-despuuxe9s-de-que-las-elecciones-gentiles-se-conviertan-y-todo-seruxe1-usado-para-la-justificaciuxf3n-y-glorificaciuxf3n-de-dios}}

\bibleverse{25} Hermanos y hermanas, no quiero que pasen por alto esta
verdad que estaba oculta anteriormente, pues de lo contrario podrían
volverse arrogantes. El pueblo de Israel en parte se ha vuelto terco,
hasta que se complete la venida de los extranjeros. \footnote{\textbf{11:25}
  Juan 10,16} \bibleverse{26} Así es como Israel se salvará.\footnote{\textbf{11:26}
  Esto no busca enseñar sobre una salvación universal, sino que a este
  punto todo Israel (que está conformado tanto por extranjeros como por
  judíos que aceptan la salvación por medio de la gracia de Dios) serán
  salvados.} Como dice la Escritura: ``El Salvador vendrá de Sión, y él
hará volver a Jacob de su rebeldía contra Dios. \footnote{\textbf{11:26}
  Mat 23,39; Sal 14,7} \bibleverse{27} Mi promesa para ellos es que
borraré sus pecados''.\footnote{\textbf{11:27} Citando Isaías 59:20-21;
  y Isaías 27:9.}

\bibleverse{28} Aunque ellos son enemigos de la buena noticia, ---y esto
los beneficia a ustedes---aún son el pueblo escogido y amado por causa
de sus ancestros. \bibleverse{29} Los dones de Dios y su llamado no
pueden retirarse. \footnote{\textbf{11:29} Núm 23,19} \bibleverse{30} En
un tiempo ustedes desobedecieron a Dios, pero ahora Dios les ha mostrado
misericordia como resultado de la desobediencia de ellos.
\bibleverse{31} De la misma manera que ellos ahora son desobedientes
como lo eran ustedes, a ellos también se les mostrará misericordia como
la que ustedes recibieron. \bibleverse{32} Porque Dios trató a todos
como prisioneros por causa de su desobediencia, a fin de poder ser
misericordioso con todos.

\bibleverse{33} ¡Oh cuán profundas son las riquezas, la sabiduría y el
conocimiento de Dios! ¡Cuán increíbles son sus decisiones, y cuán
extraordinarios son sus métodos! \footnote{\textbf{11:33} Is 45,15; Is
  55,8-9} \bibleverse{34} ¿Quién puede conocer los pensamientos de
Dios?\footnote{\textbf{11:34} Citando Isaías 40:13.} ¿Quién puede darle
consejo? \footnote{\textbf{11:34} Jer 23,18; 1Cor 2,16} \bibleverse{35}
¿Quién le ha dado alguna vez a Dios algo que luego él tuviera la
obligación de pagárselo?\footnote{\textbf{11:35} Citando Job 41:11.}

\bibleverse{36} Todo proviene de él, todo existe por medio de él, y todo
es para él. ¡Gloria a Dios para siempre, amén!

\hypertarget{advertencia-general-como-entrada-santificaciuxf3n-de-la-vida-personal-a-travuxe9s-de-la-entrega-completa-a-dios}{%
\subsection{Advertencia general como entrada: santificación de la vida
personal a través de la entrega completa a
Dios}\label{advertencia-general-como-entrada-santificaciuxf3n-de-la-vida-personal-a-travuxe9s-de-la-entrega-completa-a-dios}}

\hypertarget{section-11}{%
\section{12}\label{section-11}}

\bibleverse{1} Así que yo los animo, mis hermanos y hermanas, por la
compasión de Dios\footnote{\textbf{12:1} O ``misericordia''.} por
ustedes, que dediquen sus cuerpos como una ofrenda viva que es santa y
agradable a Dios. Esta es la manera lógica de adorar. \footnote{\textbf{12:1}
  Rom 6,13} \bibleverse{2} No sigan los caminos de este mundo; por el
contrario, sean transformados por la renovación espiritual de sus
mentes, a fin de que puedan demostrar cómo es realmente la voluntad de
Dios: buena, agradable, y perfecta. \footnote{\textbf{12:2} Efes 4,23;
  Efes 5,10; Efes 5,17}

\hypertarget{exhortaciuxf3n-a-la-humildad-del-individuo-y-al-uso-fiel-de-los-dones-recibidos-al-servicio-de-la-comunidad}{%
\subsection{Exhortación a la humildad del individuo y al uso fiel de los
dones recibidos al servicio de la
comunidad}\label{exhortaciuxf3n-a-la-humildad-del-individuo-y-al-uso-fiel-de-los-dones-recibidos-al-servicio-de-la-comunidad}}

\bibleverse{3} Déjenme explicarles a todos ustedes, por la gracia que se
ha dado, que ninguno debería tener un concepto de sí mismo más alto que
el que debería tener. Ustedes deben tener un autoconcepto realista,
conforme a la medida de fe que Dios les ha dado. \footnote{\textbf{12:3}
  1Cor 4,6; 1Cor 12,11; Efes 4,7; Mat 20,26} \bibleverse{4} Así como hay
muchas partes del cuerpo, pero no todas hacen lo mismo, \footnote{\textbf{12:4}
  1Cor 12,12} \bibleverse{5} del mismo modo nosotros somos un cuerpo en
Cristo, aunque somos muchos. Y todos somos parte de los otros.
\footnote{\textbf{12:5} 1Cor 12,27; Efes 4,4; Efes 4,25} \bibleverse{6}
Cada uno tiene dones diferentes, que varían conforme a la gracia que se
nos ha dado. De modo que si el don consiste en hablar de Dios, entonces
debes hacerlo conforme a tu medida de fe en Dios. \footnote{\textbf{12:6}
  1Cor 4,7; 1Cor 12,4} \bibleverse{7} Si se trata del ministerio del
servicio, entonces debes servir; si se trata de enseñar, debes enseñar;
\footnote{\textbf{12:7} 1Pe 4,10-11} \bibleverse{8} si el don consiste
en animar a otros, entonces debes animar; si el don consiste en dar,
entonces da generosamente; si es el don del liderazgo, entonces lidera
con compromiso; si el don consiste en ser misericordioso, entonces hazlo
con alegría. \footnote{\textbf{12:8} Mat 6,3; 2Cor 8,2; 2Cor 9,7}

\hypertarget{exhortaciuxf3n-a-amar-fraternalmente-y-a-ejercitar-sentimientos-cristianos-contra-amigos-y-enemigos}{%
\subsection{Exhortación a amar fraternalmente y a ejercitar sentimientos
cristianos contra amigos y
enemigos}\label{exhortaciuxf3n-a-amar-fraternalmente-y-a-ejercitar-sentimientos-cristianos-contra-amigos-y-enemigos}}

\bibleverse{9} El amor debe ser genuino. Odien lo malo; aférrense a lo
bueno. \footnote{\textbf{12:9} 1Tim 1,5; Am 5,15} \bibleverse{10}
Dedíquense por completo unos a otros en su amor como familia, valorando
a los demás más que a ustedes mismos. \footnote{\textbf{12:10} Juan
  13,4-15; Fil 2,3} \bibleverse{11} No sean perezosos para el trabajo
arduo; sirvan al Señor con un espíritu entusiasta. \footnote{\textbf{12:11}
  Apoc 3,15; Hech 18,25; Col 3,23} \bibleverse{12} Permanezcan alegres
en la esperanza que tienen, soporten las pruebas que se presenten, y no
dejen de orar. \footnote{\textbf{12:12} 1Tes 5,17; Luc 18,1-8; Col 4,2}
\bibleverse{13} Participen en la provisión para las necesidades del
pueblo de Dios, y reciban con hospitalidad a los extranjeros.
\footnote{\textbf{12:13} Heb 13,2; 3Jn 1,5-8}

\bibleverse{14} Bendigan a quienes los persiguen, bendíganlos y no los
maldigan. \footnote{\textbf{12:14} Mat 5,44; 1Cor 4,12; Hech 7,59}
\bibleverse{15} Alégrense con los que estén alegres; lloren con los que
lloran. \footnote{\textbf{12:15} Sal 35,13-14; 2Cor 11,29}
\bibleverse{16} Piensen los unos en los otros.\footnote{\textbf{12:16}
  O, ``Vivan en armonía unos con otros''.} No se consideren ustedes
mismos más importantes que los demás; vivan humildemente. No sean
arrogantes. \footnote{\textbf{12:16} Rom 15,5; Fil 2,2} \bibleverse{17}
No paguen mal por mal. Asegúrense de demostrar a todos que lo que hacen
es bueno, \footnote{\textbf{12:17} Is 5,21; 1Tes 5,15; Prov 20,22; 2Cor
  8,21} \bibleverse{18} y en cuanto esté de parte de ustedes, vivan en
paz con todos. \footnote{\textbf{12:18} Mar 9,50; Heb 12,14}
\bibleverse{19} Queridos amigos, no procuren la venganza, más bien dejen
que Dios sea quien haga juicio\footnote{\textbf{12:19} Literalmente,
  ``dar lugar a la ira''.} ---tal como señala la Escritura: ``\,`Es a mí
a quien corresponde administrar la justicia, yo pagaré,' dice el
Señor''.\footnote{\textbf{12:19} Deuteronomio 32:35.} \footnote{\textbf{12:19}
  Lev 19,18; Mat 5,38-44} \bibleverse{20} Si quien los odia tiene
hambre, denle de comer; si tiene sed, denle de beber; pues al hacer esto
acumulan carbones ardientes sobre sus cabezas.\footnote{\textbf{12:20}
  Queriendo decir que esto les causará gran vergüenza y remordimiento.
  Ver Proverbios 25:21-22.} \footnote{\textbf{12:20} 2Re 6,22}

\bibleverse{21} No sean vencidos por el mal, sino conquisten el mal con
el bien.

\hypertarget{exhortaciuxf3n-a-obedecer-a-las-autoridades-designadas-por-dios}{%
\subsection{Exhortación a obedecer a las autoridades designadas por
Dios}\label{exhortaciuxf3n-a-obedecer-a-las-autoridades-designadas-por-dios}}

\hypertarget{section-12}{%
\section{13}\label{section-12}}

\bibleverse{1} Todos deben obedecer a las autoridades de gobierno,
porque nadie tiene el poder de gobernar a menos que Dios se lo permita.
Estas autoridades han sido puestas allí por Dios. \footnote{\textbf{13:1}
  Tit 3,1; Juan 19,11; Prov 8,15} \bibleverse{2} Y quien quiera que se
resista a estas autoridades, se opone a lo que Dios ha establecido, y
los que lo hacen recibirán el merecido juicio por esto. \bibleverse{3}
Porque los gobernantes no producen temor a los que hacen el bien, sino a
los que hacen el mal. De modo que si ustedes no quieren vivir temerosos
de las autoridades, entonces hagan lo recto, y tendrán su aceptación.
\bibleverse{4} Los que están en el poder son siervos de Dios, que han
sido puestos allí para el propio bien de ustedes. De modo que si ustedes
hacen mal, deben tener temor, ¡no en vano las autoridades tienen el
poder para castigar! Ellos son siervos de Dios, que castigan a los
infractores. \footnote{\textbf{13:4} 2Cró 19,6-7} \bibleverse{5} Por eso
es importante que ustedes hagan lo que se les dice, no por la amenaza de
castigo, sino por lo que sus propias conciencias les dicen.
\bibleverse{6} Por ello es que ustedes tienen que pagar impuestos,
porque las autoridades son siervos de Dios que se ocupan de estas cosas.

\hypertarget{exhortaciones-al-cumplimiento-integral-de-los-deberes-especialmente-a-la-caridad-como-cumplimiento-de-la-ley}{%
\subsection{Exhortaciones al cumplimiento integral de los deberes,
especialmente a la caridad como cumplimiento de la
ley}\label{exhortaciones-al-cumplimiento-integral-de-los-deberes-especialmente-a-la-caridad-como-cumplimiento-de-la-ley}}

\bibleverse{7} Paguen todo lo que deban: los impuestos a las autoridades
de impuestos; muestren respeto a los que deben recibir respeto, y rindan
honra a los que deban recibir honra.

\bibleverse{8} No le deban nada a nadie, excepto amor unos a otros,
porque los que aman a su prójimo están cumpliendo la ley. \footnote{\textbf{13:8}
  Gal 5,14; 1Tim 1,5} \bibleverse{9} ``No cometan adulterio, no maten,
no roben, no deseen para ustedes las cosas con envidia''\footnote{\textbf{13:9}
  Literalmente, ``codicia''. Éxodo 20:13-17 o Deuteronomio 5:17-21.}
---los demás mandamientos están resumidos en esta declaración: ``Ama a
tu prójimo como a ti mismo''.\footnote{\textbf{13:9} Citando Levítico
  19:18.} \bibleverse{10} El amor no hace daño a nadie,\footnote{\textbf{13:10}
  O, ``no lastima a nadie''.} y de esta manera el amor cumple la ley.

\hypertarget{el-fin-cercano-del-mundo-advierte-caminar-en-luz-y-santificar-la-vida-personal}{%
\subsection{El fin cercano del mundo advierte caminar en luz y
santificar la vida
personal}\label{el-fin-cercano-del-mundo-advierte-caminar-en-luz-y-santificar-la-vida-personal}}

\bibleverse{11} Ustedes deben hacer esto porque pueden darse cuenta de
cuán urgente es este tiempo, que ha llegado la hora de que despierten de
su sueño. Porque la salvación está más cerca de nosotros ahora que
cuando por primera vez creímos en Dios. \footnote{\textbf{13:11} Efes
  5,14; 1Tes 5,6-8} \bibleverse{12} ¡La noche casi termina, el día casi
está aquí! Así que despojémonos de nuestras malas obras y vistámonos de
la armadura de la luz. \footnote{\textbf{13:12} 1Jn 2,8; Efes 5,11}
\bibleverse{13} Tengamos una conducta apropiada, demostrando que somos
personas que vivimos en la luz. No debemos perder el tiempo yendo a
fiestas y embriagándonos, o teniendo amoríos y actuando de manera
inmoral, o metiéndonos en peleas y andar con celos. \footnote{\textbf{13:13}
  Luc 21,34; Efes 5,18} \bibleverse{14} Por el contrario, vístanse del
Señor Jesucristo y olvídense de seguir sus deseos
pecaminosos.\footnote{\textbf{13:14} Gal 3,27; 1Cor 9,27; Col 2,23}

\hypertarget{juicio-sobre-el-tema-que-conmueve-a-la-comunidad-y-advierte-contra-la-condena-sin-amor-del-modo-de-vida-externo-del-pruxf3jimo}{%
\subsection{Juicio sobre el tema que conmueve a la comunidad y advierte
contra la condena sin amor del modo de vida externo del
prójimo}\label{juicio-sobre-el-tema-que-conmueve-a-la-comunidad-y-advierte-contra-la-condena-sin-amor-del-modo-de-vida-externo-del-pruxf3jimo}}

\hypertarget{section-13}{%
\section{14}\label{section-13}}

\bibleverse{1} Acepten a los que todavía están luchando por creer en
Dios, y no tengan discusiones por causa de opiniones personales.
\footnote{\textbf{14:1} Rom 15,1; 1Cor 8,9} \bibleverse{2} Es posible
que una persona crea que puede comer de todo, mientras otra, con una fe
más débil, solo come vegetales.\footnote{\textbf{14:2} 14:1, 2. Esto no
  guarda relación alguna con el tema del vegetarianismo o la dieta, sino
  con la comida ofrecida a ídolos. (Tal como también sucede en 1
  Corintios 8).} \footnote{\textbf{14:2} Gén 1,29; Gén 9,3}
\bibleverse{3} Los que comen de todo no deben menospreciar a los que no,
y los que no comen de todo no deben criticar a los que sí lo hacen,
porque Dios ha aceptado a ambos. \footnote{\textbf{14:3} Col 2,16}
\bibleverse{4} ¿Qué derecho tienes tú para juzgar al siervo de otro? Es
su propio amo quien decide si está haciendo bien o mal. Con ayuda de
Dios, ellos podrán discernir lo correcto. \footnote{\textbf{14:4} Mat
  7,1; Sant 4,11; Sant 1,4-12}

\bibleverse{5} Hay quienes consideran que algunos días son más
importantes que otros, mientras que otros piensan que todos los días son
iguales. Todos deben estar plenamente convencidos en su propia mente.
\footnote{\textbf{14:5} Gal 4,10} \bibleverse{6} Los que respetan un día
especial, lo hacen para el Señor; y los que comen sin
preocupaciones,\footnote{\textbf{14:6} Comer o no comer se refiere a si
  era correcto o no comer alimentos que habían sido llevados como
  ofrenda a ídolos paganos.} lo hacen también, puesto que dan las
gracias a Dios; mientras tanto, los que evitan comer ciertas cosas,
también lo hacen para el Señor, y del mismo modo, dan gracias a Dios.
\bibleverse{7} Ninguno de nosotros vive para sí mismo, o muere para sí
mismo. \bibleverse{8} Si vivimos, vivimos para el Señor, o si morimos,
morimos para el Señor. De modo que ya sea que vivamos o muramos,
pertenecemos al Señor. \bibleverse{9} Esta fue la razón por la que
Cristo murió y volvió a la vida, para así ser Señor tanto de los muertos
como de los vivos.

\bibleverse{10} ¿Por qué, entonces, criticas a tu hermano creyente? Pues
todos estaremos en pie delante del trono en el juicio de Dios.
\footnote{\textbf{14:10} Mat 25,31-32; Hech 17,31; 2Cor 5,10}
\bibleverse{11} Pues las Escrituras dicen: ``\,`Tan cierto como yo estoy
vivo,' dice el Señor, `toda rodilla se doblará delante de mí, y toda
lengua declarará que yo soy Dios'\,''.\footnote{\textbf{14:11} Citando
  Isaías 45:23.} \footnote{\textbf{14:11} Fil 2,10-11}

\bibleverse{12} Así que cada uno de nosotros tendrá que rendir cuenta de
sí mismo a Dios. \footnote{\textbf{14:12} Gal 6,5}

\hypertarget{exhortaciuxf3n-a-los-que-tienen-una-fe-fuerte-a-no-ofender-a-los-que-tienen-una-fe-duxe9bil-y-a-esforzarse-por-tener-una-conciencia-segura-en-todo-lo-que-hacen}{%
\subsection{Exhortación a los que tienen una fe fuerte a no ofender a
los que tienen una fe débil y a esforzarse por tener una conciencia
segura en todo lo que
hacen}\label{exhortaciuxf3n-a-los-que-tienen-una-fe-fuerte-a-no-ofender-a-los-que-tienen-una-fe-duxe9bil-y-a-esforzarse-por-tener-una-conciencia-segura-en-todo-lo-que-hacen}}

\bibleverse{13} Por lo tanto, no nos juzguemos más unos a otros. Por el
contrario, decidamos no poner obstáculos en el camino de nuestros
hermanos creyentes, ni hacerlos caer. \footnote{\textbf{14:13} 1Cor
  10,33} \bibleverse{14} Yo estoy seguro---persuadido por el Señor
Jesús---que nada es, en sí mismo, ceremonialmente impuro. Pero si alguno
considera que es impuro, para él es impuro. \footnote{\textbf{14:14} Mat
  15,11; Hech 10,15; Tit 1,15} \bibleverse{15} Si tu hermano creyente se
siente ofendido por ti, en términos de comidas, entonces ya tu conducta
no es de amor. No destruyas a alguien por quien Cristo murió por la
comida que eliges comer. \footnote{\textbf{14:15} 1Cor 8,11-13}
\bibleverse{16} No permitas que las cosas buenas que haces sean
malinterpretadas--- \bibleverse{17} porque el reino de Dios no tiene que
ver con la comida o la bebida, sino con vivir bien, tener paz y gozo en
el Espíritu Santo. \footnote{\textbf{14:17} 1Cor 8,8; Heb 13,9}
\bibleverse{18} Todo el que sirve a Cristo de este modo, agrada a Dios,
y es apreciado por los demás. \bibleverse{19} Así que sigamos el camino
de la paz, y busquemos formas de animarnos unos a otros. \bibleverse{20}
No destruyas la obra de Dios con discusiones sobre la comida. Todo es
limpio, pero estaría mal comer y ofender a otros. \bibleverse{21} Es
mejor no comer carne, o no beber vino ni nada que pueda ser causa del
tropiezo de tu hermano creyente.

\bibleverse{22} Lo que tú crees, de manera personal, es algo entre tú y
Dios. ¡Cuán felices son los que no se condenan a sí mismos por hacer lo
que creen que es correcto! \footnote{\textbf{14:22} Rom 14,2; 1Cor
  10,25-27}

\bibleverse{23} Pero si tienes dudas en cuanto a si está bien o mal
comer algo, entonces no debes hacerlo, porque no estás convencido de que
es correcto. Todo lo que no está basado en la convicción\footnote{\textbf{14:23}
  O, ``fe''.} es pecado.\footnote{\textbf{14:23} O, ``Pecado es hacer
  algo que no crees que es correcto''.}

\hypertarget{exhortaciuxf3n-a-ser-pacientes-con-los-duxe9biles-y-a-la-unidad-de-los-cristianos-basada-en-el-ejemplo-de-cristo}{%
\subsection{Exhortación a ser pacientes con los débiles y a la unidad de
los cristianos basada en el ejemplo de
Cristo}\label{exhortaciuxf3n-a-ser-pacientes-con-los-duxe9biles-y-a-la-unidad-de-los-cristianos-basada-en-el-ejemplo-de-cristo}}

\hypertarget{section-14}{%
\section{15}\label{section-14}}

\bibleverse{1} Los que de nosotros son espiritualmente fuertes deben
apoyar a los que son espiritualmente débiles. No debemos simplemente
complacernos a nosotros mismos. \bibleverse{2} Todos debemos animar a
otros a hacer lo recto, edificándolos. \footnote{\textbf{15:2} 1Cor
  9,19; 1Cor 10,24; 1Cor 10,33} \bibleverse{3} Así como Cristo no vivió
para complacerse a sí mismo, sino que, como la Escritura dice de él:
``Las ofensas de los que te insultaban han caído sobre mí''.\footnote{\textbf{15:3}
  Citando Salmos 69:9.} \bibleverse{4} Estas Escrituras fueron escritas
en el pasado para ayudarnos a entender, y para animarnos a fin de que
pudiéramos esperar pacientemente en esperanza. \bibleverse{5} ¡Que Dios,
quien nos da paciencia y ánimo, los ayude a estar en armonía unos con
otros como seguidores de Jesucristo, \footnote{\textbf{15:5} Fil 2,2}
\bibleverse{6} a fin de que puedan, con una sola mente y una sola voz,
glorificar juntos a Dios, el Padre de nuestro Señor Jesucristo!

\hypertarget{un-recordatorio-para-que-ambas-partes-de-la-comunidad-estuxe9n-unidas-y-tengan-una-fe-gozosa}{%
\subsection{Un recordatorio para que ambas partes de la comunidad estén
unidas y tengan una fe
gozosa}\label{un-recordatorio-para-que-ambas-partes-de-la-comunidad-estuxe9n-unidas-y-tengan-una-fe-gozosa}}

\bibleverse{7} Así que acéptense unos a otros, así como Cristo los
aceptó a ustedes, y denle la gloria a Dios. \bibleverse{8} Siempre digo
que Cristo vino como siervo a los judíos\footnote{\textbf{15:8}
  Literalmente, ``de la circuncisión''.} para mostrar que Dios dice la
verdad, manteniendo las promesas hechas a sus antepasados.
\bibleverse{9} También vino para que los extranjeros pudieran alabar a
Dios por su misericordia, como dice la Escritura, ``Por lo tanto te
alabaré entre los extranjeros; cantaré alabanzas a tu
nombre''.\footnote{\textbf{15:9} Citando Salmos 18:49.}

\bibleverse{10} Y también dice: ``¡Extranjeros, celebren con este
pueblo!''\footnote{\textbf{15:10} Citando Deuteronomio 32:43.}

\bibleverse{11} Y una vez más, dice: ``Todos ustedes, extranjeros,
alaben al Señor, que todos los pueblos le alaben''.\footnote{\textbf{15:11}
  Citando Salmos 117:1.}

\bibleverse{12} Y otra vez, Isaías dice: ``El descendiente de Isaí
vendrá a gobernar las naciones, y los extranjeros pondrán su esperanza
en él''.\footnote{\textbf{15:12} Citando Isaías 11:10. ``Descendiente de
  Isaí''. Se refiere a Isaí, el padre del Rey David, quien inició el
  linaje real.} \footnote{\textbf{15:12} Apoc 5,5}

\bibleverse{13} ¡Que el Dios de esperanza los llene por completo de todo
gozo y paz, como sus creyentes, a fin de que puedan rebosar de esperanza
por el poder del Espíritu Santo!

\hypertarget{revisiuxf3n-justificativa-del-apuxf3stol-de-la-carta-y-referencia-a-su-oficio-apostuxf3lico-para-los-gentiles}{%
\subsection{Revisión justificativa del apóstol de la carta y referencia
a su oficio apostólico para los
gentiles}\label{revisiuxf3n-justificativa-del-apuxf3stol-de-la-carta-y-referencia-a-su-oficio-apostuxf3lico-para-los-gentiles}}

\bibleverse{14} Estoy convencido de que ustedes, mis hermanos y
hermanas, están llenos de bondad, y que están llenos de todo tipo de
conocimiento, de modo que están bien capacitados para enseñarse unos a
otros. \bibleverse{15} He sido muy directo en la forma como les he
escrito sobre algunas de estas cosas, pero es solo para recordarles.
Porque Dios me dio la gracia \bibleverse{16} de ser un ministro de
Jesucristo para los extranjeros, como un sacerdote que predica la buena
noticia de Dios, a fin de que puedan convertirse en una ofrenda
agradable, santificada por el Espíritu Santo. \footnote{\textbf{15:16}
  Rom 11,13} \bibleverse{17} Así que, aunque tenga algo de qué jactarme
por mi servicio a Dios, \bibleverse{18} (no me atrevería a hablar de
ninguna de estas cosas, excepto cuando Cristo mismo lo ha hecho a través
de mi), he conducido a los extranjeros a la obediencia a través de mi
enseñanza y ejemplo, \bibleverse{19} a través del poder de señales y
milagros realizados por el poder del Espíritu Santo. Desde Jerusalén
hasta Ilírico, por todos lados he compartido enteramente la buena
noticia de Cristo. \footnote{\textbf{15:19} Mar 16,17; 2Cor 12,12}
\bibleverse{20} De hecho, con mucho agrado compartí el evangelio en
lugares donde no habían escuchado el nombre de Cristo, para no construir
sobre lo que otros habían hecho. \footnote{\textbf{15:20} 2Cor 10,15-16}
\bibleverse{21} Como dice la Escritura: ``Los que no han oído de la
buena noticia la descubrirán, y los que no han oído
entenderán''.\footnote{\textbf{15:21} Citando Isaías 52:15.}

\hypertarget{anuncio-de-los-pruxf3ximos-planes-de-viaje-del-apuxf3stol}{%
\subsection{Anuncio de los próximos planes de viaje del
apóstol}\label{anuncio-de-los-pruxf3ximos-planes-de-viaje-del-apuxf3stol}}

\bibleverse{22} Por ello muchas veces me fue imposible venir a verlos.
\footnote{\textbf{15:22} Rom 1,13} \bibleverse{23} Pero ahora, como no
hay más lugar aquí donde trabajar, y como he deseado visitarlos desde
hace años, \footnote{\textbf{15:23} Rom 1,10-11} \bibleverse{24} espero
verlos cuando vaya de camino a España. Quizás puedan brindarme ayuda
para el viaje, después de pasar juntos por algún tiempo. \bibleverse{25}
Ahora voy de camino a Jerusalén para ayudar a los creyentes que están
allá, \footnote{\textbf{15:25} Hech 18,21; Hech 19,21; Hech 20,22; Hech
  24,17} \bibleverse{26} porque los creyentes en Macedonia y Acaya
pensaron que sería bueno enviar una ayuda a los pobres que están entre
los creyentes de Jerusalén. \footnote{\textbf{15:26} 1Cor 16,1; 2Cor
  8,1-4; 2Cor 8,9} \bibleverse{27} Estuvieron felices de ayudarlos
porque tienen esta deuda con ellos\footnote{\textbf{15:27} Queriendo
  decir que los extranjeros están en deuda con los judíos por compartir
  la buena noticia de Dios. Este ejemplo en particular se aplica de
  manera específica a los creyentes en Jerusalén, es decir, que los
  extranjeros están felices de enviarles un regalo para ayudarlos.} .
Ahora que los extranjeros son partícipes de sus beneficios espirituales,
están en deuda con los creyentes judíos\footnote{\textbf{15:27}
  Implícito.} para ayudarlos con cosas materiales. \footnote{\textbf{15:27}
  1Cor 9,11; Gal 6,6} \bibleverse{28} De modo que cuando haya terminado
con esto, y les haya entregado de manera segura esta contribución, los
visitaré a ustedes de camino a España. \bibleverse{29} Sé que cuando
venga, Cristo nos dará su plena bendición.

\hypertarget{la-amonestaciuxf3n-del-apuxf3stol-a-la-iglesia-de-que-interceda-por-uxe9l}{%
\subsection{La amonestación del apóstol a la iglesia de que interceda
por
él}\label{la-amonestaciuxf3n-del-apuxf3stol-a-la-iglesia-de-que-interceda-por-uxe9l}}

\bibleverse{30} Deseo animarlos, mis hermanos y hermanas, mediante
nuestro Señor Jesucristo y mediante el amor del Espíritu, a que se unan
y oren mucho por mí. \bibleverse{31} Oren para que pueda estar a salvo
de los no creyentes de Judea. Oren para que mi labor en Jerusalén sea
bien recibida por los creyentes de allí. \footnote{\textbf{15:31} 1Tes
  2,15}

\bibleverse{32} Oren para que pueda regresar a ustedes con alegría,
conforme a la voluntad de Dios, para que podamos disfrutar juntos, en
compañía. \bibleverse{33} Que el Dios de paz esté con todos ustedes.
Amén.

\hypertarget{recomendaciuxf3n-de-phuxf6be-portador-de-la-carta-saludos-del-apuxf3stol-a-los-hermanos-en-roma}{%
\subsection{Recomendación de Phöbe, portador de la carta; Saludos del
Apóstol a los hermanos en
Roma}\label{recomendaciuxf3n-de-phuxf6be-portador-de-la-carta-saludos-del-apuxf3stol-a-los-hermanos-en-roma}}

\hypertarget{section-15}{%
\section{16}\label{section-15}}

\bibleverse{1} Les encomiendo a nuestra hermana Febe, quien es diaconisa
en la iglesia de Cencrea. \bibleverse{2} Por favor, recíbanla en el
Señor, como deben hacerlo los creyentes, y ayúdenla en todo lo que
necesite, porque ha sido de gran ayuda para mucha gente, incluyéndome a
mí.

\bibleverse{3} Envíen mi saludo a Prisca\footnote{\textbf{16:3} Llamada
  Priscila en Hechos 18:2. También 1 Corintios 16:19.} y Aquila, mis
compañeros de trabajo en Cristo Jesús, \bibleverse{4} quienes
arriesgaron su vida por mí. No solo yo estoy agradecido con ellos, sino
con todas las iglesias de los extranjeros también.\footnote{\textbf{16:4}
  Refiriéndose a las Iglesias no judías.} \bibleverse{5} Por favor,
también salúdenme a la iglesia que se reúne en su hogar. Den mis mejores
deseos a mi buen amigo Epeneto, la primera persona en seguir a Cristo en
la provincia de Asia. \bibleverse{6} Envíen mis saludos a María, que ha
trabajado mucho por ustedes, \bibleverse{7} y también a Andrónico y a
Junías, judíos como yo, y compañeros en la cárcel. Ellos son muy bien
conocidos entre los apóstoles y se convirtieron en seguidores de Cristo
antes que yo. \bibleverse{8} Envíen mis mejores deseos a Amplias, mi
buen amigo en el Señor; \bibleverse{9} a Urbano, nuestro compañero de
trabajo en Cristo; y a mi querido amigo Estaquis. \bibleverse{10}
Saludos a Apeles, un hombre fiel en Cristo. Saludos a la familia de
Aristóbulo, \bibleverse{11} a mi conciudadano Herodión, y a los de la
familia de Narciso, que pertenecen al Señor. \bibleverse{12} Mis mejores
deseos a Trifaena y Trifosa, trabajadores diligentes del Señor, y a mi
amiga Pérsida, que ha trabajado mucho en el Señor. \bibleverse{13} Den
mis saludos a Rufo, un trabajador excepcional,\footnote{\textbf{16:13}
  O, ``uno del pueblo especial de Dios''.} y a su madre, a quien
considero como mi madre también. \footnote{\textbf{16:13} Mar 15,21}
\bibleverse{14} Saludos a Asíncrito, a Flegontes, a Hermes, a Patrobas,
a Hermas, y a los creyentes que están con ellos. \bibleverse{15} Mis
mejores deseos a Filólogo y Julia, a Nereo y a su hermana, a Olimpas y a
todos los creyentes que están con ellos. \bibleverse{16} Salúdense unos
a otros con afecto. Todas las iglesias de Cristo les envían saludos.

\hypertarget{advertencia-a-los-engauxf1adores-que-causan-divisiones-y-errores-en-la-iglesia}{%
\subsection{Advertencia a los engañadores que causan divisiones y
errores en la
iglesia}\label{advertencia-a-los-engauxf1adores-que-causan-divisiones-y-errores-en-la-iglesia}}

\bibleverse{17} Ahora les ruego, mis hermanos creyentes: cuídense de los
que causan discusiones y confunden a las personas de la enseñanza que
han aprendido. ¡Aléjense de ellos! \footnote{\textbf{16:17} Mat 7,15;
  Tit 3,10; 2Tes 3,6} \bibleverse{18} Estas personas no sirven a Cristo
nuestro Señor sino a sus propios apetitos, y con su forma de hablar
lisonjera y palabras agradables engañan las mentes de las personas
desprevenidas. \footnote{\textbf{16:18} Fil 3,19; Col 2,4}
\bibleverse{19} Todos saben cuán fieles son ustedes y eso me llena de
alegría. Sin embargo, quiero que sean sabios en cuanto a lo que es
bueno, e inocentes de lo malo. \footnote{\textbf{16:19} Rom 1,8; 1Cor
  14,20} \bibleverse{20} El Dios de paz pronto quebrantará el poder de
Satanás y lo someterá a ustedes. Que la gracia de nuestro Señor
Jesucristo esté con ustedes.

\hypertarget{saludos-de-los-amigos-de-pablo-a-roma-y-finalmente-alabanza-a-dios}{%
\subsection{Saludos de los amigos de Pablo a Roma y finalmente alabanza
a
Dios}\label{saludos-de-los-amigos-de-pablo-a-roma-y-finalmente-alabanza-a-dios}}

\bibleverse{21} Timoteo, mi compañero de trabajo, envía sus saludos, así
como Lucio, Jasón y Sosípater, quienes también son judíos.
\bibleverse{22} Tercio---quien escribe esta carta---también los saluda
en el Señor. \bibleverse{23} Gayo, quien me dio hospedaje, y toda la
iglesia de aquí también los saludan. Erasto, el tesorero de la ciudad,
envía sus mejores deseos a ustedes, así como nuestro hermano Cuarto.
\bibleverse{24} \footnote{\textbf{16:24} Los primeros manuscritos no
  incluyen el versículo 24.} \bibleverse{25} Ahora, a Aquél que puede
fortalecerlos, mediante la buena noticia que yo comparto y el mensaje de
Jesucristo, Conforme al misterio de verdad\footnote{\textbf{16:25}
  Literalmente, ``misterio'', un término que se refiere a una verdad
  secreta o a un plan secreto que es conocido solo por los creyentes
  religiosos. Ver también, versículo 26.} que ha sido revelado, El
misterio de verdad, oculto por la eternidad, \bibleverse{26} y ahora
visible. A través de los escritos de los profetas, y siguiendo el
mandato del Dios eterno, El misterio de la verdad es dado a conocer a
todos, en todos lados a fin de que puedan creer y obedecerle;
\bibleverse{27} Al único Dios sabio, A través de Jesucristo. A él sea la
gloria para siempre. Amén.\footnote{\textbf{16:27} Estos últimos
  versículos parecen ser un poema o canción, por ello están
  estructurados de esta manera.}
