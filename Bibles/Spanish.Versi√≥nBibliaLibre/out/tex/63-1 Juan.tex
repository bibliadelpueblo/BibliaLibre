\hypertarget{contenido-fiabilidad-y-finalidad-del-mensaje-apostuxf3lico-de-la-palabra-de-vida}{%
\subsection{Contenido, fiabilidad y finalidad del mensaje apostólico de
la palabra de
vida}\label{contenido-fiabilidad-y-finalidad-del-mensaje-apostuxf3lico-de-la-palabra-de-vida}}

\hypertarget{section}{%
\section{1}\label{section}}

\bibleverse{1} Esta carta trata sobre la Palabra de vida que existía
desde el principio, que hemos escuchado, que hemos visto con nuestros
propios ojos y le hemos contemplado, y que hemos tocado con nuestras
manos.\footnote{\textbf{1:1} La estructura griega de la oración se ha
  ajustado para darle sentido.} \footnote{\textbf{1:1} Juan 1,1; Juan
  1,4; Juan 1,14} \bibleverse{2} Esta Vida nos fue revelada. La vimos y
damos testimonio de ella. Estamos hablándoles de Aquél que es la Vida
Eterna, que estaba con el Padre, y que nos fue revelado. \bibleverse{3}
Los que hemos visto y oído eso mismo les contamos, para que también
puedan participar de esta amistad\footnote{\textbf{1:3} Literalmente,
  ``compañerismo''.} junto a nosotros. Esta amistad con el Padre y su
Hijo Jesucristo. \bibleverse{4} Escribimos para decirles esto, a fin de
que nuestra felicidad sea completa.

\hypertarget{el-caminar-en-la-luz-la-verdad-versus-el-caminar-en-la-oscuridad-la-mentira-conocimiento-y-confesiuxf3n-del-pecado}{%
\subsection{El caminar en la luz (la verdad) versus el caminar en la
oscuridad (la mentira); Conocimiento y confesión del
pecado}\label{el-caminar-en-la-luz-la-verdad-versus-el-caminar-en-la-oscuridad-la-mentira-conocimiento-y-confesiuxf3n-del-pecado}}

\bibleverse{5} Este es el mensaje que recibimos de él y que nosotros les
declaramos a ustedes: Dios es luz, y no hay ningún vestigio de oscuridad
en él.\footnote{\textbf{1:5} En griego hay una doble negación para hacer
  énfasis, literalmente, ``la oscuridad en él no existe, de ninguna
  manera''.} \footnote{\textbf{1:5} Sant 1,17} \bibleverse{6} Si decimos
ser sus amigos, y seguimos viviendo\footnote{\textbf{1:6} Literalmente,
  ``caminando''. Ver también 1:7.} en oscuridad, estamos mintiendo, y no
vivimos en la verdad. \footnote{\textbf{1:6} 1Jn 2,4} \bibleverse{7}
Pero si vivimos en la luz, así como él está en la luz, entonces somos
amigos unos con otros, y la sangre de Jesús, su Hijo, nos limpia de todo
pecado. \footnote{\textbf{1:7} Heb 9,14; Apoc 1,5} \bibleverse{8} Si
decimos que no pecamos, nos engañamos a nosotros mismos, y la verdad no
está en nosotros. \bibleverse{9} Pero si confesamos nuestros pecados, él
es fiel y justo para perdonar nuestros pecados y limpiarnos de todo lo
malo que hay dentro de nosotros. \bibleverse{10} Si decimos que no hemos
pecado, estamos llamando a Dios mentiroso, y su palabra no está en
nosotros.\footnote{\textbf{1:10} Rom 3,10-18}

\hypertarget{el-fruto-del-conocimiento-de-dios-se-manifiesta-al-andar-seguxfan-los-mandamientos-divinos}{%
\subsection{El fruto del conocimiento de Dios se manifiesta al andar
según los mandamientos
divinos}\label{el-fruto-del-conocimiento-de-dios-se-manifiesta-al-andar-seguxfan-los-mandamientos-divinos}}

\hypertarget{section-1}{%
\section{2}\label{section-1}}

\bibleverse{1} Queridos hijos míos, les escribo esto para que no pequen.
Pero si alguno peca, tenemos a alguien que nos defiende ante el Padre, a
Jesucristo, que es verdaderamente justo. \footnote{\textbf{2:1} Rom
  8,34; Heb 7,25} \bibleverse{2} Por él son perdonados nuestros pecados,
y no solo los nuestros, sino los de todo el mundo. \footnote{\textbf{2:2}
  1Jn 4,10; Col 1,20; Juan 11,51-52} \bibleverse{3} Podemos estar
seguros de que lo conocemos si seguimos sus mandamientos. \bibleverse{4}
Todo el que dice: ``Yo conozco a Dios'', pero no hace su voluntad, es
mentiroso, y no tiene la verdad. \bibleverse{5} Pero los que siguen la
palabra de Dios permiten que su amor llene sus corazones por completo.
Así es como sabemos que vivimos en él. \bibleverse{6} Todo el que dice
vivir en él, debe vivir como Jesús vivió. \footnote{\textbf{2:6} Juan
  13,15; 1Pe 2,21-23}

\hypertarget{el-nuevo-mandamiento-del-amor-fraternal-por-los-hijos-de-la-luz}{%
\subsection{El nuevo mandamiento del amor fraternal por los hijos de la
luz}\label{el-nuevo-mandamiento-del-amor-fraternal-por-los-hijos-de-la-luz}}

\bibleverse{7} Amigos, no les escribo para darles un nuevo mandamiento,
sino un mandamiento antiguo que ya teníamos desde el principio. Este
mandamiento antiguo ya lo han escuchado. \footnote{\textbf{2:7} Juan
  13,34; 2Jn 1,-1} \bibleverse{8} Pero en cierto sentido les estoy dando
un nuevo mandamiento. Su verdad se revela en Jesús y en ustedes, pues
viene el fin de la oscuridad y la luz verdadera ya está brillando.
\footnote{\textbf{2:8} Juan 8,12; Rom 13,12} \bibleverse{9} Los que
dicen que viven en la luz pero aborrecen a un hermano
cristiano\footnote{\textbf{2:9} Literalmente, ``hermano''.} todavía
tienen tinieblas dentro de sí. \footnote{\textbf{2:9} 1Jn 4,20}
\bibleverse{10} Los que aman a sus hermanos cristianos viven en la luz,
y no hacen pecar a otros.\footnote{\textbf{2:10} Literalmente, ``no hay
  engaño''. En otras palabras, algo que hace tropezar a otros.}
\bibleverse{11} Los que aborrecen a un hermano cristiano están en
oscuridad. Tropiezan en la oscuridad, sin saber hacia dónde van porque
la oscuridad los ha cegado.

\hypertarget{las-diferentes-edades-del-desarrollo-espiritual-los-hijos-de-la-luz-evitan-el-amor-al-mundo}{%
\subsection{Las diferentes edades del desarrollo espiritual; los hijos
de la luz evitan el amor al
mundo}\label{las-diferentes-edades-del-desarrollo-espiritual-los-hijos-de-la-luz-evitan-el-amor-al-mundo}}

\bibleverse{12} Queridos amigos, les escribo a ustedes,
hijos,\footnote{\textbf{2:12} Juan identifica tres grupos: hijos, padres
  y jóvenes. Probablemente se refiere a las distintas edades de la vida
  cristiana, más que a grupos literales.} porque sus pecados han sido
perdonados por el nombre de Jesús.

\bibleverse{13} Les escribo a ustedes, padres, porque ustedes lo conocen
a él, que ha existido desde el principio. Les escribo a ustedes,
jóvenes, porque han vencido el mal. \footnote{\textbf{2:13} Juan 1,1}

\bibleverse{14} Les escribo a ustedes, pequeñitos, porque ustedes
conocen al Padre. Les escribo a ustedes, Padres, porque conocen al que
ha existido desde el principio. Les escribo a ustedes, jóvenes, porque
son fuertes. Porque la palabra de Dios vive en ustedes, y han vencido al
maligno. \footnote{\textbf{2:14} Efes 6,10}

\bibleverse{15} No amen al mundo, ni anhelen las cosas que hay en él. Si
aman al mundo, no tendrán el amor del Padre en ustedes. \footnote{\textbf{2:15}
  Sant 4,4} \bibleverse{16} Porque todas las cosas de este mundo,
nuestros deseos pecaminosos, nuestro deseo por todo lo que vemos,
nuestra jactancia por lo que hemos logrado en la vida, ninguna de esas
cosas viene del Padre, sino del mundo. \bibleverse{17} El mundo y sus
malos deseos acabarán, pero los que hacen la voluntad de Dios vivirán
para siempre.

\hypertarget{instrucciuxf3n-sobre-falsos-maestros-advertencia-de-los-anticristos-recordatorio-para-aferrarse-a-la-enseuxf1anza-correcta}{%
\subsection{Instrucción sobre falsos maestros; Advertencia de los
anticristos; Recordatorio para aferrarse a la enseñanza
correcta}\label{instrucciuxf3n-sobre-falsos-maestros-advertencia-de-los-anticristos-recordatorio-para-aferrarse-a-la-enseuxf1anza-correcta}}

\bibleverse{18} Queridos amigos, esta es la última hora. Como han
escuchado, el anticristo viene. Y ya han venido muchos anticristos. Así
es como sabemos que esta es la última hora. \bibleverse{19} Ellos se
fueron, pero no eran parte de nosotros, porque si así hubiera sido,
habrían permanecido aquí. Pero cuando se fueron demostraron que ninguno
de ellos hacía parte de nosotros. \footnote{\textbf{2:19} Hech 20,30;
  1Cor 11,19} \bibleverse{20} Pero ustedes han sido ungidos\footnote{\textbf{2:20}
  Ungir es el acto de derramar un líquido (a menudo aceite) sobre la
  cabeza de alguien para indicar que esa persona tiene una bendición
  especial y está apartada para un rol particular (como el reinado en el
  Antiguo Testamento). Aquí el ungimiento se refiere a la bendición del
  Espíritu Santo, quien en palabras de Jesús, nos conduce a toda verdad.}
con la bendición del Espíritu Santo, y todos ustedes saben lo que es
verdad. \bibleverse{21} No les escribo porque no conozcan la verdad,
sino precisamente porque la conocen, y porque no hay engaño en ella.
\bibleverse{22} ¿Quién es el mentiroso? Todo aquél que niega que Jesús
es el Cristo.\footnote{\textbf{2:22} Quiere decir Mesías. (Cristo, en
  griego). Ambos se refieren al que es ungido.} El anticristo es todo
aquél que niega al Padre y al Hijo. \bibleverse{23} Todo aquél que niega
al Hijo, tampoco tiene al Padre; y todo el que reconoce al Hijo, tiene
al Padre también.

\bibleverse{24} En cuanto a ustedes, asegúrense de que lo que oyeron
desde el principio siga vivo en ustedes. Si lo que oyeron desde el
principio vive en ustedes, también vivirán en el Hijo y en el Padre.
\bibleverse{25} La vida eterna. ¡Eso es lo que nos ha prometido!

\bibleverse{26} Escribo esto para advertirles contra las cosas que
quieren descarriarlos. \bibleverse{27} Pero el ungimiento que recibieron
de él por medio del Espíritu\footnote{\textbf{2:27} El espíritu, según
  el versículo 20.} vive en ustedes, y no necesitan que nadie los
enseñe. El ungimiento del Espíritu les enseña todas las cosas. Esa es la
verdad. No es una mentira. Así que vivan en Cristo, como se les ha
enseñado. \footnote{\textbf{2:27} Juan 16,13; 2Cor 1,21-22; Jer 31,34}

\hypertarget{permanecer-en-cristo-y-ejercer-la-justicia-de-fe-da-gozo-en-el-juicio}{%
\subsection{Permanecer en Cristo y ejercer la justicia de fe da gozo en
el
juicio}\label{permanecer-en-cristo-y-ejercer-la-justicia-de-fe-da-gozo-en-el-juicio}}

\bibleverse{28} Ahora, mis queridos amigos, sigan viviendo en Cristo,
para que cuando aparezca, podamos estar seguros y no tengamos vergüenza
delante él en su venida. \footnote{\textbf{2:28} 1Jn 4,17}
\bibleverse{29} Si ustedes saben que él es bueno y justo,\footnote{\textbf{2:29}
  Literalmente, ``justo''. Sin embargo, esta palabra a menudo solo se
  usa en un sentido religioso hoy y no tiene mucho significado en el
  hablar cotidiano.} entonces también deben saber que todo el que hace
lo justo ha nacido de Dios.\footnote{\textbf{2:29} 1Jn 3,7; 1Jn 1,3-10}

\hypertarget{felicidad-y-esperanza-de-gloria-para-los-hijos-de-dios}{%
\subsection{Felicidad y esperanza de gloria para los hijos de
Dios}\label{felicidad-y-esperanza-de-gloria-para-los-hijos-de-dios}}

\hypertarget{section-2}{%
\section{3}\label{section-2}}

\bibleverse{1} ¡Miren el amor que tiene el Padre para con nosotros! Por
eso podemos ser llamados hijos de Dios, ¡porque eso es lo que somos! La
razón por la que el mundo no nos reconoce como hijos de Dios es porque
no lo reconocen a él. \footnote{\textbf{3:1} Juan 1,12; Juan 16,3}
\bibleverse{2} Amigos míos, ya somos hijos de Dios, pero lo que
llegaremos a ser no se ha revelado todavía. Pero sabemos que cuando él
aparezca seremos como él, porque lo veremos como él es realmente.
\footnote{\textbf{3:2} Rom 8,17; Col 3,4; Fil 3,21} \bibleverse{3} Todos
los que tienen esta esperanza en él, asegúrense de ser puros, como él lo
es.

\hypertarget{los-nacidos-de-dios-estuxe1n-obligados-a-evitar-el-pecado-y-practicar-la-justicia-especialmente-el-amor-fraternal}{%
\subsection{Los nacidos de Dios están obligados a evitar el pecado y
practicar la justicia, especialmente el amor
fraternal}\label{los-nacidos-de-dios-estuxe1n-obligados-a-evitar-el-pecado-y-practicar-la-justicia-especialmente-el-amor-fraternal}}

\bibleverse{4} Todos los que pecan son violadores de la ley de Dios.
\bibleverse{5} Pero desde luego ustedes saben que Jesús vino para
eliminar los pecados, y en él no hay pecado. \bibleverse{6} Todos los
que viven en él, ya no pecan más; todos los que siguen pecando es porque
no lo han visto y no lo han conocido. \footnote{\textbf{3:6} Rom 6,11;
  Rom 6,14}

\bibleverse{7} Queridos amigos, no dejen que nadie los engañe: los que
hacen justicia son justos, así como Jesús es justo. \footnote{\textbf{3:7}
  1Jn 2,29} \bibleverse{8} Los que pecan son del diablo, porque el
diablo ha estado pecando desde el principio. Por eso vino el Hijo de
Dios, para destruir lo que el diablo ha hecho. \footnote{\textbf{3:8}
  Juan 8,44} \bibleverse{9} Y todos los que son nacidos de Dios ya no
pecan más, porque la naturaleza de Dios\footnote{\textbf{3:9}
  Literalmente, ``su semilla''.} habita en ellos. Y no pueden seguir
pecando porque han nacido de Dios. \footnote{\textbf{3:9} 1Jn 5,18}
\bibleverse{10} Así es como podemos distinguir a los hijos de Dios y los
hijos del diablo: todos aquellos que no obran con justicia, no
pertenecen a Dios, ni aquellos que no aman a sus hermanos cristianos.
\bibleverse{11} El mensaje que han escuchado desde el principio es este:
debemos amarnos unos a otros. \footnote{\textbf{3:11} Juan 13,34}
\bibleverse{12} No podemos ser como Caín, que pertenecía al maligno, y
mató a su hermano. ¿Por qué lo mató? Porque Caín era malo, pero su
hermano era justo. \footnote{\textbf{3:12} Gén 4,8}

\hypertarget{el-amor-fraternal-es-un-fruto-importante-en-la-pruxe1ctica-de-la-justicia-el-odio-viene-del-mal}{%
\subsection{El amor fraternal es un fruto importante en la práctica de
la justicia; el odio viene del
mal}\label{el-amor-fraternal-es-un-fruto-importante-en-la-pruxe1ctica-de-la-justicia-el-odio-viene-del-mal}}

\bibleverse{13} Así que no se sorprendan si este mundo los aborrece.
\footnote{\textbf{3:13} Mat 5,11; Juan 15,18-19} \bibleverse{14} La
razón por la que sabemos que hemos ido de la muerte a la vida es porque
amamos a nuestros hermanos y hermanas en la fe. Porque el que no ama
sigue muerto. \footnote{\textbf{3:14} Juan 5,24} \bibleverse{15} Los que
odian a sus hermanos cristianos son asesinos, y ustedes saben que los
asesinos no tendrán vida eterna con ellos. \footnote{\textbf{3:15} Mat
  5,21-22}

\bibleverse{16} Así es como sabemos qué es el amor: Jesús entregó su
vida por nosotros, y nosotros debemos entregar nuestras vidas por
nuestros hermanos en la fe. \footnote{\textbf{3:16} Juan 15,13}
\bibleverse{17} Si alguno de ustedes vive cómodamente en este mundo, y
ve a su hermano o hermana en Cristo padeciendo necesidad, pero no tiene
compasión, ¿cómo podemos decir que el amor vive en ustedes? \footnote{\textbf{3:17}
  Deut 15,7; 1Jn 4,20}

\bibleverse{18} Queridos amigos, no digamos que amamos solo con
palabras, sino mostremos nuestro amor en lo que hacemos y en la manera
como demostramos la verdad. \footnote{\textbf{3:18} Sant 2,15; Sant
  1,2-16}

\hypertarget{el-fruto-de-practicar-la-justicia-y-el-amor-fraternal-es-el-gozo-en-dios-por-la-certeza-de-la-unidad-con-uxe9l}{%
\subsection{El fruto de practicar la justicia y el amor fraternal es el
gozo en Dios por la certeza de la unidad con
él}\label{el-fruto-de-practicar-la-justicia-y-el-amor-fraternal-es-el-gozo-en-dios-por-la-certeza-de-la-unidad-con-uxe9l}}

\bibleverse{19} Así es como sabremos que pertenecemos a la verdad, y
pondremos nuestras mentes\footnote{\textbf{3:19} Literalmente,
  ``corazones''. Se creía que el corazón era el órgano con el que se
  pensaba.} en paz con Dios \bibleverse{20} cuando pensemos que estamos
en error. Dios es más grande de lo que creemos, y lo sabe todo.
\footnote{\textbf{3:20} Luc 15,20-22} \bibleverse{21} Así que, queridos
amigos, si tenemos la tranquilidad de que no estamos en el error,
podemos tener confianza ante Dios. \bibleverse{22} Pues recibiremos de
él cualquier cosa que le pidamos, porque seguimos sus mandamientos y
hacemos lo que le agrada. \bibleverse{23} Y esto es lo que él manda: que
debemos confiar en el nombre\footnote{\textbf{3:23} Nombre, en el
  sentido del carácter y la reputación, más que un nombre asignado.} de
su Hijo Jesucristo, y amarnos unos a otros, así como él nos mandó.
\footnote{\textbf{3:23} Juan 6,29; Juan 15,17} \bibleverse{24} Los que
guardan sus mandamientos siguen viviendo en él, y él vive en ellos. Y
sabemos que él vive en nosotros por el Espíritu que nos ha
dado.\footnote{\textbf{3:24} 1Jn 4,13; Rom 8,9}

\hypertarget{pon-a-prueba-los-espuxedritus-el-espuxedritu-de-dios-le-confiesa-a-jesuxfas-como-el-cristo-que-apareciuxf3-en-carne}{%
\subsection{¡Pon a prueba los espíritus! El espíritu de Dios le confiesa
a Jesús como el Cristo que apareció en
carne}\label{pon-a-prueba-los-espuxedritus-el-espuxedritu-de-dios-le-confiesa-a-jesuxfas-como-el-cristo-que-apareciuxf3-en-carne}}

\hypertarget{section-3}{%
\section{4}\label{section-3}}

\bibleverse{1} Queridos amigos, no confien en todos los espíritus, sino
pruébenlos para saber si son o no de Dios, porque hay muchos falsos
profetas en este mundo. \footnote{\textbf{4:1} Mat 7,15} \bibleverse{2}
¿Cómo pueden reconocer el Espíritu de Dios? Pues todo espíritu que
acepta que Jesús vino en carne humana, es de Dios; \bibleverse{3} pero
todo espíritu que no acepta a Jesús, ese espíritu no es de Dios. De
hecho, es el espíritu del anticristo, del cual oyeron que vendrá, y que
ya está en el mundo. \bibleverse{4} Pero ustedes pertenecen a Dios, mis
amigos, y los han vencido,\footnote{\textbf{4:4} Refiriéndose de nuevo a
  los falsos profetas y al espíritu que los inspira.} porque el que está
en ustedes es más grande que el que está en el mundo. \bibleverse{5}
Ellos pertenecen al mundo, y hablan como personas del mundo, y el mundo
los oye. \bibleverse{6} Sin embargo, nosotros pertenecemos a Dios y todo
el que conoce a Dios, nos escucha; pero los que no pertenecen a Dios, no
nos escuchan. Así es como podemos distinguir el espíritu de verdad del
espíritu de engaño. \footnote{\textbf{4:6} Juan 8,47; 1Cor 14,37}

\hypertarget{amor-verdadero-y-falso-el-amor-fraternal-se-basa-en-la-fe-en-el-amor-de-dios-por-nosotros-en-cristo}{%
\subsection{Amor verdadero y falso; El amor fraternal se basa en la fe
en el amor de Dios por nosotros en
Cristo}\label{amor-verdadero-y-falso-el-amor-fraternal-se-basa-en-la-fe-en-el-amor-de-dios-por-nosotros-en-cristo}}

\bibleverse{7} Queridos amigos, sigamos amándonos unos a otros, porque
el amor viene de Dios. Todos los que aman son nacidos de Dios y conocen
a Dios. \bibleverse{8} Los que no aman, no conocen a Dios, porque Dios
es amor. \bibleverse{9} ¿Cómo nos fue demostrado el amor de Dios? Dios
envió a su único Hijo para que viviéramos por él. \bibleverse{10} ¡Eso
es amor! No es que nosotros hayamos amado a Dios, sino que él nos amó y
envió a su Hijo para ser la reconciliación por nuestros pecados.
\bibleverse{11} Amigos, si esta es la manera como Dios nos ama, debemos
amarnos unos a otros de esta misma manera. \bibleverse{12} Nadie ha
visto a Dios. Sin embargo, si nos amamos unos a otros, entonces Dios
vive en nosotros, y su amor se cumple en nosotros. \footnote{\textbf{4:12}
  Juan 1,18}

\hypertarget{ademuxe1s-del-amor-fraternal-la-posesiuxf3n-del-espuxedritu-y-el-amor-por-dios-que-conocemos-testifican-de-nuestra-comuniuxf3n-con-dios}{%
\subsection{Además del amor fraternal, la posesión del espíritu y el
amor por Dios que conocemos testifican de nuestra comunión con
Dios}\label{ademuxe1s-del-amor-fraternal-la-posesiuxf3n-del-espuxedritu-y-el-amor-por-dios-que-conocemos-testifican-de-nuestra-comuniuxf3n-con-dios}}

\bibleverse{13} ¿Cómo podemos saber que él vive en nosotros? En que nos
ha dado el poder de amar\footnote{\textbf{4:13} Implícito.} por su
Espíritu. \footnote{\textbf{4:13} 1Jn 3,24} \bibleverse{14} Porque somos
testigos de lo que hemos visto y testificamos que el Padre envió al Hijo
como Salvador del mundo. \footnote{\textbf{4:14} Juan 3,17}
\bibleverse{15} Dios vive en todos los que declaran que Jesús es el Hijo
de Dios, y ellos viven en Dios. \footnote{\textbf{4:15} 1Jn 5,5}
\bibleverse{16} Hemos experimentado y creído en el amor que Dios tiene
por nosotros. Dios es amor, y los que viven en amor, viven en Dios, y
Dios en ellos.

\hypertarget{el-fruto-de-esta-comuniuxf3n-de-amor-con-dios-es-la-confianza-gozosa-en-el-duxeda-del-juicio-y-la-pruxe1ctica-del-amor-fraterno}{%
\subsection{El fruto de esta comunión de amor con Dios es la confianza
gozosa en el día del juicio y la práctica del amor
fraterno}\label{el-fruto-de-esta-comuniuxf3n-de-amor-con-dios-es-la-confianza-gozosa-en-el-duxeda-del-juicio-y-la-pruxe1ctica-del-amor-fraterno}}

\bibleverse{17} Es así como el amor se completa en nosotros, para que
podamos estar seguros en el día del juicio: por el hecho de que vivimos
como él en este mundo. \footnote{\textbf{4:17} 1Jn 2,28} \bibleverse{18}
Donde hay amor no puede haber temor. Y Dios nos ama por completo, y este
amor echa fuera todos nuestros miedos. Si tememos, es porque tememos ser
castigados, y eso muestra que no hemos sido plenamente transformados por
la plenitud del amor de Dios. \bibleverse{19} Nosotros amamos porque él
nos amó primero. \bibleverse{20} Los que dicen: ``Yo amo a Dios'', pero
odian a su hermano o hermana en la fe, son mentirosos. Los que no aman a
un hermano al que pueden ver, no pueden amar a Dios, a quien no ven.
\bibleverse{21} Este es el mandamiento que nos dio: los que aman a Dios,
amen también a sus hermanos.\footnote{\textbf{4:21} Mar 12,29-31}

\hypertarget{fe-y-amor-en-su-uniuxf3n}{%
\subsection{Fe y amor en su unión}\label{fe-y-amor-en-su-uniuxf3n}}

\hypertarget{section-4}{%
\section{5}\label{section-4}}

\bibleverse{1} Todo el que cree que Jesús es el Cristo nacido de Dios, y
el que ama al Padre también ama a su hijo. \bibleverse{2} ¿Cómo sabemos
que amamos a los hijos de Dios? Cuando amamos a Dios y seguimos sus
mandamientos. \bibleverse{3} Amar a Dios quiere decir que seguimos sus
mandamientos, y sus mandamientos no son una carga pesada. \bibleverse{4}
Todo el que nace de Dios vence al mundo. La manera como obtenemos la
victoria y vencemos al mundo es por la fe en Dios. \footnote{\textbf{5:4}
  Juan 16,33; 1Cor 15,57}

\hypertarget{el-agua-la-sangre-y-el-espuxedritu-santo-establecen-la-fe-en-jesuxfas-a-travuxe9s-de-su-testimonio}{%
\subsection{El agua, la sangre y el espíritu santo establecen la fe en
Jesús a través de su
testimonio}\label{el-agua-la-sangre-y-el-espuxedritu-santo-establecen-la-fe-en-jesuxfas-a-travuxe9s-de-su-testimonio}}

\bibleverse{5} ¿Quién puede vencer al mundo? Solo los que creen en
Jesús, creyendo que él es el Hijo de Dios. \footnote{\textbf{5:5} 1Jn
  4,4}

\bibleverse{6} Él es el que vino por agua y sangre, Jesucristo. No solo
vino por agua, sino por agua y sangre.\footnote{\textbf{5:6} Esto a
  menudo se interpreta con el fin de dar el significado del agua del
  bautismo y la sangre que significa su muerte.} El Espíritu prueba y
confirma esto, porque el Espíritu es la verdad. \footnote{\textbf{5:6}
  Juan 1,33; Juan 19,34-35; 1Jn 1,7} \bibleverse{7} Asó que hay tres que
dan evidencia de ello: \bibleverse{8} el Espíritu, el agua, y la sangre,
y los tres están de acuerdo como si fueran uno.\footnote{\textbf{5:8}
  5:7, 8. Se debate sobre la autenticidad de los versículos 7 y 8.}
\bibleverse{9} Si aceptamos la evidencia que dan los testigos humanos,
entonces la evidencia que da Dios es más importante. La evidencia que
Dios da es su testimonio sobre su Hijo. \bibleverse{10} Los que creen en
el Hijo de Dios han aceptado y se han aferrado a esta evidencia. Los que
no creen en Dios, llaman a Dios mentirosos, porque no creen la evidencia
que Dios da sobre su Hijo. \bibleverse{11} Y la evidencia es esta: Dios
nos ha dado vida eterna por medio de su Hijo. \bibleverse{12} Todo el
que tiene al Hijo tiene vida; y quien no tiene al Hijo no tiene vida.

\hypertarget{la-oraciuxf3n-e-intercesiuxf3n-de-los-creyentes-es-gozosa-y-eficaz-para-el-perduxf3n-de-los-pecados-que-no-sean-los-mortales}{%
\subsection{La oración e intercesión de los creyentes es gozosa y eficaz
para el perdón de los pecados que no sean los
mortales}\label{la-oraciuxf3n-e-intercesiuxf3n-de-los-creyentes-es-gozosa-y-eficaz-para-el-perduxf3n-de-los-pecados-que-no-sean-los-mortales}}

\bibleverse{13} Escribo para decirles a los que entre ustedes creen en
el nombre del Hijo de Dios, para que puedan estar seguros que tienen la
vida eterna. \footnote{\textbf{5:13} Juan 20,31}

\bibleverse{14} Podemos estar seguros de que él nos escuchará siempre y
cuando pidamos conforme a su voluntad. \footnote{\textbf{5:14} Juan
  14,13} \bibleverse{15} Si sabemos que él oye nuestras peticiones,
podemos estar seguros de que recibiremos lo que le pedimos.

\bibleverse{16} Si ves a tu hermano en la fe cometiendo un pecado que no
es mortal,\footnote{\textbf{5:16} Pecado mortal, literalmente ``un
  pecado para muerte''.} debes orar y Dios le otorgará vida al que ha
pecado. (Pero no por un pecado mortal. Porque hay un pecado que es
mortal, y no quiero decir que la gente deba orar por eso. \footnote{\textbf{5:16}
  Mar 3,20-30} \bibleverse{17} Sí, todo lo que no es recto es pecado,
pero hay un pecado que no es mortal).

\hypertarget{a-travuxe9s-de-la-comuniuxf3n-con-dios-y-jesuxfas-el-creyente-estuxe1-protegido-del-pecado-y-el-sentido-del-mundo}{%
\subsection{A través de la comunión con Dios y Jesús, el creyente está
protegido del pecado y el sentido del
mundo}\label{a-travuxe9s-de-la-comuniuxf3n-con-dios-y-jesuxfas-el-creyente-estuxe1-protegido-del-pecado-y-el-sentido-del-mundo}}

\bibleverse{18} Reconocemos que los que nacen de Dios no siguen pecando
más. El Hijo de Dios\footnote{\textbf{5:18} Literalmente, El Único que
  es nacido de Dios, siguiendo el concepto que está al principio del
  versículo. En el siguiente versículo queda clara su identificación.}
los protege y el diablo no puede hacerles daño. \bibleverse{19} Pues
sabemos que pertenecemos a Dios, y que el mundo está bajo control del
maligno. \footnote{\textbf{5:19} Gal 1,4} \bibleverse{20} También
sabemos que el Hijo de Dios ha venido, y nos ha ayudado a entender, para
que podamos reconocer al que es verdadero. Vivimos en él, que es
verdadero, en su Hijo Jesucristo. Él es el verdadero Dios, y es vida
eterna.\footnote{\textbf{5:20} Puede entenderse de manera que él da la
  vida eterna pero también que él vive eternamente.} \footnote{\textbf{5:20}
  Juan 17,3}

\bibleverse{21} Amigos queridos, aléjense del culto a los ídolos.
