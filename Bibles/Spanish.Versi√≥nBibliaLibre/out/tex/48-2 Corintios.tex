\hypertarget{bendiciones}{%
\subsection{Bendiciones}\label{bendiciones}}

\hypertarget{section}{%
\section{1}\label{section}}

\bibleverse{1} Esta carta viene de parte de Pablo, apóstol de
Jesucristo, conforme a la voluntad de Dios, y de parte de Timoteo,
nuestro hermano. Es enviada a la iglesia de Dios en Corinto, así como a
todo el pueblo de Dios que está por toda la región de Acaya.
\bibleverse{2} Reciban gracia y paz de Dios, nuestro Padre, y del Señor
Jesucristo.

\hypertarget{la-oraciuxf3n-de-acciuxf3n-de-gracias-del-apuxf3stol-por-el-consuelo-que-tanto-uxe9l-como-los-lectores-reciben-de-dios-en-el-sufrimiento}{%
\subsection{La oración de acción de gracias del apóstol por el consuelo
que tanto él como los lectores reciben de Dios en el
sufrimiento}\label{la-oraciuxf3n-de-acciuxf3n-de-gracias-del-apuxf3stol-por-el-consuelo-que-tanto-uxe9l-como-los-lectores-reciben-de-dios-en-el-sufrimiento}}

\bibleverse{3} ¡Alaben a Dios, el padre de nuestro Señor Jesucristo! Él
es el Padre misericordioso, y Dios de toda consolación. \footnote{\textbf{1:3}
  Rom 15,5} \bibleverse{4} Él nos consuela en todas nuestras
aflicciones, para que podamos consolar también a otros con el consuelo
que recibimos de Dios. \bibleverse{5} Cuanto más participamos de los
sufrimientos de Cristo, tanto más abundante es el consuelo que recibimos
de él. \bibleverse{6} Si estamos angustiados, es para su consuelo y
salvación. Si estamos siendo consolados, es para consuelo de ustedes,
que los ayuda a soportar con paciencia los mismos sufrimientos que
nosotros padecemos. \footnote{\textbf{1:6} 2Cor 4,8-11; 2Cor 4,15}
\bibleverse{7} Confiamos en gran manera en ustedes,\footnote{\textbf{1:7}
  Literalmente, ``nuestra esperanza en ustedes está firme''.} sabiendo
que así como participan de nuestros sufrimientos, también participan de
nuestro consuelo.

\hypertarget{mensaje-sobre-la-salvaciuxf3n-de-pablo-y-sus-colaboradores-del-peligro-de-muerte}{%
\subsection{Mensaje sobre la salvación de Pablo y sus colaboradores del
peligro de
muerte}\label{mensaje-sobre-la-salvaciuxf3n-de-pablo-y-sus-colaboradores-del-peligro-de-muerte}}

\bibleverse{8} Hermanos y hermanas, no les ocultaremos los problemas que
tuvimos en Asia. Estábamos tan agobiados que temíamos no tener las
fuerzas para continuar, tanto así que dudábamos de que pudiéramos salir
con vida. \bibleverse{9} De hecho, era como una sentencia de muerte
dentro de nosotros. Esto nos sirvió para dejar de depender de nosotros
mismos y comenzar a confiar en Dios, quien levanta a los muertos.
\bibleverse{10} Él nos salvó de la muerte, y pronto lo hará otra vez.
Tenemos plena confianza en que Dios seguirá salvándonos. \bibleverse{11}
Ustedes nos ayudan con sus oraciones. De este modo, muchos agradecerán a
Dios por la bendición que Dios nos dará en respuesta a las oraciones de
muchos. \footnote{\textbf{1:11} Fil 1,19}

\hypertarget{el-modo-de-vida-honesto-del-apuxf3stol-y-su-veracidad-en-la-correspondencia}{%
\subsection{El modo de vida honesto del apóstol y su veracidad en la
correspondencia}\label{el-modo-de-vida-honesto-del-apuxf3stol-y-su-veracidad-en-la-correspondencia}}

\bibleverse{12} Nos enorgullecemos en el hecho---y nuestra conciencia lo
confirma---de que hemos actuado de manera apropiada con las personas,
especialmente con ustedes. Hemos seguido los principios de Dios de
santidad y sinceridad, no conforme a la sabiduría mundanal, sino por la
gracia de Dios. \footnote{\textbf{1:12} 2Cor 2,17; Heb 13,18; 1Cor 1,17}
\bibleverse{13} Porque no escribimos ninguna cosa complicada que ustedes
no puedan leer o comprender. Espero que ustedes al final entiendan,
\bibleverse{14} aunque ahora solo entiendan en parte, a fin de que
cuando el Señor venga, ustedes estén orgullosos de nosotros, como
nosotros de ustedes. \footnote{\textbf{1:14} 2Cor 5,12; Fil 2,16}

\hypertarget{el-relato-del-apuxf3stol-del-cambio-en-sus-planes-de-viaje-indicaciuxf3n-de-su-fiabilidad-como-apuxf3stol-de-cristo-y-de-dios-fiel}{%
\subsection{El relato del apóstol del cambio en sus planes de viaje;
Indicación de su fiabilidad como apóstol de Cristo y de Dios
fiel}\label{el-relato-del-apuxf3stol-del-cambio-en-sus-planes-de-viaje-indicaciuxf3n-de-su-fiabilidad-como-apuxf3stol-de-cristo-y-de-dios-fiel}}

\bibleverse{15} Como yo estaba tan seguro de su confianza en mí, hice
planes para venir a visitarlos primero. Así ustedes se habrían
beneficiado doblemente, \bibleverse{16} pues iría desde donde están
ustedes a Macedonia, y luego volvería desde Macedonia a donde ustedes
nuevamente. Luego yo les habría pedido que me enviaran de camino a
Judea. \bibleverse{17} ¿Por qué cambié mi plan original? ¿Creen que tomo
decisiones a la ligera? ¿Creen que cuando hago planes soy como cualquier
persona del mundo que dice Sí y No al mismo tiempo? \bibleverse{18} Así
como Dios es digno de confianza, cuando nosotros les damos nuestra
palabra, no es Sí y No a la vez. \bibleverse{19} La verdad del Hijo de
Dios, Jesucristo, fue anunciada a ustedes por medio de
nosotros---Silvano, Timoteo y yo---y no fue Sí y No.~¡En Cristo la
respuesta es definitivamente Sí! \footnote{\textbf{1:19} Hech 18,5}
\bibleverse{20} No importa cuántas promesas Dios haya hecho, en Cristo
la respuesta siempre es Sí. Por él, respondemos diciendo Sí\footnote{\textbf{1:20}
  Literalmente, ``Amén'', que significa ``Sí'', o ``Estoy de acuerdo''.}
a la gloria de Dios. \footnote{\textbf{1:20} Apoc 3,14}

\bibleverse{21} Él nos ha dado a nosotros y también a ustedes la fuerza
para permanecer firmes en Cristo. Dios nos ha ungido, \footnote{\textbf{1:21}
  1Jn 2,27} \bibleverse{22} ha puesto su sello de aprobación sobre
nosotros, y nos ha dado la garantía del Espíritu en nuestros corazones.
\footnote{\textbf{1:22} 2Cor 5,5; Rom 8,16; Efes 1,13}

\hypertarget{declaraciuxf3n-de-la-verdadera-razuxf3n-por-la-que-pablo-no-vino-a-corinto}{%
\subsection{Declaración de la verdadera razón por la que Pablo no vino a
Corinto}\label{declaraciuxf3n-de-la-verdadera-razuxf3n-por-la-que-pablo-no-vino-a-corinto}}

\bibleverse{23} Pongo a Dios como mi testigo que la razón por la que
decidí no ir a Corinto fue para no causarles dolor. \bibleverse{24} El
propósito de esto no es dictarles la manera en que deben relacionarse
con Dios, sino porque queremos ayudarlos a tener una experiencia de
gozo, porque es a través de la fe en Dios que permanecemos
firmes.\footnote{\textbf{1:24} 1Pe 5,3; 2Cor 4,5}

\hypertarget{section-1}{%
\section{2}\label{section-1}}

\bibleverse{1} Por eso decidí que evitaría otra visita triste con
ustedes. \footnote{\textbf{2:1} 1Cor 4,21; 2Cor 12,21} \bibleverse{2}
Porque si les causo tristeza, ¿quién estará allí para alegrarme a mí?
¡No serán ustedes mismos, a quienes entristecí! \bibleverse{3} Por eso
escribí lo que escribí, para no estar triste por los que deberían
causarme alegría. Estaba muy seguro de que todos ustedes participarían
de mi felicidad. \bibleverse{4} Lloré mucho cuando les escribí, en gran
angustia y con un corazón cargado, no para entristecerlos, sino para que
supieran cuánto los amo.

\hypertarget{eliminaciuxf3n-de-la-brecha-entre-pablo-y-los-corintios-recomendaciuxf3n-de-indulgencia-contra-el-malhechor-arrepentido}{%
\subsection{Eliminación de la brecha entre Pablo y los Corintios;
Recomendación de indulgencia contra el malhechor
arrepentido}\label{eliminaciuxf3n-de-la-brecha-entre-pablo-y-los-corintios-recomendaciuxf3n-de-indulgencia-contra-el-malhechor-arrepentido}}

\bibleverse{5} Sin exagerar, pero la persona que causó mi tristeza,
provocó más dolor a todos ustedes que a mí. \bibleverse{6} Esta persona
sufrió suficiente castigo por parte de la mayoría de ustedes,
\bibleverse{7} así que ahora deben perdonarlo y ser amables con él. De
lo contrario, podría hundirse en el remordimiento. \bibleverse{8} Así
que yo los animo a que públicamente confirmen su amor hacia él.
\bibleverse{9} Por eso escribí, para poder Conocer el carácter de
ustedes y comprobar si están haciendo lo que se les enseñó.
\bibleverse{10} A todo el que ustedes perdonen, yo también perdono. Lo
que he perdonado, sea lo que sea, lo he perdonado ante Cristo, en
beneficio de ustedes. \footnote{\textbf{2:10} Juan 20,23}
\bibleverse{11} De este modo, Satanás no podrá llevarnos hacia el
pecado, porque conocemos las trampas que él inventa. \footnote{\textbf{2:11}
  Luc 22,31; 1Pe 5,8}

\hypertarget{las-experiencias-del-apuxf3stol-en-troas-y-macedonia-su-alabanza-a-dios-por-el-efecto-victorioso-de-la-proclamaciuxf3n-de-la-salvaciuxf3n}{%
\subsection{Las experiencias del apóstol en Troas y Macedonia; Su
alabanza a Dios por el efecto victorioso de la proclamación de la
salvación}\label{las-experiencias-del-apuxf3stol-en-troas-y-macedonia-su-alabanza-a-dios-por-el-efecto-victorioso-de-la-proclamaciuxf3n-de-la-salvaciuxf3n}}

\bibleverse{12} Cuando llegué a Troas para predicar la buena noticia de
Cristo, el Señor puso delante de mí una oportunidad. \footnote{\textbf{2:12}
  Hech 14,27; 1Cor 16,9} \bibleverse{13} Pero mi mente no estaba en paz
porque no podía encontrar a mi hermano Tito. De modo que me despedí y me
fui hacia Macedonia.\footnote{\textbf{2:13} Viajar de Troas a Macedonia
  implicaba realizar un cruce por el mar.} \footnote{\textbf{2:13} Hech
  20,1; 2Cor 7,6}

\bibleverse{14} ¡Pero gloria a Dios, que siempre nos guía hacia la
victoria en Cristo, y revela un dulce aroma de su conocimiento a través
de nosotros, dondequiera que vamos! \bibleverse{15} Somos como una
fragancia de Cristo para Dios, que se eleva entre los que son salvos así
como entre los que mueren. \footnote{\textbf{2:15} Éxod 29,18; 1Cor 1,18}
\bibleverse{16} Para los que mueren, es el aroma de la descomposición,
pero para los que son salvos, es el aroma de la vida. ¿Pero de quién
depende esta tarea? \footnote{\textbf{2:16} Luc 2,34; 2Cor 3,5}
\bibleverse{17} No somos como la mayoría, que hacen negocios con la
palabra de Dios por conveniencia. Muy por el contrario: somos sinceros
al predicar la palabra de Dios en Cristo, sabiendo que él nos
ve.\footnote{\textbf{2:17} 2Cor 1,12; 2Cor 4,2; 1Pe 4,11}

\hypertarget{la-iglesia-de-corinto-como-carta-de-recomendaciuxf3n-para-pablo-y-dios-como-base-segura-de-confianza-para-el-apuxf3stol}{%
\subsection{La iglesia de Corinto como carta de recomendación para Pablo
y Dios como base segura de confianza para el
apóstol}\label{la-iglesia-de-corinto-como-carta-de-recomendaciuxf3n-para-pablo-y-dios-como-base-segura-de-confianza-para-el-apuxf3stol}}

\hypertarget{section-2}{%
\section{3}\label{section-2}}

\bibleverse{1} ¿Acaso estamos empezando a hablar bien de nosotros mismos
una vez más? ¿O necesitamos una carta de recomendación para ustedes, o
de parte de ustedes, como algunos? \footnote{\textbf{3:1} 2Cor 5,12}
\bibleverse{2} Ustedes son nuestra carta de recomendación, escrita en
nuestros corazones, la cual todo el mundo conoce y puede leer.
\footnote{\textbf{3:2} 1Cor 9,2} \bibleverse{3} Ustedes demuestran que
son una carta de Cristo, entregada por nosotros; no escrita con tinta,
sino con el Espíritu del Dios vivo; no escrita sobre piedras, sino en
corazones humanos. \footnote{\textbf{3:3} Éxod 24,12}

\bibleverse{4} Tenemos plena confianza ante Dios por medio de Cristo.
\bibleverse{5} No porque consideremos que nosotros mismos podemos
hacerlo, sino que Dios nos da este poder. \footnote{\textbf{3:5} 2Cor
  2,16}

\hypertarget{la-gloria-del-nuevo-pacto-y-el-ministerio-apostuxf3lico-sobre-el-antiguo-pacto-y-el-ministerio-de-moisuxe9s}{%
\subsection{La gloria del nuevo pacto y el ministerio apostólico sobre
el antiguo pacto y el ministerio de
Moisés}\label{la-gloria-del-nuevo-pacto-y-el-ministerio-apostuxf3lico-sobre-el-antiguo-pacto-y-el-ministerio-de-moisuxe9s}}

\bibleverse{6} También nos da la capacidad de ser ministros de un nuevo
acuerdo,\footnote{\textbf{3:6} O ``pacto''.} no basado en la letra de la
ley, sino en el Espíritu. La letra de la ley mata, pero el Espíritu da
vida. \footnote{\textbf{3:6} Jer 31,31; 1Cor 11,25; Rom 7,6; Juan 6,63}

\bibleverse{7} Sin embargo, la antigua forma de relacionarnos con Dios,
escrita en piedras, terminó en muerte, aunque fue entregada con la
gloria de Dios, tanto así, que los israelitas no pudieron soportar ver
el rostro de Moisés porque era muy brillante, aunque esa gloria se
estaba desvaneciendo. \footnote{\textbf{3:7} Éxod 34,29-35}
\bibleverse{8} Si fue así, ¿no debería venir con mayor gloria la nueva
forma de relacionarnos con Dios en el Espíritu? \footnote{\textbf{3:8}
  Gal 3,2; Gal 3,5} \bibleverse{9} ¡Si la antigua forma que nos condena
trae gloria, la nueva forma, que nos justifica, trae consigo mucha más
gloria todavía! \footnote{\textbf{3:9} Deut 27,26; Rom 1,17; Rom 3,21}
\bibleverse{10} Porque las cosas viejas que una vez fueron gloriosas, no
tienen gloria en comparación con la increíble gloria de lo nuevo.
\bibleverse{11} Si lo viejo, que se desvanece, tenía gloria, lo nuevo,
que no se acaba, tiene mucha más gloria.

\hypertarget{la-diferencia-entre-los-dos-tipos-de-servicios-es-evidente-tanto-en-sus-servidores-como-en-sus-efectos}{%
\subsection{La diferencia entre los dos tipos de servicios es evidente
tanto en sus servidores como en sus
efectos}\label{la-diferencia-entre-los-dos-tipos-de-servicios-es-evidente-tanto-en-sus-servidores-como-en-sus-efectos}}

\bibleverse{12} ¡Y como tenemos esta esperanza segura, hablamos sin
temor! \bibleverse{13} No tenemos que ser como Moisés, que tuvo que
ponerse un velo para cubrir su rostro y así los israelitas no fueran
enceguecidos por la gloria, aunque ya se estaba desvaneciendo.
\bibleverse{14} No obstante, sus corazones se endurecieron. Porque desde
ese entonces hasta ahora, cuando se lee el antiguo pacto, permanece el
mismo ``velo''. \footnote{\textbf{3:14} Rom 11,25; Hech 28,27}
\bibleverse{15} Incluso hoy, cada vez que se leen los libros de Moisés,
un velo cubre sus mentes. \bibleverse{16} Pero cuando se convierten y
aceptan al Señor, el velo se quita. \bibleverse{17} Ahora bien, el Señor
es el Espíritu, y dondequiera está el Espíritu del Señor, hay libertad.
\bibleverse{18} Así que todos nosotros, con nuestros rostros
descubiertos, vemos y reflejamos al Señor como en un espejo. Estamos
siendo transformados conforme a la misma imagen del espejo, cuya gloria
es cada vez más brillante. Esto es lo que hace el Señor, que es el
Espíritu.\footnote{\textbf{3:18} 2Cor 4,6}

\hypertarget{pablo-y-sus-seguidores-aparecen-como-verdaderos-mensajeros-de-cristo-con-valentuxeda-veracidad-e-iluminaciuxf3n-divina}{%
\subsection{Pablo y sus seguidores aparecen como verdaderos mensajeros
de Cristo con valentía, veracidad e iluminación
divina}\label{pablo-y-sus-seguidores-aparecen-como-verdaderos-mensajeros-de-cristo-con-valentuxeda-veracidad-e-iluminaciuxf3n-divina}}

\hypertarget{section-3}{%
\section{4}\label{section-3}}

\bibleverse{1} Así pues, como Dios en su misericordia nos ha
proporcionado esta nueva manera de relacionarnos con él, no nos
rendimos. \footnote{\textbf{4:1} 2Cor 3,6; 1Cor 7,25} \bibleverse{2}
Pero sí hemos renunciado a los actos secretos y vergonzosos. No actuamos
con engaño ni distorsionamos la Palabra de Dios. Nosotros demostramos lo
que somos al revelar la verdad ante Dios, a fin de que todos puedan
decidirse a conciencia. \footnote{\textbf{4:2} 2Cor 2,17; 1Tes 2,5}
\bibleverse{3} Aún si la nueva noticia que compartimos está velada, lo
está para los que mueren. \footnote{\textbf{4:3} 1Cor 1,18}
\bibleverse{4} El dios de este mundo ha cegado las mentes de los que no
creen en Dios. Ellos no pueden ver la luz de la buena noticia de la
gloria de Cristo, quien es la imagen de Dios. \footnote{\textbf{4:4} Heb
  1,3} \bibleverse{5} No nos anunciamos\footnote{\textbf{4:5}
  Literalmente, ``predicamos''.} a nosotros mismos, sino a Cristo Jesús
como Señor. De hecho, somos siervos de ustedes por causa de Jesús.
\footnote{\textbf{4:5} 2Cor 1,24} \bibleverse{6} Porque el Dios que
dijo: ``Que brille la luz en medio de la oscuridad'',\footnote{\textbf{4:6}
  Citando Génesis 1:3.} brilló en nuestros corazones para iluminar el
conocimiento de la gloria de Dios en el rostro de Jesucristo.
\footnote{\textbf{4:6} Gén 1,3; 2Cor 3,18}

\hypertarget{el-sufrimiento-externo-de-los-apuxf3stoles-ademuxe1s-de-su-confianza-en-la-fe}{%
\subsection{El sufrimiento externo de los apóstoles además de su
confianza en la
fe}\label{el-sufrimiento-externo-de-los-apuxf3stoles-ademuxe1s-de-su-confianza-en-la-fe}}

\bibleverse{7} Pero tenemos este tesoro en vasijas de barro, para
demostrar que este poder supremo proviene de Dios y no de nosotros.
\footnote{\textbf{4:7} 1Cor 4,11-13; 2Cor 11,23-27} \bibleverse{8} Nos
atacan por todos lados, pero no estamos derrotados. Estamos confundidos
en cuanto a qué hacer, pero nunca desesperados. \bibleverse{9} Estamos
perseguidos, pero nunca abandonados por Dios. ¡Estamos derribados, pero
no destruidos! \bibleverse{10} En nuestros cuerpos siempre participamos
de la muerte de Jesús, para así también poder demostrar la vida de Jesús
en nuestros cuerpos. \footnote{\textbf{4:10} 1Cor 15,31; Gal 6,17}
\bibleverse{11} Aunque vivimos, estamos siempre bajo amenaza de muerte
por causa de Jesús, a fin de que la vida de Jesús pueda revelarse en
nuestros cuerpos mortales. \footnote{\textbf{4:11} Rom 8,36}
\bibleverse{12} En consecuencia, enfrentamos la muerte para que ustedes
tengan vida.

\bibleverse{13} Como tenemos el mismo espíritu de confianza en Dios al
que se refiere la Escritura cuando dice: ``Creí en Dios, por tanto
hablé'',\footnote{\textbf{4:13} Citando Salmos 116:10.} nosotros también
creemos en Dios y hablamos de él. \bibleverse{14} Sabemos que Dios,
quien resucitó a Jesús, también nos resucitará con él, y nos llevará a
su presencia con ustedes. \footnote{\textbf{4:14} 1Cor 6,14}
\bibleverse{15} ¡Todo es por ustedes! Cuantos más alcance la gracia de
Dios, mayor será nuestro agradecimiento a él, a su gloria. \footnote{\textbf{4:15}
  2Cor 1,6; 2Cor 1,11}

\hypertarget{la-renovaciuxf3n-del-hombre-espiritual-tiene-lugar-en-la-muerte-del-hombre-exterior}{%
\subsection{La renovación del hombre espiritual tiene lugar en la muerte
del hombre
exterior}\label{la-renovaciuxf3n-del-hombre-espiritual-tiene-lugar-en-la-muerte-del-hombre-exterior}}

\bibleverse{16} Por eso no nos rendimos. Aunque nuestros cuerpos físicos
están cayéndose a pedazos, nuestro interior se renueva cada día.
\footnote{\textbf{4:16} Efes 3,16} \bibleverse{17} Estas tribulaciones
triviales que tenemos, apenas duran un poco de tiempo, pero producen
para nosotros gloria eterna. \footnote{\textbf{4:17} Rom 8,17-18; 1Pe
  1,6} \bibleverse{18} No nos interesa lo visible, porque aspiramos a lo
invisible. Lo que vemos es temporal, pero lo que no vemos es
eterno.\footnote{\textbf{4:18} Heb 11,1}

\hypertarget{la-esperanza-y-el-anhelo-de-pablo-por-la-corporalidad-celestial-y-el-hogar-celestial}{%
\subsection{La esperanza y el anhelo de Pablo por la corporalidad
celestial y el hogar
celestial}\label{la-esperanza-y-el-anhelo-de-pablo-por-la-corporalidad-celestial-y-el-hogar-celestial}}

\hypertarget{section-4}{%
\section{5}\label{section-4}}

\bibleverse{1} Sabemos que cuando esta ``tienda de campaña''\footnote{\textbf{5:1}
  El simbolismo que vemos aquí es que el cuerpo terrenal es como una
  tienda de campaña, y un cuerpo celestial es una casa, y ambos
  ``visten'' a la persona.} terrenal en la que vivimos sea derribada,
tenemos una casa preparada por Dios, no hecha por manos humanas. Es
eterna, y está en el cielo. \footnote{\textbf{5:1} Job 4,19; 2Pe 1,14}
\bibleverse{2} Suspiramos en nuestro anhelo por esto, deseando con
ansias ser vestidos de este nuevo hogar celestial. \bibleverse{3} Cuando
tengamos este vestido, ya no nos veremos desnudos. \bibleverse{4} Aunque
estamos en esta ``tienda'' suspiramos, agobiados por esta vida. No
deseamos tanto ser desvestidos de lo que nos ofrece esta vida, sino que
ansiamos aquello con lo que seremos revestidos, para que lo mortal sea
aplastado por la vida. \footnote{\textbf{5:4} 1Cor 15,51-53}
\bibleverse{5} Dios mismo preparó todo esto para nosotros, y nos dio al
Espíritu como garantía. \footnote{\textbf{5:5} 2Cor 1,22; Rom 8,16; Rom
  8,23; Efes 1,13-14}

\bibleverse{6} Por ello mantenemos la fe, sabiendo que aunque estamos en
casa, con nuestros cuerpos físicos, estamos lejos del Señor. \footnote{\textbf{5:6}
  Heb 11,13} \bibleverse{7} (Pues vivimos por la fe en el Señor, y no
por vista). \footnote{\textbf{5:7} Rom 8,24; 1Pe 1,8} \bibleverse{8}
Como les digo, estamos seguros, deseando estar lejos del cuerpo para
poder estar en casa con el Señor. \footnote{\textbf{5:8} Fil 1,23}
\bibleverse{9} Por eso nuestra meta, ya sea que estemos en nuestro
cuerpo o no, es agradarle. \footnote{\textbf{5:9} Sal 39,13}
\bibleverse{10} Porque todos debemos comparecer ante el tribunal de
Cristo. Y cada uno de nosotros recibirá lo que merece por lo que hayamos
hecho en esta vida, ya sea bueno o malo. \footnote{\textbf{5:10} Juan
  5,29; Hech 17,31; Rom 2,16; Rom 14,10; Efes 6,8}

\hypertarget{comentarios-personales-especialmente-sobre-su-relaciuxf3n-con-la-comunidad}{%
\subsection{Comentarios personales, especialmente sobre su relación con
la
comunidad}\label{comentarios-personales-especialmente-sobre-su-relaciuxf3n-con-la-comunidad}}

\bibleverse{11} Sabiendo lo que es el temor al Señor, tratamos de
convencer a otros. Para Dios es claro lo que somos, y espero que esté
claro en sus mentes también. \bibleverse{12} Una vez más, no intentamos
hablar bien de nosotros mismos, sino que tratamos de darles a ustedes la
oportunidad de que se sientan orgullosos de nosotros, a fin de que
puedan responderle a los que se enorgullecen de lo exterior y no de lo
interior.\footnote{\textbf{5:12} Literalmente, ``en el corazón''.}
\bibleverse{13} Si estamos ``locos''\footnote{\textbf{5:13} Eso era
  posiblemente una crítica hecha por los de corinto respecto a Pablo y
  sus compañeros.} es por Dios. Si somos sensatos, es por ustedes.

\hypertarget{referencia-al-contenido-peculiar-de-su-sermuxf3n-y-la-gloria-de-su-servicio-de-reconciliaciuxf3n}{%
\subsection{Referencia al contenido peculiar de su sermón y la gloria de
su servicio de
reconciliación}\label{referencia-al-contenido-peculiar-de-su-sermuxf3n-y-la-gloria-de-su-servicio-de-reconciliaciuxf3n}}

\bibleverse{14} El amor de Cristo nos obliga, porque estamos
completamente seguros de que él murió por todos y así todos murieron.
\bibleverse{15} Cristo murió por todos para que ya no vivieran para sí
mismos, sino para él, quien murió y resucitó para ellos.

\bibleverse{16} De ahora en adelante ya no miramos a nadie desde el
punto de vista humano. Aunque una vez vimos a Cristo de esta manera, ya
no lo hacemos. \bibleverse{17} Por eso todo el que está en Cristo es un
nuevo ser. ¡Lo viejo ya se ha ido y ha llegado lo nuevo! \footnote{\textbf{5:17}
  Rom 8,10; Gal 2,20; Gal 6,15; Apoc 21,5} \bibleverse{18} Dios lo hizo
transformándonos de enemigos en amigos por medio de Cristo. Dios nos
encomendó este mismo trabajo de convertir a sus enemigos en sus amigos.
\footnote{\textbf{5:18} Rom 5,10} \bibleverse{19} Porque Dios estaba en
Cristo trayendo al mundo de regreso de la hostilidad a la amistad con
él, sin contar sus pecados, y dándonos este mensaje para convertir a sus
enemigos en sus amigos. \footnote{\textbf{5:19} Rom 3,24-25; Col 1,19-20}

\bibleverse{20} De modo que somos embajadores de Cristo, como si él
rogara por nosotros: ``Por favor, vuelvan a él y sean sus amigos''
\footnote{\textbf{5:20} Luc 10,16} \bibleverse{21} Dios hizo que Jesús,
quien nunca pecó, experimentara las consecuencias del pecado para que
nosotros pudiéramos tener un carácter recto, así como Dios es
recto.\textsuperscript{{[}\textbf{5:21} O, ``pudiéramos llegar a ser
rectos como él es recto''.{]}}{[}\textbf{5:21} Is 53,6; Juan 8,46; Rom
1,17{]}

\hypertarget{pablo-como-apuxf3stol-es-ejemplar-por-su-abnegaciuxf3n-y-su-realizaciuxf3n-profesional-desinteresada-en-el-servicio-de-dios}{%
\subsection{Pablo, como apóstol, es ejemplar por su abnegación y su
realización profesional desinteresada en el servicio de
Dios}\label{pablo-como-apuxf3stol-es-ejemplar-por-su-abnegaciuxf3n-y-su-realizaciuxf3n-profesional-desinteresada-en-el-servicio-de-dios}}

\hypertarget{section-5}{%
\section{6}\label{section-5}}

\bibleverse{1} Como colaboradores de Dios, también les rogamos que no
acepten la gracia de Dios en vano. \footnote{\textbf{6:1} 2Cor 1,24}
\bibleverse{2} Tal como Dios dijo: ``En el momento apropiado te escuché,
y en el día de salvación te salvé''.\footnote{\textbf{6:2} Citando
  Isaías 49:8.} Créanme, ¡ahora es el momento apropiado! ¡Ahora es el
día de salvación! \footnote{\textbf{6:2} Luc 4,19; Luc 4,21}

\bibleverse{3} Nosotros no ponemos obstáculos en el camino de nadie para
que ningunno tropiece, asegurándonos de que nadie critique la obra que
hacemos. \bibleverse{4} En lugar de ello tratamos de demostrar que somos
buenos siervos de Dios en todas las formas posibles. Con mucha paciencia
soportamos todo tipo de problemas, dificultades y angustias.
\bibleverse{5} Hemos sido azotados, llevados a la cárcel y atacados por
turbas. Nos han hecho trabajar hasta el cansancio, soportando noches sin
dormir y con hambre. \bibleverse{6} Viviendo vidas irreprensibles en el
conocimiento de Dios, con mucha paciencia, siendo amables y llenos del
Espíritu Santo, mostrando amor sincero. \footnote{\textbf{6:6} 1Tim 4,12}
\bibleverse{7} Hablamos con fidelidad,\footnote{\textbf{6:7} O ``palabra
  de verdad'', refiriéndose al evangelio.} viviendo en el poder de Dios.
Nuestras armas son lo verdadero y lo recto; atacamos con nuestra mano
derecha y nos defendemos con la izquierda,\footnote{\textbf{6:7}
  Literalmente, ``armas de derecha e izquierda''. Esto posiblemente se
  refiere al uso de una espada en la mano derecha, y un escudo en la
  mano izquierda.} \footnote{\textbf{6:7} 2Cor 4,2; 1Cor 2,4; Efes
  6,14-17} \bibleverse{8} Nosotros seguimos, no importa si recibimos
honra o deshonra, si somos maldecidos o alabados. La gente nos llama
fraude, pero nosotros decimos la verdad. \bibleverse{9} Somos
menospreciados, aunque somos reconocidos; nos han dado por muertos, pero
aún estamos vivos; nos han dado latigazos pero no hemos muerto.
\footnote{\textbf{6:9} 2Cor 4,10-11; Sal 118,18; Hech 14,19}
\bibleverse{10} ¡Nos han considerado como miserables, pero siempre
estamos gozosos; como pobres, pero hacemos ricos a muchos; nos han
considerado como desamparados, pero lo tenemos todo! \footnote{\textbf{6:10}
  Fil 4,12-13}

\hypertarget{peticiuxf3n-solemne-y-amorosa-a-los-corintios-para-la-restauraciuxf3n-completa-de-la-comuniuxf3n}{%
\subsection{Petición solemne y amorosa a los corintios para la
restauración completa de la
comunión}\label{peticiuxf3n-solemne-y-amorosa-a-los-corintios-para-la-restauraciuxf3n-completa-de-la-comuniuxf3n}}

\bibleverse{11} Les he hablado con franqueza, mis amigos de Corinto,
abriéndoles todo mi corazón. \bibleverse{12} No les hemos negado nuestro
amor, pero ustedes sí lo han hecho. \bibleverse{13} ¡Como si fueran mis
hijos, les ruego que correspondan, y amen con todo el corazón!
\footnote{\textbf{6:13} 1Cor 4,14}

\hypertarget{advertencia-contra-los-seres-paganos-y-demanda-de-perfecta-santificaciuxf3n}{%
\subsection{Advertencia contra los seres paganos y demanda de perfecta
santificación}\label{advertencia-contra-los-seres-paganos-y-demanda-de-perfecta-santificaciuxf3n}}

\bibleverse{14} No se junten con los que no creen. ¿Acaso qué relación
tiene el bien con el mal? O ¿qué tienen en común la luz con las
tinieblas? \footnote{\textbf{6:14} Efes 5,11} \bibleverse{15} ¿Podrían
alguna vez estar de acuerdo Cristo y el Diablo?\footnote{\textbf{6:15}
  Literalmente, ``Belial''.} ¿Cómo podrían compartir juntos un creyente
con un incrédulo? \bibleverse{16} ¿Qué compromiso podría existir entre
el Templo de Dios con los ídolos? Pues nosotros somos Templo del Dios
vivo, tal como Dios dijo: ``Viviré en ellos y caminaré en medio de
ellos. Yo seré su Dios, y ellos serán mi pueblo''.\footnote{\textbf{6:16}
  Citando Levítico 26:12 y Ezequiel 37:27.} \footnote{\textbf{6:16} 1Cor
  3,16} \bibleverse{17} ``Así que abandónenlos y apártense de ellos,
dice el Señor. No toquen nada impuro, y los aceptaré''.\footnote{\textbf{6:17}
  Citando Isaías 52:11, Ezequiel 20:34 y Ezequiel 20:41.} \footnote{\textbf{6:17}
  Apoc 18,14}

\bibleverse{18} ``Seré como un Padre para ustedes, y ustedes serán mis
hijos e hijas, dice el Señor Todopoderoso''.\footnote{\textbf{6:18}
  Citando 2 Samuel 7:14 o 1 Crónicas 17:13.}

\hypertarget{section-6}{%
\section{7}\label{section-6}}

\bibleverse{1} Queridos amigos, dado que tenemos estas promesas,
limpiémonos de todo lo que contamina nuestro cuerpo y espíritu,
procurando la santidad que nace de la reverencia a Dios.

\hypertarget{la-peticiuxf3n-del-apuxf3stol-de-amor-afirmaciuxf3n-de-amor-y-testimonio-de-confianza}{%
\subsection{La petición del apóstol de amor, afirmación de amor y
testimonio de
confianza}\label{la-peticiuxf3n-del-apuxf3stol-de-amor-afirmaciuxf3n-de-amor-y-testimonio-de-confianza}}

\bibleverse{2} ¡Por favor, abran un espacio para nosotros en sus
corazones! No le hemos hecho mal a nadie, no hemos corrompido a nadie,
ni nos hemos aprovechado de nadie. \footnote{\textbf{7:2} 2Cor 12,17;
  Hech 20,33} \bibleverse{3} No lo digo para condenarlos a ustedes, pues
como ya les dije, ustedes son muy importantes para nosotros, tanto, que
estamos dispuestos a vivir y morir con ustedes. \footnote{\textbf{7:3}
  2Cor 6,11-13; Rom 6,8} \bibleverse{4} Les hablo con confianza porque
estoy orgulloso de ustedes. Son una fuente de ánimo para mí. Y estoy muy
contento de ustedes a pesar de todas nuestras dificultades.

\hypertarget{alegruxeda-del-apuxf3stol-por-la-llegada-y-el-mensaje-de-tito}{%
\subsection{Alegría del apóstol por la llegada y el mensaje de
Tito}\label{alegruxeda-del-apuxf3stol-por-la-llegada-y-el-mensaje-de-tito}}

\bibleverse{5} Cuando llegamos a Macedonia, no tuvimos ni un minuto de
paz. Recibimos ataques por todas partes, por causa de conflictos
externos así como de miedos internos. \footnote{\textbf{7:5} Hech 20,1-2}
\bibleverse{6} Aun así, Dios, quien alienta a los abatidos de corazón,
nos animó con la llegada de Tito. \footnote{\textbf{7:6} 2Cor 2,13; 2Cor
  4,8} \bibleverse{7} Y no solo con su llegada, sino con el ánimo que
ustedes le dieron a él. Él nos contó cuánto deseaban verme, cuán tristes
y preocupados estaban por mí, lo cual me hizo aún más feliz.

\hypertarget{el-gozo-del-apuxf3stol-por-el-efecto-saludable-de-la-carta-penal-por-el-entendimiento-completamente-restaurado-y-por-el-informe-favorable-de-tito}{%
\subsection{El gozo del apóstol por el efecto saludable de la carta
penal, por el entendimiento completamente restaurado y por el informe
favorable de
Tito}\label{el-gozo-del-apuxf3stol-por-el-efecto-saludable-de-la-carta-penal-por-el-entendimiento-completamente-restaurado-y-por-el-informe-favorable-de-tito}}

\bibleverse{8} Aunque los hice entristecer con la carta que les escribí,
no me arrepiento, aunque sí me arrepiento porque la carta los haya
entristecido, pero fue solo por un poco tiempo. \footnote{\textbf{7:8}
  2Cor 2,4} \bibleverse{9} Ahora estoy feliz, no por entristecerlos,
sino porque esa tristeza los hizo cambiar. Llegaron a sentir la tristeza
de una manera que Dios aprueba, por lo tanto no les hicimos daño de
ninguna manera. \bibleverse{10} La tristeza que Dios quiere que sintamos
es la que nos lleva al arrepentimiento y trae salvación. Esta clase de
tristeza no trae consigo ningún tipo de remordimiento, pero la tristeza
mundanal trae muerte. \bibleverse{11} Miren, por ejemplo, lo que ocurrió
cuando tuvieron esta misma experiencia de tristeza que viene de Dios.
Recuerden cuán empeñados y afanados se volvieron por defenderse, cuánto
enojo sintieron por lo que había sucedido, con cuanta seriedad asumieron
las cosas, y cuánto anhelo tenían por hacer lo recto; estaban muy
preocupados y deseosos de que se hiciera justicia. En todo esto ustedes
demostraron que eran sinceros en su deseo de hacer las cosas
rectamente.\footnote{\textbf{7:11} Pareciera que Pablo se está
  refiriendo a problemas anteriores, que necesitaban atención. Por
  ejemplo, el capítulo 2.} \bibleverse{12} Así que cuando les escribí,
no era para hablarles respecto al agresor ni del agredido, sino para
mostrarles cuán fieles son ustedes a nosotros, ante los ojos de Dios.
\bibleverse{13} Esto nos anima en gran manera. Además de este ánimo, nos
alegró ver cuán feliz estaba Tito porque ustedes le dieron fortaleza.

\bibleverse{14} Me enorgullecí\footnote{\textbf{7:14} Aquí y en el resto
  de esta carta, Pablo habla de su jactancia. Esto debe tomarse como un
  cumplido dirigido a los otros, más que como orgullo respecto a sí
  mismo.} de ustedes al hablar con él, y no me defraudaron. Así como
todas las demás cosas que les digo son verdaderas, mis elogios sobre
ustedes hacia Tito resultaron ser verdaderos también. \bibleverse{15} Él
se preocupa por ustedes aún más al recordar que ustedes hicieron todo lo
que él les pidió y lo recibieron con mucho respeto. \bibleverse{16} Me
siento muy feliz de poder confiar plenamente en ustedes.

\hypertarget{el-gratificante-ejemplar-uxe9xito-de-la-colecciuxf3n-con-las-comunidades-macedonias}{%
\subsection{El gratificante (ejemplar) éxito de la colección con las
comunidades
macedonias}\label{el-gratificante-ejemplar-uxe9xito-de-la-colecciuxf3n-con-las-comunidades-macedonias}}

\hypertarget{section-7}{%
\section{8}\label{section-7}}

\bibleverse{1} Hermanos y hermanas, queremos contarles sobre la gracia
de Dios hacia las iglesias de Macedonia. \footnote{\textbf{8:1} Rom
  15,26} \bibleverse{2} Aunque han sufrido mucha angustia, rebosan de
felicidad; y aunque son muy pobres, también rebosan de generosidad.
\bibleverse{3} Puedo dar testimonio de que dieron todo lo que pudieron
y, de hecho, más que eso. Por decisión propia \bibleverse{4} siguieron
rogando con nosotros para tener parte en este privilegio de participar
en el ministerio al pueblo de Dios. \bibleverse{5} No solo hicieron lo
que esperábamos que hicieran, sino que se entregaron completamente al
Señor y luego a nosotros, como Dios lo quería.

\hypertarget{invitaciuxf3n-a-los-corintios-a-participar-activamente-en-la-colecta}{%
\subsection{Invitación a los corintios a participar activamente en la
colecta}\label{invitaciuxf3n-a-los-corintios-a-participar-activamente-en-la-colecta}}

\bibleverse{6} Así que hemos animado a Tito---ya que él fue quien inició
esta obra con ustedes---para que regrese y termine con ustedes este
ministerio de gracia. \bibleverse{7} Ya que ustedes tienen abundancia en
todas las cosas---confianza en Dios, conocimiento espiritual, total
dedicación, y amor por nosotros--- asegúrense de que esta abundancia que
poseen también llegue a este ministerio de dadivosidad. \footnote{\textbf{8:7}
  1Cor 1,5; 1Cor 16,1-2}

\bibleverse{8} No los estoy obligando a hacer esto, sino a que
demuestren la sinceridad de su amor, comparado con la dedicación de los
otros.\footnote{\textbf{8:8} Se presume que se refiere a las otras
  Iglesias, como las de Macedonia.} \bibleverse{9} Porque ustedes
conocen la gracia de nuestro Señor Jesucristo. Que aunque era rico, se
volvió pobre por ustedes, a fin de que a través de su pobreza ustedes
pudieran llegar a ser ricos. \bibleverse{10} Este es mi consejo: sería
bueno que terminaran lo que comenzaron. El año pasado ustedes fueron no
solo los primeros en dar sino también los primeros en querer hacerlo.
\bibleverse{11} Ahora, terminen los planes que hicieron. Sean prestos
para terminar así como lo fueron para hacer planes, y den según lo que
puedan dar. \bibleverse{12} Si hay disposición, es bueno que den de lo
que tengan, y no lo que no tienen. \footnote{\textbf{8:12} Prov 3,27-28;
  Mar 12,43} \bibleverse{13} El propósito no es hacer que las cosas sean
fáciles para los demás y difíciles para ustedes, sino justas.
\bibleverse{14} En este momento ustedes tienen más que suficiente para
suplir sus necesidades, y a la vez, cuando ellos tengan más que
suficiente podrán satisfacer las necesidades de ustedes. De esta manera
todos reciben un trato justo. \bibleverse{15} Como dice la Escritura:
``El que tenía mucho, no tenía en exceso, y el que no tenía mucho,
tampoco tenía muy poco''.\footnote{\textbf{8:15} Esto hace referencia a
  la recolección del maná, en Éxodo 16:8.}

\hypertarget{recomendaciuxf3n-de-tito-y-los-otros-dos-diputados-de-pablo}{%
\subsection{Recomendación de Tito y los otros dos diputados de
Pablo}\label{recomendaciuxf3n-de-tito-y-los-otros-dos-diputados-de-pablo}}

\bibleverse{16} Gracias a Dios que le dio a Tito la misma devoción que
yo tengo por ustedes. \bibleverse{17} Aunque aceptó hacer lo que le
dijimos, viene a verlos porque realmente desea hacerlo, y porque ya lo
había decidido. \bibleverse{18} También enviamos con él a un hermano que
es elogiado por todas las iglesias por su obra en la predicación de la
buena noticia. \footnote{\textbf{8:18} 2Cor 12,18} \bibleverse{19}
También fue designado por las iglesias para que fuera con nosotros a
entregar esta ofrenda que llevamos con nosotros. Lo hacemos para honrar
al Señor y para mostrar nuestro ferviente deseo de ayudar a
otros.\footnote{\textbf{8:19} Ver 1 Corintios 16:3-4.} \footnote{\textbf{8:19}
  Gal 2,10} \bibleverse{20} Queremos evitar que alguno pueda criticar la
manera como usamos este regalo. \bibleverse{21} Nos interesa hacer las
cosas de manera correcta, no solo a los ojos del Señor, sino también
ante los ojos de todos. \bibleverse{22} También enviamos con ellos a
otro hermano que ha demostrado en muchas ocasiones ser un hombre de
confianza, y que está dispuesto a ayudar. Ahora tiene aún más
disposición de ayudar por la gran confianza que tiene en ustedes.
\bibleverse{23} Si alguno pregunta sobre Tito, digan que es mi
compañero. Trabaja conmigo en favor de ustedes. Los otros hermanos son
representantes de las iglesias y que honran a Cristo. \footnote{\textbf{8:23}
  2Cor 7,13; 2Cor 12,18} \bibleverse{24} Así que les ruego que los
reciban antes que todas las demás iglesias y les muestren su amor,
demostrando así que tenemos razón en estar muy orgullosos de
ustedes.\footnote{\textbf{8:24} 2Cor 7,14}

\hypertarget{lo-que-pablo-ha-elogiado-hasta-ahora-de-los-corintios-y-ahora-espera-y-quuxe9-razones-lo-han-determinado-a-enviar-a-los-hermanos-por-delante}{%
\subsection{Lo que Pablo ha elogiado hasta ahora de los corintios y
ahora espera y qué razones lo han determinado a enviar a los hermanos
por
delante}\label{lo-que-pablo-ha-elogiado-hasta-ahora-de-los-corintios-y-ahora-espera-y-quuxe9-razones-lo-han-determinado-a-enviar-a-los-hermanos-por-delante}}

\hypertarget{section-8}{%
\section{9}\label{section-8}}

\bibleverse{1} Realmente no necesito escribirles sobre esta ofrenda para
el pueblo de Dios. \bibleverse{2} Sé cuán prestos están para ayudar. De
hecho, elogié esto en Macedonia, diciendo que en Acaya ustedes han
estado prestos por más de un año, y que su entusiasmo ha animado a
muchos de ellos a dar. \footnote{\textbf{9:2} 2Cor 8,19} \bibleverse{3}
Pero envío a estos hermanos para que los elogios que hago de ustedes no
sean hallados falsos, y que estén preparados, tal como dijeron que lo
harían. \bibleverse{4} Esto lo digo en caso de que algunos de Macedonia
lleguen conmigo y ustedes no estén listos. Nosotros, -- y sabemos que
ustedes también -- nos sentiríamos muy avergonzados de que este proyecto
fracasara. \bibleverse{5} Por eso decidí pedir a estos hermanos que los
visiten antes, y finalicen los arreglos necesarios para recoger esta
ofrenda, de tal modo que esté lista como un regalo y no como una
obligación.

\hypertarget{otra-invitaciuxf3n-a-participar-activamente-en-la-colecciuxf3n-en-referencia-a-los-efectos-benuxe9ficos-de-la-obra-de-amor}{%
\subsection{Otra invitación a participar activamente en la colección en
referencia a los efectos benéficos de la obra de
amor}\label{otra-invitaciuxf3n-a-participar-activamente-en-la-colecciuxf3n-en-referencia-a-los-efectos-benuxe9ficos-de-la-obra-de-amor}}

\bibleverse{6} Quisiera recordarles esto: Si siembran poco, cosecharán
poco; pero si siembran con abundancia, cosecharán abundancia.
\bibleverse{7} Cada uno debe dar según lo que haya decidido dar, y no de
mala gana o por obligación, porque Dios ama a los que dan con espíritu
alegre.\footnote{\textbf{9:7} Ver Proverbios 22:8.} \footnote{\textbf{9:7}
  Rom 12,8} \bibleverse{8} Dios puede proveerles todo para que nunca les
falte nada; con abundancia, para que ayuden a otros también.
\bibleverse{9} Como dice la Escritura: ``Él da con generosidad a los
pobres. Su generosidad es eterna''.\footnote{\textbf{9:9} Salmos 112:9.
  En el contexto del salmo, se refiere a un hombre generoso.}

\bibleverse{10} Dios, quien provee la semilla para el sembrador y da el
pan para la comida, proveerá y multiplicará su ``semilla'' y aumentará
sus cosechas de generosidad. \bibleverse{11} Serán ricos en todas las
cosas, a fin de que puedan ser siempre generosos y su generosidad lleve
a otros a estar agradecidos con Dios. \bibleverse{12} Cuando sirvan de
esta forma, no solo se satisfacen las necesidades del pueblo de Dios,
sino que muchos darán gracias a él. \bibleverse{13} Al dar esta ofrenda,
demuestran su carácter y los que la reciben agradecerán a Dios por su
obediencia, pues ella demuestra su compromiso con la buena nueva de
Cristo y su generosidad al darles a ellos y a todos los demás.
\bibleverse{14} Entonces ellos orarán por ustedes con más amor, por la
abundante gracia de Dios obrando por medio de ustedes. \bibleverse{15}
¡Gracias a Dios porque su don es más grande que lo que las palabras
pueden expresar!

\hypertarget{en-contraste-con-la-acusaciuxf3n-de-debilidad-de-caruxe1cter-y-cambio-carnal-pablo-seuxf1ala-el-poder-probado-y-comprobado-de-su-trabajo-a-sus-oponentes}{%
\subsection{En contraste con la acusación de debilidad de carácter y
cambio carnal, Pablo señala el poder probado y comprobado de su trabajo
a sus
oponentes}\label{en-contraste-con-la-acusaciuxf3n-de-debilidad-de-caruxe1cter-y-cambio-carnal-pablo-seuxf1ala-el-poder-probado-y-comprobado-de-su-trabajo-a-sus-oponentes}}

\hypertarget{section-9}{%
\section{10}\label{section-9}}

\bibleverse{1} Yo mismo, Pablo, los insto personalmente, por la bondad y
la ternura de Cristo. El mismo Pablo que es ``tímido'' cuando está con
ustedes, pero que es ``osado'' cuando no está allá.\footnote{\textbf{10:1}
  Pablo pareciera estar enfrentando alguna acusación que se había hecho
  contra él.} \bibleverse{2} Les ruego para que la próxima vez que esté
con ustedes, no tenga que ser tan duro como pienso que tendré que ser,
confrontando abiertamente a los que piensan que nosotros nos comportamos
de forma mundana. \footnote{\textbf{10:2} 2Cor 13,1-2; 1Cor 4,21}
\bibleverse{3} Aunque vivimos en este mundo, no peleamos como el mundo.
\bibleverse{4} Nuestras armas no son de este mundo, pero tenemos el
poder de Dios que destruye fortalezas del pensamiento humano, y derriba
teorías engañosas. \bibleverse{5} Todo muro que se interpone contra el
conocimiento de Dios es derribado. Todo pensamiento rebelde es capturado
y conducido a un acuerdo de obediencia a Cristo. \bibleverse{6} Cuando
ustedes estén obedeciendo a Cristo por completo, entonces estaremos
listos para castigar cualquier desobediencia.

\hypertarget{el-derecho-del-apuxf3stol-a-jactarse-en-su-oficio-y-defenderse-de-los-cargos-de-falta-de-valor-personal}{%
\subsection{El derecho del apóstol a jactarse en su oficio y defenderse
de los cargos de falta de valor
personal}\label{el-derecho-del-apuxf3stol-a-jactarse-en-su-oficio-y-defenderse-de-los-cargos-de-falta-de-valor-personal}}

\bibleverse{7} ¡Miren lo que tienen delante de sus ojos! Todo el que
crea que pertenece a Cristo debe pensarlo dos veces, porque así como
ellos pertenecen a Cristo, nosotros también le pertenecemos.
\bibleverse{8} Aunque pareciera que me enorgullezco mucho de nuestra
autoridad, no me avergüenzo de ello. El Señor nos dio esta autoridad
para edificarlos a ustedes, no para destruirlos. \footnote{\textbf{10:8}
  2Cor 13,10; 1Cor 5,4-5} \bibleverse{9} No intento asustarlos con mis
cartas. \bibleverse{10} La gente dice: ``Sus cartas son duras y severas,
pero en persona es débil, y es un orador inútil''. \bibleverse{11} Este
tipo de personas deberían comprender que lo que decimos por cartas
cuando no estamos allá, lo haremos cuando sí estemos allá.

\hypertarget{la-diferencia-entre-la-auto-fama-practicada-correctamente-por-pablo-y-la-presunciuxf3n-de-sus-oponentes}{%
\subsection{La diferencia entre la auto-fama practicada correctamente
por Pablo y la presunción de sus
oponentes}\label{la-diferencia-entre-la-auto-fama-practicada-correctamente-por-pablo-y-la-presunciuxf3n-de-sus-oponentes}}

\bibleverse{12} No somos tan arrogantes como para compararnos con los
que se tienen en un concepto muy alto. ¡Los que se miden a sí mismos, y
se comparan consigo mismos, son totalmente necios! \footnote{\textbf{10:12}
  2Cor 3,1; 2Cor 5,12} \bibleverse{13} Pero no nos jactamos con términos
extravagantes que no puedan medirse. Sencillamente medimos lo que hemos
hecho usando el sistema de medida que Dios nos ha dado, y eso los
incluye a ustedes. \footnote{\textbf{10:13} Rom 12,3; Rom 15,20; Gal 2,7}
\bibleverse{14} No estamos abusando de nuestra autoridad al decir esto,
como si no hubiéramos estado entre ustedes, porque realmente sí
estuvimos allí y compartimos con ustedes la buena noticia de
Cristo.\footnote{\textbf{10:14} Pablo está diciendo que él estaba
  trabajando dentro del marco de su comisión para predicar el evangelio
  cuando vino a Corinto. Puede ser que algunos estaban diciendo que
  Corinto realmente no era parte de la jurisdicción de Pablo.}
\bibleverse{15} Nosotros no nos estamos jactando con términos
extravagantes que no puedan medirse, reclamando crédito por lo que otros
han hecho. Por el contrario, esperamos que a medida que su fe en Dios
aumenta, nuestra obra entre ustedes crezca en gran manera.
\bibleverse{16} Entonces podremos compartir la buena noticia en lugares
que están más allá, sin jactarnos de lo que ya ha sido hecho por
otros.\footnote{\textbf{10:16} Pablo desea evitar problemas en cuento a
  quién recibe crédito por hacer una cosa y otra, y preferiría seguir
  hacia adelante con la obra de la predicación de la buena noticia.}
\bibleverse{17} ``Si alguno quiere jactarse, que se jacte en el
Señor''.\footnote{\textbf{10:17} Citando Jeremías 9:24.} \footnote{\textbf{10:17}
  1Cor 1,31} \bibleverse{18} No reciben respeto los que se elogian a sí
mismos, sino a los que el Señor elogia.\footnote{\textbf{10:18} 1Cor 4,5}

\hypertarget{por-quuxe9-y-con-quuxe9-derecho-se-alaba-a-suxed-mismo-el-apuxf3stol}{%
\subsection{Por qué y con qué derecho se alaba a sí mismo el
apóstol}\label{por-quuxe9-y-con-quuxe9-derecho-se-alaba-a-suxed-mismo-el-apuxf3stol}}

\hypertarget{section-10}{%
\section{11}\label{section-10}}

\bibleverse{1} Espero que puedan soportarme unas cuantas necedades más.
¡Bueno, de hecho, ya me soportan a mí mismo! \bibleverse{2} Sufro de una
agonía por el celo divino que siento por ustedes, pues les prometí un
solo esposo---Cristo---a fin de presentarlos a ustedes como una mujer
virgen y pura para él. \footnote{\textbf{11:2} Efes 5,26-27}
\bibleverse{3} Me preocupa que, de algún modo, así como la serpiente
engañó a Eva con su astucia, ustedes puedan ser descarriados en su forma
de pensar sobre su compromiso sincero y puro con Cristo. \footnote{\textbf{11:3}
  Gén 3,4; Gén 3,13} \bibleverse{4} Si alguno llega a hablarles sobre un
Jesús distinto al que nosotros hemos compartido con ustedes, fácilmente
ustedes concuerdan con ellos,\footnote{\textbf{11:4} En otras palabras,
  son muy tolerantes con los que traen una comprensión muy distinta de
  la buena noticia.} aceptando un espíritu diferente al que han
recibido, y una buena noticia distinta a la que creyeron. \footnote{\textbf{11:4}
  Gal 1,8-9} \bibleverse{5} No me considero inferior a estos ``súper
apóstoles''. \footnote{\textbf{11:5} 2Cor 12,11; 1Cor 15,10; Gal 2,6;
  Gal 2,9} \bibleverse{6} Aunque no sea muy talentoso para dar
discursos, sé de lo que hablo. Les hemos explicado esto claramente y de
todas las maneras posibles. \footnote{\textbf{11:6} 1Cor 2,1-2; 1Cor
  2,13; Efes 3,4}

\hypertarget{la-gloria-de-su-eficacia-desinteresada-gratuita-en-contraste-con-los-oponentes-que-trabajan-al-servicio-de-satanuxe1s}{%
\subsection{La gloria de su eficacia desinteresada (gratuita) en
contraste con los oponentes que trabajan al servicio de
Satanás}\label{la-gloria-de-su-eficacia-desinteresada-gratuita-en-contraste-con-los-oponentes-que-trabajan-al-servicio-de-satanuxe1s}}

\bibleverse{7} ¿Fue un error que me humillara para exaltarlos a ustedes,
siendo que compartí la buena noticia con ustedes sin beneficio económico
alguno? \footnote{\textbf{11:7} 2Cor 12,13; 1Cor 9,12-18; Mat 10,8}
\bibleverse{8} Despojé a otras iglesias, recibiendo pago de ellas para
poder trabajar en favor de ustedes. \footnote{\textbf{11:8} Fil 4,10;
  Fil 4,15} \bibleverse{9} Cuando estuve allá con ustedes y necesité
algo, no fui carga para nadie, porque los creyentes que venían de
Macedonia se hicieron cargo de mis necesidades. Estuve decidido a no ser
carga para ustedes y nunca lo seré. \bibleverse{10} Esto es tan cierto
como la verdad de que Cristo está en mí: ¡No hay nadie en toda Acaya que
me impida jactarme de esto! \bibleverse{11} ¿Y por qué? ¿Acaso es porque
no los amo? ¡Dios mismo sabe que sí los amo!

\bibleverse{12} Y seguiré haciendo lo que siempre he hecho, para
eliminar cualquier oportunidad que otros puedan tener de jactarse de que
su obra es igual a la nuestra. \bibleverse{13} Estas personas son falsos
apóstoles, obreros deshonestos, que fingen\footnote{\textbf{11:13}
  Literalmente, ``se transforman en''. También aparece en el versículo
  14.} ser apóstoles de Cristo. \bibleverse{14} No se sorprendan de esto
porque incluso Satanás mismo finge ser un ángel de luz. \bibleverse{15}
Así que no se extrañen de que los que le sirven finjan ser agentes del
bien. Pero su final será conforme a sus obras.

\hypertarget{otra-peticiuxf3n-del-apuxf3stol-por-su-tonta-fama-propia}{%
\subsection{Otra petición del apóstol por su tonta fama
propia}\label{otra-peticiuxf3n-del-apuxf3stol-por-su-tonta-fama-propia}}

\bibleverse{16} Permítanme decirlo nuevamente: por favor, no crean que
estoy siendo necio. No obstante, si así lo creen, acéptenme como un
necio, y permítanme jactarme un poco.\footnote{\textbf{11:16} Pablo
  sugiere que a él también debería permitírsele jactarse como lo hacían
  los falsos apóstoles.} \footnote{\textbf{11:16} 2Cor 12,6}
\bibleverse{17} Lo que estoy diciendo no es como lo diría el Señor, con
todo este orgullo. \bibleverse{18} Pero como muchos andan por ahí
jactándose como lo hace el mundo, entonces permítanme hacerlo también.
\bibleverse{19} (Ustedes son felices de soportar necios, pues son muy
sabios\footnote{\textbf{11:19} Evidentemente, es un comentario
  sarcástico o irónico, así como lo que sigue al versículo \ldots{}} )
\bibleverse{20} Soportan a personas que los esclavizan, que les roban,
que los explotan, que los humillan con su arrogancia, y que los
abofetean.

\hypertarget{el-apuxf3stol-se-jacta-de-su-ascendencia-de-su-oficio-de-la-plenitud-de-sus-sufrimientos-en-el-servicio-apostuxf3lico}{%
\subsection{El apóstol se jacta de su ascendencia, de su oficio, de la
plenitud de sus sufrimientos en el servicio
apostólico}\label{el-apuxf3stol-se-jacta-de-su-ascendencia-de-su-oficio-de-la-plenitud-de-sus-sufrimientos-en-el-servicio-apostuxf3lico}}

\bibleverse{21} ¡Lamento tanto que nosotros fuimos muy débiles para
soportar algo así! Pero sean cuales sean las razones por las cuales la
gente se jacta, me atrevo a hacerlo también. (En esto hablo como necio
una vez más). \bibleverse{22} ¿Es porque son hebreos? Yo también. ¿Es
porque son israelitas? Yo también. ¿Es porque son descendientes de
Abrahán? Yo también lo soy. \footnote{\textbf{11:22} Fil 3,5}
\bibleverse{23} ¿Es porque son siervos de Cristo? (Esto podría sonar
como una locura). Pero yo he hecho mucho más. He trabajado con más
esfuerzo, me han llevado preso en muchas más ocasiones, me han azotado
más veces de las que puedo contar, he enfrentado la muerte una y otra
vez. \footnote{\textbf{11:23} 2Cor 6,4-5; 1Cor 15,10} \bibleverse{24}
Cinco veces he recibido de los judíos cuarenta latigazos menos uno.
\footnote{\textbf{11:24} Deut 25,3} \bibleverse{25} Tres veces fui
golpeado con palos, una vez fui apedreado, tres veces naufragué. Una vez
duré veinticuatro horas a la deriva en el océano. \footnote{\textbf{11:25}
  Hech 16,22; Hech 14,19} \bibleverse{26} Durante muchas ocasiones he
afrontado los peligros de cruzar ríos, encontrarme con pandillas de
atracadores, ataques de mis propios conciudadanos, así como de
extranjeros.\footnote{\textbf{11:26} Literalmente, ``gentiles''.} He
enfrentado peligros en las ciudades, en los desiertos, y en el mar. He
enfrentado el peligro de parte de personas que fingen ser cristianos.
\bibleverse{27} He enfrentado trabajo duro y luchas, muchas noches sin
dormir, hambre y sed, a menudo he estado sin comida, con frío, y sin
ropa para cubrirme del frío. \footnote{\textbf{11:27} 2Cor 6,5}

\bibleverse{28} Aparte de todo esto, cada día enfrento las
preocupaciones de ocuparme de todas las iglesias. \footnote{\textbf{11:28}
  Hech 20,18-21; Hech 20,31} \bibleverse{29} ¿Quién es débil? ¿Acaso no
me siento débil también? ¿Quién es conducido a pecar sin que yo arda de
enojo?

\bibleverse{30} Si tengo que jactarme, me jactaré en lo débil que soy.
\footnote{\textbf{11:30} 2Cor 12,5}

\bibleverse{31} El Dios y Padre del Señor Jesús---sea él alabado por
siempre---sabe que no miento. \bibleverse{32} Mientras estaba en
Damasco, el gobernador que estaba bajo autoridad del Rey Aretas mandó a
custodiar la ciudad para capturarme. \bibleverse{33} Pero me ayudaron a
descender en una canasta por el muro de la ciudad, y hui de él.

\hypertarget{el-apuxf3stol-se-jacta-de-las-muxe1s-altas-gracias-a-travuxe9s-de-revelaciones-celestiales-y-la-muxe1s-profunda-humillaciuxf3n-a-travuxe9s-del-sufrimiento-fuxedsico}{%
\subsection{El apóstol se jacta de las más altas gracias (a través de
revelaciones celestiales) y la más profunda humillación (a través del
sufrimiento
físico)}\label{el-apuxf3stol-se-jacta-de-las-muxe1s-altas-gracias-a-travuxe9s-de-revelaciones-celestiales-y-la-muxe1s-profunda-humillaciuxf3n-a-travuxe9s-del-sufrimiento-fuxedsico}}

\hypertarget{section-11}{%
\section{12}\label{section-11}}

\bibleverse{1} Supongo que tengo que jactarme, aunque eso no ayuda
realmente. Permítanme hablarles ahora de las visiones y revelaciones de
parte del Señor. \bibleverse{2} Conozco a un hombre en Cristo que hace
catorce años fue llevado al tercer cielo (si fue físicamente con su
cuerpo, o si fue fuera del cuerpo, no lo sé, pero Dios sabe).
\bibleverse{3} Sé que este hombre (si fue físicamente con su cuerpo, o
fuera de él, no lo sé, pero Dios lo sabe), \bibleverse{4} fue llevado al
Paraíso, y escuchó cosas tan maravillosas que no se pueden explicar, en
palabras tan sagradas que ningún ser humano podría decir. \bibleverse{5}
De algo como eso me jactaría, pero no me jactaré de mí mismo, sino de
mis debilidades. \footnote{\textbf{12:5} 2Cor 11,30} \bibleverse{6} No
sería un necio si quisiera jactarme, porque estaría diciendo la verdad.
Pero no me jactaré, para que nadie me tenga en un concepto más alto que
lo que ve que hago o me oyen decir. \footnote{\textbf{12:6} 2Cor 10,8}
\bibleverse{7} Además, como las revelaciones fueron tan asombrosas, y
para que no pudiera enorgullecerme de ello, se me dio una ``espina en la
carne''\footnote{\textbf{12:7} Probablemente se refiere a algún problema
  físico en el cuerpo de Pablo.} ---un mensajero de Satanás, para
herirme a fin de que no me volviera orgulloso. \bibleverse{8} Le rogué
al Señor tres veces para deshacerme de este problema. \bibleverse{9}
Pero él me dijo: ``Mi gracia te bastará, pues mi poder se hace eficaz en
la debilidad''. Por eso me jacto felizmente de mis debilidades, para que
habite en mí el poder de Cristo.

\bibleverse{10} Por lo tanto valoro las debilidades, los insultos, los
problemas, las persecuciones y las dificultades que sufro por causa de
Cristo. ¡Porque cuando soy débil, entonces soy fuerte! \footnote{\textbf{12:10}
  Fil 4,13}

\hypertarget{referencia-a-la-injusticia-de-los-corintios}{%
\subsection{Referencia a la injusticia de los
corintios}\label{referencia-a-la-injusticia-de-los-corintios}}

\bibleverse{11} Estoy hablando como necio, pero ustedes me obligaron a
hacerlo. Ustedes deberían haber estado hablando bien de mí, pues de
ninguna manera soy inferior a estos ``súper apóstoles'',\footnote{\textbf{12:11}
  Ver 11:5.} aunque no soy nada. \footnote{\textbf{12:11} 2Cor 11,5}
\bibleverse{12} Sin embargo, las señales de apostolado fueron
presentadas pacientemente ante ustedes: señales, maravillas, y milagros
poderosos. \footnote{\textbf{12:12} Rom 15,19; Heb 2,4} \bibleverse{13}
¿Acaso en qué fueron ustedes inferiores a las demás iglesias, sino en el
hecho de que no fui una carga para ustedes? ¡Les ruego que me perdonen
por hacerles mal!\footnote{\textbf{12:13} Otra vez, una afirmación que
  debería considerarse como irónica; tal como en el versículo 16.}
\footnote{\textbf{12:13} 2Cor 11,7-9}

\hypertarget{anuncio-de-la-inminente-llegada-del-apuxf3stol-rechazo-de-un-libelo}{%
\subsection{Anuncio de la inminente llegada del apóstol; Rechazo de un
libelo}\label{anuncio-de-la-inminente-llegada-del-apuxf3stol-rechazo-de-un-libelo}}

\bibleverse{14} Estoy preparándome para visitarlos por tercera vez y no
seré carga para ustedes. ¡No quiero las cosas que tienen, los quiero a
ustedes! Después de todo, los niños no deben cuidar de los padres, sino
los padres de los hijos. \bibleverse{15} Gustosamente me gastaré y me
desgastaré por ustedes. Si los amo mucho más, ¿acaso me amarán menos
ustedes? \footnote{\textbf{12:15} Fil 2,17} \bibleverse{16} Pues,
incluso si es así, no fui carga para ustedes. ¡Quizás estaba siendo
taimado y los engañé con mis estrategias astutas! \bibleverse{17} ¿Pero
acaso me aproveché de ustedes mediante alguno de los que envié?
\bibleverse{18} Obligué a Tito para que fuera a verlos, y envié a otro
hermano con él. ¿Acaso Tito se aprovechó de ustedes? No, porque ambos
tenemos el mismo espíritu y usamos los mismos métodos.

\hypertarget{rectificaciuxf3n-de-una-opiniuxf3n-de-los-corintios-miedo-del-apuxf3stol-por-el-estatus-moral-de-la-comunidad}{%
\subsection{Rectificación de una opinión de los corintios; Miedo del
apóstol por el estatus moral de la
comunidad}\label{rectificaciuxf3n-de-una-opiniuxf3n-de-los-corintios-miedo-del-apuxf3stol-por-el-estatus-moral-de-la-comunidad}}

\bibleverse{19} Quizás ustedes están pensando que todo este tiempo hemos
estado tratando de defendernos a nosotros mismos. No, hablamos de Cristo
ante Dios. Todo lo que hacemos, amigos, es por beneficio de ustedes.
\bibleverse{20} Cuando voy de visita, me preocupo de no encontrarlos
como quisiera, y de que ustedes no me vean como quisieran verme. Me temo
que habrá discusiones, celos, enojo, calumnia, chisme, arrogancia, y
desorden. \footnote{\textbf{12:20} 2Cor 10,2} \bibleverse{21} Me temo
que cuando vaya de visita, mi Dios me humillará en presencia de ustedes,
y que estaré lamentándome por muchos que han pecado antes, y que aún no
se han arrepentido de impureza, inmoralidad sexual, y los actos
indecentes que cometieron.\footnote{\textbf{12:21} 2Cor 2,1; 2Cor 13,2}

\hypertarget{anuncio-de-juicio-imparcial-y-juicio-despiadado}{%
\subsection{Anuncio de juicio imparcial y juicio
despiadado}\label{anuncio-de-juicio-imparcial-y-juicio-despiadado}}

\hypertarget{section-12}{%
\section{13}\label{section-12}}

\bibleverse{1} Esta es mi tercera visita. ``Todo cargo debe ser
verificado por dos o tres testigos''.\footnote{\textbf{13:1} Citando
  Deuteronomio 19:15.} \footnote{\textbf{13:1} 2Cor 10,2; Mat 18,16}
\bibleverse{2} Ya advertí a los que entre ustedes estaban en pecado
cuando fui por segunda vez. Aunque no estoy allí, les advierto a ellos
una vez más---y al resto de ustedes---que cuando los visite no dudaré en
tomar medidas contra ellos, \bibleverse{3} puesto que están demandando
una prueba de que Dios está hablando a través de mí. Él no es débil para
tratarlos; más bien obra con poder en medio de ustedes. \bibleverse{4}
Aunque fue crucificado en debilidad, ahora vive mediante el poder de
Dios. Nosotros también somos débiles en él, pero ustedes podrán ver que
vivimos con él mediante el poder de Dios.

\bibleverse{5} Examínense ustedes mismos y vean si están confiando en
Dios. Pónganse a prueba. ¿No se dan cuenta de que Jesucristo está
en\footnote{\textbf{13:5} O ``unido a''.} ustedes? A menos que hayan
fallado en la prueba\ldots{} \bibleverse{6} No obstante, espero que
comprendan que nosotros no hemos fallado.

\bibleverse{7} Rogamos a Dios que ustedes no hagan nada malo, no para
que nosotros podamos mostrar que pasamos la prueba, sino para que
ustedes puedan hacer lo recto, aunque nos haga parecer como un fracaso.
\bibleverse{8} No podemos hacer nada contra la verdad, solo en favor de
la verdad. \bibleverse{9} Nos alegra cuando somos débiles, y ustedes son
fuertes. Oramos para que sigan mejorando. \bibleverse{10} Por eso les
escribo sobre esto ahora que no estoy con ustedes, para que cuando sí
esté allá, no tenga necesidad de tratarlos con dureza e imponiendo mi
autoridad. El Señor me dio autoridad para edificar, no para destruir.

\hypertarget{advertencias-finales-saludos-y-bendiciones}{%
\subsection{Advertencias finales, saludos y
bendiciones}\label{advertencias-finales-saludos-y-bendiciones}}

\bibleverse{11} Finalmente, hermanos y hermanas, me despido. Sigan
mejorando espiritualmente. Anímense unos a otros. Estén en armonía.
Vivan en paz, y que el Dios de amor y paz esté con ustedes. \footnote{\textbf{13:11}
  Rom 15,33; Fil 4,4} \bibleverse{12} Salúdense unos a otros con amor
cristiano.

\bibleverse{13} Todos los creyentes aquí les envían su saludo.
\bibleverse{14} Que la gracia del Señor Jesucristo, el amor de Dios, y
la comunión del Espíritu Santo esté con todos ustedes.
