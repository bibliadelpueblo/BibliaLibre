\hypertarget{la-fiesta-del-rey-persa-assuero-en-susa-para-los-dignatarios-y-altos-funcionarios-de-su-imperio}{%
\subsection{La fiesta del rey persa Assuero en Susa para los dignatarios
y altos funcionarios de su
imperio}\label{la-fiesta-del-rey-persa-assuero-en-susa-para-los-dignatarios-y-altos-funcionarios-de-su-imperio}}

\hypertarget{section}{%
\section{1}\label{section}}

\bibleverse{1} Este es un relato de lo que sucedió durante la época del
rey Jerjes, el Jerjes\footnote{\textbf{1:1} Está claro que el escritor
  era consciente de que había más de un ``Jerjes''.} que gobernaba 127
provincias desde la India hasta Etiopía. \bibleverse{2} En ese momento
el rey Jerjes gobernaba desde su trono real en la fortaleza de
Susa.\footnote{\textbf{1:2} Esto es significativo, ya que el rey tenía
  palacios de verano y de invierno. Esta era su residencia de invierno.}
\bibleverse{3} En el tercer año de su reinado organizó una fiesta para
sus funcionarios y administradores. Los comandantes del ejército de
Persia y Media, los nobles y los funcionarios provinciales estaban allí
con él. \bibleverse{4} Durante ciento ochenta días exhibió sus riquezas
y la gloria de su reino, mostrando lo majestuoso, espléndido y glorioso
que era.

\hypertarget{la-comida-de-los-habitantes-de-la-ciudad-real-susa-la-fiesta-de-la-reina-wasthi}{%
\subsection{La comida de los habitantes de la ciudad real Susa; la
fiesta de la reina
Wasthi}\label{la-comida-de-los-habitantes-de-la-ciudad-real-susa-la-fiesta-de-la-reina-wasthi}}

\bibleverse{5} Después de eso, el rey dio un banquete que duró siete
días para todo el pueblo, grande y pequeño, que estaba allí en la
fortaleza de Susa, en el patio del jardín del pabellón del rey.
\bibleverse{6} Estaba decorado con cortinas de algodón blanco y azul
atadas con cordones de lino fino e hilo de púrpura sobre anillos de
plata, sostenidos por pilares de mármol. Sobre un pavimento de pórfido
púrpura, mármol, nácar y piedras costosas se colocaron sofás de oro y
plata. \bibleverse{7} Las bebidas se servían en copas de oro de
diferentes tipos, y el vino real fluía libremente debido a la
generosidad del rey. \bibleverse{8} El rey había ordenado que no se
limitara la cantidad de bebida de los invitados; había dicho a sus
servidores que dieran a cada uno lo que quisiera.

\bibleverse{9} La reina Vasti también preparó un banquete para las
mujeres del palacio que pertenecía al rey Jerjes.

\hypertarget{wasthi-se-niega-a-aparecer-en-el-saluxf3n-de-baile}{%
\subsection{Wasthi se niega a aparecer en el salón de
baile}\label{wasthi-se-niega-a-aparecer-en-el-saluxf3n-de-baile}}

\bibleverse{10} El séptimo día del banquete, el rey, sintiéndose feliz
por haber bebido vino, ordenó a los siete eunucos que eran sus
asistentes, Mehumán, Bizta, Harbona, Bigta, Abagta, Zetar y Carcas,
\bibleverse{11} que le trajeran a la reina Vasti con su tocado
real,\footnote{\textbf{1:11} La palabra aquí sólo se utiliza en Ester y
  se refiere al tocado real persa, no a lo que normalmente se considera
  una corona. Sin embargo, servía para el mismo propósito que una
  corona, ya que era usada por la realeza. La palabra es probablemente
  un préstamo de la lengua persa.} para que pudiera mostrar su belleza
al pueblo y a los funcionarios, pues era muy atractiva. \bibleverse{12}
Pero cuando los eunucos le entregaron la orden del rey, la reina Vastise
negó a venir. El rey se enfadó muchísimo; estaba absolutamente furioso.

\hypertarget{asesoramiento-y-toma-de-decisiones-sobre-quuxe9-castigo-anuncio-del-repudio-en-todo-el-imperio}{%
\subsection{Asesoramiento y toma de decisiones sobre qué castigo;
Anuncio del repudio en todo el
imperio}\label{asesoramiento-y-toma-de-decisiones-sobre-quuxe9-castigo-anuncio-del-repudio-en-todo-el-imperio}}

\bibleverse{13} Entonces el rey habló con los sabios que sabrían qué
hacer, pues era costumbre que pidiera la opinión de expertos en
procedimientos y asuntos legales. \footnote{\textbf{1:13} 1Cró 12,32}
\bibleverse{14} Los más cercanos a él eran Carsena, Setar, Admata,
Tarsis, Meres, Marsena, y Memucán. Eran los siete nobles de Persia y
Media que se reunían frecuentemente con el rey y ocupaban los más altos
cargos del reino. \bibleverse{15} ``¿Qué dice la ley que debe hacerse
con la reina Vasti?'' , preguntó. ``¡Ella se negó a obedecer la orden
directa del rey Jerjes dictada por los eunucos!''

\bibleverse{16} Memucán dio su respuesta ante el rey y los nobles: ``La
reina Vasti no sólo ha insultado al rey, sino a todos los nobles y a
todo el pueblo de todas las provincias del rey Jerjes. \bibleverse{17}
Cuando se sepa lo que ha hecho la reina, todas las esposas
menospreciarán a sus maridos, los mirarán con desprecio y les dirán:
`¡El rey Jerjes ordenó que le trajeran a la reina Vasti, pero no vino!'.
\bibleverse{18} ¡Al final del día, las esposas de todos los nobles de
toda Persia y de Media que hayan oído lo que hizo la reina, tratarán a
sus nobles maridos con airado desprecio!

\bibleverse{19} ``Si le place a Su Majestad, emita un decreto real, de
acuerdo con las leyes de Persia y de Media que no pueden ser cambiadas,
para que Vasti sea desterrada de la presencia del rey Jerjes, y para que
Su Majestad le dé su posición real a otra, una que sea mejor que ella.
\bibleverse{20} Cuando el decreto de Su Majestad sea proclamado en todo
su vasto imperio, todas las esposas respetarán a sus maridos, sean de
alta o baja cuna''.

\bibleverse{21} Este consejo les pareció bien al rey y a los nobles, así
que el rey hizo lo que Memucán había dicho. \bibleverse{22} Envió cartas
a todas las provincias del imperio, en la escritura y la lengua de cada
una de ellas, para que cada hombre gobernara su propia casa y utilizara
su propia lengua materna.\textsuperscript{{[}\textbf{1:22} El
significado de esta última frase es incierto.{]}}{[}\textbf{1:22} Est
3,12; Est 8,9; Gén 3,16{]}

\hypertarget{organizaciuxf3n-de-un-gran-espectuxe1culo-nupcial-para-el-rey}{%
\subsection{Organización de un gran espectáculo nupcial para el
rey}\label{organizaciuxf3n-de-un-gran-espectuxe1culo-nupcial-para-el-rey}}

\hypertarget{section-1}{%
\section{2}\label{section-1}}

\bibleverse{1} Más tarde, después de todo lo ocurrido, la ira del rey
Jerjes se calmó y pensó en Vasti y en lo que había hecho, y en el
decreto emitido contra ella. \bibleverse{2} Sus consejeros le
sugirieron,\footnote{\textbf{2:2} Esta sugerencia puede haber sido más
  para ellos mismos, ya que si el rey traía a Vasti de vuelta, sus vidas
  podrían haber estado en peligro como los arquitectos de su caída.}
``¿Por qué no ordenar una búsqueda para encontrar hermosas jóvenes
vírgenes para Su Majestad? \bibleverse{3} Su Majestad debería poner
oficiales a cargo en cada provincia de su imperio para reunir a todas
las jóvenes hermosas y llevarlas al harén del rey en la fortaleza de
Susa. Que las pongan bajo la supervisión de Hegai, el eunuco del rey
encargado de las mujeres, y que les hagan tratamientos de belleza.
\bibleverse{4} La joven que el rey encuentre más atractiva puede
convertirse en reina en lugar de Vasti''. Al rey le pareció una buena
idea y la puso en práctica.

\hypertarget{informaciuxf3n-sobre-la-prehistoria-de-esther}{%
\subsection{Información sobre la prehistoria de
Esther}\label{informaciuxf3n-sobre-la-prehistoria-de-esther}}

\bibleverse{5} En la fortaleza de Susa vivía un judío llamado Mardoqueo,
hijo de Jair, hijo de Simei, hijo de Cis, un benjamita \bibleverse{6}
que estaba entre los que fueron tomados como prisioneros con el rey
Joaquín de Judá y llevados al exilio desde Jerusalén por el rey
Nabucodonosor de Babilonia. \footnote{\textbf{2:6} 2Re 24,15-16}
\bibleverse{7} Él había criado a Hadasa (o Ester),\footnote{\textbf{2:7}
  Hadasa era su nombre hebreo, Ester su nombre persa.} la hija de su
tío, porque ella no tenía padre ni madre. La joven tenía una hermosa
figura y era muy atractiva. Después de la muerte de su padre y de su
madre, Mardoqueo la había adoptado como su propia hija. \footnote{\textbf{2:7}
  Est 2,15}

\hypertarget{el-auxf1o-de-preparaciuxf3n-de-ester-en-el-palacio-real-y-su-elevaciuxf3n-a-reina}{%
\subsection{El año de preparación de Ester en el palacio real y su
elevación a
reina}\label{el-auxf1o-de-preparaciuxf3n-de-ester-en-el-palacio-real-y-su-elevaciuxf3n-a-reina}}

\bibleverse{8} Cuando se anunció la orden y el decreto del rey, muchas
jóvenes fueron llevadas a la fortaleza de Susa bajo la supervisión de
Hegai. Ester también fue llevada al palacio del rey y puesta bajo el
cuidado de Hegai, quien estaba a cargo de las mujeres. \bibleverse{9}
Ester llamó su atención y la trató favorablemente. Rápidamente le
preparó tratamientos de belleza y comida especial. También le
proporcionó siete sirvientas especialmente elegidas del palacio del rey,
y la trasladó a ella y a sus sirvientas al mejor lugar del harén.
\bibleverse{10} Ester no había dejado que nadie supiera su nacionalidad
o quién era su familia, porque Mardoqueo le había ordenado que no lo
hiciera. \bibleverse{11} Todos los días Mardoqueo se paseaba frente al
patio del harén para saber cómo estaba Ester y qué le ocurría.

\bibleverse{12} Antes de que le llegara el turno a la joven de ir a ver
al rey Jerjes, tenía que cumplir doce meses de tratamientos de belleza
para mujeres que eran obligatorios: seis meses con aceite de mirra y
seis con aceites y ungüentos perfumados. \bibleverse{13} Cuando llegaba
el momento de que la joven fuera a ver al rey, se le daba lo
que\footnote{\textbf{2:13} Probablemente se refiera a la ropa y las
  joyas.} ella pidiera para ir del harén al palacio del rey.
\bibleverse{14} Al anochecer iba, y por la mañana volvía a otro harén
bajo la supervisión de Saasgaz, que era el eunuco del rey encargado de
las concubinas. No volvería a estar con el rey a menos que éste se
sintiera especialmente atraído por ella y la llamara por su nombre.

\bibleverse{15} (Ester era hija de Abihail, tío de Mardoqueo. Mardoqueo
la había adoptado como su propia hija). Cuando le tocó a Ester ir a ver
al rey, no pidió nada para llevar, excepto lo que le aconsejó Hegai. (Él
era el eunuco del rey encargado de las mujeres). Y Ester fue vista con
admiración por todos.

\bibleverse{16} Entonces Ester fue llevada ante el rey Jerjes a su
palacio real, en el décimo mes, el mes de Tebet, en el séptimo año de su
reinado. \bibleverse{17} El rey amó a Ester más que a todas las demás
mujeres. La trató más favorablemente y con mayor bondad que a todas las
demás vírgenes. Así que colocó la corona real sobre su cabeza y la
nombró reina en lugar de Vasti.

\bibleverse{18} Entonces el rey dio un gran banquete a todos sus
funcionarios y administradores: el banquete de Ester.\footnote{\textbf{2:18}
  En la Septuaginta se identifica como una fiesta de bodas.} También lo
declaró festivo en todas las provincias y repartió generosos regalos.

\hypertarget{mardochai-descubre-una-conspiraciuxf3n-contra-el-rey-su-muxe9rito-estuxe1-registrado-en-las-cruxf3nicas-del-reino}{%
\subsection{Mardochai descubre una conspiración contra el rey; su mérito
está registrado en las crónicas del
reino}\label{mardochai-descubre-una-conspiraciuxf3n-contra-el-rey-su-muxe9rito-estuxe1-registrado-en-las-cruxf3nicas-del-reino}}

\bibleverse{19} Aunque hubo una segunda reunión de vírgenes,\footnote{\textbf{2:19}
  Se han dado varias interpretaciones a esta frase. Sin embargo, 2:3
  registra el decreto de ``reunir a las vírgenes'' y esto podría ser una
  segunda fase de este proceso. Obsérvese también que no hay artículo
  definido antes de las vírgenes en este versículo, por lo que
  probablemente no se referiría al grupo existente.} y Mardoqueo había
recibido un puesto del rey,\footnote{\textbf{2:19} ``Recibido un puesto
  del rey'': literalmente, ``sentado a la puerta del rey''. También en
  el versículo 21 y posteriormente.} \bibleverse{20} Ester seguía sin
dejar que nadie supiera de su familia o de su nacionalidad, como le
había ordenado Mardoqueo. Siguió las instrucciones de Mardoqueo tal como
lo hizo cuando la educó. \footnote{\textbf{2:20} Est 2,10}

\bibleverse{21} En ese momento, mientras Mardoqueo hacía su trabajo en
la puerta del palacio, Bigtán y Teres, dos eunucos que custodiaban la
entrada a las habitaciones del rey, se enfurecieron con el rey Jerjes y
buscaron la manera de asesinarlo. \bibleverse{22} Mardoqueo se enteró
del complot y se lo comunicó a la reina Ester. Ester, a su vez, se lo
comunicó al rey en nombre de Mardoqueo. \bibleverse{23} Cuando se
investigó el complot y se comprobó que era cierto, ambos fueron
empalados en postes.\footnote{\textbf{2:23} La ejecución por
  empalamiento era el método habitual, no por ahorcamiento con lazo.}
Esto fue registrado en el libro oficial de registros por orden del rey.

\hypertarget{promociuxf3n-de-amuxe1n-al-muxe1s-alto-honor-mardochai-se-niega-a-doblar-las-rodillas-amuxe1n-decide-exterminar-a-todos-los-juduxedos}{%
\subsection{Promoción de Amán al más alto honor; Mardochai se niega a
doblar las rodillas; Amán decide exterminar a todos los
judíos}\label{promociuxf3n-de-amuxe1n-al-muxe1s-alto-honor-mardochai-se-niega-a-doblar-las-rodillas-amuxe1n-decide-exterminar-a-todos-los-juduxedos}}

\hypertarget{section-2}{%
\section{3}\label{section-2}}

\bibleverse{1} Algún tiempo después de esto, el rey Jerjes honró a Amán,
hijo de Hamedata, el agagueo, dándole un puesto más alto que el de todos
sus compañeros. \bibleverse{2} Todos los funcionarios de la realeza se
inclinaban y le mostraban respeto a Amán, porque así lo había ordenado
el rey. Pero Mardoqueo no quería inclinarse ni mostrarle respeto.
\bibleverse{3} Los funcionarios del rey le preguntaban a Mardoqueo:
``¿Por qué desobedeces la orden del rey?'' \bibleverse{4} Le hablaban de
ello día tras día, pero él se negaba a escuchar. Así que se lo contaron
a Amán para ver si aguantaba lo que Mardoqueo estaba
haciendo,\footnote{\textbf{3:4} ``Si podía soportar lo que hacía
  Mardoqueo'': Alternativamente, ``si Mardoqueo continuara con lo que
  estaba haciendo''.} pues Mardoqueo les había dicho que era judío.
\bibleverse{5} Amán se puso furioso cuando vio que Mardoqueo no se
inclinaba ni le mostraba respeto. \bibleverse{6} Al saber quiénes eran
los de Mardoqueo, descartó la idea de matar sólo a Mardoqueo. Decidió
matar a todos los judíos de todo el imperio persa, ¡a todo el pueblo de
Mardoqueo!

\hypertarget{amuxe1n-hace-cumplir-su-resoluciuxf3n-con-el-rey}{%
\subsection{Amán hace cumplir su resolución con el
rey}\label{amuxe1n-hace-cumplir-su-resoluciuxf3n-con-el-rey}}

\bibleverse{7} En el duodécimo año del rey Jerjes, en el primer mes, el
mes de Nisán, se echó ``pur'' (que significa ``suerte'') en presencia de
Amán para elegir un día y un mes,\footnote{\textbf{3:7} Echar suertes
  era una forma antigua de determinar el momento más ``favorable'' para
  una acción concreta, en este caso el plan de Amán para destruir a los
  judíos.} tomando cada día y cada mes de uno en uno. La suerte cayó en
el duodécimo mes, el mes de Adar. \footnote{\textbf{3:7} Est 9,24}
\bibleverse{8} Amán fue a ver al rey Jerjes y le dijo: ``Hay un pueblo
particular que vive entre otros en muchos lugares diferentes de las
provincias de tu imperio y que se separa de todos los demás. Tienen sus
propias leyes, que son diferentes a las de cualquier otro pueblo, y
además no obedecen las leyes del rey. Así que no es buena idea que Su
Majestad los ignore. \bibleverse{9} ``Si le place a Su Majestad, emita
un decreto para destruirlos, y yo personalmente contribuiré con 10. 000
talentos de plata a los que llevan a cabo los negocios del rey para que
sean depositados en el tesoro real''.

\bibleverse{10} El rey se quitó su anillo de sello y lo
entregó\footnote{\textbf{3:10} Una señal de que el rey aceptó la
  propuesta.} a Amán, hijo de Hamedata, el agagueo, enemigo de los
judíos. \bibleverse{11} El rey le dijo a Amán: ``Puedes quedarte con el
dinero y hacer con el pueblo lo que quieras''.

\hypertarget{el-exterminio-de-los-juduxedos-en-todo-el-imperio-ordenado-por-el-rey}{%
\subsection{El exterminio de los judíos en todo el imperio ordenado por
el
rey}\label{el-exterminio-de-los-juduxedos-en-todo-el-imperio-ordenado-por-el-rey}}

\bibleverse{12} El día trece del primer mes fueron convocados los
secretarios del rey. Se emitió un decreto de acuerdo con todo lo que
Amán exigía y se envió a los principales funcionarios del
rey,\footnote{\textbf{3:12} ``Oficialesprincipales'': literalmente,
  ``Sátrapas''.} a los gobernadores de las distintas provincias y a los
nobles de los distintos pueblos de las provincias. Se envió en la
escritura de cada provincia y en la lengua de cada pueblo, con la
autorización del rey Jerjes y sellada con su anillo de sello.
\footnote{\textbf{3:12} Est 1,22}

\bibleverse{13} Se enviaron cartas por mensajero a todas las provincias
del imperio del rey con órdenes de destruir, matar y aniquilar a todos
los judíos, jóvenes y ancianos, mujeres y niños, y confiscar sus
posesiones, todo en un solo día: el día trece del duodécimo mes, el mes
de Adar. \bibleverse{14} Una copia del decreto debía ser emitida como
ley en cada provincia y publicitada al pueblo para que estuviera
preparado para ese día. \bibleverse{15} Por orden del rey, los
mensajeros se apresuraron a seguir su camino. El decreto se emitió
también en la fortaleza de Susa. El rey y Amán se sentaron a beber
mientras la gente de la ciudad de Susa estaba muy turbada.\footnote{\textbf{3:15}
  No sólo los judíos que vivían allí, sino también otras minorías
  étnicas/religiosas debían estar preocupados por tal precedente.}

\hypertarget{el-dolor-de-mardochai-sus-esfuerzos-para-mover-a-ester-a-salvar-a-los-juduxedos}{%
\subsection{El dolor de Mardochai; sus esfuerzos para mover a Ester a
salvar a los
judíos}\label{el-dolor-de-mardochai-sus-esfuerzos-para-mover-a-ester-a-salvar-a-los-juduxedos}}

\hypertarget{section-3}{%
\section{4}\label{section-3}}

\bibleverse{1} Cuando Mardoqueo se enteró de todo lo que había sucedido,
rasgó sus ropas y se vistió de saco y ceniza, y recorrió la ciudad
llorando y lamentándose de dolor. \bibleverse{2} Llegó hasta la puerta
del palacio, porque a nadie se le permitía entrar en la puerta del
palacio vestido de cilicio. \bibleverse{3} Cuando el decreto y las
órdenes del rey llegaron a todas las provincias, los judíos se pusieron
a llorar con terrible angustia. Ayunaron, lloraron y se lamentaron, y
muchos se acostaron con saco y ceniza.

\hypertarget{ester-es-informada-por-mardochai-sobre-el-desastre-inminente-y-le-pide-que-ruegue-al-rey-por-misericordia}{%
\subsection{Ester es informada por Mardochai sobre el desastre inminente
y le pide que ruegue al rey por
misericordia}\label{ester-es-informada-por-mardochai-sobre-el-desastre-inminente-y-le-pide-que-ruegue-al-rey-por-misericordia}}

\bibleverse{4} Las doncellas y los eunucos de Ester vinieron y le
dijeron,\footnote{\textbf{4:4} Claramente le dijeron a Ester lo que su
  primo estaba haciendo, pero no le dieron ninguna explicación.} y la
reina estaba muy disgustada. Le envió ropa para que se quitara el
cilicio, pero él se negó a aceptarla. \bibleverse{5} Llamó a Hatac, uno
de los eunucos del rey asignados para atenderla, y le ordenó que fuera a
ver a Mardoqueo y averiguara qué estaba haciendo y por qué.
\bibleverse{6} Hatac fue a ver a Mardoqueo en la plaza de la ciudad,
frente a la puerta del palacio. \bibleverse{7} Mardoqueo le explicó todo
lo que le había sucedido,\footnote{\textbf{4:7} Esto seguramente habría
  incluido también el problema de Amán con Mardoqueo que había
  precipitado la crisis.} incluyendo la cantidad exacta de dinero que
Amán había prometido pagar al tesoro real por la destrucción de los
judíos. \bibleverse{8} Mardoqueo también le dio una copia del decreto
que se había emitido en Susa para su destrucción, para que se lo
mostrara a Ester y se lo explicara, y le pidió que la instruyera para
que fuera a ver al rey y le pidiera clemencia y le rogara por su pueblo.

\hypertarget{la-negativa-de-esther-es-derrotada-por-mardochai-sin-embargo-requiere-que-los-juduxedos-mantengan-un-ayuno-estricto-a-su-favor.}{%
\subsection{La negativa de Esther es derrotada por Mardochai; Sin
embargo, requiere que los judíos mantengan un ayuno estricto a su
favor.}\label{la-negativa-de-esther-es-derrotada-por-mardochai-sin-embargo-requiere-que-los-juduxedos-mantengan-un-ayuno-estricto-a-su-favor.}}

\bibleverse{9} Hatac regresó y le contó a Ester lo que Mardoqueo había
dicho. \bibleverse{10} Entonces Ester habló con Hatac y le ordenó que
entregara este mensaje a Mardoqueo. \bibleverse{11} ``Todos los
funcionarios del rey, e incluso la gente de las provincias del imperio
del rey, saben que cualquier hombre o cualquier mujer que se dirija al
rey, entrando en su corte interior sin ser convocado, es condenado a
muerte -esa es la única ley del rey- a menos que el rey les tienda su
cetro de oro para que puedan vivir. En mi caso, hace treinta días que no
me llaman para ir al rey''. \footnote{\textbf{4:11} Est 5,2; Est 8,4}

\bibleverse{12} Cuando le contaron a Mardoqueo lo que dijo Ester,
\bibleverse{13} Mardoqueo le devolvió el mensaje a Ester, diciendo:
``¡No creas que porque vives en el palacio del rey tu vida es la única
que se salvará de todos los judíos! \bibleverse{14} Si te quedas callada
ahora, la ayuda y el rescate llegarán a los judíos desde algún otro
lugar, y tú y tus parientes morirán. Quién sabe: ¡podría ser que hayas
venido a ser reina para un momento como éste!''

\bibleverse{15} Ester le respondió a Mardoqueo diciendo: \bibleverse{16}
``Haz que todos los judíos de Susa se reúnan y ayunen por mí. No coman
ni beban nada durante tres días y tres noches. Yo y mis doncellas
también ayunaremos. Después iré a ver al rey, aunque sea contra la ley,
y si muero, que muera''. \footnote{\textbf{4:16} 2Re 7,4}

\bibleverse{17} Mardoqueo fue e hizo todo lo que Ester le había dicho
que hiciera.

\hypertarget{la-recepciuxf3n-amistosa-de-ester-por-parte-del-rey-y-el-engauxf1o-de-amuxe1n}{%
\subsection{La recepción amistosa de Ester por parte del rey y el engaño
de
Amán}\label{la-recepciuxf3n-amistosa-de-ester-por-parte-del-rey-y-el-engauxf1o-de-amuxe1n}}

\hypertarget{section-4}{%
\section{5}\label{section-4}}

\bibleverse{1} Tres días después, Ester se vistió con sus ropas reales y
fue a situarse en el patio interior del palacio real, frente al salón
del rey. El rey estaba sentado en su trono real en el salón del rey,
frente a la entrada. \bibleverse{2} Cuando el rey vio a la reina Ester
de pie en el patio interior, se ganó su aprobación, así que actuó
favorablemente tendiéndole su cetro. Entonces Ester se acercó y tocó el
extremo del cetro.

\bibleverse{3} El rey le preguntó: ``¿Qué pasa, reina Ester? ¿Qué
quieres? Te lo daré, tanto como la mitad de mi imperio''.

\bibleverse{4} Ester respondió: ``Si le place a Su Majestad, que el rey
y Amán vengan hoy a una cena que he preparado para él''. \footnote{\textbf{5:4}
  Est 1,19}

\hypertarget{el-rey-invitado-por-ester-a-cenar-con-amuxe1n-acepta-otra-invitaciuxf3n-a-cenar}{%
\subsection{El rey, invitado por Ester a cenar con Amán, acepta otra
invitación a
cenar}\label{el-rey-invitado-por-ester-a-cenar-con-amuxe1n-acepta-otra-invitaciuxf3n-a-cenar}}

\bibleverse{5} ``Trae a Amán de inmediato para que podamos hacer lo que
Ester ha pedido'', ordenó el rey. El rey y Amán fueron a la cena que
Ester había preparado.

\bibleverse{6} Mientras bebían el vino, el rey le preguntó a Ester:
``¿Qué es lo que realmente pides? Se te dará. ¿Qué quieres? Lo tendrás,
tanto como la mitad de mi imperio''.

\bibleverse{7} Ester respondió: ``Esto es lo que pido y esto es lo que
quiero. \bibleverse{8} Si el rey me mira con buenos ojos, y si le place
a Su Majestad conceder mi petición y hacer lo que pido, que el rey y
Amán vengan a una cena que les prepararé. Mañana responderé a la
pregunta de Su Majestad''.

\hypertarget{el-altivo-engauxf1o-de-amuxe1n-su-intenciuxf3n-de-deshacerse-de-mardochai}{%
\subsection{El altivo engaño de Amán; su intención de deshacerse de
Mardochai}\label{el-altivo-engauxf1o-de-amuxe1n-su-intenciuxf3n-de-deshacerse-de-mardochai}}

\bibleverse{9} Cuando Amán se marchó aquel día estaba muy contento y
satisfecho de sí mismo. Pero cuando vio a Mardoqueo en la puerta del
palacio y que no se levantó ni tembló de miedo ante él, Amán se
enfureció con Mardoqueo. \bibleverse{10} Sin embargo, Amán se controló y
se fue a su casa. Allí invitó a sus amigos. Una vez reunidos ellos y su
esposa Zeres, \bibleverse{11} Amán se explayó sobre la cantidad de
dinero y posesiones que tenía, y sobre la cantidad de hijos, y sobre
cómo el rey lo había hecho tan importante al promoverlo por encima de
todos los demás nobles y funcionarios.

\bibleverse{12} ``Además de todo eso'', continuó Amán, ``fui la única
persona a la que la reina Ester invitó a venir a una cena que había
preparado para el rey. También he sido invitado por ella a comer junto
al rey mañana''. \bibleverse{13} Entonces dijo: ``Pero todo esto no vale
nada\footnote{\textbf{5:13} ``No vale nada'': en otras palabras, no le
  aportaba ninguna satisfacción.} a mí mientras sigo viendo al judío
Mardoqueo sentado a la puerta del palacio''.

\bibleverse{14} Su esposa Zeres y sus amigos le dijeron: ``Haz que se
levante un poste de cincuenta codos de altura. Luego, por la mañana, ve
y pide al rey que haga empalar a Mardoqueo en él. Después, serás feliz
mientras vas con el rey a la cena''. A Amán le pareció un buen consejo,
así que hizo colocar el poste.

\hypertarget{mardochai-criado-en-alto-honor-por-amuxe1n}{%
\subsection{Mardochai criado en alto honor por
Amán}\label{mardochai-criado-en-alto-honor-por-amuxe1n}}

\hypertarget{section-5}{%
\section{6}\label{section-5}}

\bibleverse{1} Esa noche el rey no pudo dormir, así que ordenó que le
trajeran el Libro de Registros del Reinado para que se lo leyeran.
\bibleverse{2} Allí descubrió el relato de lo que Mardoqueo había
informado sobre Bigtana y Teres, los dos eunucos del rey que eran
porteros y que habían conspirado para asesinar al rey Jerjes.
\footnote{\textbf{6:2} Est 2,21-23} \bibleverse{3} ``¿Qué honor o
posición recibió Mardoqueo como recompensa por hacer esto?'' , preguntó
el rey. ``No se ha hecho nada por él'', respondieron los asistentes del
rey.

\bibleverse{4} ``¿Quién está aquí en la corte?'' , preguntó el rey.
Casualmente, Amán había llegado al patio exterior del palacio real para
pedirle al rey que hiciera empalar a Mardoqueo en el poste que le había
colocado.

\hypertarget{amuxe1n-involuntariamente-hace-que-el-rey-decida-sobre-un-honor-extraordinario-para-mardochai-y-que-lo-lleve-a-cabo-personalmente.}{%
\subsection{Amán involuntariamente hace que el rey decida sobre un honor
extraordinario para Mardochai y que lo lleve a cabo
personalmente.}\label{amuxe1n-involuntariamente-hace-que-el-rey-decida-sobre-un-honor-extraordinario-para-mardochai-y-que-lo-lleve-a-cabo-personalmente.}}

\bibleverse{5} Los asistentes del rey le dijeron: ``Amán está esperando
en el patio''. ``Dile que entre'', ordenó el rey.

\bibleverse{6} Cuando Amán entró, el rey le preguntó: ``¿Qué hay que
hacer por un hombre al que el rey quiere honrar?'' Amán se dijo a sí
mismo: ``¿A quién querría honrar el rey sino a mí?'' .

\bibleverse{7} Entonces Amán le dijo al rey: ``A un hombre al que el rey
quiere honrar \bibleverse{8} hay que traerle las ropas reales que el rey
ha usado,\footnote{\textbf{6:8} Esto solía ser un delito castigado con
  la muerte, ya que se acercaba a la pretensión de ser rey. Sólo el
  reypodíaautorizar un acto tan presuntuoso.} un caballo que el rey haya
montado y que tenga un tocado real en la cabeza. \bibleverse{9} Haz que
las vestimentas y el caballo sean entregados a uno de los más altos
funcionarios y nobles del rey. Que se asegure de que el hombre al que el
rey desea honrar se vista con los trajes reales y que monte en el
caballo por las calles de la ciudad, y que el funcionario anuncie ante
él: `¡Esto es lo que se hace por el hombre al que el rey desea
honrar!'\,''

\bibleverse{10} Entonces el rey le dijo a Amán: ``¡Bien! ¡Vete! Trae
rápidamente las vestiduras reales y el caballo, y haz lo que has dicho
para el judío Mardoqueo, que está sentado a la puerta del palacio. No
omitas nada de lo que has mencionado''.

\bibleverse{11} Amán fue a buscar las túnicas y el caballo. Vistió a
Mardoqueo, lo colocó en el caballo y lo condujo por las calles de la
ciudad, gritando delante de él: ``¡Esto es lo que se hace por el hombre
que el rey desea honrar!''

\hypertarget{el-dolor-de-amuxe1n-lleno-de-presentimientos-fue-al-banquete-de-la-reina}{%
\subsection{El dolor de Amán; Lleno de presentimientos, fue al banquete
de la
reina}\label{el-dolor-de-amuxe1n-lleno-de-presentimientos-fue-al-banquete-de-la-reina}}

\bibleverse{12} Mardoqueo regresó a la puerta del palacio, pero Amán
corrió a su casa, llorando y cubriendo su cabeza de vergüenza.
\bibleverse{13} Amán explicó a su esposa Zeres y a todos sus amigos lo
que le había sucedido. Estos sabios amigos y su esposa Zeres le dijeron:
``Si Mardoqueo es del pueblo judío, y ya has empezado a perder la
categoría ante él, no podrás vencerlo. Vas a perder ante él, vas a
caer!''\footnote{\textbf{6:13} Literalmente esta frase dice: ``para caer
  caerás ante él''.} \bibleverse{14} Mientras seguían hablando con él,
llegaron los eunucos del rey y llevaron rápidamente a Amán a la cena que
Ester había preparado.\footnote{\textbf{6:14} Est 5,8}

\hypertarget{durante-la-cena-ester-revela-los-planes-de-amuxe1n-de-matar-al-rey-el-rey-se-levanta-enojado-de-la-cena}{%
\subsection{Durante la cena, Ester revela los planes de Amán de matar al
rey; el rey se levanta enojado de la
cena}\label{durante-la-cena-ester-revela-los-planes-de-amuxe1n-de-matar-al-rey-el-rey-se-levanta-enojado-de-la-cena}}

\hypertarget{section-6}{%
\section{7}\label{section-6}}

\bibleverse{1} El rey y Amán fueron a la cena de la reina Ester.
\footnote{\textbf{7:1} Est 5,8; Est 6,14} \bibleverse{2} En esta segunda
cena, mientras bebían vino, el rey volvió a preguntar a Ester: ``¿Qué es
lo que realmente pides, reina Ester? Se te dará. ¿Qué quieres? Lo
tendrás, tanto como la mitad de mi imperio''.

\bibleverse{3} La reina Ester respondió: ``Si el rey me mira con buenos
ojos, y si le place a Su Majestad concederme la vida, eso es lo que
pido; y la vida de mi pueblo, eso es lo que pido. \bibleverse{4} Porque
mi pueblo y yo hemos sido vendidos\footnote{\textbf{7:4} ``Vendidos'':
  o, ``entregados''.} para ser destruidos, asesinados y aniquilados. Si
sólo hubiéramos sido vendidos como esclavos, me habría callado, porque
nuestro sufrimiento no habría justificado molestar al rey''.\footnote{\textbf{7:4}
  O ``aunque nuestro sufrimiento no podría haber compensado lo que el
  rey perdió''.}

\bibleverse{5} El rey preguntó a la reina Ester, exigiendo saber:
``¿Quién es éste? ¿Dónde está el hombre que se ha atrevido a hacer
esto?''

\bibleverse{6} ``¡El hombre, el adversario, el enemigo, es este malvado
Amán!'' respondió Ester. Amán tembló de terror ante el rey y la reina.

\bibleverse{7} El rey estaba furioso. Se levantó, dejando el vino, y
salió al jardín del palacio. Amán se quedó para suplicar por su vida a
la reina Ester, pues se dio cuenta de que el rey planeaba un
mal\footnote{\textbf{7:7} Aquí se utiliza la misma palabra que Ester usa
  para describir a Amán en el versículo 6.} fin para él.

\hypertarget{a-su-regreso-el-rey-condenuxf3-a-muerte-a-amuxe1n-e-inmediatamente-lo-hizo-colgar-en-la-estaca-erigida-para-mardochai}{%
\subsection{A su regreso, el rey condenó a muerte a Amán e
inmediatamente lo hizo colgar en la estaca erigida para
Mardochai}\label{a-su-regreso-el-rey-condenuxf3-a-muerte-a-amuxe1n-e-inmediatamente-lo-hizo-colgar-en-la-estaca-erigida-para-mardochai}}

\bibleverse{8} Cuando el rey regresó del jardín del palacio al comedor,
Amán se había tirado\footnote{\textbf{7:8} ``Se había tirado'':
  literalmente, ``había caído'', pero no fue una caída accidental sino
  un intento deliberado de pedir clemencia. Sin embargo, esto no hizo
  más que agravar su culpabilidad a los ojos del rey.} en el sofá donde
estaba la reina Ester. El rey gritó: ``¿Acaso va a violar a la reina
aquí en el palacio, delante de mí?'' . En cuanto el rey dijo esto, los
sirvientes le cubrieron la cara a Amán.

\bibleverse{9} Entonces Harbona, uno de los eunucos que asistían al rey,
dijo ``Amán levantó un poste junto a su casa para Mardoqueo, aquel cuyo
informe salvó la vida del rey. El poste tiene cincuenta codos de
altura''. ``¡Empaladlo en él!'', ordenó el rey.

\bibleverse{10} Así que empalaron a Amán en el poste que había colocado
para Mardoqueo. Entonces se calmó la ira del rey.

\hypertarget{el-regalo-de-ester-y-la-exaltaciuxf3n-de-mardochai-por-parte-del-rey}{%
\subsection{El regalo de Ester y la exaltación de Mardochai por parte
del
rey}\label{el-regalo-de-ester-y-la-exaltaciuxf3n-de-mardochai-por-parte-del-rey}}

\hypertarget{section-7}{%
\section{8}\label{section-7}}

\bibleverse{1} Ese mismo día el rey Jerjes entregó a la reina Ester la
propiedad que había pertenecido a Amán, el enemigo de los judíos.
Además, Mardoqueo se presentó ante el rey, porque Ester le había
explicado quién era. \bibleverse{2} El rey le quitó el anillo de sello
que le había quitado a Amán y se lo dio a Mardoqueo. Ester puso a
Mardoqueo a cargo de los bienes de Amán. \footnote{\textbf{8:2} Est 3,10}

\hypertarget{establecer-y-promulgar-medidas-de-protecciuxf3n-para-los-juduxedos-contra-sus-enemigos}{%
\subsection{Establecer y promulgar medidas de protección para los judíos
contra sus
enemigos}\label{establecer-y-promulgar-medidas-de-protecciuxf3n-para-los-juduxedos-contra-sus-enemigos}}

\bibleverse{3} Ester fue a hablar de nuevo con el rey, cayendo a sus
pies y llorando, suplicándole que acabara con el malvado plan de Amán el
agagueo que había ideado para destruir a los judíos. \bibleverse{4} Una
vez más, el rey le tendió a Ester el cetro de oro. Ella se levantó y se
puso de pie ante él. \bibleverse{5} Ester le dijo: ``Si le place a Su
Majestad, y si me ve con buenos ojos, y si el rey cree que es lo
correcto, y si se complace conmigo, que se emita una orden que revoque
las cartas enviadas por Amán, hijo de Hamedata, el agagueo, con su
artero plan para destruir a los judíos en todas las provincias del rey.
\bibleverse{6} ¿Cómo podré soportar ver el desastre que está a punto de
caer sobre mi pueblo? ¿Cómo podré soportar ver la destrucción de mi
familia?''

\bibleverse{7} El rey Jerjes dijo a la reina Ester y al judío Mardoqueo:
``Le entrego a Ester la hacienda de Amán, que fue empalado en un poste
porque quería matar a los judíos. \bibleverse{8} Ahora pueden escribir
una orden con respecto a los judíos de la manera que ustedes quieran, en
nombre del rey, y sellarla con el anillo de sello del rey. Porque ningún
decreto escrito en nombre del rey y sellado con su anillo de sello puede
ser revocado''.\footnote{\textbf{8:8} Ester ha pedido (versículo 5) que
  se revoque el decreto anterior del rey. El rey Jerjes le recuerda que
  ningún decreto puede ser revocado, sin embargo un nuevo decreto
  tampoco puede ser revocado, y esto puede contrarrestar el efecto del
  decreto anterior.}

\bibleverse{9} Los secretarios del rey fueron convocados y el día
veintitrés del tercer mes, el mes de Siván, y escribieron un decreto con
todas las órdenes de Mardoqueo a los judíos y a los oficiales
principales del rey, los gobernadores y los nobles de las 127 provincias
desde la India hasta Etiopía. Escribió a cada provincia en su propia
escritura, a cada pueblo en su propia lengua, y a los judíos en su
propia escritura y lengua. \bibleverse{10} Escribió en nombre del rey
Jerjes y las selló con el anillo del rey. Envió las cartas por medio de
un mensajero a caballo, que montaba veloces caballos de pura sangre del
rey. \bibleverse{11} Las cartas del rey autorizaban a los judíos de cada
ciudad a reunirse en defensa propia y a destruir, matar y aniquilar a
cualquier grupo armado de un pueblo o provincia que los atacara,
incluyendo a las mujeres y los niños, y a confiscar sus posesiones.
\bibleverse{12} Esto debía ocurrir en un día en todas las provincias del
rey Jerjes, el día trece del duodécimo mes, el mes de Adar.\footnote{\textbf{8:12}
  La misma fecha del decreto original. Ver 3:13.} \bibleverse{13} Una
copia del decreto debía emitirse como ley en cada provincia y darse a
conocer al pueblo para que los judíos estuvieran listos en ese día para
pagar a sus enemigos. \bibleverse{14} Por orden del rey, los mensajeros
montados en los caballos de relevo del rey salieron a toda prisa. El
decreto se emitió también en la fortaleza de Susa.

\hypertarget{mardochai-aparece-en-susa-con-un-traje-principesco-alegruxeda-de-los-juduxedos-en-todo-el-imperio}{%
\subsection{Mardochai aparece en Susa con un traje principesco; Alegría
de los judíos en todo el
imperio}\label{mardochai-aparece-en-susa-con-un-traje-principesco-alegruxeda-de-los-juduxedos-en-todo-el-imperio}}

\bibleverse{15} Entonces Mardoqueo salió del rey, vestido con ropas
reales de azul y blanco, con una gran corona de oro y un manto de
púrpura de lino fino. La ciudad de Susa gritó de alegría.
\bibleverse{16} Para los judíos fue un tiempo brillante de felicidad,
alegría y respeto. \bibleverse{17} En todas las provincias y en todas
las ciudades, dondequiera que la orden y el decreto del rey habían
llegado, los judíos estaban alegres y felices; hacían fiestas y
celebraciones. Mucha gente se hizo judía, porque les habían cogido
miedo.\footnote{\textbf{8:17} Éxod 15,14-16}

\hypertarget{exterminio-de-enemigos-de-los-juduxedos-en-todo-el-imperio-el-duxeda-13-del-mes-de-adar}{%
\subsection{Exterminio de enemigos de los judíos en todo el imperio el
día 13 del mes de
Adar}\label{exterminio-de-enemigos-de-los-juduxedos-en-todo-el-imperio-el-duxeda-13-del-mes-de-adar}}

\hypertarget{section-8}{%
\section{9}\label{section-8}}

\bibleverse{1} El decimotercer día del duodécimo mes, el mes de Adar,
debía cumplirse la orden y el decreto del rey. Ese día los enemigos de
los judíos pensaron que los aplastarían, pero sucedió exactamente lo
contrario: los judíos aplastaron a sus enemigos. \bibleverse{2} Los
judíos se reunieron en sus ciudades por todas las provincias del rey
Jerjes para atacar a los que querían destruirlos. Nadie podía oponerse a
ellos, porque todos los demás pueblos les tenían miedo. \bibleverse{3}
Todos los funcionarios de las provincias, los jefes, los gobernadores y
los funcionarios del rey ayudaron a los judíos, porque tenían miedo de
Mardoqueo. \bibleverse{4} Mardoqueo tenía mucho poder en el palacio
real, y su reputación se extendía por las provincias a medida que
aumentaba su poder. \bibleverse{5} Los judíos atacaban a sus enemigos
con espadas, matándolos y destruyéndolos, y hacían lo que querían con
sus enemigos. \bibleverse{6} En la fortaleza de Susa, los judíos mataron
y destruyeron a quinientos hombres. \bibleverse{7} Entre ellos estaban
Parsandata, Dalfón, Aspata, \bibleverse{8} Porata, Adalia, Aridata,
\bibleverse{9} Parmasta, Arisai, Aridai y Vaizata, \bibleverse{10} los
diez hijos de Amán, hijo de Hamedata, el enemigo de los judíos, pero no
tomaron sus posesiones.

\hypertarget{continuaciuxf3n-de-la-matanza-el-duxeda-14-del-mes-regocijo-de-los-juduxedos-para-celebrar-su-salvaciuxf3n}{%
\subsection{Continuación de la matanza el día 14 del mes; Regocijo de
los judíos para celebrar su
salvación}\label{continuaciuxf3n-de-la-matanza-el-duxeda-14-del-mes-regocijo-de-los-juduxedos-para-celebrar-su-salvaciuxf3n}}

\bibleverse{11} Ese mismo día, cuando se informó al rey del número de
los muertos en la fortaleza de Susa, \bibleverse{12} éste dijo a la
reina Ester: ``Los judíos han matado y destruido a quinientos hombres en
la fortaleza de Susa, incluidos los diez hijos de Amán. ¡Imagina lo que
han hecho en el resto de las provincias reales! Ahora, ¿qué es lo que
quieres pedir? Se te dará. ¿Qué más quieres? Se te concederá''.
\footnote{\textbf{9:12} Est 5,6; Est 7,2}

\bibleverse{13} ``Si le place a Su Majestad'', respondió Ester,
``permita que los judíos de Susa hagan mañana lo mismo que han hecho
hoy, siguiendo el decreto. Además, que los diez hijos de Amán sean
empalados en postes''.

\bibleverse{14} El rey ordenó que se hiciera esto. Se emitió un decreto
en Susa, y empalaron los cuerpos de los diez hijos de Amán.
\bibleverse{15} El día catorce del mes de Adar, los judíos de Susa
volvieron a reunirse y mataron allí a trescientos hombres, pero de nuevo
no tomaron sus posesiones.

\bibleverse{16} Los demás judíos de las provincias del rey también se
reunieron para defenderse y librarse de sus enemigos. Mataron a setenta
y cinco mil que los odiaban, pero no tocaron sus posesiones.
\bibleverse{17} Esto sucedió el día trece del mes de Adar, y el día
catorce descansaron y lo convirtieron en un día de fiesta y celebración.

\bibleverse{18} Sin embargo, los judíos de Susa se habían reunido para
luchar los días trece y catorce del mes. Así que descansaron el día
quince, y lo convirtieron en un día de fiesta y celebración.
\bibleverse{19} Hasta el día de hoy, los judíos rurales que viven en las
aldeas observan el decimocuarto día del mes de Adar como un día de
celebración y fiesta, un día festivo en el que se envían regalos unos a
otros.

\hypertarget{mardochai-ordena-la-celebraciuxf3n-de-la-fiesta-de-purim-para-todos-los-futuros}{%
\subsection{Mardochai ordena la celebración de la fiesta de Purim para
todos los
futuros}\label{mardochai-ordena-la-celebraciuxf3n-de-la-fiesta-de-purim-para-todos-los-futuros}}

\bibleverse{20} Mardoqueo registró estos sucesos y envió cartas a todos
los judíos de las provincias gobernadas por el rey Jerjes, cerca y
lejos, \bibleverse{21} exigiéndoles que celebraran todos los años los
días catorce y quince del mes de Adar \bibleverse{22} como el momento en
que los judíos descansaban de su victoria sobre sus enemigos, y como el
mes en que su tristeza se convertía en alegría y su luto en un tiempo de
celebración. Les escribió que observaran los días como días de fiesta y
alegría y que se dieran regalos de comida unos a otros y regalos a los
pobres. \bibleverse{23} Los judíos acordaron continuar con lo que ya
habían comenzado a hacer, siguiendo lo que Mardoqueo les había escrito.
\bibleverse{24} Porque Amán, hijo de Hamedata, el agagueo, enemigo de
todos los judíos, había tramado destruir a los judíos, y había echado
``pur'' (es decir, una ``suerte'') para aplastarlos y destruirlos.
\bibleverse{25} Pero cuando llegó a conocimiento del rey, éste envió
cartas ordenando que el malvado plan que Amán había planeado contra los
judíos recayera sobre él, y que él y sus hijos fueran empalados en
postes. \footnote{\textbf{9:25} Est 9,14; Est 7,10}

\bibleverse{26} (Por eso estos días se llaman Purim, de la palabra
Pur.\footnote{\textbf{9:26} ``Purim'' es el plural de ``Pur''.} ) Como
resultado de todas las instrucciones de la carta de Mardoqueo, y de lo
que habían visto, y de lo que les había sucedido, \bibleverse{27} los
judíos se comprometieron a adoptar la práctica de que ellos y sus
descendientes, y todos los que se unieran a ellos, no se olvidaran de
celebrar estos dos días tal como se había establecido, y en el momento
adecuado cada año. \bibleverse{28} Estos días debían ser recordados y
celebrados por cada generación, familia, provincia y ciudad, para que
estos días de Purim fueran siempre observados entre los judíos y no
fueran olvidados por sus descendientes. \bibleverse{29} Entonces la
reina Ester, hija de Abihail, escribió una carta, junto con Mardoqueo el
judío, dando en su carta plena autoridad a la carta de Mardoqueo sobre
Purim. \bibleverse{30} También se enviaron cartas expresando paz y
tranquilidad a todos los judíos de las 127 provincias del imperio del
rey Jerjes. \bibleverse{31} Establecieron estos días de Purim en su
momento, tal como lo habían ordenado Mardoqueo el judío y la reina
Ester, comprometiéndose ellos y sus descendientes a los tiempos de ayuno
y luto. \bibleverse{32} De este modo, el decreto de Ester confirmó estas
prácticas relativas a Purim, que se inscribieron en el registro oficial.

\hypertarget{posiciuxf3n-de-poder-y-servicios-de-mardochai-para-el-bienestar-de-los-juduxedos}{%
\subsection{Posición de poder y servicios de Mardochai para el bienestar
de los
judíos}\label{posiciuxf3n-de-poder-y-servicios-de-mardochai-para-el-bienestar-de-los-juduxedos}}

\hypertarget{section-9}{%
\section{10}\label{section-9}}

\bibleverse{1} El rey Jerjes impuso impuestos en todo el imperio,
incluso en sus costas más lejanas. \bibleverse{2} Todo lo que logró con
su poder y su fuerza, así como el relato completo de la alta posición a
la que el rey ascendió a Mardoqueo, están escritos en el Libro de las
Actas de los reyes de Media y Persia. \bibleverse{3} Porque el judío
Mardoqueo era el segundo al mando del rey Jerjes, líder de los judíos y
muy respetado en la comunidad judía, trabajó para ayudar a su pueblo y
para mejorar la seguridad de todos los judíos.
